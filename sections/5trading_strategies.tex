\chapter{Trading Strategies}\label{ch:predictions}

\section{Trading Strategies - Introduction and Theory}
\subsection{Idea and Process}
We compare different Trading Strategies. With the Goal of Predicting Stocks this is interesting in and of itself. For the purpose of the project it also serves as a baseline to compare our Trading Strategy against. 

\subsection{Mean Reversion Models and Momentum Trading - Theoretical Background}

A vast body of scientific literature has tried to develop models that allow to understand and explain movements in stock markets. An even vaster community of traders has tried to implement these theories to do actual forecasting. While many of the theories are much more complex, two basic ideas can be summarized as "Momentum Based Trading" and "Mean Reversion Based Trading". The former theory hypothesizes that stocks that do well now will likely continue to do so in the future while the latter states that especially good or past performance is an exception and that stocks will eventually return to their average performance. A third strategy, Pairs Trading, will be detailed later. 

In 1985 [Bondt and Thaler] were one of the early scholars to analyze mean reversion behaviour when they examined the hypothesis that markets tend to overreact. They looked at monthly returns of assets listed on the New York Stock Exchange in between 1926 and 1982 and constructed portfolios  of winners and losers that were updated every three years. Winners and losers were those stocks that had performed the best / worst over the previous years. Their idea was that "if stock prices systematically overshoot, then their reversal should be predictable from past return data alone, with no use of any accounting data such as earnings. [...] Extreme movements in stock prices will be followed by subsequent price movements in the opposite direction." Indeed they observed that the portfolio of losers outperformed the market significantly. While they focus on a very long time horizon of three years, Jagadeesh(1991) suggests that this behaviour could also be observed in the shorter term:  "These papers show that contrarian strategies that select stocks based on their returns in the previous week or month generate significant abnormal returns." 
While mean reversion behaviour has sometimes been observed in practice, it is not trivial to derive it from economic theory. Many economic theories like the famous model from Fama and French describe the return of the stocks of a company as the result of market properties like the baseline market return rate and inherent properties of that company, like for example its book-to-market-ratio [Fama and French 1993]. If such a relationship exists than this in turn implies that the actual observed stock movements should be random fluctuations around some much more slowly changing true return rate. However, it is unclear whether this mean reversion should happen in the form of a pendulum that swings forth and back or more in the form of coin tosses reverting to their equilibrium by flooding past observations with new random ones. 
Something about multiplicative connections?

On the opposite side of the spectrum of trading strategies lies the idea of momentum. [Japateesh and CX] in 1993 were one of the first to describe "that strategies which buy stocks that have performed well in the past and sell stocks that have performed poorly in the past generate significant positive returns over 3- to 12-month holding periods." [Japateesh and CX] looked at portfolios and saw mean reversion over 12 to 48 months. Thaler and Bondt observe mean reversion over the time frame of 36 months. Many scholares agree that mean reversion and momentum behaviour are not necessarily contradictions, but that the time frame determines in which way a stock will behave [Balvers and Wu]. 


PAIRS TRADING



\section{Trading Strategies - Implementation}
\subsection{Mean Reversion Portfolio}
While their analysis is focused on monthly returns over a much longer time frame, the basic idea that was replicated in our trading strategy was the same. While they found excessive returns of that strategy from 1932 to 1977 our implementation was unforturnately much less successful.

\begin{figure}
    \centering
    \begin{adjustbox}{width=.7\textwidth,center}
        %% Creator: Matplotlib, PGF backend
%%
%% To include the figure in your LaTeX document, write
%%   \input{<filename>.pgf}
%%
%% Make sure the required packages are loaded in your preamble
%%   \usepackage{pgf}
%%
%% Figures using additional raster images can only be included by \input if
%% they are in the same directory as the main LaTeX file. For loading figures
%% from other directories you can use the `import` package
%%   \usepackage{import}
%% and then include the figures with
%%   \import{<path to file>}{<filename>.pgf}
%%
%% Matplotlib used the following preamble
%%   \usepackage{fontspec}
%%   \setmainfont{DejaVuSerif.ttf}[Path=/opt/tljh/user/lib/python3.6/site-packages/matplotlib/mpl-data/fonts/ttf/]
%%   \setsansfont{DejaVuSans.ttf}[Path=/opt/tljh/user/lib/python3.6/site-packages/matplotlib/mpl-data/fonts/ttf/]
%%   \setmonofont{DejaVuSansMono.ttf}[Path=/opt/tljh/user/lib/python3.6/site-packages/matplotlib/mpl-data/fonts/ttf/]
%%
\begingroup%
\makeatletter%
\begin{pgfpicture}%
\pgfpathrectangle{\pgfpointorigin}{\pgfqpoint{6.806467in}{2.713314in}}%
\pgfusepath{use as bounding box, clip}%
\begin{pgfscope}%
\pgfsetbuttcap%
\pgfsetmiterjoin%
\definecolor{currentfill}{rgb}{1.000000,1.000000,1.000000}%
\pgfsetfillcolor{currentfill}%
\pgfsetlinewidth{0.000000pt}%
\definecolor{currentstroke}{rgb}{1.000000,1.000000,1.000000}%
\pgfsetstrokecolor{currentstroke}%
\pgfsetdash{}{0pt}%
\pgfpathmoveto{\pgfqpoint{0.000000in}{0.000000in}}%
\pgfpathlineto{\pgfqpoint{6.806467in}{0.000000in}}%
\pgfpathlineto{\pgfqpoint{6.806467in}{2.713314in}}%
\pgfpathlineto{\pgfqpoint{0.000000in}{2.713314in}}%
\pgfpathclose%
\pgfusepath{fill}%
\end{pgfscope}%
\begin{pgfscope}%
\pgfsetbuttcap%
\pgfsetmiterjoin%
\definecolor{currentfill}{rgb}{0.917647,0.917647,0.949020}%
\pgfsetfillcolor{currentfill}%
\pgfsetlinewidth{0.000000pt}%
\definecolor{currentstroke}{rgb}{0.000000,0.000000,0.000000}%
\pgfsetstrokecolor{currentstroke}%
\pgfsetstrokeopacity{0.000000}%
\pgfsetdash{}{0pt}%
\pgfpathmoveto{\pgfqpoint{0.506467in}{0.331635in}}%
\pgfpathlineto{\pgfqpoint{6.706467in}{0.331635in}}%
\pgfpathlineto{\pgfqpoint{6.706467in}{2.596635in}}%
\pgfpathlineto{\pgfqpoint{0.506467in}{2.596635in}}%
\pgfpathclose%
\pgfusepath{fill}%
\end{pgfscope}%
\begin{pgfscope}%
\pgfpathrectangle{\pgfqpoint{0.506467in}{0.331635in}}{\pgfqpoint{6.200000in}{2.265000in}}%
\pgfusepath{clip}%
\pgfsetroundcap%
\pgfsetroundjoin%
\pgfsetlinewidth{0.803000pt}%
\definecolor{currentstroke}{rgb}{1.000000,1.000000,1.000000}%
\pgfsetstrokecolor{currentstroke}%
\pgfsetdash{}{0pt}%
\pgfpathmoveto{\pgfqpoint{0.783131in}{0.331635in}}%
\pgfpathlineto{\pgfqpoint{0.783131in}{2.596635in}}%
\pgfusepath{stroke}%
\end{pgfscope}%
\begin{pgfscope}%
\definecolor{textcolor}{rgb}{0.150000,0.150000,0.150000}%
\pgfsetstrokecolor{textcolor}%
\pgfsetfillcolor{textcolor}%
\pgftext[x=0.783131in,y=0.234413in,,top]{\color{textcolor}\rmfamily\fontsize{10.000000}{12.000000}\selectfont 2012}%
\end{pgfscope}%
\begin{pgfscope}%
\pgfpathrectangle{\pgfqpoint{0.506467in}{0.331635in}}{\pgfqpoint{6.200000in}{2.265000in}}%
\pgfusepath{clip}%
\pgfsetroundcap%
\pgfsetroundjoin%
\pgfsetlinewidth{0.803000pt}%
\definecolor{currentstroke}{rgb}{1.000000,1.000000,1.000000}%
\pgfsetstrokecolor{currentstroke}%
\pgfsetdash{}{0pt}%
\pgfpathmoveto{\pgfqpoint{1.726391in}{0.331635in}}%
\pgfpathlineto{\pgfqpoint{1.726391in}{2.596635in}}%
\pgfusepath{stroke}%
\end{pgfscope}%
\begin{pgfscope}%
\definecolor{textcolor}{rgb}{0.150000,0.150000,0.150000}%
\pgfsetstrokecolor{textcolor}%
\pgfsetfillcolor{textcolor}%
\pgftext[x=1.726391in,y=0.234413in,,top]{\color{textcolor}\rmfamily\fontsize{10.000000}{12.000000}\selectfont 2013}%
\end{pgfscope}%
\begin{pgfscope}%
\pgfpathrectangle{\pgfqpoint{0.506467in}{0.331635in}}{\pgfqpoint{6.200000in}{2.265000in}}%
\pgfusepath{clip}%
\pgfsetroundcap%
\pgfsetroundjoin%
\pgfsetlinewidth{0.803000pt}%
\definecolor{currentstroke}{rgb}{1.000000,1.000000,1.000000}%
\pgfsetstrokecolor{currentstroke}%
\pgfsetdash{}{0pt}%
\pgfpathmoveto{\pgfqpoint{2.667073in}{0.331635in}}%
\pgfpathlineto{\pgfqpoint{2.667073in}{2.596635in}}%
\pgfusepath{stroke}%
\end{pgfscope}%
\begin{pgfscope}%
\definecolor{textcolor}{rgb}{0.150000,0.150000,0.150000}%
\pgfsetstrokecolor{textcolor}%
\pgfsetfillcolor{textcolor}%
\pgftext[x=2.667073in,y=0.234413in,,top]{\color{textcolor}\rmfamily\fontsize{10.000000}{12.000000}\selectfont 2014}%
\end{pgfscope}%
\begin{pgfscope}%
\pgfpathrectangle{\pgfqpoint{0.506467in}{0.331635in}}{\pgfqpoint{6.200000in}{2.265000in}}%
\pgfusepath{clip}%
\pgfsetroundcap%
\pgfsetroundjoin%
\pgfsetlinewidth{0.803000pt}%
\definecolor{currentstroke}{rgb}{1.000000,1.000000,1.000000}%
\pgfsetstrokecolor{currentstroke}%
\pgfsetdash{}{0pt}%
\pgfpathmoveto{\pgfqpoint{3.607756in}{0.331635in}}%
\pgfpathlineto{\pgfqpoint{3.607756in}{2.596635in}}%
\pgfusepath{stroke}%
\end{pgfscope}%
\begin{pgfscope}%
\definecolor{textcolor}{rgb}{0.150000,0.150000,0.150000}%
\pgfsetstrokecolor{textcolor}%
\pgfsetfillcolor{textcolor}%
\pgftext[x=3.607756in,y=0.234413in,,top]{\color{textcolor}\rmfamily\fontsize{10.000000}{12.000000}\selectfont 2015}%
\end{pgfscope}%
\begin{pgfscope}%
\pgfpathrectangle{\pgfqpoint{0.506467in}{0.331635in}}{\pgfqpoint{6.200000in}{2.265000in}}%
\pgfusepath{clip}%
\pgfsetroundcap%
\pgfsetroundjoin%
\pgfsetlinewidth{0.803000pt}%
\definecolor{currentstroke}{rgb}{1.000000,1.000000,1.000000}%
\pgfsetstrokecolor{currentstroke}%
\pgfsetdash{}{0pt}%
\pgfpathmoveto{\pgfqpoint{4.548438in}{0.331635in}}%
\pgfpathlineto{\pgfqpoint{4.548438in}{2.596635in}}%
\pgfusepath{stroke}%
\end{pgfscope}%
\begin{pgfscope}%
\definecolor{textcolor}{rgb}{0.150000,0.150000,0.150000}%
\pgfsetstrokecolor{textcolor}%
\pgfsetfillcolor{textcolor}%
\pgftext[x=4.548438in,y=0.234413in,,top]{\color{textcolor}\rmfamily\fontsize{10.000000}{12.000000}\selectfont 2016}%
\end{pgfscope}%
\begin{pgfscope}%
\pgfpathrectangle{\pgfqpoint{0.506467in}{0.331635in}}{\pgfqpoint{6.200000in}{2.265000in}}%
\pgfusepath{clip}%
\pgfsetroundcap%
\pgfsetroundjoin%
\pgfsetlinewidth{0.803000pt}%
\definecolor{currentstroke}{rgb}{1.000000,1.000000,1.000000}%
\pgfsetstrokecolor{currentstroke}%
\pgfsetdash{}{0pt}%
\pgfpathmoveto{\pgfqpoint{5.491698in}{0.331635in}}%
\pgfpathlineto{\pgfqpoint{5.491698in}{2.596635in}}%
\pgfusepath{stroke}%
\end{pgfscope}%
\begin{pgfscope}%
\definecolor{textcolor}{rgb}{0.150000,0.150000,0.150000}%
\pgfsetstrokecolor{textcolor}%
\pgfsetfillcolor{textcolor}%
\pgftext[x=5.491698in,y=0.234413in,,top]{\color{textcolor}\rmfamily\fontsize{10.000000}{12.000000}\selectfont 2017}%
\end{pgfscope}%
\begin{pgfscope}%
\pgfpathrectangle{\pgfqpoint{0.506467in}{0.331635in}}{\pgfqpoint{6.200000in}{2.265000in}}%
\pgfusepath{clip}%
\pgfsetroundcap%
\pgfsetroundjoin%
\pgfsetlinewidth{0.803000pt}%
\definecolor{currentstroke}{rgb}{1.000000,1.000000,1.000000}%
\pgfsetstrokecolor{currentstroke}%
\pgfsetdash{}{0pt}%
\pgfpathmoveto{\pgfqpoint{6.432381in}{0.331635in}}%
\pgfpathlineto{\pgfqpoint{6.432381in}{2.596635in}}%
\pgfusepath{stroke}%
\end{pgfscope}%
\begin{pgfscope}%
\definecolor{textcolor}{rgb}{0.150000,0.150000,0.150000}%
\pgfsetstrokecolor{textcolor}%
\pgfsetfillcolor{textcolor}%
\pgftext[x=6.432381in,y=0.234413in,,top]{\color{textcolor}\rmfamily\fontsize{10.000000}{12.000000}\selectfont 2018}%
\end{pgfscope}%
\begin{pgfscope}%
\pgfpathrectangle{\pgfqpoint{0.506467in}{0.331635in}}{\pgfqpoint{6.200000in}{2.265000in}}%
\pgfusepath{clip}%
\pgfsetroundcap%
\pgfsetroundjoin%
\pgfsetlinewidth{0.803000pt}%
\definecolor{currentstroke}{rgb}{1.000000,1.000000,1.000000}%
\pgfsetstrokecolor{currentstroke}%
\pgfsetdash{}{0pt}%
\pgfpathmoveto{\pgfqpoint{0.506467in}{0.574286in}}%
\pgfpathlineto{\pgfqpoint{6.706467in}{0.574286in}}%
\pgfusepath{stroke}%
\end{pgfscope}%
\begin{pgfscope}%
\definecolor{textcolor}{rgb}{0.150000,0.150000,0.150000}%
\pgfsetstrokecolor{textcolor}%
\pgfsetfillcolor{textcolor}%
\pgftext[x=0.100000in,y=0.521524in,left,base]{\color{textcolor}\rmfamily\fontsize{10.000000}{12.000000}\selectfont 1.00}%
\end{pgfscope}%
\begin{pgfscope}%
\pgfpathrectangle{\pgfqpoint{0.506467in}{0.331635in}}{\pgfqpoint{6.200000in}{2.265000in}}%
\pgfusepath{clip}%
\pgfsetroundcap%
\pgfsetroundjoin%
\pgfsetlinewidth{0.803000pt}%
\definecolor{currentstroke}{rgb}{1.000000,1.000000,1.000000}%
\pgfsetstrokecolor{currentstroke}%
\pgfsetdash{}{0pt}%
\pgfpathmoveto{\pgfqpoint{0.506467in}{0.905330in}}%
\pgfpathlineto{\pgfqpoint{6.706467in}{0.905330in}}%
\pgfusepath{stroke}%
\end{pgfscope}%
\begin{pgfscope}%
\definecolor{textcolor}{rgb}{0.150000,0.150000,0.150000}%
\pgfsetstrokecolor{textcolor}%
\pgfsetfillcolor{textcolor}%
\pgftext[x=0.100000in,y=0.852569in,left,base]{\color{textcolor}\rmfamily\fontsize{10.000000}{12.000000}\selectfont 1.25}%
\end{pgfscope}%
\begin{pgfscope}%
\pgfpathrectangle{\pgfqpoint{0.506467in}{0.331635in}}{\pgfqpoint{6.200000in}{2.265000in}}%
\pgfusepath{clip}%
\pgfsetroundcap%
\pgfsetroundjoin%
\pgfsetlinewidth{0.803000pt}%
\definecolor{currentstroke}{rgb}{1.000000,1.000000,1.000000}%
\pgfsetstrokecolor{currentstroke}%
\pgfsetdash{}{0pt}%
\pgfpathmoveto{\pgfqpoint{0.506467in}{1.236375in}}%
\pgfpathlineto{\pgfqpoint{6.706467in}{1.236375in}}%
\pgfusepath{stroke}%
\end{pgfscope}%
\begin{pgfscope}%
\definecolor{textcolor}{rgb}{0.150000,0.150000,0.150000}%
\pgfsetstrokecolor{textcolor}%
\pgfsetfillcolor{textcolor}%
\pgftext[x=0.100000in,y=1.183613in,left,base]{\color{textcolor}\rmfamily\fontsize{10.000000}{12.000000}\selectfont 1.50}%
\end{pgfscope}%
\begin{pgfscope}%
\pgfpathrectangle{\pgfqpoint{0.506467in}{0.331635in}}{\pgfqpoint{6.200000in}{2.265000in}}%
\pgfusepath{clip}%
\pgfsetroundcap%
\pgfsetroundjoin%
\pgfsetlinewidth{0.803000pt}%
\definecolor{currentstroke}{rgb}{1.000000,1.000000,1.000000}%
\pgfsetstrokecolor{currentstroke}%
\pgfsetdash{}{0pt}%
\pgfpathmoveto{\pgfqpoint{0.506467in}{1.567419in}}%
\pgfpathlineto{\pgfqpoint{6.706467in}{1.567419in}}%
\pgfusepath{stroke}%
\end{pgfscope}%
\begin{pgfscope}%
\definecolor{textcolor}{rgb}{0.150000,0.150000,0.150000}%
\pgfsetstrokecolor{textcolor}%
\pgfsetfillcolor{textcolor}%
\pgftext[x=0.100000in,y=1.514658in,left,base]{\color{textcolor}\rmfamily\fontsize{10.000000}{12.000000}\selectfont 1.75}%
\end{pgfscope}%
\begin{pgfscope}%
\pgfpathrectangle{\pgfqpoint{0.506467in}{0.331635in}}{\pgfqpoint{6.200000in}{2.265000in}}%
\pgfusepath{clip}%
\pgfsetroundcap%
\pgfsetroundjoin%
\pgfsetlinewidth{0.803000pt}%
\definecolor{currentstroke}{rgb}{1.000000,1.000000,1.000000}%
\pgfsetstrokecolor{currentstroke}%
\pgfsetdash{}{0pt}%
\pgfpathmoveto{\pgfqpoint{0.506467in}{1.898464in}}%
\pgfpathlineto{\pgfqpoint{6.706467in}{1.898464in}}%
\pgfusepath{stroke}%
\end{pgfscope}%
\begin{pgfscope}%
\definecolor{textcolor}{rgb}{0.150000,0.150000,0.150000}%
\pgfsetstrokecolor{textcolor}%
\pgfsetfillcolor{textcolor}%
\pgftext[x=0.100000in,y=1.845702in,left,base]{\color{textcolor}\rmfamily\fontsize{10.000000}{12.000000}\selectfont 2.00}%
\end{pgfscope}%
\begin{pgfscope}%
\pgfpathrectangle{\pgfqpoint{0.506467in}{0.331635in}}{\pgfqpoint{6.200000in}{2.265000in}}%
\pgfusepath{clip}%
\pgfsetroundcap%
\pgfsetroundjoin%
\pgfsetlinewidth{0.803000pt}%
\definecolor{currentstroke}{rgb}{1.000000,1.000000,1.000000}%
\pgfsetstrokecolor{currentstroke}%
\pgfsetdash{}{0pt}%
\pgfpathmoveto{\pgfqpoint{0.506467in}{2.229508in}}%
\pgfpathlineto{\pgfqpoint{6.706467in}{2.229508in}}%
\pgfusepath{stroke}%
\end{pgfscope}%
\begin{pgfscope}%
\definecolor{textcolor}{rgb}{0.150000,0.150000,0.150000}%
\pgfsetstrokecolor{textcolor}%
\pgfsetfillcolor{textcolor}%
\pgftext[x=0.100000in,y=2.176747in,left,base]{\color{textcolor}\rmfamily\fontsize{10.000000}{12.000000}\selectfont 2.25}%
\end{pgfscope}%
\begin{pgfscope}%
\pgfpathrectangle{\pgfqpoint{0.506467in}{0.331635in}}{\pgfqpoint{6.200000in}{2.265000in}}%
\pgfusepath{clip}%
\pgfsetroundcap%
\pgfsetroundjoin%
\pgfsetlinewidth{0.803000pt}%
\definecolor{currentstroke}{rgb}{1.000000,1.000000,1.000000}%
\pgfsetstrokecolor{currentstroke}%
\pgfsetdash{}{0pt}%
\pgfpathmoveto{\pgfqpoint{0.506467in}{2.560553in}}%
\pgfpathlineto{\pgfqpoint{6.706467in}{2.560553in}}%
\pgfusepath{stroke}%
\end{pgfscope}%
\begin{pgfscope}%
\definecolor{textcolor}{rgb}{0.150000,0.150000,0.150000}%
\pgfsetstrokecolor{textcolor}%
\pgfsetfillcolor{textcolor}%
\pgftext[x=0.100000in,y=2.507791in,left,base]{\color{textcolor}\rmfamily\fontsize{10.000000}{12.000000}\selectfont 2.50}%
\end{pgfscope}%
\begin{pgfscope}%
\pgfpathrectangle{\pgfqpoint{0.506467in}{0.331635in}}{\pgfqpoint{6.200000in}{2.265000in}}%
\pgfusepath{clip}%
\pgfsetroundcap%
\pgfsetroundjoin%
\pgfsetlinewidth{1.505625pt}%
\definecolor{currentstroke}{rgb}{0.121569,0.466667,0.705882}%
\pgfsetstrokecolor{currentstroke}%
\pgfsetdash{}{0pt}%
\pgfpathmoveto{\pgfqpoint{0.788285in}{0.574286in}}%
\pgfpathlineto{\pgfqpoint{0.793440in}{0.580332in}}%
\pgfpathlineto{\pgfqpoint{0.796017in}{0.575356in}}%
\pgfpathlineto{\pgfqpoint{0.803749in}{0.578843in}}%
\pgfpathlineto{\pgfqpoint{0.806326in}{0.583427in}}%
\pgfpathlineto{\pgfqpoint{0.808903in}{0.582453in}}%
\pgfpathlineto{\pgfqpoint{0.811480in}{0.589768in}}%
\pgfpathlineto{\pgfqpoint{0.814057in}{0.581286in}}%
\pgfpathlineto{\pgfqpoint{0.824366in}{0.587588in}}%
\pgfpathlineto{\pgfqpoint{0.826943in}{0.599090in}}%
\pgfpathlineto{\pgfqpoint{0.829521in}{0.602193in}}%
\pgfpathlineto{\pgfqpoint{0.832098in}{0.600206in}}%
\pgfpathlineto{\pgfqpoint{0.839830in}{0.592477in}}%
\pgfpathlineto{\pgfqpoint{0.842407in}{0.592994in}}%
\pgfpathlineto{\pgfqpoint{0.844984in}{0.599796in}}%
\pgfpathlineto{\pgfqpoint{0.847561in}{0.598436in}}%
\pgfpathlineto{\pgfqpoint{0.850138in}{0.596004in}}%
\pgfpathlineto{\pgfqpoint{0.857870in}{0.590095in}}%
\pgfpathlineto{\pgfqpoint{0.860447in}{0.591048in}}%
\pgfpathlineto{\pgfqpoint{0.863024in}{0.602088in}}%
\pgfpathlineto{\pgfqpoint{0.865602in}{0.605275in}}%
\pgfpathlineto{\pgfqpoint{0.868179in}{0.619682in}}%
\pgfpathlineto{\pgfqpoint{0.875910in}{0.621292in}}%
\pgfpathlineto{\pgfqpoint{0.878488in}{0.623766in}}%
\pgfpathlineto{\pgfqpoint{0.881065in}{0.629163in}}%
\pgfpathlineto{\pgfqpoint{0.883642in}{0.639947in}}%
\pgfpathlineto{\pgfqpoint{0.886219in}{0.633754in}}%
\pgfpathlineto{\pgfqpoint{0.893951in}{0.641744in}}%
\pgfpathlineto{\pgfqpoint{0.896528in}{0.643115in}}%
\pgfpathlineto{\pgfqpoint{0.899105in}{0.635567in}}%
\pgfpathlineto{\pgfqpoint{0.904260in}{0.655285in}}%
\pgfpathlineto{\pgfqpoint{0.914569in}{0.653133in}}%
\pgfpathlineto{\pgfqpoint{0.917146in}{0.651427in}}%
\pgfpathlineto{\pgfqpoint{0.922300in}{0.659901in}}%
\pgfpathlineto{\pgfqpoint{0.930032in}{0.661372in}}%
\pgfpathlineto{\pgfqpoint{0.932609in}{0.668657in}}%
\pgfpathlineto{\pgfqpoint{0.935186in}{0.661402in}}%
\pgfpathlineto{\pgfqpoint{0.937764in}{0.664856in}}%
\pgfpathlineto{\pgfqpoint{0.940341in}{0.662665in}}%
\pgfpathlineto{\pgfqpoint{0.948072in}{0.660182in}}%
\pgfpathlineto{\pgfqpoint{0.950650in}{0.640882in}}%
\pgfpathlineto{\pgfqpoint{0.953227in}{0.649494in}}%
\pgfpathlineto{\pgfqpoint{0.955804in}{0.663347in}}%
\pgfpathlineto{\pgfqpoint{0.966113in}{0.668621in}}%
\pgfpathlineto{\pgfqpoint{0.968690in}{0.692952in}}%
\pgfpathlineto{\pgfqpoint{0.971267in}{0.696795in}}%
\pgfpathlineto{\pgfqpoint{0.973845in}{0.704630in}}%
\pgfpathlineto{\pgfqpoint{0.976422in}{0.699237in}}%
\pgfpathlineto{\pgfqpoint{0.984153in}{0.703985in}}%
\pgfpathlineto{\pgfqpoint{0.986731in}{0.695034in}}%
\pgfpathlineto{\pgfqpoint{0.989308in}{0.694265in}}%
\pgfpathlineto{\pgfqpoint{0.991885in}{0.692614in}}%
\pgfpathlineto{\pgfqpoint{0.994462in}{0.693452in}}%
\pgfpathlineto{\pgfqpoint{1.002194in}{0.709642in}}%
\pgfpathlineto{\pgfqpoint{1.004771in}{0.704327in}}%
\pgfpathlineto{\pgfqpoint{1.007348in}{0.696531in}}%
\pgfpathlineto{\pgfqpoint{1.009926in}{0.693716in}}%
\pgfpathlineto{\pgfqpoint{1.012503in}{0.699408in}}%
\pgfpathlineto{\pgfqpoint{1.020234in}{0.704264in}}%
\pgfpathlineto{\pgfqpoint{1.022812in}{0.700145in}}%
\pgfpathlineto{\pgfqpoint{1.025389in}{0.689166in}}%
\pgfpathlineto{\pgfqpoint{1.027966in}{0.690501in}}%
\pgfpathlineto{\pgfqpoint{1.038275in}{0.672643in}}%
\pgfpathlineto{\pgfqpoint{1.040852in}{0.647125in}}%
\pgfpathlineto{\pgfqpoint{1.043429in}{0.660722in}}%
\pgfpathlineto{\pgfqpoint{1.046007in}{0.680981in}}%
\pgfpathlineto{\pgfqpoint{1.048584in}{0.668108in}}%
\pgfpathlineto{\pgfqpoint{1.056315in}{0.672850in}}%
\pgfpathlineto{\pgfqpoint{1.058893in}{0.688765in}}%
\pgfpathlineto{\pgfqpoint{1.061470in}{0.679254in}}%
\pgfpathlineto{\pgfqpoint{1.064047in}{0.674052in}}%
\pgfpathlineto{\pgfqpoint{1.066624in}{0.683518in}}%
\pgfpathlineto{\pgfqpoint{1.074356in}{0.669740in}}%
\pgfpathlineto{\pgfqpoint{1.079510in}{0.695744in}}%
\pgfpathlineto{\pgfqpoint{1.082087in}{0.710981in}}%
\pgfpathlineto{\pgfqpoint{1.084665in}{0.711430in}}%
\pgfpathlineto{\pgfqpoint{1.092396in}{0.707350in}}%
\pgfpathlineto{\pgfqpoint{1.094974in}{0.716831in}}%
\pgfpathlineto{\pgfqpoint{1.097551in}{0.717504in}}%
\pgfpathlineto{\pgfqpoint{1.100128in}{0.707752in}}%
\pgfpathlineto{\pgfqpoint{1.102705in}{0.692472in}}%
\pgfpathlineto{\pgfqpoint{1.110437in}{0.694315in}}%
\pgfpathlineto{\pgfqpoint{1.113014in}{0.691786in}}%
\pgfpathlineto{\pgfqpoint{1.115591in}{0.681215in}}%
\pgfpathlineto{\pgfqpoint{1.118168in}{0.687549in}}%
\pgfpathlineto{\pgfqpoint{1.120746in}{0.689914in}}%
\pgfpathlineto{\pgfqpoint{1.128477in}{0.671973in}}%
\pgfpathlineto{\pgfqpoint{1.131055in}{0.668413in}}%
\pgfpathlineto{\pgfqpoint{1.133632in}{0.671501in}}%
\pgfpathlineto{\pgfqpoint{1.136209in}{0.655626in}}%
\pgfpathlineto{\pgfqpoint{1.138786in}{0.646140in}}%
\pgfpathlineto{\pgfqpoint{1.146518in}{0.659943in}}%
\pgfpathlineto{\pgfqpoint{1.149095in}{0.661993in}}%
\pgfpathlineto{\pgfqpoint{1.151672in}{0.656948in}}%
\pgfpathlineto{\pgfqpoint{1.154249in}{0.662120in}}%
\pgfpathlineto{\pgfqpoint{1.156827in}{0.658003in}}%
\pgfpathlineto{\pgfqpoint{1.167135in}{0.674870in}}%
\pgfpathlineto{\pgfqpoint{1.169713in}{0.657756in}}%
\pgfpathlineto{\pgfqpoint{1.172290in}{0.658205in}}%
\pgfpathlineto{\pgfqpoint{1.174867in}{0.624860in}}%
\pgfpathlineto{\pgfqpoint{1.182599in}{0.624402in}}%
\pgfpathlineto{\pgfqpoint{1.185176in}{0.626821in}}%
\pgfpathlineto{\pgfqpoint{1.187753in}{0.657458in}}%
\pgfpathlineto{\pgfqpoint{1.190330in}{0.664708in}}%
\pgfpathlineto{\pgfqpoint{1.192908in}{0.676097in}}%
\pgfpathlineto{\pgfqpoint{1.200639in}{0.664514in}}%
\pgfpathlineto{\pgfqpoint{1.203216in}{0.682426in}}%
\pgfpathlineto{\pgfqpoint{1.205794in}{0.675378in}}%
\pgfpathlineto{\pgfqpoint{1.208371in}{0.693504in}}%
\pgfpathlineto{\pgfqpoint{1.210948in}{0.704769in}}%
\pgfpathlineto{\pgfqpoint{1.218680in}{0.705296in}}%
\pgfpathlineto{\pgfqpoint{1.221257in}{0.717305in}}%
\pgfpathlineto{\pgfqpoint{1.223834in}{0.715588in}}%
\pgfpathlineto{\pgfqpoint{1.226411in}{0.693884in}}%
\pgfpathlineto{\pgfqpoint{1.228989in}{0.709001in}}%
\pgfpathlineto{\pgfqpoint{1.236720in}{0.686438in}}%
\pgfpathlineto{\pgfqpoint{1.239297in}{0.693127in}}%
\pgfpathlineto{\pgfqpoint{1.241875in}{0.705976in}}%
\pgfpathlineto{\pgfqpoint{1.244452in}{0.699914in}}%
\pgfpathlineto{\pgfqpoint{1.247029in}{0.732698in}}%
\pgfpathlineto{\pgfqpoint{1.254761in}{0.737642in}}%
\pgfpathlineto{\pgfqpoint{1.257338in}{0.741682in}}%
\pgfpathlineto{\pgfqpoint{1.262492in}{0.736438in}}%
\pgfpathlineto{\pgfqpoint{1.265070in}{0.724108in}}%
\pgfpathlineto{\pgfqpoint{1.272801in}{0.724303in}}%
\pgfpathlineto{\pgfqpoint{1.275378in}{0.713171in}}%
\pgfpathlineto{\pgfqpoint{1.277956in}{0.705685in}}%
\pgfpathlineto{\pgfqpoint{1.280533in}{0.700658in}}%
\pgfpathlineto{\pgfqpoint{1.283110in}{0.725153in}}%
\pgfpathlineto{\pgfqpoint{1.290842in}{0.726752in}}%
\pgfpathlineto{\pgfqpoint{1.295996in}{0.752811in}}%
\pgfpathlineto{\pgfqpoint{1.301151in}{0.728958in}}%
\pgfpathlineto{\pgfqpoint{1.308882in}{0.720153in}}%
\pgfpathlineto{\pgfqpoint{1.311459in}{0.706946in}}%
\pgfpathlineto{\pgfqpoint{1.314037in}{0.711395in}}%
\pgfpathlineto{\pgfqpoint{1.316614in}{0.741826in}}%
\pgfpathlineto{\pgfqpoint{1.319191in}{0.761853in}}%
\pgfpathlineto{\pgfqpoint{1.326923in}{0.761216in}}%
\pgfpathlineto{\pgfqpoint{1.329500in}{0.752556in}}%
\pgfpathlineto{\pgfqpoint{1.332077in}{0.748675in}}%
\pgfpathlineto{\pgfqpoint{1.334654in}{0.741098in}}%
\pgfpathlineto{\pgfqpoint{1.337232in}{0.767757in}}%
\pgfpathlineto{\pgfqpoint{1.344963in}{0.766808in}}%
\pgfpathlineto{\pgfqpoint{1.347540in}{0.772527in}}%
\pgfpathlineto{\pgfqpoint{1.350118in}{0.774880in}}%
\pgfpathlineto{\pgfqpoint{1.352695in}{0.766938in}}%
\pgfpathlineto{\pgfqpoint{1.355272in}{0.770881in}}%
\pgfpathlineto{\pgfqpoint{1.363004in}{0.766760in}}%
\pgfpathlineto{\pgfqpoint{1.368158in}{0.769674in}}%
\pgfpathlineto{\pgfqpoint{1.370735in}{0.779629in}}%
\pgfpathlineto{\pgfqpoint{1.373312in}{0.780585in}}%
\pgfpathlineto{\pgfqpoint{1.381044in}{0.773883in}}%
\pgfpathlineto{\pgfqpoint{1.383621in}{0.763383in}}%
\pgfpathlineto{\pgfqpoint{1.386199in}{0.762021in}}%
\pgfpathlineto{\pgfqpoint{1.388776in}{0.749513in}}%
\pgfpathlineto{\pgfqpoint{1.391353in}{0.761488in}}%
\pgfpathlineto{\pgfqpoint{1.404239in}{0.761529in}}%
\pgfpathlineto{\pgfqpoint{1.406816in}{0.748986in}}%
\pgfpathlineto{\pgfqpoint{1.409393in}{0.762558in}}%
\pgfpathlineto{\pgfqpoint{1.422280in}{0.757972in}}%
\pgfpathlineto{\pgfqpoint{1.424857in}{0.782863in}}%
\pgfpathlineto{\pgfqpoint{1.435166in}{0.766277in}}%
\pgfpathlineto{\pgfqpoint{1.440320in}{0.776731in}}%
\pgfpathlineto{\pgfqpoint{1.442897in}{0.799366in}}%
\pgfpathlineto{\pgfqpoint{1.445474in}{0.801748in}}%
\pgfpathlineto{\pgfqpoint{1.455783in}{0.798023in}}%
\pgfpathlineto{\pgfqpoint{1.458361in}{0.803792in}}%
\pgfpathlineto{\pgfqpoint{1.460938in}{0.804428in}}%
\pgfpathlineto{\pgfqpoint{1.463515in}{0.803873in}}%
\pgfpathlineto{\pgfqpoint{1.471247in}{0.799335in}}%
\pgfpathlineto{\pgfqpoint{1.473824in}{0.791775in}}%
\pgfpathlineto{\pgfqpoint{1.476401in}{0.781987in}}%
\pgfpathlineto{\pgfqpoint{1.478978in}{0.794563in}}%
\pgfpathlineto{\pgfqpoint{1.481555in}{0.790443in}}%
\pgfpathlineto{\pgfqpoint{1.489287in}{0.799289in}}%
\pgfpathlineto{\pgfqpoint{1.491864in}{0.795024in}}%
\pgfpathlineto{\pgfqpoint{1.499596in}{0.820975in}}%
\pgfpathlineto{\pgfqpoint{1.507328in}{0.812627in}}%
\pgfpathlineto{\pgfqpoint{1.509905in}{0.791054in}}%
\pgfpathlineto{\pgfqpoint{1.512482in}{0.781433in}}%
\pgfpathlineto{\pgfqpoint{1.515059in}{0.778362in}}%
\pgfpathlineto{\pgfqpoint{1.517636in}{0.773740in}}%
\pgfpathlineto{\pgfqpoint{1.525368in}{0.780349in}}%
\pgfpathlineto{\pgfqpoint{1.530522in}{0.812153in}}%
\pgfpathlineto{\pgfqpoint{1.533100in}{0.813911in}}%
\pgfpathlineto{\pgfqpoint{1.535677in}{0.788987in}}%
\pgfpathlineto{\pgfqpoint{1.543409in}{0.783784in}}%
\pgfpathlineto{\pgfqpoint{1.545986in}{0.757841in}}%
\pgfpathlineto{\pgfqpoint{1.548563in}{0.758721in}}%
\pgfpathlineto{\pgfqpoint{1.551140in}{0.764041in}}%
\pgfpathlineto{\pgfqpoint{1.553717in}{0.767783in}}%
\pgfpathlineto{\pgfqpoint{1.566603in}{0.761671in}}%
\pgfpathlineto{\pgfqpoint{1.569181in}{0.785848in}}%
\pgfpathlineto{\pgfqpoint{1.571758in}{0.777689in}}%
\pgfpathlineto{\pgfqpoint{1.579489in}{0.775029in}}%
\pgfpathlineto{\pgfqpoint{1.582067in}{0.788426in}}%
\pgfpathlineto{\pgfqpoint{1.584644in}{0.758389in}}%
\pgfpathlineto{\pgfqpoint{1.587221in}{0.744921in}}%
\pgfpathlineto{\pgfqpoint{1.589798in}{0.738903in}}%
\pgfpathlineto{\pgfqpoint{1.597530in}{0.740806in}}%
\pgfpathlineto{\pgfqpoint{1.600107in}{0.733495in}}%
\pgfpathlineto{\pgfqpoint{1.602684in}{0.710988in}}%
\pgfpathlineto{\pgfqpoint{1.605262in}{0.711650in}}%
\pgfpathlineto{\pgfqpoint{1.615570in}{0.742960in}}%
\pgfpathlineto{\pgfqpoint{1.618148in}{0.743249in}}%
\pgfpathlineto{\pgfqpoint{1.620725in}{0.746521in}}%
\pgfpathlineto{\pgfqpoint{1.625879in}{0.767202in}}%
\pgfpathlineto{\pgfqpoint{1.633611in}{0.761627in}}%
\pgfpathlineto{\pgfqpoint{1.636188in}{0.753357in}}%
\pgfpathlineto{\pgfqpoint{1.638765in}{0.768767in}}%
\pgfpathlineto{\pgfqpoint{1.641343in}{0.769068in}}%
\pgfpathlineto{\pgfqpoint{1.643920in}{0.774089in}}%
\pgfpathlineto{\pgfqpoint{1.651651in}{0.767254in}}%
\pgfpathlineto{\pgfqpoint{1.654229in}{0.768100in}}%
\pgfpathlineto{\pgfqpoint{1.659383in}{0.783794in}}%
\pgfpathlineto{\pgfqpoint{1.661960in}{0.789120in}}%
\pgfpathlineto{\pgfqpoint{1.669692in}{0.788263in}}%
\pgfpathlineto{\pgfqpoint{1.672269in}{0.801884in}}%
\pgfpathlineto{\pgfqpoint{1.674846in}{0.805034in}}%
\pgfpathlineto{\pgfqpoint{1.677424in}{0.795219in}}%
\pgfpathlineto{\pgfqpoint{1.680001in}{0.788179in}}%
\pgfpathlineto{\pgfqpoint{1.687732in}{0.798204in}}%
\pgfpathlineto{\pgfqpoint{1.690310in}{0.810488in}}%
\pgfpathlineto{\pgfqpoint{1.692887in}{0.797587in}}%
\pgfpathlineto{\pgfqpoint{1.695464in}{0.811891in}}%
\pgfpathlineto{\pgfqpoint{1.698041in}{0.797553in}}%
\pgfpathlineto{\pgfqpoint{1.705773in}{0.794335in}}%
\pgfpathlineto{\pgfqpoint{1.710927in}{0.789808in}}%
\pgfpathlineto{\pgfqpoint{1.713505in}{0.784690in}}%
\pgfpathlineto{\pgfqpoint{1.716082in}{0.769727in}}%
\pgfpathlineto{\pgfqpoint{1.723813in}{0.792325in}}%
\pgfpathlineto{\pgfqpoint{1.728968in}{0.827985in}}%
\pgfpathlineto{\pgfqpoint{1.731545in}{0.825292in}}%
\pgfpathlineto{\pgfqpoint{1.734122in}{0.836209in}}%
\pgfpathlineto{\pgfqpoint{1.741854in}{0.833873in}}%
\pgfpathlineto{\pgfqpoint{1.744431in}{0.826698in}}%
\pgfpathlineto{\pgfqpoint{1.749586in}{0.847076in}}%
\pgfpathlineto{\pgfqpoint{1.752163in}{0.847921in}}%
\pgfpathlineto{\pgfqpoint{1.759894in}{0.848576in}}%
\pgfpathlineto{\pgfqpoint{1.762472in}{0.847030in}}%
\pgfpathlineto{\pgfqpoint{1.765049in}{0.847383in}}%
\pgfpathlineto{\pgfqpoint{1.767626in}{0.861878in}}%
\pgfpathlineto{\pgfqpoint{1.770203in}{0.857873in}}%
\pgfpathlineto{\pgfqpoint{1.780512in}{0.860290in}}%
\pgfpathlineto{\pgfqpoint{1.783089in}{0.864700in}}%
\pgfpathlineto{\pgfqpoint{1.785666in}{0.866746in}}%
\pgfpathlineto{\pgfqpoint{1.788244in}{0.882487in}}%
\pgfpathlineto{\pgfqpoint{1.795975in}{0.880855in}}%
\pgfpathlineto{\pgfqpoint{1.798553in}{0.891949in}}%
\pgfpathlineto{\pgfqpoint{1.801130in}{0.885009in}}%
\pgfpathlineto{\pgfqpoint{1.803707in}{0.881337in}}%
\pgfpathlineto{\pgfqpoint{1.806284in}{0.903858in}}%
\pgfpathlineto{\pgfqpoint{1.814016in}{0.891058in}}%
\pgfpathlineto{\pgfqpoint{1.816593in}{0.905289in}}%
\pgfpathlineto{\pgfqpoint{1.819170in}{0.910249in}}%
\pgfpathlineto{\pgfqpoint{1.821747in}{0.907695in}}%
\pgfpathlineto{\pgfqpoint{1.824325in}{0.910682in}}%
\pgfpathlineto{\pgfqpoint{1.832056in}{0.908753in}}%
\pgfpathlineto{\pgfqpoint{1.837211in}{0.923200in}}%
\pgfpathlineto{\pgfqpoint{1.839788in}{0.925139in}}%
\pgfpathlineto{\pgfqpoint{1.842365in}{0.929424in}}%
\pgfpathlineto{\pgfqpoint{1.852674in}{0.940125in}}%
\pgfpathlineto{\pgfqpoint{1.855251in}{0.924933in}}%
\pgfpathlineto{\pgfqpoint{1.857828in}{0.920226in}}%
\pgfpathlineto{\pgfqpoint{1.860406in}{0.931984in}}%
\pgfpathlineto{\pgfqpoint{1.868137in}{0.909495in}}%
\pgfpathlineto{\pgfqpoint{1.870714in}{0.921626in}}%
\pgfpathlineto{\pgfqpoint{1.873292in}{0.940393in}}%
\pgfpathlineto{\pgfqpoint{1.875869in}{0.936541in}}%
\pgfpathlineto{\pgfqpoint{1.878446in}{0.941516in}}%
\pgfpathlineto{\pgfqpoint{1.886178in}{0.948643in}}%
\pgfpathlineto{\pgfqpoint{1.888755in}{0.968466in}}%
\pgfpathlineto{\pgfqpoint{1.891332in}{0.971642in}}%
\pgfpathlineto{\pgfqpoint{1.893909in}{0.973709in}}%
\pgfpathlineto{\pgfqpoint{1.896487in}{0.981758in}}%
\pgfpathlineto{\pgfqpoint{1.904218in}{0.987167in}}%
\pgfpathlineto{\pgfqpoint{1.906795in}{0.982629in}}%
\pgfpathlineto{\pgfqpoint{1.909373in}{0.981273in}}%
\pgfpathlineto{\pgfqpoint{1.911950in}{0.992074in}}%
\pgfpathlineto{\pgfqpoint{1.914527in}{0.983786in}}%
\pgfpathlineto{\pgfqpoint{1.924836in}{0.973758in}}%
\pgfpathlineto{\pgfqpoint{1.927413in}{0.985506in}}%
\pgfpathlineto{\pgfqpoint{1.929990in}{0.973947in}}%
\pgfpathlineto{\pgfqpoint{1.932568in}{0.989945in}}%
\pgfpathlineto{\pgfqpoint{1.940299in}{0.985378in}}%
\pgfpathlineto{\pgfqpoint{1.942876in}{1.003757in}}%
\pgfpathlineto{\pgfqpoint{1.945454in}{1.001633in}}%
\pgfpathlineto{\pgfqpoint{1.948031in}{1.009400in}}%
\pgfpathlineto{\pgfqpoint{1.958340in}{1.002952in}}%
\pgfpathlineto{\pgfqpoint{1.960917in}{1.016242in}}%
\pgfpathlineto{\pgfqpoint{1.963494in}{0.997103in}}%
\pgfpathlineto{\pgfqpoint{1.966071in}{1.005550in}}%
\pgfpathlineto{\pgfqpoint{1.968649in}{0.996327in}}%
\pgfpathlineto{\pgfqpoint{1.976380in}{1.007022in}}%
\pgfpathlineto{\pgfqpoint{1.978957in}{1.011573in}}%
\pgfpathlineto{\pgfqpoint{1.981535in}{1.034774in}}%
\pgfpathlineto{\pgfqpoint{1.984112in}{1.040365in}}%
\pgfpathlineto{\pgfqpoint{1.986689in}{1.037849in}}%
\pgfpathlineto{\pgfqpoint{1.994421in}{1.005223in}}%
\pgfpathlineto{\pgfqpoint{1.996998in}{1.029118in}}%
\pgfpathlineto{\pgfqpoint{1.999575in}{1.013593in}}%
\pgfpathlineto{\pgfqpoint{2.002152in}{1.016788in}}%
\pgfpathlineto{\pgfqpoint{2.004730in}{1.039988in}}%
\pgfpathlineto{\pgfqpoint{2.012461in}{1.040850in}}%
\pgfpathlineto{\pgfqpoint{2.015038in}{1.057731in}}%
\pgfpathlineto{\pgfqpoint{2.017616in}{1.047672in}}%
\pgfpathlineto{\pgfqpoint{2.020193in}{1.050105in}}%
\pgfpathlineto{\pgfqpoint{2.022770in}{1.048083in}}%
\pgfpathlineto{\pgfqpoint{2.033079in}{1.061061in}}%
\pgfpathlineto{\pgfqpoint{2.035656in}{1.050932in}}%
\pgfpathlineto{\pgfqpoint{2.040811in}{1.096764in}}%
\pgfpathlineto{\pgfqpoint{2.048542in}{1.091331in}}%
\pgfpathlineto{\pgfqpoint{2.051119in}{1.104683in}}%
\pgfpathlineto{\pgfqpoint{2.053697in}{1.110521in}}%
\pgfpathlineto{\pgfqpoint{2.056274in}{1.111622in}}%
\pgfpathlineto{\pgfqpoint{2.058851in}{1.120185in}}%
\pgfpathlineto{\pgfqpoint{2.066583in}{1.114316in}}%
\pgfpathlineto{\pgfqpoint{2.069160in}{1.129067in}}%
\pgfpathlineto{\pgfqpoint{2.071737in}{1.148744in}}%
\pgfpathlineto{\pgfqpoint{2.074314in}{1.135055in}}%
\pgfpathlineto{\pgfqpoint{2.076891in}{1.151833in}}%
\pgfpathlineto{\pgfqpoint{2.084623in}{1.147369in}}%
\pgfpathlineto{\pgfqpoint{2.087200in}{1.148840in}}%
\pgfpathlineto{\pgfqpoint{2.089778in}{1.140020in}}%
\pgfpathlineto{\pgfqpoint{2.092355in}{1.133530in}}%
\pgfpathlineto{\pgfqpoint{2.094932in}{1.140041in}}%
\pgfpathlineto{\pgfqpoint{2.105241in}{1.148490in}}%
\pgfpathlineto{\pgfqpoint{2.107818in}{1.130276in}}%
\pgfpathlineto{\pgfqpoint{2.110395in}{1.130238in}}%
\pgfpathlineto{\pgfqpoint{2.112972in}{1.104066in}}%
\pgfpathlineto{\pgfqpoint{2.120704in}{1.125309in}}%
\pgfpathlineto{\pgfqpoint{2.123281in}{1.122779in}}%
\pgfpathlineto{\pgfqpoint{2.125859in}{1.095581in}}%
\pgfpathlineto{\pgfqpoint{2.128436in}{1.109788in}}%
\pgfpathlineto{\pgfqpoint{2.131013in}{1.135139in}}%
\pgfpathlineto{\pgfqpoint{2.138745in}{1.138950in}}%
\pgfpathlineto{\pgfqpoint{2.141322in}{1.123932in}}%
\pgfpathlineto{\pgfqpoint{2.143899in}{1.105567in}}%
\pgfpathlineto{\pgfqpoint{2.146476in}{1.133512in}}%
\pgfpathlineto{\pgfqpoint{2.149053in}{1.123257in}}%
\pgfpathlineto{\pgfqpoint{2.156785in}{1.138207in}}%
\pgfpathlineto{\pgfqpoint{2.159362in}{1.161514in}}%
\pgfpathlineto{\pgfqpoint{2.161940in}{1.130323in}}%
\pgfpathlineto{\pgfqpoint{2.164517in}{1.082329in}}%
\pgfpathlineto{\pgfqpoint{2.167094in}{1.097340in}}%
\pgfpathlineto{\pgfqpoint{2.174826in}{1.078470in}}%
\pgfpathlineto{\pgfqpoint{2.179980in}{1.115399in}}%
\pgfpathlineto{\pgfqpoint{2.182557in}{1.125824in}}%
\pgfpathlineto{\pgfqpoint{2.185134in}{1.113562in}}%
\pgfpathlineto{\pgfqpoint{2.192866in}{1.126902in}}%
\pgfpathlineto{\pgfqpoint{2.195443in}{1.117237in}}%
\pgfpathlineto{\pgfqpoint{2.198020in}{1.125489in}}%
\pgfpathlineto{\pgfqpoint{2.203175in}{1.149353in}}%
\pgfpathlineto{\pgfqpoint{2.210907in}{1.150917in}}%
\pgfpathlineto{\pgfqpoint{2.213484in}{1.160069in}}%
\pgfpathlineto{\pgfqpoint{2.216061in}{1.155909in}}%
\pgfpathlineto{\pgfqpoint{2.218638in}{1.186281in}}%
\pgfpathlineto{\pgfqpoint{2.221215in}{1.190456in}}%
\pgfpathlineto{\pgfqpoint{2.228947in}{1.184295in}}%
\pgfpathlineto{\pgfqpoint{2.231524in}{1.181339in}}%
\pgfpathlineto{\pgfqpoint{2.234101in}{1.181759in}}%
\pgfpathlineto{\pgfqpoint{2.236679in}{1.170913in}}%
\pgfpathlineto{\pgfqpoint{2.239256in}{1.185479in}}%
\pgfpathlineto{\pgfqpoint{2.246988in}{1.185581in}}%
\pgfpathlineto{\pgfqpoint{2.249565in}{1.187312in}}%
\pgfpathlineto{\pgfqpoint{2.252142in}{1.186759in}}%
\pgfpathlineto{\pgfqpoint{2.254719in}{1.199369in}}%
\pgfpathlineto{\pgfqpoint{2.257296in}{1.202195in}}%
\pgfpathlineto{\pgfqpoint{2.265028in}{1.199133in}}%
\pgfpathlineto{\pgfqpoint{2.267605in}{1.196891in}}%
\pgfpathlineto{\pgfqpoint{2.270182in}{1.176323in}}%
\pgfpathlineto{\pgfqpoint{2.272760in}{1.196926in}}%
\pgfpathlineto{\pgfqpoint{2.275337in}{1.208691in}}%
\pgfpathlineto{\pgfqpoint{2.283068in}{1.202732in}}%
\pgfpathlineto{\pgfqpoint{2.288223in}{1.192486in}}%
\pgfpathlineto{\pgfqpoint{2.290800in}{1.192233in}}%
\pgfpathlineto{\pgfqpoint{2.293377in}{1.180010in}}%
\pgfpathlineto{\pgfqpoint{2.301109in}{1.179338in}}%
\pgfpathlineto{\pgfqpoint{2.303686in}{1.182309in}}%
\pgfpathlineto{\pgfqpoint{2.306263in}{1.170536in}}%
\pgfpathlineto{\pgfqpoint{2.308841in}{1.141087in}}%
\pgfpathlineto{\pgfqpoint{2.311418in}{1.134095in}}%
\pgfpathlineto{\pgfqpoint{2.319149in}{1.134830in}}%
\pgfpathlineto{\pgfqpoint{2.321727in}{1.133093in}}%
\pgfpathlineto{\pgfqpoint{2.324304in}{1.122323in}}%
\pgfpathlineto{\pgfqpoint{2.326881in}{1.130669in}}%
\pgfpathlineto{\pgfqpoint{2.329458in}{1.136095in}}%
\pgfpathlineto{\pgfqpoint{2.337190in}{1.118779in}}%
\pgfpathlineto{\pgfqpoint{2.339767in}{1.096333in}}%
\pgfpathlineto{\pgfqpoint{2.342344in}{1.096066in}}%
\pgfpathlineto{\pgfqpoint{2.344922in}{1.102613in}}%
\pgfpathlineto{\pgfqpoint{2.347499in}{1.098740in}}%
\pgfpathlineto{\pgfqpoint{2.357808in}{1.101739in}}%
\pgfpathlineto{\pgfqpoint{2.360385in}{1.115747in}}%
\pgfpathlineto{\pgfqpoint{2.362962in}{1.117714in}}%
\pgfpathlineto{\pgfqpoint{2.365539in}{1.116453in}}%
\pgfpathlineto{\pgfqpoint{2.373271in}{1.130226in}}%
\pgfpathlineto{\pgfqpoint{2.375848in}{1.155642in}}%
\pgfpathlineto{\pgfqpoint{2.378425in}{1.169857in}}%
\pgfpathlineto{\pgfqpoint{2.381003in}{1.172143in}}%
\pgfpathlineto{\pgfqpoint{2.383580in}{1.190326in}}%
\pgfpathlineto{\pgfqpoint{2.391311in}{1.204043in}}%
\pgfpathlineto{\pgfqpoint{2.396466in}{1.233967in}}%
\pgfpathlineto{\pgfqpoint{2.399043in}{1.230407in}}%
\pgfpathlineto{\pgfqpoint{2.401620in}{1.214544in}}%
\pgfpathlineto{\pgfqpoint{2.409352in}{1.210830in}}%
\pgfpathlineto{\pgfqpoint{2.414506in}{1.189948in}}%
\pgfpathlineto{\pgfqpoint{2.417084in}{1.198180in}}%
\pgfpathlineto{\pgfqpoint{2.419661in}{1.184465in}}%
\pgfpathlineto{\pgfqpoint{2.427392in}{1.168250in}}%
\pgfpathlineto{\pgfqpoint{2.429970in}{1.177854in}}%
\pgfpathlineto{\pgfqpoint{2.432547in}{1.168787in}}%
\pgfpathlineto{\pgfqpoint{2.435124in}{1.153103in}}%
\pgfpathlineto{\pgfqpoint{2.437701in}{1.165733in}}%
\pgfpathlineto{\pgfqpoint{2.445433in}{1.151872in}}%
\pgfpathlineto{\pgfqpoint{2.448010in}{1.131290in}}%
\pgfpathlineto{\pgfqpoint{2.450587in}{1.135223in}}%
\pgfpathlineto{\pgfqpoint{2.453165in}{1.179947in}}%
\pgfpathlineto{\pgfqpoint{2.455742in}{1.199759in}}%
\pgfpathlineto{\pgfqpoint{2.463473in}{1.207210in}}%
\pgfpathlineto{\pgfqpoint{2.466051in}{1.190582in}}%
\pgfpathlineto{\pgfqpoint{2.468628in}{1.214182in}}%
\pgfpathlineto{\pgfqpoint{2.471205in}{1.248678in}}%
\pgfpathlineto{\pgfqpoint{2.473782in}{1.264161in}}%
\pgfpathlineto{\pgfqpoint{2.481514in}{1.272233in}}%
\pgfpathlineto{\pgfqpoint{2.484091in}{1.282575in}}%
\pgfpathlineto{\pgfqpoint{2.486668in}{1.272019in}}%
\pgfpathlineto{\pgfqpoint{2.491823in}{1.291681in}}%
\pgfpathlineto{\pgfqpoint{2.499554in}{1.295278in}}%
\pgfpathlineto{\pgfqpoint{2.502132in}{1.308824in}}%
\pgfpathlineto{\pgfqpoint{2.504709in}{1.299722in}}%
\pgfpathlineto{\pgfqpoint{2.507286in}{1.287481in}}%
\pgfpathlineto{\pgfqpoint{2.509863in}{1.299254in}}%
\pgfpathlineto{\pgfqpoint{2.517595in}{1.297478in}}%
\pgfpathlineto{\pgfqpoint{2.520172in}{1.293408in}}%
\pgfpathlineto{\pgfqpoint{2.522749in}{1.311312in}}%
\pgfpathlineto{\pgfqpoint{2.525326in}{1.288237in}}%
\pgfpathlineto{\pgfqpoint{2.527904in}{1.309254in}}%
\pgfpathlineto{\pgfqpoint{2.538213in}{1.305816in}}%
\pgfpathlineto{\pgfqpoint{2.543367in}{1.328514in}}%
\pgfpathlineto{\pgfqpoint{2.545944in}{1.337808in}}%
\pgfpathlineto{\pgfqpoint{2.553676in}{1.336204in}}%
\pgfpathlineto{\pgfqpoint{2.556253in}{1.332722in}}%
\pgfpathlineto{\pgfqpoint{2.558830in}{1.330684in}}%
\pgfpathlineto{\pgfqpoint{2.561407in}{1.349698in}}%
\pgfpathlineto{\pgfqpoint{2.563985in}{1.344230in}}%
\pgfpathlineto{\pgfqpoint{2.571716in}{1.342421in}}%
\pgfpathlineto{\pgfqpoint{2.574293in}{1.348853in}}%
\pgfpathlineto{\pgfqpoint{2.576871in}{1.353601in}}%
\pgfpathlineto{\pgfqpoint{2.582025in}{1.347708in}}%
\pgfpathlineto{\pgfqpoint{2.589757in}{1.333265in}}%
\pgfpathlineto{\pgfqpoint{2.592334in}{1.322127in}}%
\pgfpathlineto{\pgfqpoint{2.597488in}{1.319558in}}%
\pgfpathlineto{\pgfqpoint{2.600066in}{1.352387in}}%
\pgfpathlineto{\pgfqpoint{2.607797in}{1.354585in}}%
\pgfpathlineto{\pgfqpoint{2.610374in}{1.343544in}}%
\pgfpathlineto{\pgfqpoint{2.615529in}{1.311235in}}%
\pgfpathlineto{\pgfqpoint{2.618106in}{1.314666in}}%
\pgfpathlineto{\pgfqpoint{2.628415in}{1.331371in}}%
\pgfpathlineto{\pgfqpoint{2.630992in}{1.373172in}}%
\pgfpathlineto{\pgfqpoint{2.636147in}{1.377615in}}%
\pgfpathlineto{\pgfqpoint{2.643878in}{1.387642in}}%
\pgfpathlineto{\pgfqpoint{2.646455in}{1.396895in}}%
\pgfpathlineto{\pgfqpoint{2.651610in}{1.412991in}}%
\pgfpathlineto{\pgfqpoint{2.654187in}{1.413276in}}%
\pgfpathlineto{\pgfqpoint{2.661919in}{1.423302in}}%
\pgfpathlineto{\pgfqpoint{2.664496in}{1.430356in}}%
\pgfpathlineto{\pgfqpoint{2.669650in}{1.409549in}}%
\pgfpathlineto{\pgfqpoint{2.672228in}{1.410217in}}%
\pgfpathlineto{\pgfqpoint{2.679959in}{1.404941in}}%
\pgfpathlineto{\pgfqpoint{2.682536in}{1.419299in}}%
\pgfpathlineto{\pgfqpoint{2.685114in}{1.409538in}}%
\pgfpathlineto{\pgfqpoint{2.687691in}{1.403329in}}%
\pgfpathlineto{\pgfqpoint{2.690268in}{1.403619in}}%
\pgfpathlineto{\pgfqpoint{2.698000in}{1.380645in}}%
\pgfpathlineto{\pgfqpoint{2.700577in}{1.405378in}}%
\pgfpathlineto{\pgfqpoint{2.703154in}{1.422840in}}%
\pgfpathlineto{\pgfqpoint{2.705731in}{1.417326in}}%
\pgfpathlineto{\pgfqpoint{2.708309in}{1.421044in}}%
\pgfpathlineto{\pgfqpoint{2.718617in}{1.412591in}}%
\pgfpathlineto{\pgfqpoint{2.721195in}{1.412681in}}%
\pgfpathlineto{\pgfqpoint{2.723772in}{1.389981in}}%
\pgfpathlineto{\pgfqpoint{2.726349in}{1.346106in}}%
\pgfpathlineto{\pgfqpoint{2.734081in}{1.335906in}}%
\pgfpathlineto{\pgfqpoint{2.736658in}{1.351310in}}%
\pgfpathlineto{\pgfqpoint{2.739235in}{1.331786in}}%
\pgfpathlineto{\pgfqpoint{2.741812in}{1.343724in}}%
\pgfpathlineto{\pgfqpoint{2.752121in}{1.272492in}}%
\pgfpathlineto{\pgfqpoint{2.754698in}{1.285195in}}%
\pgfpathlineto{\pgfqpoint{2.757276in}{1.287332in}}%
\pgfpathlineto{\pgfqpoint{2.759853in}{1.323629in}}%
\pgfpathlineto{\pgfqpoint{2.762430in}{1.343095in}}%
\pgfpathlineto{\pgfqpoint{2.770162in}{1.353916in}}%
\pgfpathlineto{\pgfqpoint{2.772739in}{1.375510in}}%
\pgfpathlineto{\pgfqpoint{2.777893in}{1.380370in}}%
\pgfpathlineto{\pgfqpoint{2.780470in}{1.394695in}}%
\pgfpathlineto{\pgfqpoint{2.790779in}{1.386579in}}%
\pgfpathlineto{\pgfqpoint{2.793357in}{1.378209in}}%
\pgfpathlineto{\pgfqpoint{2.795934in}{1.392510in}}%
\pgfpathlineto{\pgfqpoint{2.798511in}{1.386565in}}%
\pgfpathlineto{\pgfqpoint{2.806243in}{1.395559in}}%
\pgfpathlineto{\pgfqpoint{2.811397in}{1.395824in}}%
\pgfpathlineto{\pgfqpoint{2.813974in}{1.409849in}}%
\pgfpathlineto{\pgfqpoint{2.816551in}{1.417434in}}%
\pgfpathlineto{\pgfqpoint{2.824283in}{1.390891in}}%
\pgfpathlineto{\pgfqpoint{2.826860in}{1.424369in}}%
\pgfpathlineto{\pgfqpoint{2.829438in}{1.420420in}}%
\pgfpathlineto{\pgfqpoint{2.832015in}{1.432760in}}%
\pgfpathlineto{\pgfqpoint{2.834592in}{1.433166in}}%
\pgfpathlineto{\pgfqpoint{2.842324in}{1.431323in}}%
\pgfpathlineto{\pgfqpoint{2.844901in}{1.421231in}}%
\pgfpathlineto{\pgfqpoint{2.847478in}{1.419430in}}%
\pgfpathlineto{\pgfqpoint{2.850055in}{1.388638in}}%
\pgfpathlineto{\pgfqpoint{2.852632in}{1.382388in}}%
\pgfpathlineto{\pgfqpoint{2.862941in}{1.419942in}}%
\pgfpathlineto{\pgfqpoint{2.865519in}{1.401334in}}%
\pgfpathlineto{\pgfqpoint{2.868096in}{1.411340in}}%
\pgfpathlineto{\pgfqpoint{2.870673in}{1.414260in}}%
\pgfpathlineto{\pgfqpoint{2.878405in}{1.407977in}}%
\pgfpathlineto{\pgfqpoint{2.880982in}{1.423555in}}%
\pgfpathlineto{\pgfqpoint{2.883559in}{1.408792in}}%
\pgfpathlineto{\pgfqpoint{2.886136in}{1.412526in}}%
\pgfpathlineto{\pgfqpoint{2.888713in}{1.418126in}}%
\pgfpathlineto{\pgfqpoint{2.896445in}{1.436556in}}%
\pgfpathlineto{\pgfqpoint{2.899022in}{1.446423in}}%
\pgfpathlineto{\pgfqpoint{2.901599in}{1.449197in}}%
\pgfpathlineto{\pgfqpoint{2.904177in}{1.458406in}}%
\pgfpathlineto{\pgfqpoint{2.906754in}{1.434935in}}%
\pgfpathlineto{\pgfqpoint{2.914486in}{1.417064in}}%
\pgfpathlineto{\pgfqpoint{2.917063in}{1.425175in}}%
\pgfpathlineto{\pgfqpoint{2.919640in}{1.446134in}}%
\pgfpathlineto{\pgfqpoint{2.922217in}{1.399330in}}%
\pgfpathlineto{\pgfqpoint{2.924794in}{1.381726in}}%
\pgfpathlineto{\pgfqpoint{2.932526in}{1.400627in}}%
\pgfpathlineto{\pgfqpoint{2.935103in}{1.414293in}}%
\pgfpathlineto{\pgfqpoint{2.937680in}{1.441332in}}%
\pgfpathlineto{\pgfqpoint{2.940258in}{1.449800in}}%
\pgfpathlineto{\pgfqpoint{2.950567in}{1.453765in}}%
\pgfpathlineto{\pgfqpoint{2.953144in}{1.457040in}}%
\pgfpathlineto{\pgfqpoint{2.958298in}{1.445375in}}%
\pgfpathlineto{\pgfqpoint{2.960875in}{1.422192in}}%
\pgfpathlineto{\pgfqpoint{2.971184in}{1.444120in}}%
\pgfpathlineto{\pgfqpoint{2.973761in}{1.452437in}}%
\pgfpathlineto{\pgfqpoint{2.976339in}{1.452297in}}%
\pgfpathlineto{\pgfqpoint{2.978916in}{1.445862in}}%
\pgfpathlineto{\pgfqpoint{2.986647in}{1.455022in}}%
\pgfpathlineto{\pgfqpoint{2.989225in}{1.440746in}}%
\pgfpathlineto{\pgfqpoint{2.991802in}{1.464389in}}%
\pgfpathlineto{\pgfqpoint{2.994379in}{1.471679in}}%
\pgfpathlineto{\pgfqpoint{3.004688in}{1.484678in}}%
\pgfpathlineto{\pgfqpoint{3.007265in}{1.483019in}}%
\pgfpathlineto{\pgfqpoint{3.009842in}{1.471642in}}%
\pgfpathlineto{\pgfqpoint{3.012420in}{1.454332in}}%
\pgfpathlineto{\pgfqpoint{3.014997in}{1.458567in}}%
\pgfpathlineto{\pgfqpoint{3.022728in}{1.464952in}}%
\pgfpathlineto{\pgfqpoint{3.025306in}{1.449216in}}%
\pgfpathlineto{\pgfqpoint{3.027883in}{1.469717in}}%
\pgfpathlineto{\pgfqpoint{3.030460in}{1.472527in}}%
\pgfpathlineto{\pgfqpoint{3.033037in}{1.484065in}}%
\pgfpathlineto{\pgfqpoint{3.043346in}{1.497286in}}%
\pgfpathlineto{\pgfqpoint{3.045923in}{1.497363in}}%
\pgfpathlineto{\pgfqpoint{3.051078in}{1.513150in}}%
\pgfpathlineto{\pgfqpoint{3.058809in}{1.515974in}}%
\pgfpathlineto{\pgfqpoint{3.063964in}{1.509069in}}%
\pgfpathlineto{\pgfqpoint{3.069118in}{1.539368in}}%
\pgfpathlineto{\pgfqpoint{3.076850in}{1.547235in}}%
\pgfpathlineto{\pgfqpoint{3.079427in}{1.548262in}}%
\pgfpathlineto{\pgfqpoint{3.082004in}{1.535148in}}%
\pgfpathlineto{\pgfqpoint{3.084582in}{1.518892in}}%
\pgfpathlineto{\pgfqpoint{3.087159in}{1.535365in}}%
\pgfpathlineto{\pgfqpoint{3.097468in}{1.534227in}}%
\pgfpathlineto{\pgfqpoint{3.100045in}{1.542985in}}%
\pgfpathlineto{\pgfqpoint{3.102622in}{1.549470in}}%
\pgfpathlineto{\pgfqpoint{3.105199in}{1.552210in}}%
\pgfpathlineto{\pgfqpoint{3.112931in}{1.545512in}}%
\pgfpathlineto{\pgfqpoint{3.115508in}{1.533885in}}%
\pgfpathlineto{\pgfqpoint{3.118085in}{1.546328in}}%
\pgfpathlineto{\pgfqpoint{3.120663in}{1.541249in}}%
\pgfpathlineto{\pgfqpoint{3.123240in}{1.549610in}}%
\pgfpathlineto{\pgfqpoint{3.130971in}{1.543755in}}%
\pgfpathlineto{\pgfqpoint{3.133549in}{1.562310in}}%
\pgfpathlineto{\pgfqpoint{3.136126in}{1.566012in}}%
\pgfpathlineto{\pgfqpoint{3.138703in}{1.575446in}}%
\pgfpathlineto{\pgfqpoint{3.149012in}{1.573537in}}%
\pgfpathlineto{\pgfqpoint{3.151589in}{1.557564in}}%
\pgfpathlineto{\pgfqpoint{3.154166in}{1.567730in}}%
\pgfpathlineto{\pgfqpoint{3.156744in}{1.568069in}}%
\pgfpathlineto{\pgfqpoint{3.159321in}{1.574955in}}%
\pgfpathlineto{\pgfqpoint{3.167052in}{1.586607in}}%
\pgfpathlineto{\pgfqpoint{3.169630in}{1.582686in}}%
\pgfpathlineto{\pgfqpoint{3.172207in}{1.607284in}}%
\pgfpathlineto{\pgfqpoint{3.174784in}{1.571228in}}%
\pgfpathlineto{\pgfqpoint{3.177361in}{1.585424in}}%
\pgfpathlineto{\pgfqpoint{3.185093in}{1.576557in}}%
\pgfpathlineto{\pgfqpoint{3.187670in}{1.587029in}}%
\pgfpathlineto{\pgfqpoint{3.190247in}{1.578966in}}%
\pgfpathlineto{\pgfqpoint{3.192824in}{1.581296in}}%
\pgfpathlineto{\pgfqpoint{3.195402in}{1.565735in}}%
\pgfpathlineto{\pgfqpoint{3.203133in}{1.567013in}}%
\pgfpathlineto{\pgfqpoint{3.205711in}{1.555317in}}%
\pgfpathlineto{\pgfqpoint{3.208288in}{1.555382in}}%
\pgfpathlineto{\pgfqpoint{3.210865in}{1.513025in}}%
\pgfpathlineto{\pgfqpoint{3.213442in}{1.511068in}}%
\pgfpathlineto{\pgfqpoint{3.221174in}{1.521784in}}%
\pgfpathlineto{\pgfqpoint{3.223751in}{1.503948in}}%
\pgfpathlineto{\pgfqpoint{3.226328in}{1.509055in}}%
\pgfpathlineto{\pgfqpoint{3.228905in}{1.496376in}}%
\pgfpathlineto{\pgfqpoint{3.231483in}{1.517884in}}%
\pgfpathlineto{\pgfqpoint{3.239214in}{1.522907in}}%
\pgfpathlineto{\pgfqpoint{3.241792in}{1.519527in}}%
\pgfpathlineto{\pgfqpoint{3.244369in}{1.540184in}}%
\pgfpathlineto{\pgfqpoint{3.246946in}{1.546770in}}%
\pgfpathlineto{\pgfqpoint{3.249523in}{1.539282in}}%
\pgfpathlineto{\pgfqpoint{3.257255in}{1.566635in}}%
\pgfpathlineto{\pgfqpoint{3.259832in}{1.571629in}}%
\pgfpathlineto{\pgfqpoint{3.264986in}{1.595056in}}%
\pgfpathlineto{\pgfqpoint{3.267564in}{1.587438in}}%
\pgfpathlineto{\pgfqpoint{3.275295in}{1.593526in}}%
\pgfpathlineto{\pgfqpoint{3.277872in}{1.590431in}}%
\pgfpathlineto{\pgfqpoint{3.280450in}{1.590885in}}%
\pgfpathlineto{\pgfqpoint{3.283027in}{1.583801in}}%
\pgfpathlineto{\pgfqpoint{3.285604in}{1.585094in}}%
\pgfpathlineto{\pgfqpoint{3.295913in}{1.588055in}}%
\pgfpathlineto{\pgfqpoint{3.298490in}{1.591522in}}%
\pgfpathlineto{\pgfqpoint{3.301067in}{1.591168in}}%
\pgfpathlineto{\pgfqpoint{3.303645in}{1.597436in}}%
\pgfpathlineto{\pgfqpoint{3.311376in}{1.595301in}}%
\pgfpathlineto{\pgfqpoint{3.313953in}{1.578994in}}%
\pgfpathlineto{\pgfqpoint{3.316531in}{1.587892in}}%
\pgfpathlineto{\pgfqpoint{3.319108in}{1.587093in}}%
\pgfpathlineto{\pgfqpoint{3.321685in}{1.574699in}}%
\pgfpathlineto{\pgfqpoint{3.329417in}{1.579254in}}%
\pgfpathlineto{\pgfqpoint{3.331994in}{1.596688in}}%
\pgfpathlineto{\pgfqpoint{3.334571in}{1.599448in}}%
\pgfpathlineto{\pgfqpoint{3.337148in}{1.614358in}}%
\pgfpathlineto{\pgfqpoint{3.339726in}{1.616764in}}%
\pgfpathlineto{\pgfqpoint{3.347457in}{1.600538in}}%
\pgfpathlineto{\pgfqpoint{3.350034in}{1.584427in}}%
\pgfpathlineto{\pgfqpoint{3.352612in}{1.599348in}}%
\pgfpathlineto{\pgfqpoint{3.355189in}{1.564028in}}%
\pgfpathlineto{\pgfqpoint{3.357766in}{1.575999in}}%
\pgfpathlineto{\pgfqpoint{3.365498in}{1.573942in}}%
\pgfpathlineto{\pgfqpoint{3.368075in}{1.577009in}}%
\pgfpathlineto{\pgfqpoint{3.370652in}{1.538409in}}%
\pgfpathlineto{\pgfqpoint{3.373229in}{1.528887in}}%
\pgfpathlineto{\pgfqpoint{3.375807in}{1.557055in}}%
\pgfpathlineto{\pgfqpoint{3.383538in}{1.555993in}}%
\pgfpathlineto{\pgfqpoint{3.386115in}{1.516608in}}%
\pgfpathlineto{\pgfqpoint{3.388693in}{1.559442in}}%
\pgfpathlineto{\pgfqpoint{3.391270in}{1.512844in}}%
\pgfpathlineto{\pgfqpoint{3.393847in}{1.485955in}}%
\pgfpathlineto{\pgfqpoint{3.401579in}{1.453444in}}%
\pgfpathlineto{\pgfqpoint{3.404156in}{1.457421in}}%
\pgfpathlineto{\pgfqpoint{3.406733in}{1.437943in}}%
\pgfpathlineto{\pgfqpoint{3.409310in}{1.430159in}}%
\pgfpathlineto{\pgfqpoint{3.411888in}{1.471963in}}%
\pgfpathlineto{\pgfqpoint{3.419619in}{1.491218in}}%
\pgfpathlineto{\pgfqpoint{3.422196in}{1.529095in}}%
\pgfpathlineto{\pgfqpoint{3.424774in}{1.512153in}}%
\pgfpathlineto{\pgfqpoint{3.429928in}{1.566332in}}%
\pgfpathlineto{\pgfqpoint{3.437660in}{1.575248in}}%
\pgfpathlineto{\pgfqpoint{3.440237in}{1.605903in}}%
\pgfpathlineto{\pgfqpoint{3.442814in}{1.603068in}}%
\pgfpathlineto{\pgfqpoint{3.445391in}{1.628360in}}%
\pgfpathlineto{\pgfqpoint{3.447969in}{1.659535in}}%
\pgfpathlineto{\pgfqpoint{3.455700in}{1.661961in}}%
\pgfpathlineto{\pgfqpoint{3.463432in}{1.694269in}}%
\pgfpathlineto{\pgfqpoint{3.466009in}{1.695307in}}%
\pgfpathlineto{\pgfqpoint{3.473741in}{1.694281in}}%
\pgfpathlineto{\pgfqpoint{3.476318in}{1.689901in}}%
\pgfpathlineto{\pgfqpoint{3.478895in}{1.697091in}}%
\pgfpathlineto{\pgfqpoint{3.481472in}{1.699001in}}%
\pgfpathlineto{\pgfqpoint{3.484049in}{1.696239in}}%
\pgfpathlineto{\pgfqpoint{3.491781in}{1.696894in}}%
\pgfpathlineto{\pgfqpoint{3.494358in}{1.710889in}}%
\pgfpathlineto{\pgfqpoint{3.496936in}{1.703618in}}%
\pgfpathlineto{\pgfqpoint{3.502090in}{1.718439in}}%
\pgfpathlineto{\pgfqpoint{3.509822in}{1.718831in}}%
\pgfpathlineto{\pgfqpoint{3.512399in}{1.728942in}}%
\pgfpathlineto{\pgfqpoint{3.514976in}{1.735822in}}%
\pgfpathlineto{\pgfqpoint{3.520130in}{1.751927in}}%
\pgfpathlineto{\pgfqpoint{3.527862in}{1.738529in}}%
\pgfpathlineto{\pgfqpoint{3.530439in}{1.750877in}}%
\pgfpathlineto{\pgfqpoint{3.533017in}{1.748035in}}%
\pgfpathlineto{\pgfqpoint{3.535594in}{1.748596in}}%
\pgfpathlineto{\pgfqpoint{3.538171in}{1.755997in}}%
\pgfpathlineto{\pgfqpoint{3.545903in}{1.756583in}}%
\pgfpathlineto{\pgfqpoint{3.548480in}{1.744314in}}%
\pgfpathlineto{\pgfqpoint{3.551057in}{1.713373in}}%
\pgfpathlineto{\pgfqpoint{3.553634in}{1.728276in}}%
\pgfpathlineto{\pgfqpoint{3.556211in}{1.688566in}}%
\pgfpathlineto{\pgfqpoint{3.563943in}{1.675460in}}%
\pgfpathlineto{\pgfqpoint{3.566520in}{1.666611in}}%
\pgfpathlineto{\pgfqpoint{3.569098in}{1.702919in}}%
\pgfpathlineto{\pgfqpoint{3.571675in}{1.757832in}}%
\pgfpathlineto{\pgfqpoint{3.574252in}{1.750225in}}%
\pgfpathlineto{\pgfqpoint{3.581984in}{1.777667in}}%
\pgfpathlineto{\pgfqpoint{3.584561in}{1.780277in}}%
\pgfpathlineto{\pgfqpoint{3.587138in}{1.781744in}}%
\pgfpathlineto{\pgfqpoint{3.592292in}{1.786021in}}%
\pgfpathlineto{\pgfqpoint{3.600024in}{1.779484in}}%
\pgfpathlineto{\pgfqpoint{3.602601in}{1.767874in}}%
\pgfpathlineto{\pgfqpoint{3.605178in}{1.741577in}}%
\pgfpathlineto{\pgfqpoint{3.610333in}{1.740106in}}%
\pgfpathlineto{\pgfqpoint{3.618065in}{1.702259in}}%
\pgfpathlineto{\pgfqpoint{3.620642in}{1.678366in}}%
\pgfpathlineto{\pgfqpoint{3.623219in}{1.704149in}}%
\pgfpathlineto{\pgfqpoint{3.625796in}{1.741008in}}%
\pgfpathlineto{\pgfqpoint{3.628373in}{1.719934in}}%
\pgfpathlineto{\pgfqpoint{3.636105in}{1.715412in}}%
\pgfpathlineto{\pgfqpoint{3.638682in}{1.716943in}}%
\pgfpathlineto{\pgfqpoint{3.641259in}{1.695088in}}%
\pgfpathlineto{\pgfqpoint{3.643837in}{1.683074in}}%
\pgfpathlineto{\pgfqpoint{3.646414in}{1.710553in}}%
\pgfpathlineto{\pgfqpoint{3.656723in}{1.712080in}}%
\pgfpathlineto{\pgfqpoint{3.661877in}{1.734866in}}%
\pgfpathlineto{\pgfqpoint{3.664454in}{1.717627in}}%
\pgfpathlineto{\pgfqpoint{3.672186in}{1.706482in}}%
\pgfpathlineto{\pgfqpoint{3.677340in}{1.641657in}}%
\pgfpathlineto{\pgfqpoint{3.679918in}{1.657082in}}%
\pgfpathlineto{\pgfqpoint{3.682495in}{1.623768in}}%
\pgfpathlineto{\pgfqpoint{3.690226in}{1.655798in}}%
\pgfpathlineto{\pgfqpoint{3.692804in}{1.690889in}}%
\pgfpathlineto{\pgfqpoint{3.695381in}{1.704787in}}%
\pgfpathlineto{\pgfqpoint{3.697958in}{1.736773in}}%
\pgfpathlineto{\pgfqpoint{3.700535in}{1.727286in}}%
\pgfpathlineto{\pgfqpoint{3.708267in}{1.714384in}}%
\pgfpathlineto{\pgfqpoint{3.710844in}{1.729114in}}%
\pgfpathlineto{\pgfqpoint{3.713421in}{1.733751in}}%
\pgfpathlineto{\pgfqpoint{3.715999in}{1.730958in}}%
\pgfpathlineto{\pgfqpoint{3.728885in}{1.742002in}}%
\pgfpathlineto{\pgfqpoint{3.731462in}{1.741468in}}%
\pgfpathlineto{\pgfqpoint{3.734039in}{1.736656in}}%
\pgfpathlineto{\pgfqpoint{3.736616in}{1.753781in}}%
\pgfpathlineto{\pgfqpoint{3.744348in}{1.751817in}}%
\pgfpathlineto{\pgfqpoint{3.746925in}{1.763755in}}%
\pgfpathlineto{\pgfqpoint{3.749502in}{1.772118in}}%
\pgfpathlineto{\pgfqpoint{3.752080in}{1.774929in}}%
\pgfpathlineto{\pgfqpoint{3.754657in}{1.761133in}}%
\pgfpathlineto{\pgfqpoint{3.762388in}{1.788620in}}%
\pgfpathlineto{\pgfqpoint{3.764966in}{1.775601in}}%
\pgfpathlineto{\pgfqpoint{3.767543in}{1.756884in}}%
\pgfpathlineto{\pgfqpoint{3.770120in}{1.758567in}}%
\pgfpathlineto{\pgfqpoint{3.772697in}{1.718342in}}%
\pgfpathlineto{\pgfqpoint{3.780429in}{1.732946in}}%
\pgfpathlineto{\pgfqpoint{3.783006in}{1.679432in}}%
\pgfpathlineto{\pgfqpoint{3.785583in}{1.683227in}}%
\pgfpathlineto{\pgfqpoint{3.788161in}{1.716083in}}%
\pgfpathlineto{\pgfqpoint{3.790738in}{1.695376in}}%
\pgfpathlineto{\pgfqpoint{3.798469in}{1.727038in}}%
\pgfpathlineto{\pgfqpoint{3.801047in}{1.707246in}}%
\pgfpathlineto{\pgfqpoint{3.803624in}{1.734384in}}%
\pgfpathlineto{\pgfqpoint{3.806201in}{1.722571in}}%
\pgfpathlineto{\pgfqpoint{3.808778in}{1.744457in}}%
\pgfpathlineto{\pgfqpoint{3.816510in}{1.741083in}}%
\pgfpathlineto{\pgfqpoint{3.819087in}{1.725226in}}%
\pgfpathlineto{\pgfqpoint{3.821664in}{1.680237in}}%
\pgfpathlineto{\pgfqpoint{3.824242in}{1.670241in}}%
\pgfpathlineto{\pgfqpoint{3.826819in}{1.689925in}}%
\pgfpathlineto{\pgfqpoint{3.834550in}{1.705102in}}%
\pgfpathlineto{\pgfqpoint{3.837128in}{1.685805in}}%
\pgfpathlineto{\pgfqpoint{3.839705in}{1.678404in}}%
\pgfpathlineto{\pgfqpoint{3.842282in}{1.690645in}}%
\pgfpathlineto{\pgfqpoint{3.852591in}{1.700443in}}%
\pgfpathlineto{\pgfqpoint{3.855168in}{1.697338in}}%
\pgfpathlineto{\pgfqpoint{3.860323in}{1.719866in}}%
\pgfpathlineto{\pgfqpoint{3.862900in}{1.759189in}}%
\pgfpathlineto{\pgfqpoint{3.870631in}{1.736587in}}%
\pgfpathlineto{\pgfqpoint{3.873209in}{1.738307in}}%
\pgfpathlineto{\pgfqpoint{3.875786in}{1.750615in}}%
\pgfpathlineto{\pgfqpoint{3.878363in}{1.751111in}}%
\pgfpathlineto{\pgfqpoint{3.880940in}{1.712520in}}%
\pgfpathlineto{\pgfqpoint{3.888672in}{1.728660in}}%
\pgfpathlineto{\pgfqpoint{3.891249in}{1.724117in}}%
\pgfpathlineto{\pgfqpoint{3.893826in}{1.746549in}}%
\pgfpathlineto{\pgfqpoint{3.896403in}{1.734923in}}%
\pgfpathlineto{\pgfqpoint{3.898981in}{1.730362in}}%
\pgfpathlineto{\pgfqpoint{3.906712in}{1.728133in}}%
\pgfpathlineto{\pgfqpoint{3.909290in}{1.736017in}}%
\pgfpathlineto{\pgfqpoint{3.911867in}{1.728836in}}%
\pgfpathlineto{\pgfqpoint{3.914444in}{1.711331in}}%
\pgfpathlineto{\pgfqpoint{3.917021in}{1.734178in}}%
\pgfpathlineto{\pgfqpoint{3.924753in}{1.741713in}}%
\pgfpathlineto{\pgfqpoint{3.927330in}{1.724124in}}%
\pgfpathlineto{\pgfqpoint{3.929907in}{1.714044in}}%
\pgfpathlineto{\pgfqpoint{3.932484in}{1.725103in}}%
\pgfpathlineto{\pgfqpoint{3.935062in}{1.763867in}}%
\pgfpathlineto{\pgfqpoint{3.942793in}{1.748610in}}%
\pgfpathlineto{\pgfqpoint{3.945371in}{1.740858in}}%
\pgfpathlineto{\pgfqpoint{3.947948in}{1.751460in}}%
\pgfpathlineto{\pgfqpoint{3.950525in}{1.778979in}}%
\pgfpathlineto{\pgfqpoint{3.953102in}{1.776033in}}%
\pgfpathlineto{\pgfqpoint{3.960834in}{1.781380in}}%
\pgfpathlineto{\pgfqpoint{3.963411in}{1.787546in}}%
\pgfpathlineto{\pgfqpoint{3.965988in}{1.786400in}}%
\pgfpathlineto{\pgfqpoint{3.968565in}{1.787508in}}%
\pgfpathlineto{\pgfqpoint{3.971143in}{1.780223in}}%
\pgfpathlineto{\pgfqpoint{3.981451in}{1.756146in}}%
\pgfpathlineto{\pgfqpoint{3.984029in}{1.772369in}}%
\pgfpathlineto{\pgfqpoint{3.986606in}{1.775645in}}%
\pgfpathlineto{\pgfqpoint{3.989183in}{1.761090in}}%
\pgfpathlineto{\pgfqpoint{3.996915in}{1.758958in}}%
\pgfpathlineto{\pgfqpoint{3.999492in}{1.754892in}}%
\pgfpathlineto{\pgfqpoint{4.002069in}{1.759421in}}%
\pgfpathlineto{\pgfqpoint{4.004646in}{1.731012in}}%
\pgfpathlineto{\pgfqpoint{4.007224in}{1.717740in}}%
\pgfpathlineto{\pgfqpoint{4.014955in}{1.704402in}}%
\pgfpathlineto{\pgfqpoint{4.017532in}{1.709934in}}%
\pgfpathlineto{\pgfqpoint{4.020110in}{1.739822in}}%
\pgfpathlineto{\pgfqpoint{4.022687in}{1.745808in}}%
\pgfpathlineto{\pgfqpoint{4.025264in}{1.724618in}}%
\pgfpathlineto{\pgfqpoint{4.032996in}{1.704973in}}%
\pgfpathlineto{\pgfqpoint{4.035573in}{1.719963in}}%
\pgfpathlineto{\pgfqpoint{4.038150in}{1.730106in}}%
\pgfpathlineto{\pgfqpoint{4.040727in}{1.759468in}}%
\pgfpathlineto{\pgfqpoint{4.043305in}{1.742750in}}%
\pgfpathlineto{\pgfqpoint{4.051036in}{1.755258in}}%
\pgfpathlineto{\pgfqpoint{4.053613in}{1.757222in}}%
\pgfpathlineto{\pgfqpoint{4.056191in}{1.737080in}}%
\pgfpathlineto{\pgfqpoint{4.058768in}{1.729376in}}%
\pgfpathlineto{\pgfqpoint{4.061345in}{1.728392in}}%
\pgfpathlineto{\pgfqpoint{4.069077in}{1.679570in}}%
\pgfpathlineto{\pgfqpoint{4.071654in}{1.681374in}}%
\pgfpathlineto{\pgfqpoint{4.074231in}{1.699559in}}%
\pgfpathlineto{\pgfqpoint{4.076808in}{1.702539in}}%
\pgfpathlineto{\pgfqpoint{4.087117in}{1.691260in}}%
\pgfpathlineto{\pgfqpoint{4.089694in}{1.703376in}}%
\pgfpathlineto{\pgfqpoint{4.092272in}{1.668671in}}%
\pgfpathlineto{\pgfqpoint{4.094849in}{1.669742in}}%
\pgfpathlineto{\pgfqpoint{4.097426in}{1.695974in}}%
\pgfpathlineto{\pgfqpoint{4.105158in}{1.725056in}}%
\pgfpathlineto{\pgfqpoint{4.107735in}{1.729819in}}%
\pgfpathlineto{\pgfqpoint{4.110312in}{1.730079in}}%
\pgfpathlineto{\pgfqpoint{4.112889in}{1.745779in}}%
\pgfpathlineto{\pgfqpoint{4.115467in}{1.739769in}}%
\pgfpathlineto{\pgfqpoint{4.123198in}{1.747223in}}%
\pgfpathlineto{\pgfqpoint{4.125775in}{1.709912in}}%
\pgfpathlineto{\pgfqpoint{4.128353in}{1.700001in}}%
\pgfpathlineto{\pgfqpoint{4.130930in}{1.677327in}}%
\pgfpathlineto{\pgfqpoint{4.133507in}{1.665129in}}%
\pgfpathlineto{\pgfqpoint{4.141239in}{1.655738in}}%
\pgfpathlineto{\pgfqpoint{4.146393in}{1.691923in}}%
\pgfpathlineto{\pgfqpoint{4.151548in}{1.681930in}}%
\pgfpathlineto{\pgfqpoint{4.159279in}{1.679365in}}%
\pgfpathlineto{\pgfqpoint{4.161856in}{1.674948in}}%
\pgfpathlineto{\pgfqpoint{4.167011in}{1.638560in}}%
\pgfpathlineto{\pgfqpoint{4.169588in}{1.649049in}}%
\pgfpathlineto{\pgfqpoint{4.177320in}{1.684928in}}%
\pgfpathlineto{\pgfqpoint{4.179897in}{1.657461in}}%
\pgfpathlineto{\pgfqpoint{4.182474in}{1.663550in}}%
\pgfpathlineto{\pgfqpoint{4.185051in}{1.653851in}}%
\pgfpathlineto{\pgfqpoint{4.187628in}{1.660080in}}%
\pgfpathlineto{\pgfqpoint{4.195360in}{1.672472in}}%
\pgfpathlineto{\pgfqpoint{4.197937in}{1.661831in}}%
\pgfpathlineto{\pgfqpoint{4.200515in}{1.643179in}}%
\pgfpathlineto{\pgfqpoint{4.205669in}{1.538283in}}%
\pgfpathlineto{\pgfqpoint{4.213401in}{1.471754in}}%
\pgfpathlineto{\pgfqpoint{4.215978in}{1.436416in}}%
\pgfpathlineto{\pgfqpoint{4.218555in}{1.523068in}}%
\pgfpathlineto{\pgfqpoint{4.221132in}{1.567875in}}%
\pgfpathlineto{\pgfqpoint{4.223709in}{1.571689in}}%
\pgfpathlineto{\pgfqpoint{4.231441in}{1.552347in}}%
\pgfpathlineto{\pgfqpoint{4.234018in}{1.490120in}}%
\pgfpathlineto{\pgfqpoint{4.236596in}{1.529815in}}%
\pgfpathlineto{\pgfqpoint{4.239173in}{1.539744in}}%
\pgfpathlineto{\pgfqpoint{4.241750in}{1.503001in}}%
\pgfpathlineto{\pgfqpoint{4.252059in}{1.562182in}}%
\pgfpathlineto{\pgfqpoint{4.254636in}{1.529243in}}%
\pgfpathlineto{\pgfqpoint{4.257213in}{1.534636in}}%
\pgfpathlineto{\pgfqpoint{4.259790in}{1.550553in}}%
\pgfpathlineto{\pgfqpoint{4.267522in}{1.540630in}}%
\pgfpathlineto{\pgfqpoint{4.270099in}{1.570742in}}%
\pgfpathlineto{\pgfqpoint{4.272677in}{1.587903in}}%
\pgfpathlineto{\pgfqpoint{4.275254in}{1.576579in}}%
\pgfpathlineto{\pgfqpoint{4.277831in}{1.536896in}}%
\pgfpathlineto{\pgfqpoint{4.285563in}{1.552391in}}%
\pgfpathlineto{\pgfqpoint{4.288140in}{1.526918in}}%
\pgfpathlineto{\pgfqpoint{4.290717in}{1.521280in}}%
\pgfpathlineto{\pgfqpoint{4.293294in}{1.512344in}}%
\pgfpathlineto{\pgfqpoint{4.295871in}{1.522612in}}%
\pgfpathlineto{\pgfqpoint{4.303603in}{1.488539in}}%
\pgfpathlineto{\pgfqpoint{4.308757in}{1.541863in}}%
\pgfpathlineto{\pgfqpoint{4.311335in}{1.535504in}}%
\pgfpathlineto{\pgfqpoint{4.313912in}{1.556681in}}%
\pgfpathlineto{\pgfqpoint{4.321644in}{1.612469in}}%
\pgfpathlineto{\pgfqpoint{4.324221in}{1.616166in}}%
\pgfpathlineto{\pgfqpoint{4.326798in}{1.641472in}}%
\pgfpathlineto{\pgfqpoint{4.329375in}{1.659969in}}%
\pgfpathlineto{\pgfqpoint{4.331952in}{1.664444in}}%
\pgfpathlineto{\pgfqpoint{4.339684in}{1.672459in}}%
\pgfpathlineto{\pgfqpoint{4.344838in}{1.648240in}}%
\pgfpathlineto{\pgfqpoint{4.347416in}{1.676589in}}%
\pgfpathlineto{\pgfqpoint{4.349993in}{1.695084in}}%
\pgfpathlineto{\pgfqpoint{4.357725in}{1.701039in}}%
\pgfpathlineto{\pgfqpoint{4.360302in}{1.707525in}}%
\pgfpathlineto{\pgfqpoint{4.362879in}{1.708842in}}%
\pgfpathlineto{\pgfqpoint{4.365456in}{1.749199in}}%
\pgfpathlineto{\pgfqpoint{4.368033in}{1.773656in}}%
\pgfpathlineto{\pgfqpoint{4.375765in}{1.776843in}}%
\pgfpathlineto{\pgfqpoint{4.378342in}{1.768999in}}%
\pgfpathlineto{\pgfqpoint{4.380919in}{1.780446in}}%
\pgfpathlineto{\pgfqpoint{4.383497in}{1.776628in}}%
\pgfpathlineto{\pgfqpoint{4.386074in}{1.760124in}}%
\pgfpathlineto{\pgfqpoint{4.393805in}{1.773572in}}%
\pgfpathlineto{\pgfqpoint{4.396383in}{1.785930in}}%
\pgfpathlineto{\pgfqpoint{4.398960in}{1.782199in}}%
\pgfpathlineto{\pgfqpoint{4.401537in}{1.784101in}}%
\pgfpathlineto{\pgfqpoint{4.404114in}{1.785264in}}%
\pgfpathlineto{\pgfqpoint{4.411846in}{1.764976in}}%
\pgfpathlineto{\pgfqpoint{4.414423in}{1.775071in}}%
\pgfpathlineto{\pgfqpoint{4.417000in}{1.776595in}}%
\pgfpathlineto{\pgfqpoint{4.422155in}{1.727287in}}%
\pgfpathlineto{\pgfqpoint{4.429886in}{1.754204in}}%
\pgfpathlineto{\pgfqpoint{4.432464in}{1.753817in}}%
\pgfpathlineto{\pgfqpoint{4.435041in}{1.782558in}}%
\pgfpathlineto{\pgfqpoint{4.437618in}{1.797522in}}%
\pgfpathlineto{\pgfqpoint{4.440195in}{1.804407in}}%
\pgfpathlineto{\pgfqpoint{4.447927in}{1.798494in}}%
\pgfpathlineto{\pgfqpoint{4.450504in}{1.789389in}}%
\pgfpathlineto{\pgfqpoint{4.453081in}{1.784850in}}%
\pgfpathlineto{\pgfqpoint{4.458236in}{1.780155in}}%
\pgfpathlineto{\pgfqpoint{4.465967in}{1.763228in}}%
\pgfpathlineto{\pgfqpoint{4.468545in}{1.785890in}}%
\pgfpathlineto{\pgfqpoint{4.473699in}{1.738551in}}%
\pgfpathlineto{\pgfqpoint{4.476276in}{1.790820in}}%
\pgfpathlineto{\pgfqpoint{4.484008in}{1.789216in}}%
\pgfpathlineto{\pgfqpoint{4.486585in}{1.766697in}}%
\pgfpathlineto{\pgfqpoint{4.489162in}{1.759146in}}%
\pgfpathlineto{\pgfqpoint{4.491740in}{1.764780in}}%
\pgfpathlineto{\pgfqpoint{4.494317in}{1.729909in}}%
\pgfpathlineto{\pgfqpoint{4.502048in}{1.746727in}}%
\pgfpathlineto{\pgfqpoint{4.504626in}{1.764687in}}%
\pgfpathlineto{\pgfqpoint{4.507203in}{1.800986in}}%
\pgfpathlineto{\pgfqpoint{4.509780in}{1.770941in}}%
\pgfpathlineto{\pgfqpoint{4.512357in}{1.716642in}}%
\pgfpathlineto{\pgfqpoint{4.520089in}{1.730320in}}%
\pgfpathlineto{\pgfqpoint{4.522666in}{1.750248in}}%
\pgfpathlineto{\pgfqpoint{4.525243in}{1.774736in}}%
\pgfpathlineto{\pgfqpoint{4.527821in}{1.772969in}}%
\pgfpathlineto{\pgfqpoint{4.538129in}{1.774863in}}%
\pgfpathlineto{\pgfqpoint{4.540707in}{1.797965in}}%
\pgfpathlineto{\pgfqpoint{4.543284in}{1.780471in}}%
\pgfpathlineto{\pgfqpoint{4.545861in}{1.757625in}}%
\pgfpathlineto{\pgfqpoint{4.556170in}{1.714347in}}%
\pgfpathlineto{\pgfqpoint{4.558747in}{1.713206in}}%
\pgfpathlineto{\pgfqpoint{4.563902in}{1.631073in}}%
\pgfpathlineto{\pgfqpoint{4.566479in}{1.606893in}}%
\pgfpathlineto{\pgfqpoint{4.574210in}{1.620866in}}%
\pgfpathlineto{\pgfqpoint{4.576788in}{1.634545in}}%
\pgfpathlineto{\pgfqpoint{4.579365in}{1.590310in}}%
\pgfpathlineto{\pgfqpoint{4.581942in}{1.622446in}}%
\pgfpathlineto{\pgfqpoint{4.584519in}{1.552313in}}%
\pgfpathlineto{\pgfqpoint{4.594828in}{1.557778in}}%
\pgfpathlineto{\pgfqpoint{4.597405in}{1.537160in}}%
\pgfpathlineto{\pgfqpoint{4.599982in}{1.561535in}}%
\pgfpathlineto{\pgfqpoint{4.602560in}{1.558340in}}%
\pgfpathlineto{\pgfqpoint{4.610291in}{1.537665in}}%
\pgfpathlineto{\pgfqpoint{4.612869in}{1.582616in}}%
\pgfpathlineto{\pgfqpoint{4.615446in}{1.576572in}}%
\pgfpathlineto{\pgfqpoint{4.618023in}{1.573098in}}%
\pgfpathlineto{\pgfqpoint{4.620600in}{1.638359in}}%
\pgfpathlineto{\pgfqpoint{4.628332in}{1.634420in}}%
\pgfpathlineto{\pgfqpoint{4.630909in}{1.594338in}}%
\pgfpathlineto{\pgfqpoint{4.633486in}{1.622832in}}%
\pgfpathlineto{\pgfqpoint{4.636063in}{1.635043in}}%
\pgfpathlineto{\pgfqpoint{4.638641in}{1.607087in}}%
\pgfpathlineto{\pgfqpoint{4.646372in}{1.582685in}}%
\pgfpathlineto{\pgfqpoint{4.648950in}{1.586789in}}%
\pgfpathlineto{\pgfqpoint{4.651527in}{1.572051in}}%
\pgfpathlineto{\pgfqpoint{4.654104in}{1.541277in}}%
\pgfpathlineto{\pgfqpoint{4.656681in}{1.582331in}}%
\pgfpathlineto{\pgfqpoint{4.666990in}{1.604504in}}%
\pgfpathlineto{\pgfqpoint{4.669567in}{1.637540in}}%
\pgfpathlineto{\pgfqpoint{4.672144in}{1.640604in}}%
\pgfpathlineto{\pgfqpoint{4.674722in}{1.637069in}}%
\pgfpathlineto{\pgfqpoint{4.682453in}{1.674502in}}%
\pgfpathlineto{\pgfqpoint{4.685030in}{1.651699in}}%
\pgfpathlineto{\pgfqpoint{4.687608in}{1.658975in}}%
\pgfpathlineto{\pgfqpoint{4.690185in}{1.696484in}}%
\pgfpathlineto{\pgfqpoint{4.692762in}{1.689135in}}%
\pgfpathlineto{\pgfqpoint{4.700494in}{1.675554in}}%
\pgfpathlineto{\pgfqpoint{4.703071in}{1.717548in}}%
\pgfpathlineto{\pgfqpoint{4.708225in}{1.736370in}}%
\pgfpathlineto{\pgfqpoint{4.710803in}{1.743358in}}%
\pgfpathlineto{\pgfqpoint{4.718534in}{1.745102in}}%
\pgfpathlineto{\pgfqpoint{4.721111in}{1.732615in}}%
\pgfpathlineto{\pgfqpoint{4.723689in}{1.732998in}}%
\pgfpathlineto{\pgfqpoint{4.726266in}{1.727851in}}%
\pgfpathlineto{\pgfqpoint{4.728843in}{1.750631in}}%
\pgfpathlineto{\pgfqpoint{4.736575in}{1.747911in}}%
\pgfpathlineto{\pgfqpoint{4.739152in}{1.749196in}}%
\pgfpathlineto{\pgfqpoint{4.741729in}{1.758490in}}%
\pgfpathlineto{\pgfqpoint{4.744306in}{1.784063in}}%
\pgfpathlineto{\pgfqpoint{4.746884in}{1.799515in}}%
\pgfpathlineto{\pgfqpoint{4.754615in}{1.796715in}}%
\pgfpathlineto{\pgfqpoint{4.757192in}{1.784943in}}%
\pgfpathlineto{\pgfqpoint{4.759770in}{1.785589in}}%
\pgfpathlineto{\pgfqpoint{4.762347in}{1.789083in}}%
\pgfpathlineto{\pgfqpoint{4.772656in}{1.794253in}}%
\pgfpathlineto{\pgfqpoint{4.777810in}{1.824253in}}%
\pgfpathlineto{\pgfqpoint{4.780387in}{1.822316in}}%
\pgfpathlineto{\pgfqpoint{4.782965in}{1.833045in}}%
\pgfpathlineto{\pgfqpoint{4.790696in}{1.820163in}}%
\pgfpathlineto{\pgfqpoint{4.793273in}{1.805442in}}%
\pgfpathlineto{\pgfqpoint{4.795851in}{1.817432in}}%
\pgfpathlineto{\pgfqpoint{4.798428in}{1.793574in}}%
\pgfpathlineto{\pgfqpoint{4.801005in}{1.801069in}}%
\pgfpathlineto{\pgfqpoint{4.808737in}{1.798327in}}%
\pgfpathlineto{\pgfqpoint{4.813891in}{1.837948in}}%
\pgfpathlineto{\pgfqpoint{4.816468in}{1.837303in}}%
\pgfpathlineto{\pgfqpoint{4.819046in}{1.834544in}}%
\pgfpathlineto{\pgfqpoint{4.826777in}{1.858137in}}%
\pgfpathlineto{\pgfqpoint{4.829354in}{1.870450in}}%
\pgfpathlineto{\pgfqpoint{4.831932in}{1.877813in}}%
\pgfpathlineto{\pgfqpoint{4.834509in}{1.864541in}}%
\pgfpathlineto{\pgfqpoint{4.837086in}{1.861369in}}%
\pgfpathlineto{\pgfqpoint{4.844818in}{1.857939in}}%
\pgfpathlineto{\pgfqpoint{4.847395in}{1.850632in}}%
\pgfpathlineto{\pgfqpoint{4.849972in}{1.867936in}}%
\pgfpathlineto{\pgfqpoint{4.852549in}{1.844375in}}%
\pgfpathlineto{\pgfqpoint{4.855127in}{1.831989in}}%
\pgfpathlineto{\pgfqpoint{4.862858in}{1.851519in}}%
\pgfpathlineto{\pgfqpoint{4.865435in}{1.829126in}}%
\pgfpathlineto{\pgfqpoint{4.868013in}{1.815592in}}%
\pgfpathlineto{\pgfqpoint{4.870590in}{1.817770in}}%
\pgfpathlineto{\pgfqpoint{4.873167in}{1.833608in}}%
\pgfpathlineto{\pgfqpoint{4.880899in}{1.827233in}}%
\pgfpathlineto{\pgfqpoint{4.883476in}{1.858784in}}%
\pgfpathlineto{\pgfqpoint{4.886053in}{1.833583in}}%
\pgfpathlineto{\pgfqpoint{4.888630in}{1.830901in}}%
\pgfpathlineto{\pgfqpoint{4.891207in}{1.807232in}}%
\pgfpathlineto{\pgfqpoint{4.898939in}{1.827150in}}%
\pgfpathlineto{\pgfqpoint{4.901516in}{1.798512in}}%
\pgfpathlineto{\pgfqpoint{4.904094in}{1.794939in}}%
\pgfpathlineto{\pgfqpoint{4.906671in}{1.774254in}}%
\pgfpathlineto{\pgfqpoint{4.909248in}{1.793611in}}%
\pgfpathlineto{\pgfqpoint{4.916980in}{1.787237in}}%
\pgfpathlineto{\pgfqpoint{4.919557in}{1.822758in}}%
\pgfpathlineto{\pgfqpoint{4.922134in}{1.836893in}}%
\pgfpathlineto{\pgfqpoint{4.924711in}{1.834893in}}%
\pgfpathlineto{\pgfqpoint{4.927288in}{1.845108in}}%
\pgfpathlineto{\pgfqpoint{4.940175in}{1.839481in}}%
\pgfpathlineto{\pgfqpoint{4.942752in}{1.850914in}}%
\pgfpathlineto{\pgfqpoint{4.945329in}{1.847942in}}%
\pgfpathlineto{\pgfqpoint{4.953061in}{1.861016in}}%
\pgfpathlineto{\pgfqpoint{4.955638in}{1.867061in}}%
\pgfpathlineto{\pgfqpoint{4.958215in}{1.875090in}}%
\pgfpathlineto{\pgfqpoint{4.960792in}{1.876210in}}%
\pgfpathlineto{\pgfqpoint{4.963369in}{1.867950in}}%
\pgfpathlineto{\pgfqpoint{4.971101in}{1.849563in}}%
\pgfpathlineto{\pgfqpoint{4.973678in}{1.852751in}}%
\pgfpathlineto{\pgfqpoint{4.976256in}{1.844679in}}%
\pgfpathlineto{\pgfqpoint{4.978833in}{1.860622in}}%
\pgfpathlineto{\pgfqpoint{4.981410in}{1.854298in}}%
\pgfpathlineto{\pgfqpoint{4.989142in}{1.871059in}}%
\pgfpathlineto{\pgfqpoint{4.991719in}{1.873090in}}%
\pgfpathlineto{\pgfqpoint{4.994296in}{1.867289in}}%
\pgfpathlineto{\pgfqpoint{4.996873in}{1.902410in}}%
\pgfpathlineto{\pgfqpoint{4.999450in}{1.818377in}}%
\pgfpathlineto{\pgfqpoint{5.007182in}{1.780666in}}%
\pgfpathlineto{\pgfqpoint{5.014914in}{1.895580in}}%
\pgfpathlineto{\pgfqpoint{5.017491in}{1.899867in}}%
\pgfpathlineto{\pgfqpoint{5.027800in}{1.892212in}}%
\pgfpathlineto{\pgfqpoint{5.030377in}{1.900711in}}%
\pgfpathlineto{\pgfqpoint{5.032954in}{1.904124in}}%
\pgfpathlineto{\pgfqpoint{5.035531in}{1.945206in}}%
\pgfpathlineto{\pgfqpoint{5.043263in}{1.954911in}}%
\pgfpathlineto{\pgfqpoint{5.045840in}{1.968331in}}%
\pgfpathlineto{\pgfqpoint{5.048417in}{1.974031in}}%
\pgfpathlineto{\pgfqpoint{5.050995in}{1.984929in}}%
\pgfpathlineto{\pgfqpoint{5.053572in}{1.988637in}}%
\pgfpathlineto{\pgfqpoint{5.061304in}{1.990585in}}%
\pgfpathlineto{\pgfqpoint{5.063881in}{1.996669in}}%
\pgfpathlineto{\pgfqpoint{5.066458in}{2.000015in}}%
\pgfpathlineto{\pgfqpoint{5.069035in}{1.976685in}}%
\pgfpathlineto{\pgfqpoint{5.071612in}{1.985317in}}%
\pgfpathlineto{\pgfqpoint{5.079344in}{1.974303in}}%
\pgfpathlineto{\pgfqpoint{5.081921in}{1.971760in}}%
\pgfpathlineto{\pgfqpoint{5.084498in}{1.967834in}}%
\pgfpathlineto{\pgfqpoint{5.087076in}{1.966429in}}%
\pgfpathlineto{\pgfqpoint{5.089653in}{1.968538in}}%
\pgfpathlineto{\pgfqpoint{5.097384in}{1.963503in}}%
\pgfpathlineto{\pgfqpoint{5.099962in}{1.951913in}}%
\pgfpathlineto{\pgfqpoint{5.102539in}{1.953332in}}%
\pgfpathlineto{\pgfqpoint{5.105116in}{1.956438in}}%
\pgfpathlineto{\pgfqpoint{5.107693in}{1.974666in}}%
\pgfpathlineto{\pgfqpoint{5.115425in}{1.973933in}}%
\pgfpathlineto{\pgfqpoint{5.118002in}{1.976097in}}%
\pgfpathlineto{\pgfqpoint{5.120579in}{1.973923in}}%
\pgfpathlineto{\pgfqpoint{5.123157in}{1.987997in}}%
\pgfpathlineto{\pgfqpoint{5.125734in}{1.980219in}}%
\pgfpathlineto{\pgfqpoint{5.133465in}{1.987884in}}%
\pgfpathlineto{\pgfqpoint{5.136043in}{1.974021in}}%
\pgfpathlineto{\pgfqpoint{5.138620in}{1.984801in}}%
\pgfpathlineto{\pgfqpoint{5.143774in}{1.979213in}}%
\pgfpathlineto{\pgfqpoint{5.151506in}{1.975391in}}%
\pgfpathlineto{\pgfqpoint{5.154083in}{1.978416in}}%
\pgfpathlineto{\pgfqpoint{5.156660in}{1.968910in}}%
\pgfpathlineto{\pgfqpoint{5.159238in}{1.969290in}}%
\pgfpathlineto{\pgfqpoint{5.161815in}{1.965119in}}%
\pgfpathlineto{\pgfqpoint{5.169546in}{1.981335in}}%
\pgfpathlineto{\pgfqpoint{5.174701in}{1.971047in}}%
\pgfpathlineto{\pgfqpoint{5.177278in}{1.975075in}}%
\pgfpathlineto{\pgfqpoint{5.179855in}{1.984052in}}%
\pgfpathlineto{\pgfqpoint{5.190164in}{1.990801in}}%
\pgfpathlineto{\pgfqpoint{5.192741in}{1.990976in}}%
\pgfpathlineto{\pgfqpoint{5.195319in}{1.990062in}}%
\pgfpathlineto{\pgfqpoint{5.197896in}{1.927860in}}%
\pgfpathlineto{\pgfqpoint{5.205627in}{1.967694in}}%
\pgfpathlineto{\pgfqpoint{5.208205in}{1.925262in}}%
\pgfpathlineto{\pgfqpoint{5.210782in}{1.915986in}}%
\pgfpathlineto{\pgfqpoint{5.213359in}{1.941201in}}%
\pgfpathlineto{\pgfqpoint{5.215936in}{1.937785in}}%
\pgfpathlineto{\pgfqpoint{5.223668in}{1.934777in}}%
\pgfpathlineto{\pgfqpoint{5.226245in}{1.938652in}}%
\pgfpathlineto{\pgfqpoint{5.231400in}{1.974296in}}%
\pgfpathlineto{\pgfqpoint{5.233977in}{1.954894in}}%
\pgfpathlineto{\pgfqpoint{5.241708in}{1.932491in}}%
\pgfpathlineto{\pgfqpoint{5.244286in}{1.950980in}}%
\pgfpathlineto{\pgfqpoint{5.246863in}{1.960364in}}%
\pgfpathlineto{\pgfqpoint{5.249440in}{1.933049in}}%
\pgfpathlineto{\pgfqpoint{5.252017in}{1.953968in}}%
\pgfpathlineto{\pgfqpoint{5.259749in}{1.950091in}}%
\pgfpathlineto{\pgfqpoint{5.262326in}{1.936372in}}%
\pgfpathlineto{\pgfqpoint{5.264903in}{1.947887in}}%
\pgfpathlineto{\pgfqpoint{5.267481in}{1.935820in}}%
\pgfpathlineto{\pgfqpoint{5.270058in}{1.928195in}}%
\pgfpathlineto{\pgfqpoint{5.277789in}{1.924929in}}%
\pgfpathlineto{\pgfqpoint{5.280367in}{1.897004in}}%
\pgfpathlineto{\pgfqpoint{5.282944in}{1.897922in}}%
\pgfpathlineto{\pgfqpoint{5.285521in}{1.894925in}}%
\pgfpathlineto{\pgfqpoint{5.288098in}{1.900900in}}%
\pgfpathlineto{\pgfqpoint{5.295830in}{1.895674in}}%
\pgfpathlineto{\pgfqpoint{5.298407in}{1.891621in}}%
\pgfpathlineto{\pgfqpoint{5.300984in}{1.882240in}}%
\pgfpathlineto{\pgfqpoint{5.303561in}{1.895701in}}%
\pgfpathlineto{\pgfqpoint{5.306139in}{1.885115in}}%
\pgfpathlineto{\pgfqpoint{5.313870in}{1.892642in}}%
\pgfpathlineto{\pgfqpoint{5.316448in}{1.887120in}}%
\pgfpathlineto{\pgfqpoint{5.319025in}{1.890264in}}%
\pgfpathlineto{\pgfqpoint{5.321602in}{1.891352in}}%
\pgfpathlineto{\pgfqpoint{5.324179in}{1.897744in}}%
\pgfpathlineto{\pgfqpoint{5.331911in}{1.896147in}}%
\pgfpathlineto{\pgfqpoint{5.337065in}{1.866162in}}%
\pgfpathlineto{\pgfqpoint{5.339642in}{1.865049in}}%
\pgfpathlineto{\pgfqpoint{5.342220in}{1.858130in}}%
\pgfpathlineto{\pgfqpoint{5.349951in}{1.912649in}}%
\pgfpathlineto{\pgfqpoint{5.352529in}{1.923049in}}%
\pgfpathlineto{\pgfqpoint{5.355106in}{1.940174in}}%
\pgfpathlineto{\pgfqpoint{5.357683in}{1.944990in}}%
\pgfpathlineto{\pgfqpoint{5.360260in}{1.959776in}}%
\pgfpathlineto{\pgfqpoint{5.367992in}{1.938462in}}%
\pgfpathlineto{\pgfqpoint{5.370569in}{1.954047in}}%
\pgfpathlineto{\pgfqpoint{5.375723in}{1.966531in}}%
\pgfpathlineto{\pgfqpoint{5.378301in}{1.954707in}}%
\pgfpathlineto{\pgfqpoint{5.386032in}{1.961219in}}%
\pgfpathlineto{\pgfqpoint{5.388609in}{1.965424in}}%
\pgfpathlineto{\pgfqpoint{5.391187in}{1.977490in}}%
\pgfpathlineto{\pgfqpoint{5.396341in}{1.994741in}}%
\pgfpathlineto{\pgfqpoint{5.404073in}{1.984063in}}%
\pgfpathlineto{\pgfqpoint{5.406650in}{1.979015in}}%
\pgfpathlineto{\pgfqpoint{5.409227in}{1.951502in}}%
\pgfpathlineto{\pgfqpoint{5.411804in}{1.943445in}}%
\pgfpathlineto{\pgfqpoint{5.414382in}{1.947867in}}%
\pgfpathlineto{\pgfqpoint{5.422113in}{1.956281in}}%
\pgfpathlineto{\pgfqpoint{5.424690in}{1.965318in}}%
\pgfpathlineto{\pgfqpoint{5.427268in}{2.013720in}}%
\pgfpathlineto{\pgfqpoint{5.429845in}{2.015482in}}%
\pgfpathlineto{\pgfqpoint{5.432422in}{2.036727in}}%
\pgfpathlineto{\pgfqpoint{5.440154in}{2.045131in}}%
\pgfpathlineto{\pgfqpoint{5.442731in}{2.056675in}}%
\pgfpathlineto{\pgfqpoint{5.445308in}{2.038930in}}%
\pgfpathlineto{\pgfqpoint{5.447885in}{2.044541in}}%
\pgfpathlineto{\pgfqpoint{5.450463in}{2.045807in}}%
\pgfpathlineto{\pgfqpoint{5.458194in}{2.063881in}}%
\pgfpathlineto{\pgfqpoint{5.460771in}{2.072623in}}%
\pgfpathlineto{\pgfqpoint{5.463349in}{2.066906in}}%
\pgfpathlineto{\pgfqpoint{5.465926in}{2.067723in}}%
\pgfpathlineto{\pgfqpoint{5.468503in}{2.072980in}}%
\pgfpathlineto{\pgfqpoint{5.478812in}{2.074407in}}%
\pgfpathlineto{\pgfqpoint{5.481389in}{2.056267in}}%
\pgfpathlineto{\pgfqpoint{5.483966in}{2.060643in}}%
\pgfpathlineto{\pgfqpoint{5.486544in}{2.049588in}}%
\pgfpathlineto{\pgfqpoint{5.496852in}{2.078540in}}%
\pgfpathlineto{\pgfqpoint{5.499430in}{2.088639in}}%
\pgfpathlineto{\pgfqpoint{5.502007in}{2.091918in}}%
\pgfpathlineto{\pgfqpoint{5.504584in}{2.100628in}}%
\pgfpathlineto{\pgfqpoint{5.514893in}{2.084569in}}%
\pgfpathlineto{\pgfqpoint{5.517470in}{2.091428in}}%
\pgfpathlineto{\pgfqpoint{5.520047in}{2.082180in}}%
\pgfpathlineto{\pgfqpoint{5.522625in}{2.080606in}}%
\pgfpathlineto{\pgfqpoint{5.532933in}{2.084772in}}%
\pgfpathlineto{\pgfqpoint{5.535511in}{2.090337in}}%
\pgfpathlineto{\pgfqpoint{5.538088in}{2.083579in}}%
\pgfpathlineto{\pgfqpoint{5.540665in}{2.090822in}}%
\pgfpathlineto{\pgfqpoint{5.548397in}{2.076059in}}%
\pgfpathlineto{\pgfqpoint{5.553551in}{2.084182in}}%
\pgfpathlineto{\pgfqpoint{5.556128in}{2.070398in}}%
\pgfpathlineto{\pgfqpoint{5.558706in}{2.081791in}}%
\pgfpathlineto{\pgfqpoint{5.566437in}{2.076261in}}%
\pgfpathlineto{\pgfqpoint{5.569014in}{2.063238in}}%
\pgfpathlineto{\pgfqpoint{5.571592in}{2.055377in}}%
\pgfpathlineto{\pgfqpoint{5.574169in}{2.053754in}}%
\pgfpathlineto{\pgfqpoint{5.576746in}{2.078019in}}%
\pgfpathlineto{\pgfqpoint{5.584478in}{2.071036in}}%
\pgfpathlineto{\pgfqpoint{5.587055in}{2.073501in}}%
\pgfpathlineto{\pgfqpoint{5.589632in}{2.073249in}}%
\pgfpathlineto{\pgfqpoint{5.592209in}{2.078171in}}%
\pgfpathlineto{\pgfqpoint{5.594786in}{2.085448in}}%
\pgfpathlineto{\pgfqpoint{5.605095in}{2.108589in}}%
\pgfpathlineto{\pgfqpoint{5.607673in}{2.129430in}}%
\pgfpathlineto{\pgfqpoint{5.610250in}{2.140411in}}%
\pgfpathlineto{\pgfqpoint{5.612827in}{2.146076in}}%
\pgfpathlineto{\pgfqpoint{5.623136in}{2.156842in}}%
\pgfpathlineto{\pgfqpoint{5.625713in}{2.155987in}}%
\pgfpathlineto{\pgfqpoint{5.628290in}{2.165560in}}%
\pgfpathlineto{\pgfqpoint{5.630867in}{2.178461in}}%
\pgfpathlineto{\pgfqpoint{5.638599in}{2.170715in}}%
\pgfpathlineto{\pgfqpoint{5.641176in}{2.162765in}}%
\pgfpathlineto{\pgfqpoint{5.643754in}{2.192373in}}%
\pgfpathlineto{\pgfqpoint{5.646331in}{2.178459in}}%
\pgfpathlineto{\pgfqpoint{5.648908in}{2.177417in}}%
\pgfpathlineto{\pgfqpoint{5.656640in}{2.167752in}}%
\pgfpathlineto{\pgfqpoint{5.659217in}{2.167981in}}%
\pgfpathlineto{\pgfqpoint{5.661794in}{2.161236in}}%
\pgfpathlineto{\pgfqpoint{5.664371in}{2.170339in}}%
\pgfpathlineto{\pgfqpoint{5.666948in}{2.185090in}}%
\pgfpathlineto{\pgfqpoint{5.674680in}{2.180591in}}%
\pgfpathlineto{\pgfqpoint{5.677257in}{2.175781in}}%
\pgfpathlineto{\pgfqpoint{5.679835in}{2.190733in}}%
\pgfpathlineto{\pgfqpoint{5.682412in}{2.187958in}}%
\pgfpathlineto{\pgfqpoint{5.684989in}{2.194982in}}%
\pgfpathlineto{\pgfqpoint{5.692721in}{2.193572in}}%
\pgfpathlineto{\pgfqpoint{5.695298in}{2.168398in}}%
\pgfpathlineto{\pgfqpoint{5.697875in}{2.169414in}}%
\pgfpathlineto{\pgfqpoint{5.703029in}{2.169166in}}%
\pgfpathlineto{\pgfqpoint{5.710761in}{2.161626in}}%
\pgfpathlineto{\pgfqpoint{5.713338in}{2.176137in}}%
\pgfpathlineto{\pgfqpoint{5.715915in}{2.168488in}}%
\pgfpathlineto{\pgfqpoint{5.718493in}{2.176074in}}%
\pgfpathlineto{\pgfqpoint{5.721070in}{2.173902in}}%
\pgfpathlineto{\pgfqpoint{5.728802in}{2.175249in}}%
\pgfpathlineto{\pgfqpoint{5.731379in}{2.177548in}}%
\pgfpathlineto{\pgfqpoint{5.736533in}{2.171135in}}%
\pgfpathlineto{\pgfqpoint{5.739110in}{2.170875in}}%
\pgfpathlineto{\pgfqpoint{5.746842in}{2.166267in}}%
\pgfpathlineto{\pgfqpoint{5.749419in}{2.169835in}}%
\pgfpathlineto{\pgfqpoint{5.751996in}{2.166305in}}%
\pgfpathlineto{\pgfqpoint{5.754574in}{2.153148in}}%
\pgfpathlineto{\pgfqpoint{5.764883in}{2.173359in}}%
\pgfpathlineto{\pgfqpoint{5.767460in}{2.170492in}}%
\pgfpathlineto{\pgfqpoint{5.770037in}{2.164513in}}%
\pgfpathlineto{\pgfqpoint{5.772614in}{2.195593in}}%
\pgfpathlineto{\pgfqpoint{5.775191in}{2.180840in}}%
\pgfpathlineto{\pgfqpoint{5.782923in}{2.199955in}}%
\pgfpathlineto{\pgfqpoint{5.785500in}{2.208636in}}%
\pgfpathlineto{\pgfqpoint{5.788077in}{2.206990in}}%
\pgfpathlineto{\pgfqpoint{5.790655in}{2.207693in}}%
\pgfpathlineto{\pgfqpoint{5.793232in}{2.184216in}}%
\pgfpathlineto{\pgfqpoint{5.800963in}{2.179827in}}%
\pgfpathlineto{\pgfqpoint{5.803541in}{2.191732in}}%
\pgfpathlineto{\pgfqpoint{5.808695in}{2.192329in}}%
\pgfpathlineto{\pgfqpoint{5.811272in}{2.197817in}}%
\pgfpathlineto{\pgfqpoint{5.819004in}{2.188825in}}%
\pgfpathlineto{\pgfqpoint{5.821581in}{2.186959in}}%
\pgfpathlineto{\pgfqpoint{5.824158in}{2.173145in}}%
\pgfpathlineto{\pgfqpoint{5.826736in}{2.167187in}}%
\pgfpathlineto{\pgfqpoint{5.829313in}{2.159133in}}%
\pgfpathlineto{\pgfqpoint{5.837044in}{2.169375in}}%
\pgfpathlineto{\pgfqpoint{5.839622in}{2.168159in}}%
\pgfpathlineto{\pgfqpoint{5.842199in}{2.126910in}}%
\pgfpathlineto{\pgfqpoint{5.844776in}{2.133081in}}%
\pgfpathlineto{\pgfqpoint{5.847353in}{2.155819in}}%
\pgfpathlineto{\pgfqpoint{5.855085in}{2.169670in}}%
\pgfpathlineto{\pgfqpoint{5.857662in}{2.172417in}}%
\pgfpathlineto{\pgfqpoint{5.860239in}{2.172757in}}%
\pgfpathlineto{\pgfqpoint{5.862817in}{2.179319in}}%
\pgfpathlineto{\pgfqpoint{5.865394in}{2.181734in}}%
\pgfpathlineto{\pgfqpoint{5.875703in}{2.186441in}}%
\pgfpathlineto{\pgfqpoint{5.880857in}{2.208486in}}%
\pgfpathlineto{\pgfqpoint{5.883434in}{2.220317in}}%
\pgfpathlineto{\pgfqpoint{5.891166in}{2.220082in}}%
\pgfpathlineto{\pgfqpoint{5.893743in}{2.210611in}}%
\pgfpathlineto{\pgfqpoint{5.896320in}{2.213386in}}%
\pgfpathlineto{\pgfqpoint{5.898898in}{2.206559in}}%
\pgfpathlineto{\pgfqpoint{5.901475in}{2.214238in}}%
\pgfpathlineto{\pgfqpoint{5.909206in}{2.227836in}}%
\pgfpathlineto{\pgfqpoint{5.911784in}{2.227433in}}%
\pgfpathlineto{\pgfqpoint{5.916938in}{2.238961in}}%
\pgfpathlineto{\pgfqpoint{5.919515in}{2.248560in}}%
\pgfpathlineto{\pgfqpoint{5.927247in}{2.255642in}}%
\pgfpathlineto{\pgfqpoint{5.929824in}{2.234764in}}%
\pgfpathlineto{\pgfqpoint{5.934979in}{2.221406in}}%
\pgfpathlineto{\pgfqpoint{5.937556in}{2.228651in}}%
\pgfpathlineto{\pgfqpoint{5.945287in}{2.233833in}}%
\pgfpathlineto{\pgfqpoint{5.947865in}{2.208722in}}%
\pgfpathlineto{\pgfqpoint{5.950442in}{2.223484in}}%
\pgfpathlineto{\pgfqpoint{5.953019in}{2.191123in}}%
\pgfpathlineto{\pgfqpoint{5.955596in}{2.197369in}}%
\pgfpathlineto{\pgfqpoint{5.963328in}{2.212510in}}%
\pgfpathlineto{\pgfqpoint{5.968482in}{2.217394in}}%
\pgfpathlineto{\pgfqpoint{5.971060in}{2.177477in}}%
\pgfpathlineto{\pgfqpoint{5.973637in}{2.186690in}}%
\pgfpathlineto{\pgfqpoint{5.981368in}{2.184215in}}%
\pgfpathlineto{\pgfqpoint{5.983946in}{2.187668in}}%
\pgfpathlineto{\pgfqpoint{5.986523in}{2.206732in}}%
\pgfpathlineto{\pgfqpoint{5.989100in}{2.208292in}}%
\pgfpathlineto{\pgfqpoint{5.991677in}{2.222286in}}%
\pgfpathlineto{\pgfqpoint{5.999409in}{2.219929in}}%
\pgfpathlineto{\pgfqpoint{6.004563in}{2.245520in}}%
\pgfpathlineto{\pgfqpoint{6.007140in}{2.251593in}}%
\pgfpathlineto{\pgfqpoint{6.017449in}{2.225287in}}%
\pgfpathlineto{\pgfqpoint{6.020027in}{2.206515in}}%
\pgfpathlineto{\pgfqpoint{6.022604in}{2.211103in}}%
\pgfpathlineto{\pgfqpoint{6.025181in}{2.241418in}}%
\pgfpathlineto{\pgfqpoint{6.027758in}{2.244842in}}%
\pgfpathlineto{\pgfqpoint{6.035490in}{2.258064in}}%
\pgfpathlineto{\pgfqpoint{6.038067in}{2.276747in}}%
\pgfpathlineto{\pgfqpoint{6.040644in}{2.278151in}}%
\pgfpathlineto{\pgfqpoint{6.043221in}{2.291728in}}%
\pgfpathlineto{\pgfqpoint{6.045799in}{2.290785in}}%
\pgfpathlineto{\pgfqpoint{6.053530in}{2.281502in}}%
\pgfpathlineto{\pgfqpoint{6.056108in}{2.277223in}}%
\pgfpathlineto{\pgfqpoint{6.058685in}{2.265534in}}%
\pgfpathlineto{\pgfqpoint{6.061262in}{2.240457in}}%
\pgfpathlineto{\pgfqpoint{6.063839in}{2.238727in}}%
\pgfpathlineto{\pgfqpoint{6.071571in}{2.261391in}}%
\pgfpathlineto{\pgfqpoint{6.074148in}{2.259642in}}%
\pgfpathlineto{\pgfqpoint{6.076725in}{2.274543in}}%
\pgfpathlineto{\pgfqpoint{6.079302in}{2.239008in}}%
\pgfpathlineto{\pgfqpoint{6.081880in}{2.226636in}}%
\pgfpathlineto{\pgfqpoint{6.089611in}{2.235113in}}%
\pgfpathlineto{\pgfqpoint{6.092188in}{2.247560in}}%
\pgfpathlineto{\pgfqpoint{6.094766in}{2.234064in}}%
\pgfpathlineto{\pgfqpoint{6.097343in}{2.228562in}}%
\pgfpathlineto{\pgfqpoint{6.099920in}{2.235464in}}%
\pgfpathlineto{\pgfqpoint{6.107652in}{2.237334in}}%
\pgfpathlineto{\pgfqpoint{6.110229in}{2.247083in}}%
\pgfpathlineto{\pgfqpoint{6.112806in}{2.245168in}}%
\pgfpathlineto{\pgfqpoint{6.115383in}{2.250555in}}%
\pgfpathlineto{\pgfqpoint{6.117961in}{2.251932in}}%
\pgfpathlineto{\pgfqpoint{6.128269in}{2.215132in}}%
\pgfpathlineto{\pgfqpoint{6.130847in}{2.219693in}}%
\pgfpathlineto{\pgfqpoint{6.133424in}{2.199545in}}%
\pgfpathlineto{\pgfqpoint{6.136001in}{2.190173in}}%
\pgfpathlineto{\pgfqpoint{6.146310in}{2.231042in}}%
\pgfpathlineto{\pgfqpoint{6.148887in}{2.241193in}}%
\pgfpathlineto{\pgfqpoint{6.154042in}{2.266864in}}%
\pgfpathlineto{\pgfqpoint{6.161773in}{2.276851in}}%
\pgfpathlineto{\pgfqpoint{6.164350in}{2.294992in}}%
\pgfpathlineto{\pgfqpoint{6.166928in}{2.288407in}}%
\pgfpathlineto{\pgfqpoint{6.169505in}{2.283489in}}%
\pgfpathlineto{\pgfqpoint{6.172082in}{2.290106in}}%
\pgfpathlineto{\pgfqpoint{6.179814in}{2.287019in}}%
\pgfpathlineto{\pgfqpoint{6.184968in}{2.281218in}}%
\pgfpathlineto{\pgfqpoint{6.187545in}{2.278035in}}%
\pgfpathlineto{\pgfqpoint{6.190123in}{2.286954in}}%
\pgfpathlineto{\pgfqpoint{6.197854in}{2.317113in}}%
\pgfpathlineto{\pgfqpoint{6.200431in}{2.338096in}}%
\pgfpathlineto{\pgfqpoint{6.203009in}{2.336603in}}%
\pgfpathlineto{\pgfqpoint{6.205586in}{2.341707in}}%
\pgfpathlineto{\pgfqpoint{6.208163in}{2.342311in}}%
\pgfpathlineto{\pgfqpoint{6.221049in}{2.328664in}}%
\pgfpathlineto{\pgfqpoint{6.223626in}{2.324091in}}%
\pgfpathlineto{\pgfqpoint{6.226204in}{2.333054in}}%
\pgfpathlineto{\pgfqpoint{6.233935in}{2.340515in}}%
\pgfpathlineto{\pgfqpoint{6.239090in}{2.352993in}}%
\pgfpathlineto{\pgfqpoint{6.241667in}{2.364589in}}%
\pgfpathlineto{\pgfqpoint{6.244244in}{2.372141in}}%
\pgfpathlineto{\pgfqpoint{6.251976in}{2.350258in}}%
\pgfpathlineto{\pgfqpoint{6.254553in}{2.361032in}}%
\pgfpathlineto{\pgfqpoint{6.257130in}{2.356032in}}%
\pgfpathlineto{\pgfqpoint{6.262285in}{2.380064in}}%
\pgfpathlineto{\pgfqpoint{6.270016in}{2.355498in}}%
\pgfpathlineto{\pgfqpoint{6.272593in}{2.358004in}}%
\pgfpathlineto{\pgfqpoint{6.275171in}{2.375961in}}%
\pgfpathlineto{\pgfqpoint{6.277748in}{2.377912in}}%
\pgfpathlineto{\pgfqpoint{6.280325in}{2.378692in}}%
\pgfpathlineto{\pgfqpoint{6.288057in}{2.370854in}}%
\pgfpathlineto{\pgfqpoint{6.290634in}{2.377472in}}%
\pgfpathlineto{\pgfqpoint{6.293211in}{2.377909in}}%
\pgfpathlineto{\pgfqpoint{6.295788in}{2.361376in}}%
\pgfpathlineto{\pgfqpoint{6.298365in}{2.365546in}}%
\pgfpathlineto{\pgfqpoint{6.306097in}{2.344764in}}%
\pgfpathlineto{\pgfqpoint{6.308674in}{2.325248in}}%
\pgfpathlineto{\pgfqpoint{6.311252in}{2.314198in}}%
\pgfpathlineto{\pgfqpoint{6.313829in}{2.331745in}}%
\pgfpathlineto{\pgfqpoint{6.316406in}{2.317020in}}%
\pgfpathlineto{\pgfqpoint{6.324138in}{2.319308in}}%
\pgfpathlineto{\pgfqpoint{6.326715in}{2.332923in}}%
\pgfpathlineto{\pgfqpoint{6.329292in}{2.331779in}}%
\pgfpathlineto{\pgfqpoint{6.334446in}{2.336565in}}%
\pgfpathlineto{\pgfqpoint{6.342178in}{2.347222in}}%
\pgfpathlineto{\pgfqpoint{6.344755in}{2.381556in}}%
\pgfpathlineto{\pgfqpoint{6.347333in}{2.388436in}}%
\pgfpathlineto{\pgfqpoint{6.349910in}{2.424444in}}%
\pgfpathlineto{\pgfqpoint{6.352487in}{2.411665in}}%
\pgfpathlineto{\pgfqpoint{6.360219in}{2.421154in}}%
\pgfpathlineto{\pgfqpoint{6.362796in}{2.401197in}}%
\pgfpathlineto{\pgfqpoint{6.367950in}{2.403367in}}%
\pgfpathlineto{\pgfqpoint{6.370527in}{2.412705in}}%
\pgfpathlineto{\pgfqpoint{6.378259in}{2.430970in}}%
\pgfpathlineto{\pgfqpoint{6.380836in}{2.446308in}}%
\pgfpathlineto{\pgfqpoint{6.383414in}{2.446739in}}%
\pgfpathlineto{\pgfqpoint{6.385991in}{2.441188in}}%
\pgfpathlineto{\pgfqpoint{6.388568in}{2.475983in}}%
\pgfpathlineto{\pgfqpoint{6.396300in}{2.493680in}}%
\pgfpathlineto{\pgfqpoint{6.398877in}{2.489288in}}%
\pgfpathlineto{\pgfqpoint{6.401454in}{2.478274in}}%
\pgfpathlineto{\pgfqpoint{6.404031in}{2.473392in}}%
\pgfpathlineto{\pgfqpoint{6.416917in}{2.470043in}}%
\pgfpathlineto{\pgfqpoint{6.422072in}{2.480106in}}%
\pgfpathlineto{\pgfqpoint{6.424649in}{2.470848in}}%
\pgfpathlineto{\pgfqpoint{6.424649in}{2.470848in}}%
\pgfusepath{stroke}%
\end{pgfscope}%
\begin{pgfscope}%
\pgfpathrectangle{\pgfqpoint{0.506467in}{0.331635in}}{\pgfqpoint{6.200000in}{2.265000in}}%
\pgfusepath{clip}%
\pgfsetroundcap%
\pgfsetroundjoin%
\pgfsetlinewidth{1.505625pt}%
\definecolor{currentstroke}{rgb}{1.000000,0.498039,0.054902}%
\pgfsetstrokecolor{currentstroke}%
\pgfsetdash{}{0pt}%
\pgfpathmoveto{\pgfqpoint{0.788285in}{0.574286in}}%
\pgfpathlineto{\pgfqpoint{0.790862in}{0.578546in}}%
\pgfpathlineto{\pgfqpoint{0.793440in}{0.569348in}}%
\pgfpathlineto{\pgfqpoint{0.796017in}{0.565016in}}%
\pgfpathlineto{\pgfqpoint{0.803749in}{0.560783in}}%
\pgfpathlineto{\pgfqpoint{0.806326in}{0.560642in}}%
\pgfpathlineto{\pgfqpoint{0.808903in}{0.555509in}}%
\pgfpathlineto{\pgfqpoint{0.811480in}{0.553916in}}%
\pgfpathlineto{\pgfqpoint{0.814057in}{0.547385in}}%
\pgfpathlineto{\pgfqpoint{0.824366in}{0.547875in}}%
\pgfpathlineto{\pgfqpoint{0.826943in}{0.550729in}}%
\pgfpathlineto{\pgfqpoint{0.829521in}{0.544822in}}%
\pgfpathlineto{\pgfqpoint{0.832098in}{0.528667in}}%
\pgfpathlineto{\pgfqpoint{0.839830in}{0.523571in}}%
\pgfpathlineto{\pgfqpoint{0.842407in}{0.510046in}}%
\pgfpathlineto{\pgfqpoint{0.844984in}{0.513425in}}%
\pgfpathlineto{\pgfqpoint{0.857870in}{0.515581in}}%
\pgfpathlineto{\pgfqpoint{0.860447in}{0.519966in}}%
\pgfpathlineto{\pgfqpoint{0.863024in}{0.512155in}}%
\pgfpathlineto{\pgfqpoint{0.865602in}{0.502075in}}%
\pgfpathlineto{\pgfqpoint{0.868179in}{0.503656in}}%
\pgfpathlineto{\pgfqpoint{0.875910in}{0.504350in}}%
\pgfpathlineto{\pgfqpoint{0.878488in}{0.501622in}}%
\pgfpathlineto{\pgfqpoint{0.881065in}{0.504255in}}%
\pgfpathlineto{\pgfqpoint{0.883642in}{0.490657in}}%
\pgfpathlineto{\pgfqpoint{0.886219in}{0.482213in}}%
\pgfpathlineto{\pgfqpoint{0.893951in}{0.489016in}}%
\pgfpathlineto{\pgfqpoint{0.896528in}{0.498620in}}%
\pgfpathlineto{\pgfqpoint{0.899105in}{0.490410in}}%
\pgfpathlineto{\pgfqpoint{0.901683in}{0.497355in}}%
\pgfpathlineto{\pgfqpoint{0.914569in}{0.494708in}}%
\pgfpathlineto{\pgfqpoint{0.917146in}{0.490397in}}%
\pgfpathlineto{\pgfqpoint{0.919723in}{0.488239in}}%
\pgfpathlineto{\pgfqpoint{0.922300in}{0.490330in}}%
\pgfpathlineto{\pgfqpoint{0.930032in}{0.486938in}}%
\pgfpathlineto{\pgfqpoint{0.932609in}{0.494227in}}%
\pgfpathlineto{\pgfqpoint{0.935186in}{0.499384in}}%
\pgfpathlineto{\pgfqpoint{0.937764in}{0.508975in}}%
\pgfpathlineto{\pgfqpoint{0.940341in}{0.512385in}}%
\pgfpathlineto{\pgfqpoint{0.950650in}{0.515655in}}%
\pgfpathlineto{\pgfqpoint{0.953227in}{0.518406in}}%
\pgfpathlineto{\pgfqpoint{0.955804in}{0.513854in}}%
\pgfpathlineto{\pgfqpoint{0.958381in}{0.518624in}}%
\pgfpathlineto{\pgfqpoint{0.966113in}{0.519008in}}%
\pgfpathlineto{\pgfqpoint{0.968690in}{0.524730in}}%
\pgfpathlineto{\pgfqpoint{0.971267in}{0.526262in}}%
\pgfpathlineto{\pgfqpoint{0.973845in}{0.517741in}}%
\pgfpathlineto{\pgfqpoint{0.976422in}{0.514654in}}%
\pgfpathlineto{\pgfqpoint{0.984153in}{0.512718in}}%
\pgfpathlineto{\pgfqpoint{0.986731in}{0.515753in}}%
\pgfpathlineto{\pgfqpoint{0.989308in}{0.514101in}}%
\pgfpathlineto{\pgfqpoint{0.991885in}{0.509923in}}%
\pgfpathlineto{\pgfqpoint{0.994462in}{0.504264in}}%
\pgfpathlineto{\pgfqpoint{1.002194in}{0.499130in}}%
\pgfpathlineto{\pgfqpoint{1.004771in}{0.496268in}}%
\pgfpathlineto{\pgfqpoint{1.007348in}{0.498810in}}%
\pgfpathlineto{\pgfqpoint{1.009926in}{0.504882in}}%
\pgfpathlineto{\pgfqpoint{1.012503in}{0.508345in}}%
\pgfpathlineto{\pgfqpoint{1.020234in}{0.514178in}}%
\pgfpathlineto{\pgfqpoint{1.022812in}{0.510688in}}%
\pgfpathlineto{\pgfqpoint{1.025389in}{0.515318in}}%
\pgfpathlineto{\pgfqpoint{1.027966in}{0.526396in}}%
\pgfpathlineto{\pgfqpoint{1.038275in}{0.527359in}}%
\pgfpathlineto{\pgfqpoint{1.040852in}{0.520963in}}%
\pgfpathlineto{\pgfqpoint{1.043429in}{0.523438in}}%
\pgfpathlineto{\pgfqpoint{1.046007in}{0.515465in}}%
\pgfpathlineto{\pgfqpoint{1.048584in}{0.509895in}}%
\pgfpathlineto{\pgfqpoint{1.056315in}{0.511745in}}%
\pgfpathlineto{\pgfqpoint{1.058893in}{0.520956in}}%
\pgfpathlineto{\pgfqpoint{1.061470in}{0.519314in}}%
\pgfpathlineto{\pgfqpoint{1.066624in}{0.510195in}}%
\pgfpathlineto{\pgfqpoint{1.074356in}{0.513179in}}%
\pgfpathlineto{\pgfqpoint{1.076933in}{0.509169in}}%
\pgfpathlineto{\pgfqpoint{1.079510in}{0.516977in}}%
\pgfpathlineto{\pgfqpoint{1.082087in}{0.517807in}}%
\pgfpathlineto{\pgfqpoint{1.084665in}{0.503194in}}%
\pgfpathlineto{\pgfqpoint{1.092396in}{0.498650in}}%
\pgfpathlineto{\pgfqpoint{1.097551in}{0.501513in}}%
\pgfpathlineto{\pgfqpoint{1.100128in}{0.490550in}}%
\pgfpathlineto{\pgfqpoint{1.102705in}{0.494411in}}%
\pgfpathlineto{\pgfqpoint{1.110437in}{0.497380in}}%
\pgfpathlineto{\pgfqpoint{1.115591in}{0.487217in}}%
\pgfpathlineto{\pgfqpoint{1.120746in}{0.493317in}}%
\pgfpathlineto{\pgfqpoint{1.128477in}{0.498490in}}%
\pgfpathlineto{\pgfqpoint{1.131055in}{0.494473in}}%
\pgfpathlineto{\pgfqpoint{1.133632in}{0.507171in}}%
\pgfpathlineto{\pgfqpoint{1.136209in}{0.506433in}}%
\pgfpathlineto{\pgfqpoint{1.138786in}{0.498425in}}%
\pgfpathlineto{\pgfqpoint{1.146518in}{0.512710in}}%
\pgfpathlineto{\pgfqpoint{1.151672in}{0.497959in}}%
\pgfpathlineto{\pgfqpoint{1.154249in}{0.499132in}}%
\pgfpathlineto{\pgfqpoint{1.156827in}{0.498166in}}%
\pgfpathlineto{\pgfqpoint{1.167135in}{0.496218in}}%
\pgfpathlineto{\pgfqpoint{1.169713in}{0.499351in}}%
\pgfpathlineto{\pgfqpoint{1.172290in}{0.497638in}}%
\pgfpathlineto{\pgfqpoint{1.174867in}{0.503601in}}%
\pgfpathlineto{\pgfqpoint{1.182599in}{0.495269in}}%
\pgfpathlineto{\pgfqpoint{1.185176in}{0.496031in}}%
\pgfpathlineto{\pgfqpoint{1.187753in}{0.498784in}}%
\pgfpathlineto{\pgfqpoint{1.190330in}{0.494213in}}%
\pgfpathlineto{\pgfqpoint{1.192908in}{0.505320in}}%
\pgfpathlineto{\pgfqpoint{1.200639in}{0.507728in}}%
\pgfpathlineto{\pgfqpoint{1.203216in}{0.513873in}}%
\pgfpathlineto{\pgfqpoint{1.205794in}{0.516848in}}%
\pgfpathlineto{\pgfqpoint{1.208371in}{0.510370in}}%
\pgfpathlineto{\pgfqpoint{1.210948in}{0.520947in}}%
\pgfpathlineto{\pgfqpoint{1.218680in}{0.526294in}}%
\pgfpathlineto{\pgfqpoint{1.221257in}{0.520595in}}%
\pgfpathlineto{\pgfqpoint{1.223834in}{0.507445in}}%
\pgfpathlineto{\pgfqpoint{1.228989in}{0.530721in}}%
\pgfpathlineto{\pgfqpoint{1.236720in}{0.535930in}}%
\pgfpathlineto{\pgfqpoint{1.239297in}{0.539312in}}%
\pgfpathlineto{\pgfqpoint{1.241875in}{0.538051in}}%
\pgfpathlineto{\pgfqpoint{1.244452in}{0.534468in}}%
\pgfpathlineto{\pgfqpoint{1.247029in}{0.544477in}}%
\pgfpathlineto{\pgfqpoint{1.254761in}{0.552157in}}%
\pgfpathlineto{\pgfqpoint{1.257338in}{0.551987in}}%
\pgfpathlineto{\pgfqpoint{1.262492in}{0.549480in}}%
\pgfpathlineto{\pgfqpoint{1.272801in}{0.546930in}}%
\pgfpathlineto{\pgfqpoint{1.275378in}{0.542570in}}%
\pgfpathlineto{\pgfqpoint{1.277956in}{0.543036in}}%
\pgfpathlineto{\pgfqpoint{1.280533in}{0.550793in}}%
\pgfpathlineto{\pgfqpoint{1.283110in}{0.552268in}}%
\pgfpathlineto{\pgfqpoint{1.290842in}{0.563817in}}%
\pgfpathlineto{\pgfqpoint{1.293419in}{0.573176in}}%
\pgfpathlineto{\pgfqpoint{1.295996in}{0.569615in}}%
\pgfpathlineto{\pgfqpoint{1.298573in}{0.553718in}}%
\pgfpathlineto{\pgfqpoint{1.301151in}{0.561912in}}%
\pgfpathlineto{\pgfqpoint{1.311459in}{0.551984in}}%
\pgfpathlineto{\pgfqpoint{1.314037in}{0.550487in}}%
\pgfpathlineto{\pgfqpoint{1.316614in}{0.556701in}}%
\pgfpathlineto{\pgfqpoint{1.319191in}{0.554941in}}%
\pgfpathlineto{\pgfqpoint{1.326923in}{0.553104in}}%
\pgfpathlineto{\pgfqpoint{1.329500in}{0.558455in}}%
\pgfpathlineto{\pgfqpoint{1.332077in}{0.551810in}}%
\pgfpathlineto{\pgfqpoint{1.334654in}{0.556493in}}%
\pgfpathlineto{\pgfqpoint{1.337232in}{0.549036in}}%
\pgfpathlineto{\pgfqpoint{1.350118in}{0.548008in}}%
\pgfpathlineto{\pgfqpoint{1.352695in}{0.550692in}}%
\pgfpathlineto{\pgfqpoint{1.355272in}{0.544433in}}%
\pgfpathlineto{\pgfqpoint{1.363004in}{0.548754in}}%
\pgfpathlineto{\pgfqpoint{1.365581in}{0.547934in}}%
\pgfpathlineto{\pgfqpoint{1.373312in}{0.534084in}}%
\pgfpathlineto{\pgfqpoint{1.381044in}{0.533966in}}%
\pgfpathlineto{\pgfqpoint{1.386199in}{0.528946in}}%
\pgfpathlineto{\pgfqpoint{1.388776in}{0.521734in}}%
\pgfpathlineto{\pgfqpoint{1.391353in}{0.518876in}}%
\pgfpathlineto{\pgfqpoint{1.401662in}{0.525361in}}%
\pgfpathlineto{\pgfqpoint{1.404239in}{0.533895in}}%
\pgfpathlineto{\pgfqpoint{1.406816in}{0.530580in}}%
\pgfpathlineto{\pgfqpoint{1.409393in}{0.538595in}}%
\pgfpathlineto{\pgfqpoint{1.419702in}{0.540375in}}%
\pgfpathlineto{\pgfqpoint{1.422280in}{0.545784in}}%
\pgfpathlineto{\pgfqpoint{1.424857in}{0.534890in}}%
\pgfpathlineto{\pgfqpoint{1.427434in}{0.540847in}}%
\pgfpathlineto{\pgfqpoint{1.435166in}{0.533906in}}%
\pgfpathlineto{\pgfqpoint{1.437743in}{0.535299in}}%
\pgfpathlineto{\pgfqpoint{1.440320in}{0.531284in}}%
\pgfpathlineto{\pgfqpoint{1.442897in}{0.529478in}}%
\pgfpathlineto{\pgfqpoint{1.445474in}{0.521962in}}%
\pgfpathlineto{\pgfqpoint{1.453206in}{0.523119in}}%
\pgfpathlineto{\pgfqpoint{1.455783in}{0.516797in}}%
\pgfpathlineto{\pgfqpoint{1.463515in}{0.503142in}}%
\pgfpathlineto{\pgfqpoint{1.471247in}{0.505951in}}%
\pgfpathlineto{\pgfqpoint{1.473824in}{0.508194in}}%
\pgfpathlineto{\pgfqpoint{1.476401in}{0.515107in}}%
\pgfpathlineto{\pgfqpoint{1.478978in}{0.513555in}}%
\pgfpathlineto{\pgfqpoint{1.481555in}{0.519716in}}%
\pgfpathlineto{\pgfqpoint{1.489287in}{0.509239in}}%
\pgfpathlineto{\pgfqpoint{1.491864in}{0.511427in}}%
\pgfpathlineto{\pgfqpoint{1.494441in}{0.518822in}}%
\pgfpathlineto{\pgfqpoint{1.497019in}{0.515405in}}%
\pgfpathlineto{\pgfqpoint{1.499596in}{0.515092in}}%
\pgfpathlineto{\pgfqpoint{1.507328in}{0.520969in}}%
\pgfpathlineto{\pgfqpoint{1.509905in}{0.519421in}}%
\pgfpathlineto{\pgfqpoint{1.512482in}{0.519694in}}%
\pgfpathlineto{\pgfqpoint{1.515059in}{0.518307in}}%
\pgfpathlineto{\pgfqpoint{1.517636in}{0.518793in}}%
\pgfpathlineto{\pgfqpoint{1.525368in}{0.513310in}}%
\pgfpathlineto{\pgfqpoint{1.527945in}{0.505685in}}%
\pgfpathlineto{\pgfqpoint{1.530522in}{0.518112in}}%
\pgfpathlineto{\pgfqpoint{1.533100in}{0.508469in}}%
\pgfpathlineto{\pgfqpoint{1.535677in}{0.505803in}}%
\pgfpathlineto{\pgfqpoint{1.543409in}{0.500805in}}%
\pgfpathlineto{\pgfqpoint{1.545986in}{0.491887in}}%
\pgfpathlineto{\pgfqpoint{1.548563in}{0.488285in}}%
\pgfpathlineto{\pgfqpoint{1.551140in}{0.482866in}}%
\pgfpathlineto{\pgfqpoint{1.553717in}{0.484469in}}%
\pgfpathlineto{\pgfqpoint{1.566603in}{0.487502in}}%
\pgfpathlineto{\pgfqpoint{1.569181in}{0.484503in}}%
\pgfpathlineto{\pgfqpoint{1.571758in}{0.485460in}}%
\pgfpathlineto{\pgfqpoint{1.579489in}{0.482052in}}%
\pgfpathlineto{\pgfqpoint{1.582067in}{0.483065in}}%
\pgfpathlineto{\pgfqpoint{1.587221in}{0.469512in}}%
\pgfpathlineto{\pgfqpoint{1.589798in}{0.485366in}}%
\pgfpathlineto{\pgfqpoint{1.597530in}{0.493460in}}%
\pgfpathlineto{\pgfqpoint{1.600107in}{0.482458in}}%
\pgfpathlineto{\pgfqpoint{1.602684in}{0.483836in}}%
\pgfpathlineto{\pgfqpoint{1.605262in}{0.487250in}}%
\pgfpathlineto{\pgfqpoint{1.607839in}{0.486027in}}%
\pgfpathlineto{\pgfqpoint{1.615570in}{0.488064in}}%
\pgfpathlineto{\pgfqpoint{1.618148in}{0.479060in}}%
\pgfpathlineto{\pgfqpoint{1.620725in}{0.475490in}}%
\pgfpathlineto{\pgfqpoint{1.625879in}{0.474560in}}%
\pgfpathlineto{\pgfqpoint{1.633611in}{0.465347in}}%
\pgfpathlineto{\pgfqpoint{1.636188in}{0.454977in}}%
\pgfpathlineto{\pgfqpoint{1.638765in}{0.460140in}}%
\pgfpathlineto{\pgfqpoint{1.641343in}{0.456901in}}%
\pgfpathlineto{\pgfqpoint{1.643920in}{0.458092in}}%
\pgfpathlineto{\pgfqpoint{1.651651in}{0.458910in}}%
\pgfpathlineto{\pgfqpoint{1.654229in}{0.463277in}}%
\pgfpathlineto{\pgfqpoint{1.661960in}{0.479512in}}%
\pgfpathlineto{\pgfqpoint{1.669692in}{0.473789in}}%
\pgfpathlineto{\pgfqpoint{1.672269in}{0.477530in}}%
\pgfpathlineto{\pgfqpoint{1.677424in}{0.474580in}}%
\pgfpathlineto{\pgfqpoint{1.680001in}{0.476873in}}%
\pgfpathlineto{\pgfqpoint{1.687732in}{0.479177in}}%
\pgfpathlineto{\pgfqpoint{1.690310in}{0.477857in}}%
\pgfpathlineto{\pgfqpoint{1.692887in}{0.459745in}}%
\pgfpathlineto{\pgfqpoint{1.695464in}{0.466933in}}%
\pgfpathlineto{\pgfqpoint{1.698041in}{0.470092in}}%
\pgfpathlineto{\pgfqpoint{1.705773in}{0.469256in}}%
\pgfpathlineto{\pgfqpoint{1.713505in}{0.473503in}}%
\pgfpathlineto{\pgfqpoint{1.716082in}{0.476950in}}%
\pgfpathlineto{\pgfqpoint{1.723813in}{0.476981in}}%
\pgfpathlineto{\pgfqpoint{1.728968in}{0.474441in}}%
\pgfpathlineto{\pgfqpoint{1.734122in}{0.467297in}}%
\pgfpathlineto{\pgfqpoint{1.744431in}{0.477115in}}%
\pgfpathlineto{\pgfqpoint{1.747008in}{0.475171in}}%
\pgfpathlineto{\pgfqpoint{1.749586in}{0.476962in}}%
\pgfpathlineto{\pgfqpoint{1.759894in}{0.473879in}}%
\pgfpathlineto{\pgfqpoint{1.767626in}{0.456582in}}%
\pgfpathlineto{\pgfqpoint{1.770203in}{0.468153in}}%
\pgfpathlineto{\pgfqpoint{1.780512in}{0.462851in}}%
\pgfpathlineto{\pgfqpoint{1.783089in}{0.455142in}}%
\pgfpathlineto{\pgfqpoint{1.785666in}{0.456788in}}%
\pgfpathlineto{\pgfqpoint{1.788244in}{0.453553in}}%
\pgfpathlineto{\pgfqpoint{1.795975in}{0.454651in}}%
\pgfpathlineto{\pgfqpoint{1.798553in}{0.452141in}}%
\pgfpathlineto{\pgfqpoint{1.801130in}{0.447708in}}%
\pgfpathlineto{\pgfqpoint{1.803707in}{0.458601in}}%
\pgfpathlineto{\pgfqpoint{1.806284in}{0.465037in}}%
\pgfpathlineto{\pgfqpoint{1.814016in}{0.462394in}}%
\pgfpathlineto{\pgfqpoint{1.816593in}{0.465202in}}%
\pgfpathlineto{\pgfqpoint{1.819170in}{0.464928in}}%
\pgfpathlineto{\pgfqpoint{1.821747in}{0.465672in}}%
\pgfpathlineto{\pgfqpoint{1.824325in}{0.470979in}}%
\pgfpathlineto{\pgfqpoint{1.832056in}{0.471131in}}%
\pgfpathlineto{\pgfqpoint{1.834634in}{0.469985in}}%
\pgfpathlineto{\pgfqpoint{1.839788in}{0.473798in}}%
\pgfpathlineto{\pgfqpoint{1.842365in}{0.477562in}}%
\pgfpathlineto{\pgfqpoint{1.852674in}{0.480179in}}%
\pgfpathlineto{\pgfqpoint{1.855251in}{0.476243in}}%
\pgfpathlineto{\pgfqpoint{1.857828in}{0.465446in}}%
\pgfpathlineto{\pgfqpoint{1.860406in}{0.465962in}}%
\pgfpathlineto{\pgfqpoint{1.868137in}{0.468477in}}%
\pgfpathlineto{\pgfqpoint{1.870714in}{0.472401in}}%
\pgfpathlineto{\pgfqpoint{1.873292in}{0.469030in}}%
\pgfpathlineto{\pgfqpoint{1.875869in}{0.470100in}}%
\pgfpathlineto{\pgfqpoint{1.878446in}{0.469656in}}%
\pgfpathlineto{\pgfqpoint{1.886178in}{0.464528in}}%
\pgfpathlineto{\pgfqpoint{1.888755in}{0.469584in}}%
\pgfpathlineto{\pgfqpoint{1.891332in}{0.468881in}}%
\pgfpathlineto{\pgfqpoint{1.893909in}{0.470015in}}%
\pgfpathlineto{\pgfqpoint{1.896487in}{0.474582in}}%
\pgfpathlineto{\pgfqpoint{1.904218in}{0.475463in}}%
\pgfpathlineto{\pgfqpoint{1.906795in}{0.476385in}}%
\pgfpathlineto{\pgfqpoint{1.909373in}{0.481607in}}%
\pgfpathlineto{\pgfqpoint{1.922259in}{0.486826in}}%
\pgfpathlineto{\pgfqpoint{1.924836in}{0.478261in}}%
\pgfpathlineto{\pgfqpoint{1.927413in}{0.489200in}}%
\pgfpathlineto{\pgfqpoint{1.932568in}{0.499145in}}%
\pgfpathlineto{\pgfqpoint{1.940299in}{0.491112in}}%
\pgfpathlineto{\pgfqpoint{1.945454in}{0.484082in}}%
\pgfpathlineto{\pgfqpoint{1.948031in}{0.485241in}}%
\pgfpathlineto{\pgfqpoint{1.958340in}{0.485745in}}%
\pgfpathlineto{\pgfqpoint{1.960917in}{0.479361in}}%
\pgfpathlineto{\pgfqpoint{1.963494in}{0.478502in}}%
\pgfpathlineto{\pgfqpoint{1.966071in}{0.481409in}}%
\pgfpathlineto{\pgfqpoint{1.968649in}{0.488599in}}%
\pgfpathlineto{\pgfqpoint{1.981535in}{0.480424in}}%
\pgfpathlineto{\pgfqpoint{1.984112in}{0.489012in}}%
\pgfpathlineto{\pgfqpoint{1.986689in}{0.482908in}}%
\pgfpathlineto{\pgfqpoint{1.994421in}{0.475636in}}%
\pgfpathlineto{\pgfqpoint{1.996998in}{0.488673in}}%
\pgfpathlineto{\pgfqpoint{1.999575in}{0.481316in}}%
\pgfpathlineto{\pgfqpoint{2.002152in}{0.483858in}}%
\pgfpathlineto{\pgfqpoint{2.004730in}{0.479094in}}%
\pgfpathlineto{\pgfqpoint{2.012461in}{0.476496in}}%
\pgfpathlineto{\pgfqpoint{2.015038in}{0.472783in}}%
\pgfpathlineto{\pgfqpoint{2.017616in}{0.461784in}}%
\pgfpathlineto{\pgfqpoint{2.020193in}{0.466218in}}%
\pgfpathlineto{\pgfqpoint{2.022770in}{0.462747in}}%
\pgfpathlineto{\pgfqpoint{2.030502in}{0.464502in}}%
\pgfpathlineto{\pgfqpoint{2.033079in}{0.469017in}}%
\pgfpathlineto{\pgfqpoint{2.035656in}{0.467403in}}%
\pgfpathlineto{\pgfqpoint{2.038233in}{0.479107in}}%
\pgfpathlineto{\pgfqpoint{2.040811in}{0.469921in}}%
\pgfpathlineto{\pgfqpoint{2.048542in}{0.466712in}}%
\pgfpathlineto{\pgfqpoint{2.053697in}{0.470697in}}%
\pgfpathlineto{\pgfqpoint{2.056274in}{0.476601in}}%
\pgfpathlineto{\pgfqpoint{2.058851in}{0.475210in}}%
\pgfpathlineto{\pgfqpoint{2.066583in}{0.473371in}}%
\pgfpathlineto{\pgfqpoint{2.069160in}{0.470895in}}%
\pgfpathlineto{\pgfqpoint{2.071737in}{0.470246in}}%
\pgfpathlineto{\pgfqpoint{2.074314in}{0.468352in}}%
\pgfpathlineto{\pgfqpoint{2.076891in}{0.465369in}}%
\pgfpathlineto{\pgfqpoint{2.084623in}{0.464998in}}%
\pgfpathlineto{\pgfqpoint{2.087200in}{0.459748in}}%
\pgfpathlineto{\pgfqpoint{2.089778in}{0.458268in}}%
\pgfpathlineto{\pgfqpoint{2.094932in}{0.464212in}}%
\pgfpathlineto{\pgfqpoint{2.105241in}{0.467510in}}%
\pgfpathlineto{\pgfqpoint{2.107818in}{0.455782in}}%
\pgfpathlineto{\pgfqpoint{2.110395in}{0.454778in}}%
\pgfpathlineto{\pgfqpoint{2.120704in}{0.443973in}}%
\pgfpathlineto{\pgfqpoint{2.123281in}{0.437638in}}%
\pgfpathlineto{\pgfqpoint{2.125859in}{0.445517in}}%
\pgfpathlineto{\pgfqpoint{2.131013in}{0.437723in}}%
\pgfpathlineto{\pgfqpoint{2.138745in}{0.450499in}}%
\pgfpathlineto{\pgfqpoint{2.141322in}{0.456856in}}%
\pgfpathlineto{\pgfqpoint{2.143899in}{0.454654in}}%
\pgfpathlineto{\pgfqpoint{2.146476in}{0.455374in}}%
\pgfpathlineto{\pgfqpoint{2.149053in}{0.447508in}}%
\pgfpathlineto{\pgfqpoint{2.156785in}{0.453875in}}%
\pgfpathlineto{\pgfqpoint{2.159362in}{0.458265in}}%
\pgfpathlineto{\pgfqpoint{2.161940in}{0.460811in}}%
\pgfpathlineto{\pgfqpoint{2.164517in}{0.460627in}}%
\pgfpathlineto{\pgfqpoint{2.167094in}{0.463356in}}%
\pgfpathlineto{\pgfqpoint{2.174826in}{0.458007in}}%
\pgfpathlineto{\pgfqpoint{2.179980in}{0.462547in}}%
\pgfpathlineto{\pgfqpoint{2.182557in}{0.464533in}}%
\pgfpathlineto{\pgfqpoint{2.185134in}{0.467760in}}%
\pgfpathlineto{\pgfqpoint{2.192866in}{0.471320in}}%
\pgfpathlineto{\pgfqpoint{2.195443in}{0.468958in}}%
\pgfpathlineto{\pgfqpoint{2.198020in}{0.467809in}}%
\pgfpathlineto{\pgfqpoint{2.203175in}{0.467363in}}%
\pgfpathlineto{\pgfqpoint{2.210907in}{0.474636in}}%
\pgfpathlineto{\pgfqpoint{2.213484in}{0.469240in}}%
\pgfpathlineto{\pgfqpoint{2.216061in}{0.466796in}}%
\pgfpathlineto{\pgfqpoint{2.218638in}{0.462124in}}%
\pgfpathlineto{\pgfqpoint{2.221215in}{0.467895in}}%
\pgfpathlineto{\pgfqpoint{2.228947in}{0.465471in}}%
\pgfpathlineto{\pgfqpoint{2.231524in}{0.463975in}}%
\pgfpathlineto{\pgfqpoint{2.234101in}{0.465245in}}%
\pgfpathlineto{\pgfqpoint{2.236679in}{0.445825in}}%
\pgfpathlineto{\pgfqpoint{2.239256in}{0.443298in}}%
\pgfpathlineto{\pgfqpoint{2.246988in}{0.434590in}}%
\pgfpathlineto{\pgfqpoint{2.249565in}{0.439808in}}%
\pgfpathlineto{\pgfqpoint{2.252142in}{0.437293in}}%
\pgfpathlineto{\pgfqpoint{2.254719in}{0.447069in}}%
\pgfpathlineto{\pgfqpoint{2.257296in}{0.449946in}}%
\pgfpathlineto{\pgfqpoint{2.265028in}{0.448946in}}%
\pgfpathlineto{\pgfqpoint{2.267605in}{0.454468in}}%
\pgfpathlineto{\pgfqpoint{2.270182in}{0.451601in}}%
\pgfpathlineto{\pgfqpoint{2.272760in}{0.456990in}}%
\pgfpathlineto{\pgfqpoint{2.275337in}{0.460169in}}%
\pgfpathlineto{\pgfqpoint{2.283068in}{0.461008in}}%
\pgfpathlineto{\pgfqpoint{2.288223in}{0.469027in}}%
\pgfpathlineto{\pgfqpoint{2.290800in}{0.469669in}}%
\pgfpathlineto{\pgfqpoint{2.293377in}{0.472312in}}%
\pgfpathlineto{\pgfqpoint{2.301109in}{0.465435in}}%
\pgfpathlineto{\pgfqpoint{2.303686in}{0.473627in}}%
\pgfpathlineto{\pgfqpoint{2.306263in}{0.483863in}}%
\pgfpathlineto{\pgfqpoint{2.308841in}{0.488985in}}%
\pgfpathlineto{\pgfqpoint{2.311418in}{0.492498in}}%
\pgfpathlineto{\pgfqpoint{2.319149in}{0.494771in}}%
\pgfpathlineto{\pgfqpoint{2.321727in}{0.493843in}}%
\pgfpathlineto{\pgfqpoint{2.324304in}{0.508855in}}%
\pgfpathlineto{\pgfqpoint{2.326881in}{0.505880in}}%
\pgfpathlineto{\pgfqpoint{2.329458in}{0.513996in}}%
\pgfpathlineto{\pgfqpoint{2.337190in}{0.521262in}}%
\pgfpathlineto{\pgfqpoint{2.339767in}{0.529192in}}%
\pgfpathlineto{\pgfqpoint{2.342344in}{0.530959in}}%
\pgfpathlineto{\pgfqpoint{2.344922in}{0.544470in}}%
\pgfpathlineto{\pgfqpoint{2.347499in}{0.544366in}}%
\pgfpathlineto{\pgfqpoint{2.357808in}{0.541449in}}%
\pgfpathlineto{\pgfqpoint{2.360385in}{0.547672in}}%
\pgfpathlineto{\pgfqpoint{2.362962in}{0.547706in}}%
\pgfpathlineto{\pgfqpoint{2.365539in}{0.549029in}}%
\pgfpathlineto{\pgfqpoint{2.373271in}{0.540438in}}%
\pgfpathlineto{\pgfqpoint{2.375848in}{0.549416in}}%
\pgfpathlineto{\pgfqpoint{2.378425in}{0.542280in}}%
\pgfpathlineto{\pgfqpoint{2.381003in}{0.537702in}}%
\pgfpathlineto{\pgfqpoint{2.383580in}{0.539516in}}%
\pgfpathlineto{\pgfqpoint{2.391311in}{0.546309in}}%
\pgfpathlineto{\pgfqpoint{2.393889in}{0.552288in}}%
\pgfpathlineto{\pgfqpoint{2.396466in}{0.552092in}}%
\pgfpathlineto{\pgfqpoint{2.399043in}{0.546011in}}%
\pgfpathlineto{\pgfqpoint{2.401620in}{0.536678in}}%
\pgfpathlineto{\pgfqpoint{2.409352in}{0.547734in}}%
\pgfpathlineto{\pgfqpoint{2.411929in}{0.541707in}}%
\pgfpathlineto{\pgfqpoint{2.414506in}{0.538663in}}%
\pgfpathlineto{\pgfqpoint{2.417084in}{0.539923in}}%
\pgfpathlineto{\pgfqpoint{2.419661in}{0.537495in}}%
\pgfpathlineto{\pgfqpoint{2.427392in}{0.541117in}}%
\pgfpathlineto{\pgfqpoint{2.429970in}{0.540645in}}%
\pgfpathlineto{\pgfqpoint{2.432547in}{0.534509in}}%
\pgfpathlineto{\pgfqpoint{2.435124in}{0.535330in}}%
\pgfpathlineto{\pgfqpoint{2.437701in}{0.543450in}}%
\pgfpathlineto{\pgfqpoint{2.445433in}{0.553935in}}%
\pgfpathlineto{\pgfqpoint{2.448010in}{0.553951in}}%
\pgfpathlineto{\pgfqpoint{2.450587in}{0.557250in}}%
\pgfpathlineto{\pgfqpoint{2.453165in}{0.563620in}}%
\pgfpathlineto{\pgfqpoint{2.455742in}{0.561378in}}%
\pgfpathlineto{\pgfqpoint{2.466051in}{0.554446in}}%
\pgfpathlineto{\pgfqpoint{2.468628in}{0.554523in}}%
\pgfpathlineto{\pgfqpoint{2.471205in}{0.543863in}}%
\pgfpathlineto{\pgfqpoint{2.473782in}{0.537705in}}%
\pgfpathlineto{\pgfqpoint{2.481514in}{0.524700in}}%
\pgfpathlineto{\pgfqpoint{2.484091in}{0.532304in}}%
\pgfpathlineto{\pgfqpoint{2.486668in}{0.531014in}}%
\pgfpathlineto{\pgfqpoint{2.489245in}{0.535010in}}%
\pgfpathlineto{\pgfqpoint{2.491823in}{0.533300in}}%
\pgfpathlineto{\pgfqpoint{2.499554in}{0.537447in}}%
\pgfpathlineto{\pgfqpoint{2.502132in}{0.532276in}}%
\pgfpathlineto{\pgfqpoint{2.504709in}{0.534703in}}%
\pgfpathlineto{\pgfqpoint{2.507286in}{0.534728in}}%
\pgfpathlineto{\pgfqpoint{2.509863in}{0.536294in}}%
\pgfpathlineto{\pgfqpoint{2.517595in}{0.544836in}}%
\pgfpathlineto{\pgfqpoint{2.520172in}{0.553314in}}%
\pgfpathlineto{\pgfqpoint{2.522749in}{0.551360in}}%
\pgfpathlineto{\pgfqpoint{2.525326in}{0.546974in}}%
\pgfpathlineto{\pgfqpoint{2.527904in}{0.548753in}}%
\pgfpathlineto{\pgfqpoint{2.535635in}{0.549862in}}%
\pgfpathlineto{\pgfqpoint{2.538213in}{0.547785in}}%
\pgfpathlineto{\pgfqpoint{2.540790in}{0.547886in}}%
\pgfpathlineto{\pgfqpoint{2.543367in}{0.546251in}}%
\pgfpathlineto{\pgfqpoint{2.545944in}{0.550941in}}%
\pgfpathlineto{\pgfqpoint{2.553676in}{0.555328in}}%
\pgfpathlineto{\pgfqpoint{2.556253in}{0.549416in}}%
\pgfpathlineto{\pgfqpoint{2.558830in}{0.550756in}}%
\pgfpathlineto{\pgfqpoint{2.561407in}{0.553876in}}%
\pgfpathlineto{\pgfqpoint{2.563985in}{0.572898in}}%
\pgfpathlineto{\pgfqpoint{2.571716in}{0.572840in}}%
\pgfpathlineto{\pgfqpoint{2.574293in}{0.581363in}}%
\pgfpathlineto{\pgfqpoint{2.582025in}{0.580306in}}%
\pgfpathlineto{\pgfqpoint{2.589757in}{0.594387in}}%
\pgfpathlineto{\pgfqpoint{2.592334in}{0.603731in}}%
\pgfpathlineto{\pgfqpoint{2.594911in}{0.609148in}}%
\pgfpathlineto{\pgfqpoint{2.597488in}{0.600941in}}%
\pgfpathlineto{\pgfqpoint{2.600066in}{0.598777in}}%
\pgfpathlineto{\pgfqpoint{2.607797in}{0.596470in}}%
\pgfpathlineto{\pgfqpoint{2.610374in}{0.598903in}}%
\pgfpathlineto{\pgfqpoint{2.612952in}{0.622927in}}%
\pgfpathlineto{\pgfqpoint{2.615529in}{0.631371in}}%
\pgfpathlineto{\pgfqpoint{2.618106in}{0.634934in}}%
\pgfpathlineto{\pgfqpoint{2.625838in}{0.638701in}}%
\pgfpathlineto{\pgfqpoint{2.628415in}{0.634855in}}%
\pgfpathlineto{\pgfqpoint{2.630992in}{0.632800in}}%
\pgfpathlineto{\pgfqpoint{2.633569in}{0.632464in}}%
\pgfpathlineto{\pgfqpoint{2.636147in}{0.634847in}}%
\pgfpathlineto{\pgfqpoint{2.643878in}{0.639245in}}%
\pgfpathlineto{\pgfqpoint{2.646455in}{0.633883in}}%
\pgfpathlineto{\pgfqpoint{2.651610in}{0.632518in}}%
\pgfpathlineto{\pgfqpoint{2.654187in}{0.634222in}}%
\pgfpathlineto{\pgfqpoint{2.661919in}{0.646194in}}%
\pgfpathlineto{\pgfqpoint{2.664496in}{0.644452in}}%
\pgfpathlineto{\pgfqpoint{2.669650in}{0.647495in}}%
\pgfpathlineto{\pgfqpoint{2.672228in}{0.653064in}}%
\pgfpathlineto{\pgfqpoint{2.679959in}{0.652266in}}%
\pgfpathlineto{\pgfqpoint{2.682536in}{0.642559in}}%
\pgfpathlineto{\pgfqpoint{2.685114in}{0.643432in}}%
\pgfpathlineto{\pgfqpoint{2.687691in}{0.645215in}}%
\pgfpathlineto{\pgfqpoint{2.690268in}{0.646256in}}%
\pgfpathlineto{\pgfqpoint{2.698000in}{0.647201in}}%
\pgfpathlineto{\pgfqpoint{2.700577in}{0.638782in}}%
\pgfpathlineto{\pgfqpoint{2.703154in}{0.647250in}}%
\pgfpathlineto{\pgfqpoint{2.705731in}{0.646008in}}%
\pgfpathlineto{\pgfqpoint{2.708309in}{0.655775in}}%
\pgfpathlineto{\pgfqpoint{2.718617in}{0.650324in}}%
\pgfpathlineto{\pgfqpoint{2.721195in}{0.646755in}}%
\pgfpathlineto{\pgfqpoint{2.723772in}{0.648070in}}%
\pgfpathlineto{\pgfqpoint{2.726349in}{0.642138in}}%
\pgfpathlineto{\pgfqpoint{2.734081in}{0.643159in}}%
\pgfpathlineto{\pgfqpoint{2.736658in}{0.648873in}}%
\pgfpathlineto{\pgfqpoint{2.739235in}{0.657777in}}%
\pgfpathlineto{\pgfqpoint{2.744390in}{0.680980in}}%
\pgfpathlineto{\pgfqpoint{2.752121in}{0.692692in}}%
\pgfpathlineto{\pgfqpoint{2.754698in}{0.691035in}}%
\pgfpathlineto{\pgfqpoint{2.759853in}{0.669174in}}%
\pgfpathlineto{\pgfqpoint{2.762430in}{0.666170in}}%
\pgfpathlineto{\pgfqpoint{2.770162in}{0.663884in}}%
\pgfpathlineto{\pgfqpoint{2.772739in}{0.667817in}}%
\pgfpathlineto{\pgfqpoint{2.775316in}{0.673329in}}%
\pgfpathlineto{\pgfqpoint{2.777893in}{0.675972in}}%
\pgfpathlineto{\pgfqpoint{2.780470in}{0.672783in}}%
\pgfpathlineto{\pgfqpoint{2.790779in}{0.673605in}}%
\pgfpathlineto{\pgfqpoint{2.793357in}{0.685301in}}%
\pgfpathlineto{\pgfqpoint{2.795934in}{0.672699in}}%
\pgfpathlineto{\pgfqpoint{2.798511in}{0.681097in}}%
\pgfpathlineto{\pgfqpoint{2.806243in}{0.670201in}}%
\pgfpathlineto{\pgfqpoint{2.808820in}{0.670910in}}%
\pgfpathlineto{\pgfqpoint{2.811397in}{0.672818in}}%
\pgfpathlineto{\pgfqpoint{2.813974in}{0.673614in}}%
\pgfpathlineto{\pgfqpoint{2.816551in}{0.671461in}}%
\pgfpathlineto{\pgfqpoint{2.824283in}{0.666876in}}%
\pgfpathlineto{\pgfqpoint{2.826860in}{0.667649in}}%
\pgfpathlineto{\pgfqpoint{2.829438in}{0.666851in}}%
\pgfpathlineto{\pgfqpoint{2.832015in}{0.662190in}}%
\pgfpathlineto{\pgfqpoint{2.834592in}{0.659851in}}%
\pgfpathlineto{\pgfqpoint{2.842324in}{0.659662in}}%
\pgfpathlineto{\pgfqpoint{2.844901in}{0.650257in}}%
\pgfpathlineto{\pgfqpoint{2.847478in}{0.652547in}}%
\pgfpathlineto{\pgfqpoint{2.852632in}{0.649417in}}%
\pgfpathlineto{\pgfqpoint{2.862941in}{0.653841in}}%
\pgfpathlineto{\pgfqpoint{2.865519in}{0.657209in}}%
\pgfpathlineto{\pgfqpoint{2.868096in}{0.650595in}}%
\pgfpathlineto{\pgfqpoint{2.870673in}{0.656871in}}%
\pgfpathlineto{\pgfqpoint{2.878405in}{0.662187in}}%
\pgfpathlineto{\pgfqpoint{2.883559in}{0.648287in}}%
\pgfpathlineto{\pgfqpoint{2.886136in}{0.648013in}}%
\pgfpathlineto{\pgfqpoint{2.888713in}{0.656585in}}%
\pgfpathlineto{\pgfqpoint{2.896445in}{0.657608in}}%
\pgfpathlineto{\pgfqpoint{2.899022in}{0.660351in}}%
\pgfpathlineto{\pgfqpoint{2.904177in}{0.661259in}}%
\pgfpathlineto{\pgfqpoint{2.906754in}{0.661193in}}%
\pgfpathlineto{\pgfqpoint{2.914486in}{0.645107in}}%
\pgfpathlineto{\pgfqpoint{2.917063in}{0.633956in}}%
\pgfpathlineto{\pgfqpoint{2.919640in}{0.645859in}}%
\pgfpathlineto{\pgfqpoint{2.922217in}{0.663784in}}%
\pgfpathlineto{\pgfqpoint{2.924794in}{0.662486in}}%
\pgfpathlineto{\pgfqpoint{2.932526in}{0.671141in}}%
\pgfpathlineto{\pgfqpoint{2.935103in}{0.670534in}}%
\pgfpathlineto{\pgfqpoint{2.937680in}{0.667822in}}%
\pgfpathlineto{\pgfqpoint{2.940258in}{0.672195in}}%
\pgfpathlineto{\pgfqpoint{2.950567in}{0.679508in}}%
\pgfpathlineto{\pgfqpoint{2.953144in}{0.679483in}}%
\pgfpathlineto{\pgfqpoint{2.955721in}{0.675473in}}%
\pgfpathlineto{\pgfqpoint{2.958298in}{0.664997in}}%
\pgfpathlineto{\pgfqpoint{2.960875in}{0.662540in}}%
\pgfpathlineto{\pgfqpoint{2.968607in}{0.658969in}}%
\pgfpathlineto{\pgfqpoint{2.971184in}{0.669020in}}%
\pgfpathlineto{\pgfqpoint{2.973761in}{0.668697in}}%
\pgfpathlineto{\pgfqpoint{2.976339in}{0.664544in}}%
\pgfpathlineto{\pgfqpoint{2.978916in}{0.667757in}}%
\pgfpathlineto{\pgfqpoint{2.986647in}{0.669116in}}%
\pgfpathlineto{\pgfqpoint{2.989225in}{0.665305in}}%
\pgfpathlineto{\pgfqpoint{2.991802in}{0.670181in}}%
\pgfpathlineto{\pgfqpoint{2.994379in}{0.669264in}}%
\pgfpathlineto{\pgfqpoint{2.996956in}{0.669051in}}%
\pgfpathlineto{\pgfqpoint{3.004688in}{0.675112in}}%
\pgfpathlineto{\pgfqpoint{3.007265in}{0.671021in}}%
\pgfpathlineto{\pgfqpoint{3.009842in}{0.671828in}}%
\pgfpathlineto{\pgfqpoint{3.012420in}{0.664545in}}%
\pgfpathlineto{\pgfqpoint{3.014997in}{0.650867in}}%
\pgfpathlineto{\pgfqpoint{3.022728in}{0.653740in}}%
\pgfpathlineto{\pgfqpoint{3.025306in}{0.650838in}}%
\pgfpathlineto{\pgfqpoint{3.027883in}{0.653448in}}%
\pgfpathlineto{\pgfqpoint{3.030460in}{0.652183in}}%
\pgfpathlineto{\pgfqpoint{3.033037in}{0.656630in}}%
\pgfpathlineto{\pgfqpoint{3.043346in}{0.649763in}}%
\pgfpathlineto{\pgfqpoint{3.048501in}{0.650146in}}%
\pgfpathlineto{\pgfqpoint{3.051078in}{0.652369in}}%
\pgfpathlineto{\pgfqpoint{3.058809in}{0.655313in}}%
\pgfpathlineto{\pgfqpoint{3.061387in}{0.647503in}}%
\pgfpathlineto{\pgfqpoint{3.066541in}{0.650743in}}%
\pgfpathlineto{\pgfqpoint{3.069118in}{0.647954in}}%
\pgfpathlineto{\pgfqpoint{3.076850in}{0.652555in}}%
\pgfpathlineto{\pgfqpoint{3.079427in}{0.665173in}}%
\pgfpathlineto{\pgfqpoint{3.082004in}{0.667318in}}%
\pgfpathlineto{\pgfqpoint{3.084582in}{0.667718in}}%
\pgfpathlineto{\pgfqpoint{3.087159in}{0.642988in}}%
\pgfpathlineto{\pgfqpoint{3.094890in}{0.638933in}}%
\pgfpathlineto{\pgfqpoint{3.097468in}{0.641902in}}%
\pgfpathlineto{\pgfqpoint{3.100045in}{0.642460in}}%
\pgfpathlineto{\pgfqpoint{3.102622in}{0.646111in}}%
\pgfpathlineto{\pgfqpoint{3.105199in}{0.640924in}}%
\pgfpathlineto{\pgfqpoint{3.112931in}{0.640738in}}%
\pgfpathlineto{\pgfqpoint{3.115508in}{0.636963in}}%
\pgfpathlineto{\pgfqpoint{3.118085in}{0.629676in}}%
\pgfpathlineto{\pgfqpoint{3.120663in}{0.625469in}}%
\pgfpathlineto{\pgfqpoint{3.123240in}{0.625772in}}%
\pgfpathlineto{\pgfqpoint{3.130971in}{0.625045in}}%
\pgfpathlineto{\pgfqpoint{3.136126in}{0.617691in}}%
\pgfpathlineto{\pgfqpoint{3.138703in}{0.617443in}}%
\pgfpathlineto{\pgfqpoint{3.149012in}{0.622177in}}%
\pgfpathlineto{\pgfqpoint{3.151589in}{0.615159in}}%
\pgfpathlineto{\pgfqpoint{3.154166in}{0.610490in}}%
\pgfpathlineto{\pgfqpoint{3.156744in}{0.609566in}}%
\pgfpathlineto{\pgfqpoint{3.159321in}{0.610464in}}%
\pgfpathlineto{\pgfqpoint{3.167052in}{0.609152in}}%
\pgfpathlineto{\pgfqpoint{3.169630in}{0.602950in}}%
\pgfpathlineto{\pgfqpoint{3.172207in}{0.563272in}}%
\pgfpathlineto{\pgfqpoint{3.174784in}{0.570142in}}%
\pgfpathlineto{\pgfqpoint{3.177361in}{0.570028in}}%
\pgfpathlineto{\pgfqpoint{3.185093in}{0.570780in}}%
\pgfpathlineto{\pgfqpoint{3.187670in}{0.564107in}}%
\pgfpathlineto{\pgfqpoint{3.190247in}{0.563942in}}%
\pgfpathlineto{\pgfqpoint{3.192824in}{0.557373in}}%
\pgfpathlineto{\pgfqpoint{3.195402in}{0.570992in}}%
\pgfpathlineto{\pgfqpoint{3.203133in}{0.568085in}}%
\pgfpathlineto{\pgfqpoint{3.205711in}{0.564770in}}%
\pgfpathlineto{\pgfqpoint{3.208288in}{0.565478in}}%
\pgfpathlineto{\pgfqpoint{3.210865in}{0.562393in}}%
\pgfpathlineto{\pgfqpoint{3.213442in}{0.544212in}}%
\pgfpathlineto{\pgfqpoint{3.221174in}{0.551135in}}%
\pgfpathlineto{\pgfqpoint{3.223751in}{0.560042in}}%
\pgfpathlineto{\pgfqpoint{3.226328in}{0.553646in}}%
\pgfpathlineto{\pgfqpoint{3.231483in}{0.555777in}}%
\pgfpathlineto{\pgfqpoint{3.239214in}{0.562424in}}%
\pgfpathlineto{\pgfqpoint{3.241792in}{0.560333in}}%
\pgfpathlineto{\pgfqpoint{3.244369in}{0.556890in}}%
\pgfpathlineto{\pgfqpoint{3.246946in}{0.559130in}}%
\pgfpathlineto{\pgfqpoint{3.249523in}{0.557067in}}%
\pgfpathlineto{\pgfqpoint{3.257255in}{0.564137in}}%
\pgfpathlineto{\pgfqpoint{3.259832in}{0.564702in}}%
\pgfpathlineto{\pgfqpoint{3.262409in}{0.562246in}}%
\pgfpathlineto{\pgfqpoint{3.264986in}{0.564119in}}%
\pgfpathlineto{\pgfqpoint{3.267564in}{0.566974in}}%
\pgfpathlineto{\pgfqpoint{3.275295in}{0.568450in}}%
\pgfpathlineto{\pgfqpoint{3.277872in}{0.567957in}}%
\pgfpathlineto{\pgfqpoint{3.283027in}{0.572083in}}%
\pgfpathlineto{\pgfqpoint{3.285604in}{0.571996in}}%
\pgfpathlineto{\pgfqpoint{3.295913in}{0.582230in}}%
\pgfpathlineto{\pgfqpoint{3.298490in}{0.582689in}}%
\pgfpathlineto{\pgfqpoint{3.301067in}{0.586995in}}%
\pgfpathlineto{\pgfqpoint{3.303645in}{0.588635in}}%
\pgfpathlineto{\pgfqpoint{3.311376in}{0.587527in}}%
\pgfpathlineto{\pgfqpoint{3.313953in}{0.585696in}}%
\pgfpathlineto{\pgfqpoint{3.316531in}{0.582450in}}%
\pgfpathlineto{\pgfqpoint{3.319108in}{0.588600in}}%
\pgfpathlineto{\pgfqpoint{3.321685in}{0.592562in}}%
\pgfpathlineto{\pgfqpoint{3.329417in}{0.592541in}}%
\pgfpathlineto{\pgfqpoint{3.331994in}{0.598637in}}%
\pgfpathlineto{\pgfqpoint{3.334571in}{0.599596in}}%
\pgfpathlineto{\pgfqpoint{3.337148in}{0.595330in}}%
\pgfpathlineto{\pgfqpoint{3.339726in}{0.596497in}}%
\pgfpathlineto{\pgfqpoint{3.347457in}{0.594353in}}%
\pgfpathlineto{\pgfqpoint{3.350034in}{0.590157in}}%
\pgfpathlineto{\pgfqpoint{3.352612in}{0.593985in}}%
\pgfpathlineto{\pgfqpoint{3.355189in}{0.592223in}}%
\pgfpathlineto{\pgfqpoint{3.357766in}{0.597693in}}%
\pgfpathlineto{\pgfqpoint{3.365498in}{0.596664in}}%
\pgfpathlineto{\pgfqpoint{3.370652in}{0.602800in}}%
\pgfpathlineto{\pgfqpoint{3.373229in}{0.597933in}}%
\pgfpathlineto{\pgfqpoint{3.375807in}{0.601531in}}%
\pgfpathlineto{\pgfqpoint{3.383538in}{0.601895in}}%
\pgfpathlineto{\pgfqpoint{3.386115in}{0.603811in}}%
\pgfpathlineto{\pgfqpoint{3.388693in}{0.606886in}}%
\pgfpathlineto{\pgfqpoint{3.391270in}{0.604568in}}%
\pgfpathlineto{\pgfqpoint{3.393847in}{0.603411in}}%
\pgfpathlineto{\pgfqpoint{3.401579in}{0.610768in}}%
\pgfpathlineto{\pgfqpoint{3.404156in}{0.611390in}}%
\pgfpathlineto{\pgfqpoint{3.406733in}{0.620549in}}%
\pgfpathlineto{\pgfqpoint{3.409310in}{0.619704in}}%
\pgfpathlineto{\pgfqpoint{3.411888in}{0.621428in}}%
\pgfpathlineto{\pgfqpoint{3.419619in}{0.615233in}}%
\pgfpathlineto{\pgfqpoint{3.422196in}{0.610450in}}%
\pgfpathlineto{\pgfqpoint{3.424774in}{0.612391in}}%
\pgfpathlineto{\pgfqpoint{3.427351in}{0.633334in}}%
\pgfpathlineto{\pgfqpoint{3.429928in}{0.634668in}}%
\pgfpathlineto{\pgfqpoint{3.437660in}{0.632213in}}%
\pgfpathlineto{\pgfqpoint{3.442814in}{0.642674in}}%
\pgfpathlineto{\pgfqpoint{3.445391in}{0.687335in}}%
\pgfpathlineto{\pgfqpoint{3.447969in}{0.695380in}}%
\pgfpathlineto{\pgfqpoint{3.455700in}{0.690982in}}%
\pgfpathlineto{\pgfqpoint{3.458277in}{0.688414in}}%
\pgfpathlineto{\pgfqpoint{3.460855in}{0.688137in}}%
\pgfpathlineto{\pgfqpoint{3.463432in}{0.690784in}}%
\pgfpathlineto{\pgfqpoint{3.466009in}{0.704836in}}%
\pgfpathlineto{\pgfqpoint{3.473741in}{0.710315in}}%
\pgfpathlineto{\pgfqpoint{3.476318in}{0.711069in}}%
\pgfpathlineto{\pgfqpoint{3.478895in}{0.713001in}}%
\pgfpathlineto{\pgfqpoint{3.481472in}{0.710343in}}%
\pgfpathlineto{\pgfqpoint{3.484049in}{0.701757in}}%
\pgfpathlineto{\pgfqpoint{3.491781in}{0.701177in}}%
\pgfpathlineto{\pgfqpoint{3.494358in}{0.692060in}}%
\pgfpathlineto{\pgfqpoint{3.496936in}{0.690186in}}%
\pgfpathlineto{\pgfqpoint{3.499513in}{0.703297in}}%
\pgfpathlineto{\pgfqpoint{3.502090in}{0.703522in}}%
\pgfpathlineto{\pgfqpoint{3.514976in}{0.725538in}}%
\pgfpathlineto{\pgfqpoint{3.520130in}{0.718981in}}%
\pgfpathlineto{\pgfqpoint{3.527862in}{0.717002in}}%
\pgfpathlineto{\pgfqpoint{3.530439in}{0.717320in}}%
\pgfpathlineto{\pgfqpoint{3.533017in}{0.717067in}}%
\pgfpathlineto{\pgfqpoint{3.535594in}{0.725115in}}%
\pgfpathlineto{\pgfqpoint{3.538171in}{0.724552in}}%
\pgfpathlineto{\pgfqpoint{3.545903in}{0.718704in}}%
\pgfpathlineto{\pgfqpoint{3.548480in}{0.708610in}}%
\pgfpathlineto{\pgfqpoint{3.551057in}{0.705804in}}%
\pgfpathlineto{\pgfqpoint{3.553634in}{0.706358in}}%
\pgfpathlineto{\pgfqpoint{3.556211in}{0.697543in}}%
\pgfpathlineto{\pgfqpoint{3.563943in}{0.697345in}}%
\pgfpathlineto{\pgfqpoint{3.566520in}{0.688250in}}%
\pgfpathlineto{\pgfqpoint{3.569098in}{0.699592in}}%
\pgfpathlineto{\pgfqpoint{3.574252in}{0.703204in}}%
\pgfpathlineto{\pgfqpoint{3.581984in}{0.709296in}}%
\pgfpathlineto{\pgfqpoint{3.584561in}{0.710413in}}%
\pgfpathlineto{\pgfqpoint{3.587138in}{0.713103in}}%
\pgfpathlineto{\pgfqpoint{3.592292in}{0.716201in}}%
\pgfpathlineto{\pgfqpoint{3.600024in}{0.712792in}}%
\pgfpathlineto{\pgfqpoint{3.602601in}{0.712668in}}%
\pgfpathlineto{\pgfqpoint{3.605178in}{0.713518in}}%
\pgfpathlineto{\pgfqpoint{3.610333in}{0.712715in}}%
\pgfpathlineto{\pgfqpoint{3.618065in}{0.715457in}}%
\pgfpathlineto{\pgfqpoint{3.620642in}{0.707211in}}%
\pgfpathlineto{\pgfqpoint{3.623219in}{0.715777in}}%
\pgfpathlineto{\pgfqpoint{3.628373in}{0.724475in}}%
\pgfpathlineto{\pgfqpoint{3.636105in}{0.723743in}}%
\pgfpathlineto{\pgfqpoint{3.638682in}{0.721970in}}%
\pgfpathlineto{\pgfqpoint{3.643837in}{0.712129in}}%
\pgfpathlineto{\pgfqpoint{3.646414in}{0.708261in}}%
\pgfpathlineto{\pgfqpoint{3.656723in}{0.711530in}}%
\pgfpathlineto{\pgfqpoint{3.659300in}{0.711271in}}%
\pgfpathlineto{\pgfqpoint{3.661877in}{0.703017in}}%
\pgfpathlineto{\pgfqpoint{3.664454in}{0.706574in}}%
\pgfpathlineto{\pgfqpoint{3.672186in}{0.705628in}}%
\pgfpathlineto{\pgfqpoint{3.674763in}{0.697187in}}%
\pgfpathlineto{\pgfqpoint{3.677340in}{0.696205in}}%
\pgfpathlineto{\pgfqpoint{3.679918in}{0.690356in}}%
\pgfpathlineto{\pgfqpoint{3.682495in}{0.702597in}}%
\pgfpathlineto{\pgfqpoint{3.690226in}{0.707793in}}%
\pgfpathlineto{\pgfqpoint{3.692804in}{0.712979in}}%
\pgfpathlineto{\pgfqpoint{3.695381in}{0.687571in}}%
\pgfpathlineto{\pgfqpoint{3.697958in}{0.682441in}}%
\pgfpathlineto{\pgfqpoint{3.700535in}{0.693204in}}%
\pgfpathlineto{\pgfqpoint{3.708267in}{0.688424in}}%
\pgfpathlineto{\pgfqpoint{3.715999in}{0.704070in}}%
\pgfpathlineto{\pgfqpoint{3.718576in}{0.696767in}}%
\pgfpathlineto{\pgfqpoint{3.728885in}{0.699847in}}%
\pgfpathlineto{\pgfqpoint{3.731462in}{0.703066in}}%
\pgfpathlineto{\pgfqpoint{3.734039in}{0.713029in}}%
\pgfpathlineto{\pgfqpoint{3.736616in}{0.713673in}}%
\pgfpathlineto{\pgfqpoint{3.744348in}{0.712281in}}%
\pgfpathlineto{\pgfqpoint{3.749502in}{0.728493in}}%
\pgfpathlineto{\pgfqpoint{3.752080in}{0.718686in}}%
\pgfpathlineto{\pgfqpoint{3.754657in}{0.721010in}}%
\pgfpathlineto{\pgfqpoint{3.764966in}{0.732957in}}%
\pgfpathlineto{\pgfqpoint{3.767543in}{0.734816in}}%
\pgfpathlineto{\pgfqpoint{3.770120in}{0.741110in}}%
\pgfpathlineto{\pgfqpoint{3.772697in}{0.743507in}}%
\pgfpathlineto{\pgfqpoint{3.780429in}{0.739322in}}%
\pgfpathlineto{\pgfqpoint{3.783006in}{0.743047in}}%
\pgfpathlineto{\pgfqpoint{3.785583in}{0.756654in}}%
\pgfpathlineto{\pgfqpoint{3.788161in}{0.777402in}}%
\pgfpathlineto{\pgfqpoint{3.790738in}{0.780643in}}%
\pgfpathlineto{\pgfqpoint{3.798469in}{0.788367in}}%
\pgfpathlineto{\pgfqpoint{3.801047in}{0.791803in}}%
\pgfpathlineto{\pgfqpoint{3.803624in}{0.796665in}}%
\pgfpathlineto{\pgfqpoint{3.808778in}{0.787754in}}%
\pgfpathlineto{\pgfqpoint{3.816510in}{0.787196in}}%
\pgfpathlineto{\pgfqpoint{3.819087in}{0.790888in}}%
\pgfpathlineto{\pgfqpoint{3.821664in}{0.791835in}}%
\pgfpathlineto{\pgfqpoint{3.824242in}{0.800549in}}%
\pgfpathlineto{\pgfqpoint{3.826819in}{0.773196in}}%
\pgfpathlineto{\pgfqpoint{3.834550in}{0.774432in}}%
\pgfpathlineto{\pgfqpoint{3.837128in}{0.781444in}}%
\pgfpathlineto{\pgfqpoint{3.842282in}{0.767791in}}%
\pgfpathlineto{\pgfqpoint{3.852591in}{0.769801in}}%
\pgfpathlineto{\pgfqpoint{3.855168in}{0.773525in}}%
\pgfpathlineto{\pgfqpoint{3.857745in}{0.772721in}}%
\pgfpathlineto{\pgfqpoint{3.860323in}{0.783821in}}%
\pgfpathlineto{\pgfqpoint{3.862900in}{0.749986in}}%
\pgfpathlineto{\pgfqpoint{3.870631in}{0.757117in}}%
\pgfpathlineto{\pgfqpoint{3.873209in}{0.756695in}}%
\pgfpathlineto{\pgfqpoint{3.875786in}{0.776048in}}%
\pgfpathlineto{\pgfqpoint{3.878363in}{0.774110in}}%
\pgfpathlineto{\pgfqpoint{3.880940in}{0.794896in}}%
\pgfpathlineto{\pgfqpoint{3.888672in}{0.802088in}}%
\pgfpathlineto{\pgfqpoint{3.891249in}{0.799893in}}%
\pgfpathlineto{\pgfqpoint{3.893826in}{0.791316in}}%
\pgfpathlineto{\pgfqpoint{3.898981in}{0.785403in}}%
\pgfpathlineto{\pgfqpoint{3.906712in}{0.788752in}}%
\pgfpathlineto{\pgfqpoint{3.909290in}{0.784279in}}%
\pgfpathlineto{\pgfqpoint{3.911867in}{0.786899in}}%
\pgfpathlineto{\pgfqpoint{3.914444in}{0.795071in}}%
\pgfpathlineto{\pgfqpoint{3.917021in}{0.794619in}}%
\pgfpathlineto{\pgfqpoint{3.924753in}{0.792443in}}%
\pgfpathlineto{\pgfqpoint{3.927330in}{0.796995in}}%
\pgfpathlineto{\pgfqpoint{3.929907in}{0.796047in}}%
\pgfpathlineto{\pgfqpoint{3.932484in}{0.792634in}}%
\pgfpathlineto{\pgfqpoint{3.935062in}{0.777935in}}%
\pgfpathlineto{\pgfqpoint{3.942793in}{0.781037in}}%
\pgfpathlineto{\pgfqpoint{3.945371in}{0.792942in}}%
\pgfpathlineto{\pgfqpoint{3.947948in}{0.799076in}}%
\pgfpathlineto{\pgfqpoint{3.950525in}{0.797732in}}%
\pgfpathlineto{\pgfqpoint{3.953102in}{0.802154in}}%
\pgfpathlineto{\pgfqpoint{3.960834in}{0.801598in}}%
\pgfpathlineto{\pgfqpoint{3.963411in}{0.802631in}}%
\pgfpathlineto{\pgfqpoint{3.965988in}{0.809608in}}%
\pgfpathlineto{\pgfqpoint{3.968565in}{0.803093in}}%
\pgfpathlineto{\pgfqpoint{3.971143in}{0.803874in}}%
\pgfpathlineto{\pgfqpoint{3.981451in}{0.809509in}}%
\pgfpathlineto{\pgfqpoint{3.984029in}{0.814998in}}%
\pgfpathlineto{\pgfqpoint{3.986606in}{0.812666in}}%
\pgfpathlineto{\pgfqpoint{3.989183in}{0.807415in}}%
\pgfpathlineto{\pgfqpoint{3.996915in}{0.814048in}}%
\pgfpathlineto{\pgfqpoint{3.999492in}{0.807909in}}%
\pgfpathlineto{\pgfqpoint{4.002069in}{0.794824in}}%
\pgfpathlineto{\pgfqpoint{4.004646in}{0.796223in}}%
\pgfpathlineto{\pgfqpoint{4.007224in}{0.793025in}}%
\pgfpathlineto{\pgfqpoint{4.014955in}{0.792729in}}%
\pgfpathlineto{\pgfqpoint{4.017532in}{0.788577in}}%
\pgfpathlineto{\pgfqpoint{4.020110in}{0.786848in}}%
\pgfpathlineto{\pgfqpoint{4.022687in}{0.790038in}}%
\pgfpathlineto{\pgfqpoint{4.025264in}{0.792213in}}%
\pgfpathlineto{\pgfqpoint{4.032996in}{0.794444in}}%
\pgfpathlineto{\pgfqpoint{4.035573in}{0.790901in}}%
\pgfpathlineto{\pgfqpoint{4.038150in}{0.790055in}}%
\pgfpathlineto{\pgfqpoint{4.040727in}{0.792590in}}%
\pgfpathlineto{\pgfqpoint{4.043305in}{0.789993in}}%
\pgfpathlineto{\pgfqpoint{4.051036in}{0.793168in}}%
\pgfpathlineto{\pgfqpoint{4.053613in}{0.788421in}}%
\pgfpathlineto{\pgfqpoint{4.056191in}{0.790896in}}%
\pgfpathlineto{\pgfqpoint{4.058768in}{0.786078in}}%
\pgfpathlineto{\pgfqpoint{4.061345in}{0.785713in}}%
\pgfpathlineto{\pgfqpoint{4.069077in}{0.788164in}}%
\pgfpathlineto{\pgfqpoint{4.071654in}{0.793135in}}%
\pgfpathlineto{\pgfqpoint{4.074231in}{0.786346in}}%
\pgfpathlineto{\pgfqpoint{4.076808in}{0.786278in}}%
\pgfpathlineto{\pgfqpoint{4.087117in}{0.789755in}}%
\pgfpathlineto{\pgfqpoint{4.089694in}{0.777793in}}%
\pgfpathlineto{\pgfqpoint{4.092272in}{0.776336in}}%
\pgfpathlineto{\pgfqpoint{4.094849in}{0.784404in}}%
\pgfpathlineto{\pgfqpoint{4.097426in}{0.786196in}}%
\pgfpathlineto{\pgfqpoint{4.105158in}{0.788214in}}%
\pgfpathlineto{\pgfqpoint{4.107735in}{0.785902in}}%
\pgfpathlineto{\pgfqpoint{4.110312in}{0.787726in}}%
\pgfpathlineto{\pgfqpoint{4.112889in}{0.785905in}}%
\pgfpathlineto{\pgfqpoint{4.115467in}{0.782919in}}%
\pgfpathlineto{\pgfqpoint{4.123198in}{0.770610in}}%
\pgfpathlineto{\pgfqpoint{4.125775in}{0.773946in}}%
\pgfpathlineto{\pgfqpoint{4.128353in}{0.766096in}}%
\pgfpathlineto{\pgfqpoint{4.130930in}{0.774845in}}%
\pgfpathlineto{\pgfqpoint{4.133507in}{0.779963in}}%
\pgfpathlineto{\pgfqpoint{4.141239in}{0.793697in}}%
\pgfpathlineto{\pgfqpoint{4.143816in}{0.788703in}}%
\pgfpathlineto{\pgfqpoint{4.146393in}{0.794235in}}%
\pgfpathlineto{\pgfqpoint{4.148970in}{0.790418in}}%
\pgfpathlineto{\pgfqpoint{4.151548in}{0.788685in}}%
\pgfpathlineto{\pgfqpoint{4.161856in}{0.788339in}}%
\pgfpathlineto{\pgfqpoint{4.164434in}{0.818241in}}%
\pgfpathlineto{\pgfqpoint{4.167011in}{0.806439in}}%
\pgfpathlineto{\pgfqpoint{4.169588in}{0.817619in}}%
\pgfpathlineto{\pgfqpoint{4.177320in}{0.820938in}}%
\pgfpathlineto{\pgfqpoint{4.179897in}{0.820975in}}%
\pgfpathlineto{\pgfqpoint{4.182474in}{0.823396in}}%
\pgfpathlineto{\pgfqpoint{4.185051in}{0.833196in}}%
\pgfpathlineto{\pgfqpoint{4.195360in}{0.836901in}}%
\pgfpathlineto{\pgfqpoint{4.197937in}{0.843994in}}%
\pgfpathlineto{\pgfqpoint{4.200515in}{0.840151in}}%
\pgfpathlineto{\pgfqpoint{4.203092in}{0.834251in}}%
\pgfpathlineto{\pgfqpoint{4.205669in}{0.836726in}}%
\pgfpathlineto{\pgfqpoint{4.213401in}{0.840009in}}%
\pgfpathlineto{\pgfqpoint{4.215978in}{0.842610in}}%
\pgfpathlineto{\pgfqpoint{4.218555in}{0.841723in}}%
\pgfpathlineto{\pgfqpoint{4.221132in}{0.839880in}}%
\pgfpathlineto{\pgfqpoint{4.223709in}{0.835590in}}%
\pgfpathlineto{\pgfqpoint{4.231441in}{0.831323in}}%
\pgfpathlineto{\pgfqpoint{4.234018in}{0.834108in}}%
\pgfpathlineto{\pgfqpoint{4.236596in}{0.838554in}}%
\pgfpathlineto{\pgfqpoint{4.239173in}{0.840712in}}%
\pgfpathlineto{\pgfqpoint{4.241750in}{0.841417in}}%
\pgfpathlineto{\pgfqpoint{4.252059in}{0.845264in}}%
\pgfpathlineto{\pgfqpoint{4.254636in}{0.844815in}}%
\pgfpathlineto{\pgfqpoint{4.257213in}{0.845262in}}%
\pgfpathlineto{\pgfqpoint{4.259790in}{0.839661in}}%
\pgfpathlineto{\pgfqpoint{4.267522in}{0.844536in}}%
\pgfpathlineto{\pgfqpoint{4.270099in}{0.845050in}}%
\pgfpathlineto{\pgfqpoint{4.272677in}{0.836202in}}%
\pgfpathlineto{\pgfqpoint{4.275254in}{0.835280in}}%
\pgfpathlineto{\pgfqpoint{4.277831in}{0.832165in}}%
\pgfpathlineto{\pgfqpoint{4.285563in}{0.829214in}}%
\pgfpathlineto{\pgfqpoint{4.288140in}{0.826629in}}%
\pgfpathlineto{\pgfqpoint{4.290717in}{0.822713in}}%
\pgfpathlineto{\pgfqpoint{4.293294in}{0.830487in}}%
\pgfpathlineto{\pgfqpoint{4.295871in}{0.826698in}}%
\pgfpathlineto{\pgfqpoint{4.303603in}{0.829429in}}%
\pgfpathlineto{\pgfqpoint{4.308757in}{0.825876in}}%
\pgfpathlineto{\pgfqpoint{4.311335in}{0.821006in}}%
\pgfpathlineto{\pgfqpoint{4.313912in}{0.821096in}}%
\pgfpathlineto{\pgfqpoint{4.321644in}{0.817284in}}%
\pgfpathlineto{\pgfqpoint{4.324221in}{0.806547in}}%
\pgfpathlineto{\pgfqpoint{4.326798in}{0.801944in}}%
\pgfpathlineto{\pgfqpoint{4.329375in}{0.804099in}}%
\pgfpathlineto{\pgfqpoint{4.331952in}{0.801389in}}%
\pgfpathlineto{\pgfqpoint{4.339684in}{0.800340in}}%
\pgfpathlineto{\pgfqpoint{4.344838in}{0.790103in}}%
\pgfpathlineto{\pgfqpoint{4.347416in}{0.800496in}}%
\pgfpathlineto{\pgfqpoint{4.349993in}{0.801642in}}%
\pgfpathlineto{\pgfqpoint{4.357725in}{0.796927in}}%
\pgfpathlineto{\pgfqpoint{4.360302in}{0.820202in}}%
\pgfpathlineto{\pgfqpoint{4.365456in}{0.799338in}}%
\pgfpathlineto{\pgfqpoint{4.368033in}{0.808640in}}%
\pgfpathlineto{\pgfqpoint{4.375765in}{0.810373in}}%
\pgfpathlineto{\pgfqpoint{4.378342in}{0.812474in}}%
\pgfpathlineto{\pgfqpoint{4.383497in}{0.822685in}}%
\pgfpathlineto{\pgfqpoint{4.386074in}{0.820679in}}%
\pgfpathlineto{\pgfqpoint{4.393805in}{0.824842in}}%
\pgfpathlineto{\pgfqpoint{4.396383in}{0.836443in}}%
\pgfpathlineto{\pgfqpoint{4.398960in}{0.832523in}}%
\pgfpathlineto{\pgfqpoint{4.401537in}{0.825741in}}%
\pgfpathlineto{\pgfqpoint{4.404114in}{0.822261in}}%
\pgfpathlineto{\pgfqpoint{4.411846in}{0.816907in}}%
\pgfpathlineto{\pgfqpoint{4.417000in}{0.801796in}}%
\pgfpathlineto{\pgfqpoint{4.419578in}{0.808462in}}%
\pgfpathlineto{\pgfqpoint{4.422155in}{0.818570in}}%
\pgfpathlineto{\pgfqpoint{4.429886in}{0.826289in}}%
\pgfpathlineto{\pgfqpoint{4.432464in}{0.833091in}}%
\pgfpathlineto{\pgfqpoint{4.437618in}{0.830308in}}%
\pgfpathlineto{\pgfqpoint{4.440195in}{0.834726in}}%
\pgfpathlineto{\pgfqpoint{4.447927in}{0.835121in}}%
\pgfpathlineto{\pgfqpoint{4.450504in}{0.832188in}}%
\pgfpathlineto{\pgfqpoint{4.453081in}{0.840094in}}%
\pgfpathlineto{\pgfqpoint{4.458236in}{0.850895in}}%
\pgfpathlineto{\pgfqpoint{4.465967in}{0.844553in}}%
\pgfpathlineto{\pgfqpoint{4.468545in}{0.849419in}}%
\pgfpathlineto{\pgfqpoint{4.471122in}{0.846219in}}%
\pgfpathlineto{\pgfqpoint{4.473699in}{0.845838in}}%
\pgfpathlineto{\pgfqpoint{4.476276in}{0.847539in}}%
\pgfpathlineto{\pgfqpoint{4.486585in}{0.846834in}}%
\pgfpathlineto{\pgfqpoint{4.489162in}{0.850792in}}%
\pgfpathlineto{\pgfqpoint{4.491740in}{0.850214in}}%
\pgfpathlineto{\pgfqpoint{4.494317in}{0.853857in}}%
\pgfpathlineto{\pgfqpoint{4.502048in}{0.861175in}}%
\pgfpathlineto{\pgfqpoint{4.504626in}{0.893197in}}%
\pgfpathlineto{\pgfqpoint{4.507203in}{0.894825in}}%
\pgfpathlineto{\pgfqpoint{4.509780in}{0.890086in}}%
\pgfpathlineto{\pgfqpoint{4.512357in}{0.873284in}}%
\pgfpathlineto{\pgfqpoint{4.522666in}{0.867455in}}%
\pgfpathlineto{\pgfqpoint{4.525243in}{0.859959in}}%
\pgfpathlineto{\pgfqpoint{4.527821in}{0.860831in}}%
\pgfpathlineto{\pgfqpoint{4.538129in}{0.859539in}}%
\pgfpathlineto{\pgfqpoint{4.540707in}{0.864377in}}%
\pgfpathlineto{\pgfqpoint{4.543284in}{0.866850in}}%
\pgfpathlineto{\pgfqpoint{4.545861in}{0.864059in}}%
\pgfpathlineto{\pgfqpoint{4.556170in}{0.866580in}}%
\pgfpathlineto{\pgfqpoint{4.558747in}{0.855945in}}%
\pgfpathlineto{\pgfqpoint{4.561324in}{0.851486in}}%
\pgfpathlineto{\pgfqpoint{4.563902in}{0.850688in}}%
\pgfpathlineto{\pgfqpoint{4.566479in}{0.844431in}}%
\pgfpathlineto{\pgfqpoint{4.574210in}{0.841337in}}%
\pgfpathlineto{\pgfqpoint{4.576788in}{0.833020in}}%
\pgfpathlineto{\pgfqpoint{4.579365in}{0.843583in}}%
\pgfpathlineto{\pgfqpoint{4.581942in}{0.839638in}}%
\pgfpathlineto{\pgfqpoint{4.584519in}{0.862331in}}%
\pgfpathlineto{\pgfqpoint{4.597405in}{0.860227in}}%
\pgfpathlineto{\pgfqpoint{4.599982in}{0.867215in}}%
\pgfpathlineto{\pgfqpoint{4.602560in}{0.816458in}}%
\pgfpathlineto{\pgfqpoint{4.610291in}{0.820032in}}%
\pgfpathlineto{\pgfqpoint{4.612869in}{0.816877in}}%
\pgfpathlineto{\pgfqpoint{4.618023in}{0.789523in}}%
\pgfpathlineto{\pgfqpoint{4.620600in}{0.803983in}}%
\pgfpathlineto{\pgfqpoint{4.628332in}{0.815249in}}%
\pgfpathlineto{\pgfqpoint{4.630909in}{0.821427in}}%
\pgfpathlineto{\pgfqpoint{4.633486in}{0.805972in}}%
\pgfpathlineto{\pgfqpoint{4.636063in}{0.818101in}}%
\pgfpathlineto{\pgfqpoint{4.638641in}{0.819001in}}%
\pgfpathlineto{\pgfqpoint{4.646372in}{0.798967in}}%
\pgfpathlineto{\pgfqpoint{4.648950in}{0.803839in}}%
\pgfpathlineto{\pgfqpoint{4.651527in}{0.786686in}}%
\pgfpathlineto{\pgfqpoint{4.654104in}{0.813695in}}%
\pgfpathlineto{\pgfqpoint{4.656681in}{0.827105in}}%
\pgfpathlineto{\pgfqpoint{4.666990in}{0.824345in}}%
\pgfpathlineto{\pgfqpoint{4.669567in}{0.808242in}}%
\pgfpathlineto{\pgfqpoint{4.672144in}{0.819920in}}%
\pgfpathlineto{\pgfqpoint{4.674722in}{0.818806in}}%
\pgfpathlineto{\pgfqpoint{4.682453in}{0.820302in}}%
\pgfpathlineto{\pgfqpoint{4.685030in}{0.826025in}}%
\pgfpathlineto{\pgfqpoint{4.687608in}{0.834602in}}%
\pgfpathlineto{\pgfqpoint{4.690185in}{0.822335in}}%
\pgfpathlineto{\pgfqpoint{4.692762in}{0.824754in}}%
\pgfpathlineto{\pgfqpoint{4.700494in}{0.825329in}}%
\pgfpathlineto{\pgfqpoint{4.703071in}{0.806123in}}%
\pgfpathlineto{\pgfqpoint{4.705648in}{0.813939in}}%
\pgfpathlineto{\pgfqpoint{4.708225in}{0.819618in}}%
\pgfpathlineto{\pgfqpoint{4.710803in}{0.818926in}}%
\pgfpathlineto{\pgfqpoint{4.718534in}{0.828164in}}%
\pgfpathlineto{\pgfqpoint{4.721111in}{0.819929in}}%
\pgfpathlineto{\pgfqpoint{4.723689in}{0.823026in}}%
\pgfpathlineto{\pgfqpoint{4.726266in}{0.813475in}}%
\pgfpathlineto{\pgfqpoint{4.728843in}{0.810382in}}%
\pgfpathlineto{\pgfqpoint{4.736575in}{0.812534in}}%
\pgfpathlineto{\pgfqpoint{4.739152in}{0.820436in}}%
\pgfpathlineto{\pgfqpoint{4.741729in}{0.821328in}}%
\pgfpathlineto{\pgfqpoint{4.744306in}{0.820494in}}%
\pgfpathlineto{\pgfqpoint{4.746884in}{0.825576in}}%
\pgfpathlineto{\pgfqpoint{4.754615in}{0.828298in}}%
\pgfpathlineto{\pgfqpoint{4.757192in}{0.826502in}}%
\pgfpathlineto{\pgfqpoint{4.759770in}{0.829194in}}%
\pgfpathlineto{\pgfqpoint{4.762347in}{0.828766in}}%
\pgfpathlineto{\pgfqpoint{4.772656in}{0.825452in}}%
\pgfpathlineto{\pgfqpoint{4.775233in}{0.833062in}}%
\pgfpathlineto{\pgfqpoint{4.777810in}{0.831829in}}%
\pgfpathlineto{\pgfqpoint{4.782965in}{0.845837in}}%
\pgfpathlineto{\pgfqpoint{4.790696in}{0.841495in}}%
\pgfpathlineto{\pgfqpoint{4.793273in}{0.843830in}}%
\pgfpathlineto{\pgfqpoint{4.795851in}{0.847407in}}%
\pgfpathlineto{\pgfqpoint{4.798428in}{0.843080in}}%
\pgfpathlineto{\pgfqpoint{4.801005in}{0.848295in}}%
\pgfpathlineto{\pgfqpoint{4.808737in}{0.846830in}}%
\pgfpathlineto{\pgfqpoint{4.811314in}{0.840361in}}%
\pgfpathlineto{\pgfqpoint{4.813891in}{0.822362in}}%
\pgfpathlineto{\pgfqpoint{4.816468in}{0.825134in}}%
\pgfpathlineto{\pgfqpoint{4.819046in}{0.824268in}}%
\pgfpathlineto{\pgfqpoint{4.829354in}{0.825416in}}%
\pgfpathlineto{\pgfqpoint{4.831932in}{0.820220in}}%
\pgfpathlineto{\pgfqpoint{4.834509in}{0.800359in}}%
\pgfpathlineto{\pgfqpoint{4.837086in}{0.804073in}}%
\pgfpathlineto{\pgfqpoint{4.844818in}{0.791772in}}%
\pgfpathlineto{\pgfqpoint{4.847395in}{0.801472in}}%
\pgfpathlineto{\pgfqpoint{4.849972in}{0.806224in}}%
\pgfpathlineto{\pgfqpoint{4.852549in}{0.815786in}}%
\pgfpathlineto{\pgfqpoint{4.855127in}{0.806088in}}%
\pgfpathlineto{\pgfqpoint{4.862858in}{0.808111in}}%
\pgfpathlineto{\pgfqpoint{4.865435in}{0.804362in}}%
\pgfpathlineto{\pgfqpoint{4.868013in}{0.797372in}}%
\pgfpathlineto{\pgfqpoint{4.870590in}{0.794960in}}%
\pgfpathlineto{\pgfqpoint{4.873167in}{0.799737in}}%
\pgfpathlineto{\pgfqpoint{4.880899in}{0.803214in}}%
\pgfpathlineto{\pgfqpoint{4.883476in}{0.811044in}}%
\pgfpathlineto{\pgfqpoint{4.886053in}{0.812349in}}%
\pgfpathlineto{\pgfqpoint{4.891207in}{0.822978in}}%
\pgfpathlineto{\pgfqpoint{4.898939in}{0.823044in}}%
\pgfpathlineto{\pgfqpoint{4.901516in}{0.825536in}}%
\pgfpathlineto{\pgfqpoint{4.904094in}{0.834396in}}%
\pgfpathlineto{\pgfqpoint{4.909248in}{0.840752in}}%
\pgfpathlineto{\pgfqpoint{4.916980in}{0.845309in}}%
\pgfpathlineto{\pgfqpoint{4.919557in}{0.833207in}}%
\pgfpathlineto{\pgfqpoint{4.922134in}{0.834341in}}%
\pgfpathlineto{\pgfqpoint{4.924711in}{0.833357in}}%
\pgfpathlineto{\pgfqpoint{4.927288in}{0.829358in}}%
\pgfpathlineto{\pgfqpoint{4.937597in}{0.828168in}}%
\pgfpathlineto{\pgfqpoint{4.940175in}{0.832034in}}%
\pgfpathlineto{\pgfqpoint{4.945329in}{0.829675in}}%
\pgfpathlineto{\pgfqpoint{4.953061in}{0.833405in}}%
\pgfpathlineto{\pgfqpoint{4.955638in}{0.837515in}}%
\pgfpathlineto{\pgfqpoint{4.958215in}{0.839470in}}%
\pgfpathlineto{\pgfqpoint{4.960792in}{0.842887in}}%
\pgfpathlineto{\pgfqpoint{4.963369in}{0.829210in}}%
\pgfpathlineto{\pgfqpoint{4.971101in}{0.811551in}}%
\pgfpathlineto{\pgfqpoint{4.973678in}{0.793544in}}%
\pgfpathlineto{\pgfqpoint{4.978833in}{0.785706in}}%
\pgfpathlineto{\pgfqpoint{4.981410in}{0.784288in}}%
\pgfpathlineto{\pgfqpoint{4.989142in}{0.787448in}}%
\pgfpathlineto{\pgfqpoint{4.994296in}{0.791060in}}%
\pgfpathlineto{\pgfqpoint{4.999450in}{0.812477in}}%
\pgfpathlineto{\pgfqpoint{5.007182in}{0.784979in}}%
\pgfpathlineto{\pgfqpoint{5.009759in}{0.786281in}}%
\pgfpathlineto{\pgfqpoint{5.012336in}{0.785497in}}%
\pgfpathlineto{\pgfqpoint{5.014914in}{0.779482in}}%
\pgfpathlineto{\pgfqpoint{5.017491in}{0.782384in}}%
\pgfpathlineto{\pgfqpoint{5.027800in}{0.768852in}}%
\pgfpathlineto{\pgfqpoint{5.030377in}{0.774561in}}%
\pgfpathlineto{\pgfqpoint{5.032954in}{0.772374in}}%
\pgfpathlineto{\pgfqpoint{5.035531in}{0.766195in}}%
\pgfpathlineto{\pgfqpoint{5.043263in}{0.762024in}}%
\pgfpathlineto{\pgfqpoint{5.045840in}{0.751164in}}%
\pgfpathlineto{\pgfqpoint{5.048417in}{0.753699in}}%
\pgfpathlineto{\pgfqpoint{5.050995in}{0.758251in}}%
\pgfpathlineto{\pgfqpoint{5.053572in}{0.757048in}}%
\pgfpathlineto{\pgfqpoint{5.061304in}{0.757893in}}%
\pgfpathlineto{\pgfqpoint{5.063881in}{0.760758in}}%
\pgfpathlineto{\pgfqpoint{5.066458in}{0.752776in}}%
\pgfpathlineto{\pgfqpoint{5.069035in}{0.767452in}}%
\pgfpathlineto{\pgfqpoint{5.071612in}{0.775183in}}%
\pgfpathlineto{\pgfqpoint{5.079344in}{0.772759in}}%
\pgfpathlineto{\pgfqpoint{5.081921in}{0.766439in}}%
\pgfpathlineto{\pgfqpoint{5.084498in}{0.774818in}}%
\pgfpathlineto{\pgfqpoint{5.087076in}{0.779506in}}%
\pgfpathlineto{\pgfqpoint{5.089653in}{0.785518in}}%
\pgfpathlineto{\pgfqpoint{5.097384in}{0.787215in}}%
\pgfpathlineto{\pgfqpoint{5.099962in}{0.779594in}}%
\pgfpathlineto{\pgfqpoint{5.102539in}{0.784530in}}%
\pgfpathlineto{\pgfqpoint{5.105116in}{0.787859in}}%
\pgfpathlineto{\pgfqpoint{5.107693in}{0.784546in}}%
\pgfpathlineto{\pgfqpoint{5.115425in}{0.782127in}}%
\pgfpathlineto{\pgfqpoint{5.118002in}{0.785851in}}%
\pgfpathlineto{\pgfqpoint{5.120579in}{0.773823in}}%
\pgfpathlineto{\pgfqpoint{5.123157in}{0.778291in}}%
\pgfpathlineto{\pgfqpoint{5.133465in}{0.770269in}}%
\pgfpathlineto{\pgfqpoint{5.136043in}{0.755269in}}%
\pgfpathlineto{\pgfqpoint{5.138620in}{0.762555in}}%
\pgfpathlineto{\pgfqpoint{5.141197in}{0.764523in}}%
\pgfpathlineto{\pgfqpoint{5.143774in}{0.765223in}}%
\pgfpathlineto{\pgfqpoint{5.151506in}{0.762780in}}%
\pgfpathlineto{\pgfqpoint{5.154083in}{0.762787in}}%
\pgfpathlineto{\pgfqpoint{5.156660in}{0.763644in}}%
\pgfpathlineto{\pgfqpoint{5.159238in}{0.762997in}}%
\pgfpathlineto{\pgfqpoint{5.161815in}{0.766720in}}%
\pgfpathlineto{\pgfqpoint{5.169546in}{0.766333in}}%
\pgfpathlineto{\pgfqpoint{5.172124in}{0.771339in}}%
\pgfpathlineto{\pgfqpoint{5.174701in}{0.766239in}}%
\pgfpathlineto{\pgfqpoint{5.177278in}{0.771632in}}%
\pgfpathlineto{\pgfqpoint{5.192741in}{0.771154in}}%
\pgfpathlineto{\pgfqpoint{5.195319in}{0.772133in}}%
\pgfpathlineto{\pgfqpoint{5.197896in}{0.768210in}}%
\pgfpathlineto{\pgfqpoint{5.205627in}{0.770933in}}%
\pgfpathlineto{\pgfqpoint{5.208205in}{0.767121in}}%
\pgfpathlineto{\pgfqpoint{5.210782in}{0.764945in}}%
\pgfpathlineto{\pgfqpoint{5.213359in}{0.764885in}}%
\pgfpathlineto{\pgfqpoint{5.215936in}{0.750474in}}%
\pgfpathlineto{\pgfqpoint{5.223668in}{0.758373in}}%
\pgfpathlineto{\pgfqpoint{5.226245in}{0.758798in}}%
\pgfpathlineto{\pgfqpoint{5.228822in}{0.762120in}}%
\pgfpathlineto{\pgfqpoint{5.231400in}{0.764196in}}%
\pgfpathlineto{\pgfqpoint{5.233977in}{0.761051in}}%
\pgfpathlineto{\pgfqpoint{5.244286in}{0.768540in}}%
\pgfpathlineto{\pgfqpoint{5.246863in}{0.768249in}}%
\pgfpathlineto{\pgfqpoint{5.252017in}{0.779540in}}%
\pgfpathlineto{\pgfqpoint{5.259749in}{0.784697in}}%
\pgfpathlineto{\pgfqpoint{5.264903in}{0.774246in}}%
\pgfpathlineto{\pgfqpoint{5.267481in}{0.789266in}}%
\pgfpathlineto{\pgfqpoint{5.270058in}{0.783654in}}%
\pgfpathlineto{\pgfqpoint{5.277789in}{0.785670in}}%
\pgfpathlineto{\pgfqpoint{5.280367in}{0.793545in}}%
\pgfpathlineto{\pgfqpoint{5.282944in}{0.793348in}}%
\pgfpathlineto{\pgfqpoint{5.285521in}{0.789667in}}%
\pgfpathlineto{\pgfqpoint{5.288098in}{0.791836in}}%
\pgfpathlineto{\pgfqpoint{5.295830in}{0.796392in}}%
\pgfpathlineto{\pgfqpoint{5.298407in}{0.807471in}}%
\pgfpathlineto{\pgfqpoint{5.300984in}{0.832634in}}%
\pgfpathlineto{\pgfqpoint{5.303561in}{0.794616in}}%
\pgfpathlineto{\pgfqpoint{5.306139in}{0.785993in}}%
\pgfpathlineto{\pgfqpoint{5.313870in}{0.786796in}}%
\pgfpathlineto{\pgfqpoint{5.316448in}{0.814513in}}%
\pgfpathlineto{\pgfqpoint{5.319025in}{0.813314in}}%
\pgfpathlineto{\pgfqpoint{5.321602in}{0.820576in}}%
\pgfpathlineto{\pgfqpoint{5.324179in}{0.832608in}}%
\pgfpathlineto{\pgfqpoint{5.331911in}{0.831653in}}%
\pgfpathlineto{\pgfqpoint{5.334488in}{0.833546in}}%
\pgfpathlineto{\pgfqpoint{5.339642in}{0.830318in}}%
\pgfpathlineto{\pgfqpoint{5.342220in}{0.833524in}}%
\pgfpathlineto{\pgfqpoint{5.349951in}{0.833447in}}%
\pgfpathlineto{\pgfqpoint{5.352529in}{0.834233in}}%
\pgfpathlineto{\pgfqpoint{5.355106in}{0.853588in}}%
\pgfpathlineto{\pgfqpoint{5.357683in}{0.843453in}}%
\pgfpathlineto{\pgfqpoint{5.360260in}{0.840775in}}%
\pgfpathlineto{\pgfqpoint{5.367992in}{0.831803in}}%
\pgfpathlineto{\pgfqpoint{5.370569in}{0.832414in}}%
\pgfpathlineto{\pgfqpoint{5.373146in}{0.834501in}}%
\pgfpathlineto{\pgfqpoint{5.375723in}{0.830889in}}%
\pgfpathlineto{\pgfqpoint{5.378301in}{0.834146in}}%
\pgfpathlineto{\pgfqpoint{5.386032in}{0.836110in}}%
\pgfpathlineto{\pgfqpoint{5.388609in}{0.845323in}}%
\pgfpathlineto{\pgfqpoint{5.391187in}{0.841930in}}%
\pgfpathlineto{\pgfqpoint{5.396341in}{0.844017in}}%
\pgfpathlineto{\pgfqpoint{5.404073in}{0.842953in}}%
\pgfpathlineto{\pgfqpoint{5.406650in}{0.843328in}}%
\pgfpathlineto{\pgfqpoint{5.409227in}{0.848832in}}%
\pgfpathlineto{\pgfqpoint{5.411804in}{0.839047in}}%
\pgfpathlineto{\pgfqpoint{5.414382in}{0.849651in}}%
\pgfpathlineto{\pgfqpoint{5.422113in}{0.856791in}}%
\pgfpathlineto{\pgfqpoint{5.424690in}{0.855805in}}%
\pgfpathlineto{\pgfqpoint{5.427268in}{0.856187in}}%
\pgfpathlineto{\pgfqpoint{5.432422in}{0.865415in}}%
\pgfpathlineto{\pgfqpoint{5.440154in}{0.856196in}}%
\pgfpathlineto{\pgfqpoint{5.442731in}{0.859068in}}%
\pgfpathlineto{\pgfqpoint{5.445308in}{0.859427in}}%
\pgfpathlineto{\pgfqpoint{5.447885in}{0.853476in}}%
\pgfpathlineto{\pgfqpoint{5.450463in}{0.860947in}}%
\pgfpathlineto{\pgfqpoint{5.458194in}{0.858940in}}%
\pgfpathlineto{\pgfqpoint{5.463349in}{0.861535in}}%
\pgfpathlineto{\pgfqpoint{5.465926in}{0.864686in}}%
\pgfpathlineto{\pgfqpoint{5.468503in}{0.868859in}}%
\pgfpathlineto{\pgfqpoint{5.478812in}{0.870987in}}%
\pgfpathlineto{\pgfqpoint{5.481389in}{0.874704in}}%
\pgfpathlineto{\pgfqpoint{5.483966in}{0.873675in}}%
\pgfpathlineto{\pgfqpoint{5.486544in}{0.876814in}}%
\pgfpathlineto{\pgfqpoint{5.496852in}{0.879276in}}%
\pgfpathlineto{\pgfqpoint{5.499430in}{0.878477in}}%
\pgfpathlineto{\pgfqpoint{5.502007in}{0.887355in}}%
\pgfpathlineto{\pgfqpoint{5.504584in}{0.885618in}}%
\pgfpathlineto{\pgfqpoint{5.512316in}{0.885683in}}%
\pgfpathlineto{\pgfqpoint{5.514893in}{0.882023in}}%
\pgfpathlineto{\pgfqpoint{5.517470in}{0.886628in}}%
\pgfpathlineto{\pgfqpoint{5.520047in}{0.897782in}}%
\pgfpathlineto{\pgfqpoint{5.522625in}{0.900882in}}%
\pgfpathlineto{\pgfqpoint{5.532933in}{0.900089in}}%
\pgfpathlineto{\pgfqpoint{5.538088in}{0.906045in}}%
\pgfpathlineto{\pgfqpoint{5.540665in}{0.901820in}}%
\pgfpathlineto{\pgfqpoint{5.548397in}{0.894302in}}%
\pgfpathlineto{\pgfqpoint{5.550974in}{0.880234in}}%
\pgfpathlineto{\pgfqpoint{5.556128in}{0.883921in}}%
\pgfpathlineto{\pgfqpoint{5.558706in}{0.892418in}}%
\pgfpathlineto{\pgfqpoint{5.569014in}{0.882299in}}%
\pgfpathlineto{\pgfqpoint{5.571592in}{0.884584in}}%
\pgfpathlineto{\pgfqpoint{5.574169in}{0.888162in}}%
\pgfpathlineto{\pgfqpoint{5.576746in}{0.889450in}}%
\pgfpathlineto{\pgfqpoint{5.584478in}{0.889057in}}%
\pgfpathlineto{\pgfqpoint{5.587055in}{0.886425in}}%
\pgfpathlineto{\pgfqpoint{5.589632in}{0.886181in}}%
\pgfpathlineto{\pgfqpoint{5.592209in}{0.881103in}}%
\pgfpathlineto{\pgfqpoint{5.594786in}{0.880026in}}%
\pgfpathlineto{\pgfqpoint{5.602518in}{0.881031in}}%
\pgfpathlineto{\pgfqpoint{5.605095in}{0.878335in}}%
\pgfpathlineto{\pgfqpoint{5.607673in}{0.889951in}}%
\pgfpathlineto{\pgfqpoint{5.610250in}{0.897299in}}%
\pgfpathlineto{\pgfqpoint{5.612827in}{0.899222in}}%
\pgfpathlineto{\pgfqpoint{5.623136in}{0.896278in}}%
\pgfpathlineto{\pgfqpoint{5.625713in}{0.893118in}}%
\pgfpathlineto{\pgfqpoint{5.628290in}{0.885584in}}%
\pgfpathlineto{\pgfqpoint{5.630867in}{0.880590in}}%
\pgfpathlineto{\pgfqpoint{5.638599in}{0.883270in}}%
\pgfpathlineto{\pgfqpoint{5.641176in}{0.880325in}}%
\pgfpathlineto{\pgfqpoint{5.643754in}{0.871279in}}%
\pgfpathlineto{\pgfqpoint{5.646331in}{0.881321in}}%
\pgfpathlineto{\pgfqpoint{5.648908in}{0.879274in}}%
\pgfpathlineto{\pgfqpoint{5.656640in}{0.877670in}}%
\pgfpathlineto{\pgfqpoint{5.659217in}{0.881526in}}%
\pgfpathlineto{\pgfqpoint{5.661794in}{0.882288in}}%
\pgfpathlineto{\pgfqpoint{5.664371in}{0.877718in}}%
\pgfpathlineto{\pgfqpoint{5.666948in}{0.884991in}}%
\pgfpathlineto{\pgfqpoint{5.674680in}{0.890824in}}%
\pgfpathlineto{\pgfqpoint{5.677257in}{0.886270in}}%
\pgfpathlineto{\pgfqpoint{5.679835in}{0.895011in}}%
\pgfpathlineto{\pgfqpoint{5.682412in}{0.899669in}}%
\pgfpathlineto{\pgfqpoint{5.684989in}{0.909998in}}%
\pgfpathlineto{\pgfqpoint{5.692721in}{0.907226in}}%
\pgfpathlineto{\pgfqpoint{5.695298in}{0.903233in}}%
\pgfpathlineto{\pgfqpoint{5.697875in}{0.910732in}}%
\pgfpathlineto{\pgfqpoint{5.700452in}{0.908529in}}%
\pgfpathlineto{\pgfqpoint{5.703029in}{0.902804in}}%
\pgfpathlineto{\pgfqpoint{5.710761in}{0.910929in}}%
\pgfpathlineto{\pgfqpoint{5.715915in}{0.915251in}}%
\pgfpathlineto{\pgfqpoint{5.718493in}{0.913921in}}%
\pgfpathlineto{\pgfqpoint{5.721070in}{0.913551in}}%
\pgfpathlineto{\pgfqpoint{5.728802in}{0.916173in}}%
\pgfpathlineto{\pgfqpoint{5.731379in}{0.914807in}}%
\pgfpathlineto{\pgfqpoint{5.733956in}{0.917580in}}%
\pgfpathlineto{\pgfqpoint{5.736533in}{0.916064in}}%
\pgfpathlineto{\pgfqpoint{5.746842in}{0.917784in}}%
\pgfpathlineto{\pgfqpoint{5.749419in}{0.915318in}}%
\pgfpathlineto{\pgfqpoint{5.751996in}{0.908535in}}%
\pgfpathlineto{\pgfqpoint{5.754574in}{0.907664in}}%
\pgfpathlineto{\pgfqpoint{5.764883in}{0.908507in}}%
\pgfpathlineto{\pgfqpoint{5.767460in}{0.919925in}}%
\pgfpathlineto{\pgfqpoint{5.770037in}{0.917808in}}%
\pgfpathlineto{\pgfqpoint{5.775191in}{0.894177in}}%
\pgfpathlineto{\pgfqpoint{5.782923in}{0.882845in}}%
\pgfpathlineto{\pgfqpoint{5.785500in}{0.881788in}}%
\pgfpathlineto{\pgfqpoint{5.788077in}{0.893838in}}%
\pgfpathlineto{\pgfqpoint{5.790655in}{0.895101in}}%
\pgfpathlineto{\pgfqpoint{5.793232in}{0.901754in}}%
\pgfpathlineto{\pgfqpoint{5.800963in}{0.905646in}}%
\pgfpathlineto{\pgfqpoint{5.806118in}{0.882978in}}%
\pgfpathlineto{\pgfqpoint{5.808695in}{0.876664in}}%
\pgfpathlineto{\pgfqpoint{5.811272in}{0.882921in}}%
\pgfpathlineto{\pgfqpoint{5.819004in}{0.883969in}}%
\pgfpathlineto{\pgfqpoint{5.821581in}{0.886988in}}%
\pgfpathlineto{\pgfqpoint{5.826736in}{0.888066in}}%
\pgfpathlineto{\pgfqpoint{5.829313in}{0.891111in}}%
\pgfpathlineto{\pgfqpoint{5.837044in}{0.880940in}}%
\pgfpathlineto{\pgfqpoint{5.839622in}{0.874554in}}%
\pgfpathlineto{\pgfqpoint{5.842199in}{0.871285in}}%
\pgfpathlineto{\pgfqpoint{5.847353in}{0.875634in}}%
\pgfpathlineto{\pgfqpoint{5.855085in}{0.878678in}}%
\pgfpathlineto{\pgfqpoint{5.857662in}{0.876282in}}%
\pgfpathlineto{\pgfqpoint{5.860239in}{0.880735in}}%
\pgfpathlineto{\pgfqpoint{5.862817in}{0.875515in}}%
\pgfpathlineto{\pgfqpoint{5.865394in}{0.878957in}}%
\pgfpathlineto{\pgfqpoint{5.875703in}{0.876479in}}%
\pgfpathlineto{\pgfqpoint{5.878280in}{0.867772in}}%
\pgfpathlineto{\pgfqpoint{5.880857in}{0.877174in}}%
\pgfpathlineto{\pgfqpoint{5.883434in}{0.877849in}}%
\pgfpathlineto{\pgfqpoint{5.891166in}{0.874829in}}%
\pgfpathlineto{\pgfqpoint{5.893743in}{0.871851in}}%
\pgfpathlineto{\pgfqpoint{5.896320in}{0.874512in}}%
\pgfpathlineto{\pgfqpoint{5.898898in}{0.879130in}}%
\pgfpathlineto{\pgfqpoint{5.901475in}{0.892263in}}%
\pgfpathlineto{\pgfqpoint{5.909206in}{0.867755in}}%
\pgfpathlineto{\pgfqpoint{5.911784in}{0.883876in}}%
\pgfpathlineto{\pgfqpoint{5.914361in}{0.886636in}}%
\pgfpathlineto{\pgfqpoint{5.916938in}{0.875652in}}%
\pgfpathlineto{\pgfqpoint{5.919515in}{0.868932in}}%
\pgfpathlineto{\pgfqpoint{5.927247in}{0.876781in}}%
\pgfpathlineto{\pgfqpoint{5.929824in}{0.876301in}}%
\pgfpathlineto{\pgfqpoint{5.932401in}{0.867276in}}%
\pgfpathlineto{\pgfqpoint{5.934979in}{0.862858in}}%
\pgfpathlineto{\pgfqpoint{5.937556in}{0.859919in}}%
\pgfpathlineto{\pgfqpoint{5.945287in}{0.863574in}}%
\pgfpathlineto{\pgfqpoint{5.947865in}{0.857175in}}%
\pgfpathlineto{\pgfqpoint{5.950442in}{0.845434in}}%
\pgfpathlineto{\pgfqpoint{5.953019in}{0.856257in}}%
\pgfpathlineto{\pgfqpoint{5.955596in}{0.853183in}}%
\pgfpathlineto{\pgfqpoint{5.963328in}{0.852500in}}%
\pgfpathlineto{\pgfqpoint{5.968482in}{0.874109in}}%
\pgfpathlineto{\pgfqpoint{5.971060in}{0.874880in}}%
\pgfpathlineto{\pgfqpoint{5.973637in}{0.871894in}}%
\pgfpathlineto{\pgfqpoint{5.981368in}{0.868683in}}%
\pgfpathlineto{\pgfqpoint{5.983946in}{0.871270in}}%
\pgfpathlineto{\pgfqpoint{5.986523in}{0.870579in}}%
\pgfpathlineto{\pgfqpoint{5.989100in}{0.868384in}}%
\pgfpathlineto{\pgfqpoint{5.991677in}{0.871950in}}%
\pgfpathlineto{\pgfqpoint{5.999409in}{0.875222in}}%
\pgfpathlineto{\pgfqpoint{6.001986in}{0.880779in}}%
\pgfpathlineto{\pgfqpoint{6.004563in}{0.878976in}}%
\pgfpathlineto{\pgfqpoint{6.007140in}{0.887007in}}%
\pgfpathlineto{\pgfqpoint{6.009718in}{0.879849in}}%
\pgfpathlineto{\pgfqpoint{6.017449in}{0.863343in}}%
\pgfpathlineto{\pgfqpoint{6.020027in}{0.863826in}}%
\pgfpathlineto{\pgfqpoint{6.022604in}{0.855632in}}%
\pgfpathlineto{\pgfqpoint{6.025181in}{0.826654in}}%
\pgfpathlineto{\pgfqpoint{6.027758in}{0.828525in}}%
\pgfpathlineto{\pgfqpoint{6.035490in}{0.828582in}}%
\pgfpathlineto{\pgfqpoint{6.040644in}{0.819399in}}%
\pgfpathlineto{\pgfqpoint{6.043221in}{0.817540in}}%
\pgfpathlineto{\pgfqpoint{6.045799in}{0.817465in}}%
\pgfpathlineto{\pgfqpoint{6.053530in}{0.815068in}}%
\pgfpathlineto{\pgfqpoint{6.056108in}{0.816492in}}%
\pgfpathlineto{\pgfqpoint{6.058685in}{0.827339in}}%
\pgfpathlineto{\pgfqpoint{6.061262in}{0.826965in}}%
\pgfpathlineto{\pgfqpoint{6.063839in}{0.828295in}}%
\pgfpathlineto{\pgfqpoint{6.071571in}{0.830105in}}%
\pgfpathlineto{\pgfqpoint{6.074148in}{0.826063in}}%
\pgfpathlineto{\pgfqpoint{6.076725in}{0.827382in}}%
\pgfpathlineto{\pgfqpoint{6.079302in}{0.825578in}}%
\pgfpathlineto{\pgfqpoint{6.081880in}{0.822673in}}%
\pgfpathlineto{\pgfqpoint{6.089611in}{0.819870in}}%
\pgfpathlineto{\pgfqpoint{6.092188in}{0.813538in}}%
\pgfpathlineto{\pgfqpoint{6.094766in}{0.822556in}}%
\pgfpathlineto{\pgfqpoint{6.097343in}{0.828124in}}%
\pgfpathlineto{\pgfqpoint{6.099920in}{0.830367in}}%
\pgfpathlineto{\pgfqpoint{6.107652in}{0.831449in}}%
\pgfpathlineto{\pgfqpoint{6.110229in}{0.819183in}}%
\pgfpathlineto{\pgfqpoint{6.112806in}{0.814593in}}%
\pgfpathlineto{\pgfqpoint{6.115383in}{0.814842in}}%
\pgfpathlineto{\pgfqpoint{6.117961in}{0.810069in}}%
\pgfpathlineto{\pgfqpoint{6.128269in}{0.793958in}}%
\pgfpathlineto{\pgfqpoint{6.130847in}{0.791531in}}%
\pgfpathlineto{\pgfqpoint{6.133424in}{0.803273in}}%
\pgfpathlineto{\pgfqpoint{6.136001in}{0.798104in}}%
\pgfpathlineto{\pgfqpoint{6.146310in}{0.806088in}}%
\pgfpathlineto{\pgfqpoint{6.148887in}{0.801504in}}%
\pgfpathlineto{\pgfqpoint{6.151464in}{0.811678in}}%
\pgfpathlineto{\pgfqpoint{6.154042in}{0.817235in}}%
\pgfpathlineto{\pgfqpoint{6.161773in}{0.822302in}}%
\pgfpathlineto{\pgfqpoint{6.164350in}{0.824913in}}%
\pgfpathlineto{\pgfqpoint{6.166928in}{0.822421in}}%
\pgfpathlineto{\pgfqpoint{6.169505in}{0.812164in}}%
\pgfpathlineto{\pgfqpoint{6.172082in}{0.805804in}}%
\pgfpathlineto{\pgfqpoint{6.179814in}{0.807949in}}%
\pgfpathlineto{\pgfqpoint{6.182391in}{0.815677in}}%
\pgfpathlineto{\pgfqpoint{6.184968in}{0.813887in}}%
\pgfpathlineto{\pgfqpoint{6.187545in}{0.813726in}}%
\pgfpathlineto{\pgfqpoint{6.190123in}{0.812601in}}%
\pgfpathlineto{\pgfqpoint{6.197854in}{0.811827in}}%
\pgfpathlineto{\pgfqpoint{6.203009in}{0.812717in}}%
\pgfpathlineto{\pgfqpoint{6.205586in}{0.815117in}}%
\pgfpathlineto{\pgfqpoint{6.208163in}{0.812723in}}%
\pgfpathlineto{\pgfqpoint{6.215895in}{0.797692in}}%
\pgfpathlineto{\pgfqpoint{6.218472in}{0.797889in}}%
\pgfpathlineto{\pgfqpoint{6.221049in}{0.781778in}}%
\pgfpathlineto{\pgfqpoint{6.226204in}{0.770264in}}%
\pgfpathlineto{\pgfqpoint{6.233935in}{0.774187in}}%
\pgfpathlineto{\pgfqpoint{6.239090in}{0.770522in}}%
\pgfpathlineto{\pgfqpoint{6.241667in}{0.776124in}}%
\pgfpathlineto{\pgfqpoint{6.244244in}{0.757561in}}%
\pgfpathlineto{\pgfqpoint{6.251976in}{0.778688in}}%
\pgfpathlineto{\pgfqpoint{6.254553in}{0.774895in}}%
\pgfpathlineto{\pgfqpoint{6.257130in}{0.766605in}}%
\pgfpathlineto{\pgfqpoint{6.259707in}{0.772834in}}%
\pgfpathlineto{\pgfqpoint{6.262285in}{0.737811in}}%
\pgfpathlineto{\pgfqpoint{6.270016in}{0.739553in}}%
\pgfpathlineto{\pgfqpoint{6.272593in}{0.735679in}}%
\pgfpathlineto{\pgfqpoint{6.275171in}{0.723563in}}%
\pgfpathlineto{\pgfqpoint{6.277748in}{0.718714in}}%
\pgfpathlineto{\pgfqpoint{6.280325in}{0.720727in}}%
\pgfpathlineto{\pgfqpoint{6.288057in}{0.724495in}}%
\pgfpathlineto{\pgfqpoint{6.290634in}{0.720127in}}%
\pgfpathlineto{\pgfqpoint{6.293211in}{0.717805in}}%
\pgfpathlineto{\pgfqpoint{6.295788in}{0.722524in}}%
\pgfpathlineto{\pgfqpoint{6.298365in}{0.711652in}}%
\pgfpathlineto{\pgfqpoint{6.306097in}{0.739982in}}%
\pgfpathlineto{\pgfqpoint{6.308674in}{0.724613in}}%
\pgfpathlineto{\pgfqpoint{6.311252in}{0.743343in}}%
\pgfpathlineto{\pgfqpoint{6.313829in}{0.750412in}}%
\pgfpathlineto{\pgfqpoint{6.316406in}{0.745513in}}%
\pgfpathlineto{\pgfqpoint{6.324138in}{0.737409in}}%
\pgfpathlineto{\pgfqpoint{6.326715in}{0.731154in}}%
\pgfpathlineto{\pgfqpoint{6.329292in}{0.749268in}}%
\pgfpathlineto{\pgfqpoint{6.334446in}{0.751836in}}%
\pgfpathlineto{\pgfqpoint{6.342178in}{0.753055in}}%
\pgfpathlineto{\pgfqpoint{6.344755in}{0.749438in}}%
\pgfpathlineto{\pgfqpoint{6.347333in}{0.737306in}}%
\pgfpathlineto{\pgfqpoint{6.349910in}{0.748148in}}%
\pgfpathlineto{\pgfqpoint{6.352487in}{0.751534in}}%
\pgfpathlineto{\pgfqpoint{6.360219in}{0.741088in}}%
\pgfpathlineto{\pgfqpoint{6.362796in}{0.753680in}}%
\pgfpathlineto{\pgfqpoint{6.365373in}{0.739812in}}%
\pgfpathlineto{\pgfqpoint{6.370527in}{0.735796in}}%
\pgfpathlineto{\pgfqpoint{6.378259in}{0.742173in}}%
\pgfpathlineto{\pgfqpoint{6.380836in}{0.739484in}}%
\pgfpathlineto{\pgfqpoint{6.383414in}{0.748822in}}%
\pgfpathlineto{\pgfqpoint{6.385991in}{0.745017in}}%
\pgfpathlineto{\pgfqpoint{6.388568in}{0.743430in}}%
\pgfpathlineto{\pgfqpoint{6.396300in}{0.726517in}}%
\pgfpathlineto{\pgfqpoint{6.398877in}{0.718680in}}%
\pgfpathlineto{\pgfqpoint{6.401454in}{0.718588in}}%
\pgfpathlineto{\pgfqpoint{6.404031in}{0.723894in}}%
\pgfpathlineto{\pgfqpoint{6.406608in}{0.721129in}}%
\pgfpathlineto{\pgfqpoint{6.419494in}{0.716519in}}%
\pgfpathlineto{\pgfqpoint{6.422072in}{0.713614in}}%
\pgfpathlineto{\pgfqpoint{6.424649in}{0.717870in}}%
\pgfpathlineto{\pgfqpoint{6.424649in}{0.717870in}}%
\pgfusepath{stroke}%
\end{pgfscope}%
\begin{pgfscope}%
\pgfsetrectcap%
\pgfsetmiterjoin%
\pgfsetlinewidth{0.803000pt}%
\definecolor{currentstroke}{rgb}{1.000000,1.000000,1.000000}%
\pgfsetstrokecolor{currentstroke}%
\pgfsetdash{}{0pt}%
\pgfpathmoveto{\pgfqpoint{0.506467in}{0.331635in}}%
\pgfpathlineto{\pgfqpoint{0.506467in}{2.596635in}}%
\pgfusepath{stroke}%
\end{pgfscope}%
\begin{pgfscope}%
\pgfsetrectcap%
\pgfsetmiterjoin%
\pgfsetlinewidth{0.803000pt}%
\definecolor{currentstroke}{rgb}{1.000000,1.000000,1.000000}%
\pgfsetstrokecolor{currentstroke}%
\pgfsetdash{}{0pt}%
\pgfpathmoveto{\pgfqpoint{6.706467in}{0.331635in}}%
\pgfpathlineto{\pgfqpoint{6.706467in}{2.596635in}}%
\pgfusepath{stroke}%
\end{pgfscope}%
\begin{pgfscope}%
\pgfsetrectcap%
\pgfsetmiterjoin%
\pgfsetlinewidth{0.803000pt}%
\definecolor{currentstroke}{rgb}{1.000000,1.000000,1.000000}%
\pgfsetstrokecolor{currentstroke}%
\pgfsetdash{}{0pt}%
\pgfpathmoveto{\pgfqpoint{0.506467in}{0.331635in}}%
\pgfpathlineto{\pgfqpoint{6.706467in}{0.331635in}}%
\pgfusepath{stroke}%
\end{pgfscope}%
\begin{pgfscope}%
\pgfsetrectcap%
\pgfsetmiterjoin%
\pgfsetlinewidth{0.803000pt}%
\definecolor{currentstroke}{rgb}{1.000000,1.000000,1.000000}%
\pgfsetstrokecolor{currentstroke}%
\pgfsetdash{}{0pt}%
\pgfpathmoveto{\pgfqpoint{0.506467in}{2.596635in}}%
\pgfpathlineto{\pgfqpoint{6.706467in}{2.596635in}}%
\pgfusepath{stroke}%
\end{pgfscope}%
\begin{pgfscope}%
\pgfsetbuttcap%
\pgfsetmiterjoin%
\definecolor{currentfill}{rgb}{0.917647,0.917647,0.949020}%
\pgfsetfillcolor{currentfill}%
\pgfsetfillopacity{0.800000}%
\pgfsetlinewidth{1.003750pt}%
\definecolor{currentstroke}{rgb}{0.800000,0.800000,0.800000}%
\pgfsetstrokecolor{currentstroke}%
\pgfsetstrokeopacity{0.800000}%
\pgfsetdash{}{0pt}%
\pgfpathmoveto{\pgfqpoint{0.603689in}{2.075843in}}%
\pgfpathlineto{\pgfqpoint{1.653670in}{2.075843in}}%
\pgfpathquadraticcurveto{\pgfqpoint{1.681448in}{2.075843in}}{\pgfqpoint{1.681448in}{2.103621in}}%
\pgfpathlineto{\pgfqpoint{1.681448in}{2.499413in}}%
\pgfpathquadraticcurveto{\pgfqpoint{1.681448in}{2.527191in}}{\pgfqpoint{1.653670in}{2.527191in}}%
\pgfpathlineto{\pgfqpoint{0.603689in}{2.527191in}}%
\pgfpathquadraticcurveto{\pgfqpoint{0.575912in}{2.527191in}}{\pgfqpoint{0.575912in}{2.499413in}}%
\pgfpathlineto{\pgfqpoint{0.575912in}{2.103621in}}%
\pgfpathquadraticcurveto{\pgfqpoint{0.575912in}{2.075843in}}{\pgfqpoint{0.603689in}{2.075843in}}%
\pgfpathclose%
\pgfusepath{stroke,fill}%
\end{pgfscope}%
\begin{pgfscope}%
\pgfsetroundcap%
\pgfsetroundjoin%
\pgfsetlinewidth{1.505625pt}%
\definecolor{currentstroke}{rgb}{0.121569,0.466667,0.705882}%
\pgfsetstrokecolor{currentstroke}%
\pgfsetdash{}{0pt}%
\pgfpathmoveto{\pgfqpoint{0.631467in}{2.414723in}}%
\pgfpathlineto{\pgfqpoint{0.909245in}{2.414723in}}%
\pgfusepath{stroke}%
\end{pgfscope}%
\begin{pgfscope}%
\definecolor{textcolor}{rgb}{0.150000,0.150000,0.150000}%
\pgfsetstrokecolor{textcolor}%
\pgfsetfillcolor{textcolor}%
\pgftext[x=1.020356in,y=2.366112in,left,base]{\color{textcolor}\rmfamily\fontsize{10.000000}{12.000000}\selectfont Baseline}%
\end{pgfscope}%
\begin{pgfscope}%
\pgfsetroundcap%
\pgfsetroundjoin%
\pgfsetlinewidth{1.505625pt}%
\definecolor{currentstroke}{rgb}{1.000000,0.498039,0.054902}%
\pgfsetstrokecolor{currentstroke}%
\pgfsetdash{}{0pt}%
\pgfpathmoveto{\pgfqpoint{0.631467in}{2.210866in}}%
\pgfpathlineto{\pgfqpoint{0.909245in}{2.210866in}}%
\pgfusepath{stroke}%
\end{pgfscope}%
\begin{pgfscope}%
\definecolor{textcolor}{rgb}{0.150000,0.150000,0.150000}%
\pgfsetstrokecolor{textcolor}%
\pgfsetfillcolor{textcolor}%
\pgftext[x=1.020356in,y=2.162255in,left,base]{\color{textcolor}\rmfamily\fontsize{10.000000}{12.000000}\selectfont Strategy}%
\end{pgfscope}%
\end{pgfpicture}%
\makeatother%
\endgroup%

    \end{adjustbox}  
    \caption{Caption}
    \label{fig:mean reversion}
\end{figure}{}

Implementation algorithm: rank portfolio everyday along their returns from the last day. Then buy the two worst performers and sell the two best performers. Interestingly, reversing that does not produce better results. The time frame is obviously not right. 

\subsection{Mean Reversion - single stocks}

\subsubsection{Cumulative Mean - Stock prices}
Idea: 
Plot: Cumulative Mean vs. Actual Time Series
--> We see that the cumulative mean does not capture the time series well. Trend is always behind the current development, since we have a trend

\begin{figure}
    \centering
    \begin{minipage}[b]{0.49\textwidth}
        \centering
            \begin{adjustbox}{width=\textwidth,center}
                %% Creator: Matplotlib, PGF backend
%%
%% To include the figure in your LaTeX document, write
%%   \input{<filename>.pgf}
%%
%% Make sure the required packages are loaded in your preamble
%%   \usepackage{pgf}
%%
%% Figures using additional raster images can only be included by \input if
%% they are in the same directory as the main LaTeX file. For loading figures
%% from other directories you can use the `import` package
%%   \usepackage{import}
%% and then include the figures with
%%   \import{<path to file>}{<filename>.pgf}
%%
%% Matplotlib used the following preamble
%%   \usepackage{fontspec}
%%   \setmainfont{DejaVuSerif.ttf}[Path=/opt/tljh/user/lib/python3.6/site-packages/matplotlib/mpl-data/fonts/ttf/]
%%   \setsansfont{DejaVuSans.ttf}[Path=/opt/tljh/user/lib/python3.6/site-packages/matplotlib/mpl-data/fonts/ttf/]
%%   \setmonofont{DejaVuSansMono.ttf}[Path=/opt/tljh/user/lib/python3.6/site-packages/matplotlib/mpl-data/fonts/ttf/]
%%
\begingroup%
\makeatletter%
\begin{pgfpicture}%
\pgfpathrectangle{\pgfpointorigin}{\pgfqpoint{7.001577in}{6.786012in}}%
\pgfusepath{use as bounding box, clip}%
\begin{pgfscope}%
\pgfsetbuttcap%
\pgfsetmiterjoin%
\definecolor{currentfill}{rgb}{1.000000,1.000000,1.000000}%
\pgfsetfillcolor{currentfill}%
\pgfsetlinewidth{0.000000pt}%
\definecolor{currentstroke}{rgb}{1.000000,1.000000,1.000000}%
\pgfsetstrokecolor{currentstroke}%
\pgfsetdash{}{0pt}%
\pgfpathmoveto{\pgfqpoint{0.000000in}{0.000000in}}%
\pgfpathlineto{\pgfqpoint{7.001577in}{0.000000in}}%
\pgfpathlineto{\pgfqpoint{7.001577in}{6.786012in}}%
\pgfpathlineto{\pgfqpoint{0.000000in}{6.786012in}}%
\pgfpathclose%
\pgfusepath{fill}%
\end{pgfscope}%
\begin{pgfscope}%
\pgfsetbuttcap%
\pgfsetmiterjoin%
\definecolor{currentfill}{rgb}{0.917647,0.917647,0.949020}%
\pgfsetfillcolor{currentfill}%
\pgfsetlinewidth{0.000000pt}%
\definecolor{currentstroke}{rgb}{0.000000,0.000000,0.000000}%
\pgfsetstrokecolor{currentstroke}%
\pgfsetstrokeopacity{0.000000}%
\pgfsetdash{}{0pt}%
\pgfpathmoveto{\pgfqpoint{0.568357in}{5.688815in}}%
\pgfpathlineto{\pgfqpoint{3.151690in}{5.688815in}}%
\pgfpathlineto{\pgfqpoint{3.151690in}{6.425400in}}%
\pgfpathlineto{\pgfqpoint{0.568357in}{6.425400in}}%
\pgfpathclose%
\pgfusepath{fill}%
\end{pgfscope}%
\begin{pgfscope}%
\pgfpathrectangle{\pgfqpoint{0.568357in}{5.688815in}}{\pgfqpoint{2.583333in}{0.736585in}}%
\pgfusepath{clip}%
\pgfsetroundcap%
\pgfsetroundjoin%
\pgfsetlinewidth{0.803000pt}%
\definecolor{currentstroke}{rgb}{1.000000,1.000000,1.000000}%
\pgfsetstrokecolor{currentstroke}%
\pgfsetdash{}{0pt}%
\pgfpathmoveto{\pgfqpoint{0.683633in}{5.688815in}}%
\pgfpathlineto{\pgfqpoint{0.683633in}{6.425400in}}%
\pgfusepath{stroke}%
\end{pgfscope}%
\begin{pgfscope}%
\definecolor{textcolor}{rgb}{0.150000,0.150000,0.150000}%
\pgfsetstrokecolor{textcolor}%
\pgfsetfillcolor{textcolor}%
\pgftext[x=0.683633in,y=5.591593in,,top]{\color{textcolor}\rmfamily\fontsize{14.000000}{16.800000}\selectfont 2012}%
\end{pgfscope}%
\begin{pgfscope}%
\pgfpathrectangle{\pgfqpoint{0.568357in}{5.688815in}}{\pgfqpoint{2.583333in}{0.736585in}}%
\pgfusepath{clip}%
\pgfsetroundcap%
\pgfsetroundjoin%
\pgfsetlinewidth{0.803000pt}%
\definecolor{currentstroke}{rgb}{1.000000,1.000000,1.000000}%
\pgfsetstrokecolor{currentstroke}%
\pgfsetdash{}{0pt}%
\pgfpathmoveto{\pgfqpoint{1.076658in}{5.688815in}}%
\pgfpathlineto{\pgfqpoint{1.076658in}{6.425400in}}%
\pgfusepath{stroke}%
\end{pgfscope}%
\begin{pgfscope}%
\definecolor{textcolor}{rgb}{0.150000,0.150000,0.150000}%
\pgfsetstrokecolor{textcolor}%
\pgfsetfillcolor{textcolor}%
\pgftext[x=1.076658in,y=5.591593in,,top]{\color{textcolor}\rmfamily\fontsize{14.000000}{16.800000}\selectfont 2013}%
\end{pgfscope}%
\begin{pgfscope}%
\pgfpathrectangle{\pgfqpoint{0.568357in}{5.688815in}}{\pgfqpoint{2.583333in}{0.736585in}}%
\pgfusepath{clip}%
\pgfsetroundcap%
\pgfsetroundjoin%
\pgfsetlinewidth{0.803000pt}%
\definecolor{currentstroke}{rgb}{1.000000,1.000000,1.000000}%
\pgfsetstrokecolor{currentstroke}%
\pgfsetdash{}{0pt}%
\pgfpathmoveto{\pgfqpoint{1.468609in}{5.688815in}}%
\pgfpathlineto{\pgfqpoint{1.468609in}{6.425400in}}%
\pgfusepath{stroke}%
\end{pgfscope}%
\begin{pgfscope}%
\definecolor{textcolor}{rgb}{0.150000,0.150000,0.150000}%
\pgfsetstrokecolor{textcolor}%
\pgfsetfillcolor{textcolor}%
\pgftext[x=1.468609in,y=5.591593in,,top]{\color{textcolor}\rmfamily\fontsize{14.000000}{16.800000}\selectfont 2014}%
\end{pgfscope}%
\begin{pgfscope}%
\pgfpathrectangle{\pgfqpoint{0.568357in}{5.688815in}}{\pgfqpoint{2.583333in}{0.736585in}}%
\pgfusepath{clip}%
\pgfsetroundcap%
\pgfsetroundjoin%
\pgfsetlinewidth{0.803000pt}%
\definecolor{currentstroke}{rgb}{1.000000,1.000000,1.000000}%
\pgfsetstrokecolor{currentstroke}%
\pgfsetdash{}{0pt}%
\pgfpathmoveto{\pgfqpoint{1.860560in}{5.688815in}}%
\pgfpathlineto{\pgfqpoint{1.860560in}{6.425400in}}%
\pgfusepath{stroke}%
\end{pgfscope}%
\begin{pgfscope}%
\definecolor{textcolor}{rgb}{0.150000,0.150000,0.150000}%
\pgfsetstrokecolor{textcolor}%
\pgfsetfillcolor{textcolor}%
\pgftext[x=1.860560in,y=5.591593in,,top]{\color{textcolor}\rmfamily\fontsize{14.000000}{16.800000}\selectfont 2015}%
\end{pgfscope}%
\begin{pgfscope}%
\pgfpathrectangle{\pgfqpoint{0.568357in}{5.688815in}}{\pgfqpoint{2.583333in}{0.736585in}}%
\pgfusepath{clip}%
\pgfsetroundcap%
\pgfsetroundjoin%
\pgfsetlinewidth{0.803000pt}%
\definecolor{currentstroke}{rgb}{1.000000,1.000000,1.000000}%
\pgfsetstrokecolor{currentstroke}%
\pgfsetdash{}{0pt}%
\pgfpathmoveto{\pgfqpoint{2.252511in}{5.688815in}}%
\pgfpathlineto{\pgfqpoint{2.252511in}{6.425400in}}%
\pgfusepath{stroke}%
\end{pgfscope}%
\begin{pgfscope}%
\definecolor{textcolor}{rgb}{0.150000,0.150000,0.150000}%
\pgfsetstrokecolor{textcolor}%
\pgfsetfillcolor{textcolor}%
\pgftext[x=2.252511in,y=5.591593in,,top]{\color{textcolor}\rmfamily\fontsize{14.000000}{16.800000}\selectfont 2016}%
\end{pgfscope}%
\begin{pgfscope}%
\pgfpathrectangle{\pgfqpoint{0.568357in}{5.688815in}}{\pgfqpoint{2.583333in}{0.736585in}}%
\pgfusepath{clip}%
\pgfsetroundcap%
\pgfsetroundjoin%
\pgfsetlinewidth{0.803000pt}%
\definecolor{currentstroke}{rgb}{1.000000,1.000000,1.000000}%
\pgfsetstrokecolor{currentstroke}%
\pgfsetdash{}{0pt}%
\pgfpathmoveto{\pgfqpoint{2.645536in}{5.688815in}}%
\pgfpathlineto{\pgfqpoint{2.645536in}{6.425400in}}%
\pgfusepath{stroke}%
\end{pgfscope}%
\begin{pgfscope}%
\definecolor{textcolor}{rgb}{0.150000,0.150000,0.150000}%
\pgfsetstrokecolor{textcolor}%
\pgfsetfillcolor{textcolor}%
\pgftext[x=2.645536in,y=5.591593in,,top]{\color{textcolor}\rmfamily\fontsize{14.000000}{16.800000}\selectfont 2017}%
\end{pgfscope}%
\begin{pgfscope}%
\pgfpathrectangle{\pgfqpoint{0.568357in}{5.688815in}}{\pgfqpoint{2.583333in}{0.736585in}}%
\pgfusepath{clip}%
\pgfsetroundcap%
\pgfsetroundjoin%
\pgfsetlinewidth{0.803000pt}%
\definecolor{currentstroke}{rgb}{1.000000,1.000000,1.000000}%
\pgfsetstrokecolor{currentstroke}%
\pgfsetdash{}{0pt}%
\pgfpathmoveto{\pgfqpoint{3.037487in}{5.688815in}}%
\pgfpathlineto{\pgfqpoint{3.037487in}{6.425400in}}%
\pgfusepath{stroke}%
\end{pgfscope}%
\begin{pgfscope}%
\definecolor{textcolor}{rgb}{0.150000,0.150000,0.150000}%
\pgfsetstrokecolor{textcolor}%
\pgfsetfillcolor{textcolor}%
\pgftext[x=3.037487in,y=5.591593in,,top]{\color{textcolor}\rmfamily\fontsize{14.000000}{16.800000}\selectfont 2018}%
\end{pgfscope}%
\begin{pgfscope}%
\pgfpathrectangle{\pgfqpoint{0.568357in}{5.688815in}}{\pgfqpoint{2.583333in}{0.736585in}}%
\pgfusepath{clip}%
\pgfsetroundcap%
\pgfsetroundjoin%
\pgfsetlinewidth{0.803000pt}%
\definecolor{currentstroke}{rgb}{1.000000,1.000000,1.000000}%
\pgfsetstrokecolor{currentstroke}%
\pgfsetdash{}{0pt}%
\pgfpathmoveto{\pgfqpoint{0.568357in}{5.852601in}}%
\pgfpathlineto{\pgfqpoint{3.151690in}{5.852601in}}%
\pgfusepath{stroke}%
\end{pgfscope}%
\begin{pgfscope}%
\definecolor{textcolor}{rgb}{0.150000,0.150000,0.150000}%
\pgfsetstrokecolor{textcolor}%
\pgfsetfillcolor{textcolor}%
\pgftext[x=0.100000in,y=5.778735in,left,base]{\color{textcolor}\rmfamily\fontsize{14.000000}{16.800000}\selectfont 100}%
\end{pgfscope}%
\begin{pgfscope}%
\pgfpathrectangle{\pgfqpoint{0.568357in}{5.688815in}}{\pgfqpoint{2.583333in}{0.736585in}}%
\pgfusepath{clip}%
\pgfsetroundcap%
\pgfsetroundjoin%
\pgfsetlinewidth{0.803000pt}%
\definecolor{currentstroke}{rgb}{1.000000,1.000000,1.000000}%
\pgfsetstrokecolor{currentstroke}%
\pgfsetdash{}{0pt}%
\pgfpathmoveto{\pgfqpoint{0.568357in}{6.263917in}}%
\pgfpathlineto{\pgfqpoint{3.151690in}{6.263917in}}%
\pgfusepath{stroke}%
\end{pgfscope}%
\begin{pgfscope}%
\definecolor{textcolor}{rgb}{0.150000,0.150000,0.150000}%
\pgfsetstrokecolor{textcolor}%
\pgfsetfillcolor{textcolor}%
\pgftext[x=0.100000in,y=6.190051in,left,base]{\color{textcolor}\rmfamily\fontsize{14.000000}{16.800000}\selectfont 200}%
\end{pgfscope}%
\begin{pgfscope}%
\pgfpathrectangle{\pgfqpoint{0.568357in}{5.688815in}}{\pgfqpoint{2.583333in}{0.736585in}}%
\pgfusepath{clip}%
\pgfsetroundcap%
\pgfsetroundjoin%
\pgfsetlinewidth{1.505625pt}%
\definecolor{currentstroke}{rgb}{0.121569,0.466667,0.705882}%
\pgfsetstrokecolor{currentstroke}%
\pgfsetdash{}{0pt}%
\pgfpathmoveto{\pgfqpoint{0.685781in}{5.722666in}}%
\pgfpathlineto{\pgfqpoint{0.686855in}{5.725011in}}%
\pgfpathlineto{\pgfqpoint{0.689002in}{5.722296in}}%
\pgfpathlineto{\pgfqpoint{0.692224in}{5.723982in}}%
\pgfpathlineto{\pgfqpoint{0.693298in}{5.725422in}}%
\pgfpathlineto{\pgfqpoint{0.694372in}{5.723612in}}%
\pgfpathlineto{\pgfqpoint{0.695445in}{5.725340in}}%
\pgfpathlineto{\pgfqpoint{0.696519in}{5.723036in}}%
\pgfpathlineto{\pgfqpoint{0.700815in}{5.725175in}}%
\pgfpathlineto{\pgfqpoint{0.702962in}{5.730481in}}%
\pgfpathlineto{\pgfqpoint{0.704036in}{5.729947in}}%
\pgfpathlineto{\pgfqpoint{0.707258in}{5.729823in}}%
\pgfpathlineto{\pgfqpoint{0.709405in}{5.732744in}}%
\pgfpathlineto{\pgfqpoint{0.710479in}{5.736486in}}%
\pgfpathlineto{\pgfqpoint{0.711553in}{5.736075in}}%
\pgfpathlineto{\pgfqpoint{0.714775in}{5.735664in}}%
\pgfpathlineto{\pgfqpoint{0.715848in}{5.733525in}}%
\pgfpathlineto{\pgfqpoint{0.716922in}{5.735705in}}%
\pgfpathlineto{\pgfqpoint{0.717996in}{5.735952in}}%
\pgfpathlineto{\pgfqpoint{0.719070in}{5.736980in}}%
\pgfpathlineto{\pgfqpoint{0.722291in}{5.736404in}}%
\pgfpathlineto{\pgfqpoint{0.723365in}{5.737515in}}%
\pgfpathlineto{\pgfqpoint{0.725513in}{5.737967in}}%
\pgfpathlineto{\pgfqpoint{0.726587in}{5.735006in}}%
\pgfpathlineto{\pgfqpoint{0.729808in}{5.737967in}}%
\pgfpathlineto{\pgfqpoint{0.730882in}{5.737844in}}%
\pgfpathlineto{\pgfqpoint{0.731956in}{5.736528in}}%
\pgfpathlineto{\pgfqpoint{0.733030in}{5.738708in}}%
\pgfpathlineto{\pgfqpoint{0.734104in}{5.738379in}}%
\pgfpathlineto{\pgfqpoint{0.739473in}{5.739078in}}%
\pgfpathlineto{\pgfqpoint{0.741621in}{5.740559in}}%
\pgfpathlineto{\pgfqpoint{0.744842in}{5.740106in}}%
\pgfpathlineto{\pgfqpoint{0.746990in}{5.738543in}}%
\pgfpathlineto{\pgfqpoint{0.752359in}{5.736692in}}%
\pgfpathlineto{\pgfqpoint{0.753433in}{5.729535in}}%
\pgfpathlineto{\pgfqpoint{0.754507in}{5.731304in}}%
\pgfpathlineto{\pgfqpoint{0.755580in}{5.735499in}}%
\pgfpathlineto{\pgfqpoint{0.756654in}{5.735828in}}%
\pgfpathlineto{\pgfqpoint{0.759876in}{5.738337in}}%
\pgfpathlineto{\pgfqpoint{0.760950in}{5.742451in}}%
\pgfpathlineto{\pgfqpoint{0.762023in}{5.742821in}}%
\pgfpathlineto{\pgfqpoint{0.763097in}{5.746687in}}%
\pgfpathlineto{\pgfqpoint{0.764171in}{5.745165in}}%
\pgfpathlineto{\pgfqpoint{0.767393in}{5.745782in}}%
\pgfpathlineto{\pgfqpoint{0.770614in}{5.741834in}}%
\pgfpathlineto{\pgfqpoint{0.771688in}{5.741463in}}%
\pgfpathlineto{\pgfqpoint{0.775983in}{5.743767in}}%
\pgfpathlineto{\pgfqpoint{0.777057in}{5.741422in}}%
\pgfpathlineto{\pgfqpoint{0.779205in}{5.743972in}}%
\pgfpathlineto{\pgfqpoint{0.782426in}{5.744055in}}%
\pgfpathlineto{\pgfqpoint{0.783500in}{5.742574in}}%
\pgfpathlineto{\pgfqpoint{0.785648in}{5.737391in}}%
\pgfpathlineto{\pgfqpoint{0.789943in}{5.734224in}}%
\pgfpathlineto{\pgfqpoint{0.791017in}{5.728301in}}%
\pgfpathlineto{\pgfqpoint{0.792091in}{5.730893in}}%
\pgfpathlineto{\pgfqpoint{0.793165in}{5.736034in}}%
\pgfpathlineto{\pgfqpoint{0.794239in}{5.732044in}}%
\pgfpathlineto{\pgfqpoint{0.797460in}{5.734471in}}%
\pgfpathlineto{\pgfqpoint{0.798534in}{5.738008in}}%
\pgfpathlineto{\pgfqpoint{0.800682in}{5.735828in}}%
\pgfpathlineto{\pgfqpoint{0.801755in}{5.738132in}}%
\pgfpathlineto{\pgfqpoint{0.804977in}{5.736939in}}%
\pgfpathlineto{\pgfqpoint{0.806051in}{5.741546in}}%
\pgfpathlineto{\pgfqpoint{0.809272in}{5.744507in}}%
\pgfpathlineto{\pgfqpoint{0.812494in}{5.744507in}}%
\pgfpathlineto{\pgfqpoint{0.813568in}{5.745330in}}%
\pgfpathlineto{\pgfqpoint{0.815715in}{5.744589in}}%
\pgfpathlineto{\pgfqpoint{0.816789in}{5.742163in}}%
\pgfpathlineto{\pgfqpoint{0.820011in}{5.739942in}}%
\pgfpathlineto{\pgfqpoint{0.822158in}{5.737391in}}%
\pgfpathlineto{\pgfqpoint{0.823232in}{5.736939in}}%
\pgfpathlineto{\pgfqpoint{0.824306in}{5.735582in}}%
\pgfpathlineto{\pgfqpoint{0.827528in}{5.732538in}}%
\pgfpathlineto{\pgfqpoint{0.828601in}{5.732373in}}%
\pgfpathlineto{\pgfqpoint{0.829675in}{5.732908in}}%
\pgfpathlineto{\pgfqpoint{0.831823in}{5.726615in}}%
\pgfpathlineto{\pgfqpoint{0.835044in}{5.729864in}}%
\pgfpathlineto{\pgfqpoint{0.836118in}{5.728713in}}%
\pgfpathlineto{\pgfqpoint{0.837192in}{5.731016in}}%
\pgfpathlineto{\pgfqpoint{0.838266in}{5.731633in}}%
\pgfpathlineto{\pgfqpoint{0.839340in}{5.730934in}}%
\pgfpathlineto{\pgfqpoint{0.843635in}{5.734183in}}%
\pgfpathlineto{\pgfqpoint{0.844709in}{5.729823in}}%
\pgfpathlineto{\pgfqpoint{0.845783in}{5.729700in}}%
\pgfpathlineto{\pgfqpoint{0.846857in}{5.724353in}}%
\pgfpathlineto{\pgfqpoint{0.851152in}{5.723201in}}%
\pgfpathlineto{\pgfqpoint{0.852226in}{5.730440in}}%
\pgfpathlineto{\pgfqpoint{0.854374in}{5.735129in}}%
\pgfpathlineto{\pgfqpoint{0.857595in}{5.732332in}}%
\pgfpathlineto{\pgfqpoint{0.858669in}{5.737597in}}%
\pgfpathlineto{\pgfqpoint{0.859743in}{5.735582in}}%
\pgfpathlineto{\pgfqpoint{0.861890in}{5.740024in}}%
\pgfpathlineto{\pgfqpoint{0.865112in}{5.739612in}}%
\pgfpathlineto{\pgfqpoint{0.866186in}{5.741340in}}%
\pgfpathlineto{\pgfqpoint{0.867260in}{5.740394in}}%
\pgfpathlineto{\pgfqpoint{0.868333in}{5.737597in}}%
\pgfpathlineto{\pgfqpoint{0.869407in}{5.737967in}}%
\pgfpathlineto{\pgfqpoint{0.872629in}{5.734594in}}%
\pgfpathlineto{\pgfqpoint{0.873703in}{5.735746in}}%
\pgfpathlineto{\pgfqpoint{0.874776in}{5.739078in}}%
\pgfpathlineto{\pgfqpoint{0.875850in}{5.739078in}}%
\pgfpathlineto{\pgfqpoint{0.876924in}{5.747427in}}%
\pgfpathlineto{\pgfqpoint{0.880146in}{5.746317in}}%
\pgfpathlineto{\pgfqpoint{0.881220in}{5.747757in}}%
\pgfpathlineto{\pgfqpoint{0.883367in}{5.747304in}}%
\pgfpathlineto{\pgfqpoint{0.884441in}{5.745330in}}%
\pgfpathlineto{\pgfqpoint{0.887663in}{5.745248in}}%
\pgfpathlineto{\pgfqpoint{0.889810in}{5.741422in}}%
\pgfpathlineto{\pgfqpoint{0.890884in}{5.736528in}}%
\pgfpathlineto{\pgfqpoint{0.891958in}{5.740559in}}%
\pgfpathlineto{\pgfqpoint{0.895179in}{5.742286in}}%
\pgfpathlineto{\pgfqpoint{0.896253in}{5.745289in}}%
\pgfpathlineto{\pgfqpoint{0.897327in}{5.751746in}}%
\pgfpathlineto{\pgfqpoint{0.898401in}{5.751623in}}%
\pgfpathlineto{\pgfqpoint{0.899475in}{5.748744in}}%
\pgfpathlineto{\pgfqpoint{0.902696in}{5.746605in}}%
\pgfpathlineto{\pgfqpoint{0.903770in}{5.742780in}}%
\pgfpathlineto{\pgfqpoint{0.904844in}{5.744507in}}%
\pgfpathlineto{\pgfqpoint{0.906992in}{5.754626in}}%
\pgfpathlineto{\pgfqpoint{0.911287in}{5.752980in}}%
\pgfpathlineto{\pgfqpoint{0.912361in}{5.752898in}}%
\pgfpathlineto{\pgfqpoint{0.913435in}{5.749073in}}%
\pgfpathlineto{\pgfqpoint{0.914509in}{5.754584in}}%
\pgfpathlineto{\pgfqpoint{0.917730in}{5.753597in}}%
\pgfpathlineto{\pgfqpoint{0.918804in}{5.754584in}}%
\pgfpathlineto{\pgfqpoint{0.920952in}{5.754214in}}%
\pgfpathlineto{\pgfqpoint{0.922025in}{5.756600in}}%
\pgfpathlineto{\pgfqpoint{0.927395in}{5.757464in}}%
\pgfpathlineto{\pgfqpoint{0.928468in}{5.761577in}}%
\pgfpathlineto{\pgfqpoint{0.929542in}{5.763263in}}%
\pgfpathlineto{\pgfqpoint{0.932764in}{5.762070in}}%
\pgfpathlineto{\pgfqpoint{0.933838in}{5.759767in}}%
\pgfpathlineto{\pgfqpoint{0.934911in}{5.759973in}}%
\pgfpathlineto{\pgfqpoint{0.935985in}{5.757546in}}%
\pgfpathlineto{\pgfqpoint{0.937059in}{5.760466in}}%
\pgfpathlineto{\pgfqpoint{0.940281in}{5.759644in}}%
\pgfpathlineto{\pgfqpoint{0.941354in}{5.758656in}}%
\pgfpathlineto{\pgfqpoint{0.942428in}{5.759109in}}%
\pgfpathlineto{\pgfqpoint{0.943502in}{5.756806in}}%
\pgfpathlineto{\pgfqpoint{0.944576in}{5.759685in}}%
\pgfpathlineto{\pgfqpoint{0.948871in}{5.756518in}}%
\pgfpathlineto{\pgfqpoint{0.949945in}{5.756764in}}%
\pgfpathlineto{\pgfqpoint{0.951019in}{5.762029in}}%
\pgfpathlineto{\pgfqpoint{0.952093in}{5.760425in}}%
\pgfpathlineto{\pgfqpoint{0.955314in}{5.753063in}}%
\pgfpathlineto{\pgfqpoint{0.956388in}{5.754790in}}%
\pgfpathlineto{\pgfqpoint{0.957462in}{5.753515in}}%
\pgfpathlineto{\pgfqpoint{0.959610in}{5.764415in}}%
\pgfpathlineto{\pgfqpoint{0.962831in}{5.763757in}}%
\pgfpathlineto{\pgfqpoint{0.963905in}{5.762523in}}%
\pgfpathlineto{\pgfqpoint{0.964979in}{5.763222in}}%
\pgfpathlineto{\pgfqpoint{0.966053in}{5.763058in}}%
\pgfpathlineto{\pgfqpoint{0.967127in}{5.761782in}}%
\pgfpathlineto{\pgfqpoint{0.970348in}{5.763592in}}%
\pgfpathlineto{\pgfqpoint{0.971422in}{5.760466in}}%
\pgfpathlineto{\pgfqpoint{0.972496in}{5.759644in}}%
\pgfpathlineto{\pgfqpoint{0.973570in}{5.760425in}}%
\pgfpathlineto{\pgfqpoint{0.974643in}{5.759068in}}%
\pgfpathlineto{\pgfqpoint{0.980013in}{5.763757in}}%
\pgfpathlineto{\pgfqpoint{0.982160in}{5.767788in}}%
\pgfpathlineto{\pgfqpoint{0.985382in}{5.769227in}}%
\pgfpathlineto{\pgfqpoint{0.986456in}{5.764291in}}%
\pgfpathlineto{\pgfqpoint{0.988603in}{5.760466in}}%
\pgfpathlineto{\pgfqpoint{0.989677in}{5.760219in}}%
\pgfpathlineto{\pgfqpoint{0.992899in}{5.760343in}}%
\pgfpathlineto{\pgfqpoint{0.993973in}{5.765279in}}%
\pgfpathlineto{\pgfqpoint{0.995046in}{5.767294in}}%
\pgfpathlineto{\pgfqpoint{0.996120in}{5.767047in}}%
\pgfpathlineto{\pgfqpoint{0.997194in}{5.760836in}}%
\pgfpathlineto{\pgfqpoint{1.000416in}{5.759438in}}%
\pgfpathlineto{\pgfqpoint{1.001489in}{5.746399in}}%
\pgfpathlineto{\pgfqpoint{1.003637in}{5.743232in}}%
\pgfpathlineto{\pgfqpoint{1.004711in}{5.743972in}}%
\pgfpathlineto{\pgfqpoint{1.010080in}{5.742492in}}%
\pgfpathlineto{\pgfqpoint{1.011154in}{5.748168in}}%
\pgfpathlineto{\pgfqpoint{1.012228in}{5.747222in}}%
\pgfpathlineto{\pgfqpoint{1.015449in}{5.749361in}}%
\pgfpathlineto{\pgfqpoint{1.016523in}{5.753721in}}%
\pgfpathlineto{\pgfqpoint{1.018671in}{5.745741in}}%
\pgfpathlineto{\pgfqpoint{1.019745in}{5.746646in}}%
\pgfpathlineto{\pgfqpoint{1.022966in}{5.747921in}}%
\pgfpathlineto{\pgfqpoint{1.024040in}{5.747592in}}%
\pgfpathlineto{\pgfqpoint{1.025114in}{5.741505in}}%
\pgfpathlineto{\pgfqpoint{1.027262in}{5.745577in}}%
\pgfpathlineto{\pgfqpoint{1.030483in}{5.749278in}}%
\pgfpathlineto{\pgfqpoint{1.032631in}{5.749114in}}%
\pgfpathlineto{\pgfqpoint{1.034778in}{5.753762in}}%
\pgfpathlineto{\pgfqpoint{1.038000in}{5.753104in}}%
\pgfpathlineto{\pgfqpoint{1.039074in}{5.753885in}}%
\pgfpathlineto{\pgfqpoint{1.040148in}{5.756024in}}%
\pgfpathlineto{\pgfqpoint{1.041221in}{5.755037in}}%
\pgfpathlineto{\pgfqpoint{1.042295in}{5.756106in}}%
\pgfpathlineto{\pgfqpoint{1.046591in}{5.753227in}}%
\pgfpathlineto{\pgfqpoint{1.047664in}{5.755448in}}%
\pgfpathlineto{\pgfqpoint{1.048738in}{5.756106in}}%
\pgfpathlineto{\pgfqpoint{1.049812in}{5.758039in}}%
\pgfpathlineto{\pgfqpoint{1.053034in}{5.759273in}}%
\pgfpathlineto{\pgfqpoint{1.054108in}{5.765525in}}%
\pgfpathlineto{\pgfqpoint{1.057329in}{5.760672in}}%
\pgfpathlineto{\pgfqpoint{1.060551in}{5.763304in}}%
\pgfpathlineto{\pgfqpoint{1.061624in}{5.766142in}}%
\pgfpathlineto{\pgfqpoint{1.062698in}{5.763263in}}%
\pgfpathlineto{\pgfqpoint{1.063772in}{5.767088in}}%
\pgfpathlineto{\pgfqpoint{1.064846in}{5.763510in}}%
\pgfpathlineto{\pgfqpoint{1.068067in}{5.763839in}}%
\pgfpathlineto{\pgfqpoint{1.070215in}{5.763428in}}%
\pgfpathlineto{\pgfqpoint{1.071289in}{5.761947in}}%
\pgfpathlineto{\pgfqpoint{1.072363in}{5.758944in}}%
\pgfpathlineto{\pgfqpoint{1.075584in}{5.762646in}}%
\pgfpathlineto{\pgfqpoint{1.077732in}{5.769351in}}%
\pgfpathlineto{\pgfqpoint{1.078806in}{5.768980in}}%
\pgfpathlineto{\pgfqpoint{1.079880in}{5.771366in}}%
\pgfpathlineto{\pgfqpoint{1.084175in}{5.771819in}}%
\pgfpathlineto{\pgfqpoint{1.086323in}{5.776631in}}%
\pgfpathlineto{\pgfqpoint{1.087397in}{5.774533in}}%
\pgfpathlineto{\pgfqpoint{1.092766in}{5.779099in}}%
\pgfpathlineto{\pgfqpoint{1.094913in}{5.783047in}}%
\pgfpathlineto{\pgfqpoint{1.101356in}{5.786256in}}%
\pgfpathlineto{\pgfqpoint{1.102430in}{5.789464in}}%
\pgfpathlineto{\pgfqpoint{1.105652in}{5.789670in}}%
\pgfpathlineto{\pgfqpoint{1.106726in}{5.793659in}}%
\pgfpathlineto{\pgfqpoint{1.107799in}{5.790163in}}%
\pgfpathlineto{\pgfqpoint{1.108873in}{5.789300in}}%
\pgfpathlineto{\pgfqpoint{1.109947in}{5.792796in}}%
\pgfpathlineto{\pgfqpoint{1.113169in}{5.790081in}}%
\pgfpathlineto{\pgfqpoint{1.114242in}{5.792549in}}%
\pgfpathlineto{\pgfqpoint{1.115316in}{5.796703in}}%
\pgfpathlineto{\pgfqpoint{1.116390in}{5.795099in}}%
\pgfpathlineto{\pgfqpoint{1.117464in}{5.796621in}}%
\pgfpathlineto{\pgfqpoint{1.120686in}{5.796498in}}%
\pgfpathlineto{\pgfqpoint{1.121759in}{5.799377in}}%
\pgfpathlineto{\pgfqpoint{1.123907in}{5.799212in}}%
\pgfpathlineto{\pgfqpoint{1.124981in}{5.800816in}}%
\pgfpathlineto{\pgfqpoint{1.129276in}{5.804107in}}%
\pgfpathlineto{\pgfqpoint{1.130350in}{5.800528in}}%
\pgfpathlineto{\pgfqpoint{1.131424in}{5.799007in}}%
\pgfpathlineto{\pgfqpoint{1.132498in}{5.801886in}}%
\pgfpathlineto{\pgfqpoint{1.135719in}{5.795634in}}%
\pgfpathlineto{\pgfqpoint{1.136793in}{5.797608in}}%
\pgfpathlineto{\pgfqpoint{1.137867in}{5.801968in}}%
\pgfpathlineto{\pgfqpoint{1.138941in}{5.803490in}}%
\pgfpathlineto{\pgfqpoint{1.140015in}{5.802667in}}%
\pgfpathlineto{\pgfqpoint{1.143236in}{5.800981in}}%
\pgfpathlineto{\pgfqpoint{1.144310in}{5.805053in}}%
\pgfpathlineto{\pgfqpoint{1.145384in}{5.805793in}}%
\pgfpathlineto{\pgfqpoint{1.146458in}{5.805341in}}%
\pgfpathlineto{\pgfqpoint{1.147531in}{5.809454in}}%
\pgfpathlineto{\pgfqpoint{1.150753in}{5.809783in}}%
\pgfpathlineto{\pgfqpoint{1.151827in}{5.807397in}}%
\pgfpathlineto{\pgfqpoint{1.152901in}{5.807274in}}%
\pgfpathlineto{\pgfqpoint{1.153975in}{5.810523in}}%
\pgfpathlineto{\pgfqpoint{1.155048in}{5.811840in}}%
\pgfpathlineto{\pgfqpoint{1.159344in}{5.807603in}}%
\pgfpathlineto{\pgfqpoint{1.160418in}{5.809248in}}%
\pgfpathlineto{\pgfqpoint{1.161491in}{5.806739in}}%
\pgfpathlineto{\pgfqpoint{1.162565in}{5.811922in}}%
\pgfpathlineto{\pgfqpoint{1.165787in}{5.807562in}}%
\pgfpathlineto{\pgfqpoint{1.166861in}{5.810688in}}%
\pgfpathlineto{\pgfqpoint{1.167934in}{5.807973in}}%
\pgfpathlineto{\pgfqpoint{1.169008in}{5.811511in}}%
\pgfpathlineto{\pgfqpoint{1.173304in}{5.809207in}}%
\pgfpathlineto{\pgfqpoint{1.174377in}{5.812251in}}%
\pgfpathlineto{\pgfqpoint{1.175451in}{5.809331in}}%
\pgfpathlineto{\pgfqpoint{1.176525in}{5.809824in}}%
\pgfpathlineto{\pgfqpoint{1.180820in}{5.809577in}}%
\pgfpathlineto{\pgfqpoint{1.181894in}{5.810277in}}%
\pgfpathlineto{\pgfqpoint{1.182968in}{5.816323in}}%
\pgfpathlineto{\pgfqpoint{1.184042in}{5.818297in}}%
\pgfpathlineto{\pgfqpoint{1.188337in}{5.809619in}}%
\pgfpathlineto{\pgfqpoint{1.189411in}{5.810976in}}%
\pgfpathlineto{\pgfqpoint{1.191559in}{5.806904in}}%
\pgfpathlineto{\pgfqpoint{1.192633in}{5.809454in}}%
\pgfpathlineto{\pgfqpoint{1.195854in}{5.809742in}}%
\pgfpathlineto{\pgfqpoint{1.196928in}{5.815295in}}%
\pgfpathlineto{\pgfqpoint{1.198002in}{5.816940in}}%
\pgfpathlineto{\pgfqpoint{1.199076in}{5.806534in}}%
\pgfpathlineto{\pgfqpoint{1.200150in}{5.802750in}}%
\pgfpathlineto{\pgfqpoint{1.203371in}{5.802873in}}%
\pgfpathlineto{\pgfqpoint{1.204445in}{5.805958in}}%
\pgfpathlineto{\pgfqpoint{1.205519in}{5.805382in}}%
\pgfpathlineto{\pgfqpoint{1.207666in}{5.816817in}}%
\pgfpathlineto{\pgfqpoint{1.210888in}{5.816858in}}%
\pgfpathlineto{\pgfqpoint{1.213036in}{5.818092in}}%
\pgfpathlineto{\pgfqpoint{1.214109in}{5.824097in}}%
\pgfpathlineto{\pgfqpoint{1.215183in}{5.826030in}}%
\pgfpathlineto{\pgfqpoint{1.219479in}{5.826441in}}%
\pgfpathlineto{\pgfqpoint{1.220552in}{5.829609in}}%
\pgfpathlineto{\pgfqpoint{1.221626in}{5.828087in}}%
\pgfpathlineto{\pgfqpoint{1.222700in}{5.829197in}}%
\pgfpathlineto{\pgfqpoint{1.225922in}{5.830349in}}%
\pgfpathlineto{\pgfqpoint{1.226996in}{5.831377in}}%
\pgfpathlineto{\pgfqpoint{1.229143in}{5.828004in}}%
\pgfpathlineto{\pgfqpoint{1.230217in}{5.827511in}}%
\pgfpathlineto{\pgfqpoint{1.234512in}{5.832118in}}%
\pgfpathlineto{\pgfqpoint{1.235586in}{5.830513in}}%
\pgfpathlineto{\pgfqpoint{1.236660in}{5.831542in}}%
\pgfpathlineto{\pgfqpoint{1.237734in}{5.827511in}}%
\pgfpathlineto{\pgfqpoint{1.240955in}{5.828704in}}%
\pgfpathlineto{\pgfqpoint{1.242029in}{5.826647in}}%
\pgfpathlineto{\pgfqpoint{1.243103in}{5.821629in}}%
\pgfpathlineto{\pgfqpoint{1.244177in}{5.821917in}}%
\pgfpathlineto{\pgfqpoint{1.245251in}{5.830431in}}%
\pgfpathlineto{\pgfqpoint{1.248472in}{5.829403in}}%
\pgfpathlineto{\pgfqpoint{1.249546in}{5.827346in}}%
\pgfpathlineto{\pgfqpoint{1.250620in}{5.823151in}}%
\pgfpathlineto{\pgfqpoint{1.251694in}{5.830760in}}%
\pgfpathlineto{\pgfqpoint{1.252768in}{5.830184in}}%
\pgfpathlineto{\pgfqpoint{1.255989in}{5.833269in}}%
\pgfpathlineto{\pgfqpoint{1.257063in}{5.836971in}}%
\pgfpathlineto{\pgfqpoint{1.258137in}{5.832076in}}%
\pgfpathlineto{\pgfqpoint{1.259211in}{5.822287in}}%
\pgfpathlineto{\pgfqpoint{1.260285in}{5.825125in}}%
\pgfpathlineto{\pgfqpoint{1.263506in}{5.817804in}}%
\pgfpathlineto{\pgfqpoint{1.264580in}{5.820395in}}%
\pgfpathlineto{\pgfqpoint{1.265654in}{5.825454in}}%
\pgfpathlineto{\pgfqpoint{1.266728in}{5.827387in}}%
\pgfpathlineto{\pgfqpoint{1.267801in}{5.824303in}}%
\pgfpathlineto{\pgfqpoint{1.271023in}{5.824138in}}%
\pgfpathlineto{\pgfqpoint{1.272097in}{5.822123in}}%
\pgfpathlineto{\pgfqpoint{1.273171in}{5.824632in}}%
\pgfpathlineto{\pgfqpoint{1.275318in}{5.831953in}}%
\pgfpathlineto{\pgfqpoint{1.278540in}{5.834010in}}%
\pgfpathlineto{\pgfqpoint{1.279614in}{5.838287in}}%
\pgfpathlineto{\pgfqpoint{1.280687in}{5.838575in}}%
\pgfpathlineto{\pgfqpoint{1.282835in}{5.844580in}}%
\pgfpathlineto{\pgfqpoint{1.286057in}{5.843552in}}%
\pgfpathlineto{\pgfqpoint{1.287130in}{5.841742in}}%
\pgfpathlineto{\pgfqpoint{1.288204in}{5.842812in}}%
\pgfpathlineto{\pgfqpoint{1.290352in}{5.848241in}}%
\pgfpathlineto{\pgfqpoint{1.293574in}{5.848570in}}%
\pgfpathlineto{\pgfqpoint{1.294647in}{5.850215in}}%
\pgfpathlineto{\pgfqpoint{1.295721in}{5.848735in}}%
\pgfpathlineto{\pgfqpoint{1.297869in}{5.850750in}}%
\pgfpathlineto{\pgfqpoint{1.301090in}{5.849722in}}%
\pgfpathlineto{\pgfqpoint{1.302164in}{5.850503in}}%
\pgfpathlineto{\pgfqpoint{1.303238in}{5.852601in}}%
\pgfpathlineto{\pgfqpoint{1.304312in}{5.856015in}}%
\pgfpathlineto{\pgfqpoint{1.309681in}{5.852848in}}%
\pgfpathlineto{\pgfqpoint{1.310755in}{5.854000in}}%
\pgfpathlineto{\pgfqpoint{1.311829in}{5.856961in}}%
\pgfpathlineto{\pgfqpoint{1.312903in}{5.855686in}}%
\pgfpathlineto{\pgfqpoint{1.316124in}{5.855892in}}%
\pgfpathlineto{\pgfqpoint{1.317198in}{5.856920in}}%
\pgfpathlineto{\pgfqpoint{1.318272in}{5.853259in}}%
\pgfpathlineto{\pgfqpoint{1.319346in}{5.847048in}}%
\pgfpathlineto{\pgfqpoint{1.320419in}{5.847213in}}%
\pgfpathlineto{\pgfqpoint{1.324715in}{5.845526in}}%
\pgfpathlineto{\pgfqpoint{1.325789in}{5.840755in}}%
\pgfpathlineto{\pgfqpoint{1.326863in}{5.845156in}}%
\pgfpathlineto{\pgfqpoint{1.327936in}{5.844169in}}%
\pgfpathlineto{\pgfqpoint{1.331158in}{5.843922in}}%
\pgfpathlineto{\pgfqpoint{1.332232in}{5.838287in}}%
\pgfpathlineto{\pgfqpoint{1.335453in}{5.841290in}}%
\pgfpathlineto{\pgfqpoint{1.339749in}{5.840015in}}%
\pgfpathlineto{\pgfqpoint{1.340822in}{5.844704in}}%
\pgfpathlineto{\pgfqpoint{1.342970in}{5.846431in}}%
\pgfpathlineto{\pgfqpoint{1.347265in}{5.855439in}}%
\pgfpathlineto{\pgfqpoint{1.348339in}{5.859347in}}%
\pgfpathlineto{\pgfqpoint{1.349413in}{5.857578in}}%
\pgfpathlineto{\pgfqpoint{1.350487in}{5.858976in}}%
\pgfpathlineto{\pgfqpoint{1.353708in}{5.861198in}}%
\pgfpathlineto{\pgfqpoint{1.354782in}{5.863666in}}%
\pgfpathlineto{\pgfqpoint{1.355856in}{5.868437in}}%
\pgfpathlineto{\pgfqpoint{1.356930in}{5.869424in}}%
\pgfpathlineto{\pgfqpoint{1.358004in}{5.863953in}}%
\pgfpathlineto{\pgfqpoint{1.361225in}{5.867820in}}%
\pgfpathlineto{\pgfqpoint{1.362299in}{5.866709in}}%
\pgfpathlineto{\pgfqpoint{1.363373in}{5.864612in}}%
\pgfpathlineto{\pgfqpoint{1.364447in}{5.866216in}}%
\pgfpathlineto{\pgfqpoint{1.365521in}{5.864694in}}%
\pgfpathlineto{\pgfqpoint{1.368742in}{5.861815in}}%
\pgfpathlineto{\pgfqpoint{1.369816in}{5.862555in}}%
\pgfpathlineto{\pgfqpoint{1.371964in}{5.858812in}}%
\pgfpathlineto{\pgfqpoint{1.373038in}{5.861815in}}%
\pgfpathlineto{\pgfqpoint{1.376259in}{5.859306in}}%
\pgfpathlineto{\pgfqpoint{1.377333in}{5.853917in}}%
\pgfpathlineto{\pgfqpoint{1.378407in}{5.855357in}}%
\pgfpathlineto{\pgfqpoint{1.380554in}{5.866462in}}%
\pgfpathlineto{\pgfqpoint{1.383776in}{5.868848in}}%
\pgfpathlineto{\pgfqpoint{1.384850in}{5.863295in}}%
\pgfpathlineto{\pgfqpoint{1.388071in}{5.873907in}}%
\pgfpathlineto{\pgfqpoint{1.391293in}{5.875347in}}%
\pgfpathlineto{\pgfqpoint{1.392367in}{5.877280in}}%
\pgfpathlineto{\pgfqpoint{1.393440in}{5.875182in}}%
\pgfpathlineto{\pgfqpoint{1.394514in}{5.876211in}}%
\pgfpathlineto{\pgfqpoint{1.395588in}{5.879460in}}%
\pgfpathlineto{\pgfqpoint{1.398810in}{5.881229in}}%
\pgfpathlineto{\pgfqpoint{1.399884in}{5.882504in}}%
\pgfpathlineto{\pgfqpoint{1.400957in}{5.880776in}}%
\pgfpathlineto{\pgfqpoint{1.402031in}{5.884519in}}%
\pgfpathlineto{\pgfqpoint{1.403105in}{5.884684in}}%
\pgfpathlineto{\pgfqpoint{1.406327in}{5.886165in}}%
\pgfpathlineto{\pgfqpoint{1.407400in}{5.885465in}}%
\pgfpathlineto{\pgfqpoint{1.408474in}{5.889003in}}%
\pgfpathlineto{\pgfqpoint{1.409548in}{5.886494in}}%
\pgfpathlineto{\pgfqpoint{1.410622in}{5.892046in}}%
\pgfpathlineto{\pgfqpoint{1.413843in}{5.891923in}}%
\pgfpathlineto{\pgfqpoint{1.415991in}{5.894144in}}%
\pgfpathlineto{\pgfqpoint{1.417065in}{5.898381in}}%
\pgfpathlineto{\pgfqpoint{1.422434in}{5.899286in}}%
\pgfpathlineto{\pgfqpoint{1.423508in}{5.898463in}}%
\pgfpathlineto{\pgfqpoint{1.425656in}{5.905003in}}%
\pgfpathlineto{\pgfqpoint{1.428877in}{5.905743in}}%
\pgfpathlineto{\pgfqpoint{1.431025in}{5.913764in}}%
\pgfpathlineto{\pgfqpoint{1.433173in}{5.913805in}}%
\pgfpathlineto{\pgfqpoint{1.437468in}{5.889332in}}%
\pgfpathlineto{\pgfqpoint{1.438542in}{5.888838in}}%
\pgfpathlineto{\pgfqpoint{1.439616in}{5.890154in}}%
\pgfpathlineto{\pgfqpoint{1.440689in}{5.896447in}}%
\pgfpathlineto{\pgfqpoint{1.443911in}{5.896324in}}%
\pgfpathlineto{\pgfqpoint{1.446059in}{5.889949in}}%
\pgfpathlineto{\pgfqpoint{1.448206in}{5.888756in}}%
\pgfpathlineto{\pgfqpoint{1.451428in}{5.893075in}}%
\pgfpathlineto{\pgfqpoint{1.453575in}{5.921908in}}%
\pgfpathlineto{\pgfqpoint{1.455723in}{5.925157in}}%
\pgfpathlineto{\pgfqpoint{1.458945in}{5.925445in}}%
\pgfpathlineto{\pgfqpoint{1.460018in}{5.926103in}}%
\pgfpathlineto{\pgfqpoint{1.462166in}{5.930710in}}%
\pgfpathlineto{\pgfqpoint{1.463240in}{5.934453in}}%
\pgfpathlineto{\pgfqpoint{1.466462in}{5.934700in}}%
\pgfpathlineto{\pgfqpoint{1.467535in}{5.937661in}}%
\pgfpathlineto{\pgfqpoint{1.469683in}{5.930134in}}%
\pgfpathlineto{\pgfqpoint{1.470757in}{5.931286in}}%
\pgfpathlineto{\pgfqpoint{1.473978in}{5.928366in}}%
\pgfpathlineto{\pgfqpoint{1.475052in}{5.928448in}}%
\pgfpathlineto{\pgfqpoint{1.476126in}{5.924828in}}%
\pgfpathlineto{\pgfqpoint{1.478274in}{5.923265in}}%
\pgfpathlineto{\pgfqpoint{1.481495in}{5.917959in}}%
\pgfpathlineto{\pgfqpoint{1.482569in}{5.927584in}}%
\pgfpathlineto{\pgfqpoint{1.483643in}{5.931245in}}%
\pgfpathlineto{\pgfqpoint{1.484717in}{5.930258in}}%
\pgfpathlineto{\pgfqpoint{1.485791in}{5.927255in}}%
\pgfpathlineto{\pgfqpoint{1.490086in}{5.926103in}}%
\pgfpathlineto{\pgfqpoint{1.491160in}{5.924294in}}%
\pgfpathlineto{\pgfqpoint{1.492234in}{5.918042in}}%
\pgfpathlineto{\pgfqpoint{1.493307in}{5.902165in}}%
\pgfpathlineto{\pgfqpoint{1.496529in}{5.897681in}}%
\pgfpathlineto{\pgfqpoint{1.498677in}{5.902247in}}%
\pgfpathlineto{\pgfqpoint{1.499751in}{5.894473in}}%
\pgfpathlineto{\pgfqpoint{1.500824in}{5.894967in}}%
\pgfpathlineto{\pgfqpoint{1.504046in}{5.879789in}}%
\pgfpathlineto{\pgfqpoint{1.505120in}{5.889784in}}%
\pgfpathlineto{\pgfqpoint{1.506194in}{5.892046in}}%
\pgfpathlineto{\pgfqpoint{1.508341in}{5.902535in}}%
\pgfpathlineto{\pgfqpoint{1.511563in}{5.900314in}}%
\pgfpathlineto{\pgfqpoint{1.512637in}{5.904838in}}%
\pgfpathlineto{\pgfqpoint{1.513710in}{5.905949in}}%
\pgfpathlineto{\pgfqpoint{1.514784in}{5.904879in}}%
\pgfpathlineto{\pgfqpoint{1.515858in}{5.911954in}}%
\pgfpathlineto{\pgfqpoint{1.520153in}{5.910802in}}%
\pgfpathlineto{\pgfqpoint{1.521227in}{5.906401in}}%
\pgfpathlineto{\pgfqpoint{1.522301in}{5.909939in}}%
\pgfpathlineto{\pgfqpoint{1.523375in}{5.909980in}}%
\pgfpathlineto{\pgfqpoint{1.526596in}{5.912242in}}%
\pgfpathlineto{\pgfqpoint{1.527670in}{5.914833in}}%
\pgfpathlineto{\pgfqpoint{1.528744in}{5.914586in}}%
\pgfpathlineto{\pgfqpoint{1.529818in}{5.919851in}}%
\pgfpathlineto{\pgfqpoint{1.530892in}{5.921250in}}%
\pgfpathlineto{\pgfqpoint{1.534113in}{5.912283in}}%
\pgfpathlineto{\pgfqpoint{1.535187in}{5.913928in}}%
\pgfpathlineto{\pgfqpoint{1.536261in}{5.918124in}}%
\pgfpathlineto{\pgfqpoint{1.537335in}{5.918946in}}%
\pgfpathlineto{\pgfqpoint{1.538409in}{5.919029in}}%
\pgfpathlineto{\pgfqpoint{1.541630in}{5.917096in}}%
\pgfpathlineto{\pgfqpoint{1.542704in}{5.913353in}}%
\pgfpathlineto{\pgfqpoint{1.543778in}{5.913558in}}%
\pgfpathlineto{\pgfqpoint{1.545926in}{5.903810in}}%
\pgfpathlineto{\pgfqpoint{1.550221in}{5.914175in}}%
\pgfpathlineto{\pgfqpoint{1.551295in}{5.908787in}}%
\pgfpathlineto{\pgfqpoint{1.553442in}{5.915532in}}%
\pgfpathlineto{\pgfqpoint{1.556664in}{5.913023in}}%
\pgfpathlineto{\pgfqpoint{1.557738in}{5.918864in}}%
\pgfpathlineto{\pgfqpoint{1.558812in}{5.915409in}}%
\pgfpathlineto{\pgfqpoint{1.559885in}{5.914504in}}%
\pgfpathlineto{\pgfqpoint{1.560959in}{5.919358in}}%
\pgfpathlineto{\pgfqpoint{1.564181in}{5.924581in}}%
\pgfpathlineto{\pgfqpoint{1.565255in}{5.927666in}}%
\pgfpathlineto{\pgfqpoint{1.566328in}{5.925774in}}%
\pgfpathlineto{\pgfqpoint{1.567402in}{5.926268in}}%
\pgfpathlineto{\pgfqpoint{1.568476in}{5.925281in}}%
\pgfpathlineto{\pgfqpoint{1.571698in}{5.920057in}}%
\pgfpathlineto{\pgfqpoint{1.572772in}{5.921661in}}%
\pgfpathlineto{\pgfqpoint{1.573845in}{5.925198in}}%
\pgfpathlineto{\pgfqpoint{1.575993in}{5.912900in}}%
\pgfpathlineto{\pgfqpoint{1.579215in}{5.915615in}}%
\pgfpathlineto{\pgfqpoint{1.580288in}{5.918988in}}%
\pgfpathlineto{\pgfqpoint{1.581362in}{5.928530in}}%
\pgfpathlineto{\pgfqpoint{1.582436in}{5.931944in}}%
\pgfpathlineto{\pgfqpoint{1.587805in}{5.936057in}}%
\pgfpathlineto{\pgfqpoint{1.591027in}{5.927749in}}%
\pgfpathlineto{\pgfqpoint{1.595322in}{5.931286in}}%
\pgfpathlineto{\pgfqpoint{1.597470in}{5.942926in}}%
\pgfpathlineto{\pgfqpoint{1.598544in}{5.940458in}}%
\pgfpathlineto{\pgfqpoint{1.601765in}{5.942186in}}%
\pgfpathlineto{\pgfqpoint{1.602839in}{5.937497in}}%
\pgfpathlineto{\pgfqpoint{1.603913in}{5.944078in}}%
\pgfpathlineto{\pgfqpoint{1.604987in}{5.942967in}}%
\pgfpathlineto{\pgfqpoint{1.609282in}{5.950083in}}%
\pgfpathlineto{\pgfqpoint{1.610356in}{5.948685in}}%
\pgfpathlineto{\pgfqpoint{1.612504in}{5.943502in}}%
\pgfpathlineto{\pgfqpoint{1.616799in}{5.945970in}}%
\pgfpathlineto{\pgfqpoint{1.617873in}{5.940828in}}%
\pgfpathlineto{\pgfqpoint{1.618947in}{5.945435in}}%
\pgfpathlineto{\pgfqpoint{1.620020in}{5.944201in}}%
\pgfpathlineto{\pgfqpoint{1.621094in}{5.947163in}}%
\pgfpathlineto{\pgfqpoint{1.626463in}{5.948191in}}%
\pgfpathlineto{\pgfqpoint{1.627537in}{5.951646in}}%
\pgfpathlineto{\pgfqpoint{1.628611in}{5.952222in}}%
\pgfpathlineto{\pgfqpoint{1.631833in}{5.951399in}}%
\pgfpathlineto{\pgfqpoint{1.632906in}{5.953456in}}%
\pgfpathlineto{\pgfqpoint{1.633980in}{5.951194in}}%
\pgfpathlineto{\pgfqpoint{1.636128in}{5.959708in}}%
\pgfpathlineto{\pgfqpoint{1.639350in}{5.962135in}}%
\pgfpathlineto{\pgfqpoint{1.641497in}{5.958885in}}%
\pgfpathlineto{\pgfqpoint{1.642571in}{5.954073in}}%
\pgfpathlineto{\pgfqpoint{1.643645in}{5.955142in}}%
\pgfpathlineto{\pgfqpoint{1.646866in}{5.954978in}}%
\pgfpathlineto{\pgfqpoint{1.647940in}{5.956294in}}%
\pgfpathlineto{\pgfqpoint{1.649014in}{5.958597in}}%
\pgfpathlineto{\pgfqpoint{1.650088in}{5.959173in}}%
\pgfpathlineto{\pgfqpoint{1.651162in}{5.961559in}}%
\pgfpathlineto{\pgfqpoint{1.654383in}{5.957734in}}%
\pgfpathlineto{\pgfqpoint{1.655457in}{5.954155in}}%
\pgfpathlineto{\pgfqpoint{1.656531in}{5.956171in}}%
\pgfpathlineto{\pgfqpoint{1.658679in}{5.956294in}}%
\pgfpathlineto{\pgfqpoint{1.661900in}{5.954690in}}%
\pgfpathlineto{\pgfqpoint{1.664048in}{5.961847in}}%
\pgfpathlineto{\pgfqpoint{1.665122in}{5.962464in}}%
\pgfpathlineto{\pgfqpoint{1.669417in}{5.960695in}}%
\pgfpathlineto{\pgfqpoint{1.670491in}{5.959420in}}%
\pgfpathlineto{\pgfqpoint{1.671565in}{5.959872in}}%
\pgfpathlineto{\pgfqpoint{1.672639in}{5.957034in}}%
\pgfpathlineto{\pgfqpoint{1.673712in}{5.958515in}}%
\pgfpathlineto{\pgfqpoint{1.678008in}{5.961230in}}%
\pgfpathlineto{\pgfqpoint{1.679082in}{5.965178in}}%
\pgfpathlineto{\pgfqpoint{1.680155in}{5.955759in}}%
\pgfpathlineto{\pgfqpoint{1.681229in}{5.960407in}}%
\pgfpathlineto{\pgfqpoint{1.684451in}{5.958515in}}%
\pgfpathlineto{\pgfqpoint{1.685525in}{5.961435in}}%
\pgfpathlineto{\pgfqpoint{1.686598in}{5.959872in}}%
\pgfpathlineto{\pgfqpoint{1.687672in}{5.961477in}}%
\pgfpathlineto{\pgfqpoint{1.688746in}{5.961435in}}%
\pgfpathlineto{\pgfqpoint{1.691968in}{5.962916in}}%
\pgfpathlineto{\pgfqpoint{1.693041in}{5.957487in}}%
\pgfpathlineto{\pgfqpoint{1.694115in}{5.956500in}}%
\pgfpathlineto{\pgfqpoint{1.695189in}{5.946258in}}%
\pgfpathlineto{\pgfqpoint{1.696263in}{5.943461in}}%
\pgfpathlineto{\pgfqpoint{1.699484in}{5.945764in}}%
\pgfpathlineto{\pgfqpoint{1.700558in}{5.942309in}}%
\pgfpathlineto{\pgfqpoint{1.702706in}{5.939965in}}%
\pgfpathlineto{\pgfqpoint{1.703780in}{5.946134in}}%
\pgfpathlineto{\pgfqpoint{1.707001in}{5.945147in}}%
\pgfpathlineto{\pgfqpoint{1.708075in}{5.946217in}}%
\pgfpathlineto{\pgfqpoint{1.710223in}{5.951482in}}%
\pgfpathlineto{\pgfqpoint{1.711297in}{5.949960in}}%
\pgfpathlineto{\pgfqpoint{1.714518in}{5.957857in}}%
\pgfpathlineto{\pgfqpoint{1.715592in}{5.958309in}}%
\pgfpathlineto{\pgfqpoint{1.716666in}{5.962669in}}%
\pgfpathlineto{\pgfqpoint{1.717740in}{5.962299in}}%
\pgfpathlineto{\pgfqpoint{1.718814in}{5.960983in}}%
\pgfpathlineto{\pgfqpoint{1.722035in}{5.963040in}}%
\pgfpathlineto{\pgfqpoint{1.723109in}{5.962669in}}%
\pgfpathlineto{\pgfqpoint{1.724183in}{5.960284in}}%
\pgfpathlineto{\pgfqpoint{1.730626in}{5.961271in}}%
\pgfpathlineto{\pgfqpoint{1.732773in}{5.959091in}}%
\pgfpathlineto{\pgfqpoint{1.733847in}{5.961230in}}%
\pgfpathlineto{\pgfqpoint{1.737069in}{5.963698in}}%
\pgfpathlineto{\pgfqpoint{1.738143in}{5.962176in}}%
\pgfpathlineto{\pgfqpoint{1.739216in}{5.962793in}}%
\pgfpathlineto{\pgfqpoint{1.741364in}{5.960284in}}%
\pgfpathlineto{\pgfqpoint{1.744586in}{5.962217in}}%
\pgfpathlineto{\pgfqpoint{1.746733in}{5.965425in}}%
\pgfpathlineto{\pgfqpoint{1.747807in}{5.970731in}}%
\pgfpathlineto{\pgfqpoint{1.748881in}{5.970196in}}%
\pgfpathlineto{\pgfqpoint{1.752103in}{5.966659in}}%
\pgfpathlineto{\pgfqpoint{1.753176in}{5.961847in}}%
\pgfpathlineto{\pgfqpoint{1.754250in}{5.963657in}}%
\pgfpathlineto{\pgfqpoint{1.755324in}{5.955019in}}%
\pgfpathlineto{\pgfqpoint{1.759619in}{5.953991in}}%
\pgfpathlineto{\pgfqpoint{1.760693in}{5.952140in}}%
\pgfpathlineto{\pgfqpoint{1.761767in}{5.943132in}}%
\pgfpathlineto{\pgfqpoint{1.762841in}{5.941281in}}%
\pgfpathlineto{\pgfqpoint{1.763915in}{5.946669in}}%
\pgfpathlineto{\pgfqpoint{1.767136in}{5.947286in}}%
\pgfpathlineto{\pgfqpoint{1.768210in}{5.937702in}}%
\pgfpathlineto{\pgfqpoint{1.769284in}{5.951152in}}%
\pgfpathlineto{\pgfqpoint{1.770358in}{5.941158in}}%
\pgfpathlineto{\pgfqpoint{1.771432in}{5.923841in}}%
\pgfpathlineto{\pgfqpoint{1.774653in}{5.920468in}}%
\pgfpathlineto{\pgfqpoint{1.775727in}{5.925075in}}%
\pgfpathlineto{\pgfqpoint{1.776801in}{5.925240in}}%
\pgfpathlineto{\pgfqpoint{1.777875in}{5.928242in}}%
\pgfpathlineto{\pgfqpoint{1.778949in}{5.936715in}}%
\pgfpathlineto{\pgfqpoint{1.782170in}{5.937415in}}%
\pgfpathlineto{\pgfqpoint{1.783244in}{5.949425in}}%
\pgfpathlineto{\pgfqpoint{1.784318in}{5.942268in}}%
\pgfpathlineto{\pgfqpoint{1.786465in}{5.977065in}}%
\pgfpathlineto{\pgfqpoint{1.789687in}{5.980562in}}%
\pgfpathlineto{\pgfqpoint{1.790761in}{5.985950in}}%
\pgfpathlineto{\pgfqpoint{1.791835in}{5.985785in}}%
\pgfpathlineto{\pgfqpoint{1.792908in}{5.989693in}}%
\pgfpathlineto{\pgfqpoint{1.793982in}{5.995739in}}%
\pgfpathlineto{\pgfqpoint{1.797204in}{5.994012in}}%
\pgfpathlineto{\pgfqpoint{1.798278in}{6.000346in}}%
\pgfpathlineto{\pgfqpoint{1.800425in}{6.003924in}}%
\pgfpathlineto{\pgfqpoint{1.801499in}{6.005734in}}%
\pgfpathlineto{\pgfqpoint{1.804721in}{6.009765in}}%
\pgfpathlineto{\pgfqpoint{1.805794in}{6.007750in}}%
\pgfpathlineto{\pgfqpoint{1.809016in}{6.014043in}}%
\pgfpathlineto{\pgfqpoint{1.812238in}{6.013467in}}%
\pgfpathlineto{\pgfqpoint{1.813311in}{6.018732in}}%
\pgfpathlineto{\pgfqpoint{1.814385in}{6.017045in}}%
\pgfpathlineto{\pgfqpoint{1.816533in}{6.021858in}}%
\pgfpathlineto{\pgfqpoint{1.819754in}{6.020829in}}%
\pgfpathlineto{\pgfqpoint{1.820828in}{6.014248in}}%
\pgfpathlineto{\pgfqpoint{1.821902in}{6.015153in}}%
\pgfpathlineto{\pgfqpoint{1.824050in}{6.021611in}}%
\pgfpathlineto{\pgfqpoint{1.827271in}{6.014619in}}%
\pgfpathlineto{\pgfqpoint{1.829419in}{6.029467in}}%
\pgfpathlineto{\pgfqpoint{1.831567in}{6.029508in}}%
\pgfpathlineto{\pgfqpoint{1.834788in}{6.024655in}}%
\pgfpathlineto{\pgfqpoint{1.835862in}{6.024285in}}%
\pgfpathlineto{\pgfqpoint{1.836936in}{6.014907in}}%
\pgfpathlineto{\pgfqpoint{1.838010in}{6.018197in}}%
\pgfpathlineto{\pgfqpoint{1.839083in}{6.010835in}}%
\pgfpathlineto{\pgfqpoint{1.842305in}{6.009847in}}%
\pgfpathlineto{\pgfqpoint{1.844453in}{6.023462in}}%
\pgfpathlineto{\pgfqpoint{1.845527in}{6.040490in}}%
\pgfpathlineto{\pgfqpoint{1.846600in}{6.041149in}}%
\pgfpathlineto{\pgfqpoint{1.849822in}{6.047647in}}%
\pgfpathlineto{\pgfqpoint{1.850896in}{6.046208in}}%
\pgfpathlineto{\pgfqpoint{1.851970in}{6.046496in}}%
\pgfpathlineto{\pgfqpoint{1.854117in}{6.043987in}}%
\pgfpathlineto{\pgfqpoint{1.857339in}{6.045591in}}%
\pgfpathlineto{\pgfqpoint{1.858413in}{6.042465in}}%
\pgfpathlineto{\pgfqpoint{1.859486in}{6.036953in}}%
\pgfpathlineto{\pgfqpoint{1.861634in}{6.036007in}}%
\pgfpathlineto{\pgfqpoint{1.865929in}{6.016387in}}%
\pgfpathlineto{\pgfqpoint{1.867003in}{6.020542in}}%
\pgfpathlineto{\pgfqpoint{1.868077in}{6.034444in}}%
\pgfpathlineto{\pgfqpoint{1.869151in}{6.027164in}}%
\pgfpathlineto{\pgfqpoint{1.873446in}{6.023544in}}%
\pgfpathlineto{\pgfqpoint{1.874520in}{6.020706in}}%
\pgfpathlineto{\pgfqpoint{1.875594in}{6.020048in}}%
\pgfpathlineto{\pgfqpoint{1.876668in}{6.028521in}}%
\pgfpathlineto{\pgfqpoint{1.880963in}{6.028315in}}%
\pgfpathlineto{\pgfqpoint{1.882037in}{6.030495in}}%
\pgfpathlineto{\pgfqpoint{1.883111in}{6.042629in}}%
\pgfpathlineto{\pgfqpoint{1.884185in}{6.035843in}}%
\pgfpathlineto{\pgfqpoint{1.887406in}{6.036665in}}%
\pgfpathlineto{\pgfqpoint{1.888480in}{6.034444in}}%
\pgfpathlineto{\pgfqpoint{1.889554in}{6.035555in}}%
\pgfpathlineto{\pgfqpoint{1.890628in}{6.043493in}}%
\pgfpathlineto{\pgfqpoint{1.891702in}{6.029632in}}%
\pgfpathlineto{\pgfqpoint{1.894923in}{6.037323in}}%
\pgfpathlineto{\pgfqpoint{1.895997in}{6.042835in}}%
\pgfpathlineto{\pgfqpoint{1.897071in}{6.038722in}}%
\pgfpathlineto{\pgfqpoint{1.898145in}{6.045015in}}%
\pgfpathlineto{\pgfqpoint{1.902440in}{6.038845in}}%
\pgfpathlineto{\pgfqpoint{1.903514in}{6.042382in}}%
\pgfpathlineto{\pgfqpoint{1.904588in}{6.041231in}}%
\pgfpathlineto{\pgfqpoint{1.905661in}{6.046455in}}%
\pgfpathlineto{\pgfqpoint{1.906735in}{6.046578in}}%
\pgfpathlineto{\pgfqpoint{1.911031in}{6.050115in}}%
\pgfpathlineto{\pgfqpoint{1.912105in}{6.051884in}}%
\pgfpathlineto{\pgfqpoint{1.913178in}{6.050732in}}%
\pgfpathlineto{\pgfqpoint{1.914252in}{6.054516in}}%
\pgfpathlineto{\pgfqpoint{1.918548in}{6.058177in}}%
\pgfpathlineto{\pgfqpoint{1.919621in}{6.057313in}}%
\pgfpathlineto{\pgfqpoint{1.920695in}{6.059946in}}%
\pgfpathlineto{\pgfqpoint{1.921769in}{6.056449in}}%
\pgfpathlineto{\pgfqpoint{1.924991in}{6.063195in}}%
\pgfpathlineto{\pgfqpoint{1.927138in}{6.051020in}}%
\pgfpathlineto{\pgfqpoint{1.928212in}{6.052501in}}%
\pgfpathlineto{\pgfqpoint{1.929286in}{6.040778in}}%
\pgfpathlineto{\pgfqpoint{1.932507in}{6.048141in}}%
\pgfpathlineto{\pgfqpoint{1.933581in}{6.033128in}}%
\pgfpathlineto{\pgfqpoint{1.934655in}{6.031195in}}%
\pgfpathlineto{\pgfqpoint{1.935729in}{6.041190in}}%
\pgfpathlineto{\pgfqpoint{1.936803in}{6.034897in}}%
\pgfpathlineto{\pgfqpoint{1.940024in}{6.047524in}}%
\pgfpathlineto{\pgfqpoint{1.941098in}{6.040326in}}%
\pgfpathlineto{\pgfqpoint{1.942172in}{6.048511in}}%
\pgfpathlineto{\pgfqpoint{1.943246in}{6.045550in}}%
\pgfpathlineto{\pgfqpoint{1.944320in}{6.048552in}}%
\pgfpathlineto{\pgfqpoint{1.947541in}{6.047236in}}%
\pgfpathlineto{\pgfqpoint{1.948615in}{6.047935in}}%
\pgfpathlineto{\pgfqpoint{1.949689in}{6.034732in}}%
\pgfpathlineto{\pgfqpoint{1.950763in}{6.034321in}}%
\pgfpathlineto{\pgfqpoint{1.955058in}{6.046948in}}%
\pgfpathlineto{\pgfqpoint{1.956132in}{6.042958in}}%
\pgfpathlineto{\pgfqpoint{1.957206in}{6.034074in}}%
\pgfpathlineto{\pgfqpoint{1.958280in}{6.035102in}}%
\pgfpathlineto{\pgfqpoint{1.963649in}{6.047524in}}%
\pgfpathlineto{\pgfqpoint{1.964723in}{6.047771in}}%
\pgfpathlineto{\pgfqpoint{1.966870in}{6.050691in}}%
\pgfpathlineto{\pgfqpoint{1.970092in}{6.046208in}}%
\pgfpathlineto{\pgfqpoint{1.971166in}{6.046742in}}%
\pgfpathlineto{\pgfqpoint{1.972239in}{6.048388in}}%
\pgfpathlineto{\pgfqpoint{1.973313in}{6.046372in}}%
\pgfpathlineto{\pgfqpoint{1.974387in}{6.031112in}}%
\pgfpathlineto{\pgfqpoint{1.977609in}{6.041354in}}%
\pgfpathlineto{\pgfqpoint{1.978682in}{6.039462in}}%
\pgfpathlineto{\pgfqpoint{1.979756in}{6.041930in}}%
\pgfpathlineto{\pgfqpoint{1.980830in}{6.023668in}}%
\pgfpathlineto{\pgfqpoint{1.981904in}{6.021200in}}%
\pgfpathlineto{\pgfqpoint{1.985126in}{6.017539in}}%
\pgfpathlineto{\pgfqpoint{1.986199in}{6.018814in}}%
\pgfpathlineto{\pgfqpoint{1.987273in}{6.013837in}}%
\pgfpathlineto{\pgfqpoint{1.988347in}{6.011739in}}%
\pgfpathlineto{\pgfqpoint{1.989421in}{6.016428in}}%
\pgfpathlineto{\pgfqpoint{1.992642in}{6.021323in}}%
\pgfpathlineto{\pgfqpoint{1.993716in}{6.017251in}}%
\pgfpathlineto{\pgfqpoint{1.994790in}{6.016264in}}%
\pgfpathlineto{\pgfqpoint{1.995864in}{6.019801in}}%
\pgfpathlineto{\pgfqpoint{1.996938in}{6.027081in}}%
\pgfpathlineto{\pgfqpoint{2.000159in}{6.024531in}}%
\pgfpathlineto{\pgfqpoint{2.001233in}{6.025148in}}%
\pgfpathlineto{\pgfqpoint{2.004455in}{6.036912in}}%
\pgfpathlineto{\pgfqpoint{2.007676in}{6.035349in}}%
\pgfpathlineto{\pgfqpoint{2.008750in}{6.036254in}}%
\pgfpathlineto{\pgfqpoint{2.009824in}{6.035555in}}%
\pgfpathlineto{\pgfqpoint{2.010898in}{6.036130in}}%
\pgfpathlineto{\pgfqpoint{2.011971in}{6.032182in}}%
\pgfpathlineto{\pgfqpoint{2.016267in}{6.027081in}}%
\pgfpathlineto{\pgfqpoint{2.017341in}{6.032305in}}%
\pgfpathlineto{\pgfqpoint{2.018415in}{6.031771in}}%
\pgfpathlineto{\pgfqpoint{2.019488in}{6.025189in}}%
\pgfpathlineto{\pgfqpoint{2.023784in}{6.025189in}}%
\pgfpathlineto{\pgfqpoint{2.024858in}{6.029179in}}%
\pgfpathlineto{\pgfqpoint{2.027005in}{6.017868in}}%
\pgfpathlineto{\pgfqpoint{2.030227in}{6.016058in}}%
\pgfpathlineto{\pgfqpoint{2.031301in}{6.017498in}}%
\pgfpathlineto{\pgfqpoint{2.032374in}{6.025066in}}%
\pgfpathlineto{\pgfqpoint{2.033448in}{6.028069in}}%
\pgfpathlineto{\pgfqpoint{2.034522in}{6.021405in}}%
\pgfpathlineto{\pgfqpoint{2.037744in}{6.013426in}}%
\pgfpathlineto{\pgfqpoint{2.039891in}{6.017374in}}%
\pgfpathlineto{\pgfqpoint{2.040965in}{6.027205in}}%
\pgfpathlineto{\pgfqpoint{2.042039in}{6.024737in}}%
\pgfpathlineto{\pgfqpoint{2.046334in}{6.027986in}}%
\pgfpathlineto{\pgfqpoint{2.048482in}{6.013590in}}%
\pgfpathlineto{\pgfqpoint{2.049556in}{6.017909in}}%
\pgfpathlineto{\pgfqpoint{2.052777in}{6.006392in}}%
\pgfpathlineto{\pgfqpoint{2.053851in}{6.007667in}}%
\pgfpathlineto{\pgfqpoint{2.054925in}{6.012603in}}%
\pgfpathlineto{\pgfqpoint{2.055999in}{6.011616in}}%
\pgfpathlineto{\pgfqpoint{2.060294in}{6.010505in}}%
\pgfpathlineto{\pgfqpoint{2.061368in}{6.011822in}}%
\pgfpathlineto{\pgfqpoint{2.062442in}{6.001827in}}%
\pgfpathlineto{\pgfqpoint{2.064590in}{6.010053in}}%
\pgfpathlineto{\pgfqpoint{2.068885in}{6.016757in}}%
\pgfpathlineto{\pgfqpoint{2.069959in}{6.013919in}}%
\pgfpathlineto{\pgfqpoint{2.071033in}{6.018197in}}%
\pgfpathlineto{\pgfqpoint{2.072106in}{6.016428in}}%
\pgfpathlineto{\pgfqpoint{2.075328in}{6.017991in}}%
\pgfpathlineto{\pgfqpoint{2.076402in}{6.013014in}}%
\pgfpathlineto{\pgfqpoint{2.077476in}{6.011739in}}%
\pgfpathlineto{\pgfqpoint{2.078549in}{5.990022in}}%
\pgfpathlineto{\pgfqpoint{2.082845in}{5.987143in}}%
\pgfpathlineto{\pgfqpoint{2.083919in}{5.995945in}}%
\pgfpathlineto{\pgfqpoint{2.086066in}{5.997631in}}%
\pgfpathlineto{\pgfqpoint{2.087140in}{5.996809in}}%
\pgfpathlineto{\pgfqpoint{2.090362in}{5.992325in}}%
\pgfpathlineto{\pgfqpoint{2.091436in}{5.993354in}}%
\pgfpathlineto{\pgfqpoint{2.092509in}{5.995575in}}%
\pgfpathlineto{\pgfqpoint{2.093583in}{5.989158in}}%
\pgfpathlineto{\pgfqpoint{2.094657in}{5.987801in}}%
\pgfpathlineto{\pgfqpoint{2.097879in}{5.996438in}}%
\pgfpathlineto{\pgfqpoint{2.098952in}{5.986279in}}%
\pgfpathlineto{\pgfqpoint{2.100026in}{5.986443in}}%
\pgfpathlineto{\pgfqpoint{2.101100in}{5.982330in}}%
\pgfpathlineto{\pgfqpoint{2.102174in}{5.985580in}}%
\pgfpathlineto{\pgfqpoint{2.105395in}{5.989076in}}%
\pgfpathlineto{\pgfqpoint{2.107543in}{5.980438in}}%
\pgfpathlineto{\pgfqpoint{2.109691in}{5.966453in}}%
\pgfpathlineto{\pgfqpoint{2.113986in}{5.950042in}}%
\pgfpathlineto{\pgfqpoint{2.115060in}{5.968798in}}%
\pgfpathlineto{\pgfqpoint{2.116134in}{5.973158in}}%
\pgfpathlineto{\pgfqpoint{2.117208in}{5.974310in}}%
\pgfpathlineto{\pgfqpoint{2.120429in}{5.966659in}}%
\pgfpathlineto{\pgfqpoint{2.121503in}{5.953209in}}%
\pgfpathlineto{\pgfqpoint{2.122577in}{5.963327in}}%
\pgfpathlineto{\pgfqpoint{2.123651in}{5.965178in}}%
\pgfpathlineto{\pgfqpoint{2.124725in}{5.958145in}}%
\pgfpathlineto{\pgfqpoint{2.129020in}{5.971430in}}%
\pgfpathlineto{\pgfqpoint{2.130094in}{5.961929in}}%
\pgfpathlineto{\pgfqpoint{2.131168in}{5.961641in}}%
\pgfpathlineto{\pgfqpoint{2.132241in}{5.963410in}}%
\pgfpathlineto{\pgfqpoint{2.135463in}{5.961723in}}%
\pgfpathlineto{\pgfqpoint{2.136537in}{5.972047in}}%
\pgfpathlineto{\pgfqpoint{2.137611in}{5.974268in}}%
\pgfpathlineto{\pgfqpoint{2.138684in}{5.969662in}}%
\pgfpathlineto{\pgfqpoint{2.139758in}{5.957322in}}%
\pgfpathlineto{\pgfqpoint{2.142980in}{5.958885in}}%
\pgfpathlineto{\pgfqpoint{2.144054in}{5.951440in}}%
\pgfpathlineto{\pgfqpoint{2.145127in}{5.950083in}}%
\pgfpathlineto{\pgfqpoint{2.146201in}{5.949795in}}%
\pgfpathlineto{\pgfqpoint{2.147275in}{5.957117in}}%
\pgfpathlineto{\pgfqpoint{2.150497in}{5.952757in}}%
\pgfpathlineto{\pgfqpoint{2.151570in}{5.964479in}}%
\pgfpathlineto{\pgfqpoint{2.152644in}{5.965302in}}%
\pgfpathlineto{\pgfqpoint{2.153718in}{5.961723in}}%
\pgfpathlineto{\pgfqpoint{2.154792in}{5.970567in}}%
\pgfpathlineto{\pgfqpoint{2.158014in}{5.982125in}}%
\pgfpathlineto{\pgfqpoint{2.159087in}{5.980150in}}%
\pgfpathlineto{\pgfqpoint{2.161235in}{5.993765in}}%
\pgfpathlineto{\pgfqpoint{2.162309in}{5.995328in}}%
\pgfpathlineto{\pgfqpoint{2.165530in}{5.995945in}}%
\pgfpathlineto{\pgfqpoint{2.167678in}{5.989487in}}%
\pgfpathlineto{\pgfqpoint{2.168752in}{5.992695in}}%
\pgfpathlineto{\pgfqpoint{2.169826in}{5.990886in}}%
\pgfpathlineto{\pgfqpoint{2.173047in}{5.988335in}}%
\pgfpathlineto{\pgfqpoint{2.175195in}{5.995040in}}%
\pgfpathlineto{\pgfqpoint{2.176269in}{6.017868in}}%
\pgfpathlineto{\pgfqpoint{2.177343in}{6.017128in}}%
\pgfpathlineto{\pgfqpoint{2.181638in}{6.020583in}}%
\pgfpathlineto{\pgfqpoint{2.182712in}{6.025560in}}%
\pgfpathlineto{\pgfqpoint{2.184859in}{6.022351in}}%
\pgfpathlineto{\pgfqpoint{2.188081in}{6.032387in}}%
\pgfpathlineto{\pgfqpoint{2.189155in}{6.028069in}}%
\pgfpathlineto{\pgfqpoint{2.192376in}{6.029920in}}%
\pgfpathlineto{\pgfqpoint{2.195598in}{6.023297in}}%
\pgfpathlineto{\pgfqpoint{2.196672in}{6.023750in}}%
\pgfpathlineto{\pgfqpoint{2.197746in}{6.029138in}}%
\pgfpathlineto{\pgfqpoint{2.198819in}{6.018979in}}%
\pgfpathlineto{\pgfqpoint{2.199893in}{6.016593in}}%
\pgfpathlineto{\pgfqpoint{2.203115in}{6.026053in}}%
\pgfpathlineto{\pgfqpoint{2.204189in}{6.021693in}}%
\pgfpathlineto{\pgfqpoint{2.206336in}{6.030537in}}%
\pgfpathlineto{\pgfqpoint{2.207410in}{6.032634in}}%
\pgfpathlineto{\pgfqpoint{2.210632in}{6.031524in}}%
\pgfpathlineto{\pgfqpoint{2.211705in}{6.028192in}}%
\pgfpathlineto{\pgfqpoint{2.212779in}{6.027740in}}%
\pgfpathlineto{\pgfqpoint{2.214927in}{6.029056in}}%
\pgfpathlineto{\pgfqpoint{2.218148in}{6.023832in}}%
\pgfpathlineto{\pgfqpoint{2.219222in}{6.025025in}}%
\pgfpathlineto{\pgfqpoint{2.221370in}{6.015647in}}%
\pgfpathlineto{\pgfqpoint{2.222444in}{6.029920in}}%
\pgfpathlineto{\pgfqpoint{2.225665in}{6.028192in}}%
\pgfpathlineto{\pgfqpoint{2.227813in}{6.021858in}}%
\pgfpathlineto{\pgfqpoint{2.228887in}{6.027328in}}%
\pgfpathlineto{\pgfqpoint{2.229961in}{6.017333in}}%
\pgfpathlineto{\pgfqpoint{2.233182in}{6.027740in}}%
\pgfpathlineto{\pgfqpoint{2.234256in}{5.992408in}}%
\pgfpathlineto{\pgfqpoint{2.235330in}{5.999153in}}%
\pgfpathlineto{\pgfqpoint{2.236404in}{5.995081in}}%
\pgfpathlineto{\pgfqpoint{2.237478in}{5.987883in}}%
\pgfpathlineto{\pgfqpoint{2.240699in}{5.989981in}}%
\pgfpathlineto{\pgfqpoint{2.243921in}{6.003760in}}%
\pgfpathlineto{\pgfqpoint{2.248216in}{6.003883in}}%
\pgfpathlineto{\pgfqpoint{2.249290in}{6.010094in}}%
\pgfpathlineto{\pgfqpoint{2.255733in}{5.987513in}}%
\pgfpathlineto{\pgfqpoint{2.256807in}{5.989898in}}%
\pgfpathlineto{\pgfqpoint{2.258954in}{5.965754in}}%
\pgfpathlineto{\pgfqpoint{2.260028in}{5.963986in}}%
\pgfpathlineto{\pgfqpoint{2.263250in}{5.963862in}}%
\pgfpathlineto{\pgfqpoint{2.264324in}{5.965343in}}%
\pgfpathlineto{\pgfqpoint{2.265397in}{5.957363in}}%
\pgfpathlineto{\pgfqpoint{2.266471in}{5.966536in}}%
\pgfpathlineto{\pgfqpoint{2.267545in}{5.957281in}}%
\pgfpathlineto{\pgfqpoint{2.271840in}{5.956171in}}%
\pgfpathlineto{\pgfqpoint{2.272914in}{5.950823in}}%
\pgfpathlineto{\pgfqpoint{2.273988in}{5.953826in}}%
\pgfpathlineto{\pgfqpoint{2.275062in}{5.960366in}}%
\pgfpathlineto{\pgfqpoint{2.278283in}{5.953086in}}%
\pgfpathlineto{\pgfqpoint{2.279357in}{5.979945in}}%
\pgfpathlineto{\pgfqpoint{2.280431in}{5.982783in}}%
\pgfpathlineto{\pgfqpoint{2.281505in}{5.989364in}}%
\pgfpathlineto{\pgfqpoint{2.282579in}{6.003061in}}%
\pgfpathlineto{\pgfqpoint{2.286874in}{5.991420in}}%
\pgfpathlineto{\pgfqpoint{2.287948in}{6.008737in}}%
\pgfpathlineto{\pgfqpoint{2.289022in}{6.012151in}}%
\pgfpathlineto{\pgfqpoint{2.290096in}{6.012274in}}%
\pgfpathlineto{\pgfqpoint{2.293317in}{6.013878in}}%
\pgfpathlineto{\pgfqpoint{2.294391in}{6.017004in}}%
\pgfpathlineto{\pgfqpoint{2.296539in}{6.005899in}}%
\pgfpathlineto{\pgfqpoint{2.297613in}{6.018238in}}%
\pgfpathlineto{\pgfqpoint{2.301908in}{6.024120in}}%
\pgfpathlineto{\pgfqpoint{2.302982in}{6.028192in}}%
\pgfpathlineto{\pgfqpoint{2.304056in}{6.028809in}}%
\pgfpathlineto{\pgfqpoint{2.305129in}{6.027534in}}%
\pgfpathlineto{\pgfqpoint{2.308351in}{6.032223in}}%
\pgfpathlineto{\pgfqpoint{2.309425in}{6.026506in}}%
\pgfpathlineto{\pgfqpoint{2.310499in}{6.030454in}}%
\pgfpathlineto{\pgfqpoint{2.311572in}{6.037076in}}%
\pgfpathlineto{\pgfqpoint{2.312646in}{6.034362in}}%
\pgfpathlineto{\pgfqpoint{2.315868in}{6.029138in}}%
\pgfpathlineto{\pgfqpoint{2.316942in}{6.039339in}}%
\pgfpathlineto{\pgfqpoint{2.319089in}{6.038516in}}%
\pgfpathlineto{\pgfqpoint{2.320163in}{6.041107in}}%
\pgfpathlineto{\pgfqpoint{2.323385in}{6.042999in}}%
\pgfpathlineto{\pgfqpoint{2.324458in}{6.041601in}}%
\pgfpathlineto{\pgfqpoint{2.326606in}{6.040449in}}%
\pgfpathlineto{\pgfqpoint{2.327680in}{6.047894in}}%
\pgfpathlineto{\pgfqpoint{2.330902in}{6.047647in}}%
\pgfpathlineto{\pgfqpoint{2.334123in}{6.054928in}}%
\pgfpathlineto{\pgfqpoint{2.335197in}{6.060809in}}%
\pgfpathlineto{\pgfqpoint{2.338418in}{6.059000in}}%
\pgfpathlineto{\pgfqpoint{2.339492in}{6.059246in}}%
\pgfpathlineto{\pgfqpoint{2.340566in}{6.056861in}}%
\pgfpathlineto{\pgfqpoint{2.341640in}{6.057560in}}%
\pgfpathlineto{\pgfqpoint{2.345935in}{6.064388in}}%
\pgfpathlineto{\pgfqpoint{2.347009in}{6.057313in}}%
\pgfpathlineto{\pgfqpoint{2.348083in}{6.066157in}}%
\pgfpathlineto{\pgfqpoint{2.349157in}{6.065704in}}%
\pgfpathlineto{\pgfqpoint{2.350231in}{6.069077in}}%
\pgfpathlineto{\pgfqpoint{2.354526in}{6.062619in}}%
\pgfpathlineto{\pgfqpoint{2.355600in}{6.066362in}}%
\pgfpathlineto{\pgfqpoint{2.356674in}{6.067678in}}%
\pgfpathlineto{\pgfqpoint{2.357747in}{6.065704in}}%
\pgfpathlineto{\pgfqpoint{2.360969in}{6.065540in}}%
\pgfpathlineto{\pgfqpoint{2.362043in}{6.070928in}}%
\pgfpathlineto{\pgfqpoint{2.363117in}{6.073026in}}%
\pgfpathlineto{\pgfqpoint{2.364191in}{6.071421in}}%
\pgfpathlineto{\pgfqpoint{2.365264in}{6.073766in}}%
\pgfpathlineto{\pgfqpoint{2.369560in}{6.077015in}}%
\pgfpathlineto{\pgfqpoint{2.370634in}{6.074342in}}%
\pgfpathlineto{\pgfqpoint{2.371707in}{6.073313in}}%
\pgfpathlineto{\pgfqpoint{2.372781in}{6.073313in}}%
\pgfpathlineto{\pgfqpoint{2.376003in}{6.072244in}}%
\pgfpathlineto{\pgfqpoint{2.377077in}{6.064018in}}%
\pgfpathlineto{\pgfqpoint{2.378150in}{6.069941in}}%
\pgfpathlineto{\pgfqpoint{2.379224in}{6.067103in}}%
\pgfpathlineto{\pgfqpoint{2.383520in}{6.072121in}}%
\pgfpathlineto{\pgfqpoint{2.384593in}{6.070722in}}%
\pgfpathlineto{\pgfqpoint{2.385667in}{6.067678in}}%
\pgfpathlineto{\pgfqpoint{2.386741in}{6.069982in}}%
\pgfpathlineto{\pgfqpoint{2.387815in}{6.074177in}}%
\pgfpathlineto{\pgfqpoint{2.391036in}{6.072779in}}%
\pgfpathlineto{\pgfqpoint{2.392110in}{6.079319in}}%
\pgfpathlineto{\pgfqpoint{2.393184in}{6.077468in}}%
\pgfpathlineto{\pgfqpoint{2.394258in}{6.078907in}}%
\pgfpathlineto{\pgfqpoint{2.395332in}{6.072038in}}%
\pgfpathlineto{\pgfqpoint{2.398553in}{6.076645in}}%
\pgfpathlineto{\pgfqpoint{2.399627in}{6.070105in}}%
\pgfpathlineto{\pgfqpoint{2.400701in}{6.070558in}}%
\pgfpathlineto{\pgfqpoint{2.401775in}{6.064141in}}%
\pgfpathlineto{\pgfqpoint{2.402849in}{6.063730in}}%
\pgfpathlineto{\pgfqpoint{2.406070in}{6.067884in}}%
\pgfpathlineto{\pgfqpoint{2.408218in}{6.082527in}}%
\pgfpathlineto{\pgfqpoint{2.409292in}{6.078455in}}%
\pgfpathlineto{\pgfqpoint{2.410366in}{6.078373in}}%
\pgfpathlineto{\pgfqpoint{2.414661in}{6.076234in}}%
\pgfpathlineto{\pgfqpoint{2.415735in}{6.077632in}}%
\pgfpathlineto{\pgfqpoint{2.416809in}{6.075164in}}%
\pgfpathlineto{\pgfqpoint{2.417882in}{6.076439in}}%
\pgfpathlineto{\pgfqpoint{2.423252in}{6.087915in}}%
\pgfpathlineto{\pgfqpoint{2.425399in}{6.077139in}}%
\pgfpathlineto{\pgfqpoint{2.428621in}{6.071750in}}%
\pgfpathlineto{\pgfqpoint{2.429695in}{6.073560in}}%
\pgfpathlineto{\pgfqpoint{2.430769in}{6.074259in}}%
\pgfpathlineto{\pgfqpoint{2.431842in}{6.082321in}}%
\pgfpathlineto{\pgfqpoint{2.432916in}{6.078578in}}%
\pgfpathlineto{\pgfqpoint{2.436138in}{6.087216in}}%
\pgfpathlineto{\pgfqpoint{2.437212in}{6.087751in}}%
\pgfpathlineto{\pgfqpoint{2.438285in}{6.087216in}}%
\pgfpathlineto{\pgfqpoint{2.439359in}{6.098075in}}%
\pgfpathlineto{\pgfqpoint{2.440433in}{6.079236in}}%
\pgfpathlineto{\pgfqpoint{2.443655in}{6.071956in}}%
\pgfpathlineto{\pgfqpoint{2.447950in}{6.103463in}}%
\pgfpathlineto{\pgfqpoint{2.452245in}{6.103792in}}%
\pgfpathlineto{\pgfqpoint{2.454393in}{6.100954in}}%
\pgfpathlineto{\pgfqpoint{2.455467in}{6.109427in}}%
\pgfpathlineto{\pgfqpoint{2.458688in}{6.112800in}}%
\pgfpathlineto{\pgfqpoint{2.459762in}{6.117119in}}%
\pgfpathlineto{\pgfqpoint{2.460836in}{6.117324in}}%
\pgfpathlineto{\pgfqpoint{2.461910in}{6.123700in}}%
\pgfpathlineto{\pgfqpoint{2.462984in}{6.125592in}}%
\pgfpathlineto{\pgfqpoint{2.466205in}{6.124481in}}%
\pgfpathlineto{\pgfqpoint{2.468353in}{6.125633in}}%
\pgfpathlineto{\pgfqpoint{2.469427in}{6.121273in}}%
\pgfpathlineto{\pgfqpoint{2.470501in}{6.121972in}}%
\pgfpathlineto{\pgfqpoint{2.473722in}{6.118887in}}%
\pgfpathlineto{\pgfqpoint{2.474796in}{6.111484in}}%
\pgfpathlineto{\pgfqpoint{2.475870in}{6.113787in}}%
\pgfpathlineto{\pgfqpoint{2.476944in}{6.112676in}}%
\pgfpathlineto{\pgfqpoint{2.478017in}{6.114116in}}%
\pgfpathlineto{\pgfqpoint{2.482313in}{6.114116in}}%
\pgfpathlineto{\pgfqpoint{2.483387in}{6.114198in}}%
\pgfpathlineto{\pgfqpoint{2.484460in}{6.112018in}}%
\pgfpathlineto{\pgfqpoint{2.485534in}{6.114898in}}%
\pgfpathlineto{\pgfqpoint{2.488756in}{6.114939in}}%
\pgfpathlineto{\pgfqpoint{2.489830in}{6.114239in}}%
\pgfpathlineto{\pgfqpoint{2.490903in}{6.115844in}}%
\pgfpathlineto{\pgfqpoint{2.491977in}{6.124070in}}%
\pgfpathlineto{\pgfqpoint{2.493051in}{6.121314in}}%
\pgfpathlineto{\pgfqpoint{2.496273in}{6.122425in}}%
\pgfpathlineto{\pgfqpoint{2.497346in}{6.117448in}}%
\pgfpathlineto{\pgfqpoint{2.498420in}{6.124029in}}%
\pgfpathlineto{\pgfqpoint{2.499494in}{6.121396in}}%
\pgfpathlineto{\pgfqpoint{2.500568in}{6.123042in}}%
\pgfpathlineto{\pgfqpoint{2.503790in}{6.120985in}}%
\pgfpathlineto{\pgfqpoint{2.504863in}{6.123659in}}%
\pgfpathlineto{\pgfqpoint{2.505937in}{6.122301in}}%
\pgfpathlineto{\pgfqpoint{2.507011in}{6.122918in}}%
\pgfpathlineto{\pgfqpoint{2.508085in}{6.122301in}}%
\pgfpathlineto{\pgfqpoint{2.511306in}{6.126414in}}%
\pgfpathlineto{\pgfqpoint{2.512380in}{6.125098in}}%
\pgfpathlineto{\pgfqpoint{2.513454in}{6.121643in}}%
\pgfpathlineto{\pgfqpoint{2.515602in}{6.127607in}}%
\pgfpathlineto{\pgfqpoint{2.519897in}{6.126250in}}%
\pgfpathlineto{\pgfqpoint{2.520971in}{6.123576in}}%
\pgfpathlineto{\pgfqpoint{2.522045in}{6.124893in}}%
\pgfpathlineto{\pgfqpoint{2.523119in}{6.107987in}}%
\pgfpathlineto{\pgfqpoint{2.526340in}{6.117324in}}%
\pgfpathlineto{\pgfqpoint{2.527414in}{6.109592in}}%
\pgfpathlineto{\pgfqpoint{2.528488in}{6.107905in}}%
\pgfpathlineto{\pgfqpoint{2.529562in}{6.111566in}}%
\pgfpathlineto{\pgfqpoint{2.530635in}{6.105766in}}%
\pgfpathlineto{\pgfqpoint{2.533857in}{6.112059in}}%
\pgfpathlineto{\pgfqpoint{2.534931in}{6.115432in}}%
\pgfpathlineto{\pgfqpoint{2.536005in}{6.122959in}}%
\pgfpathlineto{\pgfqpoint{2.537079in}{6.123947in}}%
\pgfpathlineto{\pgfqpoint{2.538152in}{6.114610in}}%
\pgfpathlineto{\pgfqpoint{2.541374in}{6.109139in}}%
\pgfpathlineto{\pgfqpoint{2.542448in}{6.110496in}}%
\pgfpathlineto{\pgfqpoint{2.543522in}{6.115391in}}%
\pgfpathlineto{\pgfqpoint{2.544595in}{6.106959in}}%
\pgfpathlineto{\pgfqpoint{2.545669in}{6.110209in}}%
\pgfpathlineto{\pgfqpoint{2.548891in}{6.105684in}}%
\pgfpathlineto{\pgfqpoint{2.549965in}{6.092933in}}%
\pgfpathlineto{\pgfqpoint{2.551038in}{6.095648in}}%
\pgfpathlineto{\pgfqpoint{2.552112in}{6.092810in}}%
\pgfpathlineto{\pgfqpoint{2.553186in}{6.091617in}}%
\pgfpathlineto{\pgfqpoint{2.556408in}{6.090877in}}%
\pgfpathlineto{\pgfqpoint{2.557481in}{6.085365in}}%
\pgfpathlineto{\pgfqpoint{2.558555in}{6.085488in}}%
\pgfpathlineto{\pgfqpoint{2.560703in}{6.087833in}}%
\pgfpathlineto{\pgfqpoint{2.563924in}{6.087380in}}%
\pgfpathlineto{\pgfqpoint{2.564998in}{6.086393in}}%
\pgfpathlineto{\pgfqpoint{2.567146in}{6.086023in}}%
\pgfpathlineto{\pgfqpoint{2.568220in}{6.084666in}}%
\pgfpathlineto{\pgfqpoint{2.571441in}{6.091370in}}%
\pgfpathlineto{\pgfqpoint{2.572515in}{6.072244in}}%
\pgfpathlineto{\pgfqpoint{2.573589in}{6.073313in}}%
\pgfpathlineto{\pgfqpoint{2.574663in}{6.070475in}}%
\pgfpathlineto{\pgfqpoint{2.575737in}{6.070517in}}%
\pgfpathlineto{\pgfqpoint{2.578958in}{6.068707in}}%
\pgfpathlineto{\pgfqpoint{2.580032in}{6.064758in}}%
\pgfpathlineto{\pgfqpoint{2.582180in}{6.074547in}}%
\pgfpathlineto{\pgfqpoint{2.583254in}{6.073190in}}%
\pgfpathlineto{\pgfqpoint{2.587549in}{6.090465in}}%
\pgfpathlineto{\pgfqpoint{2.588623in}{6.088039in}}%
\pgfpathlineto{\pgfqpoint{2.589697in}{6.102805in}}%
\pgfpathlineto{\pgfqpoint{2.590770in}{6.105849in}}%
\pgfpathlineto{\pgfqpoint{2.593992in}{6.097622in}}%
\pgfpathlineto{\pgfqpoint{2.595066in}{6.102640in}}%
\pgfpathlineto{\pgfqpoint{2.596140in}{6.098363in}}%
\pgfpathlineto{\pgfqpoint{2.597213in}{6.101283in}}%
\pgfpathlineto{\pgfqpoint{2.598287in}{6.102023in}}%
\pgfpathlineto{\pgfqpoint{2.601509in}{6.096512in}}%
\pgfpathlineto{\pgfqpoint{2.603657in}{6.099309in}}%
\pgfpathlineto{\pgfqpoint{2.605804in}{6.104039in}}%
\pgfpathlineto{\pgfqpoint{2.609026in}{6.100255in}}%
\pgfpathlineto{\pgfqpoint{2.610100in}{6.101283in}}%
\pgfpathlineto{\pgfqpoint{2.611173in}{6.097334in}}%
\pgfpathlineto{\pgfqpoint{2.612247in}{6.100748in}}%
\pgfpathlineto{\pgfqpoint{2.613321in}{6.099967in}}%
\pgfpathlineto{\pgfqpoint{2.616543in}{6.096882in}}%
\pgfpathlineto{\pgfqpoint{2.617616in}{6.097581in}}%
\pgfpathlineto{\pgfqpoint{2.618690in}{6.113828in}}%
\pgfpathlineto{\pgfqpoint{2.619764in}{6.113170in}}%
\pgfpathlineto{\pgfqpoint{2.620838in}{6.123124in}}%
\pgfpathlineto{\pgfqpoint{2.624059in}{6.127525in}}%
\pgfpathlineto{\pgfqpoint{2.625133in}{6.124440in}}%
\pgfpathlineto{\pgfqpoint{2.626207in}{6.115926in}}%
\pgfpathlineto{\pgfqpoint{2.627281in}{6.113705in}}%
\pgfpathlineto{\pgfqpoint{2.628355in}{6.119175in}}%
\pgfpathlineto{\pgfqpoint{2.631576in}{6.121890in}}%
\pgfpathlineto{\pgfqpoint{2.632650in}{6.123741in}}%
\pgfpathlineto{\pgfqpoint{2.633724in}{6.122918in}}%
\pgfpathlineto{\pgfqpoint{2.634798in}{6.125839in}}%
\pgfpathlineto{\pgfqpoint{2.635872in}{6.124111in}}%
\pgfpathlineto{\pgfqpoint{2.640167in}{6.124769in}}%
\pgfpathlineto{\pgfqpoint{2.641241in}{6.121561in}}%
\pgfpathlineto{\pgfqpoint{2.643389in}{6.123453in}}%
\pgfpathlineto{\pgfqpoint{2.647684in}{6.121437in}}%
\pgfpathlineto{\pgfqpoint{2.648758in}{6.122507in}}%
\pgfpathlineto{\pgfqpoint{2.649832in}{6.120162in}}%
\pgfpathlineto{\pgfqpoint{2.650905in}{6.122137in}}%
\pgfpathlineto{\pgfqpoint{2.654127in}{6.118476in}}%
\pgfpathlineto{\pgfqpoint{2.655201in}{6.115844in}}%
\pgfpathlineto{\pgfqpoint{2.656275in}{6.120862in}}%
\pgfpathlineto{\pgfqpoint{2.657348in}{6.119134in}}%
\pgfpathlineto{\pgfqpoint{2.662718in}{6.118435in}}%
\pgfpathlineto{\pgfqpoint{2.663791in}{6.123124in}}%
\pgfpathlineto{\pgfqpoint{2.664865in}{6.123864in}}%
\pgfpathlineto{\pgfqpoint{2.665939in}{6.123124in}}%
\pgfpathlineto{\pgfqpoint{2.669161in}{6.123206in}}%
\pgfpathlineto{\pgfqpoint{2.670234in}{6.113499in}}%
\pgfpathlineto{\pgfqpoint{2.671308in}{6.116419in}}%
\pgfpathlineto{\pgfqpoint{2.672382in}{6.116748in}}%
\pgfpathlineto{\pgfqpoint{2.673456in}{6.119257in}}%
\pgfpathlineto{\pgfqpoint{2.677751in}{6.109098in}}%
\pgfpathlineto{\pgfqpoint{2.678825in}{6.110455in}}%
\pgfpathlineto{\pgfqpoint{2.679899in}{6.106671in}}%
\pgfpathlineto{\pgfqpoint{2.680973in}{6.109962in}}%
\pgfpathlineto{\pgfqpoint{2.684194in}{6.110167in}}%
\pgfpathlineto{\pgfqpoint{2.685268in}{6.112718in}}%
\pgfpathlineto{\pgfqpoint{2.688490in}{6.125098in}}%
\pgfpathlineto{\pgfqpoint{2.692785in}{6.135011in}}%
\pgfpathlineto{\pgfqpoint{2.694933in}{6.146487in}}%
\pgfpathlineto{\pgfqpoint{2.696007in}{6.144718in}}%
\pgfpathlineto{\pgfqpoint{2.700302in}{6.146322in}}%
\pgfpathlineto{\pgfqpoint{2.701376in}{6.156605in}}%
\pgfpathlineto{\pgfqpoint{2.702450in}{6.161047in}}%
\pgfpathlineto{\pgfqpoint{2.703523in}{6.161870in}}%
\pgfpathlineto{\pgfqpoint{2.706745in}{6.159937in}}%
\pgfpathlineto{\pgfqpoint{2.707819in}{6.157798in}}%
\pgfpathlineto{\pgfqpoint{2.708893in}{6.171289in}}%
\pgfpathlineto{\pgfqpoint{2.709967in}{6.171412in}}%
\pgfpathlineto{\pgfqpoint{2.711040in}{6.169191in}}%
\pgfpathlineto{\pgfqpoint{2.714262in}{6.167587in}}%
\pgfpathlineto{\pgfqpoint{2.715336in}{6.168328in}}%
\pgfpathlineto{\pgfqpoint{2.717483in}{6.171454in}}%
\pgfpathlineto{\pgfqpoint{2.718557in}{6.176472in}}%
\pgfpathlineto{\pgfqpoint{2.721779in}{6.177664in}}%
\pgfpathlineto{\pgfqpoint{2.722853in}{6.173099in}}%
\pgfpathlineto{\pgfqpoint{2.723926in}{6.176430in}}%
\pgfpathlineto{\pgfqpoint{2.725000in}{6.173017in}}%
\pgfpathlineto{\pgfqpoint{2.726074in}{6.180914in}}%
\pgfpathlineto{\pgfqpoint{2.729296in}{6.183423in}}%
\pgfpathlineto{\pgfqpoint{2.730369in}{6.180009in}}%
\pgfpathlineto{\pgfqpoint{2.731443in}{6.180379in}}%
\pgfpathlineto{\pgfqpoint{2.732517in}{6.180050in}}%
\pgfpathlineto{\pgfqpoint{2.733591in}{6.177623in}}%
\pgfpathlineto{\pgfqpoint{2.736812in}{6.173469in}}%
\pgfpathlineto{\pgfqpoint{2.737886in}{6.175649in}}%
\pgfpathlineto{\pgfqpoint{2.738960in}{6.174456in}}%
\pgfpathlineto{\pgfqpoint{2.740034in}{6.176760in}}%
\pgfpathlineto{\pgfqpoint{2.741108in}{6.176965in}}%
\pgfpathlineto{\pgfqpoint{2.744329in}{6.174621in}}%
\pgfpathlineto{\pgfqpoint{2.745403in}{6.172564in}}%
\pgfpathlineto{\pgfqpoint{2.746477in}{6.172770in}}%
\pgfpathlineto{\pgfqpoint{2.747551in}{6.171412in}}%
\pgfpathlineto{\pgfqpoint{2.748625in}{6.171783in}}%
\pgfpathlineto{\pgfqpoint{2.751846in}{6.170713in}}%
\pgfpathlineto{\pgfqpoint{2.752920in}{6.172112in}}%
\pgfpathlineto{\pgfqpoint{2.753994in}{6.170672in}}%
\pgfpathlineto{\pgfqpoint{2.755068in}{6.166641in}}%
\pgfpathlineto{\pgfqpoint{2.759363in}{6.173222in}}%
\pgfpathlineto{\pgfqpoint{2.760437in}{6.172687in}}%
\pgfpathlineto{\pgfqpoint{2.761511in}{6.171166in}}%
\pgfpathlineto{\pgfqpoint{2.762585in}{6.176307in}}%
\pgfpathlineto{\pgfqpoint{2.763658in}{6.177582in}}%
\pgfpathlineto{\pgfqpoint{2.767954in}{6.191567in}}%
\pgfpathlineto{\pgfqpoint{2.769028in}{6.191073in}}%
\pgfpathlineto{\pgfqpoint{2.770101in}{6.195310in}}%
\pgfpathlineto{\pgfqpoint{2.774397in}{6.190333in}}%
\pgfpathlineto{\pgfqpoint{2.777618in}{6.208390in}}%
\pgfpathlineto{\pgfqpoint{2.778692in}{6.207978in}}%
\pgfpathlineto{\pgfqpoint{2.781914in}{6.205099in}}%
\pgfpathlineto{\pgfqpoint{2.782988in}{6.202426in}}%
\pgfpathlineto{\pgfqpoint{2.784061in}{6.197366in}}%
\pgfpathlineto{\pgfqpoint{2.785135in}{6.197778in}}%
\pgfpathlineto{\pgfqpoint{2.786209in}{6.197079in}}%
\pgfpathlineto{\pgfqpoint{2.790504in}{6.202261in}}%
\pgfpathlineto{\pgfqpoint{2.791578in}{6.194899in}}%
\pgfpathlineto{\pgfqpoint{2.793726in}{6.198642in}}%
\pgfpathlineto{\pgfqpoint{2.796947in}{6.209130in}}%
\pgfpathlineto{\pgfqpoint{2.798021in}{6.206086in}}%
\pgfpathlineto{\pgfqpoint{2.799095in}{6.205181in}}%
\pgfpathlineto{\pgfqpoint{2.801243in}{6.217480in}}%
\pgfpathlineto{\pgfqpoint{2.805538in}{6.224308in}}%
\pgfpathlineto{\pgfqpoint{2.806612in}{6.232164in}}%
\pgfpathlineto{\pgfqpoint{2.807686in}{6.231711in}}%
\pgfpathlineto{\pgfqpoint{2.808760in}{6.240801in}}%
\pgfpathlineto{\pgfqpoint{2.811981in}{6.238951in}}%
\pgfpathlineto{\pgfqpoint{2.814129in}{6.234262in}}%
\pgfpathlineto{\pgfqpoint{2.816277in}{6.241665in}}%
\pgfpathlineto{\pgfqpoint{2.819498in}{6.243475in}}%
\pgfpathlineto{\pgfqpoint{2.821646in}{6.253593in}}%
\pgfpathlineto{\pgfqpoint{2.822720in}{6.258529in}}%
\pgfpathlineto{\pgfqpoint{2.823793in}{6.266097in}}%
\pgfpathlineto{\pgfqpoint{2.828089in}{6.266550in}}%
\pgfpathlineto{\pgfqpoint{2.830236in}{6.262149in}}%
\pgfpathlineto{\pgfqpoint{2.831310in}{6.264781in}}%
\pgfpathlineto{\pgfqpoint{2.834532in}{6.263630in}}%
\pgfpathlineto{\pgfqpoint{2.835606in}{6.252689in}}%
\pgfpathlineto{\pgfqpoint{2.836679in}{6.255938in}}%
\pgfpathlineto{\pgfqpoint{2.837753in}{6.245244in}}%
\pgfpathlineto{\pgfqpoint{2.838827in}{6.246560in}}%
\pgfpathlineto{\pgfqpoint{2.842049in}{6.252894in}}%
\pgfpathlineto{\pgfqpoint{2.844196in}{6.252606in}}%
\pgfpathlineto{\pgfqpoint{2.845270in}{6.245902in}}%
\pgfpathlineto{\pgfqpoint{2.846344in}{6.251948in}}%
\pgfpathlineto{\pgfqpoint{2.849566in}{6.255444in}}%
\pgfpathlineto{\pgfqpoint{2.850639in}{6.252236in}}%
\pgfpathlineto{\pgfqpoint{2.851713in}{6.258570in}}%
\pgfpathlineto{\pgfqpoint{2.852787in}{6.257789in}}%
\pgfpathlineto{\pgfqpoint{2.853861in}{6.260421in}}%
\pgfpathlineto{\pgfqpoint{2.857082in}{6.260051in}}%
\pgfpathlineto{\pgfqpoint{2.858156in}{6.258611in}}%
\pgfpathlineto{\pgfqpoint{2.859230in}{6.261696in}}%
\pgfpathlineto{\pgfqpoint{2.860304in}{6.263013in}}%
\pgfpathlineto{\pgfqpoint{2.861378in}{6.258036in}}%
\pgfpathlineto{\pgfqpoint{2.864599in}{6.253552in}}%
\pgfpathlineto{\pgfqpoint{2.865673in}{6.212503in}}%
\pgfpathlineto{\pgfqpoint{2.866747in}{6.211104in}}%
\pgfpathlineto{\pgfqpoint{2.867821in}{6.215053in}}%
\pgfpathlineto{\pgfqpoint{2.868895in}{6.213778in}}%
\pgfpathlineto{\pgfqpoint{2.872116in}{6.219413in}}%
\pgfpathlineto{\pgfqpoint{2.875338in}{6.244339in}}%
\pgfpathlineto{\pgfqpoint{2.876411in}{6.244462in}}%
\pgfpathlineto{\pgfqpoint{2.879633in}{6.243640in}}%
\pgfpathlineto{\pgfqpoint{2.880707in}{6.239732in}}%
\pgfpathlineto{\pgfqpoint{2.881781in}{6.239938in}}%
\pgfpathlineto{\pgfqpoint{2.883928in}{6.238005in}}%
\pgfpathlineto{\pgfqpoint{2.887150in}{6.243393in}}%
\pgfpathlineto{\pgfqpoint{2.888224in}{6.242652in}}%
\pgfpathlineto{\pgfqpoint{2.889298in}{6.245490in}}%
\pgfpathlineto{\pgfqpoint{2.891445in}{6.228544in}}%
\pgfpathlineto{\pgfqpoint{2.894667in}{6.232493in}}%
\pgfpathlineto{\pgfqpoint{2.895741in}{6.235989in}}%
\pgfpathlineto{\pgfqpoint{2.896814in}{6.229696in}}%
\pgfpathlineto{\pgfqpoint{2.897888in}{6.227557in}}%
\pgfpathlineto{\pgfqpoint{2.898962in}{6.227598in}}%
\pgfpathlineto{\pgfqpoint{2.902184in}{6.228914in}}%
\pgfpathlineto{\pgfqpoint{2.903257in}{6.230683in}}%
\pgfpathlineto{\pgfqpoint{2.905405in}{6.236112in}}%
\pgfpathlineto{\pgfqpoint{2.906479in}{6.233192in}}%
\pgfpathlineto{\pgfqpoint{2.910774in}{6.223156in}}%
\pgfpathlineto{\pgfqpoint{2.911848in}{6.227310in}}%
\pgfpathlineto{\pgfqpoint{2.913996in}{6.241460in}}%
\pgfpathlineto{\pgfqpoint{2.917217in}{6.256514in}}%
\pgfpathlineto{\pgfqpoint{2.918291in}{6.256843in}}%
\pgfpathlineto{\pgfqpoint{2.919365in}{6.256226in}}%
\pgfpathlineto{\pgfqpoint{2.921513in}{6.271239in}}%
\pgfpathlineto{\pgfqpoint{2.924734in}{6.272843in}}%
\pgfpathlineto{\pgfqpoint{2.925808in}{6.272062in}}%
\pgfpathlineto{\pgfqpoint{2.926882in}{6.260216in}}%
\pgfpathlineto{\pgfqpoint{2.927956in}{6.259928in}}%
\pgfpathlineto{\pgfqpoint{2.929030in}{6.261203in}}%
\pgfpathlineto{\pgfqpoint{2.932251in}{6.261079in}}%
\pgfpathlineto{\pgfqpoint{2.933325in}{6.262231in}}%
\pgfpathlineto{\pgfqpoint{2.934399in}{6.256185in}}%
\pgfpathlineto{\pgfqpoint{2.935473in}{6.256473in}}%
\pgfpathlineto{\pgfqpoint{2.936546in}{6.257830in}}%
\pgfpathlineto{\pgfqpoint{2.939768in}{6.268977in}}%
\pgfpathlineto{\pgfqpoint{2.941916in}{6.283578in}}%
\pgfpathlineto{\pgfqpoint{2.942989in}{6.283003in}}%
\pgfpathlineto{\pgfqpoint{2.944063in}{6.283578in}}%
\pgfpathlineto{\pgfqpoint{2.948359in}{6.284483in}}%
\pgfpathlineto{\pgfqpoint{2.949433in}{6.283537in}}%
\pgfpathlineto{\pgfqpoint{2.950506in}{6.287733in}}%
\pgfpathlineto{\pgfqpoint{2.951580in}{6.288267in}}%
\pgfpathlineto{\pgfqpoint{2.954802in}{6.292134in}}%
\pgfpathlineto{\pgfqpoint{2.955876in}{6.288391in}}%
\pgfpathlineto{\pgfqpoint{2.956949in}{6.290406in}}%
\pgfpathlineto{\pgfqpoint{2.958023in}{6.294190in}}%
\pgfpathlineto{\pgfqpoint{2.959097in}{6.302252in}}%
\pgfpathlineto{\pgfqpoint{2.962319in}{6.303157in}}%
\pgfpathlineto{\pgfqpoint{2.963392in}{6.354119in}}%
\pgfpathlineto{\pgfqpoint{2.964466in}{6.365924in}}%
\pgfpathlineto{\pgfqpoint{2.965540in}{6.347456in}}%
\pgfpathlineto{\pgfqpoint{2.966614in}{6.354489in}}%
\pgfpathlineto{\pgfqpoint{2.970909in}{6.336762in}}%
\pgfpathlineto{\pgfqpoint{2.971983in}{6.336720in}}%
\pgfpathlineto{\pgfqpoint{2.973057in}{6.344700in}}%
\pgfpathlineto{\pgfqpoint{2.974131in}{6.344659in}}%
\pgfpathlineto{\pgfqpoint{2.977352in}{6.337255in}}%
\pgfpathlineto{\pgfqpoint{2.979500in}{6.335363in}}%
\pgfpathlineto{\pgfqpoint{2.981648in}{6.326109in}}%
\pgfpathlineto{\pgfqpoint{2.984869in}{6.329111in}}%
\pgfpathlineto{\pgfqpoint{2.985943in}{6.333430in}}%
\pgfpathlineto{\pgfqpoint{2.987017in}{6.325903in}}%
\pgfpathlineto{\pgfqpoint{2.988091in}{6.333759in}}%
\pgfpathlineto{\pgfqpoint{2.989165in}{6.333553in}}%
\pgfpathlineto{\pgfqpoint{2.992386in}{6.341821in}}%
\pgfpathlineto{\pgfqpoint{2.993460in}{6.351939in}}%
\pgfpathlineto{\pgfqpoint{2.994534in}{6.346715in}}%
\pgfpathlineto{\pgfqpoint{2.996681in}{6.345934in}}%
\pgfpathlineto{\pgfqpoint{2.999903in}{6.356176in}}%
\pgfpathlineto{\pgfqpoint{3.002051in}{6.371847in}}%
\pgfpathlineto{\pgfqpoint{3.003124in}{6.391919in}}%
\pgfpathlineto{\pgfqpoint{3.004198in}{6.384145in}}%
\pgfpathlineto{\pgfqpoint{3.007420in}{6.376742in}}%
\pgfpathlineto{\pgfqpoint{3.008494in}{6.372834in}}%
\pgfpathlineto{\pgfqpoint{3.009567in}{6.374438in}}%
\pgfpathlineto{\pgfqpoint{3.010641in}{6.380814in}}%
\pgfpathlineto{\pgfqpoint{3.011715in}{6.372340in}}%
\pgfpathlineto{\pgfqpoint{3.014937in}{6.376906in}}%
\pgfpathlineto{\pgfqpoint{3.016011in}{6.366253in}}%
\pgfpathlineto{\pgfqpoint{3.017084in}{6.376166in}}%
\pgfpathlineto{\pgfqpoint{3.018158in}{6.372135in}}%
\pgfpathlineto{\pgfqpoint{3.019232in}{6.371806in}}%
\pgfpathlineto{\pgfqpoint{3.022454in}{6.373451in}}%
\pgfpathlineto{\pgfqpoint{3.023527in}{6.373328in}}%
\pgfpathlineto{\pgfqpoint{3.026749in}{6.359055in}}%
\pgfpathlineto{\pgfqpoint{3.031044in}{6.361852in}}%
\pgfpathlineto{\pgfqpoint{3.032118in}{6.364772in}}%
\pgfpathlineto{\pgfqpoint{3.034266in}{6.361523in}}%
\pgfpathlineto{\pgfqpoint{3.034266in}{6.361523in}}%
\pgfusepath{stroke}%
\end{pgfscope}%
\begin{pgfscope}%
\pgfpathrectangle{\pgfqpoint{0.568357in}{5.688815in}}{\pgfqpoint{2.583333in}{0.736585in}}%
\pgfusepath{clip}%
\pgfsetroundcap%
\pgfsetroundjoin%
\pgfsetlinewidth{1.505625pt}%
\definecolor{currentstroke}{rgb}{1.000000,0.647059,0.000000}%
\pgfsetstrokecolor{currentstroke}%
\pgfsetdash{}{0pt}%
\pgfpathmoveto{\pgfqpoint{0.685781in}{5.722666in}}%
\pgfpathlineto{\pgfqpoint{0.686855in}{5.723839in}}%
\pgfpathlineto{\pgfqpoint{0.696519in}{5.723900in}}%
\pgfpathlineto{\pgfqpoint{0.701888in}{5.724390in}}%
\pgfpathlineto{\pgfqpoint{0.704036in}{5.725286in}}%
\pgfpathlineto{\pgfqpoint{0.708332in}{5.725962in}}%
\pgfpathlineto{\pgfqpoint{0.714775in}{5.727916in}}%
\pgfpathlineto{\pgfqpoint{0.726587in}{5.730616in}}%
\pgfpathlineto{\pgfqpoint{0.739473in}{5.732123in}}%
\pgfpathlineto{\pgfqpoint{0.746990in}{5.733054in}}%
\pgfpathlineto{\pgfqpoint{0.753433in}{5.733291in}}%
\pgfpathlineto{\pgfqpoint{0.759876in}{5.733454in}}%
\pgfpathlineto{\pgfqpoint{0.771688in}{5.735078in}}%
\pgfpathlineto{\pgfqpoint{0.798534in}{5.735782in}}%
\pgfpathlineto{\pgfqpoint{0.814642in}{5.736522in}}%
\pgfpathlineto{\pgfqpoint{0.823232in}{5.736747in}}%
\pgfpathlineto{\pgfqpoint{0.845783in}{5.735987in}}%
\pgfpathlineto{\pgfqpoint{0.854374in}{5.735570in}}%
\pgfpathlineto{\pgfqpoint{0.876924in}{5.735913in}}%
\pgfpathlineto{\pgfqpoint{0.906992in}{5.737232in}}%
\pgfpathlineto{\pgfqpoint{0.922025in}{5.738295in}}%
\pgfpathlineto{\pgfqpoint{0.933838in}{5.739225in}}%
\pgfpathlineto{\pgfqpoint{0.944576in}{5.740160in}}%
\pgfpathlineto{\pgfqpoint{0.959610in}{5.741047in}}%
\pgfpathlineto{\pgfqpoint{0.978939in}{5.742363in}}%
\pgfpathlineto{\pgfqpoint{0.997194in}{5.743761in}}%
\pgfpathlineto{\pgfqpoint{1.086323in}{5.746429in}}%
\pgfpathlineto{\pgfqpoint{1.184042in}{5.756866in}}%
\pgfpathlineto{\pgfqpoint{1.192633in}{5.757853in}}%
\pgfpathlineto{\pgfqpoint{1.204445in}{5.758922in}}%
\pgfpathlineto{\pgfqpoint{1.245251in}{5.763884in}}%
\pgfpathlineto{\pgfqpoint{1.255989in}{5.764955in}}%
\pgfpathlineto{\pgfqpoint{1.263506in}{5.765791in}}%
\pgfpathlineto{\pgfqpoint{1.275318in}{5.767046in}}%
\pgfpathlineto{\pgfqpoint{1.286057in}{5.768189in}}%
\pgfpathlineto{\pgfqpoint{1.297869in}{5.770003in}}%
\pgfpathlineto{\pgfqpoint{1.304312in}{5.770831in}}%
\pgfpathlineto{\pgfqpoint{1.312903in}{5.772082in}}%
\pgfpathlineto{\pgfqpoint{1.323641in}{5.773241in}}%
\pgfpathlineto{\pgfqpoint{1.335453in}{5.774724in}}%
\pgfpathlineto{\pgfqpoint{1.346192in}{5.775565in}}%
\pgfpathlineto{\pgfqpoint{1.358004in}{5.777366in}}%
\pgfpathlineto{\pgfqpoint{1.368742in}{5.778570in}}%
\pgfpathlineto{\pgfqpoint{1.380554in}{5.780212in}}%
\pgfpathlineto{\pgfqpoint{1.391293in}{5.781405in}}%
\pgfpathlineto{\pgfqpoint{1.403105in}{5.783329in}}%
\pgfpathlineto{\pgfqpoint{1.409548in}{5.784217in}}%
\pgfpathlineto{\pgfqpoint{1.418139in}{5.785622in}}%
\pgfpathlineto{\pgfqpoint{1.424582in}{5.786583in}}%
\pgfpathlineto{\pgfqpoint{1.431025in}{5.787598in}}%
\pgfpathlineto{\pgfqpoint{1.454649in}{5.790952in}}%
\pgfpathlineto{\pgfqpoint{1.460018in}{5.791763in}}%
\pgfpathlineto{\pgfqpoint{1.477200in}{5.794494in}}%
\pgfpathlineto{\pgfqpoint{1.485791in}{5.796032in}}%
\pgfpathlineto{\pgfqpoint{1.496529in}{5.797165in}}%
\pgfpathlineto{\pgfqpoint{1.508341in}{5.798831in}}%
\pgfpathlineto{\pgfqpoint{1.520153in}{5.800040in}}%
\pgfpathlineto{\pgfqpoint{1.530892in}{5.801717in}}%
\pgfpathlineto{\pgfqpoint{1.537335in}{5.802553in}}%
\pgfpathlineto{\pgfqpoint{1.545926in}{5.803746in}}%
\pgfpathlineto{\pgfqpoint{1.556664in}{5.804919in}}%
\pgfpathlineto{\pgfqpoint{1.568476in}{5.806777in}}%
\pgfpathlineto{\pgfqpoint{1.579215in}{5.807952in}}%
\pgfpathlineto{\pgfqpoint{1.591027in}{5.809631in}}%
\pgfpathlineto{\pgfqpoint{1.597470in}{5.810491in}}%
\pgfpathlineto{\pgfqpoint{1.606061in}{5.811826in}}%
\pgfpathlineto{\pgfqpoint{1.612504in}{5.812735in}}%
\pgfpathlineto{\pgfqpoint{1.621094in}{5.814051in}}%
\pgfpathlineto{\pgfqpoint{1.628611in}{5.814950in}}%
\pgfpathlineto{\pgfqpoint{1.635054in}{5.815858in}}%
\pgfpathlineto{\pgfqpoint{1.643645in}{5.817249in}}%
\pgfpathlineto{\pgfqpoint{1.650088in}{5.818154in}}%
\pgfpathlineto{\pgfqpoint{1.658679in}{5.819487in}}%
\pgfpathlineto{\pgfqpoint{1.670491in}{5.820821in}}%
\pgfpathlineto{\pgfqpoint{1.681229in}{5.822562in}}%
\pgfpathlineto{\pgfqpoint{1.687672in}{5.823419in}}%
\pgfpathlineto{\pgfqpoint{1.696263in}{5.824633in}}%
\pgfpathlineto{\pgfqpoint{1.707001in}{5.825721in}}%
\pgfpathlineto{\pgfqpoint{1.718814in}{5.827480in}}%
\pgfpathlineto{\pgfqpoint{1.730626in}{5.828679in}}%
\pgfpathlineto{\pgfqpoint{1.741364in}{5.830244in}}%
\pgfpathlineto{\pgfqpoint{1.752103in}{5.831439in}}%
\pgfpathlineto{\pgfqpoint{1.763915in}{5.833012in}}%
\pgfpathlineto{\pgfqpoint{1.776801in}{5.834164in}}%
\pgfpathlineto{\pgfqpoint{1.813311in}{5.839193in}}%
\pgfpathlineto{\pgfqpoint{1.821902in}{5.840660in}}%
\pgfpathlineto{\pgfqpoint{1.836936in}{5.842885in}}%
\pgfpathlineto{\pgfqpoint{1.846600in}{5.844574in}}%
\pgfpathlineto{\pgfqpoint{1.857339in}{5.845913in}}%
\pgfpathlineto{\pgfqpoint{1.861634in}{5.846678in}}%
\pgfpathlineto{\pgfqpoint{1.868077in}{5.847610in}}%
\pgfpathlineto{\pgfqpoint{1.876668in}{5.848993in}}%
\pgfpathlineto{\pgfqpoint{1.884185in}{5.849957in}}%
\pgfpathlineto{\pgfqpoint{1.890628in}{5.850928in}}%
\pgfpathlineto{\pgfqpoint{1.899218in}{5.852380in}}%
\pgfpathlineto{\pgfqpoint{1.905661in}{5.853350in}}%
\pgfpathlineto{\pgfqpoint{1.906735in}{5.853596in}}%
\pgfpathlineto{\pgfqpoint{1.913178in}{5.854348in}}%
\pgfpathlineto{\pgfqpoint{1.921769in}{5.855884in}}%
\pgfpathlineto{\pgfqpoint{1.928212in}{5.856885in}}%
\pgfpathlineto{\pgfqpoint{1.936803in}{5.858240in}}%
\pgfpathlineto{\pgfqpoint{1.943246in}{5.859168in}}%
\pgfpathlineto{\pgfqpoint{1.951837in}{5.860515in}}%
\pgfpathlineto{\pgfqpoint{1.963649in}{5.861844in}}%
\pgfpathlineto{\pgfqpoint{1.974387in}{5.863625in}}%
\pgfpathlineto{\pgfqpoint{1.985126in}{5.864829in}}%
\pgfpathlineto{\pgfqpoint{1.996938in}{5.866467in}}%
\pgfpathlineto{\pgfqpoint{2.007676in}{5.867633in}}%
\pgfpathlineto{\pgfqpoint{2.011971in}{5.868419in}}%
\pgfpathlineto{\pgfqpoint{2.022710in}{5.869352in}}%
\pgfpathlineto{\pgfqpoint{2.034522in}{5.870944in}}%
\pgfpathlineto{\pgfqpoint{2.045260in}{5.871976in}}%
\pgfpathlineto{\pgfqpoint{2.055999in}{5.873273in}}%
\pgfpathlineto{\pgfqpoint{2.067811in}{5.874194in}}%
\pgfpathlineto{\pgfqpoint{2.079623in}{5.875557in}}%
\pgfpathlineto{\pgfqpoint{2.092509in}{5.876611in}}%
\pgfpathlineto{\pgfqpoint{2.108617in}{5.877921in}}%
\pgfpathlineto{\pgfqpoint{2.124725in}{5.878932in}}%
\pgfpathlineto{\pgfqpoint{2.144054in}{5.879932in}}%
\pgfpathlineto{\pgfqpoint{2.191303in}{5.883707in}}%
\pgfpathlineto{\pgfqpoint{2.199893in}{5.884569in}}%
\pgfpathlineto{\pgfqpoint{2.211705in}{5.885594in}}%
\pgfpathlineto{\pgfqpoint{2.222444in}{5.886578in}}%
\pgfpathlineto{\pgfqpoint{2.236404in}{5.887736in}}%
\pgfpathlineto{\pgfqpoint{2.251437in}{5.888744in}}%
\pgfpathlineto{\pgfqpoint{2.278283in}{5.889860in}}%
\pgfpathlineto{\pgfqpoint{2.362043in}{5.897249in}}%
\pgfpathlineto{\pgfqpoint{2.372781in}{5.898556in}}%
\pgfpathlineto{\pgfqpoint{2.384593in}{5.899652in}}%
\pgfpathlineto{\pgfqpoint{2.395332in}{5.900922in}}%
\pgfpathlineto{\pgfqpoint{2.407144in}{5.901994in}}%
\pgfpathlineto{\pgfqpoint{2.417882in}{5.903101in}}%
\pgfpathlineto{\pgfqpoint{2.429695in}{5.904209in}}%
\pgfpathlineto{\pgfqpoint{2.447950in}{5.906300in}}%
\pgfpathlineto{\pgfqpoint{2.458688in}{5.907177in}}%
\pgfpathlineto{\pgfqpoint{2.470501in}{5.908868in}}%
\pgfpathlineto{\pgfqpoint{2.481239in}{5.909937in}}%
\pgfpathlineto{\pgfqpoint{2.493051in}{5.911536in}}%
\pgfpathlineto{\pgfqpoint{2.503790in}{5.912616in}}%
\pgfpathlineto{\pgfqpoint{2.515602in}{5.914234in}}%
\pgfpathlineto{\pgfqpoint{2.527414in}{5.915269in}}%
\pgfpathlineto{\pgfqpoint{2.538152in}{5.916607in}}%
\pgfpathlineto{\pgfqpoint{2.549965in}{5.917722in}}%
\pgfpathlineto{\pgfqpoint{2.560703in}{5.918863in}}%
\pgfpathlineto{\pgfqpoint{2.573589in}{5.919948in}}%
\pgfpathlineto{\pgfqpoint{2.590770in}{5.921520in}}%
\pgfpathlineto{\pgfqpoint{2.602583in}{5.922532in}}%
\pgfpathlineto{\pgfqpoint{2.613321in}{5.923537in}}%
\pgfpathlineto{\pgfqpoint{2.624059in}{5.924445in}}%
\pgfpathlineto{\pgfqpoint{2.635872in}{5.925858in}}%
\pgfpathlineto{\pgfqpoint{2.648758in}{5.926795in}}%
\pgfpathlineto{\pgfqpoint{2.658422in}{5.927859in}}%
\pgfpathlineto{\pgfqpoint{2.670234in}{5.928769in}}%
\pgfpathlineto{\pgfqpoint{2.680973in}{5.929917in}}%
\pgfpathlineto{\pgfqpoint{2.692785in}{5.930962in}}%
\pgfpathlineto{\pgfqpoint{2.703523in}{5.932151in}}%
\pgfpathlineto{\pgfqpoint{2.714262in}{5.933230in}}%
\pgfpathlineto{\pgfqpoint{2.726074in}{5.934886in}}%
\pgfpathlineto{\pgfqpoint{2.736812in}{5.935999in}}%
\pgfpathlineto{\pgfqpoint{2.748625in}{5.937617in}}%
\pgfpathlineto{\pgfqpoint{2.761511in}{5.938843in}}%
\pgfpathlineto{\pgfqpoint{2.771175in}{5.940144in}}%
\pgfpathlineto{\pgfqpoint{2.781914in}{5.941311in}}%
\pgfpathlineto{\pgfqpoint{2.793726in}{5.943022in}}%
\pgfpathlineto{\pgfqpoint{2.805538in}{5.944211in}}%
\pgfpathlineto{\pgfqpoint{2.816277in}{5.945921in}}%
\pgfpathlineto{\pgfqpoint{2.822720in}{5.946812in}}%
\pgfpathlineto{\pgfqpoint{2.831310in}{5.948198in}}%
\pgfpathlineto{\pgfqpoint{2.837753in}{5.949084in}}%
\pgfpathlineto{\pgfqpoint{2.842049in}{5.949518in}}%
\pgfpathlineto{\pgfqpoint{2.858156in}{5.951712in}}%
\pgfpathlineto{\pgfqpoint{2.867821in}{5.953149in}}%
\pgfpathlineto{\pgfqpoint{2.876411in}{5.954333in}}%
\pgfpathlineto{\pgfqpoint{2.887150in}{5.955549in}}%
\pgfpathlineto{\pgfqpoint{2.898962in}{5.957311in}}%
\pgfpathlineto{\pgfqpoint{2.910774in}{5.958461in}}%
\pgfpathlineto{\pgfqpoint{2.921513in}{5.960092in}}%
\pgfpathlineto{\pgfqpoint{2.932251in}{5.961359in}}%
\pgfpathlineto{\pgfqpoint{2.944063in}{5.963272in}}%
\pgfpathlineto{\pgfqpoint{2.950506in}{5.964156in}}%
\pgfpathlineto{\pgfqpoint{2.959097in}{5.965505in}}%
\pgfpathlineto{\pgfqpoint{2.964466in}{5.966274in}}%
\pgfpathlineto{\pgfqpoint{2.974131in}{5.968070in}}%
\pgfpathlineto{\pgfqpoint{2.980574in}{5.969064in}}%
\pgfpathlineto{\pgfqpoint{2.989165in}{5.970527in}}%
\pgfpathlineto{\pgfqpoint{2.999903in}{5.971799in}}%
\pgfpathlineto{\pgfqpoint{3.011715in}{5.974235in}}%
\pgfpathlineto{\pgfqpoint{3.018158in}{5.975299in}}%
\pgfpathlineto{\pgfqpoint{3.026749in}{5.976862in}}%
\pgfpathlineto{\pgfqpoint{3.033192in}{5.977630in}}%
\pgfpathlineto{\pgfqpoint{3.034266in}{5.977885in}}%
\pgfpathlineto{\pgfqpoint{3.034266in}{5.977885in}}%
\pgfusepath{stroke}%
\end{pgfscope}%
\begin{pgfscope}%
\pgfsetrectcap%
\pgfsetmiterjoin%
\pgfsetlinewidth{0.803000pt}%
\definecolor{currentstroke}{rgb}{1.000000,1.000000,1.000000}%
\pgfsetstrokecolor{currentstroke}%
\pgfsetdash{}{0pt}%
\pgfpathmoveto{\pgfqpoint{0.568357in}{5.688815in}}%
\pgfpathlineto{\pgfqpoint{0.568357in}{6.425400in}}%
\pgfusepath{stroke}%
\end{pgfscope}%
\begin{pgfscope}%
\pgfsetrectcap%
\pgfsetmiterjoin%
\pgfsetlinewidth{0.803000pt}%
\definecolor{currentstroke}{rgb}{1.000000,1.000000,1.000000}%
\pgfsetstrokecolor{currentstroke}%
\pgfsetdash{}{0pt}%
\pgfpathmoveto{\pgfqpoint{3.151690in}{5.688815in}}%
\pgfpathlineto{\pgfqpoint{3.151690in}{6.425400in}}%
\pgfusepath{stroke}%
\end{pgfscope}%
\begin{pgfscope}%
\pgfsetrectcap%
\pgfsetmiterjoin%
\pgfsetlinewidth{0.803000pt}%
\definecolor{currentstroke}{rgb}{1.000000,1.000000,1.000000}%
\pgfsetstrokecolor{currentstroke}%
\pgfsetdash{}{0pt}%
\pgfpathmoveto{\pgfqpoint{0.568357in}{5.688815in}}%
\pgfpathlineto{\pgfqpoint{3.151690in}{5.688815in}}%
\pgfusepath{stroke}%
\end{pgfscope}%
\begin{pgfscope}%
\pgfsetrectcap%
\pgfsetmiterjoin%
\pgfsetlinewidth{0.803000pt}%
\definecolor{currentstroke}{rgb}{1.000000,1.000000,1.000000}%
\pgfsetstrokecolor{currentstroke}%
\pgfsetdash{}{0pt}%
\pgfpathmoveto{\pgfqpoint{0.568357in}{6.425400in}}%
\pgfpathlineto{\pgfqpoint{3.151690in}{6.425400in}}%
\pgfusepath{stroke}%
\end{pgfscope}%
\begin{pgfscope}%
\definecolor{textcolor}{rgb}{0.150000,0.150000,0.150000}%
\pgfsetstrokecolor{textcolor}%
\pgfsetfillcolor{textcolor}%
\pgftext[x=1.860023in,y=6.508734in,,base]{\color{textcolor}\rmfamily\fontsize{16.800000}{20.160000}\selectfont MMM}%
\end{pgfscope}%
\begin{pgfscope}%
\pgfsetbuttcap%
\pgfsetmiterjoin%
\definecolor{currentfill}{rgb}{0.917647,0.917647,0.949020}%
\pgfsetfillcolor{currentfill}%
\pgfsetlinewidth{0.000000pt}%
\definecolor{currentstroke}{rgb}{0.000000,0.000000,0.000000}%
\pgfsetstrokecolor{currentstroke}%
\pgfsetstrokeopacity{0.000000}%
\pgfsetdash{}{0pt}%
\pgfpathmoveto{\pgfqpoint{4.185023in}{5.688815in}}%
\pgfpathlineto{\pgfqpoint{6.768357in}{5.688815in}}%
\pgfpathlineto{\pgfqpoint{6.768357in}{6.425400in}}%
\pgfpathlineto{\pgfqpoint{4.185023in}{6.425400in}}%
\pgfpathclose%
\pgfusepath{fill}%
\end{pgfscope}%
\begin{pgfscope}%
\pgfpathrectangle{\pgfqpoint{4.185023in}{5.688815in}}{\pgfqpoint{2.583333in}{0.736585in}}%
\pgfusepath{clip}%
\pgfsetroundcap%
\pgfsetroundjoin%
\pgfsetlinewidth{0.803000pt}%
\definecolor{currentstroke}{rgb}{1.000000,1.000000,1.000000}%
\pgfsetstrokecolor{currentstroke}%
\pgfsetdash{}{0pt}%
\pgfpathmoveto{\pgfqpoint{4.300300in}{5.688815in}}%
\pgfpathlineto{\pgfqpoint{4.300300in}{6.425400in}}%
\pgfusepath{stroke}%
\end{pgfscope}%
\begin{pgfscope}%
\definecolor{textcolor}{rgb}{0.150000,0.150000,0.150000}%
\pgfsetstrokecolor{textcolor}%
\pgfsetfillcolor{textcolor}%
\pgftext[x=4.300300in,y=5.591593in,,top]{\color{textcolor}\rmfamily\fontsize{14.000000}{16.800000}\selectfont 2012}%
\end{pgfscope}%
\begin{pgfscope}%
\pgfpathrectangle{\pgfqpoint{4.185023in}{5.688815in}}{\pgfqpoint{2.583333in}{0.736585in}}%
\pgfusepath{clip}%
\pgfsetroundcap%
\pgfsetroundjoin%
\pgfsetlinewidth{0.803000pt}%
\definecolor{currentstroke}{rgb}{1.000000,1.000000,1.000000}%
\pgfsetstrokecolor{currentstroke}%
\pgfsetdash{}{0pt}%
\pgfpathmoveto{\pgfqpoint{4.693325in}{5.688815in}}%
\pgfpathlineto{\pgfqpoint{4.693325in}{6.425400in}}%
\pgfusepath{stroke}%
\end{pgfscope}%
\begin{pgfscope}%
\definecolor{textcolor}{rgb}{0.150000,0.150000,0.150000}%
\pgfsetstrokecolor{textcolor}%
\pgfsetfillcolor{textcolor}%
\pgftext[x=4.693325in,y=5.591593in,,top]{\color{textcolor}\rmfamily\fontsize{14.000000}{16.800000}\selectfont 2013}%
\end{pgfscope}%
\begin{pgfscope}%
\pgfpathrectangle{\pgfqpoint{4.185023in}{5.688815in}}{\pgfqpoint{2.583333in}{0.736585in}}%
\pgfusepath{clip}%
\pgfsetroundcap%
\pgfsetroundjoin%
\pgfsetlinewidth{0.803000pt}%
\definecolor{currentstroke}{rgb}{1.000000,1.000000,1.000000}%
\pgfsetstrokecolor{currentstroke}%
\pgfsetdash{}{0pt}%
\pgfpathmoveto{\pgfqpoint{5.085276in}{5.688815in}}%
\pgfpathlineto{\pgfqpoint{5.085276in}{6.425400in}}%
\pgfusepath{stroke}%
\end{pgfscope}%
\begin{pgfscope}%
\definecolor{textcolor}{rgb}{0.150000,0.150000,0.150000}%
\pgfsetstrokecolor{textcolor}%
\pgfsetfillcolor{textcolor}%
\pgftext[x=5.085276in,y=5.591593in,,top]{\color{textcolor}\rmfamily\fontsize{14.000000}{16.800000}\selectfont 2014}%
\end{pgfscope}%
\begin{pgfscope}%
\pgfpathrectangle{\pgfqpoint{4.185023in}{5.688815in}}{\pgfqpoint{2.583333in}{0.736585in}}%
\pgfusepath{clip}%
\pgfsetroundcap%
\pgfsetroundjoin%
\pgfsetlinewidth{0.803000pt}%
\definecolor{currentstroke}{rgb}{1.000000,1.000000,1.000000}%
\pgfsetstrokecolor{currentstroke}%
\pgfsetdash{}{0pt}%
\pgfpathmoveto{\pgfqpoint{5.477227in}{5.688815in}}%
\pgfpathlineto{\pgfqpoint{5.477227in}{6.425400in}}%
\pgfusepath{stroke}%
\end{pgfscope}%
\begin{pgfscope}%
\definecolor{textcolor}{rgb}{0.150000,0.150000,0.150000}%
\pgfsetstrokecolor{textcolor}%
\pgfsetfillcolor{textcolor}%
\pgftext[x=5.477227in,y=5.591593in,,top]{\color{textcolor}\rmfamily\fontsize{14.000000}{16.800000}\selectfont 2015}%
\end{pgfscope}%
\begin{pgfscope}%
\pgfpathrectangle{\pgfqpoint{4.185023in}{5.688815in}}{\pgfqpoint{2.583333in}{0.736585in}}%
\pgfusepath{clip}%
\pgfsetroundcap%
\pgfsetroundjoin%
\pgfsetlinewidth{0.803000pt}%
\definecolor{currentstroke}{rgb}{1.000000,1.000000,1.000000}%
\pgfsetstrokecolor{currentstroke}%
\pgfsetdash{}{0pt}%
\pgfpathmoveto{\pgfqpoint{5.869178in}{5.688815in}}%
\pgfpathlineto{\pgfqpoint{5.869178in}{6.425400in}}%
\pgfusepath{stroke}%
\end{pgfscope}%
\begin{pgfscope}%
\definecolor{textcolor}{rgb}{0.150000,0.150000,0.150000}%
\pgfsetstrokecolor{textcolor}%
\pgfsetfillcolor{textcolor}%
\pgftext[x=5.869178in,y=5.591593in,,top]{\color{textcolor}\rmfamily\fontsize{14.000000}{16.800000}\selectfont 2016}%
\end{pgfscope}%
\begin{pgfscope}%
\pgfpathrectangle{\pgfqpoint{4.185023in}{5.688815in}}{\pgfqpoint{2.583333in}{0.736585in}}%
\pgfusepath{clip}%
\pgfsetroundcap%
\pgfsetroundjoin%
\pgfsetlinewidth{0.803000pt}%
\definecolor{currentstroke}{rgb}{1.000000,1.000000,1.000000}%
\pgfsetstrokecolor{currentstroke}%
\pgfsetdash{}{0pt}%
\pgfpathmoveto{\pgfqpoint{6.262203in}{5.688815in}}%
\pgfpathlineto{\pgfqpoint{6.262203in}{6.425400in}}%
\pgfusepath{stroke}%
\end{pgfscope}%
\begin{pgfscope}%
\definecolor{textcolor}{rgb}{0.150000,0.150000,0.150000}%
\pgfsetstrokecolor{textcolor}%
\pgfsetfillcolor{textcolor}%
\pgftext[x=6.262203in,y=5.591593in,,top]{\color{textcolor}\rmfamily\fontsize{14.000000}{16.800000}\selectfont 2017}%
\end{pgfscope}%
\begin{pgfscope}%
\pgfpathrectangle{\pgfqpoint{4.185023in}{5.688815in}}{\pgfqpoint{2.583333in}{0.736585in}}%
\pgfusepath{clip}%
\pgfsetroundcap%
\pgfsetroundjoin%
\pgfsetlinewidth{0.803000pt}%
\definecolor{currentstroke}{rgb}{1.000000,1.000000,1.000000}%
\pgfsetstrokecolor{currentstroke}%
\pgfsetdash{}{0pt}%
\pgfpathmoveto{\pgfqpoint{6.654154in}{5.688815in}}%
\pgfpathlineto{\pgfqpoint{6.654154in}{6.425400in}}%
\pgfusepath{stroke}%
\end{pgfscope}%
\begin{pgfscope}%
\definecolor{textcolor}{rgb}{0.150000,0.150000,0.150000}%
\pgfsetstrokecolor{textcolor}%
\pgfsetfillcolor{textcolor}%
\pgftext[x=6.654154in,y=5.591593in,,top]{\color{textcolor}\rmfamily\fontsize{14.000000}{16.800000}\selectfont 2018}%
\end{pgfscope}%
\begin{pgfscope}%
\pgfpathrectangle{\pgfqpoint{4.185023in}{5.688815in}}{\pgfqpoint{2.583333in}{0.736585in}}%
\pgfusepath{clip}%
\pgfsetroundcap%
\pgfsetroundjoin%
\pgfsetlinewidth{0.803000pt}%
\definecolor{currentstroke}{rgb}{1.000000,1.000000,1.000000}%
\pgfsetstrokecolor{currentstroke}%
\pgfsetdash{}{0pt}%
\pgfpathmoveto{\pgfqpoint{4.185023in}{5.805720in}}%
\pgfpathlineto{\pgfqpoint{6.768357in}{5.805720in}}%
\pgfusepath{stroke}%
\end{pgfscope}%
\begin{pgfscope}%
\definecolor{textcolor}{rgb}{0.150000,0.150000,0.150000}%
\pgfsetstrokecolor{textcolor}%
\pgfsetfillcolor{textcolor}%
\pgftext[x=3.840378in,y=5.731854in,left,base]{\color{textcolor}\rmfamily\fontsize{14.000000}{16.800000}\selectfont 50}%
\end{pgfscope}%
\begin{pgfscope}%
\pgfpathrectangle{\pgfqpoint{4.185023in}{5.688815in}}{\pgfqpoint{2.583333in}{0.736585in}}%
\pgfusepath{clip}%
\pgfsetroundcap%
\pgfsetroundjoin%
\pgfsetlinewidth{0.803000pt}%
\definecolor{currentstroke}{rgb}{1.000000,1.000000,1.000000}%
\pgfsetstrokecolor{currentstroke}%
\pgfsetdash{}{0pt}%
\pgfpathmoveto{\pgfqpoint{4.185023in}{6.116075in}}%
\pgfpathlineto{\pgfqpoint{6.768357in}{6.116075in}}%
\pgfusepath{stroke}%
\end{pgfscope}%
\begin{pgfscope}%
\definecolor{textcolor}{rgb}{0.150000,0.150000,0.150000}%
\pgfsetstrokecolor{textcolor}%
\pgfsetfillcolor{textcolor}%
\pgftext[x=3.840378in,y=6.042209in,left,base]{\color{textcolor}\rmfamily\fontsize{14.000000}{16.800000}\selectfont 75}%
\end{pgfscope}%
\begin{pgfscope}%
\pgfpathrectangle{\pgfqpoint{4.185023in}{5.688815in}}{\pgfqpoint{2.583333in}{0.736585in}}%
\pgfusepath{clip}%
\pgfsetroundcap%
\pgfsetroundjoin%
\pgfsetlinewidth{1.505625pt}%
\definecolor{currentstroke}{rgb}{1.000000,0.498039,0.054902}%
\pgfsetstrokecolor{currentstroke}%
\pgfsetdash{}{0pt}%
\pgfpathmoveto{\pgfqpoint{4.302448in}{5.722296in}}%
\pgfpathlineto{\pgfqpoint{4.303521in}{5.722668in}}%
\pgfpathlineto{\pgfqpoint{4.304595in}{5.728876in}}%
\pgfpathlineto{\pgfqpoint{4.305669in}{5.722917in}}%
\pgfpathlineto{\pgfqpoint{4.308891in}{5.724282in}}%
\pgfpathlineto{\pgfqpoint{4.311038in}{5.730489in}}%
\pgfpathlineto{\pgfqpoint{4.312112in}{5.738310in}}%
\pgfpathlineto{\pgfqpoint{4.317481in}{5.744642in}}%
\pgfpathlineto{\pgfqpoint{4.319629in}{5.752835in}}%
\pgfpathlineto{\pgfqpoint{4.320703in}{5.742655in}}%
\pgfpathlineto{\pgfqpoint{4.324998in}{5.733593in}}%
\pgfpathlineto{\pgfqpoint{4.326072in}{5.744145in}}%
\pgfpathlineto{\pgfqpoint{4.331441in}{5.732476in}}%
\pgfpathlineto{\pgfqpoint{4.332515in}{5.743773in}}%
\pgfpathlineto{\pgfqpoint{4.334663in}{5.755194in}}%
\pgfpathlineto{\pgfqpoint{4.335737in}{5.767360in}}%
\pgfpathlineto{\pgfqpoint{4.338958in}{5.762394in}}%
\pgfpathlineto{\pgfqpoint{4.340032in}{5.765994in}}%
\pgfpathlineto{\pgfqpoint{4.341106in}{5.760532in}}%
\pgfpathlineto{\pgfqpoint{4.342180in}{5.767980in}}%
\pgfpathlineto{\pgfqpoint{4.343253in}{5.762394in}}%
\pgfpathlineto{\pgfqpoint{4.346475in}{5.765249in}}%
\pgfpathlineto{\pgfqpoint{4.347549in}{5.764132in}}%
\pgfpathlineto{\pgfqpoint{4.348623in}{5.759166in}}%
\pgfpathlineto{\pgfqpoint{4.349696in}{5.774188in}}%
\pgfpathlineto{\pgfqpoint{4.356140in}{5.773319in}}%
\pgfpathlineto{\pgfqpoint{4.357213in}{5.771953in}}%
\pgfpathlineto{\pgfqpoint{4.358287in}{5.779401in}}%
\pgfpathlineto{\pgfqpoint{4.361509in}{5.788836in}}%
\pgfpathlineto{\pgfqpoint{4.362583in}{5.784119in}}%
\pgfpathlineto{\pgfqpoint{4.363656in}{5.774436in}}%
\pgfpathlineto{\pgfqpoint{4.364730in}{5.782008in}}%
\pgfpathlineto{\pgfqpoint{4.365804in}{5.775553in}}%
\pgfpathlineto{\pgfqpoint{4.369026in}{5.775305in}}%
\pgfpathlineto{\pgfqpoint{4.370099in}{5.761649in}}%
\pgfpathlineto{\pgfqpoint{4.373321in}{5.777912in}}%
\pgfpathlineto{\pgfqpoint{4.376542in}{5.773070in}}%
\pgfpathlineto{\pgfqpoint{4.379764in}{5.817141in}}%
\pgfpathlineto{\pgfqpoint{4.380838in}{5.815279in}}%
\pgfpathlineto{\pgfqpoint{4.384059in}{5.823224in}}%
\pgfpathlineto{\pgfqpoint{4.385133in}{5.819251in}}%
\pgfpathlineto{\pgfqpoint{4.386207in}{5.820741in}}%
\pgfpathlineto{\pgfqpoint{4.387281in}{5.823472in}}%
\pgfpathlineto{\pgfqpoint{4.388355in}{5.822975in}}%
\pgfpathlineto{\pgfqpoint{4.391576in}{5.838741in}}%
\pgfpathlineto{\pgfqpoint{4.392650in}{5.833900in}}%
\pgfpathlineto{\pgfqpoint{4.393724in}{5.843211in}}%
\pgfpathlineto{\pgfqpoint{4.394798in}{5.830176in}}%
\pgfpathlineto{\pgfqpoint{4.395872in}{5.829803in}}%
\pgfpathlineto{\pgfqpoint{4.399093in}{5.831541in}}%
\pgfpathlineto{\pgfqpoint{4.400167in}{5.837997in}}%
\pgfpathlineto{\pgfqpoint{4.401241in}{5.826576in}}%
\pgfpathlineto{\pgfqpoint{4.402315in}{5.834769in}}%
\pgfpathlineto{\pgfqpoint{4.406610in}{5.824217in}}%
\pgfpathlineto{\pgfqpoint{4.407684in}{5.813292in}}%
\pgfpathlineto{\pgfqpoint{4.409831in}{5.834024in}}%
\pgfpathlineto{\pgfqpoint{4.410905in}{5.825582in}}%
\pgfpathlineto{\pgfqpoint{4.414127in}{5.831665in}}%
\pgfpathlineto{\pgfqpoint{4.415201in}{5.835638in}}%
\pgfpathlineto{\pgfqpoint{4.416274in}{5.834024in}}%
\pgfpathlineto{\pgfqpoint{4.417348in}{5.828810in}}%
\pgfpathlineto{\pgfqpoint{4.418422in}{5.827445in}}%
\pgfpathlineto{\pgfqpoint{4.421644in}{5.826079in}}%
\pgfpathlineto{\pgfqpoint{4.422717in}{5.829431in}}%
\pgfpathlineto{\pgfqpoint{4.424865in}{5.851404in}}%
\pgfpathlineto{\pgfqpoint{4.425939in}{5.857859in}}%
\pgfpathlineto{\pgfqpoint{4.429161in}{5.858356in}}%
\pgfpathlineto{\pgfqpoint{4.430234in}{5.866798in}}%
\pgfpathlineto{\pgfqpoint{4.431308in}{5.867791in}}%
\pgfpathlineto{\pgfqpoint{4.432382in}{5.865556in}}%
\pgfpathlineto{\pgfqpoint{4.433456in}{5.857115in}}%
\pgfpathlineto{\pgfqpoint{4.436677in}{5.857115in}}%
\pgfpathlineto{\pgfqpoint{4.437751in}{5.854756in}}%
\pgfpathlineto{\pgfqpoint{4.438825in}{5.849914in}}%
\pgfpathlineto{\pgfqpoint{4.439899in}{5.849542in}}%
\pgfpathlineto{\pgfqpoint{4.440973in}{5.852025in}}%
\pgfpathlineto{\pgfqpoint{4.445268in}{5.832907in}}%
\pgfpathlineto{\pgfqpoint{4.446342in}{5.826824in}}%
\pgfpathlineto{\pgfqpoint{4.447416in}{5.807333in}}%
\pgfpathlineto{\pgfqpoint{4.448490in}{5.804602in}}%
\pgfpathlineto{\pgfqpoint{4.451711in}{5.814906in}}%
\pgfpathlineto{\pgfqpoint{4.452785in}{5.815154in}}%
\pgfpathlineto{\pgfqpoint{4.453859in}{5.810934in}}%
\pgfpathlineto{\pgfqpoint{4.454933in}{5.815030in}}%
\pgfpathlineto{\pgfqpoint{4.456006in}{5.809071in}}%
\pgfpathlineto{\pgfqpoint{4.460302in}{5.817513in}}%
\pgfpathlineto{\pgfqpoint{4.461376in}{5.805223in}}%
\pgfpathlineto{\pgfqpoint{4.462450in}{5.809320in}}%
\pgfpathlineto{\pgfqpoint{4.463523in}{5.782505in}}%
\pgfpathlineto{\pgfqpoint{4.466745in}{5.783374in}}%
\pgfpathlineto{\pgfqpoint{4.467819in}{5.788712in}}%
\pgfpathlineto{\pgfqpoint{4.468893in}{5.804354in}}%
\pgfpathlineto{\pgfqpoint{4.469966in}{5.802740in}}%
\pgfpathlineto{\pgfqpoint{4.471040in}{5.809692in}}%
\pgfpathlineto{\pgfqpoint{4.474262in}{5.801251in}}%
\pgfpathlineto{\pgfqpoint{4.475336in}{5.816644in}}%
\pgfpathlineto{\pgfqpoint{4.476409in}{5.801251in}}%
\pgfpathlineto{\pgfqpoint{4.477483in}{5.800630in}}%
\pgfpathlineto{\pgfqpoint{4.478557in}{5.814410in}}%
\pgfpathlineto{\pgfqpoint{4.481779in}{5.809568in}}%
\pgfpathlineto{\pgfqpoint{4.482852in}{5.821734in}}%
\pgfpathlineto{\pgfqpoint{4.483926in}{5.827320in}}%
\pgfpathlineto{\pgfqpoint{4.485000in}{5.814782in}}%
\pgfpathlineto{\pgfqpoint{4.486074in}{5.820120in}}%
\pgfpathlineto{\pgfqpoint{4.489295in}{5.811803in}}%
\pgfpathlineto{\pgfqpoint{4.490369in}{5.812547in}}%
\pgfpathlineto{\pgfqpoint{4.491443in}{5.821113in}}%
\pgfpathlineto{\pgfqpoint{4.492517in}{5.819003in}}%
\pgfpathlineto{\pgfqpoint{4.493591in}{5.836010in}}%
\pgfpathlineto{\pgfqpoint{4.496812in}{5.845321in}}%
\pgfpathlineto{\pgfqpoint{4.497886in}{5.851652in}}%
\pgfpathlineto{\pgfqpoint{4.500034in}{5.849914in}}%
\pgfpathlineto{\pgfqpoint{4.501108in}{5.842838in}}%
\pgfpathlineto{\pgfqpoint{4.505403in}{5.840231in}}%
\pgfpathlineto{\pgfqpoint{4.506477in}{5.836755in}}%
\pgfpathlineto{\pgfqpoint{4.507551in}{5.823844in}}%
\pgfpathlineto{\pgfqpoint{4.508625in}{5.835017in}}%
\pgfpathlineto{\pgfqpoint{4.511846in}{5.842962in}}%
\pgfpathlineto{\pgfqpoint{4.512920in}{5.843459in}}%
\pgfpathlineto{\pgfqpoint{4.513994in}{5.839114in}}%
\pgfpathlineto{\pgfqpoint{4.515068in}{5.816023in}}%
\pgfpathlineto{\pgfqpoint{4.516141in}{5.811306in}}%
\pgfpathlineto{\pgfqpoint{4.519363in}{5.810313in}}%
\pgfpathlineto{\pgfqpoint{4.520437in}{5.809196in}}%
\pgfpathlineto{\pgfqpoint{4.521511in}{5.813913in}}%
\pgfpathlineto{\pgfqpoint{4.522584in}{5.833155in}}%
\pgfpathlineto{\pgfqpoint{4.523658in}{5.841721in}}%
\pgfpathlineto{\pgfqpoint{4.526880in}{5.839114in}}%
\pgfpathlineto{\pgfqpoint{4.527954in}{5.832534in}}%
\pgfpathlineto{\pgfqpoint{4.529028in}{5.822355in}}%
\pgfpathlineto{\pgfqpoint{4.530101in}{5.818879in}}%
\pgfpathlineto{\pgfqpoint{4.531175in}{5.831417in}}%
\pgfpathlineto{\pgfqpoint{4.534397in}{5.825707in}}%
\pgfpathlineto{\pgfqpoint{4.536544in}{5.834769in}}%
\pgfpathlineto{\pgfqpoint{4.537618in}{5.818630in}}%
\pgfpathlineto{\pgfqpoint{4.538692in}{5.811678in}}%
\pgfpathlineto{\pgfqpoint{4.541914in}{5.814782in}}%
\pgfpathlineto{\pgfqpoint{4.542987in}{5.814285in}}%
\pgfpathlineto{\pgfqpoint{4.546209in}{5.831169in}}%
\pgfpathlineto{\pgfqpoint{4.550504in}{5.820120in}}%
\pgfpathlineto{\pgfqpoint{4.551578in}{5.822603in}}%
\pgfpathlineto{\pgfqpoint{4.552652in}{5.818134in}}%
\pgfpathlineto{\pgfqpoint{4.553726in}{5.830051in}}%
\pgfpathlineto{\pgfqpoint{4.556947in}{5.829307in}}%
\pgfpathlineto{\pgfqpoint{4.558021in}{5.830920in}}%
\pgfpathlineto{\pgfqpoint{4.559095in}{5.829307in}}%
\pgfpathlineto{\pgfqpoint{4.560169in}{5.826576in}}%
\pgfpathlineto{\pgfqpoint{4.561243in}{5.839238in}}%
\pgfpathlineto{\pgfqpoint{4.565538in}{5.842714in}}%
\pgfpathlineto{\pgfqpoint{4.566612in}{5.826700in}}%
\pgfpathlineto{\pgfqpoint{4.568760in}{5.832783in}}%
\pgfpathlineto{\pgfqpoint{4.571981in}{5.830424in}}%
\pgfpathlineto{\pgfqpoint{4.573055in}{5.827196in}}%
\pgfpathlineto{\pgfqpoint{4.574129in}{5.827693in}}%
\pgfpathlineto{\pgfqpoint{4.575203in}{5.847556in}}%
\pgfpathlineto{\pgfqpoint{4.576276in}{5.850038in}}%
\pgfpathlineto{\pgfqpoint{4.579498in}{5.849045in}}%
\pgfpathlineto{\pgfqpoint{4.580572in}{5.843335in}}%
\pgfpathlineto{\pgfqpoint{4.581646in}{5.843335in}}%
\pgfpathlineto{\pgfqpoint{4.583793in}{5.834272in}}%
\pgfpathlineto{\pgfqpoint{4.587015in}{5.832162in}}%
\pgfpathlineto{\pgfqpoint{4.588089in}{5.826079in}}%
\pgfpathlineto{\pgfqpoint{4.589162in}{5.814906in}}%
\pgfpathlineto{\pgfqpoint{4.591310in}{5.822975in}}%
\pgfpathlineto{\pgfqpoint{4.594532in}{5.832783in}}%
\pgfpathlineto{\pgfqpoint{4.595605in}{5.826576in}}%
\pgfpathlineto{\pgfqpoint{4.596679in}{5.831417in}}%
\pgfpathlineto{\pgfqpoint{4.597753in}{5.842217in}}%
\pgfpathlineto{\pgfqpoint{4.598827in}{5.844452in}}%
\pgfpathlineto{\pgfqpoint{4.602049in}{5.847307in}}%
\pgfpathlineto{\pgfqpoint{4.604196in}{5.837748in}}%
\pgfpathlineto{\pgfqpoint{4.605270in}{5.843459in}}%
\pgfpathlineto{\pgfqpoint{4.606344in}{5.836879in}}%
\pgfpathlineto{\pgfqpoint{4.609565in}{5.833527in}}%
\pgfpathlineto{\pgfqpoint{4.611713in}{5.853514in}}%
\pgfpathlineto{\pgfqpoint{4.612787in}{5.833776in}}%
\pgfpathlineto{\pgfqpoint{4.613861in}{5.825334in}}%
\pgfpathlineto{\pgfqpoint{4.617082in}{5.823969in}}%
\pgfpathlineto{\pgfqpoint{4.618156in}{5.808575in}}%
\pgfpathlineto{\pgfqpoint{4.619230in}{5.806961in}}%
\pgfpathlineto{\pgfqpoint{4.621378in}{5.812796in}}%
\pgfpathlineto{\pgfqpoint{4.626747in}{5.815279in}}%
\pgfpathlineto{\pgfqpoint{4.627821in}{5.825210in}}%
\pgfpathlineto{\pgfqpoint{4.632116in}{5.819624in}}%
\pgfpathlineto{\pgfqpoint{4.633190in}{5.828934in}}%
\pgfpathlineto{\pgfqpoint{4.634264in}{5.810809in}}%
\pgfpathlineto{\pgfqpoint{4.635338in}{5.810685in}}%
\pgfpathlineto{\pgfqpoint{4.636411in}{5.813665in}}%
\pgfpathlineto{\pgfqpoint{4.639633in}{5.810189in}}%
\pgfpathlineto{\pgfqpoint{4.641781in}{5.789085in}}%
\pgfpathlineto{\pgfqpoint{4.642854in}{5.788960in}}%
\pgfpathlineto{\pgfqpoint{4.643928in}{5.796409in}}%
\pgfpathlineto{\pgfqpoint{4.647150in}{5.806961in}}%
\pgfpathlineto{\pgfqpoint{4.648224in}{5.814037in}}%
\pgfpathlineto{\pgfqpoint{4.649297in}{5.815403in}}%
\pgfpathlineto{\pgfqpoint{4.651445in}{5.821362in}}%
\pgfpathlineto{\pgfqpoint{4.654667in}{5.812051in}}%
\pgfpathlineto{\pgfqpoint{4.655740in}{5.798023in}}%
\pgfpathlineto{\pgfqpoint{4.656814in}{5.810065in}}%
\pgfpathlineto{\pgfqpoint{4.657888in}{5.814658in}}%
\pgfpathlineto{\pgfqpoint{4.658962in}{5.814410in}}%
\pgfpathlineto{\pgfqpoint{4.662183in}{5.815527in}}%
\pgfpathlineto{\pgfqpoint{4.663257in}{5.813789in}}%
\pgfpathlineto{\pgfqpoint{4.664331in}{5.820120in}}%
\pgfpathlineto{\pgfqpoint{4.665405in}{5.816892in}}%
\pgfpathlineto{\pgfqpoint{4.666479in}{5.822479in}}%
\pgfpathlineto{\pgfqpoint{4.669700in}{5.824093in}}%
\pgfpathlineto{\pgfqpoint{4.670774in}{5.827569in}}%
\pgfpathlineto{\pgfqpoint{4.671848in}{5.834396in}}%
\pgfpathlineto{\pgfqpoint{4.672922in}{5.835017in}}%
\pgfpathlineto{\pgfqpoint{4.673996in}{5.822851in}}%
\pgfpathlineto{\pgfqpoint{4.677217in}{5.829927in}}%
\pgfpathlineto{\pgfqpoint{4.678291in}{5.836134in}}%
\pgfpathlineto{\pgfqpoint{4.679365in}{5.824465in}}%
\pgfpathlineto{\pgfqpoint{4.680439in}{5.831417in}}%
\pgfpathlineto{\pgfqpoint{4.681513in}{5.834148in}}%
\pgfpathlineto{\pgfqpoint{4.684734in}{5.832783in}}%
\pgfpathlineto{\pgfqpoint{4.686882in}{5.828314in}}%
\pgfpathlineto{\pgfqpoint{4.687956in}{5.823348in}}%
\pgfpathlineto{\pgfqpoint{4.689029in}{5.823100in}}%
\pgfpathlineto{\pgfqpoint{4.692251in}{5.832286in}}%
\pgfpathlineto{\pgfqpoint{4.694399in}{5.848921in}}%
\pgfpathlineto{\pgfqpoint{4.695472in}{5.851652in}}%
\pgfpathlineto{\pgfqpoint{4.696546in}{5.858604in}}%
\pgfpathlineto{\pgfqpoint{4.699768in}{5.861459in}}%
\pgfpathlineto{\pgfqpoint{4.700842in}{5.865308in}}%
\pgfpathlineto{\pgfqpoint{4.701916in}{5.865929in}}%
\pgfpathlineto{\pgfqpoint{4.704063in}{5.876977in}}%
\pgfpathlineto{\pgfqpoint{4.707285in}{5.876605in}}%
\pgfpathlineto{\pgfqpoint{4.708359in}{5.871639in}}%
\pgfpathlineto{\pgfqpoint{4.709432in}{5.870025in}}%
\pgfpathlineto{\pgfqpoint{4.710506in}{5.871391in}}%
\pgfpathlineto{\pgfqpoint{4.711580in}{5.860466in}}%
\pgfpathlineto{\pgfqpoint{4.715875in}{5.856246in}}%
\pgfpathlineto{\pgfqpoint{4.716949in}{5.850907in}}%
\pgfpathlineto{\pgfqpoint{4.719097in}{5.857363in}}%
\pgfpathlineto{\pgfqpoint{4.722318in}{5.852521in}}%
\pgfpathlineto{\pgfqpoint{4.723392in}{5.856742in}}%
\pgfpathlineto{\pgfqpoint{4.724466in}{5.855377in}}%
\pgfpathlineto{\pgfqpoint{4.725540in}{5.849542in}}%
\pgfpathlineto{\pgfqpoint{4.726614in}{5.861956in}}%
\pgfpathlineto{\pgfqpoint{4.729835in}{5.856370in}}%
\pgfpathlineto{\pgfqpoint{4.730909in}{5.870398in}}%
\pgfpathlineto{\pgfqpoint{4.731983in}{5.869280in}}%
\pgfpathlineto{\pgfqpoint{4.733057in}{5.887157in}}%
\pgfpathlineto{\pgfqpoint{4.734131in}{5.883309in}}%
\pgfpathlineto{\pgfqpoint{4.737352in}{5.885419in}}%
\pgfpathlineto{\pgfqpoint{4.738426in}{5.887902in}}%
\pgfpathlineto{\pgfqpoint{4.739500in}{5.886660in}}%
\pgfpathlineto{\pgfqpoint{4.740574in}{5.889391in}}%
\pgfpathlineto{\pgfqpoint{4.741648in}{5.882067in}}%
\pgfpathlineto{\pgfqpoint{4.745943in}{5.888771in}}%
\pgfpathlineto{\pgfqpoint{4.748091in}{5.880453in}}%
\pgfpathlineto{\pgfqpoint{4.749164in}{5.891998in}}%
\pgfpathlineto{\pgfqpoint{4.753460in}{5.884674in}}%
\pgfpathlineto{\pgfqpoint{4.754534in}{5.891254in}}%
\pgfpathlineto{\pgfqpoint{4.755607in}{5.887281in}}%
\pgfpathlineto{\pgfqpoint{4.756681in}{5.889764in}}%
\pgfpathlineto{\pgfqpoint{4.759903in}{5.895599in}}%
\pgfpathlineto{\pgfqpoint{4.760977in}{5.909503in}}%
\pgfpathlineto{\pgfqpoint{4.762050in}{5.915585in}}%
\pgfpathlineto{\pgfqpoint{4.763124in}{5.914468in}}%
\pgfpathlineto{\pgfqpoint{4.764198in}{5.916082in}}%
\pgfpathlineto{\pgfqpoint{4.767420in}{5.925144in}}%
\pgfpathlineto{\pgfqpoint{4.768493in}{5.922786in}}%
\pgfpathlineto{\pgfqpoint{4.769567in}{5.922910in}}%
\pgfpathlineto{\pgfqpoint{4.770641in}{5.923779in}}%
\pgfpathlineto{\pgfqpoint{4.771715in}{5.931848in}}%
\pgfpathlineto{\pgfqpoint{4.774937in}{5.928248in}}%
\pgfpathlineto{\pgfqpoint{4.776010in}{5.920924in}}%
\pgfpathlineto{\pgfqpoint{4.777084in}{5.930731in}}%
\pgfpathlineto{\pgfqpoint{4.778158in}{5.924151in}}%
\pgfpathlineto{\pgfqpoint{4.779232in}{5.933214in}}%
\pgfpathlineto{\pgfqpoint{4.782453in}{5.931972in}}%
\pgfpathlineto{\pgfqpoint{4.783527in}{5.944014in}}%
\pgfpathlineto{\pgfqpoint{4.784601in}{5.943766in}}%
\pgfpathlineto{\pgfqpoint{4.785675in}{5.947242in}}%
\pgfpathlineto{\pgfqpoint{4.789970in}{5.945007in}}%
\pgfpathlineto{\pgfqpoint{4.791044in}{5.949352in}}%
\pgfpathlineto{\pgfqpoint{4.792118in}{5.935821in}}%
\pgfpathlineto{\pgfqpoint{4.793192in}{5.941283in}}%
\pgfpathlineto{\pgfqpoint{4.794266in}{5.925020in}}%
\pgfpathlineto{\pgfqpoint{4.797487in}{5.928496in}}%
\pgfpathlineto{\pgfqpoint{4.798561in}{5.924151in}}%
\pgfpathlineto{\pgfqpoint{4.799635in}{5.926262in}}%
\pgfpathlineto{\pgfqpoint{4.800709in}{5.930110in}}%
\pgfpathlineto{\pgfqpoint{4.801782in}{5.929365in}}%
\pgfpathlineto{\pgfqpoint{4.805004in}{5.911489in}}%
\pgfpathlineto{\pgfqpoint{4.806078in}{5.917075in}}%
\pgfpathlineto{\pgfqpoint{4.807152in}{5.911861in}}%
\pgfpathlineto{\pgfqpoint{4.808226in}{5.922165in}}%
\pgfpathlineto{\pgfqpoint{4.809299in}{5.946993in}}%
\pgfpathlineto{\pgfqpoint{4.812521in}{5.940538in}}%
\pgfpathlineto{\pgfqpoint{4.813595in}{5.949476in}}%
\pgfpathlineto{\pgfqpoint{4.814669in}{5.949104in}}%
\pgfpathlineto{\pgfqpoint{4.815742in}{5.957173in}}%
\pgfpathlineto{\pgfqpoint{4.816816in}{5.952828in}}%
\pgfpathlineto{\pgfqpoint{4.820038in}{5.951463in}}%
\pgfpathlineto{\pgfqpoint{4.821112in}{5.960277in}}%
\pgfpathlineto{\pgfqpoint{4.822185in}{5.958787in}}%
\pgfpathlineto{\pgfqpoint{4.824333in}{5.980884in}}%
\pgfpathlineto{\pgfqpoint{4.827555in}{5.979022in}}%
\pgfpathlineto{\pgfqpoint{4.829702in}{5.981877in}}%
\pgfpathlineto{\pgfqpoint{4.835071in}{5.975919in}}%
\pgfpathlineto{\pgfqpoint{4.837219in}{6.009809in}}%
\pgfpathlineto{\pgfqpoint{4.838293in}{6.003602in}}%
\pgfpathlineto{\pgfqpoint{4.839367in}{6.016017in}}%
\pgfpathlineto{\pgfqpoint{4.842588in}{6.028182in}}%
\pgfpathlineto{\pgfqpoint{4.843662in}{6.036252in}}%
\pgfpathlineto{\pgfqpoint{4.844736in}{6.028679in}}%
\pgfpathlineto{\pgfqpoint{4.845810in}{6.031534in}}%
\pgfpathlineto{\pgfqpoint{4.846884in}{6.038114in}}%
\pgfpathlineto{\pgfqpoint{4.851179in}{6.048169in}}%
\pgfpathlineto{\pgfqpoint{4.852253in}{6.044445in}}%
\pgfpathlineto{\pgfqpoint{4.853327in}{6.047921in}}%
\pgfpathlineto{\pgfqpoint{4.854401in}{6.043080in}}%
\pgfpathlineto{\pgfqpoint{4.857622in}{6.051645in}}%
\pgfpathlineto{\pgfqpoint{4.858696in}{6.047052in}}%
\pgfpathlineto{\pgfqpoint{4.859770in}{6.032279in}}%
\pgfpathlineto{\pgfqpoint{4.861917in}{6.069398in}}%
\pgfpathlineto{\pgfqpoint{4.865139in}{6.072129in}}%
\pgfpathlineto{\pgfqpoint{4.867287in}{6.031783in}}%
\pgfpathlineto{\pgfqpoint{4.868360in}{6.037369in}}%
\pgfpathlineto{\pgfqpoint{4.869434in}{6.012044in}}%
\pgfpathlineto{\pgfqpoint{4.872656in}{6.021851in}}%
\pgfpathlineto{\pgfqpoint{4.873730in}{6.034762in}}%
\pgfpathlineto{\pgfqpoint{4.874804in}{6.026320in}}%
\pgfpathlineto{\pgfqpoint{4.875877in}{6.011423in}}%
\pgfpathlineto{\pgfqpoint{4.876951in}{6.015892in}}%
\pgfpathlineto{\pgfqpoint{4.880173in}{6.001119in}}%
\pgfpathlineto{\pgfqpoint{4.883394in}{6.036376in}}%
\pgfpathlineto{\pgfqpoint{4.884468in}{6.032279in}}%
\pgfpathlineto{\pgfqpoint{4.887690in}{6.042211in}}%
\pgfpathlineto{\pgfqpoint{4.888763in}{6.033272in}}%
\pgfpathlineto{\pgfqpoint{4.889837in}{6.032776in}}%
\pgfpathlineto{\pgfqpoint{4.891985in}{6.052514in}}%
\pgfpathlineto{\pgfqpoint{4.895206in}{6.060832in}}%
\pgfpathlineto{\pgfqpoint{4.896280in}{6.067287in}}%
\pgfpathlineto{\pgfqpoint{4.897354in}{6.052763in}}%
\pgfpathlineto{\pgfqpoint{4.898428in}{6.059590in}}%
\pgfpathlineto{\pgfqpoint{4.899502in}{6.075481in}}%
\pgfpathlineto{\pgfqpoint{4.902723in}{6.070018in}}%
\pgfpathlineto{\pgfqpoint{4.903797in}{6.074736in}}%
\pgfpathlineto{\pgfqpoint{4.904871in}{6.058101in}}%
\pgfpathlineto{\pgfqpoint{4.905945in}{6.026320in}}%
\pgfpathlineto{\pgfqpoint{4.907019in}{6.026941in}}%
\pgfpathlineto{\pgfqpoint{4.910240in}{6.034390in}}%
\pgfpathlineto{\pgfqpoint{4.911314in}{6.030914in}}%
\pgfpathlineto{\pgfqpoint{4.912388in}{6.041714in}}%
\pgfpathlineto{\pgfqpoint{4.913462in}{6.046307in}}%
\pgfpathlineto{\pgfqpoint{4.914536in}{6.041466in}}%
\pgfpathlineto{\pgfqpoint{4.917757in}{6.038238in}}%
\pgfpathlineto{\pgfqpoint{4.918831in}{6.039728in}}%
\pgfpathlineto{\pgfqpoint{4.919905in}{6.023589in}}%
\pgfpathlineto{\pgfqpoint{4.920979in}{6.044693in}}%
\pgfpathlineto{\pgfqpoint{4.925274in}{6.048169in}}%
\pgfpathlineto{\pgfqpoint{4.926348in}{6.046804in}}%
\pgfpathlineto{\pgfqpoint{4.927422in}{6.039231in}}%
\pgfpathlineto{\pgfqpoint{4.928495in}{6.051149in}}%
\pgfpathlineto{\pgfqpoint{4.929569in}{6.043328in}}%
\pgfpathlineto{\pgfqpoint{4.932791in}{6.043204in}}%
\pgfpathlineto{\pgfqpoint{4.933865in}{6.051521in}}%
\pgfpathlineto{\pgfqpoint{4.934938in}{6.047549in}}%
\pgfpathlineto{\pgfqpoint{4.936012in}{6.036252in}}%
\pgfpathlineto{\pgfqpoint{4.937086in}{6.039479in}}%
\pgfpathlineto{\pgfqpoint{4.940308in}{6.030045in}}%
\pgfpathlineto{\pgfqpoint{4.941381in}{6.029300in}}%
\pgfpathlineto{\pgfqpoint{4.942455in}{6.019865in}}%
\pgfpathlineto{\pgfqpoint{4.943529in}{6.024582in}}%
\pgfpathlineto{\pgfqpoint{4.944603in}{6.022224in}}%
\pgfpathlineto{\pgfqpoint{4.947825in}{6.021851in}}%
\pgfpathlineto{\pgfqpoint{4.948898in}{6.002485in}}%
\pgfpathlineto{\pgfqpoint{4.949972in}{6.003602in}}%
\pgfpathlineto{\pgfqpoint{4.951046in}{6.005961in}}%
\pgfpathlineto{\pgfqpoint{4.952120in}{6.002485in}}%
\pgfpathlineto{\pgfqpoint{4.956415in}{6.008320in}}%
\pgfpathlineto{\pgfqpoint{4.958563in}{6.023962in}}%
\pgfpathlineto{\pgfqpoint{4.959637in}{6.018872in}}%
\pgfpathlineto{\pgfqpoint{4.962858in}{6.022348in}}%
\pgfpathlineto{\pgfqpoint{4.965006in}{6.039604in}}%
\pgfpathlineto{\pgfqpoint{4.966080in}{6.040845in}}%
\pgfpathlineto{\pgfqpoint{4.967154in}{6.040969in}}%
\pgfpathlineto{\pgfqpoint{4.970375in}{6.044445in}}%
\pgfpathlineto{\pgfqpoint{4.972523in}{6.067660in}}%
\pgfpathlineto{\pgfqpoint{4.973597in}{6.066791in}}%
\pgfpathlineto{\pgfqpoint{4.977892in}{6.054004in}}%
\pgfpathlineto{\pgfqpoint{4.978966in}{6.049783in}}%
\pgfpathlineto{\pgfqpoint{4.980040in}{6.048790in}}%
\pgfpathlineto{\pgfqpoint{4.981114in}{6.052638in}}%
\pgfpathlineto{\pgfqpoint{4.982187in}{6.047673in}}%
\pgfpathlineto{\pgfqpoint{4.985409in}{6.043452in}}%
\pgfpathlineto{\pgfqpoint{4.986483in}{6.048169in}}%
\pgfpathlineto{\pgfqpoint{4.988630in}{6.029051in}}%
\pgfpathlineto{\pgfqpoint{4.989704in}{6.032279in}}%
\pgfpathlineto{\pgfqpoint{4.992926in}{6.016637in}}%
\pgfpathlineto{\pgfqpoint{4.994000in}{6.008444in}}%
\pgfpathlineto{\pgfqpoint{4.995073in}{6.008444in}}%
\pgfpathlineto{\pgfqpoint{4.996147in}{6.036252in}}%
\pgfpathlineto{\pgfqpoint{4.997221in}{6.044693in}}%
\pgfpathlineto{\pgfqpoint{5.000443in}{6.052390in}}%
\pgfpathlineto{\pgfqpoint{5.001516in}{6.043080in}}%
\pgfpathlineto{\pgfqpoint{5.002590in}{6.055245in}}%
\pgfpathlineto{\pgfqpoint{5.003664in}{6.099813in}}%
\pgfpathlineto{\pgfqpoint{5.004738in}{6.103164in}}%
\pgfpathlineto{\pgfqpoint{5.007959in}{6.101799in}}%
\pgfpathlineto{\pgfqpoint{5.009033in}{6.107137in}}%
\pgfpathlineto{\pgfqpoint{5.010107in}{6.104406in}}%
\pgfpathlineto{\pgfqpoint{5.011181in}{6.107509in}}%
\pgfpathlineto{\pgfqpoint{5.012255in}{6.127000in}}%
\pgfpathlineto{\pgfqpoint{5.015476in}{6.128614in}}%
\pgfpathlineto{\pgfqpoint{5.016550in}{6.138917in}}%
\pgfpathlineto{\pgfqpoint{5.017624in}{6.132586in}}%
\pgfpathlineto{\pgfqpoint{5.018698in}{6.117689in}}%
\pgfpathlineto{\pgfqpoint{5.019772in}{6.121786in}}%
\pgfpathlineto{\pgfqpoint{5.024067in}{6.118806in}}%
\pgfpathlineto{\pgfqpoint{5.025141in}{6.121413in}}%
\pgfpathlineto{\pgfqpoint{5.026215in}{6.107633in}}%
\pgfpathlineto{\pgfqpoint{5.027289in}{6.117441in}}%
\pgfpathlineto{\pgfqpoint{5.030510in}{6.113220in}}%
\pgfpathlineto{\pgfqpoint{5.031584in}{6.108999in}}%
\pgfpathlineto{\pgfqpoint{5.033732in}{6.118806in}}%
\pgfpathlineto{\pgfqpoint{5.034805in}{6.129110in}}%
\pgfpathlineto{\pgfqpoint{5.038027in}{6.123896in}}%
\pgfpathlineto{\pgfqpoint{5.039101in}{6.125262in}}%
\pgfpathlineto{\pgfqpoint{5.040175in}{6.122903in}}%
\pgfpathlineto{\pgfqpoint{5.041248in}{6.142269in}}%
\pgfpathlineto{\pgfqpoint{5.042322in}{6.141152in}}%
\pgfpathlineto{\pgfqpoint{5.045544in}{6.149594in}}%
\pgfpathlineto{\pgfqpoint{5.047692in}{6.160890in}}%
\pgfpathlineto{\pgfqpoint{5.049839in}{6.163373in}}%
\pgfpathlineto{\pgfqpoint{5.053061in}{6.157539in}}%
\pgfpathlineto{\pgfqpoint{5.054135in}{6.149718in}}%
\pgfpathlineto{\pgfqpoint{5.055208in}{6.148476in}}%
\pgfpathlineto{\pgfqpoint{5.056282in}{6.148973in}}%
\pgfpathlineto{\pgfqpoint{5.057356in}{6.164863in}}%
\pgfpathlineto{\pgfqpoint{5.060578in}{6.162753in}}%
\pgfpathlineto{\pgfqpoint{5.061651in}{6.157539in}}%
\pgfpathlineto{\pgfqpoint{5.062725in}{6.141648in}}%
\pgfpathlineto{\pgfqpoint{5.063799in}{6.134821in}}%
\pgfpathlineto{\pgfqpoint{5.064873in}{6.139166in}}%
\pgfpathlineto{\pgfqpoint{5.068094in}{6.148725in}}%
\pgfpathlineto{\pgfqpoint{5.069168in}{6.144007in}}%
\pgfpathlineto{\pgfqpoint{5.070242in}{6.165484in}}%
\pgfpathlineto{\pgfqpoint{5.071316in}{6.170325in}}%
\pgfpathlineto{\pgfqpoint{5.072390in}{6.183484in}}%
\pgfpathlineto{\pgfqpoint{5.079907in}{6.201982in}}%
\pgfpathlineto{\pgfqpoint{5.083128in}{6.206575in}}%
\pgfpathlineto{\pgfqpoint{5.084202in}{6.219486in}}%
\pgfpathlineto{\pgfqpoint{5.086350in}{6.204961in}}%
\pgfpathlineto{\pgfqpoint{5.087424in}{6.208313in}}%
\pgfpathlineto{\pgfqpoint{5.090645in}{6.207816in}}%
\pgfpathlineto{\pgfqpoint{5.091719in}{6.203968in}}%
\pgfpathlineto{\pgfqpoint{5.092793in}{6.207071in}}%
\pgfpathlineto{\pgfqpoint{5.098162in}{6.179512in}}%
\pgfpathlineto{\pgfqpoint{5.099236in}{6.181002in}}%
\pgfpathlineto{\pgfqpoint{5.100310in}{6.193912in}}%
\pgfpathlineto{\pgfqpoint{5.101383in}{6.188450in}}%
\pgfpathlineto{\pgfqpoint{5.102457in}{6.224948in}}%
\pgfpathlineto{\pgfqpoint{5.106753in}{6.220603in}}%
\pgfpathlineto{\pgfqpoint{5.107826in}{6.226810in}}%
\pgfpathlineto{\pgfqpoint{5.109974in}{6.179015in}}%
\pgfpathlineto{\pgfqpoint{5.113196in}{6.164987in}}%
\pgfpathlineto{\pgfqpoint{5.114269in}{6.175415in}}%
\pgfpathlineto{\pgfqpoint{5.115343in}{6.163001in}}%
\pgfpathlineto{\pgfqpoint{5.116417in}{6.175167in}}%
\pgfpathlineto{\pgfqpoint{5.117491in}{6.156918in}}%
\pgfpathlineto{\pgfqpoint{5.120713in}{6.132214in}}%
\pgfpathlineto{\pgfqpoint{5.121786in}{6.145249in}}%
\pgfpathlineto{\pgfqpoint{5.122860in}{6.142021in}}%
\pgfpathlineto{\pgfqpoint{5.125008in}{6.179512in}}%
\pgfpathlineto{\pgfqpoint{5.130377in}{6.202602in}}%
\pgfpathlineto{\pgfqpoint{5.131451in}{6.201237in}}%
\pgfpathlineto{\pgfqpoint{5.132525in}{6.202478in}}%
\pgfpathlineto{\pgfqpoint{5.136820in}{6.202602in}}%
\pgfpathlineto{\pgfqpoint{5.137894in}{6.200616in}}%
\pgfpathlineto{\pgfqpoint{5.138968in}{6.202602in}}%
\pgfpathlineto{\pgfqpoint{5.140042in}{6.199623in}}%
\pgfpathlineto{\pgfqpoint{5.143263in}{6.212906in}}%
\pgfpathlineto{\pgfqpoint{5.144337in}{6.212782in}}%
\pgfpathlineto{\pgfqpoint{5.145411in}{6.210796in}}%
\pgfpathlineto{\pgfqpoint{5.146485in}{6.217127in}}%
\pgfpathlineto{\pgfqpoint{5.147558in}{6.228548in}}%
\pgfpathlineto{\pgfqpoint{5.150780in}{6.213527in}}%
\pgfpathlineto{\pgfqpoint{5.151854in}{6.243693in}}%
\pgfpathlineto{\pgfqpoint{5.152928in}{6.238355in}}%
\pgfpathlineto{\pgfqpoint{5.154002in}{6.254121in}}%
\pgfpathlineto{\pgfqpoint{5.155075in}{6.257970in}}%
\pgfpathlineto{\pgfqpoint{5.158297in}{6.256108in}}%
\pgfpathlineto{\pgfqpoint{5.160445in}{6.247045in}}%
\pgfpathlineto{\pgfqpoint{5.161518in}{6.221720in}}%
\pgfpathlineto{\pgfqpoint{5.162592in}{6.215761in}}%
\pgfpathlineto{\pgfqpoint{5.166888in}{6.232024in}}%
\pgfpathlineto{\pgfqpoint{5.167961in}{6.222217in}}%
\pgfpathlineto{\pgfqpoint{5.169035in}{6.233141in}}%
\pgfpathlineto{\pgfqpoint{5.174404in}{6.223458in}}%
\pgfpathlineto{\pgfqpoint{5.175478in}{6.209927in}}%
\pgfpathlineto{\pgfqpoint{5.177626in}{6.219113in}}%
\pgfpathlineto{\pgfqpoint{5.180847in}{6.214147in}}%
\pgfpathlineto{\pgfqpoint{5.181921in}{6.227182in}}%
\pgfpathlineto{\pgfqpoint{5.182995in}{6.221099in}}%
\pgfpathlineto{\pgfqpoint{5.184069in}{6.227679in}}%
\pgfpathlineto{\pgfqpoint{5.185143in}{6.206947in}}%
\pgfpathlineto{\pgfqpoint{5.188364in}{6.177526in}}%
\pgfpathlineto{\pgfqpoint{5.189438in}{6.176284in}}%
\pgfpathlineto{\pgfqpoint{5.190512in}{6.201857in}}%
\pgfpathlineto{\pgfqpoint{5.191586in}{6.163249in}}%
\pgfpathlineto{\pgfqpoint{5.192660in}{6.153939in}}%
\pgfpathlineto{\pgfqpoint{5.195881in}{6.164863in}}%
\pgfpathlineto{\pgfqpoint{5.196955in}{6.171070in}}%
\pgfpathlineto{\pgfqpoint{5.198029in}{6.186712in}}%
\pgfpathlineto{\pgfqpoint{5.199103in}{6.173181in}}%
\pgfpathlineto{\pgfqpoint{5.203398in}{6.178270in}}%
\pgfpathlineto{\pgfqpoint{5.204472in}{6.182864in}}%
\pgfpathlineto{\pgfqpoint{5.205546in}{6.183609in}}%
\pgfpathlineto{\pgfqpoint{5.206620in}{6.186836in}}%
\pgfpathlineto{\pgfqpoint{5.207693in}{6.182491in}}%
\pgfpathlineto{\pgfqpoint{5.210915in}{6.182740in}}%
\pgfpathlineto{\pgfqpoint{5.211989in}{6.191181in}}%
\pgfpathlineto{\pgfqpoint{5.213063in}{6.186960in}}%
\pgfpathlineto{\pgfqpoint{5.214136in}{6.180008in}}%
\pgfpathlineto{\pgfqpoint{5.215210in}{6.181250in}}%
\pgfpathlineto{\pgfqpoint{5.218432in}{6.186215in}}%
\pgfpathlineto{\pgfqpoint{5.219506in}{6.172932in}}%
\pgfpathlineto{\pgfqpoint{5.220580in}{6.193292in}}%
\pgfpathlineto{\pgfqpoint{5.221653in}{6.200616in}}%
\pgfpathlineto{\pgfqpoint{5.222727in}{6.203223in}}%
\pgfpathlineto{\pgfqpoint{5.225949in}{6.212534in}}%
\pgfpathlineto{\pgfqpoint{5.228096in}{6.198878in}}%
\pgfpathlineto{\pgfqpoint{5.229170in}{6.188947in}}%
\pgfpathlineto{\pgfqpoint{5.230244in}{6.187829in}}%
\pgfpathlineto{\pgfqpoint{5.233466in}{6.194905in}}%
\pgfpathlineto{\pgfqpoint{5.234539in}{6.183112in}}%
\pgfpathlineto{\pgfqpoint{5.235613in}{6.191926in}}%
\pgfpathlineto{\pgfqpoint{5.236687in}{6.195278in}}%
\pgfpathlineto{\pgfqpoint{5.242056in}{6.232396in}}%
\pgfpathlineto{\pgfqpoint{5.243130in}{6.228672in}}%
\pgfpathlineto{\pgfqpoint{5.245278in}{6.233638in}}%
\pgfpathlineto{\pgfqpoint{5.248499in}{6.238107in}}%
\pgfpathlineto{\pgfqpoint{5.249573in}{6.236245in}}%
\pgfpathlineto{\pgfqpoint{5.250647in}{6.237238in}}%
\pgfpathlineto{\pgfqpoint{5.251721in}{6.248535in}}%
\pgfpathlineto{\pgfqpoint{5.252795in}{6.272743in}}%
\pgfpathlineto{\pgfqpoint{5.256016in}{6.280315in}}%
\pgfpathlineto{\pgfqpoint{5.259238in}{6.271005in}}%
\pgfpathlineto{\pgfqpoint{5.260312in}{6.271998in}}%
\pgfpathlineto{\pgfqpoint{5.263533in}{6.266535in}}%
\pgfpathlineto{\pgfqpoint{5.264607in}{6.269887in}}%
\pgfpathlineto{\pgfqpoint{5.265681in}{6.280191in}}%
\pgfpathlineto{\pgfqpoint{5.266755in}{6.274605in}}%
\pgfpathlineto{\pgfqpoint{5.267828in}{6.279943in}}%
\pgfpathlineto{\pgfqpoint{5.271050in}{6.279570in}}%
\pgfpathlineto{\pgfqpoint{5.272124in}{6.267280in}}%
\pgfpathlineto{\pgfqpoint{5.273198in}{6.269639in}}%
\pgfpathlineto{\pgfqpoint{5.274271in}{6.265791in}}%
\pgfpathlineto{\pgfqpoint{5.275345in}{6.272991in}}%
\pgfpathlineto{\pgfqpoint{5.278567in}{6.272246in}}%
\pgfpathlineto{\pgfqpoint{5.279641in}{6.277832in}}%
\pgfpathlineto{\pgfqpoint{5.280714in}{6.276219in}}%
\pgfpathlineto{\pgfqpoint{5.281788in}{6.283419in}}%
\pgfpathlineto{\pgfqpoint{5.286084in}{6.278329in}}%
\pgfpathlineto{\pgfqpoint{5.287158in}{6.268894in}}%
\pgfpathlineto{\pgfqpoint{5.288231in}{6.273736in}}%
\pgfpathlineto{\pgfqpoint{5.289305in}{6.270508in}}%
\pgfpathlineto{\pgfqpoint{5.293601in}{6.271501in}}%
\pgfpathlineto{\pgfqpoint{5.294674in}{6.270508in}}%
\pgfpathlineto{\pgfqpoint{5.295748in}{6.270632in}}%
\pgfpathlineto{\pgfqpoint{5.296822in}{6.253625in}}%
\pgfpathlineto{\pgfqpoint{5.297896in}{6.259832in}}%
\pgfpathlineto{\pgfqpoint{5.301117in}{6.252383in}}%
\pgfpathlineto{\pgfqpoint{5.302191in}{6.258094in}}%
\pgfpathlineto{\pgfqpoint{5.304339in}{6.255487in}}%
\pgfpathlineto{\pgfqpoint{5.305413in}{6.241459in}}%
\pgfpathlineto{\pgfqpoint{5.308634in}{6.240714in}}%
\pgfpathlineto{\pgfqpoint{5.309708in}{6.238976in}}%
\pgfpathlineto{\pgfqpoint{5.310782in}{6.229789in}}%
\pgfpathlineto{\pgfqpoint{5.312930in}{6.178767in}}%
\pgfpathlineto{\pgfqpoint{5.316151in}{6.183981in}}%
\pgfpathlineto{\pgfqpoint{5.317225in}{6.177526in}}%
\pgfpathlineto{\pgfqpoint{5.318299in}{6.177898in}}%
\pgfpathlineto{\pgfqpoint{5.319373in}{6.173553in}}%
\pgfpathlineto{\pgfqpoint{5.320446in}{6.190188in}}%
\pgfpathlineto{\pgfqpoint{5.323668in}{6.184229in}}%
\pgfpathlineto{\pgfqpoint{5.324742in}{6.185719in}}%
\pgfpathlineto{\pgfqpoint{5.325816in}{6.189443in}}%
\pgfpathlineto{\pgfqpoint{5.326890in}{6.187953in}}%
\pgfpathlineto{\pgfqpoint{5.327963in}{6.180257in}}%
\pgfpathlineto{\pgfqpoint{5.331185in}{6.186712in}}%
\pgfpathlineto{\pgfqpoint{5.332259in}{6.197761in}}%
\pgfpathlineto{\pgfqpoint{5.333333in}{6.201982in}}%
\pgfpathlineto{\pgfqpoint{5.334406in}{6.209430in}}%
\pgfpathlineto{\pgfqpoint{5.335480in}{6.206451in}}%
\pgfpathlineto{\pgfqpoint{5.338702in}{6.214644in}}%
\pgfpathlineto{\pgfqpoint{5.339776in}{6.209554in}}%
\pgfpathlineto{\pgfqpoint{5.340849in}{6.210547in}}%
\pgfpathlineto{\pgfqpoint{5.341923in}{6.208065in}}%
\pgfpathlineto{\pgfqpoint{5.342997in}{6.214147in}}%
\pgfpathlineto{\pgfqpoint{5.347292in}{6.216134in}}%
\pgfpathlineto{\pgfqpoint{5.348366in}{6.220975in}}%
\pgfpathlineto{\pgfqpoint{5.349440in}{6.215265in}}%
\pgfpathlineto{\pgfqpoint{5.350514in}{6.214768in}}%
\pgfpathlineto{\pgfqpoint{5.353735in}{6.207071in}}%
\pgfpathlineto{\pgfqpoint{5.354809in}{6.195154in}}%
\pgfpathlineto{\pgfqpoint{5.355883in}{6.200988in}}%
\pgfpathlineto{\pgfqpoint{5.356957in}{6.201113in}}%
\pgfpathlineto{\pgfqpoint{5.358031in}{6.192174in}}%
\pgfpathlineto{\pgfqpoint{5.361252in}{6.189195in}}%
\pgfpathlineto{\pgfqpoint{5.364474in}{6.220479in}}%
\pgfpathlineto{\pgfqpoint{5.365548in}{6.215885in}}%
\pgfpathlineto{\pgfqpoint{5.368769in}{6.207692in}}%
\pgfpathlineto{\pgfqpoint{5.369843in}{6.199871in}}%
\pgfpathlineto{\pgfqpoint{5.370917in}{6.202354in}}%
\pgfpathlineto{\pgfqpoint{5.371991in}{6.182119in}}%
\pgfpathlineto{\pgfqpoint{5.373065in}{6.200616in}}%
\pgfpathlineto{\pgfqpoint{5.376286in}{6.195774in}}%
\pgfpathlineto{\pgfqpoint{5.377360in}{6.191057in}}%
\pgfpathlineto{\pgfqpoint{5.378434in}{6.174050in}}%
\pgfpathlineto{\pgfqpoint{5.379508in}{6.174794in}}%
\pgfpathlineto{\pgfqpoint{5.380581in}{6.189691in}}%
\pgfpathlineto{\pgfqpoint{5.383803in}{6.188202in}}%
\pgfpathlineto{\pgfqpoint{5.384877in}{6.168836in}}%
\pgfpathlineto{\pgfqpoint{5.385951in}{6.192547in}}%
\pgfpathlineto{\pgfqpoint{5.388098in}{6.164615in}}%
\pgfpathlineto{\pgfqpoint{5.391320in}{6.139166in}}%
\pgfpathlineto{\pgfqpoint{5.392394in}{6.138669in}}%
\pgfpathlineto{\pgfqpoint{5.393468in}{6.117813in}}%
\pgfpathlineto{\pgfqpoint{5.394541in}{6.109868in}}%
\pgfpathlineto{\pgfqpoint{5.395615in}{6.136807in}}%
\pgfpathlineto{\pgfqpoint{5.398837in}{6.153318in}}%
\pgfpathlineto{\pgfqpoint{5.399911in}{6.172187in}}%
\pgfpathlineto{\pgfqpoint{5.400984in}{6.152821in}}%
\pgfpathlineto{\pgfqpoint{5.402058in}{6.171815in}}%
\pgfpathlineto{\pgfqpoint{5.403132in}{6.180877in}}%
\pgfpathlineto{\pgfqpoint{5.406354in}{6.183609in}}%
\pgfpathlineto{\pgfqpoint{5.407427in}{6.199499in}}%
\pgfpathlineto{\pgfqpoint{5.409575in}{6.207816in}}%
\pgfpathlineto{\pgfqpoint{5.410649in}{6.221844in}}%
\pgfpathlineto{\pgfqpoint{5.413870in}{6.232148in}}%
\pgfpathlineto{\pgfqpoint{5.414944in}{6.238355in}}%
\pgfpathlineto{\pgfqpoint{5.416018in}{6.250273in}}%
\pgfpathlineto{\pgfqpoint{5.417092in}{6.240590in}}%
\pgfpathlineto{\pgfqpoint{5.418166in}{6.248411in}}%
\pgfpathlineto{\pgfqpoint{5.421387in}{6.250025in}}%
\pgfpathlineto{\pgfqpoint{5.422461in}{6.242452in}}%
\pgfpathlineto{\pgfqpoint{5.423535in}{6.240217in}}%
\pgfpathlineto{\pgfqpoint{5.425683in}{6.230162in}}%
\pgfpathlineto{\pgfqpoint{5.428904in}{6.223831in}}%
\pgfpathlineto{\pgfqpoint{5.429978in}{6.229045in}}%
\pgfpathlineto{\pgfqpoint{5.431052in}{6.228176in}}%
\pgfpathlineto{\pgfqpoint{5.432126in}{6.229541in}}%
\pgfpathlineto{\pgfqpoint{5.433200in}{6.226934in}}%
\pgfpathlineto{\pgfqpoint{5.436421in}{6.234383in}}%
\pgfpathlineto{\pgfqpoint{5.437495in}{6.238479in}}%
\pgfpathlineto{\pgfqpoint{5.438569in}{6.239100in}}%
\pgfpathlineto{\pgfqpoint{5.440716in}{6.250273in}}%
\pgfpathlineto{\pgfqpoint{5.443938in}{6.247045in}}%
\pgfpathlineto{\pgfqpoint{5.445012in}{6.256977in}}%
\pgfpathlineto{\pgfqpoint{5.446086in}{6.236741in}}%
\pgfpathlineto{\pgfqpoint{5.448233in}{6.252880in}}%
\pgfpathlineto{\pgfqpoint{5.451455in}{6.263432in}}%
\pgfpathlineto{\pgfqpoint{5.452529in}{6.261073in}}%
\pgfpathlineto{\pgfqpoint{5.453602in}{6.254121in}}%
\pgfpathlineto{\pgfqpoint{5.454676in}{6.258466in}}%
\pgfpathlineto{\pgfqpoint{5.455750in}{6.233886in}}%
\pgfpathlineto{\pgfqpoint{5.458972in}{6.222837in}}%
\pgfpathlineto{\pgfqpoint{5.460046in}{6.202106in}}%
\pgfpathlineto{\pgfqpoint{5.462193in}{6.258963in}}%
\pgfpathlineto{\pgfqpoint{5.463267in}{6.255859in}}%
\pgfpathlineto{\pgfqpoint{5.468636in}{6.269391in}}%
\pgfpathlineto{\pgfqpoint{5.470784in}{6.271874in}}%
\pgfpathlineto{\pgfqpoint{5.475079in}{6.271625in}}%
\pgfpathlineto{\pgfqpoint{5.476153in}{6.257473in}}%
\pgfpathlineto{\pgfqpoint{5.478301in}{6.257225in}}%
\pgfpathlineto{\pgfqpoint{5.481522in}{6.228796in}}%
\pgfpathlineto{\pgfqpoint{5.482596in}{6.206575in}}%
\pgfpathlineto{\pgfqpoint{5.484744in}{6.243693in}}%
\pgfpathlineto{\pgfqpoint{5.485818in}{6.230286in}}%
\pgfpathlineto{\pgfqpoint{5.490113in}{6.216506in}}%
\pgfpathlineto{\pgfqpoint{5.492261in}{6.177774in}}%
\pgfpathlineto{\pgfqpoint{5.493334in}{6.179636in}}%
\pgfpathlineto{\pgfqpoint{5.498704in}{6.198506in}}%
\pgfpathlineto{\pgfqpoint{5.499778in}{6.160394in}}%
\pgfpathlineto{\pgfqpoint{5.504073in}{6.147980in}}%
\pgfpathlineto{\pgfqpoint{5.506221in}{6.129855in}}%
\pgfpathlineto{\pgfqpoint{5.507294in}{6.132958in}}%
\pgfpathlineto{\pgfqpoint{5.508368in}{6.117813in}}%
\pgfpathlineto{\pgfqpoint{5.511590in}{6.134448in}}%
\pgfpathlineto{\pgfqpoint{5.512664in}{6.152945in}}%
\pgfpathlineto{\pgfqpoint{5.513737in}{6.151580in}}%
\pgfpathlineto{\pgfqpoint{5.514811in}{6.164491in}}%
\pgfpathlineto{\pgfqpoint{5.515885in}{6.167718in}}%
\pgfpathlineto{\pgfqpoint{5.519107in}{6.167470in}}%
\pgfpathlineto{\pgfqpoint{5.520180in}{6.177401in}}%
\pgfpathlineto{\pgfqpoint{5.521254in}{6.179264in}}%
\pgfpathlineto{\pgfqpoint{5.522328in}{6.115330in}}%
\pgfpathlineto{\pgfqpoint{5.523402in}{6.087647in}}%
\pgfpathlineto{\pgfqpoint{5.527697in}{6.099192in}}%
\pgfpathlineto{\pgfqpoint{5.528771in}{6.107261in}}%
\pgfpathlineto{\pgfqpoint{5.529845in}{6.091371in}}%
\pgfpathlineto{\pgfqpoint{5.530919in}{6.107882in}}%
\pgfpathlineto{\pgfqpoint{5.534140in}{6.113344in}}%
\pgfpathlineto{\pgfqpoint{5.535214in}{6.119799in}}%
\pgfpathlineto{\pgfqpoint{5.537362in}{6.147359in}}%
\pgfpathlineto{\pgfqpoint{5.538436in}{6.128241in}}%
\pgfpathlineto{\pgfqpoint{5.541657in}{6.133331in}}%
\pgfpathlineto{\pgfqpoint{5.542731in}{6.131965in}}%
\pgfpathlineto{\pgfqpoint{5.543805in}{6.117068in}}%
\pgfpathlineto{\pgfqpoint{5.544879in}{6.123151in}}%
\pgfpathlineto{\pgfqpoint{5.545953in}{6.113468in}}%
\pgfpathlineto{\pgfqpoint{5.549174in}{6.115703in}}%
\pgfpathlineto{\pgfqpoint{5.550248in}{6.099564in}}%
\pgfpathlineto{\pgfqpoint{5.551322in}{6.103537in}}%
\pgfpathlineto{\pgfqpoint{5.552396in}{6.127869in}}%
\pgfpathlineto{\pgfqpoint{5.553469in}{6.116820in}}%
\pgfpathlineto{\pgfqpoint{5.556691in}{6.127124in}}%
\pgfpathlineto{\pgfqpoint{5.557765in}{6.122034in}}%
\pgfpathlineto{\pgfqpoint{5.558839in}{6.131345in}}%
\pgfpathlineto{\pgfqpoint{5.559912in}{6.127620in}}%
\pgfpathlineto{\pgfqpoint{5.560986in}{6.141028in}}%
\pgfpathlineto{\pgfqpoint{5.564208in}{6.135317in}}%
\pgfpathlineto{\pgfqpoint{5.566356in}{6.111109in}}%
\pgfpathlineto{\pgfqpoint{5.567429in}{6.092240in}}%
\pgfpathlineto{\pgfqpoint{5.568503in}{6.086405in}}%
\pgfpathlineto{\pgfqpoint{5.571725in}{6.087274in}}%
\pgfpathlineto{\pgfqpoint{5.572799in}{6.091123in}}%
\pgfpathlineto{\pgfqpoint{5.574946in}{6.109496in}}%
\pgfpathlineto{\pgfqpoint{5.579242in}{6.108627in}}%
\pgfpathlineto{\pgfqpoint{5.580315in}{6.093605in}}%
\pgfpathlineto{\pgfqpoint{5.582463in}{6.103164in}}%
\pgfpathlineto{\pgfqpoint{5.583537in}{6.108130in}}%
\pgfpathlineto{\pgfqpoint{5.586758in}{6.104406in}}%
\pgfpathlineto{\pgfqpoint{5.588906in}{6.109992in}}%
\pgfpathlineto{\pgfqpoint{5.589980in}{6.123524in}}%
\pgfpathlineto{\pgfqpoint{5.591054in}{6.081812in}}%
\pgfpathlineto{\pgfqpoint{5.594275in}{6.080819in}}%
\pgfpathlineto{\pgfqpoint{5.595349in}{6.081439in}}%
\pgfpathlineto{\pgfqpoint{5.596423in}{6.094847in}}%
\pgfpathlineto{\pgfqpoint{5.598571in}{6.089633in}}%
\pgfpathlineto{\pgfqpoint{5.601792in}{6.083798in}}%
\pgfpathlineto{\pgfqpoint{5.602866in}{6.083798in}}%
\pgfpathlineto{\pgfqpoint{5.603940in}{6.079950in}}%
\pgfpathlineto{\pgfqpoint{5.606088in}{6.086033in}}%
\pgfpathlineto{\pgfqpoint{5.609309in}{6.092985in}}%
\pgfpathlineto{\pgfqpoint{5.610383in}{6.087895in}}%
\pgfpathlineto{\pgfqpoint{5.611457in}{6.088019in}}%
\pgfpathlineto{\pgfqpoint{5.613604in}{6.101054in}}%
\pgfpathlineto{\pgfqpoint{5.616826in}{6.109496in}}%
\pgfpathlineto{\pgfqpoint{5.617900in}{6.102295in}}%
\pgfpathlineto{\pgfqpoint{5.620047in}{6.121662in}}%
\pgfpathlineto{\pgfqpoint{5.621121in}{6.115454in}}%
\pgfpathlineto{\pgfqpoint{5.624343in}{6.114958in}}%
\pgfpathlineto{\pgfqpoint{5.625417in}{6.128738in}}%
\pgfpathlineto{\pgfqpoint{5.627564in}{6.121786in}}%
\pgfpathlineto{\pgfqpoint{5.628638in}{6.127372in}}%
\pgfpathlineto{\pgfqpoint{5.632934in}{6.115082in}}%
\pgfpathlineto{\pgfqpoint{5.635081in}{6.113965in}}%
\pgfpathlineto{\pgfqpoint{5.636155in}{6.109744in}}%
\pgfpathlineto{\pgfqpoint{5.639377in}{6.106764in}}%
\pgfpathlineto{\pgfqpoint{5.641524in}{6.119924in}}%
\pgfpathlineto{\pgfqpoint{5.642598in}{6.105523in}}%
\pgfpathlineto{\pgfqpoint{5.643672in}{6.105895in}}%
\pgfpathlineto{\pgfqpoint{5.646893in}{6.099068in}}%
\pgfpathlineto{\pgfqpoint{5.647967in}{6.103288in}}%
\pgfpathlineto{\pgfqpoint{5.649041in}{6.115082in}}%
\pgfpathlineto{\pgfqpoint{5.650115in}{6.116448in}}%
\pgfpathlineto{\pgfqpoint{5.651189in}{6.107509in}}%
\pgfpathlineto{\pgfqpoint{5.654410in}{6.104282in}}%
\pgfpathlineto{\pgfqpoint{5.655484in}{6.105523in}}%
\pgfpathlineto{\pgfqpoint{5.656558in}{6.116696in}}%
\pgfpathlineto{\pgfqpoint{5.657632in}{6.121910in}}%
\pgfpathlineto{\pgfqpoint{5.658706in}{6.115082in}}%
\pgfpathlineto{\pgfqpoint{5.661927in}{6.127372in}}%
\pgfpathlineto{\pgfqpoint{5.663001in}{6.128738in}}%
\pgfpathlineto{\pgfqpoint{5.665149in}{6.111482in}}%
\pgfpathlineto{\pgfqpoint{5.666222in}{6.111482in}}%
\pgfpathlineto{\pgfqpoint{5.669444in}{6.087398in}}%
\pgfpathlineto{\pgfqpoint{5.670518in}{6.089881in}}%
\pgfpathlineto{\pgfqpoint{5.671592in}{6.097826in}}%
\pgfpathlineto{\pgfqpoint{5.672666in}{6.095592in}}%
\pgfpathlineto{\pgfqpoint{5.676961in}{6.088391in}}%
\pgfpathlineto{\pgfqpoint{5.678035in}{6.087895in}}%
\pgfpathlineto{\pgfqpoint{5.679109in}{6.069025in}}%
\pgfpathlineto{\pgfqpoint{5.680182in}{6.073743in}}%
\pgfpathlineto{\pgfqpoint{5.681256in}{6.085040in}}%
\pgfpathlineto{\pgfqpoint{5.685552in}{6.104406in}}%
\pgfpathlineto{\pgfqpoint{5.686625in}{6.099440in}}%
\pgfpathlineto{\pgfqpoint{5.688773in}{6.107261in}}%
\pgfpathlineto{\pgfqpoint{5.691995in}{6.108254in}}%
\pgfpathlineto{\pgfqpoint{5.693068in}{6.104157in}}%
\pgfpathlineto{\pgfqpoint{5.694142in}{6.104654in}}%
\pgfpathlineto{\pgfqpoint{5.696290in}{6.068653in}}%
\pgfpathlineto{\pgfqpoint{5.699511in}{6.057232in}}%
\pgfpathlineto{\pgfqpoint{5.700585in}{6.059466in}}%
\pgfpathlineto{\pgfqpoint{5.702733in}{6.071260in}}%
\pgfpathlineto{\pgfqpoint{5.703807in}{6.070515in}}%
\pgfpathlineto{\pgfqpoint{5.707028in}{6.069646in}}%
\pgfpathlineto{\pgfqpoint{5.709176in}{6.065053in}}%
\pgfpathlineto{\pgfqpoint{5.710250in}{6.058225in}}%
\pgfpathlineto{\pgfqpoint{5.711324in}{6.113096in}}%
\pgfpathlineto{\pgfqpoint{5.714545in}{6.131096in}}%
\pgfpathlineto{\pgfqpoint{5.715619in}{6.131965in}}%
\pgfpathlineto{\pgfqpoint{5.717767in}{6.125262in}}%
\pgfpathlineto{\pgfqpoint{5.718841in}{6.127000in}}%
\pgfpathlineto{\pgfqpoint{5.722062in}{6.127993in}}%
\pgfpathlineto{\pgfqpoint{5.723136in}{6.130848in}}%
\pgfpathlineto{\pgfqpoint{5.724210in}{6.127496in}}%
\pgfpathlineto{\pgfqpoint{5.726357in}{6.081812in}}%
\pgfpathlineto{\pgfqpoint{5.730653in}{6.040969in}}%
\pgfpathlineto{\pgfqpoint{5.732800in}{6.080074in}}%
\pgfpathlineto{\pgfqpoint{5.733874in}{6.077343in}}%
\pgfpathlineto{\pgfqpoint{5.737096in}{6.078212in}}%
\pgfpathlineto{\pgfqpoint{5.738170in}{6.046556in}}%
\pgfpathlineto{\pgfqpoint{5.739244in}{6.057480in}}%
\pgfpathlineto{\pgfqpoint{5.740317in}{6.061204in}}%
\pgfpathlineto{\pgfqpoint{5.741391in}{6.047425in}}%
\pgfpathlineto{\pgfqpoint{5.745687in}{6.063935in}}%
\pgfpathlineto{\pgfqpoint{5.746760in}{6.059590in}}%
\pgfpathlineto{\pgfqpoint{5.748908in}{6.064060in}}%
\pgfpathlineto{\pgfqpoint{5.752130in}{6.059839in}}%
\pgfpathlineto{\pgfqpoint{5.754277in}{6.085164in}}%
\pgfpathlineto{\pgfqpoint{5.755351in}{6.082184in}}%
\pgfpathlineto{\pgfqpoint{5.756425in}{6.069274in}}%
\pgfpathlineto{\pgfqpoint{5.759646in}{6.078460in}}%
\pgfpathlineto{\pgfqpoint{5.760720in}{6.066418in}}%
\pgfpathlineto{\pgfqpoint{5.761794in}{6.065549in}}%
\pgfpathlineto{\pgfqpoint{5.762868in}{6.054625in}}%
\pgfpathlineto{\pgfqpoint{5.763942in}{6.059342in}}%
\pgfpathlineto{\pgfqpoint{5.767163in}{6.038859in}}%
\pgfpathlineto{\pgfqpoint{5.768237in}{6.036128in}}%
\pgfpathlineto{\pgfqpoint{5.769311in}{6.048045in}}%
\pgfpathlineto{\pgfqpoint{5.770385in}{6.045314in}}%
\pgfpathlineto{\pgfqpoint{5.771459in}{6.051273in}}%
\pgfpathlineto{\pgfqpoint{5.774680in}{6.084915in}}%
\pgfpathlineto{\pgfqpoint{5.775754in}{6.080446in}}%
\pgfpathlineto{\pgfqpoint{5.776828in}{6.087026in}}%
\pgfpathlineto{\pgfqpoint{5.777902in}{6.087026in}}%
\pgfpathlineto{\pgfqpoint{5.778976in}{6.088764in}}%
\pgfpathlineto{\pgfqpoint{5.782197in}{6.088516in}}%
\pgfpathlineto{\pgfqpoint{5.784345in}{6.075108in}}%
\pgfpathlineto{\pgfqpoint{5.786492in}{6.087274in}}%
\pgfpathlineto{\pgfqpoint{5.790788in}{6.084295in}}%
\pgfpathlineto{\pgfqpoint{5.791862in}{6.079205in}}%
\pgfpathlineto{\pgfqpoint{5.792935in}{6.032279in}}%
\pgfpathlineto{\pgfqpoint{5.794009in}{6.056735in}}%
\pgfpathlineto{\pgfqpoint{5.798305in}{6.049907in}}%
\pgfpathlineto{\pgfqpoint{5.799378in}{6.055121in}}%
\pgfpathlineto{\pgfqpoint{5.800452in}{6.052390in}}%
\pgfpathlineto{\pgfqpoint{5.801526in}{6.041217in}}%
\pgfpathlineto{\pgfqpoint{5.804748in}{6.049162in}}%
\pgfpathlineto{\pgfqpoint{5.806895in}{6.050652in}}%
\pgfpathlineto{\pgfqpoint{5.807969in}{6.049038in}}%
\pgfpathlineto{\pgfqpoint{5.809043in}{6.053259in}}%
\pgfpathlineto{\pgfqpoint{5.812265in}{6.043080in}}%
\pgfpathlineto{\pgfqpoint{5.813338in}{6.042459in}}%
\pgfpathlineto{\pgfqpoint{5.814412in}{6.037121in}}%
\pgfpathlineto{\pgfqpoint{5.816560in}{6.017134in}}%
\pgfpathlineto{\pgfqpoint{5.819781in}{6.022472in}}%
\pgfpathlineto{\pgfqpoint{5.820855in}{6.016017in}}%
\pgfpathlineto{\pgfqpoint{5.823003in}{6.035134in}}%
\pgfpathlineto{\pgfqpoint{5.824077in}{6.031286in}}%
\pgfpathlineto{\pgfqpoint{5.827298in}{6.029300in}}%
\pgfpathlineto{\pgfqpoint{5.828372in}{6.022099in}}%
\pgfpathlineto{\pgfqpoint{5.831594in}{6.024706in}}%
\pgfpathlineto{\pgfqpoint{5.834815in}{6.022224in}}%
\pgfpathlineto{\pgfqpoint{5.835889in}{6.028803in}}%
\pgfpathlineto{\pgfqpoint{5.838037in}{6.007947in}}%
\pgfpathlineto{\pgfqpoint{5.839111in}{6.015892in}}%
\pgfpathlineto{\pgfqpoint{5.842332in}{6.010182in}}%
\pgfpathlineto{\pgfqpoint{5.843406in}{6.001988in}}%
\pgfpathlineto{\pgfqpoint{5.844480in}{6.001368in}}%
\pgfpathlineto{\pgfqpoint{5.845554in}{6.004347in}}%
\pgfpathlineto{\pgfqpoint{5.846627in}{5.989698in}}%
\pgfpathlineto{\pgfqpoint{5.849849in}{5.989450in}}%
\pgfpathlineto{\pgfqpoint{5.850923in}{6.004844in}}%
\pgfpathlineto{\pgfqpoint{5.851997in}{6.011423in}}%
\pgfpathlineto{\pgfqpoint{5.854144in}{5.978153in}}%
\pgfpathlineto{\pgfqpoint{5.857366in}{5.984360in}}%
\pgfpathlineto{\pgfqpoint{5.858440in}{5.989574in}}%
\pgfpathlineto{\pgfqpoint{5.859513in}{6.002733in}}%
\pgfpathlineto{\pgfqpoint{5.860587in}{6.004968in}}%
\pgfpathlineto{\pgfqpoint{5.864883in}{6.000250in}}%
\pgfpathlineto{\pgfqpoint{5.865956in}{6.009437in}}%
\pgfpathlineto{\pgfqpoint{5.867030in}{6.004968in}}%
\pgfpathlineto{\pgfqpoint{5.868104in}{5.997768in}}%
\pgfpathlineto{\pgfqpoint{5.872399in}{5.974925in}}%
\pgfpathlineto{\pgfqpoint{5.873473in}{5.962760in}}%
\pgfpathlineto{\pgfqpoint{5.874547in}{5.941159in}}%
\pgfpathlineto{\pgfqpoint{5.875621in}{5.934331in}}%
\pgfpathlineto{\pgfqpoint{5.876695in}{5.931848in}}%
\pgfpathlineto{\pgfqpoint{5.879916in}{5.936814in}}%
\pgfpathlineto{\pgfqpoint{5.880990in}{5.940910in}}%
\pgfpathlineto{\pgfqpoint{5.882064in}{5.922662in}}%
\pgfpathlineto{\pgfqpoint{5.883138in}{5.927876in}}%
\pgfpathlineto{\pgfqpoint{5.884212in}{5.923406in}}%
\pgfpathlineto{\pgfqpoint{5.888507in}{5.920303in}}%
\pgfpathlineto{\pgfqpoint{5.889581in}{5.924772in}}%
\pgfpathlineto{\pgfqpoint{5.890655in}{5.920303in}}%
\pgfpathlineto{\pgfqpoint{5.891729in}{5.831293in}}%
\pgfpathlineto{\pgfqpoint{5.894950in}{5.830796in}}%
\pgfpathlineto{\pgfqpoint{5.896024in}{5.831665in}}%
\pgfpathlineto{\pgfqpoint{5.897098in}{5.824962in}}%
\pgfpathlineto{\pgfqpoint{5.898172in}{5.805720in}}%
\pgfpathlineto{\pgfqpoint{5.899245in}{5.812920in}}%
\pgfpathlineto{\pgfqpoint{5.902467in}{5.827072in}}%
\pgfpathlineto{\pgfqpoint{5.903541in}{5.814906in}}%
\pgfpathlineto{\pgfqpoint{5.905688in}{5.823348in}}%
\pgfpathlineto{\pgfqpoint{5.906762in}{5.818630in}}%
\pgfpathlineto{\pgfqpoint{5.909984in}{5.800009in}}%
\pgfpathlineto{\pgfqpoint{5.911058in}{5.802740in}}%
\pgfpathlineto{\pgfqpoint{5.912132in}{5.798768in}}%
\pgfpathlineto{\pgfqpoint{5.913205in}{5.784864in}}%
\pgfpathlineto{\pgfqpoint{5.914279in}{5.803113in}}%
\pgfpathlineto{\pgfqpoint{5.918575in}{5.809196in}}%
\pgfpathlineto{\pgfqpoint{5.920722in}{5.820617in}}%
\pgfpathlineto{\pgfqpoint{5.921796in}{5.827196in}}%
\pgfpathlineto{\pgfqpoint{5.925018in}{5.837997in}}%
\pgfpathlineto{\pgfqpoint{5.927165in}{5.826327in}}%
\pgfpathlineto{\pgfqpoint{5.928239in}{5.835141in}}%
\pgfpathlineto{\pgfqpoint{5.929313in}{5.835017in}}%
\pgfpathlineto{\pgfqpoint{5.932534in}{5.837376in}}%
\pgfpathlineto{\pgfqpoint{5.933608in}{5.851652in}}%
\pgfpathlineto{\pgfqpoint{5.934682in}{5.855501in}}%
\pgfpathlineto{\pgfqpoint{5.935756in}{5.866798in}}%
\pgfpathlineto{\pgfqpoint{5.940051in}{5.877474in}}%
\pgfpathlineto{\pgfqpoint{5.941125in}{5.882564in}}%
\pgfpathlineto{\pgfqpoint{5.943273in}{5.874619in}}%
\pgfpathlineto{\pgfqpoint{5.944347in}{5.882936in}}%
\pgfpathlineto{\pgfqpoint{5.947568in}{5.884178in}}%
\pgfpathlineto{\pgfqpoint{5.948642in}{5.880205in}}%
\pgfpathlineto{\pgfqpoint{5.950790in}{5.890260in}}%
\pgfpathlineto{\pgfqpoint{5.951864in}{5.903544in}}%
\pgfpathlineto{\pgfqpoint{5.955085in}{5.903420in}}%
\pgfpathlineto{\pgfqpoint{5.956159in}{5.896468in}}%
\pgfpathlineto{\pgfqpoint{5.957233in}{5.896716in}}%
\pgfpathlineto{\pgfqpoint{5.958307in}{5.894730in}}%
\pgfpathlineto{\pgfqpoint{5.962602in}{5.892495in}}%
\pgfpathlineto{\pgfqpoint{5.963676in}{5.896343in}}%
\pgfpathlineto{\pgfqpoint{5.964750in}{5.892619in}}%
\pgfpathlineto{\pgfqpoint{5.965823in}{5.905654in}}%
\pgfpathlineto{\pgfqpoint{5.966897in}{5.902178in}}%
\pgfpathlineto{\pgfqpoint{5.970119in}{5.896592in}}%
\pgfpathlineto{\pgfqpoint{5.971193in}{5.891502in}}%
\pgfpathlineto{\pgfqpoint{5.972266in}{5.891254in}}%
\pgfpathlineto{\pgfqpoint{5.973340in}{5.879212in}}%
\pgfpathlineto{\pgfqpoint{5.974414in}{5.886784in}}%
\pgfpathlineto{\pgfqpoint{5.977636in}{5.890260in}}%
\pgfpathlineto{\pgfqpoint{5.978710in}{5.900688in}}%
\pgfpathlineto{\pgfqpoint{5.979783in}{5.918192in}}%
\pgfpathlineto{\pgfqpoint{5.980857in}{5.922289in}}%
\pgfpathlineto{\pgfqpoint{5.981931in}{5.917944in}}%
\pgfpathlineto{\pgfqpoint{5.985153in}{5.923282in}}%
\pgfpathlineto{\pgfqpoint{5.988374in}{5.958787in}}%
\pgfpathlineto{\pgfqpoint{5.989448in}{5.962635in}}%
\pgfpathlineto{\pgfqpoint{5.992669in}{5.959904in}}%
\pgfpathlineto{\pgfqpoint{5.993743in}{5.965863in}}%
\pgfpathlineto{\pgfqpoint{5.994817in}{5.966111in}}%
\pgfpathlineto{\pgfqpoint{5.996965in}{5.956677in}}%
\pgfpathlineto{\pgfqpoint{6.000186in}{5.959656in}}%
\pgfpathlineto{\pgfqpoint{6.002334in}{5.942897in}}%
\pgfpathlineto{\pgfqpoint{6.003408in}{5.938924in}}%
\pgfpathlineto{\pgfqpoint{6.004482in}{5.945876in}}%
\pgfpathlineto{\pgfqpoint{6.007703in}{5.939669in}}%
\pgfpathlineto{\pgfqpoint{6.008777in}{5.949725in}}%
\pgfpathlineto{\pgfqpoint{6.011999in}{5.941283in}}%
\pgfpathlineto{\pgfqpoint{6.015220in}{5.940662in}}%
\pgfpathlineto{\pgfqpoint{6.016294in}{5.925269in}}%
\pgfpathlineto{\pgfqpoint{6.017368in}{5.934207in}}%
\pgfpathlineto{\pgfqpoint{6.018442in}{5.925020in}}%
\pgfpathlineto{\pgfqpoint{6.019515in}{5.938924in}}%
\pgfpathlineto{\pgfqpoint{6.022737in}{5.935076in}}%
\pgfpathlineto{\pgfqpoint{6.023811in}{5.950097in}}%
\pgfpathlineto{\pgfqpoint{6.024885in}{5.955311in}}%
\pgfpathlineto{\pgfqpoint{6.025958in}{5.954318in}}%
\pgfpathlineto{\pgfqpoint{6.027032in}{5.957794in}}%
\pgfpathlineto{\pgfqpoint{6.031328in}{5.960649in}}%
\pgfpathlineto{\pgfqpoint{6.032401in}{5.962387in}}%
\pgfpathlineto{\pgfqpoint{6.033475in}{5.968222in}}%
\pgfpathlineto{\pgfqpoint{6.034549in}{5.957421in}}%
\pgfpathlineto{\pgfqpoint{6.037771in}{5.962760in}}%
\pgfpathlineto{\pgfqpoint{6.038844in}{5.962139in}}%
\pgfpathlineto{\pgfqpoint{6.039918in}{5.966484in}}%
\pgfpathlineto{\pgfqpoint{6.045287in}{5.935945in}}%
\pgfpathlineto{\pgfqpoint{6.046361in}{5.905282in}}%
\pgfpathlineto{\pgfqpoint{6.048509in}{5.915461in}}%
\pgfpathlineto{\pgfqpoint{6.049583in}{5.914592in}}%
\pgfpathlineto{\pgfqpoint{6.052804in}{5.919806in}}%
\pgfpathlineto{\pgfqpoint{6.053878in}{5.919682in}}%
\pgfpathlineto{\pgfqpoint{6.054952in}{5.915710in}}%
\pgfpathlineto{\pgfqpoint{6.056026in}{5.930979in}}%
\pgfpathlineto{\pgfqpoint{6.057100in}{5.893364in}}%
\pgfpathlineto{\pgfqpoint{6.060321in}{5.865184in}}%
\pgfpathlineto{\pgfqpoint{6.061395in}{5.867915in}}%
\pgfpathlineto{\pgfqpoint{6.063543in}{5.905282in}}%
\pgfpathlineto{\pgfqpoint{6.064617in}{5.904413in}}%
\pgfpathlineto{\pgfqpoint{6.068912in}{5.886164in}}%
\pgfpathlineto{\pgfqpoint{6.071060in}{5.893985in}}%
\pgfpathlineto{\pgfqpoint{6.072133in}{5.913847in}}%
\pgfpathlineto{\pgfqpoint{6.075355in}{5.921793in}}%
\pgfpathlineto{\pgfqpoint{6.076429in}{5.931848in}}%
\pgfpathlineto{\pgfqpoint{6.077503in}{5.932965in}}%
\pgfpathlineto{\pgfqpoint{6.078576in}{5.939048in}}%
\pgfpathlineto{\pgfqpoint{6.079650in}{5.941035in}}%
\pgfpathlineto{\pgfqpoint{6.082872in}{5.943517in}}%
\pgfpathlineto{\pgfqpoint{6.083946in}{5.945504in}}%
\pgfpathlineto{\pgfqpoint{6.085020in}{5.949352in}}%
\pgfpathlineto{\pgfqpoint{6.086093in}{5.936938in}}%
\pgfpathlineto{\pgfqpoint{6.087167in}{5.946993in}}%
\pgfpathlineto{\pgfqpoint{6.091463in}{5.947987in}}%
\pgfpathlineto{\pgfqpoint{6.093610in}{5.953076in}}%
\pgfpathlineto{\pgfqpoint{6.094684in}{5.949104in}}%
\pgfpathlineto{\pgfqpoint{6.097906in}{5.945255in}}%
\pgfpathlineto{\pgfqpoint{6.098979in}{5.937435in}}%
\pgfpathlineto{\pgfqpoint{6.100053in}{5.941655in}}%
\pgfpathlineto{\pgfqpoint{6.101127in}{5.942897in}}%
\pgfpathlineto{\pgfqpoint{6.102201in}{5.961642in}}%
\pgfpathlineto{\pgfqpoint{6.105422in}{5.965367in}}%
\pgfpathlineto{\pgfqpoint{6.107570in}{5.952456in}}%
\pgfpathlineto{\pgfqpoint{6.108644in}{5.961146in}}%
\pgfpathlineto{\pgfqpoint{6.109718in}{5.960028in}}%
\pgfpathlineto{\pgfqpoint{6.112939in}{5.963008in}}%
\pgfpathlineto{\pgfqpoint{6.114013in}{5.959035in}}%
\pgfpathlineto{\pgfqpoint{6.115087in}{5.963629in}}%
\pgfpathlineto{\pgfqpoint{6.117235in}{5.961766in}}%
\pgfpathlineto{\pgfqpoint{6.120456in}{5.959780in}}%
\pgfpathlineto{\pgfqpoint{6.121530in}{5.963629in}}%
\pgfpathlineto{\pgfqpoint{6.122604in}{5.957421in}}%
\pgfpathlineto{\pgfqpoint{6.124752in}{5.953076in}}%
\pgfpathlineto{\pgfqpoint{6.127973in}{5.961642in}}%
\pgfpathlineto{\pgfqpoint{6.129047in}{5.960897in}}%
\pgfpathlineto{\pgfqpoint{6.130121in}{5.962387in}}%
\pgfpathlineto{\pgfqpoint{6.131195in}{5.953821in}}%
\pgfpathlineto{\pgfqpoint{6.132268in}{5.957794in}}%
\pgfpathlineto{\pgfqpoint{6.136564in}{5.964498in}}%
\pgfpathlineto{\pgfqpoint{6.137638in}{5.969587in}}%
\pgfpathlineto{\pgfqpoint{6.138711in}{5.970208in}}%
\pgfpathlineto{\pgfqpoint{6.139785in}{5.956925in}}%
\pgfpathlineto{\pgfqpoint{6.143007in}{5.966235in}}%
\pgfpathlineto{\pgfqpoint{6.145154in}{5.937435in}}%
\pgfpathlineto{\pgfqpoint{6.146228in}{5.941655in}}%
\pgfpathlineto{\pgfqpoint{6.147302in}{5.939669in}}%
\pgfpathlineto{\pgfqpoint{6.150524in}{5.944386in}}%
\pgfpathlineto{\pgfqpoint{6.151598in}{5.940290in}}%
\pgfpathlineto{\pgfqpoint{6.153745in}{5.951090in}}%
\pgfpathlineto{\pgfqpoint{6.154819in}{5.941904in}}%
\pgfpathlineto{\pgfqpoint{6.158041in}{5.936814in}}%
\pgfpathlineto{\pgfqpoint{6.159114in}{5.946993in}}%
\pgfpathlineto{\pgfqpoint{6.160188in}{5.946249in}}%
\pgfpathlineto{\pgfqpoint{6.161262in}{5.936193in}}%
\pgfpathlineto{\pgfqpoint{6.162336in}{5.944138in}}%
\pgfpathlineto{\pgfqpoint{6.165557in}{5.941407in}}%
\pgfpathlineto{\pgfqpoint{6.166631in}{5.942648in}}%
\pgfpathlineto{\pgfqpoint{6.167705in}{5.951711in}}%
\pgfpathlineto{\pgfqpoint{6.168779in}{5.922910in}}%
\pgfpathlineto{\pgfqpoint{6.169853in}{5.920799in}}%
\pgfpathlineto{\pgfqpoint{6.173074in}{5.922413in}}%
\pgfpathlineto{\pgfqpoint{6.174148in}{5.910123in}}%
\pgfpathlineto{\pgfqpoint{6.175222in}{5.908013in}}%
\pgfpathlineto{\pgfqpoint{6.177370in}{5.901557in}}%
\pgfpathlineto{\pgfqpoint{6.180591in}{5.898578in}}%
\pgfpathlineto{\pgfqpoint{6.181665in}{5.900813in}}%
\pgfpathlineto{\pgfqpoint{6.182739in}{5.914716in}}%
\pgfpathlineto{\pgfqpoint{6.183813in}{5.980636in}}%
\pgfpathlineto{\pgfqpoint{6.184887in}{5.987464in}}%
\pgfpathlineto{\pgfqpoint{6.188108in}{5.984236in}}%
\pgfpathlineto{\pgfqpoint{6.189182in}{5.980139in}}%
\pgfpathlineto{\pgfqpoint{6.190256in}{5.980884in}}%
\pgfpathlineto{\pgfqpoint{6.191330in}{5.982374in}}%
\pgfpathlineto{\pgfqpoint{6.192403in}{5.976663in}}%
\pgfpathlineto{\pgfqpoint{6.195625in}{5.976291in}}%
\pgfpathlineto{\pgfqpoint{6.196699in}{5.974305in}}%
\pgfpathlineto{\pgfqpoint{6.197773in}{5.964746in}}%
\pgfpathlineto{\pgfqpoint{6.198846in}{5.963380in}}%
\pgfpathlineto{\pgfqpoint{6.199920in}{5.965491in}}%
\pgfpathlineto{\pgfqpoint{6.203142in}{5.983243in}}%
\pgfpathlineto{\pgfqpoint{6.204216in}{5.983988in}}%
\pgfpathlineto{\pgfqpoint{6.206363in}{6.020237in}}%
\pgfpathlineto{\pgfqpoint{6.207437in}{6.024955in}}%
\pgfpathlineto{\pgfqpoint{6.210659in}{6.047797in}}%
\pgfpathlineto{\pgfqpoint{6.211732in}{6.048418in}}%
\pgfpathlineto{\pgfqpoint{6.212806in}{6.038983in}}%
\pgfpathlineto{\pgfqpoint{6.213880in}{6.040100in}}%
\pgfpathlineto{\pgfqpoint{6.214954in}{6.030914in}}%
\pgfpathlineto{\pgfqpoint{6.219249in}{6.039479in}}%
\pgfpathlineto{\pgfqpoint{6.220323in}{6.053259in}}%
\pgfpathlineto{\pgfqpoint{6.222471in}{6.053011in}}%
\pgfpathlineto{\pgfqpoint{6.225692in}{6.044321in}}%
\pgfpathlineto{\pgfqpoint{6.226766in}{6.036748in}}%
\pgfpathlineto{\pgfqpoint{6.228914in}{6.049038in}}%
\pgfpathlineto{\pgfqpoint{6.229988in}{6.041093in}}%
\pgfpathlineto{\pgfqpoint{6.233209in}{6.043080in}}%
\pgfpathlineto{\pgfqpoint{6.234283in}{6.046431in}}%
\pgfpathlineto{\pgfqpoint{6.235357in}{6.070018in}}%
\pgfpathlineto{\pgfqpoint{6.236431in}{6.077467in}}%
\pgfpathlineto{\pgfqpoint{6.237505in}{6.075729in}}%
\pgfpathlineto{\pgfqpoint{6.240726in}{6.061577in}}%
\pgfpathlineto{\pgfqpoint{6.242874in}{6.067411in}}%
\pgfpathlineto{\pgfqpoint{6.243948in}{6.077839in}}%
\pgfpathlineto{\pgfqpoint{6.245021in}{6.078460in}}%
\pgfpathlineto{\pgfqpoint{6.248243in}{6.073122in}}%
\pgfpathlineto{\pgfqpoint{6.250391in}{6.082308in}}%
\pgfpathlineto{\pgfqpoint{6.251464in}{6.073494in}}%
\pgfpathlineto{\pgfqpoint{6.252538in}{6.078212in}}%
\pgfpathlineto{\pgfqpoint{6.256834in}{6.078336in}}%
\pgfpathlineto{\pgfqpoint{6.258981in}{6.065673in}}%
\pgfpathlineto{\pgfqpoint{6.260055in}{6.067536in}}%
\pgfpathlineto{\pgfqpoint{6.264351in}{6.082681in}}%
\pgfpathlineto{\pgfqpoint{6.265424in}{6.097454in}}%
\pgfpathlineto{\pgfqpoint{6.266498in}{6.086157in}}%
\pgfpathlineto{\pgfqpoint{6.270794in}{6.092612in}}%
\pgfpathlineto{\pgfqpoint{6.271867in}{6.102047in}}%
\pgfpathlineto{\pgfqpoint{6.272941in}{6.105151in}}%
\pgfpathlineto{\pgfqpoint{6.274015in}{6.104778in}}%
\pgfpathlineto{\pgfqpoint{6.275089in}{6.101675in}}%
\pgfpathlineto{\pgfqpoint{6.279384in}{6.101426in}}%
\pgfpathlineto{\pgfqpoint{6.280458in}{6.112103in}}%
\pgfpathlineto{\pgfqpoint{6.282606in}{6.096709in}}%
\pgfpathlineto{\pgfqpoint{6.285827in}{6.093978in}}%
\pgfpathlineto{\pgfqpoint{6.286901in}{6.111358in}}%
\pgfpathlineto{\pgfqpoint{6.287975in}{6.104902in}}%
\pgfpathlineto{\pgfqpoint{6.289049in}{6.105399in}}%
\pgfpathlineto{\pgfqpoint{6.290123in}{6.104406in}}%
\pgfpathlineto{\pgfqpoint{6.293344in}{6.109620in}}%
\pgfpathlineto{\pgfqpoint{6.294418in}{6.098819in}}%
\pgfpathlineto{\pgfqpoint{6.295492in}{6.103413in}}%
\pgfpathlineto{\pgfqpoint{6.296566in}{6.100433in}}%
\pgfpathlineto{\pgfqpoint{6.297640in}{6.118682in}}%
\pgfpathlineto{\pgfqpoint{6.301935in}{6.114834in}}%
\pgfpathlineto{\pgfqpoint{6.303009in}{6.115827in}}%
\pgfpathlineto{\pgfqpoint{6.305156in}{6.124020in}}%
\pgfpathlineto{\pgfqpoint{6.308378in}{6.129110in}}%
\pgfpathlineto{\pgfqpoint{6.309452in}{6.135069in}}%
\pgfpathlineto{\pgfqpoint{6.310526in}{6.137428in}}%
\pgfpathlineto{\pgfqpoint{6.311599in}{6.136310in}}%
\pgfpathlineto{\pgfqpoint{6.312673in}{6.138669in}}%
\pgfpathlineto{\pgfqpoint{6.316969in}{6.141773in}}%
\pgfpathlineto{\pgfqpoint{6.318042in}{6.140531in}}%
\pgfpathlineto{\pgfqpoint{6.319116in}{6.142766in}}%
\pgfpathlineto{\pgfqpoint{6.320190in}{6.139290in}}%
\pgfpathlineto{\pgfqpoint{6.323412in}{6.144131in}}%
\pgfpathlineto{\pgfqpoint{6.324486in}{6.142890in}}%
\pgfpathlineto{\pgfqpoint{6.325559in}{6.165111in}}%
\pgfpathlineto{\pgfqpoint{6.326633in}{6.143386in}}%
\pgfpathlineto{\pgfqpoint{6.327707in}{6.140779in}}%
\pgfpathlineto{\pgfqpoint{6.330929in}{6.136186in}}%
\pgfpathlineto{\pgfqpoint{6.332002in}{6.137179in}}%
\pgfpathlineto{\pgfqpoint{6.333076in}{6.130724in}}%
\pgfpathlineto{\pgfqpoint{6.334150in}{6.133827in}}%
\pgfpathlineto{\pgfqpoint{6.335224in}{6.134696in}}%
\pgfpathlineto{\pgfqpoint{6.338445in}{6.132710in}}%
\pgfpathlineto{\pgfqpoint{6.339519in}{6.138421in}}%
\pgfpathlineto{\pgfqpoint{6.340593in}{6.132958in}}%
\pgfpathlineto{\pgfqpoint{6.341667in}{6.139414in}}%
\pgfpathlineto{\pgfqpoint{6.342741in}{6.133207in}}%
\pgfpathlineto{\pgfqpoint{6.345962in}{6.128365in}}%
\pgfpathlineto{\pgfqpoint{6.347036in}{6.112599in}}%
\pgfpathlineto{\pgfqpoint{6.349184in}{6.116323in}}%
\pgfpathlineto{\pgfqpoint{6.350258in}{6.120668in}}%
\pgfpathlineto{\pgfqpoint{6.353479in}{6.113468in}}%
\pgfpathlineto{\pgfqpoint{6.354553in}{6.125882in}}%
\pgfpathlineto{\pgfqpoint{6.355627in}{6.121165in}}%
\pgfpathlineto{\pgfqpoint{6.356701in}{6.132710in}}%
\pgfpathlineto{\pgfqpoint{6.357775in}{6.131469in}}%
\pgfpathlineto{\pgfqpoint{6.360996in}{6.125262in}}%
\pgfpathlineto{\pgfqpoint{6.363144in}{6.119179in}}%
\pgfpathlineto{\pgfqpoint{6.364218in}{6.121041in}}%
\pgfpathlineto{\pgfqpoint{6.365291in}{6.119303in}}%
\pgfpathlineto{\pgfqpoint{6.368513in}{6.115951in}}%
\pgfpathlineto{\pgfqpoint{6.369587in}{6.113220in}}%
\pgfpathlineto{\pgfqpoint{6.371734in}{6.095592in}}%
\pgfpathlineto{\pgfqpoint{6.376030in}{6.106020in}}%
\pgfpathlineto{\pgfqpoint{6.377104in}{6.095468in}}%
\pgfpathlineto{\pgfqpoint{6.378177in}{6.092612in}}%
\pgfpathlineto{\pgfqpoint{6.379251in}{6.146366in}}%
\pgfpathlineto{\pgfqpoint{6.380325in}{6.141152in}}%
\pgfpathlineto{\pgfqpoint{6.384620in}{6.153690in}}%
\pgfpathlineto{\pgfqpoint{6.386768in}{6.150090in}}%
\pgfpathlineto{\pgfqpoint{6.387842in}{6.137055in}}%
\pgfpathlineto{\pgfqpoint{6.391064in}{6.136807in}}%
\pgfpathlineto{\pgfqpoint{6.392137in}{6.140531in}}%
\pgfpathlineto{\pgfqpoint{6.394285in}{6.126007in}}%
\pgfpathlineto{\pgfqpoint{6.395359in}{6.125882in}}%
\pgfpathlineto{\pgfqpoint{6.398580in}{6.124020in}}%
\pgfpathlineto{\pgfqpoint{6.400728in}{6.129855in}}%
\pgfpathlineto{\pgfqpoint{6.402876in}{6.115951in}}%
\pgfpathlineto{\pgfqpoint{6.406097in}{6.126007in}}%
\pgfpathlineto{\pgfqpoint{6.407171in}{6.123648in}}%
\pgfpathlineto{\pgfqpoint{6.408245in}{6.102420in}}%
\pgfpathlineto{\pgfqpoint{6.409319in}{6.102544in}}%
\pgfpathlineto{\pgfqpoint{6.410393in}{6.107633in}}%
\pgfpathlineto{\pgfqpoint{6.413614in}{6.109744in}}%
\pgfpathlineto{\pgfqpoint{6.414688in}{6.112475in}}%
\pgfpathlineto{\pgfqpoint{6.415762in}{6.111482in}}%
\pgfpathlineto{\pgfqpoint{6.416836in}{6.115330in}}%
\pgfpathlineto{\pgfqpoint{6.417909in}{6.115579in}}%
\pgfpathlineto{\pgfqpoint{6.423279in}{6.109371in}}%
\pgfpathlineto{\pgfqpoint{6.424352in}{6.125386in}}%
\pgfpathlineto{\pgfqpoint{6.425426in}{6.127993in}}%
\pgfpathlineto{\pgfqpoint{6.428648in}{6.133703in}}%
\pgfpathlineto{\pgfqpoint{6.429722in}{6.132214in}}%
\pgfpathlineto{\pgfqpoint{6.430796in}{6.143759in}}%
\pgfpathlineto{\pgfqpoint{6.431869in}{6.145497in}}%
\pgfpathlineto{\pgfqpoint{6.432943in}{6.149842in}}%
\pgfpathlineto{\pgfqpoint{6.436165in}{6.148104in}}%
\pgfpathlineto{\pgfqpoint{6.438312in}{6.156173in}}%
\pgfpathlineto{\pgfqpoint{6.439386in}{6.154435in}}%
\pgfpathlineto{\pgfqpoint{6.440460in}{6.163497in}}%
\pgfpathlineto{\pgfqpoint{6.443682in}{6.168711in}}%
\pgfpathlineto{\pgfqpoint{6.444755in}{6.176284in}}%
\pgfpathlineto{\pgfqpoint{6.445829in}{6.172684in}}%
\pgfpathlineto{\pgfqpoint{6.446903in}{6.173056in}}%
\pgfpathlineto{\pgfqpoint{6.447977in}{6.172808in}}%
\pgfpathlineto{\pgfqpoint{6.452272in}{6.183112in}}%
\pgfpathlineto{\pgfqpoint{6.453346in}{6.193788in}}%
\pgfpathlineto{\pgfqpoint{6.454420in}{6.189691in}}%
\pgfpathlineto{\pgfqpoint{6.455494in}{6.197016in}}%
\pgfpathlineto{\pgfqpoint{6.458715in}{6.207320in}}%
\pgfpathlineto{\pgfqpoint{6.460863in}{6.208685in}}%
\pgfpathlineto{\pgfqpoint{6.461937in}{6.194781in}}%
\pgfpathlineto{\pgfqpoint{6.463011in}{6.201982in}}%
\pgfpathlineto{\pgfqpoint{6.466232in}{6.201609in}}%
\pgfpathlineto{\pgfqpoint{6.467306in}{6.199871in}}%
\pgfpathlineto{\pgfqpoint{6.469454in}{6.214520in}}%
\pgfpathlineto{\pgfqpoint{6.470528in}{6.213403in}}%
\pgfpathlineto{\pgfqpoint{6.473749in}{6.212658in}}%
\pgfpathlineto{\pgfqpoint{6.475897in}{6.221224in}}%
\pgfpathlineto{\pgfqpoint{6.476971in}{6.214272in}}%
\pgfpathlineto{\pgfqpoint{6.478044in}{6.217127in}}%
\pgfpathlineto{\pgfqpoint{6.481266in}{6.210051in}}%
\pgfpathlineto{\pgfqpoint{6.482340in}{6.214892in}}%
\pgfpathlineto{\pgfqpoint{6.483414in}{6.213651in}}%
\pgfpathlineto{\pgfqpoint{6.484487in}{6.196147in}}%
\pgfpathlineto{\pgfqpoint{6.485561in}{6.207568in}}%
\pgfpathlineto{\pgfqpoint{6.488783in}{6.212782in}}%
\pgfpathlineto{\pgfqpoint{6.489857in}{6.212906in}}%
\pgfpathlineto{\pgfqpoint{6.490930in}{6.213651in}}%
\pgfpathlineto{\pgfqpoint{6.492004in}{6.216506in}}%
\pgfpathlineto{\pgfqpoint{6.493078in}{6.221720in}}%
\pgfpathlineto{\pgfqpoint{6.496300in}{6.220230in}}%
\pgfpathlineto{\pgfqpoint{6.497374in}{6.221348in}}%
\pgfpathlineto{\pgfqpoint{6.498447in}{6.218368in}}%
\pgfpathlineto{\pgfqpoint{6.499521in}{6.204713in}}%
\pgfpathlineto{\pgfqpoint{6.500595in}{6.201485in}}%
\pgfpathlineto{\pgfqpoint{6.503817in}{6.215637in}}%
\pgfpathlineto{\pgfqpoint{6.504890in}{6.231652in}}%
\pgfpathlineto{\pgfqpoint{6.505964in}{6.238852in}}%
\pgfpathlineto{\pgfqpoint{6.508112in}{6.214644in}}%
\pgfpathlineto{\pgfqpoint{6.513481in}{6.213403in}}%
\pgfpathlineto{\pgfqpoint{6.515629in}{6.215637in}}%
\pgfpathlineto{\pgfqpoint{6.519924in}{6.215016in}}%
\pgfpathlineto{\pgfqpoint{6.523146in}{6.223706in}}%
\pgfpathlineto{\pgfqpoint{6.527441in}{6.213775in}}%
\pgfpathlineto{\pgfqpoint{6.528515in}{6.212782in}}%
\pgfpathlineto{\pgfqpoint{6.529589in}{6.202851in}}%
\pgfpathlineto{\pgfqpoint{6.530663in}{6.200988in}}%
\pgfpathlineto{\pgfqpoint{6.533884in}{6.218368in}}%
\pgfpathlineto{\pgfqpoint{6.534958in}{6.228672in}}%
\pgfpathlineto{\pgfqpoint{6.536032in}{6.229789in}}%
\pgfpathlineto{\pgfqpoint{6.537106in}{6.224327in}}%
\pgfpathlineto{\pgfqpoint{6.538179in}{6.234010in}}%
\pgfpathlineto{\pgfqpoint{6.541401in}{6.244562in}}%
\pgfpathlineto{\pgfqpoint{6.542475in}{6.258218in}}%
\pgfpathlineto{\pgfqpoint{6.543549in}{6.251390in}}%
\pgfpathlineto{\pgfqpoint{6.545696in}{6.251018in}}%
\pgfpathlineto{\pgfqpoint{6.548918in}{6.248659in}}%
\pgfpathlineto{\pgfqpoint{6.549992in}{6.254494in}}%
\pgfpathlineto{\pgfqpoint{6.552139in}{6.271998in}}%
\pgfpathlineto{\pgfqpoint{6.553213in}{6.275846in}}%
\pgfpathlineto{\pgfqpoint{6.556435in}{6.276839in}}%
\pgfpathlineto{\pgfqpoint{6.557508in}{6.287516in}}%
\pgfpathlineto{\pgfqpoint{6.558582in}{6.282426in}}%
\pgfpathlineto{\pgfqpoint{6.560730in}{6.293226in}}%
\pgfpathlineto{\pgfqpoint{6.563952in}{6.294964in}}%
\pgfpathlineto{\pgfqpoint{6.565025in}{6.297323in}}%
\pgfpathlineto{\pgfqpoint{6.566099in}{6.298192in}}%
\pgfpathlineto{\pgfqpoint{6.567173in}{6.293971in}}%
\pgfpathlineto{\pgfqpoint{6.568247in}{6.309116in}}%
\pgfpathlineto{\pgfqpoint{6.572542in}{6.294964in}}%
\pgfpathlineto{\pgfqpoint{6.573616in}{6.299681in}}%
\pgfpathlineto{\pgfqpoint{6.574690in}{6.297447in}}%
\pgfpathlineto{\pgfqpoint{6.575764in}{6.299806in}}%
\pgfpathlineto{\pgfqpoint{6.578985in}{6.303282in}}%
\pgfpathlineto{\pgfqpoint{6.580059in}{6.321158in}}%
\pgfpathlineto{\pgfqpoint{6.581133in}{6.317186in}}%
\pgfpathlineto{\pgfqpoint{6.582207in}{6.343380in}}%
\pgfpathlineto{\pgfqpoint{6.583281in}{6.344621in}}%
\pgfpathlineto{\pgfqpoint{6.586502in}{6.335807in}}%
\pgfpathlineto{\pgfqpoint{6.588650in}{6.344621in}}%
\pgfpathlineto{\pgfqpoint{6.589724in}{6.346856in}}%
\pgfpathlineto{\pgfqpoint{6.590797in}{6.352318in}}%
\pgfpathlineto{\pgfqpoint{6.594019in}{6.350580in}}%
\pgfpathlineto{\pgfqpoint{6.595093in}{6.339531in}}%
\pgfpathlineto{\pgfqpoint{6.596167in}{6.336552in}}%
\pgfpathlineto{\pgfqpoint{6.597240in}{6.319917in}}%
\pgfpathlineto{\pgfqpoint{6.598314in}{6.317061in}}%
\pgfpathlineto{\pgfqpoint{6.601536in}{6.321655in}}%
\pgfpathlineto{\pgfqpoint{6.602610in}{6.320041in}}%
\pgfpathlineto{\pgfqpoint{6.603684in}{6.313958in}}%
\pgfpathlineto{\pgfqpoint{6.604757in}{6.317558in}}%
\pgfpathlineto{\pgfqpoint{6.605831in}{6.319172in}}%
\pgfpathlineto{\pgfqpoint{6.609053in}{6.322275in}}%
\pgfpathlineto{\pgfqpoint{6.610127in}{6.327862in}}%
\pgfpathlineto{\pgfqpoint{6.611200in}{6.320786in}}%
\pgfpathlineto{\pgfqpoint{6.613348in}{6.316565in}}%
\pgfpathlineto{\pgfqpoint{6.616570in}{6.316441in}}%
\pgfpathlineto{\pgfqpoint{6.618717in}{6.354428in}}%
\pgfpathlineto{\pgfqpoint{6.619791in}{6.367836in}}%
\pgfpathlineto{\pgfqpoint{6.620865in}{6.369698in}}%
\pgfpathlineto{\pgfqpoint{6.625160in}{6.379877in}}%
\pgfpathlineto{\pgfqpoint{6.626234in}{6.373919in}}%
\pgfpathlineto{\pgfqpoint{6.627308in}{6.378388in}}%
\pgfpathlineto{\pgfqpoint{6.628382in}{6.378015in}}%
\pgfpathlineto{\pgfqpoint{6.631603in}{6.383477in}}%
\pgfpathlineto{\pgfqpoint{6.632677in}{6.387947in}}%
\pgfpathlineto{\pgfqpoint{6.633751in}{6.368705in}}%
\pgfpathlineto{\pgfqpoint{6.634825in}{6.361008in}}%
\pgfpathlineto{\pgfqpoint{6.635899in}{6.377643in}}%
\pgfpathlineto{\pgfqpoint{6.639120in}{6.391671in}}%
\pgfpathlineto{\pgfqpoint{6.641268in}{6.377519in}}%
\pgfpathlineto{\pgfqpoint{6.642342in}{6.377394in}}%
\pgfpathlineto{\pgfqpoint{6.643416in}{6.380250in}}%
\pgfpathlineto{\pgfqpoint{6.647711in}{6.378263in}}%
\pgfpathlineto{\pgfqpoint{6.649859in}{6.391919in}}%
\pgfpathlineto{\pgfqpoint{6.650932in}{6.387202in}}%
\pgfpathlineto{\pgfqpoint{6.650932in}{6.387202in}}%
\pgfusepath{stroke}%
\end{pgfscope}%
\begin{pgfscope}%
\pgfpathrectangle{\pgfqpoint{4.185023in}{5.688815in}}{\pgfqpoint{2.583333in}{0.736585in}}%
\pgfusepath{clip}%
\pgfsetroundcap%
\pgfsetroundjoin%
\pgfsetlinewidth{1.505625pt}%
\definecolor{currentstroke}{rgb}{1.000000,0.647059,0.000000}%
\pgfsetstrokecolor{currentstroke}%
\pgfsetdash{}{0pt}%
\pgfpathmoveto{\pgfqpoint{4.302448in}{5.722296in}}%
\pgfpathlineto{\pgfqpoint{4.303521in}{5.722482in}}%
\pgfpathlineto{\pgfqpoint{4.304595in}{5.724613in}}%
\pgfpathlineto{\pgfqpoint{4.305669in}{5.724189in}}%
\pgfpathlineto{\pgfqpoint{4.309964in}{5.724738in}}%
\pgfpathlineto{\pgfqpoint{4.311038in}{5.725559in}}%
\pgfpathlineto{\pgfqpoint{4.313186in}{5.728531in}}%
\pgfpathlineto{\pgfqpoint{4.317481in}{5.730142in}}%
\pgfpathlineto{\pgfqpoint{4.320703in}{5.734261in}}%
\pgfpathlineto{\pgfqpoint{4.326072in}{5.734943in}}%
\pgfpathlineto{\pgfqpoint{4.328220in}{5.735648in}}%
\pgfpathlineto{\pgfqpoint{4.332515in}{5.735896in}}%
\pgfpathlineto{\pgfqpoint{4.334663in}{5.737374in}}%
\pgfpathlineto{\pgfqpoint{4.335737in}{5.738677in}}%
\pgfpathlineto{\pgfqpoint{4.338958in}{5.739666in}}%
\pgfpathlineto{\pgfqpoint{4.343253in}{5.743174in}}%
\pgfpathlineto{\pgfqpoint{4.347549in}{5.744609in}}%
\pgfpathlineto{\pgfqpoint{4.349696in}{5.745988in}}%
\pgfpathlineto{\pgfqpoint{4.350770in}{5.746839in}}%
\pgfpathlineto{\pgfqpoint{4.356140in}{5.748369in}}%
\pgfpathlineto{\pgfqpoint{4.358287in}{5.749846in}}%
\pgfpathlineto{\pgfqpoint{4.362583in}{5.751724in}}%
\pgfpathlineto{\pgfqpoint{4.365804in}{5.753553in}}%
\pgfpathlineto{\pgfqpoint{4.372247in}{5.754972in}}%
\pgfpathlineto{\pgfqpoint{4.373321in}{5.755461in}}%
\pgfpathlineto{\pgfqpoint{4.376542in}{5.755827in}}%
\pgfpathlineto{\pgfqpoint{4.378690in}{5.757602in}}%
\pgfpathlineto{\pgfqpoint{4.380838in}{5.759856in}}%
\pgfpathlineto{\pgfqpoint{4.384059in}{5.761052in}}%
\pgfpathlineto{\pgfqpoint{4.388355in}{5.765302in}}%
\pgfpathlineto{\pgfqpoint{4.391576in}{5.766568in}}%
\pgfpathlineto{\pgfqpoint{4.395872in}{5.770936in}}%
\pgfpathlineto{\pgfqpoint{4.399093in}{5.771898in}}%
\pgfpathlineto{\pgfqpoint{4.402315in}{5.774680in}}%
\pgfpathlineto{\pgfqpoint{4.407684in}{5.775977in}}%
\pgfpathlineto{\pgfqpoint{4.410905in}{5.778139in}}%
\pgfpathlineto{\pgfqpoint{4.415201in}{5.779660in}}%
\pgfpathlineto{\pgfqpoint{4.418422in}{5.781651in}}%
\pgfpathlineto{\pgfqpoint{4.422717in}{5.782833in}}%
\pgfpathlineto{\pgfqpoint{4.425939in}{5.785359in}}%
\pgfpathlineto{\pgfqpoint{4.429161in}{5.786249in}}%
\pgfpathlineto{\pgfqpoint{4.433456in}{5.789880in}}%
\pgfpathlineto{\pgfqpoint{4.437751in}{5.791381in}}%
\pgfpathlineto{\pgfqpoint{4.440973in}{5.793330in}}%
\pgfpathlineto{\pgfqpoint{4.462450in}{5.796156in}}%
\pgfpathlineto{\pgfqpoint{4.468893in}{5.795918in}}%
\pgfpathlineto{\pgfqpoint{4.485000in}{5.797304in}}%
\pgfpathlineto{\pgfqpoint{4.493591in}{5.798398in}}%
\pgfpathlineto{\pgfqpoint{4.500034in}{5.799583in}}%
\pgfpathlineto{\pgfqpoint{4.501108in}{5.799919in}}%
\pgfpathlineto{\pgfqpoint{4.507551in}{5.800984in}}%
\pgfpathlineto{\pgfqpoint{4.515068in}{5.802227in}}%
\pgfpathlineto{\pgfqpoint{4.530101in}{5.803648in}}%
\pgfpathlineto{\pgfqpoint{4.537618in}{5.804459in}}%
\pgfpathlineto{\pgfqpoint{4.558021in}{5.805893in}}%
\pgfpathlineto{\pgfqpoint{4.568760in}{5.806964in}}%
\pgfpathlineto{\pgfqpoint{4.576276in}{5.807796in}}%
\pgfpathlineto{\pgfqpoint{4.604196in}{5.810288in}}%
\pgfpathlineto{\pgfqpoint{4.613861in}{5.811272in}}%
\pgfpathlineto{\pgfqpoint{4.642854in}{5.811305in}}%
\pgfpathlineto{\pgfqpoint{4.651445in}{5.811295in}}%
\pgfpathlineto{\pgfqpoint{4.677217in}{5.811839in}}%
\pgfpathlineto{\pgfqpoint{4.689029in}{5.812397in}}%
\pgfpathlineto{\pgfqpoint{4.699768in}{5.813149in}}%
\pgfpathlineto{\pgfqpoint{4.711580in}{5.815095in}}%
\pgfpathlineto{\pgfqpoint{4.723392in}{5.815981in}}%
\pgfpathlineto{\pgfqpoint{4.762050in}{5.821663in}}%
\pgfpathlineto{\pgfqpoint{4.764198in}{5.822296in}}%
\pgfpathlineto{\pgfqpoint{4.769567in}{5.823313in}}%
\pgfpathlineto{\pgfqpoint{4.771715in}{5.824007in}}%
\pgfpathlineto{\pgfqpoint{4.777084in}{5.825020in}}%
\pgfpathlineto{\pgfqpoint{4.779232in}{5.825697in}}%
\pgfpathlineto{\pgfqpoint{4.784601in}{5.826806in}}%
\pgfpathlineto{\pgfqpoint{4.785675in}{5.827195in}}%
\pgfpathlineto{\pgfqpoint{4.792118in}{5.828308in}}%
\pgfpathlineto{\pgfqpoint{4.794266in}{5.828974in}}%
\pgfpathlineto{\pgfqpoint{4.799635in}{5.829892in}}%
\pgfpathlineto{\pgfqpoint{4.801782in}{5.830516in}}%
\pgfpathlineto{\pgfqpoint{4.808226in}{5.831567in}}%
\pgfpathlineto{\pgfqpoint{4.809299in}{5.831922in}}%
\pgfpathlineto{\pgfqpoint{4.814669in}{5.832969in}}%
\pgfpathlineto{\pgfqpoint{4.816816in}{5.833709in}}%
\pgfpathlineto{\pgfqpoint{4.822185in}{5.834818in}}%
\pgfpathlineto{\pgfqpoint{4.824333in}{5.835662in}}%
\pgfpathlineto{\pgfqpoint{4.828628in}{5.836518in}}%
\pgfpathlineto{\pgfqpoint{4.831850in}{5.837790in}}%
\pgfpathlineto{\pgfqpoint{4.836145in}{5.838655in}}%
\pgfpathlineto{\pgfqpoint{4.839367in}{5.840143in}}%
\pgfpathlineto{\pgfqpoint{4.843662in}{5.841250in}}%
\pgfpathlineto{\pgfqpoint{4.846884in}{5.842892in}}%
\pgfpathlineto{\pgfqpoint{4.852253in}{5.844047in}}%
\pgfpathlineto{\pgfqpoint{4.854401in}{5.845186in}}%
\pgfpathlineto{\pgfqpoint{4.858696in}{5.846333in}}%
\pgfpathlineto{\pgfqpoint{4.861917in}{5.848037in}}%
\pgfpathlineto{\pgfqpoint{4.866213in}{5.849223in}}%
\pgfpathlineto{\pgfqpoint{4.869434in}{5.850689in}}%
\pgfpathlineto{\pgfqpoint{4.873730in}{5.851659in}}%
\pgfpathlineto{\pgfqpoint{4.876951in}{5.853011in}}%
\pgfpathlineto{\pgfqpoint{4.881247in}{5.853846in}}%
\pgfpathlineto{\pgfqpoint{4.884468in}{5.855262in}}%
\pgfpathlineto{\pgfqpoint{4.888763in}{5.856233in}}%
\pgfpathlineto{\pgfqpoint{4.891985in}{5.857219in}}%
\pgfpathlineto{\pgfqpoint{4.896280in}{5.858308in}}%
\pgfpathlineto{\pgfqpoint{4.899502in}{5.859908in}}%
\pgfpathlineto{\pgfqpoint{4.903797in}{5.861012in}}%
\pgfpathlineto{\pgfqpoint{4.907019in}{5.862374in}}%
\pgfpathlineto{\pgfqpoint{4.911314in}{5.863247in}}%
\pgfpathlineto{\pgfqpoint{4.914536in}{5.864620in}}%
\pgfpathlineto{\pgfqpoint{4.919905in}{5.865902in}}%
\pgfpathlineto{\pgfqpoint{4.922052in}{5.866803in}}%
\pgfpathlineto{\pgfqpoint{4.926348in}{5.867706in}}%
\pgfpathlineto{\pgfqpoint{4.929569in}{5.869023in}}%
\pgfpathlineto{\pgfqpoint{4.933865in}{5.869903in}}%
\pgfpathlineto{\pgfqpoint{4.937086in}{5.871162in}}%
\pgfpathlineto{\pgfqpoint{4.942455in}{5.872295in}}%
\pgfpathlineto{\pgfqpoint{4.944603in}{5.873027in}}%
\pgfpathlineto{\pgfqpoint{4.949972in}{5.874010in}}%
\pgfpathlineto{\pgfqpoint{4.952120in}{5.874633in}}%
\pgfpathlineto{\pgfqpoint{4.958563in}{5.875642in}}%
\pgfpathlineto{\pgfqpoint{4.959637in}{5.875981in}}%
\pgfpathlineto{\pgfqpoint{4.965006in}{5.877080in}}%
\pgfpathlineto{\pgfqpoint{4.967154in}{5.877847in}}%
\pgfpathlineto{\pgfqpoint{4.971449in}{5.878659in}}%
\pgfpathlineto{\pgfqpoint{4.974670in}{5.879961in}}%
\pgfpathlineto{\pgfqpoint{4.980040in}{5.881140in}}%
\pgfpathlineto{\pgfqpoint{4.982187in}{5.881913in}}%
\pgfpathlineto{\pgfqpoint{4.987557in}{5.883007in}}%
\pgfpathlineto{\pgfqpoint{4.989704in}{5.883675in}}%
\pgfpathlineto{\pgfqpoint{4.995073in}{5.884535in}}%
\pgfpathlineto{\pgfqpoint{4.997221in}{5.885232in}}%
\pgfpathlineto{\pgfqpoint{5.002590in}{5.886332in}}%
\pgfpathlineto{\pgfqpoint{5.004738in}{5.887284in}}%
\pgfpathlineto{\pgfqpoint{5.009033in}{5.888241in}}%
\pgfpathlineto{\pgfqpoint{5.012255in}{5.889716in}}%
\pgfpathlineto{\pgfqpoint{5.016550in}{5.890780in}}%
\pgfpathlineto{\pgfqpoint{5.019772in}{5.892294in}}%
\pgfpathlineto{\pgfqpoint{5.024067in}{5.893271in}}%
\pgfpathlineto{\pgfqpoint{5.027289in}{5.894699in}}%
\pgfpathlineto{\pgfqpoint{5.031584in}{5.895622in}}%
\pgfpathlineto{\pgfqpoint{5.034805in}{5.897051in}}%
\pgfpathlineto{\pgfqpoint{5.039101in}{5.898011in}}%
\pgfpathlineto{\pgfqpoint{5.042322in}{5.899505in}}%
\pgfpathlineto{\pgfqpoint{5.046618in}{5.900560in}}%
\pgfpathlineto{\pgfqpoint{5.049839in}{5.901648in}}%
\pgfpathlineto{\pgfqpoint{5.054135in}{5.902691in}}%
\pgfpathlineto{\pgfqpoint{5.057356in}{5.904243in}}%
\pgfpathlineto{\pgfqpoint{5.061651in}{5.905292in}}%
\pgfpathlineto{\pgfqpoint{5.064873in}{5.906717in}}%
\pgfpathlineto{\pgfqpoint{5.069168in}{5.907689in}}%
\pgfpathlineto{\pgfqpoint{5.072390in}{5.909295in}}%
\pgfpathlineto{\pgfqpoint{5.076685in}{5.910439in}}%
\pgfpathlineto{\pgfqpoint{5.086350in}{5.913382in}}%
\pgfpathlineto{\pgfqpoint{5.087424in}{5.913967in}}%
\pgfpathlineto{\pgfqpoint{5.091719in}{5.915121in}}%
\pgfpathlineto{\pgfqpoint{5.094940in}{5.916811in}}%
\pgfpathlineto{\pgfqpoint{5.099236in}{5.917842in}}%
\pgfpathlineto{\pgfqpoint{5.102457in}{5.919503in}}%
\pgfpathlineto{\pgfqpoint{5.107826in}{5.920682in}}%
\pgfpathlineto{\pgfqpoint{5.109974in}{5.921729in}}%
\pgfpathlineto{\pgfqpoint{5.114269in}{5.922684in}}%
\pgfpathlineto{\pgfqpoint{5.117491in}{5.924074in}}%
\pgfpathlineto{\pgfqpoint{5.121786in}{5.924892in}}%
\pgfpathlineto{\pgfqpoint{5.125008in}{5.926240in}}%
\pgfpathlineto{\pgfqpoint{5.129303in}{5.927259in}}%
\pgfpathlineto{\pgfqpoint{5.132525in}{5.928806in}}%
\pgfpathlineto{\pgfqpoint{5.137894in}{5.929825in}}%
\pgfpathlineto{\pgfqpoint{5.140042in}{5.930836in}}%
\pgfpathlineto{\pgfqpoint{5.144337in}{5.931882in}}%
\pgfpathlineto{\pgfqpoint{5.147558in}{5.933470in}}%
\pgfpathlineto{\pgfqpoint{5.151854in}{5.934556in}}%
\pgfpathlineto{\pgfqpoint{5.155075in}{5.936286in}}%
\pgfpathlineto{\pgfqpoint{5.159371in}{5.937441in}}%
\pgfpathlineto{\pgfqpoint{5.162592in}{5.939021in}}%
\pgfpathlineto{\pgfqpoint{5.166888in}{5.940072in}}%
\pgfpathlineto{\pgfqpoint{5.170109in}{5.941628in}}%
\pgfpathlineto{\pgfqpoint{5.174404in}{5.942640in}}%
\pgfpathlineto{\pgfqpoint{5.177626in}{5.944090in}}%
\pgfpathlineto{\pgfqpoint{5.181921in}{5.945070in}}%
\pgfpathlineto{\pgfqpoint{5.185143in}{5.946517in}}%
\pgfpathlineto{\pgfqpoint{5.189438in}{5.947327in}}%
\pgfpathlineto{\pgfqpoint{5.192660in}{5.948511in}}%
\pgfpathlineto{\pgfqpoint{5.198029in}{5.949689in}}%
\pgfpathlineto{\pgfqpoint{5.199103in}{5.950077in}}%
\pgfpathlineto{\pgfqpoint{5.204472in}{5.950874in}}%
\pgfpathlineto{\pgfqpoint{5.207693in}{5.952079in}}%
\pgfpathlineto{\pgfqpoint{5.213063in}{5.953286in}}%
\pgfpathlineto{\pgfqpoint{5.215210in}{5.954062in}}%
\pgfpathlineto{\pgfqpoint{5.220580in}{5.955234in}}%
\pgfpathlineto{\pgfqpoint{5.222727in}{5.956069in}}%
\pgfpathlineto{\pgfqpoint{5.228096in}{5.957330in}}%
\pgfpathlineto{\pgfqpoint{5.230244in}{5.958105in}}%
\pgfpathlineto{\pgfqpoint{5.235613in}{5.959267in}}%
\pgfpathlineto{\pgfqpoint{5.237761in}{5.960064in}}%
\pgfpathlineto{\pgfqpoint{5.243130in}{5.960961in}}%
\pgfpathlineto{\pgfqpoint{5.245278in}{5.961859in}}%
\pgfpathlineto{\pgfqpoint{5.249573in}{5.962766in}}%
\pgfpathlineto{\pgfqpoint{5.252795in}{5.964192in}}%
\pgfpathlineto{\pgfqpoint{5.257090in}{5.965220in}}%
\pgfpathlineto{\pgfqpoint{5.260312in}{5.966719in}}%
\pgfpathlineto{\pgfqpoint{5.264607in}{5.967696in}}%
\pgfpathlineto{\pgfqpoint{5.267828in}{5.969199in}}%
\pgfpathlineto{\pgfqpoint{5.272124in}{5.970177in}}%
\pgfpathlineto{\pgfqpoint{5.275345in}{5.971614in}}%
\pgfpathlineto{\pgfqpoint{5.279641in}{5.972582in}}%
\pgfpathlineto{\pgfqpoint{5.281788in}{5.973559in}}%
\pgfpathlineto{\pgfqpoint{5.287158in}{5.974510in}}%
\pgfpathlineto{\pgfqpoint{5.290379in}{5.975916in}}%
\pgfpathlineto{\pgfqpoint{5.294674in}{5.976844in}}%
\pgfpathlineto{\pgfqpoint{5.297896in}{5.978179in}}%
\pgfpathlineto{\pgfqpoint{5.302191in}{5.979044in}}%
\pgfpathlineto{\pgfqpoint{5.305413in}{5.980312in}}%
\pgfpathlineto{\pgfqpoint{5.310782in}{5.981500in}}%
\pgfpathlineto{\pgfqpoint{5.312930in}{5.982135in}}%
\pgfpathlineto{\pgfqpoint{5.318299in}{5.983044in}}%
\pgfpathlineto{\pgfqpoint{5.327963in}{5.985184in}}%
\pgfpathlineto{\pgfqpoint{5.333333in}{5.986137in}}%
\pgfpathlineto{\pgfqpoint{5.335480in}{5.986805in}}%
\pgfpathlineto{\pgfqpoint{5.340849in}{5.987816in}}%
\pgfpathlineto{\pgfqpoint{5.342997in}{5.988484in}}%
\pgfpathlineto{\pgfqpoint{5.349440in}{5.989506in}}%
\pgfpathlineto{\pgfqpoint{5.350514in}{5.989841in}}%
\pgfpathlineto{\pgfqpoint{5.355883in}{5.990778in}}%
\pgfpathlineto{\pgfqpoint{5.358031in}{5.991385in}}%
\pgfpathlineto{\pgfqpoint{5.363400in}{5.992302in}}%
\pgfpathlineto{\pgfqpoint{5.365548in}{5.992963in}}%
\pgfpathlineto{\pgfqpoint{5.370917in}{5.993883in}}%
\pgfpathlineto{\pgfqpoint{5.380581in}{5.995833in}}%
\pgfpathlineto{\pgfqpoint{5.387024in}{5.996896in}}%
\pgfpathlineto{\pgfqpoint{5.392394in}{5.997542in}}%
\pgfpathlineto{\pgfqpoint{5.417092in}{6.002025in}}%
\pgfpathlineto{\pgfqpoint{5.418166in}{6.002368in}}%
\pgfpathlineto{\pgfqpoint{5.423535in}{6.003375in}}%
\pgfpathlineto{\pgfqpoint{5.425683in}{6.004008in}}%
\pgfpathlineto{\pgfqpoint{5.431052in}{6.004930in}}%
\pgfpathlineto{\pgfqpoint{5.433200in}{6.005543in}}%
\pgfpathlineto{\pgfqpoint{5.440716in}{6.006827in}}%
\pgfpathlineto{\pgfqpoint{5.446086in}{6.007807in}}%
\pgfpathlineto{\pgfqpoint{5.448233in}{6.008460in}}%
\pgfpathlineto{\pgfqpoint{5.453602in}{6.009478in}}%
\pgfpathlineto{\pgfqpoint{5.455750in}{6.010116in}}%
\pgfpathlineto{\pgfqpoint{5.461119in}{6.010951in}}%
\pgfpathlineto{\pgfqpoint{5.463267in}{6.011610in}}%
\pgfpathlineto{\pgfqpoint{5.468636in}{6.012631in}}%
\pgfpathlineto{\pgfqpoint{5.508368in}{6.018479in}}%
\pgfpathlineto{\pgfqpoint{5.515885in}{6.019350in}}%
\pgfpathlineto{\pgfqpoint{5.537362in}{6.021114in}}%
\pgfpathlineto{\pgfqpoint{5.545953in}{6.021891in}}%
\pgfpathlineto{\pgfqpoint{5.558839in}{6.022845in}}%
\pgfpathlineto{\pgfqpoint{5.568503in}{6.023656in}}%
\pgfpathlineto{\pgfqpoint{5.586758in}{6.024589in}}%
\pgfpathlineto{\pgfqpoint{5.598571in}{6.025357in}}%
\pgfpathlineto{\pgfqpoint{5.617900in}{6.026292in}}%
\pgfpathlineto{\pgfqpoint{5.636155in}{6.027580in}}%
\pgfpathlineto{\pgfqpoint{5.654410in}{6.028611in}}%
\pgfpathlineto{\pgfqpoint{5.672666in}{6.029814in}}%
\pgfpathlineto{\pgfqpoint{5.693068in}{6.030680in}}%
\pgfpathlineto{\pgfqpoint{5.703807in}{6.031052in}}%
\pgfpathlineto{\pgfqpoint{5.717767in}{6.031722in}}%
\pgfpathlineto{\pgfqpoint{5.726357in}{6.032276in}}%
\pgfpathlineto{\pgfqpoint{5.771459in}{6.033164in}}%
\pgfpathlineto{\pgfqpoint{5.842332in}{6.033861in}}%
\pgfpathlineto{\pgfqpoint{5.890655in}{6.032016in}}%
\pgfpathlineto{\pgfqpoint{5.899245in}{6.030791in}}%
\pgfpathlineto{\pgfqpoint{5.906762in}{6.029772in}}%
\pgfpathlineto{\pgfqpoint{5.913205in}{6.028870in}}%
\pgfpathlineto{\pgfqpoint{5.914279in}{6.028652in}}%
\pgfpathlineto{\pgfqpoint{5.925018in}{6.027658in}}%
\pgfpathlineto{\pgfqpoint{5.936830in}{6.026094in}}%
\pgfpathlineto{\pgfqpoint{5.949716in}{6.024994in}}%
\pgfpathlineto{\pgfqpoint{5.958307in}{6.024275in}}%
\pgfpathlineto{\pgfqpoint{5.972266in}{6.023317in}}%
\pgfpathlineto{\pgfqpoint{5.981931in}{6.022531in}}%
\pgfpathlineto{\pgfqpoint{6.017368in}{6.020970in}}%
\pgfpathlineto{\pgfqpoint{6.027032in}{6.020492in}}%
\pgfpathlineto{\pgfqpoint{6.047435in}{6.019738in}}%
\pgfpathlineto{\pgfqpoint{6.072133in}{6.018077in}}%
\pgfpathlineto{\pgfqpoint{6.115087in}{6.016397in}}%
\pgfpathlineto{\pgfqpoint{6.196699in}{6.013467in}}%
\pgfpathlineto{\pgfqpoint{6.212806in}{6.013383in}}%
\pgfpathlineto{\pgfqpoint{6.285827in}{6.015410in}}%
\pgfpathlineto{\pgfqpoint{6.312673in}{6.016907in}}%
\pgfpathlineto{\pgfqpoint{6.327707in}{6.017790in}}%
\pgfpathlineto{\pgfqpoint{6.348110in}{6.018912in}}%
\pgfpathlineto{\pgfqpoint{6.371734in}{6.020113in}}%
\pgfpathlineto{\pgfqpoint{6.386768in}{6.020867in}}%
\pgfpathlineto{\pgfqpoint{6.402876in}{6.021747in}}%
\pgfpathlineto{\pgfqpoint{6.430796in}{6.022948in}}%
\pgfpathlineto{\pgfqpoint{6.566099in}{6.034725in}}%
\pgfpathlineto{\pgfqpoint{6.575764in}{6.035991in}}%
\pgfpathlineto{\pgfqpoint{6.582207in}{6.036770in}}%
\pgfpathlineto{\pgfqpoint{6.590797in}{6.038024in}}%
\pgfpathlineto{\pgfqpoint{6.601536in}{6.039214in}}%
\pgfpathlineto{\pgfqpoint{6.611200in}{6.040539in}}%
\pgfpathlineto{\pgfqpoint{6.625160in}{6.042213in}}%
\pgfpathlineto{\pgfqpoint{6.635899in}{6.043994in}}%
\pgfpathlineto{\pgfqpoint{6.642342in}{6.044896in}}%
\pgfpathlineto{\pgfqpoint{6.643416in}{6.045119in}}%
\pgfpathlineto{\pgfqpoint{6.650932in}{6.046021in}}%
\pgfpathlineto{\pgfqpoint{6.650932in}{6.046021in}}%
\pgfusepath{stroke}%
\end{pgfscope}%
\begin{pgfscope}%
\pgfsetrectcap%
\pgfsetmiterjoin%
\pgfsetlinewidth{0.803000pt}%
\definecolor{currentstroke}{rgb}{1.000000,1.000000,1.000000}%
\pgfsetstrokecolor{currentstroke}%
\pgfsetdash{}{0pt}%
\pgfpathmoveto{\pgfqpoint{4.185023in}{5.688815in}}%
\pgfpathlineto{\pgfqpoint{4.185023in}{6.425400in}}%
\pgfusepath{stroke}%
\end{pgfscope}%
\begin{pgfscope}%
\pgfsetrectcap%
\pgfsetmiterjoin%
\pgfsetlinewidth{0.803000pt}%
\definecolor{currentstroke}{rgb}{1.000000,1.000000,1.000000}%
\pgfsetstrokecolor{currentstroke}%
\pgfsetdash{}{0pt}%
\pgfpathmoveto{\pgfqpoint{6.768357in}{5.688815in}}%
\pgfpathlineto{\pgfqpoint{6.768357in}{6.425400in}}%
\pgfusepath{stroke}%
\end{pgfscope}%
\begin{pgfscope}%
\pgfsetrectcap%
\pgfsetmiterjoin%
\pgfsetlinewidth{0.803000pt}%
\definecolor{currentstroke}{rgb}{1.000000,1.000000,1.000000}%
\pgfsetstrokecolor{currentstroke}%
\pgfsetdash{}{0pt}%
\pgfpathmoveto{\pgfqpoint{4.185023in}{5.688815in}}%
\pgfpathlineto{\pgfqpoint{6.768357in}{5.688815in}}%
\pgfusepath{stroke}%
\end{pgfscope}%
\begin{pgfscope}%
\pgfsetrectcap%
\pgfsetmiterjoin%
\pgfsetlinewidth{0.803000pt}%
\definecolor{currentstroke}{rgb}{1.000000,1.000000,1.000000}%
\pgfsetstrokecolor{currentstroke}%
\pgfsetdash{}{0pt}%
\pgfpathmoveto{\pgfqpoint{4.185023in}{6.425400in}}%
\pgfpathlineto{\pgfqpoint{6.768357in}{6.425400in}}%
\pgfusepath{stroke}%
\end{pgfscope}%
\begin{pgfscope}%
\definecolor{textcolor}{rgb}{0.150000,0.150000,0.150000}%
\pgfsetstrokecolor{textcolor}%
\pgfsetfillcolor{textcolor}%
\pgftext[x=5.476690in,y=6.508734in,,base]{\color{textcolor}\rmfamily\fontsize{16.800000}{20.160000}\selectfont AXP}%
\end{pgfscope}%
\begin{pgfscope}%
\pgfsetbuttcap%
\pgfsetmiterjoin%
\definecolor{currentfill}{rgb}{0.917647,0.917647,0.949020}%
\pgfsetfillcolor{currentfill}%
\pgfsetlinewidth{0.000000pt}%
\definecolor{currentstroke}{rgb}{0.000000,0.000000,0.000000}%
\pgfsetstrokecolor{currentstroke}%
\pgfsetstrokeopacity{0.000000}%
\pgfsetdash{}{0pt}%
\pgfpathmoveto{\pgfqpoint{0.568357in}{4.362961in}}%
\pgfpathlineto{\pgfqpoint{3.151690in}{4.362961in}}%
\pgfpathlineto{\pgfqpoint{3.151690in}{5.099547in}}%
\pgfpathlineto{\pgfqpoint{0.568357in}{5.099547in}}%
\pgfpathclose%
\pgfusepath{fill}%
\end{pgfscope}%
\begin{pgfscope}%
\pgfpathrectangle{\pgfqpoint{0.568357in}{4.362961in}}{\pgfqpoint{2.583333in}{0.736585in}}%
\pgfusepath{clip}%
\pgfsetroundcap%
\pgfsetroundjoin%
\pgfsetlinewidth{0.803000pt}%
\definecolor{currentstroke}{rgb}{1.000000,1.000000,1.000000}%
\pgfsetstrokecolor{currentstroke}%
\pgfsetdash{}{0pt}%
\pgfpathmoveto{\pgfqpoint{0.683633in}{4.362961in}}%
\pgfpathlineto{\pgfqpoint{0.683633in}{5.099547in}}%
\pgfusepath{stroke}%
\end{pgfscope}%
\begin{pgfscope}%
\definecolor{textcolor}{rgb}{0.150000,0.150000,0.150000}%
\pgfsetstrokecolor{textcolor}%
\pgfsetfillcolor{textcolor}%
\pgftext[x=0.683633in,y=4.265739in,,top]{\color{textcolor}\rmfamily\fontsize{14.000000}{16.800000}\selectfont 2012}%
\end{pgfscope}%
\begin{pgfscope}%
\pgfpathrectangle{\pgfqpoint{0.568357in}{4.362961in}}{\pgfqpoint{2.583333in}{0.736585in}}%
\pgfusepath{clip}%
\pgfsetroundcap%
\pgfsetroundjoin%
\pgfsetlinewidth{0.803000pt}%
\definecolor{currentstroke}{rgb}{1.000000,1.000000,1.000000}%
\pgfsetstrokecolor{currentstroke}%
\pgfsetdash{}{0pt}%
\pgfpathmoveto{\pgfqpoint{1.076658in}{4.362961in}}%
\pgfpathlineto{\pgfqpoint{1.076658in}{5.099547in}}%
\pgfusepath{stroke}%
\end{pgfscope}%
\begin{pgfscope}%
\definecolor{textcolor}{rgb}{0.150000,0.150000,0.150000}%
\pgfsetstrokecolor{textcolor}%
\pgfsetfillcolor{textcolor}%
\pgftext[x=1.076658in,y=4.265739in,,top]{\color{textcolor}\rmfamily\fontsize{14.000000}{16.800000}\selectfont 2013}%
\end{pgfscope}%
\begin{pgfscope}%
\pgfpathrectangle{\pgfqpoint{0.568357in}{4.362961in}}{\pgfqpoint{2.583333in}{0.736585in}}%
\pgfusepath{clip}%
\pgfsetroundcap%
\pgfsetroundjoin%
\pgfsetlinewidth{0.803000pt}%
\definecolor{currentstroke}{rgb}{1.000000,1.000000,1.000000}%
\pgfsetstrokecolor{currentstroke}%
\pgfsetdash{}{0pt}%
\pgfpathmoveto{\pgfqpoint{1.468609in}{4.362961in}}%
\pgfpathlineto{\pgfqpoint{1.468609in}{5.099547in}}%
\pgfusepath{stroke}%
\end{pgfscope}%
\begin{pgfscope}%
\definecolor{textcolor}{rgb}{0.150000,0.150000,0.150000}%
\pgfsetstrokecolor{textcolor}%
\pgfsetfillcolor{textcolor}%
\pgftext[x=1.468609in,y=4.265739in,,top]{\color{textcolor}\rmfamily\fontsize{14.000000}{16.800000}\selectfont 2014}%
\end{pgfscope}%
\begin{pgfscope}%
\pgfpathrectangle{\pgfqpoint{0.568357in}{4.362961in}}{\pgfqpoint{2.583333in}{0.736585in}}%
\pgfusepath{clip}%
\pgfsetroundcap%
\pgfsetroundjoin%
\pgfsetlinewidth{0.803000pt}%
\definecolor{currentstroke}{rgb}{1.000000,1.000000,1.000000}%
\pgfsetstrokecolor{currentstroke}%
\pgfsetdash{}{0pt}%
\pgfpathmoveto{\pgfqpoint{1.860560in}{4.362961in}}%
\pgfpathlineto{\pgfqpoint{1.860560in}{5.099547in}}%
\pgfusepath{stroke}%
\end{pgfscope}%
\begin{pgfscope}%
\definecolor{textcolor}{rgb}{0.150000,0.150000,0.150000}%
\pgfsetstrokecolor{textcolor}%
\pgfsetfillcolor{textcolor}%
\pgftext[x=1.860560in,y=4.265739in,,top]{\color{textcolor}\rmfamily\fontsize{14.000000}{16.800000}\selectfont 2015}%
\end{pgfscope}%
\begin{pgfscope}%
\pgfpathrectangle{\pgfqpoint{0.568357in}{4.362961in}}{\pgfqpoint{2.583333in}{0.736585in}}%
\pgfusepath{clip}%
\pgfsetroundcap%
\pgfsetroundjoin%
\pgfsetlinewidth{0.803000pt}%
\definecolor{currentstroke}{rgb}{1.000000,1.000000,1.000000}%
\pgfsetstrokecolor{currentstroke}%
\pgfsetdash{}{0pt}%
\pgfpathmoveto{\pgfqpoint{2.252511in}{4.362961in}}%
\pgfpathlineto{\pgfqpoint{2.252511in}{5.099547in}}%
\pgfusepath{stroke}%
\end{pgfscope}%
\begin{pgfscope}%
\definecolor{textcolor}{rgb}{0.150000,0.150000,0.150000}%
\pgfsetstrokecolor{textcolor}%
\pgfsetfillcolor{textcolor}%
\pgftext[x=2.252511in,y=4.265739in,,top]{\color{textcolor}\rmfamily\fontsize{14.000000}{16.800000}\selectfont 2016}%
\end{pgfscope}%
\begin{pgfscope}%
\pgfpathrectangle{\pgfqpoint{0.568357in}{4.362961in}}{\pgfqpoint{2.583333in}{0.736585in}}%
\pgfusepath{clip}%
\pgfsetroundcap%
\pgfsetroundjoin%
\pgfsetlinewidth{0.803000pt}%
\definecolor{currentstroke}{rgb}{1.000000,1.000000,1.000000}%
\pgfsetstrokecolor{currentstroke}%
\pgfsetdash{}{0pt}%
\pgfpathmoveto{\pgfqpoint{2.645536in}{4.362961in}}%
\pgfpathlineto{\pgfqpoint{2.645536in}{5.099547in}}%
\pgfusepath{stroke}%
\end{pgfscope}%
\begin{pgfscope}%
\definecolor{textcolor}{rgb}{0.150000,0.150000,0.150000}%
\pgfsetstrokecolor{textcolor}%
\pgfsetfillcolor{textcolor}%
\pgftext[x=2.645536in,y=4.265739in,,top]{\color{textcolor}\rmfamily\fontsize{14.000000}{16.800000}\selectfont 2017}%
\end{pgfscope}%
\begin{pgfscope}%
\pgfpathrectangle{\pgfqpoint{0.568357in}{4.362961in}}{\pgfqpoint{2.583333in}{0.736585in}}%
\pgfusepath{clip}%
\pgfsetroundcap%
\pgfsetroundjoin%
\pgfsetlinewidth{0.803000pt}%
\definecolor{currentstroke}{rgb}{1.000000,1.000000,1.000000}%
\pgfsetstrokecolor{currentstroke}%
\pgfsetdash{}{0pt}%
\pgfpathmoveto{\pgfqpoint{3.037487in}{4.362961in}}%
\pgfpathlineto{\pgfqpoint{3.037487in}{5.099547in}}%
\pgfusepath{stroke}%
\end{pgfscope}%
\begin{pgfscope}%
\definecolor{textcolor}{rgb}{0.150000,0.150000,0.150000}%
\pgfsetstrokecolor{textcolor}%
\pgfsetfillcolor{textcolor}%
\pgftext[x=3.037487in,y=4.265739in,,top]{\color{textcolor}\rmfamily\fontsize{14.000000}{16.800000}\selectfont 2018}%
\end{pgfscope}%
\begin{pgfscope}%
\pgfpathrectangle{\pgfqpoint{0.568357in}{4.362961in}}{\pgfqpoint{2.583333in}{0.736585in}}%
\pgfusepath{clip}%
\pgfsetroundcap%
\pgfsetroundjoin%
\pgfsetlinewidth{0.803000pt}%
\definecolor{currentstroke}{rgb}{1.000000,1.000000,1.000000}%
\pgfsetstrokecolor{currentstroke}%
\pgfsetdash{}{0pt}%
\pgfpathmoveto{\pgfqpoint{0.568357in}{4.451985in}}%
\pgfpathlineto{\pgfqpoint{3.151690in}{4.451985in}}%
\pgfusepath{stroke}%
\end{pgfscope}%
\begin{pgfscope}%
\definecolor{textcolor}{rgb}{0.150000,0.150000,0.150000}%
\pgfsetstrokecolor{textcolor}%
\pgfsetfillcolor{textcolor}%
\pgftext[x=0.223712in,y=4.378119in,left,base]{\color{textcolor}\rmfamily\fontsize{14.000000}{16.800000}\selectfont 15}%
\end{pgfscope}%
\begin{pgfscope}%
\pgfpathrectangle{\pgfqpoint{0.568357in}{4.362961in}}{\pgfqpoint{2.583333in}{0.736585in}}%
\pgfusepath{clip}%
\pgfsetroundcap%
\pgfsetroundjoin%
\pgfsetlinewidth{0.803000pt}%
\definecolor{currentstroke}{rgb}{1.000000,1.000000,1.000000}%
\pgfsetstrokecolor{currentstroke}%
\pgfsetdash{}{0pt}%
\pgfpathmoveto{\pgfqpoint{0.568357in}{4.674156in}}%
\pgfpathlineto{\pgfqpoint{3.151690in}{4.674156in}}%
\pgfusepath{stroke}%
\end{pgfscope}%
\begin{pgfscope}%
\definecolor{textcolor}{rgb}{0.150000,0.150000,0.150000}%
\pgfsetstrokecolor{textcolor}%
\pgfsetfillcolor{textcolor}%
\pgftext[x=0.223712in,y=4.600290in,left,base]{\color{textcolor}\rmfamily\fontsize{14.000000}{16.800000}\selectfont 20}%
\end{pgfscope}%
\begin{pgfscope}%
\pgfpathrectangle{\pgfqpoint{0.568357in}{4.362961in}}{\pgfqpoint{2.583333in}{0.736585in}}%
\pgfusepath{clip}%
\pgfsetroundcap%
\pgfsetroundjoin%
\pgfsetlinewidth{0.803000pt}%
\definecolor{currentstroke}{rgb}{1.000000,1.000000,1.000000}%
\pgfsetstrokecolor{currentstroke}%
\pgfsetdash{}{0pt}%
\pgfpathmoveto{\pgfqpoint{0.568357in}{4.896327in}}%
\pgfpathlineto{\pgfqpoint{3.151690in}{4.896327in}}%
\pgfusepath{stroke}%
\end{pgfscope}%
\begin{pgfscope}%
\definecolor{textcolor}{rgb}{0.150000,0.150000,0.150000}%
\pgfsetstrokecolor{textcolor}%
\pgfsetfillcolor{textcolor}%
\pgftext[x=0.223712in,y=4.822461in,left,base]{\color{textcolor}\rmfamily\fontsize{14.000000}{16.800000}\selectfont 25}%
\end{pgfscope}%
\begin{pgfscope}%
\pgfpathrectangle{\pgfqpoint{0.568357in}{4.362961in}}{\pgfqpoint{2.583333in}{0.736585in}}%
\pgfusepath{clip}%
\pgfsetroundcap%
\pgfsetroundjoin%
\pgfsetlinewidth{1.505625pt}%
\definecolor{currentstroke}{rgb}{0.172549,0.627451,0.172549}%
\pgfsetstrokecolor{currentstroke}%
\pgfsetdash{}{0pt}%
\pgfpathmoveto{\pgfqpoint{0.685781in}{4.398220in}}%
\pgfpathlineto{\pgfqpoint{0.686855in}{4.404885in}}%
\pgfpathlineto{\pgfqpoint{0.687929in}{4.404441in}}%
\pgfpathlineto{\pgfqpoint{0.689002in}{4.407995in}}%
\pgfpathlineto{\pgfqpoint{0.692224in}{4.415105in}}%
\pgfpathlineto{\pgfqpoint{0.693298in}{4.410217in}}%
\pgfpathlineto{\pgfqpoint{0.694372in}{4.415549in}}%
\pgfpathlineto{\pgfqpoint{0.695445in}{4.417326in}}%
\pgfpathlineto{\pgfqpoint{0.696519in}{4.414216in}}%
\pgfpathlineto{\pgfqpoint{0.700815in}{4.411106in}}%
\pgfpathlineto{\pgfqpoint{0.701888in}{4.420437in}}%
\pgfpathlineto{\pgfqpoint{0.702962in}{4.424436in}}%
\pgfpathlineto{\pgfqpoint{0.704036in}{4.424436in}}%
\pgfpathlineto{\pgfqpoint{0.707258in}{4.417771in}}%
\pgfpathlineto{\pgfqpoint{0.708332in}{4.414216in}}%
\pgfpathlineto{\pgfqpoint{0.709405in}{4.423992in}}%
\pgfpathlineto{\pgfqpoint{0.711553in}{4.420437in}}%
\pgfpathlineto{\pgfqpoint{0.714775in}{4.416438in}}%
\pgfpathlineto{\pgfqpoint{0.715848in}{4.409773in}}%
\pgfpathlineto{\pgfqpoint{0.716922in}{4.411994in}}%
\pgfpathlineto{\pgfqpoint{0.717996in}{4.411106in}}%
\pgfpathlineto{\pgfqpoint{0.719070in}{4.420437in}}%
\pgfpathlineto{\pgfqpoint{0.722291in}{4.421326in}}%
\pgfpathlineto{\pgfqpoint{0.723365in}{4.425769in}}%
\pgfpathlineto{\pgfqpoint{0.724439in}{4.427546in}}%
\pgfpathlineto{\pgfqpoint{0.725513in}{4.423992in}}%
\pgfpathlineto{\pgfqpoint{0.726587in}{4.415549in}}%
\pgfpathlineto{\pgfqpoint{0.729808in}{4.421770in}}%
\pgfpathlineto{\pgfqpoint{0.730882in}{4.417771in}}%
\pgfpathlineto{\pgfqpoint{0.731956in}{4.411550in}}%
\pgfpathlineto{\pgfqpoint{0.734104in}{4.428879in}}%
\pgfpathlineto{\pgfqpoint{0.738399in}{4.433323in}}%
\pgfpathlineto{\pgfqpoint{0.739473in}{4.432434in}}%
\pgfpathlineto{\pgfqpoint{0.740547in}{4.435544in}}%
\pgfpathlineto{\pgfqpoint{0.744842in}{4.427546in}}%
\pgfpathlineto{\pgfqpoint{0.745916in}{4.430657in}}%
\pgfpathlineto{\pgfqpoint{0.746990in}{4.427102in}}%
\pgfpathlineto{\pgfqpoint{0.748064in}{4.429324in}}%
\pgfpathlineto{\pgfqpoint{0.749137in}{4.424436in}}%
\pgfpathlineto{\pgfqpoint{0.752359in}{4.419993in}}%
\pgfpathlineto{\pgfqpoint{0.753433in}{4.405774in}}%
\pgfpathlineto{\pgfqpoint{0.755580in}{4.426213in}}%
\pgfpathlineto{\pgfqpoint{0.756654in}{4.426658in}}%
\pgfpathlineto{\pgfqpoint{0.759876in}{4.429768in}}%
\pgfpathlineto{\pgfqpoint{0.760950in}{4.445320in}}%
\pgfpathlineto{\pgfqpoint{0.762023in}{4.451985in}}%
\pgfpathlineto{\pgfqpoint{0.763097in}{4.464427in}}%
\pgfpathlineto{\pgfqpoint{0.764171in}{4.465760in}}%
\pgfpathlineto{\pgfqpoint{0.767393in}{4.466204in}}%
\pgfpathlineto{\pgfqpoint{0.768466in}{4.461316in}}%
\pgfpathlineto{\pgfqpoint{0.769540in}{4.461316in}}%
\pgfpathlineto{\pgfqpoint{0.770614in}{4.453762in}}%
\pgfpathlineto{\pgfqpoint{0.771688in}{4.451541in}}%
\pgfpathlineto{\pgfqpoint{0.774910in}{4.460428in}}%
\pgfpathlineto{\pgfqpoint{0.775983in}{4.460428in}}%
\pgfpathlineto{\pgfqpoint{0.778131in}{4.457317in}}%
\pgfpathlineto{\pgfqpoint{0.779205in}{4.461316in}}%
\pgfpathlineto{\pgfqpoint{0.782426in}{4.459539in}}%
\pgfpathlineto{\pgfqpoint{0.783500in}{4.457762in}}%
\pgfpathlineto{\pgfqpoint{0.785648in}{4.441765in}}%
\pgfpathlineto{\pgfqpoint{0.789943in}{4.431990in}}%
\pgfpathlineto{\pgfqpoint{0.791017in}{4.416438in}}%
\pgfpathlineto{\pgfqpoint{0.793165in}{4.435544in}}%
\pgfpathlineto{\pgfqpoint{0.794239in}{4.421326in}}%
\pgfpathlineto{\pgfqpoint{0.797460in}{4.421770in}}%
\pgfpathlineto{\pgfqpoint{0.798534in}{4.436878in}}%
\pgfpathlineto{\pgfqpoint{0.799608in}{4.428435in}}%
\pgfpathlineto{\pgfqpoint{0.800682in}{4.429768in}}%
\pgfpathlineto{\pgfqpoint{0.801755in}{4.437322in}}%
\pgfpathlineto{\pgfqpoint{0.804977in}{4.427546in}}%
\pgfpathlineto{\pgfqpoint{0.806051in}{4.443543in}}%
\pgfpathlineto{\pgfqpoint{0.807125in}{4.440432in}}%
\pgfpathlineto{\pgfqpoint{0.809272in}{4.451541in}}%
\pgfpathlineto{\pgfqpoint{0.812494in}{4.444876in}}%
\pgfpathlineto{\pgfqpoint{0.813568in}{4.451985in}}%
\pgfpathlineto{\pgfqpoint{0.814642in}{4.451096in}}%
\pgfpathlineto{\pgfqpoint{0.815715in}{4.445764in}}%
\pgfpathlineto{\pgfqpoint{0.816789in}{4.436878in}}%
\pgfpathlineto{\pgfqpoint{0.820011in}{4.435989in}}%
\pgfpathlineto{\pgfqpoint{0.821085in}{4.433767in}}%
\pgfpathlineto{\pgfqpoint{0.822158in}{4.422214in}}%
\pgfpathlineto{\pgfqpoint{0.823232in}{4.428435in}}%
\pgfpathlineto{\pgfqpoint{0.824306in}{4.425769in}}%
\pgfpathlineto{\pgfqpoint{0.828601in}{4.404885in}}%
\pgfpathlineto{\pgfqpoint{0.829675in}{4.425325in}}%
\pgfpathlineto{\pgfqpoint{0.830749in}{4.421326in}}%
\pgfpathlineto{\pgfqpoint{0.835044in}{4.429324in}}%
\pgfpathlineto{\pgfqpoint{0.836118in}{4.431101in}}%
\pgfpathlineto{\pgfqpoint{0.837192in}{4.431101in}}%
\pgfpathlineto{\pgfqpoint{0.838266in}{4.433767in}}%
\pgfpathlineto{\pgfqpoint{0.839340in}{4.431990in}}%
\pgfpathlineto{\pgfqpoint{0.843635in}{4.436878in}}%
\pgfpathlineto{\pgfqpoint{0.844709in}{4.426658in}}%
\pgfpathlineto{\pgfqpoint{0.845783in}{4.428435in}}%
\pgfpathlineto{\pgfqpoint{0.846857in}{4.409773in}}%
\pgfpathlineto{\pgfqpoint{0.850078in}{4.396442in}}%
\pgfpathlineto{\pgfqpoint{0.851152in}{4.399553in}}%
\pgfpathlineto{\pgfqpoint{0.852226in}{4.421326in}}%
\pgfpathlineto{\pgfqpoint{0.853300in}{4.425325in}}%
\pgfpathlineto{\pgfqpoint{0.854374in}{4.431990in}}%
\pgfpathlineto{\pgfqpoint{0.857595in}{4.428879in}}%
\pgfpathlineto{\pgfqpoint{0.858669in}{4.441321in}}%
\pgfpathlineto{\pgfqpoint{0.859743in}{4.437766in}}%
\pgfpathlineto{\pgfqpoint{0.861890in}{4.460872in}}%
\pgfpathlineto{\pgfqpoint{0.865112in}{4.452429in}}%
\pgfpathlineto{\pgfqpoint{0.866186in}{4.460872in}}%
\pgfpathlineto{\pgfqpoint{0.867260in}{4.463982in}}%
\pgfpathlineto{\pgfqpoint{0.868333in}{4.450652in}}%
\pgfpathlineto{\pgfqpoint{0.869407in}{4.459983in}}%
\pgfpathlineto{\pgfqpoint{0.872629in}{4.450208in}}%
\pgfpathlineto{\pgfqpoint{0.874776in}{4.471092in}}%
\pgfpathlineto{\pgfqpoint{0.875850in}{4.473314in}}%
\pgfpathlineto{\pgfqpoint{0.876924in}{4.495086in}}%
\pgfpathlineto{\pgfqpoint{0.880146in}{4.483089in}}%
\pgfpathlineto{\pgfqpoint{0.883367in}{4.477757in}}%
\pgfpathlineto{\pgfqpoint{0.884441in}{4.466648in}}%
\pgfpathlineto{\pgfqpoint{0.887663in}{4.467981in}}%
\pgfpathlineto{\pgfqpoint{0.888736in}{4.453762in}}%
\pgfpathlineto{\pgfqpoint{0.889810in}{4.455540in}}%
\pgfpathlineto{\pgfqpoint{0.890884in}{4.447542in}}%
\pgfpathlineto{\pgfqpoint{0.891958in}{4.458650in}}%
\pgfpathlineto{\pgfqpoint{0.895179in}{4.452429in}}%
\pgfpathlineto{\pgfqpoint{0.897327in}{4.461316in}}%
\pgfpathlineto{\pgfqpoint{0.898401in}{4.459539in}}%
\pgfpathlineto{\pgfqpoint{0.902696in}{4.469759in}}%
\pgfpathlineto{\pgfqpoint{0.903770in}{4.465315in}}%
\pgfpathlineto{\pgfqpoint{0.904844in}{4.466648in}}%
\pgfpathlineto{\pgfqpoint{0.906992in}{4.497752in}}%
\pgfpathlineto{\pgfqpoint{0.912361in}{4.491532in}}%
\pgfpathlineto{\pgfqpoint{0.913435in}{4.484422in}}%
\pgfpathlineto{\pgfqpoint{0.914509in}{4.499085in}}%
\pgfpathlineto{\pgfqpoint{0.917730in}{4.499530in}}%
\pgfpathlineto{\pgfqpoint{0.918804in}{4.504862in}}%
\pgfpathlineto{\pgfqpoint{0.919878in}{4.500863in}}%
\pgfpathlineto{\pgfqpoint{0.922025in}{4.503973in}}%
\pgfpathlineto{\pgfqpoint{0.926321in}{4.498641in}}%
\pgfpathlineto{\pgfqpoint{0.927395in}{4.499085in}}%
\pgfpathlineto{\pgfqpoint{0.928468in}{4.502196in}}%
\pgfpathlineto{\pgfqpoint{0.929542in}{4.500418in}}%
\pgfpathlineto{\pgfqpoint{0.932764in}{4.498197in}}%
\pgfpathlineto{\pgfqpoint{0.934911in}{4.493309in}}%
\pgfpathlineto{\pgfqpoint{0.935985in}{4.488421in}}%
\pgfpathlineto{\pgfqpoint{0.937059in}{4.493753in}}%
\pgfpathlineto{\pgfqpoint{0.940281in}{4.495531in}}%
\pgfpathlineto{\pgfqpoint{0.941354in}{4.494198in}}%
\pgfpathlineto{\pgfqpoint{0.942428in}{4.494642in}}%
\pgfpathlineto{\pgfqpoint{0.943502in}{4.488421in}}%
\pgfpathlineto{\pgfqpoint{0.944576in}{4.490643in}}%
\pgfpathlineto{\pgfqpoint{0.948871in}{4.483978in}}%
\pgfpathlineto{\pgfqpoint{0.949945in}{4.488865in}}%
\pgfpathlineto{\pgfqpoint{0.951019in}{4.511083in}}%
\pgfpathlineto{\pgfqpoint{0.952093in}{4.520858in}}%
\pgfpathlineto{\pgfqpoint{0.955314in}{4.516859in}}%
\pgfpathlineto{\pgfqpoint{0.956388in}{4.520858in}}%
\pgfpathlineto{\pgfqpoint{0.957462in}{4.531078in}}%
\pgfpathlineto{\pgfqpoint{0.959610in}{4.538187in}}%
\pgfpathlineto{\pgfqpoint{0.962831in}{4.536410in}}%
\pgfpathlineto{\pgfqpoint{0.967127in}{4.558627in}}%
\pgfpathlineto{\pgfqpoint{0.971422in}{4.551073in}}%
\pgfpathlineto{\pgfqpoint{0.972496in}{4.543964in}}%
\pgfpathlineto{\pgfqpoint{0.973570in}{4.565292in}}%
\pgfpathlineto{\pgfqpoint{0.974643in}{4.564848in}}%
\pgfpathlineto{\pgfqpoint{0.977865in}{4.567958in}}%
\pgfpathlineto{\pgfqpoint{0.978939in}{4.567514in}}%
\pgfpathlineto{\pgfqpoint{0.980013in}{4.571513in}}%
\pgfpathlineto{\pgfqpoint{0.981087in}{4.572846in}}%
\pgfpathlineto{\pgfqpoint{0.982160in}{4.578623in}}%
\pgfpathlineto{\pgfqpoint{0.985382in}{4.571957in}}%
\pgfpathlineto{\pgfqpoint{0.987530in}{4.555072in}}%
\pgfpathlineto{\pgfqpoint{0.988603in}{4.557738in}}%
\pgfpathlineto{\pgfqpoint{0.989677in}{4.556850in}}%
\pgfpathlineto{\pgfqpoint{0.992899in}{4.562182in}}%
\pgfpathlineto{\pgfqpoint{0.993973in}{4.562182in}}%
\pgfpathlineto{\pgfqpoint{0.995046in}{4.571513in}}%
\pgfpathlineto{\pgfqpoint{0.996120in}{4.567958in}}%
\pgfpathlineto{\pgfqpoint{0.997194in}{4.541298in}}%
\pgfpathlineto{\pgfqpoint{1.000416in}{4.530189in}}%
\pgfpathlineto{\pgfqpoint{1.001489in}{4.515526in}}%
\pgfpathlineto{\pgfqpoint{1.003637in}{4.515082in}}%
\pgfpathlineto{\pgfqpoint{1.004711in}{4.509750in}}%
\pgfpathlineto{\pgfqpoint{1.010080in}{4.507972in}}%
\pgfpathlineto{\pgfqpoint{1.011154in}{4.517748in}}%
\pgfpathlineto{\pgfqpoint{1.012228in}{4.516859in}}%
\pgfpathlineto{\pgfqpoint{1.015449in}{4.519969in}}%
\pgfpathlineto{\pgfqpoint{1.016523in}{4.526190in}}%
\pgfpathlineto{\pgfqpoint{1.018671in}{4.502196in}}%
\pgfpathlineto{\pgfqpoint{1.019745in}{4.506195in}}%
\pgfpathlineto{\pgfqpoint{1.022966in}{4.502196in}}%
\pgfpathlineto{\pgfqpoint{1.024040in}{4.495086in}}%
\pgfpathlineto{\pgfqpoint{1.025114in}{4.471981in}}%
\pgfpathlineto{\pgfqpoint{1.026188in}{4.473758in}}%
\pgfpathlineto{\pgfqpoint{1.027262in}{4.476868in}}%
\pgfpathlineto{\pgfqpoint{1.030483in}{4.494198in}}%
\pgfpathlineto{\pgfqpoint{1.031557in}{4.492865in}}%
\pgfpathlineto{\pgfqpoint{1.032631in}{4.495086in}}%
\pgfpathlineto{\pgfqpoint{1.034778in}{4.507528in}}%
\pgfpathlineto{\pgfqpoint{1.038000in}{4.507972in}}%
\pgfpathlineto{\pgfqpoint{1.039074in}{4.501751in}}%
\pgfpathlineto{\pgfqpoint{1.040148in}{4.511083in}}%
\pgfpathlineto{\pgfqpoint{1.042295in}{4.510638in}}%
\pgfpathlineto{\pgfqpoint{1.045517in}{4.499974in}}%
\pgfpathlineto{\pgfqpoint{1.046591in}{4.501307in}}%
\pgfpathlineto{\pgfqpoint{1.047664in}{4.513749in}}%
\pgfpathlineto{\pgfqpoint{1.049812in}{4.521747in}}%
\pgfpathlineto{\pgfqpoint{1.053034in}{4.519525in}}%
\pgfpathlineto{\pgfqpoint{1.054108in}{4.523524in}}%
\pgfpathlineto{\pgfqpoint{1.055181in}{4.532855in}}%
\pgfpathlineto{\pgfqpoint{1.056255in}{4.527523in}}%
\pgfpathlineto{\pgfqpoint{1.057329in}{4.527523in}}%
\pgfpathlineto{\pgfqpoint{1.060551in}{4.538187in}}%
\pgfpathlineto{\pgfqpoint{1.061624in}{4.529745in}}%
\pgfpathlineto{\pgfqpoint{1.062698in}{4.506195in}}%
\pgfpathlineto{\pgfqpoint{1.063772in}{4.514193in}}%
\pgfpathlineto{\pgfqpoint{1.064846in}{4.508417in}}%
\pgfpathlineto{\pgfqpoint{1.070215in}{4.504862in}}%
\pgfpathlineto{\pgfqpoint{1.071289in}{4.501751in}}%
\pgfpathlineto{\pgfqpoint{1.072363in}{4.493309in}}%
\pgfpathlineto{\pgfqpoint{1.077732in}{4.524413in}}%
\pgfpathlineto{\pgfqpoint{1.078806in}{4.515970in}}%
\pgfpathlineto{\pgfqpoint{1.079880in}{4.519525in}}%
\pgfpathlineto{\pgfqpoint{1.083101in}{4.517303in}}%
\pgfpathlineto{\pgfqpoint{1.084175in}{4.509305in}}%
\pgfpathlineto{\pgfqpoint{1.085249in}{4.511083in}}%
\pgfpathlineto{\pgfqpoint{1.086323in}{4.518636in}}%
\pgfpathlineto{\pgfqpoint{1.087397in}{4.517303in}}%
\pgfpathlineto{\pgfqpoint{1.090618in}{4.516859in}}%
\pgfpathlineto{\pgfqpoint{1.091692in}{4.519525in}}%
\pgfpathlineto{\pgfqpoint{1.092766in}{4.516859in}}%
\pgfpathlineto{\pgfqpoint{1.093840in}{4.523080in}}%
\pgfpathlineto{\pgfqpoint{1.094913in}{4.548852in}}%
\pgfpathlineto{\pgfqpoint{1.099209in}{4.547519in}}%
\pgfpathlineto{\pgfqpoint{1.100283in}{4.545297in}}%
\pgfpathlineto{\pgfqpoint{1.101356in}{4.548852in}}%
\pgfpathlineto{\pgfqpoint{1.102430in}{4.557294in}}%
\pgfpathlineto{\pgfqpoint{1.105652in}{4.564404in}}%
\pgfpathlineto{\pgfqpoint{1.106726in}{4.564404in}}%
\pgfpathlineto{\pgfqpoint{1.107799in}{4.555072in}}%
\pgfpathlineto{\pgfqpoint{1.108873in}{4.556850in}}%
\pgfpathlineto{\pgfqpoint{1.109947in}{4.568847in}}%
\pgfpathlineto{\pgfqpoint{1.113169in}{4.558183in}}%
\pgfpathlineto{\pgfqpoint{1.114242in}{4.566181in}}%
\pgfpathlineto{\pgfqpoint{1.115316in}{4.562626in}}%
\pgfpathlineto{\pgfqpoint{1.116390in}{4.563959in}}%
\pgfpathlineto{\pgfqpoint{1.117464in}{4.564404in}}%
\pgfpathlineto{\pgfqpoint{1.120686in}{4.563071in}}%
\pgfpathlineto{\pgfqpoint{1.121759in}{4.567514in}}%
\pgfpathlineto{\pgfqpoint{1.122833in}{4.595508in}}%
\pgfpathlineto{\pgfqpoint{1.123907in}{4.595952in}}%
\pgfpathlineto{\pgfqpoint{1.124981in}{4.591953in}}%
\pgfpathlineto{\pgfqpoint{1.129276in}{4.607949in}}%
\pgfpathlineto{\pgfqpoint{1.130350in}{4.595952in}}%
\pgfpathlineto{\pgfqpoint{1.131424in}{4.597285in}}%
\pgfpathlineto{\pgfqpoint{1.132498in}{4.602173in}}%
\pgfpathlineto{\pgfqpoint{1.135719in}{4.581733in}}%
\pgfpathlineto{\pgfqpoint{1.137867in}{4.601284in}}%
\pgfpathlineto{\pgfqpoint{1.138941in}{4.595952in}}%
\pgfpathlineto{\pgfqpoint{1.140015in}{4.595063in}}%
\pgfpathlineto{\pgfqpoint{1.143236in}{4.597729in}}%
\pgfpathlineto{\pgfqpoint{1.144310in}{4.608838in}}%
\pgfpathlineto{\pgfqpoint{1.145384in}{4.611948in}}%
\pgfpathlineto{\pgfqpoint{1.146458in}{4.611948in}}%
\pgfpathlineto{\pgfqpoint{1.147531in}{4.615503in}}%
\pgfpathlineto{\pgfqpoint{1.150753in}{4.610171in}}%
\pgfpathlineto{\pgfqpoint{1.151827in}{4.602617in}}%
\pgfpathlineto{\pgfqpoint{1.152901in}{4.605727in}}%
\pgfpathlineto{\pgfqpoint{1.153975in}{4.612393in}}%
\pgfpathlineto{\pgfqpoint{1.155048in}{4.603950in}}%
\pgfpathlineto{\pgfqpoint{1.158270in}{4.597285in}}%
\pgfpathlineto{\pgfqpoint{1.159344in}{4.599507in}}%
\pgfpathlineto{\pgfqpoint{1.160418in}{4.604394in}}%
\pgfpathlineto{\pgfqpoint{1.161491in}{4.598618in}}%
\pgfpathlineto{\pgfqpoint{1.162565in}{4.601284in}}%
\pgfpathlineto{\pgfqpoint{1.165787in}{4.596841in}}%
\pgfpathlineto{\pgfqpoint{1.166861in}{4.592397in}}%
\pgfpathlineto{\pgfqpoint{1.167934in}{4.591953in}}%
\pgfpathlineto{\pgfqpoint{1.169008in}{4.592397in}}%
\pgfpathlineto{\pgfqpoint{1.173304in}{4.591064in}}%
\pgfpathlineto{\pgfqpoint{1.174377in}{4.600395in}}%
\pgfpathlineto{\pgfqpoint{1.175451in}{4.588398in}}%
\pgfpathlineto{\pgfqpoint{1.176525in}{4.591064in}}%
\pgfpathlineto{\pgfqpoint{1.177599in}{4.586176in}}%
\pgfpathlineto{\pgfqpoint{1.180820in}{4.592397in}}%
\pgfpathlineto{\pgfqpoint{1.181894in}{4.590620in}}%
\pgfpathlineto{\pgfqpoint{1.182968in}{4.608838in}}%
\pgfpathlineto{\pgfqpoint{1.184042in}{4.608838in}}%
\pgfpathlineto{\pgfqpoint{1.185116in}{4.604394in}}%
\pgfpathlineto{\pgfqpoint{1.188337in}{4.581733in}}%
\pgfpathlineto{\pgfqpoint{1.189411in}{4.591953in}}%
\pgfpathlineto{\pgfqpoint{1.190485in}{4.579956in}}%
\pgfpathlineto{\pgfqpoint{1.191559in}{4.576845in}}%
\pgfpathlineto{\pgfqpoint{1.192633in}{4.544853in}}%
\pgfpathlineto{\pgfqpoint{1.195854in}{4.530634in}}%
\pgfpathlineto{\pgfqpoint{1.196928in}{4.535966in}}%
\pgfpathlineto{\pgfqpoint{1.198002in}{4.551962in}}%
\pgfpathlineto{\pgfqpoint{1.199076in}{4.551962in}}%
\pgfpathlineto{\pgfqpoint{1.200150in}{4.560849in}}%
\pgfpathlineto{\pgfqpoint{1.204445in}{4.563515in}}%
\pgfpathlineto{\pgfqpoint{1.205519in}{4.558627in}}%
\pgfpathlineto{\pgfqpoint{1.207666in}{4.573290in}}%
\pgfpathlineto{\pgfqpoint{1.210888in}{4.573735in}}%
\pgfpathlineto{\pgfqpoint{1.211962in}{4.577290in}}%
\pgfpathlineto{\pgfqpoint{1.213036in}{4.588842in}}%
\pgfpathlineto{\pgfqpoint{1.214109in}{4.580844in}}%
\pgfpathlineto{\pgfqpoint{1.215183in}{4.584843in}}%
\pgfpathlineto{\pgfqpoint{1.218405in}{4.583066in}}%
\pgfpathlineto{\pgfqpoint{1.222700in}{4.604394in}}%
\pgfpathlineto{\pgfqpoint{1.225922in}{4.608393in}}%
\pgfpathlineto{\pgfqpoint{1.226996in}{4.611504in}}%
\pgfpathlineto{\pgfqpoint{1.228069in}{4.618613in}}%
\pgfpathlineto{\pgfqpoint{1.230217in}{4.607060in}}%
\pgfpathlineto{\pgfqpoint{1.234512in}{4.609282in}}%
\pgfpathlineto{\pgfqpoint{1.235586in}{4.610615in}}%
\pgfpathlineto{\pgfqpoint{1.236660in}{4.609282in}}%
\pgfpathlineto{\pgfqpoint{1.237734in}{4.599507in}}%
\pgfpathlineto{\pgfqpoint{1.240955in}{4.610615in}}%
\pgfpathlineto{\pgfqpoint{1.242029in}{4.611504in}}%
\pgfpathlineto{\pgfqpoint{1.243103in}{4.599507in}}%
\pgfpathlineto{\pgfqpoint{1.244177in}{4.601728in}}%
\pgfpathlineto{\pgfqpoint{1.245251in}{4.618613in}}%
\pgfpathlineto{\pgfqpoint{1.248472in}{4.615503in}}%
\pgfpathlineto{\pgfqpoint{1.249546in}{4.608838in}}%
\pgfpathlineto{\pgfqpoint{1.250620in}{4.605727in}}%
\pgfpathlineto{\pgfqpoint{1.251694in}{4.611948in}}%
\pgfpathlineto{\pgfqpoint{1.252768in}{4.606616in}}%
\pgfpathlineto{\pgfqpoint{1.255989in}{4.615503in}}%
\pgfpathlineto{\pgfqpoint{1.257063in}{4.635054in}}%
\pgfpathlineto{\pgfqpoint{1.258137in}{4.622612in}}%
\pgfpathlineto{\pgfqpoint{1.259211in}{4.603506in}}%
\pgfpathlineto{\pgfqpoint{1.260285in}{4.607505in}}%
\pgfpathlineto{\pgfqpoint{1.263506in}{4.592397in}}%
\pgfpathlineto{\pgfqpoint{1.265654in}{4.603506in}}%
\pgfpathlineto{\pgfqpoint{1.266728in}{4.606172in}}%
\pgfpathlineto{\pgfqpoint{1.267801in}{4.601728in}}%
\pgfpathlineto{\pgfqpoint{1.271023in}{4.606616in}}%
\pgfpathlineto{\pgfqpoint{1.272097in}{4.591508in}}%
\pgfpathlineto{\pgfqpoint{1.273171in}{4.591508in}}%
\pgfpathlineto{\pgfqpoint{1.275318in}{4.603061in}}%
\pgfpathlineto{\pgfqpoint{1.278540in}{4.606172in}}%
\pgfpathlineto{\pgfqpoint{1.279614in}{4.616836in}}%
\pgfpathlineto{\pgfqpoint{1.280687in}{4.613726in}}%
\pgfpathlineto{\pgfqpoint{1.281761in}{4.627944in}}%
\pgfpathlineto{\pgfqpoint{1.282835in}{4.621724in}}%
\pgfpathlineto{\pgfqpoint{1.286057in}{4.616836in}}%
\pgfpathlineto{\pgfqpoint{1.287130in}{4.610171in}}%
\pgfpathlineto{\pgfqpoint{1.289278in}{4.616836in}}%
\pgfpathlineto{\pgfqpoint{1.290352in}{4.655494in}}%
\pgfpathlineto{\pgfqpoint{1.293574in}{4.660381in}}%
\pgfpathlineto{\pgfqpoint{1.295721in}{4.651939in}}%
\pgfpathlineto{\pgfqpoint{1.296795in}{4.654161in}}%
\pgfpathlineto{\pgfqpoint{1.297869in}{4.652828in}}%
\pgfpathlineto{\pgfqpoint{1.301090in}{4.647051in}}%
\pgfpathlineto{\pgfqpoint{1.302164in}{4.647051in}}%
\pgfpathlineto{\pgfqpoint{1.303238in}{4.643052in}}%
\pgfpathlineto{\pgfqpoint{1.304312in}{4.651939in}}%
\pgfpathlineto{\pgfqpoint{1.305386in}{4.654605in}}%
\pgfpathlineto{\pgfqpoint{1.308607in}{4.648384in}}%
\pgfpathlineto{\pgfqpoint{1.309681in}{4.640830in}}%
\pgfpathlineto{\pgfqpoint{1.310755in}{4.642163in}}%
\pgfpathlineto{\pgfqpoint{1.311829in}{4.641719in}}%
\pgfpathlineto{\pgfqpoint{1.312903in}{4.638609in}}%
\pgfpathlineto{\pgfqpoint{1.316124in}{4.639497in}}%
\pgfpathlineto{\pgfqpoint{1.317198in}{4.637276in}}%
\pgfpathlineto{\pgfqpoint{1.318272in}{4.632388in}}%
\pgfpathlineto{\pgfqpoint{1.320419in}{4.628389in}}%
\pgfpathlineto{\pgfqpoint{1.323641in}{4.624834in}}%
\pgfpathlineto{\pgfqpoint{1.325789in}{4.616392in}}%
\pgfpathlineto{\pgfqpoint{1.326863in}{4.622168in}}%
\pgfpathlineto{\pgfqpoint{1.327936in}{4.622168in}}%
\pgfpathlineto{\pgfqpoint{1.331158in}{4.616392in}}%
\pgfpathlineto{\pgfqpoint{1.332232in}{4.601284in}}%
\pgfpathlineto{\pgfqpoint{1.333306in}{4.601728in}}%
\pgfpathlineto{\pgfqpoint{1.334379in}{4.598618in}}%
\pgfpathlineto{\pgfqpoint{1.335453in}{4.599951in}}%
\pgfpathlineto{\pgfqpoint{1.339749in}{4.596841in}}%
\pgfpathlineto{\pgfqpoint{1.340822in}{4.600840in}}%
\pgfpathlineto{\pgfqpoint{1.342970in}{4.600395in}}%
\pgfpathlineto{\pgfqpoint{1.346192in}{4.608393in}}%
\pgfpathlineto{\pgfqpoint{1.347265in}{4.625278in}}%
\pgfpathlineto{\pgfqpoint{1.348339in}{4.633277in}}%
\pgfpathlineto{\pgfqpoint{1.349413in}{4.624834in}}%
\pgfpathlineto{\pgfqpoint{1.350487in}{4.622168in}}%
\pgfpathlineto{\pgfqpoint{1.353708in}{4.635054in}}%
\pgfpathlineto{\pgfqpoint{1.355856in}{4.660381in}}%
\pgfpathlineto{\pgfqpoint{1.356930in}{4.652828in}}%
\pgfpathlineto{\pgfqpoint{1.358004in}{4.636831in}}%
\pgfpathlineto{\pgfqpoint{1.361225in}{4.646607in}}%
\pgfpathlineto{\pgfqpoint{1.362299in}{4.647940in}}%
\pgfpathlineto{\pgfqpoint{1.363373in}{4.644829in}}%
\pgfpathlineto{\pgfqpoint{1.364447in}{4.645274in}}%
\pgfpathlineto{\pgfqpoint{1.365521in}{4.638164in}}%
\pgfpathlineto{\pgfqpoint{1.368742in}{4.632832in}}%
\pgfpathlineto{\pgfqpoint{1.370890in}{4.648384in}}%
\pgfpathlineto{\pgfqpoint{1.371964in}{4.639942in}}%
\pgfpathlineto{\pgfqpoint{1.373038in}{4.638164in}}%
\pgfpathlineto{\pgfqpoint{1.376259in}{4.634610in}}%
\pgfpathlineto{\pgfqpoint{1.377333in}{4.624834in}}%
\pgfpathlineto{\pgfqpoint{1.378407in}{4.621279in}}%
\pgfpathlineto{\pgfqpoint{1.379481in}{4.645274in}}%
\pgfpathlineto{\pgfqpoint{1.380554in}{4.650606in}}%
\pgfpathlineto{\pgfqpoint{1.383776in}{4.650162in}}%
\pgfpathlineto{\pgfqpoint{1.384850in}{4.643052in}}%
\pgfpathlineto{\pgfqpoint{1.385924in}{4.649273in}}%
\pgfpathlineto{\pgfqpoint{1.386997in}{4.660826in}}%
\pgfpathlineto{\pgfqpoint{1.388071in}{4.691485in}}%
\pgfpathlineto{\pgfqpoint{1.391293in}{4.712369in}}%
\pgfpathlineto{\pgfqpoint{1.392367in}{4.708370in}}%
\pgfpathlineto{\pgfqpoint{1.393440in}{4.696817in}}%
\pgfpathlineto{\pgfqpoint{1.394514in}{4.705260in}}%
\pgfpathlineto{\pgfqpoint{1.395588in}{4.703038in}}%
\pgfpathlineto{\pgfqpoint{1.398810in}{4.710592in}}%
\pgfpathlineto{\pgfqpoint{1.400957in}{4.720368in}}%
\pgfpathlineto{\pgfqpoint{1.402031in}{4.712369in}}%
\pgfpathlineto{\pgfqpoint{1.403105in}{4.726588in}}%
\pgfpathlineto{\pgfqpoint{1.406327in}{4.722589in}}%
\pgfpathlineto{\pgfqpoint{1.407400in}{4.722145in}}%
\pgfpathlineto{\pgfqpoint{1.408474in}{4.739474in}}%
\pgfpathlineto{\pgfqpoint{1.409548in}{4.728810in}}%
\pgfpathlineto{\pgfqpoint{1.410622in}{4.744806in}}%
\pgfpathlineto{\pgfqpoint{1.413843in}{4.743473in}}%
\pgfpathlineto{\pgfqpoint{1.414917in}{4.744806in}}%
\pgfpathlineto{\pgfqpoint{1.415991in}{4.748361in}}%
\pgfpathlineto{\pgfqpoint{1.417065in}{4.742585in}}%
\pgfpathlineto{\pgfqpoint{1.418139in}{4.750138in}}%
\pgfpathlineto{\pgfqpoint{1.421360in}{4.750583in}}%
\pgfpathlineto{\pgfqpoint{1.422434in}{4.743918in}}%
\pgfpathlineto{\pgfqpoint{1.424582in}{4.739919in}}%
\pgfpathlineto{\pgfqpoint{1.425656in}{4.745695in}}%
\pgfpathlineto{\pgfqpoint{1.428877in}{4.733253in}}%
\pgfpathlineto{\pgfqpoint{1.431025in}{4.736808in}}%
\pgfpathlineto{\pgfqpoint{1.433173in}{4.731032in}}%
\pgfpathlineto{\pgfqpoint{1.436394in}{4.731032in}}%
\pgfpathlineto{\pgfqpoint{1.437468in}{4.727477in}}%
\pgfpathlineto{\pgfqpoint{1.438542in}{4.730143in}}%
\pgfpathlineto{\pgfqpoint{1.439616in}{4.723478in}}%
\pgfpathlineto{\pgfqpoint{1.440689in}{4.740807in}}%
\pgfpathlineto{\pgfqpoint{1.443911in}{4.749694in}}%
\pgfpathlineto{\pgfqpoint{1.444985in}{4.747917in}}%
\pgfpathlineto{\pgfqpoint{1.446059in}{4.727921in}}%
\pgfpathlineto{\pgfqpoint{1.447132in}{4.726588in}}%
\pgfpathlineto{\pgfqpoint{1.448206in}{4.737253in}}%
\pgfpathlineto{\pgfqpoint{1.452502in}{4.743918in}}%
\pgfpathlineto{\pgfqpoint{1.453575in}{4.757248in}}%
\pgfpathlineto{\pgfqpoint{1.454649in}{4.762136in}}%
\pgfpathlineto{\pgfqpoint{1.455723in}{4.763469in}}%
\pgfpathlineto{\pgfqpoint{1.458945in}{4.764802in}}%
\pgfpathlineto{\pgfqpoint{1.460018in}{4.772356in}}%
\pgfpathlineto{\pgfqpoint{1.462166in}{4.780354in}}%
\pgfpathlineto{\pgfqpoint{1.463240in}{4.780354in}}%
\pgfpathlineto{\pgfqpoint{1.466462in}{4.782575in}}%
\pgfpathlineto{\pgfqpoint{1.467535in}{4.787463in}}%
\pgfpathlineto{\pgfqpoint{1.469683in}{4.768356in}}%
\pgfpathlineto{\pgfqpoint{1.470757in}{4.767912in}}%
\pgfpathlineto{\pgfqpoint{1.473978in}{4.759914in}}%
\pgfpathlineto{\pgfqpoint{1.475052in}{4.760803in}}%
\pgfpathlineto{\pgfqpoint{1.476126in}{4.758137in}}%
\pgfpathlineto{\pgfqpoint{1.477200in}{4.758581in}}%
\pgfpathlineto{\pgfqpoint{1.478274in}{4.749250in}}%
\pgfpathlineto{\pgfqpoint{1.481495in}{4.740807in}}%
\pgfpathlineto{\pgfqpoint{1.483643in}{4.763024in}}%
\pgfpathlineto{\pgfqpoint{1.484717in}{4.757692in}}%
\pgfpathlineto{\pgfqpoint{1.485791in}{4.735475in}}%
\pgfpathlineto{\pgfqpoint{1.490086in}{4.725255in}}%
\pgfpathlineto{\pgfqpoint{1.492234in}{4.708370in}}%
\pgfpathlineto{\pgfqpoint{1.493307in}{4.677266in}}%
\pgfpathlineto{\pgfqpoint{1.496529in}{4.681710in}}%
\pgfpathlineto{\pgfqpoint{1.497603in}{4.695484in}}%
\pgfpathlineto{\pgfqpoint{1.498677in}{4.689708in}}%
\pgfpathlineto{\pgfqpoint{1.499751in}{4.696817in}}%
\pgfpathlineto{\pgfqpoint{1.500824in}{4.683932in}}%
\pgfpathlineto{\pgfqpoint{1.504046in}{4.655938in}}%
\pgfpathlineto{\pgfqpoint{1.505120in}{4.663936in}}%
\pgfpathlineto{\pgfqpoint{1.506194in}{4.662159in}}%
\pgfpathlineto{\pgfqpoint{1.508341in}{4.686153in}}%
\pgfpathlineto{\pgfqpoint{1.511563in}{4.680821in}}%
\pgfpathlineto{\pgfqpoint{1.512637in}{4.694596in}}%
\pgfpathlineto{\pgfqpoint{1.513710in}{4.693263in}}%
\pgfpathlineto{\pgfqpoint{1.514784in}{4.695040in}}%
\pgfpathlineto{\pgfqpoint{1.515858in}{4.705704in}}%
\pgfpathlineto{\pgfqpoint{1.520153in}{4.702594in}}%
\pgfpathlineto{\pgfqpoint{1.521227in}{4.693263in}}%
\pgfpathlineto{\pgfqpoint{1.522301in}{4.691485in}}%
\pgfpathlineto{\pgfqpoint{1.523375in}{4.684820in}}%
\pgfpathlineto{\pgfqpoint{1.526596in}{4.697262in}}%
\pgfpathlineto{\pgfqpoint{1.527670in}{4.696817in}}%
\pgfpathlineto{\pgfqpoint{1.528744in}{4.697706in}}%
\pgfpathlineto{\pgfqpoint{1.529818in}{4.704816in}}%
\pgfpathlineto{\pgfqpoint{1.530892in}{4.703927in}}%
\pgfpathlineto{\pgfqpoint{1.534113in}{4.691485in}}%
\pgfpathlineto{\pgfqpoint{1.536261in}{4.720368in}}%
\pgfpathlineto{\pgfqpoint{1.537335in}{4.731032in}}%
\pgfpathlineto{\pgfqpoint{1.538409in}{4.727921in}}%
\pgfpathlineto{\pgfqpoint{1.541630in}{4.724367in}}%
\pgfpathlineto{\pgfqpoint{1.543778in}{4.714591in}}%
\pgfpathlineto{\pgfqpoint{1.545926in}{4.691041in}}%
\pgfpathlineto{\pgfqpoint{1.549147in}{4.702594in}}%
\pgfpathlineto{\pgfqpoint{1.550221in}{4.710592in}}%
\pgfpathlineto{\pgfqpoint{1.551295in}{4.697262in}}%
\pgfpathlineto{\pgfqpoint{1.552369in}{4.696817in}}%
\pgfpathlineto{\pgfqpoint{1.553442in}{4.701261in}}%
\pgfpathlineto{\pgfqpoint{1.556664in}{4.701705in}}%
\pgfpathlineto{\pgfqpoint{1.557738in}{4.712369in}}%
\pgfpathlineto{\pgfqpoint{1.558812in}{4.709259in}}%
\pgfpathlineto{\pgfqpoint{1.559885in}{4.716368in}}%
\pgfpathlineto{\pgfqpoint{1.560959in}{4.718590in}}%
\pgfpathlineto{\pgfqpoint{1.564181in}{4.719035in}}%
\pgfpathlineto{\pgfqpoint{1.565255in}{4.718146in}}%
\pgfpathlineto{\pgfqpoint{1.567402in}{4.731476in}}%
\pgfpathlineto{\pgfqpoint{1.568476in}{4.723922in}}%
\pgfpathlineto{\pgfqpoint{1.571698in}{4.717702in}}%
\pgfpathlineto{\pgfqpoint{1.572772in}{4.714147in}}%
\pgfpathlineto{\pgfqpoint{1.573845in}{4.721256in}}%
\pgfpathlineto{\pgfqpoint{1.574919in}{4.707926in}}%
\pgfpathlineto{\pgfqpoint{1.575993in}{4.702594in}}%
\pgfpathlineto{\pgfqpoint{1.580288in}{4.716368in}}%
\pgfpathlineto{\pgfqpoint{1.582436in}{4.743029in}}%
\pgfpathlineto{\pgfqpoint{1.586731in}{4.744362in}}%
\pgfpathlineto{\pgfqpoint{1.587805in}{4.743918in}}%
\pgfpathlineto{\pgfqpoint{1.588879in}{4.738141in}}%
\pgfpathlineto{\pgfqpoint{1.589953in}{4.739474in}}%
\pgfpathlineto{\pgfqpoint{1.591027in}{4.744806in}}%
\pgfpathlineto{\pgfqpoint{1.594248in}{4.751027in}}%
\pgfpathlineto{\pgfqpoint{1.595322in}{4.750583in}}%
\pgfpathlineto{\pgfqpoint{1.596396in}{4.755026in}}%
\pgfpathlineto{\pgfqpoint{1.598544in}{4.747472in}}%
\pgfpathlineto{\pgfqpoint{1.601765in}{4.743918in}}%
\pgfpathlineto{\pgfqpoint{1.602839in}{4.729699in}}%
\pgfpathlineto{\pgfqpoint{1.603913in}{4.742140in}}%
\pgfpathlineto{\pgfqpoint{1.604987in}{4.739030in}}%
\pgfpathlineto{\pgfqpoint{1.606061in}{4.738141in}}%
\pgfpathlineto{\pgfqpoint{1.610356in}{4.756359in}}%
\pgfpathlineto{\pgfqpoint{1.612504in}{4.744806in}}%
\pgfpathlineto{\pgfqpoint{1.613577in}{4.747028in}}%
\pgfpathlineto{\pgfqpoint{1.616799in}{4.744806in}}%
\pgfpathlineto{\pgfqpoint{1.617873in}{4.733698in}}%
\pgfpathlineto{\pgfqpoint{1.618947in}{4.740807in}}%
\pgfpathlineto{\pgfqpoint{1.625390in}{4.743473in}}%
\pgfpathlineto{\pgfqpoint{1.627537in}{4.749694in}}%
\pgfpathlineto{\pgfqpoint{1.628611in}{4.751471in}}%
\pgfpathlineto{\pgfqpoint{1.631833in}{4.752805in}}%
\pgfpathlineto{\pgfqpoint{1.632906in}{4.751471in}}%
\pgfpathlineto{\pgfqpoint{1.633980in}{4.743029in}}%
\pgfpathlineto{\pgfqpoint{1.635054in}{4.750583in}}%
\pgfpathlineto{\pgfqpoint{1.636128in}{4.765690in}}%
\pgfpathlineto{\pgfqpoint{1.639350in}{4.775022in}}%
\pgfpathlineto{\pgfqpoint{1.640423in}{4.773689in}}%
\pgfpathlineto{\pgfqpoint{1.642571in}{4.757692in}}%
\pgfpathlineto{\pgfqpoint{1.643645in}{4.760358in}}%
\pgfpathlineto{\pgfqpoint{1.646866in}{4.752805in}}%
\pgfpathlineto{\pgfqpoint{1.647940in}{4.754582in}}%
\pgfpathlineto{\pgfqpoint{1.649014in}{4.755026in}}%
\pgfpathlineto{\pgfqpoint{1.650088in}{4.764357in}}%
\pgfpathlineto{\pgfqpoint{1.651162in}{4.766135in}}%
\pgfpathlineto{\pgfqpoint{1.655457in}{4.751916in}}%
\pgfpathlineto{\pgfqpoint{1.657605in}{4.741252in}}%
\pgfpathlineto{\pgfqpoint{1.658679in}{4.746584in}}%
\pgfpathlineto{\pgfqpoint{1.661900in}{4.740807in}}%
\pgfpathlineto{\pgfqpoint{1.662974in}{4.745251in}}%
\pgfpathlineto{\pgfqpoint{1.665122in}{4.762136in}}%
\pgfpathlineto{\pgfqpoint{1.669417in}{4.758137in}}%
\pgfpathlineto{\pgfqpoint{1.670491in}{4.744362in}}%
\pgfpathlineto{\pgfqpoint{1.671565in}{4.742585in}}%
\pgfpathlineto{\pgfqpoint{1.672639in}{4.738141in}}%
\pgfpathlineto{\pgfqpoint{1.673712in}{4.750583in}}%
\pgfpathlineto{\pgfqpoint{1.676934in}{4.754582in}}%
\pgfpathlineto{\pgfqpoint{1.678008in}{4.752805in}}%
\pgfpathlineto{\pgfqpoint{1.679082in}{4.767912in}}%
\pgfpathlineto{\pgfqpoint{1.680155in}{4.752805in}}%
\pgfpathlineto{\pgfqpoint{1.684451in}{4.730143in}}%
\pgfpathlineto{\pgfqpoint{1.685525in}{4.731476in}}%
\pgfpathlineto{\pgfqpoint{1.686598in}{4.727477in}}%
\pgfpathlineto{\pgfqpoint{1.687672in}{4.728366in}}%
\pgfpathlineto{\pgfqpoint{1.688746in}{4.723034in}}%
\pgfpathlineto{\pgfqpoint{1.691968in}{4.715924in}}%
\pgfpathlineto{\pgfqpoint{1.693041in}{4.710592in}}%
\pgfpathlineto{\pgfqpoint{1.694115in}{4.717702in}}%
\pgfpathlineto{\pgfqpoint{1.695189in}{4.699928in}}%
\pgfpathlineto{\pgfqpoint{1.696263in}{4.707037in}}%
\pgfpathlineto{\pgfqpoint{1.699484in}{4.704371in}}%
\pgfpathlineto{\pgfqpoint{1.700558in}{4.695040in}}%
\pgfpathlineto{\pgfqpoint{1.701632in}{4.710592in}}%
\pgfpathlineto{\pgfqpoint{1.702706in}{4.712369in}}%
\pgfpathlineto{\pgfqpoint{1.703780in}{4.718590in}}%
\pgfpathlineto{\pgfqpoint{1.707001in}{4.723034in}}%
\pgfpathlineto{\pgfqpoint{1.708075in}{4.716368in}}%
\pgfpathlineto{\pgfqpoint{1.709149in}{4.724367in}}%
\pgfpathlineto{\pgfqpoint{1.710223in}{4.726588in}}%
\pgfpathlineto{\pgfqpoint{1.711297in}{4.717702in}}%
\pgfpathlineto{\pgfqpoint{1.714518in}{4.733253in}}%
\pgfpathlineto{\pgfqpoint{1.715592in}{4.732365in}}%
\pgfpathlineto{\pgfqpoint{1.716666in}{4.743918in}}%
\pgfpathlineto{\pgfqpoint{1.717740in}{4.746584in}}%
\pgfpathlineto{\pgfqpoint{1.718814in}{4.736364in}}%
\pgfpathlineto{\pgfqpoint{1.722035in}{4.738141in}}%
\pgfpathlineto{\pgfqpoint{1.723109in}{4.731032in}}%
\pgfpathlineto{\pgfqpoint{1.724183in}{4.735475in}}%
\pgfpathlineto{\pgfqpoint{1.725257in}{4.731032in}}%
\pgfpathlineto{\pgfqpoint{1.726330in}{4.730143in}}%
\pgfpathlineto{\pgfqpoint{1.730626in}{4.725255in}}%
\pgfpathlineto{\pgfqpoint{1.731700in}{4.728810in}}%
\pgfpathlineto{\pgfqpoint{1.732773in}{4.729254in}}%
\pgfpathlineto{\pgfqpoint{1.733847in}{4.734586in}}%
\pgfpathlineto{\pgfqpoint{1.737069in}{4.733698in}}%
\pgfpathlineto{\pgfqpoint{1.738143in}{4.727033in}}%
\pgfpathlineto{\pgfqpoint{1.740290in}{4.731476in}}%
\pgfpathlineto{\pgfqpoint{1.741364in}{4.726144in}}%
\pgfpathlineto{\pgfqpoint{1.744586in}{4.727921in}}%
\pgfpathlineto{\pgfqpoint{1.745660in}{4.738586in}}%
\pgfpathlineto{\pgfqpoint{1.746733in}{4.740363in}}%
\pgfpathlineto{\pgfqpoint{1.747807in}{4.746584in}}%
\pgfpathlineto{\pgfqpoint{1.748881in}{4.749250in}}%
\pgfpathlineto{\pgfqpoint{1.754250in}{4.736364in}}%
\pgfpathlineto{\pgfqpoint{1.755324in}{4.722145in}}%
\pgfpathlineto{\pgfqpoint{1.756398in}{4.725255in}}%
\pgfpathlineto{\pgfqpoint{1.759619in}{4.717257in}}%
\pgfpathlineto{\pgfqpoint{1.760693in}{4.724811in}}%
\pgfpathlineto{\pgfqpoint{1.761767in}{4.707926in}}%
\pgfpathlineto{\pgfqpoint{1.762841in}{4.706593in}}%
\pgfpathlineto{\pgfqpoint{1.763915in}{4.716813in}}%
\pgfpathlineto{\pgfqpoint{1.767136in}{4.710148in}}%
\pgfpathlineto{\pgfqpoint{1.768210in}{4.695040in}}%
\pgfpathlineto{\pgfqpoint{1.769284in}{4.711036in}}%
\pgfpathlineto{\pgfqpoint{1.771432in}{4.675489in}}%
\pgfpathlineto{\pgfqpoint{1.774653in}{4.663492in}}%
\pgfpathlineto{\pgfqpoint{1.776801in}{4.675489in}}%
\pgfpathlineto{\pgfqpoint{1.777875in}{4.674600in}}%
\pgfpathlineto{\pgfqpoint{1.778949in}{4.695484in}}%
\pgfpathlineto{\pgfqpoint{1.782170in}{4.703038in}}%
\pgfpathlineto{\pgfqpoint{1.783244in}{4.718590in}}%
\pgfpathlineto{\pgfqpoint{1.784318in}{4.709259in}}%
\pgfpathlineto{\pgfqpoint{1.786465in}{4.725700in}}%
\pgfpathlineto{\pgfqpoint{1.789687in}{4.721256in}}%
\pgfpathlineto{\pgfqpoint{1.790761in}{4.734142in}}%
\pgfpathlineto{\pgfqpoint{1.791835in}{4.726144in}}%
\pgfpathlineto{\pgfqpoint{1.792908in}{4.726588in}}%
\pgfpathlineto{\pgfqpoint{1.793982in}{4.731920in}}%
\pgfpathlineto{\pgfqpoint{1.797204in}{4.727921in}}%
\pgfpathlineto{\pgfqpoint{1.798278in}{4.727921in}}%
\pgfpathlineto{\pgfqpoint{1.799351in}{4.732365in}}%
\pgfpathlineto{\pgfqpoint{1.800425in}{4.751916in}}%
\pgfpathlineto{\pgfqpoint{1.801499in}{4.753693in}}%
\pgfpathlineto{\pgfqpoint{1.804721in}{4.755915in}}%
\pgfpathlineto{\pgfqpoint{1.805794in}{4.752805in}}%
\pgfpathlineto{\pgfqpoint{1.806868in}{4.757692in}}%
\pgfpathlineto{\pgfqpoint{1.807942in}{4.754138in}}%
\pgfpathlineto{\pgfqpoint{1.809016in}{4.755471in}}%
\pgfpathlineto{\pgfqpoint{1.812238in}{4.761247in}}%
\pgfpathlineto{\pgfqpoint{1.813311in}{4.775910in}}%
\pgfpathlineto{\pgfqpoint{1.815459in}{4.769690in}}%
\pgfpathlineto{\pgfqpoint{1.816533in}{4.775022in}}%
\pgfpathlineto{\pgfqpoint{1.819754in}{4.775466in}}%
\pgfpathlineto{\pgfqpoint{1.820828in}{4.770134in}}%
\pgfpathlineto{\pgfqpoint{1.821902in}{4.770578in}}%
\pgfpathlineto{\pgfqpoint{1.827271in}{4.739474in}}%
\pgfpathlineto{\pgfqpoint{1.828345in}{4.740363in}}%
\pgfpathlineto{\pgfqpoint{1.829419in}{4.752805in}}%
\pgfpathlineto{\pgfqpoint{1.830493in}{4.742140in}}%
\pgfpathlineto{\pgfqpoint{1.831567in}{4.739030in}}%
\pgfpathlineto{\pgfqpoint{1.835862in}{4.723478in}}%
\pgfpathlineto{\pgfqpoint{1.836936in}{4.711925in}}%
\pgfpathlineto{\pgfqpoint{1.838010in}{4.717257in}}%
\pgfpathlineto{\pgfqpoint{1.839083in}{4.698150in}}%
\pgfpathlineto{\pgfqpoint{1.843379in}{4.683487in}}%
\pgfpathlineto{\pgfqpoint{1.844453in}{4.689708in}}%
\pgfpathlineto{\pgfqpoint{1.846600in}{4.733698in}}%
\pgfpathlineto{\pgfqpoint{1.849822in}{4.736808in}}%
\pgfpathlineto{\pgfqpoint{1.850896in}{4.743473in}}%
\pgfpathlineto{\pgfqpoint{1.851970in}{4.741252in}}%
\pgfpathlineto{\pgfqpoint{1.857339in}{4.736808in}}%
\pgfpathlineto{\pgfqpoint{1.858413in}{4.731920in}}%
\pgfpathlineto{\pgfqpoint{1.859486in}{4.720812in}}%
\pgfpathlineto{\pgfqpoint{1.861634in}{4.712814in}}%
\pgfpathlineto{\pgfqpoint{1.864856in}{4.695929in}}%
\pgfpathlineto{\pgfqpoint{1.865929in}{4.676378in}}%
\pgfpathlineto{\pgfqpoint{1.867003in}{4.676822in}}%
\pgfpathlineto{\pgfqpoint{1.868077in}{4.687486in}}%
\pgfpathlineto{\pgfqpoint{1.869151in}{4.674600in}}%
\pgfpathlineto{\pgfqpoint{1.872372in}{4.672823in}}%
\pgfpathlineto{\pgfqpoint{1.875594in}{4.658160in}}%
\pgfpathlineto{\pgfqpoint{1.876668in}{4.658604in}}%
\pgfpathlineto{\pgfqpoint{1.880963in}{4.667935in}}%
\pgfpathlineto{\pgfqpoint{1.884185in}{4.691485in}}%
\pgfpathlineto{\pgfqpoint{1.887406in}{4.695484in}}%
\pgfpathlineto{\pgfqpoint{1.888480in}{4.687931in}}%
\pgfpathlineto{\pgfqpoint{1.889554in}{4.667935in}}%
\pgfpathlineto{\pgfqpoint{1.890628in}{4.676822in}}%
\pgfpathlineto{\pgfqpoint{1.891702in}{4.669713in}}%
\pgfpathlineto{\pgfqpoint{1.894923in}{4.681265in}}%
\pgfpathlineto{\pgfqpoint{1.895997in}{4.691041in}}%
\pgfpathlineto{\pgfqpoint{1.897071in}{4.679488in}}%
\pgfpathlineto{\pgfqpoint{1.898145in}{4.692374in}}%
\pgfpathlineto{\pgfqpoint{1.899218in}{4.692818in}}%
\pgfpathlineto{\pgfqpoint{1.902440in}{4.697262in}}%
\pgfpathlineto{\pgfqpoint{1.905661in}{4.706593in}}%
\pgfpathlineto{\pgfqpoint{1.906735in}{4.716368in}}%
\pgfpathlineto{\pgfqpoint{1.911031in}{4.716813in}}%
\pgfpathlineto{\pgfqpoint{1.912105in}{4.719923in}}%
\pgfpathlineto{\pgfqpoint{1.913178in}{4.719479in}}%
\pgfpathlineto{\pgfqpoint{1.914252in}{4.727033in}}%
\pgfpathlineto{\pgfqpoint{1.917474in}{4.725700in}}%
\pgfpathlineto{\pgfqpoint{1.918548in}{4.733698in}}%
\pgfpathlineto{\pgfqpoint{1.919621in}{4.753249in}}%
\pgfpathlineto{\pgfqpoint{1.920695in}{4.752360in}}%
\pgfpathlineto{\pgfqpoint{1.921769in}{4.756359in}}%
\pgfpathlineto{\pgfqpoint{1.924991in}{4.760803in}}%
\pgfpathlineto{\pgfqpoint{1.927138in}{4.743918in}}%
\pgfpathlineto{\pgfqpoint{1.928212in}{4.749694in}}%
\pgfpathlineto{\pgfqpoint{1.929286in}{4.735031in}}%
\pgfpathlineto{\pgfqpoint{1.932507in}{4.743029in}}%
\pgfpathlineto{\pgfqpoint{1.933581in}{4.725700in}}%
\pgfpathlineto{\pgfqpoint{1.934655in}{4.726144in}}%
\pgfpathlineto{\pgfqpoint{1.935729in}{4.734142in}}%
\pgfpathlineto{\pgfqpoint{1.936803in}{4.720812in}}%
\pgfpathlineto{\pgfqpoint{1.940024in}{4.735920in}}%
\pgfpathlineto{\pgfqpoint{1.941098in}{4.730587in}}%
\pgfpathlineto{\pgfqpoint{1.942172in}{4.743029in}}%
\pgfpathlineto{\pgfqpoint{1.943246in}{4.731476in}}%
\pgfpathlineto{\pgfqpoint{1.944320in}{4.734142in}}%
\pgfpathlineto{\pgfqpoint{1.947541in}{4.736808in}}%
\pgfpathlineto{\pgfqpoint{1.948615in}{4.729254in}}%
\pgfpathlineto{\pgfqpoint{1.949689in}{4.715924in}}%
\pgfpathlineto{\pgfqpoint{1.950763in}{4.711925in}}%
\pgfpathlineto{\pgfqpoint{1.951837in}{4.714147in}}%
\pgfpathlineto{\pgfqpoint{1.955058in}{4.723478in}}%
\pgfpathlineto{\pgfqpoint{1.956132in}{4.711925in}}%
\pgfpathlineto{\pgfqpoint{1.957206in}{4.713258in}}%
\pgfpathlineto{\pgfqpoint{1.958280in}{4.716813in}}%
\pgfpathlineto{\pgfqpoint{1.962575in}{4.726144in}}%
\pgfpathlineto{\pgfqpoint{1.963649in}{4.719923in}}%
\pgfpathlineto{\pgfqpoint{1.964723in}{4.719479in}}%
\pgfpathlineto{\pgfqpoint{1.965796in}{4.746584in}}%
\pgfpathlineto{\pgfqpoint{1.966870in}{4.850115in}}%
\pgfpathlineto{\pgfqpoint{1.970092in}{4.817234in}}%
\pgfpathlineto{\pgfqpoint{1.971166in}{4.821233in}}%
\pgfpathlineto{\pgfqpoint{1.973313in}{4.804348in}}%
\pgfpathlineto{\pgfqpoint{1.974387in}{4.803459in}}%
\pgfpathlineto{\pgfqpoint{1.977609in}{4.794573in}}%
\pgfpathlineto{\pgfqpoint{1.978682in}{4.779909in}}%
\pgfpathlineto{\pgfqpoint{1.979756in}{4.790574in}}%
\pgfpathlineto{\pgfqpoint{1.981904in}{4.786574in}}%
\pgfpathlineto{\pgfqpoint{1.985126in}{4.789241in}}%
\pgfpathlineto{\pgfqpoint{1.986199in}{4.798572in}}%
\pgfpathlineto{\pgfqpoint{1.987273in}{4.797239in}}%
\pgfpathlineto{\pgfqpoint{1.988347in}{4.796794in}}%
\pgfpathlineto{\pgfqpoint{1.989421in}{4.805681in}}%
\pgfpathlineto{\pgfqpoint{1.992642in}{4.803904in}}%
\pgfpathlineto{\pgfqpoint{1.993716in}{4.791018in}}%
\pgfpathlineto{\pgfqpoint{1.994790in}{4.787019in}}%
\pgfpathlineto{\pgfqpoint{1.996938in}{4.807459in}}%
\pgfpathlineto{\pgfqpoint{2.000159in}{4.791018in}}%
\pgfpathlineto{\pgfqpoint{2.001233in}{4.795017in}}%
\pgfpathlineto{\pgfqpoint{2.003381in}{4.809236in}}%
\pgfpathlineto{\pgfqpoint{2.004455in}{4.803904in}}%
\pgfpathlineto{\pgfqpoint{2.007676in}{4.805681in}}%
\pgfpathlineto{\pgfqpoint{2.008750in}{4.807014in}}%
\pgfpathlineto{\pgfqpoint{2.009824in}{4.817678in}}%
\pgfpathlineto{\pgfqpoint{2.010898in}{4.820789in}}%
\pgfpathlineto{\pgfqpoint{2.011971in}{4.819456in}}%
\pgfpathlineto{\pgfqpoint{2.016267in}{4.813235in}}%
\pgfpathlineto{\pgfqpoint{2.017341in}{4.813235in}}%
\pgfpathlineto{\pgfqpoint{2.018415in}{4.817234in}}%
\pgfpathlineto{\pgfqpoint{2.019488in}{4.803904in}}%
\pgfpathlineto{\pgfqpoint{2.022710in}{4.804348in}}%
\pgfpathlineto{\pgfqpoint{2.023784in}{4.806126in}}%
\pgfpathlineto{\pgfqpoint{2.024858in}{4.813679in}}%
\pgfpathlineto{\pgfqpoint{2.025931in}{4.803459in}}%
\pgfpathlineto{\pgfqpoint{2.027005in}{4.804793in}}%
\pgfpathlineto{\pgfqpoint{2.030227in}{4.803015in}}%
\pgfpathlineto{\pgfqpoint{2.031301in}{4.806126in}}%
\pgfpathlineto{\pgfqpoint{2.032374in}{4.817234in}}%
\pgfpathlineto{\pgfqpoint{2.034522in}{4.808347in}}%
\pgfpathlineto{\pgfqpoint{2.037744in}{4.801682in}}%
\pgfpathlineto{\pgfqpoint{2.038817in}{4.802126in}}%
\pgfpathlineto{\pgfqpoint{2.039891in}{4.803904in}}%
\pgfpathlineto{\pgfqpoint{2.040965in}{4.816345in}}%
\pgfpathlineto{\pgfqpoint{2.042039in}{4.811458in}}%
\pgfpathlineto{\pgfqpoint{2.045260in}{4.818123in}}%
\pgfpathlineto{\pgfqpoint{2.046334in}{4.823011in}}%
\pgfpathlineto{\pgfqpoint{2.048482in}{4.803904in}}%
\pgfpathlineto{\pgfqpoint{2.049556in}{4.805681in}}%
\pgfpathlineto{\pgfqpoint{2.053851in}{4.786130in}}%
\pgfpathlineto{\pgfqpoint{2.055999in}{4.794128in}}%
\pgfpathlineto{\pgfqpoint{2.060294in}{4.776355in}}%
\pgfpathlineto{\pgfqpoint{2.061368in}{4.782575in}}%
\pgfpathlineto{\pgfqpoint{2.062442in}{4.760803in}}%
\pgfpathlineto{\pgfqpoint{2.063516in}{4.765690in}}%
\pgfpathlineto{\pgfqpoint{2.064590in}{4.775022in}}%
\pgfpathlineto{\pgfqpoint{2.067811in}{4.782575in}}%
\pgfpathlineto{\pgfqpoint{2.072106in}{4.811458in}}%
\pgfpathlineto{\pgfqpoint{2.075328in}{4.807903in}}%
\pgfpathlineto{\pgfqpoint{2.078549in}{4.774577in}}%
\pgfpathlineto{\pgfqpoint{2.079623in}{4.755471in}}%
\pgfpathlineto{\pgfqpoint{2.082845in}{4.763024in}}%
\pgfpathlineto{\pgfqpoint{2.084993in}{4.774577in}}%
\pgfpathlineto{\pgfqpoint{2.086066in}{4.769245in}}%
\pgfpathlineto{\pgfqpoint{2.087140in}{4.768801in}}%
\pgfpathlineto{\pgfqpoint{2.090362in}{4.759914in}}%
\pgfpathlineto{\pgfqpoint{2.091436in}{4.761247in}}%
\pgfpathlineto{\pgfqpoint{2.092509in}{4.768801in}}%
\pgfpathlineto{\pgfqpoint{2.093583in}{4.766135in}}%
\pgfpathlineto{\pgfqpoint{2.094657in}{4.756804in}}%
\pgfpathlineto{\pgfqpoint{2.097879in}{4.773689in}}%
\pgfpathlineto{\pgfqpoint{2.098952in}{4.753693in}}%
\pgfpathlineto{\pgfqpoint{2.100026in}{4.759470in}}%
\pgfpathlineto{\pgfqpoint{2.101100in}{4.756804in}}%
\pgfpathlineto{\pgfqpoint{2.102174in}{4.767912in}}%
\pgfpathlineto{\pgfqpoint{2.105395in}{4.772800in}}%
\pgfpathlineto{\pgfqpoint{2.106469in}{4.767468in}}%
\pgfpathlineto{\pgfqpoint{2.107543in}{4.754582in}}%
\pgfpathlineto{\pgfqpoint{2.109691in}{4.711925in}}%
\pgfpathlineto{\pgfqpoint{2.112912in}{4.684376in}}%
\pgfpathlineto{\pgfqpoint{2.113986in}{4.662159in}}%
\pgfpathlineto{\pgfqpoint{2.117208in}{4.733253in}}%
\pgfpathlineto{\pgfqpoint{2.120429in}{4.720368in}}%
\pgfpathlineto{\pgfqpoint{2.121503in}{4.684820in}}%
\pgfpathlineto{\pgfqpoint{2.122577in}{4.711036in}}%
\pgfpathlineto{\pgfqpoint{2.123651in}{4.708815in}}%
\pgfpathlineto{\pgfqpoint{2.124725in}{4.689708in}}%
\pgfpathlineto{\pgfqpoint{2.129020in}{4.725700in}}%
\pgfpathlineto{\pgfqpoint{2.130094in}{4.710148in}}%
\pgfpathlineto{\pgfqpoint{2.131168in}{4.715035in}}%
\pgfpathlineto{\pgfqpoint{2.132241in}{4.725255in}}%
\pgfpathlineto{\pgfqpoint{2.135463in}{4.718590in}}%
\pgfpathlineto{\pgfqpoint{2.137611in}{4.762136in}}%
\pgfpathlineto{\pgfqpoint{2.138684in}{4.748805in}}%
\pgfpathlineto{\pgfqpoint{2.139758in}{4.727921in}}%
\pgfpathlineto{\pgfqpoint{2.142980in}{4.739030in}}%
\pgfpathlineto{\pgfqpoint{2.145127in}{4.740807in}}%
\pgfpathlineto{\pgfqpoint{2.146201in}{4.732365in}}%
\pgfpathlineto{\pgfqpoint{2.147275in}{4.732365in}}%
\pgfpathlineto{\pgfqpoint{2.150497in}{4.709259in}}%
\pgfpathlineto{\pgfqpoint{2.151570in}{4.719035in}}%
\pgfpathlineto{\pgfqpoint{2.152644in}{4.743918in}}%
\pgfpathlineto{\pgfqpoint{2.153718in}{4.742585in}}%
\pgfpathlineto{\pgfqpoint{2.154792in}{4.753249in}}%
\pgfpathlineto{\pgfqpoint{2.161235in}{4.850560in}}%
\pgfpathlineto{\pgfqpoint{2.162309in}{4.852337in}}%
\pgfpathlineto{\pgfqpoint{2.165530in}{4.853226in}}%
\pgfpathlineto{\pgfqpoint{2.167678in}{4.834563in}}%
\pgfpathlineto{\pgfqpoint{2.168752in}{4.850560in}}%
\pgfpathlineto{\pgfqpoint{2.169826in}{4.886996in}}%
\pgfpathlineto{\pgfqpoint{2.173047in}{4.887440in}}%
\pgfpathlineto{\pgfqpoint{2.174121in}{4.879442in}}%
\pgfpathlineto{\pgfqpoint{2.175195in}{4.882108in}}%
\pgfpathlineto{\pgfqpoint{2.176269in}{4.909657in}}%
\pgfpathlineto{\pgfqpoint{2.177343in}{4.906991in}}%
\pgfpathlineto{\pgfqpoint{2.180564in}{4.908324in}}%
\pgfpathlineto{\pgfqpoint{2.183786in}{4.900326in}}%
\pgfpathlineto{\pgfqpoint{2.184859in}{4.884774in}}%
\pgfpathlineto{\pgfqpoint{2.189155in}{4.910102in}}%
\pgfpathlineto{\pgfqpoint{2.190229in}{4.908324in}}%
\pgfpathlineto{\pgfqpoint{2.191303in}{4.911879in}}%
\pgfpathlineto{\pgfqpoint{2.192376in}{4.922543in}}%
\pgfpathlineto{\pgfqpoint{2.195598in}{4.916322in}}%
\pgfpathlineto{\pgfqpoint{2.197746in}{4.950981in}}%
\pgfpathlineto{\pgfqpoint{2.198819in}{4.931874in}}%
\pgfpathlineto{\pgfqpoint{2.199893in}{4.936318in}}%
\pgfpathlineto{\pgfqpoint{2.203115in}{4.939428in}}%
\pgfpathlineto{\pgfqpoint{2.204189in}{4.937651in}}%
\pgfpathlineto{\pgfqpoint{2.205262in}{4.945205in}}%
\pgfpathlineto{\pgfqpoint{2.206336in}{4.935873in}}%
\pgfpathlineto{\pgfqpoint{2.207410in}{4.950537in}}%
\pgfpathlineto{\pgfqpoint{2.210632in}{4.947871in}}%
\pgfpathlineto{\pgfqpoint{2.211705in}{4.950537in}}%
\pgfpathlineto{\pgfqpoint{2.212779in}{4.939428in}}%
\pgfpathlineto{\pgfqpoint{2.214927in}{4.939428in}}%
\pgfpathlineto{\pgfqpoint{2.218148in}{4.923432in}}%
\pgfpathlineto{\pgfqpoint{2.219222in}{4.931874in}}%
\pgfpathlineto{\pgfqpoint{2.220296in}{4.924320in}}%
\pgfpathlineto{\pgfqpoint{2.221370in}{4.926542in}}%
\pgfpathlineto{\pgfqpoint{2.222444in}{4.944316in}}%
\pgfpathlineto{\pgfqpoint{2.225665in}{4.939872in}}%
\pgfpathlineto{\pgfqpoint{2.226739in}{4.932763in}}%
\pgfpathlineto{\pgfqpoint{2.228887in}{4.950092in}}%
\pgfpathlineto{\pgfqpoint{2.229961in}{4.935429in}}%
\pgfpathlineto{\pgfqpoint{2.233182in}{4.935429in}}%
\pgfpathlineto{\pgfqpoint{2.234256in}{4.937651in}}%
\pgfpathlineto{\pgfqpoint{2.235330in}{4.962978in}}%
\pgfpathlineto{\pgfqpoint{2.237478in}{4.944760in}}%
\pgfpathlineto{\pgfqpoint{2.240699in}{4.949648in}}%
\pgfpathlineto{\pgfqpoint{2.241773in}{4.952758in}}%
\pgfpathlineto{\pgfqpoint{2.242847in}{4.970532in}}%
\pgfpathlineto{\pgfqpoint{2.243921in}{4.966089in}}%
\pgfpathlineto{\pgfqpoint{2.248216in}{4.968755in}}%
\pgfpathlineto{\pgfqpoint{2.249290in}{4.982974in}}%
\pgfpathlineto{\pgfqpoint{2.250364in}{4.974531in}}%
\pgfpathlineto{\pgfqpoint{2.251437in}{4.978086in}}%
\pgfpathlineto{\pgfqpoint{2.255733in}{4.961201in}}%
\pgfpathlineto{\pgfqpoint{2.256807in}{4.962534in}}%
\pgfpathlineto{\pgfqpoint{2.257881in}{4.943871in}}%
\pgfpathlineto{\pgfqpoint{2.258954in}{4.894550in}}%
\pgfpathlineto{\pgfqpoint{2.260028in}{4.874999in}}%
\pgfpathlineto{\pgfqpoint{2.264324in}{4.882108in}}%
\pgfpathlineto{\pgfqpoint{2.265397in}{4.866556in}}%
\pgfpathlineto{\pgfqpoint{2.266471in}{4.898104in}}%
\pgfpathlineto{\pgfqpoint{2.267545in}{4.876332in}}%
\pgfpathlineto{\pgfqpoint{2.271840in}{4.876332in}}%
\pgfpathlineto{\pgfqpoint{2.272914in}{4.857669in}}%
\pgfpathlineto{\pgfqpoint{2.273988in}{4.880331in}}%
\pgfpathlineto{\pgfqpoint{2.275062in}{4.866556in}}%
\pgfpathlineto{\pgfqpoint{2.278283in}{4.859002in}}%
\pgfpathlineto{\pgfqpoint{2.279357in}{4.869222in}}%
\pgfpathlineto{\pgfqpoint{2.280431in}{4.857669in}}%
\pgfpathlineto{\pgfqpoint{2.281505in}{4.865667in}}%
\pgfpathlineto{\pgfqpoint{2.282579in}{4.899882in}}%
\pgfpathlineto{\pgfqpoint{2.285800in}{4.882108in}}%
\pgfpathlineto{\pgfqpoint{2.286874in}{4.866556in}}%
\pgfpathlineto{\pgfqpoint{2.289022in}{4.902548in}}%
\pgfpathlineto{\pgfqpoint{2.290096in}{4.878109in}}%
\pgfpathlineto{\pgfqpoint{2.293317in}{4.863890in}}%
\pgfpathlineto{\pgfqpoint{2.294391in}{4.868333in}}%
\pgfpathlineto{\pgfqpoint{2.295465in}{4.869222in}}%
\pgfpathlineto{\pgfqpoint{2.296539in}{4.836341in}}%
\pgfpathlineto{\pgfqpoint{2.297613in}{4.867445in}}%
\pgfpathlineto{\pgfqpoint{2.301908in}{4.890550in}}%
\pgfpathlineto{\pgfqpoint{2.302982in}{4.908768in}}%
\pgfpathlineto{\pgfqpoint{2.304056in}{4.898993in}}%
\pgfpathlineto{\pgfqpoint{2.305129in}{4.896771in}}%
\pgfpathlineto{\pgfqpoint{2.308351in}{4.911435in}}%
\pgfpathlineto{\pgfqpoint{2.310499in}{4.894105in}}%
\pgfpathlineto{\pgfqpoint{2.311572in}{4.913656in}}%
\pgfpathlineto{\pgfqpoint{2.312646in}{4.920321in}}%
\pgfpathlineto{\pgfqpoint{2.315868in}{4.910102in}}%
\pgfpathlineto{\pgfqpoint{2.316942in}{4.938539in}}%
\pgfpathlineto{\pgfqpoint{2.318015in}{4.950092in}}%
\pgfpathlineto{\pgfqpoint{2.319089in}{4.951870in}}%
\pgfpathlineto{\pgfqpoint{2.320163in}{4.961201in}}%
\pgfpathlineto{\pgfqpoint{2.323385in}{4.954536in}}%
\pgfpathlineto{\pgfqpoint{2.324458in}{4.945649in}}%
\pgfpathlineto{\pgfqpoint{2.325532in}{4.945205in}}%
\pgfpathlineto{\pgfqpoint{2.326606in}{4.941205in}}%
\pgfpathlineto{\pgfqpoint{2.327680in}{4.956313in}}%
\pgfpathlineto{\pgfqpoint{2.330902in}{4.953647in}}%
\pgfpathlineto{\pgfqpoint{2.331975in}{4.954091in}}%
\pgfpathlineto{\pgfqpoint{2.333049in}{4.950092in}}%
\pgfpathlineto{\pgfqpoint{2.334123in}{4.980308in}}%
\pgfpathlineto{\pgfqpoint{2.335197in}{4.978974in}}%
\pgfpathlineto{\pgfqpoint{2.338418in}{4.985195in}}%
\pgfpathlineto{\pgfqpoint{2.339492in}{4.984307in}}%
\pgfpathlineto{\pgfqpoint{2.340566in}{4.984751in}}%
\pgfpathlineto{\pgfqpoint{2.341640in}{4.986084in}}%
\pgfpathlineto{\pgfqpoint{2.345935in}{5.000747in}}%
\pgfpathlineto{\pgfqpoint{2.347009in}{5.000303in}}%
\pgfpathlineto{\pgfqpoint{2.348083in}{5.014077in}}%
\pgfpathlineto{\pgfqpoint{2.349157in}{5.012300in}}%
\pgfpathlineto{\pgfqpoint{2.350231in}{5.017632in}}%
\pgfpathlineto{\pgfqpoint{2.354526in}{4.981196in}}%
\pgfpathlineto{\pgfqpoint{2.355600in}{4.978086in}}%
\pgfpathlineto{\pgfqpoint{2.356674in}{4.967866in}}%
\pgfpathlineto{\pgfqpoint{2.357747in}{4.973642in}}%
\pgfpathlineto{\pgfqpoint{2.360969in}{4.970532in}}%
\pgfpathlineto{\pgfqpoint{2.362043in}{4.974531in}}%
\pgfpathlineto{\pgfqpoint{2.363117in}{4.981196in}}%
\pgfpathlineto{\pgfqpoint{2.364191in}{4.982529in}}%
\pgfpathlineto{\pgfqpoint{2.368486in}{4.984307in}}%
\pgfpathlineto{\pgfqpoint{2.369560in}{4.987861in}}%
\pgfpathlineto{\pgfqpoint{2.370634in}{4.987861in}}%
\pgfpathlineto{\pgfqpoint{2.372781in}{4.972754in}}%
\pgfpathlineto{\pgfqpoint{2.376003in}{4.969643in}}%
\pgfpathlineto{\pgfqpoint{2.377077in}{4.978086in}}%
\pgfpathlineto{\pgfqpoint{2.378150in}{4.979419in}}%
\pgfpathlineto{\pgfqpoint{2.379224in}{4.978086in}}%
\pgfpathlineto{\pgfqpoint{2.380298in}{4.972309in}}%
\pgfpathlineto{\pgfqpoint{2.383520in}{4.977641in}}%
\pgfpathlineto{\pgfqpoint{2.384593in}{4.967866in}}%
\pgfpathlineto{\pgfqpoint{2.385667in}{4.946093in}}%
\pgfpathlineto{\pgfqpoint{2.386741in}{4.938984in}}%
\pgfpathlineto{\pgfqpoint{2.387815in}{4.947871in}}%
\pgfpathlineto{\pgfqpoint{2.391036in}{4.938095in}}%
\pgfpathlineto{\pgfqpoint{2.392110in}{4.961645in}}%
\pgfpathlineto{\pgfqpoint{2.393184in}{4.956313in}}%
\pgfpathlineto{\pgfqpoint{2.394258in}{4.946982in}}%
\pgfpathlineto{\pgfqpoint{2.395332in}{4.929653in}}%
\pgfpathlineto{\pgfqpoint{2.398553in}{4.941650in}}%
\pgfpathlineto{\pgfqpoint{2.399627in}{4.932319in}}%
\pgfpathlineto{\pgfqpoint{2.400701in}{4.928320in}}%
\pgfpathlineto{\pgfqpoint{2.401775in}{4.918544in}}%
\pgfpathlineto{\pgfqpoint{2.402849in}{4.926542in}}%
\pgfpathlineto{\pgfqpoint{2.406070in}{4.923432in}}%
\pgfpathlineto{\pgfqpoint{2.408218in}{4.946982in}}%
\pgfpathlineto{\pgfqpoint{2.409292in}{4.944316in}}%
\pgfpathlineto{\pgfqpoint{2.410366in}{4.947871in}}%
\pgfpathlineto{\pgfqpoint{2.414661in}{4.952314in}}%
\pgfpathlineto{\pgfqpoint{2.415735in}{4.947426in}}%
\pgfpathlineto{\pgfqpoint{2.416809in}{4.945205in}}%
\pgfpathlineto{\pgfqpoint{2.417882in}{4.941205in}}%
\pgfpathlineto{\pgfqpoint{2.421104in}{4.947871in}}%
\pgfpathlineto{\pgfqpoint{2.422178in}{4.948759in}}%
\pgfpathlineto{\pgfqpoint{2.423252in}{4.955424in}}%
\pgfpathlineto{\pgfqpoint{2.424325in}{4.952758in}}%
\pgfpathlineto{\pgfqpoint{2.425399in}{4.944760in}}%
\pgfpathlineto{\pgfqpoint{2.428621in}{4.936762in}}%
\pgfpathlineto{\pgfqpoint{2.429695in}{4.960312in}}%
\pgfpathlineto{\pgfqpoint{2.430769in}{4.966089in}}%
\pgfpathlineto{\pgfqpoint{2.431842in}{4.977197in}}%
\pgfpathlineto{\pgfqpoint{2.432916in}{4.975420in}}%
\pgfpathlineto{\pgfqpoint{2.437212in}{4.988750in}}%
\pgfpathlineto{\pgfqpoint{2.438285in}{4.982529in}}%
\pgfpathlineto{\pgfqpoint{2.439359in}{4.998526in}}%
\pgfpathlineto{\pgfqpoint{2.440433in}{4.945205in}}%
\pgfpathlineto{\pgfqpoint{2.443655in}{4.925653in}}%
\pgfpathlineto{\pgfqpoint{2.447950in}{5.010078in}}%
\pgfpathlineto{\pgfqpoint{2.452245in}{5.008301in}}%
\pgfpathlineto{\pgfqpoint{2.453319in}{5.019854in}}%
\pgfpathlineto{\pgfqpoint{2.454393in}{5.022964in}}%
\pgfpathlineto{\pgfqpoint{2.455467in}{5.037628in}}%
\pgfpathlineto{\pgfqpoint{2.458688in}{5.038072in}}%
\pgfpathlineto{\pgfqpoint{2.459762in}{5.039849in}}%
\pgfpathlineto{\pgfqpoint{2.460836in}{5.043848in}}%
\pgfpathlineto{\pgfqpoint{2.462984in}{5.064288in}}%
\pgfpathlineto{\pgfqpoint{2.466205in}{5.065177in}}%
\pgfpathlineto{\pgfqpoint{2.467279in}{5.066065in}}%
\pgfpathlineto{\pgfqpoint{2.469427in}{5.052735in}}%
\pgfpathlineto{\pgfqpoint{2.470501in}{5.032295in}}%
\pgfpathlineto{\pgfqpoint{2.478017in}{4.996304in}}%
\pgfpathlineto{\pgfqpoint{2.481239in}{4.996748in}}%
\pgfpathlineto{\pgfqpoint{2.482313in}{4.992749in}}%
\pgfpathlineto{\pgfqpoint{2.485534in}{5.002080in}}%
\pgfpathlineto{\pgfqpoint{2.488756in}{5.001636in}}%
\pgfpathlineto{\pgfqpoint{2.489830in}{5.002525in}}%
\pgfpathlineto{\pgfqpoint{2.490903in}{5.001636in}}%
\pgfpathlineto{\pgfqpoint{2.491977in}{5.002080in}}%
\pgfpathlineto{\pgfqpoint{2.493051in}{5.000303in}}%
\pgfpathlineto{\pgfqpoint{2.496273in}{5.000303in}}%
\pgfpathlineto{\pgfqpoint{2.497346in}{4.998526in}}%
\pgfpathlineto{\pgfqpoint{2.498420in}{5.002080in}}%
\pgfpathlineto{\pgfqpoint{2.499494in}{5.007857in}}%
\pgfpathlineto{\pgfqpoint{2.500568in}{5.000747in}}%
\pgfpathlineto{\pgfqpoint{2.503790in}{5.003413in}}%
\pgfpathlineto{\pgfqpoint{2.504863in}{4.999859in}}%
\pgfpathlineto{\pgfqpoint{2.507011in}{4.998970in}}%
\pgfpathlineto{\pgfqpoint{2.508085in}{4.999859in}}%
\pgfpathlineto{\pgfqpoint{2.511306in}{5.005191in}}%
\pgfpathlineto{\pgfqpoint{2.512380in}{5.005191in}}%
\pgfpathlineto{\pgfqpoint{2.513454in}{5.000303in}}%
\pgfpathlineto{\pgfqpoint{2.514528in}{4.998970in}}%
\pgfpathlineto{\pgfqpoint{2.515602in}{5.002080in}}%
\pgfpathlineto{\pgfqpoint{2.519897in}{4.992749in}}%
\pgfpathlineto{\pgfqpoint{2.520971in}{4.993193in}}%
\pgfpathlineto{\pgfqpoint{2.522045in}{4.992749in}}%
\pgfpathlineto{\pgfqpoint{2.523119in}{4.956313in}}%
\pgfpathlineto{\pgfqpoint{2.526340in}{4.970976in}}%
\pgfpathlineto{\pgfqpoint{2.527414in}{4.946093in}}%
\pgfpathlineto{\pgfqpoint{2.528488in}{4.940317in}}%
\pgfpathlineto{\pgfqpoint{2.529562in}{4.951425in}}%
\pgfpathlineto{\pgfqpoint{2.530635in}{4.948759in}}%
\pgfpathlineto{\pgfqpoint{2.533857in}{4.938984in}}%
\pgfpathlineto{\pgfqpoint{2.537079in}{4.962978in}}%
\pgfpathlineto{\pgfqpoint{2.538152in}{4.956757in}}%
\pgfpathlineto{\pgfqpoint{2.541374in}{4.942983in}}%
\pgfpathlineto{\pgfqpoint{2.542448in}{4.956313in}}%
\pgfpathlineto{\pgfqpoint{2.543522in}{4.957202in}}%
\pgfpathlineto{\pgfqpoint{2.544595in}{4.942983in}}%
\pgfpathlineto{\pgfqpoint{2.545669in}{4.946093in}}%
\pgfpathlineto{\pgfqpoint{2.548891in}{4.946982in}}%
\pgfpathlineto{\pgfqpoint{2.549965in}{4.941650in}}%
\pgfpathlineto{\pgfqpoint{2.551038in}{4.941650in}}%
\pgfpathlineto{\pgfqpoint{2.553186in}{4.925209in}}%
\pgfpathlineto{\pgfqpoint{2.556408in}{4.916322in}}%
\pgfpathlineto{\pgfqpoint{2.557481in}{4.918988in}}%
\pgfpathlineto{\pgfqpoint{2.558555in}{4.918100in}}%
\pgfpathlineto{\pgfqpoint{2.559629in}{4.913212in}}%
\pgfpathlineto{\pgfqpoint{2.560703in}{4.917655in}}%
\pgfpathlineto{\pgfqpoint{2.563924in}{4.916322in}}%
\pgfpathlineto{\pgfqpoint{2.566072in}{4.924320in}}%
\pgfpathlineto{\pgfqpoint{2.567146in}{4.924765in}}%
\pgfpathlineto{\pgfqpoint{2.568220in}{4.921210in}}%
\pgfpathlineto{\pgfqpoint{2.571441in}{4.918988in}}%
\pgfpathlineto{\pgfqpoint{2.572515in}{4.908324in}}%
\pgfpathlineto{\pgfqpoint{2.573589in}{4.916767in}}%
\pgfpathlineto{\pgfqpoint{2.574663in}{4.907435in}}%
\pgfpathlineto{\pgfqpoint{2.575737in}{4.930541in}}%
\pgfpathlineto{\pgfqpoint{2.578958in}{4.926098in}}%
\pgfpathlineto{\pgfqpoint{2.580032in}{4.917211in}}%
\pgfpathlineto{\pgfqpoint{2.582180in}{4.893661in}}%
\pgfpathlineto{\pgfqpoint{2.583254in}{4.899882in}}%
\pgfpathlineto{\pgfqpoint{2.586475in}{4.934096in}}%
\pgfpathlineto{\pgfqpoint{2.587549in}{4.938539in}}%
\pgfpathlineto{\pgfqpoint{2.588623in}{4.946538in}}%
\pgfpathlineto{\pgfqpoint{2.589697in}{4.977197in}}%
\pgfpathlineto{\pgfqpoint{2.590770in}{4.989194in}}%
\pgfpathlineto{\pgfqpoint{2.593992in}{4.981196in}}%
\pgfpathlineto{\pgfqpoint{2.595066in}{4.990527in}}%
\pgfpathlineto{\pgfqpoint{2.596140in}{4.990083in}}%
\pgfpathlineto{\pgfqpoint{2.597213in}{4.992305in}}%
\pgfpathlineto{\pgfqpoint{2.598287in}{4.987417in}}%
\pgfpathlineto{\pgfqpoint{2.601509in}{4.995415in}}%
\pgfpathlineto{\pgfqpoint{2.603657in}{5.013633in}}%
\pgfpathlineto{\pgfqpoint{2.605804in}{5.017632in}}%
\pgfpathlineto{\pgfqpoint{2.609026in}{5.010078in}}%
\pgfpathlineto{\pgfqpoint{2.610100in}{5.002525in}}%
\pgfpathlineto{\pgfqpoint{2.611173in}{4.990972in}}%
\pgfpathlineto{\pgfqpoint{2.612247in}{5.015855in}}%
\pgfpathlineto{\pgfqpoint{2.613321in}{5.013633in}}%
\pgfpathlineto{\pgfqpoint{2.616543in}{5.004746in}}%
\pgfpathlineto{\pgfqpoint{2.617616in}{5.006968in}}%
\pgfpathlineto{\pgfqpoint{2.618690in}{5.023853in}}%
\pgfpathlineto{\pgfqpoint{2.619764in}{5.021187in}}%
\pgfpathlineto{\pgfqpoint{2.620838in}{5.030962in}}%
\pgfpathlineto{\pgfqpoint{2.624059in}{5.034073in}}%
\pgfpathlineto{\pgfqpoint{2.625133in}{5.029185in}}%
\pgfpathlineto{\pgfqpoint{2.627281in}{5.010523in}}%
\pgfpathlineto{\pgfqpoint{2.628355in}{5.029629in}}%
\pgfpathlineto{\pgfqpoint{2.631576in}{5.036295in}}%
\pgfpathlineto{\pgfqpoint{2.632650in}{5.049180in}}%
\pgfpathlineto{\pgfqpoint{2.634798in}{5.042071in}}%
\pgfpathlineto{\pgfqpoint{2.635872in}{5.044293in}}%
\pgfpathlineto{\pgfqpoint{2.640167in}{5.045181in}}%
\pgfpathlineto{\pgfqpoint{2.641241in}{5.037183in}}%
\pgfpathlineto{\pgfqpoint{2.642315in}{5.037628in}}%
\pgfpathlineto{\pgfqpoint{2.643389in}{5.033184in}}%
\pgfpathlineto{\pgfqpoint{2.648758in}{5.037183in}}%
\pgfpathlineto{\pgfqpoint{2.649832in}{5.030074in}}%
\pgfpathlineto{\pgfqpoint{2.650905in}{5.033629in}}%
\pgfpathlineto{\pgfqpoint{2.654127in}{5.027852in}}%
\pgfpathlineto{\pgfqpoint{2.655201in}{5.024297in}}%
\pgfpathlineto{\pgfqpoint{2.656275in}{5.028296in}}%
\pgfpathlineto{\pgfqpoint{2.657348in}{5.024742in}}%
\pgfpathlineto{\pgfqpoint{2.662718in}{5.020298in}}%
\pgfpathlineto{\pgfqpoint{2.664865in}{5.017632in}}%
\pgfpathlineto{\pgfqpoint{2.665939in}{4.990972in}}%
\pgfpathlineto{\pgfqpoint{2.669161in}{4.960312in}}%
\pgfpathlineto{\pgfqpoint{2.671308in}{4.984751in}}%
\pgfpathlineto{\pgfqpoint{2.672382in}{4.982529in}}%
\pgfpathlineto{\pgfqpoint{2.673456in}{4.970532in}}%
\pgfpathlineto{\pgfqpoint{2.676678in}{4.968310in}}%
\pgfpathlineto{\pgfqpoint{2.677751in}{4.958090in}}%
\pgfpathlineto{\pgfqpoint{2.679899in}{4.957202in}}%
\pgfpathlineto{\pgfqpoint{2.680973in}{4.958090in}}%
\pgfpathlineto{\pgfqpoint{2.684194in}{4.956757in}}%
\pgfpathlineto{\pgfqpoint{2.686342in}{4.947426in}}%
\pgfpathlineto{\pgfqpoint{2.688490in}{4.958979in}}%
\pgfpathlineto{\pgfqpoint{2.691711in}{4.971421in}}%
\pgfpathlineto{\pgfqpoint{2.692785in}{4.981196in}}%
\pgfpathlineto{\pgfqpoint{2.694933in}{4.987861in}}%
\pgfpathlineto{\pgfqpoint{2.696007in}{4.984751in}}%
\pgfpathlineto{\pgfqpoint{2.700302in}{4.990527in}}%
\pgfpathlineto{\pgfqpoint{2.701376in}{4.982974in}}%
\pgfpathlineto{\pgfqpoint{2.702450in}{4.980308in}}%
\pgfpathlineto{\pgfqpoint{2.703523in}{4.986973in}}%
\pgfpathlineto{\pgfqpoint{2.706745in}{4.977197in}}%
\pgfpathlineto{\pgfqpoint{2.707819in}{4.971865in}}%
\pgfpathlineto{\pgfqpoint{2.708893in}{4.986973in}}%
\pgfpathlineto{\pgfqpoint{2.709967in}{4.986973in}}%
\pgfpathlineto{\pgfqpoint{2.711040in}{4.984307in}}%
\pgfpathlineto{\pgfqpoint{2.714262in}{4.979419in}}%
\pgfpathlineto{\pgfqpoint{2.717483in}{4.966089in}}%
\pgfpathlineto{\pgfqpoint{2.718557in}{4.990527in}}%
\pgfpathlineto{\pgfqpoint{2.721779in}{4.974087in}}%
\pgfpathlineto{\pgfqpoint{2.722853in}{4.961201in}}%
\pgfpathlineto{\pgfqpoint{2.723926in}{4.970088in}}%
\pgfpathlineto{\pgfqpoint{2.725000in}{4.969643in}}%
\pgfpathlineto{\pgfqpoint{2.726074in}{4.974531in}}%
\pgfpathlineto{\pgfqpoint{2.729296in}{4.969199in}}%
\pgfpathlineto{\pgfqpoint{2.730369in}{4.955424in}}%
\pgfpathlineto{\pgfqpoint{2.732517in}{4.964311in}}%
\pgfpathlineto{\pgfqpoint{2.733591in}{4.968310in}}%
\pgfpathlineto{\pgfqpoint{2.736812in}{4.957202in}}%
\pgfpathlineto{\pgfqpoint{2.737886in}{4.964311in}}%
\pgfpathlineto{\pgfqpoint{2.738960in}{4.966977in}}%
\pgfpathlineto{\pgfqpoint{2.740034in}{4.974531in}}%
\pgfpathlineto{\pgfqpoint{2.741108in}{4.971421in}}%
\pgfpathlineto{\pgfqpoint{2.744329in}{4.974531in}}%
\pgfpathlineto{\pgfqpoint{2.745403in}{4.980308in}}%
\pgfpathlineto{\pgfqpoint{2.747551in}{4.976753in}}%
\pgfpathlineto{\pgfqpoint{2.748625in}{4.978974in}}%
\pgfpathlineto{\pgfqpoint{2.751846in}{4.979863in}}%
\pgfpathlineto{\pgfqpoint{2.752920in}{4.981196in}}%
\pgfpathlineto{\pgfqpoint{2.755068in}{4.962089in}}%
\pgfpathlineto{\pgfqpoint{2.759363in}{4.965200in}}%
\pgfpathlineto{\pgfqpoint{2.762585in}{4.990083in}}%
\pgfpathlineto{\pgfqpoint{2.763658in}{4.961645in}}%
\pgfpathlineto{\pgfqpoint{2.766880in}{4.961645in}}%
\pgfpathlineto{\pgfqpoint{2.767954in}{4.957646in}}%
\pgfpathlineto{\pgfqpoint{2.770101in}{4.942983in}}%
\pgfpathlineto{\pgfqpoint{2.771175in}{4.939428in}}%
\pgfpathlineto{\pgfqpoint{2.774397in}{4.937206in}}%
\pgfpathlineto{\pgfqpoint{2.775471in}{4.939428in}}%
\pgfpathlineto{\pgfqpoint{2.776545in}{4.948759in}}%
\pgfpathlineto{\pgfqpoint{2.777618in}{4.947871in}}%
\pgfpathlineto{\pgfqpoint{2.778692in}{4.948315in}}%
\pgfpathlineto{\pgfqpoint{2.781914in}{4.942538in}}%
\pgfpathlineto{\pgfqpoint{2.782988in}{4.936762in}}%
\pgfpathlineto{\pgfqpoint{2.784061in}{4.927875in}}%
\pgfpathlineto{\pgfqpoint{2.785135in}{4.934540in}}%
\pgfpathlineto{\pgfqpoint{2.786209in}{4.910546in}}%
\pgfpathlineto{\pgfqpoint{2.789431in}{4.906991in}}%
\pgfpathlineto{\pgfqpoint{2.790504in}{4.901659in}}%
\pgfpathlineto{\pgfqpoint{2.791578in}{4.876332in}}%
\pgfpathlineto{\pgfqpoint{2.792652in}{4.878998in}}%
\pgfpathlineto{\pgfqpoint{2.793726in}{4.902103in}}%
\pgfpathlineto{\pgfqpoint{2.796947in}{4.906991in}}%
\pgfpathlineto{\pgfqpoint{2.798021in}{4.910990in}}%
\pgfpathlineto{\pgfqpoint{2.800169in}{4.879442in}}%
\pgfpathlineto{\pgfqpoint{2.801243in}{4.878109in}}%
\pgfpathlineto{\pgfqpoint{2.805538in}{4.874554in}}%
\pgfpathlineto{\pgfqpoint{2.806612in}{4.875443in}}%
\pgfpathlineto{\pgfqpoint{2.807686in}{4.888773in}}%
\pgfpathlineto{\pgfqpoint{2.808760in}{4.894994in}}%
\pgfpathlineto{\pgfqpoint{2.811981in}{4.898993in}}%
\pgfpathlineto{\pgfqpoint{2.813055in}{4.897216in}}%
\pgfpathlineto{\pgfqpoint{2.814129in}{4.886996in}}%
\pgfpathlineto{\pgfqpoint{2.815203in}{4.883441in}}%
\pgfpathlineto{\pgfqpoint{2.819498in}{4.937206in}}%
\pgfpathlineto{\pgfqpoint{2.820572in}{4.917655in}}%
\pgfpathlineto{\pgfqpoint{2.821646in}{4.927431in}}%
\pgfpathlineto{\pgfqpoint{2.823793in}{4.959423in}}%
\pgfpathlineto{\pgfqpoint{2.827015in}{4.951425in}}%
\pgfpathlineto{\pgfqpoint{2.829163in}{4.910102in}}%
\pgfpathlineto{\pgfqpoint{2.830236in}{4.900770in}}%
\pgfpathlineto{\pgfqpoint{2.831310in}{4.901659in}}%
\pgfpathlineto{\pgfqpoint{2.834532in}{4.903436in}}%
\pgfpathlineto{\pgfqpoint{2.835606in}{4.886996in}}%
\pgfpathlineto{\pgfqpoint{2.836679in}{4.881664in}}%
\pgfpathlineto{\pgfqpoint{2.837753in}{4.879442in}}%
\pgfpathlineto{\pgfqpoint{2.838827in}{4.878998in}}%
\pgfpathlineto{\pgfqpoint{2.842049in}{4.896771in}}%
\pgfpathlineto{\pgfqpoint{2.844196in}{4.892772in}}%
\pgfpathlineto{\pgfqpoint{2.845270in}{4.850560in}}%
\pgfpathlineto{\pgfqpoint{2.846344in}{4.844339in}}%
\pgfpathlineto{\pgfqpoint{2.849566in}{4.839896in}}%
\pgfpathlineto{\pgfqpoint{2.851713in}{4.861668in}}%
\pgfpathlineto{\pgfqpoint{2.852787in}{4.870111in}}%
\pgfpathlineto{\pgfqpoint{2.853861in}{4.869666in}}%
\pgfpathlineto{\pgfqpoint{2.857082in}{4.871444in}}%
\pgfpathlineto{\pgfqpoint{2.859230in}{4.876332in}}%
\pgfpathlineto{\pgfqpoint{2.860304in}{4.866112in}}%
\pgfpathlineto{\pgfqpoint{2.861378in}{4.834563in}}%
\pgfpathlineto{\pgfqpoint{2.864599in}{4.815012in}}%
\pgfpathlineto{\pgfqpoint{2.865673in}{4.815457in}}%
\pgfpathlineto{\pgfqpoint{2.867821in}{4.829676in}}%
\pgfpathlineto{\pgfqpoint{2.868895in}{4.819011in}}%
\pgfpathlineto{\pgfqpoint{2.872116in}{4.822122in}}%
\pgfpathlineto{\pgfqpoint{2.873190in}{4.815457in}}%
\pgfpathlineto{\pgfqpoint{2.874264in}{4.818567in}}%
\pgfpathlineto{\pgfqpoint{2.875338in}{4.828343in}}%
\pgfpathlineto{\pgfqpoint{2.876411in}{4.829231in}}%
\pgfpathlineto{\pgfqpoint{2.880707in}{4.820344in}}%
\pgfpathlineto{\pgfqpoint{2.881781in}{4.826565in}}%
\pgfpathlineto{\pgfqpoint{2.882855in}{4.809680in}}%
\pgfpathlineto{\pgfqpoint{2.883928in}{4.805681in}}%
\pgfpathlineto{\pgfqpoint{2.887150in}{4.812346in}}%
\pgfpathlineto{\pgfqpoint{2.888224in}{4.803459in}}%
\pgfpathlineto{\pgfqpoint{2.889298in}{4.801682in}}%
\pgfpathlineto{\pgfqpoint{2.891445in}{4.779465in}}%
\pgfpathlineto{\pgfqpoint{2.894667in}{4.776799in}}%
\pgfpathlineto{\pgfqpoint{2.895741in}{4.781242in}}%
\pgfpathlineto{\pgfqpoint{2.896814in}{4.772800in}}%
\pgfpathlineto{\pgfqpoint{2.897888in}{4.769245in}}%
\pgfpathlineto{\pgfqpoint{2.898962in}{4.776799in}}%
\pgfpathlineto{\pgfqpoint{2.902184in}{4.776355in}}%
\pgfpathlineto{\pgfqpoint{2.903257in}{4.775022in}}%
\pgfpathlineto{\pgfqpoint{2.904331in}{4.768356in}}%
\pgfpathlineto{\pgfqpoint{2.905405in}{4.779465in}}%
\pgfpathlineto{\pgfqpoint{2.906479in}{4.803459in}}%
\pgfpathlineto{\pgfqpoint{2.910774in}{4.787463in}}%
\pgfpathlineto{\pgfqpoint{2.911848in}{4.794573in}}%
\pgfpathlineto{\pgfqpoint{2.912922in}{4.758137in}}%
\pgfpathlineto{\pgfqpoint{2.913996in}{4.749694in}}%
\pgfpathlineto{\pgfqpoint{2.917217in}{4.745695in}}%
\pgfpathlineto{\pgfqpoint{2.920439in}{4.767468in}}%
\pgfpathlineto{\pgfqpoint{2.921513in}{4.763913in}}%
\pgfpathlineto{\pgfqpoint{2.924734in}{4.785686in}}%
\pgfpathlineto{\pgfqpoint{2.925808in}{4.775022in}}%
\pgfpathlineto{\pgfqpoint{2.926882in}{4.779909in}}%
\pgfpathlineto{\pgfqpoint{2.927956in}{4.797683in}}%
\pgfpathlineto{\pgfqpoint{2.929030in}{4.802571in}}%
\pgfpathlineto{\pgfqpoint{2.932251in}{4.812346in}}%
\pgfpathlineto{\pgfqpoint{2.933325in}{4.804793in}}%
\pgfpathlineto{\pgfqpoint{2.934399in}{4.782131in}}%
\pgfpathlineto{\pgfqpoint{2.936546in}{4.774133in}}%
\pgfpathlineto{\pgfqpoint{2.939768in}{4.790129in}}%
\pgfpathlineto{\pgfqpoint{2.940842in}{4.799460in}}%
\pgfpathlineto{\pgfqpoint{2.941916in}{4.786574in}}%
\pgfpathlineto{\pgfqpoint{2.942989in}{4.788796in}}%
\pgfpathlineto{\pgfqpoint{2.944063in}{4.782575in}}%
\pgfpathlineto{\pgfqpoint{2.947285in}{4.743473in}}%
\pgfpathlineto{\pgfqpoint{2.948359in}{4.740807in}}%
\pgfpathlineto{\pgfqpoint{2.949433in}{4.728810in}}%
\pgfpathlineto{\pgfqpoint{2.950506in}{4.727921in}}%
\pgfpathlineto{\pgfqpoint{2.951580in}{4.725255in}}%
\pgfpathlineto{\pgfqpoint{2.954802in}{4.740807in}}%
\pgfpathlineto{\pgfqpoint{2.955876in}{4.733698in}}%
\pgfpathlineto{\pgfqpoint{2.956949in}{4.731032in}}%
\pgfpathlineto{\pgfqpoint{2.959097in}{4.759914in}}%
\pgfpathlineto{\pgfqpoint{2.964466in}{4.664825in}}%
\pgfpathlineto{\pgfqpoint{2.965540in}{4.657271in}}%
\pgfpathlineto{\pgfqpoint{2.966614in}{4.635498in}}%
\pgfpathlineto{\pgfqpoint{2.969835in}{4.619946in}}%
\pgfpathlineto{\pgfqpoint{2.971983in}{4.603950in}}%
\pgfpathlineto{\pgfqpoint{2.973057in}{4.600840in}}%
\pgfpathlineto{\pgfqpoint{2.974131in}{4.608838in}}%
\pgfpathlineto{\pgfqpoint{2.977352in}{4.608393in}}%
\pgfpathlineto{\pgfqpoint{2.978426in}{4.611948in}}%
\pgfpathlineto{\pgfqpoint{2.980574in}{4.603061in}}%
\pgfpathlineto{\pgfqpoint{2.981648in}{4.623501in}}%
\pgfpathlineto{\pgfqpoint{2.984869in}{4.563071in}}%
\pgfpathlineto{\pgfqpoint{2.985943in}{4.517303in}}%
\pgfpathlineto{\pgfqpoint{2.987017in}{4.531967in}}%
\pgfpathlineto{\pgfqpoint{2.988091in}{4.531522in}}%
\pgfpathlineto{\pgfqpoint{2.989165in}{4.530189in}}%
\pgfpathlineto{\pgfqpoint{2.992386in}{4.520858in}}%
\pgfpathlineto{\pgfqpoint{2.993460in}{4.514637in}}%
\pgfpathlineto{\pgfqpoint{2.994534in}{4.527523in}}%
\pgfpathlineto{\pgfqpoint{2.996681in}{4.529301in}}%
\pgfpathlineto{\pgfqpoint{2.999903in}{4.526190in}}%
\pgfpathlineto{\pgfqpoint{3.000977in}{4.538187in}}%
\pgfpathlineto{\pgfqpoint{3.002051in}{4.541298in}}%
\pgfpathlineto{\pgfqpoint{3.003124in}{4.533300in}}%
\pgfpathlineto{\pgfqpoint{3.004198in}{4.516415in}}%
\pgfpathlineto{\pgfqpoint{3.007420in}{4.519525in}}%
\pgfpathlineto{\pgfqpoint{3.008494in}{4.511527in}}%
\pgfpathlineto{\pgfqpoint{3.010641in}{4.514637in}}%
\pgfpathlineto{\pgfqpoint{3.011715in}{4.514637in}}%
\pgfpathlineto{\pgfqpoint{3.014937in}{4.512416in}}%
\pgfpathlineto{\pgfqpoint{3.016011in}{4.523080in}}%
\pgfpathlineto{\pgfqpoint{3.018158in}{4.511971in}}%
\pgfpathlineto{\pgfqpoint{3.019232in}{4.519081in}}%
\pgfpathlineto{\pgfqpoint{3.022454in}{4.516859in}}%
\pgfpathlineto{\pgfqpoint{3.024601in}{4.503973in}}%
\pgfpathlineto{\pgfqpoint{3.026749in}{4.506195in}}%
\pgfpathlineto{\pgfqpoint{3.031044in}{4.507972in}}%
\pgfpathlineto{\pgfqpoint{3.033192in}{4.505306in}}%
\pgfpathlineto{\pgfqpoint{3.034266in}{4.508861in}}%
\pgfpathlineto{\pgfqpoint{3.034266in}{4.508861in}}%
\pgfusepath{stroke}%
\end{pgfscope}%
\begin{pgfscope}%
\pgfpathrectangle{\pgfqpoint{0.568357in}{4.362961in}}{\pgfqpoint{2.583333in}{0.736585in}}%
\pgfusepath{clip}%
\pgfsetroundcap%
\pgfsetroundjoin%
\pgfsetlinewidth{1.505625pt}%
\definecolor{currentstroke}{rgb}{1.000000,0.647059,0.000000}%
\pgfsetstrokecolor{currentstroke}%
\pgfsetdash{}{0pt}%
\pgfpathmoveto{\pgfqpoint{0.685781in}{4.398220in}}%
\pgfpathlineto{\pgfqpoint{0.686855in}{4.401552in}}%
\pgfpathlineto{\pgfqpoint{0.696519in}{4.409773in}}%
\pgfpathlineto{\pgfqpoint{0.700815in}{4.409906in}}%
\pgfpathlineto{\pgfqpoint{0.704036in}{4.412951in}}%
\pgfpathlineto{\pgfqpoint{0.709405in}{4.414022in}}%
\pgfpathlineto{\pgfqpoint{0.711553in}{4.414809in}}%
\pgfpathlineto{\pgfqpoint{0.716922in}{4.414512in}}%
\pgfpathlineto{\pgfqpoint{0.719070in}{4.414622in}}%
\pgfpathlineto{\pgfqpoint{0.723365in}{4.415336in}}%
\pgfpathlineto{\pgfqpoint{0.725513in}{4.416109in}}%
\pgfpathlineto{\pgfqpoint{0.733030in}{4.416299in}}%
\pgfpathlineto{\pgfqpoint{0.734104in}{4.416680in}}%
\pgfpathlineto{\pgfqpoint{0.739473in}{4.417606in}}%
\pgfpathlineto{\pgfqpoint{0.741621in}{4.418515in}}%
\pgfpathlineto{\pgfqpoint{0.760950in}{4.420310in}}%
\pgfpathlineto{\pgfqpoint{0.764171in}{4.422641in}}%
\pgfpathlineto{\pgfqpoint{0.768466in}{4.424164in}}%
\pgfpathlineto{\pgfqpoint{0.771688in}{4.425816in}}%
\pgfpathlineto{\pgfqpoint{0.775983in}{4.426989in}}%
\pgfpathlineto{\pgfqpoint{0.779205in}{4.428550in}}%
\pgfpathlineto{\pgfqpoint{0.791017in}{4.429820in}}%
\pgfpathlineto{\pgfqpoint{0.800682in}{4.429697in}}%
\pgfpathlineto{\pgfqpoint{0.823232in}{4.431387in}}%
\pgfpathlineto{\pgfqpoint{0.858669in}{4.429832in}}%
\pgfpathlineto{\pgfqpoint{0.866186in}{4.430790in}}%
\pgfpathlineto{\pgfqpoint{0.869407in}{4.431475in}}%
\pgfpathlineto{\pgfqpoint{0.874776in}{4.432178in}}%
\pgfpathlineto{\pgfqpoint{0.881220in}{4.433785in}}%
\pgfpathlineto{\pgfqpoint{0.891958in}{4.435213in}}%
\pgfpathlineto{\pgfqpoint{0.899475in}{4.436050in}}%
\pgfpathlineto{\pgfqpoint{0.905918in}{4.437052in}}%
\pgfpathlineto{\pgfqpoint{0.906992in}{4.437473in}}%
\pgfpathlineto{\pgfqpoint{0.912361in}{4.438594in}}%
\pgfpathlineto{\pgfqpoint{0.929542in}{4.443204in}}%
\pgfpathlineto{\pgfqpoint{0.935985in}{4.444448in}}%
\pgfpathlineto{\pgfqpoint{0.937059in}{4.444748in}}%
\pgfpathlineto{\pgfqpoint{0.943502in}{4.445902in}}%
\pgfpathlineto{\pgfqpoint{0.944576in}{4.446167in}}%
\pgfpathlineto{\pgfqpoint{0.951019in}{4.447012in}}%
\pgfpathlineto{\pgfqpoint{0.952093in}{4.447439in}}%
\pgfpathlineto{\pgfqpoint{0.956388in}{4.448255in}}%
\pgfpathlineto{\pgfqpoint{0.959610in}{4.449716in}}%
\pgfpathlineto{\pgfqpoint{0.963905in}{4.450714in}}%
\pgfpathlineto{\pgfqpoint{0.967127in}{4.452412in}}%
\pgfpathlineto{\pgfqpoint{0.971422in}{4.453489in}}%
\pgfpathlineto{\pgfqpoint{0.974643in}{4.455157in}}%
\pgfpathlineto{\pgfqpoint{0.978939in}{4.456342in}}%
\pgfpathlineto{\pgfqpoint{0.982160in}{4.458176in}}%
\pgfpathlineto{\pgfqpoint{0.986456in}{4.459291in}}%
\pgfpathlineto{\pgfqpoint{0.989677in}{4.460764in}}%
\pgfpathlineto{\pgfqpoint{0.993973in}{4.461778in}}%
\pgfpathlineto{\pgfqpoint{0.997194in}{4.463234in}}%
\pgfpathlineto{\pgfqpoint{1.015449in}{4.465494in}}%
\pgfpathlineto{\pgfqpoint{1.019745in}{4.466342in}}%
\pgfpathlineto{\pgfqpoint{1.041221in}{4.467995in}}%
\pgfpathlineto{\pgfqpoint{1.049812in}{4.469093in}}%
\pgfpathlineto{\pgfqpoint{1.056255in}{4.470043in}}%
\pgfpathlineto{\pgfqpoint{1.062698in}{4.470955in}}%
\pgfpathlineto{\pgfqpoint{1.068067in}{4.471426in}}%
\pgfpathlineto{\pgfqpoint{1.085249in}{4.472972in}}%
\pgfpathlineto{\pgfqpoint{1.094913in}{4.474304in}}%
\pgfpathlineto{\pgfqpoint{1.101356in}{4.475126in}}%
\pgfpathlineto{\pgfqpoint{1.102430in}{4.475434in}}%
\pgfpathlineto{\pgfqpoint{1.107799in}{4.476388in}}%
\pgfpathlineto{\pgfqpoint{1.117464in}{4.478577in}}%
\pgfpathlineto{\pgfqpoint{1.122833in}{4.479614in}}%
\pgfpathlineto{\pgfqpoint{1.124981in}{4.480425in}}%
\pgfpathlineto{\pgfqpoint{1.130350in}{4.481280in}}%
\pgfpathlineto{\pgfqpoint{1.132498in}{4.482109in}}%
\pgfpathlineto{\pgfqpoint{1.137867in}{4.483240in}}%
\pgfpathlineto{\pgfqpoint{1.140015in}{4.484011in}}%
\pgfpathlineto{\pgfqpoint{1.144310in}{4.484825in}}%
\pgfpathlineto{\pgfqpoint{1.147531in}{4.486126in}}%
\pgfpathlineto{\pgfqpoint{1.152901in}{4.487330in}}%
\pgfpathlineto{\pgfqpoint{1.155048in}{4.488133in}}%
\pgfpathlineto{\pgfqpoint{1.160418in}{4.489241in}}%
\pgfpathlineto{\pgfqpoint{1.162565in}{4.489965in}}%
\pgfpathlineto{\pgfqpoint{1.167934in}{4.490972in}}%
\pgfpathlineto{\pgfqpoint{1.169008in}{4.491299in}}%
\pgfpathlineto{\pgfqpoint{1.175451in}{4.492277in}}%
\pgfpathlineto{\pgfqpoint{1.177599in}{4.492889in}}%
\pgfpathlineto{\pgfqpoint{1.182968in}{4.493873in}}%
\pgfpathlineto{\pgfqpoint{1.185116in}{4.494578in}}%
\pgfpathlineto{\pgfqpoint{1.191559in}{4.495665in}}%
\pgfpathlineto{\pgfqpoint{1.207666in}{4.497597in}}%
\pgfpathlineto{\pgfqpoint{1.214109in}{4.498572in}}%
\pgfpathlineto{\pgfqpoint{1.222700in}{4.500207in}}%
\pgfpathlineto{\pgfqpoint{1.228069in}{4.501178in}}%
\pgfpathlineto{\pgfqpoint{1.230217in}{4.501796in}}%
\pgfpathlineto{\pgfqpoint{1.236660in}{4.502713in}}%
\pgfpathlineto{\pgfqpoint{1.245251in}{4.504455in}}%
\pgfpathlineto{\pgfqpoint{1.251694in}{4.505623in}}%
\pgfpathlineto{\pgfqpoint{1.260285in}{4.507404in}}%
\pgfpathlineto{\pgfqpoint{1.266728in}{4.508399in}}%
\pgfpathlineto{\pgfqpoint{1.273171in}{4.509348in}}%
\pgfpathlineto{\pgfqpoint{1.286057in}{4.511277in}}%
\pgfpathlineto{\pgfqpoint{1.312903in}{4.517510in}}%
\pgfpathlineto{\pgfqpoint{1.319346in}{4.518663in}}%
\pgfpathlineto{\pgfqpoint{1.324715in}{4.519436in}}%
\pgfpathlineto{\pgfqpoint{1.333306in}{4.520791in}}%
\pgfpathlineto{\pgfqpoint{1.342970in}{4.521910in}}%
\pgfpathlineto{\pgfqpoint{1.349413in}{4.522859in}}%
\pgfpathlineto{\pgfqpoint{1.358004in}{4.524516in}}%
\pgfpathlineto{\pgfqpoint{1.364447in}{4.525632in}}%
\pgfpathlineto{\pgfqpoint{1.373038in}{4.527184in}}%
\pgfpathlineto{\pgfqpoint{1.379481in}{4.528120in}}%
\pgfpathlineto{\pgfqpoint{1.384850in}{4.528921in}}%
\pgfpathlineto{\pgfqpoint{1.415991in}{4.537217in}}%
\pgfpathlineto{\pgfqpoint{1.418139in}{4.538104in}}%
\pgfpathlineto{\pgfqpoint{1.423508in}{4.539413in}}%
\pgfpathlineto{\pgfqpoint{1.425656in}{4.540266in}}%
\pgfpathlineto{\pgfqpoint{1.431025in}{4.541483in}}%
\pgfpathlineto{\pgfqpoint{1.443911in}{4.544241in}}%
\pgfpathlineto{\pgfqpoint{1.448206in}{4.545795in}}%
\pgfpathlineto{\pgfqpoint{1.452502in}{4.546595in}}%
\pgfpathlineto{\pgfqpoint{1.455723in}{4.547891in}}%
\pgfpathlineto{\pgfqpoint{1.462166in}{4.549242in}}%
\pgfpathlineto{\pgfqpoint{1.463240in}{4.549704in}}%
\pgfpathlineto{\pgfqpoint{1.469683in}{4.551074in}}%
\pgfpathlineto{\pgfqpoint{1.470757in}{4.551504in}}%
\pgfpathlineto{\pgfqpoint{1.476126in}{4.552736in}}%
\pgfpathlineto{\pgfqpoint{1.478274in}{4.553526in}}%
\pgfpathlineto{\pgfqpoint{1.483643in}{4.554684in}}%
\pgfpathlineto{\pgfqpoint{1.485791in}{4.555431in}}%
\pgfpathlineto{\pgfqpoint{1.492234in}{4.556363in}}%
\pgfpathlineto{\pgfqpoint{1.500824in}{4.557868in}}%
\pgfpathlineto{\pgfqpoint{1.507267in}{4.558679in}}%
\pgfpathlineto{\pgfqpoint{1.515858in}{4.560187in}}%
\pgfpathlineto{\pgfqpoint{1.523375in}{4.561177in}}%
\pgfpathlineto{\pgfqpoint{1.529818in}{4.562197in}}%
\pgfpathlineto{\pgfqpoint{1.538409in}{4.563864in}}%
\pgfpathlineto{\pgfqpoint{1.544852in}{4.564957in}}%
\pgfpathlineto{\pgfqpoint{1.553442in}{4.566411in}}%
\pgfpathlineto{\pgfqpoint{1.559885in}{4.567434in}}%
\pgfpathlineto{\pgfqpoint{1.568476in}{4.569076in}}%
\pgfpathlineto{\pgfqpoint{1.574919in}{4.570100in}}%
\pgfpathlineto{\pgfqpoint{1.582436in}{4.571404in}}%
\pgfpathlineto{\pgfqpoint{1.588879in}{4.572289in}}%
\pgfpathlineto{\pgfqpoint{1.598544in}{4.574393in}}%
\pgfpathlineto{\pgfqpoint{1.604987in}{4.575507in}}%
\pgfpathlineto{\pgfqpoint{1.613577in}{4.577248in}}%
\pgfpathlineto{\pgfqpoint{1.620020in}{4.578334in}}%
\pgfpathlineto{\pgfqpoint{1.621094in}{4.578605in}}%
\pgfpathlineto{\pgfqpoint{1.627537in}{4.579440in}}%
\pgfpathlineto{\pgfqpoint{1.632906in}{4.580292in}}%
\pgfpathlineto{\pgfqpoint{1.643645in}{4.582647in}}%
\pgfpathlineto{\pgfqpoint{1.649014in}{4.583479in}}%
\pgfpathlineto{\pgfqpoint{1.651162in}{4.584066in}}%
\pgfpathlineto{\pgfqpoint{1.657605in}{4.585121in}}%
\pgfpathlineto{\pgfqpoint{1.665122in}{4.586428in}}%
\pgfpathlineto{\pgfqpoint{1.672639in}{4.587435in}}%
\pgfpathlineto{\pgfqpoint{1.681229in}{4.589002in}}%
\pgfpathlineto{\pgfqpoint{1.687672in}{4.589876in}}%
\pgfpathlineto{\pgfqpoint{1.696263in}{4.591008in}}%
\pgfpathlineto{\pgfqpoint{1.707001in}{4.592104in}}%
\pgfpathlineto{\pgfqpoint{1.726330in}{4.595025in}}%
\pgfpathlineto{\pgfqpoint{1.737069in}{4.596028in}}%
\pgfpathlineto{\pgfqpoint{1.748881in}{4.597861in}}%
\pgfpathlineto{\pgfqpoint{1.759619in}{4.599015in}}%
\pgfpathlineto{\pgfqpoint{1.769284in}{4.600135in}}%
\pgfpathlineto{\pgfqpoint{1.778949in}{4.600913in}}%
\pgfpathlineto{\pgfqpoint{1.789687in}{4.601887in}}%
\pgfpathlineto{\pgfqpoint{1.801499in}{4.603552in}}%
\pgfpathlineto{\pgfqpoint{1.809016in}{4.604601in}}%
\pgfpathlineto{\pgfqpoint{1.815459in}{4.605510in}}%
\pgfpathlineto{\pgfqpoint{1.821902in}{4.606425in}}%
\pgfpathlineto{\pgfqpoint{1.874520in}{4.610927in}}%
\pgfpathlineto{\pgfqpoint{1.920695in}{4.614114in}}%
\pgfpathlineto{\pgfqpoint{1.929286in}{4.615132in}}%
\pgfpathlineto{\pgfqpoint{1.942172in}{4.616296in}}%
\pgfpathlineto{\pgfqpoint{1.951837in}{4.617231in}}%
\pgfpathlineto{\pgfqpoint{1.965796in}{4.618253in}}%
\pgfpathlineto{\pgfqpoint{1.966870in}{4.618535in}}%
\pgfpathlineto{\pgfqpoint{1.973313in}{4.619479in}}%
\pgfpathlineto{\pgfqpoint{1.981904in}{4.620713in}}%
\pgfpathlineto{\pgfqpoint{1.988347in}{4.621549in}}%
\pgfpathlineto{\pgfqpoint{1.996938in}{4.622810in}}%
\pgfpathlineto{\pgfqpoint{2.003381in}{4.623644in}}%
\pgfpathlineto{\pgfqpoint{2.011971in}{4.624973in}}%
\pgfpathlineto{\pgfqpoint{2.022710in}{4.626055in}}%
\pgfpathlineto{\pgfqpoint{2.034522in}{4.627950in}}%
\pgfpathlineto{\pgfqpoint{2.040965in}{4.628769in}}%
\pgfpathlineto{\pgfqpoint{2.049556in}{4.630027in}}%
\pgfpathlineto{\pgfqpoint{2.062442in}{4.631237in}}%
\pgfpathlineto{\pgfqpoint{2.079623in}{4.633326in}}%
\pgfpathlineto{\pgfqpoint{2.091436in}{4.634360in}}%
\pgfpathlineto{\pgfqpoint{2.102174in}{4.635490in}}%
\pgfpathlineto{\pgfqpoint{2.145127in}{4.638037in}}%
\pgfpathlineto{\pgfqpoint{2.173047in}{4.641154in}}%
\pgfpathlineto{\pgfqpoint{2.184859in}{4.643549in}}%
\pgfpathlineto{\pgfqpoint{2.191303in}{4.644643in}}%
\pgfpathlineto{\pgfqpoint{2.192376in}{4.644930in}}%
\pgfpathlineto{\pgfqpoint{2.197746in}{4.645817in}}%
\pgfpathlineto{\pgfqpoint{2.199893in}{4.646409in}}%
\pgfpathlineto{\pgfqpoint{2.205262in}{4.647313in}}%
\pgfpathlineto{\pgfqpoint{2.212779in}{4.648828in}}%
\pgfpathlineto{\pgfqpoint{2.227813in}{4.651418in}}%
\pgfpathlineto{\pgfqpoint{2.229961in}{4.652005in}}%
\pgfpathlineto{\pgfqpoint{2.235330in}{4.652888in}}%
\pgfpathlineto{\pgfqpoint{2.237478in}{4.653483in}}%
\pgfpathlineto{\pgfqpoint{2.242847in}{4.654395in}}%
\pgfpathlineto{\pgfqpoint{2.243921in}{4.654706in}}%
\pgfpathlineto{\pgfqpoint{2.250364in}{4.655663in}}%
\pgfpathlineto{\pgfqpoint{2.251437in}{4.655984in}}%
\pgfpathlineto{\pgfqpoint{2.257881in}{4.656876in}}%
\pgfpathlineto{\pgfqpoint{2.267545in}{4.658425in}}%
\pgfpathlineto{\pgfqpoint{2.278283in}{4.659451in}}%
\pgfpathlineto{\pgfqpoint{2.290096in}{4.661364in}}%
\pgfpathlineto{\pgfqpoint{2.301908in}{4.662549in}}%
\pgfpathlineto{\pgfqpoint{2.312646in}{4.664415in}}%
\pgfpathlineto{\pgfqpoint{2.318015in}{4.665184in}}%
\pgfpathlineto{\pgfqpoint{2.327680in}{4.667081in}}%
\pgfpathlineto{\pgfqpoint{2.333049in}{4.667892in}}%
\pgfpathlineto{\pgfqpoint{2.335197in}{4.668480in}}%
\pgfpathlineto{\pgfqpoint{2.340566in}{4.669374in}}%
\pgfpathlineto{\pgfqpoint{2.341640in}{4.669672in}}%
\pgfpathlineto{\pgfqpoint{2.348083in}{4.670615in}}%
\pgfpathlineto{\pgfqpoint{2.350231in}{4.671260in}}%
\pgfpathlineto{\pgfqpoint{2.356674in}{4.672411in}}%
\pgfpathlineto{\pgfqpoint{2.365264in}{4.674109in}}%
\pgfpathlineto{\pgfqpoint{2.370634in}{4.674976in}}%
\pgfpathlineto{\pgfqpoint{2.372781in}{4.675534in}}%
\pgfpathlineto{\pgfqpoint{2.378150in}{4.676363in}}%
\pgfpathlineto{\pgfqpoint{2.384593in}{4.677456in}}%
\pgfpathlineto{\pgfqpoint{2.395332in}{4.679410in}}%
\pgfpathlineto{\pgfqpoint{2.401775in}{4.680320in}}%
\pgfpathlineto{\pgfqpoint{2.410366in}{4.681715in}}%
\pgfpathlineto{\pgfqpoint{2.417882in}{4.682667in}}%
\pgfpathlineto{\pgfqpoint{2.424325in}{4.683630in}}%
\pgfpathlineto{\pgfqpoint{2.432916in}{4.685108in}}%
\pgfpathlineto{\pgfqpoint{2.439359in}{4.686186in}}%
\pgfpathlineto{\pgfqpoint{2.447950in}{4.687685in}}%
\pgfpathlineto{\pgfqpoint{2.454393in}{4.688555in}}%
\pgfpathlineto{\pgfqpoint{2.455467in}{4.688863in}}%
\pgfpathlineto{\pgfqpoint{2.460836in}{4.689789in}}%
\pgfpathlineto{\pgfqpoint{2.462984in}{4.690437in}}%
\pgfpathlineto{\pgfqpoint{2.468353in}{4.691416in}}%
\pgfpathlineto{\pgfqpoint{2.470501in}{4.692029in}}%
\pgfpathlineto{\pgfqpoint{2.476944in}{4.693124in}}%
\pgfpathlineto{\pgfqpoint{2.485534in}{4.694701in}}%
\pgfpathlineto{\pgfqpoint{2.491977in}{4.695761in}}%
\pgfpathlineto{\pgfqpoint{2.500568in}{4.697335in}}%
\pgfpathlineto{\pgfqpoint{2.507011in}{4.698371in}}%
\pgfpathlineto{\pgfqpoint{2.515602in}{4.699920in}}%
\pgfpathlineto{\pgfqpoint{2.523119in}{4.700882in}}%
\pgfpathlineto{\pgfqpoint{2.533857in}{4.702139in}}%
\pgfpathlineto{\pgfqpoint{2.545669in}{4.704022in}}%
\pgfpathlineto{\pgfqpoint{2.556408in}{4.705171in}}%
\pgfpathlineto{\pgfqpoint{2.568220in}{4.706766in}}%
\pgfpathlineto{\pgfqpoint{2.578958in}{4.707808in}}%
\pgfpathlineto{\pgfqpoint{2.598287in}{4.710597in}}%
\pgfpathlineto{\pgfqpoint{2.609026in}{4.711805in}}%
\pgfpathlineto{\pgfqpoint{2.620838in}{4.713979in}}%
\pgfpathlineto{\pgfqpoint{2.627281in}{4.714970in}}%
\pgfpathlineto{\pgfqpoint{2.635872in}{4.716530in}}%
\pgfpathlineto{\pgfqpoint{2.642315in}{4.717302in}}%
\pgfpathlineto{\pgfqpoint{2.643389in}{4.717554in}}%
\pgfpathlineto{\pgfqpoint{2.649832in}{4.718308in}}%
\pgfpathlineto{\pgfqpoint{2.658422in}{4.719770in}}%
\pgfpathlineto{\pgfqpoint{2.669161in}{4.720878in}}%
\pgfpathlineto{\pgfqpoint{2.680973in}{4.722610in}}%
\pgfpathlineto{\pgfqpoint{2.691711in}{4.723702in}}%
\pgfpathlineto{\pgfqpoint{2.696007in}{4.724510in}}%
\pgfpathlineto{\pgfqpoint{2.706745in}{4.725510in}}%
\pgfpathlineto{\pgfqpoint{2.718557in}{4.727258in}}%
\pgfpathlineto{\pgfqpoint{2.729296in}{4.728368in}}%
\pgfpathlineto{\pgfqpoint{2.741108in}{4.729980in}}%
\pgfpathlineto{\pgfqpoint{2.751846in}{4.731103in}}%
\pgfpathlineto{\pgfqpoint{2.755068in}{4.731645in}}%
\pgfpathlineto{\pgfqpoint{2.762585in}{4.732381in}}%
\pgfpathlineto{\pgfqpoint{2.771175in}{4.733367in}}%
\pgfpathlineto{\pgfqpoint{2.782988in}{4.734457in}}%
\pgfpathlineto{\pgfqpoint{2.793726in}{4.735464in}}%
\pgfpathlineto{\pgfqpoint{2.813055in}{4.736715in}}%
\pgfpathlineto{\pgfqpoint{2.838827in}{4.738968in}}%
\pgfpathlineto{\pgfqpoint{2.944063in}{4.742163in}}%
\pgfpathlineto{\pgfqpoint{2.970909in}{4.741707in}}%
\pgfpathlineto{\pgfqpoint{3.034266in}{4.736145in}}%
\pgfpathlineto{\pgfqpoint{3.034266in}{4.736145in}}%
\pgfusepath{stroke}%
\end{pgfscope}%
\begin{pgfscope}%
\pgfsetrectcap%
\pgfsetmiterjoin%
\pgfsetlinewidth{0.803000pt}%
\definecolor{currentstroke}{rgb}{1.000000,1.000000,1.000000}%
\pgfsetstrokecolor{currentstroke}%
\pgfsetdash{}{0pt}%
\pgfpathmoveto{\pgfqpoint{0.568357in}{4.362961in}}%
\pgfpathlineto{\pgfqpoint{0.568357in}{5.099547in}}%
\pgfusepath{stroke}%
\end{pgfscope}%
\begin{pgfscope}%
\pgfsetrectcap%
\pgfsetmiterjoin%
\pgfsetlinewidth{0.803000pt}%
\definecolor{currentstroke}{rgb}{1.000000,1.000000,1.000000}%
\pgfsetstrokecolor{currentstroke}%
\pgfsetdash{}{0pt}%
\pgfpathmoveto{\pgfqpoint{3.151690in}{4.362961in}}%
\pgfpathlineto{\pgfqpoint{3.151690in}{5.099547in}}%
\pgfusepath{stroke}%
\end{pgfscope}%
\begin{pgfscope}%
\pgfsetrectcap%
\pgfsetmiterjoin%
\pgfsetlinewidth{0.803000pt}%
\definecolor{currentstroke}{rgb}{1.000000,1.000000,1.000000}%
\pgfsetstrokecolor{currentstroke}%
\pgfsetdash{}{0pt}%
\pgfpathmoveto{\pgfqpoint{0.568357in}{4.362961in}}%
\pgfpathlineto{\pgfqpoint{3.151690in}{4.362961in}}%
\pgfusepath{stroke}%
\end{pgfscope}%
\begin{pgfscope}%
\pgfsetrectcap%
\pgfsetmiterjoin%
\pgfsetlinewidth{0.803000pt}%
\definecolor{currentstroke}{rgb}{1.000000,1.000000,1.000000}%
\pgfsetstrokecolor{currentstroke}%
\pgfsetdash{}{0pt}%
\pgfpathmoveto{\pgfqpoint{0.568357in}{5.099547in}}%
\pgfpathlineto{\pgfqpoint{3.151690in}{5.099547in}}%
\pgfusepath{stroke}%
\end{pgfscope}%
\begin{pgfscope}%
\definecolor{textcolor}{rgb}{0.150000,0.150000,0.150000}%
\pgfsetstrokecolor{textcolor}%
\pgfsetfillcolor{textcolor}%
\pgftext[x=1.860023in,y=5.182880in,,base]{\color{textcolor}\rmfamily\fontsize{16.800000}{20.160000}\selectfont GE}%
\end{pgfscope}%
\begin{pgfscope}%
\pgfsetbuttcap%
\pgfsetmiterjoin%
\definecolor{currentfill}{rgb}{0.917647,0.917647,0.949020}%
\pgfsetfillcolor{currentfill}%
\pgfsetlinewidth{0.000000pt}%
\definecolor{currentstroke}{rgb}{0.000000,0.000000,0.000000}%
\pgfsetstrokecolor{currentstroke}%
\pgfsetstrokeopacity{0.000000}%
\pgfsetdash{}{0pt}%
\pgfpathmoveto{\pgfqpoint{4.185023in}{4.362961in}}%
\pgfpathlineto{\pgfqpoint{6.768357in}{4.362961in}}%
\pgfpathlineto{\pgfqpoint{6.768357in}{5.099547in}}%
\pgfpathlineto{\pgfqpoint{4.185023in}{5.099547in}}%
\pgfpathclose%
\pgfusepath{fill}%
\end{pgfscope}%
\begin{pgfscope}%
\pgfpathrectangle{\pgfqpoint{4.185023in}{4.362961in}}{\pgfqpoint{2.583333in}{0.736585in}}%
\pgfusepath{clip}%
\pgfsetroundcap%
\pgfsetroundjoin%
\pgfsetlinewidth{0.803000pt}%
\definecolor{currentstroke}{rgb}{1.000000,1.000000,1.000000}%
\pgfsetstrokecolor{currentstroke}%
\pgfsetdash{}{0pt}%
\pgfpathmoveto{\pgfqpoint{4.300300in}{4.362961in}}%
\pgfpathlineto{\pgfqpoint{4.300300in}{5.099547in}}%
\pgfusepath{stroke}%
\end{pgfscope}%
\begin{pgfscope}%
\definecolor{textcolor}{rgb}{0.150000,0.150000,0.150000}%
\pgfsetstrokecolor{textcolor}%
\pgfsetfillcolor{textcolor}%
\pgftext[x=4.300300in,y=4.265739in,,top]{\color{textcolor}\rmfamily\fontsize{14.000000}{16.800000}\selectfont 2012}%
\end{pgfscope}%
\begin{pgfscope}%
\pgfpathrectangle{\pgfqpoint{4.185023in}{4.362961in}}{\pgfqpoint{2.583333in}{0.736585in}}%
\pgfusepath{clip}%
\pgfsetroundcap%
\pgfsetroundjoin%
\pgfsetlinewidth{0.803000pt}%
\definecolor{currentstroke}{rgb}{1.000000,1.000000,1.000000}%
\pgfsetstrokecolor{currentstroke}%
\pgfsetdash{}{0pt}%
\pgfpathmoveto{\pgfqpoint{4.693325in}{4.362961in}}%
\pgfpathlineto{\pgfqpoint{4.693325in}{5.099547in}}%
\pgfusepath{stroke}%
\end{pgfscope}%
\begin{pgfscope}%
\definecolor{textcolor}{rgb}{0.150000,0.150000,0.150000}%
\pgfsetstrokecolor{textcolor}%
\pgfsetfillcolor{textcolor}%
\pgftext[x=4.693325in,y=4.265739in,,top]{\color{textcolor}\rmfamily\fontsize{14.000000}{16.800000}\selectfont 2013}%
\end{pgfscope}%
\begin{pgfscope}%
\pgfpathrectangle{\pgfqpoint{4.185023in}{4.362961in}}{\pgfqpoint{2.583333in}{0.736585in}}%
\pgfusepath{clip}%
\pgfsetroundcap%
\pgfsetroundjoin%
\pgfsetlinewidth{0.803000pt}%
\definecolor{currentstroke}{rgb}{1.000000,1.000000,1.000000}%
\pgfsetstrokecolor{currentstroke}%
\pgfsetdash{}{0pt}%
\pgfpathmoveto{\pgfqpoint{5.085276in}{4.362961in}}%
\pgfpathlineto{\pgfqpoint{5.085276in}{5.099547in}}%
\pgfusepath{stroke}%
\end{pgfscope}%
\begin{pgfscope}%
\definecolor{textcolor}{rgb}{0.150000,0.150000,0.150000}%
\pgfsetstrokecolor{textcolor}%
\pgfsetfillcolor{textcolor}%
\pgftext[x=5.085276in,y=4.265739in,,top]{\color{textcolor}\rmfamily\fontsize{14.000000}{16.800000}\selectfont 2014}%
\end{pgfscope}%
\begin{pgfscope}%
\pgfpathrectangle{\pgfqpoint{4.185023in}{4.362961in}}{\pgfqpoint{2.583333in}{0.736585in}}%
\pgfusepath{clip}%
\pgfsetroundcap%
\pgfsetroundjoin%
\pgfsetlinewidth{0.803000pt}%
\definecolor{currentstroke}{rgb}{1.000000,1.000000,1.000000}%
\pgfsetstrokecolor{currentstroke}%
\pgfsetdash{}{0pt}%
\pgfpathmoveto{\pgfqpoint{5.477227in}{4.362961in}}%
\pgfpathlineto{\pgfqpoint{5.477227in}{5.099547in}}%
\pgfusepath{stroke}%
\end{pgfscope}%
\begin{pgfscope}%
\definecolor{textcolor}{rgb}{0.150000,0.150000,0.150000}%
\pgfsetstrokecolor{textcolor}%
\pgfsetfillcolor{textcolor}%
\pgftext[x=5.477227in,y=4.265739in,,top]{\color{textcolor}\rmfamily\fontsize{14.000000}{16.800000}\selectfont 2015}%
\end{pgfscope}%
\begin{pgfscope}%
\pgfpathrectangle{\pgfqpoint{4.185023in}{4.362961in}}{\pgfqpoint{2.583333in}{0.736585in}}%
\pgfusepath{clip}%
\pgfsetroundcap%
\pgfsetroundjoin%
\pgfsetlinewidth{0.803000pt}%
\definecolor{currentstroke}{rgb}{1.000000,1.000000,1.000000}%
\pgfsetstrokecolor{currentstroke}%
\pgfsetdash{}{0pt}%
\pgfpathmoveto{\pgfqpoint{5.869178in}{4.362961in}}%
\pgfpathlineto{\pgfqpoint{5.869178in}{5.099547in}}%
\pgfusepath{stroke}%
\end{pgfscope}%
\begin{pgfscope}%
\definecolor{textcolor}{rgb}{0.150000,0.150000,0.150000}%
\pgfsetstrokecolor{textcolor}%
\pgfsetfillcolor{textcolor}%
\pgftext[x=5.869178in,y=4.265739in,,top]{\color{textcolor}\rmfamily\fontsize{14.000000}{16.800000}\selectfont 2016}%
\end{pgfscope}%
\begin{pgfscope}%
\pgfpathrectangle{\pgfqpoint{4.185023in}{4.362961in}}{\pgfqpoint{2.583333in}{0.736585in}}%
\pgfusepath{clip}%
\pgfsetroundcap%
\pgfsetroundjoin%
\pgfsetlinewidth{0.803000pt}%
\definecolor{currentstroke}{rgb}{1.000000,1.000000,1.000000}%
\pgfsetstrokecolor{currentstroke}%
\pgfsetdash{}{0pt}%
\pgfpathmoveto{\pgfqpoint{6.262203in}{4.362961in}}%
\pgfpathlineto{\pgfqpoint{6.262203in}{5.099547in}}%
\pgfusepath{stroke}%
\end{pgfscope}%
\begin{pgfscope}%
\definecolor{textcolor}{rgb}{0.150000,0.150000,0.150000}%
\pgfsetstrokecolor{textcolor}%
\pgfsetfillcolor{textcolor}%
\pgftext[x=6.262203in,y=4.265739in,,top]{\color{textcolor}\rmfamily\fontsize{14.000000}{16.800000}\selectfont 2017}%
\end{pgfscope}%
\begin{pgfscope}%
\pgfpathrectangle{\pgfqpoint{4.185023in}{4.362961in}}{\pgfqpoint{2.583333in}{0.736585in}}%
\pgfusepath{clip}%
\pgfsetroundcap%
\pgfsetroundjoin%
\pgfsetlinewidth{0.803000pt}%
\definecolor{currentstroke}{rgb}{1.000000,1.000000,1.000000}%
\pgfsetstrokecolor{currentstroke}%
\pgfsetdash{}{0pt}%
\pgfpathmoveto{\pgfqpoint{6.654154in}{4.362961in}}%
\pgfpathlineto{\pgfqpoint{6.654154in}{5.099547in}}%
\pgfusepath{stroke}%
\end{pgfscope}%
\begin{pgfscope}%
\definecolor{textcolor}{rgb}{0.150000,0.150000,0.150000}%
\pgfsetstrokecolor{textcolor}%
\pgfsetfillcolor{textcolor}%
\pgftext[x=6.654154in,y=4.265739in,,top]{\color{textcolor}\rmfamily\fontsize{14.000000}{16.800000}\selectfont 2018}%
\end{pgfscope}%
\begin{pgfscope}%
\pgfpathrectangle{\pgfqpoint{4.185023in}{4.362961in}}{\pgfqpoint{2.583333in}{0.736585in}}%
\pgfusepath{clip}%
\pgfsetroundcap%
\pgfsetroundjoin%
\pgfsetlinewidth{0.803000pt}%
\definecolor{currentstroke}{rgb}{1.000000,1.000000,1.000000}%
\pgfsetstrokecolor{currentstroke}%
\pgfsetdash{}{0pt}%
\pgfpathmoveto{\pgfqpoint{4.185023in}{4.491491in}}%
\pgfpathlineto{\pgfqpoint{6.768357in}{4.491491in}}%
\pgfusepath{stroke}%
\end{pgfscope}%
\begin{pgfscope}%
\definecolor{textcolor}{rgb}{0.150000,0.150000,0.150000}%
\pgfsetstrokecolor{textcolor}%
\pgfsetfillcolor{textcolor}%
\pgftext[x=3.840378in,y=4.417625in,left,base]{\color{textcolor}\rmfamily\fontsize{14.000000}{16.800000}\selectfont 20}%
\end{pgfscope}%
\begin{pgfscope}%
\pgfpathrectangle{\pgfqpoint{4.185023in}{4.362961in}}{\pgfqpoint{2.583333in}{0.736585in}}%
\pgfusepath{clip}%
\pgfsetroundcap%
\pgfsetroundjoin%
\pgfsetlinewidth{0.803000pt}%
\definecolor{currentstroke}{rgb}{1.000000,1.000000,1.000000}%
\pgfsetstrokecolor{currentstroke}%
\pgfsetdash{}{0pt}%
\pgfpathmoveto{\pgfqpoint{4.185023in}{4.941961in}}%
\pgfpathlineto{\pgfqpoint{6.768357in}{4.941961in}}%
\pgfusepath{stroke}%
\end{pgfscope}%
\begin{pgfscope}%
\definecolor{textcolor}{rgb}{0.150000,0.150000,0.150000}%
\pgfsetstrokecolor{textcolor}%
\pgfsetfillcolor{textcolor}%
\pgftext[x=3.840378in,y=4.868095in,left,base]{\color{textcolor}\rmfamily\fontsize{14.000000}{16.800000}\selectfont 40}%
\end{pgfscope}%
\begin{pgfscope}%
\pgfpathrectangle{\pgfqpoint{4.185023in}{4.362961in}}{\pgfqpoint{2.583333in}{0.736585in}}%
\pgfusepath{clip}%
\pgfsetroundcap%
\pgfsetroundjoin%
\pgfsetlinewidth{1.505625pt}%
\definecolor{currentstroke}{rgb}{0.839216,0.152941,0.156863}%
\pgfsetstrokecolor{currentstroke}%
\pgfsetdash{}{0pt}%
\pgfpathmoveto{\pgfqpoint{4.302448in}{4.476401in}}%
\pgfpathlineto{\pgfqpoint{4.304595in}{4.491717in}}%
\pgfpathlineto{\pgfqpoint{4.305669in}{4.489014in}}%
\pgfpathlineto{\pgfqpoint{4.309964in}{4.495095in}}%
\pgfpathlineto{\pgfqpoint{4.311038in}{4.498924in}}%
\pgfpathlineto{\pgfqpoint{4.312112in}{4.498023in}}%
\pgfpathlineto{\pgfqpoint{4.313186in}{4.487212in}}%
\pgfpathlineto{\pgfqpoint{4.317481in}{4.485410in}}%
\pgfpathlineto{\pgfqpoint{4.319629in}{4.495771in}}%
\pgfpathlineto{\pgfqpoint{4.320703in}{4.509060in}}%
\pgfpathlineto{\pgfqpoint{4.323924in}{4.514916in}}%
\pgfpathlineto{\pgfqpoint{4.324998in}{4.518294in}}%
\pgfpathlineto{\pgfqpoint{4.326072in}{4.518294in}}%
\pgfpathlineto{\pgfqpoint{4.327146in}{4.515592in}}%
\pgfpathlineto{\pgfqpoint{4.328220in}{4.515366in}}%
\pgfpathlineto{\pgfqpoint{4.331441in}{4.515592in}}%
\pgfpathlineto{\pgfqpoint{4.332515in}{4.509736in}}%
\pgfpathlineto{\pgfqpoint{4.333589in}{4.512213in}}%
\pgfpathlineto{\pgfqpoint{4.334663in}{4.511087in}}%
\pgfpathlineto{\pgfqpoint{4.335737in}{4.519195in}}%
\pgfpathlineto{\pgfqpoint{4.338958in}{4.518970in}}%
\pgfpathlineto{\pgfqpoint{4.340032in}{4.517393in}}%
\pgfpathlineto{\pgfqpoint{4.341106in}{4.521222in}}%
\pgfpathlineto{\pgfqpoint{4.342180in}{4.521448in}}%
\pgfpathlineto{\pgfqpoint{4.343253in}{4.518520in}}%
\pgfpathlineto{\pgfqpoint{4.346475in}{4.518520in}}%
\pgfpathlineto{\pgfqpoint{4.347549in}{4.520096in}}%
\pgfpathlineto{\pgfqpoint{4.348623in}{4.516493in}}%
\pgfpathlineto{\pgfqpoint{4.349696in}{4.520997in}}%
\pgfpathlineto{\pgfqpoint{4.350770in}{4.530457in}}%
\pgfpathlineto{\pgfqpoint{4.355066in}{4.526853in}}%
\pgfpathlineto{\pgfqpoint{4.356140in}{4.519195in}}%
\pgfpathlineto{\pgfqpoint{4.357213in}{4.517844in}}%
\pgfpathlineto{\pgfqpoint{4.358287in}{4.518520in}}%
\pgfpathlineto{\pgfqpoint{4.361509in}{4.521898in}}%
\pgfpathlineto{\pgfqpoint{4.362583in}{4.528205in}}%
\pgfpathlineto{\pgfqpoint{4.363656in}{4.521898in}}%
\pgfpathlineto{\pgfqpoint{4.364730in}{4.521448in}}%
\pgfpathlineto{\pgfqpoint{4.365804in}{4.522574in}}%
\pgfpathlineto{\pgfqpoint{4.369026in}{4.515817in}}%
\pgfpathlineto{\pgfqpoint{4.370099in}{4.516943in}}%
\pgfpathlineto{\pgfqpoint{4.371173in}{4.522349in}}%
\pgfpathlineto{\pgfqpoint{4.372247in}{4.520997in}}%
\pgfpathlineto{\pgfqpoint{4.373321in}{4.525277in}}%
\pgfpathlineto{\pgfqpoint{4.376542in}{4.523700in}}%
\pgfpathlineto{\pgfqpoint{4.377616in}{4.532709in}}%
\pgfpathlineto{\pgfqpoint{4.378690in}{4.532259in}}%
\pgfpathlineto{\pgfqpoint{4.379764in}{4.537439in}}%
\pgfpathlineto{\pgfqpoint{4.380838in}{4.536989in}}%
\pgfpathlineto{\pgfqpoint{4.386207in}{4.537890in}}%
\pgfpathlineto{\pgfqpoint{4.387281in}{4.540142in}}%
\pgfpathlineto{\pgfqpoint{4.388355in}{4.539692in}}%
\pgfpathlineto{\pgfqpoint{4.391576in}{4.545323in}}%
\pgfpathlineto{\pgfqpoint{4.392650in}{4.545323in}}%
\pgfpathlineto{\pgfqpoint{4.393724in}{4.538340in}}%
\pgfpathlineto{\pgfqpoint{4.394798in}{4.544647in}}%
\pgfpathlineto{\pgfqpoint{4.395872in}{4.543971in}}%
\pgfpathlineto{\pgfqpoint{4.399093in}{4.548701in}}%
\pgfpathlineto{\pgfqpoint{4.401241in}{4.540593in}}%
\pgfpathlineto{\pgfqpoint{4.402315in}{4.543070in}}%
\pgfpathlineto{\pgfqpoint{4.406610in}{4.537439in}}%
\pgfpathlineto{\pgfqpoint{4.407684in}{4.532034in}}%
\pgfpathlineto{\pgfqpoint{4.408758in}{4.539241in}}%
\pgfpathlineto{\pgfqpoint{4.409831in}{4.550503in}}%
\pgfpathlineto{\pgfqpoint{4.410905in}{4.543521in}}%
\pgfpathlineto{\pgfqpoint{4.415201in}{4.550278in}}%
\pgfpathlineto{\pgfqpoint{4.416274in}{4.540818in}}%
\pgfpathlineto{\pgfqpoint{4.417348in}{4.536313in}}%
\pgfpathlineto{\pgfqpoint{4.418422in}{4.534737in}}%
\pgfpathlineto{\pgfqpoint{4.421644in}{4.532034in}}%
\pgfpathlineto{\pgfqpoint{4.422717in}{4.529556in}}%
\pgfpathlineto{\pgfqpoint{4.424865in}{4.545773in}}%
\pgfpathlineto{\pgfqpoint{4.425939in}{4.548701in}}%
\pgfpathlineto{\pgfqpoint{4.429161in}{4.548926in}}%
\pgfpathlineto{\pgfqpoint{4.430234in}{4.558837in}}%
\pgfpathlineto{\pgfqpoint{4.431308in}{4.562891in}}%
\pgfpathlineto{\pgfqpoint{4.432382in}{4.555458in}}%
\pgfpathlineto{\pgfqpoint{4.433456in}{4.543746in}}%
\pgfpathlineto{\pgfqpoint{4.436677in}{4.541043in}}%
\pgfpathlineto{\pgfqpoint{4.437751in}{4.534061in}}%
\pgfpathlineto{\pgfqpoint{4.438825in}{4.530908in}}%
\pgfpathlineto{\pgfqpoint{4.439899in}{4.531809in}}%
\pgfpathlineto{\pgfqpoint{4.440973in}{4.538791in}}%
\pgfpathlineto{\pgfqpoint{4.446342in}{4.518520in}}%
\pgfpathlineto{\pgfqpoint{4.447416in}{4.512889in}}%
\pgfpathlineto{\pgfqpoint{4.448490in}{4.510636in}}%
\pgfpathlineto{\pgfqpoint{4.451711in}{4.512213in}}%
\pgfpathlineto{\pgfqpoint{4.452785in}{4.509961in}}%
\pgfpathlineto{\pgfqpoint{4.453859in}{4.499375in}}%
\pgfpathlineto{\pgfqpoint{4.454933in}{4.503204in}}%
\pgfpathlineto{\pgfqpoint{4.461376in}{4.511763in}}%
\pgfpathlineto{\pgfqpoint{4.462450in}{4.506582in}}%
\pgfpathlineto{\pgfqpoint{4.463523in}{4.493969in}}%
\pgfpathlineto{\pgfqpoint{4.466745in}{4.492167in}}%
\pgfpathlineto{\pgfqpoint{4.467819in}{4.499149in}}%
\pgfpathlineto{\pgfqpoint{4.468893in}{4.510636in}}%
\pgfpathlineto{\pgfqpoint{4.469966in}{4.508384in}}%
\pgfpathlineto{\pgfqpoint{4.471040in}{4.516718in}}%
\pgfpathlineto{\pgfqpoint{4.474262in}{4.509285in}}%
\pgfpathlineto{\pgfqpoint{4.475336in}{4.518745in}}%
\pgfpathlineto{\pgfqpoint{4.476409in}{4.519195in}}%
\pgfpathlineto{\pgfqpoint{4.478557in}{4.533610in}}%
\pgfpathlineto{\pgfqpoint{4.481779in}{4.534962in}}%
\pgfpathlineto{\pgfqpoint{4.483926in}{4.539016in}}%
\pgfpathlineto{\pgfqpoint{4.485000in}{4.522123in}}%
\pgfpathlineto{\pgfqpoint{4.486074in}{4.526403in}}%
\pgfpathlineto{\pgfqpoint{4.489295in}{4.510411in}}%
\pgfpathlineto{\pgfqpoint{4.490369in}{4.509510in}}%
\pgfpathlineto{\pgfqpoint{4.491443in}{4.513339in}}%
\pgfpathlineto{\pgfqpoint{4.492517in}{4.506357in}}%
\pgfpathlineto{\pgfqpoint{4.493591in}{4.521222in}}%
\pgfpathlineto{\pgfqpoint{4.496812in}{4.521448in}}%
\pgfpathlineto{\pgfqpoint{4.497886in}{4.524826in}}%
\pgfpathlineto{\pgfqpoint{4.500034in}{4.519421in}}%
\pgfpathlineto{\pgfqpoint{4.501108in}{4.512213in}}%
\pgfpathlineto{\pgfqpoint{4.504329in}{4.512438in}}%
\pgfpathlineto{\pgfqpoint{4.505403in}{4.501402in}}%
\pgfpathlineto{\pgfqpoint{4.506477in}{4.498474in}}%
\pgfpathlineto{\pgfqpoint{4.507551in}{4.486762in}}%
\pgfpathlineto{\pgfqpoint{4.508625in}{4.495996in}}%
\pgfpathlineto{\pgfqpoint{4.511846in}{4.493744in}}%
\pgfpathlineto{\pgfqpoint{4.512920in}{4.498249in}}%
\pgfpathlineto{\pgfqpoint{4.513994in}{4.513114in}}%
\pgfpathlineto{\pgfqpoint{4.515068in}{4.510411in}}%
\pgfpathlineto{\pgfqpoint{4.516141in}{4.500726in}}%
\pgfpathlineto{\pgfqpoint{4.519363in}{4.495996in}}%
\pgfpathlineto{\pgfqpoint{4.520437in}{4.491491in}}%
\pgfpathlineto{\pgfqpoint{4.521511in}{4.493744in}}%
\pgfpathlineto{\pgfqpoint{4.523658in}{4.509736in}}%
\pgfpathlineto{\pgfqpoint{4.527954in}{4.504105in}}%
\pgfpathlineto{\pgfqpoint{4.529028in}{4.508159in}}%
\pgfpathlineto{\pgfqpoint{4.530101in}{4.507708in}}%
\pgfpathlineto{\pgfqpoint{4.531175in}{4.517619in}}%
\pgfpathlineto{\pgfqpoint{4.534397in}{4.519195in}}%
\pgfpathlineto{\pgfqpoint{4.536544in}{4.524376in}}%
\pgfpathlineto{\pgfqpoint{4.537618in}{4.526178in}}%
\pgfpathlineto{\pgfqpoint{4.538692in}{4.529556in}}%
\pgfpathlineto{\pgfqpoint{4.541914in}{4.526178in}}%
\pgfpathlineto{\pgfqpoint{4.544061in}{4.518520in}}%
\pgfpathlineto{\pgfqpoint{4.545135in}{4.524151in}}%
\pgfpathlineto{\pgfqpoint{4.546209in}{4.519421in}}%
\pgfpathlineto{\pgfqpoint{4.549430in}{4.517619in}}%
\pgfpathlineto{\pgfqpoint{4.550504in}{4.515592in}}%
\pgfpathlineto{\pgfqpoint{4.551578in}{4.508609in}}%
\pgfpathlineto{\pgfqpoint{4.552652in}{4.495996in}}%
\pgfpathlineto{\pgfqpoint{4.553726in}{4.493744in}}%
\pgfpathlineto{\pgfqpoint{4.556947in}{4.492392in}}%
\pgfpathlineto{\pgfqpoint{4.558021in}{4.495320in}}%
\pgfpathlineto{\pgfqpoint{4.560169in}{4.482032in}}%
\pgfpathlineto{\pgfqpoint{4.561243in}{4.492167in}}%
\pgfpathlineto{\pgfqpoint{4.565538in}{4.484734in}}%
\pgfpathlineto{\pgfqpoint{4.566612in}{4.484284in}}%
\pgfpathlineto{\pgfqpoint{4.567686in}{4.497122in}}%
\pgfpathlineto{\pgfqpoint{4.568760in}{4.480680in}}%
\pgfpathlineto{\pgfqpoint{4.571981in}{4.463788in}}%
\pgfpathlineto{\pgfqpoint{4.573055in}{4.465139in}}%
\pgfpathlineto{\pgfqpoint{4.574129in}{4.462436in}}%
\pgfpathlineto{\pgfqpoint{4.575203in}{4.465589in}}%
\pgfpathlineto{\pgfqpoint{4.576276in}{4.465815in}}%
\pgfpathlineto{\pgfqpoint{4.579498in}{4.464689in}}%
\pgfpathlineto{\pgfqpoint{4.580572in}{4.465815in}}%
\pgfpathlineto{\pgfqpoint{4.581646in}{4.461760in}}%
\pgfpathlineto{\pgfqpoint{4.582719in}{4.462211in}}%
\pgfpathlineto{\pgfqpoint{4.583793in}{4.461310in}}%
\pgfpathlineto{\pgfqpoint{4.587015in}{4.455454in}}%
\pgfpathlineto{\pgfqpoint{4.588089in}{4.450724in}}%
\pgfpathlineto{\pgfqpoint{4.589162in}{4.452751in}}%
\pgfpathlineto{\pgfqpoint{4.590236in}{4.460634in}}%
\pgfpathlineto{\pgfqpoint{4.591310in}{4.452751in}}%
\pgfpathlineto{\pgfqpoint{4.594532in}{4.454553in}}%
\pgfpathlineto{\pgfqpoint{4.595605in}{4.456130in}}%
\pgfpathlineto{\pgfqpoint{4.596679in}{4.450724in}}%
\pgfpathlineto{\pgfqpoint{4.597753in}{4.449373in}}%
\pgfpathlineto{\pgfqpoint{4.598827in}{4.453202in}}%
\pgfpathlineto{\pgfqpoint{4.602049in}{4.450048in}}%
\pgfpathlineto{\pgfqpoint{4.603122in}{4.439012in}}%
\pgfpathlineto{\pgfqpoint{4.606344in}{4.431354in}}%
\pgfpathlineto{\pgfqpoint{4.609565in}{4.435858in}}%
\pgfpathlineto{\pgfqpoint{4.610639in}{4.447120in}}%
\pgfpathlineto{\pgfqpoint{4.611713in}{4.436985in}}%
\pgfpathlineto{\pgfqpoint{4.612787in}{4.434732in}}%
\pgfpathlineto{\pgfqpoint{4.613861in}{4.427525in}}%
\pgfpathlineto{\pgfqpoint{4.617082in}{4.431129in}}%
\pgfpathlineto{\pgfqpoint{4.618156in}{4.433381in}}%
\pgfpathlineto{\pgfqpoint{4.619230in}{4.431129in}}%
\pgfpathlineto{\pgfqpoint{4.621378in}{4.439913in}}%
\pgfpathlineto{\pgfqpoint{4.626747in}{4.434057in}}%
\pgfpathlineto{\pgfqpoint{4.627821in}{4.445544in}}%
\pgfpathlineto{\pgfqpoint{4.628894in}{4.441940in}}%
\pgfpathlineto{\pgfqpoint{4.632116in}{4.441940in}}%
\pgfpathlineto{\pgfqpoint{4.633190in}{4.439913in}}%
\pgfpathlineto{\pgfqpoint{4.634264in}{4.425047in}}%
\pgfpathlineto{\pgfqpoint{4.635338in}{4.423471in}}%
\pgfpathlineto{\pgfqpoint{4.636411in}{4.423020in}}%
\pgfpathlineto{\pgfqpoint{4.639633in}{4.422344in}}%
\pgfpathlineto{\pgfqpoint{4.641781in}{4.407479in}}%
\pgfpathlineto{\pgfqpoint{4.642854in}{4.408830in}}%
\pgfpathlineto{\pgfqpoint{4.643928in}{4.411758in}}%
\pgfpathlineto{\pgfqpoint{4.647150in}{4.412885in}}%
\pgfpathlineto{\pgfqpoint{4.648224in}{4.399145in}}%
\pgfpathlineto{\pgfqpoint{4.649297in}{4.396442in}}%
\pgfpathlineto{\pgfqpoint{4.651445in}{4.403199in}}%
\pgfpathlineto{\pgfqpoint{4.655740in}{4.407028in}}%
\pgfpathlineto{\pgfqpoint{4.656814in}{4.409956in}}%
\pgfpathlineto{\pgfqpoint{4.657888in}{4.399596in}}%
\pgfpathlineto{\pgfqpoint{4.658962in}{4.400271in}}%
\pgfpathlineto{\pgfqpoint{4.662183in}{4.399821in}}%
\pgfpathlineto{\pgfqpoint{4.663257in}{4.407704in}}%
\pgfpathlineto{\pgfqpoint{4.664331in}{4.405452in}}%
\pgfpathlineto{\pgfqpoint{4.665405in}{4.411083in}}%
\pgfpathlineto{\pgfqpoint{4.666479in}{4.411083in}}%
\pgfpathlineto{\pgfqpoint{4.669700in}{4.409731in}}%
\pgfpathlineto{\pgfqpoint{4.670774in}{4.420092in}}%
\pgfpathlineto{\pgfqpoint{4.671848in}{4.420543in}}%
\pgfpathlineto{\pgfqpoint{4.672922in}{4.417164in}}%
\pgfpathlineto{\pgfqpoint{4.673996in}{4.418065in}}%
\pgfpathlineto{\pgfqpoint{4.677217in}{4.418741in}}%
\pgfpathlineto{\pgfqpoint{4.678291in}{4.425948in}}%
\pgfpathlineto{\pgfqpoint{4.679365in}{4.428426in}}%
\pgfpathlineto{\pgfqpoint{4.680439in}{4.427074in}}%
\pgfpathlineto{\pgfqpoint{4.681513in}{4.422344in}}%
\pgfpathlineto{\pgfqpoint{4.684734in}{4.420092in}}%
\pgfpathlineto{\pgfqpoint{4.686882in}{4.420092in}}%
\pgfpathlineto{\pgfqpoint{4.687956in}{4.417614in}}%
\pgfpathlineto{\pgfqpoint{4.689029in}{4.412434in}}%
\pgfpathlineto{\pgfqpoint{4.692251in}{4.419642in}}%
\pgfpathlineto{\pgfqpoint{4.694399in}{4.433606in}}%
\pgfpathlineto{\pgfqpoint{4.695472in}{4.432480in}}%
\pgfpathlineto{\pgfqpoint{4.696546in}{4.429552in}}%
\pgfpathlineto{\pgfqpoint{4.699768in}{4.431129in}}%
\pgfpathlineto{\pgfqpoint{4.700842in}{4.428201in}}%
\pgfpathlineto{\pgfqpoint{4.702989in}{4.441264in}}%
\pgfpathlineto{\pgfqpoint{4.704063in}{4.444868in}}%
\pgfpathlineto{\pgfqpoint{4.707285in}{4.444868in}}%
\pgfpathlineto{\pgfqpoint{4.708359in}{4.442841in}}%
\pgfpathlineto{\pgfqpoint{4.709432in}{4.446895in}}%
\pgfpathlineto{\pgfqpoint{4.710506in}{4.457481in}}%
\pgfpathlineto{\pgfqpoint{4.711580in}{4.431129in}}%
\pgfpathlineto{\pgfqpoint{4.715875in}{4.429777in}}%
\pgfpathlineto{\pgfqpoint{4.716949in}{4.428651in}}%
\pgfpathlineto{\pgfqpoint{4.718023in}{4.425723in}}%
\pgfpathlineto{\pgfqpoint{4.719097in}{4.425948in}}%
\pgfpathlineto{\pgfqpoint{4.722318in}{4.427525in}}%
\pgfpathlineto{\pgfqpoint{4.723392in}{4.431804in}}%
\pgfpathlineto{\pgfqpoint{4.724466in}{4.433381in}}%
\pgfpathlineto{\pgfqpoint{4.725540in}{4.427300in}}%
\pgfpathlineto{\pgfqpoint{4.726614in}{4.433156in}}%
\pgfpathlineto{\pgfqpoint{4.729835in}{4.429552in}}%
\pgfpathlineto{\pgfqpoint{4.730909in}{4.434057in}}%
\pgfpathlineto{\pgfqpoint{4.733057in}{4.427300in}}%
\pgfpathlineto{\pgfqpoint{4.734131in}{4.430678in}}%
\pgfpathlineto{\pgfqpoint{4.737352in}{4.431354in}}%
\pgfpathlineto{\pgfqpoint{4.738426in}{4.434282in}}%
\pgfpathlineto{\pgfqpoint{4.739500in}{4.435408in}}%
\pgfpathlineto{\pgfqpoint{4.740574in}{4.434958in}}%
\pgfpathlineto{\pgfqpoint{4.741648in}{4.432930in}}%
\pgfpathlineto{\pgfqpoint{4.745943in}{4.432480in}}%
\pgfpathlineto{\pgfqpoint{4.748091in}{4.416714in}}%
\pgfpathlineto{\pgfqpoint{4.749164in}{4.419867in}}%
\pgfpathlineto{\pgfqpoint{4.752386in}{4.416488in}}%
\pgfpathlineto{\pgfqpoint{4.754534in}{4.429327in}}%
\pgfpathlineto{\pgfqpoint{4.755607in}{4.428426in}}%
\pgfpathlineto{\pgfqpoint{4.756681in}{4.431354in}}%
\pgfpathlineto{\pgfqpoint{4.759903in}{4.435858in}}%
\pgfpathlineto{\pgfqpoint{4.763124in}{4.447345in}}%
\pgfpathlineto{\pgfqpoint{4.764198in}{4.441489in}}%
\pgfpathlineto{\pgfqpoint{4.767420in}{4.443516in}}%
\pgfpathlineto{\pgfqpoint{4.768493in}{4.442616in}}%
\pgfpathlineto{\pgfqpoint{4.769567in}{4.443066in}}%
\pgfpathlineto{\pgfqpoint{4.770641in}{4.442841in}}%
\pgfpathlineto{\pgfqpoint{4.771715in}{4.437886in}}%
\pgfpathlineto{\pgfqpoint{4.774937in}{4.435633in}}%
\pgfpathlineto{\pgfqpoint{4.776010in}{4.433381in}}%
\pgfpathlineto{\pgfqpoint{4.777084in}{4.434057in}}%
\pgfpathlineto{\pgfqpoint{4.778158in}{4.431579in}}%
\pgfpathlineto{\pgfqpoint{4.779232in}{4.436759in}}%
\pgfpathlineto{\pgfqpoint{4.782453in}{4.433606in}}%
\pgfpathlineto{\pgfqpoint{4.783527in}{4.445093in}}%
\pgfpathlineto{\pgfqpoint{4.784601in}{4.446219in}}%
\pgfpathlineto{\pgfqpoint{4.785675in}{4.446219in}}%
\pgfpathlineto{\pgfqpoint{4.789970in}{4.438787in}}%
\pgfpathlineto{\pgfqpoint{4.791044in}{4.439237in}}%
\pgfpathlineto{\pgfqpoint{4.792118in}{4.431579in}}%
\pgfpathlineto{\pgfqpoint{4.793192in}{4.433381in}}%
\pgfpathlineto{\pgfqpoint{4.794266in}{4.429552in}}%
\pgfpathlineto{\pgfqpoint{4.797487in}{4.432480in}}%
\pgfpathlineto{\pgfqpoint{4.799635in}{4.454103in}}%
\pgfpathlineto{\pgfqpoint{4.800709in}{4.446219in}}%
\pgfpathlineto{\pgfqpoint{4.801782in}{4.443291in}}%
\pgfpathlineto{\pgfqpoint{4.805004in}{4.437886in}}%
\pgfpathlineto{\pgfqpoint{4.806078in}{4.447796in}}%
\pgfpathlineto{\pgfqpoint{4.807152in}{4.448021in}}%
\pgfpathlineto{\pgfqpoint{4.809299in}{4.457481in}}%
\pgfpathlineto{\pgfqpoint{4.812521in}{4.465589in}}%
\pgfpathlineto{\pgfqpoint{4.814669in}{4.480005in}}%
\pgfpathlineto{\pgfqpoint{4.815742in}{4.474824in}}%
\pgfpathlineto{\pgfqpoint{4.816816in}{4.475275in}}%
\pgfpathlineto{\pgfqpoint{4.824333in}{4.489915in}}%
\pgfpathlineto{\pgfqpoint{4.827555in}{4.489014in}}%
\pgfpathlineto{\pgfqpoint{4.828628in}{4.493519in}}%
\pgfpathlineto{\pgfqpoint{4.830776in}{4.497348in}}%
\pgfpathlineto{\pgfqpoint{4.831850in}{4.500050in}}%
\pgfpathlineto{\pgfqpoint{4.835071in}{4.492167in}}%
\pgfpathlineto{\pgfqpoint{4.836145in}{4.487663in}}%
\pgfpathlineto{\pgfqpoint{4.837219in}{4.494420in}}%
\pgfpathlineto{\pgfqpoint{4.838293in}{4.489464in}}%
\pgfpathlineto{\pgfqpoint{4.839367in}{4.491266in}}%
\pgfpathlineto{\pgfqpoint{4.842588in}{4.492167in}}%
\pgfpathlineto{\pgfqpoint{4.843662in}{4.493519in}}%
\pgfpathlineto{\pgfqpoint{4.844736in}{4.491942in}}%
\pgfpathlineto{\pgfqpoint{4.845810in}{4.491491in}}%
\pgfpathlineto{\pgfqpoint{4.846884in}{4.489014in}}%
\pgfpathlineto{\pgfqpoint{4.851179in}{4.492167in}}%
\pgfpathlineto{\pgfqpoint{4.852253in}{4.495771in}}%
\pgfpathlineto{\pgfqpoint{4.853327in}{4.494645in}}%
\pgfpathlineto{\pgfqpoint{4.854401in}{4.495771in}}%
\pgfpathlineto{\pgfqpoint{4.857622in}{4.513790in}}%
\pgfpathlineto{\pgfqpoint{4.858696in}{4.516042in}}%
\pgfpathlineto{\pgfqpoint{4.859770in}{4.503654in}}%
\pgfpathlineto{\pgfqpoint{4.861917in}{4.501627in}}%
\pgfpathlineto{\pgfqpoint{4.865139in}{4.509510in}}%
\pgfpathlineto{\pgfqpoint{4.867287in}{4.499149in}}%
\pgfpathlineto{\pgfqpoint{4.868360in}{4.509060in}}%
\pgfpathlineto{\pgfqpoint{4.869434in}{4.507934in}}%
\pgfpathlineto{\pgfqpoint{4.872656in}{4.511312in}}%
\pgfpathlineto{\pgfqpoint{4.873730in}{4.518069in}}%
\pgfpathlineto{\pgfqpoint{4.874804in}{4.509285in}}%
\pgfpathlineto{\pgfqpoint{4.875877in}{4.494194in}}%
\pgfpathlineto{\pgfqpoint{4.876951in}{4.494420in}}%
\pgfpathlineto{\pgfqpoint{4.880173in}{4.482707in}}%
\pgfpathlineto{\pgfqpoint{4.881247in}{4.488338in}}%
\pgfpathlineto{\pgfqpoint{4.882320in}{4.490816in}}%
\pgfpathlineto{\pgfqpoint{4.883394in}{4.491491in}}%
\pgfpathlineto{\pgfqpoint{4.884468in}{4.494870in}}%
\pgfpathlineto{\pgfqpoint{4.888763in}{4.485410in}}%
\pgfpathlineto{\pgfqpoint{4.889837in}{4.486086in}}%
\pgfpathlineto{\pgfqpoint{4.891985in}{4.491717in}}%
\pgfpathlineto{\pgfqpoint{4.895206in}{4.475500in}}%
\pgfpathlineto{\pgfqpoint{4.896280in}{4.474599in}}%
\pgfpathlineto{\pgfqpoint{4.897354in}{4.476626in}}%
\pgfpathlineto{\pgfqpoint{4.898428in}{4.490365in}}%
\pgfpathlineto{\pgfqpoint{4.899502in}{4.488789in}}%
\pgfpathlineto{\pgfqpoint{4.902723in}{4.489464in}}%
\pgfpathlineto{\pgfqpoint{4.903797in}{4.495320in}}%
\pgfpathlineto{\pgfqpoint{4.904871in}{4.493519in}}%
\pgfpathlineto{\pgfqpoint{4.905945in}{4.476401in}}%
\pgfpathlineto{\pgfqpoint{4.907019in}{4.472572in}}%
\pgfpathlineto{\pgfqpoint{4.910240in}{4.467617in}}%
\pgfpathlineto{\pgfqpoint{4.911314in}{4.467166in}}%
\pgfpathlineto{\pgfqpoint{4.914536in}{4.476851in}}%
\pgfpathlineto{\pgfqpoint{4.917757in}{4.476401in}}%
\pgfpathlineto{\pgfqpoint{4.918831in}{4.479104in}}%
\pgfpathlineto{\pgfqpoint{4.919905in}{4.478203in}}%
\pgfpathlineto{\pgfqpoint{4.920979in}{4.475725in}}%
\pgfpathlineto{\pgfqpoint{4.922052in}{4.475950in}}%
\pgfpathlineto{\pgfqpoint{4.925274in}{4.474599in}}%
\pgfpathlineto{\pgfqpoint{4.927422in}{4.470319in}}%
\pgfpathlineto{\pgfqpoint{4.928495in}{4.465589in}}%
\pgfpathlineto{\pgfqpoint{4.929569in}{4.466716in}}%
\pgfpathlineto{\pgfqpoint{4.932791in}{4.469193in}}%
\pgfpathlineto{\pgfqpoint{4.933865in}{4.466941in}}%
\pgfpathlineto{\pgfqpoint{4.934938in}{4.468067in}}%
\pgfpathlineto{\pgfqpoint{4.936012in}{4.457706in}}%
\pgfpathlineto{\pgfqpoint{4.937086in}{4.455454in}}%
\pgfpathlineto{\pgfqpoint{4.940308in}{4.462436in}}%
\pgfpathlineto{\pgfqpoint{4.941381in}{4.466941in}}%
\pgfpathlineto{\pgfqpoint{4.942455in}{4.460409in}}%
\pgfpathlineto{\pgfqpoint{4.943529in}{4.461986in}}%
\pgfpathlineto{\pgfqpoint{4.944603in}{4.465589in}}%
\pgfpathlineto{\pgfqpoint{4.948898in}{4.460860in}}%
\pgfpathlineto{\pgfqpoint{4.949972in}{4.462661in}}%
\pgfpathlineto{\pgfqpoint{4.951046in}{4.458382in}}%
\pgfpathlineto{\pgfqpoint{4.952120in}{4.456805in}}%
\pgfpathlineto{\pgfqpoint{4.956415in}{4.458607in}}%
\pgfpathlineto{\pgfqpoint{4.957489in}{4.469193in}}%
\pgfpathlineto{\pgfqpoint{4.958563in}{4.468518in}}%
\pgfpathlineto{\pgfqpoint{4.963932in}{4.475725in}}%
\pgfpathlineto{\pgfqpoint{4.966080in}{4.469193in}}%
\pgfpathlineto{\pgfqpoint{4.967154in}{4.484509in}}%
\pgfpathlineto{\pgfqpoint{4.970375in}{4.483383in}}%
\pgfpathlineto{\pgfqpoint{4.971449in}{4.490140in}}%
\pgfpathlineto{\pgfqpoint{4.972523in}{4.493068in}}%
\pgfpathlineto{\pgfqpoint{4.973597in}{4.493293in}}%
\pgfpathlineto{\pgfqpoint{4.974670in}{4.490591in}}%
\pgfpathlineto{\pgfqpoint{4.977892in}{4.487888in}}%
\pgfpathlineto{\pgfqpoint{4.978966in}{4.489239in}}%
\pgfpathlineto{\pgfqpoint{4.980040in}{4.489239in}}%
\pgfpathlineto{\pgfqpoint{4.982187in}{4.475725in}}%
\pgfpathlineto{\pgfqpoint{4.985409in}{4.474599in}}%
\pgfpathlineto{\pgfqpoint{4.986483in}{4.472797in}}%
\pgfpathlineto{\pgfqpoint{4.987557in}{4.473923in}}%
\pgfpathlineto{\pgfqpoint{4.988630in}{4.468518in}}%
\pgfpathlineto{\pgfqpoint{4.989704in}{4.472572in}}%
\pgfpathlineto{\pgfqpoint{4.992926in}{4.472797in}}%
\pgfpathlineto{\pgfqpoint{4.994000in}{4.466265in}}%
\pgfpathlineto{\pgfqpoint{4.995073in}{4.468292in}}%
\pgfpathlineto{\pgfqpoint{4.996147in}{4.477977in}}%
\pgfpathlineto{\pgfqpoint{4.997221in}{4.480905in}}%
\pgfpathlineto{\pgfqpoint{5.000443in}{4.484509in}}%
\pgfpathlineto{\pgfqpoint{5.001516in}{4.483383in}}%
\pgfpathlineto{\pgfqpoint{5.003664in}{4.493519in}}%
\pgfpathlineto{\pgfqpoint{5.004738in}{4.492843in}}%
\pgfpathlineto{\pgfqpoint{5.007959in}{4.497573in}}%
\pgfpathlineto{\pgfqpoint{5.009033in}{4.496221in}}%
\pgfpathlineto{\pgfqpoint{5.010107in}{4.490140in}}%
\pgfpathlineto{\pgfqpoint{5.011181in}{4.490816in}}%
\pgfpathlineto{\pgfqpoint{5.012255in}{4.499600in}}%
\pgfpathlineto{\pgfqpoint{5.015476in}{4.501852in}}%
\pgfpathlineto{\pgfqpoint{5.016550in}{4.504780in}}%
\pgfpathlineto{\pgfqpoint{5.018698in}{4.503879in}}%
\pgfpathlineto{\pgfqpoint{5.019772in}{4.501177in}}%
\pgfpathlineto{\pgfqpoint{5.024067in}{4.499825in}}%
\pgfpathlineto{\pgfqpoint{5.025141in}{4.504105in}}%
\pgfpathlineto{\pgfqpoint{5.026215in}{4.500501in}}%
\pgfpathlineto{\pgfqpoint{5.030510in}{4.502528in}}%
\pgfpathlineto{\pgfqpoint{5.032658in}{4.510636in}}%
\pgfpathlineto{\pgfqpoint{5.033732in}{4.506807in}}%
\pgfpathlineto{\pgfqpoint{5.034805in}{4.509285in}}%
\pgfpathlineto{\pgfqpoint{5.038027in}{4.510636in}}%
\pgfpathlineto{\pgfqpoint{5.039101in}{4.512664in}}%
\pgfpathlineto{\pgfqpoint{5.040175in}{4.509961in}}%
\pgfpathlineto{\pgfqpoint{5.041248in}{4.522799in}}%
\pgfpathlineto{\pgfqpoint{5.042322in}{4.496672in}}%
\pgfpathlineto{\pgfqpoint{5.045544in}{4.494420in}}%
\pgfpathlineto{\pgfqpoint{5.046618in}{4.492618in}}%
\pgfpathlineto{\pgfqpoint{5.047692in}{4.497348in}}%
\pgfpathlineto{\pgfqpoint{5.049839in}{4.496221in}}%
\pgfpathlineto{\pgfqpoint{5.053061in}{4.493519in}}%
\pgfpathlineto{\pgfqpoint{5.054135in}{4.490591in}}%
\pgfpathlineto{\pgfqpoint{5.055208in}{4.494194in}}%
\pgfpathlineto{\pgfqpoint{5.057356in}{4.514916in}}%
\pgfpathlineto{\pgfqpoint{5.060578in}{4.516943in}}%
\pgfpathlineto{\pgfqpoint{5.061651in}{4.514916in}}%
\pgfpathlineto{\pgfqpoint{5.062725in}{4.507258in}}%
\pgfpathlineto{\pgfqpoint{5.063799in}{4.508159in}}%
\pgfpathlineto{\pgfqpoint{5.064873in}{4.504780in}}%
\pgfpathlineto{\pgfqpoint{5.068094in}{4.507934in}}%
\pgfpathlineto{\pgfqpoint{5.069168in}{4.511763in}}%
\pgfpathlineto{\pgfqpoint{5.070242in}{4.521222in}}%
\pgfpathlineto{\pgfqpoint{5.071316in}{4.520997in}}%
\pgfpathlineto{\pgfqpoint{5.072390in}{4.519421in}}%
\pgfpathlineto{\pgfqpoint{5.076685in}{4.526628in}}%
\pgfpathlineto{\pgfqpoint{5.078833in}{4.531809in}}%
\pgfpathlineto{\pgfqpoint{5.079907in}{4.529781in}}%
\pgfpathlineto{\pgfqpoint{5.084202in}{4.536764in}}%
\pgfpathlineto{\pgfqpoint{5.086350in}{4.533385in}}%
\pgfpathlineto{\pgfqpoint{5.087424in}{4.533160in}}%
\pgfpathlineto{\pgfqpoint{5.090645in}{4.527079in}}%
\pgfpathlineto{\pgfqpoint{5.091719in}{4.529556in}}%
\pgfpathlineto{\pgfqpoint{5.093867in}{4.524376in}}%
\pgfpathlineto{\pgfqpoint{5.094940in}{4.528430in}}%
\pgfpathlineto{\pgfqpoint{5.098162in}{4.527980in}}%
\pgfpathlineto{\pgfqpoint{5.099236in}{4.547124in}}%
\pgfpathlineto{\pgfqpoint{5.100310in}{4.550278in}}%
\pgfpathlineto{\pgfqpoint{5.101383in}{4.547800in}}%
\pgfpathlineto{\pgfqpoint{5.102457in}{4.534511in}}%
\pgfpathlineto{\pgfqpoint{5.106753in}{4.529556in}}%
\pgfpathlineto{\pgfqpoint{5.109974in}{4.514691in}}%
\pgfpathlineto{\pgfqpoint{5.113196in}{4.513114in}}%
\pgfpathlineto{\pgfqpoint{5.114269in}{4.516493in}}%
\pgfpathlineto{\pgfqpoint{5.115343in}{4.512213in}}%
\pgfpathlineto{\pgfqpoint{5.116417in}{4.513339in}}%
\pgfpathlineto{\pgfqpoint{5.121786in}{4.495771in}}%
\pgfpathlineto{\pgfqpoint{5.122860in}{4.494420in}}%
\pgfpathlineto{\pgfqpoint{5.123934in}{4.503429in}}%
\pgfpathlineto{\pgfqpoint{5.125008in}{4.507708in}}%
\pgfpathlineto{\pgfqpoint{5.128229in}{4.509285in}}%
\pgfpathlineto{\pgfqpoint{5.130377in}{4.514240in}}%
\pgfpathlineto{\pgfqpoint{5.131451in}{4.517168in}}%
\pgfpathlineto{\pgfqpoint{5.132525in}{4.518294in}}%
\pgfpathlineto{\pgfqpoint{5.136820in}{4.518294in}}%
\pgfpathlineto{\pgfqpoint{5.137894in}{4.513339in}}%
\pgfpathlineto{\pgfqpoint{5.138968in}{4.517844in}}%
\pgfpathlineto{\pgfqpoint{5.140042in}{4.511763in}}%
\pgfpathlineto{\pgfqpoint{5.143263in}{4.515817in}}%
\pgfpathlineto{\pgfqpoint{5.144337in}{4.515592in}}%
\pgfpathlineto{\pgfqpoint{5.145411in}{4.518970in}}%
\pgfpathlineto{\pgfqpoint{5.146485in}{4.518294in}}%
\pgfpathlineto{\pgfqpoint{5.147558in}{4.518294in}}%
\pgfpathlineto{\pgfqpoint{5.150780in}{4.513339in}}%
\pgfpathlineto{\pgfqpoint{5.151854in}{4.515366in}}%
\pgfpathlineto{\pgfqpoint{5.152928in}{4.513339in}}%
\pgfpathlineto{\pgfqpoint{5.154002in}{4.515817in}}%
\pgfpathlineto{\pgfqpoint{5.155075in}{4.516042in}}%
\pgfpathlineto{\pgfqpoint{5.158297in}{4.519871in}}%
\pgfpathlineto{\pgfqpoint{5.159371in}{4.517619in}}%
\pgfpathlineto{\pgfqpoint{5.160445in}{4.518294in}}%
\pgfpathlineto{\pgfqpoint{5.161518in}{4.514691in}}%
\pgfpathlineto{\pgfqpoint{5.162592in}{4.513339in}}%
\pgfpathlineto{\pgfqpoint{5.165814in}{4.517168in}}%
\pgfpathlineto{\pgfqpoint{5.166888in}{4.519421in}}%
\pgfpathlineto{\pgfqpoint{5.167961in}{4.523250in}}%
\pgfpathlineto{\pgfqpoint{5.169035in}{4.531133in}}%
\pgfpathlineto{\pgfqpoint{5.170109in}{4.526178in}}%
\pgfpathlineto{\pgfqpoint{5.173331in}{4.525277in}}%
\pgfpathlineto{\pgfqpoint{5.174404in}{4.531809in}}%
\pgfpathlineto{\pgfqpoint{5.176552in}{4.528880in}}%
\pgfpathlineto{\pgfqpoint{5.177626in}{4.534962in}}%
\pgfpathlineto{\pgfqpoint{5.180847in}{4.538566in}}%
\pgfpathlineto{\pgfqpoint{5.181921in}{4.541944in}}%
\pgfpathlineto{\pgfqpoint{5.182995in}{4.540142in}}%
\pgfpathlineto{\pgfqpoint{5.184069in}{4.550053in}}%
\pgfpathlineto{\pgfqpoint{5.185143in}{4.545323in}}%
\pgfpathlineto{\pgfqpoint{5.188364in}{4.551629in}}%
\pgfpathlineto{\pgfqpoint{5.189438in}{4.559738in}}%
\pgfpathlineto{\pgfqpoint{5.190512in}{4.561089in}}%
\pgfpathlineto{\pgfqpoint{5.191586in}{4.550503in}}%
\pgfpathlineto{\pgfqpoint{5.192660in}{4.545548in}}%
\pgfpathlineto{\pgfqpoint{5.195881in}{4.552981in}}%
\pgfpathlineto{\pgfqpoint{5.198029in}{4.560188in}}%
\pgfpathlineto{\pgfqpoint{5.199103in}{4.562215in}}%
\pgfpathlineto{\pgfqpoint{5.203398in}{4.560413in}}%
\pgfpathlineto{\pgfqpoint{5.205546in}{4.556584in}}%
\pgfpathlineto{\pgfqpoint{5.206620in}{4.556584in}}%
\pgfpathlineto{\pgfqpoint{5.207693in}{4.547124in}}%
\pgfpathlineto{\pgfqpoint{5.210915in}{4.548476in}}%
\pgfpathlineto{\pgfqpoint{5.213063in}{4.555458in}}%
\pgfpathlineto{\pgfqpoint{5.214136in}{4.550953in}}%
\pgfpathlineto{\pgfqpoint{5.215210in}{4.550053in}}%
\pgfpathlineto{\pgfqpoint{5.219506in}{4.550278in}}%
\pgfpathlineto{\pgfqpoint{5.220580in}{4.553656in}}%
\pgfpathlineto{\pgfqpoint{5.222727in}{4.552305in}}%
\pgfpathlineto{\pgfqpoint{5.225949in}{4.553656in}}%
\pgfpathlineto{\pgfqpoint{5.227023in}{4.555233in}}%
\pgfpathlineto{\pgfqpoint{5.228096in}{4.552981in}}%
\pgfpathlineto{\pgfqpoint{5.230244in}{4.543070in}}%
\pgfpathlineto{\pgfqpoint{5.233466in}{4.547350in}}%
\pgfpathlineto{\pgfqpoint{5.234539in}{4.547350in}}%
\pgfpathlineto{\pgfqpoint{5.235613in}{4.550278in}}%
\pgfpathlineto{\pgfqpoint{5.236687in}{4.549377in}}%
\pgfpathlineto{\pgfqpoint{5.237761in}{4.552080in}}%
\pgfpathlineto{\pgfqpoint{5.242056in}{4.560188in}}%
\pgfpathlineto{\pgfqpoint{5.245278in}{4.572126in}}%
\pgfpathlineto{\pgfqpoint{5.248499in}{4.570999in}}%
\pgfpathlineto{\pgfqpoint{5.249573in}{4.578657in}}%
\pgfpathlineto{\pgfqpoint{5.250647in}{4.577531in}}%
\pgfpathlineto{\pgfqpoint{5.251721in}{4.578657in}}%
\pgfpathlineto{\pgfqpoint{5.252795in}{4.588568in}}%
\pgfpathlineto{\pgfqpoint{5.256016in}{4.583613in}}%
\pgfpathlineto{\pgfqpoint{5.257090in}{4.589919in}}%
\pgfpathlineto{\pgfqpoint{5.258164in}{4.584063in}}%
\pgfpathlineto{\pgfqpoint{5.259238in}{4.584513in}}%
\pgfpathlineto{\pgfqpoint{5.260312in}{4.621677in}}%
\pgfpathlineto{\pgfqpoint{5.263533in}{4.624380in}}%
\pgfpathlineto{\pgfqpoint{5.265681in}{4.622803in}}%
\pgfpathlineto{\pgfqpoint{5.267828in}{4.628209in}}%
\pgfpathlineto{\pgfqpoint{5.271050in}{4.628659in}}%
\pgfpathlineto{\pgfqpoint{5.273198in}{4.641273in}}%
\pgfpathlineto{\pgfqpoint{5.274271in}{4.639471in}}%
\pgfpathlineto{\pgfqpoint{5.275345in}{4.642399in}}%
\pgfpathlineto{\pgfqpoint{5.278567in}{4.641723in}}%
\pgfpathlineto{\pgfqpoint{5.279641in}{4.643300in}}%
\pgfpathlineto{\pgfqpoint{5.280714in}{4.643300in}}%
\pgfpathlineto{\pgfqpoint{5.281788in}{4.646453in}}%
\pgfpathlineto{\pgfqpoint{5.286084in}{4.644201in}}%
\pgfpathlineto{\pgfqpoint{5.287158in}{4.639696in}}%
\pgfpathlineto{\pgfqpoint{5.288231in}{4.641498in}}%
\pgfpathlineto{\pgfqpoint{5.289305in}{4.648705in}}%
\pgfpathlineto{\pgfqpoint{5.290379in}{4.648480in}}%
\pgfpathlineto{\pgfqpoint{5.293601in}{4.653210in}}%
\pgfpathlineto{\pgfqpoint{5.294674in}{4.657490in}}%
\pgfpathlineto{\pgfqpoint{5.295748in}{4.714699in}}%
\pgfpathlineto{\pgfqpoint{5.296822in}{4.696230in}}%
\pgfpathlineto{\pgfqpoint{5.297896in}{4.696230in}}%
\pgfpathlineto{\pgfqpoint{5.301117in}{4.703212in}}%
\pgfpathlineto{\pgfqpoint{5.302191in}{4.717402in}}%
\pgfpathlineto{\pgfqpoint{5.304339in}{4.706816in}}%
\pgfpathlineto{\pgfqpoint{5.308634in}{4.706591in}}%
\pgfpathlineto{\pgfqpoint{5.309708in}{4.705690in}}%
\pgfpathlineto{\pgfqpoint{5.310782in}{4.708843in}}%
\pgfpathlineto{\pgfqpoint{5.311856in}{4.699834in}}%
\pgfpathlineto{\pgfqpoint{5.312930in}{4.696906in}}%
\pgfpathlineto{\pgfqpoint{5.316151in}{4.702987in}}%
\pgfpathlineto{\pgfqpoint{5.317225in}{4.683392in}}%
\pgfpathlineto{\pgfqpoint{5.318299in}{4.683842in}}%
\pgfpathlineto{\pgfqpoint{5.319373in}{4.680463in}}%
\pgfpathlineto{\pgfqpoint{5.320446in}{4.679112in}}%
\pgfpathlineto{\pgfqpoint{5.324742in}{4.689473in}}%
\pgfpathlineto{\pgfqpoint{5.325816in}{4.708393in}}%
\pgfpathlineto{\pgfqpoint{5.326890in}{4.705239in}}%
\pgfpathlineto{\pgfqpoint{5.327963in}{4.709744in}}%
\pgfpathlineto{\pgfqpoint{5.331185in}{4.714474in}}%
\pgfpathlineto{\pgfqpoint{5.332259in}{4.713123in}}%
\pgfpathlineto{\pgfqpoint{5.333333in}{4.716276in}}%
\pgfpathlineto{\pgfqpoint{5.334406in}{4.728889in}}%
\pgfpathlineto{\pgfqpoint{5.335480in}{4.724835in}}%
\pgfpathlineto{\pgfqpoint{5.339776in}{4.722132in}}%
\pgfpathlineto{\pgfqpoint{5.340849in}{4.721907in}}%
\pgfpathlineto{\pgfqpoint{5.341923in}{4.719204in}}%
\pgfpathlineto{\pgfqpoint{5.342997in}{4.724384in}}%
\pgfpathlineto{\pgfqpoint{5.347292in}{4.717627in}}%
\pgfpathlineto{\pgfqpoint{5.348366in}{4.717627in}}%
\pgfpathlineto{\pgfqpoint{5.349440in}{4.724159in}}%
\pgfpathlineto{\pgfqpoint{5.350514in}{4.725961in}}%
\pgfpathlineto{\pgfqpoint{5.353735in}{4.732493in}}%
\pgfpathlineto{\pgfqpoint{5.354809in}{4.724159in}}%
\pgfpathlineto{\pgfqpoint{5.355883in}{4.726411in}}%
\pgfpathlineto{\pgfqpoint{5.356957in}{4.726411in}}%
\pgfpathlineto{\pgfqpoint{5.358031in}{4.718528in}}%
\pgfpathlineto{\pgfqpoint{5.361252in}{4.716952in}}%
\pgfpathlineto{\pgfqpoint{5.362326in}{4.724610in}}%
\pgfpathlineto{\pgfqpoint{5.363400in}{4.725510in}}%
\pgfpathlineto{\pgfqpoint{5.364474in}{4.729339in}}%
\pgfpathlineto{\pgfqpoint{5.365548in}{4.722357in}}%
\pgfpathlineto{\pgfqpoint{5.368769in}{4.720330in}}%
\pgfpathlineto{\pgfqpoint{5.369843in}{4.714699in}}%
\pgfpathlineto{\pgfqpoint{5.370917in}{4.721006in}}%
\pgfpathlineto{\pgfqpoint{5.371991in}{4.709068in}}%
\pgfpathlineto{\pgfqpoint{5.373065in}{4.711546in}}%
\pgfpathlineto{\pgfqpoint{5.376286in}{4.723934in}}%
\pgfpathlineto{\pgfqpoint{5.377360in}{4.722357in}}%
\pgfpathlineto{\pgfqpoint{5.379508in}{4.696906in}}%
\pgfpathlineto{\pgfqpoint{5.380581in}{4.707041in}}%
\pgfpathlineto{\pgfqpoint{5.383803in}{4.708618in}}%
\pgfpathlineto{\pgfqpoint{5.384877in}{4.696005in}}%
\pgfpathlineto{\pgfqpoint{5.385951in}{4.711771in}}%
\pgfpathlineto{\pgfqpoint{5.387024in}{4.698933in}}%
\pgfpathlineto{\pgfqpoint{5.388098in}{4.665598in}}%
\pgfpathlineto{\pgfqpoint{5.391320in}{4.656814in}}%
\pgfpathlineto{\pgfqpoint{5.392394in}{4.670103in}}%
\pgfpathlineto{\pgfqpoint{5.394541in}{4.644651in}}%
\pgfpathlineto{\pgfqpoint{5.395615in}{4.655237in}}%
\pgfpathlineto{\pgfqpoint{5.398837in}{4.659066in}}%
\pgfpathlineto{\pgfqpoint{5.399911in}{4.679112in}}%
\pgfpathlineto{\pgfqpoint{5.400984in}{4.672580in}}%
\pgfpathlineto{\pgfqpoint{5.403132in}{4.690374in}}%
\pgfpathlineto{\pgfqpoint{5.406354in}{4.690824in}}%
\pgfpathlineto{\pgfqpoint{5.407427in}{4.701410in}}%
\pgfpathlineto{\pgfqpoint{5.408501in}{4.704789in}}%
\pgfpathlineto{\pgfqpoint{5.409575in}{4.678662in}}%
\pgfpathlineto{\pgfqpoint{5.410649in}{4.706591in}}%
\pgfpathlineto{\pgfqpoint{5.413870in}{4.712447in}}%
\pgfpathlineto{\pgfqpoint{5.414944in}{4.716952in}}%
\pgfpathlineto{\pgfqpoint{5.416018in}{4.706140in}}%
\pgfpathlineto{\pgfqpoint{5.417092in}{4.707266in}}%
\pgfpathlineto{\pgfqpoint{5.418166in}{4.702536in}}%
\pgfpathlineto{\pgfqpoint{5.421387in}{4.696230in}}%
\pgfpathlineto{\pgfqpoint{5.423535in}{4.698482in}}%
\pgfpathlineto{\pgfqpoint{5.425683in}{4.709744in}}%
\pgfpathlineto{\pgfqpoint{5.428904in}{4.715600in}}%
\pgfpathlineto{\pgfqpoint{5.429978in}{4.724835in}}%
\pgfpathlineto{\pgfqpoint{5.431052in}{4.717627in}}%
\pgfpathlineto{\pgfqpoint{5.432126in}{4.749160in}}%
\pgfpathlineto{\pgfqpoint{5.433200in}{4.742178in}}%
\pgfpathlineto{\pgfqpoint{5.436421in}{4.755016in}}%
\pgfpathlineto{\pgfqpoint{5.437495in}{4.756368in}}%
\pgfpathlineto{\pgfqpoint{5.438569in}{4.767855in}}%
\pgfpathlineto{\pgfqpoint{5.440716in}{4.774837in}}%
\pgfpathlineto{\pgfqpoint{5.443938in}{4.773260in}}%
\pgfpathlineto{\pgfqpoint{5.445012in}{4.781594in}}%
\pgfpathlineto{\pgfqpoint{5.446086in}{4.778441in}}%
\pgfpathlineto{\pgfqpoint{5.447159in}{4.778891in}}%
\pgfpathlineto{\pgfqpoint{5.448233in}{4.783171in}}%
\pgfpathlineto{\pgfqpoint{5.451455in}{4.773711in}}%
\pgfpathlineto{\pgfqpoint{5.453602in}{4.758395in}}%
\pgfpathlineto{\pgfqpoint{5.454676in}{4.764026in}}%
\pgfpathlineto{\pgfqpoint{5.455750in}{4.754791in}}%
\pgfpathlineto{\pgfqpoint{5.458972in}{4.748484in}}%
\pgfpathlineto{\pgfqpoint{5.460046in}{4.741502in}}%
\pgfpathlineto{\pgfqpoint{5.462193in}{4.770332in}}%
\pgfpathlineto{\pgfqpoint{5.463267in}{4.757494in}}%
\pgfpathlineto{\pgfqpoint{5.467562in}{4.778441in}}%
\pgfpathlineto{\pgfqpoint{5.468636in}{4.778441in}}%
\pgfpathlineto{\pgfqpoint{5.470784in}{4.780693in}}%
\pgfpathlineto{\pgfqpoint{5.474005in}{4.773485in}}%
\pgfpathlineto{\pgfqpoint{5.476153in}{4.755917in}}%
\pgfpathlineto{\pgfqpoint{5.478301in}{4.757269in}}%
\pgfpathlineto{\pgfqpoint{5.481522in}{4.749160in}}%
\pgfpathlineto{\pgfqpoint{5.482596in}{4.735871in}}%
\pgfpathlineto{\pgfqpoint{5.484744in}{4.763800in}}%
\pgfpathlineto{\pgfqpoint{5.485818in}{4.765152in}}%
\pgfpathlineto{\pgfqpoint{5.489039in}{4.761998in}}%
\pgfpathlineto{\pgfqpoint{5.491187in}{4.757043in}}%
\pgfpathlineto{\pgfqpoint{5.492261in}{4.753890in}}%
\pgfpathlineto{\pgfqpoint{5.493334in}{4.759070in}}%
\pgfpathlineto{\pgfqpoint{5.497630in}{4.751863in}}%
\pgfpathlineto{\pgfqpoint{5.499778in}{4.768080in}}%
\pgfpathlineto{\pgfqpoint{5.500851in}{4.759070in}}%
\pgfpathlineto{\pgfqpoint{5.504073in}{4.746457in}}%
\pgfpathlineto{\pgfqpoint{5.505147in}{4.714474in}}%
\pgfpathlineto{\pgfqpoint{5.506221in}{4.706365in}}%
\pgfpathlineto{\pgfqpoint{5.507294in}{4.714924in}}%
\pgfpathlineto{\pgfqpoint{5.508368in}{4.691950in}}%
\pgfpathlineto{\pgfqpoint{5.511590in}{4.703888in}}%
\pgfpathlineto{\pgfqpoint{5.512664in}{4.704789in}}%
\pgfpathlineto{\pgfqpoint{5.513737in}{4.707717in}}%
\pgfpathlineto{\pgfqpoint{5.514811in}{4.714474in}}%
\pgfpathlineto{\pgfqpoint{5.515885in}{4.701636in}}%
\pgfpathlineto{\pgfqpoint{5.519107in}{4.694428in}}%
\pgfpathlineto{\pgfqpoint{5.520180in}{4.709519in}}%
\pgfpathlineto{\pgfqpoint{5.521254in}{4.706591in}}%
\pgfpathlineto{\pgfqpoint{5.522328in}{4.718078in}}%
\pgfpathlineto{\pgfqpoint{5.523402in}{4.722808in}}%
\pgfpathlineto{\pgfqpoint{5.527697in}{4.730240in}}%
\pgfpathlineto{\pgfqpoint{5.528771in}{4.721006in}}%
\pgfpathlineto{\pgfqpoint{5.529845in}{4.719654in}}%
\pgfpathlineto{\pgfqpoint{5.530919in}{4.723709in}}%
\pgfpathlineto{\pgfqpoint{5.534140in}{4.710870in}}%
\pgfpathlineto{\pgfqpoint{5.535214in}{4.723709in}}%
\pgfpathlineto{\pgfqpoint{5.538436in}{4.700735in}}%
\pgfpathlineto{\pgfqpoint{5.541657in}{4.716726in}}%
\pgfpathlineto{\pgfqpoint{5.543805in}{4.717852in}}%
\pgfpathlineto{\pgfqpoint{5.545953in}{4.699608in}}%
\pgfpathlineto{\pgfqpoint{5.549174in}{4.690149in}}%
\pgfpathlineto{\pgfqpoint{5.550248in}{4.669877in}}%
\pgfpathlineto{\pgfqpoint{5.551322in}{4.682491in}}%
\pgfpathlineto{\pgfqpoint{5.552396in}{4.652084in}}%
\pgfpathlineto{\pgfqpoint{5.553469in}{4.654561in}}%
\pgfpathlineto{\pgfqpoint{5.556691in}{4.652760in}}%
\pgfpathlineto{\pgfqpoint{5.557765in}{4.648030in}}%
\pgfpathlineto{\pgfqpoint{5.558839in}{4.653886in}}%
\pgfpathlineto{\pgfqpoint{5.559912in}{4.650958in}}%
\pgfpathlineto{\pgfqpoint{5.560986in}{4.662219in}}%
\pgfpathlineto{\pgfqpoint{5.564208in}{4.659967in}}%
\pgfpathlineto{\pgfqpoint{5.565282in}{4.651859in}}%
\pgfpathlineto{\pgfqpoint{5.566356in}{4.634065in}}%
\pgfpathlineto{\pgfqpoint{5.567429in}{4.637894in}}%
\pgfpathlineto{\pgfqpoint{5.568503in}{4.675959in}}%
\pgfpathlineto{\pgfqpoint{5.572799in}{4.661319in}}%
\pgfpathlineto{\pgfqpoint{5.573872in}{4.652309in}}%
\pgfpathlineto{\pgfqpoint{5.574946in}{4.652309in}}%
\pgfpathlineto{\pgfqpoint{5.579242in}{4.656814in}}%
\pgfpathlineto{\pgfqpoint{5.580315in}{4.661319in}}%
\pgfpathlineto{\pgfqpoint{5.581389in}{4.662219in}}%
\pgfpathlineto{\pgfqpoint{5.582463in}{4.660868in}}%
\pgfpathlineto{\pgfqpoint{5.583537in}{4.674607in}}%
\pgfpathlineto{\pgfqpoint{5.586758in}{4.670553in}}%
\pgfpathlineto{\pgfqpoint{5.587832in}{4.665823in}}%
\pgfpathlineto{\pgfqpoint{5.588906in}{4.692401in}}%
\pgfpathlineto{\pgfqpoint{5.589980in}{4.693077in}}%
\pgfpathlineto{\pgfqpoint{5.591054in}{4.685193in}}%
\pgfpathlineto{\pgfqpoint{5.594275in}{4.690374in}}%
\pgfpathlineto{\pgfqpoint{5.595349in}{4.684518in}}%
\pgfpathlineto{\pgfqpoint{5.596423in}{4.689698in}}%
\pgfpathlineto{\pgfqpoint{5.598571in}{4.677535in}}%
\pgfpathlineto{\pgfqpoint{5.601792in}{4.685869in}}%
\pgfpathlineto{\pgfqpoint{5.602866in}{4.696230in}}%
\pgfpathlineto{\pgfqpoint{5.603940in}{4.693527in}}%
\pgfpathlineto{\pgfqpoint{5.605014in}{4.686770in}}%
\pgfpathlineto{\pgfqpoint{5.606088in}{4.704113in}}%
\pgfpathlineto{\pgfqpoint{5.609309in}{4.704338in}}%
\pgfpathlineto{\pgfqpoint{5.611457in}{4.684968in}}%
\pgfpathlineto{\pgfqpoint{5.612531in}{4.685193in}}%
\pgfpathlineto{\pgfqpoint{5.613604in}{4.696455in}}%
\pgfpathlineto{\pgfqpoint{5.616826in}{4.694203in}}%
\pgfpathlineto{\pgfqpoint{5.617900in}{4.685419in}}%
\pgfpathlineto{\pgfqpoint{5.620047in}{4.699834in}}%
\pgfpathlineto{\pgfqpoint{5.621121in}{4.700284in}}%
\pgfpathlineto{\pgfqpoint{5.624343in}{4.708618in}}%
\pgfpathlineto{\pgfqpoint{5.625417in}{4.703437in}}%
\pgfpathlineto{\pgfqpoint{5.627564in}{4.711546in}}%
\pgfpathlineto{\pgfqpoint{5.632934in}{4.702536in}}%
\pgfpathlineto{\pgfqpoint{5.635081in}{4.720555in}}%
\pgfpathlineto{\pgfqpoint{5.636155in}{4.729565in}}%
\pgfpathlineto{\pgfqpoint{5.639377in}{4.718753in}}%
\pgfpathlineto{\pgfqpoint{5.642598in}{4.686770in}}%
\pgfpathlineto{\pgfqpoint{5.643672in}{4.677310in}}%
\pgfpathlineto{\pgfqpoint{5.646893in}{4.666499in}}%
\pgfpathlineto{\pgfqpoint{5.647967in}{4.665598in}}%
\pgfpathlineto{\pgfqpoint{5.649041in}{4.676860in}}%
\pgfpathlineto{\pgfqpoint{5.650115in}{4.677535in}}%
\pgfpathlineto{\pgfqpoint{5.651189in}{4.666949in}}%
\pgfpathlineto{\pgfqpoint{5.654410in}{4.668301in}}%
\pgfpathlineto{\pgfqpoint{5.656558in}{4.679563in}}%
\pgfpathlineto{\pgfqpoint{5.657632in}{4.688121in}}%
\pgfpathlineto{\pgfqpoint{5.658706in}{4.681815in}}%
\pgfpathlineto{\pgfqpoint{5.661927in}{4.685644in}}%
\pgfpathlineto{\pgfqpoint{5.664075in}{4.678662in}}%
\pgfpathlineto{\pgfqpoint{5.665149in}{4.680238in}}%
\pgfpathlineto{\pgfqpoint{5.666222in}{4.660868in}}%
\pgfpathlineto{\pgfqpoint{5.669444in}{4.648255in}}%
\pgfpathlineto{\pgfqpoint{5.670518in}{4.648931in}}%
\pgfpathlineto{\pgfqpoint{5.671592in}{4.644201in}}%
\pgfpathlineto{\pgfqpoint{5.672666in}{4.651633in}}%
\pgfpathlineto{\pgfqpoint{5.678035in}{4.638570in}}%
\pgfpathlineto{\pgfqpoint{5.680182in}{4.619200in}}%
\pgfpathlineto{\pgfqpoint{5.681256in}{4.623930in}}%
\pgfpathlineto{\pgfqpoint{5.684478in}{4.635191in}}%
\pgfpathlineto{\pgfqpoint{5.685552in}{4.633615in}}%
\pgfpathlineto{\pgfqpoint{5.686625in}{4.634290in}}%
\pgfpathlineto{\pgfqpoint{5.687699in}{4.638570in}}%
\pgfpathlineto{\pgfqpoint{5.688773in}{4.630011in}}%
\pgfpathlineto{\pgfqpoint{5.691995in}{4.622578in}}%
\pgfpathlineto{\pgfqpoint{5.693068in}{4.614920in}}%
\pgfpathlineto{\pgfqpoint{5.694142in}{4.612668in}}%
\pgfpathlineto{\pgfqpoint{5.695216in}{4.612443in}}%
\pgfpathlineto{\pgfqpoint{5.696290in}{4.601857in}}%
\pgfpathlineto{\pgfqpoint{5.699511in}{4.607487in}}%
\pgfpathlineto{\pgfqpoint{5.700585in}{4.619650in}}%
\pgfpathlineto{\pgfqpoint{5.701659in}{4.620776in}}%
\pgfpathlineto{\pgfqpoint{5.702733in}{4.618749in}}%
\pgfpathlineto{\pgfqpoint{5.707028in}{4.621452in}}%
\pgfpathlineto{\pgfqpoint{5.708102in}{4.623029in}}%
\pgfpathlineto{\pgfqpoint{5.709176in}{4.627759in}}%
\pgfpathlineto{\pgfqpoint{5.711324in}{4.623029in}}%
\pgfpathlineto{\pgfqpoint{5.714545in}{4.638345in}}%
\pgfpathlineto{\pgfqpoint{5.715619in}{4.624830in}}%
\pgfpathlineto{\pgfqpoint{5.716693in}{4.634516in}}%
\pgfpathlineto{\pgfqpoint{5.717767in}{4.622803in}}%
\pgfpathlineto{\pgfqpoint{5.718841in}{4.625731in}}%
\pgfpathlineto{\pgfqpoint{5.722062in}{4.626858in}}%
\pgfpathlineto{\pgfqpoint{5.723136in}{4.623479in}}%
\pgfpathlineto{\pgfqpoint{5.725284in}{4.595775in}}%
\pgfpathlineto{\pgfqpoint{5.726357in}{4.576180in}}%
\pgfpathlineto{\pgfqpoint{5.729579in}{4.569873in}}%
\pgfpathlineto{\pgfqpoint{5.730653in}{4.562215in}}%
\pgfpathlineto{\pgfqpoint{5.731727in}{4.591045in}}%
\pgfpathlineto{\pgfqpoint{5.732800in}{4.599604in}}%
\pgfpathlineto{\pgfqpoint{5.733874in}{4.613569in}}%
\pgfpathlineto{\pgfqpoint{5.737096in}{4.616046in}}%
\pgfpathlineto{\pgfqpoint{5.738170in}{4.601631in}}%
\pgfpathlineto{\pgfqpoint{5.740317in}{4.626858in}}%
\pgfpathlineto{\pgfqpoint{5.741391in}{4.615596in}}%
\pgfpathlineto{\pgfqpoint{5.745687in}{4.635417in}}%
\pgfpathlineto{\pgfqpoint{5.746760in}{4.630236in}}%
\pgfpathlineto{\pgfqpoint{5.747834in}{4.630687in}}%
\pgfpathlineto{\pgfqpoint{5.748908in}{4.634741in}}%
\pgfpathlineto{\pgfqpoint{5.752130in}{4.633164in}}%
\pgfpathlineto{\pgfqpoint{5.753203in}{4.640146in}}%
\pgfpathlineto{\pgfqpoint{5.754277in}{4.640822in}}%
\pgfpathlineto{\pgfqpoint{5.755351in}{4.639696in}}%
\pgfpathlineto{\pgfqpoint{5.756425in}{4.625731in}}%
\pgfpathlineto{\pgfqpoint{5.759646in}{4.628659in}}%
\pgfpathlineto{\pgfqpoint{5.760720in}{4.618749in}}%
\pgfpathlineto{\pgfqpoint{5.761794in}{4.620101in}}%
\pgfpathlineto{\pgfqpoint{5.762868in}{4.614920in}}%
\pgfpathlineto{\pgfqpoint{5.763942in}{4.621452in}}%
\pgfpathlineto{\pgfqpoint{5.767163in}{4.620551in}}%
\pgfpathlineto{\pgfqpoint{5.768237in}{4.630236in}}%
\pgfpathlineto{\pgfqpoint{5.769311in}{4.648255in}}%
\pgfpathlineto{\pgfqpoint{5.770385in}{4.645552in}}%
\pgfpathlineto{\pgfqpoint{5.771459in}{4.655688in}}%
\pgfpathlineto{\pgfqpoint{5.774680in}{4.669877in}}%
\pgfpathlineto{\pgfqpoint{5.777902in}{4.696230in}}%
\pgfpathlineto{\pgfqpoint{5.778976in}{4.688572in}}%
\pgfpathlineto{\pgfqpoint{5.782197in}{4.689923in}}%
\pgfpathlineto{\pgfqpoint{5.783271in}{4.686545in}}%
\pgfpathlineto{\pgfqpoint{5.784345in}{4.701861in}}%
\pgfpathlineto{\pgfqpoint{5.785419in}{4.700960in}}%
\pgfpathlineto{\pgfqpoint{5.786492in}{4.706816in}}%
\pgfpathlineto{\pgfqpoint{5.789714in}{4.717852in}}%
\pgfpathlineto{\pgfqpoint{5.790788in}{4.714699in}}%
\pgfpathlineto{\pgfqpoint{5.791862in}{4.713348in}}%
\pgfpathlineto{\pgfqpoint{5.794009in}{4.744205in}}%
\pgfpathlineto{\pgfqpoint{5.798305in}{4.735646in}}%
\pgfpathlineto{\pgfqpoint{5.799378in}{4.740376in}}%
\pgfpathlineto{\pgfqpoint{5.800452in}{4.726637in}}%
\pgfpathlineto{\pgfqpoint{5.801526in}{4.723258in}}%
\pgfpathlineto{\pgfqpoint{5.804748in}{4.728213in}}%
\pgfpathlineto{\pgfqpoint{5.805822in}{4.732267in}}%
\pgfpathlineto{\pgfqpoint{5.806895in}{4.734069in}}%
\pgfpathlineto{\pgfqpoint{5.813338in}{4.714924in}}%
\pgfpathlineto{\pgfqpoint{5.816560in}{4.692626in}}%
\pgfpathlineto{\pgfqpoint{5.819781in}{4.692401in}}%
\pgfpathlineto{\pgfqpoint{5.821929in}{4.713798in}}%
\pgfpathlineto{\pgfqpoint{5.823003in}{4.736997in}}%
\pgfpathlineto{\pgfqpoint{5.824077in}{4.744205in}}%
\pgfpathlineto{\pgfqpoint{5.827298in}{4.740601in}}%
\pgfpathlineto{\pgfqpoint{5.828372in}{4.738124in}}%
\pgfpathlineto{\pgfqpoint{5.829446in}{4.740151in}}%
\pgfpathlineto{\pgfqpoint{5.831594in}{4.740151in}}%
\pgfpathlineto{\pgfqpoint{5.834815in}{4.746457in}}%
\pgfpathlineto{\pgfqpoint{5.835889in}{4.752989in}}%
\pgfpathlineto{\pgfqpoint{5.836963in}{4.747809in}}%
\pgfpathlineto{\pgfqpoint{5.838037in}{4.731817in}}%
\pgfpathlineto{\pgfqpoint{5.839111in}{4.750061in}}%
\pgfpathlineto{\pgfqpoint{5.842332in}{4.750962in}}%
\pgfpathlineto{\pgfqpoint{5.843406in}{4.746232in}}%
\pgfpathlineto{\pgfqpoint{5.844480in}{4.747358in}}%
\pgfpathlineto{\pgfqpoint{5.845554in}{4.746457in}}%
\pgfpathlineto{\pgfqpoint{5.846627in}{4.736322in}}%
\pgfpathlineto{\pgfqpoint{5.849849in}{4.740376in}}%
\pgfpathlineto{\pgfqpoint{5.850923in}{4.754791in}}%
\pgfpathlineto{\pgfqpoint{5.851997in}{4.757269in}}%
\pgfpathlineto{\pgfqpoint{5.853070in}{4.749385in}}%
\pgfpathlineto{\pgfqpoint{5.854144in}{4.728213in}}%
\pgfpathlineto{\pgfqpoint{5.857366in}{4.735871in}}%
\pgfpathlineto{\pgfqpoint{5.859513in}{4.751187in}}%
\pgfpathlineto{\pgfqpoint{5.860587in}{4.750737in}}%
\pgfpathlineto{\pgfqpoint{5.864883in}{4.749836in}}%
\pgfpathlineto{\pgfqpoint{5.865956in}{4.760197in}}%
\pgfpathlineto{\pgfqpoint{5.868104in}{4.740151in}}%
\pgfpathlineto{\pgfqpoint{5.873473in}{4.727538in}}%
\pgfpathlineto{\pgfqpoint{5.874547in}{4.712222in}}%
\pgfpathlineto{\pgfqpoint{5.875621in}{4.686995in}}%
\pgfpathlineto{\pgfqpoint{5.876695in}{4.680463in}}%
\pgfpathlineto{\pgfqpoint{5.879916in}{4.691500in}}%
\pgfpathlineto{\pgfqpoint{5.880990in}{4.704113in}}%
\pgfpathlineto{\pgfqpoint{5.882064in}{4.688572in}}%
\pgfpathlineto{\pgfqpoint{5.883138in}{4.705239in}}%
\pgfpathlineto{\pgfqpoint{5.884212in}{4.644876in}}%
\pgfpathlineto{\pgfqpoint{5.888507in}{4.645777in}}%
\pgfpathlineto{\pgfqpoint{5.889581in}{4.641498in}}%
\pgfpathlineto{\pgfqpoint{5.890655in}{4.642849in}}%
\pgfpathlineto{\pgfqpoint{5.891729in}{4.648255in}}%
\pgfpathlineto{\pgfqpoint{5.894950in}{4.641723in}}%
\pgfpathlineto{\pgfqpoint{5.896024in}{4.648480in}}%
\pgfpathlineto{\pgfqpoint{5.897098in}{4.646003in}}%
\pgfpathlineto{\pgfqpoint{5.898172in}{4.649156in}}%
\pgfpathlineto{\pgfqpoint{5.899245in}{4.670553in}}%
\pgfpathlineto{\pgfqpoint{5.902467in}{4.666499in}}%
\pgfpathlineto{\pgfqpoint{5.903541in}{4.645777in}}%
\pgfpathlineto{\pgfqpoint{5.904615in}{4.641498in}}%
\pgfpathlineto{\pgfqpoint{5.905688in}{4.650507in}}%
\pgfpathlineto{\pgfqpoint{5.906762in}{4.635417in}}%
\pgfpathlineto{\pgfqpoint{5.909984in}{4.630912in}}%
\pgfpathlineto{\pgfqpoint{5.911058in}{4.630687in}}%
\pgfpathlineto{\pgfqpoint{5.912132in}{4.618974in}}%
\pgfpathlineto{\pgfqpoint{5.913205in}{4.618749in}}%
\pgfpathlineto{\pgfqpoint{5.914279in}{4.627308in}}%
\pgfpathlineto{\pgfqpoint{5.918575in}{4.630236in}}%
\pgfpathlineto{\pgfqpoint{5.919648in}{4.644201in}}%
\pgfpathlineto{\pgfqpoint{5.920722in}{4.643300in}}%
\pgfpathlineto{\pgfqpoint{5.921796in}{4.628659in}}%
\pgfpathlineto{\pgfqpoint{5.925018in}{4.641723in}}%
\pgfpathlineto{\pgfqpoint{5.926091in}{4.630461in}}%
\pgfpathlineto{\pgfqpoint{5.929313in}{4.650958in}}%
\pgfpathlineto{\pgfqpoint{5.932534in}{4.646678in}}%
\pgfpathlineto{\pgfqpoint{5.933608in}{4.662670in}}%
\pgfpathlineto{\pgfqpoint{5.934682in}{4.666274in}}%
\pgfpathlineto{\pgfqpoint{5.936830in}{4.668076in}}%
\pgfpathlineto{\pgfqpoint{5.940051in}{4.674382in}}%
\pgfpathlineto{\pgfqpoint{5.941125in}{4.666499in}}%
\pgfpathlineto{\pgfqpoint{5.943273in}{4.680689in}}%
\pgfpathlineto{\pgfqpoint{5.944347in}{4.691050in}}%
\pgfpathlineto{\pgfqpoint{5.947568in}{4.684292in}}%
\pgfpathlineto{\pgfqpoint{5.948642in}{4.688797in}}%
\pgfpathlineto{\pgfqpoint{5.949716in}{4.689698in}}%
\pgfpathlineto{\pgfqpoint{5.950790in}{4.695554in}}%
\pgfpathlineto{\pgfqpoint{5.951864in}{4.709969in}}%
\pgfpathlineto{\pgfqpoint{5.955085in}{4.702987in}}%
\pgfpathlineto{\pgfqpoint{5.956159in}{4.702536in}}%
\pgfpathlineto{\pgfqpoint{5.957233in}{4.696005in}}%
\pgfpathlineto{\pgfqpoint{5.958307in}{4.693527in}}%
\pgfpathlineto{\pgfqpoint{5.962602in}{4.693978in}}%
\pgfpathlineto{\pgfqpoint{5.964750in}{4.710645in}}%
\pgfpathlineto{\pgfqpoint{5.965823in}{4.703212in}}%
\pgfpathlineto{\pgfqpoint{5.966897in}{4.705239in}}%
\pgfpathlineto{\pgfqpoint{5.971193in}{4.693978in}}%
\pgfpathlineto{\pgfqpoint{5.972266in}{4.697581in}}%
\pgfpathlineto{\pgfqpoint{5.973340in}{4.686770in}}%
\pgfpathlineto{\pgfqpoint{5.974414in}{4.688572in}}%
\pgfpathlineto{\pgfqpoint{5.977636in}{4.689248in}}%
\pgfpathlineto{\pgfqpoint{5.979783in}{4.698708in}}%
\pgfpathlineto{\pgfqpoint{5.981931in}{4.684968in}}%
\pgfpathlineto{\pgfqpoint{5.985153in}{4.688797in}}%
\pgfpathlineto{\pgfqpoint{5.986226in}{4.687896in}}%
\pgfpathlineto{\pgfqpoint{5.987300in}{4.696005in}}%
\pgfpathlineto{\pgfqpoint{5.988374in}{4.695329in}}%
\pgfpathlineto{\pgfqpoint{5.989448in}{4.688572in}}%
\pgfpathlineto{\pgfqpoint{5.992669in}{4.683617in}}%
\pgfpathlineto{\pgfqpoint{5.993743in}{4.683842in}}%
\pgfpathlineto{\pgfqpoint{5.994817in}{4.690824in}}%
\pgfpathlineto{\pgfqpoint{5.996965in}{4.660868in}}%
\pgfpathlineto{\pgfqpoint{6.000186in}{4.667625in}}%
\pgfpathlineto{\pgfqpoint{6.002334in}{4.657264in}}%
\pgfpathlineto{\pgfqpoint{6.003408in}{4.658390in}}%
\pgfpathlineto{\pgfqpoint{6.004482in}{4.661319in}}%
\pgfpathlineto{\pgfqpoint{6.007703in}{4.656363in}}%
\pgfpathlineto{\pgfqpoint{6.008777in}{4.663346in}}%
\pgfpathlineto{\pgfqpoint{6.009851in}{4.661544in}}%
\pgfpathlineto{\pgfqpoint{6.010925in}{4.655462in}}%
\pgfpathlineto{\pgfqpoint{6.015220in}{4.668526in}}%
\pgfpathlineto{\pgfqpoint{6.016294in}{4.659967in}}%
\pgfpathlineto{\pgfqpoint{6.017368in}{4.660192in}}%
\pgfpathlineto{\pgfqpoint{6.018442in}{4.652760in}}%
\pgfpathlineto{\pgfqpoint{6.019515in}{4.663571in}}%
\pgfpathlineto{\pgfqpoint{6.022737in}{4.665148in}}%
\pgfpathlineto{\pgfqpoint{6.023811in}{4.682265in}}%
\pgfpathlineto{\pgfqpoint{6.024885in}{4.689022in}}%
\pgfpathlineto{\pgfqpoint{6.027032in}{4.692851in}}%
\pgfpathlineto{\pgfqpoint{6.031328in}{4.693302in}}%
\pgfpathlineto{\pgfqpoint{6.033475in}{4.696680in}}%
\pgfpathlineto{\pgfqpoint{6.034549in}{4.693752in}}%
\pgfpathlineto{\pgfqpoint{6.037771in}{4.695104in}}%
\pgfpathlineto{\pgfqpoint{6.038844in}{4.699158in}}%
\pgfpathlineto{\pgfqpoint{6.039918in}{4.699383in}}%
\pgfpathlineto{\pgfqpoint{6.040992in}{4.700509in}}%
\pgfpathlineto{\pgfqpoint{6.042066in}{4.702536in}}%
\pgfpathlineto{\pgfqpoint{6.045287in}{4.705239in}}%
\pgfpathlineto{\pgfqpoint{6.046361in}{4.704564in}}%
\pgfpathlineto{\pgfqpoint{6.047435in}{4.693752in}}%
\pgfpathlineto{\pgfqpoint{6.049583in}{4.696680in}}%
\pgfpathlineto{\pgfqpoint{6.053878in}{4.708393in}}%
\pgfpathlineto{\pgfqpoint{6.054952in}{4.707717in}}%
\pgfpathlineto{\pgfqpoint{6.056026in}{4.722132in}}%
\pgfpathlineto{\pgfqpoint{6.057100in}{4.692401in}}%
\pgfpathlineto{\pgfqpoint{6.060321in}{4.675283in}}%
\pgfpathlineto{\pgfqpoint{6.061395in}{4.684968in}}%
\pgfpathlineto{\pgfqpoint{6.063543in}{4.718303in}}%
\pgfpathlineto{\pgfqpoint{6.064617in}{4.717177in}}%
\pgfpathlineto{\pgfqpoint{6.068912in}{4.715825in}}%
\pgfpathlineto{\pgfqpoint{6.071060in}{4.726411in}}%
\pgfpathlineto{\pgfqpoint{6.072133in}{4.743079in}}%
\pgfpathlineto{\pgfqpoint{6.075355in}{4.750737in}}%
\pgfpathlineto{\pgfqpoint{6.076429in}{4.762449in}}%
\pgfpathlineto{\pgfqpoint{6.077503in}{4.763800in}}%
\pgfpathlineto{\pgfqpoint{6.078576in}{4.767855in}}%
\pgfpathlineto{\pgfqpoint{6.079650in}{4.765152in}}%
\pgfpathlineto{\pgfqpoint{6.082872in}{4.764701in}}%
\pgfpathlineto{\pgfqpoint{6.083946in}{4.766728in}}%
\pgfpathlineto{\pgfqpoint{6.085020in}{4.777990in}}%
\pgfpathlineto{\pgfqpoint{6.086093in}{4.748484in}}%
\pgfpathlineto{\pgfqpoint{6.087167in}{4.756593in}}%
\pgfpathlineto{\pgfqpoint{6.090389in}{4.757269in}}%
\pgfpathlineto{\pgfqpoint{6.091463in}{4.765602in}}%
\pgfpathlineto{\pgfqpoint{6.092536in}{4.760197in}}%
\pgfpathlineto{\pgfqpoint{6.093610in}{4.758845in}}%
\pgfpathlineto{\pgfqpoint{6.094684in}{4.760647in}}%
\pgfpathlineto{\pgfqpoint{6.097906in}{4.760647in}}%
\pgfpathlineto{\pgfqpoint{6.098979in}{4.754566in}}%
\pgfpathlineto{\pgfqpoint{6.100053in}{4.753440in}}%
\pgfpathlineto{\pgfqpoint{6.102201in}{4.768756in}}%
\pgfpathlineto{\pgfqpoint{6.105422in}{4.769882in}}%
\pgfpathlineto{\pgfqpoint{6.106496in}{4.767404in}}%
\pgfpathlineto{\pgfqpoint{6.107570in}{4.759296in}}%
\pgfpathlineto{\pgfqpoint{6.108644in}{4.762449in}}%
\pgfpathlineto{\pgfqpoint{6.109718in}{4.760197in}}%
\pgfpathlineto{\pgfqpoint{6.112939in}{4.767179in}}%
\pgfpathlineto{\pgfqpoint{6.114013in}{4.773485in}}%
\pgfpathlineto{\pgfqpoint{6.115087in}{4.769431in}}%
\pgfpathlineto{\pgfqpoint{6.116161in}{4.768530in}}%
\pgfpathlineto{\pgfqpoint{6.117235in}{4.774161in}}%
\pgfpathlineto{\pgfqpoint{6.121530in}{4.777540in}}%
\pgfpathlineto{\pgfqpoint{6.122604in}{4.772359in}}%
\pgfpathlineto{\pgfqpoint{6.123678in}{4.771008in}}%
\pgfpathlineto{\pgfqpoint{6.124752in}{4.774612in}}%
\pgfpathlineto{\pgfqpoint{6.127973in}{4.780468in}}%
\pgfpathlineto{\pgfqpoint{6.132268in}{4.791504in}}%
\pgfpathlineto{\pgfqpoint{6.136564in}{4.801865in}}%
\pgfpathlineto{\pgfqpoint{6.137638in}{4.799387in}}%
\pgfpathlineto{\pgfqpoint{6.138711in}{4.799162in}}%
\pgfpathlineto{\pgfqpoint{6.139785in}{4.778215in}}%
\pgfpathlineto{\pgfqpoint{6.143007in}{4.791504in}}%
\pgfpathlineto{\pgfqpoint{6.144081in}{4.781819in}}%
\pgfpathlineto{\pgfqpoint{6.145154in}{4.782044in}}%
\pgfpathlineto{\pgfqpoint{6.147302in}{4.824614in}}%
\pgfpathlineto{\pgfqpoint{6.150524in}{4.814028in}}%
\pgfpathlineto{\pgfqpoint{6.151598in}{4.813577in}}%
\pgfpathlineto{\pgfqpoint{6.152671in}{4.820109in}}%
\pgfpathlineto{\pgfqpoint{6.153745in}{4.822136in}}%
\pgfpathlineto{\pgfqpoint{6.154819in}{4.814703in}}%
\pgfpathlineto{\pgfqpoint{6.158041in}{4.803442in}}%
\pgfpathlineto{\pgfqpoint{6.160188in}{4.819884in}}%
\pgfpathlineto{\pgfqpoint{6.161262in}{4.817406in}}%
\pgfpathlineto{\pgfqpoint{6.162336in}{4.826416in}}%
\pgfpathlineto{\pgfqpoint{6.165557in}{4.824389in}}%
\pgfpathlineto{\pgfqpoint{6.166631in}{4.821911in}}%
\pgfpathlineto{\pgfqpoint{6.167705in}{4.831371in}}%
\pgfpathlineto{\pgfqpoint{6.169853in}{4.833623in}}%
\pgfpathlineto{\pgfqpoint{6.173074in}{4.832047in}}%
\pgfpathlineto{\pgfqpoint{6.174148in}{4.816280in}}%
\pgfpathlineto{\pgfqpoint{6.176296in}{4.810199in}}%
\pgfpathlineto{\pgfqpoint{6.177370in}{4.820109in}}%
\pgfpathlineto{\pgfqpoint{6.180591in}{4.816731in}}%
\pgfpathlineto{\pgfqpoint{6.181665in}{4.826416in}}%
\pgfpathlineto{\pgfqpoint{6.182739in}{4.779792in}}%
\pgfpathlineto{\pgfqpoint{6.183813in}{4.777990in}}%
\pgfpathlineto{\pgfqpoint{6.184887in}{4.772359in}}%
\pgfpathlineto{\pgfqpoint{6.188108in}{4.774612in}}%
\pgfpathlineto{\pgfqpoint{6.191330in}{4.765152in}}%
\pgfpathlineto{\pgfqpoint{6.192403in}{4.763800in}}%
\pgfpathlineto{\pgfqpoint{6.195625in}{4.766503in}}%
\pgfpathlineto{\pgfqpoint{6.196699in}{4.759070in}}%
\pgfpathlineto{\pgfqpoint{6.197773in}{4.760872in}}%
\pgfpathlineto{\pgfqpoint{6.199920in}{4.745556in}}%
\pgfpathlineto{\pgfqpoint{6.203142in}{4.768080in}}%
\pgfpathlineto{\pgfqpoint{6.204216in}{4.769206in}}%
\pgfpathlineto{\pgfqpoint{6.205289in}{4.769431in}}%
\pgfpathlineto{\pgfqpoint{6.206363in}{4.764251in}}%
\pgfpathlineto{\pgfqpoint{6.207437in}{4.766503in}}%
\pgfpathlineto{\pgfqpoint{6.210659in}{4.763800in}}%
\pgfpathlineto{\pgfqpoint{6.211732in}{4.772810in}}%
\pgfpathlineto{\pgfqpoint{6.212806in}{4.771233in}}%
\pgfpathlineto{\pgfqpoint{6.213880in}{4.775062in}}%
\pgfpathlineto{\pgfqpoint{6.214954in}{4.773485in}}%
\pgfpathlineto{\pgfqpoint{6.218175in}{4.774161in}}%
\pgfpathlineto{\pgfqpoint{6.219249in}{4.784747in}}%
\pgfpathlineto{\pgfqpoint{6.220323in}{4.778891in}}%
\pgfpathlineto{\pgfqpoint{6.222471in}{4.783846in}}%
\pgfpathlineto{\pgfqpoint{6.225692in}{4.785423in}}%
\pgfpathlineto{\pgfqpoint{6.226766in}{4.781143in}}%
\pgfpathlineto{\pgfqpoint{6.227840in}{4.768305in}}%
\pgfpathlineto{\pgfqpoint{6.228914in}{4.748710in}}%
\pgfpathlineto{\pgfqpoint{6.229988in}{4.757043in}}%
\pgfpathlineto{\pgfqpoint{6.233209in}{4.761773in}}%
\pgfpathlineto{\pgfqpoint{6.234283in}{4.768756in}}%
\pgfpathlineto{\pgfqpoint{6.235357in}{4.785198in}}%
\pgfpathlineto{\pgfqpoint{6.236431in}{4.789252in}}%
\pgfpathlineto{\pgfqpoint{6.240726in}{4.794883in}}%
\pgfpathlineto{\pgfqpoint{6.241800in}{4.812451in}}%
\pgfpathlineto{\pgfqpoint{6.242874in}{4.807045in}}%
\pgfpathlineto{\pgfqpoint{6.243948in}{4.812226in}}%
\pgfpathlineto{\pgfqpoint{6.245021in}{4.802090in}}%
\pgfpathlineto{\pgfqpoint{6.248243in}{4.814253in}}%
\pgfpathlineto{\pgfqpoint{6.249317in}{4.821010in}}%
\pgfpathlineto{\pgfqpoint{6.250391in}{4.816055in}}%
\pgfpathlineto{\pgfqpoint{6.251464in}{4.815154in}}%
\pgfpathlineto{\pgfqpoint{6.256834in}{4.818082in}}%
\pgfpathlineto{\pgfqpoint{6.257908in}{4.808847in}}%
\pgfpathlineto{\pgfqpoint{6.258981in}{4.809523in}}%
\pgfpathlineto{\pgfqpoint{6.260055in}{4.801189in}}%
\pgfpathlineto{\pgfqpoint{6.264351in}{4.808172in}}%
\pgfpathlineto{\pgfqpoint{6.265424in}{4.804117in}}%
\pgfpathlineto{\pgfqpoint{6.266498in}{4.802991in}}%
\pgfpathlineto{\pgfqpoint{6.267572in}{4.805694in}}%
\pgfpathlineto{\pgfqpoint{6.270794in}{4.808397in}}%
\pgfpathlineto{\pgfqpoint{6.271867in}{4.806820in}}%
\pgfpathlineto{\pgfqpoint{6.272941in}{4.815604in}}%
\pgfpathlineto{\pgfqpoint{6.274015in}{4.810424in}}%
\pgfpathlineto{\pgfqpoint{6.275089in}{4.812226in}}%
\pgfpathlineto{\pgfqpoint{6.279384in}{4.812451in}}%
\pgfpathlineto{\pgfqpoint{6.280458in}{4.811550in}}%
\pgfpathlineto{\pgfqpoint{6.281532in}{4.807496in}}%
\pgfpathlineto{\pgfqpoint{6.282606in}{4.815379in}}%
\pgfpathlineto{\pgfqpoint{6.285827in}{4.811775in}}%
\pgfpathlineto{\pgfqpoint{6.286901in}{4.829569in}}%
\pgfpathlineto{\pgfqpoint{6.287975in}{4.833398in}}%
\pgfpathlineto{\pgfqpoint{6.289049in}{4.828218in}}%
\pgfpathlineto{\pgfqpoint{6.290123in}{4.837002in}}%
\pgfpathlineto{\pgfqpoint{6.293344in}{4.825289in}}%
\pgfpathlineto{\pgfqpoint{6.295492in}{4.806595in}}%
\pgfpathlineto{\pgfqpoint{6.297640in}{4.812001in}}%
\pgfpathlineto{\pgfqpoint{6.300861in}{4.806595in}}%
\pgfpathlineto{\pgfqpoint{6.301935in}{4.808397in}}%
\pgfpathlineto{\pgfqpoint{6.303009in}{4.809073in}}%
\pgfpathlineto{\pgfqpoint{6.304083in}{4.789477in}}%
\pgfpathlineto{\pgfqpoint{6.305156in}{4.787000in}}%
\pgfpathlineto{\pgfqpoint{6.310526in}{4.802090in}}%
\pgfpathlineto{\pgfqpoint{6.311599in}{4.809748in}}%
\pgfpathlineto{\pgfqpoint{6.312673in}{4.811100in}}%
\pgfpathlineto{\pgfqpoint{6.316969in}{4.812001in}}%
\pgfpathlineto{\pgfqpoint{6.318042in}{4.802541in}}%
\pgfpathlineto{\pgfqpoint{6.319116in}{4.804793in}}%
\pgfpathlineto{\pgfqpoint{6.320190in}{4.812226in}}%
\pgfpathlineto{\pgfqpoint{6.323412in}{4.811775in}}%
\pgfpathlineto{\pgfqpoint{6.325559in}{4.799613in}}%
\pgfpathlineto{\pgfqpoint{6.327707in}{4.798937in}}%
\pgfpathlineto{\pgfqpoint{6.330929in}{4.791955in}}%
\pgfpathlineto{\pgfqpoint{6.332002in}{4.796685in}}%
\pgfpathlineto{\pgfqpoint{6.333076in}{4.792856in}}%
\pgfpathlineto{\pgfqpoint{6.335224in}{4.799162in}}%
\pgfpathlineto{\pgfqpoint{6.338445in}{4.783171in}}%
\pgfpathlineto{\pgfqpoint{6.339519in}{4.783621in}}%
\pgfpathlineto{\pgfqpoint{6.340593in}{4.782044in}}%
\pgfpathlineto{\pgfqpoint{6.341667in}{4.782720in}}%
\pgfpathlineto{\pgfqpoint{6.342741in}{4.785648in}}%
\pgfpathlineto{\pgfqpoint{6.345962in}{4.789027in}}%
\pgfpathlineto{\pgfqpoint{6.347036in}{4.780693in}}%
\pgfpathlineto{\pgfqpoint{6.348110in}{4.787675in}}%
\pgfpathlineto{\pgfqpoint{6.350258in}{4.783171in}}%
\pgfpathlineto{\pgfqpoint{6.353479in}{4.788126in}}%
\pgfpathlineto{\pgfqpoint{6.354553in}{4.792630in}}%
\pgfpathlineto{\pgfqpoint{6.355627in}{4.791955in}}%
\pgfpathlineto{\pgfqpoint{6.356701in}{4.795784in}}%
\pgfpathlineto{\pgfqpoint{6.357775in}{4.802541in}}%
\pgfpathlineto{\pgfqpoint{6.360996in}{4.804343in}}%
\pgfpathlineto{\pgfqpoint{6.362070in}{4.806820in}}%
\pgfpathlineto{\pgfqpoint{6.363144in}{4.805694in}}%
\pgfpathlineto{\pgfqpoint{6.364218in}{4.801640in}}%
\pgfpathlineto{\pgfqpoint{6.365291in}{4.801640in}}%
\pgfpathlineto{\pgfqpoint{6.370661in}{4.793081in}}%
\pgfpathlineto{\pgfqpoint{6.371734in}{4.785198in}}%
\pgfpathlineto{\pgfqpoint{6.376030in}{4.789928in}}%
\pgfpathlineto{\pgfqpoint{6.377104in}{4.796234in}}%
\pgfpathlineto{\pgfqpoint{6.378177in}{4.799162in}}%
\pgfpathlineto{\pgfqpoint{6.380325in}{4.807721in}}%
\pgfpathlineto{\pgfqpoint{6.384620in}{4.819433in}}%
\pgfpathlineto{\pgfqpoint{6.385694in}{4.820560in}}%
\pgfpathlineto{\pgfqpoint{6.386768in}{4.831146in}}%
\pgfpathlineto{\pgfqpoint{6.387842in}{4.804117in}}%
\pgfpathlineto{\pgfqpoint{6.391064in}{4.807496in}}%
\pgfpathlineto{\pgfqpoint{6.392137in}{4.821460in}}%
\pgfpathlineto{\pgfqpoint{6.393211in}{4.827542in}}%
\pgfpathlineto{\pgfqpoint{6.394285in}{4.824614in}}%
\pgfpathlineto{\pgfqpoint{6.395359in}{4.824163in}}%
\pgfpathlineto{\pgfqpoint{6.398580in}{4.818082in}}%
\pgfpathlineto{\pgfqpoint{6.399654in}{4.814478in}}%
\pgfpathlineto{\pgfqpoint{6.401802in}{4.800063in}}%
\pgfpathlineto{\pgfqpoint{6.402876in}{4.796685in}}%
\pgfpathlineto{\pgfqpoint{6.406097in}{4.798712in}}%
\pgfpathlineto{\pgfqpoint{6.407171in}{4.802766in}}%
\pgfpathlineto{\pgfqpoint{6.408245in}{4.786324in}}%
\pgfpathlineto{\pgfqpoint{6.410393in}{4.793982in}}%
\pgfpathlineto{\pgfqpoint{6.414688in}{4.803667in}}%
\pgfpathlineto{\pgfqpoint{6.416836in}{4.812226in}}%
\pgfpathlineto{\pgfqpoint{6.417909in}{4.812226in}}%
\pgfpathlineto{\pgfqpoint{6.422205in}{4.810424in}}%
\pgfpathlineto{\pgfqpoint{6.423279in}{4.809073in}}%
\pgfpathlineto{\pgfqpoint{6.424352in}{4.809298in}}%
\pgfpathlineto{\pgfqpoint{6.425426in}{4.813352in}}%
\pgfpathlineto{\pgfqpoint{6.428648in}{4.813802in}}%
\pgfpathlineto{\pgfqpoint{6.429722in}{4.809298in}}%
\pgfpathlineto{\pgfqpoint{6.430796in}{4.812226in}}%
\pgfpathlineto{\pgfqpoint{6.431869in}{4.816956in}}%
\pgfpathlineto{\pgfqpoint{6.432943in}{4.800514in}}%
\pgfpathlineto{\pgfqpoint{6.436165in}{4.800964in}}%
\pgfpathlineto{\pgfqpoint{6.437239in}{4.804117in}}%
\pgfpathlineto{\pgfqpoint{6.439386in}{4.791955in}}%
\pgfpathlineto{\pgfqpoint{6.440460in}{4.789928in}}%
\pgfpathlineto{\pgfqpoint{6.443682in}{4.796234in}}%
\pgfpathlineto{\pgfqpoint{6.444755in}{4.782495in}}%
\pgfpathlineto{\pgfqpoint{6.446903in}{4.771684in}}%
\pgfpathlineto{\pgfqpoint{6.447977in}{4.768080in}}%
\pgfpathlineto{\pgfqpoint{6.451198in}{4.765602in}}%
\pgfpathlineto{\pgfqpoint{6.452272in}{4.756593in}}%
\pgfpathlineto{\pgfqpoint{6.453346in}{4.768305in}}%
\pgfpathlineto{\pgfqpoint{6.454420in}{4.754341in}}%
\pgfpathlineto{\pgfqpoint{6.455494in}{4.758620in}}%
\pgfpathlineto{\pgfqpoint{6.458715in}{4.752539in}}%
\pgfpathlineto{\pgfqpoint{6.460863in}{4.771233in}}%
\pgfpathlineto{\pgfqpoint{6.461937in}{4.756142in}}%
\pgfpathlineto{\pgfqpoint{6.463011in}{4.761548in}}%
\pgfpathlineto{\pgfqpoint{6.466232in}{4.756593in}}%
\pgfpathlineto{\pgfqpoint{6.468380in}{4.769431in}}%
\pgfpathlineto{\pgfqpoint{6.469454in}{4.769206in}}%
\pgfpathlineto{\pgfqpoint{6.470528in}{4.778666in}}%
\pgfpathlineto{\pgfqpoint{6.473749in}{4.774161in}}%
\pgfpathlineto{\pgfqpoint{6.475897in}{4.775963in}}%
\pgfpathlineto{\pgfqpoint{6.476971in}{4.780017in}}%
\pgfpathlineto{\pgfqpoint{6.478044in}{4.779567in}}%
\pgfpathlineto{\pgfqpoint{6.481266in}{4.774837in}}%
\pgfpathlineto{\pgfqpoint{6.482340in}{4.778441in}}%
\pgfpathlineto{\pgfqpoint{6.483414in}{4.780017in}}%
\pgfpathlineto{\pgfqpoint{6.485561in}{4.791955in}}%
\pgfpathlineto{\pgfqpoint{6.488783in}{4.795333in}}%
\pgfpathlineto{\pgfqpoint{6.489857in}{4.814028in}}%
\pgfpathlineto{\pgfqpoint{6.490930in}{4.820334in}}%
\pgfpathlineto{\pgfqpoint{6.492004in}{4.822812in}}%
\pgfpathlineto{\pgfqpoint{6.493078in}{4.818758in}}%
\pgfpathlineto{\pgfqpoint{6.496300in}{4.821686in}}%
\pgfpathlineto{\pgfqpoint{6.497374in}{4.821235in}}%
\pgfpathlineto{\pgfqpoint{6.498447in}{4.825064in}}%
\pgfpathlineto{\pgfqpoint{6.500595in}{4.809523in}}%
\pgfpathlineto{\pgfqpoint{6.503817in}{4.819659in}}%
\pgfpathlineto{\pgfqpoint{6.508112in}{4.791279in}}%
\pgfpathlineto{\pgfqpoint{6.511333in}{4.789252in}}%
\pgfpathlineto{\pgfqpoint{6.512407in}{4.783396in}}%
\pgfpathlineto{\pgfqpoint{6.513481in}{4.783621in}}%
\pgfpathlineto{\pgfqpoint{6.514555in}{4.784747in}}%
\pgfpathlineto{\pgfqpoint{6.515629in}{4.783846in}}%
\pgfpathlineto{\pgfqpoint{6.518850in}{4.783396in}}%
\pgfpathlineto{\pgfqpoint{6.519924in}{4.785198in}}%
\pgfpathlineto{\pgfqpoint{6.522072in}{4.792405in}}%
\pgfpathlineto{\pgfqpoint{6.523146in}{4.792856in}}%
\pgfpathlineto{\pgfqpoint{6.527441in}{4.791279in}}%
\pgfpathlineto{\pgfqpoint{6.528515in}{4.807271in}}%
\pgfpathlineto{\pgfqpoint{6.529589in}{4.802541in}}%
\pgfpathlineto{\pgfqpoint{6.530663in}{4.795108in}}%
\pgfpathlineto{\pgfqpoint{6.533884in}{4.807496in}}%
\pgfpathlineto{\pgfqpoint{6.536032in}{4.819433in}}%
\pgfpathlineto{\pgfqpoint{6.537106in}{4.822587in}}%
\pgfpathlineto{\pgfqpoint{6.538179in}{4.833848in}}%
\pgfpathlineto{\pgfqpoint{6.541401in}{4.833848in}}%
\pgfpathlineto{\pgfqpoint{6.542475in}{4.838804in}}%
\pgfpathlineto{\pgfqpoint{6.543549in}{4.835425in}}%
\pgfpathlineto{\pgfqpoint{6.544622in}{4.838128in}}%
\pgfpathlineto{\pgfqpoint{6.545696in}{4.837677in}}%
\pgfpathlineto{\pgfqpoint{6.548918in}{4.837227in}}%
\pgfpathlineto{\pgfqpoint{6.549992in}{4.843984in}}%
\pgfpathlineto{\pgfqpoint{6.551065in}{4.845335in}}%
\pgfpathlineto{\pgfqpoint{6.557508in}{4.884751in}}%
\pgfpathlineto{\pgfqpoint{6.558582in}{4.883851in}}%
\pgfpathlineto{\pgfqpoint{6.560730in}{4.890157in}}%
\pgfpathlineto{\pgfqpoint{6.563952in}{4.895112in}}%
\pgfpathlineto{\pgfqpoint{6.565025in}{4.890608in}}%
\pgfpathlineto{\pgfqpoint{6.566099in}{4.883175in}}%
\pgfpathlineto{\pgfqpoint{6.567173in}{4.880697in}}%
\pgfpathlineto{\pgfqpoint{6.568247in}{4.891058in}}%
\pgfpathlineto{\pgfqpoint{6.572542in}{4.893536in}}%
\pgfpathlineto{\pgfqpoint{6.573616in}{4.903446in}}%
\pgfpathlineto{\pgfqpoint{6.574690in}{4.900067in}}%
\pgfpathlineto{\pgfqpoint{6.575764in}{4.907275in}}%
\pgfpathlineto{\pgfqpoint{6.580059in}{4.918537in}}%
\pgfpathlineto{\pgfqpoint{6.581133in}{4.914708in}}%
\pgfpathlineto{\pgfqpoint{6.582207in}{4.927096in}}%
\pgfpathlineto{\pgfqpoint{6.583281in}{4.992414in}}%
\pgfpathlineto{\pgfqpoint{6.586502in}{4.991738in}}%
\pgfpathlineto{\pgfqpoint{6.588650in}{5.041965in}}%
\pgfpathlineto{\pgfqpoint{6.589724in}{5.050299in}}%
\pgfpathlineto{\pgfqpoint{6.590797in}{5.033857in}}%
\pgfpathlineto{\pgfqpoint{6.594019in}{5.047596in}}%
\pgfpathlineto{\pgfqpoint{6.595093in}{5.049398in}}%
\pgfpathlineto{\pgfqpoint{6.596167in}{5.047596in}}%
\pgfpathlineto{\pgfqpoint{6.597240in}{5.039037in}}%
\pgfpathlineto{\pgfqpoint{6.598314in}{5.023496in}}%
\pgfpathlineto{\pgfqpoint{6.601536in}{5.027100in}}%
\pgfpathlineto{\pgfqpoint{6.602610in}{5.029577in}}%
\pgfpathlineto{\pgfqpoint{6.603684in}{5.020793in}}%
\pgfpathlineto{\pgfqpoint{6.604757in}{5.024847in}}%
\pgfpathlineto{\pgfqpoint{6.605831in}{5.003000in}}%
\pgfpathlineto{\pgfqpoint{6.609053in}{5.002774in}}%
\pgfpathlineto{\pgfqpoint{6.610127in}{5.009757in}}%
\pgfpathlineto{\pgfqpoint{6.611200in}{5.003450in}}%
\pgfpathlineto{\pgfqpoint{6.613348in}{5.005477in}}%
\pgfpathlineto{\pgfqpoint{6.616570in}{4.999846in}}%
\pgfpathlineto{\pgfqpoint{6.617643in}{5.005027in}}%
\pgfpathlineto{\pgfqpoint{6.618717in}{4.988359in}}%
\pgfpathlineto{\pgfqpoint{6.619791in}{5.007504in}}%
\pgfpathlineto{\pgfqpoint{6.620865in}{5.004126in}}%
\pgfpathlineto{\pgfqpoint{6.624086in}{4.999846in}}%
\pgfpathlineto{\pgfqpoint{6.625160in}{4.977323in}}%
\pgfpathlineto{\pgfqpoint{6.626234in}{4.977548in}}%
\pgfpathlineto{\pgfqpoint{6.627308in}{4.969665in}}%
\pgfpathlineto{\pgfqpoint{6.628382in}{4.975296in}}%
\pgfpathlineto{\pgfqpoint{6.631603in}{4.982053in}}%
\pgfpathlineto{\pgfqpoint{6.632677in}{4.974845in}}%
\pgfpathlineto{\pgfqpoint{6.633751in}{4.975071in}}%
\pgfpathlineto{\pgfqpoint{6.634825in}{4.973494in}}%
\pgfpathlineto{\pgfqpoint{6.635899in}{5.001423in}}%
\pgfpathlineto{\pgfqpoint{6.641268in}{5.066065in}}%
\pgfpathlineto{\pgfqpoint{6.642342in}{5.048948in}}%
\pgfpathlineto{\pgfqpoint{6.643416in}{5.047596in}}%
\pgfpathlineto{\pgfqpoint{6.647711in}{5.034307in}}%
\pgfpathlineto{\pgfqpoint{6.648785in}{5.034758in}}%
\pgfpathlineto{\pgfqpoint{6.649859in}{5.037235in}}%
\pgfpathlineto{\pgfqpoint{6.650932in}{5.035884in}}%
\pgfpathlineto{\pgfqpoint{6.650932in}{5.035884in}}%
\pgfusepath{stroke}%
\end{pgfscope}%
\begin{pgfscope}%
\pgfpathrectangle{\pgfqpoint{4.185023in}{4.362961in}}{\pgfqpoint{2.583333in}{0.736585in}}%
\pgfusepath{clip}%
\pgfsetroundcap%
\pgfsetroundjoin%
\pgfsetlinewidth{1.505625pt}%
\definecolor{currentstroke}{rgb}{1.000000,0.647059,0.000000}%
\pgfsetstrokecolor{currentstroke}%
\pgfsetdash{}{0pt}%
\pgfpathmoveto{\pgfqpoint{4.302448in}{4.476401in}}%
\pgfpathlineto{\pgfqpoint{4.304595in}{4.484885in}}%
\pgfpathlineto{\pgfqpoint{4.305669in}{4.485917in}}%
\pgfpathlineto{\pgfqpoint{4.308891in}{4.487347in}}%
\pgfpathlineto{\pgfqpoint{4.312112in}{4.491097in}}%
\pgfpathlineto{\pgfqpoint{4.313186in}{4.490666in}}%
\pgfpathlineto{\pgfqpoint{4.318555in}{4.490263in}}%
\pgfpathlineto{\pgfqpoint{4.319629in}{4.490722in}}%
\pgfpathlineto{\pgfqpoint{4.320703in}{4.492133in}}%
\pgfpathlineto{\pgfqpoint{4.323924in}{4.493760in}}%
\pgfpathlineto{\pgfqpoint{4.327146in}{4.497931in}}%
\pgfpathlineto{\pgfqpoint{4.328220in}{4.498899in}}%
\pgfpathlineto{\pgfqpoint{4.333589in}{4.500844in}}%
\pgfpathlineto{\pgfqpoint{4.343253in}{4.505199in}}%
\pgfpathlineto{\pgfqpoint{4.348623in}{4.506473in}}%
\pgfpathlineto{\pgfqpoint{4.350770in}{4.507640in}}%
\pgfpathlineto{\pgfqpoint{4.361509in}{4.509380in}}%
\pgfpathlineto{\pgfqpoint{4.365804in}{4.510728in}}%
\pgfpathlineto{\pgfqpoint{4.372247in}{4.511449in}}%
\pgfpathlineto{\pgfqpoint{4.373321in}{4.511743in}}%
\pgfpathlineto{\pgfqpoint{4.377616in}{4.512415in}}%
\pgfpathlineto{\pgfqpoint{4.380838in}{4.513751in}}%
\pgfpathlineto{\pgfqpoint{4.386207in}{4.515047in}}%
\pgfpathlineto{\pgfqpoint{4.388355in}{4.515920in}}%
\pgfpathlineto{\pgfqpoint{4.393724in}{4.517273in}}%
\pgfpathlineto{\pgfqpoint{4.395872in}{4.518145in}}%
\pgfpathlineto{\pgfqpoint{4.401241in}{4.519355in}}%
\pgfpathlineto{\pgfqpoint{4.402315in}{4.519714in}}%
\pgfpathlineto{\pgfqpoint{4.408758in}{4.520433in}}%
\pgfpathlineto{\pgfqpoint{4.410905in}{4.521181in}}%
\pgfpathlineto{\pgfqpoint{4.424865in}{4.523272in}}%
\pgfpathlineto{\pgfqpoint{4.425939in}{4.523586in}}%
\pgfpathlineto{\pgfqpoint{4.430234in}{4.524316in}}%
\pgfpathlineto{\pgfqpoint{4.433456in}{4.525353in}}%
\pgfpathlineto{\pgfqpoint{4.447416in}{4.525699in}}%
\pgfpathlineto{\pgfqpoint{4.456006in}{4.524570in}}%
\pgfpathlineto{\pgfqpoint{4.463523in}{4.523857in}}%
\pgfpathlineto{\pgfqpoint{4.478557in}{4.522956in}}%
\pgfpathlineto{\pgfqpoint{4.492517in}{4.522892in}}%
\pgfpathlineto{\pgfqpoint{4.506477in}{4.522349in}}%
\pgfpathlineto{\pgfqpoint{4.508625in}{4.521886in}}%
\pgfpathlineto{\pgfqpoint{4.542987in}{4.520215in}}%
\pgfpathlineto{\pgfqpoint{4.556947in}{4.519634in}}%
\pgfpathlineto{\pgfqpoint{4.581646in}{4.515803in}}%
\pgfpathlineto{\pgfqpoint{4.591310in}{4.513597in}}%
\pgfpathlineto{\pgfqpoint{4.596679in}{4.512658in}}%
\pgfpathlineto{\pgfqpoint{4.598827in}{4.512022in}}%
\pgfpathlineto{\pgfqpoint{4.604196in}{4.510948in}}%
\pgfpathlineto{\pgfqpoint{4.606344in}{4.510162in}}%
\pgfpathlineto{\pgfqpoint{4.611713in}{4.509115in}}%
\pgfpathlineto{\pgfqpoint{4.613861in}{4.508346in}}%
\pgfpathlineto{\pgfqpoint{4.619230in}{4.507233in}}%
\pgfpathlineto{\pgfqpoint{4.621378in}{4.506563in}}%
\pgfpathlineto{\pgfqpoint{4.632116in}{4.505323in}}%
\pgfpathlineto{\pgfqpoint{4.651445in}{4.500068in}}%
\pgfpathlineto{\pgfqpoint{4.656814in}{4.498853in}}%
\pgfpathlineto{\pgfqpoint{4.658962in}{4.497993in}}%
\pgfpathlineto{\pgfqpoint{4.664331in}{4.496787in}}%
\pgfpathlineto{\pgfqpoint{4.666479in}{4.496058in}}%
\pgfpathlineto{\pgfqpoint{4.671848in}{4.495058in}}%
\pgfpathlineto{\pgfqpoint{4.673996in}{4.494413in}}%
\pgfpathlineto{\pgfqpoint{4.680439in}{4.493276in}}%
\pgfpathlineto{\pgfqpoint{4.684734in}{4.492690in}}%
\pgfpathlineto{\pgfqpoint{4.762050in}{4.482650in}}%
\pgfpathlineto{\pgfqpoint{4.785675in}{4.480453in}}%
\pgfpathlineto{\pgfqpoint{4.797487in}{4.479574in}}%
\pgfpathlineto{\pgfqpoint{4.809299in}{4.478703in}}%
\pgfpathlineto{\pgfqpoint{4.830776in}{4.478928in}}%
\pgfpathlineto{\pgfqpoint{4.846884in}{4.479342in}}%
\pgfpathlineto{\pgfqpoint{4.866213in}{4.480053in}}%
\pgfpathlineto{\pgfqpoint{4.891985in}{4.480801in}}%
\pgfpathlineto{\pgfqpoint{4.949972in}{4.480015in}}%
\pgfpathlineto{\pgfqpoint{4.967154in}{4.479729in}}%
\pgfpathlineto{\pgfqpoint{5.025141in}{4.480402in}}%
\pgfpathlineto{\pgfqpoint{5.191586in}{4.487605in}}%
\pgfpathlineto{\pgfqpoint{5.215210in}{4.489310in}}%
\pgfpathlineto{\pgfqpoint{5.233466in}{4.490442in}}%
\pgfpathlineto{\pgfqpoint{5.272124in}{4.494372in}}%
\pgfpathlineto{\pgfqpoint{5.281788in}{4.496021in}}%
\pgfpathlineto{\pgfqpoint{5.289305in}{4.496953in}}%
\pgfpathlineto{\pgfqpoint{5.297896in}{4.498650in}}%
\pgfpathlineto{\pgfqpoint{5.303265in}{4.499642in}}%
\pgfpathlineto{\pgfqpoint{5.305413in}{4.500285in}}%
\pgfpathlineto{\pgfqpoint{5.310782in}{4.501244in}}%
\pgfpathlineto{\pgfqpoint{5.312930in}{4.501851in}}%
\pgfpathlineto{\pgfqpoint{5.319373in}{4.502990in}}%
\pgfpathlineto{\pgfqpoint{5.327963in}{4.504752in}}%
\pgfpathlineto{\pgfqpoint{5.333333in}{4.505703in}}%
\pgfpathlineto{\pgfqpoint{5.335480in}{4.506369in}}%
\pgfpathlineto{\pgfqpoint{5.340849in}{4.507339in}}%
\pgfpathlineto{\pgfqpoint{5.342997in}{4.507980in}}%
\pgfpathlineto{\pgfqpoint{5.349440in}{4.508926in}}%
\pgfpathlineto{\pgfqpoint{5.350514in}{4.509249in}}%
\pgfpathlineto{\pgfqpoint{5.355883in}{4.510218in}}%
\pgfpathlineto{\pgfqpoint{5.358031in}{4.510844in}}%
\pgfpathlineto{\pgfqpoint{5.363400in}{4.511776in}}%
\pgfpathlineto{\pgfqpoint{5.365548in}{4.512403in}}%
\pgfpathlineto{\pgfqpoint{5.370917in}{4.513305in}}%
\pgfpathlineto{\pgfqpoint{5.380581in}{4.515302in}}%
\pgfpathlineto{\pgfqpoint{5.387024in}{4.516384in}}%
\pgfpathlineto{\pgfqpoint{5.395615in}{4.517589in}}%
\pgfpathlineto{\pgfqpoint{5.402058in}{4.518467in}}%
\pgfpathlineto{\pgfqpoint{5.410649in}{4.519957in}}%
\pgfpathlineto{\pgfqpoint{5.417092in}{4.521021in}}%
\pgfpathlineto{\pgfqpoint{5.425683in}{4.522518in}}%
\pgfpathlineto{\pgfqpoint{5.431052in}{4.523332in}}%
\pgfpathlineto{\pgfqpoint{5.433200in}{4.523942in}}%
\pgfpathlineto{\pgfqpoint{5.438569in}{4.524910in}}%
\pgfpathlineto{\pgfqpoint{5.452529in}{4.527633in}}%
\pgfpathlineto{\pgfqpoint{5.455750in}{4.528569in}}%
\pgfpathlineto{\pgfqpoint{5.461119in}{4.529453in}}%
\pgfpathlineto{\pgfqpoint{5.463267in}{4.530081in}}%
\pgfpathlineto{\pgfqpoint{5.468636in}{4.531069in}}%
\pgfpathlineto{\pgfqpoint{5.492261in}{4.535250in}}%
\pgfpathlineto{\pgfqpoint{5.493334in}{4.535543in}}%
\pgfpathlineto{\pgfqpoint{5.499778in}{4.536419in}}%
\pgfpathlineto{\pgfqpoint{5.500851in}{4.536709in}}%
\pgfpathlineto{\pgfqpoint{5.511590in}{4.538074in}}%
\pgfpathlineto{\pgfqpoint{5.523402in}{4.540034in}}%
\pgfpathlineto{\pgfqpoint{5.534140in}{4.541182in}}%
\pgfpathlineto{\pgfqpoint{5.543805in}{4.542704in}}%
\pgfpathlineto{\pgfqpoint{5.551322in}{4.543626in}}%
\pgfpathlineto{\pgfqpoint{5.566356in}{4.544963in}}%
\pgfpathlineto{\pgfqpoint{5.672666in}{4.555772in}}%
\pgfpathlineto{\pgfqpoint{5.693068in}{4.556773in}}%
\pgfpathlineto{\pgfqpoint{5.726357in}{4.558313in}}%
\pgfpathlineto{\pgfqpoint{5.746760in}{4.558939in}}%
\pgfpathlineto{\pgfqpoint{5.771459in}{4.560260in}}%
\pgfpathlineto{\pgfqpoint{5.783271in}{4.561187in}}%
\pgfpathlineto{\pgfqpoint{5.801526in}{4.563368in}}%
\pgfpathlineto{\pgfqpoint{5.813338in}{4.564544in}}%
\pgfpathlineto{\pgfqpoint{5.829446in}{4.566274in}}%
\pgfpathlineto{\pgfqpoint{5.845554in}{4.568086in}}%
\pgfpathlineto{\pgfqpoint{5.854144in}{4.569146in}}%
\pgfpathlineto{\pgfqpoint{5.865956in}{4.570220in}}%
\pgfpathlineto{\pgfqpoint{5.868104in}{4.570569in}}%
\pgfpathlineto{\pgfqpoint{5.882064in}{4.571612in}}%
\pgfpathlineto{\pgfqpoint{5.888507in}{4.571888in}}%
\pgfpathlineto{\pgfqpoint{5.914279in}{4.573106in}}%
\pgfpathlineto{\pgfqpoint{5.934682in}{4.573922in}}%
\pgfpathlineto{\pgfqpoint{5.974414in}{4.576757in}}%
\pgfpathlineto{\pgfqpoint{5.988374in}{4.577713in}}%
\pgfpathlineto{\pgfqpoint{6.002334in}{4.578514in}}%
\pgfpathlineto{\pgfqpoint{6.100053in}{4.585706in}}%
\pgfpathlineto{\pgfqpoint{6.109718in}{4.586781in}}%
\pgfpathlineto{\pgfqpoint{6.121530in}{4.587894in}}%
\pgfpathlineto{\pgfqpoint{6.132268in}{4.589210in}}%
\pgfpathlineto{\pgfqpoint{6.144081in}{4.590240in}}%
\pgfpathlineto{\pgfqpoint{6.162336in}{4.592668in}}%
\pgfpathlineto{\pgfqpoint{6.173074in}{4.593850in}}%
\pgfpathlineto{\pgfqpoint{6.184887in}{4.595411in}}%
\pgfpathlineto{\pgfqpoint{6.197773in}{4.596532in}}%
\pgfpathlineto{\pgfqpoint{6.214954in}{4.598185in}}%
\pgfpathlineto{\pgfqpoint{6.228914in}{4.599332in}}%
\pgfpathlineto{\pgfqpoint{6.252538in}{4.601863in}}%
\pgfpathlineto{\pgfqpoint{6.265424in}{4.602847in}}%
\pgfpathlineto{\pgfqpoint{6.275089in}{4.603985in}}%
\pgfpathlineto{\pgfqpoint{6.286901in}{4.604978in}}%
\pgfpathlineto{\pgfqpoint{6.297640in}{4.606325in}}%
\pgfpathlineto{\pgfqpoint{6.309452in}{4.607375in}}%
\pgfpathlineto{\pgfqpoint{6.320190in}{4.608458in}}%
\pgfpathlineto{\pgfqpoint{6.333076in}{4.609631in}}%
\pgfpathlineto{\pgfqpoint{6.342741in}{4.610582in}}%
\pgfpathlineto{\pgfqpoint{6.355627in}{4.611655in}}%
\pgfpathlineto{\pgfqpoint{6.371734in}{4.613206in}}%
\pgfpathlineto{\pgfqpoint{6.383547in}{4.614057in}}%
\pgfpathlineto{\pgfqpoint{6.395359in}{4.615436in}}%
\pgfpathlineto{\pgfqpoint{6.408245in}{4.616546in}}%
\pgfpathlineto{\pgfqpoint{6.417909in}{4.617508in}}%
\pgfpathlineto{\pgfqpoint{6.429722in}{4.618358in}}%
\pgfpathlineto{\pgfqpoint{6.440460in}{4.619426in}}%
\pgfpathlineto{\pgfqpoint{6.461937in}{4.620815in}}%
\pgfpathlineto{\pgfqpoint{6.485561in}{4.622570in}}%
\pgfpathlineto{\pgfqpoint{6.497374in}{4.623532in}}%
\pgfpathlineto{\pgfqpoint{6.508112in}{4.624583in}}%
\pgfpathlineto{\pgfqpoint{6.522072in}{4.625602in}}%
\pgfpathlineto{\pgfqpoint{6.530663in}{4.626203in}}%
\pgfpathlineto{\pgfqpoint{6.542475in}{4.627168in}}%
\pgfpathlineto{\pgfqpoint{6.560730in}{4.629247in}}%
\pgfpathlineto{\pgfqpoint{6.571468in}{4.630317in}}%
\pgfpathlineto{\pgfqpoint{6.620865in}{4.638334in}}%
\pgfpathlineto{\pgfqpoint{6.627308in}{4.639251in}}%
\pgfpathlineto{\pgfqpoint{6.635899in}{4.640616in}}%
\pgfpathlineto{\pgfqpoint{6.641268in}{4.641439in}}%
\pgfpathlineto{\pgfqpoint{6.643416in}{4.641979in}}%
\pgfpathlineto{\pgfqpoint{6.649859in}{4.642762in}}%
\pgfpathlineto{\pgfqpoint{6.650932in}{4.643023in}}%
\pgfpathlineto{\pgfqpoint{6.650932in}{4.643023in}}%
\pgfusepath{stroke}%
\end{pgfscope}%
\begin{pgfscope}%
\pgfsetrectcap%
\pgfsetmiterjoin%
\pgfsetlinewidth{0.803000pt}%
\definecolor{currentstroke}{rgb}{1.000000,1.000000,1.000000}%
\pgfsetstrokecolor{currentstroke}%
\pgfsetdash{}{0pt}%
\pgfpathmoveto{\pgfqpoint{4.185023in}{4.362961in}}%
\pgfpathlineto{\pgfqpoint{4.185023in}{5.099547in}}%
\pgfusepath{stroke}%
\end{pgfscope}%
\begin{pgfscope}%
\pgfsetrectcap%
\pgfsetmiterjoin%
\pgfsetlinewidth{0.803000pt}%
\definecolor{currentstroke}{rgb}{1.000000,1.000000,1.000000}%
\pgfsetstrokecolor{currentstroke}%
\pgfsetdash{}{0pt}%
\pgfpathmoveto{\pgfqpoint{6.768357in}{4.362961in}}%
\pgfpathlineto{\pgfqpoint{6.768357in}{5.099547in}}%
\pgfusepath{stroke}%
\end{pgfscope}%
\begin{pgfscope}%
\pgfsetrectcap%
\pgfsetmiterjoin%
\pgfsetlinewidth{0.803000pt}%
\definecolor{currentstroke}{rgb}{1.000000,1.000000,1.000000}%
\pgfsetstrokecolor{currentstroke}%
\pgfsetdash{}{0pt}%
\pgfpathmoveto{\pgfqpoint{4.185023in}{4.362961in}}%
\pgfpathlineto{\pgfqpoint{6.768357in}{4.362961in}}%
\pgfusepath{stroke}%
\end{pgfscope}%
\begin{pgfscope}%
\pgfsetrectcap%
\pgfsetmiterjoin%
\pgfsetlinewidth{0.803000pt}%
\definecolor{currentstroke}{rgb}{1.000000,1.000000,1.000000}%
\pgfsetstrokecolor{currentstroke}%
\pgfsetdash{}{0pt}%
\pgfpathmoveto{\pgfqpoint{4.185023in}{5.099547in}}%
\pgfpathlineto{\pgfqpoint{6.768357in}{5.099547in}}%
\pgfusepath{stroke}%
\end{pgfscope}%
\begin{pgfscope}%
\definecolor{textcolor}{rgb}{0.150000,0.150000,0.150000}%
\pgfsetstrokecolor{textcolor}%
\pgfsetfillcolor{textcolor}%
\pgftext[x=5.476690in,y=5.182880in,,base]{\color{textcolor}\rmfamily\fontsize{16.800000}{20.160000}\selectfont INTC}%
\end{pgfscope}%
\begin{pgfscope}%
\pgfsetbuttcap%
\pgfsetmiterjoin%
\definecolor{currentfill}{rgb}{0.917647,0.917647,0.949020}%
\pgfsetfillcolor{currentfill}%
\pgfsetlinewidth{0.000000pt}%
\definecolor{currentstroke}{rgb}{0.000000,0.000000,0.000000}%
\pgfsetstrokecolor{currentstroke}%
\pgfsetstrokeopacity{0.000000}%
\pgfsetdash{}{0pt}%
\pgfpathmoveto{\pgfqpoint{0.568357in}{3.037108in}}%
\pgfpathlineto{\pgfqpoint{3.151690in}{3.037108in}}%
\pgfpathlineto{\pgfqpoint{3.151690in}{3.773693in}}%
\pgfpathlineto{\pgfqpoint{0.568357in}{3.773693in}}%
\pgfpathclose%
\pgfusepath{fill}%
\end{pgfscope}%
\begin{pgfscope}%
\pgfpathrectangle{\pgfqpoint{0.568357in}{3.037108in}}{\pgfqpoint{2.583333in}{0.736585in}}%
\pgfusepath{clip}%
\pgfsetroundcap%
\pgfsetroundjoin%
\pgfsetlinewidth{0.803000pt}%
\definecolor{currentstroke}{rgb}{1.000000,1.000000,1.000000}%
\pgfsetstrokecolor{currentstroke}%
\pgfsetdash{}{0pt}%
\pgfpathmoveto{\pgfqpoint{0.683633in}{3.037108in}}%
\pgfpathlineto{\pgfqpoint{0.683633in}{3.773693in}}%
\pgfusepath{stroke}%
\end{pgfscope}%
\begin{pgfscope}%
\definecolor{textcolor}{rgb}{0.150000,0.150000,0.150000}%
\pgfsetstrokecolor{textcolor}%
\pgfsetfillcolor{textcolor}%
\pgftext[x=0.683633in,y=2.939885in,,top]{\color{textcolor}\rmfamily\fontsize{14.000000}{16.800000}\selectfont 2012}%
\end{pgfscope}%
\begin{pgfscope}%
\pgfpathrectangle{\pgfqpoint{0.568357in}{3.037108in}}{\pgfqpoint{2.583333in}{0.736585in}}%
\pgfusepath{clip}%
\pgfsetroundcap%
\pgfsetroundjoin%
\pgfsetlinewidth{0.803000pt}%
\definecolor{currentstroke}{rgb}{1.000000,1.000000,1.000000}%
\pgfsetstrokecolor{currentstroke}%
\pgfsetdash{}{0pt}%
\pgfpathmoveto{\pgfqpoint{1.076658in}{3.037108in}}%
\pgfpathlineto{\pgfqpoint{1.076658in}{3.773693in}}%
\pgfusepath{stroke}%
\end{pgfscope}%
\begin{pgfscope}%
\definecolor{textcolor}{rgb}{0.150000,0.150000,0.150000}%
\pgfsetstrokecolor{textcolor}%
\pgfsetfillcolor{textcolor}%
\pgftext[x=1.076658in,y=2.939885in,,top]{\color{textcolor}\rmfamily\fontsize{14.000000}{16.800000}\selectfont 2013}%
\end{pgfscope}%
\begin{pgfscope}%
\pgfpathrectangle{\pgfqpoint{0.568357in}{3.037108in}}{\pgfqpoint{2.583333in}{0.736585in}}%
\pgfusepath{clip}%
\pgfsetroundcap%
\pgfsetroundjoin%
\pgfsetlinewidth{0.803000pt}%
\definecolor{currentstroke}{rgb}{1.000000,1.000000,1.000000}%
\pgfsetstrokecolor{currentstroke}%
\pgfsetdash{}{0pt}%
\pgfpathmoveto{\pgfqpoint{1.468609in}{3.037108in}}%
\pgfpathlineto{\pgfqpoint{1.468609in}{3.773693in}}%
\pgfusepath{stroke}%
\end{pgfscope}%
\begin{pgfscope}%
\definecolor{textcolor}{rgb}{0.150000,0.150000,0.150000}%
\pgfsetstrokecolor{textcolor}%
\pgfsetfillcolor{textcolor}%
\pgftext[x=1.468609in,y=2.939885in,,top]{\color{textcolor}\rmfamily\fontsize{14.000000}{16.800000}\selectfont 2014}%
\end{pgfscope}%
\begin{pgfscope}%
\pgfpathrectangle{\pgfqpoint{0.568357in}{3.037108in}}{\pgfqpoint{2.583333in}{0.736585in}}%
\pgfusepath{clip}%
\pgfsetroundcap%
\pgfsetroundjoin%
\pgfsetlinewidth{0.803000pt}%
\definecolor{currentstroke}{rgb}{1.000000,1.000000,1.000000}%
\pgfsetstrokecolor{currentstroke}%
\pgfsetdash{}{0pt}%
\pgfpathmoveto{\pgfqpoint{1.860560in}{3.037108in}}%
\pgfpathlineto{\pgfqpoint{1.860560in}{3.773693in}}%
\pgfusepath{stroke}%
\end{pgfscope}%
\begin{pgfscope}%
\definecolor{textcolor}{rgb}{0.150000,0.150000,0.150000}%
\pgfsetstrokecolor{textcolor}%
\pgfsetfillcolor{textcolor}%
\pgftext[x=1.860560in,y=2.939885in,,top]{\color{textcolor}\rmfamily\fontsize{14.000000}{16.800000}\selectfont 2015}%
\end{pgfscope}%
\begin{pgfscope}%
\pgfpathrectangle{\pgfqpoint{0.568357in}{3.037108in}}{\pgfqpoint{2.583333in}{0.736585in}}%
\pgfusepath{clip}%
\pgfsetroundcap%
\pgfsetroundjoin%
\pgfsetlinewidth{0.803000pt}%
\definecolor{currentstroke}{rgb}{1.000000,1.000000,1.000000}%
\pgfsetstrokecolor{currentstroke}%
\pgfsetdash{}{0pt}%
\pgfpathmoveto{\pgfqpoint{2.252511in}{3.037108in}}%
\pgfpathlineto{\pgfqpoint{2.252511in}{3.773693in}}%
\pgfusepath{stroke}%
\end{pgfscope}%
\begin{pgfscope}%
\definecolor{textcolor}{rgb}{0.150000,0.150000,0.150000}%
\pgfsetstrokecolor{textcolor}%
\pgfsetfillcolor{textcolor}%
\pgftext[x=2.252511in,y=2.939885in,,top]{\color{textcolor}\rmfamily\fontsize{14.000000}{16.800000}\selectfont 2016}%
\end{pgfscope}%
\begin{pgfscope}%
\pgfpathrectangle{\pgfqpoint{0.568357in}{3.037108in}}{\pgfqpoint{2.583333in}{0.736585in}}%
\pgfusepath{clip}%
\pgfsetroundcap%
\pgfsetroundjoin%
\pgfsetlinewidth{0.803000pt}%
\definecolor{currentstroke}{rgb}{1.000000,1.000000,1.000000}%
\pgfsetstrokecolor{currentstroke}%
\pgfsetdash{}{0pt}%
\pgfpathmoveto{\pgfqpoint{2.645536in}{3.037108in}}%
\pgfpathlineto{\pgfqpoint{2.645536in}{3.773693in}}%
\pgfusepath{stroke}%
\end{pgfscope}%
\begin{pgfscope}%
\definecolor{textcolor}{rgb}{0.150000,0.150000,0.150000}%
\pgfsetstrokecolor{textcolor}%
\pgfsetfillcolor{textcolor}%
\pgftext[x=2.645536in,y=2.939885in,,top]{\color{textcolor}\rmfamily\fontsize{14.000000}{16.800000}\selectfont 2017}%
\end{pgfscope}%
\begin{pgfscope}%
\pgfpathrectangle{\pgfqpoint{0.568357in}{3.037108in}}{\pgfqpoint{2.583333in}{0.736585in}}%
\pgfusepath{clip}%
\pgfsetroundcap%
\pgfsetroundjoin%
\pgfsetlinewidth{0.803000pt}%
\definecolor{currentstroke}{rgb}{1.000000,1.000000,1.000000}%
\pgfsetstrokecolor{currentstroke}%
\pgfsetdash{}{0pt}%
\pgfpathmoveto{\pgfqpoint{3.037487in}{3.037108in}}%
\pgfpathlineto{\pgfqpoint{3.037487in}{3.773693in}}%
\pgfusepath{stroke}%
\end{pgfscope}%
\begin{pgfscope}%
\definecolor{textcolor}{rgb}{0.150000,0.150000,0.150000}%
\pgfsetstrokecolor{textcolor}%
\pgfsetfillcolor{textcolor}%
\pgftext[x=3.037487in,y=2.939885in,,top]{\color{textcolor}\rmfamily\fontsize{14.000000}{16.800000}\selectfont 2018}%
\end{pgfscope}%
\begin{pgfscope}%
\pgfpathrectangle{\pgfqpoint{0.568357in}{3.037108in}}{\pgfqpoint{2.583333in}{0.736585in}}%
\pgfusepath{clip}%
\pgfsetroundcap%
\pgfsetroundjoin%
\pgfsetlinewidth{0.803000pt}%
\definecolor{currentstroke}{rgb}{1.000000,1.000000,1.000000}%
\pgfsetstrokecolor{currentstroke}%
\pgfsetdash{}{0pt}%
\pgfpathmoveto{\pgfqpoint{0.568357in}{3.065330in}}%
\pgfpathlineto{\pgfqpoint{3.151690in}{3.065330in}}%
\pgfusepath{stroke}%
\end{pgfscope}%
\begin{pgfscope}%
\definecolor{textcolor}{rgb}{0.150000,0.150000,0.150000}%
\pgfsetstrokecolor{textcolor}%
\pgfsetfillcolor{textcolor}%
\pgftext[x=0.223712in,y=2.991463in,left,base]{\color{textcolor}\rmfamily\fontsize{14.000000}{16.800000}\selectfont 50}%
\end{pgfscope}%
\begin{pgfscope}%
\pgfpathrectangle{\pgfqpoint{0.568357in}{3.037108in}}{\pgfqpoint{2.583333in}{0.736585in}}%
\pgfusepath{clip}%
\pgfsetroundcap%
\pgfsetroundjoin%
\pgfsetlinewidth{0.803000pt}%
\definecolor{currentstroke}{rgb}{1.000000,1.000000,1.000000}%
\pgfsetstrokecolor{currentstroke}%
\pgfsetdash{}{0pt}%
\pgfpathmoveto{\pgfqpoint{0.568357in}{3.452037in}}%
\pgfpathlineto{\pgfqpoint{3.151690in}{3.452037in}}%
\pgfusepath{stroke}%
\end{pgfscope}%
\begin{pgfscope}%
\definecolor{textcolor}{rgb}{0.150000,0.150000,0.150000}%
\pgfsetstrokecolor{textcolor}%
\pgfsetfillcolor{textcolor}%
\pgftext[x=0.100000in,y=3.378171in,left,base]{\color{textcolor}\rmfamily\fontsize{14.000000}{16.800000}\selectfont 100}%
\end{pgfscope}%
\begin{pgfscope}%
\pgfpathrectangle{\pgfqpoint{0.568357in}{3.037108in}}{\pgfqpoint{2.583333in}{0.736585in}}%
\pgfusepath{clip}%
\pgfsetroundcap%
\pgfsetroundjoin%
\pgfsetlinewidth{1.505625pt}%
\definecolor{currentstroke}{rgb}{0.580392,0.403922,0.741176}%
\pgfsetstrokecolor{currentstroke}%
\pgfsetdash{}{0pt}%
\pgfpathmoveto{\pgfqpoint{0.685781in}{3.088919in}}%
\pgfpathlineto{\pgfqpoint{0.686855in}{3.086444in}}%
\pgfpathlineto{\pgfqpoint{0.687929in}{3.085980in}}%
\pgfpathlineto{\pgfqpoint{0.689002in}{3.082422in}}%
\pgfpathlineto{\pgfqpoint{0.692224in}{3.083041in}}%
\pgfpathlineto{\pgfqpoint{0.693298in}{3.084742in}}%
\pgfpathlineto{\pgfqpoint{0.694372in}{3.084278in}}%
\pgfpathlineto{\pgfqpoint{0.696519in}{3.085052in}}%
\pgfpathlineto{\pgfqpoint{0.700815in}{3.084201in}}%
\pgfpathlineto{\pgfqpoint{0.701888in}{3.085206in}}%
\pgfpathlineto{\pgfqpoint{0.702962in}{3.084665in}}%
\pgfpathlineto{\pgfqpoint{0.704036in}{3.085129in}}%
\pgfpathlineto{\pgfqpoint{0.708332in}{3.083505in}}%
\pgfpathlineto{\pgfqpoint{0.709405in}{3.084820in}}%
\pgfpathlineto{\pgfqpoint{0.710479in}{3.087836in}}%
\pgfpathlineto{\pgfqpoint{0.711553in}{3.086985in}}%
\pgfpathlineto{\pgfqpoint{0.714775in}{3.087913in}}%
\pgfpathlineto{\pgfqpoint{0.715848in}{3.089151in}}%
\pgfpathlineto{\pgfqpoint{0.717996in}{3.087140in}}%
\pgfpathlineto{\pgfqpoint{0.719070in}{3.087449in}}%
\pgfpathlineto{\pgfqpoint{0.722291in}{3.084665in}}%
\pgfpathlineto{\pgfqpoint{0.724439in}{3.084974in}}%
\pgfpathlineto{\pgfqpoint{0.726587in}{3.080953in}}%
\pgfpathlineto{\pgfqpoint{0.729808in}{3.081494in}}%
\pgfpathlineto{\pgfqpoint{0.730882in}{3.081030in}}%
\pgfpathlineto{\pgfqpoint{0.731956in}{3.081262in}}%
\pgfpathlineto{\pgfqpoint{0.733030in}{3.082963in}}%
\pgfpathlineto{\pgfqpoint{0.734104in}{3.083427in}}%
\pgfpathlineto{\pgfqpoint{0.741621in}{3.083659in}}%
\pgfpathlineto{\pgfqpoint{0.744842in}{3.083582in}}%
\pgfpathlineto{\pgfqpoint{0.745916in}{3.088145in}}%
\pgfpathlineto{\pgfqpoint{0.746990in}{3.087527in}}%
\pgfpathlineto{\pgfqpoint{0.748064in}{3.085980in}}%
\pgfpathlineto{\pgfqpoint{0.749137in}{3.085593in}}%
\pgfpathlineto{\pgfqpoint{0.752359in}{3.086444in}}%
\pgfpathlineto{\pgfqpoint{0.753433in}{3.082963in}}%
\pgfpathlineto{\pgfqpoint{0.754507in}{3.082654in}}%
\pgfpathlineto{\pgfqpoint{0.755580in}{3.086134in}}%
\pgfpathlineto{\pgfqpoint{0.756654in}{3.085438in}}%
\pgfpathlineto{\pgfqpoint{0.759876in}{3.087681in}}%
\pgfpathlineto{\pgfqpoint{0.760950in}{3.089151in}}%
\pgfpathlineto{\pgfqpoint{0.762023in}{3.087527in}}%
\pgfpathlineto{\pgfqpoint{0.764171in}{3.087836in}}%
\pgfpathlineto{\pgfqpoint{0.767393in}{3.088377in}}%
\pgfpathlineto{\pgfqpoint{0.770614in}{3.083659in}}%
\pgfpathlineto{\pgfqpoint{0.771688in}{3.084201in}}%
\pgfpathlineto{\pgfqpoint{0.777057in}{3.090930in}}%
\pgfpathlineto{\pgfqpoint{0.778131in}{3.090466in}}%
\pgfpathlineto{\pgfqpoint{0.779205in}{3.093095in}}%
\pgfpathlineto{\pgfqpoint{0.782426in}{3.094642in}}%
\pgfpathlineto{\pgfqpoint{0.785648in}{3.089151in}}%
\pgfpathlineto{\pgfqpoint{0.789943in}{3.086521in}}%
\pgfpathlineto{\pgfqpoint{0.791017in}{3.082035in}}%
\pgfpathlineto{\pgfqpoint{0.792091in}{3.081571in}}%
\pgfpathlineto{\pgfqpoint{0.793165in}{3.081726in}}%
\pgfpathlineto{\pgfqpoint{0.794239in}{3.077859in}}%
\pgfpathlineto{\pgfqpoint{0.798534in}{3.082113in}}%
\pgfpathlineto{\pgfqpoint{0.799608in}{3.076080in}}%
\pgfpathlineto{\pgfqpoint{0.800682in}{3.074688in}}%
\pgfpathlineto{\pgfqpoint{0.801755in}{3.078942in}}%
\pgfpathlineto{\pgfqpoint{0.804977in}{3.076776in}}%
\pgfpathlineto{\pgfqpoint{0.806051in}{3.079328in}}%
\pgfpathlineto{\pgfqpoint{0.808199in}{3.085438in}}%
\pgfpathlineto{\pgfqpoint{0.815715in}{3.089151in}}%
\pgfpathlineto{\pgfqpoint{0.816789in}{3.085438in}}%
\pgfpathlineto{\pgfqpoint{0.820011in}{3.085825in}}%
\pgfpathlineto{\pgfqpoint{0.821085in}{3.086908in}}%
\pgfpathlineto{\pgfqpoint{0.822158in}{3.082499in}}%
\pgfpathlineto{\pgfqpoint{0.823232in}{3.084356in}}%
\pgfpathlineto{\pgfqpoint{0.824306in}{3.082886in}}%
\pgfpathlineto{\pgfqpoint{0.827528in}{3.080334in}}%
\pgfpathlineto{\pgfqpoint{0.828601in}{3.078323in}}%
\pgfpathlineto{\pgfqpoint{0.829675in}{3.078942in}}%
\pgfpathlineto{\pgfqpoint{0.831823in}{3.076699in}}%
\pgfpathlineto{\pgfqpoint{0.836118in}{3.077782in}}%
\pgfpathlineto{\pgfqpoint{0.837192in}{3.076157in}}%
\pgfpathlineto{\pgfqpoint{0.838266in}{3.078942in}}%
\pgfpathlineto{\pgfqpoint{0.839340in}{3.075229in}}%
\pgfpathlineto{\pgfqpoint{0.843635in}{3.075771in}}%
\pgfpathlineto{\pgfqpoint{0.844709in}{3.073296in}}%
\pgfpathlineto{\pgfqpoint{0.845783in}{3.074688in}}%
\pgfpathlineto{\pgfqpoint{0.846857in}{3.070589in}}%
\pgfpathlineto{\pgfqpoint{0.850078in}{3.074146in}}%
\pgfpathlineto{\pgfqpoint{0.851152in}{3.073296in}}%
\pgfpathlineto{\pgfqpoint{0.852226in}{3.077085in}}%
\pgfpathlineto{\pgfqpoint{0.853300in}{3.077085in}}%
\pgfpathlineto{\pgfqpoint{0.854374in}{3.078168in}}%
\pgfpathlineto{\pgfqpoint{0.857595in}{3.072754in}}%
\pgfpathlineto{\pgfqpoint{0.861890in}{3.097426in}}%
\pgfpathlineto{\pgfqpoint{0.865112in}{3.099282in}}%
\pgfpathlineto{\pgfqpoint{0.867260in}{3.103691in}}%
\pgfpathlineto{\pgfqpoint{0.868333in}{3.099824in}}%
\pgfpathlineto{\pgfqpoint{0.869407in}{3.101371in}}%
\pgfpathlineto{\pgfqpoint{0.873703in}{3.100056in}}%
\pgfpathlineto{\pgfqpoint{0.874776in}{3.102685in}}%
\pgfpathlineto{\pgfqpoint{0.875850in}{3.103304in}}%
\pgfpathlineto{\pgfqpoint{0.876924in}{3.107249in}}%
\pgfpathlineto{\pgfqpoint{0.881220in}{3.110342in}}%
\pgfpathlineto{\pgfqpoint{0.884441in}{3.107790in}}%
\pgfpathlineto{\pgfqpoint{0.889810in}{3.109414in}}%
\pgfpathlineto{\pgfqpoint{0.890884in}{3.108254in}}%
\pgfpathlineto{\pgfqpoint{0.891958in}{3.113900in}}%
\pgfpathlineto{\pgfqpoint{0.895179in}{3.112895in}}%
\pgfpathlineto{\pgfqpoint{0.897327in}{3.118773in}}%
\pgfpathlineto{\pgfqpoint{0.898401in}{3.119778in}}%
\pgfpathlineto{\pgfqpoint{0.899475in}{3.114055in}}%
\pgfpathlineto{\pgfqpoint{0.902696in}{3.110729in}}%
\pgfpathlineto{\pgfqpoint{0.903770in}{3.105934in}}%
\pgfpathlineto{\pgfqpoint{0.904844in}{3.107094in}}%
\pgfpathlineto{\pgfqpoint{0.906992in}{3.119701in}}%
\pgfpathlineto{\pgfqpoint{0.910213in}{3.119237in}}%
\pgfpathlineto{\pgfqpoint{0.911287in}{3.117767in}}%
\pgfpathlineto{\pgfqpoint{0.912361in}{3.118850in}}%
\pgfpathlineto{\pgfqpoint{0.913435in}{3.112895in}}%
\pgfpathlineto{\pgfqpoint{0.914509in}{3.117148in}}%
\pgfpathlineto{\pgfqpoint{0.917730in}{3.115369in}}%
\pgfpathlineto{\pgfqpoint{0.918804in}{3.111889in}}%
\pgfpathlineto{\pgfqpoint{0.920952in}{3.112121in}}%
\pgfpathlineto{\pgfqpoint{0.922025in}{3.114132in}}%
\pgfpathlineto{\pgfqpoint{0.925247in}{3.112972in}}%
\pgfpathlineto{\pgfqpoint{0.926321in}{3.114132in}}%
\pgfpathlineto{\pgfqpoint{0.929542in}{3.108795in}}%
\pgfpathlineto{\pgfqpoint{0.932764in}{3.108177in}}%
\pgfpathlineto{\pgfqpoint{0.933838in}{3.108641in}}%
\pgfpathlineto{\pgfqpoint{0.935985in}{3.108409in}}%
\pgfpathlineto{\pgfqpoint{0.937059in}{3.111425in}}%
\pgfpathlineto{\pgfqpoint{0.942428in}{3.109956in}}%
\pgfpathlineto{\pgfqpoint{0.943502in}{3.108950in}}%
\pgfpathlineto{\pgfqpoint{0.944576in}{3.110342in}}%
\pgfpathlineto{\pgfqpoint{0.949945in}{3.109337in}}%
\pgfpathlineto{\pgfqpoint{0.951019in}{3.112972in}}%
\pgfpathlineto{\pgfqpoint{0.952093in}{3.113204in}}%
\pgfpathlineto{\pgfqpoint{0.956388in}{3.115292in}}%
\pgfpathlineto{\pgfqpoint{0.957462in}{3.114905in}}%
\pgfpathlineto{\pgfqpoint{0.958536in}{3.120319in}}%
\pgfpathlineto{\pgfqpoint{0.959610in}{3.116994in}}%
\pgfpathlineto{\pgfqpoint{0.962831in}{3.115602in}}%
\pgfpathlineto{\pgfqpoint{0.963905in}{3.117535in}}%
\pgfpathlineto{\pgfqpoint{0.964979in}{3.117844in}}%
\pgfpathlineto{\pgfqpoint{0.967127in}{3.120783in}}%
\pgfpathlineto{\pgfqpoint{0.970348in}{3.120397in}}%
\pgfpathlineto{\pgfqpoint{0.971422in}{3.122408in}}%
\pgfpathlineto{\pgfqpoint{0.972496in}{3.120397in}}%
\pgfpathlineto{\pgfqpoint{0.973570in}{3.120474in}}%
\pgfpathlineto{\pgfqpoint{0.974643in}{3.119778in}}%
\pgfpathlineto{\pgfqpoint{0.977865in}{3.121247in}}%
\pgfpathlineto{\pgfqpoint{0.978939in}{3.120087in}}%
\pgfpathlineto{\pgfqpoint{0.980013in}{3.120397in}}%
\pgfpathlineto{\pgfqpoint{0.982160in}{3.124573in}}%
\pgfpathlineto{\pgfqpoint{0.985382in}{3.123181in}}%
\pgfpathlineto{\pgfqpoint{0.986456in}{3.116607in}}%
\pgfpathlineto{\pgfqpoint{0.988603in}{3.113823in}}%
\pgfpathlineto{\pgfqpoint{0.989677in}{3.113823in}}%
\pgfpathlineto{\pgfqpoint{0.992899in}{3.117844in}}%
\pgfpathlineto{\pgfqpoint{0.993973in}{3.123877in}}%
\pgfpathlineto{\pgfqpoint{0.996120in}{3.142903in}}%
\pgfpathlineto{\pgfqpoint{0.997194in}{3.138727in}}%
\pgfpathlineto{\pgfqpoint{1.000416in}{3.138185in}}%
\pgfpathlineto{\pgfqpoint{1.001489in}{3.132462in}}%
\pgfpathlineto{\pgfqpoint{1.002563in}{3.131534in}}%
\pgfpathlineto{\pgfqpoint{1.003637in}{3.134086in}}%
\pgfpathlineto{\pgfqpoint{1.004711in}{3.132539in}}%
\pgfpathlineto{\pgfqpoint{1.010080in}{3.131998in}}%
\pgfpathlineto{\pgfqpoint{1.011154in}{3.136406in}}%
\pgfpathlineto{\pgfqpoint{1.012228in}{3.132539in}}%
\pgfpathlineto{\pgfqpoint{1.015449in}{3.131843in}}%
\pgfpathlineto{\pgfqpoint{1.016523in}{3.133235in}}%
\pgfpathlineto{\pgfqpoint{1.018671in}{3.124573in}}%
\pgfpathlineto{\pgfqpoint{1.019745in}{3.125965in}}%
\pgfpathlineto{\pgfqpoint{1.022966in}{3.124728in}}%
\pgfpathlineto{\pgfqpoint{1.026188in}{3.120861in}}%
\pgfpathlineto{\pgfqpoint{1.027262in}{3.121634in}}%
\pgfpathlineto{\pgfqpoint{1.030483in}{3.121944in}}%
\pgfpathlineto{\pgfqpoint{1.031557in}{3.124650in}}%
\pgfpathlineto{\pgfqpoint{1.032631in}{3.124186in}}%
\pgfpathlineto{\pgfqpoint{1.034778in}{3.127899in}}%
\pgfpathlineto{\pgfqpoint{1.038000in}{3.124883in}}%
\pgfpathlineto{\pgfqpoint{1.039074in}{3.123026in}}%
\pgfpathlineto{\pgfqpoint{1.040148in}{3.126197in}}%
\pgfpathlineto{\pgfqpoint{1.041221in}{3.125733in}}%
\pgfpathlineto{\pgfqpoint{1.042295in}{3.128982in}}%
\pgfpathlineto{\pgfqpoint{1.045517in}{3.128518in}}%
\pgfpathlineto{\pgfqpoint{1.047664in}{3.130528in}}%
\pgfpathlineto{\pgfqpoint{1.048738in}{3.131070in}}%
\pgfpathlineto{\pgfqpoint{1.049812in}{3.133622in}}%
\pgfpathlineto{\pgfqpoint{1.053034in}{3.134628in}}%
\pgfpathlineto{\pgfqpoint{1.054108in}{3.137876in}}%
\pgfpathlineto{\pgfqpoint{1.057329in}{3.135169in}}%
\pgfpathlineto{\pgfqpoint{1.061624in}{3.136870in}}%
\pgfpathlineto{\pgfqpoint{1.062698in}{3.134782in}}%
\pgfpathlineto{\pgfqpoint{1.063772in}{3.135556in}}%
\pgfpathlineto{\pgfqpoint{1.064846in}{3.132462in}}%
\pgfpathlineto{\pgfqpoint{1.068067in}{3.130838in}}%
\pgfpathlineto{\pgfqpoint{1.070215in}{3.131843in}}%
\pgfpathlineto{\pgfqpoint{1.071289in}{3.131302in}}%
\pgfpathlineto{\pgfqpoint{1.072363in}{3.127357in}}%
\pgfpathlineto{\pgfqpoint{1.075584in}{3.131379in}}%
\pgfpathlineto{\pgfqpoint{1.077732in}{3.136174in}}%
\pgfpathlineto{\pgfqpoint{1.078806in}{3.135556in}}%
\pgfpathlineto{\pgfqpoint{1.079880in}{3.140738in}}%
\pgfpathlineto{\pgfqpoint{1.084175in}{3.139887in}}%
\pgfpathlineto{\pgfqpoint{1.087397in}{3.145919in}}%
\pgfpathlineto{\pgfqpoint{1.090618in}{3.147312in}}%
\pgfpathlineto{\pgfqpoint{1.091692in}{3.146074in}}%
\pgfpathlineto{\pgfqpoint{1.093840in}{3.149477in}}%
\pgfpathlineto{\pgfqpoint{1.094913in}{3.151643in}}%
\pgfpathlineto{\pgfqpoint{1.099209in}{3.148085in}}%
\pgfpathlineto{\pgfqpoint{1.101356in}{3.150792in}}%
\pgfpathlineto{\pgfqpoint{1.102430in}{3.156051in}}%
\pgfpathlineto{\pgfqpoint{1.105652in}{3.154118in}}%
\pgfpathlineto{\pgfqpoint{1.106726in}{3.159222in}}%
\pgfpathlineto{\pgfqpoint{1.108873in}{3.156051in}}%
\pgfpathlineto{\pgfqpoint{1.109947in}{3.157753in}}%
\pgfpathlineto{\pgfqpoint{1.113169in}{3.157289in}}%
\pgfpathlineto{\pgfqpoint{1.115316in}{3.165564in}}%
\pgfpathlineto{\pgfqpoint{1.116390in}{3.163399in}}%
\pgfpathlineto{\pgfqpoint{1.117464in}{3.166106in}}%
\pgfpathlineto{\pgfqpoint{1.120686in}{3.165719in}}%
\pgfpathlineto{\pgfqpoint{1.121759in}{3.168194in}}%
\pgfpathlineto{\pgfqpoint{1.122833in}{3.167266in}}%
\pgfpathlineto{\pgfqpoint{1.123907in}{3.168271in}}%
\pgfpathlineto{\pgfqpoint{1.124981in}{3.170514in}}%
\pgfpathlineto{\pgfqpoint{1.129276in}{3.175696in}}%
\pgfpathlineto{\pgfqpoint{1.130350in}{3.173685in}}%
\pgfpathlineto{\pgfqpoint{1.131424in}{3.175155in}}%
\pgfpathlineto{\pgfqpoint{1.132498in}{3.175077in}}%
\pgfpathlineto{\pgfqpoint{1.135719in}{3.170591in}}%
\pgfpathlineto{\pgfqpoint{1.136793in}{3.171829in}}%
\pgfpathlineto{\pgfqpoint{1.137867in}{3.175541in}}%
\pgfpathlineto{\pgfqpoint{1.138941in}{3.174149in}}%
\pgfpathlineto{\pgfqpoint{1.140015in}{3.178016in}}%
\pgfpathlineto{\pgfqpoint{1.143236in}{3.181264in}}%
\pgfpathlineto{\pgfqpoint{1.144310in}{3.184203in}}%
\pgfpathlineto{\pgfqpoint{1.145384in}{3.182502in}}%
\pgfpathlineto{\pgfqpoint{1.147531in}{3.187684in}}%
\pgfpathlineto{\pgfqpoint{1.152901in}{3.190004in}}%
\pgfpathlineto{\pgfqpoint{1.153975in}{3.193639in}}%
\pgfpathlineto{\pgfqpoint{1.155048in}{3.194181in}}%
\pgfpathlineto{\pgfqpoint{1.158270in}{3.191706in}}%
\pgfpathlineto{\pgfqpoint{1.159344in}{3.192015in}}%
\pgfpathlineto{\pgfqpoint{1.160418in}{3.195882in}}%
\pgfpathlineto{\pgfqpoint{1.161491in}{3.193020in}}%
\pgfpathlineto{\pgfqpoint{1.162565in}{3.197738in}}%
\pgfpathlineto{\pgfqpoint{1.165787in}{3.197352in}}%
\pgfpathlineto{\pgfqpoint{1.166861in}{3.205008in}}%
\pgfpathlineto{\pgfqpoint{1.169008in}{3.209417in}}%
\pgfpathlineto{\pgfqpoint{1.173304in}{3.212046in}}%
\pgfpathlineto{\pgfqpoint{1.174377in}{3.216996in}}%
\pgfpathlineto{\pgfqpoint{1.175451in}{3.212975in}}%
\pgfpathlineto{\pgfqpoint{1.176525in}{3.215140in}}%
\pgfpathlineto{\pgfqpoint{1.180820in}{3.206710in}}%
\pgfpathlineto{\pgfqpoint{1.185116in}{3.217306in}}%
\pgfpathlineto{\pgfqpoint{1.188337in}{3.210577in}}%
\pgfpathlineto{\pgfqpoint{1.189411in}{3.221869in}}%
\pgfpathlineto{\pgfqpoint{1.190485in}{3.224885in}}%
\pgfpathlineto{\pgfqpoint{1.191559in}{3.220167in}}%
\pgfpathlineto{\pgfqpoint{1.192633in}{3.228675in}}%
\pgfpathlineto{\pgfqpoint{1.195854in}{3.230918in}}%
\pgfpathlineto{\pgfqpoint{1.196928in}{3.234940in}}%
\pgfpathlineto{\pgfqpoint{1.198002in}{3.228056in}}%
\pgfpathlineto{\pgfqpoint{1.199076in}{3.233470in}}%
\pgfpathlineto{\pgfqpoint{1.200150in}{3.232774in}}%
\pgfpathlineto{\pgfqpoint{1.203371in}{3.235790in}}%
\pgfpathlineto{\pgfqpoint{1.204445in}{3.233547in}}%
\pgfpathlineto{\pgfqpoint{1.205519in}{3.226973in}}%
\pgfpathlineto{\pgfqpoint{1.207666in}{3.236873in}}%
\pgfpathlineto{\pgfqpoint{1.210888in}{3.229912in}}%
\pgfpathlineto{\pgfqpoint{1.211962in}{3.235481in}}%
\pgfpathlineto{\pgfqpoint{1.213036in}{3.235017in}}%
\pgfpathlineto{\pgfqpoint{1.214109in}{3.233006in}}%
\pgfpathlineto{\pgfqpoint{1.215183in}{3.236950in}}%
\pgfpathlineto{\pgfqpoint{1.218405in}{3.237569in}}%
\pgfpathlineto{\pgfqpoint{1.220552in}{3.249170in}}%
\pgfpathlineto{\pgfqpoint{1.221626in}{3.247933in}}%
\pgfpathlineto{\pgfqpoint{1.222700in}{3.252109in}}%
\pgfpathlineto{\pgfqpoint{1.225922in}{3.251645in}}%
\pgfpathlineto{\pgfqpoint{1.226996in}{3.255358in}}%
\pgfpathlineto{\pgfqpoint{1.228069in}{3.254584in}}%
\pgfpathlineto{\pgfqpoint{1.230217in}{3.248088in}}%
\pgfpathlineto{\pgfqpoint{1.234512in}{3.253269in}}%
\pgfpathlineto{\pgfqpoint{1.235586in}{3.240431in}}%
\pgfpathlineto{\pgfqpoint{1.236660in}{3.242674in}}%
\pgfpathlineto{\pgfqpoint{1.237734in}{3.230763in}}%
\pgfpathlineto{\pgfqpoint{1.240955in}{3.234243in}}%
\pgfpathlineto{\pgfqpoint{1.243103in}{3.227515in}}%
\pgfpathlineto{\pgfqpoint{1.245251in}{3.235558in}}%
\pgfpathlineto{\pgfqpoint{1.248472in}{3.237105in}}%
\pgfpathlineto{\pgfqpoint{1.249546in}{3.234166in}}%
\pgfpathlineto{\pgfqpoint{1.250620in}{3.227901in}}%
\pgfpathlineto{\pgfqpoint{1.251694in}{3.235558in}}%
\pgfpathlineto{\pgfqpoint{1.252768in}{3.235558in}}%
\pgfpathlineto{\pgfqpoint{1.255989in}{3.240276in}}%
\pgfpathlineto{\pgfqpoint{1.257063in}{3.245071in}}%
\pgfpathlineto{\pgfqpoint{1.259211in}{3.220554in}}%
\pgfpathlineto{\pgfqpoint{1.260285in}{3.224344in}}%
\pgfpathlineto{\pgfqpoint{1.263506in}{3.233625in}}%
\pgfpathlineto{\pgfqpoint{1.264580in}{3.238497in}}%
\pgfpathlineto{\pgfqpoint{1.265654in}{3.249248in}}%
\pgfpathlineto{\pgfqpoint{1.266728in}{3.247392in}}%
\pgfpathlineto{\pgfqpoint{1.267801in}{3.241823in}}%
\pgfpathlineto{\pgfqpoint{1.271023in}{3.246850in}}%
\pgfpathlineto{\pgfqpoint{1.272097in}{3.246463in}}%
\pgfpathlineto{\pgfqpoint{1.273171in}{3.247856in}}%
\pgfpathlineto{\pgfqpoint{1.275318in}{3.254971in}}%
\pgfpathlineto{\pgfqpoint{1.279614in}{3.261622in}}%
\pgfpathlineto{\pgfqpoint{1.282835in}{3.268892in}}%
\pgfpathlineto{\pgfqpoint{1.286057in}{3.271599in}}%
\pgfpathlineto{\pgfqpoint{1.287130in}{3.271599in}}%
\pgfpathlineto{\pgfqpoint{1.288204in}{3.269821in}}%
\pgfpathlineto{\pgfqpoint{1.289278in}{3.270130in}}%
\pgfpathlineto{\pgfqpoint{1.290352in}{3.283587in}}%
\pgfpathlineto{\pgfqpoint{1.293574in}{3.283897in}}%
\pgfpathlineto{\pgfqpoint{1.294647in}{3.284825in}}%
\pgfpathlineto{\pgfqpoint{1.295721in}{3.284438in}}%
\pgfpathlineto{\pgfqpoint{1.297869in}{3.287532in}}%
\pgfpathlineto{\pgfqpoint{1.301090in}{3.290007in}}%
\pgfpathlineto{\pgfqpoint{1.302164in}{3.289775in}}%
\pgfpathlineto{\pgfqpoint{1.304312in}{3.293719in}}%
\pgfpathlineto{\pgfqpoint{1.305386in}{3.297741in}}%
\pgfpathlineto{\pgfqpoint{1.308607in}{3.293874in}}%
\pgfpathlineto{\pgfqpoint{1.309681in}{3.293951in}}%
\pgfpathlineto{\pgfqpoint{1.310755in}{3.293023in}}%
\pgfpathlineto{\pgfqpoint{1.311829in}{3.290857in}}%
\pgfpathlineto{\pgfqpoint{1.312903in}{3.284438in}}%
\pgfpathlineto{\pgfqpoint{1.316124in}{3.282118in}}%
\pgfpathlineto{\pgfqpoint{1.317198in}{3.288692in}}%
\pgfpathlineto{\pgfqpoint{1.319346in}{3.266031in}}%
\pgfpathlineto{\pgfqpoint{1.320419in}{3.264871in}}%
\pgfpathlineto{\pgfqpoint{1.323641in}{3.271909in}}%
\pgfpathlineto{\pgfqpoint{1.326863in}{3.257601in}}%
\pgfpathlineto{\pgfqpoint{1.327936in}{3.262937in}}%
\pgfpathlineto{\pgfqpoint{1.331158in}{3.257137in}}%
\pgfpathlineto{\pgfqpoint{1.332232in}{3.248088in}}%
\pgfpathlineto{\pgfqpoint{1.333306in}{3.250485in}}%
\pgfpathlineto{\pgfqpoint{1.334379in}{3.250795in}}%
\pgfpathlineto{\pgfqpoint{1.335453in}{3.249712in}}%
\pgfpathlineto{\pgfqpoint{1.339749in}{3.249789in}}%
\pgfpathlineto{\pgfqpoint{1.340822in}{3.252960in}}%
\pgfpathlineto{\pgfqpoint{1.346192in}{3.257291in}}%
\pgfpathlineto{\pgfqpoint{1.348339in}{3.268351in}}%
\pgfpathlineto{\pgfqpoint{1.349413in}{3.266882in}}%
\pgfpathlineto{\pgfqpoint{1.350487in}{3.263943in}}%
\pgfpathlineto{\pgfqpoint{1.353708in}{3.267036in}}%
\pgfpathlineto{\pgfqpoint{1.354782in}{3.267191in}}%
\pgfpathlineto{\pgfqpoint{1.355856in}{3.272837in}}%
\pgfpathlineto{\pgfqpoint{1.356930in}{3.273920in}}%
\pgfpathlineto{\pgfqpoint{1.358004in}{3.271290in}}%
\pgfpathlineto{\pgfqpoint{1.361225in}{3.267423in}}%
\pgfpathlineto{\pgfqpoint{1.363373in}{3.254120in}}%
\pgfpathlineto{\pgfqpoint{1.364447in}{3.254043in}}%
\pgfpathlineto{\pgfqpoint{1.365521in}{3.251800in}}%
\pgfpathlineto{\pgfqpoint{1.368742in}{3.251568in}}%
\pgfpathlineto{\pgfqpoint{1.369816in}{3.256672in}}%
\pgfpathlineto{\pgfqpoint{1.370890in}{3.255512in}}%
\pgfpathlineto{\pgfqpoint{1.371964in}{3.250795in}}%
\pgfpathlineto{\pgfqpoint{1.373038in}{3.255667in}}%
\pgfpathlineto{\pgfqpoint{1.376259in}{3.250872in}}%
\pgfpathlineto{\pgfqpoint{1.377333in}{3.244375in}}%
\pgfpathlineto{\pgfqpoint{1.378407in}{3.246695in}}%
\pgfpathlineto{\pgfqpoint{1.380554in}{3.269821in}}%
\pgfpathlineto{\pgfqpoint{1.384850in}{3.272992in}}%
\pgfpathlineto{\pgfqpoint{1.386997in}{3.286449in}}%
\pgfpathlineto{\pgfqpoint{1.388071in}{3.284206in}}%
\pgfpathlineto{\pgfqpoint{1.391293in}{3.281344in}}%
\pgfpathlineto{\pgfqpoint{1.392367in}{3.289001in}}%
\pgfpathlineto{\pgfqpoint{1.393440in}{3.287300in}}%
\pgfpathlineto{\pgfqpoint{1.394514in}{3.288924in}}%
\pgfpathlineto{\pgfqpoint{1.395588in}{3.287222in}}%
\pgfpathlineto{\pgfqpoint{1.398810in}{3.289233in}}%
\pgfpathlineto{\pgfqpoint{1.399884in}{3.294183in}}%
\pgfpathlineto{\pgfqpoint{1.402031in}{3.290703in}}%
\pgfpathlineto{\pgfqpoint{1.403105in}{3.295730in}}%
\pgfpathlineto{\pgfqpoint{1.406327in}{3.293487in}}%
\pgfpathlineto{\pgfqpoint{1.407400in}{3.292018in}}%
\pgfpathlineto{\pgfqpoint{1.408474in}{3.293487in}}%
\pgfpathlineto{\pgfqpoint{1.409548in}{3.291167in}}%
\pgfpathlineto{\pgfqpoint{1.410622in}{3.300216in}}%
\pgfpathlineto{\pgfqpoint{1.413843in}{3.301763in}}%
\pgfpathlineto{\pgfqpoint{1.414917in}{3.296967in}}%
\pgfpathlineto{\pgfqpoint{1.415991in}{3.295498in}}%
\pgfpathlineto{\pgfqpoint{1.418139in}{3.302459in}}%
\pgfpathlineto{\pgfqpoint{1.421360in}{3.301840in}}%
\pgfpathlineto{\pgfqpoint{1.423508in}{3.307486in}}%
\pgfpathlineto{\pgfqpoint{1.424582in}{3.307795in}}%
\pgfpathlineto{\pgfqpoint{1.425656in}{3.312513in}}%
\pgfpathlineto{\pgfqpoint{1.428877in}{3.315065in}}%
\pgfpathlineto{\pgfqpoint{1.429951in}{3.311276in}}%
\pgfpathlineto{\pgfqpoint{1.431025in}{3.310734in}}%
\pgfpathlineto{\pgfqpoint{1.437468in}{3.304006in}}%
\pgfpathlineto{\pgfqpoint{1.438542in}{3.301763in}}%
\pgfpathlineto{\pgfqpoint{1.439616in}{3.297354in}}%
\pgfpathlineto{\pgfqpoint{1.440689in}{3.307099in}}%
\pgfpathlineto{\pgfqpoint{1.443911in}{3.307099in}}%
\pgfpathlineto{\pgfqpoint{1.444985in}{3.305166in}}%
\pgfpathlineto{\pgfqpoint{1.446059in}{3.298282in}}%
\pgfpathlineto{\pgfqpoint{1.447132in}{3.285289in}}%
\pgfpathlineto{\pgfqpoint{1.448206in}{3.286526in}}%
\pgfpathlineto{\pgfqpoint{1.451428in}{3.286681in}}%
\pgfpathlineto{\pgfqpoint{1.452502in}{3.281963in}}%
\pgfpathlineto{\pgfqpoint{1.453575in}{3.295111in}}%
\pgfpathlineto{\pgfqpoint{1.454649in}{3.290780in}}%
\pgfpathlineto{\pgfqpoint{1.455723in}{3.291476in}}%
\pgfpathlineto{\pgfqpoint{1.460018in}{3.291322in}}%
\pgfpathlineto{\pgfqpoint{1.462166in}{3.294493in}}%
\pgfpathlineto{\pgfqpoint{1.463240in}{3.293255in}}%
\pgfpathlineto{\pgfqpoint{1.466462in}{3.292868in}}%
\pgfpathlineto{\pgfqpoint{1.467535in}{3.288151in}}%
\pgfpathlineto{\pgfqpoint{1.469683in}{3.284438in}}%
\pgfpathlineto{\pgfqpoint{1.470757in}{3.289852in}}%
\pgfpathlineto{\pgfqpoint{1.473978in}{3.293100in}}%
\pgfpathlineto{\pgfqpoint{1.475052in}{3.306094in}}%
\pgfpathlineto{\pgfqpoint{1.476126in}{3.305243in}}%
\pgfpathlineto{\pgfqpoint{1.477200in}{3.309033in}}%
\pgfpathlineto{\pgfqpoint{1.478274in}{3.309110in}}%
\pgfpathlineto{\pgfqpoint{1.481495in}{3.307563in}}%
\pgfpathlineto{\pgfqpoint{1.483643in}{3.309497in}}%
\pgfpathlineto{\pgfqpoint{1.484717in}{3.308491in}}%
\pgfpathlineto{\pgfqpoint{1.485791in}{3.311276in}}%
\pgfpathlineto{\pgfqpoint{1.490086in}{3.304392in}}%
\pgfpathlineto{\pgfqpoint{1.491160in}{3.306326in}}%
\pgfpathlineto{\pgfqpoint{1.493307in}{3.281654in}}%
\pgfpathlineto{\pgfqpoint{1.496529in}{3.277168in}}%
\pgfpathlineto{\pgfqpoint{1.497603in}{3.278251in}}%
\pgfpathlineto{\pgfqpoint{1.498677in}{3.270285in}}%
\pgfpathlineto{\pgfqpoint{1.499751in}{3.274229in}}%
\pgfpathlineto{\pgfqpoint{1.500824in}{3.267423in}}%
\pgfpathlineto{\pgfqpoint{1.504046in}{3.256131in}}%
\pgfpathlineto{\pgfqpoint{1.505120in}{3.255048in}}%
\pgfpathlineto{\pgfqpoint{1.506194in}{3.259457in}}%
\pgfpathlineto{\pgfqpoint{1.508341in}{3.277864in}}%
\pgfpathlineto{\pgfqpoint{1.511563in}{3.284670in}}%
\pgfpathlineto{\pgfqpoint{1.512637in}{3.297354in}}%
\pgfpathlineto{\pgfqpoint{1.513710in}{3.293719in}}%
\pgfpathlineto{\pgfqpoint{1.515858in}{3.295962in}}%
\pgfpathlineto{\pgfqpoint{1.520153in}{3.292018in}}%
\pgfpathlineto{\pgfqpoint{1.521227in}{3.288460in}}%
\pgfpathlineto{\pgfqpoint{1.522301in}{3.293410in}}%
\pgfpathlineto{\pgfqpoint{1.523375in}{3.292095in}}%
\pgfpathlineto{\pgfqpoint{1.526596in}{3.289311in}}%
\pgfpathlineto{\pgfqpoint{1.528744in}{3.289311in}}%
\pgfpathlineto{\pgfqpoint{1.529818in}{3.291012in}}%
\pgfpathlineto{\pgfqpoint{1.530892in}{3.296117in}}%
\pgfpathlineto{\pgfqpoint{1.534113in}{3.292327in}}%
\pgfpathlineto{\pgfqpoint{1.535187in}{3.304238in}}%
\pgfpathlineto{\pgfqpoint{1.536261in}{3.299210in}}%
\pgfpathlineto{\pgfqpoint{1.538409in}{3.304160in}}%
\pgfpathlineto{\pgfqpoint{1.543778in}{3.306016in}}%
\pgfpathlineto{\pgfqpoint{1.544852in}{3.301995in}}%
\pgfpathlineto{\pgfqpoint{1.545926in}{3.300757in}}%
\pgfpathlineto{\pgfqpoint{1.549147in}{3.308259in}}%
\pgfpathlineto{\pgfqpoint{1.550221in}{3.308259in}}%
\pgfpathlineto{\pgfqpoint{1.551295in}{3.305939in}}%
\pgfpathlineto{\pgfqpoint{1.552369in}{3.309497in}}%
\pgfpathlineto{\pgfqpoint{1.553442in}{3.321639in}}%
\pgfpathlineto{\pgfqpoint{1.556664in}{3.316767in}}%
\pgfpathlineto{\pgfqpoint{1.557738in}{3.331384in}}%
\pgfpathlineto{\pgfqpoint{1.558812in}{3.329142in}}%
\pgfpathlineto{\pgfqpoint{1.564181in}{3.337030in}}%
\pgfpathlineto{\pgfqpoint{1.565255in}{3.335097in}}%
\pgfpathlineto{\pgfqpoint{1.566328in}{3.337030in}}%
\pgfpathlineto{\pgfqpoint{1.567402in}{3.337262in}}%
\pgfpathlineto{\pgfqpoint{1.568476in}{3.338345in}}%
\pgfpathlineto{\pgfqpoint{1.571698in}{3.334787in}}%
\pgfpathlineto{\pgfqpoint{1.572772in}{3.335948in}}%
\pgfpathlineto{\pgfqpoint{1.573845in}{3.341903in}}%
\pgfpathlineto{\pgfqpoint{1.574919in}{3.325738in}}%
\pgfpathlineto{\pgfqpoint{1.575993in}{3.327904in}}%
\pgfpathlineto{\pgfqpoint{1.579215in}{3.329760in}}%
\pgfpathlineto{\pgfqpoint{1.580288in}{3.343527in}}%
\pgfpathlineto{\pgfqpoint{1.581362in}{3.340511in}}%
\pgfpathlineto{\pgfqpoint{1.582436in}{3.341980in}}%
\pgfpathlineto{\pgfqpoint{1.587805in}{3.350101in}}%
\pgfpathlineto{\pgfqpoint{1.588879in}{3.350410in}}%
\pgfpathlineto{\pgfqpoint{1.591027in}{3.347471in}}%
\pgfpathlineto{\pgfqpoint{1.594248in}{3.357913in}}%
\pgfpathlineto{\pgfqpoint{1.595322in}{3.355824in}}%
\pgfpathlineto{\pgfqpoint{1.596396in}{3.357526in}}%
\pgfpathlineto{\pgfqpoint{1.597470in}{3.352499in}}%
\pgfpathlineto{\pgfqpoint{1.598544in}{3.344300in}}%
\pgfpathlineto{\pgfqpoint{1.601765in}{3.348941in}}%
\pgfpathlineto{\pgfqpoint{1.602839in}{3.345615in}}%
\pgfpathlineto{\pgfqpoint{1.603913in}{3.355051in}}%
\pgfpathlineto{\pgfqpoint{1.604987in}{3.352267in}}%
\pgfpathlineto{\pgfqpoint{1.606061in}{3.355051in}}%
\pgfpathlineto{\pgfqpoint{1.609282in}{3.352421in}}%
\pgfpathlineto{\pgfqpoint{1.610356in}{3.355747in}}%
\pgfpathlineto{\pgfqpoint{1.613577in}{3.352808in}}%
\pgfpathlineto{\pgfqpoint{1.616799in}{3.353195in}}%
\pgfpathlineto{\pgfqpoint{1.617873in}{3.350565in}}%
\pgfpathlineto{\pgfqpoint{1.620020in}{3.360078in}}%
\pgfpathlineto{\pgfqpoint{1.621094in}{3.360233in}}%
\pgfpathlineto{\pgfqpoint{1.625390in}{3.359073in}}%
\pgfpathlineto{\pgfqpoint{1.626463in}{3.355592in}}%
\pgfpathlineto{\pgfqpoint{1.627537in}{3.358686in}}%
\pgfpathlineto{\pgfqpoint{1.628611in}{3.363481in}}%
\pgfpathlineto{\pgfqpoint{1.633980in}{3.371679in}}%
\pgfpathlineto{\pgfqpoint{1.635054in}{3.375314in}}%
\pgfpathlineto{\pgfqpoint{1.636128in}{3.375082in}}%
\pgfpathlineto{\pgfqpoint{1.639350in}{3.375314in}}%
\pgfpathlineto{\pgfqpoint{1.640423in}{3.381270in}}%
\pgfpathlineto{\pgfqpoint{1.642571in}{3.370674in}}%
\pgfpathlineto{\pgfqpoint{1.646866in}{3.370133in}}%
\pgfpathlineto{\pgfqpoint{1.647940in}{3.366652in}}%
\pgfpathlineto{\pgfqpoint{1.650088in}{3.379336in}}%
\pgfpathlineto{\pgfqpoint{1.651162in}{3.389159in}}%
\pgfpathlineto{\pgfqpoint{1.655457in}{3.384595in}}%
\pgfpathlineto{\pgfqpoint{1.656531in}{3.392484in}}%
\pgfpathlineto{\pgfqpoint{1.657605in}{3.391711in}}%
\pgfpathlineto{\pgfqpoint{1.658679in}{3.387302in}}%
\pgfpathlineto{\pgfqpoint{1.661900in}{3.384750in}}%
\pgfpathlineto{\pgfqpoint{1.662974in}{3.393180in}}%
\pgfpathlineto{\pgfqpoint{1.664048in}{3.393103in}}%
\pgfpathlineto{\pgfqpoint{1.665122in}{3.390164in}}%
\pgfpathlineto{\pgfqpoint{1.669417in}{3.397279in}}%
\pgfpathlineto{\pgfqpoint{1.670491in}{3.392175in}}%
\pgfpathlineto{\pgfqpoint{1.671565in}{3.394340in}}%
\pgfpathlineto{\pgfqpoint{1.672639in}{3.392716in}}%
\pgfpathlineto{\pgfqpoint{1.673712in}{3.387998in}}%
\pgfpathlineto{\pgfqpoint{1.676934in}{3.389932in}}%
\pgfpathlineto{\pgfqpoint{1.679082in}{3.368586in}}%
\pgfpathlineto{\pgfqpoint{1.680155in}{3.356056in}}%
\pgfpathlineto{\pgfqpoint{1.681229in}{3.365724in}}%
\pgfpathlineto{\pgfqpoint{1.684451in}{3.362166in}}%
\pgfpathlineto{\pgfqpoint{1.685525in}{3.370287in}}%
\pgfpathlineto{\pgfqpoint{1.686598in}{3.368354in}}%
\pgfpathlineto{\pgfqpoint{1.687672in}{3.368431in}}%
\pgfpathlineto{\pgfqpoint{1.688746in}{3.367812in}}%
\pgfpathlineto{\pgfqpoint{1.691968in}{3.367812in}}%
\pgfpathlineto{\pgfqpoint{1.693041in}{3.366807in}}%
\pgfpathlineto{\pgfqpoint{1.694115in}{3.369127in}}%
\pgfpathlineto{\pgfqpoint{1.695189in}{3.354200in}}%
\pgfpathlineto{\pgfqpoint{1.696263in}{3.352885in}}%
\pgfpathlineto{\pgfqpoint{1.699484in}{3.354664in}}%
\pgfpathlineto{\pgfqpoint{1.700558in}{3.352344in}}%
\pgfpathlineto{\pgfqpoint{1.701632in}{3.358377in}}%
\pgfpathlineto{\pgfqpoint{1.702706in}{3.353117in}}%
\pgfpathlineto{\pgfqpoint{1.703780in}{3.360852in}}%
\pgfpathlineto{\pgfqpoint{1.707001in}{3.361393in}}%
\pgfpathlineto{\pgfqpoint{1.708075in}{3.357835in}}%
\pgfpathlineto{\pgfqpoint{1.709149in}{3.365337in}}%
\pgfpathlineto{\pgfqpoint{1.710223in}{3.367194in}}%
\pgfpathlineto{\pgfqpoint{1.711297in}{3.361470in}}%
\pgfpathlineto{\pgfqpoint{1.715592in}{3.373536in}}%
\pgfpathlineto{\pgfqpoint{1.716666in}{3.375237in}}%
\pgfpathlineto{\pgfqpoint{1.717740in}{3.381888in}}%
\pgfpathlineto{\pgfqpoint{1.718814in}{3.379181in}}%
\pgfpathlineto{\pgfqpoint{1.722035in}{3.380110in}}%
\pgfpathlineto{\pgfqpoint{1.723109in}{3.381502in}}%
\pgfpathlineto{\pgfqpoint{1.725257in}{3.378176in}}%
\pgfpathlineto{\pgfqpoint{1.726330in}{3.383513in}}%
\pgfpathlineto{\pgfqpoint{1.730626in}{3.380960in}}%
\pgfpathlineto{\pgfqpoint{1.731700in}{3.383667in}}%
\pgfpathlineto{\pgfqpoint{1.732773in}{3.384286in}}%
\pgfpathlineto{\pgfqpoint{1.733847in}{3.388153in}}%
\pgfpathlineto{\pgfqpoint{1.737069in}{3.385678in}}%
\pgfpathlineto{\pgfqpoint{1.738143in}{3.383977in}}%
\pgfpathlineto{\pgfqpoint{1.739216in}{3.392098in}}%
\pgfpathlineto{\pgfqpoint{1.740290in}{3.389081in}}%
\pgfpathlineto{\pgfqpoint{1.744586in}{3.390241in}}%
\pgfpathlineto{\pgfqpoint{1.745660in}{3.398130in}}%
\pgfpathlineto{\pgfqpoint{1.746733in}{3.400218in}}%
\pgfpathlineto{\pgfqpoint{1.748881in}{3.412438in}}%
\pgfpathlineto{\pgfqpoint{1.752103in}{3.411665in}}%
\pgfpathlineto{\pgfqpoint{1.753176in}{3.408881in}}%
\pgfpathlineto{\pgfqpoint{1.754250in}{3.416847in}}%
\pgfpathlineto{\pgfqpoint{1.755324in}{3.406406in}}%
\pgfpathlineto{\pgfqpoint{1.756398in}{3.406406in}}%
\pgfpathlineto{\pgfqpoint{1.759619in}{3.402616in}}%
\pgfpathlineto{\pgfqpoint{1.760693in}{3.402925in}}%
\pgfpathlineto{\pgfqpoint{1.761767in}{3.387380in}}%
\pgfpathlineto{\pgfqpoint{1.762841in}{3.384286in}}%
\pgfpathlineto{\pgfqpoint{1.763915in}{3.393026in}}%
\pgfpathlineto{\pgfqpoint{1.767136in}{3.391169in}}%
\pgfpathlineto{\pgfqpoint{1.768210in}{3.374386in}}%
\pgfpathlineto{\pgfqpoint{1.769284in}{3.391556in}}%
\pgfpathlineto{\pgfqpoint{1.770358in}{3.372298in}}%
\pgfpathlineto{\pgfqpoint{1.771432in}{3.366497in}}%
\pgfpathlineto{\pgfqpoint{1.774653in}{3.352189in}}%
\pgfpathlineto{\pgfqpoint{1.775727in}{3.337804in}}%
\pgfpathlineto{\pgfqpoint{1.776801in}{3.346002in}}%
\pgfpathlineto{\pgfqpoint{1.777875in}{3.336257in}}%
\pgfpathlineto{\pgfqpoint{1.778949in}{3.349328in}}%
\pgfpathlineto{\pgfqpoint{1.782170in}{3.352731in}}%
\pgfpathlineto{\pgfqpoint{1.785392in}{3.376010in}}%
\pgfpathlineto{\pgfqpoint{1.786465in}{3.379414in}}%
\pgfpathlineto{\pgfqpoint{1.789687in}{3.385833in}}%
\pgfpathlineto{\pgfqpoint{1.791835in}{3.395965in}}%
\pgfpathlineto{\pgfqpoint{1.793982in}{3.411046in}}%
\pgfpathlineto{\pgfqpoint{1.797204in}{3.408881in}}%
\pgfpathlineto{\pgfqpoint{1.798278in}{3.416692in}}%
\pgfpathlineto{\pgfqpoint{1.800425in}{3.419399in}}%
\pgfpathlineto{\pgfqpoint{1.801499in}{3.413908in}}%
\pgfpathlineto{\pgfqpoint{1.805794in}{3.418703in}}%
\pgfpathlineto{\pgfqpoint{1.806868in}{3.417620in}}%
\pgfpathlineto{\pgfqpoint{1.807942in}{3.419786in}}%
\pgfpathlineto{\pgfqpoint{1.809016in}{3.413598in}}%
\pgfpathlineto{\pgfqpoint{1.812238in}{3.414527in}}%
\pgfpathlineto{\pgfqpoint{1.813311in}{3.418162in}}%
\pgfpathlineto{\pgfqpoint{1.814385in}{3.417698in}}%
\pgfpathlineto{\pgfqpoint{1.815459in}{3.413753in}}%
\pgfpathlineto{\pgfqpoint{1.816533in}{3.416305in}}%
\pgfpathlineto{\pgfqpoint{1.820828in}{3.408417in}}%
\pgfpathlineto{\pgfqpoint{1.824050in}{3.419012in}}%
\pgfpathlineto{\pgfqpoint{1.827271in}{3.417466in}}%
\pgfpathlineto{\pgfqpoint{1.828345in}{3.420791in}}%
\pgfpathlineto{\pgfqpoint{1.829419in}{3.415377in}}%
\pgfpathlineto{\pgfqpoint{1.830493in}{3.414295in}}%
\pgfpathlineto{\pgfqpoint{1.831567in}{3.420791in}}%
\pgfpathlineto{\pgfqpoint{1.834788in}{3.420869in}}%
\pgfpathlineto{\pgfqpoint{1.835862in}{3.417620in}}%
\pgfpathlineto{\pgfqpoint{1.836936in}{3.405246in}}%
\pgfpathlineto{\pgfqpoint{1.838010in}{3.408571in}}%
\pgfpathlineto{\pgfqpoint{1.839083in}{3.392871in}}%
\pgfpathlineto{\pgfqpoint{1.842305in}{3.389700in}}%
\pgfpathlineto{\pgfqpoint{1.843379in}{3.381424in}}%
\pgfpathlineto{\pgfqpoint{1.844453in}{3.390396in}}%
\pgfpathlineto{\pgfqpoint{1.845527in}{3.409190in}}%
\pgfpathlineto{\pgfqpoint{1.846600in}{3.400528in}}%
\pgfpathlineto{\pgfqpoint{1.849822in}{3.408649in}}%
\pgfpathlineto{\pgfqpoint{1.850896in}{3.391866in}}%
\pgfpathlineto{\pgfqpoint{1.854117in}{3.397202in}}%
\pgfpathlineto{\pgfqpoint{1.858413in}{3.399213in}}%
\pgfpathlineto{\pgfqpoint{1.859486in}{3.393799in}}%
\pgfpathlineto{\pgfqpoint{1.861634in}{3.393490in}}%
\pgfpathlineto{\pgfqpoint{1.864856in}{3.388540in}}%
\pgfpathlineto{\pgfqpoint{1.865929in}{3.384982in}}%
\pgfpathlineto{\pgfqpoint{1.867003in}{3.400605in}}%
\pgfpathlineto{\pgfqpoint{1.868077in}{3.406251in}}%
\pgfpathlineto{\pgfqpoint{1.869151in}{3.396351in}}%
\pgfpathlineto{\pgfqpoint{1.872372in}{3.393876in}}%
\pgfpathlineto{\pgfqpoint{1.873446in}{3.395114in}}%
\pgfpathlineto{\pgfqpoint{1.874520in}{3.389932in}}%
\pgfpathlineto{\pgfqpoint{1.875594in}{3.379646in}}%
\pgfpathlineto{\pgfqpoint{1.876668in}{3.390241in}}%
\pgfpathlineto{\pgfqpoint{1.880963in}{3.371370in}}%
\pgfpathlineto{\pgfqpoint{1.882037in}{3.375546in}}%
\pgfpathlineto{\pgfqpoint{1.883111in}{3.388308in}}%
\pgfpathlineto{\pgfqpoint{1.884185in}{3.377635in}}%
\pgfpathlineto{\pgfqpoint{1.887406in}{3.378021in}}%
\pgfpathlineto{\pgfqpoint{1.888480in}{3.376861in}}%
\pgfpathlineto{\pgfqpoint{1.889554in}{3.372685in}}%
\pgfpathlineto{\pgfqpoint{1.890628in}{3.378872in}}%
\pgfpathlineto{\pgfqpoint{1.891702in}{3.363559in}}%
\pgfpathlineto{\pgfqpoint{1.894923in}{3.368276in}}%
\pgfpathlineto{\pgfqpoint{1.895997in}{3.379414in}}%
\pgfpathlineto{\pgfqpoint{1.897071in}{3.371911in}}%
\pgfpathlineto{\pgfqpoint{1.898145in}{3.379414in}}%
\pgfpathlineto{\pgfqpoint{1.899218in}{3.370133in}}%
\pgfpathlineto{\pgfqpoint{1.902440in}{3.361084in}}%
\pgfpathlineto{\pgfqpoint{1.903514in}{3.364951in}}%
\pgfpathlineto{\pgfqpoint{1.904588in}{3.365183in}}%
\pgfpathlineto{\pgfqpoint{1.905661in}{3.351880in}}%
\pgfpathlineto{\pgfqpoint{1.906735in}{3.360001in}}%
\pgfpathlineto{\pgfqpoint{1.911031in}{3.365569in}}%
\pgfpathlineto{\pgfqpoint{1.912105in}{3.362321in}}%
\pgfpathlineto{\pgfqpoint{1.913178in}{3.367348in}}%
\pgfpathlineto{\pgfqpoint{1.914252in}{3.369127in}}%
\pgfpathlineto{\pgfqpoint{1.917474in}{3.368586in}}%
\pgfpathlineto{\pgfqpoint{1.919621in}{3.375701in}}%
\pgfpathlineto{\pgfqpoint{1.920695in}{3.386684in}}%
\pgfpathlineto{\pgfqpoint{1.921769in}{3.384673in}}%
\pgfpathlineto{\pgfqpoint{1.924991in}{3.389545in}}%
\pgfpathlineto{\pgfqpoint{1.927138in}{3.378717in}}%
\pgfpathlineto{\pgfqpoint{1.928212in}{3.384750in}}%
\pgfpathlineto{\pgfqpoint{1.929286in}{3.368122in}}%
\pgfpathlineto{\pgfqpoint{1.932507in}{3.371911in}}%
\pgfpathlineto{\pgfqpoint{1.934655in}{3.355824in}}%
\pgfpathlineto{\pgfqpoint{1.935729in}{3.366188in}}%
\pgfpathlineto{\pgfqpoint{1.936803in}{3.361934in}}%
\pgfpathlineto{\pgfqpoint{1.940024in}{3.374696in}}%
\pgfpathlineto{\pgfqpoint{1.941098in}{3.366575in}}%
\pgfpathlineto{\pgfqpoint{1.942172in}{3.377403in}}%
\pgfpathlineto{\pgfqpoint{1.943246in}{3.379027in}}%
\pgfpathlineto{\pgfqpoint{1.944320in}{3.383899in}}%
\pgfpathlineto{\pgfqpoint{1.947541in}{3.387921in}}%
\pgfpathlineto{\pgfqpoint{1.948615in}{3.380883in}}%
\pgfpathlineto{\pgfqpoint{1.949689in}{3.369746in}}%
\pgfpathlineto{\pgfqpoint{1.950763in}{3.368354in}}%
\pgfpathlineto{\pgfqpoint{1.951837in}{3.369746in}}%
\pgfpathlineto{\pgfqpoint{1.955058in}{3.378021in}}%
\pgfpathlineto{\pgfqpoint{1.956132in}{3.371525in}}%
\pgfpathlineto{\pgfqpoint{1.957206in}{3.361548in}}%
\pgfpathlineto{\pgfqpoint{1.958280in}{3.364873in}}%
\pgfpathlineto{\pgfqpoint{1.962575in}{3.361548in}}%
\pgfpathlineto{\pgfqpoint{1.963649in}{3.368044in}}%
\pgfpathlineto{\pgfqpoint{1.964723in}{3.368431in}}%
\pgfpathlineto{\pgfqpoint{1.966870in}{3.381579in}}%
\pgfpathlineto{\pgfqpoint{1.970092in}{3.371138in}}%
\pgfpathlineto{\pgfqpoint{1.971166in}{3.370983in}}%
\pgfpathlineto{\pgfqpoint{1.972239in}{3.371525in}}%
\pgfpathlineto{\pgfqpoint{1.973313in}{3.365956in}}%
\pgfpathlineto{\pgfqpoint{1.974387in}{3.364487in}}%
\pgfpathlineto{\pgfqpoint{1.978682in}{3.369436in}}%
\pgfpathlineto{\pgfqpoint{1.979756in}{3.370365in}}%
\pgfpathlineto{\pgfqpoint{1.980830in}{3.370519in}}%
\pgfpathlineto{\pgfqpoint{1.981904in}{3.374773in}}%
\pgfpathlineto{\pgfqpoint{1.985126in}{3.371370in}}%
\pgfpathlineto{\pgfqpoint{1.986199in}{3.372453in}}%
\pgfpathlineto{\pgfqpoint{1.987273in}{3.370055in}}%
\pgfpathlineto{\pgfqpoint{1.988347in}{3.361857in}}%
\pgfpathlineto{\pgfqpoint{1.989421in}{3.368276in}}%
\pgfpathlineto{\pgfqpoint{1.992642in}{3.369746in}}%
\pgfpathlineto{\pgfqpoint{1.993716in}{3.363945in}}%
\pgfpathlineto{\pgfqpoint{1.994790in}{3.361625in}}%
\pgfpathlineto{\pgfqpoint{1.995864in}{3.365028in}}%
\pgfpathlineto{\pgfqpoint{1.996938in}{3.377480in}}%
\pgfpathlineto{\pgfqpoint{2.000159in}{3.374464in}}%
\pgfpathlineto{\pgfqpoint{2.001233in}{3.370597in}}%
\pgfpathlineto{\pgfqpoint{2.002307in}{3.371138in}}%
\pgfpathlineto{\pgfqpoint{2.003381in}{3.379955in}}%
\pgfpathlineto{\pgfqpoint{2.004455in}{3.383203in}}%
\pgfpathlineto{\pgfqpoint{2.008750in}{3.394650in}}%
\pgfpathlineto{\pgfqpoint{2.010898in}{3.389159in}}%
\pgfpathlineto{\pgfqpoint{2.011971in}{3.381734in}}%
\pgfpathlineto{\pgfqpoint{2.016267in}{3.378099in}}%
\pgfpathlineto{\pgfqpoint{2.017341in}{3.380264in}}%
\pgfpathlineto{\pgfqpoint{2.018415in}{3.380342in}}%
\pgfpathlineto{\pgfqpoint{2.019488in}{3.373381in}}%
\pgfpathlineto{\pgfqpoint{2.023784in}{3.372221in}}%
\pgfpathlineto{\pgfqpoint{2.024858in}{3.372994in}}%
\pgfpathlineto{\pgfqpoint{2.027005in}{3.362630in}}%
\pgfpathlineto{\pgfqpoint{2.030227in}{3.358222in}}%
\pgfpathlineto{\pgfqpoint{2.031301in}{3.360001in}}%
\pgfpathlineto{\pgfqpoint{2.033448in}{3.367116in}}%
\pgfpathlineto{\pgfqpoint{2.034522in}{3.360929in}}%
\pgfpathlineto{\pgfqpoint{2.037744in}{3.354974in}}%
\pgfpathlineto{\pgfqpoint{2.038817in}{3.361006in}}%
\pgfpathlineto{\pgfqpoint{2.039891in}{3.363559in}}%
\pgfpathlineto{\pgfqpoint{2.040965in}{3.374850in}}%
\pgfpathlineto{\pgfqpoint{2.042039in}{3.371447in}}%
\pgfpathlineto{\pgfqpoint{2.045260in}{3.372994in}}%
\pgfpathlineto{\pgfqpoint{2.048482in}{3.366265in}}%
\pgfpathlineto{\pgfqpoint{2.049556in}{3.369901in}}%
\pgfpathlineto{\pgfqpoint{2.052777in}{3.356288in}}%
\pgfpathlineto{\pgfqpoint{2.053851in}{3.354742in}}%
\pgfpathlineto{\pgfqpoint{2.054925in}{3.361780in}}%
\pgfpathlineto{\pgfqpoint{2.055999in}{3.361548in}}%
\pgfpathlineto{\pgfqpoint{2.060294in}{3.359923in}}%
\pgfpathlineto{\pgfqpoint{2.061368in}{3.364873in}}%
\pgfpathlineto{\pgfqpoint{2.062442in}{3.356907in}}%
\pgfpathlineto{\pgfqpoint{2.063516in}{3.361316in}}%
\pgfpathlineto{\pgfqpoint{2.064590in}{3.369127in}}%
\pgfpathlineto{\pgfqpoint{2.067811in}{3.374232in}}%
\pgfpathlineto{\pgfqpoint{2.068885in}{3.370829in}}%
\pgfpathlineto{\pgfqpoint{2.071033in}{3.380110in}}%
\pgfpathlineto{\pgfqpoint{2.072106in}{3.372917in}}%
\pgfpathlineto{\pgfqpoint{2.075328in}{3.374928in}}%
\pgfpathlineto{\pgfqpoint{2.076402in}{3.374773in}}%
\pgfpathlineto{\pgfqpoint{2.077476in}{3.373613in}}%
\pgfpathlineto{\pgfqpoint{2.078549in}{3.373768in}}%
\pgfpathlineto{\pgfqpoint{2.079623in}{3.366497in}}%
\pgfpathlineto{\pgfqpoint{2.082845in}{3.360465in}}%
\pgfpathlineto{\pgfqpoint{2.084993in}{3.370519in}}%
\pgfpathlineto{\pgfqpoint{2.086066in}{3.371293in}}%
\pgfpathlineto{\pgfqpoint{2.087140in}{3.373845in}}%
\pgfpathlineto{\pgfqpoint{2.090362in}{3.372530in}}%
\pgfpathlineto{\pgfqpoint{2.091436in}{3.370983in}}%
\pgfpathlineto{\pgfqpoint{2.092509in}{3.376010in}}%
\pgfpathlineto{\pgfqpoint{2.093583in}{3.365956in}}%
\pgfpathlineto{\pgfqpoint{2.094657in}{3.364409in}}%
\pgfpathlineto{\pgfqpoint{2.097879in}{3.370906in}}%
\pgfpathlineto{\pgfqpoint{2.098952in}{3.365492in}}%
\pgfpathlineto{\pgfqpoint{2.101100in}{3.361857in}}%
\pgfpathlineto{\pgfqpoint{2.105395in}{3.371525in}}%
\pgfpathlineto{\pgfqpoint{2.106469in}{3.368044in}}%
\pgfpathlineto{\pgfqpoint{2.107543in}{3.367580in}}%
\pgfpathlineto{\pgfqpoint{2.108617in}{3.364023in}}%
\pgfpathlineto{\pgfqpoint{2.109691in}{3.346621in}}%
\pgfpathlineto{\pgfqpoint{2.112912in}{3.327517in}}%
\pgfpathlineto{\pgfqpoint{2.113986in}{3.312900in}}%
\pgfpathlineto{\pgfqpoint{2.115060in}{3.343527in}}%
\pgfpathlineto{\pgfqpoint{2.116134in}{3.351261in}}%
\pgfpathlineto{\pgfqpoint{2.117208in}{3.343914in}}%
\pgfpathlineto{\pgfqpoint{2.120429in}{3.335638in}}%
\pgfpathlineto{\pgfqpoint{2.121503in}{3.322413in}}%
\pgfpathlineto{\pgfqpoint{2.122577in}{3.331230in}}%
\pgfpathlineto{\pgfqpoint{2.123651in}{3.326280in}}%
\pgfpathlineto{\pgfqpoint{2.124725in}{3.316922in}}%
\pgfpathlineto{\pgfqpoint{2.129020in}{3.335329in}}%
\pgfpathlineto{\pgfqpoint{2.130094in}{3.323341in}}%
\pgfpathlineto{\pgfqpoint{2.131168in}{3.326821in}}%
\pgfpathlineto{\pgfqpoint{2.132241in}{3.328291in}}%
\pgfpathlineto{\pgfqpoint{2.135463in}{3.330998in}}%
\pgfpathlineto{\pgfqpoint{2.136537in}{3.338577in}}%
\pgfpathlineto{\pgfqpoint{2.138684in}{3.341594in}}%
\pgfpathlineto{\pgfqpoint{2.139758in}{3.331462in}}%
\pgfpathlineto{\pgfqpoint{2.142980in}{3.329683in}}%
\pgfpathlineto{\pgfqpoint{2.144054in}{3.330456in}}%
\pgfpathlineto{\pgfqpoint{2.145127in}{3.328677in}}%
\pgfpathlineto{\pgfqpoint{2.146201in}{3.325120in}}%
\pgfpathlineto{\pgfqpoint{2.147275in}{3.314756in}}%
\pgfpathlineto{\pgfqpoint{2.150497in}{3.317386in}}%
\pgfpathlineto{\pgfqpoint{2.151570in}{3.329064in}}%
\pgfpathlineto{\pgfqpoint{2.152644in}{3.331230in}}%
\pgfpathlineto{\pgfqpoint{2.153718in}{3.329915in}}%
\pgfpathlineto{\pgfqpoint{2.154792in}{3.335252in}}%
\pgfpathlineto{\pgfqpoint{2.158014in}{3.341052in}}%
\pgfpathlineto{\pgfqpoint{2.159087in}{3.331694in}}%
\pgfpathlineto{\pgfqpoint{2.160161in}{3.342599in}}%
\pgfpathlineto{\pgfqpoint{2.161235in}{3.343295in}}%
\pgfpathlineto{\pgfqpoint{2.162309in}{3.345306in}}%
\pgfpathlineto{\pgfqpoint{2.165530in}{3.349637in}}%
\pgfpathlineto{\pgfqpoint{2.166604in}{3.345925in}}%
\pgfpathlineto{\pgfqpoint{2.167678in}{3.339428in}}%
\pgfpathlineto{\pgfqpoint{2.168752in}{3.357758in}}%
\pgfpathlineto{\pgfqpoint{2.169826in}{3.365415in}}%
\pgfpathlineto{\pgfqpoint{2.173047in}{3.363326in}}%
\pgfpathlineto{\pgfqpoint{2.174121in}{3.360852in}}%
\pgfpathlineto{\pgfqpoint{2.175195in}{3.361161in}}%
\pgfpathlineto{\pgfqpoint{2.176269in}{3.374386in}}%
\pgfpathlineto{\pgfqpoint{2.177343in}{3.379878in}}%
\pgfpathlineto{\pgfqpoint{2.180564in}{3.377248in}}%
\pgfpathlineto{\pgfqpoint{2.182712in}{3.381038in}}%
\pgfpathlineto{\pgfqpoint{2.183786in}{3.387302in}}%
\pgfpathlineto{\pgfqpoint{2.184859in}{3.384905in}}%
\pgfpathlineto{\pgfqpoint{2.188081in}{3.392871in}}%
\pgfpathlineto{\pgfqpoint{2.189155in}{3.391401in}}%
\pgfpathlineto{\pgfqpoint{2.190229in}{3.391247in}}%
\pgfpathlineto{\pgfqpoint{2.191303in}{3.393954in}}%
\pgfpathlineto{\pgfqpoint{2.195598in}{3.383590in}}%
\pgfpathlineto{\pgfqpoint{2.197746in}{3.390705in}}%
\pgfpathlineto{\pgfqpoint{2.198819in}{3.379568in}}%
\pgfpathlineto{\pgfqpoint{2.199893in}{3.376861in}}%
\pgfpathlineto{\pgfqpoint{2.204189in}{3.388153in}}%
\pgfpathlineto{\pgfqpoint{2.205262in}{3.396351in}}%
\pgfpathlineto{\pgfqpoint{2.206336in}{3.395114in}}%
\pgfpathlineto{\pgfqpoint{2.207410in}{3.400296in}}%
\pgfpathlineto{\pgfqpoint{2.210632in}{3.402384in}}%
\pgfpathlineto{\pgfqpoint{2.211705in}{3.397202in}}%
\pgfpathlineto{\pgfqpoint{2.212779in}{3.396661in}}%
\pgfpathlineto{\pgfqpoint{2.214927in}{3.399522in}}%
\pgfpathlineto{\pgfqpoint{2.218148in}{3.391556in}}%
\pgfpathlineto{\pgfqpoint{2.219222in}{3.399445in}}%
\pgfpathlineto{\pgfqpoint{2.220296in}{3.397279in}}%
\pgfpathlineto{\pgfqpoint{2.221370in}{3.388385in}}%
\pgfpathlineto{\pgfqpoint{2.222444in}{3.403621in}}%
\pgfpathlineto{\pgfqpoint{2.225665in}{3.406251in}}%
\pgfpathlineto{\pgfqpoint{2.226739in}{3.399909in}}%
\pgfpathlineto{\pgfqpoint{2.227813in}{3.397975in}}%
\pgfpathlineto{\pgfqpoint{2.228887in}{3.401456in}}%
\pgfpathlineto{\pgfqpoint{2.229961in}{3.394650in}}%
\pgfpathlineto{\pgfqpoint{2.233182in}{3.397975in}}%
\pgfpathlineto{\pgfqpoint{2.235330in}{3.419786in}}%
\pgfpathlineto{\pgfqpoint{2.237478in}{3.396583in}}%
\pgfpathlineto{\pgfqpoint{2.240699in}{3.393954in}}%
\pgfpathlineto{\pgfqpoint{2.242847in}{3.407411in}}%
\pgfpathlineto{\pgfqpoint{2.243921in}{3.409035in}}%
\pgfpathlineto{\pgfqpoint{2.248216in}{3.405555in}}%
\pgfpathlineto{\pgfqpoint{2.249290in}{3.411201in}}%
\pgfpathlineto{\pgfqpoint{2.250364in}{3.409499in}}%
\pgfpathlineto{\pgfqpoint{2.251437in}{3.401997in}}%
\pgfpathlineto{\pgfqpoint{2.255733in}{3.386220in}}%
\pgfpathlineto{\pgfqpoint{2.256807in}{3.389159in}}%
\pgfpathlineto{\pgfqpoint{2.257881in}{3.385601in}}%
\pgfpathlineto{\pgfqpoint{2.260028in}{3.369901in}}%
\pgfpathlineto{\pgfqpoint{2.263250in}{3.365724in}}%
\pgfpathlineto{\pgfqpoint{2.264324in}{3.370442in}}%
\pgfpathlineto{\pgfqpoint{2.265397in}{3.361857in}}%
\pgfpathlineto{\pgfqpoint{2.266471in}{3.375005in}}%
\pgfpathlineto{\pgfqpoint{2.267545in}{3.361702in}}%
\pgfpathlineto{\pgfqpoint{2.271840in}{3.365260in}}%
\pgfpathlineto{\pgfqpoint{2.272914in}{3.352885in}}%
\pgfpathlineto{\pgfqpoint{2.273988in}{3.354278in}}%
\pgfpathlineto{\pgfqpoint{2.275062in}{3.360001in}}%
\pgfpathlineto{\pgfqpoint{2.278283in}{3.357526in}}%
\pgfpathlineto{\pgfqpoint{2.279357in}{3.391169in}}%
\pgfpathlineto{\pgfqpoint{2.280431in}{3.398053in}}%
\pgfpathlineto{\pgfqpoint{2.281505in}{3.398826in}}%
\pgfpathlineto{\pgfqpoint{2.282579in}{3.414140in}}%
\pgfpathlineto{\pgfqpoint{2.285800in}{3.413598in}}%
\pgfpathlineto{\pgfqpoint{2.286874in}{3.406870in}}%
\pgfpathlineto{\pgfqpoint{2.287948in}{3.411974in}}%
\pgfpathlineto{\pgfqpoint{2.289022in}{3.410350in}}%
\pgfpathlineto{\pgfqpoint{2.290096in}{3.386684in}}%
\pgfpathlineto{\pgfqpoint{2.293317in}{3.396970in}}%
\pgfpathlineto{\pgfqpoint{2.294391in}{3.396738in}}%
\pgfpathlineto{\pgfqpoint{2.295465in}{3.395037in}}%
\pgfpathlineto{\pgfqpoint{2.296539in}{3.394804in}}%
\pgfpathlineto{\pgfqpoint{2.302982in}{3.400450in}}%
\pgfpathlineto{\pgfqpoint{2.304056in}{3.412748in}}%
\pgfpathlineto{\pgfqpoint{2.305129in}{3.417466in}}%
\pgfpathlineto{\pgfqpoint{2.308351in}{3.421642in}}%
\pgfpathlineto{\pgfqpoint{2.309425in}{3.416924in}}%
\pgfpathlineto{\pgfqpoint{2.310499in}{3.423111in}}%
\pgfpathlineto{\pgfqpoint{2.311572in}{3.433243in}}%
\pgfpathlineto{\pgfqpoint{2.312646in}{3.428989in}}%
\pgfpathlineto{\pgfqpoint{2.315868in}{3.424890in}}%
\pgfpathlineto{\pgfqpoint{2.316942in}{3.439199in}}%
\pgfpathlineto{\pgfqpoint{2.318015in}{3.437884in}}%
\pgfpathlineto{\pgfqpoint{2.319089in}{3.435099in}}%
\pgfpathlineto{\pgfqpoint{2.320163in}{3.434094in}}%
\pgfpathlineto{\pgfqpoint{2.323385in}{3.435796in}}%
\pgfpathlineto{\pgfqpoint{2.324458in}{3.432083in}}%
\pgfpathlineto{\pgfqpoint{2.326606in}{3.438580in}}%
\pgfpathlineto{\pgfqpoint{2.327680in}{3.442679in}}%
\pgfpathlineto{\pgfqpoint{2.330902in}{3.442292in}}%
\pgfpathlineto{\pgfqpoint{2.331975in}{3.442988in}}%
\pgfpathlineto{\pgfqpoint{2.333049in}{3.440513in}}%
\pgfpathlineto{\pgfqpoint{2.334123in}{3.435796in}}%
\pgfpathlineto{\pgfqpoint{2.335197in}{3.441132in}}%
\pgfpathlineto{\pgfqpoint{2.338418in}{3.439585in}}%
\pgfpathlineto{\pgfqpoint{2.339492in}{3.440281in}}%
\pgfpathlineto{\pgfqpoint{2.340566in}{3.448247in}}%
\pgfpathlineto{\pgfqpoint{2.341640in}{3.446933in}}%
\pgfpathlineto{\pgfqpoint{2.345935in}{3.446314in}}%
\pgfpathlineto{\pgfqpoint{2.347009in}{3.452811in}}%
\pgfpathlineto{\pgfqpoint{2.348083in}{3.451651in}}%
\pgfpathlineto{\pgfqpoint{2.349157in}{3.446082in}}%
\pgfpathlineto{\pgfqpoint{2.350231in}{3.453120in}}%
\pgfpathlineto{\pgfqpoint{2.353452in}{3.448866in}}%
\pgfpathlineto{\pgfqpoint{2.355600in}{3.454744in}}%
\pgfpathlineto{\pgfqpoint{2.357747in}{3.452501in}}%
\pgfpathlineto{\pgfqpoint{2.360969in}{3.451573in}}%
\pgfpathlineto{\pgfqpoint{2.362043in}{3.456059in}}%
\pgfpathlineto{\pgfqpoint{2.363117in}{3.457993in}}%
\pgfpathlineto{\pgfqpoint{2.364191in}{3.457761in}}%
\pgfpathlineto{\pgfqpoint{2.365264in}{3.460158in}}%
\pgfpathlineto{\pgfqpoint{2.368486in}{3.465495in}}%
\pgfpathlineto{\pgfqpoint{2.370634in}{3.484289in}}%
\pgfpathlineto{\pgfqpoint{2.371707in}{3.484211in}}%
\pgfpathlineto{\pgfqpoint{2.372781in}{3.482432in}}%
\pgfpathlineto{\pgfqpoint{2.376003in}{3.483747in}}%
\pgfpathlineto{\pgfqpoint{2.377077in}{3.479493in}}%
\pgfpathlineto{\pgfqpoint{2.378150in}{3.478565in}}%
\pgfpathlineto{\pgfqpoint{2.380298in}{3.473616in}}%
\pgfpathlineto{\pgfqpoint{2.383520in}{3.478411in}}%
\pgfpathlineto{\pgfqpoint{2.384593in}{3.477947in}}%
\pgfpathlineto{\pgfqpoint{2.385667in}{3.474621in}}%
\pgfpathlineto{\pgfqpoint{2.386741in}{3.479339in}}%
\pgfpathlineto{\pgfqpoint{2.387815in}{3.478333in}}%
\pgfpathlineto{\pgfqpoint{2.391036in}{3.485294in}}%
\pgfpathlineto{\pgfqpoint{2.392110in}{3.492023in}}%
\pgfpathlineto{\pgfqpoint{2.394258in}{3.488929in}}%
\pgfpathlineto{\pgfqpoint{2.395332in}{3.484134in}}%
\pgfpathlineto{\pgfqpoint{2.398553in}{3.490399in}}%
\pgfpathlineto{\pgfqpoint{2.399627in}{3.486068in}}%
\pgfpathlineto{\pgfqpoint{2.400701in}{3.484366in}}%
\pgfpathlineto{\pgfqpoint{2.401775in}{3.479107in}}%
\pgfpathlineto{\pgfqpoint{2.402849in}{3.483283in}}%
\pgfpathlineto{\pgfqpoint{2.406070in}{3.479880in}}%
\pgfpathlineto{\pgfqpoint{2.408218in}{3.488388in}}%
\pgfpathlineto{\pgfqpoint{2.409292in}{3.485139in}}%
\pgfpathlineto{\pgfqpoint{2.410366in}{3.486300in}}%
\pgfpathlineto{\pgfqpoint{2.414661in}{3.483670in}}%
\pgfpathlineto{\pgfqpoint{2.415735in}{3.484289in}}%
\pgfpathlineto{\pgfqpoint{2.416809in}{3.496509in}}%
\pgfpathlineto{\pgfqpoint{2.417882in}{3.498442in}}%
\pgfpathlineto{\pgfqpoint{2.421104in}{3.505712in}}%
\pgfpathlineto{\pgfqpoint{2.422178in}{3.505403in}}%
\pgfpathlineto{\pgfqpoint{2.423252in}{3.505867in}}%
\pgfpathlineto{\pgfqpoint{2.424325in}{3.514452in}}%
\pgfpathlineto{\pgfqpoint{2.425399in}{3.514529in}}%
\pgfpathlineto{\pgfqpoint{2.428621in}{3.512982in}}%
\pgfpathlineto{\pgfqpoint{2.429695in}{3.515303in}}%
\pgfpathlineto{\pgfqpoint{2.430769in}{3.510275in}}%
\pgfpathlineto{\pgfqpoint{2.431842in}{3.511900in}}%
\pgfpathlineto{\pgfqpoint{2.432916in}{3.503624in}}%
\pgfpathlineto{\pgfqpoint{2.436138in}{3.511204in}}%
\pgfpathlineto{\pgfqpoint{2.437212in}{3.508574in}}%
\pgfpathlineto{\pgfqpoint{2.438285in}{3.510585in}}%
\pgfpathlineto{\pgfqpoint{2.439359in}{3.517159in}}%
\pgfpathlineto{\pgfqpoint{2.440433in}{3.504707in}}%
\pgfpathlineto{\pgfqpoint{2.443655in}{3.511281in}}%
\pgfpathlineto{\pgfqpoint{2.446876in}{3.545156in}}%
\pgfpathlineto{\pgfqpoint{2.447950in}{3.545079in}}%
\pgfpathlineto{\pgfqpoint{2.453319in}{3.554747in}}%
\pgfpathlineto{\pgfqpoint{2.454393in}{3.553819in}}%
\pgfpathlineto{\pgfqpoint{2.455467in}{3.556294in}}%
\pgfpathlineto{\pgfqpoint{2.460836in}{3.557299in}}%
\pgfpathlineto{\pgfqpoint{2.461910in}{3.558614in}}%
\pgfpathlineto{\pgfqpoint{2.462984in}{3.557299in}}%
\pgfpathlineto{\pgfqpoint{2.466205in}{3.558305in}}%
\pgfpathlineto{\pgfqpoint{2.467279in}{3.573386in}}%
\pgfpathlineto{\pgfqpoint{2.468353in}{3.572613in}}%
\pgfpathlineto{\pgfqpoint{2.469427in}{3.572690in}}%
\pgfpathlineto{\pgfqpoint{2.470501in}{3.571839in}}%
\pgfpathlineto{\pgfqpoint{2.473722in}{3.570834in}}%
\pgfpathlineto{\pgfqpoint{2.474796in}{3.572690in}}%
\pgfpathlineto{\pgfqpoint{2.476944in}{3.567740in}}%
\pgfpathlineto{\pgfqpoint{2.478017in}{3.573231in}}%
\pgfpathlineto{\pgfqpoint{2.481239in}{3.574469in}}%
\pgfpathlineto{\pgfqpoint{2.482313in}{3.570524in}}%
\pgfpathlineto{\pgfqpoint{2.483387in}{3.563796in}}%
\pgfpathlineto{\pgfqpoint{2.484460in}{3.563486in}}%
\pgfpathlineto{\pgfqpoint{2.485534in}{3.566193in}}%
\pgfpathlineto{\pgfqpoint{2.490903in}{3.559929in}}%
\pgfpathlineto{\pgfqpoint{2.491977in}{3.562868in}}%
\pgfpathlineto{\pgfqpoint{2.493051in}{3.558923in}}%
\pgfpathlineto{\pgfqpoint{2.496273in}{3.552427in}}%
\pgfpathlineto{\pgfqpoint{2.497346in}{3.538273in}}%
\pgfpathlineto{\pgfqpoint{2.498420in}{3.545234in}}%
\pgfpathlineto{\pgfqpoint{2.499494in}{3.541057in}}%
\pgfpathlineto{\pgfqpoint{2.500568in}{3.541057in}}%
\pgfpathlineto{\pgfqpoint{2.503790in}{3.535334in}}%
\pgfpathlineto{\pgfqpoint{2.504863in}{3.537577in}}%
\pgfpathlineto{\pgfqpoint{2.505937in}{3.532318in}}%
\pgfpathlineto{\pgfqpoint{2.507011in}{3.531312in}}%
\pgfpathlineto{\pgfqpoint{2.508085in}{3.534715in}}%
\pgfpathlineto{\pgfqpoint{2.511306in}{3.541057in}}%
\pgfpathlineto{\pgfqpoint{2.512380in}{3.537809in}}%
\pgfpathlineto{\pgfqpoint{2.513454in}{3.536881in}}%
\pgfpathlineto{\pgfqpoint{2.514528in}{3.535025in}}%
\pgfpathlineto{\pgfqpoint{2.515602in}{3.536726in}}%
\pgfpathlineto{\pgfqpoint{2.519897in}{3.539820in}}%
\pgfpathlineto{\pgfqpoint{2.522045in}{3.537809in}}%
\pgfpathlineto{\pgfqpoint{2.523119in}{3.528915in}}%
\pgfpathlineto{\pgfqpoint{2.526340in}{3.535566in}}%
\pgfpathlineto{\pgfqpoint{2.527414in}{3.524429in}}%
\pgfpathlineto{\pgfqpoint{2.528488in}{3.526208in}}%
\pgfpathlineto{\pgfqpoint{2.529562in}{3.531776in}}%
\pgfpathlineto{\pgfqpoint{2.530635in}{3.529069in}}%
\pgfpathlineto{\pgfqpoint{2.533857in}{3.524816in}}%
\pgfpathlineto{\pgfqpoint{2.534931in}{3.526904in}}%
\pgfpathlineto{\pgfqpoint{2.537079in}{3.537732in}}%
\pgfpathlineto{\pgfqpoint{2.538152in}{3.533091in}}%
\pgfpathlineto{\pgfqpoint{2.541374in}{3.525666in}}%
\pgfpathlineto{\pgfqpoint{2.542448in}{3.536030in}}%
\pgfpathlineto{\pgfqpoint{2.543522in}{3.537268in}}%
\pgfpathlineto{\pgfqpoint{2.544595in}{3.521954in}}%
\pgfpathlineto{\pgfqpoint{2.545669in}{3.528141in}}%
\pgfpathlineto{\pgfqpoint{2.548891in}{3.533091in}}%
\pgfpathlineto{\pgfqpoint{2.549965in}{3.533169in}}%
\pgfpathlineto{\pgfqpoint{2.551038in}{3.535721in}}%
\pgfpathlineto{\pgfqpoint{2.552112in}{3.532627in}}%
\pgfpathlineto{\pgfqpoint{2.553186in}{3.536185in}}%
\pgfpathlineto{\pgfqpoint{2.556408in}{3.540207in}}%
\pgfpathlineto{\pgfqpoint{2.557481in}{3.524661in}}%
\pgfpathlineto{\pgfqpoint{2.559629in}{3.529069in}}%
\pgfpathlineto{\pgfqpoint{2.560703in}{3.524042in}}%
\pgfpathlineto{\pgfqpoint{2.563924in}{3.530771in}}%
\pgfpathlineto{\pgfqpoint{2.564998in}{3.508574in}}%
\pgfpathlineto{\pgfqpoint{2.566072in}{3.502696in}}%
\pgfpathlineto{\pgfqpoint{2.567146in}{3.504707in}}%
\pgfpathlineto{\pgfqpoint{2.568220in}{3.494420in}}%
\pgfpathlineto{\pgfqpoint{2.571441in}{3.495658in}}%
\pgfpathlineto{\pgfqpoint{2.572515in}{3.498210in}}%
\pgfpathlineto{\pgfqpoint{2.573589in}{3.502464in}}%
\pgfpathlineto{\pgfqpoint{2.574663in}{3.510662in}}%
\pgfpathlineto{\pgfqpoint{2.575737in}{3.508033in}}%
\pgfpathlineto{\pgfqpoint{2.578958in}{3.512750in}}%
\pgfpathlineto{\pgfqpoint{2.581106in}{3.504629in}}%
\pgfpathlineto{\pgfqpoint{2.583254in}{3.506486in}}%
\pgfpathlineto{\pgfqpoint{2.587549in}{3.520407in}}%
\pgfpathlineto{\pgfqpoint{2.588623in}{3.543842in}}%
\pgfpathlineto{\pgfqpoint{2.593992in}{3.517159in}}%
\pgfpathlineto{\pgfqpoint{2.595066in}{3.515148in}}%
\pgfpathlineto{\pgfqpoint{2.596140in}{3.515457in}}%
\pgfpathlineto{\pgfqpoint{2.597213in}{3.516927in}}%
\pgfpathlineto{\pgfqpoint{2.598287in}{3.513988in}}%
\pgfpathlineto{\pgfqpoint{2.601509in}{3.511358in}}%
\pgfpathlineto{\pgfqpoint{2.602583in}{3.495039in}}%
\pgfpathlineto{\pgfqpoint{2.603657in}{3.497359in}}%
\pgfpathlineto{\pgfqpoint{2.605804in}{3.505094in}}%
\pgfpathlineto{\pgfqpoint{2.609026in}{3.497823in}}%
\pgfpathlineto{\pgfqpoint{2.610100in}{3.493106in}}%
\pgfpathlineto{\pgfqpoint{2.611173in}{3.484598in}}%
\pgfpathlineto{\pgfqpoint{2.612247in}{3.485139in}}%
\pgfpathlineto{\pgfqpoint{2.613321in}{3.489393in}}%
\pgfpathlineto{\pgfqpoint{2.616543in}{3.489239in}}%
\pgfpathlineto{\pgfqpoint{2.617616in}{3.490089in}}%
\pgfpathlineto{\pgfqpoint{2.618690in}{3.483129in}}%
\pgfpathlineto{\pgfqpoint{2.619764in}{3.482355in}}%
\pgfpathlineto{\pgfqpoint{2.620838in}{3.491559in}}%
\pgfpathlineto{\pgfqpoint{2.625133in}{3.517855in}}%
\pgfpathlineto{\pgfqpoint{2.626207in}{3.511281in}}%
\pgfpathlineto{\pgfqpoint{2.627281in}{3.517855in}}%
\pgfpathlineto{\pgfqpoint{2.628355in}{3.517778in}}%
\pgfpathlineto{\pgfqpoint{2.631576in}{3.518783in}}%
\pgfpathlineto{\pgfqpoint{2.633724in}{3.513601in}}%
\pgfpathlineto{\pgfqpoint{2.634798in}{3.514529in}}%
\pgfpathlineto{\pgfqpoint{2.635872in}{3.518319in}}%
\pgfpathlineto{\pgfqpoint{2.640167in}{3.517932in}}%
\pgfpathlineto{\pgfqpoint{2.641241in}{3.512132in}}%
\pgfpathlineto{\pgfqpoint{2.642315in}{3.514916in}}%
\pgfpathlineto{\pgfqpoint{2.643389in}{3.512905in}}%
\pgfpathlineto{\pgfqpoint{2.647684in}{3.517468in}}%
\pgfpathlineto{\pgfqpoint{2.648758in}{3.516076in}}%
\pgfpathlineto{\pgfqpoint{2.649832in}{3.524816in}}%
\pgfpathlineto{\pgfqpoint{2.650905in}{3.520794in}}%
\pgfpathlineto{\pgfqpoint{2.654127in}{3.520639in}}%
\pgfpathlineto{\pgfqpoint{2.655201in}{3.519788in}}%
\pgfpathlineto{\pgfqpoint{2.656275in}{3.509425in}}%
\pgfpathlineto{\pgfqpoint{2.658422in}{3.508497in}}%
\pgfpathlineto{\pgfqpoint{2.662718in}{3.510430in}}%
\pgfpathlineto{\pgfqpoint{2.663791in}{3.509193in}}%
\pgfpathlineto{\pgfqpoint{2.664865in}{3.505558in}}%
\pgfpathlineto{\pgfqpoint{2.665939in}{3.505248in}}%
\pgfpathlineto{\pgfqpoint{2.669161in}{3.503469in}}%
\pgfpathlineto{\pgfqpoint{2.670234in}{3.487924in}}%
\pgfpathlineto{\pgfqpoint{2.671308in}{3.495426in}}%
\pgfpathlineto{\pgfqpoint{2.672382in}{3.488465in}}%
\pgfpathlineto{\pgfqpoint{2.673456in}{3.499680in}}%
\pgfpathlineto{\pgfqpoint{2.676678in}{3.497823in}}%
\pgfpathlineto{\pgfqpoint{2.677751in}{3.498674in}}%
\pgfpathlineto{\pgfqpoint{2.678825in}{3.498519in}}%
\pgfpathlineto{\pgfqpoint{2.679899in}{3.500994in}}%
\pgfpathlineto{\pgfqpoint{2.680973in}{3.501536in}}%
\pgfpathlineto{\pgfqpoint{2.684194in}{3.499757in}}%
\pgfpathlineto{\pgfqpoint{2.685268in}{3.500376in}}%
\pgfpathlineto{\pgfqpoint{2.686342in}{3.499757in}}%
\pgfpathlineto{\pgfqpoint{2.687416in}{3.504707in}}%
\pgfpathlineto{\pgfqpoint{2.688490in}{3.513137in}}%
\pgfpathlineto{\pgfqpoint{2.691711in}{3.517778in}}%
\pgfpathlineto{\pgfqpoint{2.692785in}{3.521258in}}%
\pgfpathlineto{\pgfqpoint{2.696007in}{3.539356in}}%
\pgfpathlineto{\pgfqpoint{2.700302in}{3.545156in}}%
\pgfpathlineto{\pgfqpoint{2.701376in}{3.544074in}}%
\pgfpathlineto{\pgfqpoint{2.703523in}{3.573231in}}%
\pgfpathlineto{\pgfqpoint{2.707819in}{3.569442in}}%
\pgfpathlineto{\pgfqpoint{2.708893in}{3.581430in}}%
\pgfpathlineto{\pgfqpoint{2.709967in}{3.579805in}}%
\pgfpathlineto{\pgfqpoint{2.711040in}{3.580966in}}%
\pgfpathlineto{\pgfqpoint{2.714262in}{3.580347in}}%
\pgfpathlineto{\pgfqpoint{2.715336in}{3.581275in}}%
\pgfpathlineto{\pgfqpoint{2.716410in}{3.583208in}}%
\pgfpathlineto{\pgfqpoint{2.717483in}{3.596666in}}%
\pgfpathlineto{\pgfqpoint{2.718557in}{3.598599in}}%
\pgfpathlineto{\pgfqpoint{2.721779in}{3.602002in}}%
\pgfpathlineto{\pgfqpoint{2.722853in}{3.604709in}}%
\pgfpathlineto{\pgfqpoint{2.723926in}{3.618631in}}%
\pgfpathlineto{\pgfqpoint{2.726074in}{3.612057in}}%
\pgfpathlineto{\pgfqpoint{2.729296in}{3.612134in}}%
\pgfpathlineto{\pgfqpoint{2.732517in}{3.596357in}}%
\pgfpathlineto{\pgfqpoint{2.733591in}{3.593263in}}%
\pgfpathlineto{\pgfqpoint{2.736812in}{3.595583in}}%
\pgfpathlineto{\pgfqpoint{2.737886in}{3.594578in}}%
\pgfpathlineto{\pgfqpoint{2.738960in}{3.589164in}}%
\pgfpathlineto{\pgfqpoint{2.741108in}{3.586534in}}%
\pgfpathlineto{\pgfqpoint{2.746477in}{3.588313in}}%
\pgfpathlineto{\pgfqpoint{2.747551in}{3.590169in}}%
\pgfpathlineto{\pgfqpoint{2.748625in}{3.589164in}}%
\pgfpathlineto{\pgfqpoint{2.752920in}{3.584059in}}%
\pgfpathlineto{\pgfqpoint{2.753994in}{3.592721in}}%
\pgfpathlineto{\pgfqpoint{2.755068in}{3.589705in}}%
\pgfpathlineto{\pgfqpoint{2.759363in}{3.595042in}}%
\pgfpathlineto{\pgfqpoint{2.760437in}{3.566580in}}%
\pgfpathlineto{\pgfqpoint{2.761511in}{3.563332in}}%
\pgfpathlineto{\pgfqpoint{2.762585in}{3.566967in}}%
\pgfpathlineto{\pgfqpoint{2.763658in}{3.566193in}}%
\pgfpathlineto{\pgfqpoint{2.769028in}{3.578877in}}%
\pgfpathlineto{\pgfqpoint{2.770101in}{3.580579in}}%
\pgfpathlineto{\pgfqpoint{2.771175in}{3.578645in}}%
\pgfpathlineto{\pgfqpoint{2.774397in}{3.577640in}}%
\pgfpathlineto{\pgfqpoint{2.775471in}{3.580270in}}%
\pgfpathlineto{\pgfqpoint{2.776545in}{3.577640in}}%
\pgfpathlineto{\pgfqpoint{2.777618in}{3.582126in}}%
\pgfpathlineto{\pgfqpoint{2.778692in}{3.578877in}}%
\pgfpathlineto{\pgfqpoint{2.782988in}{3.576712in}}%
\pgfpathlineto{\pgfqpoint{2.784061in}{3.573850in}}%
\pgfpathlineto{\pgfqpoint{2.786209in}{3.579883in}}%
\pgfpathlineto{\pgfqpoint{2.790504in}{3.609969in}}%
\pgfpathlineto{\pgfqpoint{2.791578in}{3.601925in}}%
\pgfpathlineto{\pgfqpoint{2.792652in}{3.604168in}}%
\pgfpathlineto{\pgfqpoint{2.793726in}{3.604323in}}%
\pgfpathlineto{\pgfqpoint{2.796947in}{3.606256in}}%
\pgfpathlineto{\pgfqpoint{2.798021in}{3.608112in}}%
\pgfpathlineto{\pgfqpoint{2.799095in}{3.608035in}}%
\pgfpathlineto{\pgfqpoint{2.800169in}{3.614377in}}%
\pgfpathlineto{\pgfqpoint{2.801243in}{3.609891in}}%
\pgfpathlineto{\pgfqpoint{2.805538in}{3.611283in}}%
\pgfpathlineto{\pgfqpoint{2.806612in}{3.619636in}}%
\pgfpathlineto{\pgfqpoint{2.807686in}{3.623581in}}%
\pgfpathlineto{\pgfqpoint{2.808760in}{3.633094in}}%
\pgfpathlineto{\pgfqpoint{2.811981in}{3.635182in}}%
\pgfpathlineto{\pgfqpoint{2.813055in}{3.638585in}}%
\pgfpathlineto{\pgfqpoint{2.814129in}{3.638044in}}%
\pgfpathlineto{\pgfqpoint{2.815203in}{3.636574in}}%
\pgfpathlineto{\pgfqpoint{2.816277in}{3.643767in}}%
\pgfpathlineto{\pgfqpoint{2.819498in}{3.645855in}}%
\pgfpathlineto{\pgfqpoint{2.820572in}{3.647325in}}%
\pgfpathlineto{\pgfqpoint{2.821646in}{3.652816in}}%
\pgfpathlineto{\pgfqpoint{2.822720in}{3.654827in}}%
\pgfpathlineto{\pgfqpoint{2.823793in}{3.664417in}}%
\pgfpathlineto{\pgfqpoint{2.827015in}{3.662406in}}%
\pgfpathlineto{\pgfqpoint{2.828089in}{3.663489in}}%
\pgfpathlineto{\pgfqpoint{2.829163in}{3.668516in}}%
\pgfpathlineto{\pgfqpoint{2.830236in}{3.676869in}}%
\pgfpathlineto{\pgfqpoint{2.831310in}{3.679653in}}%
\pgfpathlineto{\pgfqpoint{2.834532in}{3.679035in}}%
\pgfpathlineto{\pgfqpoint{2.837753in}{3.651888in}}%
\pgfpathlineto{\pgfqpoint{2.838827in}{3.649335in}}%
\pgfpathlineto{\pgfqpoint{2.842049in}{3.653821in}}%
\pgfpathlineto{\pgfqpoint{2.844196in}{3.659235in}}%
\pgfpathlineto{\pgfqpoint{2.845270in}{3.650960in}}%
\pgfpathlineto{\pgfqpoint{2.846344in}{3.651114in}}%
\pgfpathlineto{\pgfqpoint{2.850639in}{3.641524in}}%
\pgfpathlineto{\pgfqpoint{2.851713in}{3.648871in}}%
\pgfpathlineto{\pgfqpoint{2.852787in}{3.646164in}}%
\pgfpathlineto{\pgfqpoint{2.853861in}{3.651578in}}%
\pgfpathlineto{\pgfqpoint{2.857082in}{3.648253in}}%
\pgfpathlineto{\pgfqpoint{2.858156in}{3.665268in}}%
\pgfpathlineto{\pgfqpoint{2.859230in}{3.670759in}}%
\pgfpathlineto{\pgfqpoint{2.860304in}{3.680736in}}%
\pgfpathlineto{\pgfqpoint{2.861378in}{3.671455in}}%
\pgfpathlineto{\pgfqpoint{2.865673in}{3.646319in}}%
\pgfpathlineto{\pgfqpoint{2.866747in}{3.639513in}}%
\pgfpathlineto{\pgfqpoint{2.867821in}{3.638585in}}%
\pgfpathlineto{\pgfqpoint{2.868895in}{3.646087in}}%
\pgfpathlineto{\pgfqpoint{2.872116in}{3.652429in}}%
\pgfpathlineto{\pgfqpoint{2.873190in}{3.650960in}}%
\pgfpathlineto{\pgfqpoint{2.874264in}{3.648330in}}%
\pgfpathlineto{\pgfqpoint{2.875338in}{3.657070in}}%
\pgfpathlineto{\pgfqpoint{2.876411in}{3.655832in}}%
\pgfpathlineto{\pgfqpoint{2.879633in}{3.653512in}}%
\pgfpathlineto{\pgfqpoint{2.880707in}{3.649181in}}%
\pgfpathlineto{\pgfqpoint{2.881781in}{3.656219in}}%
\pgfpathlineto{\pgfqpoint{2.882855in}{3.655213in}}%
\pgfpathlineto{\pgfqpoint{2.883928in}{3.655291in}}%
\pgfpathlineto{\pgfqpoint{2.887150in}{3.658075in}}%
\pgfpathlineto{\pgfqpoint{2.888224in}{3.657302in}}%
\pgfpathlineto{\pgfqpoint{2.889298in}{3.663102in}}%
\pgfpathlineto{\pgfqpoint{2.890371in}{3.654827in}}%
\pgfpathlineto{\pgfqpoint{2.891445in}{3.651810in}}%
\pgfpathlineto{\pgfqpoint{2.894667in}{3.657843in}}%
\pgfpathlineto{\pgfqpoint{2.895741in}{3.666969in}}%
\pgfpathlineto{\pgfqpoint{2.896814in}{3.653048in}}%
\pgfpathlineto{\pgfqpoint{2.897888in}{3.653744in}}%
\pgfpathlineto{\pgfqpoint{2.898962in}{3.650960in}}%
\pgfpathlineto{\pgfqpoint{2.902184in}{3.651424in}}%
\pgfpathlineto{\pgfqpoint{2.903257in}{3.654904in}}%
\pgfpathlineto{\pgfqpoint{2.904331in}{3.646474in}}%
\pgfpathlineto{\pgfqpoint{2.905405in}{3.656064in}}%
\pgfpathlineto{\pgfqpoint{2.906479in}{3.646164in}}%
\pgfpathlineto{\pgfqpoint{2.910774in}{3.637812in}}%
\pgfpathlineto{\pgfqpoint{2.911848in}{3.643535in}}%
\pgfpathlineto{\pgfqpoint{2.912922in}{3.654749in}}%
\pgfpathlineto{\pgfqpoint{2.913996in}{3.645778in}}%
\pgfpathlineto{\pgfqpoint{2.917217in}{3.662252in}}%
\pgfpathlineto{\pgfqpoint{2.918291in}{3.657998in}}%
\pgfpathlineto{\pgfqpoint{2.919365in}{3.656683in}}%
\pgfpathlineto{\pgfqpoint{2.920439in}{3.669444in}}%
\pgfpathlineto{\pgfqpoint{2.924734in}{3.678339in}}%
\pgfpathlineto{\pgfqpoint{2.925808in}{3.677101in}}%
\pgfpathlineto{\pgfqpoint{2.927956in}{3.651501in}}%
\pgfpathlineto{\pgfqpoint{2.929030in}{3.648871in}}%
\pgfpathlineto{\pgfqpoint{2.932251in}{3.647247in}}%
\pgfpathlineto{\pgfqpoint{2.933325in}{3.645546in}}%
\pgfpathlineto{\pgfqpoint{2.934399in}{3.636729in}}%
\pgfpathlineto{\pgfqpoint{2.935473in}{3.634641in}}%
\pgfpathlineto{\pgfqpoint{2.936546in}{3.638662in}}%
\pgfpathlineto{\pgfqpoint{2.939768in}{3.647557in}}%
\pgfpathlineto{\pgfqpoint{2.941916in}{3.659931in}}%
\pgfpathlineto{\pgfqpoint{2.942989in}{3.662097in}}%
\pgfpathlineto{\pgfqpoint{2.944063in}{3.662329in}}%
\pgfpathlineto{\pgfqpoint{2.947285in}{3.664030in}}%
\pgfpathlineto{\pgfqpoint{2.948359in}{3.667356in}}%
\pgfpathlineto{\pgfqpoint{2.949433in}{3.687697in}}%
\pgfpathlineto{\pgfqpoint{2.950506in}{3.689012in}}%
\pgfpathlineto{\pgfqpoint{2.951580in}{3.686073in}}%
\pgfpathlineto{\pgfqpoint{2.954802in}{3.683752in}}%
\pgfpathlineto{\pgfqpoint{2.955876in}{3.718247in}}%
\pgfpathlineto{\pgfqpoint{2.956949in}{3.717473in}}%
\pgfpathlineto{\pgfqpoint{2.958023in}{3.727450in}}%
\pgfpathlineto{\pgfqpoint{2.962319in}{3.739129in}}%
\pgfpathlineto{\pgfqpoint{2.963392in}{3.724511in}}%
\pgfpathlineto{\pgfqpoint{2.964466in}{3.729848in}}%
\pgfpathlineto{\pgfqpoint{2.965540in}{3.725749in}}%
\pgfpathlineto{\pgfqpoint{2.966614in}{3.725594in}}%
\pgfpathlineto{\pgfqpoint{2.970909in}{3.708038in}}%
\pgfpathlineto{\pgfqpoint{2.971983in}{3.712292in}}%
\pgfpathlineto{\pgfqpoint{2.973057in}{3.711905in}}%
\pgfpathlineto{\pgfqpoint{2.974131in}{3.712988in}}%
\pgfpathlineto{\pgfqpoint{2.977352in}{3.710667in}}%
\pgfpathlineto{\pgfqpoint{2.978426in}{3.710745in}}%
\pgfpathlineto{\pgfqpoint{2.979500in}{3.722191in}}%
\pgfpathlineto{\pgfqpoint{2.981648in}{3.709198in}}%
\pgfpathlineto{\pgfqpoint{2.984869in}{3.710667in}}%
\pgfpathlineto{\pgfqpoint{2.988091in}{3.704093in}}%
\pgfpathlineto{\pgfqpoint{2.989165in}{3.697674in}}%
\pgfpathlineto{\pgfqpoint{2.992386in}{3.697133in}}%
\pgfpathlineto{\pgfqpoint{2.993460in}{3.700149in}}%
\pgfpathlineto{\pgfqpoint{2.994534in}{3.692415in}}%
\pgfpathlineto{\pgfqpoint{2.999903in}{3.704635in}}%
\pgfpathlineto{\pgfqpoint{3.000977in}{3.718866in}}%
\pgfpathlineto{\pgfqpoint{3.002051in}{3.717319in}}%
\pgfpathlineto{\pgfqpoint{3.003124in}{3.713761in}}%
\pgfpathlineto{\pgfqpoint{3.004198in}{3.718634in}}%
\pgfpathlineto{\pgfqpoint{3.007420in}{3.711363in}}%
\pgfpathlineto{\pgfqpoint{3.008494in}{3.716313in}}%
\pgfpathlineto{\pgfqpoint{3.009567in}{3.726600in}}%
\pgfpathlineto{\pgfqpoint{3.010641in}{3.718788in}}%
\pgfpathlineto{\pgfqpoint{3.011715in}{3.723119in}}%
\pgfpathlineto{\pgfqpoint{3.014937in}{3.727218in}}%
\pgfpathlineto{\pgfqpoint{3.016011in}{3.738046in}}%
\pgfpathlineto{\pgfqpoint{3.017084in}{3.740212in}}%
\pgfpathlineto{\pgfqpoint{3.018158in}{3.731008in}}%
\pgfpathlineto{\pgfqpoint{3.019232in}{3.737041in}}%
\pgfpathlineto{\pgfqpoint{3.022454in}{3.732091in}}%
\pgfpathlineto{\pgfqpoint{3.023527in}{3.731936in}}%
\pgfpathlineto{\pgfqpoint{3.024601in}{3.727373in}}%
\pgfpathlineto{\pgfqpoint{3.025675in}{3.726600in}}%
\pgfpathlineto{\pgfqpoint{3.026749in}{3.719639in}}%
\pgfpathlineto{\pgfqpoint{3.031044in}{3.719407in}}%
\pgfpathlineto{\pgfqpoint{3.032118in}{3.722965in}}%
\pgfpathlineto{\pgfqpoint{3.033192in}{3.722887in}}%
\pgfpathlineto{\pgfqpoint{3.034266in}{3.716700in}}%
\pgfpathlineto{\pgfqpoint{3.034266in}{3.716700in}}%
\pgfusepath{stroke}%
\end{pgfscope}%
\begin{pgfscope}%
\pgfpathrectangle{\pgfqpoint{0.568357in}{3.037108in}}{\pgfqpoint{2.583333in}{0.736585in}}%
\pgfusepath{clip}%
\pgfsetroundcap%
\pgfsetroundjoin%
\pgfsetlinewidth{1.505625pt}%
\definecolor{currentstroke}{rgb}{1.000000,0.647059,0.000000}%
\pgfsetstrokecolor{currentstroke}%
\pgfsetdash{}{0pt}%
\pgfpathmoveto{\pgfqpoint{0.685781in}{3.088919in}}%
\pgfpathlineto{\pgfqpoint{0.689002in}{3.085941in}}%
\pgfpathlineto{\pgfqpoint{0.696519in}{3.085086in}}%
\pgfpathlineto{\pgfqpoint{0.714775in}{3.085239in}}%
\pgfpathlineto{\pgfqpoint{0.719070in}{3.085697in}}%
\pgfpathlineto{\pgfqpoint{0.730882in}{3.085067in}}%
\pgfpathlineto{\pgfqpoint{0.734104in}{3.084838in}}%
\pgfpathlineto{\pgfqpoint{0.763097in}{3.085117in}}%
\pgfpathlineto{\pgfqpoint{0.791017in}{3.085940in}}%
\pgfpathlineto{\pgfqpoint{0.813568in}{3.085117in}}%
\pgfpathlineto{\pgfqpoint{0.824306in}{3.085175in}}%
\pgfpathlineto{\pgfqpoint{0.843635in}{3.084357in}}%
\pgfpathlineto{\pgfqpoint{0.854374in}{3.083662in}}%
\pgfpathlineto{\pgfqpoint{0.861890in}{3.083767in}}%
\pgfpathlineto{\pgfqpoint{0.873703in}{3.084753in}}%
\pgfpathlineto{\pgfqpoint{0.891958in}{3.086861in}}%
\pgfpathlineto{\pgfqpoint{0.898401in}{3.087733in}}%
\pgfpathlineto{\pgfqpoint{0.906992in}{3.088746in}}%
\pgfpathlineto{\pgfqpoint{0.918804in}{3.090017in}}%
\pgfpathlineto{\pgfqpoint{0.929542in}{3.091136in}}%
\pgfpathlineto{\pgfqpoint{0.943502in}{3.092121in}}%
\pgfpathlineto{\pgfqpoint{0.959610in}{3.093337in}}%
\pgfpathlineto{\pgfqpoint{0.971422in}{3.094315in}}%
\pgfpathlineto{\pgfqpoint{0.982160in}{3.095427in}}%
\pgfpathlineto{\pgfqpoint{0.995046in}{3.096393in}}%
\pgfpathlineto{\pgfqpoint{1.000416in}{3.097033in}}%
\pgfpathlineto{\pgfqpoint{1.004711in}{3.097719in}}%
\pgfpathlineto{\pgfqpoint{1.016523in}{3.098552in}}%
\pgfpathlineto{\pgfqpoint{1.027262in}{3.099475in}}%
\pgfpathlineto{\pgfqpoint{1.042295in}{3.100485in}}%
\pgfpathlineto{\pgfqpoint{1.054108in}{3.101424in}}%
\pgfpathlineto{\pgfqpoint{1.064846in}{3.102537in}}%
\pgfpathlineto{\pgfqpoint{1.083101in}{3.103648in}}%
\pgfpathlineto{\pgfqpoint{1.102430in}{3.105788in}}%
\pgfpathlineto{\pgfqpoint{1.113169in}{3.106912in}}%
\pgfpathlineto{\pgfqpoint{1.124981in}{3.108804in}}%
\pgfpathlineto{\pgfqpoint{1.135719in}{3.109941in}}%
\pgfpathlineto{\pgfqpoint{1.147531in}{3.112071in}}%
\pgfpathlineto{\pgfqpoint{1.153975in}{3.113120in}}%
\pgfpathlineto{\pgfqpoint{1.162565in}{3.114708in}}%
\pgfpathlineto{\pgfqpoint{1.167934in}{3.115568in}}%
\pgfpathlineto{\pgfqpoint{1.169008in}{3.115871in}}%
\pgfpathlineto{\pgfqpoint{1.175451in}{3.116812in}}%
\pgfpathlineto{\pgfqpoint{1.177599in}{3.117428in}}%
\pgfpathlineto{\pgfqpoint{1.182968in}{3.118297in}}%
\pgfpathlineto{\pgfqpoint{1.185116in}{3.118908in}}%
\pgfpathlineto{\pgfqpoint{1.190485in}{3.119838in}}%
\pgfpathlineto{\pgfqpoint{1.200150in}{3.122172in}}%
\pgfpathlineto{\pgfqpoint{1.205519in}{3.123162in}}%
\pgfpathlineto{\pgfqpoint{1.207666in}{3.123830in}}%
\pgfpathlineto{\pgfqpoint{1.213036in}{3.124803in}}%
\pgfpathlineto{\pgfqpoint{1.215183in}{3.125451in}}%
\pgfpathlineto{\pgfqpoint{1.220552in}{3.126487in}}%
\pgfpathlineto{\pgfqpoint{1.222700in}{3.127203in}}%
\pgfpathlineto{\pgfqpoint{1.228069in}{3.128295in}}%
\pgfpathlineto{\pgfqpoint{1.230217in}{3.128987in}}%
\pgfpathlineto{\pgfqpoint{1.236660in}{3.129977in}}%
\pgfpathlineto{\pgfqpoint{1.242029in}{3.130834in}}%
\pgfpathlineto{\pgfqpoint{1.252768in}{3.133085in}}%
\pgfpathlineto{\pgfqpoint{1.259211in}{3.134196in}}%
\pgfpathlineto{\pgfqpoint{1.267801in}{3.135880in}}%
\pgfpathlineto{\pgfqpoint{1.278540in}{3.137401in}}%
\pgfpathlineto{\pgfqpoint{1.282835in}{3.138737in}}%
\pgfpathlineto{\pgfqpoint{1.288204in}{3.139765in}}%
\pgfpathlineto{\pgfqpoint{1.290352in}{3.140472in}}%
\pgfpathlineto{\pgfqpoint{1.295721in}{3.141576in}}%
\pgfpathlineto{\pgfqpoint{1.297869in}{3.142314in}}%
\pgfpathlineto{\pgfqpoint{1.303238in}{3.143438in}}%
\pgfpathlineto{\pgfqpoint{1.305386in}{3.144203in}}%
\pgfpathlineto{\pgfqpoint{1.310755in}{3.145321in}}%
\pgfpathlineto{\pgfqpoint{1.312903in}{3.146027in}}%
\pgfpathlineto{\pgfqpoint{1.318272in}{3.147028in}}%
\pgfpathlineto{\pgfqpoint{1.320419in}{3.147608in}}%
\pgfpathlineto{\pgfqpoint{1.326863in}{3.148743in}}%
\pgfpathlineto{\pgfqpoint{1.331158in}{3.149280in}}%
\pgfpathlineto{\pgfqpoint{1.335453in}{3.150242in}}%
\pgfpathlineto{\pgfqpoint{1.341896in}{3.150968in}}%
\pgfpathlineto{\pgfqpoint{1.350487in}{3.152535in}}%
\pgfpathlineto{\pgfqpoint{1.356930in}{3.153628in}}%
\pgfpathlineto{\pgfqpoint{1.361225in}{3.154162in}}%
\pgfpathlineto{\pgfqpoint{1.365521in}{3.155089in}}%
\pgfpathlineto{\pgfqpoint{1.371964in}{3.155983in}}%
\pgfpathlineto{\pgfqpoint{1.380554in}{3.157303in}}%
\pgfpathlineto{\pgfqpoint{1.385924in}{3.158090in}}%
\pgfpathlineto{\pgfqpoint{1.388071in}{3.158653in}}%
\pgfpathlineto{\pgfqpoint{1.393440in}{3.159492in}}%
\pgfpathlineto{\pgfqpoint{1.403105in}{3.161486in}}%
\pgfpathlineto{\pgfqpoint{1.409548in}{3.162611in}}%
\pgfpathlineto{\pgfqpoint{1.410622in}{3.162906in}}%
\pgfpathlineto{\pgfqpoint{1.417065in}{3.164057in}}%
\pgfpathlineto{\pgfqpoint{1.418139in}{3.164350in}}%
\pgfpathlineto{\pgfqpoint{1.423508in}{3.165238in}}%
\pgfpathlineto{\pgfqpoint{1.425656in}{3.165845in}}%
\pgfpathlineto{\pgfqpoint{1.436394in}{3.167344in}}%
\pgfpathlineto{\pgfqpoint{1.440689in}{3.168457in}}%
\pgfpathlineto{\pgfqpoint{1.447132in}{3.169523in}}%
\pgfpathlineto{\pgfqpoint{1.455723in}{3.170965in}}%
\pgfpathlineto{\pgfqpoint{1.466462in}{3.172179in}}%
\pgfpathlineto{\pgfqpoint{1.470757in}{3.172865in}}%
\pgfpathlineto{\pgfqpoint{1.477200in}{3.173893in}}%
\pgfpathlineto{\pgfqpoint{1.485791in}{3.175472in}}%
\pgfpathlineto{\pgfqpoint{1.496529in}{3.176605in}}%
\pgfpathlineto{\pgfqpoint{1.505120in}{3.177637in}}%
\pgfpathlineto{\pgfqpoint{1.589953in}{3.190468in}}%
\pgfpathlineto{\pgfqpoint{1.598544in}{3.192128in}}%
\pgfpathlineto{\pgfqpoint{1.604987in}{3.193202in}}%
\pgfpathlineto{\pgfqpoint{1.613577in}{3.194821in}}%
\pgfpathlineto{\pgfqpoint{1.620020in}{3.195889in}}%
\pgfpathlineto{\pgfqpoint{1.621094in}{3.196163in}}%
\pgfpathlineto{\pgfqpoint{1.627537in}{3.196966in}}%
\pgfpathlineto{\pgfqpoint{1.636128in}{3.198674in}}%
\pgfpathlineto{\pgfqpoint{1.641497in}{3.199549in}}%
\pgfpathlineto{\pgfqpoint{1.651162in}{3.201521in}}%
\pgfpathlineto{\pgfqpoint{1.656531in}{3.202417in}}%
\pgfpathlineto{\pgfqpoint{1.658679in}{3.203016in}}%
\pgfpathlineto{\pgfqpoint{1.664048in}{3.203910in}}%
\pgfpathlineto{\pgfqpoint{1.665122in}{3.204207in}}%
\pgfpathlineto{\pgfqpoint{1.671565in}{3.205110in}}%
\pgfpathlineto{\pgfqpoint{1.673712in}{3.205695in}}%
\pgfpathlineto{\pgfqpoint{1.680155in}{3.206741in}}%
\pgfpathlineto{\pgfqpoint{1.688746in}{3.208235in}}%
\pgfpathlineto{\pgfqpoint{1.695189in}{3.209200in}}%
\pgfpathlineto{\pgfqpoint{1.703780in}{3.210541in}}%
\pgfpathlineto{\pgfqpoint{1.710223in}{3.211467in}}%
\pgfpathlineto{\pgfqpoint{1.718814in}{3.212935in}}%
\pgfpathlineto{\pgfqpoint{1.725257in}{3.213935in}}%
\pgfpathlineto{\pgfqpoint{1.726330in}{3.214188in}}%
\pgfpathlineto{\pgfqpoint{1.732773in}{3.214942in}}%
\pgfpathlineto{\pgfqpoint{1.741364in}{3.216474in}}%
\pgfpathlineto{\pgfqpoint{1.746733in}{3.217265in}}%
\pgfpathlineto{\pgfqpoint{1.756398in}{3.219227in}}%
\pgfpathlineto{\pgfqpoint{1.762841in}{3.220239in}}%
\pgfpathlineto{\pgfqpoint{1.770358in}{3.221418in}}%
\pgfpathlineto{\pgfqpoint{1.777875in}{3.222317in}}%
\pgfpathlineto{\pgfqpoint{1.793982in}{3.224742in}}%
\pgfpathlineto{\pgfqpoint{1.800425in}{3.225808in}}%
\pgfpathlineto{\pgfqpoint{1.809016in}{3.227394in}}%
\pgfpathlineto{\pgfqpoint{1.815459in}{3.228432in}}%
\pgfpathlineto{\pgfqpoint{1.821902in}{3.229434in}}%
\pgfpathlineto{\pgfqpoint{1.836936in}{3.231713in}}%
\pgfpathlineto{\pgfqpoint{1.843379in}{3.232581in}}%
\pgfpathlineto{\pgfqpoint{1.851970in}{3.233913in}}%
\pgfpathlineto{\pgfqpoint{1.875594in}{3.236849in}}%
\pgfpathlineto{\pgfqpoint{1.876668in}{3.237050in}}%
\pgfpathlineto{\pgfqpoint{1.887406in}{3.237966in}}%
\pgfpathlineto{\pgfqpoint{1.899218in}{3.239531in}}%
\pgfpathlineto{\pgfqpoint{1.912105in}{3.240618in}}%
\pgfpathlineto{\pgfqpoint{1.929286in}{3.242675in}}%
\pgfpathlineto{\pgfqpoint{1.941098in}{3.243747in}}%
\pgfpathlineto{\pgfqpoint{1.951837in}{3.245059in}}%
\pgfpathlineto{\pgfqpoint{1.965796in}{3.246265in}}%
\pgfpathlineto{\pgfqpoint{1.974387in}{3.247170in}}%
\pgfpathlineto{\pgfqpoint{1.986199in}{3.248210in}}%
\pgfpathlineto{\pgfqpoint{1.996938in}{3.249341in}}%
\pgfpathlineto{\pgfqpoint{2.008750in}{3.250424in}}%
\pgfpathlineto{\pgfqpoint{2.017341in}{3.251208in}}%
\pgfpathlineto{\pgfqpoint{2.027005in}{3.252187in}}%
\pgfpathlineto{\pgfqpoint{2.039891in}{3.253191in}}%
\pgfpathlineto{\pgfqpoint{2.049556in}{3.254129in}}%
\pgfpathlineto{\pgfqpoint{2.064590in}{3.255213in}}%
\pgfpathlineto{\pgfqpoint{2.077476in}{3.256283in}}%
\pgfpathlineto{\pgfqpoint{2.087140in}{3.257158in}}%
\pgfpathlineto{\pgfqpoint{2.101100in}{3.258255in}}%
\pgfpathlineto{\pgfqpoint{2.112912in}{3.259021in}}%
\pgfpathlineto{\pgfqpoint{2.196672in}{3.264118in}}%
\pgfpathlineto{\pgfqpoint{2.212779in}{3.265548in}}%
\pgfpathlineto{\pgfqpoint{2.236404in}{3.267589in}}%
\pgfpathlineto{\pgfqpoint{2.243921in}{3.268259in}}%
\pgfpathlineto{\pgfqpoint{2.258954in}{3.269269in}}%
\pgfpathlineto{\pgfqpoint{2.273988in}{3.270107in}}%
\pgfpathlineto{\pgfqpoint{2.340566in}{3.276091in}}%
\pgfpathlineto{\pgfqpoint{2.350231in}{3.277065in}}%
\pgfpathlineto{\pgfqpoint{2.362043in}{3.278209in}}%
\pgfpathlineto{\pgfqpoint{2.380298in}{3.280548in}}%
\pgfpathlineto{\pgfqpoint{2.391036in}{3.281637in}}%
\pgfpathlineto{\pgfqpoint{2.402849in}{3.283308in}}%
\pgfpathlineto{\pgfqpoint{2.414661in}{3.284397in}}%
\pgfpathlineto{\pgfqpoint{2.425399in}{3.285963in}}%
\pgfpathlineto{\pgfqpoint{2.436138in}{3.287165in}}%
\pgfpathlineto{\pgfqpoint{2.447950in}{3.289031in}}%
\pgfpathlineto{\pgfqpoint{2.455467in}{3.289965in}}%
\pgfpathlineto{\pgfqpoint{2.461910in}{3.290903in}}%
\pgfpathlineto{\pgfqpoint{2.470501in}{3.292352in}}%
\pgfpathlineto{\pgfqpoint{2.476944in}{3.293319in}}%
\pgfpathlineto{\pgfqpoint{2.485534in}{3.294748in}}%
\pgfpathlineto{\pgfqpoint{2.491977in}{3.295667in}}%
\pgfpathlineto{\pgfqpoint{2.500568in}{3.296956in}}%
\pgfpathlineto{\pgfqpoint{2.511306in}{3.298177in}}%
\pgfpathlineto{\pgfqpoint{2.515602in}{3.298988in}}%
\pgfpathlineto{\pgfqpoint{2.526340in}{3.299992in}}%
\pgfpathlineto{\pgfqpoint{2.538152in}{3.301730in}}%
\pgfpathlineto{\pgfqpoint{2.548891in}{3.302877in}}%
\pgfpathlineto{\pgfqpoint{2.560703in}{3.304584in}}%
\pgfpathlineto{\pgfqpoint{2.572515in}{3.305741in}}%
\pgfpathlineto{\pgfqpoint{2.583254in}{3.307063in}}%
\pgfpathlineto{\pgfqpoint{2.593992in}{3.308144in}}%
\pgfpathlineto{\pgfqpoint{2.605804in}{3.309445in}}%
\pgfpathlineto{\pgfqpoint{2.618690in}{3.310602in}}%
\pgfpathlineto{\pgfqpoint{2.635872in}{3.312522in}}%
\pgfpathlineto{\pgfqpoint{2.648758in}{3.313487in}}%
\pgfpathlineto{\pgfqpoint{2.658422in}{3.314606in}}%
\pgfpathlineto{\pgfqpoint{2.670234in}{3.315497in}}%
\pgfpathlineto{\pgfqpoint{2.688490in}{3.317361in}}%
\pgfpathlineto{\pgfqpoint{2.701376in}{3.318527in}}%
\pgfpathlineto{\pgfqpoint{2.711040in}{3.319901in}}%
\pgfpathlineto{\pgfqpoint{2.717483in}{3.320715in}}%
\pgfpathlineto{\pgfqpoint{2.726074in}{3.322033in}}%
\pgfpathlineto{\pgfqpoint{2.736812in}{3.323302in}}%
\pgfpathlineto{\pgfqpoint{2.748625in}{3.325107in}}%
\pgfpathlineto{\pgfqpoint{2.760437in}{3.326281in}}%
\pgfpathlineto{\pgfqpoint{2.771175in}{3.327757in}}%
\pgfpathlineto{\pgfqpoint{2.781914in}{3.328878in}}%
\pgfpathlineto{\pgfqpoint{2.793726in}{3.330629in}}%
\pgfpathlineto{\pgfqpoint{2.805538in}{3.331860in}}%
\pgfpathlineto{\pgfqpoint{2.816277in}{3.333625in}}%
\pgfpathlineto{\pgfqpoint{2.822720in}{3.334548in}}%
\pgfpathlineto{\pgfqpoint{2.831310in}{3.336005in}}%
\pgfpathlineto{\pgfqpoint{2.837753in}{3.336958in}}%
\pgfpathlineto{\pgfqpoint{2.842049in}{3.337412in}}%
\pgfpathlineto{\pgfqpoint{2.858156in}{3.339659in}}%
\pgfpathlineto{\pgfqpoint{2.868895in}{3.341465in}}%
\pgfpathlineto{\pgfqpoint{2.875338in}{3.342349in}}%
\pgfpathlineto{\pgfqpoint{2.883928in}{3.343674in}}%
\pgfpathlineto{\pgfqpoint{2.890371in}{3.344563in}}%
\pgfpathlineto{\pgfqpoint{2.898962in}{3.345876in}}%
\pgfpathlineto{\pgfqpoint{2.905405in}{3.346735in}}%
\pgfpathlineto{\pgfqpoint{2.906479in}{3.346945in}}%
\pgfpathlineto{\pgfqpoint{2.917217in}{3.347999in}}%
\pgfpathlineto{\pgfqpoint{2.929030in}{3.349971in}}%
\pgfpathlineto{\pgfqpoint{2.939768in}{3.351181in}}%
\pgfpathlineto{\pgfqpoint{2.951580in}{3.353153in}}%
\pgfpathlineto{\pgfqpoint{2.958023in}{3.354135in}}%
\pgfpathlineto{\pgfqpoint{2.966614in}{3.355670in}}%
\pgfpathlineto{\pgfqpoint{2.973057in}{3.356638in}}%
\pgfpathlineto{\pgfqpoint{2.981648in}{3.358088in}}%
\pgfpathlineto{\pgfqpoint{2.988091in}{3.359032in}}%
\pgfpathlineto{\pgfqpoint{2.994534in}{3.359943in}}%
\pgfpathlineto{\pgfqpoint{3.009567in}{3.362076in}}%
\pgfpathlineto{\pgfqpoint{3.019232in}{3.363797in}}%
\pgfpathlineto{\pgfqpoint{3.025675in}{3.364770in}}%
\pgfpathlineto{\pgfqpoint{3.026749in}{3.365005in}}%
\pgfpathlineto{\pgfqpoint{3.034266in}{3.365948in}}%
\pgfpathlineto{\pgfqpoint{3.034266in}{3.365948in}}%
\pgfusepath{stroke}%
\end{pgfscope}%
\begin{pgfscope}%
\pgfsetrectcap%
\pgfsetmiterjoin%
\pgfsetlinewidth{0.803000pt}%
\definecolor{currentstroke}{rgb}{1.000000,1.000000,1.000000}%
\pgfsetstrokecolor{currentstroke}%
\pgfsetdash{}{0pt}%
\pgfpathmoveto{\pgfqpoint{0.568357in}{3.037108in}}%
\pgfpathlineto{\pgfqpoint{0.568357in}{3.773693in}}%
\pgfusepath{stroke}%
\end{pgfscope}%
\begin{pgfscope}%
\pgfsetrectcap%
\pgfsetmiterjoin%
\pgfsetlinewidth{0.803000pt}%
\definecolor{currentstroke}{rgb}{1.000000,1.000000,1.000000}%
\pgfsetstrokecolor{currentstroke}%
\pgfsetdash{}{0pt}%
\pgfpathmoveto{\pgfqpoint{3.151690in}{3.037108in}}%
\pgfpathlineto{\pgfqpoint{3.151690in}{3.773693in}}%
\pgfusepath{stroke}%
\end{pgfscope}%
\begin{pgfscope}%
\pgfsetrectcap%
\pgfsetmiterjoin%
\pgfsetlinewidth{0.803000pt}%
\definecolor{currentstroke}{rgb}{1.000000,1.000000,1.000000}%
\pgfsetstrokecolor{currentstroke}%
\pgfsetdash{}{0pt}%
\pgfpathmoveto{\pgfqpoint{0.568357in}{3.037108in}}%
\pgfpathlineto{\pgfqpoint{3.151690in}{3.037108in}}%
\pgfusepath{stroke}%
\end{pgfscope}%
\begin{pgfscope}%
\pgfsetrectcap%
\pgfsetmiterjoin%
\pgfsetlinewidth{0.803000pt}%
\definecolor{currentstroke}{rgb}{1.000000,1.000000,1.000000}%
\pgfsetstrokecolor{currentstroke}%
\pgfsetdash{}{0pt}%
\pgfpathmoveto{\pgfqpoint{0.568357in}{3.773693in}}%
\pgfpathlineto{\pgfqpoint{3.151690in}{3.773693in}}%
\pgfusepath{stroke}%
\end{pgfscope}%
\begin{pgfscope}%
\definecolor{textcolor}{rgb}{0.150000,0.150000,0.150000}%
\pgfsetstrokecolor{textcolor}%
\pgfsetfillcolor{textcolor}%
\pgftext[x=1.860023in,y=3.857026in,,base]{\color{textcolor}\rmfamily\fontsize{16.800000}{20.160000}\selectfont JNJ}%
\end{pgfscope}%
\begin{pgfscope}%
\pgfsetbuttcap%
\pgfsetmiterjoin%
\definecolor{currentfill}{rgb}{0.917647,0.917647,0.949020}%
\pgfsetfillcolor{currentfill}%
\pgfsetlinewidth{0.000000pt}%
\definecolor{currentstroke}{rgb}{0.000000,0.000000,0.000000}%
\pgfsetstrokecolor{currentstroke}%
\pgfsetstrokeopacity{0.000000}%
\pgfsetdash{}{0pt}%
\pgfpathmoveto{\pgfqpoint{4.185023in}{3.037108in}}%
\pgfpathlineto{\pgfqpoint{6.768357in}{3.037108in}}%
\pgfpathlineto{\pgfqpoint{6.768357in}{3.773693in}}%
\pgfpathlineto{\pgfqpoint{4.185023in}{3.773693in}}%
\pgfpathclose%
\pgfusepath{fill}%
\end{pgfscope}%
\begin{pgfscope}%
\pgfpathrectangle{\pgfqpoint{4.185023in}{3.037108in}}{\pgfqpoint{2.583333in}{0.736585in}}%
\pgfusepath{clip}%
\pgfsetroundcap%
\pgfsetroundjoin%
\pgfsetlinewidth{0.803000pt}%
\definecolor{currentstroke}{rgb}{1.000000,1.000000,1.000000}%
\pgfsetstrokecolor{currentstroke}%
\pgfsetdash{}{0pt}%
\pgfpathmoveto{\pgfqpoint{4.300300in}{3.037108in}}%
\pgfpathlineto{\pgfqpoint{4.300300in}{3.773693in}}%
\pgfusepath{stroke}%
\end{pgfscope}%
\begin{pgfscope}%
\definecolor{textcolor}{rgb}{0.150000,0.150000,0.150000}%
\pgfsetstrokecolor{textcolor}%
\pgfsetfillcolor{textcolor}%
\pgftext[x=4.300300in,y=2.939885in,,top]{\color{textcolor}\rmfamily\fontsize{14.000000}{16.800000}\selectfont 2012}%
\end{pgfscope}%
\begin{pgfscope}%
\pgfpathrectangle{\pgfqpoint{4.185023in}{3.037108in}}{\pgfqpoint{2.583333in}{0.736585in}}%
\pgfusepath{clip}%
\pgfsetroundcap%
\pgfsetroundjoin%
\pgfsetlinewidth{0.803000pt}%
\definecolor{currentstroke}{rgb}{1.000000,1.000000,1.000000}%
\pgfsetstrokecolor{currentstroke}%
\pgfsetdash{}{0pt}%
\pgfpathmoveto{\pgfqpoint{4.693325in}{3.037108in}}%
\pgfpathlineto{\pgfqpoint{4.693325in}{3.773693in}}%
\pgfusepath{stroke}%
\end{pgfscope}%
\begin{pgfscope}%
\definecolor{textcolor}{rgb}{0.150000,0.150000,0.150000}%
\pgfsetstrokecolor{textcolor}%
\pgfsetfillcolor{textcolor}%
\pgftext[x=4.693325in,y=2.939885in,,top]{\color{textcolor}\rmfamily\fontsize{14.000000}{16.800000}\selectfont 2013}%
\end{pgfscope}%
\begin{pgfscope}%
\pgfpathrectangle{\pgfqpoint{4.185023in}{3.037108in}}{\pgfqpoint{2.583333in}{0.736585in}}%
\pgfusepath{clip}%
\pgfsetroundcap%
\pgfsetroundjoin%
\pgfsetlinewidth{0.803000pt}%
\definecolor{currentstroke}{rgb}{1.000000,1.000000,1.000000}%
\pgfsetstrokecolor{currentstroke}%
\pgfsetdash{}{0pt}%
\pgfpathmoveto{\pgfqpoint{5.085276in}{3.037108in}}%
\pgfpathlineto{\pgfqpoint{5.085276in}{3.773693in}}%
\pgfusepath{stroke}%
\end{pgfscope}%
\begin{pgfscope}%
\definecolor{textcolor}{rgb}{0.150000,0.150000,0.150000}%
\pgfsetstrokecolor{textcolor}%
\pgfsetfillcolor{textcolor}%
\pgftext[x=5.085276in,y=2.939885in,,top]{\color{textcolor}\rmfamily\fontsize{14.000000}{16.800000}\selectfont 2014}%
\end{pgfscope}%
\begin{pgfscope}%
\pgfpathrectangle{\pgfqpoint{4.185023in}{3.037108in}}{\pgfqpoint{2.583333in}{0.736585in}}%
\pgfusepath{clip}%
\pgfsetroundcap%
\pgfsetroundjoin%
\pgfsetlinewidth{0.803000pt}%
\definecolor{currentstroke}{rgb}{1.000000,1.000000,1.000000}%
\pgfsetstrokecolor{currentstroke}%
\pgfsetdash{}{0pt}%
\pgfpathmoveto{\pgfqpoint{5.477227in}{3.037108in}}%
\pgfpathlineto{\pgfqpoint{5.477227in}{3.773693in}}%
\pgfusepath{stroke}%
\end{pgfscope}%
\begin{pgfscope}%
\definecolor{textcolor}{rgb}{0.150000,0.150000,0.150000}%
\pgfsetstrokecolor{textcolor}%
\pgfsetfillcolor{textcolor}%
\pgftext[x=5.477227in,y=2.939885in,,top]{\color{textcolor}\rmfamily\fontsize{14.000000}{16.800000}\selectfont 2015}%
\end{pgfscope}%
\begin{pgfscope}%
\pgfpathrectangle{\pgfqpoint{4.185023in}{3.037108in}}{\pgfqpoint{2.583333in}{0.736585in}}%
\pgfusepath{clip}%
\pgfsetroundcap%
\pgfsetroundjoin%
\pgfsetlinewidth{0.803000pt}%
\definecolor{currentstroke}{rgb}{1.000000,1.000000,1.000000}%
\pgfsetstrokecolor{currentstroke}%
\pgfsetdash{}{0pt}%
\pgfpathmoveto{\pgfqpoint{5.869178in}{3.037108in}}%
\pgfpathlineto{\pgfqpoint{5.869178in}{3.773693in}}%
\pgfusepath{stroke}%
\end{pgfscope}%
\begin{pgfscope}%
\definecolor{textcolor}{rgb}{0.150000,0.150000,0.150000}%
\pgfsetstrokecolor{textcolor}%
\pgfsetfillcolor{textcolor}%
\pgftext[x=5.869178in,y=2.939885in,,top]{\color{textcolor}\rmfamily\fontsize{14.000000}{16.800000}\selectfont 2016}%
\end{pgfscope}%
\begin{pgfscope}%
\pgfpathrectangle{\pgfqpoint{4.185023in}{3.037108in}}{\pgfqpoint{2.583333in}{0.736585in}}%
\pgfusepath{clip}%
\pgfsetroundcap%
\pgfsetroundjoin%
\pgfsetlinewidth{0.803000pt}%
\definecolor{currentstroke}{rgb}{1.000000,1.000000,1.000000}%
\pgfsetstrokecolor{currentstroke}%
\pgfsetdash{}{0pt}%
\pgfpathmoveto{\pgfqpoint{6.262203in}{3.037108in}}%
\pgfpathlineto{\pgfqpoint{6.262203in}{3.773693in}}%
\pgfusepath{stroke}%
\end{pgfscope}%
\begin{pgfscope}%
\definecolor{textcolor}{rgb}{0.150000,0.150000,0.150000}%
\pgfsetstrokecolor{textcolor}%
\pgfsetfillcolor{textcolor}%
\pgftext[x=6.262203in,y=2.939885in,,top]{\color{textcolor}\rmfamily\fontsize{14.000000}{16.800000}\selectfont 2017}%
\end{pgfscope}%
\begin{pgfscope}%
\pgfpathrectangle{\pgfqpoint{4.185023in}{3.037108in}}{\pgfqpoint{2.583333in}{0.736585in}}%
\pgfusepath{clip}%
\pgfsetroundcap%
\pgfsetroundjoin%
\pgfsetlinewidth{0.803000pt}%
\definecolor{currentstroke}{rgb}{1.000000,1.000000,1.000000}%
\pgfsetstrokecolor{currentstroke}%
\pgfsetdash{}{0pt}%
\pgfpathmoveto{\pgfqpoint{6.654154in}{3.037108in}}%
\pgfpathlineto{\pgfqpoint{6.654154in}{3.773693in}}%
\pgfusepath{stroke}%
\end{pgfscope}%
\begin{pgfscope}%
\definecolor{textcolor}{rgb}{0.150000,0.150000,0.150000}%
\pgfsetstrokecolor{textcolor}%
\pgfsetfillcolor{textcolor}%
\pgftext[x=6.654154in,y=2.939885in,,top]{\color{textcolor}\rmfamily\fontsize{14.000000}{16.800000}\selectfont 2018}%
\end{pgfscope}%
\begin{pgfscope}%
\pgfpathrectangle{\pgfqpoint{4.185023in}{3.037108in}}{\pgfqpoint{2.583333in}{0.736585in}}%
\pgfusepath{clip}%
\pgfsetroundcap%
\pgfsetroundjoin%
\pgfsetlinewidth{0.803000pt}%
\definecolor{currentstroke}{rgb}{1.000000,1.000000,1.000000}%
\pgfsetstrokecolor{currentstroke}%
\pgfsetdash{}{0pt}%
\pgfpathmoveto{\pgfqpoint{4.185023in}{3.117393in}}%
\pgfpathlineto{\pgfqpoint{6.768357in}{3.117393in}}%
\pgfusepath{stroke}%
\end{pgfscope}%
\begin{pgfscope}%
\definecolor{textcolor}{rgb}{0.150000,0.150000,0.150000}%
\pgfsetstrokecolor{textcolor}%
\pgfsetfillcolor{textcolor}%
\pgftext[x=3.840378in,y=3.043527in,left,base]{\color{textcolor}\rmfamily\fontsize{14.000000}{16.800000}\selectfont 50}%
\end{pgfscope}%
\begin{pgfscope}%
\pgfpathrectangle{\pgfqpoint{4.185023in}{3.037108in}}{\pgfqpoint{2.583333in}{0.736585in}}%
\pgfusepath{clip}%
\pgfsetroundcap%
\pgfsetroundjoin%
\pgfsetlinewidth{0.803000pt}%
\definecolor{currentstroke}{rgb}{1.000000,1.000000,1.000000}%
\pgfsetstrokecolor{currentstroke}%
\pgfsetdash{}{0pt}%
\pgfpathmoveto{\pgfqpoint{4.185023in}{3.520878in}}%
\pgfpathlineto{\pgfqpoint{6.768357in}{3.520878in}}%
\pgfusepath{stroke}%
\end{pgfscope}%
\begin{pgfscope}%
\definecolor{textcolor}{rgb}{0.150000,0.150000,0.150000}%
\pgfsetstrokecolor{textcolor}%
\pgfsetfillcolor{textcolor}%
\pgftext[x=3.840378in,y=3.447011in,left,base]{\color{textcolor}\rmfamily\fontsize{14.000000}{16.800000}\selectfont 75}%
\end{pgfscope}%
\begin{pgfscope}%
\pgfpathrectangle{\pgfqpoint{4.185023in}{3.037108in}}{\pgfqpoint{2.583333in}{0.736585in}}%
\pgfusepath{clip}%
\pgfsetroundcap%
\pgfsetroundjoin%
\pgfsetlinewidth{1.505625pt}%
\definecolor{currentstroke}{rgb}{0.549020,0.337255,0.294118}%
\pgfsetstrokecolor{currentstroke}%
\pgfsetdash{}{0pt}%
\pgfpathmoveto{\pgfqpoint{4.302448in}{3.153707in}}%
\pgfpathlineto{\pgfqpoint{4.303521in}{3.153222in}}%
\pgfpathlineto{\pgfqpoint{4.305669in}{3.147735in}}%
\pgfpathlineto{\pgfqpoint{4.308891in}{3.151286in}}%
\pgfpathlineto{\pgfqpoint{4.309964in}{3.147251in}}%
\pgfpathlineto{\pgfqpoint{4.311038in}{3.139181in}}%
\pgfpathlineto{\pgfqpoint{4.312112in}{3.140795in}}%
\pgfpathlineto{\pgfqpoint{4.313186in}{3.140795in}}%
\pgfpathlineto{\pgfqpoint{4.317481in}{3.146444in}}%
\pgfpathlineto{\pgfqpoint{4.318555in}{3.150156in}}%
\pgfpathlineto{\pgfqpoint{4.319629in}{3.150801in}}%
\pgfpathlineto{\pgfqpoint{4.320703in}{3.152738in}}%
\pgfpathlineto{\pgfqpoint{4.324998in}{3.130789in}}%
\pgfpathlineto{\pgfqpoint{4.326072in}{3.136922in}}%
\pgfpathlineto{\pgfqpoint{4.327146in}{3.134501in}}%
\pgfpathlineto{\pgfqpoint{4.328220in}{3.128206in}}%
\pgfpathlineto{\pgfqpoint{4.331441in}{3.114326in}}%
\pgfpathlineto{\pgfqpoint{4.332515in}{3.112228in}}%
\pgfpathlineto{\pgfqpoint{4.334663in}{3.115779in}}%
\pgfpathlineto{\pgfqpoint{4.335737in}{3.108678in}}%
\pgfpathlineto{\pgfqpoint{4.340032in}{3.120621in}}%
\pgfpathlineto{\pgfqpoint{4.341106in}{3.119814in}}%
\pgfpathlineto{\pgfqpoint{4.342180in}{3.124817in}}%
\pgfpathlineto{\pgfqpoint{4.343253in}{3.122880in}}%
\pgfpathlineto{\pgfqpoint{4.346475in}{3.127238in}}%
\pgfpathlineto{\pgfqpoint{4.347549in}{3.130466in}}%
\pgfpathlineto{\pgfqpoint{4.348623in}{3.131434in}}%
\pgfpathlineto{\pgfqpoint{4.349696in}{3.139665in}}%
\pgfpathlineto{\pgfqpoint{4.350770in}{3.135953in}}%
\pgfpathlineto{\pgfqpoint{4.355066in}{3.129659in}}%
\pgfpathlineto{\pgfqpoint{4.356140in}{3.129982in}}%
\pgfpathlineto{\pgfqpoint{4.357213in}{3.155159in}}%
\pgfpathlineto{\pgfqpoint{4.358287in}{3.158871in}}%
\pgfpathlineto{\pgfqpoint{4.361509in}{3.158710in}}%
\pgfpathlineto{\pgfqpoint{4.362583in}{3.167425in}}%
\pgfpathlineto{\pgfqpoint{4.363656in}{3.170492in}}%
\pgfpathlineto{\pgfqpoint{4.364730in}{3.158226in}}%
\pgfpathlineto{\pgfqpoint{4.365804in}{3.158387in}}%
\pgfpathlineto{\pgfqpoint{4.369026in}{3.161938in}}%
\pgfpathlineto{\pgfqpoint{4.370099in}{3.160485in}}%
\pgfpathlineto{\pgfqpoint{4.371173in}{3.157257in}}%
\pgfpathlineto{\pgfqpoint{4.372247in}{3.161292in}}%
\pgfpathlineto{\pgfqpoint{4.373321in}{3.161615in}}%
\pgfpathlineto{\pgfqpoint{4.377616in}{3.174042in}}%
\pgfpathlineto{\pgfqpoint{4.378690in}{3.173397in}}%
\pgfpathlineto{\pgfqpoint{4.379764in}{3.171137in}}%
\pgfpathlineto{\pgfqpoint{4.380838in}{3.165650in}}%
\pgfpathlineto{\pgfqpoint{4.386207in}{3.165004in}}%
\pgfpathlineto{\pgfqpoint{4.387281in}{3.169200in}}%
\pgfpathlineto{\pgfqpoint{4.388355in}{3.168071in}}%
\pgfpathlineto{\pgfqpoint{4.391576in}{3.168393in}}%
\pgfpathlineto{\pgfqpoint{4.392650in}{3.164520in}}%
\pgfpathlineto{\pgfqpoint{4.393724in}{3.165004in}}%
\pgfpathlineto{\pgfqpoint{4.394798in}{3.162583in}}%
\pgfpathlineto{\pgfqpoint{4.395872in}{3.165166in}}%
\pgfpathlineto{\pgfqpoint{4.399093in}{3.169685in}}%
\pgfpathlineto{\pgfqpoint{4.400167in}{3.163713in}}%
\pgfpathlineto{\pgfqpoint{4.401241in}{3.165811in}}%
\pgfpathlineto{\pgfqpoint{4.402315in}{3.166457in}}%
\pgfpathlineto{\pgfqpoint{4.406610in}{3.160162in}}%
\pgfpathlineto{\pgfqpoint{4.407684in}{3.154191in}}%
\pgfpathlineto{\pgfqpoint{4.408758in}{3.155482in}}%
\pgfpathlineto{\pgfqpoint{4.410905in}{3.147412in}}%
\pgfpathlineto{\pgfqpoint{4.415201in}{3.162745in}}%
\pgfpathlineto{\pgfqpoint{4.417348in}{3.157096in}}%
\pgfpathlineto{\pgfqpoint{4.418422in}{3.169039in}}%
\pgfpathlineto{\pgfqpoint{4.421644in}{3.158064in}}%
\pgfpathlineto{\pgfqpoint{4.423791in}{3.168393in}}%
\pgfpathlineto{\pgfqpoint{4.424865in}{3.168071in}}%
\pgfpathlineto{\pgfqpoint{4.425939in}{3.136922in}}%
\pgfpathlineto{\pgfqpoint{4.429161in}{3.126592in}}%
\pgfpathlineto{\pgfqpoint{4.430234in}{3.125785in}}%
\pgfpathlineto{\pgfqpoint{4.432382in}{3.137729in}}%
\pgfpathlineto{\pgfqpoint{4.433456in}{3.134823in}}%
\pgfpathlineto{\pgfqpoint{4.436677in}{3.134501in}}%
\pgfpathlineto{\pgfqpoint{4.437751in}{3.133371in}}%
\pgfpathlineto{\pgfqpoint{4.438825in}{3.127077in}}%
\pgfpathlineto{\pgfqpoint{4.439899in}{3.133048in}}%
\pgfpathlineto{\pgfqpoint{4.440973in}{3.127238in}}%
\pgfpathlineto{\pgfqpoint{4.444194in}{3.125947in}}%
\pgfpathlineto{\pgfqpoint{4.445268in}{3.127722in}}%
\pgfpathlineto{\pgfqpoint{4.446342in}{3.134985in}}%
\pgfpathlineto{\pgfqpoint{4.448490in}{3.125140in}}%
\pgfpathlineto{\pgfqpoint{4.451711in}{3.123365in}}%
\pgfpathlineto{\pgfqpoint{4.452785in}{3.120298in}}%
\pgfpathlineto{\pgfqpoint{4.453859in}{3.110614in}}%
\pgfpathlineto{\pgfqpoint{4.454933in}{3.112874in}}%
\pgfpathlineto{\pgfqpoint{4.456006in}{3.111906in}}%
\pgfpathlineto{\pgfqpoint{4.460302in}{3.117877in}}%
\pgfpathlineto{\pgfqpoint{4.461376in}{3.109646in}}%
\pgfpathlineto{\pgfqpoint{4.462450in}{3.109323in}}%
\pgfpathlineto{\pgfqpoint{4.463523in}{3.099801in}}%
\pgfpathlineto{\pgfqpoint{4.466745in}{3.097864in}}%
\pgfpathlineto{\pgfqpoint{4.467819in}{3.094959in}}%
\pgfpathlineto{\pgfqpoint{4.471040in}{3.115295in}}%
\pgfpathlineto{\pgfqpoint{4.474262in}{3.112551in}}%
\pgfpathlineto{\pgfqpoint{4.475336in}{3.115295in}}%
\pgfpathlineto{\pgfqpoint{4.476409in}{3.112874in}}%
\pgfpathlineto{\pgfqpoint{4.477483in}{3.120782in}}%
\pgfpathlineto{\pgfqpoint{4.478557in}{3.116909in}}%
\pgfpathlineto{\pgfqpoint{4.481779in}{3.109323in}}%
\pgfpathlineto{\pgfqpoint{4.482852in}{3.108355in}}%
\pgfpathlineto{\pgfqpoint{4.483926in}{3.084953in}}%
\pgfpathlineto{\pgfqpoint{4.485000in}{3.076722in}}%
\pgfpathlineto{\pgfqpoint{4.486074in}{3.077851in}}%
\pgfpathlineto{\pgfqpoint{4.489295in}{3.071073in}}%
\pgfpathlineto{\pgfqpoint{4.490369in}{3.070589in}}%
\pgfpathlineto{\pgfqpoint{4.491443in}{3.079788in}}%
\pgfpathlineto{\pgfqpoint{4.492517in}{3.083662in}}%
\pgfpathlineto{\pgfqpoint{4.493591in}{3.095928in}}%
\pgfpathlineto{\pgfqpoint{4.496812in}{3.095282in}}%
\pgfpathlineto{\pgfqpoint{4.497886in}{3.097380in}}%
\pgfpathlineto{\pgfqpoint{4.500034in}{3.097219in}}%
\pgfpathlineto{\pgfqpoint{4.501108in}{3.096412in}}%
\pgfpathlineto{\pgfqpoint{4.504329in}{3.099801in}}%
\pgfpathlineto{\pgfqpoint{4.505403in}{3.102222in}}%
\pgfpathlineto{\pgfqpoint{4.506477in}{3.097864in}}%
\pgfpathlineto{\pgfqpoint{4.508625in}{3.145314in}}%
\pgfpathlineto{\pgfqpoint{4.511846in}{3.141602in}}%
\pgfpathlineto{\pgfqpoint{4.512920in}{3.148542in}}%
\pgfpathlineto{\pgfqpoint{4.513994in}{3.149026in}}%
\pgfpathlineto{\pgfqpoint{4.515068in}{3.150317in}}%
\pgfpathlineto{\pgfqpoint{4.516141in}{3.147896in}}%
\pgfpathlineto{\pgfqpoint{4.519363in}{3.143377in}}%
\pgfpathlineto{\pgfqpoint{4.520437in}{3.138697in}}%
\pgfpathlineto{\pgfqpoint{4.521511in}{3.138697in}}%
\pgfpathlineto{\pgfqpoint{4.523658in}{3.152415in}}%
\pgfpathlineto{\pgfqpoint{4.526880in}{3.152577in}}%
\pgfpathlineto{\pgfqpoint{4.530101in}{3.132080in}}%
\pgfpathlineto{\pgfqpoint{4.531175in}{3.157741in}}%
\pgfpathlineto{\pgfqpoint{4.534397in}{3.161776in}}%
\pgfpathlineto{\pgfqpoint{4.536544in}{3.173558in}}%
\pgfpathlineto{\pgfqpoint{4.538692in}{3.174204in}}%
\pgfpathlineto{\pgfqpoint{4.541914in}{3.170492in}}%
\pgfpathlineto{\pgfqpoint{4.542987in}{3.173719in}}%
\pgfpathlineto{\pgfqpoint{4.544061in}{3.172590in}}%
\pgfpathlineto{\pgfqpoint{4.545135in}{3.177270in}}%
\pgfpathlineto{\pgfqpoint{4.546209in}{3.177270in}}%
\pgfpathlineto{\pgfqpoint{4.549430in}{3.174042in}}%
\pgfpathlineto{\pgfqpoint{4.550504in}{3.174204in}}%
\pgfpathlineto{\pgfqpoint{4.551578in}{3.175333in}}%
\pgfpathlineto{\pgfqpoint{4.552652in}{3.173074in}}%
\pgfpathlineto{\pgfqpoint{4.553726in}{3.177431in}}%
\pgfpathlineto{\pgfqpoint{4.556947in}{3.178561in}}%
\pgfpathlineto{\pgfqpoint{4.559095in}{3.175495in}}%
\pgfpathlineto{\pgfqpoint{4.560169in}{3.175656in}}%
\pgfpathlineto{\pgfqpoint{4.561243in}{3.179691in}}%
\pgfpathlineto{\pgfqpoint{4.565538in}{3.182435in}}%
\pgfpathlineto{\pgfqpoint{4.566612in}{3.180982in}}%
\pgfpathlineto{\pgfqpoint{4.567686in}{3.193248in}}%
\pgfpathlineto{\pgfqpoint{4.568760in}{3.196799in}}%
\pgfpathlineto{\pgfqpoint{4.571981in}{3.196799in}}%
\pgfpathlineto{\pgfqpoint{4.574129in}{3.191473in}}%
\pgfpathlineto{\pgfqpoint{4.575203in}{3.201963in}}%
\pgfpathlineto{\pgfqpoint{4.576276in}{3.205191in}}%
\pgfpathlineto{\pgfqpoint{4.579498in}{3.206321in}}%
\pgfpathlineto{\pgfqpoint{4.580572in}{3.205998in}}%
\pgfpathlineto{\pgfqpoint{4.581646in}{3.206482in}}%
\pgfpathlineto{\pgfqpoint{4.582719in}{3.210356in}}%
\pgfpathlineto{\pgfqpoint{4.583793in}{3.208580in}}%
\pgfpathlineto{\pgfqpoint{4.587015in}{3.212938in}}%
\pgfpathlineto{\pgfqpoint{4.588089in}{3.210679in}}%
\pgfpathlineto{\pgfqpoint{4.589162in}{3.206967in}}%
\pgfpathlineto{\pgfqpoint{4.590236in}{3.206967in}}%
\pgfpathlineto{\pgfqpoint{4.591310in}{3.207774in}}%
\pgfpathlineto{\pgfqpoint{4.594532in}{3.208903in}}%
\pgfpathlineto{\pgfqpoint{4.595605in}{3.200349in}}%
\pgfpathlineto{\pgfqpoint{4.597753in}{3.207935in}}%
\pgfpathlineto{\pgfqpoint{4.598827in}{3.211163in}}%
\pgfpathlineto{\pgfqpoint{4.602049in}{3.204384in}}%
\pgfpathlineto{\pgfqpoint{4.605270in}{3.190182in}}%
\pgfpathlineto{\pgfqpoint{4.606344in}{3.189375in}}%
\pgfpathlineto{\pgfqpoint{4.610639in}{3.203093in}}%
\pgfpathlineto{\pgfqpoint{4.611713in}{3.216489in}}%
\pgfpathlineto{\pgfqpoint{4.612787in}{3.216489in}}%
\pgfpathlineto{\pgfqpoint{4.613861in}{3.204868in}}%
\pgfpathlineto{\pgfqpoint{4.617082in}{3.203900in}}%
\pgfpathlineto{\pgfqpoint{4.618156in}{3.190020in}}%
\pgfpathlineto{\pgfqpoint{4.619230in}{3.198413in}}%
\pgfpathlineto{\pgfqpoint{4.620304in}{3.224397in}}%
\pgfpathlineto{\pgfqpoint{4.621378in}{3.216166in}}%
\pgfpathlineto{\pgfqpoint{4.628894in}{3.212938in}}%
\pgfpathlineto{\pgfqpoint{4.632116in}{3.206160in}}%
\pgfpathlineto{\pgfqpoint{4.633190in}{3.209549in}}%
\pgfpathlineto{\pgfqpoint{4.635338in}{3.183080in}}%
\pgfpathlineto{\pgfqpoint{4.636411in}{3.184371in}}%
\pgfpathlineto{\pgfqpoint{4.639633in}{3.185340in}}%
\pgfpathlineto{\pgfqpoint{4.642854in}{3.175495in}}%
\pgfpathlineto{\pgfqpoint{4.643928in}{3.181951in}}%
\pgfpathlineto{\pgfqpoint{4.649297in}{3.203416in}}%
\pgfpathlineto{\pgfqpoint{4.651445in}{3.218103in}}%
\pgfpathlineto{\pgfqpoint{4.654667in}{3.216650in}}%
\pgfpathlineto{\pgfqpoint{4.655740in}{3.210194in}}%
\pgfpathlineto{\pgfqpoint{4.656814in}{3.216166in}}%
\pgfpathlineto{\pgfqpoint{4.657888in}{3.216973in}}%
\pgfpathlineto{\pgfqpoint{4.658962in}{3.221169in}}%
\pgfpathlineto{\pgfqpoint{4.662183in}{3.217941in}}%
\pgfpathlineto{\pgfqpoint{4.663257in}{3.214391in}}%
\pgfpathlineto{\pgfqpoint{4.664331in}{3.215682in}}%
\pgfpathlineto{\pgfqpoint{4.666479in}{3.227302in}}%
\pgfpathlineto{\pgfqpoint{4.669700in}{3.226334in}}%
\pgfpathlineto{\pgfqpoint{4.670774in}{3.231983in}}%
\pgfpathlineto{\pgfqpoint{4.671848in}{3.233435in}}%
\pgfpathlineto{\pgfqpoint{4.672922in}{3.225204in}}%
\pgfpathlineto{\pgfqpoint{4.673996in}{3.222460in}}%
\pgfpathlineto{\pgfqpoint{4.677217in}{3.222460in}}%
\pgfpathlineto{\pgfqpoint{4.678291in}{3.223106in}}%
\pgfpathlineto{\pgfqpoint{4.679365in}{3.214875in}}%
\pgfpathlineto{\pgfqpoint{4.680439in}{3.221169in}}%
\pgfpathlineto{\pgfqpoint{4.681513in}{3.206805in}}%
\pgfpathlineto{\pgfqpoint{4.684734in}{3.204061in}}%
\pgfpathlineto{\pgfqpoint{4.686882in}{3.197283in}}%
\pgfpathlineto{\pgfqpoint{4.687956in}{3.196960in}}%
\pgfpathlineto{\pgfqpoint{4.689029in}{3.186308in}}%
\pgfpathlineto{\pgfqpoint{4.692251in}{3.195992in}}%
\pgfpathlineto{\pgfqpoint{4.694399in}{3.215520in}}%
\pgfpathlineto{\pgfqpoint{4.695472in}{3.209710in}}%
\pgfpathlineto{\pgfqpoint{4.696546in}{3.211647in}}%
\pgfpathlineto{\pgfqpoint{4.700842in}{3.204061in}}%
\pgfpathlineto{\pgfqpoint{4.702989in}{3.213906in}}%
\pgfpathlineto{\pgfqpoint{4.704063in}{3.213261in}}%
\pgfpathlineto{\pgfqpoint{4.707285in}{3.218587in}}%
\pgfpathlineto{\pgfqpoint{4.708359in}{3.221815in}}%
\pgfpathlineto{\pgfqpoint{4.709432in}{3.222138in}}%
\pgfpathlineto{\pgfqpoint{4.711580in}{3.230046in}}%
\pgfpathlineto{\pgfqpoint{4.715875in}{3.230207in}}%
\pgfpathlineto{\pgfqpoint{4.716949in}{3.239891in}}%
\pgfpathlineto{\pgfqpoint{4.718023in}{3.236340in}}%
\pgfpathlineto{\pgfqpoint{4.719097in}{3.273622in}}%
\pgfpathlineto{\pgfqpoint{4.722318in}{3.280401in}}%
\pgfpathlineto{\pgfqpoint{4.723392in}{3.296540in}}%
\pgfpathlineto{\pgfqpoint{4.725540in}{3.298638in}}%
\pgfpathlineto{\pgfqpoint{4.726614in}{3.308645in}}%
\pgfpathlineto{\pgfqpoint{4.729835in}{3.299929in}}%
\pgfpathlineto{\pgfqpoint{4.731983in}{3.311711in}}%
\pgfpathlineto{\pgfqpoint{4.733057in}{3.311711in}}%
\pgfpathlineto{\pgfqpoint{4.734131in}{3.306385in}}%
\pgfpathlineto{\pgfqpoint{4.737352in}{3.307192in}}%
\pgfpathlineto{\pgfqpoint{4.738426in}{3.309452in}}%
\pgfpathlineto{\pgfqpoint{4.739500in}{3.317037in}}%
\pgfpathlineto{\pgfqpoint{4.740574in}{3.319942in}}%
\pgfpathlineto{\pgfqpoint{4.741648in}{3.316876in}}%
\pgfpathlineto{\pgfqpoint{4.745943in}{3.327851in}}%
\pgfpathlineto{\pgfqpoint{4.747017in}{3.323977in}}%
\pgfpathlineto{\pgfqpoint{4.749164in}{3.322686in}}%
\pgfpathlineto{\pgfqpoint{4.752386in}{3.308645in}}%
\pgfpathlineto{\pgfqpoint{4.753460in}{3.310743in}}%
\pgfpathlineto{\pgfqpoint{4.754534in}{3.319619in}}%
\pgfpathlineto{\pgfqpoint{4.755607in}{3.312034in}}%
\pgfpathlineto{\pgfqpoint{4.756681in}{3.316230in}}%
\pgfpathlineto{\pgfqpoint{4.759903in}{3.318651in}}%
\pgfpathlineto{\pgfqpoint{4.760977in}{3.323493in}}%
\pgfpathlineto{\pgfqpoint{4.762050in}{3.325591in}}%
\pgfpathlineto{\pgfqpoint{4.763124in}{3.321556in}}%
\pgfpathlineto{\pgfqpoint{4.764198in}{3.325268in}}%
\pgfpathlineto{\pgfqpoint{4.767420in}{3.327528in}}%
\pgfpathlineto{\pgfqpoint{4.768493in}{3.325107in}}%
\pgfpathlineto{\pgfqpoint{4.769567in}{3.320265in}}%
\pgfpathlineto{\pgfqpoint{4.770641in}{3.328012in}}%
\pgfpathlineto{\pgfqpoint{4.771715in}{3.314132in}}%
\pgfpathlineto{\pgfqpoint{4.774937in}{3.311873in}}%
\pgfpathlineto{\pgfqpoint{4.777084in}{3.330594in}}%
\pgfpathlineto{\pgfqpoint{4.778158in}{3.325591in}}%
\pgfpathlineto{\pgfqpoint{4.779232in}{3.326398in}}%
\pgfpathlineto{\pgfqpoint{4.782453in}{3.318651in}}%
\pgfpathlineto{\pgfqpoint{4.783527in}{3.328173in}}%
\pgfpathlineto{\pgfqpoint{4.784601in}{3.323654in}}%
\pgfpathlineto{\pgfqpoint{4.785675in}{3.323654in}}%
\pgfpathlineto{\pgfqpoint{4.789970in}{3.332047in}}%
\pgfpathlineto{\pgfqpoint{4.791044in}{3.348670in}}%
\pgfpathlineto{\pgfqpoint{4.792118in}{3.337696in}}%
\pgfpathlineto{\pgfqpoint{4.793192in}{3.343183in}}%
\pgfpathlineto{\pgfqpoint{4.794266in}{3.338987in}}%
\pgfpathlineto{\pgfqpoint{4.797487in}{3.346411in}}%
\pgfpathlineto{\pgfqpoint{4.798561in}{3.339471in}}%
\pgfpathlineto{\pgfqpoint{4.799635in}{3.352382in}}%
\pgfpathlineto{\pgfqpoint{4.801782in}{3.363357in}}%
\pgfpathlineto{\pgfqpoint{4.805004in}{3.357708in}}%
\pgfpathlineto{\pgfqpoint{4.806078in}{3.363680in}}%
\pgfpathlineto{\pgfqpoint{4.807152in}{3.349961in}}%
\pgfpathlineto{\pgfqpoint{4.808226in}{3.360613in}}%
\pgfpathlineto{\pgfqpoint{4.809299in}{3.381111in}}%
\pgfpathlineto{\pgfqpoint{4.812521in}{3.380788in}}%
\pgfpathlineto{\pgfqpoint{4.813595in}{3.395797in}}%
\pgfpathlineto{\pgfqpoint{4.814669in}{3.331885in}}%
\pgfpathlineto{\pgfqpoint{4.815742in}{3.324784in}}%
\pgfpathlineto{\pgfqpoint{4.816816in}{3.331724in}}%
\pgfpathlineto{\pgfqpoint{4.820038in}{3.339309in}}%
\pgfpathlineto{\pgfqpoint{4.821112in}{3.327205in}}%
\pgfpathlineto{\pgfqpoint{4.822185in}{3.330110in}}%
\pgfpathlineto{\pgfqpoint{4.824333in}{3.346088in}}%
\pgfpathlineto{\pgfqpoint{4.827555in}{3.340278in}}%
\pgfpathlineto{\pgfqpoint{4.828628in}{3.342699in}}%
\pgfpathlineto{\pgfqpoint{4.829702in}{3.349316in}}%
\pgfpathlineto{\pgfqpoint{4.830776in}{3.346411in}}%
\pgfpathlineto{\pgfqpoint{4.831850in}{3.353674in}}%
\pgfpathlineto{\pgfqpoint{4.835071in}{3.351414in}}%
\pgfpathlineto{\pgfqpoint{4.837219in}{3.379012in}}%
\pgfpathlineto{\pgfqpoint{4.838293in}{3.372718in}}%
\pgfpathlineto{\pgfqpoint{4.839367in}{3.370297in}}%
\pgfpathlineto{\pgfqpoint{4.843662in}{3.354158in}}%
\pgfpathlineto{\pgfqpoint{4.844736in}{3.354481in}}%
\pgfpathlineto{\pgfqpoint{4.845810in}{3.352867in}}%
\pgfpathlineto{\pgfqpoint{4.846884in}{3.394990in}}%
\pgfpathlineto{\pgfqpoint{4.851179in}{3.381433in}}%
\pgfpathlineto{\pgfqpoint{4.852253in}{3.355449in}}%
\pgfpathlineto{\pgfqpoint{4.853327in}{3.358031in}}%
\pgfpathlineto{\pgfqpoint{4.854401in}{3.327205in}}%
\pgfpathlineto{\pgfqpoint{4.857622in}{3.338987in}}%
\pgfpathlineto{\pgfqpoint{4.858696in}{3.335275in}}%
\pgfpathlineto{\pgfqpoint{4.859770in}{3.325752in}}%
\pgfpathlineto{\pgfqpoint{4.860844in}{3.328012in}}%
\pgfpathlineto{\pgfqpoint{4.861917in}{3.340278in}}%
\pgfpathlineto{\pgfqpoint{4.866213in}{3.345120in}}%
\pgfpathlineto{\pgfqpoint{4.867287in}{3.338341in}}%
\pgfpathlineto{\pgfqpoint{4.868360in}{3.349316in}}%
\pgfpathlineto{\pgfqpoint{4.869434in}{3.343990in}}%
\pgfpathlineto{\pgfqpoint{4.872656in}{3.356256in}}%
\pgfpathlineto{\pgfqpoint{4.873730in}{3.357386in}}%
\pgfpathlineto{\pgfqpoint{4.874804in}{3.338341in}}%
\pgfpathlineto{\pgfqpoint{4.875877in}{3.307192in}}%
\pgfpathlineto{\pgfqpoint{4.876951in}{3.336082in}}%
\pgfpathlineto{\pgfqpoint{4.880173in}{3.324784in}}%
\pgfpathlineto{\pgfqpoint{4.881247in}{3.326237in}}%
\pgfpathlineto{\pgfqpoint{4.882320in}{3.335597in}}%
\pgfpathlineto{\pgfqpoint{4.883394in}{3.339148in}}%
\pgfpathlineto{\pgfqpoint{4.884468in}{3.330110in}}%
\pgfpathlineto{\pgfqpoint{4.889837in}{3.351091in}}%
\pgfpathlineto{\pgfqpoint{4.891985in}{3.348025in}}%
\pgfpathlineto{\pgfqpoint{4.895206in}{3.353674in}}%
\pgfpathlineto{\pgfqpoint{4.896280in}{3.364326in}}%
\pgfpathlineto{\pgfqpoint{4.897354in}{3.367392in}}%
\pgfpathlineto{\pgfqpoint{4.899502in}{3.390633in}}%
\pgfpathlineto{\pgfqpoint{4.902723in}{3.389987in}}%
\pgfpathlineto{\pgfqpoint{4.904871in}{3.379174in}}%
\pgfpathlineto{\pgfqpoint{4.905945in}{3.381917in}}%
\pgfpathlineto{\pgfqpoint{4.907019in}{3.396282in}}%
\pgfpathlineto{\pgfqpoint{4.910240in}{3.394345in}}%
\pgfpathlineto{\pgfqpoint{4.911314in}{3.390633in}}%
\pgfpathlineto{\pgfqpoint{4.912388in}{3.381595in}}%
\pgfpathlineto{\pgfqpoint{4.913462in}{3.383531in}}%
\pgfpathlineto{\pgfqpoint{4.914536in}{3.383209in}}%
\pgfpathlineto{\pgfqpoint{4.917757in}{3.379174in}}%
\pgfpathlineto{\pgfqpoint{4.918831in}{3.383854in}}%
\pgfpathlineto{\pgfqpoint{4.919905in}{3.381917in}}%
\pgfpathlineto{\pgfqpoint{4.920979in}{3.399832in}}%
\pgfpathlineto{\pgfqpoint{4.922052in}{3.395152in}}%
\pgfpathlineto{\pgfqpoint{4.925274in}{3.396604in}}%
\pgfpathlineto{\pgfqpoint{4.927422in}{3.404190in}}%
\pgfpathlineto{\pgfqpoint{4.928495in}{3.406934in}}%
\pgfpathlineto{\pgfqpoint{4.929569in}{3.399832in}}%
\pgfpathlineto{\pgfqpoint{4.933865in}{3.400155in}}%
\pgfpathlineto{\pgfqpoint{4.934938in}{3.394668in}}%
\pgfpathlineto{\pgfqpoint{4.937086in}{3.376591in}}%
\pgfpathlineto{\pgfqpoint{4.942455in}{3.369652in}}%
\pgfpathlineto{\pgfqpoint{4.944603in}{3.378044in}}%
\pgfpathlineto{\pgfqpoint{4.948898in}{3.350930in}}%
\pgfpathlineto{\pgfqpoint{4.949972in}{3.335920in}}%
\pgfpathlineto{\pgfqpoint{4.952120in}{3.349800in}}%
\pgfpathlineto{\pgfqpoint{4.956415in}{3.348025in}}%
\pgfpathlineto{\pgfqpoint{4.958563in}{3.339794in}}%
\pgfpathlineto{\pgfqpoint{4.959637in}{3.339955in}}%
\pgfpathlineto{\pgfqpoint{4.962858in}{3.353512in}}%
\pgfpathlineto{\pgfqpoint{4.963932in}{3.350607in}}%
\pgfpathlineto{\pgfqpoint{4.965006in}{3.354965in}}%
\pgfpathlineto{\pgfqpoint{4.966080in}{3.354803in}}%
\pgfpathlineto{\pgfqpoint{4.967154in}{3.365294in}}%
\pgfpathlineto{\pgfqpoint{4.970375in}{3.380142in}}%
\pgfpathlineto{\pgfqpoint{4.971449in}{3.375785in}}%
\pgfpathlineto{\pgfqpoint{4.972523in}{3.381917in}}%
\pgfpathlineto{\pgfqpoint{4.973597in}{3.379658in}}%
\pgfpathlineto{\pgfqpoint{4.974670in}{3.369813in}}%
\pgfpathlineto{\pgfqpoint{4.977892in}{3.368360in}}%
\pgfpathlineto{\pgfqpoint{4.980040in}{3.347541in}}%
\pgfpathlineto{\pgfqpoint{4.981114in}{3.351898in}}%
\pgfpathlineto{\pgfqpoint{4.982187in}{3.340762in}}%
\pgfpathlineto{\pgfqpoint{4.985409in}{3.319135in}}%
\pgfpathlineto{\pgfqpoint{4.986483in}{3.326721in}}%
\pgfpathlineto{\pgfqpoint{4.987557in}{3.323654in}}%
\pgfpathlineto{\pgfqpoint{4.988630in}{3.322525in}}%
\pgfpathlineto{\pgfqpoint{4.989704in}{3.324945in}}%
\pgfpathlineto{\pgfqpoint{4.992926in}{3.319942in}}%
\pgfpathlineto{\pgfqpoint{4.997221in}{3.357708in}}%
\pgfpathlineto{\pgfqpoint{5.000443in}{3.361098in}}%
\pgfpathlineto{\pgfqpoint{5.001516in}{3.345927in}}%
\pgfpathlineto{\pgfqpoint{5.003664in}{3.378528in}}%
\pgfpathlineto{\pgfqpoint{5.004738in}{3.378367in}}%
\pgfpathlineto{\pgfqpoint{5.007959in}{3.372557in}}%
\pgfpathlineto{\pgfqpoint{5.009033in}{3.391440in}}%
\pgfpathlineto{\pgfqpoint{5.010107in}{3.398541in}}%
\pgfpathlineto{\pgfqpoint{5.011181in}{3.394506in}}%
\pgfpathlineto{\pgfqpoint{5.012255in}{3.386275in}}%
\pgfpathlineto{\pgfqpoint{5.015476in}{3.403867in}}%
\pgfpathlineto{\pgfqpoint{5.016550in}{3.419361in}}%
\pgfpathlineto{\pgfqpoint{5.018698in}{3.396443in}}%
\pgfpathlineto{\pgfqpoint{5.019772in}{3.401769in}}%
\pgfpathlineto{\pgfqpoint{5.024067in}{3.405481in}}%
\pgfpathlineto{\pgfqpoint{5.025141in}{3.424041in}}%
\pgfpathlineto{\pgfqpoint{5.026215in}{3.417586in}}%
\pgfpathlineto{\pgfqpoint{5.027289in}{3.420006in}}%
\pgfpathlineto{\pgfqpoint{5.030510in}{3.416779in}}%
\pgfpathlineto{\pgfqpoint{5.032658in}{3.433402in}}%
\pgfpathlineto{\pgfqpoint{5.034805in}{3.451478in}}%
\pgfpathlineto{\pgfqpoint{5.038027in}{3.447766in}}%
\pgfpathlineto{\pgfqpoint{5.039101in}{3.444700in}}%
\pgfpathlineto{\pgfqpoint{5.040175in}{3.449380in}}%
\pgfpathlineto{\pgfqpoint{5.041248in}{3.449057in}}%
\pgfpathlineto{\pgfqpoint{5.042322in}{3.452931in}}%
\pgfpathlineto{\pgfqpoint{5.045544in}{3.459064in}}%
\pgfpathlineto{\pgfqpoint{5.046618in}{3.448735in}}%
\pgfpathlineto{\pgfqpoint{5.047692in}{3.443893in}}%
\pgfpathlineto{\pgfqpoint{5.049839in}{3.443086in}}%
\pgfpathlineto{\pgfqpoint{5.053061in}{3.431304in}}%
\pgfpathlineto{\pgfqpoint{5.054135in}{3.437760in}}%
\pgfpathlineto{\pgfqpoint{5.056282in}{3.422427in}}%
\pgfpathlineto{\pgfqpoint{5.057356in}{3.447121in}}%
\pgfpathlineto{\pgfqpoint{5.060578in}{3.450671in}}%
\pgfpathlineto{\pgfqpoint{5.061651in}{3.435339in}}%
\pgfpathlineto{\pgfqpoint{5.062725in}{3.440342in}}%
\pgfpathlineto{\pgfqpoint{5.063799in}{3.417263in}}%
\pgfpathlineto{\pgfqpoint{5.064873in}{3.418231in}}%
\pgfpathlineto{\pgfqpoint{5.068094in}{3.409032in}}%
\pgfpathlineto{\pgfqpoint{5.069168in}{3.398541in}}%
\pgfpathlineto{\pgfqpoint{5.070242in}{3.418392in}}%
\pgfpathlineto{\pgfqpoint{5.071316in}{3.411937in}}%
\pgfpathlineto{\pgfqpoint{5.072390in}{3.411130in}}%
\pgfpathlineto{\pgfqpoint{5.075611in}{3.403867in}}%
\pgfpathlineto{\pgfqpoint{5.076685in}{3.403867in}}%
\pgfpathlineto{\pgfqpoint{5.079907in}{3.413389in}}%
\pgfpathlineto{\pgfqpoint{5.083128in}{3.413228in}}%
\pgfpathlineto{\pgfqpoint{5.086350in}{3.393538in}}%
\pgfpathlineto{\pgfqpoint{5.087424in}{3.392408in}}%
\pgfpathlineto{\pgfqpoint{5.090645in}{3.394990in}}%
\pgfpathlineto{\pgfqpoint{5.091719in}{3.405481in}}%
\pgfpathlineto{\pgfqpoint{5.092793in}{3.389503in}}%
\pgfpathlineto{\pgfqpoint{5.093867in}{3.391924in}}%
\pgfpathlineto{\pgfqpoint{5.098162in}{3.386437in}}%
\pgfpathlineto{\pgfqpoint{5.099236in}{3.398057in}}%
\pgfpathlineto{\pgfqpoint{5.100310in}{3.396927in}}%
\pgfpathlineto{\pgfqpoint{5.101383in}{3.393861in}}%
\pgfpathlineto{\pgfqpoint{5.102457in}{3.384661in}}%
\pgfpathlineto{\pgfqpoint{5.106753in}{3.388696in}}%
\pgfpathlineto{\pgfqpoint{5.107826in}{3.384016in}}%
\pgfpathlineto{\pgfqpoint{5.108900in}{3.370620in}}%
\pgfpathlineto{\pgfqpoint{5.109974in}{3.383370in}}%
\pgfpathlineto{\pgfqpoint{5.113196in}{3.373686in}}%
\pgfpathlineto{\pgfqpoint{5.114269in}{3.382402in}}%
\pgfpathlineto{\pgfqpoint{5.116417in}{3.352060in}}%
\pgfpathlineto{\pgfqpoint{5.117491in}{3.348670in}}%
\pgfpathlineto{\pgfqpoint{5.120713in}{3.336243in}}%
\pgfpathlineto{\pgfqpoint{5.125008in}{3.358031in}}%
\pgfpathlineto{\pgfqpoint{5.128229in}{3.367715in}}%
\pgfpathlineto{\pgfqpoint{5.129303in}{3.378690in}}%
\pgfpathlineto{\pgfqpoint{5.130377in}{3.360452in}}%
\pgfpathlineto{\pgfqpoint{5.131451in}{3.364648in}}%
\pgfpathlineto{\pgfqpoint{5.132525in}{3.386275in}}%
\pgfpathlineto{\pgfqpoint{5.136820in}{3.366908in}}%
\pgfpathlineto{\pgfqpoint{5.137894in}{3.369329in}}%
\pgfpathlineto{\pgfqpoint{5.138968in}{3.366262in}}%
\pgfpathlineto{\pgfqpoint{5.140042in}{3.366908in}}%
\pgfpathlineto{\pgfqpoint{5.143263in}{3.365455in}}%
\pgfpathlineto{\pgfqpoint{5.144337in}{3.369006in}}%
\pgfpathlineto{\pgfqpoint{5.145411in}{3.365455in}}%
\pgfpathlineto{\pgfqpoint{5.147558in}{3.376269in}}%
\pgfpathlineto{\pgfqpoint{5.150780in}{3.360291in}}%
\pgfpathlineto{\pgfqpoint{5.151854in}{3.373525in}}%
\pgfpathlineto{\pgfqpoint{5.152928in}{3.364971in}}%
\pgfpathlineto{\pgfqpoint{5.155075in}{3.372557in}}%
\pgfpathlineto{\pgfqpoint{5.158297in}{3.374493in}}%
\pgfpathlineto{\pgfqpoint{5.160445in}{3.384177in}}%
\pgfpathlineto{\pgfqpoint{5.161518in}{3.383531in}}%
\pgfpathlineto{\pgfqpoint{5.162592in}{3.380626in}}%
\pgfpathlineto{\pgfqpoint{5.165814in}{3.392247in}}%
\pgfpathlineto{\pgfqpoint{5.166888in}{3.391278in}}%
\pgfpathlineto{\pgfqpoint{5.167961in}{3.377883in}}%
\pgfpathlineto{\pgfqpoint{5.170109in}{3.365778in}}%
\pgfpathlineto{\pgfqpoint{5.174404in}{3.391924in}}%
\pgfpathlineto{\pgfqpoint{5.175478in}{3.387728in}}%
\pgfpathlineto{\pgfqpoint{5.177626in}{3.391278in}}%
\pgfpathlineto{\pgfqpoint{5.180847in}{3.402576in}}%
\pgfpathlineto{\pgfqpoint{5.182995in}{3.396282in}}%
\pgfpathlineto{\pgfqpoint{5.184069in}{3.395797in}}%
\pgfpathlineto{\pgfqpoint{5.185143in}{3.391278in}}%
\pgfpathlineto{\pgfqpoint{5.188364in}{3.401123in}}%
\pgfpathlineto{\pgfqpoint{5.189438in}{3.412744in}}%
\pgfpathlineto{\pgfqpoint{5.190512in}{3.414680in}}%
\pgfpathlineto{\pgfqpoint{5.192660in}{3.404835in}}%
\pgfpathlineto{\pgfqpoint{5.196955in}{3.405804in}}%
\pgfpathlineto{\pgfqpoint{5.198029in}{3.416779in}}%
\pgfpathlineto{\pgfqpoint{5.199103in}{3.418231in}}%
\pgfpathlineto{\pgfqpoint{5.203398in}{3.415649in}}%
\pgfpathlineto{\pgfqpoint{5.205546in}{3.408063in}}%
\pgfpathlineto{\pgfqpoint{5.206620in}{3.418877in}}%
\pgfpathlineto{\pgfqpoint{5.207693in}{3.422427in}}%
\pgfpathlineto{\pgfqpoint{5.210915in}{3.443247in}}%
\pgfpathlineto{\pgfqpoint{5.211989in}{3.436469in}}%
\pgfpathlineto{\pgfqpoint{5.213063in}{3.437921in}}%
\pgfpathlineto{\pgfqpoint{5.214136in}{3.435016in}}%
\pgfpathlineto{\pgfqpoint{5.215210in}{3.429367in}}%
\pgfpathlineto{\pgfqpoint{5.218432in}{3.426301in}}%
\pgfpathlineto{\pgfqpoint{5.219506in}{3.418554in}}%
\pgfpathlineto{\pgfqpoint{5.220580in}{3.431627in}}%
\pgfpathlineto{\pgfqpoint{5.221653in}{3.432595in}}%
\pgfpathlineto{\pgfqpoint{5.222727in}{3.435823in}}%
\pgfpathlineto{\pgfqpoint{5.227023in}{3.425171in}}%
\pgfpathlineto{\pgfqpoint{5.230244in}{3.407579in}}%
\pgfpathlineto{\pgfqpoint{5.233466in}{3.402092in}}%
\pgfpathlineto{\pgfqpoint{5.235613in}{3.409839in}}%
\pgfpathlineto{\pgfqpoint{5.236687in}{3.411937in}}%
\pgfpathlineto{\pgfqpoint{5.237761in}{3.410161in}}%
\pgfpathlineto{\pgfqpoint{5.242056in}{3.404190in}}%
\pgfpathlineto{\pgfqpoint{5.243130in}{3.404513in}}%
\pgfpathlineto{\pgfqpoint{5.245278in}{3.413873in}}%
\pgfpathlineto{\pgfqpoint{5.248499in}{3.408063in}}%
\pgfpathlineto{\pgfqpoint{5.249573in}{3.402092in}}%
\pgfpathlineto{\pgfqpoint{5.250647in}{3.401285in}}%
\pgfpathlineto{\pgfqpoint{5.251721in}{3.404674in}}%
\pgfpathlineto{\pgfqpoint{5.252795in}{3.403544in}}%
\pgfpathlineto{\pgfqpoint{5.257090in}{3.405158in}}%
\pgfpathlineto{\pgfqpoint{5.258164in}{3.403867in}}%
\pgfpathlineto{\pgfqpoint{5.259238in}{3.399832in}}%
\pgfpathlineto{\pgfqpoint{5.260312in}{3.398218in}}%
\pgfpathlineto{\pgfqpoint{5.263533in}{3.398864in}}%
\pgfpathlineto{\pgfqpoint{5.264607in}{3.397411in}}%
\pgfpathlineto{\pgfqpoint{5.265681in}{3.400316in}}%
\pgfpathlineto{\pgfqpoint{5.266755in}{3.406449in}}%
\pgfpathlineto{\pgfqpoint{5.267828in}{3.402092in}}%
\pgfpathlineto{\pgfqpoint{5.271050in}{3.396604in}}%
\pgfpathlineto{\pgfqpoint{5.272124in}{3.389664in}}%
\pgfpathlineto{\pgfqpoint{5.273198in}{3.393861in}}%
\pgfpathlineto{\pgfqpoint{5.274271in}{3.384338in}}%
\pgfpathlineto{\pgfqpoint{5.275345in}{3.389664in}}%
\pgfpathlineto{\pgfqpoint{5.278567in}{3.383854in}}%
\pgfpathlineto{\pgfqpoint{5.279641in}{3.393215in}}%
\pgfpathlineto{\pgfqpoint{5.281788in}{3.402899in}}%
\pgfpathlineto{\pgfqpoint{5.286084in}{3.405642in}}%
\pgfpathlineto{\pgfqpoint{5.287158in}{3.410807in}}%
\pgfpathlineto{\pgfqpoint{5.288231in}{3.425978in}}%
\pgfpathlineto{\pgfqpoint{5.289305in}{3.425171in}}%
\pgfpathlineto{\pgfqpoint{5.290379in}{3.419038in}}%
\pgfpathlineto{\pgfqpoint{5.293601in}{3.421136in}}%
\pgfpathlineto{\pgfqpoint{5.294674in}{3.420329in}}%
\pgfpathlineto{\pgfqpoint{5.295748in}{3.424848in}}%
\pgfpathlineto{\pgfqpoint{5.296822in}{3.417424in}}%
\pgfpathlineto{\pgfqpoint{5.297896in}{3.419361in}}%
\pgfpathlineto{\pgfqpoint{5.301117in}{3.415649in}}%
\pgfpathlineto{\pgfqpoint{5.303265in}{3.411775in}}%
\pgfpathlineto{\pgfqpoint{5.304339in}{3.415487in}}%
\pgfpathlineto{\pgfqpoint{5.305413in}{3.405804in}}%
\pgfpathlineto{\pgfqpoint{5.308634in}{3.401608in}}%
\pgfpathlineto{\pgfqpoint{5.311856in}{3.374978in}}%
\pgfpathlineto{\pgfqpoint{5.312930in}{3.407095in}}%
\pgfpathlineto{\pgfqpoint{5.316151in}{3.401123in}}%
\pgfpathlineto{\pgfqpoint{5.317225in}{3.403706in}}%
\pgfpathlineto{\pgfqpoint{5.318299in}{3.426785in}}%
\pgfpathlineto{\pgfqpoint{5.319373in}{3.413712in}}%
\pgfpathlineto{\pgfqpoint{5.320446in}{3.424848in}}%
\pgfpathlineto{\pgfqpoint{5.323668in}{3.432272in}}%
\pgfpathlineto{\pgfqpoint{5.324742in}{3.431465in}}%
\pgfpathlineto{\pgfqpoint{5.325816in}{3.432272in}}%
\pgfpathlineto{\pgfqpoint{5.326890in}{3.438728in}}%
\pgfpathlineto{\pgfqpoint{5.327963in}{3.436307in}}%
\pgfpathlineto{\pgfqpoint{5.333333in}{3.450510in}}%
\pgfpathlineto{\pgfqpoint{5.334406in}{3.456966in}}%
\pgfpathlineto{\pgfqpoint{5.335480in}{3.458580in}}%
\pgfpathlineto{\pgfqpoint{5.338702in}{3.460516in}}%
\pgfpathlineto{\pgfqpoint{5.341923in}{3.453576in}}%
\pgfpathlineto{\pgfqpoint{5.342997in}{3.454706in}}%
\pgfpathlineto{\pgfqpoint{5.347292in}{3.452931in}}%
\pgfpathlineto{\pgfqpoint{5.348366in}{3.451801in}}%
\pgfpathlineto{\pgfqpoint{5.349440in}{3.462776in}}%
\pgfpathlineto{\pgfqpoint{5.350514in}{3.463744in}}%
\pgfpathlineto{\pgfqpoint{5.353735in}{3.457611in}}%
\pgfpathlineto{\pgfqpoint{5.354809in}{3.452931in}}%
\pgfpathlineto{\pgfqpoint{5.355883in}{3.461969in}}%
\pgfpathlineto{\pgfqpoint{5.358031in}{3.456643in}}%
\pgfpathlineto{\pgfqpoint{5.362326in}{3.467940in}}%
\pgfpathlineto{\pgfqpoint{5.364474in}{3.469554in}}%
\pgfpathlineto{\pgfqpoint{5.365548in}{3.473428in}}%
\pgfpathlineto{\pgfqpoint{5.368769in}{3.478108in}}%
\pgfpathlineto{\pgfqpoint{5.369843in}{3.472944in}}%
\pgfpathlineto{\pgfqpoint{5.370917in}{3.483918in}}%
\pgfpathlineto{\pgfqpoint{5.371991in}{3.471491in}}%
\pgfpathlineto{\pgfqpoint{5.373065in}{3.474880in}}%
\pgfpathlineto{\pgfqpoint{5.376286in}{3.472944in}}%
\pgfpathlineto{\pgfqpoint{5.378434in}{3.455029in}}%
\pgfpathlineto{\pgfqpoint{5.379508in}{3.453899in}}%
\pgfpathlineto{\pgfqpoint{5.380581in}{3.464067in}}%
\pgfpathlineto{\pgfqpoint{5.383803in}{3.461000in}}%
\pgfpathlineto{\pgfqpoint{5.384877in}{3.455352in}}%
\pgfpathlineto{\pgfqpoint{5.385951in}{3.469393in}}%
\pgfpathlineto{\pgfqpoint{5.387024in}{3.462292in}}%
\pgfpathlineto{\pgfqpoint{5.388098in}{3.476333in}}%
\pgfpathlineto{\pgfqpoint{5.391320in}{3.458257in}}%
\pgfpathlineto{\pgfqpoint{5.392394in}{3.460678in}}%
\pgfpathlineto{\pgfqpoint{5.394541in}{3.442602in}}%
\pgfpathlineto{\pgfqpoint{5.395615in}{3.456804in}}%
\pgfpathlineto{\pgfqpoint{5.400984in}{3.478915in}}%
\pgfpathlineto{\pgfqpoint{5.402058in}{3.465035in}}%
\pgfpathlineto{\pgfqpoint{5.403132in}{3.491827in}}%
\pgfpathlineto{\pgfqpoint{5.406354in}{3.502801in}}%
\pgfpathlineto{\pgfqpoint{5.407427in}{3.510064in}}%
\pgfpathlineto{\pgfqpoint{5.408501in}{3.511033in}}%
\pgfpathlineto{\pgfqpoint{5.410649in}{3.521200in}}%
\pgfpathlineto{\pgfqpoint{5.413870in}{3.522653in}}%
\pgfpathlineto{\pgfqpoint{5.414944in}{3.540083in}}%
\pgfpathlineto{\pgfqpoint{5.416018in}{3.545087in}}%
\pgfpathlineto{\pgfqpoint{5.417092in}{3.543795in}}%
\pgfpathlineto{\pgfqpoint{5.418166in}{3.547023in}}%
\pgfpathlineto{\pgfqpoint{5.421387in}{3.551381in}}%
\pgfpathlineto{\pgfqpoint{5.422461in}{3.554447in}}%
\pgfpathlineto{\pgfqpoint{5.423535in}{3.551865in}}%
\pgfpathlineto{\pgfqpoint{5.425683in}{3.532821in}}%
\pgfpathlineto{\pgfqpoint{5.428904in}{3.529109in}}%
\pgfpathlineto{\pgfqpoint{5.429978in}{3.530561in}}%
\pgfpathlineto{\pgfqpoint{5.431052in}{3.541375in}}%
\pgfpathlineto{\pgfqpoint{5.432126in}{3.537824in}}%
\pgfpathlineto{\pgfqpoint{5.433200in}{3.539599in}}%
\pgfpathlineto{\pgfqpoint{5.436421in}{3.532821in}}%
\pgfpathlineto{\pgfqpoint{5.437495in}{3.542343in}}%
\pgfpathlineto{\pgfqpoint{5.438569in}{3.543473in}}%
\pgfpathlineto{\pgfqpoint{5.440716in}{3.564938in}}%
\pgfpathlineto{\pgfqpoint{5.443938in}{3.560096in}}%
\pgfpathlineto{\pgfqpoint{5.445012in}{3.573815in}}%
\pgfpathlineto{\pgfqpoint{5.446086in}{3.558967in}}%
\pgfpathlineto{\pgfqpoint{5.447159in}{3.567036in}}%
\pgfpathlineto{\pgfqpoint{5.448233in}{3.564293in}}%
\pgfpathlineto{\pgfqpoint{5.451455in}{3.569619in}}%
\pgfpathlineto{\pgfqpoint{5.452529in}{3.568812in}}%
\pgfpathlineto{\pgfqpoint{5.453602in}{3.558967in}}%
\pgfpathlineto{\pgfqpoint{5.454676in}{3.564777in}}%
\pgfpathlineto{\pgfqpoint{5.455750in}{3.552834in}}%
\pgfpathlineto{\pgfqpoint{5.458972in}{3.547992in}}%
\pgfpathlineto{\pgfqpoint{5.460046in}{3.550090in}}%
\pgfpathlineto{\pgfqpoint{5.462193in}{3.586726in}}%
\pgfpathlineto{\pgfqpoint{5.463267in}{3.587533in}}%
\pgfpathlineto{\pgfqpoint{5.466489in}{3.595119in}}%
\pgfpathlineto{\pgfqpoint{5.467562in}{3.604641in}}%
\pgfpathlineto{\pgfqpoint{5.468636in}{3.602543in}}%
\pgfpathlineto{\pgfqpoint{5.470784in}{3.607062in}}%
\pgfpathlineto{\pgfqpoint{5.475079in}{3.592375in}}%
\pgfpathlineto{\pgfqpoint{5.476153in}{3.574138in}}%
\pgfpathlineto{\pgfqpoint{5.478301in}{3.565099in}}%
\pgfpathlineto{\pgfqpoint{5.481522in}{3.559128in}}%
\pgfpathlineto{\pgfqpoint{5.482596in}{3.553479in}}%
\pgfpathlineto{\pgfqpoint{5.483670in}{3.559935in}}%
\pgfpathlineto{\pgfqpoint{5.484744in}{3.574299in}}%
\pgfpathlineto{\pgfqpoint{5.485818in}{3.562517in}}%
\pgfpathlineto{\pgfqpoint{5.489039in}{3.557837in}}%
\pgfpathlineto{\pgfqpoint{5.490113in}{3.563163in}}%
\pgfpathlineto{\pgfqpoint{5.491187in}{3.558805in}}%
\pgfpathlineto{\pgfqpoint{5.492261in}{3.557030in}}%
\pgfpathlineto{\pgfqpoint{5.493334in}{3.576397in}}%
\pgfpathlineto{\pgfqpoint{5.497630in}{3.575590in}}%
\pgfpathlineto{\pgfqpoint{5.498704in}{3.578011in}}%
\pgfpathlineto{\pgfqpoint{5.499778in}{3.590600in}}%
\pgfpathlineto{\pgfqpoint{5.500851in}{3.568973in}}%
\pgfpathlineto{\pgfqpoint{5.504073in}{3.562033in}}%
\pgfpathlineto{\pgfqpoint{5.505147in}{3.518779in}}%
\pgfpathlineto{\pgfqpoint{5.506221in}{3.500058in}}%
\pgfpathlineto{\pgfqpoint{5.507294in}{3.507320in}}%
\pgfpathlineto{\pgfqpoint{5.508368in}{3.488115in}}%
\pgfpathlineto{\pgfqpoint{5.511590in}{3.499735in}}%
\pgfpathlineto{\pgfqpoint{5.512664in}{3.511355in}}%
\pgfpathlineto{\pgfqpoint{5.513737in}{3.509096in}}%
\pgfpathlineto{\pgfqpoint{5.514811in}{3.521846in}}%
\pgfpathlineto{\pgfqpoint{5.515885in}{3.506514in}}%
\pgfpathlineto{\pgfqpoint{5.519107in}{3.498767in}}%
\pgfpathlineto{\pgfqpoint{5.522328in}{3.512485in}}%
\pgfpathlineto{\pgfqpoint{5.523402in}{3.510548in}}%
\pgfpathlineto{\pgfqpoint{5.527697in}{3.504900in}}%
\pgfpathlineto{\pgfqpoint{5.528771in}{3.515713in}}%
\pgfpathlineto{\pgfqpoint{5.529845in}{3.501026in}}%
\pgfpathlineto{\pgfqpoint{5.530919in}{3.496184in}}%
\pgfpathlineto{\pgfqpoint{5.535214in}{3.505061in}}%
\pgfpathlineto{\pgfqpoint{5.536288in}{3.504415in}}%
\pgfpathlineto{\pgfqpoint{5.537362in}{3.500381in}}%
\pgfpathlineto{\pgfqpoint{5.538436in}{3.499896in}}%
\pgfpathlineto{\pgfqpoint{5.541657in}{3.503770in}}%
\pgfpathlineto{\pgfqpoint{5.542731in}{3.500219in}}%
\pgfpathlineto{\pgfqpoint{5.543805in}{3.488922in}}%
\pgfpathlineto{\pgfqpoint{5.544879in}{3.492795in}}%
\pgfpathlineto{\pgfqpoint{5.545953in}{3.465358in}}%
\pgfpathlineto{\pgfqpoint{5.549174in}{3.471330in}}%
\pgfpathlineto{\pgfqpoint{5.550248in}{3.449703in}}%
\pgfpathlineto{\pgfqpoint{5.551322in}{3.447605in}}%
\pgfpathlineto{\pgfqpoint{5.552396in}{3.457450in}}%
\pgfpathlineto{\pgfqpoint{5.553469in}{3.453738in}}%
\pgfpathlineto{\pgfqpoint{5.556691in}{3.477947in}}%
\pgfpathlineto{\pgfqpoint{5.557765in}{3.467940in}}%
\pgfpathlineto{\pgfqpoint{5.558839in}{3.480529in}}%
\pgfpathlineto{\pgfqpoint{5.559912in}{3.475365in}}%
\pgfpathlineto{\pgfqpoint{5.560986in}{3.494409in}}%
\pgfpathlineto{\pgfqpoint{5.564208in}{3.496023in}}%
\pgfpathlineto{\pgfqpoint{5.567429in}{3.458257in}}%
\pgfpathlineto{\pgfqpoint{5.571725in}{3.466165in}}%
\pgfpathlineto{\pgfqpoint{5.572799in}{3.455352in}}%
\pgfpathlineto{\pgfqpoint{5.573872in}{3.460516in}}%
\pgfpathlineto{\pgfqpoint{5.574946in}{3.462130in}}%
\pgfpathlineto{\pgfqpoint{5.579242in}{3.470684in}}%
\pgfpathlineto{\pgfqpoint{5.580315in}{3.461485in}}%
\pgfpathlineto{\pgfqpoint{5.581389in}{3.466811in}}%
\pgfpathlineto{\pgfqpoint{5.582463in}{3.468586in}}%
\pgfpathlineto{\pgfqpoint{5.583537in}{3.475042in}}%
\pgfpathlineto{\pgfqpoint{5.586758in}{3.476171in}}%
\pgfpathlineto{\pgfqpoint{5.587832in}{3.478431in}}%
\pgfpathlineto{\pgfqpoint{5.588906in}{3.477140in}}%
\pgfpathlineto{\pgfqpoint{5.589980in}{3.477140in}}%
\pgfpathlineto{\pgfqpoint{5.591054in}{3.463583in}}%
\pgfpathlineto{\pgfqpoint{5.594275in}{3.468263in}}%
\pgfpathlineto{\pgfqpoint{5.595349in}{3.471168in}}%
\pgfpathlineto{\pgfqpoint{5.596423in}{3.471330in}}%
\pgfpathlineto{\pgfqpoint{5.597497in}{3.450510in}}%
\pgfpathlineto{\pgfqpoint{5.598571in}{3.451317in}}%
\pgfpathlineto{\pgfqpoint{5.602866in}{3.443086in}}%
\pgfpathlineto{\pgfqpoint{5.605014in}{3.430336in}}%
\pgfpathlineto{\pgfqpoint{5.606088in}{3.441310in}}%
\pgfpathlineto{\pgfqpoint{5.609309in}{3.442117in}}%
\pgfpathlineto{\pgfqpoint{5.610383in}{3.438083in}}%
\pgfpathlineto{\pgfqpoint{5.611457in}{3.442763in}}%
\pgfpathlineto{\pgfqpoint{5.612531in}{3.440019in}}%
\pgfpathlineto{\pgfqpoint{5.613604in}{3.450671in}}%
\pgfpathlineto{\pgfqpoint{5.618974in}{3.432918in}}%
\pgfpathlineto{\pgfqpoint{5.621121in}{3.451962in}}%
\pgfpathlineto{\pgfqpoint{5.624343in}{3.447605in}}%
\pgfpathlineto{\pgfqpoint{5.625417in}{3.448896in}}%
\pgfpathlineto{\pgfqpoint{5.626490in}{3.443893in}}%
\pgfpathlineto{\pgfqpoint{5.627564in}{3.442924in}}%
\pgfpathlineto{\pgfqpoint{5.628638in}{3.436469in}}%
\pgfpathlineto{\pgfqpoint{5.632934in}{3.425010in}}%
\pgfpathlineto{\pgfqpoint{5.634007in}{3.428560in}}%
\pgfpathlineto{\pgfqpoint{5.635081in}{3.427753in}}%
\pgfpathlineto{\pgfqpoint{5.636155in}{3.414519in}}%
\pgfpathlineto{\pgfqpoint{5.639377in}{3.420975in}}%
\pgfpathlineto{\pgfqpoint{5.640450in}{3.416617in}}%
\pgfpathlineto{\pgfqpoint{5.641524in}{3.416940in}}%
\pgfpathlineto{\pgfqpoint{5.642598in}{3.411130in}}%
\pgfpathlineto{\pgfqpoint{5.643672in}{3.400962in}}%
\pgfpathlineto{\pgfqpoint{5.646893in}{3.404835in}}%
\pgfpathlineto{\pgfqpoint{5.649041in}{3.430658in}}%
\pgfpathlineto{\pgfqpoint{5.650115in}{3.428883in}}%
\pgfpathlineto{\pgfqpoint{5.651189in}{3.421298in}}%
\pgfpathlineto{\pgfqpoint{5.654410in}{3.410646in}}%
\pgfpathlineto{\pgfqpoint{5.657632in}{3.448735in}}%
\pgfpathlineto{\pgfqpoint{5.658706in}{3.444700in}}%
\pgfpathlineto{\pgfqpoint{5.661927in}{3.443570in}}%
\pgfpathlineto{\pgfqpoint{5.663001in}{3.434209in}}%
\pgfpathlineto{\pgfqpoint{5.664075in}{3.430336in}}%
\pgfpathlineto{\pgfqpoint{5.665149in}{3.428560in}}%
\pgfpathlineto{\pgfqpoint{5.666222in}{3.427915in}}%
\pgfpathlineto{\pgfqpoint{5.669444in}{3.413551in}}%
\pgfpathlineto{\pgfqpoint{5.670518in}{3.412421in}}%
\pgfpathlineto{\pgfqpoint{5.671592in}{3.433241in}}%
\pgfpathlineto{\pgfqpoint{5.672666in}{3.436146in}}%
\pgfpathlineto{\pgfqpoint{5.676961in}{3.437921in}}%
\pgfpathlineto{\pgfqpoint{5.678035in}{3.461323in}}%
\pgfpathlineto{\pgfqpoint{5.679109in}{3.451155in}}%
\pgfpathlineto{\pgfqpoint{5.680182in}{3.446475in}}%
\pgfpathlineto{\pgfqpoint{5.685552in}{3.465842in}}%
\pgfpathlineto{\pgfqpoint{5.687699in}{3.469554in}}%
\pgfpathlineto{\pgfqpoint{5.688773in}{3.468747in}}%
\pgfpathlineto{\pgfqpoint{5.691995in}{3.467940in}}%
\pgfpathlineto{\pgfqpoint{5.693068in}{3.460193in}}%
\pgfpathlineto{\pgfqpoint{5.695216in}{3.456320in}}%
\pgfpathlineto{\pgfqpoint{5.696290in}{3.450510in}}%
\pgfpathlineto{\pgfqpoint{5.699511in}{3.445991in}}%
\pgfpathlineto{\pgfqpoint{5.701659in}{3.455190in}}%
\pgfpathlineto{\pgfqpoint{5.702733in}{3.409354in}}%
\pgfpathlineto{\pgfqpoint{5.703807in}{3.399509in}}%
\pgfpathlineto{\pgfqpoint{5.707028in}{3.395313in}}%
\pgfpathlineto{\pgfqpoint{5.708102in}{3.388373in}}%
\pgfpathlineto{\pgfqpoint{5.709176in}{3.386275in}}%
\pgfpathlineto{\pgfqpoint{5.710250in}{3.385952in}}%
\pgfpathlineto{\pgfqpoint{5.711324in}{3.382240in}}%
\pgfpathlineto{\pgfqpoint{5.714545in}{3.394990in}}%
\pgfpathlineto{\pgfqpoint{5.715619in}{3.392731in}}%
\pgfpathlineto{\pgfqpoint{5.716693in}{3.395152in}}%
\pgfpathlineto{\pgfqpoint{5.717767in}{3.386437in}}%
\pgfpathlineto{\pgfqpoint{5.718841in}{3.384177in}}%
\pgfpathlineto{\pgfqpoint{5.722062in}{3.382886in}}%
\pgfpathlineto{\pgfqpoint{5.723136in}{3.377237in}}%
\pgfpathlineto{\pgfqpoint{5.724210in}{3.362873in}}%
\pgfpathlineto{\pgfqpoint{5.725284in}{3.359968in}}%
\pgfpathlineto{\pgfqpoint{5.726357in}{3.330594in}}%
\pgfpathlineto{\pgfqpoint{5.730653in}{3.282015in}}%
\pgfpathlineto{\pgfqpoint{5.731727in}{3.317199in}}%
\pgfpathlineto{\pgfqpoint{5.732800in}{3.325430in}}%
\pgfpathlineto{\pgfqpoint{5.733874in}{3.321556in}}%
\pgfpathlineto{\pgfqpoint{5.737096in}{3.313971in}}%
\pgfpathlineto{\pgfqpoint{5.738170in}{3.288793in}}%
\pgfpathlineto{\pgfqpoint{5.739244in}{3.301705in}}%
\pgfpathlineto{\pgfqpoint{5.740317in}{3.303319in}}%
\pgfpathlineto{\pgfqpoint{5.741391in}{3.286856in}}%
\pgfpathlineto{\pgfqpoint{5.745687in}{3.304126in}}%
\pgfpathlineto{\pgfqpoint{5.746760in}{3.282822in}}%
\pgfpathlineto{\pgfqpoint{5.747834in}{3.280562in}}%
\pgfpathlineto{\pgfqpoint{5.748908in}{3.282015in}}%
\pgfpathlineto{\pgfqpoint{5.752130in}{3.276850in}}%
\pgfpathlineto{\pgfqpoint{5.753203in}{3.296540in}}%
\pgfpathlineto{\pgfqpoint{5.754277in}{3.305740in}}%
\pgfpathlineto{\pgfqpoint{5.755351in}{3.307838in}}%
\pgfpathlineto{\pgfqpoint{5.756425in}{3.303480in}}%
\pgfpathlineto{\pgfqpoint{5.759646in}{3.313648in}}%
\pgfpathlineto{\pgfqpoint{5.760720in}{3.307031in}}%
\pgfpathlineto{\pgfqpoint{5.761794in}{3.308160in}}%
\pgfpathlineto{\pgfqpoint{5.763942in}{3.342376in}}%
\pgfpathlineto{\pgfqpoint{5.767163in}{3.329464in}}%
\pgfpathlineto{\pgfqpoint{5.768237in}{3.336727in}}%
\pgfpathlineto{\pgfqpoint{5.769311in}{3.331885in}}%
\pgfpathlineto{\pgfqpoint{5.770385in}{3.332047in}}%
\pgfpathlineto{\pgfqpoint{5.771459in}{3.338825in}}%
\pgfpathlineto{\pgfqpoint{5.776828in}{3.357224in}}%
\pgfpathlineto{\pgfqpoint{5.777902in}{3.366908in}}%
\pgfpathlineto{\pgfqpoint{5.778976in}{3.368038in}}%
\pgfpathlineto{\pgfqpoint{5.782197in}{3.365939in}}%
\pgfpathlineto{\pgfqpoint{5.783271in}{3.362712in}}%
\pgfpathlineto{\pgfqpoint{5.785419in}{3.364971in}}%
\pgfpathlineto{\pgfqpoint{5.786492in}{3.374009in}}%
\pgfpathlineto{\pgfqpoint{5.789714in}{3.377721in}}%
\pgfpathlineto{\pgfqpoint{5.790788in}{3.367231in}}%
\pgfpathlineto{\pgfqpoint{5.791862in}{3.364810in}}%
\pgfpathlineto{\pgfqpoint{5.792935in}{3.382886in}}%
\pgfpathlineto{\pgfqpoint{5.794009in}{3.414035in}}%
\pgfpathlineto{\pgfqpoint{5.797231in}{3.420652in}}%
\pgfpathlineto{\pgfqpoint{5.798305in}{3.417747in}}%
\pgfpathlineto{\pgfqpoint{5.799378in}{3.406611in}}%
\pgfpathlineto{\pgfqpoint{5.800452in}{3.413873in}}%
\pgfpathlineto{\pgfqpoint{5.801526in}{3.404674in}}%
\pgfpathlineto{\pgfqpoint{5.804748in}{3.407902in}}%
\pgfpathlineto{\pgfqpoint{5.805822in}{3.414358in}}%
\pgfpathlineto{\pgfqpoint{5.806895in}{3.414519in}}%
\pgfpathlineto{\pgfqpoint{5.809043in}{3.393054in}}%
\pgfpathlineto{\pgfqpoint{5.812265in}{3.390633in}}%
\pgfpathlineto{\pgfqpoint{5.814412in}{3.398702in}}%
\pgfpathlineto{\pgfqpoint{5.816560in}{3.370136in}}%
\pgfpathlineto{\pgfqpoint{5.819781in}{3.389180in}}%
\pgfpathlineto{\pgfqpoint{5.820855in}{3.385952in}}%
\pgfpathlineto{\pgfqpoint{5.821929in}{3.397895in}}%
\pgfpathlineto{\pgfqpoint{5.823003in}{3.402414in}}%
\pgfpathlineto{\pgfqpoint{5.824077in}{3.396766in}}%
\pgfpathlineto{\pgfqpoint{5.827298in}{3.398864in}}%
\pgfpathlineto{\pgfqpoint{5.828372in}{3.405804in}}%
\pgfpathlineto{\pgfqpoint{5.829446in}{3.397895in}}%
\pgfpathlineto{\pgfqpoint{5.831594in}{3.394990in}}%
\pgfpathlineto{\pgfqpoint{5.834815in}{3.382724in}}%
\pgfpathlineto{\pgfqpoint{5.835889in}{3.398380in}}%
\pgfpathlineto{\pgfqpoint{5.836963in}{3.396282in}}%
\pgfpathlineto{\pgfqpoint{5.838037in}{3.395636in}}%
\pgfpathlineto{\pgfqpoint{5.839111in}{3.425494in}}%
\pgfpathlineto{\pgfqpoint{5.842332in}{3.433241in}}%
\pgfpathlineto{\pgfqpoint{5.843406in}{3.424364in}}%
\pgfpathlineto{\pgfqpoint{5.844480in}{3.423718in}}%
\pgfpathlineto{\pgfqpoint{5.845554in}{3.425010in}}%
\pgfpathlineto{\pgfqpoint{5.846627in}{3.424848in}}%
\pgfpathlineto{\pgfqpoint{5.849849in}{3.432111in}}%
\pgfpathlineto{\pgfqpoint{5.851997in}{3.470845in}}%
\pgfpathlineto{\pgfqpoint{5.853070in}{3.460678in}}%
\pgfpathlineto{\pgfqpoint{5.854144in}{3.429851in}}%
\pgfpathlineto{\pgfqpoint{5.857366in}{3.441472in}}%
\pgfpathlineto{\pgfqpoint{5.859513in}{3.455513in}}%
\pgfpathlineto{\pgfqpoint{5.860587in}{3.453576in}}%
\pgfpathlineto{\pgfqpoint{5.864883in}{3.455513in}}%
\pgfpathlineto{\pgfqpoint{5.865956in}{3.461807in}}%
\pgfpathlineto{\pgfqpoint{5.867030in}{3.457611in}}%
\pgfpathlineto{\pgfqpoint{5.868104in}{3.448089in}}%
\pgfpathlineto{\pgfqpoint{5.872399in}{3.433241in}}%
\pgfpathlineto{\pgfqpoint{5.873473in}{3.436791in}}%
\pgfpathlineto{\pgfqpoint{5.876695in}{3.398864in}}%
\pgfpathlineto{\pgfqpoint{5.879916in}{3.408870in}}%
\pgfpathlineto{\pgfqpoint{5.880990in}{3.406611in}}%
\pgfpathlineto{\pgfqpoint{5.882064in}{3.397088in}}%
\pgfpathlineto{\pgfqpoint{5.883138in}{3.401446in}}%
\pgfpathlineto{\pgfqpoint{5.884212in}{3.384661in}}%
\pgfpathlineto{\pgfqpoint{5.888507in}{3.409677in}}%
\pgfpathlineto{\pgfqpoint{5.889581in}{3.406288in}}%
\pgfpathlineto{\pgfqpoint{5.891729in}{3.428399in}}%
\pgfpathlineto{\pgfqpoint{5.894950in}{3.421136in}}%
\pgfpathlineto{\pgfqpoint{5.896024in}{3.449380in}}%
\pgfpathlineto{\pgfqpoint{5.897098in}{3.449219in}}%
\pgfpathlineto{\pgfqpoint{5.898172in}{3.463906in}}%
\pgfpathlineto{\pgfqpoint{5.899245in}{3.491020in}}%
\pgfpathlineto{\pgfqpoint{5.902467in}{3.482789in}}%
\pgfpathlineto{\pgfqpoint{5.903541in}{3.469716in}}%
\pgfpathlineto{\pgfqpoint{5.904615in}{3.482466in}}%
\pgfpathlineto{\pgfqpoint{5.905688in}{3.476656in}}%
\pgfpathlineto{\pgfqpoint{5.909984in}{3.504415in}}%
\pgfpathlineto{\pgfqpoint{5.911058in}{3.504738in}}%
\pgfpathlineto{\pgfqpoint{5.912132in}{3.490051in}}%
\pgfpathlineto{\pgfqpoint{5.913205in}{3.465197in}}%
\pgfpathlineto{\pgfqpoint{5.914279in}{3.480852in}}%
\pgfpathlineto{\pgfqpoint{5.918575in}{3.487792in}}%
\pgfpathlineto{\pgfqpoint{5.919648in}{3.501994in}}%
\pgfpathlineto{\pgfqpoint{5.920722in}{3.495216in}}%
\pgfpathlineto{\pgfqpoint{5.921796in}{3.492472in}}%
\pgfpathlineto{\pgfqpoint{5.925018in}{3.497314in}}%
\pgfpathlineto{\pgfqpoint{5.927165in}{3.489083in}}%
\pgfpathlineto{\pgfqpoint{5.928239in}{3.500381in}}%
\pgfpathlineto{\pgfqpoint{5.929313in}{3.482466in}}%
\pgfpathlineto{\pgfqpoint{5.932534in}{3.470845in}}%
\pgfpathlineto{\pgfqpoint{5.935756in}{3.507643in}}%
\pgfpathlineto{\pgfqpoint{5.936830in}{3.517004in}}%
\pgfpathlineto{\pgfqpoint{5.940051in}{3.511355in}}%
\pgfpathlineto{\pgfqpoint{5.941125in}{3.510871in}}%
\pgfpathlineto{\pgfqpoint{5.942199in}{3.509419in}}%
\pgfpathlineto{\pgfqpoint{5.944347in}{3.491827in}}%
\pgfpathlineto{\pgfqpoint{5.947568in}{3.483434in}}%
\pgfpathlineto{\pgfqpoint{5.948642in}{3.485532in}}%
\pgfpathlineto{\pgfqpoint{5.949716in}{3.486017in}}%
\pgfpathlineto{\pgfqpoint{5.950790in}{3.506352in}}%
\pgfpathlineto{\pgfqpoint{5.951864in}{3.512162in}}%
\pgfpathlineto{\pgfqpoint{5.955085in}{3.514583in}}%
\pgfpathlineto{\pgfqpoint{5.956159in}{3.506191in}}%
\pgfpathlineto{\pgfqpoint{5.958307in}{3.508289in}}%
\pgfpathlineto{\pgfqpoint{5.962602in}{3.504415in}}%
\pgfpathlineto{\pgfqpoint{5.963676in}{3.506998in}}%
\pgfpathlineto{\pgfqpoint{5.964750in}{3.505384in}}%
\pgfpathlineto{\pgfqpoint{5.965823in}{3.499896in}}%
\pgfpathlineto{\pgfqpoint{5.966897in}{3.517650in}}%
\pgfpathlineto{\pgfqpoint{5.971193in}{3.512324in}}%
\pgfpathlineto{\pgfqpoint{5.972266in}{3.521685in}}%
\pgfpathlineto{\pgfqpoint{5.973340in}{3.513453in}}%
\pgfpathlineto{\pgfqpoint{5.974414in}{3.512808in}}%
\pgfpathlineto{\pgfqpoint{5.977636in}{3.506029in}}%
\pgfpathlineto{\pgfqpoint{5.978710in}{3.507482in}}%
\pgfpathlineto{\pgfqpoint{5.979783in}{3.502156in}}%
\pgfpathlineto{\pgfqpoint{5.985153in}{3.517327in}}%
\pgfpathlineto{\pgfqpoint{5.986226in}{3.523783in}}%
\pgfpathlineto{\pgfqpoint{5.987300in}{3.498605in}}%
\pgfpathlineto{\pgfqpoint{5.988374in}{3.487630in}}%
\pgfpathlineto{\pgfqpoint{5.992669in}{3.496668in}}%
\pgfpathlineto{\pgfqpoint{5.993743in}{3.469554in}}%
\pgfpathlineto{\pgfqpoint{5.994817in}{3.474396in}}%
\pgfpathlineto{\pgfqpoint{5.995891in}{3.472621in}}%
\pgfpathlineto{\pgfqpoint{5.996965in}{3.477785in}}%
\pgfpathlineto{\pgfqpoint{6.002334in}{3.499412in}}%
\pgfpathlineto{\pgfqpoint{6.003408in}{3.495055in}}%
\pgfpathlineto{\pgfqpoint{6.004482in}{3.507159in}}%
\pgfpathlineto{\pgfqpoint{6.007703in}{3.506998in}}%
\pgfpathlineto{\pgfqpoint{6.008777in}{3.512162in}}%
\pgfpathlineto{\pgfqpoint{6.009851in}{3.507320in}}%
\pgfpathlineto{\pgfqpoint{6.010925in}{3.511194in}}%
\pgfpathlineto{\pgfqpoint{6.011999in}{3.493925in}}%
\pgfpathlineto{\pgfqpoint{6.015220in}{3.499735in}}%
\pgfpathlineto{\pgfqpoint{6.017368in}{3.473912in}}%
\pgfpathlineto{\pgfqpoint{6.018442in}{3.478754in}}%
\pgfpathlineto{\pgfqpoint{6.019515in}{3.476333in}}%
\pgfpathlineto{\pgfqpoint{6.022737in}{3.478915in}}%
\pgfpathlineto{\pgfqpoint{6.024885in}{3.497637in}}%
\pgfpathlineto{\pgfqpoint{6.025958in}{3.493763in}}%
\pgfpathlineto{\pgfqpoint{6.027032in}{3.496830in}}%
\pgfpathlineto{\pgfqpoint{6.031328in}{3.491181in}}%
\pgfpathlineto{\pgfqpoint{6.032401in}{3.502156in}}%
\pgfpathlineto{\pgfqpoint{6.033475in}{3.504415in}}%
\pgfpathlineto{\pgfqpoint{6.034549in}{3.512001in}}%
\pgfpathlineto{\pgfqpoint{6.037771in}{3.516359in}}%
\pgfpathlineto{\pgfqpoint{6.038844in}{3.509903in}}%
\pgfpathlineto{\pgfqpoint{6.039918in}{3.514583in}}%
\pgfpathlineto{\pgfqpoint{6.040992in}{3.522169in}}%
\pgfpathlineto{\pgfqpoint{6.042066in}{3.522653in}}%
\pgfpathlineto{\pgfqpoint{6.045287in}{3.513453in}}%
\pgfpathlineto{\pgfqpoint{6.046361in}{3.524912in}}%
\pgfpathlineto{\pgfqpoint{6.047435in}{3.519102in}}%
\pgfpathlineto{\pgfqpoint{6.048509in}{3.525719in}}%
\pgfpathlineto{\pgfqpoint{6.049583in}{3.521685in}}%
\pgfpathlineto{\pgfqpoint{6.052804in}{3.520393in}}%
\pgfpathlineto{\pgfqpoint{6.053878in}{3.525719in}}%
\pgfpathlineto{\pgfqpoint{6.054952in}{3.528140in}}%
\pgfpathlineto{\pgfqpoint{6.056026in}{3.537340in}}%
\pgfpathlineto{\pgfqpoint{6.057100in}{3.508934in}}%
\pgfpathlineto{\pgfqpoint{6.060321in}{3.493925in}}%
\pgfpathlineto{\pgfqpoint{6.063543in}{3.544118in}}%
\pgfpathlineto{\pgfqpoint{6.064617in}{3.545732in}}%
\pgfpathlineto{\pgfqpoint{6.068912in}{3.555254in}}%
\pgfpathlineto{\pgfqpoint{6.071060in}{3.546378in}}%
\pgfpathlineto{\pgfqpoint{6.072133in}{3.560096in}}%
\pgfpathlineto{\pgfqpoint{6.076429in}{3.559773in}}%
\pgfpathlineto{\pgfqpoint{6.077503in}{3.561872in}}%
\pgfpathlineto{\pgfqpoint{6.078576in}{3.561549in}}%
\pgfpathlineto{\pgfqpoint{6.079650in}{3.563647in}}%
\pgfpathlineto{\pgfqpoint{6.082872in}{3.562517in}}%
\pgfpathlineto{\pgfqpoint{6.083946in}{3.565906in}}%
\pgfpathlineto{\pgfqpoint{6.085020in}{3.563324in}}%
\pgfpathlineto{\pgfqpoint{6.086093in}{3.562356in}}%
\pgfpathlineto{\pgfqpoint{6.087167in}{3.569134in}}%
\pgfpathlineto{\pgfqpoint{6.090389in}{3.570264in}}%
\pgfpathlineto{\pgfqpoint{6.092536in}{3.550735in}}%
\pgfpathlineto{\pgfqpoint{6.093610in}{3.555739in}}%
\pgfpathlineto{\pgfqpoint{6.094684in}{3.567198in}}%
\pgfpathlineto{\pgfqpoint{6.098979in}{3.584467in}}%
\pgfpathlineto{\pgfqpoint{6.100053in}{3.572846in}}%
\pgfpathlineto{\pgfqpoint{6.101127in}{3.573976in}}%
\pgfpathlineto{\pgfqpoint{6.102201in}{3.570103in}}%
\pgfpathlineto{\pgfqpoint{6.105422in}{3.569780in}}%
\pgfpathlineto{\pgfqpoint{6.107570in}{3.577850in}}%
\pgfpathlineto{\pgfqpoint{6.109718in}{3.588502in}}%
\pgfpathlineto{\pgfqpoint{6.112939in}{3.588179in}}%
\pgfpathlineto{\pgfqpoint{6.114013in}{3.581723in}}%
\pgfpathlineto{\pgfqpoint{6.116161in}{3.594473in}}%
\pgfpathlineto{\pgfqpoint{6.120456in}{3.585758in}}%
\pgfpathlineto{\pgfqpoint{6.121530in}{3.593828in}}%
\pgfpathlineto{\pgfqpoint{6.122604in}{3.592536in}}%
\pgfpathlineto{\pgfqpoint{6.123678in}{3.601252in}}%
\pgfpathlineto{\pgfqpoint{6.124752in}{3.596410in}}%
\pgfpathlineto{\pgfqpoint{6.127973in}{3.607062in}}%
\pgfpathlineto{\pgfqpoint{6.129047in}{3.595926in}}%
\pgfpathlineto{\pgfqpoint{6.130121in}{3.592536in}}%
\pgfpathlineto{\pgfqpoint{6.131195in}{3.607223in}}%
\pgfpathlineto{\pgfqpoint{6.132268in}{3.605609in}}%
\pgfpathlineto{\pgfqpoint{6.136564in}{3.612065in}}%
\pgfpathlineto{\pgfqpoint{6.137638in}{3.601897in}}%
\pgfpathlineto{\pgfqpoint{6.138711in}{3.599476in}}%
\pgfpathlineto{\pgfqpoint{6.139785in}{3.576881in}}%
\pgfpathlineto{\pgfqpoint{6.143007in}{3.606255in}}%
\pgfpathlineto{\pgfqpoint{6.144081in}{3.588663in}}%
\pgfpathlineto{\pgfqpoint{6.145154in}{3.588179in}}%
\pgfpathlineto{\pgfqpoint{6.146228in}{3.603511in}}%
\pgfpathlineto{\pgfqpoint{6.147302in}{3.603350in}}%
\pgfpathlineto{\pgfqpoint{6.150524in}{3.608030in}}%
\pgfpathlineto{\pgfqpoint{6.151598in}{3.611097in}}%
\pgfpathlineto{\pgfqpoint{6.152671in}{3.599638in}}%
\pgfpathlineto{\pgfqpoint{6.153745in}{3.617230in}}%
\pgfpathlineto{\pgfqpoint{6.154819in}{3.599154in}}%
\pgfpathlineto{\pgfqpoint{6.158041in}{3.600445in}}%
\pgfpathlineto{\pgfqpoint{6.159114in}{3.607869in}}%
\pgfpathlineto{\pgfqpoint{6.160188in}{3.624008in}}%
\pgfpathlineto{\pgfqpoint{6.161262in}{3.606094in}}%
\pgfpathlineto{\pgfqpoint{6.162336in}{3.628366in}}%
\pgfpathlineto{\pgfqpoint{6.166631in}{3.607869in}}%
\pgfpathlineto{\pgfqpoint{6.169853in}{3.632078in}}%
\pgfpathlineto{\pgfqpoint{6.173074in}{3.618198in}}%
\pgfpathlineto{\pgfqpoint{6.174148in}{3.610613in}}%
\pgfpathlineto{\pgfqpoint{6.175222in}{3.611097in}}%
\pgfpathlineto{\pgfqpoint{6.176296in}{3.606094in}}%
\pgfpathlineto{\pgfqpoint{6.177370in}{3.608999in}}%
\pgfpathlineto{\pgfqpoint{6.180591in}{3.600122in}}%
\pgfpathlineto{\pgfqpoint{6.181665in}{3.594635in}}%
\pgfpathlineto{\pgfqpoint{6.183813in}{3.567198in}}%
\pgfpathlineto{\pgfqpoint{6.184887in}{3.558321in}}%
\pgfpathlineto{\pgfqpoint{6.188108in}{3.554932in}}%
\pgfpathlineto{\pgfqpoint{6.189182in}{3.597378in}}%
\pgfpathlineto{\pgfqpoint{6.190256in}{3.603673in}}%
\pgfpathlineto{\pgfqpoint{6.191330in}{3.591568in}}%
\pgfpathlineto{\pgfqpoint{6.192403in}{3.595442in}}%
\pgfpathlineto{\pgfqpoint{6.195625in}{3.594796in}}%
\pgfpathlineto{\pgfqpoint{6.196699in}{3.595603in}}%
\pgfpathlineto{\pgfqpoint{6.198846in}{3.591891in}}%
\pgfpathlineto{\pgfqpoint{6.199920in}{3.569457in}}%
\pgfpathlineto{\pgfqpoint{6.203142in}{3.591245in}}%
\pgfpathlineto{\pgfqpoint{6.204216in}{3.604641in}}%
\pgfpathlineto{\pgfqpoint{6.205289in}{3.582046in}}%
\pgfpathlineto{\pgfqpoint{6.206363in}{3.537985in}}%
\pgfpathlineto{\pgfqpoint{6.207437in}{3.547185in}}%
\pgfpathlineto{\pgfqpoint{6.210659in}{3.538631in}}%
\pgfpathlineto{\pgfqpoint{6.211732in}{3.547830in}}%
\pgfpathlineto{\pgfqpoint{6.212806in}{3.541375in}}%
\pgfpathlineto{\pgfqpoint{6.213880in}{3.539599in}}%
\pgfpathlineto{\pgfqpoint{6.214954in}{3.523783in}}%
\pgfpathlineto{\pgfqpoint{6.219249in}{3.535080in}}%
\pgfpathlineto{\pgfqpoint{6.220323in}{3.533950in}}%
\pgfpathlineto{\pgfqpoint{6.222471in}{3.545409in}}%
\pgfpathlineto{\pgfqpoint{6.226766in}{3.537017in}}%
\pgfpathlineto{\pgfqpoint{6.228914in}{3.521685in}}%
\pgfpathlineto{\pgfqpoint{6.229988in}{3.529754in}}%
\pgfpathlineto{\pgfqpoint{6.233209in}{3.538470in}}%
\pgfpathlineto{\pgfqpoint{6.234283in}{3.537340in}}%
\pgfpathlineto{\pgfqpoint{6.235357in}{3.556061in}}%
\pgfpathlineto{\pgfqpoint{6.236431in}{3.546055in}}%
\pgfpathlineto{\pgfqpoint{6.237505in}{3.558967in}}%
\pgfpathlineto{\pgfqpoint{6.240726in}{3.570103in}}%
\pgfpathlineto{\pgfqpoint{6.241800in}{3.570910in}}%
\pgfpathlineto{\pgfqpoint{6.242874in}{3.558967in}}%
\pgfpathlineto{\pgfqpoint{6.243948in}{3.563486in}}%
\pgfpathlineto{\pgfqpoint{6.248243in}{3.563970in}}%
\pgfpathlineto{\pgfqpoint{6.249317in}{3.561872in}}%
\pgfpathlineto{\pgfqpoint{6.250391in}{3.557514in}}%
\pgfpathlineto{\pgfqpoint{6.251464in}{3.560419in}}%
\pgfpathlineto{\pgfqpoint{6.252538in}{3.567682in}}%
\pgfpathlineto{\pgfqpoint{6.256834in}{3.562356in}}%
\pgfpathlineto{\pgfqpoint{6.257908in}{3.554447in}}%
\pgfpathlineto{\pgfqpoint{6.258981in}{3.558644in}}%
\pgfpathlineto{\pgfqpoint{6.260055in}{3.554609in}}%
\pgfpathlineto{\pgfqpoint{6.264351in}{3.556384in}}%
\pgfpathlineto{\pgfqpoint{6.265424in}{3.560903in}}%
\pgfpathlineto{\pgfqpoint{6.266498in}{3.569134in}}%
\pgfpathlineto{\pgfqpoint{6.267572in}{3.568650in}}%
\pgfpathlineto{\pgfqpoint{6.270794in}{3.559289in}}%
\pgfpathlineto{\pgfqpoint{6.271867in}{3.545894in}}%
\pgfpathlineto{\pgfqpoint{6.272941in}{3.549767in}}%
\pgfpathlineto{\pgfqpoint{6.274015in}{3.551058in}}%
\pgfpathlineto{\pgfqpoint{6.275089in}{3.553641in}}%
\pgfpathlineto{\pgfqpoint{6.280458in}{3.577204in}}%
\pgfpathlineto{\pgfqpoint{6.281532in}{3.573653in}}%
\pgfpathlineto{\pgfqpoint{6.282606in}{3.614647in}}%
\pgfpathlineto{\pgfqpoint{6.285827in}{3.607385in}}%
\pgfpathlineto{\pgfqpoint{6.286901in}{3.620780in}}%
\pgfpathlineto{\pgfqpoint{6.289049in}{3.602059in}}%
\pgfpathlineto{\pgfqpoint{6.290123in}{3.603834in}}%
\pgfpathlineto{\pgfqpoint{6.293344in}{3.604318in}}%
\pgfpathlineto{\pgfqpoint{6.294418in}{3.616907in}}%
\pgfpathlineto{\pgfqpoint{6.295492in}{3.612872in}}%
\pgfpathlineto{\pgfqpoint{6.296566in}{3.619328in}}%
\pgfpathlineto{\pgfqpoint{6.297640in}{3.614163in}}%
\pgfpathlineto{\pgfqpoint{6.300861in}{3.614002in}}%
\pgfpathlineto{\pgfqpoint{6.303009in}{3.627882in}}%
\pgfpathlineto{\pgfqpoint{6.304083in}{3.632885in}}%
\pgfpathlineto{\pgfqpoint{6.305156in}{3.622556in}}%
\pgfpathlineto{\pgfqpoint{6.308378in}{3.627559in}}%
\pgfpathlineto{\pgfqpoint{6.309452in}{3.620780in}}%
\pgfpathlineto{\pgfqpoint{6.310526in}{3.669521in}}%
\pgfpathlineto{\pgfqpoint{6.311599in}{3.664518in}}%
\pgfpathlineto{\pgfqpoint{6.312673in}{3.669037in}}%
\pgfpathlineto{\pgfqpoint{6.316969in}{3.677591in}}%
\pgfpathlineto{\pgfqpoint{6.320190in}{3.668392in}}%
\pgfpathlineto{\pgfqpoint{6.323412in}{3.665971in}}%
\pgfpathlineto{\pgfqpoint{6.324486in}{3.668714in}}%
\pgfpathlineto{\pgfqpoint{6.325559in}{3.677591in}}%
\pgfpathlineto{\pgfqpoint{6.327707in}{3.660160in}}%
\pgfpathlineto{\pgfqpoint{6.332002in}{3.657094in}}%
\pgfpathlineto{\pgfqpoint{6.333076in}{3.654834in}}%
\pgfpathlineto{\pgfqpoint{6.334150in}{3.657901in}}%
\pgfpathlineto{\pgfqpoint{6.335224in}{3.668714in}}%
\pgfpathlineto{\pgfqpoint{6.338445in}{3.672265in}}%
\pgfpathlineto{\pgfqpoint{6.339519in}{3.667746in}}%
\pgfpathlineto{\pgfqpoint{6.340593in}{3.673718in}}%
\pgfpathlineto{\pgfqpoint{6.341667in}{3.674202in}}%
\pgfpathlineto{\pgfqpoint{6.342741in}{3.667746in}}%
\pgfpathlineto{\pgfqpoint{6.345962in}{3.670974in}}%
\pgfpathlineto{\pgfqpoint{6.347036in}{3.670490in}}%
\pgfpathlineto{\pgfqpoint{6.350258in}{3.661290in}}%
\pgfpathlineto{\pgfqpoint{6.353479in}{3.659999in}}%
\pgfpathlineto{\pgfqpoint{6.354553in}{3.664034in}}%
\pgfpathlineto{\pgfqpoint{6.355627in}{3.661774in}}%
\pgfpathlineto{\pgfqpoint{6.357775in}{3.650477in}}%
\pgfpathlineto{\pgfqpoint{6.360996in}{3.648056in}}%
\pgfpathlineto{\pgfqpoint{6.362070in}{3.651445in}}%
\pgfpathlineto{\pgfqpoint{6.363144in}{3.652252in}}%
\pgfpathlineto{\pgfqpoint{6.364218in}{3.643860in}}%
\pgfpathlineto{\pgfqpoint{6.365291in}{3.641277in}}%
\pgfpathlineto{\pgfqpoint{6.368513in}{3.645151in}}%
\pgfpathlineto{\pgfqpoint{6.369587in}{3.649831in}}%
\pgfpathlineto{\pgfqpoint{6.370661in}{3.657417in}}%
\pgfpathlineto{\pgfqpoint{6.371734in}{3.653221in}}%
\pgfpathlineto{\pgfqpoint{6.376030in}{3.658547in}}%
\pgfpathlineto{\pgfqpoint{6.377104in}{3.664679in}}%
\pgfpathlineto{\pgfqpoint{6.380325in}{3.642246in}}%
\pgfpathlineto{\pgfqpoint{6.384620in}{3.663066in}}%
\pgfpathlineto{\pgfqpoint{6.385694in}{3.629011in}}%
\pgfpathlineto{\pgfqpoint{6.386768in}{3.628366in}}%
\pgfpathlineto{\pgfqpoint{6.387842in}{3.622878in}}%
\pgfpathlineto{\pgfqpoint{6.391064in}{3.619328in}}%
\pgfpathlineto{\pgfqpoint{6.392137in}{3.606255in}}%
\pgfpathlineto{\pgfqpoint{6.393211in}{3.609321in}}%
\pgfpathlineto{\pgfqpoint{6.398580in}{3.611258in}}%
\pgfpathlineto{\pgfqpoint{6.399654in}{3.608999in}}%
\pgfpathlineto{\pgfqpoint{6.400728in}{3.610290in}}%
\pgfpathlineto{\pgfqpoint{6.401802in}{3.605448in}}%
\pgfpathlineto{\pgfqpoint{6.402876in}{3.605771in}}%
\pgfpathlineto{\pgfqpoint{6.406097in}{3.607869in}}%
\pgfpathlineto{\pgfqpoint{6.407171in}{3.606578in}}%
\pgfpathlineto{\pgfqpoint{6.408245in}{3.606900in}}%
\pgfpathlineto{\pgfqpoint{6.409319in}{3.600929in}}%
\pgfpathlineto{\pgfqpoint{6.410393in}{3.606578in}}%
\pgfpathlineto{\pgfqpoint{6.413614in}{3.606094in}}%
\pgfpathlineto{\pgfqpoint{6.414688in}{3.604157in}}%
\pgfpathlineto{\pgfqpoint{6.417909in}{3.621749in}}%
\pgfpathlineto{\pgfqpoint{6.422205in}{3.624008in}}%
\pgfpathlineto{\pgfqpoint{6.423279in}{3.634337in}}%
\pgfpathlineto{\pgfqpoint{6.424352in}{3.634983in}}%
\pgfpathlineto{\pgfqpoint{6.425426in}{3.641762in}}%
\pgfpathlineto{\pgfqpoint{6.429722in}{3.644989in}}%
\pgfpathlineto{\pgfqpoint{6.430796in}{3.644505in}}%
\pgfpathlineto{\pgfqpoint{6.431869in}{3.630787in}}%
\pgfpathlineto{\pgfqpoint{6.432943in}{3.635306in}}%
\pgfpathlineto{\pgfqpoint{6.436165in}{3.636597in}}%
\pgfpathlineto{\pgfqpoint{6.437239in}{3.633853in}}%
\pgfpathlineto{\pgfqpoint{6.438312in}{3.639663in}}%
\pgfpathlineto{\pgfqpoint{6.439386in}{3.653705in}}%
\pgfpathlineto{\pgfqpoint{6.440460in}{3.657901in}}%
\pgfpathlineto{\pgfqpoint{6.443682in}{3.660967in}}%
\pgfpathlineto{\pgfqpoint{6.445829in}{3.654027in}}%
\pgfpathlineto{\pgfqpoint{6.446903in}{3.648379in}}%
\pgfpathlineto{\pgfqpoint{6.447977in}{3.654350in}}%
\pgfpathlineto{\pgfqpoint{6.451198in}{3.653382in}}%
\pgfpathlineto{\pgfqpoint{6.452272in}{3.642084in}}%
\pgfpathlineto{\pgfqpoint{6.453346in}{3.638534in}}%
\pgfpathlineto{\pgfqpoint{6.454420in}{3.617875in}}%
\pgfpathlineto{\pgfqpoint{6.455494in}{3.620135in}}%
\pgfpathlineto{\pgfqpoint{6.458715in}{3.629011in}}%
\pgfpathlineto{\pgfqpoint{6.460863in}{3.628043in}}%
\pgfpathlineto{\pgfqpoint{6.461937in}{3.623524in}}%
\pgfpathlineto{\pgfqpoint{6.463011in}{3.627720in}}%
\pgfpathlineto{\pgfqpoint{6.466232in}{3.619328in}}%
\pgfpathlineto{\pgfqpoint{6.467306in}{3.614486in}}%
\pgfpathlineto{\pgfqpoint{6.468380in}{3.617391in}}%
\pgfpathlineto{\pgfqpoint{6.469454in}{3.613518in}}%
\pgfpathlineto{\pgfqpoint{6.470528in}{3.619489in}}%
\pgfpathlineto{\pgfqpoint{6.473749in}{3.626268in}}%
\pgfpathlineto{\pgfqpoint{6.474823in}{3.642569in}}%
\pgfpathlineto{\pgfqpoint{6.476971in}{3.652414in}}%
\pgfpathlineto{\pgfqpoint{6.478044in}{3.652575in}}%
\pgfpathlineto{\pgfqpoint{6.481266in}{3.646119in}}%
\pgfpathlineto{\pgfqpoint{6.482340in}{3.660645in}}%
\pgfpathlineto{\pgfqpoint{6.483414in}{3.663066in}}%
\pgfpathlineto{\pgfqpoint{6.484487in}{3.683885in}}%
\pgfpathlineto{\pgfqpoint{6.485561in}{3.676784in}}%
\pgfpathlineto{\pgfqpoint{6.489857in}{3.690341in}}%
\pgfpathlineto{\pgfqpoint{6.490930in}{3.689373in}}%
\pgfpathlineto{\pgfqpoint{6.493078in}{3.683724in}}%
\pgfpathlineto{\pgfqpoint{6.498447in}{3.702768in}}%
\pgfpathlineto{\pgfqpoint{6.499521in}{3.700025in}}%
\pgfpathlineto{\pgfqpoint{6.500595in}{3.693892in}}%
\pgfpathlineto{\pgfqpoint{6.503817in}{3.698572in}}%
\pgfpathlineto{\pgfqpoint{6.504890in}{3.706965in}}%
\pgfpathlineto{\pgfqpoint{6.505964in}{3.710677in}}%
\pgfpathlineto{\pgfqpoint{6.507038in}{3.705028in}}%
\pgfpathlineto{\pgfqpoint{6.508112in}{3.711000in}}%
\pgfpathlineto{\pgfqpoint{6.511333in}{3.716971in}}%
\pgfpathlineto{\pgfqpoint{6.512407in}{3.715841in}}%
\pgfpathlineto{\pgfqpoint{6.514555in}{3.708094in}}%
\pgfpathlineto{\pgfqpoint{6.515629in}{3.711645in}}%
\pgfpathlineto{\pgfqpoint{6.518850in}{3.711000in}}%
\pgfpathlineto{\pgfqpoint{6.519924in}{3.708740in}}%
\pgfpathlineto{\pgfqpoint{6.520998in}{3.701961in}}%
\pgfpathlineto{\pgfqpoint{6.523146in}{3.711968in}}%
\pgfpathlineto{\pgfqpoint{6.527441in}{3.714873in}}%
\pgfpathlineto{\pgfqpoint{6.528515in}{3.714873in}}%
\pgfpathlineto{\pgfqpoint{6.529589in}{3.718585in}}%
\pgfpathlineto{\pgfqpoint{6.530663in}{3.716648in}}%
\pgfpathlineto{\pgfqpoint{6.533884in}{3.734079in}}%
\pgfpathlineto{\pgfqpoint{6.534958in}{3.726816in}}%
\pgfpathlineto{\pgfqpoint{6.537106in}{3.727462in}}%
\pgfpathlineto{\pgfqpoint{6.538179in}{3.723104in}}%
\pgfpathlineto{\pgfqpoint{6.541401in}{3.721329in}}%
\pgfpathlineto{\pgfqpoint{6.542475in}{3.736823in}}%
\pgfpathlineto{\pgfqpoint{6.543549in}{3.740212in}}%
\pgfpathlineto{\pgfqpoint{6.544622in}{3.713582in}}%
\pgfpathlineto{\pgfqpoint{6.545696in}{3.707610in}}%
\pgfpathlineto{\pgfqpoint{6.548918in}{3.714873in}}%
\pgfpathlineto{\pgfqpoint{6.549992in}{3.713743in}}%
\pgfpathlineto{\pgfqpoint{6.551065in}{3.686790in}}%
\pgfpathlineto{\pgfqpoint{6.552139in}{3.687113in}}%
\pgfpathlineto{\pgfqpoint{6.553213in}{3.688404in}}%
\pgfpathlineto{\pgfqpoint{6.558582in}{3.710354in}}%
\pgfpathlineto{\pgfqpoint{6.559656in}{3.704382in}}%
\pgfpathlineto{\pgfqpoint{6.560730in}{3.708901in}}%
\pgfpathlineto{\pgfqpoint{6.563952in}{3.705674in}}%
\pgfpathlineto{\pgfqpoint{6.565025in}{3.698249in}}%
\pgfpathlineto{\pgfqpoint{6.566099in}{3.695829in}}%
\pgfpathlineto{\pgfqpoint{6.568247in}{3.719715in}}%
\pgfpathlineto{\pgfqpoint{6.571468in}{3.721167in}}%
\pgfpathlineto{\pgfqpoint{6.572542in}{3.716003in}}%
\pgfpathlineto{\pgfqpoint{6.573616in}{3.715519in}}%
\pgfpathlineto{\pgfqpoint{6.574690in}{3.708094in}}%
\pgfpathlineto{\pgfqpoint{6.575764in}{3.657094in}}%
\pgfpathlineto{\pgfqpoint{6.580059in}{3.637727in}}%
\pgfpathlineto{\pgfqpoint{6.581133in}{3.635951in}}%
\pgfpathlineto{\pgfqpoint{6.582207in}{3.645635in}}%
\pgfpathlineto{\pgfqpoint{6.583281in}{3.638695in}}%
\pgfpathlineto{\pgfqpoint{6.586502in}{3.626913in}}%
\pgfpathlineto{\pgfqpoint{6.587576in}{3.628043in}}%
\pgfpathlineto{\pgfqpoint{6.588650in}{3.636436in}}%
\pgfpathlineto{\pgfqpoint{6.589724in}{3.630625in}}%
\pgfpathlineto{\pgfqpoint{6.590797in}{3.631594in}}%
\pgfpathlineto{\pgfqpoint{6.594019in}{3.623524in}}%
\pgfpathlineto{\pgfqpoint{6.596167in}{3.646926in}}%
\pgfpathlineto{\pgfqpoint{6.597240in}{3.649993in}}%
\pgfpathlineto{\pgfqpoint{6.598314in}{3.655803in}}%
\pgfpathlineto{\pgfqpoint{6.601536in}{3.668553in}}%
\pgfpathlineto{\pgfqpoint{6.602610in}{3.666616in}}%
\pgfpathlineto{\pgfqpoint{6.603684in}{3.656771in}}%
\pgfpathlineto{\pgfqpoint{6.604757in}{3.672426in}}%
\pgfpathlineto{\pgfqpoint{6.605831in}{3.659838in}}%
\pgfpathlineto{\pgfqpoint{6.609053in}{3.657417in}}%
\pgfpathlineto{\pgfqpoint{6.610127in}{3.664357in}}%
\pgfpathlineto{\pgfqpoint{6.611200in}{3.658385in}}%
\pgfpathlineto{\pgfqpoint{6.613348in}{3.660160in}}%
\pgfpathlineto{\pgfqpoint{6.616570in}{3.667907in}}%
\pgfpathlineto{\pgfqpoint{6.617643in}{3.674686in}}%
\pgfpathlineto{\pgfqpoint{6.618717in}{3.674363in}}%
\pgfpathlineto{\pgfqpoint{6.620865in}{3.689373in}}%
\pgfpathlineto{\pgfqpoint{6.624086in}{3.705351in}}%
\pgfpathlineto{\pgfqpoint{6.625160in}{3.705189in}}%
\pgfpathlineto{\pgfqpoint{6.626234in}{3.702930in}}%
\pgfpathlineto{\pgfqpoint{6.627308in}{3.685338in}}%
\pgfpathlineto{\pgfqpoint{6.628382in}{3.689534in}}%
\pgfpathlineto{\pgfqpoint{6.631603in}{3.687436in}}%
\pgfpathlineto{\pgfqpoint{6.632677in}{3.681626in}}%
\pgfpathlineto{\pgfqpoint{6.633751in}{3.697281in}}%
\pgfpathlineto{\pgfqpoint{6.634825in}{3.699056in}}%
\pgfpathlineto{\pgfqpoint{6.635899in}{3.712613in}}%
\pgfpathlineto{\pgfqpoint{6.639120in}{3.712613in}}%
\pgfpathlineto{\pgfqpoint{6.641268in}{3.707126in}}%
\pgfpathlineto{\pgfqpoint{6.642342in}{3.709386in}}%
\pgfpathlineto{\pgfqpoint{6.643416in}{3.716326in}}%
\pgfpathlineto{\pgfqpoint{6.647711in}{3.721652in}}%
\pgfpathlineto{\pgfqpoint{6.648785in}{3.715841in}}%
\pgfpathlineto{\pgfqpoint{6.649859in}{3.715357in}}%
\pgfpathlineto{\pgfqpoint{6.650932in}{3.712613in}}%
\pgfpathlineto{\pgfqpoint{6.650932in}{3.712613in}}%
\pgfusepath{stroke}%
\end{pgfscope}%
\begin{pgfscope}%
\pgfpathrectangle{\pgfqpoint{4.185023in}{3.037108in}}{\pgfqpoint{2.583333in}{0.736585in}}%
\pgfusepath{clip}%
\pgfsetroundcap%
\pgfsetroundjoin%
\pgfsetlinewidth{1.505625pt}%
\definecolor{currentstroke}{rgb}{1.000000,0.647059,0.000000}%
\pgfsetstrokecolor{currentstroke}%
\pgfsetdash{}{0pt}%
\pgfpathmoveto{\pgfqpoint{4.302448in}{3.153707in}}%
\pgfpathlineto{\pgfqpoint{4.303521in}{3.153464in}}%
\pgfpathlineto{\pgfqpoint{4.305669in}{3.151084in}}%
\pgfpathlineto{\pgfqpoint{4.308891in}{3.151124in}}%
\pgfpathlineto{\pgfqpoint{4.309964in}{3.150479in}}%
\pgfpathlineto{\pgfqpoint{4.312112in}{3.147856in}}%
\pgfpathlineto{\pgfqpoint{4.313186in}{3.147071in}}%
\pgfpathlineto{\pgfqpoint{4.318555in}{3.147295in}}%
\pgfpathlineto{\pgfqpoint{4.320703in}{3.147983in}}%
\pgfpathlineto{\pgfqpoint{4.323924in}{3.147205in}}%
\pgfpathlineto{\pgfqpoint{4.326072in}{3.145536in}}%
\pgfpathlineto{\pgfqpoint{4.331441in}{3.142401in}}%
\pgfpathlineto{\pgfqpoint{4.335737in}{3.137244in}}%
\pgfpathlineto{\pgfqpoint{4.340032in}{3.135818in}}%
\pgfpathlineto{\pgfqpoint{4.343253in}{3.134391in}}%
\pgfpathlineto{\pgfqpoint{4.356140in}{3.133924in}}%
\pgfpathlineto{\pgfqpoint{4.358287in}{3.135172in}}%
\pgfpathlineto{\pgfqpoint{4.361509in}{3.135792in}}%
\pgfpathlineto{\pgfqpoint{4.365804in}{3.138443in}}%
\pgfpathlineto{\pgfqpoint{4.371173in}{3.139873in}}%
\pgfpathlineto{\pgfqpoint{4.373321in}{3.140792in}}%
\pgfpathlineto{\pgfqpoint{4.377616in}{3.142099in}}%
\pgfpathlineto{\pgfqpoint{4.380838in}{3.143713in}}%
\pgfpathlineto{\pgfqpoint{4.386207in}{3.144880in}}%
\pgfpathlineto{\pgfqpoint{4.388355in}{3.145713in}}%
\pgfpathlineto{\pgfqpoint{4.394798in}{3.146986in}}%
\pgfpathlineto{\pgfqpoint{4.400167in}{3.147886in}}%
\pgfpathlineto{\pgfqpoint{4.402315in}{3.148439in}}%
\pgfpathlineto{\pgfqpoint{4.421644in}{3.149756in}}%
\pgfpathlineto{\pgfqpoint{4.425939in}{3.150212in}}%
\pgfpathlineto{\pgfqpoint{4.438825in}{3.148517in}}%
\pgfpathlineto{\pgfqpoint{4.466745in}{3.143740in}}%
\pgfpathlineto{\pgfqpoint{4.471040in}{3.142409in}}%
\pgfpathlineto{\pgfqpoint{4.481779in}{3.140972in}}%
\pgfpathlineto{\pgfqpoint{4.485000in}{3.139687in}}%
\pgfpathlineto{\pgfqpoint{4.486074in}{3.139172in}}%
\pgfpathlineto{\pgfqpoint{4.490369in}{3.138051in}}%
\pgfpathlineto{\pgfqpoint{4.493591in}{3.136813in}}%
\pgfpathlineto{\pgfqpoint{4.504329in}{3.135290in}}%
\pgfpathlineto{\pgfqpoint{4.508625in}{3.134780in}}%
\pgfpathlineto{\pgfqpoint{4.523658in}{3.135533in}}%
\pgfpathlineto{\pgfqpoint{4.536544in}{3.136483in}}%
\pgfpathlineto{\pgfqpoint{4.546209in}{3.138143in}}%
\pgfpathlineto{\pgfqpoint{4.552652in}{3.139027in}}%
\pgfpathlineto{\pgfqpoint{4.561243in}{3.140386in}}%
\pgfpathlineto{\pgfqpoint{4.567686in}{3.141174in}}%
\pgfpathlineto{\pgfqpoint{4.568760in}{3.141496in}}%
\pgfpathlineto{\pgfqpoint{4.574129in}{3.142390in}}%
\pgfpathlineto{\pgfqpoint{4.576276in}{3.143077in}}%
\pgfpathlineto{\pgfqpoint{4.581646in}{3.144125in}}%
\pgfpathlineto{\pgfqpoint{4.583793in}{3.144839in}}%
\pgfpathlineto{\pgfqpoint{4.589162in}{3.145893in}}%
\pgfpathlineto{\pgfqpoint{4.591310in}{3.146547in}}%
\pgfpathlineto{\pgfqpoint{4.596679in}{3.147462in}}%
\pgfpathlineto{\pgfqpoint{4.598827in}{3.148105in}}%
\pgfpathlineto{\pgfqpoint{4.609565in}{3.149542in}}%
\pgfpathlineto{\pgfqpoint{4.621378in}{3.152080in}}%
\pgfpathlineto{\pgfqpoint{4.632116in}{3.153203in}}%
\pgfpathlineto{\pgfqpoint{4.636411in}{3.153954in}}%
\pgfpathlineto{\pgfqpoint{4.648224in}{3.154956in}}%
\pgfpathlineto{\pgfqpoint{4.658962in}{3.156773in}}%
\pgfpathlineto{\pgfqpoint{4.665405in}{3.157815in}}%
\pgfpathlineto{\pgfqpoint{4.666479in}{3.158110in}}%
\pgfpathlineto{\pgfqpoint{4.671848in}{3.159024in}}%
\pgfpathlineto{\pgfqpoint{4.673996in}{3.159564in}}%
\pgfpathlineto{\pgfqpoint{4.680439in}{3.160561in}}%
\pgfpathlineto{\pgfqpoint{4.684734in}{3.160926in}}%
\pgfpathlineto{\pgfqpoint{4.701916in}{3.162577in}}%
\pgfpathlineto{\pgfqpoint{4.711580in}{3.164129in}}%
\pgfpathlineto{\pgfqpoint{4.718023in}{3.164934in}}%
\pgfpathlineto{\pgfqpoint{4.719097in}{3.165341in}}%
\pgfpathlineto{\pgfqpoint{4.723392in}{3.166256in}}%
\pgfpathlineto{\pgfqpoint{4.726614in}{3.167750in}}%
\pgfpathlineto{\pgfqpoint{4.730909in}{3.168736in}}%
\pgfpathlineto{\pgfqpoint{4.734131in}{3.170265in}}%
\pgfpathlineto{\pgfqpoint{4.738426in}{3.171255in}}%
\pgfpathlineto{\pgfqpoint{4.741648in}{3.172815in}}%
\pgfpathlineto{\pgfqpoint{4.747017in}{3.173893in}}%
\pgfpathlineto{\pgfqpoint{4.749164in}{3.174937in}}%
\pgfpathlineto{\pgfqpoint{4.753460in}{3.175872in}}%
\pgfpathlineto{\pgfqpoint{4.756681in}{3.177317in}}%
\pgfpathlineto{\pgfqpoint{4.760977in}{3.178298in}}%
\pgfpathlineto{\pgfqpoint{4.764198in}{3.179776in}}%
\pgfpathlineto{\pgfqpoint{4.768493in}{3.180760in}}%
\pgfpathlineto{\pgfqpoint{4.771715in}{3.182155in}}%
\pgfpathlineto{\pgfqpoint{4.776010in}{3.183053in}}%
\pgfpathlineto{\pgfqpoint{4.779232in}{3.184469in}}%
\pgfpathlineto{\pgfqpoint{4.783527in}{3.185371in}}%
\pgfpathlineto{\pgfqpoint{4.785675in}{3.186263in}}%
\pgfpathlineto{\pgfqpoint{4.791044in}{3.187251in}}%
\pgfpathlineto{\pgfqpoint{4.794266in}{3.188705in}}%
\pgfpathlineto{\pgfqpoint{4.798561in}{3.189679in}}%
\pgfpathlineto{\pgfqpoint{4.801782in}{3.191256in}}%
\pgfpathlineto{\pgfqpoint{4.806078in}{3.192308in}}%
\pgfpathlineto{\pgfqpoint{4.809299in}{3.193892in}}%
\pgfpathlineto{\pgfqpoint{4.813595in}{3.195081in}}%
\pgfpathlineto{\pgfqpoint{4.816816in}{3.196303in}}%
\pgfpathlineto{\pgfqpoint{4.822185in}{3.197527in}}%
\pgfpathlineto{\pgfqpoint{4.824333in}{3.198397in}}%
\pgfpathlineto{\pgfqpoint{4.828628in}{3.199246in}}%
\pgfpathlineto{\pgfqpoint{4.831850in}{3.200575in}}%
\pgfpathlineto{\pgfqpoint{4.836145in}{3.201491in}}%
\pgfpathlineto{\pgfqpoint{4.839367in}{3.202991in}}%
\pgfpathlineto{\pgfqpoint{4.843662in}{3.203874in}}%
\pgfpathlineto{\pgfqpoint{4.846884in}{3.205276in}}%
\pgfpathlineto{\pgfqpoint{4.852253in}{3.206203in}}%
\pgfpathlineto{\pgfqpoint{4.854401in}{3.206973in}}%
\pgfpathlineto{\pgfqpoint{4.859770in}{3.208035in}}%
\pgfpathlineto{\pgfqpoint{4.861917in}{3.208738in}}%
\pgfpathlineto{\pgfqpoint{4.867287in}{3.209848in}}%
\pgfpathlineto{\pgfqpoint{4.869434in}{3.210599in}}%
\pgfpathlineto{\pgfqpoint{4.874804in}{3.211744in}}%
\pgfpathlineto{\pgfqpoint{4.884468in}{3.213928in}}%
\pgfpathlineto{\pgfqpoint{4.889837in}{3.214996in}}%
\pgfpathlineto{\pgfqpoint{4.898428in}{3.216930in}}%
\pgfpathlineto{\pgfqpoint{4.899502in}{3.217383in}}%
\pgfpathlineto{\pgfqpoint{4.904871in}{3.218678in}}%
\pgfpathlineto{\pgfqpoint{4.907019in}{3.219557in}}%
\pgfpathlineto{\pgfqpoint{4.912388in}{3.220856in}}%
\pgfpathlineto{\pgfqpoint{4.914536in}{3.221683in}}%
\pgfpathlineto{\pgfqpoint{4.919905in}{3.222895in}}%
\pgfpathlineto{\pgfqpoint{4.922052in}{3.223772in}}%
\pgfpathlineto{\pgfqpoint{4.926348in}{3.224648in}}%
\pgfpathlineto{\pgfqpoint{4.929569in}{3.225981in}}%
\pgfpathlineto{\pgfqpoint{4.934938in}{3.227253in}}%
\pgfpathlineto{\pgfqpoint{4.937086in}{3.228004in}}%
\pgfpathlineto{\pgfqpoint{4.942455in}{3.229050in}}%
\pgfpathlineto{\pgfqpoint{4.944603in}{3.229764in}}%
\pgfpathlineto{\pgfqpoint{4.951046in}{3.230887in}}%
\pgfpathlineto{\pgfqpoint{4.952120in}{3.231172in}}%
\pgfpathlineto{\pgfqpoint{4.958563in}{3.231976in}}%
\pgfpathlineto{\pgfqpoint{4.967154in}{3.233680in}}%
\pgfpathlineto{\pgfqpoint{4.972523in}{3.234696in}}%
\pgfpathlineto{\pgfqpoint{4.974670in}{3.235344in}}%
\pgfpathlineto{\pgfqpoint{4.981114in}{3.236459in}}%
\pgfpathlineto{\pgfqpoint{4.986483in}{3.237090in}}%
\pgfpathlineto{\pgfqpoint{4.997221in}{3.238808in}}%
\pgfpathlineto{\pgfqpoint{5.003664in}{3.239904in}}%
\pgfpathlineto{\pgfqpoint{5.004738in}{3.240211in}}%
\pgfpathlineto{\pgfqpoint{5.010107in}{3.241182in}}%
\pgfpathlineto{\pgfqpoint{5.012255in}{3.241835in}}%
\pgfpathlineto{\pgfqpoint{5.017624in}{3.242932in}}%
\pgfpathlineto{\pgfqpoint{5.019772in}{3.243608in}}%
\pgfpathlineto{\pgfqpoint{5.025141in}{3.244689in}}%
\pgfpathlineto{\pgfqpoint{5.027289in}{3.245435in}}%
\pgfpathlineto{\pgfqpoint{5.032658in}{3.246579in}}%
\pgfpathlineto{\pgfqpoint{5.034805in}{3.247433in}}%
\pgfpathlineto{\pgfqpoint{5.040175in}{3.248695in}}%
\pgfpathlineto{\pgfqpoint{5.042322in}{3.249543in}}%
\pgfpathlineto{\pgfqpoint{5.047692in}{3.250799in}}%
\pgfpathlineto{\pgfqpoint{5.061651in}{3.253858in}}%
\pgfpathlineto{\pgfqpoint{5.064873in}{3.254905in}}%
\pgfpathlineto{\pgfqpoint{5.070242in}{3.255839in}}%
\pgfpathlineto{\pgfqpoint{5.072390in}{3.256467in}}%
\pgfpathlineto{\pgfqpoint{5.083128in}{3.257988in}}%
\pgfpathlineto{\pgfqpoint{5.087424in}{3.258816in}}%
\pgfpathlineto{\pgfqpoint{5.093867in}{3.259892in}}%
\pgfpathlineto{\pgfqpoint{5.102457in}{3.261431in}}%
\pgfpathlineto{\pgfqpoint{5.113196in}{3.262574in}}%
\pgfpathlineto{\pgfqpoint{5.117491in}{3.263330in}}%
\pgfpathlineto{\pgfqpoint{5.128229in}{3.264317in}}%
\pgfpathlineto{\pgfqpoint{5.132525in}{3.265129in}}%
\pgfpathlineto{\pgfqpoint{5.143263in}{3.266075in}}%
\pgfpathlineto{\pgfqpoint{5.155075in}{3.267767in}}%
\pgfpathlineto{\pgfqpoint{5.161518in}{3.268587in}}%
\pgfpathlineto{\pgfqpoint{5.170109in}{3.269786in}}%
\pgfpathlineto{\pgfqpoint{5.176552in}{3.270633in}}%
\pgfpathlineto{\pgfqpoint{5.185143in}{3.271960in}}%
\pgfpathlineto{\pgfqpoint{5.191586in}{3.272923in}}%
\pgfpathlineto{\pgfqpoint{5.199103in}{3.274115in}}%
\pgfpathlineto{\pgfqpoint{5.206620in}{3.275076in}}%
\pgfpathlineto{\pgfqpoint{5.215210in}{3.276704in}}%
\pgfpathlineto{\pgfqpoint{5.221653in}{3.277725in}}%
\pgfpathlineto{\pgfqpoint{5.227023in}{3.278492in}}%
\pgfpathlineto{\pgfqpoint{5.234539in}{3.279584in}}%
\pgfpathlineto{\pgfqpoint{5.237761in}{3.280238in}}%
\pgfpathlineto{\pgfqpoint{5.248499in}{3.281291in}}%
\pgfpathlineto{\pgfqpoint{5.260312in}{3.283065in}}%
\pgfpathlineto{\pgfqpoint{5.271050in}{3.284198in}}%
\pgfpathlineto{\pgfqpoint{5.281788in}{3.285567in}}%
\pgfpathlineto{\pgfqpoint{5.289305in}{3.286397in}}%
\pgfpathlineto{\pgfqpoint{5.297896in}{3.287654in}}%
\pgfpathlineto{\pgfqpoint{5.308634in}{3.288798in}}%
\pgfpathlineto{\pgfqpoint{5.409575in}{3.304487in}}%
\pgfpathlineto{\pgfqpoint{5.410649in}{3.304791in}}%
\pgfpathlineto{\pgfqpoint{5.416018in}{3.305760in}}%
\pgfpathlineto{\pgfqpoint{5.418166in}{3.306427in}}%
\pgfpathlineto{\pgfqpoint{5.423535in}{3.307451in}}%
\pgfpathlineto{\pgfqpoint{5.425683in}{3.308084in}}%
\pgfpathlineto{\pgfqpoint{5.431052in}{3.309016in}}%
\pgfpathlineto{\pgfqpoint{5.433200in}{3.309647in}}%
\pgfpathlineto{\pgfqpoint{5.438569in}{3.310591in}}%
\pgfpathlineto{\pgfqpoint{5.452529in}{3.313355in}}%
\pgfpathlineto{\pgfqpoint{5.455750in}{3.314348in}}%
\pgfpathlineto{\pgfqpoint{5.461119in}{3.315319in}}%
\pgfpathlineto{\pgfqpoint{5.463267in}{3.316047in}}%
\pgfpathlineto{\pgfqpoint{5.468636in}{3.317186in}}%
\pgfpathlineto{\pgfqpoint{5.530919in}{3.327962in}}%
\pgfpathlineto{\pgfqpoint{5.537362in}{3.328848in}}%
\pgfpathlineto{\pgfqpoint{5.545953in}{3.330073in}}%
\pgfpathlineto{\pgfqpoint{5.557765in}{3.331210in}}%
\pgfpathlineto{\pgfqpoint{5.568503in}{3.332648in}}%
\pgfpathlineto{\pgfqpoint{5.581389in}{3.333763in}}%
\pgfpathlineto{\pgfqpoint{5.591054in}{3.334948in}}%
\pgfpathlineto{\pgfqpoint{5.603940in}{3.336094in}}%
\pgfpathlineto{\pgfqpoint{5.628638in}{3.338205in}}%
\pgfpathlineto{\pgfqpoint{5.647967in}{3.339212in}}%
\pgfpathlineto{\pgfqpoint{5.666222in}{3.340584in}}%
\pgfpathlineto{\pgfqpoint{5.680182in}{3.341453in}}%
\pgfpathlineto{\pgfqpoint{5.696290in}{3.342931in}}%
\pgfpathlineto{\pgfqpoint{5.733874in}{3.343833in}}%
\pgfpathlineto{\pgfqpoint{5.761794in}{3.342999in}}%
\pgfpathlineto{\pgfqpoint{5.784345in}{3.343081in}}%
\pgfpathlineto{\pgfqpoint{5.823003in}{3.344446in}}%
\pgfpathlineto{\pgfqpoint{5.919648in}{3.349798in}}%
\pgfpathlineto{\pgfqpoint{5.929313in}{3.350757in}}%
\pgfpathlineto{\pgfqpoint{5.940051in}{3.351604in}}%
\pgfpathlineto{\pgfqpoint{5.951864in}{3.352852in}}%
\pgfpathlineto{\pgfqpoint{5.965823in}{3.354005in}}%
\pgfpathlineto{\pgfqpoint{5.974414in}{3.354906in}}%
\pgfpathlineto{\pgfqpoint{5.986226in}{3.355913in}}%
\pgfpathlineto{\pgfqpoint{5.996965in}{3.356850in}}%
\pgfpathlineto{\pgfqpoint{6.008777in}{3.357768in}}%
\pgfpathlineto{\pgfqpoint{6.019515in}{3.358733in}}%
\pgfpathlineto{\pgfqpoint{6.033475in}{3.359709in}}%
\pgfpathlineto{\pgfqpoint{6.049583in}{3.361265in}}%
\pgfpathlineto{\pgfqpoint{6.061395in}{3.362237in}}%
\pgfpathlineto{\pgfqpoint{6.072133in}{3.363380in}}%
\pgfpathlineto{\pgfqpoint{6.082872in}{3.364421in}}%
\pgfpathlineto{\pgfqpoint{6.094684in}{3.365974in}}%
\pgfpathlineto{\pgfqpoint{6.105422in}{3.367058in}}%
\pgfpathlineto{\pgfqpoint{6.117235in}{3.368743in}}%
\pgfpathlineto{\pgfqpoint{6.127973in}{3.369907in}}%
\pgfpathlineto{\pgfqpoint{6.138711in}{3.371286in}}%
\pgfpathlineto{\pgfqpoint{6.147302in}{3.372416in}}%
\pgfpathlineto{\pgfqpoint{6.158041in}{3.373592in}}%
\pgfpathlineto{\pgfqpoint{6.169853in}{3.375419in}}%
\pgfpathlineto{\pgfqpoint{6.180591in}{3.376582in}}%
\pgfpathlineto{\pgfqpoint{6.192403in}{3.378105in}}%
\pgfpathlineto{\pgfqpoint{6.203142in}{3.379144in}}%
\pgfpathlineto{\pgfqpoint{6.210659in}{3.379890in}}%
\pgfpathlineto{\pgfqpoint{6.222471in}{3.380912in}}%
\pgfpathlineto{\pgfqpoint{6.235357in}{3.381913in}}%
\pgfpathlineto{\pgfqpoint{6.252538in}{3.383636in}}%
\pgfpathlineto{\pgfqpoint{6.266498in}{3.384612in}}%
\pgfpathlineto{\pgfqpoint{6.275089in}{3.385417in}}%
\pgfpathlineto{\pgfqpoint{6.285827in}{3.386217in}}%
\pgfpathlineto{\pgfqpoint{6.297640in}{3.387801in}}%
\pgfpathlineto{\pgfqpoint{6.308378in}{3.388905in}}%
\pgfpathlineto{\pgfqpoint{6.320190in}{3.390606in}}%
\pgfpathlineto{\pgfqpoint{6.326633in}{3.391465in}}%
\pgfpathlineto{\pgfqpoint{6.335224in}{3.392697in}}%
\pgfpathlineto{\pgfqpoint{6.341667in}{3.393551in}}%
\pgfpathlineto{\pgfqpoint{6.350258in}{3.394799in}}%
\pgfpathlineto{\pgfqpoint{6.360996in}{3.395988in}}%
\pgfpathlineto{\pgfqpoint{6.371734in}{3.397513in}}%
\pgfpathlineto{\pgfqpoint{6.379251in}{3.398296in}}%
\pgfpathlineto{\pgfqpoint{6.387842in}{3.399380in}}%
\pgfpathlineto{\pgfqpoint{6.399654in}{3.400479in}}%
\pgfpathlineto{\pgfqpoint{6.410393in}{3.401695in}}%
\pgfpathlineto{\pgfqpoint{6.423279in}{3.402801in}}%
\pgfpathlineto{\pgfqpoint{6.432943in}{3.404012in}}%
\pgfpathlineto{\pgfqpoint{6.443682in}{3.405074in}}%
\pgfpathlineto{\pgfqpoint{6.455494in}{3.406621in}}%
\pgfpathlineto{\pgfqpoint{6.469454in}{3.407858in}}%
\pgfpathlineto{\pgfqpoint{6.485561in}{3.409774in}}%
\pgfpathlineto{\pgfqpoint{6.496300in}{3.410962in}}%
\pgfpathlineto{\pgfqpoint{6.508112in}{3.412816in}}%
\pgfpathlineto{\pgfqpoint{6.518850in}{3.414080in}}%
\pgfpathlineto{\pgfqpoint{6.523146in}{3.414903in}}%
\pgfpathlineto{\pgfqpoint{6.533884in}{3.415967in}}%
\pgfpathlineto{\pgfqpoint{6.545696in}{3.417897in}}%
\pgfpathlineto{\pgfqpoint{6.556435in}{3.419061in}}%
\pgfpathlineto{\pgfqpoint{6.568247in}{3.420835in}}%
\pgfpathlineto{\pgfqpoint{6.580059in}{3.422101in}}%
\pgfpathlineto{\pgfqpoint{6.590797in}{3.423255in}}%
\pgfpathlineto{\pgfqpoint{6.601536in}{3.424164in}}%
\pgfpathlineto{\pgfqpoint{6.613348in}{3.425445in}}%
\pgfpathlineto{\pgfqpoint{6.624086in}{3.426480in}}%
\pgfpathlineto{\pgfqpoint{6.635899in}{3.428095in}}%
\pgfpathlineto{\pgfqpoint{6.647711in}{3.429229in}}%
\pgfpathlineto{\pgfqpoint{6.650932in}{3.429797in}}%
\pgfpathlineto{\pgfqpoint{6.650932in}{3.429797in}}%
\pgfusepath{stroke}%
\end{pgfscope}%
\begin{pgfscope}%
\pgfsetrectcap%
\pgfsetmiterjoin%
\pgfsetlinewidth{0.803000pt}%
\definecolor{currentstroke}{rgb}{1.000000,1.000000,1.000000}%
\pgfsetstrokecolor{currentstroke}%
\pgfsetdash{}{0pt}%
\pgfpathmoveto{\pgfqpoint{4.185023in}{3.037108in}}%
\pgfpathlineto{\pgfqpoint{4.185023in}{3.773693in}}%
\pgfusepath{stroke}%
\end{pgfscope}%
\begin{pgfscope}%
\pgfsetrectcap%
\pgfsetmiterjoin%
\pgfsetlinewidth{0.803000pt}%
\definecolor{currentstroke}{rgb}{1.000000,1.000000,1.000000}%
\pgfsetstrokecolor{currentstroke}%
\pgfsetdash{}{0pt}%
\pgfpathmoveto{\pgfqpoint{6.768357in}{3.037108in}}%
\pgfpathlineto{\pgfqpoint{6.768357in}{3.773693in}}%
\pgfusepath{stroke}%
\end{pgfscope}%
\begin{pgfscope}%
\pgfsetrectcap%
\pgfsetmiterjoin%
\pgfsetlinewidth{0.803000pt}%
\definecolor{currentstroke}{rgb}{1.000000,1.000000,1.000000}%
\pgfsetstrokecolor{currentstroke}%
\pgfsetdash{}{0pt}%
\pgfpathmoveto{\pgfqpoint{4.185023in}{3.037108in}}%
\pgfpathlineto{\pgfqpoint{6.768357in}{3.037108in}}%
\pgfusepath{stroke}%
\end{pgfscope}%
\begin{pgfscope}%
\pgfsetrectcap%
\pgfsetmiterjoin%
\pgfsetlinewidth{0.803000pt}%
\definecolor{currentstroke}{rgb}{1.000000,1.000000,1.000000}%
\pgfsetstrokecolor{currentstroke}%
\pgfsetdash{}{0pt}%
\pgfpathmoveto{\pgfqpoint{4.185023in}{3.773693in}}%
\pgfpathlineto{\pgfqpoint{6.768357in}{3.773693in}}%
\pgfusepath{stroke}%
\end{pgfscope}%
\begin{pgfscope}%
\definecolor{textcolor}{rgb}{0.150000,0.150000,0.150000}%
\pgfsetstrokecolor{textcolor}%
\pgfsetfillcolor{textcolor}%
\pgftext[x=5.476690in,y=3.857026in,,base]{\color{textcolor}\rmfamily\fontsize{16.800000}{20.160000}\selectfont PG}%
\end{pgfscope}%
\begin{pgfscope}%
\pgfsetbuttcap%
\pgfsetmiterjoin%
\definecolor{currentfill}{rgb}{0.917647,0.917647,0.949020}%
\pgfsetfillcolor{currentfill}%
\pgfsetlinewidth{0.000000pt}%
\definecolor{currentstroke}{rgb}{0.000000,0.000000,0.000000}%
\pgfsetstrokecolor{currentstroke}%
\pgfsetstrokeopacity{0.000000}%
\pgfsetdash{}{0pt}%
\pgfpathmoveto{\pgfqpoint{0.568357in}{1.711254in}}%
\pgfpathlineto{\pgfqpoint{3.151690in}{1.711254in}}%
\pgfpathlineto{\pgfqpoint{3.151690in}{2.447839in}}%
\pgfpathlineto{\pgfqpoint{0.568357in}{2.447839in}}%
\pgfpathclose%
\pgfusepath{fill}%
\end{pgfscope}%
\begin{pgfscope}%
\pgfpathrectangle{\pgfqpoint{0.568357in}{1.711254in}}{\pgfqpoint{2.583333in}{0.736585in}}%
\pgfusepath{clip}%
\pgfsetroundcap%
\pgfsetroundjoin%
\pgfsetlinewidth{0.803000pt}%
\definecolor{currentstroke}{rgb}{1.000000,1.000000,1.000000}%
\pgfsetstrokecolor{currentstroke}%
\pgfsetdash{}{0pt}%
\pgfpathmoveto{\pgfqpoint{0.683633in}{1.711254in}}%
\pgfpathlineto{\pgfqpoint{0.683633in}{2.447839in}}%
\pgfusepath{stroke}%
\end{pgfscope}%
\begin{pgfscope}%
\definecolor{textcolor}{rgb}{0.150000,0.150000,0.150000}%
\pgfsetstrokecolor{textcolor}%
\pgfsetfillcolor{textcolor}%
\pgftext[x=0.683633in,y=1.614032in,,top]{\color{textcolor}\rmfamily\fontsize{14.000000}{16.800000}\selectfont 2012}%
\end{pgfscope}%
\begin{pgfscope}%
\pgfpathrectangle{\pgfqpoint{0.568357in}{1.711254in}}{\pgfqpoint{2.583333in}{0.736585in}}%
\pgfusepath{clip}%
\pgfsetroundcap%
\pgfsetroundjoin%
\pgfsetlinewidth{0.803000pt}%
\definecolor{currentstroke}{rgb}{1.000000,1.000000,1.000000}%
\pgfsetstrokecolor{currentstroke}%
\pgfsetdash{}{0pt}%
\pgfpathmoveto{\pgfqpoint{1.076658in}{1.711254in}}%
\pgfpathlineto{\pgfqpoint{1.076658in}{2.447839in}}%
\pgfusepath{stroke}%
\end{pgfscope}%
\begin{pgfscope}%
\definecolor{textcolor}{rgb}{0.150000,0.150000,0.150000}%
\pgfsetstrokecolor{textcolor}%
\pgfsetfillcolor{textcolor}%
\pgftext[x=1.076658in,y=1.614032in,,top]{\color{textcolor}\rmfamily\fontsize{14.000000}{16.800000}\selectfont 2013}%
\end{pgfscope}%
\begin{pgfscope}%
\pgfpathrectangle{\pgfqpoint{0.568357in}{1.711254in}}{\pgfqpoint{2.583333in}{0.736585in}}%
\pgfusepath{clip}%
\pgfsetroundcap%
\pgfsetroundjoin%
\pgfsetlinewidth{0.803000pt}%
\definecolor{currentstroke}{rgb}{1.000000,1.000000,1.000000}%
\pgfsetstrokecolor{currentstroke}%
\pgfsetdash{}{0pt}%
\pgfpathmoveto{\pgfqpoint{1.468609in}{1.711254in}}%
\pgfpathlineto{\pgfqpoint{1.468609in}{2.447839in}}%
\pgfusepath{stroke}%
\end{pgfscope}%
\begin{pgfscope}%
\definecolor{textcolor}{rgb}{0.150000,0.150000,0.150000}%
\pgfsetstrokecolor{textcolor}%
\pgfsetfillcolor{textcolor}%
\pgftext[x=1.468609in,y=1.614032in,,top]{\color{textcolor}\rmfamily\fontsize{14.000000}{16.800000}\selectfont 2014}%
\end{pgfscope}%
\begin{pgfscope}%
\pgfpathrectangle{\pgfqpoint{0.568357in}{1.711254in}}{\pgfqpoint{2.583333in}{0.736585in}}%
\pgfusepath{clip}%
\pgfsetroundcap%
\pgfsetroundjoin%
\pgfsetlinewidth{0.803000pt}%
\definecolor{currentstroke}{rgb}{1.000000,1.000000,1.000000}%
\pgfsetstrokecolor{currentstroke}%
\pgfsetdash{}{0pt}%
\pgfpathmoveto{\pgfqpoint{1.860560in}{1.711254in}}%
\pgfpathlineto{\pgfqpoint{1.860560in}{2.447839in}}%
\pgfusepath{stroke}%
\end{pgfscope}%
\begin{pgfscope}%
\definecolor{textcolor}{rgb}{0.150000,0.150000,0.150000}%
\pgfsetstrokecolor{textcolor}%
\pgfsetfillcolor{textcolor}%
\pgftext[x=1.860560in,y=1.614032in,,top]{\color{textcolor}\rmfamily\fontsize{14.000000}{16.800000}\selectfont 2015}%
\end{pgfscope}%
\begin{pgfscope}%
\pgfpathrectangle{\pgfqpoint{0.568357in}{1.711254in}}{\pgfqpoint{2.583333in}{0.736585in}}%
\pgfusepath{clip}%
\pgfsetroundcap%
\pgfsetroundjoin%
\pgfsetlinewidth{0.803000pt}%
\definecolor{currentstroke}{rgb}{1.000000,1.000000,1.000000}%
\pgfsetstrokecolor{currentstroke}%
\pgfsetdash{}{0pt}%
\pgfpathmoveto{\pgfqpoint{2.252511in}{1.711254in}}%
\pgfpathlineto{\pgfqpoint{2.252511in}{2.447839in}}%
\pgfusepath{stroke}%
\end{pgfscope}%
\begin{pgfscope}%
\definecolor{textcolor}{rgb}{0.150000,0.150000,0.150000}%
\pgfsetstrokecolor{textcolor}%
\pgfsetfillcolor{textcolor}%
\pgftext[x=2.252511in,y=1.614032in,,top]{\color{textcolor}\rmfamily\fontsize{14.000000}{16.800000}\selectfont 2016}%
\end{pgfscope}%
\begin{pgfscope}%
\pgfpathrectangle{\pgfqpoint{0.568357in}{1.711254in}}{\pgfqpoint{2.583333in}{0.736585in}}%
\pgfusepath{clip}%
\pgfsetroundcap%
\pgfsetroundjoin%
\pgfsetlinewidth{0.803000pt}%
\definecolor{currentstroke}{rgb}{1.000000,1.000000,1.000000}%
\pgfsetstrokecolor{currentstroke}%
\pgfsetdash{}{0pt}%
\pgfpathmoveto{\pgfqpoint{2.645536in}{1.711254in}}%
\pgfpathlineto{\pgfqpoint{2.645536in}{2.447839in}}%
\pgfusepath{stroke}%
\end{pgfscope}%
\begin{pgfscope}%
\definecolor{textcolor}{rgb}{0.150000,0.150000,0.150000}%
\pgfsetstrokecolor{textcolor}%
\pgfsetfillcolor{textcolor}%
\pgftext[x=2.645536in,y=1.614032in,,top]{\color{textcolor}\rmfamily\fontsize{14.000000}{16.800000}\selectfont 2017}%
\end{pgfscope}%
\begin{pgfscope}%
\pgfpathrectangle{\pgfqpoint{0.568357in}{1.711254in}}{\pgfqpoint{2.583333in}{0.736585in}}%
\pgfusepath{clip}%
\pgfsetroundcap%
\pgfsetroundjoin%
\pgfsetlinewidth{0.803000pt}%
\definecolor{currentstroke}{rgb}{1.000000,1.000000,1.000000}%
\pgfsetstrokecolor{currentstroke}%
\pgfsetdash{}{0pt}%
\pgfpathmoveto{\pgfqpoint{3.037487in}{1.711254in}}%
\pgfpathlineto{\pgfqpoint{3.037487in}{2.447839in}}%
\pgfusepath{stroke}%
\end{pgfscope}%
\begin{pgfscope}%
\definecolor{textcolor}{rgb}{0.150000,0.150000,0.150000}%
\pgfsetstrokecolor{textcolor}%
\pgfsetfillcolor{textcolor}%
\pgftext[x=3.037487in,y=1.614032in,,top]{\color{textcolor}\rmfamily\fontsize{14.000000}{16.800000}\selectfont 2018}%
\end{pgfscope}%
\begin{pgfscope}%
\pgfpathrectangle{\pgfqpoint{0.568357in}{1.711254in}}{\pgfqpoint{2.583333in}{0.736585in}}%
\pgfusepath{clip}%
\pgfsetroundcap%
\pgfsetroundjoin%
\pgfsetlinewidth{0.803000pt}%
\definecolor{currentstroke}{rgb}{1.000000,1.000000,1.000000}%
\pgfsetstrokecolor{currentstroke}%
\pgfsetdash{}{0pt}%
\pgfpathmoveto{\pgfqpoint{0.568357in}{1.905563in}}%
\pgfpathlineto{\pgfqpoint{3.151690in}{1.905563in}}%
\pgfusepath{stroke}%
\end{pgfscope}%
\begin{pgfscope}%
\definecolor{textcolor}{rgb}{0.150000,0.150000,0.150000}%
\pgfsetstrokecolor{textcolor}%
\pgfsetfillcolor{textcolor}%
\pgftext[x=0.223712in,y=1.831697in,left,base]{\color{textcolor}\rmfamily\fontsize{14.000000}{16.800000}\selectfont 75}%
\end{pgfscope}%
\begin{pgfscope}%
\pgfpathrectangle{\pgfqpoint{0.568357in}{1.711254in}}{\pgfqpoint{2.583333in}{0.736585in}}%
\pgfusepath{clip}%
\pgfsetroundcap%
\pgfsetroundjoin%
\pgfsetlinewidth{0.803000pt}%
\definecolor{currentstroke}{rgb}{1.000000,1.000000,1.000000}%
\pgfsetstrokecolor{currentstroke}%
\pgfsetdash{}{0pt}%
\pgfpathmoveto{\pgfqpoint{0.568357in}{2.169735in}}%
\pgfpathlineto{\pgfqpoint{3.151690in}{2.169735in}}%
\pgfusepath{stroke}%
\end{pgfscope}%
\begin{pgfscope}%
\definecolor{textcolor}{rgb}{0.150000,0.150000,0.150000}%
\pgfsetstrokecolor{textcolor}%
\pgfsetfillcolor{textcolor}%
\pgftext[x=0.100000in,y=2.095869in,left,base]{\color{textcolor}\rmfamily\fontsize{14.000000}{16.800000}\selectfont 100}%
\end{pgfscope}%
\begin{pgfscope}%
\pgfpathrectangle{\pgfqpoint{0.568357in}{1.711254in}}{\pgfqpoint{2.583333in}{0.736585in}}%
\pgfusepath{clip}%
\pgfsetroundcap%
\pgfsetroundjoin%
\pgfsetlinewidth{0.803000pt}%
\definecolor{currentstroke}{rgb}{1.000000,1.000000,1.000000}%
\pgfsetstrokecolor{currentstroke}%
\pgfsetdash{}{0pt}%
\pgfpathmoveto{\pgfqpoint{0.568357in}{2.433907in}}%
\pgfpathlineto{\pgfqpoint{3.151690in}{2.433907in}}%
\pgfusepath{stroke}%
\end{pgfscope}%
\begin{pgfscope}%
\definecolor{textcolor}{rgb}{0.150000,0.150000,0.150000}%
\pgfsetstrokecolor{textcolor}%
\pgfsetfillcolor{textcolor}%
\pgftext[x=0.100000in,y=2.360041in,left,base]{\color{textcolor}\rmfamily\fontsize{14.000000}{16.800000}\selectfont 125}%
\end{pgfscope}%
\begin{pgfscope}%
\pgfpathrectangle{\pgfqpoint{0.568357in}{1.711254in}}{\pgfqpoint{2.583333in}{0.736585in}}%
\pgfusepath{clip}%
\pgfsetroundcap%
\pgfsetroundjoin%
\pgfsetlinewidth{1.505625pt}%
\definecolor{currentstroke}{rgb}{0.890196,0.466667,0.760784}%
\pgfsetstrokecolor{currentstroke}%
\pgfsetdash{}{0pt}%
\pgfpathmoveto{\pgfqpoint{0.685781in}{1.770518in}}%
\pgfpathlineto{\pgfqpoint{0.686855in}{1.773900in}}%
\pgfpathlineto{\pgfqpoint{0.689002in}{1.763755in}}%
\pgfpathlineto{\pgfqpoint{0.692224in}{1.765552in}}%
\pgfpathlineto{\pgfqpoint{0.693298in}{1.782565in}}%
\pgfpathlineto{\pgfqpoint{0.695445in}{1.793237in}}%
\pgfpathlineto{\pgfqpoint{0.696519in}{1.782987in}}%
\pgfpathlineto{\pgfqpoint{0.700815in}{1.791441in}}%
\pgfpathlineto{\pgfqpoint{0.701888in}{1.796407in}}%
\pgfpathlineto{\pgfqpoint{0.704036in}{1.788376in}}%
\pgfpathlineto{\pgfqpoint{0.707258in}{1.789856in}}%
\pgfpathlineto{\pgfqpoint{0.708332in}{1.797992in}}%
\pgfpathlineto{\pgfqpoint{0.709405in}{1.796830in}}%
\pgfpathlineto{\pgfqpoint{0.710479in}{1.794716in}}%
\pgfpathlineto{\pgfqpoint{0.711553in}{1.796513in}}%
\pgfpathlineto{\pgfqpoint{0.714775in}{1.796407in}}%
\pgfpathlineto{\pgfqpoint{0.715848in}{1.802959in}}%
\pgfpathlineto{\pgfqpoint{0.716922in}{1.819443in}}%
\pgfpathlineto{\pgfqpoint{0.717996in}{1.817647in}}%
\pgfpathlineto{\pgfqpoint{0.719070in}{1.826734in}}%
\pgfpathlineto{\pgfqpoint{0.722291in}{1.822507in}}%
\pgfpathlineto{\pgfqpoint{0.723365in}{1.819971in}}%
\pgfpathlineto{\pgfqpoint{0.725513in}{1.850827in}}%
\pgfpathlineto{\pgfqpoint{0.726587in}{1.848290in}}%
\pgfpathlineto{\pgfqpoint{0.729808in}{1.860442in}}%
\pgfpathlineto{\pgfqpoint{0.730882in}{1.858329in}}%
\pgfpathlineto{\pgfqpoint{0.731956in}{1.844592in}}%
\pgfpathlineto{\pgfqpoint{0.734104in}{1.854948in}}%
\pgfpathlineto{\pgfqpoint{0.739473in}{1.856533in}}%
\pgfpathlineto{\pgfqpoint{0.740547in}{1.852412in}}%
\pgfpathlineto{\pgfqpoint{0.741621in}{1.856638in}}%
\pgfpathlineto{\pgfqpoint{0.745916in}{1.852200in}}%
\pgfpathlineto{\pgfqpoint{0.749137in}{1.861710in}}%
\pgfpathlineto{\pgfqpoint{0.752359in}{1.850615in}}%
\pgfpathlineto{\pgfqpoint{0.753433in}{1.833814in}}%
\pgfpathlineto{\pgfqpoint{0.755580in}{1.852834in}}%
\pgfpathlineto{\pgfqpoint{0.756654in}{1.852940in}}%
\pgfpathlineto{\pgfqpoint{0.759876in}{1.856850in}}%
\pgfpathlineto{\pgfqpoint{0.760950in}{1.879674in}}%
\pgfpathlineto{\pgfqpoint{0.762023in}{1.881787in}}%
\pgfpathlineto{\pgfqpoint{0.763097in}{1.882527in}}%
\pgfpathlineto{\pgfqpoint{0.764171in}{1.870058in}}%
\pgfpathlineto{\pgfqpoint{0.767393in}{1.862450in}}%
\pgfpathlineto{\pgfqpoint{0.768466in}{1.850404in}}%
\pgfpathlineto{\pgfqpoint{0.771688in}{1.837512in}}%
\pgfpathlineto{\pgfqpoint{0.774910in}{1.852517in}}%
\pgfpathlineto{\pgfqpoint{0.775983in}{1.849030in}}%
\pgfpathlineto{\pgfqpoint{0.777057in}{1.834976in}}%
\pgfpathlineto{\pgfqpoint{0.779205in}{1.847551in}}%
\pgfpathlineto{\pgfqpoint{0.782426in}{1.845649in}}%
\pgfpathlineto{\pgfqpoint{0.784574in}{1.838569in}}%
\pgfpathlineto{\pgfqpoint{0.789943in}{1.823564in}}%
\pgfpathlineto{\pgfqpoint{0.791017in}{1.808031in}}%
\pgfpathlineto{\pgfqpoint{0.793165in}{1.831912in}}%
\pgfpathlineto{\pgfqpoint{0.794239in}{1.819760in}}%
\pgfpathlineto{\pgfqpoint{0.797460in}{1.820500in}}%
\pgfpathlineto{\pgfqpoint{0.798534in}{1.832546in}}%
\pgfpathlineto{\pgfqpoint{0.799608in}{1.832123in}}%
\pgfpathlineto{\pgfqpoint{0.800682in}{1.825043in}}%
\pgfpathlineto{\pgfqpoint{0.801755in}{1.830327in}}%
\pgfpathlineto{\pgfqpoint{0.804977in}{1.819337in}}%
\pgfpathlineto{\pgfqpoint{0.806051in}{1.820183in}}%
\pgfpathlineto{\pgfqpoint{0.807125in}{1.819866in}}%
\pgfpathlineto{\pgfqpoint{0.809272in}{1.839203in}}%
\pgfpathlineto{\pgfqpoint{0.813568in}{1.834237in}}%
\pgfpathlineto{\pgfqpoint{0.814642in}{1.835293in}}%
\pgfpathlineto{\pgfqpoint{0.815715in}{1.829587in}}%
\pgfpathlineto{\pgfqpoint{0.816789in}{1.816907in}}%
\pgfpathlineto{\pgfqpoint{0.821085in}{1.811095in}}%
\pgfpathlineto{\pgfqpoint{0.822158in}{1.795033in}}%
\pgfpathlineto{\pgfqpoint{0.823232in}{1.797569in}}%
\pgfpathlineto{\pgfqpoint{0.824306in}{1.796513in}}%
\pgfpathlineto{\pgfqpoint{0.827528in}{1.785629in}}%
\pgfpathlineto{\pgfqpoint{0.828601in}{1.787637in}}%
\pgfpathlineto{\pgfqpoint{0.831823in}{1.758155in}}%
\pgfpathlineto{\pgfqpoint{0.836118in}{1.771998in}}%
\pgfpathlineto{\pgfqpoint{0.837192in}{1.773054in}}%
\pgfpathlineto{\pgfqpoint{0.839340in}{1.763755in}}%
\pgfpathlineto{\pgfqpoint{0.843635in}{1.782248in}}%
\pgfpathlineto{\pgfqpoint{0.844709in}{1.770201in}}%
\pgfpathlineto{\pgfqpoint{0.845783in}{1.773477in}}%
\pgfpathlineto{\pgfqpoint{0.846857in}{1.754879in}}%
\pgfpathlineto{\pgfqpoint{0.850078in}{1.749807in}}%
\pgfpathlineto{\pgfqpoint{0.851152in}{1.744735in}}%
\pgfpathlineto{\pgfqpoint{0.853300in}{1.784995in}}%
\pgfpathlineto{\pgfqpoint{0.854374in}{1.785946in}}%
\pgfpathlineto{\pgfqpoint{0.858669in}{1.775696in}}%
\pgfpathlineto{\pgfqpoint{0.859743in}{1.768405in}}%
\pgfpathlineto{\pgfqpoint{0.861890in}{1.776858in}}%
\pgfpathlineto{\pgfqpoint{0.865112in}{1.781402in}}%
\pgfpathlineto{\pgfqpoint{0.866186in}{1.793977in}}%
\pgfpathlineto{\pgfqpoint{0.867260in}{1.789539in}}%
\pgfpathlineto{\pgfqpoint{0.868333in}{1.779817in}}%
\pgfpathlineto{\pgfqpoint{0.869407in}{1.783199in}}%
\pgfpathlineto{\pgfqpoint{0.872629in}{1.770307in}}%
\pgfpathlineto{\pgfqpoint{0.873703in}{1.769250in}}%
\pgfpathlineto{\pgfqpoint{0.874776in}{1.774005in}}%
\pgfpathlineto{\pgfqpoint{0.875850in}{1.760057in}}%
\pgfpathlineto{\pgfqpoint{0.876924in}{1.786157in}}%
\pgfpathlineto{\pgfqpoint{0.880146in}{1.781825in}}%
\pgfpathlineto{\pgfqpoint{0.881220in}{1.788165in}}%
\pgfpathlineto{\pgfqpoint{0.883367in}{1.784889in}}%
\pgfpathlineto{\pgfqpoint{0.884441in}{1.773371in}}%
\pgfpathlineto{\pgfqpoint{0.887663in}{1.775485in}}%
\pgfpathlineto{\pgfqpoint{0.888736in}{1.773900in}}%
\pgfpathlineto{\pgfqpoint{0.889810in}{1.759212in}}%
\pgfpathlineto{\pgfqpoint{0.890884in}{1.753294in}}%
\pgfpathlineto{\pgfqpoint{0.891958in}{1.768933in}}%
\pgfpathlineto{\pgfqpoint{0.895179in}{1.765552in}}%
\pgfpathlineto{\pgfqpoint{0.896253in}{1.768933in}}%
\pgfpathlineto{\pgfqpoint{0.898401in}{1.788799in}}%
\pgfpathlineto{\pgfqpoint{0.899475in}{1.774639in}}%
\pgfpathlineto{\pgfqpoint{0.902696in}{1.766080in}}%
\pgfpathlineto{\pgfqpoint{0.903770in}{1.754140in}}%
\pgfpathlineto{\pgfqpoint{0.904844in}{1.760163in}}%
\pgfpathlineto{\pgfqpoint{0.905918in}{1.763016in}}%
\pgfpathlineto{\pgfqpoint{0.906992in}{1.775062in}}%
\pgfpathlineto{\pgfqpoint{0.910213in}{1.781191in}}%
\pgfpathlineto{\pgfqpoint{0.911287in}{1.776436in}}%
\pgfpathlineto{\pgfqpoint{0.912361in}{1.779606in}}%
\pgfpathlineto{\pgfqpoint{0.913435in}{1.776436in}}%
\pgfpathlineto{\pgfqpoint{0.914509in}{1.797569in}}%
\pgfpathlineto{\pgfqpoint{0.917730in}{1.796196in}}%
\pgfpathlineto{\pgfqpoint{0.918804in}{1.809299in}}%
\pgfpathlineto{\pgfqpoint{0.920952in}{1.799577in}}%
\pgfpathlineto{\pgfqpoint{0.922025in}{1.807185in}}%
\pgfpathlineto{\pgfqpoint{0.925247in}{1.803487in}}%
\pgfpathlineto{\pgfqpoint{0.926321in}{1.806234in}}%
\pgfpathlineto{\pgfqpoint{0.928468in}{1.819866in}}%
\pgfpathlineto{\pgfqpoint{0.929542in}{1.834237in}}%
\pgfpathlineto{\pgfqpoint{0.932764in}{1.831700in}}%
\pgfpathlineto{\pgfqpoint{0.933838in}{1.824409in}}%
\pgfpathlineto{\pgfqpoint{0.934911in}{1.827896in}}%
\pgfpathlineto{\pgfqpoint{0.935985in}{1.823775in}}%
\pgfpathlineto{\pgfqpoint{0.937059in}{1.831700in}}%
\pgfpathlineto{\pgfqpoint{0.941354in}{1.836667in}}%
\pgfpathlineto{\pgfqpoint{0.942428in}{1.832440in}}%
\pgfpathlineto{\pgfqpoint{0.943502in}{1.822507in}}%
\pgfpathlineto{\pgfqpoint{0.944576in}{1.829587in}}%
\pgfpathlineto{\pgfqpoint{0.949945in}{1.813314in}}%
\pgfpathlineto{\pgfqpoint{0.951019in}{1.825572in}}%
\pgfpathlineto{\pgfqpoint{0.952093in}{1.825677in}}%
\pgfpathlineto{\pgfqpoint{0.955314in}{1.818492in}}%
\pgfpathlineto{\pgfqpoint{0.957462in}{1.820500in}}%
\pgfpathlineto{\pgfqpoint{0.959610in}{1.852940in}}%
\pgfpathlineto{\pgfqpoint{0.962831in}{1.850298in}}%
\pgfpathlineto{\pgfqpoint{0.963905in}{1.844275in}}%
\pgfpathlineto{\pgfqpoint{0.964979in}{1.846388in}}%
\pgfpathlineto{\pgfqpoint{0.966053in}{1.839203in}}%
\pgfpathlineto{\pgfqpoint{0.967127in}{1.837724in}}%
\pgfpathlineto{\pgfqpoint{0.970348in}{1.831172in}}%
\pgfpathlineto{\pgfqpoint{0.971422in}{1.819866in}}%
\pgfpathlineto{\pgfqpoint{0.973570in}{1.816167in}}%
\pgfpathlineto{\pgfqpoint{0.974643in}{1.815639in}}%
\pgfpathlineto{\pgfqpoint{0.978939in}{1.816484in}}%
\pgfpathlineto{\pgfqpoint{0.980013in}{1.818069in}}%
\pgfpathlineto{\pgfqpoint{0.985382in}{1.816273in}}%
\pgfpathlineto{\pgfqpoint{0.987530in}{1.795245in}}%
\pgfpathlineto{\pgfqpoint{0.988603in}{1.795456in}}%
\pgfpathlineto{\pgfqpoint{0.989677in}{1.794716in}}%
\pgfpathlineto{\pgfqpoint{0.992899in}{1.796407in}}%
\pgfpathlineto{\pgfqpoint{0.995046in}{1.817858in}}%
\pgfpathlineto{\pgfqpoint{0.996120in}{1.824092in}}%
\pgfpathlineto{\pgfqpoint{0.997194in}{1.812891in}}%
\pgfpathlineto{\pgfqpoint{1.000416in}{1.811518in}}%
\pgfpathlineto{\pgfqpoint{1.001489in}{1.804649in}}%
\pgfpathlineto{\pgfqpoint{1.002563in}{1.812152in}}%
\pgfpathlineto{\pgfqpoint{1.003637in}{1.806657in}}%
\pgfpathlineto{\pgfqpoint{1.004711in}{1.814793in}}%
\pgfpathlineto{\pgfqpoint{1.010080in}{1.814476in}}%
\pgfpathlineto{\pgfqpoint{1.011154in}{1.822507in}}%
\pgfpathlineto{\pgfqpoint{1.012228in}{1.813631in}}%
\pgfpathlineto{\pgfqpoint{1.015449in}{1.812152in}}%
\pgfpathlineto{\pgfqpoint{1.016523in}{1.830644in}}%
\pgfpathlineto{\pgfqpoint{1.018671in}{1.796513in}}%
\pgfpathlineto{\pgfqpoint{1.019745in}{1.793660in}}%
\pgfpathlineto{\pgfqpoint{1.022966in}{1.803593in}}%
\pgfpathlineto{\pgfqpoint{1.024040in}{1.804015in}}%
\pgfpathlineto{\pgfqpoint{1.025114in}{1.787637in}}%
\pgfpathlineto{\pgfqpoint{1.026188in}{1.789327in}}%
\pgfpathlineto{\pgfqpoint{1.032631in}{1.811835in}}%
\pgfpathlineto{\pgfqpoint{1.034778in}{1.823458in}}%
\pgfpathlineto{\pgfqpoint{1.038000in}{1.824092in}}%
\pgfpathlineto{\pgfqpoint{1.039074in}{1.825360in}}%
\pgfpathlineto{\pgfqpoint{1.040148in}{1.834131in}}%
\pgfpathlineto{\pgfqpoint{1.041221in}{1.833814in}}%
\pgfpathlineto{\pgfqpoint{1.042295in}{1.836984in}}%
\pgfpathlineto{\pgfqpoint{1.045517in}{1.834131in}}%
\pgfpathlineto{\pgfqpoint{1.046591in}{1.837195in}}%
\pgfpathlineto{\pgfqpoint{1.047664in}{1.838041in}}%
\pgfpathlineto{\pgfqpoint{1.048738in}{1.843852in}}%
\pgfpathlineto{\pgfqpoint{1.049812in}{1.844909in}}%
\pgfpathlineto{\pgfqpoint{1.053034in}{1.845120in}}%
\pgfpathlineto{\pgfqpoint{1.054108in}{1.846705in}}%
\pgfpathlineto{\pgfqpoint{1.055181in}{1.845015in}}%
\pgfpathlineto{\pgfqpoint{1.057329in}{1.835822in}}%
\pgfpathlineto{\pgfqpoint{1.060551in}{1.835927in}}%
\pgfpathlineto{\pgfqpoint{1.061624in}{1.856110in}}%
\pgfpathlineto{\pgfqpoint{1.062698in}{1.863401in}}%
\pgfpathlineto{\pgfqpoint{1.063772in}{1.864669in}}%
\pgfpathlineto{\pgfqpoint{1.064846in}{1.858963in}}%
\pgfpathlineto{\pgfqpoint{1.070215in}{1.855053in}}%
\pgfpathlineto{\pgfqpoint{1.071289in}{1.854631in}}%
\pgfpathlineto{\pgfqpoint{1.072363in}{1.843324in}}%
\pgfpathlineto{\pgfqpoint{1.075584in}{1.854102in}}%
\pgfpathlineto{\pgfqpoint{1.077732in}{1.872172in}}%
\pgfpathlineto{\pgfqpoint{1.078806in}{1.874919in}}%
\pgfpathlineto{\pgfqpoint{1.079880in}{1.880942in}}%
\pgfpathlineto{\pgfqpoint{1.083101in}{1.877244in}}%
\pgfpathlineto{\pgfqpoint{1.084175in}{1.868051in}}%
\pgfpathlineto{\pgfqpoint{1.085249in}{1.877138in}}%
\pgfpathlineto{\pgfqpoint{1.087397in}{1.882739in}}%
\pgfpathlineto{\pgfqpoint{1.091692in}{1.889818in}}%
\pgfpathlineto{\pgfqpoint{1.092766in}{1.886331in}}%
\pgfpathlineto{\pgfqpoint{1.094913in}{1.898800in}}%
\pgfpathlineto{\pgfqpoint{1.099209in}{1.903450in}}%
\pgfpathlineto{\pgfqpoint{1.101356in}{1.916553in}}%
\pgfpathlineto{\pgfqpoint{1.102430in}{1.924478in}}%
\pgfpathlineto{\pgfqpoint{1.106726in}{1.924161in}}%
\pgfpathlineto{\pgfqpoint{1.107799in}{1.918243in}}%
\pgfpathlineto{\pgfqpoint{1.108873in}{1.904401in}}%
\pgfpathlineto{\pgfqpoint{1.109947in}{1.924900in}}%
\pgfpathlineto{\pgfqpoint{1.113169in}{1.922153in}}%
\pgfpathlineto{\pgfqpoint{1.114242in}{1.919617in}}%
\pgfpathlineto{\pgfqpoint{1.115316in}{1.920674in}}%
\pgfpathlineto{\pgfqpoint{1.116390in}{1.925851in}}%
\pgfpathlineto{\pgfqpoint{1.117464in}{1.927119in}}%
\pgfpathlineto{\pgfqpoint{1.120686in}{1.922787in}}%
\pgfpathlineto{\pgfqpoint{1.121759in}{1.926168in}}%
\pgfpathlineto{\pgfqpoint{1.122833in}{1.926697in}}%
\pgfpathlineto{\pgfqpoint{1.123907in}{1.928387in}}%
\pgfpathlineto{\pgfqpoint{1.124981in}{1.938320in}}%
\pgfpathlineto{\pgfqpoint{1.129276in}{1.940434in}}%
\pgfpathlineto{\pgfqpoint{1.131424in}{1.925851in}}%
\pgfpathlineto{\pgfqpoint{1.132498in}{1.935679in}}%
\pgfpathlineto{\pgfqpoint{1.135719in}{1.916341in}}%
\pgfpathlineto{\pgfqpoint{1.136793in}{1.923949in}}%
\pgfpathlineto{\pgfqpoint{1.137867in}{1.936418in}}%
\pgfpathlineto{\pgfqpoint{1.138941in}{1.936207in}}%
\pgfpathlineto{\pgfqpoint{1.140015in}{1.932403in}}%
\pgfpathlineto{\pgfqpoint{1.143236in}{1.923315in}}%
\pgfpathlineto{\pgfqpoint{1.144310in}{1.940434in}}%
\pgfpathlineto{\pgfqpoint{1.145384in}{1.940962in}}%
\pgfpathlineto{\pgfqpoint{1.147531in}{1.950050in}}%
\pgfpathlineto{\pgfqpoint{1.151827in}{1.959771in}}%
\pgfpathlineto{\pgfqpoint{1.152901in}{1.959243in}}%
\pgfpathlineto{\pgfqpoint{1.153975in}{1.962518in}}%
\pgfpathlineto{\pgfqpoint{1.158270in}{1.956390in}}%
\pgfpathlineto{\pgfqpoint{1.160418in}{1.962518in}}%
\pgfpathlineto{\pgfqpoint{1.161491in}{1.953325in}}%
\pgfpathlineto{\pgfqpoint{1.162565in}{1.963892in}}%
\pgfpathlineto{\pgfqpoint{1.166861in}{1.955333in}}%
\pgfpathlineto{\pgfqpoint{1.167934in}{1.954910in}}%
\pgfpathlineto{\pgfqpoint{1.169008in}{1.962413in}}%
\pgfpathlineto{\pgfqpoint{1.173304in}{1.957763in}}%
\pgfpathlineto{\pgfqpoint{1.174377in}{1.958397in}}%
\pgfpathlineto{\pgfqpoint{1.175451in}{1.960299in}}%
\pgfpathlineto{\pgfqpoint{1.176525in}{1.959982in}}%
\pgfpathlineto{\pgfqpoint{1.177599in}{1.955227in}}%
\pgfpathlineto{\pgfqpoint{1.180820in}{1.964949in}}%
\pgfpathlineto{\pgfqpoint{1.184042in}{1.984075in}}%
\pgfpathlineto{\pgfqpoint{1.185116in}{1.982913in}}%
\pgfpathlineto{\pgfqpoint{1.188337in}{1.963152in}}%
\pgfpathlineto{\pgfqpoint{1.189411in}{1.972240in}}%
\pgfpathlineto{\pgfqpoint{1.191559in}{1.945294in}}%
\pgfpathlineto{\pgfqpoint{1.192633in}{1.960511in}}%
\pgfpathlineto{\pgfqpoint{1.195854in}{1.964209in}}%
\pgfpathlineto{\pgfqpoint{1.198002in}{1.949416in}}%
\pgfpathlineto{\pgfqpoint{1.199076in}{1.950367in}}%
\pgfpathlineto{\pgfqpoint{1.200150in}{1.941702in}}%
\pgfpathlineto{\pgfqpoint{1.203371in}{1.945928in}}%
\pgfpathlineto{\pgfqpoint{1.205519in}{1.940751in}}%
\pgfpathlineto{\pgfqpoint{1.206593in}{1.947091in}}%
\pgfpathlineto{\pgfqpoint{1.207666in}{1.959454in}}%
\pgfpathlineto{\pgfqpoint{1.210888in}{1.962941in}}%
\pgfpathlineto{\pgfqpoint{1.214109in}{1.974459in}}%
\pgfpathlineto{\pgfqpoint{1.215183in}{1.978897in}}%
\pgfpathlineto{\pgfqpoint{1.218405in}{1.976255in}}%
\pgfpathlineto{\pgfqpoint{1.219479in}{1.985132in}}%
\pgfpathlineto{\pgfqpoint{1.220552in}{1.988936in}}%
\pgfpathlineto{\pgfqpoint{1.221626in}{1.983018in}}%
\pgfpathlineto{\pgfqpoint{1.222700in}{2.002990in}}%
\pgfpathlineto{\pgfqpoint{1.225922in}{2.001933in}}%
\pgfpathlineto{\pgfqpoint{1.226996in}{2.004786in}}%
\pgfpathlineto{\pgfqpoint{1.229143in}{1.985977in}}%
\pgfpathlineto{\pgfqpoint{1.230217in}{1.981856in}}%
\pgfpathlineto{\pgfqpoint{1.234512in}{1.990415in}}%
\pgfpathlineto{\pgfqpoint{1.235586in}{1.982596in}}%
\pgfpathlineto{\pgfqpoint{1.236660in}{1.990309in}}%
\pgfpathlineto{\pgfqpoint{1.237734in}{1.980588in}}%
\pgfpathlineto{\pgfqpoint{1.240955in}{1.983652in}}%
\pgfpathlineto{\pgfqpoint{1.244177in}{1.960933in}}%
\pgfpathlineto{\pgfqpoint{1.245251in}{1.976889in}}%
\pgfpathlineto{\pgfqpoint{1.248472in}{1.973614in}}%
\pgfpathlineto{\pgfqpoint{1.249546in}{1.969387in}}%
\pgfpathlineto{\pgfqpoint{1.250620in}{1.960299in}}%
\pgfpathlineto{\pgfqpoint{1.251694in}{1.974987in}}%
\pgfpathlineto{\pgfqpoint{1.252768in}{1.972557in}}%
\pgfpathlineto{\pgfqpoint{1.255989in}{1.981327in}}%
\pgfpathlineto{\pgfqpoint{1.257063in}{1.992211in}}%
\pgfpathlineto{\pgfqpoint{1.259211in}{1.957235in}}%
\pgfpathlineto{\pgfqpoint{1.260285in}{1.955756in}}%
\pgfpathlineto{\pgfqpoint{1.263506in}{1.949416in}}%
\pgfpathlineto{\pgfqpoint{1.264580in}{1.952057in}}%
\pgfpathlineto{\pgfqpoint{1.265654in}{1.963047in}}%
\pgfpathlineto{\pgfqpoint{1.266728in}{1.967908in}}%
\pgfpathlineto{\pgfqpoint{1.267801in}{1.962624in}}%
\pgfpathlineto{\pgfqpoint{1.271023in}{1.979108in}}%
\pgfpathlineto{\pgfqpoint{1.272097in}{1.970549in}}%
\pgfpathlineto{\pgfqpoint{1.275318in}{1.995487in}}%
\pgfpathlineto{\pgfqpoint{1.278540in}{1.999608in}}%
\pgfpathlineto{\pgfqpoint{1.279614in}{2.009118in}}%
\pgfpathlineto{\pgfqpoint{1.280687in}{2.007005in}}%
\pgfpathlineto{\pgfqpoint{1.281761in}{2.024335in}}%
\pgfpathlineto{\pgfqpoint{1.286057in}{2.029407in}}%
\pgfpathlineto{\pgfqpoint{1.287130in}{2.027082in}}%
\pgfpathlineto{\pgfqpoint{1.290352in}{2.049907in}}%
\pgfpathlineto{\pgfqpoint{1.293574in}{2.046525in}}%
\pgfpathlineto{\pgfqpoint{1.294647in}{2.073999in}}%
\pgfpathlineto{\pgfqpoint{1.296795in}{2.071040in}}%
\pgfpathlineto{\pgfqpoint{1.297869in}{2.072625in}}%
\pgfpathlineto{\pgfqpoint{1.301090in}{2.073576in}}%
\pgfpathlineto{\pgfqpoint{1.302164in}{2.078120in}}%
\pgfpathlineto{\pgfqpoint{1.303238in}{2.078120in}}%
\pgfpathlineto{\pgfqpoint{1.304312in}{2.092808in}}%
\pgfpathlineto{\pgfqpoint{1.305386in}{2.098197in}}%
\pgfpathlineto{\pgfqpoint{1.308607in}{2.087947in}}%
\pgfpathlineto{\pgfqpoint{1.309681in}{2.074739in}}%
\pgfpathlineto{\pgfqpoint{1.310755in}{2.082347in}}%
\pgfpathlineto{\pgfqpoint{1.311829in}{2.084460in}}%
\pgfpathlineto{\pgfqpoint{1.312903in}{2.078543in}}%
\pgfpathlineto{\pgfqpoint{1.316124in}{2.077803in}}%
\pgfpathlineto{\pgfqpoint{1.317198in}{2.089321in}}%
\pgfpathlineto{\pgfqpoint{1.318272in}{2.078331in}}%
\pgfpathlineto{\pgfqpoint{1.319346in}{2.059311in}}%
\pgfpathlineto{\pgfqpoint{1.320419in}{2.060156in}}%
\pgfpathlineto{\pgfqpoint{1.324715in}{2.053816in}}%
\pgfpathlineto{\pgfqpoint{1.325789in}{2.047476in}}%
\pgfpathlineto{\pgfqpoint{1.326863in}{2.058994in}}%
\pgfpathlineto{\pgfqpoint{1.331158in}{2.052971in}}%
\pgfpathlineto{\pgfqpoint{1.332232in}{2.031309in}}%
\pgfpathlineto{\pgfqpoint{1.333306in}{2.031626in}}%
\pgfpathlineto{\pgfqpoint{1.334379in}{2.036064in}}%
\pgfpathlineto{\pgfqpoint{1.335453in}{2.032683in}}%
\pgfpathlineto{\pgfqpoint{1.341896in}{2.065651in}}%
\pgfpathlineto{\pgfqpoint{1.342970in}{2.061530in}}%
\pgfpathlineto{\pgfqpoint{1.346192in}{2.072942in}}%
\pgfpathlineto{\pgfqpoint{1.348339in}{2.103798in}}%
\pgfpathlineto{\pgfqpoint{1.349413in}{2.103798in}}%
\pgfpathlineto{\pgfqpoint{1.350487in}{2.108870in}}%
\pgfpathlineto{\pgfqpoint{1.353708in}{2.120916in}}%
\pgfpathlineto{\pgfqpoint{1.356930in}{2.142050in}}%
\pgfpathlineto{\pgfqpoint{1.358004in}{2.119859in}}%
\pgfpathlineto{\pgfqpoint{1.361225in}{2.118380in}}%
\pgfpathlineto{\pgfqpoint{1.362299in}{2.123663in}}%
\pgfpathlineto{\pgfqpoint{1.363373in}{2.116901in}}%
\pgfpathlineto{\pgfqpoint{1.364447in}{2.120599in}}%
\pgfpathlineto{\pgfqpoint{1.365521in}{2.117852in}}%
\pgfpathlineto{\pgfqpoint{1.369816in}{2.099571in}}%
\pgfpathlineto{\pgfqpoint{1.371964in}{2.065757in}}%
\pgfpathlineto{\pgfqpoint{1.373038in}{2.071040in}}%
\pgfpathlineto{\pgfqpoint{1.376259in}{2.068821in}}%
\pgfpathlineto{\pgfqpoint{1.377333in}{2.057198in}}%
\pgfpathlineto{\pgfqpoint{1.378407in}{2.057937in}}%
\pgfpathlineto{\pgfqpoint{1.379481in}{2.086151in}}%
\pgfpathlineto{\pgfqpoint{1.380554in}{2.096189in}}%
\pgfpathlineto{\pgfqpoint{1.383776in}{2.095238in}}%
\pgfpathlineto{\pgfqpoint{1.384850in}{2.085094in}}%
\pgfpathlineto{\pgfqpoint{1.385924in}{2.090906in}}%
\pgfpathlineto{\pgfqpoint{1.386997in}{2.105488in}}%
\pgfpathlineto{\pgfqpoint{1.388071in}{2.102952in}}%
\pgfpathlineto{\pgfqpoint{1.391293in}{2.101790in}}%
\pgfpathlineto{\pgfqpoint{1.392367in}{2.088159in}}%
\pgfpathlineto{\pgfqpoint{1.393440in}{2.090483in}}%
\pgfpathlineto{\pgfqpoint{1.395588in}{2.100945in}}%
\pgfpathlineto{\pgfqpoint{1.398810in}{2.086891in}}%
\pgfpathlineto{\pgfqpoint{1.399884in}{2.090800in}}%
\pgfpathlineto{\pgfqpoint{1.400957in}{2.086468in}}%
\pgfpathlineto{\pgfqpoint{1.402031in}{2.089215in}}%
\pgfpathlineto{\pgfqpoint{1.403105in}{2.100839in}}%
\pgfpathlineto{\pgfqpoint{1.406327in}{2.105171in}}%
\pgfpathlineto{\pgfqpoint{1.407400in}{2.102424in}}%
\pgfpathlineto{\pgfqpoint{1.408474in}{2.111511in}}%
\pgfpathlineto{\pgfqpoint{1.409548in}{2.098620in}}%
\pgfpathlineto{\pgfqpoint{1.410622in}{2.110349in}}%
\pgfpathlineto{\pgfqpoint{1.413843in}{2.106017in}}%
\pgfpathlineto{\pgfqpoint{1.414917in}{2.100311in}}%
\pgfpathlineto{\pgfqpoint{1.415991in}{2.105805in}}%
\pgfpathlineto{\pgfqpoint{1.417065in}{2.117218in}}%
\pgfpathlineto{\pgfqpoint{1.418139in}{2.116267in}}%
\pgfpathlineto{\pgfqpoint{1.421360in}{2.121233in}}%
\pgfpathlineto{\pgfqpoint{1.422434in}{2.120916in}}%
\pgfpathlineto{\pgfqpoint{1.423508in}{2.118486in}}%
\pgfpathlineto{\pgfqpoint{1.424582in}{2.127150in}}%
\pgfpathlineto{\pgfqpoint{1.425656in}{2.131166in}}%
\pgfpathlineto{\pgfqpoint{1.428877in}{2.132117in}}%
\pgfpathlineto{\pgfqpoint{1.431025in}{2.143212in}}%
\pgfpathlineto{\pgfqpoint{1.433173in}{2.137189in}}%
\pgfpathlineto{\pgfqpoint{1.436394in}{2.132117in}}%
\pgfpathlineto{\pgfqpoint{1.438542in}{2.119437in}}%
\pgfpathlineto{\pgfqpoint{1.439616in}{2.120493in}}%
\pgfpathlineto{\pgfqpoint{1.440689in}{2.139619in}}%
\pgfpathlineto{\pgfqpoint{1.443911in}{2.140148in}}%
\pgfpathlineto{\pgfqpoint{1.444985in}{2.138351in}}%
\pgfpathlineto{\pgfqpoint{1.446059in}{2.116901in}}%
\pgfpathlineto{\pgfqpoint{1.448206in}{2.104749in}}%
\pgfpathlineto{\pgfqpoint{1.451428in}{2.115738in}}%
\pgfpathlineto{\pgfqpoint{1.452502in}{2.107285in}}%
\pgfpathlineto{\pgfqpoint{1.453575in}{2.127467in}}%
\pgfpathlineto{\pgfqpoint{1.454649in}{2.124614in}}%
\pgfpathlineto{\pgfqpoint{1.455723in}{2.135393in}}%
\pgfpathlineto{\pgfqpoint{1.458945in}{2.136872in}}%
\pgfpathlineto{\pgfqpoint{1.462166in}{2.154096in}}%
\pgfpathlineto{\pgfqpoint{1.463240in}{2.155153in}}%
\pgfpathlineto{\pgfqpoint{1.466462in}{2.154519in}}%
\pgfpathlineto{\pgfqpoint{1.467535in}{2.164346in}}%
\pgfpathlineto{\pgfqpoint{1.469683in}{2.152300in}}%
\pgfpathlineto{\pgfqpoint{1.470757in}{2.156104in}}%
\pgfpathlineto{\pgfqpoint{1.473978in}{2.155047in}}%
\pgfpathlineto{\pgfqpoint{1.475052in}{2.161704in}}%
\pgfpathlineto{\pgfqpoint{1.478274in}{2.164663in}}%
\pgfpathlineto{\pgfqpoint{1.482569in}{2.153568in}}%
\pgfpathlineto{\pgfqpoint{1.483643in}{2.166882in}}%
\pgfpathlineto{\pgfqpoint{1.484717in}{2.168256in}}%
\pgfpathlineto{\pgfqpoint{1.485791in}{2.168150in}}%
\pgfpathlineto{\pgfqpoint{1.490086in}{2.175335in}}%
\pgfpathlineto{\pgfqpoint{1.491160in}{2.185797in}}%
\pgfpathlineto{\pgfqpoint{1.492234in}{2.173962in}}%
\pgfpathlineto{\pgfqpoint{1.493307in}{2.145854in}}%
\pgfpathlineto{\pgfqpoint{1.496529in}{2.164240in}}%
\pgfpathlineto{\pgfqpoint{1.497603in}{2.164980in}}%
\pgfpathlineto{\pgfqpoint{1.498677in}{2.159485in}}%
\pgfpathlineto{\pgfqpoint{1.499751in}{2.172482in}}%
\pgfpathlineto{\pgfqpoint{1.500824in}{2.166459in}}%
\pgfpathlineto{\pgfqpoint{1.506194in}{2.109926in}}%
\pgfpathlineto{\pgfqpoint{1.508341in}{2.136027in}}%
\pgfpathlineto{\pgfqpoint{1.511563in}{2.144057in}}%
\pgfpathlineto{\pgfqpoint{1.512637in}{2.156632in}}%
\pgfpathlineto{\pgfqpoint{1.514784in}{2.165297in}}%
\pgfpathlineto{\pgfqpoint{1.515858in}{2.170580in}}%
\pgfpathlineto{\pgfqpoint{1.520153in}{2.169312in}}%
\pgfpathlineto{\pgfqpoint{1.521227in}{2.172165in}}%
\pgfpathlineto{\pgfqpoint{1.522301in}{2.181464in}}%
\pgfpathlineto{\pgfqpoint{1.526596in}{2.194884in}}%
\pgfpathlineto{\pgfqpoint{1.527670in}{2.189284in}}%
\pgfpathlineto{\pgfqpoint{1.528744in}{2.191397in}}%
\pgfpathlineto{\pgfqpoint{1.530892in}{2.199851in}}%
\pgfpathlineto{\pgfqpoint{1.534113in}{2.197420in}}%
\pgfpathlineto{\pgfqpoint{1.535187in}{2.205134in}}%
\pgfpathlineto{\pgfqpoint{1.536261in}{2.203443in}}%
\pgfpathlineto{\pgfqpoint{1.537335in}{2.206719in}}%
\pgfpathlineto{\pgfqpoint{1.538409in}{2.211791in}}%
\pgfpathlineto{\pgfqpoint{1.541630in}{2.206719in}}%
\pgfpathlineto{\pgfqpoint{1.542704in}{2.186959in}}%
\pgfpathlineto{\pgfqpoint{1.543778in}{2.188544in}}%
\pgfpathlineto{\pgfqpoint{1.544852in}{2.161493in}}%
\pgfpathlineto{\pgfqpoint{1.545926in}{2.158745in}}%
\pgfpathlineto{\pgfqpoint{1.549147in}{2.175547in}}%
\pgfpathlineto{\pgfqpoint{1.550221in}{2.178083in}}%
\pgfpathlineto{\pgfqpoint{1.551295in}{2.171848in}}%
\pgfpathlineto{\pgfqpoint{1.552369in}{2.169629in}}%
\pgfpathlineto{\pgfqpoint{1.553442in}{2.177026in}}%
\pgfpathlineto{\pgfqpoint{1.556664in}{2.169418in}}%
\pgfpathlineto{\pgfqpoint{1.557738in}{2.182944in}}%
\pgfpathlineto{\pgfqpoint{1.559885in}{2.169946in}}%
\pgfpathlineto{\pgfqpoint{1.560959in}{2.179245in}}%
\pgfpathlineto{\pgfqpoint{1.564181in}{2.198160in}}%
\pgfpathlineto{\pgfqpoint{1.565255in}{2.208198in}}%
\pgfpathlineto{\pgfqpoint{1.566328in}{2.226268in}}%
\pgfpathlineto{\pgfqpoint{1.567402in}{2.225317in}}%
\pgfpathlineto{\pgfqpoint{1.568476in}{2.210523in}}%
\pgfpathlineto{\pgfqpoint{1.572772in}{2.188227in}}%
\pgfpathlineto{\pgfqpoint{1.573845in}{2.199956in}}%
\pgfpathlineto{\pgfqpoint{1.574919in}{2.176920in}}%
\pgfpathlineto{\pgfqpoint{1.575993in}{2.171109in}}%
\pgfpathlineto{\pgfqpoint{1.579215in}{2.180513in}}%
\pgfpathlineto{\pgfqpoint{1.580288in}{2.188861in}}%
\pgfpathlineto{\pgfqpoint{1.581362in}{2.209572in}}%
\pgfpathlineto{\pgfqpoint{1.582436in}{2.214221in}}%
\pgfpathlineto{\pgfqpoint{1.586731in}{2.211685in}}%
\pgfpathlineto{\pgfqpoint{1.588879in}{2.224260in}}%
\pgfpathlineto{\pgfqpoint{1.589953in}{2.218026in}}%
\pgfpathlineto{\pgfqpoint{1.591027in}{2.201541in}}%
\pgfpathlineto{\pgfqpoint{1.594248in}{2.206085in}}%
\pgfpathlineto{\pgfqpoint{1.595322in}{2.204817in}}%
\pgfpathlineto{\pgfqpoint{1.596396in}{2.212002in}}%
\pgfpathlineto{\pgfqpoint{1.597470in}{2.197737in}}%
\pgfpathlineto{\pgfqpoint{1.598544in}{2.195201in}}%
\pgfpathlineto{\pgfqpoint{1.601765in}{2.197948in}}%
\pgfpathlineto{\pgfqpoint{1.602839in}{2.190446in}}%
\pgfpathlineto{\pgfqpoint{1.603913in}{2.198794in}}%
\pgfpathlineto{\pgfqpoint{1.604987in}{2.199851in}}%
\pgfpathlineto{\pgfqpoint{1.606061in}{2.199534in}}%
\pgfpathlineto{\pgfqpoint{1.609282in}{2.214750in}}%
\pgfpathlineto{\pgfqpoint{1.610356in}{2.215807in}}%
\pgfpathlineto{\pgfqpoint{1.611430in}{2.207987in}}%
\pgfpathlineto{\pgfqpoint{1.613577in}{2.182415in}}%
\pgfpathlineto{\pgfqpoint{1.616799in}{2.186536in}}%
\pgfpathlineto{\pgfqpoint{1.617873in}{2.168678in}}%
\pgfpathlineto{\pgfqpoint{1.618947in}{2.184846in}}%
\pgfpathlineto{\pgfqpoint{1.620020in}{2.186748in}}%
\pgfpathlineto{\pgfqpoint{1.621094in}{2.191397in}}%
\pgfpathlineto{\pgfqpoint{1.626463in}{2.195201in}}%
\pgfpathlineto{\pgfqpoint{1.627537in}{2.199005in}}%
\pgfpathlineto{\pgfqpoint{1.628611in}{2.197737in}}%
\pgfpathlineto{\pgfqpoint{1.632906in}{2.212742in}}%
\pgfpathlineto{\pgfqpoint{1.633980in}{2.206296in}}%
\pgfpathlineto{\pgfqpoint{1.636128in}{2.222781in}}%
\pgfpathlineto{\pgfqpoint{1.639350in}{2.233876in}}%
\pgfpathlineto{\pgfqpoint{1.642571in}{2.203655in}}%
\pgfpathlineto{\pgfqpoint{1.643645in}{2.203126in}}%
\pgfpathlineto{\pgfqpoint{1.646866in}{2.203866in}}%
\pgfpathlineto{\pgfqpoint{1.649014in}{2.207142in}}%
\pgfpathlineto{\pgfqpoint{1.651162in}{2.214327in}}%
\pgfpathlineto{\pgfqpoint{1.654383in}{2.206296in}}%
\pgfpathlineto{\pgfqpoint{1.655457in}{2.192771in}}%
\pgfpathlineto{\pgfqpoint{1.656531in}{2.196786in}}%
\pgfpathlineto{\pgfqpoint{1.657605in}{2.193405in}}%
\pgfpathlineto{\pgfqpoint{1.658679in}{2.201119in}}%
\pgfpathlineto{\pgfqpoint{1.661900in}{2.190552in}}%
\pgfpathlineto{\pgfqpoint{1.662974in}{2.195095in}}%
\pgfpathlineto{\pgfqpoint{1.664048in}{2.187699in}}%
\pgfpathlineto{\pgfqpoint{1.665122in}{2.191291in}}%
\pgfpathlineto{\pgfqpoint{1.669417in}{2.187276in}}%
\pgfpathlineto{\pgfqpoint{1.670491in}{2.177660in}}%
\pgfpathlineto{\pgfqpoint{1.672639in}{2.172482in}}%
\pgfpathlineto{\pgfqpoint{1.673712in}{2.178294in}}%
\pgfpathlineto{\pgfqpoint{1.676934in}{2.185268in}}%
\pgfpathlineto{\pgfqpoint{1.678008in}{2.184846in}}%
\pgfpathlineto{\pgfqpoint{1.679082in}{2.180407in}}%
\pgfpathlineto{\pgfqpoint{1.680155in}{2.165508in}}%
\pgfpathlineto{\pgfqpoint{1.681229in}{2.173011in}}%
\pgfpathlineto{\pgfqpoint{1.684451in}{2.167516in}}%
\pgfpathlineto{\pgfqpoint{1.686598in}{2.137189in}}%
\pgfpathlineto{\pgfqpoint{1.687672in}{2.129898in}}%
\pgfpathlineto{\pgfqpoint{1.688746in}{2.129264in}}%
\pgfpathlineto{\pgfqpoint{1.691968in}{2.130320in}}%
\pgfpathlineto{\pgfqpoint{1.694115in}{2.105911in}}%
\pgfpathlineto{\pgfqpoint{1.695189in}{2.094393in}}%
\pgfpathlineto{\pgfqpoint{1.696263in}{2.090695in}}%
\pgfpathlineto{\pgfqpoint{1.699484in}{2.093548in}}%
\pgfpathlineto{\pgfqpoint{1.700558in}{2.093231in}}%
\pgfpathlineto{\pgfqpoint{1.701632in}{2.081713in}}%
\pgfpathlineto{\pgfqpoint{1.702706in}{2.085728in}}%
\pgfpathlineto{\pgfqpoint{1.703780in}{2.102001in}}%
\pgfpathlineto{\pgfqpoint{1.707001in}{2.099782in}}%
\pgfpathlineto{\pgfqpoint{1.708075in}{2.092174in}}%
\pgfpathlineto{\pgfqpoint{1.709149in}{2.103798in}}%
\pgfpathlineto{\pgfqpoint{1.710223in}{2.105805in}}%
\pgfpathlineto{\pgfqpoint{1.711297in}{2.104537in}}%
\pgfpathlineto{\pgfqpoint{1.716666in}{2.142789in}}%
\pgfpathlineto{\pgfqpoint{1.717740in}{2.146065in}}%
\pgfpathlineto{\pgfqpoint{1.718814in}{2.139831in}}%
\pgfpathlineto{\pgfqpoint{1.722035in}{2.143846in}}%
\pgfpathlineto{\pgfqpoint{1.723109in}{2.142472in}}%
\pgfpathlineto{\pgfqpoint{1.724183in}{2.136978in}}%
\pgfpathlineto{\pgfqpoint{1.725257in}{2.137189in}}%
\pgfpathlineto{\pgfqpoint{1.726330in}{2.126516in}}%
\pgfpathlineto{\pgfqpoint{1.731700in}{2.138034in}}%
\pgfpathlineto{\pgfqpoint{1.732773in}{2.138140in}}%
\pgfpathlineto{\pgfqpoint{1.733847in}{2.133279in}}%
\pgfpathlineto{\pgfqpoint{1.737069in}{2.132328in}}%
\pgfpathlineto{\pgfqpoint{1.738143in}{2.133068in}}%
\pgfpathlineto{\pgfqpoint{1.739216in}{2.131483in}}%
\pgfpathlineto{\pgfqpoint{1.740290in}{2.131483in}}%
\pgfpathlineto{\pgfqpoint{1.741364in}{2.130003in}}%
\pgfpathlineto{\pgfqpoint{1.744586in}{2.129792in}}%
\pgfpathlineto{\pgfqpoint{1.745660in}{2.131800in}}%
\pgfpathlineto{\pgfqpoint{1.746733in}{2.127045in}}%
\pgfpathlineto{\pgfqpoint{1.747807in}{2.131588in}}%
\pgfpathlineto{\pgfqpoint{1.748881in}{2.130954in}}%
\pgfpathlineto{\pgfqpoint{1.752103in}{2.112357in}}%
\pgfpathlineto{\pgfqpoint{1.753176in}{2.102530in}}%
\pgfpathlineto{\pgfqpoint{1.754250in}{2.108658in}}%
\pgfpathlineto{\pgfqpoint{1.755324in}{2.093759in}}%
\pgfpathlineto{\pgfqpoint{1.756398in}{2.100839in}}%
\pgfpathlineto{\pgfqpoint{1.759619in}{2.099360in}}%
\pgfpathlineto{\pgfqpoint{1.760693in}{2.104220in}}%
\pgfpathlineto{\pgfqpoint{1.761767in}{2.087736in}}%
\pgfpathlineto{\pgfqpoint{1.762841in}{2.082030in}}%
\pgfpathlineto{\pgfqpoint{1.763915in}{2.093125in}}%
\pgfpathlineto{\pgfqpoint{1.767136in}{2.091223in}}%
\pgfpathlineto{\pgfqpoint{1.768210in}{2.065017in}}%
\pgfpathlineto{\pgfqpoint{1.769284in}{2.076324in}}%
\pgfpathlineto{\pgfqpoint{1.770358in}{2.051069in}}%
\pgfpathlineto{\pgfqpoint{1.771432in}{2.051069in}}%
\pgfpathlineto{\pgfqpoint{1.774653in}{2.045151in}}%
\pgfpathlineto{\pgfqpoint{1.775727in}{2.052760in}}%
\pgfpathlineto{\pgfqpoint{1.776801in}{2.043883in}}%
\pgfpathlineto{\pgfqpoint{1.777875in}{2.044412in}}%
\pgfpathlineto{\pgfqpoint{1.778949in}{2.065968in}}%
\pgfpathlineto{\pgfqpoint{1.782170in}{2.065546in}}%
\pgfpathlineto{\pgfqpoint{1.783244in}{2.070195in}}%
\pgfpathlineto{\pgfqpoint{1.784318in}{2.062692in}}%
\pgfpathlineto{\pgfqpoint{1.785392in}{2.081501in}}%
\pgfpathlineto{\pgfqpoint{1.786465in}{2.087525in}}%
\pgfpathlineto{\pgfqpoint{1.789687in}{2.091012in}}%
\pgfpathlineto{\pgfqpoint{1.790761in}{2.110455in}}%
\pgfpathlineto{\pgfqpoint{1.791835in}{2.106545in}}%
\pgfpathlineto{\pgfqpoint{1.793982in}{2.117323in}}%
\pgfpathlineto{\pgfqpoint{1.797204in}{2.110772in}}%
\pgfpathlineto{\pgfqpoint{1.798278in}{2.116267in}}%
\pgfpathlineto{\pgfqpoint{1.800425in}{2.132223in}}%
\pgfpathlineto{\pgfqpoint{1.801499in}{2.136872in}}%
\pgfpathlineto{\pgfqpoint{1.804721in}{2.136132in}}%
\pgfpathlineto{\pgfqpoint{1.805794in}{2.129158in}}%
\pgfpathlineto{\pgfqpoint{1.806868in}{2.133596in}}%
\pgfpathlineto{\pgfqpoint{1.807942in}{2.133596in}}%
\pgfpathlineto{\pgfqpoint{1.809016in}{2.127150in}}%
\pgfpathlineto{\pgfqpoint{1.812238in}{2.126305in}}%
\pgfpathlineto{\pgfqpoint{1.813311in}{2.139831in}}%
\pgfpathlineto{\pgfqpoint{1.814385in}{2.138563in}}%
\pgfpathlineto{\pgfqpoint{1.815459in}{2.140042in}}%
\pgfpathlineto{\pgfqpoint{1.816533in}{2.153990in}}%
\pgfpathlineto{\pgfqpoint{1.819754in}{2.139725in}}%
\pgfpathlineto{\pgfqpoint{1.820828in}{2.167305in}}%
\pgfpathlineto{\pgfqpoint{1.821902in}{2.152722in}}%
\pgfpathlineto{\pgfqpoint{1.824050in}{2.151877in}}%
\pgfpathlineto{\pgfqpoint{1.827271in}{2.148284in}}%
\pgfpathlineto{\pgfqpoint{1.828345in}{2.148178in}}%
\pgfpathlineto{\pgfqpoint{1.829419in}{2.160542in}}%
\pgfpathlineto{\pgfqpoint{1.831567in}{2.163395in}}%
\pgfpathlineto{\pgfqpoint{1.834788in}{2.179985in}}%
\pgfpathlineto{\pgfqpoint{1.835862in}{2.198265in}}%
\pgfpathlineto{\pgfqpoint{1.836936in}{2.184212in}}%
\pgfpathlineto{\pgfqpoint{1.838010in}{2.189284in}}%
\pgfpathlineto{\pgfqpoint{1.839083in}{2.171426in}}%
\pgfpathlineto{\pgfqpoint{1.842305in}{2.169207in}}%
\pgfpathlineto{\pgfqpoint{1.844453in}{2.188016in}}%
\pgfpathlineto{\pgfqpoint{1.845527in}{2.216863in}}%
\pgfpathlineto{\pgfqpoint{1.846600in}{2.203972in}}%
\pgfpathlineto{\pgfqpoint{1.849822in}{2.218554in}}%
\pgfpathlineto{\pgfqpoint{1.850896in}{2.219188in}}%
\pgfpathlineto{\pgfqpoint{1.851970in}{2.216335in}}%
\pgfpathlineto{\pgfqpoint{1.854117in}{2.219505in}}%
\pgfpathlineto{\pgfqpoint{1.857339in}{2.215912in}}%
\pgfpathlineto{\pgfqpoint{1.858413in}{2.209678in}}%
\pgfpathlineto{\pgfqpoint{1.859486in}{2.198371in}}%
\pgfpathlineto{\pgfqpoint{1.861634in}{2.198688in}}%
\pgfpathlineto{\pgfqpoint{1.864856in}{2.180619in}}%
\pgfpathlineto{\pgfqpoint{1.865929in}{2.165508in}}%
\pgfpathlineto{\pgfqpoint{1.867003in}{2.176920in}}%
\pgfpathlineto{\pgfqpoint{1.868077in}{2.195095in}}%
\pgfpathlineto{\pgfqpoint{1.869151in}{2.189072in}}%
\pgfpathlineto{\pgfqpoint{1.872372in}{2.193193in}}%
\pgfpathlineto{\pgfqpoint{1.873446in}{2.192137in}}%
\pgfpathlineto{\pgfqpoint{1.874520in}{2.184423in}}%
\pgfpathlineto{\pgfqpoint{1.875594in}{2.184423in}}%
\pgfpathlineto{\pgfqpoint{1.876668in}{2.209255in}}%
\pgfpathlineto{\pgfqpoint{1.880963in}{2.222252in}}%
\pgfpathlineto{\pgfqpoint{1.883111in}{2.249832in}}%
\pgfpathlineto{\pgfqpoint{1.887406in}{2.233770in}}%
\pgfpathlineto{\pgfqpoint{1.888480in}{2.237574in}}%
\pgfpathlineto{\pgfqpoint{1.889554in}{2.216441in}}%
\pgfpathlineto{\pgfqpoint{1.890628in}{2.212002in}}%
\pgfpathlineto{\pgfqpoint{1.891702in}{2.196258in}}%
\pgfpathlineto{\pgfqpoint{1.894923in}{2.213270in}}%
\pgfpathlineto{\pgfqpoint{1.895997in}{2.235038in}}%
\pgfpathlineto{\pgfqpoint{1.897071in}{2.224683in}}%
\pgfpathlineto{\pgfqpoint{1.898145in}{2.246556in}}%
\pgfpathlineto{\pgfqpoint{1.899218in}{2.243703in}}%
\pgfpathlineto{\pgfqpoint{1.902440in}{2.239159in}}%
\pgfpathlineto{\pgfqpoint{1.903514in}{2.239899in}}%
\pgfpathlineto{\pgfqpoint{1.904588in}{2.239265in}}%
\pgfpathlineto{\pgfqpoint{1.905661in}{2.247718in}}%
\pgfpathlineto{\pgfqpoint{1.906735in}{2.263463in}}%
\pgfpathlineto{\pgfqpoint{1.911031in}{2.264203in}}%
\pgfpathlineto{\pgfqpoint{1.913178in}{2.279208in}}%
\pgfpathlineto{\pgfqpoint{1.914252in}{2.290620in}}%
\pgfpathlineto{\pgfqpoint{1.917474in}{2.286922in}}%
\pgfpathlineto{\pgfqpoint{1.918548in}{2.287661in}}%
\pgfpathlineto{\pgfqpoint{1.919621in}{2.283646in}}%
\pgfpathlineto{\pgfqpoint{1.921769in}{2.269803in}}%
\pgfpathlineto{\pgfqpoint{1.924991in}{2.281955in}}%
\pgfpathlineto{\pgfqpoint{1.926064in}{2.267796in}}%
\pgfpathlineto{\pgfqpoint{1.927138in}{2.261561in}}%
\pgfpathlineto{\pgfqpoint{1.928212in}{2.259131in}}%
\pgfpathlineto{\pgfqpoint{1.929286in}{2.246239in}}%
\pgfpathlineto{\pgfqpoint{1.932507in}{2.268218in}}%
\pgfpathlineto{\pgfqpoint{1.933581in}{2.227007in}}%
\pgfpathlineto{\pgfqpoint{1.934655in}{2.235884in}}%
\pgfpathlineto{\pgfqpoint{1.935729in}{2.263463in}}%
\pgfpathlineto{\pgfqpoint{1.936803in}{2.239688in}}%
\pgfpathlineto{\pgfqpoint{1.940024in}{2.252262in}}%
\pgfpathlineto{\pgfqpoint{1.941098in}{2.250360in}}%
\pgfpathlineto{\pgfqpoint{1.942172in}{2.254587in}}%
\pgfpathlineto{\pgfqpoint{1.943246in}{2.245816in}}%
\pgfpathlineto{\pgfqpoint{1.944320in}{2.246556in}}%
\pgfpathlineto{\pgfqpoint{1.947541in}{2.239159in}}%
\pgfpathlineto{\pgfqpoint{1.948615in}{2.241378in}}%
\pgfpathlineto{\pgfqpoint{1.949689in}{2.218026in}}%
\pgfpathlineto{\pgfqpoint{1.950763in}{2.214116in}}%
\pgfpathlineto{\pgfqpoint{1.951837in}{2.222252in}}%
\pgfpathlineto{\pgfqpoint{1.955058in}{2.240533in}}%
\pgfpathlineto{\pgfqpoint{1.957206in}{2.212953in}}%
\pgfpathlineto{\pgfqpoint{1.958280in}{2.224366in}}%
\pgfpathlineto{\pgfqpoint{1.962575in}{2.231340in}}%
\pgfpathlineto{\pgfqpoint{1.963649in}{2.227853in}}%
\pgfpathlineto{\pgfqpoint{1.964723in}{2.231128in}}%
\pgfpathlineto{\pgfqpoint{1.965796in}{2.231551in}}%
\pgfpathlineto{\pgfqpoint{1.966870in}{2.237046in}}%
\pgfpathlineto{\pgfqpoint{1.970092in}{2.227113in}}%
\pgfpathlineto{\pgfqpoint{1.972239in}{2.231445in}}%
\pgfpathlineto{\pgfqpoint{1.973313in}{2.227747in}}%
\pgfpathlineto{\pgfqpoint{1.974387in}{2.205240in}}%
\pgfpathlineto{\pgfqpoint{1.978682in}{2.222675in}}%
\pgfpathlineto{\pgfqpoint{1.979756in}{2.222781in}}%
\pgfpathlineto{\pgfqpoint{1.980830in}{2.225634in}}%
\pgfpathlineto{\pgfqpoint{1.981904in}{2.215172in}}%
\pgfpathlineto{\pgfqpoint{1.985126in}{2.211263in}}%
\pgfpathlineto{\pgfqpoint{1.986199in}{2.214433in}}%
\pgfpathlineto{\pgfqpoint{1.987273in}{2.207776in}}%
\pgfpathlineto{\pgfqpoint{1.988347in}{2.192348in}}%
\pgfpathlineto{\pgfqpoint{1.989421in}{2.208410in}}%
\pgfpathlineto{\pgfqpoint{1.992642in}{2.217920in}}%
\pgfpathlineto{\pgfqpoint{1.993716in}{2.204923in}}%
\pgfpathlineto{\pgfqpoint{1.994790in}{2.204923in}}%
\pgfpathlineto{\pgfqpoint{1.995864in}{2.214116in}}%
\pgfpathlineto{\pgfqpoint{1.996938in}{2.236729in}}%
\pgfpathlineto{\pgfqpoint{2.001233in}{2.226585in}}%
\pgfpathlineto{\pgfqpoint{2.002307in}{2.232819in}}%
\pgfpathlineto{\pgfqpoint{2.003381in}{2.249726in}}%
\pgfpathlineto{\pgfqpoint{2.004455in}{2.243492in}}%
\pgfpathlineto{\pgfqpoint{2.007676in}{2.243703in}}%
\pgfpathlineto{\pgfqpoint{2.008750in}{2.248987in}}%
\pgfpathlineto{\pgfqpoint{2.009824in}{2.247084in}}%
\pgfpathlineto{\pgfqpoint{2.010898in}{2.249515in}}%
\pgfpathlineto{\pgfqpoint{2.016267in}{2.228064in}}%
\pgfpathlineto{\pgfqpoint{2.017341in}{2.235355in}}%
\pgfpathlineto{\pgfqpoint{2.018415in}{2.235884in}}%
\pgfpathlineto{\pgfqpoint{2.019488in}{2.230917in}}%
\pgfpathlineto{\pgfqpoint{2.022710in}{2.229226in}}%
\pgfpathlineto{\pgfqpoint{2.023784in}{2.232502in}}%
\pgfpathlineto{\pgfqpoint{2.024858in}{2.243703in}}%
\pgfpathlineto{\pgfqpoint{2.025931in}{2.230600in}}%
\pgfpathlineto{\pgfqpoint{2.027005in}{2.229226in}}%
\pgfpathlineto{\pgfqpoint{2.030227in}{2.221407in}}%
\pgfpathlineto{\pgfqpoint{2.031301in}{2.222569in}}%
\pgfpathlineto{\pgfqpoint{2.033448in}{2.241907in}}%
\pgfpathlineto{\pgfqpoint{2.034522in}{2.235038in}}%
\pgfpathlineto{\pgfqpoint{2.037744in}{2.206508in}}%
\pgfpathlineto{\pgfqpoint{2.038817in}{2.209572in}}%
\pgfpathlineto{\pgfqpoint{2.039891in}{2.210840in}}%
\pgfpathlineto{\pgfqpoint{2.040965in}{2.218765in}}%
\pgfpathlineto{\pgfqpoint{2.042039in}{2.209149in}}%
\pgfpathlineto{\pgfqpoint{2.045260in}{2.212848in}}%
\pgfpathlineto{\pgfqpoint{2.046334in}{2.212425in}}%
\pgfpathlineto{\pgfqpoint{2.048482in}{2.193933in}}%
\pgfpathlineto{\pgfqpoint{2.049556in}{2.196046in}}%
\pgfpathlineto{\pgfqpoint{2.053851in}{2.171320in}}%
\pgfpathlineto{\pgfqpoint{2.054925in}{2.170263in}}%
\pgfpathlineto{\pgfqpoint{2.055999in}{2.158323in}}%
\pgfpathlineto{\pgfqpoint{2.060294in}{2.156421in}}%
\pgfpathlineto{\pgfqpoint{2.061368in}{2.163395in}}%
\pgfpathlineto{\pgfqpoint{2.062442in}{2.149764in}}%
\pgfpathlineto{\pgfqpoint{2.063516in}{2.152194in}}%
\pgfpathlineto{\pgfqpoint{2.064590in}{2.164240in}}%
\pgfpathlineto{\pgfqpoint{2.067811in}{2.177554in}}%
\pgfpathlineto{\pgfqpoint{2.068885in}{2.176920in}}%
\pgfpathlineto{\pgfqpoint{2.069959in}{2.174384in}}%
\pgfpathlineto{\pgfqpoint{2.071033in}{2.174490in}}%
\pgfpathlineto{\pgfqpoint{2.072106in}{2.169524in}}%
\pgfpathlineto{\pgfqpoint{2.075328in}{2.167093in}}%
\pgfpathlineto{\pgfqpoint{2.076402in}{2.092914in}}%
\pgfpathlineto{\pgfqpoint{2.077476in}{2.081924in}}%
\pgfpathlineto{\pgfqpoint{2.078549in}{2.077909in}}%
\pgfpathlineto{\pgfqpoint{2.079623in}{2.060473in}}%
\pgfpathlineto{\pgfqpoint{2.082845in}{2.056352in}}%
\pgfpathlineto{\pgfqpoint{2.083919in}{2.057303in}}%
\pgfpathlineto{\pgfqpoint{2.084993in}{2.061002in}}%
\pgfpathlineto{\pgfqpoint{2.086066in}{2.073999in}}%
\pgfpathlineto{\pgfqpoint{2.087140in}{2.070089in}}%
\pgfpathlineto{\pgfqpoint{2.092509in}{2.056352in}}%
\pgfpathlineto{\pgfqpoint{2.093583in}{2.057303in}}%
\pgfpathlineto{\pgfqpoint{2.094657in}{2.050752in}}%
\pgfpathlineto{\pgfqpoint{2.097879in}{2.062798in}}%
\pgfpathlineto{\pgfqpoint{2.098952in}{2.053816in}}%
\pgfpathlineto{\pgfqpoint{2.100026in}{2.060685in}}%
\pgfpathlineto{\pgfqpoint{2.101100in}{2.057726in}}%
\pgfpathlineto{\pgfqpoint{2.102174in}{2.060790in}}%
\pgfpathlineto{\pgfqpoint{2.105395in}{2.067131in}}%
\pgfpathlineto{\pgfqpoint{2.106469in}{2.067659in}}%
\pgfpathlineto{\pgfqpoint{2.107543in}{2.056141in}}%
\pgfpathlineto{\pgfqpoint{2.109691in}{2.006160in}}%
\pgfpathlineto{\pgfqpoint{2.112912in}{1.985660in}}%
\pgfpathlineto{\pgfqpoint{2.113986in}{1.965794in}}%
\pgfpathlineto{\pgfqpoint{2.116134in}{2.008696in}}%
\pgfpathlineto{\pgfqpoint{2.117208in}{2.008379in}}%
\pgfpathlineto{\pgfqpoint{2.120429in}{1.992740in}}%
\pgfpathlineto{\pgfqpoint{2.121503in}{1.974565in}}%
\pgfpathlineto{\pgfqpoint{2.122577in}{1.989147in}}%
\pgfpathlineto{\pgfqpoint{2.123651in}{1.995593in}}%
\pgfpathlineto{\pgfqpoint{2.124725in}{1.983864in}}%
\pgfpathlineto{\pgfqpoint{2.129020in}{2.004363in}}%
\pgfpathlineto{\pgfqpoint{2.130094in}{1.995064in}}%
\pgfpathlineto{\pgfqpoint{2.131168in}{1.990732in}}%
\pgfpathlineto{\pgfqpoint{2.132241in}{1.999714in}}%
\pgfpathlineto{\pgfqpoint{2.135463in}{1.994853in}}%
\pgfpathlineto{\pgfqpoint{2.137611in}{2.015141in}}%
\pgfpathlineto{\pgfqpoint{2.138684in}{2.009118in}}%
\pgfpathlineto{\pgfqpoint{2.139758in}{1.987562in}}%
\pgfpathlineto{\pgfqpoint{2.142980in}{1.992211in}}%
\pgfpathlineto{\pgfqpoint{2.144054in}{1.959982in}}%
\pgfpathlineto{\pgfqpoint{2.145127in}{1.948147in}}%
\pgfpathlineto{\pgfqpoint{2.146201in}{1.946774in}}%
\pgfpathlineto{\pgfqpoint{2.147275in}{1.951318in}}%
\pgfpathlineto{\pgfqpoint{2.150497in}{1.946879in}}%
\pgfpathlineto{\pgfqpoint{2.152644in}{1.967591in}}%
\pgfpathlineto{\pgfqpoint{2.153718in}{1.961567in}}%
\pgfpathlineto{\pgfqpoint{2.154792in}{1.975093in}}%
\pgfpathlineto{\pgfqpoint{2.158014in}{1.998974in}}%
\pgfpathlineto{\pgfqpoint{2.159087in}{2.001827in}}%
\pgfpathlineto{\pgfqpoint{2.162309in}{2.028878in}}%
\pgfpathlineto{\pgfqpoint{2.165530in}{2.029407in}}%
\pgfpathlineto{\pgfqpoint{2.166604in}{2.018417in}}%
\pgfpathlineto{\pgfqpoint{2.167678in}{1.998129in}}%
\pgfpathlineto{\pgfqpoint{2.168752in}{2.007850in}}%
\pgfpathlineto{\pgfqpoint{2.169826in}{2.006054in}}%
\pgfpathlineto{\pgfqpoint{2.173047in}{1.996966in}}%
\pgfpathlineto{\pgfqpoint{2.175195in}{2.053816in}}%
\pgfpathlineto{\pgfqpoint{2.176269in}{2.071252in}}%
\pgfpathlineto{\pgfqpoint{2.177343in}{2.079282in}}%
\pgfpathlineto{\pgfqpoint{2.180564in}{2.075478in}}%
\pgfpathlineto{\pgfqpoint{2.181638in}{2.062587in}}%
\pgfpathlineto{\pgfqpoint{2.182712in}{2.066814in}}%
\pgfpathlineto{\pgfqpoint{2.183786in}{2.064172in}}%
\pgfpathlineto{\pgfqpoint{2.184859in}{2.058043in}}%
\pgfpathlineto{\pgfqpoint{2.188081in}{2.067448in}}%
\pgfpathlineto{\pgfqpoint{2.190229in}{2.076535in}}%
\pgfpathlineto{\pgfqpoint{2.191303in}{2.080973in}}%
\pgfpathlineto{\pgfqpoint{2.192376in}{2.080973in}}%
\pgfpathlineto{\pgfqpoint{2.197746in}{2.062692in}}%
\pgfpathlineto{\pgfqpoint{2.198819in}{2.071991in}}%
\pgfpathlineto{\pgfqpoint{2.199893in}{2.045468in}}%
\pgfpathlineto{\pgfqpoint{2.203115in}{2.058571in}}%
\pgfpathlineto{\pgfqpoint{2.204189in}{2.055824in}}%
\pgfpathlineto{\pgfqpoint{2.205262in}{2.057409in}}%
\pgfpathlineto{\pgfqpoint{2.206336in}{2.063115in}}%
\pgfpathlineto{\pgfqpoint{2.207410in}{2.062270in}}%
\pgfpathlineto{\pgfqpoint{2.210632in}{2.061107in}}%
\pgfpathlineto{\pgfqpoint{2.211705in}{2.054028in}}%
\pgfpathlineto{\pgfqpoint{2.212779in}{2.053182in}}%
\pgfpathlineto{\pgfqpoint{2.218148in}{2.041347in}}%
\pgfpathlineto{\pgfqpoint{2.219222in}{2.047582in}}%
\pgfpathlineto{\pgfqpoint{2.220296in}{2.033634in}}%
\pgfpathlineto{\pgfqpoint{2.221370in}{2.028350in}}%
\pgfpathlineto{\pgfqpoint{2.222444in}{2.038283in}}%
\pgfpathlineto{\pgfqpoint{2.225665in}{2.039974in}}%
\pgfpathlineto{\pgfqpoint{2.226739in}{2.023806in}}%
\pgfpathlineto{\pgfqpoint{2.227813in}{2.023067in}}%
\pgfpathlineto{\pgfqpoint{2.228887in}{2.020319in}}%
\pgfpathlineto{\pgfqpoint{2.229961in}{2.014719in}}%
\pgfpathlineto{\pgfqpoint{2.233182in}{2.012183in}}%
\pgfpathlineto{\pgfqpoint{2.234256in}{2.014402in}}%
\pgfpathlineto{\pgfqpoint{2.235330in}{2.032471in}}%
\pgfpathlineto{\pgfqpoint{2.237478in}{2.005526in}}%
\pgfpathlineto{\pgfqpoint{2.240699in}{2.017783in}}%
\pgfpathlineto{\pgfqpoint{2.242847in}{2.043778in}}%
\pgfpathlineto{\pgfqpoint{2.243921in}{2.043778in}}%
\pgfpathlineto{\pgfqpoint{2.248216in}{2.041770in}}%
\pgfpathlineto{\pgfqpoint{2.249290in}{2.053182in}}%
\pgfpathlineto{\pgfqpoint{2.250364in}{2.049590in}}%
\pgfpathlineto{\pgfqpoint{2.251437in}{2.041559in}}%
\pgfpathlineto{\pgfqpoint{2.255733in}{2.036698in}}%
\pgfpathlineto{\pgfqpoint{2.256807in}{2.038177in}}%
\pgfpathlineto{\pgfqpoint{2.257881in}{2.013028in}}%
\pgfpathlineto{\pgfqpoint{2.260028in}{1.986717in}}%
\pgfpathlineto{\pgfqpoint{2.264324in}{1.987668in}}%
\pgfpathlineto{\pgfqpoint{2.265397in}{1.971606in}}%
\pgfpathlineto{\pgfqpoint{2.266471in}{1.973297in}}%
\pgfpathlineto{\pgfqpoint{2.267545in}{1.940645in}}%
\pgfpathlineto{\pgfqpoint{2.271840in}{1.936841in}}%
\pgfpathlineto{\pgfqpoint{2.272914in}{1.932931in}}%
\pgfpathlineto{\pgfqpoint{2.275062in}{1.947619in}}%
\pgfpathlineto{\pgfqpoint{2.278283in}{1.933565in}}%
\pgfpathlineto{\pgfqpoint{2.279357in}{1.940856in}}%
\pgfpathlineto{\pgfqpoint{2.280431in}{1.942336in}}%
\pgfpathlineto{\pgfqpoint{2.281505in}{1.948570in}}%
\pgfpathlineto{\pgfqpoint{2.282579in}{1.960511in}}%
\pgfpathlineto{\pgfqpoint{2.285800in}{1.959243in}}%
\pgfpathlineto{\pgfqpoint{2.286874in}{1.938637in}}%
\pgfpathlineto{\pgfqpoint{2.287948in}{1.943815in}}%
\pgfpathlineto{\pgfqpoint{2.289022in}{1.964737in}}%
\pgfpathlineto{\pgfqpoint{2.290096in}{1.961990in}}%
\pgfpathlineto{\pgfqpoint{2.293317in}{1.951740in}}%
\pgfpathlineto{\pgfqpoint{2.294391in}{1.956178in}}%
\pgfpathlineto{\pgfqpoint{2.295465in}{1.953537in}}%
\pgfpathlineto{\pgfqpoint{2.296539in}{1.931240in}}%
\pgfpathlineto{\pgfqpoint{2.297613in}{1.943709in}}%
\pgfpathlineto{\pgfqpoint{2.301908in}{1.948887in}}%
\pgfpathlineto{\pgfqpoint{2.302982in}{1.971078in}}%
\pgfpathlineto{\pgfqpoint{2.304056in}{1.973402in}}%
\pgfpathlineto{\pgfqpoint{2.305129in}{1.972134in}}%
\pgfpathlineto{\pgfqpoint{2.308351in}{2.012394in}}%
\pgfpathlineto{\pgfqpoint{2.309425in}{2.004892in}}%
\pgfpathlineto{\pgfqpoint{2.310499in}{2.024546in}}%
\pgfpathlineto{\pgfqpoint{2.311572in}{2.067976in}}%
\pgfpathlineto{\pgfqpoint{2.315868in}{2.053816in}}%
\pgfpathlineto{\pgfqpoint{2.316942in}{2.038494in}}%
\pgfpathlineto{\pgfqpoint{2.319089in}{2.048850in}}%
\pgfpathlineto{\pgfqpoint{2.320163in}{2.057515in}}%
\pgfpathlineto{\pgfqpoint{2.324458in}{2.056564in}}%
\pgfpathlineto{\pgfqpoint{2.326606in}{2.049484in}}%
\pgfpathlineto{\pgfqpoint{2.327680in}{2.055084in}}%
\pgfpathlineto{\pgfqpoint{2.330902in}{2.055824in}}%
\pgfpathlineto{\pgfqpoint{2.331975in}{2.050646in}}%
\pgfpathlineto{\pgfqpoint{2.334123in}{2.075478in}}%
\pgfpathlineto{\pgfqpoint{2.335197in}{2.077486in}}%
\pgfpathlineto{\pgfqpoint{2.338418in}{2.078648in}}%
\pgfpathlineto{\pgfqpoint{2.339492in}{2.073682in}}%
\pgfpathlineto{\pgfqpoint{2.340566in}{2.078014in}}%
\pgfpathlineto{\pgfqpoint{2.345935in}{2.076112in}}%
\pgfpathlineto{\pgfqpoint{2.347009in}{2.087947in}}%
\pgfpathlineto{\pgfqpoint{2.348083in}{2.089215in}}%
\pgfpathlineto{\pgfqpoint{2.350231in}{2.086468in}}%
\pgfpathlineto{\pgfqpoint{2.353452in}{2.089321in}}%
\pgfpathlineto{\pgfqpoint{2.354526in}{2.085411in}}%
\pgfpathlineto{\pgfqpoint{2.357747in}{2.099571in}}%
\pgfpathlineto{\pgfqpoint{2.360969in}{2.107602in}}%
\pgfpathlineto{\pgfqpoint{2.362043in}{2.116689in}}%
\pgfpathlineto{\pgfqpoint{2.363117in}{2.131588in}}%
\pgfpathlineto{\pgfqpoint{2.364191in}{2.132328in}}%
\pgfpathlineto{\pgfqpoint{2.365264in}{2.131271in}}%
\pgfpathlineto{\pgfqpoint{2.368486in}{2.136661in}}%
\pgfpathlineto{\pgfqpoint{2.369560in}{2.135393in}}%
\pgfpathlineto{\pgfqpoint{2.370634in}{2.140570in}}%
\pgfpathlineto{\pgfqpoint{2.371707in}{2.139619in}}%
\pgfpathlineto{\pgfqpoint{2.372781in}{2.142261in}}%
\pgfpathlineto{\pgfqpoint{2.376003in}{2.137400in}}%
\pgfpathlineto{\pgfqpoint{2.377077in}{2.133491in}}%
\pgfpathlineto{\pgfqpoint{2.378150in}{2.144057in}}%
\pgfpathlineto{\pgfqpoint{2.379224in}{2.127890in}}%
\pgfpathlineto{\pgfqpoint{2.380298in}{2.129264in}}%
\pgfpathlineto{\pgfqpoint{2.383520in}{2.129264in}}%
\pgfpathlineto{\pgfqpoint{2.385667in}{2.093231in}}%
\pgfpathlineto{\pgfqpoint{2.386741in}{2.089532in}}%
\pgfpathlineto{\pgfqpoint{2.387815in}{2.097669in}}%
\pgfpathlineto{\pgfqpoint{2.391036in}{2.087630in}}%
\pgfpathlineto{\pgfqpoint{2.392110in}{2.107813in}}%
\pgfpathlineto{\pgfqpoint{2.393184in}{2.102318in}}%
\pgfpathlineto{\pgfqpoint{2.394258in}{2.100945in}}%
\pgfpathlineto{\pgfqpoint{2.395332in}{2.089321in}}%
\pgfpathlineto{\pgfqpoint{2.398553in}{2.104432in}}%
\pgfpathlineto{\pgfqpoint{2.399627in}{2.087313in}}%
\pgfpathlineto{\pgfqpoint{2.400701in}{2.086257in}}%
\pgfpathlineto{\pgfqpoint{2.401775in}{2.078543in}}%
\pgfpathlineto{\pgfqpoint{2.402849in}{2.084355in}}%
\pgfpathlineto{\pgfqpoint{2.406070in}{2.082453in}}%
\pgfpathlineto{\pgfqpoint{2.407144in}{2.092280in}}%
\pgfpathlineto{\pgfqpoint{2.408218in}{2.096295in}}%
\pgfpathlineto{\pgfqpoint{2.409292in}{2.097246in}}%
\pgfpathlineto{\pgfqpoint{2.410366in}{2.100628in}}%
\pgfpathlineto{\pgfqpoint{2.414661in}{2.098831in}}%
\pgfpathlineto{\pgfqpoint{2.415735in}{2.096612in}}%
\pgfpathlineto{\pgfqpoint{2.416809in}{2.100945in}}%
\pgfpathlineto{\pgfqpoint{2.417882in}{2.097457in}}%
\pgfpathlineto{\pgfqpoint{2.421104in}{2.106545in}}%
\pgfpathlineto{\pgfqpoint{2.422178in}{2.107390in}}%
\pgfpathlineto{\pgfqpoint{2.423252in}{2.113519in}}%
\pgfpathlineto{\pgfqpoint{2.424325in}{2.115950in}}%
\pgfpathlineto{\pgfqpoint{2.425399in}{2.112779in}}%
\pgfpathlineto{\pgfqpoint{2.428621in}{2.105066in}}%
\pgfpathlineto{\pgfqpoint{2.429695in}{2.104854in}}%
\pgfpathlineto{\pgfqpoint{2.430769in}{2.098514in}}%
\pgfpathlineto{\pgfqpoint{2.431842in}{2.104220in}}%
\pgfpathlineto{\pgfqpoint{2.432916in}{2.104960in}}%
\pgfpathlineto{\pgfqpoint{2.436138in}{2.109821in}}%
\pgfpathlineto{\pgfqpoint{2.438285in}{2.106439in}}%
\pgfpathlineto{\pgfqpoint{2.439359in}{2.116055in}}%
\pgfpathlineto{\pgfqpoint{2.440433in}{2.082347in}}%
\pgfpathlineto{\pgfqpoint{2.443655in}{2.065863in}}%
\pgfpathlineto{\pgfqpoint{2.446876in}{2.118169in}}%
\pgfpathlineto{\pgfqpoint{2.447950in}{2.119965in}}%
\pgfpathlineto{\pgfqpoint{2.452245in}{2.099254in}}%
\pgfpathlineto{\pgfqpoint{2.454393in}{2.112462in}}%
\pgfpathlineto{\pgfqpoint{2.455467in}{2.129052in}}%
\pgfpathlineto{\pgfqpoint{2.458688in}{2.132011in}}%
\pgfpathlineto{\pgfqpoint{2.460836in}{2.143106in}}%
\pgfpathlineto{\pgfqpoint{2.461910in}{2.143529in}}%
\pgfpathlineto{\pgfqpoint{2.462984in}{2.147122in}}%
\pgfpathlineto{\pgfqpoint{2.466205in}{2.147016in}}%
\pgfpathlineto{\pgfqpoint{2.467279in}{2.148284in}}%
\pgfpathlineto{\pgfqpoint{2.468353in}{2.152405in}}%
\pgfpathlineto{\pgfqpoint{2.469427in}{2.150503in}}%
\pgfpathlineto{\pgfqpoint{2.470501in}{2.143423in}}%
\pgfpathlineto{\pgfqpoint{2.473722in}{2.138774in}}%
\pgfpathlineto{\pgfqpoint{2.474796in}{2.170475in}}%
\pgfpathlineto{\pgfqpoint{2.476944in}{2.167516in}}%
\pgfpathlineto{\pgfqpoint{2.478017in}{2.168150in}}%
\pgfpathlineto{\pgfqpoint{2.481239in}{2.161176in}}%
\pgfpathlineto{\pgfqpoint{2.482313in}{2.154624in}}%
\pgfpathlineto{\pgfqpoint{2.484460in}{2.155575in}}%
\pgfpathlineto{\pgfqpoint{2.485534in}{2.169101in}}%
\pgfpathlineto{\pgfqpoint{2.488756in}{2.169524in}}%
\pgfpathlineto{\pgfqpoint{2.489830in}{2.174913in}}%
\pgfpathlineto{\pgfqpoint{2.490903in}{2.172799in}}%
\pgfpathlineto{\pgfqpoint{2.491977in}{2.183155in}}%
\pgfpathlineto{\pgfqpoint{2.493051in}{2.180090in}}%
\pgfpathlineto{\pgfqpoint{2.496273in}{2.188121in}}%
\pgfpathlineto{\pgfqpoint{2.497346in}{2.182944in}}%
\pgfpathlineto{\pgfqpoint{2.499494in}{2.191186in}}%
\pgfpathlineto{\pgfqpoint{2.503790in}{2.182944in}}%
\pgfpathlineto{\pgfqpoint{2.504863in}{2.178083in}}%
\pgfpathlineto{\pgfqpoint{2.505937in}{2.177660in}}%
\pgfpathlineto{\pgfqpoint{2.507011in}{2.175335in}}%
\pgfpathlineto{\pgfqpoint{2.508085in}{2.171214in}}%
\pgfpathlineto{\pgfqpoint{2.511306in}{2.177766in}}%
\pgfpathlineto{\pgfqpoint{2.513454in}{2.162549in}}%
\pgfpathlineto{\pgfqpoint{2.515602in}{2.167516in}}%
\pgfpathlineto{\pgfqpoint{2.520971in}{2.156843in}}%
\pgfpathlineto{\pgfqpoint{2.522045in}{2.155892in}}%
\pgfpathlineto{\pgfqpoint{2.523119in}{2.125565in}}%
\pgfpathlineto{\pgfqpoint{2.526340in}{2.138880in}}%
\pgfpathlineto{\pgfqpoint{2.527414in}{2.122078in}}%
\pgfpathlineto{\pgfqpoint{2.528488in}{2.115527in}}%
\pgfpathlineto{\pgfqpoint{2.529562in}{2.125882in}}%
\pgfpathlineto{\pgfqpoint{2.530635in}{2.100099in}}%
\pgfpathlineto{\pgfqpoint{2.533857in}{2.103481in}}%
\pgfpathlineto{\pgfqpoint{2.534931in}{2.101473in}}%
\pgfpathlineto{\pgfqpoint{2.537079in}{2.128735in}}%
\pgfpathlineto{\pgfqpoint{2.538152in}{2.124403in}}%
\pgfpathlineto{\pgfqpoint{2.541374in}{2.121127in}}%
\pgfpathlineto{\pgfqpoint{2.542448in}{2.122290in}}%
\pgfpathlineto{\pgfqpoint{2.543522in}{2.122290in}}%
\pgfpathlineto{\pgfqpoint{2.544595in}{2.109609in}}%
\pgfpathlineto{\pgfqpoint{2.545669in}{2.114893in}}%
\pgfpathlineto{\pgfqpoint{2.548891in}{2.123135in}}%
\pgfpathlineto{\pgfqpoint{2.549965in}{2.113308in}}%
\pgfpathlineto{\pgfqpoint{2.551038in}{2.121339in}}%
\pgfpathlineto{\pgfqpoint{2.552112in}{2.119648in}}%
\pgfpathlineto{\pgfqpoint{2.553186in}{2.104854in}}%
\pgfpathlineto{\pgfqpoint{2.556408in}{2.098937in}}%
\pgfpathlineto{\pgfqpoint{2.557481in}{2.086996in}}%
\pgfpathlineto{\pgfqpoint{2.558555in}{2.088581in}}%
\pgfpathlineto{\pgfqpoint{2.559629in}{2.097669in}}%
\pgfpathlineto{\pgfqpoint{2.560703in}{2.100733in}}%
\pgfpathlineto{\pgfqpoint{2.563924in}{2.096295in}}%
\pgfpathlineto{\pgfqpoint{2.564998in}{2.099043in}}%
\pgfpathlineto{\pgfqpoint{2.566072in}{2.096823in}}%
\pgfpathlineto{\pgfqpoint{2.568220in}{2.086045in}}%
\pgfpathlineto{\pgfqpoint{2.571441in}{2.094393in}}%
\pgfpathlineto{\pgfqpoint{2.572515in}{2.112568in}}%
\pgfpathlineto{\pgfqpoint{2.573589in}{2.109081in}}%
\pgfpathlineto{\pgfqpoint{2.574663in}{2.099888in}}%
\pgfpathlineto{\pgfqpoint{2.575737in}{2.117323in}}%
\pgfpathlineto{\pgfqpoint{2.578958in}{2.120810in}}%
\pgfpathlineto{\pgfqpoint{2.580032in}{2.119014in}}%
\pgfpathlineto{\pgfqpoint{2.582180in}{2.109504in}}%
\pgfpathlineto{\pgfqpoint{2.583254in}{2.112251in}}%
\pgfpathlineto{\pgfqpoint{2.587549in}{2.132434in}}%
\pgfpathlineto{\pgfqpoint{2.588623in}{2.146593in}}%
\pgfpathlineto{\pgfqpoint{2.589697in}{2.182098in}}%
\pgfpathlineto{\pgfqpoint{2.590770in}{2.186536in}}%
\pgfpathlineto{\pgfqpoint{2.595066in}{2.174913in}}%
\pgfpathlineto{\pgfqpoint{2.596140in}{2.174279in}}%
\pgfpathlineto{\pgfqpoint{2.598287in}{2.169735in}}%
\pgfpathlineto{\pgfqpoint{2.602583in}{2.174067in}}%
\pgfpathlineto{\pgfqpoint{2.603657in}{2.185691in}}%
\pgfpathlineto{\pgfqpoint{2.605804in}{2.192348in}}%
\pgfpathlineto{\pgfqpoint{2.609026in}{2.189072in}}%
\pgfpathlineto{\pgfqpoint{2.610100in}{2.192982in}}%
\pgfpathlineto{\pgfqpoint{2.611173in}{2.181887in}}%
\pgfpathlineto{\pgfqpoint{2.612247in}{2.179456in}}%
\pgfpathlineto{\pgfqpoint{2.613321in}{2.186748in}}%
\pgfpathlineto{\pgfqpoint{2.616543in}{2.178822in}}%
\pgfpathlineto{\pgfqpoint{2.617616in}{2.178928in}}%
\pgfpathlineto{\pgfqpoint{2.618690in}{2.201330in}}%
\pgfpathlineto{\pgfqpoint{2.619764in}{2.188861in}}%
\pgfpathlineto{\pgfqpoint{2.620838in}{2.202387in}}%
\pgfpathlineto{\pgfqpoint{2.624059in}{2.208304in}}%
\pgfpathlineto{\pgfqpoint{2.625133in}{2.207564in}}%
\pgfpathlineto{\pgfqpoint{2.627281in}{2.185902in}}%
\pgfpathlineto{\pgfqpoint{2.628355in}{2.189812in}}%
\pgfpathlineto{\pgfqpoint{2.631576in}{2.212636in}}%
\pgfpathlineto{\pgfqpoint{2.633724in}{2.209361in}}%
\pgfpathlineto{\pgfqpoint{2.634798in}{2.209044in}}%
\pgfpathlineto{\pgfqpoint{2.635872in}{2.210946in}}%
\pgfpathlineto{\pgfqpoint{2.640167in}{2.214221in}}%
\pgfpathlineto{\pgfqpoint{2.641241in}{2.206296in}}%
\pgfpathlineto{\pgfqpoint{2.642315in}{2.209889in}}%
\pgfpathlineto{\pgfqpoint{2.643389in}{2.200696in}}%
\pgfpathlineto{\pgfqpoint{2.647684in}{2.212742in}}%
\pgfpathlineto{\pgfqpoint{2.648758in}{2.213376in}}%
\pgfpathlineto{\pgfqpoint{2.649832in}{2.217814in}}%
\pgfpathlineto{\pgfqpoint{2.650905in}{2.229755in}}%
\pgfpathlineto{\pgfqpoint{2.655201in}{2.216969in}}%
\pgfpathlineto{\pgfqpoint{2.657348in}{2.212636in}}%
\pgfpathlineto{\pgfqpoint{2.658422in}{2.206613in}}%
\pgfpathlineto{\pgfqpoint{2.662718in}{2.204183in}}%
\pgfpathlineto{\pgfqpoint{2.664865in}{2.211685in}}%
\pgfpathlineto{\pgfqpoint{2.665939in}{2.212319in}}%
\pgfpathlineto{\pgfqpoint{2.669161in}{2.207881in}}%
\pgfpathlineto{\pgfqpoint{2.670234in}{2.220456in}}%
\pgfpathlineto{\pgfqpoint{2.673456in}{2.201436in}}%
\pgfpathlineto{\pgfqpoint{2.676678in}{2.196258in}}%
\pgfpathlineto{\pgfqpoint{2.677751in}{2.201224in}}%
\pgfpathlineto{\pgfqpoint{2.678825in}{2.186431in}}%
\pgfpathlineto{\pgfqpoint{2.679899in}{2.188227in}}%
\pgfpathlineto{\pgfqpoint{2.680973in}{2.201013in}}%
\pgfpathlineto{\pgfqpoint{2.684194in}{2.211157in}}%
\pgfpathlineto{\pgfqpoint{2.685268in}{2.216863in}}%
\pgfpathlineto{\pgfqpoint{2.686342in}{2.208410in}}%
\pgfpathlineto{\pgfqpoint{2.687416in}{2.205345in}}%
\pgfpathlineto{\pgfqpoint{2.688490in}{2.214855in}}%
\pgfpathlineto{\pgfqpoint{2.691711in}{2.224471in}}%
\pgfpathlineto{\pgfqpoint{2.692785in}{2.220033in}}%
\pgfpathlineto{\pgfqpoint{2.693859in}{2.229966in}}%
\pgfpathlineto{\pgfqpoint{2.696007in}{2.232397in}}%
\pgfpathlineto{\pgfqpoint{2.701376in}{2.237469in}}%
\pgfpathlineto{\pgfqpoint{2.702450in}{2.231762in}}%
\pgfpathlineto{\pgfqpoint{2.703523in}{2.235461in}}%
\pgfpathlineto{\pgfqpoint{2.706745in}{2.238948in}}%
\pgfpathlineto{\pgfqpoint{2.707819in}{2.236412in}}%
\pgfpathlineto{\pgfqpoint{2.708893in}{2.247718in}}%
\pgfpathlineto{\pgfqpoint{2.709967in}{2.237786in}}%
\pgfpathlineto{\pgfqpoint{2.711040in}{2.234299in}}%
\pgfpathlineto{\pgfqpoint{2.714262in}{2.227536in}}%
\pgfpathlineto{\pgfqpoint{2.715336in}{2.233665in}}%
\pgfpathlineto{\pgfqpoint{2.716410in}{2.228381in}}%
\pgfpathlineto{\pgfqpoint{2.718557in}{2.232291in}}%
\pgfpathlineto{\pgfqpoint{2.721779in}{2.233982in}}%
\pgfpathlineto{\pgfqpoint{2.722853in}{2.229226in}}%
\pgfpathlineto{\pgfqpoint{2.723926in}{2.241695in}}%
\pgfpathlineto{\pgfqpoint{2.725000in}{2.234299in}}%
\pgfpathlineto{\pgfqpoint{2.726074in}{2.245394in}}%
\pgfpathlineto{\pgfqpoint{2.729296in}{2.246028in}}%
\pgfpathlineto{\pgfqpoint{2.730369in}{2.232502in}}%
\pgfpathlineto{\pgfqpoint{2.731443in}{2.230177in}}%
\pgfpathlineto{\pgfqpoint{2.733591in}{2.228909in}}%
\pgfpathlineto{\pgfqpoint{2.736812in}{2.229226in}}%
\pgfpathlineto{\pgfqpoint{2.737886in}{2.238314in}}%
\pgfpathlineto{\pgfqpoint{2.738960in}{2.231551in}}%
\pgfpathlineto{\pgfqpoint{2.740034in}{2.235355in}}%
\pgfpathlineto{\pgfqpoint{2.741108in}{2.233031in}}%
\pgfpathlineto{\pgfqpoint{2.744329in}{2.230177in}}%
\pgfpathlineto{\pgfqpoint{2.745403in}{2.240744in}}%
\pgfpathlineto{\pgfqpoint{2.746477in}{2.235884in}}%
\pgfpathlineto{\pgfqpoint{2.747551in}{2.235038in}}%
\pgfpathlineto{\pgfqpoint{2.748625in}{2.240744in}}%
\pgfpathlineto{\pgfqpoint{2.751846in}{2.239688in}}%
\pgfpathlineto{\pgfqpoint{2.752920in}{2.242329in}}%
\pgfpathlineto{\pgfqpoint{2.753994in}{2.233348in}}%
\pgfpathlineto{\pgfqpoint{2.755068in}{2.231762in}}%
\pgfpathlineto{\pgfqpoint{2.760437in}{2.246450in}}%
\pgfpathlineto{\pgfqpoint{2.761511in}{2.240110in}}%
\pgfpathlineto{\pgfqpoint{2.763658in}{2.260716in}}%
\pgfpathlineto{\pgfqpoint{2.767954in}{2.279525in}}%
\pgfpathlineto{\pgfqpoint{2.769028in}{2.292839in}}%
\pgfpathlineto{\pgfqpoint{2.770101in}{2.298545in}}%
\pgfpathlineto{\pgfqpoint{2.771175in}{2.300659in}}%
\pgfpathlineto{\pgfqpoint{2.774397in}{2.300025in}}%
\pgfpathlineto{\pgfqpoint{2.775471in}{2.303934in}}%
\pgfpathlineto{\pgfqpoint{2.777618in}{2.320736in}}%
\pgfpathlineto{\pgfqpoint{2.778692in}{2.324223in}}%
\pgfpathlineto{\pgfqpoint{2.781914in}{2.321581in}}%
\pgfpathlineto{\pgfqpoint{2.782988in}{2.325385in}}%
\pgfpathlineto{\pgfqpoint{2.784061in}{2.320947in}}%
\pgfpathlineto{\pgfqpoint{2.785135in}{2.323589in}}%
\pgfpathlineto{\pgfqpoint{2.786209in}{2.319045in}}%
\pgfpathlineto{\pgfqpoint{2.789431in}{2.319890in}}%
\pgfpathlineto{\pgfqpoint{2.790504in}{2.324751in}}%
\pgfpathlineto{\pgfqpoint{2.791578in}{2.313127in}}%
\pgfpathlineto{\pgfqpoint{2.792652in}{2.311014in}}%
\pgfpathlineto{\pgfqpoint{2.793726in}{2.328978in}}%
\pgfpathlineto{\pgfqpoint{2.796947in}{2.333944in}}%
\pgfpathlineto{\pgfqpoint{2.798021in}{2.338065in}}%
\pgfpathlineto{\pgfqpoint{2.799095in}{2.338171in}}%
\pgfpathlineto{\pgfqpoint{2.800169in}{2.339862in}}%
\pgfpathlineto{\pgfqpoint{2.801243in}{2.335846in}}%
\pgfpathlineto{\pgfqpoint{2.805538in}{2.330034in}}%
\pgfpathlineto{\pgfqpoint{2.806612in}{2.330140in}}%
\pgfpathlineto{\pgfqpoint{2.808760in}{2.338594in}}%
\pgfpathlineto{\pgfqpoint{2.811981in}{2.327181in}}%
\pgfpathlineto{\pgfqpoint{2.813055in}{2.318517in}}%
\pgfpathlineto{\pgfqpoint{2.814129in}{2.315135in}}%
\pgfpathlineto{\pgfqpoint{2.815203in}{2.316720in}}%
\pgfpathlineto{\pgfqpoint{2.816277in}{2.323694in}}%
\pgfpathlineto{\pgfqpoint{2.819498in}{2.316086in}}%
\pgfpathlineto{\pgfqpoint{2.820572in}{2.317883in}}%
\pgfpathlineto{\pgfqpoint{2.821646in}{2.317460in}}%
\pgfpathlineto{\pgfqpoint{2.822720in}{2.324857in}}%
\pgfpathlineto{\pgfqpoint{2.823793in}{2.321581in}}%
\pgfpathlineto{\pgfqpoint{2.827015in}{2.335846in}}%
\pgfpathlineto{\pgfqpoint{2.828089in}{2.332465in}}%
\pgfpathlineto{\pgfqpoint{2.829163in}{2.334895in}}%
\pgfpathlineto{\pgfqpoint{2.830236in}{2.339122in}}%
\pgfpathlineto{\pgfqpoint{2.831310in}{2.339545in}}%
\pgfpathlineto{\pgfqpoint{2.834532in}{2.336058in}}%
\pgfpathlineto{\pgfqpoint{2.835606in}{2.333310in}}%
\pgfpathlineto{\pgfqpoint{2.836679in}{2.342398in}}%
\pgfpathlineto{\pgfqpoint{2.837753in}{2.333839in}}%
\pgfpathlineto{\pgfqpoint{2.838827in}{2.338488in}}%
\pgfpathlineto{\pgfqpoint{2.842049in}{2.338171in}}%
\pgfpathlineto{\pgfqpoint{2.844196in}{2.345356in}}%
\pgfpathlineto{\pgfqpoint{2.845270in}{2.336269in}}%
\pgfpathlineto{\pgfqpoint{2.846344in}{2.343454in}}%
\pgfpathlineto{\pgfqpoint{2.849566in}{2.347998in}}%
\pgfpathlineto{\pgfqpoint{2.850639in}{2.352859in}}%
\pgfpathlineto{\pgfqpoint{2.851713in}{2.354550in}}%
\pgfpathlineto{\pgfqpoint{2.852787in}{2.348104in}}%
\pgfpathlineto{\pgfqpoint{2.853861in}{2.351485in}}%
\pgfpathlineto{\pgfqpoint{2.857082in}{2.348526in}}%
\pgfpathlineto{\pgfqpoint{2.858156in}{2.343666in}}%
\pgfpathlineto{\pgfqpoint{2.859230in}{2.348209in}}%
\pgfpathlineto{\pgfqpoint{2.860304in}{2.341869in}}%
\pgfpathlineto{\pgfqpoint{2.861378in}{2.352331in}}%
\pgfpathlineto{\pgfqpoint{2.864599in}{2.348738in}}%
\pgfpathlineto{\pgfqpoint{2.865673in}{2.321475in}}%
\pgfpathlineto{\pgfqpoint{2.867821in}{2.304146in}}%
\pgfpathlineto{\pgfqpoint{2.868895in}{2.305942in}}%
\pgfpathlineto{\pgfqpoint{2.872116in}{2.302983in}}%
\pgfpathlineto{\pgfqpoint{2.873190in}{2.306048in}}%
\pgfpathlineto{\pgfqpoint{2.875338in}{2.328344in}}%
\pgfpathlineto{\pgfqpoint{2.876411in}{2.332254in}}%
\pgfpathlineto{\pgfqpoint{2.879633in}{2.302455in}}%
\pgfpathlineto{\pgfqpoint{2.880707in}{2.299179in}}%
\pgfpathlineto{\pgfqpoint{2.881781in}{2.289246in}}%
\pgfpathlineto{\pgfqpoint{2.882855in}{2.284808in}}%
\pgfpathlineto{\pgfqpoint{2.883928in}{2.286076in}}%
\pgfpathlineto{\pgfqpoint{2.887150in}{2.288718in}}%
\pgfpathlineto{\pgfqpoint{2.888224in}{2.276777in}}%
\pgfpathlineto{\pgfqpoint{2.889298in}{2.304463in}}%
\pgfpathlineto{\pgfqpoint{2.890371in}{2.285231in}}%
\pgfpathlineto{\pgfqpoint{2.891445in}{2.278996in}}%
\pgfpathlineto{\pgfqpoint{2.894667in}{2.276989in}}%
\pgfpathlineto{\pgfqpoint{2.895741in}{2.281110in}}%
\pgfpathlineto{\pgfqpoint{2.896814in}{2.294635in}}%
\pgfpathlineto{\pgfqpoint{2.897888in}{2.277200in}}%
\pgfpathlineto{\pgfqpoint{2.898962in}{2.274875in}}%
\pgfpathlineto{\pgfqpoint{2.902184in}{2.277517in}}%
\pgfpathlineto{\pgfqpoint{2.903257in}{2.311542in}}%
\pgfpathlineto{\pgfqpoint{2.904331in}{2.320630in}}%
\pgfpathlineto{\pgfqpoint{2.905405in}{2.321792in}}%
\pgfpathlineto{\pgfqpoint{2.911848in}{2.219716in}}%
\pgfpathlineto{\pgfqpoint{2.912922in}{2.223415in}}%
\pgfpathlineto{\pgfqpoint{2.913996in}{2.219082in}}%
\pgfpathlineto{\pgfqpoint{2.917217in}{2.220033in}}%
\pgfpathlineto{\pgfqpoint{2.918291in}{2.222252in}}%
\pgfpathlineto{\pgfqpoint{2.919365in}{2.226479in}}%
\pgfpathlineto{\pgfqpoint{2.920439in}{2.255327in}}%
\pgfpathlineto{\pgfqpoint{2.921513in}{2.254798in}}%
\pgfpathlineto{\pgfqpoint{2.924734in}{2.251628in}}%
\pgfpathlineto{\pgfqpoint{2.926882in}{2.265048in}}%
\pgfpathlineto{\pgfqpoint{2.929030in}{2.274241in}}%
\pgfpathlineto{\pgfqpoint{2.932251in}{2.266950in}}%
\pgfpathlineto{\pgfqpoint{2.933325in}{2.271283in}}%
\pgfpathlineto{\pgfqpoint{2.934399in}{2.296115in}}%
\pgfpathlineto{\pgfqpoint{2.935473in}{2.282589in}}%
\pgfpathlineto{\pgfqpoint{2.936546in}{2.285020in}}%
\pgfpathlineto{\pgfqpoint{2.939768in}{2.300236in}}%
\pgfpathlineto{\pgfqpoint{2.940842in}{2.301504in}}%
\pgfpathlineto{\pgfqpoint{2.941916in}{2.300659in}}%
\pgfpathlineto{\pgfqpoint{2.942989in}{2.306153in}}%
\pgfpathlineto{\pgfqpoint{2.944063in}{2.306787in}}%
\pgfpathlineto{\pgfqpoint{2.947285in}{2.310803in}}%
\pgfpathlineto{\pgfqpoint{2.949433in}{2.301927in}}%
\pgfpathlineto{\pgfqpoint{2.950506in}{2.312705in}}%
\pgfpathlineto{\pgfqpoint{2.951580in}{2.311648in}}%
\pgfpathlineto{\pgfqpoint{2.954802in}{2.314607in}}%
\pgfpathlineto{\pgfqpoint{2.955876in}{2.318200in}}%
\pgfpathlineto{\pgfqpoint{2.956949in}{2.316298in}}%
\pgfpathlineto{\pgfqpoint{2.958023in}{2.319468in}}%
\pgfpathlineto{\pgfqpoint{2.959097in}{2.334050in}}%
\pgfpathlineto{\pgfqpoint{2.962319in}{2.333627in}}%
\pgfpathlineto{\pgfqpoint{2.964466in}{2.314395in}}%
\pgfpathlineto{\pgfqpoint{2.965540in}{2.323906in}}%
\pgfpathlineto{\pgfqpoint{2.966614in}{2.315241in}}%
\pgfpathlineto{\pgfqpoint{2.969835in}{2.322955in}}%
\pgfpathlineto{\pgfqpoint{2.970909in}{2.322215in}}%
\pgfpathlineto{\pgfqpoint{2.971983in}{2.325808in}}%
\pgfpathlineto{\pgfqpoint{2.973057in}{2.339333in}}%
\pgfpathlineto{\pgfqpoint{2.974131in}{2.335424in}}%
\pgfpathlineto{\pgfqpoint{2.977352in}{2.328555in}}%
\pgfpathlineto{\pgfqpoint{2.978426in}{2.332148in}}%
\pgfpathlineto{\pgfqpoint{2.979500in}{2.327498in}}%
\pgfpathlineto{\pgfqpoint{2.980574in}{2.308161in}}%
\pgfpathlineto{\pgfqpoint{2.981648in}{2.305519in}}%
\pgfpathlineto{\pgfqpoint{2.984869in}{2.295058in}}%
\pgfpathlineto{\pgfqpoint{2.985943in}{2.312493in}}%
\pgfpathlineto{\pgfqpoint{2.987017in}{2.300130in}}%
\pgfpathlineto{\pgfqpoint{2.988091in}{2.310486in}}%
\pgfpathlineto{\pgfqpoint{2.989165in}{2.296643in}}%
\pgfpathlineto{\pgfqpoint{2.992386in}{2.295058in}}%
\pgfpathlineto{\pgfqpoint{2.993460in}{2.301821in}}%
\pgfpathlineto{\pgfqpoint{2.994534in}{2.298651in}}%
\pgfpathlineto{\pgfqpoint{2.999903in}{2.302666in}}%
\pgfpathlineto{\pgfqpoint{3.002051in}{2.313233in}}%
\pgfpathlineto{\pgfqpoint{3.003124in}{2.346624in}}%
\pgfpathlineto{\pgfqpoint{3.004198in}{2.333099in}}%
\pgfpathlineto{\pgfqpoint{3.007420in}{2.332254in}}%
\pgfpathlineto{\pgfqpoint{3.008494in}{2.334790in}}%
\pgfpathlineto{\pgfqpoint{3.011715in}{2.360467in}}%
\pgfpathlineto{\pgfqpoint{3.016011in}{2.367230in}}%
\pgfpathlineto{\pgfqpoint{3.017084in}{2.375578in}}%
\pgfpathlineto{\pgfqpoint{3.018158in}{2.370083in}}%
\pgfpathlineto{\pgfqpoint{3.019232in}{2.394598in}}%
\pgfpathlineto{\pgfqpoint{3.025675in}{2.406116in}}%
\pgfpathlineto{\pgfqpoint{3.026749in}{2.405271in}}%
\pgfpathlineto{\pgfqpoint{3.031044in}{2.404425in}}%
\pgfpathlineto{\pgfqpoint{3.033192in}{2.414358in}}%
\pgfpathlineto{\pgfqpoint{3.034266in}{2.408758in}}%
\pgfpathlineto{\pgfqpoint{3.034266in}{2.408758in}}%
\pgfusepath{stroke}%
\end{pgfscope}%
\begin{pgfscope}%
\pgfpathrectangle{\pgfqpoint{0.568357in}{1.711254in}}{\pgfqpoint{2.583333in}{0.736585in}}%
\pgfusepath{clip}%
\pgfsetroundcap%
\pgfsetroundjoin%
\pgfsetlinewidth{1.505625pt}%
\definecolor{currentstroke}{rgb}{1.000000,0.647059,0.000000}%
\pgfsetstrokecolor{currentstroke}%
\pgfsetdash{}{0pt}%
\pgfpathmoveto{\pgfqpoint{0.685781in}{1.770518in}}%
\pgfpathlineto{\pgfqpoint{0.686855in}{1.772209in}}%
\pgfpathlineto{\pgfqpoint{0.689002in}{1.768933in}}%
\pgfpathlineto{\pgfqpoint{0.692224in}{1.768257in}}%
\pgfpathlineto{\pgfqpoint{0.696519in}{1.776506in}}%
\pgfpathlineto{\pgfqpoint{0.700815in}{1.778000in}}%
\pgfpathlineto{\pgfqpoint{0.702962in}{1.780759in}}%
\pgfpathlineto{\pgfqpoint{0.704036in}{1.781345in}}%
\pgfpathlineto{\pgfqpoint{0.707258in}{1.781953in}}%
\pgfpathlineto{\pgfqpoint{0.710479in}{1.784522in}}%
\pgfpathlineto{\pgfqpoint{0.711553in}{1.785189in}}%
\pgfpathlineto{\pgfqpoint{0.714775in}{1.785779in}}%
\pgfpathlineto{\pgfqpoint{0.716922in}{1.788200in}}%
\pgfpathlineto{\pgfqpoint{0.719070in}{1.791156in}}%
\pgfpathlineto{\pgfqpoint{0.722291in}{1.792462in}}%
\pgfpathlineto{\pgfqpoint{0.724439in}{1.795074in}}%
\pgfpathlineto{\pgfqpoint{0.726587in}{1.798966in}}%
\pgfpathlineto{\pgfqpoint{0.729808in}{1.801086in}}%
\pgfpathlineto{\pgfqpoint{0.733030in}{1.805765in}}%
\pgfpathlineto{\pgfqpoint{0.734104in}{1.807256in}}%
\pgfpathlineto{\pgfqpoint{0.738399in}{1.808708in}}%
\pgfpathlineto{\pgfqpoint{0.741621in}{1.812477in}}%
\pgfpathlineto{\pgfqpoint{0.744842in}{1.813562in}}%
\pgfpathlineto{\pgfqpoint{0.749137in}{1.817699in}}%
\pgfpathlineto{\pgfqpoint{0.754507in}{1.819380in}}%
\pgfpathlineto{\pgfqpoint{0.756654in}{1.820805in}}%
\pgfpathlineto{\pgfqpoint{0.759876in}{1.821556in}}%
\pgfpathlineto{\pgfqpoint{0.764171in}{1.825937in}}%
\pgfpathlineto{\pgfqpoint{0.784574in}{1.829784in}}%
\pgfpathlineto{\pgfqpoint{0.789943in}{1.829780in}}%
\pgfpathlineto{\pgfqpoint{0.793165in}{1.829335in}}%
\pgfpathlineto{\pgfqpoint{0.798534in}{1.829127in}}%
\pgfpathlineto{\pgfqpoint{0.806051in}{1.828888in}}%
\pgfpathlineto{\pgfqpoint{0.809272in}{1.828949in}}%
\pgfpathlineto{\pgfqpoint{0.820011in}{1.828837in}}%
\pgfpathlineto{\pgfqpoint{0.828601in}{1.826691in}}%
\pgfpathlineto{\pgfqpoint{0.831823in}{1.824852in}}%
\pgfpathlineto{\pgfqpoint{0.836118in}{1.823752in}}%
\pgfpathlineto{\pgfqpoint{0.839340in}{1.822105in}}%
\pgfpathlineto{\pgfqpoint{0.844709in}{1.821214in}}%
\pgfpathlineto{\pgfqpoint{0.853300in}{1.818002in}}%
\pgfpathlineto{\pgfqpoint{0.861890in}{1.815835in}}%
\pgfpathlineto{\pgfqpoint{0.868333in}{1.814838in}}%
\pgfpathlineto{\pgfqpoint{0.876924in}{1.812869in}}%
\pgfpathlineto{\pgfqpoint{0.887663in}{1.811634in}}%
\pgfpathlineto{\pgfqpoint{0.899475in}{1.808963in}}%
\pgfpathlineto{\pgfqpoint{0.904844in}{1.807931in}}%
\pgfpathlineto{\pgfqpoint{0.906992in}{1.807391in}}%
\pgfpathlineto{\pgfqpoint{0.928468in}{1.806543in}}%
\pgfpathlineto{\pgfqpoint{0.935985in}{1.807214in}}%
\pgfpathlineto{\pgfqpoint{0.944576in}{1.808075in}}%
\pgfpathlineto{\pgfqpoint{0.958536in}{1.808701in}}%
\pgfpathlineto{\pgfqpoint{0.963905in}{1.809375in}}%
\pgfpathlineto{\pgfqpoint{0.967127in}{1.809895in}}%
\pgfpathlineto{\pgfqpoint{0.988603in}{1.810210in}}%
\pgfpathlineto{\pgfqpoint{0.996120in}{1.810168in}}%
\pgfpathlineto{\pgfqpoint{1.025114in}{1.810071in}}%
\pgfpathlineto{\pgfqpoint{1.032631in}{1.809870in}}%
\pgfpathlineto{\pgfqpoint{1.048738in}{1.810863in}}%
\pgfpathlineto{\pgfqpoint{1.057329in}{1.811661in}}%
\pgfpathlineto{\pgfqpoint{1.063772in}{1.812372in}}%
\pgfpathlineto{\pgfqpoint{1.068067in}{1.812740in}}%
\pgfpathlineto{\pgfqpoint{1.084175in}{1.814566in}}%
\pgfpathlineto{\pgfqpoint{1.094913in}{1.816777in}}%
\pgfpathlineto{\pgfqpoint{1.101356in}{1.817824in}}%
\pgfpathlineto{\pgfqpoint{1.102430in}{1.818224in}}%
\pgfpathlineto{\pgfqpoint{1.107799in}{1.819380in}}%
\pgfpathlineto{\pgfqpoint{1.117464in}{1.821940in}}%
\pgfpathlineto{\pgfqpoint{1.122833in}{1.823047in}}%
\pgfpathlineto{\pgfqpoint{1.124981in}{1.823829in}}%
\pgfpathlineto{\pgfqpoint{1.131424in}{1.824978in}}%
\pgfpathlineto{\pgfqpoint{1.132498in}{1.825365in}}%
\pgfpathlineto{\pgfqpoint{1.137867in}{1.826405in}}%
\pgfpathlineto{\pgfqpoint{1.140015in}{1.827147in}}%
\pgfpathlineto{\pgfqpoint{1.145384in}{1.828246in}}%
\pgfpathlineto{\pgfqpoint{1.147531in}{1.829058in}}%
\pgfpathlineto{\pgfqpoint{1.151827in}{1.829928in}}%
\pgfpathlineto{\pgfqpoint{1.155048in}{1.831234in}}%
\pgfpathlineto{\pgfqpoint{1.159344in}{1.832070in}}%
\pgfpathlineto{\pgfqpoint{1.162565in}{1.833324in}}%
\pgfpathlineto{\pgfqpoint{1.167934in}{1.834511in}}%
\pgfpathlineto{\pgfqpoint{1.169008in}{1.834924in}}%
\pgfpathlineto{\pgfqpoint{1.174377in}{1.835713in}}%
\pgfpathlineto{\pgfqpoint{1.177599in}{1.836883in}}%
\pgfpathlineto{\pgfqpoint{1.181894in}{1.837712in}}%
\pgfpathlineto{\pgfqpoint{1.185116in}{1.839062in}}%
\pgfpathlineto{\pgfqpoint{1.190485in}{1.840224in}}%
\pgfpathlineto{\pgfqpoint{1.200150in}{1.842610in}}%
\pgfpathlineto{\pgfqpoint{1.205519in}{1.843517in}}%
\pgfpathlineto{\pgfqpoint{1.215183in}{1.846033in}}%
\pgfpathlineto{\pgfqpoint{1.219479in}{1.846821in}}%
\pgfpathlineto{\pgfqpoint{1.222700in}{1.848080in}}%
\pgfpathlineto{\pgfqpoint{1.226996in}{1.848975in}}%
\pgfpathlineto{\pgfqpoint{1.230217in}{1.850158in}}%
\pgfpathlineto{\pgfqpoint{1.235586in}{1.850932in}}%
\pgfpathlineto{\pgfqpoint{1.237734in}{1.851692in}}%
\pgfpathlineto{\pgfqpoint{1.243103in}{1.852719in}}%
\pgfpathlineto{\pgfqpoint{1.252768in}{1.854971in}}%
\pgfpathlineto{\pgfqpoint{1.258137in}{1.856014in}}%
\pgfpathlineto{\pgfqpoint{1.260285in}{1.856558in}}%
\pgfpathlineto{\pgfqpoint{1.265654in}{1.857351in}}%
\pgfpathlineto{\pgfqpoint{1.267801in}{1.857928in}}%
\pgfpathlineto{\pgfqpoint{1.275318in}{1.859231in}}%
\pgfpathlineto{\pgfqpoint{1.280687in}{1.860381in}}%
\pgfpathlineto{\pgfqpoint{1.282835in}{1.861240in}}%
\pgfpathlineto{\pgfqpoint{1.287130in}{1.862108in}}%
\pgfpathlineto{\pgfqpoint{1.290352in}{1.863495in}}%
\pgfpathlineto{\pgfqpoint{1.294647in}{1.864504in}}%
\pgfpathlineto{\pgfqpoint{1.297869in}{1.866089in}}%
\pgfpathlineto{\pgfqpoint{1.302164in}{1.867151in}}%
\pgfpathlineto{\pgfqpoint{1.305386in}{1.868829in}}%
\pgfpathlineto{\pgfqpoint{1.309681in}{1.869891in}}%
\pgfpathlineto{\pgfqpoint{1.312903in}{1.871469in}}%
\pgfpathlineto{\pgfqpoint{1.317198in}{1.872516in}}%
\pgfpathlineto{\pgfqpoint{1.320419in}{1.873938in}}%
\pgfpathlineto{\pgfqpoint{1.324715in}{1.874821in}}%
\pgfpathlineto{\pgfqpoint{1.327936in}{1.876127in}}%
\pgfpathlineto{\pgfqpoint{1.333306in}{1.877299in}}%
\pgfpathlineto{\pgfqpoint{1.335453in}{1.878051in}}%
\pgfpathlineto{\pgfqpoint{1.340822in}{1.878915in}}%
\pgfpathlineto{\pgfqpoint{1.342970in}{1.879790in}}%
\pgfpathlineto{\pgfqpoint{1.347265in}{1.880740in}}%
\pgfpathlineto{\pgfqpoint{1.350487in}{1.882319in}}%
\pgfpathlineto{\pgfqpoint{1.354782in}{1.883446in}}%
\pgfpathlineto{\pgfqpoint{1.358004in}{1.885169in}}%
\pgfpathlineto{\pgfqpoint{1.362299in}{1.886256in}}%
\pgfpathlineto{\pgfqpoint{1.365521in}{1.887850in}}%
\pgfpathlineto{\pgfqpoint{1.369816in}{1.888824in}}%
\pgfpathlineto{\pgfqpoint{1.373038in}{1.890064in}}%
\pgfpathlineto{\pgfqpoint{1.378407in}{1.891218in}}%
\pgfpathlineto{\pgfqpoint{1.380554in}{1.892113in}}%
\pgfpathlineto{\pgfqpoint{1.384850in}{1.892995in}}%
\pgfpathlineto{\pgfqpoint{1.388071in}{1.894367in}}%
\pgfpathlineto{\pgfqpoint{1.393440in}{1.895680in}}%
\pgfpathlineto{\pgfqpoint{1.395588in}{1.896569in}}%
\pgfpathlineto{\pgfqpoint{1.400957in}{1.897818in}}%
\pgfpathlineto{\pgfqpoint{1.403105in}{1.898672in}}%
\pgfpathlineto{\pgfqpoint{1.407400in}{1.899556in}}%
\pgfpathlineto{\pgfqpoint{1.410622in}{1.900888in}}%
\pgfpathlineto{\pgfqpoint{1.414917in}{1.901750in}}%
\pgfpathlineto{\pgfqpoint{1.418139in}{1.903093in}}%
\pgfpathlineto{\pgfqpoint{1.422434in}{1.904013in}}%
\pgfpathlineto{\pgfqpoint{1.425656in}{1.905407in}}%
\pgfpathlineto{\pgfqpoint{1.429951in}{1.906366in}}%
\pgfpathlineto{\pgfqpoint{1.433173in}{1.907338in}}%
\pgfpathlineto{\pgfqpoint{1.437468in}{1.908257in}}%
\pgfpathlineto{\pgfqpoint{1.440689in}{1.909604in}}%
\pgfpathlineto{\pgfqpoint{1.444985in}{1.910545in}}%
\pgfpathlineto{\pgfqpoint{1.448206in}{1.911771in}}%
\pgfpathlineto{\pgfqpoint{1.453575in}{1.913016in}}%
\pgfpathlineto{\pgfqpoint{1.455723in}{1.913891in}}%
\pgfpathlineto{\pgfqpoint{1.462166in}{1.915278in}}%
\pgfpathlineto{\pgfqpoint{1.463240in}{1.915758in}}%
\pgfpathlineto{\pgfqpoint{1.469683in}{1.917197in}}%
\pgfpathlineto{\pgfqpoint{1.470757in}{1.917671in}}%
\pgfpathlineto{\pgfqpoint{1.475052in}{1.918622in}}%
\pgfpathlineto{\pgfqpoint{1.478274in}{1.920067in}}%
\pgfpathlineto{\pgfqpoint{1.482569in}{1.920984in}}%
\pgfpathlineto{\pgfqpoint{1.485791in}{1.922424in}}%
\pgfpathlineto{\pgfqpoint{1.491160in}{1.923425in}}%
\pgfpathlineto{\pgfqpoint{1.493307in}{1.924338in}}%
\pgfpathlineto{\pgfqpoint{1.497603in}{1.925262in}}%
\pgfpathlineto{\pgfqpoint{1.500824in}{1.926644in}}%
\pgfpathlineto{\pgfqpoint{1.506194in}{1.927749in}}%
\pgfpathlineto{\pgfqpoint{1.508341in}{1.928516in}}%
\pgfpathlineto{\pgfqpoint{1.512637in}{1.929353in}}%
\pgfpathlineto{\pgfqpoint{1.515858in}{1.930682in}}%
\pgfpathlineto{\pgfqpoint{1.521227in}{1.931579in}}%
\pgfpathlineto{\pgfqpoint{1.523375in}{1.932516in}}%
\pgfpathlineto{\pgfqpoint{1.527670in}{1.933479in}}%
\pgfpathlineto{\pgfqpoint{1.530892in}{1.934930in}}%
\pgfpathlineto{\pgfqpoint{1.535187in}{1.935910in}}%
\pgfpathlineto{\pgfqpoint{1.538409in}{1.937398in}}%
\pgfpathlineto{\pgfqpoint{1.542704in}{1.938343in}}%
\pgfpathlineto{\pgfqpoint{1.545926in}{1.939600in}}%
\pgfpathlineto{\pgfqpoint{1.550221in}{1.940456in}}%
\pgfpathlineto{\pgfqpoint{1.553442in}{1.941708in}}%
\pgfpathlineto{\pgfqpoint{1.557738in}{1.942547in}}%
\pgfpathlineto{\pgfqpoint{1.560959in}{1.943790in}}%
\pgfpathlineto{\pgfqpoint{1.565255in}{1.944710in}}%
\pgfpathlineto{\pgfqpoint{1.568476in}{1.946170in}}%
\pgfpathlineto{\pgfqpoint{1.572772in}{1.947028in}}%
\pgfpathlineto{\pgfqpoint{1.575993in}{1.948264in}}%
\pgfpathlineto{\pgfqpoint{1.580288in}{1.949087in}}%
\pgfpathlineto{\pgfqpoint{1.582436in}{1.950000in}}%
\pgfpathlineto{\pgfqpoint{1.587805in}{1.950920in}}%
\pgfpathlineto{\pgfqpoint{1.591027in}{1.952281in}}%
\pgfpathlineto{\pgfqpoint{1.595322in}{1.953150in}}%
\pgfpathlineto{\pgfqpoint{1.598544in}{1.954422in}}%
\pgfpathlineto{\pgfqpoint{1.603913in}{1.955651in}}%
\pgfpathlineto{\pgfqpoint{1.606061in}{1.956477in}}%
\pgfpathlineto{\pgfqpoint{1.610356in}{1.957350in}}%
\pgfpathlineto{\pgfqpoint{1.613577in}{1.958543in}}%
\pgfpathlineto{\pgfqpoint{1.618947in}{1.959652in}}%
\pgfpathlineto{\pgfqpoint{1.621094in}{1.960416in}}%
\pgfpathlineto{\pgfqpoint{1.626463in}{1.961193in}}%
\pgfpathlineto{\pgfqpoint{1.628611in}{1.961977in}}%
\pgfpathlineto{\pgfqpoint{1.633980in}{1.963197in}}%
\pgfpathlineto{\pgfqpoint{1.636128in}{1.964037in}}%
\pgfpathlineto{\pgfqpoint{1.641497in}{1.965316in}}%
\pgfpathlineto{\pgfqpoint{1.643645in}{1.966090in}}%
\pgfpathlineto{\pgfqpoint{1.649014in}{1.967251in}}%
\pgfpathlineto{\pgfqpoint{1.651162in}{1.968042in}}%
\pgfpathlineto{\pgfqpoint{1.656531in}{1.969153in}}%
\pgfpathlineto{\pgfqpoint{1.658679in}{1.969882in}}%
\pgfpathlineto{\pgfqpoint{1.664048in}{1.970939in}}%
\pgfpathlineto{\pgfqpoint{1.665122in}{1.971290in}}%
\pgfpathlineto{\pgfqpoint{1.671565in}{1.972281in}}%
\pgfpathlineto{\pgfqpoint{1.673712in}{1.972922in}}%
\pgfpathlineto{\pgfqpoint{1.679082in}{1.973914in}}%
\pgfpathlineto{\pgfqpoint{1.681229in}{1.974525in}}%
\pgfpathlineto{\pgfqpoint{1.687672in}{1.975589in}}%
\pgfpathlineto{\pgfqpoint{1.694115in}{1.976482in}}%
\pgfpathlineto{\pgfqpoint{1.703780in}{1.977715in}}%
\pgfpathlineto{\pgfqpoint{1.714518in}{1.978877in}}%
\pgfpathlineto{\pgfqpoint{1.726330in}{1.981027in}}%
\pgfpathlineto{\pgfqpoint{1.733847in}{1.981951in}}%
\pgfpathlineto{\pgfqpoint{1.740290in}{1.982838in}}%
\pgfpathlineto{\pgfqpoint{1.748881in}{1.984132in}}%
\pgfpathlineto{\pgfqpoint{1.760693in}{1.985339in}}%
\pgfpathlineto{\pgfqpoint{1.771432in}{1.986363in}}%
\pgfpathlineto{\pgfqpoint{1.784318in}{1.987156in}}%
\pgfpathlineto{\pgfqpoint{1.838010in}{1.994284in}}%
\pgfpathlineto{\pgfqpoint{1.846600in}{1.995844in}}%
\pgfpathlineto{\pgfqpoint{1.857339in}{1.997321in}}%
\pgfpathlineto{\pgfqpoint{1.861634in}{1.998135in}}%
\pgfpathlineto{\pgfqpoint{1.868077in}{1.999091in}}%
\pgfpathlineto{\pgfqpoint{1.876668in}{2.000605in}}%
\pgfpathlineto{\pgfqpoint{1.883111in}{2.001525in}}%
\pgfpathlineto{\pgfqpoint{1.884185in}{2.001842in}}%
\pgfpathlineto{\pgfqpoint{1.890628in}{2.002997in}}%
\pgfpathlineto{\pgfqpoint{1.899218in}{2.004719in}}%
\pgfpathlineto{\pgfqpoint{1.904588in}{2.005619in}}%
\pgfpathlineto{\pgfqpoint{1.906735in}{2.006257in}}%
\pgfpathlineto{\pgfqpoint{1.913178in}{2.007267in}}%
\pgfpathlineto{\pgfqpoint{1.914252in}{2.007627in}}%
\pgfpathlineto{\pgfqpoint{1.919621in}{2.008683in}}%
\pgfpathlineto{\pgfqpoint{1.921769in}{2.009351in}}%
\pgfpathlineto{\pgfqpoint{1.927138in}{2.010334in}}%
\pgfpathlineto{\pgfqpoint{1.929286in}{2.010942in}}%
\pgfpathlineto{\pgfqpoint{1.935729in}{2.012127in}}%
\pgfpathlineto{\pgfqpoint{1.944320in}{2.013881in}}%
\pgfpathlineto{\pgfqpoint{1.950763in}{2.014936in}}%
\pgfpathlineto{\pgfqpoint{1.958280in}{2.016222in}}%
\pgfpathlineto{\pgfqpoint{1.964723in}{2.017004in}}%
\pgfpathlineto{\pgfqpoint{1.973313in}{2.018557in}}%
\pgfpathlineto{\pgfqpoint{1.981904in}{2.019998in}}%
\pgfpathlineto{\pgfqpoint{1.988347in}{2.020890in}}%
\pgfpathlineto{\pgfqpoint{1.996938in}{2.022270in}}%
\pgfpathlineto{\pgfqpoint{2.003381in}{2.023274in}}%
\pgfpathlineto{\pgfqpoint{2.011971in}{2.024845in}}%
\pgfpathlineto{\pgfqpoint{2.019488in}{2.025815in}}%
\pgfpathlineto{\pgfqpoint{2.025931in}{2.026784in}}%
\pgfpathlineto{\pgfqpoint{2.034522in}{2.028197in}}%
\pgfpathlineto{\pgfqpoint{2.040965in}{2.029039in}}%
\pgfpathlineto{\pgfqpoint{2.049556in}{2.030238in}}%
\pgfpathlineto{\pgfqpoint{2.063516in}{2.031434in}}%
\pgfpathlineto{\pgfqpoint{2.075328in}{2.032538in}}%
\pgfpathlineto{\pgfqpoint{2.090362in}{2.032947in}}%
\pgfpathlineto{\pgfqpoint{2.116134in}{2.033064in}}%
\pgfpathlineto{\pgfqpoint{2.188081in}{2.031797in}}%
\pgfpathlineto{\pgfqpoint{2.214927in}{2.032377in}}%
\pgfpathlineto{\pgfqpoint{2.266471in}{2.032059in}}%
\pgfpathlineto{\pgfqpoint{2.282579in}{2.031185in}}%
\pgfpathlineto{\pgfqpoint{2.335197in}{2.030542in}}%
\pgfpathlineto{\pgfqpoint{2.362043in}{2.031408in}}%
\pgfpathlineto{\pgfqpoint{2.384593in}{2.032809in}}%
\pgfpathlineto{\pgfqpoint{2.417882in}{2.034024in}}%
\pgfpathlineto{\pgfqpoint{2.445802in}{2.035114in}}%
\pgfpathlineto{\pgfqpoint{2.538152in}{2.040960in}}%
\pgfpathlineto{\pgfqpoint{2.578958in}{2.042362in}}%
\pgfpathlineto{\pgfqpoint{2.632650in}{2.046067in}}%
\pgfpathlineto{\pgfqpoint{2.643389in}{2.046972in}}%
\pgfpathlineto{\pgfqpoint{2.656275in}{2.047917in}}%
\pgfpathlineto{\pgfqpoint{2.665939in}{2.048679in}}%
\pgfpathlineto{\pgfqpoint{2.679899in}{2.049762in}}%
\pgfpathlineto{\pgfqpoint{2.696007in}{2.051194in}}%
\pgfpathlineto{\pgfqpoint{2.707819in}{2.052049in}}%
\pgfpathlineto{\pgfqpoint{2.718557in}{2.053164in}}%
\pgfpathlineto{\pgfqpoint{2.730369in}{2.054148in}}%
\pgfpathlineto{\pgfqpoint{2.748625in}{2.055910in}}%
\pgfpathlineto{\pgfqpoint{2.762585in}{2.057020in}}%
\pgfpathlineto{\pgfqpoint{2.778692in}{2.058986in}}%
\pgfpathlineto{\pgfqpoint{2.789431in}{2.060154in}}%
\pgfpathlineto{\pgfqpoint{2.801243in}{2.061936in}}%
\pgfpathlineto{\pgfqpoint{2.811981in}{2.062927in}}%
\pgfpathlineto{\pgfqpoint{2.823793in}{2.064606in}}%
\pgfpathlineto{\pgfqpoint{2.834532in}{2.065789in}}%
\pgfpathlineto{\pgfqpoint{2.846344in}{2.067364in}}%
\pgfpathlineto{\pgfqpoint{2.857082in}{2.068584in}}%
\pgfpathlineto{\pgfqpoint{2.868895in}{2.070268in}}%
\pgfpathlineto{\pgfqpoint{2.879633in}{2.071313in}}%
\pgfpathlineto{\pgfqpoint{2.891445in}{2.072691in}}%
\pgfpathlineto{\pgfqpoint{2.903257in}{2.073734in}}%
\pgfpathlineto{\pgfqpoint{2.906479in}{2.074242in}}%
\pgfpathlineto{\pgfqpoint{2.920439in}{2.075097in}}%
\pgfpathlineto{\pgfqpoint{2.936546in}{2.076585in}}%
\pgfpathlineto{\pgfqpoint{2.948359in}{2.077683in}}%
\pgfpathlineto{\pgfqpoint{2.966614in}{2.079817in}}%
\pgfpathlineto{\pgfqpoint{2.977352in}{2.080833in}}%
\pgfpathlineto{\pgfqpoint{2.989165in}{2.082224in}}%
\pgfpathlineto{\pgfqpoint{3.002051in}{2.083263in}}%
\pgfpathlineto{\pgfqpoint{3.011715in}{2.084483in}}%
\pgfpathlineto{\pgfqpoint{3.022454in}{2.085660in}}%
\pgfpathlineto{\pgfqpoint{3.026749in}{2.086505in}}%
\pgfpathlineto{\pgfqpoint{3.034266in}{2.087360in}}%
\pgfpathlineto{\pgfqpoint{3.034266in}{2.087360in}}%
\pgfusepath{stroke}%
\end{pgfscope}%
\begin{pgfscope}%
\pgfsetrectcap%
\pgfsetmiterjoin%
\pgfsetlinewidth{0.803000pt}%
\definecolor{currentstroke}{rgb}{1.000000,1.000000,1.000000}%
\pgfsetstrokecolor{currentstroke}%
\pgfsetdash{}{0pt}%
\pgfpathmoveto{\pgfqpoint{0.568357in}{1.711254in}}%
\pgfpathlineto{\pgfqpoint{0.568357in}{2.447839in}}%
\pgfusepath{stroke}%
\end{pgfscope}%
\begin{pgfscope}%
\pgfsetrectcap%
\pgfsetmiterjoin%
\pgfsetlinewidth{0.803000pt}%
\definecolor{currentstroke}{rgb}{1.000000,1.000000,1.000000}%
\pgfsetstrokecolor{currentstroke}%
\pgfsetdash{}{0pt}%
\pgfpathmoveto{\pgfqpoint{3.151690in}{1.711254in}}%
\pgfpathlineto{\pgfqpoint{3.151690in}{2.447839in}}%
\pgfusepath{stroke}%
\end{pgfscope}%
\begin{pgfscope}%
\pgfsetrectcap%
\pgfsetmiterjoin%
\pgfsetlinewidth{0.803000pt}%
\definecolor{currentstroke}{rgb}{1.000000,1.000000,1.000000}%
\pgfsetstrokecolor{currentstroke}%
\pgfsetdash{}{0pt}%
\pgfpathmoveto{\pgfqpoint{0.568357in}{1.711254in}}%
\pgfpathlineto{\pgfqpoint{3.151690in}{1.711254in}}%
\pgfusepath{stroke}%
\end{pgfscope}%
\begin{pgfscope}%
\pgfsetrectcap%
\pgfsetmiterjoin%
\pgfsetlinewidth{0.803000pt}%
\definecolor{currentstroke}{rgb}{1.000000,1.000000,1.000000}%
\pgfsetstrokecolor{currentstroke}%
\pgfsetdash{}{0pt}%
\pgfpathmoveto{\pgfqpoint{0.568357in}{2.447839in}}%
\pgfpathlineto{\pgfqpoint{3.151690in}{2.447839in}}%
\pgfusepath{stroke}%
\end{pgfscope}%
\begin{pgfscope}%
\definecolor{textcolor}{rgb}{0.150000,0.150000,0.150000}%
\pgfsetstrokecolor{textcolor}%
\pgfsetfillcolor{textcolor}%
\pgftext[x=1.860023in,y=2.531173in,,base]{\color{textcolor}\rmfamily\fontsize{16.800000}{20.160000}\selectfont UTX}%
\end{pgfscope}%
\begin{pgfscope}%
\pgfsetbuttcap%
\pgfsetmiterjoin%
\definecolor{currentfill}{rgb}{0.917647,0.917647,0.949020}%
\pgfsetfillcolor{currentfill}%
\pgfsetlinewidth{0.000000pt}%
\definecolor{currentstroke}{rgb}{0.000000,0.000000,0.000000}%
\pgfsetstrokecolor{currentstroke}%
\pgfsetstrokeopacity{0.000000}%
\pgfsetdash{}{0pt}%
\pgfpathmoveto{\pgfqpoint{4.185023in}{1.711254in}}%
\pgfpathlineto{\pgfqpoint{6.768357in}{1.711254in}}%
\pgfpathlineto{\pgfqpoint{6.768357in}{2.447839in}}%
\pgfpathlineto{\pgfqpoint{4.185023in}{2.447839in}}%
\pgfpathclose%
\pgfusepath{fill}%
\end{pgfscope}%
\begin{pgfscope}%
\pgfpathrectangle{\pgfqpoint{4.185023in}{1.711254in}}{\pgfqpoint{2.583333in}{0.736585in}}%
\pgfusepath{clip}%
\pgfsetroundcap%
\pgfsetroundjoin%
\pgfsetlinewidth{0.803000pt}%
\definecolor{currentstroke}{rgb}{1.000000,1.000000,1.000000}%
\pgfsetstrokecolor{currentstroke}%
\pgfsetdash{}{0pt}%
\pgfpathmoveto{\pgfqpoint{4.300300in}{1.711254in}}%
\pgfpathlineto{\pgfqpoint{4.300300in}{2.447839in}}%
\pgfusepath{stroke}%
\end{pgfscope}%
\begin{pgfscope}%
\definecolor{textcolor}{rgb}{0.150000,0.150000,0.150000}%
\pgfsetstrokecolor{textcolor}%
\pgfsetfillcolor{textcolor}%
\pgftext[x=4.300300in,y=1.614032in,,top]{\color{textcolor}\rmfamily\fontsize{14.000000}{16.800000}\selectfont 2012}%
\end{pgfscope}%
\begin{pgfscope}%
\pgfpathrectangle{\pgfqpoint{4.185023in}{1.711254in}}{\pgfqpoint{2.583333in}{0.736585in}}%
\pgfusepath{clip}%
\pgfsetroundcap%
\pgfsetroundjoin%
\pgfsetlinewidth{0.803000pt}%
\definecolor{currentstroke}{rgb}{1.000000,1.000000,1.000000}%
\pgfsetstrokecolor{currentstroke}%
\pgfsetdash{}{0pt}%
\pgfpathmoveto{\pgfqpoint{4.693325in}{1.711254in}}%
\pgfpathlineto{\pgfqpoint{4.693325in}{2.447839in}}%
\pgfusepath{stroke}%
\end{pgfscope}%
\begin{pgfscope}%
\definecolor{textcolor}{rgb}{0.150000,0.150000,0.150000}%
\pgfsetstrokecolor{textcolor}%
\pgfsetfillcolor{textcolor}%
\pgftext[x=4.693325in,y=1.614032in,,top]{\color{textcolor}\rmfamily\fontsize{14.000000}{16.800000}\selectfont 2013}%
\end{pgfscope}%
\begin{pgfscope}%
\pgfpathrectangle{\pgfqpoint{4.185023in}{1.711254in}}{\pgfqpoint{2.583333in}{0.736585in}}%
\pgfusepath{clip}%
\pgfsetroundcap%
\pgfsetroundjoin%
\pgfsetlinewidth{0.803000pt}%
\definecolor{currentstroke}{rgb}{1.000000,1.000000,1.000000}%
\pgfsetstrokecolor{currentstroke}%
\pgfsetdash{}{0pt}%
\pgfpathmoveto{\pgfqpoint{5.085276in}{1.711254in}}%
\pgfpathlineto{\pgfqpoint{5.085276in}{2.447839in}}%
\pgfusepath{stroke}%
\end{pgfscope}%
\begin{pgfscope}%
\definecolor{textcolor}{rgb}{0.150000,0.150000,0.150000}%
\pgfsetstrokecolor{textcolor}%
\pgfsetfillcolor{textcolor}%
\pgftext[x=5.085276in,y=1.614032in,,top]{\color{textcolor}\rmfamily\fontsize{14.000000}{16.800000}\selectfont 2014}%
\end{pgfscope}%
\begin{pgfscope}%
\pgfpathrectangle{\pgfqpoint{4.185023in}{1.711254in}}{\pgfqpoint{2.583333in}{0.736585in}}%
\pgfusepath{clip}%
\pgfsetroundcap%
\pgfsetroundjoin%
\pgfsetlinewidth{0.803000pt}%
\definecolor{currentstroke}{rgb}{1.000000,1.000000,1.000000}%
\pgfsetstrokecolor{currentstroke}%
\pgfsetdash{}{0pt}%
\pgfpathmoveto{\pgfqpoint{5.477227in}{1.711254in}}%
\pgfpathlineto{\pgfqpoint{5.477227in}{2.447839in}}%
\pgfusepath{stroke}%
\end{pgfscope}%
\begin{pgfscope}%
\definecolor{textcolor}{rgb}{0.150000,0.150000,0.150000}%
\pgfsetstrokecolor{textcolor}%
\pgfsetfillcolor{textcolor}%
\pgftext[x=5.477227in,y=1.614032in,,top]{\color{textcolor}\rmfamily\fontsize{14.000000}{16.800000}\selectfont 2015}%
\end{pgfscope}%
\begin{pgfscope}%
\pgfpathrectangle{\pgfqpoint{4.185023in}{1.711254in}}{\pgfqpoint{2.583333in}{0.736585in}}%
\pgfusepath{clip}%
\pgfsetroundcap%
\pgfsetroundjoin%
\pgfsetlinewidth{0.803000pt}%
\definecolor{currentstroke}{rgb}{1.000000,1.000000,1.000000}%
\pgfsetstrokecolor{currentstroke}%
\pgfsetdash{}{0pt}%
\pgfpathmoveto{\pgfqpoint{5.869178in}{1.711254in}}%
\pgfpathlineto{\pgfqpoint{5.869178in}{2.447839in}}%
\pgfusepath{stroke}%
\end{pgfscope}%
\begin{pgfscope}%
\definecolor{textcolor}{rgb}{0.150000,0.150000,0.150000}%
\pgfsetstrokecolor{textcolor}%
\pgfsetfillcolor{textcolor}%
\pgftext[x=5.869178in,y=1.614032in,,top]{\color{textcolor}\rmfamily\fontsize{14.000000}{16.800000}\selectfont 2016}%
\end{pgfscope}%
\begin{pgfscope}%
\pgfpathrectangle{\pgfqpoint{4.185023in}{1.711254in}}{\pgfqpoint{2.583333in}{0.736585in}}%
\pgfusepath{clip}%
\pgfsetroundcap%
\pgfsetroundjoin%
\pgfsetlinewidth{0.803000pt}%
\definecolor{currentstroke}{rgb}{1.000000,1.000000,1.000000}%
\pgfsetstrokecolor{currentstroke}%
\pgfsetdash{}{0pt}%
\pgfpathmoveto{\pgfqpoint{6.262203in}{1.711254in}}%
\pgfpathlineto{\pgfqpoint{6.262203in}{2.447839in}}%
\pgfusepath{stroke}%
\end{pgfscope}%
\begin{pgfscope}%
\definecolor{textcolor}{rgb}{0.150000,0.150000,0.150000}%
\pgfsetstrokecolor{textcolor}%
\pgfsetfillcolor{textcolor}%
\pgftext[x=6.262203in,y=1.614032in,,top]{\color{textcolor}\rmfamily\fontsize{14.000000}{16.800000}\selectfont 2017}%
\end{pgfscope}%
\begin{pgfscope}%
\pgfpathrectangle{\pgfqpoint{4.185023in}{1.711254in}}{\pgfqpoint{2.583333in}{0.736585in}}%
\pgfusepath{clip}%
\pgfsetroundcap%
\pgfsetroundjoin%
\pgfsetlinewidth{0.803000pt}%
\definecolor{currentstroke}{rgb}{1.000000,1.000000,1.000000}%
\pgfsetstrokecolor{currentstroke}%
\pgfsetdash{}{0pt}%
\pgfpathmoveto{\pgfqpoint{6.654154in}{1.711254in}}%
\pgfpathlineto{\pgfqpoint{6.654154in}{2.447839in}}%
\pgfusepath{stroke}%
\end{pgfscope}%
\begin{pgfscope}%
\definecolor{textcolor}{rgb}{0.150000,0.150000,0.150000}%
\pgfsetstrokecolor{textcolor}%
\pgfsetfillcolor{textcolor}%
\pgftext[x=6.654154in,y=1.614032in,,top]{\color{textcolor}\rmfamily\fontsize{14.000000}{16.800000}\selectfont 2018}%
\end{pgfscope}%
\begin{pgfscope}%
\pgfpathrectangle{\pgfqpoint{4.185023in}{1.711254in}}{\pgfqpoint{2.583333in}{0.736585in}}%
\pgfusepath{clip}%
\pgfsetroundcap%
\pgfsetroundjoin%
\pgfsetlinewidth{0.803000pt}%
\definecolor{currentstroke}{rgb}{1.000000,1.000000,1.000000}%
\pgfsetstrokecolor{currentstroke}%
\pgfsetdash{}{0pt}%
\pgfpathmoveto{\pgfqpoint{4.185023in}{1.846609in}}%
\pgfpathlineto{\pgfqpoint{6.768357in}{1.846609in}}%
\pgfusepath{stroke}%
\end{pgfscope}%
\begin{pgfscope}%
\definecolor{textcolor}{rgb}{0.150000,0.150000,0.150000}%
\pgfsetstrokecolor{textcolor}%
\pgfsetfillcolor{textcolor}%
\pgftext[x=3.840378in,y=1.772743in,left,base]{\color{textcolor}\rmfamily\fontsize{14.000000}{16.800000}\selectfont 30}%
\end{pgfscope}%
\begin{pgfscope}%
\pgfpathrectangle{\pgfqpoint{4.185023in}{1.711254in}}{\pgfqpoint{2.583333in}{0.736585in}}%
\pgfusepath{clip}%
\pgfsetroundcap%
\pgfsetroundjoin%
\pgfsetlinewidth{0.803000pt}%
\definecolor{currentstroke}{rgb}{1.000000,1.000000,1.000000}%
\pgfsetstrokecolor{currentstroke}%
\pgfsetdash{}{0pt}%
\pgfpathmoveto{\pgfqpoint{4.185023in}{2.138511in}}%
\pgfpathlineto{\pgfqpoint{6.768357in}{2.138511in}}%
\pgfusepath{stroke}%
\end{pgfscope}%
\begin{pgfscope}%
\definecolor{textcolor}{rgb}{0.150000,0.150000,0.150000}%
\pgfsetstrokecolor{textcolor}%
\pgfsetfillcolor{textcolor}%
\pgftext[x=3.840378in,y=2.064645in,left,base]{\color{textcolor}\rmfamily\fontsize{14.000000}{16.800000}\selectfont 40}%
\end{pgfscope}%
\begin{pgfscope}%
\pgfpathrectangle{\pgfqpoint{4.185023in}{1.711254in}}{\pgfqpoint{2.583333in}{0.736585in}}%
\pgfusepath{clip}%
\pgfsetroundcap%
\pgfsetroundjoin%
\pgfsetlinewidth{0.803000pt}%
\definecolor{currentstroke}{rgb}{1.000000,1.000000,1.000000}%
\pgfsetstrokecolor{currentstroke}%
\pgfsetdash{}{0pt}%
\pgfpathmoveto{\pgfqpoint{4.185023in}{2.430413in}}%
\pgfpathlineto{\pgfqpoint{6.768357in}{2.430413in}}%
\pgfusepath{stroke}%
\end{pgfscope}%
\begin{pgfscope}%
\definecolor{textcolor}{rgb}{0.150000,0.150000,0.150000}%
\pgfsetstrokecolor{textcolor}%
\pgfsetfillcolor{textcolor}%
\pgftext[x=3.840378in,y=2.356547in,left,base]{\color{textcolor}\rmfamily\fontsize{14.000000}{16.800000}\selectfont 50}%
\end{pgfscope}%
\begin{pgfscope}%
\pgfpathrectangle{\pgfqpoint{4.185023in}{1.711254in}}{\pgfqpoint{2.583333in}{0.736585in}}%
\pgfusepath{clip}%
\pgfsetroundcap%
\pgfsetroundjoin%
\pgfsetlinewidth{1.505625pt}%
\definecolor{currentstroke}{rgb}{0.498039,0.498039,0.498039}%
\pgfsetstrokecolor{currentstroke}%
\pgfsetdash{}{0pt}%
\pgfpathmoveto{\pgfqpoint{4.302448in}{1.786477in}}%
\pgfpathlineto{\pgfqpoint{4.304595in}{1.770422in}}%
\pgfpathlineto{\pgfqpoint{4.305669in}{1.768087in}}%
\pgfpathlineto{\pgfqpoint{4.308891in}{1.768963in}}%
\pgfpathlineto{\pgfqpoint{4.309964in}{1.773050in}}%
\pgfpathlineto{\pgfqpoint{4.311038in}{1.780055in}}%
\pgfpathlineto{\pgfqpoint{4.313186in}{1.780347in}}%
\pgfpathlineto{\pgfqpoint{4.317481in}{1.782390in}}%
\pgfpathlineto{\pgfqpoint{4.320703in}{1.781515in}}%
\pgfpathlineto{\pgfqpoint{4.323924in}{1.769547in}}%
\pgfpathlineto{\pgfqpoint{4.324998in}{1.756995in}}%
\pgfpathlineto{\pgfqpoint{4.326072in}{1.754660in}}%
\pgfpathlineto{\pgfqpoint{4.327146in}{1.747362in}}%
\pgfpathlineto{\pgfqpoint{4.328220in}{1.744735in}}%
\pgfpathlineto{\pgfqpoint{4.331441in}{1.753200in}}%
\pgfpathlineto{\pgfqpoint{4.332515in}{1.754076in}}%
\pgfpathlineto{\pgfqpoint{4.333589in}{1.756995in}}%
\pgfpathlineto{\pgfqpoint{4.334663in}{1.752033in}}%
\pgfpathlineto{\pgfqpoint{4.335737in}{1.757871in}}%
\pgfpathlineto{\pgfqpoint{4.338958in}{1.764001in}}%
\pgfpathlineto{\pgfqpoint{4.340032in}{1.759622in}}%
\pgfpathlineto{\pgfqpoint{4.342180in}{1.759622in}}%
\pgfpathlineto{\pgfqpoint{4.343253in}{1.754660in}}%
\pgfpathlineto{\pgfqpoint{4.346475in}{1.764001in}}%
\pgfpathlineto{\pgfqpoint{4.347549in}{1.761957in}}%
\pgfpathlineto{\pgfqpoint{4.348623in}{1.757579in}}%
\pgfpathlineto{\pgfqpoint{4.349696in}{1.762249in}}%
\pgfpathlineto{\pgfqpoint{4.350770in}{1.770714in}}%
\pgfpathlineto{\pgfqpoint{4.355066in}{1.771298in}}%
\pgfpathlineto{\pgfqpoint{4.356140in}{1.765460in}}%
\pgfpathlineto{\pgfqpoint{4.357213in}{1.764001in}}%
\pgfpathlineto{\pgfqpoint{4.362583in}{1.764293in}}%
\pgfpathlineto{\pgfqpoint{4.363656in}{1.763417in}}%
\pgfpathlineto{\pgfqpoint{4.365804in}{1.775093in}}%
\pgfpathlineto{\pgfqpoint{4.369026in}{1.782099in}}%
\pgfpathlineto{\pgfqpoint{4.370099in}{1.775677in}}%
\pgfpathlineto{\pgfqpoint{4.371173in}{1.779471in}}%
\pgfpathlineto{\pgfqpoint{4.372247in}{1.786477in}}%
\pgfpathlineto{\pgfqpoint{4.373321in}{1.784142in}}%
\pgfpathlineto{\pgfqpoint{4.376542in}{1.788812in}}%
\pgfpathlineto{\pgfqpoint{4.377616in}{1.792023in}}%
\pgfpathlineto{\pgfqpoint{4.378690in}{1.791731in}}%
\pgfpathlineto{\pgfqpoint{4.380838in}{1.793775in}}%
\pgfpathlineto{\pgfqpoint{4.384059in}{1.795526in}}%
\pgfpathlineto{\pgfqpoint{4.385133in}{1.795234in}}%
\pgfpathlineto{\pgfqpoint{4.386207in}{1.798153in}}%
\pgfpathlineto{\pgfqpoint{4.387281in}{1.795818in}}%
\pgfpathlineto{\pgfqpoint{4.388355in}{1.790856in}}%
\pgfpathlineto{\pgfqpoint{4.391576in}{1.788812in}}%
\pgfpathlineto{\pgfqpoint{4.393724in}{1.766628in}}%
\pgfpathlineto{\pgfqpoint{4.394798in}{1.762833in}}%
\pgfpathlineto{\pgfqpoint{4.395872in}{1.766044in}}%
\pgfpathlineto{\pgfqpoint{4.399093in}{1.772174in}}%
\pgfpathlineto{\pgfqpoint{4.401241in}{1.769255in}}%
\pgfpathlineto{\pgfqpoint{4.402315in}{1.764584in}}%
\pgfpathlineto{\pgfqpoint{4.406610in}{1.760206in}}%
\pgfpathlineto{\pgfqpoint{4.407684in}{1.746487in}}%
\pgfpathlineto{\pgfqpoint{4.408758in}{1.758454in}}%
\pgfpathlineto{\pgfqpoint{4.409831in}{1.762249in}}%
\pgfpathlineto{\pgfqpoint{4.410905in}{1.756119in}}%
\pgfpathlineto{\pgfqpoint{4.414127in}{1.759622in}}%
\pgfpathlineto{\pgfqpoint{4.415201in}{1.766044in}}%
\pgfpathlineto{\pgfqpoint{4.416274in}{1.764584in}}%
\pgfpathlineto{\pgfqpoint{4.418422in}{1.787061in}}%
\pgfpathlineto{\pgfqpoint{4.421644in}{1.783558in}}%
\pgfpathlineto{\pgfqpoint{4.422717in}{1.803407in}}%
\pgfpathlineto{\pgfqpoint{4.423791in}{1.802824in}}%
\pgfpathlineto{\pgfqpoint{4.424865in}{1.816835in}}%
\pgfpathlineto{\pgfqpoint{4.425939in}{1.818586in}}%
\pgfpathlineto{\pgfqpoint{4.429161in}{1.821797in}}%
\pgfpathlineto{\pgfqpoint{4.430234in}{1.825592in}}%
\pgfpathlineto{\pgfqpoint{4.432382in}{1.827343in}}%
\pgfpathlineto{\pgfqpoint{4.433456in}{1.819170in}}%
\pgfpathlineto{\pgfqpoint{4.436677in}{1.825008in}}%
\pgfpathlineto{\pgfqpoint{4.437751in}{1.825300in}}%
\pgfpathlineto{\pgfqpoint{4.438825in}{1.819170in}}%
\pgfpathlineto{\pgfqpoint{4.439899in}{1.825300in}}%
\pgfpathlineto{\pgfqpoint{4.440973in}{1.838144in}}%
\pgfpathlineto{\pgfqpoint{4.444194in}{1.832598in}}%
\pgfpathlineto{\pgfqpoint{4.445268in}{1.835808in}}%
\pgfpathlineto{\pgfqpoint{4.446342in}{1.832306in}}%
\pgfpathlineto{\pgfqpoint{4.447416in}{1.842814in}}%
\pgfpathlineto{\pgfqpoint{4.448490in}{1.846025in}}%
\pgfpathlineto{\pgfqpoint{4.451711in}{1.841938in}}%
\pgfpathlineto{\pgfqpoint{4.452785in}{1.843106in}}%
\pgfpathlineto{\pgfqpoint{4.453859in}{1.840771in}}%
\pgfpathlineto{\pgfqpoint{4.456006in}{1.844274in}}%
\pgfpathlineto{\pgfqpoint{4.460302in}{1.850695in}}%
\pgfpathlineto{\pgfqpoint{4.461376in}{1.843690in}}%
\pgfpathlineto{\pgfqpoint{4.462450in}{1.848360in}}%
\pgfpathlineto{\pgfqpoint{4.463523in}{1.835517in}}%
\pgfpathlineto{\pgfqpoint{4.466745in}{1.841938in}}%
\pgfpathlineto{\pgfqpoint{4.467819in}{1.839019in}}%
\pgfpathlineto{\pgfqpoint{4.468893in}{1.850987in}}%
\pgfpathlineto{\pgfqpoint{4.469966in}{1.848360in}}%
\pgfpathlineto{\pgfqpoint{4.471040in}{1.865291in}}%
\pgfpathlineto{\pgfqpoint{4.474262in}{1.867626in}}%
\pgfpathlineto{\pgfqpoint{4.475336in}{1.875799in}}%
\pgfpathlineto{\pgfqpoint{4.476409in}{1.876675in}}%
\pgfpathlineto{\pgfqpoint{4.477483in}{1.892729in}}%
\pgfpathlineto{\pgfqpoint{4.478557in}{1.888643in}}%
\pgfpathlineto{\pgfqpoint{4.481779in}{1.894189in}}%
\pgfpathlineto{\pgfqpoint{4.482852in}{1.892437in}}%
\pgfpathlineto{\pgfqpoint{4.483926in}{1.883389in}}%
\pgfpathlineto{\pgfqpoint{4.485000in}{1.883972in}}%
\pgfpathlineto{\pgfqpoint{4.486074in}{1.897108in}}%
\pgfpathlineto{\pgfqpoint{4.489295in}{1.890686in}}%
\pgfpathlineto{\pgfqpoint{4.490369in}{1.894773in}}%
\pgfpathlineto{\pgfqpoint{4.491443in}{1.893605in}}%
\pgfpathlineto{\pgfqpoint{4.492517in}{1.897400in}}%
\pgfpathlineto{\pgfqpoint{4.493591in}{1.907324in}}%
\pgfpathlineto{\pgfqpoint{4.496812in}{1.917833in}}%
\pgfpathlineto{\pgfqpoint{4.497886in}{1.918125in}}%
\pgfpathlineto{\pgfqpoint{4.500034in}{1.916957in}}%
\pgfpathlineto{\pgfqpoint{4.501108in}{1.917541in}}%
\pgfpathlineto{\pgfqpoint{4.504329in}{1.924255in}}%
\pgfpathlineto{\pgfqpoint{4.505403in}{1.923379in}}%
\pgfpathlineto{\pgfqpoint{4.506477in}{1.927758in}}%
\pgfpathlineto{\pgfqpoint{4.507551in}{1.922795in}}%
\pgfpathlineto{\pgfqpoint{4.508625in}{1.934471in}}%
\pgfpathlineto{\pgfqpoint{4.511846in}{1.935639in}}%
\pgfpathlineto{\pgfqpoint{4.513994in}{1.948775in}}%
\pgfpathlineto{\pgfqpoint{4.515068in}{1.920168in}}%
\pgfpathlineto{\pgfqpoint{4.519363in}{1.916082in}}%
\pgfpathlineto{\pgfqpoint{4.520437in}{1.903530in}}%
\pgfpathlineto{\pgfqpoint{4.521511in}{1.902654in}}%
\pgfpathlineto{\pgfqpoint{4.523658in}{1.927758in}}%
\pgfpathlineto{\pgfqpoint{4.526880in}{1.928925in}}%
\pgfpathlineto{\pgfqpoint{4.527954in}{1.932720in}}%
\pgfpathlineto{\pgfqpoint{4.529028in}{1.934471in}}%
\pgfpathlineto{\pgfqpoint{4.530101in}{1.921628in}}%
\pgfpathlineto{\pgfqpoint{4.531175in}{1.918417in}}%
\pgfpathlineto{\pgfqpoint{4.534397in}{1.923379in}}%
\pgfpathlineto{\pgfqpoint{4.536544in}{1.912287in}}%
\pgfpathlineto{\pgfqpoint{4.537618in}{1.915790in}}%
\pgfpathlineto{\pgfqpoint{4.538692in}{1.921336in}}%
\pgfpathlineto{\pgfqpoint{4.541914in}{1.914330in}}%
\pgfpathlineto{\pgfqpoint{4.542987in}{1.915498in}}%
\pgfpathlineto{\pgfqpoint{4.545135in}{1.911119in}}%
\pgfpathlineto{\pgfqpoint{4.546209in}{1.909952in}}%
\pgfpathlineto{\pgfqpoint{4.549430in}{1.902070in}}%
\pgfpathlineto{\pgfqpoint{4.550504in}{1.884848in}}%
\pgfpathlineto{\pgfqpoint{4.552652in}{1.871129in}}%
\pgfpathlineto{\pgfqpoint{4.553726in}{1.890978in}}%
\pgfpathlineto{\pgfqpoint{4.558021in}{1.880178in}}%
\pgfpathlineto{\pgfqpoint{4.559095in}{1.889227in}}%
\pgfpathlineto{\pgfqpoint{4.560169in}{1.882221in}}%
\pgfpathlineto{\pgfqpoint{4.566612in}{1.904405in}}%
\pgfpathlineto{\pgfqpoint{4.567686in}{1.911703in}}%
\pgfpathlineto{\pgfqpoint{4.568760in}{1.902654in}}%
\pgfpathlineto{\pgfqpoint{4.573055in}{1.913746in}}%
\pgfpathlineto{\pgfqpoint{4.575203in}{1.942353in}}%
\pgfpathlineto{\pgfqpoint{4.576276in}{1.919876in}}%
\pgfpathlineto{\pgfqpoint{4.579498in}{1.921044in}}%
\pgfpathlineto{\pgfqpoint{4.582719in}{1.940309in}}%
\pgfpathlineto{\pgfqpoint{4.583793in}{1.943520in}}%
\pgfpathlineto{\pgfqpoint{4.587015in}{1.944396in}}%
\pgfpathlineto{\pgfqpoint{4.589162in}{1.942353in}}%
\pgfpathlineto{\pgfqpoint{4.590236in}{1.946147in}}%
\pgfpathlineto{\pgfqpoint{4.591310in}{1.942061in}}%
\pgfpathlineto{\pgfqpoint{4.595605in}{1.948191in}}%
\pgfpathlineto{\pgfqpoint{4.596679in}{1.957240in}}%
\pgfpathlineto{\pgfqpoint{4.597753in}{1.977965in}}%
\pgfpathlineto{\pgfqpoint{4.598827in}{1.984678in}}%
\pgfpathlineto{\pgfqpoint{4.602049in}{1.974170in}}%
\pgfpathlineto{\pgfqpoint{4.606344in}{1.932136in}}%
\pgfpathlineto{\pgfqpoint{4.609565in}{1.929509in}}%
\pgfpathlineto{\pgfqpoint{4.610639in}{1.920460in}}%
\pgfpathlineto{\pgfqpoint{4.611713in}{1.934471in}}%
\pgfpathlineto{\pgfqpoint{4.612787in}{1.957240in}}%
\pgfpathlineto{\pgfqpoint{4.613861in}{1.943812in}}%
\pgfpathlineto{\pgfqpoint{4.617082in}{1.935347in}}%
\pgfpathlineto{\pgfqpoint{4.618156in}{1.920460in}}%
\pgfpathlineto{\pgfqpoint{4.619230in}{1.923963in}}%
\pgfpathlineto{\pgfqpoint{4.620304in}{1.924255in}}%
\pgfpathlineto{\pgfqpoint{4.621378in}{1.934471in}}%
\pgfpathlineto{\pgfqpoint{4.626747in}{1.932720in}}%
\pgfpathlineto{\pgfqpoint{4.627821in}{1.943520in}}%
\pgfpathlineto{\pgfqpoint{4.628894in}{1.930093in}}%
\pgfpathlineto{\pgfqpoint{4.632116in}{1.923087in}}%
\pgfpathlineto{\pgfqpoint{4.633190in}{1.925422in}}%
\pgfpathlineto{\pgfqpoint{4.635338in}{1.888935in}}%
\pgfpathlineto{\pgfqpoint{4.636411in}{1.889518in}}%
\pgfpathlineto{\pgfqpoint{4.640707in}{1.887475in}}%
\pgfpathlineto{\pgfqpoint{4.641781in}{1.881053in}}%
\pgfpathlineto{\pgfqpoint{4.643928in}{1.862955in}}%
\pgfpathlineto{\pgfqpoint{4.647150in}{1.893313in}}%
\pgfpathlineto{\pgfqpoint{4.648224in}{1.893313in}}%
\pgfpathlineto{\pgfqpoint{4.651445in}{1.913746in}}%
\pgfpathlineto{\pgfqpoint{4.654667in}{1.903822in}}%
\pgfpathlineto{\pgfqpoint{4.655740in}{1.896816in}}%
\pgfpathlineto{\pgfqpoint{4.657888in}{1.919001in}}%
\pgfpathlineto{\pgfqpoint{4.658962in}{1.921336in}}%
\pgfpathlineto{\pgfqpoint{4.662183in}{1.921044in}}%
\pgfpathlineto{\pgfqpoint{4.663257in}{1.911703in}}%
\pgfpathlineto{\pgfqpoint{4.665405in}{1.928633in}}%
\pgfpathlineto{\pgfqpoint{4.666479in}{1.927758in}}%
\pgfpathlineto{\pgfqpoint{4.669700in}{1.919584in}}%
\pgfpathlineto{\pgfqpoint{4.671848in}{1.935931in}}%
\pgfpathlineto{\pgfqpoint{4.673996in}{1.923379in}}%
\pgfpathlineto{\pgfqpoint{4.677217in}{1.921336in}}%
\pgfpathlineto{\pgfqpoint{4.678291in}{1.916082in}}%
\pgfpathlineto{\pgfqpoint{4.679365in}{1.905573in}}%
\pgfpathlineto{\pgfqpoint{4.680439in}{1.914914in}}%
\pgfpathlineto{\pgfqpoint{4.681513in}{1.909660in}}%
\pgfpathlineto{\pgfqpoint{4.684734in}{1.909076in}}%
\pgfpathlineto{\pgfqpoint{4.686882in}{1.907033in}}%
\pgfpathlineto{\pgfqpoint{4.687956in}{1.907616in}}%
\pgfpathlineto{\pgfqpoint{4.689029in}{1.895065in}}%
\pgfpathlineto{\pgfqpoint{4.692251in}{1.903238in}}%
\pgfpathlineto{\pgfqpoint{4.694399in}{1.924547in}}%
\pgfpathlineto{\pgfqpoint{4.695472in}{1.920168in}}%
\pgfpathlineto{\pgfqpoint{4.696546in}{1.925422in}}%
\pgfpathlineto{\pgfqpoint{4.699768in}{1.933596in}}%
\pgfpathlineto{\pgfqpoint{4.700842in}{1.910243in}}%
\pgfpathlineto{\pgfqpoint{4.701916in}{1.908200in}}%
\pgfpathlineto{\pgfqpoint{4.702989in}{1.921044in}}%
\pgfpathlineto{\pgfqpoint{4.704063in}{1.914622in}}%
\pgfpathlineto{\pgfqpoint{4.707285in}{1.899151in}}%
\pgfpathlineto{\pgfqpoint{4.709432in}{1.875507in}}%
\pgfpathlineto{\pgfqpoint{4.711580in}{1.897984in}}%
\pgfpathlineto{\pgfqpoint{4.715875in}{1.906741in}}%
\pgfpathlineto{\pgfqpoint{4.718023in}{1.899151in}}%
\pgfpathlineto{\pgfqpoint{4.719097in}{1.900903in}}%
\pgfpathlineto{\pgfqpoint{4.722318in}{1.903238in}}%
\pgfpathlineto{\pgfqpoint{4.723392in}{1.919001in}}%
\pgfpathlineto{\pgfqpoint{4.724466in}{1.921336in}}%
\pgfpathlineto{\pgfqpoint{4.725540in}{1.921336in}}%
\pgfpathlineto{\pgfqpoint{4.726614in}{1.942061in}}%
\pgfpathlineto{\pgfqpoint{4.729835in}{1.941185in}}%
\pgfpathlineto{\pgfqpoint{4.730909in}{1.942061in}}%
\pgfpathlineto{\pgfqpoint{4.731983in}{1.946147in}}%
\pgfpathlineto{\pgfqpoint{4.733057in}{1.940017in}}%
\pgfpathlineto{\pgfqpoint{4.734131in}{1.937682in}}%
\pgfpathlineto{\pgfqpoint{4.737352in}{1.936807in}}%
\pgfpathlineto{\pgfqpoint{4.739500in}{1.941185in}}%
\pgfpathlineto{\pgfqpoint{4.740574in}{1.937098in}}%
\pgfpathlineto{\pgfqpoint{4.741648in}{1.938558in}}%
\pgfpathlineto{\pgfqpoint{4.745943in}{1.940893in}}%
\pgfpathlineto{\pgfqpoint{4.748091in}{1.954321in}}%
\pgfpathlineto{\pgfqpoint{4.749164in}{1.960451in}}%
\pgfpathlineto{\pgfqpoint{4.752386in}{1.967456in}}%
\pgfpathlineto{\pgfqpoint{4.754534in}{1.981176in}}%
\pgfpathlineto{\pgfqpoint{4.756681in}{1.989349in}}%
\pgfpathlineto{\pgfqpoint{4.759903in}{1.997814in}}%
\pgfpathlineto{\pgfqpoint{4.760977in}{2.010366in}}%
\pgfpathlineto{\pgfqpoint{4.762050in}{2.001317in}}%
\pgfpathlineto{\pgfqpoint{4.763124in}{2.005695in}}%
\pgfpathlineto{\pgfqpoint{4.764198in}{2.016204in}}%
\pgfpathlineto{\pgfqpoint{4.767420in}{2.012993in}}%
\pgfpathlineto{\pgfqpoint{4.768493in}{2.024961in}}%
\pgfpathlineto{\pgfqpoint{4.769567in}{2.015912in}}%
\pgfpathlineto{\pgfqpoint{4.770641in}{2.027588in}}%
\pgfpathlineto{\pgfqpoint{4.771715in}{2.017663in}}%
\pgfpathlineto{\pgfqpoint{4.776010in}{2.037805in}}%
\pgfpathlineto{\pgfqpoint{4.777084in}{2.030215in}}%
\pgfpathlineto{\pgfqpoint{4.779232in}{2.039264in}}%
\pgfpathlineto{\pgfqpoint{4.782453in}{2.042475in}}%
\pgfpathlineto{\pgfqpoint{4.783527in}{2.049481in}}%
\pgfpathlineto{\pgfqpoint{4.784601in}{2.037513in}}%
\pgfpathlineto{\pgfqpoint{4.785675in}{2.042183in}}%
\pgfpathlineto{\pgfqpoint{4.789970in}{2.043643in}}%
\pgfpathlineto{\pgfqpoint{4.791044in}{2.049773in}}%
\pgfpathlineto{\pgfqpoint{4.792118in}{2.038680in}}%
\pgfpathlineto{\pgfqpoint{4.794266in}{2.051232in}}%
\pgfpathlineto{\pgfqpoint{4.797487in}{2.059697in}}%
\pgfpathlineto{\pgfqpoint{4.798561in}{2.057946in}}%
\pgfpathlineto{\pgfqpoint{4.801782in}{2.091223in}}%
\pgfpathlineto{\pgfqpoint{4.805004in}{2.086260in}}%
\pgfpathlineto{\pgfqpoint{4.806078in}{2.082174in}}%
\pgfpathlineto{\pgfqpoint{4.807152in}{2.062033in}}%
\pgfpathlineto{\pgfqpoint{4.809299in}{2.121580in}}%
\pgfpathlineto{\pgfqpoint{4.812521in}{2.120413in}}%
\pgfpathlineto{\pgfqpoint{4.813595in}{2.123332in}}%
\pgfpathlineto{\pgfqpoint{4.814669in}{2.111656in}}%
\pgfpathlineto{\pgfqpoint{4.815742in}{2.143181in}}%
\pgfpathlineto{\pgfqpoint{4.816816in}{2.152230in}}%
\pgfpathlineto{\pgfqpoint{4.820038in}{2.148435in}}%
\pgfpathlineto{\pgfqpoint{4.821112in}{2.158360in}}%
\pgfpathlineto{\pgfqpoint{4.822185in}{2.125083in}}%
\pgfpathlineto{\pgfqpoint{4.824333in}{2.131213in}}%
\pgfpathlineto{\pgfqpoint{4.827555in}{2.117202in}}%
\pgfpathlineto{\pgfqpoint{4.828628in}{2.136467in}}%
\pgfpathlineto{\pgfqpoint{4.829702in}{2.140554in}}%
\pgfpathlineto{\pgfqpoint{4.830776in}{2.131797in}}%
\pgfpathlineto{\pgfqpoint{4.831850in}{2.135884in}}%
\pgfpathlineto{\pgfqpoint{4.835071in}{2.128294in}}%
\pgfpathlineto{\pgfqpoint{4.837219in}{2.151354in}}%
\pgfpathlineto{\pgfqpoint{4.838293in}{2.142597in}}%
\pgfpathlineto{\pgfqpoint{4.839367in}{2.145808in}}%
\pgfpathlineto{\pgfqpoint{4.842588in}{2.132381in}}%
\pgfpathlineto{\pgfqpoint{4.844736in}{2.104650in}}%
\pgfpathlineto{\pgfqpoint{4.845810in}{2.113699in}}%
\pgfpathlineto{\pgfqpoint{4.846884in}{2.102899in}}%
\pgfpathlineto{\pgfqpoint{4.851179in}{2.090347in}}%
\pgfpathlineto{\pgfqpoint{4.852253in}{2.062616in}}%
\pgfpathlineto{\pgfqpoint{4.854401in}{2.038680in}}%
\pgfpathlineto{\pgfqpoint{4.857622in}{2.042767in}}%
\pgfpathlineto{\pgfqpoint{4.858696in}{2.046562in}}%
\pgfpathlineto{\pgfqpoint{4.859770in}{2.034594in}}%
\pgfpathlineto{\pgfqpoint{4.860844in}{2.071373in}}%
\pgfpathlineto{\pgfqpoint{4.861917in}{2.077503in}}%
\pgfpathlineto{\pgfqpoint{4.865139in}{2.083925in}}%
\pgfpathlineto{\pgfqpoint{4.867287in}{2.069914in}}%
\pgfpathlineto{\pgfqpoint{4.869434in}{2.095601in}}%
\pgfpathlineto{\pgfqpoint{4.872656in}{2.087720in}}%
\pgfpathlineto{\pgfqpoint{4.873730in}{2.106402in}}%
\pgfpathlineto{\pgfqpoint{4.875877in}{2.049189in}}%
\pgfpathlineto{\pgfqpoint{4.876951in}{2.061449in}}%
\pgfpathlineto{\pgfqpoint{4.880173in}{2.052692in}}%
\pgfpathlineto{\pgfqpoint{4.881247in}{2.081882in}}%
\pgfpathlineto{\pgfqpoint{4.882320in}{2.086552in}}%
\pgfpathlineto{\pgfqpoint{4.883394in}{2.094142in}}%
\pgfpathlineto{\pgfqpoint{4.884468in}{2.079547in}}%
\pgfpathlineto{\pgfqpoint{4.887690in}{2.080130in}}%
\pgfpathlineto{\pgfqpoint{4.891985in}{2.100855in}}%
\pgfpathlineto{\pgfqpoint{4.895206in}{2.109321in}}%
\pgfpathlineto{\pgfqpoint{4.896280in}{2.104650in}}%
\pgfpathlineto{\pgfqpoint{4.897354in}{2.095309in}}%
\pgfpathlineto{\pgfqpoint{4.898428in}{2.110196in}}%
\pgfpathlineto{\pgfqpoint{4.899502in}{2.092390in}}%
\pgfpathlineto{\pgfqpoint{4.902723in}{2.082466in}}%
\pgfpathlineto{\pgfqpoint{4.904871in}{2.099688in}}%
\pgfpathlineto{\pgfqpoint{4.905945in}{2.082758in}}%
\pgfpathlineto{\pgfqpoint{4.907019in}{2.082174in}}%
\pgfpathlineto{\pgfqpoint{4.911314in}{2.091223in}}%
\pgfpathlineto{\pgfqpoint{4.912388in}{2.091807in}}%
\pgfpathlineto{\pgfqpoint{4.914536in}{2.106110in}}%
\pgfpathlineto{\pgfqpoint{4.917757in}{2.116326in}}%
\pgfpathlineto{\pgfqpoint{4.919905in}{2.071665in}}%
\pgfpathlineto{\pgfqpoint{4.920979in}{2.083633in}}%
\pgfpathlineto{\pgfqpoint{4.922052in}{2.088887in}}%
\pgfpathlineto{\pgfqpoint{4.925274in}{2.088012in}}%
\pgfpathlineto{\pgfqpoint{4.927422in}{2.081882in}}%
\pgfpathlineto{\pgfqpoint{4.929569in}{2.068162in}}%
\pgfpathlineto{\pgfqpoint{4.932791in}{2.075460in}}%
\pgfpathlineto{\pgfqpoint{4.934938in}{2.058530in}}%
\pgfpathlineto{\pgfqpoint{4.936012in}{2.050940in}}%
\pgfpathlineto{\pgfqpoint{4.937086in}{2.032259in}}%
\pgfpathlineto{\pgfqpoint{4.940308in}{2.027588in}}%
\pgfpathlineto{\pgfqpoint{4.941381in}{2.036929in}}%
\pgfpathlineto{\pgfqpoint{4.942455in}{2.022626in}}%
\pgfpathlineto{\pgfqpoint{4.943529in}{2.017080in}}%
\pgfpathlineto{\pgfqpoint{4.944603in}{2.030215in}}%
\pgfpathlineto{\pgfqpoint{4.947825in}{2.015328in}}%
\pgfpathlineto{\pgfqpoint{4.948898in}{2.015328in}}%
\pgfpathlineto{\pgfqpoint{4.949972in}{2.006863in}}%
\pgfpathlineto{\pgfqpoint{4.951046in}{2.034886in}}%
\pgfpathlineto{\pgfqpoint{4.952120in}{2.024961in}}%
\pgfpathlineto{\pgfqpoint{4.956415in}{1.994603in}}%
\pgfpathlineto{\pgfqpoint{4.957489in}{2.011533in}}%
\pgfpathlineto{\pgfqpoint{4.958563in}{2.008614in}}%
\pgfpathlineto{\pgfqpoint{4.959637in}{2.001901in}}%
\pgfpathlineto{\pgfqpoint{4.962858in}{1.992268in}}%
\pgfpathlineto{\pgfqpoint{4.963932in}{2.004820in}}%
\pgfpathlineto{\pgfqpoint{4.965006in}{2.005987in}}%
\pgfpathlineto{\pgfqpoint{4.967154in}{2.033426in}}%
\pgfpathlineto{\pgfqpoint{4.970375in}{2.045394in}}%
\pgfpathlineto{\pgfqpoint{4.972523in}{2.054735in}}%
\pgfpathlineto{\pgfqpoint{4.973597in}{2.050065in}}%
\pgfpathlineto{\pgfqpoint{4.974670in}{2.034010in}}%
\pgfpathlineto{\pgfqpoint{4.977892in}{2.038388in}}%
\pgfpathlineto{\pgfqpoint{4.978966in}{2.022626in}}%
\pgfpathlineto{\pgfqpoint{4.980040in}{2.015328in}}%
\pgfpathlineto{\pgfqpoint{4.981114in}{2.031383in}}%
\pgfpathlineto{\pgfqpoint{4.982187in}{2.016496in}}%
\pgfpathlineto{\pgfqpoint{4.985409in}{2.009198in}}%
\pgfpathlineto{\pgfqpoint{4.986483in}{2.016204in}}%
\pgfpathlineto{\pgfqpoint{4.987557in}{2.011825in}}%
\pgfpathlineto{\pgfqpoint{4.988630in}{2.016788in}}%
\pgfpathlineto{\pgfqpoint{4.989704in}{2.018831in}}%
\pgfpathlineto{\pgfqpoint{4.992926in}{2.022918in}}%
\pgfpathlineto{\pgfqpoint{4.994000in}{2.007155in}}%
\pgfpathlineto{\pgfqpoint{4.995073in}{2.010366in}}%
\pgfpathlineto{\pgfqpoint{4.996147in}{2.025253in}}%
\pgfpathlineto{\pgfqpoint{4.997221in}{2.030507in}}%
\pgfpathlineto{\pgfqpoint{5.000443in}{2.024085in}}%
\pgfpathlineto{\pgfqpoint{5.001516in}{2.012993in}}%
\pgfpathlineto{\pgfqpoint{5.002590in}{2.034010in}}%
\pgfpathlineto{\pgfqpoint{5.004738in}{2.096185in}}%
\pgfpathlineto{\pgfqpoint{5.007959in}{2.109029in}}%
\pgfpathlineto{\pgfqpoint{5.009033in}{2.121580in}}%
\pgfpathlineto{\pgfqpoint{5.011181in}{2.105234in}}%
\pgfpathlineto{\pgfqpoint{5.012255in}{2.111948in}}%
\pgfpathlineto{\pgfqpoint{5.015476in}{2.108737in}}%
\pgfpathlineto{\pgfqpoint{5.016550in}{2.120413in}}%
\pgfpathlineto{\pgfqpoint{5.017624in}{2.107861in}}%
\pgfpathlineto{\pgfqpoint{5.019772in}{2.106985in}}%
\pgfpathlineto{\pgfqpoint{5.022993in}{2.120121in}}%
\pgfpathlineto{\pgfqpoint{5.024067in}{2.098228in}}%
\pgfpathlineto{\pgfqpoint{5.025141in}{2.109613in}}%
\pgfpathlineto{\pgfqpoint{5.026215in}{2.099688in}}%
\pgfpathlineto{\pgfqpoint{5.027289in}{2.100564in}}%
\pgfpathlineto{\pgfqpoint{5.030510in}{2.095017in}}%
\pgfpathlineto{\pgfqpoint{5.031584in}{2.099396in}}%
\pgfpathlineto{\pgfqpoint{5.032658in}{2.095601in}}%
\pgfpathlineto{\pgfqpoint{5.033732in}{2.102023in}}%
\pgfpathlineto{\pgfqpoint{5.034805in}{2.102899in}}%
\pgfpathlineto{\pgfqpoint{5.038027in}{2.113115in}}%
\pgfpathlineto{\pgfqpoint{5.039101in}{2.113407in}}%
\pgfpathlineto{\pgfqpoint{5.040175in}{2.104942in}}%
\pgfpathlineto{\pgfqpoint{5.041248in}{2.104358in}}%
\pgfpathlineto{\pgfqpoint{5.042322in}{2.100855in}}%
\pgfpathlineto{\pgfqpoint{5.045544in}{2.096185in}}%
\pgfpathlineto{\pgfqpoint{5.046618in}{2.097061in}}%
\pgfpathlineto{\pgfqpoint{5.047692in}{2.094434in}}%
\pgfpathlineto{\pgfqpoint{5.049839in}{2.087428in}}%
\pgfpathlineto{\pgfqpoint{5.053061in}{2.079255in}}%
\pgfpathlineto{\pgfqpoint{5.054135in}{2.086844in}}%
\pgfpathlineto{\pgfqpoint{5.055208in}{2.081882in}}%
\pgfpathlineto{\pgfqpoint{5.056282in}{2.071373in}}%
\pgfpathlineto{\pgfqpoint{5.057356in}{2.084217in}}%
\pgfpathlineto{\pgfqpoint{5.060578in}{2.086260in}}%
\pgfpathlineto{\pgfqpoint{5.063799in}{2.053859in}}%
\pgfpathlineto{\pgfqpoint{5.064873in}{2.047437in}}%
\pgfpathlineto{\pgfqpoint{5.068094in}{2.056778in}}%
\pgfpathlineto{\pgfqpoint{5.069168in}{2.041016in}}%
\pgfpathlineto{\pgfqpoint{5.070242in}{2.061449in}}%
\pgfpathlineto{\pgfqpoint{5.071316in}{2.060573in}}%
\pgfpathlineto{\pgfqpoint{5.072390in}{2.052692in}}%
\pgfpathlineto{\pgfqpoint{5.075611in}{2.064076in}}%
\pgfpathlineto{\pgfqpoint{5.076685in}{2.072833in}}%
\pgfpathlineto{\pgfqpoint{5.078833in}{2.077503in}}%
\pgfpathlineto{\pgfqpoint{5.079907in}{2.077211in}}%
\pgfpathlineto{\pgfqpoint{5.084202in}{2.076628in}}%
\pgfpathlineto{\pgfqpoint{5.086350in}{2.073417in}}%
\pgfpathlineto{\pgfqpoint{5.087424in}{2.060281in}}%
\pgfpathlineto{\pgfqpoint{5.090645in}{2.066411in}}%
\pgfpathlineto{\pgfqpoint{5.091719in}{2.080130in}}%
\pgfpathlineto{\pgfqpoint{5.092793in}{2.074000in}}%
\pgfpathlineto{\pgfqpoint{5.093867in}{2.051232in}}%
\pgfpathlineto{\pgfqpoint{5.094940in}{2.057070in}}%
\pgfpathlineto{\pgfqpoint{5.098162in}{2.040724in}}%
\pgfpathlineto{\pgfqpoint{5.099236in}{2.041599in}}%
\pgfpathlineto{\pgfqpoint{5.100310in}{2.068746in}}%
\pgfpathlineto{\pgfqpoint{5.101383in}{2.074584in}}%
\pgfpathlineto{\pgfqpoint{5.106753in}{2.055903in}}%
\pgfpathlineto{\pgfqpoint{5.107826in}{2.047437in}}%
\pgfpathlineto{\pgfqpoint{5.108900in}{2.059405in}}%
\pgfpathlineto{\pgfqpoint{5.109974in}{2.054151in}}%
\pgfpathlineto{\pgfqpoint{5.113196in}{2.055611in}}%
\pgfpathlineto{\pgfqpoint{5.114269in}{2.048021in}}%
\pgfpathlineto{\pgfqpoint{5.115343in}{2.055611in}}%
\pgfpathlineto{\pgfqpoint{5.116417in}{2.054151in}}%
\pgfpathlineto{\pgfqpoint{5.117491in}{2.063200in}}%
\pgfpathlineto{\pgfqpoint{5.120713in}{2.026420in}}%
\pgfpathlineto{\pgfqpoint{5.121786in}{2.035761in}}%
\pgfpathlineto{\pgfqpoint{5.122860in}{2.032842in}}%
\pgfpathlineto{\pgfqpoint{5.123934in}{2.032550in}}%
\pgfpathlineto{\pgfqpoint{5.125008in}{2.035469in}}%
\pgfpathlineto{\pgfqpoint{5.128229in}{2.037805in}}%
\pgfpathlineto{\pgfqpoint{5.130377in}{2.048313in}}%
\pgfpathlineto{\pgfqpoint{5.131451in}{2.046854in}}%
\pgfpathlineto{\pgfqpoint{5.132525in}{2.028756in}}%
\pgfpathlineto{\pgfqpoint{5.136820in}{2.016788in}}%
\pgfpathlineto{\pgfqpoint{5.137894in}{2.029339in}}%
\pgfpathlineto{\pgfqpoint{5.138968in}{2.065243in}}%
\pgfpathlineto{\pgfqpoint{5.140042in}{2.045978in}}%
\pgfpathlineto{\pgfqpoint{5.143263in}{2.022334in}}%
\pgfpathlineto{\pgfqpoint{5.145411in}{2.025253in}}%
\pgfpathlineto{\pgfqpoint{5.146485in}{2.051232in}}%
\pgfpathlineto{\pgfqpoint{5.147558in}{2.052984in}}%
\pgfpathlineto{\pgfqpoint{5.150780in}{2.046854in}}%
\pgfpathlineto{\pgfqpoint{5.151854in}{2.060281in}}%
\pgfpathlineto{\pgfqpoint{5.152928in}{2.048605in}}%
\pgfpathlineto{\pgfqpoint{5.154002in}{2.050356in}}%
\pgfpathlineto{\pgfqpoint{5.155075in}{2.043351in}}%
\pgfpathlineto{\pgfqpoint{5.158297in}{2.040724in}}%
\pgfpathlineto{\pgfqpoint{5.161518in}{2.017955in}}%
\pgfpathlineto{\pgfqpoint{5.162592in}{2.019123in}}%
\pgfpathlineto{\pgfqpoint{5.165814in}{2.024085in}}%
\pgfpathlineto{\pgfqpoint{5.166888in}{2.033134in}}%
\pgfpathlineto{\pgfqpoint{5.167961in}{2.025253in}}%
\pgfpathlineto{\pgfqpoint{5.169035in}{2.044810in}}%
\pgfpathlineto{\pgfqpoint{5.170109in}{2.037805in}}%
\pgfpathlineto{\pgfqpoint{5.173331in}{2.040140in}}%
\pgfpathlineto{\pgfqpoint{5.174404in}{2.044810in}}%
\pgfpathlineto{\pgfqpoint{5.175478in}{2.040140in}}%
\pgfpathlineto{\pgfqpoint{5.176552in}{2.055611in}}%
\pgfpathlineto{\pgfqpoint{5.177626in}{2.049481in}}%
\pgfpathlineto{\pgfqpoint{5.180847in}{2.052984in}}%
\pgfpathlineto{\pgfqpoint{5.181921in}{2.057070in}}%
\pgfpathlineto{\pgfqpoint{5.182995in}{2.058530in}}%
\pgfpathlineto{\pgfqpoint{5.184069in}{2.065243in}}%
\pgfpathlineto{\pgfqpoint{5.185143in}{2.063492in}}%
\pgfpathlineto{\pgfqpoint{5.188364in}{2.065243in}}%
\pgfpathlineto{\pgfqpoint{5.189438in}{2.079547in}}%
\pgfpathlineto{\pgfqpoint{5.190512in}{2.074292in}}%
\pgfpathlineto{\pgfqpoint{5.192660in}{2.053567in}}%
\pgfpathlineto{\pgfqpoint{5.195881in}{2.057946in}}%
\pgfpathlineto{\pgfqpoint{5.196955in}{2.050065in}}%
\pgfpathlineto{\pgfqpoint{5.198029in}{2.054151in}}%
\pgfpathlineto{\pgfqpoint{5.199103in}{2.065535in}}%
\pgfpathlineto{\pgfqpoint{5.203398in}{2.074292in}}%
\pgfpathlineto{\pgfqpoint{5.204472in}{2.072833in}}%
\pgfpathlineto{\pgfqpoint{5.205546in}{2.061741in}}%
\pgfpathlineto{\pgfqpoint{5.206620in}{2.035178in}}%
\pgfpathlineto{\pgfqpoint{5.207693in}{2.027296in}}%
\pgfpathlineto{\pgfqpoint{5.211989in}{2.046562in}}%
\pgfpathlineto{\pgfqpoint{5.213063in}{2.045686in}}%
\pgfpathlineto{\pgfqpoint{5.214136in}{2.056778in}}%
\pgfpathlineto{\pgfqpoint{5.215210in}{2.054443in}}%
\pgfpathlineto{\pgfqpoint{5.219506in}{2.062616in}}%
\pgfpathlineto{\pgfqpoint{5.221653in}{2.085968in}}%
\pgfpathlineto{\pgfqpoint{5.222727in}{2.085968in}}%
\pgfpathlineto{\pgfqpoint{5.225949in}{2.078379in}}%
\pgfpathlineto{\pgfqpoint{5.227023in}{2.071373in}}%
\pgfpathlineto{\pgfqpoint{5.228096in}{2.075168in}}%
\pgfpathlineto{\pgfqpoint{5.229170in}{2.074000in}}%
\pgfpathlineto{\pgfqpoint{5.230244in}{2.099396in}}%
\pgfpathlineto{\pgfqpoint{5.233466in}{2.101147in}}%
\pgfpathlineto{\pgfqpoint{5.234539in}{2.089763in}}%
\pgfpathlineto{\pgfqpoint{5.236687in}{2.108153in}}%
\pgfpathlineto{\pgfqpoint{5.237761in}{2.114867in}}%
\pgfpathlineto{\pgfqpoint{5.242056in}{2.111948in}}%
\pgfpathlineto{\pgfqpoint{5.243130in}{2.114867in}}%
\pgfpathlineto{\pgfqpoint{5.244204in}{2.114283in}}%
\pgfpathlineto{\pgfqpoint{5.245278in}{2.119829in}}%
\pgfpathlineto{\pgfqpoint{5.248499in}{2.121872in}}%
\pgfpathlineto{\pgfqpoint{5.249573in}{2.104358in}}%
\pgfpathlineto{\pgfqpoint{5.250647in}{2.101147in}}%
\pgfpathlineto{\pgfqpoint{5.252795in}{2.107569in}}%
\pgfpathlineto{\pgfqpoint{5.256016in}{2.110780in}}%
\pgfpathlineto{\pgfqpoint{5.257090in}{2.109904in}}%
\pgfpathlineto{\pgfqpoint{5.258164in}{2.106402in}}%
\pgfpathlineto{\pgfqpoint{5.259238in}{2.098812in}}%
\pgfpathlineto{\pgfqpoint{5.260312in}{2.102023in}}%
\pgfpathlineto{\pgfqpoint{5.263533in}{2.104942in}}%
\pgfpathlineto{\pgfqpoint{5.264607in}{2.102607in}}%
\pgfpathlineto{\pgfqpoint{5.265681in}{2.107569in}}%
\pgfpathlineto{\pgfqpoint{5.266755in}{2.108737in}}%
\pgfpathlineto{\pgfqpoint{5.267828in}{2.106694in}}%
\pgfpathlineto{\pgfqpoint{5.271050in}{2.114575in}}%
\pgfpathlineto{\pgfqpoint{5.272124in}{2.104358in}}%
\pgfpathlineto{\pgfqpoint{5.273198in}{2.107277in}}%
\pgfpathlineto{\pgfqpoint{5.274271in}{2.102607in}}%
\pgfpathlineto{\pgfqpoint{5.275345in}{2.105234in}}%
\pgfpathlineto{\pgfqpoint{5.278567in}{2.096185in}}%
\pgfpathlineto{\pgfqpoint{5.280714in}{2.113115in}}%
\pgfpathlineto{\pgfqpoint{5.281788in}{2.114283in}}%
\pgfpathlineto{\pgfqpoint{5.286084in}{2.115159in}}%
\pgfpathlineto{\pgfqpoint{5.287158in}{2.104358in}}%
\pgfpathlineto{\pgfqpoint{5.288231in}{2.107569in}}%
\pgfpathlineto{\pgfqpoint{5.290379in}{2.140554in}}%
\pgfpathlineto{\pgfqpoint{5.293601in}{2.145516in}}%
\pgfpathlineto{\pgfqpoint{5.295748in}{2.155441in}}%
\pgfpathlineto{\pgfqpoint{5.296822in}{2.140554in}}%
\pgfpathlineto{\pgfqpoint{5.297896in}{2.150771in}}%
\pgfpathlineto{\pgfqpoint{5.301117in}{2.149603in}}%
\pgfpathlineto{\pgfqpoint{5.302191in}{2.156025in}}%
\pgfpathlineto{\pgfqpoint{5.303265in}{2.154274in}}%
\pgfpathlineto{\pgfqpoint{5.304339in}{2.157484in}}%
\pgfpathlineto{\pgfqpoint{5.305413in}{2.163031in}}%
\pgfpathlineto{\pgfqpoint{5.308634in}{2.170036in}}%
\pgfpathlineto{\pgfqpoint{5.309708in}{2.179085in}}%
\pgfpathlineto{\pgfqpoint{5.310782in}{2.174123in}}%
\pgfpathlineto{\pgfqpoint{5.311856in}{2.142889in}}%
\pgfpathlineto{\pgfqpoint{5.312930in}{2.129170in}}%
\pgfpathlineto{\pgfqpoint{5.316151in}{2.138219in}}%
\pgfpathlineto{\pgfqpoint{5.319373in}{2.101731in}}%
\pgfpathlineto{\pgfqpoint{5.320446in}{2.102899in}}%
\pgfpathlineto{\pgfqpoint{5.323668in}{2.102315in}}%
\pgfpathlineto{\pgfqpoint{5.325816in}{2.108153in}}%
\pgfpathlineto{\pgfqpoint{5.326890in}{2.109904in}}%
\pgfpathlineto{\pgfqpoint{5.327963in}{2.105234in}}%
\pgfpathlineto{\pgfqpoint{5.331185in}{2.104942in}}%
\pgfpathlineto{\pgfqpoint{5.332259in}{2.102899in}}%
\pgfpathlineto{\pgfqpoint{5.333333in}{2.105818in}}%
\pgfpathlineto{\pgfqpoint{5.334406in}{2.106985in}}%
\pgfpathlineto{\pgfqpoint{5.335480in}{2.101439in}}%
\pgfpathlineto{\pgfqpoint{5.340849in}{2.119829in}}%
\pgfpathlineto{\pgfqpoint{5.341923in}{2.119537in}}%
\pgfpathlineto{\pgfqpoint{5.342997in}{2.128878in}}%
\pgfpathlineto{\pgfqpoint{5.347292in}{2.127710in}}%
\pgfpathlineto{\pgfqpoint{5.348366in}{2.130338in}}%
\pgfpathlineto{\pgfqpoint{5.349440in}{2.126543in}}%
\pgfpathlineto{\pgfqpoint{5.350514in}{2.131797in}}%
\pgfpathlineto{\pgfqpoint{5.353735in}{2.122164in}}%
\pgfpathlineto{\pgfqpoint{5.354809in}{2.107569in}}%
\pgfpathlineto{\pgfqpoint{5.355883in}{2.104066in}}%
\pgfpathlineto{\pgfqpoint{5.356957in}{2.110196in}}%
\pgfpathlineto{\pgfqpoint{5.358031in}{2.095893in}}%
\pgfpathlineto{\pgfqpoint{5.361252in}{2.099688in}}%
\pgfpathlineto{\pgfqpoint{5.365548in}{2.141430in}}%
\pgfpathlineto{\pgfqpoint{5.368769in}{2.137343in}}%
\pgfpathlineto{\pgfqpoint{5.369843in}{2.131213in}}%
\pgfpathlineto{\pgfqpoint{5.370917in}{2.134716in}}%
\pgfpathlineto{\pgfqpoint{5.371991in}{2.124208in}}%
\pgfpathlineto{\pgfqpoint{5.373065in}{2.127710in}}%
\pgfpathlineto{\pgfqpoint{5.376286in}{2.127419in}}%
\pgfpathlineto{\pgfqpoint{5.377360in}{2.132965in}}%
\pgfpathlineto{\pgfqpoint{5.378434in}{2.119829in}}%
\pgfpathlineto{\pgfqpoint{5.379508in}{2.116618in}}%
\pgfpathlineto{\pgfqpoint{5.380581in}{2.126543in}}%
\pgfpathlineto{\pgfqpoint{5.383803in}{2.135008in}}%
\pgfpathlineto{\pgfqpoint{5.384877in}{2.125667in}}%
\pgfpathlineto{\pgfqpoint{5.385951in}{2.143181in}}%
\pgfpathlineto{\pgfqpoint{5.387024in}{2.120997in}}%
\pgfpathlineto{\pgfqpoint{5.388098in}{2.121289in}}%
\pgfpathlineto{\pgfqpoint{5.393468in}{2.097353in}}%
\pgfpathlineto{\pgfqpoint{5.394541in}{2.091515in}}%
\pgfpathlineto{\pgfqpoint{5.395615in}{2.100855in}}%
\pgfpathlineto{\pgfqpoint{5.399911in}{2.115451in}}%
\pgfpathlineto{\pgfqpoint{5.400984in}{2.106402in}}%
\pgfpathlineto{\pgfqpoint{5.402058in}{2.104358in}}%
\pgfpathlineto{\pgfqpoint{5.403132in}{2.117202in}}%
\pgfpathlineto{\pgfqpoint{5.406354in}{2.132673in}}%
\pgfpathlineto{\pgfqpoint{5.407427in}{2.145225in}}%
\pgfpathlineto{\pgfqpoint{5.408501in}{2.142306in}}%
\pgfpathlineto{\pgfqpoint{5.409575in}{2.143765in}}%
\pgfpathlineto{\pgfqpoint{5.410649in}{2.151938in}}%
\pgfpathlineto{\pgfqpoint{5.413870in}{2.155441in}}%
\pgfpathlineto{\pgfqpoint{5.414944in}{2.153982in}}%
\pgfpathlineto{\pgfqpoint{5.416018in}{2.154274in}}%
\pgfpathlineto{\pgfqpoint{5.417092in}{2.152814in}}%
\pgfpathlineto{\pgfqpoint{5.418166in}{2.166533in}}%
\pgfpathlineto{\pgfqpoint{5.421387in}{2.163031in}}%
\pgfpathlineto{\pgfqpoint{5.422461in}{2.160403in}}%
\pgfpathlineto{\pgfqpoint{5.423535in}{2.165366in}}%
\pgfpathlineto{\pgfqpoint{5.425683in}{2.181420in}}%
\pgfpathlineto{\pgfqpoint{5.428904in}{2.179085in}}%
\pgfpathlineto{\pgfqpoint{5.429978in}{2.174999in}}%
\pgfpathlineto{\pgfqpoint{5.431052in}{2.157776in}}%
\pgfpathlineto{\pgfqpoint{5.432126in}{2.150771in}}%
\pgfpathlineto{\pgfqpoint{5.433200in}{2.151063in}}%
\pgfpathlineto{\pgfqpoint{5.437495in}{2.130629in}}%
\pgfpathlineto{\pgfqpoint{5.438569in}{2.147268in}}%
\pgfpathlineto{\pgfqpoint{5.440716in}{2.160112in}}%
\pgfpathlineto{\pgfqpoint{5.443938in}{2.146976in}}%
\pgfpathlineto{\pgfqpoint{5.445012in}{2.125375in}}%
\pgfpathlineto{\pgfqpoint{5.446086in}{2.117786in}}%
\pgfpathlineto{\pgfqpoint{5.447159in}{2.117494in}}%
\pgfpathlineto{\pgfqpoint{5.448233in}{2.113407in}}%
\pgfpathlineto{\pgfqpoint{5.451455in}{2.120413in}}%
\pgfpathlineto{\pgfqpoint{5.452529in}{2.073709in}}%
\pgfpathlineto{\pgfqpoint{5.453602in}{2.056486in}}%
\pgfpathlineto{\pgfqpoint{5.454676in}{2.060573in}}%
\pgfpathlineto{\pgfqpoint{5.455750in}{2.042183in}}%
\pgfpathlineto{\pgfqpoint{5.458972in}{2.038680in}}%
\pgfpathlineto{\pgfqpoint{5.460046in}{2.041016in}}%
\pgfpathlineto{\pgfqpoint{5.462193in}{2.076920in}}%
\pgfpathlineto{\pgfqpoint{5.463267in}{2.076044in}}%
\pgfpathlineto{\pgfqpoint{5.467562in}{2.091515in}}%
\pgfpathlineto{\pgfqpoint{5.468636in}{2.091515in}}%
\pgfpathlineto{\pgfqpoint{5.470784in}{2.095893in}}%
\pgfpathlineto{\pgfqpoint{5.474005in}{2.088596in}}%
\pgfpathlineto{\pgfqpoint{5.475079in}{2.083341in}}%
\pgfpathlineto{\pgfqpoint{5.476153in}{2.070498in}}%
\pgfpathlineto{\pgfqpoint{5.478301in}{2.074876in}}%
\pgfpathlineto{\pgfqpoint{5.481522in}{2.065535in}}%
\pgfpathlineto{\pgfqpoint{5.482596in}{2.076628in}}%
\pgfpathlineto{\pgfqpoint{5.483670in}{2.069622in}}%
\pgfpathlineto{\pgfqpoint{5.484744in}{2.092974in}}%
\pgfpathlineto{\pgfqpoint{5.485818in}{2.083049in}}%
\pgfpathlineto{\pgfqpoint{5.490113in}{2.092974in}}%
\pgfpathlineto{\pgfqpoint{5.491187in}{2.088012in}}%
\pgfpathlineto{\pgfqpoint{5.492261in}{2.091223in}}%
\pgfpathlineto{\pgfqpoint{5.493334in}{2.112240in}}%
\pgfpathlineto{\pgfqpoint{5.498704in}{2.118370in}}%
\pgfpathlineto{\pgfqpoint{5.500851in}{2.092390in}}%
\pgfpathlineto{\pgfqpoint{5.504073in}{2.087720in}}%
\pgfpathlineto{\pgfqpoint{5.506221in}{2.066119in}}%
\pgfpathlineto{\pgfqpoint{5.507294in}{2.067579in}}%
\pgfpathlineto{\pgfqpoint{5.508368in}{2.057946in}}%
\pgfpathlineto{\pgfqpoint{5.511590in}{2.088304in}}%
\pgfpathlineto{\pgfqpoint{5.512664in}{2.108445in}}%
\pgfpathlineto{\pgfqpoint{5.513737in}{2.107861in}}%
\pgfpathlineto{\pgfqpoint{5.514811in}{2.109321in}}%
\pgfpathlineto{\pgfqpoint{5.515885in}{2.144057in}}%
\pgfpathlineto{\pgfqpoint{5.519107in}{2.138511in}}%
\pgfpathlineto{\pgfqpoint{5.521254in}{2.155733in}}%
\pgfpathlineto{\pgfqpoint{5.523402in}{2.143765in}}%
\pgfpathlineto{\pgfqpoint{5.527697in}{2.140554in}}%
\pgfpathlineto{\pgfqpoint{5.528771in}{2.135008in}}%
\pgfpathlineto{\pgfqpoint{5.529845in}{2.134132in}}%
\pgfpathlineto{\pgfqpoint{5.530919in}{2.135592in}}%
\pgfpathlineto{\pgfqpoint{5.534140in}{2.130921in}}%
\pgfpathlineto{\pgfqpoint{5.535214in}{2.141430in}}%
\pgfpathlineto{\pgfqpoint{5.536288in}{2.141138in}}%
\pgfpathlineto{\pgfqpoint{5.537362in}{2.145225in}}%
\pgfpathlineto{\pgfqpoint{5.538436in}{2.146976in}}%
\pgfpathlineto{\pgfqpoint{5.541657in}{2.147560in}}%
\pgfpathlineto{\pgfqpoint{5.542731in}{2.149311in}}%
\pgfpathlineto{\pgfqpoint{5.543805in}{2.137927in}}%
\pgfpathlineto{\pgfqpoint{5.544879in}{2.134424in}}%
\pgfpathlineto{\pgfqpoint{5.545953in}{2.119537in}}%
\pgfpathlineto{\pgfqpoint{5.549174in}{2.118078in}}%
\pgfpathlineto{\pgfqpoint{5.550248in}{2.100855in}}%
\pgfpathlineto{\pgfqpoint{5.551322in}{2.104942in}}%
\pgfpathlineto{\pgfqpoint{5.552396in}{2.130046in}}%
\pgfpathlineto{\pgfqpoint{5.553469in}{2.132381in}}%
\pgfpathlineto{\pgfqpoint{5.556691in}{2.142889in}}%
\pgfpathlineto{\pgfqpoint{5.557765in}{2.135008in}}%
\pgfpathlineto{\pgfqpoint{5.558839in}{2.149311in}}%
\pgfpathlineto{\pgfqpoint{5.559912in}{2.143473in}}%
\pgfpathlineto{\pgfqpoint{5.560986in}{2.149603in}}%
\pgfpathlineto{\pgfqpoint{5.564208in}{2.151646in}}%
\pgfpathlineto{\pgfqpoint{5.565282in}{2.146100in}}%
\pgfpathlineto{\pgfqpoint{5.566356in}{2.130046in}}%
\pgfpathlineto{\pgfqpoint{5.567429in}{2.122456in}}%
\pgfpathlineto{\pgfqpoint{5.568503in}{2.125959in}}%
\pgfpathlineto{\pgfqpoint{5.571725in}{2.139095in}}%
\pgfpathlineto{\pgfqpoint{5.572799in}{2.127419in}}%
\pgfpathlineto{\pgfqpoint{5.573872in}{2.134424in}}%
\pgfpathlineto{\pgfqpoint{5.574946in}{2.147560in}}%
\pgfpathlineto{\pgfqpoint{5.579242in}{2.151354in}}%
\pgfpathlineto{\pgfqpoint{5.580315in}{2.142889in}}%
\pgfpathlineto{\pgfqpoint{5.581389in}{2.152522in}}%
\pgfpathlineto{\pgfqpoint{5.582463in}{2.149603in}}%
\pgfpathlineto{\pgfqpoint{5.583537in}{2.154857in}}%
\pgfpathlineto{\pgfqpoint{5.586758in}{2.150479in}}%
\pgfpathlineto{\pgfqpoint{5.587832in}{2.153690in}}%
\pgfpathlineto{\pgfqpoint{5.588906in}{2.158944in}}%
\pgfpathlineto{\pgfqpoint{5.589980in}{2.156025in}}%
\pgfpathlineto{\pgfqpoint{5.591054in}{2.146976in}}%
\pgfpathlineto{\pgfqpoint{5.594275in}{2.158652in}}%
\pgfpathlineto{\pgfqpoint{5.595349in}{2.153690in}}%
\pgfpathlineto{\pgfqpoint{5.597497in}{2.174707in}}%
\pgfpathlineto{\pgfqpoint{5.598571in}{2.174123in}}%
\pgfpathlineto{\pgfqpoint{5.601792in}{2.175582in}}%
\pgfpathlineto{\pgfqpoint{5.602866in}{2.186675in}}%
\pgfpathlineto{\pgfqpoint{5.605014in}{2.184048in}}%
\pgfpathlineto{\pgfqpoint{5.606088in}{2.183464in}}%
\pgfpathlineto{\pgfqpoint{5.609309in}{2.186091in}}%
\pgfpathlineto{\pgfqpoint{5.611457in}{2.164782in}}%
\pgfpathlineto{\pgfqpoint{5.612531in}{2.167117in}}%
\pgfpathlineto{\pgfqpoint{5.613604in}{2.177042in}}%
\pgfpathlineto{\pgfqpoint{5.616826in}{2.168869in}}%
\pgfpathlineto{\pgfqpoint{5.617900in}{2.164490in}}%
\pgfpathlineto{\pgfqpoint{5.618974in}{2.167117in}}%
\pgfpathlineto{\pgfqpoint{5.620047in}{2.172955in}}%
\pgfpathlineto{\pgfqpoint{5.621121in}{2.168577in}}%
\pgfpathlineto{\pgfqpoint{5.625417in}{2.162739in}}%
\pgfpathlineto{\pgfqpoint{5.626490in}{2.165950in}}%
\pgfpathlineto{\pgfqpoint{5.627564in}{2.171204in}}%
\pgfpathlineto{\pgfqpoint{5.628638in}{2.164198in}}%
\pgfpathlineto{\pgfqpoint{5.632934in}{2.159528in}}%
\pgfpathlineto{\pgfqpoint{5.634007in}{2.163322in}}%
\pgfpathlineto{\pgfqpoint{5.635081in}{2.162447in}}%
\pgfpathlineto{\pgfqpoint{5.636155in}{2.160112in}}%
\pgfpathlineto{\pgfqpoint{5.641524in}{2.151354in}}%
\pgfpathlineto{\pgfqpoint{5.643672in}{2.106985in}}%
\pgfpathlineto{\pgfqpoint{5.646893in}{2.111948in}}%
\pgfpathlineto{\pgfqpoint{5.647967in}{2.109613in}}%
\pgfpathlineto{\pgfqpoint{5.649041in}{2.112823in}}%
\pgfpathlineto{\pgfqpoint{5.650115in}{2.119537in}}%
\pgfpathlineto{\pgfqpoint{5.651189in}{2.107277in}}%
\pgfpathlineto{\pgfqpoint{5.654410in}{2.101439in}}%
\pgfpathlineto{\pgfqpoint{5.655484in}{2.111364in}}%
\pgfpathlineto{\pgfqpoint{5.656558in}{2.107861in}}%
\pgfpathlineto{\pgfqpoint{5.657632in}{2.119829in}}%
\pgfpathlineto{\pgfqpoint{5.658706in}{2.112532in}}%
\pgfpathlineto{\pgfqpoint{5.661927in}{2.113991in}}%
\pgfpathlineto{\pgfqpoint{5.663001in}{2.119829in}}%
\pgfpathlineto{\pgfqpoint{5.664075in}{2.108445in}}%
\pgfpathlineto{\pgfqpoint{5.666222in}{2.116326in}}%
\pgfpathlineto{\pgfqpoint{5.670518in}{2.092098in}}%
\pgfpathlineto{\pgfqpoint{5.672666in}{2.106110in}}%
\pgfpathlineto{\pgfqpoint{5.676961in}{2.102315in}}%
\pgfpathlineto{\pgfqpoint{5.678035in}{2.107277in}}%
\pgfpathlineto{\pgfqpoint{5.679109in}{2.104358in}}%
\pgfpathlineto{\pgfqpoint{5.680182in}{2.096769in}}%
\pgfpathlineto{\pgfqpoint{5.681256in}{2.114575in}}%
\pgfpathlineto{\pgfqpoint{5.684478in}{2.119537in}}%
\pgfpathlineto{\pgfqpoint{5.685552in}{2.124791in}}%
\pgfpathlineto{\pgfqpoint{5.686625in}{2.122748in}}%
\pgfpathlineto{\pgfqpoint{5.687699in}{2.135008in}}%
\pgfpathlineto{\pgfqpoint{5.688773in}{2.129170in}}%
\pgfpathlineto{\pgfqpoint{5.691995in}{2.141430in}}%
\pgfpathlineto{\pgfqpoint{5.693068in}{2.113991in}}%
\pgfpathlineto{\pgfqpoint{5.694142in}{2.101439in}}%
\pgfpathlineto{\pgfqpoint{5.695216in}{2.098812in}}%
\pgfpathlineto{\pgfqpoint{5.696290in}{2.091223in}}%
\pgfpathlineto{\pgfqpoint{5.699511in}{2.086260in}}%
\pgfpathlineto{\pgfqpoint{5.700585in}{2.087720in}}%
\pgfpathlineto{\pgfqpoint{5.701659in}{2.104066in}}%
\pgfpathlineto{\pgfqpoint{5.703807in}{2.109613in}}%
\pgfpathlineto{\pgfqpoint{5.707028in}{2.113991in}}%
\pgfpathlineto{\pgfqpoint{5.708102in}{2.106694in}}%
\pgfpathlineto{\pgfqpoint{5.709176in}{2.105818in}}%
\pgfpathlineto{\pgfqpoint{5.710250in}{2.105818in}}%
\pgfpathlineto{\pgfqpoint{5.711324in}{2.099104in}}%
\pgfpathlineto{\pgfqpoint{5.716693in}{2.135300in}}%
\pgfpathlineto{\pgfqpoint{5.718841in}{2.126543in}}%
\pgfpathlineto{\pgfqpoint{5.722062in}{2.127419in}}%
\pgfpathlineto{\pgfqpoint{5.724210in}{2.125375in}}%
\pgfpathlineto{\pgfqpoint{5.729579in}{2.059697in}}%
\pgfpathlineto{\pgfqpoint{5.730653in}{2.029631in}}%
\pgfpathlineto{\pgfqpoint{5.732800in}{2.095017in}}%
\pgfpathlineto{\pgfqpoint{5.733874in}{2.092098in}}%
\pgfpathlineto{\pgfqpoint{5.737096in}{2.090639in}}%
\pgfpathlineto{\pgfqpoint{5.738170in}{2.063492in}}%
\pgfpathlineto{\pgfqpoint{5.740317in}{2.083633in}}%
\pgfpathlineto{\pgfqpoint{5.741391in}{2.061741in}}%
\pgfpathlineto{\pgfqpoint{5.745687in}{2.087428in}}%
\pgfpathlineto{\pgfqpoint{5.746760in}{2.075460in}}%
\pgfpathlineto{\pgfqpoint{5.747834in}{2.077211in}}%
\pgfpathlineto{\pgfqpoint{5.748908in}{2.083925in}}%
\pgfpathlineto{\pgfqpoint{5.752130in}{2.081882in}}%
\pgfpathlineto{\pgfqpoint{5.753203in}{2.099396in}}%
\pgfpathlineto{\pgfqpoint{5.754277in}{2.095017in}}%
\pgfpathlineto{\pgfqpoint{5.756425in}{2.055611in}}%
\pgfpathlineto{\pgfqpoint{5.759646in}{2.061157in}}%
\pgfpathlineto{\pgfqpoint{5.761794in}{2.041307in}}%
\pgfpathlineto{\pgfqpoint{5.763942in}{2.047146in}}%
\pgfpathlineto{\pgfqpoint{5.769311in}{2.029631in}}%
\pgfpathlineto{\pgfqpoint{5.770385in}{2.016496in}}%
\pgfpathlineto{\pgfqpoint{5.771459in}{2.013577in}}%
\pgfpathlineto{\pgfqpoint{5.774680in}{2.041599in}}%
\pgfpathlineto{\pgfqpoint{5.775754in}{2.043059in}}%
\pgfpathlineto{\pgfqpoint{5.777902in}{2.061157in}}%
\pgfpathlineto{\pgfqpoint{5.778976in}{2.059405in}}%
\pgfpathlineto{\pgfqpoint{5.783271in}{2.064368in}}%
\pgfpathlineto{\pgfqpoint{5.784345in}{2.055319in}}%
\pgfpathlineto{\pgfqpoint{5.785419in}{2.072249in}}%
\pgfpathlineto{\pgfqpoint{5.786492in}{2.072833in}}%
\pgfpathlineto{\pgfqpoint{5.789714in}{2.072833in}}%
\pgfpathlineto{\pgfqpoint{5.790788in}{2.086260in}}%
\pgfpathlineto{\pgfqpoint{5.791862in}{2.076920in}}%
\pgfpathlineto{\pgfqpoint{5.792935in}{2.102315in}}%
\pgfpathlineto{\pgfqpoint{5.794009in}{2.108737in}}%
\pgfpathlineto{\pgfqpoint{5.797231in}{2.113699in}}%
\pgfpathlineto{\pgfqpoint{5.798305in}{2.109029in}}%
\pgfpathlineto{\pgfqpoint{5.799378in}{2.116618in}}%
\pgfpathlineto{\pgfqpoint{5.800452in}{2.114867in}}%
\pgfpathlineto{\pgfqpoint{5.801526in}{2.126543in}}%
\pgfpathlineto{\pgfqpoint{5.804748in}{2.124208in}}%
\pgfpathlineto{\pgfqpoint{5.806895in}{2.108445in}}%
\pgfpathlineto{\pgfqpoint{5.807969in}{2.109904in}}%
\pgfpathlineto{\pgfqpoint{5.809043in}{2.099396in}}%
\pgfpathlineto{\pgfqpoint{5.813338in}{2.082758in}}%
\pgfpathlineto{\pgfqpoint{5.814412in}{2.088012in}}%
\pgfpathlineto{\pgfqpoint{5.816560in}{2.061157in}}%
\pgfpathlineto{\pgfqpoint{5.819781in}{2.081298in}}%
\pgfpathlineto{\pgfqpoint{5.820855in}{2.082174in}}%
\pgfpathlineto{\pgfqpoint{5.823003in}{2.099104in}}%
\pgfpathlineto{\pgfqpoint{5.824077in}{2.089763in}}%
\pgfpathlineto{\pgfqpoint{5.827298in}{2.080130in}}%
\pgfpathlineto{\pgfqpoint{5.828372in}{2.084801in}}%
\pgfpathlineto{\pgfqpoint{5.829446in}{2.078379in}}%
\pgfpathlineto{\pgfqpoint{5.831594in}{2.085968in}}%
\pgfpathlineto{\pgfqpoint{5.834815in}{2.091223in}}%
\pgfpathlineto{\pgfqpoint{5.835889in}{2.094434in}}%
\pgfpathlineto{\pgfqpoint{5.838037in}{2.069330in}}%
\pgfpathlineto{\pgfqpoint{5.839111in}{2.097645in}}%
\pgfpathlineto{\pgfqpoint{5.842332in}{2.106402in}}%
\pgfpathlineto{\pgfqpoint{5.844480in}{2.089179in}}%
\pgfpathlineto{\pgfqpoint{5.845554in}{2.088012in}}%
\pgfpathlineto{\pgfqpoint{5.846627in}{2.075752in}}%
\pgfpathlineto{\pgfqpoint{5.850923in}{2.093850in}}%
\pgfpathlineto{\pgfqpoint{5.851997in}{2.117494in}}%
\pgfpathlineto{\pgfqpoint{5.854144in}{2.094142in}}%
\pgfpathlineto{\pgfqpoint{5.857366in}{2.102315in}}%
\pgfpathlineto{\pgfqpoint{5.859513in}{2.128294in}}%
\pgfpathlineto{\pgfqpoint{5.860587in}{2.122456in}}%
\pgfpathlineto{\pgfqpoint{5.864883in}{2.123332in}}%
\pgfpathlineto{\pgfqpoint{5.865956in}{2.134716in}}%
\pgfpathlineto{\pgfqpoint{5.868104in}{2.110196in}}%
\pgfpathlineto{\pgfqpoint{5.872399in}{2.101731in}}%
\pgfpathlineto{\pgfqpoint{5.873473in}{2.117202in}}%
\pgfpathlineto{\pgfqpoint{5.876695in}{2.089763in}}%
\pgfpathlineto{\pgfqpoint{5.879916in}{2.096185in}}%
\pgfpathlineto{\pgfqpoint{5.880990in}{2.092098in}}%
\pgfpathlineto{\pgfqpoint{5.882064in}{2.072541in}}%
\pgfpathlineto{\pgfqpoint{5.883138in}{2.090639in}}%
\pgfpathlineto{\pgfqpoint{5.884212in}{2.079547in}}%
\pgfpathlineto{\pgfqpoint{5.888507in}{2.090639in}}%
\pgfpathlineto{\pgfqpoint{5.889581in}{2.079547in}}%
\pgfpathlineto{\pgfqpoint{5.891729in}{2.144933in}}%
\pgfpathlineto{\pgfqpoint{5.894950in}{2.144641in}}%
\pgfpathlineto{\pgfqpoint{5.897098in}{2.194556in}}%
\pgfpathlineto{\pgfqpoint{5.898172in}{2.193972in}}%
\pgfpathlineto{\pgfqpoint{5.899245in}{2.217908in}}%
\pgfpathlineto{\pgfqpoint{5.902467in}{2.237757in}}%
\pgfpathlineto{\pgfqpoint{5.903541in}{2.216449in}}%
\pgfpathlineto{\pgfqpoint{5.904615in}{2.234255in}}%
\pgfpathlineto{\pgfqpoint{5.905688in}{2.229292in}}%
\pgfpathlineto{\pgfqpoint{5.906762in}{2.243012in}}%
\pgfpathlineto{\pgfqpoint{5.909984in}{2.237174in}}%
\pgfpathlineto{\pgfqpoint{5.911058in}{2.222287in}}%
\pgfpathlineto{\pgfqpoint{5.912132in}{2.218200in}}%
\pgfpathlineto{\pgfqpoint{5.913205in}{2.203313in}}%
\pgfpathlineto{\pgfqpoint{5.914279in}{2.221411in}}%
\pgfpathlineto{\pgfqpoint{5.918575in}{2.224622in}}%
\pgfpathlineto{\pgfqpoint{5.919648in}{2.226665in}}%
\pgfpathlineto{\pgfqpoint{5.920722in}{2.242136in}}%
\pgfpathlineto{\pgfqpoint{5.921796in}{2.240093in}}%
\pgfpathlineto{\pgfqpoint{5.925018in}{2.245347in}}%
\pgfpathlineto{\pgfqpoint{5.926091in}{2.234255in}}%
\pgfpathlineto{\pgfqpoint{5.928239in}{2.246223in}}%
\pgfpathlineto{\pgfqpoint{5.932534in}{2.236882in}}%
\pgfpathlineto{\pgfqpoint{5.934682in}{2.271618in}}%
\pgfpathlineto{\pgfqpoint{5.935756in}{2.265780in}}%
\pgfpathlineto{\pgfqpoint{5.936830in}{2.263737in}}%
\pgfpathlineto{\pgfqpoint{5.940051in}{2.273953in}}%
\pgfpathlineto{\pgfqpoint{5.941125in}{2.280083in}}%
\pgfpathlineto{\pgfqpoint{5.942199in}{2.277164in}}%
\pgfpathlineto{\pgfqpoint{5.943273in}{2.276580in}}%
\pgfpathlineto{\pgfqpoint{5.944347in}{2.281835in}}%
\pgfpathlineto{\pgfqpoint{5.947568in}{2.282127in}}%
\pgfpathlineto{\pgfqpoint{5.948642in}{2.285337in}}%
\pgfpathlineto{\pgfqpoint{5.950790in}{2.309273in}}%
\pgfpathlineto{\pgfqpoint{5.951864in}{2.299641in}}%
\pgfpathlineto{\pgfqpoint{5.955085in}{2.304603in}}%
\pgfpathlineto{\pgfqpoint{5.957233in}{2.291176in}}%
\pgfpathlineto{\pgfqpoint{5.958307in}{2.307522in}}%
\pgfpathlineto{\pgfqpoint{5.962602in}{2.303435in}}%
\pgfpathlineto{\pgfqpoint{5.963676in}{2.319782in}}%
\pgfpathlineto{\pgfqpoint{5.964750in}{2.319490in}}%
\pgfpathlineto{\pgfqpoint{5.965823in}{2.320366in}}%
\pgfpathlineto{\pgfqpoint{5.966897in}{2.318614in}}%
\pgfpathlineto{\pgfqpoint{5.970119in}{2.328831in}}%
\pgfpathlineto{\pgfqpoint{5.971193in}{2.320658in}}%
\pgfpathlineto{\pgfqpoint{5.972266in}{2.320658in}}%
\pgfpathlineto{\pgfqpoint{5.973340in}{2.282418in}}%
\pgfpathlineto{\pgfqpoint{5.974414in}{2.286797in}}%
\pgfpathlineto{\pgfqpoint{5.977636in}{2.272494in}}%
\pgfpathlineto{\pgfqpoint{5.978710in}{2.280959in}}%
\pgfpathlineto{\pgfqpoint{5.979783in}{2.264321in}}%
\pgfpathlineto{\pgfqpoint{5.980857in}{2.266072in}}%
\pgfpathlineto{\pgfqpoint{5.981931in}{2.265780in}}%
\pgfpathlineto{\pgfqpoint{5.985153in}{2.275413in}}%
\pgfpathlineto{\pgfqpoint{5.986226in}{2.284170in}}%
\pgfpathlineto{\pgfqpoint{5.987300in}{2.275997in}}%
\pgfpathlineto{\pgfqpoint{5.988374in}{2.232503in}}%
\pgfpathlineto{\pgfqpoint{5.989448in}{2.245639in}}%
\pgfpathlineto{\pgfqpoint{5.992669in}{2.250893in}}%
\pgfpathlineto{\pgfqpoint{5.993743in}{2.243012in}}%
\pgfpathlineto{\pgfqpoint{5.994817in}{2.274537in}}%
\pgfpathlineto{\pgfqpoint{5.995891in}{2.257607in}}%
\pgfpathlineto{\pgfqpoint{5.996965in}{2.255563in}}%
\pgfpathlineto{\pgfqpoint{6.000186in}{2.265196in}}%
\pgfpathlineto{\pgfqpoint{6.001260in}{2.248850in}}%
\pgfpathlineto{\pgfqpoint{6.002334in}{2.252936in}}%
\pgfpathlineto{\pgfqpoint{6.003408in}{2.252936in}}%
\pgfpathlineto{\pgfqpoint{6.004482in}{2.259942in}}%
\pgfpathlineto{\pgfqpoint{6.007703in}{2.259066in}}%
\pgfpathlineto{\pgfqpoint{6.008777in}{2.270742in}}%
\pgfpathlineto{\pgfqpoint{6.009851in}{2.260818in}}%
\pgfpathlineto{\pgfqpoint{6.010925in}{2.268991in}}%
\pgfpathlineto{\pgfqpoint{6.011999in}{2.255563in}}%
\pgfpathlineto{\pgfqpoint{6.015220in}{2.262861in}}%
\pgfpathlineto{\pgfqpoint{6.017368in}{2.241552in}}%
\pgfpathlineto{\pgfqpoint{6.018442in}{2.222579in}}%
\pgfpathlineto{\pgfqpoint{6.019515in}{2.223162in}}%
\pgfpathlineto{\pgfqpoint{6.022737in}{2.210027in}}%
\pgfpathlineto{\pgfqpoint{6.024885in}{2.228125in}}%
\pgfpathlineto{\pgfqpoint{6.027032in}{2.247390in}}%
\pgfpathlineto{\pgfqpoint{6.031328in}{2.254396in}}%
\pgfpathlineto{\pgfqpoint{6.032401in}{2.242720in}}%
\pgfpathlineto{\pgfqpoint{6.033475in}{2.251185in}}%
\pgfpathlineto{\pgfqpoint{6.034549in}{2.254980in}}%
\pgfpathlineto{\pgfqpoint{6.037771in}{2.249725in}}%
\pgfpathlineto{\pgfqpoint{6.038844in}{2.275997in}}%
\pgfpathlineto{\pgfqpoint{6.039918in}{2.270159in}}%
\pgfpathlineto{\pgfqpoint{6.040992in}{2.280959in}}%
\pgfpathlineto{\pgfqpoint{6.042066in}{2.299057in}}%
\pgfpathlineto{\pgfqpoint{6.045287in}{2.296722in}}%
\pgfpathlineto{\pgfqpoint{6.046361in}{2.307230in}}%
\pgfpathlineto{\pgfqpoint{6.047435in}{2.303435in}}%
\pgfpathlineto{\pgfqpoint{6.049583in}{2.327079in}}%
\pgfpathlineto{\pgfqpoint{6.052804in}{2.326788in}}%
\pgfpathlineto{\pgfqpoint{6.053878in}{2.335253in}}%
\pgfpathlineto{\pgfqpoint{6.054952in}{2.333501in}}%
\pgfpathlineto{\pgfqpoint{6.056026in}{2.349556in}}%
\pgfpathlineto{\pgfqpoint{6.057100in}{2.343426in}}%
\pgfpathlineto{\pgfqpoint{6.061395in}{2.353351in}}%
\pgfpathlineto{\pgfqpoint{6.062469in}{2.359481in}}%
\pgfpathlineto{\pgfqpoint{6.064617in}{2.388963in}}%
\pgfpathlineto{\pgfqpoint{6.068912in}{2.396552in}}%
\pgfpathlineto{\pgfqpoint{6.069986in}{2.404142in}}%
\pgfpathlineto{\pgfqpoint{6.071060in}{2.381665in}}%
\pgfpathlineto{\pgfqpoint{6.072133in}{2.394801in}}%
\pgfpathlineto{\pgfqpoint{6.075355in}{2.395676in}}%
\pgfpathlineto{\pgfqpoint{6.076429in}{2.384000in}}%
\pgfpathlineto{\pgfqpoint{6.077503in}{2.397428in}}%
\pgfpathlineto{\pgfqpoint{6.078576in}{2.393341in}}%
\pgfpathlineto{\pgfqpoint{6.079650in}{2.393341in}}%
\pgfpathlineto{\pgfqpoint{6.082872in}{2.395676in}}%
\pgfpathlineto{\pgfqpoint{6.083946in}{2.389838in}}%
\pgfpathlineto{\pgfqpoint{6.085020in}{2.387795in}}%
\pgfpathlineto{\pgfqpoint{6.086093in}{2.381373in}}%
\pgfpathlineto{\pgfqpoint{6.087167in}{2.400055in}}%
\pgfpathlineto{\pgfqpoint{6.090389in}{2.394217in}}%
\pgfpathlineto{\pgfqpoint{6.091463in}{2.367070in}}%
\pgfpathlineto{\pgfqpoint{6.092536in}{2.380206in}}%
\pgfpathlineto{\pgfqpoint{6.093610in}{2.368238in}}%
\pgfpathlineto{\pgfqpoint{6.094684in}{2.382249in}}%
\pgfpathlineto{\pgfqpoint{6.097906in}{2.359189in}}%
\pgfpathlineto{\pgfqpoint{6.098979in}{2.346345in}}%
\pgfpathlineto{\pgfqpoint{6.100053in}{2.344010in}}%
\pgfpathlineto{\pgfqpoint{6.101127in}{2.344594in}}%
\pgfpathlineto{\pgfqpoint{6.102201in}{2.337296in}}%
\pgfpathlineto{\pgfqpoint{6.105422in}{2.336128in}}%
\pgfpathlineto{\pgfqpoint{6.106496in}{2.337880in}}%
\pgfpathlineto{\pgfqpoint{6.107570in}{2.341675in}}%
\pgfpathlineto{\pgfqpoint{6.108644in}{2.342842in}}%
\pgfpathlineto{\pgfqpoint{6.109718in}{2.337588in}}%
\pgfpathlineto{\pgfqpoint{6.112939in}{2.336420in}}%
\pgfpathlineto{\pgfqpoint{6.114013in}{2.314820in}}%
\pgfpathlineto{\pgfqpoint{6.115087in}{2.325328in}}%
\pgfpathlineto{\pgfqpoint{6.117235in}{2.306938in}}%
\pgfpathlineto{\pgfqpoint{6.120456in}{2.309565in}}%
\pgfpathlineto{\pgfqpoint{6.121530in}{2.311609in}}%
\pgfpathlineto{\pgfqpoint{6.122604in}{2.308982in}}%
\pgfpathlineto{\pgfqpoint{6.123678in}{2.314528in}}%
\pgfpathlineto{\pgfqpoint{6.124752in}{2.297305in}}%
\pgfpathlineto{\pgfqpoint{6.127973in}{2.308106in}}%
\pgfpathlineto{\pgfqpoint{6.129047in}{2.302268in}}%
\pgfpathlineto{\pgfqpoint{6.130121in}{2.304019in}}%
\pgfpathlineto{\pgfqpoint{6.132268in}{2.318031in}}%
\pgfpathlineto{\pgfqpoint{6.137638in}{2.339047in}}%
\pgfpathlineto{\pgfqpoint{6.138711in}{2.336128in}}%
\pgfpathlineto{\pgfqpoint{6.139785in}{2.290884in}}%
\pgfpathlineto{\pgfqpoint{6.143007in}{2.310149in}}%
\pgfpathlineto{\pgfqpoint{6.144081in}{2.281543in}}%
\pgfpathlineto{\pgfqpoint{6.145154in}{2.282418in}}%
\pgfpathlineto{\pgfqpoint{6.146228in}{2.294970in}}%
\pgfpathlineto{\pgfqpoint{6.147302in}{2.292343in}}%
\pgfpathlineto{\pgfqpoint{6.150524in}{2.275121in}}%
\pgfpathlineto{\pgfqpoint{6.151598in}{2.276872in}}%
\pgfpathlineto{\pgfqpoint{6.153745in}{2.304311in}}%
\pgfpathlineto{\pgfqpoint{6.154819in}{2.309857in}}%
\pgfpathlineto{\pgfqpoint{6.158041in}{2.299349in}}%
\pgfpathlineto{\pgfqpoint{6.159114in}{2.308106in}}%
\pgfpathlineto{\pgfqpoint{6.160188in}{2.297014in}}%
\pgfpathlineto{\pgfqpoint{6.161262in}{2.298473in}}%
\pgfpathlineto{\pgfqpoint{6.162336in}{2.294970in}}%
\pgfpathlineto{\pgfqpoint{6.165557in}{2.292343in}}%
\pgfpathlineto{\pgfqpoint{6.167705in}{2.266072in}}%
\pgfpathlineto{\pgfqpoint{6.168779in}{2.265780in}}%
\pgfpathlineto{\pgfqpoint{6.169853in}{2.257023in}}%
\pgfpathlineto{\pgfqpoint{6.173074in}{2.264029in}}%
\pgfpathlineto{\pgfqpoint{6.174148in}{2.256439in}}%
\pgfpathlineto{\pgfqpoint{6.175222in}{2.266948in}}%
\pgfpathlineto{\pgfqpoint{6.177370in}{2.266364in}}%
\pgfpathlineto{\pgfqpoint{6.180591in}{2.270159in}}%
\pgfpathlineto{\pgfqpoint{6.181665in}{2.266072in}}%
\pgfpathlineto{\pgfqpoint{6.182739in}{2.268991in}}%
\pgfpathlineto{\pgfqpoint{6.184887in}{2.212654in}}%
\pgfpathlineto{\pgfqpoint{6.188108in}{2.212946in}}%
\pgfpathlineto{\pgfqpoint{6.190256in}{2.198059in}}%
\pgfpathlineto{\pgfqpoint{6.191330in}{2.221411in}}%
\pgfpathlineto{\pgfqpoint{6.192403in}{2.212946in}}%
\pgfpathlineto{\pgfqpoint{6.195625in}{2.210027in}}%
\pgfpathlineto{\pgfqpoint{6.196699in}{2.198935in}}%
\pgfpathlineto{\pgfqpoint{6.197773in}{2.180253in}}%
\pgfpathlineto{\pgfqpoint{6.198846in}{2.178501in}}%
\pgfpathlineto{\pgfqpoint{6.199920in}{2.183756in}}%
\pgfpathlineto{\pgfqpoint{6.204216in}{2.198643in}}%
\pgfpathlineto{\pgfqpoint{6.205289in}{2.203897in}}%
\pgfpathlineto{\pgfqpoint{6.206363in}{2.173831in}}%
\pgfpathlineto{\pgfqpoint{6.207437in}{2.173831in}}%
\pgfpathlineto{\pgfqpoint{6.210659in}{2.160695in}}%
\pgfpathlineto{\pgfqpoint{6.211732in}{2.191345in}}%
\pgfpathlineto{\pgfqpoint{6.212806in}{2.205648in}}%
\pgfpathlineto{\pgfqpoint{6.213880in}{2.203313in}}%
\pgfpathlineto{\pgfqpoint{6.214954in}{2.209443in}}%
\pgfpathlineto{\pgfqpoint{6.218175in}{2.215573in}}%
\pgfpathlineto{\pgfqpoint{6.220323in}{2.264904in}}%
\pgfpathlineto{\pgfqpoint{6.222471in}{2.276289in}}%
\pgfpathlineto{\pgfqpoint{6.225692in}{2.287965in}}%
\pgfpathlineto{\pgfqpoint{6.226766in}{2.283878in}}%
\pgfpathlineto{\pgfqpoint{6.227840in}{2.256439in}}%
\pgfpathlineto{\pgfqpoint{6.228914in}{2.255855in}}%
\pgfpathlineto{\pgfqpoint{6.229988in}{2.254104in}}%
\pgfpathlineto{\pgfqpoint{6.233209in}{2.252644in}}%
\pgfpathlineto{\pgfqpoint{6.234283in}{2.268407in}}%
\pgfpathlineto{\pgfqpoint{6.235357in}{2.294678in}}%
\pgfpathlineto{\pgfqpoint{6.236431in}{2.288257in}}%
\pgfpathlineto{\pgfqpoint{6.237505in}{2.297597in}}%
\pgfpathlineto{\pgfqpoint{6.240726in}{2.304311in}}%
\pgfpathlineto{\pgfqpoint{6.241800in}{2.319782in}}%
\pgfpathlineto{\pgfqpoint{6.242874in}{2.301100in}}%
\pgfpathlineto{\pgfqpoint{6.243948in}{2.305771in}}%
\pgfpathlineto{\pgfqpoint{6.245021in}{2.317447in}}%
\pgfpathlineto{\pgfqpoint{6.249317in}{2.339339in}}%
\pgfpathlineto{\pgfqpoint{6.250391in}{2.335545in}}%
\pgfpathlineto{\pgfqpoint{6.251464in}{2.353059in}}%
\pgfpathlineto{\pgfqpoint{6.252538in}{2.353934in}}%
\pgfpathlineto{\pgfqpoint{6.256834in}{2.352767in}}%
\pgfpathlineto{\pgfqpoint{6.257908in}{2.347805in}}%
\pgfpathlineto{\pgfqpoint{6.258981in}{2.355394in}}%
\pgfpathlineto{\pgfqpoint{6.260055in}{2.346053in}}%
\pgfpathlineto{\pgfqpoint{6.264351in}{2.376995in}}%
\pgfpathlineto{\pgfqpoint{6.265424in}{2.375535in}}%
\pgfpathlineto{\pgfqpoint{6.266498in}{2.378746in}}%
\pgfpathlineto{\pgfqpoint{6.267572in}{2.357729in}}%
\pgfpathlineto{\pgfqpoint{6.270794in}{2.342550in}}%
\pgfpathlineto{\pgfqpoint{6.271867in}{2.344594in}}%
\pgfpathlineto{\pgfqpoint{6.272941in}{2.337004in}}%
\pgfpathlineto{\pgfqpoint{6.274015in}{2.342550in}}%
\pgfpathlineto{\pgfqpoint{6.275089in}{2.339339in}}%
\pgfpathlineto{\pgfqpoint{6.279384in}{2.344302in}}%
\pgfpathlineto{\pgfqpoint{6.280458in}{2.331458in}}%
\pgfpathlineto{\pgfqpoint{6.281532in}{2.334377in}}%
\pgfpathlineto{\pgfqpoint{6.282606in}{2.343718in}}%
\pgfpathlineto{\pgfqpoint{6.285827in}{2.335545in}}%
\pgfpathlineto{\pgfqpoint{6.286901in}{2.275997in}}%
\pgfpathlineto{\pgfqpoint{6.287975in}{2.266948in}}%
\pgfpathlineto{\pgfqpoint{6.289049in}{2.250017in}}%
\pgfpathlineto{\pgfqpoint{6.290123in}{2.262277in}}%
\pgfpathlineto{\pgfqpoint{6.293344in}{2.256439in}}%
\pgfpathlineto{\pgfqpoint{6.294418in}{2.247098in}}%
\pgfpathlineto{\pgfqpoint{6.295492in}{2.231044in}}%
\pgfpathlineto{\pgfqpoint{6.296566in}{2.228125in}}%
\pgfpathlineto{\pgfqpoint{6.297640in}{2.236006in}}%
\pgfpathlineto{\pgfqpoint{6.300861in}{2.221411in}}%
\pgfpathlineto{\pgfqpoint{6.301935in}{2.221703in}}%
\pgfpathlineto{\pgfqpoint{6.305156in}{2.246223in}}%
\pgfpathlineto{\pgfqpoint{6.308378in}{2.235130in}}%
\pgfpathlineto{\pgfqpoint{6.310526in}{2.222870in}}%
\pgfpathlineto{\pgfqpoint{6.311599in}{2.232795in}}%
\pgfpathlineto{\pgfqpoint{6.312673in}{2.251769in}}%
\pgfpathlineto{\pgfqpoint{6.316969in}{2.257899in}}%
\pgfpathlineto{\pgfqpoint{6.318042in}{2.264321in}}%
\pgfpathlineto{\pgfqpoint{6.319116in}{2.280959in}}%
\pgfpathlineto{\pgfqpoint{6.320190in}{2.288548in}}%
\pgfpathlineto{\pgfqpoint{6.324486in}{2.263153in}}%
\pgfpathlineto{\pgfqpoint{6.327707in}{2.275121in}}%
\pgfpathlineto{\pgfqpoint{6.330929in}{2.273661in}}%
\pgfpathlineto{\pgfqpoint{6.332002in}{2.258191in}}%
\pgfpathlineto{\pgfqpoint{6.333076in}{2.250893in}}%
\pgfpathlineto{\pgfqpoint{6.335224in}{2.255855in}}%
\pgfpathlineto{\pgfqpoint{6.338445in}{2.259066in}}%
\pgfpathlineto{\pgfqpoint{6.339519in}{2.256147in}}%
\pgfpathlineto{\pgfqpoint{6.340593in}{2.276580in}}%
\pgfpathlineto{\pgfqpoint{6.341667in}{2.273953in}}%
\pgfpathlineto{\pgfqpoint{6.342741in}{2.283002in}}%
\pgfpathlineto{\pgfqpoint{6.345962in}{2.279208in}}%
\pgfpathlineto{\pgfqpoint{6.347036in}{2.276872in}}%
\pgfpathlineto{\pgfqpoint{6.348110in}{2.265196in}}%
\pgfpathlineto{\pgfqpoint{6.349184in}{2.263445in}}%
\pgfpathlineto{\pgfqpoint{6.350258in}{2.264612in}}%
\pgfpathlineto{\pgfqpoint{6.353479in}{2.250309in}}%
\pgfpathlineto{\pgfqpoint{6.354553in}{2.254688in}}%
\pgfpathlineto{\pgfqpoint{6.355627in}{2.250309in}}%
\pgfpathlineto{\pgfqpoint{6.356701in}{2.248266in}}%
\pgfpathlineto{\pgfqpoint{6.357775in}{2.240385in}}%
\pgfpathlineto{\pgfqpoint{6.362070in}{2.254980in}}%
\pgfpathlineto{\pgfqpoint{6.363144in}{2.247390in}}%
\pgfpathlineto{\pgfqpoint{6.364218in}{2.247098in}}%
\pgfpathlineto{\pgfqpoint{6.365291in}{2.252936in}}%
\pgfpathlineto{\pgfqpoint{6.368513in}{2.250017in}}%
\pgfpathlineto{\pgfqpoint{6.370661in}{2.259942in}}%
\pgfpathlineto{\pgfqpoint{6.371734in}{2.252061in}}%
\pgfpathlineto{\pgfqpoint{6.376030in}{2.257023in}}%
\pgfpathlineto{\pgfqpoint{6.377104in}{2.267823in}}%
\pgfpathlineto{\pgfqpoint{6.378177in}{2.260526in}}%
\pgfpathlineto{\pgfqpoint{6.379251in}{2.246515in}}%
\pgfpathlineto{\pgfqpoint{6.380325in}{2.215865in}}%
\pgfpathlineto{\pgfqpoint{6.383547in}{2.210611in}}%
\pgfpathlineto{\pgfqpoint{6.384620in}{2.201270in}}%
\pgfpathlineto{\pgfqpoint{6.385694in}{2.218784in}}%
\pgfpathlineto{\pgfqpoint{6.387842in}{2.180545in}}%
\pgfpathlineto{\pgfqpoint{6.391064in}{2.179669in}}%
\pgfpathlineto{\pgfqpoint{6.392137in}{2.180545in}}%
\pgfpathlineto{\pgfqpoint{6.393211in}{2.186967in}}%
\pgfpathlineto{\pgfqpoint{6.394285in}{2.179669in}}%
\pgfpathlineto{\pgfqpoint{6.395359in}{2.201270in}}%
\pgfpathlineto{\pgfqpoint{6.398580in}{2.199518in}}%
\pgfpathlineto{\pgfqpoint{6.399654in}{2.193972in}}%
\pgfpathlineto{\pgfqpoint{6.400728in}{2.193096in}}%
\pgfpathlineto{\pgfqpoint{6.402876in}{2.178793in}}%
\pgfpathlineto{\pgfqpoint{6.406097in}{2.166533in}}%
\pgfpathlineto{\pgfqpoint{6.407171in}{2.164782in}}%
\pgfpathlineto{\pgfqpoint{6.408245in}{2.142889in}}%
\pgfpathlineto{\pgfqpoint{6.410393in}{2.167701in}}%
\pgfpathlineto{\pgfqpoint{6.414688in}{2.169161in}}%
\pgfpathlineto{\pgfqpoint{6.415762in}{2.157776in}}%
\pgfpathlineto{\pgfqpoint{6.416836in}{2.164782in}}%
\pgfpathlineto{\pgfqpoint{6.417909in}{2.165074in}}%
\pgfpathlineto{\pgfqpoint{6.422205in}{2.188134in}}%
\pgfpathlineto{\pgfqpoint{6.423279in}{2.199810in}}%
\pgfpathlineto{\pgfqpoint{6.425426in}{2.194556in}}%
\pgfpathlineto{\pgfqpoint{6.428648in}{2.192805in}}%
\pgfpathlineto{\pgfqpoint{6.430796in}{2.196015in}}%
\pgfpathlineto{\pgfqpoint{6.431869in}{2.187842in}}%
\pgfpathlineto{\pgfqpoint{6.432943in}{2.201854in}}%
\pgfpathlineto{\pgfqpoint{6.436165in}{2.214405in}}%
\pgfpathlineto{\pgfqpoint{6.437239in}{2.195140in}}%
\pgfpathlineto{\pgfqpoint{6.438312in}{2.201270in}}%
\pgfpathlineto{\pgfqpoint{6.439386in}{2.199810in}}%
\pgfpathlineto{\pgfqpoint{6.440460in}{2.199518in}}%
\pgfpathlineto{\pgfqpoint{6.443682in}{2.198059in}}%
\pgfpathlineto{\pgfqpoint{6.445829in}{2.167409in}}%
\pgfpathlineto{\pgfqpoint{6.446903in}{2.167701in}}%
\pgfpathlineto{\pgfqpoint{6.447977in}{2.166825in}}%
\pgfpathlineto{\pgfqpoint{6.451198in}{2.176458in}}%
\pgfpathlineto{\pgfqpoint{6.452272in}{2.152522in}}%
\pgfpathlineto{\pgfqpoint{6.453346in}{2.152522in}}%
\pgfpathlineto{\pgfqpoint{6.454420in}{2.141138in}}%
\pgfpathlineto{\pgfqpoint{6.455494in}{2.147560in}}%
\pgfpathlineto{\pgfqpoint{6.458715in}{2.156609in}}%
\pgfpathlineto{\pgfqpoint{6.460863in}{2.147268in}}%
\pgfpathlineto{\pgfqpoint{6.461937in}{2.132673in}}%
\pgfpathlineto{\pgfqpoint{6.463011in}{2.131505in}}%
\pgfpathlineto{\pgfqpoint{6.466232in}{2.124208in}}%
\pgfpathlineto{\pgfqpoint{6.467306in}{2.115742in}}%
\pgfpathlineto{\pgfqpoint{6.469454in}{2.131797in}}%
\pgfpathlineto{\pgfqpoint{6.470528in}{2.133840in}}%
\pgfpathlineto{\pgfqpoint{6.473749in}{2.136467in}}%
\pgfpathlineto{\pgfqpoint{6.474823in}{2.128294in}}%
\pgfpathlineto{\pgfqpoint{6.475897in}{2.130629in}}%
\pgfpathlineto{\pgfqpoint{6.476971in}{2.151354in}}%
\pgfpathlineto{\pgfqpoint{6.478044in}{2.151646in}}%
\pgfpathlineto{\pgfqpoint{6.481266in}{2.137635in}}%
\pgfpathlineto{\pgfqpoint{6.483414in}{2.156025in}}%
\pgfpathlineto{\pgfqpoint{6.484487in}{2.247098in}}%
\pgfpathlineto{\pgfqpoint{6.488783in}{2.262861in}}%
\pgfpathlineto{\pgfqpoint{6.489857in}{2.275997in}}%
\pgfpathlineto{\pgfqpoint{6.490930in}{2.257899in}}%
\pgfpathlineto{\pgfqpoint{6.493078in}{2.276580in}}%
\pgfpathlineto{\pgfqpoint{6.496300in}{2.275121in}}%
\pgfpathlineto{\pgfqpoint{6.499521in}{2.252644in}}%
\pgfpathlineto{\pgfqpoint{6.500595in}{2.253812in}}%
\pgfpathlineto{\pgfqpoint{6.503817in}{2.273078in}}%
\pgfpathlineto{\pgfqpoint{6.504890in}{2.264904in}}%
\pgfpathlineto{\pgfqpoint{6.505964in}{2.263153in}}%
\pgfpathlineto{\pgfqpoint{6.507038in}{2.249434in}}%
\pgfpathlineto{\pgfqpoint{6.508112in}{2.243887in}}%
\pgfpathlineto{\pgfqpoint{6.512407in}{2.261693in}}%
\pgfpathlineto{\pgfqpoint{6.513481in}{2.259650in}}%
\pgfpathlineto{\pgfqpoint{6.514555in}{2.260526in}}%
\pgfpathlineto{\pgfqpoint{6.515629in}{2.270450in}}%
\pgfpathlineto{\pgfqpoint{6.518850in}{2.268407in}}%
\pgfpathlineto{\pgfqpoint{6.519924in}{2.265780in}}%
\pgfpathlineto{\pgfqpoint{6.520998in}{2.255563in}}%
\pgfpathlineto{\pgfqpoint{6.522072in}{2.251477in}}%
\pgfpathlineto{\pgfqpoint{6.523146in}{2.250017in}}%
\pgfpathlineto{\pgfqpoint{6.527441in}{2.235130in}}%
\pgfpathlineto{\pgfqpoint{6.528515in}{2.223162in}}%
\pgfpathlineto{\pgfqpoint{6.529589in}{2.204189in}}%
\pgfpathlineto{\pgfqpoint{6.530663in}{2.201854in}}%
\pgfpathlineto{\pgfqpoint{6.533884in}{2.206816in}}%
\pgfpathlineto{\pgfqpoint{6.536032in}{2.232211in}}%
\pgfpathlineto{\pgfqpoint{6.537106in}{2.230460in}}%
\pgfpathlineto{\pgfqpoint{6.538179in}{2.248558in}}%
\pgfpathlineto{\pgfqpoint{6.541401in}{2.254688in}}%
\pgfpathlineto{\pgfqpoint{6.542475in}{2.287965in}}%
\pgfpathlineto{\pgfqpoint{6.543549in}{2.291759in}}%
\pgfpathlineto{\pgfqpoint{6.544622in}{2.277456in}}%
\pgfpathlineto{\pgfqpoint{6.545696in}{2.302852in}}%
\pgfpathlineto{\pgfqpoint{6.548918in}{2.302852in}}%
\pgfpathlineto{\pgfqpoint{6.549992in}{2.292635in}}%
\pgfpathlineto{\pgfqpoint{6.551065in}{2.292635in}}%
\pgfpathlineto{\pgfqpoint{6.552139in}{2.290008in}}%
\pgfpathlineto{\pgfqpoint{6.553213in}{2.292051in}}%
\pgfpathlineto{\pgfqpoint{6.556435in}{2.288548in}}%
\pgfpathlineto{\pgfqpoint{6.557508in}{2.301684in}}%
\pgfpathlineto{\pgfqpoint{6.558582in}{2.302852in}}%
\pgfpathlineto{\pgfqpoint{6.559656in}{2.299349in}}%
\pgfpathlineto{\pgfqpoint{6.560730in}{2.289424in}}%
\pgfpathlineto{\pgfqpoint{6.565025in}{2.298473in}}%
\pgfpathlineto{\pgfqpoint{6.566099in}{2.290884in}}%
\pgfpathlineto{\pgfqpoint{6.568247in}{2.263737in}}%
\pgfpathlineto{\pgfqpoint{6.571468in}{2.270159in}}%
\pgfpathlineto{\pgfqpoint{6.573616in}{2.285046in}}%
\pgfpathlineto{\pgfqpoint{6.575764in}{2.308982in}}%
\pgfpathlineto{\pgfqpoint{6.578985in}{2.294386in}}%
\pgfpathlineto{\pgfqpoint{6.580059in}{2.292927in}}%
\pgfpathlineto{\pgfqpoint{6.581133in}{2.284754in}}%
\pgfpathlineto{\pgfqpoint{6.582207in}{2.291759in}}%
\pgfpathlineto{\pgfqpoint{6.583281in}{2.291176in}}%
\pgfpathlineto{\pgfqpoint{6.586502in}{2.263153in}}%
\pgfpathlineto{\pgfqpoint{6.587576in}{2.264029in}}%
\pgfpathlineto{\pgfqpoint{6.588650in}{2.263153in}}%
\pgfpathlineto{\pgfqpoint{6.589724in}{2.252936in}}%
\pgfpathlineto{\pgfqpoint{6.590797in}{2.252061in}}%
\pgfpathlineto{\pgfqpoint{6.594019in}{2.200978in}}%
\pgfpathlineto{\pgfqpoint{6.595093in}{2.201854in}}%
\pgfpathlineto{\pgfqpoint{6.596167in}{2.198935in}}%
\pgfpathlineto{\pgfqpoint{6.597240in}{2.188426in}}%
\pgfpathlineto{\pgfqpoint{6.598314in}{2.183464in}}%
\pgfpathlineto{\pgfqpoint{6.601536in}{2.179669in}}%
\pgfpathlineto{\pgfqpoint{6.602610in}{2.165366in}}%
\pgfpathlineto{\pgfqpoint{6.603684in}{2.162447in}}%
\pgfpathlineto{\pgfqpoint{6.605831in}{2.197767in}}%
\pgfpathlineto{\pgfqpoint{6.609053in}{2.219076in}}%
\pgfpathlineto{\pgfqpoint{6.610127in}{2.218492in}}%
\pgfpathlineto{\pgfqpoint{6.611200in}{2.243304in}}%
\pgfpathlineto{\pgfqpoint{6.613348in}{2.240968in}}%
\pgfpathlineto{\pgfqpoint{6.616570in}{2.258483in}}%
\pgfpathlineto{\pgfqpoint{6.619791in}{2.345761in}}%
\pgfpathlineto{\pgfqpoint{6.620865in}{2.355394in}}%
\pgfpathlineto{\pgfqpoint{6.624086in}{2.368238in}}%
\pgfpathlineto{\pgfqpoint{6.625160in}{2.346345in}}%
\pgfpathlineto{\pgfqpoint{6.627308in}{2.332918in}}%
\pgfpathlineto{\pgfqpoint{6.628382in}{2.351015in}}%
\pgfpathlineto{\pgfqpoint{6.631603in}{2.371449in}}%
\pgfpathlineto{\pgfqpoint{6.632677in}{2.407936in}}%
\pgfpathlineto{\pgfqpoint{6.633751in}{2.399763in}}%
\pgfpathlineto{\pgfqpoint{6.634825in}{2.384876in}}%
\pgfpathlineto{\pgfqpoint{6.635899in}{2.393633in}}%
\pgfpathlineto{\pgfqpoint{6.639120in}{2.409396in}}%
\pgfpathlineto{\pgfqpoint{6.640194in}{2.398012in}}%
\pgfpathlineto{\pgfqpoint{6.641268in}{2.396552in}}%
\pgfpathlineto{\pgfqpoint{6.643416in}{2.407936in}}%
\pgfpathlineto{\pgfqpoint{6.647711in}{2.408520in}}%
\pgfpathlineto{\pgfqpoint{6.648785in}{2.410272in}}%
\pgfpathlineto{\pgfqpoint{6.649859in}{2.414358in}}%
\pgfpathlineto{\pgfqpoint{6.650932in}{2.400931in}}%
\pgfpathlineto{\pgfqpoint{6.650932in}{2.400931in}}%
\pgfusepath{stroke}%
\end{pgfscope}%
\begin{pgfscope}%
\pgfpathrectangle{\pgfqpoint{4.185023in}{1.711254in}}{\pgfqpoint{2.583333in}{0.736585in}}%
\pgfusepath{clip}%
\pgfsetroundcap%
\pgfsetroundjoin%
\pgfsetlinewidth{1.505625pt}%
\definecolor{currentstroke}{rgb}{1.000000,0.647059,0.000000}%
\pgfsetstrokecolor{currentstroke}%
\pgfsetdash{}{0pt}%
\pgfpathmoveto{\pgfqpoint{4.302448in}{1.786477in}}%
\pgfpathlineto{\pgfqpoint{4.304595in}{1.777623in}}%
\pgfpathlineto{\pgfqpoint{4.305669in}{1.775239in}}%
\pgfpathlineto{\pgfqpoint{4.309964in}{1.773828in}}%
\pgfpathlineto{\pgfqpoint{4.313186in}{1.775969in}}%
\pgfpathlineto{\pgfqpoint{4.318555in}{1.777110in}}%
\pgfpathlineto{\pgfqpoint{4.320703in}{1.777832in}}%
\pgfpathlineto{\pgfqpoint{4.323924in}{1.777240in}}%
\pgfpathlineto{\pgfqpoint{4.328220in}{1.771395in}}%
\pgfpathlineto{\pgfqpoint{4.332515in}{1.769620in}}%
\pgfpathlineto{\pgfqpoint{4.335737in}{1.767795in}}%
\pgfpathlineto{\pgfqpoint{4.341106in}{1.767021in}}%
\pgfpathlineto{\pgfqpoint{4.343253in}{1.766315in}}%
\pgfpathlineto{\pgfqpoint{4.365804in}{1.766072in}}%
\pgfpathlineto{\pgfqpoint{4.372247in}{1.767364in}}%
\pgfpathlineto{\pgfqpoint{4.373321in}{1.767721in}}%
\pgfpathlineto{\pgfqpoint{4.377616in}{1.768647in}}%
\pgfpathlineto{\pgfqpoint{4.380838in}{1.770046in}}%
\pgfpathlineto{\pgfqpoint{4.385133in}{1.770985in}}%
\pgfpathlineto{\pgfqpoint{4.388355in}{1.772246in}}%
\pgfpathlineto{\pgfqpoint{4.394798in}{1.772317in}}%
\pgfpathlineto{\pgfqpoint{4.402315in}{1.772032in}}%
\pgfpathlineto{\pgfqpoint{4.408758in}{1.771294in}}%
\pgfpathlineto{\pgfqpoint{4.417348in}{1.770703in}}%
\pgfpathlineto{\pgfqpoint{4.424865in}{1.772455in}}%
\pgfpathlineto{\pgfqpoint{4.425939in}{1.773024in}}%
\pgfpathlineto{\pgfqpoint{4.430234in}{1.774245in}}%
\pgfpathlineto{\pgfqpoint{4.433456in}{1.775992in}}%
\pgfpathlineto{\pgfqpoint{4.437751in}{1.777110in}}%
\pgfpathlineto{\pgfqpoint{4.448490in}{1.781852in}}%
\pgfpathlineto{\pgfqpoint{4.452785in}{1.783090in}}%
\pgfpathlineto{\pgfqpoint{4.456006in}{1.784861in}}%
\pgfpathlineto{\pgfqpoint{4.461376in}{1.786072in}}%
\pgfpathlineto{\pgfqpoint{4.463523in}{1.787136in}}%
\pgfpathlineto{\pgfqpoint{4.467819in}{1.788133in}}%
\pgfpathlineto{\pgfqpoint{4.471040in}{1.789953in}}%
\pgfpathlineto{\pgfqpoint{4.475336in}{1.791413in}}%
\pgfpathlineto{\pgfqpoint{4.478557in}{1.793881in}}%
\pgfpathlineto{\pgfqpoint{4.482852in}{1.795581in}}%
\pgfpathlineto{\pgfqpoint{4.486074in}{1.797895in}}%
\pgfpathlineto{\pgfqpoint{4.490369in}{1.799450in}}%
\pgfpathlineto{\pgfqpoint{4.493591in}{1.801850in}}%
\pgfpathlineto{\pgfqpoint{4.496812in}{1.802770in}}%
\pgfpathlineto{\pgfqpoint{4.497886in}{1.803679in}}%
\pgfpathlineto{\pgfqpoint{4.505403in}{1.807247in}}%
\pgfpathlineto{\pgfqpoint{4.508625in}{1.809958in}}%
\pgfpathlineto{\pgfqpoint{4.511846in}{1.810889in}}%
\pgfpathlineto{\pgfqpoint{4.516141in}{1.814401in}}%
\pgfpathlineto{\pgfqpoint{4.520437in}{1.815754in}}%
\pgfpathlineto{\pgfqpoint{4.523658in}{1.817848in}}%
\pgfpathlineto{\pgfqpoint{4.527954in}{1.819396in}}%
\pgfpathlineto{\pgfqpoint{4.531175in}{1.821519in}}%
\pgfpathlineto{\pgfqpoint{4.535471in}{1.822835in}}%
\pgfpathlineto{\pgfqpoint{4.538692in}{1.824659in}}%
\pgfpathlineto{\pgfqpoint{4.542987in}{1.825816in}}%
\pgfpathlineto{\pgfqpoint{4.546209in}{1.827428in}}%
\pgfpathlineto{\pgfqpoint{4.552652in}{1.828823in}}%
\pgfpathlineto{\pgfqpoint{4.553726in}{1.829202in}}%
\pgfpathlineto{\pgfqpoint{4.559095in}{1.830184in}}%
\pgfpathlineto{\pgfqpoint{4.561243in}{1.830822in}}%
\pgfpathlineto{\pgfqpoint{4.566612in}{1.831669in}}%
\pgfpathlineto{\pgfqpoint{4.568760in}{1.832542in}}%
\pgfpathlineto{\pgfqpoint{4.573055in}{1.833448in}}%
\pgfpathlineto{\pgfqpoint{4.576276in}{1.835074in}}%
\pgfpathlineto{\pgfqpoint{4.580572in}{1.836068in}}%
\pgfpathlineto{\pgfqpoint{4.583793in}{1.837769in}}%
\pgfpathlineto{\pgfqpoint{4.588089in}{1.838914in}}%
\pgfpathlineto{\pgfqpoint{4.591310in}{1.840583in}}%
\pgfpathlineto{\pgfqpoint{4.595605in}{1.841710in}}%
\pgfpathlineto{\pgfqpoint{4.598827in}{1.843755in}}%
\pgfpathlineto{\pgfqpoint{4.603122in}{1.845042in}}%
\pgfpathlineto{\pgfqpoint{4.606344in}{1.846551in}}%
\pgfpathlineto{\pgfqpoint{4.611713in}{1.847769in}}%
\pgfpathlineto{\pgfqpoint{4.613861in}{1.848782in}}%
\pgfpathlineto{\pgfqpoint{4.619230in}{1.849915in}}%
\pgfpathlineto{\pgfqpoint{4.621378in}{1.850679in}}%
\pgfpathlineto{\pgfqpoint{4.627821in}{1.851511in}}%
\pgfpathlineto{\pgfqpoint{4.628894in}{1.851884in}}%
\pgfpathlineto{\pgfqpoint{4.657888in}{1.855436in}}%
\pgfpathlineto{\pgfqpoint{4.658962in}{1.855723in}}%
\pgfpathlineto{\pgfqpoint{4.665405in}{1.856832in}}%
\pgfpathlineto{\pgfqpoint{4.666479in}{1.857133in}}%
\pgfpathlineto{\pgfqpoint{4.671848in}{1.858026in}}%
\pgfpathlineto{\pgfqpoint{4.673996in}{1.858599in}}%
\pgfpathlineto{\pgfqpoint{4.686882in}{1.860111in}}%
\pgfpathlineto{\pgfqpoint{4.692251in}{1.860613in}}%
\pgfpathlineto{\pgfqpoint{4.738426in}{1.866521in}}%
\pgfpathlineto{\pgfqpoint{4.741648in}{1.867291in}}%
\pgfpathlineto{\pgfqpoint{4.748091in}{1.868145in}}%
\pgfpathlineto{\pgfqpoint{4.749164in}{1.868468in}}%
\pgfpathlineto{\pgfqpoint{4.754534in}{1.869573in}}%
\pgfpathlineto{\pgfqpoint{4.756681in}{1.870381in}}%
\pgfpathlineto{\pgfqpoint{4.760977in}{1.871294in}}%
\pgfpathlineto{\pgfqpoint{4.764198in}{1.872677in}}%
\pgfpathlineto{\pgfqpoint{4.768493in}{1.873659in}}%
\pgfpathlineto{\pgfqpoint{4.771715in}{1.875121in}}%
\pgfpathlineto{\pgfqpoint{4.776010in}{1.876181in}}%
\pgfpathlineto{\pgfqpoint{4.779232in}{1.877735in}}%
\pgfpathlineto{\pgfqpoint{4.783527in}{1.878827in}}%
\pgfpathlineto{\pgfqpoint{4.785675in}{1.879866in}}%
\pgfpathlineto{\pgfqpoint{4.791044in}{1.880935in}}%
\pgfpathlineto{\pgfqpoint{4.794266in}{1.882499in}}%
\pgfpathlineto{\pgfqpoint{4.798561in}{1.883611in}}%
\pgfpathlineto{\pgfqpoint{4.801782in}{1.885464in}}%
\pgfpathlineto{\pgfqpoint{4.806078in}{1.886698in}}%
\pgfpathlineto{\pgfqpoint{4.809299in}{1.888592in}}%
\pgfpathlineto{\pgfqpoint{4.813595in}{1.890019in}}%
\pgfpathlineto{\pgfqpoint{4.816816in}{1.892253in}}%
\pgfpathlineto{\pgfqpoint{4.821112in}{1.893826in}}%
\pgfpathlineto{\pgfqpoint{4.824333in}{1.895924in}}%
\pgfpathlineto{\pgfqpoint{4.828628in}{1.897294in}}%
\pgfpathlineto{\pgfqpoint{4.831850in}{1.899401in}}%
\pgfpathlineto{\pgfqpoint{4.836145in}{1.900780in}}%
\pgfpathlineto{\pgfqpoint{4.839367in}{1.902917in}}%
\pgfpathlineto{\pgfqpoint{4.843662in}{1.904198in}}%
\pgfpathlineto{\pgfqpoint{4.846884in}{1.905937in}}%
\pgfpathlineto{\pgfqpoint{4.852253in}{1.906906in}}%
\pgfpathlineto{\pgfqpoint{4.854401in}{1.907687in}}%
\pgfpathlineto{\pgfqpoint{4.859770in}{1.908810in}}%
\pgfpathlineto{\pgfqpoint{4.861917in}{1.909733in}}%
\pgfpathlineto{\pgfqpoint{4.866213in}{1.910676in}}%
\pgfpathlineto{\pgfqpoint{4.869434in}{1.912104in}}%
\pgfpathlineto{\pgfqpoint{4.873730in}{1.913115in}}%
\pgfpathlineto{\pgfqpoint{4.876951in}{1.914319in}}%
\pgfpathlineto{\pgfqpoint{4.881247in}{1.915144in}}%
\pgfpathlineto{\pgfqpoint{4.884468in}{1.916520in}}%
\pgfpathlineto{\pgfqpoint{4.888763in}{1.917406in}}%
\pgfpathlineto{\pgfqpoint{4.891985in}{1.918360in}}%
\pgfpathlineto{\pgfqpoint{4.896280in}{1.919352in}}%
\pgfpathlineto{\pgfqpoint{4.899502in}{1.920762in}}%
\pgfpathlineto{\pgfqpoint{4.903797in}{1.921620in}}%
\pgfpathlineto{\pgfqpoint{4.907019in}{1.922908in}}%
\pgfpathlineto{\pgfqpoint{4.911314in}{1.923766in}}%
\pgfpathlineto{\pgfqpoint{4.914536in}{1.925104in}}%
\pgfpathlineto{\pgfqpoint{4.919905in}{1.926380in}}%
\pgfpathlineto{\pgfqpoint{4.922052in}{1.927183in}}%
\pgfpathlineto{\pgfqpoint{4.927422in}{1.928365in}}%
\pgfpathlineto{\pgfqpoint{4.929569in}{1.929075in}}%
\pgfpathlineto{\pgfqpoint{4.934938in}{1.930088in}}%
\pgfpathlineto{\pgfqpoint{4.937086in}{1.930634in}}%
\pgfpathlineto{\pgfqpoint{4.943529in}{1.931561in}}%
\pgfpathlineto{\pgfqpoint{4.948898in}{1.932202in}}%
\pgfpathlineto{\pgfqpoint{4.959637in}{1.933524in}}%
\pgfpathlineto{\pgfqpoint{4.966080in}{1.934213in}}%
\pgfpathlineto{\pgfqpoint{4.974670in}{1.935750in}}%
\pgfpathlineto{\pgfqpoint{4.986483in}{1.937115in}}%
\pgfpathlineto{\pgfqpoint{4.997221in}{1.938562in}}%
\pgfpathlineto{\pgfqpoint{5.003664in}{1.939422in}}%
\pgfpathlineto{\pgfqpoint{5.004738in}{1.939769in}}%
\pgfpathlineto{\pgfqpoint{5.010107in}{1.940924in}}%
\pgfpathlineto{\pgfqpoint{5.012255in}{1.941658in}}%
\pgfpathlineto{\pgfqpoint{5.017624in}{1.942771in}}%
\pgfpathlineto{\pgfqpoint{5.019772in}{1.943482in}}%
\pgfpathlineto{\pgfqpoint{5.025141in}{1.944552in}}%
\pgfpathlineto{\pgfqpoint{5.027289in}{1.945219in}}%
\pgfpathlineto{\pgfqpoint{5.032658in}{1.946185in}}%
\pgfpathlineto{\pgfqpoint{5.034805in}{1.946848in}}%
\pgfpathlineto{\pgfqpoint{5.040175in}{1.947881in}}%
\pgfpathlineto{\pgfqpoint{5.042322in}{1.948530in}}%
\pgfpathlineto{\pgfqpoint{5.053061in}{1.950006in}}%
\pgfpathlineto{\pgfqpoint{5.057356in}{1.951085in}}%
\pgfpathlineto{\pgfqpoint{5.069168in}{1.952636in}}%
\pgfpathlineto{\pgfqpoint{5.076685in}{1.953737in}}%
\pgfpathlineto{\pgfqpoint{5.160445in}{1.962859in}}%
\pgfpathlineto{\pgfqpoint{5.228096in}{1.969663in}}%
\pgfpathlineto{\pgfqpoint{5.245278in}{1.972116in}}%
\pgfpathlineto{\pgfqpoint{5.256016in}{1.973453in}}%
\pgfpathlineto{\pgfqpoint{5.267828in}{1.975367in}}%
\pgfpathlineto{\pgfqpoint{5.278567in}{1.976610in}}%
\pgfpathlineto{\pgfqpoint{5.281788in}{1.977252in}}%
\pgfpathlineto{\pgfqpoint{5.289305in}{1.978110in}}%
\pgfpathlineto{\pgfqpoint{5.297896in}{1.979697in}}%
\pgfpathlineto{\pgfqpoint{5.304339in}{1.980783in}}%
\pgfpathlineto{\pgfqpoint{5.311856in}{1.982211in}}%
\pgfpathlineto{\pgfqpoint{5.320446in}{1.983460in}}%
\pgfpathlineto{\pgfqpoint{5.331185in}{1.984575in}}%
\pgfpathlineto{\pgfqpoint{5.342997in}{1.986299in}}%
\pgfpathlineto{\pgfqpoint{5.353735in}{1.987348in}}%
\pgfpathlineto{\pgfqpoint{5.365548in}{1.988992in}}%
\pgfpathlineto{\pgfqpoint{5.376286in}{1.990223in}}%
\pgfpathlineto{\pgfqpoint{5.388098in}{1.991986in}}%
\pgfpathlineto{\pgfqpoint{5.399911in}{1.993098in}}%
\pgfpathlineto{\pgfqpoint{5.440716in}{1.998896in}}%
\pgfpathlineto{\pgfqpoint{5.513737in}{2.003991in}}%
\pgfpathlineto{\pgfqpoint{5.538436in}{2.006735in}}%
\pgfpathlineto{\pgfqpoint{5.551322in}{2.007932in}}%
\pgfpathlineto{\pgfqpoint{5.568503in}{2.009855in}}%
\pgfpathlineto{\pgfqpoint{5.581389in}{2.010985in}}%
\pgfpathlineto{\pgfqpoint{5.591054in}{2.012186in}}%
\pgfpathlineto{\pgfqpoint{5.601792in}{2.013299in}}%
\pgfpathlineto{\pgfqpoint{5.613604in}{2.015069in}}%
\pgfpathlineto{\pgfqpoint{5.624343in}{2.016148in}}%
\pgfpathlineto{\pgfqpoint{5.636155in}{2.017527in}}%
\pgfpathlineto{\pgfqpoint{5.656558in}{2.019078in}}%
\pgfpathlineto{\pgfqpoint{5.669444in}{2.019929in}}%
\pgfpathlineto{\pgfqpoint{5.711324in}{2.022659in}}%
\pgfpathlineto{\pgfqpoint{5.725284in}{2.023675in}}%
\pgfpathlineto{\pgfqpoint{5.741391in}{2.024272in}}%
\pgfpathlineto{\pgfqpoint{5.794009in}{2.025633in}}%
\pgfpathlineto{\pgfqpoint{5.820855in}{2.026927in}}%
\pgfpathlineto{\pgfqpoint{5.846627in}{2.027972in}}%
\pgfpathlineto{\pgfqpoint{5.865956in}{2.028889in}}%
\pgfpathlineto{\pgfqpoint{5.876695in}{2.029431in}}%
\pgfpathlineto{\pgfqpoint{5.896024in}{2.030270in}}%
\pgfpathlineto{\pgfqpoint{5.914279in}{2.032663in}}%
\pgfpathlineto{\pgfqpoint{5.925018in}{2.033639in}}%
\pgfpathlineto{\pgfqpoint{5.936830in}{2.035502in}}%
\pgfpathlineto{\pgfqpoint{5.943273in}{2.036419in}}%
\pgfpathlineto{\pgfqpoint{5.951864in}{2.037872in}}%
\pgfpathlineto{\pgfqpoint{5.962602in}{2.039109in}}%
\pgfpathlineto{\pgfqpoint{5.973340in}{2.041178in}}%
\pgfpathlineto{\pgfqpoint{5.981931in}{2.042467in}}%
\pgfpathlineto{\pgfqpoint{5.992669in}{2.043675in}}%
\pgfpathlineto{\pgfqpoint{6.004482in}{2.045429in}}%
\pgfpathlineto{\pgfqpoint{6.015220in}{2.046617in}}%
\pgfpathlineto{\pgfqpoint{6.024885in}{2.047770in}}%
\pgfpathlineto{\pgfqpoint{6.034549in}{2.048849in}}%
\pgfpathlineto{\pgfqpoint{6.040992in}{2.049639in}}%
\pgfpathlineto{\pgfqpoint{6.049583in}{2.051025in}}%
\pgfpathlineto{\pgfqpoint{6.056026in}{2.052038in}}%
\pgfpathlineto{\pgfqpoint{6.064617in}{2.053684in}}%
\pgfpathlineto{\pgfqpoint{6.071060in}{2.054584in}}%
\pgfpathlineto{\pgfqpoint{6.072133in}{2.054883in}}%
\pgfpathlineto{\pgfqpoint{6.077503in}{2.055772in}}%
\pgfpathlineto{\pgfqpoint{6.079650in}{2.056364in}}%
\pgfpathlineto{\pgfqpoint{6.085020in}{2.057242in}}%
\pgfpathlineto{\pgfqpoint{6.094684in}{2.059216in}}%
\pgfpathlineto{\pgfqpoint{6.101127in}{2.060218in}}%
\pgfpathlineto{\pgfqpoint{6.109718in}{2.061658in}}%
\pgfpathlineto{\pgfqpoint{6.116161in}{2.062558in}}%
\pgfpathlineto{\pgfqpoint{6.124752in}{2.063816in}}%
\pgfpathlineto{\pgfqpoint{6.136564in}{2.065085in}}%
\pgfpathlineto{\pgfqpoint{6.143007in}{2.065945in}}%
\pgfpathlineto{\pgfqpoint{6.154819in}{2.067639in}}%
\pgfpathlineto{\pgfqpoint{6.165557in}{2.068797in}}%
\pgfpathlineto{\pgfqpoint{6.177370in}{2.070263in}}%
\pgfpathlineto{\pgfqpoint{6.196699in}{2.071799in}}%
\pgfpathlineto{\pgfqpoint{6.250391in}{2.076786in}}%
\pgfpathlineto{\pgfqpoint{6.252538in}{2.077227in}}%
\pgfpathlineto{\pgfqpoint{6.264351in}{2.078334in}}%
\pgfpathlineto{\pgfqpoint{6.271867in}{2.079447in}}%
\pgfpathlineto{\pgfqpoint{6.275089in}{2.080063in}}%
\pgfpathlineto{\pgfqpoint{6.285827in}{2.081077in}}%
\pgfpathlineto{\pgfqpoint{6.297640in}{2.082267in}}%
\pgfpathlineto{\pgfqpoint{6.310526in}{2.083189in}}%
\pgfpathlineto{\pgfqpoint{6.335224in}{2.085402in}}%
\pgfpathlineto{\pgfqpoint{6.347036in}{2.086398in}}%
\pgfpathlineto{\pgfqpoint{6.357775in}{2.087418in}}%
\pgfpathlineto{\pgfqpoint{6.370661in}{2.088411in}}%
\pgfpathlineto{\pgfqpoint{6.380325in}{2.089138in}}%
\pgfpathlineto{\pgfqpoint{6.409319in}{2.090488in}}%
\pgfpathlineto{\pgfqpoint{6.455494in}{2.092435in}}%
\pgfpathlineto{\pgfqpoint{6.489857in}{2.093454in}}%
\pgfpathlineto{\pgfqpoint{6.508112in}{2.095000in}}%
\pgfpathlineto{\pgfqpoint{6.520998in}{2.095939in}}%
\pgfpathlineto{\pgfqpoint{6.530663in}{2.096491in}}%
\pgfpathlineto{\pgfqpoint{6.543549in}{2.097325in}}%
\pgfpathlineto{\pgfqpoint{6.560730in}{2.098955in}}%
\pgfpathlineto{\pgfqpoint{6.574690in}{2.100098in}}%
\pgfpathlineto{\pgfqpoint{6.583281in}{2.100892in}}%
\pgfpathlineto{\pgfqpoint{6.617643in}{2.102577in}}%
\pgfpathlineto{\pgfqpoint{6.643416in}{2.105822in}}%
\pgfpathlineto{\pgfqpoint{6.650932in}{2.106625in}}%
\pgfpathlineto{\pgfqpoint{6.650932in}{2.106625in}}%
\pgfusepath{stroke}%
\end{pgfscope}%
\begin{pgfscope}%
\pgfsetrectcap%
\pgfsetmiterjoin%
\pgfsetlinewidth{0.803000pt}%
\definecolor{currentstroke}{rgb}{1.000000,1.000000,1.000000}%
\pgfsetstrokecolor{currentstroke}%
\pgfsetdash{}{0pt}%
\pgfpathmoveto{\pgfqpoint{4.185023in}{1.711254in}}%
\pgfpathlineto{\pgfqpoint{4.185023in}{2.447839in}}%
\pgfusepath{stroke}%
\end{pgfscope}%
\begin{pgfscope}%
\pgfsetrectcap%
\pgfsetmiterjoin%
\pgfsetlinewidth{0.803000pt}%
\definecolor{currentstroke}{rgb}{1.000000,1.000000,1.000000}%
\pgfsetstrokecolor{currentstroke}%
\pgfsetdash{}{0pt}%
\pgfpathmoveto{\pgfqpoint{6.768357in}{1.711254in}}%
\pgfpathlineto{\pgfqpoint{6.768357in}{2.447839in}}%
\pgfusepath{stroke}%
\end{pgfscope}%
\begin{pgfscope}%
\pgfsetrectcap%
\pgfsetmiterjoin%
\pgfsetlinewidth{0.803000pt}%
\definecolor{currentstroke}{rgb}{1.000000,1.000000,1.000000}%
\pgfsetstrokecolor{currentstroke}%
\pgfsetdash{}{0pt}%
\pgfpathmoveto{\pgfqpoint{4.185023in}{1.711254in}}%
\pgfpathlineto{\pgfqpoint{6.768357in}{1.711254in}}%
\pgfusepath{stroke}%
\end{pgfscope}%
\begin{pgfscope}%
\pgfsetrectcap%
\pgfsetmiterjoin%
\pgfsetlinewidth{0.803000pt}%
\definecolor{currentstroke}{rgb}{1.000000,1.000000,1.000000}%
\pgfsetstrokecolor{currentstroke}%
\pgfsetdash{}{0pt}%
\pgfpathmoveto{\pgfqpoint{4.185023in}{2.447839in}}%
\pgfpathlineto{\pgfqpoint{6.768357in}{2.447839in}}%
\pgfusepath{stroke}%
\end{pgfscope}%
\begin{pgfscope}%
\definecolor{textcolor}{rgb}{0.150000,0.150000,0.150000}%
\pgfsetstrokecolor{textcolor}%
\pgfsetfillcolor{textcolor}%
\pgftext[x=5.476690in,y=2.531173in,,base]{\color{textcolor}\rmfamily\fontsize{16.800000}{20.160000}\selectfont VZ}%
\end{pgfscope}%
\begin{pgfscope}%
\pgfsetbuttcap%
\pgfsetmiterjoin%
\definecolor{currentfill}{rgb}{0.917647,0.917647,0.949020}%
\pgfsetfillcolor{currentfill}%
\pgfsetlinewidth{0.000000pt}%
\definecolor{currentstroke}{rgb}{0.000000,0.000000,0.000000}%
\pgfsetstrokecolor{currentstroke}%
\pgfsetstrokeopacity{0.000000}%
\pgfsetdash{}{0pt}%
\pgfpathmoveto{\pgfqpoint{0.568357in}{0.385400in}}%
\pgfpathlineto{\pgfqpoint{3.151690in}{0.385400in}}%
\pgfpathlineto{\pgfqpoint{3.151690in}{1.121986in}}%
\pgfpathlineto{\pgfqpoint{0.568357in}{1.121986in}}%
\pgfpathclose%
\pgfusepath{fill}%
\end{pgfscope}%
\begin{pgfscope}%
\pgfpathrectangle{\pgfqpoint{0.568357in}{0.385400in}}{\pgfqpoint{2.583333in}{0.736585in}}%
\pgfusepath{clip}%
\pgfsetroundcap%
\pgfsetroundjoin%
\pgfsetlinewidth{0.803000pt}%
\definecolor{currentstroke}{rgb}{1.000000,1.000000,1.000000}%
\pgfsetstrokecolor{currentstroke}%
\pgfsetdash{}{0pt}%
\pgfpathmoveto{\pgfqpoint{0.683633in}{0.385400in}}%
\pgfpathlineto{\pgfqpoint{0.683633in}{1.121986in}}%
\pgfusepath{stroke}%
\end{pgfscope}%
\begin{pgfscope}%
\definecolor{textcolor}{rgb}{0.150000,0.150000,0.150000}%
\pgfsetstrokecolor{textcolor}%
\pgfsetfillcolor{textcolor}%
\pgftext[x=0.683633in,y=0.288178in,,top]{\color{textcolor}\rmfamily\fontsize{14.000000}{16.800000}\selectfont 2012}%
\end{pgfscope}%
\begin{pgfscope}%
\pgfpathrectangle{\pgfqpoint{0.568357in}{0.385400in}}{\pgfqpoint{2.583333in}{0.736585in}}%
\pgfusepath{clip}%
\pgfsetroundcap%
\pgfsetroundjoin%
\pgfsetlinewidth{0.803000pt}%
\definecolor{currentstroke}{rgb}{1.000000,1.000000,1.000000}%
\pgfsetstrokecolor{currentstroke}%
\pgfsetdash{}{0pt}%
\pgfpathmoveto{\pgfqpoint{1.076658in}{0.385400in}}%
\pgfpathlineto{\pgfqpoint{1.076658in}{1.121986in}}%
\pgfusepath{stroke}%
\end{pgfscope}%
\begin{pgfscope}%
\definecolor{textcolor}{rgb}{0.150000,0.150000,0.150000}%
\pgfsetstrokecolor{textcolor}%
\pgfsetfillcolor{textcolor}%
\pgftext[x=1.076658in,y=0.288178in,,top]{\color{textcolor}\rmfamily\fontsize{14.000000}{16.800000}\selectfont 2013}%
\end{pgfscope}%
\begin{pgfscope}%
\pgfpathrectangle{\pgfqpoint{0.568357in}{0.385400in}}{\pgfqpoint{2.583333in}{0.736585in}}%
\pgfusepath{clip}%
\pgfsetroundcap%
\pgfsetroundjoin%
\pgfsetlinewidth{0.803000pt}%
\definecolor{currentstroke}{rgb}{1.000000,1.000000,1.000000}%
\pgfsetstrokecolor{currentstroke}%
\pgfsetdash{}{0pt}%
\pgfpathmoveto{\pgfqpoint{1.468609in}{0.385400in}}%
\pgfpathlineto{\pgfqpoint{1.468609in}{1.121986in}}%
\pgfusepath{stroke}%
\end{pgfscope}%
\begin{pgfscope}%
\definecolor{textcolor}{rgb}{0.150000,0.150000,0.150000}%
\pgfsetstrokecolor{textcolor}%
\pgfsetfillcolor{textcolor}%
\pgftext[x=1.468609in,y=0.288178in,,top]{\color{textcolor}\rmfamily\fontsize{14.000000}{16.800000}\selectfont 2014}%
\end{pgfscope}%
\begin{pgfscope}%
\pgfpathrectangle{\pgfqpoint{0.568357in}{0.385400in}}{\pgfqpoint{2.583333in}{0.736585in}}%
\pgfusepath{clip}%
\pgfsetroundcap%
\pgfsetroundjoin%
\pgfsetlinewidth{0.803000pt}%
\definecolor{currentstroke}{rgb}{1.000000,1.000000,1.000000}%
\pgfsetstrokecolor{currentstroke}%
\pgfsetdash{}{0pt}%
\pgfpathmoveto{\pgfqpoint{1.860560in}{0.385400in}}%
\pgfpathlineto{\pgfqpoint{1.860560in}{1.121986in}}%
\pgfusepath{stroke}%
\end{pgfscope}%
\begin{pgfscope}%
\definecolor{textcolor}{rgb}{0.150000,0.150000,0.150000}%
\pgfsetstrokecolor{textcolor}%
\pgfsetfillcolor{textcolor}%
\pgftext[x=1.860560in,y=0.288178in,,top]{\color{textcolor}\rmfamily\fontsize{14.000000}{16.800000}\selectfont 2015}%
\end{pgfscope}%
\begin{pgfscope}%
\pgfpathrectangle{\pgfqpoint{0.568357in}{0.385400in}}{\pgfqpoint{2.583333in}{0.736585in}}%
\pgfusepath{clip}%
\pgfsetroundcap%
\pgfsetroundjoin%
\pgfsetlinewidth{0.803000pt}%
\definecolor{currentstroke}{rgb}{1.000000,1.000000,1.000000}%
\pgfsetstrokecolor{currentstroke}%
\pgfsetdash{}{0pt}%
\pgfpathmoveto{\pgfqpoint{2.252511in}{0.385400in}}%
\pgfpathlineto{\pgfqpoint{2.252511in}{1.121986in}}%
\pgfusepath{stroke}%
\end{pgfscope}%
\begin{pgfscope}%
\definecolor{textcolor}{rgb}{0.150000,0.150000,0.150000}%
\pgfsetstrokecolor{textcolor}%
\pgfsetfillcolor{textcolor}%
\pgftext[x=2.252511in,y=0.288178in,,top]{\color{textcolor}\rmfamily\fontsize{14.000000}{16.800000}\selectfont 2016}%
\end{pgfscope}%
\begin{pgfscope}%
\pgfpathrectangle{\pgfqpoint{0.568357in}{0.385400in}}{\pgfqpoint{2.583333in}{0.736585in}}%
\pgfusepath{clip}%
\pgfsetroundcap%
\pgfsetroundjoin%
\pgfsetlinewidth{0.803000pt}%
\definecolor{currentstroke}{rgb}{1.000000,1.000000,1.000000}%
\pgfsetstrokecolor{currentstroke}%
\pgfsetdash{}{0pt}%
\pgfpathmoveto{\pgfqpoint{2.645536in}{0.385400in}}%
\pgfpathlineto{\pgfqpoint{2.645536in}{1.121986in}}%
\pgfusepath{stroke}%
\end{pgfscope}%
\begin{pgfscope}%
\definecolor{textcolor}{rgb}{0.150000,0.150000,0.150000}%
\pgfsetstrokecolor{textcolor}%
\pgfsetfillcolor{textcolor}%
\pgftext[x=2.645536in,y=0.288178in,,top]{\color{textcolor}\rmfamily\fontsize{14.000000}{16.800000}\selectfont 2017}%
\end{pgfscope}%
\begin{pgfscope}%
\pgfpathrectangle{\pgfqpoint{0.568357in}{0.385400in}}{\pgfqpoint{2.583333in}{0.736585in}}%
\pgfusepath{clip}%
\pgfsetroundcap%
\pgfsetroundjoin%
\pgfsetlinewidth{0.803000pt}%
\definecolor{currentstroke}{rgb}{1.000000,1.000000,1.000000}%
\pgfsetstrokecolor{currentstroke}%
\pgfsetdash{}{0pt}%
\pgfpathmoveto{\pgfqpoint{3.037487in}{0.385400in}}%
\pgfpathlineto{\pgfqpoint{3.037487in}{1.121986in}}%
\pgfusepath{stroke}%
\end{pgfscope}%
\begin{pgfscope}%
\definecolor{textcolor}{rgb}{0.150000,0.150000,0.150000}%
\pgfsetstrokecolor{textcolor}%
\pgfsetfillcolor{textcolor}%
\pgftext[x=3.037487in,y=0.288178in,,top]{\color{textcolor}\rmfamily\fontsize{14.000000}{16.800000}\selectfont 2018}%
\end{pgfscope}%
\begin{pgfscope}%
\pgfpathrectangle{\pgfqpoint{0.568357in}{0.385400in}}{\pgfqpoint{2.583333in}{0.736585in}}%
\pgfusepath{clip}%
\pgfsetroundcap%
\pgfsetroundjoin%
\pgfsetlinewidth{0.803000pt}%
\definecolor{currentstroke}{rgb}{1.000000,1.000000,1.000000}%
\pgfsetstrokecolor{currentstroke}%
\pgfsetdash{}{0pt}%
\pgfpathmoveto{\pgfqpoint{0.568357in}{0.626357in}}%
\pgfpathlineto{\pgfqpoint{3.151690in}{0.626357in}}%
\pgfusepath{stroke}%
\end{pgfscope}%
\begin{pgfscope}%
\definecolor{textcolor}{rgb}{0.150000,0.150000,0.150000}%
\pgfsetstrokecolor{textcolor}%
\pgfsetfillcolor{textcolor}%
\pgftext[x=0.223712in,y=0.552491in,left,base]{\color{textcolor}\rmfamily\fontsize{14.000000}{16.800000}\selectfont 50}%
\end{pgfscope}%
\begin{pgfscope}%
\pgfpathrectangle{\pgfqpoint{0.568357in}{0.385400in}}{\pgfqpoint{2.583333in}{0.736585in}}%
\pgfusepath{clip}%
\pgfsetroundcap%
\pgfsetroundjoin%
\pgfsetlinewidth{0.803000pt}%
\definecolor{currentstroke}{rgb}{1.000000,1.000000,1.000000}%
\pgfsetstrokecolor{currentstroke}%
\pgfsetdash{}{0pt}%
\pgfpathmoveto{\pgfqpoint{0.568357in}{0.992792in}}%
\pgfpathlineto{\pgfqpoint{3.151690in}{0.992792in}}%
\pgfusepath{stroke}%
\end{pgfscope}%
\begin{pgfscope}%
\definecolor{textcolor}{rgb}{0.150000,0.150000,0.150000}%
\pgfsetstrokecolor{textcolor}%
\pgfsetfillcolor{textcolor}%
\pgftext[x=0.100000in,y=0.918926in,left,base]{\color{textcolor}\rmfamily\fontsize{14.000000}{16.800000}\selectfont 100}%
\end{pgfscope}%
\begin{pgfscope}%
\pgfpathrectangle{\pgfqpoint{0.568357in}{0.385400in}}{\pgfqpoint{2.583333in}{0.736585in}}%
\pgfusepath{clip}%
\pgfsetroundcap%
\pgfsetroundjoin%
\pgfsetlinewidth{1.505625pt}%
\definecolor{currentstroke}{rgb}{0.737255,0.741176,0.133333}%
\pgfsetstrokecolor{currentstroke}%
\pgfsetdash{}{0pt}%
\pgfpathmoveto{\pgfqpoint{0.685781in}{0.425038in}}%
\pgfpathlineto{\pgfqpoint{0.686855in}{0.422106in}}%
\pgfpathlineto{\pgfqpoint{0.687929in}{0.423352in}}%
\pgfpathlineto{\pgfqpoint{0.689002in}{0.421446in}}%
\pgfpathlineto{\pgfqpoint{0.694372in}{0.418881in}}%
\pgfpathlineto{\pgfqpoint{0.695445in}{0.422399in}}%
\pgfpathlineto{\pgfqpoint{0.696519in}{0.421446in}}%
\pgfpathlineto{\pgfqpoint{0.700815in}{0.424305in}}%
\pgfpathlineto{\pgfqpoint{0.701888in}{0.426357in}}%
\pgfpathlineto{\pgfqpoint{0.704036in}{0.421227in}}%
\pgfpathlineto{\pgfqpoint{0.707258in}{0.419614in}}%
\pgfpathlineto{\pgfqpoint{0.708332in}{0.421886in}}%
\pgfpathlineto{\pgfqpoint{0.709405in}{0.421153in}}%
\pgfpathlineto{\pgfqpoint{0.711553in}{0.421959in}}%
\pgfpathlineto{\pgfqpoint{0.714775in}{0.420054in}}%
\pgfpathlineto{\pgfqpoint{0.715848in}{0.421300in}}%
\pgfpathlineto{\pgfqpoint{0.716922in}{0.424305in}}%
\pgfpathlineto{\pgfqpoint{0.717996in}{0.430021in}}%
\pgfpathlineto{\pgfqpoint{0.719070in}{0.431560in}}%
\pgfpathlineto{\pgfqpoint{0.722291in}{0.432073in}}%
\pgfpathlineto{\pgfqpoint{0.723365in}{0.431487in}}%
\pgfpathlineto{\pgfqpoint{0.724439in}{0.433685in}}%
\pgfpathlineto{\pgfqpoint{0.725513in}{0.440208in}}%
\pgfpathlineto{\pgfqpoint{0.726587in}{0.442553in}}%
\pgfpathlineto{\pgfqpoint{0.729808in}{0.440648in}}%
\pgfpathlineto{\pgfqpoint{0.730882in}{0.444678in}}%
\pgfpathlineto{\pgfqpoint{0.731956in}{0.445924in}}%
\pgfpathlineto{\pgfqpoint{0.733030in}{0.443799in}}%
\pgfpathlineto{\pgfqpoint{0.734104in}{0.445778in}}%
\pgfpathlineto{\pgfqpoint{0.738399in}{0.444459in}}%
\pgfpathlineto{\pgfqpoint{0.739473in}{0.447976in}}%
\pgfpathlineto{\pgfqpoint{0.740547in}{0.448123in}}%
\pgfpathlineto{\pgfqpoint{0.741621in}{0.449882in}}%
\pgfpathlineto{\pgfqpoint{0.744842in}{0.448783in}}%
\pgfpathlineto{\pgfqpoint{0.745916in}{0.452154in}}%
\pgfpathlineto{\pgfqpoint{0.746990in}{0.447976in}}%
\pgfpathlineto{\pgfqpoint{0.748064in}{0.449296in}}%
\pgfpathlineto{\pgfqpoint{0.749137in}{0.447610in}}%
\pgfpathlineto{\pgfqpoint{0.752359in}{0.447756in}}%
\pgfpathlineto{\pgfqpoint{0.753433in}{0.445485in}}%
\pgfpathlineto{\pgfqpoint{0.754507in}{0.446804in}}%
\pgfpathlineto{\pgfqpoint{0.755580in}{0.450541in}}%
\pgfpathlineto{\pgfqpoint{0.756654in}{0.449222in}}%
\pgfpathlineto{\pgfqpoint{0.759876in}{0.448196in}}%
\pgfpathlineto{\pgfqpoint{0.760950in}{0.449442in}}%
\pgfpathlineto{\pgfqpoint{0.762023in}{0.448563in}}%
\pgfpathlineto{\pgfqpoint{0.763097in}{0.449002in}}%
\pgfpathlineto{\pgfqpoint{0.764171in}{0.448489in}}%
\pgfpathlineto{\pgfqpoint{0.767393in}{0.451934in}}%
\pgfpathlineto{\pgfqpoint{0.768466in}{0.448196in}}%
\pgfpathlineto{\pgfqpoint{0.770614in}{0.449442in}}%
\pgfpathlineto{\pgfqpoint{0.771688in}{0.451861in}}%
\pgfpathlineto{\pgfqpoint{0.774910in}{0.453913in}}%
\pgfpathlineto{\pgfqpoint{0.778131in}{0.452227in}}%
\pgfpathlineto{\pgfqpoint{0.779205in}{0.450615in}}%
\pgfpathlineto{\pgfqpoint{0.782426in}{0.452154in}}%
\pgfpathlineto{\pgfqpoint{0.783500in}{0.454426in}}%
\pgfpathlineto{\pgfqpoint{0.784574in}{0.452154in}}%
\pgfpathlineto{\pgfqpoint{0.785648in}{0.455452in}}%
\pgfpathlineto{\pgfqpoint{0.789943in}{0.453033in}}%
\pgfpathlineto{\pgfqpoint{0.791017in}{0.448563in}}%
\pgfpathlineto{\pgfqpoint{0.792091in}{0.449589in}}%
\pgfpathlineto{\pgfqpoint{0.794239in}{0.458896in}}%
\pgfpathlineto{\pgfqpoint{0.797460in}{0.455158in}}%
\pgfpathlineto{\pgfqpoint{0.798534in}{0.457211in}}%
\pgfpathlineto{\pgfqpoint{0.801755in}{0.455452in}}%
\pgfpathlineto{\pgfqpoint{0.804977in}{0.450761in}}%
\pgfpathlineto{\pgfqpoint{0.806051in}{0.452080in}}%
\pgfpathlineto{\pgfqpoint{0.808199in}{0.458969in}}%
\pgfpathlineto{\pgfqpoint{0.809272in}{0.459482in}}%
\pgfpathlineto{\pgfqpoint{0.813568in}{0.458823in}}%
\pgfpathlineto{\pgfqpoint{0.814642in}{0.457357in}}%
\pgfpathlineto{\pgfqpoint{0.815715in}{0.448050in}}%
\pgfpathlineto{\pgfqpoint{0.816789in}{0.450248in}}%
\pgfpathlineto{\pgfqpoint{0.820011in}{0.451421in}}%
\pgfpathlineto{\pgfqpoint{0.822158in}{0.449662in}}%
\pgfpathlineto{\pgfqpoint{0.823232in}{0.450835in}}%
\pgfpathlineto{\pgfqpoint{0.824306in}{0.450468in}}%
\pgfpathlineto{\pgfqpoint{0.828601in}{0.448343in}}%
\pgfpathlineto{\pgfqpoint{0.829675in}{0.451274in}}%
\pgfpathlineto{\pgfqpoint{0.831823in}{0.443286in}}%
\pgfpathlineto{\pgfqpoint{0.835044in}{0.449149in}}%
\pgfpathlineto{\pgfqpoint{0.836118in}{0.452667in}}%
\pgfpathlineto{\pgfqpoint{0.838266in}{0.454939in}}%
\pgfpathlineto{\pgfqpoint{0.839340in}{0.454279in}}%
\pgfpathlineto{\pgfqpoint{0.843635in}{0.455745in}}%
\pgfpathlineto{\pgfqpoint{0.846857in}{0.442700in}}%
\pgfpathlineto{\pgfqpoint{0.850078in}{0.445778in}}%
\pgfpathlineto{\pgfqpoint{0.851152in}{0.445191in}}%
\pgfpathlineto{\pgfqpoint{0.852226in}{0.449369in}}%
\pgfpathlineto{\pgfqpoint{0.853300in}{0.450102in}}%
\pgfpathlineto{\pgfqpoint{0.854374in}{0.449735in}}%
\pgfpathlineto{\pgfqpoint{0.857595in}{0.449369in}}%
\pgfpathlineto{\pgfqpoint{0.858669in}{0.450688in}}%
\pgfpathlineto{\pgfqpoint{0.859743in}{0.447024in}}%
\pgfpathlineto{\pgfqpoint{0.861890in}{0.452667in}}%
\pgfpathlineto{\pgfqpoint{0.865112in}{0.455452in}}%
\pgfpathlineto{\pgfqpoint{0.867260in}{0.459482in}}%
\pgfpathlineto{\pgfqpoint{0.868333in}{0.454426in}}%
\pgfpathlineto{\pgfqpoint{0.869407in}{0.463367in}}%
\pgfpathlineto{\pgfqpoint{0.872629in}{0.457357in}}%
\pgfpathlineto{\pgfqpoint{0.873703in}{0.460435in}}%
\pgfpathlineto{\pgfqpoint{0.874776in}{0.460875in}}%
\pgfpathlineto{\pgfqpoint{0.875850in}{0.457797in}}%
\pgfpathlineto{\pgfqpoint{0.876924in}{0.461241in}}%
\pgfpathlineto{\pgfqpoint{0.880146in}{0.466078in}}%
\pgfpathlineto{\pgfqpoint{0.881220in}{0.465639in}}%
\pgfpathlineto{\pgfqpoint{0.883367in}{0.466738in}}%
\pgfpathlineto{\pgfqpoint{0.884441in}{0.463880in}}%
\pgfpathlineto{\pgfqpoint{0.887663in}{0.461241in}}%
\pgfpathlineto{\pgfqpoint{0.889810in}{0.456478in}}%
\pgfpathlineto{\pgfqpoint{0.891958in}{0.461974in}}%
\pgfpathlineto{\pgfqpoint{0.896253in}{0.468717in}}%
\pgfpathlineto{\pgfqpoint{0.897327in}{0.468057in}}%
\pgfpathlineto{\pgfqpoint{0.898401in}{0.464099in}}%
\pgfpathlineto{\pgfqpoint{0.899475in}{0.464686in}}%
\pgfpathlineto{\pgfqpoint{0.902696in}{0.462560in}}%
\pgfpathlineto{\pgfqpoint{0.903770in}{0.459629in}}%
\pgfpathlineto{\pgfqpoint{0.904844in}{0.458896in}}%
\pgfpathlineto{\pgfqpoint{0.906992in}{0.470182in}}%
\pgfpathlineto{\pgfqpoint{0.910213in}{0.472967in}}%
\pgfpathlineto{\pgfqpoint{0.912361in}{0.467471in}}%
\pgfpathlineto{\pgfqpoint{0.914509in}{0.473187in}}%
\pgfpathlineto{\pgfqpoint{0.917730in}{0.473260in}}%
\pgfpathlineto{\pgfqpoint{0.918804in}{0.472234in}}%
\pgfpathlineto{\pgfqpoint{0.919878in}{0.473480in}}%
\pgfpathlineto{\pgfqpoint{0.920952in}{0.469156in}}%
\pgfpathlineto{\pgfqpoint{0.922025in}{0.470109in}}%
\pgfpathlineto{\pgfqpoint{0.925247in}{0.468863in}}%
\pgfpathlineto{\pgfqpoint{0.926321in}{0.471428in}}%
\pgfpathlineto{\pgfqpoint{0.927395in}{0.471501in}}%
\pgfpathlineto{\pgfqpoint{0.928468in}{0.472527in}}%
\pgfpathlineto{\pgfqpoint{0.929542in}{0.470842in}}%
\pgfpathlineto{\pgfqpoint{0.932764in}{0.470036in}}%
\pgfpathlineto{\pgfqpoint{0.933838in}{0.468936in}}%
\pgfpathlineto{\pgfqpoint{0.934911in}{0.469889in}}%
\pgfpathlineto{\pgfqpoint{0.937059in}{0.467544in}}%
\pgfpathlineto{\pgfqpoint{0.940281in}{0.469156in}}%
\pgfpathlineto{\pgfqpoint{0.941354in}{0.468790in}}%
\pgfpathlineto{\pgfqpoint{0.942428in}{0.469743in}}%
\pgfpathlineto{\pgfqpoint{0.943502in}{0.467617in}}%
\pgfpathlineto{\pgfqpoint{0.944576in}{0.470182in}}%
\pgfpathlineto{\pgfqpoint{0.948871in}{0.470622in}}%
\pgfpathlineto{\pgfqpoint{0.949945in}{0.469083in}}%
\pgfpathlineto{\pgfqpoint{0.951019in}{0.472234in}}%
\pgfpathlineto{\pgfqpoint{0.952093in}{0.472527in}}%
\pgfpathlineto{\pgfqpoint{0.955314in}{0.470842in}}%
\pgfpathlineto{\pgfqpoint{0.957462in}{0.478684in}}%
\pgfpathlineto{\pgfqpoint{0.958536in}{0.481175in}}%
\pgfpathlineto{\pgfqpoint{0.959610in}{0.480003in}}%
\pgfpathlineto{\pgfqpoint{0.963905in}{0.479050in}}%
\pgfpathlineto{\pgfqpoint{0.964979in}{0.480882in}}%
\pgfpathlineto{\pgfqpoint{0.966053in}{0.480589in}}%
\pgfpathlineto{\pgfqpoint{0.967127in}{0.481249in}}%
\pgfpathlineto{\pgfqpoint{0.970348in}{0.479123in}}%
\pgfpathlineto{\pgfqpoint{0.971422in}{0.480516in}}%
\pgfpathlineto{\pgfqpoint{0.972496in}{0.477145in}}%
\pgfpathlineto{\pgfqpoint{0.973570in}{0.479416in}}%
\pgfpathlineto{\pgfqpoint{0.974643in}{0.480076in}}%
\pgfpathlineto{\pgfqpoint{0.977865in}{0.484034in}}%
\pgfpathlineto{\pgfqpoint{0.978939in}{0.482861in}}%
\pgfpathlineto{\pgfqpoint{0.980013in}{0.486965in}}%
\pgfpathlineto{\pgfqpoint{0.982160in}{0.489823in}}%
\pgfpathlineto{\pgfqpoint{0.985382in}{0.487331in}}%
\pgfpathlineto{\pgfqpoint{0.986456in}{0.484107in}}%
\pgfpathlineto{\pgfqpoint{0.987530in}{0.485206in}}%
\pgfpathlineto{\pgfqpoint{0.988603in}{0.487844in}}%
\pgfpathlineto{\pgfqpoint{0.989677in}{0.487991in}}%
\pgfpathlineto{\pgfqpoint{0.992899in}{0.489237in}}%
\pgfpathlineto{\pgfqpoint{0.995046in}{0.493634in}}%
\pgfpathlineto{\pgfqpoint{0.996120in}{0.492535in}}%
\pgfpathlineto{\pgfqpoint{0.997194in}{0.489383in}}%
\pgfpathlineto{\pgfqpoint{1.000416in}{0.487771in}}%
\pgfpathlineto{\pgfqpoint{1.001489in}{0.483887in}}%
\pgfpathlineto{\pgfqpoint{1.002563in}{0.483667in}}%
\pgfpathlineto{\pgfqpoint{1.004711in}{0.486672in}}%
\pgfpathlineto{\pgfqpoint{1.010080in}{0.487405in}}%
\pgfpathlineto{\pgfqpoint{1.011154in}{0.495759in}}%
\pgfpathlineto{\pgfqpoint{1.015449in}{0.491875in}}%
\pgfpathlineto{\pgfqpoint{1.016523in}{0.495759in}}%
\pgfpathlineto{\pgfqpoint{1.017597in}{0.493634in}}%
\pgfpathlineto{\pgfqpoint{1.018671in}{0.492828in}}%
\pgfpathlineto{\pgfqpoint{1.019745in}{0.494220in}}%
\pgfpathlineto{\pgfqpoint{1.022966in}{0.494733in}}%
\pgfpathlineto{\pgfqpoint{1.026188in}{0.491582in}}%
\pgfpathlineto{\pgfqpoint{1.027262in}{0.496053in}}%
\pgfpathlineto{\pgfqpoint{1.031557in}{0.502062in}}%
\pgfpathlineto{\pgfqpoint{1.032631in}{0.502575in}}%
\pgfpathlineto{\pgfqpoint{1.034778in}{0.504994in}}%
\pgfpathlineto{\pgfqpoint{1.039074in}{0.502795in}}%
\pgfpathlineto{\pgfqpoint{1.041221in}{0.505067in}}%
\pgfpathlineto{\pgfqpoint{1.042295in}{0.507632in}}%
\pgfpathlineto{\pgfqpoint{1.045517in}{0.505873in}}%
\pgfpathlineto{\pgfqpoint{1.046591in}{0.504114in}}%
\pgfpathlineto{\pgfqpoint{1.048738in}{0.505580in}}%
\pgfpathlineto{\pgfqpoint{1.054108in}{0.506313in}}%
\pgfpathlineto{\pgfqpoint{1.056255in}{0.503455in}}%
\pgfpathlineto{\pgfqpoint{1.057329in}{0.502868in}}%
\pgfpathlineto{\pgfqpoint{1.060551in}{0.506679in}}%
\pgfpathlineto{\pgfqpoint{1.061624in}{0.509318in}}%
\pgfpathlineto{\pgfqpoint{1.062698in}{0.506753in}}%
\pgfpathlineto{\pgfqpoint{1.063772in}{0.512176in}}%
\pgfpathlineto{\pgfqpoint{1.064846in}{0.509391in}}%
\pgfpathlineto{\pgfqpoint{1.068067in}{0.510050in}}%
\pgfpathlineto{\pgfqpoint{1.072363in}{0.505873in}}%
\pgfpathlineto{\pgfqpoint{1.075584in}{0.510710in}}%
\pgfpathlineto{\pgfqpoint{1.077732in}{0.517013in}}%
\pgfpathlineto{\pgfqpoint{1.078806in}{0.517233in}}%
\pgfpathlineto{\pgfqpoint{1.079880in}{0.519285in}}%
\pgfpathlineto{\pgfqpoint{1.083101in}{0.521190in}}%
\pgfpathlineto{\pgfqpoint{1.084175in}{0.523609in}}%
\pgfpathlineto{\pgfqpoint{1.085249in}{0.527639in}}%
\pgfpathlineto{\pgfqpoint{1.086323in}{0.525514in}}%
\pgfpathlineto{\pgfqpoint{1.087397in}{0.526540in}}%
\pgfpathlineto{\pgfqpoint{1.093840in}{0.524488in}}%
\pgfpathlineto{\pgfqpoint{1.094913in}{0.521776in}}%
\pgfpathlineto{\pgfqpoint{1.102430in}{0.524415in}}%
\pgfpathlineto{\pgfqpoint{1.105652in}{0.518625in}}%
\pgfpathlineto{\pgfqpoint{1.106726in}{0.519138in}}%
\pgfpathlineto{\pgfqpoint{1.107799in}{0.516426in}}%
\pgfpathlineto{\pgfqpoint{1.108873in}{0.521190in}}%
\pgfpathlineto{\pgfqpoint{1.109947in}{0.522289in}}%
\pgfpathlineto{\pgfqpoint{1.113169in}{0.519138in}}%
\pgfpathlineto{\pgfqpoint{1.114242in}{0.523902in}}%
\pgfpathlineto{\pgfqpoint{1.115316in}{0.526027in}}%
\pgfpathlineto{\pgfqpoint{1.116390in}{0.519798in}}%
\pgfpathlineto{\pgfqpoint{1.117464in}{0.521043in}}%
\pgfpathlineto{\pgfqpoint{1.120686in}{0.518552in}}%
\pgfpathlineto{\pgfqpoint{1.121759in}{0.519358in}}%
\pgfpathlineto{\pgfqpoint{1.122833in}{0.518185in}}%
\pgfpathlineto{\pgfqpoint{1.123907in}{0.520164in}}%
\pgfpathlineto{\pgfqpoint{1.124981in}{0.523535in}}%
\pgfpathlineto{\pgfqpoint{1.129276in}{0.522949in}}%
\pgfpathlineto{\pgfqpoint{1.130350in}{0.519285in}}%
\pgfpathlineto{\pgfqpoint{1.132498in}{0.525954in}}%
\pgfpathlineto{\pgfqpoint{1.135719in}{0.520017in}}%
\pgfpathlineto{\pgfqpoint{1.137867in}{0.526687in}}%
\pgfpathlineto{\pgfqpoint{1.140015in}{0.523682in}}%
\pgfpathlineto{\pgfqpoint{1.144310in}{0.527419in}}%
\pgfpathlineto{\pgfqpoint{1.145384in}{0.528665in}}%
\pgfpathlineto{\pgfqpoint{1.146458in}{0.527786in}}%
\pgfpathlineto{\pgfqpoint{1.150753in}{0.528885in}}%
\pgfpathlineto{\pgfqpoint{1.151827in}{0.526540in}}%
\pgfpathlineto{\pgfqpoint{1.152901in}{0.525880in}}%
\pgfpathlineto{\pgfqpoint{1.153975in}{0.528006in}}%
\pgfpathlineto{\pgfqpoint{1.155048in}{0.524488in}}%
\pgfpathlineto{\pgfqpoint{1.158270in}{0.523682in}}%
\pgfpathlineto{\pgfqpoint{1.159344in}{0.520237in}}%
\pgfpathlineto{\pgfqpoint{1.160418in}{0.525807in}}%
\pgfpathlineto{\pgfqpoint{1.161491in}{0.523242in}}%
\pgfpathlineto{\pgfqpoint{1.162565in}{0.527053in}}%
\pgfpathlineto{\pgfqpoint{1.165787in}{0.533576in}}%
\pgfpathlineto{\pgfqpoint{1.166861in}{0.539365in}}%
\pgfpathlineto{\pgfqpoint{1.169008in}{0.543323in}}%
\pgfpathlineto{\pgfqpoint{1.173304in}{0.539145in}}%
\pgfpathlineto{\pgfqpoint{1.174377in}{0.540171in}}%
\pgfpathlineto{\pgfqpoint{1.175451in}{0.534821in}}%
\pgfpathlineto{\pgfqpoint{1.176525in}{0.537680in}}%
\pgfpathlineto{\pgfqpoint{1.177599in}{0.535481in}}%
\pgfpathlineto{\pgfqpoint{1.180820in}{0.537386in}}%
\pgfpathlineto{\pgfqpoint{1.181894in}{0.534821in}}%
\pgfpathlineto{\pgfqpoint{1.182968in}{0.538559in}}%
\pgfpathlineto{\pgfqpoint{1.184042in}{0.539585in}}%
\pgfpathlineto{\pgfqpoint{1.185116in}{0.536507in}}%
\pgfpathlineto{\pgfqpoint{1.188337in}{0.529032in}}%
\pgfpathlineto{\pgfqpoint{1.189411in}{0.534748in}}%
\pgfpathlineto{\pgfqpoint{1.190485in}{0.530644in}}%
\pgfpathlineto{\pgfqpoint{1.191559in}{0.529252in}}%
\pgfpathlineto{\pgfqpoint{1.192633in}{0.533502in}}%
\pgfpathlineto{\pgfqpoint{1.195854in}{0.532769in}}%
\pgfpathlineto{\pgfqpoint{1.198002in}{0.538706in}}%
\pgfpathlineto{\pgfqpoint{1.199076in}{0.541491in}}%
\pgfpathlineto{\pgfqpoint{1.200150in}{0.538999in}}%
\pgfpathlineto{\pgfqpoint{1.204445in}{0.541051in}}%
\pgfpathlineto{\pgfqpoint{1.205519in}{0.536947in}}%
\pgfpathlineto{\pgfqpoint{1.206593in}{0.552630in}}%
\pgfpathlineto{\pgfqpoint{1.207666in}{0.559519in}}%
\pgfpathlineto{\pgfqpoint{1.210888in}{0.558420in}}%
\pgfpathlineto{\pgfqpoint{1.211962in}{0.559959in}}%
\pgfpathlineto{\pgfqpoint{1.214109in}{0.557907in}}%
\pgfpathlineto{\pgfqpoint{1.215183in}{0.558200in}}%
\pgfpathlineto{\pgfqpoint{1.218405in}{0.558493in}}%
\pgfpathlineto{\pgfqpoint{1.219479in}{0.560838in}}%
\pgfpathlineto{\pgfqpoint{1.220552in}{0.565455in}}%
\pgfpathlineto{\pgfqpoint{1.221626in}{0.562157in}}%
\pgfpathlineto{\pgfqpoint{1.222700in}{0.570146in}}%
\pgfpathlineto{\pgfqpoint{1.225922in}{0.565162in}}%
\pgfpathlineto{\pgfqpoint{1.226996in}{0.565089in}}%
\pgfpathlineto{\pgfqpoint{1.229143in}{0.559299in}}%
\pgfpathlineto{\pgfqpoint{1.230217in}{0.563257in}}%
\pgfpathlineto{\pgfqpoint{1.234512in}{0.562524in}}%
\pgfpathlineto{\pgfqpoint{1.235586in}{0.559373in}}%
\pgfpathlineto{\pgfqpoint{1.236660in}{0.564209in}}%
\pgfpathlineto{\pgfqpoint{1.237734in}{0.559373in}}%
\pgfpathlineto{\pgfqpoint{1.240955in}{0.562817in}}%
\pgfpathlineto{\pgfqpoint{1.242029in}{0.562817in}}%
\pgfpathlineto{\pgfqpoint{1.243103in}{0.557834in}}%
\pgfpathlineto{\pgfqpoint{1.244177in}{0.561425in}}%
\pgfpathlineto{\pgfqpoint{1.245251in}{0.562377in}}%
\pgfpathlineto{\pgfqpoint{1.248472in}{0.566408in}}%
\pgfpathlineto{\pgfqpoint{1.249546in}{0.561938in}}%
\pgfpathlineto{\pgfqpoint{1.250620in}{0.561058in}}%
\pgfpathlineto{\pgfqpoint{1.251694in}{0.566262in}}%
\pgfpathlineto{\pgfqpoint{1.252768in}{0.564063in}}%
\pgfpathlineto{\pgfqpoint{1.255989in}{0.566408in}}%
\pgfpathlineto{\pgfqpoint{1.257063in}{0.569266in}}%
\pgfpathlineto{\pgfqpoint{1.258137in}{0.566921in}}%
\pgfpathlineto{\pgfqpoint{1.259211in}{0.559886in}}%
\pgfpathlineto{\pgfqpoint{1.260285in}{0.561644in}}%
\pgfpathlineto{\pgfqpoint{1.263506in}{0.559592in}}%
\pgfpathlineto{\pgfqpoint{1.266728in}{0.569633in}}%
\pgfpathlineto{\pgfqpoint{1.267801in}{0.567141in}}%
\pgfpathlineto{\pgfqpoint{1.271023in}{0.571245in}}%
\pgfpathlineto{\pgfqpoint{1.272097in}{0.571318in}}%
\pgfpathlineto{\pgfqpoint{1.275318in}{0.580626in}}%
\pgfpathlineto{\pgfqpoint{1.280687in}{0.573737in}}%
\pgfpathlineto{\pgfqpoint{1.281761in}{0.579233in}}%
\pgfpathlineto{\pgfqpoint{1.282835in}{0.580479in}}%
\pgfpathlineto{\pgfqpoint{1.286057in}{0.579746in}}%
\pgfpathlineto{\pgfqpoint{1.287130in}{0.578281in}}%
\pgfpathlineto{\pgfqpoint{1.288204in}{0.578720in}}%
\pgfpathlineto{\pgfqpoint{1.289278in}{0.580992in}}%
\pgfpathlineto{\pgfqpoint{1.290352in}{0.579160in}}%
\pgfpathlineto{\pgfqpoint{1.293574in}{0.581359in}}%
\pgfpathlineto{\pgfqpoint{1.295721in}{0.573810in}}%
\pgfpathlineto{\pgfqpoint{1.296795in}{0.587075in}}%
\pgfpathlineto{\pgfqpoint{1.297869in}{0.584657in}}%
\pgfpathlineto{\pgfqpoint{1.302164in}{0.581725in}}%
\pgfpathlineto{\pgfqpoint{1.303238in}{0.557467in}}%
\pgfpathlineto{\pgfqpoint{1.304312in}{0.561058in}}%
\pgfpathlineto{\pgfqpoint{1.305386in}{0.569193in}}%
\pgfpathlineto{\pgfqpoint{1.308607in}{0.570146in}}%
\pgfpathlineto{\pgfqpoint{1.310755in}{0.564796in}}%
\pgfpathlineto{\pgfqpoint{1.312903in}{0.562157in}}%
\pgfpathlineto{\pgfqpoint{1.316124in}{0.562084in}}%
\pgfpathlineto{\pgfqpoint{1.317198in}{0.561205in}}%
\pgfpathlineto{\pgfqpoint{1.318272in}{0.561938in}}%
\pgfpathlineto{\pgfqpoint{1.319346in}{0.554462in}}%
\pgfpathlineto{\pgfqpoint{1.320419in}{0.553070in}}%
\pgfpathlineto{\pgfqpoint{1.323641in}{0.556221in}}%
\pgfpathlineto{\pgfqpoint{1.324715in}{0.553216in}}%
\pgfpathlineto{\pgfqpoint{1.325789in}{0.562011in}}%
\pgfpathlineto{\pgfqpoint{1.327936in}{0.563183in}}%
\pgfpathlineto{\pgfqpoint{1.332232in}{0.554829in}}%
\pgfpathlineto{\pgfqpoint{1.333306in}{0.556588in}}%
\pgfpathlineto{\pgfqpoint{1.334379in}{0.556881in}}%
\pgfpathlineto{\pgfqpoint{1.335453in}{0.555268in}}%
\pgfpathlineto{\pgfqpoint{1.339749in}{0.559666in}}%
\pgfpathlineto{\pgfqpoint{1.340822in}{0.558273in}}%
\pgfpathlineto{\pgfqpoint{1.341896in}{0.558347in}}%
\pgfpathlineto{\pgfqpoint{1.342970in}{0.559079in}}%
\pgfpathlineto{\pgfqpoint{1.346192in}{0.562304in}}%
\pgfpathlineto{\pgfqpoint{1.347265in}{0.572491in}}%
\pgfpathlineto{\pgfqpoint{1.348339in}{0.575422in}}%
\pgfpathlineto{\pgfqpoint{1.349413in}{0.573297in}}%
\pgfpathlineto{\pgfqpoint{1.350487in}{0.579966in}}%
\pgfpathlineto{\pgfqpoint{1.353708in}{0.580626in}}%
\pgfpathlineto{\pgfqpoint{1.355856in}{0.587881in}}%
\pgfpathlineto{\pgfqpoint{1.356930in}{0.589640in}}%
\pgfpathlineto{\pgfqpoint{1.358004in}{0.596602in}}%
\pgfpathlineto{\pgfqpoint{1.361225in}{0.592205in}}%
\pgfpathlineto{\pgfqpoint{1.363373in}{0.584290in}}%
\pgfpathlineto{\pgfqpoint{1.364447in}{0.587661in}}%
\pgfpathlineto{\pgfqpoint{1.365521in}{0.586855in}}%
\pgfpathlineto{\pgfqpoint{1.368742in}{0.583557in}}%
\pgfpathlineto{\pgfqpoint{1.369816in}{0.587148in}}%
\pgfpathlineto{\pgfqpoint{1.370890in}{0.584730in}}%
\pgfpathlineto{\pgfqpoint{1.371964in}{0.579380in}}%
\pgfpathlineto{\pgfqpoint{1.373038in}{0.582458in}}%
\pgfpathlineto{\pgfqpoint{1.376259in}{0.575422in}}%
\pgfpathlineto{\pgfqpoint{1.377333in}{0.569046in}}%
\pgfpathlineto{\pgfqpoint{1.378407in}{0.571245in}}%
\pgfpathlineto{\pgfqpoint{1.380554in}{0.585389in}}%
\pgfpathlineto{\pgfqpoint{1.383776in}{0.587515in}}%
\pgfpathlineto{\pgfqpoint{1.384850in}{0.583997in}}%
\pgfpathlineto{\pgfqpoint{1.386997in}{0.595723in}}%
\pgfpathlineto{\pgfqpoint{1.388071in}{0.599387in}}%
\pgfpathlineto{\pgfqpoint{1.392367in}{0.598581in}}%
\pgfpathlineto{\pgfqpoint{1.393440in}{0.596749in}}%
\pgfpathlineto{\pgfqpoint{1.394514in}{0.603565in}}%
\pgfpathlineto{\pgfqpoint{1.395588in}{0.603784in}}%
\pgfpathlineto{\pgfqpoint{1.398810in}{0.603784in}}%
\pgfpathlineto{\pgfqpoint{1.399884in}{0.605763in}}%
\pgfpathlineto{\pgfqpoint{1.400957in}{0.605104in}}%
\pgfpathlineto{\pgfqpoint{1.402031in}{0.592938in}}%
\pgfpathlineto{\pgfqpoint{1.403105in}{0.597189in}}%
\pgfpathlineto{\pgfqpoint{1.406327in}{0.592572in}}%
\pgfpathlineto{\pgfqpoint{1.407400in}{0.594184in}}%
\pgfpathlineto{\pgfqpoint{1.408474in}{0.596969in}}%
\pgfpathlineto{\pgfqpoint{1.409548in}{0.591985in}}%
\pgfpathlineto{\pgfqpoint{1.410622in}{0.595723in}}%
\pgfpathlineto{\pgfqpoint{1.413843in}{0.597335in}}%
\pgfpathlineto{\pgfqpoint{1.414917in}{0.595650in}}%
\pgfpathlineto{\pgfqpoint{1.415991in}{0.601000in}}%
\pgfpathlineto{\pgfqpoint{1.417065in}{0.601586in}}%
\pgfpathlineto{\pgfqpoint{1.418139in}{0.604737in}}%
\pgfpathlineto{\pgfqpoint{1.421360in}{0.601219in}}%
\pgfpathlineto{\pgfqpoint{1.422434in}{0.596895in}}%
\pgfpathlineto{\pgfqpoint{1.423508in}{0.598141in}}%
\pgfpathlineto{\pgfqpoint{1.424582in}{0.604078in}}%
\pgfpathlineto{\pgfqpoint{1.425656in}{0.604957in}}%
\pgfpathlineto{\pgfqpoint{1.428877in}{0.604664in}}%
\pgfpathlineto{\pgfqpoint{1.429951in}{0.607376in}}%
\pgfpathlineto{\pgfqpoint{1.431025in}{0.608182in}}%
\pgfpathlineto{\pgfqpoint{1.433173in}{0.607229in}}%
\pgfpathlineto{\pgfqpoint{1.436394in}{0.610087in}}%
\pgfpathlineto{\pgfqpoint{1.437468in}{0.604371in}}%
\pgfpathlineto{\pgfqpoint{1.438542in}{0.606056in}}%
\pgfpathlineto{\pgfqpoint{1.439616in}{0.604297in}}%
\pgfpathlineto{\pgfqpoint{1.440689in}{0.604517in}}%
\pgfpathlineto{\pgfqpoint{1.443911in}{0.604078in}}%
\pgfpathlineto{\pgfqpoint{1.444985in}{0.600413in}}%
\pgfpathlineto{\pgfqpoint{1.446059in}{0.611040in}}%
\pgfpathlineto{\pgfqpoint{1.447132in}{0.607229in}}%
\pgfpathlineto{\pgfqpoint{1.448206in}{0.613898in}}%
\pgfpathlineto{\pgfqpoint{1.451428in}{0.614558in}}%
\pgfpathlineto{\pgfqpoint{1.452502in}{0.624012in}}%
\pgfpathlineto{\pgfqpoint{1.453575in}{0.627529in}}%
\pgfpathlineto{\pgfqpoint{1.454649in}{0.628775in}}%
\pgfpathlineto{\pgfqpoint{1.455723in}{0.628629in}}%
\pgfpathlineto{\pgfqpoint{1.462166in}{0.635811in}}%
\pgfpathlineto{\pgfqpoint{1.463240in}{0.634931in}}%
\pgfpathlineto{\pgfqpoint{1.466462in}{0.637057in}}%
\pgfpathlineto{\pgfqpoint{1.467535in}{0.640062in}}%
\pgfpathlineto{\pgfqpoint{1.469683in}{0.637203in}}%
\pgfpathlineto{\pgfqpoint{1.470757in}{0.637496in}}%
\pgfpathlineto{\pgfqpoint{1.473978in}{0.635225in}}%
\pgfpathlineto{\pgfqpoint{1.475052in}{0.638083in}}%
\pgfpathlineto{\pgfqpoint{1.476126in}{0.639255in}}%
\pgfpathlineto{\pgfqpoint{1.477200in}{0.638742in}}%
\pgfpathlineto{\pgfqpoint{1.481495in}{0.633612in}}%
\pgfpathlineto{\pgfqpoint{1.482569in}{0.640062in}}%
\pgfpathlineto{\pgfqpoint{1.483643in}{0.641894in}}%
\pgfpathlineto{\pgfqpoint{1.484717in}{0.638522in}}%
\pgfpathlineto{\pgfqpoint{1.485791in}{0.656331in}}%
\pgfpathlineto{\pgfqpoint{1.490086in}{0.655891in}}%
\pgfpathlineto{\pgfqpoint{1.491160in}{0.657650in}}%
\pgfpathlineto{\pgfqpoint{1.493307in}{0.637643in}}%
\pgfpathlineto{\pgfqpoint{1.496529in}{0.629068in}}%
\pgfpathlineto{\pgfqpoint{1.497603in}{0.637130in}}%
\pgfpathlineto{\pgfqpoint{1.498677in}{0.630607in}}%
\pgfpathlineto{\pgfqpoint{1.499751in}{0.636983in}}%
\pgfpathlineto{\pgfqpoint{1.500824in}{0.627676in}}%
\pgfpathlineto{\pgfqpoint{1.504046in}{0.624378in}}%
\pgfpathlineto{\pgfqpoint{1.506194in}{0.628042in}}%
\pgfpathlineto{\pgfqpoint{1.508341in}{0.638522in}}%
\pgfpathlineto{\pgfqpoint{1.511563in}{0.636470in}}%
\pgfpathlineto{\pgfqpoint{1.512637in}{0.639475in}}%
\pgfpathlineto{\pgfqpoint{1.513710in}{0.645411in}}%
\pgfpathlineto{\pgfqpoint{1.514784in}{0.645192in}}%
\pgfpathlineto{\pgfqpoint{1.515858in}{0.648563in}}%
\pgfpathlineto{\pgfqpoint{1.520153in}{0.648636in}}%
\pgfpathlineto{\pgfqpoint{1.521227in}{0.644898in}}%
\pgfpathlineto{\pgfqpoint{1.523375in}{0.644019in}}%
\pgfpathlineto{\pgfqpoint{1.527670in}{0.650468in}}%
\pgfpathlineto{\pgfqpoint{1.528744in}{0.648709in}}%
\pgfpathlineto{\pgfqpoint{1.529818in}{0.649003in}}%
\pgfpathlineto{\pgfqpoint{1.530892in}{0.648416in}}%
\pgfpathlineto{\pgfqpoint{1.534113in}{0.640721in}}%
\pgfpathlineto{\pgfqpoint{1.535187in}{0.647683in}}%
\pgfpathlineto{\pgfqpoint{1.536261in}{0.643066in}}%
\pgfpathlineto{\pgfqpoint{1.537335in}{0.644825in}}%
\pgfpathlineto{\pgfqpoint{1.538409in}{0.647757in}}%
\pgfpathlineto{\pgfqpoint{1.541630in}{0.647683in}}%
\pgfpathlineto{\pgfqpoint{1.542704in}{0.650542in}}%
\pgfpathlineto{\pgfqpoint{1.543778in}{0.648709in}}%
\pgfpathlineto{\pgfqpoint{1.544852in}{0.639549in}}%
\pgfpathlineto{\pgfqpoint{1.545926in}{0.639549in}}%
\pgfpathlineto{\pgfqpoint{1.549147in}{0.644752in}}%
\pgfpathlineto{\pgfqpoint{1.550221in}{0.649222in}}%
\pgfpathlineto{\pgfqpoint{1.552369in}{0.641381in}}%
\pgfpathlineto{\pgfqpoint{1.553442in}{0.644019in}}%
\pgfpathlineto{\pgfqpoint{1.556664in}{0.639549in}}%
\pgfpathlineto{\pgfqpoint{1.558812in}{0.630827in}}%
\pgfpathlineto{\pgfqpoint{1.559885in}{0.630974in}}%
\pgfpathlineto{\pgfqpoint{1.560959in}{0.624598in}}%
\pgfpathlineto{\pgfqpoint{1.564181in}{0.631121in}}%
\pgfpathlineto{\pgfqpoint{1.565255in}{0.629142in}}%
\pgfpathlineto{\pgfqpoint{1.566328in}{0.629068in}}%
\pgfpathlineto{\pgfqpoint{1.567402in}{0.629655in}}%
\pgfpathlineto{\pgfqpoint{1.568476in}{0.617049in}}%
\pgfpathlineto{\pgfqpoint{1.572772in}{0.608255in}}%
\pgfpathlineto{\pgfqpoint{1.573845in}{0.616830in}}%
\pgfpathlineto{\pgfqpoint{1.575993in}{0.598068in}}%
\pgfpathlineto{\pgfqpoint{1.579215in}{0.605543in}}%
\pgfpathlineto{\pgfqpoint{1.580288in}{0.610820in}}%
\pgfpathlineto{\pgfqpoint{1.581362in}{0.619908in}}%
\pgfpathlineto{\pgfqpoint{1.582436in}{0.617489in}}%
\pgfpathlineto{\pgfqpoint{1.586731in}{0.619541in}}%
\pgfpathlineto{\pgfqpoint{1.587805in}{0.621007in}}%
\pgfpathlineto{\pgfqpoint{1.588879in}{0.619028in}}%
\pgfpathlineto{\pgfqpoint{1.589953in}{0.619981in}}%
\pgfpathlineto{\pgfqpoint{1.591027in}{0.602026in}}%
\pgfpathlineto{\pgfqpoint{1.596396in}{0.608328in}}%
\pgfpathlineto{\pgfqpoint{1.597470in}{0.614338in}}%
\pgfpathlineto{\pgfqpoint{1.598544in}{0.611406in}}%
\pgfpathlineto{\pgfqpoint{1.601765in}{0.616097in}}%
\pgfpathlineto{\pgfqpoint{1.602839in}{0.613019in}}%
\pgfpathlineto{\pgfqpoint{1.604987in}{0.622546in}}%
\pgfpathlineto{\pgfqpoint{1.606061in}{0.622399in}}%
\pgfpathlineto{\pgfqpoint{1.610356in}{0.624598in}}%
\pgfpathlineto{\pgfqpoint{1.611430in}{0.623499in}}%
\pgfpathlineto{\pgfqpoint{1.612504in}{0.619321in}}%
\pgfpathlineto{\pgfqpoint{1.613577in}{0.623425in}}%
\pgfpathlineto{\pgfqpoint{1.616799in}{0.624378in}}%
\pgfpathlineto{\pgfqpoint{1.617873in}{0.620201in}}%
\pgfpathlineto{\pgfqpoint{1.618947in}{0.623938in}}%
\pgfpathlineto{\pgfqpoint{1.620020in}{0.622692in}}%
\pgfpathlineto{\pgfqpoint{1.621094in}{0.627383in}}%
\pgfpathlineto{\pgfqpoint{1.625390in}{0.631560in}}%
\pgfpathlineto{\pgfqpoint{1.626463in}{0.630388in}}%
\pgfpathlineto{\pgfqpoint{1.627537in}{0.631853in}}%
\pgfpathlineto{\pgfqpoint{1.628611in}{0.632147in}}%
\pgfpathlineto{\pgfqpoint{1.631833in}{0.629875in}}%
\pgfpathlineto{\pgfqpoint{1.632906in}{0.626064in}}%
\pgfpathlineto{\pgfqpoint{1.633980in}{0.626284in}}%
\pgfpathlineto{\pgfqpoint{1.636128in}{0.628995in}}%
\pgfpathlineto{\pgfqpoint{1.639350in}{0.628262in}}%
\pgfpathlineto{\pgfqpoint{1.640423in}{0.631121in}}%
\pgfpathlineto{\pgfqpoint{1.642571in}{0.626870in}}%
\pgfpathlineto{\pgfqpoint{1.643645in}{0.625990in}}%
\pgfpathlineto{\pgfqpoint{1.646866in}{0.624158in}}%
\pgfpathlineto{\pgfqpoint{1.649014in}{0.625917in}}%
\pgfpathlineto{\pgfqpoint{1.651162in}{0.622912in}}%
\pgfpathlineto{\pgfqpoint{1.654383in}{0.622839in}}%
\pgfpathlineto{\pgfqpoint{1.655457in}{0.619981in}}%
\pgfpathlineto{\pgfqpoint{1.656531in}{0.622179in}}%
\pgfpathlineto{\pgfqpoint{1.657605in}{0.621960in}}%
\pgfpathlineto{\pgfqpoint{1.658679in}{0.622546in}}%
\pgfpathlineto{\pgfqpoint{1.661900in}{0.625038in}}%
\pgfpathlineto{\pgfqpoint{1.662974in}{0.631121in}}%
\pgfpathlineto{\pgfqpoint{1.664048in}{0.632147in}}%
\pgfpathlineto{\pgfqpoint{1.665122in}{0.635078in}}%
\pgfpathlineto{\pgfqpoint{1.669417in}{0.635444in}}%
\pgfpathlineto{\pgfqpoint{1.670491in}{0.632879in}}%
\pgfpathlineto{\pgfqpoint{1.671565in}{0.634418in}}%
\pgfpathlineto{\pgfqpoint{1.672639in}{0.633392in}}%
\pgfpathlineto{\pgfqpoint{1.676934in}{0.642920in}}%
\pgfpathlineto{\pgfqpoint{1.679082in}{0.645705in}}%
\pgfpathlineto{\pgfqpoint{1.680155in}{0.637423in}}%
\pgfpathlineto{\pgfqpoint{1.681229in}{0.641454in}}%
\pgfpathlineto{\pgfqpoint{1.684451in}{0.639768in}}%
\pgfpathlineto{\pgfqpoint{1.685525in}{0.643286in}}%
\pgfpathlineto{\pgfqpoint{1.686598in}{0.643213in}}%
\pgfpathlineto{\pgfqpoint{1.687672in}{0.645851in}}%
\pgfpathlineto{\pgfqpoint{1.688746in}{0.632073in}}%
\pgfpathlineto{\pgfqpoint{1.691968in}{0.631121in}}%
\pgfpathlineto{\pgfqpoint{1.693041in}{0.629801in}}%
\pgfpathlineto{\pgfqpoint{1.694115in}{0.630827in}}%
\pgfpathlineto{\pgfqpoint{1.695189in}{0.625551in}}%
\pgfpathlineto{\pgfqpoint{1.696263in}{0.626943in}}%
\pgfpathlineto{\pgfqpoint{1.699484in}{0.627529in}}%
\pgfpathlineto{\pgfqpoint{1.700558in}{0.624745in}}%
\pgfpathlineto{\pgfqpoint{1.701632in}{0.624964in}}%
\pgfpathlineto{\pgfqpoint{1.702706in}{0.622106in}}%
\pgfpathlineto{\pgfqpoint{1.703780in}{0.624598in}}%
\pgfpathlineto{\pgfqpoint{1.708075in}{0.624305in}}%
\pgfpathlineto{\pgfqpoint{1.709149in}{0.629068in}}%
\pgfpathlineto{\pgfqpoint{1.710223in}{0.630974in}}%
\pgfpathlineto{\pgfqpoint{1.711297in}{0.626870in}}%
\pgfpathlineto{\pgfqpoint{1.715592in}{0.635298in}}%
\pgfpathlineto{\pgfqpoint{1.716666in}{0.637423in}}%
\pgfpathlineto{\pgfqpoint{1.717740in}{0.636617in}}%
\pgfpathlineto{\pgfqpoint{1.718814in}{0.637203in}}%
\pgfpathlineto{\pgfqpoint{1.722035in}{0.637057in}}%
\pgfpathlineto{\pgfqpoint{1.724183in}{0.639035in}}%
\pgfpathlineto{\pgfqpoint{1.726330in}{0.630974in}}%
\pgfpathlineto{\pgfqpoint{1.731700in}{0.635151in}}%
\pgfpathlineto{\pgfqpoint{1.732773in}{0.633979in}}%
\pgfpathlineto{\pgfqpoint{1.733847in}{0.633905in}}%
\pgfpathlineto{\pgfqpoint{1.737069in}{0.636617in}}%
\pgfpathlineto{\pgfqpoint{1.738143in}{0.634052in}}%
\pgfpathlineto{\pgfqpoint{1.739216in}{0.638596in}}%
\pgfpathlineto{\pgfqpoint{1.741364in}{0.633612in}}%
\pgfpathlineto{\pgfqpoint{1.744586in}{0.634638in}}%
\pgfpathlineto{\pgfqpoint{1.745660in}{0.639329in}}%
\pgfpathlineto{\pgfqpoint{1.746733in}{0.636251in}}%
\pgfpathlineto{\pgfqpoint{1.747807in}{0.637790in}}%
\pgfpathlineto{\pgfqpoint{1.748881in}{0.637496in}}%
\pgfpathlineto{\pgfqpoint{1.752103in}{0.633319in}}%
\pgfpathlineto{\pgfqpoint{1.753176in}{0.630827in}}%
\pgfpathlineto{\pgfqpoint{1.754250in}{0.634272in}}%
\pgfpathlineto{\pgfqpoint{1.755324in}{0.627823in}}%
\pgfpathlineto{\pgfqpoint{1.756398in}{0.629948in}}%
\pgfpathlineto{\pgfqpoint{1.759619in}{0.628189in}}%
\pgfpathlineto{\pgfqpoint{1.760693in}{0.632440in}}%
\pgfpathlineto{\pgfqpoint{1.761767in}{0.626577in}}%
\pgfpathlineto{\pgfqpoint{1.762841in}{0.626064in}}%
\pgfpathlineto{\pgfqpoint{1.763915in}{0.630021in}}%
\pgfpathlineto{\pgfqpoint{1.767136in}{0.629655in}}%
\pgfpathlineto{\pgfqpoint{1.768210in}{0.623132in}}%
\pgfpathlineto{\pgfqpoint{1.769284in}{0.630681in}}%
\pgfpathlineto{\pgfqpoint{1.771432in}{0.617782in}}%
\pgfpathlineto{\pgfqpoint{1.774653in}{0.616610in}}%
\pgfpathlineto{\pgfqpoint{1.776801in}{0.609574in}}%
\pgfpathlineto{\pgfqpoint{1.778949in}{0.619614in}}%
\pgfpathlineto{\pgfqpoint{1.782170in}{0.622839in}}%
\pgfpathlineto{\pgfqpoint{1.783244in}{0.632366in}}%
\pgfpathlineto{\pgfqpoint{1.784318in}{0.628189in}}%
\pgfpathlineto{\pgfqpoint{1.785392in}{0.634052in}}%
\pgfpathlineto{\pgfqpoint{1.786465in}{0.632660in}}%
\pgfpathlineto{\pgfqpoint{1.789687in}{0.632513in}}%
\pgfpathlineto{\pgfqpoint{1.790761in}{0.638303in}}%
\pgfpathlineto{\pgfqpoint{1.791835in}{0.634712in}}%
\pgfpathlineto{\pgfqpoint{1.792908in}{0.673114in}}%
\pgfpathlineto{\pgfqpoint{1.793982in}{0.681469in}}%
\pgfpathlineto{\pgfqpoint{1.797204in}{0.681615in}}%
\pgfpathlineto{\pgfqpoint{1.798278in}{0.684107in}}%
\pgfpathlineto{\pgfqpoint{1.799351in}{0.695613in}}%
\pgfpathlineto{\pgfqpoint{1.800425in}{0.696566in}}%
\pgfpathlineto{\pgfqpoint{1.801499in}{0.700670in}}%
\pgfpathlineto{\pgfqpoint{1.805794in}{0.695906in}}%
\pgfpathlineto{\pgfqpoint{1.806868in}{0.703235in}}%
\pgfpathlineto{\pgfqpoint{1.807942in}{0.701403in}}%
\pgfpathlineto{\pgfqpoint{1.809016in}{0.697738in}}%
\pgfpathlineto{\pgfqpoint{1.812238in}{0.699424in}}%
\pgfpathlineto{\pgfqpoint{1.814385in}{0.699571in}}%
\pgfpathlineto{\pgfqpoint{1.816533in}{0.707192in}}%
\pgfpathlineto{\pgfqpoint{1.819754in}{0.707925in}}%
\pgfpathlineto{\pgfqpoint{1.820828in}{0.712542in}}%
\pgfpathlineto{\pgfqpoint{1.821902in}{0.712542in}}%
\pgfpathlineto{\pgfqpoint{1.824050in}{0.714155in}}%
\pgfpathlineto{\pgfqpoint{1.827271in}{0.714155in}}%
\pgfpathlineto{\pgfqpoint{1.829419in}{0.720164in}}%
\pgfpathlineto{\pgfqpoint{1.830493in}{0.719358in}}%
\pgfpathlineto{\pgfqpoint{1.831567in}{0.723242in}}%
\pgfpathlineto{\pgfqpoint{1.834788in}{0.722803in}}%
\pgfpathlineto{\pgfqpoint{1.835862in}{0.724635in}}%
\pgfpathlineto{\pgfqpoint{1.836936in}{0.720384in}}%
\pgfpathlineto{\pgfqpoint{1.838010in}{0.722876in}}%
\pgfpathlineto{\pgfqpoint{1.839083in}{0.711736in}}%
\pgfpathlineto{\pgfqpoint{1.842305in}{0.711590in}}%
\pgfpathlineto{\pgfqpoint{1.843379in}{0.705800in}}%
\pgfpathlineto{\pgfqpoint{1.845527in}{0.724708in}}%
\pgfpathlineto{\pgfqpoint{1.846600in}{0.720311in}}%
\pgfpathlineto{\pgfqpoint{1.850896in}{0.726613in}}%
\pgfpathlineto{\pgfqpoint{1.851970in}{0.730791in}}%
\pgfpathlineto{\pgfqpoint{1.858413in}{0.725368in}}%
\pgfpathlineto{\pgfqpoint{1.859486in}{0.721263in}}%
\pgfpathlineto{\pgfqpoint{1.861634in}{0.726174in}}%
\pgfpathlineto{\pgfqpoint{1.865929in}{0.712982in}}%
\pgfpathlineto{\pgfqpoint{1.868077in}{0.725221in}}%
\pgfpathlineto{\pgfqpoint{1.869151in}{0.718332in}}%
\pgfpathlineto{\pgfqpoint{1.872372in}{0.717379in}}%
\pgfpathlineto{\pgfqpoint{1.873446in}{0.718772in}}%
\pgfpathlineto{\pgfqpoint{1.874520in}{0.709538in}}%
\pgfpathlineto{\pgfqpoint{1.875594in}{0.705287in}}%
\pgfpathlineto{\pgfqpoint{1.876668in}{0.708512in}}%
\pgfpathlineto{\pgfqpoint{1.884185in}{0.714375in}}%
\pgfpathlineto{\pgfqpoint{1.887406in}{0.711296in}}%
\pgfpathlineto{\pgfqpoint{1.889554in}{0.693341in}}%
\pgfpathlineto{\pgfqpoint{1.890628in}{0.696273in}}%
\pgfpathlineto{\pgfqpoint{1.891702in}{0.708438in}}%
\pgfpathlineto{\pgfqpoint{1.894923in}{0.709171in}}%
\pgfpathlineto{\pgfqpoint{1.897071in}{0.725954in}}%
\pgfpathlineto{\pgfqpoint{1.898145in}{0.738119in}}%
\pgfpathlineto{\pgfqpoint{1.899218in}{0.730424in}}%
\pgfpathlineto{\pgfqpoint{1.903514in}{0.725368in}}%
\pgfpathlineto{\pgfqpoint{1.905661in}{0.740025in}}%
\pgfpathlineto{\pgfqpoint{1.906735in}{0.737753in}}%
\pgfpathlineto{\pgfqpoint{1.911031in}{0.739952in}}%
\pgfpathlineto{\pgfqpoint{1.912105in}{0.736874in}}%
\pgfpathlineto{\pgfqpoint{1.913178in}{0.736874in}}%
\pgfpathlineto{\pgfqpoint{1.914252in}{0.743763in}}%
\pgfpathlineto{\pgfqpoint{1.917474in}{0.743763in}}%
\pgfpathlineto{\pgfqpoint{1.918548in}{0.742810in}}%
\pgfpathlineto{\pgfqpoint{1.920695in}{0.745082in}}%
\pgfpathlineto{\pgfqpoint{1.921769in}{0.740758in}}%
\pgfpathlineto{\pgfqpoint{1.924991in}{0.753143in}}%
\pgfpathlineto{\pgfqpoint{1.927138in}{0.745082in}}%
\pgfpathlineto{\pgfqpoint{1.928212in}{0.745741in}}%
\pgfpathlineto{\pgfqpoint{1.929286in}{0.737240in}}%
\pgfpathlineto{\pgfqpoint{1.932507in}{0.740978in}}%
\pgfpathlineto{\pgfqpoint{1.933581in}{0.729911in}}%
\pgfpathlineto{\pgfqpoint{1.934655in}{0.729105in}}%
\pgfpathlineto{\pgfqpoint{1.935729in}{0.737680in}}%
\pgfpathlineto{\pgfqpoint{1.936803in}{0.729618in}}%
\pgfpathlineto{\pgfqpoint{1.940024in}{0.736727in}}%
\pgfpathlineto{\pgfqpoint{1.941098in}{0.728665in}}%
\pgfpathlineto{\pgfqpoint{1.942172in}{0.734309in}}%
\pgfpathlineto{\pgfqpoint{1.943246in}{0.733576in}}%
\pgfpathlineto{\pgfqpoint{1.944320in}{0.737826in}}%
\pgfpathlineto{\pgfqpoint{1.947541in}{0.735481in}}%
\pgfpathlineto{\pgfqpoint{1.948615in}{0.735628in}}%
\pgfpathlineto{\pgfqpoint{1.949689in}{0.725881in}}%
\pgfpathlineto{\pgfqpoint{1.951837in}{0.724561in}}%
\pgfpathlineto{\pgfqpoint{1.955058in}{0.725368in}}%
\pgfpathlineto{\pgfqpoint{1.957206in}{0.721996in}}%
\pgfpathlineto{\pgfqpoint{1.958280in}{0.722803in}}%
\pgfpathlineto{\pgfqpoint{1.962575in}{0.722143in}}%
\pgfpathlineto{\pgfqpoint{1.964723in}{0.731377in}}%
\pgfpathlineto{\pgfqpoint{1.966870in}{0.730204in}}%
\pgfpathlineto{\pgfqpoint{1.970092in}{0.725074in}}%
\pgfpathlineto{\pgfqpoint{1.971166in}{0.724488in}}%
\pgfpathlineto{\pgfqpoint{1.972239in}{0.725514in}}%
\pgfpathlineto{\pgfqpoint{1.973313in}{0.725294in}}%
\pgfpathlineto{\pgfqpoint{1.974387in}{0.717306in}}%
\pgfpathlineto{\pgfqpoint{1.977609in}{0.718698in}}%
\pgfpathlineto{\pgfqpoint{1.978682in}{0.723169in}}%
\pgfpathlineto{\pgfqpoint{1.979756in}{0.742077in}}%
\pgfpathlineto{\pgfqpoint{1.981904in}{0.738266in}}%
\pgfpathlineto{\pgfqpoint{1.985126in}{0.735628in}}%
\pgfpathlineto{\pgfqpoint{1.986199in}{0.733283in}}%
\pgfpathlineto{\pgfqpoint{1.987273in}{0.737313in}}%
\pgfpathlineto{\pgfqpoint{1.988347in}{0.728152in}}%
\pgfpathlineto{\pgfqpoint{1.989421in}{0.726174in}}%
\pgfpathlineto{\pgfqpoint{1.992642in}{0.724855in}}%
\pgfpathlineto{\pgfqpoint{1.993716in}{0.727493in}}%
\pgfpathlineto{\pgfqpoint{1.994790in}{0.725441in}}%
\pgfpathlineto{\pgfqpoint{1.995864in}{0.731890in}}%
\pgfpathlineto{\pgfqpoint{1.996938in}{0.752410in}}%
\pgfpathlineto{\pgfqpoint{2.000159in}{0.749186in}}%
\pgfpathlineto{\pgfqpoint{2.001233in}{0.746914in}}%
\pgfpathlineto{\pgfqpoint{2.002307in}{0.747427in}}%
\pgfpathlineto{\pgfqpoint{2.003381in}{0.757028in}}%
\pgfpathlineto{\pgfqpoint{2.004455in}{0.753949in}}%
\pgfpathlineto{\pgfqpoint{2.008750in}{0.758200in}}%
\pgfpathlineto{\pgfqpoint{2.010898in}{0.752557in}}%
\pgfpathlineto{\pgfqpoint{2.011971in}{0.754316in}}%
\pgfpathlineto{\pgfqpoint{2.016267in}{0.746694in}}%
\pgfpathlineto{\pgfqpoint{2.017341in}{0.753436in}}%
\pgfpathlineto{\pgfqpoint{2.018415in}{0.753876in}}%
\pgfpathlineto{\pgfqpoint{2.019488in}{0.747647in}}%
\pgfpathlineto{\pgfqpoint{2.022710in}{0.750725in}}%
\pgfpathlineto{\pgfqpoint{2.024858in}{0.749552in}}%
\pgfpathlineto{\pgfqpoint{2.025931in}{0.744349in}}%
\pgfpathlineto{\pgfqpoint{2.027005in}{0.745448in}}%
\pgfpathlineto{\pgfqpoint{2.030227in}{0.740538in}}%
\pgfpathlineto{\pgfqpoint{2.031301in}{0.742443in}}%
\pgfpathlineto{\pgfqpoint{2.032374in}{0.754389in}}%
\pgfpathlineto{\pgfqpoint{2.033448in}{0.754462in}}%
\pgfpathlineto{\pgfqpoint{2.037744in}{0.746914in}}%
\pgfpathlineto{\pgfqpoint{2.038817in}{0.749992in}}%
\pgfpathlineto{\pgfqpoint{2.039891in}{0.748160in}}%
\pgfpathlineto{\pgfqpoint{2.040965in}{0.753510in}}%
\pgfpathlineto{\pgfqpoint{2.042039in}{0.747720in}}%
\pgfpathlineto{\pgfqpoint{2.045260in}{0.750578in}}%
\pgfpathlineto{\pgfqpoint{2.046334in}{0.752923in}}%
\pgfpathlineto{\pgfqpoint{2.047408in}{0.748966in}}%
\pgfpathlineto{\pgfqpoint{2.048482in}{0.747207in}}%
\pgfpathlineto{\pgfqpoint{2.049556in}{0.748160in}}%
\pgfpathlineto{\pgfqpoint{2.052777in}{0.733722in}}%
\pgfpathlineto{\pgfqpoint{2.053851in}{0.736800in}}%
\pgfpathlineto{\pgfqpoint{2.055999in}{0.744569in}}%
\pgfpathlineto{\pgfqpoint{2.060294in}{0.743689in}}%
\pgfpathlineto{\pgfqpoint{2.061368in}{0.741198in}}%
\pgfpathlineto{\pgfqpoint{2.062442in}{0.733796in}}%
\pgfpathlineto{\pgfqpoint{2.063516in}{0.736067in}}%
\pgfpathlineto{\pgfqpoint{2.064590in}{0.745815in}}%
\pgfpathlineto{\pgfqpoint{2.067811in}{0.753656in}}%
\pgfpathlineto{\pgfqpoint{2.068885in}{0.757760in}}%
\pgfpathlineto{\pgfqpoint{2.069959in}{0.757174in}}%
\pgfpathlineto{\pgfqpoint{2.075328in}{0.776229in}}%
\pgfpathlineto{\pgfqpoint{2.076402in}{0.771392in}}%
\pgfpathlineto{\pgfqpoint{2.077476in}{0.771025in}}%
\pgfpathlineto{\pgfqpoint{2.078549in}{0.769486in}}%
\pgfpathlineto{\pgfqpoint{2.079623in}{0.791106in}}%
\pgfpathlineto{\pgfqpoint{2.082845in}{0.784437in}}%
\pgfpathlineto{\pgfqpoint{2.086066in}{0.802319in}}%
\pgfpathlineto{\pgfqpoint{2.087140in}{0.794990in}}%
\pgfpathlineto{\pgfqpoint{2.090362in}{0.797995in}}%
\pgfpathlineto{\pgfqpoint{2.092509in}{0.792425in}}%
\pgfpathlineto{\pgfqpoint{2.093583in}{0.782458in}}%
\pgfpathlineto{\pgfqpoint{2.094657in}{0.786929in}}%
\pgfpathlineto{\pgfqpoint{2.097879in}{0.788174in}}%
\pgfpathlineto{\pgfqpoint{2.098952in}{0.780919in}}%
\pgfpathlineto{\pgfqpoint{2.102174in}{0.787881in}}%
\pgfpathlineto{\pgfqpoint{2.106469in}{0.789640in}}%
\pgfpathlineto{\pgfqpoint{2.107543in}{0.789127in}}%
\pgfpathlineto{\pgfqpoint{2.108617in}{0.785976in}}%
\pgfpathlineto{\pgfqpoint{2.109691in}{0.766335in}}%
\pgfpathlineto{\pgfqpoint{2.113986in}{0.736287in}}%
\pgfpathlineto{\pgfqpoint{2.115060in}{0.762744in}}%
\pgfpathlineto{\pgfqpoint{2.116134in}{0.774910in}}%
\pgfpathlineto{\pgfqpoint{2.117208in}{0.775349in}}%
\pgfpathlineto{\pgfqpoint{2.120429in}{0.767068in}}%
\pgfpathlineto{\pgfqpoint{2.121503in}{0.750432in}}%
\pgfpathlineto{\pgfqpoint{2.123651in}{0.760692in}}%
\pgfpathlineto{\pgfqpoint{2.124725in}{0.751897in}}%
\pgfpathlineto{\pgfqpoint{2.129020in}{0.761645in}}%
\pgfpathlineto{\pgfqpoint{2.130094in}{0.754975in}}%
\pgfpathlineto{\pgfqpoint{2.132241in}{0.763257in}}%
\pgfpathlineto{\pgfqpoint{2.135463in}{0.757907in}}%
\pgfpathlineto{\pgfqpoint{2.137611in}{0.764869in}}%
\pgfpathlineto{\pgfqpoint{2.138684in}{0.764649in}}%
\pgfpathlineto{\pgfqpoint{2.139758in}{0.756368in}}%
\pgfpathlineto{\pgfqpoint{2.142980in}{0.763403in}}%
\pgfpathlineto{\pgfqpoint{2.144054in}{0.759593in}}%
\pgfpathlineto{\pgfqpoint{2.145127in}{0.764576in}}%
\pgfpathlineto{\pgfqpoint{2.146201in}{0.759519in}}%
\pgfpathlineto{\pgfqpoint{2.147275in}{0.762744in}}%
\pgfpathlineto{\pgfqpoint{2.150497in}{0.737900in}}%
\pgfpathlineto{\pgfqpoint{2.152644in}{0.755415in}}%
\pgfpathlineto{\pgfqpoint{2.153718in}{0.757760in}}%
\pgfpathlineto{\pgfqpoint{2.154792in}{0.762597in}}%
\pgfpathlineto{\pgfqpoint{2.158014in}{0.773737in}}%
\pgfpathlineto{\pgfqpoint{2.159087in}{0.772638in}}%
\pgfpathlineto{\pgfqpoint{2.161235in}{0.785609in}}%
\pgfpathlineto{\pgfqpoint{2.162309in}{0.786196in}}%
\pgfpathlineto{\pgfqpoint{2.165530in}{0.793378in}}%
\pgfpathlineto{\pgfqpoint{2.166604in}{0.793451in}}%
\pgfpathlineto{\pgfqpoint{2.167678in}{0.787735in}}%
\pgfpathlineto{\pgfqpoint{2.169826in}{0.800560in}}%
\pgfpathlineto{\pgfqpoint{2.173047in}{0.807596in}}%
\pgfpathlineto{\pgfqpoint{2.175195in}{0.796676in}}%
\pgfpathlineto{\pgfqpoint{2.177343in}{0.808109in}}%
\pgfpathlineto{\pgfqpoint{2.180564in}{0.816024in}}%
\pgfpathlineto{\pgfqpoint{2.181638in}{0.811333in}}%
\pgfpathlineto{\pgfqpoint{2.182712in}{0.820934in}}%
\pgfpathlineto{\pgfqpoint{2.183786in}{0.818369in}}%
\pgfpathlineto{\pgfqpoint{2.188081in}{0.794990in}}%
\pgfpathlineto{\pgfqpoint{2.189155in}{0.814045in}}%
\pgfpathlineto{\pgfqpoint{2.190229in}{0.817269in}}%
\pgfpathlineto{\pgfqpoint{2.191303in}{0.823719in}}%
\pgfpathlineto{\pgfqpoint{2.192376in}{0.820054in}}%
\pgfpathlineto{\pgfqpoint{2.195598in}{0.815071in}}%
\pgfpathlineto{\pgfqpoint{2.196672in}{0.826284in}}%
\pgfpathlineto{\pgfqpoint{2.197746in}{0.824158in}}%
\pgfpathlineto{\pgfqpoint{2.198819in}{0.817929in}}%
\pgfpathlineto{\pgfqpoint{2.199893in}{0.816537in}}%
\pgfpathlineto{\pgfqpoint{2.203115in}{0.822180in}}%
\pgfpathlineto{\pgfqpoint{2.204189in}{0.821667in}}%
\pgfpathlineto{\pgfqpoint{2.205262in}{0.833246in}}%
\pgfpathlineto{\pgfqpoint{2.206336in}{0.831047in}}%
\pgfpathlineto{\pgfqpoint{2.207410in}{0.831341in}}%
\pgfpathlineto{\pgfqpoint{2.210632in}{0.830754in}}%
\pgfpathlineto{\pgfqpoint{2.212779in}{0.826870in}}%
\pgfpathlineto{\pgfqpoint{2.214927in}{0.828849in}}%
\pgfpathlineto{\pgfqpoint{2.218148in}{0.822913in}}%
\pgfpathlineto{\pgfqpoint{2.219222in}{0.829362in}}%
\pgfpathlineto{\pgfqpoint{2.221370in}{0.817782in}}%
\pgfpathlineto{\pgfqpoint{2.222444in}{0.832880in}}%
\pgfpathlineto{\pgfqpoint{2.225665in}{0.826797in}}%
\pgfpathlineto{\pgfqpoint{2.226739in}{0.823352in}}%
\pgfpathlineto{\pgfqpoint{2.227813in}{0.814924in}}%
\pgfpathlineto{\pgfqpoint{2.228887in}{0.816610in}}%
\pgfpathlineto{\pgfqpoint{2.229961in}{0.802319in}}%
\pgfpathlineto{\pgfqpoint{2.233182in}{0.807669in}}%
\pgfpathlineto{\pgfqpoint{2.235330in}{0.828995in}}%
\pgfpathlineto{\pgfqpoint{2.236404in}{0.820641in}}%
\pgfpathlineto{\pgfqpoint{2.237478in}{0.803785in}}%
\pgfpathlineto{\pgfqpoint{2.241773in}{0.811333in}}%
\pgfpathlineto{\pgfqpoint{2.242847in}{0.819761in}}%
\pgfpathlineto{\pgfqpoint{2.243921in}{0.817636in}}%
\pgfpathlineto{\pgfqpoint{2.248216in}{0.819541in}}%
\pgfpathlineto{\pgfqpoint{2.249290in}{0.824378in}}%
\pgfpathlineto{\pgfqpoint{2.251437in}{0.812579in}}%
\pgfpathlineto{\pgfqpoint{2.255733in}{0.799387in}}%
\pgfpathlineto{\pgfqpoint{2.256807in}{0.803418in}}%
\pgfpathlineto{\pgfqpoint{2.260028in}{0.779233in}}%
\pgfpathlineto{\pgfqpoint{2.263250in}{0.786709in}}%
\pgfpathlineto{\pgfqpoint{2.264324in}{0.792645in}}%
\pgfpathlineto{\pgfqpoint{2.265397in}{0.780846in}}%
\pgfpathlineto{\pgfqpoint{2.266471in}{0.785829in}}%
\pgfpathlineto{\pgfqpoint{2.267545in}{0.771758in}}%
\pgfpathlineto{\pgfqpoint{2.271840in}{0.768680in}}%
\pgfpathlineto{\pgfqpoint{2.272914in}{0.763623in}}%
\pgfpathlineto{\pgfqpoint{2.275062in}{0.777988in}}%
\pgfpathlineto{\pgfqpoint{2.278283in}{0.771172in}}%
\pgfpathlineto{\pgfqpoint{2.279357in}{0.772125in}}%
\pgfpathlineto{\pgfqpoint{2.280431in}{0.765236in}}%
\pgfpathlineto{\pgfqpoint{2.281505in}{0.753949in}}%
\pgfpathlineto{\pgfqpoint{2.282579in}{0.790740in}}%
\pgfpathlineto{\pgfqpoint{2.285800in}{0.789933in}}%
\pgfpathlineto{\pgfqpoint{2.286874in}{0.782971in}}%
\pgfpathlineto{\pgfqpoint{2.287948in}{0.789933in}}%
\pgfpathlineto{\pgfqpoint{2.289022in}{0.784950in}}%
\pgfpathlineto{\pgfqpoint{2.290096in}{0.769706in}}%
\pgfpathlineto{\pgfqpoint{2.293317in}{0.742883in}}%
\pgfpathlineto{\pgfqpoint{2.294391in}{0.746841in}}%
\pgfpathlineto{\pgfqpoint{2.295465in}{0.760838in}}%
\pgfpathlineto{\pgfqpoint{2.296539in}{0.748819in}}%
\pgfpathlineto{\pgfqpoint{2.297613in}{0.762744in}}%
\pgfpathlineto{\pgfqpoint{2.301908in}{0.767654in}}%
\pgfpathlineto{\pgfqpoint{2.302982in}{0.775276in}}%
\pgfpathlineto{\pgfqpoint{2.304056in}{0.769633in}}%
\pgfpathlineto{\pgfqpoint{2.305129in}{0.771685in}}%
\pgfpathlineto{\pgfqpoint{2.308351in}{0.782678in}}%
\pgfpathlineto{\pgfqpoint{2.309425in}{0.776155in}}%
\pgfpathlineto{\pgfqpoint{2.310499in}{0.773957in}}%
\pgfpathlineto{\pgfqpoint{2.311572in}{0.784144in}}%
\pgfpathlineto{\pgfqpoint{2.312646in}{0.780259in}}%
\pgfpathlineto{\pgfqpoint{2.315868in}{0.777841in}}%
\pgfpathlineto{\pgfqpoint{2.316942in}{0.793964in}}%
\pgfpathlineto{\pgfqpoint{2.319089in}{0.788761in}}%
\pgfpathlineto{\pgfqpoint{2.320163in}{0.788687in}}%
\pgfpathlineto{\pgfqpoint{2.323385in}{0.774616in}}%
\pgfpathlineto{\pgfqpoint{2.324458in}{0.765162in}}%
\pgfpathlineto{\pgfqpoint{2.325532in}{0.765602in}}%
\pgfpathlineto{\pgfqpoint{2.326606in}{0.762304in}}%
\pgfpathlineto{\pgfqpoint{2.327680in}{0.772418in}}%
\pgfpathlineto{\pgfqpoint{2.330902in}{0.771392in}}%
\pgfpathlineto{\pgfqpoint{2.333049in}{0.777694in}}%
\pgfpathlineto{\pgfqpoint{2.335197in}{0.787808in}}%
\pgfpathlineto{\pgfqpoint{2.338418in}{0.787735in}}%
\pgfpathlineto{\pgfqpoint{2.339492in}{0.781945in}}%
\pgfpathlineto{\pgfqpoint{2.340566in}{0.788761in}}%
\pgfpathlineto{\pgfqpoint{2.341640in}{0.790373in}}%
\pgfpathlineto{\pgfqpoint{2.345935in}{0.789933in}}%
\pgfpathlineto{\pgfqpoint{2.348083in}{0.809281in}}%
\pgfpathlineto{\pgfqpoint{2.349157in}{0.807083in}}%
\pgfpathlineto{\pgfqpoint{2.350231in}{0.815071in}}%
\pgfpathlineto{\pgfqpoint{2.353452in}{0.816756in}}%
\pgfpathlineto{\pgfqpoint{2.354526in}{0.810600in}}%
\pgfpathlineto{\pgfqpoint{2.355600in}{0.819468in}}%
\pgfpathlineto{\pgfqpoint{2.356674in}{0.814998in}}%
\pgfpathlineto{\pgfqpoint{2.357747in}{0.818222in}}%
\pgfpathlineto{\pgfqpoint{2.360969in}{0.816683in}}%
\pgfpathlineto{\pgfqpoint{2.362043in}{0.821740in}}%
\pgfpathlineto{\pgfqpoint{2.364191in}{0.834638in}}%
\pgfpathlineto{\pgfqpoint{2.365264in}{0.832880in}}%
\pgfpathlineto{\pgfqpoint{2.368486in}{0.842700in}}%
\pgfpathlineto{\pgfqpoint{2.369560in}{0.837643in}}%
\pgfpathlineto{\pgfqpoint{2.370634in}{0.840501in}}%
\pgfpathlineto{\pgfqpoint{2.371707in}{0.837936in}}%
\pgfpathlineto{\pgfqpoint{2.372781in}{0.825917in}}%
\pgfpathlineto{\pgfqpoint{2.376003in}{0.819028in}}%
\pgfpathlineto{\pgfqpoint{2.378150in}{0.823426in}}%
\pgfpathlineto{\pgfqpoint{2.379224in}{0.815730in}}%
\pgfpathlineto{\pgfqpoint{2.380298in}{0.812579in}}%
\pgfpathlineto{\pgfqpoint{2.383520in}{0.821300in}}%
\pgfpathlineto{\pgfqpoint{2.384593in}{0.812213in}}%
\pgfpathlineto{\pgfqpoint{2.385667in}{0.811333in}}%
\pgfpathlineto{\pgfqpoint{2.387815in}{0.816024in}}%
\pgfpathlineto{\pgfqpoint{2.391036in}{0.819541in}}%
\pgfpathlineto{\pgfqpoint{2.392110in}{0.826577in}}%
\pgfpathlineto{\pgfqpoint{2.393184in}{0.813971in}}%
\pgfpathlineto{\pgfqpoint{2.394258in}{0.818222in}}%
\pgfpathlineto{\pgfqpoint{2.395332in}{0.810600in}}%
\pgfpathlineto{\pgfqpoint{2.398553in}{0.817563in}}%
\pgfpathlineto{\pgfqpoint{2.399627in}{0.810307in}}%
\pgfpathlineto{\pgfqpoint{2.400701in}{0.814924in}}%
\pgfpathlineto{\pgfqpoint{2.401775in}{0.810967in}}%
\pgfpathlineto{\pgfqpoint{2.402849in}{0.816610in}}%
\pgfpathlineto{\pgfqpoint{2.406070in}{0.813312in}}%
\pgfpathlineto{\pgfqpoint{2.407144in}{0.828849in}}%
\pgfpathlineto{\pgfqpoint{2.408218in}{0.826577in}}%
\pgfpathlineto{\pgfqpoint{2.409292in}{0.826137in}}%
\pgfpathlineto{\pgfqpoint{2.410366in}{0.830901in}}%
\pgfpathlineto{\pgfqpoint{2.414661in}{0.825697in}}%
\pgfpathlineto{\pgfqpoint{2.415735in}{0.827749in}}%
\pgfpathlineto{\pgfqpoint{2.416809in}{0.832953in}}%
\pgfpathlineto{\pgfqpoint{2.417882in}{0.832880in}}%
\pgfpathlineto{\pgfqpoint{2.421104in}{0.837203in}}%
\pgfpathlineto{\pgfqpoint{2.422178in}{0.837643in}}%
\pgfpathlineto{\pgfqpoint{2.423252in}{0.845118in}}%
\pgfpathlineto{\pgfqpoint{2.424325in}{0.842334in}}%
\pgfpathlineto{\pgfqpoint{2.425399in}{0.834638in}}%
\pgfpathlineto{\pgfqpoint{2.428621in}{0.821813in}}%
\pgfpathlineto{\pgfqpoint{2.429695in}{0.823059in}}%
\pgfpathlineto{\pgfqpoint{2.430769in}{0.820201in}}%
\pgfpathlineto{\pgfqpoint{2.431842in}{0.821520in}}%
\pgfpathlineto{\pgfqpoint{2.432916in}{0.811773in}}%
\pgfpathlineto{\pgfqpoint{2.436138in}{0.814265in}}%
\pgfpathlineto{\pgfqpoint{2.437212in}{0.814191in}}%
\pgfpathlineto{\pgfqpoint{2.438285in}{0.808475in}}%
\pgfpathlineto{\pgfqpoint{2.439359in}{0.820641in}}%
\pgfpathlineto{\pgfqpoint{2.440433in}{0.797848in}}%
\pgfpathlineto{\pgfqpoint{2.443655in}{0.785609in}}%
\pgfpathlineto{\pgfqpoint{2.445802in}{0.809941in}}%
\pgfpathlineto{\pgfqpoint{2.446876in}{0.791546in}}%
\pgfpathlineto{\pgfqpoint{2.447950in}{0.793744in}}%
\pgfpathlineto{\pgfqpoint{2.452245in}{0.795137in}}%
\pgfpathlineto{\pgfqpoint{2.453319in}{0.790740in}}%
\pgfpathlineto{\pgfqpoint{2.454393in}{0.793964in}}%
\pgfpathlineto{\pgfqpoint{2.455467in}{0.807669in}}%
\pgfpathlineto{\pgfqpoint{2.458688in}{0.808402in}}%
\pgfpathlineto{\pgfqpoint{2.459762in}{0.815217in}}%
\pgfpathlineto{\pgfqpoint{2.460836in}{0.815144in}}%
\pgfpathlineto{\pgfqpoint{2.461910in}{0.819981in}}%
\pgfpathlineto{\pgfqpoint{2.462984in}{0.821154in}}%
\pgfpathlineto{\pgfqpoint{2.466205in}{0.821227in}}%
\pgfpathlineto{\pgfqpoint{2.468353in}{0.828702in}}%
\pgfpathlineto{\pgfqpoint{2.469427in}{0.824671in}}%
\pgfpathlineto{\pgfqpoint{2.470501in}{0.832660in}}%
\pgfpathlineto{\pgfqpoint{2.474796in}{0.822473in}}%
\pgfpathlineto{\pgfqpoint{2.475870in}{0.822766in}}%
\pgfpathlineto{\pgfqpoint{2.476944in}{0.827530in}}%
\pgfpathlineto{\pgfqpoint{2.478017in}{0.819321in}}%
\pgfpathlineto{\pgfqpoint{2.482313in}{0.821007in}}%
\pgfpathlineto{\pgfqpoint{2.483387in}{0.824085in}}%
\pgfpathlineto{\pgfqpoint{2.485534in}{0.834345in}}%
\pgfpathlineto{\pgfqpoint{2.488756in}{0.832953in}}%
\pgfpathlineto{\pgfqpoint{2.489830in}{0.833759in}}%
\pgfpathlineto{\pgfqpoint{2.490903in}{0.831047in}}%
\pgfpathlineto{\pgfqpoint{2.491977in}{0.834199in}}%
\pgfpathlineto{\pgfqpoint{2.493051in}{0.833466in}}%
\pgfpathlineto{\pgfqpoint{2.496273in}{0.839842in}}%
\pgfpathlineto{\pgfqpoint{2.497346in}{0.839109in}}%
\pgfpathlineto{\pgfqpoint{2.498420in}{0.840355in}}%
\pgfpathlineto{\pgfqpoint{2.499494in}{0.836251in}}%
\pgfpathlineto{\pgfqpoint{2.503790in}{0.841894in}}%
\pgfpathlineto{\pgfqpoint{2.504863in}{0.840062in}}%
\pgfpathlineto{\pgfqpoint{2.505937in}{0.836397in}}%
\pgfpathlineto{\pgfqpoint{2.507011in}{0.836617in}}%
\pgfpathlineto{\pgfqpoint{2.508085in}{0.838376in}}%
\pgfpathlineto{\pgfqpoint{2.511306in}{0.840575in}}%
\pgfpathlineto{\pgfqpoint{2.512380in}{0.842700in}}%
\pgfpathlineto{\pgfqpoint{2.513454in}{0.840795in}}%
\pgfpathlineto{\pgfqpoint{2.514528in}{0.843653in}}%
\pgfpathlineto{\pgfqpoint{2.515602in}{0.848636in}}%
\pgfpathlineto{\pgfqpoint{2.519897in}{0.852447in}}%
\pgfpathlineto{\pgfqpoint{2.520971in}{0.857577in}}%
\pgfpathlineto{\pgfqpoint{2.522045in}{0.855598in}}%
\pgfpathlineto{\pgfqpoint{2.523119in}{0.843873in}}%
\pgfpathlineto{\pgfqpoint{2.526340in}{0.855598in}}%
\pgfpathlineto{\pgfqpoint{2.527414in}{0.847903in}}%
\pgfpathlineto{\pgfqpoint{2.528488in}{0.844972in}}%
\pgfpathlineto{\pgfqpoint{2.529562in}{0.848710in}}%
\pgfpathlineto{\pgfqpoint{2.530635in}{0.849149in}}%
\pgfpathlineto{\pgfqpoint{2.533857in}{0.852301in}}%
\pgfpathlineto{\pgfqpoint{2.534931in}{0.852081in}}%
\pgfpathlineto{\pgfqpoint{2.536005in}{0.857431in}}%
\pgfpathlineto{\pgfqpoint{2.537079in}{0.858457in}}%
\pgfpathlineto{\pgfqpoint{2.538152in}{0.852520in}}%
\pgfpathlineto{\pgfqpoint{2.541374in}{0.847317in}}%
\pgfpathlineto{\pgfqpoint{2.542448in}{0.849809in}}%
\pgfpathlineto{\pgfqpoint{2.543522in}{0.855598in}}%
\pgfpathlineto{\pgfqpoint{2.544595in}{0.848123in}}%
\pgfpathlineto{\pgfqpoint{2.545669in}{0.853693in}}%
\pgfpathlineto{\pgfqpoint{2.548891in}{0.854866in}}%
\pgfpathlineto{\pgfqpoint{2.549965in}{0.853913in}}%
\pgfpathlineto{\pgfqpoint{2.551038in}{0.858383in}}%
\pgfpathlineto{\pgfqpoint{2.552112in}{0.858457in}}%
\pgfpathlineto{\pgfqpoint{2.553186in}{0.855012in}}%
\pgfpathlineto{\pgfqpoint{2.556408in}{0.856698in}}%
\pgfpathlineto{\pgfqpoint{2.557481in}{0.848929in}}%
\pgfpathlineto{\pgfqpoint{2.558555in}{0.850468in}}%
\pgfpathlineto{\pgfqpoint{2.559629in}{0.847757in}}%
\pgfpathlineto{\pgfqpoint{2.560703in}{0.851934in}}%
\pgfpathlineto{\pgfqpoint{2.563924in}{0.849736in}}%
\pgfpathlineto{\pgfqpoint{2.564998in}{0.845631in}}%
\pgfpathlineto{\pgfqpoint{2.566072in}{0.854499in}}%
\pgfpathlineto{\pgfqpoint{2.568220in}{0.851201in}}%
\pgfpathlineto{\pgfqpoint{2.571441in}{0.857064in}}%
\pgfpathlineto{\pgfqpoint{2.572515in}{0.848856in}}%
\pgfpathlineto{\pgfqpoint{2.573589in}{0.846877in}}%
\pgfpathlineto{\pgfqpoint{2.574663in}{0.848123in}}%
\pgfpathlineto{\pgfqpoint{2.575737in}{0.850249in}}%
\pgfpathlineto{\pgfqpoint{2.578958in}{0.852301in}}%
\pgfpathlineto{\pgfqpoint{2.581106in}{0.838449in}}%
\pgfpathlineto{\pgfqpoint{2.582180in}{0.838962in}}%
\pgfpathlineto{\pgfqpoint{2.583254in}{0.836910in}}%
\pgfpathlineto{\pgfqpoint{2.587549in}{0.855012in}}%
\pgfpathlineto{\pgfqpoint{2.588623in}{0.857577in}}%
\pgfpathlineto{\pgfqpoint{2.589697in}{0.847757in}}%
\pgfpathlineto{\pgfqpoint{2.590770in}{0.847830in}}%
\pgfpathlineto{\pgfqpoint{2.593992in}{0.822693in}}%
\pgfpathlineto{\pgfqpoint{2.595066in}{0.824012in}}%
\pgfpathlineto{\pgfqpoint{2.597213in}{0.843213in}}%
\pgfpathlineto{\pgfqpoint{2.598287in}{0.841454in}}%
\pgfpathlineto{\pgfqpoint{2.601509in}{0.847684in}}%
\pgfpathlineto{\pgfqpoint{2.602583in}{0.835005in}}%
\pgfpathlineto{\pgfqpoint{2.603657in}{0.832440in}}%
\pgfpathlineto{\pgfqpoint{2.605804in}{0.836471in}}%
\pgfpathlineto{\pgfqpoint{2.609026in}{0.828922in}}%
\pgfpathlineto{\pgfqpoint{2.610100in}{0.829435in}}%
\pgfpathlineto{\pgfqpoint{2.612247in}{0.802612in}}%
\pgfpathlineto{\pgfqpoint{2.613321in}{0.804737in}}%
\pgfpathlineto{\pgfqpoint{2.616543in}{0.816170in}}%
\pgfpathlineto{\pgfqpoint{2.617616in}{0.814704in}}%
\pgfpathlineto{\pgfqpoint{2.618690in}{0.830534in}}%
\pgfpathlineto{\pgfqpoint{2.619764in}{0.830314in}}%
\pgfpathlineto{\pgfqpoint{2.620838in}{0.829362in}}%
\pgfpathlineto{\pgfqpoint{2.624059in}{0.824745in}}%
\pgfpathlineto{\pgfqpoint{2.625133in}{0.829728in}}%
\pgfpathlineto{\pgfqpoint{2.626207in}{0.829288in}}%
\pgfpathlineto{\pgfqpoint{2.627281in}{0.831927in}}%
\pgfpathlineto{\pgfqpoint{2.628355in}{0.823645in}}%
\pgfpathlineto{\pgfqpoint{2.631576in}{0.821813in}}%
\pgfpathlineto{\pgfqpoint{2.632650in}{0.823719in}}%
\pgfpathlineto{\pgfqpoint{2.634798in}{0.820421in}}%
\pgfpathlineto{\pgfqpoint{2.635872in}{0.822180in}}%
\pgfpathlineto{\pgfqpoint{2.640167in}{0.823645in}}%
\pgfpathlineto{\pgfqpoint{2.642315in}{0.823499in}}%
\pgfpathlineto{\pgfqpoint{2.643389in}{0.821300in}}%
\pgfpathlineto{\pgfqpoint{2.647684in}{0.831927in}}%
\pgfpathlineto{\pgfqpoint{2.649832in}{0.843360in}}%
\pgfpathlineto{\pgfqpoint{2.650905in}{0.851421in}}%
\pgfpathlineto{\pgfqpoint{2.654127in}{0.848123in}}%
\pgfpathlineto{\pgfqpoint{2.655201in}{0.844972in}}%
\pgfpathlineto{\pgfqpoint{2.656275in}{0.848490in}}%
\pgfpathlineto{\pgfqpoint{2.658422in}{0.843946in}}%
\pgfpathlineto{\pgfqpoint{2.662718in}{0.844605in}}%
\pgfpathlineto{\pgfqpoint{2.664865in}{0.847977in}}%
\pgfpathlineto{\pgfqpoint{2.669161in}{0.850981in}}%
\pgfpathlineto{\pgfqpoint{2.671308in}{0.863587in}}%
\pgfpathlineto{\pgfqpoint{2.672382in}{0.858823in}}%
\pgfpathlineto{\pgfqpoint{2.673456in}{0.862634in}}%
\pgfpathlineto{\pgfqpoint{2.676678in}{0.862121in}}%
\pgfpathlineto{\pgfqpoint{2.677751in}{0.855012in}}%
\pgfpathlineto{\pgfqpoint{2.679899in}{0.852081in}}%
\pgfpathlineto{\pgfqpoint{2.680973in}{0.879270in}}%
\pgfpathlineto{\pgfqpoint{2.685268in}{0.877072in}}%
\pgfpathlineto{\pgfqpoint{2.686342in}{0.872161in}}%
\pgfpathlineto{\pgfqpoint{2.688490in}{0.877951in}}%
\pgfpathlineto{\pgfqpoint{2.691711in}{0.881835in}}%
\pgfpathlineto{\pgfqpoint{2.692785in}{0.884840in}}%
\pgfpathlineto{\pgfqpoint{2.693859in}{0.890996in}}%
\pgfpathlineto{\pgfqpoint{2.694933in}{0.890043in}}%
\pgfpathlineto{\pgfqpoint{2.696007in}{0.890410in}}%
\pgfpathlineto{\pgfqpoint{2.700302in}{0.893708in}}%
\pgfpathlineto{\pgfqpoint{2.701376in}{0.892828in}}%
\pgfpathlineto{\pgfqpoint{2.703523in}{0.897372in}}%
\pgfpathlineto{\pgfqpoint{2.707819in}{0.893854in}}%
\pgfpathlineto{\pgfqpoint{2.708893in}{0.901403in}}%
\pgfpathlineto{\pgfqpoint{2.709967in}{0.898105in}}%
\pgfpathlineto{\pgfqpoint{2.711040in}{0.900010in}}%
\pgfpathlineto{\pgfqpoint{2.717483in}{0.902282in}}%
\pgfpathlineto{\pgfqpoint{2.718557in}{0.906753in}}%
\pgfpathlineto{\pgfqpoint{2.721779in}{0.909464in}}%
\pgfpathlineto{\pgfqpoint{2.722853in}{0.905434in}}%
\pgfpathlineto{\pgfqpoint{2.723926in}{0.908145in}}%
\pgfpathlineto{\pgfqpoint{2.726074in}{0.910417in}}%
\pgfpathlineto{\pgfqpoint{2.729296in}{0.902795in}}%
\pgfpathlineto{\pgfqpoint{2.730369in}{0.894734in}}%
\pgfpathlineto{\pgfqpoint{2.733591in}{0.902869in}}%
\pgfpathlineto{\pgfqpoint{2.736812in}{0.901110in}}%
\pgfpathlineto{\pgfqpoint{2.738960in}{0.902795in}}%
\pgfpathlineto{\pgfqpoint{2.741108in}{0.900523in}}%
\pgfpathlineto{\pgfqpoint{2.744329in}{0.904114in}}%
\pgfpathlineto{\pgfqpoint{2.745403in}{0.899937in}}%
\pgfpathlineto{\pgfqpoint{2.746477in}{0.901696in}}%
\pgfpathlineto{\pgfqpoint{2.747551in}{0.902136in}}%
\pgfpathlineto{\pgfqpoint{2.748625in}{0.899644in}}%
\pgfpathlineto{\pgfqpoint{2.752920in}{0.900377in}}%
\pgfpathlineto{\pgfqpoint{2.753994in}{0.899204in}}%
\pgfpathlineto{\pgfqpoint{2.755068in}{0.900523in}}%
\pgfpathlineto{\pgfqpoint{2.759363in}{0.907339in}}%
\pgfpathlineto{\pgfqpoint{2.760437in}{0.906753in}}%
\pgfpathlineto{\pgfqpoint{2.761511in}{0.906899in}}%
\pgfpathlineto{\pgfqpoint{2.762585in}{0.917013in}}%
\pgfpathlineto{\pgfqpoint{2.763658in}{0.917013in}}%
\pgfpathlineto{\pgfqpoint{2.767954in}{0.923902in}}%
\pgfpathlineto{\pgfqpoint{2.771175in}{0.917526in}}%
\pgfpathlineto{\pgfqpoint{2.774397in}{0.917819in}}%
\pgfpathlineto{\pgfqpoint{2.775471in}{0.926980in}}%
\pgfpathlineto{\pgfqpoint{2.776545in}{0.926320in}}%
\pgfpathlineto{\pgfqpoint{2.777618in}{0.927566in}}%
\pgfpathlineto{\pgfqpoint{2.778692in}{0.923755in}}%
\pgfpathlineto{\pgfqpoint{2.781914in}{0.922509in}}%
\pgfpathlineto{\pgfqpoint{2.782988in}{0.923096in}}%
\pgfpathlineto{\pgfqpoint{2.784061in}{0.924928in}}%
\pgfpathlineto{\pgfqpoint{2.785135in}{0.923829in}}%
\pgfpathlineto{\pgfqpoint{2.786209in}{0.928372in}}%
\pgfpathlineto{\pgfqpoint{2.789431in}{0.931964in}}%
\pgfpathlineto{\pgfqpoint{2.790504in}{0.931377in}}%
\pgfpathlineto{\pgfqpoint{2.791578in}{0.922583in}}%
\pgfpathlineto{\pgfqpoint{2.792652in}{0.922216in}}%
\pgfpathlineto{\pgfqpoint{2.793726in}{0.927786in}}%
\pgfpathlineto{\pgfqpoint{2.796947in}{0.933722in}}%
\pgfpathlineto{\pgfqpoint{2.798021in}{0.937753in}}%
\pgfpathlineto{\pgfqpoint{2.799095in}{0.944569in}}%
\pgfpathlineto{\pgfqpoint{2.800169in}{0.946254in}}%
\pgfpathlineto{\pgfqpoint{2.801243in}{0.943543in}}%
\pgfpathlineto{\pgfqpoint{2.805538in}{0.943909in}}%
\pgfpathlineto{\pgfqpoint{2.806612in}{0.947647in}}%
\pgfpathlineto{\pgfqpoint{2.807686in}{0.948820in}}%
\pgfpathlineto{\pgfqpoint{2.808760in}{0.954243in}}%
\pgfpathlineto{\pgfqpoint{2.811981in}{0.957174in}}%
\pgfpathlineto{\pgfqpoint{2.813055in}{0.951678in}}%
\pgfpathlineto{\pgfqpoint{2.814129in}{0.953803in}}%
\pgfpathlineto{\pgfqpoint{2.815203in}{0.953803in}}%
\pgfpathlineto{\pgfqpoint{2.816277in}{0.942810in}}%
\pgfpathlineto{\pgfqpoint{2.819498in}{0.935115in}}%
\pgfpathlineto{\pgfqpoint{2.820572in}{0.946548in}}%
\pgfpathlineto{\pgfqpoint{2.821646in}{0.948307in}}%
\pgfpathlineto{\pgfqpoint{2.822720in}{0.939952in}}%
\pgfpathlineto{\pgfqpoint{2.823793in}{0.939952in}}%
\pgfpathlineto{\pgfqpoint{2.827015in}{0.944422in}}%
\pgfpathlineto{\pgfqpoint{2.828089in}{0.941491in}}%
\pgfpathlineto{\pgfqpoint{2.829163in}{0.942663in}}%
\pgfpathlineto{\pgfqpoint{2.830236in}{0.938413in}}%
\pgfpathlineto{\pgfqpoint{2.831310in}{0.950139in}}%
\pgfpathlineto{\pgfqpoint{2.834532in}{0.947574in}}%
\pgfpathlineto{\pgfqpoint{2.835606in}{0.945228in}}%
\pgfpathlineto{\pgfqpoint{2.836679in}{0.954829in}}%
\pgfpathlineto{\pgfqpoint{2.837753in}{0.941784in}}%
\pgfpathlineto{\pgfqpoint{2.838827in}{0.937167in}}%
\pgfpathlineto{\pgfqpoint{2.842049in}{0.934016in}}%
\pgfpathlineto{\pgfqpoint{2.844196in}{0.938779in}}%
\pgfpathlineto{\pgfqpoint{2.845270in}{0.933356in}}%
\pgfpathlineto{\pgfqpoint{2.846344in}{0.938193in}}%
\pgfpathlineto{\pgfqpoint{2.850639in}{0.948893in}}%
\pgfpathlineto{\pgfqpoint{2.851713in}{0.954463in}}%
\pgfpathlineto{\pgfqpoint{2.852787in}{0.952777in}}%
\pgfpathlineto{\pgfqpoint{2.853861in}{0.959886in}}%
\pgfpathlineto{\pgfqpoint{2.857082in}{0.959153in}}%
\pgfpathlineto{\pgfqpoint{2.859230in}{0.969413in}}%
\pgfpathlineto{\pgfqpoint{2.860304in}{0.968387in}}%
\pgfpathlineto{\pgfqpoint{2.861378in}{0.979160in}}%
\pgfpathlineto{\pgfqpoint{2.864599in}{0.984730in}}%
\pgfpathlineto{\pgfqpoint{2.865673in}{0.982019in}}%
\pgfpathlineto{\pgfqpoint{2.866747in}{0.988175in}}%
\pgfpathlineto{\pgfqpoint{2.867821in}{0.978940in}}%
\pgfpathlineto{\pgfqpoint{2.868895in}{0.975936in}}%
\pgfpathlineto{\pgfqpoint{2.872116in}{0.978867in}}%
\pgfpathlineto{\pgfqpoint{2.873190in}{0.988321in}}%
\pgfpathlineto{\pgfqpoint{2.874264in}{0.991326in}}%
\pgfpathlineto{\pgfqpoint{2.875338in}{0.986342in}}%
\pgfpathlineto{\pgfqpoint{2.876411in}{0.988468in}}%
\pgfpathlineto{\pgfqpoint{2.879633in}{0.992792in}}%
\pgfpathlineto{\pgfqpoint{2.881781in}{0.989054in}}%
\pgfpathlineto{\pgfqpoint{2.882855in}{0.978354in}}%
\pgfpathlineto{\pgfqpoint{2.888224in}{1.001073in}}%
\pgfpathlineto{\pgfqpoint{2.889298in}{1.007229in}}%
\pgfpathlineto{\pgfqpoint{2.890371in}{0.998875in}}%
\pgfpathlineto{\pgfqpoint{2.891445in}{1.001366in}}%
\pgfpathlineto{\pgfqpoint{2.894667in}{1.006496in}}%
\pgfpathlineto{\pgfqpoint{2.895741in}{1.012652in}}%
\pgfpathlineto{\pgfqpoint{2.896814in}{1.006203in}}%
\pgfpathlineto{\pgfqpoint{2.897888in}{1.006496in}}%
\pgfpathlineto{\pgfqpoint{2.903257in}{1.010527in}}%
\pgfpathlineto{\pgfqpoint{2.904331in}{1.010234in}}%
\pgfpathlineto{\pgfqpoint{2.905405in}{1.008695in}}%
\pgfpathlineto{\pgfqpoint{2.906479in}{1.011407in}}%
\pgfpathlineto{\pgfqpoint{2.910774in}{1.005031in}}%
\pgfpathlineto{\pgfqpoint{2.911848in}{1.006203in}}%
\pgfpathlineto{\pgfqpoint{2.912922in}{1.016244in}}%
\pgfpathlineto{\pgfqpoint{2.913996in}{1.015291in}}%
\pgfpathlineto{\pgfqpoint{2.917217in}{1.027676in}}%
\pgfpathlineto{\pgfqpoint{2.918291in}{1.028116in}}%
\pgfpathlineto{\pgfqpoint{2.919365in}{1.025404in}}%
\pgfpathlineto{\pgfqpoint{2.920439in}{1.027237in}}%
\pgfpathlineto{\pgfqpoint{2.921513in}{1.021593in}}%
\pgfpathlineto{\pgfqpoint{2.924734in}{1.018149in}}%
\pgfpathlineto{\pgfqpoint{2.925808in}{1.022326in}}%
\pgfpathlineto{\pgfqpoint{2.926882in}{1.019175in}}%
\pgfpathlineto{\pgfqpoint{2.929030in}{1.023426in}}%
\pgfpathlineto{\pgfqpoint{2.932251in}{1.005104in}}%
\pgfpathlineto{\pgfqpoint{2.933325in}{1.004518in}}%
\pgfpathlineto{\pgfqpoint{2.935473in}{1.016390in}}%
\pgfpathlineto{\pgfqpoint{2.936546in}{1.021154in}}%
\pgfpathlineto{\pgfqpoint{2.939768in}{1.022546in}}%
\pgfpathlineto{\pgfqpoint{2.940842in}{1.023646in}}%
\pgfpathlineto{\pgfqpoint{2.941916in}{1.021667in}}%
\pgfpathlineto{\pgfqpoint{2.944063in}{1.031927in}}%
\pgfpathlineto{\pgfqpoint{2.947285in}{1.033906in}}%
\pgfpathlineto{\pgfqpoint{2.948359in}{1.036104in}}%
\pgfpathlineto{\pgfqpoint{2.949433in}{1.044312in}}%
\pgfpathlineto{\pgfqpoint{2.950506in}{1.041894in}}%
\pgfpathlineto{\pgfqpoint{2.951580in}{1.045851in}}%
\pgfpathlineto{\pgfqpoint{2.954802in}{1.043286in}}%
\pgfpathlineto{\pgfqpoint{2.955876in}{1.037790in}}%
\pgfpathlineto{\pgfqpoint{2.956949in}{1.039622in}}%
\pgfpathlineto{\pgfqpoint{2.958023in}{1.033979in}}%
\pgfpathlineto{\pgfqpoint{2.959097in}{1.037863in}}%
\pgfpathlineto{\pgfqpoint{2.962319in}{1.037717in}}%
\pgfpathlineto{\pgfqpoint{2.965540in}{1.054133in}}%
\pgfpathlineto{\pgfqpoint{2.966614in}{1.053473in}}%
\pgfpathlineto{\pgfqpoint{2.969835in}{1.055819in}}%
\pgfpathlineto{\pgfqpoint{2.970909in}{1.055379in}}%
\pgfpathlineto{\pgfqpoint{2.971983in}{1.063294in}}%
\pgfpathlineto{\pgfqpoint{2.973057in}{1.062634in}}%
\pgfpathlineto{\pgfqpoint{2.974131in}{1.065419in}}%
\pgfpathlineto{\pgfqpoint{2.978426in}{1.070696in}}%
\pgfpathlineto{\pgfqpoint{2.979500in}{1.073407in}}%
\pgfpathlineto{\pgfqpoint{2.981648in}{1.069157in}}%
\pgfpathlineto{\pgfqpoint{2.984869in}{1.065712in}}%
\pgfpathlineto{\pgfqpoint{2.985943in}{1.069890in}}%
\pgfpathlineto{\pgfqpoint{2.987017in}{1.057358in}}%
\pgfpathlineto{\pgfqpoint{2.988091in}{1.064393in}}%
\pgfpathlineto{\pgfqpoint{2.989165in}{1.055672in}}%
\pgfpathlineto{\pgfqpoint{2.992386in}{1.056625in}}%
\pgfpathlineto{\pgfqpoint{2.993460in}{1.067471in}}%
\pgfpathlineto{\pgfqpoint{2.994534in}{1.062927in}}%
\pgfpathlineto{\pgfqpoint{2.996681in}{1.071209in}}%
\pgfpathlineto{\pgfqpoint{2.999903in}{1.074214in}}%
\pgfpathlineto{\pgfqpoint{3.000977in}{1.081322in}}%
\pgfpathlineto{\pgfqpoint{3.002051in}{1.055965in}}%
\pgfpathlineto{\pgfqpoint{3.003124in}{1.075753in}}%
\pgfpathlineto{\pgfqpoint{3.004198in}{1.062268in}}%
\pgfpathlineto{\pgfqpoint{3.007420in}{1.038376in}}%
\pgfpathlineto{\pgfqpoint{3.010641in}{1.067105in}}%
\pgfpathlineto{\pgfqpoint{3.011715in}{1.075826in}}%
\pgfpathlineto{\pgfqpoint{3.014937in}{1.074140in}}%
\pgfpathlineto{\pgfqpoint{3.016011in}{1.082055in}}%
\pgfpathlineto{\pgfqpoint{3.017084in}{1.080956in}}%
\pgfpathlineto{\pgfqpoint{3.018158in}{1.078098in}}%
\pgfpathlineto{\pgfqpoint{3.019232in}{1.084620in}}%
\pgfpathlineto{\pgfqpoint{3.022454in}{1.082715in}}%
\pgfpathlineto{\pgfqpoint{3.023527in}{1.072455in}}%
\pgfpathlineto{\pgfqpoint{3.024601in}{1.072308in}}%
\pgfpathlineto{\pgfqpoint{3.026749in}{1.076485in}}%
\pgfpathlineto{\pgfqpoint{3.031044in}{1.078611in}}%
\pgfpathlineto{\pgfqpoint{3.032118in}{1.086086in}}%
\pgfpathlineto{\pgfqpoint{3.033192in}{1.088504in}}%
\pgfpathlineto{\pgfqpoint{3.034266in}{1.086086in}}%
\pgfpathlineto{\pgfqpoint{3.034266in}{1.086086in}}%
\pgfusepath{stroke}%
\end{pgfscope}%
\begin{pgfscope}%
\pgfpathrectangle{\pgfqpoint{0.568357in}{0.385400in}}{\pgfqpoint{2.583333in}{0.736585in}}%
\pgfusepath{clip}%
\pgfsetroundcap%
\pgfsetroundjoin%
\pgfsetlinewidth{1.505625pt}%
\definecolor{currentstroke}{rgb}{1.000000,0.647059,0.000000}%
\pgfsetstrokecolor{currentstroke}%
\pgfsetdash{}{0pt}%
\pgfpathmoveto{\pgfqpoint{0.685781in}{0.425038in}}%
\pgfpathlineto{\pgfqpoint{0.686855in}{0.423572in}}%
\pgfpathlineto{\pgfqpoint{0.689002in}{0.422986in}}%
\pgfpathlineto{\pgfqpoint{0.696519in}{0.421642in}}%
\pgfpathlineto{\pgfqpoint{0.711553in}{0.422057in}}%
\pgfpathlineto{\pgfqpoint{0.717996in}{0.422396in}}%
\pgfpathlineto{\pgfqpoint{0.719070in}{0.422794in}}%
\pgfpathlineto{\pgfqpoint{0.724439in}{0.423904in}}%
\pgfpathlineto{\pgfqpoint{0.726587in}{0.425153in}}%
\pgfpathlineto{\pgfqpoint{0.730882in}{0.426320in}}%
\pgfpathlineto{\pgfqpoint{0.734104in}{0.428033in}}%
\pgfpathlineto{\pgfqpoint{0.739473in}{0.429073in}}%
\pgfpathlineto{\pgfqpoint{0.741621in}{0.430150in}}%
\pgfpathlineto{\pgfqpoint{0.745916in}{0.431192in}}%
\pgfpathlineto{\pgfqpoint{0.749137in}{0.432413in}}%
\pgfpathlineto{\pgfqpoint{0.754507in}{0.433365in}}%
\pgfpathlineto{\pgfqpoint{0.756654in}{0.434067in}}%
\pgfpathlineto{\pgfqpoint{0.762023in}{0.434947in}}%
\pgfpathlineto{\pgfqpoint{0.767393in}{0.435789in}}%
\pgfpathlineto{\pgfqpoint{0.779205in}{0.438031in}}%
\pgfpathlineto{\pgfqpoint{0.792091in}{0.439468in}}%
\pgfpathlineto{\pgfqpoint{0.801755in}{0.441025in}}%
\pgfpathlineto{\pgfqpoint{0.808199in}{0.441705in}}%
\pgfpathlineto{\pgfqpoint{0.814642in}{0.442508in}}%
\pgfpathlineto{\pgfqpoint{0.846857in}{0.444027in}}%
\pgfpathlineto{\pgfqpoint{0.866186in}{0.444660in}}%
\pgfpathlineto{\pgfqpoint{0.881220in}{0.445921in}}%
\pgfpathlineto{\pgfqpoint{0.928468in}{0.450048in}}%
\pgfpathlineto{\pgfqpoint{0.937059in}{0.450751in}}%
\pgfpathlineto{\pgfqpoint{0.955314in}{0.451862in}}%
\pgfpathlineto{\pgfqpoint{0.974643in}{0.453919in}}%
\pgfpathlineto{\pgfqpoint{0.985382in}{0.454931in}}%
\pgfpathlineto{\pgfqpoint{1.004711in}{0.457141in}}%
\pgfpathlineto{\pgfqpoint{1.016523in}{0.457986in}}%
\pgfpathlineto{\pgfqpoint{1.027262in}{0.459272in}}%
\pgfpathlineto{\pgfqpoint{1.039074in}{0.460422in}}%
\pgfpathlineto{\pgfqpoint{1.049812in}{0.461950in}}%
\pgfpathlineto{\pgfqpoint{1.060551in}{0.463021in}}%
\pgfpathlineto{\pgfqpoint{1.068067in}{0.463967in}}%
\pgfpathlineto{\pgfqpoint{1.084175in}{0.465747in}}%
\pgfpathlineto{\pgfqpoint{1.092766in}{0.467132in}}%
\pgfpathlineto{\pgfqpoint{1.094913in}{0.467558in}}%
\pgfpathlineto{\pgfqpoint{1.105652in}{0.468585in}}%
\pgfpathlineto{\pgfqpoint{1.117464in}{0.470288in}}%
\pgfpathlineto{\pgfqpoint{1.130350in}{0.471520in}}%
\pgfpathlineto{\pgfqpoint{1.140015in}{0.472779in}}%
\pgfpathlineto{\pgfqpoint{1.150753in}{0.473891in}}%
\pgfpathlineto{\pgfqpoint{1.162565in}{0.475394in}}%
\pgfpathlineto{\pgfqpoint{1.174377in}{0.476626in}}%
\pgfpathlineto{\pgfqpoint{1.185116in}{0.478132in}}%
\pgfpathlineto{\pgfqpoint{1.195854in}{0.479117in}}%
\pgfpathlineto{\pgfqpoint{1.215183in}{0.481975in}}%
\pgfpathlineto{\pgfqpoint{1.221626in}{0.482903in}}%
\pgfpathlineto{\pgfqpoint{1.228069in}{0.483855in}}%
\pgfpathlineto{\pgfqpoint{1.230217in}{0.484298in}}%
\pgfpathlineto{\pgfqpoint{1.240955in}{0.485387in}}%
\pgfpathlineto{\pgfqpoint{1.252768in}{0.487298in}}%
\pgfpathlineto{\pgfqpoint{1.263506in}{0.488541in}}%
\pgfpathlineto{\pgfqpoint{1.275318in}{0.490277in}}%
\pgfpathlineto{\pgfqpoint{1.281761in}{0.491174in}}%
\pgfpathlineto{\pgfqpoint{1.290352in}{0.492541in}}%
\pgfpathlineto{\pgfqpoint{1.296795in}{0.493431in}}%
\pgfpathlineto{\pgfqpoint{1.303238in}{0.494271in}}%
\pgfpathlineto{\pgfqpoint{1.312903in}{0.495509in}}%
\pgfpathlineto{\pgfqpoint{1.324715in}{0.496566in}}%
\pgfpathlineto{\pgfqpoint{1.335453in}{0.497751in}}%
\pgfpathlineto{\pgfqpoint{1.347265in}{0.498656in}}%
\pgfpathlineto{\pgfqpoint{1.358004in}{0.500227in}}%
\pgfpathlineto{\pgfqpoint{1.368742in}{0.501415in}}%
\pgfpathlineto{\pgfqpoint{1.378407in}{0.502628in}}%
\pgfpathlineto{\pgfqpoint{1.395588in}{0.505019in}}%
\pgfpathlineto{\pgfqpoint{1.406327in}{0.506245in}}%
\pgfpathlineto{\pgfqpoint{1.418139in}{0.507988in}}%
\pgfpathlineto{\pgfqpoint{1.428877in}{0.509164in}}%
\pgfpathlineto{\pgfqpoint{1.433173in}{0.509778in}}%
\pgfpathlineto{\pgfqpoint{1.443911in}{0.510958in}}%
\pgfpathlineto{\pgfqpoint{1.463240in}{0.513858in}}%
\pgfpathlineto{\pgfqpoint{1.473978in}{0.515081in}}%
\pgfpathlineto{\pgfqpoint{1.485791in}{0.517276in}}%
\pgfpathlineto{\pgfqpoint{1.493307in}{0.518302in}}%
\pgfpathlineto{\pgfqpoint{1.499751in}{0.519185in}}%
\pgfpathlineto{\pgfqpoint{1.508341in}{0.520441in}}%
\pgfpathlineto{\pgfqpoint{1.514784in}{0.521353in}}%
\pgfpathlineto{\pgfqpoint{1.515858in}{0.521591in}}%
\pgfpathlineto{\pgfqpoint{1.523375in}{0.522514in}}%
\pgfpathlineto{\pgfqpoint{1.529818in}{0.523451in}}%
\pgfpathlineto{\pgfqpoint{1.538409in}{0.524789in}}%
\pgfpathlineto{\pgfqpoint{1.544852in}{0.525673in}}%
\pgfpathlineto{\pgfqpoint{1.553442in}{0.526947in}}%
\pgfpathlineto{\pgfqpoint{1.564181in}{0.528068in}}%
\pgfpathlineto{\pgfqpoint{1.574919in}{0.529332in}}%
\pgfpathlineto{\pgfqpoint{1.591027in}{0.530778in}}%
\pgfpathlineto{\pgfqpoint{1.602839in}{0.531734in}}%
\pgfpathlineto{\pgfqpoint{1.621094in}{0.533705in}}%
\pgfpathlineto{\pgfqpoint{1.632906in}{0.534660in}}%
\pgfpathlineto{\pgfqpoint{1.643645in}{0.535873in}}%
\pgfpathlineto{\pgfqpoint{1.656531in}{0.536998in}}%
\pgfpathlineto{\pgfqpoint{1.665122in}{0.537866in}}%
\pgfpathlineto{\pgfqpoint{1.676934in}{0.538791in}}%
\pgfpathlineto{\pgfqpoint{1.688746in}{0.540226in}}%
\pgfpathlineto{\pgfqpoint{1.701632in}{0.541299in}}%
\pgfpathlineto{\pgfqpoint{1.718814in}{0.542907in}}%
\pgfpathlineto{\pgfqpoint{1.732773in}{0.544009in}}%
\pgfpathlineto{\pgfqpoint{1.748881in}{0.545493in}}%
\pgfpathlineto{\pgfqpoint{1.762841in}{0.546591in}}%
\pgfpathlineto{\pgfqpoint{1.774653in}{0.547370in}}%
\pgfpathlineto{\pgfqpoint{1.806868in}{0.550671in}}%
\pgfpathlineto{\pgfqpoint{1.816533in}{0.552117in}}%
\pgfpathlineto{\pgfqpoint{1.828345in}{0.553433in}}%
\pgfpathlineto{\pgfqpoint{1.839083in}{0.555235in}}%
\pgfpathlineto{\pgfqpoint{1.845527in}{0.556089in}}%
\pgfpathlineto{\pgfqpoint{1.851970in}{0.556993in}}%
\pgfpathlineto{\pgfqpoint{1.889554in}{0.561506in}}%
\pgfpathlineto{\pgfqpoint{1.906735in}{0.563992in}}%
\pgfpathlineto{\pgfqpoint{1.914252in}{0.564882in}}%
\pgfpathlineto{\pgfqpoint{1.920695in}{0.565786in}}%
\pgfpathlineto{\pgfqpoint{1.929286in}{0.567133in}}%
\pgfpathlineto{\pgfqpoint{1.940024in}{0.568379in}}%
\pgfpathlineto{\pgfqpoint{1.951837in}{0.570181in}}%
\pgfpathlineto{\pgfqpoint{1.963649in}{0.571306in}}%
\pgfpathlineto{\pgfqpoint{1.974387in}{0.572805in}}%
\pgfpathlineto{\pgfqpoint{1.981904in}{0.573765in}}%
\pgfpathlineto{\pgfqpoint{1.992642in}{0.574890in}}%
\pgfpathlineto{\pgfqpoint{2.011971in}{0.577731in}}%
\pgfpathlineto{\pgfqpoint{2.022710in}{0.578739in}}%
\pgfpathlineto{\pgfqpoint{2.034522in}{0.580500in}}%
\pgfpathlineto{\pgfqpoint{2.045260in}{0.581663in}}%
\pgfpathlineto{\pgfqpoint{2.055999in}{0.583141in}}%
\pgfpathlineto{\pgfqpoint{2.064590in}{0.584028in}}%
\pgfpathlineto{\pgfqpoint{2.072106in}{0.585008in}}%
\pgfpathlineto{\pgfqpoint{2.078549in}{0.585845in}}%
\pgfpathlineto{\pgfqpoint{2.087140in}{0.587232in}}%
\pgfpathlineto{\pgfqpoint{2.093583in}{0.588137in}}%
\pgfpathlineto{\pgfqpoint{2.102174in}{0.589438in}}%
\pgfpathlineto{\pgfqpoint{2.112912in}{0.590672in}}%
\pgfpathlineto{\pgfqpoint{2.132241in}{0.593035in}}%
\pgfpathlineto{\pgfqpoint{2.142980in}{0.594115in}}%
\pgfpathlineto{\pgfqpoint{2.154792in}{0.595662in}}%
\pgfpathlineto{\pgfqpoint{2.165530in}{0.596840in}}%
\pgfpathlineto{\pgfqpoint{2.177343in}{0.598742in}}%
\pgfpathlineto{\pgfqpoint{2.183786in}{0.599647in}}%
\pgfpathlineto{\pgfqpoint{2.192376in}{0.600972in}}%
\pgfpathlineto{\pgfqpoint{2.198819in}{0.601876in}}%
\pgfpathlineto{\pgfqpoint{2.207410in}{0.603250in}}%
\pgfpathlineto{\pgfqpoint{2.219222in}{0.604618in}}%
\pgfpathlineto{\pgfqpoint{2.229961in}{0.606351in}}%
\pgfpathlineto{\pgfqpoint{2.236404in}{0.607206in}}%
\pgfpathlineto{\pgfqpoint{2.243921in}{0.608230in}}%
\pgfpathlineto{\pgfqpoint{2.255733in}{0.609256in}}%
\pgfpathlineto{\pgfqpoint{2.267545in}{0.610830in}}%
\pgfpathlineto{\pgfqpoint{2.279357in}{0.611768in}}%
\pgfpathlineto{\pgfqpoint{2.295465in}{0.613460in}}%
\pgfpathlineto{\pgfqpoint{2.312646in}{0.615132in}}%
\pgfpathlineto{\pgfqpoint{2.324458in}{0.616250in}}%
\pgfpathlineto{\pgfqpoint{2.341640in}{0.618083in}}%
\pgfpathlineto{\pgfqpoint{2.350231in}{0.618954in}}%
\pgfpathlineto{\pgfqpoint{2.360969in}{0.620055in}}%
\pgfpathlineto{\pgfqpoint{2.372781in}{0.621831in}}%
\pgfpathlineto{\pgfqpoint{2.383520in}{0.622918in}}%
\pgfpathlineto{\pgfqpoint{2.395332in}{0.624498in}}%
\pgfpathlineto{\pgfqpoint{2.406070in}{0.625528in}}%
\pgfpathlineto{\pgfqpoint{2.417882in}{0.626992in}}%
\pgfpathlineto{\pgfqpoint{2.428621in}{0.628116in}}%
\pgfpathlineto{\pgfqpoint{2.440433in}{0.629605in}}%
\pgfpathlineto{\pgfqpoint{2.453319in}{0.630627in}}%
\pgfpathlineto{\pgfqpoint{2.470501in}{0.632586in}}%
\pgfpathlineto{\pgfqpoint{2.482313in}{0.633741in}}%
\pgfpathlineto{\pgfqpoint{2.493051in}{0.635105in}}%
\pgfpathlineto{\pgfqpoint{2.503790in}{0.636154in}}%
\pgfpathlineto{\pgfqpoint{2.515602in}{0.637721in}}%
\pgfpathlineto{\pgfqpoint{2.526340in}{0.638632in}}%
\pgfpathlineto{\pgfqpoint{2.538152in}{0.640242in}}%
\pgfpathlineto{\pgfqpoint{2.548891in}{0.641302in}}%
\pgfpathlineto{\pgfqpoint{2.560703in}{0.642887in}}%
\pgfpathlineto{\pgfqpoint{2.571441in}{0.643922in}}%
\pgfpathlineto{\pgfqpoint{2.583254in}{0.645407in}}%
\pgfpathlineto{\pgfqpoint{2.595066in}{0.646539in}}%
\pgfpathlineto{\pgfqpoint{2.605804in}{0.647630in}}%
\pgfpathlineto{\pgfqpoint{2.618690in}{0.648727in}}%
\pgfpathlineto{\pgfqpoint{2.628355in}{0.649734in}}%
\pgfpathlineto{\pgfqpoint{2.642315in}{0.650834in}}%
\pgfpathlineto{\pgfqpoint{2.650905in}{0.651572in}}%
\pgfpathlineto{\pgfqpoint{2.664865in}{0.652798in}}%
\pgfpathlineto{\pgfqpoint{2.673456in}{0.653760in}}%
\pgfpathlineto{\pgfqpoint{2.685268in}{0.654913in}}%
\pgfpathlineto{\pgfqpoint{2.696007in}{0.656327in}}%
\pgfpathlineto{\pgfqpoint{2.706745in}{0.657246in}}%
\pgfpathlineto{\pgfqpoint{2.718557in}{0.658926in}}%
\pgfpathlineto{\pgfqpoint{2.729296in}{0.660063in}}%
\pgfpathlineto{\pgfqpoint{2.741108in}{0.661703in}}%
\pgfpathlineto{\pgfqpoint{2.751846in}{0.662787in}}%
\pgfpathlineto{\pgfqpoint{2.763658in}{0.664251in}}%
\pgfpathlineto{\pgfqpoint{2.774397in}{0.665399in}}%
\pgfpathlineto{\pgfqpoint{2.786209in}{0.667132in}}%
\pgfpathlineto{\pgfqpoint{2.796947in}{0.668289in}}%
\pgfpathlineto{\pgfqpoint{2.801243in}{0.669097in}}%
\pgfpathlineto{\pgfqpoint{2.811981in}{0.670128in}}%
\pgfpathlineto{\pgfqpoint{2.823793in}{0.671935in}}%
\pgfpathlineto{\pgfqpoint{2.834532in}{0.673119in}}%
\pgfpathlineto{\pgfqpoint{2.842049in}{0.674093in}}%
\pgfpathlineto{\pgfqpoint{2.858156in}{0.676070in}}%
\pgfpathlineto{\pgfqpoint{2.868895in}{0.677795in}}%
\pgfpathlineto{\pgfqpoint{2.875338in}{0.678672in}}%
\pgfpathlineto{\pgfqpoint{2.883928in}{0.679983in}}%
\pgfpathlineto{\pgfqpoint{2.890371in}{0.680889in}}%
\pgfpathlineto{\pgfqpoint{2.898962in}{0.682264in}}%
\pgfpathlineto{\pgfqpoint{2.905405in}{0.683183in}}%
\pgfpathlineto{\pgfqpoint{2.906479in}{0.683413in}}%
\pgfpathlineto{\pgfqpoint{2.913996in}{0.684328in}}%
\pgfpathlineto{\pgfqpoint{2.920439in}{0.685284in}}%
\pgfpathlineto{\pgfqpoint{2.929030in}{0.686682in}}%
\pgfpathlineto{\pgfqpoint{2.935473in}{0.687575in}}%
\pgfpathlineto{\pgfqpoint{2.944063in}{0.688970in}}%
\pgfpathlineto{\pgfqpoint{2.950506in}{0.689932in}}%
\pgfpathlineto{\pgfqpoint{2.959097in}{0.691369in}}%
\pgfpathlineto{\pgfqpoint{2.965540in}{0.692339in}}%
\pgfpathlineto{\pgfqpoint{2.974131in}{0.693836in}}%
\pgfpathlineto{\pgfqpoint{2.980574in}{0.694860in}}%
\pgfpathlineto{\pgfqpoint{2.989165in}{0.696354in}}%
\pgfpathlineto{\pgfqpoint{2.999903in}{0.697599in}}%
\pgfpathlineto{\pgfqpoint{3.004198in}{0.698596in}}%
\pgfpathlineto{\pgfqpoint{3.010641in}{0.699541in}}%
\pgfpathlineto{\pgfqpoint{3.019232in}{0.701060in}}%
\pgfpathlineto{\pgfqpoint{3.025675in}{0.702056in}}%
\pgfpathlineto{\pgfqpoint{3.026749in}{0.702305in}}%
\pgfpathlineto{\pgfqpoint{3.033192in}{0.703065in}}%
\pgfpathlineto{\pgfqpoint{3.034266in}{0.703319in}}%
\pgfpathlineto{\pgfqpoint{3.034266in}{0.703319in}}%
\pgfusepath{stroke}%
\end{pgfscope}%
\begin{pgfscope}%
\pgfsetrectcap%
\pgfsetmiterjoin%
\pgfsetlinewidth{0.803000pt}%
\definecolor{currentstroke}{rgb}{1.000000,1.000000,1.000000}%
\pgfsetstrokecolor{currentstroke}%
\pgfsetdash{}{0pt}%
\pgfpathmoveto{\pgfqpoint{0.568357in}{0.385400in}}%
\pgfpathlineto{\pgfqpoint{0.568357in}{1.121986in}}%
\pgfusepath{stroke}%
\end{pgfscope}%
\begin{pgfscope}%
\pgfsetrectcap%
\pgfsetmiterjoin%
\pgfsetlinewidth{0.803000pt}%
\definecolor{currentstroke}{rgb}{1.000000,1.000000,1.000000}%
\pgfsetstrokecolor{currentstroke}%
\pgfsetdash{}{0pt}%
\pgfpathmoveto{\pgfqpoint{3.151690in}{0.385400in}}%
\pgfpathlineto{\pgfqpoint{3.151690in}{1.121986in}}%
\pgfusepath{stroke}%
\end{pgfscope}%
\begin{pgfscope}%
\pgfsetrectcap%
\pgfsetmiterjoin%
\pgfsetlinewidth{0.803000pt}%
\definecolor{currentstroke}{rgb}{1.000000,1.000000,1.000000}%
\pgfsetstrokecolor{currentstroke}%
\pgfsetdash{}{0pt}%
\pgfpathmoveto{\pgfqpoint{0.568357in}{0.385400in}}%
\pgfpathlineto{\pgfqpoint{3.151690in}{0.385400in}}%
\pgfusepath{stroke}%
\end{pgfscope}%
\begin{pgfscope}%
\pgfsetrectcap%
\pgfsetmiterjoin%
\pgfsetlinewidth{0.803000pt}%
\definecolor{currentstroke}{rgb}{1.000000,1.000000,1.000000}%
\pgfsetstrokecolor{currentstroke}%
\pgfsetdash{}{0pt}%
\pgfpathmoveto{\pgfqpoint{0.568357in}{1.121986in}}%
\pgfpathlineto{\pgfqpoint{3.151690in}{1.121986in}}%
\pgfusepath{stroke}%
\end{pgfscope}%
\begin{pgfscope}%
\definecolor{textcolor}{rgb}{0.150000,0.150000,0.150000}%
\pgfsetstrokecolor{textcolor}%
\pgfsetfillcolor{textcolor}%
\pgftext[x=1.860023in,y=1.205319in,,base]{\color{textcolor}\rmfamily\fontsize{16.800000}{20.160000}\selectfont V}%
\end{pgfscope}%
\begin{pgfscope}%
\pgfsetbuttcap%
\pgfsetmiterjoin%
\definecolor{currentfill}{rgb}{0.917647,0.917647,0.949020}%
\pgfsetfillcolor{currentfill}%
\pgfsetlinewidth{0.000000pt}%
\definecolor{currentstroke}{rgb}{0.000000,0.000000,0.000000}%
\pgfsetstrokecolor{currentstroke}%
\pgfsetstrokeopacity{0.000000}%
\pgfsetdash{}{0pt}%
\pgfpathmoveto{\pgfqpoint{4.185023in}{0.385400in}}%
\pgfpathlineto{\pgfqpoint{6.768357in}{0.385400in}}%
\pgfpathlineto{\pgfqpoint{6.768357in}{1.121986in}}%
\pgfpathlineto{\pgfqpoint{4.185023in}{1.121986in}}%
\pgfpathclose%
\pgfusepath{fill}%
\end{pgfscope}%
\begin{pgfscope}%
\pgfpathrectangle{\pgfqpoint{4.185023in}{0.385400in}}{\pgfqpoint{2.583333in}{0.736585in}}%
\pgfusepath{clip}%
\pgfsetroundcap%
\pgfsetroundjoin%
\pgfsetlinewidth{0.803000pt}%
\definecolor{currentstroke}{rgb}{1.000000,1.000000,1.000000}%
\pgfsetstrokecolor{currentstroke}%
\pgfsetdash{}{0pt}%
\pgfpathmoveto{\pgfqpoint{4.300300in}{0.385400in}}%
\pgfpathlineto{\pgfqpoint{4.300300in}{1.121986in}}%
\pgfusepath{stroke}%
\end{pgfscope}%
\begin{pgfscope}%
\definecolor{textcolor}{rgb}{0.150000,0.150000,0.150000}%
\pgfsetstrokecolor{textcolor}%
\pgfsetfillcolor{textcolor}%
\pgftext[x=4.300300in,y=0.288178in,,top]{\color{textcolor}\rmfamily\fontsize{14.000000}{16.800000}\selectfont 2012}%
\end{pgfscope}%
\begin{pgfscope}%
\pgfpathrectangle{\pgfqpoint{4.185023in}{0.385400in}}{\pgfqpoint{2.583333in}{0.736585in}}%
\pgfusepath{clip}%
\pgfsetroundcap%
\pgfsetroundjoin%
\pgfsetlinewidth{0.803000pt}%
\definecolor{currentstroke}{rgb}{1.000000,1.000000,1.000000}%
\pgfsetstrokecolor{currentstroke}%
\pgfsetdash{}{0pt}%
\pgfpathmoveto{\pgfqpoint{4.693325in}{0.385400in}}%
\pgfpathlineto{\pgfqpoint{4.693325in}{1.121986in}}%
\pgfusepath{stroke}%
\end{pgfscope}%
\begin{pgfscope}%
\definecolor{textcolor}{rgb}{0.150000,0.150000,0.150000}%
\pgfsetstrokecolor{textcolor}%
\pgfsetfillcolor{textcolor}%
\pgftext[x=4.693325in,y=0.288178in,,top]{\color{textcolor}\rmfamily\fontsize{14.000000}{16.800000}\selectfont 2013}%
\end{pgfscope}%
\begin{pgfscope}%
\pgfpathrectangle{\pgfqpoint{4.185023in}{0.385400in}}{\pgfqpoint{2.583333in}{0.736585in}}%
\pgfusepath{clip}%
\pgfsetroundcap%
\pgfsetroundjoin%
\pgfsetlinewidth{0.803000pt}%
\definecolor{currentstroke}{rgb}{1.000000,1.000000,1.000000}%
\pgfsetstrokecolor{currentstroke}%
\pgfsetdash{}{0pt}%
\pgfpathmoveto{\pgfqpoint{5.085276in}{0.385400in}}%
\pgfpathlineto{\pgfqpoint{5.085276in}{1.121986in}}%
\pgfusepath{stroke}%
\end{pgfscope}%
\begin{pgfscope}%
\definecolor{textcolor}{rgb}{0.150000,0.150000,0.150000}%
\pgfsetstrokecolor{textcolor}%
\pgfsetfillcolor{textcolor}%
\pgftext[x=5.085276in,y=0.288178in,,top]{\color{textcolor}\rmfamily\fontsize{14.000000}{16.800000}\selectfont 2014}%
\end{pgfscope}%
\begin{pgfscope}%
\pgfpathrectangle{\pgfqpoint{4.185023in}{0.385400in}}{\pgfqpoint{2.583333in}{0.736585in}}%
\pgfusepath{clip}%
\pgfsetroundcap%
\pgfsetroundjoin%
\pgfsetlinewidth{0.803000pt}%
\definecolor{currentstroke}{rgb}{1.000000,1.000000,1.000000}%
\pgfsetstrokecolor{currentstroke}%
\pgfsetdash{}{0pt}%
\pgfpathmoveto{\pgfqpoint{5.477227in}{0.385400in}}%
\pgfpathlineto{\pgfqpoint{5.477227in}{1.121986in}}%
\pgfusepath{stroke}%
\end{pgfscope}%
\begin{pgfscope}%
\definecolor{textcolor}{rgb}{0.150000,0.150000,0.150000}%
\pgfsetstrokecolor{textcolor}%
\pgfsetfillcolor{textcolor}%
\pgftext[x=5.477227in,y=0.288178in,,top]{\color{textcolor}\rmfamily\fontsize{14.000000}{16.800000}\selectfont 2015}%
\end{pgfscope}%
\begin{pgfscope}%
\pgfpathrectangle{\pgfqpoint{4.185023in}{0.385400in}}{\pgfqpoint{2.583333in}{0.736585in}}%
\pgfusepath{clip}%
\pgfsetroundcap%
\pgfsetroundjoin%
\pgfsetlinewidth{0.803000pt}%
\definecolor{currentstroke}{rgb}{1.000000,1.000000,1.000000}%
\pgfsetstrokecolor{currentstroke}%
\pgfsetdash{}{0pt}%
\pgfpathmoveto{\pgfqpoint{5.869178in}{0.385400in}}%
\pgfpathlineto{\pgfqpoint{5.869178in}{1.121986in}}%
\pgfusepath{stroke}%
\end{pgfscope}%
\begin{pgfscope}%
\definecolor{textcolor}{rgb}{0.150000,0.150000,0.150000}%
\pgfsetstrokecolor{textcolor}%
\pgfsetfillcolor{textcolor}%
\pgftext[x=5.869178in,y=0.288178in,,top]{\color{textcolor}\rmfamily\fontsize{14.000000}{16.800000}\selectfont 2016}%
\end{pgfscope}%
\begin{pgfscope}%
\pgfpathrectangle{\pgfqpoint{4.185023in}{0.385400in}}{\pgfqpoint{2.583333in}{0.736585in}}%
\pgfusepath{clip}%
\pgfsetroundcap%
\pgfsetroundjoin%
\pgfsetlinewidth{0.803000pt}%
\definecolor{currentstroke}{rgb}{1.000000,1.000000,1.000000}%
\pgfsetstrokecolor{currentstroke}%
\pgfsetdash{}{0pt}%
\pgfpathmoveto{\pgfqpoint{6.262203in}{0.385400in}}%
\pgfpathlineto{\pgfqpoint{6.262203in}{1.121986in}}%
\pgfusepath{stroke}%
\end{pgfscope}%
\begin{pgfscope}%
\definecolor{textcolor}{rgb}{0.150000,0.150000,0.150000}%
\pgfsetstrokecolor{textcolor}%
\pgfsetfillcolor{textcolor}%
\pgftext[x=6.262203in,y=0.288178in,,top]{\color{textcolor}\rmfamily\fontsize{14.000000}{16.800000}\selectfont 2017}%
\end{pgfscope}%
\begin{pgfscope}%
\pgfpathrectangle{\pgfqpoint{4.185023in}{0.385400in}}{\pgfqpoint{2.583333in}{0.736585in}}%
\pgfusepath{clip}%
\pgfsetroundcap%
\pgfsetroundjoin%
\pgfsetlinewidth{0.803000pt}%
\definecolor{currentstroke}{rgb}{1.000000,1.000000,1.000000}%
\pgfsetstrokecolor{currentstroke}%
\pgfsetdash{}{0pt}%
\pgfpathmoveto{\pgfqpoint{6.654154in}{0.385400in}}%
\pgfpathlineto{\pgfqpoint{6.654154in}{1.121986in}}%
\pgfusepath{stroke}%
\end{pgfscope}%
\begin{pgfscope}%
\definecolor{textcolor}{rgb}{0.150000,0.150000,0.150000}%
\pgfsetstrokecolor{textcolor}%
\pgfsetfillcolor{textcolor}%
\pgftext[x=6.654154in,y=0.288178in,,top]{\color{textcolor}\rmfamily\fontsize{14.000000}{16.800000}\selectfont 2018}%
\end{pgfscope}%
\begin{pgfscope}%
\pgfpathrectangle{\pgfqpoint{4.185023in}{0.385400in}}{\pgfqpoint{2.583333in}{0.736585in}}%
\pgfusepath{clip}%
\pgfsetroundcap%
\pgfsetroundjoin%
\pgfsetlinewidth{0.803000pt}%
\definecolor{currentstroke}{rgb}{1.000000,1.000000,1.000000}%
\pgfsetstrokecolor{currentstroke}%
\pgfsetdash{}{0pt}%
\pgfpathmoveto{\pgfqpoint{4.185023in}{0.548133in}}%
\pgfpathlineto{\pgfqpoint{6.768357in}{0.548133in}}%
\pgfusepath{stroke}%
\end{pgfscope}%
\begin{pgfscope}%
\definecolor{textcolor}{rgb}{0.150000,0.150000,0.150000}%
\pgfsetstrokecolor{textcolor}%
\pgfsetfillcolor{textcolor}%
\pgftext[x=3.840378in,y=0.474267in,left,base]{\color{textcolor}\rmfamily\fontsize{14.000000}{16.800000}\selectfont 50}%
\end{pgfscope}%
\begin{pgfscope}%
\pgfpathrectangle{\pgfqpoint{4.185023in}{0.385400in}}{\pgfqpoint{2.583333in}{0.736585in}}%
\pgfusepath{clip}%
\pgfsetroundcap%
\pgfsetroundjoin%
\pgfsetlinewidth{0.803000pt}%
\definecolor{currentstroke}{rgb}{1.000000,1.000000,1.000000}%
\pgfsetstrokecolor{currentstroke}%
\pgfsetdash{}{0pt}%
\pgfpathmoveto{\pgfqpoint{4.185023in}{0.965344in}}%
\pgfpathlineto{\pgfqpoint{6.768357in}{0.965344in}}%
\pgfusepath{stroke}%
\end{pgfscope}%
\begin{pgfscope}%
\definecolor{textcolor}{rgb}{0.150000,0.150000,0.150000}%
\pgfsetstrokecolor{textcolor}%
\pgfsetfillcolor{textcolor}%
\pgftext[x=3.716667in,y=0.891478in,left,base]{\color{textcolor}\rmfamily\fontsize{14.000000}{16.800000}\selectfont 100}%
\end{pgfscope}%
\begin{pgfscope}%
\pgfpathrectangle{\pgfqpoint{4.185023in}{0.385400in}}{\pgfqpoint{2.583333in}{0.736585in}}%
\pgfusepath{clip}%
\pgfsetroundcap%
\pgfsetroundjoin%
\pgfsetlinewidth{1.505625pt}%
\definecolor{currentstroke}{rgb}{0.090196,0.745098,0.811765}%
\pgfsetstrokecolor{currentstroke}%
\pgfsetdash{}{0pt}%
\pgfpathmoveto{\pgfqpoint{4.302448in}{0.418881in}}%
\pgfpathlineto{\pgfqpoint{4.305669in}{0.430897in}}%
\pgfpathlineto{\pgfqpoint{4.309964in}{0.428811in}}%
\pgfpathlineto{\pgfqpoint{4.311038in}{0.421802in}}%
\pgfpathlineto{\pgfqpoint{4.312112in}{0.421969in}}%
\pgfpathlineto{\pgfqpoint{4.313186in}{0.419549in}}%
\pgfpathlineto{\pgfqpoint{4.317481in}{0.420133in}}%
\pgfpathlineto{\pgfqpoint{4.319629in}{0.427309in}}%
\pgfpathlineto{\pgfqpoint{4.320703in}{0.426391in}}%
\pgfpathlineto{\pgfqpoint{4.324998in}{0.425891in}}%
\pgfpathlineto{\pgfqpoint{4.326072in}{0.428227in}}%
\pgfpathlineto{\pgfqpoint{4.328220in}{0.425891in}}%
\pgfpathlineto{\pgfqpoint{4.332515in}{0.423304in}}%
\pgfpathlineto{\pgfqpoint{4.333589in}{0.426475in}}%
\pgfpathlineto{\pgfqpoint{4.334663in}{0.423387in}}%
\pgfpathlineto{\pgfqpoint{4.335737in}{0.431565in}}%
\pgfpathlineto{\pgfqpoint{4.338958in}{0.434986in}}%
\pgfpathlineto{\pgfqpoint{4.341106in}{0.441077in}}%
\pgfpathlineto{\pgfqpoint{4.342180in}{0.443080in}}%
\pgfpathlineto{\pgfqpoint{4.343253in}{0.442412in}}%
\pgfpathlineto{\pgfqpoint{4.346475in}{0.444999in}}%
\pgfpathlineto{\pgfqpoint{4.347549in}{0.443580in}}%
\pgfpathlineto{\pgfqpoint{4.348623in}{0.440910in}}%
\pgfpathlineto{\pgfqpoint{4.350770in}{0.444665in}}%
\pgfpathlineto{\pgfqpoint{4.355066in}{0.443330in}}%
\pgfpathlineto{\pgfqpoint{4.356140in}{0.441077in}}%
\pgfpathlineto{\pgfqpoint{4.357213in}{0.442662in}}%
\pgfpathlineto{\pgfqpoint{4.358287in}{0.441411in}}%
\pgfpathlineto{\pgfqpoint{4.361509in}{0.443914in}}%
\pgfpathlineto{\pgfqpoint{4.362583in}{0.446084in}}%
\pgfpathlineto{\pgfqpoint{4.363656in}{0.446501in}}%
\pgfpathlineto{\pgfqpoint{4.364730in}{0.449505in}}%
\pgfpathlineto{\pgfqpoint{4.365804in}{0.449254in}}%
\pgfpathlineto{\pgfqpoint{4.369026in}{0.451841in}}%
\pgfpathlineto{\pgfqpoint{4.370099in}{0.446584in}}%
\pgfpathlineto{\pgfqpoint{4.371173in}{0.444665in}}%
\pgfpathlineto{\pgfqpoint{4.373321in}{0.448420in}}%
\pgfpathlineto{\pgfqpoint{4.376542in}{0.449171in}}%
\pgfpathlineto{\pgfqpoint{4.377616in}{0.461687in}}%
\pgfpathlineto{\pgfqpoint{4.378690in}{0.457682in}}%
\pgfpathlineto{\pgfqpoint{4.379764in}{0.457599in}}%
\pgfpathlineto{\pgfqpoint{4.380838in}{0.455513in}}%
\pgfpathlineto{\pgfqpoint{4.384059in}{0.457432in}}%
\pgfpathlineto{\pgfqpoint{4.385133in}{0.455930in}}%
\pgfpathlineto{\pgfqpoint{4.387281in}{0.456264in}}%
\pgfpathlineto{\pgfqpoint{4.388355in}{0.459017in}}%
\pgfpathlineto{\pgfqpoint{4.391576in}{0.464441in}}%
\pgfpathlineto{\pgfqpoint{4.392650in}{0.462772in}}%
\pgfpathlineto{\pgfqpoint{4.394798in}{0.454177in}}%
\pgfpathlineto{\pgfqpoint{4.395872in}{0.459935in}}%
\pgfpathlineto{\pgfqpoint{4.399093in}{0.460436in}}%
\pgfpathlineto{\pgfqpoint{4.401241in}{0.453593in}}%
\pgfpathlineto{\pgfqpoint{4.402315in}{0.454678in}}%
\pgfpathlineto{\pgfqpoint{4.406610in}{0.447419in}}%
\pgfpathlineto{\pgfqpoint{4.407684in}{0.438991in}}%
\pgfpathlineto{\pgfqpoint{4.408758in}{0.441828in}}%
\pgfpathlineto{\pgfqpoint{4.409831in}{0.447669in}}%
\pgfpathlineto{\pgfqpoint{4.410905in}{0.445416in}}%
\pgfpathlineto{\pgfqpoint{4.414127in}{0.443998in}}%
\pgfpathlineto{\pgfqpoint{4.415201in}{0.451674in}}%
\pgfpathlineto{\pgfqpoint{4.416274in}{0.450256in}}%
\pgfpathlineto{\pgfqpoint{4.417348in}{0.447168in}}%
\pgfpathlineto{\pgfqpoint{4.418422in}{0.449171in}}%
\pgfpathlineto{\pgfqpoint{4.421644in}{0.446668in}}%
\pgfpathlineto{\pgfqpoint{4.422717in}{0.447919in}}%
\pgfpathlineto{\pgfqpoint{4.424865in}{0.456764in}}%
\pgfpathlineto{\pgfqpoint{4.425939in}{0.456764in}}%
\pgfpathlineto{\pgfqpoint{4.429161in}{0.454928in}}%
\pgfpathlineto{\pgfqpoint{4.430234in}{0.460018in}}%
\pgfpathlineto{\pgfqpoint{4.431308in}{0.458183in}}%
\pgfpathlineto{\pgfqpoint{4.432382in}{0.460185in}}%
\pgfpathlineto{\pgfqpoint{4.433456in}{0.453593in}}%
\pgfpathlineto{\pgfqpoint{4.436677in}{0.460269in}}%
\pgfpathlineto{\pgfqpoint{4.440973in}{0.473369in}}%
\pgfpathlineto{\pgfqpoint{4.445268in}{0.469197in}}%
\pgfpathlineto{\pgfqpoint{4.446342in}{0.469698in}}%
\pgfpathlineto{\pgfqpoint{4.448490in}{0.460185in}}%
\pgfpathlineto{\pgfqpoint{4.451711in}{0.464524in}}%
\pgfpathlineto{\pgfqpoint{4.452785in}{0.464524in}}%
\pgfpathlineto{\pgfqpoint{4.453859in}{0.463189in}}%
\pgfpathlineto{\pgfqpoint{4.454933in}{0.464941in}}%
\pgfpathlineto{\pgfqpoint{4.456006in}{0.465359in}}%
\pgfpathlineto{\pgfqpoint{4.460302in}{0.472702in}}%
\pgfpathlineto{\pgfqpoint{4.461376in}{0.470616in}}%
\pgfpathlineto{\pgfqpoint{4.462450in}{0.474454in}}%
\pgfpathlineto{\pgfqpoint{4.463523in}{0.464608in}}%
\pgfpathlineto{\pgfqpoint{4.466745in}{0.464691in}}%
\pgfpathlineto{\pgfqpoint{4.467819in}{0.467945in}}%
\pgfpathlineto{\pgfqpoint{4.468893in}{0.473119in}}%
\pgfpathlineto{\pgfqpoint{4.469966in}{0.473870in}}%
\pgfpathlineto{\pgfqpoint{4.471040in}{0.478459in}}%
\pgfpathlineto{\pgfqpoint{4.474262in}{0.475121in}}%
\pgfpathlineto{\pgfqpoint{4.475336in}{0.479460in}}%
\pgfpathlineto{\pgfqpoint{4.476409in}{0.478376in}}%
\pgfpathlineto{\pgfqpoint{4.477483in}{0.485552in}}%
\pgfpathlineto{\pgfqpoint{4.478557in}{0.484801in}}%
\pgfpathlineto{\pgfqpoint{4.481779in}{0.484884in}}%
\pgfpathlineto{\pgfqpoint{4.483926in}{0.489640in}}%
\pgfpathlineto{\pgfqpoint{4.485000in}{0.487137in}}%
\pgfpathlineto{\pgfqpoint{4.486074in}{0.487721in}}%
\pgfpathlineto{\pgfqpoint{4.489295in}{0.481880in}}%
\pgfpathlineto{\pgfqpoint{4.491443in}{0.490725in}}%
\pgfpathlineto{\pgfqpoint{4.492517in}{0.490308in}}%
\pgfpathlineto{\pgfqpoint{4.493591in}{0.495398in}}%
\pgfpathlineto{\pgfqpoint{4.496812in}{0.497067in}}%
\pgfpathlineto{\pgfqpoint{4.497886in}{0.496149in}}%
\pgfpathlineto{\pgfqpoint{4.500034in}{0.492477in}}%
\pgfpathlineto{\pgfqpoint{4.501108in}{0.491977in}}%
\pgfpathlineto{\pgfqpoint{4.504329in}{0.491643in}}%
\pgfpathlineto{\pgfqpoint{4.505403in}{0.486887in}}%
\pgfpathlineto{\pgfqpoint{4.506477in}{0.486219in}}%
\pgfpathlineto{\pgfqpoint{4.507551in}{0.487221in}}%
\pgfpathlineto{\pgfqpoint{4.508625in}{0.493145in}}%
\pgfpathlineto{\pgfqpoint{4.511846in}{0.490642in}}%
\pgfpathlineto{\pgfqpoint{4.512920in}{0.501823in}}%
\pgfpathlineto{\pgfqpoint{4.513994in}{0.501823in}}%
\pgfpathlineto{\pgfqpoint{4.516141in}{0.496149in}}%
\pgfpathlineto{\pgfqpoint{4.519363in}{0.491560in}}%
\pgfpathlineto{\pgfqpoint{4.521511in}{0.493896in}}%
\pgfpathlineto{\pgfqpoint{4.522584in}{0.504493in}}%
\pgfpathlineto{\pgfqpoint{4.523658in}{0.506245in}}%
\pgfpathlineto{\pgfqpoint{4.526880in}{0.505244in}}%
\pgfpathlineto{\pgfqpoint{4.529028in}{0.497818in}}%
\pgfpathlineto{\pgfqpoint{4.530101in}{0.499069in}}%
\pgfpathlineto{\pgfqpoint{4.531175in}{0.504994in}}%
\pgfpathlineto{\pgfqpoint{4.534397in}{0.504076in}}%
\pgfpathlineto{\pgfqpoint{4.535471in}{0.505244in}}%
\pgfpathlineto{\pgfqpoint{4.536544in}{0.510417in}}%
\pgfpathlineto{\pgfqpoint{4.538692in}{0.504076in}}%
\pgfpathlineto{\pgfqpoint{4.541914in}{0.505661in}}%
\pgfpathlineto{\pgfqpoint{4.542987in}{0.504410in}}%
\pgfpathlineto{\pgfqpoint{4.544061in}{0.505912in}}%
\pgfpathlineto{\pgfqpoint{4.546209in}{0.510167in}}%
\pgfpathlineto{\pgfqpoint{4.549430in}{0.510084in}}%
\pgfpathlineto{\pgfqpoint{4.550504in}{0.503992in}}%
\pgfpathlineto{\pgfqpoint{4.551578in}{0.504159in}}%
\pgfpathlineto{\pgfqpoint{4.552652in}{0.500488in}}%
\pgfpathlineto{\pgfqpoint{4.553726in}{0.503408in}}%
\pgfpathlineto{\pgfqpoint{4.558021in}{0.503909in}}%
\pgfpathlineto{\pgfqpoint{4.559095in}{0.506996in}}%
\pgfpathlineto{\pgfqpoint{4.560169in}{0.502324in}}%
\pgfpathlineto{\pgfqpoint{4.565538in}{0.504159in}}%
\pgfpathlineto{\pgfqpoint{4.567686in}{0.520681in}}%
\pgfpathlineto{\pgfqpoint{4.568760in}{0.519763in}}%
\pgfpathlineto{\pgfqpoint{4.571981in}{0.518094in}}%
\pgfpathlineto{\pgfqpoint{4.573055in}{0.518428in}}%
\pgfpathlineto{\pgfqpoint{4.574129in}{0.519513in}}%
\pgfpathlineto{\pgfqpoint{4.575203in}{0.526271in}}%
\pgfpathlineto{\pgfqpoint{4.576276in}{0.524352in}}%
\pgfpathlineto{\pgfqpoint{4.579498in}{0.522850in}}%
\pgfpathlineto{\pgfqpoint{4.580572in}{0.521015in}}%
\pgfpathlineto{\pgfqpoint{4.581646in}{0.527022in}}%
\pgfpathlineto{\pgfqpoint{4.582719in}{0.526689in}}%
\pgfpathlineto{\pgfqpoint{4.587015in}{0.528691in}}%
\pgfpathlineto{\pgfqpoint{4.589162in}{0.521265in}}%
\pgfpathlineto{\pgfqpoint{4.590236in}{0.526105in}}%
\pgfpathlineto{\pgfqpoint{4.591310in}{0.523852in}}%
\pgfpathlineto{\pgfqpoint{4.594532in}{0.522266in}}%
\pgfpathlineto{\pgfqpoint{4.595605in}{0.519012in}}%
\pgfpathlineto{\pgfqpoint{4.596679in}{0.525270in}}%
\pgfpathlineto{\pgfqpoint{4.597753in}{0.526438in}}%
\pgfpathlineto{\pgfqpoint{4.598827in}{0.529025in}}%
\pgfpathlineto{\pgfqpoint{4.602049in}{0.524185in}}%
\pgfpathlineto{\pgfqpoint{4.603122in}{0.518094in}}%
\pgfpathlineto{\pgfqpoint{4.604196in}{0.515841in}}%
\pgfpathlineto{\pgfqpoint{4.605270in}{0.509249in}}%
\pgfpathlineto{\pgfqpoint{4.606344in}{0.511168in}}%
\pgfpathlineto{\pgfqpoint{4.609565in}{0.512670in}}%
\pgfpathlineto{\pgfqpoint{4.610639in}{0.516008in}}%
\pgfpathlineto{\pgfqpoint{4.611713in}{0.523935in}}%
\pgfpathlineto{\pgfqpoint{4.612787in}{0.524936in}}%
\pgfpathlineto{\pgfqpoint{4.613861in}{0.521015in}}%
\pgfpathlineto{\pgfqpoint{4.617082in}{0.520180in}}%
\pgfpathlineto{\pgfqpoint{4.618156in}{0.512420in}}%
\pgfpathlineto{\pgfqpoint{4.619230in}{0.511586in}}%
\pgfpathlineto{\pgfqpoint{4.621378in}{0.507330in}}%
\pgfpathlineto{\pgfqpoint{4.626747in}{0.500071in}}%
\pgfpathlineto{\pgfqpoint{4.627821in}{0.505077in}}%
\pgfpathlineto{\pgfqpoint{4.628894in}{0.505661in}}%
\pgfpathlineto{\pgfqpoint{4.633190in}{0.510251in}}%
\pgfpathlineto{\pgfqpoint{4.634264in}{0.507330in}}%
\pgfpathlineto{\pgfqpoint{4.635338in}{0.506996in}}%
\pgfpathlineto{\pgfqpoint{4.636411in}{0.484634in}}%
\pgfpathlineto{\pgfqpoint{4.639633in}{0.487554in}}%
\pgfpathlineto{\pgfqpoint{4.640707in}{0.491393in}}%
\pgfpathlineto{\pgfqpoint{4.641781in}{0.485468in}}%
\pgfpathlineto{\pgfqpoint{4.642854in}{0.487721in}}%
\pgfpathlineto{\pgfqpoint{4.643928in}{0.487304in}}%
\pgfpathlineto{\pgfqpoint{4.647150in}{0.490975in}}%
\pgfpathlineto{\pgfqpoint{4.648224in}{0.494981in}}%
\pgfpathlineto{\pgfqpoint{4.651445in}{0.501155in}}%
\pgfpathlineto{\pgfqpoint{4.654667in}{0.499403in}}%
\pgfpathlineto{\pgfqpoint{4.655740in}{0.496149in}}%
\pgfpathlineto{\pgfqpoint{4.657888in}{0.504576in}}%
\pgfpathlineto{\pgfqpoint{4.658962in}{0.504159in}}%
\pgfpathlineto{\pgfqpoint{4.662183in}{0.501406in}}%
\pgfpathlineto{\pgfqpoint{4.663257in}{0.501406in}}%
\pgfpathlineto{\pgfqpoint{4.665405in}{0.505327in}}%
\pgfpathlineto{\pgfqpoint{4.666479in}{0.506663in}}%
\pgfpathlineto{\pgfqpoint{4.669700in}{0.507163in}}%
\pgfpathlineto{\pgfqpoint{4.671848in}{0.509750in}}%
\pgfpathlineto{\pgfqpoint{4.673996in}{0.502324in}}%
\pgfpathlineto{\pgfqpoint{4.677217in}{0.506996in}}%
\pgfpathlineto{\pgfqpoint{4.678291in}{0.514172in}}%
\pgfpathlineto{\pgfqpoint{4.679365in}{0.512003in}}%
\pgfpathlineto{\pgfqpoint{4.680439in}{0.519596in}}%
\pgfpathlineto{\pgfqpoint{4.681513in}{0.512503in}}%
\pgfpathlineto{\pgfqpoint{4.686882in}{0.511335in}}%
\pgfpathlineto{\pgfqpoint{4.689029in}{0.505995in}}%
\pgfpathlineto{\pgfqpoint{4.692251in}{0.510835in}}%
\pgfpathlineto{\pgfqpoint{4.694399in}{0.520848in}}%
\pgfpathlineto{\pgfqpoint{4.695472in}{0.521682in}}%
\pgfpathlineto{\pgfqpoint{4.696546in}{0.529192in}}%
\pgfpathlineto{\pgfqpoint{4.700842in}{0.518261in}}%
\pgfpathlineto{\pgfqpoint{4.702989in}{0.518511in}}%
\pgfpathlineto{\pgfqpoint{4.704063in}{0.516926in}}%
\pgfpathlineto{\pgfqpoint{4.707285in}{0.517009in}}%
\pgfpathlineto{\pgfqpoint{4.709432in}{0.524185in}}%
\pgfpathlineto{\pgfqpoint{4.710506in}{0.530861in}}%
\pgfpathlineto{\pgfqpoint{4.711580in}{0.530360in}}%
\pgfpathlineto{\pgfqpoint{4.715875in}{0.533281in}}%
\pgfpathlineto{\pgfqpoint{4.716949in}{0.542626in}}%
\pgfpathlineto{\pgfqpoint{4.718023in}{0.542626in}}%
\pgfpathlineto{\pgfqpoint{4.719097in}{0.545880in}}%
\pgfpathlineto{\pgfqpoint{4.722318in}{0.545713in}}%
\pgfpathlineto{\pgfqpoint{4.724466in}{0.541374in}}%
\pgfpathlineto{\pgfqpoint{4.725540in}{0.542125in}}%
\pgfpathlineto{\pgfqpoint{4.726614in}{0.547466in}}%
\pgfpathlineto{\pgfqpoint{4.729835in}{0.542209in}}%
\pgfpathlineto{\pgfqpoint{4.731983in}{0.546965in}}%
\pgfpathlineto{\pgfqpoint{4.733057in}{0.545713in}}%
\pgfpathlineto{\pgfqpoint{4.734131in}{0.548050in}}%
\pgfpathlineto{\pgfqpoint{4.737352in}{0.548717in}}%
\pgfpathlineto{\pgfqpoint{4.738426in}{0.550219in}}%
\pgfpathlineto{\pgfqpoint{4.739500in}{0.550303in}}%
\pgfpathlineto{\pgfqpoint{4.740574in}{0.549719in}}%
\pgfpathlineto{\pgfqpoint{4.741648in}{0.555309in}}%
\pgfpathlineto{\pgfqpoint{4.745943in}{0.556227in}}%
\pgfpathlineto{\pgfqpoint{4.747017in}{0.547549in}}%
\pgfpathlineto{\pgfqpoint{4.748091in}{0.544295in}}%
\pgfpathlineto{\pgfqpoint{4.749164in}{0.544879in}}%
\pgfpathlineto{\pgfqpoint{4.752386in}{0.539873in}}%
\pgfpathlineto{\pgfqpoint{4.753460in}{0.542209in}}%
\pgfpathlineto{\pgfqpoint{4.754534in}{0.546631in}}%
\pgfpathlineto{\pgfqpoint{4.755607in}{0.547466in}}%
\pgfpathlineto{\pgfqpoint{4.756681in}{0.553140in}}%
\pgfpathlineto{\pgfqpoint{4.759903in}{0.556728in}}%
\pgfpathlineto{\pgfqpoint{4.760977in}{0.561901in}}%
\pgfpathlineto{\pgfqpoint{4.763124in}{0.560733in}}%
\pgfpathlineto{\pgfqpoint{4.764198in}{0.568910in}}%
\pgfpathlineto{\pgfqpoint{4.767420in}{0.570913in}}%
\pgfpathlineto{\pgfqpoint{4.768493in}{0.566741in}}%
\pgfpathlineto{\pgfqpoint{4.769567in}{0.568493in}}%
\pgfpathlineto{\pgfqpoint{4.770641in}{0.571581in}}%
\pgfpathlineto{\pgfqpoint{4.771715in}{0.570329in}}%
\pgfpathlineto{\pgfqpoint{4.774937in}{0.564571in}}%
\pgfpathlineto{\pgfqpoint{4.776010in}{0.560650in}}%
\pgfpathlineto{\pgfqpoint{4.777084in}{0.565406in}}%
\pgfpathlineto{\pgfqpoint{4.778158in}{0.560650in}}%
\pgfpathlineto{\pgfqpoint{4.779232in}{0.564238in}}%
\pgfpathlineto{\pgfqpoint{4.782453in}{0.559899in}}%
\pgfpathlineto{\pgfqpoint{4.783527in}{0.563069in}}%
\pgfpathlineto{\pgfqpoint{4.784601in}{0.561818in}}%
\pgfpathlineto{\pgfqpoint{4.785675in}{0.564404in}}%
\pgfpathlineto{\pgfqpoint{4.789970in}{0.563570in}}%
\pgfpathlineto{\pgfqpoint{4.791044in}{0.569411in}}%
\pgfpathlineto{\pgfqpoint{4.792118in}{0.567826in}}%
\pgfpathlineto{\pgfqpoint{4.793192in}{0.570412in}}%
\pgfpathlineto{\pgfqpoint{4.794266in}{0.571247in}}%
\pgfpathlineto{\pgfqpoint{4.798561in}{0.582261in}}%
\pgfpathlineto{\pgfqpoint{4.799635in}{0.589604in}}%
\pgfpathlineto{\pgfqpoint{4.800709in}{0.593025in}}%
\pgfpathlineto{\pgfqpoint{4.801782in}{0.593025in}}%
\pgfpathlineto{\pgfqpoint{4.805004in}{0.580259in}}%
\pgfpathlineto{\pgfqpoint{4.806078in}{0.594527in}}%
\pgfpathlineto{\pgfqpoint{4.807152in}{0.593943in}}%
\pgfpathlineto{\pgfqpoint{4.808226in}{0.588686in}}%
\pgfpathlineto{\pgfqpoint{4.809299in}{0.600702in}}%
\pgfpathlineto{\pgfqpoint{4.812521in}{0.604123in}}%
\pgfpathlineto{\pgfqpoint{4.813595in}{0.608545in}}%
\pgfpathlineto{\pgfqpoint{4.814669in}{0.603622in}}%
\pgfpathlineto{\pgfqpoint{4.815742in}{0.604040in}}%
\pgfpathlineto{\pgfqpoint{4.816816in}{0.603038in}}%
\pgfpathlineto{\pgfqpoint{4.820038in}{0.611716in}}%
\pgfpathlineto{\pgfqpoint{4.821112in}{0.610465in}}%
\pgfpathlineto{\pgfqpoint{4.822185in}{0.613302in}}%
\pgfpathlineto{\pgfqpoint{4.824333in}{0.625401in}}%
\pgfpathlineto{\pgfqpoint{4.827555in}{0.627403in}}%
\pgfpathlineto{\pgfqpoint{4.828628in}{0.635080in}}%
\pgfpathlineto{\pgfqpoint{4.829702in}{0.634496in}}%
\pgfpathlineto{\pgfqpoint{4.831850in}{0.643758in}}%
\pgfpathlineto{\pgfqpoint{4.835071in}{0.644676in}}%
\pgfpathlineto{\pgfqpoint{4.837219in}{0.647346in}}%
\pgfpathlineto{\pgfqpoint{4.838293in}{0.638167in}}%
\pgfpathlineto{\pgfqpoint{4.839367in}{0.639002in}}%
\pgfpathlineto{\pgfqpoint{4.842588in}{0.635497in}}%
\pgfpathlineto{\pgfqpoint{4.845810in}{0.628738in}}%
\pgfpathlineto{\pgfqpoint{4.846884in}{0.630658in}}%
\pgfpathlineto{\pgfqpoint{4.851179in}{0.639836in}}%
\pgfpathlineto{\pgfqpoint{4.852253in}{0.636582in}}%
\pgfpathlineto{\pgfqpoint{4.854401in}{0.612300in}}%
\pgfpathlineto{\pgfqpoint{4.857622in}{0.617807in}}%
\pgfpathlineto{\pgfqpoint{4.858696in}{0.621980in}}%
\pgfpathlineto{\pgfqpoint{4.859770in}{0.612634in}}%
\pgfpathlineto{\pgfqpoint{4.860844in}{0.612717in}}%
\pgfpathlineto{\pgfqpoint{4.861917in}{0.625818in}}%
\pgfpathlineto{\pgfqpoint{4.865139in}{0.618058in}}%
\pgfpathlineto{\pgfqpoint{4.866213in}{0.617807in}}%
\pgfpathlineto{\pgfqpoint{4.867287in}{0.611633in}}%
\pgfpathlineto{\pgfqpoint{4.868360in}{0.621562in}}%
\pgfpathlineto{\pgfqpoint{4.869434in}{0.617807in}}%
\pgfpathlineto{\pgfqpoint{4.872656in}{0.623064in}}%
\pgfpathlineto{\pgfqpoint{4.873730in}{0.629156in}}%
\pgfpathlineto{\pgfqpoint{4.874804in}{0.621813in}}%
\pgfpathlineto{\pgfqpoint{4.875877in}{0.603873in}}%
\pgfpathlineto{\pgfqpoint{4.876951in}{0.609630in}}%
\pgfpathlineto{\pgfqpoint{4.880173in}{0.607377in}}%
\pgfpathlineto{\pgfqpoint{4.881247in}{0.608378in}}%
\pgfpathlineto{\pgfqpoint{4.883394in}{0.617140in}}%
\pgfpathlineto{\pgfqpoint{4.884468in}{0.612801in}}%
\pgfpathlineto{\pgfqpoint{4.887690in}{0.618809in}}%
\pgfpathlineto{\pgfqpoint{4.888763in}{0.613635in}}%
\pgfpathlineto{\pgfqpoint{4.889837in}{0.616306in}}%
\pgfpathlineto{\pgfqpoint{4.891985in}{0.617974in}}%
\pgfpathlineto{\pgfqpoint{4.896280in}{0.626485in}}%
\pgfpathlineto{\pgfqpoint{4.897354in}{0.626235in}}%
\pgfpathlineto{\pgfqpoint{4.898428in}{0.639002in}}%
\pgfpathlineto{\pgfqpoint{4.899502in}{0.642089in}}%
\pgfpathlineto{\pgfqpoint{4.902723in}{0.634079in}}%
\pgfpathlineto{\pgfqpoint{4.903797in}{0.627153in}}%
\pgfpathlineto{\pgfqpoint{4.905945in}{0.633161in}}%
\pgfpathlineto{\pgfqpoint{4.907019in}{0.628154in}}%
\pgfpathlineto{\pgfqpoint{4.910240in}{0.622397in}}%
\pgfpathlineto{\pgfqpoint{4.911314in}{0.622731in}}%
\pgfpathlineto{\pgfqpoint{4.912388in}{0.624233in}}%
\pgfpathlineto{\pgfqpoint{4.913462in}{0.623482in}}%
\pgfpathlineto{\pgfqpoint{4.914536in}{0.626819in}}%
\pgfpathlineto{\pgfqpoint{4.917757in}{0.623982in}}%
\pgfpathlineto{\pgfqpoint{4.918831in}{0.620728in}}%
\pgfpathlineto{\pgfqpoint{4.919905in}{0.624233in}}%
\pgfpathlineto{\pgfqpoint{4.920979in}{0.629656in}}%
\pgfpathlineto{\pgfqpoint{4.922052in}{0.638501in}}%
\pgfpathlineto{\pgfqpoint{4.925274in}{0.634746in}}%
\pgfpathlineto{\pgfqpoint{4.926348in}{0.642590in}}%
\pgfpathlineto{\pgfqpoint{4.927422in}{0.633912in}}%
\pgfpathlineto{\pgfqpoint{4.928495in}{0.632660in}}%
\pgfpathlineto{\pgfqpoint{4.929569in}{0.624900in}}%
\pgfpathlineto{\pgfqpoint{4.932791in}{0.618809in}}%
\pgfpathlineto{\pgfqpoint{4.934938in}{0.618976in}}%
\pgfpathlineto{\pgfqpoint{4.936012in}{0.606960in}}%
\pgfpathlineto{\pgfqpoint{4.937086in}{0.605375in}}%
\pgfpathlineto{\pgfqpoint{4.940308in}{0.602788in}}%
\pgfpathlineto{\pgfqpoint{4.941381in}{0.603122in}}%
\pgfpathlineto{\pgfqpoint{4.942455in}{0.597448in}}%
\pgfpathlineto{\pgfqpoint{4.943529in}{0.601286in}}%
\pgfpathlineto{\pgfqpoint{4.944603in}{0.601953in}}%
\pgfpathlineto{\pgfqpoint{4.947825in}{0.599116in}}%
\pgfpathlineto{\pgfqpoint{4.948898in}{0.594026in}}%
\pgfpathlineto{\pgfqpoint{4.949972in}{0.594944in}}%
\pgfpathlineto{\pgfqpoint{4.951046in}{0.597030in}}%
\pgfpathlineto{\pgfqpoint{4.952120in}{0.595111in}}%
\pgfpathlineto{\pgfqpoint{4.956415in}{0.595862in}}%
\pgfpathlineto{\pgfqpoint{4.959637in}{0.599367in}}%
\pgfpathlineto{\pgfqpoint{4.962858in}{0.600952in}}%
\pgfpathlineto{\pgfqpoint{4.967154in}{0.639836in}}%
\pgfpathlineto{\pgfqpoint{4.972523in}{0.643007in}}%
\pgfpathlineto{\pgfqpoint{4.974670in}{0.626986in}}%
\pgfpathlineto{\pgfqpoint{4.977892in}{0.625067in}}%
\pgfpathlineto{\pgfqpoint{4.978966in}{0.621729in}}%
\pgfpathlineto{\pgfqpoint{4.980040in}{0.622731in}}%
\pgfpathlineto{\pgfqpoint{4.981114in}{0.628738in}}%
\pgfpathlineto{\pgfqpoint{4.982187in}{0.628405in}}%
\pgfpathlineto{\pgfqpoint{4.985409in}{0.623064in}}%
\pgfpathlineto{\pgfqpoint{4.986483in}{0.625651in}}%
\pgfpathlineto{\pgfqpoint{4.987557in}{0.626068in}}%
\pgfpathlineto{\pgfqpoint{4.988630in}{0.619476in}}%
\pgfpathlineto{\pgfqpoint{4.989704in}{0.629239in}}%
\pgfpathlineto{\pgfqpoint{4.992926in}{0.623815in}}%
\pgfpathlineto{\pgfqpoint{4.995073in}{0.616222in}}%
\pgfpathlineto{\pgfqpoint{4.996147in}{0.631409in}}%
\pgfpathlineto{\pgfqpoint{4.997221in}{0.636165in}}%
\pgfpathlineto{\pgfqpoint{5.000443in}{0.640921in}}%
\pgfpathlineto{\pgfqpoint{5.001516in}{0.637917in}}%
\pgfpathlineto{\pgfqpoint{5.002590in}{0.637249in}}%
\pgfpathlineto{\pgfqpoint{5.003664in}{0.637750in}}%
\pgfpathlineto{\pgfqpoint{5.004738in}{0.643341in}}%
\pgfpathlineto{\pgfqpoint{5.007959in}{0.646845in}}%
\pgfpathlineto{\pgfqpoint{5.009033in}{0.657442in}}%
\pgfpathlineto{\pgfqpoint{5.010107in}{0.650767in}}%
\pgfpathlineto{\pgfqpoint{5.011181in}{0.657860in}}%
\pgfpathlineto{\pgfqpoint{5.012255in}{0.659445in}}%
\pgfpathlineto{\pgfqpoint{5.016550in}{0.656858in}}%
\pgfpathlineto{\pgfqpoint{5.017624in}{0.653437in}}%
\pgfpathlineto{\pgfqpoint{5.018698in}{0.654355in}}%
\pgfpathlineto{\pgfqpoint{5.019772in}{0.657526in}}%
\pgfpathlineto{\pgfqpoint{5.022993in}{0.656024in}}%
\pgfpathlineto{\pgfqpoint{5.024067in}{0.656358in}}%
\pgfpathlineto{\pgfqpoint{5.025141in}{0.657442in}}%
\pgfpathlineto{\pgfqpoint{5.026215in}{0.643341in}}%
\pgfpathlineto{\pgfqpoint{5.027289in}{0.654272in}}%
\pgfpathlineto{\pgfqpoint{5.030510in}{0.652436in}}%
\pgfpathlineto{\pgfqpoint{5.031584in}{0.648097in}}%
\pgfpathlineto{\pgfqpoint{5.033732in}{0.665286in}}%
\pgfpathlineto{\pgfqpoint{5.034805in}{0.665119in}}%
\pgfpathlineto{\pgfqpoint{5.038027in}{0.661281in}}%
\pgfpathlineto{\pgfqpoint{5.039101in}{0.658360in}}%
\pgfpathlineto{\pgfqpoint{5.040175in}{0.659195in}}%
\pgfpathlineto{\pgfqpoint{5.041248in}{0.664618in}}%
\pgfpathlineto{\pgfqpoint{5.042322in}{0.666621in}}%
\pgfpathlineto{\pgfqpoint{5.045544in}{0.663117in}}%
\pgfpathlineto{\pgfqpoint{5.046618in}{0.674131in}}%
\pgfpathlineto{\pgfqpoint{5.047692in}{0.670960in}}%
\pgfpathlineto{\pgfqpoint{5.049839in}{0.669208in}}%
\pgfpathlineto{\pgfqpoint{5.053061in}{0.672045in}}%
\pgfpathlineto{\pgfqpoint{5.054135in}{0.664368in}}%
\pgfpathlineto{\pgfqpoint{5.055208in}{0.664869in}}%
\pgfpathlineto{\pgfqpoint{5.056282in}{0.666871in}}%
\pgfpathlineto{\pgfqpoint{5.057356in}{0.676217in}}%
\pgfpathlineto{\pgfqpoint{5.060578in}{0.673547in}}%
\pgfpathlineto{\pgfqpoint{5.061651in}{0.677051in}}%
\pgfpathlineto{\pgfqpoint{5.062725in}{0.669041in}}%
\pgfpathlineto{\pgfqpoint{5.064873in}{0.668791in}}%
\pgfpathlineto{\pgfqpoint{5.069168in}{0.676801in}}%
\pgfpathlineto{\pgfqpoint{5.071316in}{0.694658in}}%
\pgfpathlineto{\pgfqpoint{5.072390in}{0.690235in}}%
\pgfpathlineto{\pgfqpoint{5.075611in}{0.696994in}}%
\pgfpathlineto{\pgfqpoint{5.076685in}{0.701416in}}%
\pgfpathlineto{\pgfqpoint{5.078833in}{0.707424in}}%
\pgfpathlineto{\pgfqpoint{5.079907in}{0.705338in}}%
\pgfpathlineto{\pgfqpoint{5.083128in}{0.719857in}}%
\pgfpathlineto{\pgfqpoint{5.084202in}{0.721109in}}%
\pgfpathlineto{\pgfqpoint{5.086350in}{0.720108in}}%
\pgfpathlineto{\pgfqpoint{5.087424in}{0.718856in}}%
\pgfpathlineto{\pgfqpoint{5.090645in}{0.716686in}}%
\pgfpathlineto{\pgfqpoint{5.091719in}{0.720692in}}%
\pgfpathlineto{\pgfqpoint{5.092793in}{0.712014in}}%
\pgfpathlineto{\pgfqpoint{5.093867in}{0.709510in}}%
\pgfpathlineto{\pgfqpoint{5.094940in}{0.713349in}}%
\pgfpathlineto{\pgfqpoint{5.098162in}{0.696994in}}%
\pgfpathlineto{\pgfqpoint{5.099236in}{0.706089in}}%
\pgfpathlineto{\pgfqpoint{5.102457in}{0.702418in}}%
\pgfpathlineto{\pgfqpoint{5.106753in}{0.704170in}}%
\pgfpathlineto{\pgfqpoint{5.107826in}{0.712681in}}%
\pgfpathlineto{\pgfqpoint{5.108900in}{0.708676in}}%
\pgfpathlineto{\pgfqpoint{5.109974in}{0.692738in}}%
\pgfpathlineto{\pgfqpoint{5.113196in}{0.689067in}}%
\pgfpathlineto{\pgfqpoint{5.114269in}{0.693907in}}%
\pgfpathlineto{\pgfqpoint{5.115343in}{0.681974in}}%
\pgfpathlineto{\pgfqpoint{5.116417in}{0.696577in}}%
\pgfpathlineto{\pgfqpoint{5.117491in}{0.691821in}}%
\pgfpathlineto{\pgfqpoint{5.120713in}{0.671628in}}%
\pgfpathlineto{\pgfqpoint{5.122860in}{0.685312in}}%
\pgfpathlineto{\pgfqpoint{5.123934in}{0.714684in}}%
\pgfpathlineto{\pgfqpoint{5.125008in}{0.715518in}}%
\pgfpathlineto{\pgfqpoint{5.128229in}{0.726199in}}%
\pgfpathlineto{\pgfqpoint{5.129303in}{0.731873in}}%
\pgfpathlineto{\pgfqpoint{5.130377in}{0.732791in}}%
\pgfpathlineto{\pgfqpoint{5.131451in}{0.732707in}}%
\pgfpathlineto{\pgfqpoint{5.132525in}{0.742971in}}%
\pgfpathlineto{\pgfqpoint{5.136820in}{0.745724in}}%
\pgfpathlineto{\pgfqpoint{5.137894in}{0.740217in}}%
\pgfpathlineto{\pgfqpoint{5.138968in}{0.742720in}}%
\pgfpathlineto{\pgfqpoint{5.140042in}{0.749980in}}%
\pgfpathlineto{\pgfqpoint{5.143263in}{0.754569in}}%
\pgfpathlineto{\pgfqpoint{5.144337in}{0.750564in}}%
\pgfpathlineto{\pgfqpoint{5.145411in}{0.749563in}}%
\pgfpathlineto{\pgfqpoint{5.147558in}{0.755237in}}%
\pgfpathlineto{\pgfqpoint{5.150780in}{0.744806in}}%
\pgfpathlineto{\pgfqpoint{5.151854in}{0.762162in}}%
\pgfpathlineto{\pgfqpoint{5.154002in}{0.774762in}}%
\pgfpathlineto{\pgfqpoint{5.155075in}{0.766001in}}%
\pgfpathlineto{\pgfqpoint{5.158297in}{0.764165in}}%
\pgfpathlineto{\pgfqpoint{5.159371in}{0.757072in}}%
\pgfpathlineto{\pgfqpoint{5.160445in}{0.759576in}}%
\pgfpathlineto{\pgfqpoint{5.161518in}{0.748394in}}%
\pgfpathlineto{\pgfqpoint{5.162592in}{0.749479in}}%
\pgfpathlineto{\pgfqpoint{5.166888in}{0.764332in}}%
\pgfpathlineto{\pgfqpoint{5.167961in}{0.752984in}}%
\pgfpathlineto{\pgfqpoint{5.169035in}{0.755237in}}%
\pgfpathlineto{\pgfqpoint{5.170109in}{0.751649in}}%
\pgfpathlineto{\pgfqpoint{5.173331in}{0.744973in}}%
\pgfpathlineto{\pgfqpoint{5.174404in}{0.745474in}}%
\pgfpathlineto{\pgfqpoint{5.175478in}{0.738298in}}%
\pgfpathlineto{\pgfqpoint{5.176552in}{0.737213in}}%
\pgfpathlineto{\pgfqpoint{5.177626in}{0.741135in}}%
\pgfpathlineto{\pgfqpoint{5.180847in}{0.749479in}}%
\pgfpathlineto{\pgfqpoint{5.181921in}{0.761078in}}%
\pgfpathlineto{\pgfqpoint{5.184069in}{0.761995in}}%
\pgfpathlineto{\pgfqpoint{5.185143in}{0.752233in}}%
\pgfpathlineto{\pgfqpoint{5.188364in}{0.742220in}}%
\pgfpathlineto{\pgfqpoint{5.189438in}{0.745641in}}%
\pgfpathlineto{\pgfqpoint{5.190512in}{0.752566in}}%
\pgfpathlineto{\pgfqpoint{5.191586in}{0.729703in}}%
\pgfpathlineto{\pgfqpoint{5.192660in}{0.725865in}}%
\pgfpathlineto{\pgfqpoint{5.195881in}{0.730538in}}%
\pgfpathlineto{\pgfqpoint{5.196955in}{0.730872in}}%
\pgfpathlineto{\pgfqpoint{5.199103in}{0.748895in}}%
\pgfpathlineto{\pgfqpoint{5.203398in}{0.742053in}}%
\pgfpathlineto{\pgfqpoint{5.204472in}{0.744723in}}%
\pgfpathlineto{\pgfqpoint{5.206620in}{0.745808in}}%
\pgfpathlineto{\pgfqpoint{5.207693in}{0.735294in}}%
\pgfpathlineto{\pgfqpoint{5.210915in}{0.731789in}}%
\pgfpathlineto{\pgfqpoint{5.213063in}{0.743889in}}%
\pgfpathlineto{\pgfqpoint{5.214136in}{0.745557in}}%
\pgfpathlineto{\pgfqpoint{5.215210in}{0.751315in}}%
\pgfpathlineto{\pgfqpoint{5.218432in}{0.758407in}}%
\pgfpathlineto{\pgfqpoint{5.219506in}{0.756905in}}%
\pgfpathlineto{\pgfqpoint{5.220580in}{0.751148in}}%
\pgfpathlineto{\pgfqpoint{5.221653in}{0.761328in}}%
\pgfpathlineto{\pgfqpoint{5.222727in}{0.763998in}}%
\pgfpathlineto{\pgfqpoint{5.225949in}{0.767670in}}%
\pgfpathlineto{\pgfqpoint{5.227023in}{0.764999in}}%
\pgfpathlineto{\pgfqpoint{5.229170in}{0.750147in}}%
\pgfpathlineto{\pgfqpoint{5.233466in}{0.757072in}}%
\pgfpathlineto{\pgfqpoint{5.234539in}{0.757406in}}%
\pgfpathlineto{\pgfqpoint{5.235613in}{0.765834in}}%
\pgfpathlineto{\pgfqpoint{5.236687in}{0.767085in}}%
\pgfpathlineto{\pgfqpoint{5.237761in}{0.774595in}}%
\pgfpathlineto{\pgfqpoint{5.242056in}{0.777849in}}%
\pgfpathlineto{\pgfqpoint{5.243130in}{0.776765in}}%
\pgfpathlineto{\pgfqpoint{5.244204in}{0.780102in}}%
\pgfpathlineto{\pgfqpoint{5.245278in}{0.779936in}}%
\pgfpathlineto{\pgfqpoint{5.248499in}{0.781938in}}%
\pgfpathlineto{\pgfqpoint{5.249573in}{0.778934in}}%
\pgfpathlineto{\pgfqpoint{5.251721in}{0.785860in}}%
\pgfpathlineto{\pgfqpoint{5.252795in}{0.784525in}}%
\pgfpathlineto{\pgfqpoint{5.256016in}{0.791284in}}%
\pgfpathlineto{\pgfqpoint{5.258164in}{0.782272in}}%
\pgfpathlineto{\pgfqpoint{5.259238in}{0.770590in}}%
\pgfpathlineto{\pgfqpoint{5.260312in}{0.770590in}}%
\pgfpathlineto{\pgfqpoint{5.263533in}{0.774428in}}%
\pgfpathlineto{\pgfqpoint{5.264607in}{0.773344in}}%
\pgfpathlineto{\pgfqpoint{5.266755in}{0.778100in}}%
\pgfpathlineto{\pgfqpoint{5.267828in}{0.770757in}}%
\pgfpathlineto{\pgfqpoint{5.272124in}{0.769672in}}%
\pgfpathlineto{\pgfqpoint{5.273198in}{0.779101in}}%
\pgfpathlineto{\pgfqpoint{5.275345in}{0.789865in}}%
\pgfpathlineto{\pgfqpoint{5.278567in}{0.793286in}}%
\pgfpathlineto{\pgfqpoint{5.279641in}{0.798960in}}%
\pgfpathlineto{\pgfqpoint{5.280714in}{0.798710in}}%
\pgfpathlineto{\pgfqpoint{5.281788in}{0.801797in}}%
\pgfpathlineto{\pgfqpoint{5.286084in}{0.799878in}}%
\pgfpathlineto{\pgfqpoint{5.287158in}{0.794204in}}%
\pgfpathlineto{\pgfqpoint{5.288231in}{0.804718in}}%
\pgfpathlineto{\pgfqpoint{5.289305in}{0.801964in}}%
\pgfpathlineto{\pgfqpoint{5.290379in}{0.802215in}}%
\pgfpathlineto{\pgfqpoint{5.293601in}{0.801213in}}%
\pgfpathlineto{\pgfqpoint{5.296822in}{0.787779in}}%
\pgfpathlineto{\pgfqpoint{5.297896in}{0.793870in}}%
\pgfpathlineto{\pgfqpoint{5.301117in}{0.793286in}}%
\pgfpathlineto{\pgfqpoint{5.302191in}{0.796958in}}%
\pgfpathlineto{\pgfqpoint{5.303265in}{0.795623in}}%
\pgfpathlineto{\pgfqpoint{5.304339in}{0.801464in}}%
\pgfpathlineto{\pgfqpoint{5.305413in}{0.797041in}}%
\pgfpathlineto{\pgfqpoint{5.308634in}{0.804217in}}%
\pgfpathlineto{\pgfqpoint{5.309708in}{0.796874in}}%
\pgfpathlineto{\pgfqpoint{5.310782in}{0.804634in}}%
\pgfpathlineto{\pgfqpoint{5.311856in}{0.794371in}}%
\pgfpathlineto{\pgfqpoint{5.312930in}{0.790533in}}%
\pgfpathlineto{\pgfqpoint{5.316151in}{0.804885in}}%
\pgfpathlineto{\pgfqpoint{5.317225in}{0.801130in}}%
\pgfpathlineto{\pgfqpoint{5.318299in}{0.799878in}}%
\pgfpathlineto{\pgfqpoint{5.319373in}{0.791534in}}%
\pgfpathlineto{\pgfqpoint{5.320446in}{0.801881in}}%
\pgfpathlineto{\pgfqpoint{5.323668in}{0.806887in}}%
\pgfpathlineto{\pgfqpoint{5.324742in}{0.804634in}}%
\pgfpathlineto{\pgfqpoint{5.325816in}{0.807638in}}%
\pgfpathlineto{\pgfqpoint{5.327963in}{0.820655in}}%
\pgfpathlineto{\pgfqpoint{5.332259in}{0.826913in}}%
\pgfpathlineto{\pgfqpoint{5.333333in}{0.824994in}}%
\pgfpathlineto{\pgfqpoint{5.334406in}{0.829166in}}%
\pgfpathlineto{\pgfqpoint{5.335480in}{0.830001in}}%
\pgfpathlineto{\pgfqpoint{5.338702in}{0.829166in}}%
\pgfpathlineto{\pgfqpoint{5.339776in}{0.826329in}}%
\pgfpathlineto{\pgfqpoint{5.340849in}{0.829083in}}%
\pgfpathlineto{\pgfqpoint{5.341923in}{0.827998in}}%
\pgfpathlineto{\pgfqpoint{5.342997in}{0.825245in}}%
\pgfpathlineto{\pgfqpoint{5.348366in}{0.833422in}}%
\pgfpathlineto{\pgfqpoint{5.349440in}{0.827247in}}%
\pgfpathlineto{\pgfqpoint{5.350514in}{0.833422in}}%
\pgfpathlineto{\pgfqpoint{5.353735in}{0.830501in}}%
\pgfpathlineto{\pgfqpoint{5.354809in}{0.823075in}}%
\pgfpathlineto{\pgfqpoint{5.355883in}{0.822491in}}%
\pgfpathlineto{\pgfqpoint{5.356957in}{0.825996in}}%
\pgfpathlineto{\pgfqpoint{5.358031in}{0.823659in}}%
\pgfpathlineto{\pgfqpoint{5.363400in}{0.828749in}}%
\pgfpathlineto{\pgfqpoint{5.364474in}{0.828833in}}%
\pgfpathlineto{\pgfqpoint{5.365548in}{0.830001in}}%
\pgfpathlineto{\pgfqpoint{5.368769in}{0.820739in}}%
\pgfpathlineto{\pgfqpoint{5.369843in}{0.813145in}}%
\pgfpathlineto{\pgfqpoint{5.370917in}{0.821990in}}%
\pgfpathlineto{\pgfqpoint{5.371991in}{0.811310in}}%
\pgfpathlineto{\pgfqpoint{5.373065in}{0.816483in}}%
\pgfpathlineto{\pgfqpoint{5.376286in}{0.817151in}}%
\pgfpathlineto{\pgfqpoint{5.377360in}{0.818736in}}%
\pgfpathlineto{\pgfqpoint{5.378434in}{0.806804in}}%
\pgfpathlineto{\pgfqpoint{5.379508in}{0.801380in}}%
\pgfpathlineto{\pgfqpoint{5.380581in}{0.814230in}}%
\pgfpathlineto{\pgfqpoint{5.383803in}{0.815065in}}%
\pgfpathlineto{\pgfqpoint{5.384877in}{0.803883in}}%
\pgfpathlineto{\pgfqpoint{5.385951in}{0.811560in}}%
\pgfpathlineto{\pgfqpoint{5.387024in}{0.793036in}}%
\pgfpathlineto{\pgfqpoint{5.388098in}{0.797375in}}%
\pgfpathlineto{\pgfqpoint{5.391320in}{0.779185in}}%
\pgfpathlineto{\pgfqpoint{5.392394in}{0.780937in}}%
\pgfpathlineto{\pgfqpoint{5.393468in}{0.764999in}}%
\pgfpathlineto{\pgfqpoint{5.394541in}{0.762413in}}%
\pgfpathlineto{\pgfqpoint{5.395615in}{0.778517in}}%
\pgfpathlineto{\pgfqpoint{5.398837in}{0.791617in}}%
\pgfpathlineto{\pgfqpoint{5.399911in}{0.807221in}}%
\pgfpathlineto{\pgfqpoint{5.400984in}{0.803800in}}%
\pgfpathlineto{\pgfqpoint{5.403132in}{0.815482in}}%
\pgfpathlineto{\pgfqpoint{5.406354in}{0.814230in}}%
\pgfpathlineto{\pgfqpoint{5.407427in}{0.825662in}}%
\pgfpathlineto{\pgfqpoint{5.408501in}{0.822574in}}%
\pgfpathlineto{\pgfqpoint{5.409575in}{0.827915in}}%
\pgfpathlineto{\pgfqpoint{5.410649in}{0.836843in}}%
\pgfpathlineto{\pgfqpoint{5.413870in}{0.839430in}}%
\pgfpathlineto{\pgfqpoint{5.414944in}{0.828916in}}%
\pgfpathlineto{\pgfqpoint{5.416018in}{0.833923in}}%
\pgfpathlineto{\pgfqpoint{5.417092in}{0.841683in}}%
\pgfpathlineto{\pgfqpoint{5.418166in}{0.826162in}}%
\pgfpathlineto{\pgfqpoint{5.421387in}{0.824494in}}%
\pgfpathlineto{\pgfqpoint{5.422461in}{0.826079in}}%
\pgfpathlineto{\pgfqpoint{5.423535in}{0.825411in}}%
\pgfpathlineto{\pgfqpoint{5.425683in}{0.832421in}}%
\pgfpathlineto{\pgfqpoint{5.429978in}{0.828332in}}%
\pgfpathlineto{\pgfqpoint{5.431052in}{0.824827in}}%
\pgfpathlineto{\pgfqpoint{5.432126in}{0.817735in}}%
\pgfpathlineto{\pgfqpoint{5.433200in}{0.818152in}}%
\pgfpathlineto{\pgfqpoint{5.436421in}{0.830668in}}%
\pgfpathlineto{\pgfqpoint{5.437495in}{0.838929in}}%
\pgfpathlineto{\pgfqpoint{5.440716in}{0.845604in}}%
\pgfpathlineto{\pgfqpoint{5.443938in}{0.847023in}}%
\pgfpathlineto{\pgfqpoint{5.445012in}{0.853031in}}%
\pgfpathlineto{\pgfqpoint{5.446086in}{0.850194in}}%
\pgfpathlineto{\pgfqpoint{5.447159in}{0.851195in}}%
\pgfpathlineto{\pgfqpoint{5.448233in}{0.855284in}}%
\pgfpathlineto{\pgfqpoint{5.451455in}{0.855534in}}%
\pgfpathlineto{\pgfqpoint{5.452529in}{0.848942in}}%
\pgfpathlineto{\pgfqpoint{5.453602in}{0.838762in}}%
\pgfpathlineto{\pgfqpoint{5.454676in}{0.848692in}}%
\pgfpathlineto{\pgfqpoint{5.455750in}{0.846689in}}%
\pgfpathlineto{\pgfqpoint{5.458972in}{0.842100in}}%
\pgfpathlineto{\pgfqpoint{5.460046in}{0.836342in}}%
\pgfpathlineto{\pgfqpoint{5.462193in}{0.855451in}}%
\pgfpathlineto{\pgfqpoint{5.463267in}{0.857620in}}%
\pgfpathlineto{\pgfqpoint{5.467562in}{0.871722in}}%
\pgfpathlineto{\pgfqpoint{5.468636in}{0.869636in}}%
\pgfpathlineto{\pgfqpoint{5.470784in}{0.874392in}}%
\pgfpathlineto{\pgfqpoint{5.474005in}{0.878063in}}%
\pgfpathlineto{\pgfqpoint{5.476153in}{0.867800in}}%
\pgfpathlineto{\pgfqpoint{5.478301in}{0.864379in}}%
\pgfpathlineto{\pgfqpoint{5.482596in}{0.849860in}}%
\pgfpathlineto{\pgfqpoint{5.485818in}{0.868301in}}%
\pgfpathlineto{\pgfqpoint{5.489039in}{0.869970in}}%
\pgfpathlineto{\pgfqpoint{5.490113in}{0.875644in}}%
\pgfpathlineto{\pgfqpoint{5.491187in}{0.868134in}}%
\pgfpathlineto{\pgfqpoint{5.492261in}{0.869052in}}%
\pgfpathlineto{\pgfqpoint{5.493334in}{0.875560in}}%
\pgfpathlineto{\pgfqpoint{5.497630in}{0.872139in}}%
\pgfpathlineto{\pgfqpoint{5.498704in}{0.867633in}}%
\pgfpathlineto{\pgfqpoint{5.499778in}{0.875310in}}%
\pgfpathlineto{\pgfqpoint{5.500851in}{0.871972in}}%
\pgfpathlineto{\pgfqpoint{5.504073in}{0.873891in}}%
\pgfpathlineto{\pgfqpoint{5.506221in}{0.855951in}}%
\pgfpathlineto{\pgfqpoint{5.507294in}{0.860207in}}%
\pgfpathlineto{\pgfqpoint{5.508368in}{0.842517in}}%
\pgfpathlineto{\pgfqpoint{5.511590in}{0.850110in}}%
\pgfpathlineto{\pgfqpoint{5.512664in}{0.867133in}}%
\pgfpathlineto{\pgfqpoint{5.513737in}{0.923289in}}%
\pgfpathlineto{\pgfqpoint{5.514811in}{0.933886in}}%
\pgfpathlineto{\pgfqpoint{5.515885in}{0.929047in}}%
\pgfpathlineto{\pgfqpoint{5.519107in}{0.926794in}}%
\pgfpathlineto{\pgfqpoint{5.520180in}{0.928296in}}%
\pgfpathlineto{\pgfqpoint{5.521254in}{0.927878in}}%
\pgfpathlineto{\pgfqpoint{5.522328in}{0.941313in}}%
\pgfpathlineto{\pgfqpoint{5.523402in}{0.945902in}}%
\pgfpathlineto{\pgfqpoint{5.527697in}{0.945485in}}%
\pgfpathlineto{\pgfqpoint{5.528771in}{0.943482in}}%
\pgfpathlineto{\pgfqpoint{5.529845in}{0.943732in}}%
\pgfpathlineto{\pgfqpoint{5.530919in}{0.948906in}}%
\pgfpathlineto{\pgfqpoint{5.534140in}{0.952327in}}%
\pgfpathlineto{\pgfqpoint{5.535214in}{0.949824in}}%
\pgfpathlineto{\pgfqpoint{5.536288in}{0.956833in}}%
\pgfpathlineto{\pgfqpoint{5.538436in}{0.945234in}}%
\pgfpathlineto{\pgfqpoint{5.542731in}{0.962924in}}%
\pgfpathlineto{\pgfqpoint{5.545953in}{0.943148in}}%
\pgfpathlineto{\pgfqpoint{5.549174in}{0.954413in}}%
\pgfpathlineto{\pgfqpoint{5.550248in}{0.937474in}}%
\pgfpathlineto{\pgfqpoint{5.551322in}{0.935889in}}%
\pgfpathlineto{\pgfqpoint{5.552396in}{0.969349in}}%
\pgfpathlineto{\pgfqpoint{5.553469in}{0.963675in}}%
\pgfpathlineto{\pgfqpoint{5.556691in}{0.970935in}}%
\pgfpathlineto{\pgfqpoint{5.557765in}{0.967764in}}%
\pgfpathlineto{\pgfqpoint{5.558839in}{0.975607in}}%
\pgfpathlineto{\pgfqpoint{5.559912in}{0.970935in}}%
\pgfpathlineto{\pgfqpoint{5.560986in}{0.979195in}}%
\pgfpathlineto{\pgfqpoint{5.564208in}{0.977610in}}%
\pgfpathlineto{\pgfqpoint{5.565282in}{0.968932in}}%
\pgfpathlineto{\pgfqpoint{5.566356in}{0.952410in}}%
\pgfpathlineto{\pgfqpoint{5.571725in}{0.961172in}}%
\pgfpathlineto{\pgfqpoint{5.572799in}{0.951493in}}%
\pgfpathlineto{\pgfqpoint{5.574946in}{0.960254in}}%
\pgfpathlineto{\pgfqpoint{5.579242in}{0.957333in}}%
\pgfpathlineto{\pgfqpoint{5.580315in}{0.955748in}}%
\pgfpathlineto{\pgfqpoint{5.581389in}{0.963008in}}%
\pgfpathlineto{\pgfqpoint{5.582463in}{0.966262in}}%
\pgfpathlineto{\pgfqpoint{5.583537in}{0.967680in}}%
\pgfpathlineto{\pgfqpoint{5.586758in}{0.964176in}}%
\pgfpathlineto{\pgfqpoint{5.587832in}{0.965344in}}%
\pgfpathlineto{\pgfqpoint{5.588906in}{0.967847in}}%
\pgfpathlineto{\pgfqpoint{5.589980in}{0.976609in}}%
\pgfpathlineto{\pgfqpoint{5.591054in}{0.965594in}}%
\pgfpathlineto{\pgfqpoint{5.594275in}{0.977610in}}%
\pgfpathlineto{\pgfqpoint{5.595349in}{0.973354in}}%
\pgfpathlineto{\pgfqpoint{5.596423in}{0.975357in}}%
\pgfpathlineto{\pgfqpoint{5.598571in}{0.987873in}}%
\pgfpathlineto{\pgfqpoint{5.601792in}{0.992796in}}%
\pgfpathlineto{\pgfqpoint{5.603940in}{0.990043in}}%
\pgfpathlineto{\pgfqpoint{5.605014in}{0.981532in}}%
\pgfpathlineto{\pgfqpoint{5.606088in}{0.995550in}}%
\pgfpathlineto{\pgfqpoint{5.609309in}{0.999555in}}%
\pgfpathlineto{\pgfqpoint{5.610383in}{0.997886in}}%
\pgfpathlineto{\pgfqpoint{5.611457in}{0.989292in}}%
\pgfpathlineto{\pgfqpoint{5.612531in}{0.985704in}}%
\pgfpathlineto{\pgfqpoint{5.613604in}{0.992379in}}%
\pgfpathlineto{\pgfqpoint{5.616826in}{0.980530in}}%
\pgfpathlineto{\pgfqpoint{5.617900in}{0.985537in}}%
\pgfpathlineto{\pgfqpoint{5.618974in}{0.985203in}}%
\pgfpathlineto{\pgfqpoint{5.621121in}{0.993881in}}%
\pgfpathlineto{\pgfqpoint{5.624343in}{0.994131in}}%
\pgfpathlineto{\pgfqpoint{5.625417in}{0.995884in}}%
\pgfpathlineto{\pgfqpoint{5.626490in}{0.993047in}}%
\pgfpathlineto{\pgfqpoint{5.627564in}{0.994549in}}%
\pgfpathlineto{\pgfqpoint{5.628638in}{0.993547in}}%
\pgfpathlineto{\pgfqpoint{5.632934in}{0.987122in}}%
\pgfpathlineto{\pgfqpoint{5.634007in}{0.994382in}}%
\pgfpathlineto{\pgfqpoint{5.635081in}{0.995633in}}%
\pgfpathlineto{\pgfqpoint{5.636155in}{0.994382in}}%
\pgfpathlineto{\pgfqpoint{5.639377in}{0.999055in}}%
\pgfpathlineto{\pgfqpoint{5.640450in}{0.997386in}}%
\pgfpathlineto{\pgfqpoint{5.641524in}{1.000640in}}%
\pgfpathlineto{\pgfqpoint{5.642598in}{0.993881in}}%
\pgfpathlineto{\pgfqpoint{5.643672in}{0.993881in}}%
\pgfpathlineto{\pgfqpoint{5.646893in}{0.985954in}}%
\pgfpathlineto{\pgfqpoint{5.647967in}{0.979946in}}%
\pgfpathlineto{\pgfqpoint{5.649041in}{0.991545in}}%
\pgfpathlineto{\pgfqpoint{5.650115in}{0.996384in}}%
\pgfpathlineto{\pgfqpoint{5.651189in}{0.991128in}}%
\pgfpathlineto{\pgfqpoint{5.654410in}{0.992880in}}%
\pgfpathlineto{\pgfqpoint{5.657632in}{1.016661in}}%
\pgfpathlineto{\pgfqpoint{5.658706in}{1.011988in}}%
\pgfpathlineto{\pgfqpoint{5.661927in}{1.019164in}}%
\pgfpathlineto{\pgfqpoint{5.663001in}{1.026006in}}%
\pgfpathlineto{\pgfqpoint{5.664075in}{1.021000in}}%
\pgfpathlineto{\pgfqpoint{5.666222in}{1.030512in}}%
\pgfpathlineto{\pgfqpoint{5.669444in}{1.015409in}}%
\pgfpathlineto{\pgfqpoint{5.671592in}{1.036854in}}%
\pgfpathlineto{\pgfqpoint{5.672666in}{1.035602in}}%
\pgfpathlineto{\pgfqpoint{5.676961in}{1.041360in}}%
\pgfpathlineto{\pgfqpoint{5.678035in}{1.052374in}}%
\pgfpathlineto{\pgfqpoint{5.679109in}{1.037354in}}%
\pgfpathlineto{\pgfqpoint{5.680182in}{1.040609in}}%
\pgfpathlineto{\pgfqpoint{5.681256in}{1.047201in}}%
\pgfpathlineto{\pgfqpoint{5.684478in}{1.059884in}}%
\pgfpathlineto{\pgfqpoint{5.685552in}{1.058298in}}%
\pgfpathlineto{\pgfqpoint{5.686625in}{1.061803in}}%
\pgfpathlineto{\pgfqpoint{5.687699in}{1.067894in}}%
\pgfpathlineto{\pgfqpoint{5.688773in}{1.066225in}}%
\pgfpathlineto{\pgfqpoint{5.691995in}{1.071900in}}%
\pgfpathlineto{\pgfqpoint{5.693068in}{1.069813in}}%
\pgfpathlineto{\pgfqpoint{5.694142in}{1.069897in}}%
\pgfpathlineto{\pgfqpoint{5.695216in}{1.065725in}}%
\pgfpathlineto{\pgfqpoint{5.696290in}{1.066643in}}%
\pgfpathlineto{\pgfqpoint{5.699511in}{1.061469in}}%
\pgfpathlineto{\pgfqpoint{5.700585in}{1.063055in}}%
\pgfpathlineto{\pgfqpoint{5.701659in}{1.073986in}}%
\pgfpathlineto{\pgfqpoint{5.702733in}{1.075404in}}%
\pgfpathlineto{\pgfqpoint{5.703807in}{1.075237in}}%
\pgfpathlineto{\pgfqpoint{5.708102in}{1.088504in}}%
\pgfpathlineto{\pgfqpoint{5.709176in}{1.000723in}}%
\pgfpathlineto{\pgfqpoint{5.710250in}{0.985120in}}%
\pgfpathlineto{\pgfqpoint{5.711324in}{0.991378in}}%
\pgfpathlineto{\pgfqpoint{5.714545in}{1.004395in}}%
\pgfpathlineto{\pgfqpoint{5.715619in}{0.980781in}}%
\pgfpathlineto{\pgfqpoint{5.716693in}{0.972854in}}%
\pgfpathlineto{\pgfqpoint{5.717767in}{0.977026in}}%
\pgfpathlineto{\pgfqpoint{5.718841in}{0.974189in}}%
\pgfpathlineto{\pgfqpoint{5.722062in}{0.989041in}}%
\pgfpathlineto{\pgfqpoint{5.723136in}{0.972436in}}%
\pgfpathlineto{\pgfqpoint{5.724210in}{0.968598in}}%
\pgfpathlineto{\pgfqpoint{5.725284in}{0.917949in}}%
\pgfpathlineto{\pgfqpoint{5.729579in}{0.881318in}}%
\pgfpathlineto{\pgfqpoint{5.730653in}{0.885490in}}%
\pgfpathlineto{\pgfqpoint{5.732800in}{0.934888in}}%
\pgfpathlineto{\pgfqpoint{5.733874in}{0.937307in}}%
\pgfpathlineto{\pgfqpoint{5.737096in}{0.932635in}}%
\pgfpathlineto{\pgfqpoint{5.738170in}{0.913944in}}%
\pgfpathlineto{\pgfqpoint{5.739244in}{0.932718in}}%
\pgfpathlineto{\pgfqpoint{5.740317in}{0.933469in}}%
\pgfpathlineto{\pgfqpoint{5.741391in}{0.925459in}}%
\pgfpathlineto{\pgfqpoint{5.745687in}{0.949406in}}%
\pgfpathlineto{\pgfqpoint{5.746760in}{0.932885in}}%
\pgfpathlineto{\pgfqpoint{5.747834in}{0.938309in}}%
\pgfpathlineto{\pgfqpoint{5.748908in}{0.953078in}}%
\pgfpathlineto{\pgfqpoint{5.752130in}{0.947905in}}%
\pgfpathlineto{\pgfqpoint{5.753203in}{0.944817in}}%
\pgfpathlineto{\pgfqpoint{5.754277in}{0.948989in}}%
\pgfpathlineto{\pgfqpoint{5.755351in}{0.950908in}}%
\pgfpathlineto{\pgfqpoint{5.756425in}{0.940144in}}%
\pgfpathlineto{\pgfqpoint{5.759646in}{0.944650in}}%
\pgfpathlineto{\pgfqpoint{5.762868in}{0.922705in}}%
\pgfpathlineto{\pgfqpoint{5.763942in}{0.920202in}}%
\pgfpathlineto{\pgfqpoint{5.767163in}{0.905933in}}%
\pgfpathlineto{\pgfqpoint{5.768237in}{0.913276in}}%
\pgfpathlineto{\pgfqpoint{5.769311in}{0.935138in}}%
\pgfpathlineto{\pgfqpoint{5.771459in}{0.941396in}}%
\pgfpathlineto{\pgfqpoint{5.774680in}{0.948155in}}%
\pgfpathlineto{\pgfqpoint{5.775754in}{0.947487in}}%
\pgfpathlineto{\pgfqpoint{5.776828in}{0.944483in}}%
\pgfpathlineto{\pgfqpoint{5.778976in}{0.961589in}}%
\pgfpathlineto{\pgfqpoint{5.783271in}{0.969683in}}%
\pgfpathlineto{\pgfqpoint{5.784345in}{0.962924in}}%
\pgfpathlineto{\pgfqpoint{5.785419in}{0.979946in}}%
\pgfpathlineto{\pgfqpoint{5.791862in}{0.997219in}}%
\pgfpathlineto{\pgfqpoint{5.792935in}{1.022085in}}%
\pgfpathlineto{\pgfqpoint{5.794009in}{1.020833in}}%
\pgfpathlineto{\pgfqpoint{5.798305in}{1.026173in}}%
\pgfpathlineto{\pgfqpoint{5.800452in}{1.036186in}}%
\pgfpathlineto{\pgfqpoint{5.801526in}{1.025923in}}%
\pgfpathlineto{\pgfqpoint{5.805822in}{1.040108in}}%
\pgfpathlineto{\pgfqpoint{5.806895in}{1.022085in}}%
\pgfpathlineto{\pgfqpoint{5.807969in}{1.020082in}}%
\pgfpathlineto{\pgfqpoint{5.809043in}{1.041109in}}%
\pgfpathlineto{\pgfqpoint{5.812265in}{1.047034in}}%
\pgfpathlineto{\pgfqpoint{5.813338in}{1.054877in}}%
\pgfpathlineto{\pgfqpoint{5.814412in}{1.047785in}}%
\pgfpathlineto{\pgfqpoint{5.815486in}{1.045365in}}%
\pgfpathlineto{\pgfqpoint{5.816560in}{1.034601in}}%
\pgfpathlineto{\pgfqpoint{5.820855in}{1.044781in}}%
\pgfpathlineto{\pgfqpoint{5.821929in}{1.060551in}}%
\pgfpathlineto{\pgfqpoint{5.823003in}{1.065057in}}%
\pgfpathlineto{\pgfqpoint{5.824077in}{1.075738in}}%
\pgfpathlineto{\pgfqpoint{5.827298in}{1.070648in}}%
\pgfpathlineto{\pgfqpoint{5.828372in}{1.059049in}}%
\pgfpathlineto{\pgfqpoint{5.829446in}{1.064723in}}%
\pgfpathlineto{\pgfqpoint{5.831594in}{1.036854in}}%
\pgfpathlineto{\pgfqpoint{5.834815in}{1.023837in}}%
\pgfpathlineto{\pgfqpoint{5.835889in}{1.038940in}}%
\pgfpathlineto{\pgfqpoint{5.838037in}{1.011404in}}%
\pgfpathlineto{\pgfqpoint{5.839111in}{1.029845in}}%
\pgfpathlineto{\pgfqpoint{5.842332in}{1.026674in}}%
\pgfpathlineto{\pgfqpoint{5.844480in}{1.008066in}}%
\pgfpathlineto{\pgfqpoint{5.845554in}{1.008066in}}%
\pgfpathlineto{\pgfqpoint{5.846627in}{0.986538in}}%
\pgfpathlineto{\pgfqpoint{5.849849in}{0.996885in}}%
\pgfpathlineto{\pgfqpoint{5.851997in}{1.032098in}}%
\pgfpathlineto{\pgfqpoint{5.853070in}{1.017996in}}%
\pgfpathlineto{\pgfqpoint{5.854144in}{0.984035in}}%
\pgfpathlineto{\pgfqpoint{5.857366in}{0.975023in}}%
\pgfpathlineto{\pgfqpoint{5.858440in}{0.976275in}}%
\pgfpathlineto{\pgfqpoint{5.859513in}{0.966929in}}%
\pgfpathlineto{\pgfqpoint{5.860587in}{0.969266in}}%
\pgfpathlineto{\pgfqpoint{5.864883in}{0.980280in}}%
\pgfpathlineto{\pgfqpoint{5.865956in}{0.978945in}}%
\pgfpathlineto{\pgfqpoint{5.867030in}{0.973104in}}%
\pgfpathlineto{\pgfqpoint{5.868104in}{0.963091in}}%
\pgfpathlineto{\pgfqpoint{5.872399in}{0.946486in}}%
\pgfpathlineto{\pgfqpoint{5.873473in}{0.929964in}}%
\pgfpathlineto{\pgfqpoint{5.874547in}{0.925709in}}%
\pgfpathlineto{\pgfqpoint{5.875621in}{0.918950in}}%
\pgfpathlineto{\pgfqpoint{5.876695in}{0.916947in}}%
\pgfpathlineto{\pgfqpoint{5.879916in}{0.922204in}}%
\pgfpathlineto{\pgfqpoint{5.880990in}{0.934470in}}%
\pgfpathlineto{\pgfqpoint{5.882064in}{0.910856in}}%
\pgfpathlineto{\pgfqpoint{5.883138in}{0.915779in}}%
\pgfpathlineto{\pgfqpoint{5.884212in}{0.874559in}}%
\pgfpathlineto{\pgfqpoint{5.888507in}{0.875143in}}%
\pgfpathlineto{\pgfqpoint{5.889581in}{0.863795in}}%
\pgfpathlineto{\pgfqpoint{5.890655in}{0.875477in}}%
\pgfpathlineto{\pgfqpoint{5.891729in}{0.898340in}}%
\pgfpathlineto{\pgfqpoint{5.894950in}{0.885573in}}%
\pgfpathlineto{\pgfqpoint{5.896024in}{0.893333in}}%
\pgfpathlineto{\pgfqpoint{5.897098in}{0.877897in}}%
\pgfpathlineto{\pgfqpoint{5.898172in}{0.871638in}}%
\pgfpathlineto{\pgfqpoint{5.899245in}{0.889745in}}%
\pgfpathlineto{\pgfqpoint{5.902467in}{0.884488in}}%
\pgfpathlineto{\pgfqpoint{5.903541in}{0.868384in}}%
\pgfpathlineto{\pgfqpoint{5.904615in}{0.884405in}}%
\pgfpathlineto{\pgfqpoint{5.905688in}{0.886658in}}%
\pgfpathlineto{\pgfqpoint{5.906762in}{0.874559in}}%
\pgfpathlineto{\pgfqpoint{5.909984in}{0.860457in}}%
\pgfpathlineto{\pgfqpoint{5.911058in}{0.862043in}}%
\pgfpathlineto{\pgfqpoint{5.912132in}{0.834590in}}%
\pgfpathlineto{\pgfqpoint{5.914279in}{0.852780in}}%
\pgfpathlineto{\pgfqpoint{5.918575in}{0.866715in}}%
\pgfpathlineto{\pgfqpoint{5.919648in}{0.887242in}}%
\pgfpathlineto{\pgfqpoint{5.921796in}{0.883320in}}%
\pgfpathlineto{\pgfqpoint{5.925018in}{0.894084in}}%
\pgfpathlineto{\pgfqpoint{5.926091in}{0.886241in}}%
\pgfpathlineto{\pgfqpoint{5.927165in}{0.886658in}}%
\pgfpathlineto{\pgfqpoint{5.928239in}{0.888410in}}%
\pgfpathlineto{\pgfqpoint{5.929313in}{0.885740in}}%
\pgfpathlineto{\pgfqpoint{5.932534in}{0.887409in}}%
\pgfpathlineto{\pgfqpoint{5.933608in}{0.904264in}}%
\pgfpathlineto{\pgfqpoint{5.934682in}{0.899091in}}%
\pgfpathlineto{\pgfqpoint{5.935756in}{0.913526in}}%
\pgfpathlineto{\pgfqpoint{5.936830in}{0.910856in}}%
\pgfpathlineto{\pgfqpoint{5.940051in}{0.918032in}}%
\pgfpathlineto{\pgfqpoint{5.941125in}{0.905599in}}%
\pgfpathlineto{\pgfqpoint{5.942199in}{0.904348in}}%
\pgfpathlineto{\pgfqpoint{5.943273in}{0.899425in}}%
\pgfpathlineto{\pgfqpoint{5.944347in}{0.906517in}}%
\pgfpathlineto{\pgfqpoint{5.947568in}{0.913443in}}%
\pgfpathlineto{\pgfqpoint{5.948642in}{0.908937in}}%
\pgfpathlineto{\pgfqpoint{5.949716in}{0.910522in}}%
\pgfpathlineto{\pgfqpoint{5.950790in}{0.919701in}}%
\pgfpathlineto{\pgfqpoint{5.951864in}{0.916530in}}%
\pgfpathlineto{\pgfqpoint{5.955085in}{0.910689in}}%
\pgfpathlineto{\pgfqpoint{5.957233in}{0.897756in}}%
\pgfpathlineto{\pgfqpoint{5.958307in}{0.900843in}}%
\pgfpathlineto{\pgfqpoint{5.962602in}{0.907769in}}%
\pgfpathlineto{\pgfqpoint{5.963676in}{0.908269in}}%
\pgfpathlineto{\pgfqpoint{5.965823in}{0.917365in}}%
\pgfpathlineto{\pgfqpoint{5.966897in}{0.915529in}}%
\pgfpathlineto{\pgfqpoint{5.970119in}{0.912442in}}%
\pgfpathlineto{\pgfqpoint{5.971193in}{0.899091in}}%
\pgfpathlineto{\pgfqpoint{5.972266in}{0.902929in}}%
\pgfpathlineto{\pgfqpoint{5.973340in}{0.892499in}}%
\pgfpathlineto{\pgfqpoint{5.974414in}{0.894502in}}%
\pgfpathlineto{\pgfqpoint{5.977636in}{0.893333in}}%
\pgfpathlineto{\pgfqpoint{5.978710in}{0.901844in}}%
\pgfpathlineto{\pgfqpoint{5.979783in}{0.918783in}}%
\pgfpathlineto{\pgfqpoint{5.980857in}{0.912024in}}%
\pgfpathlineto{\pgfqpoint{5.981931in}{0.911691in}}%
\pgfpathlineto{\pgfqpoint{5.987300in}{0.948739in}}%
\pgfpathlineto{\pgfqpoint{5.988374in}{0.945902in}}%
\pgfpathlineto{\pgfqpoint{5.989448in}{0.952744in}}%
\pgfpathlineto{\pgfqpoint{5.993743in}{0.961589in}}%
\pgfpathlineto{\pgfqpoint{5.994817in}{0.964676in}}%
\pgfpathlineto{\pgfqpoint{5.996965in}{0.948655in}}%
\pgfpathlineto{\pgfqpoint{6.000186in}{0.957417in}}%
\pgfpathlineto{\pgfqpoint{6.001260in}{0.952828in}}%
\pgfpathlineto{\pgfqpoint{6.002334in}{0.951910in}}%
\pgfpathlineto{\pgfqpoint{6.004482in}{0.966762in}}%
\pgfpathlineto{\pgfqpoint{6.007703in}{0.965177in}}%
\pgfpathlineto{\pgfqpoint{6.008777in}{0.975107in}}%
\pgfpathlineto{\pgfqpoint{6.009851in}{0.940979in}}%
\pgfpathlineto{\pgfqpoint{6.010925in}{0.936389in}}%
\pgfpathlineto{\pgfqpoint{6.011999in}{0.926961in}}%
\pgfpathlineto{\pgfqpoint{6.015220in}{0.925709in}}%
\pgfpathlineto{\pgfqpoint{6.016294in}{0.922371in}}%
\pgfpathlineto{\pgfqpoint{6.018442in}{0.910272in}}%
\pgfpathlineto{\pgfqpoint{6.019515in}{0.921120in}}%
\pgfpathlineto{\pgfqpoint{6.022737in}{0.916363in}}%
\pgfpathlineto{\pgfqpoint{6.024885in}{0.921787in}}%
\pgfpathlineto{\pgfqpoint{6.025958in}{0.921370in}}%
\pgfpathlineto{\pgfqpoint{6.027032in}{0.925125in}}%
\pgfpathlineto{\pgfqpoint{6.031328in}{0.916697in}}%
\pgfpathlineto{\pgfqpoint{6.032401in}{0.911107in}}%
\pgfpathlineto{\pgfqpoint{6.033475in}{0.912692in}}%
\pgfpathlineto{\pgfqpoint{6.037771in}{0.913193in}}%
\pgfpathlineto{\pgfqpoint{6.039918in}{0.907352in}}%
\pgfpathlineto{\pgfqpoint{6.040992in}{0.905766in}}%
\pgfpathlineto{\pgfqpoint{6.042066in}{0.901761in}}%
\pgfpathlineto{\pgfqpoint{6.045287in}{0.903597in}}%
\pgfpathlineto{\pgfqpoint{6.046361in}{0.910189in}}%
\pgfpathlineto{\pgfqpoint{6.047435in}{0.909187in}}%
\pgfpathlineto{\pgfqpoint{6.048509in}{0.910022in}}%
\pgfpathlineto{\pgfqpoint{6.049583in}{0.914945in}}%
\pgfpathlineto{\pgfqpoint{6.052804in}{0.919451in}}%
\pgfpathlineto{\pgfqpoint{6.053878in}{0.913526in}}%
\pgfpathlineto{\pgfqpoint{6.054952in}{0.913276in}}%
\pgfpathlineto{\pgfqpoint{6.056026in}{0.915112in}}%
\pgfpathlineto{\pgfqpoint{6.057100in}{0.888994in}}%
\pgfpathlineto{\pgfqpoint{6.060321in}{0.878397in}}%
\pgfpathlineto{\pgfqpoint{6.062469in}{0.898924in}}%
\pgfpathlineto{\pgfqpoint{6.063543in}{0.905599in}}%
\pgfpathlineto{\pgfqpoint{6.064617in}{0.907268in}}%
\pgfpathlineto{\pgfqpoint{6.068912in}{0.904348in}}%
\pgfpathlineto{\pgfqpoint{6.072133in}{0.925625in}}%
\pgfpathlineto{\pgfqpoint{6.075355in}{0.928462in}}%
\pgfpathlineto{\pgfqpoint{6.076429in}{0.930215in}}%
\pgfpathlineto{\pgfqpoint{6.077503in}{0.927628in}}%
\pgfpathlineto{\pgfqpoint{6.078576in}{0.928379in}}%
\pgfpathlineto{\pgfqpoint{6.079650in}{0.927044in}}%
\pgfpathlineto{\pgfqpoint{6.082872in}{0.929798in}}%
\pgfpathlineto{\pgfqpoint{6.083946in}{0.924374in}}%
\pgfpathlineto{\pgfqpoint{6.085020in}{0.914444in}}%
\pgfpathlineto{\pgfqpoint{6.087167in}{0.910356in}}%
\pgfpathlineto{\pgfqpoint{6.090389in}{0.907769in}}%
\pgfpathlineto{\pgfqpoint{6.092536in}{0.899341in}}%
\pgfpathlineto{\pgfqpoint{6.093610in}{0.896004in}}%
\pgfpathlineto{\pgfqpoint{6.094684in}{0.896337in}}%
\pgfpathlineto{\pgfqpoint{6.097906in}{0.893083in}}%
\pgfpathlineto{\pgfqpoint{6.098979in}{0.888827in}}%
\pgfpathlineto{\pgfqpoint{6.100053in}{0.897422in}}%
\pgfpathlineto{\pgfqpoint{6.101127in}{0.889996in}}%
\pgfpathlineto{\pgfqpoint{6.102201in}{0.895336in}}%
\pgfpathlineto{\pgfqpoint{6.105422in}{0.894752in}}%
\pgfpathlineto{\pgfqpoint{6.107570in}{0.911524in}}%
\pgfpathlineto{\pgfqpoint{6.108644in}{0.910856in}}%
\pgfpathlineto{\pgfqpoint{6.109718in}{0.903430in}}%
\pgfpathlineto{\pgfqpoint{6.112939in}{0.905516in}}%
\pgfpathlineto{\pgfqpoint{6.114013in}{0.903764in}}%
\pgfpathlineto{\pgfqpoint{6.115087in}{0.903680in}}%
\pgfpathlineto{\pgfqpoint{6.120456in}{0.895670in}}%
\pgfpathlineto{\pgfqpoint{6.121530in}{0.896504in}}%
\pgfpathlineto{\pgfqpoint{6.122604in}{0.895253in}}%
\pgfpathlineto{\pgfqpoint{6.124752in}{0.890413in}}%
\pgfpathlineto{\pgfqpoint{6.127973in}{0.887659in}}%
\pgfpathlineto{\pgfqpoint{6.129047in}{0.887659in}}%
\pgfpathlineto{\pgfqpoint{6.131195in}{0.882820in}}%
\pgfpathlineto{\pgfqpoint{6.132268in}{0.884071in}}%
\pgfpathlineto{\pgfqpoint{6.137638in}{0.878481in}}%
\pgfpathlineto{\pgfqpoint{6.138711in}{0.881485in}}%
\pgfpathlineto{\pgfqpoint{6.139785in}{0.868134in}}%
\pgfpathlineto{\pgfqpoint{6.143007in}{0.877897in}}%
\pgfpathlineto{\pgfqpoint{6.144081in}{0.870387in}}%
\pgfpathlineto{\pgfqpoint{6.145154in}{0.866882in}}%
\pgfpathlineto{\pgfqpoint{6.146228in}{0.868801in}}%
\pgfpathlineto{\pgfqpoint{6.147302in}{0.869302in}}%
\pgfpathlineto{\pgfqpoint{6.150524in}{0.869803in}}%
\pgfpathlineto{\pgfqpoint{6.151598in}{0.872389in}}%
\pgfpathlineto{\pgfqpoint{6.152671in}{0.867884in}}%
\pgfpathlineto{\pgfqpoint{6.153745in}{0.876061in}}%
\pgfpathlineto{\pgfqpoint{6.154819in}{0.874976in}}%
\pgfpathlineto{\pgfqpoint{6.159114in}{0.862543in}}%
\pgfpathlineto{\pgfqpoint{6.160188in}{0.866382in}}%
\pgfpathlineto{\pgfqpoint{6.161262in}{0.863211in}}%
\pgfpathlineto{\pgfqpoint{6.162336in}{0.871638in}}%
\pgfpathlineto{\pgfqpoint{6.165557in}{0.868718in}}%
\pgfpathlineto{\pgfqpoint{6.166631in}{0.869552in}}%
\pgfpathlineto{\pgfqpoint{6.167705in}{0.868384in}}%
\pgfpathlineto{\pgfqpoint{6.168779in}{0.871388in}}%
\pgfpathlineto{\pgfqpoint{6.169853in}{0.868718in}}%
\pgfpathlineto{\pgfqpoint{6.173074in}{0.868718in}}%
\pgfpathlineto{\pgfqpoint{6.175222in}{0.860123in}}%
\pgfpathlineto{\pgfqpoint{6.176296in}{0.857787in}}%
\pgfpathlineto{\pgfqpoint{6.177370in}{0.859206in}}%
\pgfpathlineto{\pgfqpoint{6.180591in}{0.855451in}}%
\pgfpathlineto{\pgfqpoint{6.181665in}{0.858204in}}%
\pgfpathlineto{\pgfqpoint{6.182739in}{0.864212in}}%
\pgfpathlineto{\pgfqpoint{6.183813in}{0.865046in}}%
\pgfpathlineto{\pgfqpoint{6.184887in}{0.873057in}}%
\pgfpathlineto{\pgfqpoint{6.188108in}{0.875727in}}%
\pgfpathlineto{\pgfqpoint{6.189182in}{0.870804in}}%
\pgfpathlineto{\pgfqpoint{6.191330in}{0.880900in}}%
\pgfpathlineto{\pgfqpoint{6.192403in}{0.879565in}}%
\pgfpathlineto{\pgfqpoint{6.197773in}{0.864129in}}%
\pgfpathlineto{\pgfqpoint{6.198846in}{0.875727in}}%
\pgfpathlineto{\pgfqpoint{6.199920in}{0.868384in}}%
\pgfpathlineto{\pgfqpoint{6.203142in}{0.884155in}}%
\pgfpathlineto{\pgfqpoint{6.204216in}{0.883821in}}%
\pgfpathlineto{\pgfqpoint{6.206363in}{0.888410in}}%
\pgfpathlineto{\pgfqpoint{6.207437in}{0.910105in}}%
\pgfpathlineto{\pgfqpoint{6.210659in}{0.912024in}}%
\pgfpathlineto{\pgfqpoint{6.211732in}{0.910272in}}%
\pgfpathlineto{\pgfqpoint{6.212806in}{0.921620in}}%
\pgfpathlineto{\pgfqpoint{6.213880in}{0.923623in}}%
\pgfpathlineto{\pgfqpoint{6.214954in}{0.914611in}}%
\pgfpathlineto{\pgfqpoint{6.218175in}{0.909688in}}%
\pgfpathlineto{\pgfqpoint{6.219249in}{0.910356in}}%
\pgfpathlineto{\pgfqpoint{6.220323in}{0.914778in}}%
\pgfpathlineto{\pgfqpoint{6.222471in}{0.919200in}}%
\pgfpathlineto{\pgfqpoint{6.225692in}{0.920369in}}%
\pgfpathlineto{\pgfqpoint{6.226766in}{0.925959in}}%
\pgfpathlineto{\pgfqpoint{6.227840in}{0.921620in}}%
\pgfpathlineto{\pgfqpoint{6.228914in}{0.920202in}}%
\pgfpathlineto{\pgfqpoint{6.229988in}{0.916697in}}%
\pgfpathlineto{\pgfqpoint{6.234283in}{0.933886in}}%
\pgfpathlineto{\pgfqpoint{6.235357in}{0.944483in}}%
\pgfpathlineto{\pgfqpoint{6.237505in}{0.973855in}}%
\pgfpathlineto{\pgfqpoint{6.241800in}{0.965678in}}%
\pgfpathlineto{\pgfqpoint{6.243948in}{0.970017in}}%
\pgfpathlineto{\pgfqpoint{6.245021in}{0.966178in}}%
\pgfpathlineto{\pgfqpoint{6.248243in}{0.977360in}}%
\pgfpathlineto{\pgfqpoint{6.250391in}{0.979446in}}%
\pgfpathlineto{\pgfqpoint{6.251464in}{0.978361in}}%
\pgfpathlineto{\pgfqpoint{6.252538in}{0.976191in}}%
\pgfpathlineto{\pgfqpoint{6.256834in}{0.976358in}}%
\pgfpathlineto{\pgfqpoint{6.257908in}{0.969349in}}%
\pgfpathlineto{\pgfqpoint{6.258981in}{0.971435in}}%
\pgfpathlineto{\pgfqpoint{6.260055in}{0.968682in}}%
\pgfpathlineto{\pgfqpoint{6.264351in}{0.983618in}}%
\pgfpathlineto{\pgfqpoint{6.265424in}{0.994549in}}%
\pgfpathlineto{\pgfqpoint{6.266498in}{0.994131in}}%
\pgfpathlineto{\pgfqpoint{6.267572in}{1.006982in}}%
\pgfpathlineto{\pgfqpoint{6.270794in}{1.001975in}}%
\pgfpathlineto{\pgfqpoint{6.271867in}{1.002142in}}%
\pgfpathlineto{\pgfqpoint{6.272941in}{1.010653in}}%
\pgfpathlineto{\pgfqpoint{6.274015in}{0.995300in}}%
\pgfpathlineto{\pgfqpoint{6.275089in}{0.999555in}}%
\pgfpathlineto{\pgfqpoint{6.279384in}{0.998804in}}%
\pgfpathlineto{\pgfqpoint{6.280458in}{1.000390in}}%
\pgfpathlineto{\pgfqpoint{6.281532in}{0.993297in}}%
\pgfpathlineto{\pgfqpoint{6.282606in}{0.996384in}}%
\pgfpathlineto{\pgfqpoint{6.285827in}{0.991962in}}%
\pgfpathlineto{\pgfqpoint{6.286901in}{0.998304in}}%
\pgfpathlineto{\pgfqpoint{6.287975in}{0.999388in}}%
\pgfpathlineto{\pgfqpoint{6.289049in}{0.999555in}}%
\pgfpathlineto{\pgfqpoint{6.290123in}{1.009485in}}%
\pgfpathlineto{\pgfqpoint{6.293344in}{1.022669in}}%
\pgfpathlineto{\pgfqpoint{6.294418in}{1.020416in}}%
\pgfpathlineto{\pgfqpoint{6.295492in}{1.025589in}}%
\pgfpathlineto{\pgfqpoint{6.296566in}{1.020082in}}%
\pgfpathlineto{\pgfqpoint{6.297640in}{1.017579in}}%
\pgfpathlineto{\pgfqpoint{6.300861in}{1.011738in}}%
\pgfpathlineto{\pgfqpoint{6.301935in}{1.007148in}}%
\pgfpathlineto{\pgfqpoint{6.303009in}{1.007148in}}%
\pgfpathlineto{\pgfqpoint{6.304083in}{1.011070in}}%
\pgfpathlineto{\pgfqpoint{6.305156in}{1.009234in}}%
\pgfpathlineto{\pgfqpoint{6.308378in}{1.012322in}}%
\pgfpathlineto{\pgfqpoint{6.309452in}{1.017829in}}%
\pgfpathlineto{\pgfqpoint{6.310526in}{1.016577in}}%
\pgfpathlineto{\pgfqpoint{6.311599in}{1.020833in}}%
\pgfpathlineto{\pgfqpoint{6.312673in}{1.015660in}}%
\pgfpathlineto{\pgfqpoint{6.316969in}{1.015242in}}%
\pgfpathlineto{\pgfqpoint{6.318042in}{1.016077in}}%
\pgfpathlineto{\pgfqpoint{6.319116in}{1.012989in}}%
\pgfpathlineto{\pgfqpoint{6.320190in}{1.017746in}}%
\pgfpathlineto{\pgfqpoint{6.323412in}{1.016995in}}%
\pgfpathlineto{\pgfqpoint{6.324486in}{1.015910in}}%
\pgfpathlineto{\pgfqpoint{6.325559in}{1.023503in}}%
\pgfpathlineto{\pgfqpoint{6.326633in}{1.019915in}}%
\pgfpathlineto{\pgfqpoint{6.327707in}{1.025088in}}%
\pgfpathlineto{\pgfqpoint{6.330929in}{1.020499in}}%
\pgfpathlineto{\pgfqpoint{6.332002in}{1.022085in}}%
\pgfpathlineto{\pgfqpoint{6.333076in}{1.021918in}}%
\pgfpathlineto{\pgfqpoint{6.334150in}{1.023420in}}%
\pgfpathlineto{\pgfqpoint{6.335224in}{1.022585in}}%
\pgfpathlineto{\pgfqpoint{6.338445in}{1.027341in}}%
\pgfpathlineto{\pgfqpoint{6.339519in}{1.033683in}}%
\pgfpathlineto{\pgfqpoint{6.340593in}{1.030178in}}%
\pgfpathlineto{\pgfqpoint{6.341667in}{1.028927in}}%
\pgfpathlineto{\pgfqpoint{6.342741in}{1.029261in}}%
\pgfpathlineto{\pgfqpoint{6.345962in}{1.036937in}}%
\pgfpathlineto{\pgfqpoint{6.347036in}{1.029261in}}%
\pgfpathlineto{\pgfqpoint{6.349184in}{1.033182in}}%
\pgfpathlineto{\pgfqpoint{6.350258in}{1.032348in}}%
\pgfpathlineto{\pgfqpoint{6.353479in}{1.034267in}}%
\pgfpathlineto{\pgfqpoint{6.354553in}{1.039691in}}%
\pgfpathlineto{\pgfqpoint{6.355627in}{1.035519in}}%
\pgfpathlineto{\pgfqpoint{6.356701in}{1.040692in}}%
\pgfpathlineto{\pgfqpoint{6.357775in}{1.042444in}}%
\pgfpathlineto{\pgfqpoint{6.360996in}{1.040859in}}%
\pgfpathlineto{\pgfqpoint{6.362070in}{1.039607in}}%
\pgfpathlineto{\pgfqpoint{6.363144in}{1.039274in}}%
\pgfpathlineto{\pgfqpoint{6.364218in}{1.039691in}}%
\pgfpathlineto{\pgfqpoint{6.365291in}{1.035853in}}%
\pgfpathlineto{\pgfqpoint{6.368513in}{1.034684in}}%
\pgfpathlineto{\pgfqpoint{6.369587in}{1.039858in}}%
\pgfpathlineto{\pgfqpoint{6.370661in}{1.039607in}}%
\pgfpathlineto{\pgfqpoint{6.376030in}{1.045532in}}%
\pgfpathlineto{\pgfqpoint{6.377104in}{1.048869in}}%
\pgfpathlineto{\pgfqpoint{6.378177in}{1.045115in}}%
\pgfpathlineto{\pgfqpoint{6.379251in}{1.053626in}}%
\pgfpathlineto{\pgfqpoint{6.380325in}{1.050872in}}%
\pgfpathlineto{\pgfqpoint{6.383547in}{1.044864in}}%
\pgfpathlineto{\pgfqpoint{6.384620in}{1.056796in}}%
\pgfpathlineto{\pgfqpoint{6.386768in}{1.062137in}}%
\pgfpathlineto{\pgfqpoint{6.391064in}{1.053208in}}%
\pgfpathlineto{\pgfqpoint{6.392137in}{1.050288in}}%
\pgfpathlineto{\pgfqpoint{6.393211in}{1.028176in}}%
\pgfpathlineto{\pgfqpoint{6.394285in}{1.024588in}}%
\pgfpathlineto{\pgfqpoint{6.395359in}{1.031180in}}%
\pgfpathlineto{\pgfqpoint{6.398580in}{1.026590in}}%
\pgfpathlineto{\pgfqpoint{6.399654in}{1.031764in}}%
\pgfpathlineto{\pgfqpoint{6.400728in}{1.012405in}}%
\pgfpathlineto{\pgfqpoint{6.401802in}{1.011738in}}%
\pgfpathlineto{\pgfqpoint{6.402876in}{1.012656in}}%
\pgfpathlineto{\pgfqpoint{6.406097in}{1.008150in}}%
\pgfpathlineto{\pgfqpoint{6.408245in}{0.985453in}}%
\pgfpathlineto{\pgfqpoint{6.409319in}{0.988457in}}%
\pgfpathlineto{\pgfqpoint{6.410393in}{0.995216in}}%
\pgfpathlineto{\pgfqpoint{6.413614in}{0.996134in}}%
\pgfpathlineto{\pgfqpoint{6.414688in}{0.991211in}}%
\pgfpathlineto{\pgfqpoint{6.415762in}{0.996718in}}%
\pgfpathlineto{\pgfqpoint{6.416836in}{0.993214in}}%
\pgfpathlineto{\pgfqpoint{6.417909in}{1.002392in}}%
\pgfpathlineto{\pgfqpoint{6.422205in}{1.001808in}}%
\pgfpathlineto{\pgfqpoint{6.423279in}{0.998554in}}%
\pgfpathlineto{\pgfqpoint{6.424352in}{1.000723in}}%
\pgfpathlineto{\pgfqpoint{6.425426in}{0.992463in}}%
\pgfpathlineto{\pgfqpoint{6.428648in}{0.987206in}}%
\pgfpathlineto{\pgfqpoint{6.429722in}{0.978945in}}%
\pgfpathlineto{\pgfqpoint{6.430796in}{0.982366in}}%
\pgfpathlineto{\pgfqpoint{6.431869in}{0.969516in}}%
\pgfpathlineto{\pgfqpoint{6.432943in}{0.979946in}}%
\pgfpathlineto{\pgfqpoint{6.436165in}{0.991378in}}%
\pgfpathlineto{\pgfqpoint{6.438312in}{0.984118in}}%
\pgfpathlineto{\pgfqpoint{6.439386in}{0.982867in}}%
\pgfpathlineto{\pgfqpoint{6.440460in}{0.979028in}}%
\pgfpathlineto{\pgfqpoint{6.443682in}{0.977944in}}%
\pgfpathlineto{\pgfqpoint{6.444755in}{0.966429in}}%
\pgfpathlineto{\pgfqpoint{6.445829in}{0.973354in}}%
\pgfpathlineto{\pgfqpoint{6.446903in}{0.968682in}}%
\pgfpathlineto{\pgfqpoint{6.447977in}{0.969850in}}%
\pgfpathlineto{\pgfqpoint{6.451198in}{0.979529in}}%
\pgfpathlineto{\pgfqpoint{6.452272in}{0.978277in}}%
\pgfpathlineto{\pgfqpoint{6.453346in}{0.989876in}}%
\pgfpathlineto{\pgfqpoint{6.454420in}{0.980697in}}%
\pgfpathlineto{\pgfqpoint{6.455494in}{0.985036in}}%
\pgfpathlineto{\pgfqpoint{6.458715in}{0.994716in}}%
\pgfpathlineto{\pgfqpoint{6.460863in}{0.980030in}}%
\pgfpathlineto{\pgfqpoint{6.461937in}{0.967847in}}%
\pgfpathlineto{\pgfqpoint{6.463011in}{0.967597in}}%
\pgfpathlineto{\pgfqpoint{6.466232in}{0.969850in}}%
\pgfpathlineto{\pgfqpoint{6.467306in}{0.971852in}}%
\pgfpathlineto{\pgfqpoint{6.468380in}{0.976191in}}%
\pgfpathlineto{\pgfqpoint{6.469454in}{0.975524in}}%
\pgfpathlineto{\pgfqpoint{6.470528in}{0.981949in}}%
\pgfpathlineto{\pgfqpoint{6.473749in}{0.979529in}}%
\pgfpathlineto{\pgfqpoint{6.476971in}{1.000390in}}%
\pgfpathlineto{\pgfqpoint{6.478044in}{0.998137in}}%
\pgfpathlineto{\pgfqpoint{6.481266in}{0.997469in}}%
\pgfpathlineto{\pgfqpoint{6.482340in}{0.992713in}}%
\pgfpathlineto{\pgfqpoint{6.483414in}{0.996968in}}%
\pgfpathlineto{\pgfqpoint{6.484487in}{1.021751in}}%
\pgfpathlineto{\pgfqpoint{6.485561in}{1.021417in}}%
\pgfpathlineto{\pgfqpoint{6.488783in}{1.021167in}}%
\pgfpathlineto{\pgfqpoint{6.489857in}{1.026674in}}%
\pgfpathlineto{\pgfqpoint{6.490930in}{1.010987in}}%
\pgfpathlineto{\pgfqpoint{6.492004in}{1.014575in}}%
\pgfpathlineto{\pgfqpoint{6.493078in}{1.003060in}}%
\pgfpathlineto{\pgfqpoint{6.496300in}{0.992212in}}%
\pgfpathlineto{\pgfqpoint{6.497374in}{0.997302in}}%
\pgfpathlineto{\pgfqpoint{6.498447in}{0.963675in}}%
\pgfpathlineto{\pgfqpoint{6.499521in}{0.951659in}}%
\pgfpathlineto{\pgfqpoint{6.500595in}{0.956833in}}%
\pgfpathlineto{\pgfqpoint{6.503817in}{0.952077in}}%
\pgfpathlineto{\pgfqpoint{6.504890in}{0.952994in}}%
\pgfpathlineto{\pgfqpoint{6.505964in}{0.958585in}}%
\pgfpathlineto{\pgfqpoint{6.508112in}{0.946403in}}%
\pgfpathlineto{\pgfqpoint{6.511333in}{0.950241in}}%
\pgfpathlineto{\pgfqpoint{6.512407in}{0.963174in}}%
\pgfpathlineto{\pgfqpoint{6.513481in}{0.952911in}}%
\pgfpathlineto{\pgfqpoint{6.514555in}{0.953078in}}%
\pgfpathlineto{\pgfqpoint{6.515629in}{0.960254in}}%
\pgfpathlineto{\pgfqpoint{6.519924in}{0.961589in}}%
\pgfpathlineto{\pgfqpoint{6.520998in}{0.964009in}}%
\pgfpathlineto{\pgfqpoint{6.522072in}{0.950491in}}%
\pgfpathlineto{\pgfqpoint{6.523146in}{0.952911in}}%
\pgfpathlineto{\pgfqpoint{6.527441in}{0.953745in}}%
\pgfpathlineto{\pgfqpoint{6.528515in}{0.952911in}}%
\pgfpathlineto{\pgfqpoint{6.529589in}{0.916947in}}%
\pgfpathlineto{\pgfqpoint{6.533884in}{0.917198in}}%
\pgfpathlineto{\pgfqpoint{6.536032in}{0.931216in}}%
\pgfpathlineto{\pgfqpoint{6.537106in}{0.923706in}}%
\pgfpathlineto{\pgfqpoint{6.538179in}{0.928796in}}%
\pgfpathlineto{\pgfqpoint{6.541401in}{0.925375in}}%
\pgfpathlineto{\pgfqpoint{6.542475in}{0.928045in}}%
\pgfpathlineto{\pgfqpoint{6.543549in}{0.934387in}}%
\pgfpathlineto{\pgfqpoint{6.545696in}{0.929380in}}%
\pgfpathlineto{\pgfqpoint{6.548918in}{0.937307in}}%
\pgfpathlineto{\pgfqpoint{6.549992in}{0.929631in}}%
\pgfpathlineto{\pgfqpoint{6.551065in}{0.934554in}}%
\pgfpathlineto{\pgfqpoint{6.552139in}{0.924958in}}%
\pgfpathlineto{\pgfqpoint{6.557508in}{0.947154in}}%
\pgfpathlineto{\pgfqpoint{6.558582in}{0.945234in}}%
\pgfpathlineto{\pgfqpoint{6.559656in}{0.941646in}}%
\pgfpathlineto{\pgfqpoint{6.560730in}{0.941313in}}%
\pgfpathlineto{\pgfqpoint{6.563952in}{0.937307in}}%
\pgfpathlineto{\pgfqpoint{6.565025in}{0.937307in}}%
\pgfpathlineto{\pgfqpoint{6.566099in}{0.929047in}}%
\pgfpathlineto{\pgfqpoint{6.567173in}{0.915863in}}%
\pgfpathlineto{\pgfqpoint{6.568247in}{0.919534in}}%
\pgfpathlineto{\pgfqpoint{6.572542in}{0.927461in}}%
\pgfpathlineto{\pgfqpoint{6.573616in}{0.926543in}}%
\pgfpathlineto{\pgfqpoint{6.575764in}{0.935889in}}%
\pgfpathlineto{\pgfqpoint{6.578985in}{0.930215in}}%
\pgfpathlineto{\pgfqpoint{6.581133in}{0.922955in}}%
\pgfpathlineto{\pgfqpoint{6.582207in}{0.929047in}}%
\pgfpathlineto{\pgfqpoint{6.583281in}{0.927044in}}%
\pgfpathlineto{\pgfqpoint{6.586502in}{0.924874in}}%
\pgfpathlineto{\pgfqpoint{6.587576in}{0.923039in}}%
\pgfpathlineto{\pgfqpoint{6.588650in}{0.932885in}}%
\pgfpathlineto{\pgfqpoint{6.589724in}{0.927378in}}%
\pgfpathlineto{\pgfqpoint{6.590797in}{0.929714in}}%
\pgfpathlineto{\pgfqpoint{6.595093in}{0.953745in}}%
\pgfpathlineto{\pgfqpoint{6.596167in}{0.950324in}}%
\pgfpathlineto{\pgfqpoint{6.598314in}{0.979446in}}%
\pgfpathlineto{\pgfqpoint{6.601536in}{0.979112in}}%
\pgfpathlineto{\pgfqpoint{6.602610in}{0.966429in}}%
\pgfpathlineto{\pgfqpoint{6.603684in}{0.970601in}}%
\pgfpathlineto{\pgfqpoint{6.604757in}{0.969933in}}%
\pgfpathlineto{\pgfqpoint{6.605831in}{0.968598in}}%
\pgfpathlineto{\pgfqpoint{6.609053in}{0.963008in}}%
\pgfpathlineto{\pgfqpoint{6.610127in}{0.965010in}}%
\pgfpathlineto{\pgfqpoint{6.611200in}{0.962924in}}%
\pgfpathlineto{\pgfqpoint{6.613348in}{0.962090in}}%
\pgfpathlineto{\pgfqpoint{6.616570in}{0.963341in}}%
\pgfpathlineto{\pgfqpoint{6.617643in}{0.968348in}}%
\pgfpathlineto{\pgfqpoint{6.618717in}{0.983201in}}%
\pgfpathlineto{\pgfqpoint{6.619791in}{0.979779in}}%
\pgfpathlineto{\pgfqpoint{6.620865in}{0.983284in}}%
\pgfpathlineto{\pgfqpoint{6.624086in}{1.023503in}}%
\pgfpathlineto{\pgfqpoint{6.626234in}{0.984953in}}%
\pgfpathlineto{\pgfqpoint{6.628382in}{0.981782in}}%
\pgfpathlineto{\pgfqpoint{6.632677in}{1.007899in}}%
\pgfpathlineto{\pgfqpoint{6.633751in}{1.009401in}}%
\pgfpathlineto{\pgfqpoint{6.634825in}{1.033516in}}%
\pgfpathlineto{\pgfqpoint{6.635899in}{1.039274in}}%
\pgfpathlineto{\pgfqpoint{6.639120in}{1.037271in}}%
\pgfpathlineto{\pgfqpoint{6.640194in}{1.043696in}}%
\pgfpathlineto{\pgfqpoint{6.641268in}{1.026340in}}%
\pgfpathlineto{\pgfqpoint{6.642342in}{1.025422in}}%
\pgfpathlineto{\pgfqpoint{6.643416in}{1.018079in}}%
\pgfpathlineto{\pgfqpoint{6.647711in}{1.013573in}}%
\pgfpathlineto{\pgfqpoint{6.648785in}{1.009652in}}%
\pgfpathlineto{\pgfqpoint{6.649859in}{1.010653in}}%
\pgfpathlineto{\pgfqpoint{6.650932in}{1.008567in}}%
\pgfpathlineto{\pgfqpoint{6.650932in}{1.008567in}}%
\pgfusepath{stroke}%
\end{pgfscope}%
\begin{pgfscope}%
\pgfpathrectangle{\pgfqpoint{4.185023in}{0.385400in}}{\pgfqpoint{2.583333in}{0.736585in}}%
\pgfusepath{clip}%
\pgfsetroundcap%
\pgfsetroundjoin%
\pgfsetlinewidth{1.505625pt}%
\definecolor{currentstroke}{rgb}{1.000000,0.647059,0.000000}%
\pgfsetstrokecolor{currentstroke}%
\pgfsetdash{}{0pt}%
\pgfpathmoveto{\pgfqpoint{4.302448in}{0.418881in}}%
\pgfpathlineto{\pgfqpoint{4.305669in}{0.425119in}}%
\pgfpathlineto{\pgfqpoint{4.309964in}{0.426489in}}%
\pgfpathlineto{\pgfqpoint{4.313186in}{0.424695in}}%
\pgfpathlineto{\pgfqpoint{4.319629in}{0.424493in}}%
\pgfpathlineto{\pgfqpoint{4.328220in}{0.425158in}}%
\pgfpathlineto{\pgfqpoint{4.335737in}{0.425285in}}%
\pgfpathlineto{\pgfqpoint{4.340032in}{0.426218in}}%
\pgfpathlineto{\pgfqpoint{4.343253in}{0.427929in}}%
\pgfpathlineto{\pgfqpoint{4.348623in}{0.429403in}}%
\pgfpathlineto{\pgfqpoint{4.350770in}{0.430283in}}%
\pgfpathlineto{\pgfqpoint{4.357213in}{0.431289in}}%
\pgfpathlineto{\pgfqpoint{4.373321in}{0.434922in}}%
\pgfpathlineto{\pgfqpoint{4.377616in}{0.435759in}}%
\pgfpathlineto{\pgfqpoint{4.380838in}{0.436980in}}%
\pgfpathlineto{\pgfqpoint{4.386207in}{0.438044in}}%
\pgfpathlineto{\pgfqpoint{4.395872in}{0.440435in}}%
\pgfpathlineto{\pgfqpoint{4.410905in}{0.441595in}}%
\pgfpathlineto{\pgfqpoint{4.422717in}{0.442181in}}%
\pgfpathlineto{\pgfqpoint{4.439899in}{0.444522in}}%
\pgfpathlineto{\pgfqpoint{4.440973in}{0.444839in}}%
\pgfpathlineto{\pgfqpoint{4.452785in}{0.446357in}}%
\pgfpathlineto{\pgfqpoint{4.463523in}{0.447798in}}%
\pgfpathlineto{\pgfqpoint{4.469966in}{0.448610in}}%
\pgfpathlineto{\pgfqpoint{4.471040in}{0.448881in}}%
\pgfpathlineto{\pgfqpoint{4.476409in}{0.449645in}}%
\pgfpathlineto{\pgfqpoint{4.478557in}{0.450263in}}%
\pgfpathlineto{\pgfqpoint{4.483926in}{0.451210in}}%
\pgfpathlineto{\pgfqpoint{4.486074in}{0.451813in}}%
\pgfpathlineto{\pgfqpoint{4.491443in}{0.452655in}}%
\pgfpathlineto{\pgfqpoint{4.497886in}{0.453980in}}%
\pgfpathlineto{\pgfqpoint{4.513994in}{0.456784in}}%
\pgfpathlineto{\pgfqpoint{4.516141in}{0.457372in}}%
\pgfpathlineto{\pgfqpoint{4.521511in}{0.458120in}}%
\pgfpathlineto{\pgfqpoint{4.523658in}{0.458776in}}%
\pgfpathlineto{\pgfqpoint{4.530101in}{0.459906in}}%
\pgfpathlineto{\pgfqpoint{4.531175in}{0.460209in}}%
\pgfpathlineto{\pgfqpoint{4.536544in}{0.461124in}}%
\pgfpathlineto{\pgfqpoint{4.538692in}{0.461697in}}%
\pgfpathlineto{\pgfqpoint{4.544061in}{0.462531in}}%
\pgfpathlineto{\pgfqpoint{4.549430in}{0.463413in}}%
\pgfpathlineto{\pgfqpoint{4.556947in}{0.464619in}}%
\pgfpathlineto{\pgfqpoint{4.561243in}{0.465551in}}%
\pgfpathlineto{\pgfqpoint{4.567686in}{0.466370in}}%
\pgfpathlineto{\pgfqpoint{4.568760in}{0.466678in}}%
\pgfpathlineto{\pgfqpoint{4.574129in}{0.467565in}}%
\pgfpathlineto{\pgfqpoint{4.576276in}{0.468214in}}%
\pgfpathlineto{\pgfqpoint{4.581646in}{0.469132in}}%
\pgfpathlineto{\pgfqpoint{4.583793in}{0.469764in}}%
\pgfpathlineto{\pgfqpoint{4.590236in}{0.470956in}}%
\pgfpathlineto{\pgfqpoint{4.594532in}{0.471507in}}%
\pgfpathlineto{\pgfqpoint{4.606344in}{0.473702in}}%
\pgfpathlineto{\pgfqpoint{4.612787in}{0.474607in}}%
\pgfpathlineto{\pgfqpoint{4.619230in}{0.475417in}}%
\pgfpathlineto{\pgfqpoint{4.621378in}{0.475730in}}%
\pgfpathlineto{\pgfqpoint{4.633190in}{0.476441in}}%
\pgfpathlineto{\pgfqpoint{4.639633in}{0.476813in}}%
\pgfpathlineto{\pgfqpoint{4.700842in}{0.481117in}}%
\pgfpathlineto{\pgfqpoint{4.711580in}{0.482365in}}%
\pgfpathlineto{\pgfqpoint{4.719097in}{0.483245in}}%
\pgfpathlineto{\pgfqpoint{4.725540in}{0.484127in}}%
\pgfpathlineto{\pgfqpoint{4.734131in}{0.485466in}}%
\pgfpathlineto{\pgfqpoint{4.740574in}{0.486381in}}%
\pgfpathlineto{\pgfqpoint{4.741648in}{0.486625in}}%
\pgfpathlineto{\pgfqpoint{4.752386in}{0.487670in}}%
\pgfpathlineto{\pgfqpoint{4.785675in}{0.493138in}}%
\pgfpathlineto{\pgfqpoint{4.793192in}{0.494089in}}%
\pgfpathlineto{\pgfqpoint{4.801782in}{0.495790in}}%
\pgfpathlineto{\pgfqpoint{4.807152in}{0.496661in}}%
\pgfpathlineto{\pgfqpoint{4.816816in}{0.498892in}}%
\pgfpathlineto{\pgfqpoint{4.822185in}{0.499909in}}%
\pgfpathlineto{\pgfqpoint{4.824333in}{0.500638in}}%
\pgfpathlineto{\pgfqpoint{4.829702in}{0.501806in}}%
\pgfpathlineto{\pgfqpoint{4.831850in}{0.502629in}}%
\pgfpathlineto{\pgfqpoint{4.837219in}{0.503883in}}%
\pgfpathlineto{\pgfqpoint{4.839367in}{0.504664in}}%
\pgfpathlineto{\pgfqpoint{4.844736in}{0.505773in}}%
\pgfpathlineto{\pgfqpoint{4.846884in}{0.506481in}}%
\pgfpathlineto{\pgfqpoint{4.853327in}{0.507561in}}%
\pgfpathlineto{\pgfqpoint{4.854401in}{0.507857in}}%
\pgfpathlineto{\pgfqpoint{4.859770in}{0.508778in}}%
\pgfpathlineto{\pgfqpoint{4.869434in}{0.510877in}}%
\pgfpathlineto{\pgfqpoint{4.874804in}{0.511807in}}%
\pgfpathlineto{\pgfqpoint{4.884468in}{0.513652in}}%
\pgfpathlineto{\pgfqpoint{4.895206in}{0.515032in}}%
\pgfpathlineto{\pgfqpoint{4.907019in}{0.517740in}}%
\pgfpathlineto{\pgfqpoint{4.913462in}{0.518816in}}%
\pgfpathlineto{\pgfqpoint{4.917757in}{0.519357in}}%
\pgfpathlineto{\pgfqpoint{4.929569in}{0.521858in}}%
\pgfpathlineto{\pgfqpoint{4.936012in}{0.522783in}}%
\pgfpathlineto{\pgfqpoint{4.944603in}{0.523934in}}%
\pgfpathlineto{\pgfqpoint{4.957489in}{0.525138in}}%
\pgfpathlineto{\pgfqpoint{4.989704in}{0.530075in}}%
\pgfpathlineto{\pgfqpoint{4.996147in}{0.530906in}}%
\pgfpathlineto{\pgfqpoint{5.004738in}{0.532339in}}%
\pgfpathlineto{\pgfqpoint{5.011181in}{0.533400in}}%
\pgfpathlineto{\pgfqpoint{5.016550in}{0.534213in}}%
\pgfpathlineto{\pgfqpoint{5.027289in}{0.536267in}}%
\pgfpathlineto{\pgfqpoint{5.033732in}{0.537282in}}%
\pgfpathlineto{\pgfqpoint{5.039101in}{0.538069in}}%
\pgfpathlineto{\pgfqpoint{5.047692in}{0.539673in}}%
\pgfpathlineto{\pgfqpoint{5.062725in}{0.542079in}}%
\pgfpathlineto{\pgfqpoint{5.072390in}{0.544032in}}%
\pgfpathlineto{\pgfqpoint{5.079907in}{0.545302in}}%
\pgfpathlineto{\pgfqpoint{5.086350in}{0.546346in}}%
\pgfpathlineto{\pgfqpoint{5.087424in}{0.546689in}}%
\pgfpathlineto{\pgfqpoint{5.092793in}{0.547693in}}%
\pgfpathlineto{\pgfqpoint{5.094940in}{0.548337in}}%
\pgfpathlineto{\pgfqpoint{5.100310in}{0.549241in}}%
\pgfpathlineto{\pgfqpoint{5.102457in}{0.549840in}}%
\pgfpathlineto{\pgfqpoint{5.108900in}{0.550761in}}%
\pgfpathlineto{\pgfqpoint{5.114269in}{0.551575in}}%
\pgfpathlineto{\pgfqpoint{5.125008in}{0.553705in}}%
\pgfpathlineto{\pgfqpoint{5.130377in}{0.554703in}}%
\pgfpathlineto{\pgfqpoint{5.132525in}{0.555390in}}%
\pgfpathlineto{\pgfqpoint{5.138968in}{0.556440in}}%
\pgfpathlineto{\pgfqpoint{5.140042in}{0.556800in}}%
\pgfpathlineto{\pgfqpoint{5.145411in}{0.557882in}}%
\pgfpathlineto{\pgfqpoint{5.147558in}{0.558606in}}%
\pgfpathlineto{\pgfqpoint{5.152928in}{0.559708in}}%
\pgfpathlineto{\pgfqpoint{5.155075in}{0.560478in}}%
\pgfpathlineto{\pgfqpoint{5.160445in}{0.561568in}}%
\pgfpathlineto{\pgfqpoint{5.162592in}{0.562247in}}%
\pgfpathlineto{\pgfqpoint{5.167961in}{0.563310in}}%
\pgfpathlineto{\pgfqpoint{5.170109in}{0.563993in}}%
\pgfpathlineto{\pgfqpoint{5.175478in}{0.564952in}}%
\pgfpathlineto{\pgfqpoint{5.177626in}{0.565572in}}%
\pgfpathlineto{\pgfqpoint{5.182995in}{0.566591in}}%
\pgfpathlineto{\pgfqpoint{5.185143in}{0.567263in}}%
\pgfpathlineto{\pgfqpoint{5.190512in}{0.568208in}}%
\pgfpathlineto{\pgfqpoint{5.195881in}{0.569048in}}%
\pgfpathlineto{\pgfqpoint{5.199103in}{0.569939in}}%
\pgfpathlineto{\pgfqpoint{5.205546in}{0.570841in}}%
\pgfpathlineto{\pgfqpoint{5.207693in}{0.571425in}}%
\pgfpathlineto{\pgfqpoint{5.213063in}{0.572281in}}%
\pgfpathlineto{\pgfqpoint{5.222727in}{0.574452in}}%
\pgfpathlineto{\pgfqpoint{5.228096in}{0.575403in}}%
\pgfpathlineto{\pgfqpoint{5.237761in}{0.577560in}}%
\pgfpathlineto{\pgfqpoint{5.244204in}{0.578557in}}%
\pgfpathlineto{\pgfqpoint{5.245278in}{0.578890in}}%
\pgfpathlineto{\pgfqpoint{5.250647in}{0.579886in}}%
\pgfpathlineto{\pgfqpoint{5.252795in}{0.580560in}}%
\pgfpathlineto{\pgfqpoint{5.258164in}{0.581567in}}%
\pgfpathlineto{\pgfqpoint{5.260312in}{0.582182in}}%
\pgfpathlineto{\pgfqpoint{5.265681in}{0.583117in}}%
\pgfpathlineto{\pgfqpoint{5.267828in}{0.583734in}}%
\pgfpathlineto{\pgfqpoint{5.273198in}{0.584645in}}%
\pgfpathlineto{\pgfqpoint{5.275345in}{0.585291in}}%
\pgfpathlineto{\pgfqpoint{5.280714in}{0.586303in}}%
\pgfpathlineto{\pgfqpoint{5.281788in}{0.586645in}}%
\pgfpathlineto{\pgfqpoint{5.288231in}{0.587656in}}%
\pgfpathlineto{\pgfqpoint{5.290379in}{0.588333in}}%
\pgfpathlineto{\pgfqpoint{5.295748in}{0.589311in}}%
\pgfpathlineto{\pgfqpoint{5.297896in}{0.589941in}}%
\pgfpathlineto{\pgfqpoint{5.303265in}{0.590901in}}%
\pgfpathlineto{\pgfqpoint{5.305413in}{0.591548in}}%
\pgfpathlineto{\pgfqpoint{5.310782in}{0.592523in}}%
\pgfpathlineto{\pgfqpoint{5.312930in}{0.593139in}}%
\pgfpathlineto{\pgfqpoint{5.318299in}{0.594100in}}%
\pgfpathlineto{\pgfqpoint{5.327963in}{0.596361in}}%
\pgfpathlineto{\pgfqpoint{5.333333in}{0.597402in}}%
\pgfpathlineto{\pgfqpoint{5.335480in}{0.598101in}}%
\pgfpathlineto{\pgfqpoint{5.340849in}{0.599136in}}%
\pgfpathlineto{\pgfqpoint{5.342997in}{0.599816in}}%
\pgfpathlineto{\pgfqpoint{5.349440in}{0.600848in}}%
\pgfpathlineto{\pgfqpoint{5.350514in}{0.601194in}}%
\pgfpathlineto{\pgfqpoint{5.355883in}{0.602189in}}%
\pgfpathlineto{\pgfqpoint{5.358031in}{0.602845in}}%
\pgfpathlineto{\pgfqpoint{5.363400in}{0.603837in}}%
\pgfpathlineto{\pgfqpoint{5.365548in}{0.604498in}}%
\pgfpathlineto{\pgfqpoint{5.370917in}{0.605434in}}%
\pgfpathlineto{\pgfqpoint{5.373065in}{0.606040in}}%
\pgfpathlineto{\pgfqpoint{5.378434in}{0.606944in}}%
\pgfpathlineto{\pgfqpoint{5.388098in}{0.608932in}}%
\pgfpathlineto{\pgfqpoint{5.394541in}{0.609861in}}%
\pgfpathlineto{\pgfqpoint{5.403132in}{0.611482in}}%
\pgfpathlineto{\pgfqpoint{5.408501in}{0.612366in}}%
\pgfpathlineto{\pgfqpoint{5.410649in}{0.612983in}}%
\pgfpathlineto{\pgfqpoint{5.416018in}{0.613909in}}%
\pgfpathlineto{\pgfqpoint{5.418166in}{0.614522in}}%
\pgfpathlineto{\pgfqpoint{5.423535in}{0.615399in}}%
\pgfpathlineto{\pgfqpoint{5.425683in}{0.615996in}}%
\pgfpathlineto{\pgfqpoint{5.431052in}{0.616870in}}%
\pgfpathlineto{\pgfqpoint{5.438569in}{0.618323in}}%
\pgfpathlineto{\pgfqpoint{5.452529in}{0.620841in}}%
\pgfpathlineto{\pgfqpoint{5.463267in}{0.623257in}}%
\pgfpathlineto{\pgfqpoint{5.470784in}{0.624576in}}%
\pgfpathlineto{\pgfqpoint{5.481522in}{0.626180in}}%
\pgfpathlineto{\pgfqpoint{5.493334in}{0.629007in}}%
\pgfpathlineto{\pgfqpoint{5.499778in}{0.629955in}}%
\pgfpathlineto{\pgfqpoint{5.500851in}{0.630270in}}%
\pgfpathlineto{\pgfqpoint{5.506221in}{0.631183in}}%
\pgfpathlineto{\pgfqpoint{5.508368in}{0.631752in}}%
\pgfpathlineto{\pgfqpoint{5.513737in}{0.632711in}}%
\pgfpathlineto{\pgfqpoint{5.515885in}{0.633478in}}%
\pgfpathlineto{\pgfqpoint{5.521254in}{0.634607in}}%
\pgfpathlineto{\pgfqpoint{5.523402in}{0.635395in}}%
\pgfpathlineto{\pgfqpoint{5.528771in}{0.636182in}}%
\pgfpathlineto{\pgfqpoint{5.530919in}{0.636969in}}%
\pgfpathlineto{\pgfqpoint{5.536288in}{0.638167in}}%
\pgfpathlineto{\pgfqpoint{5.538436in}{0.638947in}}%
\pgfpathlineto{\pgfqpoint{5.543805in}{0.640155in}}%
\pgfpathlineto{\pgfqpoint{5.545953in}{0.640927in}}%
\pgfpathlineto{\pgfqpoint{5.551322in}{0.642057in}}%
\pgfpathlineto{\pgfqpoint{5.553469in}{0.642865in}}%
\pgfpathlineto{\pgfqpoint{5.558839in}{0.644088in}}%
\pgfpathlineto{\pgfqpoint{5.560986in}{0.644907in}}%
\pgfpathlineto{\pgfqpoint{5.566356in}{0.646096in}}%
\pgfpathlineto{\pgfqpoint{5.568503in}{0.646856in}}%
\pgfpathlineto{\pgfqpoint{5.573872in}{0.647993in}}%
\pgfpathlineto{\pgfqpoint{5.574946in}{0.648376in}}%
\pgfpathlineto{\pgfqpoint{5.580315in}{0.649128in}}%
\pgfpathlineto{\pgfqpoint{5.583537in}{0.650283in}}%
\pgfpathlineto{\pgfqpoint{5.588906in}{0.651431in}}%
\pgfpathlineto{\pgfqpoint{5.591054in}{0.652204in}}%
\pgfpathlineto{\pgfqpoint{5.596423in}{0.653372in}}%
\pgfpathlineto{\pgfqpoint{5.598571in}{0.654171in}}%
\pgfpathlineto{\pgfqpoint{5.603940in}{0.655382in}}%
\pgfpathlineto{\pgfqpoint{5.606088in}{0.656178in}}%
\pgfpathlineto{\pgfqpoint{5.611457in}{0.657390in}}%
\pgfpathlineto{\pgfqpoint{5.613604in}{0.658178in}}%
\pgfpathlineto{\pgfqpoint{5.618974in}{0.659334in}}%
\pgfpathlineto{\pgfqpoint{5.621121in}{0.660121in}}%
\pgfpathlineto{\pgfqpoint{5.626490in}{0.661300in}}%
\pgfpathlineto{\pgfqpoint{5.628638in}{0.662081in}}%
\pgfpathlineto{\pgfqpoint{5.634007in}{0.662851in}}%
\pgfpathlineto{\pgfqpoint{5.636155in}{0.663627in}}%
\pgfpathlineto{\pgfqpoint{5.641524in}{0.664798in}}%
\pgfpathlineto{\pgfqpoint{5.643672in}{0.665563in}}%
\pgfpathlineto{\pgfqpoint{5.649041in}{0.666675in}}%
\pgfpathlineto{\pgfqpoint{5.651189in}{0.667430in}}%
\pgfpathlineto{\pgfqpoint{5.656558in}{0.668574in}}%
\pgfpathlineto{\pgfqpoint{5.658706in}{0.669368in}}%
\pgfpathlineto{\pgfqpoint{5.664075in}{0.670578in}}%
\pgfpathlineto{\pgfqpoint{5.666222in}{0.671395in}}%
\pgfpathlineto{\pgfqpoint{5.671592in}{0.672603in}}%
\pgfpathlineto{\pgfqpoint{5.672666in}{0.673016in}}%
\pgfpathlineto{\pgfqpoint{5.678035in}{0.673864in}}%
\pgfpathlineto{\pgfqpoint{5.681256in}{0.675111in}}%
\pgfpathlineto{\pgfqpoint{5.685552in}{0.675976in}}%
\pgfpathlineto{\pgfqpoint{5.688773in}{0.677289in}}%
\pgfpathlineto{\pgfqpoint{5.693068in}{0.678171in}}%
\pgfpathlineto{\pgfqpoint{5.696290in}{0.679476in}}%
\pgfpathlineto{\pgfqpoint{5.700585in}{0.680329in}}%
\pgfpathlineto{\pgfqpoint{5.703807in}{0.681644in}}%
\pgfpathlineto{\pgfqpoint{5.709176in}{0.682894in}}%
\pgfpathlineto{\pgfqpoint{5.711324in}{0.683569in}}%
\pgfpathlineto{\pgfqpoint{5.716693in}{0.684568in}}%
\pgfpathlineto{\pgfqpoint{5.718841in}{0.685208in}}%
\pgfpathlineto{\pgfqpoint{5.725284in}{0.686419in}}%
\pgfpathlineto{\pgfqpoint{5.733874in}{0.687877in}}%
\pgfpathlineto{\pgfqpoint{5.740317in}{0.688917in}}%
\pgfpathlineto{\pgfqpoint{5.741391in}{0.689173in}}%
\pgfpathlineto{\pgfqpoint{5.747834in}{0.689984in}}%
\pgfpathlineto{\pgfqpoint{5.752130in}{0.690544in}}%
\pgfpathlineto{\pgfqpoint{5.760720in}{0.692172in}}%
\pgfpathlineto{\pgfqpoint{5.767163in}{0.693141in}}%
\pgfpathlineto{\pgfqpoint{5.778976in}{0.695508in}}%
\pgfpathlineto{\pgfqpoint{5.784345in}{0.696362in}}%
\pgfpathlineto{\pgfqpoint{5.786492in}{0.696960in}}%
\pgfpathlineto{\pgfqpoint{5.791862in}{0.697894in}}%
\pgfpathlineto{\pgfqpoint{5.794009in}{0.698569in}}%
\pgfpathlineto{\pgfqpoint{5.799378in}{0.699593in}}%
\pgfpathlineto{\pgfqpoint{5.801526in}{0.700281in}}%
\pgfpathlineto{\pgfqpoint{5.806895in}{0.701312in}}%
\pgfpathlineto{\pgfqpoint{5.809043in}{0.701992in}}%
\pgfpathlineto{\pgfqpoint{5.814412in}{0.703066in}}%
\pgfpathlineto{\pgfqpoint{5.816560in}{0.703758in}}%
\pgfpathlineto{\pgfqpoint{5.821929in}{0.704819in}}%
\pgfpathlineto{\pgfqpoint{5.824077in}{0.705566in}}%
\pgfpathlineto{\pgfqpoint{5.829446in}{0.706663in}}%
\pgfpathlineto{\pgfqpoint{5.844480in}{0.709549in}}%
\pgfpathlineto{\pgfqpoint{5.846627in}{0.710129in}}%
\pgfpathlineto{\pgfqpoint{5.851997in}{0.711050in}}%
\pgfpathlineto{\pgfqpoint{5.854144in}{0.711631in}}%
\pgfpathlineto{\pgfqpoint{5.864883in}{0.712937in}}%
\pgfpathlineto{\pgfqpoint{5.868104in}{0.713709in}}%
\pgfpathlineto{\pgfqpoint{5.875621in}{0.714566in}}%
\pgfpathlineto{\pgfqpoint{5.884212in}{0.715735in}}%
\pgfpathlineto{\pgfqpoint{5.896024in}{0.716711in}}%
\pgfpathlineto{\pgfqpoint{5.906762in}{0.717977in}}%
\pgfpathlineto{\pgfqpoint{5.920722in}{0.719086in}}%
\pgfpathlineto{\pgfqpoint{5.929313in}{0.720053in}}%
\pgfpathlineto{\pgfqpoint{5.940051in}{0.721113in}}%
\pgfpathlineto{\pgfqpoint{5.951864in}{0.722713in}}%
\pgfpathlineto{\pgfqpoint{5.964750in}{0.723918in}}%
\pgfpathlineto{\pgfqpoint{5.974414in}{0.725099in}}%
\pgfpathlineto{\pgfqpoint{5.985153in}{0.726138in}}%
\pgfpathlineto{\pgfqpoint{5.996965in}{0.728018in}}%
\pgfpathlineto{\pgfqpoint{6.003408in}{0.728853in}}%
\pgfpathlineto{\pgfqpoint{6.011999in}{0.730072in}}%
\pgfpathlineto{\pgfqpoint{6.023811in}{0.731266in}}%
\pgfpathlineto{\pgfqpoint{6.034549in}{0.732437in}}%
\pgfpathlineto{\pgfqpoint{6.046361in}{0.733532in}}%
\pgfpathlineto{\pgfqpoint{6.057100in}{0.734788in}}%
\pgfpathlineto{\pgfqpoint{6.069986in}{0.735805in}}%
\pgfpathlineto{\pgfqpoint{6.079650in}{0.736973in}}%
\pgfpathlineto{\pgfqpoint{6.092536in}{0.738196in}}%
\pgfpathlineto{\pgfqpoint{6.102201in}{0.739139in}}%
\pgfpathlineto{\pgfqpoint{6.114013in}{0.740134in}}%
\pgfpathlineto{\pgfqpoint{6.124752in}{0.741206in}}%
\pgfpathlineto{\pgfqpoint{6.144081in}{0.742500in}}%
\pgfpathlineto{\pgfqpoint{6.162336in}{0.743874in}}%
\pgfpathlineto{\pgfqpoint{6.180591in}{0.744976in}}%
\pgfpathlineto{\pgfqpoint{6.272941in}{0.753576in}}%
\pgfpathlineto{\pgfqpoint{6.275089in}{0.753961in}}%
\pgfpathlineto{\pgfqpoint{6.285827in}{0.754913in}}%
\pgfpathlineto{\pgfqpoint{6.297640in}{0.756723in}}%
\pgfpathlineto{\pgfqpoint{6.308378in}{0.757903in}}%
\pgfpathlineto{\pgfqpoint{6.312673in}{0.758708in}}%
\pgfpathlineto{\pgfqpoint{6.323412in}{0.759700in}}%
\pgfpathlineto{\pgfqpoint{6.335224in}{0.761507in}}%
\pgfpathlineto{\pgfqpoint{6.345962in}{0.762740in}}%
\pgfpathlineto{\pgfqpoint{6.357775in}{0.764600in}}%
\pgfpathlineto{\pgfqpoint{6.368513in}{0.765838in}}%
\pgfpathlineto{\pgfqpoint{6.371734in}{0.766458in}}%
\pgfpathlineto{\pgfqpoint{6.379251in}{0.767303in}}%
\pgfpathlineto{\pgfqpoint{6.387842in}{0.768596in}}%
\pgfpathlineto{\pgfqpoint{6.398580in}{0.769787in}}%
\pgfpathlineto{\pgfqpoint{6.409319in}{0.771185in}}%
\pgfpathlineto{\pgfqpoint{6.417909in}{0.772177in}}%
\pgfpathlineto{\pgfqpoint{6.429722in}{0.773149in}}%
\pgfpathlineto{\pgfqpoint{6.440460in}{0.774366in}}%
\pgfpathlineto{\pgfqpoint{6.452272in}{0.775376in}}%
\pgfpathlineto{\pgfqpoint{6.463011in}{0.776413in}}%
\pgfpathlineto{\pgfqpoint{6.474823in}{0.777422in}}%
\pgfpathlineto{\pgfqpoint{6.485561in}{0.778710in}}%
\pgfpathlineto{\pgfqpoint{6.497374in}{0.779856in}}%
\pgfpathlineto{\pgfqpoint{6.508112in}{0.780841in}}%
\pgfpathlineto{\pgfqpoint{6.520998in}{0.781838in}}%
\pgfpathlineto{\pgfqpoint{6.530663in}{0.782504in}}%
\pgfpathlineto{\pgfqpoint{6.544622in}{0.783408in}}%
\pgfpathlineto{\pgfqpoint{6.560730in}{0.784567in}}%
\pgfpathlineto{\pgfqpoint{6.580059in}{0.785749in}}%
\pgfpathlineto{\pgfqpoint{6.613348in}{0.788194in}}%
\pgfpathlineto{\pgfqpoint{6.625160in}{0.789121in}}%
\pgfpathlineto{\pgfqpoint{6.643416in}{0.791071in}}%
\pgfpathlineto{\pgfqpoint{6.650932in}{0.791653in}}%
\pgfpathlineto{\pgfqpoint{6.650932in}{0.791653in}}%
\pgfusepath{stroke}%
\end{pgfscope}%
\begin{pgfscope}%
\pgfsetrectcap%
\pgfsetmiterjoin%
\pgfsetlinewidth{0.803000pt}%
\definecolor{currentstroke}{rgb}{1.000000,1.000000,1.000000}%
\pgfsetstrokecolor{currentstroke}%
\pgfsetdash{}{0pt}%
\pgfpathmoveto{\pgfqpoint{4.185023in}{0.385400in}}%
\pgfpathlineto{\pgfqpoint{4.185023in}{1.121986in}}%
\pgfusepath{stroke}%
\end{pgfscope}%
\begin{pgfscope}%
\pgfsetrectcap%
\pgfsetmiterjoin%
\pgfsetlinewidth{0.803000pt}%
\definecolor{currentstroke}{rgb}{1.000000,1.000000,1.000000}%
\pgfsetstrokecolor{currentstroke}%
\pgfsetdash{}{0pt}%
\pgfpathmoveto{\pgfqpoint{6.768357in}{0.385400in}}%
\pgfpathlineto{\pgfqpoint{6.768357in}{1.121986in}}%
\pgfusepath{stroke}%
\end{pgfscope}%
\begin{pgfscope}%
\pgfsetrectcap%
\pgfsetmiterjoin%
\pgfsetlinewidth{0.803000pt}%
\definecolor{currentstroke}{rgb}{1.000000,1.000000,1.000000}%
\pgfsetstrokecolor{currentstroke}%
\pgfsetdash{}{0pt}%
\pgfpathmoveto{\pgfqpoint{4.185023in}{0.385400in}}%
\pgfpathlineto{\pgfqpoint{6.768357in}{0.385400in}}%
\pgfusepath{stroke}%
\end{pgfscope}%
\begin{pgfscope}%
\pgfsetrectcap%
\pgfsetmiterjoin%
\pgfsetlinewidth{0.803000pt}%
\definecolor{currentstroke}{rgb}{1.000000,1.000000,1.000000}%
\pgfsetstrokecolor{currentstroke}%
\pgfsetdash{}{0pt}%
\pgfpathmoveto{\pgfqpoint{4.185023in}{1.121986in}}%
\pgfpathlineto{\pgfqpoint{6.768357in}{1.121986in}}%
\pgfusepath{stroke}%
\end{pgfscope}%
\begin{pgfscope}%
\definecolor{textcolor}{rgb}{0.150000,0.150000,0.150000}%
\pgfsetstrokecolor{textcolor}%
\pgfsetfillcolor{textcolor}%
\pgftext[x=5.476690in,y=1.205319in,,base]{\color{textcolor}\rmfamily\fontsize{16.800000}{20.160000}\selectfont DIS}%
\end{pgfscope}%
\end{pgfpicture}%
\makeatother%
\endgroup%

            \end{adjustbox}
        %\caption{Flower one.}
    \end{minipage}
    \hfill
    \begin{minipage}[b]{0.49\textwidth}
        \centering
        \begin{adjustbox}{width=\textwidth,center}
            %% Creator: Matplotlib, PGF backend
%%
%% To include the figure in your LaTeX document, write
%%   \input{<filename>.pgf}
%%
%% Make sure the required packages are loaded in your preamble
%%   \usepackage{pgf}
%%
%% Figures using additional raster images can only be included by \input if
%% they are in the same directory as the main LaTeX file. For loading figures
%% from other directories you can use the `import` package
%%   \usepackage{import}
%% and then include the figures with
%%   \import{<path to file>}{<filename>.pgf}
%%
%% Matplotlib used the following preamble
%%   \usepackage{fontspec}
%%   \setmainfont{DejaVuSerif.ttf}[Path=/opt/tljh/user/lib/python3.6/site-packages/matplotlib/mpl-data/fonts/ttf/]
%%   \setsansfont{DejaVuSans.ttf}[Path=/opt/tljh/user/lib/python3.6/site-packages/matplotlib/mpl-data/fonts/ttf/]
%%   \setmonofont{DejaVuSansMono.ttf}[Path=/opt/tljh/user/lib/python3.6/site-packages/matplotlib/mpl-data/fonts/ttf/]
%%
\begingroup%
\makeatletter%
\begin{pgfpicture}%
\pgfpathrectangle{\pgfpointorigin}{\pgfqpoint{6.648115in}{4.890977in}}%
\pgfusepath{use as bounding box, clip}%
\begin{pgfscope}%
\pgfsetbuttcap%
\pgfsetmiterjoin%
\definecolor{currentfill}{rgb}{1.000000,1.000000,1.000000}%
\pgfsetfillcolor{currentfill}%
\pgfsetlinewidth{0.000000pt}%
\definecolor{currentstroke}{rgb}{1.000000,1.000000,1.000000}%
\pgfsetstrokecolor{currentstroke}%
\pgfsetdash{}{0pt}%
\pgfpathmoveto{\pgfqpoint{0.000000in}{0.000000in}}%
\pgfpathlineto{\pgfqpoint{6.648115in}{0.000000in}}%
\pgfpathlineto{\pgfqpoint{6.648115in}{4.890977in}}%
\pgfpathlineto{\pgfqpoint{0.000000in}{4.890977in}}%
\pgfpathclose%
\pgfusepath{fill}%
\end{pgfscope}%
\begin{pgfscope}%
\pgfsetbuttcap%
\pgfsetmiterjoin%
\definecolor{currentfill}{rgb}{0.917647,0.917647,0.949020}%
\pgfsetfillcolor{currentfill}%
\pgfsetlinewidth{0.000000pt}%
\definecolor{currentstroke}{rgb}{0.000000,0.000000,0.000000}%
\pgfsetstrokecolor{currentstroke}%
\pgfsetstrokeopacity{0.000000}%
\pgfsetdash{}{0pt}%
\pgfpathmoveto{\pgfqpoint{0.285588in}{4.180131in}}%
\pgfpathlineto{\pgfqpoint{2.868921in}{4.180131in}}%
\pgfpathlineto{\pgfqpoint{2.868921in}{4.581016in}}%
\pgfpathlineto{\pgfqpoint{0.285588in}{4.581016in}}%
\pgfpathclose%
\pgfusepath{fill}%
\end{pgfscope}%
\begin{pgfscope}%
\pgfpathrectangle{\pgfqpoint{0.285588in}{4.180131in}}{\pgfqpoint{2.583333in}{0.400885in}}%
\pgfusepath{clip}%
\pgfsetroundcap%
\pgfsetroundjoin%
\pgfsetlinewidth{0.803000pt}%
\definecolor{currentstroke}{rgb}{1.000000,1.000000,1.000000}%
\pgfsetstrokecolor{currentstroke}%
\pgfsetdash{}{0pt}%
\pgfpathmoveto{\pgfqpoint{0.400864in}{4.180131in}}%
\pgfpathlineto{\pgfqpoint{0.400864in}{4.581016in}}%
\pgfusepath{stroke}%
\end{pgfscope}%
\begin{pgfscope}%
\definecolor{textcolor}{rgb}{0.150000,0.150000,0.150000}%
\pgfsetstrokecolor{textcolor}%
\pgfsetfillcolor{textcolor}%
\pgftext[x=0.400864in,y=4.082908in,,top]{\color{textcolor}\rmfamily\fontsize{10.000000}{12.000000}\selectfont 2012}%
\end{pgfscope}%
\begin{pgfscope}%
\pgfpathrectangle{\pgfqpoint{0.285588in}{4.180131in}}{\pgfqpoint{2.583333in}{0.400885in}}%
\pgfusepath{clip}%
\pgfsetroundcap%
\pgfsetroundjoin%
\pgfsetlinewidth{0.803000pt}%
\definecolor{currentstroke}{rgb}{1.000000,1.000000,1.000000}%
\pgfsetstrokecolor{currentstroke}%
\pgfsetdash{}{0pt}%
\pgfpathmoveto{\pgfqpoint{0.793889in}{4.180131in}}%
\pgfpathlineto{\pgfqpoint{0.793889in}{4.581016in}}%
\pgfusepath{stroke}%
\end{pgfscope}%
\begin{pgfscope}%
\definecolor{textcolor}{rgb}{0.150000,0.150000,0.150000}%
\pgfsetstrokecolor{textcolor}%
\pgfsetfillcolor{textcolor}%
\pgftext[x=0.793889in,y=4.082908in,,top]{\color{textcolor}\rmfamily\fontsize{10.000000}{12.000000}\selectfont 2013}%
\end{pgfscope}%
\begin{pgfscope}%
\pgfpathrectangle{\pgfqpoint{0.285588in}{4.180131in}}{\pgfqpoint{2.583333in}{0.400885in}}%
\pgfusepath{clip}%
\pgfsetroundcap%
\pgfsetroundjoin%
\pgfsetlinewidth{0.803000pt}%
\definecolor{currentstroke}{rgb}{1.000000,1.000000,1.000000}%
\pgfsetstrokecolor{currentstroke}%
\pgfsetdash{}{0pt}%
\pgfpathmoveto{\pgfqpoint{1.185840in}{4.180131in}}%
\pgfpathlineto{\pgfqpoint{1.185840in}{4.581016in}}%
\pgfusepath{stroke}%
\end{pgfscope}%
\begin{pgfscope}%
\definecolor{textcolor}{rgb}{0.150000,0.150000,0.150000}%
\pgfsetstrokecolor{textcolor}%
\pgfsetfillcolor{textcolor}%
\pgftext[x=1.185840in,y=4.082908in,,top]{\color{textcolor}\rmfamily\fontsize{10.000000}{12.000000}\selectfont 2014}%
\end{pgfscope}%
\begin{pgfscope}%
\pgfpathrectangle{\pgfqpoint{0.285588in}{4.180131in}}{\pgfqpoint{2.583333in}{0.400885in}}%
\pgfusepath{clip}%
\pgfsetroundcap%
\pgfsetroundjoin%
\pgfsetlinewidth{0.803000pt}%
\definecolor{currentstroke}{rgb}{1.000000,1.000000,1.000000}%
\pgfsetstrokecolor{currentstroke}%
\pgfsetdash{}{0pt}%
\pgfpathmoveto{\pgfqpoint{1.577791in}{4.180131in}}%
\pgfpathlineto{\pgfqpoint{1.577791in}{4.581016in}}%
\pgfusepath{stroke}%
\end{pgfscope}%
\begin{pgfscope}%
\definecolor{textcolor}{rgb}{0.150000,0.150000,0.150000}%
\pgfsetstrokecolor{textcolor}%
\pgfsetfillcolor{textcolor}%
\pgftext[x=1.577791in,y=4.082908in,,top]{\color{textcolor}\rmfamily\fontsize{10.000000}{12.000000}\selectfont 2015}%
\end{pgfscope}%
\begin{pgfscope}%
\pgfpathrectangle{\pgfqpoint{0.285588in}{4.180131in}}{\pgfqpoint{2.583333in}{0.400885in}}%
\pgfusepath{clip}%
\pgfsetroundcap%
\pgfsetroundjoin%
\pgfsetlinewidth{0.803000pt}%
\definecolor{currentstroke}{rgb}{1.000000,1.000000,1.000000}%
\pgfsetstrokecolor{currentstroke}%
\pgfsetdash{}{0pt}%
\pgfpathmoveto{\pgfqpoint{1.969742in}{4.180131in}}%
\pgfpathlineto{\pgfqpoint{1.969742in}{4.581016in}}%
\pgfusepath{stroke}%
\end{pgfscope}%
\begin{pgfscope}%
\definecolor{textcolor}{rgb}{0.150000,0.150000,0.150000}%
\pgfsetstrokecolor{textcolor}%
\pgfsetfillcolor{textcolor}%
\pgftext[x=1.969742in,y=4.082908in,,top]{\color{textcolor}\rmfamily\fontsize{10.000000}{12.000000}\selectfont 2016}%
\end{pgfscope}%
\begin{pgfscope}%
\pgfpathrectangle{\pgfqpoint{0.285588in}{4.180131in}}{\pgfqpoint{2.583333in}{0.400885in}}%
\pgfusepath{clip}%
\pgfsetroundcap%
\pgfsetroundjoin%
\pgfsetlinewidth{0.803000pt}%
\definecolor{currentstroke}{rgb}{1.000000,1.000000,1.000000}%
\pgfsetstrokecolor{currentstroke}%
\pgfsetdash{}{0pt}%
\pgfpathmoveto{\pgfqpoint{2.362767in}{4.180131in}}%
\pgfpathlineto{\pgfqpoint{2.362767in}{4.581016in}}%
\pgfusepath{stroke}%
\end{pgfscope}%
\begin{pgfscope}%
\definecolor{textcolor}{rgb}{0.150000,0.150000,0.150000}%
\pgfsetstrokecolor{textcolor}%
\pgfsetfillcolor{textcolor}%
\pgftext[x=2.362767in,y=4.082908in,,top]{\color{textcolor}\rmfamily\fontsize{10.000000}{12.000000}\selectfont 2017}%
\end{pgfscope}%
\begin{pgfscope}%
\pgfpathrectangle{\pgfqpoint{0.285588in}{4.180131in}}{\pgfqpoint{2.583333in}{0.400885in}}%
\pgfusepath{clip}%
\pgfsetroundcap%
\pgfsetroundjoin%
\pgfsetlinewidth{0.803000pt}%
\definecolor{currentstroke}{rgb}{1.000000,1.000000,1.000000}%
\pgfsetstrokecolor{currentstroke}%
\pgfsetdash{}{0pt}%
\pgfpathmoveto{\pgfqpoint{2.754718in}{4.180131in}}%
\pgfpathlineto{\pgfqpoint{2.754718in}{4.581016in}}%
\pgfusepath{stroke}%
\end{pgfscope}%
\begin{pgfscope}%
\definecolor{textcolor}{rgb}{0.150000,0.150000,0.150000}%
\pgfsetstrokecolor{textcolor}%
\pgfsetfillcolor{textcolor}%
\pgftext[x=2.754718in,y=4.082908in,,top]{\color{textcolor}\rmfamily\fontsize{10.000000}{12.000000}\selectfont 2018}%
\end{pgfscope}%
\begin{pgfscope}%
\pgfpathrectangle{\pgfqpoint{0.285588in}{4.180131in}}{\pgfqpoint{2.583333in}{0.400885in}}%
\pgfusepath{clip}%
\pgfsetroundcap%
\pgfsetroundjoin%
\pgfsetlinewidth{0.803000pt}%
\definecolor{currentstroke}{rgb}{1.000000,1.000000,1.000000}%
\pgfsetstrokecolor{currentstroke}%
\pgfsetdash{}{0pt}%
\pgfpathmoveto{\pgfqpoint{0.285588in}{4.289571in}}%
\pgfpathlineto{\pgfqpoint{2.868921in}{4.289571in}}%
\pgfusepath{stroke}%
\end{pgfscope}%
\begin{pgfscope}%
\definecolor{textcolor}{rgb}{0.150000,0.150000,0.150000}%
\pgfsetstrokecolor{textcolor}%
\pgfsetfillcolor{textcolor}%
\pgftext[x=0.100000in,y=4.236809in,left,base]{\color{textcolor}\rmfamily\fontsize{10.000000}{12.000000}\selectfont 1}%
\end{pgfscope}%
\begin{pgfscope}%
\pgfpathrectangle{\pgfqpoint{0.285588in}{4.180131in}}{\pgfqpoint{2.583333in}{0.400885in}}%
\pgfusepath{clip}%
\pgfsetroundcap%
\pgfsetroundjoin%
\pgfsetlinewidth{0.803000pt}%
\definecolor{currentstroke}{rgb}{1.000000,1.000000,1.000000}%
\pgfsetstrokecolor{currentstroke}%
\pgfsetdash{}{0pt}%
\pgfpathmoveto{\pgfqpoint{0.285588in}{4.404445in}}%
\pgfpathlineto{\pgfqpoint{2.868921in}{4.404445in}}%
\pgfusepath{stroke}%
\end{pgfscope}%
\begin{pgfscope}%
\definecolor{textcolor}{rgb}{0.150000,0.150000,0.150000}%
\pgfsetstrokecolor{textcolor}%
\pgfsetfillcolor{textcolor}%
\pgftext[x=0.100000in,y=4.351683in,left,base]{\color{textcolor}\rmfamily\fontsize{10.000000}{12.000000}\selectfont 2}%
\end{pgfscope}%
\begin{pgfscope}%
\pgfpathrectangle{\pgfqpoint{0.285588in}{4.180131in}}{\pgfqpoint{2.583333in}{0.400885in}}%
\pgfusepath{clip}%
\pgfsetroundcap%
\pgfsetroundjoin%
\pgfsetlinewidth{0.803000pt}%
\definecolor{currentstroke}{rgb}{1.000000,1.000000,1.000000}%
\pgfsetstrokecolor{currentstroke}%
\pgfsetdash{}{0pt}%
\pgfpathmoveto{\pgfqpoint{0.285588in}{4.519319in}}%
\pgfpathlineto{\pgfqpoint{2.868921in}{4.519319in}}%
\pgfusepath{stroke}%
\end{pgfscope}%
\begin{pgfscope}%
\definecolor{textcolor}{rgb}{0.150000,0.150000,0.150000}%
\pgfsetstrokecolor{textcolor}%
\pgfsetfillcolor{textcolor}%
\pgftext[x=0.100000in,y=4.466557in,left,base]{\color{textcolor}\rmfamily\fontsize{10.000000}{12.000000}\selectfont 3}%
\end{pgfscope}%
\begin{pgfscope}%
\pgfpathrectangle{\pgfqpoint{0.285588in}{4.180131in}}{\pgfqpoint{2.583333in}{0.400885in}}%
\pgfusepath{clip}%
\pgfsetroundcap%
\pgfsetroundjoin%
\pgfsetlinewidth{1.505625pt}%
\definecolor{currentstroke}{rgb}{0.121569,0.466667,0.705882}%
\pgfsetstrokecolor{currentstroke}%
\pgfsetstrokeopacity{0.300000}%
\pgfsetdash{}{0pt}%
\pgfpathmoveto{\pgfqpoint{0.403012in}{4.289571in}}%
\pgfpathlineto{\pgfqpoint{0.404086in}{4.290528in}}%
\pgfpathlineto{\pgfqpoint{0.406233in}{4.289420in}}%
\pgfpathlineto{\pgfqpoint{0.410529in}{4.290696in}}%
\pgfpathlineto{\pgfqpoint{0.411602in}{4.289957in}}%
\pgfpathlineto{\pgfqpoint{0.412676in}{4.290662in}}%
\pgfpathlineto{\pgfqpoint{0.413750in}{4.289722in}}%
\pgfpathlineto{\pgfqpoint{0.418045in}{4.290595in}}%
\pgfpathlineto{\pgfqpoint{0.420193in}{4.292761in}}%
\pgfpathlineto{\pgfqpoint{0.421267in}{4.292543in}}%
\pgfpathlineto{\pgfqpoint{0.425562in}{4.292929in}}%
\pgfpathlineto{\pgfqpoint{0.426636in}{4.293685in}}%
\pgfpathlineto{\pgfqpoint{0.427710in}{4.295213in}}%
\pgfpathlineto{\pgfqpoint{0.433079in}{4.294004in}}%
\pgfpathlineto{\pgfqpoint{0.434153in}{4.294894in}}%
\pgfpathlineto{\pgfqpoint{0.442744in}{4.295817in}}%
\pgfpathlineto{\pgfqpoint{0.443818in}{4.294608in}}%
\pgfpathlineto{\pgfqpoint{0.448113in}{4.295767in}}%
\pgfpathlineto{\pgfqpoint{0.449187in}{4.295230in}}%
\pgfpathlineto{\pgfqpoint{0.450261in}{4.296120in}}%
\pgfpathlineto{\pgfqpoint{0.451334in}{4.295985in}}%
\pgfpathlineto{\pgfqpoint{0.457778in}{4.296456in}}%
\pgfpathlineto{\pgfqpoint{0.458851in}{4.296875in}}%
\pgfpathlineto{\pgfqpoint{0.464221in}{4.296053in}}%
\pgfpathlineto{\pgfqpoint{0.469590in}{4.295297in}}%
\pgfpathlineto{\pgfqpoint{0.470664in}{4.292375in}}%
\pgfpathlineto{\pgfqpoint{0.471737in}{4.293097in}}%
\pgfpathlineto{\pgfqpoint{0.472811in}{4.294810in}}%
\pgfpathlineto{\pgfqpoint{0.473885in}{4.294944in}}%
\pgfpathlineto{\pgfqpoint{0.477107in}{4.295969in}}%
\pgfpathlineto{\pgfqpoint{0.478180in}{4.297648in}}%
\pgfpathlineto{\pgfqpoint{0.479254in}{4.297799in}}%
\pgfpathlineto{\pgfqpoint{0.480328in}{4.299377in}}%
\pgfpathlineto{\pgfqpoint{0.481402in}{4.298756in}}%
\pgfpathlineto{\pgfqpoint{0.484623in}{4.299008in}}%
\pgfpathlineto{\pgfqpoint{0.488919in}{4.297245in}}%
\pgfpathlineto{\pgfqpoint{0.493214in}{4.298185in}}%
\pgfpathlineto{\pgfqpoint{0.494288in}{4.297228in}}%
\pgfpathlineto{\pgfqpoint{0.496436in}{4.298269in}}%
\pgfpathlineto{\pgfqpoint{0.499657in}{4.298303in}}%
\pgfpathlineto{\pgfqpoint{0.500731in}{4.297698in}}%
\pgfpathlineto{\pgfqpoint{0.502879in}{4.295582in}}%
\pgfpathlineto{\pgfqpoint{0.507174in}{4.294289in}}%
\pgfpathlineto{\pgfqpoint{0.508248in}{4.291871in}}%
\pgfpathlineto{\pgfqpoint{0.509322in}{4.292929in}}%
\pgfpathlineto{\pgfqpoint{0.510396in}{4.295028in}}%
\pgfpathlineto{\pgfqpoint{0.511469in}{4.293399in}}%
\pgfpathlineto{\pgfqpoint{0.514691in}{4.294390in}}%
\pgfpathlineto{\pgfqpoint{0.515765in}{4.295834in}}%
\pgfpathlineto{\pgfqpoint{0.517912in}{4.294944in}}%
\pgfpathlineto{\pgfqpoint{0.518986in}{4.295885in}}%
\pgfpathlineto{\pgfqpoint{0.522208in}{4.295398in}}%
\pgfpathlineto{\pgfqpoint{0.523282in}{4.297278in}}%
\pgfpathlineto{\pgfqpoint{0.526503in}{4.298487in}}%
\pgfpathlineto{\pgfqpoint{0.532946in}{4.298521in}}%
\pgfpathlineto{\pgfqpoint{0.534020in}{4.297530in}}%
\pgfpathlineto{\pgfqpoint{0.541537in}{4.294844in}}%
\pgfpathlineto{\pgfqpoint{0.545832in}{4.293534in}}%
\pgfpathlineto{\pgfqpoint{0.546906in}{4.293752in}}%
\pgfpathlineto{\pgfqpoint{0.549054in}{4.291183in}}%
\pgfpathlineto{\pgfqpoint{0.552275in}{4.292509in}}%
\pgfpathlineto{\pgfqpoint{0.553349in}{4.292039in}}%
\pgfpathlineto{\pgfqpoint{0.555497in}{4.293232in}}%
\pgfpathlineto{\pgfqpoint{0.556571in}{4.292946in}}%
\pgfpathlineto{\pgfqpoint{0.560866in}{4.294273in}}%
\pgfpathlineto{\pgfqpoint{0.561940in}{4.292493in}}%
\pgfpathlineto{\pgfqpoint{0.563014in}{4.292442in}}%
\pgfpathlineto{\pgfqpoint{0.564088in}{4.290259in}}%
\pgfpathlineto{\pgfqpoint{0.568383in}{4.289789in}}%
\pgfpathlineto{\pgfqpoint{0.569457in}{4.292745in}}%
\pgfpathlineto{\pgfqpoint{0.571604in}{4.294659in}}%
\pgfpathlineto{\pgfqpoint{0.574826in}{4.293517in}}%
\pgfpathlineto{\pgfqpoint{0.575900in}{4.295666in}}%
\pgfpathlineto{\pgfqpoint{0.576974in}{4.294844in}}%
\pgfpathlineto{\pgfqpoint{0.579121in}{4.296657in}}%
\pgfpathlineto{\pgfqpoint{0.582343in}{4.296489in}}%
\pgfpathlineto{\pgfqpoint{0.583417in}{4.297194in}}%
\pgfpathlineto{\pgfqpoint{0.584490in}{4.296808in}}%
\pgfpathlineto{\pgfqpoint{0.585564in}{4.295666in}}%
\pgfpathlineto{\pgfqpoint{0.586638in}{4.295817in}}%
\pgfpathlineto{\pgfqpoint{0.589860in}{4.294441in}}%
\pgfpathlineto{\pgfqpoint{0.590933in}{4.294911in}}%
\pgfpathlineto{\pgfqpoint{0.592007in}{4.296271in}}%
\pgfpathlineto{\pgfqpoint{0.593081in}{4.296271in}}%
\pgfpathlineto{\pgfqpoint{0.594155in}{4.299680in}}%
\pgfpathlineto{\pgfqpoint{0.597377in}{4.299226in}}%
\pgfpathlineto{\pgfqpoint{0.598450in}{4.299814in}}%
\pgfpathlineto{\pgfqpoint{0.600598in}{4.299629in}}%
\pgfpathlineto{\pgfqpoint{0.601672in}{4.298823in}}%
\pgfpathlineto{\pgfqpoint{0.604893in}{4.298790in}}%
\pgfpathlineto{\pgfqpoint{0.608115in}{4.295230in}}%
\pgfpathlineto{\pgfqpoint{0.609189in}{4.296875in}}%
\pgfpathlineto{\pgfqpoint{0.612410in}{4.297581in}}%
\pgfpathlineto{\pgfqpoint{0.613484in}{4.298806in}}%
\pgfpathlineto{\pgfqpoint{0.614558in}{4.301443in}}%
\pgfpathlineto{\pgfqpoint{0.615632in}{4.301392in}}%
\pgfpathlineto{\pgfqpoint{0.616706in}{4.300217in}}%
\pgfpathlineto{\pgfqpoint{0.619927in}{4.299344in}}%
\pgfpathlineto{\pgfqpoint{0.621001in}{4.297782in}}%
\pgfpathlineto{\pgfqpoint{0.622075in}{4.298487in}}%
\pgfpathlineto{\pgfqpoint{0.624222in}{4.302618in}}%
\pgfpathlineto{\pgfqpoint{0.629592in}{4.301913in}}%
\pgfpathlineto{\pgfqpoint{0.630666in}{4.300351in}}%
\pgfpathlineto{\pgfqpoint{0.631739in}{4.302601in}}%
\pgfpathlineto{\pgfqpoint{0.636035in}{4.302601in}}%
\pgfpathlineto{\pgfqpoint{0.638182in}{4.302450in}}%
\pgfpathlineto{\pgfqpoint{0.639256in}{4.303424in}}%
\pgfpathlineto{\pgfqpoint{0.644625in}{4.303777in}}%
\pgfpathlineto{\pgfqpoint{0.646773in}{4.306145in}}%
\pgfpathlineto{\pgfqpoint{0.649995in}{4.305658in}}%
\pgfpathlineto{\pgfqpoint{0.651068in}{4.304717in}}%
\pgfpathlineto{\pgfqpoint{0.652142in}{4.304801in}}%
\pgfpathlineto{\pgfqpoint{0.653216in}{4.303810in}}%
\pgfpathlineto{\pgfqpoint{0.654290in}{4.305003in}}%
\pgfpathlineto{\pgfqpoint{0.659659in}{4.304449in}}%
\pgfpathlineto{\pgfqpoint{0.660733in}{4.303508in}}%
\pgfpathlineto{\pgfqpoint{0.661807in}{4.304684in}}%
\pgfpathlineto{\pgfqpoint{0.667176in}{4.303491in}}%
\pgfpathlineto{\pgfqpoint{0.668250in}{4.305641in}}%
\pgfpathlineto{\pgfqpoint{0.669324in}{4.304986in}}%
\pgfpathlineto{\pgfqpoint{0.672545in}{4.301980in}}%
\pgfpathlineto{\pgfqpoint{0.673619in}{4.302685in}}%
\pgfpathlineto{\pgfqpoint{0.674693in}{4.302165in}}%
\pgfpathlineto{\pgfqpoint{0.676841in}{4.306615in}}%
\pgfpathlineto{\pgfqpoint{0.684357in}{4.305540in}}%
\pgfpathlineto{\pgfqpoint{0.687579in}{4.306279in}}%
\pgfpathlineto{\pgfqpoint{0.688653in}{4.305003in}}%
\pgfpathlineto{\pgfqpoint{0.689727in}{4.304667in}}%
\pgfpathlineto{\pgfqpoint{0.690800in}{4.304986in}}%
\pgfpathlineto{\pgfqpoint{0.691874in}{4.304432in}}%
\pgfpathlineto{\pgfqpoint{0.698317in}{4.307253in}}%
\pgfpathlineto{\pgfqpoint{0.699391in}{4.307992in}}%
\pgfpathlineto{\pgfqpoint{0.702613in}{4.308579in}}%
\pgfpathlineto{\pgfqpoint{0.704760in}{4.305641in}}%
\pgfpathlineto{\pgfqpoint{0.706908in}{4.304902in}}%
\pgfpathlineto{\pgfqpoint{0.710130in}{4.304952in}}%
\pgfpathlineto{\pgfqpoint{0.711203in}{4.306967in}}%
\pgfpathlineto{\pgfqpoint{0.712277in}{4.307790in}}%
\pgfpathlineto{\pgfqpoint{0.713351in}{4.307689in}}%
\pgfpathlineto{\pgfqpoint{0.714425in}{4.305154in}}%
\pgfpathlineto{\pgfqpoint{0.717646in}{4.304583in}}%
\pgfpathlineto{\pgfqpoint{0.718720in}{4.299260in}}%
\pgfpathlineto{\pgfqpoint{0.721942in}{4.298269in}}%
\pgfpathlineto{\pgfqpoint{0.727311in}{4.297665in}}%
\pgfpathlineto{\pgfqpoint{0.728385in}{4.299982in}}%
\pgfpathlineto{\pgfqpoint{0.729459in}{4.299596in}}%
\pgfpathlineto{\pgfqpoint{0.732680in}{4.300469in}}%
\pgfpathlineto{\pgfqpoint{0.733754in}{4.302249in}}%
\pgfpathlineto{\pgfqpoint{0.735902in}{4.298991in}}%
\pgfpathlineto{\pgfqpoint{0.736976in}{4.299361in}}%
\pgfpathlineto{\pgfqpoint{0.741271in}{4.299747in}}%
\pgfpathlineto{\pgfqpoint{0.742345in}{4.297262in}}%
\pgfpathlineto{\pgfqpoint{0.744492in}{4.298924in}}%
\pgfpathlineto{\pgfqpoint{0.748788in}{4.300486in}}%
\pgfpathlineto{\pgfqpoint{0.749862in}{4.300368in}}%
\pgfpathlineto{\pgfqpoint{0.752009in}{4.302266in}}%
\pgfpathlineto{\pgfqpoint{0.756305in}{4.302316in}}%
\pgfpathlineto{\pgfqpoint{0.757378in}{4.303189in}}%
\pgfpathlineto{\pgfqpoint{0.758452in}{4.302786in}}%
\pgfpathlineto{\pgfqpoint{0.759526in}{4.303223in}}%
\pgfpathlineto{\pgfqpoint{0.763821in}{4.302047in}}%
\pgfpathlineto{\pgfqpoint{0.765969in}{4.303223in}}%
\pgfpathlineto{\pgfqpoint{0.767043in}{4.304012in}}%
\pgfpathlineto{\pgfqpoint{0.770265in}{4.304516in}}%
\pgfpathlineto{\pgfqpoint{0.771338in}{4.307068in}}%
\pgfpathlineto{\pgfqpoint{0.774560in}{4.305087in}}%
\pgfpathlineto{\pgfqpoint{0.777781in}{4.306161in}}%
\pgfpathlineto{\pgfqpoint{0.778855in}{4.307320in}}%
\pgfpathlineto{\pgfqpoint{0.779929in}{4.306145in}}%
\pgfpathlineto{\pgfqpoint{0.781003in}{4.307706in}}%
\pgfpathlineto{\pgfqpoint{0.782077in}{4.306245in}}%
\pgfpathlineto{\pgfqpoint{0.787446in}{4.306212in}}%
\pgfpathlineto{\pgfqpoint{0.789594in}{4.304381in}}%
\pgfpathlineto{\pgfqpoint{0.792815in}{4.305893in}}%
\pgfpathlineto{\pgfqpoint{0.794963in}{4.308630in}}%
\pgfpathlineto{\pgfqpoint{0.796037in}{4.308479in}}%
\pgfpathlineto{\pgfqpoint{0.797110in}{4.309453in}}%
\pgfpathlineto{\pgfqpoint{0.801406in}{4.309637in}}%
\pgfpathlineto{\pgfqpoint{0.803554in}{4.311602in}}%
\pgfpathlineto{\pgfqpoint{0.804627in}{4.310746in}}%
\pgfpathlineto{\pgfqpoint{0.811070in}{4.313298in}}%
\pgfpathlineto{\pgfqpoint{0.812144in}{4.314222in}}%
\pgfpathlineto{\pgfqpoint{0.818587in}{4.315531in}}%
\pgfpathlineto{\pgfqpoint{0.819661in}{4.316841in}}%
\pgfpathlineto{\pgfqpoint{0.822883in}{4.316925in}}%
\pgfpathlineto{\pgfqpoint{0.823956in}{4.318554in}}%
\pgfpathlineto{\pgfqpoint{0.825030in}{4.317127in}}%
\pgfpathlineto{\pgfqpoint{0.826104in}{4.316774in}}%
\pgfpathlineto{\pgfqpoint{0.827178in}{4.318201in}}%
\pgfpathlineto{\pgfqpoint{0.830399in}{4.317093in}}%
\pgfpathlineto{\pgfqpoint{0.832547in}{4.319796in}}%
\pgfpathlineto{\pgfqpoint{0.833621in}{4.319142in}}%
\pgfpathlineto{\pgfqpoint{0.834695in}{4.319763in}}%
\pgfpathlineto{\pgfqpoint{0.837916in}{4.319712in}}%
\pgfpathlineto{\pgfqpoint{0.838990in}{4.320888in}}%
\pgfpathlineto{\pgfqpoint{0.842212in}{4.321476in}}%
\pgfpathlineto{\pgfqpoint{0.846507in}{4.322819in}}%
\pgfpathlineto{\pgfqpoint{0.848655in}{4.320737in}}%
\pgfpathlineto{\pgfqpoint{0.849729in}{4.321912in}}%
\pgfpathlineto{\pgfqpoint{0.852950in}{4.319360in}}%
\pgfpathlineto{\pgfqpoint{0.854024in}{4.320166in}}%
\pgfpathlineto{\pgfqpoint{0.855098in}{4.321946in}}%
\pgfpathlineto{\pgfqpoint{0.856172in}{4.322567in}}%
\pgfpathlineto{\pgfqpoint{0.860467in}{4.321543in}}%
\pgfpathlineto{\pgfqpoint{0.861541in}{4.323205in}}%
\pgfpathlineto{\pgfqpoint{0.863688in}{4.323323in}}%
\pgfpathlineto{\pgfqpoint{0.864762in}{4.325002in}}%
\pgfpathlineto{\pgfqpoint{0.867984in}{4.325136in}}%
\pgfpathlineto{\pgfqpoint{0.869058in}{4.324162in}}%
\pgfpathlineto{\pgfqpoint{0.870131in}{4.324112in}}%
\pgfpathlineto{\pgfqpoint{0.872279in}{4.325976in}}%
\pgfpathlineto{\pgfqpoint{0.876575in}{4.324246in}}%
\pgfpathlineto{\pgfqpoint{0.877648in}{4.324918in}}%
\pgfpathlineto{\pgfqpoint{0.878722in}{4.323894in}}%
\pgfpathlineto{\pgfqpoint{0.879796in}{4.326009in}}%
\pgfpathlineto{\pgfqpoint{0.883018in}{4.324230in}}%
\pgfpathlineto{\pgfqpoint{0.884091in}{4.325506in}}%
\pgfpathlineto{\pgfqpoint{0.885165in}{4.324397in}}%
\pgfpathlineto{\pgfqpoint{0.886239in}{4.325842in}}%
\pgfpathlineto{\pgfqpoint{0.890534in}{4.324901in}}%
\pgfpathlineto{\pgfqpoint{0.891608in}{4.326144in}}%
\pgfpathlineto{\pgfqpoint{0.892682in}{4.324952in}}%
\pgfpathlineto{\pgfqpoint{0.894830in}{4.325086in}}%
\pgfpathlineto{\pgfqpoint{0.899125in}{4.325338in}}%
\pgfpathlineto{\pgfqpoint{0.900199in}{4.327806in}}%
\pgfpathlineto{\pgfqpoint{0.901273in}{4.328612in}}%
\pgfpathlineto{\pgfqpoint{0.905568in}{4.325069in}}%
\pgfpathlineto{\pgfqpoint{0.906642in}{4.325623in}}%
\pgfpathlineto{\pgfqpoint{0.908790in}{4.323961in}}%
\pgfpathlineto{\pgfqpoint{0.909864in}{4.325002in}}%
\pgfpathlineto{\pgfqpoint{0.913085in}{4.325120in}}%
\pgfpathlineto{\pgfqpoint{0.914159in}{4.327386in}}%
\pgfpathlineto{\pgfqpoint{0.915233in}{4.328058in}}%
\pgfpathlineto{\pgfqpoint{0.916307in}{4.323810in}}%
\pgfpathlineto{\pgfqpoint{0.917380in}{4.322265in}}%
\pgfpathlineto{\pgfqpoint{0.920602in}{4.322315in}}%
\pgfpathlineto{\pgfqpoint{0.921676in}{4.323575in}}%
\pgfpathlineto{\pgfqpoint{0.922750in}{4.323340in}}%
\pgfpathlineto{\pgfqpoint{0.924897in}{4.328008in}}%
\pgfpathlineto{\pgfqpoint{0.930266in}{4.328528in}}%
\pgfpathlineto{\pgfqpoint{0.931340in}{4.330980in}}%
\pgfpathlineto{\pgfqpoint{0.932414in}{4.331769in}}%
\pgfpathlineto{\pgfqpoint{0.936709in}{4.331937in}}%
\pgfpathlineto{\pgfqpoint{0.937783in}{4.333230in}}%
\pgfpathlineto{\pgfqpoint{0.938857in}{4.332609in}}%
\pgfpathlineto{\pgfqpoint{0.939931in}{4.333062in}}%
\pgfpathlineto{\pgfqpoint{0.944226in}{4.333952in}}%
\pgfpathlineto{\pgfqpoint{0.946374in}{4.332575in}}%
\pgfpathlineto{\pgfqpoint{0.947448in}{4.332374in}}%
\pgfpathlineto{\pgfqpoint{0.951743in}{4.334254in}}%
\pgfpathlineto{\pgfqpoint{0.952817in}{4.333599in}}%
\pgfpathlineto{\pgfqpoint{0.953891in}{4.334019in}}%
\pgfpathlineto{\pgfqpoint{0.954965in}{4.332374in}}%
\pgfpathlineto{\pgfqpoint{0.958186in}{4.332861in}}%
\pgfpathlineto{\pgfqpoint{0.959260in}{4.332021in}}%
\pgfpathlineto{\pgfqpoint{0.960334in}{4.329972in}}%
\pgfpathlineto{\pgfqpoint{0.961408in}{4.330090in}}%
\pgfpathlineto{\pgfqpoint{0.962482in}{4.333566in}}%
\pgfpathlineto{\pgfqpoint{0.965703in}{4.333146in}}%
\pgfpathlineto{\pgfqpoint{0.966777in}{4.332306in}}%
\pgfpathlineto{\pgfqpoint{0.967851in}{4.330594in}}%
\pgfpathlineto{\pgfqpoint{0.968925in}{4.333700in}}%
\pgfpathlineto{\pgfqpoint{0.969998in}{4.333465in}}%
\pgfpathlineto{\pgfqpoint{0.973220in}{4.334725in}}%
\pgfpathlineto{\pgfqpoint{0.974294in}{4.336236in}}%
\pgfpathlineto{\pgfqpoint{0.975368in}{4.334238in}}%
\pgfpathlineto{\pgfqpoint{0.976442in}{4.330241in}}%
\pgfpathlineto{\pgfqpoint{0.977515in}{4.331400in}}%
\pgfpathlineto{\pgfqpoint{0.980737in}{4.328411in}}%
\pgfpathlineto{\pgfqpoint{0.981811in}{4.329469in}}%
\pgfpathlineto{\pgfqpoint{0.982885in}{4.331534in}}%
\pgfpathlineto{\pgfqpoint{0.983958in}{4.332323in}}%
\pgfpathlineto{\pgfqpoint{0.985032in}{4.331064in}}%
\pgfpathlineto{\pgfqpoint{0.988254in}{4.330997in}}%
\pgfpathlineto{\pgfqpoint{0.989328in}{4.330174in}}%
\pgfpathlineto{\pgfqpoint{0.990401in}{4.331198in}}%
\pgfpathlineto{\pgfqpoint{0.992549in}{4.334187in}}%
\pgfpathlineto{\pgfqpoint{0.995771in}{4.335027in}}%
\pgfpathlineto{\pgfqpoint{0.996844in}{4.336773in}}%
\pgfpathlineto{\pgfqpoint{0.997918in}{4.336891in}}%
\pgfpathlineto{\pgfqpoint{1.000066in}{4.339342in}}%
\pgfpathlineto{\pgfqpoint{1.003287in}{4.338923in}}%
\pgfpathlineto{\pgfqpoint{1.004361in}{4.338184in}}%
\pgfpathlineto{\pgfqpoint{1.005435in}{4.338620in}}%
\pgfpathlineto{\pgfqpoint{1.007583in}{4.340837in}}%
\pgfpathlineto{\pgfqpoint{1.010804in}{4.340971in}}%
\pgfpathlineto{\pgfqpoint{1.011878in}{4.341643in}}%
\pgfpathlineto{\pgfqpoint{1.012952in}{4.341038in}}%
\pgfpathlineto{\pgfqpoint{1.015100in}{4.341861in}}%
\pgfpathlineto{\pgfqpoint{1.019395in}{4.341760in}}%
\pgfpathlineto{\pgfqpoint{1.022617in}{4.343792in}}%
\pgfpathlineto{\pgfqpoint{1.027986in}{4.343188in}}%
\pgfpathlineto{\pgfqpoint{1.029060in}{4.344397in}}%
\pgfpathlineto{\pgfqpoint{1.030133in}{4.343876in}}%
\pgfpathlineto{\pgfqpoint{1.034429in}{4.344380in}}%
\pgfpathlineto{\pgfqpoint{1.035503in}{4.342885in}}%
\pgfpathlineto{\pgfqpoint{1.036576in}{4.340350in}}%
\pgfpathlineto{\pgfqpoint{1.037650in}{4.340417in}}%
\pgfpathlineto{\pgfqpoint{1.041946in}{4.339729in}}%
\pgfpathlineto{\pgfqpoint{1.043020in}{4.337781in}}%
\pgfpathlineto{\pgfqpoint{1.044093in}{4.339577in}}%
\pgfpathlineto{\pgfqpoint{1.045167in}{4.339174in}}%
\pgfpathlineto{\pgfqpoint{1.048389in}{4.339074in}}%
\pgfpathlineto{\pgfqpoint{1.049463in}{4.336773in}}%
\pgfpathlineto{\pgfqpoint{1.052684in}{4.337999in}}%
\pgfpathlineto{\pgfqpoint{1.056979in}{4.337478in}}%
\pgfpathlineto{\pgfqpoint{1.058053in}{4.339393in}}%
\pgfpathlineto{\pgfqpoint{1.060201in}{4.340098in}}%
\pgfpathlineto{\pgfqpoint{1.064496in}{4.343775in}}%
\pgfpathlineto{\pgfqpoint{1.065570in}{4.345371in}}%
\pgfpathlineto{\pgfqpoint{1.066644in}{4.344649in}}%
\pgfpathlineto{\pgfqpoint{1.067718in}{4.345220in}}%
\pgfpathlineto{\pgfqpoint{1.070939in}{4.346126in}}%
\pgfpathlineto{\pgfqpoint{1.072013in}{4.347134in}}%
\pgfpathlineto{\pgfqpoint{1.073087in}{4.349082in}}%
\pgfpathlineto{\pgfqpoint{1.074161in}{4.349485in}}%
\pgfpathlineto{\pgfqpoint{1.075235in}{4.347251in}}%
\pgfpathlineto{\pgfqpoint{1.078456in}{4.348830in}}%
\pgfpathlineto{\pgfqpoint{1.080604in}{4.347520in}}%
\pgfpathlineto{\pgfqpoint{1.081678in}{4.348175in}}%
\pgfpathlineto{\pgfqpoint{1.082752in}{4.347554in}}%
\pgfpathlineto{\pgfqpoint{1.085973in}{4.346378in}}%
\pgfpathlineto{\pgfqpoint{1.087047in}{4.346680in}}%
\pgfpathlineto{\pgfqpoint{1.089195in}{4.345152in}}%
\pgfpathlineto{\pgfqpoint{1.090268in}{4.346378in}}%
\pgfpathlineto{\pgfqpoint{1.093490in}{4.345354in}}%
\pgfpathlineto{\pgfqpoint{1.094564in}{4.343154in}}%
\pgfpathlineto{\pgfqpoint{1.095638in}{4.343742in}}%
\pgfpathlineto{\pgfqpoint{1.097785in}{4.348276in}}%
\pgfpathlineto{\pgfqpoint{1.101007in}{4.349250in}}%
\pgfpathlineto{\pgfqpoint{1.102081in}{4.346983in}}%
\pgfpathlineto{\pgfqpoint{1.105302in}{4.351315in}}%
\pgfpathlineto{\pgfqpoint{1.108524in}{4.351903in}}%
\pgfpathlineto{\pgfqpoint{1.109597in}{4.352692in}}%
\pgfpathlineto{\pgfqpoint{1.110671in}{4.351836in}}%
\pgfpathlineto{\pgfqpoint{1.111745in}{4.352255in}}%
\pgfpathlineto{\pgfqpoint{1.112819in}{4.353582in}}%
\pgfpathlineto{\pgfqpoint{1.117114in}{4.354825in}}%
\pgfpathlineto{\pgfqpoint{1.118188in}{4.354119in}}%
\pgfpathlineto{\pgfqpoint{1.119262in}{4.355647in}}%
\pgfpathlineto{\pgfqpoint{1.124631in}{4.356034in}}%
\pgfpathlineto{\pgfqpoint{1.125705in}{4.357478in}}%
\pgfpathlineto{\pgfqpoint{1.126779in}{4.356453in}}%
\pgfpathlineto{\pgfqpoint{1.127853in}{4.358720in}}%
\pgfpathlineto{\pgfqpoint{1.131074in}{4.358670in}}%
\pgfpathlineto{\pgfqpoint{1.133222in}{4.359577in}}%
\pgfpathlineto{\pgfqpoint{1.134296in}{4.361306in}}%
\pgfpathlineto{\pgfqpoint{1.140739in}{4.361340in}}%
\pgfpathlineto{\pgfqpoint{1.142886in}{4.364010in}}%
\pgfpathlineto{\pgfqpoint{1.146108in}{4.364312in}}%
\pgfpathlineto{\pgfqpoint{1.148256in}{4.367586in}}%
\pgfpathlineto{\pgfqpoint{1.150403in}{4.367603in}}%
\pgfpathlineto{\pgfqpoint{1.154699in}{4.357612in}}%
\pgfpathlineto{\pgfqpoint{1.155773in}{4.357411in}}%
\pgfpathlineto{\pgfqpoint{1.156846in}{4.357948in}}%
\pgfpathlineto{\pgfqpoint{1.157920in}{4.360517in}}%
\pgfpathlineto{\pgfqpoint{1.161142in}{4.360467in}}%
\pgfpathlineto{\pgfqpoint{1.163289in}{4.357864in}}%
\pgfpathlineto{\pgfqpoint{1.165437in}{4.357377in}}%
\pgfpathlineto{\pgfqpoint{1.168659in}{4.359140in}}%
\pgfpathlineto{\pgfqpoint{1.170806in}{4.370911in}}%
\pgfpathlineto{\pgfqpoint{1.172954in}{4.372238in}}%
\pgfpathlineto{\pgfqpoint{1.177249in}{4.372624in}}%
\pgfpathlineto{\pgfqpoint{1.180471in}{4.376033in}}%
\pgfpathlineto{\pgfqpoint{1.183692in}{4.376134in}}%
\pgfpathlineto{\pgfqpoint{1.184766in}{4.377343in}}%
\pgfpathlineto{\pgfqpoint{1.186914in}{4.374270in}}%
\pgfpathlineto{\pgfqpoint{1.187988in}{4.374740in}}%
\pgfpathlineto{\pgfqpoint{1.192283in}{4.373581in}}%
\pgfpathlineto{\pgfqpoint{1.193357in}{4.372104in}}%
\pgfpathlineto{\pgfqpoint{1.195505in}{4.371465in}}%
\pgfpathlineto{\pgfqpoint{1.198726in}{4.369299in}}%
\pgfpathlineto{\pgfqpoint{1.199800in}{4.373229in}}%
\pgfpathlineto{\pgfqpoint{1.200874in}{4.374723in}}%
\pgfpathlineto{\pgfqpoint{1.201948in}{4.374320in}}%
\pgfpathlineto{\pgfqpoint{1.203021in}{4.373094in}}%
\pgfpathlineto{\pgfqpoint{1.207317in}{4.372624in}}%
\pgfpathlineto{\pgfqpoint{1.208391in}{4.371885in}}%
\pgfpathlineto{\pgfqpoint{1.209464in}{4.369333in}}%
\pgfpathlineto{\pgfqpoint{1.210538in}{4.362851in}}%
\pgfpathlineto{\pgfqpoint{1.213760in}{4.361021in}}%
\pgfpathlineto{\pgfqpoint{1.215908in}{4.362885in}}%
\pgfpathlineto{\pgfqpoint{1.216981in}{4.359711in}}%
\pgfpathlineto{\pgfqpoint{1.218055in}{4.359913in}}%
\pgfpathlineto{\pgfqpoint{1.221277in}{4.353716in}}%
\pgfpathlineto{\pgfqpoint{1.222351in}{4.357797in}}%
\pgfpathlineto{\pgfqpoint{1.223424in}{4.358720in}}%
\pgfpathlineto{\pgfqpoint{1.225572in}{4.363002in}}%
\pgfpathlineto{\pgfqpoint{1.228794in}{4.362096in}}%
\pgfpathlineto{\pgfqpoint{1.229867in}{4.363943in}}%
\pgfpathlineto{\pgfqpoint{1.230941in}{4.364396in}}%
\pgfpathlineto{\pgfqpoint{1.232015in}{4.363959in}}%
\pgfpathlineto{\pgfqpoint{1.233089in}{4.366848in}}%
\pgfpathlineto{\pgfqpoint{1.237384in}{4.366377in}}%
\pgfpathlineto{\pgfqpoint{1.238458in}{4.364581in}}%
\pgfpathlineto{\pgfqpoint{1.239532in}{4.366025in}}%
\pgfpathlineto{\pgfqpoint{1.240606in}{4.366042in}}%
\pgfpathlineto{\pgfqpoint{1.243827in}{4.366965in}}%
\pgfpathlineto{\pgfqpoint{1.244901in}{4.368023in}}%
\pgfpathlineto{\pgfqpoint{1.245975in}{4.367922in}}%
\pgfpathlineto{\pgfqpoint{1.247049in}{4.370072in}}%
\pgfpathlineto{\pgfqpoint{1.248123in}{4.370643in}}%
\pgfpathlineto{\pgfqpoint{1.251344in}{4.366982in}}%
\pgfpathlineto{\pgfqpoint{1.252418in}{4.367654in}}%
\pgfpathlineto{\pgfqpoint{1.253492in}{4.369366in}}%
\pgfpathlineto{\pgfqpoint{1.255640in}{4.369736in}}%
\pgfpathlineto{\pgfqpoint{1.258861in}{4.368947in}}%
\pgfpathlineto{\pgfqpoint{1.259935in}{4.367419in}}%
\pgfpathlineto{\pgfqpoint{1.261009in}{4.367503in}}%
\pgfpathlineto{\pgfqpoint{1.263156in}{4.363523in}}%
\pgfpathlineto{\pgfqpoint{1.267452in}{4.367754in}}%
\pgfpathlineto{\pgfqpoint{1.268526in}{4.365555in}}%
\pgfpathlineto{\pgfqpoint{1.270673in}{4.368309in}}%
\pgfpathlineto{\pgfqpoint{1.273895in}{4.367284in}}%
\pgfpathlineto{\pgfqpoint{1.274969in}{4.369669in}}%
\pgfpathlineto{\pgfqpoint{1.276042in}{4.368258in}}%
\pgfpathlineto{\pgfqpoint{1.277116in}{4.367889in}}%
\pgfpathlineto{\pgfqpoint{1.278190in}{4.369870in}}%
\pgfpathlineto{\pgfqpoint{1.282485in}{4.373262in}}%
\pgfpathlineto{\pgfqpoint{1.283559in}{4.372490in}}%
\pgfpathlineto{\pgfqpoint{1.284633in}{4.372691in}}%
\pgfpathlineto{\pgfqpoint{1.285707in}{4.372288in}}%
\pgfpathlineto{\pgfqpoint{1.288929in}{4.370156in}}%
\pgfpathlineto{\pgfqpoint{1.290002in}{4.370811in}}%
\pgfpathlineto{\pgfqpoint{1.291076in}{4.372255in}}%
\pgfpathlineto{\pgfqpoint{1.293224in}{4.367234in}}%
\pgfpathlineto{\pgfqpoint{1.296445in}{4.368342in}}%
\pgfpathlineto{\pgfqpoint{1.297519in}{4.369719in}}%
\pgfpathlineto{\pgfqpoint{1.298593in}{4.373615in}}%
\pgfpathlineto{\pgfqpoint{1.299667in}{4.375009in}}%
\pgfpathlineto{\pgfqpoint{1.305036in}{4.376688in}}%
\pgfpathlineto{\pgfqpoint{1.308258in}{4.373296in}}%
\pgfpathlineto{\pgfqpoint{1.312553in}{4.374740in}}%
\pgfpathlineto{\pgfqpoint{1.314701in}{4.379492in}}%
\pgfpathlineto{\pgfqpoint{1.315774in}{4.378484in}}%
\pgfpathlineto{\pgfqpoint{1.318996in}{4.379190in}}%
\pgfpathlineto{\pgfqpoint{1.320070in}{4.377275in}}%
\pgfpathlineto{\pgfqpoint{1.321144in}{4.379962in}}%
\pgfpathlineto{\pgfqpoint{1.322218in}{4.379509in}}%
\pgfpathlineto{\pgfqpoint{1.326513in}{4.382414in}}%
\pgfpathlineto{\pgfqpoint{1.327587in}{4.381843in}}%
\pgfpathlineto{\pgfqpoint{1.329734in}{4.379727in}}%
\pgfpathlineto{\pgfqpoint{1.334030in}{4.380735in}}%
\pgfpathlineto{\pgfqpoint{1.335104in}{4.378636in}}%
\pgfpathlineto{\pgfqpoint{1.336177in}{4.380516in}}%
\pgfpathlineto{\pgfqpoint{1.337251in}{4.380013in}}%
\pgfpathlineto{\pgfqpoint{1.338325in}{4.381222in}}%
\pgfpathlineto{\pgfqpoint{1.343694in}{4.381641in}}%
\pgfpathlineto{\pgfqpoint{1.344768in}{4.383052in}}%
\pgfpathlineto{\pgfqpoint{1.345842in}{4.383287in}}%
\pgfpathlineto{\pgfqpoint{1.349063in}{4.382951in}}%
\pgfpathlineto{\pgfqpoint{1.350137in}{4.383791in}}%
\pgfpathlineto{\pgfqpoint{1.351211in}{4.382867in}}%
\pgfpathlineto{\pgfqpoint{1.353359in}{4.386343in}}%
\pgfpathlineto{\pgfqpoint{1.356580in}{4.387334in}}%
\pgfpathlineto{\pgfqpoint{1.358728in}{4.386007in}}%
\pgfpathlineto{\pgfqpoint{1.359802in}{4.384043in}}%
\pgfpathlineto{\pgfqpoint{1.360876in}{4.384479in}}%
\pgfpathlineto{\pgfqpoint{1.364097in}{4.384412in}}%
\pgfpathlineto{\pgfqpoint{1.368393in}{4.387099in}}%
\pgfpathlineto{\pgfqpoint{1.371614in}{4.385537in}}%
\pgfpathlineto{\pgfqpoint{1.372688in}{4.384076in}}%
\pgfpathlineto{\pgfqpoint{1.373762in}{4.384899in}}%
\pgfpathlineto{\pgfqpoint{1.375909in}{4.384949in}}%
\pgfpathlineto{\pgfqpoint{1.379131in}{4.384295in}}%
\pgfpathlineto{\pgfqpoint{1.381279in}{4.387216in}}%
\pgfpathlineto{\pgfqpoint{1.382352in}{4.387468in}}%
\pgfpathlineto{\pgfqpoint{1.388796in}{4.386410in}}%
\pgfpathlineto{\pgfqpoint{1.389869in}{4.385252in}}%
\pgfpathlineto{\pgfqpoint{1.390943in}{4.385856in}}%
\pgfpathlineto{\pgfqpoint{1.395239in}{4.386964in}}%
\pgfpathlineto{\pgfqpoint{1.396312in}{4.388576in}}%
\pgfpathlineto{\pgfqpoint{1.397386in}{4.384731in}}%
\pgfpathlineto{\pgfqpoint{1.398460in}{4.386629in}}%
\pgfpathlineto{\pgfqpoint{1.401682in}{4.385856in}}%
\pgfpathlineto{\pgfqpoint{1.402755in}{4.387048in}}%
\pgfpathlineto{\pgfqpoint{1.403829in}{4.386410in}}%
\pgfpathlineto{\pgfqpoint{1.404903in}{4.387065in}}%
\pgfpathlineto{\pgfqpoint{1.405977in}{4.387048in}}%
\pgfpathlineto{\pgfqpoint{1.409198in}{4.387653in}}%
\pgfpathlineto{\pgfqpoint{1.410272in}{4.385436in}}%
\pgfpathlineto{\pgfqpoint{1.411346in}{4.385033in}}%
\pgfpathlineto{\pgfqpoint{1.412420in}{4.380852in}}%
\pgfpathlineto{\pgfqpoint{1.413494in}{4.379710in}}%
\pgfpathlineto{\pgfqpoint{1.416715in}{4.380651in}}%
\pgfpathlineto{\pgfqpoint{1.417789in}{4.379240in}}%
\pgfpathlineto{\pgfqpoint{1.419937in}{4.378283in}}%
\pgfpathlineto{\pgfqpoint{1.421011in}{4.380802in}}%
\pgfpathlineto{\pgfqpoint{1.424232in}{4.380399in}}%
\pgfpathlineto{\pgfqpoint{1.425306in}{4.380835in}}%
\pgfpathlineto{\pgfqpoint{1.427454in}{4.382985in}}%
\pgfpathlineto{\pgfqpoint{1.428528in}{4.382363in}}%
\pgfpathlineto{\pgfqpoint{1.431749in}{4.385588in}}%
\pgfpathlineto{\pgfqpoint{1.432823in}{4.385772in}}%
\pgfpathlineto{\pgfqpoint{1.433897in}{4.387552in}}%
\pgfpathlineto{\pgfqpoint{1.436044in}{4.386864in}}%
\pgfpathlineto{\pgfqpoint{1.440340in}{4.387552in}}%
\pgfpathlineto{\pgfqpoint{1.441414in}{4.386578in}}%
\pgfpathlineto{\pgfqpoint{1.448930in}{4.386394in}}%
\pgfpathlineto{\pgfqpoint{1.450004in}{4.386091in}}%
\pgfpathlineto{\pgfqpoint{1.451078in}{4.386964in}}%
\pgfpathlineto{\pgfqpoint{1.454300in}{4.387972in}}%
\pgfpathlineto{\pgfqpoint{1.455373in}{4.387351in}}%
\pgfpathlineto{\pgfqpoint{1.456447in}{4.387603in}}%
\pgfpathlineto{\pgfqpoint{1.458595in}{4.386578in}}%
\pgfpathlineto{\pgfqpoint{1.462890in}{4.387922in}}%
\pgfpathlineto{\pgfqpoint{1.463964in}{4.388677in}}%
\pgfpathlineto{\pgfqpoint{1.465038in}{4.390843in}}%
\pgfpathlineto{\pgfqpoint{1.466112in}{4.390625in}}%
\pgfpathlineto{\pgfqpoint{1.469333in}{4.389181in}}%
\pgfpathlineto{\pgfqpoint{1.470407in}{4.387216in}}%
\pgfpathlineto{\pgfqpoint{1.471481in}{4.387955in}}%
\pgfpathlineto{\pgfqpoint{1.472555in}{4.384429in}}%
\pgfpathlineto{\pgfqpoint{1.476850in}{4.384009in}}%
\pgfpathlineto{\pgfqpoint{1.477924in}{4.383253in}}%
\pgfpathlineto{\pgfqpoint{1.478998in}{4.379576in}}%
\pgfpathlineto{\pgfqpoint{1.480072in}{4.378820in}}%
\pgfpathlineto{\pgfqpoint{1.481146in}{4.381020in}}%
\pgfpathlineto{\pgfqpoint{1.484367in}{4.381272in}}%
\pgfpathlineto{\pgfqpoint{1.485441in}{4.377359in}}%
\pgfpathlineto{\pgfqpoint{1.486515in}{4.382850in}}%
\pgfpathlineto{\pgfqpoint{1.487589in}{4.378770in}}%
\pgfpathlineto{\pgfqpoint{1.488662in}{4.371701in}}%
\pgfpathlineto{\pgfqpoint{1.491884in}{4.370324in}}%
\pgfpathlineto{\pgfqpoint{1.492958in}{4.372204in}}%
\pgfpathlineto{\pgfqpoint{1.494032in}{4.372271in}}%
\pgfpathlineto{\pgfqpoint{1.495106in}{4.373497in}}%
\pgfpathlineto{\pgfqpoint{1.496179in}{4.376956in}}%
\pgfpathlineto{\pgfqpoint{1.499401in}{4.377242in}}%
\pgfpathlineto{\pgfqpoint{1.500475in}{4.382145in}}%
\pgfpathlineto{\pgfqpoint{1.501549in}{4.379223in}}%
\pgfpathlineto{\pgfqpoint{1.503696in}{4.393429in}}%
\pgfpathlineto{\pgfqpoint{1.506918in}{4.394857in}}%
\pgfpathlineto{\pgfqpoint{1.507992in}{4.397056in}}%
\pgfpathlineto{\pgfqpoint{1.509065in}{4.396989in}}%
\pgfpathlineto{\pgfqpoint{1.511213in}{4.401053in}}%
\pgfpathlineto{\pgfqpoint{1.514435in}{4.400348in}}%
\pgfpathlineto{\pgfqpoint{1.515508in}{4.402934in}}%
\pgfpathlineto{\pgfqpoint{1.518730in}{4.405133in}}%
\pgfpathlineto{\pgfqpoint{1.521951in}{4.406779in}}%
\pgfpathlineto{\pgfqpoint{1.523025in}{4.405956in}}%
\pgfpathlineto{\pgfqpoint{1.526247in}{4.408525in}}%
\pgfpathlineto{\pgfqpoint{1.529468in}{4.408290in}}%
\pgfpathlineto{\pgfqpoint{1.530542in}{4.410440in}}%
\pgfpathlineto{\pgfqpoint{1.531616in}{4.409751in}}%
\pgfpathlineto{\pgfqpoint{1.533764in}{4.411716in}}%
\pgfpathlineto{\pgfqpoint{1.536985in}{4.411296in}}%
\pgfpathlineto{\pgfqpoint{1.538059in}{4.408609in}}%
\pgfpathlineto{\pgfqpoint{1.539133in}{4.408979in}}%
\pgfpathlineto{\pgfqpoint{1.541281in}{4.411615in}}%
\pgfpathlineto{\pgfqpoint{1.544502in}{4.408760in}}%
\pgfpathlineto{\pgfqpoint{1.546650in}{4.414822in}}%
\pgfpathlineto{\pgfqpoint{1.548797in}{4.414839in}}%
\pgfpathlineto{\pgfqpoint{1.553093in}{4.412707in}}%
\pgfpathlineto{\pgfqpoint{1.554167in}{4.408878in}}%
\pgfpathlineto{\pgfqpoint{1.555240in}{4.410221in}}%
\pgfpathlineto{\pgfqpoint{1.556314in}{4.407216in}}%
\pgfpathlineto{\pgfqpoint{1.559536in}{4.406813in}}%
\pgfpathlineto{\pgfqpoint{1.561684in}{4.412371in}}%
\pgfpathlineto{\pgfqpoint{1.562757in}{4.419323in}}%
\pgfpathlineto{\pgfqpoint{1.563831in}{4.419591in}}%
\pgfpathlineto{\pgfqpoint{1.567053in}{4.422244in}}%
\pgfpathlineto{\pgfqpoint{1.568127in}{4.421657in}}%
\pgfpathlineto{\pgfqpoint{1.569200in}{4.421774in}}%
\pgfpathlineto{\pgfqpoint{1.571348in}{4.420750in}}%
\pgfpathlineto{\pgfqpoint{1.574570in}{4.421405in}}%
\pgfpathlineto{\pgfqpoint{1.575643in}{4.420129in}}%
\pgfpathlineto{\pgfqpoint{1.576717in}{4.417879in}}%
\pgfpathlineto{\pgfqpoint{1.578865in}{4.417492in}}%
\pgfpathlineto{\pgfqpoint{1.583160in}{4.409483in}}%
\pgfpathlineto{\pgfqpoint{1.584234in}{4.411179in}}%
\pgfpathlineto{\pgfqpoint{1.585308in}{4.416854in}}%
\pgfpathlineto{\pgfqpoint{1.586382in}{4.413882in}}%
\pgfpathlineto{\pgfqpoint{1.592825in}{4.410977in}}%
\pgfpathlineto{\pgfqpoint{1.593899in}{4.414436in}}%
\pgfpathlineto{\pgfqpoint{1.598194in}{4.414352in}}%
\pgfpathlineto{\pgfqpoint{1.599268in}{4.415242in}}%
\pgfpathlineto{\pgfqpoint{1.600342in}{4.420196in}}%
\pgfpathlineto{\pgfqpoint{1.601416in}{4.417425in}}%
\pgfpathlineto{\pgfqpoint{1.604637in}{4.417761in}}%
\pgfpathlineto{\pgfqpoint{1.605711in}{4.416854in}}%
\pgfpathlineto{\pgfqpoint{1.606785in}{4.417308in}}%
\pgfpathlineto{\pgfqpoint{1.607859in}{4.420548in}}%
\pgfpathlineto{\pgfqpoint{1.608932in}{4.414890in}}%
\pgfpathlineto{\pgfqpoint{1.612154in}{4.418030in}}%
\pgfpathlineto{\pgfqpoint{1.613228in}{4.420280in}}%
\pgfpathlineto{\pgfqpoint{1.614302in}{4.418601in}}%
\pgfpathlineto{\pgfqpoint{1.615375in}{4.421170in}}%
\pgfpathlineto{\pgfqpoint{1.619671in}{4.418651in}}%
\pgfpathlineto{\pgfqpoint{1.620745in}{4.420095in}}%
\pgfpathlineto{\pgfqpoint{1.621818in}{4.419625in}}%
\pgfpathlineto{\pgfqpoint{1.622892in}{4.421757in}}%
\pgfpathlineto{\pgfqpoint{1.623966in}{4.421808in}}%
\pgfpathlineto{\pgfqpoint{1.630409in}{4.423504in}}%
\pgfpathlineto{\pgfqpoint{1.631483in}{4.425049in}}%
\pgfpathlineto{\pgfqpoint{1.637926in}{4.427265in}}%
\pgfpathlineto{\pgfqpoint{1.639000in}{4.425838in}}%
\pgfpathlineto{\pgfqpoint{1.642221in}{4.428592in}}%
\pgfpathlineto{\pgfqpoint{1.644369in}{4.423621in}}%
\pgfpathlineto{\pgfqpoint{1.645443in}{4.424226in}}%
\pgfpathlineto{\pgfqpoint{1.646517in}{4.419440in}}%
\pgfpathlineto{\pgfqpoint{1.649738in}{4.422446in}}%
\pgfpathlineto{\pgfqpoint{1.650812in}{4.416317in}}%
\pgfpathlineto{\pgfqpoint{1.651886in}{4.415528in}}%
\pgfpathlineto{\pgfqpoint{1.652960in}{4.419608in}}%
\pgfpathlineto{\pgfqpoint{1.654034in}{4.417039in}}%
\pgfpathlineto{\pgfqpoint{1.657255in}{4.422194in}}%
\pgfpathlineto{\pgfqpoint{1.658329in}{4.419255in}}%
\pgfpathlineto{\pgfqpoint{1.659403in}{4.422597in}}%
\pgfpathlineto{\pgfqpoint{1.660477in}{4.421388in}}%
\pgfpathlineto{\pgfqpoint{1.661550in}{4.422614in}}%
\pgfpathlineto{\pgfqpoint{1.665846in}{4.422362in}}%
\pgfpathlineto{\pgfqpoint{1.666920in}{4.416972in}}%
\pgfpathlineto{\pgfqpoint{1.667994in}{4.416804in}}%
\pgfpathlineto{\pgfqpoint{1.672289in}{4.421959in}}%
\pgfpathlineto{\pgfqpoint{1.673363in}{4.420330in}}%
\pgfpathlineto{\pgfqpoint{1.674437in}{4.416703in}}%
\pgfpathlineto{\pgfqpoint{1.675510in}{4.417123in}}%
\pgfpathlineto{\pgfqpoint{1.680880in}{4.422194in}}%
\pgfpathlineto{\pgfqpoint{1.681953in}{4.422295in}}%
\pgfpathlineto{\pgfqpoint{1.684101in}{4.423487in}}%
\pgfpathlineto{\pgfqpoint{1.687323in}{4.421657in}}%
\pgfpathlineto{\pgfqpoint{1.689470in}{4.422547in}}%
\pgfpathlineto{\pgfqpoint{1.690544in}{4.421724in}}%
\pgfpathlineto{\pgfqpoint{1.691618in}{4.415494in}}%
\pgfpathlineto{\pgfqpoint{1.694839in}{4.419675in}}%
\pgfpathlineto{\pgfqpoint{1.695913in}{4.418903in}}%
\pgfpathlineto{\pgfqpoint{1.696987in}{4.419910in}}%
\pgfpathlineto{\pgfqpoint{1.698061in}{4.412455in}}%
\pgfpathlineto{\pgfqpoint{1.699135in}{4.411447in}}%
\pgfpathlineto{\pgfqpoint{1.702356in}{4.409953in}}%
\pgfpathlineto{\pgfqpoint{1.703430in}{4.410473in}}%
\pgfpathlineto{\pgfqpoint{1.705578in}{4.407585in}}%
\pgfpathlineto{\pgfqpoint{1.706652in}{4.409499in}}%
\pgfpathlineto{\pgfqpoint{1.709873in}{4.411498in}}%
\pgfpathlineto{\pgfqpoint{1.710947in}{4.409835in}}%
\pgfpathlineto{\pgfqpoint{1.712021in}{4.409432in}}%
\pgfpathlineto{\pgfqpoint{1.713095in}{4.410876in}}%
\pgfpathlineto{\pgfqpoint{1.714169in}{4.413848in}}%
\pgfpathlineto{\pgfqpoint{1.717390in}{4.412807in}}%
\pgfpathlineto{\pgfqpoint{1.718464in}{4.413059in}}%
\pgfpathlineto{\pgfqpoint{1.721685in}{4.417862in}}%
\pgfpathlineto{\pgfqpoint{1.724907in}{4.417224in}}%
\pgfpathlineto{\pgfqpoint{1.725981in}{4.417593in}}%
\pgfpathlineto{\pgfqpoint{1.728128in}{4.417543in}}%
\pgfpathlineto{\pgfqpoint{1.729202in}{4.415931in}}%
\pgfpathlineto{\pgfqpoint{1.733498in}{4.413848in}}%
\pgfpathlineto{\pgfqpoint{1.734572in}{4.415981in}}%
\pgfpathlineto{\pgfqpoint{1.735645in}{4.415763in}}%
\pgfpathlineto{\pgfqpoint{1.736719in}{4.413076in}}%
\pgfpathlineto{\pgfqpoint{1.741015in}{4.413076in}}%
\pgfpathlineto{\pgfqpoint{1.742088in}{4.414705in}}%
\pgfpathlineto{\pgfqpoint{1.744236in}{4.410087in}}%
\pgfpathlineto{\pgfqpoint{1.747458in}{4.409348in}}%
\pgfpathlineto{\pgfqpoint{1.748531in}{4.409936in}}%
\pgfpathlineto{\pgfqpoint{1.749605in}{4.413026in}}%
\pgfpathlineto{\pgfqpoint{1.750679in}{4.414251in}}%
\pgfpathlineto{\pgfqpoint{1.751753in}{4.411531in}}%
\pgfpathlineto{\pgfqpoint{1.754974in}{4.408274in}}%
\pgfpathlineto{\pgfqpoint{1.757122in}{4.409886in}}%
\pgfpathlineto{\pgfqpoint{1.758196in}{4.413899in}}%
\pgfpathlineto{\pgfqpoint{1.759270in}{4.412891in}}%
\pgfpathlineto{\pgfqpoint{1.763565in}{4.414218in}}%
\pgfpathlineto{\pgfqpoint{1.765713in}{4.408341in}}%
\pgfpathlineto{\pgfqpoint{1.766787in}{4.410104in}}%
\pgfpathlineto{\pgfqpoint{1.770008in}{4.405402in}}%
\pgfpathlineto{\pgfqpoint{1.771082in}{4.405923in}}%
\pgfpathlineto{\pgfqpoint{1.772156in}{4.407938in}}%
\pgfpathlineto{\pgfqpoint{1.773230in}{4.407535in}}%
\pgfpathlineto{\pgfqpoint{1.777525in}{4.407081in}}%
\pgfpathlineto{\pgfqpoint{1.778599in}{4.407619in}}%
\pgfpathlineto{\pgfqpoint{1.779673in}{4.403538in}}%
\pgfpathlineto{\pgfqpoint{1.781820in}{4.406897in}}%
\pgfpathlineto{\pgfqpoint{1.786116in}{4.409634in}}%
\pgfpathlineto{\pgfqpoint{1.787190in}{4.408475in}}%
\pgfpathlineto{\pgfqpoint{1.788263in}{4.410221in}}%
\pgfpathlineto{\pgfqpoint{1.789337in}{4.409499in}}%
\pgfpathlineto{\pgfqpoint{1.792559in}{4.410137in}}%
\pgfpathlineto{\pgfqpoint{1.793633in}{4.408106in}}%
\pgfpathlineto{\pgfqpoint{1.794706in}{4.407585in}}%
\pgfpathlineto{\pgfqpoint{1.795780in}{4.398719in}}%
\pgfpathlineto{\pgfqpoint{1.800076in}{4.397543in}}%
\pgfpathlineto{\pgfqpoint{1.801149in}{4.401137in}}%
\pgfpathlineto{\pgfqpoint{1.803297in}{4.401825in}}%
\pgfpathlineto{\pgfqpoint{1.804371in}{4.401490in}}%
\pgfpathlineto{\pgfqpoint{1.807593in}{4.399659in}}%
\pgfpathlineto{\pgfqpoint{1.809740in}{4.400986in}}%
\pgfpathlineto{\pgfqpoint{1.810814in}{4.398366in}}%
\pgfpathlineto{\pgfqpoint{1.811888in}{4.397812in}}%
\pgfpathlineto{\pgfqpoint{1.815109in}{4.401338in}}%
\pgfpathlineto{\pgfqpoint{1.816183in}{4.397191in}}%
\pgfpathlineto{\pgfqpoint{1.817257in}{4.397258in}}%
\pgfpathlineto{\pgfqpoint{1.818331in}{4.395579in}}%
\pgfpathlineto{\pgfqpoint{1.819405in}{4.396905in}}%
\pgfpathlineto{\pgfqpoint{1.822626in}{4.398333in}}%
\pgfpathlineto{\pgfqpoint{1.824774in}{4.394806in}}%
\pgfpathlineto{\pgfqpoint{1.826922in}{4.389097in}}%
\pgfpathlineto{\pgfqpoint{1.831217in}{4.382397in}}%
\pgfpathlineto{\pgfqpoint{1.832291in}{4.390054in}}%
\pgfpathlineto{\pgfqpoint{1.833365in}{4.391834in}}%
\pgfpathlineto{\pgfqpoint{1.834438in}{4.392304in}}%
\pgfpathlineto{\pgfqpoint{1.837660in}{4.389181in}}%
\pgfpathlineto{\pgfqpoint{1.838734in}{4.383690in}}%
\pgfpathlineto{\pgfqpoint{1.839808in}{4.387821in}}%
\pgfpathlineto{\pgfqpoint{1.840882in}{4.388576in}}%
\pgfpathlineto{\pgfqpoint{1.841955in}{4.385705in}}%
\pgfpathlineto{\pgfqpoint{1.846251in}{4.391129in}}%
\pgfpathlineto{\pgfqpoint{1.847325in}{4.387250in}}%
\pgfpathlineto{\pgfqpoint{1.848398in}{4.387132in}}%
\pgfpathlineto{\pgfqpoint{1.849472in}{4.387854in}}%
\pgfpathlineto{\pgfqpoint{1.852694in}{4.387166in}}%
\pgfpathlineto{\pgfqpoint{1.853768in}{4.391381in}}%
\pgfpathlineto{\pgfqpoint{1.854841in}{4.392288in}}%
\pgfpathlineto{\pgfqpoint{1.855915in}{4.390407in}}%
\pgfpathlineto{\pgfqpoint{1.856989in}{4.385369in}}%
\pgfpathlineto{\pgfqpoint{1.860211in}{4.386007in}}%
\pgfpathlineto{\pgfqpoint{1.861284in}{4.382968in}}%
\pgfpathlineto{\pgfqpoint{1.863432in}{4.382296in}}%
\pgfpathlineto{\pgfqpoint{1.864506in}{4.385285in}}%
\pgfpathlineto{\pgfqpoint{1.867727in}{4.383505in}}%
\pgfpathlineto{\pgfqpoint{1.868801in}{4.388291in}}%
\pgfpathlineto{\pgfqpoint{1.869875in}{4.388627in}}%
\pgfpathlineto{\pgfqpoint{1.870949in}{4.387166in}}%
\pgfpathlineto{\pgfqpoint{1.872023in}{4.390776in}}%
\pgfpathlineto{\pgfqpoint{1.875244in}{4.395495in}}%
\pgfpathlineto{\pgfqpoint{1.876318in}{4.394689in}}%
\pgfpathlineto{\pgfqpoint{1.878466in}{4.400247in}}%
\pgfpathlineto{\pgfqpoint{1.879540in}{4.400885in}}%
\pgfpathlineto{\pgfqpoint{1.882761in}{4.401137in}}%
\pgfpathlineto{\pgfqpoint{1.884909in}{4.398501in}}%
\pgfpathlineto{\pgfqpoint{1.885983in}{4.399810in}}%
\pgfpathlineto{\pgfqpoint{1.887057in}{4.399071in}}%
\pgfpathlineto{\pgfqpoint{1.890278in}{4.398030in}}%
\pgfpathlineto{\pgfqpoint{1.892426in}{4.400767in}}%
\pgfpathlineto{\pgfqpoint{1.893500in}{4.410087in}}%
\pgfpathlineto{\pgfqpoint{1.894573in}{4.409785in}}%
\pgfpathlineto{\pgfqpoint{1.898869in}{4.411195in}}%
\pgfpathlineto{\pgfqpoint{1.899943in}{4.413227in}}%
\pgfpathlineto{\pgfqpoint{1.902090in}{4.411917in}}%
\pgfpathlineto{\pgfqpoint{1.905312in}{4.416015in}}%
\pgfpathlineto{\pgfqpoint{1.906386in}{4.414251in}}%
\pgfpathlineto{\pgfqpoint{1.909607in}{4.415007in}}%
\pgfpathlineto{\pgfqpoint{1.912829in}{4.412304in}}%
\pgfpathlineto{\pgfqpoint{1.913903in}{4.412488in}}%
\pgfpathlineto{\pgfqpoint{1.914976in}{4.414688in}}%
\pgfpathlineto{\pgfqpoint{1.916050in}{4.410540in}}%
\pgfpathlineto{\pgfqpoint{1.917124in}{4.409566in}}%
\pgfpathlineto{\pgfqpoint{1.920346in}{4.413429in}}%
\pgfpathlineto{\pgfqpoint{1.921419in}{4.411649in}}%
\pgfpathlineto{\pgfqpoint{1.923567in}{4.415259in}}%
\pgfpathlineto{\pgfqpoint{1.924641in}{4.416115in}}%
\pgfpathlineto{\pgfqpoint{1.927862in}{4.415662in}}%
\pgfpathlineto{\pgfqpoint{1.928936in}{4.414302in}}%
\pgfpathlineto{\pgfqpoint{1.930010in}{4.414117in}}%
\pgfpathlineto{\pgfqpoint{1.932158in}{4.414654in}}%
\pgfpathlineto{\pgfqpoint{1.935379in}{4.412522in}}%
\pgfpathlineto{\pgfqpoint{1.936453in}{4.413009in}}%
\pgfpathlineto{\pgfqpoint{1.938601in}{4.409180in}}%
\pgfpathlineto{\pgfqpoint{1.939675in}{4.415007in}}%
\pgfpathlineto{\pgfqpoint{1.942896in}{4.414302in}}%
\pgfpathlineto{\pgfqpoint{1.945044in}{4.411716in}}%
\pgfpathlineto{\pgfqpoint{1.946118in}{4.413949in}}%
\pgfpathlineto{\pgfqpoint{1.947192in}{4.409869in}}%
\pgfpathlineto{\pgfqpoint{1.950413in}{4.414117in}}%
\pgfpathlineto{\pgfqpoint{1.951487in}{4.399693in}}%
\pgfpathlineto{\pgfqpoint{1.952561in}{4.402447in}}%
\pgfpathlineto{\pgfqpoint{1.953635in}{4.400784in}}%
\pgfpathlineto{\pgfqpoint{1.954708in}{4.397846in}}%
\pgfpathlineto{\pgfqpoint{1.957930in}{4.398702in}}%
\pgfpathlineto{\pgfqpoint{1.961151in}{4.404327in}}%
\pgfpathlineto{\pgfqpoint{1.965447in}{4.404378in}}%
\pgfpathlineto{\pgfqpoint{1.966521in}{4.406913in}}%
\pgfpathlineto{\pgfqpoint{1.972964in}{4.397695in}}%
\pgfpathlineto{\pgfqpoint{1.974038in}{4.398668in}}%
\pgfpathlineto{\pgfqpoint{1.976185in}{4.388812in}}%
\pgfpathlineto{\pgfqpoint{1.977259in}{4.388090in}}%
\pgfpathlineto{\pgfqpoint{1.980481in}{4.388039in}}%
\pgfpathlineto{\pgfqpoint{1.981554in}{4.388644in}}%
\pgfpathlineto{\pgfqpoint{1.982628in}{4.385386in}}%
\pgfpathlineto{\pgfqpoint{1.983702in}{4.389131in}}%
\pgfpathlineto{\pgfqpoint{1.984776in}{4.385352in}}%
\pgfpathlineto{\pgfqpoint{1.989071in}{4.384899in}}%
\pgfpathlineto{\pgfqpoint{1.990145in}{4.382716in}}%
\pgfpathlineto{\pgfqpoint{1.991219in}{4.383942in}}%
\pgfpathlineto{\pgfqpoint{1.992293in}{4.386612in}}%
\pgfpathlineto{\pgfqpoint{1.995514in}{4.383640in}}%
\pgfpathlineto{\pgfqpoint{1.996588in}{4.394605in}}%
\pgfpathlineto{\pgfqpoint{1.997662in}{4.395763in}}%
\pgfpathlineto{\pgfqpoint{1.998736in}{4.398450in}}%
\pgfpathlineto{\pgfqpoint{1.999810in}{4.404042in}}%
\pgfpathlineto{\pgfqpoint{2.004105in}{4.399290in}}%
\pgfpathlineto{\pgfqpoint{2.005179in}{4.406359in}}%
\pgfpathlineto{\pgfqpoint{2.006253in}{4.407753in}}%
\pgfpathlineto{\pgfqpoint{2.010548in}{4.408458in}}%
\pgfpathlineto{\pgfqpoint{2.011622in}{4.409734in}}%
\pgfpathlineto{\pgfqpoint{2.013770in}{4.405201in}}%
\pgfpathlineto{\pgfqpoint{2.014843in}{4.410238in}}%
\pgfpathlineto{\pgfqpoint{2.019139in}{4.412639in}}%
\pgfpathlineto{\pgfqpoint{2.020213in}{4.414302in}}%
\pgfpathlineto{\pgfqpoint{2.021286in}{4.414554in}}%
\pgfpathlineto{\pgfqpoint{2.022360in}{4.414033in}}%
\pgfpathlineto{\pgfqpoint{2.025582in}{4.415947in}}%
\pgfpathlineto{\pgfqpoint{2.026656in}{4.413613in}}%
\pgfpathlineto{\pgfqpoint{2.027729in}{4.415225in}}%
\pgfpathlineto{\pgfqpoint{2.028803in}{4.417929in}}%
\pgfpathlineto{\pgfqpoint{2.029877in}{4.416821in}}%
\pgfpathlineto{\pgfqpoint{2.033099in}{4.414688in}}%
\pgfpathlineto{\pgfqpoint{2.034172in}{4.418852in}}%
\pgfpathlineto{\pgfqpoint{2.036320in}{4.418517in}}%
\pgfpathlineto{\pgfqpoint{2.037394in}{4.419575in}}%
\pgfpathlineto{\pgfqpoint{2.040615in}{4.420347in}}%
\pgfpathlineto{\pgfqpoint{2.042763in}{4.419608in}}%
\pgfpathlineto{\pgfqpoint{2.043837in}{4.419306in}}%
\pgfpathlineto{\pgfqpoint{2.044911in}{4.422345in}}%
\pgfpathlineto{\pgfqpoint{2.048132in}{4.422244in}}%
\pgfpathlineto{\pgfqpoint{2.051354in}{4.425217in}}%
\pgfpathlineto{\pgfqpoint{2.052428in}{4.427618in}}%
\pgfpathlineto{\pgfqpoint{2.058871in}{4.426291in}}%
\pgfpathlineto{\pgfqpoint{2.063166in}{4.429079in}}%
\pgfpathlineto{\pgfqpoint{2.064240in}{4.426191in}}%
\pgfpathlineto{\pgfqpoint{2.065314in}{4.429801in}}%
\pgfpathlineto{\pgfqpoint{2.066388in}{4.429616in}}%
\pgfpathlineto{\pgfqpoint{2.067461in}{4.430993in}}%
\pgfpathlineto{\pgfqpoint{2.071757in}{4.428357in}}%
\pgfpathlineto{\pgfqpoint{2.072831in}{4.429885in}}%
\pgfpathlineto{\pgfqpoint{2.073904in}{4.430422in}}%
\pgfpathlineto{\pgfqpoint{2.074978in}{4.429616in}}%
\pgfpathlineto{\pgfqpoint{2.078200in}{4.429549in}}%
\pgfpathlineto{\pgfqpoint{2.079274in}{4.431749in}}%
\pgfpathlineto{\pgfqpoint{2.080348in}{4.432605in}}%
\pgfpathlineto{\pgfqpoint{2.081421in}{4.431950in}}%
\pgfpathlineto{\pgfqpoint{2.082495in}{4.432907in}}%
\pgfpathlineto{\pgfqpoint{2.086791in}{4.434234in}}%
\pgfpathlineto{\pgfqpoint{2.088938in}{4.432723in}}%
\pgfpathlineto{\pgfqpoint{2.093234in}{4.432286in}}%
\pgfpathlineto{\pgfqpoint{2.094307in}{4.428928in}}%
\pgfpathlineto{\pgfqpoint{2.095381in}{4.431346in}}%
\pgfpathlineto{\pgfqpoint{2.096455in}{4.430187in}}%
\pgfpathlineto{\pgfqpoint{2.100750in}{4.432236in}}%
\pgfpathlineto{\pgfqpoint{2.101824in}{4.431665in}}%
\pgfpathlineto{\pgfqpoint{2.102898in}{4.430422in}}%
\pgfpathlineto{\pgfqpoint{2.103972in}{4.431363in}}%
\pgfpathlineto{\pgfqpoint{2.105046in}{4.433075in}}%
\pgfpathlineto{\pgfqpoint{2.108267in}{4.432504in}}%
\pgfpathlineto{\pgfqpoint{2.109341in}{4.435174in}}%
\pgfpathlineto{\pgfqpoint{2.110415in}{4.434419in}}%
\pgfpathlineto{\pgfqpoint{2.111489in}{4.435006in}}%
\pgfpathlineto{\pgfqpoint{2.112563in}{4.432202in}}%
\pgfpathlineto{\pgfqpoint{2.115784in}{4.434083in}}%
\pgfpathlineto{\pgfqpoint{2.116858in}{4.431413in}}%
\pgfpathlineto{\pgfqpoint{2.117932in}{4.431598in}}%
\pgfpathlineto{\pgfqpoint{2.119006in}{4.428978in}}%
\pgfpathlineto{\pgfqpoint{2.120080in}{4.428810in}}%
\pgfpathlineto{\pgfqpoint{2.123301in}{4.430506in}}%
\pgfpathlineto{\pgfqpoint{2.125449in}{4.436484in}}%
\pgfpathlineto{\pgfqpoint{2.126523in}{4.434822in}}%
\pgfpathlineto{\pgfqpoint{2.127596in}{4.434788in}}%
\pgfpathlineto{\pgfqpoint{2.131892in}{4.433915in}}%
\pgfpathlineto{\pgfqpoint{2.132966in}{4.434486in}}%
\pgfpathlineto{\pgfqpoint{2.134039in}{4.433478in}}%
\pgfpathlineto{\pgfqpoint{2.135113in}{4.433999in}}%
\pgfpathlineto{\pgfqpoint{2.140482in}{4.438684in}}%
\pgfpathlineto{\pgfqpoint{2.142630in}{4.434284in}}%
\pgfpathlineto{\pgfqpoint{2.145852in}{4.432085in}}%
\pgfpathlineto{\pgfqpoint{2.147999in}{4.433109in}}%
\pgfpathlineto{\pgfqpoint{2.149073in}{4.436400in}}%
\pgfpathlineto{\pgfqpoint{2.150147in}{4.434872in}}%
\pgfpathlineto{\pgfqpoint{2.153369in}{4.438398in}}%
\pgfpathlineto{\pgfqpoint{2.155516in}{4.438398in}}%
\pgfpathlineto{\pgfqpoint{2.156590in}{4.442831in}}%
\pgfpathlineto{\pgfqpoint{2.157664in}{4.435141in}}%
\pgfpathlineto{\pgfqpoint{2.160885in}{4.432169in}}%
\pgfpathlineto{\pgfqpoint{2.165181in}{4.445031in}}%
\pgfpathlineto{\pgfqpoint{2.169476in}{4.445166in}}%
\pgfpathlineto{\pgfqpoint{2.171624in}{4.444007in}}%
\pgfpathlineto{\pgfqpoint{2.172698in}{4.447466in}}%
\pgfpathlineto{\pgfqpoint{2.175919in}{4.448843in}}%
\pgfpathlineto{\pgfqpoint{2.176993in}{4.450606in}}%
\pgfpathlineto{\pgfqpoint{2.178067in}{4.450690in}}%
\pgfpathlineto{\pgfqpoint{2.179141in}{4.453293in}}%
\pgfpathlineto{\pgfqpoint{2.180214in}{4.454065in}}%
\pgfpathlineto{\pgfqpoint{2.184510in}{4.453813in}}%
\pgfpathlineto{\pgfqpoint{2.185584in}{4.454082in}}%
\pgfpathlineto{\pgfqpoint{2.186658in}{4.452302in}}%
\pgfpathlineto{\pgfqpoint{2.187731in}{4.452588in}}%
\pgfpathlineto{\pgfqpoint{2.190953in}{4.451328in}}%
\pgfpathlineto{\pgfqpoint{2.192027in}{4.448306in}}%
\pgfpathlineto{\pgfqpoint{2.193101in}{4.449246in}}%
\pgfpathlineto{\pgfqpoint{2.194174in}{4.448793in}}%
\pgfpathlineto{\pgfqpoint{2.195248in}{4.449380in}}%
\pgfpathlineto{\pgfqpoint{2.200617in}{4.449414in}}%
\pgfpathlineto{\pgfqpoint{2.201691in}{4.448524in}}%
\pgfpathlineto{\pgfqpoint{2.202765in}{4.449699in}}%
\pgfpathlineto{\pgfqpoint{2.208134in}{4.450086in}}%
\pgfpathlineto{\pgfqpoint{2.209208in}{4.453444in}}%
\pgfpathlineto{\pgfqpoint{2.210282in}{4.452319in}}%
\pgfpathlineto{\pgfqpoint{2.213503in}{4.452772in}}%
\pgfpathlineto{\pgfqpoint{2.214577in}{4.450740in}}%
\pgfpathlineto{\pgfqpoint{2.215651in}{4.453427in}}%
\pgfpathlineto{\pgfqpoint{2.216725in}{4.452353in}}%
\pgfpathlineto{\pgfqpoint{2.217799in}{4.453024in}}%
\pgfpathlineto{\pgfqpoint{2.221020in}{4.452185in}}%
\pgfpathlineto{\pgfqpoint{2.222094in}{4.453276in}}%
\pgfpathlineto{\pgfqpoint{2.223168in}{4.452722in}}%
\pgfpathlineto{\pgfqpoint{2.229611in}{4.453864in}}%
\pgfpathlineto{\pgfqpoint{2.230685in}{4.452453in}}%
\pgfpathlineto{\pgfqpoint{2.232833in}{4.454888in}}%
\pgfpathlineto{\pgfqpoint{2.237128in}{4.454334in}}%
\pgfpathlineto{\pgfqpoint{2.238202in}{4.453242in}}%
\pgfpathlineto{\pgfqpoint{2.239276in}{4.453780in}}%
\pgfpathlineto{\pgfqpoint{2.240349in}{4.446878in}}%
\pgfpathlineto{\pgfqpoint{2.243571in}{4.450690in}}%
\pgfpathlineto{\pgfqpoint{2.244645in}{4.447533in}}%
\pgfpathlineto{\pgfqpoint{2.245719in}{4.446845in}}%
\pgfpathlineto{\pgfqpoint{2.246792in}{4.448339in}}%
\pgfpathlineto{\pgfqpoint{2.247866in}{4.445972in}}%
\pgfpathlineto{\pgfqpoint{2.252162in}{4.449918in}}%
\pgfpathlineto{\pgfqpoint{2.253236in}{4.452991in}}%
\pgfpathlineto{\pgfqpoint{2.254309in}{4.453394in}}%
\pgfpathlineto{\pgfqpoint{2.255383in}{4.449582in}}%
\pgfpathlineto{\pgfqpoint{2.258605in}{4.447348in}}%
\pgfpathlineto{\pgfqpoint{2.259679in}{4.447903in}}%
\pgfpathlineto{\pgfqpoint{2.260752in}{4.449901in}}%
\pgfpathlineto{\pgfqpoint{2.261826in}{4.446459in}}%
\pgfpathlineto{\pgfqpoint{2.262900in}{4.447785in}}%
\pgfpathlineto{\pgfqpoint{2.266122in}{4.445938in}}%
\pgfpathlineto{\pgfqpoint{2.267195in}{4.440732in}}%
\pgfpathlineto{\pgfqpoint{2.268269in}{4.441841in}}%
\pgfpathlineto{\pgfqpoint{2.270417in}{4.440195in}}%
\pgfpathlineto{\pgfqpoint{2.273638in}{4.439893in}}%
\pgfpathlineto{\pgfqpoint{2.274712in}{4.437643in}}%
\pgfpathlineto{\pgfqpoint{2.276860in}{4.438163in}}%
\pgfpathlineto{\pgfqpoint{2.277934in}{4.438650in}}%
\pgfpathlineto{\pgfqpoint{2.284377in}{4.437911in}}%
\pgfpathlineto{\pgfqpoint{2.285451in}{4.437357in}}%
\pgfpathlineto{\pgfqpoint{2.288672in}{4.440094in}}%
\pgfpathlineto{\pgfqpoint{2.289746in}{4.432286in}}%
\pgfpathlineto{\pgfqpoint{2.290820in}{4.432723in}}%
\pgfpathlineto{\pgfqpoint{2.291894in}{4.431564in}}%
\pgfpathlineto{\pgfqpoint{2.292968in}{4.431581in}}%
\pgfpathlineto{\pgfqpoint{2.296189in}{4.430842in}}%
\pgfpathlineto{\pgfqpoint{2.297263in}{4.429230in}}%
\pgfpathlineto{\pgfqpoint{2.299411in}{4.433226in}}%
\pgfpathlineto{\pgfqpoint{2.300484in}{4.432672in}}%
\pgfpathlineto{\pgfqpoint{2.304780in}{4.439725in}}%
\pgfpathlineto{\pgfqpoint{2.305854in}{4.438734in}}%
\pgfpathlineto{\pgfqpoint{2.306927in}{4.444763in}}%
\pgfpathlineto{\pgfqpoint{2.308001in}{4.446005in}}%
\pgfpathlineto{\pgfqpoint{2.311223in}{4.442647in}}%
\pgfpathlineto{\pgfqpoint{2.312297in}{4.444695in}}%
\pgfpathlineto{\pgfqpoint{2.313370in}{4.442949in}}%
\pgfpathlineto{\pgfqpoint{2.314444in}{4.444141in}}%
\pgfpathlineto{\pgfqpoint{2.315518in}{4.444443in}}%
\pgfpathlineto{\pgfqpoint{2.318740in}{4.442193in}}%
\pgfpathlineto{\pgfqpoint{2.320887in}{4.443335in}}%
\pgfpathlineto{\pgfqpoint{2.323035in}{4.445266in}}%
\pgfpathlineto{\pgfqpoint{2.326257in}{4.443721in}}%
\pgfpathlineto{\pgfqpoint{2.327330in}{4.444141in}}%
\pgfpathlineto{\pgfqpoint{2.328404in}{4.442529in}}%
\pgfpathlineto{\pgfqpoint{2.329478in}{4.443923in}}%
\pgfpathlineto{\pgfqpoint{2.334847in}{4.442630in}}%
\pgfpathlineto{\pgfqpoint{2.335921in}{4.449263in}}%
\pgfpathlineto{\pgfqpoint{2.336995in}{4.448994in}}%
\pgfpathlineto{\pgfqpoint{2.338069in}{4.453058in}}%
\pgfpathlineto{\pgfqpoint{2.341290in}{4.454855in}}%
\pgfpathlineto{\pgfqpoint{2.342364in}{4.453595in}}%
\pgfpathlineto{\pgfqpoint{2.343438in}{4.450119in}}%
\pgfpathlineto{\pgfqpoint{2.344512in}{4.449212in}}%
\pgfpathlineto{\pgfqpoint{2.345586in}{4.451446in}}%
\pgfpathlineto{\pgfqpoint{2.353102in}{4.453461in}}%
\pgfpathlineto{\pgfqpoint{2.357398in}{4.453729in}}%
\pgfpathlineto{\pgfqpoint{2.358472in}{4.452420in}}%
\pgfpathlineto{\pgfqpoint{2.360619in}{4.453192in}}%
\pgfpathlineto{\pgfqpoint{2.364915in}{4.452369in}}%
\pgfpathlineto{\pgfqpoint{2.365989in}{4.452806in}}%
\pgfpathlineto{\pgfqpoint{2.367062in}{4.451849in}}%
\pgfpathlineto{\pgfqpoint{2.368136in}{4.452655in}}%
\pgfpathlineto{\pgfqpoint{2.372432in}{4.450086in}}%
\pgfpathlineto{\pgfqpoint{2.373505in}{4.452134in}}%
\pgfpathlineto{\pgfqpoint{2.375653in}{4.451345in}}%
\pgfpathlineto{\pgfqpoint{2.379948in}{4.451143in}}%
\pgfpathlineto{\pgfqpoint{2.381022in}{4.453058in}}%
\pgfpathlineto{\pgfqpoint{2.382096in}{4.453360in}}%
\pgfpathlineto{\pgfqpoint{2.383170in}{4.453058in}}%
\pgfpathlineto{\pgfqpoint{2.386391in}{4.453091in}}%
\pgfpathlineto{\pgfqpoint{2.387465in}{4.449128in}}%
\pgfpathlineto{\pgfqpoint{2.388539in}{4.450321in}}%
\pgfpathlineto{\pgfqpoint{2.389613in}{4.450455in}}%
\pgfpathlineto{\pgfqpoint{2.390687in}{4.451479in}}%
\pgfpathlineto{\pgfqpoint{2.394982in}{4.447332in}}%
\pgfpathlineto{\pgfqpoint{2.396056in}{4.447886in}}%
\pgfpathlineto{\pgfqpoint{2.397130in}{4.446341in}}%
\pgfpathlineto{\pgfqpoint{2.398204in}{4.447684in}}%
\pgfpathlineto{\pgfqpoint{2.401425in}{4.447768in}}%
\pgfpathlineto{\pgfqpoint{2.402499in}{4.448809in}}%
\pgfpathlineto{\pgfqpoint{2.405721in}{4.453864in}}%
\pgfpathlineto{\pgfqpoint{2.410016in}{4.457911in}}%
\pgfpathlineto{\pgfqpoint{2.412164in}{4.462596in}}%
\pgfpathlineto{\pgfqpoint{2.413237in}{4.461874in}}%
\pgfpathlineto{\pgfqpoint{2.417533in}{4.462528in}}%
\pgfpathlineto{\pgfqpoint{2.418607in}{4.466726in}}%
\pgfpathlineto{\pgfqpoint{2.419680in}{4.468540in}}%
\pgfpathlineto{\pgfqpoint{2.420754in}{4.468876in}}%
\pgfpathlineto{\pgfqpoint{2.423976in}{4.468087in}}%
\pgfpathlineto{\pgfqpoint{2.425050in}{4.467213in}}%
\pgfpathlineto{\pgfqpoint{2.426124in}{4.472721in}}%
\pgfpathlineto{\pgfqpoint{2.427197in}{4.472772in}}%
\pgfpathlineto{\pgfqpoint{2.428271in}{4.471865in}}%
\pgfpathlineto{\pgfqpoint{2.432567in}{4.471512in}}%
\pgfpathlineto{\pgfqpoint{2.434714in}{4.472788in}}%
\pgfpathlineto{\pgfqpoint{2.435788in}{4.474837in}}%
\pgfpathlineto{\pgfqpoint{2.439010in}{4.475324in}}%
\pgfpathlineto{\pgfqpoint{2.440083in}{4.473460in}}%
\pgfpathlineto{\pgfqpoint{2.441157in}{4.474820in}}%
\pgfpathlineto{\pgfqpoint{2.442231in}{4.473426in}}%
\pgfpathlineto{\pgfqpoint{2.443305in}{4.476651in}}%
\pgfpathlineto{\pgfqpoint{2.446526in}{4.477675in}}%
\pgfpathlineto{\pgfqpoint{2.447600in}{4.476281in}}%
\pgfpathlineto{\pgfqpoint{2.449748in}{4.476298in}}%
\pgfpathlineto{\pgfqpoint{2.450822in}{4.475307in}}%
\pgfpathlineto{\pgfqpoint{2.454043in}{4.473611in}}%
\pgfpathlineto{\pgfqpoint{2.455117in}{4.474501in}}%
\pgfpathlineto{\pgfqpoint{2.456191in}{4.474014in}}%
\pgfpathlineto{\pgfqpoint{2.457265in}{4.474955in}}%
\pgfpathlineto{\pgfqpoint{2.458339in}{4.475038in}}%
\pgfpathlineto{\pgfqpoint{2.465856in}{4.472923in}}%
\pgfpathlineto{\pgfqpoint{2.469077in}{4.472486in}}%
\pgfpathlineto{\pgfqpoint{2.470151in}{4.473057in}}%
\pgfpathlineto{\pgfqpoint{2.471225in}{4.472469in}}%
\pgfpathlineto{\pgfqpoint{2.472299in}{4.470824in}}%
\pgfpathlineto{\pgfqpoint{2.476594in}{4.473510in}}%
\pgfpathlineto{\pgfqpoint{2.478742in}{4.472671in}}%
\pgfpathlineto{\pgfqpoint{2.479815in}{4.474770in}}%
\pgfpathlineto{\pgfqpoint{2.480889in}{4.475290in}}%
\pgfpathlineto{\pgfqpoint{2.485185in}{4.481000in}}%
\pgfpathlineto{\pgfqpoint{2.486258in}{4.480798in}}%
\pgfpathlineto{\pgfqpoint{2.487332in}{4.482528in}}%
\pgfpathlineto{\pgfqpoint{2.491628in}{4.480496in}}%
\pgfpathlineto{\pgfqpoint{2.494849in}{4.487868in}}%
\pgfpathlineto{\pgfqpoint{2.495923in}{4.487700in}}%
\pgfpathlineto{\pgfqpoint{2.499145in}{4.486524in}}%
\pgfpathlineto{\pgfqpoint{2.500218in}{4.485433in}}%
\pgfpathlineto{\pgfqpoint{2.501292in}{4.483367in}}%
\pgfpathlineto{\pgfqpoint{2.503440in}{4.483250in}}%
\pgfpathlineto{\pgfqpoint{2.507735in}{4.485366in}}%
\pgfpathlineto{\pgfqpoint{2.508809in}{4.482360in}}%
\pgfpathlineto{\pgfqpoint{2.510957in}{4.483888in}}%
\pgfpathlineto{\pgfqpoint{2.514178in}{4.488170in}}%
\pgfpathlineto{\pgfqpoint{2.515252in}{4.486927in}}%
\pgfpathlineto{\pgfqpoint{2.516326in}{4.486558in}}%
\pgfpathlineto{\pgfqpoint{2.518474in}{4.491579in}}%
\pgfpathlineto{\pgfqpoint{2.522769in}{4.494366in}}%
\pgfpathlineto{\pgfqpoint{2.523843in}{4.497573in}}%
\pgfpathlineto{\pgfqpoint{2.524917in}{4.497389in}}%
\pgfpathlineto{\pgfqpoint{2.525991in}{4.501100in}}%
\pgfpathlineto{\pgfqpoint{2.529212in}{4.500344in}}%
\pgfpathlineto{\pgfqpoint{2.531360in}{4.498430in}}%
\pgfpathlineto{\pgfqpoint{2.533507in}{4.501452in}}%
\pgfpathlineto{\pgfqpoint{2.536729in}{4.502191in}}%
\pgfpathlineto{\pgfqpoint{2.538877in}{4.506322in}}%
\pgfpathlineto{\pgfqpoint{2.541024in}{4.511427in}}%
\pgfpathlineto{\pgfqpoint{2.545320in}{4.511611in}}%
\pgfpathlineto{\pgfqpoint{2.547467in}{4.509815in}}%
\pgfpathlineto{\pgfqpoint{2.548541in}{4.510889in}}%
\pgfpathlineto{\pgfqpoint{2.551763in}{4.510419in}}%
\pgfpathlineto{\pgfqpoint{2.552836in}{4.505953in}}%
\pgfpathlineto{\pgfqpoint{2.553910in}{4.507279in}}%
\pgfpathlineto{\pgfqpoint{2.554984in}{4.502913in}}%
\pgfpathlineto{\pgfqpoint{2.556058in}{4.503451in}}%
\pgfpathlineto{\pgfqpoint{2.559279in}{4.506037in}}%
\pgfpathlineto{\pgfqpoint{2.561427in}{4.505919in}}%
\pgfpathlineto{\pgfqpoint{2.562501in}{4.503182in}}%
\pgfpathlineto{\pgfqpoint{2.563575in}{4.505650in}}%
\pgfpathlineto{\pgfqpoint{2.566796in}{4.507078in}}%
\pgfpathlineto{\pgfqpoint{2.567870in}{4.505768in}}%
\pgfpathlineto{\pgfqpoint{2.568944in}{4.508354in}}%
\pgfpathlineto{\pgfqpoint{2.570018in}{4.508035in}}%
\pgfpathlineto{\pgfqpoint{2.571092in}{4.509109in}}%
\pgfpathlineto{\pgfqpoint{2.575387in}{4.508371in}}%
\pgfpathlineto{\pgfqpoint{2.577535in}{4.510167in}}%
\pgfpathlineto{\pgfqpoint{2.578609in}{4.508136in}}%
\pgfpathlineto{\pgfqpoint{2.581830in}{4.506305in}}%
\pgfpathlineto{\pgfqpoint{2.582904in}{4.489547in}}%
\pgfpathlineto{\pgfqpoint{2.583978in}{4.488976in}}%
\pgfpathlineto{\pgfqpoint{2.585052in}{4.490588in}}%
\pgfpathlineto{\pgfqpoint{2.586125in}{4.490067in}}%
\pgfpathlineto{\pgfqpoint{2.589347in}{4.492368in}}%
\pgfpathlineto{\pgfqpoint{2.592568in}{4.502544in}}%
\pgfpathlineto{\pgfqpoint{2.596864in}{4.502258in}}%
\pgfpathlineto{\pgfqpoint{2.597938in}{4.500663in}}%
\pgfpathlineto{\pgfqpoint{2.600085in}{4.500361in}}%
\pgfpathlineto{\pgfqpoint{2.601159in}{4.499958in}}%
\pgfpathlineto{\pgfqpoint{2.604381in}{4.502158in}}%
\pgfpathlineto{\pgfqpoint{2.605455in}{4.501855in}}%
\pgfpathlineto{\pgfqpoint{2.606528in}{4.503014in}}%
\pgfpathlineto{\pgfqpoint{2.608676in}{4.496096in}}%
\pgfpathlineto{\pgfqpoint{2.611898in}{4.497708in}}%
\pgfpathlineto{\pgfqpoint{2.612971in}{4.499135in}}%
\pgfpathlineto{\pgfqpoint{2.614045in}{4.496566in}}%
\pgfpathlineto{\pgfqpoint{2.615119in}{4.495693in}}%
\pgfpathlineto{\pgfqpoint{2.619414in}{4.496247in}}%
\pgfpathlineto{\pgfqpoint{2.621562in}{4.498195in}}%
\pgfpathlineto{\pgfqpoint{2.622636in}{4.499185in}}%
\pgfpathlineto{\pgfqpoint{2.628005in}{4.493896in}}%
\pgfpathlineto{\pgfqpoint{2.629079in}{4.495592in}}%
\pgfpathlineto{\pgfqpoint{2.631227in}{4.501368in}}%
\pgfpathlineto{\pgfqpoint{2.634448in}{4.507514in}}%
\pgfpathlineto{\pgfqpoint{2.636596in}{4.507397in}}%
\pgfpathlineto{\pgfqpoint{2.638744in}{4.513526in}}%
\pgfpathlineto{\pgfqpoint{2.643039in}{4.513862in}}%
\pgfpathlineto{\pgfqpoint{2.644113in}{4.509026in}}%
\pgfpathlineto{\pgfqpoint{2.645187in}{4.508908in}}%
\pgfpathlineto{\pgfqpoint{2.646260in}{4.509429in}}%
\pgfpathlineto{\pgfqpoint{2.650556in}{4.509848in}}%
\pgfpathlineto{\pgfqpoint{2.651630in}{4.507380in}}%
\pgfpathlineto{\pgfqpoint{2.653777in}{4.508052in}}%
\pgfpathlineto{\pgfqpoint{2.656999in}{4.512602in}}%
\pgfpathlineto{\pgfqpoint{2.659146in}{4.518563in}}%
\pgfpathlineto{\pgfqpoint{2.661294in}{4.518563in}}%
\pgfpathlineto{\pgfqpoint{2.666663in}{4.518547in}}%
\pgfpathlineto{\pgfqpoint{2.667737in}{4.520259in}}%
\pgfpathlineto{\pgfqpoint{2.668811in}{4.520478in}}%
\pgfpathlineto{\pgfqpoint{2.672033in}{4.522056in}}%
\pgfpathlineto{\pgfqpoint{2.673106in}{4.520528in}}%
\pgfpathlineto{\pgfqpoint{2.674180in}{4.521351in}}%
\pgfpathlineto{\pgfqpoint{2.675254in}{4.522896in}}%
\pgfpathlineto{\pgfqpoint{2.676328in}{4.526187in}}%
\pgfpathlineto{\pgfqpoint{2.679549in}{4.526556in}}%
\pgfpathlineto{\pgfqpoint{2.680623in}{4.547362in}}%
\pgfpathlineto{\pgfqpoint{2.681697in}{4.552181in}}%
\pgfpathlineto{\pgfqpoint{2.682771in}{4.544641in}}%
\pgfpathlineto{\pgfqpoint{2.683845in}{4.547513in}}%
\pgfpathlineto{\pgfqpoint{2.688140in}{4.540275in}}%
\pgfpathlineto{\pgfqpoint{2.689214in}{4.540259in}}%
\pgfpathlineto{\pgfqpoint{2.690288in}{4.543516in}}%
\pgfpathlineto{\pgfqpoint{2.691362in}{4.543499in}}%
\pgfpathlineto{\pgfqpoint{2.695657in}{4.540057in}}%
\pgfpathlineto{\pgfqpoint{2.696731in}{4.539704in}}%
\pgfpathlineto{\pgfqpoint{2.698879in}{4.535926in}}%
\pgfpathlineto{\pgfqpoint{2.702100in}{4.537152in}}%
\pgfpathlineto{\pgfqpoint{2.703174in}{4.538915in}}%
\pgfpathlineto{\pgfqpoint{2.704248in}{4.535842in}}%
\pgfpathlineto{\pgfqpoint{2.705322in}{4.539050in}}%
\pgfpathlineto{\pgfqpoint{2.706395in}{4.538966in}}%
\pgfpathlineto{\pgfqpoint{2.709617in}{4.542341in}}%
\pgfpathlineto{\pgfqpoint{2.710691in}{4.546472in}}%
\pgfpathlineto{\pgfqpoint{2.711765in}{4.544339in}}%
\pgfpathlineto{\pgfqpoint{2.713912in}{4.544020in}}%
\pgfpathlineto{\pgfqpoint{2.717134in}{4.548201in}}%
\pgfpathlineto{\pgfqpoint{2.719281in}{4.554599in}}%
\pgfpathlineto{\pgfqpoint{2.720355in}{4.562793in}}%
\pgfpathlineto{\pgfqpoint{2.721429in}{4.559620in}}%
\pgfpathlineto{\pgfqpoint{2.725724in}{4.555002in}}%
\pgfpathlineto{\pgfqpoint{2.726798in}{4.555657in}}%
\pgfpathlineto{\pgfqpoint{2.727872in}{4.558260in}}%
\pgfpathlineto{\pgfqpoint{2.728946in}{4.554800in}}%
\pgfpathlineto{\pgfqpoint{2.732167in}{4.556664in}}%
\pgfpathlineto{\pgfqpoint{2.733241in}{4.552315in}}%
\pgfpathlineto{\pgfqpoint{2.734315in}{4.556362in}}%
\pgfpathlineto{\pgfqpoint{2.735389in}{4.554717in}}%
\pgfpathlineto{\pgfqpoint{2.736463in}{4.554582in}}%
\pgfpathlineto{\pgfqpoint{2.740758in}{4.555204in}}%
\pgfpathlineto{\pgfqpoint{2.743980in}{4.549377in}}%
\pgfpathlineto{\pgfqpoint{2.748275in}{4.550519in}}%
\pgfpathlineto{\pgfqpoint{2.749349in}{4.551711in}}%
\pgfpathlineto{\pgfqpoint{2.751497in}{4.550384in}}%
\pgfpathlineto{\pgfqpoint{2.751497in}{4.550384in}}%
\pgfusepath{stroke}%
\end{pgfscope}%
\begin{pgfscope}%
\pgfpathrectangle{\pgfqpoint{0.285588in}{4.180131in}}{\pgfqpoint{2.583333in}{0.400885in}}%
\pgfusepath{clip}%
\pgfsetroundcap%
\pgfsetroundjoin%
\pgfsetlinewidth{1.505625pt}%
\definecolor{currentstroke}{rgb}{0.121569,0.466667,0.705882}%
\pgfsetstrokecolor{currentstroke}%
\pgfsetdash{}{0pt}%
\pgfpathmoveto{\pgfqpoint{0.403012in}{4.289571in}}%
\pgfpathlineto{\pgfqpoint{0.405159in}{4.288102in}}%
\pgfpathlineto{\pgfqpoint{0.406233in}{4.287524in}}%
\pgfpathlineto{\pgfqpoint{0.410529in}{4.286276in}}%
\pgfpathlineto{\pgfqpoint{0.413750in}{4.283993in}}%
\pgfpathlineto{\pgfqpoint{0.418045in}{4.283163in}}%
\pgfpathlineto{\pgfqpoint{0.420193in}{4.281155in}}%
\pgfpathlineto{\pgfqpoint{0.424489in}{4.281397in}}%
\pgfpathlineto{\pgfqpoint{0.426636in}{4.280322in}}%
\pgfpathlineto{\pgfqpoint{0.427710in}{4.278966in}}%
\pgfpathlineto{\pgfqpoint{0.433079in}{4.280016in}}%
\pgfpathlineto{\pgfqpoint{0.434153in}{4.279231in}}%
\pgfpathlineto{\pgfqpoint{0.442744in}{4.278430in}}%
\pgfpathlineto{\pgfqpoint{0.443818in}{4.279466in}}%
\pgfpathlineto{\pgfqpoint{0.448113in}{4.278453in}}%
\pgfpathlineto{\pgfqpoint{0.449187in}{4.278913in}}%
\pgfpathlineto{\pgfqpoint{0.450261in}{4.278144in}}%
\pgfpathlineto{\pgfqpoint{0.455630in}{4.278201in}}%
\pgfpathlineto{\pgfqpoint{0.458851in}{4.277502in}}%
\pgfpathlineto{\pgfqpoint{0.466368in}{4.278297in}}%
\pgfpathlineto{\pgfqpoint{0.469590in}{4.278843in}}%
\pgfpathlineto{\pgfqpoint{0.470664in}{4.276319in}}%
\pgfpathlineto{\pgfqpoint{0.471737in}{4.276943in}}%
\pgfpathlineto{\pgfqpoint{0.472811in}{4.275464in}}%
\pgfpathlineto{\pgfqpoint{0.477107in}{4.274494in}}%
\pgfpathlineto{\pgfqpoint{0.478180in}{4.273112in}}%
\pgfpathlineto{\pgfqpoint{0.479254in}{4.272991in}}%
\pgfpathlineto{\pgfqpoint{0.480328in}{4.271731in}}%
\pgfpathlineto{\pgfqpoint{0.481402in}{4.272214in}}%
\pgfpathlineto{\pgfqpoint{0.485697in}{4.272437in}}%
\pgfpathlineto{\pgfqpoint{0.488919in}{4.273411in}}%
\pgfpathlineto{\pgfqpoint{0.493214in}{4.272653in}}%
\pgfpathlineto{\pgfqpoint{0.494288in}{4.273413in}}%
\pgfpathlineto{\pgfqpoint{0.496436in}{4.272577in}}%
\pgfpathlineto{\pgfqpoint{0.500731in}{4.273029in}}%
\pgfpathlineto{\pgfqpoint{0.502879in}{4.274735in}}%
\pgfpathlineto{\pgfqpoint{0.507174in}{4.273665in}}%
\pgfpathlineto{\pgfqpoint{0.508248in}{4.271664in}}%
\pgfpathlineto{\pgfqpoint{0.509322in}{4.272539in}}%
\pgfpathlineto{\pgfqpoint{0.511469in}{4.269501in}}%
\pgfpathlineto{\pgfqpoint{0.514691in}{4.270292in}}%
\pgfpathlineto{\pgfqpoint{0.515765in}{4.269139in}}%
\pgfpathlineto{\pgfqpoint{0.517912in}{4.269835in}}%
\pgfpathlineto{\pgfqpoint{0.518986in}{4.269091in}}%
\pgfpathlineto{\pgfqpoint{0.522208in}{4.269471in}}%
\pgfpathlineto{\pgfqpoint{0.523282in}{4.267994in}}%
\pgfpathlineto{\pgfqpoint{0.526503in}{4.267078in}}%
\pgfpathlineto{\pgfqpoint{0.532946in}{4.267052in}}%
\pgfpathlineto{\pgfqpoint{0.534020in}{4.267791in}}%
\pgfpathlineto{\pgfqpoint{0.541537in}{4.268992in}}%
\pgfpathlineto{\pgfqpoint{0.545832in}{4.267964in}}%
\pgfpathlineto{\pgfqpoint{0.546906in}{4.268136in}}%
\pgfpathlineto{\pgfqpoint{0.549054in}{4.266119in}}%
\pgfpathlineto{\pgfqpoint{0.552275in}{4.267160in}}%
\pgfpathlineto{\pgfqpoint{0.553349in}{4.266791in}}%
\pgfpathlineto{\pgfqpoint{0.555497in}{4.267727in}}%
\pgfpathlineto{\pgfqpoint{0.556571in}{4.267503in}}%
\pgfpathlineto{\pgfqpoint{0.560866in}{4.268544in}}%
\pgfpathlineto{\pgfqpoint{0.561940in}{4.267147in}}%
\pgfpathlineto{\pgfqpoint{0.563014in}{4.267108in}}%
\pgfpathlineto{\pgfqpoint{0.564088in}{4.265394in}}%
\pgfpathlineto{\pgfqpoint{0.568383in}{4.265025in}}%
\pgfpathlineto{\pgfqpoint{0.569457in}{4.267345in}}%
\pgfpathlineto{\pgfqpoint{0.571604in}{4.268847in}}%
\pgfpathlineto{\pgfqpoint{0.574826in}{4.267951in}}%
\pgfpathlineto{\pgfqpoint{0.575900in}{4.266264in}}%
\pgfpathlineto{\pgfqpoint{0.576974in}{4.266887in}}%
\pgfpathlineto{\pgfqpoint{0.579121in}{4.265506in}}%
\pgfpathlineto{\pgfqpoint{0.582343in}{4.265631in}}%
\pgfpathlineto{\pgfqpoint{0.583417in}{4.265104in}}%
\pgfpathlineto{\pgfqpoint{0.586638in}{4.266123in}}%
\pgfpathlineto{\pgfqpoint{0.589860in}{4.265084in}}%
\pgfpathlineto{\pgfqpoint{0.590933in}{4.265439in}}%
\pgfpathlineto{\pgfqpoint{0.592007in}{4.264412in}}%
\pgfpathlineto{\pgfqpoint{0.593081in}{4.264412in}}%
\pgfpathlineto{\pgfqpoint{0.594155in}{4.261896in}}%
\pgfpathlineto{\pgfqpoint{0.598450in}{4.261800in}}%
\pgfpathlineto{\pgfqpoint{0.604893in}{4.262515in}}%
\pgfpathlineto{\pgfqpoint{0.608115in}{4.265077in}}%
\pgfpathlineto{\pgfqpoint{0.609189in}{4.263843in}}%
\pgfpathlineto{\pgfqpoint{0.612410in}{4.263328in}}%
\pgfpathlineto{\pgfqpoint{0.613484in}{4.262444in}}%
\pgfpathlineto{\pgfqpoint{0.614558in}{4.260580in}}%
\pgfpathlineto{\pgfqpoint{0.615632in}{4.260614in}}%
\pgfpathlineto{\pgfqpoint{0.616706in}{4.261411in}}%
\pgfpathlineto{\pgfqpoint{0.619927in}{4.262015in}}%
\pgfpathlineto{\pgfqpoint{0.621001in}{4.263109in}}%
\pgfpathlineto{\pgfqpoint{0.622075in}{4.262602in}}%
\pgfpathlineto{\pgfqpoint{0.624222in}{4.259714in}}%
\pgfpathlineto{\pgfqpoint{0.629592in}{4.260183in}}%
\pgfpathlineto{\pgfqpoint{0.630666in}{4.261232in}}%
\pgfpathlineto{\pgfqpoint{0.631739in}{4.259683in}}%
\pgfpathlineto{\pgfqpoint{0.638182in}{4.259782in}}%
\pgfpathlineto{\pgfqpoint{0.639256in}{4.259133in}}%
\pgfpathlineto{\pgfqpoint{0.644625in}{4.258902in}}%
\pgfpathlineto{\pgfqpoint{0.646773in}{4.257369in}}%
\pgfpathlineto{\pgfqpoint{0.652142in}{4.258217in}}%
\pgfpathlineto{\pgfqpoint{0.653216in}{4.258853in}}%
\pgfpathlineto{\pgfqpoint{0.654290in}{4.258076in}}%
\pgfpathlineto{\pgfqpoint{0.667176in}{4.259033in}}%
\pgfpathlineto{\pgfqpoint{0.668250in}{4.257625in}}%
\pgfpathlineto{\pgfqpoint{0.669324in}{4.258040in}}%
\pgfpathlineto{\pgfqpoint{0.672545in}{4.259963in}}%
\pgfpathlineto{\pgfqpoint{0.673619in}{4.259490in}}%
\pgfpathlineto{\pgfqpoint{0.674693in}{4.259835in}}%
\pgfpathlineto{\pgfqpoint{0.676841in}{4.256912in}}%
\pgfpathlineto{\pgfqpoint{0.690800in}{4.257925in}}%
\pgfpathlineto{\pgfqpoint{0.691874in}{4.258279in}}%
\pgfpathlineto{\pgfqpoint{0.702613in}{4.255673in}}%
\pgfpathlineto{\pgfqpoint{0.704760in}{4.257468in}}%
\pgfpathlineto{\pgfqpoint{0.706908in}{4.257935in}}%
\pgfpathlineto{\pgfqpoint{0.710130in}{4.257903in}}%
\pgfpathlineto{\pgfqpoint{0.712277in}{4.256106in}}%
\pgfpathlineto{\pgfqpoint{0.713351in}{4.256168in}}%
\pgfpathlineto{\pgfqpoint{0.714425in}{4.257721in}}%
\pgfpathlineto{\pgfqpoint{0.717646in}{4.258085in}}%
\pgfpathlineto{\pgfqpoint{0.718720in}{4.261502in}}%
\pgfpathlineto{\pgfqpoint{0.719794in}{4.261841in}}%
\pgfpathlineto{\pgfqpoint{0.721942in}{4.261063in}}%
\pgfpathlineto{\pgfqpoint{0.727311in}{4.260640in}}%
\pgfpathlineto{\pgfqpoint{0.728385in}{4.259021in}}%
\pgfpathlineto{\pgfqpoint{0.729459in}{4.259281in}}%
\pgfpathlineto{\pgfqpoint{0.732680in}{4.258689in}}%
\pgfpathlineto{\pgfqpoint{0.733754in}{4.257501in}}%
\pgfpathlineto{\pgfqpoint{0.735902in}{4.259641in}}%
\pgfpathlineto{\pgfqpoint{0.740197in}{4.259035in}}%
\pgfpathlineto{\pgfqpoint{0.741271in}{4.259125in}}%
\pgfpathlineto{\pgfqpoint{0.742345in}{4.257447in}}%
\pgfpathlineto{\pgfqpoint{0.743419in}{4.258116in}}%
\pgfpathlineto{\pgfqpoint{0.747714in}{4.256653in}}%
\pgfpathlineto{\pgfqpoint{0.749862in}{4.256697in}}%
\pgfpathlineto{\pgfqpoint{0.752009in}{4.255459in}}%
\pgfpathlineto{\pgfqpoint{0.756305in}{4.255426in}}%
\pgfpathlineto{\pgfqpoint{0.757378in}{4.254874in}}%
\pgfpathlineto{\pgfqpoint{0.759526in}{4.254851in}}%
\pgfpathlineto{\pgfqpoint{0.763821in}{4.255587in}}%
\pgfpathlineto{\pgfqpoint{0.767043in}{4.254350in}}%
\pgfpathlineto{\pgfqpoint{0.770265in}{4.254040in}}%
\pgfpathlineto{\pgfqpoint{0.771338in}{4.252480in}}%
\pgfpathlineto{\pgfqpoint{0.774560in}{4.253655in}}%
\pgfpathlineto{\pgfqpoint{0.781003in}{4.252072in}}%
\pgfpathlineto{\pgfqpoint{0.782077in}{4.252922in}}%
\pgfpathlineto{\pgfqpoint{0.788520in}{4.253301in}}%
\pgfpathlineto{\pgfqpoint{0.789594in}{4.254037in}}%
\pgfpathlineto{\pgfqpoint{0.792815in}{4.253113in}}%
\pgfpathlineto{\pgfqpoint{0.794963in}{4.251477in}}%
\pgfpathlineto{\pgfqpoint{0.796037in}{4.251563in}}%
\pgfpathlineto{\pgfqpoint{0.797110in}{4.251004in}}%
\pgfpathlineto{\pgfqpoint{0.801406in}{4.250899in}}%
\pgfpathlineto{\pgfqpoint{0.803554in}{4.249797in}}%
\pgfpathlineto{\pgfqpoint{0.804627in}{4.250267in}}%
\pgfpathlineto{\pgfqpoint{0.825030in}{4.246844in}}%
\pgfpathlineto{\pgfqpoint{0.826104in}{4.247023in}}%
\pgfpathlineto{\pgfqpoint{0.827178in}{4.246296in}}%
\pgfpathlineto{\pgfqpoint{0.830399in}{4.246849in}}%
\pgfpathlineto{\pgfqpoint{0.833621in}{4.245811in}}%
\pgfpathlineto{\pgfqpoint{0.834695in}{4.245505in}}%
\pgfpathlineto{\pgfqpoint{0.837916in}{4.245530in}}%
\pgfpathlineto{\pgfqpoint{0.840064in}{4.244931in}}%
\pgfpathlineto{\pgfqpoint{0.842212in}{4.244673in}}%
\pgfpathlineto{\pgfqpoint{0.846507in}{4.244032in}}%
\pgfpathlineto{\pgfqpoint{0.848655in}{4.245013in}}%
\pgfpathlineto{\pgfqpoint{0.849729in}{4.244447in}}%
\pgfpathlineto{\pgfqpoint{0.852950in}{4.245656in}}%
\pgfpathlineto{\pgfqpoint{0.857245in}{4.244261in}}%
\pgfpathlineto{\pgfqpoint{0.860467in}{4.244585in}}%
\pgfpathlineto{\pgfqpoint{0.862615in}{4.243654in}}%
\pgfpathlineto{\pgfqpoint{0.863688in}{4.243739in}}%
\pgfpathlineto{\pgfqpoint{0.864762in}{4.242959in}}%
\pgfpathlineto{\pgfqpoint{0.877648in}{4.242981in}}%
\pgfpathlineto{\pgfqpoint{0.878722in}{4.243446in}}%
\pgfpathlineto{\pgfqpoint{0.879796in}{4.242471in}}%
\pgfpathlineto{\pgfqpoint{0.883018in}{4.243268in}}%
\pgfpathlineto{\pgfqpoint{0.884091in}{4.242683in}}%
\pgfpathlineto{\pgfqpoint{0.885165in}{4.243183in}}%
\pgfpathlineto{\pgfqpoint{0.886239in}{4.242522in}}%
\pgfpathlineto{\pgfqpoint{0.890534in}{4.242944in}}%
\pgfpathlineto{\pgfqpoint{0.891608in}{4.242380in}}%
\pgfpathlineto{\pgfqpoint{0.892682in}{4.242912in}}%
\pgfpathlineto{\pgfqpoint{0.899125in}{4.242737in}}%
\pgfpathlineto{\pgfqpoint{0.901273in}{4.241270in}}%
\pgfpathlineto{\pgfqpoint{0.908790in}{4.243316in}}%
\pgfpathlineto{\pgfqpoint{0.909864in}{4.242837in}}%
\pgfpathlineto{\pgfqpoint{0.913085in}{4.242784in}}%
\pgfpathlineto{\pgfqpoint{0.915233in}{4.241463in}}%
\pgfpathlineto{\pgfqpoint{0.916307in}{4.243312in}}%
\pgfpathlineto{\pgfqpoint{0.917380in}{4.244023in}}%
\pgfpathlineto{\pgfqpoint{0.920602in}{4.243999in}}%
\pgfpathlineto{\pgfqpoint{0.921676in}{4.243408in}}%
\pgfpathlineto{\pgfqpoint{0.922750in}{4.243517in}}%
\pgfpathlineto{\pgfqpoint{0.924897in}{4.241388in}}%
\pgfpathlineto{\pgfqpoint{0.930266in}{4.241162in}}%
\pgfpathlineto{\pgfqpoint{0.932414in}{4.239773in}}%
\pgfpathlineto{\pgfqpoint{0.936709in}{4.239703in}}%
\pgfpathlineto{\pgfqpoint{0.937783in}{4.239169in}}%
\pgfpathlineto{\pgfqpoint{0.939931in}{4.239236in}}%
\pgfpathlineto{\pgfqpoint{0.945300in}{4.239212in}}%
\pgfpathlineto{\pgfqpoint{0.947448in}{4.239514in}}%
\pgfpathlineto{\pgfqpoint{0.953891in}{4.238834in}}%
\pgfpathlineto{\pgfqpoint{0.954965in}{4.239496in}}%
\pgfpathlineto{\pgfqpoint{0.959260in}{4.239639in}}%
\pgfpathlineto{\pgfqpoint{0.960334in}{4.240484in}}%
\pgfpathlineto{\pgfqpoint{0.961408in}{4.240435in}}%
\pgfpathlineto{\pgfqpoint{0.962482in}{4.238964in}}%
\pgfpathlineto{\pgfqpoint{0.966777in}{4.239475in}}%
\pgfpathlineto{\pgfqpoint{0.967851in}{4.240179in}}%
\pgfpathlineto{\pgfqpoint{0.968925in}{4.238875in}}%
\pgfpathlineto{\pgfqpoint{0.969998in}{4.238969in}}%
\pgfpathlineto{\pgfqpoint{0.975368in}{4.238639in}}%
\pgfpathlineto{\pgfqpoint{0.976442in}{4.240240in}}%
\pgfpathlineto{\pgfqpoint{0.977515in}{4.239752in}}%
\pgfpathlineto{\pgfqpoint{0.980737in}{4.240993in}}%
\pgfpathlineto{\pgfqpoint{0.983958in}{4.239331in}}%
\pgfpathlineto{\pgfqpoint{0.985032in}{4.239848in}}%
\pgfpathlineto{\pgfqpoint{0.990401in}{4.239787in}}%
\pgfpathlineto{\pgfqpoint{0.992549in}{4.238544in}}%
\pgfpathlineto{\pgfqpoint{0.995771in}{4.238208in}}%
\pgfpathlineto{\pgfqpoint{0.997918in}{4.237471in}}%
\pgfpathlineto{\pgfqpoint{1.000066in}{4.236528in}}%
\pgfpathlineto{\pgfqpoint{1.006509in}{4.236340in}}%
\pgfpathlineto{\pgfqpoint{1.007583in}{4.235964in}}%
\pgfpathlineto{\pgfqpoint{1.020469in}{4.235310in}}%
\pgfpathlineto{\pgfqpoint{1.022617in}{4.234885in}}%
\pgfpathlineto{\pgfqpoint{1.027986in}{4.235099in}}%
\pgfpathlineto{\pgfqpoint{1.029060in}{4.234665in}}%
\pgfpathlineto{\pgfqpoint{1.030133in}{4.234849in}}%
\pgfpathlineto{\pgfqpoint{1.035503in}{4.235199in}}%
\pgfpathlineto{\pgfqpoint{1.036576in}{4.236111in}}%
\pgfpathlineto{\pgfqpoint{1.041946in}{4.236341in}}%
\pgfpathlineto{\pgfqpoint{1.043020in}{4.237069in}}%
\pgfpathlineto{\pgfqpoint{1.044093in}{4.236382in}}%
\pgfpathlineto{\pgfqpoint{1.051610in}{4.237106in}}%
\pgfpathlineto{\pgfqpoint{1.052684in}{4.236964in}}%
\pgfpathlineto{\pgfqpoint{1.056979in}{4.237163in}}%
\pgfpathlineto{\pgfqpoint{1.059127in}{4.236252in}}%
\pgfpathlineto{\pgfqpoint{1.060201in}{4.236164in}}%
\pgfpathlineto{\pgfqpoint{1.067718in}{4.234295in}}%
\pgfpathlineto{\pgfqpoint{1.072013in}{4.233629in}}%
\pgfpathlineto{\pgfqpoint{1.074161in}{4.232829in}}%
\pgfpathlineto{\pgfqpoint{1.075235in}{4.233572in}}%
\pgfpathlineto{\pgfqpoint{1.079530in}{4.233185in}}%
\pgfpathlineto{\pgfqpoint{1.081678in}{4.233251in}}%
\pgfpathlineto{\pgfqpoint{1.089195in}{4.234283in}}%
\pgfpathlineto{\pgfqpoint{1.090268in}{4.233854in}}%
\pgfpathlineto{\pgfqpoint{1.093490in}{4.234207in}}%
\pgfpathlineto{\pgfqpoint{1.094564in}{4.234975in}}%
\pgfpathlineto{\pgfqpoint{1.095638in}{4.234764in}}%
\pgfpathlineto{\pgfqpoint{1.097785in}{4.233174in}}%
\pgfpathlineto{\pgfqpoint{1.101007in}{4.232846in}}%
\pgfpathlineto{\pgfqpoint{1.102081in}{4.233601in}}%
\pgfpathlineto{\pgfqpoint{1.105302in}{4.232140in}}%
\pgfpathlineto{\pgfqpoint{1.118188in}{4.231234in}}%
\pgfpathlineto{\pgfqpoint{1.120336in}{4.230731in}}%
\pgfpathlineto{\pgfqpoint{1.133222in}{4.229545in}}%
\pgfpathlineto{\pgfqpoint{1.135370in}{4.229008in}}%
\pgfpathlineto{\pgfqpoint{1.146108in}{4.228165in}}%
\pgfpathlineto{\pgfqpoint{1.150403in}{4.227244in}}%
\pgfpathlineto{\pgfqpoint{1.154699in}{4.230005in}}%
\pgfpathlineto{\pgfqpoint{1.156846in}{4.229903in}}%
\pgfpathlineto{\pgfqpoint{1.157920in}{4.229129in}}%
\pgfpathlineto{\pgfqpoint{1.162216in}{4.229577in}}%
\pgfpathlineto{\pgfqpoint{1.165437in}{4.230059in}}%
\pgfpathlineto{\pgfqpoint{1.168659in}{4.229524in}}%
\pgfpathlineto{\pgfqpoint{1.170806in}{4.226133in}}%
\pgfpathlineto{\pgfqpoint{1.172954in}{4.225786in}}%
\pgfpathlineto{\pgfqpoint{1.179397in}{4.225202in}}%
\pgfpathlineto{\pgfqpoint{1.180471in}{4.224816in}}%
\pgfpathlineto{\pgfqpoint{1.187988in}{4.225126in}}%
\pgfpathlineto{\pgfqpoint{1.198726in}{4.226525in}}%
\pgfpathlineto{\pgfqpoint{1.200874in}{4.225096in}}%
\pgfpathlineto{\pgfqpoint{1.208391in}{4.225818in}}%
\pgfpathlineto{\pgfqpoint{1.209464in}{4.226480in}}%
\pgfpathlineto{\pgfqpoint{1.210538in}{4.228204in}}%
\pgfpathlineto{\pgfqpoint{1.214834in}{4.228369in}}%
\pgfpathlineto{\pgfqpoint{1.215908in}{4.228187in}}%
\pgfpathlineto{\pgfqpoint{1.216981in}{4.229089in}}%
\pgfpathlineto{\pgfqpoint{1.218055in}{4.229030in}}%
\pgfpathlineto{\pgfqpoint{1.221277in}{4.230847in}}%
\pgfpathlineto{\pgfqpoint{1.222351in}{4.229567in}}%
\pgfpathlineto{\pgfqpoint{1.224498in}{4.228633in}}%
\pgfpathlineto{\pgfqpoint{1.225572in}{4.228035in}}%
\pgfpathlineto{\pgfqpoint{1.228794in}{4.228292in}}%
\pgfpathlineto{\pgfqpoint{1.230941in}{4.227636in}}%
\pgfpathlineto{\pgfqpoint{1.232015in}{4.227758in}}%
\pgfpathlineto{\pgfqpoint{1.233089in}{4.226948in}}%
\pgfpathlineto{\pgfqpoint{1.243827in}{4.226907in}}%
\pgfpathlineto{\pgfqpoint{1.248123in}{4.225919in}}%
\pgfpathlineto{\pgfqpoint{1.252418in}{4.226694in}}%
\pgfpathlineto{\pgfqpoint{1.254566in}{4.226143in}}%
\pgfpathlineto{\pgfqpoint{1.258861in}{4.226343in}}%
\pgfpathlineto{\pgfqpoint{1.263156in}{4.227810in}}%
\pgfpathlineto{\pgfqpoint{1.267452in}{4.226627in}}%
\pgfpathlineto{\pgfqpoint{1.268526in}{4.227219in}}%
\pgfpathlineto{\pgfqpoint{1.270673in}{4.226466in}}%
\pgfpathlineto{\pgfqpoint{1.273895in}{4.226740in}}%
\pgfpathlineto{\pgfqpoint{1.274969in}{4.226096in}}%
\pgfpathlineto{\pgfqpoint{1.277116in}{4.226566in}}%
\pgfpathlineto{\pgfqpoint{1.278190in}{4.226034in}}%
\pgfpathlineto{\pgfqpoint{1.284633in}{4.225294in}}%
\pgfpathlineto{\pgfqpoint{1.292150in}{4.226094in}}%
\pgfpathlineto{\pgfqpoint{1.293224in}{4.226701in}}%
\pgfpathlineto{\pgfqpoint{1.297519in}{4.226034in}}%
\pgfpathlineto{\pgfqpoint{1.299667in}{4.224656in}}%
\pgfpathlineto{\pgfqpoint{1.306110in}{4.224560in}}%
\pgfpathlineto{\pgfqpoint{1.308258in}{4.225078in}}%
\pgfpathlineto{\pgfqpoint{1.313627in}{4.224150in}}%
\pgfpathlineto{\pgfqpoint{1.314701in}{4.223539in}}%
\pgfpathlineto{\pgfqpoint{1.315774in}{4.223779in}}%
\pgfpathlineto{\pgfqpoint{1.334030in}{4.223219in}}%
\pgfpathlineto{\pgfqpoint{1.335104in}{4.223714in}}%
\pgfpathlineto{\pgfqpoint{1.337251in}{4.223381in}}%
\pgfpathlineto{\pgfqpoint{1.338325in}{4.223094in}}%
\pgfpathlineto{\pgfqpoint{1.352285in}{4.222219in}}%
\pgfpathlineto{\pgfqpoint{1.353359in}{4.221912in}}%
\pgfpathlineto{\pgfqpoint{1.358728in}{4.221985in}}%
\pgfpathlineto{\pgfqpoint{1.360876in}{4.222325in}}%
\pgfpathlineto{\pgfqpoint{1.367319in}{4.221953in}}%
\pgfpathlineto{\pgfqpoint{1.368393in}{4.221735in}}%
\pgfpathlineto{\pgfqpoint{1.379131in}{4.222358in}}%
\pgfpathlineto{\pgfqpoint{1.381279in}{4.221698in}}%
\pgfpathlineto{\pgfqpoint{1.387722in}{4.221917in}}%
\pgfpathlineto{\pgfqpoint{1.396312in}{4.221393in}}%
\pgfpathlineto{\pgfqpoint{1.397386in}{4.222232in}}%
\pgfpathlineto{\pgfqpoint{1.398460in}{4.221803in}}%
\pgfpathlineto{\pgfqpoint{1.405977in}{4.221707in}}%
\pgfpathlineto{\pgfqpoint{1.410272in}{4.222061in}}%
\pgfpathlineto{\pgfqpoint{1.411346in}{4.222151in}}%
\pgfpathlineto{\pgfqpoint{1.413494in}{4.223363in}}%
\pgfpathlineto{\pgfqpoint{1.417789in}{4.223471in}}%
\pgfpathlineto{\pgfqpoint{1.419937in}{4.223700in}}%
\pgfpathlineto{\pgfqpoint{1.421011in}{4.223094in}}%
\pgfpathlineto{\pgfqpoint{1.426380in}{4.222766in}}%
\pgfpathlineto{\pgfqpoint{1.434971in}{4.221574in}}%
\pgfpathlineto{\pgfqpoint{1.439266in}{4.221506in}}%
\pgfpathlineto{\pgfqpoint{1.458595in}{4.221750in}}%
\pgfpathlineto{\pgfqpoint{1.463964in}{4.221287in}}%
\pgfpathlineto{\pgfqpoint{1.466112in}{4.220862in}}%
\pgfpathlineto{\pgfqpoint{1.471481in}{4.221434in}}%
\pgfpathlineto{\pgfqpoint{1.472555in}{4.222206in}}%
\pgfpathlineto{\pgfqpoint{1.477924in}{4.222474in}}%
\pgfpathlineto{\pgfqpoint{1.480072in}{4.223495in}}%
\pgfpathlineto{\pgfqpoint{1.481146in}{4.222969in}}%
\pgfpathlineto{\pgfqpoint{1.484367in}{4.222910in}}%
\pgfpathlineto{\pgfqpoint{1.485441in}{4.223824in}}%
\pgfpathlineto{\pgfqpoint{1.486515in}{4.222493in}}%
\pgfpathlineto{\pgfqpoint{1.487589in}{4.223430in}}%
\pgfpathlineto{\pgfqpoint{1.488662in}{4.225118in}}%
\pgfpathlineto{\pgfqpoint{1.491884in}{4.225470in}}%
\pgfpathlineto{\pgfqpoint{1.494032in}{4.224965in}}%
\pgfpathlineto{\pgfqpoint{1.501549in}{4.223202in}}%
\pgfpathlineto{\pgfqpoint{1.503696in}{4.219937in}}%
\pgfpathlineto{\pgfqpoint{1.510139in}{4.218886in}}%
\pgfpathlineto{\pgfqpoint{1.511213in}{4.218399in}}%
\pgfpathlineto{\pgfqpoint{1.515508in}{4.218033in}}%
\pgfpathlineto{\pgfqpoint{1.521951in}{4.217311in}}%
\pgfpathlineto{\pgfqpoint{1.524099in}{4.217307in}}%
\pgfpathlineto{\pgfqpoint{1.526247in}{4.216991in}}%
\pgfpathlineto{\pgfqpoint{1.532690in}{4.216625in}}%
\pgfpathlineto{\pgfqpoint{1.533764in}{4.216416in}}%
\pgfpathlineto{\pgfqpoint{1.545576in}{4.216274in}}%
\pgfpathlineto{\pgfqpoint{1.547724in}{4.215845in}}%
\pgfpathlineto{\pgfqpoint{1.553093in}{4.216208in}}%
\pgfpathlineto{\pgfqpoint{1.554167in}{4.216876in}}%
\pgfpathlineto{\pgfqpoint{1.555240in}{4.216634in}}%
\pgfpathlineto{\pgfqpoint{1.556314in}{4.217169in}}%
\pgfpathlineto{\pgfqpoint{1.559536in}{4.217243in}}%
\pgfpathlineto{\pgfqpoint{1.563831in}{4.214977in}}%
\pgfpathlineto{\pgfqpoint{1.569200in}{4.214616in}}%
\pgfpathlineto{\pgfqpoint{1.578865in}{4.215314in}}%
\pgfpathlineto{\pgfqpoint{1.584234in}{4.216371in}}%
\pgfpathlineto{\pgfqpoint{1.585308in}{4.215370in}}%
\pgfpathlineto{\pgfqpoint{1.586382in}{4.215870in}}%
\pgfpathlineto{\pgfqpoint{1.592825in}{4.216373in}}%
\pgfpathlineto{\pgfqpoint{1.593899in}{4.215763in}}%
\pgfpathlineto{\pgfqpoint{1.599268in}{4.215625in}}%
\pgfpathlineto{\pgfqpoint{1.600342in}{4.214782in}}%
\pgfpathlineto{\pgfqpoint{1.601416in}{4.215235in}}%
\pgfpathlineto{\pgfqpoint{1.628261in}{4.214231in}}%
\pgfpathlineto{\pgfqpoint{1.637926in}{4.213599in}}%
\pgfpathlineto{\pgfqpoint{1.639000in}{4.213819in}}%
\pgfpathlineto{\pgfqpoint{1.642221in}{4.213390in}}%
\pgfpathlineto{\pgfqpoint{1.644369in}{4.214154in}}%
\pgfpathlineto{\pgfqpoint{1.645443in}{4.214058in}}%
\pgfpathlineto{\pgfqpoint{1.646517in}{4.214813in}}%
\pgfpathlineto{\pgfqpoint{1.649738in}{4.214321in}}%
\pgfpathlineto{\pgfqpoint{1.650812in}{4.215301in}}%
\pgfpathlineto{\pgfqpoint{1.651886in}{4.215433in}}%
\pgfpathlineto{\pgfqpoint{1.652960in}{4.214743in}}%
\pgfpathlineto{\pgfqpoint{1.654034in}{4.215163in}}%
\pgfpathlineto{\pgfqpoint{1.657255in}{4.214302in}}%
\pgfpathlineto{\pgfqpoint{1.658329in}{4.214773in}}%
\pgfpathlineto{\pgfqpoint{1.659403in}{4.214225in}}%
\pgfpathlineto{\pgfqpoint{1.661550in}{4.214221in}}%
\pgfpathlineto{\pgfqpoint{1.665846in}{4.214261in}}%
\pgfpathlineto{\pgfqpoint{1.666920in}{4.215122in}}%
\pgfpathlineto{\pgfqpoint{1.669067in}{4.214922in}}%
\pgfpathlineto{\pgfqpoint{1.673363in}{4.214556in}}%
\pgfpathlineto{\pgfqpoint{1.674437in}{4.215145in}}%
\pgfpathlineto{\pgfqpoint{1.690544in}{4.214308in}}%
\pgfpathlineto{\pgfqpoint{1.691618in}{4.215307in}}%
\pgfpathlineto{\pgfqpoint{1.696987in}{4.214562in}}%
\pgfpathlineto{\pgfqpoint{1.698061in}{4.215774in}}%
\pgfpathlineto{\pgfqpoint{1.705578in}{4.216627in}}%
\pgfpathlineto{\pgfqpoint{1.706652in}{4.216282in}}%
\pgfpathlineto{\pgfqpoint{1.710947in}{4.216218in}}%
\pgfpathlineto{\pgfqpoint{1.713095in}{4.216033in}}%
\pgfpathlineto{\pgfqpoint{1.714169in}{4.215513in}}%
\pgfpathlineto{\pgfqpoint{1.719538in}{4.215330in}}%
\pgfpathlineto{\pgfqpoint{1.721685in}{4.214831in}}%
\pgfpathlineto{\pgfqpoint{1.743162in}{4.215804in}}%
\pgfpathlineto{\pgfqpoint{1.744236in}{4.216132in}}%
\pgfpathlineto{\pgfqpoint{1.748531in}{4.216158in}}%
\pgfpathlineto{\pgfqpoint{1.750679in}{4.215403in}}%
\pgfpathlineto{\pgfqpoint{1.751753in}{4.215866in}}%
\pgfpathlineto{\pgfqpoint{1.756048in}{4.216267in}}%
\pgfpathlineto{\pgfqpoint{1.757122in}{4.216145in}}%
\pgfpathlineto{\pgfqpoint{1.758196in}{4.215438in}}%
\pgfpathlineto{\pgfqpoint{1.759270in}{4.215609in}}%
\pgfpathlineto{\pgfqpoint{1.763565in}{4.215382in}}%
\pgfpathlineto{\pgfqpoint{1.765713in}{4.216392in}}%
\pgfpathlineto{\pgfqpoint{1.766787in}{4.216077in}}%
\pgfpathlineto{\pgfqpoint{1.771082in}{4.216809in}}%
\pgfpathlineto{\pgfqpoint{1.773230in}{4.216514in}}%
\pgfpathlineto{\pgfqpoint{1.778599in}{4.216498in}}%
\pgfpathlineto{\pgfqpoint{1.779673in}{4.217231in}}%
\pgfpathlineto{\pgfqpoint{1.785042in}{4.216202in}}%
\pgfpathlineto{\pgfqpoint{1.792559in}{4.216026in}}%
\pgfpathlineto{\pgfqpoint{1.794706in}{4.216476in}}%
\pgfpathlineto{\pgfqpoint{1.795780in}{4.218066in}}%
\pgfpathlineto{\pgfqpoint{1.800076in}{4.218294in}}%
\pgfpathlineto{\pgfqpoint{1.802223in}{4.217496in}}%
\pgfpathlineto{\pgfqpoint{1.809740in}{4.217616in}}%
\pgfpathlineto{\pgfqpoint{1.811888in}{4.218220in}}%
\pgfpathlineto{\pgfqpoint{1.815109in}{4.217532in}}%
\pgfpathlineto{\pgfqpoint{1.816183in}{4.218316in}}%
\pgfpathlineto{\pgfqpoint{1.823700in}{4.218358in}}%
\pgfpathlineto{\pgfqpoint{1.831217in}{4.221362in}}%
\pgfpathlineto{\pgfqpoint{1.832291in}{4.219642in}}%
\pgfpathlineto{\pgfqpoint{1.834438in}{4.219174in}}%
\pgfpathlineto{\pgfqpoint{1.837660in}{4.219812in}}%
\pgfpathlineto{\pgfqpoint{1.838734in}{4.220967in}}%
\pgfpathlineto{\pgfqpoint{1.839808in}{4.220053in}}%
\pgfpathlineto{\pgfqpoint{1.840882in}{4.219892in}}%
\pgfpathlineto{\pgfqpoint{1.841955in}{4.220499in}}%
\pgfpathlineto{\pgfqpoint{1.846251in}{4.219322in}}%
\pgfpathlineto{\pgfqpoint{1.847325in}{4.220121in}}%
\pgfpathlineto{\pgfqpoint{1.852694in}{4.220138in}}%
\pgfpathlineto{\pgfqpoint{1.854841in}{4.219050in}}%
\pgfpathlineto{\pgfqpoint{1.855915in}{4.219434in}}%
\pgfpathlineto{\pgfqpoint{1.856989in}{4.220479in}}%
\pgfpathlineto{\pgfqpoint{1.860211in}{4.220340in}}%
\pgfpathlineto{\pgfqpoint{1.862358in}{4.221120in}}%
\pgfpathlineto{\pgfqpoint{1.863432in}{4.221146in}}%
\pgfpathlineto{\pgfqpoint{1.864506in}{4.220477in}}%
\pgfpathlineto{\pgfqpoint{1.867727in}{4.220864in}}%
\pgfpathlineto{\pgfqpoint{1.868801in}{4.219806in}}%
\pgfpathlineto{\pgfqpoint{1.872023in}{4.219272in}}%
\pgfpathlineto{\pgfqpoint{1.879540in}{4.217245in}}%
\pgfpathlineto{\pgfqpoint{1.887057in}{4.217584in}}%
\pgfpathlineto{\pgfqpoint{1.891352in}{4.217462in}}%
\pgfpathlineto{\pgfqpoint{1.892426in}{4.217258in}}%
\pgfpathlineto{\pgfqpoint{1.893500in}{4.215504in}}%
\pgfpathlineto{\pgfqpoint{1.898869in}{4.215311in}}%
\pgfpathlineto{\pgfqpoint{1.901016in}{4.215081in}}%
\pgfpathlineto{\pgfqpoint{1.902090in}{4.215184in}}%
\pgfpathlineto{\pgfqpoint{1.906386in}{4.214776in}}%
\pgfpathlineto{\pgfqpoint{1.909607in}{4.214649in}}%
\pgfpathlineto{\pgfqpoint{1.913903in}{4.215067in}}%
\pgfpathlineto{\pgfqpoint{1.914976in}{4.214694in}}%
\pgfpathlineto{\pgfqpoint{1.917124in}{4.215553in}}%
\pgfpathlineto{\pgfqpoint{1.924641in}{4.214427in}}%
\pgfpathlineto{\pgfqpoint{1.937527in}{4.215249in}}%
\pgfpathlineto{\pgfqpoint{1.938601in}{4.215591in}}%
\pgfpathlineto{\pgfqpoint{1.939675in}{4.214575in}}%
\pgfpathlineto{\pgfqpoint{1.946118in}{4.214745in}}%
\pgfpathlineto{\pgfqpoint{1.947192in}{4.215428in}}%
\pgfpathlineto{\pgfqpoint{1.950413in}{4.214692in}}%
\pgfpathlineto{\pgfqpoint{1.951487in}{4.217102in}}%
\pgfpathlineto{\pgfqpoint{1.952561in}{4.216583in}}%
\pgfpathlineto{\pgfqpoint{1.954708in}{4.217437in}}%
\pgfpathlineto{\pgfqpoint{1.959004in}{4.216842in}}%
\pgfpathlineto{\pgfqpoint{1.961151in}{4.216217in}}%
\pgfpathlineto{\pgfqpoint{1.968668in}{4.216357in}}%
\pgfpathlineto{\pgfqpoint{1.975111in}{4.218086in}}%
\pgfpathlineto{\pgfqpoint{1.976185in}{4.219141in}}%
\pgfpathlineto{\pgfqpoint{1.980481in}{4.219302in}}%
\pgfpathlineto{\pgfqpoint{1.981554in}{4.219175in}}%
\pgfpathlineto{\pgfqpoint{1.982628in}{4.219853in}}%
\pgfpathlineto{\pgfqpoint{1.983702in}{4.219050in}}%
\pgfpathlineto{\pgfqpoint{1.984776in}{4.219832in}}%
\pgfpathlineto{\pgfqpoint{1.995514in}{4.220178in}}%
\pgfpathlineto{\pgfqpoint{1.996588in}{4.217792in}}%
\pgfpathlineto{\pgfqpoint{1.998736in}{4.217044in}}%
\pgfpathlineto{\pgfqpoint{1.999810in}{4.215985in}}%
\pgfpathlineto{\pgfqpoint{2.004105in}{4.216848in}}%
\pgfpathlineto{\pgfqpoint{2.005179in}{4.215521in}}%
\pgfpathlineto{\pgfqpoint{2.007326in}{4.215267in}}%
\pgfpathlineto{\pgfqpoint{2.012696in}{4.215242in}}%
\pgfpathlineto{\pgfqpoint{2.013770in}{4.215715in}}%
\pgfpathlineto{\pgfqpoint{2.014843in}{4.214819in}}%
\pgfpathlineto{\pgfqpoint{2.029877in}{4.213708in}}%
\pgfpathlineto{\pgfqpoint{2.033099in}{4.214052in}}%
\pgfpathlineto{\pgfqpoint{2.034172in}{4.213369in}}%
\pgfpathlineto{\pgfqpoint{2.043837in}{4.213296in}}%
\pgfpathlineto{\pgfqpoint{2.044911in}{4.212816in}}%
\pgfpathlineto{\pgfqpoint{2.051354in}{4.212377in}}%
\pgfpathlineto{\pgfqpoint{2.052428in}{4.212016in}}%
\pgfpathlineto{\pgfqpoint{2.058871in}{4.212212in}}%
\pgfpathlineto{\pgfqpoint{2.067461in}{4.211506in}}%
\pgfpathlineto{\pgfqpoint{2.074978in}{4.211701in}}%
\pgfpathlineto{\pgfqpoint{2.082495in}{4.211225in}}%
\pgfpathlineto{\pgfqpoint{2.090012in}{4.211250in}}%
\pgfpathlineto{\pgfqpoint{2.108267in}{4.211266in}}%
\pgfpathlineto{\pgfqpoint{2.110415in}{4.210992in}}%
\pgfpathlineto{\pgfqpoint{2.115784in}{4.211033in}}%
\pgfpathlineto{\pgfqpoint{2.119006in}{4.211754in}}%
\pgfpathlineto{\pgfqpoint{2.123301in}{4.211531in}}%
\pgfpathlineto{\pgfqpoint{2.125449in}{4.210679in}}%
\pgfpathlineto{\pgfqpoint{2.127596in}{4.210913in}}%
\pgfpathlineto{\pgfqpoint{2.135113in}{4.211022in}}%
\pgfpathlineto{\pgfqpoint{2.141556in}{4.210640in}}%
\pgfpathlineto{\pgfqpoint{2.142630in}{4.210972in}}%
\pgfpathlineto{\pgfqpoint{2.147999in}{4.211134in}}%
\pgfpathlineto{\pgfqpoint{2.149073in}{4.210670in}}%
\pgfpathlineto{\pgfqpoint{2.150147in}{4.210880in}}%
\pgfpathlineto{\pgfqpoint{2.156590in}{4.209789in}}%
\pgfpathlineto{\pgfqpoint{2.157664in}{4.210796in}}%
\pgfpathlineto{\pgfqpoint{2.160885in}{4.211208in}}%
\pgfpathlineto{\pgfqpoint{2.165181in}{4.209443in}}%
\pgfpathlineto{\pgfqpoint{2.175919in}{4.208953in}}%
\pgfpathlineto{\pgfqpoint{2.180214in}{4.208308in}}%
\pgfpathlineto{\pgfqpoint{2.221020in}{4.208516in}}%
\pgfpathlineto{\pgfqpoint{2.225316in}{4.208450in}}%
\pgfpathlineto{\pgfqpoint{2.237128in}{4.208252in}}%
\pgfpathlineto{\pgfqpoint{2.247866in}{4.209249in}}%
\pgfpathlineto{\pgfqpoint{2.255383in}{4.208781in}}%
\pgfpathlineto{\pgfqpoint{2.277934in}{4.210163in}}%
\pgfpathlineto{\pgfqpoint{2.288672in}{4.209966in}}%
\pgfpathlineto{\pgfqpoint{2.289746in}{4.211003in}}%
\pgfpathlineto{\pgfqpoint{2.292968in}{4.211102in}}%
\pgfpathlineto{\pgfqpoint{2.300484in}{4.210942in}}%
\pgfpathlineto{\pgfqpoint{2.308001in}{4.209126in}}%
\pgfpathlineto{\pgfqpoint{2.313370in}{4.209510in}}%
\pgfpathlineto{\pgfqpoint{2.315518in}{4.209316in}}%
\pgfpathlineto{\pgfqpoint{2.320887in}{4.209456in}}%
\pgfpathlineto{\pgfqpoint{2.323035in}{4.209206in}}%
\pgfpathlineto{\pgfqpoint{2.334847in}{4.209541in}}%
\pgfpathlineto{\pgfqpoint{2.335921in}{4.208679in}}%
\pgfpathlineto{\pgfqpoint{2.338069in}{4.208208in}}%
\pgfpathlineto{\pgfqpoint{2.342364in}{4.208142in}}%
\pgfpathlineto{\pgfqpoint{2.344512in}{4.208670in}}%
\pgfpathlineto{\pgfqpoint{2.345586in}{4.208393in}}%
\pgfpathlineto{\pgfqpoint{2.360619in}{4.208180in}}%
\pgfpathlineto{\pgfqpoint{2.383170in}{4.208189in}}%
\pgfpathlineto{\pgfqpoint{2.390687in}{4.208373in}}%
\pgfpathlineto{\pgfqpoint{2.398204in}{4.208834in}}%
\pgfpathlineto{\pgfqpoint{2.403573in}{4.208441in}}%
\pgfpathlineto{\pgfqpoint{2.412164in}{4.207052in}}%
\pgfpathlineto{\pgfqpoint{2.417533in}{4.207060in}}%
\pgfpathlineto{\pgfqpoint{2.419680in}{4.206390in}}%
\pgfpathlineto{\pgfqpoint{2.450822in}{4.205655in}}%
\pgfpathlineto{\pgfqpoint{2.464782in}{4.205916in}}%
\pgfpathlineto{\pgfqpoint{2.486258in}{4.205085in}}%
\pgfpathlineto{\pgfqpoint{2.488406in}{4.204956in}}%
\pgfpathlineto{\pgfqpoint{2.493775in}{4.204677in}}%
\pgfpathlineto{\pgfqpoint{2.495923in}{4.204407in}}%
\pgfpathlineto{\pgfqpoint{2.510957in}{4.204767in}}%
\pgfpathlineto{\pgfqpoint{2.525991in}{4.203163in}}%
\pgfpathlineto{\pgfqpoint{2.536729in}{4.203066in}}%
\pgfpathlineto{\pgfqpoint{2.541024in}{4.202282in}}%
\pgfpathlineto{\pgfqpoint{2.553910in}{4.202620in}}%
\pgfpathlineto{\pgfqpoint{2.556058in}{4.202940in}}%
\pgfpathlineto{\pgfqpoint{2.566796in}{4.202626in}}%
\pgfpathlineto{\pgfqpoint{2.570018in}{4.202544in}}%
\pgfpathlineto{\pgfqpoint{2.577535in}{4.202366in}}%
\pgfpathlineto{\pgfqpoint{2.581830in}{4.202686in}}%
\pgfpathlineto{\pgfqpoint{2.582904in}{4.204101in}}%
\pgfpathlineto{\pgfqpoint{2.586125in}{4.204051in}}%
\pgfpathlineto{\pgfqpoint{2.591495in}{4.203230in}}%
\pgfpathlineto{\pgfqpoint{2.593642in}{4.202919in}}%
\pgfpathlineto{\pgfqpoint{2.631227in}{4.203001in}}%
\pgfpathlineto{\pgfqpoint{2.638744in}{4.201971in}}%
\pgfpathlineto{\pgfqpoint{2.650556in}{4.202264in}}%
\pgfpathlineto{\pgfqpoint{2.653777in}{4.202411in}}%
\pgfpathlineto{\pgfqpoint{2.665590in}{4.201526in}}%
\pgfpathlineto{\pgfqpoint{2.674180in}{4.201337in}}%
\pgfpathlineto{\pgfqpoint{2.679549in}{4.200940in}}%
\pgfpathlineto{\pgfqpoint{2.680623in}{4.199388in}}%
\pgfpathlineto{\pgfqpoint{2.681697in}{4.199069in}}%
\pgfpathlineto{\pgfqpoint{2.682771in}{4.199556in}}%
\pgfpathlineto{\pgfqpoint{2.683845in}{4.199363in}}%
\pgfpathlineto{\pgfqpoint{2.690288in}{4.199621in}}%
\pgfpathlineto{\pgfqpoint{2.718208in}{4.199121in}}%
\pgfpathlineto{\pgfqpoint{2.721429in}{4.198546in}}%
\pgfpathlineto{\pgfqpoint{2.733241in}{4.198999in}}%
\pgfpathlineto{\pgfqpoint{2.735389in}{4.198842in}}%
\pgfpathlineto{\pgfqpoint{2.751497in}{4.199117in}}%
\pgfpathlineto{\pgfqpoint{2.751497in}{4.199117in}}%
\pgfusepath{stroke}%
\end{pgfscope}%
\begin{pgfscope}%
\pgfsetrectcap%
\pgfsetmiterjoin%
\pgfsetlinewidth{0.803000pt}%
\definecolor{currentstroke}{rgb}{1.000000,1.000000,1.000000}%
\pgfsetstrokecolor{currentstroke}%
\pgfsetdash{}{0pt}%
\pgfpathmoveto{\pgfqpoint{0.285588in}{4.180131in}}%
\pgfpathlineto{\pgfqpoint{0.285588in}{4.581016in}}%
\pgfusepath{stroke}%
\end{pgfscope}%
\begin{pgfscope}%
\pgfsetrectcap%
\pgfsetmiterjoin%
\pgfsetlinewidth{0.803000pt}%
\definecolor{currentstroke}{rgb}{1.000000,1.000000,1.000000}%
\pgfsetstrokecolor{currentstroke}%
\pgfsetdash{}{0pt}%
\pgfpathmoveto{\pgfqpoint{2.868921in}{4.180131in}}%
\pgfpathlineto{\pgfqpoint{2.868921in}{4.581016in}}%
\pgfusepath{stroke}%
\end{pgfscope}%
\begin{pgfscope}%
\pgfsetrectcap%
\pgfsetmiterjoin%
\pgfsetlinewidth{0.803000pt}%
\definecolor{currentstroke}{rgb}{1.000000,1.000000,1.000000}%
\pgfsetstrokecolor{currentstroke}%
\pgfsetdash{}{0pt}%
\pgfpathmoveto{\pgfqpoint{0.285587in}{4.180131in}}%
\pgfpathlineto{\pgfqpoint{2.868921in}{4.180131in}}%
\pgfusepath{stroke}%
\end{pgfscope}%
\begin{pgfscope}%
\pgfsetrectcap%
\pgfsetmiterjoin%
\pgfsetlinewidth{0.803000pt}%
\definecolor{currentstroke}{rgb}{1.000000,1.000000,1.000000}%
\pgfsetstrokecolor{currentstroke}%
\pgfsetdash{}{0pt}%
\pgfpathmoveto{\pgfqpoint{0.285587in}{4.581016in}}%
\pgfpathlineto{\pgfqpoint{2.868921in}{4.581016in}}%
\pgfusepath{stroke}%
\end{pgfscope}%
\begin{pgfscope}%
\definecolor{textcolor}{rgb}{0.150000,0.150000,0.150000}%
\pgfsetstrokecolor{textcolor}%
\pgfsetfillcolor{textcolor}%
\pgftext[x=1.577254in,y=4.664349in,,base]{\color{textcolor}\rmfamily\fontsize{12.000000}{14.400000}\selectfont MMM}%
\end{pgfscope}%
\begin{pgfscope}%
\pgfsetbuttcap%
\pgfsetmiterjoin%
\definecolor{currentfill}{rgb}{0.917647,0.917647,0.949020}%
\pgfsetfillcolor{currentfill}%
\pgfsetlinewidth{0.000000pt}%
\definecolor{currentstroke}{rgb}{0.000000,0.000000,0.000000}%
\pgfsetstrokecolor{currentstroke}%
\pgfsetstrokeopacity{0.000000}%
\pgfsetdash{}{0pt}%
\pgfpathmoveto{\pgfqpoint{3.902254in}{4.180131in}}%
\pgfpathlineto{\pgfqpoint{6.485588in}{4.180131in}}%
\pgfpathlineto{\pgfqpoint{6.485588in}{4.581016in}}%
\pgfpathlineto{\pgfqpoint{3.902254in}{4.581016in}}%
\pgfpathclose%
\pgfusepath{fill}%
\end{pgfscope}%
\begin{pgfscope}%
\pgfpathrectangle{\pgfqpoint{3.902254in}{4.180131in}}{\pgfqpoint{2.583333in}{0.400885in}}%
\pgfusepath{clip}%
\pgfsetroundcap%
\pgfsetroundjoin%
\pgfsetlinewidth{0.803000pt}%
\definecolor{currentstroke}{rgb}{1.000000,1.000000,1.000000}%
\pgfsetstrokecolor{currentstroke}%
\pgfsetdash{}{0pt}%
\pgfpathmoveto{\pgfqpoint{4.017531in}{4.180131in}}%
\pgfpathlineto{\pgfqpoint{4.017531in}{4.581016in}}%
\pgfusepath{stroke}%
\end{pgfscope}%
\begin{pgfscope}%
\definecolor{textcolor}{rgb}{0.150000,0.150000,0.150000}%
\pgfsetstrokecolor{textcolor}%
\pgfsetfillcolor{textcolor}%
\pgftext[x=4.017531in,y=4.082908in,,top]{\color{textcolor}\rmfamily\fontsize{10.000000}{12.000000}\selectfont 2012}%
\end{pgfscope}%
\begin{pgfscope}%
\pgfpathrectangle{\pgfqpoint{3.902254in}{4.180131in}}{\pgfqpoint{2.583333in}{0.400885in}}%
\pgfusepath{clip}%
\pgfsetroundcap%
\pgfsetroundjoin%
\pgfsetlinewidth{0.803000pt}%
\definecolor{currentstroke}{rgb}{1.000000,1.000000,1.000000}%
\pgfsetstrokecolor{currentstroke}%
\pgfsetdash{}{0pt}%
\pgfpathmoveto{\pgfqpoint{4.410556in}{4.180131in}}%
\pgfpathlineto{\pgfqpoint{4.410556in}{4.581016in}}%
\pgfusepath{stroke}%
\end{pgfscope}%
\begin{pgfscope}%
\definecolor{textcolor}{rgb}{0.150000,0.150000,0.150000}%
\pgfsetstrokecolor{textcolor}%
\pgfsetfillcolor{textcolor}%
\pgftext[x=4.410556in,y=4.082908in,,top]{\color{textcolor}\rmfamily\fontsize{10.000000}{12.000000}\selectfont 2013}%
\end{pgfscope}%
\begin{pgfscope}%
\pgfpathrectangle{\pgfqpoint{3.902254in}{4.180131in}}{\pgfqpoint{2.583333in}{0.400885in}}%
\pgfusepath{clip}%
\pgfsetroundcap%
\pgfsetroundjoin%
\pgfsetlinewidth{0.803000pt}%
\definecolor{currentstroke}{rgb}{1.000000,1.000000,1.000000}%
\pgfsetstrokecolor{currentstroke}%
\pgfsetdash{}{0pt}%
\pgfpathmoveto{\pgfqpoint{4.802507in}{4.180131in}}%
\pgfpathlineto{\pgfqpoint{4.802507in}{4.581016in}}%
\pgfusepath{stroke}%
\end{pgfscope}%
\begin{pgfscope}%
\definecolor{textcolor}{rgb}{0.150000,0.150000,0.150000}%
\pgfsetstrokecolor{textcolor}%
\pgfsetfillcolor{textcolor}%
\pgftext[x=4.802507in,y=4.082908in,,top]{\color{textcolor}\rmfamily\fontsize{10.000000}{12.000000}\selectfont 2014}%
\end{pgfscope}%
\begin{pgfscope}%
\pgfpathrectangle{\pgfqpoint{3.902254in}{4.180131in}}{\pgfqpoint{2.583333in}{0.400885in}}%
\pgfusepath{clip}%
\pgfsetroundcap%
\pgfsetroundjoin%
\pgfsetlinewidth{0.803000pt}%
\definecolor{currentstroke}{rgb}{1.000000,1.000000,1.000000}%
\pgfsetstrokecolor{currentstroke}%
\pgfsetdash{}{0pt}%
\pgfpathmoveto{\pgfqpoint{5.194458in}{4.180131in}}%
\pgfpathlineto{\pgfqpoint{5.194458in}{4.581016in}}%
\pgfusepath{stroke}%
\end{pgfscope}%
\begin{pgfscope}%
\definecolor{textcolor}{rgb}{0.150000,0.150000,0.150000}%
\pgfsetstrokecolor{textcolor}%
\pgfsetfillcolor{textcolor}%
\pgftext[x=5.194458in,y=4.082908in,,top]{\color{textcolor}\rmfamily\fontsize{10.000000}{12.000000}\selectfont 2015}%
\end{pgfscope}%
\begin{pgfscope}%
\pgfpathrectangle{\pgfqpoint{3.902254in}{4.180131in}}{\pgfqpoint{2.583333in}{0.400885in}}%
\pgfusepath{clip}%
\pgfsetroundcap%
\pgfsetroundjoin%
\pgfsetlinewidth{0.803000pt}%
\definecolor{currentstroke}{rgb}{1.000000,1.000000,1.000000}%
\pgfsetstrokecolor{currentstroke}%
\pgfsetdash{}{0pt}%
\pgfpathmoveto{\pgfqpoint{5.586409in}{4.180131in}}%
\pgfpathlineto{\pgfqpoint{5.586409in}{4.581016in}}%
\pgfusepath{stroke}%
\end{pgfscope}%
\begin{pgfscope}%
\definecolor{textcolor}{rgb}{0.150000,0.150000,0.150000}%
\pgfsetstrokecolor{textcolor}%
\pgfsetfillcolor{textcolor}%
\pgftext[x=5.586409in,y=4.082908in,,top]{\color{textcolor}\rmfamily\fontsize{10.000000}{12.000000}\selectfont 2016}%
\end{pgfscope}%
\begin{pgfscope}%
\pgfpathrectangle{\pgfqpoint{3.902254in}{4.180131in}}{\pgfqpoint{2.583333in}{0.400885in}}%
\pgfusepath{clip}%
\pgfsetroundcap%
\pgfsetroundjoin%
\pgfsetlinewidth{0.803000pt}%
\definecolor{currentstroke}{rgb}{1.000000,1.000000,1.000000}%
\pgfsetstrokecolor{currentstroke}%
\pgfsetdash{}{0pt}%
\pgfpathmoveto{\pgfqpoint{5.979434in}{4.180131in}}%
\pgfpathlineto{\pgfqpoint{5.979434in}{4.581016in}}%
\pgfusepath{stroke}%
\end{pgfscope}%
\begin{pgfscope}%
\definecolor{textcolor}{rgb}{0.150000,0.150000,0.150000}%
\pgfsetstrokecolor{textcolor}%
\pgfsetfillcolor{textcolor}%
\pgftext[x=5.979434in,y=4.082908in,,top]{\color{textcolor}\rmfamily\fontsize{10.000000}{12.000000}\selectfont 2017}%
\end{pgfscope}%
\begin{pgfscope}%
\pgfpathrectangle{\pgfqpoint{3.902254in}{4.180131in}}{\pgfqpoint{2.583333in}{0.400885in}}%
\pgfusepath{clip}%
\pgfsetroundcap%
\pgfsetroundjoin%
\pgfsetlinewidth{0.803000pt}%
\definecolor{currentstroke}{rgb}{1.000000,1.000000,1.000000}%
\pgfsetstrokecolor{currentstroke}%
\pgfsetdash{}{0pt}%
\pgfpathmoveto{\pgfqpoint{6.371385in}{4.180131in}}%
\pgfpathlineto{\pgfqpoint{6.371385in}{4.581016in}}%
\pgfusepath{stroke}%
\end{pgfscope}%
\begin{pgfscope}%
\definecolor{textcolor}{rgb}{0.150000,0.150000,0.150000}%
\pgfsetstrokecolor{textcolor}%
\pgfsetfillcolor{textcolor}%
\pgftext[x=6.371385in,y=4.082908in,,top]{\color{textcolor}\rmfamily\fontsize{10.000000}{12.000000}\selectfont 2018}%
\end{pgfscope}%
\begin{pgfscope}%
\pgfpathrectangle{\pgfqpoint{3.902254in}{4.180131in}}{\pgfqpoint{2.583333in}{0.400885in}}%
\pgfusepath{clip}%
\pgfsetroundcap%
\pgfsetroundjoin%
\pgfsetlinewidth{0.803000pt}%
\definecolor{currentstroke}{rgb}{1.000000,1.000000,1.000000}%
\pgfsetstrokecolor{currentstroke}%
\pgfsetdash{}{0pt}%
\pgfpathmoveto{\pgfqpoint{3.902254in}{4.336522in}}%
\pgfpathlineto{\pgfqpoint{6.485588in}{4.336522in}}%
\pgfusepath{stroke}%
\end{pgfscope}%
\begin{pgfscope}%
\definecolor{textcolor}{rgb}{0.150000,0.150000,0.150000}%
\pgfsetstrokecolor{textcolor}%
\pgfsetfillcolor{textcolor}%
\pgftext[x=3.716667in,y=4.283760in,left,base]{\color{textcolor}\rmfamily\fontsize{10.000000}{12.000000}\selectfont 1}%
\end{pgfscope}%
\begin{pgfscope}%
\pgfpathrectangle{\pgfqpoint{3.902254in}{4.180131in}}{\pgfqpoint{2.583333in}{0.400885in}}%
\pgfusepath{clip}%
\pgfsetroundcap%
\pgfsetroundjoin%
\pgfsetlinewidth{0.803000pt}%
\definecolor{currentstroke}{rgb}{1.000000,1.000000,1.000000}%
\pgfsetstrokecolor{currentstroke}%
\pgfsetdash{}{0pt}%
\pgfpathmoveto{\pgfqpoint{3.902254in}{4.518076in}}%
\pgfpathlineto{\pgfqpoint{6.485588in}{4.518076in}}%
\pgfusepath{stroke}%
\end{pgfscope}%
\begin{pgfscope}%
\definecolor{textcolor}{rgb}{0.150000,0.150000,0.150000}%
\pgfsetstrokecolor{textcolor}%
\pgfsetfillcolor{textcolor}%
\pgftext[x=3.716667in,y=4.465315in,left,base]{\color{textcolor}\rmfamily\fontsize{10.000000}{12.000000}\selectfont 2}%
\end{pgfscope}%
\begin{pgfscope}%
\pgfpathrectangle{\pgfqpoint{3.902254in}{4.180131in}}{\pgfqpoint{2.583333in}{0.400885in}}%
\pgfusepath{clip}%
\pgfsetroundcap%
\pgfsetroundjoin%
\pgfsetlinewidth{1.505625pt}%
\definecolor{currentstroke}{rgb}{1.000000,0.498039,0.054902}%
\pgfsetstrokecolor{currentstroke}%
\pgfsetstrokeopacity{0.300000}%
\pgfsetdash{}{0pt}%
\pgfpathmoveto{\pgfqpoint{4.019678in}{4.336522in}}%
\pgfpathlineto{\pgfqpoint{4.020752in}{4.336647in}}%
\pgfpathlineto{\pgfqpoint{4.021826in}{4.338745in}}%
\pgfpathlineto{\pgfqpoint{4.022900in}{4.336731in}}%
\pgfpathlineto{\pgfqpoint{4.026121in}{4.337193in}}%
\pgfpathlineto{\pgfqpoint{4.028269in}{4.339290in}}%
\pgfpathlineto{\pgfqpoint{4.029343in}{4.341933in}}%
\pgfpathlineto{\pgfqpoint{4.034712in}{4.344072in}}%
\pgfpathlineto{\pgfqpoint{4.036860in}{4.346841in}}%
\pgfpathlineto{\pgfqpoint{4.037934in}{4.343401in}}%
\pgfpathlineto{\pgfqpoint{4.042229in}{4.340339in}}%
\pgfpathlineto{\pgfqpoint{4.043303in}{4.343905in}}%
\pgfpathlineto{\pgfqpoint{4.048672in}{4.339961in}}%
\pgfpathlineto{\pgfqpoint{4.050820in}{4.345582in}}%
\pgfpathlineto{\pgfqpoint{4.051894in}{4.347638in}}%
\pgfpathlineto{\pgfqpoint{4.052967in}{4.351749in}}%
\pgfpathlineto{\pgfqpoint{4.056189in}{4.350071in}}%
\pgfpathlineto{\pgfqpoint{4.057263in}{4.351288in}}%
\pgfpathlineto{\pgfqpoint{4.058337in}{4.349442in}}%
\pgfpathlineto{\pgfqpoint{4.059410in}{4.351959in}}%
\pgfpathlineto{\pgfqpoint{4.060484in}{4.350071in}}%
\pgfpathlineto{\pgfqpoint{4.063706in}{4.351036in}}%
\pgfpathlineto{\pgfqpoint{4.064780in}{4.350658in}}%
\pgfpathlineto{\pgfqpoint{4.065853in}{4.348980in}}%
\pgfpathlineto{\pgfqpoint{4.066927in}{4.354056in}}%
\pgfpathlineto{\pgfqpoint{4.074444in}{4.353301in}}%
\pgfpathlineto{\pgfqpoint{4.075518in}{4.355818in}}%
\pgfpathlineto{\pgfqpoint{4.078740in}{4.359006in}}%
\pgfpathlineto{\pgfqpoint{4.079813in}{4.357412in}}%
\pgfpathlineto{\pgfqpoint{4.080887in}{4.354140in}}%
\pgfpathlineto{\pgfqpoint{4.081961in}{4.356699in}}%
\pgfpathlineto{\pgfqpoint{4.083035in}{4.354518in}}%
\pgfpathlineto{\pgfqpoint{4.086256in}{4.354434in}}%
\pgfpathlineto{\pgfqpoint{4.087330in}{4.349819in}}%
\pgfpathlineto{\pgfqpoint{4.090552in}{4.355315in}}%
\pgfpathlineto{\pgfqpoint{4.093773in}{4.353679in}}%
\pgfpathlineto{\pgfqpoint{4.096995in}{4.368570in}}%
\pgfpathlineto{\pgfqpoint{4.098069in}{4.367941in}}%
\pgfpathlineto{\pgfqpoint{4.101290in}{4.370626in}}%
\pgfpathlineto{\pgfqpoint{4.102364in}{4.369284in}}%
\pgfpathlineto{\pgfqpoint{4.105585in}{4.370542in}}%
\pgfpathlineto{\pgfqpoint{4.108807in}{4.375870in}}%
\pgfpathlineto{\pgfqpoint{4.109881in}{4.374234in}}%
\pgfpathlineto{\pgfqpoint{4.110955in}{4.377380in}}%
\pgfpathlineto{\pgfqpoint{4.112029in}{4.372975in}}%
\pgfpathlineto{\pgfqpoint{4.113102in}{4.372849in}}%
\pgfpathlineto{\pgfqpoint{4.116324in}{4.373436in}}%
\pgfpathlineto{\pgfqpoint{4.117398in}{4.375618in}}%
\pgfpathlineto{\pgfqpoint{4.118472in}{4.371759in}}%
\pgfpathlineto{\pgfqpoint{4.119545in}{4.374527in}}%
\pgfpathlineto{\pgfqpoint{4.123841in}{4.370962in}}%
\pgfpathlineto{\pgfqpoint{4.124915in}{4.367270in}}%
\pgfpathlineto{\pgfqpoint{4.127062in}{4.374275in}}%
\pgfpathlineto{\pgfqpoint{4.128136in}{4.371423in}}%
\pgfpathlineto{\pgfqpoint{4.133505in}{4.374275in}}%
\pgfpathlineto{\pgfqpoint{4.134579in}{4.372514in}}%
\pgfpathlineto{\pgfqpoint{4.135653in}{4.372052in}}%
\pgfpathlineto{\pgfqpoint{4.138874in}{4.371591in}}%
\pgfpathlineto{\pgfqpoint{4.139948in}{4.372723in}}%
\pgfpathlineto{\pgfqpoint{4.142096in}{4.380148in}}%
\pgfpathlineto{\pgfqpoint{4.143170in}{4.382330in}}%
\pgfpathlineto{\pgfqpoint{4.146391in}{4.382497in}}%
\pgfpathlineto{\pgfqpoint{4.147465in}{4.385350in}}%
\pgfpathlineto{\pgfqpoint{4.148539in}{4.385686in}}%
\pgfpathlineto{\pgfqpoint{4.149613in}{4.384930in}}%
\pgfpathlineto{\pgfqpoint{4.150687in}{4.382078in}}%
\pgfpathlineto{\pgfqpoint{4.153908in}{4.382078in}}%
\pgfpathlineto{\pgfqpoint{4.154982in}{4.381281in}}%
\pgfpathlineto{\pgfqpoint{4.156056in}{4.379645in}}%
\pgfpathlineto{\pgfqpoint{4.157130in}{4.379519in}}%
\pgfpathlineto{\pgfqpoint{4.158204in}{4.380358in}}%
\pgfpathlineto{\pgfqpoint{4.163573in}{4.371842in}}%
\pgfpathlineto{\pgfqpoint{4.164647in}{4.365256in}}%
\pgfpathlineto{\pgfqpoint{4.165720in}{4.364334in}}%
\pgfpathlineto{\pgfqpoint{4.168942in}{4.367815in}}%
\pgfpathlineto{\pgfqpoint{4.170016in}{4.367899in}}%
\pgfpathlineto{\pgfqpoint{4.171090in}{4.366473in}}%
\pgfpathlineto{\pgfqpoint{4.172163in}{4.367857in}}%
\pgfpathlineto{\pgfqpoint{4.173237in}{4.365844in}}%
\pgfpathlineto{\pgfqpoint{4.177533in}{4.368696in}}%
\pgfpathlineto{\pgfqpoint{4.178607in}{4.364543in}}%
\pgfpathlineto{\pgfqpoint{4.179680in}{4.365928in}}%
\pgfpathlineto{\pgfqpoint{4.180754in}{4.356867in}}%
\pgfpathlineto{\pgfqpoint{4.183976in}{4.357160in}}%
\pgfpathlineto{\pgfqpoint{4.185050in}{4.358964in}}%
\pgfpathlineto{\pgfqpoint{4.186123in}{4.364250in}}%
\pgfpathlineto{\pgfqpoint{4.187197in}{4.363704in}}%
\pgfpathlineto{\pgfqpoint{4.188271in}{4.366053in}}%
\pgfpathlineto{\pgfqpoint{4.191493in}{4.363201in}}%
\pgfpathlineto{\pgfqpoint{4.192566in}{4.368403in}}%
\pgfpathlineto{\pgfqpoint{4.193640in}{4.363201in}}%
\pgfpathlineto{\pgfqpoint{4.194714in}{4.362991in}}%
\pgfpathlineto{\pgfqpoint{4.195788in}{4.367648in}}%
\pgfpathlineto{\pgfqpoint{4.199009in}{4.366012in}}%
\pgfpathlineto{\pgfqpoint{4.200083in}{4.370123in}}%
\pgfpathlineto{\pgfqpoint{4.201157in}{4.372010in}}%
\pgfpathlineto{\pgfqpoint{4.202231in}{4.367773in}}%
\pgfpathlineto{\pgfqpoint{4.203305in}{4.369577in}}%
\pgfpathlineto{\pgfqpoint{4.206526in}{4.366767in}}%
\pgfpathlineto{\pgfqpoint{4.207600in}{4.367018in}}%
\pgfpathlineto{\pgfqpoint{4.208674in}{4.369913in}}%
\pgfpathlineto{\pgfqpoint{4.209748in}{4.369200in}}%
\pgfpathlineto{\pgfqpoint{4.210822in}{4.374947in}}%
\pgfpathlineto{\pgfqpoint{4.214043in}{4.378093in}}%
\pgfpathlineto{\pgfqpoint{4.215117in}{4.380232in}}%
\pgfpathlineto{\pgfqpoint{4.217265in}{4.379645in}}%
\pgfpathlineto{\pgfqpoint{4.218339in}{4.377254in}}%
\pgfpathlineto{\pgfqpoint{4.222634in}{4.376373in}}%
\pgfpathlineto{\pgfqpoint{4.223708in}{4.375198in}}%
\pgfpathlineto{\pgfqpoint{4.224782in}{4.370836in}}%
\pgfpathlineto{\pgfqpoint{4.225855in}{4.374611in}}%
\pgfpathlineto{\pgfqpoint{4.229077in}{4.377296in}}%
\pgfpathlineto{\pgfqpoint{4.230151in}{4.377464in}}%
\pgfpathlineto{\pgfqpoint{4.231225in}{4.375995in}}%
\pgfpathlineto{\pgfqpoint{4.232298in}{4.368193in}}%
\pgfpathlineto{\pgfqpoint{4.233372in}{4.366599in}}%
\pgfpathlineto{\pgfqpoint{4.237668in}{4.365886in}}%
\pgfpathlineto{\pgfqpoint{4.238741in}{4.367480in}}%
\pgfpathlineto{\pgfqpoint{4.239815in}{4.373982in}}%
\pgfpathlineto{\pgfqpoint{4.240889in}{4.376876in}}%
\pgfpathlineto{\pgfqpoint{4.244111in}{4.375995in}}%
\pgfpathlineto{\pgfqpoint{4.247332in}{4.369158in}}%
\pgfpathlineto{\pgfqpoint{4.248406in}{4.373395in}}%
\pgfpathlineto{\pgfqpoint{4.251628in}{4.371465in}}%
\pgfpathlineto{\pgfqpoint{4.253775in}{4.374527in}}%
\pgfpathlineto{\pgfqpoint{4.254849in}{4.369074in}}%
\pgfpathlineto{\pgfqpoint{4.255923in}{4.366725in}}%
\pgfpathlineto{\pgfqpoint{4.260218in}{4.367606in}}%
\pgfpathlineto{\pgfqpoint{4.263440in}{4.373311in}}%
\pgfpathlineto{\pgfqpoint{4.267735in}{4.369577in}}%
\pgfpathlineto{\pgfqpoint{4.268809in}{4.370416in}}%
\pgfpathlineto{\pgfqpoint{4.269883in}{4.368906in}}%
\pgfpathlineto{\pgfqpoint{4.270957in}{4.372933in}}%
\pgfpathlineto{\pgfqpoint{4.274178in}{4.372681in}}%
\pgfpathlineto{\pgfqpoint{4.275252in}{4.373227in}}%
\pgfpathlineto{\pgfqpoint{4.277400in}{4.371759in}}%
\pgfpathlineto{\pgfqpoint{4.278473in}{4.376037in}}%
\pgfpathlineto{\pgfqpoint{4.282769in}{4.377212in}}%
\pgfpathlineto{\pgfqpoint{4.283843in}{4.371800in}}%
\pgfpathlineto{\pgfqpoint{4.285990in}{4.373856in}}%
\pgfpathlineto{\pgfqpoint{4.289212in}{4.373059in}}%
\pgfpathlineto{\pgfqpoint{4.290286in}{4.371968in}}%
\pgfpathlineto{\pgfqpoint{4.291360in}{4.372136in}}%
\pgfpathlineto{\pgfqpoint{4.292433in}{4.378848in}}%
\pgfpathlineto{\pgfqpoint{4.293507in}{4.379687in}}%
\pgfpathlineto{\pgfqpoint{4.296729in}{4.379351in}}%
\pgfpathlineto{\pgfqpoint{4.297803in}{4.377422in}}%
\pgfpathlineto{\pgfqpoint{4.298876in}{4.377422in}}%
\pgfpathlineto{\pgfqpoint{4.301024in}{4.374359in}}%
\pgfpathlineto{\pgfqpoint{4.304246in}{4.373646in}}%
\pgfpathlineto{\pgfqpoint{4.305319in}{4.371591in}}%
\pgfpathlineto{\pgfqpoint{4.306393in}{4.367815in}}%
\pgfpathlineto{\pgfqpoint{4.308541in}{4.370542in}}%
\pgfpathlineto{\pgfqpoint{4.311762in}{4.373856in}}%
\pgfpathlineto{\pgfqpoint{4.312836in}{4.371759in}}%
\pgfpathlineto{\pgfqpoint{4.313910in}{4.373395in}}%
\pgfpathlineto{\pgfqpoint{4.314984in}{4.377044in}}%
\pgfpathlineto{\pgfqpoint{4.316058in}{4.377799in}}%
\pgfpathlineto{\pgfqpoint{4.319279in}{4.378764in}}%
\pgfpathlineto{\pgfqpoint{4.321427in}{4.375534in}}%
\pgfpathlineto{\pgfqpoint{4.322501in}{4.377464in}}%
\pgfpathlineto{\pgfqpoint{4.323575in}{4.375240in}}%
\pgfpathlineto{\pgfqpoint{4.326796in}{4.374108in}}%
\pgfpathlineto{\pgfqpoint{4.328944in}{4.380861in}}%
\pgfpathlineto{\pgfqpoint{4.330018in}{4.374192in}}%
\pgfpathlineto{\pgfqpoint{4.331092in}{4.371339in}}%
\pgfpathlineto{\pgfqpoint{4.334313in}{4.370878in}}%
\pgfpathlineto{\pgfqpoint{4.335387in}{4.365676in}}%
\pgfpathlineto{\pgfqpoint{4.336461in}{4.365131in}}%
\pgfpathlineto{\pgfqpoint{4.338608in}{4.367102in}}%
\pgfpathlineto{\pgfqpoint{4.343978in}{4.367941in}}%
\pgfpathlineto{\pgfqpoint{4.345051in}{4.371297in}}%
\pgfpathlineto{\pgfqpoint{4.349347in}{4.369409in}}%
\pgfpathlineto{\pgfqpoint{4.350421in}{4.372556in}}%
\pgfpathlineto{\pgfqpoint{4.351495in}{4.366431in}}%
\pgfpathlineto{\pgfqpoint{4.352568in}{4.366389in}}%
\pgfpathlineto{\pgfqpoint{4.353642in}{4.367396in}}%
\pgfpathlineto{\pgfqpoint{4.356864in}{4.366221in}}%
\pgfpathlineto{\pgfqpoint{4.359011in}{4.359090in}}%
\pgfpathlineto{\pgfqpoint{4.360085in}{4.359048in}}%
\pgfpathlineto{\pgfqpoint{4.361159in}{4.361565in}}%
\pgfpathlineto{\pgfqpoint{4.364381in}{4.365131in}}%
\pgfpathlineto{\pgfqpoint{4.365454in}{4.367522in}}%
\pgfpathlineto{\pgfqpoint{4.366528in}{4.367983in}}%
\pgfpathlineto{\pgfqpoint{4.368676in}{4.369997in}}%
\pgfpathlineto{\pgfqpoint{4.371897in}{4.366851in}}%
\pgfpathlineto{\pgfqpoint{4.372971in}{4.362110in}}%
\pgfpathlineto{\pgfqpoint{4.374045in}{4.366179in}}%
\pgfpathlineto{\pgfqpoint{4.375119in}{4.367731in}}%
\pgfpathlineto{\pgfqpoint{4.376193in}{4.367648in}}%
\pgfpathlineto{\pgfqpoint{4.379414in}{4.368025in}}%
\pgfpathlineto{\pgfqpoint{4.380488in}{4.367438in}}%
\pgfpathlineto{\pgfqpoint{4.381562in}{4.369577in}}%
\pgfpathlineto{\pgfqpoint{4.382636in}{4.368487in}}%
\pgfpathlineto{\pgfqpoint{4.383710in}{4.370374in}}%
\pgfpathlineto{\pgfqpoint{4.386931in}{4.370920in}}%
\pgfpathlineto{\pgfqpoint{4.388005in}{4.372094in}}%
\pgfpathlineto{\pgfqpoint{4.389079in}{4.374401in}}%
\pgfpathlineto{\pgfqpoint{4.390153in}{4.374611in}}%
\pgfpathlineto{\pgfqpoint{4.391227in}{4.370500in}}%
\pgfpathlineto{\pgfqpoint{4.394448in}{4.372891in}}%
\pgfpathlineto{\pgfqpoint{4.395522in}{4.374989in}}%
\pgfpathlineto{\pgfqpoint{4.396596in}{4.371045in}}%
\pgfpathlineto{\pgfqpoint{4.397670in}{4.373395in}}%
\pgfpathlineto{\pgfqpoint{4.398743in}{4.374317in}}%
\pgfpathlineto{\pgfqpoint{4.401965in}{4.373856in}}%
\pgfpathlineto{\pgfqpoint{4.404113in}{4.372346in}}%
\pgfpathlineto{\pgfqpoint{4.405186in}{4.370668in}}%
\pgfpathlineto{\pgfqpoint{4.406260in}{4.370584in}}%
\pgfpathlineto{\pgfqpoint{4.409482in}{4.373688in}}%
\pgfpathlineto{\pgfqpoint{4.411629in}{4.379309in}}%
\pgfpathlineto{\pgfqpoint{4.412703in}{4.380232in}}%
\pgfpathlineto{\pgfqpoint{4.413777in}{4.382581in}}%
\pgfpathlineto{\pgfqpoint{4.416999in}{4.383546in}}%
\pgfpathlineto{\pgfqpoint{4.418073in}{4.384847in}}%
\pgfpathlineto{\pgfqpoint{4.419146in}{4.385056in}}%
\pgfpathlineto{\pgfqpoint{4.421294in}{4.388790in}}%
\pgfpathlineto{\pgfqpoint{4.424516in}{4.388664in}}%
\pgfpathlineto{\pgfqpoint{4.425589in}{4.386986in}}%
\pgfpathlineto{\pgfqpoint{4.426663in}{4.386441in}}%
\pgfpathlineto{\pgfqpoint{4.427737in}{4.386902in}}%
\pgfpathlineto{\pgfqpoint{4.428811in}{4.383211in}}%
\pgfpathlineto{\pgfqpoint{4.433106in}{4.381784in}}%
\pgfpathlineto{\pgfqpoint{4.434180in}{4.379980in}}%
\pgfpathlineto{\pgfqpoint{4.436328in}{4.382162in}}%
\pgfpathlineto{\pgfqpoint{4.439549in}{4.380526in}}%
\pgfpathlineto{\pgfqpoint{4.440623in}{4.381952in}}%
\pgfpathlineto{\pgfqpoint{4.441697in}{4.381491in}}%
\pgfpathlineto{\pgfqpoint{4.442771in}{4.379519in}}%
\pgfpathlineto{\pgfqpoint{4.443845in}{4.383714in}}%
\pgfpathlineto{\pgfqpoint{4.447066in}{4.381826in}}%
\pgfpathlineto{\pgfqpoint{4.448140in}{4.386566in}}%
\pgfpathlineto{\pgfqpoint{4.449214in}{4.386189in}}%
\pgfpathlineto{\pgfqpoint{4.450288in}{4.392230in}}%
\pgfpathlineto{\pgfqpoint{4.451361in}{4.390929in}}%
\pgfpathlineto{\pgfqpoint{4.454583in}{4.391642in}}%
\pgfpathlineto{\pgfqpoint{4.455657in}{4.392481in}}%
\pgfpathlineto{\pgfqpoint{4.456731in}{4.392062in}}%
\pgfpathlineto{\pgfqpoint{4.457805in}{4.392985in}}%
\pgfpathlineto{\pgfqpoint{4.458878in}{4.390510in}}%
\pgfpathlineto{\pgfqpoint{4.463174in}{4.392775in}}%
\pgfpathlineto{\pgfqpoint{4.465321in}{4.389964in}}%
\pgfpathlineto{\pgfqpoint{4.466395in}{4.393866in}}%
\pgfpathlineto{\pgfqpoint{4.470691in}{4.391391in}}%
\pgfpathlineto{\pgfqpoint{4.471764in}{4.393614in}}%
\pgfpathlineto{\pgfqpoint{4.472838in}{4.392272in}}%
\pgfpathlineto{\pgfqpoint{4.477134in}{4.395082in}}%
\pgfpathlineto{\pgfqpoint{4.478207in}{4.399780in}}%
\pgfpathlineto{\pgfqpoint{4.479281in}{4.401836in}}%
\pgfpathlineto{\pgfqpoint{4.480355in}{4.401458in}}%
\pgfpathlineto{\pgfqpoint{4.481429in}{4.402004in}}%
\pgfpathlineto{\pgfqpoint{4.484650in}{4.405066in}}%
\pgfpathlineto{\pgfqpoint{4.485724in}{4.404269in}}%
\pgfpathlineto{\pgfqpoint{4.487872in}{4.404604in}}%
\pgfpathlineto{\pgfqpoint{4.488946in}{4.407331in}}%
\pgfpathlineto{\pgfqpoint{4.492167in}{4.406115in}}%
\pgfpathlineto{\pgfqpoint{4.493241in}{4.403640in}}%
\pgfpathlineto{\pgfqpoint{4.494315in}{4.406954in}}%
\pgfpathlineto{\pgfqpoint{4.495389in}{4.404730in}}%
\pgfpathlineto{\pgfqpoint{4.496463in}{4.407793in}}%
\pgfpathlineto{\pgfqpoint{4.499684in}{4.407373in}}%
\pgfpathlineto{\pgfqpoint{4.500758in}{4.411442in}}%
\pgfpathlineto{\pgfqpoint{4.501832in}{4.411358in}}%
\pgfpathlineto{\pgfqpoint{4.502906in}{4.412533in}}%
\pgfpathlineto{\pgfqpoint{4.507201in}{4.411778in}}%
\pgfpathlineto{\pgfqpoint{4.508275in}{4.413246in}}%
\pgfpathlineto{\pgfqpoint{4.509349in}{4.408673in}}%
\pgfpathlineto{\pgfqpoint{4.510423in}{4.410519in}}%
\pgfpathlineto{\pgfqpoint{4.511496in}{4.405024in}}%
\pgfpathlineto{\pgfqpoint{4.514718in}{4.406199in}}%
\pgfpathlineto{\pgfqpoint{4.515792in}{4.404730in}}%
\pgfpathlineto{\pgfqpoint{4.519013in}{4.406492in}}%
\pgfpathlineto{\pgfqpoint{4.522235in}{4.400452in}}%
\pgfpathlineto{\pgfqpoint{4.523309in}{4.402339in}}%
\pgfpathlineto{\pgfqpoint{4.524383in}{4.400577in}}%
\pgfpathlineto{\pgfqpoint{4.525456in}{4.404059in}}%
\pgfpathlineto{\pgfqpoint{4.526530in}{4.412449in}}%
\pgfpathlineto{\pgfqpoint{4.529752in}{4.410268in}}%
\pgfpathlineto{\pgfqpoint{4.530826in}{4.413288in}}%
\pgfpathlineto{\pgfqpoint{4.531899in}{4.413162in}}%
\pgfpathlineto{\pgfqpoint{4.532973in}{4.415889in}}%
\pgfpathlineto{\pgfqpoint{4.534047in}{4.414420in}}%
\pgfpathlineto{\pgfqpoint{4.537269in}{4.413959in}}%
\pgfpathlineto{\pgfqpoint{4.538342in}{4.416937in}}%
\pgfpathlineto{\pgfqpoint{4.539416in}{4.416434in}}%
\pgfpathlineto{\pgfqpoint{4.541564in}{4.423901in}}%
\pgfpathlineto{\pgfqpoint{4.544785in}{4.423272in}}%
\pgfpathlineto{\pgfqpoint{4.546933in}{4.424237in}}%
\pgfpathlineto{\pgfqpoint{4.552302in}{4.422223in}}%
\pgfpathlineto{\pgfqpoint{4.554450in}{4.433675in}}%
\pgfpathlineto{\pgfqpoint{4.555524in}{4.431578in}}%
\pgfpathlineto{\pgfqpoint{4.556598in}{4.435772in}}%
\pgfpathlineto{\pgfqpoint{4.559819in}{4.439883in}}%
\pgfpathlineto{\pgfqpoint{4.560893in}{4.442610in}}%
\pgfpathlineto{\pgfqpoint{4.561967in}{4.440051in}}%
\pgfpathlineto{\pgfqpoint{4.563041in}{4.441016in}}%
\pgfpathlineto{\pgfqpoint{4.564115in}{4.443239in}}%
\pgfpathlineto{\pgfqpoint{4.568410in}{4.446637in}}%
\pgfpathlineto{\pgfqpoint{4.569484in}{4.445379in}}%
\pgfpathlineto{\pgfqpoint{4.570558in}{4.446553in}}%
\pgfpathlineto{\pgfqpoint{4.571631in}{4.444917in}}%
\pgfpathlineto{\pgfqpoint{4.574853in}{4.447812in}}%
\pgfpathlineto{\pgfqpoint{4.575927in}{4.446260in}}%
\pgfpathlineto{\pgfqpoint{4.577001in}{4.441268in}}%
\pgfpathlineto{\pgfqpoint{4.579148in}{4.453810in}}%
\pgfpathlineto{\pgfqpoint{4.582370in}{4.454733in}}%
\pgfpathlineto{\pgfqpoint{4.584517in}{4.441100in}}%
\pgfpathlineto{\pgfqpoint{4.585591in}{4.442988in}}%
\pgfpathlineto{\pgfqpoint{4.586665in}{4.434430in}}%
\pgfpathlineto{\pgfqpoint{4.589887in}{4.437744in}}%
\pgfpathlineto{\pgfqpoint{4.590961in}{4.442107in}}%
\pgfpathlineto{\pgfqpoint{4.592034in}{4.439254in}}%
\pgfpathlineto{\pgfqpoint{4.593108in}{4.434220in}}%
\pgfpathlineto{\pgfqpoint{4.594182in}{4.435730in}}%
\pgfpathlineto{\pgfqpoint{4.597404in}{4.430739in}}%
\pgfpathlineto{\pgfqpoint{4.600625in}{4.442652in}}%
\pgfpathlineto{\pgfqpoint{4.601699in}{4.441268in}}%
\pgfpathlineto{\pgfqpoint{4.604920in}{4.444624in}}%
\pgfpathlineto{\pgfqpoint{4.605994in}{4.441603in}}%
\pgfpathlineto{\pgfqpoint{4.607068in}{4.441436in}}%
\pgfpathlineto{\pgfqpoint{4.609216in}{4.448105in}}%
\pgfpathlineto{\pgfqpoint{4.612437in}{4.450916in}}%
\pgfpathlineto{\pgfqpoint{4.613511in}{4.453097in}}%
\pgfpathlineto{\pgfqpoint{4.614585in}{4.448189in}}%
\pgfpathlineto{\pgfqpoint{4.615659in}{4.450496in}}%
\pgfpathlineto{\pgfqpoint{4.616733in}{4.455866in}}%
\pgfpathlineto{\pgfqpoint{4.619954in}{4.454020in}}%
\pgfpathlineto{\pgfqpoint{4.621028in}{4.455614in}}%
\pgfpathlineto{\pgfqpoint{4.622102in}{4.449993in}}%
\pgfpathlineto{\pgfqpoint{4.623176in}{4.439254in}}%
\pgfpathlineto{\pgfqpoint{4.624249in}{4.439464in}}%
\pgfpathlineto{\pgfqpoint{4.627471in}{4.441981in}}%
\pgfpathlineto{\pgfqpoint{4.628545in}{4.440806in}}%
\pgfpathlineto{\pgfqpoint{4.629619in}{4.444456in}}%
\pgfpathlineto{\pgfqpoint{4.630693in}{4.446008in}}%
\pgfpathlineto{\pgfqpoint{4.631766in}{4.444372in}}%
\pgfpathlineto{\pgfqpoint{4.634988in}{4.443281in}}%
\pgfpathlineto{\pgfqpoint{4.636062in}{4.443785in}}%
\pgfpathlineto{\pgfqpoint{4.637136in}{4.438331in}}%
\pgfpathlineto{\pgfqpoint{4.638209in}{4.445463in}}%
\pgfpathlineto{\pgfqpoint{4.643579in}{4.446176in}}%
\pgfpathlineto{\pgfqpoint{4.644652in}{4.443617in}}%
\pgfpathlineto{\pgfqpoint{4.645726in}{4.447644in}}%
\pgfpathlineto{\pgfqpoint{4.646800in}{4.445001in}}%
\pgfpathlineto{\pgfqpoint{4.650022in}{4.444959in}}%
\pgfpathlineto{\pgfqpoint{4.651095in}{4.447770in}}%
\pgfpathlineto{\pgfqpoint{4.652169in}{4.446427in}}%
\pgfpathlineto{\pgfqpoint{4.653243in}{4.442610in}}%
\pgfpathlineto{\pgfqpoint{4.654317in}{4.443701in}}%
\pgfpathlineto{\pgfqpoint{4.657538in}{4.440513in}}%
\pgfpathlineto{\pgfqpoint{4.658612in}{4.440261in}}%
\pgfpathlineto{\pgfqpoint{4.659686in}{4.437073in}}%
\pgfpathlineto{\pgfqpoint{4.660760in}{4.438667in}}%
\pgfpathlineto{\pgfqpoint{4.661834in}{4.437870in}}%
\pgfpathlineto{\pgfqpoint{4.665055in}{4.437744in}}%
\pgfpathlineto{\pgfqpoint{4.666129in}{4.431200in}}%
\pgfpathlineto{\pgfqpoint{4.668277in}{4.432375in}}%
\pgfpathlineto{\pgfqpoint{4.669351in}{4.431200in}}%
\pgfpathlineto{\pgfqpoint{4.673646in}{4.433172in}}%
\pgfpathlineto{\pgfqpoint{4.675794in}{4.438457in}}%
\pgfpathlineto{\pgfqpoint{4.676868in}{4.436737in}}%
\pgfpathlineto{\pgfqpoint{4.680089in}{4.437912in}}%
\pgfpathlineto{\pgfqpoint{4.682237in}{4.443743in}}%
\pgfpathlineto{\pgfqpoint{4.687606in}{4.445379in}}%
\pgfpathlineto{\pgfqpoint{4.689754in}{4.453223in}}%
\pgfpathlineto{\pgfqpoint{4.690827in}{4.452930in}}%
\pgfpathlineto{\pgfqpoint{4.695123in}{4.448609in}}%
\pgfpathlineto{\pgfqpoint{4.696197in}{4.447183in}}%
\pgfpathlineto{\pgfqpoint{4.697271in}{4.446847in}}%
\pgfpathlineto{\pgfqpoint{4.698344in}{4.448147in}}%
\pgfpathlineto{\pgfqpoint{4.699418in}{4.446469in}}%
\pgfpathlineto{\pgfqpoint{4.702640in}{4.445043in}}%
\pgfpathlineto{\pgfqpoint{4.703714in}{4.446637in}}%
\pgfpathlineto{\pgfqpoint{4.705861in}{4.440177in}}%
\pgfpathlineto{\pgfqpoint{4.706935in}{4.441268in}}%
\pgfpathlineto{\pgfqpoint{4.712304in}{4.433214in}}%
\pgfpathlineto{\pgfqpoint{4.713378in}{4.442610in}}%
\pgfpathlineto{\pgfqpoint{4.714452in}{4.445463in}}%
\pgfpathlineto{\pgfqpoint{4.717673in}{4.448063in}}%
\pgfpathlineto{\pgfqpoint{4.718747in}{4.444917in}}%
\pgfpathlineto{\pgfqpoint{4.719821in}{4.449028in}}%
\pgfpathlineto{\pgfqpoint{4.720895in}{4.464088in}}%
\pgfpathlineto{\pgfqpoint{4.721969in}{4.465221in}}%
\pgfpathlineto{\pgfqpoint{4.725190in}{4.464759in}}%
\pgfpathlineto{\pgfqpoint{4.726264in}{4.466563in}}%
\pgfpathlineto{\pgfqpoint{4.727338in}{4.465640in}}%
\pgfpathlineto{\pgfqpoint{4.728412in}{4.466689in}}%
\pgfpathlineto{\pgfqpoint{4.729486in}{4.473275in}}%
\pgfpathlineto{\pgfqpoint{4.732707in}{4.473820in}}%
\pgfpathlineto{\pgfqpoint{4.733781in}{4.477302in}}%
\pgfpathlineto{\pgfqpoint{4.734855in}{4.475162in}}%
\pgfpathlineto{\pgfqpoint{4.735929in}{4.470129in}}%
\pgfpathlineto{\pgfqpoint{4.737003in}{4.471513in}}%
\pgfpathlineto{\pgfqpoint{4.741298in}{4.470506in}}%
\pgfpathlineto{\pgfqpoint{4.742372in}{4.471387in}}%
\pgfpathlineto{\pgfqpoint{4.743446in}{4.466731in}}%
\pgfpathlineto{\pgfqpoint{4.744519in}{4.470045in}}%
\pgfpathlineto{\pgfqpoint{4.747741in}{4.468618in}}%
\pgfpathlineto{\pgfqpoint{4.748815in}{4.467192in}}%
\pgfpathlineto{\pgfqpoint{4.750962in}{4.470506in}}%
\pgfpathlineto{\pgfqpoint{4.752036in}{4.473988in}}%
\pgfpathlineto{\pgfqpoint{4.755258in}{4.472226in}}%
\pgfpathlineto{\pgfqpoint{4.756332in}{4.472687in}}%
\pgfpathlineto{\pgfqpoint{4.757405in}{4.471890in}}%
\pgfpathlineto{\pgfqpoint{4.758479in}{4.478434in}}%
\pgfpathlineto{\pgfqpoint{4.759553in}{4.478057in}}%
\pgfpathlineto{\pgfqpoint{4.762775in}{4.480909in}}%
\pgfpathlineto{\pgfqpoint{4.764922in}{4.484727in}}%
\pgfpathlineto{\pgfqpoint{4.767070in}{4.485566in}}%
\pgfpathlineto{\pgfqpoint{4.770292in}{4.483594in}}%
\pgfpathlineto{\pgfqpoint{4.771365in}{4.480951in}}%
\pgfpathlineto{\pgfqpoint{4.773513in}{4.480700in}}%
\pgfpathlineto{\pgfqpoint{4.774587in}{4.486069in}}%
\pgfpathlineto{\pgfqpoint{4.777808in}{4.485356in}}%
\pgfpathlineto{\pgfqpoint{4.778882in}{4.483594in}}%
\pgfpathlineto{\pgfqpoint{4.779956in}{4.478225in}}%
\pgfpathlineto{\pgfqpoint{4.781030in}{4.475917in}}%
\pgfpathlineto{\pgfqpoint{4.782104in}{4.477386in}}%
\pgfpathlineto{\pgfqpoint{4.785325in}{4.480616in}}%
\pgfpathlineto{\pgfqpoint{4.786399in}{4.479022in}}%
\pgfpathlineto{\pgfqpoint{4.787473in}{4.486279in}}%
\pgfpathlineto{\pgfqpoint{4.788547in}{4.487915in}}%
\pgfpathlineto{\pgfqpoint{4.789621in}{4.492361in}}%
\pgfpathlineto{\pgfqpoint{4.797137in}{4.498612in}}%
\pgfpathlineto{\pgfqpoint{4.800359in}{4.500164in}}%
\pgfpathlineto{\pgfqpoint{4.801433in}{4.504527in}}%
\pgfpathlineto{\pgfqpoint{4.803581in}{4.499619in}}%
\pgfpathlineto{\pgfqpoint{4.804654in}{4.500751in}}%
\pgfpathlineto{\pgfqpoint{4.807876in}{4.500583in}}%
\pgfpathlineto{\pgfqpoint{4.808950in}{4.499283in}}%
\pgfpathlineto{\pgfqpoint{4.810024in}{4.500332in}}%
\pgfpathlineto{\pgfqpoint{4.815393in}{4.491019in}}%
\pgfpathlineto{\pgfqpoint{4.816467in}{4.491522in}}%
\pgfpathlineto{\pgfqpoint{4.817540in}{4.495885in}}%
\pgfpathlineto{\pgfqpoint{4.818614in}{4.494039in}}%
\pgfpathlineto{\pgfqpoint{4.819688in}{4.506372in}}%
\pgfpathlineto{\pgfqpoint{4.823983in}{4.504904in}}%
\pgfpathlineto{\pgfqpoint{4.825057in}{4.507002in}}%
\pgfpathlineto{\pgfqpoint{4.827205in}{4.490851in}}%
\pgfpathlineto{\pgfqpoint{4.830426in}{4.486111in}}%
\pgfpathlineto{\pgfqpoint{4.831500in}{4.489635in}}%
\pgfpathlineto{\pgfqpoint{4.832574in}{4.485440in}}%
\pgfpathlineto{\pgfqpoint{4.833648in}{4.489551in}}%
\pgfpathlineto{\pgfqpoint{4.834722in}{4.483384in}}%
\pgfpathlineto{\pgfqpoint{4.837943in}{4.475037in}}%
\pgfpathlineto{\pgfqpoint{4.839017in}{4.479441in}}%
\pgfpathlineto{\pgfqpoint{4.840091in}{4.478350in}}%
\pgfpathlineto{\pgfqpoint{4.842239in}{4.491019in}}%
\pgfpathlineto{\pgfqpoint{4.847608in}{4.498822in}}%
\pgfpathlineto{\pgfqpoint{4.848682in}{4.498360in}}%
\pgfpathlineto{\pgfqpoint{4.849756in}{4.498780in}}%
\pgfpathlineto{\pgfqpoint{4.854051in}{4.498822in}}%
\pgfpathlineto{\pgfqpoint{4.855125in}{4.498150in}}%
\pgfpathlineto{\pgfqpoint{4.856199in}{4.498822in}}%
\pgfpathlineto{\pgfqpoint{4.857272in}{4.497815in}}%
\pgfpathlineto{\pgfqpoint{4.860494in}{4.502303in}}%
\pgfpathlineto{\pgfqpoint{4.861568in}{4.502261in}}%
\pgfpathlineto{\pgfqpoint{4.862642in}{4.501590in}}%
\pgfpathlineto{\pgfqpoint{4.863715in}{4.503730in}}%
\pgfpathlineto{\pgfqpoint{4.864789in}{4.507589in}}%
\pgfpathlineto{\pgfqpoint{4.868011in}{4.502513in}}%
\pgfpathlineto{\pgfqpoint{4.869085in}{4.512707in}}%
\pgfpathlineto{\pgfqpoint{4.870159in}{4.510903in}}%
\pgfpathlineto{\pgfqpoint{4.871232in}{4.516230in}}%
\pgfpathlineto{\pgfqpoint{4.872306in}{4.517531in}}%
\pgfpathlineto{\pgfqpoint{4.875528in}{4.516901in}}%
\pgfpathlineto{\pgfqpoint{4.877675in}{4.513839in}}%
\pgfpathlineto{\pgfqpoint{4.878749in}{4.505282in}}%
\pgfpathlineto{\pgfqpoint{4.879823in}{4.503268in}}%
\pgfpathlineto{\pgfqpoint{4.884118in}{4.508763in}}%
\pgfpathlineto{\pgfqpoint{4.885192in}{4.505449in}}%
\pgfpathlineto{\pgfqpoint{4.886266in}{4.509141in}}%
\pgfpathlineto{\pgfqpoint{4.891635in}{4.505869in}}%
\pgfpathlineto{\pgfqpoint{4.892709in}{4.501297in}}%
\pgfpathlineto{\pgfqpoint{4.894857in}{4.504401in}}%
\pgfpathlineto{\pgfqpoint{4.898078in}{4.502723in}}%
\pgfpathlineto{\pgfqpoint{4.899152in}{4.507127in}}%
\pgfpathlineto{\pgfqpoint{4.900226in}{4.505072in}}%
\pgfpathlineto{\pgfqpoint{4.901300in}{4.507295in}}%
\pgfpathlineto{\pgfqpoint{4.902374in}{4.500290in}}%
\pgfpathlineto{\pgfqpoint{4.905595in}{4.490348in}}%
\pgfpathlineto{\pgfqpoint{4.906669in}{4.489928in}}%
\pgfpathlineto{\pgfqpoint{4.907743in}{4.498570in}}%
\pgfpathlineto{\pgfqpoint{4.908817in}{4.485524in}}%
\pgfpathlineto{\pgfqpoint{4.909891in}{4.482378in}}%
\pgfpathlineto{\pgfqpoint{4.913112in}{4.486069in}}%
\pgfpathlineto{\pgfqpoint{4.914186in}{4.488167in}}%
\pgfpathlineto{\pgfqpoint{4.915260in}{4.493452in}}%
\pgfpathlineto{\pgfqpoint{4.916334in}{4.488880in}}%
\pgfpathlineto{\pgfqpoint{4.920629in}{4.490600in}}%
\pgfpathlineto{\pgfqpoint{4.921703in}{4.492152in}}%
\pgfpathlineto{\pgfqpoint{4.922777in}{4.492403in}}%
\pgfpathlineto{\pgfqpoint{4.923850in}{4.493494in}}%
\pgfpathlineto{\pgfqpoint{4.924924in}{4.492026in}}%
\pgfpathlineto{\pgfqpoint{4.928146in}{4.492110in}}%
\pgfpathlineto{\pgfqpoint{4.929220in}{4.494962in}}%
\pgfpathlineto{\pgfqpoint{4.932441in}{4.491606in}}%
\pgfpathlineto{\pgfqpoint{4.935663in}{4.493284in}}%
\pgfpathlineto{\pgfqpoint{4.936737in}{4.488796in}}%
\pgfpathlineto{\pgfqpoint{4.937810in}{4.495675in}}%
\pgfpathlineto{\pgfqpoint{4.938884in}{4.498150in}}%
\pgfpathlineto{\pgfqpoint{4.943180in}{4.502177in}}%
\pgfpathlineto{\pgfqpoint{4.945327in}{4.497563in}}%
\pgfpathlineto{\pgfqpoint{4.946401in}{4.494207in}}%
\pgfpathlineto{\pgfqpoint{4.947475in}{4.493830in}}%
\pgfpathlineto{\pgfqpoint{4.950696in}{4.496221in}}%
\pgfpathlineto{\pgfqpoint{4.951770in}{4.492236in}}%
\pgfpathlineto{\pgfqpoint{4.952844in}{4.495214in}}%
\pgfpathlineto{\pgfqpoint{4.953918in}{4.496347in}}%
\pgfpathlineto{\pgfqpoint{4.959287in}{4.508889in}}%
\pgfpathlineto{\pgfqpoint{4.960361in}{4.507631in}}%
\pgfpathlineto{\pgfqpoint{4.962509in}{4.509309in}}%
\pgfpathlineto{\pgfqpoint{4.965730in}{4.510819in}}%
\pgfpathlineto{\pgfqpoint{4.966804in}{4.510190in}}%
\pgfpathlineto{\pgfqpoint{4.967878in}{4.510525in}}%
\pgfpathlineto{\pgfqpoint{4.968952in}{4.514343in}}%
\pgfpathlineto{\pgfqpoint{4.970026in}{4.522523in}}%
\pgfpathlineto{\pgfqpoint{4.973247in}{4.525081in}}%
\pgfpathlineto{\pgfqpoint{4.976469in}{4.521935in}}%
\pgfpathlineto{\pgfqpoint{4.977542in}{4.522271in}}%
\pgfpathlineto{\pgfqpoint{4.980764in}{4.520425in}}%
\pgfpathlineto{\pgfqpoint{4.981838in}{4.521558in}}%
\pgfpathlineto{\pgfqpoint{4.982912in}{4.525040in}}%
\pgfpathlineto{\pgfqpoint{4.983985in}{4.523152in}}%
\pgfpathlineto{\pgfqpoint{4.985059in}{4.524956in}}%
\pgfpathlineto{\pgfqpoint{4.988281in}{4.524830in}}%
\pgfpathlineto{\pgfqpoint{4.989355in}{4.520677in}}%
\pgfpathlineto{\pgfqpoint{4.990428in}{4.521474in}}%
\pgfpathlineto{\pgfqpoint{4.991502in}{4.520173in}}%
\pgfpathlineto{\pgfqpoint{4.992576in}{4.522607in}}%
\pgfpathlineto{\pgfqpoint{4.995798in}{4.522355in}}%
\pgfpathlineto{\pgfqpoint{4.996871in}{4.524243in}}%
\pgfpathlineto{\pgfqpoint{4.997945in}{4.523697in}}%
\pgfpathlineto{\pgfqpoint{4.999019in}{4.526130in}}%
\pgfpathlineto{\pgfqpoint{5.003314in}{4.524410in}}%
\pgfpathlineto{\pgfqpoint{5.004388in}{4.521222in}}%
\pgfpathlineto{\pgfqpoint{5.005462in}{4.522858in}}%
\pgfpathlineto{\pgfqpoint{5.006536in}{4.521768in}}%
\pgfpathlineto{\pgfqpoint{5.012979in}{4.521809in}}%
\pgfpathlineto{\pgfqpoint{5.014053in}{4.516062in}}%
\pgfpathlineto{\pgfqpoint{5.015127in}{4.518160in}}%
\pgfpathlineto{\pgfqpoint{5.018348in}{4.515643in}}%
\pgfpathlineto{\pgfqpoint{5.019422in}{4.517573in}}%
\pgfpathlineto{\pgfqpoint{5.021570in}{4.516692in}}%
\pgfpathlineto{\pgfqpoint{5.022644in}{4.511952in}}%
\pgfpathlineto{\pgfqpoint{5.026939in}{4.511113in}}%
\pgfpathlineto{\pgfqpoint{5.028013in}{4.508008in}}%
\pgfpathlineto{\pgfqpoint{5.030160in}{4.490767in}}%
\pgfpathlineto{\pgfqpoint{5.033382in}{4.492529in}}%
\pgfpathlineto{\pgfqpoint{5.034456in}{4.490348in}}%
\pgfpathlineto{\pgfqpoint{5.035530in}{4.490474in}}%
\pgfpathlineto{\pgfqpoint{5.036603in}{4.489006in}}%
\pgfpathlineto{\pgfqpoint{5.037677in}{4.494627in}}%
\pgfpathlineto{\pgfqpoint{5.040899in}{4.492613in}}%
\pgfpathlineto{\pgfqpoint{5.041973in}{4.493116in}}%
\pgfpathlineto{\pgfqpoint{5.043047in}{4.494375in}}%
\pgfpathlineto{\pgfqpoint{5.044120in}{4.493872in}}%
\pgfpathlineto{\pgfqpoint{5.045194in}{4.491271in}}%
\pgfpathlineto{\pgfqpoint{5.048416in}{4.493452in}}%
\pgfpathlineto{\pgfqpoint{5.049490in}{4.497186in}}%
\pgfpathlineto{\pgfqpoint{5.050563in}{4.498612in}}%
\pgfpathlineto{\pgfqpoint{5.051637in}{4.501129in}}%
\pgfpathlineto{\pgfqpoint{5.052711in}{4.500122in}}%
\pgfpathlineto{\pgfqpoint{5.055933in}{4.502891in}}%
\pgfpathlineto{\pgfqpoint{5.057006in}{4.501171in}}%
\pgfpathlineto{\pgfqpoint{5.058080in}{4.501506in}}%
\pgfpathlineto{\pgfqpoint{5.059154in}{4.500667in}}%
\pgfpathlineto{\pgfqpoint{5.060228in}{4.502723in}}%
\pgfpathlineto{\pgfqpoint{5.064523in}{4.503394in}}%
\pgfpathlineto{\pgfqpoint{5.065597in}{4.505030in}}%
\pgfpathlineto{\pgfqpoint{5.066671in}{4.503100in}}%
\pgfpathlineto{\pgfqpoint{5.067745in}{4.502933in}}%
\pgfpathlineto{\pgfqpoint{5.070966in}{4.500332in}}%
\pgfpathlineto{\pgfqpoint{5.072040in}{4.496305in}}%
\pgfpathlineto{\pgfqpoint{5.073114in}{4.498276in}}%
\pgfpathlineto{\pgfqpoint{5.074188in}{4.498318in}}%
\pgfpathlineto{\pgfqpoint{5.075262in}{4.495298in}}%
\pgfpathlineto{\pgfqpoint{5.078483in}{4.494291in}}%
\pgfpathlineto{\pgfqpoint{5.081705in}{4.504862in}}%
\pgfpathlineto{\pgfqpoint{5.082779in}{4.503310in}}%
\pgfpathlineto{\pgfqpoint{5.086000in}{4.500541in}}%
\pgfpathlineto{\pgfqpoint{5.087074in}{4.497899in}}%
\pgfpathlineto{\pgfqpoint{5.088148in}{4.498738in}}%
\pgfpathlineto{\pgfqpoint{5.089222in}{4.491900in}}%
\pgfpathlineto{\pgfqpoint{5.090295in}{4.498150in}}%
\pgfpathlineto{\pgfqpoint{5.093517in}{4.496514in}}%
\pgfpathlineto{\pgfqpoint{5.094591in}{4.494920in}}%
\pgfpathlineto{\pgfqpoint{5.095665in}{4.489173in}}%
\pgfpathlineto{\pgfqpoint{5.096738in}{4.489425in}}%
\pgfpathlineto{\pgfqpoint{5.097812in}{4.494459in}}%
\pgfpathlineto{\pgfqpoint{5.101034in}{4.493955in}}%
\pgfpathlineto{\pgfqpoint{5.102108in}{4.487411in}}%
\pgfpathlineto{\pgfqpoint{5.103181in}{4.495424in}}%
\pgfpathlineto{\pgfqpoint{5.105329in}{4.485985in}}%
\pgfpathlineto{\pgfqpoint{5.108551in}{4.477386in}}%
\pgfpathlineto{\pgfqpoint{5.109625in}{4.477218in}}%
\pgfpathlineto{\pgfqpoint{5.110698in}{4.470170in}}%
\pgfpathlineto{\pgfqpoint{5.111772in}{4.467486in}}%
\pgfpathlineto{\pgfqpoint{5.112846in}{4.476589in}}%
\pgfpathlineto{\pgfqpoint{5.116068in}{4.482168in}}%
\pgfpathlineto{\pgfqpoint{5.117141in}{4.488544in}}%
\pgfpathlineto{\pgfqpoint{5.118215in}{4.482000in}}%
\pgfpathlineto{\pgfqpoint{5.120363in}{4.491480in}}%
\pgfpathlineto{\pgfqpoint{5.123584in}{4.492403in}}%
\pgfpathlineto{\pgfqpoint{5.124658in}{4.497773in}}%
\pgfpathlineto{\pgfqpoint{5.126806in}{4.500583in}}%
\pgfpathlineto{\pgfqpoint{5.127880in}{4.505324in}}%
\pgfpathlineto{\pgfqpoint{5.131101in}{4.508805in}}%
\pgfpathlineto{\pgfqpoint{5.132175in}{4.510903in}}%
\pgfpathlineto{\pgfqpoint{5.133249in}{4.514930in}}%
\pgfpathlineto{\pgfqpoint{5.134323in}{4.511658in}}%
\pgfpathlineto{\pgfqpoint{5.135397in}{4.514301in}}%
\pgfpathlineto{\pgfqpoint{5.138618in}{4.514846in}}%
\pgfpathlineto{\pgfqpoint{5.139692in}{4.512287in}}%
\pgfpathlineto{\pgfqpoint{5.140766in}{4.511532in}}%
\pgfpathlineto{\pgfqpoint{5.142914in}{4.508134in}}%
\pgfpathlineto{\pgfqpoint{5.146135in}{4.505995in}}%
\pgfpathlineto{\pgfqpoint{5.147209in}{4.507757in}}%
\pgfpathlineto{\pgfqpoint{5.148283in}{4.507463in}}%
\pgfpathlineto{\pgfqpoint{5.149357in}{4.507924in}}%
\pgfpathlineto{\pgfqpoint{5.150430in}{4.507043in}}%
\pgfpathlineto{\pgfqpoint{5.155800in}{4.511154in}}%
\pgfpathlineto{\pgfqpoint{5.157947in}{4.514930in}}%
\pgfpathlineto{\pgfqpoint{5.161169in}{4.513839in}}%
\pgfpathlineto{\pgfqpoint{5.162243in}{4.517195in}}%
\pgfpathlineto{\pgfqpoint{5.163316in}{4.510357in}}%
\pgfpathlineto{\pgfqpoint{5.165464in}{4.515811in}}%
\pgfpathlineto{\pgfqpoint{5.168686in}{4.519376in}}%
\pgfpathlineto{\pgfqpoint{5.169759in}{4.518579in}}%
\pgfpathlineto{\pgfqpoint{5.170833in}{4.516230in}}%
\pgfpathlineto{\pgfqpoint{5.171907in}{4.517698in}}%
\pgfpathlineto{\pgfqpoint{5.172981in}{4.509393in}}%
\pgfpathlineto{\pgfqpoint{5.176202in}{4.505659in}}%
\pgfpathlineto{\pgfqpoint{5.177276in}{4.498654in}}%
\pgfpathlineto{\pgfqpoint{5.179424in}{4.517866in}}%
\pgfpathlineto{\pgfqpoint{5.180498in}{4.516818in}}%
\pgfpathlineto{\pgfqpoint{5.185867in}{4.521390in}}%
\pgfpathlineto{\pgfqpoint{5.188015in}{4.522229in}}%
\pgfpathlineto{\pgfqpoint{5.192310in}{4.522145in}}%
\pgfpathlineto{\pgfqpoint{5.193384in}{4.517363in}}%
\pgfpathlineto{\pgfqpoint{5.195532in}{4.517279in}}%
\pgfpathlineto{\pgfqpoint{5.198753in}{4.507673in}}%
\pgfpathlineto{\pgfqpoint{5.199827in}{4.500164in}}%
\pgfpathlineto{\pgfqpoint{5.201975in}{4.512707in}}%
\pgfpathlineto{\pgfqpoint{5.203048in}{4.508176in}}%
\pgfpathlineto{\pgfqpoint{5.207344in}{4.503520in}}%
\pgfpathlineto{\pgfqpoint{5.209491in}{4.490432in}}%
\pgfpathlineto{\pgfqpoint{5.210565in}{4.491061in}}%
\pgfpathlineto{\pgfqpoint{5.215935in}{4.497437in}}%
\pgfpathlineto{\pgfqpoint{5.217008in}{4.484559in}}%
\pgfpathlineto{\pgfqpoint{5.221304in}{4.480364in}}%
\pgfpathlineto{\pgfqpoint{5.223451in}{4.474240in}}%
\pgfpathlineto{\pgfqpoint{5.224525in}{4.475288in}}%
\pgfpathlineto{\pgfqpoint{5.225599in}{4.470170in}}%
\pgfpathlineto{\pgfqpoint{5.228821in}{4.475792in}}%
\pgfpathlineto{\pgfqpoint{5.229894in}{4.482042in}}%
\pgfpathlineto{\pgfqpoint{5.230968in}{4.481581in}}%
\pgfpathlineto{\pgfqpoint{5.232042in}{4.485943in}}%
\pgfpathlineto{\pgfqpoint{5.233116in}{4.487034in}}%
\pgfpathlineto{\pgfqpoint{5.236337in}{4.486950in}}%
\pgfpathlineto{\pgfqpoint{5.237411in}{4.490306in}}%
\pgfpathlineto{\pgfqpoint{5.238485in}{4.490935in}}%
\pgfpathlineto{\pgfqpoint{5.239559in}{4.469332in}}%
\pgfpathlineto{\pgfqpoint{5.240633in}{4.459977in}}%
\pgfpathlineto{\pgfqpoint{5.244928in}{4.463878in}}%
\pgfpathlineto{\pgfqpoint{5.246002in}{4.466605in}}%
\pgfpathlineto{\pgfqpoint{5.247076in}{4.461235in}}%
\pgfpathlineto{\pgfqpoint{5.248150in}{4.466815in}}%
\pgfpathlineto{\pgfqpoint{5.251371in}{4.468660in}}%
\pgfpathlineto{\pgfqpoint{5.252445in}{4.470842in}}%
\pgfpathlineto{\pgfqpoint{5.254593in}{4.480154in}}%
\pgfpathlineto{\pgfqpoint{5.255667in}{4.473694in}}%
\pgfpathlineto{\pgfqpoint{5.258888in}{4.475414in}}%
\pgfpathlineto{\pgfqpoint{5.259962in}{4.474953in}}%
\pgfpathlineto{\pgfqpoint{5.261036in}{4.469919in}}%
\pgfpathlineto{\pgfqpoint{5.262110in}{4.471974in}}%
\pgfpathlineto{\pgfqpoint{5.263183in}{4.468702in}}%
\pgfpathlineto{\pgfqpoint{5.266405in}{4.469457in}}%
\pgfpathlineto{\pgfqpoint{5.267479in}{4.464004in}}%
\pgfpathlineto{\pgfqpoint{5.268553in}{4.465346in}}%
\pgfpathlineto{\pgfqpoint{5.269626in}{4.473568in}}%
\pgfpathlineto{\pgfqpoint{5.270700in}{4.469835in}}%
\pgfpathlineto{\pgfqpoint{5.273922in}{4.473317in}}%
\pgfpathlineto{\pgfqpoint{5.274996in}{4.471597in}}%
\pgfpathlineto{\pgfqpoint{5.276069in}{4.474743in}}%
\pgfpathlineto{\pgfqpoint{5.277143in}{4.473484in}}%
\pgfpathlineto{\pgfqpoint{5.278217in}{4.478015in}}%
\pgfpathlineto{\pgfqpoint{5.281439in}{4.476085in}}%
\pgfpathlineto{\pgfqpoint{5.283586in}{4.467905in}}%
\pgfpathlineto{\pgfqpoint{5.284660in}{4.461529in}}%
\pgfpathlineto{\pgfqpoint{5.285734in}{4.459557in}}%
\pgfpathlineto{\pgfqpoint{5.288956in}{4.459851in}}%
\pgfpathlineto{\pgfqpoint{5.290029in}{4.461151in}}%
\pgfpathlineto{\pgfqpoint{5.292177in}{4.467360in}}%
\pgfpathlineto{\pgfqpoint{5.296472in}{4.467066in}}%
\pgfpathlineto{\pgfqpoint{5.297546in}{4.461990in}}%
\pgfpathlineto{\pgfqpoint{5.300768in}{4.466898in}}%
\pgfpathlineto{\pgfqpoint{5.303989in}{4.465640in}}%
\pgfpathlineto{\pgfqpoint{5.306137in}{4.467528in}}%
\pgfpathlineto{\pgfqpoint{5.307211in}{4.472100in}}%
\pgfpathlineto{\pgfqpoint{5.308285in}{4.458005in}}%
\pgfpathlineto{\pgfqpoint{5.312580in}{4.457879in}}%
\pgfpathlineto{\pgfqpoint{5.313654in}{4.462410in}}%
\pgfpathlineto{\pgfqpoint{5.315802in}{4.460648in}}%
\pgfpathlineto{\pgfqpoint{5.319023in}{4.458676in}}%
\pgfpathlineto{\pgfqpoint{5.320097in}{4.458676in}}%
\pgfpathlineto{\pgfqpoint{5.321171in}{4.457376in}}%
\pgfpathlineto{\pgfqpoint{5.323318in}{4.459432in}}%
\pgfpathlineto{\pgfqpoint{5.326540in}{4.461781in}}%
\pgfpathlineto{\pgfqpoint{5.327614in}{4.460061in}}%
\pgfpathlineto{\pgfqpoint{5.328688in}{4.460103in}}%
\pgfpathlineto{\pgfqpoint{5.330835in}{4.464507in}}%
\pgfpathlineto{\pgfqpoint{5.334057in}{4.467360in}}%
\pgfpathlineto{\pgfqpoint{5.335131in}{4.464927in}}%
\pgfpathlineto{\pgfqpoint{5.337278in}{4.471471in}}%
\pgfpathlineto{\pgfqpoint{5.338352in}{4.469373in}}%
\pgfpathlineto{\pgfqpoint{5.341574in}{4.469206in}}%
\pgfpathlineto{\pgfqpoint{5.342647in}{4.473862in}}%
\pgfpathlineto{\pgfqpoint{5.344795in}{4.471513in}}%
\pgfpathlineto{\pgfqpoint{5.345869in}{4.473401in}}%
\pgfpathlineto{\pgfqpoint{5.351238in}{4.468954in}}%
\pgfpathlineto{\pgfqpoint{5.352312in}{4.468870in}}%
\pgfpathlineto{\pgfqpoint{5.353386in}{4.467444in}}%
\pgfpathlineto{\pgfqpoint{5.356607in}{4.466437in}}%
\pgfpathlineto{\pgfqpoint{5.358755in}{4.470884in}}%
\pgfpathlineto{\pgfqpoint{5.359829in}{4.466018in}}%
\pgfpathlineto{\pgfqpoint{5.360903in}{4.466143in}}%
\pgfpathlineto{\pgfqpoint{5.364124in}{4.463836in}}%
\pgfpathlineto{\pgfqpoint{5.365198in}{4.465262in}}%
\pgfpathlineto{\pgfqpoint{5.366272in}{4.469248in}}%
\pgfpathlineto{\pgfqpoint{5.367346in}{4.469709in}}%
\pgfpathlineto{\pgfqpoint{5.368420in}{4.466689in}}%
\pgfpathlineto{\pgfqpoint{5.371641in}{4.465598in}}%
\pgfpathlineto{\pgfqpoint{5.372715in}{4.466018in}}%
\pgfpathlineto{\pgfqpoint{5.374863in}{4.471555in}}%
\pgfpathlineto{\pgfqpoint{5.375936in}{4.469248in}}%
\pgfpathlineto{\pgfqpoint{5.379158in}{4.473401in}}%
\pgfpathlineto{\pgfqpoint{5.380232in}{4.473862in}}%
\pgfpathlineto{\pgfqpoint{5.382379in}{4.468031in}}%
\pgfpathlineto{\pgfqpoint{5.383453in}{4.468031in}}%
\pgfpathlineto{\pgfqpoint{5.386675in}{4.459893in}}%
\pgfpathlineto{\pgfqpoint{5.387749in}{4.460732in}}%
\pgfpathlineto{\pgfqpoint{5.388823in}{4.463417in}}%
\pgfpathlineto{\pgfqpoint{5.394192in}{4.460229in}}%
\pgfpathlineto{\pgfqpoint{5.395266in}{4.460061in}}%
\pgfpathlineto{\pgfqpoint{5.396339in}{4.453685in}}%
\pgfpathlineto{\pgfqpoint{5.397413in}{4.455279in}}%
\pgfpathlineto{\pgfqpoint{5.398487in}{4.459096in}}%
\pgfpathlineto{\pgfqpoint{5.402782in}{4.465640in}}%
\pgfpathlineto{\pgfqpoint{5.403856in}{4.463962in}}%
\pgfpathlineto{\pgfqpoint{5.406004in}{4.466605in}}%
\pgfpathlineto{\pgfqpoint{5.409225in}{4.466940in}}%
\pgfpathlineto{\pgfqpoint{5.410299in}{4.465556in}}%
\pgfpathlineto{\pgfqpoint{5.411373in}{4.465724in}}%
\pgfpathlineto{\pgfqpoint{5.413521in}{4.453559in}}%
\pgfpathlineto{\pgfqpoint{5.416742in}{4.449699in}}%
\pgfpathlineto{\pgfqpoint{5.417816in}{4.450455in}}%
\pgfpathlineto{\pgfqpoint{5.419964in}{4.454440in}}%
\pgfpathlineto{\pgfqpoint{5.425333in}{4.452846in}}%
\pgfpathlineto{\pgfqpoint{5.426407in}{4.452342in}}%
\pgfpathlineto{\pgfqpoint{5.427481in}{4.450035in}}%
\pgfpathlineto{\pgfqpoint{5.428555in}{4.468576in}}%
\pgfpathlineto{\pgfqpoint{5.431776in}{4.474659in}}%
\pgfpathlineto{\pgfqpoint{5.432850in}{4.474953in}}%
\pgfpathlineto{\pgfqpoint{5.434998in}{4.472687in}}%
\pgfpathlineto{\pgfqpoint{5.436071in}{4.473275in}}%
\pgfpathlineto{\pgfqpoint{5.439293in}{4.473610in}}%
\pgfpathlineto{\pgfqpoint{5.440367in}{4.474575in}}%
\pgfpathlineto{\pgfqpoint{5.441441in}{4.473442in}}%
\pgfpathlineto{\pgfqpoint{5.443588in}{4.458005in}}%
\pgfpathlineto{\pgfqpoint{5.447884in}{4.444204in}}%
\pgfpathlineto{\pgfqpoint{5.450031in}{4.457418in}}%
\pgfpathlineto{\pgfqpoint{5.451105in}{4.456495in}}%
\pgfpathlineto{\pgfqpoint{5.454327in}{4.456789in}}%
\pgfpathlineto{\pgfqpoint{5.455401in}{4.446092in}}%
\pgfpathlineto{\pgfqpoint{5.456474in}{4.449783in}}%
\pgfpathlineto{\pgfqpoint{5.457548in}{4.451042in}}%
\pgfpathlineto{\pgfqpoint{5.458622in}{4.446385in}}%
\pgfpathlineto{\pgfqpoint{5.462917in}{4.451965in}}%
\pgfpathlineto{\pgfqpoint{5.463991in}{4.450496in}}%
\pgfpathlineto{\pgfqpoint{5.466139in}{4.452007in}}%
\pgfpathlineto{\pgfqpoint{5.469360in}{4.450580in}}%
\pgfpathlineto{\pgfqpoint{5.471508in}{4.459138in}}%
\pgfpathlineto{\pgfqpoint{5.472582in}{4.458131in}}%
\pgfpathlineto{\pgfqpoint{5.473656in}{4.453768in}}%
\pgfpathlineto{\pgfqpoint{5.476877in}{4.456873in}}%
\pgfpathlineto{\pgfqpoint{5.477951in}{4.452804in}}%
\pgfpathlineto{\pgfqpoint{5.479025in}{4.452510in}}%
\pgfpathlineto{\pgfqpoint{5.480099in}{4.448819in}}%
\pgfpathlineto{\pgfqpoint{5.481173in}{4.450413in}}%
\pgfpathlineto{\pgfqpoint{5.484394in}{4.443491in}}%
\pgfpathlineto{\pgfqpoint{5.485468in}{4.442568in}}%
\pgfpathlineto{\pgfqpoint{5.486542in}{4.446595in}}%
\pgfpathlineto{\pgfqpoint{5.487616in}{4.445672in}}%
\pgfpathlineto{\pgfqpoint{5.488690in}{4.447686in}}%
\pgfpathlineto{\pgfqpoint{5.491911in}{4.459054in}}%
\pgfpathlineto{\pgfqpoint{5.492985in}{4.457544in}}%
\pgfpathlineto{\pgfqpoint{5.494059in}{4.459767in}}%
\pgfpathlineto{\pgfqpoint{5.495133in}{4.459767in}}%
\pgfpathlineto{\pgfqpoint{5.496206in}{4.460354in}}%
\pgfpathlineto{\pgfqpoint{5.499428in}{4.460271in}}%
\pgfpathlineto{\pgfqpoint{5.501576in}{4.455740in}}%
\pgfpathlineto{\pgfqpoint{5.503723in}{4.459851in}}%
\pgfpathlineto{\pgfqpoint{5.508019in}{4.458844in}}%
\pgfpathlineto{\pgfqpoint{5.509092in}{4.457124in}}%
\pgfpathlineto{\pgfqpoint{5.510166in}{4.441268in}}%
\pgfpathlineto{\pgfqpoint{5.511240in}{4.449532in}}%
\pgfpathlineto{\pgfqpoint{5.515535in}{4.447224in}}%
\pgfpathlineto{\pgfqpoint{5.516609in}{4.448986in}}%
\pgfpathlineto{\pgfqpoint{5.517683in}{4.448063in}}%
\pgfpathlineto{\pgfqpoint{5.518757in}{4.444288in}}%
\pgfpathlineto{\pgfqpoint{5.523052in}{4.447308in}}%
\pgfpathlineto{\pgfqpoint{5.524126in}{4.447476in}}%
\pgfpathlineto{\pgfqpoint{5.525200in}{4.446931in}}%
\pgfpathlineto{\pgfqpoint{5.526274in}{4.448357in}}%
\pgfpathlineto{\pgfqpoint{5.529495in}{4.444917in}}%
\pgfpathlineto{\pgfqpoint{5.530569in}{4.444708in}}%
\pgfpathlineto{\pgfqpoint{5.531643in}{4.442904in}}%
\pgfpathlineto{\pgfqpoint{5.533791in}{4.436150in}}%
\pgfpathlineto{\pgfqpoint{5.537012in}{4.437954in}}%
\pgfpathlineto{\pgfqpoint{5.538086in}{4.435772in}}%
\pgfpathlineto{\pgfqpoint{5.540234in}{4.442233in}}%
\pgfpathlineto{\pgfqpoint{5.541308in}{4.440932in}}%
\pgfpathlineto{\pgfqpoint{5.544529in}{4.440261in}}%
\pgfpathlineto{\pgfqpoint{5.545603in}{4.437828in}}%
\pgfpathlineto{\pgfqpoint{5.548824in}{4.438709in}}%
\pgfpathlineto{\pgfqpoint{5.552046in}{4.437870in}}%
\pgfpathlineto{\pgfqpoint{5.553120in}{4.440093in}}%
\pgfpathlineto{\pgfqpoint{5.555267in}{4.433046in}}%
\pgfpathlineto{\pgfqpoint{5.556341in}{4.435730in}}%
\pgfpathlineto{\pgfqpoint{5.559563in}{4.433801in}}%
\pgfpathlineto{\pgfqpoint{5.560637in}{4.431032in}}%
\pgfpathlineto{\pgfqpoint{5.561711in}{4.430822in}}%
\pgfpathlineto{\pgfqpoint{5.562784in}{4.431829in}}%
\pgfpathlineto{\pgfqpoint{5.563858in}{4.426879in}}%
\pgfpathlineto{\pgfqpoint{5.567080in}{4.426795in}}%
\pgfpathlineto{\pgfqpoint{5.568154in}{4.431997in}}%
\pgfpathlineto{\pgfqpoint{5.569227in}{4.434220in}}%
\pgfpathlineto{\pgfqpoint{5.571375in}{4.422978in}}%
\pgfpathlineto{\pgfqpoint{5.574597in}{4.425075in}}%
\pgfpathlineto{\pgfqpoint{5.575670in}{4.426837in}}%
\pgfpathlineto{\pgfqpoint{5.576744in}{4.431284in}}%
\pgfpathlineto{\pgfqpoint{5.577818in}{4.432039in}}%
\pgfpathlineto{\pgfqpoint{5.582113in}{4.430445in}}%
\pgfpathlineto{\pgfqpoint{5.583187in}{4.433549in}}%
\pgfpathlineto{\pgfqpoint{5.585335in}{4.429606in}}%
\pgfpathlineto{\pgfqpoint{5.589630in}{4.421887in}}%
\pgfpathlineto{\pgfqpoint{5.590704in}{4.417776in}}%
\pgfpathlineto{\pgfqpoint{5.591778in}{4.410477in}}%
\pgfpathlineto{\pgfqpoint{5.592852in}{4.408170in}}%
\pgfpathlineto{\pgfqpoint{5.593926in}{4.407331in}}%
\pgfpathlineto{\pgfqpoint{5.597147in}{4.409009in}}%
\pgfpathlineto{\pgfqpoint{5.598221in}{4.410393in}}%
\pgfpathlineto{\pgfqpoint{5.599295in}{4.404227in}}%
\pgfpathlineto{\pgfqpoint{5.600369in}{4.405989in}}%
\pgfpathlineto{\pgfqpoint{5.601443in}{4.404479in}}%
\pgfpathlineto{\pgfqpoint{5.605738in}{4.403430in}}%
\pgfpathlineto{\pgfqpoint{5.606812in}{4.404940in}}%
\pgfpathlineto{\pgfqpoint{5.607886in}{4.403430in}}%
\pgfpathlineto{\pgfqpoint{5.608959in}{4.373353in}}%
\pgfpathlineto{\pgfqpoint{5.613255in}{4.373478in}}%
\pgfpathlineto{\pgfqpoint{5.614329in}{4.371213in}}%
\pgfpathlineto{\pgfqpoint{5.615402in}{4.364711in}}%
\pgfpathlineto{\pgfqpoint{5.616476in}{4.367144in}}%
\pgfpathlineto{\pgfqpoint{5.619698in}{4.371926in}}%
\pgfpathlineto{\pgfqpoint{5.620772in}{4.367815in}}%
\pgfpathlineto{\pgfqpoint{5.622919in}{4.370668in}}%
\pgfpathlineto{\pgfqpoint{5.623993in}{4.369074in}}%
\pgfpathlineto{\pgfqpoint{5.627215in}{4.362781in}}%
\pgfpathlineto{\pgfqpoint{5.628289in}{4.363704in}}%
\pgfpathlineto{\pgfqpoint{5.629362in}{4.362362in}}%
\pgfpathlineto{\pgfqpoint{5.630436in}{4.357664in}}%
\pgfpathlineto{\pgfqpoint{5.631510in}{4.363830in}}%
\pgfpathlineto{\pgfqpoint{5.635805in}{4.365886in}}%
\pgfpathlineto{\pgfqpoint{5.642248in}{4.375618in}}%
\pgfpathlineto{\pgfqpoint{5.644396in}{4.371675in}}%
\pgfpathlineto{\pgfqpoint{5.645470in}{4.374653in}}%
\pgfpathlineto{\pgfqpoint{5.646544in}{4.374611in}}%
\pgfpathlineto{\pgfqpoint{5.649765in}{4.375408in}}%
\pgfpathlineto{\pgfqpoint{5.650839in}{4.380232in}}%
\pgfpathlineto{\pgfqpoint{5.651913in}{4.381533in}}%
\pgfpathlineto{\pgfqpoint{5.652987in}{4.385350in}}%
\pgfpathlineto{\pgfqpoint{5.657282in}{4.388958in}}%
\pgfpathlineto{\pgfqpoint{5.658356in}{4.390677in}}%
\pgfpathlineto{\pgfqpoint{5.660504in}{4.387993in}}%
\pgfpathlineto{\pgfqpoint{5.661578in}{4.390803in}}%
\pgfpathlineto{\pgfqpoint{5.664799in}{4.391223in}}%
\pgfpathlineto{\pgfqpoint{5.665873in}{4.389880in}}%
\pgfpathlineto{\pgfqpoint{5.668021in}{4.393278in}}%
\pgfpathlineto{\pgfqpoint{5.669094in}{4.397767in}}%
\pgfpathlineto{\pgfqpoint{5.672316in}{4.397725in}}%
\pgfpathlineto{\pgfqpoint{5.673390in}{4.395376in}}%
\pgfpathlineto{\pgfqpoint{5.674464in}{4.395460in}}%
\pgfpathlineto{\pgfqpoint{5.675537in}{4.394788in}}%
\pgfpathlineto{\pgfqpoint{5.679833in}{4.394033in}}%
\pgfpathlineto{\pgfqpoint{5.680907in}{4.395334in}}%
\pgfpathlineto{\pgfqpoint{5.681980in}{4.394075in}}%
\pgfpathlineto{\pgfqpoint{5.683054in}{4.398480in}}%
\pgfpathlineto{\pgfqpoint{5.684128in}{4.397305in}}%
\pgfpathlineto{\pgfqpoint{5.687350in}{4.395418in}}%
\pgfpathlineto{\pgfqpoint{5.688423in}{4.393698in}}%
\pgfpathlineto{\pgfqpoint{5.689497in}{4.393614in}}%
\pgfpathlineto{\pgfqpoint{5.690571in}{4.389545in}}%
\pgfpathlineto{\pgfqpoint{5.691645in}{4.392104in}}%
\pgfpathlineto{\pgfqpoint{5.694867in}{4.393278in}}%
\pgfpathlineto{\pgfqpoint{5.695940in}{4.396802in}}%
\pgfpathlineto{\pgfqpoint{5.697014in}{4.402717in}}%
\pgfpathlineto{\pgfqpoint{5.698088in}{4.404101in}}%
\pgfpathlineto{\pgfqpoint{5.699162in}{4.402633in}}%
\pgfpathlineto{\pgfqpoint{5.702383in}{4.404437in}}%
\pgfpathlineto{\pgfqpoint{5.705605in}{4.416434in}}%
\pgfpathlineto{\pgfqpoint{5.706679in}{4.417734in}}%
\pgfpathlineto{\pgfqpoint{5.709900in}{4.416812in}}%
\pgfpathlineto{\pgfqpoint{5.710974in}{4.418825in}}%
\pgfpathlineto{\pgfqpoint{5.712048in}{4.418909in}}%
\pgfpathlineto{\pgfqpoint{5.714196in}{4.415721in}}%
\pgfpathlineto{\pgfqpoint{5.717417in}{4.416728in}}%
\pgfpathlineto{\pgfqpoint{5.719565in}{4.411065in}}%
\pgfpathlineto{\pgfqpoint{5.720639in}{4.409722in}}%
\pgfpathlineto{\pgfqpoint{5.721712in}{4.412071in}}%
\pgfpathlineto{\pgfqpoint{5.724934in}{4.409974in}}%
\pgfpathlineto{\pgfqpoint{5.726008in}{4.413372in}}%
\pgfpathlineto{\pgfqpoint{5.729229in}{4.410519in}}%
\pgfpathlineto{\pgfqpoint{5.732451in}{4.410310in}}%
\pgfpathlineto{\pgfqpoint{5.733525in}{4.405108in}}%
\pgfpathlineto{\pgfqpoint{5.734599in}{4.408128in}}%
\pgfpathlineto{\pgfqpoint{5.735672in}{4.405024in}}%
\pgfpathlineto{\pgfqpoint{5.736746in}{4.409722in}}%
\pgfpathlineto{\pgfqpoint{5.739968in}{4.408422in}}%
\pgfpathlineto{\pgfqpoint{5.741042in}{4.413498in}}%
\pgfpathlineto{\pgfqpoint{5.742115in}{4.415259in}}%
\pgfpathlineto{\pgfqpoint{5.743189in}{4.414924in}}%
\pgfpathlineto{\pgfqpoint{5.744263in}{4.416098in}}%
\pgfpathlineto{\pgfqpoint{5.749632in}{4.417651in}}%
\pgfpathlineto{\pgfqpoint{5.750706in}{4.419622in}}%
\pgfpathlineto{\pgfqpoint{5.751780in}{4.415973in}}%
\pgfpathlineto{\pgfqpoint{5.755001in}{4.417776in}}%
\pgfpathlineto{\pgfqpoint{5.756075in}{4.417567in}}%
\pgfpathlineto{\pgfqpoint{5.757149in}{4.419035in}}%
\pgfpathlineto{\pgfqpoint{5.762518in}{4.408715in}}%
\pgfpathlineto{\pgfqpoint{5.763592in}{4.398354in}}%
\pgfpathlineto{\pgfqpoint{5.765740in}{4.401794in}}%
\pgfpathlineto{\pgfqpoint{5.766814in}{4.401500in}}%
\pgfpathlineto{\pgfqpoint{5.771109in}{4.403220in}}%
\pgfpathlineto{\pgfqpoint{5.772183in}{4.401878in}}%
\pgfpathlineto{\pgfqpoint{5.773257in}{4.407037in}}%
\pgfpathlineto{\pgfqpoint{5.774331in}{4.394327in}}%
\pgfpathlineto{\pgfqpoint{5.777552in}{4.384805in}}%
\pgfpathlineto{\pgfqpoint{5.778626in}{4.385727in}}%
\pgfpathlineto{\pgfqpoint{5.780774in}{4.398354in}}%
\pgfpathlineto{\pgfqpoint{5.781847in}{4.398060in}}%
\pgfpathlineto{\pgfqpoint{5.786143in}{4.391894in}}%
\pgfpathlineto{\pgfqpoint{5.788290in}{4.394537in}}%
\pgfpathlineto{\pgfqpoint{5.789364in}{4.401249in}}%
\pgfpathlineto{\pgfqpoint{5.792586in}{4.403933in}}%
\pgfpathlineto{\pgfqpoint{5.793660in}{4.407331in}}%
\pgfpathlineto{\pgfqpoint{5.794733in}{4.407709in}}%
\pgfpathlineto{\pgfqpoint{5.795807in}{4.409764in}}%
\pgfpathlineto{\pgfqpoint{5.796881in}{4.410435in}}%
\pgfpathlineto{\pgfqpoint{5.801177in}{4.411946in}}%
\pgfpathlineto{\pgfqpoint{5.802250in}{4.413246in}}%
\pgfpathlineto{\pgfqpoint{5.803324in}{4.409051in}}%
\pgfpathlineto{\pgfqpoint{5.804398in}{4.412449in}}%
\pgfpathlineto{\pgfqpoint{5.808693in}{4.412784in}}%
\pgfpathlineto{\pgfqpoint{5.810841in}{4.414504in}}%
\pgfpathlineto{\pgfqpoint{5.811915in}{4.413162in}}%
\pgfpathlineto{\pgfqpoint{5.815136in}{4.411862in}}%
\pgfpathlineto{\pgfqpoint{5.816210in}{4.409219in}}%
\pgfpathlineto{\pgfqpoint{5.817284in}{4.410645in}}%
\pgfpathlineto{\pgfqpoint{5.818358in}{4.411065in}}%
\pgfpathlineto{\pgfqpoint{5.819432in}{4.417399in}}%
\pgfpathlineto{\pgfqpoint{5.822653in}{4.418657in}}%
\pgfpathlineto{\pgfqpoint{5.824801in}{4.414295in}}%
\pgfpathlineto{\pgfqpoint{5.825875in}{4.417231in}}%
\pgfpathlineto{\pgfqpoint{5.826949in}{4.416854in}}%
\pgfpathlineto{\pgfqpoint{5.830170in}{4.417860in}}%
\pgfpathlineto{\pgfqpoint{5.831244in}{4.416518in}}%
\pgfpathlineto{\pgfqpoint{5.832318in}{4.418070in}}%
\pgfpathlineto{\pgfqpoint{5.837687in}{4.416770in}}%
\pgfpathlineto{\pgfqpoint{5.838761in}{4.418070in}}%
\pgfpathlineto{\pgfqpoint{5.839835in}{4.415973in}}%
\pgfpathlineto{\pgfqpoint{5.841982in}{4.414504in}}%
\pgfpathlineto{\pgfqpoint{5.845204in}{4.417399in}}%
\pgfpathlineto{\pgfqpoint{5.846278in}{4.417147in}}%
\pgfpathlineto{\pgfqpoint{5.847352in}{4.417651in}}%
\pgfpathlineto{\pgfqpoint{5.848425in}{4.414756in}}%
\pgfpathlineto{\pgfqpoint{5.849499in}{4.416098in}}%
\pgfpathlineto{\pgfqpoint{5.853795in}{4.418364in}}%
\pgfpathlineto{\pgfqpoint{5.854868in}{4.420084in}}%
\pgfpathlineto{\pgfqpoint{5.855942in}{4.420293in}}%
\pgfpathlineto{\pgfqpoint{5.857016in}{4.415805in}}%
\pgfpathlineto{\pgfqpoint{5.860238in}{4.418951in}}%
\pgfpathlineto{\pgfqpoint{5.862385in}{4.409219in}}%
\pgfpathlineto{\pgfqpoint{5.863459in}{4.410645in}}%
\pgfpathlineto{\pgfqpoint{5.864533in}{4.409974in}}%
\pgfpathlineto{\pgfqpoint{5.867755in}{4.411568in}}%
\pgfpathlineto{\pgfqpoint{5.868828in}{4.410184in}}%
\pgfpathlineto{\pgfqpoint{5.870976in}{4.413833in}}%
\pgfpathlineto{\pgfqpoint{5.872050in}{4.410729in}}%
\pgfpathlineto{\pgfqpoint{5.875271in}{4.409009in}}%
\pgfpathlineto{\pgfqpoint{5.876345in}{4.412449in}}%
\pgfpathlineto{\pgfqpoint{5.877419in}{4.412197in}}%
\pgfpathlineto{\pgfqpoint{5.878493in}{4.408799in}}%
\pgfpathlineto{\pgfqpoint{5.879567in}{4.411484in}}%
\pgfpathlineto{\pgfqpoint{5.882788in}{4.410561in}}%
\pgfpathlineto{\pgfqpoint{5.883862in}{4.410981in}}%
\pgfpathlineto{\pgfqpoint{5.884936in}{4.414043in}}%
\pgfpathlineto{\pgfqpoint{5.886010in}{4.404311in}}%
\pgfpathlineto{\pgfqpoint{5.887084in}{4.403598in}}%
\pgfpathlineto{\pgfqpoint{5.890305in}{4.404143in}}%
\pgfpathlineto{\pgfqpoint{5.891379in}{4.399990in}}%
\pgfpathlineto{\pgfqpoint{5.893527in}{4.398144in}}%
\pgfpathlineto{\pgfqpoint{5.894600in}{4.397096in}}%
\pgfpathlineto{\pgfqpoint{5.897822in}{4.396089in}}%
\pgfpathlineto{\pgfqpoint{5.898896in}{4.396844in}}%
\pgfpathlineto{\pgfqpoint{5.899970in}{4.401542in}}%
\pgfpathlineto{\pgfqpoint{5.901044in}{4.423817in}}%
\pgfpathlineto{\pgfqpoint{5.902117in}{4.426124in}}%
\pgfpathlineto{\pgfqpoint{5.905339in}{4.425034in}}%
\pgfpathlineto{\pgfqpoint{5.906413in}{4.423649in}}%
\pgfpathlineto{\pgfqpoint{5.908560in}{4.424404in}}%
\pgfpathlineto{\pgfqpoint{5.909634in}{4.422475in}}%
\pgfpathlineto{\pgfqpoint{5.912856in}{4.422349in}}%
\pgfpathlineto{\pgfqpoint{5.913930in}{4.421678in}}%
\pgfpathlineto{\pgfqpoint{5.915003in}{4.418448in}}%
\pgfpathlineto{\pgfqpoint{5.916077in}{4.417986in}}%
\pgfpathlineto{\pgfqpoint{5.917151in}{4.418699in}}%
\pgfpathlineto{\pgfqpoint{5.920373in}{4.424698in}}%
\pgfpathlineto{\pgfqpoint{5.921446in}{4.424950in}}%
\pgfpathlineto{\pgfqpoint{5.923594in}{4.437199in}}%
\pgfpathlineto{\pgfqpoint{5.924668in}{4.438793in}}%
\pgfpathlineto{\pgfqpoint{5.927889in}{4.446511in}}%
\pgfpathlineto{\pgfqpoint{5.928963in}{4.446721in}}%
\pgfpathlineto{\pgfqpoint{5.930037in}{4.443533in}}%
\pgfpathlineto{\pgfqpoint{5.931111in}{4.443911in}}%
\pgfpathlineto{\pgfqpoint{5.932185in}{4.440806in}}%
\pgfpathlineto{\pgfqpoint{5.936480in}{4.443701in}}%
\pgfpathlineto{\pgfqpoint{5.937554in}{4.448357in}}%
\pgfpathlineto{\pgfqpoint{5.939702in}{4.448273in}}%
\pgfpathlineto{\pgfqpoint{5.942923in}{4.445337in}}%
\pgfpathlineto{\pgfqpoint{5.943997in}{4.442778in}}%
\pgfpathlineto{\pgfqpoint{5.946145in}{4.446931in}}%
\pgfpathlineto{\pgfqpoint{5.947219in}{4.444246in}}%
\pgfpathlineto{\pgfqpoint{5.950440in}{4.444917in}}%
\pgfpathlineto{\pgfqpoint{5.951514in}{4.446050in}}%
\pgfpathlineto{\pgfqpoint{5.952588in}{4.454020in}}%
\pgfpathlineto{\pgfqpoint{5.953662in}{4.456537in}}%
\pgfpathlineto{\pgfqpoint{5.954735in}{4.455950in}}%
\pgfpathlineto{\pgfqpoint{5.957957in}{4.451168in}}%
\pgfpathlineto{\pgfqpoint{5.960105in}{4.453139in}}%
\pgfpathlineto{\pgfqpoint{5.961178in}{4.456663in}}%
\pgfpathlineto{\pgfqpoint{5.962252in}{4.456873in}}%
\pgfpathlineto{\pgfqpoint{5.965474in}{4.455069in}}%
\pgfpathlineto{\pgfqpoint{5.967621in}{4.458173in}}%
\pgfpathlineto{\pgfqpoint{5.968695in}{4.455195in}}%
\pgfpathlineto{\pgfqpoint{5.969769in}{4.456789in}}%
\pgfpathlineto{\pgfqpoint{5.974065in}{4.456831in}}%
\pgfpathlineto{\pgfqpoint{5.976212in}{4.452552in}}%
\pgfpathlineto{\pgfqpoint{5.977286in}{4.453181in}}%
\pgfpathlineto{\pgfqpoint{5.981581in}{4.458299in}}%
\pgfpathlineto{\pgfqpoint{5.982655in}{4.463291in}}%
\pgfpathlineto{\pgfqpoint{5.983729in}{4.459474in}}%
\pgfpathlineto{\pgfqpoint{5.988024in}{4.461655in}}%
\pgfpathlineto{\pgfqpoint{5.989098in}{4.464843in}}%
\pgfpathlineto{\pgfqpoint{5.990172in}{4.465892in}}%
\pgfpathlineto{\pgfqpoint{5.991246in}{4.465766in}}%
\pgfpathlineto{\pgfqpoint{5.992320in}{4.464717in}}%
\pgfpathlineto{\pgfqpoint{5.996615in}{4.464633in}}%
\pgfpathlineto{\pgfqpoint{5.997689in}{4.468241in}}%
\pgfpathlineto{\pgfqpoint{5.999837in}{4.463039in}}%
\pgfpathlineto{\pgfqpoint{6.003058in}{4.462116in}}%
\pgfpathlineto{\pgfqpoint{6.004132in}{4.467989in}}%
\pgfpathlineto{\pgfqpoint{6.005206in}{4.465808in}}%
\pgfpathlineto{\pgfqpoint{6.007354in}{4.465640in}}%
\pgfpathlineto{\pgfqpoint{6.010575in}{4.467402in}}%
\pgfpathlineto{\pgfqpoint{6.011649in}{4.463752in}}%
\pgfpathlineto{\pgfqpoint{6.012723in}{4.465304in}}%
\pgfpathlineto{\pgfqpoint{6.013797in}{4.464298in}}%
\pgfpathlineto{\pgfqpoint{6.014870in}{4.470464in}}%
\pgfpathlineto{\pgfqpoint{6.020240in}{4.469499in}}%
\pgfpathlineto{\pgfqpoint{6.022387in}{4.472268in}}%
\pgfpathlineto{\pgfqpoint{6.025609in}{4.473988in}}%
\pgfpathlineto{\pgfqpoint{6.026683in}{4.476001in}}%
\pgfpathlineto{\pgfqpoint{6.027756in}{4.476798in}}%
\pgfpathlineto{\pgfqpoint{6.028830in}{4.476421in}}%
\pgfpathlineto{\pgfqpoint{6.029904in}{4.477218in}}%
\pgfpathlineto{\pgfqpoint{6.034199in}{4.478267in}}%
\pgfpathlineto{\pgfqpoint{6.035273in}{4.477847in}}%
\pgfpathlineto{\pgfqpoint{6.036347in}{4.478602in}}%
\pgfpathlineto{\pgfqpoint{6.037421in}{4.477428in}}%
\pgfpathlineto{\pgfqpoint{6.040643in}{4.479064in}}%
\pgfpathlineto{\pgfqpoint{6.041716in}{4.478644in}}%
\pgfpathlineto{\pgfqpoint{6.042790in}{4.486153in}}%
\pgfpathlineto{\pgfqpoint{6.043864in}{4.478812in}}%
\pgfpathlineto{\pgfqpoint{6.044938in}{4.477931in}}%
\pgfpathlineto{\pgfqpoint{6.048159in}{4.476379in}}%
\pgfpathlineto{\pgfqpoint{6.049233in}{4.476714in}}%
\pgfpathlineto{\pgfqpoint{6.050307in}{4.474533in}}%
\pgfpathlineto{\pgfqpoint{6.052455in}{4.475876in}}%
\pgfpathlineto{\pgfqpoint{6.055676in}{4.475204in}}%
\pgfpathlineto{\pgfqpoint{6.056750in}{4.477134in}}%
\pgfpathlineto{\pgfqpoint{6.057824in}{4.475288in}}%
\pgfpathlineto{\pgfqpoint{6.058898in}{4.477470in}}%
\pgfpathlineto{\pgfqpoint{6.059972in}{4.475372in}}%
\pgfpathlineto{\pgfqpoint{6.063193in}{4.473736in}}%
\pgfpathlineto{\pgfqpoint{6.064267in}{4.468409in}}%
\pgfpathlineto{\pgfqpoint{6.066415in}{4.469667in}}%
\pgfpathlineto{\pgfqpoint{6.067488in}{4.471135in}}%
\pgfpathlineto{\pgfqpoint{6.070710in}{4.468702in}}%
\pgfpathlineto{\pgfqpoint{6.071784in}{4.472897in}}%
\pgfpathlineto{\pgfqpoint{6.072858in}{4.471303in}}%
\pgfpathlineto{\pgfqpoint{6.073932in}{4.475204in}}%
\pgfpathlineto{\pgfqpoint{6.075005in}{4.474785in}}%
\pgfpathlineto{\pgfqpoint{6.080375in}{4.470632in}}%
\pgfpathlineto{\pgfqpoint{6.081448in}{4.471261in}}%
\pgfpathlineto{\pgfqpoint{6.086818in}{4.468618in}}%
\pgfpathlineto{\pgfqpoint{6.088965in}{4.462662in}}%
\pgfpathlineto{\pgfqpoint{6.093261in}{4.466185in}}%
\pgfpathlineto{\pgfqpoint{6.094334in}{4.462620in}}%
\pgfpathlineto{\pgfqpoint{6.095408in}{4.461655in}}%
\pgfpathlineto{\pgfqpoint{6.096482in}{4.479819in}}%
\pgfpathlineto{\pgfqpoint{6.097556in}{4.478057in}}%
\pgfpathlineto{\pgfqpoint{6.101851in}{4.482294in}}%
\pgfpathlineto{\pgfqpoint{6.103999in}{4.481077in}}%
\pgfpathlineto{\pgfqpoint{6.105073in}{4.476673in}}%
\pgfpathlineto{\pgfqpoint{6.108294in}{4.476589in}}%
\pgfpathlineto{\pgfqpoint{6.109368in}{4.477847in}}%
\pgfpathlineto{\pgfqpoint{6.111516in}{4.472939in}}%
\pgfpathlineto{\pgfqpoint{6.115811in}{4.472268in}}%
\pgfpathlineto{\pgfqpoint{6.117959in}{4.474240in}}%
\pgfpathlineto{\pgfqpoint{6.120107in}{4.469541in}}%
\pgfpathlineto{\pgfqpoint{6.123328in}{4.472939in}}%
\pgfpathlineto{\pgfqpoint{6.124402in}{4.472142in}}%
\pgfpathlineto{\pgfqpoint{6.125476in}{4.464969in}}%
\pgfpathlineto{\pgfqpoint{6.126550in}{4.465011in}}%
\pgfpathlineto{\pgfqpoint{6.127623in}{4.466731in}}%
\pgfpathlineto{\pgfqpoint{6.130845in}{4.467444in}}%
\pgfpathlineto{\pgfqpoint{6.131919in}{4.468367in}}%
\pgfpathlineto{\pgfqpoint{6.132993in}{4.468031in}}%
\pgfpathlineto{\pgfqpoint{6.134066in}{4.469332in}}%
\pgfpathlineto{\pgfqpoint{6.135140in}{4.469415in}}%
\pgfpathlineto{\pgfqpoint{6.140509in}{4.467318in}}%
\pgfpathlineto{\pgfqpoint{6.141583in}{4.472729in}}%
\pgfpathlineto{\pgfqpoint{6.145879in}{4.475540in}}%
\pgfpathlineto{\pgfqpoint{6.146953in}{4.475037in}}%
\pgfpathlineto{\pgfqpoint{6.148026in}{4.478938in}}%
\pgfpathlineto{\pgfqpoint{6.149100in}{4.479525in}}%
\pgfpathlineto{\pgfqpoint{6.150174in}{4.480993in}}%
\pgfpathlineto{\pgfqpoint{6.153396in}{4.480406in}}%
\pgfpathlineto{\pgfqpoint{6.155543in}{4.483133in}}%
\pgfpathlineto{\pgfqpoint{6.156617in}{4.482545in}}%
\pgfpathlineto{\pgfqpoint{6.157691in}{4.485608in}}%
\pgfpathlineto{\pgfqpoint{6.160912in}{4.487369in}}%
\pgfpathlineto{\pgfqpoint{6.161986in}{4.489928in}}%
\pgfpathlineto{\pgfqpoint{6.163060in}{4.488712in}}%
\pgfpathlineto{\pgfqpoint{6.165208in}{4.488754in}}%
\pgfpathlineto{\pgfqpoint{6.169503in}{4.492236in}}%
\pgfpathlineto{\pgfqpoint{6.170577in}{4.495843in}}%
\pgfpathlineto{\pgfqpoint{6.171651in}{4.494459in}}%
\pgfpathlineto{\pgfqpoint{6.172725in}{4.496934in}}%
\pgfpathlineto{\pgfqpoint{6.175946in}{4.500416in}}%
\pgfpathlineto{\pgfqpoint{6.178094in}{4.500877in}}%
\pgfpathlineto{\pgfqpoint{6.179168in}{4.496179in}}%
\pgfpathlineto{\pgfqpoint{6.180242in}{4.498612in}}%
\pgfpathlineto{\pgfqpoint{6.184537in}{4.497899in}}%
\pgfpathlineto{\pgfqpoint{6.186685in}{4.502849in}}%
\pgfpathlineto{\pgfqpoint{6.187758in}{4.502471in}}%
\pgfpathlineto{\pgfqpoint{6.190980in}{4.502219in}}%
\pgfpathlineto{\pgfqpoint{6.193128in}{4.505114in}}%
\pgfpathlineto{\pgfqpoint{6.194201in}{4.502765in}}%
\pgfpathlineto{\pgfqpoint{6.195275in}{4.503730in}}%
\pgfpathlineto{\pgfqpoint{6.198497in}{4.501338in}}%
\pgfpathlineto{\pgfqpoint{6.199571in}{4.502974in}}%
\pgfpathlineto{\pgfqpoint{6.200644in}{4.502555in}}%
\pgfpathlineto{\pgfqpoint{6.201718in}{4.496640in}}%
\pgfpathlineto{\pgfqpoint{6.202792in}{4.500499in}}%
\pgfpathlineto{\pgfqpoint{6.207087in}{4.502303in}}%
\pgfpathlineto{\pgfqpoint{6.208161in}{4.502555in}}%
\pgfpathlineto{\pgfqpoint{6.209235in}{4.503520in}}%
\pgfpathlineto{\pgfqpoint{6.210309in}{4.505282in}}%
\pgfpathlineto{\pgfqpoint{6.215678in}{4.504149in}}%
\pgfpathlineto{\pgfqpoint{6.216752in}{4.499535in}}%
\pgfpathlineto{\pgfqpoint{6.217826in}{4.498444in}}%
\pgfpathlineto{\pgfqpoint{6.221047in}{4.503226in}}%
\pgfpathlineto{\pgfqpoint{6.222121in}{4.508638in}}%
\pgfpathlineto{\pgfqpoint{6.223195in}{4.511071in}}%
\pgfpathlineto{\pgfqpoint{6.225343in}{4.502891in}}%
\pgfpathlineto{\pgfqpoint{6.230712in}{4.502471in}}%
\pgfpathlineto{\pgfqpoint{6.232860in}{4.503226in}}%
\pgfpathlineto{\pgfqpoint{6.237155in}{4.503016in}}%
\pgfpathlineto{\pgfqpoint{6.240376in}{4.505953in}}%
\pgfpathlineto{\pgfqpoint{6.245746in}{4.502261in}}%
\pgfpathlineto{\pgfqpoint{6.246820in}{4.498905in}}%
\pgfpathlineto{\pgfqpoint{6.247893in}{4.498276in}}%
\pgfpathlineto{\pgfqpoint{6.251115in}{4.504149in}}%
\pgfpathlineto{\pgfqpoint{6.252189in}{4.507631in}}%
\pgfpathlineto{\pgfqpoint{6.253263in}{4.508008in}}%
\pgfpathlineto{\pgfqpoint{6.254336in}{4.506163in}}%
\pgfpathlineto{\pgfqpoint{6.255410in}{4.509435in}}%
\pgfpathlineto{\pgfqpoint{6.258632in}{4.513000in}}%
\pgfpathlineto{\pgfqpoint{6.259706in}{4.517615in}}%
\pgfpathlineto{\pgfqpoint{6.260779in}{4.515307in}}%
\pgfpathlineto{\pgfqpoint{6.262927in}{4.515182in}}%
\pgfpathlineto{\pgfqpoint{6.266149in}{4.514385in}}%
\pgfpathlineto{\pgfqpoint{6.267222in}{4.516356in}}%
\pgfpathlineto{\pgfqpoint{6.269370in}{4.522271in}}%
\pgfpathlineto{\pgfqpoint{6.270444in}{4.523571in}}%
\pgfpathlineto{\pgfqpoint{6.273665in}{4.523907in}}%
\pgfpathlineto{\pgfqpoint{6.274739in}{4.527515in}}%
\pgfpathlineto{\pgfqpoint{6.275813in}{4.525795in}}%
\pgfpathlineto{\pgfqpoint{6.277961in}{4.529444in}}%
\pgfpathlineto{\pgfqpoint{6.281182in}{4.530031in}}%
\pgfpathlineto{\pgfqpoint{6.283330in}{4.531122in}}%
\pgfpathlineto{\pgfqpoint{6.284404in}{4.529696in}}%
\pgfpathlineto{\pgfqpoint{6.285478in}{4.534814in}}%
\pgfpathlineto{\pgfqpoint{6.289773in}{4.530031in}}%
\pgfpathlineto{\pgfqpoint{6.290847in}{4.531626in}}%
\pgfpathlineto{\pgfqpoint{6.291921in}{4.530870in}}%
\pgfpathlineto{\pgfqpoint{6.292995in}{4.531667in}}%
\pgfpathlineto{\pgfqpoint{6.296216in}{4.532842in}}%
\pgfpathlineto{\pgfqpoint{6.297290in}{4.538883in}}%
\pgfpathlineto{\pgfqpoint{6.298364in}{4.537540in}}%
\pgfpathlineto{\pgfqpoint{6.299438in}{4.546391in}}%
\pgfpathlineto{\pgfqpoint{6.300511in}{4.546811in}}%
\pgfpathlineto{\pgfqpoint{6.303733in}{4.543833in}}%
\pgfpathlineto{\pgfqpoint{6.305881in}{4.546811in}}%
\pgfpathlineto{\pgfqpoint{6.306954in}{4.547566in}}%
\pgfpathlineto{\pgfqpoint{6.308028in}{4.549412in}}%
\pgfpathlineto{\pgfqpoint{6.311250in}{4.548825in}}%
\pgfpathlineto{\pgfqpoint{6.312324in}{4.545091in}}%
\pgfpathlineto{\pgfqpoint{6.313397in}{4.544084in}}%
\pgfpathlineto{\pgfqpoint{6.314471in}{4.538463in}}%
\pgfpathlineto{\pgfqpoint{6.315545in}{4.537498in}}%
\pgfpathlineto{\pgfqpoint{6.318767in}{4.539050in}}%
\pgfpathlineto{\pgfqpoint{6.319841in}{4.538505in}}%
\pgfpathlineto{\pgfqpoint{6.320914in}{4.536450in}}%
\pgfpathlineto{\pgfqpoint{6.323062in}{4.538211in}}%
\pgfpathlineto{\pgfqpoint{6.326284in}{4.539260in}}%
\pgfpathlineto{\pgfqpoint{6.327357in}{4.541148in}}%
\pgfpathlineto{\pgfqpoint{6.328431in}{4.538757in}}%
\pgfpathlineto{\pgfqpoint{6.330579in}{4.537331in}}%
\pgfpathlineto{\pgfqpoint{6.333800in}{4.537289in}}%
\pgfpathlineto{\pgfqpoint{6.335948in}{4.550125in}}%
\pgfpathlineto{\pgfqpoint{6.337022in}{4.554655in}}%
\pgfpathlineto{\pgfqpoint{6.338096in}{4.555285in}}%
\pgfpathlineto{\pgfqpoint{6.342391in}{4.558724in}}%
\pgfpathlineto{\pgfqpoint{6.343465in}{4.556711in}}%
\pgfpathlineto{\pgfqpoint{6.344539in}{4.558221in}}%
\pgfpathlineto{\pgfqpoint{6.345613in}{4.558095in}}%
\pgfpathlineto{\pgfqpoint{6.348834in}{4.559941in}}%
\pgfpathlineto{\pgfqpoint{6.349908in}{4.561451in}}%
\pgfpathlineto{\pgfqpoint{6.350982in}{4.554949in}}%
\pgfpathlineto{\pgfqpoint{6.352056in}{4.552348in}}%
\pgfpathlineto{\pgfqpoint{6.353130in}{4.557969in}}%
\pgfpathlineto{\pgfqpoint{6.356351in}{4.562710in}}%
\pgfpathlineto{\pgfqpoint{6.358499in}{4.557927in}}%
\pgfpathlineto{\pgfqpoint{6.359573in}{4.557885in}}%
\pgfpathlineto{\pgfqpoint{6.360646in}{4.558850in}}%
\pgfpathlineto{\pgfqpoint{6.364942in}{4.558179in}}%
\pgfpathlineto{\pgfqpoint{6.367089in}{4.562793in}}%
\pgfpathlineto{\pgfqpoint{6.368163in}{4.561199in}}%
\pgfpathlineto{\pgfqpoint{6.368163in}{4.561199in}}%
\pgfusepath{stroke}%
\end{pgfscope}%
\begin{pgfscope}%
\pgfpathrectangle{\pgfqpoint{3.902254in}{4.180131in}}{\pgfqpoint{2.583333in}{0.400885in}}%
\pgfusepath{clip}%
\pgfsetroundcap%
\pgfsetroundjoin%
\pgfsetlinewidth{1.505625pt}%
\definecolor{currentstroke}{rgb}{1.000000,0.498039,0.054902}%
\pgfsetstrokecolor{currentstroke}%
\pgfsetdash{}{0pt}%
\pgfpathmoveto{\pgfqpoint{4.019678in}{4.336522in}}%
\pgfpathlineto{\pgfqpoint{4.020752in}{4.336396in}}%
\pgfpathlineto{\pgfqpoint{4.022900in}{4.332336in}}%
\pgfpathlineto{\pgfqpoint{4.026121in}{4.331886in}}%
\pgfpathlineto{\pgfqpoint{4.028269in}{4.329861in}}%
\pgfpathlineto{\pgfqpoint{4.029343in}{4.327354in}}%
\pgfpathlineto{\pgfqpoint{4.034712in}{4.325388in}}%
\pgfpathlineto{\pgfqpoint{4.036860in}{4.322911in}}%
\pgfpathlineto{\pgfqpoint{4.037934in}{4.325922in}}%
\pgfpathlineto{\pgfqpoint{4.041155in}{4.327977in}}%
\pgfpathlineto{\pgfqpoint{4.042229in}{4.327236in}}%
\pgfpathlineto{\pgfqpoint{4.043303in}{4.323923in}}%
\pgfpathlineto{\pgfqpoint{4.045451in}{4.325014in}}%
\pgfpathlineto{\pgfqpoint{4.048672in}{4.322544in}}%
\pgfpathlineto{\pgfqpoint{4.049746in}{4.319086in}}%
\pgfpathlineto{\pgfqpoint{4.051894in}{4.315765in}}%
\pgfpathlineto{\pgfqpoint{4.052967in}{4.312335in}}%
\pgfpathlineto{\pgfqpoint{4.056189in}{4.313676in}}%
\pgfpathlineto{\pgfqpoint{4.057263in}{4.312687in}}%
\pgfpathlineto{\pgfqpoint{4.058337in}{4.314170in}}%
\pgfpathlineto{\pgfqpoint{4.059410in}{4.312109in}}%
\pgfpathlineto{\pgfqpoint{4.060484in}{4.313615in}}%
\pgfpathlineto{\pgfqpoint{4.064780in}{4.313134in}}%
\pgfpathlineto{\pgfqpoint{4.065853in}{4.314491in}}%
\pgfpathlineto{\pgfqpoint{4.066927in}{4.310317in}}%
\pgfpathlineto{\pgfqpoint{4.074444in}{4.310908in}}%
\pgfpathlineto{\pgfqpoint{4.075518in}{4.308929in}}%
\pgfpathlineto{\pgfqpoint{4.078740in}{4.306485in}}%
\pgfpathlineto{\pgfqpoint{4.079813in}{4.307669in}}%
\pgfpathlineto{\pgfqpoint{4.080887in}{4.310137in}}%
\pgfpathlineto{\pgfqpoint{4.081961in}{4.308143in}}%
\pgfpathlineto{\pgfqpoint{4.083035in}{4.309799in}}%
\pgfpathlineto{\pgfqpoint{4.086256in}{4.309864in}}%
\pgfpathlineto{\pgfqpoint{4.087330in}{4.313448in}}%
\pgfpathlineto{\pgfqpoint{4.090552in}{4.309055in}}%
\pgfpathlineto{\pgfqpoint{4.093773in}{4.310314in}}%
\pgfpathlineto{\pgfqpoint{4.096995in}{4.299177in}}%
\pgfpathlineto{\pgfqpoint{4.098069in}{4.299601in}}%
\pgfpathlineto{\pgfqpoint{4.101290in}{4.297778in}}%
\pgfpathlineto{\pgfqpoint{4.102364in}{4.298667in}}%
\pgfpathlineto{\pgfqpoint{4.105585in}{4.297825in}}%
\pgfpathlineto{\pgfqpoint{4.108807in}{4.294294in}}%
\pgfpathlineto{\pgfqpoint{4.109881in}{4.295326in}}%
\pgfpathlineto{\pgfqpoint{4.110955in}{4.293312in}}%
\pgfpathlineto{\pgfqpoint{4.112029in}{4.296052in}}%
\pgfpathlineto{\pgfqpoint{4.113102in}{4.296133in}}%
\pgfpathlineto{\pgfqpoint{4.116324in}{4.295753in}}%
\pgfpathlineto{\pgfqpoint{4.117398in}{4.294347in}}%
\pgfpathlineto{\pgfqpoint{4.118472in}{4.296785in}}%
\pgfpathlineto{\pgfqpoint{4.119545in}{4.294974in}}%
\pgfpathlineto{\pgfqpoint{4.123841in}{4.297248in}}%
\pgfpathlineto{\pgfqpoint{4.124915in}{4.299679in}}%
\pgfpathlineto{\pgfqpoint{4.127062in}{4.294979in}}%
\pgfpathlineto{\pgfqpoint{4.128136in}{4.296801in}}%
\pgfpathlineto{\pgfqpoint{4.133505in}{4.294937in}}%
\pgfpathlineto{\pgfqpoint{4.134579in}{4.296061in}}%
\pgfpathlineto{\pgfqpoint{4.135653in}{4.296361in}}%
\pgfpathlineto{\pgfqpoint{4.138874in}{4.296661in}}%
\pgfpathlineto{\pgfqpoint{4.139948in}{4.295920in}}%
\pgfpathlineto{\pgfqpoint{4.142096in}{4.291187in}}%
\pgfpathlineto{\pgfqpoint{4.143170in}{4.289867in}}%
\pgfpathlineto{\pgfqpoint{4.146391in}{4.289768in}}%
\pgfpathlineto{\pgfqpoint{4.147465in}{4.288078in}}%
\pgfpathlineto{\pgfqpoint{4.148539in}{4.287884in}}%
\pgfpathlineto{\pgfqpoint{4.149613in}{4.288319in}}%
\pgfpathlineto{\pgfqpoint{4.150687in}{4.289973in}}%
\pgfpathlineto{\pgfqpoint{4.154982in}{4.290447in}}%
\pgfpathlineto{\pgfqpoint{4.156056in}{4.291426in}}%
\pgfpathlineto{\pgfqpoint{4.157130in}{4.291502in}}%
\pgfpathlineto{\pgfqpoint{4.158204in}{4.290992in}}%
\pgfpathlineto{\pgfqpoint{4.163573in}{4.296252in}}%
\pgfpathlineto{\pgfqpoint{4.164647in}{4.300542in}}%
\pgfpathlineto{\pgfqpoint{4.165720in}{4.301181in}}%
\pgfpathlineto{\pgfqpoint{4.168942in}{4.298750in}}%
\pgfpathlineto{\pgfqpoint{4.170016in}{4.298693in}}%
\pgfpathlineto{\pgfqpoint{4.171090in}{4.299656in}}%
\pgfpathlineto{\pgfqpoint{4.172163in}{4.298709in}}%
\pgfpathlineto{\pgfqpoint{4.173237in}{4.300068in}}%
\pgfpathlineto{\pgfqpoint{4.177533in}{4.298105in}}%
\pgfpathlineto{\pgfqpoint{4.178607in}{4.300887in}}%
\pgfpathlineto{\pgfqpoint{4.179680in}{4.299923in}}%
\pgfpathlineto{\pgfqpoint{4.180754in}{4.293697in}}%
\pgfpathlineto{\pgfqpoint{4.183976in}{4.293899in}}%
\pgfpathlineto{\pgfqpoint{4.185050in}{4.295138in}}%
\pgfpathlineto{\pgfqpoint{4.186123in}{4.291506in}}%
\pgfpathlineto{\pgfqpoint{4.187197in}{4.291862in}}%
\pgfpathlineto{\pgfqpoint{4.188271in}{4.290321in}}%
\pgfpathlineto{\pgfqpoint{4.191493in}{4.292150in}}%
\pgfpathlineto{\pgfqpoint{4.192566in}{4.288724in}}%
\pgfpathlineto{\pgfqpoint{4.193640in}{4.291983in}}%
\pgfpathlineto{\pgfqpoint{4.194714in}{4.292121in}}%
\pgfpathlineto{\pgfqpoint{4.195788in}{4.289051in}}%
\pgfpathlineto{\pgfqpoint{4.199009in}{4.290083in}}%
\pgfpathlineto{\pgfqpoint{4.201157in}{4.286289in}}%
\pgfpathlineto{\pgfqpoint{4.202231in}{4.288852in}}%
\pgfpathlineto{\pgfqpoint{4.203305in}{4.287717in}}%
\pgfpathlineto{\pgfqpoint{4.206526in}{4.289456in}}%
\pgfpathlineto{\pgfqpoint{4.207600in}{4.289296in}}%
\pgfpathlineto{\pgfqpoint{4.208674in}{4.287462in}}%
\pgfpathlineto{\pgfqpoint{4.209748in}{4.287902in}}%
\pgfpathlineto{\pgfqpoint{4.210822in}{4.284336in}}%
\pgfpathlineto{\pgfqpoint{4.215117in}{4.281263in}}%
\pgfpathlineto{\pgfqpoint{4.217265in}{4.281592in}}%
\pgfpathlineto{\pgfqpoint{4.218339in}{4.282940in}}%
\pgfpathlineto{\pgfqpoint{4.223708in}{4.284129in}}%
\pgfpathlineto{\pgfqpoint{4.224782in}{4.286687in}}%
\pgfpathlineto{\pgfqpoint{4.225855in}{4.284384in}}%
\pgfpathlineto{\pgfqpoint{4.230151in}{4.282705in}}%
\pgfpathlineto{\pgfqpoint{4.231225in}{4.283548in}}%
\pgfpathlineto{\pgfqpoint{4.232298in}{4.288087in}}%
\pgfpathlineto{\pgfqpoint{4.233372in}{4.289083in}}%
\pgfpathlineto{\pgfqpoint{4.237668in}{4.289535in}}%
\pgfpathlineto{\pgfqpoint{4.238741in}{4.288518in}}%
\pgfpathlineto{\pgfqpoint{4.239815in}{4.284432in}}%
\pgfpathlineto{\pgfqpoint{4.240889in}{4.282721in}}%
\pgfpathlineto{\pgfqpoint{4.244111in}{4.283228in}}%
\pgfpathlineto{\pgfqpoint{4.247332in}{4.287273in}}%
\pgfpathlineto{\pgfqpoint{4.248406in}{4.284656in}}%
\pgfpathlineto{\pgfqpoint{4.251628in}{4.285801in}}%
\pgfpathlineto{\pgfqpoint{4.253775in}{4.283962in}}%
\pgfpathlineto{\pgfqpoint{4.254849in}{4.287166in}}%
\pgfpathlineto{\pgfqpoint{4.255923in}{4.288617in}}%
\pgfpathlineto{\pgfqpoint{4.260218in}{4.288060in}}%
\pgfpathlineto{\pgfqpoint{4.263440in}{4.284546in}}%
\pgfpathlineto{\pgfqpoint{4.267735in}{4.286778in}}%
\pgfpathlineto{\pgfqpoint{4.268809in}{4.286262in}}%
\pgfpathlineto{\pgfqpoint{4.269883in}{4.287183in}}%
\pgfpathlineto{\pgfqpoint{4.270957in}{4.284694in}}%
\pgfpathlineto{\pgfqpoint{4.277400in}{4.285393in}}%
\pgfpathlineto{\pgfqpoint{4.278473in}{4.282818in}}%
\pgfpathlineto{\pgfqpoint{4.282769in}{4.282139in}}%
\pgfpathlineto{\pgfqpoint{4.283843in}{4.285236in}}%
\pgfpathlineto{\pgfqpoint{4.285990in}{4.284006in}}%
\pgfpathlineto{\pgfqpoint{4.291360in}{4.285023in}}%
\pgfpathlineto{\pgfqpoint{4.292433in}{4.281004in}}%
\pgfpathlineto{\pgfqpoint{4.293507in}{4.280531in}}%
\pgfpathlineto{\pgfqpoint{4.296729in}{4.280719in}}%
\pgfpathlineto{\pgfqpoint{4.297803in}{4.281800in}}%
\pgfpathlineto{\pgfqpoint{4.298876in}{4.281800in}}%
\pgfpathlineto{\pgfqpoint{4.301024in}{4.283558in}}%
\pgfpathlineto{\pgfqpoint{4.304246in}{4.283976in}}%
\pgfpathlineto{\pgfqpoint{4.305319in}{4.285189in}}%
\pgfpathlineto{\pgfqpoint{4.306393in}{4.287458in}}%
\pgfpathlineto{\pgfqpoint{4.308541in}{4.285771in}}%
\pgfpathlineto{\pgfqpoint{4.311762in}{4.283760in}}%
\pgfpathlineto{\pgfqpoint{4.312836in}{4.284995in}}%
\pgfpathlineto{\pgfqpoint{4.313910in}{4.284013in}}%
\pgfpathlineto{\pgfqpoint{4.314984in}{4.281857in}}%
\pgfpathlineto{\pgfqpoint{4.316058in}{4.281426in}}%
\pgfpathlineto{\pgfqpoint{4.319279in}{4.280878in}}%
\pgfpathlineto{\pgfqpoint{4.321427in}{4.282707in}}%
\pgfpathlineto{\pgfqpoint{4.322501in}{4.281590in}}%
\pgfpathlineto{\pgfqpoint{4.323575in}{4.282855in}}%
\pgfpathlineto{\pgfqpoint{4.326796in}{4.283512in}}%
\pgfpathlineto{\pgfqpoint{4.328944in}{4.279609in}}%
\pgfpathlineto{\pgfqpoint{4.330018in}{4.283289in}}%
\pgfpathlineto{\pgfqpoint{4.331092in}{4.284959in}}%
\pgfpathlineto{\pgfqpoint{4.334313in}{4.285236in}}%
\pgfpathlineto{\pgfqpoint{4.335387in}{4.282098in}}%
\pgfpathlineto{\pgfqpoint{4.336461in}{4.281769in}}%
\pgfpathlineto{\pgfqpoint{4.337535in}{4.282250in}}%
\pgfpathlineto{\pgfqpoint{4.338608in}{4.281541in}}%
\pgfpathlineto{\pgfqpoint{4.343978in}{4.281040in}}%
\pgfpathlineto{\pgfqpoint{4.345051in}{4.279054in}}%
\pgfpathlineto{\pgfqpoint{4.349347in}{4.280140in}}%
\pgfpathlineto{\pgfqpoint{4.350421in}{4.278304in}}%
\pgfpathlineto{\pgfqpoint{4.351495in}{4.274832in}}%
\pgfpathlineto{\pgfqpoint{4.353642in}{4.274238in}}%
\pgfpathlineto{\pgfqpoint{4.356864in}{4.273578in}}%
\pgfpathlineto{\pgfqpoint{4.360085in}{4.269551in}}%
\pgfpathlineto{\pgfqpoint{4.361159in}{4.270964in}}%
\pgfpathlineto{\pgfqpoint{4.364381in}{4.272966in}}%
\pgfpathlineto{\pgfqpoint{4.365454in}{4.271624in}}%
\pgfpathlineto{\pgfqpoint{4.366528in}{4.271370in}}%
\pgfpathlineto{\pgfqpoint{4.368676in}{4.270270in}}%
\pgfpathlineto{\pgfqpoint{4.371897in}{4.271957in}}%
\pgfpathlineto{\pgfqpoint{4.372971in}{4.269340in}}%
\pgfpathlineto{\pgfqpoint{4.374045in}{4.271586in}}%
\pgfpathlineto{\pgfqpoint{4.375119in}{4.270729in}}%
\pgfpathlineto{\pgfqpoint{4.380488in}{4.270888in}}%
\pgfpathlineto{\pgfqpoint{4.381562in}{4.269721in}}%
\pgfpathlineto{\pgfqpoint{4.382636in}{4.270304in}}%
\pgfpathlineto{\pgfqpoint{4.383710in}{4.269284in}}%
\pgfpathlineto{\pgfqpoint{4.388005in}{4.268375in}}%
\pgfpathlineto{\pgfqpoint{4.389079in}{4.267170in}}%
\pgfpathlineto{\pgfqpoint{4.390153in}{4.267063in}}%
\pgfpathlineto{\pgfqpoint{4.391227in}{4.269161in}}%
\pgfpathlineto{\pgfqpoint{4.394448in}{4.267894in}}%
\pgfpathlineto{\pgfqpoint{4.395522in}{4.266807in}}%
\pgfpathlineto{\pgfqpoint{4.396596in}{4.268811in}}%
\pgfpathlineto{\pgfqpoint{4.398743in}{4.267098in}}%
\pgfpathlineto{\pgfqpoint{4.401965in}{4.267334in}}%
\pgfpathlineto{\pgfqpoint{4.406260in}{4.269027in}}%
\pgfpathlineto{\pgfqpoint{4.409482in}{4.267385in}}%
\pgfpathlineto{\pgfqpoint{4.411629in}{4.264495in}}%
\pgfpathlineto{\pgfqpoint{4.412703in}{4.264045in}}%
\pgfpathlineto{\pgfqpoint{4.413777in}{4.262907in}}%
\pgfpathlineto{\pgfqpoint{4.419146in}{4.261741in}}%
\pgfpathlineto{\pgfqpoint{4.421294in}{4.260022in}}%
\pgfpathlineto{\pgfqpoint{4.424516in}{4.260079in}}%
\pgfpathlineto{\pgfqpoint{4.426663in}{4.261082in}}%
\pgfpathlineto{\pgfqpoint{4.427737in}{4.260871in}}%
\pgfpathlineto{\pgfqpoint{4.428811in}{4.262556in}}%
\pgfpathlineto{\pgfqpoint{4.433106in}{4.263229in}}%
\pgfpathlineto{\pgfqpoint{4.434180in}{4.264090in}}%
\pgfpathlineto{\pgfqpoint{4.436328in}{4.263036in}}%
\pgfpathlineto{\pgfqpoint{4.439549in}{4.263814in}}%
\pgfpathlineto{\pgfqpoint{4.440623in}{4.263126in}}%
\pgfpathlineto{\pgfqpoint{4.441697in}{4.263346in}}%
\pgfpathlineto{\pgfqpoint{4.442771in}{4.264289in}}%
\pgfpathlineto{\pgfqpoint{4.443845in}{4.262247in}}%
\pgfpathlineto{\pgfqpoint{4.447066in}{4.263132in}}%
\pgfpathlineto{\pgfqpoint{4.448140in}{4.260872in}}%
\pgfpathlineto{\pgfqpoint{4.449214in}{4.261045in}}%
\pgfpathlineto{\pgfqpoint{4.450288in}{4.258274in}}%
\pgfpathlineto{\pgfqpoint{4.451361in}{4.258840in}}%
\pgfpathlineto{\pgfqpoint{4.457805in}{4.257939in}}%
\pgfpathlineto{\pgfqpoint{4.458878in}{4.259009in}}%
\pgfpathlineto{\pgfqpoint{4.463174in}{4.258009in}}%
\pgfpathlineto{\pgfqpoint{4.465321in}{4.259234in}}%
\pgfpathlineto{\pgfqpoint{4.466395in}{4.257503in}}%
\pgfpathlineto{\pgfqpoint{4.470691in}{4.258568in}}%
\pgfpathlineto{\pgfqpoint{4.471764in}{4.257594in}}%
\pgfpathlineto{\pgfqpoint{4.472838in}{4.258171in}}%
\pgfpathlineto{\pgfqpoint{4.477134in}{4.256955in}}%
\pgfpathlineto{\pgfqpoint{4.479281in}{4.254120in}}%
\pgfpathlineto{\pgfqpoint{4.481429in}{4.254052in}}%
\pgfpathlineto{\pgfqpoint{4.484650in}{4.252823in}}%
\pgfpathlineto{\pgfqpoint{4.486798in}{4.253119in}}%
\pgfpathlineto{\pgfqpoint{4.487872in}{4.253003in}}%
\pgfpathlineto{\pgfqpoint{4.488946in}{4.251932in}}%
\pgfpathlineto{\pgfqpoint{4.492167in}{4.252400in}}%
\pgfpathlineto{\pgfqpoint{4.493241in}{4.253360in}}%
\pgfpathlineto{\pgfqpoint{4.494315in}{4.252049in}}%
\pgfpathlineto{\pgfqpoint{4.495389in}{4.252905in}}%
\pgfpathlineto{\pgfqpoint{4.496463in}{4.251704in}}%
\pgfpathlineto{\pgfqpoint{4.499684in}{4.251865in}}%
\pgfpathlineto{\pgfqpoint{4.500758in}{4.250303in}}%
\pgfpathlineto{\pgfqpoint{4.507201in}{4.250175in}}%
\pgfpathlineto{\pgfqpoint{4.508275in}{4.249631in}}%
\pgfpathlineto{\pgfqpoint{4.509349in}{4.251307in}}%
\pgfpathlineto{\pgfqpoint{4.510423in}{4.250606in}}%
\pgfpathlineto{\pgfqpoint{4.511496in}{4.252663in}}%
\pgfpathlineto{\pgfqpoint{4.514718in}{4.252204in}}%
\pgfpathlineto{\pgfqpoint{4.515792in}{4.252772in}}%
\pgfpathlineto{\pgfqpoint{4.519013in}{4.252084in}}%
\pgfpathlineto{\pgfqpoint{4.522235in}{4.254416in}}%
\pgfpathlineto{\pgfqpoint{4.523309in}{4.253651in}}%
\pgfpathlineto{\pgfqpoint{4.524383in}{4.254354in}}%
\pgfpathlineto{\pgfqpoint{4.525456in}{4.252945in}}%
\pgfpathlineto{\pgfqpoint{4.526530in}{4.249645in}}%
\pgfpathlineto{\pgfqpoint{4.529752in}{4.250447in}}%
\pgfpathlineto{\pgfqpoint{4.530826in}{4.249318in}}%
\pgfpathlineto{\pgfqpoint{4.531899in}{4.249364in}}%
\pgfpathlineto{\pgfqpoint{4.532973in}{4.248367in}}%
\pgfpathlineto{\pgfqpoint{4.534047in}{4.248892in}}%
\pgfpathlineto{\pgfqpoint{4.537269in}{4.249059in}}%
\pgfpathlineto{\pgfqpoint{4.538342in}{4.247977in}}%
\pgfpathlineto{\pgfqpoint{4.539416in}{4.248156in}}%
\pgfpathlineto{\pgfqpoint{4.541564in}{4.245531in}}%
\pgfpathlineto{\pgfqpoint{4.552302in}{4.246097in}}%
\pgfpathlineto{\pgfqpoint{4.554450in}{4.242272in}}%
\pgfpathlineto{\pgfqpoint{4.555524in}{4.242929in}}%
\pgfpathlineto{\pgfqpoint{4.556598in}{4.241595in}}%
\pgfpathlineto{\pgfqpoint{4.560893in}{4.239509in}}%
\pgfpathlineto{\pgfqpoint{4.561967in}{4.240262in}}%
\pgfpathlineto{\pgfqpoint{4.570558in}{4.238339in}}%
\pgfpathlineto{\pgfqpoint{4.571631in}{4.238807in}}%
\pgfpathlineto{\pgfqpoint{4.574853in}{4.237970in}}%
\pgfpathlineto{\pgfqpoint{4.575927in}{4.238410in}}%
\pgfpathlineto{\pgfqpoint{4.577001in}{4.239840in}}%
\pgfpathlineto{\pgfqpoint{4.579148in}{4.236201in}}%
\pgfpathlineto{\pgfqpoint{4.582370in}{4.235950in}}%
\pgfpathlineto{\pgfqpoint{4.584517in}{4.239719in}}%
\pgfpathlineto{\pgfqpoint{4.585591in}{4.239159in}}%
\pgfpathlineto{\pgfqpoint{4.586665in}{4.241661in}}%
\pgfpathlineto{\pgfqpoint{4.589887in}{4.240633in}}%
\pgfpathlineto{\pgfqpoint{4.590961in}{4.239311in}}%
\pgfpathlineto{\pgfqpoint{4.594182in}{4.241189in}}%
\pgfpathlineto{\pgfqpoint{4.597404in}{4.242722in}}%
\pgfpathlineto{\pgfqpoint{4.600625in}{4.239033in}}%
\pgfpathlineto{\pgfqpoint{4.601699in}{4.239438in}}%
\pgfpathlineto{\pgfqpoint{4.604920in}{4.238448in}}%
\pgfpathlineto{\pgfqpoint{4.605994in}{4.239318in}}%
\pgfpathlineto{\pgfqpoint{4.607068in}{4.239368in}}%
\pgfpathlineto{\pgfqpoint{4.609216in}{4.237403in}}%
\pgfpathlineto{\pgfqpoint{4.613511in}{4.236010in}}%
\pgfpathlineto{\pgfqpoint{4.614585in}{4.237345in}}%
\pgfpathlineto{\pgfqpoint{4.615659in}{4.236696in}}%
\pgfpathlineto{\pgfqpoint{4.616733in}{4.235212in}}%
\pgfpathlineto{\pgfqpoint{4.619954in}{4.235704in}}%
\pgfpathlineto{\pgfqpoint{4.621028in}{4.235273in}}%
\pgfpathlineto{\pgfqpoint{4.622102in}{4.236775in}}%
\pgfpathlineto{\pgfqpoint{4.623176in}{4.239753in}}%
\pgfpathlineto{\pgfqpoint{4.624249in}{4.239690in}}%
\pgfpathlineto{\pgfqpoint{4.627471in}{4.238941in}}%
\pgfpathlineto{\pgfqpoint{4.628545in}{4.239284in}}%
\pgfpathlineto{\pgfqpoint{4.630693in}{4.237761in}}%
\pgfpathlineto{\pgfqpoint{4.631766in}{4.238227in}}%
\pgfpathlineto{\pgfqpoint{4.636062in}{4.238395in}}%
\pgfpathlineto{\pgfqpoint{4.637136in}{4.239970in}}%
\pgfpathlineto{\pgfqpoint{4.638209in}{4.237831in}}%
\pgfpathlineto{\pgfqpoint{4.643579in}{4.237627in}}%
\pgfpathlineto{\pgfqpoint{4.644652in}{4.238353in}}%
\pgfpathlineto{\pgfqpoint{4.645726in}{4.237190in}}%
\pgfpathlineto{\pgfqpoint{4.646800in}{4.237932in}}%
\pgfpathlineto{\pgfqpoint{4.650022in}{4.237944in}}%
\pgfpathlineto{\pgfqpoint{4.651095in}{4.237140in}}%
\pgfpathlineto{\pgfqpoint{4.652169in}{4.237517in}}%
\pgfpathlineto{\pgfqpoint{4.653243in}{4.238598in}}%
\pgfpathlineto{\pgfqpoint{4.654317in}{4.238281in}}%
\pgfpathlineto{\pgfqpoint{4.661834in}{4.239973in}}%
\pgfpathlineto{\pgfqpoint{4.665055in}{4.240011in}}%
\pgfpathlineto{\pgfqpoint{4.666129in}{4.241979in}}%
\pgfpathlineto{\pgfqpoint{4.674720in}{4.240509in}}%
\pgfpathlineto{\pgfqpoint{4.675794in}{4.239730in}}%
\pgfpathlineto{\pgfqpoint{4.676868in}{4.240244in}}%
\pgfpathlineto{\pgfqpoint{4.680089in}{4.239889in}}%
\pgfpathlineto{\pgfqpoint{4.682237in}{4.238155in}}%
\pgfpathlineto{\pgfqpoint{4.687606in}{4.237685in}}%
\pgfpathlineto{\pgfqpoint{4.689754in}{4.235478in}}%
\pgfpathlineto{\pgfqpoint{4.691901in}{4.235818in}}%
\pgfpathlineto{\pgfqpoint{4.697271in}{4.237223in}}%
\pgfpathlineto{\pgfqpoint{4.698344in}{4.236857in}}%
\pgfpathlineto{\pgfqpoint{4.699418in}{4.237326in}}%
\pgfpathlineto{\pgfqpoint{4.702640in}{4.237728in}}%
\pgfpathlineto{\pgfqpoint{4.703714in}{4.237274in}}%
\pgfpathlineto{\pgfqpoint{4.705861in}{4.239115in}}%
\pgfpathlineto{\pgfqpoint{4.706935in}{4.238793in}}%
\pgfpathlineto{\pgfqpoint{4.712304in}{4.241182in}}%
\pgfpathlineto{\pgfqpoint{4.713378in}{4.238270in}}%
\pgfpathlineto{\pgfqpoint{4.714452in}{4.237444in}}%
\pgfpathlineto{\pgfqpoint{4.717673in}{4.236706in}}%
\pgfpathlineto{\pgfqpoint{4.718747in}{4.237583in}}%
\pgfpathlineto{\pgfqpoint{4.719821in}{4.236412in}}%
\pgfpathlineto{\pgfqpoint{4.720895in}{4.232241in}}%
\pgfpathlineto{\pgfqpoint{4.721969in}{4.231958in}}%
\pgfpathlineto{\pgfqpoint{4.727338in}{4.231850in}}%
\pgfpathlineto{\pgfqpoint{4.728412in}{4.231591in}}%
\pgfpathlineto{\pgfqpoint{4.729486in}{4.229972in}}%
\pgfpathlineto{\pgfqpoint{4.732707in}{4.229843in}}%
\pgfpathlineto{\pgfqpoint{4.733781in}{4.229026in}}%
\pgfpathlineto{\pgfqpoint{4.734855in}{4.229517in}}%
\pgfpathlineto{\pgfqpoint{4.735929in}{4.230689in}}%
\pgfpathlineto{\pgfqpoint{4.737003in}{4.230357in}}%
\pgfpathlineto{\pgfqpoint{4.742372in}{4.230385in}}%
\pgfpathlineto{\pgfqpoint{4.743446in}{4.231495in}}%
\pgfpathlineto{\pgfqpoint{4.744519in}{4.230682in}}%
\pgfpathlineto{\pgfqpoint{4.749889in}{4.230991in}}%
\pgfpathlineto{\pgfqpoint{4.752036in}{4.229730in}}%
\pgfpathlineto{\pgfqpoint{4.757405in}{4.230221in}}%
\pgfpathlineto{\pgfqpoint{4.758479in}{4.228667in}}%
\pgfpathlineto{\pgfqpoint{4.759553in}{4.228754in}}%
\pgfpathlineto{\pgfqpoint{4.767070in}{4.227067in}}%
\pgfpathlineto{\pgfqpoint{4.773513in}{4.228136in}}%
\pgfpathlineto{\pgfqpoint{4.774587in}{4.226930in}}%
\pgfpathlineto{\pgfqpoint{4.778882in}{4.227470in}}%
\pgfpathlineto{\pgfqpoint{4.781030in}{4.229180in}}%
\pgfpathlineto{\pgfqpoint{4.785325in}{4.228101in}}%
\pgfpathlineto{\pgfqpoint{4.786399in}{4.228459in}}%
\pgfpathlineto{\pgfqpoint{4.787473in}{4.226813in}}%
\pgfpathlineto{\pgfqpoint{4.788547in}{4.226458in}}%
\pgfpathlineto{\pgfqpoint{4.789621in}{4.225503in}}%
\pgfpathlineto{\pgfqpoint{4.801433in}{4.223029in}}%
\pgfpathlineto{\pgfqpoint{4.803581in}{4.223985in}}%
\pgfpathlineto{\pgfqpoint{4.804654in}{4.223758in}}%
\pgfpathlineto{\pgfqpoint{4.811097in}{4.224241in}}%
\pgfpathlineto{\pgfqpoint{4.816467in}{4.225617in}}%
\pgfpathlineto{\pgfqpoint{4.817540in}{4.224701in}}%
\pgfpathlineto{\pgfqpoint{4.818614in}{4.225079in}}%
\pgfpathlineto{\pgfqpoint{4.819688in}{4.222529in}}%
\pgfpathlineto{\pgfqpoint{4.825057in}{4.222404in}}%
\pgfpathlineto{\pgfqpoint{4.827205in}{4.225570in}}%
\pgfpathlineto{\pgfqpoint{4.830426in}{4.226567in}}%
\pgfpathlineto{\pgfqpoint{4.831500in}{4.225805in}}%
\pgfpathlineto{\pgfqpoint{4.832574in}{4.226693in}}%
\pgfpathlineto{\pgfqpoint{4.833648in}{4.225801in}}%
\pgfpathlineto{\pgfqpoint{4.834722in}{4.227106in}}%
\pgfpathlineto{\pgfqpoint{4.837943in}{4.228940in}}%
\pgfpathlineto{\pgfqpoint{4.839017in}{4.227922in}}%
\pgfpathlineto{\pgfqpoint{4.840091in}{4.228167in}}%
\pgfpathlineto{\pgfqpoint{4.842239in}{4.225352in}}%
\pgfpathlineto{\pgfqpoint{4.847608in}{4.223736in}}%
\pgfpathlineto{\pgfqpoint{4.855125in}{4.223870in}}%
\pgfpathlineto{\pgfqpoint{4.863715in}{4.222753in}}%
\pgfpathlineto{\pgfqpoint{4.864789in}{4.222003in}}%
\pgfpathlineto{\pgfqpoint{4.868011in}{4.222967in}}%
\pgfpathlineto{\pgfqpoint{4.869085in}{4.220973in}}%
\pgfpathlineto{\pgfqpoint{4.870159in}{4.221306in}}%
\pgfpathlineto{\pgfqpoint{4.871232in}{4.220313in}}%
\pgfpathlineto{\pgfqpoint{4.872306in}{4.220078in}}%
\pgfpathlineto{\pgfqpoint{4.877675in}{4.220745in}}%
\pgfpathlineto{\pgfqpoint{4.878749in}{4.222313in}}%
\pgfpathlineto{\pgfqpoint{4.879823in}{4.222700in}}%
\pgfpathlineto{\pgfqpoint{4.884118in}{4.221638in}}%
\pgfpathlineto{\pgfqpoint{4.885192in}{4.222262in}}%
\pgfpathlineto{\pgfqpoint{4.886266in}{4.221553in}}%
\pgfpathlineto{\pgfqpoint{4.891635in}{4.222172in}}%
\pgfpathlineto{\pgfqpoint{4.892709in}{4.223047in}}%
\pgfpathlineto{\pgfqpoint{4.894857in}{4.222440in}}%
\pgfpathlineto{\pgfqpoint{4.898078in}{4.222764in}}%
\pgfpathlineto{\pgfqpoint{4.899152in}{4.221905in}}%
\pgfpathlineto{\pgfqpoint{4.900226in}{4.222296in}}%
\pgfpathlineto{\pgfqpoint{4.901300in}{4.221868in}}%
\pgfpathlineto{\pgfqpoint{4.902374in}{4.223198in}}%
\pgfpathlineto{\pgfqpoint{4.906669in}{4.225251in}}%
\pgfpathlineto{\pgfqpoint{4.907743in}{4.223437in}}%
\pgfpathlineto{\pgfqpoint{4.908817in}{4.226037in}}%
\pgfpathlineto{\pgfqpoint{4.909891in}{4.226714in}}%
\pgfpathlineto{\pgfqpoint{4.914186in}{4.225455in}}%
\pgfpathlineto{\pgfqpoint{4.915260in}{4.224337in}}%
\pgfpathlineto{\pgfqpoint{4.916334in}{4.225274in}}%
\pgfpathlineto{\pgfqpoint{4.930293in}{4.224296in}}%
\pgfpathlineto{\pgfqpoint{4.932441in}{4.224690in}}%
\pgfpathlineto{\pgfqpoint{4.935663in}{4.224342in}}%
\pgfpathlineto{\pgfqpoint{4.936737in}{4.225262in}}%
\pgfpathlineto{\pgfqpoint{4.937810in}{4.223814in}}%
\pgfpathlineto{\pgfqpoint{4.939958in}{4.223138in}}%
\pgfpathlineto{\pgfqpoint{4.943180in}{4.222515in}}%
\pgfpathlineto{\pgfqpoint{4.947475in}{4.224166in}}%
\pgfpathlineto{\pgfqpoint{4.950696in}{4.223678in}}%
\pgfpathlineto{\pgfqpoint{4.951770in}{4.224480in}}%
\pgfpathlineto{\pgfqpoint{4.954992in}{4.223147in}}%
\pgfpathlineto{\pgfqpoint{4.959287in}{4.221143in}}%
\pgfpathlineto{\pgfqpoint{4.961435in}{4.221220in}}%
\pgfpathlineto{\pgfqpoint{4.968952in}{4.220128in}}%
\pgfpathlineto{\pgfqpoint{4.970026in}{4.218645in}}%
\pgfpathlineto{\pgfqpoint{4.974321in}{4.218395in}}%
\pgfpathlineto{\pgfqpoint{4.981838in}{4.218805in}}%
\pgfpathlineto{\pgfqpoint{4.982912in}{4.218199in}}%
\pgfpathlineto{\pgfqpoint{4.983985in}{4.218522in}}%
\pgfpathlineto{\pgfqpoint{4.985059in}{4.218210in}}%
\pgfpathlineto{\pgfqpoint{4.988281in}{4.218232in}}%
\pgfpathlineto{\pgfqpoint{4.989355in}{4.218942in}}%
\pgfpathlineto{\pgfqpoint{4.996871in}{4.218319in}}%
\pgfpathlineto{\pgfqpoint{4.999019in}{4.217994in}}%
\pgfpathlineto{\pgfqpoint{5.012979in}{4.218728in}}%
\pgfpathlineto{\pgfqpoint{5.014053in}{4.219727in}}%
\pgfpathlineto{\pgfqpoint{5.015127in}{4.219351in}}%
\pgfpathlineto{\pgfqpoint{5.019422in}{4.219450in}}%
\pgfpathlineto{\pgfqpoint{5.021570in}{4.219607in}}%
\pgfpathlineto{\pgfqpoint{5.022644in}{4.220454in}}%
\pgfpathlineto{\pgfqpoint{5.026939in}{4.220608in}}%
\pgfpathlineto{\pgfqpoint{5.028013in}{4.221180in}}%
\pgfpathlineto{\pgfqpoint{5.030160in}{4.224487in}}%
\pgfpathlineto{\pgfqpoint{5.033382in}{4.224122in}}%
\pgfpathlineto{\pgfqpoint{5.035530in}{4.224543in}}%
\pgfpathlineto{\pgfqpoint{5.036603in}{4.224848in}}%
\pgfpathlineto{\pgfqpoint{5.037677in}{4.223672in}}%
\pgfpathlineto{\pgfqpoint{5.041973in}{4.223976in}}%
\pgfpathlineto{\pgfqpoint{5.044120in}{4.223821in}}%
\pgfpathlineto{\pgfqpoint{5.045194in}{4.224350in}}%
\pgfpathlineto{\pgfqpoint{5.048416in}{4.223900in}}%
\pgfpathlineto{\pgfqpoint{5.050563in}{4.222855in}}%
\pgfpathlineto{\pgfqpoint{5.051637in}{4.222358in}}%
\pgfpathlineto{\pgfqpoint{5.052711in}{4.222554in}}%
\pgfpathlineto{\pgfqpoint{5.057006in}{4.222343in}}%
\pgfpathlineto{\pgfqpoint{5.067745in}{4.221995in}}%
\pgfpathlineto{\pgfqpoint{5.070966in}{4.222496in}}%
\pgfpathlineto{\pgfqpoint{5.072040in}{4.223284in}}%
\pgfpathlineto{\pgfqpoint{5.074188in}{4.222881in}}%
\pgfpathlineto{\pgfqpoint{5.075262in}{4.223478in}}%
\pgfpathlineto{\pgfqpoint{5.078483in}{4.223681in}}%
\pgfpathlineto{\pgfqpoint{5.081705in}{4.221583in}}%
\pgfpathlineto{\pgfqpoint{5.087074in}{4.222926in}}%
\pgfpathlineto{\pgfqpoint{5.088148in}{4.222760in}}%
\pgfpathlineto{\pgfqpoint{5.089222in}{4.224109in}}%
\pgfpathlineto{\pgfqpoint{5.090295in}{4.222826in}}%
\pgfpathlineto{\pgfqpoint{5.094591in}{4.223468in}}%
\pgfpathlineto{\pgfqpoint{5.095665in}{4.224626in}}%
\pgfpathlineto{\pgfqpoint{5.096738in}{4.224573in}}%
\pgfpathlineto{\pgfqpoint{5.097812in}{4.223526in}}%
\pgfpathlineto{\pgfqpoint{5.101034in}{4.223627in}}%
\pgfpathlineto{\pgfqpoint{5.102108in}{4.224953in}}%
\pgfpathlineto{\pgfqpoint{5.103181in}{4.223266in}}%
\pgfpathlineto{\pgfqpoint{5.105329in}{4.225185in}}%
\pgfpathlineto{\pgfqpoint{5.111772in}{4.229250in}}%
\pgfpathlineto{\pgfqpoint{5.112846in}{4.227086in}}%
\pgfpathlineto{\pgfqpoint{5.116068in}{4.225835in}}%
\pgfpathlineto{\pgfqpoint{5.117141in}{4.224454in}}%
\pgfpathlineto{\pgfqpoint{5.118215in}{4.225817in}}%
\pgfpathlineto{\pgfqpoint{5.120363in}{4.223789in}}%
\pgfpathlineto{\pgfqpoint{5.123584in}{4.223600in}}%
\pgfpathlineto{\pgfqpoint{5.124658in}{4.222508in}}%
\pgfpathlineto{\pgfqpoint{5.126806in}{4.221957in}}%
\pgfpathlineto{\pgfqpoint{5.127880in}{4.221038in}}%
\pgfpathlineto{\pgfqpoint{5.132175in}{4.219993in}}%
\pgfpathlineto{\pgfqpoint{5.133249in}{4.219258in}}%
\pgfpathlineto{\pgfqpoint{5.134323in}{4.219842in}}%
\pgfpathlineto{\pgfqpoint{5.135397in}{4.219361in}}%
\pgfpathlineto{\pgfqpoint{5.139692in}{4.219721in}}%
\pgfpathlineto{\pgfqpoint{5.150430in}{4.220677in}}%
\pgfpathlineto{\pgfqpoint{5.162243in}{4.218818in}}%
\pgfpathlineto{\pgfqpoint{5.163316in}{4.220023in}}%
\pgfpathlineto{\pgfqpoint{5.168686in}{4.218400in}}%
\pgfpathlineto{\pgfqpoint{5.171907in}{4.218689in}}%
\pgfpathlineto{\pgfqpoint{5.172981in}{4.220148in}}%
\pgfpathlineto{\pgfqpoint{5.176202in}{4.220835in}}%
\pgfpathlineto{\pgfqpoint{5.177276in}{4.222150in}}%
\pgfpathlineto{\pgfqpoint{5.179424in}{4.218496in}}%
\pgfpathlineto{\pgfqpoint{5.180498in}{4.218680in}}%
\pgfpathlineto{\pgfqpoint{5.188015in}{4.217736in}}%
\pgfpathlineto{\pgfqpoint{5.192310in}{4.217750in}}%
\pgfpathlineto{\pgfqpoint{5.193384in}{4.218568in}}%
\pgfpathlineto{\pgfqpoint{5.195532in}{4.218583in}}%
\pgfpathlineto{\pgfqpoint{5.198753in}{4.220269in}}%
\pgfpathlineto{\pgfqpoint{5.199827in}{4.221660in}}%
\pgfpathlineto{\pgfqpoint{5.201975in}{4.219278in}}%
\pgfpathlineto{\pgfqpoint{5.203048in}{4.220092in}}%
\pgfpathlineto{\pgfqpoint{5.207344in}{4.220955in}}%
\pgfpathlineto{\pgfqpoint{5.209491in}{4.223476in}}%
\pgfpathlineto{\pgfqpoint{5.214861in}{4.222375in}}%
\pgfpathlineto{\pgfqpoint{5.215935in}{4.222059in}}%
\pgfpathlineto{\pgfqpoint{5.217008in}{4.224582in}}%
\pgfpathlineto{\pgfqpoint{5.221304in}{4.225472in}}%
\pgfpathlineto{\pgfqpoint{5.223451in}{4.226812in}}%
\pgfpathlineto{\pgfqpoint{5.224525in}{4.226576in}}%
\pgfpathlineto{\pgfqpoint{5.225599in}{4.227720in}}%
\pgfpathlineto{\pgfqpoint{5.228821in}{4.226422in}}%
\pgfpathlineto{\pgfqpoint{5.229894in}{4.225030in}}%
\pgfpathlineto{\pgfqpoint{5.230968in}{4.225129in}}%
\pgfpathlineto{\pgfqpoint{5.233116in}{4.223964in}}%
\pgfpathlineto{\pgfqpoint{5.236337in}{4.223981in}}%
\pgfpathlineto{\pgfqpoint{5.238485in}{4.223156in}}%
\pgfpathlineto{\pgfqpoint{5.240633in}{4.229700in}}%
\pgfpathlineto{\pgfqpoint{5.246002in}{4.228093in}}%
\pgfpathlineto{\pgfqpoint{5.247076in}{4.229353in}}%
\pgfpathlineto{\pgfqpoint{5.248150in}{4.227998in}}%
\pgfpathlineto{\pgfqpoint{5.252445in}{4.227061in}}%
\pgfpathlineto{\pgfqpoint{5.254593in}{4.224966in}}%
\pgfpathlineto{\pgfqpoint{5.255667in}{4.226356in}}%
\pgfpathlineto{\pgfqpoint{5.259962in}{4.226073in}}%
\pgfpathlineto{\pgfqpoint{5.261036in}{4.227192in}}%
\pgfpathlineto{\pgfqpoint{5.262110in}{4.226720in}}%
\pgfpathlineto{\pgfqpoint{5.263183in}{4.227461in}}%
\pgfpathlineto{\pgfqpoint{5.266405in}{4.227287in}}%
\pgfpathlineto{\pgfqpoint{5.267479in}{4.228541in}}%
\pgfpathlineto{\pgfqpoint{5.268553in}{4.228221in}}%
\pgfpathlineto{\pgfqpoint{5.269626in}{4.226281in}}%
\pgfpathlineto{\pgfqpoint{5.270700in}{4.227116in}}%
\pgfpathlineto{\pgfqpoint{5.273922in}{4.226318in}}%
\pgfpathlineto{\pgfqpoint{5.274996in}{4.226704in}}%
\pgfpathlineto{\pgfqpoint{5.276069in}{4.225991in}}%
\pgfpathlineto{\pgfqpoint{5.277143in}{4.226271in}}%
\pgfpathlineto{\pgfqpoint{5.278217in}{4.225256in}}%
\pgfpathlineto{\pgfqpoint{5.281439in}{4.225676in}}%
\pgfpathlineto{\pgfqpoint{5.283586in}{4.227500in}}%
\pgfpathlineto{\pgfqpoint{5.284660in}{4.228978in}}%
\pgfpathlineto{\pgfqpoint{5.285734in}{4.229454in}}%
\pgfpathlineto{\pgfqpoint{5.290029in}{4.229064in}}%
\pgfpathlineto{\pgfqpoint{5.292177in}{4.227577in}}%
\pgfpathlineto{\pgfqpoint{5.296472in}{4.227645in}}%
\pgfpathlineto{\pgfqpoint{5.297546in}{4.228827in}}%
\pgfpathlineto{\pgfqpoint{5.300768in}{4.227659in}}%
\pgfpathlineto{\pgfqpoint{5.305063in}{4.227745in}}%
\pgfpathlineto{\pgfqpoint{5.306137in}{4.227510in}}%
\pgfpathlineto{\pgfqpoint{5.307211in}{4.226449in}}%
\pgfpathlineto{\pgfqpoint{5.308285in}{4.229626in}}%
\pgfpathlineto{\pgfqpoint{5.312580in}{4.229657in}}%
\pgfpathlineto{\pgfqpoint{5.313654in}{4.228540in}}%
\pgfpathlineto{\pgfqpoint{5.319023in}{4.229440in}}%
\pgfpathlineto{\pgfqpoint{5.323318in}{4.229252in}}%
\pgfpathlineto{\pgfqpoint{5.326540in}{4.228679in}}%
\pgfpathlineto{\pgfqpoint{5.328688in}{4.229082in}}%
\pgfpathlineto{\pgfqpoint{5.330835in}{4.228019in}}%
\pgfpathlineto{\pgfqpoint{5.334057in}{4.227346in}}%
\pgfpathlineto{\pgfqpoint{5.335131in}{4.227910in}}%
\pgfpathlineto{\pgfqpoint{5.337278in}{4.226386in}}%
\pgfpathlineto{\pgfqpoint{5.338352in}{4.226859in}}%
\pgfpathlineto{\pgfqpoint{5.341574in}{4.226898in}}%
\pgfpathlineto{\pgfqpoint{5.342647in}{4.225832in}}%
\pgfpathlineto{\pgfqpoint{5.344795in}{4.226355in}}%
\pgfpathlineto{\pgfqpoint{5.345869in}{4.225930in}}%
\pgfpathlineto{\pgfqpoint{5.353386in}{4.227269in}}%
\pgfpathlineto{\pgfqpoint{5.357681in}{4.227023in}}%
\pgfpathlineto{\pgfqpoint{5.358755in}{4.226473in}}%
\pgfpathlineto{\pgfqpoint{5.359829in}{4.227575in}}%
\pgfpathlineto{\pgfqpoint{5.360903in}{4.227545in}}%
\pgfpathlineto{\pgfqpoint{5.365198in}{4.227746in}}%
\pgfpathlineto{\pgfqpoint{5.366272in}{4.226811in}}%
\pgfpathlineto{\pgfqpoint{5.367346in}{4.226706in}}%
\pgfpathlineto{\pgfqpoint{5.368420in}{4.227394in}}%
\pgfpathlineto{\pgfqpoint{5.372715in}{4.227549in}}%
\pgfpathlineto{\pgfqpoint{5.374863in}{4.226267in}}%
\pgfpathlineto{\pgfqpoint{5.375936in}{4.226787in}}%
\pgfpathlineto{\pgfqpoint{5.380232in}{4.225735in}}%
\pgfpathlineto{\pgfqpoint{5.386675in}{4.228914in}}%
\pgfpathlineto{\pgfqpoint{5.389896in}{4.228242in}}%
\pgfpathlineto{\pgfqpoint{5.395266in}{4.228862in}}%
\pgfpathlineto{\pgfqpoint{5.396339in}{4.230407in}}%
\pgfpathlineto{\pgfqpoint{5.406004in}{4.227240in}}%
\pgfpathlineto{\pgfqpoint{5.411373in}{4.227443in}}%
\pgfpathlineto{\pgfqpoint{5.413521in}{4.230333in}}%
\pgfpathlineto{\pgfqpoint{5.417816in}{4.231111in}}%
\pgfpathlineto{\pgfqpoint{5.419964in}{4.230091in}}%
\pgfpathlineto{\pgfqpoint{5.426407in}{4.230620in}}%
\pgfpathlineto{\pgfqpoint{5.427481in}{4.231207in}}%
\pgfpathlineto{\pgfqpoint{5.428555in}{4.226416in}}%
\pgfpathlineto{\pgfqpoint{5.432850in}{4.224966in}}%
\pgfpathlineto{\pgfqpoint{5.436071in}{4.225333in}}%
\pgfpathlineto{\pgfqpoint{5.441441in}{4.225294in}}%
\pgfpathlineto{\pgfqpoint{5.443588in}{4.228787in}}%
\pgfpathlineto{\pgfqpoint{5.447884in}{4.232218in}}%
\pgfpathlineto{\pgfqpoint{5.450031in}{4.228762in}}%
\pgfpathlineto{\pgfqpoint{5.451105in}{4.228987in}}%
\pgfpathlineto{\pgfqpoint{5.454327in}{4.228915in}}%
\pgfpathlineto{\pgfqpoint{5.455401in}{4.231536in}}%
\pgfpathlineto{\pgfqpoint{5.457548in}{4.230242in}}%
\pgfpathlineto{\pgfqpoint{5.458622in}{4.231426in}}%
\pgfpathlineto{\pgfqpoint{5.462917in}{4.229962in}}%
\pgfpathlineto{\pgfqpoint{5.465065in}{4.230140in}}%
\pgfpathlineto{\pgfqpoint{5.466139in}{4.229949in}}%
\pgfpathlineto{\pgfqpoint{5.469360in}{4.230309in}}%
\pgfpathlineto{\pgfqpoint{5.471508in}{4.228157in}}%
\pgfpathlineto{\pgfqpoint{5.472582in}{4.228399in}}%
\pgfpathlineto{\pgfqpoint{5.473656in}{4.229456in}}%
\pgfpathlineto{\pgfqpoint{5.476877in}{4.228682in}}%
\pgfpathlineto{\pgfqpoint{5.477951in}{4.229675in}}%
\pgfpathlineto{\pgfqpoint{5.479025in}{4.229749in}}%
\pgfpathlineto{\pgfqpoint{5.480099in}{4.230677in}}%
\pgfpathlineto{\pgfqpoint{5.481173in}{4.230266in}}%
\pgfpathlineto{\pgfqpoint{5.485468in}{4.232277in}}%
\pgfpathlineto{\pgfqpoint{5.486542in}{4.231194in}}%
\pgfpathlineto{\pgfqpoint{5.487616in}{4.231435in}}%
\pgfpathlineto{\pgfqpoint{5.488690in}{4.230906in}}%
\pgfpathlineto{\pgfqpoint{5.491911in}{4.227957in}}%
\pgfpathlineto{\pgfqpoint{5.492985in}{4.228319in}}%
\pgfpathlineto{\pgfqpoint{5.495133in}{4.227780in}}%
\pgfpathlineto{\pgfqpoint{5.499428in}{4.227660in}}%
\pgfpathlineto{\pgfqpoint{5.501576in}{4.228746in}}%
\pgfpathlineto{\pgfqpoint{5.503723in}{4.227744in}}%
\pgfpathlineto{\pgfqpoint{5.509092in}{4.228398in}}%
\pgfpathlineto{\pgfqpoint{5.511240in}{4.222536in}}%
\pgfpathlineto{\pgfqpoint{5.518757in}{4.223744in}}%
\pgfpathlineto{\pgfqpoint{5.524126in}{4.222988in}}%
\pgfpathlineto{\pgfqpoint{5.531643in}{4.224054in}}%
\pgfpathlineto{\pgfqpoint{5.533791in}{4.222434in}}%
\pgfpathlineto{\pgfqpoint{5.537012in}{4.222867in}}%
\pgfpathlineto{\pgfqpoint{5.538086in}{4.222343in}}%
\pgfpathlineto{\pgfqpoint{5.539160in}{4.223330in}}%
\pgfpathlineto{\pgfqpoint{5.541308in}{4.222459in}}%
\pgfpathlineto{\pgfqpoint{5.554194in}{4.221331in}}%
\pgfpathlineto{\pgfqpoint{5.555267in}{4.220598in}}%
\pgfpathlineto{\pgfqpoint{5.556341in}{4.221232in}}%
\pgfpathlineto{\pgfqpoint{5.562784in}{4.220311in}}%
\pgfpathlineto{\pgfqpoint{5.563858in}{4.219142in}}%
\pgfpathlineto{\pgfqpoint{5.567080in}{4.219123in}}%
\pgfpathlineto{\pgfqpoint{5.569227in}{4.220875in}}%
\pgfpathlineto{\pgfqpoint{5.571375in}{4.218222in}}%
\pgfpathlineto{\pgfqpoint{5.575670in}{4.219133in}}%
\pgfpathlineto{\pgfqpoint{5.576744in}{4.220182in}}%
\pgfpathlineto{\pgfqpoint{5.577818in}{4.220360in}}%
\pgfpathlineto{\pgfqpoint{5.582113in}{4.219984in}}%
\pgfpathlineto{\pgfqpoint{5.583187in}{4.220717in}}%
\pgfpathlineto{\pgfqpoint{5.585335in}{4.219786in}}%
\pgfpathlineto{\pgfqpoint{5.590704in}{4.216994in}}%
\pgfpathlineto{\pgfqpoint{5.591778in}{4.215271in}}%
\pgfpathlineto{\pgfqpoint{5.593926in}{4.214529in}}%
\pgfpathlineto{\pgfqpoint{5.598221in}{4.215252in}}%
\pgfpathlineto{\pgfqpoint{5.599295in}{4.213796in}}%
\pgfpathlineto{\pgfqpoint{5.600369in}{4.214212in}}%
\pgfpathlineto{\pgfqpoint{5.601443in}{4.213856in}}%
\pgfpathlineto{\pgfqpoint{5.607886in}{4.213608in}}%
\pgfpathlineto{\pgfqpoint{5.608959in}{4.206509in}}%
\pgfpathlineto{\pgfqpoint{5.614329in}{4.206004in}}%
\pgfpathlineto{\pgfqpoint{5.615402in}{4.204470in}}%
\pgfpathlineto{\pgfqpoint{5.619698in}{4.206173in}}%
\pgfpathlineto{\pgfqpoint{5.620772in}{4.205202in}}%
\pgfpathlineto{\pgfqpoint{5.622919in}{4.205876in}}%
\pgfpathlineto{\pgfqpoint{5.630436in}{4.202807in}}%
\pgfpathlineto{\pgfqpoint{5.631510in}{4.204262in}}%
\pgfpathlineto{\pgfqpoint{5.636879in}{4.205153in}}%
\pgfpathlineto{\pgfqpoint{5.639027in}{4.206183in}}%
\pgfpathlineto{\pgfqpoint{5.642248in}{4.207044in}}%
\pgfpathlineto{\pgfqpoint{5.644396in}{4.206113in}}%
\pgfpathlineto{\pgfqpoint{5.645470in}{4.206816in}}%
\pgfpathlineto{\pgfqpoint{5.649765in}{4.206994in}}%
\pgfpathlineto{\pgfqpoint{5.650839in}{4.208133in}}%
\pgfpathlineto{\pgfqpoint{5.651913in}{4.208440in}}%
\pgfpathlineto{\pgfqpoint{5.654061in}{4.209529in}}%
\pgfpathlineto{\pgfqpoint{5.659430in}{4.210242in}}%
\pgfpathlineto{\pgfqpoint{5.660504in}{4.209965in}}%
\pgfpathlineto{\pgfqpoint{5.661578in}{4.210628in}}%
\pgfpathlineto{\pgfqpoint{5.668021in}{4.211212in}}%
\pgfpathlineto{\pgfqpoint{5.669094in}{4.212272in}}%
\pgfpathlineto{\pgfqpoint{5.674464in}{4.211727in}}%
\pgfpathlineto{\pgfqpoint{5.679833in}{4.211390in}}%
\pgfpathlineto{\pgfqpoint{5.680907in}{4.211697in}}%
\pgfpathlineto{\pgfqpoint{5.681980in}{4.211400in}}%
\pgfpathlineto{\pgfqpoint{5.683054in}{4.212440in}}%
\pgfpathlineto{\pgfqpoint{5.690571in}{4.210331in}}%
\pgfpathlineto{\pgfqpoint{5.691645in}{4.210935in}}%
\pgfpathlineto{\pgfqpoint{5.694867in}{4.211212in}}%
\pgfpathlineto{\pgfqpoint{5.698088in}{4.213766in}}%
\pgfpathlineto{\pgfqpoint{5.699162in}{4.213420in}}%
\pgfpathlineto{\pgfqpoint{5.702383in}{4.213846in}}%
\pgfpathlineto{\pgfqpoint{5.706679in}{4.216984in}}%
\pgfpathlineto{\pgfqpoint{5.710974in}{4.217242in}}%
\pgfpathlineto{\pgfqpoint{5.713122in}{4.216895in}}%
\pgfpathlineto{\pgfqpoint{5.714196in}{4.216509in}}%
\pgfpathlineto{\pgfqpoint{5.717417in}{4.216747in}}%
\pgfpathlineto{\pgfqpoint{5.719565in}{4.215410in}}%
\pgfpathlineto{\pgfqpoint{5.720639in}{4.215093in}}%
\pgfpathlineto{\pgfqpoint{5.721712in}{4.215648in}}%
\pgfpathlineto{\pgfqpoint{5.724934in}{4.215153in}}%
\pgfpathlineto{\pgfqpoint{5.726008in}{4.215955in}}%
\pgfpathlineto{\pgfqpoint{5.729229in}{4.215281in}}%
\pgfpathlineto{\pgfqpoint{5.732451in}{4.215232in}}%
\pgfpathlineto{\pgfqpoint{5.733525in}{4.214004in}}%
\pgfpathlineto{\pgfqpoint{5.734599in}{4.214717in}}%
\pgfpathlineto{\pgfqpoint{5.735672in}{4.213984in}}%
\pgfpathlineto{\pgfqpoint{5.736746in}{4.215093in}}%
\pgfpathlineto{\pgfqpoint{5.739968in}{4.214786in}}%
\pgfpathlineto{\pgfqpoint{5.742115in}{4.216400in}}%
\pgfpathlineto{\pgfqpoint{5.744263in}{4.216598in}}%
\pgfpathlineto{\pgfqpoint{5.757149in}{4.217291in}}%
\pgfpathlineto{\pgfqpoint{5.762518in}{4.214856in}}%
\pgfpathlineto{\pgfqpoint{5.763592in}{4.212410in}}%
\pgfpathlineto{\pgfqpoint{5.766814in}{4.213153in}}%
\pgfpathlineto{\pgfqpoint{5.771109in}{4.213559in}}%
\pgfpathlineto{\pgfqpoint{5.772183in}{4.213242in}}%
\pgfpathlineto{\pgfqpoint{5.773257in}{4.214460in}}%
\pgfpathlineto{\pgfqpoint{5.774331in}{4.211460in}}%
\pgfpathlineto{\pgfqpoint{5.777552in}{4.209212in}}%
\pgfpathlineto{\pgfqpoint{5.778626in}{4.209430in}}%
\pgfpathlineto{\pgfqpoint{5.780774in}{4.212410in}}%
\pgfpathlineto{\pgfqpoint{5.781847in}{4.212341in}}%
\pgfpathlineto{\pgfqpoint{5.786143in}{4.210885in}}%
\pgfpathlineto{\pgfqpoint{5.788290in}{4.211509in}}%
\pgfpathlineto{\pgfqpoint{5.789364in}{4.213093in}}%
\pgfpathlineto{\pgfqpoint{5.792586in}{4.213727in}}%
\pgfpathlineto{\pgfqpoint{5.793660in}{4.214529in}}%
\pgfpathlineto{\pgfqpoint{5.801177in}{4.215618in}}%
\pgfpathlineto{\pgfqpoint{5.802250in}{4.215925in}}%
\pgfpathlineto{\pgfqpoint{5.803324in}{4.214935in}}%
\pgfpathlineto{\pgfqpoint{5.804398in}{4.215737in}}%
\pgfpathlineto{\pgfqpoint{5.815136in}{4.215598in}}%
\pgfpathlineto{\pgfqpoint{5.816210in}{4.214974in}}%
\pgfpathlineto{\pgfqpoint{5.818358in}{4.215410in}}%
\pgfpathlineto{\pgfqpoint{5.819432in}{4.216905in}}%
\pgfpathlineto{\pgfqpoint{5.823727in}{4.216786in}}%
\pgfpathlineto{\pgfqpoint{5.824801in}{4.216172in}}%
\pgfpathlineto{\pgfqpoint{5.825875in}{4.216865in}}%
\pgfpathlineto{\pgfqpoint{5.831244in}{4.216697in}}%
\pgfpathlineto{\pgfqpoint{5.833392in}{4.217014in}}%
\pgfpathlineto{\pgfqpoint{5.841982in}{4.216222in}}%
\pgfpathlineto{\pgfqpoint{5.847352in}{4.216964in}}%
\pgfpathlineto{\pgfqpoint{5.848425in}{4.216281in}}%
\pgfpathlineto{\pgfqpoint{5.849499in}{4.216598in}}%
\pgfpathlineto{\pgfqpoint{5.855942in}{4.217588in}}%
\pgfpathlineto{\pgfqpoint{5.857016in}{4.216529in}}%
\pgfpathlineto{\pgfqpoint{5.860238in}{4.217271in}}%
\pgfpathlineto{\pgfqpoint{5.862385in}{4.214974in}}%
\pgfpathlineto{\pgfqpoint{5.864533in}{4.215153in}}%
\pgfpathlineto{\pgfqpoint{5.870976in}{4.216063in}}%
\pgfpathlineto{\pgfqpoint{5.872050in}{4.215331in}}%
\pgfpathlineto{\pgfqpoint{5.875271in}{4.214925in}}%
\pgfpathlineto{\pgfqpoint{5.876345in}{4.215737in}}%
\pgfpathlineto{\pgfqpoint{5.877419in}{4.215677in}}%
\pgfpathlineto{\pgfqpoint{5.878493in}{4.214875in}}%
\pgfpathlineto{\pgfqpoint{5.879567in}{4.215509in}}%
\pgfpathlineto{\pgfqpoint{5.883862in}{4.215390in}}%
\pgfpathlineto{\pgfqpoint{5.884936in}{4.216113in}}%
\pgfpathlineto{\pgfqpoint{5.886010in}{4.213816in}}%
\pgfpathlineto{\pgfqpoint{5.887084in}{4.213648in}}%
\pgfpathlineto{\pgfqpoint{5.890305in}{4.213776in}}%
\pgfpathlineto{\pgfqpoint{5.891379in}{4.212796in}}%
\pgfpathlineto{\pgfqpoint{5.898896in}{4.212054in}}%
\pgfpathlineto{\pgfqpoint{5.899970in}{4.213163in}}%
\pgfpathlineto{\pgfqpoint{5.901044in}{4.218420in}}%
\pgfpathlineto{\pgfqpoint{5.902117in}{4.218964in}}%
\pgfpathlineto{\pgfqpoint{5.913930in}{4.217915in}}%
\pgfpathlineto{\pgfqpoint{5.916077in}{4.217044in}}%
\pgfpathlineto{\pgfqpoint{5.917151in}{4.217212in}}%
\pgfpathlineto{\pgfqpoint{5.921446in}{4.218687in}}%
\pgfpathlineto{\pgfqpoint{5.922520in}{4.220281in}}%
\pgfpathlineto{\pgfqpoint{5.923594in}{4.218984in}}%
\pgfpathlineto{\pgfqpoint{5.924668in}{4.218623in}}%
\pgfpathlineto{\pgfqpoint{5.928963in}{4.216847in}}%
\pgfpathlineto{\pgfqpoint{5.930037in}{4.217523in}}%
\pgfpathlineto{\pgfqpoint{5.931111in}{4.217441in}}%
\pgfpathlineto{\pgfqpoint{5.932185in}{4.218112in}}%
\pgfpathlineto{\pgfqpoint{5.936480in}{4.217475in}}%
\pgfpathlineto{\pgfqpoint{5.937554in}{4.216467in}}%
\pgfpathlineto{\pgfqpoint{5.939702in}{4.216485in}}%
\pgfpathlineto{\pgfqpoint{5.945071in}{4.217173in}}%
\pgfpathlineto{\pgfqpoint{5.946145in}{4.216750in}}%
\pgfpathlineto{\pgfqpoint{5.947219in}{4.217319in}}%
\pgfpathlineto{\pgfqpoint{5.951514in}{4.216931in}}%
\pgfpathlineto{\pgfqpoint{5.952588in}{4.215234in}}%
\pgfpathlineto{\pgfqpoint{5.953662in}{4.214727in}}%
\pgfpathlineto{\pgfqpoint{5.954735in}{4.214843in}}%
\pgfpathlineto{\pgfqpoint{5.959031in}{4.215579in}}%
\pgfpathlineto{\pgfqpoint{5.967621in}{4.214374in}}%
\pgfpathlineto{\pgfqpoint{5.968695in}{4.214958in}}%
\pgfpathlineto{\pgfqpoint{5.969769in}{4.214639in}}%
\pgfpathlineto{\pgfqpoint{5.975138in}{4.215120in}}%
\pgfpathlineto{\pgfqpoint{5.977286in}{4.215354in}}%
\pgfpathlineto{\pgfqpoint{5.981581in}{4.214318in}}%
\pgfpathlineto{\pgfqpoint{5.982655in}{4.213341in}}%
\pgfpathlineto{\pgfqpoint{5.983729in}{4.214064in}}%
\pgfpathlineto{\pgfqpoint{5.998763in}{4.212985in}}%
\pgfpathlineto{\pgfqpoint{5.999837in}{4.213354in}}%
\pgfpathlineto{\pgfqpoint{6.003058in}{4.213529in}}%
\pgfpathlineto{\pgfqpoint{6.004132in}{4.212409in}}%
\pgfpathlineto{\pgfqpoint{6.006280in}{4.212778in}}%
\pgfpathlineto{\pgfqpoint{6.013797in}{4.213080in}}%
\pgfpathlineto{\pgfqpoint{6.014870in}{4.211922in}}%
\pgfpathlineto{\pgfqpoint{6.021313in}{4.211814in}}%
\pgfpathlineto{\pgfqpoint{6.027756in}{4.210794in}}%
\pgfpathlineto{\pgfqpoint{6.029904in}{4.210720in}}%
\pgfpathlineto{\pgfqpoint{6.041716in}{4.210472in}}%
\pgfpathlineto{\pgfqpoint{6.042790in}{4.209184in}}%
\pgfpathlineto{\pgfqpoint{6.043864in}{4.210386in}}%
\pgfpathlineto{\pgfqpoint{6.051381in}{4.210939in}}%
\pgfpathlineto{\pgfqpoint{6.059972in}{4.210967in}}%
\pgfpathlineto{\pgfqpoint{6.063193in}{4.211253in}}%
\pgfpathlineto{\pgfqpoint{6.064267in}{4.212193in}}%
\pgfpathlineto{\pgfqpoint{6.067488in}{4.211698in}}%
\pgfpathlineto{\pgfqpoint{6.070710in}{4.212135in}}%
\pgfpathlineto{\pgfqpoint{6.071784in}{4.211370in}}%
\pgfpathlineto{\pgfqpoint{6.072858in}{4.211653in}}%
\pgfpathlineto{\pgfqpoint{6.073932in}{4.210954in}}%
\pgfpathlineto{\pgfqpoint{6.086818in}{4.212124in}}%
\pgfpathlineto{\pgfqpoint{6.088965in}{4.213219in}}%
\pgfpathlineto{\pgfqpoint{6.093261in}{4.212552in}}%
\pgfpathlineto{\pgfqpoint{6.095408in}{4.213395in}}%
\pgfpathlineto{\pgfqpoint{6.096482in}{4.209934in}}%
\pgfpathlineto{\pgfqpoint{6.097556in}{4.210232in}}%
\pgfpathlineto{\pgfqpoint{6.102925in}{4.209587in}}%
\pgfpathlineto{\pgfqpoint{6.103999in}{4.209714in}}%
\pgfpathlineto{\pgfqpoint{6.105073in}{4.210453in}}%
\pgfpathlineto{\pgfqpoint{6.110442in}{4.210739in}}%
\pgfpathlineto{\pgfqpoint{6.112590in}{4.211104in}}%
\pgfpathlineto{\pgfqpoint{6.124402in}{4.211223in}}%
\pgfpathlineto{\pgfqpoint{6.125476in}{4.212495in}}%
\pgfpathlineto{\pgfqpoint{6.130845in}{4.212038in}}%
\pgfpathlineto{\pgfqpoint{6.135140in}{4.211678in}}%
\pgfpathlineto{\pgfqpoint{6.140509in}{4.212058in}}%
\pgfpathlineto{\pgfqpoint{6.141583in}{4.211069in}}%
\pgfpathlineto{\pgfqpoint{6.164134in}{4.208338in}}%
\pgfpathlineto{\pgfqpoint{6.168429in}{4.207875in}}%
\pgfpathlineto{\pgfqpoint{6.172725in}{4.207060in}}%
\pgfpathlineto{\pgfqpoint{6.178094in}{4.206461in}}%
\pgfpathlineto{\pgfqpoint{6.179168in}{4.207160in}}%
\pgfpathlineto{\pgfqpoint{6.180242in}{4.206788in}}%
\pgfpathlineto{\pgfqpoint{6.185611in}{4.206489in}}%
\pgfpathlineto{\pgfqpoint{6.187758in}{4.206207in}}%
\pgfpathlineto{\pgfqpoint{6.200644in}{4.206187in}}%
\pgfpathlineto{\pgfqpoint{6.201718in}{4.207058in}}%
\pgfpathlineto{\pgfqpoint{6.202792in}{4.206470in}}%
\pgfpathlineto{\pgfqpoint{6.215678in}{4.205928in}}%
\pgfpathlineto{\pgfqpoint{6.217826in}{4.206765in}}%
\pgfpathlineto{\pgfqpoint{6.222121in}{4.205250in}}%
\pgfpathlineto{\pgfqpoint{6.223195in}{4.204904in}}%
\pgfpathlineto{\pgfqpoint{6.225343in}{4.206063in}}%
\pgfpathlineto{\pgfqpoint{6.238229in}{4.205879in}}%
\pgfpathlineto{\pgfqpoint{6.240376in}{4.205616in}}%
\pgfpathlineto{\pgfqpoint{6.247893in}{4.206739in}}%
\pgfpathlineto{\pgfqpoint{6.253263in}{4.205292in}}%
\pgfpathlineto{\pgfqpoint{6.254336in}{4.205555in}}%
\pgfpathlineto{\pgfqpoint{6.255410in}{4.205083in}}%
\pgfpathlineto{\pgfqpoint{6.262927in}{4.204269in}}%
\pgfpathlineto{\pgfqpoint{6.267222in}{4.204107in}}%
\pgfpathlineto{\pgfqpoint{6.270444in}{4.203138in}}%
\pgfpathlineto{\pgfqpoint{6.276887in}{4.202578in}}%
\pgfpathlineto{\pgfqpoint{6.281182in}{4.202300in}}%
\pgfpathlineto{\pgfqpoint{6.288699in}{4.202149in}}%
\pgfpathlineto{\pgfqpoint{6.291921in}{4.202179in}}%
\pgfpathlineto{\pgfqpoint{6.298364in}{4.201343in}}%
\pgfpathlineto{\pgfqpoint{6.299438in}{4.200270in}}%
\pgfpathlineto{\pgfqpoint{6.311250in}{4.199984in}}%
\pgfpathlineto{\pgfqpoint{6.315545in}{4.201303in}}%
\pgfpathlineto{\pgfqpoint{6.327357in}{4.200861in}}%
\pgfpathlineto{\pgfqpoint{6.330579in}{4.201317in}}%
\pgfpathlineto{\pgfqpoint{6.333800in}{4.201322in}}%
\pgfpathlineto{\pgfqpoint{6.337022in}{4.199276in}}%
\pgfpathlineto{\pgfqpoint{6.349908in}{4.198530in}}%
\pgfpathlineto{\pgfqpoint{6.352056in}{4.199514in}}%
\pgfpathlineto{\pgfqpoint{6.353130in}{4.198884in}}%
\pgfpathlineto{\pgfqpoint{6.357425in}{4.198600in}}%
\pgfpathlineto{\pgfqpoint{6.359573in}{4.198884in}}%
\pgfpathlineto{\pgfqpoint{6.368163in}{4.198522in}}%
\pgfpathlineto{\pgfqpoint{6.368163in}{4.198522in}}%
\pgfusepath{stroke}%
\end{pgfscope}%
\begin{pgfscope}%
\pgfsetrectcap%
\pgfsetmiterjoin%
\pgfsetlinewidth{0.803000pt}%
\definecolor{currentstroke}{rgb}{1.000000,1.000000,1.000000}%
\pgfsetstrokecolor{currentstroke}%
\pgfsetdash{}{0pt}%
\pgfpathmoveto{\pgfqpoint{3.902254in}{4.180131in}}%
\pgfpathlineto{\pgfqpoint{3.902254in}{4.581016in}}%
\pgfusepath{stroke}%
\end{pgfscope}%
\begin{pgfscope}%
\pgfsetrectcap%
\pgfsetmiterjoin%
\pgfsetlinewidth{0.803000pt}%
\definecolor{currentstroke}{rgb}{1.000000,1.000000,1.000000}%
\pgfsetstrokecolor{currentstroke}%
\pgfsetdash{}{0pt}%
\pgfpathmoveto{\pgfqpoint{6.485588in}{4.180131in}}%
\pgfpathlineto{\pgfqpoint{6.485588in}{4.581016in}}%
\pgfusepath{stroke}%
\end{pgfscope}%
\begin{pgfscope}%
\pgfsetrectcap%
\pgfsetmiterjoin%
\pgfsetlinewidth{0.803000pt}%
\definecolor{currentstroke}{rgb}{1.000000,1.000000,1.000000}%
\pgfsetstrokecolor{currentstroke}%
\pgfsetdash{}{0pt}%
\pgfpathmoveto{\pgfqpoint{3.902254in}{4.180131in}}%
\pgfpathlineto{\pgfqpoint{6.485588in}{4.180131in}}%
\pgfusepath{stroke}%
\end{pgfscope}%
\begin{pgfscope}%
\pgfsetrectcap%
\pgfsetmiterjoin%
\pgfsetlinewidth{0.803000pt}%
\definecolor{currentstroke}{rgb}{1.000000,1.000000,1.000000}%
\pgfsetstrokecolor{currentstroke}%
\pgfsetdash{}{0pt}%
\pgfpathmoveto{\pgfqpoint{3.902254in}{4.581016in}}%
\pgfpathlineto{\pgfqpoint{6.485588in}{4.581016in}}%
\pgfusepath{stroke}%
\end{pgfscope}%
\begin{pgfscope}%
\definecolor{textcolor}{rgb}{0.150000,0.150000,0.150000}%
\pgfsetstrokecolor{textcolor}%
\pgfsetfillcolor{textcolor}%
\pgftext[x=5.193921in,y=4.664349in,,base]{\color{textcolor}\rmfamily\fontsize{12.000000}{14.400000}\selectfont AXP}%
\end{pgfscope}%
\begin{pgfscope}%
\pgfsetbuttcap%
\pgfsetmiterjoin%
\definecolor{currentfill}{rgb}{0.917647,0.917647,0.949020}%
\pgfsetfillcolor{currentfill}%
\pgfsetlinewidth{0.000000pt}%
\definecolor{currentstroke}{rgb}{0.000000,0.000000,0.000000}%
\pgfsetstrokecolor{currentstroke}%
\pgfsetstrokeopacity{0.000000}%
\pgfsetdash{}{0pt}%
\pgfpathmoveto{\pgfqpoint{0.285588in}{3.218007in}}%
\pgfpathlineto{\pgfqpoint{2.868921in}{3.218007in}}%
\pgfpathlineto{\pgfqpoint{2.868921in}{3.618892in}}%
\pgfpathlineto{\pgfqpoint{0.285588in}{3.618892in}}%
\pgfpathclose%
\pgfusepath{fill}%
\end{pgfscope}%
\begin{pgfscope}%
\pgfpathrectangle{\pgfqpoint{0.285588in}{3.218007in}}{\pgfqpoint{2.583333in}{0.400885in}}%
\pgfusepath{clip}%
\pgfsetroundcap%
\pgfsetroundjoin%
\pgfsetlinewidth{0.803000pt}%
\definecolor{currentstroke}{rgb}{1.000000,1.000000,1.000000}%
\pgfsetstrokecolor{currentstroke}%
\pgfsetdash{}{0pt}%
\pgfpathmoveto{\pgfqpoint{0.400864in}{3.218007in}}%
\pgfpathlineto{\pgfqpoint{0.400864in}{3.618892in}}%
\pgfusepath{stroke}%
\end{pgfscope}%
\begin{pgfscope}%
\definecolor{textcolor}{rgb}{0.150000,0.150000,0.150000}%
\pgfsetstrokecolor{textcolor}%
\pgfsetfillcolor{textcolor}%
\pgftext[x=0.400864in,y=3.120784in,,top]{\color{textcolor}\rmfamily\fontsize{10.000000}{12.000000}\selectfont 2012}%
\end{pgfscope}%
\begin{pgfscope}%
\pgfpathrectangle{\pgfqpoint{0.285588in}{3.218007in}}{\pgfqpoint{2.583333in}{0.400885in}}%
\pgfusepath{clip}%
\pgfsetroundcap%
\pgfsetroundjoin%
\pgfsetlinewidth{0.803000pt}%
\definecolor{currentstroke}{rgb}{1.000000,1.000000,1.000000}%
\pgfsetstrokecolor{currentstroke}%
\pgfsetdash{}{0pt}%
\pgfpathmoveto{\pgfqpoint{0.793889in}{3.218007in}}%
\pgfpathlineto{\pgfqpoint{0.793889in}{3.618892in}}%
\pgfusepath{stroke}%
\end{pgfscope}%
\begin{pgfscope}%
\definecolor{textcolor}{rgb}{0.150000,0.150000,0.150000}%
\pgfsetstrokecolor{textcolor}%
\pgfsetfillcolor{textcolor}%
\pgftext[x=0.793889in,y=3.120784in,,top]{\color{textcolor}\rmfamily\fontsize{10.000000}{12.000000}\selectfont 2013}%
\end{pgfscope}%
\begin{pgfscope}%
\pgfpathrectangle{\pgfqpoint{0.285588in}{3.218007in}}{\pgfqpoint{2.583333in}{0.400885in}}%
\pgfusepath{clip}%
\pgfsetroundcap%
\pgfsetroundjoin%
\pgfsetlinewidth{0.803000pt}%
\definecolor{currentstroke}{rgb}{1.000000,1.000000,1.000000}%
\pgfsetstrokecolor{currentstroke}%
\pgfsetdash{}{0pt}%
\pgfpathmoveto{\pgfqpoint{1.185840in}{3.218007in}}%
\pgfpathlineto{\pgfqpoint{1.185840in}{3.618892in}}%
\pgfusepath{stroke}%
\end{pgfscope}%
\begin{pgfscope}%
\definecolor{textcolor}{rgb}{0.150000,0.150000,0.150000}%
\pgfsetstrokecolor{textcolor}%
\pgfsetfillcolor{textcolor}%
\pgftext[x=1.185840in,y=3.120784in,,top]{\color{textcolor}\rmfamily\fontsize{10.000000}{12.000000}\selectfont 2014}%
\end{pgfscope}%
\begin{pgfscope}%
\pgfpathrectangle{\pgfqpoint{0.285588in}{3.218007in}}{\pgfqpoint{2.583333in}{0.400885in}}%
\pgfusepath{clip}%
\pgfsetroundcap%
\pgfsetroundjoin%
\pgfsetlinewidth{0.803000pt}%
\definecolor{currentstroke}{rgb}{1.000000,1.000000,1.000000}%
\pgfsetstrokecolor{currentstroke}%
\pgfsetdash{}{0pt}%
\pgfpathmoveto{\pgfqpoint{1.577791in}{3.218007in}}%
\pgfpathlineto{\pgfqpoint{1.577791in}{3.618892in}}%
\pgfusepath{stroke}%
\end{pgfscope}%
\begin{pgfscope}%
\definecolor{textcolor}{rgb}{0.150000,0.150000,0.150000}%
\pgfsetstrokecolor{textcolor}%
\pgfsetfillcolor{textcolor}%
\pgftext[x=1.577791in,y=3.120784in,,top]{\color{textcolor}\rmfamily\fontsize{10.000000}{12.000000}\selectfont 2015}%
\end{pgfscope}%
\begin{pgfscope}%
\pgfpathrectangle{\pgfqpoint{0.285588in}{3.218007in}}{\pgfqpoint{2.583333in}{0.400885in}}%
\pgfusepath{clip}%
\pgfsetroundcap%
\pgfsetroundjoin%
\pgfsetlinewidth{0.803000pt}%
\definecolor{currentstroke}{rgb}{1.000000,1.000000,1.000000}%
\pgfsetstrokecolor{currentstroke}%
\pgfsetdash{}{0pt}%
\pgfpathmoveto{\pgfqpoint{1.969742in}{3.218007in}}%
\pgfpathlineto{\pgfqpoint{1.969742in}{3.618892in}}%
\pgfusepath{stroke}%
\end{pgfscope}%
\begin{pgfscope}%
\definecolor{textcolor}{rgb}{0.150000,0.150000,0.150000}%
\pgfsetstrokecolor{textcolor}%
\pgfsetfillcolor{textcolor}%
\pgftext[x=1.969742in,y=3.120784in,,top]{\color{textcolor}\rmfamily\fontsize{10.000000}{12.000000}\selectfont 2016}%
\end{pgfscope}%
\begin{pgfscope}%
\pgfpathrectangle{\pgfqpoint{0.285588in}{3.218007in}}{\pgfqpoint{2.583333in}{0.400885in}}%
\pgfusepath{clip}%
\pgfsetroundcap%
\pgfsetroundjoin%
\pgfsetlinewidth{0.803000pt}%
\definecolor{currentstroke}{rgb}{1.000000,1.000000,1.000000}%
\pgfsetstrokecolor{currentstroke}%
\pgfsetdash{}{0pt}%
\pgfpathmoveto{\pgfqpoint{2.362767in}{3.218007in}}%
\pgfpathlineto{\pgfqpoint{2.362767in}{3.618892in}}%
\pgfusepath{stroke}%
\end{pgfscope}%
\begin{pgfscope}%
\definecolor{textcolor}{rgb}{0.150000,0.150000,0.150000}%
\pgfsetstrokecolor{textcolor}%
\pgfsetfillcolor{textcolor}%
\pgftext[x=2.362767in,y=3.120784in,,top]{\color{textcolor}\rmfamily\fontsize{10.000000}{12.000000}\selectfont 2017}%
\end{pgfscope}%
\begin{pgfscope}%
\pgfpathrectangle{\pgfqpoint{0.285588in}{3.218007in}}{\pgfqpoint{2.583333in}{0.400885in}}%
\pgfusepath{clip}%
\pgfsetroundcap%
\pgfsetroundjoin%
\pgfsetlinewidth{0.803000pt}%
\definecolor{currentstroke}{rgb}{1.000000,1.000000,1.000000}%
\pgfsetstrokecolor{currentstroke}%
\pgfsetdash{}{0pt}%
\pgfpathmoveto{\pgfqpoint{2.754718in}{3.218007in}}%
\pgfpathlineto{\pgfqpoint{2.754718in}{3.618892in}}%
\pgfusepath{stroke}%
\end{pgfscope}%
\begin{pgfscope}%
\definecolor{textcolor}{rgb}{0.150000,0.150000,0.150000}%
\pgfsetstrokecolor{textcolor}%
\pgfsetfillcolor{textcolor}%
\pgftext[x=2.754718in,y=3.120784in,,top]{\color{textcolor}\rmfamily\fontsize{10.000000}{12.000000}\selectfont 2018}%
\end{pgfscope}%
\begin{pgfscope}%
\pgfpathrectangle{\pgfqpoint{0.285588in}{3.218007in}}{\pgfqpoint{2.583333in}{0.400885in}}%
\pgfusepath{clip}%
\pgfsetroundcap%
\pgfsetroundjoin%
\pgfsetlinewidth{0.803000pt}%
\definecolor{currentstroke}{rgb}{1.000000,1.000000,1.000000}%
\pgfsetstrokecolor{currentstroke}%
\pgfsetdash{}{0pt}%
\pgfpathmoveto{\pgfqpoint{0.285588in}{3.379446in}}%
\pgfpathlineto{\pgfqpoint{2.868921in}{3.379446in}}%
\pgfusepath{stroke}%
\end{pgfscope}%
\begin{pgfscope}%
\definecolor{textcolor}{rgb}{0.150000,0.150000,0.150000}%
\pgfsetstrokecolor{textcolor}%
\pgfsetfillcolor{textcolor}%
\pgftext[x=0.100000in,y=3.326684in,left,base]{\color{textcolor}\rmfamily\fontsize{10.000000}{12.000000}\selectfont 1}%
\end{pgfscope}%
\begin{pgfscope}%
\pgfpathrectangle{\pgfqpoint{0.285588in}{3.218007in}}{\pgfqpoint{2.583333in}{0.400885in}}%
\pgfusepath{clip}%
\pgfsetroundcap%
\pgfsetroundjoin%
\pgfsetlinewidth{0.803000pt}%
\definecolor{currentstroke}{rgb}{1.000000,1.000000,1.000000}%
\pgfsetstrokecolor{currentstroke}%
\pgfsetdash{}{0pt}%
\pgfpathmoveto{\pgfqpoint{0.285588in}{3.582418in}}%
\pgfpathlineto{\pgfqpoint{2.868921in}{3.582418in}}%
\pgfusepath{stroke}%
\end{pgfscope}%
\begin{pgfscope}%
\definecolor{textcolor}{rgb}{0.150000,0.150000,0.150000}%
\pgfsetstrokecolor{textcolor}%
\pgfsetfillcolor{textcolor}%
\pgftext[x=0.100000in,y=3.529657in,left,base]{\color{textcolor}\rmfamily\fontsize{10.000000}{12.000000}\selectfont 2}%
\end{pgfscope}%
\begin{pgfscope}%
\pgfpathrectangle{\pgfqpoint{0.285588in}{3.218007in}}{\pgfqpoint{2.583333in}{0.400885in}}%
\pgfusepath{clip}%
\pgfsetroundcap%
\pgfsetroundjoin%
\pgfsetlinewidth{1.505625pt}%
\definecolor{currentstroke}{rgb}{0.172549,0.627451,0.172549}%
\pgfsetstrokecolor{currentstroke}%
\pgfsetstrokeopacity{0.300000}%
\pgfsetdash{}{0pt}%
\pgfpathmoveto{\pgfqpoint{0.403012in}{3.379446in}}%
\pgfpathlineto{\pgfqpoint{0.404086in}{3.381654in}}%
\pgfpathlineto{\pgfqpoint{0.405159in}{3.381507in}}%
\pgfpathlineto{\pgfqpoint{0.406233in}{3.382684in}}%
\pgfpathlineto{\pgfqpoint{0.409455in}{3.385039in}}%
\pgfpathlineto{\pgfqpoint{0.410529in}{3.383420in}}%
\pgfpathlineto{\pgfqpoint{0.411602in}{3.385186in}}%
\pgfpathlineto{\pgfqpoint{0.412676in}{3.385775in}}%
\pgfpathlineto{\pgfqpoint{0.413750in}{3.384745in}}%
\pgfpathlineto{\pgfqpoint{0.418045in}{3.383714in}}%
\pgfpathlineto{\pgfqpoint{0.419119in}{3.386805in}}%
\pgfpathlineto{\pgfqpoint{0.420193in}{3.388130in}}%
\pgfpathlineto{\pgfqpoint{0.421267in}{3.388130in}}%
\pgfpathlineto{\pgfqpoint{0.425562in}{3.384745in}}%
\pgfpathlineto{\pgfqpoint{0.426636in}{3.387983in}}%
\pgfpathlineto{\pgfqpoint{0.428784in}{3.386805in}}%
\pgfpathlineto{\pgfqpoint{0.432005in}{3.385481in}}%
\pgfpathlineto{\pgfqpoint{0.433079in}{3.383273in}}%
\pgfpathlineto{\pgfqpoint{0.434153in}{3.384009in}}%
\pgfpathlineto{\pgfqpoint{0.435227in}{3.383714in}}%
\pgfpathlineto{\pgfqpoint{0.436301in}{3.386805in}}%
\pgfpathlineto{\pgfqpoint{0.439522in}{3.387100in}}%
\pgfpathlineto{\pgfqpoint{0.441670in}{3.389160in}}%
\pgfpathlineto{\pgfqpoint{0.442744in}{3.387983in}}%
\pgfpathlineto{\pgfqpoint{0.443818in}{3.385186in}}%
\pgfpathlineto{\pgfqpoint{0.447039in}{3.387247in}}%
\pgfpathlineto{\pgfqpoint{0.449187in}{3.383862in}}%
\pgfpathlineto{\pgfqpoint{0.451334in}{3.389602in}}%
\pgfpathlineto{\pgfqpoint{0.455630in}{3.391074in}}%
\pgfpathlineto{\pgfqpoint{0.456704in}{3.390779in}}%
\pgfpathlineto{\pgfqpoint{0.457778in}{3.391810in}}%
\pgfpathlineto{\pgfqpoint{0.462073in}{3.389160in}}%
\pgfpathlineto{\pgfqpoint{0.463147in}{3.390191in}}%
\pgfpathlineto{\pgfqpoint{0.464221in}{3.389013in}}%
\pgfpathlineto{\pgfqpoint{0.465294in}{3.389749in}}%
\pgfpathlineto{\pgfqpoint{0.466368in}{3.388130in}}%
\pgfpathlineto{\pgfqpoint{0.469590in}{3.386658in}}%
\pgfpathlineto{\pgfqpoint{0.470664in}{3.381948in}}%
\pgfpathlineto{\pgfqpoint{0.472811in}{3.388719in}}%
\pgfpathlineto{\pgfqpoint{0.473885in}{3.388866in}}%
\pgfpathlineto{\pgfqpoint{0.477107in}{3.389896in}}%
\pgfpathlineto{\pgfqpoint{0.478180in}{3.395048in}}%
\pgfpathlineto{\pgfqpoint{0.479254in}{3.397256in}}%
\pgfpathlineto{\pgfqpoint{0.480328in}{3.401377in}}%
\pgfpathlineto{\pgfqpoint{0.481402in}{3.401819in}}%
\pgfpathlineto{\pgfqpoint{0.484623in}{3.401966in}}%
\pgfpathlineto{\pgfqpoint{0.485697in}{3.400347in}}%
\pgfpathlineto{\pgfqpoint{0.486771in}{3.400347in}}%
\pgfpathlineto{\pgfqpoint{0.487845in}{3.397844in}}%
\pgfpathlineto{\pgfqpoint{0.488919in}{3.397108in}}%
\pgfpathlineto{\pgfqpoint{0.492140in}{3.400052in}}%
\pgfpathlineto{\pgfqpoint{0.494288in}{3.399611in}}%
\pgfpathlineto{\pgfqpoint{0.495362in}{3.399022in}}%
\pgfpathlineto{\pgfqpoint{0.496436in}{3.400347in}}%
\pgfpathlineto{\pgfqpoint{0.500731in}{3.399169in}}%
\pgfpathlineto{\pgfqpoint{0.502879in}{3.393870in}}%
\pgfpathlineto{\pgfqpoint{0.507174in}{3.390632in}}%
\pgfpathlineto{\pgfqpoint{0.508248in}{3.385481in}}%
\pgfpathlineto{\pgfqpoint{0.510396in}{3.391810in}}%
\pgfpathlineto{\pgfqpoint{0.511469in}{3.387100in}}%
\pgfpathlineto{\pgfqpoint{0.514691in}{3.387247in}}%
\pgfpathlineto{\pgfqpoint{0.515765in}{3.392251in}}%
\pgfpathlineto{\pgfqpoint{0.516839in}{3.389455in}}%
\pgfpathlineto{\pgfqpoint{0.517912in}{3.389896in}}%
\pgfpathlineto{\pgfqpoint{0.518986in}{3.392398in}}%
\pgfpathlineto{\pgfqpoint{0.522208in}{3.389160in}}%
\pgfpathlineto{\pgfqpoint{0.523282in}{3.394459in}}%
\pgfpathlineto{\pgfqpoint{0.524355in}{3.393429in}}%
\pgfpathlineto{\pgfqpoint{0.526503in}{3.397108in}}%
\pgfpathlineto{\pgfqpoint{0.529725in}{3.394901in}}%
\pgfpathlineto{\pgfqpoint{0.530799in}{3.397256in}}%
\pgfpathlineto{\pgfqpoint{0.531872in}{3.396961in}}%
\pgfpathlineto{\pgfqpoint{0.532946in}{3.395195in}}%
\pgfpathlineto{\pgfqpoint{0.534020in}{3.392251in}}%
\pgfpathlineto{\pgfqpoint{0.537242in}{3.391957in}}%
\pgfpathlineto{\pgfqpoint{0.538315in}{3.391221in}}%
\pgfpathlineto{\pgfqpoint{0.539389in}{3.387394in}}%
\pgfpathlineto{\pgfqpoint{0.540463in}{3.389455in}}%
\pgfpathlineto{\pgfqpoint{0.541537in}{3.388572in}}%
\pgfpathlineto{\pgfqpoint{0.545832in}{3.381654in}}%
\pgfpathlineto{\pgfqpoint{0.546906in}{3.388424in}}%
\pgfpathlineto{\pgfqpoint{0.547980in}{3.387100in}}%
\pgfpathlineto{\pgfqpoint{0.553349in}{3.390338in}}%
\pgfpathlineto{\pgfqpoint{0.554423in}{3.390338in}}%
\pgfpathlineto{\pgfqpoint{0.555497in}{3.391221in}}%
\pgfpathlineto{\pgfqpoint{0.556571in}{3.390632in}}%
\pgfpathlineto{\pgfqpoint{0.560866in}{3.392251in}}%
\pgfpathlineto{\pgfqpoint{0.561940in}{3.388866in}}%
\pgfpathlineto{\pgfqpoint{0.563014in}{3.389455in}}%
\pgfpathlineto{\pgfqpoint{0.564088in}{3.383273in}}%
\pgfpathlineto{\pgfqpoint{0.567309in}{3.378857in}}%
\pgfpathlineto{\pgfqpoint{0.568383in}{3.379887in}}%
\pgfpathlineto{\pgfqpoint{0.569457in}{3.387100in}}%
\pgfpathlineto{\pgfqpoint{0.570531in}{3.388424in}}%
\pgfpathlineto{\pgfqpoint{0.571604in}{3.390632in}}%
\pgfpathlineto{\pgfqpoint{0.574826in}{3.389602in}}%
\pgfpathlineto{\pgfqpoint{0.575900in}{3.393723in}}%
\pgfpathlineto{\pgfqpoint{0.576974in}{3.392546in}}%
\pgfpathlineto{\pgfqpoint{0.579121in}{3.400199in}}%
\pgfpathlineto{\pgfqpoint{0.582343in}{3.397403in}}%
\pgfpathlineto{\pgfqpoint{0.583417in}{3.400199in}}%
\pgfpathlineto{\pgfqpoint{0.584490in}{3.401230in}}%
\pgfpathlineto{\pgfqpoint{0.585564in}{3.396814in}}%
\pgfpathlineto{\pgfqpoint{0.586638in}{3.399905in}}%
\pgfpathlineto{\pgfqpoint{0.589860in}{3.396667in}}%
\pgfpathlineto{\pgfqpoint{0.593081in}{3.404321in}}%
\pgfpathlineto{\pgfqpoint{0.594155in}{3.411533in}}%
\pgfpathlineto{\pgfqpoint{0.598450in}{3.406970in}}%
\pgfpathlineto{\pgfqpoint{0.600598in}{3.405793in}}%
\pgfpathlineto{\pgfqpoint{0.601672in}{3.402113in}}%
\pgfpathlineto{\pgfqpoint{0.604893in}{3.402554in}}%
\pgfpathlineto{\pgfqpoint{0.605967in}{3.397844in}}%
\pgfpathlineto{\pgfqpoint{0.607041in}{3.398433in}}%
\pgfpathlineto{\pgfqpoint{0.608115in}{3.395784in}}%
\pgfpathlineto{\pgfqpoint{0.609189in}{3.399463in}}%
\pgfpathlineto{\pgfqpoint{0.612410in}{3.397403in}}%
\pgfpathlineto{\pgfqpoint{0.614558in}{3.400347in}}%
\pgfpathlineto{\pgfqpoint{0.615632in}{3.399758in}}%
\pgfpathlineto{\pgfqpoint{0.619927in}{3.403143in}}%
\pgfpathlineto{\pgfqpoint{0.621001in}{3.401671in}}%
\pgfpathlineto{\pgfqpoint{0.622075in}{3.402113in}}%
\pgfpathlineto{\pgfqpoint{0.624222in}{3.412416in}}%
\pgfpathlineto{\pgfqpoint{0.629592in}{3.410355in}}%
\pgfpathlineto{\pgfqpoint{0.630666in}{3.408000in}}%
\pgfpathlineto{\pgfqpoint{0.631739in}{3.412858in}}%
\pgfpathlineto{\pgfqpoint{0.634961in}{3.413005in}}%
\pgfpathlineto{\pgfqpoint{0.636035in}{3.414771in}}%
\pgfpathlineto{\pgfqpoint{0.637109in}{3.413446in}}%
\pgfpathlineto{\pgfqpoint{0.639256in}{3.414477in}}%
\pgfpathlineto{\pgfqpoint{0.644625in}{3.412858in}}%
\pgfpathlineto{\pgfqpoint{0.645699in}{3.413888in}}%
\pgfpathlineto{\pgfqpoint{0.646773in}{3.413299in}}%
\pgfpathlineto{\pgfqpoint{0.649995in}{3.412563in}}%
\pgfpathlineto{\pgfqpoint{0.653216in}{3.409325in}}%
\pgfpathlineto{\pgfqpoint{0.654290in}{3.411091in}}%
\pgfpathlineto{\pgfqpoint{0.657511in}{3.411680in}}%
\pgfpathlineto{\pgfqpoint{0.659659in}{3.411386in}}%
\pgfpathlineto{\pgfqpoint{0.660733in}{3.409325in}}%
\pgfpathlineto{\pgfqpoint{0.661807in}{3.410061in}}%
\pgfpathlineto{\pgfqpoint{0.666102in}{3.407853in}}%
\pgfpathlineto{\pgfqpoint{0.667176in}{3.409472in}}%
\pgfpathlineto{\pgfqpoint{0.668250in}{3.416832in}}%
\pgfpathlineto{\pgfqpoint{0.669324in}{3.420070in}}%
\pgfpathlineto{\pgfqpoint{0.672545in}{3.418745in}}%
\pgfpathlineto{\pgfqpoint{0.673619in}{3.420070in}}%
\pgfpathlineto{\pgfqpoint{0.674693in}{3.423455in}}%
\pgfpathlineto{\pgfqpoint{0.676841in}{3.425810in}}%
\pgfpathlineto{\pgfqpoint{0.680062in}{3.425221in}}%
\pgfpathlineto{\pgfqpoint{0.684357in}{3.432581in}}%
\pgfpathlineto{\pgfqpoint{0.688653in}{3.430079in}}%
\pgfpathlineto{\pgfqpoint{0.689727in}{3.427724in}}%
\pgfpathlineto{\pgfqpoint{0.690800in}{3.434789in}}%
\pgfpathlineto{\pgfqpoint{0.691874in}{3.434641in}}%
\pgfpathlineto{\pgfqpoint{0.697243in}{3.436849in}}%
\pgfpathlineto{\pgfqpoint{0.698317in}{3.437291in}}%
\pgfpathlineto{\pgfqpoint{0.699391in}{3.439204in}}%
\pgfpathlineto{\pgfqpoint{0.702613in}{3.436996in}}%
\pgfpathlineto{\pgfqpoint{0.704760in}{3.431403in}}%
\pgfpathlineto{\pgfqpoint{0.705834in}{3.432286in}}%
\pgfpathlineto{\pgfqpoint{0.706908in}{3.431992in}}%
\pgfpathlineto{\pgfqpoint{0.711203in}{3.433758in}}%
\pgfpathlineto{\pgfqpoint{0.712277in}{3.436849in}}%
\pgfpathlineto{\pgfqpoint{0.713351in}{3.435672in}}%
\pgfpathlineto{\pgfqpoint{0.714425in}{3.426840in}}%
\pgfpathlineto{\pgfqpoint{0.717646in}{3.423161in}}%
\pgfpathlineto{\pgfqpoint{0.718720in}{3.418304in}}%
\pgfpathlineto{\pgfqpoint{0.720868in}{3.418156in}}%
\pgfpathlineto{\pgfqpoint{0.721942in}{3.416390in}}%
\pgfpathlineto{\pgfqpoint{0.727311in}{3.415801in}}%
\pgfpathlineto{\pgfqpoint{0.728385in}{3.419040in}}%
\pgfpathlineto{\pgfqpoint{0.729459in}{3.418745in}}%
\pgfpathlineto{\pgfqpoint{0.732680in}{3.419775in}}%
\pgfpathlineto{\pgfqpoint{0.733754in}{3.421836in}}%
\pgfpathlineto{\pgfqpoint{0.735902in}{3.413888in}}%
\pgfpathlineto{\pgfqpoint{0.736976in}{3.415213in}}%
\pgfpathlineto{\pgfqpoint{0.740197in}{3.413888in}}%
\pgfpathlineto{\pgfqpoint{0.741271in}{3.411533in}}%
\pgfpathlineto{\pgfqpoint{0.742345in}{3.403879in}}%
\pgfpathlineto{\pgfqpoint{0.744492in}{3.405498in}}%
\pgfpathlineto{\pgfqpoint{0.747714in}{3.411239in}}%
\pgfpathlineto{\pgfqpoint{0.748788in}{3.410797in}}%
\pgfpathlineto{\pgfqpoint{0.749862in}{3.411533in}}%
\pgfpathlineto{\pgfqpoint{0.752009in}{3.415654in}}%
\pgfpathlineto{\pgfqpoint{0.755231in}{3.415801in}}%
\pgfpathlineto{\pgfqpoint{0.756305in}{3.413741in}}%
\pgfpathlineto{\pgfqpoint{0.757378in}{3.416832in}}%
\pgfpathlineto{\pgfqpoint{0.759526in}{3.416684in}}%
\pgfpathlineto{\pgfqpoint{0.762748in}{3.413152in}}%
\pgfpathlineto{\pgfqpoint{0.763821in}{3.413594in}}%
\pgfpathlineto{\pgfqpoint{0.764895in}{3.417715in}}%
\pgfpathlineto{\pgfqpoint{0.767043in}{3.420364in}}%
\pgfpathlineto{\pgfqpoint{0.770265in}{3.419628in}}%
\pgfpathlineto{\pgfqpoint{0.771338in}{3.420953in}}%
\pgfpathlineto{\pgfqpoint{0.772412in}{3.424044in}}%
\pgfpathlineto{\pgfqpoint{0.773486in}{3.422278in}}%
\pgfpathlineto{\pgfqpoint{0.774560in}{3.422278in}}%
\pgfpathlineto{\pgfqpoint{0.777781in}{3.425810in}}%
\pgfpathlineto{\pgfqpoint{0.778855in}{3.423014in}}%
\pgfpathlineto{\pgfqpoint{0.779929in}{3.415213in}}%
\pgfpathlineto{\pgfqpoint{0.781003in}{3.417862in}}%
\pgfpathlineto{\pgfqpoint{0.782077in}{3.415949in}}%
\pgfpathlineto{\pgfqpoint{0.787446in}{3.414771in}}%
\pgfpathlineto{\pgfqpoint{0.788520in}{3.413741in}}%
\pgfpathlineto{\pgfqpoint{0.789594in}{3.410944in}}%
\pgfpathlineto{\pgfqpoint{0.794963in}{3.421247in}}%
\pgfpathlineto{\pgfqpoint{0.796037in}{3.418451in}}%
\pgfpathlineto{\pgfqpoint{0.797110in}{3.419628in}}%
\pgfpathlineto{\pgfqpoint{0.800332in}{3.418892in}}%
\pgfpathlineto{\pgfqpoint{0.801406in}{3.416243in}}%
\pgfpathlineto{\pgfqpoint{0.802480in}{3.416832in}}%
\pgfpathlineto{\pgfqpoint{0.803554in}{3.419334in}}%
\pgfpathlineto{\pgfqpoint{0.804627in}{3.418892in}}%
\pgfpathlineto{\pgfqpoint{0.807849in}{3.418745in}}%
\pgfpathlineto{\pgfqpoint{0.808923in}{3.419628in}}%
\pgfpathlineto{\pgfqpoint{0.809997in}{3.418745in}}%
\pgfpathlineto{\pgfqpoint{0.811070in}{3.420806in}}%
\pgfpathlineto{\pgfqpoint{0.812144in}{3.429343in}}%
\pgfpathlineto{\pgfqpoint{0.816440in}{3.428901in}}%
\pgfpathlineto{\pgfqpoint{0.817513in}{3.428165in}}%
\pgfpathlineto{\pgfqpoint{0.818587in}{3.429343in}}%
\pgfpathlineto{\pgfqpoint{0.819661in}{3.432139in}}%
\pgfpathlineto{\pgfqpoint{0.822883in}{3.434494in}}%
\pgfpathlineto{\pgfqpoint{0.823956in}{3.434494in}}%
\pgfpathlineto{\pgfqpoint{0.825030in}{3.431403in}}%
\pgfpathlineto{\pgfqpoint{0.826104in}{3.431992in}}%
\pgfpathlineto{\pgfqpoint{0.827178in}{3.435966in}}%
\pgfpathlineto{\pgfqpoint{0.830399in}{3.432434in}}%
\pgfpathlineto{\pgfqpoint{0.831473in}{3.435083in}}%
\pgfpathlineto{\pgfqpoint{0.832547in}{3.433906in}}%
\pgfpathlineto{\pgfqpoint{0.834695in}{3.434494in}}%
\pgfpathlineto{\pgfqpoint{0.837916in}{3.434053in}}%
\pgfpathlineto{\pgfqpoint{0.838990in}{3.435525in}}%
\pgfpathlineto{\pgfqpoint{0.840064in}{3.444797in}}%
\pgfpathlineto{\pgfqpoint{0.841138in}{3.444945in}}%
\pgfpathlineto{\pgfqpoint{0.842212in}{3.443620in}}%
\pgfpathlineto{\pgfqpoint{0.846507in}{3.448919in}}%
\pgfpathlineto{\pgfqpoint{0.847581in}{3.444945in}}%
\pgfpathlineto{\pgfqpoint{0.848655in}{3.445386in}}%
\pgfpathlineto{\pgfqpoint{0.849729in}{3.447005in}}%
\pgfpathlineto{\pgfqpoint{0.852950in}{3.440235in}}%
\pgfpathlineto{\pgfqpoint{0.855098in}{3.446711in}}%
\pgfpathlineto{\pgfqpoint{0.856172in}{3.444945in}}%
\pgfpathlineto{\pgfqpoint{0.857245in}{3.444650in}}%
\pgfpathlineto{\pgfqpoint{0.860467in}{3.445533in}}%
\pgfpathlineto{\pgfqpoint{0.861541in}{3.449213in}}%
\pgfpathlineto{\pgfqpoint{0.862615in}{3.450243in}}%
\pgfpathlineto{\pgfqpoint{0.863688in}{3.450243in}}%
\pgfpathlineto{\pgfqpoint{0.864762in}{3.451421in}}%
\pgfpathlineto{\pgfqpoint{0.867984in}{3.449655in}}%
\pgfpathlineto{\pgfqpoint{0.869058in}{3.447152in}}%
\pgfpathlineto{\pgfqpoint{0.870131in}{3.448183in}}%
\pgfpathlineto{\pgfqpoint{0.871205in}{3.450391in}}%
\pgfpathlineto{\pgfqpoint{0.872279in}{3.447594in}}%
\pgfpathlineto{\pgfqpoint{0.875501in}{3.445386in}}%
\pgfpathlineto{\pgfqpoint{0.876575in}{3.446122in}}%
\pgfpathlineto{\pgfqpoint{0.877648in}{3.447741in}}%
\pgfpathlineto{\pgfqpoint{0.878722in}{3.445828in}}%
\pgfpathlineto{\pgfqpoint{0.879796in}{3.446711in}}%
\pgfpathlineto{\pgfqpoint{0.883018in}{3.445239in}}%
\pgfpathlineto{\pgfqpoint{0.884091in}{3.443767in}}%
\pgfpathlineto{\pgfqpoint{0.890534in}{3.443326in}}%
\pgfpathlineto{\pgfqpoint{0.891608in}{3.446416in}}%
\pgfpathlineto{\pgfqpoint{0.892682in}{3.442442in}}%
\pgfpathlineto{\pgfqpoint{0.893756in}{3.443326in}}%
\pgfpathlineto{\pgfqpoint{0.894830in}{3.441706in}}%
\pgfpathlineto{\pgfqpoint{0.898051in}{3.443767in}}%
\pgfpathlineto{\pgfqpoint{0.899125in}{3.443178in}}%
\pgfpathlineto{\pgfqpoint{0.900199in}{3.449213in}}%
\pgfpathlineto{\pgfqpoint{0.901273in}{3.449213in}}%
\pgfpathlineto{\pgfqpoint{0.902347in}{3.447741in}}%
\pgfpathlineto{\pgfqpoint{0.905568in}{3.440235in}}%
\pgfpathlineto{\pgfqpoint{0.906642in}{3.443620in}}%
\pgfpathlineto{\pgfqpoint{0.907716in}{3.439646in}}%
\pgfpathlineto{\pgfqpoint{0.908790in}{3.438616in}}%
\pgfpathlineto{\pgfqpoint{0.909864in}{3.428018in}}%
\pgfpathlineto{\pgfqpoint{0.913085in}{3.423308in}}%
\pgfpathlineto{\pgfqpoint{0.914159in}{3.425074in}}%
\pgfpathlineto{\pgfqpoint{0.915233in}{3.430373in}}%
\pgfpathlineto{\pgfqpoint{0.916307in}{3.430373in}}%
\pgfpathlineto{\pgfqpoint{0.917380in}{3.433317in}}%
\pgfpathlineto{\pgfqpoint{0.921676in}{3.434200in}}%
\pgfpathlineto{\pgfqpoint{0.922750in}{3.432581in}}%
\pgfpathlineto{\pgfqpoint{0.924897in}{3.437438in}}%
\pgfpathlineto{\pgfqpoint{0.928119in}{3.437585in}}%
\pgfpathlineto{\pgfqpoint{0.929193in}{3.438763in}}%
\pgfpathlineto{\pgfqpoint{0.930266in}{3.442590in}}%
\pgfpathlineto{\pgfqpoint{0.931340in}{3.439940in}}%
\pgfpathlineto{\pgfqpoint{0.932414in}{3.441265in}}%
\pgfpathlineto{\pgfqpoint{0.935636in}{3.440676in}}%
\pgfpathlineto{\pgfqpoint{0.939931in}{3.447741in}}%
\pgfpathlineto{\pgfqpoint{0.944226in}{3.450096in}}%
\pgfpathlineto{\pgfqpoint{0.945300in}{3.452451in}}%
\pgfpathlineto{\pgfqpoint{0.947448in}{3.448624in}}%
\pgfpathlineto{\pgfqpoint{0.953891in}{3.449360in}}%
\pgfpathlineto{\pgfqpoint{0.954965in}{3.446122in}}%
\pgfpathlineto{\pgfqpoint{0.958186in}{3.449802in}}%
\pgfpathlineto{\pgfqpoint{0.959260in}{3.450096in}}%
\pgfpathlineto{\pgfqpoint{0.960334in}{3.446122in}}%
\pgfpathlineto{\pgfqpoint{0.961408in}{3.446858in}}%
\pgfpathlineto{\pgfqpoint{0.962482in}{3.452451in}}%
\pgfpathlineto{\pgfqpoint{0.965703in}{3.451421in}}%
\pgfpathlineto{\pgfqpoint{0.967851in}{3.448183in}}%
\pgfpathlineto{\pgfqpoint{0.968925in}{3.450243in}}%
\pgfpathlineto{\pgfqpoint{0.969998in}{3.448477in}}%
\pgfpathlineto{\pgfqpoint{0.973220in}{3.451421in}}%
\pgfpathlineto{\pgfqpoint{0.974294in}{3.457897in}}%
\pgfpathlineto{\pgfqpoint{0.975368in}{3.453776in}}%
\pgfpathlineto{\pgfqpoint{0.976442in}{3.447447in}}%
\pgfpathlineto{\pgfqpoint{0.977515in}{3.448771in}}%
\pgfpathlineto{\pgfqpoint{0.980737in}{3.443767in}}%
\pgfpathlineto{\pgfqpoint{0.982885in}{3.447447in}}%
\pgfpathlineto{\pgfqpoint{0.983958in}{3.448330in}}%
\pgfpathlineto{\pgfqpoint{0.985032in}{3.446858in}}%
\pgfpathlineto{\pgfqpoint{0.988254in}{3.448477in}}%
\pgfpathlineto{\pgfqpoint{0.989328in}{3.443473in}}%
\pgfpathlineto{\pgfqpoint{0.990401in}{3.443473in}}%
\pgfpathlineto{\pgfqpoint{0.992549in}{3.447300in}}%
\pgfpathlineto{\pgfqpoint{0.995771in}{3.448330in}}%
\pgfpathlineto{\pgfqpoint{0.996844in}{3.451862in}}%
\pgfpathlineto{\pgfqpoint{0.997918in}{3.450832in}}%
\pgfpathlineto{\pgfqpoint{0.998992in}{3.455542in}}%
\pgfpathlineto{\pgfqpoint{1.000066in}{3.453482in}}%
\pgfpathlineto{\pgfqpoint{1.003287in}{3.451862in}}%
\pgfpathlineto{\pgfqpoint{1.004361in}{3.449655in}}%
\pgfpathlineto{\pgfqpoint{1.006509in}{3.451862in}}%
\pgfpathlineto{\pgfqpoint{1.007583in}{3.464668in}}%
\pgfpathlineto{\pgfqpoint{1.010804in}{3.466287in}}%
\pgfpathlineto{\pgfqpoint{1.012952in}{3.463490in}}%
\pgfpathlineto{\pgfqpoint{1.014026in}{3.464226in}}%
\pgfpathlineto{\pgfqpoint{1.019395in}{3.461871in}}%
\pgfpathlineto{\pgfqpoint{1.020469in}{3.460547in}}%
\pgfpathlineto{\pgfqpoint{1.021543in}{3.463490in}}%
\pgfpathlineto{\pgfqpoint{1.022617in}{3.464373in}}%
\pgfpathlineto{\pgfqpoint{1.025838in}{3.462313in}}%
\pgfpathlineto{\pgfqpoint{1.026912in}{3.459811in}}%
\pgfpathlineto{\pgfqpoint{1.029060in}{3.460105in}}%
\pgfpathlineto{\pgfqpoint{1.030133in}{3.459075in}}%
\pgfpathlineto{\pgfqpoint{1.033355in}{3.459369in}}%
\pgfpathlineto{\pgfqpoint{1.034429in}{3.458633in}}%
\pgfpathlineto{\pgfqpoint{1.036576in}{3.456278in}}%
\pgfpathlineto{\pgfqpoint{1.041946in}{3.452893in}}%
\pgfpathlineto{\pgfqpoint{1.043020in}{3.451715in}}%
\pgfpathlineto{\pgfqpoint{1.044093in}{3.453629in}}%
\pgfpathlineto{\pgfqpoint{1.045167in}{3.453629in}}%
\pgfpathlineto{\pgfqpoint{1.048389in}{3.451715in}}%
\pgfpathlineto{\pgfqpoint{1.049463in}{3.446711in}}%
\pgfpathlineto{\pgfqpoint{1.050536in}{3.446858in}}%
\pgfpathlineto{\pgfqpoint{1.051610in}{3.445828in}}%
\pgfpathlineto{\pgfqpoint{1.052684in}{3.446269in}}%
\pgfpathlineto{\pgfqpoint{1.056979in}{3.445239in}}%
\pgfpathlineto{\pgfqpoint{1.058053in}{3.446564in}}%
\pgfpathlineto{\pgfqpoint{1.060201in}{3.446416in}}%
\pgfpathlineto{\pgfqpoint{1.063422in}{3.449066in}}%
\pgfpathlineto{\pgfqpoint{1.065570in}{3.457308in}}%
\pgfpathlineto{\pgfqpoint{1.066644in}{3.454512in}}%
\pgfpathlineto{\pgfqpoint{1.067718in}{3.453629in}}%
\pgfpathlineto{\pgfqpoint{1.070939in}{3.457897in}}%
\pgfpathlineto{\pgfqpoint{1.073087in}{3.466287in}}%
\pgfpathlineto{\pgfqpoint{1.074161in}{3.463785in}}%
\pgfpathlineto{\pgfqpoint{1.075235in}{3.458486in}}%
\pgfpathlineto{\pgfqpoint{1.079530in}{3.462166in}}%
\pgfpathlineto{\pgfqpoint{1.080604in}{3.461135in}}%
\pgfpathlineto{\pgfqpoint{1.081678in}{3.461282in}}%
\pgfpathlineto{\pgfqpoint{1.082752in}{3.458927in}}%
\pgfpathlineto{\pgfqpoint{1.085973in}{3.457161in}}%
\pgfpathlineto{\pgfqpoint{1.088121in}{3.462313in}}%
\pgfpathlineto{\pgfqpoint{1.089195in}{3.459516in}}%
\pgfpathlineto{\pgfqpoint{1.093490in}{3.457750in}}%
\pgfpathlineto{\pgfqpoint{1.094564in}{3.454512in}}%
\pgfpathlineto{\pgfqpoint{1.095638in}{3.453334in}}%
\pgfpathlineto{\pgfqpoint{1.096711in}{3.461282in}}%
\pgfpathlineto{\pgfqpoint{1.097785in}{3.463049in}}%
\pgfpathlineto{\pgfqpoint{1.101007in}{3.462902in}}%
\pgfpathlineto{\pgfqpoint{1.102081in}{3.460547in}}%
\pgfpathlineto{\pgfqpoint{1.103154in}{3.462607in}}%
\pgfpathlineto{\pgfqpoint{1.104228in}{3.466434in}}%
\pgfpathlineto{\pgfqpoint{1.105302in}{3.476590in}}%
\pgfpathlineto{\pgfqpoint{1.108524in}{3.483508in}}%
\pgfpathlineto{\pgfqpoint{1.109597in}{3.482183in}}%
\pgfpathlineto{\pgfqpoint{1.110671in}{3.478356in}}%
\pgfpathlineto{\pgfqpoint{1.111745in}{3.481153in}}%
\pgfpathlineto{\pgfqpoint{1.112819in}{3.480417in}}%
\pgfpathlineto{\pgfqpoint{1.117114in}{3.484391in}}%
\pgfpathlineto{\pgfqpoint{1.118188in}{3.486157in}}%
\pgfpathlineto{\pgfqpoint{1.119262in}{3.483508in}}%
\pgfpathlineto{\pgfqpoint{1.120336in}{3.488218in}}%
\pgfpathlineto{\pgfqpoint{1.124631in}{3.486746in}}%
\pgfpathlineto{\pgfqpoint{1.125705in}{3.492486in}}%
\pgfpathlineto{\pgfqpoint{1.126779in}{3.488954in}}%
\pgfpathlineto{\pgfqpoint{1.127853in}{3.494253in}}%
\pgfpathlineto{\pgfqpoint{1.131074in}{3.493811in}}%
\pgfpathlineto{\pgfqpoint{1.132148in}{3.494253in}}%
\pgfpathlineto{\pgfqpoint{1.133222in}{3.495430in}}%
\pgfpathlineto{\pgfqpoint{1.134296in}{3.493517in}}%
\pgfpathlineto{\pgfqpoint{1.135370in}{3.496019in}}%
\pgfpathlineto{\pgfqpoint{1.138591in}{3.496166in}}%
\pgfpathlineto{\pgfqpoint{1.139665in}{3.493958in}}%
\pgfpathlineto{\pgfqpoint{1.141813in}{3.492634in}}%
\pgfpathlineto{\pgfqpoint{1.142886in}{3.494547in}}%
\pgfpathlineto{\pgfqpoint{1.146108in}{3.490426in}}%
\pgfpathlineto{\pgfqpoint{1.148256in}{3.491603in}}%
\pgfpathlineto{\pgfqpoint{1.150403in}{3.489690in}}%
\pgfpathlineto{\pgfqpoint{1.153625in}{3.489690in}}%
\pgfpathlineto{\pgfqpoint{1.154699in}{3.488512in}}%
\pgfpathlineto{\pgfqpoint{1.155773in}{3.489395in}}%
\pgfpathlineto{\pgfqpoint{1.156846in}{3.487188in}}%
\pgfpathlineto{\pgfqpoint{1.157920in}{3.492928in}}%
\pgfpathlineto{\pgfqpoint{1.161142in}{3.495872in}}%
\pgfpathlineto{\pgfqpoint{1.162216in}{3.495283in}}%
\pgfpathlineto{\pgfqpoint{1.163289in}{3.488659in}}%
\pgfpathlineto{\pgfqpoint{1.164363in}{3.488218in}}%
\pgfpathlineto{\pgfqpoint{1.165437in}{3.491750in}}%
\pgfpathlineto{\pgfqpoint{1.169732in}{3.493958in}}%
\pgfpathlineto{\pgfqpoint{1.170806in}{3.498374in}}%
\pgfpathlineto{\pgfqpoint{1.171880in}{3.499993in}}%
\pgfpathlineto{\pgfqpoint{1.172954in}{3.500435in}}%
\pgfpathlineto{\pgfqpoint{1.176175in}{3.500876in}}%
\pgfpathlineto{\pgfqpoint{1.177249in}{3.503378in}}%
\pgfpathlineto{\pgfqpoint{1.179397in}{3.506028in}}%
\pgfpathlineto{\pgfqpoint{1.180471in}{3.506028in}}%
\pgfpathlineto{\pgfqpoint{1.183692in}{3.506764in}}%
\pgfpathlineto{\pgfqpoint{1.184766in}{3.508383in}}%
\pgfpathlineto{\pgfqpoint{1.186914in}{3.502054in}}%
\pgfpathlineto{\pgfqpoint{1.187988in}{3.501906in}}%
\pgfpathlineto{\pgfqpoint{1.191209in}{3.499257in}}%
\pgfpathlineto{\pgfqpoint{1.192283in}{3.499551in}}%
\pgfpathlineto{\pgfqpoint{1.193357in}{3.498668in}}%
\pgfpathlineto{\pgfqpoint{1.194431in}{3.498815in}}%
\pgfpathlineto{\pgfqpoint{1.195505in}{3.495724in}}%
\pgfpathlineto{\pgfqpoint{1.198726in}{3.492928in}}%
\pgfpathlineto{\pgfqpoint{1.200874in}{3.500287in}}%
\pgfpathlineto{\pgfqpoint{1.201948in}{3.498521in}}%
\pgfpathlineto{\pgfqpoint{1.203021in}{3.491162in}}%
\pgfpathlineto{\pgfqpoint{1.207317in}{3.487776in}}%
\pgfpathlineto{\pgfqpoint{1.209464in}{3.482183in}}%
\pgfpathlineto{\pgfqpoint{1.210538in}{3.471880in}}%
\pgfpathlineto{\pgfqpoint{1.213760in}{3.473352in}}%
\pgfpathlineto{\pgfqpoint{1.214834in}{3.477915in}}%
\pgfpathlineto{\pgfqpoint{1.215908in}{3.476001in}}%
\pgfpathlineto{\pgfqpoint{1.216981in}{3.478356in}}%
\pgfpathlineto{\pgfqpoint{1.218055in}{3.474088in}}%
\pgfpathlineto{\pgfqpoint{1.221277in}{3.464815in}}%
\pgfpathlineto{\pgfqpoint{1.222351in}{3.467464in}}%
\pgfpathlineto{\pgfqpoint{1.223424in}{3.466876in}}%
\pgfpathlineto{\pgfqpoint{1.225572in}{3.474824in}}%
\pgfpathlineto{\pgfqpoint{1.228794in}{3.473058in}}%
\pgfpathlineto{\pgfqpoint{1.229867in}{3.477620in}}%
\pgfpathlineto{\pgfqpoint{1.230941in}{3.477179in}}%
\pgfpathlineto{\pgfqpoint{1.232015in}{3.477768in}}%
\pgfpathlineto{\pgfqpoint{1.233089in}{3.481300in}}%
\pgfpathlineto{\pgfqpoint{1.237384in}{3.480270in}}%
\pgfpathlineto{\pgfqpoint{1.238458in}{3.477179in}}%
\pgfpathlineto{\pgfqpoint{1.239532in}{3.476590in}}%
\pgfpathlineto{\pgfqpoint{1.240606in}{3.474382in}}%
\pgfpathlineto{\pgfqpoint{1.243827in}{3.478503in}}%
\pgfpathlineto{\pgfqpoint{1.245975in}{3.478651in}}%
\pgfpathlineto{\pgfqpoint{1.247049in}{3.481006in}}%
\pgfpathlineto{\pgfqpoint{1.248123in}{3.480711in}}%
\pgfpathlineto{\pgfqpoint{1.251344in}{3.476590in}}%
\pgfpathlineto{\pgfqpoint{1.253492in}{3.486157in}}%
\pgfpathlineto{\pgfqpoint{1.254566in}{3.489690in}}%
\pgfpathlineto{\pgfqpoint{1.255640in}{3.488659in}}%
\pgfpathlineto{\pgfqpoint{1.258861in}{3.487482in}}%
\pgfpathlineto{\pgfqpoint{1.261009in}{3.484244in}}%
\pgfpathlineto{\pgfqpoint{1.263156in}{3.476443in}}%
\pgfpathlineto{\pgfqpoint{1.266378in}{3.480270in}}%
\pgfpathlineto{\pgfqpoint{1.267452in}{3.482919in}}%
\pgfpathlineto{\pgfqpoint{1.268526in}{3.478503in}}%
\pgfpathlineto{\pgfqpoint{1.269599in}{3.478356in}}%
\pgfpathlineto{\pgfqpoint{1.270673in}{3.479828in}}%
\pgfpathlineto{\pgfqpoint{1.273895in}{3.479975in}}%
\pgfpathlineto{\pgfqpoint{1.274969in}{3.483508in}}%
\pgfpathlineto{\pgfqpoint{1.276042in}{3.482478in}}%
\pgfpathlineto{\pgfqpoint{1.277116in}{3.484833in}}%
\pgfpathlineto{\pgfqpoint{1.278190in}{3.485569in}}%
\pgfpathlineto{\pgfqpoint{1.282485in}{3.485421in}}%
\pgfpathlineto{\pgfqpoint{1.284633in}{3.489837in}}%
\pgfpathlineto{\pgfqpoint{1.285707in}{3.487335in}}%
\pgfpathlineto{\pgfqpoint{1.290002in}{3.484097in}}%
\pgfpathlineto{\pgfqpoint{1.291076in}{3.486452in}}%
\pgfpathlineto{\pgfqpoint{1.292150in}{3.482036in}}%
\pgfpathlineto{\pgfqpoint{1.293224in}{3.480270in}}%
\pgfpathlineto{\pgfqpoint{1.297519in}{3.484833in}}%
\pgfpathlineto{\pgfqpoint{1.299667in}{3.493664in}}%
\pgfpathlineto{\pgfqpoint{1.305036in}{3.493958in}}%
\pgfpathlineto{\pgfqpoint{1.306110in}{3.492045in}}%
\pgfpathlineto{\pgfqpoint{1.307184in}{3.492486in}}%
\pgfpathlineto{\pgfqpoint{1.308258in}{3.494253in}}%
\pgfpathlineto{\pgfqpoint{1.311479in}{3.496313in}}%
\pgfpathlineto{\pgfqpoint{1.312553in}{3.496166in}}%
\pgfpathlineto{\pgfqpoint{1.313627in}{3.497638in}}%
\pgfpathlineto{\pgfqpoint{1.315774in}{3.495136in}}%
\pgfpathlineto{\pgfqpoint{1.318996in}{3.493958in}}%
\pgfpathlineto{\pgfqpoint{1.320070in}{3.489248in}}%
\pgfpathlineto{\pgfqpoint{1.321144in}{3.493369in}}%
\pgfpathlineto{\pgfqpoint{1.323291in}{3.492045in}}%
\pgfpathlineto{\pgfqpoint{1.327587in}{3.498080in}}%
\pgfpathlineto{\pgfqpoint{1.329734in}{3.494253in}}%
\pgfpathlineto{\pgfqpoint{1.330808in}{3.494989in}}%
\pgfpathlineto{\pgfqpoint{1.334030in}{3.494253in}}%
\pgfpathlineto{\pgfqpoint{1.335104in}{3.490573in}}%
\pgfpathlineto{\pgfqpoint{1.336177in}{3.492928in}}%
\pgfpathlineto{\pgfqpoint{1.342620in}{3.493811in}}%
\pgfpathlineto{\pgfqpoint{1.344768in}{3.495872in}}%
\pgfpathlineto{\pgfqpoint{1.345842in}{3.496460in}}%
\pgfpathlineto{\pgfqpoint{1.349063in}{3.496902in}}%
\pgfpathlineto{\pgfqpoint{1.350137in}{3.496460in}}%
\pgfpathlineto{\pgfqpoint{1.351211in}{3.493664in}}%
\pgfpathlineto{\pgfqpoint{1.352285in}{3.496166in}}%
\pgfpathlineto{\pgfqpoint{1.353359in}{3.501170in}}%
\pgfpathlineto{\pgfqpoint{1.356580in}{3.504261in}}%
\pgfpathlineto{\pgfqpoint{1.357654in}{3.503820in}}%
\pgfpathlineto{\pgfqpoint{1.359802in}{3.498521in}}%
\pgfpathlineto{\pgfqpoint{1.360876in}{3.499404in}}%
\pgfpathlineto{\pgfqpoint{1.364097in}{3.496902in}}%
\pgfpathlineto{\pgfqpoint{1.366245in}{3.497638in}}%
\pgfpathlineto{\pgfqpoint{1.367319in}{3.500729in}}%
\pgfpathlineto{\pgfqpoint{1.368393in}{3.501318in}}%
\pgfpathlineto{\pgfqpoint{1.373762in}{3.494694in}}%
\pgfpathlineto{\pgfqpoint{1.374836in}{3.493075in}}%
\pgfpathlineto{\pgfqpoint{1.375909in}{3.494841in}}%
\pgfpathlineto{\pgfqpoint{1.379131in}{3.492928in}}%
\pgfpathlineto{\pgfqpoint{1.380205in}{3.494400in}}%
\pgfpathlineto{\pgfqpoint{1.382352in}{3.499993in}}%
\pgfpathlineto{\pgfqpoint{1.386648in}{3.498668in}}%
\pgfpathlineto{\pgfqpoint{1.387722in}{3.494105in}}%
\pgfpathlineto{\pgfqpoint{1.388796in}{3.493517in}}%
\pgfpathlineto{\pgfqpoint{1.389869in}{3.492045in}}%
\pgfpathlineto{\pgfqpoint{1.390943in}{3.496166in}}%
\pgfpathlineto{\pgfqpoint{1.394165in}{3.497491in}}%
\pgfpathlineto{\pgfqpoint{1.395239in}{3.496902in}}%
\pgfpathlineto{\pgfqpoint{1.396312in}{3.501906in}}%
\pgfpathlineto{\pgfqpoint{1.397386in}{3.496902in}}%
\pgfpathlineto{\pgfqpoint{1.401682in}{3.489395in}}%
\pgfpathlineto{\pgfqpoint{1.402755in}{3.489837in}}%
\pgfpathlineto{\pgfqpoint{1.403829in}{3.488512in}}%
\pgfpathlineto{\pgfqpoint{1.404903in}{3.488807in}}%
\pgfpathlineto{\pgfqpoint{1.405977in}{3.487040in}}%
\pgfpathlineto{\pgfqpoint{1.409198in}{3.484685in}}%
\pgfpathlineto{\pgfqpoint{1.410272in}{3.482919in}}%
\pgfpathlineto{\pgfqpoint{1.411346in}{3.485274in}}%
\pgfpathlineto{\pgfqpoint{1.412420in}{3.479387in}}%
\pgfpathlineto{\pgfqpoint{1.413494in}{3.481742in}}%
\pgfpathlineto{\pgfqpoint{1.416715in}{3.480858in}}%
\pgfpathlineto{\pgfqpoint{1.417789in}{3.477768in}}%
\pgfpathlineto{\pgfqpoint{1.418863in}{3.482919in}}%
\pgfpathlineto{\pgfqpoint{1.419937in}{3.483508in}}%
\pgfpathlineto{\pgfqpoint{1.421011in}{3.485569in}}%
\pgfpathlineto{\pgfqpoint{1.424232in}{3.487040in}}%
\pgfpathlineto{\pgfqpoint{1.425306in}{3.484833in}}%
\pgfpathlineto{\pgfqpoint{1.426380in}{3.487482in}}%
\pgfpathlineto{\pgfqpoint{1.427454in}{3.488218in}}%
\pgfpathlineto{\pgfqpoint{1.428528in}{3.485274in}}%
\pgfpathlineto{\pgfqpoint{1.431749in}{3.490426in}}%
\pgfpathlineto{\pgfqpoint{1.432823in}{3.490131in}}%
\pgfpathlineto{\pgfqpoint{1.433897in}{3.493958in}}%
\pgfpathlineto{\pgfqpoint{1.434971in}{3.494841in}}%
\pgfpathlineto{\pgfqpoint{1.436044in}{3.491456in}}%
\pgfpathlineto{\pgfqpoint{1.439266in}{3.492045in}}%
\pgfpathlineto{\pgfqpoint{1.440340in}{3.489690in}}%
\pgfpathlineto{\pgfqpoint{1.441414in}{3.491162in}}%
\pgfpathlineto{\pgfqpoint{1.442487in}{3.489690in}}%
\pgfpathlineto{\pgfqpoint{1.447857in}{3.487776in}}%
\pgfpathlineto{\pgfqpoint{1.448930in}{3.488954in}}%
\pgfpathlineto{\pgfqpoint{1.450004in}{3.489101in}}%
\pgfpathlineto{\pgfqpoint{1.451078in}{3.490867in}}%
\pgfpathlineto{\pgfqpoint{1.454300in}{3.490573in}}%
\pgfpathlineto{\pgfqpoint{1.455373in}{3.488365in}}%
\pgfpathlineto{\pgfqpoint{1.457521in}{3.489837in}}%
\pgfpathlineto{\pgfqpoint{1.458595in}{3.488071in}}%
\pgfpathlineto{\pgfqpoint{1.461817in}{3.488659in}}%
\pgfpathlineto{\pgfqpoint{1.462890in}{3.492192in}}%
\pgfpathlineto{\pgfqpoint{1.463964in}{3.492781in}}%
\pgfpathlineto{\pgfqpoint{1.466112in}{3.495724in}}%
\pgfpathlineto{\pgfqpoint{1.471481in}{3.491456in}}%
\pgfpathlineto{\pgfqpoint{1.472555in}{3.486746in}}%
\pgfpathlineto{\pgfqpoint{1.473629in}{3.487776in}}%
\pgfpathlineto{\pgfqpoint{1.476850in}{3.485127in}}%
\pgfpathlineto{\pgfqpoint{1.477924in}{3.487629in}}%
\pgfpathlineto{\pgfqpoint{1.478998in}{3.482036in}}%
\pgfpathlineto{\pgfqpoint{1.480072in}{3.481594in}}%
\pgfpathlineto{\pgfqpoint{1.481146in}{3.484980in}}%
\pgfpathlineto{\pgfqpoint{1.484367in}{3.482772in}}%
\pgfpathlineto{\pgfqpoint{1.485441in}{3.477768in}}%
\pgfpathlineto{\pgfqpoint{1.486515in}{3.483066in}}%
\pgfpathlineto{\pgfqpoint{1.488662in}{3.471291in}}%
\pgfpathlineto{\pgfqpoint{1.491884in}{3.467317in}}%
\pgfpathlineto{\pgfqpoint{1.494032in}{3.471291in}}%
\pgfpathlineto{\pgfqpoint{1.495106in}{3.470997in}}%
\pgfpathlineto{\pgfqpoint{1.496179in}{3.477915in}}%
\pgfpathlineto{\pgfqpoint{1.499401in}{3.480417in}}%
\pgfpathlineto{\pgfqpoint{1.500475in}{3.485569in}}%
\pgfpathlineto{\pgfqpoint{1.501549in}{3.482478in}}%
\pgfpathlineto{\pgfqpoint{1.503696in}{3.487924in}}%
\pgfpathlineto{\pgfqpoint{1.506918in}{3.486452in}}%
\pgfpathlineto{\pgfqpoint{1.507992in}{3.490720in}}%
\pgfpathlineto{\pgfqpoint{1.509065in}{3.488071in}}%
\pgfpathlineto{\pgfqpoint{1.510139in}{3.488218in}}%
\pgfpathlineto{\pgfqpoint{1.511213in}{3.489984in}}%
\pgfpathlineto{\pgfqpoint{1.515508in}{3.488659in}}%
\pgfpathlineto{\pgfqpoint{1.516582in}{3.490131in}}%
\pgfpathlineto{\pgfqpoint{1.517656in}{3.496608in}}%
\pgfpathlineto{\pgfqpoint{1.518730in}{3.497196in}}%
\pgfpathlineto{\pgfqpoint{1.521951in}{3.497932in}}%
\pgfpathlineto{\pgfqpoint{1.523025in}{3.496902in}}%
\pgfpathlineto{\pgfqpoint{1.524099in}{3.498521in}}%
\pgfpathlineto{\pgfqpoint{1.525173in}{3.497344in}}%
\pgfpathlineto{\pgfqpoint{1.529468in}{3.499699in}}%
\pgfpathlineto{\pgfqpoint{1.530542in}{3.504556in}}%
\pgfpathlineto{\pgfqpoint{1.532690in}{3.502495in}}%
\pgfpathlineto{\pgfqpoint{1.533764in}{3.504261in}}%
\pgfpathlineto{\pgfqpoint{1.536985in}{3.504409in}}%
\pgfpathlineto{\pgfqpoint{1.538059in}{3.502642in}}%
\pgfpathlineto{\pgfqpoint{1.539133in}{3.502790in}}%
\pgfpathlineto{\pgfqpoint{1.544502in}{3.492486in}}%
\pgfpathlineto{\pgfqpoint{1.545576in}{3.492781in}}%
\pgfpathlineto{\pgfqpoint{1.546650in}{3.496902in}}%
\pgfpathlineto{\pgfqpoint{1.547724in}{3.493369in}}%
\pgfpathlineto{\pgfqpoint{1.552019in}{3.488512in}}%
\pgfpathlineto{\pgfqpoint{1.553093in}{3.487188in}}%
\pgfpathlineto{\pgfqpoint{1.554167in}{3.483361in}}%
\pgfpathlineto{\pgfqpoint{1.555240in}{3.485127in}}%
\pgfpathlineto{\pgfqpoint{1.556314in}{3.478798in}}%
\pgfpathlineto{\pgfqpoint{1.560610in}{3.473941in}}%
\pgfpathlineto{\pgfqpoint{1.561684in}{3.476001in}}%
\pgfpathlineto{\pgfqpoint{1.563831in}{3.490573in}}%
\pgfpathlineto{\pgfqpoint{1.567053in}{3.491603in}}%
\pgfpathlineto{\pgfqpoint{1.568127in}{3.493811in}}%
\pgfpathlineto{\pgfqpoint{1.569200in}{3.493075in}}%
\pgfpathlineto{\pgfqpoint{1.574570in}{3.491603in}}%
\pgfpathlineto{\pgfqpoint{1.575643in}{3.489984in}}%
\pgfpathlineto{\pgfqpoint{1.576717in}{3.486304in}}%
\pgfpathlineto{\pgfqpoint{1.578865in}{3.483655in}}%
\pgfpathlineto{\pgfqpoint{1.582086in}{3.478062in}}%
\pgfpathlineto{\pgfqpoint{1.583160in}{3.471586in}}%
\pgfpathlineto{\pgfqpoint{1.584234in}{3.471733in}}%
\pgfpathlineto{\pgfqpoint{1.585308in}{3.475265in}}%
\pgfpathlineto{\pgfqpoint{1.586382in}{3.470997in}}%
\pgfpathlineto{\pgfqpoint{1.589603in}{3.470408in}}%
\pgfpathlineto{\pgfqpoint{1.592825in}{3.465551in}}%
\pgfpathlineto{\pgfqpoint{1.593899in}{3.465698in}}%
\pgfpathlineto{\pgfqpoint{1.598194in}{3.468789in}}%
\pgfpathlineto{\pgfqpoint{1.601416in}{3.476590in}}%
\pgfpathlineto{\pgfqpoint{1.604637in}{3.477915in}}%
\pgfpathlineto{\pgfqpoint{1.605711in}{3.475413in}}%
\pgfpathlineto{\pgfqpoint{1.606785in}{3.468789in}}%
\pgfpathlineto{\pgfqpoint{1.607859in}{3.471733in}}%
\pgfpathlineto{\pgfqpoint{1.608932in}{3.469378in}}%
\pgfpathlineto{\pgfqpoint{1.612154in}{3.473205in}}%
\pgfpathlineto{\pgfqpoint{1.613228in}{3.476443in}}%
\pgfpathlineto{\pgfqpoint{1.614302in}{3.472616in}}%
\pgfpathlineto{\pgfqpoint{1.615375in}{3.476884in}}%
\pgfpathlineto{\pgfqpoint{1.616449in}{3.477032in}}%
\pgfpathlineto{\pgfqpoint{1.620745in}{3.479534in}}%
\pgfpathlineto{\pgfqpoint{1.621818in}{3.480123in}}%
\pgfpathlineto{\pgfqpoint{1.622892in}{3.481594in}}%
\pgfpathlineto{\pgfqpoint{1.623966in}{3.484833in}}%
\pgfpathlineto{\pgfqpoint{1.628261in}{3.484980in}}%
\pgfpathlineto{\pgfqpoint{1.629335in}{3.486010in}}%
\pgfpathlineto{\pgfqpoint{1.630409in}{3.485863in}}%
\pgfpathlineto{\pgfqpoint{1.631483in}{3.488365in}}%
\pgfpathlineto{\pgfqpoint{1.634705in}{3.487924in}}%
\pgfpathlineto{\pgfqpoint{1.635778in}{3.490573in}}%
\pgfpathlineto{\pgfqpoint{1.636852in}{3.497049in}}%
\pgfpathlineto{\pgfqpoint{1.637926in}{3.496755in}}%
\pgfpathlineto{\pgfqpoint{1.639000in}{3.498080in}}%
\pgfpathlineto{\pgfqpoint{1.642221in}{3.499551in}}%
\pgfpathlineto{\pgfqpoint{1.644369in}{3.493958in}}%
\pgfpathlineto{\pgfqpoint{1.645443in}{3.495872in}}%
\pgfpathlineto{\pgfqpoint{1.646517in}{3.491014in}}%
\pgfpathlineto{\pgfqpoint{1.649738in}{3.493664in}}%
\pgfpathlineto{\pgfqpoint{1.650812in}{3.487924in}}%
\pgfpathlineto{\pgfqpoint{1.651886in}{3.488071in}}%
\pgfpathlineto{\pgfqpoint{1.652960in}{3.490720in}}%
\pgfpathlineto{\pgfqpoint{1.654034in}{3.486304in}}%
\pgfpathlineto{\pgfqpoint{1.657255in}{3.491309in}}%
\pgfpathlineto{\pgfqpoint{1.658329in}{3.489543in}}%
\pgfpathlineto{\pgfqpoint{1.659403in}{3.493664in}}%
\pgfpathlineto{\pgfqpoint{1.660477in}{3.489837in}}%
\pgfpathlineto{\pgfqpoint{1.661550in}{3.490720in}}%
\pgfpathlineto{\pgfqpoint{1.664772in}{3.491603in}}%
\pgfpathlineto{\pgfqpoint{1.665846in}{3.489101in}}%
\pgfpathlineto{\pgfqpoint{1.666920in}{3.484685in}}%
\pgfpathlineto{\pgfqpoint{1.667994in}{3.483361in}}%
\pgfpathlineto{\pgfqpoint{1.669067in}{3.484097in}}%
\pgfpathlineto{\pgfqpoint{1.672289in}{3.487188in}}%
\pgfpathlineto{\pgfqpoint{1.673363in}{3.483361in}}%
\pgfpathlineto{\pgfqpoint{1.674437in}{3.483802in}}%
\pgfpathlineto{\pgfqpoint{1.675510in}{3.484980in}}%
\pgfpathlineto{\pgfqpoint{1.679806in}{3.488071in}}%
\pgfpathlineto{\pgfqpoint{1.680880in}{3.486010in}}%
\pgfpathlineto{\pgfqpoint{1.681953in}{3.485863in}}%
\pgfpathlineto{\pgfqpoint{1.683027in}{3.494841in}}%
\pgfpathlineto{\pgfqpoint{1.684101in}{3.529136in}}%
\pgfpathlineto{\pgfqpoint{1.687323in}{3.518244in}}%
\pgfpathlineto{\pgfqpoint{1.688396in}{3.519569in}}%
\pgfpathlineto{\pgfqpoint{1.690544in}{3.513976in}}%
\pgfpathlineto{\pgfqpoint{1.691618in}{3.513681in}}%
\pgfpathlineto{\pgfqpoint{1.694839in}{3.510738in}}%
\pgfpathlineto{\pgfqpoint{1.695913in}{3.505880in}}%
\pgfpathlineto{\pgfqpoint{1.696987in}{3.509413in}}%
\pgfpathlineto{\pgfqpoint{1.699135in}{3.508088in}}%
\pgfpathlineto{\pgfqpoint{1.702356in}{3.508971in}}%
\pgfpathlineto{\pgfqpoint{1.703430in}{3.512062in}}%
\pgfpathlineto{\pgfqpoint{1.705578in}{3.511474in}}%
\pgfpathlineto{\pgfqpoint{1.706652in}{3.514417in}}%
\pgfpathlineto{\pgfqpoint{1.709873in}{3.513829in}}%
\pgfpathlineto{\pgfqpoint{1.710947in}{3.509560in}}%
\pgfpathlineto{\pgfqpoint{1.712021in}{3.508235in}}%
\pgfpathlineto{\pgfqpoint{1.714169in}{3.515006in}}%
\pgfpathlineto{\pgfqpoint{1.717390in}{3.509560in}}%
\pgfpathlineto{\pgfqpoint{1.718464in}{3.510885in}}%
\pgfpathlineto{\pgfqpoint{1.720612in}{3.515595in}}%
\pgfpathlineto{\pgfqpoint{1.721685in}{3.513829in}}%
\pgfpathlineto{\pgfqpoint{1.725981in}{3.514859in}}%
\pgfpathlineto{\pgfqpoint{1.727055in}{3.518391in}}%
\pgfpathlineto{\pgfqpoint{1.728128in}{3.519422in}}%
\pgfpathlineto{\pgfqpoint{1.734572in}{3.516920in}}%
\pgfpathlineto{\pgfqpoint{1.735645in}{3.518244in}}%
\pgfpathlineto{\pgfqpoint{1.736719in}{3.513829in}}%
\pgfpathlineto{\pgfqpoint{1.741015in}{3.514565in}}%
\pgfpathlineto{\pgfqpoint{1.742088in}{3.517067in}}%
\pgfpathlineto{\pgfqpoint{1.743162in}{3.513681in}}%
\pgfpathlineto{\pgfqpoint{1.744236in}{3.514123in}}%
\pgfpathlineto{\pgfqpoint{1.747458in}{3.513534in}}%
\pgfpathlineto{\pgfqpoint{1.748531in}{3.514565in}}%
\pgfpathlineto{\pgfqpoint{1.749605in}{3.518244in}}%
\pgfpathlineto{\pgfqpoint{1.751753in}{3.515301in}}%
\pgfpathlineto{\pgfqpoint{1.754974in}{3.513093in}}%
\pgfpathlineto{\pgfqpoint{1.757122in}{3.513829in}}%
\pgfpathlineto{\pgfqpoint{1.758196in}{3.517950in}}%
\pgfpathlineto{\pgfqpoint{1.759270in}{3.516331in}}%
\pgfpathlineto{\pgfqpoint{1.762491in}{3.518539in}}%
\pgfpathlineto{\pgfqpoint{1.763565in}{3.520158in}}%
\pgfpathlineto{\pgfqpoint{1.765713in}{3.513829in}}%
\pgfpathlineto{\pgfqpoint{1.766787in}{3.514417in}}%
\pgfpathlineto{\pgfqpoint{1.771082in}{3.507941in}}%
\pgfpathlineto{\pgfqpoint{1.773230in}{3.510590in}}%
\pgfpathlineto{\pgfqpoint{1.777525in}{3.504703in}}%
\pgfpathlineto{\pgfqpoint{1.778599in}{3.506764in}}%
\pgfpathlineto{\pgfqpoint{1.779673in}{3.499551in}}%
\pgfpathlineto{\pgfqpoint{1.780747in}{3.501170in}}%
\pgfpathlineto{\pgfqpoint{1.781820in}{3.504261in}}%
\pgfpathlineto{\pgfqpoint{1.785042in}{3.506764in}}%
\pgfpathlineto{\pgfqpoint{1.789337in}{3.516331in}}%
\pgfpathlineto{\pgfqpoint{1.792559in}{3.515153in}}%
\pgfpathlineto{\pgfqpoint{1.795780in}{3.504114in}}%
\pgfpathlineto{\pgfqpoint{1.796854in}{3.497785in}}%
\pgfpathlineto{\pgfqpoint{1.800076in}{3.500287in}}%
\pgfpathlineto{\pgfqpoint{1.802223in}{3.504114in}}%
\pgfpathlineto{\pgfqpoint{1.803297in}{3.502348in}}%
\pgfpathlineto{\pgfqpoint{1.804371in}{3.502201in}}%
\pgfpathlineto{\pgfqpoint{1.807593in}{3.499257in}}%
\pgfpathlineto{\pgfqpoint{1.808666in}{3.499699in}}%
\pgfpathlineto{\pgfqpoint{1.809740in}{3.502201in}}%
\pgfpathlineto{\pgfqpoint{1.810814in}{3.501318in}}%
\pgfpathlineto{\pgfqpoint{1.811888in}{3.498227in}}%
\pgfpathlineto{\pgfqpoint{1.815109in}{3.503820in}}%
\pgfpathlineto{\pgfqpoint{1.816183in}{3.497196in}}%
\pgfpathlineto{\pgfqpoint{1.817257in}{3.499110in}}%
\pgfpathlineto{\pgfqpoint{1.818331in}{3.498227in}}%
\pgfpathlineto{\pgfqpoint{1.819405in}{3.501906in}}%
\pgfpathlineto{\pgfqpoint{1.822626in}{3.503525in}}%
\pgfpathlineto{\pgfqpoint{1.823700in}{3.501759in}}%
\pgfpathlineto{\pgfqpoint{1.824774in}{3.497491in}}%
\pgfpathlineto{\pgfqpoint{1.826922in}{3.483361in}}%
\pgfpathlineto{\pgfqpoint{1.830143in}{3.474235in}}%
\pgfpathlineto{\pgfqpoint{1.831217in}{3.466876in}}%
\pgfpathlineto{\pgfqpoint{1.834438in}{3.490426in}}%
\pgfpathlineto{\pgfqpoint{1.837660in}{3.486157in}}%
\pgfpathlineto{\pgfqpoint{1.838734in}{3.474382in}}%
\pgfpathlineto{\pgfqpoint{1.839808in}{3.483066in}}%
\pgfpathlineto{\pgfqpoint{1.840882in}{3.482330in}}%
\pgfpathlineto{\pgfqpoint{1.841955in}{3.476001in}}%
\pgfpathlineto{\pgfqpoint{1.846251in}{3.487924in}}%
\pgfpathlineto{\pgfqpoint{1.847325in}{3.482772in}}%
\pgfpathlineto{\pgfqpoint{1.848398in}{3.484391in}}%
\pgfpathlineto{\pgfqpoint{1.849472in}{3.487776in}}%
\pgfpathlineto{\pgfqpoint{1.852694in}{3.485569in}}%
\pgfpathlineto{\pgfqpoint{1.854841in}{3.499993in}}%
\pgfpathlineto{\pgfqpoint{1.855915in}{3.495577in}}%
\pgfpathlineto{\pgfqpoint{1.856989in}{3.488659in}}%
\pgfpathlineto{\pgfqpoint{1.860211in}{3.492339in}}%
\pgfpathlineto{\pgfqpoint{1.862358in}{3.492928in}}%
\pgfpathlineto{\pgfqpoint{1.863432in}{3.490131in}}%
\pgfpathlineto{\pgfqpoint{1.864506in}{3.490131in}}%
\pgfpathlineto{\pgfqpoint{1.867727in}{3.482478in}}%
\pgfpathlineto{\pgfqpoint{1.868801in}{3.485716in}}%
\pgfpathlineto{\pgfqpoint{1.869875in}{3.493958in}}%
\pgfpathlineto{\pgfqpoint{1.870949in}{3.493517in}}%
\pgfpathlineto{\pgfqpoint{1.872023in}{3.497049in}}%
\pgfpathlineto{\pgfqpoint{1.878466in}{3.529283in}}%
\pgfpathlineto{\pgfqpoint{1.879540in}{3.529872in}}%
\pgfpathlineto{\pgfqpoint{1.882761in}{3.530167in}}%
\pgfpathlineto{\pgfqpoint{1.884909in}{3.523985in}}%
\pgfpathlineto{\pgfqpoint{1.885983in}{3.529283in}}%
\pgfpathlineto{\pgfqpoint{1.887057in}{3.541353in}}%
\pgfpathlineto{\pgfqpoint{1.890278in}{3.541500in}}%
\pgfpathlineto{\pgfqpoint{1.891352in}{3.538851in}}%
\pgfpathlineto{\pgfqpoint{1.892426in}{3.539734in}}%
\pgfpathlineto{\pgfqpoint{1.893500in}{3.548859in}}%
\pgfpathlineto{\pgfqpoint{1.894573in}{3.547976in}}%
\pgfpathlineto{\pgfqpoint{1.897795in}{3.548418in}}%
\pgfpathlineto{\pgfqpoint{1.901016in}{3.545768in}}%
\pgfpathlineto{\pgfqpoint{1.902090in}{3.540617in}}%
\pgfpathlineto{\pgfqpoint{1.906386in}{3.549007in}}%
\pgfpathlineto{\pgfqpoint{1.907460in}{3.548418in}}%
\pgfpathlineto{\pgfqpoint{1.908533in}{3.549595in}}%
\pgfpathlineto{\pgfqpoint{1.909607in}{3.553128in}}%
\pgfpathlineto{\pgfqpoint{1.912829in}{3.551067in}}%
\pgfpathlineto{\pgfqpoint{1.914976in}{3.562548in}}%
\pgfpathlineto{\pgfqpoint{1.916050in}{3.556219in}}%
\pgfpathlineto{\pgfqpoint{1.917124in}{3.557691in}}%
\pgfpathlineto{\pgfqpoint{1.920346in}{3.558721in}}%
\pgfpathlineto{\pgfqpoint{1.921419in}{3.558132in}}%
\pgfpathlineto{\pgfqpoint{1.922493in}{3.560634in}}%
\pgfpathlineto{\pgfqpoint{1.923567in}{3.557543in}}%
\pgfpathlineto{\pgfqpoint{1.924641in}{3.562401in}}%
\pgfpathlineto{\pgfqpoint{1.927862in}{3.561518in}}%
\pgfpathlineto{\pgfqpoint{1.928936in}{3.562401in}}%
\pgfpathlineto{\pgfqpoint{1.930010in}{3.558721in}}%
\pgfpathlineto{\pgfqpoint{1.932158in}{3.558721in}}%
\pgfpathlineto{\pgfqpoint{1.935379in}{3.553422in}}%
\pgfpathlineto{\pgfqpoint{1.936453in}{3.556219in}}%
\pgfpathlineto{\pgfqpoint{1.937527in}{3.553717in}}%
\pgfpathlineto{\pgfqpoint{1.938601in}{3.554453in}}%
\pgfpathlineto{\pgfqpoint{1.939675in}{3.560340in}}%
\pgfpathlineto{\pgfqpoint{1.942896in}{3.558868in}}%
\pgfpathlineto{\pgfqpoint{1.943970in}{3.556513in}}%
\pgfpathlineto{\pgfqpoint{1.946118in}{3.562254in}}%
\pgfpathlineto{\pgfqpoint{1.947192in}{3.557396in}}%
\pgfpathlineto{\pgfqpoint{1.950413in}{3.557396in}}%
\pgfpathlineto{\pgfqpoint{1.951487in}{3.558132in}}%
\pgfpathlineto{\pgfqpoint{1.952561in}{3.566522in}}%
\pgfpathlineto{\pgfqpoint{1.954708in}{3.560487in}}%
\pgfpathlineto{\pgfqpoint{1.959004in}{3.563137in}}%
\pgfpathlineto{\pgfqpoint{1.960078in}{3.569024in}}%
\pgfpathlineto{\pgfqpoint{1.961151in}{3.567552in}}%
\pgfpathlineto{\pgfqpoint{1.965447in}{3.568435in}}%
\pgfpathlineto{\pgfqpoint{1.966521in}{3.573145in}}%
\pgfpathlineto{\pgfqpoint{1.967594in}{3.570349in}}%
\pgfpathlineto{\pgfqpoint{1.968668in}{3.571526in}}%
\pgfpathlineto{\pgfqpoint{1.972964in}{3.565933in}}%
\pgfpathlineto{\pgfqpoint{1.974038in}{3.566375in}}%
\pgfpathlineto{\pgfqpoint{1.975111in}{3.560193in}}%
\pgfpathlineto{\pgfqpoint{1.976185in}{3.543855in}}%
\pgfpathlineto{\pgfqpoint{1.977259in}{3.537379in}}%
\pgfpathlineto{\pgfqpoint{1.981554in}{3.539734in}}%
\pgfpathlineto{\pgfqpoint{1.982628in}{3.534582in}}%
\pgfpathlineto{\pgfqpoint{1.983702in}{3.545033in}}%
\pgfpathlineto{\pgfqpoint{1.984776in}{3.537820in}}%
\pgfpathlineto{\pgfqpoint{1.989071in}{3.537820in}}%
\pgfpathlineto{\pgfqpoint{1.990145in}{3.531638in}}%
\pgfpathlineto{\pgfqpoint{1.991219in}{3.539145in}}%
\pgfpathlineto{\pgfqpoint{1.992293in}{3.534582in}}%
\pgfpathlineto{\pgfqpoint{1.995514in}{3.532080in}}%
\pgfpathlineto{\pgfqpoint{1.996588in}{3.535465in}}%
\pgfpathlineto{\pgfqpoint{1.997662in}{3.531638in}}%
\pgfpathlineto{\pgfqpoint{1.998736in}{3.534288in}}%
\pgfpathlineto{\pgfqpoint{1.999810in}{3.545621in}}%
\pgfpathlineto{\pgfqpoint{2.003031in}{3.539734in}}%
\pgfpathlineto{\pgfqpoint{2.004105in}{3.534582in}}%
\pgfpathlineto{\pgfqpoint{2.006253in}{3.546504in}}%
\pgfpathlineto{\pgfqpoint{2.007326in}{3.538409in}}%
\pgfpathlineto{\pgfqpoint{2.010548in}{3.533699in}}%
\pgfpathlineto{\pgfqpoint{2.011622in}{3.535171in}}%
\pgfpathlineto{\pgfqpoint{2.012696in}{3.535465in}}%
\pgfpathlineto{\pgfqpoint{2.013770in}{3.524573in}}%
\pgfpathlineto{\pgfqpoint{2.014843in}{3.534877in}}%
\pgfpathlineto{\pgfqpoint{2.019139in}{3.542530in}}%
\pgfpathlineto{\pgfqpoint{2.020213in}{3.548565in}}%
\pgfpathlineto{\pgfqpoint{2.021286in}{3.545327in}}%
\pgfpathlineto{\pgfqpoint{2.022360in}{3.544591in}}%
\pgfpathlineto{\pgfqpoint{2.025582in}{3.549448in}}%
\pgfpathlineto{\pgfqpoint{2.027729in}{3.543708in}}%
\pgfpathlineto{\pgfqpoint{2.028803in}{3.550184in}}%
\pgfpathlineto{\pgfqpoint{2.029877in}{3.552392in}}%
\pgfpathlineto{\pgfqpoint{2.033099in}{3.549007in}}%
\pgfpathlineto{\pgfqpoint{2.034172in}{3.558427in}}%
\pgfpathlineto{\pgfqpoint{2.035246in}{3.562254in}}%
\pgfpathlineto{\pgfqpoint{2.036320in}{3.562842in}}%
\pgfpathlineto{\pgfqpoint{2.037394in}{3.565933in}}%
\pgfpathlineto{\pgfqpoint{2.040615in}{3.563725in}}%
\pgfpathlineto{\pgfqpoint{2.041689in}{3.560782in}}%
\pgfpathlineto{\pgfqpoint{2.042763in}{3.560634in}}%
\pgfpathlineto{\pgfqpoint{2.043837in}{3.559310in}}%
\pgfpathlineto{\pgfqpoint{2.044911in}{3.564314in}}%
\pgfpathlineto{\pgfqpoint{2.049206in}{3.563578in}}%
\pgfpathlineto{\pgfqpoint{2.050280in}{3.562254in}}%
\pgfpathlineto{\pgfqpoint{2.051354in}{3.572262in}}%
\pgfpathlineto{\pgfqpoint{2.052428in}{3.571821in}}%
\pgfpathlineto{\pgfqpoint{2.055649in}{3.573881in}}%
\pgfpathlineto{\pgfqpoint{2.057797in}{3.573734in}}%
\pgfpathlineto{\pgfqpoint{2.058871in}{3.574176in}}%
\pgfpathlineto{\pgfqpoint{2.063166in}{3.579033in}}%
\pgfpathlineto{\pgfqpoint{2.064240in}{3.578886in}}%
\pgfpathlineto{\pgfqpoint{2.065314in}{3.583449in}}%
\pgfpathlineto{\pgfqpoint{2.066388in}{3.582860in}}%
\pgfpathlineto{\pgfqpoint{2.067461in}{3.584626in}}%
\pgfpathlineto{\pgfqpoint{2.073904in}{3.568141in}}%
\pgfpathlineto{\pgfqpoint{2.074978in}{3.570054in}}%
\pgfpathlineto{\pgfqpoint{2.078200in}{3.569024in}}%
\pgfpathlineto{\pgfqpoint{2.079274in}{3.570349in}}%
\pgfpathlineto{\pgfqpoint{2.080348in}{3.572557in}}%
\pgfpathlineto{\pgfqpoint{2.082495in}{3.573145in}}%
\pgfpathlineto{\pgfqpoint{2.085717in}{3.573587in}}%
\pgfpathlineto{\pgfqpoint{2.086791in}{3.574764in}}%
\pgfpathlineto{\pgfqpoint{2.087864in}{3.574764in}}%
\pgfpathlineto{\pgfqpoint{2.090012in}{3.569760in}}%
\pgfpathlineto{\pgfqpoint{2.093234in}{3.568730in}}%
\pgfpathlineto{\pgfqpoint{2.094307in}{3.571526in}}%
\pgfpathlineto{\pgfqpoint{2.095381in}{3.571968in}}%
\pgfpathlineto{\pgfqpoint{2.096455in}{3.571526in}}%
\pgfpathlineto{\pgfqpoint{2.097529in}{3.569613in}}%
\pgfpathlineto{\pgfqpoint{2.100750in}{3.571379in}}%
\pgfpathlineto{\pgfqpoint{2.101824in}{3.568141in}}%
\pgfpathlineto{\pgfqpoint{2.102898in}{3.560929in}}%
\pgfpathlineto{\pgfqpoint{2.103972in}{3.558574in}}%
\pgfpathlineto{\pgfqpoint{2.105046in}{3.561518in}}%
\pgfpathlineto{\pgfqpoint{2.108267in}{3.558279in}}%
\pgfpathlineto{\pgfqpoint{2.109341in}{3.566080in}}%
\pgfpathlineto{\pgfqpoint{2.110415in}{3.564314in}}%
\pgfpathlineto{\pgfqpoint{2.111489in}{3.561223in}}%
\pgfpathlineto{\pgfqpoint{2.112563in}{3.555483in}}%
\pgfpathlineto{\pgfqpoint{2.115784in}{3.559457in}}%
\pgfpathlineto{\pgfqpoint{2.116858in}{3.556366in}}%
\pgfpathlineto{\pgfqpoint{2.117932in}{3.555041in}}%
\pgfpathlineto{\pgfqpoint{2.119006in}{3.551803in}}%
\pgfpathlineto{\pgfqpoint{2.120080in}{3.554453in}}%
\pgfpathlineto{\pgfqpoint{2.123301in}{3.553422in}}%
\pgfpathlineto{\pgfqpoint{2.125449in}{3.561223in}}%
\pgfpathlineto{\pgfqpoint{2.126523in}{3.560340in}}%
\pgfpathlineto{\pgfqpoint{2.127596in}{3.561518in}}%
\pgfpathlineto{\pgfqpoint{2.131892in}{3.562989in}}%
\pgfpathlineto{\pgfqpoint{2.134039in}{3.560634in}}%
\pgfpathlineto{\pgfqpoint{2.135113in}{3.559310in}}%
\pgfpathlineto{\pgfqpoint{2.139409in}{3.561812in}}%
\pgfpathlineto{\pgfqpoint{2.140482in}{3.564020in}}%
\pgfpathlineto{\pgfqpoint{2.141556in}{3.563137in}}%
\pgfpathlineto{\pgfqpoint{2.142630in}{3.560487in}}%
\pgfpathlineto{\pgfqpoint{2.145852in}{3.557838in}}%
\pgfpathlineto{\pgfqpoint{2.146926in}{3.565639in}}%
\pgfpathlineto{\pgfqpoint{2.147999in}{3.567552in}}%
\pgfpathlineto{\pgfqpoint{2.149073in}{3.571232in}}%
\pgfpathlineto{\pgfqpoint{2.150147in}{3.570643in}}%
\pgfpathlineto{\pgfqpoint{2.154442in}{3.575059in}}%
\pgfpathlineto{\pgfqpoint{2.155516in}{3.572998in}}%
\pgfpathlineto{\pgfqpoint{2.156590in}{3.578297in}}%
\pgfpathlineto{\pgfqpoint{2.157664in}{3.560634in}}%
\pgfpathlineto{\pgfqpoint{2.160885in}{3.554158in}}%
\pgfpathlineto{\pgfqpoint{2.165181in}{3.582124in}}%
\pgfpathlineto{\pgfqpoint{2.169476in}{3.581535in}}%
\pgfpathlineto{\pgfqpoint{2.170550in}{3.585362in}}%
\pgfpathlineto{\pgfqpoint{2.171624in}{3.586392in}}%
\pgfpathlineto{\pgfqpoint{2.172698in}{3.591250in}}%
\pgfpathlineto{\pgfqpoint{2.176993in}{3.591985in}}%
\pgfpathlineto{\pgfqpoint{2.178067in}{3.593310in}}%
\pgfpathlineto{\pgfqpoint{2.180214in}{3.600081in}}%
\pgfpathlineto{\pgfqpoint{2.184510in}{3.600670in}}%
\pgfpathlineto{\pgfqpoint{2.186658in}{3.596254in}}%
\pgfpathlineto{\pgfqpoint{2.187731in}{3.589483in}}%
\pgfpathlineto{\pgfqpoint{2.195248in}{3.577561in}}%
\pgfpathlineto{\pgfqpoint{2.198470in}{3.577708in}}%
\pgfpathlineto{\pgfqpoint{2.199544in}{3.576384in}}%
\pgfpathlineto{\pgfqpoint{2.202765in}{3.579475in}}%
\pgfpathlineto{\pgfqpoint{2.213503in}{3.578886in}}%
\pgfpathlineto{\pgfqpoint{2.214577in}{3.578297in}}%
\pgfpathlineto{\pgfqpoint{2.216725in}{3.581388in}}%
\pgfpathlineto{\pgfqpoint{2.217799in}{3.579033in}}%
\pgfpathlineto{\pgfqpoint{2.221020in}{3.579916in}}%
\pgfpathlineto{\pgfqpoint{2.222094in}{3.578739in}}%
\pgfpathlineto{\pgfqpoint{2.225316in}{3.578739in}}%
\pgfpathlineto{\pgfqpoint{2.229611in}{3.580505in}}%
\pgfpathlineto{\pgfqpoint{2.230685in}{3.578886in}}%
\pgfpathlineto{\pgfqpoint{2.231759in}{3.578444in}}%
\pgfpathlineto{\pgfqpoint{2.232833in}{3.579475in}}%
\pgfpathlineto{\pgfqpoint{2.237128in}{3.576384in}}%
\pgfpathlineto{\pgfqpoint{2.239276in}{3.576384in}}%
\pgfpathlineto{\pgfqpoint{2.240349in}{3.564314in}}%
\pgfpathlineto{\pgfqpoint{2.243571in}{3.569171in}}%
\pgfpathlineto{\pgfqpoint{2.244645in}{3.560929in}}%
\pgfpathlineto{\pgfqpoint{2.245719in}{3.559015in}}%
\pgfpathlineto{\pgfqpoint{2.246792in}{3.562695in}}%
\pgfpathlineto{\pgfqpoint{2.251088in}{3.558574in}}%
\pgfpathlineto{\pgfqpoint{2.254309in}{3.566522in}}%
\pgfpathlineto{\pgfqpoint{2.255383in}{3.564461in}}%
\pgfpathlineto{\pgfqpoint{2.258605in}{3.559898in}}%
\pgfpathlineto{\pgfqpoint{2.259679in}{3.564314in}}%
\pgfpathlineto{\pgfqpoint{2.260752in}{3.564609in}}%
\pgfpathlineto{\pgfqpoint{2.261826in}{3.559898in}}%
\pgfpathlineto{\pgfqpoint{2.262900in}{3.560929in}}%
\pgfpathlineto{\pgfqpoint{2.266122in}{3.561223in}}%
\pgfpathlineto{\pgfqpoint{2.267195in}{3.559457in}}%
\pgfpathlineto{\pgfqpoint{2.268269in}{3.559457in}}%
\pgfpathlineto{\pgfqpoint{2.270417in}{3.554011in}}%
\pgfpathlineto{\pgfqpoint{2.273638in}{3.551067in}}%
\pgfpathlineto{\pgfqpoint{2.274712in}{3.551950in}}%
\pgfpathlineto{\pgfqpoint{2.275786in}{3.551656in}}%
\pgfpathlineto{\pgfqpoint{2.276860in}{3.550037in}}%
\pgfpathlineto{\pgfqpoint{2.277934in}{3.551509in}}%
\pgfpathlineto{\pgfqpoint{2.281155in}{3.551067in}}%
\pgfpathlineto{\pgfqpoint{2.283303in}{3.553717in}}%
\pgfpathlineto{\pgfqpoint{2.284377in}{3.553864in}}%
\pgfpathlineto{\pgfqpoint{2.285451in}{3.552686in}}%
\pgfpathlineto{\pgfqpoint{2.288672in}{3.551950in}}%
\pgfpathlineto{\pgfqpoint{2.289746in}{3.548418in}}%
\pgfpathlineto{\pgfqpoint{2.290820in}{3.551214in}}%
\pgfpathlineto{\pgfqpoint{2.291894in}{3.548123in}}%
\pgfpathlineto{\pgfqpoint{2.292968in}{3.555777in}}%
\pgfpathlineto{\pgfqpoint{2.296189in}{3.554305in}}%
\pgfpathlineto{\pgfqpoint{2.297263in}{3.551362in}}%
\pgfpathlineto{\pgfqpoint{2.299411in}{3.543561in}}%
\pgfpathlineto{\pgfqpoint{2.300484in}{3.545621in}}%
\pgfpathlineto{\pgfqpoint{2.303706in}{3.556955in}}%
\pgfpathlineto{\pgfqpoint{2.304780in}{3.558427in}}%
\pgfpathlineto{\pgfqpoint{2.305854in}{3.561076in}}%
\pgfpathlineto{\pgfqpoint{2.306927in}{3.571232in}}%
\pgfpathlineto{\pgfqpoint{2.308001in}{3.575206in}}%
\pgfpathlineto{\pgfqpoint{2.311223in}{3.572557in}}%
\pgfpathlineto{\pgfqpoint{2.312297in}{3.575648in}}%
\pgfpathlineto{\pgfqpoint{2.313370in}{3.575500in}}%
\pgfpathlineto{\pgfqpoint{2.314444in}{3.576236in}}%
\pgfpathlineto{\pgfqpoint{2.315518in}{3.574617in}}%
\pgfpathlineto{\pgfqpoint{2.318740in}{3.577267in}}%
\pgfpathlineto{\pgfqpoint{2.320887in}{3.583301in}}%
\pgfpathlineto{\pgfqpoint{2.323035in}{3.584626in}}%
\pgfpathlineto{\pgfqpoint{2.326257in}{3.582124in}}%
\pgfpathlineto{\pgfqpoint{2.328404in}{3.575795in}}%
\pgfpathlineto{\pgfqpoint{2.329478in}{3.584037in}}%
\pgfpathlineto{\pgfqpoint{2.330552in}{3.583301in}}%
\pgfpathlineto{\pgfqpoint{2.333773in}{3.580358in}}%
\pgfpathlineto{\pgfqpoint{2.334847in}{3.581094in}}%
\pgfpathlineto{\pgfqpoint{2.335921in}{3.586687in}}%
\pgfpathlineto{\pgfqpoint{2.336995in}{3.585804in}}%
\pgfpathlineto{\pgfqpoint{2.338069in}{3.589042in}}%
\pgfpathlineto{\pgfqpoint{2.341290in}{3.590072in}}%
\pgfpathlineto{\pgfqpoint{2.342364in}{3.588453in}}%
\pgfpathlineto{\pgfqpoint{2.344512in}{3.582271in}}%
\pgfpathlineto{\pgfqpoint{2.345586in}{3.588600in}}%
\pgfpathlineto{\pgfqpoint{2.348807in}{3.590808in}}%
\pgfpathlineto{\pgfqpoint{2.349881in}{3.595076in}}%
\pgfpathlineto{\pgfqpoint{2.352029in}{3.592721in}}%
\pgfpathlineto{\pgfqpoint{2.353102in}{3.593457in}}%
\pgfpathlineto{\pgfqpoint{2.357398in}{3.593752in}}%
\pgfpathlineto{\pgfqpoint{2.358472in}{3.591102in}}%
\pgfpathlineto{\pgfqpoint{2.359546in}{3.591250in}}%
\pgfpathlineto{\pgfqpoint{2.360619in}{3.589778in}}%
\pgfpathlineto{\pgfqpoint{2.365989in}{3.591102in}}%
\pgfpathlineto{\pgfqpoint{2.367062in}{3.588747in}}%
\pgfpathlineto{\pgfqpoint{2.368136in}{3.589925in}}%
\pgfpathlineto{\pgfqpoint{2.372432in}{3.586834in}}%
\pgfpathlineto{\pgfqpoint{2.373505in}{3.588159in}}%
\pgfpathlineto{\pgfqpoint{2.374579in}{3.586981in}}%
\pgfpathlineto{\pgfqpoint{2.382096in}{3.584626in}}%
\pgfpathlineto{\pgfqpoint{2.383170in}{3.575795in}}%
\pgfpathlineto{\pgfqpoint{2.386391in}{3.565639in}}%
\pgfpathlineto{\pgfqpoint{2.388539in}{3.573734in}}%
\pgfpathlineto{\pgfqpoint{2.389613in}{3.572998in}}%
\pgfpathlineto{\pgfqpoint{2.390687in}{3.569024in}}%
\pgfpathlineto{\pgfqpoint{2.393908in}{3.568288in}}%
\pgfpathlineto{\pgfqpoint{2.394982in}{3.564903in}}%
\pgfpathlineto{\pgfqpoint{2.401425in}{3.564461in}}%
\pgfpathlineto{\pgfqpoint{2.403573in}{3.561370in}}%
\pgfpathlineto{\pgfqpoint{2.405721in}{3.565197in}}%
\pgfpathlineto{\pgfqpoint{2.408942in}{3.569319in}}%
\pgfpathlineto{\pgfqpoint{2.410016in}{3.572557in}}%
\pgfpathlineto{\pgfqpoint{2.412164in}{3.574764in}}%
\pgfpathlineto{\pgfqpoint{2.413237in}{3.573734in}}%
\pgfpathlineto{\pgfqpoint{2.417533in}{3.575648in}}%
\pgfpathlineto{\pgfqpoint{2.418607in}{3.573145in}}%
\pgfpathlineto{\pgfqpoint{2.419680in}{3.572262in}}%
\pgfpathlineto{\pgfqpoint{2.420754in}{3.574470in}}%
\pgfpathlineto{\pgfqpoint{2.425050in}{3.569466in}}%
\pgfpathlineto{\pgfqpoint{2.426124in}{3.574470in}}%
\pgfpathlineto{\pgfqpoint{2.427197in}{3.574470in}}%
\pgfpathlineto{\pgfqpoint{2.428271in}{3.573587in}}%
\pgfpathlineto{\pgfqpoint{2.431493in}{3.571968in}}%
\pgfpathlineto{\pgfqpoint{2.434714in}{3.567552in}}%
\pgfpathlineto{\pgfqpoint{2.435788in}{3.575648in}}%
\pgfpathlineto{\pgfqpoint{2.439010in}{3.570202in}}%
\pgfpathlineto{\pgfqpoint{2.440083in}{3.565933in}}%
\pgfpathlineto{\pgfqpoint{2.441157in}{3.568877in}}%
\pgfpathlineto{\pgfqpoint{2.442231in}{3.568730in}}%
\pgfpathlineto{\pgfqpoint{2.443305in}{3.570349in}}%
\pgfpathlineto{\pgfqpoint{2.446526in}{3.568583in}}%
\pgfpathlineto{\pgfqpoint{2.447600in}{3.564020in}}%
\pgfpathlineto{\pgfqpoint{2.450822in}{3.568288in}}%
\pgfpathlineto{\pgfqpoint{2.454043in}{3.564609in}}%
\pgfpathlineto{\pgfqpoint{2.455117in}{3.566964in}}%
\pgfpathlineto{\pgfqpoint{2.456191in}{3.567847in}}%
\pgfpathlineto{\pgfqpoint{2.457265in}{3.570349in}}%
\pgfpathlineto{\pgfqpoint{2.458339in}{3.569319in}}%
\pgfpathlineto{\pgfqpoint{2.461560in}{3.570349in}}%
\pgfpathlineto{\pgfqpoint{2.462634in}{3.572262in}}%
\pgfpathlineto{\pgfqpoint{2.464782in}{3.571085in}}%
\pgfpathlineto{\pgfqpoint{2.465856in}{3.571821in}}%
\pgfpathlineto{\pgfqpoint{2.470151in}{3.572557in}}%
\pgfpathlineto{\pgfqpoint{2.472299in}{3.566228in}}%
\pgfpathlineto{\pgfqpoint{2.476594in}{3.567258in}}%
\pgfpathlineto{\pgfqpoint{2.479815in}{3.575500in}}%
\pgfpathlineto{\pgfqpoint{2.480889in}{3.566080in}}%
\pgfpathlineto{\pgfqpoint{2.484111in}{3.566080in}}%
\pgfpathlineto{\pgfqpoint{2.485185in}{3.564756in}}%
\pgfpathlineto{\pgfqpoint{2.487332in}{3.559898in}}%
\pgfpathlineto{\pgfqpoint{2.488406in}{3.558721in}}%
\pgfpathlineto{\pgfqpoint{2.491628in}{3.557985in}}%
\pgfpathlineto{\pgfqpoint{2.492702in}{3.558721in}}%
\pgfpathlineto{\pgfqpoint{2.493775in}{3.561812in}}%
\pgfpathlineto{\pgfqpoint{2.495923in}{3.561665in}}%
\pgfpathlineto{\pgfqpoint{2.499145in}{3.559751in}}%
\pgfpathlineto{\pgfqpoint{2.501292in}{3.554894in}}%
\pgfpathlineto{\pgfqpoint{2.502366in}{3.557102in}}%
\pgfpathlineto{\pgfqpoint{2.503440in}{3.549154in}}%
\pgfpathlineto{\pgfqpoint{2.506661in}{3.547976in}}%
\pgfpathlineto{\pgfqpoint{2.507735in}{3.546210in}}%
\pgfpathlineto{\pgfqpoint{2.508809in}{3.537820in}}%
\pgfpathlineto{\pgfqpoint{2.509883in}{3.538703in}}%
\pgfpathlineto{\pgfqpoint{2.510957in}{3.546357in}}%
\pgfpathlineto{\pgfqpoint{2.514178in}{3.547976in}}%
\pgfpathlineto{\pgfqpoint{2.515252in}{3.549301in}}%
\pgfpathlineto{\pgfqpoint{2.517400in}{3.538851in}}%
\pgfpathlineto{\pgfqpoint{2.522769in}{3.537232in}}%
\pgfpathlineto{\pgfqpoint{2.523843in}{3.537526in}}%
\pgfpathlineto{\pgfqpoint{2.524917in}{3.541942in}}%
\pgfpathlineto{\pgfqpoint{2.525991in}{3.544002in}}%
\pgfpathlineto{\pgfqpoint{2.529212in}{3.545327in}}%
\pgfpathlineto{\pgfqpoint{2.530286in}{3.544738in}}%
\pgfpathlineto{\pgfqpoint{2.531360in}{3.541353in}}%
\pgfpathlineto{\pgfqpoint{2.532434in}{3.540175in}}%
\pgfpathlineto{\pgfqpoint{2.536729in}{3.557985in}}%
\pgfpathlineto{\pgfqpoint{2.537803in}{3.551509in}}%
\pgfpathlineto{\pgfqpoint{2.538877in}{3.554747in}}%
\pgfpathlineto{\pgfqpoint{2.541024in}{3.565344in}}%
\pgfpathlineto{\pgfqpoint{2.544246in}{3.562695in}}%
\pgfpathlineto{\pgfqpoint{2.546393in}{3.549007in}}%
\pgfpathlineto{\pgfqpoint{2.547467in}{3.545916in}}%
\pgfpathlineto{\pgfqpoint{2.551763in}{3.546799in}}%
\pgfpathlineto{\pgfqpoint{2.552836in}{3.541353in}}%
\pgfpathlineto{\pgfqpoint{2.554984in}{3.538851in}}%
\pgfpathlineto{\pgfqpoint{2.556058in}{3.538703in}}%
\pgfpathlineto{\pgfqpoint{2.559279in}{3.544591in}}%
\pgfpathlineto{\pgfqpoint{2.561427in}{3.543266in}}%
\pgfpathlineto{\pgfqpoint{2.562501in}{3.529283in}}%
\pgfpathlineto{\pgfqpoint{2.563575in}{3.527223in}}%
\pgfpathlineto{\pgfqpoint{2.566796in}{3.525751in}}%
\pgfpathlineto{\pgfqpoint{2.568944in}{3.532963in}}%
\pgfpathlineto{\pgfqpoint{2.570018in}{3.535760in}}%
\pgfpathlineto{\pgfqpoint{2.571092in}{3.535612in}}%
\pgfpathlineto{\pgfqpoint{2.574313in}{3.536201in}}%
\pgfpathlineto{\pgfqpoint{2.576461in}{3.537820in}}%
\pgfpathlineto{\pgfqpoint{2.577535in}{3.534435in}}%
\pgfpathlineto{\pgfqpoint{2.578609in}{3.523985in}}%
\pgfpathlineto{\pgfqpoint{2.581830in}{3.517508in}}%
\pgfpathlineto{\pgfqpoint{2.582904in}{3.517656in}}%
\pgfpathlineto{\pgfqpoint{2.585052in}{3.522366in}}%
\pgfpathlineto{\pgfqpoint{2.586125in}{3.518833in}}%
\pgfpathlineto{\pgfqpoint{2.589347in}{3.519863in}}%
\pgfpathlineto{\pgfqpoint{2.590421in}{3.517656in}}%
\pgfpathlineto{\pgfqpoint{2.591495in}{3.518686in}}%
\pgfpathlineto{\pgfqpoint{2.592568in}{3.521924in}}%
\pgfpathlineto{\pgfqpoint{2.593642in}{3.522218in}}%
\pgfpathlineto{\pgfqpoint{2.597938in}{3.519275in}}%
\pgfpathlineto{\pgfqpoint{2.599012in}{3.521335in}}%
\pgfpathlineto{\pgfqpoint{2.600085in}{3.515742in}}%
\pgfpathlineto{\pgfqpoint{2.601159in}{3.514417in}}%
\pgfpathlineto{\pgfqpoint{2.604381in}{3.516625in}}%
\pgfpathlineto{\pgfqpoint{2.605455in}{3.513681in}}%
\pgfpathlineto{\pgfqpoint{2.606528in}{3.513093in}}%
\pgfpathlineto{\pgfqpoint{2.608676in}{3.505733in}}%
\pgfpathlineto{\pgfqpoint{2.611898in}{3.504850in}}%
\pgfpathlineto{\pgfqpoint{2.612971in}{3.506322in}}%
\pgfpathlineto{\pgfqpoint{2.614045in}{3.503525in}}%
\pgfpathlineto{\pgfqpoint{2.615119in}{3.502348in}}%
\pgfpathlineto{\pgfqpoint{2.616193in}{3.504850in}}%
\pgfpathlineto{\pgfqpoint{2.620488in}{3.504261in}}%
\pgfpathlineto{\pgfqpoint{2.621562in}{3.502054in}}%
\pgfpathlineto{\pgfqpoint{2.622636in}{3.505733in}}%
\pgfpathlineto{\pgfqpoint{2.623710in}{3.513681in}}%
\pgfpathlineto{\pgfqpoint{2.628005in}{3.508383in}}%
\pgfpathlineto{\pgfqpoint{2.629079in}{3.510738in}}%
\pgfpathlineto{\pgfqpoint{2.630153in}{3.498668in}}%
\pgfpathlineto{\pgfqpoint{2.631227in}{3.495872in}}%
\pgfpathlineto{\pgfqpoint{2.634448in}{3.494547in}}%
\pgfpathlineto{\pgfqpoint{2.637670in}{3.501759in}}%
\pgfpathlineto{\pgfqpoint{2.638744in}{3.500582in}}%
\pgfpathlineto{\pgfqpoint{2.641965in}{3.507794in}}%
\pgfpathlineto{\pgfqpoint{2.643039in}{3.504261in}}%
\pgfpathlineto{\pgfqpoint{2.644113in}{3.505880in}}%
\pgfpathlineto{\pgfqpoint{2.645187in}{3.511768in}}%
\pgfpathlineto{\pgfqpoint{2.646260in}{3.513387in}}%
\pgfpathlineto{\pgfqpoint{2.649482in}{3.516625in}}%
\pgfpathlineto{\pgfqpoint{2.650556in}{3.514123in}}%
\pgfpathlineto{\pgfqpoint{2.651630in}{3.506616in}}%
\pgfpathlineto{\pgfqpoint{2.653777in}{3.503967in}}%
\pgfpathlineto{\pgfqpoint{2.656999in}{3.509266in}}%
\pgfpathlineto{\pgfqpoint{2.658073in}{3.512357in}}%
\pgfpathlineto{\pgfqpoint{2.659146in}{3.508088in}}%
\pgfpathlineto{\pgfqpoint{2.660220in}{3.508824in}}%
\pgfpathlineto{\pgfqpoint{2.661294in}{3.506764in}}%
\pgfpathlineto{\pgfqpoint{2.664516in}{3.493811in}}%
\pgfpathlineto{\pgfqpoint{2.665590in}{3.492928in}}%
\pgfpathlineto{\pgfqpoint{2.666663in}{3.488954in}}%
\pgfpathlineto{\pgfqpoint{2.667737in}{3.488659in}}%
\pgfpathlineto{\pgfqpoint{2.668811in}{3.487776in}}%
\pgfpathlineto{\pgfqpoint{2.672033in}{3.492928in}}%
\pgfpathlineto{\pgfqpoint{2.673106in}{3.490573in}}%
\pgfpathlineto{\pgfqpoint{2.674180in}{3.489690in}}%
\pgfpathlineto{\pgfqpoint{2.676328in}{3.499257in}}%
\pgfpathlineto{\pgfqpoint{2.681697in}{3.467759in}}%
\pgfpathlineto{\pgfqpoint{2.682771in}{3.465257in}}%
\pgfpathlineto{\pgfqpoint{2.683845in}{3.458044in}}%
\pgfpathlineto{\pgfqpoint{2.687066in}{3.452893in}}%
\pgfpathlineto{\pgfqpoint{2.689214in}{3.447594in}}%
\pgfpathlineto{\pgfqpoint{2.690288in}{3.446564in}}%
\pgfpathlineto{\pgfqpoint{2.691362in}{3.449213in}}%
\pgfpathlineto{\pgfqpoint{2.694583in}{3.449066in}}%
\pgfpathlineto{\pgfqpoint{2.695657in}{3.450243in}}%
\pgfpathlineto{\pgfqpoint{2.697805in}{3.447300in}}%
\pgfpathlineto{\pgfqpoint{2.698879in}{3.454070in}}%
\pgfpathlineto{\pgfqpoint{2.702100in}{3.434053in}}%
\pgfpathlineto{\pgfqpoint{2.703174in}{3.418892in}}%
\pgfpathlineto{\pgfqpoint{2.704248in}{3.423750in}}%
\pgfpathlineto{\pgfqpoint{2.706395in}{3.423161in}}%
\pgfpathlineto{\pgfqpoint{2.709617in}{3.420070in}}%
\pgfpathlineto{\pgfqpoint{2.710691in}{3.418009in}}%
\pgfpathlineto{\pgfqpoint{2.711765in}{3.422278in}}%
\pgfpathlineto{\pgfqpoint{2.713912in}{3.422866in}}%
\pgfpathlineto{\pgfqpoint{2.717134in}{3.421836in}}%
\pgfpathlineto{\pgfqpoint{2.718208in}{3.425810in}}%
\pgfpathlineto{\pgfqpoint{2.719281in}{3.426840in}}%
\pgfpathlineto{\pgfqpoint{2.720355in}{3.424191in}}%
\pgfpathlineto{\pgfqpoint{2.721429in}{3.418598in}}%
\pgfpathlineto{\pgfqpoint{2.724651in}{3.419628in}}%
\pgfpathlineto{\pgfqpoint{2.725724in}{3.416979in}}%
\pgfpathlineto{\pgfqpoint{2.728946in}{3.418009in}}%
\pgfpathlineto{\pgfqpoint{2.732167in}{3.417273in}}%
\pgfpathlineto{\pgfqpoint{2.733241in}{3.420806in}}%
\pgfpathlineto{\pgfqpoint{2.735389in}{3.417126in}}%
\pgfpathlineto{\pgfqpoint{2.736463in}{3.419481in}}%
\pgfpathlineto{\pgfqpoint{2.739684in}{3.418745in}}%
\pgfpathlineto{\pgfqpoint{2.741832in}{3.414477in}}%
\pgfpathlineto{\pgfqpoint{2.748275in}{3.415801in}}%
\pgfpathlineto{\pgfqpoint{2.750423in}{3.414918in}}%
\pgfpathlineto{\pgfqpoint{2.751497in}{3.416096in}}%
\pgfpathlineto{\pgfqpoint{2.751497in}{3.416096in}}%
\pgfusepath{stroke}%
\end{pgfscope}%
\begin{pgfscope}%
\pgfpathrectangle{\pgfqpoint{0.285588in}{3.218007in}}{\pgfqpoint{2.583333in}{0.400885in}}%
\pgfusepath{clip}%
\pgfsetroundcap%
\pgfsetroundjoin%
\pgfsetlinewidth{1.505625pt}%
\definecolor{currentstroke}{rgb}{0.172549,0.627451,0.172549}%
\pgfsetstrokecolor{currentstroke}%
\pgfsetdash{}{0pt}%
\pgfpathmoveto{\pgfqpoint{0.403012in}{3.379446in}}%
\pgfpathlineto{\pgfqpoint{0.404086in}{3.377238in}}%
\pgfpathlineto{\pgfqpoint{0.405159in}{3.377382in}}%
\pgfpathlineto{\pgfqpoint{0.406233in}{3.376228in}}%
\pgfpathlineto{\pgfqpoint{0.409455in}{3.373947in}}%
\pgfpathlineto{\pgfqpoint{0.410529in}{3.375480in}}%
\pgfpathlineto{\pgfqpoint{0.411602in}{3.373781in}}%
\pgfpathlineto{\pgfqpoint{0.412676in}{3.373225in}}%
\pgfpathlineto{\pgfqpoint{0.413750in}{3.374193in}}%
\pgfpathlineto{\pgfqpoint{0.418045in}{3.375172in}}%
\pgfpathlineto{\pgfqpoint{0.419119in}{3.372208in}}%
\pgfpathlineto{\pgfqpoint{0.420193in}{3.370975in}}%
\pgfpathlineto{\pgfqpoint{0.421267in}{3.370975in}}%
\pgfpathlineto{\pgfqpoint{0.425562in}{3.374109in}}%
\pgfpathlineto{\pgfqpoint{0.426636in}{3.371036in}}%
\pgfpathlineto{\pgfqpoint{0.428784in}{3.372122in}}%
\pgfpathlineto{\pgfqpoint{0.432005in}{3.373354in}}%
\pgfpathlineto{\pgfqpoint{0.433079in}{3.371275in}}%
\pgfpathlineto{\pgfqpoint{0.434153in}{3.371968in}}%
\pgfpathlineto{\pgfqpoint{0.435227in}{3.371691in}}%
\pgfpathlineto{\pgfqpoint{0.436301in}{3.368779in}}%
\pgfpathlineto{\pgfqpoint{0.439522in}{3.368510in}}%
\pgfpathlineto{\pgfqpoint{0.441670in}{3.366639in}}%
\pgfpathlineto{\pgfqpoint{0.442744in}{3.367691in}}%
\pgfpathlineto{\pgfqpoint{0.443818in}{3.365163in}}%
\pgfpathlineto{\pgfqpoint{0.447039in}{3.363300in}}%
\pgfpathlineto{\pgfqpoint{0.448113in}{3.364474in}}%
\pgfpathlineto{\pgfqpoint{0.450261in}{3.360115in}}%
\pgfpathlineto{\pgfqpoint{0.451334in}{3.357543in}}%
\pgfpathlineto{\pgfqpoint{0.457778in}{3.355673in}}%
\pgfpathlineto{\pgfqpoint{0.462073in}{3.357889in}}%
\pgfpathlineto{\pgfqpoint{0.463147in}{3.357010in}}%
\pgfpathlineto{\pgfqpoint{0.464221in}{3.358005in}}%
\pgfpathlineto{\pgfqpoint{0.465294in}{3.357376in}}%
\pgfpathlineto{\pgfqpoint{0.466368in}{3.358749in}}%
\pgfpathlineto{\pgfqpoint{0.469590in}{3.360017in}}%
\pgfpathlineto{\pgfqpoint{0.470664in}{3.355904in}}%
\pgfpathlineto{\pgfqpoint{0.471737in}{3.359246in}}%
\pgfpathlineto{\pgfqpoint{0.472811in}{3.356675in}}%
\pgfpathlineto{\pgfqpoint{0.477107in}{3.355677in}}%
\pgfpathlineto{\pgfqpoint{0.478180in}{3.351351in}}%
\pgfpathlineto{\pgfqpoint{0.479254in}{3.349585in}}%
\pgfpathlineto{\pgfqpoint{0.480328in}{3.346353in}}%
\pgfpathlineto{\pgfqpoint{0.481402in}{3.346020in}}%
\pgfpathlineto{\pgfqpoint{0.484623in}{3.345909in}}%
\pgfpathlineto{\pgfqpoint{0.485697in}{3.347125in}}%
\pgfpathlineto{\pgfqpoint{0.486771in}{3.347125in}}%
\pgfpathlineto{\pgfqpoint{0.487845in}{3.349033in}}%
\pgfpathlineto{\pgfqpoint{0.488919in}{3.349606in}}%
\pgfpathlineto{\pgfqpoint{0.492140in}{3.347296in}}%
\pgfpathlineto{\pgfqpoint{0.494288in}{3.347634in}}%
\pgfpathlineto{\pgfqpoint{0.495362in}{3.348085in}}%
\pgfpathlineto{\pgfqpoint{0.496436in}{3.347064in}}%
\pgfpathlineto{\pgfqpoint{0.500731in}{3.347964in}}%
\pgfpathlineto{\pgfqpoint{0.502879in}{3.352093in}}%
\pgfpathlineto{\pgfqpoint{0.507174in}{3.354709in}}%
\pgfpathlineto{\pgfqpoint{0.508248in}{3.350421in}}%
\pgfpathlineto{\pgfqpoint{0.509322in}{3.352994in}}%
\pgfpathlineto{\pgfqpoint{0.511469in}{3.346497in}}%
\pgfpathlineto{\pgfqpoint{0.514691in}{3.346616in}}%
\pgfpathlineto{\pgfqpoint{0.516839in}{3.340423in}}%
\pgfpathlineto{\pgfqpoint{0.517912in}{3.340083in}}%
\pgfpathlineto{\pgfqpoint{0.518986in}{3.338165in}}%
\pgfpathlineto{\pgfqpoint{0.522208in}{3.335740in}}%
\pgfpathlineto{\pgfqpoint{0.523282in}{3.331772in}}%
\pgfpathlineto{\pgfqpoint{0.524355in}{3.332506in}}%
\pgfpathlineto{\pgfqpoint{0.526503in}{3.329882in}}%
\pgfpathlineto{\pgfqpoint{0.529725in}{3.331417in}}%
\pgfpathlineto{\pgfqpoint{0.530799in}{3.329747in}}%
\pgfpathlineto{\pgfqpoint{0.531872in}{3.329951in}}%
\pgfpathlineto{\pgfqpoint{0.532946in}{3.331181in}}%
\pgfpathlineto{\pgfqpoint{0.534020in}{3.333263in}}%
\pgfpathlineto{\pgfqpoint{0.538315in}{3.334013in}}%
\pgfpathlineto{\pgfqpoint{0.539389in}{3.331205in}}%
\pgfpathlineto{\pgfqpoint{0.540463in}{3.332717in}}%
\pgfpathlineto{\pgfqpoint{0.541537in}{3.332069in}}%
\pgfpathlineto{\pgfqpoint{0.545832in}{3.326994in}}%
\pgfpathlineto{\pgfqpoint{0.546906in}{3.331961in}}%
\pgfpathlineto{\pgfqpoint{0.547980in}{3.330989in}}%
\pgfpathlineto{\pgfqpoint{0.552275in}{3.332933in}}%
\pgfpathlineto{\pgfqpoint{0.555497in}{3.331857in}}%
\pgfpathlineto{\pgfqpoint{0.556571in}{3.332283in}}%
\pgfpathlineto{\pgfqpoint{0.560866in}{3.331105in}}%
\pgfpathlineto{\pgfqpoint{0.561940in}{3.328679in}}%
\pgfpathlineto{\pgfqpoint{0.563014in}{3.329101in}}%
\pgfpathlineto{\pgfqpoint{0.564088in}{3.324671in}}%
\pgfpathlineto{\pgfqpoint{0.567309in}{3.321506in}}%
\pgfpathlineto{\pgfqpoint{0.568383in}{3.322245in}}%
\pgfpathlineto{\pgfqpoint{0.569457in}{3.327413in}}%
\pgfpathlineto{\pgfqpoint{0.570531in}{3.328362in}}%
\pgfpathlineto{\pgfqpoint{0.571604in}{3.326780in}}%
\pgfpathlineto{\pgfqpoint{0.574826in}{3.326057in}}%
\pgfpathlineto{\pgfqpoint{0.575900in}{3.323165in}}%
\pgfpathlineto{\pgfqpoint{0.576974in}{3.323960in}}%
\pgfpathlineto{\pgfqpoint{0.579121in}{3.318826in}}%
\pgfpathlineto{\pgfqpoint{0.582343in}{3.320606in}}%
\pgfpathlineto{\pgfqpoint{0.583417in}{3.318781in}}%
\pgfpathlineto{\pgfqpoint{0.584490in}{3.318126in}}%
\pgfpathlineto{\pgfqpoint{0.585564in}{3.320909in}}%
\pgfpathlineto{\pgfqpoint{0.586638in}{3.318883in}}%
\pgfpathlineto{\pgfqpoint{0.589860in}{3.320947in}}%
\pgfpathlineto{\pgfqpoint{0.593081in}{3.316024in}}%
\pgfpathlineto{\pgfqpoint{0.594155in}{3.311606in}}%
\pgfpathlineto{\pgfqpoint{0.598450in}{3.314241in}}%
\pgfpathlineto{\pgfqpoint{0.600598in}{3.314945in}}%
\pgfpathlineto{\pgfqpoint{0.601672in}{3.317167in}}%
\pgfpathlineto{\pgfqpoint{0.604893in}{3.316891in}}%
\pgfpathlineto{\pgfqpoint{0.605967in}{3.319817in}}%
\pgfpathlineto{\pgfqpoint{0.607041in}{3.319436in}}%
\pgfpathlineto{\pgfqpoint{0.608115in}{3.321142in}}%
\pgfpathlineto{\pgfqpoint{0.609189in}{3.318715in}}%
\pgfpathlineto{\pgfqpoint{0.612410in}{3.320029in}}%
\pgfpathlineto{\pgfqpoint{0.614558in}{3.318129in}}%
\pgfpathlineto{\pgfqpoint{0.615632in}{3.318502in}}%
\pgfpathlineto{\pgfqpoint{0.619927in}{3.316361in}}%
\pgfpathlineto{\pgfqpoint{0.621001in}{3.317269in}}%
\pgfpathlineto{\pgfqpoint{0.622075in}{3.316993in}}%
\pgfpathlineto{\pgfqpoint{0.624222in}{3.310712in}}%
\pgfpathlineto{\pgfqpoint{0.629592in}{3.311889in}}%
\pgfpathlineto{\pgfqpoint{0.630666in}{3.313253in}}%
\pgfpathlineto{\pgfqpoint{0.631739in}{3.310383in}}%
\pgfpathlineto{\pgfqpoint{0.634961in}{3.310300in}}%
\pgfpathlineto{\pgfqpoint{0.636035in}{3.309301in}}%
\pgfpathlineto{\pgfqpoint{0.637109in}{3.310039in}}%
\pgfpathlineto{\pgfqpoint{0.639256in}{3.309459in}}%
\pgfpathlineto{\pgfqpoint{0.644625in}{3.310366in}}%
\pgfpathlineto{\pgfqpoint{0.645699in}{3.309783in}}%
\pgfpathlineto{\pgfqpoint{0.653216in}{3.312383in}}%
\pgfpathlineto{\pgfqpoint{0.654290in}{3.311352in}}%
\pgfpathlineto{\pgfqpoint{0.659659in}{3.311182in}}%
\pgfpathlineto{\pgfqpoint{0.660733in}{3.312363in}}%
\pgfpathlineto{\pgfqpoint{0.661807in}{3.311934in}}%
\pgfpathlineto{\pgfqpoint{0.666102in}{3.313214in}}%
\pgfpathlineto{\pgfqpoint{0.667176in}{3.312257in}}%
\pgfpathlineto{\pgfqpoint{0.668250in}{3.307968in}}%
\pgfpathlineto{\pgfqpoint{0.669324in}{3.306197in}}%
\pgfpathlineto{\pgfqpoint{0.672545in}{3.306902in}}%
\pgfpathlineto{\pgfqpoint{0.673619in}{3.306189in}}%
\pgfpathlineto{\pgfqpoint{0.675767in}{3.303624in}}%
\pgfpathlineto{\pgfqpoint{0.676841in}{3.303172in}}%
\pgfpathlineto{\pgfqpoint{0.680062in}{3.303471in}}%
\pgfpathlineto{\pgfqpoint{0.684357in}{3.299794in}}%
\pgfpathlineto{\pgfqpoint{0.688653in}{3.301004in}}%
\pgfpathlineto{\pgfqpoint{0.689727in}{3.302160in}}%
\pgfpathlineto{\pgfqpoint{0.690800in}{3.298626in}}%
\pgfpathlineto{\pgfqpoint{0.691874in}{3.298695in}}%
\pgfpathlineto{\pgfqpoint{0.698317in}{3.297448in}}%
\pgfpathlineto{\pgfqpoint{0.699391in}{3.296561in}}%
\pgfpathlineto{\pgfqpoint{0.702613in}{3.297570in}}%
\pgfpathlineto{\pgfqpoint{0.704760in}{3.300197in}}%
\pgfpathlineto{\pgfqpoint{0.706908in}{3.299910in}}%
\pgfpathlineto{\pgfqpoint{0.711203in}{3.299057in}}%
\pgfpathlineto{\pgfqpoint{0.712277in}{3.297584in}}%
\pgfpathlineto{\pgfqpoint{0.713351in}{3.298132in}}%
\pgfpathlineto{\pgfqpoint{0.714425in}{3.302277in}}%
\pgfpathlineto{\pgfqpoint{0.717646in}{3.304126in}}%
\pgfpathlineto{\pgfqpoint{0.718720in}{3.306639in}}%
\pgfpathlineto{\pgfqpoint{0.720868in}{3.306718in}}%
\pgfpathlineto{\pgfqpoint{0.721942in}{3.307670in}}%
\pgfpathlineto{\pgfqpoint{0.727311in}{3.307992in}}%
\pgfpathlineto{\pgfqpoint{0.728385in}{3.306213in}}%
\pgfpathlineto{\pgfqpoint{0.729459in}{3.306370in}}%
\pgfpathlineto{\pgfqpoint{0.732680in}{3.305818in}}%
\pgfpathlineto{\pgfqpoint{0.733754in}{3.304722in}}%
\pgfpathlineto{\pgfqpoint{0.735902in}{3.308940in}}%
\pgfpathlineto{\pgfqpoint{0.736976in}{3.308200in}}%
\pgfpathlineto{\pgfqpoint{0.740197in}{3.308931in}}%
\pgfpathlineto{\pgfqpoint{0.741271in}{3.310245in}}%
\pgfpathlineto{\pgfqpoint{0.742345in}{3.314601in}}%
\pgfpathlineto{\pgfqpoint{0.744492in}{3.313621in}}%
\pgfpathlineto{\pgfqpoint{0.747714in}{3.310183in}}%
\pgfpathlineto{\pgfqpoint{0.748788in}{3.310435in}}%
\pgfpathlineto{\pgfqpoint{0.749862in}{3.310014in}}%
\pgfpathlineto{\pgfqpoint{0.752009in}{3.307673in}}%
\pgfpathlineto{\pgfqpoint{0.755231in}{3.307592in}}%
\pgfpathlineto{\pgfqpoint{0.756305in}{3.308721in}}%
\pgfpathlineto{\pgfqpoint{0.757378in}{3.306998in}}%
\pgfpathlineto{\pgfqpoint{0.759526in}{3.307078in}}%
\pgfpathlineto{\pgfqpoint{0.762748in}{3.308999in}}%
\pgfpathlineto{\pgfqpoint{0.763821in}{3.308751in}}%
\pgfpathlineto{\pgfqpoint{0.764895in}{3.306452in}}%
\pgfpathlineto{\pgfqpoint{0.767043in}{3.305032in}}%
\pgfpathlineto{\pgfqpoint{0.770265in}{3.305420in}}%
\pgfpathlineto{\pgfqpoint{0.771338in}{3.304718in}}%
\pgfpathlineto{\pgfqpoint{0.772412in}{3.303096in}}%
\pgfpathlineto{\pgfqpoint{0.773486in}{3.304000in}}%
\pgfpathlineto{\pgfqpoint{0.774560in}{3.304000in}}%
\pgfpathlineto{\pgfqpoint{0.777781in}{3.302167in}}%
\pgfpathlineto{\pgfqpoint{0.778855in}{3.303577in}}%
\pgfpathlineto{\pgfqpoint{0.779929in}{3.307599in}}%
\pgfpathlineto{\pgfqpoint{0.781003in}{3.306143in}}%
\pgfpathlineto{\pgfqpoint{0.782077in}{3.307171in}}%
\pgfpathlineto{\pgfqpoint{0.788520in}{3.308383in}}%
\pgfpathlineto{\pgfqpoint{0.789594in}{3.309938in}}%
\pgfpathlineto{\pgfqpoint{0.794963in}{3.304192in}}%
\pgfpathlineto{\pgfqpoint{0.796037in}{3.305651in}}%
\pgfpathlineto{\pgfqpoint{0.797110in}{3.305023in}}%
\pgfpathlineto{\pgfqpoint{0.800332in}{3.305412in}}%
\pgfpathlineto{\pgfqpoint{0.801406in}{3.306821in}}%
\pgfpathlineto{\pgfqpoint{0.802480in}{3.306501in}}%
\pgfpathlineto{\pgfqpoint{0.803554in}{3.305147in}}%
\pgfpathlineto{\pgfqpoint{0.804627in}{3.305381in}}%
\pgfpathlineto{\pgfqpoint{0.809997in}{3.305456in}}%
\pgfpathlineto{\pgfqpoint{0.811070in}{3.304359in}}%
\pgfpathlineto{\pgfqpoint{0.812144in}{3.299891in}}%
\pgfpathlineto{\pgfqpoint{0.818587in}{3.299887in}}%
\pgfpathlineto{\pgfqpoint{0.819661in}{3.298522in}}%
\pgfpathlineto{\pgfqpoint{0.823956in}{3.297398in}}%
\pgfpathlineto{\pgfqpoint{0.825030in}{3.298846in}}%
\pgfpathlineto{\pgfqpoint{0.826104in}{3.298564in}}%
\pgfpathlineto{\pgfqpoint{0.827178in}{3.296665in}}%
\pgfpathlineto{\pgfqpoint{0.830399in}{3.298301in}}%
\pgfpathlineto{\pgfqpoint{0.831473in}{3.297040in}}%
\pgfpathlineto{\pgfqpoint{0.832547in}{3.297589in}}%
\pgfpathlineto{\pgfqpoint{0.834695in}{3.297312in}}%
\pgfpathlineto{\pgfqpoint{0.837916in}{3.297519in}}%
\pgfpathlineto{\pgfqpoint{0.838990in}{3.296827in}}%
\pgfpathlineto{\pgfqpoint{0.840064in}{3.292519in}}%
\pgfpathlineto{\pgfqpoint{0.841138in}{3.292455in}}%
\pgfpathlineto{\pgfqpoint{0.842212in}{3.293028in}}%
\pgfpathlineto{\pgfqpoint{0.846507in}{3.290716in}}%
\pgfpathlineto{\pgfqpoint{0.847581in}{3.292382in}}%
\pgfpathlineto{\pgfqpoint{0.849729in}{3.291495in}}%
\pgfpathlineto{\pgfqpoint{0.852950in}{3.294374in}}%
\pgfpathlineto{\pgfqpoint{0.855098in}{3.291513in}}%
\pgfpathlineto{\pgfqpoint{0.857245in}{3.292392in}}%
\pgfpathlineto{\pgfqpoint{0.860467in}{3.292010in}}%
\pgfpathlineto{\pgfqpoint{0.861541in}{3.290430in}}%
\pgfpathlineto{\pgfqpoint{0.864762in}{3.289512in}}%
\pgfpathlineto{\pgfqpoint{0.867984in}{3.290238in}}%
\pgfpathlineto{\pgfqpoint{0.869058in}{3.291280in}}%
\pgfpathlineto{\pgfqpoint{0.871205in}{3.289913in}}%
\pgfpathlineto{\pgfqpoint{0.872279in}{3.291072in}}%
\pgfpathlineto{\pgfqpoint{0.875501in}{3.292005in}}%
\pgfpathlineto{\pgfqpoint{0.877648in}{3.290997in}}%
\pgfpathlineto{\pgfqpoint{0.878722in}{3.291805in}}%
\pgfpathlineto{\pgfqpoint{0.879796in}{3.291426in}}%
\pgfpathlineto{\pgfqpoint{0.886239in}{3.292685in}}%
\pgfpathlineto{\pgfqpoint{0.890534in}{3.292877in}}%
\pgfpathlineto{\pgfqpoint{0.891608in}{3.291529in}}%
\pgfpathlineto{\pgfqpoint{0.892682in}{3.293223in}}%
\pgfpathlineto{\pgfqpoint{0.893756in}{3.292835in}}%
\pgfpathlineto{\pgfqpoint{0.894830in}{3.293541in}}%
\pgfpathlineto{\pgfqpoint{0.899125in}{3.292888in}}%
\pgfpathlineto{\pgfqpoint{0.900199in}{3.290254in}}%
\pgfpathlineto{\pgfqpoint{0.901273in}{3.290254in}}%
\pgfpathlineto{\pgfqpoint{0.902347in}{3.290868in}}%
\pgfpathlineto{\pgfqpoint{0.905568in}{3.294033in}}%
\pgfpathlineto{\pgfqpoint{0.906642in}{3.292524in}}%
\pgfpathlineto{\pgfqpoint{0.907716in}{3.294251in}}%
\pgfpathlineto{\pgfqpoint{0.908790in}{3.294712in}}%
\pgfpathlineto{\pgfqpoint{0.909864in}{3.299492in}}%
\pgfpathlineto{\pgfqpoint{0.913085in}{3.301795in}}%
\pgfpathlineto{\pgfqpoint{0.914159in}{3.300898in}}%
\pgfpathlineto{\pgfqpoint{0.915233in}{3.298246in}}%
\pgfpathlineto{\pgfqpoint{0.916307in}{3.298246in}}%
\pgfpathlineto{\pgfqpoint{0.917380in}{3.296835in}}%
\pgfpathlineto{\pgfqpoint{0.921676in}{3.296421in}}%
\pgfpathlineto{\pgfqpoint{0.922750in}{3.297175in}}%
\pgfpathlineto{\pgfqpoint{0.924897in}{3.294906in}}%
\pgfpathlineto{\pgfqpoint{0.929193in}{3.294306in}}%
\pgfpathlineto{\pgfqpoint{0.930266in}{3.292587in}}%
\pgfpathlineto{\pgfqpoint{0.931340in}{3.293743in}}%
\pgfpathlineto{\pgfqpoint{0.932414in}{3.293153in}}%
\pgfpathlineto{\pgfqpoint{0.935636in}{3.293412in}}%
\pgfpathlineto{\pgfqpoint{0.939931in}{3.290342in}}%
\pgfpathlineto{\pgfqpoint{0.944226in}{3.289357in}}%
\pgfpathlineto{\pgfqpoint{0.945300in}{3.288386in}}%
\pgfpathlineto{\pgfqpoint{0.947448in}{3.289948in}}%
\pgfpathlineto{\pgfqpoint{0.953891in}{3.289641in}}%
\pgfpathlineto{\pgfqpoint{0.954965in}{3.290983in}}%
\pgfpathlineto{\pgfqpoint{0.959260in}{3.289299in}}%
\pgfpathlineto{\pgfqpoint{0.960334in}{3.290938in}}%
\pgfpathlineto{\pgfqpoint{0.961408in}{3.290625in}}%
\pgfpathlineto{\pgfqpoint{0.962482in}{3.288264in}}%
\pgfpathlineto{\pgfqpoint{0.965703in}{3.288681in}}%
\pgfpathlineto{\pgfqpoint{0.967851in}{3.290010in}}%
\pgfpathlineto{\pgfqpoint{0.968925in}{3.289149in}}%
\pgfpathlineto{\pgfqpoint{0.969998in}{3.289876in}}%
\pgfpathlineto{\pgfqpoint{0.973220in}{3.288648in}}%
\pgfpathlineto{\pgfqpoint{0.974294in}{3.286006in}}%
\pgfpathlineto{\pgfqpoint{0.976442in}{3.290147in}}%
\pgfpathlineto{\pgfqpoint{0.977515in}{3.289591in}}%
\pgfpathlineto{\pgfqpoint{0.980737in}{3.291670in}}%
\pgfpathlineto{\pgfqpoint{0.983958in}{3.289724in}}%
\pgfpathlineto{\pgfqpoint{0.985032in}{3.290338in}}%
\pgfpathlineto{\pgfqpoint{0.988254in}{3.289656in}}%
\pgfpathlineto{\pgfqpoint{0.989328in}{3.291738in}}%
\pgfpathlineto{\pgfqpoint{0.990401in}{3.291738in}}%
\pgfpathlineto{\pgfqpoint{0.992549in}{3.290086in}}%
\pgfpathlineto{\pgfqpoint{0.995771in}{3.289654in}}%
\pgfpathlineto{\pgfqpoint{0.996844in}{3.288183in}}%
\pgfpathlineto{\pgfqpoint{0.997918in}{3.288601in}}%
\pgfpathlineto{\pgfqpoint{0.998992in}{3.286676in}}%
\pgfpathlineto{\pgfqpoint{1.000066in}{3.287490in}}%
\pgfpathlineto{\pgfqpoint{1.003287in}{3.288139in}}%
\pgfpathlineto{\pgfqpoint{1.004361in}{3.289034in}}%
\pgfpathlineto{\pgfqpoint{1.006509in}{3.288128in}}%
\pgfpathlineto{\pgfqpoint{1.007583in}{3.282936in}}%
\pgfpathlineto{\pgfqpoint{1.010804in}{3.282338in}}%
\pgfpathlineto{\pgfqpoint{1.012952in}{3.283364in}}%
\pgfpathlineto{\pgfqpoint{1.015100in}{3.283254in}}%
\pgfpathlineto{\pgfqpoint{1.020469in}{3.284464in}}%
\pgfpathlineto{\pgfqpoint{1.022617in}{3.283016in}}%
\pgfpathlineto{\pgfqpoint{1.025838in}{3.283778in}}%
\pgfpathlineto{\pgfqpoint{1.026912in}{3.284718in}}%
\pgfpathlineto{\pgfqpoint{1.029060in}{3.284605in}}%
\pgfpathlineto{\pgfqpoint{1.030133in}{3.284998in}}%
\pgfpathlineto{\pgfqpoint{1.034429in}{3.285167in}}%
\pgfpathlineto{\pgfqpoint{1.036576in}{3.286077in}}%
\pgfpathlineto{\pgfqpoint{1.043020in}{3.287886in}}%
\pgfpathlineto{\pgfqpoint{1.044093in}{3.287112in}}%
\pgfpathlineto{\pgfqpoint{1.045167in}{3.287112in}}%
\pgfpathlineto{\pgfqpoint{1.048389in}{3.287875in}}%
\pgfpathlineto{\pgfqpoint{1.049463in}{3.289901in}}%
\pgfpathlineto{\pgfqpoint{1.052684in}{3.290085in}}%
\pgfpathlineto{\pgfqpoint{1.056979in}{3.290518in}}%
\pgfpathlineto{\pgfqpoint{1.059127in}{3.290018in}}%
\pgfpathlineto{\pgfqpoint{1.060201in}{3.290018in}}%
\pgfpathlineto{\pgfqpoint{1.063422in}{3.288904in}}%
\pgfpathlineto{\pgfqpoint{1.065570in}{3.285548in}}%
\pgfpathlineto{\pgfqpoint{1.067718in}{3.286984in}}%
\pgfpathlineto{\pgfqpoint{1.070939in}{3.285282in}}%
\pgfpathlineto{\pgfqpoint{1.073087in}{3.282085in}}%
\pgfpathlineto{\pgfqpoint{1.074161in}{3.282997in}}%
\pgfpathlineto{\pgfqpoint{1.075235in}{3.284961in}}%
\pgfpathlineto{\pgfqpoint{1.079530in}{3.283550in}}%
\pgfpathlineto{\pgfqpoint{1.085973in}{3.285446in}}%
\pgfpathlineto{\pgfqpoint{1.088121in}{3.283462in}}%
\pgfpathlineto{\pgfqpoint{1.089195in}{3.284509in}}%
\pgfpathlineto{\pgfqpoint{1.093490in}{3.285185in}}%
\pgfpathlineto{\pgfqpoint{1.095638in}{3.286902in}}%
\pgfpathlineto{\pgfqpoint{1.096711in}{3.283732in}}%
\pgfpathlineto{\pgfqpoint{1.097785in}{3.283067in}}%
\pgfpathlineto{\pgfqpoint{1.101007in}{3.283122in}}%
\pgfpathlineto{\pgfqpoint{1.102081in}{3.283999in}}%
\pgfpathlineto{\pgfqpoint{1.104228in}{3.281791in}}%
\pgfpathlineto{\pgfqpoint{1.105302in}{3.278102in}}%
\pgfpathlineto{\pgfqpoint{1.108524in}{3.275760in}}%
\pgfpathlineto{\pgfqpoint{1.109597in}{3.276188in}}%
\pgfpathlineto{\pgfqpoint{1.110671in}{3.277436in}}%
\pgfpathlineto{\pgfqpoint{1.111745in}{3.276501in}}%
\pgfpathlineto{\pgfqpoint{1.112819in}{3.276743in}}%
\pgfpathlineto{\pgfqpoint{1.118188in}{3.274872in}}%
\pgfpathlineto{\pgfqpoint{1.119262in}{3.275714in}}%
\pgfpathlineto{\pgfqpoint{1.120336in}{3.274191in}}%
\pgfpathlineto{\pgfqpoint{1.124631in}{3.274653in}}%
\pgfpathlineto{\pgfqpoint{1.125705in}{3.272837in}}%
\pgfpathlineto{\pgfqpoint{1.126779in}{3.273914in}}%
\pgfpathlineto{\pgfqpoint{1.127853in}{3.272261in}}%
\pgfpathlineto{\pgfqpoint{1.132148in}{3.272261in}}%
\pgfpathlineto{\pgfqpoint{1.133222in}{3.271906in}}%
\pgfpathlineto{\pgfqpoint{1.134296in}{3.272479in}}%
\pgfpathlineto{\pgfqpoint{1.135370in}{3.271721in}}%
\pgfpathlineto{\pgfqpoint{1.138591in}{3.271677in}}%
\pgfpathlineto{\pgfqpoint{1.140739in}{3.272557in}}%
\pgfpathlineto{\pgfqpoint{1.141813in}{3.272735in}}%
\pgfpathlineto{\pgfqpoint{1.142886in}{3.272153in}}%
\pgfpathlineto{\pgfqpoint{1.147182in}{3.273211in}}%
\pgfpathlineto{\pgfqpoint{1.148256in}{3.273030in}}%
\pgfpathlineto{\pgfqpoint{1.150403in}{3.273616in}}%
\pgfpathlineto{\pgfqpoint{1.155773in}{3.273705in}}%
\pgfpathlineto{\pgfqpoint{1.156846in}{3.274391in}}%
\pgfpathlineto{\pgfqpoint{1.157920in}{3.272582in}}%
\pgfpathlineto{\pgfqpoint{1.162216in}{3.271864in}}%
\pgfpathlineto{\pgfqpoint{1.163289in}{3.273846in}}%
\pgfpathlineto{\pgfqpoint{1.164363in}{3.273983in}}%
\pgfpathlineto{\pgfqpoint{1.165437in}{3.272878in}}%
\pgfpathlineto{\pgfqpoint{1.169732in}{3.272205in}}%
\pgfpathlineto{\pgfqpoint{1.171880in}{3.270399in}}%
\pgfpathlineto{\pgfqpoint{1.177249in}{3.269420in}}%
\pgfpathlineto{\pgfqpoint{1.180471in}{3.268667in}}%
\pgfpathlineto{\pgfqpoint{1.184766in}{3.268010in}}%
\pgfpathlineto{\pgfqpoint{1.186914in}{3.269756in}}%
\pgfpathlineto{\pgfqpoint{1.187988in}{3.269798in}}%
\pgfpathlineto{\pgfqpoint{1.193357in}{3.270729in}}%
\pgfpathlineto{\pgfqpoint{1.194431in}{3.270686in}}%
\pgfpathlineto{\pgfqpoint{1.195505in}{3.271589in}}%
\pgfpathlineto{\pgfqpoint{1.198726in}{3.272422in}}%
\pgfpathlineto{\pgfqpoint{1.200874in}{3.270216in}}%
\pgfpathlineto{\pgfqpoint{1.201948in}{3.270727in}}%
\pgfpathlineto{\pgfqpoint{1.203021in}{3.272881in}}%
\pgfpathlineto{\pgfqpoint{1.207317in}{3.273918in}}%
\pgfpathlineto{\pgfqpoint{1.209464in}{3.275683in}}%
\pgfpathlineto{\pgfqpoint{1.210538in}{3.279027in}}%
\pgfpathlineto{\pgfqpoint{1.213760in}{3.278516in}}%
\pgfpathlineto{\pgfqpoint{1.214834in}{3.276948in}}%
\pgfpathlineto{\pgfqpoint{1.215908in}{3.277586in}}%
\pgfpathlineto{\pgfqpoint{1.216981in}{3.276791in}}%
\pgfpathlineto{\pgfqpoint{1.218055in}{3.278209in}}%
\pgfpathlineto{\pgfqpoint{1.221277in}{3.281379in}}%
\pgfpathlineto{\pgfqpoint{1.222351in}{3.280415in}}%
\pgfpathlineto{\pgfqpoint{1.223424in}{3.280625in}}%
\pgfpathlineto{\pgfqpoint{1.225572in}{3.277810in}}%
\pgfpathlineto{\pgfqpoint{1.228794in}{3.278410in}}%
\pgfpathlineto{\pgfqpoint{1.229867in}{3.276842in}}%
\pgfpathlineto{\pgfqpoint{1.232015in}{3.276792in}}%
\pgfpathlineto{\pgfqpoint{1.233089in}{3.275616in}}%
\pgfpathlineto{\pgfqpoint{1.237384in}{3.275951in}}%
\pgfpathlineto{\pgfqpoint{1.238458in}{3.276963in}}%
\pgfpathlineto{\pgfqpoint{1.240606in}{3.277901in}}%
\pgfpathlineto{\pgfqpoint{1.244901in}{3.276547in}}%
\pgfpathlineto{\pgfqpoint{1.245975in}{3.276449in}}%
\pgfpathlineto{\pgfqpoint{1.247049in}{3.275670in}}%
\pgfpathlineto{\pgfqpoint{1.248123in}{3.275766in}}%
\pgfpathlineto{\pgfqpoint{1.251344in}{3.277111in}}%
\pgfpathlineto{\pgfqpoint{1.253492in}{3.273947in}}%
\pgfpathlineto{\pgfqpoint{1.254566in}{3.272836in}}%
\pgfpathlineto{\pgfqpoint{1.255640in}{3.273153in}}%
\pgfpathlineto{\pgfqpoint{1.259935in}{3.274022in}}%
\pgfpathlineto{\pgfqpoint{1.261009in}{3.274533in}}%
\pgfpathlineto{\pgfqpoint{1.263156in}{3.277047in}}%
\pgfpathlineto{\pgfqpoint{1.267452in}{3.274898in}}%
\pgfpathlineto{\pgfqpoint{1.268526in}{3.276316in}}%
\pgfpathlineto{\pgfqpoint{1.270673in}{3.275878in}}%
\pgfpathlineto{\pgfqpoint{1.273895in}{3.275830in}}%
\pgfpathlineto{\pgfqpoint{1.274969in}{3.274673in}}%
\pgfpathlineto{\pgfqpoint{1.276042in}{3.275003in}}%
\pgfpathlineto{\pgfqpoint{1.278190in}{3.274011in}}%
\pgfpathlineto{\pgfqpoint{1.283559in}{3.273407in}}%
\pgfpathlineto{\pgfqpoint{1.284633in}{3.272673in}}%
\pgfpathlineto{\pgfqpoint{1.285707in}{3.273441in}}%
\pgfpathlineto{\pgfqpoint{1.290002in}{3.274456in}}%
\pgfpathlineto{\pgfqpoint{1.291076in}{3.273706in}}%
\pgfpathlineto{\pgfqpoint{1.293224in}{3.275661in}}%
\pgfpathlineto{\pgfqpoint{1.297519in}{3.274180in}}%
\pgfpathlineto{\pgfqpoint{1.299667in}{3.271420in}}%
\pgfpathlineto{\pgfqpoint{1.305036in}{3.271332in}}%
\pgfpathlineto{\pgfqpoint{1.306110in}{3.271904in}}%
\pgfpathlineto{\pgfqpoint{1.308258in}{3.271238in}}%
\pgfpathlineto{\pgfqpoint{1.313627in}{3.270233in}}%
\pgfpathlineto{\pgfqpoint{1.315774in}{3.270966in}}%
\pgfpathlineto{\pgfqpoint{1.318996in}{3.271315in}}%
\pgfpathlineto{\pgfqpoint{1.320070in}{3.272722in}}%
\pgfpathlineto{\pgfqpoint{1.321144in}{3.271454in}}%
\pgfpathlineto{\pgfqpoint{1.323291in}{3.271852in}}%
\pgfpathlineto{\pgfqpoint{1.327587in}{3.270036in}}%
\pgfpathlineto{\pgfqpoint{1.329734in}{3.271156in}}%
\pgfpathlineto{\pgfqpoint{1.330808in}{3.270937in}}%
\pgfpathlineto{\pgfqpoint{1.334030in}{3.271155in}}%
\pgfpathlineto{\pgfqpoint{1.335104in}{3.272252in}}%
\pgfpathlineto{\pgfqpoint{1.336177in}{3.271534in}}%
\pgfpathlineto{\pgfqpoint{1.344768in}{3.270655in}}%
\pgfpathlineto{\pgfqpoint{1.349063in}{3.270352in}}%
\pgfpathlineto{\pgfqpoint{1.350137in}{3.270481in}}%
\pgfpathlineto{\pgfqpoint{1.351211in}{3.271303in}}%
\pgfpathlineto{\pgfqpoint{1.352285in}{3.270554in}}%
\pgfpathlineto{\pgfqpoint{1.353359in}{3.269082in}}%
\pgfpathlineto{\pgfqpoint{1.357654in}{3.268324in}}%
\pgfpathlineto{\pgfqpoint{1.359802in}{3.269822in}}%
\pgfpathlineto{\pgfqpoint{1.360876in}{3.269566in}}%
\pgfpathlineto{\pgfqpoint{1.365171in}{3.270115in}}%
\pgfpathlineto{\pgfqpoint{1.366245in}{3.270072in}}%
\pgfpathlineto{\pgfqpoint{1.367319in}{3.269172in}}%
\pgfpathlineto{\pgfqpoint{1.368393in}{3.269003in}}%
\pgfpathlineto{\pgfqpoint{1.375909in}{3.270864in}}%
\pgfpathlineto{\pgfqpoint{1.379131in}{3.271432in}}%
\pgfpathlineto{\pgfqpoint{1.381279in}{3.270246in}}%
\pgfpathlineto{\pgfqpoint{1.382352in}{3.269341in}}%
\pgfpathlineto{\pgfqpoint{1.386648in}{3.269722in}}%
\pgfpathlineto{\pgfqpoint{1.387722in}{3.271042in}}%
\pgfpathlineto{\pgfqpoint{1.389869in}{3.271657in}}%
\pgfpathlineto{\pgfqpoint{1.390943in}{3.270414in}}%
\pgfpathlineto{\pgfqpoint{1.395239in}{3.270197in}}%
\pgfpathlineto{\pgfqpoint{1.396312in}{3.268733in}}%
\pgfpathlineto{\pgfqpoint{1.398460in}{3.270668in}}%
\pgfpathlineto{\pgfqpoint{1.401682in}{3.272365in}}%
\pgfpathlineto{\pgfqpoint{1.403829in}{3.272634in}}%
\pgfpathlineto{\pgfqpoint{1.404903in}{3.272544in}}%
\pgfpathlineto{\pgfqpoint{1.405977in}{3.273087in}}%
\pgfpathlineto{\pgfqpoint{1.410272in}{3.274377in}}%
\pgfpathlineto{\pgfqpoint{1.411346in}{3.273625in}}%
\pgfpathlineto{\pgfqpoint{1.412420in}{3.275477in}}%
\pgfpathlineto{\pgfqpoint{1.413494in}{3.274708in}}%
\pgfpathlineto{\pgfqpoint{1.416715in}{3.274992in}}%
\pgfpathlineto{\pgfqpoint{1.417789in}{3.275992in}}%
\pgfpathlineto{\pgfqpoint{1.418863in}{3.274291in}}%
\pgfpathlineto{\pgfqpoint{1.427454in}{3.272609in}}%
\pgfpathlineto{\pgfqpoint{1.428528in}{3.273516in}}%
\pgfpathlineto{\pgfqpoint{1.431749in}{3.271897in}}%
\pgfpathlineto{\pgfqpoint{1.432823in}{3.271987in}}%
\pgfpathlineto{\pgfqpoint{1.433897in}{3.270822in}}%
\pgfpathlineto{\pgfqpoint{1.434971in}{3.270559in}}%
\pgfpathlineto{\pgfqpoint{1.436044in}{3.271560in}}%
\pgfpathlineto{\pgfqpoint{1.439266in}{3.271382in}}%
\pgfpathlineto{\pgfqpoint{1.440340in}{3.272090in}}%
\pgfpathlineto{\pgfqpoint{1.441414in}{3.271641in}}%
\pgfpathlineto{\pgfqpoint{1.443561in}{3.272176in}}%
\pgfpathlineto{\pgfqpoint{1.448930in}{3.272307in}}%
\pgfpathlineto{\pgfqpoint{1.457521in}{3.272028in}}%
\pgfpathlineto{\pgfqpoint{1.458595in}{3.272567in}}%
\pgfpathlineto{\pgfqpoint{1.461817in}{3.272385in}}%
\pgfpathlineto{\pgfqpoint{1.462890in}{3.271300in}}%
\pgfpathlineto{\pgfqpoint{1.466112in}{3.270245in}}%
\pgfpathlineto{\pgfqpoint{1.471481in}{3.271509in}}%
\pgfpathlineto{\pgfqpoint{1.472555in}{3.272930in}}%
\pgfpathlineto{\pgfqpoint{1.473629in}{3.272610in}}%
\pgfpathlineto{\pgfqpoint{1.476850in}{3.273428in}}%
\pgfpathlineto{\pgfqpoint{1.477924in}{3.272642in}}%
\pgfpathlineto{\pgfqpoint{1.478998in}{3.274371in}}%
\pgfpathlineto{\pgfqpoint{1.480072in}{3.274512in}}%
\pgfpathlineto{\pgfqpoint{1.481146in}{3.273424in}}%
\pgfpathlineto{\pgfqpoint{1.484367in}{3.274118in}}%
\pgfpathlineto{\pgfqpoint{1.485441in}{3.275713in}}%
\pgfpathlineto{\pgfqpoint{1.486515in}{3.273968in}}%
\pgfpathlineto{\pgfqpoint{1.488662in}{3.277786in}}%
\pgfpathlineto{\pgfqpoint{1.491884in}{3.279151in}}%
\pgfpathlineto{\pgfqpoint{1.494032in}{3.277758in}}%
\pgfpathlineto{\pgfqpoint{1.495106in}{3.277859in}}%
\pgfpathlineto{\pgfqpoint{1.496179in}{3.275478in}}%
\pgfpathlineto{\pgfqpoint{1.499401in}{3.274656in}}%
\pgfpathlineto{\pgfqpoint{1.500475in}{3.272992in}}%
\pgfpathlineto{\pgfqpoint{1.501549in}{3.273957in}}%
\pgfpathlineto{\pgfqpoint{1.503696in}{3.272237in}}%
\pgfpathlineto{\pgfqpoint{1.506918in}{3.272690in}}%
\pgfpathlineto{\pgfqpoint{1.507992in}{3.271365in}}%
\pgfpathlineto{\pgfqpoint{1.509065in}{3.272165in}}%
\pgfpathlineto{\pgfqpoint{1.516582in}{3.271529in}}%
\pgfpathlineto{\pgfqpoint{1.517656in}{3.269567in}}%
\pgfpathlineto{\pgfqpoint{1.521951in}{3.269182in}}%
\pgfpathlineto{\pgfqpoint{1.523025in}{3.269479in}}%
\pgfpathlineto{\pgfqpoint{1.524099in}{3.269009in}}%
\pgfpathlineto{\pgfqpoint{1.526247in}{3.269220in}}%
\pgfpathlineto{\pgfqpoint{1.529468in}{3.268668in}}%
\pgfpathlineto{\pgfqpoint{1.530542in}{3.267282in}}%
\pgfpathlineto{\pgfqpoint{1.532690in}{3.267854in}}%
\pgfpathlineto{\pgfqpoint{1.533764in}{3.267359in}}%
\pgfpathlineto{\pgfqpoint{1.539133in}{3.267766in}}%
\pgfpathlineto{\pgfqpoint{1.544502in}{3.270694in}}%
\pgfpathlineto{\pgfqpoint{1.545576in}{3.270607in}}%
\pgfpathlineto{\pgfqpoint{1.546650in}{3.269380in}}%
\pgfpathlineto{\pgfqpoint{1.548797in}{3.270710in}}%
\pgfpathlineto{\pgfqpoint{1.553093in}{3.272256in}}%
\pgfpathlineto{\pgfqpoint{1.554167in}{3.273436in}}%
\pgfpathlineto{\pgfqpoint{1.555240in}{3.272878in}}%
\pgfpathlineto{\pgfqpoint{1.556314in}{3.274855in}}%
\pgfpathlineto{\pgfqpoint{1.560610in}{3.276445in}}%
\pgfpathlineto{\pgfqpoint{1.561684in}{3.275752in}}%
\pgfpathlineto{\pgfqpoint{1.563831in}{3.271033in}}%
\pgfpathlineto{\pgfqpoint{1.571348in}{3.270454in}}%
\pgfpathlineto{\pgfqpoint{1.575643in}{3.271200in}}%
\pgfpathlineto{\pgfqpoint{1.576717in}{3.272312in}}%
\pgfpathlineto{\pgfqpoint{1.582086in}{3.274892in}}%
\pgfpathlineto{\pgfqpoint{1.583160in}{3.277005in}}%
\pgfpathlineto{\pgfqpoint{1.584234in}{3.276955in}}%
\pgfpathlineto{\pgfqpoint{1.585308in}{3.275753in}}%
\pgfpathlineto{\pgfqpoint{1.586382in}{3.277171in}}%
\pgfpathlineto{\pgfqpoint{1.590677in}{3.277878in}}%
\pgfpathlineto{\pgfqpoint{1.593899in}{3.279004in}}%
\pgfpathlineto{\pgfqpoint{1.599268in}{3.277092in}}%
\pgfpathlineto{\pgfqpoint{1.601416in}{3.275249in}}%
\pgfpathlineto{\pgfqpoint{1.604637in}{3.274813in}}%
\pgfpathlineto{\pgfqpoint{1.605711in}{3.275629in}}%
\pgfpathlineto{\pgfqpoint{1.606785in}{3.277826in}}%
\pgfpathlineto{\pgfqpoint{1.607859in}{3.276805in}}%
\pgfpathlineto{\pgfqpoint{1.608932in}{3.277606in}}%
\pgfpathlineto{\pgfqpoint{1.612154in}{3.276284in}}%
\pgfpathlineto{\pgfqpoint{1.613228in}{3.275195in}}%
\pgfpathlineto{\pgfqpoint{1.614302in}{3.276455in}}%
\pgfpathlineto{\pgfqpoint{1.615375in}{3.275014in}}%
\pgfpathlineto{\pgfqpoint{1.619671in}{3.274483in}}%
\pgfpathlineto{\pgfqpoint{1.622892in}{3.273486in}}%
\pgfpathlineto{\pgfqpoint{1.623966in}{3.272457in}}%
\pgfpathlineto{\pgfqpoint{1.630409in}{3.272136in}}%
\pgfpathlineto{\pgfqpoint{1.631483in}{3.271362in}}%
\pgfpathlineto{\pgfqpoint{1.634705in}{3.271497in}}%
\pgfpathlineto{\pgfqpoint{1.635778in}{3.270688in}}%
\pgfpathlineto{\pgfqpoint{1.636852in}{3.268746in}}%
\pgfpathlineto{\pgfqpoint{1.639000in}{3.268448in}}%
\pgfpathlineto{\pgfqpoint{1.642221in}{3.268028in}}%
\pgfpathlineto{\pgfqpoint{1.644369in}{3.269626in}}%
\pgfpathlineto{\pgfqpoint{1.645443in}{3.269065in}}%
\pgfpathlineto{\pgfqpoint{1.646517in}{3.270473in}}%
\pgfpathlineto{\pgfqpoint{1.649738in}{3.269681in}}%
\pgfpathlineto{\pgfqpoint{1.650812in}{3.271368in}}%
\pgfpathlineto{\pgfqpoint{1.651886in}{3.271323in}}%
\pgfpathlineto{\pgfqpoint{1.652960in}{3.270517in}}%
\pgfpathlineto{\pgfqpoint{1.654034in}{3.271838in}}%
\pgfpathlineto{\pgfqpoint{1.657255in}{3.270298in}}%
\pgfpathlineto{\pgfqpoint{1.658329in}{3.270824in}}%
\pgfpathlineto{\pgfqpoint{1.659403in}{3.269582in}}%
\pgfpathlineto{\pgfqpoint{1.660477in}{3.270705in}}%
\pgfpathlineto{\pgfqpoint{1.664772in}{3.270176in}}%
\pgfpathlineto{\pgfqpoint{1.667994in}{3.272665in}}%
\pgfpathlineto{\pgfqpoint{1.672289in}{3.271471in}}%
\pgfpathlineto{\pgfqpoint{1.673363in}{3.272641in}}%
\pgfpathlineto{\pgfqpoint{1.675510in}{3.272134in}}%
\pgfpathlineto{\pgfqpoint{1.679806in}{3.271176in}}%
\pgfpathlineto{\pgfqpoint{1.681953in}{3.271847in}}%
\pgfpathlineto{\pgfqpoint{1.683027in}{3.269080in}}%
\pgfpathlineto{\pgfqpoint{1.684101in}{3.259104in}}%
\pgfpathlineto{\pgfqpoint{1.687323in}{3.261656in}}%
\pgfpathlineto{\pgfqpoint{1.688396in}{3.261326in}}%
\pgfpathlineto{\pgfqpoint{1.690544in}{3.262720in}}%
\pgfpathlineto{\pgfqpoint{1.694839in}{3.263549in}}%
\pgfpathlineto{\pgfqpoint{1.695913in}{3.264814in}}%
\pgfpathlineto{\pgfqpoint{1.696987in}{3.263867in}}%
\pgfpathlineto{\pgfqpoint{1.699135in}{3.264215in}}%
\pgfpathlineto{\pgfqpoint{1.702356in}{3.263982in}}%
\pgfpathlineto{\pgfqpoint{1.703430in}{3.263168in}}%
\pgfpathlineto{\pgfqpoint{1.705578in}{3.263320in}}%
\pgfpathlineto{\pgfqpoint{1.706652in}{3.262557in}}%
\pgfpathlineto{\pgfqpoint{1.709873in}{3.262707in}}%
\pgfpathlineto{\pgfqpoint{1.712021in}{3.264146in}}%
\pgfpathlineto{\pgfqpoint{1.714169in}{3.262374in}}%
\pgfpathlineto{\pgfqpoint{1.717390in}{3.263756in}}%
\pgfpathlineto{\pgfqpoint{1.721685in}{3.262639in}}%
\pgfpathlineto{\pgfqpoint{1.725981in}{3.262376in}}%
\pgfpathlineto{\pgfqpoint{1.728128in}{3.261223in}}%
\pgfpathlineto{\pgfqpoint{1.741015in}{3.262422in}}%
\pgfpathlineto{\pgfqpoint{1.742088in}{3.261786in}}%
\pgfpathlineto{\pgfqpoint{1.743162in}{3.262634in}}%
\pgfpathlineto{\pgfqpoint{1.751753in}{3.262207in}}%
\pgfpathlineto{\pgfqpoint{1.757122in}{3.262578in}}%
\pgfpathlineto{\pgfqpoint{1.758196in}{3.261526in}}%
\pgfpathlineto{\pgfqpoint{1.759270in}{3.261929in}}%
\pgfpathlineto{\pgfqpoint{1.763565in}{3.260972in}}%
\pgfpathlineto{\pgfqpoint{1.765713in}{3.262542in}}%
\pgfpathlineto{\pgfqpoint{1.766787in}{3.262392in}}%
\pgfpathlineto{\pgfqpoint{1.771082in}{3.264046in}}%
\pgfpathlineto{\pgfqpoint{1.773230in}{3.263349in}}%
\pgfpathlineto{\pgfqpoint{1.777525in}{3.264880in}}%
\pgfpathlineto{\pgfqpoint{1.778599in}{3.264325in}}%
\pgfpathlineto{\pgfqpoint{1.779673in}{3.266243in}}%
\pgfpathlineto{\pgfqpoint{1.789337in}{3.261777in}}%
\pgfpathlineto{\pgfqpoint{1.792559in}{3.262073in}}%
\pgfpathlineto{\pgfqpoint{1.795780in}{3.264923in}}%
\pgfpathlineto{\pgfqpoint{1.796854in}{3.266632in}}%
\pgfpathlineto{\pgfqpoint{1.801149in}{3.265401in}}%
\pgfpathlineto{\pgfqpoint{1.802223in}{3.264879in}}%
\pgfpathlineto{\pgfqpoint{1.804371in}{3.265396in}}%
\pgfpathlineto{\pgfqpoint{1.808666in}{3.266076in}}%
\pgfpathlineto{\pgfqpoint{1.809740in}{3.265383in}}%
\pgfpathlineto{\pgfqpoint{1.810814in}{3.265624in}}%
\pgfpathlineto{\pgfqpoint{1.811888in}{3.266472in}}%
\pgfpathlineto{\pgfqpoint{1.815109in}{3.264908in}}%
\pgfpathlineto{\pgfqpoint{1.816183in}{3.266697in}}%
\pgfpathlineto{\pgfqpoint{1.817257in}{3.266159in}}%
\pgfpathlineto{\pgfqpoint{1.818331in}{3.266404in}}%
\pgfpathlineto{\pgfqpoint{1.819405in}{3.265376in}}%
\pgfpathlineto{\pgfqpoint{1.822626in}{3.264933in}}%
\pgfpathlineto{\pgfqpoint{1.823700in}{3.265411in}}%
\pgfpathlineto{\pgfqpoint{1.825848in}{3.268479in}}%
\pgfpathlineto{\pgfqpoint{1.826922in}{3.270633in}}%
\pgfpathlineto{\pgfqpoint{1.830143in}{3.273433in}}%
\pgfpathlineto{\pgfqpoint{1.831217in}{3.275830in}}%
\pgfpathlineto{\pgfqpoint{1.834438in}{3.268122in}}%
\pgfpathlineto{\pgfqpoint{1.837660in}{3.269369in}}%
\pgfpathlineto{\pgfqpoint{1.838734in}{3.272901in}}%
\pgfpathlineto{\pgfqpoint{1.839808in}{3.270090in}}%
\pgfpathlineto{\pgfqpoint{1.840882in}{3.270314in}}%
\pgfpathlineto{\pgfqpoint{1.841955in}{3.272256in}}%
\pgfpathlineto{\pgfqpoint{1.846251in}{3.268444in}}%
\pgfpathlineto{\pgfqpoint{1.847325in}{3.269965in}}%
\pgfpathlineto{\pgfqpoint{1.849472in}{3.268448in}}%
\pgfpathlineto{\pgfqpoint{1.852694in}{3.269101in}}%
\pgfpathlineto{\pgfqpoint{1.854841in}{3.264876in}}%
\pgfpathlineto{\pgfqpoint{1.856989in}{3.268025in}}%
\pgfpathlineto{\pgfqpoint{1.862358in}{3.266778in}}%
\pgfpathlineto{\pgfqpoint{1.863432in}{3.267576in}}%
\pgfpathlineto{\pgfqpoint{1.864506in}{3.267576in}}%
\pgfpathlineto{\pgfqpoint{1.867727in}{3.269799in}}%
\pgfpathlineto{\pgfqpoint{1.868801in}{3.268811in}}%
\pgfpathlineto{\pgfqpoint{1.869875in}{3.266350in}}%
\pgfpathlineto{\pgfqpoint{1.870949in}{3.266475in}}%
\pgfpathlineto{\pgfqpoint{1.872023in}{3.265472in}}%
\pgfpathlineto{\pgfqpoint{1.877392in}{3.257809in}}%
\pgfpathlineto{\pgfqpoint{1.879540in}{3.256921in}}%
\pgfpathlineto{\pgfqpoint{1.882761in}{3.256854in}}%
\pgfpathlineto{\pgfqpoint{1.884909in}{3.258271in}}%
\pgfpathlineto{\pgfqpoint{1.885983in}{3.257024in}}%
\pgfpathlineto{\pgfqpoint{1.887057in}{3.254268in}}%
\pgfpathlineto{\pgfqpoint{1.890278in}{3.254237in}}%
\pgfpathlineto{\pgfqpoint{1.891352in}{3.254801in}}%
\pgfpathlineto{\pgfqpoint{1.892426in}{3.254610in}}%
\pgfpathlineto{\pgfqpoint{1.893500in}{3.252648in}}%
\pgfpathlineto{\pgfqpoint{1.894573in}{3.252828in}}%
\pgfpathlineto{\pgfqpoint{1.899943in}{3.253131in}}%
\pgfpathlineto{\pgfqpoint{1.901016in}{3.253283in}}%
\pgfpathlineto{\pgfqpoint{1.902090in}{3.254355in}}%
\pgfpathlineto{\pgfqpoint{1.907460in}{3.252697in}}%
\pgfpathlineto{\pgfqpoint{1.909607in}{3.251736in}}%
\pgfpathlineto{\pgfqpoint{1.912829in}{3.252148in}}%
\pgfpathlineto{\pgfqpoint{1.914976in}{3.249862in}}%
\pgfpathlineto{\pgfqpoint{1.916050in}{3.251066in}}%
\pgfpathlineto{\pgfqpoint{1.917124in}{3.250776in}}%
\pgfpathlineto{\pgfqpoint{1.928936in}{3.249848in}}%
\pgfpathlineto{\pgfqpoint{1.930010in}{3.250548in}}%
\pgfpathlineto{\pgfqpoint{1.932158in}{3.250548in}}%
\pgfpathlineto{\pgfqpoint{1.935379in}{3.251575in}}%
\pgfpathlineto{\pgfqpoint{1.936453in}{3.251018in}}%
\pgfpathlineto{\pgfqpoint{1.937527in}{3.251509in}}%
\pgfpathlineto{\pgfqpoint{1.938601in}{3.251362in}}%
\pgfpathlineto{\pgfqpoint{1.939675in}{3.250196in}}%
\pgfpathlineto{\pgfqpoint{1.945044in}{3.250242in}}%
\pgfpathlineto{\pgfqpoint{1.946118in}{3.249818in}}%
\pgfpathlineto{\pgfqpoint{1.947192in}{3.250741in}}%
\pgfpathlineto{\pgfqpoint{1.951487in}{3.250598in}}%
\pgfpathlineto{\pgfqpoint{1.952561in}{3.248968in}}%
\pgfpathlineto{\pgfqpoint{1.954708in}{3.250098in}}%
\pgfpathlineto{\pgfqpoint{1.959004in}{3.249592in}}%
\pgfpathlineto{\pgfqpoint{1.960078in}{3.248479in}}%
\pgfpathlineto{\pgfqpoint{1.961151in}{3.248749in}}%
\pgfpathlineto{\pgfqpoint{1.965447in}{3.248586in}}%
\pgfpathlineto{\pgfqpoint{1.966521in}{3.247719in}}%
\pgfpathlineto{\pgfqpoint{1.967594in}{3.248221in}}%
\pgfpathlineto{\pgfqpoint{1.968668in}{3.248007in}}%
\pgfpathlineto{\pgfqpoint{1.975111in}{3.250086in}}%
\pgfpathlineto{\pgfqpoint{1.976185in}{3.253220in}}%
\pgfpathlineto{\pgfqpoint{1.977259in}{3.254573in}}%
\pgfpathlineto{\pgfqpoint{1.981554in}{3.254065in}}%
\pgfpathlineto{\pgfqpoint{1.982628in}{3.255166in}}%
\pgfpathlineto{\pgfqpoint{1.983702in}{3.252869in}}%
\pgfpathlineto{\pgfqpoint{1.984776in}{3.254364in}}%
\pgfpathlineto{\pgfqpoint{1.989071in}{3.254364in}}%
\pgfpathlineto{\pgfqpoint{1.990145in}{3.255697in}}%
\pgfpathlineto{\pgfqpoint{1.991219in}{3.254022in}}%
\pgfpathlineto{\pgfqpoint{1.992293in}{3.254998in}}%
\pgfpathlineto{\pgfqpoint{1.995514in}{3.255547in}}%
\pgfpathlineto{\pgfqpoint{1.996588in}{3.254794in}}%
\pgfpathlineto{\pgfqpoint{1.997662in}{3.255629in}}%
\pgfpathlineto{\pgfqpoint{1.998736in}{3.255038in}}%
\pgfpathlineto{\pgfqpoint{1.999810in}{3.252550in}}%
\pgfpathlineto{\pgfqpoint{2.003031in}{3.253763in}}%
\pgfpathlineto{\pgfqpoint{2.004105in}{3.254859in}}%
\pgfpathlineto{\pgfqpoint{2.006253in}{3.252292in}}%
\pgfpathlineto{\pgfqpoint{2.007326in}{3.253951in}}%
\pgfpathlineto{\pgfqpoint{2.010548in}{3.254959in}}%
\pgfpathlineto{\pgfqpoint{2.012696in}{3.254571in}}%
\pgfpathlineto{\pgfqpoint{2.013770in}{3.256941in}}%
\pgfpathlineto{\pgfqpoint{2.014843in}{3.254559in}}%
\pgfpathlineto{\pgfqpoint{2.019139in}{3.252892in}}%
\pgfpathlineto{\pgfqpoint{2.020213in}{3.251632in}}%
\pgfpathlineto{\pgfqpoint{2.022360in}{3.252437in}}%
\pgfpathlineto{\pgfqpoint{2.025582in}{3.251435in}}%
\pgfpathlineto{\pgfqpoint{2.027729in}{3.252597in}}%
\pgfpathlineto{\pgfqpoint{2.028803in}{3.251255in}}%
\pgfpathlineto{\pgfqpoint{2.029877in}{3.250813in}}%
\pgfpathlineto{\pgfqpoint{2.033099in}{3.251482in}}%
\pgfpathlineto{\pgfqpoint{2.034172in}{3.249586in}}%
\pgfpathlineto{\pgfqpoint{2.036320in}{3.248743in}}%
\pgfpathlineto{\pgfqpoint{2.037394in}{3.248165in}}%
\pgfpathlineto{\pgfqpoint{2.043837in}{3.249398in}}%
\pgfpathlineto{\pgfqpoint{2.044911in}{3.248444in}}%
\pgfpathlineto{\pgfqpoint{2.050280in}{3.248827in}}%
\pgfpathlineto{\pgfqpoint{2.051354in}{3.246950in}}%
\pgfpathlineto{\pgfqpoint{2.052428in}{3.247029in}}%
\pgfpathlineto{\pgfqpoint{2.064240in}{3.245778in}}%
\pgfpathlineto{\pgfqpoint{2.065314in}{3.244992in}}%
\pgfpathlineto{\pgfqpoint{2.067461in}{3.244793in}}%
\pgfpathlineto{\pgfqpoint{2.073904in}{3.247625in}}%
\pgfpathlineto{\pgfqpoint{2.074978in}{3.247277in}}%
\pgfpathlineto{\pgfqpoint{2.079274in}{3.247223in}}%
\pgfpathlineto{\pgfqpoint{2.081421in}{3.246748in}}%
\pgfpathlineto{\pgfqpoint{2.088938in}{3.246823in}}%
\pgfpathlineto{\pgfqpoint{2.090012in}{3.247320in}}%
\pgfpathlineto{\pgfqpoint{2.093234in}{3.247506in}}%
\pgfpathlineto{\pgfqpoint{2.095381in}{3.246920in}}%
\pgfpathlineto{\pgfqpoint{2.101824in}{3.247601in}}%
\pgfpathlineto{\pgfqpoint{2.103972in}{3.249354in}}%
\pgfpathlineto{\pgfqpoint{2.105046in}{3.248793in}}%
\pgfpathlineto{\pgfqpoint{2.108267in}{3.249401in}}%
\pgfpathlineto{\pgfqpoint{2.109341in}{3.247911in}}%
\pgfpathlineto{\pgfqpoint{2.111489in}{3.248807in}}%
\pgfpathlineto{\pgfqpoint{2.112563in}{3.249886in}}%
\pgfpathlineto{\pgfqpoint{2.115784in}{3.249116in}}%
\pgfpathlineto{\pgfqpoint{2.119006in}{3.250586in}}%
\pgfpathlineto{\pgfqpoint{2.120080in}{3.250063in}}%
\pgfpathlineto{\pgfqpoint{2.123301in}{3.250264in}}%
\pgfpathlineto{\pgfqpoint{2.125449in}{3.248752in}}%
\pgfpathlineto{\pgfqpoint{2.127596in}{3.248695in}}%
\pgfpathlineto{\pgfqpoint{2.132966in}{3.248721in}}%
\pgfpathlineto{\pgfqpoint{2.135113in}{3.249108in}}%
\pgfpathlineto{\pgfqpoint{2.141556in}{3.248384in}}%
\pgfpathlineto{\pgfqpoint{2.142630in}{3.248877in}}%
\pgfpathlineto{\pgfqpoint{2.145852in}{3.249376in}}%
\pgfpathlineto{\pgfqpoint{2.146926in}{3.247885in}}%
\pgfpathlineto{\pgfqpoint{2.150147in}{3.246970in}}%
\pgfpathlineto{\pgfqpoint{2.156590in}{3.245609in}}%
\pgfpathlineto{\pgfqpoint{2.157664in}{3.248648in}}%
\pgfpathlineto{\pgfqpoint{2.160885in}{3.249864in}}%
\pgfpathlineto{\pgfqpoint{2.164107in}{3.244708in}}%
\pgfpathlineto{\pgfqpoint{2.170550in}{3.244136in}}%
\pgfpathlineto{\pgfqpoint{2.171624in}{3.243966in}}%
\pgfpathlineto{\pgfqpoint{2.172698in}{3.243166in}}%
\pgfpathlineto{\pgfqpoint{2.178067in}{3.242836in}}%
\pgfpathlineto{\pgfqpoint{2.180214in}{3.241766in}}%
\pgfpathlineto{\pgfqpoint{2.185584in}{3.241970in}}%
\pgfpathlineto{\pgfqpoint{2.186658in}{3.242358in}}%
\pgfpathlineto{\pgfqpoint{2.187731in}{3.243421in}}%
\pgfpathlineto{\pgfqpoint{2.195248in}{3.245393in}}%
\pgfpathlineto{\pgfqpoint{2.239276in}{3.245581in}}%
\pgfpathlineto{\pgfqpoint{2.240349in}{3.247667in}}%
\pgfpathlineto{\pgfqpoint{2.243571in}{3.246775in}}%
\pgfpathlineto{\pgfqpoint{2.244645in}{3.248251in}}%
\pgfpathlineto{\pgfqpoint{2.245719in}{3.248608in}}%
\pgfpathlineto{\pgfqpoint{2.246792in}{3.247914in}}%
\pgfpathlineto{\pgfqpoint{2.251088in}{3.248679in}}%
\pgfpathlineto{\pgfqpoint{2.254309in}{3.247197in}}%
\pgfpathlineto{\pgfqpoint{2.258605in}{3.248407in}}%
\pgfpathlineto{\pgfqpoint{2.259679in}{3.247579in}}%
\pgfpathlineto{\pgfqpoint{2.260752in}{3.247525in}}%
\pgfpathlineto{\pgfqpoint{2.261826in}{3.248387in}}%
\pgfpathlineto{\pgfqpoint{2.266122in}{3.248139in}}%
\pgfpathlineto{\pgfqpoint{2.273638in}{3.250068in}}%
\pgfpathlineto{\pgfqpoint{2.275786in}{3.249952in}}%
\pgfpathlineto{\pgfqpoint{2.276860in}{3.250269in}}%
\pgfpathlineto{\pgfqpoint{2.277934in}{3.249979in}}%
\pgfpathlineto{\pgfqpoint{2.282229in}{3.249747in}}%
\pgfpathlineto{\pgfqpoint{2.284377in}{3.249518in}}%
\pgfpathlineto{\pgfqpoint{2.289746in}{3.250580in}}%
\pgfpathlineto{\pgfqpoint{2.290820in}{3.250023in}}%
\pgfpathlineto{\pgfqpoint{2.291894in}{3.250629in}}%
\pgfpathlineto{\pgfqpoint{2.292968in}{3.249102in}}%
\pgfpathlineto{\pgfqpoint{2.297263in}{3.249952in}}%
\pgfpathlineto{\pgfqpoint{2.299411in}{3.251496in}}%
\pgfpathlineto{\pgfqpoint{2.300484in}{3.251075in}}%
\pgfpathlineto{\pgfqpoint{2.304780in}{3.248505in}}%
\pgfpathlineto{\pgfqpoint{2.305854in}{3.248005in}}%
\pgfpathlineto{\pgfqpoint{2.306927in}{3.246116in}}%
\pgfpathlineto{\pgfqpoint{2.308001in}{3.245415in}}%
\pgfpathlineto{\pgfqpoint{2.311223in}{3.245873in}}%
\pgfpathlineto{\pgfqpoint{2.313370in}{3.245357in}}%
\pgfpathlineto{\pgfqpoint{2.326257in}{3.244216in}}%
\pgfpathlineto{\pgfqpoint{2.328404in}{3.245281in}}%
\pgfpathlineto{\pgfqpoint{2.329478in}{3.243861in}}%
\pgfpathlineto{\pgfqpoint{2.334847in}{3.244347in}}%
\pgfpathlineto{\pgfqpoint{2.335921in}{3.243409in}}%
\pgfpathlineto{\pgfqpoint{2.336995in}{3.243553in}}%
\pgfpathlineto{\pgfqpoint{2.338069in}{3.243022in}}%
\pgfpathlineto{\pgfqpoint{2.342364in}{3.243116in}}%
\pgfpathlineto{\pgfqpoint{2.344512in}{3.244124in}}%
\pgfpathlineto{\pgfqpoint{2.345586in}{3.243068in}}%
\pgfpathlineto{\pgfqpoint{2.348807in}{3.242712in}}%
\pgfpathlineto{\pgfqpoint{2.349881in}{3.242029in}}%
\pgfpathlineto{\pgfqpoint{2.353102in}{3.242282in}}%
\pgfpathlineto{\pgfqpoint{2.359546in}{3.242630in}}%
\pgfpathlineto{\pgfqpoint{2.360619in}{3.242865in}}%
\pgfpathlineto{\pgfqpoint{2.368136in}{3.242838in}}%
\pgfpathlineto{\pgfqpoint{2.382096in}{3.243696in}}%
\pgfpathlineto{\pgfqpoint{2.383170in}{3.245150in}}%
\pgfpathlineto{\pgfqpoint{2.386391in}{3.246897in}}%
\pgfpathlineto{\pgfqpoint{2.389613in}{3.245574in}}%
\pgfpathlineto{\pgfqpoint{2.390687in}{3.246267in}}%
\pgfpathlineto{\pgfqpoint{2.393908in}{3.246397in}}%
\pgfpathlineto{\pgfqpoint{2.396056in}{3.247028in}}%
\pgfpathlineto{\pgfqpoint{2.405721in}{3.246942in}}%
\pgfpathlineto{\pgfqpoint{2.413237in}{3.245414in}}%
\pgfpathlineto{\pgfqpoint{2.418607in}{3.245512in}}%
\pgfpathlineto{\pgfqpoint{2.425050in}{3.246150in}}%
\pgfpathlineto{\pgfqpoint{2.426124in}{3.245263in}}%
\pgfpathlineto{\pgfqpoint{2.431493in}{3.245696in}}%
\pgfpathlineto{\pgfqpoint{2.434714in}{3.246475in}}%
\pgfpathlineto{\pgfqpoint{2.435788in}{3.245026in}}%
\pgfpathlineto{\pgfqpoint{2.441157in}{3.246183in}}%
\pgfpathlineto{\pgfqpoint{2.446526in}{3.246233in}}%
\pgfpathlineto{\pgfqpoint{2.447600in}{3.247045in}}%
\pgfpathlineto{\pgfqpoint{2.450822in}{3.246273in}}%
\pgfpathlineto{\pgfqpoint{2.454043in}{3.246929in}}%
\pgfpathlineto{\pgfqpoint{2.457265in}{3.245896in}}%
\pgfpathlineto{\pgfqpoint{2.458339in}{3.246078in}}%
\pgfpathlineto{\pgfqpoint{2.470151in}{3.245506in}}%
\pgfpathlineto{\pgfqpoint{2.472299in}{3.246618in}}%
\pgfpathlineto{\pgfqpoint{2.477668in}{3.245958in}}%
\pgfpathlineto{\pgfqpoint{2.479815in}{3.244977in}}%
\pgfpathlineto{\pgfqpoint{2.480889in}{3.246594in}}%
\pgfpathlineto{\pgfqpoint{2.485185in}{3.246832in}}%
\pgfpathlineto{\pgfqpoint{2.488406in}{3.247937in}}%
\pgfpathlineto{\pgfqpoint{2.492702in}{3.247936in}}%
\pgfpathlineto{\pgfqpoint{2.493775in}{3.247358in}}%
\pgfpathlineto{\pgfqpoint{2.500218in}{3.248093in}}%
\pgfpathlineto{\pgfqpoint{2.501292in}{3.248646in}}%
\pgfpathlineto{\pgfqpoint{2.502366in}{3.248225in}}%
\pgfpathlineto{\pgfqpoint{2.503440in}{3.249723in}}%
\pgfpathlineto{\pgfqpoint{2.507735in}{3.250304in}}%
\pgfpathlineto{\pgfqpoint{2.508809in}{3.251980in}}%
\pgfpathlineto{\pgfqpoint{2.509883in}{3.251795in}}%
\pgfpathlineto{\pgfqpoint{2.510957in}{3.250204in}}%
\pgfpathlineto{\pgfqpoint{2.515252in}{3.249619in}}%
\pgfpathlineto{\pgfqpoint{2.517400in}{3.251698in}}%
\pgfpathlineto{\pgfqpoint{2.523843in}{3.251973in}}%
\pgfpathlineto{\pgfqpoint{2.525991in}{3.250629in}}%
\pgfpathlineto{\pgfqpoint{2.530286in}{3.250480in}}%
\pgfpathlineto{\pgfqpoint{2.532434in}{3.251401in}}%
\pgfpathlineto{\pgfqpoint{2.536729in}{3.247801in}}%
\pgfpathlineto{\pgfqpoint{2.537803in}{3.249012in}}%
\pgfpathlineto{\pgfqpoint{2.539950in}{3.247154in}}%
\pgfpathlineto{\pgfqpoint{2.541024in}{3.246397in}}%
\pgfpathlineto{\pgfqpoint{2.544246in}{3.246874in}}%
\pgfpathlineto{\pgfqpoint{2.546393in}{3.249410in}}%
\pgfpathlineto{\pgfqpoint{2.547467in}{3.250015in}}%
\pgfpathlineto{\pgfqpoint{2.551763in}{3.249839in}}%
\pgfpathlineto{\pgfqpoint{2.553910in}{3.251279in}}%
\pgfpathlineto{\pgfqpoint{2.556058in}{3.251461in}}%
\pgfpathlineto{\pgfqpoint{2.559279in}{3.250242in}}%
\pgfpathlineto{\pgfqpoint{2.561427in}{3.250507in}}%
\pgfpathlineto{\pgfqpoint{2.562501in}{3.253330in}}%
\pgfpathlineto{\pgfqpoint{2.563575in}{3.253779in}}%
\pgfpathlineto{\pgfqpoint{2.566796in}{3.254103in}}%
\pgfpathlineto{\pgfqpoint{2.568944in}{3.252515in}}%
\pgfpathlineto{\pgfqpoint{2.571092in}{3.251950in}}%
\pgfpathlineto{\pgfqpoint{2.577535in}{3.252190in}}%
\pgfpathlineto{\pgfqpoint{2.578609in}{3.254401in}}%
\pgfpathlineto{\pgfqpoint{2.582904in}{3.255819in}}%
\pgfpathlineto{\pgfqpoint{2.585052in}{3.254731in}}%
\pgfpathlineto{\pgfqpoint{2.586125in}{3.255530in}}%
\pgfpathlineto{\pgfqpoint{2.589347in}{3.255292in}}%
\pgfpathlineto{\pgfqpoint{2.590421in}{3.255799in}}%
\pgfpathlineto{\pgfqpoint{2.593642in}{3.254744in}}%
\pgfpathlineto{\pgfqpoint{2.597938in}{3.255413in}}%
\pgfpathlineto{\pgfqpoint{2.599012in}{3.254938in}}%
\pgfpathlineto{\pgfqpoint{2.600085in}{3.256211in}}%
\pgfpathlineto{\pgfqpoint{2.601159in}{3.256522in}}%
\pgfpathlineto{\pgfqpoint{2.604381in}{3.255999in}}%
\pgfpathlineto{\pgfqpoint{2.608676in}{3.258602in}}%
\pgfpathlineto{\pgfqpoint{2.621562in}{3.259512in}}%
\pgfpathlineto{\pgfqpoint{2.622636in}{3.258574in}}%
\pgfpathlineto{\pgfqpoint{2.623710in}{3.256592in}}%
\pgfpathlineto{\pgfqpoint{2.628005in}{3.257851in}}%
\pgfpathlineto{\pgfqpoint{2.629079in}{3.257274in}}%
\pgfpathlineto{\pgfqpoint{2.630153in}{3.260191in}}%
\pgfpathlineto{\pgfqpoint{2.631227in}{3.260918in}}%
\pgfpathlineto{\pgfqpoint{2.634448in}{3.261268in}}%
\pgfpathlineto{\pgfqpoint{2.637670in}{3.259374in}}%
\pgfpathlineto{\pgfqpoint{2.638744in}{3.259674in}}%
\pgfpathlineto{\pgfqpoint{2.641965in}{3.257822in}}%
\pgfpathlineto{\pgfqpoint{2.643039in}{3.258690in}}%
\pgfpathlineto{\pgfqpoint{2.644113in}{3.258284in}}%
\pgfpathlineto{\pgfqpoint{2.645187in}{3.256821in}}%
\pgfpathlineto{\pgfqpoint{2.649482in}{3.255665in}}%
\pgfpathlineto{\pgfqpoint{2.650556in}{3.256247in}}%
\pgfpathlineto{\pgfqpoint{2.651630in}{3.258021in}}%
\pgfpathlineto{\pgfqpoint{2.653777in}{3.258678in}}%
\pgfpathlineto{\pgfqpoint{2.658073in}{3.256596in}}%
\pgfpathlineto{\pgfqpoint{2.659146in}{3.257615in}}%
\pgfpathlineto{\pgfqpoint{2.660220in}{3.257435in}}%
\pgfpathlineto{\pgfqpoint{2.661294in}{3.257937in}}%
\pgfpathlineto{\pgfqpoint{2.664516in}{3.261131in}}%
\pgfpathlineto{\pgfqpoint{2.665590in}{3.260896in}}%
\pgfpathlineto{\pgfqpoint{2.666663in}{3.259835in}}%
\pgfpathlineto{\pgfqpoint{2.668811in}{3.259521in}}%
\pgfpathlineto{\pgfqpoint{2.672033in}{3.260896in}}%
\pgfpathlineto{\pgfqpoint{2.675254in}{3.258383in}}%
\pgfpathlineto{\pgfqpoint{2.676328in}{3.257514in}}%
\pgfpathlineto{\pgfqpoint{2.681697in}{3.249606in}}%
\pgfpathlineto{\pgfqpoint{2.682771in}{3.248978in}}%
\pgfpathlineto{\pgfqpoint{2.683845in}{3.247167in}}%
\pgfpathlineto{\pgfqpoint{2.690288in}{3.244285in}}%
\pgfpathlineto{\pgfqpoint{2.691362in}{3.244950in}}%
\pgfpathlineto{\pgfqpoint{2.697805in}{3.244470in}}%
\pgfpathlineto{\pgfqpoint{2.698879in}{3.246169in}}%
\pgfpathlineto{\pgfqpoint{2.702100in}{3.241144in}}%
\pgfpathlineto{\pgfqpoint{2.703174in}{3.237337in}}%
\pgfpathlineto{\pgfqpoint{2.704248in}{3.238557in}}%
\pgfpathlineto{\pgfqpoint{2.706395in}{3.238409in}}%
\pgfpathlineto{\pgfqpoint{2.710691in}{3.237116in}}%
\pgfpathlineto{\pgfqpoint{2.711765in}{3.238187in}}%
\pgfpathlineto{\pgfqpoint{2.718208in}{3.239074in}}%
\pgfpathlineto{\pgfqpoint{2.719281in}{3.239333in}}%
\pgfpathlineto{\pgfqpoint{2.720355in}{3.238668in}}%
\pgfpathlineto{\pgfqpoint{2.721429in}{3.237263in}}%
\pgfpathlineto{\pgfqpoint{2.724651in}{3.237522in}}%
\pgfpathlineto{\pgfqpoint{2.725724in}{3.236857in}}%
\pgfpathlineto{\pgfqpoint{2.728946in}{3.237116in}}%
\pgfpathlineto{\pgfqpoint{2.732167in}{3.236931in}}%
\pgfpathlineto{\pgfqpoint{2.733241in}{3.237818in}}%
\pgfpathlineto{\pgfqpoint{2.735389in}{3.236894in}}%
\pgfpathlineto{\pgfqpoint{2.736463in}{3.237485in}}%
\pgfpathlineto{\pgfqpoint{2.740758in}{3.236709in}}%
\pgfpathlineto{\pgfqpoint{2.742906in}{3.236303in}}%
\pgfpathlineto{\pgfqpoint{2.749349in}{3.236413in}}%
\pgfpathlineto{\pgfqpoint{2.751497in}{3.236635in}}%
\pgfpathlineto{\pgfqpoint{2.751497in}{3.236635in}}%
\pgfusepath{stroke}%
\end{pgfscope}%
\begin{pgfscope}%
\pgfsetrectcap%
\pgfsetmiterjoin%
\pgfsetlinewidth{0.803000pt}%
\definecolor{currentstroke}{rgb}{1.000000,1.000000,1.000000}%
\pgfsetstrokecolor{currentstroke}%
\pgfsetdash{}{0pt}%
\pgfpathmoveto{\pgfqpoint{0.285588in}{3.218007in}}%
\pgfpathlineto{\pgfqpoint{0.285588in}{3.618892in}}%
\pgfusepath{stroke}%
\end{pgfscope}%
\begin{pgfscope}%
\pgfsetrectcap%
\pgfsetmiterjoin%
\pgfsetlinewidth{0.803000pt}%
\definecolor{currentstroke}{rgb}{1.000000,1.000000,1.000000}%
\pgfsetstrokecolor{currentstroke}%
\pgfsetdash{}{0pt}%
\pgfpathmoveto{\pgfqpoint{2.868921in}{3.218007in}}%
\pgfpathlineto{\pgfqpoint{2.868921in}{3.618892in}}%
\pgfusepath{stroke}%
\end{pgfscope}%
\begin{pgfscope}%
\pgfsetrectcap%
\pgfsetmiterjoin%
\pgfsetlinewidth{0.803000pt}%
\definecolor{currentstroke}{rgb}{1.000000,1.000000,1.000000}%
\pgfsetstrokecolor{currentstroke}%
\pgfsetdash{}{0pt}%
\pgfpathmoveto{\pgfqpoint{0.285587in}{3.218007in}}%
\pgfpathlineto{\pgfqpoint{2.868921in}{3.218007in}}%
\pgfusepath{stroke}%
\end{pgfscope}%
\begin{pgfscope}%
\pgfsetrectcap%
\pgfsetmiterjoin%
\pgfsetlinewidth{0.803000pt}%
\definecolor{currentstroke}{rgb}{1.000000,1.000000,1.000000}%
\pgfsetstrokecolor{currentstroke}%
\pgfsetdash{}{0pt}%
\pgfpathmoveto{\pgfqpoint{0.285587in}{3.618892in}}%
\pgfpathlineto{\pgfqpoint{2.868921in}{3.618892in}}%
\pgfusepath{stroke}%
\end{pgfscope}%
\begin{pgfscope}%
\definecolor{textcolor}{rgb}{0.150000,0.150000,0.150000}%
\pgfsetstrokecolor{textcolor}%
\pgfsetfillcolor{textcolor}%
\pgftext[x=1.577254in,y=3.702225in,,base]{\color{textcolor}\rmfamily\fontsize{12.000000}{14.400000}\selectfont GE}%
\end{pgfscope}%
\begin{pgfscope}%
\pgfsetbuttcap%
\pgfsetmiterjoin%
\definecolor{currentfill}{rgb}{0.917647,0.917647,0.949020}%
\pgfsetfillcolor{currentfill}%
\pgfsetlinewidth{0.000000pt}%
\definecolor{currentstroke}{rgb}{0.000000,0.000000,0.000000}%
\pgfsetstrokecolor{currentstroke}%
\pgfsetstrokeopacity{0.000000}%
\pgfsetdash{}{0pt}%
\pgfpathmoveto{\pgfqpoint{3.902254in}{3.218007in}}%
\pgfpathlineto{\pgfqpoint{6.485588in}{3.218007in}}%
\pgfpathlineto{\pgfqpoint{6.485588in}{3.618892in}}%
\pgfpathlineto{\pgfqpoint{3.902254in}{3.618892in}}%
\pgfpathclose%
\pgfusepath{fill}%
\end{pgfscope}%
\begin{pgfscope}%
\pgfpathrectangle{\pgfqpoint{3.902254in}{3.218007in}}{\pgfqpoint{2.583333in}{0.400885in}}%
\pgfusepath{clip}%
\pgfsetroundcap%
\pgfsetroundjoin%
\pgfsetlinewidth{0.803000pt}%
\definecolor{currentstroke}{rgb}{1.000000,1.000000,1.000000}%
\pgfsetstrokecolor{currentstroke}%
\pgfsetdash{}{0pt}%
\pgfpathmoveto{\pgfqpoint{4.017531in}{3.218007in}}%
\pgfpathlineto{\pgfqpoint{4.017531in}{3.618892in}}%
\pgfusepath{stroke}%
\end{pgfscope}%
\begin{pgfscope}%
\definecolor{textcolor}{rgb}{0.150000,0.150000,0.150000}%
\pgfsetstrokecolor{textcolor}%
\pgfsetfillcolor{textcolor}%
\pgftext[x=4.017531in,y=3.120784in,,top]{\color{textcolor}\rmfamily\fontsize{10.000000}{12.000000}\selectfont 2012}%
\end{pgfscope}%
\begin{pgfscope}%
\pgfpathrectangle{\pgfqpoint{3.902254in}{3.218007in}}{\pgfqpoint{2.583333in}{0.400885in}}%
\pgfusepath{clip}%
\pgfsetroundcap%
\pgfsetroundjoin%
\pgfsetlinewidth{0.803000pt}%
\definecolor{currentstroke}{rgb}{1.000000,1.000000,1.000000}%
\pgfsetstrokecolor{currentstroke}%
\pgfsetdash{}{0pt}%
\pgfpathmoveto{\pgfqpoint{4.410556in}{3.218007in}}%
\pgfpathlineto{\pgfqpoint{4.410556in}{3.618892in}}%
\pgfusepath{stroke}%
\end{pgfscope}%
\begin{pgfscope}%
\definecolor{textcolor}{rgb}{0.150000,0.150000,0.150000}%
\pgfsetstrokecolor{textcolor}%
\pgfsetfillcolor{textcolor}%
\pgftext[x=4.410556in,y=3.120784in,,top]{\color{textcolor}\rmfamily\fontsize{10.000000}{12.000000}\selectfont 2013}%
\end{pgfscope}%
\begin{pgfscope}%
\pgfpathrectangle{\pgfqpoint{3.902254in}{3.218007in}}{\pgfqpoint{2.583333in}{0.400885in}}%
\pgfusepath{clip}%
\pgfsetroundcap%
\pgfsetroundjoin%
\pgfsetlinewidth{0.803000pt}%
\definecolor{currentstroke}{rgb}{1.000000,1.000000,1.000000}%
\pgfsetstrokecolor{currentstroke}%
\pgfsetdash{}{0pt}%
\pgfpathmoveto{\pgfqpoint{4.802507in}{3.218007in}}%
\pgfpathlineto{\pgfqpoint{4.802507in}{3.618892in}}%
\pgfusepath{stroke}%
\end{pgfscope}%
\begin{pgfscope}%
\definecolor{textcolor}{rgb}{0.150000,0.150000,0.150000}%
\pgfsetstrokecolor{textcolor}%
\pgfsetfillcolor{textcolor}%
\pgftext[x=4.802507in,y=3.120784in,,top]{\color{textcolor}\rmfamily\fontsize{10.000000}{12.000000}\selectfont 2014}%
\end{pgfscope}%
\begin{pgfscope}%
\pgfpathrectangle{\pgfqpoint{3.902254in}{3.218007in}}{\pgfqpoint{2.583333in}{0.400885in}}%
\pgfusepath{clip}%
\pgfsetroundcap%
\pgfsetroundjoin%
\pgfsetlinewidth{0.803000pt}%
\definecolor{currentstroke}{rgb}{1.000000,1.000000,1.000000}%
\pgfsetstrokecolor{currentstroke}%
\pgfsetdash{}{0pt}%
\pgfpathmoveto{\pgfqpoint{5.194458in}{3.218007in}}%
\pgfpathlineto{\pgfqpoint{5.194458in}{3.618892in}}%
\pgfusepath{stroke}%
\end{pgfscope}%
\begin{pgfscope}%
\definecolor{textcolor}{rgb}{0.150000,0.150000,0.150000}%
\pgfsetstrokecolor{textcolor}%
\pgfsetfillcolor{textcolor}%
\pgftext[x=5.194458in,y=3.120784in,,top]{\color{textcolor}\rmfamily\fontsize{10.000000}{12.000000}\selectfont 2015}%
\end{pgfscope}%
\begin{pgfscope}%
\pgfpathrectangle{\pgfqpoint{3.902254in}{3.218007in}}{\pgfqpoint{2.583333in}{0.400885in}}%
\pgfusepath{clip}%
\pgfsetroundcap%
\pgfsetroundjoin%
\pgfsetlinewidth{0.803000pt}%
\definecolor{currentstroke}{rgb}{1.000000,1.000000,1.000000}%
\pgfsetstrokecolor{currentstroke}%
\pgfsetdash{}{0pt}%
\pgfpathmoveto{\pgfqpoint{5.586409in}{3.218007in}}%
\pgfpathlineto{\pgfqpoint{5.586409in}{3.618892in}}%
\pgfusepath{stroke}%
\end{pgfscope}%
\begin{pgfscope}%
\definecolor{textcolor}{rgb}{0.150000,0.150000,0.150000}%
\pgfsetstrokecolor{textcolor}%
\pgfsetfillcolor{textcolor}%
\pgftext[x=5.586409in,y=3.120784in,,top]{\color{textcolor}\rmfamily\fontsize{10.000000}{12.000000}\selectfont 2016}%
\end{pgfscope}%
\begin{pgfscope}%
\pgfpathrectangle{\pgfqpoint{3.902254in}{3.218007in}}{\pgfqpoint{2.583333in}{0.400885in}}%
\pgfusepath{clip}%
\pgfsetroundcap%
\pgfsetroundjoin%
\pgfsetlinewidth{0.803000pt}%
\definecolor{currentstroke}{rgb}{1.000000,1.000000,1.000000}%
\pgfsetstrokecolor{currentstroke}%
\pgfsetdash{}{0pt}%
\pgfpathmoveto{\pgfqpoint{5.979434in}{3.218007in}}%
\pgfpathlineto{\pgfqpoint{5.979434in}{3.618892in}}%
\pgfusepath{stroke}%
\end{pgfscope}%
\begin{pgfscope}%
\definecolor{textcolor}{rgb}{0.150000,0.150000,0.150000}%
\pgfsetstrokecolor{textcolor}%
\pgfsetfillcolor{textcolor}%
\pgftext[x=5.979434in,y=3.120784in,,top]{\color{textcolor}\rmfamily\fontsize{10.000000}{12.000000}\selectfont 2017}%
\end{pgfscope}%
\begin{pgfscope}%
\pgfpathrectangle{\pgfqpoint{3.902254in}{3.218007in}}{\pgfqpoint{2.583333in}{0.400885in}}%
\pgfusepath{clip}%
\pgfsetroundcap%
\pgfsetroundjoin%
\pgfsetlinewidth{0.803000pt}%
\definecolor{currentstroke}{rgb}{1.000000,1.000000,1.000000}%
\pgfsetstrokecolor{currentstroke}%
\pgfsetdash{}{0pt}%
\pgfpathmoveto{\pgfqpoint{6.371385in}{3.218007in}}%
\pgfpathlineto{\pgfqpoint{6.371385in}{3.618892in}}%
\pgfusepath{stroke}%
\end{pgfscope}%
\begin{pgfscope}%
\definecolor{textcolor}{rgb}{0.150000,0.150000,0.150000}%
\pgfsetstrokecolor{textcolor}%
\pgfsetfillcolor{textcolor}%
\pgftext[x=6.371385in,y=3.120784in,,top]{\color{textcolor}\rmfamily\fontsize{10.000000}{12.000000}\selectfont 2018}%
\end{pgfscope}%
\begin{pgfscope}%
\pgfpathrectangle{\pgfqpoint{3.902254in}{3.218007in}}{\pgfqpoint{2.583333in}{0.400885in}}%
\pgfusepath{clip}%
\pgfsetroundcap%
\pgfsetroundjoin%
\pgfsetlinewidth{0.803000pt}%
\definecolor{currentstroke}{rgb}{1.000000,1.000000,1.000000}%
\pgfsetstrokecolor{currentstroke}%
\pgfsetdash{}{0pt}%
\pgfpathmoveto{\pgfqpoint{3.902254in}{3.374091in}}%
\pgfpathlineto{\pgfqpoint{6.485588in}{3.374091in}}%
\pgfusepath{stroke}%
\end{pgfscope}%
\begin{pgfscope}%
\definecolor{textcolor}{rgb}{0.150000,0.150000,0.150000}%
\pgfsetstrokecolor{textcolor}%
\pgfsetfillcolor{textcolor}%
\pgftext[x=3.716667in,y=3.321329in,left,base]{\color{textcolor}\rmfamily\fontsize{10.000000}{12.000000}\selectfont 1}%
\end{pgfscope}%
\begin{pgfscope}%
\pgfpathrectangle{\pgfqpoint{3.902254in}{3.218007in}}{\pgfqpoint{2.583333in}{0.400885in}}%
\pgfusepath{clip}%
\pgfsetroundcap%
\pgfsetroundjoin%
\pgfsetlinewidth{0.803000pt}%
\definecolor{currentstroke}{rgb}{1.000000,1.000000,1.000000}%
\pgfsetstrokecolor{currentstroke}%
\pgfsetdash{}{0pt}%
\pgfpathmoveto{\pgfqpoint{3.902254in}{3.541385in}}%
\pgfpathlineto{\pgfqpoint{6.485588in}{3.541385in}}%
\pgfusepath{stroke}%
\end{pgfscope}%
\begin{pgfscope}%
\definecolor{textcolor}{rgb}{0.150000,0.150000,0.150000}%
\pgfsetstrokecolor{textcolor}%
\pgfsetfillcolor{textcolor}%
\pgftext[x=3.716667in,y=3.488624in,left,base]{\color{textcolor}\rmfamily\fontsize{10.000000}{12.000000}\selectfont 2}%
\end{pgfscope}%
\begin{pgfscope}%
\pgfpathrectangle{\pgfqpoint{3.902254in}{3.218007in}}{\pgfqpoint{2.583333in}{0.400885in}}%
\pgfusepath{clip}%
\pgfsetroundcap%
\pgfsetroundjoin%
\pgfsetlinewidth{1.505625pt}%
\definecolor{currentstroke}{rgb}{0.839216,0.152941,0.156863}%
\pgfsetstrokecolor{currentstroke}%
\pgfsetstrokeopacity{0.300000}%
\pgfsetdash{}{0pt}%
\pgfpathmoveto{\pgfqpoint{4.019678in}{3.374091in}}%
\pgfpathlineto{\pgfqpoint{4.021826in}{3.379976in}}%
\pgfpathlineto{\pgfqpoint{4.022900in}{3.378938in}}%
\pgfpathlineto{\pgfqpoint{4.027195in}{3.381274in}}%
\pgfpathlineto{\pgfqpoint{4.028269in}{3.382746in}}%
\pgfpathlineto{\pgfqpoint{4.029343in}{3.382399in}}%
\pgfpathlineto{\pgfqpoint{4.030417in}{3.378245in}}%
\pgfpathlineto{\pgfqpoint{4.034712in}{3.377553in}}%
\pgfpathlineto{\pgfqpoint{4.036860in}{3.381534in}}%
\pgfpathlineto{\pgfqpoint{4.037934in}{3.386640in}}%
\pgfpathlineto{\pgfqpoint{4.043303in}{3.390189in}}%
\pgfpathlineto{\pgfqpoint{4.044377in}{3.389150in}}%
\pgfpathlineto{\pgfqpoint{4.048672in}{3.389150in}}%
\pgfpathlineto{\pgfqpoint{4.049746in}{3.386900in}}%
\pgfpathlineto{\pgfqpoint{4.050820in}{3.387852in}}%
\pgfpathlineto{\pgfqpoint{4.051894in}{3.387419in}}%
\pgfpathlineto{\pgfqpoint{4.052967in}{3.390535in}}%
\pgfpathlineto{\pgfqpoint{4.056189in}{3.390448in}}%
\pgfpathlineto{\pgfqpoint{4.057263in}{3.389842in}}%
\pgfpathlineto{\pgfqpoint{4.058337in}{3.391314in}}%
\pgfpathlineto{\pgfqpoint{4.059410in}{3.391400in}}%
\pgfpathlineto{\pgfqpoint{4.060484in}{3.390275in}}%
\pgfpathlineto{\pgfqpoint{4.063706in}{3.390275in}}%
\pgfpathlineto{\pgfqpoint{4.064780in}{3.390881in}}%
\pgfpathlineto{\pgfqpoint{4.065853in}{3.389496in}}%
\pgfpathlineto{\pgfqpoint{4.066927in}{3.391227in}}%
\pgfpathlineto{\pgfqpoint{4.068001in}{3.394862in}}%
\pgfpathlineto{\pgfqpoint{4.072296in}{3.393477in}}%
\pgfpathlineto{\pgfqpoint{4.073370in}{3.390535in}}%
\pgfpathlineto{\pgfqpoint{4.074444in}{3.390016in}}%
\pgfpathlineto{\pgfqpoint{4.078740in}{3.391573in}}%
\pgfpathlineto{\pgfqpoint{4.079813in}{3.393997in}}%
\pgfpathlineto{\pgfqpoint{4.080887in}{3.391573in}}%
\pgfpathlineto{\pgfqpoint{4.081961in}{3.391400in}}%
\pgfpathlineto{\pgfqpoint{4.083035in}{3.391833in}}%
\pgfpathlineto{\pgfqpoint{4.086256in}{3.389237in}}%
\pgfpathlineto{\pgfqpoint{4.087330in}{3.389669in}}%
\pgfpathlineto{\pgfqpoint{4.088404in}{3.391746in}}%
\pgfpathlineto{\pgfqpoint{4.089478in}{3.391227in}}%
\pgfpathlineto{\pgfqpoint{4.090552in}{3.392872in}}%
\pgfpathlineto{\pgfqpoint{4.093773in}{3.392266in}}%
\pgfpathlineto{\pgfqpoint{4.094847in}{3.395728in}}%
\pgfpathlineto{\pgfqpoint{4.095921in}{3.395555in}}%
\pgfpathlineto{\pgfqpoint{4.096995in}{3.397545in}}%
\pgfpathlineto{\pgfqpoint{4.098069in}{3.397372in}}%
\pgfpathlineto{\pgfqpoint{4.103438in}{3.397718in}}%
\pgfpathlineto{\pgfqpoint{4.104512in}{3.398584in}}%
\pgfpathlineto{\pgfqpoint{4.105585in}{3.398411in}}%
\pgfpathlineto{\pgfqpoint{4.108807in}{3.400574in}}%
\pgfpathlineto{\pgfqpoint{4.109881in}{3.400574in}}%
\pgfpathlineto{\pgfqpoint{4.110955in}{3.397891in}}%
\pgfpathlineto{\pgfqpoint{4.112029in}{3.400315in}}%
\pgfpathlineto{\pgfqpoint{4.113102in}{3.400055in}}%
\pgfpathlineto{\pgfqpoint{4.116324in}{3.401872in}}%
\pgfpathlineto{\pgfqpoint{4.118472in}{3.398757in}}%
\pgfpathlineto{\pgfqpoint{4.119545in}{3.399709in}}%
\pgfpathlineto{\pgfqpoint{4.123841in}{3.397545in}}%
\pgfpathlineto{\pgfqpoint{4.124915in}{3.395468in}}%
\pgfpathlineto{\pgfqpoint{4.125988in}{3.398237in}}%
\pgfpathlineto{\pgfqpoint{4.127062in}{3.402565in}}%
\pgfpathlineto{\pgfqpoint{4.128136in}{3.399882in}}%
\pgfpathlineto{\pgfqpoint{4.132431in}{3.402478in}}%
\pgfpathlineto{\pgfqpoint{4.134579in}{3.397112in}}%
\pgfpathlineto{\pgfqpoint{4.135653in}{3.396507in}}%
\pgfpathlineto{\pgfqpoint{4.138874in}{3.395468in}}%
\pgfpathlineto{\pgfqpoint{4.139948in}{3.394516in}}%
\pgfpathlineto{\pgfqpoint{4.142096in}{3.400747in}}%
\pgfpathlineto{\pgfqpoint{4.143170in}{3.401872in}}%
\pgfpathlineto{\pgfqpoint{4.146391in}{3.401959in}}%
\pgfpathlineto{\pgfqpoint{4.147465in}{3.405767in}}%
\pgfpathlineto{\pgfqpoint{4.148539in}{3.407325in}}%
\pgfpathlineto{\pgfqpoint{4.149613in}{3.404469in}}%
\pgfpathlineto{\pgfqpoint{4.150687in}{3.399968in}}%
\pgfpathlineto{\pgfqpoint{4.153908in}{3.398930in}}%
\pgfpathlineto{\pgfqpoint{4.156056in}{3.395035in}}%
\pgfpathlineto{\pgfqpoint{4.157130in}{3.395381in}}%
\pgfpathlineto{\pgfqpoint{4.158204in}{3.398064in}}%
\pgfpathlineto{\pgfqpoint{4.163573in}{3.390275in}}%
\pgfpathlineto{\pgfqpoint{4.164647in}{3.388112in}}%
\pgfpathlineto{\pgfqpoint{4.165720in}{3.387246in}}%
\pgfpathlineto{\pgfqpoint{4.168942in}{3.387852in}}%
\pgfpathlineto{\pgfqpoint{4.170016in}{3.386986in}}%
\pgfpathlineto{\pgfqpoint{4.171090in}{3.382919in}}%
\pgfpathlineto{\pgfqpoint{4.173237in}{3.384996in}}%
\pgfpathlineto{\pgfqpoint{4.178607in}{3.387679in}}%
\pgfpathlineto{\pgfqpoint{4.179680in}{3.385688in}}%
\pgfpathlineto{\pgfqpoint{4.180754in}{3.380842in}}%
\pgfpathlineto{\pgfqpoint{4.183976in}{3.380149in}}%
\pgfpathlineto{\pgfqpoint{4.185050in}{3.382832in}}%
\pgfpathlineto{\pgfqpoint{4.186123in}{3.387246in}}%
\pgfpathlineto{\pgfqpoint{4.187197in}{3.386381in}}%
\pgfpathlineto{\pgfqpoint{4.188271in}{3.389583in}}%
\pgfpathlineto{\pgfqpoint{4.191493in}{3.386727in}}%
\pgfpathlineto{\pgfqpoint{4.192566in}{3.390362in}}%
\pgfpathlineto{\pgfqpoint{4.193640in}{3.390535in}}%
\pgfpathlineto{\pgfqpoint{4.195788in}{3.396074in}}%
\pgfpathlineto{\pgfqpoint{4.200083in}{3.397199in}}%
\pgfpathlineto{\pgfqpoint{4.201157in}{3.398151in}}%
\pgfpathlineto{\pgfqpoint{4.202231in}{3.391660in}}%
\pgfpathlineto{\pgfqpoint{4.203305in}{3.393304in}}%
\pgfpathlineto{\pgfqpoint{4.206526in}{3.387160in}}%
\pgfpathlineto{\pgfqpoint{4.207600in}{3.386813in}}%
\pgfpathlineto{\pgfqpoint{4.208674in}{3.388285in}}%
\pgfpathlineto{\pgfqpoint{4.209748in}{3.385602in}}%
\pgfpathlineto{\pgfqpoint{4.210822in}{3.391314in}}%
\pgfpathlineto{\pgfqpoint{4.214043in}{3.391400in}}%
\pgfpathlineto{\pgfqpoint{4.215117in}{3.392698in}}%
\pgfpathlineto{\pgfqpoint{4.217265in}{3.390621in}}%
\pgfpathlineto{\pgfqpoint{4.218339in}{3.387852in}}%
\pgfpathlineto{\pgfqpoint{4.221560in}{3.387938in}}%
\pgfpathlineto{\pgfqpoint{4.222634in}{3.383698in}}%
\pgfpathlineto{\pgfqpoint{4.223708in}{3.382573in}}%
\pgfpathlineto{\pgfqpoint{4.224782in}{3.378072in}}%
\pgfpathlineto{\pgfqpoint{4.225855in}{3.381621in}}%
\pgfpathlineto{\pgfqpoint{4.229077in}{3.380755in}}%
\pgfpathlineto{\pgfqpoint{4.230151in}{3.382486in}}%
\pgfpathlineto{\pgfqpoint{4.231225in}{3.388198in}}%
\pgfpathlineto{\pgfqpoint{4.232298in}{3.387160in}}%
\pgfpathlineto{\pgfqpoint{4.233372in}{3.383438in}}%
\pgfpathlineto{\pgfqpoint{4.236594in}{3.381621in}}%
\pgfpathlineto{\pgfqpoint{4.237668in}{3.379890in}}%
\pgfpathlineto{\pgfqpoint{4.238741in}{3.380755in}}%
\pgfpathlineto{\pgfqpoint{4.240889in}{3.386900in}}%
\pgfpathlineto{\pgfqpoint{4.245184in}{3.384736in}}%
\pgfpathlineto{\pgfqpoint{4.246258in}{3.386294in}}%
\pgfpathlineto{\pgfqpoint{4.247332in}{3.386121in}}%
\pgfpathlineto{\pgfqpoint{4.248406in}{3.389929in}}%
\pgfpathlineto{\pgfqpoint{4.251628in}{3.390535in}}%
\pgfpathlineto{\pgfqpoint{4.253775in}{3.392525in}}%
\pgfpathlineto{\pgfqpoint{4.255923in}{3.394516in}}%
\pgfpathlineto{\pgfqpoint{4.259144in}{3.393218in}}%
\pgfpathlineto{\pgfqpoint{4.261292in}{3.390275in}}%
\pgfpathlineto{\pgfqpoint{4.262366in}{3.392439in}}%
\pgfpathlineto{\pgfqpoint{4.263440in}{3.390621in}}%
\pgfpathlineto{\pgfqpoint{4.267735in}{3.389150in}}%
\pgfpathlineto{\pgfqpoint{4.268809in}{3.386467in}}%
\pgfpathlineto{\pgfqpoint{4.269883in}{3.381621in}}%
\pgfpathlineto{\pgfqpoint{4.270957in}{3.380755in}}%
\pgfpathlineto{\pgfqpoint{4.274178in}{3.380236in}}%
\pgfpathlineto{\pgfqpoint{4.275252in}{3.381361in}}%
\pgfpathlineto{\pgfqpoint{4.277400in}{3.376255in}}%
\pgfpathlineto{\pgfqpoint{4.278473in}{3.380149in}}%
\pgfpathlineto{\pgfqpoint{4.283843in}{3.377120in}}%
\pgfpathlineto{\pgfqpoint{4.284917in}{3.382053in}}%
\pgfpathlineto{\pgfqpoint{4.285990in}{3.375735in}}%
\pgfpathlineto{\pgfqpoint{4.289212in}{3.369244in}}%
\pgfpathlineto{\pgfqpoint{4.290286in}{3.369764in}}%
\pgfpathlineto{\pgfqpoint{4.291360in}{3.368725in}}%
\pgfpathlineto{\pgfqpoint{4.292433in}{3.369937in}}%
\pgfpathlineto{\pgfqpoint{4.293507in}{3.370023in}}%
\pgfpathlineto{\pgfqpoint{4.296729in}{3.369591in}}%
\pgfpathlineto{\pgfqpoint{4.297803in}{3.370023in}}%
\pgfpathlineto{\pgfqpoint{4.298876in}{3.368465in}}%
\pgfpathlineto{\pgfqpoint{4.301024in}{3.368292in}}%
\pgfpathlineto{\pgfqpoint{4.304246in}{3.366042in}}%
\pgfpathlineto{\pgfqpoint{4.305319in}{3.364225in}}%
\pgfpathlineto{\pgfqpoint{4.306393in}{3.365004in}}%
\pgfpathlineto{\pgfqpoint{4.307467in}{3.368033in}}%
\pgfpathlineto{\pgfqpoint{4.308541in}{3.365004in}}%
\pgfpathlineto{\pgfqpoint{4.312836in}{3.366302in}}%
\pgfpathlineto{\pgfqpoint{4.313910in}{3.364225in}}%
\pgfpathlineto{\pgfqpoint{4.314984in}{3.363705in}}%
\pgfpathlineto{\pgfqpoint{4.316058in}{3.365177in}}%
\pgfpathlineto{\pgfqpoint{4.319279in}{3.363965in}}%
\pgfpathlineto{\pgfqpoint{4.320353in}{3.359724in}}%
\pgfpathlineto{\pgfqpoint{4.323575in}{3.356782in}}%
\pgfpathlineto{\pgfqpoint{4.326796in}{3.358513in}}%
\pgfpathlineto{\pgfqpoint{4.327870in}{3.362840in}}%
\pgfpathlineto{\pgfqpoint{4.328944in}{3.358945in}}%
\pgfpathlineto{\pgfqpoint{4.330018in}{3.358080in}}%
\pgfpathlineto{\pgfqpoint{4.331092in}{3.355310in}}%
\pgfpathlineto{\pgfqpoint{4.335387in}{3.357561in}}%
\pgfpathlineto{\pgfqpoint{4.336461in}{3.356695in}}%
\pgfpathlineto{\pgfqpoint{4.338608in}{3.360070in}}%
\pgfpathlineto{\pgfqpoint{4.343978in}{3.357820in}}%
\pgfpathlineto{\pgfqpoint{4.345051in}{3.362234in}}%
\pgfpathlineto{\pgfqpoint{4.346125in}{3.360849in}}%
\pgfpathlineto{\pgfqpoint{4.349347in}{3.360849in}}%
\pgfpathlineto{\pgfqpoint{4.350421in}{3.360070in}}%
\pgfpathlineto{\pgfqpoint{4.351495in}{3.354358in}}%
\pgfpathlineto{\pgfqpoint{4.353642in}{3.353580in}}%
\pgfpathlineto{\pgfqpoint{4.356864in}{3.353320in}}%
\pgfpathlineto{\pgfqpoint{4.359011in}{3.347608in}}%
\pgfpathlineto{\pgfqpoint{4.364381in}{3.349685in}}%
\pgfpathlineto{\pgfqpoint{4.365454in}{3.344406in}}%
\pgfpathlineto{\pgfqpoint{4.366528in}{3.343367in}}%
\pgfpathlineto{\pgfqpoint{4.368676in}{3.345963in}}%
\pgfpathlineto{\pgfqpoint{4.372971in}{3.347435in}}%
\pgfpathlineto{\pgfqpoint{4.374045in}{3.348560in}}%
\pgfpathlineto{\pgfqpoint{4.375119in}{3.344579in}}%
\pgfpathlineto{\pgfqpoint{4.376193in}{3.344838in}}%
\pgfpathlineto{\pgfqpoint{4.379414in}{3.344665in}}%
\pgfpathlineto{\pgfqpoint{4.380488in}{3.347694in}}%
\pgfpathlineto{\pgfqpoint{4.381562in}{3.346829in}}%
\pgfpathlineto{\pgfqpoint{4.382636in}{3.348993in}}%
\pgfpathlineto{\pgfqpoint{4.386931in}{3.348473in}}%
\pgfpathlineto{\pgfqpoint{4.388005in}{3.352454in}}%
\pgfpathlineto{\pgfqpoint{4.389079in}{3.352627in}}%
\pgfpathlineto{\pgfqpoint{4.390153in}{3.351329in}}%
\pgfpathlineto{\pgfqpoint{4.391227in}{3.351675in}}%
\pgfpathlineto{\pgfqpoint{4.394448in}{3.351935in}}%
\pgfpathlineto{\pgfqpoint{4.395522in}{3.354705in}}%
\pgfpathlineto{\pgfqpoint{4.396596in}{3.355657in}}%
\pgfpathlineto{\pgfqpoint{4.397670in}{3.355137in}}%
\pgfpathlineto{\pgfqpoint{4.398743in}{3.353320in}}%
\pgfpathlineto{\pgfqpoint{4.401965in}{3.352454in}}%
\pgfpathlineto{\pgfqpoint{4.404113in}{3.352454in}}%
\pgfpathlineto{\pgfqpoint{4.405186in}{3.351502in}}%
\pgfpathlineto{\pgfqpoint{4.406260in}{3.349512in}}%
\pgfpathlineto{\pgfqpoint{4.409482in}{3.352281in}}%
\pgfpathlineto{\pgfqpoint{4.411629in}{3.357647in}}%
\pgfpathlineto{\pgfqpoint{4.412703in}{3.357214in}}%
\pgfpathlineto{\pgfqpoint{4.413777in}{3.356089in}}%
\pgfpathlineto{\pgfqpoint{4.416999in}{3.356695in}}%
\pgfpathlineto{\pgfqpoint{4.418073in}{3.355570in}}%
\pgfpathlineto{\pgfqpoint{4.421294in}{3.361975in}}%
\pgfpathlineto{\pgfqpoint{4.424516in}{3.361975in}}%
\pgfpathlineto{\pgfqpoint{4.425589in}{3.361196in}}%
\pgfpathlineto{\pgfqpoint{4.426663in}{3.362753in}}%
\pgfpathlineto{\pgfqpoint{4.427737in}{3.366821in}}%
\pgfpathlineto{\pgfqpoint{4.428811in}{3.356695in}}%
\pgfpathlineto{\pgfqpoint{4.434180in}{3.355743in}}%
\pgfpathlineto{\pgfqpoint{4.435254in}{3.354618in}}%
\pgfpathlineto{\pgfqpoint{4.439549in}{3.355310in}}%
\pgfpathlineto{\pgfqpoint{4.440623in}{3.356955in}}%
\pgfpathlineto{\pgfqpoint{4.441697in}{3.357561in}}%
\pgfpathlineto{\pgfqpoint{4.442771in}{3.355224in}}%
\pgfpathlineto{\pgfqpoint{4.443845in}{3.357474in}}%
\pgfpathlineto{\pgfqpoint{4.447066in}{3.356089in}}%
\pgfpathlineto{\pgfqpoint{4.448140in}{3.357820in}}%
\pgfpathlineto{\pgfqpoint{4.450288in}{3.355224in}}%
\pgfpathlineto{\pgfqpoint{4.451361in}{3.356522in}}%
\pgfpathlineto{\pgfqpoint{4.454583in}{3.356782in}}%
\pgfpathlineto{\pgfqpoint{4.456731in}{3.358340in}}%
\pgfpathlineto{\pgfqpoint{4.457805in}{3.358166in}}%
\pgfpathlineto{\pgfqpoint{4.458878in}{3.357388in}}%
\pgfpathlineto{\pgfqpoint{4.463174in}{3.357214in}}%
\pgfpathlineto{\pgfqpoint{4.465321in}{3.351156in}}%
\pgfpathlineto{\pgfqpoint{4.466395in}{3.352368in}}%
\pgfpathlineto{\pgfqpoint{4.469617in}{3.351070in}}%
\pgfpathlineto{\pgfqpoint{4.471764in}{3.356003in}}%
\pgfpathlineto{\pgfqpoint{4.472838in}{3.355657in}}%
\pgfpathlineto{\pgfqpoint{4.473912in}{3.356782in}}%
\pgfpathlineto{\pgfqpoint{4.477134in}{3.358513in}}%
\pgfpathlineto{\pgfqpoint{4.480355in}{3.362927in}}%
\pgfpathlineto{\pgfqpoint{4.481429in}{3.360676in}}%
\pgfpathlineto{\pgfqpoint{4.484650in}{3.361455in}}%
\pgfpathlineto{\pgfqpoint{4.486798in}{3.361282in}}%
\pgfpathlineto{\pgfqpoint{4.487872in}{3.361196in}}%
\pgfpathlineto{\pgfqpoint{4.488946in}{3.359292in}}%
\pgfpathlineto{\pgfqpoint{4.492167in}{3.358426in}}%
\pgfpathlineto{\pgfqpoint{4.493241in}{3.357561in}}%
\pgfpathlineto{\pgfqpoint{4.494315in}{3.357820in}}%
\pgfpathlineto{\pgfqpoint{4.495389in}{3.356868in}}%
\pgfpathlineto{\pgfqpoint{4.496463in}{3.358859in}}%
\pgfpathlineto{\pgfqpoint{4.499684in}{3.357647in}}%
\pgfpathlineto{\pgfqpoint{4.500758in}{3.362061in}}%
\pgfpathlineto{\pgfqpoint{4.502906in}{3.362494in}}%
\pgfpathlineto{\pgfqpoint{4.507201in}{3.359638in}}%
\pgfpathlineto{\pgfqpoint{4.508275in}{3.359811in}}%
\pgfpathlineto{\pgfqpoint{4.509349in}{3.356868in}}%
\pgfpathlineto{\pgfqpoint{4.510423in}{3.357561in}}%
\pgfpathlineto{\pgfqpoint{4.511496in}{3.356089in}}%
\pgfpathlineto{\pgfqpoint{4.514718in}{3.357214in}}%
\pgfpathlineto{\pgfqpoint{4.516866in}{3.365523in}}%
\pgfpathlineto{\pgfqpoint{4.517939in}{3.362494in}}%
\pgfpathlineto{\pgfqpoint{4.519013in}{3.361369in}}%
\pgfpathlineto{\pgfqpoint{4.522235in}{3.359292in}}%
\pgfpathlineto{\pgfqpoint{4.523309in}{3.363100in}}%
\pgfpathlineto{\pgfqpoint{4.524383in}{3.363186in}}%
\pgfpathlineto{\pgfqpoint{4.526530in}{3.366821in}}%
\pgfpathlineto{\pgfqpoint{4.529752in}{3.369937in}}%
\pgfpathlineto{\pgfqpoint{4.531899in}{3.375476in}}%
\pgfpathlineto{\pgfqpoint{4.532973in}{3.373485in}}%
\pgfpathlineto{\pgfqpoint{4.534047in}{3.373658in}}%
\pgfpathlineto{\pgfqpoint{4.541564in}{3.379284in}}%
\pgfpathlineto{\pgfqpoint{4.544785in}{3.378938in}}%
\pgfpathlineto{\pgfqpoint{4.546933in}{3.381361in}}%
\pgfpathlineto{\pgfqpoint{4.549081in}{3.383178in}}%
\pgfpathlineto{\pgfqpoint{4.553376in}{3.378418in}}%
\pgfpathlineto{\pgfqpoint{4.554450in}{3.381015in}}%
\pgfpathlineto{\pgfqpoint{4.555524in}{3.379111in}}%
\pgfpathlineto{\pgfqpoint{4.556598in}{3.379803in}}%
\pgfpathlineto{\pgfqpoint{4.561967in}{3.380063in}}%
\pgfpathlineto{\pgfqpoint{4.563041in}{3.379890in}}%
\pgfpathlineto{\pgfqpoint{4.564115in}{3.378938in}}%
\pgfpathlineto{\pgfqpoint{4.568410in}{3.380149in}}%
\pgfpathlineto{\pgfqpoint{4.569484in}{3.381534in}}%
\pgfpathlineto{\pgfqpoint{4.570558in}{3.381101in}}%
\pgfpathlineto{\pgfqpoint{4.571631in}{3.381534in}}%
\pgfpathlineto{\pgfqpoint{4.574853in}{3.388458in}}%
\pgfpathlineto{\pgfqpoint{4.575927in}{3.389323in}}%
\pgfpathlineto{\pgfqpoint{4.577001in}{3.384563in}}%
\pgfpathlineto{\pgfqpoint{4.579148in}{3.383784in}}%
\pgfpathlineto{\pgfqpoint{4.582370in}{3.386813in}}%
\pgfpathlineto{\pgfqpoint{4.584517in}{3.382832in}}%
\pgfpathlineto{\pgfqpoint{4.585591in}{3.386640in}}%
\pgfpathlineto{\pgfqpoint{4.586665in}{3.386208in}}%
\pgfpathlineto{\pgfqpoint{4.589887in}{3.387506in}}%
\pgfpathlineto{\pgfqpoint{4.590961in}{3.390102in}}%
\pgfpathlineto{\pgfqpoint{4.592034in}{3.386727in}}%
\pgfpathlineto{\pgfqpoint{4.593108in}{3.380928in}}%
\pgfpathlineto{\pgfqpoint{4.594182in}{3.381015in}}%
\pgfpathlineto{\pgfqpoint{4.597404in}{3.376514in}}%
\pgfpathlineto{\pgfqpoint{4.599551in}{3.379630in}}%
\pgfpathlineto{\pgfqpoint{4.600625in}{3.379890in}}%
\pgfpathlineto{\pgfqpoint{4.601699in}{3.381188in}}%
\pgfpathlineto{\pgfqpoint{4.607068in}{3.377813in}}%
\pgfpathlineto{\pgfqpoint{4.609216in}{3.379976in}}%
\pgfpathlineto{\pgfqpoint{4.612437in}{3.373745in}}%
\pgfpathlineto{\pgfqpoint{4.613511in}{3.373399in}}%
\pgfpathlineto{\pgfqpoint{4.614585in}{3.374178in}}%
\pgfpathlineto{\pgfqpoint{4.615659in}{3.379457in}}%
\pgfpathlineto{\pgfqpoint{4.616733in}{3.378851in}}%
\pgfpathlineto{\pgfqpoint{4.619954in}{3.379111in}}%
\pgfpathlineto{\pgfqpoint{4.621028in}{3.381361in}}%
\pgfpathlineto{\pgfqpoint{4.622102in}{3.380669in}}%
\pgfpathlineto{\pgfqpoint{4.623176in}{3.374091in}}%
\pgfpathlineto{\pgfqpoint{4.624249in}{3.372620in}}%
\pgfpathlineto{\pgfqpoint{4.628545in}{3.370543in}}%
\pgfpathlineto{\pgfqpoint{4.631766in}{3.374264in}}%
\pgfpathlineto{\pgfqpoint{4.634988in}{3.374091in}}%
\pgfpathlineto{\pgfqpoint{4.636062in}{3.375130in}}%
\pgfpathlineto{\pgfqpoint{4.637136in}{3.374783in}}%
\pgfpathlineto{\pgfqpoint{4.638209in}{3.373831in}}%
\pgfpathlineto{\pgfqpoint{4.639283in}{3.373918in}}%
\pgfpathlineto{\pgfqpoint{4.642505in}{3.373399in}}%
\pgfpathlineto{\pgfqpoint{4.644652in}{3.371754in}}%
\pgfpathlineto{\pgfqpoint{4.645726in}{3.369937in}}%
\pgfpathlineto{\pgfqpoint{4.650022in}{3.371322in}}%
\pgfpathlineto{\pgfqpoint{4.651095in}{3.370456in}}%
\pgfpathlineto{\pgfqpoint{4.652169in}{3.370889in}}%
\pgfpathlineto{\pgfqpoint{4.653243in}{3.366908in}}%
\pgfpathlineto{\pgfqpoint{4.654317in}{3.366042in}}%
\pgfpathlineto{\pgfqpoint{4.657538in}{3.368725in}}%
\pgfpathlineto{\pgfqpoint{4.658612in}{3.370456in}}%
\pgfpathlineto{\pgfqpoint{4.659686in}{3.367946in}}%
\pgfpathlineto{\pgfqpoint{4.660760in}{3.368552in}}%
\pgfpathlineto{\pgfqpoint{4.661834in}{3.369937in}}%
\pgfpathlineto{\pgfqpoint{4.666129in}{3.368119in}}%
\pgfpathlineto{\pgfqpoint{4.667203in}{3.368812in}}%
\pgfpathlineto{\pgfqpoint{4.668277in}{3.367167in}}%
\pgfpathlineto{\pgfqpoint{4.669351in}{3.366561in}}%
\pgfpathlineto{\pgfqpoint{4.673646in}{3.367254in}}%
\pgfpathlineto{\pgfqpoint{4.674720in}{3.371322in}}%
\pgfpathlineto{\pgfqpoint{4.675794in}{3.371062in}}%
\pgfpathlineto{\pgfqpoint{4.681163in}{3.373831in}}%
\pgfpathlineto{\pgfqpoint{4.683311in}{3.371322in}}%
\pgfpathlineto{\pgfqpoint{4.684384in}{3.377207in}}%
\pgfpathlineto{\pgfqpoint{4.687606in}{3.376774in}}%
\pgfpathlineto{\pgfqpoint{4.688680in}{3.379370in}}%
\pgfpathlineto{\pgfqpoint{4.689754in}{3.380495in}}%
\pgfpathlineto{\pgfqpoint{4.690827in}{3.380582in}}%
\pgfpathlineto{\pgfqpoint{4.691901in}{3.379543in}}%
\pgfpathlineto{\pgfqpoint{4.695123in}{3.378505in}}%
\pgfpathlineto{\pgfqpoint{4.697271in}{3.379024in}}%
\pgfpathlineto{\pgfqpoint{4.699418in}{3.373831in}}%
\pgfpathlineto{\pgfqpoint{4.704787in}{3.373139in}}%
\pgfpathlineto{\pgfqpoint{4.705861in}{3.371062in}}%
\pgfpathlineto{\pgfqpoint{4.706935in}{3.372620in}}%
\pgfpathlineto{\pgfqpoint{4.710157in}{3.372706in}}%
\pgfpathlineto{\pgfqpoint{4.711230in}{3.370196in}}%
\pgfpathlineto{\pgfqpoint{4.712304in}{3.370975in}}%
\pgfpathlineto{\pgfqpoint{4.713378in}{3.374697in}}%
\pgfpathlineto{\pgfqpoint{4.714452in}{3.375822in}}%
\pgfpathlineto{\pgfqpoint{4.717673in}{3.377207in}}%
\pgfpathlineto{\pgfqpoint{4.718747in}{3.376774in}}%
\pgfpathlineto{\pgfqpoint{4.720895in}{3.380669in}}%
\pgfpathlineto{\pgfqpoint{4.721969in}{3.380409in}}%
\pgfpathlineto{\pgfqpoint{4.725190in}{3.382226in}}%
\pgfpathlineto{\pgfqpoint{4.726264in}{3.381707in}}%
\pgfpathlineto{\pgfqpoint{4.727338in}{3.379370in}}%
\pgfpathlineto{\pgfqpoint{4.728412in}{3.379630in}}%
\pgfpathlineto{\pgfqpoint{4.729486in}{3.383005in}}%
\pgfpathlineto{\pgfqpoint{4.732707in}{3.383871in}}%
\pgfpathlineto{\pgfqpoint{4.733781in}{3.384996in}}%
\pgfpathlineto{\pgfqpoint{4.735929in}{3.384650in}}%
\pgfpathlineto{\pgfqpoint{4.737003in}{3.383611in}}%
\pgfpathlineto{\pgfqpoint{4.741298in}{3.383092in}}%
\pgfpathlineto{\pgfqpoint{4.742372in}{3.384736in}}%
\pgfpathlineto{\pgfqpoint{4.743446in}{3.383351in}}%
\pgfpathlineto{\pgfqpoint{4.747741in}{3.384130in}}%
\pgfpathlineto{\pgfqpoint{4.749889in}{3.387246in}}%
\pgfpathlineto{\pgfqpoint{4.750962in}{3.385775in}}%
\pgfpathlineto{\pgfqpoint{4.752036in}{3.386727in}}%
\pgfpathlineto{\pgfqpoint{4.755258in}{3.387246in}}%
\pgfpathlineto{\pgfqpoint{4.756332in}{3.388025in}}%
\pgfpathlineto{\pgfqpoint{4.757405in}{3.386986in}}%
\pgfpathlineto{\pgfqpoint{4.758479in}{3.391920in}}%
\pgfpathlineto{\pgfqpoint{4.759553in}{3.381880in}}%
\pgfpathlineto{\pgfqpoint{4.763849in}{3.380322in}}%
\pgfpathlineto{\pgfqpoint{4.764922in}{3.382140in}}%
\pgfpathlineto{\pgfqpoint{4.770292in}{3.380669in}}%
\pgfpathlineto{\pgfqpoint{4.771365in}{3.379543in}}%
\pgfpathlineto{\pgfqpoint{4.772439in}{3.380928in}}%
\pgfpathlineto{\pgfqpoint{4.774587in}{3.388890in}}%
\pgfpathlineto{\pgfqpoint{4.777808in}{3.389669in}}%
\pgfpathlineto{\pgfqpoint{4.778882in}{3.388890in}}%
\pgfpathlineto{\pgfqpoint{4.779956in}{3.385948in}}%
\pgfpathlineto{\pgfqpoint{4.781030in}{3.386294in}}%
\pgfpathlineto{\pgfqpoint{4.782104in}{3.384996in}}%
\pgfpathlineto{\pgfqpoint{4.785325in}{3.386208in}}%
\pgfpathlineto{\pgfqpoint{4.786399in}{3.387679in}}%
\pgfpathlineto{\pgfqpoint{4.787473in}{3.391314in}}%
\pgfpathlineto{\pgfqpoint{4.789621in}{3.390621in}}%
\pgfpathlineto{\pgfqpoint{4.793916in}{3.393391in}}%
\pgfpathlineto{\pgfqpoint{4.796064in}{3.395381in}}%
\pgfpathlineto{\pgfqpoint{4.797137in}{3.394603in}}%
\pgfpathlineto{\pgfqpoint{4.801433in}{3.397285in}}%
\pgfpathlineto{\pgfqpoint{4.807876in}{3.393564in}}%
\pgfpathlineto{\pgfqpoint{4.808950in}{3.394516in}}%
\pgfpathlineto{\pgfqpoint{4.811097in}{3.392525in}}%
\pgfpathlineto{\pgfqpoint{4.812171in}{3.394083in}}%
\pgfpathlineto{\pgfqpoint{4.815393in}{3.393910in}}%
\pgfpathlineto{\pgfqpoint{4.816467in}{3.401267in}}%
\pgfpathlineto{\pgfqpoint{4.817540in}{3.402478in}}%
\pgfpathlineto{\pgfqpoint{4.818614in}{3.401526in}}%
\pgfpathlineto{\pgfqpoint{4.819688in}{3.396420in}}%
\pgfpathlineto{\pgfqpoint{4.823983in}{3.394516in}}%
\pgfpathlineto{\pgfqpoint{4.827205in}{3.388804in}}%
\pgfpathlineto{\pgfqpoint{4.830426in}{3.388198in}}%
\pgfpathlineto{\pgfqpoint{4.831500in}{3.389496in}}%
\pgfpathlineto{\pgfqpoint{4.832574in}{3.387852in}}%
\pgfpathlineto{\pgfqpoint{4.833648in}{3.388285in}}%
\pgfpathlineto{\pgfqpoint{4.839017in}{3.381534in}}%
\pgfpathlineto{\pgfqpoint{4.840091in}{3.381015in}}%
\pgfpathlineto{\pgfqpoint{4.842239in}{3.386121in}}%
\pgfpathlineto{\pgfqpoint{4.845460in}{3.386727in}}%
\pgfpathlineto{\pgfqpoint{4.848682in}{3.389756in}}%
\pgfpathlineto{\pgfqpoint{4.849756in}{3.390189in}}%
\pgfpathlineto{\pgfqpoint{4.854051in}{3.390189in}}%
\pgfpathlineto{\pgfqpoint{4.855125in}{3.388285in}}%
\pgfpathlineto{\pgfqpoint{4.856199in}{3.390016in}}%
\pgfpathlineto{\pgfqpoint{4.857272in}{3.387679in}}%
\pgfpathlineto{\pgfqpoint{4.863715in}{3.390189in}}%
\pgfpathlineto{\pgfqpoint{4.864789in}{3.390189in}}%
\pgfpathlineto{\pgfqpoint{4.868011in}{3.388285in}}%
\pgfpathlineto{\pgfqpoint{4.869085in}{3.389064in}}%
\pgfpathlineto{\pgfqpoint{4.870159in}{3.388285in}}%
\pgfpathlineto{\pgfqpoint{4.871232in}{3.389237in}}%
\pgfpathlineto{\pgfqpoint{4.872306in}{3.389323in}}%
\pgfpathlineto{\pgfqpoint{4.875528in}{3.390794in}}%
\pgfpathlineto{\pgfqpoint{4.876602in}{3.389929in}}%
\pgfpathlineto{\pgfqpoint{4.877675in}{3.390189in}}%
\pgfpathlineto{\pgfqpoint{4.879823in}{3.388285in}}%
\pgfpathlineto{\pgfqpoint{4.884118in}{3.390621in}}%
\pgfpathlineto{\pgfqpoint{4.885192in}{3.392093in}}%
\pgfpathlineto{\pgfqpoint{4.886266in}{3.395122in}}%
\pgfpathlineto{\pgfqpoint{4.887340in}{3.393218in}}%
\pgfpathlineto{\pgfqpoint{4.890561in}{3.392872in}}%
\pgfpathlineto{\pgfqpoint{4.891635in}{3.395381in}}%
\pgfpathlineto{\pgfqpoint{4.893783in}{3.394256in}}%
\pgfpathlineto{\pgfqpoint{4.894857in}{3.396593in}}%
\pgfpathlineto{\pgfqpoint{4.898078in}{3.397978in}}%
\pgfpathlineto{\pgfqpoint{4.899152in}{3.399276in}}%
\pgfpathlineto{\pgfqpoint{4.900226in}{3.398584in}}%
\pgfpathlineto{\pgfqpoint{4.901300in}{3.402392in}}%
\pgfpathlineto{\pgfqpoint{4.902374in}{3.400574in}}%
\pgfpathlineto{\pgfqpoint{4.905595in}{3.402998in}}%
\pgfpathlineto{\pgfqpoint{4.906669in}{3.406113in}}%
\pgfpathlineto{\pgfqpoint{4.907743in}{3.406632in}}%
\pgfpathlineto{\pgfqpoint{4.909891in}{3.400661in}}%
\pgfpathlineto{\pgfqpoint{4.914186in}{3.405075in}}%
\pgfpathlineto{\pgfqpoint{4.916334in}{3.407065in}}%
\pgfpathlineto{\pgfqpoint{4.920629in}{3.406373in}}%
\pgfpathlineto{\pgfqpoint{4.922777in}{3.404902in}}%
\pgfpathlineto{\pgfqpoint{4.923850in}{3.404902in}}%
\pgfpathlineto{\pgfqpoint{4.924924in}{3.401267in}}%
\pgfpathlineto{\pgfqpoint{4.928146in}{3.401786in}}%
\pgfpathlineto{\pgfqpoint{4.930293in}{3.404469in}}%
\pgfpathlineto{\pgfqpoint{4.931367in}{3.402738in}}%
\pgfpathlineto{\pgfqpoint{4.932441in}{3.402392in}}%
\pgfpathlineto{\pgfqpoint{4.936737in}{3.402478in}}%
\pgfpathlineto{\pgfqpoint{4.937810in}{3.403776in}}%
\pgfpathlineto{\pgfqpoint{4.939958in}{3.403257in}}%
\pgfpathlineto{\pgfqpoint{4.944253in}{3.404382in}}%
\pgfpathlineto{\pgfqpoint{4.945327in}{3.403517in}}%
\pgfpathlineto{\pgfqpoint{4.947475in}{3.399709in}}%
\pgfpathlineto{\pgfqpoint{4.952844in}{3.402478in}}%
\pgfpathlineto{\pgfqpoint{4.953918in}{3.402132in}}%
\pgfpathlineto{\pgfqpoint{4.954992in}{3.403171in}}%
\pgfpathlineto{\pgfqpoint{4.961435in}{3.408190in}}%
\pgfpathlineto{\pgfqpoint{4.962509in}{3.410873in}}%
\pgfpathlineto{\pgfqpoint{4.965730in}{3.410441in}}%
\pgfpathlineto{\pgfqpoint{4.966804in}{3.413383in}}%
\pgfpathlineto{\pgfqpoint{4.967878in}{3.412950in}}%
\pgfpathlineto{\pgfqpoint{4.968952in}{3.413383in}}%
\pgfpathlineto{\pgfqpoint{4.970026in}{3.417191in}}%
\pgfpathlineto{\pgfqpoint{4.973247in}{3.415287in}}%
\pgfpathlineto{\pgfqpoint{4.974321in}{3.417710in}}%
\pgfpathlineto{\pgfqpoint{4.975395in}{3.415460in}}%
\pgfpathlineto{\pgfqpoint{4.976469in}{3.415633in}}%
\pgfpathlineto{\pgfqpoint{4.977542in}{3.429913in}}%
\pgfpathlineto{\pgfqpoint{4.980764in}{3.430952in}}%
\pgfpathlineto{\pgfqpoint{4.982912in}{3.430346in}}%
\pgfpathlineto{\pgfqpoint{4.985059in}{3.432423in}}%
\pgfpathlineto{\pgfqpoint{4.988281in}{3.432596in}}%
\pgfpathlineto{\pgfqpoint{4.990428in}{3.437443in}}%
\pgfpathlineto{\pgfqpoint{4.991502in}{3.436751in}}%
\pgfpathlineto{\pgfqpoint{4.992576in}{3.437876in}}%
\pgfpathlineto{\pgfqpoint{4.995798in}{3.437616in}}%
\pgfpathlineto{\pgfqpoint{4.999019in}{3.439434in}}%
\pgfpathlineto{\pgfqpoint{5.003314in}{3.438568in}}%
\pgfpathlineto{\pgfqpoint{5.004388in}{3.436837in}}%
\pgfpathlineto{\pgfqpoint{5.005462in}{3.437530in}}%
\pgfpathlineto{\pgfqpoint{5.006536in}{3.440299in}}%
\pgfpathlineto{\pgfqpoint{5.007610in}{3.440212in}}%
\pgfpathlineto{\pgfqpoint{5.010831in}{3.442030in}}%
\pgfpathlineto{\pgfqpoint{5.011905in}{3.443674in}}%
\pgfpathlineto{\pgfqpoint{5.012979in}{3.465657in}}%
\pgfpathlineto{\pgfqpoint{5.014053in}{3.458560in}}%
\pgfpathlineto{\pgfqpoint{5.015127in}{3.458560in}}%
\pgfpathlineto{\pgfqpoint{5.018348in}{3.461243in}}%
\pgfpathlineto{\pgfqpoint{5.019422in}{3.466696in}}%
\pgfpathlineto{\pgfqpoint{5.021570in}{3.462628in}}%
\pgfpathlineto{\pgfqpoint{5.026939in}{3.462195in}}%
\pgfpathlineto{\pgfqpoint{5.028013in}{3.463407in}}%
\pgfpathlineto{\pgfqpoint{5.029087in}{3.459945in}}%
\pgfpathlineto{\pgfqpoint{5.030160in}{3.458820in}}%
\pgfpathlineto{\pgfqpoint{5.033382in}{3.461157in}}%
\pgfpathlineto{\pgfqpoint{5.034456in}{3.453627in}}%
\pgfpathlineto{\pgfqpoint{5.035530in}{3.453800in}}%
\pgfpathlineto{\pgfqpoint{5.037677in}{3.451983in}}%
\pgfpathlineto{\pgfqpoint{5.041973in}{3.455964in}}%
\pgfpathlineto{\pgfqpoint{5.043047in}{3.463234in}}%
\pgfpathlineto{\pgfqpoint{5.044120in}{3.462022in}}%
\pgfpathlineto{\pgfqpoint{5.045194in}{3.463753in}}%
\pgfpathlineto{\pgfqpoint{5.048416in}{3.465571in}}%
\pgfpathlineto{\pgfqpoint{5.049490in}{3.465051in}}%
\pgfpathlineto{\pgfqpoint{5.050563in}{3.466263in}}%
\pgfpathlineto{\pgfqpoint{5.051637in}{3.471110in}}%
\pgfpathlineto{\pgfqpoint{5.052711in}{3.469552in}}%
\pgfpathlineto{\pgfqpoint{5.059154in}{3.467388in}}%
\pgfpathlineto{\pgfqpoint{5.060228in}{3.469379in}}%
\pgfpathlineto{\pgfqpoint{5.065597in}{3.466782in}}%
\pgfpathlineto{\pgfqpoint{5.066671in}{3.469292in}}%
\pgfpathlineto{\pgfqpoint{5.070966in}{3.472494in}}%
\pgfpathlineto{\pgfqpoint{5.072040in}{3.469292in}}%
\pgfpathlineto{\pgfqpoint{5.073114in}{3.470158in}}%
\pgfpathlineto{\pgfqpoint{5.074188in}{3.470158in}}%
\pgfpathlineto{\pgfqpoint{5.075262in}{3.467128in}}%
\pgfpathlineto{\pgfqpoint{5.078483in}{3.466523in}}%
\pgfpathlineto{\pgfqpoint{5.079557in}{3.469465in}}%
\pgfpathlineto{\pgfqpoint{5.080631in}{3.469811in}}%
\pgfpathlineto{\pgfqpoint{5.081705in}{3.471283in}}%
\pgfpathlineto{\pgfqpoint{5.082779in}{3.468600in}}%
\pgfpathlineto{\pgfqpoint{5.086000in}{3.467821in}}%
\pgfpathlineto{\pgfqpoint{5.087074in}{3.465657in}}%
\pgfpathlineto{\pgfqpoint{5.088148in}{3.468080in}}%
\pgfpathlineto{\pgfqpoint{5.089222in}{3.463493in}}%
\pgfpathlineto{\pgfqpoint{5.090295in}{3.464445in}}%
\pgfpathlineto{\pgfqpoint{5.093517in}{3.469206in}}%
\pgfpathlineto{\pgfqpoint{5.094591in}{3.468600in}}%
\pgfpathlineto{\pgfqpoint{5.096738in}{3.458820in}}%
\pgfpathlineto{\pgfqpoint{5.097812in}{3.462715in}}%
\pgfpathlineto{\pgfqpoint{5.101034in}{3.463320in}}%
\pgfpathlineto{\pgfqpoint{5.102108in}{3.458474in}}%
\pgfpathlineto{\pgfqpoint{5.103181in}{3.464532in}}%
\pgfpathlineto{\pgfqpoint{5.104255in}{3.459599in}}%
\pgfpathlineto{\pgfqpoint{5.105329in}{3.446790in}}%
\pgfpathlineto{\pgfqpoint{5.108551in}{3.443415in}}%
\pgfpathlineto{\pgfqpoint{5.109625in}{3.448521in}}%
\pgfpathlineto{\pgfqpoint{5.111772in}{3.438741in}}%
\pgfpathlineto{\pgfqpoint{5.112846in}{3.442809in}}%
\pgfpathlineto{\pgfqpoint{5.116068in}{3.444280in}}%
\pgfpathlineto{\pgfqpoint{5.117141in}{3.451983in}}%
\pgfpathlineto{\pgfqpoint{5.118215in}{3.449473in}}%
\pgfpathlineto{\pgfqpoint{5.120363in}{3.456310in}}%
\pgfpathlineto{\pgfqpoint{5.123584in}{3.456483in}}%
\pgfpathlineto{\pgfqpoint{5.124658in}{3.460551in}}%
\pgfpathlineto{\pgfqpoint{5.125732in}{3.461849in}}%
\pgfpathlineto{\pgfqpoint{5.126806in}{3.451810in}}%
\pgfpathlineto{\pgfqpoint{5.127880in}{3.462541in}}%
\pgfpathlineto{\pgfqpoint{5.131101in}{3.464792in}}%
\pgfpathlineto{\pgfqpoint{5.132175in}{3.466523in}}%
\pgfpathlineto{\pgfqpoint{5.133249in}{3.462368in}}%
\pgfpathlineto{\pgfqpoint{5.134323in}{3.462801in}}%
\pgfpathlineto{\pgfqpoint{5.135397in}{3.460984in}}%
\pgfpathlineto{\pgfqpoint{5.138618in}{3.458560in}}%
\pgfpathlineto{\pgfqpoint{5.140766in}{3.459426in}}%
\pgfpathlineto{\pgfqpoint{5.142914in}{3.463753in}}%
\pgfpathlineto{\pgfqpoint{5.146135in}{3.466003in}}%
\pgfpathlineto{\pgfqpoint{5.147209in}{3.469552in}}%
\pgfpathlineto{\pgfqpoint{5.148283in}{3.466782in}}%
\pgfpathlineto{\pgfqpoint{5.149357in}{3.478899in}}%
\pgfpathlineto{\pgfqpoint{5.150430in}{3.476216in}}%
\pgfpathlineto{\pgfqpoint{5.153652in}{3.481149in}}%
\pgfpathlineto{\pgfqpoint{5.154726in}{3.481668in}}%
\pgfpathlineto{\pgfqpoint{5.155800in}{3.486082in}}%
\pgfpathlineto{\pgfqpoint{5.157947in}{3.488765in}}%
\pgfpathlineto{\pgfqpoint{5.161169in}{3.488159in}}%
\pgfpathlineto{\pgfqpoint{5.162243in}{3.491361in}}%
\pgfpathlineto{\pgfqpoint{5.163316in}{3.490150in}}%
\pgfpathlineto{\pgfqpoint{5.164390in}{3.490323in}}%
\pgfpathlineto{\pgfqpoint{5.165464in}{3.491967in}}%
\pgfpathlineto{\pgfqpoint{5.168686in}{3.488332in}}%
\pgfpathlineto{\pgfqpoint{5.170833in}{3.482447in}}%
\pgfpathlineto{\pgfqpoint{5.171907in}{3.484611in}}%
\pgfpathlineto{\pgfqpoint{5.172981in}{3.481062in}}%
\pgfpathlineto{\pgfqpoint{5.176202in}{3.478639in}}%
\pgfpathlineto{\pgfqpoint{5.177276in}{3.475956in}}%
\pgfpathlineto{\pgfqpoint{5.179424in}{3.487034in}}%
\pgfpathlineto{\pgfqpoint{5.180498in}{3.482101in}}%
\pgfpathlineto{\pgfqpoint{5.184793in}{3.490150in}}%
\pgfpathlineto{\pgfqpoint{5.185867in}{3.490150in}}%
\pgfpathlineto{\pgfqpoint{5.188015in}{3.491015in}}%
\pgfpathlineto{\pgfqpoint{5.191236in}{3.488246in}}%
\pgfpathlineto{\pgfqpoint{5.193384in}{3.481495in}}%
\pgfpathlineto{\pgfqpoint{5.195532in}{3.482014in}}%
\pgfpathlineto{\pgfqpoint{5.198753in}{3.478899in}}%
\pgfpathlineto{\pgfqpoint{5.199827in}{3.473793in}}%
\pgfpathlineto{\pgfqpoint{5.201975in}{3.484524in}}%
\pgfpathlineto{\pgfqpoint{5.203048in}{3.485044in}}%
\pgfpathlineto{\pgfqpoint{5.207344in}{3.483053in}}%
\pgfpathlineto{\pgfqpoint{5.209491in}{3.480716in}}%
\pgfpathlineto{\pgfqpoint{5.210565in}{3.482707in}}%
\pgfpathlineto{\pgfqpoint{5.214861in}{3.479937in}}%
\pgfpathlineto{\pgfqpoint{5.217008in}{3.486169in}}%
\pgfpathlineto{\pgfqpoint{5.218082in}{3.482707in}}%
\pgfpathlineto{\pgfqpoint{5.221304in}{3.477860in}}%
\pgfpathlineto{\pgfqpoint{5.222378in}{3.465571in}}%
\pgfpathlineto{\pgfqpoint{5.223451in}{3.462455in}}%
\pgfpathlineto{\pgfqpoint{5.224525in}{3.465744in}}%
\pgfpathlineto{\pgfqpoint{5.225599in}{3.456916in}}%
\pgfpathlineto{\pgfqpoint{5.228821in}{3.461503in}}%
\pgfpathlineto{\pgfqpoint{5.229894in}{3.461849in}}%
\pgfpathlineto{\pgfqpoint{5.230968in}{3.462974in}}%
\pgfpathlineto{\pgfqpoint{5.232042in}{3.465571in}}%
\pgfpathlineto{\pgfqpoint{5.233116in}{3.460637in}}%
\pgfpathlineto{\pgfqpoint{5.236337in}{3.457868in}}%
\pgfpathlineto{\pgfqpoint{5.237411in}{3.463667in}}%
\pgfpathlineto{\pgfqpoint{5.238485in}{3.462541in}}%
\pgfpathlineto{\pgfqpoint{5.239559in}{3.466955in}}%
\pgfpathlineto{\pgfqpoint{5.240633in}{3.468773in}}%
\pgfpathlineto{\pgfqpoint{5.244928in}{3.471629in}}%
\pgfpathlineto{\pgfqpoint{5.246002in}{3.468080in}}%
\pgfpathlineto{\pgfqpoint{5.247076in}{3.467561in}}%
\pgfpathlineto{\pgfqpoint{5.248150in}{3.469119in}}%
\pgfpathlineto{\pgfqpoint{5.251371in}{3.464186in}}%
\pgfpathlineto{\pgfqpoint{5.252445in}{3.469119in}}%
\pgfpathlineto{\pgfqpoint{5.255667in}{3.460291in}}%
\pgfpathlineto{\pgfqpoint{5.258888in}{3.466436in}}%
\pgfpathlineto{\pgfqpoint{5.261036in}{3.466869in}}%
\pgfpathlineto{\pgfqpoint{5.263183in}{3.459859in}}%
\pgfpathlineto{\pgfqpoint{5.266405in}{3.456224in}}%
\pgfpathlineto{\pgfqpoint{5.267479in}{3.448434in}}%
\pgfpathlineto{\pgfqpoint{5.268553in}{3.453281in}}%
\pgfpathlineto{\pgfqpoint{5.269626in}{3.441597in}}%
\pgfpathlineto{\pgfqpoint{5.270700in}{3.442549in}}%
\pgfpathlineto{\pgfqpoint{5.273922in}{3.441857in}}%
\pgfpathlineto{\pgfqpoint{5.274996in}{3.440039in}}%
\pgfpathlineto{\pgfqpoint{5.276069in}{3.442290in}}%
\pgfpathlineto{\pgfqpoint{5.277143in}{3.441165in}}%
\pgfpathlineto{\pgfqpoint{5.278217in}{3.445492in}}%
\pgfpathlineto{\pgfqpoint{5.281439in}{3.444626in}}%
\pgfpathlineto{\pgfqpoint{5.282513in}{3.441511in}}%
\pgfpathlineto{\pgfqpoint{5.283586in}{3.434674in}}%
\pgfpathlineto{\pgfqpoint{5.284660in}{3.436145in}}%
\pgfpathlineto{\pgfqpoint{5.285734in}{3.450771in}}%
\pgfpathlineto{\pgfqpoint{5.290029in}{3.445146in}}%
\pgfpathlineto{\pgfqpoint{5.291103in}{3.441684in}}%
\pgfpathlineto{\pgfqpoint{5.292177in}{3.441684in}}%
\pgfpathlineto{\pgfqpoint{5.296472in}{3.443415in}}%
\pgfpathlineto{\pgfqpoint{5.297546in}{3.445146in}}%
\pgfpathlineto{\pgfqpoint{5.298620in}{3.445492in}}%
\pgfpathlineto{\pgfqpoint{5.299694in}{3.444973in}}%
\pgfpathlineto{\pgfqpoint{5.300768in}{3.450252in}}%
\pgfpathlineto{\pgfqpoint{5.303989in}{3.448694in}}%
\pgfpathlineto{\pgfqpoint{5.305063in}{3.446877in}}%
\pgfpathlineto{\pgfqpoint{5.306137in}{3.457089in}}%
\pgfpathlineto{\pgfqpoint{5.307211in}{3.457349in}}%
\pgfpathlineto{\pgfqpoint{5.308285in}{3.454320in}}%
\pgfpathlineto{\pgfqpoint{5.311506in}{3.456310in}}%
\pgfpathlineto{\pgfqpoint{5.312580in}{3.454060in}}%
\pgfpathlineto{\pgfqpoint{5.313654in}{3.456050in}}%
\pgfpathlineto{\pgfqpoint{5.315802in}{3.451377in}}%
\pgfpathlineto{\pgfqpoint{5.319023in}{3.454579in}}%
\pgfpathlineto{\pgfqpoint{5.320097in}{3.458560in}}%
\pgfpathlineto{\pgfqpoint{5.321171in}{3.457522in}}%
\pgfpathlineto{\pgfqpoint{5.322245in}{3.454925in}}%
\pgfpathlineto{\pgfqpoint{5.323318in}{3.461589in}}%
\pgfpathlineto{\pgfqpoint{5.326540in}{3.461676in}}%
\pgfpathlineto{\pgfqpoint{5.328688in}{3.454233in}}%
\pgfpathlineto{\pgfqpoint{5.329761in}{3.454320in}}%
\pgfpathlineto{\pgfqpoint{5.330835in}{3.458647in}}%
\pgfpathlineto{\pgfqpoint{5.334057in}{3.457781in}}%
\pgfpathlineto{\pgfqpoint{5.335131in}{3.454406in}}%
\pgfpathlineto{\pgfqpoint{5.337278in}{3.459945in}}%
\pgfpathlineto{\pgfqpoint{5.338352in}{3.460118in}}%
\pgfpathlineto{\pgfqpoint{5.341574in}{3.463320in}}%
\pgfpathlineto{\pgfqpoint{5.342647in}{3.461330in}}%
\pgfpathlineto{\pgfqpoint{5.344795in}{3.464445in}}%
\pgfpathlineto{\pgfqpoint{5.350164in}{3.460984in}}%
\pgfpathlineto{\pgfqpoint{5.352312in}{3.467907in}}%
\pgfpathlineto{\pgfqpoint{5.353386in}{3.471369in}}%
\pgfpathlineto{\pgfqpoint{5.356607in}{3.467215in}}%
\pgfpathlineto{\pgfqpoint{5.359829in}{3.454925in}}%
\pgfpathlineto{\pgfqpoint{5.360903in}{3.451290in}}%
\pgfpathlineto{\pgfqpoint{5.364124in}{3.447136in}}%
\pgfpathlineto{\pgfqpoint{5.365198in}{3.446790in}}%
\pgfpathlineto{\pgfqpoint{5.366272in}{3.451117in}}%
\pgfpathlineto{\pgfqpoint{5.367346in}{3.451377in}}%
\pgfpathlineto{\pgfqpoint{5.368420in}{3.447309in}}%
\pgfpathlineto{\pgfqpoint{5.371641in}{3.447829in}}%
\pgfpathlineto{\pgfqpoint{5.373789in}{3.452156in}}%
\pgfpathlineto{\pgfqpoint{5.374863in}{3.455445in}}%
\pgfpathlineto{\pgfqpoint{5.375936in}{3.453021in}}%
\pgfpathlineto{\pgfqpoint{5.379158in}{3.454493in}}%
\pgfpathlineto{\pgfqpoint{5.381306in}{3.451810in}}%
\pgfpathlineto{\pgfqpoint{5.382379in}{3.452416in}}%
\pgfpathlineto{\pgfqpoint{5.383453in}{3.444973in}}%
\pgfpathlineto{\pgfqpoint{5.386675in}{3.440126in}}%
\pgfpathlineto{\pgfqpoint{5.387749in}{3.440386in}}%
\pgfpathlineto{\pgfqpoint{5.388823in}{3.438568in}}%
\pgfpathlineto{\pgfqpoint{5.389896in}{3.441424in}}%
\pgfpathlineto{\pgfqpoint{5.395266in}{3.436404in}}%
\pgfpathlineto{\pgfqpoint{5.397413in}{3.428961in}}%
\pgfpathlineto{\pgfqpoint{5.398487in}{3.430779in}}%
\pgfpathlineto{\pgfqpoint{5.401709in}{3.435106in}}%
\pgfpathlineto{\pgfqpoint{5.402782in}{3.434500in}}%
\pgfpathlineto{\pgfqpoint{5.403856in}{3.434760in}}%
\pgfpathlineto{\pgfqpoint{5.404930in}{3.436404in}}%
\pgfpathlineto{\pgfqpoint{5.406004in}{3.433116in}}%
\pgfpathlineto{\pgfqpoint{5.409225in}{3.430260in}}%
\pgfpathlineto{\pgfqpoint{5.410299in}{3.427317in}}%
\pgfpathlineto{\pgfqpoint{5.411373in}{3.426452in}}%
\pgfpathlineto{\pgfqpoint{5.412447in}{3.426365in}}%
\pgfpathlineto{\pgfqpoint{5.413521in}{3.422297in}}%
\pgfpathlineto{\pgfqpoint{5.416742in}{3.424461in}}%
\pgfpathlineto{\pgfqpoint{5.417816in}{3.429135in}}%
\pgfpathlineto{\pgfqpoint{5.418890in}{3.429567in}}%
\pgfpathlineto{\pgfqpoint{5.419964in}{3.428788in}}%
\pgfpathlineto{\pgfqpoint{5.425333in}{3.430433in}}%
\pgfpathlineto{\pgfqpoint{5.426407in}{3.432250in}}%
\pgfpathlineto{\pgfqpoint{5.428555in}{3.430433in}}%
\pgfpathlineto{\pgfqpoint{5.431776in}{3.436318in}}%
\pgfpathlineto{\pgfqpoint{5.432850in}{3.431125in}}%
\pgfpathlineto{\pgfqpoint{5.433924in}{3.434847in}}%
\pgfpathlineto{\pgfqpoint{5.434998in}{3.430346in}}%
\pgfpathlineto{\pgfqpoint{5.436071in}{3.431471in}}%
\pgfpathlineto{\pgfqpoint{5.439293in}{3.431904in}}%
\pgfpathlineto{\pgfqpoint{5.440367in}{3.430606in}}%
\pgfpathlineto{\pgfqpoint{5.442514in}{3.419961in}}%
\pgfpathlineto{\pgfqpoint{5.443588in}{3.412431in}}%
\pgfpathlineto{\pgfqpoint{5.446810in}{3.410008in}}%
\pgfpathlineto{\pgfqpoint{5.447884in}{3.407065in}}%
\pgfpathlineto{\pgfqpoint{5.448957in}{3.418143in}}%
\pgfpathlineto{\pgfqpoint{5.450031in}{3.421432in}}%
\pgfpathlineto{\pgfqpoint{5.451105in}{3.426798in}}%
\pgfpathlineto{\pgfqpoint{5.454327in}{3.427750in}}%
\pgfpathlineto{\pgfqpoint{5.455401in}{3.422211in}}%
\pgfpathlineto{\pgfqpoint{5.457548in}{3.431904in}}%
\pgfpathlineto{\pgfqpoint{5.458622in}{3.427577in}}%
\pgfpathlineto{\pgfqpoint{5.462917in}{3.435193in}}%
\pgfpathlineto{\pgfqpoint{5.463991in}{3.433202in}}%
\pgfpathlineto{\pgfqpoint{5.465065in}{3.433375in}}%
\pgfpathlineto{\pgfqpoint{5.466139in}{3.434933in}}%
\pgfpathlineto{\pgfqpoint{5.469360in}{3.434327in}}%
\pgfpathlineto{\pgfqpoint{5.470434in}{3.437010in}}%
\pgfpathlineto{\pgfqpoint{5.471508in}{3.437270in}}%
\pgfpathlineto{\pgfqpoint{5.472582in}{3.436837in}}%
\pgfpathlineto{\pgfqpoint{5.473656in}{3.431471in}}%
\pgfpathlineto{\pgfqpoint{5.476877in}{3.432596in}}%
\pgfpathlineto{\pgfqpoint{5.477951in}{3.428788in}}%
\pgfpathlineto{\pgfqpoint{5.479025in}{3.429308in}}%
\pgfpathlineto{\pgfqpoint{5.480099in}{3.427317in}}%
\pgfpathlineto{\pgfqpoint{5.481173in}{3.429827in}}%
\pgfpathlineto{\pgfqpoint{5.484394in}{3.429481in}}%
\pgfpathlineto{\pgfqpoint{5.485468in}{3.433202in}}%
\pgfpathlineto{\pgfqpoint{5.486542in}{3.440126in}}%
\pgfpathlineto{\pgfqpoint{5.487616in}{3.439087in}}%
\pgfpathlineto{\pgfqpoint{5.488690in}{3.442982in}}%
\pgfpathlineto{\pgfqpoint{5.491911in}{3.448434in}}%
\pgfpathlineto{\pgfqpoint{5.495133in}{3.458560in}}%
\pgfpathlineto{\pgfqpoint{5.496206in}{3.455618in}}%
\pgfpathlineto{\pgfqpoint{5.499428in}{3.456137in}}%
\pgfpathlineto{\pgfqpoint{5.500502in}{3.454839in}}%
\pgfpathlineto{\pgfqpoint{5.501576in}{3.460724in}}%
\pgfpathlineto{\pgfqpoint{5.502649in}{3.460378in}}%
\pgfpathlineto{\pgfqpoint{5.503723in}{3.462628in}}%
\pgfpathlineto{\pgfqpoint{5.506945in}{3.466869in}}%
\pgfpathlineto{\pgfqpoint{5.509092in}{3.465138in}}%
\pgfpathlineto{\pgfqpoint{5.511240in}{3.476995in}}%
\pgfpathlineto{\pgfqpoint{5.515535in}{3.473706in}}%
\pgfpathlineto{\pgfqpoint{5.516609in}{3.475523in}}%
\pgfpathlineto{\pgfqpoint{5.517683in}{3.470244in}}%
\pgfpathlineto{\pgfqpoint{5.518757in}{3.468946in}}%
\pgfpathlineto{\pgfqpoint{5.521979in}{3.470850in}}%
\pgfpathlineto{\pgfqpoint{5.524126in}{3.473100in}}%
\pgfpathlineto{\pgfqpoint{5.530569in}{3.465744in}}%
\pgfpathlineto{\pgfqpoint{5.533791in}{3.457176in}}%
\pgfpathlineto{\pgfqpoint{5.537012in}{3.457089in}}%
\pgfpathlineto{\pgfqpoint{5.539160in}{3.465311in}}%
\pgfpathlineto{\pgfqpoint{5.540234in}{3.474225in}}%
\pgfpathlineto{\pgfqpoint{5.541308in}{3.476995in}}%
\pgfpathlineto{\pgfqpoint{5.545603in}{3.474658in}}%
\pgfpathlineto{\pgfqpoint{5.546677in}{3.475437in}}%
\pgfpathlineto{\pgfqpoint{5.548824in}{3.475437in}}%
\pgfpathlineto{\pgfqpoint{5.552046in}{3.477860in}}%
\pgfpathlineto{\pgfqpoint{5.553120in}{3.480370in}}%
\pgfpathlineto{\pgfqpoint{5.554194in}{3.478379in}}%
\pgfpathlineto{\pgfqpoint{5.555267in}{3.472235in}}%
\pgfpathlineto{\pgfqpoint{5.556341in}{3.479245in}}%
\pgfpathlineto{\pgfqpoint{5.559563in}{3.479591in}}%
\pgfpathlineto{\pgfqpoint{5.560637in}{3.477774in}}%
\pgfpathlineto{\pgfqpoint{5.561711in}{3.478206in}}%
\pgfpathlineto{\pgfqpoint{5.562784in}{3.477860in}}%
\pgfpathlineto{\pgfqpoint{5.563858in}{3.473966in}}%
\pgfpathlineto{\pgfqpoint{5.567080in}{3.475523in}}%
\pgfpathlineto{\pgfqpoint{5.568154in}{3.481062in}}%
\pgfpathlineto{\pgfqpoint{5.569227in}{3.482014in}}%
\pgfpathlineto{\pgfqpoint{5.570301in}{3.478985in}}%
\pgfpathlineto{\pgfqpoint{5.571375in}{3.470850in}}%
\pgfpathlineto{\pgfqpoint{5.574597in}{3.473793in}}%
\pgfpathlineto{\pgfqpoint{5.576744in}{3.479678in}}%
\pgfpathlineto{\pgfqpoint{5.582113in}{3.479158in}}%
\pgfpathlineto{\pgfqpoint{5.583187in}{3.483140in}}%
\pgfpathlineto{\pgfqpoint{5.585335in}{3.475437in}}%
\pgfpathlineto{\pgfqpoint{5.590704in}{3.470590in}}%
\pgfpathlineto{\pgfqpoint{5.591778in}{3.464705in}}%
\pgfpathlineto{\pgfqpoint{5.592852in}{3.455012in}}%
\pgfpathlineto{\pgfqpoint{5.593926in}{3.452502in}}%
\pgfpathlineto{\pgfqpoint{5.597147in}{3.456743in}}%
\pgfpathlineto{\pgfqpoint{5.598221in}{3.461589in}}%
\pgfpathlineto{\pgfqpoint{5.599295in}{3.455618in}}%
\pgfpathlineto{\pgfqpoint{5.600369in}{3.462022in}}%
\pgfpathlineto{\pgfqpoint{5.601443in}{3.438828in}}%
\pgfpathlineto{\pgfqpoint{5.605738in}{3.439174in}}%
\pgfpathlineto{\pgfqpoint{5.606812in}{3.437530in}}%
\pgfpathlineto{\pgfqpoint{5.607886in}{3.438049in}}%
\pgfpathlineto{\pgfqpoint{5.608959in}{3.440126in}}%
\pgfpathlineto{\pgfqpoint{5.612181in}{3.437616in}}%
\pgfpathlineto{\pgfqpoint{5.613255in}{3.440212in}}%
\pgfpathlineto{\pgfqpoint{5.614329in}{3.439260in}}%
\pgfpathlineto{\pgfqpoint{5.615402in}{3.440472in}}%
\pgfpathlineto{\pgfqpoint{5.616476in}{3.448694in}}%
\pgfpathlineto{\pgfqpoint{5.619698in}{3.447136in}}%
\pgfpathlineto{\pgfqpoint{5.620772in}{3.439174in}}%
\pgfpathlineto{\pgfqpoint{5.621845in}{3.437530in}}%
\pgfpathlineto{\pgfqpoint{5.622919in}{3.440991in}}%
\pgfpathlineto{\pgfqpoint{5.623993in}{3.435193in}}%
\pgfpathlineto{\pgfqpoint{5.628289in}{3.433375in}}%
\pgfpathlineto{\pgfqpoint{5.629362in}{3.428875in}}%
\pgfpathlineto{\pgfqpoint{5.630436in}{3.428788in}}%
\pgfpathlineto{\pgfqpoint{5.631510in}{3.432077in}}%
\pgfpathlineto{\pgfqpoint{5.635805in}{3.433202in}}%
\pgfpathlineto{\pgfqpoint{5.636879in}{3.438568in}}%
\pgfpathlineto{\pgfqpoint{5.637953in}{3.438222in}}%
\pgfpathlineto{\pgfqpoint{5.639027in}{3.432596in}}%
\pgfpathlineto{\pgfqpoint{5.642248in}{3.437616in}}%
\pgfpathlineto{\pgfqpoint{5.643322in}{3.433289in}}%
\pgfpathlineto{\pgfqpoint{5.646544in}{3.441165in}}%
\pgfpathlineto{\pgfqpoint{5.649765in}{3.439520in}}%
\pgfpathlineto{\pgfqpoint{5.650839in}{3.445665in}}%
\pgfpathlineto{\pgfqpoint{5.651913in}{3.447050in}}%
\pgfpathlineto{\pgfqpoint{5.654061in}{3.447742in}}%
\pgfpathlineto{\pgfqpoint{5.657282in}{3.450165in}}%
\pgfpathlineto{\pgfqpoint{5.658356in}{3.447136in}}%
\pgfpathlineto{\pgfqpoint{5.660504in}{3.452589in}}%
\pgfpathlineto{\pgfqpoint{5.661578in}{3.456570in}}%
\pgfpathlineto{\pgfqpoint{5.664799in}{3.453973in}}%
\pgfpathlineto{\pgfqpoint{5.665873in}{3.455704in}}%
\pgfpathlineto{\pgfqpoint{5.666947in}{3.456050in}}%
\pgfpathlineto{\pgfqpoint{5.668021in}{3.458301in}}%
\pgfpathlineto{\pgfqpoint{5.669094in}{3.463840in}}%
\pgfpathlineto{\pgfqpoint{5.673390in}{3.460984in}}%
\pgfpathlineto{\pgfqpoint{5.674464in}{3.458474in}}%
\pgfpathlineto{\pgfqpoint{5.675537in}{3.457522in}}%
\pgfpathlineto{\pgfqpoint{5.679833in}{3.457695in}}%
\pgfpathlineto{\pgfqpoint{5.681980in}{3.464099in}}%
\pgfpathlineto{\pgfqpoint{5.683054in}{3.461243in}}%
\pgfpathlineto{\pgfqpoint{5.684128in}{3.462022in}}%
\pgfpathlineto{\pgfqpoint{5.688423in}{3.457695in}}%
\pgfpathlineto{\pgfqpoint{5.689497in}{3.459080in}}%
\pgfpathlineto{\pgfqpoint{5.690571in}{3.454925in}}%
\pgfpathlineto{\pgfqpoint{5.691645in}{3.455618in}}%
\pgfpathlineto{\pgfqpoint{5.694867in}{3.455877in}}%
\pgfpathlineto{\pgfqpoint{5.697014in}{3.459512in}}%
\pgfpathlineto{\pgfqpoint{5.699162in}{3.454233in}}%
\pgfpathlineto{\pgfqpoint{5.702383in}{3.455704in}}%
\pgfpathlineto{\pgfqpoint{5.703457in}{3.455358in}}%
\pgfpathlineto{\pgfqpoint{5.704531in}{3.458474in}}%
\pgfpathlineto{\pgfqpoint{5.705605in}{3.458214in}}%
\pgfpathlineto{\pgfqpoint{5.706679in}{3.455618in}}%
\pgfpathlineto{\pgfqpoint{5.709900in}{3.453714in}}%
\pgfpathlineto{\pgfqpoint{5.710974in}{3.453800in}}%
\pgfpathlineto{\pgfqpoint{5.712048in}{3.456483in}}%
\pgfpathlineto{\pgfqpoint{5.714196in}{3.444973in}}%
\pgfpathlineto{\pgfqpoint{5.717417in}{3.447569in}}%
\pgfpathlineto{\pgfqpoint{5.719565in}{3.443588in}}%
\pgfpathlineto{\pgfqpoint{5.720639in}{3.444021in}}%
\pgfpathlineto{\pgfqpoint{5.721712in}{3.445146in}}%
\pgfpathlineto{\pgfqpoint{5.724934in}{3.443242in}}%
\pgfpathlineto{\pgfqpoint{5.726008in}{3.445925in}}%
\pgfpathlineto{\pgfqpoint{5.727082in}{3.445232in}}%
\pgfpathlineto{\pgfqpoint{5.728155in}{3.442895in}}%
\pgfpathlineto{\pgfqpoint{5.732451in}{3.447915in}}%
\pgfpathlineto{\pgfqpoint{5.733525in}{3.444626in}}%
\pgfpathlineto{\pgfqpoint{5.734599in}{3.444713in}}%
\pgfpathlineto{\pgfqpoint{5.735672in}{3.441857in}}%
\pgfpathlineto{\pgfqpoint{5.736746in}{3.446011in}}%
\pgfpathlineto{\pgfqpoint{5.739968in}{3.446617in}}%
\pgfpathlineto{\pgfqpoint{5.741042in}{3.453194in}}%
\pgfpathlineto{\pgfqpoint{5.742115in}{3.455791in}}%
\pgfpathlineto{\pgfqpoint{5.744263in}{3.457262in}}%
\pgfpathlineto{\pgfqpoint{5.749632in}{3.457955in}}%
\pgfpathlineto{\pgfqpoint{5.750706in}{3.458733in}}%
\pgfpathlineto{\pgfqpoint{5.751780in}{3.457608in}}%
\pgfpathlineto{\pgfqpoint{5.755001in}{3.458128in}}%
\pgfpathlineto{\pgfqpoint{5.756075in}{3.459685in}}%
\pgfpathlineto{\pgfqpoint{5.758223in}{3.460205in}}%
\pgfpathlineto{\pgfqpoint{5.759297in}{3.460984in}}%
\pgfpathlineto{\pgfqpoint{5.762518in}{3.462022in}}%
\pgfpathlineto{\pgfqpoint{5.763592in}{3.461763in}}%
\pgfpathlineto{\pgfqpoint{5.764666in}{3.457608in}}%
\pgfpathlineto{\pgfqpoint{5.766814in}{3.458733in}}%
\pgfpathlineto{\pgfqpoint{5.771109in}{3.463234in}}%
\pgfpathlineto{\pgfqpoint{5.772183in}{3.462974in}}%
\pgfpathlineto{\pgfqpoint{5.773257in}{3.468513in}}%
\pgfpathlineto{\pgfqpoint{5.774331in}{3.457089in}}%
\pgfpathlineto{\pgfqpoint{5.777552in}{3.450512in}}%
\pgfpathlineto{\pgfqpoint{5.778626in}{3.454233in}}%
\pgfpathlineto{\pgfqpoint{5.780774in}{3.467042in}}%
\pgfpathlineto{\pgfqpoint{5.781847in}{3.466609in}}%
\pgfpathlineto{\pgfqpoint{5.786143in}{3.466090in}}%
\pgfpathlineto{\pgfqpoint{5.788290in}{3.470158in}}%
\pgfpathlineto{\pgfqpoint{5.789364in}{3.476562in}}%
\pgfpathlineto{\pgfqpoint{5.792586in}{3.479505in}}%
\pgfpathlineto{\pgfqpoint{5.793660in}{3.484005in}}%
\pgfpathlineto{\pgfqpoint{5.794733in}{3.484524in}}%
\pgfpathlineto{\pgfqpoint{5.795807in}{3.486082in}}%
\pgfpathlineto{\pgfqpoint{5.796881in}{3.485044in}}%
\pgfpathlineto{\pgfqpoint{5.800103in}{3.484870in}}%
\pgfpathlineto{\pgfqpoint{5.801177in}{3.485649in}}%
\pgfpathlineto{\pgfqpoint{5.802250in}{3.489977in}}%
\pgfpathlineto{\pgfqpoint{5.803324in}{3.478639in}}%
\pgfpathlineto{\pgfqpoint{5.804398in}{3.481755in}}%
\pgfpathlineto{\pgfqpoint{5.807620in}{3.482014in}}%
\pgfpathlineto{\pgfqpoint{5.808693in}{3.485217in}}%
\pgfpathlineto{\pgfqpoint{5.809767in}{3.483140in}}%
\pgfpathlineto{\pgfqpoint{5.810841in}{3.482620in}}%
\pgfpathlineto{\pgfqpoint{5.811915in}{3.483313in}}%
\pgfpathlineto{\pgfqpoint{5.815136in}{3.483313in}}%
\pgfpathlineto{\pgfqpoint{5.816210in}{3.480976in}}%
\pgfpathlineto{\pgfqpoint{5.817284in}{3.480543in}}%
\pgfpathlineto{\pgfqpoint{5.819432in}{3.486428in}}%
\pgfpathlineto{\pgfqpoint{5.822653in}{3.486861in}}%
\pgfpathlineto{\pgfqpoint{5.823727in}{3.485909in}}%
\pgfpathlineto{\pgfqpoint{5.824801in}{3.482793in}}%
\pgfpathlineto{\pgfqpoint{5.825875in}{3.484005in}}%
\pgfpathlineto{\pgfqpoint{5.826949in}{3.483140in}}%
\pgfpathlineto{\pgfqpoint{5.830170in}{3.485822in}}%
\pgfpathlineto{\pgfqpoint{5.831244in}{3.488246in}}%
\pgfpathlineto{\pgfqpoint{5.832318in}{3.486688in}}%
\pgfpathlineto{\pgfqpoint{5.833392in}{3.486342in}}%
\pgfpathlineto{\pgfqpoint{5.834466in}{3.488505in}}%
\pgfpathlineto{\pgfqpoint{5.838761in}{3.489804in}}%
\pgfpathlineto{\pgfqpoint{5.839835in}{3.487813in}}%
\pgfpathlineto{\pgfqpoint{5.840909in}{3.487294in}}%
\pgfpathlineto{\pgfqpoint{5.841982in}{3.488678in}}%
\pgfpathlineto{\pgfqpoint{5.846278in}{3.492140in}}%
\pgfpathlineto{\pgfqpoint{5.848425in}{3.494737in}}%
\pgfpathlineto{\pgfqpoint{5.849499in}{3.495169in}}%
\pgfpathlineto{\pgfqpoint{5.853795in}{3.499151in}}%
\pgfpathlineto{\pgfqpoint{5.854868in}{3.498199in}}%
\pgfpathlineto{\pgfqpoint{5.855942in}{3.498112in}}%
\pgfpathlineto{\pgfqpoint{5.857016in}{3.490063in}}%
\pgfpathlineto{\pgfqpoint{5.860238in}{3.495169in}}%
\pgfpathlineto{\pgfqpoint{5.861311in}{3.491448in}}%
\pgfpathlineto{\pgfqpoint{5.862385in}{3.491535in}}%
\pgfpathlineto{\pgfqpoint{5.864533in}{3.507892in}}%
\pgfpathlineto{\pgfqpoint{5.867755in}{3.503824in}}%
\pgfpathlineto{\pgfqpoint{5.868828in}{3.503651in}}%
\pgfpathlineto{\pgfqpoint{5.869902in}{3.506161in}}%
\pgfpathlineto{\pgfqpoint{5.870976in}{3.506940in}}%
\pgfpathlineto{\pgfqpoint{5.872050in}{3.504084in}}%
\pgfpathlineto{\pgfqpoint{5.875271in}{3.499756in}}%
\pgfpathlineto{\pgfqpoint{5.877419in}{3.506074in}}%
\pgfpathlineto{\pgfqpoint{5.878493in}{3.505122in}}%
\pgfpathlineto{\pgfqpoint{5.879567in}{3.508584in}}%
\pgfpathlineto{\pgfqpoint{5.882788in}{3.507805in}}%
\pgfpathlineto{\pgfqpoint{5.883862in}{3.506853in}}%
\pgfpathlineto{\pgfqpoint{5.884936in}{3.510488in}}%
\pgfpathlineto{\pgfqpoint{5.887084in}{3.511354in}}%
\pgfpathlineto{\pgfqpoint{5.890305in}{3.510748in}}%
\pgfpathlineto{\pgfqpoint{5.891379in}{3.504690in}}%
\pgfpathlineto{\pgfqpoint{5.893527in}{3.502353in}}%
\pgfpathlineto{\pgfqpoint{5.894600in}{3.506161in}}%
\pgfpathlineto{\pgfqpoint{5.897822in}{3.504863in}}%
\pgfpathlineto{\pgfqpoint{5.898896in}{3.508584in}}%
\pgfpathlineto{\pgfqpoint{5.899970in}{3.490669in}}%
\pgfpathlineto{\pgfqpoint{5.901044in}{3.489977in}}%
\pgfpathlineto{\pgfqpoint{5.902117in}{3.487813in}}%
\pgfpathlineto{\pgfqpoint{5.905339in}{3.488678in}}%
\pgfpathlineto{\pgfqpoint{5.908560in}{3.485044in}}%
\pgfpathlineto{\pgfqpoint{5.909634in}{3.484524in}}%
\pgfpathlineto{\pgfqpoint{5.912856in}{3.485563in}}%
\pgfpathlineto{\pgfqpoint{5.913930in}{3.482707in}}%
\pgfpathlineto{\pgfqpoint{5.915003in}{3.483399in}}%
\pgfpathlineto{\pgfqpoint{5.917151in}{3.477514in}}%
\pgfpathlineto{\pgfqpoint{5.920373in}{3.486169in}}%
\pgfpathlineto{\pgfqpoint{5.922520in}{3.486688in}}%
\pgfpathlineto{\pgfqpoint{5.923594in}{3.484697in}}%
\pgfpathlineto{\pgfqpoint{5.924668in}{3.485563in}}%
\pgfpathlineto{\pgfqpoint{5.927889in}{3.484524in}}%
\pgfpathlineto{\pgfqpoint{5.928963in}{3.487986in}}%
\pgfpathlineto{\pgfqpoint{5.930037in}{3.487380in}}%
\pgfpathlineto{\pgfqpoint{5.931111in}{3.488852in}}%
\pgfpathlineto{\pgfqpoint{5.932185in}{3.488246in}}%
\pgfpathlineto{\pgfqpoint{5.935406in}{3.488505in}}%
\pgfpathlineto{\pgfqpoint{5.936480in}{3.492573in}}%
\pgfpathlineto{\pgfqpoint{5.937554in}{3.490323in}}%
\pgfpathlineto{\pgfqpoint{5.939702in}{3.492227in}}%
\pgfpathlineto{\pgfqpoint{5.942923in}{3.492833in}}%
\pgfpathlineto{\pgfqpoint{5.943997in}{3.491188in}}%
\pgfpathlineto{\pgfqpoint{5.946145in}{3.478726in}}%
\pgfpathlineto{\pgfqpoint{5.947219in}{3.481928in}}%
\pgfpathlineto{\pgfqpoint{5.950440in}{3.483745in}}%
\pgfpathlineto{\pgfqpoint{5.951514in}{3.486428in}}%
\pgfpathlineto{\pgfqpoint{5.952588in}{3.492746in}}%
\pgfpathlineto{\pgfqpoint{5.953662in}{3.494304in}}%
\pgfpathlineto{\pgfqpoint{5.957957in}{3.496468in}}%
\pgfpathlineto{\pgfqpoint{5.959031in}{3.503218in}}%
\pgfpathlineto{\pgfqpoint{5.960105in}{3.501141in}}%
\pgfpathlineto{\pgfqpoint{5.961178in}{3.503132in}}%
\pgfpathlineto{\pgfqpoint{5.962252in}{3.499237in}}%
\pgfpathlineto{\pgfqpoint{5.965474in}{3.503911in}}%
\pgfpathlineto{\pgfqpoint{5.966548in}{3.506507in}}%
\pgfpathlineto{\pgfqpoint{5.967621in}{3.504603in}}%
\pgfpathlineto{\pgfqpoint{5.968695in}{3.504257in}}%
\pgfpathlineto{\pgfqpoint{5.974065in}{3.505382in}}%
\pgfpathlineto{\pgfqpoint{5.975138in}{3.501834in}}%
\pgfpathlineto{\pgfqpoint{5.976212in}{3.502093in}}%
\pgfpathlineto{\pgfqpoint{5.977286in}{3.498891in}}%
\pgfpathlineto{\pgfqpoint{5.981581in}{3.501574in}}%
\pgfpathlineto{\pgfqpoint{5.982655in}{3.500016in}}%
\pgfpathlineto{\pgfqpoint{5.983729in}{3.499583in}}%
\pgfpathlineto{\pgfqpoint{5.984803in}{3.500622in}}%
\pgfpathlineto{\pgfqpoint{5.988024in}{3.501660in}}%
\pgfpathlineto{\pgfqpoint{5.989098in}{3.501055in}}%
\pgfpathlineto{\pgfqpoint{5.990172in}{3.504430in}}%
\pgfpathlineto{\pgfqpoint{5.991246in}{3.502439in}}%
\pgfpathlineto{\pgfqpoint{5.992320in}{3.503132in}}%
\pgfpathlineto{\pgfqpoint{5.997689in}{3.502872in}}%
\pgfpathlineto{\pgfqpoint{5.998763in}{3.501314in}}%
\pgfpathlineto{\pgfqpoint{5.999837in}{3.504343in}}%
\pgfpathlineto{\pgfqpoint{6.003058in}{3.502959in}}%
\pgfpathlineto{\pgfqpoint{6.004132in}{3.509796in}}%
\pgfpathlineto{\pgfqpoint{6.005206in}{3.511267in}}%
\pgfpathlineto{\pgfqpoint{6.006280in}{3.509277in}}%
\pgfpathlineto{\pgfqpoint{6.007354in}{3.512652in}}%
\pgfpathlineto{\pgfqpoint{6.010575in}{3.508151in}}%
\pgfpathlineto{\pgfqpoint{6.012723in}{3.500968in}}%
\pgfpathlineto{\pgfqpoint{6.014870in}{3.503045in}}%
\pgfpathlineto{\pgfqpoint{6.018092in}{3.500968in}}%
\pgfpathlineto{\pgfqpoint{6.020240in}{3.501920in}}%
\pgfpathlineto{\pgfqpoint{6.021313in}{3.494391in}}%
\pgfpathlineto{\pgfqpoint{6.022387in}{3.493439in}}%
\pgfpathlineto{\pgfqpoint{6.027756in}{3.499237in}}%
\pgfpathlineto{\pgfqpoint{6.028830in}{3.502180in}}%
\pgfpathlineto{\pgfqpoint{6.029904in}{3.502699in}}%
\pgfpathlineto{\pgfqpoint{6.034199in}{3.503045in}}%
\pgfpathlineto{\pgfqpoint{6.035273in}{3.499410in}}%
\pgfpathlineto{\pgfqpoint{6.036347in}{3.500276in}}%
\pgfpathlineto{\pgfqpoint{6.037421in}{3.503132in}}%
\pgfpathlineto{\pgfqpoint{6.040643in}{3.502959in}}%
\pgfpathlineto{\pgfqpoint{6.042790in}{3.498285in}}%
\pgfpathlineto{\pgfqpoint{6.044938in}{3.498026in}}%
\pgfpathlineto{\pgfqpoint{6.048159in}{3.495343in}}%
\pgfpathlineto{\pgfqpoint{6.049233in}{3.497160in}}%
\pgfpathlineto{\pgfqpoint{6.050307in}{3.495689in}}%
\pgfpathlineto{\pgfqpoint{6.052455in}{3.498112in}}%
\pgfpathlineto{\pgfqpoint{6.055676in}{3.491967in}}%
\pgfpathlineto{\pgfqpoint{6.056750in}{3.492140in}}%
\pgfpathlineto{\pgfqpoint{6.057824in}{3.491535in}}%
\pgfpathlineto{\pgfqpoint{6.058898in}{3.491794in}}%
\pgfpathlineto{\pgfqpoint{6.059972in}{3.492919in}}%
\pgfpathlineto{\pgfqpoint{6.063193in}{3.494217in}}%
\pgfpathlineto{\pgfqpoint{6.064267in}{3.491015in}}%
\pgfpathlineto{\pgfqpoint{6.065341in}{3.493698in}}%
\pgfpathlineto{\pgfqpoint{6.067488in}{3.491967in}}%
\pgfpathlineto{\pgfqpoint{6.070710in}{3.493871in}}%
\pgfpathlineto{\pgfqpoint{6.071784in}{3.495602in}}%
\pgfpathlineto{\pgfqpoint{6.072858in}{3.495343in}}%
\pgfpathlineto{\pgfqpoint{6.073932in}{3.496814in}}%
\pgfpathlineto{\pgfqpoint{6.075005in}{3.499410in}}%
\pgfpathlineto{\pgfqpoint{6.078227in}{3.500103in}}%
\pgfpathlineto{\pgfqpoint{6.079301in}{3.501055in}}%
\pgfpathlineto{\pgfqpoint{6.080375in}{3.500622in}}%
\pgfpathlineto{\pgfqpoint{6.081448in}{3.499064in}}%
\pgfpathlineto{\pgfqpoint{6.082522in}{3.499064in}}%
\pgfpathlineto{\pgfqpoint{6.087891in}{3.495775in}}%
\pgfpathlineto{\pgfqpoint{6.088965in}{3.492746in}}%
\pgfpathlineto{\pgfqpoint{6.093261in}{3.494564in}}%
\pgfpathlineto{\pgfqpoint{6.095408in}{3.498112in}}%
\pgfpathlineto{\pgfqpoint{6.097556in}{3.501401in}}%
\pgfpathlineto{\pgfqpoint{6.101851in}{3.505901in}}%
\pgfpathlineto{\pgfqpoint{6.102925in}{3.506334in}}%
\pgfpathlineto{\pgfqpoint{6.103999in}{3.510402in}}%
\pgfpathlineto{\pgfqpoint{6.105073in}{3.500016in}}%
\pgfpathlineto{\pgfqpoint{6.108294in}{3.501314in}}%
\pgfpathlineto{\pgfqpoint{6.109368in}{3.506680in}}%
\pgfpathlineto{\pgfqpoint{6.110442in}{3.509017in}}%
\pgfpathlineto{\pgfqpoint{6.111516in}{3.507892in}}%
\pgfpathlineto{\pgfqpoint{6.112590in}{3.507719in}}%
\pgfpathlineto{\pgfqpoint{6.116885in}{3.503997in}}%
\pgfpathlineto{\pgfqpoint{6.119033in}{3.498458in}}%
\pgfpathlineto{\pgfqpoint{6.120107in}{3.497160in}}%
\pgfpathlineto{\pgfqpoint{6.123328in}{3.497939in}}%
\pgfpathlineto{\pgfqpoint{6.124402in}{3.499497in}}%
\pgfpathlineto{\pgfqpoint{6.125476in}{3.493179in}}%
\pgfpathlineto{\pgfqpoint{6.127623in}{3.496121in}}%
\pgfpathlineto{\pgfqpoint{6.131919in}{3.499843in}}%
\pgfpathlineto{\pgfqpoint{6.134066in}{3.503132in}}%
\pgfpathlineto{\pgfqpoint{6.135140in}{3.503132in}}%
\pgfpathlineto{\pgfqpoint{6.141583in}{3.502007in}}%
\pgfpathlineto{\pgfqpoint{6.142657in}{3.503564in}}%
\pgfpathlineto{\pgfqpoint{6.145879in}{3.503738in}}%
\pgfpathlineto{\pgfqpoint{6.146953in}{3.502007in}}%
\pgfpathlineto{\pgfqpoint{6.149100in}{3.504949in}}%
\pgfpathlineto{\pgfqpoint{6.150174in}{3.498631in}}%
\pgfpathlineto{\pgfqpoint{6.153396in}{3.498804in}}%
\pgfpathlineto{\pgfqpoint{6.154469in}{3.500016in}}%
\pgfpathlineto{\pgfqpoint{6.156617in}{3.495343in}}%
\pgfpathlineto{\pgfqpoint{6.157691in}{3.494564in}}%
\pgfpathlineto{\pgfqpoint{6.160912in}{3.496987in}}%
\pgfpathlineto{\pgfqpoint{6.161986in}{3.491708in}}%
\pgfpathlineto{\pgfqpoint{6.164134in}{3.487553in}}%
\pgfpathlineto{\pgfqpoint{6.165208in}{3.486169in}}%
\pgfpathlineto{\pgfqpoint{6.168429in}{3.485217in}}%
\pgfpathlineto{\pgfqpoint{6.169503in}{3.481755in}}%
\pgfpathlineto{\pgfqpoint{6.170577in}{3.486255in}}%
\pgfpathlineto{\pgfqpoint{6.171651in}{3.480889in}}%
\pgfpathlineto{\pgfqpoint{6.172725in}{3.482534in}}%
\pgfpathlineto{\pgfqpoint{6.175946in}{3.480197in}}%
\pgfpathlineto{\pgfqpoint{6.178094in}{3.487380in}}%
\pgfpathlineto{\pgfqpoint{6.179168in}{3.481582in}}%
\pgfpathlineto{\pgfqpoint{6.180242in}{3.483659in}}%
\pgfpathlineto{\pgfqpoint{6.183463in}{3.481755in}}%
\pgfpathlineto{\pgfqpoint{6.185611in}{3.486688in}}%
\pgfpathlineto{\pgfqpoint{6.186685in}{3.486601in}}%
\pgfpathlineto{\pgfqpoint{6.187758in}{3.490236in}}%
\pgfpathlineto{\pgfqpoint{6.190980in}{3.488505in}}%
\pgfpathlineto{\pgfqpoint{6.193128in}{3.489198in}}%
\pgfpathlineto{\pgfqpoint{6.194201in}{3.490756in}}%
\pgfpathlineto{\pgfqpoint{6.195275in}{3.490583in}}%
\pgfpathlineto{\pgfqpoint{6.198497in}{3.488765in}}%
\pgfpathlineto{\pgfqpoint{6.201718in}{3.492573in}}%
\pgfpathlineto{\pgfqpoint{6.202792in}{3.495343in}}%
\pgfpathlineto{\pgfqpoint{6.206014in}{3.496641in}}%
\pgfpathlineto{\pgfqpoint{6.207087in}{3.503824in}}%
\pgfpathlineto{\pgfqpoint{6.208161in}{3.506247in}}%
\pgfpathlineto{\pgfqpoint{6.209235in}{3.507199in}}%
\pgfpathlineto{\pgfqpoint{6.210309in}{3.505642in}}%
\pgfpathlineto{\pgfqpoint{6.215678in}{3.508065in}}%
\pgfpathlineto{\pgfqpoint{6.217826in}{3.502093in}}%
\pgfpathlineto{\pgfqpoint{6.221047in}{3.505988in}}%
\pgfpathlineto{\pgfqpoint{6.225343in}{3.495083in}}%
\pgfpathlineto{\pgfqpoint{6.228564in}{3.494304in}}%
\pgfpathlineto{\pgfqpoint{6.229638in}{3.492054in}}%
\pgfpathlineto{\pgfqpoint{6.237155in}{3.492746in}}%
\pgfpathlineto{\pgfqpoint{6.239303in}{3.495516in}}%
\pgfpathlineto{\pgfqpoint{6.240376in}{3.495689in}}%
\pgfpathlineto{\pgfqpoint{6.244672in}{3.495083in}}%
\pgfpathlineto{\pgfqpoint{6.245746in}{3.501228in}}%
\pgfpathlineto{\pgfqpoint{6.247893in}{3.496554in}}%
\pgfpathlineto{\pgfqpoint{6.251115in}{3.501314in}}%
\pgfpathlineto{\pgfqpoint{6.253263in}{3.505901in}}%
\pgfpathlineto{\pgfqpoint{6.254336in}{3.507113in}}%
\pgfpathlineto{\pgfqpoint{6.255410in}{3.511440in}}%
\pgfpathlineto{\pgfqpoint{6.258632in}{3.511440in}}%
\pgfpathlineto{\pgfqpoint{6.259706in}{3.513344in}}%
\pgfpathlineto{\pgfqpoint{6.260779in}{3.512046in}}%
\pgfpathlineto{\pgfqpoint{6.261853in}{3.513085in}}%
\pgfpathlineto{\pgfqpoint{6.266149in}{3.512738in}}%
\pgfpathlineto{\pgfqpoint{6.267222in}{3.515335in}}%
\pgfpathlineto{\pgfqpoint{6.268296in}{3.515854in}}%
\pgfpathlineto{\pgfqpoint{6.274739in}{3.531000in}}%
\pgfpathlineto{\pgfqpoint{6.275813in}{3.530654in}}%
\pgfpathlineto{\pgfqpoint{6.277961in}{3.533077in}}%
\pgfpathlineto{\pgfqpoint{6.281182in}{3.534981in}}%
\pgfpathlineto{\pgfqpoint{6.282256in}{3.533250in}}%
\pgfpathlineto{\pgfqpoint{6.283330in}{3.530394in}}%
\pgfpathlineto{\pgfqpoint{6.284404in}{3.529442in}}%
\pgfpathlineto{\pgfqpoint{6.285478in}{3.533423in}}%
\pgfpathlineto{\pgfqpoint{6.289773in}{3.534375in}}%
\pgfpathlineto{\pgfqpoint{6.290847in}{3.538183in}}%
\pgfpathlineto{\pgfqpoint{6.291921in}{3.536885in}}%
\pgfpathlineto{\pgfqpoint{6.292995in}{3.539654in}}%
\pgfpathlineto{\pgfqpoint{6.297290in}{3.543982in}}%
\pgfpathlineto{\pgfqpoint{6.298364in}{3.542510in}}%
\pgfpathlineto{\pgfqpoint{6.299438in}{3.547270in}}%
\pgfpathlineto{\pgfqpoint{6.300511in}{3.572369in}}%
\pgfpathlineto{\pgfqpoint{6.303733in}{3.572109in}}%
\pgfpathlineto{\pgfqpoint{6.305881in}{3.591409in}}%
\pgfpathlineto{\pgfqpoint{6.306954in}{3.594611in}}%
\pgfpathlineto{\pgfqpoint{6.308028in}{3.588293in}}%
\pgfpathlineto{\pgfqpoint{6.311250in}{3.593573in}}%
\pgfpathlineto{\pgfqpoint{6.312324in}{3.594265in}}%
\pgfpathlineto{\pgfqpoint{6.313397in}{3.593573in}}%
\pgfpathlineto{\pgfqpoint{6.314471in}{3.590284in}}%
\pgfpathlineto{\pgfqpoint{6.315545in}{3.584312in}}%
\pgfpathlineto{\pgfqpoint{6.319841in}{3.586649in}}%
\pgfpathlineto{\pgfqpoint{6.320914in}{3.583274in}}%
\pgfpathlineto{\pgfqpoint{6.321988in}{3.584832in}}%
\pgfpathlineto{\pgfqpoint{6.323062in}{3.576437in}}%
\pgfpathlineto{\pgfqpoint{6.326284in}{3.576350in}}%
\pgfpathlineto{\pgfqpoint{6.327357in}{3.579033in}}%
\pgfpathlineto{\pgfqpoint{6.328431in}{3.576610in}}%
\pgfpathlineto{\pgfqpoint{6.330579in}{3.577389in}}%
\pgfpathlineto{\pgfqpoint{6.333800in}{3.575225in}}%
\pgfpathlineto{\pgfqpoint{6.334874in}{3.577216in}}%
\pgfpathlineto{\pgfqpoint{6.335948in}{3.570811in}}%
\pgfpathlineto{\pgfqpoint{6.337022in}{3.578168in}}%
\pgfpathlineto{\pgfqpoint{6.338096in}{3.576869in}}%
\pgfpathlineto{\pgfqpoint{6.341317in}{3.575225in}}%
\pgfpathlineto{\pgfqpoint{6.342391in}{3.566570in}}%
\pgfpathlineto{\pgfqpoint{6.343465in}{3.566657in}}%
\pgfpathlineto{\pgfqpoint{6.344539in}{3.563628in}}%
\pgfpathlineto{\pgfqpoint{6.345613in}{3.565791in}}%
\pgfpathlineto{\pgfqpoint{6.348834in}{3.568388in}}%
\pgfpathlineto{\pgfqpoint{6.349908in}{3.565618in}}%
\pgfpathlineto{\pgfqpoint{6.350982in}{3.565705in}}%
\pgfpathlineto{\pgfqpoint{6.352056in}{3.565099in}}%
\pgfpathlineto{\pgfqpoint{6.353130in}{3.575831in}}%
\pgfpathlineto{\pgfqpoint{6.358499in}{3.600670in}}%
\pgfpathlineto{\pgfqpoint{6.359573in}{3.594092in}}%
\pgfpathlineto{\pgfqpoint{6.360646in}{3.593573in}}%
\pgfpathlineto{\pgfqpoint{6.364942in}{3.588467in}}%
\pgfpathlineto{\pgfqpoint{6.366016in}{3.588640in}}%
\pgfpathlineto{\pgfqpoint{6.367089in}{3.589592in}}%
\pgfpathlineto{\pgfqpoint{6.368163in}{3.589072in}}%
\pgfpathlineto{\pgfqpoint{6.368163in}{3.589072in}}%
\pgfusepath{stroke}%
\end{pgfscope}%
\begin{pgfscope}%
\pgfpathrectangle{\pgfqpoint{3.902254in}{3.218007in}}{\pgfqpoint{2.583333in}{0.400885in}}%
\pgfusepath{clip}%
\pgfsetroundcap%
\pgfsetroundjoin%
\pgfsetlinewidth{1.505625pt}%
\definecolor{currentstroke}{rgb}{0.839216,0.152941,0.156863}%
\pgfsetstrokecolor{currentstroke}%
\pgfsetdash{}{0pt}%
\pgfpathmoveto{\pgfqpoint{4.019678in}{3.374091in}}%
\pgfpathlineto{\pgfqpoint{4.021826in}{3.368296in}}%
\pgfpathlineto{\pgfqpoint{4.022900in}{3.369265in}}%
\pgfpathlineto{\pgfqpoint{4.027195in}{3.367073in}}%
\pgfpathlineto{\pgfqpoint{4.028269in}{3.365721in}}%
\pgfpathlineto{\pgfqpoint{4.029343in}{3.366034in}}%
\pgfpathlineto{\pgfqpoint{4.030417in}{3.362267in}}%
\pgfpathlineto{\pgfqpoint{4.034712in}{3.361639in}}%
\pgfpathlineto{\pgfqpoint{4.036860in}{3.358069in}}%
\pgfpathlineto{\pgfqpoint{4.037934in}{3.353649in}}%
\pgfpathlineto{\pgfqpoint{4.043303in}{3.350777in}}%
\pgfpathlineto{\pgfqpoint{4.044377in}{3.351593in}}%
\pgfpathlineto{\pgfqpoint{4.048672in}{3.351593in}}%
\pgfpathlineto{\pgfqpoint{4.049746in}{3.353379in}}%
\pgfpathlineto{\pgfqpoint{4.050820in}{3.352605in}}%
\pgfpathlineto{\pgfqpoint{4.051894in}{3.352953in}}%
\pgfpathlineto{\pgfqpoint{4.052967in}{3.350432in}}%
\pgfpathlineto{\pgfqpoint{4.057263in}{3.350974in}}%
\pgfpathlineto{\pgfqpoint{4.058337in}{3.349815in}}%
\pgfpathlineto{\pgfqpoint{4.059410in}{3.349748in}}%
\pgfpathlineto{\pgfqpoint{4.060484in}{3.350619in}}%
\pgfpathlineto{\pgfqpoint{4.064780in}{3.350144in}}%
\pgfpathlineto{\pgfqpoint{4.065853in}{3.351222in}}%
\pgfpathlineto{\pgfqpoint{4.066927in}{3.349854in}}%
\pgfpathlineto{\pgfqpoint{4.068001in}{3.347034in}}%
\pgfpathlineto{\pgfqpoint{4.072296in}{3.348067in}}%
\pgfpathlineto{\pgfqpoint{4.073370in}{3.350294in}}%
\pgfpathlineto{\pgfqpoint{4.074444in}{3.350699in}}%
\pgfpathlineto{\pgfqpoint{4.078740in}{3.349479in}}%
\pgfpathlineto{\pgfqpoint{4.079813in}{3.347607in}}%
\pgfpathlineto{\pgfqpoint{4.080887in}{3.349430in}}%
\pgfpathlineto{\pgfqpoint{4.083035in}{3.349229in}}%
\pgfpathlineto{\pgfqpoint{4.086256in}{3.351228in}}%
\pgfpathlineto{\pgfqpoint{4.087330in}{3.350885in}}%
\pgfpathlineto{\pgfqpoint{4.088404in}{3.349249in}}%
\pgfpathlineto{\pgfqpoint{4.089478in}{3.349649in}}%
\pgfpathlineto{\pgfqpoint{4.090552in}{3.348375in}}%
\pgfpathlineto{\pgfqpoint{4.093773in}{3.348836in}}%
\pgfpathlineto{\pgfqpoint{4.094847in}{3.346185in}}%
\pgfpathlineto{\pgfqpoint{4.095921in}{3.346312in}}%
\pgfpathlineto{\pgfqpoint{4.096995in}{3.344841in}}%
\pgfpathlineto{\pgfqpoint{4.101290in}{3.344904in}}%
\pgfpathlineto{\pgfqpoint{4.103438in}{3.344716in}}%
\pgfpathlineto{\pgfqpoint{4.104512in}{3.344090in}}%
\pgfpathlineto{\pgfqpoint{4.105585in}{3.344214in}}%
\pgfpathlineto{\pgfqpoint{4.109881in}{3.342663in}}%
\pgfpathlineto{\pgfqpoint{4.110955in}{3.344544in}}%
\pgfpathlineto{\pgfqpoint{4.112029in}{3.342797in}}%
\pgfpathlineto{\pgfqpoint{4.113102in}{3.342979in}}%
\pgfpathlineto{\pgfqpoint{4.116324in}{3.341699in}}%
\pgfpathlineto{\pgfqpoint{4.118472in}{3.343870in}}%
\pgfpathlineto{\pgfqpoint{4.119545in}{3.343190in}}%
\pgfpathlineto{\pgfqpoint{4.123841in}{3.344720in}}%
\pgfpathlineto{\pgfqpoint{4.124915in}{3.346222in}}%
\pgfpathlineto{\pgfqpoint{4.127062in}{3.341070in}}%
\pgfpathlineto{\pgfqpoint{4.128136in}{3.342910in}}%
\pgfpathlineto{\pgfqpoint{4.132431in}{3.341086in}}%
\pgfpathlineto{\pgfqpoint{4.134579in}{3.344814in}}%
\pgfpathlineto{\pgfqpoint{4.139948in}{3.346714in}}%
\pgfpathlineto{\pgfqpoint{4.142096in}{3.342142in}}%
\pgfpathlineto{\pgfqpoint{4.143170in}{3.341357in}}%
\pgfpathlineto{\pgfqpoint{4.146391in}{3.341297in}}%
\pgfpathlineto{\pgfqpoint{4.147465in}{3.338673in}}%
\pgfpathlineto{\pgfqpoint{4.148539in}{3.337640in}}%
\pgfpathlineto{\pgfqpoint{4.149613in}{3.339504in}}%
\pgfpathlineto{\pgfqpoint{4.150687in}{3.342525in}}%
\pgfpathlineto{\pgfqpoint{4.153908in}{3.343255in}}%
\pgfpathlineto{\pgfqpoint{4.156056in}{3.346045in}}%
\pgfpathlineto{\pgfqpoint{4.157130in}{3.345789in}}%
\pgfpathlineto{\pgfqpoint{4.158204in}{3.343812in}}%
\pgfpathlineto{\pgfqpoint{4.161425in}{3.346849in}}%
\pgfpathlineto{\pgfqpoint{4.162499in}{3.346137in}}%
\pgfpathlineto{\pgfqpoint{4.164647in}{3.342572in}}%
\pgfpathlineto{\pgfqpoint{4.165720in}{3.341924in}}%
\pgfpathlineto{\pgfqpoint{4.168942in}{3.342378in}}%
\pgfpathlineto{\pgfqpoint{4.170016in}{3.341730in}}%
\pgfpathlineto{\pgfqpoint{4.171090in}{3.338683in}}%
\pgfpathlineto{\pgfqpoint{4.173237in}{3.340239in}}%
\pgfpathlineto{\pgfqpoint{4.178607in}{3.342248in}}%
\pgfpathlineto{\pgfqpoint{4.179680in}{3.340757in}}%
\pgfpathlineto{\pgfqpoint{4.180754in}{3.337128in}}%
\pgfpathlineto{\pgfqpoint{4.183976in}{3.336610in}}%
\pgfpathlineto{\pgfqpoint{4.185050in}{3.338619in}}%
\pgfpathlineto{\pgfqpoint{4.186123in}{3.341924in}}%
\pgfpathlineto{\pgfqpoint{4.187197in}{3.341276in}}%
\pgfpathlineto{\pgfqpoint{4.188271in}{3.343674in}}%
\pgfpathlineto{\pgfqpoint{4.191493in}{3.341535in}}%
\pgfpathlineto{\pgfqpoint{4.192566in}{3.344257in}}%
\pgfpathlineto{\pgfqpoint{4.193640in}{3.344387in}}%
\pgfpathlineto{\pgfqpoint{4.195788in}{3.340300in}}%
\pgfpathlineto{\pgfqpoint{4.200083in}{3.339509in}}%
\pgfpathlineto{\pgfqpoint{4.201157in}{3.338845in}}%
\pgfpathlineto{\pgfqpoint{4.202231in}{3.334366in}}%
\pgfpathlineto{\pgfqpoint{4.207600in}{3.328831in}}%
\pgfpathlineto{\pgfqpoint{4.208674in}{3.329828in}}%
\pgfpathlineto{\pgfqpoint{4.209748in}{3.328009in}}%
\pgfpathlineto{\pgfqpoint{4.210822in}{3.331882in}}%
\pgfpathlineto{\pgfqpoint{4.214043in}{3.331940in}}%
\pgfpathlineto{\pgfqpoint{4.218339in}{3.327821in}}%
\pgfpathlineto{\pgfqpoint{4.221560in}{3.327878in}}%
\pgfpathlineto{\pgfqpoint{4.222634in}{3.325044in}}%
\pgfpathlineto{\pgfqpoint{4.223708in}{3.324292in}}%
\pgfpathlineto{\pgfqpoint{4.224782in}{3.321284in}}%
\pgfpathlineto{\pgfqpoint{4.225855in}{3.323655in}}%
\pgfpathlineto{\pgfqpoint{4.229077in}{3.323077in}}%
\pgfpathlineto{\pgfqpoint{4.230151in}{3.324234in}}%
\pgfpathlineto{\pgfqpoint{4.231225in}{3.328052in}}%
\pgfpathlineto{\pgfqpoint{4.232298in}{3.327358in}}%
\pgfpathlineto{\pgfqpoint{4.233372in}{3.324870in}}%
\pgfpathlineto{\pgfqpoint{4.236594in}{3.323655in}}%
\pgfpathlineto{\pgfqpoint{4.237668in}{3.322498in}}%
\pgfpathlineto{\pgfqpoint{4.238741in}{3.323077in}}%
\pgfpathlineto{\pgfqpoint{4.240889in}{3.327184in}}%
\pgfpathlineto{\pgfqpoint{4.245184in}{3.325738in}}%
\pgfpathlineto{\pgfqpoint{4.246258in}{3.326779in}}%
\pgfpathlineto{\pgfqpoint{4.247332in}{3.326664in}}%
\pgfpathlineto{\pgfqpoint{4.248406in}{3.329209in}}%
\pgfpathlineto{\pgfqpoint{4.251628in}{3.329614in}}%
\pgfpathlineto{\pgfqpoint{4.253775in}{3.328290in}}%
\pgfpathlineto{\pgfqpoint{4.255923in}{3.326994in}}%
\pgfpathlineto{\pgfqpoint{4.259144in}{3.327825in}}%
\pgfpathlineto{\pgfqpoint{4.260218in}{3.328781in}}%
\pgfpathlineto{\pgfqpoint{4.263440in}{3.325212in}}%
\pgfpathlineto{\pgfqpoint{4.267735in}{3.324265in}}%
\pgfpathlineto{\pgfqpoint{4.268809in}{3.322536in}}%
\pgfpathlineto{\pgfqpoint{4.269883in}{3.319414in}}%
\pgfpathlineto{\pgfqpoint{4.270957in}{3.318857in}}%
\pgfpathlineto{\pgfqpoint{4.274178in}{3.318522in}}%
\pgfpathlineto{\pgfqpoint{4.275252in}{3.319247in}}%
\pgfpathlineto{\pgfqpoint{4.277400in}{3.315958in}}%
\pgfpathlineto{\pgfqpoint{4.278473in}{3.318466in}}%
\pgfpathlineto{\pgfqpoint{4.283843in}{3.316515in}}%
\pgfpathlineto{\pgfqpoint{4.284917in}{3.319693in}}%
\pgfpathlineto{\pgfqpoint{4.285990in}{3.315623in}}%
\pgfpathlineto{\pgfqpoint{4.289212in}{3.311442in}}%
\pgfpathlineto{\pgfqpoint{4.290286in}{3.311776in}}%
\pgfpathlineto{\pgfqpoint{4.291360in}{3.311107in}}%
\pgfpathlineto{\pgfqpoint{4.292433in}{3.311888in}}%
\pgfpathlineto{\pgfqpoint{4.297803in}{3.311944in}}%
\pgfpathlineto{\pgfqpoint{4.298876in}{3.310940in}}%
\pgfpathlineto{\pgfqpoint{4.301024in}{3.310829in}}%
\pgfpathlineto{\pgfqpoint{4.304246in}{3.309379in}}%
\pgfpathlineto{\pgfqpoint{4.305319in}{3.308208in}}%
\pgfpathlineto{\pgfqpoint{4.306393in}{3.308710in}}%
\pgfpathlineto{\pgfqpoint{4.307467in}{3.310661in}}%
\pgfpathlineto{\pgfqpoint{4.308541in}{3.308710in}}%
\pgfpathlineto{\pgfqpoint{4.312836in}{3.309546in}}%
\pgfpathlineto{\pgfqpoint{4.313910in}{3.308208in}}%
\pgfpathlineto{\pgfqpoint{4.314984in}{3.307874in}}%
\pgfpathlineto{\pgfqpoint{4.316058in}{3.308821in}}%
\pgfpathlineto{\pgfqpoint{4.319279in}{3.308041in}}%
\pgfpathlineto{\pgfqpoint{4.320353in}{3.305309in}}%
\pgfpathlineto{\pgfqpoint{4.323575in}{3.303414in}}%
\pgfpathlineto{\pgfqpoint{4.326796in}{3.304529in}}%
\pgfpathlineto{\pgfqpoint{4.327870in}{3.307316in}}%
\pgfpathlineto{\pgfqpoint{4.328944in}{3.304807in}}%
\pgfpathlineto{\pgfqpoint{4.330018in}{3.304250in}}%
\pgfpathlineto{\pgfqpoint{4.331092in}{3.302466in}}%
\pgfpathlineto{\pgfqpoint{4.335387in}{3.303915in}}%
\pgfpathlineto{\pgfqpoint{4.336461in}{3.303358in}}%
\pgfpathlineto{\pgfqpoint{4.338608in}{3.305532in}}%
\pgfpathlineto{\pgfqpoint{4.343978in}{3.304083in}}%
\pgfpathlineto{\pgfqpoint{4.345051in}{3.306926in}}%
\pgfpathlineto{\pgfqpoint{4.346125in}{3.306034in}}%
\pgfpathlineto{\pgfqpoint{4.350421in}{3.305532in}}%
\pgfpathlineto{\pgfqpoint{4.351495in}{3.301853in}}%
\pgfpathlineto{\pgfqpoint{4.353642in}{3.301351in}}%
\pgfpathlineto{\pgfqpoint{4.356864in}{3.301184in}}%
\pgfpathlineto{\pgfqpoint{4.359011in}{3.297504in}}%
\pgfpathlineto{\pgfqpoint{4.364381in}{3.298842in}}%
\pgfpathlineto{\pgfqpoint{4.365454in}{3.295441in}}%
\pgfpathlineto{\pgfqpoint{4.366528in}{3.294772in}}%
\pgfpathlineto{\pgfqpoint{4.368676in}{3.296445in}}%
\pgfpathlineto{\pgfqpoint{4.374045in}{3.298117in}}%
\pgfpathlineto{\pgfqpoint{4.375119in}{3.295553in}}%
\pgfpathlineto{\pgfqpoint{4.376193in}{3.295720in}}%
\pgfpathlineto{\pgfqpoint{4.379414in}{3.295608in}}%
\pgfpathlineto{\pgfqpoint{4.380488in}{3.297560in}}%
\pgfpathlineto{\pgfqpoint{4.381562in}{3.297002in}}%
\pgfpathlineto{\pgfqpoint{4.382636in}{3.298396in}}%
\pgfpathlineto{\pgfqpoint{4.386931in}{3.298062in}}%
\pgfpathlineto{\pgfqpoint{4.388005in}{3.300626in}}%
\pgfpathlineto{\pgfqpoint{4.389079in}{3.300738in}}%
\pgfpathlineto{\pgfqpoint{4.390153in}{3.299901in}}%
\pgfpathlineto{\pgfqpoint{4.394448in}{3.300292in}}%
\pgfpathlineto{\pgfqpoint{4.395522in}{3.302076in}}%
\pgfpathlineto{\pgfqpoint{4.396596in}{3.302689in}}%
\pgfpathlineto{\pgfqpoint{4.397670in}{3.302354in}}%
\pgfpathlineto{\pgfqpoint{4.398743in}{3.301184in}}%
\pgfpathlineto{\pgfqpoint{4.405186in}{3.300013in}}%
\pgfpathlineto{\pgfqpoint{4.406260in}{3.298731in}}%
\pgfpathlineto{\pgfqpoint{4.409482in}{3.300515in}}%
\pgfpathlineto{\pgfqpoint{4.411629in}{3.303971in}}%
\pgfpathlineto{\pgfqpoint{4.413777in}{3.302968in}}%
\pgfpathlineto{\pgfqpoint{4.416999in}{3.303358in}}%
\pgfpathlineto{\pgfqpoint{4.418073in}{3.302633in}}%
\pgfpathlineto{\pgfqpoint{4.421294in}{3.306759in}}%
\pgfpathlineto{\pgfqpoint{4.426663in}{3.307260in}}%
\pgfpathlineto{\pgfqpoint{4.427737in}{3.309881in}}%
\pgfpathlineto{\pgfqpoint{4.428811in}{3.303358in}}%
\pgfpathlineto{\pgfqpoint{4.434180in}{3.302745in}}%
\pgfpathlineto{\pgfqpoint{4.435254in}{3.302020in}}%
\pgfpathlineto{\pgfqpoint{4.439549in}{3.302466in}}%
\pgfpathlineto{\pgfqpoint{4.441697in}{3.303915in}}%
\pgfpathlineto{\pgfqpoint{4.442771in}{3.302410in}}%
\pgfpathlineto{\pgfqpoint{4.443845in}{3.303860in}}%
\pgfpathlineto{\pgfqpoint{4.447066in}{3.302968in}}%
\pgfpathlineto{\pgfqpoint{4.448140in}{3.304083in}}%
\pgfpathlineto{\pgfqpoint{4.450288in}{3.302410in}}%
\pgfpathlineto{\pgfqpoint{4.451361in}{3.303246in}}%
\pgfpathlineto{\pgfqpoint{4.454583in}{3.303414in}}%
\pgfpathlineto{\pgfqpoint{4.456731in}{3.304417in}}%
\pgfpathlineto{\pgfqpoint{4.463174in}{3.303692in}}%
\pgfpathlineto{\pgfqpoint{4.465321in}{3.299790in}}%
\pgfpathlineto{\pgfqpoint{4.466395in}{3.300570in}}%
\pgfpathlineto{\pgfqpoint{4.469617in}{3.299734in}}%
\pgfpathlineto{\pgfqpoint{4.471764in}{3.302912in}}%
\pgfpathlineto{\pgfqpoint{4.472838in}{3.302689in}}%
\pgfpathlineto{\pgfqpoint{4.473912in}{3.303414in}}%
\pgfpathlineto{\pgfqpoint{4.477134in}{3.304529in}}%
\pgfpathlineto{\pgfqpoint{4.480355in}{3.307372in}}%
\pgfpathlineto{\pgfqpoint{4.481429in}{3.305922in}}%
\pgfpathlineto{\pgfqpoint{4.486798in}{3.306313in}}%
\pgfpathlineto{\pgfqpoint{4.487872in}{3.306257in}}%
\pgfpathlineto{\pgfqpoint{4.488946in}{3.305030in}}%
\pgfpathlineto{\pgfqpoint{4.495389in}{3.303469in}}%
\pgfpathlineto{\pgfqpoint{4.496463in}{3.304752in}}%
\pgfpathlineto{\pgfqpoint{4.499684in}{3.303971in}}%
\pgfpathlineto{\pgfqpoint{4.500758in}{3.306814in}}%
\pgfpathlineto{\pgfqpoint{4.502906in}{3.307093in}}%
\pgfpathlineto{\pgfqpoint{4.508275in}{3.305365in}}%
\pgfpathlineto{\pgfqpoint{4.509349in}{3.303469in}}%
\pgfpathlineto{\pgfqpoint{4.510423in}{3.303915in}}%
\pgfpathlineto{\pgfqpoint{4.511496in}{3.302968in}}%
\pgfpathlineto{\pgfqpoint{4.514718in}{3.303692in}}%
\pgfpathlineto{\pgfqpoint{4.516866in}{3.309044in}}%
\pgfpathlineto{\pgfqpoint{4.517939in}{3.307093in}}%
\pgfpathlineto{\pgfqpoint{4.519013in}{3.306368in}}%
\pgfpathlineto{\pgfqpoint{4.522235in}{3.305030in}}%
\pgfpathlineto{\pgfqpoint{4.523309in}{3.307483in}}%
\pgfpathlineto{\pgfqpoint{4.524383in}{3.307539in}}%
\pgfpathlineto{\pgfqpoint{4.526530in}{3.309881in}}%
\pgfpathlineto{\pgfqpoint{4.529752in}{3.311888in}}%
\pgfpathlineto{\pgfqpoint{4.530826in}{3.314174in}}%
\pgfpathlineto{\pgfqpoint{4.532973in}{3.311639in}}%
\pgfpathlineto{\pgfqpoint{4.534047in}{3.311748in}}%
\pgfpathlineto{\pgfqpoint{4.541564in}{3.308291in}}%
\pgfpathlineto{\pgfqpoint{4.544785in}{3.308495in}}%
\pgfpathlineto{\pgfqpoint{4.546933in}{3.307071in}}%
\pgfpathlineto{\pgfqpoint{4.549081in}{3.306032in}}%
\pgfpathlineto{\pgfqpoint{4.553376in}{3.308744in}}%
\pgfpathlineto{\pgfqpoint{4.554450in}{3.307202in}}%
\pgfpathlineto{\pgfqpoint{4.555524in}{3.308299in}}%
\pgfpathlineto{\pgfqpoint{4.556598in}{3.307892in}}%
\pgfpathlineto{\pgfqpoint{4.563041in}{3.307838in}}%
\pgfpathlineto{\pgfqpoint{4.564115in}{3.308394in}}%
\pgfpathlineto{\pgfqpoint{4.568410in}{3.307679in}}%
\pgfpathlineto{\pgfqpoint{4.569484in}{3.306873in}}%
\pgfpathlineto{\pgfqpoint{4.571631in}{3.306872in}}%
\pgfpathlineto{\pgfqpoint{4.575927in}{3.302449in}}%
\pgfpathlineto{\pgfqpoint{4.577001in}{3.304943in}}%
\pgfpathlineto{\pgfqpoint{4.579148in}{3.305374in}}%
\pgfpathlineto{\pgfqpoint{4.582370in}{3.303687in}}%
\pgfpathlineto{\pgfqpoint{4.584517in}{3.305854in}}%
\pgfpathlineto{\pgfqpoint{4.585591in}{3.303711in}}%
\pgfpathlineto{\pgfqpoint{4.586665in}{3.303944in}}%
\pgfpathlineto{\pgfqpoint{4.589887in}{3.303241in}}%
\pgfpathlineto{\pgfqpoint{4.590961in}{3.301855in}}%
\pgfpathlineto{\pgfqpoint{4.592034in}{3.303606in}}%
\pgfpathlineto{\pgfqpoint{4.593108in}{3.306726in}}%
\pgfpathlineto{\pgfqpoint{4.594182in}{3.306676in}}%
\pgfpathlineto{\pgfqpoint{4.597404in}{3.309256in}}%
\pgfpathlineto{\pgfqpoint{4.599551in}{3.307390in}}%
\pgfpathlineto{\pgfqpoint{4.600625in}{3.307238in}}%
\pgfpathlineto{\pgfqpoint{4.601699in}{3.306485in}}%
\pgfpathlineto{\pgfqpoint{4.607068in}{3.308428in}}%
\pgfpathlineto{\pgfqpoint{4.609216in}{3.307142in}}%
\pgfpathlineto{\pgfqpoint{4.612437in}{3.303531in}}%
\pgfpathlineto{\pgfqpoint{4.613511in}{3.303331in}}%
\pgfpathlineto{\pgfqpoint{4.614585in}{3.303782in}}%
\pgfpathlineto{\pgfqpoint{4.615659in}{3.300723in}}%
\pgfpathlineto{\pgfqpoint{4.616733in}{3.301053in}}%
\pgfpathlineto{\pgfqpoint{4.619954in}{3.300910in}}%
\pgfpathlineto{\pgfqpoint{4.621028in}{3.299681in}}%
\pgfpathlineto{\pgfqpoint{4.622102in}{3.300050in}}%
\pgfpathlineto{\pgfqpoint{4.623176in}{3.296522in}}%
\pgfpathlineto{\pgfqpoint{4.624249in}{3.295733in}}%
\pgfpathlineto{\pgfqpoint{4.628545in}{3.294619in}}%
\pgfpathlineto{\pgfqpoint{4.631766in}{3.296615in}}%
\pgfpathlineto{\pgfqpoint{4.634988in}{3.296522in}}%
\pgfpathlineto{\pgfqpoint{4.636062in}{3.297079in}}%
\pgfpathlineto{\pgfqpoint{4.644652in}{3.295269in}}%
\pgfpathlineto{\pgfqpoint{4.645726in}{3.294294in}}%
\pgfpathlineto{\pgfqpoint{4.650022in}{3.295037in}}%
\pgfpathlineto{\pgfqpoint{4.651095in}{3.294572in}}%
\pgfpathlineto{\pgfqpoint{4.652169in}{3.294805in}}%
\pgfpathlineto{\pgfqpoint{4.653243in}{3.292669in}}%
\pgfpathlineto{\pgfqpoint{4.654317in}{3.292205in}}%
\pgfpathlineto{\pgfqpoint{4.658612in}{3.294572in}}%
\pgfpathlineto{\pgfqpoint{4.659686in}{3.293226in}}%
\pgfpathlineto{\pgfqpoint{4.661834in}{3.294294in}}%
\pgfpathlineto{\pgfqpoint{4.669351in}{3.292484in}}%
\pgfpathlineto{\pgfqpoint{4.673646in}{3.292855in}}%
\pgfpathlineto{\pgfqpoint{4.674720in}{3.295037in}}%
\pgfpathlineto{\pgfqpoint{4.676868in}{3.295176in}}%
\pgfpathlineto{\pgfqpoint{4.681163in}{3.296383in}}%
\pgfpathlineto{\pgfqpoint{4.683311in}{3.295037in}}%
\pgfpathlineto{\pgfqpoint{4.684384in}{3.291880in}}%
\pgfpathlineto{\pgfqpoint{4.687606in}{3.292096in}}%
\pgfpathlineto{\pgfqpoint{4.689754in}{3.290246in}}%
\pgfpathlineto{\pgfqpoint{4.691901in}{3.290703in}}%
\pgfpathlineto{\pgfqpoint{4.696197in}{3.290952in}}%
\pgfpathlineto{\pgfqpoint{4.697271in}{3.290952in}}%
\pgfpathlineto{\pgfqpoint{4.698344in}{3.291967in}}%
\pgfpathlineto{\pgfqpoint{4.699418in}{3.290407in}}%
\pgfpathlineto{\pgfqpoint{4.704787in}{3.290061in}}%
\pgfpathlineto{\pgfqpoint{4.705861in}{3.289021in}}%
\pgfpathlineto{\pgfqpoint{4.706935in}{3.289801in}}%
\pgfpathlineto{\pgfqpoint{4.710157in}{3.289844in}}%
\pgfpathlineto{\pgfqpoint{4.711230in}{3.288588in}}%
\pgfpathlineto{\pgfqpoint{4.712304in}{3.288978in}}%
\pgfpathlineto{\pgfqpoint{4.713378in}{3.290840in}}%
\pgfpathlineto{\pgfqpoint{4.714452in}{3.290277in}}%
\pgfpathlineto{\pgfqpoint{4.719821in}{3.288705in}}%
\pgfpathlineto{\pgfqpoint{4.720895in}{3.287923in}}%
\pgfpathlineto{\pgfqpoint{4.721969in}{3.288044in}}%
\pgfpathlineto{\pgfqpoint{4.726264in}{3.287431in}}%
\pgfpathlineto{\pgfqpoint{4.727338in}{3.288508in}}%
\pgfpathlineto{\pgfqpoint{4.728412in}{3.288385in}}%
\pgfpathlineto{\pgfqpoint{4.729486in}{3.286792in}}%
\pgfpathlineto{\pgfqpoint{4.735929in}{3.286047in}}%
\pgfpathlineto{\pgfqpoint{4.737003in}{3.286510in}}%
\pgfpathlineto{\pgfqpoint{4.741298in}{3.286744in}}%
\pgfpathlineto{\pgfqpoint{4.742372in}{3.285998in}}%
\pgfpathlineto{\pgfqpoint{4.743446in}{3.286615in}}%
\pgfpathlineto{\pgfqpoint{4.747741in}{3.286263in}}%
\pgfpathlineto{\pgfqpoint{4.749889in}{3.284878in}}%
\pgfpathlineto{\pgfqpoint{4.750962in}{3.285515in}}%
\pgfpathlineto{\pgfqpoint{4.752036in}{3.285096in}}%
\pgfpathlineto{\pgfqpoint{4.757405in}{3.284979in}}%
\pgfpathlineto{\pgfqpoint{4.758479in}{3.282838in}}%
\pgfpathlineto{\pgfqpoint{4.759553in}{3.286962in}}%
\pgfpathlineto{\pgfqpoint{4.763849in}{3.287679in}}%
\pgfpathlineto{\pgfqpoint{4.764922in}{3.286832in}}%
\pgfpathlineto{\pgfqpoint{4.770292in}{3.287505in}}%
\pgfpathlineto{\pgfqpoint{4.771365in}{3.288028in}}%
\pgfpathlineto{\pgfqpoint{4.772439in}{3.287377in}}%
\pgfpathlineto{\pgfqpoint{4.774587in}{3.283774in}}%
\pgfpathlineto{\pgfqpoint{4.778882in}{3.283772in}}%
\pgfpathlineto{\pgfqpoint{4.779956in}{3.285015in}}%
\pgfpathlineto{\pgfqpoint{4.781030in}{3.284864in}}%
\pgfpathlineto{\pgfqpoint{4.782104in}{3.285429in}}%
\pgfpathlineto{\pgfqpoint{4.786399in}{3.284254in}}%
\pgfpathlineto{\pgfqpoint{4.787473in}{3.282697in}}%
\pgfpathlineto{\pgfqpoint{4.789621in}{3.282982in}}%
\pgfpathlineto{\pgfqpoint{4.801433in}{3.280290in}}%
\pgfpathlineto{\pgfqpoint{4.807876in}{3.281739in}}%
\pgfpathlineto{\pgfqpoint{4.808950in}{3.281357in}}%
\pgfpathlineto{\pgfqpoint{4.811097in}{3.282152in}}%
\pgfpathlineto{\pgfqpoint{4.812171in}{3.281520in}}%
\pgfpathlineto{\pgfqpoint{4.815393in}{3.281589in}}%
\pgfpathlineto{\pgfqpoint{4.816467in}{3.278648in}}%
\pgfpathlineto{\pgfqpoint{4.817540in}{3.278201in}}%
\pgfpathlineto{\pgfqpoint{4.818614in}{3.278548in}}%
\pgfpathlineto{\pgfqpoint{4.819688in}{3.280429in}}%
\pgfpathlineto{\pgfqpoint{4.823983in}{3.281169in}}%
\pgfpathlineto{\pgfqpoint{4.827205in}{3.283478in}}%
\pgfpathlineto{\pgfqpoint{4.830426in}{3.283733in}}%
\pgfpathlineto{\pgfqpoint{4.831500in}{3.283182in}}%
\pgfpathlineto{\pgfqpoint{4.832574in}{3.283870in}}%
\pgfpathlineto{\pgfqpoint{4.833648in}{3.283686in}}%
\pgfpathlineto{\pgfqpoint{4.840091in}{3.286840in}}%
\pgfpathlineto{\pgfqpoint{4.842239in}{3.284523in}}%
\pgfpathlineto{\pgfqpoint{4.846534in}{3.283701in}}%
\pgfpathlineto{\pgfqpoint{4.849756in}{3.282790in}}%
\pgfpathlineto{\pgfqpoint{4.854051in}{3.282790in}}%
\pgfpathlineto{\pgfqpoint{4.855125in}{3.283579in}}%
\pgfpathlineto{\pgfqpoint{4.856199in}{3.282847in}}%
\pgfpathlineto{\pgfqpoint{4.857272in}{3.283817in}}%
\pgfpathlineto{\pgfqpoint{4.864789in}{3.282753in}}%
\pgfpathlineto{\pgfqpoint{4.868011in}{3.283541in}}%
\pgfpathlineto{\pgfqpoint{4.869085in}{3.283212in}}%
\pgfpathlineto{\pgfqpoint{4.870159in}{3.283539in}}%
\pgfpathlineto{\pgfqpoint{4.872306in}{3.283100in}}%
\pgfpathlineto{\pgfqpoint{4.876602in}{3.282841in}}%
\pgfpathlineto{\pgfqpoint{4.877675in}{3.282733in}}%
\pgfpathlineto{\pgfqpoint{4.879823in}{3.283525in}}%
\pgfpathlineto{\pgfqpoint{4.885192in}{3.281936in}}%
\pgfpathlineto{\pgfqpoint{4.886266in}{3.280708in}}%
\pgfpathlineto{\pgfqpoint{4.887340in}{3.281455in}}%
\pgfpathlineto{\pgfqpoint{4.890561in}{3.281594in}}%
\pgfpathlineto{\pgfqpoint{4.891635in}{3.280585in}}%
\pgfpathlineto{\pgfqpoint{4.893783in}{3.281027in}}%
\pgfpathlineto{\pgfqpoint{4.894857in}{3.280101in}}%
\pgfpathlineto{\pgfqpoint{4.900226in}{3.279332in}}%
\pgfpathlineto{\pgfqpoint{4.901300in}{3.277892in}}%
\pgfpathlineto{\pgfqpoint{4.902374in}{3.278553in}}%
\pgfpathlineto{\pgfqpoint{4.905595in}{3.277655in}}%
\pgfpathlineto{\pgfqpoint{4.906669in}{3.276530in}}%
\pgfpathlineto{\pgfqpoint{4.907743in}{3.276348in}}%
\pgfpathlineto{\pgfqpoint{4.909891in}{3.278454in}}%
\pgfpathlineto{\pgfqpoint{4.916334in}{3.276140in}}%
\pgfpathlineto{\pgfqpoint{4.923850in}{3.276894in}}%
\pgfpathlineto{\pgfqpoint{4.924924in}{3.278181in}}%
\pgfpathlineto{\pgfqpoint{4.929220in}{3.277579in}}%
\pgfpathlineto{\pgfqpoint{4.930293in}{3.277017in}}%
\pgfpathlineto{\pgfqpoint{4.932441in}{3.277757in}}%
\pgfpathlineto{\pgfqpoint{4.939958in}{3.277441in}}%
\pgfpathlineto{\pgfqpoint{4.945327in}{3.277345in}}%
\pgfpathlineto{\pgfqpoint{4.947475in}{3.278723in}}%
\pgfpathlineto{\pgfqpoint{4.961435in}{3.275664in}}%
\pgfpathlineto{\pgfqpoint{4.962509in}{3.274746in}}%
\pgfpathlineto{\pgfqpoint{4.965730in}{3.274890in}}%
\pgfpathlineto{\pgfqpoint{4.966804in}{3.273906in}}%
\pgfpathlineto{\pgfqpoint{4.968952in}{3.273906in}}%
\pgfpathlineto{\pgfqpoint{4.970026in}{3.272669in}}%
\pgfpathlineto{\pgfqpoint{4.973247in}{3.273265in}}%
\pgfpathlineto{\pgfqpoint{4.974321in}{3.272492in}}%
\pgfpathlineto{\pgfqpoint{4.975395in}{3.273193in}}%
\pgfpathlineto{\pgfqpoint{4.976469in}{3.273138in}}%
\pgfpathlineto{\pgfqpoint{4.977542in}{3.268602in}}%
\pgfpathlineto{\pgfqpoint{4.989355in}{3.267301in}}%
\pgfpathlineto{\pgfqpoint{4.990428in}{3.266566in}}%
\pgfpathlineto{\pgfqpoint{4.992576in}{3.266452in}}%
\pgfpathlineto{\pgfqpoint{5.005462in}{3.266535in}}%
\pgfpathlineto{\pgfqpoint{5.006536in}{3.265818in}}%
\pgfpathlineto{\pgfqpoint{5.010831in}{3.265380in}}%
\pgfpathlineto{\pgfqpoint{5.011905in}{3.264970in}}%
\pgfpathlineto{\pgfqpoint{5.012979in}{3.259572in}}%
\pgfpathlineto{\pgfqpoint{5.014053in}{3.261019in}}%
\pgfpathlineto{\pgfqpoint{5.015127in}{3.261019in}}%
\pgfpathlineto{\pgfqpoint{5.018348in}{3.260441in}}%
\pgfpathlineto{\pgfqpoint{5.019422in}{3.259291in}}%
\pgfpathlineto{\pgfqpoint{5.022644in}{3.260119in}}%
\pgfpathlineto{\pgfqpoint{5.029087in}{3.260673in}}%
\pgfpathlineto{\pgfqpoint{5.030160in}{3.260913in}}%
\pgfpathlineto{\pgfqpoint{5.033382in}{3.260411in}}%
\pgfpathlineto{\pgfqpoint{5.034456in}{3.261998in}}%
\pgfpathlineto{\pgfqpoint{5.041973in}{3.261469in}}%
\pgfpathlineto{\pgfqpoint{5.043047in}{3.259874in}}%
\pgfpathlineto{\pgfqpoint{5.044120in}{3.260125in}}%
\pgfpathlineto{\pgfqpoint{5.045194in}{3.259763in}}%
\pgfpathlineto{\pgfqpoint{5.050563in}{3.259247in}}%
\pgfpathlineto{\pgfqpoint{5.051637in}{3.258267in}}%
\pgfpathlineto{\pgfqpoint{5.052711in}{3.258570in}}%
\pgfpathlineto{\pgfqpoint{5.065597in}{3.259112in}}%
\pgfpathlineto{\pgfqpoint{5.067745in}{3.258470in}}%
\pgfpathlineto{\pgfqpoint{5.070966in}{3.257978in}}%
\pgfpathlineto{\pgfqpoint{5.072040in}{3.258595in}}%
\pgfpathlineto{\pgfqpoint{5.074188in}{3.258424in}}%
\pgfpathlineto{\pgfqpoint{5.075262in}{3.259018in}}%
\pgfpathlineto{\pgfqpoint{5.078483in}{3.259139in}}%
\pgfpathlineto{\pgfqpoint{5.080631in}{3.258478in}}%
\pgfpathlineto{\pgfqpoint{5.081705in}{3.258189in}}%
\pgfpathlineto{\pgfqpoint{5.082779in}{3.258710in}}%
\pgfpathlineto{\pgfqpoint{5.088148in}{3.258805in}}%
\pgfpathlineto{\pgfqpoint{5.089222in}{3.259718in}}%
\pgfpathlineto{\pgfqpoint{5.094591in}{3.258667in}}%
\pgfpathlineto{\pgfqpoint{5.096738in}{3.260639in}}%
\pgfpathlineto{\pgfqpoint{5.097812in}{3.259807in}}%
\pgfpathlineto{\pgfqpoint{5.101034in}{3.259681in}}%
\pgfpathlineto{\pgfqpoint{5.102108in}{3.260680in}}%
\pgfpathlineto{\pgfqpoint{5.103181in}{3.259383in}}%
\pgfpathlineto{\pgfqpoint{5.104255in}{3.260390in}}%
\pgfpathlineto{\pgfqpoint{5.105329in}{3.263105in}}%
\pgfpathlineto{\pgfqpoint{5.108551in}{3.263897in}}%
\pgfpathlineto{\pgfqpoint{5.109625in}{3.262665in}}%
\pgfpathlineto{\pgfqpoint{5.111772in}{3.264967in}}%
\pgfpathlineto{\pgfqpoint{5.112846in}{3.263947in}}%
\pgfpathlineto{\pgfqpoint{5.116068in}{3.263591in}}%
\pgfpathlineto{\pgfqpoint{5.117141in}{3.261749in}}%
\pgfpathlineto{\pgfqpoint{5.118215in}{3.262311in}}%
\pgfpathlineto{\pgfqpoint{5.120363in}{3.260769in}}%
\pgfpathlineto{\pgfqpoint{5.123584in}{3.260731in}}%
\pgfpathlineto{\pgfqpoint{5.125732in}{3.259581in}}%
\pgfpathlineto{\pgfqpoint{5.126806in}{3.261659in}}%
\pgfpathlineto{\pgfqpoint{5.127880in}{3.259256in}}%
\pgfpathlineto{\pgfqpoint{5.132175in}{3.258446in}}%
\pgfpathlineto{\pgfqpoint{5.133249in}{3.259272in}}%
\pgfpathlineto{\pgfqpoint{5.135397in}{3.259555in}}%
\pgfpathlineto{\pgfqpoint{5.140766in}{3.259875in}}%
\pgfpathlineto{\pgfqpoint{5.142914in}{3.258973in}}%
\pgfpathlineto{\pgfqpoint{5.148283in}{3.258346in}}%
\pgfpathlineto{\pgfqpoint{5.149357in}{3.255944in}}%
\pgfpathlineto{\pgfqpoint{5.150430in}{3.256428in}}%
\pgfpathlineto{\pgfqpoint{5.157947in}{3.254187in}}%
\pgfpathlineto{\pgfqpoint{5.161169in}{3.254288in}}%
\pgfpathlineto{\pgfqpoint{5.162243in}{3.253748in}}%
\pgfpathlineto{\pgfqpoint{5.164390in}{3.253919in}}%
\pgfpathlineto{\pgfqpoint{5.165464in}{3.253646in}}%
\pgfpathlineto{\pgfqpoint{5.169759in}{3.254637in}}%
\pgfpathlineto{\pgfqpoint{5.170833in}{3.255245in}}%
\pgfpathlineto{\pgfqpoint{5.171907in}{3.254864in}}%
\pgfpathlineto{\pgfqpoint{5.172981in}{3.255478in}}%
\pgfpathlineto{\pgfqpoint{5.177276in}{3.256393in}}%
\pgfpathlineto{\pgfqpoint{5.179424in}{3.254393in}}%
\pgfpathlineto{\pgfqpoint{5.180498in}{3.255231in}}%
\pgfpathlineto{\pgfqpoint{5.185867in}{3.253828in}}%
\pgfpathlineto{\pgfqpoint{5.188015in}{3.253685in}}%
\pgfpathlineto{\pgfqpoint{5.192310in}{3.254680in}}%
\pgfpathlineto{\pgfqpoint{5.193384in}{3.255291in}}%
\pgfpathlineto{\pgfqpoint{5.195532in}{3.255199in}}%
\pgfpathlineto{\pgfqpoint{5.198753in}{3.255747in}}%
\pgfpathlineto{\pgfqpoint{5.199827in}{3.256666in}}%
\pgfpathlineto{\pgfqpoint{5.201975in}{3.254701in}}%
\pgfpathlineto{\pgfqpoint{5.206270in}{3.254819in}}%
\pgfpathlineto{\pgfqpoint{5.214861in}{3.255495in}}%
\pgfpathlineto{\pgfqpoint{5.217008in}{3.254396in}}%
\pgfpathlineto{\pgfqpoint{5.218082in}{3.254986in}}%
\pgfpathlineto{\pgfqpoint{5.221304in}{3.255832in}}%
\pgfpathlineto{\pgfqpoint{5.222378in}{3.258056in}}%
\pgfpathlineto{\pgfqpoint{5.223451in}{3.258673in}}%
\pgfpathlineto{\pgfqpoint{5.224525in}{3.258005in}}%
\pgfpathlineto{\pgfqpoint{5.225599in}{3.259751in}}%
\pgfpathlineto{\pgfqpoint{5.232042in}{3.257957in}}%
\pgfpathlineto{\pgfqpoint{5.233116in}{3.258932in}}%
\pgfpathlineto{\pgfqpoint{5.236337in}{3.259501in}}%
\pgfpathlineto{\pgfqpoint{5.237411in}{3.258283in}}%
\pgfpathlineto{\pgfqpoint{5.238485in}{3.258509in}}%
\pgfpathlineto{\pgfqpoint{5.240633in}{3.257261in}}%
\pgfpathlineto{\pgfqpoint{5.244928in}{3.256711in}}%
\pgfpathlineto{\pgfqpoint{5.247076in}{3.257481in}}%
\pgfpathlineto{\pgfqpoint{5.248150in}{3.257178in}}%
\pgfpathlineto{\pgfqpoint{5.251371in}{3.258125in}}%
\pgfpathlineto{\pgfqpoint{5.252445in}{3.257142in}}%
\pgfpathlineto{\pgfqpoint{5.255667in}{3.258874in}}%
\pgfpathlineto{\pgfqpoint{5.259962in}{3.257544in}}%
\pgfpathlineto{\pgfqpoint{5.261036in}{3.257527in}}%
\pgfpathlineto{\pgfqpoint{5.263183in}{3.258913in}}%
\pgfpathlineto{\pgfqpoint{5.266405in}{3.259661in}}%
\pgfpathlineto{\pgfqpoint{5.267479in}{3.261312in}}%
\pgfpathlineto{\pgfqpoint{5.268553in}{3.260219in}}%
\pgfpathlineto{\pgfqpoint{5.269626in}{3.262751in}}%
\pgfpathlineto{\pgfqpoint{5.270700in}{3.262524in}}%
\pgfpathlineto{\pgfqpoint{5.277143in}{3.262843in}}%
\pgfpathlineto{\pgfqpoint{5.278217in}{3.261808in}}%
\pgfpathlineto{\pgfqpoint{5.281439in}{3.262008in}}%
\pgfpathlineto{\pgfqpoint{5.282513in}{3.262731in}}%
\pgfpathlineto{\pgfqpoint{5.283586in}{3.264360in}}%
\pgfpathlineto{\pgfqpoint{5.284660in}{3.263989in}}%
\pgfpathlineto{\pgfqpoint{5.285734in}{3.260341in}}%
\pgfpathlineto{\pgfqpoint{5.292177in}{3.262383in}}%
\pgfpathlineto{\pgfqpoint{5.299694in}{3.261609in}}%
\pgfpathlineto{\pgfqpoint{5.300768in}{3.260394in}}%
\pgfpathlineto{\pgfqpoint{5.305063in}{3.261142in}}%
\pgfpathlineto{\pgfqpoint{5.306137in}{3.258831in}}%
\pgfpathlineto{\pgfqpoint{5.307211in}{3.258777in}}%
\pgfpathlineto{\pgfqpoint{5.308285in}{3.259405in}}%
\pgfpathlineto{\pgfqpoint{5.311506in}{3.258982in}}%
\pgfpathlineto{\pgfqpoint{5.312580in}{3.259453in}}%
\pgfpathlineto{\pgfqpoint{5.313654in}{3.259029in}}%
\pgfpathlineto{\pgfqpoint{5.315802in}{3.260017in}}%
\pgfpathlineto{\pgfqpoint{5.319023in}{3.259320in}}%
\pgfpathlineto{\pgfqpoint{5.320097in}{3.258476in}}%
\pgfpathlineto{\pgfqpoint{5.322245in}{3.259227in}}%
\pgfpathlineto{\pgfqpoint{5.323318in}{3.257819in}}%
\pgfpathlineto{\pgfqpoint{5.326540in}{3.257802in}}%
\pgfpathlineto{\pgfqpoint{5.328688in}{3.259313in}}%
\pgfpathlineto{\pgfqpoint{5.329761in}{3.259294in}}%
\pgfpathlineto{\pgfqpoint{5.330835in}{3.258377in}}%
\pgfpathlineto{\pgfqpoint{5.334057in}{3.258554in}}%
\pgfpathlineto{\pgfqpoint{5.335131in}{3.259250in}}%
\pgfpathlineto{\pgfqpoint{5.337278in}{3.258089in}}%
\pgfpathlineto{\pgfqpoint{5.343721in}{3.257452in}}%
\pgfpathlineto{\pgfqpoint{5.345869in}{3.257331in}}%
\pgfpathlineto{\pgfqpoint{5.350164in}{3.257859in}}%
\pgfpathlineto{\pgfqpoint{5.352312in}{3.256484in}}%
\pgfpathlineto{\pgfqpoint{5.353386in}{3.255825in}}%
\pgfpathlineto{\pgfqpoint{5.356607in}{3.256595in}}%
\pgfpathlineto{\pgfqpoint{5.360903in}{3.259786in}}%
\pgfpathlineto{\pgfqpoint{5.365198in}{3.260764in}}%
\pgfpathlineto{\pgfqpoint{5.366272in}{3.259790in}}%
\pgfpathlineto{\pgfqpoint{5.367346in}{3.259734in}}%
\pgfpathlineto{\pgfqpoint{5.368420in}{3.260615in}}%
\pgfpathlineto{\pgfqpoint{5.372715in}{3.260074in}}%
\pgfpathlineto{\pgfqpoint{5.374863in}{3.258836in}}%
\pgfpathlineto{\pgfqpoint{5.375936in}{3.259343in}}%
\pgfpathlineto{\pgfqpoint{5.380232in}{3.259303in}}%
\pgfpathlineto{\pgfqpoint{5.382379in}{3.259467in}}%
\pgfpathlineto{\pgfqpoint{5.383453in}{3.261063in}}%
\pgfpathlineto{\pgfqpoint{5.388823in}{3.262536in}}%
\pgfpathlineto{\pgfqpoint{5.389896in}{3.261849in}}%
\pgfpathlineto{\pgfqpoint{5.396339in}{3.263799in}}%
\pgfpathlineto{\pgfqpoint{5.397413in}{3.264888in}}%
\pgfpathlineto{\pgfqpoint{5.401709in}{3.263299in}}%
\pgfpathlineto{\pgfqpoint{5.403856in}{3.263385in}}%
\pgfpathlineto{\pgfqpoint{5.404930in}{3.262977in}}%
\pgfpathlineto{\pgfqpoint{5.406004in}{3.263781in}}%
\pgfpathlineto{\pgfqpoint{5.412447in}{3.265513in}}%
\pgfpathlineto{\pgfqpoint{5.413521in}{3.266601in}}%
\pgfpathlineto{\pgfqpoint{5.416742in}{3.266000in}}%
\pgfpathlineto{\pgfqpoint{5.417816in}{3.264729in}}%
\pgfpathlineto{\pgfqpoint{5.427481in}{3.264142in}}%
\pgfpathlineto{\pgfqpoint{5.428555in}{3.264385in}}%
\pgfpathlineto{\pgfqpoint{5.431776in}{3.262869in}}%
\pgfpathlineto{\pgfqpoint{5.432850in}{3.264138in}}%
\pgfpathlineto{\pgfqpoint{5.433924in}{3.263187in}}%
\pgfpathlineto{\pgfqpoint{5.434998in}{3.264300in}}%
\pgfpathlineto{\pgfqpoint{5.436071in}{3.264010in}}%
\pgfpathlineto{\pgfqpoint{5.440367in}{3.264229in}}%
\pgfpathlineto{\pgfqpoint{5.442514in}{3.267024in}}%
\pgfpathlineto{\pgfqpoint{5.443588in}{3.269152in}}%
\pgfpathlineto{\pgfqpoint{5.446810in}{3.269887in}}%
\pgfpathlineto{\pgfqpoint{5.447884in}{3.270800in}}%
\pgfpathlineto{\pgfqpoint{5.448957in}{3.267260in}}%
\pgfpathlineto{\pgfqpoint{5.451105in}{3.264831in}}%
\pgfpathlineto{\pgfqpoint{5.454327in}{3.264580in}}%
\pgfpathlineto{\pgfqpoint{5.455401in}{3.266028in}}%
\pgfpathlineto{\pgfqpoint{5.457548in}{3.263418in}}%
\pgfpathlineto{\pgfqpoint{5.458622in}{3.264506in}}%
\pgfpathlineto{\pgfqpoint{5.462917in}{3.262515in}}%
\pgfpathlineto{\pgfqpoint{5.465065in}{3.262958in}}%
\pgfpathlineto{\pgfqpoint{5.466139in}{3.262572in}}%
\pgfpathlineto{\pgfqpoint{5.469360in}{3.262720in}}%
\pgfpathlineto{\pgfqpoint{5.471508in}{3.261998in}}%
\pgfpathlineto{\pgfqpoint{5.472582in}{3.262102in}}%
\pgfpathlineto{\pgfqpoint{5.473656in}{3.263392in}}%
\pgfpathlineto{\pgfqpoint{5.476877in}{3.263109in}}%
\pgfpathlineto{\pgfqpoint{5.477951in}{3.264058in}}%
\pgfpathlineto{\pgfqpoint{5.479025in}{3.263924in}}%
\pgfpathlineto{\pgfqpoint{5.480099in}{3.264435in}}%
\pgfpathlineto{\pgfqpoint{5.481173in}{3.263779in}}%
\pgfpathlineto{\pgfqpoint{5.484394in}{3.263868in}}%
\pgfpathlineto{\pgfqpoint{5.485468in}{3.262914in}}%
\pgfpathlineto{\pgfqpoint{5.486542in}{3.261198in}}%
\pgfpathlineto{\pgfqpoint{5.487616in}{3.261440in}}%
\pgfpathlineto{\pgfqpoint{5.488690in}{3.260524in}}%
\pgfpathlineto{\pgfqpoint{5.492985in}{3.258400in}}%
\pgfpathlineto{\pgfqpoint{5.495133in}{3.257139in}}%
\pgfpathlineto{\pgfqpoint{5.496206in}{3.257728in}}%
\pgfpathlineto{\pgfqpoint{5.500502in}{3.257886in}}%
\pgfpathlineto{\pgfqpoint{5.501576in}{3.256674in}}%
\pgfpathlineto{\pgfqpoint{5.503723in}{3.256299in}}%
\pgfpathlineto{\pgfqpoint{5.508019in}{3.255705in}}%
\pgfpathlineto{\pgfqpoint{5.509092in}{3.255803in}}%
\pgfpathlineto{\pgfqpoint{5.511240in}{3.253598in}}%
\pgfpathlineto{\pgfqpoint{5.521979in}{3.254658in}}%
\pgfpathlineto{\pgfqpoint{5.524126in}{3.254252in}}%
\pgfpathlineto{\pgfqpoint{5.531643in}{3.256109in}}%
\pgfpathlineto{\pgfqpoint{5.533791in}{3.257237in}}%
\pgfpathlineto{\pgfqpoint{5.537012in}{3.257254in}}%
\pgfpathlineto{\pgfqpoint{5.539160in}{3.255624in}}%
\pgfpathlineto{\pgfqpoint{5.540234in}{3.253940in}}%
\pgfpathlineto{\pgfqpoint{5.541308in}{3.253452in}}%
\pgfpathlineto{\pgfqpoint{5.548824in}{3.253720in}}%
\pgfpathlineto{\pgfqpoint{5.554194in}{3.253201in}}%
\pgfpathlineto{\pgfqpoint{5.555267in}{3.254251in}}%
\pgfpathlineto{\pgfqpoint{5.556341in}{3.252998in}}%
\pgfpathlineto{\pgfqpoint{5.562784in}{3.253232in}}%
\pgfpathlineto{\pgfqpoint{5.563858in}{3.253899in}}%
\pgfpathlineto{\pgfqpoint{5.567080in}{3.253624in}}%
\pgfpathlineto{\pgfqpoint{5.568154in}{3.252659in}}%
\pgfpathlineto{\pgfqpoint{5.569227in}{3.252500in}}%
\pgfpathlineto{\pgfqpoint{5.570301in}{3.253003in}}%
\pgfpathlineto{\pgfqpoint{5.571375in}{3.254384in}}%
\pgfpathlineto{\pgfqpoint{5.575670in}{3.253183in}}%
\pgfpathlineto{\pgfqpoint{5.577818in}{3.252856in}}%
\pgfpathlineto{\pgfqpoint{5.582113in}{3.252914in}}%
\pgfpathlineto{\pgfqpoint{5.583187in}{3.252240in}}%
\pgfpathlineto{\pgfqpoint{5.585335in}{3.253525in}}%
\pgfpathlineto{\pgfqpoint{5.590704in}{3.254374in}}%
\pgfpathlineto{\pgfqpoint{5.593926in}{3.257773in}}%
\pgfpathlineto{\pgfqpoint{5.597147in}{3.256894in}}%
\pgfpathlineto{\pgfqpoint{5.598221in}{3.255922in}}%
\pgfpathlineto{\pgfqpoint{5.599295in}{3.257073in}}%
\pgfpathlineto{\pgfqpoint{5.600369in}{3.255779in}}%
\pgfpathlineto{\pgfqpoint{5.601443in}{3.260231in}}%
\pgfpathlineto{\pgfqpoint{5.615402in}{3.259832in}}%
\pgfpathlineto{\pgfqpoint{5.616476in}{3.257966in}}%
\pgfpathlineto{\pgfqpoint{5.619698in}{3.258296in}}%
\pgfpathlineto{\pgfqpoint{5.620772in}{3.260002in}}%
\pgfpathlineto{\pgfqpoint{5.621845in}{3.260378in}}%
\pgfpathlineto{\pgfqpoint{5.622919in}{3.259575in}}%
\pgfpathlineto{\pgfqpoint{5.623993in}{3.260881in}}%
\pgfpathlineto{\pgfqpoint{5.628289in}{3.261312in}}%
\pgfpathlineto{\pgfqpoint{5.629362in}{3.262395in}}%
\pgfpathlineto{\pgfqpoint{5.630436in}{3.262416in}}%
\pgfpathlineto{\pgfqpoint{5.631510in}{3.261592in}}%
\pgfpathlineto{\pgfqpoint{5.635805in}{3.261319in}}%
\pgfpathlineto{\pgfqpoint{5.636879in}{3.260027in}}%
\pgfpathlineto{\pgfqpoint{5.637953in}{3.260106in}}%
\pgfpathlineto{\pgfqpoint{5.639027in}{3.261402in}}%
\pgfpathlineto{\pgfqpoint{5.642248in}{3.260188in}}%
\pgfpathlineto{\pgfqpoint{5.643322in}{3.261189in}}%
\pgfpathlineto{\pgfqpoint{5.646544in}{3.259338in}}%
\pgfpathlineto{\pgfqpoint{5.649765in}{3.259707in}}%
\pgfpathlineto{\pgfqpoint{5.650839in}{3.258310in}}%
\pgfpathlineto{\pgfqpoint{5.654061in}{3.257864in}}%
\pgfpathlineto{\pgfqpoint{5.657282in}{3.257350in}}%
\pgfpathlineto{\pgfqpoint{5.658356in}{3.257979in}}%
\pgfpathlineto{\pgfqpoint{5.661578in}{3.256021in}}%
\pgfpathlineto{\pgfqpoint{5.665873in}{3.256184in}}%
\pgfpathlineto{\pgfqpoint{5.668021in}{3.255670in}}%
\pgfpathlineto{\pgfqpoint{5.669094in}{3.254594in}}%
\pgfpathlineto{\pgfqpoint{5.679833in}{3.255754in}}%
\pgfpathlineto{\pgfqpoint{5.681980in}{3.254519in}}%
\pgfpathlineto{\pgfqpoint{5.683054in}{3.255049in}}%
\pgfpathlineto{\pgfqpoint{5.684128in}{3.254901in}}%
\pgfpathlineto{\pgfqpoint{5.691645in}{3.256114in}}%
\pgfpathlineto{\pgfqpoint{5.695940in}{3.255771in}}%
\pgfpathlineto{\pgfqpoint{5.697014in}{3.255348in}}%
\pgfpathlineto{\pgfqpoint{5.699162in}{3.256373in}}%
\pgfpathlineto{\pgfqpoint{5.706679in}{3.256083in}}%
\pgfpathlineto{\pgfqpoint{5.710974in}{3.256442in}}%
\pgfpathlineto{\pgfqpoint{5.712048in}{3.255903in}}%
\pgfpathlineto{\pgfqpoint{5.714196in}{3.258219in}}%
\pgfpathlineto{\pgfqpoint{5.717417in}{3.257659in}}%
\pgfpathlineto{\pgfqpoint{5.720639in}{3.258412in}}%
\pgfpathlineto{\pgfqpoint{5.721712in}{3.258167in}}%
\pgfpathlineto{\pgfqpoint{5.724934in}{3.258578in}}%
\pgfpathlineto{\pgfqpoint{5.726008in}{3.257990in}}%
\pgfpathlineto{\pgfqpoint{5.729229in}{3.258376in}}%
\pgfpathlineto{\pgfqpoint{5.732451in}{3.257548in}}%
\pgfpathlineto{\pgfqpoint{5.733525in}{3.258240in}}%
\pgfpathlineto{\pgfqpoint{5.734599in}{3.258221in}}%
\pgfpathlineto{\pgfqpoint{5.735672in}{3.258839in}}%
\pgfpathlineto{\pgfqpoint{5.736746in}{3.257919in}}%
\pgfpathlineto{\pgfqpoint{5.739968in}{3.257790in}}%
\pgfpathlineto{\pgfqpoint{5.742115in}{3.255868in}}%
\pgfpathlineto{\pgfqpoint{5.744263in}{3.255579in}}%
\pgfpathlineto{\pgfqpoint{5.758223in}{3.255009in}}%
\pgfpathlineto{\pgfqpoint{5.763592in}{3.254713in}}%
\pgfpathlineto{\pgfqpoint{5.764666in}{3.255494in}}%
\pgfpathlineto{\pgfqpoint{5.772183in}{3.254464in}}%
\pgfpathlineto{\pgfqpoint{5.773257in}{3.253433in}}%
\pgfpathlineto{\pgfqpoint{5.774331in}{3.255469in}}%
\pgfpathlineto{\pgfqpoint{5.777552in}{3.256748in}}%
\pgfpathlineto{\pgfqpoint{5.779700in}{3.254816in}}%
\pgfpathlineto{\pgfqpoint{5.780774in}{3.253503in}}%
\pgfpathlineto{\pgfqpoint{5.786143in}{3.253674in}}%
\pgfpathlineto{\pgfqpoint{5.788290in}{3.252945in}}%
\pgfpathlineto{\pgfqpoint{5.789364in}{3.251822in}}%
\pgfpathlineto{\pgfqpoint{5.792586in}{3.251331in}}%
\pgfpathlineto{\pgfqpoint{5.794733in}{3.250514in}}%
\pgfpathlineto{\pgfqpoint{5.796881in}{3.250431in}}%
\pgfpathlineto{\pgfqpoint{5.801177in}{3.250335in}}%
\pgfpathlineto{\pgfqpoint{5.802250in}{3.249660in}}%
\pgfpathlineto{\pgfqpoint{5.803324in}{3.251376in}}%
\pgfpathlineto{\pgfqpoint{5.804398in}{3.250865in}}%
\pgfpathlineto{\pgfqpoint{5.818358in}{3.250614in}}%
\pgfpathlineto{\pgfqpoint{5.819432in}{3.250107in}}%
\pgfpathlineto{\pgfqpoint{5.823727in}{3.250187in}}%
\pgfpathlineto{\pgfqpoint{5.824801in}{3.250671in}}%
\pgfpathlineto{\pgfqpoint{5.831244in}{3.249812in}}%
\pgfpathlineto{\pgfqpoint{5.833392in}{3.250104in}}%
\pgfpathlineto{\pgfqpoint{5.834466in}{3.249769in}}%
\pgfpathlineto{\pgfqpoint{5.849499in}{3.248767in}}%
\pgfpathlineto{\pgfqpoint{5.855942in}{3.248334in}}%
\pgfpathlineto{\pgfqpoint{5.857016in}{3.249482in}}%
\pgfpathlineto{\pgfqpoint{5.860238in}{3.248712in}}%
\pgfpathlineto{\pgfqpoint{5.862385in}{3.249240in}}%
\pgfpathlineto{\pgfqpoint{5.864533in}{3.246870in}}%
\pgfpathlineto{\pgfqpoint{5.868828in}{3.247435in}}%
\pgfpathlineto{\pgfqpoint{5.870976in}{3.246987in}}%
\pgfpathlineto{\pgfqpoint{5.875271in}{3.247960in}}%
\pgfpathlineto{\pgfqpoint{5.877419in}{3.247080in}}%
\pgfpathlineto{\pgfqpoint{5.878493in}{3.247208in}}%
\pgfpathlineto{\pgfqpoint{5.879567in}{3.246739in}}%
\pgfpathlineto{\pgfqpoint{5.886010in}{3.246403in}}%
\pgfpathlineto{\pgfqpoint{5.890305in}{3.246448in}}%
\pgfpathlineto{\pgfqpoint{5.892453in}{3.247391in}}%
\pgfpathlineto{\pgfqpoint{5.893527in}{3.247557in}}%
\pgfpathlineto{\pgfqpoint{5.894600in}{3.247032in}}%
\pgfpathlineto{\pgfqpoint{5.897822in}{3.247206in}}%
\pgfpathlineto{\pgfqpoint{5.898896in}{3.246702in}}%
\pgfpathlineto{\pgfqpoint{5.899970in}{3.249071in}}%
\pgfpathlineto{\pgfqpoint{5.917151in}{3.251103in}}%
\pgfpathlineto{\pgfqpoint{5.921446in}{3.249620in}}%
\pgfpathlineto{\pgfqpoint{5.935406in}{3.249320in}}%
\pgfpathlineto{\pgfqpoint{5.936480in}{3.248706in}}%
\pgfpathlineto{\pgfqpoint{5.937554in}{3.249036in}}%
\pgfpathlineto{\pgfqpoint{5.942923in}{3.248663in}}%
\pgfpathlineto{\pgfqpoint{5.945071in}{3.249634in}}%
\pgfpathlineto{\pgfqpoint{5.946145in}{3.250788in}}%
\pgfpathlineto{\pgfqpoint{5.947219in}{3.250270in}}%
\pgfpathlineto{\pgfqpoint{5.951514in}{3.249564in}}%
\pgfpathlineto{\pgfqpoint{5.953662in}{3.248370in}}%
\pgfpathlineto{\pgfqpoint{5.957957in}{3.248058in}}%
\pgfpathlineto{\pgfqpoint{5.959031in}{3.247097in}}%
\pgfpathlineto{\pgfqpoint{5.962252in}{3.247635in}}%
\pgfpathlineto{\pgfqpoint{5.967621in}{3.246884in}}%
\pgfpathlineto{\pgfqpoint{5.989098in}{3.247350in}}%
\pgfpathlineto{\pgfqpoint{5.990172in}{3.246885in}}%
\pgfpathlineto{\pgfqpoint{5.992320in}{3.247059in}}%
\pgfpathlineto{\pgfqpoint{6.003058in}{3.247076in}}%
\pgfpathlineto{\pgfqpoint{6.004132in}{3.246146in}}%
\pgfpathlineto{\pgfqpoint{6.007354in}{3.245771in}}%
\pgfpathlineto{\pgfqpoint{6.011649in}{3.246981in}}%
\pgfpathlineto{\pgfqpoint{6.013797in}{3.247131in}}%
\pgfpathlineto{\pgfqpoint{6.014870in}{3.247013in}}%
\pgfpathlineto{\pgfqpoint{6.020240in}{3.247164in}}%
\pgfpathlineto{\pgfqpoint{6.021313in}{3.248194in}}%
\pgfpathlineto{\pgfqpoint{6.022387in}{3.248331in}}%
\pgfpathlineto{\pgfqpoint{6.029904in}{3.247019in}}%
\pgfpathlineto{\pgfqpoint{6.041716in}{3.247314in}}%
\pgfpathlineto{\pgfqpoint{6.044938in}{3.247649in}}%
\pgfpathlineto{\pgfqpoint{6.050307in}{3.247973in}}%
\pgfpathlineto{\pgfqpoint{6.052455in}{3.247629in}}%
\pgfpathlineto{\pgfqpoint{6.057824in}{3.248553in}}%
\pgfpathlineto{\pgfqpoint{6.078227in}{3.247315in}}%
\pgfpathlineto{\pgfqpoint{6.081448in}{3.247457in}}%
\pgfpathlineto{\pgfqpoint{6.087891in}{3.247917in}}%
\pgfpathlineto{\pgfqpoint{6.088965in}{3.248348in}}%
\pgfpathlineto{\pgfqpoint{6.095408in}{3.247578in}}%
\pgfpathlineto{\pgfqpoint{6.097556in}{3.247120in}}%
\pgfpathlineto{\pgfqpoint{6.103999in}{3.245911in}}%
\pgfpathlineto{\pgfqpoint{6.105073in}{3.247249in}}%
\pgfpathlineto{\pgfqpoint{6.108294in}{3.247070in}}%
\pgfpathlineto{\pgfqpoint{6.110442in}{3.246028in}}%
\pgfpathlineto{\pgfqpoint{6.123328in}{3.247509in}}%
\pgfpathlineto{\pgfqpoint{6.124402in}{3.247291in}}%
\pgfpathlineto{\pgfqpoint{6.125476in}{3.248165in}}%
\pgfpathlineto{\pgfqpoint{6.135140in}{3.246766in}}%
\pgfpathlineto{\pgfqpoint{6.155543in}{3.247560in}}%
\pgfpathlineto{\pgfqpoint{6.157691in}{3.247926in}}%
\pgfpathlineto{\pgfqpoint{6.160912in}{3.247580in}}%
\pgfpathlineto{\pgfqpoint{6.163060in}{3.248663in}}%
\pgfpathlineto{\pgfqpoint{6.165208in}{3.249140in}}%
\pgfpathlineto{\pgfqpoint{6.175946in}{3.250025in}}%
\pgfpathlineto{\pgfqpoint{6.178094in}{3.248889in}}%
\pgfpathlineto{\pgfqpoint{6.179168in}{3.249759in}}%
\pgfpathlineto{\pgfqpoint{6.180242in}{3.249434in}}%
\pgfpathlineto{\pgfqpoint{6.184537in}{3.249376in}}%
\pgfpathlineto{\pgfqpoint{6.187758in}{3.248429in}}%
\pgfpathlineto{\pgfqpoint{6.193128in}{3.248581in}}%
\pgfpathlineto{\pgfqpoint{6.195275in}{3.248376in}}%
\pgfpathlineto{\pgfqpoint{6.200644in}{3.248347in}}%
\pgfpathlineto{\pgfqpoint{6.210309in}{3.246243in}}%
\pgfpathlineto{\pgfqpoint{6.216752in}{3.246408in}}%
\pgfpathlineto{\pgfqpoint{6.217826in}{3.246707in}}%
\pgfpathlineto{\pgfqpoint{6.222121in}{3.246545in}}%
\pgfpathlineto{\pgfqpoint{6.231786in}{3.248008in}}%
\pgfpathlineto{\pgfqpoint{6.238229in}{3.247796in}}%
\pgfpathlineto{\pgfqpoint{6.240376in}{3.247561in}}%
\pgfpathlineto{\pgfqpoint{6.244672in}{3.247647in}}%
\pgfpathlineto{\pgfqpoint{6.245746in}{3.246776in}}%
\pgfpathlineto{\pgfqpoint{6.247893in}{3.247415in}}%
\pgfpathlineto{\pgfqpoint{6.259706in}{3.245166in}}%
\pgfpathlineto{\pgfqpoint{6.262927in}{3.245219in}}%
\pgfpathlineto{\pgfqpoint{6.268296in}{3.244850in}}%
\pgfpathlineto{\pgfqpoint{6.276887in}{3.242897in}}%
\pgfpathlineto{\pgfqpoint{6.282256in}{3.242790in}}%
\pgfpathlineto{\pgfqpoint{6.284404in}{3.243211in}}%
\pgfpathlineto{\pgfqpoint{6.285478in}{3.242762in}}%
\pgfpathlineto{\pgfqpoint{6.292995in}{3.242081in}}%
\pgfpathlineto{\pgfqpoint{6.299438in}{3.241280in}}%
\pgfpathlineto{\pgfqpoint{6.300511in}{3.238738in}}%
\pgfpathlineto{\pgfqpoint{6.303733in}{3.238761in}}%
\pgfpathlineto{\pgfqpoint{6.306954in}{3.236863in}}%
\pgfpathlineto{\pgfqpoint{6.308028in}{3.237353in}}%
\pgfpathlineto{\pgfqpoint{6.313397in}{3.236930in}}%
\pgfpathlineto{\pgfqpoint{6.315545in}{3.237659in}}%
\pgfpathlineto{\pgfqpoint{6.321988in}{3.237613in}}%
\pgfpathlineto{\pgfqpoint{6.323062in}{3.238298in}}%
\pgfpathlineto{\pgfqpoint{6.341317in}{3.238373in}}%
\pgfpathlineto{\pgfqpoint{6.342391in}{3.239115in}}%
\pgfpathlineto{\pgfqpoint{6.352056in}{3.239240in}}%
\pgfpathlineto{\pgfqpoint{6.353130in}{3.238268in}}%
\pgfpathlineto{\pgfqpoint{6.358499in}{3.236229in}}%
\pgfpathlineto{\pgfqpoint{6.360646in}{3.236760in}}%
\pgfpathlineto{\pgfqpoint{6.368163in}{3.237108in}}%
\pgfpathlineto{\pgfqpoint{6.368163in}{3.237108in}}%
\pgfusepath{stroke}%
\end{pgfscope}%
\begin{pgfscope}%
\pgfsetrectcap%
\pgfsetmiterjoin%
\pgfsetlinewidth{0.803000pt}%
\definecolor{currentstroke}{rgb}{1.000000,1.000000,1.000000}%
\pgfsetstrokecolor{currentstroke}%
\pgfsetdash{}{0pt}%
\pgfpathmoveto{\pgfqpoint{3.902254in}{3.218007in}}%
\pgfpathlineto{\pgfqpoint{3.902254in}{3.618892in}}%
\pgfusepath{stroke}%
\end{pgfscope}%
\begin{pgfscope}%
\pgfsetrectcap%
\pgfsetmiterjoin%
\pgfsetlinewidth{0.803000pt}%
\definecolor{currentstroke}{rgb}{1.000000,1.000000,1.000000}%
\pgfsetstrokecolor{currentstroke}%
\pgfsetdash{}{0pt}%
\pgfpathmoveto{\pgfqpoint{6.485588in}{3.218007in}}%
\pgfpathlineto{\pgfqpoint{6.485588in}{3.618892in}}%
\pgfusepath{stroke}%
\end{pgfscope}%
\begin{pgfscope}%
\pgfsetrectcap%
\pgfsetmiterjoin%
\pgfsetlinewidth{0.803000pt}%
\definecolor{currentstroke}{rgb}{1.000000,1.000000,1.000000}%
\pgfsetstrokecolor{currentstroke}%
\pgfsetdash{}{0pt}%
\pgfpathmoveto{\pgfqpoint{3.902254in}{3.218007in}}%
\pgfpathlineto{\pgfqpoint{6.485588in}{3.218007in}}%
\pgfusepath{stroke}%
\end{pgfscope}%
\begin{pgfscope}%
\pgfsetrectcap%
\pgfsetmiterjoin%
\pgfsetlinewidth{0.803000pt}%
\definecolor{currentstroke}{rgb}{1.000000,1.000000,1.000000}%
\pgfsetstrokecolor{currentstroke}%
\pgfsetdash{}{0pt}%
\pgfpathmoveto{\pgfqpoint{3.902254in}{3.618892in}}%
\pgfpathlineto{\pgfqpoint{6.485588in}{3.618892in}}%
\pgfusepath{stroke}%
\end{pgfscope}%
\begin{pgfscope}%
\definecolor{textcolor}{rgb}{0.150000,0.150000,0.150000}%
\pgfsetstrokecolor{textcolor}%
\pgfsetfillcolor{textcolor}%
\pgftext[x=5.193921in,y=3.702225in,,base]{\color{textcolor}\rmfamily\fontsize{12.000000}{14.400000}\selectfont INTC}%
\end{pgfscope}%
\begin{pgfscope}%
\pgfsetbuttcap%
\pgfsetmiterjoin%
\definecolor{currentfill}{rgb}{0.917647,0.917647,0.949020}%
\pgfsetfillcolor{currentfill}%
\pgfsetlinewidth{0.000000pt}%
\definecolor{currentstroke}{rgb}{0.000000,0.000000,0.000000}%
\pgfsetstrokecolor{currentstroke}%
\pgfsetstrokeopacity{0.000000}%
\pgfsetdash{}{0pt}%
\pgfpathmoveto{\pgfqpoint{0.285588in}{2.255883in}}%
\pgfpathlineto{\pgfqpoint{2.868921in}{2.255883in}}%
\pgfpathlineto{\pgfqpoint{2.868921in}{2.656768in}}%
\pgfpathlineto{\pgfqpoint{0.285588in}{2.656768in}}%
\pgfpathclose%
\pgfusepath{fill}%
\end{pgfscope}%
\begin{pgfscope}%
\pgfpathrectangle{\pgfqpoint{0.285588in}{2.255883in}}{\pgfqpoint{2.583333in}{0.400885in}}%
\pgfusepath{clip}%
\pgfsetroundcap%
\pgfsetroundjoin%
\pgfsetlinewidth{0.803000pt}%
\definecolor{currentstroke}{rgb}{1.000000,1.000000,1.000000}%
\pgfsetstrokecolor{currentstroke}%
\pgfsetdash{}{0pt}%
\pgfpathmoveto{\pgfqpoint{0.400864in}{2.255883in}}%
\pgfpathlineto{\pgfqpoint{0.400864in}{2.656768in}}%
\pgfusepath{stroke}%
\end{pgfscope}%
\begin{pgfscope}%
\definecolor{textcolor}{rgb}{0.150000,0.150000,0.150000}%
\pgfsetstrokecolor{textcolor}%
\pgfsetfillcolor{textcolor}%
\pgftext[x=0.400864in,y=2.158661in,,top]{\color{textcolor}\rmfamily\fontsize{10.000000}{12.000000}\selectfont 2012}%
\end{pgfscope}%
\begin{pgfscope}%
\pgfpathrectangle{\pgfqpoint{0.285588in}{2.255883in}}{\pgfqpoint{2.583333in}{0.400885in}}%
\pgfusepath{clip}%
\pgfsetroundcap%
\pgfsetroundjoin%
\pgfsetlinewidth{0.803000pt}%
\definecolor{currentstroke}{rgb}{1.000000,1.000000,1.000000}%
\pgfsetstrokecolor{currentstroke}%
\pgfsetdash{}{0pt}%
\pgfpathmoveto{\pgfqpoint{0.793889in}{2.255883in}}%
\pgfpathlineto{\pgfqpoint{0.793889in}{2.656768in}}%
\pgfusepath{stroke}%
\end{pgfscope}%
\begin{pgfscope}%
\definecolor{textcolor}{rgb}{0.150000,0.150000,0.150000}%
\pgfsetstrokecolor{textcolor}%
\pgfsetfillcolor{textcolor}%
\pgftext[x=0.793889in,y=2.158661in,,top]{\color{textcolor}\rmfamily\fontsize{10.000000}{12.000000}\selectfont 2013}%
\end{pgfscope}%
\begin{pgfscope}%
\pgfpathrectangle{\pgfqpoint{0.285588in}{2.255883in}}{\pgfqpoint{2.583333in}{0.400885in}}%
\pgfusepath{clip}%
\pgfsetroundcap%
\pgfsetroundjoin%
\pgfsetlinewidth{0.803000pt}%
\definecolor{currentstroke}{rgb}{1.000000,1.000000,1.000000}%
\pgfsetstrokecolor{currentstroke}%
\pgfsetdash{}{0pt}%
\pgfpathmoveto{\pgfqpoint{1.185840in}{2.255883in}}%
\pgfpathlineto{\pgfqpoint{1.185840in}{2.656768in}}%
\pgfusepath{stroke}%
\end{pgfscope}%
\begin{pgfscope}%
\definecolor{textcolor}{rgb}{0.150000,0.150000,0.150000}%
\pgfsetstrokecolor{textcolor}%
\pgfsetfillcolor{textcolor}%
\pgftext[x=1.185840in,y=2.158661in,,top]{\color{textcolor}\rmfamily\fontsize{10.000000}{12.000000}\selectfont 2014}%
\end{pgfscope}%
\begin{pgfscope}%
\pgfpathrectangle{\pgfqpoint{0.285588in}{2.255883in}}{\pgfqpoint{2.583333in}{0.400885in}}%
\pgfusepath{clip}%
\pgfsetroundcap%
\pgfsetroundjoin%
\pgfsetlinewidth{0.803000pt}%
\definecolor{currentstroke}{rgb}{1.000000,1.000000,1.000000}%
\pgfsetstrokecolor{currentstroke}%
\pgfsetdash{}{0pt}%
\pgfpathmoveto{\pgfqpoint{1.577791in}{2.255883in}}%
\pgfpathlineto{\pgfqpoint{1.577791in}{2.656768in}}%
\pgfusepath{stroke}%
\end{pgfscope}%
\begin{pgfscope}%
\definecolor{textcolor}{rgb}{0.150000,0.150000,0.150000}%
\pgfsetstrokecolor{textcolor}%
\pgfsetfillcolor{textcolor}%
\pgftext[x=1.577791in,y=2.158661in,,top]{\color{textcolor}\rmfamily\fontsize{10.000000}{12.000000}\selectfont 2015}%
\end{pgfscope}%
\begin{pgfscope}%
\pgfpathrectangle{\pgfqpoint{0.285588in}{2.255883in}}{\pgfqpoint{2.583333in}{0.400885in}}%
\pgfusepath{clip}%
\pgfsetroundcap%
\pgfsetroundjoin%
\pgfsetlinewidth{0.803000pt}%
\definecolor{currentstroke}{rgb}{1.000000,1.000000,1.000000}%
\pgfsetstrokecolor{currentstroke}%
\pgfsetdash{}{0pt}%
\pgfpathmoveto{\pgfqpoint{1.969742in}{2.255883in}}%
\pgfpathlineto{\pgfqpoint{1.969742in}{2.656768in}}%
\pgfusepath{stroke}%
\end{pgfscope}%
\begin{pgfscope}%
\definecolor{textcolor}{rgb}{0.150000,0.150000,0.150000}%
\pgfsetstrokecolor{textcolor}%
\pgfsetfillcolor{textcolor}%
\pgftext[x=1.969742in,y=2.158661in,,top]{\color{textcolor}\rmfamily\fontsize{10.000000}{12.000000}\selectfont 2016}%
\end{pgfscope}%
\begin{pgfscope}%
\pgfpathrectangle{\pgfqpoint{0.285588in}{2.255883in}}{\pgfqpoint{2.583333in}{0.400885in}}%
\pgfusepath{clip}%
\pgfsetroundcap%
\pgfsetroundjoin%
\pgfsetlinewidth{0.803000pt}%
\definecolor{currentstroke}{rgb}{1.000000,1.000000,1.000000}%
\pgfsetstrokecolor{currentstroke}%
\pgfsetdash{}{0pt}%
\pgfpathmoveto{\pgfqpoint{2.362767in}{2.255883in}}%
\pgfpathlineto{\pgfqpoint{2.362767in}{2.656768in}}%
\pgfusepath{stroke}%
\end{pgfscope}%
\begin{pgfscope}%
\definecolor{textcolor}{rgb}{0.150000,0.150000,0.150000}%
\pgfsetstrokecolor{textcolor}%
\pgfsetfillcolor{textcolor}%
\pgftext[x=2.362767in,y=2.158661in,,top]{\color{textcolor}\rmfamily\fontsize{10.000000}{12.000000}\selectfont 2017}%
\end{pgfscope}%
\begin{pgfscope}%
\pgfpathrectangle{\pgfqpoint{0.285588in}{2.255883in}}{\pgfqpoint{2.583333in}{0.400885in}}%
\pgfusepath{clip}%
\pgfsetroundcap%
\pgfsetroundjoin%
\pgfsetlinewidth{0.803000pt}%
\definecolor{currentstroke}{rgb}{1.000000,1.000000,1.000000}%
\pgfsetstrokecolor{currentstroke}%
\pgfsetdash{}{0pt}%
\pgfpathmoveto{\pgfqpoint{2.754718in}{2.255883in}}%
\pgfpathlineto{\pgfqpoint{2.754718in}{2.656768in}}%
\pgfusepath{stroke}%
\end{pgfscope}%
\begin{pgfscope}%
\definecolor{textcolor}{rgb}{0.150000,0.150000,0.150000}%
\pgfsetstrokecolor{textcolor}%
\pgfsetfillcolor{textcolor}%
\pgftext[x=2.754718in,y=2.158661in,,top]{\color{textcolor}\rmfamily\fontsize{10.000000}{12.000000}\selectfont 2018}%
\end{pgfscope}%
\begin{pgfscope}%
\pgfpathrectangle{\pgfqpoint{0.285588in}{2.255883in}}{\pgfqpoint{2.583333in}{0.400885in}}%
\pgfusepath{clip}%
\pgfsetroundcap%
\pgfsetroundjoin%
\pgfsetlinewidth{0.803000pt}%
\definecolor{currentstroke}{rgb}{1.000000,1.000000,1.000000}%
\pgfsetstrokecolor{currentstroke}%
\pgfsetdash{}{0pt}%
\pgfpathmoveto{\pgfqpoint{0.285588in}{2.385050in}}%
\pgfpathlineto{\pgfqpoint{2.868921in}{2.385050in}}%
\pgfusepath{stroke}%
\end{pgfscope}%
\begin{pgfscope}%
\definecolor{textcolor}{rgb}{0.150000,0.150000,0.150000}%
\pgfsetstrokecolor{textcolor}%
\pgfsetfillcolor{textcolor}%
\pgftext[x=0.100000in,y=2.332289in,left,base]{\color{textcolor}\rmfamily\fontsize{10.000000}{12.000000}\selectfont 1}%
\end{pgfscope}%
\begin{pgfscope}%
\pgfpathrectangle{\pgfqpoint{0.285588in}{2.255883in}}{\pgfqpoint{2.583333in}{0.400885in}}%
\pgfusepath{clip}%
\pgfsetroundcap%
\pgfsetroundjoin%
\pgfsetlinewidth{0.803000pt}%
\definecolor{currentstroke}{rgb}{1.000000,1.000000,1.000000}%
\pgfsetstrokecolor{currentstroke}%
\pgfsetdash{}{0pt}%
\pgfpathmoveto{\pgfqpoint{0.285588in}{2.544766in}}%
\pgfpathlineto{\pgfqpoint{2.868921in}{2.544766in}}%
\pgfusepath{stroke}%
\end{pgfscope}%
\begin{pgfscope}%
\definecolor{textcolor}{rgb}{0.150000,0.150000,0.150000}%
\pgfsetstrokecolor{textcolor}%
\pgfsetfillcolor{textcolor}%
\pgftext[x=0.100000in,y=2.492005in,left,base]{\color{textcolor}\rmfamily\fontsize{10.000000}{12.000000}\selectfont 2}%
\end{pgfscope}%
\begin{pgfscope}%
\pgfpathrectangle{\pgfqpoint{0.285588in}{2.255883in}}{\pgfqpoint{2.583333in}{0.400885in}}%
\pgfusepath{clip}%
\pgfsetroundcap%
\pgfsetroundjoin%
\pgfsetlinewidth{1.505625pt}%
\definecolor{currentstroke}{rgb}{0.580392,0.403922,0.741176}%
\pgfsetstrokecolor{currentstroke}%
\pgfsetstrokeopacity{0.300000}%
\pgfsetdash{}{0pt}%
\pgfpathmoveto{\pgfqpoint{0.403012in}{2.385050in}}%
\pgfpathlineto{\pgfqpoint{0.404086in}{2.384080in}}%
\pgfpathlineto{\pgfqpoint{0.405159in}{2.383898in}}%
\pgfpathlineto{\pgfqpoint{0.406233in}{2.382503in}}%
\pgfpathlineto{\pgfqpoint{0.409455in}{2.382745in}}%
\pgfpathlineto{\pgfqpoint{0.410529in}{2.383413in}}%
\pgfpathlineto{\pgfqpoint{0.412676in}{2.383473in}}%
\pgfpathlineto{\pgfqpoint{0.413750in}{2.383534in}}%
\pgfpathlineto{\pgfqpoint{0.420193in}{2.383382in}}%
\pgfpathlineto{\pgfqpoint{0.421267in}{2.383564in}}%
\pgfpathlineto{\pgfqpoint{0.425562in}{2.382927in}}%
\pgfpathlineto{\pgfqpoint{0.426636in}{2.383443in}}%
\pgfpathlineto{\pgfqpoint{0.427710in}{2.384626in}}%
\pgfpathlineto{\pgfqpoint{0.428784in}{2.384262in}}%
\pgfpathlineto{\pgfqpoint{0.434153in}{2.384595in}}%
\pgfpathlineto{\pgfqpoint{0.440596in}{2.383534in}}%
\pgfpathlineto{\pgfqpoint{0.441670in}{2.383504in}}%
\pgfpathlineto{\pgfqpoint{0.443818in}{2.381957in}}%
\pgfpathlineto{\pgfqpoint{0.462073in}{2.382958in}}%
\pgfpathlineto{\pgfqpoint{0.463147in}{2.384747in}}%
\pgfpathlineto{\pgfqpoint{0.466368in}{2.383746in}}%
\pgfpathlineto{\pgfqpoint{0.469590in}{2.384080in}}%
\pgfpathlineto{\pgfqpoint{0.470664in}{2.382715in}}%
\pgfpathlineto{\pgfqpoint{0.471737in}{2.382594in}}%
\pgfpathlineto{\pgfqpoint{0.472811in}{2.383928in}}%
\pgfpathlineto{\pgfqpoint{0.473885in}{2.383686in}}%
\pgfpathlineto{\pgfqpoint{0.478180in}{2.385111in}}%
\pgfpathlineto{\pgfqpoint{0.480328in}{2.384474in}}%
\pgfpathlineto{\pgfqpoint{0.484623in}{2.384808in}}%
\pgfpathlineto{\pgfqpoint{0.487845in}{2.382988in}}%
\pgfpathlineto{\pgfqpoint{0.488919in}{2.383200in}}%
\pgfpathlineto{\pgfqpoint{0.496436in}{2.386658in}}%
\pgfpathlineto{\pgfqpoint{0.499657in}{2.387264in}}%
\pgfpathlineto{\pgfqpoint{0.502879in}{2.385141in}}%
\pgfpathlineto{\pgfqpoint{0.507174in}{2.384110in}}%
\pgfpathlineto{\pgfqpoint{0.508248in}{2.382351in}}%
\pgfpathlineto{\pgfqpoint{0.510396in}{2.382230in}}%
\pgfpathlineto{\pgfqpoint{0.511469in}{2.380744in}}%
\pgfpathlineto{\pgfqpoint{0.515765in}{2.382412in}}%
\pgfpathlineto{\pgfqpoint{0.516839in}{2.380046in}}%
\pgfpathlineto{\pgfqpoint{0.517912in}{2.379500in}}%
\pgfpathlineto{\pgfqpoint{0.518986in}{2.381138in}}%
\pgfpathlineto{\pgfqpoint{0.522208in}{2.380319in}}%
\pgfpathlineto{\pgfqpoint{0.526503in}{2.383928in}}%
\pgfpathlineto{\pgfqpoint{0.532946in}{2.385141in}}%
\pgfpathlineto{\pgfqpoint{0.534020in}{2.383686in}}%
\pgfpathlineto{\pgfqpoint{0.538315in}{2.384262in}}%
\pgfpathlineto{\pgfqpoint{0.539389in}{2.382533in}}%
\pgfpathlineto{\pgfqpoint{0.540463in}{2.383261in}}%
\pgfpathlineto{\pgfqpoint{0.541537in}{2.382685in}}%
\pgfpathlineto{\pgfqpoint{0.549054in}{2.380258in}}%
\pgfpathlineto{\pgfqpoint{0.553349in}{2.380683in}}%
\pgfpathlineto{\pgfqpoint{0.554423in}{2.380076in}}%
\pgfpathlineto{\pgfqpoint{0.555497in}{2.381168in}}%
\pgfpathlineto{\pgfqpoint{0.556571in}{2.379712in}}%
\pgfpathlineto{\pgfqpoint{0.560866in}{2.379894in}}%
\pgfpathlineto{\pgfqpoint{0.561940in}{2.378954in}}%
\pgfpathlineto{\pgfqpoint{0.563014in}{2.379500in}}%
\pgfpathlineto{\pgfqpoint{0.564088in}{2.377893in}}%
\pgfpathlineto{\pgfqpoint{0.567309in}{2.379288in}}%
\pgfpathlineto{\pgfqpoint{0.568383in}{2.378954in}}%
\pgfpathlineto{\pgfqpoint{0.569457in}{2.380410in}}%
\pgfpathlineto{\pgfqpoint{0.571604in}{2.380865in}}%
\pgfpathlineto{\pgfqpoint{0.574826in}{2.378742in}}%
\pgfpathlineto{\pgfqpoint{0.579121in}{2.388356in}}%
\pgfpathlineto{\pgfqpoint{0.582343in}{2.389054in}}%
\pgfpathlineto{\pgfqpoint{0.584490in}{2.390783in}}%
\pgfpathlineto{\pgfqpoint{0.585564in}{2.389297in}}%
\pgfpathlineto{\pgfqpoint{0.586638in}{2.389873in}}%
\pgfpathlineto{\pgfqpoint{0.590933in}{2.389357in}}%
\pgfpathlineto{\pgfqpoint{0.594155in}{2.392178in}}%
\pgfpathlineto{\pgfqpoint{0.598450in}{2.393361in}}%
\pgfpathlineto{\pgfqpoint{0.601672in}{2.392390in}}%
\pgfpathlineto{\pgfqpoint{0.608115in}{2.392542in}}%
\pgfpathlineto{\pgfqpoint{0.609189in}{2.394756in}}%
\pgfpathlineto{\pgfqpoint{0.612410in}{2.394362in}}%
\pgfpathlineto{\pgfqpoint{0.614558in}{2.396636in}}%
\pgfpathlineto{\pgfqpoint{0.615632in}{2.397031in}}%
\pgfpathlineto{\pgfqpoint{0.616706in}{2.394817in}}%
\pgfpathlineto{\pgfqpoint{0.619927in}{2.393543in}}%
\pgfpathlineto{\pgfqpoint{0.621001in}{2.391662in}}%
\pgfpathlineto{\pgfqpoint{0.622075in}{2.392117in}}%
\pgfpathlineto{\pgfqpoint{0.624222in}{2.397031in}}%
\pgfpathlineto{\pgfqpoint{0.629592in}{2.396667in}}%
\pgfpathlineto{\pgfqpoint{0.630666in}{2.394362in}}%
\pgfpathlineto{\pgfqpoint{0.631739in}{2.396030in}}%
\pgfpathlineto{\pgfqpoint{0.634961in}{2.395332in}}%
\pgfpathlineto{\pgfqpoint{0.636035in}{2.393967in}}%
\pgfpathlineto{\pgfqpoint{0.643552in}{2.394847in}}%
\pgfpathlineto{\pgfqpoint{0.646773in}{2.392784in}}%
\pgfpathlineto{\pgfqpoint{0.653216in}{2.392633in}}%
\pgfpathlineto{\pgfqpoint{0.654290in}{2.393785in}}%
\pgfpathlineto{\pgfqpoint{0.667176in}{2.392966in}}%
\pgfpathlineto{\pgfqpoint{0.668250in}{2.394392in}}%
\pgfpathlineto{\pgfqpoint{0.674693in}{2.395150in}}%
\pgfpathlineto{\pgfqpoint{0.675767in}{2.397243in}}%
\pgfpathlineto{\pgfqpoint{0.676841in}{2.395969in}}%
\pgfpathlineto{\pgfqpoint{0.680062in}{2.395423in}}%
\pgfpathlineto{\pgfqpoint{0.683284in}{2.397031in}}%
\pgfpathlineto{\pgfqpoint{0.684357in}{2.397425in}}%
\pgfpathlineto{\pgfqpoint{0.687579in}{2.397273in}}%
\pgfpathlineto{\pgfqpoint{0.688653in}{2.398092in}}%
\pgfpathlineto{\pgfqpoint{0.689727in}{2.397273in}}%
\pgfpathlineto{\pgfqpoint{0.698317in}{2.397940in}}%
\pgfpathlineto{\pgfqpoint{0.699391in}{2.398911in}}%
\pgfpathlineto{\pgfqpoint{0.702613in}{2.398365in}}%
\pgfpathlineto{\pgfqpoint{0.703687in}{2.395817in}}%
\pgfpathlineto{\pgfqpoint{0.706908in}{2.394726in}}%
\pgfpathlineto{\pgfqpoint{0.710130in}{2.396272in}}%
\pgfpathlineto{\pgfqpoint{0.712277in}{2.402278in}}%
\pgfpathlineto{\pgfqpoint{0.713351in}{2.406069in}}%
\pgfpathlineto{\pgfqpoint{0.714425in}{2.404401in}}%
\pgfpathlineto{\pgfqpoint{0.717646in}{2.404219in}}%
\pgfpathlineto{\pgfqpoint{0.718720in}{2.402005in}}%
\pgfpathlineto{\pgfqpoint{0.719794in}{2.401610in}}%
\pgfpathlineto{\pgfqpoint{0.720868in}{2.402611in}}%
\pgfpathlineto{\pgfqpoint{0.721942in}{2.402005in}}%
\pgfpathlineto{\pgfqpoint{0.727311in}{2.401823in}}%
\pgfpathlineto{\pgfqpoint{0.728385in}{2.403521in}}%
\pgfpathlineto{\pgfqpoint{0.729459in}{2.402005in}}%
\pgfpathlineto{\pgfqpoint{0.732680in}{2.401732in}}%
\pgfpathlineto{\pgfqpoint{0.733754in}{2.402278in}}%
\pgfpathlineto{\pgfqpoint{0.735902in}{2.398911in}}%
\pgfpathlineto{\pgfqpoint{0.736976in}{2.399457in}}%
\pgfpathlineto{\pgfqpoint{0.741271in}{2.398547in}}%
\pgfpathlineto{\pgfqpoint{0.743419in}{2.397455in}}%
\pgfpathlineto{\pgfqpoint{0.744492in}{2.397758in}}%
\pgfpathlineto{\pgfqpoint{0.747714in}{2.397910in}}%
\pgfpathlineto{\pgfqpoint{0.748788in}{2.398941in}}%
\pgfpathlineto{\pgfqpoint{0.749862in}{2.398759in}}%
\pgfpathlineto{\pgfqpoint{0.752009in}{2.400215in}}%
\pgfpathlineto{\pgfqpoint{0.756305in}{2.398335in}}%
\pgfpathlineto{\pgfqpoint{0.757378in}{2.399518in}}%
\pgfpathlineto{\pgfqpoint{0.758452in}{2.399366in}}%
\pgfpathlineto{\pgfqpoint{0.759526in}{2.400640in}}%
\pgfpathlineto{\pgfqpoint{0.762748in}{2.400428in}}%
\pgfpathlineto{\pgfqpoint{0.764895in}{2.401246in}}%
\pgfpathlineto{\pgfqpoint{0.765969in}{2.401428in}}%
\pgfpathlineto{\pgfqpoint{0.767043in}{2.402460in}}%
\pgfpathlineto{\pgfqpoint{0.770265in}{2.402824in}}%
\pgfpathlineto{\pgfqpoint{0.771338in}{2.404067in}}%
\pgfpathlineto{\pgfqpoint{0.774560in}{2.403066in}}%
\pgfpathlineto{\pgfqpoint{0.778855in}{2.403703in}}%
\pgfpathlineto{\pgfqpoint{0.779929in}{2.402915in}}%
\pgfpathlineto{\pgfqpoint{0.781003in}{2.403188in}}%
\pgfpathlineto{\pgfqpoint{0.782077in}{2.402005in}}%
\pgfpathlineto{\pgfqpoint{0.785298in}{2.401368in}}%
\pgfpathlineto{\pgfqpoint{0.788520in}{2.401550in}}%
\pgfpathlineto{\pgfqpoint{0.789594in}{2.400003in}}%
\pgfpathlineto{\pgfqpoint{0.792815in}{2.401580in}}%
\pgfpathlineto{\pgfqpoint{0.794963in}{2.403430in}}%
\pgfpathlineto{\pgfqpoint{0.796037in}{2.403188in}}%
\pgfpathlineto{\pgfqpoint{0.797110in}{2.405220in}}%
\pgfpathlineto{\pgfqpoint{0.801406in}{2.404856in}}%
\pgfpathlineto{\pgfqpoint{0.804627in}{2.407221in}}%
\pgfpathlineto{\pgfqpoint{0.807849in}{2.407767in}}%
\pgfpathlineto{\pgfqpoint{0.808923in}{2.407282in}}%
\pgfpathlineto{\pgfqpoint{0.812144in}{2.409435in}}%
\pgfpathlineto{\pgfqpoint{0.816440in}{2.408071in}}%
\pgfpathlineto{\pgfqpoint{0.818587in}{2.409102in}}%
\pgfpathlineto{\pgfqpoint{0.819661in}{2.411164in}}%
\pgfpathlineto{\pgfqpoint{0.822883in}{2.410406in}}%
\pgfpathlineto{\pgfqpoint{0.823956in}{2.412408in}}%
\pgfpathlineto{\pgfqpoint{0.826104in}{2.411164in}}%
\pgfpathlineto{\pgfqpoint{0.827178in}{2.411831in}}%
\pgfpathlineto{\pgfqpoint{0.830399in}{2.411649in}}%
\pgfpathlineto{\pgfqpoint{0.832547in}{2.414864in}}%
\pgfpathlineto{\pgfqpoint{0.833621in}{2.414046in}}%
\pgfpathlineto{\pgfqpoint{0.834695in}{2.415107in}}%
\pgfpathlineto{\pgfqpoint{0.837916in}{2.414925in}}%
\pgfpathlineto{\pgfqpoint{0.838990in}{2.415896in}}%
\pgfpathlineto{\pgfqpoint{0.840064in}{2.415532in}}%
\pgfpathlineto{\pgfqpoint{0.842212in}{2.416806in}}%
\pgfpathlineto{\pgfqpoint{0.846507in}{2.418807in}}%
\pgfpathlineto{\pgfqpoint{0.847581in}{2.418049in}}%
\pgfpathlineto{\pgfqpoint{0.849729in}{2.418565in}}%
\pgfpathlineto{\pgfqpoint{0.852950in}{2.416836in}}%
\pgfpathlineto{\pgfqpoint{0.854024in}{2.417291in}}%
\pgfpathlineto{\pgfqpoint{0.855098in}{2.418747in}}%
\pgfpathlineto{\pgfqpoint{0.856172in}{2.418201in}}%
\pgfpathlineto{\pgfqpoint{0.857245in}{2.419717in}}%
\pgfpathlineto{\pgfqpoint{0.860467in}{2.420961in}}%
\pgfpathlineto{\pgfqpoint{0.861541in}{2.422144in}}%
\pgfpathlineto{\pgfqpoint{0.862615in}{2.421446in}}%
\pgfpathlineto{\pgfqpoint{0.864762in}{2.423478in}}%
\pgfpathlineto{\pgfqpoint{0.870131in}{2.424388in}}%
\pgfpathlineto{\pgfqpoint{0.871205in}{2.425783in}}%
\pgfpathlineto{\pgfqpoint{0.872279in}{2.426026in}}%
\pgfpathlineto{\pgfqpoint{0.876575in}{2.425177in}}%
\pgfpathlineto{\pgfqpoint{0.877648in}{2.426663in}}%
\pgfpathlineto{\pgfqpoint{0.878722in}{2.425571in}}%
\pgfpathlineto{\pgfqpoint{0.879796in}{2.427421in}}%
\pgfpathlineto{\pgfqpoint{0.883018in}{2.427269in}}%
\pgfpathlineto{\pgfqpoint{0.884091in}{2.430211in}}%
\pgfpathlineto{\pgfqpoint{0.886239in}{2.431940in}}%
\pgfpathlineto{\pgfqpoint{0.890534in}{2.432971in}}%
\pgfpathlineto{\pgfqpoint{0.891608in}{2.434882in}}%
\pgfpathlineto{\pgfqpoint{0.892682in}{2.433305in}}%
\pgfpathlineto{\pgfqpoint{0.893756in}{2.434184in}}%
\pgfpathlineto{\pgfqpoint{0.898051in}{2.430878in}}%
\pgfpathlineto{\pgfqpoint{0.902347in}{2.435003in}}%
\pgfpathlineto{\pgfqpoint{0.905568in}{2.432395in}}%
\pgfpathlineto{\pgfqpoint{0.906642in}{2.436793in}}%
\pgfpathlineto{\pgfqpoint{0.907716in}{2.437945in}}%
\pgfpathlineto{\pgfqpoint{0.908790in}{2.436126in}}%
\pgfpathlineto{\pgfqpoint{0.909864in}{2.439431in}}%
\pgfpathlineto{\pgfqpoint{0.913085in}{2.440311in}}%
\pgfpathlineto{\pgfqpoint{0.914159in}{2.441888in}}%
\pgfpathlineto{\pgfqpoint{0.915233in}{2.439189in}}%
\pgfpathlineto{\pgfqpoint{0.916307in}{2.441282in}}%
\pgfpathlineto{\pgfqpoint{0.917380in}{2.441039in}}%
\pgfpathlineto{\pgfqpoint{0.920602in}{2.442191in}}%
\pgfpathlineto{\pgfqpoint{0.921676in}{2.441312in}}%
\pgfpathlineto{\pgfqpoint{0.922750in}{2.438795in}}%
\pgfpathlineto{\pgfqpoint{0.924897in}{2.442646in}}%
\pgfpathlineto{\pgfqpoint{0.928119in}{2.439917in}}%
\pgfpathlineto{\pgfqpoint{0.929193in}{2.442070in}}%
\pgfpathlineto{\pgfqpoint{0.930266in}{2.441888in}}%
\pgfpathlineto{\pgfqpoint{0.931340in}{2.441130in}}%
\pgfpathlineto{\pgfqpoint{0.932414in}{2.442677in}}%
\pgfpathlineto{\pgfqpoint{0.935636in}{2.442889in}}%
\pgfpathlineto{\pgfqpoint{0.937783in}{2.447438in}}%
\pgfpathlineto{\pgfqpoint{0.938857in}{2.446953in}}%
\pgfpathlineto{\pgfqpoint{0.939931in}{2.448561in}}%
\pgfpathlineto{\pgfqpoint{0.943153in}{2.448379in}}%
\pgfpathlineto{\pgfqpoint{0.944226in}{2.449835in}}%
\pgfpathlineto{\pgfqpoint{0.945300in}{2.449501in}}%
\pgfpathlineto{\pgfqpoint{0.947448in}{2.447014in}}%
\pgfpathlineto{\pgfqpoint{0.951743in}{2.449016in}}%
\pgfpathlineto{\pgfqpoint{0.952817in}{2.444011in}}%
\pgfpathlineto{\pgfqpoint{0.953891in}{2.444891in}}%
\pgfpathlineto{\pgfqpoint{0.954965in}{2.440250in}}%
\pgfpathlineto{\pgfqpoint{0.958186in}{2.441615in}}%
\pgfpathlineto{\pgfqpoint{0.960334in}{2.438977in}}%
\pgfpathlineto{\pgfqpoint{0.962482in}{2.442131in}}%
\pgfpathlineto{\pgfqpoint{0.965703in}{2.442707in}}%
\pgfpathlineto{\pgfqpoint{0.966777in}{2.441555in}}%
\pgfpathlineto{\pgfqpoint{0.967851in}{2.439128in}}%
\pgfpathlineto{\pgfqpoint{0.968925in}{2.442131in}}%
\pgfpathlineto{\pgfqpoint{0.969998in}{2.442131in}}%
\pgfpathlineto{\pgfqpoint{0.973220in}{2.443951in}}%
\pgfpathlineto{\pgfqpoint{0.974294in}{2.445831in}}%
\pgfpathlineto{\pgfqpoint{0.976442in}{2.436277in}}%
\pgfpathlineto{\pgfqpoint{0.981811in}{2.443253in}}%
\pgfpathlineto{\pgfqpoint{0.982885in}{2.447438in}}%
\pgfpathlineto{\pgfqpoint{0.983958in}{2.446711in}}%
\pgfpathlineto{\pgfqpoint{0.985032in}{2.444557in}}%
\pgfpathlineto{\pgfqpoint{0.988254in}{2.446529in}}%
\pgfpathlineto{\pgfqpoint{0.989328in}{2.446377in}}%
\pgfpathlineto{\pgfqpoint{0.990401in}{2.446893in}}%
\pgfpathlineto{\pgfqpoint{0.992549in}{2.449683in}}%
\pgfpathlineto{\pgfqpoint{0.997918in}{2.453171in}}%
\pgfpathlineto{\pgfqpoint{1.000066in}{2.455082in}}%
\pgfpathlineto{\pgfqpoint{1.004361in}{2.456143in}}%
\pgfpathlineto{\pgfqpoint{1.005435in}{2.455446in}}%
\pgfpathlineto{\pgfqpoint{1.006509in}{2.455567in}}%
\pgfpathlineto{\pgfqpoint{1.007583in}{2.460814in}}%
\pgfpathlineto{\pgfqpoint{1.012952in}{2.461147in}}%
\pgfpathlineto{\pgfqpoint{1.018321in}{2.463331in}}%
\pgfpathlineto{\pgfqpoint{1.019395in}{2.463210in}}%
\pgfpathlineto{\pgfqpoint{1.022617in}{2.466334in}}%
\pgfpathlineto{\pgfqpoint{1.026912in}{2.464848in}}%
\pgfpathlineto{\pgfqpoint{1.029060in}{2.463635in}}%
\pgfpathlineto{\pgfqpoint{1.030133in}{2.461147in}}%
\pgfpathlineto{\pgfqpoint{1.033355in}{2.460238in}}%
\pgfpathlineto{\pgfqpoint{1.034429in}{2.462816in}}%
\pgfpathlineto{\pgfqpoint{1.036576in}{2.453959in}}%
\pgfpathlineto{\pgfqpoint{1.037650in}{2.453504in}}%
\pgfpathlineto{\pgfqpoint{1.040872in}{2.456264in}}%
\pgfpathlineto{\pgfqpoint{1.044093in}{2.450684in}}%
\pgfpathlineto{\pgfqpoint{1.045167in}{2.452777in}}%
\pgfpathlineto{\pgfqpoint{1.048389in}{2.450502in}}%
\pgfpathlineto{\pgfqpoint{1.049463in}{2.447014in}}%
\pgfpathlineto{\pgfqpoint{1.050536in}{2.447924in}}%
\pgfpathlineto{\pgfqpoint{1.052684in}{2.447620in}}%
\pgfpathlineto{\pgfqpoint{1.056979in}{2.447651in}}%
\pgfpathlineto{\pgfqpoint{1.058053in}{2.448864in}}%
\pgfpathlineto{\pgfqpoint{1.063422in}{2.450562in}}%
\pgfpathlineto{\pgfqpoint{1.065570in}{2.454869in}}%
\pgfpathlineto{\pgfqpoint{1.067718in}{2.453171in}}%
\pgfpathlineto{\pgfqpoint{1.072013in}{2.454445in}}%
\pgfpathlineto{\pgfqpoint{1.073087in}{2.456628in}}%
\pgfpathlineto{\pgfqpoint{1.074161in}{2.457023in}}%
\pgfpathlineto{\pgfqpoint{1.075235in}{2.456022in}}%
\pgfpathlineto{\pgfqpoint{1.078456in}{2.454505in}}%
\pgfpathlineto{\pgfqpoint{1.080604in}{2.449349in}}%
\pgfpathlineto{\pgfqpoint{1.081678in}{2.449319in}}%
\pgfpathlineto{\pgfqpoint{1.082752in}{2.448439in}}%
\pgfpathlineto{\pgfqpoint{1.085973in}{2.448348in}}%
\pgfpathlineto{\pgfqpoint{1.087047in}{2.450350in}}%
\pgfpathlineto{\pgfqpoint{1.088121in}{2.449895in}}%
\pgfpathlineto{\pgfqpoint{1.089195in}{2.448045in}}%
\pgfpathlineto{\pgfqpoint{1.090268in}{2.449926in}}%
\pgfpathlineto{\pgfqpoint{1.093490in}{2.448075in}}%
\pgfpathlineto{\pgfqpoint{1.094564in}{2.445558in}}%
\pgfpathlineto{\pgfqpoint{1.095638in}{2.446468in}}%
\pgfpathlineto{\pgfqpoint{1.097785in}{2.455446in}}%
\pgfpathlineto{\pgfqpoint{1.102081in}{2.456659in}}%
\pgfpathlineto{\pgfqpoint{1.104228in}{2.461906in}}%
\pgfpathlineto{\pgfqpoint{1.105302in}{2.461057in}}%
\pgfpathlineto{\pgfqpoint{1.108524in}{2.459934in}}%
\pgfpathlineto{\pgfqpoint{1.109597in}{2.462937in}}%
\pgfpathlineto{\pgfqpoint{1.110671in}{2.462270in}}%
\pgfpathlineto{\pgfqpoint{1.111745in}{2.462907in}}%
\pgfpathlineto{\pgfqpoint{1.112819in}{2.462239in}}%
\pgfpathlineto{\pgfqpoint{1.116041in}{2.462998in}}%
\pgfpathlineto{\pgfqpoint{1.117114in}{2.464939in}}%
\pgfpathlineto{\pgfqpoint{1.119262in}{2.463574in}}%
\pgfpathlineto{\pgfqpoint{1.120336in}{2.465515in}}%
\pgfpathlineto{\pgfqpoint{1.124631in}{2.464089in}}%
\pgfpathlineto{\pgfqpoint{1.125705in}{2.464666in}}%
\pgfpathlineto{\pgfqpoint{1.126779in}{2.463786in}}%
\pgfpathlineto{\pgfqpoint{1.127853in}{2.467274in}}%
\pgfpathlineto{\pgfqpoint{1.131074in}{2.467881in}}%
\pgfpathlineto{\pgfqpoint{1.132148in}{2.466000in}}%
\pgfpathlineto{\pgfqpoint{1.133222in}{2.465454in}}%
\pgfpathlineto{\pgfqpoint{1.135370in}{2.468154in}}%
\pgfpathlineto{\pgfqpoint{1.138591in}{2.467911in}}%
\pgfpathlineto{\pgfqpoint{1.140739in}{2.470095in}}%
\pgfpathlineto{\pgfqpoint{1.141813in}{2.470216in}}%
\pgfpathlineto{\pgfqpoint{1.142886in}{2.472066in}}%
\pgfpathlineto{\pgfqpoint{1.146108in}{2.473067in}}%
\pgfpathlineto{\pgfqpoint{1.147182in}{2.471581in}}%
\pgfpathlineto{\pgfqpoint{1.150403in}{2.470550in}}%
\pgfpathlineto{\pgfqpoint{1.154699in}{2.468760in}}%
\pgfpathlineto{\pgfqpoint{1.155773in}{2.467881in}}%
\pgfpathlineto{\pgfqpoint{1.156846in}{2.466152in}}%
\pgfpathlineto{\pgfqpoint{1.157920in}{2.469973in}}%
\pgfpathlineto{\pgfqpoint{1.161142in}{2.469973in}}%
\pgfpathlineto{\pgfqpoint{1.162216in}{2.469185in}}%
\pgfpathlineto{\pgfqpoint{1.163289in}{2.466516in}}%
\pgfpathlineto{\pgfqpoint{1.164363in}{2.461481in}}%
\pgfpathlineto{\pgfqpoint{1.165437in}{2.461966in}}%
\pgfpathlineto{\pgfqpoint{1.168659in}{2.462027in}}%
\pgfpathlineto{\pgfqpoint{1.169732in}{2.460177in}}%
\pgfpathlineto{\pgfqpoint{1.170806in}{2.465303in}}%
\pgfpathlineto{\pgfqpoint{1.171880in}{2.463604in}}%
\pgfpathlineto{\pgfqpoint{1.172954in}{2.463877in}}%
\pgfpathlineto{\pgfqpoint{1.177249in}{2.463817in}}%
\pgfpathlineto{\pgfqpoint{1.179397in}{2.465060in}}%
\pgfpathlineto{\pgfqpoint{1.180471in}{2.464544in}}%
\pgfpathlineto{\pgfqpoint{1.183692in}{2.464423in}}%
\pgfpathlineto{\pgfqpoint{1.184766in}{2.462573in}}%
\pgfpathlineto{\pgfqpoint{1.186914in}{2.461147in}}%
\pgfpathlineto{\pgfqpoint{1.187988in}{2.463271in}}%
\pgfpathlineto{\pgfqpoint{1.191209in}{2.464514in}}%
\pgfpathlineto{\pgfqpoint{1.192283in}{2.469579in}}%
\pgfpathlineto{\pgfqpoint{1.193357in}{2.469246in}}%
\pgfpathlineto{\pgfqpoint{1.194431in}{2.470732in}}%
\pgfpathlineto{\pgfqpoint{1.195505in}{2.470762in}}%
\pgfpathlineto{\pgfqpoint{1.198726in}{2.470125in}}%
\pgfpathlineto{\pgfqpoint{1.200874in}{2.470914in}}%
\pgfpathlineto{\pgfqpoint{1.201948in}{2.470489in}}%
\pgfpathlineto{\pgfqpoint{1.203021in}{2.471581in}}%
\pgfpathlineto{\pgfqpoint{1.207317in}{2.468912in}}%
\pgfpathlineto{\pgfqpoint{1.208391in}{2.469670in}}%
\pgfpathlineto{\pgfqpoint{1.210538in}{2.460056in}}%
\pgfpathlineto{\pgfqpoint{1.213760in}{2.458327in}}%
\pgfpathlineto{\pgfqpoint{1.214834in}{2.458721in}}%
\pgfpathlineto{\pgfqpoint{1.215908in}{2.455627in}}%
\pgfpathlineto{\pgfqpoint{1.216981in}{2.457174in}}%
\pgfpathlineto{\pgfqpoint{1.218055in}{2.454505in}}%
\pgfpathlineto{\pgfqpoint{1.221277in}{2.450138in}}%
\pgfpathlineto{\pgfqpoint{1.222351in}{2.449713in}}%
\pgfpathlineto{\pgfqpoint{1.223424in}{2.451412in}}%
\pgfpathlineto{\pgfqpoint{1.225572in}{2.458569in}}%
\pgfpathlineto{\pgfqpoint{1.228794in}{2.461238in}}%
\pgfpathlineto{\pgfqpoint{1.229867in}{2.466152in}}%
\pgfpathlineto{\pgfqpoint{1.230941in}{2.464726in}}%
\pgfpathlineto{\pgfqpoint{1.233089in}{2.465606in}}%
\pgfpathlineto{\pgfqpoint{1.237384in}{2.464089in}}%
\pgfpathlineto{\pgfqpoint{1.238458in}{2.462725in}}%
\pgfpathlineto{\pgfqpoint{1.239532in}{2.464635in}}%
\pgfpathlineto{\pgfqpoint{1.243827in}{2.463028in}}%
\pgfpathlineto{\pgfqpoint{1.245975in}{2.463028in}}%
\pgfpathlineto{\pgfqpoint{1.247049in}{2.463695in}}%
\pgfpathlineto{\pgfqpoint{1.248123in}{2.465667in}}%
\pgfpathlineto{\pgfqpoint{1.251344in}{2.464211in}}%
\pgfpathlineto{\pgfqpoint{1.252418in}{2.468851in}}%
\pgfpathlineto{\pgfqpoint{1.253492in}{2.466910in}}%
\pgfpathlineto{\pgfqpoint{1.255640in}{2.468821in}}%
\pgfpathlineto{\pgfqpoint{1.261009in}{2.469549in}}%
\pgfpathlineto{\pgfqpoint{1.262083in}{2.467972in}}%
\pgfpathlineto{\pgfqpoint{1.263156in}{2.467486in}}%
\pgfpathlineto{\pgfqpoint{1.266378in}{2.470398in}}%
\pgfpathlineto{\pgfqpoint{1.267452in}{2.470428in}}%
\pgfpathlineto{\pgfqpoint{1.268526in}{2.469518in}}%
\pgfpathlineto{\pgfqpoint{1.269599in}{2.470883in}}%
\pgfpathlineto{\pgfqpoint{1.270673in}{2.475615in}}%
\pgfpathlineto{\pgfqpoint{1.273895in}{2.473704in}}%
\pgfpathlineto{\pgfqpoint{1.274969in}{2.479406in}}%
\pgfpathlineto{\pgfqpoint{1.276042in}{2.478526in}}%
\pgfpathlineto{\pgfqpoint{1.281412in}{2.481620in}}%
\pgfpathlineto{\pgfqpoint{1.282485in}{2.480862in}}%
\pgfpathlineto{\pgfqpoint{1.284633in}{2.481711in}}%
\pgfpathlineto{\pgfqpoint{1.285707in}{2.482105in}}%
\pgfpathlineto{\pgfqpoint{1.288929in}{2.480740in}}%
\pgfpathlineto{\pgfqpoint{1.290002in}{2.481165in}}%
\pgfpathlineto{\pgfqpoint{1.291076in}{2.483500in}}%
\pgfpathlineto{\pgfqpoint{1.292150in}{2.477222in}}%
\pgfpathlineto{\pgfqpoint{1.293224in}{2.478071in}}%
\pgfpathlineto{\pgfqpoint{1.296445in}{2.478769in}}%
\pgfpathlineto{\pgfqpoint{1.297519in}{2.484137in}}%
\pgfpathlineto{\pgfqpoint{1.298593in}{2.482985in}}%
\pgfpathlineto{\pgfqpoint{1.306110in}{2.486806in}}%
\pgfpathlineto{\pgfqpoint{1.308258in}{2.485684in}}%
\pgfpathlineto{\pgfqpoint{1.311479in}{2.489718in}}%
\pgfpathlineto{\pgfqpoint{1.312553in}{2.488929in}}%
\pgfpathlineto{\pgfqpoint{1.313627in}{2.489597in}}%
\pgfpathlineto{\pgfqpoint{1.314701in}{2.487625in}}%
\pgfpathlineto{\pgfqpoint{1.315774in}{2.484441in}}%
\pgfpathlineto{\pgfqpoint{1.318996in}{2.486230in}}%
\pgfpathlineto{\pgfqpoint{1.320070in}{2.484956in}}%
\pgfpathlineto{\pgfqpoint{1.321144in}{2.488626in}}%
\pgfpathlineto{\pgfqpoint{1.322218in}{2.487534in}}%
\pgfpathlineto{\pgfqpoint{1.323291in}{2.488626in}}%
\pgfpathlineto{\pgfqpoint{1.326513in}{2.487595in}}%
\pgfpathlineto{\pgfqpoint{1.327587in}{2.488899in}}%
\pgfpathlineto{\pgfqpoint{1.330808in}{2.487747in}}%
\pgfpathlineto{\pgfqpoint{1.334030in}{2.487898in}}%
\pgfpathlineto{\pgfqpoint{1.335104in}{2.486897in}}%
\pgfpathlineto{\pgfqpoint{1.337251in}{2.490567in}}%
\pgfpathlineto{\pgfqpoint{1.338325in}{2.490628in}}%
\pgfpathlineto{\pgfqpoint{1.342620in}{2.490173in}}%
\pgfpathlineto{\pgfqpoint{1.343694in}{2.488838in}}%
\pgfpathlineto{\pgfqpoint{1.345842in}{2.491902in}}%
\pgfpathlineto{\pgfqpoint{1.351211in}{2.495086in}}%
\pgfpathlineto{\pgfqpoint{1.352285in}{2.496512in}}%
\pgfpathlineto{\pgfqpoint{1.356580in}{2.496512in}}%
\pgfpathlineto{\pgfqpoint{1.357654in}{2.498817in}}%
\pgfpathlineto{\pgfqpoint{1.359802in}{2.494692in}}%
\pgfpathlineto{\pgfqpoint{1.364097in}{2.494480in}}%
\pgfpathlineto{\pgfqpoint{1.365171in}{2.493115in}}%
\pgfpathlineto{\pgfqpoint{1.367319in}{2.498059in}}%
\pgfpathlineto{\pgfqpoint{1.368393in}{2.501911in}}%
\pgfpathlineto{\pgfqpoint{1.372688in}{2.500121in}}%
\pgfpathlineto{\pgfqpoint{1.373762in}{2.503184in}}%
\pgfpathlineto{\pgfqpoint{1.374836in}{2.502911in}}%
\pgfpathlineto{\pgfqpoint{1.375909in}{2.501152in}}%
\pgfpathlineto{\pgfqpoint{1.379131in}{2.500182in}}%
\pgfpathlineto{\pgfqpoint{1.380205in}{2.503488in}}%
\pgfpathlineto{\pgfqpoint{1.381279in}{2.503457in}}%
\pgfpathlineto{\pgfqpoint{1.382352in}{2.502305in}}%
\pgfpathlineto{\pgfqpoint{1.386648in}{2.505065in}}%
\pgfpathlineto{\pgfqpoint{1.387722in}{2.503093in}}%
\pgfpathlineto{\pgfqpoint{1.388796in}{2.503912in}}%
\pgfpathlineto{\pgfqpoint{1.389869in}{2.503306in}}%
\pgfpathlineto{\pgfqpoint{1.390943in}{2.501456in}}%
\pgfpathlineto{\pgfqpoint{1.394165in}{2.502184in}}%
\pgfpathlineto{\pgfqpoint{1.396312in}{2.493873in}}%
\pgfpathlineto{\pgfqpoint{1.397386in}{2.489020in}}%
\pgfpathlineto{\pgfqpoint{1.398460in}{2.492781in}}%
\pgfpathlineto{\pgfqpoint{1.401682in}{2.491386in}}%
\pgfpathlineto{\pgfqpoint{1.402755in}{2.494540in}}%
\pgfpathlineto{\pgfqpoint{1.403829in}{2.493813in}}%
\pgfpathlineto{\pgfqpoint{1.409198in}{2.493600in}}%
\pgfpathlineto{\pgfqpoint{1.410272in}{2.493206in}}%
\pgfpathlineto{\pgfqpoint{1.411346in}{2.494086in}}%
\pgfpathlineto{\pgfqpoint{1.412420in}{2.488293in}}%
\pgfpathlineto{\pgfqpoint{1.413494in}{2.487807in}}%
\pgfpathlineto{\pgfqpoint{1.416715in}{2.488475in}}%
\pgfpathlineto{\pgfqpoint{1.417789in}{2.487595in}}%
\pgfpathlineto{\pgfqpoint{1.418863in}{2.489930in}}%
\pgfpathlineto{\pgfqpoint{1.419937in}{2.487868in}}%
\pgfpathlineto{\pgfqpoint{1.421011in}{2.490901in}}%
\pgfpathlineto{\pgfqpoint{1.424232in}{2.491113in}}%
\pgfpathlineto{\pgfqpoint{1.425306in}{2.489718in}}%
\pgfpathlineto{\pgfqpoint{1.426380in}{2.492630in}}%
\pgfpathlineto{\pgfqpoint{1.427454in}{2.493358in}}%
\pgfpathlineto{\pgfqpoint{1.428528in}{2.491144in}}%
\pgfpathlineto{\pgfqpoint{1.432823in}{2.495845in}}%
\pgfpathlineto{\pgfqpoint{1.433897in}{2.496482in}}%
\pgfpathlineto{\pgfqpoint{1.434971in}{2.499060in}}%
\pgfpathlineto{\pgfqpoint{1.436044in}{2.498028in}}%
\pgfpathlineto{\pgfqpoint{1.441414in}{2.498362in}}%
\pgfpathlineto{\pgfqpoint{1.442487in}{2.497634in}}%
\pgfpathlineto{\pgfqpoint{1.443561in}{2.499696in}}%
\pgfpathlineto{\pgfqpoint{1.447857in}{2.498726in}}%
\pgfpathlineto{\pgfqpoint{1.448930in}{2.499787in}}%
\pgfpathlineto{\pgfqpoint{1.450004in}{2.500000in}}%
\pgfpathlineto{\pgfqpoint{1.451078in}{2.501516in}}%
\pgfpathlineto{\pgfqpoint{1.455373in}{2.499878in}}%
\pgfpathlineto{\pgfqpoint{1.456447in}{2.503033in}}%
\pgfpathlineto{\pgfqpoint{1.457521in}{2.501880in}}%
\pgfpathlineto{\pgfqpoint{1.461817in}{2.502305in}}%
\pgfpathlineto{\pgfqpoint{1.462890in}{2.505398in}}%
\pgfpathlineto{\pgfqpoint{1.463964in}{2.506217in}}%
\pgfpathlineto{\pgfqpoint{1.466112in}{2.510979in}}%
\pgfpathlineto{\pgfqpoint{1.469333in}{2.510676in}}%
\pgfpathlineto{\pgfqpoint{1.470407in}{2.509554in}}%
\pgfpathlineto{\pgfqpoint{1.471481in}{2.512678in}}%
\pgfpathlineto{\pgfqpoint{1.472555in}{2.508613in}}%
\pgfpathlineto{\pgfqpoint{1.473629in}{2.508613in}}%
\pgfpathlineto{\pgfqpoint{1.476850in}{2.507127in}}%
\pgfpathlineto{\pgfqpoint{1.477924in}{2.507249in}}%
\pgfpathlineto{\pgfqpoint{1.478998in}{2.501213in}}%
\pgfpathlineto{\pgfqpoint{1.480072in}{2.500030in}}%
\pgfpathlineto{\pgfqpoint{1.481146in}{2.503397in}}%
\pgfpathlineto{\pgfqpoint{1.484367in}{2.502699in}}%
\pgfpathlineto{\pgfqpoint{1.485441in}{2.496148in}}%
\pgfpathlineto{\pgfqpoint{1.486515in}{2.502820in}}%
\pgfpathlineto{\pgfqpoint{1.487589in}{2.495329in}}%
\pgfpathlineto{\pgfqpoint{1.491884in}{2.487504in}}%
\pgfpathlineto{\pgfqpoint{1.492958in}{2.481923in}}%
\pgfpathlineto{\pgfqpoint{1.494032in}{2.485108in}}%
\pgfpathlineto{\pgfqpoint{1.495106in}{2.481317in}}%
\pgfpathlineto{\pgfqpoint{1.496179in}{2.486382in}}%
\pgfpathlineto{\pgfqpoint{1.499401in}{2.487716in}}%
\pgfpathlineto{\pgfqpoint{1.502622in}{2.496785in}}%
\pgfpathlineto{\pgfqpoint{1.503696in}{2.498119in}}%
\pgfpathlineto{\pgfqpoint{1.506918in}{2.500606in}}%
\pgfpathlineto{\pgfqpoint{1.509065in}{2.504549in}}%
\pgfpathlineto{\pgfqpoint{1.511213in}{2.510403in}}%
\pgfpathlineto{\pgfqpoint{1.514435in}{2.509554in}}%
\pgfpathlineto{\pgfqpoint{1.515508in}{2.512647in}}%
\pgfpathlineto{\pgfqpoint{1.517656in}{2.513678in}}%
\pgfpathlineto{\pgfqpoint{1.518730in}{2.511525in}}%
\pgfpathlineto{\pgfqpoint{1.523025in}{2.513406in}}%
\pgfpathlineto{\pgfqpoint{1.524099in}{2.512981in}}%
\pgfpathlineto{\pgfqpoint{1.525173in}{2.513830in}}%
\pgfpathlineto{\pgfqpoint{1.526247in}{2.511404in}}%
\pgfpathlineto{\pgfqpoint{1.529468in}{2.511798in}}%
\pgfpathlineto{\pgfqpoint{1.530542in}{2.513193in}}%
\pgfpathlineto{\pgfqpoint{1.531616in}{2.513011in}}%
\pgfpathlineto{\pgfqpoint{1.532690in}{2.511464in}}%
\pgfpathlineto{\pgfqpoint{1.533764in}{2.512465in}}%
\pgfpathlineto{\pgfqpoint{1.538059in}{2.509402in}}%
\pgfpathlineto{\pgfqpoint{1.541281in}{2.513527in}}%
\pgfpathlineto{\pgfqpoint{1.544502in}{2.512920in}}%
\pgfpathlineto{\pgfqpoint{1.545576in}{2.514224in}}%
\pgfpathlineto{\pgfqpoint{1.546650in}{2.512101in}}%
\pgfpathlineto{\pgfqpoint{1.547724in}{2.511677in}}%
\pgfpathlineto{\pgfqpoint{1.548797in}{2.514224in}}%
\pgfpathlineto{\pgfqpoint{1.552019in}{2.514224in}}%
\pgfpathlineto{\pgfqpoint{1.553093in}{2.512981in}}%
\pgfpathlineto{\pgfqpoint{1.554167in}{2.508158in}}%
\pgfpathlineto{\pgfqpoint{1.555240in}{2.509432in}}%
\pgfpathlineto{\pgfqpoint{1.556314in}{2.503336in}}%
\pgfpathlineto{\pgfqpoint{1.559536in}{2.502093in}}%
\pgfpathlineto{\pgfqpoint{1.560610in}{2.498908in}}%
\pgfpathlineto{\pgfqpoint{1.561684in}{2.502396in}}%
\pgfpathlineto{\pgfqpoint{1.562757in}{2.509675in}}%
\pgfpathlineto{\pgfqpoint{1.563831in}{2.506339in}}%
\pgfpathlineto{\pgfqpoint{1.567053in}{2.509493in}}%
\pgfpathlineto{\pgfqpoint{1.568127in}{2.502942in}}%
\pgfpathlineto{\pgfqpoint{1.571348in}{2.505035in}}%
\pgfpathlineto{\pgfqpoint{1.575643in}{2.505823in}}%
\pgfpathlineto{\pgfqpoint{1.576717in}{2.503730in}}%
\pgfpathlineto{\pgfqpoint{1.578865in}{2.503579in}}%
\pgfpathlineto{\pgfqpoint{1.582086in}{2.501638in}}%
\pgfpathlineto{\pgfqpoint{1.583160in}{2.500273in}}%
\pgfpathlineto{\pgfqpoint{1.584234in}{2.506369in}}%
\pgfpathlineto{\pgfqpoint{1.585308in}{2.508553in}}%
\pgfpathlineto{\pgfqpoint{1.586382in}{2.504701in}}%
\pgfpathlineto{\pgfqpoint{1.589603in}{2.503761in}}%
\pgfpathlineto{\pgfqpoint{1.590677in}{2.504216in}}%
\pgfpathlineto{\pgfqpoint{1.591751in}{2.502214in}}%
\pgfpathlineto{\pgfqpoint{1.592825in}{2.498180in}}%
\pgfpathlineto{\pgfqpoint{1.593899in}{2.502305in}}%
\pgfpathlineto{\pgfqpoint{1.598194in}{2.494995in}}%
\pgfpathlineto{\pgfqpoint{1.599268in}{2.496603in}}%
\pgfpathlineto{\pgfqpoint{1.600342in}{2.501577in}}%
\pgfpathlineto{\pgfqpoint{1.601416in}{2.497422in}}%
\pgfpathlineto{\pgfqpoint{1.605711in}{2.497118in}}%
\pgfpathlineto{\pgfqpoint{1.606785in}{2.495481in}}%
\pgfpathlineto{\pgfqpoint{1.607859in}{2.497877in}}%
\pgfpathlineto{\pgfqpoint{1.608932in}{2.491932in}}%
\pgfpathlineto{\pgfqpoint{1.612154in}{2.493752in}}%
\pgfpathlineto{\pgfqpoint{1.613228in}{2.498089in}}%
\pgfpathlineto{\pgfqpoint{1.614302in}{2.495177in}}%
\pgfpathlineto{\pgfqpoint{1.615375in}{2.498089in}}%
\pgfpathlineto{\pgfqpoint{1.616449in}{2.494480in}}%
\pgfpathlineto{\pgfqpoint{1.619671in}{2.490962in}}%
\pgfpathlineto{\pgfqpoint{1.620745in}{2.492478in}}%
\pgfpathlineto{\pgfqpoint{1.621818in}{2.492569in}}%
\pgfpathlineto{\pgfqpoint{1.622892in}{2.487413in}}%
\pgfpathlineto{\pgfqpoint{1.623966in}{2.490537in}}%
\pgfpathlineto{\pgfqpoint{1.628261in}{2.492721in}}%
\pgfpathlineto{\pgfqpoint{1.629335in}{2.491447in}}%
\pgfpathlineto{\pgfqpoint{1.630409in}{2.493418in}}%
\pgfpathlineto{\pgfqpoint{1.631483in}{2.494116in}}%
\pgfpathlineto{\pgfqpoint{1.634705in}{2.493904in}}%
\pgfpathlineto{\pgfqpoint{1.636852in}{2.496664in}}%
\pgfpathlineto{\pgfqpoint{1.637926in}{2.500940in}}%
\pgfpathlineto{\pgfqpoint{1.639000in}{2.500151in}}%
\pgfpathlineto{\pgfqpoint{1.642221in}{2.502062in}}%
\pgfpathlineto{\pgfqpoint{1.644369in}{2.497846in}}%
\pgfpathlineto{\pgfqpoint{1.645443in}{2.500182in}}%
\pgfpathlineto{\pgfqpoint{1.646517in}{2.493722in}}%
\pgfpathlineto{\pgfqpoint{1.649738in}{2.495177in}}%
\pgfpathlineto{\pgfqpoint{1.651886in}{2.488929in}}%
\pgfpathlineto{\pgfqpoint{1.652960in}{2.492963in}}%
\pgfpathlineto{\pgfqpoint{1.654034in}{2.491295in}}%
\pgfpathlineto{\pgfqpoint{1.657255in}{2.496269in}}%
\pgfpathlineto{\pgfqpoint{1.658329in}{2.493115in}}%
\pgfpathlineto{\pgfqpoint{1.659403in}{2.497331in}}%
\pgfpathlineto{\pgfqpoint{1.660477in}{2.497937in}}%
\pgfpathlineto{\pgfqpoint{1.661550in}{2.499848in}}%
\pgfpathlineto{\pgfqpoint{1.664772in}{2.501395in}}%
\pgfpathlineto{\pgfqpoint{1.665846in}{2.498665in}}%
\pgfpathlineto{\pgfqpoint{1.666920in}{2.494328in}}%
\pgfpathlineto{\pgfqpoint{1.667994in}{2.493782in}}%
\pgfpathlineto{\pgfqpoint{1.669067in}{2.494328in}}%
\pgfpathlineto{\pgfqpoint{1.672289in}{2.497573in}}%
\pgfpathlineto{\pgfqpoint{1.674437in}{2.491144in}}%
\pgfpathlineto{\pgfqpoint{1.675510in}{2.492448in}}%
\pgfpathlineto{\pgfqpoint{1.679806in}{2.491174in}}%
\pgfpathlineto{\pgfqpoint{1.680880in}{2.493691in}}%
\pgfpathlineto{\pgfqpoint{1.681953in}{2.493843in}}%
\pgfpathlineto{\pgfqpoint{1.684101in}{2.498938in}}%
\pgfpathlineto{\pgfqpoint{1.687323in}{2.494904in}}%
\pgfpathlineto{\pgfqpoint{1.689470in}{2.495026in}}%
\pgfpathlineto{\pgfqpoint{1.690544in}{2.492842in}}%
\pgfpathlineto{\pgfqpoint{1.691618in}{2.492296in}}%
\pgfpathlineto{\pgfqpoint{1.696987in}{2.494571in}}%
\pgfpathlineto{\pgfqpoint{1.698061in}{2.494662in}}%
\pgfpathlineto{\pgfqpoint{1.699135in}{2.496300in}}%
\pgfpathlineto{\pgfqpoint{1.702356in}{2.494965in}}%
\pgfpathlineto{\pgfqpoint{1.703430in}{2.495390in}}%
\pgfpathlineto{\pgfqpoint{1.704504in}{2.494449in}}%
\pgfpathlineto{\pgfqpoint{1.705578in}{2.491265in}}%
\pgfpathlineto{\pgfqpoint{1.706652in}{2.493782in}}%
\pgfpathlineto{\pgfqpoint{1.709873in}{2.494358in}}%
\pgfpathlineto{\pgfqpoint{1.710947in}{2.492084in}}%
\pgfpathlineto{\pgfqpoint{1.712021in}{2.491204in}}%
\pgfpathlineto{\pgfqpoint{1.713095in}{2.492508in}}%
\pgfpathlineto{\pgfqpoint{1.714169in}{2.497361in}}%
\pgfpathlineto{\pgfqpoint{1.717390in}{2.496178in}}%
\pgfpathlineto{\pgfqpoint{1.718464in}{2.494692in}}%
\pgfpathlineto{\pgfqpoint{1.719538in}{2.494904in}}%
\pgfpathlineto{\pgfqpoint{1.720612in}{2.498332in}}%
\pgfpathlineto{\pgfqpoint{1.725981in}{2.504034in}}%
\pgfpathlineto{\pgfqpoint{1.728128in}{2.501911in}}%
\pgfpathlineto{\pgfqpoint{1.729202in}{2.499029in}}%
\pgfpathlineto{\pgfqpoint{1.733498in}{2.497573in}}%
\pgfpathlineto{\pgfqpoint{1.734572in}{2.498423in}}%
\pgfpathlineto{\pgfqpoint{1.735645in}{2.498483in}}%
\pgfpathlineto{\pgfqpoint{1.736719in}{2.495754in}}%
\pgfpathlineto{\pgfqpoint{1.742088in}{2.495602in}}%
\pgfpathlineto{\pgfqpoint{1.744236in}{2.491568in}}%
\pgfpathlineto{\pgfqpoint{1.747458in}{2.489870in}}%
\pgfpathlineto{\pgfqpoint{1.748531in}{2.490537in}}%
\pgfpathlineto{\pgfqpoint{1.750679in}{2.493327in}}%
\pgfpathlineto{\pgfqpoint{1.751753in}{2.490931in}}%
\pgfpathlineto{\pgfqpoint{1.754974in}{2.488596in}}%
\pgfpathlineto{\pgfqpoint{1.756048in}{2.490931in}}%
\pgfpathlineto{\pgfqpoint{1.757122in}{2.491932in}}%
\pgfpathlineto{\pgfqpoint{1.758196in}{2.496360in}}%
\pgfpathlineto{\pgfqpoint{1.759270in}{2.494995in}}%
\pgfpathlineto{\pgfqpoint{1.762491in}{2.495602in}}%
\pgfpathlineto{\pgfqpoint{1.765713in}{2.492994in}}%
\pgfpathlineto{\pgfqpoint{1.766787in}{2.494389in}}%
\pgfpathlineto{\pgfqpoint{1.770008in}{2.489111in}}%
\pgfpathlineto{\pgfqpoint{1.771082in}{2.488505in}}%
\pgfpathlineto{\pgfqpoint{1.772156in}{2.491235in}}%
\pgfpathlineto{\pgfqpoint{1.777525in}{2.490507in}}%
\pgfpathlineto{\pgfqpoint{1.778599in}{2.492448in}}%
\pgfpathlineto{\pgfqpoint{1.779673in}{2.489354in}}%
\pgfpathlineto{\pgfqpoint{1.780747in}{2.491053in}}%
\pgfpathlineto{\pgfqpoint{1.781820in}{2.494116in}}%
\pgfpathlineto{\pgfqpoint{1.785042in}{2.496118in}}%
\pgfpathlineto{\pgfqpoint{1.786116in}{2.494783in}}%
\pgfpathlineto{\pgfqpoint{1.788263in}{2.498362in}}%
\pgfpathlineto{\pgfqpoint{1.789337in}{2.495602in}}%
\pgfpathlineto{\pgfqpoint{1.793633in}{2.496300in}}%
\pgfpathlineto{\pgfqpoint{1.795780in}{2.495905in}}%
\pgfpathlineto{\pgfqpoint{1.796854in}{2.493085in}}%
\pgfpathlineto{\pgfqpoint{1.800076in}{2.490719in}}%
\pgfpathlineto{\pgfqpoint{1.802223in}{2.494631in}}%
\pgfpathlineto{\pgfqpoint{1.803297in}{2.494935in}}%
\pgfpathlineto{\pgfqpoint{1.804371in}{2.495936in}}%
\pgfpathlineto{\pgfqpoint{1.808666in}{2.494844in}}%
\pgfpathlineto{\pgfqpoint{1.809740in}{2.496785in}}%
\pgfpathlineto{\pgfqpoint{1.810814in}{2.492872in}}%
\pgfpathlineto{\pgfqpoint{1.811888in}{2.492266in}}%
\pgfpathlineto{\pgfqpoint{1.815109in}{2.494813in}}%
\pgfpathlineto{\pgfqpoint{1.816183in}{2.492690in}}%
\pgfpathlineto{\pgfqpoint{1.818331in}{2.491265in}}%
\pgfpathlineto{\pgfqpoint{1.822626in}{2.495026in}}%
\pgfpathlineto{\pgfqpoint{1.823700in}{2.493661in}}%
\pgfpathlineto{\pgfqpoint{1.824774in}{2.493509in}}%
\pgfpathlineto{\pgfqpoint{1.825848in}{2.492114in}}%
\pgfpathlineto{\pgfqpoint{1.826922in}{2.485351in}}%
\pgfpathlineto{\pgfqpoint{1.830143in}{2.477889in}}%
\pgfpathlineto{\pgfqpoint{1.831217in}{2.472218in}}%
\pgfpathlineto{\pgfqpoint{1.832291in}{2.484137in}}%
\pgfpathlineto{\pgfqpoint{1.833365in}{2.487140in}}%
\pgfpathlineto{\pgfqpoint{1.834438in}{2.484289in}}%
\pgfpathlineto{\pgfqpoint{1.837660in}{2.481044in}}%
\pgfpathlineto{\pgfqpoint{1.838734in}{2.475918in}}%
\pgfpathlineto{\pgfqpoint{1.839808in}{2.479345in}}%
\pgfpathlineto{\pgfqpoint{1.840882in}{2.477404in}}%
\pgfpathlineto{\pgfqpoint{1.841955in}{2.473795in}}%
\pgfpathlineto{\pgfqpoint{1.846251in}{2.480953in}}%
\pgfpathlineto{\pgfqpoint{1.847325in}{2.476282in}}%
\pgfpathlineto{\pgfqpoint{1.849472in}{2.478193in}}%
\pgfpathlineto{\pgfqpoint{1.852694in}{2.479254in}}%
\pgfpathlineto{\pgfqpoint{1.853768in}{2.482196in}}%
\pgfpathlineto{\pgfqpoint{1.855915in}{2.483409in}}%
\pgfpathlineto{\pgfqpoint{1.856989in}{2.479467in}}%
\pgfpathlineto{\pgfqpoint{1.862358in}{2.478375in}}%
\pgfpathlineto{\pgfqpoint{1.863432in}{2.476980in}}%
\pgfpathlineto{\pgfqpoint{1.864506in}{2.472946in}}%
\pgfpathlineto{\pgfqpoint{1.867727in}{2.473947in}}%
\pgfpathlineto{\pgfqpoint{1.868801in}{2.478496in}}%
\pgfpathlineto{\pgfqpoint{1.869875in}{2.479345in}}%
\pgfpathlineto{\pgfqpoint{1.870949in}{2.478860in}}%
\pgfpathlineto{\pgfqpoint{1.872023in}{2.480922in}}%
\pgfpathlineto{\pgfqpoint{1.875244in}{2.483167in}}%
\pgfpathlineto{\pgfqpoint{1.876318in}{2.479527in}}%
\pgfpathlineto{\pgfqpoint{1.877392in}{2.483773in}}%
\pgfpathlineto{\pgfqpoint{1.879540in}{2.484835in}}%
\pgfpathlineto{\pgfqpoint{1.882761in}{2.486533in}}%
\pgfpathlineto{\pgfqpoint{1.883835in}{2.485047in}}%
\pgfpathlineto{\pgfqpoint{1.884909in}{2.482560in}}%
\pgfpathlineto{\pgfqpoint{1.885983in}{2.489688in}}%
\pgfpathlineto{\pgfqpoint{1.887057in}{2.492660in}}%
\pgfpathlineto{\pgfqpoint{1.890278in}{2.491841in}}%
\pgfpathlineto{\pgfqpoint{1.891352in}{2.490871in}}%
\pgfpathlineto{\pgfqpoint{1.892426in}{2.490992in}}%
\pgfpathlineto{\pgfqpoint{1.893500in}{2.496148in}}%
\pgfpathlineto{\pgfqpoint{1.894573in}{2.498271in}}%
\pgfpathlineto{\pgfqpoint{1.897795in}{2.497270in}}%
\pgfpathlineto{\pgfqpoint{1.899943in}{2.498756in}}%
\pgfpathlineto{\pgfqpoint{1.901016in}{2.501152in}}%
\pgfpathlineto{\pgfqpoint{1.902090in}{2.500242in}}%
\pgfpathlineto{\pgfqpoint{1.905312in}{2.503336in}}%
\pgfpathlineto{\pgfqpoint{1.907460in}{2.502729in}}%
\pgfpathlineto{\pgfqpoint{1.908533in}{2.503791in}}%
\pgfpathlineto{\pgfqpoint{1.912829in}{2.499727in}}%
\pgfpathlineto{\pgfqpoint{1.914976in}{2.502487in}}%
\pgfpathlineto{\pgfqpoint{1.916050in}{2.498180in}}%
\pgfpathlineto{\pgfqpoint{1.917124in}{2.497118in}}%
\pgfpathlineto{\pgfqpoint{1.921419in}{2.501516in}}%
\pgfpathlineto{\pgfqpoint{1.922493in}{2.504701in}}%
\pgfpathlineto{\pgfqpoint{1.923567in}{2.504216in}}%
\pgfpathlineto{\pgfqpoint{1.924641in}{2.506248in}}%
\pgfpathlineto{\pgfqpoint{1.927862in}{2.507036in}}%
\pgfpathlineto{\pgfqpoint{1.928936in}{2.505035in}}%
\pgfpathlineto{\pgfqpoint{1.930010in}{2.504822in}}%
\pgfpathlineto{\pgfqpoint{1.932158in}{2.505944in}}%
\pgfpathlineto{\pgfqpoint{1.935379in}{2.502851in}}%
\pgfpathlineto{\pgfqpoint{1.936453in}{2.505914in}}%
\pgfpathlineto{\pgfqpoint{1.937527in}{2.505065in}}%
\pgfpathlineto{\pgfqpoint{1.938601in}{2.501607in}}%
\pgfpathlineto{\pgfqpoint{1.939675in}{2.507522in}}%
\pgfpathlineto{\pgfqpoint{1.942896in}{2.508553in}}%
\pgfpathlineto{\pgfqpoint{1.943970in}{2.506066in}}%
\pgfpathlineto{\pgfqpoint{1.945044in}{2.505338in}}%
\pgfpathlineto{\pgfqpoint{1.946118in}{2.506672in}}%
\pgfpathlineto{\pgfqpoint{1.947192in}{2.504064in}}%
\pgfpathlineto{\pgfqpoint{1.950413in}{2.505338in}}%
\pgfpathlineto{\pgfqpoint{1.952561in}{2.513830in}}%
\pgfpathlineto{\pgfqpoint{1.954708in}{2.504792in}}%
\pgfpathlineto{\pgfqpoint{1.957930in}{2.503791in}}%
\pgfpathlineto{\pgfqpoint{1.960078in}{2.509008in}}%
\pgfpathlineto{\pgfqpoint{1.961151in}{2.509645in}}%
\pgfpathlineto{\pgfqpoint{1.965447in}{2.508280in}}%
\pgfpathlineto{\pgfqpoint{1.966521in}{2.510494in}}%
\pgfpathlineto{\pgfqpoint{1.967594in}{2.509796in}}%
\pgfpathlineto{\pgfqpoint{1.968668in}{2.506915in}}%
\pgfpathlineto{\pgfqpoint{1.972964in}{2.500758in}}%
\pgfpathlineto{\pgfqpoint{1.974038in}{2.501911in}}%
\pgfpathlineto{\pgfqpoint{1.975111in}{2.500515in}}%
\pgfpathlineto{\pgfqpoint{1.977259in}{2.494389in}}%
\pgfpathlineto{\pgfqpoint{1.980481in}{2.492781in}}%
\pgfpathlineto{\pgfqpoint{1.981554in}{2.494631in}}%
\pgfpathlineto{\pgfqpoint{1.982628in}{2.491265in}}%
\pgfpathlineto{\pgfqpoint{1.983702in}{2.496391in}}%
\pgfpathlineto{\pgfqpoint{1.984776in}{2.491235in}}%
\pgfpathlineto{\pgfqpoint{1.989071in}{2.492599in}}%
\pgfpathlineto{\pgfqpoint{1.990145in}{2.487807in}}%
\pgfpathlineto{\pgfqpoint{1.991219in}{2.488323in}}%
\pgfpathlineto{\pgfqpoint{1.992293in}{2.490537in}}%
\pgfpathlineto{\pgfqpoint{1.995514in}{2.489566in}}%
\pgfpathlineto{\pgfqpoint{1.996588in}{2.502669in}}%
\pgfpathlineto{\pgfqpoint{1.997662in}{2.505368in}}%
\pgfpathlineto{\pgfqpoint{1.998736in}{2.505671in}}%
\pgfpathlineto{\pgfqpoint{1.999810in}{2.511616in}}%
\pgfpathlineto{\pgfqpoint{2.003031in}{2.511434in}}%
\pgfpathlineto{\pgfqpoint{2.004105in}{2.508795in}}%
\pgfpathlineto{\pgfqpoint{2.005179in}{2.510797in}}%
\pgfpathlineto{\pgfqpoint{2.006253in}{2.510130in}}%
\pgfpathlineto{\pgfqpoint{2.007326in}{2.500940in}}%
\pgfpathlineto{\pgfqpoint{2.010548in}{2.504913in}}%
\pgfpathlineto{\pgfqpoint{2.011622in}{2.504853in}}%
\pgfpathlineto{\pgfqpoint{2.013770in}{2.504094in}}%
\pgfpathlineto{\pgfqpoint{2.020213in}{2.506308in}}%
\pgfpathlineto{\pgfqpoint{2.021286in}{2.511070in}}%
\pgfpathlineto{\pgfqpoint{2.022360in}{2.512920in}}%
\pgfpathlineto{\pgfqpoint{2.025582in}{2.514558in}}%
\pgfpathlineto{\pgfqpoint{2.026656in}{2.512708in}}%
\pgfpathlineto{\pgfqpoint{2.027729in}{2.515134in}}%
\pgfpathlineto{\pgfqpoint{2.028803in}{2.519047in}}%
\pgfpathlineto{\pgfqpoint{2.029877in}{2.517379in}}%
\pgfpathlineto{\pgfqpoint{2.033099in}{2.515802in}}%
\pgfpathlineto{\pgfqpoint{2.034172in}{2.521352in}}%
\pgfpathlineto{\pgfqpoint{2.037394in}{2.519380in}}%
\pgfpathlineto{\pgfqpoint{2.040615in}{2.520048in}}%
\pgfpathlineto{\pgfqpoint{2.041689in}{2.518592in}}%
\pgfpathlineto{\pgfqpoint{2.043837in}{2.521140in}}%
\pgfpathlineto{\pgfqpoint{2.044911in}{2.522717in}}%
\pgfpathlineto{\pgfqpoint{2.049206in}{2.522868in}}%
\pgfpathlineto{\pgfqpoint{2.050280in}{2.521898in}}%
\pgfpathlineto{\pgfqpoint{2.051354in}{2.520048in}}%
\pgfpathlineto{\pgfqpoint{2.052428in}{2.522140in}}%
\pgfpathlineto{\pgfqpoint{2.056723in}{2.521807in}}%
\pgfpathlineto{\pgfqpoint{2.057797in}{2.524900in}}%
\pgfpathlineto{\pgfqpoint{2.058871in}{2.524385in}}%
\pgfpathlineto{\pgfqpoint{2.063166in}{2.524142in}}%
\pgfpathlineto{\pgfqpoint{2.064240in}{2.526660in}}%
\pgfpathlineto{\pgfqpoint{2.065314in}{2.526235in}}%
\pgfpathlineto{\pgfqpoint{2.066388in}{2.524082in}}%
\pgfpathlineto{\pgfqpoint{2.067461in}{2.526811in}}%
\pgfpathlineto{\pgfqpoint{2.070683in}{2.525143in}}%
\pgfpathlineto{\pgfqpoint{2.072831in}{2.527448in}}%
\pgfpathlineto{\pgfqpoint{2.074978in}{2.526569in}}%
\pgfpathlineto{\pgfqpoint{2.078200in}{2.526205in}}%
\pgfpathlineto{\pgfqpoint{2.080348in}{2.528692in}}%
\pgfpathlineto{\pgfqpoint{2.081421in}{2.528601in}}%
\pgfpathlineto{\pgfqpoint{2.082495in}{2.529541in}}%
\pgfpathlineto{\pgfqpoint{2.085717in}{2.531603in}}%
\pgfpathlineto{\pgfqpoint{2.087864in}{2.538913in}}%
\pgfpathlineto{\pgfqpoint{2.088938in}{2.538913in}}%
\pgfpathlineto{\pgfqpoint{2.090012in}{2.538215in}}%
\pgfpathlineto{\pgfqpoint{2.093234in}{2.538731in}}%
\pgfpathlineto{\pgfqpoint{2.094307in}{2.537063in}}%
\pgfpathlineto{\pgfqpoint{2.096455in}{2.535880in}}%
\pgfpathlineto{\pgfqpoint{2.097529in}{2.534788in}}%
\pgfpathlineto{\pgfqpoint{2.100750in}{2.536638in}}%
\pgfpathlineto{\pgfqpoint{2.101824in}{2.536456in}}%
\pgfpathlineto{\pgfqpoint{2.102898in}{2.535182in}}%
\pgfpathlineto{\pgfqpoint{2.103972in}{2.537002in}}%
\pgfpathlineto{\pgfqpoint{2.105046in}{2.536608in}}%
\pgfpathlineto{\pgfqpoint{2.108267in}{2.539307in}}%
\pgfpathlineto{\pgfqpoint{2.109341in}{2.541946in}}%
\pgfpathlineto{\pgfqpoint{2.112563in}{2.538882in}}%
\pgfpathlineto{\pgfqpoint{2.115784in}{2.541309in}}%
\pgfpathlineto{\pgfqpoint{2.119006in}{2.536911in}}%
\pgfpathlineto{\pgfqpoint{2.120080in}{2.538549in}}%
\pgfpathlineto{\pgfqpoint{2.123301in}{2.537214in}}%
\pgfpathlineto{\pgfqpoint{2.125449in}{2.540520in}}%
\pgfpathlineto{\pgfqpoint{2.126523in}{2.539277in}}%
\pgfpathlineto{\pgfqpoint{2.127596in}{2.539732in}}%
\pgfpathlineto{\pgfqpoint{2.132966in}{2.538943in}}%
\pgfpathlineto{\pgfqpoint{2.134039in}{2.543705in}}%
\pgfpathlineto{\pgfqpoint{2.139409in}{2.547132in}}%
\pgfpathlineto{\pgfqpoint{2.140482in}{2.547344in}}%
\pgfpathlineto{\pgfqpoint{2.141556in}{2.550681in}}%
\pgfpathlineto{\pgfqpoint{2.142630in}{2.550711in}}%
\pgfpathlineto{\pgfqpoint{2.145852in}{2.550104in}}%
\pgfpathlineto{\pgfqpoint{2.146926in}{2.551014in}}%
\pgfpathlineto{\pgfqpoint{2.147999in}{2.549043in}}%
\pgfpathlineto{\pgfqpoint{2.149073in}{2.549680in}}%
\pgfpathlineto{\pgfqpoint{2.150147in}{2.546435in}}%
\pgfpathlineto{\pgfqpoint{2.153369in}{2.549407in}}%
\pgfpathlineto{\pgfqpoint{2.154442in}{2.548406in}}%
\pgfpathlineto{\pgfqpoint{2.155516in}{2.549164in}}%
\pgfpathlineto{\pgfqpoint{2.156590in}{2.551742in}}%
\pgfpathlineto{\pgfqpoint{2.157664in}{2.546859in}}%
\pgfpathlineto{\pgfqpoint{2.160885in}{2.549437in}}%
\pgfpathlineto{\pgfqpoint{2.164107in}{2.562631in}}%
\pgfpathlineto{\pgfqpoint{2.165181in}{2.562600in}}%
\pgfpathlineto{\pgfqpoint{2.170550in}{2.566361in}}%
\pgfpathlineto{\pgfqpoint{2.171624in}{2.565997in}}%
\pgfpathlineto{\pgfqpoint{2.172698in}{2.566937in}}%
\pgfpathlineto{\pgfqpoint{2.183436in}{2.567756in}}%
\pgfpathlineto{\pgfqpoint{2.184510in}{2.573610in}}%
\pgfpathlineto{\pgfqpoint{2.187731in}{2.573003in}}%
\pgfpathlineto{\pgfqpoint{2.190953in}{2.572609in}}%
\pgfpathlineto{\pgfqpoint{2.192027in}{2.573337in}}%
\pgfpathlineto{\pgfqpoint{2.194174in}{2.571426in}}%
\pgfpathlineto{\pgfqpoint{2.195248in}{2.573549in}}%
\pgfpathlineto{\pgfqpoint{2.198470in}{2.574035in}}%
\pgfpathlineto{\pgfqpoint{2.199544in}{2.572518in}}%
\pgfpathlineto{\pgfqpoint{2.200617in}{2.569849in}}%
\pgfpathlineto{\pgfqpoint{2.201691in}{2.569758in}}%
\pgfpathlineto{\pgfqpoint{2.202765in}{2.570820in}}%
\pgfpathlineto{\pgfqpoint{2.208134in}{2.568363in}}%
\pgfpathlineto{\pgfqpoint{2.209208in}{2.569485in}}%
\pgfpathlineto{\pgfqpoint{2.210282in}{2.567969in}}%
\pgfpathlineto{\pgfqpoint{2.213503in}{2.565451in}}%
\pgfpathlineto{\pgfqpoint{2.214577in}{2.559931in}}%
\pgfpathlineto{\pgfqpoint{2.215651in}{2.562661in}}%
\pgfpathlineto{\pgfqpoint{2.216725in}{2.561023in}}%
\pgfpathlineto{\pgfqpoint{2.217799in}{2.561023in}}%
\pgfpathlineto{\pgfqpoint{2.221020in}{2.558809in}}%
\pgfpathlineto{\pgfqpoint{2.222094in}{2.559689in}}%
\pgfpathlineto{\pgfqpoint{2.223168in}{2.557626in}}%
\pgfpathlineto{\pgfqpoint{2.224242in}{2.557232in}}%
\pgfpathlineto{\pgfqpoint{2.225316in}{2.558566in}}%
\pgfpathlineto{\pgfqpoint{2.228537in}{2.561023in}}%
\pgfpathlineto{\pgfqpoint{2.229611in}{2.559749in}}%
\pgfpathlineto{\pgfqpoint{2.231759in}{2.558657in}}%
\pgfpathlineto{\pgfqpoint{2.232833in}{2.559325in}}%
\pgfpathlineto{\pgfqpoint{2.237128in}{2.560538in}}%
\pgfpathlineto{\pgfqpoint{2.239276in}{2.559749in}}%
\pgfpathlineto{\pgfqpoint{2.240349in}{2.556292in}}%
\pgfpathlineto{\pgfqpoint{2.243571in}{2.558900in}}%
\pgfpathlineto{\pgfqpoint{2.244645in}{2.554563in}}%
\pgfpathlineto{\pgfqpoint{2.245719in}{2.555260in}}%
\pgfpathlineto{\pgfqpoint{2.246792in}{2.557414in}}%
\pgfpathlineto{\pgfqpoint{2.247866in}{2.556352in}}%
\pgfpathlineto{\pgfqpoint{2.251088in}{2.554684in}}%
\pgfpathlineto{\pgfqpoint{2.252162in}{2.555503in}}%
\pgfpathlineto{\pgfqpoint{2.254309in}{2.559719in}}%
\pgfpathlineto{\pgfqpoint{2.255383in}{2.557899in}}%
\pgfpathlineto{\pgfqpoint{2.258605in}{2.555018in}}%
\pgfpathlineto{\pgfqpoint{2.259679in}{2.559052in}}%
\pgfpathlineto{\pgfqpoint{2.260752in}{2.559537in}}%
\pgfpathlineto{\pgfqpoint{2.261826in}{2.553592in}}%
\pgfpathlineto{\pgfqpoint{2.262900in}{2.556019in}}%
\pgfpathlineto{\pgfqpoint{2.268269in}{2.558961in}}%
\pgfpathlineto{\pgfqpoint{2.269343in}{2.557747in}}%
\pgfpathlineto{\pgfqpoint{2.270417in}{2.559112in}}%
\pgfpathlineto{\pgfqpoint{2.273638in}{2.560689in}}%
\pgfpathlineto{\pgfqpoint{2.274712in}{2.554624in}}%
\pgfpathlineto{\pgfqpoint{2.276860in}{2.556383in}}%
\pgfpathlineto{\pgfqpoint{2.277934in}{2.554411in}}%
\pgfpathlineto{\pgfqpoint{2.281155in}{2.557020in}}%
\pgfpathlineto{\pgfqpoint{2.282229in}{2.548406in}}%
\pgfpathlineto{\pgfqpoint{2.283303in}{2.546101in}}%
\pgfpathlineto{\pgfqpoint{2.284377in}{2.546889in}}%
\pgfpathlineto{\pgfqpoint{2.285451in}{2.542886in}}%
\pgfpathlineto{\pgfqpoint{2.288672in}{2.543341in}}%
\pgfpathlineto{\pgfqpoint{2.290820in}{2.546010in}}%
\pgfpathlineto{\pgfqpoint{2.291894in}{2.549195in}}%
\pgfpathlineto{\pgfqpoint{2.292968in}{2.548163in}}%
\pgfpathlineto{\pgfqpoint{2.296189in}{2.550013in}}%
\pgfpathlineto{\pgfqpoint{2.298337in}{2.546859in}}%
\pgfpathlineto{\pgfqpoint{2.300484in}{2.547557in}}%
\pgfpathlineto{\pgfqpoint{2.304780in}{2.552986in}}%
\pgfpathlineto{\pgfqpoint{2.305854in}{2.562115in}}%
\pgfpathlineto{\pgfqpoint{2.311223in}{2.551712in}}%
\pgfpathlineto{\pgfqpoint{2.312297in}{2.550954in}}%
\pgfpathlineto{\pgfqpoint{2.314444in}{2.551651in}}%
\pgfpathlineto{\pgfqpoint{2.315518in}{2.550499in}}%
\pgfpathlineto{\pgfqpoint{2.318740in}{2.549467in}}%
\pgfpathlineto{\pgfqpoint{2.319814in}{2.543098in}}%
\pgfpathlineto{\pgfqpoint{2.320887in}{2.544038in}}%
\pgfpathlineto{\pgfqpoint{2.323035in}{2.547011in}}%
\pgfpathlineto{\pgfqpoint{2.326257in}{2.544190in}}%
\pgfpathlineto{\pgfqpoint{2.327330in}{2.542370in}}%
\pgfpathlineto{\pgfqpoint{2.328404in}{2.539034in}}%
\pgfpathlineto{\pgfqpoint{2.329478in}{2.539277in}}%
\pgfpathlineto{\pgfqpoint{2.330552in}{2.540915in}}%
\pgfpathlineto{\pgfqpoint{2.334847in}{2.541187in}}%
\pgfpathlineto{\pgfqpoint{2.335921in}{2.538488in}}%
\pgfpathlineto{\pgfqpoint{2.336995in}{2.538155in}}%
\pgfpathlineto{\pgfqpoint{2.338069in}{2.541733in}}%
\pgfpathlineto{\pgfqpoint{2.342364in}{2.551985in}}%
\pgfpathlineto{\pgfqpoint{2.343438in}{2.549437in}}%
\pgfpathlineto{\pgfqpoint{2.344512in}{2.551985in}}%
\pgfpathlineto{\pgfqpoint{2.349881in}{2.551318in}}%
\pgfpathlineto{\pgfqpoint{2.350955in}{2.550347in}}%
\pgfpathlineto{\pgfqpoint{2.352029in}{2.550711in}}%
\pgfpathlineto{\pgfqpoint{2.353102in}{2.552167in}}%
\pgfpathlineto{\pgfqpoint{2.357398in}{2.552045in}}%
\pgfpathlineto{\pgfqpoint{2.358472in}{2.549740in}}%
\pgfpathlineto{\pgfqpoint{2.359546in}{2.550863in}}%
\pgfpathlineto{\pgfqpoint{2.360619in}{2.550074in}}%
\pgfpathlineto{\pgfqpoint{2.364915in}{2.551833in}}%
\pgfpathlineto{\pgfqpoint{2.365989in}{2.551318in}}%
\pgfpathlineto{\pgfqpoint{2.367062in}{2.554715in}}%
\pgfpathlineto{\pgfqpoint{2.368136in}{2.553137in}}%
\pgfpathlineto{\pgfqpoint{2.372432in}{2.552743in}}%
\pgfpathlineto{\pgfqpoint{2.373505in}{2.548709in}}%
\pgfpathlineto{\pgfqpoint{2.375653in}{2.548345in}}%
\pgfpathlineto{\pgfqpoint{2.379948in}{2.549104in}}%
\pgfpathlineto{\pgfqpoint{2.381022in}{2.548618in}}%
\pgfpathlineto{\pgfqpoint{2.382096in}{2.547223in}}%
\pgfpathlineto{\pgfqpoint{2.386391in}{2.546404in}}%
\pgfpathlineto{\pgfqpoint{2.387465in}{2.540338in}}%
\pgfpathlineto{\pgfqpoint{2.388539in}{2.543280in}}%
\pgfpathlineto{\pgfqpoint{2.389613in}{2.540551in}}%
\pgfpathlineto{\pgfqpoint{2.390687in}{2.544918in}}%
\pgfpathlineto{\pgfqpoint{2.393908in}{2.544190in}}%
\pgfpathlineto{\pgfqpoint{2.398204in}{2.545646in}}%
\pgfpathlineto{\pgfqpoint{2.403573in}{2.544948in}}%
\pgfpathlineto{\pgfqpoint{2.404647in}{2.546889in}}%
\pgfpathlineto{\pgfqpoint{2.405721in}{2.550135in}}%
\pgfpathlineto{\pgfqpoint{2.408942in}{2.551955in}}%
\pgfpathlineto{\pgfqpoint{2.410016in}{2.553319in}}%
\pgfpathlineto{\pgfqpoint{2.413237in}{2.560356in}}%
\pgfpathlineto{\pgfqpoint{2.417533in}{2.562600in}}%
\pgfpathlineto{\pgfqpoint{2.418607in}{2.562206in}}%
\pgfpathlineto{\pgfqpoint{2.420754in}{2.573549in}}%
\pgfpathlineto{\pgfqpoint{2.425050in}{2.572063in}}%
\pgfpathlineto{\pgfqpoint{2.426124in}{2.576764in}}%
\pgfpathlineto{\pgfqpoint{2.427197in}{2.576097in}}%
\pgfpathlineto{\pgfqpoint{2.428271in}{2.576552in}}%
\pgfpathlineto{\pgfqpoint{2.432567in}{2.576673in}}%
\pgfpathlineto{\pgfqpoint{2.433640in}{2.577431in}}%
\pgfpathlineto{\pgfqpoint{2.434714in}{2.582678in}}%
\pgfpathlineto{\pgfqpoint{2.435788in}{2.583406in}}%
\pgfpathlineto{\pgfqpoint{2.439010in}{2.584741in}}%
\pgfpathlineto{\pgfqpoint{2.440083in}{2.585802in}}%
\pgfpathlineto{\pgfqpoint{2.441157in}{2.591231in}}%
\pgfpathlineto{\pgfqpoint{2.443305in}{2.588684in}}%
\pgfpathlineto{\pgfqpoint{2.446526in}{2.588684in}}%
\pgfpathlineto{\pgfqpoint{2.449748in}{2.582527in}}%
\pgfpathlineto{\pgfqpoint{2.450822in}{2.581344in}}%
\pgfpathlineto{\pgfqpoint{2.454043in}{2.582254in}}%
\pgfpathlineto{\pgfqpoint{2.455117in}{2.581860in}}%
\pgfpathlineto{\pgfqpoint{2.456191in}{2.579767in}}%
\pgfpathlineto{\pgfqpoint{2.458339in}{2.578705in}}%
\pgfpathlineto{\pgfqpoint{2.463708in}{2.579433in}}%
\pgfpathlineto{\pgfqpoint{2.464782in}{2.580131in}}%
\pgfpathlineto{\pgfqpoint{2.470151in}{2.577765in}}%
\pgfpathlineto{\pgfqpoint{2.471225in}{2.581132in}}%
\pgfpathlineto{\pgfqpoint{2.472299in}{2.579949in}}%
\pgfpathlineto{\pgfqpoint{2.476594in}{2.582042in}}%
\pgfpathlineto{\pgfqpoint{2.477668in}{2.570971in}}%
\pgfpathlineto{\pgfqpoint{2.478742in}{2.569697in}}%
\pgfpathlineto{\pgfqpoint{2.479815in}{2.571093in}}%
\pgfpathlineto{\pgfqpoint{2.480889in}{2.570789in}}%
\pgfpathlineto{\pgfqpoint{2.486258in}{2.575763in}}%
\pgfpathlineto{\pgfqpoint{2.487332in}{2.576400in}}%
\pgfpathlineto{\pgfqpoint{2.488406in}{2.575642in}}%
\pgfpathlineto{\pgfqpoint{2.491628in}{2.575278in}}%
\pgfpathlineto{\pgfqpoint{2.492702in}{2.576309in}}%
\pgfpathlineto{\pgfqpoint{2.493775in}{2.575248in}}%
\pgfpathlineto{\pgfqpoint{2.494849in}{2.577007in}}%
\pgfpathlineto{\pgfqpoint{2.495923in}{2.575763in}}%
\pgfpathlineto{\pgfqpoint{2.500218in}{2.574914in}}%
\pgfpathlineto{\pgfqpoint{2.501292in}{2.573762in}}%
\pgfpathlineto{\pgfqpoint{2.503440in}{2.576127in}}%
\pgfpathlineto{\pgfqpoint{2.507735in}{2.587835in}}%
\pgfpathlineto{\pgfqpoint{2.508809in}{2.584741in}}%
\pgfpathlineto{\pgfqpoint{2.509883in}{2.585590in}}%
\pgfpathlineto{\pgfqpoint{2.510957in}{2.585651in}}%
\pgfpathlineto{\pgfqpoint{2.516326in}{2.587107in}}%
\pgfpathlineto{\pgfqpoint{2.517400in}{2.589563in}}%
\pgfpathlineto{\pgfqpoint{2.518474in}{2.587835in}}%
\pgfpathlineto{\pgfqpoint{2.522769in}{2.588380in}}%
\pgfpathlineto{\pgfqpoint{2.525991in}{2.596842in}}%
\pgfpathlineto{\pgfqpoint{2.529212in}{2.597661in}}%
\pgfpathlineto{\pgfqpoint{2.530286in}{2.598996in}}%
\pgfpathlineto{\pgfqpoint{2.532434in}{2.598177in}}%
\pgfpathlineto{\pgfqpoint{2.533507in}{2.600998in}}%
\pgfpathlineto{\pgfqpoint{2.537803in}{2.602393in}}%
\pgfpathlineto{\pgfqpoint{2.538877in}{2.604546in}}%
\pgfpathlineto{\pgfqpoint{2.539950in}{2.605304in}}%
\pgfpathlineto{\pgfqpoint{2.541024in}{2.609035in}}%
\pgfpathlineto{\pgfqpoint{2.544246in}{2.608246in}}%
\pgfpathlineto{\pgfqpoint{2.545320in}{2.608671in}}%
\pgfpathlineto{\pgfqpoint{2.546393in}{2.610642in}}%
\pgfpathlineto{\pgfqpoint{2.547467in}{2.613888in}}%
\pgfpathlineto{\pgfqpoint{2.548541in}{2.614980in}}%
\pgfpathlineto{\pgfqpoint{2.551763in}{2.614737in}}%
\pgfpathlineto{\pgfqpoint{2.554984in}{2.604152in}}%
\pgfpathlineto{\pgfqpoint{2.556058in}{2.603151in}}%
\pgfpathlineto{\pgfqpoint{2.559279in}{2.604910in}}%
\pgfpathlineto{\pgfqpoint{2.561427in}{2.607003in}}%
\pgfpathlineto{\pgfqpoint{2.562501in}{2.603818in}}%
\pgfpathlineto{\pgfqpoint{2.563575in}{2.603879in}}%
\pgfpathlineto{\pgfqpoint{2.567870in}{2.600148in}}%
\pgfpathlineto{\pgfqpoint{2.568944in}{2.602999in}}%
\pgfpathlineto{\pgfqpoint{2.570018in}{2.601938in}}%
\pgfpathlineto{\pgfqpoint{2.571092in}{2.604061in}}%
\pgfpathlineto{\pgfqpoint{2.574313in}{2.602757in}}%
\pgfpathlineto{\pgfqpoint{2.575387in}{2.609369in}}%
\pgfpathlineto{\pgfqpoint{2.576461in}{2.611492in}}%
\pgfpathlineto{\pgfqpoint{2.577535in}{2.615374in}}%
\pgfpathlineto{\pgfqpoint{2.578609in}{2.611795in}}%
\pgfpathlineto{\pgfqpoint{2.582904in}{2.601998in}}%
\pgfpathlineto{\pgfqpoint{2.583978in}{2.599329in}}%
\pgfpathlineto{\pgfqpoint{2.585052in}{2.598996in}}%
\pgfpathlineto{\pgfqpoint{2.586125in}{2.601907in}}%
\pgfpathlineto{\pgfqpoint{2.589347in}{2.604394in}}%
\pgfpathlineto{\pgfqpoint{2.591495in}{2.602787in}}%
\pgfpathlineto{\pgfqpoint{2.592568in}{2.606184in}}%
\pgfpathlineto{\pgfqpoint{2.596864in}{2.604789in}}%
\pgfpathlineto{\pgfqpoint{2.597938in}{2.603090in}}%
\pgfpathlineto{\pgfqpoint{2.599012in}{2.605850in}}%
\pgfpathlineto{\pgfqpoint{2.601159in}{2.605486in}}%
\pgfpathlineto{\pgfqpoint{2.604381in}{2.606548in}}%
\pgfpathlineto{\pgfqpoint{2.605455in}{2.606275in}}%
\pgfpathlineto{\pgfqpoint{2.606528in}{2.608519in}}%
\pgfpathlineto{\pgfqpoint{2.607602in}{2.605304in}}%
\pgfpathlineto{\pgfqpoint{2.608676in}{2.604122in}}%
\pgfpathlineto{\pgfqpoint{2.611898in}{2.606487in}}%
\pgfpathlineto{\pgfqpoint{2.612971in}{2.610036in}}%
\pgfpathlineto{\pgfqpoint{2.614045in}{2.604607in}}%
\pgfpathlineto{\pgfqpoint{2.615119in}{2.604880in}}%
\pgfpathlineto{\pgfqpoint{2.616193in}{2.603818in}}%
\pgfpathlineto{\pgfqpoint{2.619414in}{2.603970in}}%
\pgfpathlineto{\pgfqpoint{2.620488in}{2.605335in}}%
\pgfpathlineto{\pgfqpoint{2.621562in}{2.602059in}}%
\pgfpathlineto{\pgfqpoint{2.622636in}{2.605790in}}%
\pgfpathlineto{\pgfqpoint{2.623710in}{2.601938in}}%
\pgfpathlineto{\pgfqpoint{2.628005in}{2.598693in}}%
\pgfpathlineto{\pgfqpoint{2.629079in}{2.600907in}}%
\pgfpathlineto{\pgfqpoint{2.630153in}{2.605274in}}%
\pgfpathlineto{\pgfqpoint{2.631227in}{2.601786in}}%
\pgfpathlineto{\pgfqpoint{2.634448in}{2.608216in}}%
\pgfpathlineto{\pgfqpoint{2.635522in}{2.606548in}}%
\pgfpathlineto{\pgfqpoint{2.636596in}{2.606032in}}%
\pgfpathlineto{\pgfqpoint{2.637670in}{2.611006in}}%
\pgfpathlineto{\pgfqpoint{2.641965in}{2.614434in}}%
\pgfpathlineto{\pgfqpoint{2.643039in}{2.613979in}}%
\pgfpathlineto{\pgfqpoint{2.645187in}{2.604000in}}%
\pgfpathlineto{\pgfqpoint{2.646260in}{2.602969in}}%
\pgfpathlineto{\pgfqpoint{2.650556in}{2.601695in}}%
\pgfpathlineto{\pgfqpoint{2.651630in}{2.598268in}}%
\pgfpathlineto{\pgfqpoint{2.652703in}{2.597449in}}%
\pgfpathlineto{\pgfqpoint{2.653777in}{2.598996in}}%
\pgfpathlineto{\pgfqpoint{2.656999in}{2.602484in}}%
\pgfpathlineto{\pgfqpoint{2.659146in}{2.607276in}}%
\pgfpathlineto{\pgfqpoint{2.660220in}{2.608155in}}%
\pgfpathlineto{\pgfqpoint{2.664516in}{2.608883in}}%
\pgfpathlineto{\pgfqpoint{2.665590in}{2.610187in}}%
\pgfpathlineto{\pgfqpoint{2.666663in}{2.618104in}}%
\pgfpathlineto{\pgfqpoint{2.667737in}{2.618619in}}%
\pgfpathlineto{\pgfqpoint{2.668811in}{2.617467in}}%
\pgfpathlineto{\pgfqpoint{2.672033in}{2.616587in}}%
\pgfpathlineto{\pgfqpoint{2.673106in}{2.629993in}}%
\pgfpathlineto{\pgfqpoint{2.674180in}{2.629689in}}%
\pgfpathlineto{\pgfqpoint{2.675254in}{2.633602in}}%
\pgfpathlineto{\pgfqpoint{2.679549in}{2.638121in}}%
\pgfpathlineto{\pgfqpoint{2.680623in}{2.632449in}}%
\pgfpathlineto{\pgfqpoint{2.681697in}{2.634512in}}%
\pgfpathlineto{\pgfqpoint{2.682771in}{2.632935in}}%
\pgfpathlineto{\pgfqpoint{2.683845in}{2.632844in}}%
\pgfpathlineto{\pgfqpoint{2.688140in}{2.626020in}}%
\pgfpathlineto{\pgfqpoint{2.689214in}{2.627657in}}%
\pgfpathlineto{\pgfqpoint{2.695657in}{2.627051in}}%
\pgfpathlineto{\pgfqpoint{2.696731in}{2.631509in}}%
\pgfpathlineto{\pgfqpoint{2.698879in}{2.626474in}}%
\pgfpathlineto{\pgfqpoint{2.702100in}{2.627020in}}%
\pgfpathlineto{\pgfqpoint{2.705322in}{2.624473in}}%
\pgfpathlineto{\pgfqpoint{2.706395in}{2.621986in}}%
\pgfpathlineto{\pgfqpoint{2.709617in}{2.621773in}}%
\pgfpathlineto{\pgfqpoint{2.710691in}{2.622956in}}%
\pgfpathlineto{\pgfqpoint{2.711765in}{2.619923in}}%
\pgfpathlineto{\pgfqpoint{2.717134in}{2.624685in}}%
\pgfpathlineto{\pgfqpoint{2.718208in}{2.630235in}}%
\pgfpathlineto{\pgfqpoint{2.719281in}{2.629629in}}%
\pgfpathlineto{\pgfqpoint{2.720355in}{2.628264in}}%
\pgfpathlineto{\pgfqpoint{2.721429in}{2.630144in}}%
\pgfpathlineto{\pgfqpoint{2.724651in}{2.627324in}}%
\pgfpathlineto{\pgfqpoint{2.725724in}{2.629234in}}%
\pgfpathlineto{\pgfqpoint{2.726798in}{2.633268in}}%
\pgfpathlineto{\pgfqpoint{2.727872in}{2.630205in}}%
\pgfpathlineto{\pgfqpoint{2.728946in}{2.631904in}}%
\pgfpathlineto{\pgfqpoint{2.732167in}{2.633481in}}%
\pgfpathlineto{\pgfqpoint{2.733241in}{2.637696in}}%
\pgfpathlineto{\pgfqpoint{2.734315in}{2.638546in}}%
\pgfpathlineto{\pgfqpoint{2.735389in}{2.634967in}}%
\pgfpathlineto{\pgfqpoint{2.736463in}{2.637302in}}%
\pgfpathlineto{\pgfqpoint{2.742906in}{2.633268in}}%
\pgfpathlineto{\pgfqpoint{2.743980in}{2.630539in}}%
\pgfpathlineto{\pgfqpoint{2.748275in}{2.630448in}}%
\pgfpathlineto{\pgfqpoint{2.749349in}{2.631843in}}%
\pgfpathlineto{\pgfqpoint{2.750423in}{2.631813in}}%
\pgfpathlineto{\pgfqpoint{2.751497in}{2.629386in}}%
\pgfpathlineto{\pgfqpoint{2.751497in}{2.629386in}}%
\pgfusepath{stroke}%
\end{pgfscope}%
\begin{pgfscope}%
\pgfpathrectangle{\pgfqpoint{0.285588in}{2.255883in}}{\pgfqpoint{2.583333in}{0.400885in}}%
\pgfusepath{clip}%
\pgfsetroundcap%
\pgfsetroundjoin%
\pgfsetlinewidth{1.505625pt}%
\definecolor{currentstroke}{rgb}{0.580392,0.403922,0.741176}%
\pgfsetstrokecolor{currentstroke}%
\pgfsetdash{}{0pt}%
\pgfpathmoveto{\pgfqpoint{0.403012in}{2.385050in}}%
\pgfpathlineto{\pgfqpoint{0.404086in}{2.384080in}}%
\pgfpathlineto{\pgfqpoint{0.405159in}{2.383898in}}%
\pgfpathlineto{\pgfqpoint{0.406233in}{2.382503in}}%
\pgfpathlineto{\pgfqpoint{0.409455in}{2.382745in}}%
\pgfpathlineto{\pgfqpoint{0.410529in}{2.383413in}}%
\pgfpathlineto{\pgfqpoint{0.412676in}{2.383473in}}%
\pgfpathlineto{\pgfqpoint{0.413750in}{2.383534in}}%
\pgfpathlineto{\pgfqpoint{0.421267in}{2.382414in}}%
\pgfpathlineto{\pgfqpoint{0.426636in}{2.381270in}}%
\pgfpathlineto{\pgfqpoint{0.427710in}{2.380103in}}%
\pgfpathlineto{\pgfqpoint{0.428784in}{2.380457in}}%
\pgfpathlineto{\pgfqpoint{0.434153in}{2.380128in}}%
\pgfpathlineto{\pgfqpoint{0.436301in}{2.380245in}}%
\pgfpathlineto{\pgfqpoint{0.440596in}{2.379330in}}%
\pgfpathlineto{\pgfqpoint{0.441670in}{2.379301in}}%
\pgfpathlineto{\pgfqpoint{0.443818in}{2.377795in}}%
\pgfpathlineto{\pgfqpoint{0.462073in}{2.378769in}}%
\pgfpathlineto{\pgfqpoint{0.463147in}{2.377027in}}%
\pgfpathlineto{\pgfqpoint{0.466368in}{2.377983in}}%
\pgfpathlineto{\pgfqpoint{0.469590in}{2.377661in}}%
\pgfpathlineto{\pgfqpoint{0.470664in}{2.376352in}}%
\pgfpathlineto{\pgfqpoint{0.471737in}{2.376235in}}%
\pgfpathlineto{\pgfqpoint{0.472811in}{2.374955in}}%
\pgfpathlineto{\pgfqpoint{0.473885in}{2.375184in}}%
\pgfpathlineto{\pgfqpoint{0.478180in}{2.373840in}}%
\pgfpathlineto{\pgfqpoint{0.480328in}{2.374433in}}%
\pgfpathlineto{\pgfqpoint{0.484623in}{2.374120in}}%
\pgfpathlineto{\pgfqpoint{0.486771in}{2.375143in}}%
\pgfpathlineto{\pgfqpoint{0.487845in}{2.374454in}}%
\pgfpathlineto{\pgfqpoint{0.488919in}{2.374655in}}%
\pgfpathlineto{\pgfqpoint{0.496436in}{2.371431in}}%
\pgfpathlineto{\pgfqpoint{0.499657in}{2.370882in}}%
\pgfpathlineto{\pgfqpoint{0.502879in}{2.372804in}}%
\pgfpathlineto{\pgfqpoint{0.507174in}{2.373755in}}%
\pgfpathlineto{\pgfqpoint{0.508248in}{2.372111in}}%
\pgfpathlineto{\pgfqpoint{0.510396in}{2.371997in}}%
\pgfpathlineto{\pgfqpoint{0.511469in}{2.370608in}}%
\pgfpathlineto{\pgfqpoint{0.515765in}{2.372168in}}%
\pgfpathlineto{\pgfqpoint{0.516839in}{2.369956in}}%
\pgfpathlineto{\pgfqpoint{0.517912in}{2.369446in}}%
\pgfpathlineto{\pgfqpoint{0.518986in}{2.370977in}}%
\pgfpathlineto{\pgfqpoint{0.522208in}{2.370211in}}%
\pgfpathlineto{\pgfqpoint{0.524355in}{2.372649in}}%
\pgfpathlineto{\pgfqpoint{0.526503in}{2.371716in}}%
\pgfpathlineto{\pgfqpoint{0.532946in}{2.370602in}}%
\pgfpathlineto{\pgfqpoint{0.534020in}{2.371925in}}%
\pgfpathlineto{\pgfqpoint{0.538315in}{2.371392in}}%
\pgfpathlineto{\pgfqpoint{0.539389in}{2.369803in}}%
\pgfpathlineto{\pgfqpoint{0.540463in}{2.370472in}}%
\pgfpathlineto{\pgfqpoint{0.541537in}{2.369943in}}%
\pgfpathlineto{\pgfqpoint{0.549054in}{2.367713in}}%
\pgfpathlineto{\pgfqpoint{0.553349in}{2.368103in}}%
\pgfpathlineto{\pgfqpoint{0.554423in}{2.367546in}}%
\pgfpathlineto{\pgfqpoint{0.555497in}{2.368549in}}%
\pgfpathlineto{\pgfqpoint{0.556571in}{2.367211in}}%
\pgfpathlineto{\pgfqpoint{0.560866in}{2.367378in}}%
\pgfpathlineto{\pgfqpoint{0.561940in}{2.366514in}}%
\pgfpathlineto{\pgfqpoint{0.563014in}{2.367016in}}%
\pgfpathlineto{\pgfqpoint{0.564088in}{2.365539in}}%
\pgfpathlineto{\pgfqpoint{0.567309in}{2.366821in}}%
\pgfpathlineto{\pgfqpoint{0.568383in}{2.366514in}}%
\pgfpathlineto{\pgfqpoint{0.569457in}{2.367852in}}%
\pgfpathlineto{\pgfqpoint{0.571604in}{2.368270in}}%
\pgfpathlineto{\pgfqpoint{0.574826in}{2.366319in}}%
\pgfpathlineto{\pgfqpoint{0.575900in}{2.368493in}}%
\pgfpathlineto{\pgfqpoint{0.578047in}{2.363210in}}%
\pgfpathlineto{\pgfqpoint{0.579121in}{2.362020in}}%
\pgfpathlineto{\pgfqpoint{0.582343in}{2.361435in}}%
\pgfpathlineto{\pgfqpoint{0.584490in}{2.360005in}}%
\pgfpathlineto{\pgfqpoint{0.585564in}{2.361215in}}%
\pgfpathlineto{\pgfqpoint{0.586638in}{2.360737in}}%
\pgfpathlineto{\pgfqpoint{0.590933in}{2.361162in}}%
\pgfpathlineto{\pgfqpoint{0.594155in}{2.358849in}}%
\pgfpathlineto{\pgfqpoint{0.598450in}{2.357903in}}%
\pgfpathlineto{\pgfqpoint{0.601672in}{2.358671in}}%
\pgfpathlineto{\pgfqpoint{0.608115in}{2.358548in}}%
\pgfpathlineto{\pgfqpoint{0.609189in}{2.356784in}}%
\pgfpathlineto{\pgfqpoint{0.612410in}{2.357090in}}%
\pgfpathlineto{\pgfqpoint{0.614558in}{2.355328in}}%
\pgfpathlineto{\pgfqpoint{0.615632in}{2.355029in}}%
\pgfpathlineto{\pgfqpoint{0.616706in}{2.356702in}}%
\pgfpathlineto{\pgfqpoint{0.619927in}{2.357689in}}%
\pgfpathlineto{\pgfqpoint{0.621001in}{2.359169in}}%
\pgfpathlineto{\pgfqpoint{0.622075in}{2.358803in}}%
\pgfpathlineto{\pgfqpoint{0.624222in}{2.354925in}}%
\pgfpathlineto{\pgfqpoint{0.629592in}{2.355198in}}%
\pgfpathlineto{\pgfqpoint{0.630666in}{2.356945in}}%
\pgfpathlineto{\pgfqpoint{0.631739in}{2.355646in}}%
\pgfpathlineto{\pgfqpoint{0.634961in}{2.356179in}}%
\pgfpathlineto{\pgfqpoint{0.636035in}{2.357229in}}%
\pgfpathlineto{\pgfqpoint{0.643552in}{2.356540in}}%
\pgfpathlineto{\pgfqpoint{0.646773in}{2.358149in}}%
\pgfpathlineto{\pgfqpoint{0.653216in}{2.358268in}}%
\pgfpathlineto{\pgfqpoint{0.654290in}{2.357353in}}%
\pgfpathlineto{\pgfqpoint{0.667176in}{2.357993in}}%
\pgfpathlineto{\pgfqpoint{0.668250in}{2.356865in}}%
\pgfpathlineto{\pgfqpoint{0.674693in}{2.356275in}}%
\pgfpathlineto{\pgfqpoint{0.675767in}{2.354662in}}%
\pgfpathlineto{\pgfqpoint{0.676841in}{2.355620in}}%
\pgfpathlineto{\pgfqpoint{0.680062in}{2.356037in}}%
\pgfpathlineto{\pgfqpoint{0.683284in}{2.354808in}}%
\pgfpathlineto{\pgfqpoint{0.684357in}{2.354511in}}%
\pgfpathlineto{\pgfqpoint{0.687579in}{2.354625in}}%
\pgfpathlineto{\pgfqpoint{0.688653in}{2.354009in}}%
\pgfpathlineto{\pgfqpoint{0.690800in}{2.354596in}}%
\pgfpathlineto{\pgfqpoint{0.691874in}{2.354778in}}%
\pgfpathlineto{\pgfqpoint{0.702613in}{2.353793in}}%
\pgfpathlineto{\pgfqpoint{0.703687in}{2.355685in}}%
\pgfpathlineto{\pgfqpoint{0.706908in}{2.356522in}}%
\pgfpathlineto{\pgfqpoint{0.710130in}{2.355324in}}%
\pgfpathlineto{\pgfqpoint{0.712277in}{2.350833in}}%
\pgfpathlineto{\pgfqpoint{0.713351in}{2.348144in}}%
\pgfpathlineto{\pgfqpoint{0.714425in}{2.349278in}}%
\pgfpathlineto{\pgfqpoint{0.717646in}{2.349404in}}%
\pgfpathlineto{\pgfqpoint{0.718720in}{2.350939in}}%
\pgfpathlineto{\pgfqpoint{0.719794in}{2.351220in}}%
\pgfpathlineto{\pgfqpoint{0.720868in}{2.350505in}}%
\pgfpathlineto{\pgfqpoint{0.721942in}{2.350933in}}%
\pgfpathlineto{\pgfqpoint{0.727311in}{2.351062in}}%
\pgfpathlineto{\pgfqpoint{0.728385in}{2.349853in}}%
\pgfpathlineto{\pgfqpoint{0.729459in}{2.350912in}}%
\pgfpathlineto{\pgfqpoint{0.733754in}{2.350717in}}%
\pgfpathlineto{\pgfqpoint{0.735902in}{2.353125in}}%
\pgfpathlineto{\pgfqpoint{0.736976in}{2.352724in}}%
\pgfpathlineto{\pgfqpoint{0.742345in}{2.353839in}}%
\pgfpathlineto{\pgfqpoint{0.743419in}{2.354201in}}%
\pgfpathlineto{\pgfqpoint{0.744492in}{2.353974in}}%
\pgfpathlineto{\pgfqpoint{0.747714in}{2.353861in}}%
\pgfpathlineto{\pgfqpoint{0.748788in}{2.353093in}}%
\pgfpathlineto{\pgfqpoint{0.749862in}{2.353226in}}%
\pgfpathlineto{\pgfqpoint{0.752009in}{2.352153in}}%
\pgfpathlineto{\pgfqpoint{0.756305in}{2.353523in}}%
\pgfpathlineto{\pgfqpoint{0.757378in}{2.352647in}}%
\pgfpathlineto{\pgfqpoint{0.758452in}{2.352758in}}%
\pgfpathlineto{\pgfqpoint{0.759526in}{2.351825in}}%
\pgfpathlineto{\pgfqpoint{0.763821in}{2.351583in}}%
\pgfpathlineto{\pgfqpoint{0.770265in}{2.350262in}}%
\pgfpathlineto{\pgfqpoint{0.771338in}{2.349387in}}%
\pgfpathlineto{\pgfqpoint{0.774560in}{2.350084in}}%
\pgfpathlineto{\pgfqpoint{0.778855in}{2.349637in}}%
\pgfpathlineto{\pgfqpoint{0.779929in}{2.350186in}}%
\pgfpathlineto{\pgfqpoint{0.781003in}{2.349995in}}%
\pgfpathlineto{\pgfqpoint{0.782077in}{2.350824in}}%
\pgfpathlineto{\pgfqpoint{0.785298in}{2.351276in}}%
\pgfpathlineto{\pgfqpoint{0.788520in}{2.351145in}}%
\pgfpathlineto{\pgfqpoint{0.789594in}{2.352250in}}%
\pgfpathlineto{\pgfqpoint{0.796037in}{2.349953in}}%
\pgfpathlineto{\pgfqpoint{0.797110in}{2.348529in}}%
\pgfpathlineto{\pgfqpoint{0.801406in}{2.348778in}}%
\pgfpathlineto{\pgfqpoint{0.804627in}{2.347165in}}%
\pgfpathlineto{\pgfqpoint{0.811070in}{2.346232in}}%
\pgfpathlineto{\pgfqpoint{0.812144in}{2.345692in}}%
\pgfpathlineto{\pgfqpoint{0.817513in}{2.346302in}}%
\pgfpathlineto{\pgfqpoint{0.818587in}{2.345902in}}%
\pgfpathlineto{\pgfqpoint{0.819661in}{2.344549in}}%
\pgfpathlineto{\pgfqpoint{0.822883in}{2.345035in}}%
\pgfpathlineto{\pgfqpoint{0.823956in}{2.343740in}}%
\pgfpathlineto{\pgfqpoint{0.826104in}{2.344530in}}%
\pgfpathlineto{\pgfqpoint{0.827178in}{2.344102in}}%
\pgfpathlineto{\pgfqpoint{0.830399in}{2.344218in}}%
\pgfpathlineto{\pgfqpoint{0.832547in}{2.342183in}}%
\pgfpathlineto{\pgfqpoint{0.833621in}{2.342688in}}%
\pgfpathlineto{\pgfqpoint{0.834695in}{2.342028in}}%
\pgfpathlineto{\pgfqpoint{0.837916in}{2.342140in}}%
\pgfpathlineto{\pgfqpoint{0.838990in}{2.341542in}}%
\pgfpathlineto{\pgfqpoint{0.841138in}{2.341523in}}%
\pgfpathlineto{\pgfqpoint{0.842212in}{2.340987in}}%
\pgfpathlineto{\pgfqpoint{0.846507in}{2.339777in}}%
\pgfpathlineto{\pgfqpoint{0.847581in}{2.340226in}}%
\pgfpathlineto{\pgfqpoint{0.849729in}{2.339918in}}%
\pgfpathlineto{\pgfqpoint{0.852950in}{2.340944in}}%
\pgfpathlineto{\pgfqpoint{0.854024in}{2.340669in}}%
\pgfpathlineto{\pgfqpoint{0.855098in}{2.339794in}}%
\pgfpathlineto{\pgfqpoint{0.856172in}{2.340117in}}%
\pgfpathlineto{\pgfqpoint{0.857245in}{2.339215in}}%
\pgfpathlineto{\pgfqpoint{0.872279in}{2.335612in}}%
\pgfpathlineto{\pgfqpoint{0.876575in}{2.336078in}}%
\pgfpathlineto{\pgfqpoint{0.877648in}{2.335255in}}%
\pgfpathlineto{\pgfqpoint{0.878722in}{2.335851in}}%
\pgfpathlineto{\pgfqpoint{0.879796in}{2.334830in}}%
\pgfpathlineto{\pgfqpoint{0.883018in}{2.334912in}}%
\pgfpathlineto{\pgfqpoint{0.884091in}{2.333315in}}%
\pgfpathlineto{\pgfqpoint{0.886239in}{2.332408in}}%
\pgfpathlineto{\pgfqpoint{0.890534in}{2.331873in}}%
\pgfpathlineto{\pgfqpoint{0.891608in}{2.330893in}}%
\pgfpathlineto{\pgfqpoint{0.892682in}{2.331687in}}%
\pgfpathlineto{\pgfqpoint{0.893756in}{2.331238in}}%
\pgfpathlineto{\pgfqpoint{0.898051in}{2.332925in}}%
\pgfpathlineto{\pgfqpoint{0.902347in}{2.330797in}}%
\pgfpathlineto{\pgfqpoint{0.905568in}{2.332109in}}%
\pgfpathlineto{\pgfqpoint{0.906642in}{2.329842in}}%
\pgfpathlineto{\pgfqpoint{0.907716in}{2.329272in}}%
\pgfpathlineto{\pgfqpoint{0.908790in}{2.330162in}}%
\pgfpathlineto{\pgfqpoint{0.909864in}{2.328518in}}%
\pgfpathlineto{\pgfqpoint{0.913085in}{2.328094in}}%
\pgfpathlineto{\pgfqpoint{0.914159in}{2.327340in}}%
\pgfpathlineto{\pgfqpoint{0.915233in}{2.328611in}}%
\pgfpathlineto{\pgfqpoint{0.916307in}{2.327601in}}%
\pgfpathlineto{\pgfqpoint{0.917380in}{2.327716in}}%
\pgfpathlineto{\pgfqpoint{0.920602in}{2.327169in}}%
\pgfpathlineto{\pgfqpoint{0.921676in}{2.327582in}}%
\pgfpathlineto{\pgfqpoint{0.922750in}{2.328773in}}%
\pgfpathlineto{\pgfqpoint{0.924897in}{2.326923in}}%
\pgfpathlineto{\pgfqpoint{0.928119in}{2.328199in}}%
\pgfpathlineto{\pgfqpoint{0.929193in}{2.327166in}}%
\pgfpathlineto{\pgfqpoint{0.932414in}{2.326876in}}%
\pgfpathlineto{\pgfqpoint{0.935636in}{2.326777in}}%
\pgfpathlineto{\pgfqpoint{0.937783in}{2.324676in}}%
\pgfpathlineto{\pgfqpoint{0.938857in}{2.324893in}}%
\pgfpathlineto{\pgfqpoint{0.939931in}{2.324171in}}%
\pgfpathlineto{\pgfqpoint{0.943153in}{2.324251in}}%
\pgfpathlineto{\pgfqpoint{0.944226in}{2.323606in}}%
\pgfpathlineto{\pgfqpoint{0.946374in}{2.324417in}}%
\pgfpathlineto{\pgfqpoint{0.947448in}{2.324849in}}%
\pgfpathlineto{\pgfqpoint{0.951743in}{2.323951in}}%
\pgfpathlineto{\pgfqpoint{0.952817in}{2.326157in}}%
\pgfpathlineto{\pgfqpoint{0.953891in}{2.325751in}}%
\pgfpathlineto{\pgfqpoint{0.954965in}{2.327874in}}%
\pgfpathlineto{\pgfqpoint{0.958186in}{2.327223in}}%
\pgfpathlineto{\pgfqpoint{0.960334in}{2.328473in}}%
\pgfpathlineto{\pgfqpoint{0.962482in}{2.326961in}}%
\pgfpathlineto{\pgfqpoint{0.965703in}{2.326691in}}%
\pgfpathlineto{\pgfqpoint{0.967851in}{2.328371in}}%
\pgfpathlineto{\pgfqpoint{0.968925in}{2.326924in}}%
\pgfpathlineto{\pgfqpoint{0.969998in}{2.326924in}}%
\pgfpathlineto{\pgfqpoint{0.973220in}{2.326072in}}%
\pgfpathlineto{\pgfqpoint{0.974294in}{2.325205in}}%
\pgfpathlineto{\pgfqpoint{0.976442in}{2.329624in}}%
\pgfpathlineto{\pgfqpoint{0.981811in}{2.326244in}}%
\pgfpathlineto{\pgfqpoint{0.982885in}{2.324306in}}%
\pgfpathlineto{\pgfqpoint{0.983958in}{2.324630in}}%
\pgfpathlineto{\pgfqpoint{0.985032in}{2.325596in}}%
\pgfpathlineto{\pgfqpoint{1.004361in}{2.320525in}}%
\pgfpathlineto{\pgfqpoint{1.006509in}{2.320762in}}%
\pgfpathlineto{\pgfqpoint{1.007583in}{2.318588in}}%
\pgfpathlineto{\pgfqpoint{1.015100in}{2.317978in}}%
\pgfpathlineto{\pgfqpoint{1.021543in}{2.317047in}}%
\pgfpathlineto{\pgfqpoint{1.022617in}{2.316443in}}%
\pgfpathlineto{\pgfqpoint{1.029060in}{2.317470in}}%
\pgfpathlineto{\pgfqpoint{1.030133in}{2.318431in}}%
\pgfpathlineto{\pgfqpoint{1.033355in}{2.318791in}}%
\pgfpathlineto{\pgfqpoint{1.034429in}{2.317765in}}%
\pgfpathlineto{\pgfqpoint{1.036576in}{2.321270in}}%
\pgfpathlineto{\pgfqpoint{1.037650in}{2.321461in}}%
\pgfpathlineto{\pgfqpoint{1.040872in}{2.320299in}}%
\pgfpathlineto{\pgfqpoint{1.044093in}{2.322630in}}%
\pgfpathlineto{\pgfqpoint{1.045167in}{2.321727in}}%
\pgfpathlineto{\pgfqpoint{1.048389in}{2.322691in}}%
\pgfpathlineto{\pgfqpoint{1.049463in}{2.324199in}}%
\pgfpathlineto{\pgfqpoint{1.051610in}{2.323753in}}%
\pgfpathlineto{\pgfqpoint{1.052684in}{2.323927in}}%
\pgfpathlineto{\pgfqpoint{1.058053in}{2.323375in}}%
\pgfpathlineto{\pgfqpoint{1.063422in}{2.322634in}}%
\pgfpathlineto{\pgfqpoint{1.065570in}{2.320790in}}%
\pgfpathlineto{\pgfqpoint{1.067718in}{2.321499in}}%
\pgfpathlineto{\pgfqpoint{1.072013in}{2.320962in}}%
\pgfpathlineto{\pgfqpoint{1.073087in}{2.320050in}}%
\pgfpathlineto{\pgfqpoint{1.074161in}{2.319889in}}%
\pgfpathlineto{\pgfqpoint{1.075235in}{2.320297in}}%
\pgfpathlineto{\pgfqpoint{1.078456in}{2.320921in}}%
\pgfpathlineto{\pgfqpoint{1.081678in}{2.323109in}}%
\pgfpathlineto{\pgfqpoint{1.082752in}{2.323493in}}%
\pgfpathlineto{\pgfqpoint{1.085973in}{2.323533in}}%
\pgfpathlineto{\pgfqpoint{1.087047in}{2.322652in}}%
\pgfpathlineto{\pgfqpoint{1.088121in}{2.322849in}}%
\pgfpathlineto{\pgfqpoint{1.089195in}{2.323652in}}%
\pgfpathlineto{\pgfqpoint{1.090268in}{2.322822in}}%
\pgfpathlineto{\pgfqpoint{1.093490in}{2.323625in}}%
\pgfpathlineto{\pgfqpoint{1.094564in}{2.324736in}}%
\pgfpathlineto{\pgfqpoint{1.095638in}{2.324325in}}%
\pgfpathlineto{\pgfqpoint{1.097785in}{2.320386in}}%
\pgfpathlineto{\pgfqpoint{1.102081in}{2.319886in}}%
\pgfpathlineto{\pgfqpoint{1.104228in}{2.317765in}}%
\pgfpathlineto{\pgfqpoint{1.108524in}{2.318538in}}%
\pgfpathlineto{\pgfqpoint{1.109597in}{2.317345in}}%
\pgfpathlineto{\pgfqpoint{1.112819in}{2.317614in}}%
\pgfpathlineto{\pgfqpoint{1.116041in}{2.317319in}}%
\pgfpathlineto{\pgfqpoint{1.117114in}{2.316567in}}%
\pgfpathlineto{\pgfqpoint{1.119262in}{2.317088in}}%
\pgfpathlineto{\pgfqpoint{1.120336in}{2.316341in}}%
\pgfpathlineto{\pgfqpoint{1.126779in}{2.316997in}}%
\pgfpathlineto{\pgfqpoint{1.127853in}{2.315656in}}%
\pgfpathlineto{\pgfqpoint{1.131074in}{2.315430in}}%
\pgfpathlineto{\pgfqpoint{1.133222in}{2.316334in}}%
\pgfpathlineto{\pgfqpoint{1.135370in}{2.315317in}}%
\pgfpathlineto{\pgfqpoint{1.138591in}{2.315407in}}%
\pgfpathlineto{\pgfqpoint{1.140739in}{2.314599in}}%
\pgfpathlineto{\pgfqpoint{1.141813in}{2.314555in}}%
\pgfpathlineto{\pgfqpoint{1.142886in}{2.313881in}}%
\pgfpathlineto{\pgfqpoint{1.146108in}{2.313522in}}%
\pgfpathlineto{\pgfqpoint{1.148256in}{2.314127in}}%
\pgfpathlineto{\pgfqpoint{1.155773in}{2.315400in}}%
\pgfpathlineto{\pgfqpoint{1.156846in}{2.316041in}}%
\pgfpathlineto{\pgfqpoint{1.157920in}{2.314602in}}%
\pgfpathlineto{\pgfqpoint{1.162216in}{2.314890in}}%
\pgfpathlineto{\pgfqpoint{1.163289in}{2.315870in}}%
\pgfpathlineto{\pgfqpoint{1.164363in}{2.317760in}}%
\pgfpathlineto{\pgfqpoint{1.168659in}{2.317546in}}%
\pgfpathlineto{\pgfqpoint{1.169732in}{2.318267in}}%
\pgfpathlineto{\pgfqpoint{1.170806in}{2.316239in}}%
\pgfpathlineto{\pgfqpoint{1.171880in}{2.316882in}}%
\pgfpathlineto{\pgfqpoint{1.180471in}{2.316519in}}%
\pgfpathlineto{\pgfqpoint{1.183692in}{2.316566in}}%
\pgfpathlineto{\pgfqpoint{1.184766in}{2.317272in}}%
\pgfpathlineto{\pgfqpoint{1.186914in}{2.317824in}}%
\pgfpathlineto{\pgfqpoint{1.187988in}{2.316991in}}%
\pgfpathlineto{\pgfqpoint{1.191209in}{2.316512in}}%
\pgfpathlineto{\pgfqpoint{1.192283in}{2.314581in}}%
\pgfpathlineto{\pgfqpoint{1.193357in}{2.314703in}}%
\pgfpathlineto{\pgfqpoint{1.195505in}{2.314148in}}%
\pgfpathlineto{\pgfqpoint{1.201948in}{2.314245in}}%
\pgfpathlineto{\pgfqpoint{1.203021in}{2.313849in}}%
\pgfpathlineto{\pgfqpoint{1.208391in}{2.314530in}}%
\pgfpathlineto{\pgfqpoint{1.210538in}{2.318109in}}%
\pgfpathlineto{\pgfqpoint{1.214834in}{2.318634in}}%
\pgfpathlineto{\pgfqpoint{1.215908in}{2.319870in}}%
\pgfpathlineto{\pgfqpoint{1.216981in}{2.319236in}}%
\pgfpathlineto{\pgfqpoint{1.218055in}{2.320317in}}%
\pgfpathlineto{\pgfqpoint{1.222351in}{2.322309in}}%
\pgfpathlineto{\pgfqpoint{1.223424in}{2.321575in}}%
\pgfpathlineto{\pgfqpoint{1.225572in}{2.318576in}}%
\pgfpathlineto{\pgfqpoint{1.228794in}{2.317509in}}%
\pgfpathlineto{\pgfqpoint{1.229867in}{2.315589in}}%
\pgfpathlineto{\pgfqpoint{1.230941in}{2.316123in}}%
\pgfpathlineto{\pgfqpoint{1.233089in}{2.315790in}}%
\pgfpathlineto{\pgfqpoint{1.243827in}{2.316756in}}%
\pgfpathlineto{\pgfqpoint{1.247049in}{2.316500in}}%
\pgfpathlineto{\pgfqpoint{1.248123in}{2.315746in}}%
\pgfpathlineto{\pgfqpoint{1.251344in}{2.316293in}}%
\pgfpathlineto{\pgfqpoint{1.252418in}{2.314526in}}%
\pgfpathlineto{\pgfqpoint{1.253492in}{2.315237in}}%
\pgfpathlineto{\pgfqpoint{1.255640in}{2.314529in}}%
\pgfpathlineto{\pgfqpoint{1.262083in}{2.314837in}}%
\pgfpathlineto{\pgfqpoint{1.263156in}{2.315016in}}%
\pgfpathlineto{\pgfqpoint{1.267452in}{2.313927in}}%
\pgfpathlineto{\pgfqpoint{1.268526in}{2.314256in}}%
\pgfpathlineto{\pgfqpoint{1.269599in}{2.313759in}}%
\pgfpathlineto{\pgfqpoint{1.270673in}{2.312055in}}%
\pgfpathlineto{\pgfqpoint{1.273895in}{2.312717in}}%
\pgfpathlineto{\pgfqpoint{1.274969in}{2.310711in}}%
\pgfpathlineto{\pgfqpoint{1.277116in}{2.310801in}}%
\pgfpathlineto{\pgfqpoint{1.285707in}{2.309804in}}%
\pgfpathlineto{\pgfqpoint{1.290002in}{2.310112in}}%
\pgfpathlineto{\pgfqpoint{1.291076in}{2.309338in}}%
\pgfpathlineto{\pgfqpoint{1.292150in}{2.311381in}}%
\pgfpathlineto{\pgfqpoint{1.293224in}{2.311091in}}%
\pgfpathlineto{\pgfqpoint{1.296445in}{2.310854in}}%
\pgfpathlineto{\pgfqpoint{1.297519in}{2.309043in}}%
\pgfpathlineto{\pgfqpoint{1.299667in}{2.309237in}}%
\pgfpathlineto{\pgfqpoint{1.306110in}{2.308177in}}%
\pgfpathlineto{\pgfqpoint{1.308258in}{2.308533in}}%
\pgfpathlineto{\pgfqpoint{1.311479in}{2.307244in}}%
\pgfpathlineto{\pgfqpoint{1.314701in}{2.307892in}}%
\pgfpathlineto{\pgfqpoint{1.315774in}{2.308894in}}%
\pgfpathlineto{\pgfqpoint{1.318996in}{2.308317in}}%
\pgfpathlineto{\pgfqpoint{1.320070in}{2.308722in}}%
\pgfpathlineto{\pgfqpoint{1.321144in}{2.307543in}}%
\pgfpathlineto{\pgfqpoint{1.322218in}{2.307884in}}%
\pgfpathlineto{\pgfqpoint{1.323291in}{2.307541in}}%
\pgfpathlineto{\pgfqpoint{1.330808in}{2.307812in}}%
\pgfpathlineto{\pgfqpoint{1.336177in}{2.307350in}}%
\pgfpathlineto{\pgfqpoint{1.338325in}{2.306907in}}%
\pgfpathlineto{\pgfqpoint{1.344768in}{2.307080in}}%
\pgfpathlineto{\pgfqpoint{1.345842in}{2.306509in}}%
\pgfpathlineto{\pgfqpoint{1.357654in}{2.304444in}}%
\pgfpathlineto{\pgfqpoint{1.360876in}{2.305646in}}%
\pgfpathlineto{\pgfqpoint{1.366245in}{2.305412in}}%
\pgfpathlineto{\pgfqpoint{1.368393in}{2.303519in}}%
\pgfpathlineto{\pgfqpoint{1.372688in}{2.304026in}}%
\pgfpathlineto{\pgfqpoint{1.373762in}{2.303149in}}%
\pgfpathlineto{\pgfqpoint{1.379131in}{2.303995in}}%
\pgfpathlineto{\pgfqpoint{1.380205in}{2.303048in}}%
\pgfpathlineto{\pgfqpoint{1.394165in}{2.303397in}}%
\pgfpathlineto{\pgfqpoint{1.397386in}{2.307225in}}%
\pgfpathlineto{\pgfqpoint{1.398460in}{2.306057in}}%
\pgfpathlineto{\pgfqpoint{1.401682in}{2.306478in}}%
\pgfpathlineto{\pgfqpoint{1.402755in}{2.305516in}}%
\pgfpathlineto{\pgfqpoint{1.405977in}{2.305797in}}%
\pgfpathlineto{\pgfqpoint{1.411346in}{2.305650in}}%
\pgfpathlineto{\pgfqpoint{1.412420in}{2.307382in}}%
\pgfpathlineto{\pgfqpoint{1.413494in}{2.307533in}}%
\pgfpathlineto{\pgfqpoint{1.425306in}{2.306913in}}%
\pgfpathlineto{\pgfqpoint{1.427454in}{2.305795in}}%
\pgfpathlineto{\pgfqpoint{1.428528in}{2.306460in}}%
\pgfpathlineto{\pgfqpoint{1.434971in}{2.304087in}}%
\pgfpathlineto{\pgfqpoint{1.436044in}{2.304384in}}%
\pgfpathlineto{\pgfqpoint{1.450004in}{2.303809in}}%
\pgfpathlineto{\pgfqpoint{1.451078in}{2.303375in}}%
\pgfpathlineto{\pgfqpoint{1.455373in}{2.303840in}}%
\pgfpathlineto{\pgfqpoint{1.456447in}{2.302938in}}%
\pgfpathlineto{\pgfqpoint{1.458595in}{2.303243in}}%
\pgfpathlineto{\pgfqpoint{1.461817in}{2.303140in}}%
\pgfpathlineto{\pgfqpoint{1.465038in}{2.301209in}}%
\pgfpathlineto{\pgfqpoint{1.466112in}{2.300756in}}%
\pgfpathlineto{\pgfqpoint{1.470407in}{2.301133in}}%
\pgfpathlineto{\pgfqpoint{1.471481in}{2.300299in}}%
\pgfpathlineto{\pgfqpoint{1.472555in}{2.301360in}}%
\pgfpathlineto{\pgfqpoint{1.477924in}{2.301726in}}%
\pgfpathlineto{\pgfqpoint{1.478998in}{2.303361in}}%
\pgfpathlineto{\pgfqpoint{1.480072in}{2.303696in}}%
\pgfpathlineto{\pgfqpoint{1.481146in}{2.302735in}}%
\pgfpathlineto{\pgfqpoint{1.484367in}{2.302930in}}%
\pgfpathlineto{\pgfqpoint{1.485441in}{2.304762in}}%
\pgfpathlineto{\pgfqpoint{1.486515in}{2.302805in}}%
\pgfpathlineto{\pgfqpoint{1.487589in}{2.304897in}}%
\pgfpathlineto{\pgfqpoint{1.491884in}{2.307230in}}%
\pgfpathlineto{\pgfqpoint{1.492958in}{2.308974in}}%
\pgfpathlineto{\pgfqpoint{1.494032in}{2.307935in}}%
\pgfpathlineto{\pgfqpoint{1.495106in}{2.309141in}}%
\pgfpathlineto{\pgfqpoint{1.496179in}{2.307483in}}%
\pgfpathlineto{\pgfqpoint{1.499401in}{2.307063in}}%
\pgfpathlineto{\pgfqpoint{1.503696in}{2.303912in}}%
\pgfpathlineto{\pgfqpoint{1.507992in}{2.302664in}}%
\pgfpathlineto{\pgfqpoint{1.511213in}{2.300494in}}%
\pgfpathlineto{\pgfqpoint{1.514435in}{2.300718in}}%
\pgfpathlineto{\pgfqpoint{1.516582in}{2.299764in}}%
\pgfpathlineto{\pgfqpoint{1.517656in}{2.299630in}}%
\pgfpathlineto{\pgfqpoint{1.518730in}{2.300185in}}%
\pgfpathlineto{\pgfqpoint{1.525173in}{2.299584in}}%
\pgfpathlineto{\pgfqpoint{1.526247in}{2.300208in}}%
\pgfpathlineto{\pgfqpoint{1.533764in}{2.299927in}}%
\pgfpathlineto{\pgfqpoint{1.538059in}{2.300725in}}%
\pgfpathlineto{\pgfqpoint{1.541281in}{2.299637in}}%
\pgfpathlineto{\pgfqpoint{1.547724in}{2.300111in}}%
\pgfpathlineto{\pgfqpoint{1.548797in}{2.299445in}}%
\pgfpathlineto{\pgfqpoint{1.553093in}{2.299764in}}%
\pgfpathlineto{\pgfqpoint{1.554167in}{2.301012in}}%
\pgfpathlineto{\pgfqpoint{1.555240in}{2.300671in}}%
\pgfpathlineto{\pgfqpoint{1.556314in}{2.302288in}}%
\pgfpathlineto{\pgfqpoint{1.559536in}{2.302632in}}%
\pgfpathlineto{\pgfqpoint{1.560610in}{2.303522in}}%
\pgfpathlineto{\pgfqpoint{1.561684in}{2.302525in}}%
\pgfpathlineto{\pgfqpoint{1.562757in}{2.300497in}}%
\pgfpathlineto{\pgfqpoint{1.563831in}{2.301379in}}%
\pgfpathlineto{\pgfqpoint{1.567053in}{2.300525in}}%
\pgfpathlineto{\pgfqpoint{1.568127in}{2.302259in}}%
\pgfpathlineto{\pgfqpoint{1.571348in}{2.301681in}}%
\pgfpathlineto{\pgfqpoint{1.575643in}{2.301466in}}%
\pgfpathlineto{\pgfqpoint{1.576717in}{2.302034in}}%
\pgfpathlineto{\pgfqpoint{1.582086in}{2.302611in}}%
\pgfpathlineto{\pgfqpoint{1.583160in}{2.302993in}}%
\pgfpathlineto{\pgfqpoint{1.584234in}{2.301271in}}%
\pgfpathlineto{\pgfqpoint{1.585308in}{2.300681in}}%
\pgfpathlineto{\pgfqpoint{1.586382in}{2.301705in}}%
\pgfpathlineto{\pgfqpoint{1.591751in}{2.302386in}}%
\pgfpathlineto{\pgfqpoint{1.592825in}{2.303509in}}%
\pgfpathlineto{\pgfqpoint{1.593899in}{2.302327in}}%
\pgfpathlineto{\pgfqpoint{1.598194in}{2.304359in}}%
\pgfpathlineto{\pgfqpoint{1.599268in}{2.303888in}}%
\pgfpathlineto{\pgfqpoint{1.600342in}{2.302447in}}%
\pgfpathlineto{\pgfqpoint{1.601416in}{2.303607in}}%
\pgfpathlineto{\pgfqpoint{1.605711in}{2.303695in}}%
\pgfpathlineto{\pgfqpoint{1.606785in}{2.304167in}}%
\pgfpathlineto{\pgfqpoint{1.607859in}{2.303468in}}%
\pgfpathlineto{\pgfqpoint{1.608932in}{2.305172in}}%
\pgfpathlineto{\pgfqpoint{1.612154in}{2.304627in}}%
\pgfpathlineto{\pgfqpoint{1.613228in}{2.303346in}}%
\pgfpathlineto{\pgfqpoint{1.614302in}{2.304178in}}%
\pgfpathlineto{\pgfqpoint{1.615375in}{2.303328in}}%
\pgfpathlineto{\pgfqpoint{1.616449in}{2.304360in}}%
\pgfpathlineto{\pgfqpoint{1.619671in}{2.305393in}}%
\pgfpathlineto{\pgfqpoint{1.621818in}{2.304908in}}%
\pgfpathlineto{\pgfqpoint{1.622892in}{2.306444in}}%
\pgfpathlineto{\pgfqpoint{1.623966in}{2.305477in}}%
\pgfpathlineto{\pgfqpoint{1.631483in}{2.304398in}}%
\pgfpathlineto{\pgfqpoint{1.635778in}{2.304049in}}%
\pgfpathlineto{\pgfqpoint{1.636852in}{2.303651in}}%
\pgfpathlineto{\pgfqpoint{1.637926in}{2.302417in}}%
\pgfpathlineto{\pgfqpoint{1.639000in}{2.302637in}}%
\pgfpathlineto{\pgfqpoint{1.642221in}{2.302100in}}%
\pgfpathlineto{\pgfqpoint{1.644369in}{2.303278in}}%
\pgfpathlineto{\pgfqpoint{1.645443in}{2.302610in}}%
\pgfpathlineto{\pgfqpoint{1.646517in}{2.304427in}}%
\pgfpathlineto{\pgfqpoint{1.649738in}{2.303998in}}%
\pgfpathlineto{\pgfqpoint{1.651886in}{2.305840in}}%
\pgfpathlineto{\pgfqpoint{1.652960in}{2.304608in}}%
\pgfpathlineto{\pgfqpoint{1.654034in}{2.305102in}}%
\pgfpathlineto{\pgfqpoint{1.657255in}{2.303611in}}%
\pgfpathlineto{\pgfqpoint{1.658329in}{2.304522in}}%
\pgfpathlineto{\pgfqpoint{1.659403in}{2.303275in}}%
\pgfpathlineto{\pgfqpoint{1.664772in}{2.302121in}}%
\pgfpathlineto{\pgfqpoint{1.667994in}{2.304271in}}%
\pgfpathlineto{\pgfqpoint{1.673363in}{2.303888in}}%
\pgfpathlineto{\pgfqpoint{1.674437in}{2.305019in}}%
\pgfpathlineto{\pgfqpoint{1.675510in}{2.304628in}}%
\pgfpathlineto{\pgfqpoint{1.679806in}{2.305006in}}%
\pgfpathlineto{\pgfqpoint{1.680880in}{2.304252in}}%
\pgfpathlineto{\pgfqpoint{1.681953in}{2.304207in}}%
\pgfpathlineto{\pgfqpoint{1.684101in}{2.302724in}}%
\pgfpathlineto{\pgfqpoint{1.688396in}{2.303891in}}%
\pgfpathlineto{\pgfqpoint{1.689470in}{2.303829in}}%
\pgfpathlineto{\pgfqpoint{1.691618in}{2.304626in}}%
\pgfpathlineto{\pgfqpoint{1.699135in}{2.303448in}}%
\pgfpathlineto{\pgfqpoint{1.704504in}{2.303982in}}%
\pgfpathlineto{\pgfqpoint{1.705578in}{2.304913in}}%
\pgfpathlineto{\pgfqpoint{1.706652in}{2.304160in}}%
\pgfpathlineto{\pgfqpoint{1.709873in}{2.303991in}}%
\pgfpathlineto{\pgfqpoint{1.712021in}{2.304917in}}%
\pgfpathlineto{\pgfqpoint{1.713095in}{2.304527in}}%
\pgfpathlineto{\pgfqpoint{1.714169in}{2.303088in}}%
\pgfpathlineto{\pgfqpoint{1.719538in}{2.303793in}}%
\pgfpathlineto{\pgfqpoint{1.721685in}{2.302443in}}%
\pgfpathlineto{\pgfqpoint{1.727055in}{2.301462in}}%
\pgfpathlineto{\pgfqpoint{1.733498in}{2.302985in}}%
\pgfpathlineto{\pgfqpoint{1.735645in}{2.302726in}}%
\pgfpathlineto{\pgfqpoint{1.736719in}{2.303499in}}%
\pgfpathlineto{\pgfqpoint{1.743162in}{2.304210in}}%
\pgfpathlineto{\pgfqpoint{1.744236in}{2.304719in}}%
\pgfpathlineto{\pgfqpoint{1.748531in}{2.305024in}}%
\pgfpathlineto{\pgfqpoint{1.750679in}{2.304189in}}%
\pgfpathlineto{\pgfqpoint{1.751753in}{2.304894in}}%
\pgfpathlineto{\pgfqpoint{1.754974in}{2.305594in}}%
\pgfpathlineto{\pgfqpoint{1.758196in}{2.303266in}}%
\pgfpathlineto{\pgfqpoint{1.759270in}{2.303658in}}%
\pgfpathlineto{\pgfqpoint{1.763565in}{2.303719in}}%
\pgfpathlineto{\pgfqpoint{1.765713in}{2.304241in}}%
\pgfpathlineto{\pgfqpoint{1.766787in}{2.303830in}}%
\pgfpathlineto{\pgfqpoint{1.771082in}{2.305554in}}%
\pgfpathlineto{\pgfqpoint{1.772156in}{2.304721in}}%
\pgfpathlineto{\pgfqpoint{1.780747in}{2.304757in}}%
\pgfpathlineto{\pgfqpoint{1.781820in}{2.303842in}}%
\pgfpathlineto{\pgfqpoint{1.785042in}{2.303257in}}%
\pgfpathlineto{\pgfqpoint{1.786116in}{2.303641in}}%
\pgfpathlineto{\pgfqpoint{1.788263in}{2.302608in}}%
\pgfpathlineto{\pgfqpoint{1.789337in}{2.303389in}}%
\pgfpathlineto{\pgfqpoint{1.795780in}{2.303300in}}%
\pgfpathlineto{\pgfqpoint{1.796854in}{2.304113in}}%
\pgfpathlineto{\pgfqpoint{1.800076in}{2.304809in}}%
\pgfpathlineto{\pgfqpoint{1.803297in}{2.303558in}}%
\pgfpathlineto{\pgfqpoint{1.804371in}{2.303268in}}%
\pgfpathlineto{\pgfqpoint{1.809740in}{2.303019in}}%
\pgfpathlineto{\pgfqpoint{1.810814in}{2.304139in}}%
\pgfpathlineto{\pgfqpoint{1.811888in}{2.304317in}}%
\pgfpathlineto{\pgfqpoint{1.815109in}{2.303564in}}%
\pgfpathlineto{\pgfqpoint{1.817257in}{2.304412in}}%
\pgfpathlineto{\pgfqpoint{1.819405in}{2.304330in}}%
\pgfpathlineto{\pgfqpoint{1.822626in}{2.303486in}}%
\pgfpathlineto{\pgfqpoint{1.825848in}{2.304335in}}%
\pgfpathlineto{\pgfqpoint{1.826922in}{2.306338in}}%
\pgfpathlineto{\pgfqpoint{1.830143in}{2.308662in}}%
\pgfpathlineto{\pgfqpoint{1.831217in}{2.310534in}}%
\pgfpathlineto{\pgfqpoint{1.832291in}{2.306420in}}%
\pgfpathlineto{\pgfqpoint{1.833365in}{2.305479in}}%
\pgfpathlineto{\pgfqpoint{1.834438in}{2.306352in}}%
\pgfpathlineto{\pgfqpoint{1.837660in}{2.307367in}}%
\pgfpathlineto{\pgfqpoint{1.838734in}{2.309012in}}%
\pgfpathlineto{\pgfqpoint{1.839808in}{2.307867in}}%
\pgfpathlineto{\pgfqpoint{1.841955in}{2.309689in}}%
\pgfpathlineto{\pgfqpoint{1.846251in}{2.307259in}}%
\pgfpathlineto{\pgfqpoint{1.847325in}{2.308756in}}%
\pgfpathlineto{\pgfqpoint{1.849472in}{2.308122in}}%
\pgfpathlineto{\pgfqpoint{1.852694in}{2.307775in}}%
\pgfpathlineto{\pgfqpoint{1.853768in}{2.306820in}}%
\pgfpathlineto{\pgfqpoint{1.855915in}{2.306436in}}%
\pgfpathlineto{\pgfqpoint{1.856989in}{2.307675in}}%
\pgfpathlineto{\pgfqpoint{1.863432in}{2.308485in}}%
\pgfpathlineto{\pgfqpoint{1.864506in}{2.309818in}}%
\pgfpathlineto{\pgfqpoint{1.867727in}{2.309476in}}%
\pgfpathlineto{\pgfqpoint{1.868801in}{2.307936in}}%
\pgfpathlineto{\pgfqpoint{1.875244in}{2.306427in}}%
\pgfpathlineto{\pgfqpoint{1.876318in}{2.307572in}}%
\pgfpathlineto{\pgfqpoint{1.877392in}{2.306198in}}%
\pgfpathlineto{\pgfqpoint{1.883835in}{2.305795in}}%
\pgfpathlineto{\pgfqpoint{1.884909in}{2.306565in}}%
\pgfpathlineto{\pgfqpoint{1.885983in}{2.304314in}}%
\pgfpathlineto{\pgfqpoint{1.887057in}{2.303426in}}%
\pgfpathlineto{\pgfqpoint{1.892426in}{2.303915in}}%
\pgfpathlineto{\pgfqpoint{1.894573in}{2.301786in}}%
\pgfpathlineto{\pgfqpoint{1.898869in}{2.301869in}}%
\pgfpathlineto{\pgfqpoint{1.908533in}{2.300250in}}%
\pgfpathlineto{\pgfqpoint{1.912829in}{2.301350in}}%
\pgfpathlineto{\pgfqpoint{1.914976in}{2.300589in}}%
\pgfpathlineto{\pgfqpoint{1.916050in}{2.301759in}}%
\pgfpathlineto{\pgfqpoint{1.917124in}{2.302056in}}%
\pgfpathlineto{\pgfqpoint{1.924641in}{2.299535in}}%
\pgfpathlineto{\pgfqpoint{1.927862in}{2.299327in}}%
\pgfpathlineto{\pgfqpoint{1.930010in}{2.299909in}}%
\pgfpathlineto{\pgfqpoint{1.932158in}{2.299609in}}%
\pgfpathlineto{\pgfqpoint{1.935379in}{2.300428in}}%
\pgfpathlineto{\pgfqpoint{1.936453in}{2.299599in}}%
\pgfpathlineto{\pgfqpoint{1.937527in}{2.299824in}}%
\pgfpathlineto{\pgfqpoint{1.938601in}{2.300745in}}%
\pgfpathlineto{\pgfqpoint{1.939675in}{2.299131in}}%
\pgfpathlineto{\pgfqpoint{1.942896in}{2.298861in}}%
\pgfpathlineto{\pgfqpoint{1.945044in}{2.299699in}}%
\pgfpathlineto{\pgfqpoint{1.946118in}{2.299344in}}%
\pgfpathlineto{\pgfqpoint{1.947192in}{2.300031in}}%
\pgfpathlineto{\pgfqpoint{1.950413in}{2.299689in}}%
\pgfpathlineto{\pgfqpoint{1.952561in}{2.297465in}}%
\pgfpathlineto{\pgfqpoint{1.954708in}{2.299761in}}%
\pgfpathlineto{\pgfqpoint{1.957930in}{2.300027in}}%
\pgfpathlineto{\pgfqpoint{1.960078in}{2.298641in}}%
\pgfpathlineto{\pgfqpoint{1.961151in}{2.298476in}}%
\pgfpathlineto{\pgfqpoint{1.965447in}{2.298827in}}%
\pgfpathlineto{\pgfqpoint{1.966521in}{2.298252in}}%
\pgfpathlineto{\pgfqpoint{1.968668in}{2.299171in}}%
\pgfpathlineto{\pgfqpoint{1.972964in}{2.300785in}}%
\pgfpathlineto{\pgfqpoint{1.974038in}{2.300469in}}%
\pgfpathlineto{\pgfqpoint{1.976185in}{2.301731in}}%
\pgfpathlineto{\pgfqpoint{1.977259in}{2.302549in}}%
\pgfpathlineto{\pgfqpoint{1.980481in}{2.303010in}}%
\pgfpathlineto{\pgfqpoint{1.981554in}{2.302473in}}%
\pgfpathlineto{\pgfqpoint{1.982628in}{2.303437in}}%
\pgfpathlineto{\pgfqpoint{1.983702in}{2.301932in}}%
\pgfpathlineto{\pgfqpoint{1.984776in}{2.303389in}}%
\pgfpathlineto{\pgfqpoint{1.989071in}{2.302988in}}%
\pgfpathlineto{\pgfqpoint{1.990145in}{2.304380in}}%
\pgfpathlineto{\pgfqpoint{1.995514in}{2.303847in}}%
\pgfpathlineto{\pgfqpoint{1.996588in}{2.299954in}}%
\pgfpathlineto{\pgfqpoint{1.999810in}{2.297582in}}%
\pgfpathlineto{\pgfqpoint{2.003031in}{2.297628in}}%
\pgfpathlineto{\pgfqpoint{2.004105in}{2.298295in}}%
\pgfpathlineto{\pgfqpoint{2.005179in}{2.297780in}}%
\pgfpathlineto{\pgfqpoint{2.006253in}{2.297949in}}%
\pgfpathlineto{\pgfqpoint{2.007326in}{2.300292in}}%
\pgfpathlineto{\pgfqpoint{2.011622in}{2.299228in}}%
\pgfpathlineto{\pgfqpoint{2.014843in}{2.299340in}}%
\pgfpathlineto{\pgfqpoint{2.020213in}{2.298842in}}%
\pgfpathlineto{\pgfqpoint{2.022360in}{2.297129in}}%
\pgfpathlineto{\pgfqpoint{2.025582in}{2.296720in}}%
\pgfpathlineto{\pgfqpoint{2.026656in}{2.297176in}}%
\pgfpathlineto{\pgfqpoint{2.028803in}{2.295608in}}%
\pgfpathlineto{\pgfqpoint{2.029877in}{2.296007in}}%
\pgfpathlineto{\pgfqpoint{2.033099in}{2.296389in}}%
\pgfpathlineto{\pgfqpoint{2.034172in}{2.295031in}}%
\pgfpathlineto{\pgfqpoint{2.041689in}{2.295684in}}%
\pgfpathlineto{\pgfqpoint{2.044911in}{2.294703in}}%
\pgfpathlineto{\pgfqpoint{2.050280in}{2.294894in}}%
\pgfpathlineto{\pgfqpoint{2.051354in}{2.295328in}}%
\pgfpathlineto{\pgfqpoint{2.052428in}{2.294831in}}%
\pgfpathlineto{\pgfqpoint{2.056723in}{2.294909in}}%
\pgfpathlineto{\pgfqpoint{2.057797in}{2.294183in}}%
\pgfpathlineto{\pgfqpoint{2.063166in}{2.294357in}}%
\pgfpathlineto{\pgfqpoint{2.064240in}{2.293776in}}%
\pgfpathlineto{\pgfqpoint{2.070683in}{2.294110in}}%
\pgfpathlineto{\pgfqpoint{2.073904in}{2.293679in}}%
\pgfpathlineto{\pgfqpoint{2.078200in}{2.293865in}}%
\pgfpathlineto{\pgfqpoint{2.080348in}{2.293301in}}%
\pgfpathlineto{\pgfqpoint{2.082495in}{2.293110in}}%
\pgfpathlineto{\pgfqpoint{2.085717in}{2.292651in}}%
\pgfpathlineto{\pgfqpoint{2.087864in}{2.291061in}}%
\pgfpathlineto{\pgfqpoint{2.095381in}{2.291526in}}%
\pgfpathlineto{\pgfqpoint{2.097529in}{2.291933in}}%
\pgfpathlineto{\pgfqpoint{2.101824in}{2.291574in}}%
\pgfpathlineto{\pgfqpoint{2.102898in}{2.291845in}}%
\pgfpathlineto{\pgfqpoint{2.105046in}{2.291538in}}%
\pgfpathlineto{\pgfqpoint{2.111489in}{2.290662in}}%
\pgfpathlineto{\pgfqpoint{2.112563in}{2.291046in}}%
\pgfpathlineto{\pgfqpoint{2.116858in}{2.290887in}}%
\pgfpathlineto{\pgfqpoint{2.123301in}{2.291385in}}%
\pgfpathlineto{\pgfqpoint{2.126523in}{2.290946in}}%
\pgfpathlineto{\pgfqpoint{2.127596in}{2.290851in}}%
\pgfpathlineto{\pgfqpoint{2.132966in}{2.291015in}}%
\pgfpathlineto{\pgfqpoint{2.134039in}{2.290018in}}%
\pgfpathlineto{\pgfqpoint{2.146926in}{2.288553in}}%
\pgfpathlineto{\pgfqpoint{2.150147in}{2.289446in}}%
\pgfpathlineto{\pgfqpoint{2.156590in}{2.288393in}}%
\pgfpathlineto{\pgfqpoint{2.157664in}{2.289337in}}%
\pgfpathlineto{\pgfqpoint{2.160885in}{2.288823in}}%
\pgfpathlineto{\pgfqpoint{2.164107in}{2.286305in}}%
\pgfpathlineto{\pgfqpoint{2.169476in}{2.285817in}}%
\pgfpathlineto{\pgfqpoint{2.172698in}{2.285532in}}%
\pgfpathlineto{\pgfqpoint{2.183436in}{2.285387in}}%
\pgfpathlineto{\pgfqpoint{2.184510in}{2.284361in}}%
\pgfpathlineto{\pgfqpoint{2.195248in}{2.284368in}}%
\pgfpathlineto{\pgfqpoint{2.199544in}{2.284542in}}%
\pgfpathlineto{\pgfqpoint{2.201691in}{2.285013in}}%
\pgfpathlineto{\pgfqpoint{2.202765in}{2.284829in}}%
\pgfpathlineto{\pgfqpoint{2.213503in}{2.285761in}}%
\pgfpathlineto{\pgfqpoint{2.214577in}{2.286742in}}%
\pgfpathlineto{\pgfqpoint{2.215651in}{2.286241in}}%
\pgfpathlineto{\pgfqpoint{2.217799in}{2.286536in}}%
\pgfpathlineto{\pgfqpoint{2.225316in}{2.286981in}}%
\pgfpathlineto{\pgfqpoint{2.229611in}{2.286759in}}%
\pgfpathlineto{\pgfqpoint{2.232833in}{2.286836in}}%
\pgfpathlineto{\pgfqpoint{2.239276in}{2.286757in}}%
\pgfpathlineto{\pgfqpoint{2.240349in}{2.287392in}}%
\pgfpathlineto{\pgfqpoint{2.243571in}{2.286903in}}%
\pgfpathlineto{\pgfqpoint{2.244645in}{2.287704in}}%
\pgfpathlineto{\pgfqpoint{2.247866in}{2.287363in}}%
\pgfpathlineto{\pgfqpoint{2.252162in}{2.287521in}}%
\pgfpathlineto{\pgfqpoint{2.254309in}{2.286731in}}%
\pgfpathlineto{\pgfqpoint{2.258605in}{2.287600in}}%
\pgfpathlineto{\pgfqpoint{2.260752in}{2.286749in}}%
\pgfpathlineto{\pgfqpoint{2.261826in}{2.287841in}}%
\pgfpathlineto{\pgfqpoint{2.262900in}{2.287379in}}%
\pgfpathlineto{\pgfqpoint{2.273638in}{2.286509in}}%
\pgfpathlineto{\pgfqpoint{2.274712in}{2.287616in}}%
\pgfpathlineto{\pgfqpoint{2.276860in}{2.287284in}}%
\pgfpathlineto{\pgfqpoint{2.277934in}{2.287653in}}%
\pgfpathlineto{\pgfqpoint{2.281155in}{2.287159in}}%
\pgfpathlineto{\pgfqpoint{2.282229in}{2.288765in}}%
\pgfpathlineto{\pgfqpoint{2.283303in}{2.289217in}}%
\pgfpathlineto{\pgfqpoint{2.284377in}{2.289060in}}%
\pgfpathlineto{\pgfqpoint{2.285451in}{2.289854in}}%
\pgfpathlineto{\pgfqpoint{2.290820in}{2.289222in}}%
\pgfpathlineto{\pgfqpoint{2.291894in}{2.288588in}}%
\pgfpathlineto{\pgfqpoint{2.292968in}{2.288789in}}%
\pgfpathlineto{\pgfqpoint{2.297263in}{2.288779in}}%
\pgfpathlineto{\pgfqpoint{2.299411in}{2.288945in}}%
\pgfpathlineto{\pgfqpoint{2.300484in}{2.288904in}}%
\pgfpathlineto{\pgfqpoint{2.304780in}{2.287838in}}%
\pgfpathlineto{\pgfqpoint{2.305854in}{2.286097in}}%
\pgfpathlineto{\pgfqpoint{2.312297in}{2.288156in}}%
\pgfpathlineto{\pgfqpoint{2.318740in}{2.288442in}}%
\pgfpathlineto{\pgfqpoint{2.319814in}{2.289682in}}%
\pgfpathlineto{\pgfqpoint{2.323035in}{2.288893in}}%
\pgfpathlineto{\pgfqpoint{2.327330in}{2.289817in}}%
\pgfpathlineto{\pgfqpoint{2.328404in}{2.290495in}}%
\pgfpathlineto{\pgfqpoint{2.335921in}{2.290602in}}%
\pgfpathlineto{\pgfqpoint{2.336995in}{2.290672in}}%
\pgfpathlineto{\pgfqpoint{2.341290in}{2.288135in}}%
\pgfpathlineto{\pgfqpoint{2.342364in}{2.287848in}}%
\pgfpathlineto{\pgfqpoint{2.343438in}{2.288335in}}%
\pgfpathlineto{\pgfqpoint{2.345586in}{2.287846in}}%
\pgfpathlineto{\pgfqpoint{2.357398in}{2.287826in}}%
\pgfpathlineto{\pgfqpoint{2.358472in}{2.288267in}}%
\pgfpathlineto{\pgfqpoint{2.360619in}{2.288201in}}%
\pgfpathlineto{\pgfqpoint{2.372432in}{2.287678in}}%
\pgfpathlineto{\pgfqpoint{2.373505in}{2.288446in}}%
\pgfpathlineto{\pgfqpoint{2.386391in}{2.288898in}}%
\pgfpathlineto{\pgfqpoint{2.387465in}{2.290098in}}%
\pgfpathlineto{\pgfqpoint{2.388539in}{2.289494in}}%
\pgfpathlineto{\pgfqpoint{2.389613in}{2.290044in}}%
\pgfpathlineto{\pgfqpoint{2.390687in}{2.289148in}}%
\pgfpathlineto{\pgfqpoint{2.401425in}{2.289140in}}%
\pgfpathlineto{\pgfqpoint{2.404647in}{2.288753in}}%
\pgfpathlineto{\pgfqpoint{2.405721in}{2.288113in}}%
\pgfpathlineto{\pgfqpoint{2.411090in}{2.287052in}}%
\pgfpathlineto{\pgfqpoint{2.413237in}{2.286185in}}%
\pgfpathlineto{\pgfqpoint{2.418607in}{2.285848in}}%
\pgfpathlineto{\pgfqpoint{2.420754in}{2.283844in}}%
\pgfpathlineto{\pgfqpoint{2.425050in}{2.284094in}}%
\pgfpathlineto{\pgfqpoint{2.426124in}{2.283298in}}%
\pgfpathlineto{\pgfqpoint{2.431493in}{2.283367in}}%
\pgfpathlineto{\pgfqpoint{2.433640in}{2.283187in}}%
\pgfpathlineto{\pgfqpoint{2.434714in}{2.282325in}}%
\pgfpathlineto{\pgfqpoint{2.440083in}{2.281830in}}%
\pgfpathlineto{\pgfqpoint{2.441157in}{2.280979in}}%
\pgfpathlineto{\pgfqpoint{2.443305in}{2.281368in}}%
\pgfpathlineto{\pgfqpoint{2.447600in}{2.281723in}}%
\pgfpathlineto{\pgfqpoint{2.450822in}{2.282516in}}%
\pgfpathlineto{\pgfqpoint{2.456191in}{2.282768in}}%
\pgfpathlineto{\pgfqpoint{2.461560in}{2.282876in}}%
\pgfpathlineto{\pgfqpoint{2.476594in}{2.282392in}}%
\pgfpathlineto{\pgfqpoint{2.477668in}{2.284163in}}%
\pgfpathlineto{\pgfqpoint{2.479815in}{2.284140in}}%
\pgfpathlineto{\pgfqpoint{2.480889in}{2.284192in}}%
\pgfpathlineto{\pgfqpoint{2.488406in}{2.283370in}}%
\pgfpathlineto{\pgfqpoint{2.503440in}{2.283284in}}%
\pgfpathlineto{\pgfqpoint{2.507735in}{2.281369in}}%
\pgfpathlineto{\pgfqpoint{2.508809in}{2.281847in}}%
\pgfpathlineto{\pgfqpoint{2.515252in}{2.281472in}}%
\pgfpathlineto{\pgfqpoint{2.522769in}{2.281276in}}%
\pgfpathlineto{\pgfqpoint{2.525991in}{2.279991in}}%
\pgfpathlineto{\pgfqpoint{2.537803in}{2.279182in}}%
\pgfpathlineto{\pgfqpoint{2.541024in}{2.278243in}}%
\pgfpathlineto{\pgfqpoint{2.546393in}{2.278020in}}%
\pgfpathlineto{\pgfqpoint{2.548541in}{2.277430in}}%
\pgfpathlineto{\pgfqpoint{2.552836in}{2.277974in}}%
\pgfpathlineto{\pgfqpoint{2.556058in}{2.279047in}}%
\pgfpathlineto{\pgfqpoint{2.563575in}{2.278937in}}%
\pgfpathlineto{\pgfqpoint{2.570018in}{2.279207in}}%
\pgfpathlineto{\pgfqpoint{2.571092in}{2.278903in}}%
\pgfpathlineto{\pgfqpoint{2.574313in}{2.279087in}}%
\pgfpathlineto{\pgfqpoint{2.576461in}{2.277854in}}%
\pgfpathlineto{\pgfqpoint{2.577535in}{2.277326in}}%
\pgfpathlineto{\pgfqpoint{2.582904in}{2.279148in}}%
\pgfpathlineto{\pgfqpoint{2.585052in}{2.279578in}}%
\pgfpathlineto{\pgfqpoint{2.586125in}{2.279155in}}%
\pgfpathlineto{\pgfqpoint{2.593642in}{2.278611in}}%
\pgfpathlineto{\pgfqpoint{2.605455in}{2.278526in}}%
\pgfpathlineto{\pgfqpoint{2.606528in}{2.278213in}}%
\pgfpathlineto{\pgfqpoint{2.608676in}{2.278822in}}%
\pgfpathlineto{\pgfqpoint{2.615119in}{2.278698in}}%
\pgfpathlineto{\pgfqpoint{2.619414in}{2.278826in}}%
\pgfpathlineto{\pgfqpoint{2.620488in}{2.278633in}}%
\pgfpathlineto{\pgfqpoint{2.621562in}{2.279092in}}%
\pgfpathlineto{\pgfqpoint{2.622636in}{2.278560in}}%
\pgfpathlineto{\pgfqpoint{2.623710in}{2.279099in}}%
\pgfpathlineto{\pgfqpoint{2.629079in}{2.279241in}}%
\pgfpathlineto{\pgfqpoint{2.630153in}{2.278614in}}%
\pgfpathlineto{\pgfqpoint{2.631227in}{2.279103in}}%
\pgfpathlineto{\pgfqpoint{2.635522in}{2.278415in}}%
\pgfpathlineto{\pgfqpoint{2.636596in}{2.278487in}}%
\pgfpathlineto{\pgfqpoint{2.638744in}{2.277689in}}%
\pgfpathlineto{\pgfqpoint{2.643039in}{2.277388in}}%
\pgfpathlineto{\pgfqpoint{2.645187in}{2.278742in}}%
\pgfpathlineto{\pgfqpoint{2.653777in}{2.279451in}}%
\pgfpathlineto{\pgfqpoint{2.665590in}{2.277867in}}%
\pgfpathlineto{\pgfqpoint{2.666663in}{2.276786in}}%
\pgfpathlineto{\pgfqpoint{2.672033in}{2.276985in}}%
\pgfpathlineto{\pgfqpoint{2.673106in}{2.275215in}}%
\pgfpathlineto{\pgfqpoint{2.675254in}{2.274770in}}%
\pgfpathlineto{\pgfqpoint{2.679549in}{2.274225in}}%
\pgfpathlineto{\pgfqpoint{2.680623in}{2.274896in}}%
\pgfpathlineto{\pgfqpoint{2.682771in}{2.274835in}}%
\pgfpathlineto{\pgfqpoint{2.683845in}{2.274846in}}%
\pgfpathlineto{\pgfqpoint{2.689214in}{2.275475in}}%
\pgfpathlineto{\pgfqpoint{2.695657in}{2.275550in}}%
\pgfpathlineto{\pgfqpoint{2.696731in}{2.274993in}}%
\pgfpathlineto{\pgfqpoint{2.698879in}{2.275612in}}%
\pgfpathlineto{\pgfqpoint{2.705322in}{2.275864in}}%
\pgfpathlineto{\pgfqpoint{2.706395in}{2.276179in}}%
\pgfpathlineto{\pgfqpoint{2.717134in}{2.275828in}}%
\pgfpathlineto{\pgfqpoint{2.718208in}{2.275126in}}%
\pgfpathlineto{\pgfqpoint{2.725724in}{2.275244in}}%
\pgfpathlineto{\pgfqpoint{2.726798in}{2.274746in}}%
\pgfpathlineto{\pgfqpoint{2.728946in}{2.274908in}}%
\pgfpathlineto{\pgfqpoint{2.740758in}{2.274481in}}%
\pgfpathlineto{\pgfqpoint{2.743980in}{2.275059in}}%
\pgfpathlineto{\pgfqpoint{2.751497in}{2.275198in}}%
\pgfpathlineto{\pgfqpoint{2.751497in}{2.275198in}}%
\pgfusepath{stroke}%
\end{pgfscope}%
\begin{pgfscope}%
\pgfsetrectcap%
\pgfsetmiterjoin%
\pgfsetlinewidth{0.803000pt}%
\definecolor{currentstroke}{rgb}{1.000000,1.000000,1.000000}%
\pgfsetstrokecolor{currentstroke}%
\pgfsetdash{}{0pt}%
\pgfpathmoveto{\pgfqpoint{0.285588in}{2.255883in}}%
\pgfpathlineto{\pgfqpoint{0.285588in}{2.656768in}}%
\pgfusepath{stroke}%
\end{pgfscope}%
\begin{pgfscope}%
\pgfsetrectcap%
\pgfsetmiterjoin%
\pgfsetlinewidth{0.803000pt}%
\definecolor{currentstroke}{rgb}{1.000000,1.000000,1.000000}%
\pgfsetstrokecolor{currentstroke}%
\pgfsetdash{}{0pt}%
\pgfpathmoveto{\pgfqpoint{2.868921in}{2.255883in}}%
\pgfpathlineto{\pgfqpoint{2.868921in}{2.656768in}}%
\pgfusepath{stroke}%
\end{pgfscope}%
\begin{pgfscope}%
\pgfsetrectcap%
\pgfsetmiterjoin%
\pgfsetlinewidth{0.803000pt}%
\definecolor{currentstroke}{rgb}{1.000000,1.000000,1.000000}%
\pgfsetstrokecolor{currentstroke}%
\pgfsetdash{}{0pt}%
\pgfpathmoveto{\pgfqpoint{0.285587in}{2.255883in}}%
\pgfpathlineto{\pgfqpoint{2.868921in}{2.255883in}}%
\pgfusepath{stroke}%
\end{pgfscope}%
\begin{pgfscope}%
\pgfsetrectcap%
\pgfsetmiterjoin%
\pgfsetlinewidth{0.803000pt}%
\definecolor{currentstroke}{rgb}{1.000000,1.000000,1.000000}%
\pgfsetstrokecolor{currentstroke}%
\pgfsetdash{}{0pt}%
\pgfpathmoveto{\pgfqpoint{0.285587in}{2.656768in}}%
\pgfpathlineto{\pgfqpoint{2.868921in}{2.656768in}}%
\pgfusepath{stroke}%
\end{pgfscope}%
\begin{pgfscope}%
\definecolor{textcolor}{rgb}{0.150000,0.150000,0.150000}%
\pgfsetstrokecolor{textcolor}%
\pgfsetfillcolor{textcolor}%
\pgftext[x=1.577254in,y=2.740101in,,base]{\color{textcolor}\rmfamily\fontsize{12.000000}{14.400000}\selectfont JNJ}%
\end{pgfscope}%
\begin{pgfscope}%
\pgfsetbuttcap%
\pgfsetmiterjoin%
\definecolor{currentfill}{rgb}{0.917647,0.917647,0.949020}%
\pgfsetfillcolor{currentfill}%
\pgfsetlinewidth{0.000000pt}%
\definecolor{currentstroke}{rgb}{0.000000,0.000000,0.000000}%
\pgfsetstrokecolor{currentstroke}%
\pgfsetstrokeopacity{0.000000}%
\pgfsetdash{}{0pt}%
\pgfpathmoveto{\pgfqpoint{3.902254in}{2.255883in}}%
\pgfpathlineto{\pgfqpoint{6.485588in}{2.255883in}}%
\pgfpathlineto{\pgfqpoint{6.485588in}{2.656768in}}%
\pgfpathlineto{\pgfqpoint{3.902254in}{2.656768in}}%
\pgfpathclose%
\pgfusepath{fill}%
\end{pgfscope}%
\begin{pgfscope}%
\pgfpathrectangle{\pgfqpoint{3.902254in}{2.255883in}}{\pgfqpoint{2.583333in}{0.400885in}}%
\pgfusepath{clip}%
\pgfsetroundcap%
\pgfsetroundjoin%
\pgfsetlinewidth{0.803000pt}%
\definecolor{currentstroke}{rgb}{1.000000,1.000000,1.000000}%
\pgfsetstrokecolor{currentstroke}%
\pgfsetdash{}{0pt}%
\pgfpathmoveto{\pgfqpoint{4.017531in}{2.255883in}}%
\pgfpathlineto{\pgfqpoint{4.017531in}{2.656768in}}%
\pgfusepath{stroke}%
\end{pgfscope}%
\begin{pgfscope}%
\definecolor{textcolor}{rgb}{0.150000,0.150000,0.150000}%
\pgfsetstrokecolor{textcolor}%
\pgfsetfillcolor{textcolor}%
\pgftext[x=4.017531in,y=2.158661in,,top]{\color{textcolor}\rmfamily\fontsize{10.000000}{12.000000}\selectfont 2012}%
\end{pgfscope}%
\begin{pgfscope}%
\pgfpathrectangle{\pgfqpoint{3.902254in}{2.255883in}}{\pgfqpoint{2.583333in}{0.400885in}}%
\pgfusepath{clip}%
\pgfsetroundcap%
\pgfsetroundjoin%
\pgfsetlinewidth{0.803000pt}%
\definecolor{currentstroke}{rgb}{1.000000,1.000000,1.000000}%
\pgfsetstrokecolor{currentstroke}%
\pgfsetdash{}{0pt}%
\pgfpathmoveto{\pgfqpoint{4.410556in}{2.255883in}}%
\pgfpathlineto{\pgfqpoint{4.410556in}{2.656768in}}%
\pgfusepath{stroke}%
\end{pgfscope}%
\begin{pgfscope}%
\definecolor{textcolor}{rgb}{0.150000,0.150000,0.150000}%
\pgfsetstrokecolor{textcolor}%
\pgfsetfillcolor{textcolor}%
\pgftext[x=4.410556in,y=2.158661in,,top]{\color{textcolor}\rmfamily\fontsize{10.000000}{12.000000}\selectfont 2013}%
\end{pgfscope}%
\begin{pgfscope}%
\pgfpathrectangle{\pgfqpoint{3.902254in}{2.255883in}}{\pgfqpoint{2.583333in}{0.400885in}}%
\pgfusepath{clip}%
\pgfsetroundcap%
\pgfsetroundjoin%
\pgfsetlinewidth{0.803000pt}%
\definecolor{currentstroke}{rgb}{1.000000,1.000000,1.000000}%
\pgfsetstrokecolor{currentstroke}%
\pgfsetdash{}{0pt}%
\pgfpathmoveto{\pgfqpoint{4.802507in}{2.255883in}}%
\pgfpathlineto{\pgfqpoint{4.802507in}{2.656768in}}%
\pgfusepath{stroke}%
\end{pgfscope}%
\begin{pgfscope}%
\definecolor{textcolor}{rgb}{0.150000,0.150000,0.150000}%
\pgfsetstrokecolor{textcolor}%
\pgfsetfillcolor{textcolor}%
\pgftext[x=4.802507in,y=2.158661in,,top]{\color{textcolor}\rmfamily\fontsize{10.000000}{12.000000}\selectfont 2014}%
\end{pgfscope}%
\begin{pgfscope}%
\pgfpathrectangle{\pgfqpoint{3.902254in}{2.255883in}}{\pgfqpoint{2.583333in}{0.400885in}}%
\pgfusepath{clip}%
\pgfsetroundcap%
\pgfsetroundjoin%
\pgfsetlinewidth{0.803000pt}%
\definecolor{currentstroke}{rgb}{1.000000,1.000000,1.000000}%
\pgfsetstrokecolor{currentstroke}%
\pgfsetdash{}{0pt}%
\pgfpathmoveto{\pgfqpoint{5.194458in}{2.255883in}}%
\pgfpathlineto{\pgfqpoint{5.194458in}{2.656768in}}%
\pgfusepath{stroke}%
\end{pgfscope}%
\begin{pgfscope}%
\definecolor{textcolor}{rgb}{0.150000,0.150000,0.150000}%
\pgfsetstrokecolor{textcolor}%
\pgfsetfillcolor{textcolor}%
\pgftext[x=5.194458in,y=2.158661in,,top]{\color{textcolor}\rmfamily\fontsize{10.000000}{12.000000}\selectfont 2015}%
\end{pgfscope}%
\begin{pgfscope}%
\pgfpathrectangle{\pgfqpoint{3.902254in}{2.255883in}}{\pgfqpoint{2.583333in}{0.400885in}}%
\pgfusepath{clip}%
\pgfsetroundcap%
\pgfsetroundjoin%
\pgfsetlinewidth{0.803000pt}%
\definecolor{currentstroke}{rgb}{1.000000,1.000000,1.000000}%
\pgfsetstrokecolor{currentstroke}%
\pgfsetdash{}{0pt}%
\pgfpathmoveto{\pgfqpoint{5.586409in}{2.255883in}}%
\pgfpathlineto{\pgfqpoint{5.586409in}{2.656768in}}%
\pgfusepath{stroke}%
\end{pgfscope}%
\begin{pgfscope}%
\definecolor{textcolor}{rgb}{0.150000,0.150000,0.150000}%
\pgfsetstrokecolor{textcolor}%
\pgfsetfillcolor{textcolor}%
\pgftext[x=5.586409in,y=2.158661in,,top]{\color{textcolor}\rmfamily\fontsize{10.000000}{12.000000}\selectfont 2016}%
\end{pgfscope}%
\begin{pgfscope}%
\pgfpathrectangle{\pgfqpoint{3.902254in}{2.255883in}}{\pgfqpoint{2.583333in}{0.400885in}}%
\pgfusepath{clip}%
\pgfsetroundcap%
\pgfsetroundjoin%
\pgfsetlinewidth{0.803000pt}%
\definecolor{currentstroke}{rgb}{1.000000,1.000000,1.000000}%
\pgfsetstrokecolor{currentstroke}%
\pgfsetdash{}{0pt}%
\pgfpathmoveto{\pgfqpoint{5.979434in}{2.255883in}}%
\pgfpathlineto{\pgfqpoint{5.979434in}{2.656768in}}%
\pgfusepath{stroke}%
\end{pgfscope}%
\begin{pgfscope}%
\definecolor{textcolor}{rgb}{0.150000,0.150000,0.150000}%
\pgfsetstrokecolor{textcolor}%
\pgfsetfillcolor{textcolor}%
\pgftext[x=5.979434in,y=2.158661in,,top]{\color{textcolor}\rmfamily\fontsize{10.000000}{12.000000}\selectfont 2017}%
\end{pgfscope}%
\begin{pgfscope}%
\pgfpathrectangle{\pgfqpoint{3.902254in}{2.255883in}}{\pgfqpoint{2.583333in}{0.400885in}}%
\pgfusepath{clip}%
\pgfsetroundcap%
\pgfsetroundjoin%
\pgfsetlinewidth{0.803000pt}%
\definecolor{currentstroke}{rgb}{1.000000,1.000000,1.000000}%
\pgfsetstrokecolor{currentstroke}%
\pgfsetdash{}{0pt}%
\pgfpathmoveto{\pgfqpoint{6.371385in}{2.255883in}}%
\pgfpathlineto{\pgfqpoint{6.371385in}{2.656768in}}%
\pgfusepath{stroke}%
\end{pgfscope}%
\begin{pgfscope}%
\definecolor{textcolor}{rgb}{0.150000,0.150000,0.150000}%
\pgfsetstrokecolor{textcolor}%
\pgfsetfillcolor{textcolor}%
\pgftext[x=6.371385in,y=2.158661in,,top]{\color{textcolor}\rmfamily\fontsize{10.000000}{12.000000}\selectfont 2018}%
\end{pgfscope}%
\begin{pgfscope}%
\pgfpathrectangle{\pgfqpoint{3.902254in}{2.255883in}}{\pgfqpoint{2.583333in}{0.400885in}}%
\pgfusepath{clip}%
\pgfsetroundcap%
\pgfsetroundjoin%
\pgfsetlinewidth{0.803000pt}%
\definecolor{currentstroke}{rgb}{1.000000,1.000000,1.000000}%
\pgfsetstrokecolor{currentstroke}%
\pgfsetdash{}{0pt}%
\pgfpathmoveto{\pgfqpoint{3.902254in}{2.298011in}}%
\pgfpathlineto{\pgfqpoint{6.485588in}{2.298011in}}%
\pgfusepath{stroke}%
\end{pgfscope}%
\begin{pgfscope}%
\definecolor{textcolor}{rgb}{0.150000,0.150000,0.150000}%
\pgfsetstrokecolor{textcolor}%
\pgfsetfillcolor{textcolor}%
\pgftext[x=3.584153in,y=2.245250in,left,base]{\color{textcolor}\rmfamily\fontsize{10.000000}{12.000000}\selectfont 0.5}%
\end{pgfscope}%
\begin{pgfscope}%
\pgfpathrectangle{\pgfqpoint{3.902254in}{2.255883in}}{\pgfqpoint{2.583333in}{0.400885in}}%
\pgfusepath{clip}%
\pgfsetroundcap%
\pgfsetroundjoin%
\pgfsetlinewidth{0.803000pt}%
\definecolor{currentstroke}{rgb}{1.000000,1.000000,1.000000}%
\pgfsetstrokecolor{currentstroke}%
\pgfsetdash{}{0pt}%
\pgfpathmoveto{\pgfqpoint{3.902254in}{2.440434in}}%
\pgfpathlineto{\pgfqpoint{6.485588in}{2.440434in}}%
\pgfusepath{stroke}%
\end{pgfscope}%
\begin{pgfscope}%
\definecolor{textcolor}{rgb}{0.150000,0.150000,0.150000}%
\pgfsetstrokecolor{textcolor}%
\pgfsetfillcolor{textcolor}%
\pgftext[x=3.584153in,y=2.387673in,left,base]{\color{textcolor}\rmfamily\fontsize{10.000000}{12.000000}\selectfont 1.0}%
\end{pgfscope}%
\begin{pgfscope}%
\pgfpathrectangle{\pgfqpoint{3.902254in}{2.255883in}}{\pgfqpoint{2.583333in}{0.400885in}}%
\pgfusepath{clip}%
\pgfsetroundcap%
\pgfsetroundjoin%
\pgfsetlinewidth{0.803000pt}%
\definecolor{currentstroke}{rgb}{1.000000,1.000000,1.000000}%
\pgfsetstrokecolor{currentstroke}%
\pgfsetdash{}{0pt}%
\pgfpathmoveto{\pgfqpoint{3.902254in}{2.582858in}}%
\pgfpathlineto{\pgfqpoint{6.485588in}{2.582858in}}%
\pgfusepath{stroke}%
\end{pgfscope}%
\begin{pgfscope}%
\definecolor{textcolor}{rgb}{0.150000,0.150000,0.150000}%
\pgfsetstrokecolor{textcolor}%
\pgfsetfillcolor{textcolor}%
\pgftext[x=3.584153in,y=2.530096in,left,base]{\color{textcolor}\rmfamily\fontsize{10.000000}{12.000000}\selectfont 1.5}%
\end{pgfscope}%
\begin{pgfscope}%
\pgfpathrectangle{\pgfqpoint{3.902254in}{2.255883in}}{\pgfqpoint{2.583333in}{0.400885in}}%
\pgfusepath{clip}%
\pgfsetroundcap%
\pgfsetroundjoin%
\pgfsetlinewidth{1.505625pt}%
\definecolor{currentstroke}{rgb}{0.549020,0.337255,0.294118}%
\pgfsetstrokecolor{currentstroke}%
\pgfsetstrokeopacity{0.300000}%
\pgfsetdash{}{0pt}%
\pgfpathmoveto{\pgfqpoint{4.019678in}{2.440434in}}%
\pgfpathlineto{\pgfqpoint{4.020752in}{2.440271in}}%
\pgfpathlineto{\pgfqpoint{4.022900in}{2.438417in}}%
\pgfpathlineto{\pgfqpoint{4.026121in}{2.439617in}}%
\pgfpathlineto{\pgfqpoint{4.027195in}{2.438254in}}%
\pgfpathlineto{\pgfqpoint{4.028269in}{2.435528in}}%
\pgfpathlineto{\pgfqpoint{4.030417in}{2.436073in}}%
\pgfpathlineto{\pgfqpoint{4.034712in}{2.437981in}}%
\pgfpathlineto{\pgfqpoint{4.035786in}{2.439235in}}%
\pgfpathlineto{\pgfqpoint{4.037934in}{2.440107in}}%
\pgfpathlineto{\pgfqpoint{4.042229in}{2.432693in}}%
\pgfpathlineto{\pgfqpoint{4.043303in}{2.434765in}}%
\pgfpathlineto{\pgfqpoint{4.044377in}{2.433947in}}%
\pgfpathlineto{\pgfqpoint{4.045451in}{2.431821in}}%
\pgfpathlineto{\pgfqpoint{4.049746in}{2.426424in}}%
\pgfpathlineto{\pgfqpoint{4.051894in}{2.427623in}}%
\pgfpathlineto{\pgfqpoint{4.052967in}{2.425224in}}%
\pgfpathlineto{\pgfqpoint{4.057263in}{2.429259in}}%
\pgfpathlineto{\pgfqpoint{4.058337in}{2.428986in}}%
\pgfpathlineto{\pgfqpoint{4.059410in}{2.430676in}}%
\pgfpathlineto{\pgfqpoint{4.060484in}{2.430022in}}%
\pgfpathlineto{\pgfqpoint{4.065853in}{2.432911in}}%
\pgfpathlineto{\pgfqpoint{4.066927in}{2.435691in}}%
\pgfpathlineto{\pgfqpoint{4.068001in}{2.434438in}}%
\pgfpathlineto{\pgfqpoint{4.072296in}{2.432311in}}%
\pgfpathlineto{\pgfqpoint{4.073370in}{2.432420in}}%
\pgfpathlineto{\pgfqpoint{4.074444in}{2.440925in}}%
\pgfpathlineto{\pgfqpoint{4.075518in}{2.442179in}}%
\pgfpathlineto{\pgfqpoint{4.078740in}{2.442124in}}%
\pgfpathlineto{\pgfqpoint{4.079813in}{2.445068in}}%
\pgfpathlineto{\pgfqpoint{4.080887in}{2.446104in}}%
\pgfpathlineto{\pgfqpoint{4.081961in}{2.441961in}}%
\pgfpathlineto{\pgfqpoint{4.083035in}{2.442015in}}%
\pgfpathlineto{\pgfqpoint{4.086256in}{2.443215in}}%
\pgfpathlineto{\pgfqpoint{4.088404in}{2.441634in}}%
\pgfpathlineto{\pgfqpoint{4.089478in}{2.442997in}}%
\pgfpathlineto{\pgfqpoint{4.090552in}{2.443106in}}%
\pgfpathlineto{\pgfqpoint{4.094847in}{2.447303in}}%
\pgfpathlineto{\pgfqpoint{4.096995in}{2.446322in}}%
\pgfpathlineto{\pgfqpoint{4.098069in}{2.444468in}}%
\pgfpathlineto{\pgfqpoint{4.103438in}{2.444250in}}%
\pgfpathlineto{\pgfqpoint{4.104512in}{2.445668in}}%
\pgfpathlineto{\pgfqpoint{4.105585in}{2.445286in}}%
\pgfpathlineto{\pgfqpoint{4.108807in}{2.445395in}}%
\pgfpathlineto{\pgfqpoint{4.109881in}{2.444087in}}%
\pgfpathlineto{\pgfqpoint{4.110955in}{2.444250in}}%
\pgfpathlineto{\pgfqpoint{4.112029in}{2.443433in}}%
\pgfpathlineto{\pgfqpoint{4.113102in}{2.444305in}}%
\pgfpathlineto{\pgfqpoint{4.116324in}{2.445831in}}%
\pgfpathlineto{\pgfqpoint{4.117398in}{2.443814in}}%
\pgfpathlineto{\pgfqpoint{4.119545in}{2.444741in}}%
\pgfpathlineto{\pgfqpoint{4.123841in}{2.442615in}}%
\pgfpathlineto{\pgfqpoint{4.124915in}{2.440598in}}%
\pgfpathlineto{\pgfqpoint{4.125988in}{2.441034in}}%
\pgfpathlineto{\pgfqpoint{4.128136in}{2.438308in}}%
\pgfpathlineto{\pgfqpoint{4.132431in}{2.443487in}}%
\pgfpathlineto{\pgfqpoint{4.134579in}{2.441579in}}%
\pgfpathlineto{\pgfqpoint{4.135653in}{2.445613in}}%
\pgfpathlineto{\pgfqpoint{4.138874in}{2.441906in}}%
\pgfpathlineto{\pgfqpoint{4.141022in}{2.445395in}}%
\pgfpathlineto{\pgfqpoint{4.142096in}{2.445286in}}%
\pgfpathlineto{\pgfqpoint{4.143170in}{2.434765in}}%
\pgfpathlineto{\pgfqpoint{4.146391in}{2.431276in}}%
\pgfpathlineto{\pgfqpoint{4.147465in}{2.431003in}}%
\pgfpathlineto{\pgfqpoint{4.149613in}{2.435037in}}%
\pgfpathlineto{\pgfqpoint{4.150687in}{2.434056in}}%
\pgfpathlineto{\pgfqpoint{4.154982in}{2.433565in}}%
\pgfpathlineto{\pgfqpoint{4.156056in}{2.431439in}}%
\pgfpathlineto{\pgfqpoint{4.157130in}{2.433456in}}%
\pgfpathlineto{\pgfqpoint{4.158204in}{2.431494in}}%
\pgfpathlineto{\pgfqpoint{4.161425in}{2.431058in}}%
\pgfpathlineto{\pgfqpoint{4.162499in}{2.431657in}}%
\pgfpathlineto{\pgfqpoint{4.163573in}{2.434110in}}%
\pgfpathlineto{\pgfqpoint{4.165720in}{2.430785in}}%
\pgfpathlineto{\pgfqpoint{4.168942in}{2.430185in}}%
\pgfpathlineto{\pgfqpoint{4.170016in}{2.429149in}}%
\pgfpathlineto{\pgfqpoint{4.171090in}{2.425879in}}%
\pgfpathlineto{\pgfqpoint{4.172163in}{2.426642in}}%
\pgfpathlineto{\pgfqpoint{4.173237in}{2.426315in}}%
\pgfpathlineto{\pgfqpoint{4.177533in}{2.428332in}}%
\pgfpathlineto{\pgfqpoint{4.178607in}{2.425551in}}%
\pgfpathlineto{\pgfqpoint{4.179680in}{2.425442in}}%
\pgfpathlineto{\pgfqpoint{4.180754in}{2.422226in}}%
\pgfpathlineto{\pgfqpoint{4.183976in}{2.421572in}}%
\pgfpathlineto{\pgfqpoint{4.185050in}{2.420590in}}%
\pgfpathlineto{\pgfqpoint{4.188271in}{2.427459in}}%
\pgfpathlineto{\pgfqpoint{4.191493in}{2.426533in}}%
\pgfpathlineto{\pgfqpoint{4.192566in}{2.427459in}}%
\pgfpathlineto{\pgfqpoint{4.193640in}{2.426642in}}%
\pgfpathlineto{\pgfqpoint{4.194714in}{2.429313in}}%
\pgfpathlineto{\pgfqpoint{4.195788in}{2.428005in}}%
\pgfpathlineto{\pgfqpoint{4.200083in}{2.425115in}}%
\pgfpathlineto{\pgfqpoint{4.201157in}{2.417210in}}%
\pgfpathlineto{\pgfqpoint{4.202231in}{2.414430in}}%
\pgfpathlineto{\pgfqpoint{4.203305in}{2.414812in}}%
\pgfpathlineto{\pgfqpoint{4.207600in}{2.412359in}}%
\pgfpathlineto{\pgfqpoint{4.208674in}{2.415466in}}%
\pgfpathlineto{\pgfqpoint{4.209748in}{2.416774in}}%
\pgfpathlineto{\pgfqpoint{4.210822in}{2.420918in}}%
\pgfpathlineto{\pgfqpoint{4.214043in}{2.420699in}}%
\pgfpathlineto{\pgfqpoint{4.215117in}{2.421408in}}%
\pgfpathlineto{\pgfqpoint{4.218339in}{2.421081in}}%
\pgfpathlineto{\pgfqpoint{4.222634in}{2.423044in}}%
\pgfpathlineto{\pgfqpoint{4.223708in}{2.421572in}}%
\pgfpathlineto{\pgfqpoint{4.225855in}{2.437599in}}%
\pgfpathlineto{\pgfqpoint{4.229077in}{2.436346in}}%
\pgfpathlineto{\pgfqpoint{4.230151in}{2.438690in}}%
\pgfpathlineto{\pgfqpoint{4.232298in}{2.439289in}}%
\pgfpathlineto{\pgfqpoint{4.233372in}{2.438472in}}%
\pgfpathlineto{\pgfqpoint{4.236594in}{2.436945in}}%
\pgfpathlineto{\pgfqpoint{4.237668in}{2.435364in}}%
\pgfpathlineto{\pgfqpoint{4.238741in}{2.435364in}}%
\pgfpathlineto{\pgfqpoint{4.240889in}{2.439998in}}%
\pgfpathlineto{\pgfqpoint{4.244111in}{2.440053in}}%
\pgfpathlineto{\pgfqpoint{4.247332in}{2.433129in}}%
\pgfpathlineto{\pgfqpoint{4.248406in}{2.441797in}}%
\pgfpathlineto{\pgfqpoint{4.251628in}{2.443160in}}%
\pgfpathlineto{\pgfqpoint{4.253775in}{2.447140in}}%
\pgfpathlineto{\pgfqpoint{4.255923in}{2.447358in}}%
\pgfpathlineto{\pgfqpoint{4.259144in}{2.446104in}}%
\pgfpathlineto{\pgfqpoint{4.260218in}{2.447194in}}%
\pgfpathlineto{\pgfqpoint{4.261292in}{2.446813in}}%
\pgfpathlineto{\pgfqpoint{4.262366in}{2.448394in}}%
\pgfpathlineto{\pgfqpoint{4.263440in}{2.448394in}}%
\pgfpathlineto{\pgfqpoint{4.267735in}{2.447358in}}%
\pgfpathlineto{\pgfqpoint{4.268809in}{2.447739in}}%
\pgfpathlineto{\pgfqpoint{4.269883in}{2.446976in}}%
\pgfpathlineto{\pgfqpoint{4.270957in}{2.448448in}}%
\pgfpathlineto{\pgfqpoint{4.274178in}{2.448830in}}%
\pgfpathlineto{\pgfqpoint{4.277400in}{2.447848in}}%
\pgfpathlineto{\pgfqpoint{4.278473in}{2.449211in}}%
\pgfpathlineto{\pgfqpoint{4.282769in}{2.450138in}}%
\pgfpathlineto{\pgfqpoint{4.283843in}{2.449648in}}%
\pgfpathlineto{\pgfqpoint{4.284917in}{2.453791in}}%
\pgfpathlineto{\pgfqpoint{4.285990in}{2.454990in}}%
\pgfpathlineto{\pgfqpoint{4.289212in}{2.454990in}}%
\pgfpathlineto{\pgfqpoint{4.291360in}{2.453191in}}%
\pgfpathlineto{\pgfqpoint{4.292433in}{2.456735in}}%
\pgfpathlineto{\pgfqpoint{4.293507in}{2.457825in}}%
\pgfpathlineto{\pgfqpoint{4.298876in}{2.458261in}}%
\pgfpathlineto{\pgfqpoint{4.299950in}{2.459569in}}%
\pgfpathlineto{\pgfqpoint{4.301024in}{2.458970in}}%
\pgfpathlineto{\pgfqpoint{4.304246in}{2.460442in}}%
\pgfpathlineto{\pgfqpoint{4.307467in}{2.458425in}}%
\pgfpathlineto{\pgfqpoint{4.311762in}{2.459079in}}%
\pgfpathlineto{\pgfqpoint{4.312836in}{2.456189in}}%
\pgfpathlineto{\pgfqpoint{4.314984in}{2.458752in}}%
\pgfpathlineto{\pgfqpoint{4.316058in}{2.459842in}}%
\pgfpathlineto{\pgfqpoint{4.319279in}{2.457552in}}%
\pgfpathlineto{\pgfqpoint{4.322501in}{2.452755in}}%
\pgfpathlineto{\pgfqpoint{4.323575in}{2.452482in}}%
\pgfpathlineto{\pgfqpoint{4.327870in}{2.457116in}}%
\pgfpathlineto{\pgfqpoint{4.328944in}{2.461641in}}%
\pgfpathlineto{\pgfqpoint{4.330018in}{2.461641in}}%
\pgfpathlineto{\pgfqpoint{4.331092in}{2.457716in}}%
\pgfpathlineto{\pgfqpoint{4.334313in}{2.457389in}}%
\pgfpathlineto{\pgfqpoint{4.335387in}{2.452700in}}%
\pgfpathlineto{\pgfqpoint{4.336461in}{2.455535in}}%
\pgfpathlineto{\pgfqpoint{4.337535in}{2.464312in}}%
\pgfpathlineto{\pgfqpoint{4.338608in}{2.461532in}}%
\pgfpathlineto{\pgfqpoint{4.346125in}{2.460442in}}%
\pgfpathlineto{\pgfqpoint{4.349347in}{2.458152in}}%
\pgfpathlineto{\pgfqpoint{4.350421in}{2.459297in}}%
\pgfpathlineto{\pgfqpoint{4.352568in}{2.450356in}}%
\pgfpathlineto{\pgfqpoint{4.353642in}{2.450792in}}%
\pgfpathlineto{\pgfqpoint{4.356864in}{2.451119in}}%
\pgfpathlineto{\pgfqpoint{4.360085in}{2.447794in}}%
\pgfpathlineto{\pgfqpoint{4.361159in}{2.449975in}}%
\pgfpathlineto{\pgfqpoint{4.368676in}{2.462186in}}%
\pgfpathlineto{\pgfqpoint{4.371897in}{2.461696in}}%
\pgfpathlineto{\pgfqpoint{4.372971in}{2.459515in}}%
\pgfpathlineto{\pgfqpoint{4.374045in}{2.461532in}}%
\pgfpathlineto{\pgfqpoint{4.375119in}{2.461805in}}%
\pgfpathlineto{\pgfqpoint{4.376193in}{2.463222in}}%
\pgfpathlineto{\pgfqpoint{4.379414in}{2.462132in}}%
\pgfpathlineto{\pgfqpoint{4.380488in}{2.460932in}}%
\pgfpathlineto{\pgfqpoint{4.381562in}{2.461368in}}%
\pgfpathlineto{\pgfqpoint{4.383710in}{2.465294in}}%
\pgfpathlineto{\pgfqpoint{4.386931in}{2.464967in}}%
\pgfpathlineto{\pgfqpoint{4.388005in}{2.466875in}}%
\pgfpathlineto{\pgfqpoint{4.389079in}{2.467365in}}%
\pgfpathlineto{\pgfqpoint{4.390153in}{2.464585in}}%
\pgfpathlineto{\pgfqpoint{4.391227in}{2.463658in}}%
\pgfpathlineto{\pgfqpoint{4.395522in}{2.463876in}}%
\pgfpathlineto{\pgfqpoint{4.396596in}{2.461096in}}%
\pgfpathlineto{\pgfqpoint{4.397670in}{2.463222in}}%
\pgfpathlineto{\pgfqpoint{4.398743in}{2.458370in}}%
\pgfpathlineto{\pgfqpoint{4.401965in}{2.457443in}}%
\pgfpathlineto{\pgfqpoint{4.404113in}{2.455154in}}%
\pgfpathlineto{\pgfqpoint{4.405186in}{2.455045in}}%
\pgfpathlineto{\pgfqpoint{4.406260in}{2.451447in}}%
\pgfpathlineto{\pgfqpoint{4.409482in}{2.454718in}}%
\pgfpathlineto{\pgfqpoint{4.411629in}{2.461314in}}%
\pgfpathlineto{\pgfqpoint{4.412703in}{2.459351in}}%
\pgfpathlineto{\pgfqpoint{4.413777in}{2.460006in}}%
\pgfpathlineto{\pgfqpoint{4.418073in}{2.457443in}}%
\pgfpathlineto{\pgfqpoint{4.420220in}{2.460769in}}%
\pgfpathlineto{\pgfqpoint{4.421294in}{2.460551in}}%
\pgfpathlineto{\pgfqpoint{4.426663in}{2.463549in}}%
\pgfpathlineto{\pgfqpoint{4.428811in}{2.466220in}}%
\pgfpathlineto{\pgfqpoint{4.433106in}{2.466275in}}%
\pgfpathlineto{\pgfqpoint{4.434180in}{2.469546in}}%
\pgfpathlineto{\pgfqpoint{4.435254in}{2.468347in}}%
\pgfpathlineto{\pgfqpoint{4.436328in}{2.480940in}}%
\pgfpathlineto{\pgfqpoint{4.439549in}{2.483229in}}%
\pgfpathlineto{\pgfqpoint{4.440623in}{2.488681in}}%
\pgfpathlineto{\pgfqpoint{4.442771in}{2.489390in}}%
\pgfpathlineto{\pgfqpoint{4.443845in}{2.492770in}}%
\pgfpathlineto{\pgfqpoint{4.447066in}{2.489826in}}%
\pgfpathlineto{\pgfqpoint{4.449214in}{2.493806in}}%
\pgfpathlineto{\pgfqpoint{4.450288in}{2.493806in}}%
\pgfpathlineto{\pgfqpoint{4.451361in}{2.492007in}}%
\pgfpathlineto{\pgfqpoint{4.454583in}{2.492279in}}%
\pgfpathlineto{\pgfqpoint{4.455657in}{2.493042in}}%
\pgfpathlineto{\pgfqpoint{4.456731in}{2.495605in}}%
\pgfpathlineto{\pgfqpoint{4.457805in}{2.496586in}}%
\pgfpathlineto{\pgfqpoint{4.458878in}{2.495550in}}%
\pgfpathlineto{\pgfqpoint{4.463174in}{2.499257in}}%
\pgfpathlineto{\pgfqpoint{4.464248in}{2.497949in}}%
\pgfpathlineto{\pgfqpoint{4.466395in}{2.497513in}}%
\pgfpathlineto{\pgfqpoint{4.469617in}{2.492770in}}%
\pgfpathlineto{\pgfqpoint{4.470691in}{2.493478in}}%
\pgfpathlineto{\pgfqpoint{4.471764in}{2.496477in}}%
\pgfpathlineto{\pgfqpoint{4.472838in}{2.493915in}}%
\pgfpathlineto{\pgfqpoint{4.473912in}{2.495332in}}%
\pgfpathlineto{\pgfqpoint{4.477134in}{2.496150in}}%
\pgfpathlineto{\pgfqpoint{4.479281in}{2.498494in}}%
\pgfpathlineto{\pgfqpoint{4.480355in}{2.497131in}}%
\pgfpathlineto{\pgfqpoint{4.481429in}{2.498385in}}%
\pgfpathlineto{\pgfqpoint{4.484650in}{2.499148in}}%
\pgfpathlineto{\pgfqpoint{4.485724in}{2.498330in}}%
\pgfpathlineto{\pgfqpoint{4.486798in}{2.496695in}}%
\pgfpathlineto{\pgfqpoint{4.487872in}{2.499312in}}%
\pgfpathlineto{\pgfqpoint{4.488946in}{2.494623in}}%
\pgfpathlineto{\pgfqpoint{4.492167in}{2.493860in}}%
\pgfpathlineto{\pgfqpoint{4.494315in}{2.500184in}}%
\pgfpathlineto{\pgfqpoint{4.495389in}{2.498494in}}%
\pgfpathlineto{\pgfqpoint{4.496463in}{2.498766in}}%
\pgfpathlineto{\pgfqpoint{4.499684in}{2.496150in}}%
\pgfpathlineto{\pgfqpoint{4.500758in}{2.499366in}}%
\pgfpathlineto{\pgfqpoint{4.501832in}{2.497840in}}%
\pgfpathlineto{\pgfqpoint{4.502906in}{2.497840in}}%
\pgfpathlineto{\pgfqpoint{4.507201in}{2.500675in}}%
\pgfpathlineto{\pgfqpoint{4.508275in}{2.506290in}}%
\pgfpathlineto{\pgfqpoint{4.509349in}{2.502583in}}%
\pgfpathlineto{\pgfqpoint{4.510423in}{2.504436in}}%
\pgfpathlineto{\pgfqpoint{4.511496in}{2.503019in}}%
\pgfpathlineto{\pgfqpoint{4.514718in}{2.505526in}}%
\pgfpathlineto{\pgfqpoint{4.515792in}{2.503182in}}%
\pgfpathlineto{\pgfqpoint{4.516866in}{2.507544in}}%
\pgfpathlineto{\pgfqpoint{4.519013in}{2.511251in}}%
\pgfpathlineto{\pgfqpoint{4.522235in}{2.509343in}}%
\pgfpathlineto{\pgfqpoint{4.523309in}{2.511360in}}%
\pgfpathlineto{\pgfqpoint{4.524383in}{2.506726in}}%
\pgfpathlineto{\pgfqpoint{4.525456in}{2.510324in}}%
\pgfpathlineto{\pgfqpoint{4.526530in}{2.517247in}}%
\pgfpathlineto{\pgfqpoint{4.529752in}{2.517138in}}%
\pgfpathlineto{\pgfqpoint{4.530826in}{2.522208in}}%
\pgfpathlineto{\pgfqpoint{4.531899in}{2.500620in}}%
\pgfpathlineto{\pgfqpoint{4.532973in}{2.498221in}}%
\pgfpathlineto{\pgfqpoint{4.534047in}{2.500566in}}%
\pgfpathlineto{\pgfqpoint{4.537269in}{2.503128in}}%
\pgfpathlineto{\pgfqpoint{4.538342in}{2.499039in}}%
\pgfpathlineto{\pgfqpoint{4.539416in}{2.500020in}}%
\pgfpathlineto{\pgfqpoint{4.541564in}{2.505417in}}%
\pgfpathlineto{\pgfqpoint{4.544785in}{2.503455in}}%
\pgfpathlineto{\pgfqpoint{4.545859in}{2.504273in}}%
\pgfpathlineto{\pgfqpoint{4.546933in}{2.506508in}}%
\pgfpathlineto{\pgfqpoint{4.548007in}{2.505526in}}%
\pgfpathlineto{\pgfqpoint{4.549081in}{2.507980in}}%
\pgfpathlineto{\pgfqpoint{4.552302in}{2.507216in}}%
\pgfpathlineto{\pgfqpoint{4.554450in}{2.516539in}}%
\pgfpathlineto{\pgfqpoint{4.555524in}{2.514413in}}%
\pgfpathlineto{\pgfqpoint{4.556598in}{2.513595in}}%
\pgfpathlineto{\pgfqpoint{4.560893in}{2.508143in}}%
\pgfpathlineto{\pgfqpoint{4.561967in}{2.508252in}}%
\pgfpathlineto{\pgfqpoint{4.563041in}{2.507707in}}%
\pgfpathlineto{\pgfqpoint{4.564115in}{2.521936in}}%
\pgfpathlineto{\pgfqpoint{4.568410in}{2.517356in}}%
\pgfpathlineto{\pgfqpoint{4.569484in}{2.508579in}}%
\pgfpathlineto{\pgfqpoint{4.570558in}{2.509452in}}%
\pgfpathlineto{\pgfqpoint{4.571631in}{2.499039in}}%
\pgfpathlineto{\pgfqpoint{4.574853in}{2.503019in}}%
\pgfpathlineto{\pgfqpoint{4.575927in}{2.501765in}}%
\pgfpathlineto{\pgfqpoint{4.577001in}{2.498548in}}%
\pgfpathlineto{\pgfqpoint{4.578074in}{2.499312in}}%
\pgfpathlineto{\pgfqpoint{4.579148in}{2.503455in}}%
\pgfpathlineto{\pgfqpoint{4.583444in}{2.505090in}}%
\pgfpathlineto{\pgfqpoint{4.584517in}{2.502801in}}%
\pgfpathlineto{\pgfqpoint{4.585591in}{2.506508in}}%
\pgfpathlineto{\pgfqpoint{4.586665in}{2.504709in}}%
\pgfpathlineto{\pgfqpoint{4.589887in}{2.508852in}}%
\pgfpathlineto{\pgfqpoint{4.590961in}{2.509234in}}%
\pgfpathlineto{\pgfqpoint{4.592034in}{2.502801in}}%
\pgfpathlineto{\pgfqpoint{4.593108in}{2.492279in}}%
\pgfpathlineto{\pgfqpoint{4.594182in}{2.502037in}}%
\pgfpathlineto{\pgfqpoint{4.597404in}{2.498221in}}%
\pgfpathlineto{\pgfqpoint{4.598477in}{2.498712in}}%
\pgfpathlineto{\pgfqpoint{4.599551in}{2.501874in}}%
\pgfpathlineto{\pgfqpoint{4.600625in}{2.503073in}}%
\pgfpathlineto{\pgfqpoint{4.601699in}{2.500020in}}%
\pgfpathlineto{\pgfqpoint{4.607068in}{2.507107in}}%
\pgfpathlineto{\pgfqpoint{4.609216in}{2.506072in}}%
\pgfpathlineto{\pgfqpoint{4.612437in}{2.507980in}}%
\pgfpathlineto{\pgfqpoint{4.613511in}{2.511578in}}%
\pgfpathlineto{\pgfqpoint{4.614585in}{2.512614in}}%
\pgfpathlineto{\pgfqpoint{4.616733in}{2.520464in}}%
\pgfpathlineto{\pgfqpoint{4.619954in}{2.520246in}}%
\pgfpathlineto{\pgfqpoint{4.622102in}{2.516593in}}%
\pgfpathlineto{\pgfqpoint{4.623176in}{2.517520in}}%
\pgfpathlineto{\pgfqpoint{4.624249in}{2.522372in}}%
\pgfpathlineto{\pgfqpoint{4.627471in}{2.521718in}}%
\pgfpathlineto{\pgfqpoint{4.628545in}{2.520464in}}%
\pgfpathlineto{\pgfqpoint{4.629619in}{2.517411in}}%
\pgfpathlineto{\pgfqpoint{4.630693in}{2.518065in}}%
\pgfpathlineto{\pgfqpoint{4.631766in}{2.517956in}}%
\pgfpathlineto{\pgfqpoint{4.634988in}{2.516593in}}%
\pgfpathlineto{\pgfqpoint{4.636062in}{2.518174in}}%
\pgfpathlineto{\pgfqpoint{4.637136in}{2.517520in}}%
\pgfpathlineto{\pgfqpoint{4.638209in}{2.523571in}}%
\pgfpathlineto{\pgfqpoint{4.639283in}{2.521990in}}%
\pgfpathlineto{\pgfqpoint{4.642505in}{2.522481in}}%
\pgfpathlineto{\pgfqpoint{4.644652in}{2.525043in}}%
\pgfpathlineto{\pgfqpoint{4.645726in}{2.525970in}}%
\pgfpathlineto{\pgfqpoint{4.646800in}{2.523571in}}%
\pgfpathlineto{\pgfqpoint{4.651095in}{2.523680in}}%
\pgfpathlineto{\pgfqpoint{4.652169in}{2.521827in}}%
\pgfpathlineto{\pgfqpoint{4.654317in}{2.515721in}}%
\pgfpathlineto{\pgfqpoint{4.659686in}{2.513377in}}%
\pgfpathlineto{\pgfqpoint{4.661834in}{2.516212in}}%
\pgfpathlineto{\pgfqpoint{4.666129in}{2.507053in}}%
\pgfpathlineto{\pgfqpoint{4.667203in}{2.501983in}}%
\pgfpathlineto{\pgfqpoint{4.669351in}{2.506671in}}%
\pgfpathlineto{\pgfqpoint{4.673646in}{2.506072in}}%
\pgfpathlineto{\pgfqpoint{4.675794in}{2.503291in}}%
\pgfpathlineto{\pgfqpoint{4.676868in}{2.503346in}}%
\pgfpathlineto{\pgfqpoint{4.680089in}{2.507925in}}%
\pgfpathlineto{\pgfqpoint{4.681163in}{2.506944in}}%
\pgfpathlineto{\pgfqpoint{4.682237in}{2.508416in}}%
\pgfpathlineto{\pgfqpoint{4.683311in}{2.508361in}}%
\pgfpathlineto{\pgfqpoint{4.684384in}{2.511905in}}%
\pgfpathlineto{\pgfqpoint{4.687606in}{2.516920in}}%
\pgfpathlineto{\pgfqpoint{4.688680in}{2.515448in}}%
\pgfpathlineto{\pgfqpoint{4.689754in}{2.517520in}}%
\pgfpathlineto{\pgfqpoint{4.690827in}{2.516757in}}%
\pgfpathlineto{\pgfqpoint{4.691901in}{2.513431in}}%
\pgfpathlineto{\pgfqpoint{4.695123in}{2.512941in}}%
\pgfpathlineto{\pgfqpoint{4.697271in}{2.505908in}}%
\pgfpathlineto{\pgfqpoint{4.698344in}{2.507380in}}%
\pgfpathlineto{\pgfqpoint{4.699418in}{2.503618in}}%
\pgfpathlineto{\pgfqpoint{4.702640in}{2.496313in}}%
\pgfpathlineto{\pgfqpoint{4.703714in}{2.498876in}}%
\pgfpathlineto{\pgfqpoint{4.705861in}{2.497458in}}%
\pgfpathlineto{\pgfqpoint{4.706935in}{2.498276in}}%
\pgfpathlineto{\pgfqpoint{4.710157in}{2.496586in}}%
\pgfpathlineto{\pgfqpoint{4.714452in}{2.509343in}}%
\pgfpathlineto{\pgfqpoint{4.717673in}{2.510487in}}%
\pgfpathlineto{\pgfqpoint{4.718747in}{2.505363in}}%
\pgfpathlineto{\pgfqpoint{4.720895in}{2.516375in}}%
\pgfpathlineto{\pgfqpoint{4.721969in}{2.516321in}}%
\pgfpathlineto{\pgfqpoint{4.725190in}{2.514358in}}%
\pgfpathlineto{\pgfqpoint{4.726264in}{2.520736in}}%
\pgfpathlineto{\pgfqpoint{4.727338in}{2.523135in}}%
\pgfpathlineto{\pgfqpoint{4.728412in}{2.521772in}}%
\pgfpathlineto{\pgfqpoint{4.729486in}{2.518992in}}%
\pgfpathlineto{\pgfqpoint{4.732707in}{2.524934in}}%
\pgfpathlineto{\pgfqpoint{4.733781in}{2.530168in}}%
\pgfpathlineto{\pgfqpoint{4.735929in}{2.522426in}}%
\pgfpathlineto{\pgfqpoint{4.737003in}{2.524226in}}%
\pgfpathlineto{\pgfqpoint{4.741298in}{2.525479in}}%
\pgfpathlineto{\pgfqpoint{4.742372in}{2.531749in}}%
\pgfpathlineto{\pgfqpoint{4.743446in}{2.529568in}}%
\pgfpathlineto{\pgfqpoint{4.744519in}{2.530386in}}%
\pgfpathlineto{\pgfqpoint{4.747741in}{2.529295in}}%
\pgfpathlineto{\pgfqpoint{4.749889in}{2.534911in}}%
\pgfpathlineto{\pgfqpoint{4.752036in}{2.541016in}}%
\pgfpathlineto{\pgfqpoint{4.755258in}{2.539763in}}%
\pgfpathlineto{\pgfqpoint{4.756332in}{2.538727in}}%
\pgfpathlineto{\pgfqpoint{4.757405in}{2.540308in}}%
\pgfpathlineto{\pgfqpoint{4.758479in}{2.540199in}}%
\pgfpathlineto{\pgfqpoint{4.759553in}{2.541507in}}%
\pgfpathlineto{\pgfqpoint{4.762775in}{2.543579in}}%
\pgfpathlineto{\pgfqpoint{4.764922in}{2.538454in}}%
\pgfpathlineto{\pgfqpoint{4.767070in}{2.538182in}}%
\pgfpathlineto{\pgfqpoint{4.770292in}{2.534202in}}%
\pgfpathlineto{\pgfqpoint{4.771365in}{2.536383in}}%
\pgfpathlineto{\pgfqpoint{4.773513in}{2.531204in}}%
\pgfpathlineto{\pgfqpoint{4.774587in}{2.539545in}}%
\pgfpathlineto{\pgfqpoint{4.777808in}{2.540744in}}%
\pgfpathlineto{\pgfqpoint{4.778882in}{2.535565in}}%
\pgfpathlineto{\pgfqpoint{4.779956in}{2.537255in}}%
\pgfpathlineto{\pgfqpoint{4.781030in}{2.529459in}}%
\pgfpathlineto{\pgfqpoint{4.782104in}{2.529786in}}%
\pgfpathlineto{\pgfqpoint{4.785325in}{2.526679in}}%
\pgfpathlineto{\pgfqpoint{4.786399in}{2.523135in}}%
\pgfpathlineto{\pgfqpoint{4.787473in}{2.529841in}}%
\pgfpathlineto{\pgfqpoint{4.788547in}{2.527660in}}%
\pgfpathlineto{\pgfqpoint{4.789621in}{2.527387in}}%
\pgfpathlineto{\pgfqpoint{4.792842in}{2.524934in}}%
\pgfpathlineto{\pgfqpoint{4.793916in}{2.524934in}}%
\pgfpathlineto{\pgfqpoint{4.797137in}{2.528151in}}%
\pgfpathlineto{\pgfqpoint{4.800359in}{2.528096in}}%
\pgfpathlineto{\pgfqpoint{4.803581in}{2.521445in}}%
\pgfpathlineto{\pgfqpoint{4.804654in}{2.521064in}}%
\pgfpathlineto{\pgfqpoint{4.807876in}{2.521936in}}%
\pgfpathlineto{\pgfqpoint{4.808950in}{2.525479in}}%
\pgfpathlineto{\pgfqpoint{4.810024in}{2.520082in}}%
\pgfpathlineto{\pgfqpoint{4.811097in}{2.520900in}}%
\pgfpathlineto{\pgfqpoint{4.815393in}{2.519046in}}%
\pgfpathlineto{\pgfqpoint{4.816467in}{2.522972in}}%
\pgfpathlineto{\pgfqpoint{4.817540in}{2.522590in}}%
\pgfpathlineto{\pgfqpoint{4.818614in}{2.521554in}}%
\pgfpathlineto{\pgfqpoint{4.819688in}{2.518447in}}%
\pgfpathlineto{\pgfqpoint{4.823983in}{2.519810in}}%
\pgfpathlineto{\pgfqpoint{4.825057in}{2.518229in}}%
\pgfpathlineto{\pgfqpoint{4.826131in}{2.513704in}}%
\pgfpathlineto{\pgfqpoint{4.827205in}{2.518011in}}%
\pgfpathlineto{\pgfqpoint{4.830426in}{2.514740in}}%
\pgfpathlineto{\pgfqpoint{4.831500in}{2.517684in}}%
\pgfpathlineto{\pgfqpoint{4.833648in}{2.507435in}}%
\pgfpathlineto{\pgfqpoint{4.837943in}{2.502092in}}%
\pgfpathlineto{\pgfqpoint{4.842239in}{2.509452in}}%
\pgfpathlineto{\pgfqpoint{4.845460in}{2.512723in}}%
\pgfpathlineto{\pgfqpoint{4.846534in}{2.516430in}}%
\pgfpathlineto{\pgfqpoint{4.847608in}{2.510269in}}%
\pgfpathlineto{\pgfqpoint{4.848682in}{2.511687in}}%
\pgfpathlineto{\pgfqpoint{4.849756in}{2.518992in}}%
\pgfpathlineto{\pgfqpoint{4.854051in}{2.512450in}}%
\pgfpathlineto{\pgfqpoint{4.855125in}{2.513268in}}%
\pgfpathlineto{\pgfqpoint{4.856199in}{2.512232in}}%
\pgfpathlineto{\pgfqpoint{4.857272in}{2.512450in}}%
\pgfpathlineto{\pgfqpoint{4.860494in}{2.511959in}}%
\pgfpathlineto{\pgfqpoint{4.861568in}{2.513159in}}%
\pgfpathlineto{\pgfqpoint{4.862642in}{2.511959in}}%
\pgfpathlineto{\pgfqpoint{4.864789in}{2.515612in}}%
\pgfpathlineto{\pgfqpoint{4.868011in}{2.510215in}}%
\pgfpathlineto{\pgfqpoint{4.869085in}{2.514685in}}%
\pgfpathlineto{\pgfqpoint{4.870159in}{2.511796in}}%
\pgfpathlineto{\pgfqpoint{4.872306in}{2.514358in}}%
\pgfpathlineto{\pgfqpoint{4.875528in}{2.515012in}}%
\pgfpathlineto{\pgfqpoint{4.877675in}{2.518283in}}%
\pgfpathlineto{\pgfqpoint{4.878749in}{2.518065in}}%
\pgfpathlineto{\pgfqpoint{4.879823in}{2.517084in}}%
\pgfpathlineto{\pgfqpoint{4.883045in}{2.521009in}}%
\pgfpathlineto{\pgfqpoint{4.884118in}{2.520682in}}%
\pgfpathlineto{\pgfqpoint{4.885192in}{2.516157in}}%
\pgfpathlineto{\pgfqpoint{4.887340in}{2.512068in}}%
\pgfpathlineto{\pgfqpoint{4.891635in}{2.520900in}}%
\pgfpathlineto{\pgfqpoint{4.892709in}{2.519483in}}%
\pgfpathlineto{\pgfqpoint{4.894857in}{2.520682in}}%
\pgfpathlineto{\pgfqpoint{4.898078in}{2.524498in}}%
\pgfpathlineto{\pgfqpoint{4.900226in}{2.522372in}}%
\pgfpathlineto{\pgfqpoint{4.901300in}{2.522208in}}%
\pgfpathlineto{\pgfqpoint{4.902374in}{2.520682in}}%
\pgfpathlineto{\pgfqpoint{4.905595in}{2.524007in}}%
\pgfpathlineto{\pgfqpoint{4.906669in}{2.527933in}}%
\pgfpathlineto{\pgfqpoint{4.907743in}{2.528587in}}%
\pgfpathlineto{\pgfqpoint{4.909891in}{2.525261in}}%
\pgfpathlineto{\pgfqpoint{4.914186in}{2.525588in}}%
\pgfpathlineto{\pgfqpoint{4.915260in}{2.529295in}}%
\pgfpathlineto{\pgfqpoint{4.916334in}{2.529786in}}%
\pgfpathlineto{\pgfqpoint{4.920629in}{2.528914in}}%
\pgfpathlineto{\pgfqpoint{4.922777in}{2.526352in}}%
\pgfpathlineto{\pgfqpoint{4.923850in}{2.530004in}}%
\pgfpathlineto{\pgfqpoint{4.924924in}{2.531204in}}%
\pgfpathlineto{\pgfqpoint{4.928146in}{2.538236in}}%
\pgfpathlineto{\pgfqpoint{4.929220in}{2.535946in}}%
\pgfpathlineto{\pgfqpoint{4.930293in}{2.536437in}}%
\pgfpathlineto{\pgfqpoint{4.931367in}{2.535456in}}%
\pgfpathlineto{\pgfqpoint{4.932441in}{2.533548in}}%
\pgfpathlineto{\pgfqpoint{4.935663in}{2.532512in}}%
\pgfpathlineto{\pgfqpoint{4.936737in}{2.529895in}}%
\pgfpathlineto{\pgfqpoint{4.937810in}{2.534311in}}%
\pgfpathlineto{\pgfqpoint{4.938884in}{2.534638in}}%
\pgfpathlineto{\pgfqpoint{4.939958in}{2.535728in}}%
\pgfpathlineto{\pgfqpoint{4.945327in}{2.530059in}}%
\pgfpathlineto{\pgfqpoint{4.946401in}{2.527115in}}%
\pgfpathlineto{\pgfqpoint{4.947475in}{2.526188in}}%
\pgfpathlineto{\pgfqpoint{4.950696in}{2.524335in}}%
\pgfpathlineto{\pgfqpoint{4.952844in}{2.526951in}}%
\pgfpathlineto{\pgfqpoint{4.953918in}{2.527660in}}%
\pgfpathlineto{\pgfqpoint{4.959287in}{2.525043in}}%
\pgfpathlineto{\pgfqpoint{4.960361in}{2.525152in}}%
\pgfpathlineto{\pgfqpoint{4.962509in}{2.528314in}}%
\pgfpathlineto{\pgfqpoint{4.965730in}{2.526352in}}%
\pgfpathlineto{\pgfqpoint{4.966804in}{2.524335in}}%
\pgfpathlineto{\pgfqpoint{4.967878in}{2.524062in}}%
\pgfpathlineto{\pgfqpoint{4.968952in}{2.525207in}}%
\pgfpathlineto{\pgfqpoint{4.970026in}{2.524825in}}%
\pgfpathlineto{\pgfqpoint{4.975395in}{2.524934in}}%
\pgfpathlineto{\pgfqpoint{4.977542in}{2.523026in}}%
\pgfpathlineto{\pgfqpoint{4.980764in}{2.523244in}}%
\pgfpathlineto{\pgfqpoint{4.981838in}{2.522754in}}%
\pgfpathlineto{\pgfqpoint{4.982912in}{2.523735in}}%
\pgfpathlineto{\pgfqpoint{4.983985in}{2.525806in}}%
\pgfpathlineto{\pgfqpoint{4.985059in}{2.524335in}}%
\pgfpathlineto{\pgfqpoint{4.988281in}{2.522481in}}%
\pgfpathlineto{\pgfqpoint{4.989355in}{2.520137in}}%
\pgfpathlineto{\pgfqpoint{4.990428in}{2.521554in}}%
\pgfpathlineto{\pgfqpoint{4.991502in}{2.518338in}}%
\pgfpathlineto{\pgfqpoint{4.992576in}{2.520137in}}%
\pgfpathlineto{\pgfqpoint{4.995798in}{2.518174in}}%
\pgfpathlineto{\pgfqpoint{4.996871in}{2.521336in}}%
\pgfpathlineto{\pgfqpoint{4.999019in}{2.524607in}}%
\pgfpathlineto{\pgfqpoint{5.003314in}{2.525534in}}%
\pgfpathlineto{\pgfqpoint{5.004388in}{2.527278in}}%
\pgfpathlineto{\pgfqpoint{5.005462in}{2.532403in}}%
\pgfpathlineto{\pgfqpoint{5.006536in}{2.532130in}}%
\pgfpathlineto{\pgfqpoint{5.007610in}{2.530059in}}%
\pgfpathlineto{\pgfqpoint{5.011905in}{2.530495in}}%
\pgfpathlineto{\pgfqpoint{5.012979in}{2.532021in}}%
\pgfpathlineto{\pgfqpoint{5.014053in}{2.529514in}}%
\pgfpathlineto{\pgfqpoint{5.015127in}{2.530168in}}%
\pgfpathlineto{\pgfqpoint{5.020496in}{2.527605in}}%
\pgfpathlineto{\pgfqpoint{5.021570in}{2.528859in}}%
\pgfpathlineto{\pgfqpoint{5.022644in}{2.525588in}}%
\pgfpathlineto{\pgfqpoint{5.025865in}{2.524171in}}%
\pgfpathlineto{\pgfqpoint{5.029087in}{2.515176in}}%
\pgfpathlineto{\pgfqpoint{5.030160in}{2.526025in}}%
\pgfpathlineto{\pgfqpoint{5.033382in}{2.524007in}}%
\pgfpathlineto{\pgfqpoint{5.034456in}{2.524880in}}%
\pgfpathlineto{\pgfqpoint{5.035530in}{2.532675in}}%
\pgfpathlineto{\pgfqpoint{5.036603in}{2.528260in}}%
\pgfpathlineto{\pgfqpoint{5.037677in}{2.532021in}}%
\pgfpathlineto{\pgfqpoint{5.040899in}{2.534529in}}%
\pgfpathlineto{\pgfqpoint{5.043047in}{2.534529in}}%
\pgfpathlineto{\pgfqpoint{5.044120in}{2.536710in}}%
\pgfpathlineto{\pgfqpoint{5.045194in}{2.535892in}}%
\pgfpathlineto{\pgfqpoint{5.052711in}{2.543415in}}%
\pgfpathlineto{\pgfqpoint{5.055933in}{2.544069in}}%
\pgfpathlineto{\pgfqpoint{5.059154in}{2.541725in}}%
\pgfpathlineto{\pgfqpoint{5.060228in}{2.542107in}}%
\pgfpathlineto{\pgfqpoint{5.065597in}{2.541125in}}%
\pgfpathlineto{\pgfqpoint{5.066671in}{2.544833in}}%
\pgfpathlineto{\pgfqpoint{5.067745in}{2.545160in}}%
\pgfpathlineto{\pgfqpoint{5.070966in}{2.543088in}}%
\pgfpathlineto{\pgfqpoint{5.072040in}{2.541507in}}%
\pgfpathlineto{\pgfqpoint{5.073114in}{2.544560in}}%
\pgfpathlineto{\pgfqpoint{5.075262in}{2.542761in}}%
\pgfpathlineto{\pgfqpoint{5.080631in}{2.546959in}}%
\pgfpathlineto{\pgfqpoint{5.081705in}{2.547122in}}%
\pgfpathlineto{\pgfqpoint{5.082779in}{2.548431in}}%
\pgfpathlineto{\pgfqpoint{5.086000in}{2.550012in}}%
\pgfpathlineto{\pgfqpoint{5.087074in}{2.548267in}}%
\pgfpathlineto{\pgfqpoint{5.088148in}{2.551974in}}%
\pgfpathlineto{\pgfqpoint{5.089222in}{2.547776in}}%
\pgfpathlineto{\pgfqpoint{5.090295in}{2.548921in}}%
\pgfpathlineto{\pgfqpoint{5.093517in}{2.548267in}}%
\pgfpathlineto{\pgfqpoint{5.095665in}{2.542216in}}%
\pgfpathlineto{\pgfqpoint{5.096738in}{2.541834in}}%
\pgfpathlineto{\pgfqpoint{5.097812in}{2.545269in}}%
\pgfpathlineto{\pgfqpoint{5.101034in}{2.544233in}}%
\pgfpathlineto{\pgfqpoint{5.102108in}{2.542325in}}%
\pgfpathlineto{\pgfqpoint{5.103181in}{2.547068in}}%
\pgfpathlineto{\pgfqpoint{5.104255in}{2.544669in}}%
\pgfpathlineto{\pgfqpoint{5.105329in}{2.549412in}}%
\pgfpathlineto{\pgfqpoint{5.108551in}{2.543306in}}%
\pgfpathlineto{\pgfqpoint{5.109625in}{2.544124in}}%
\pgfpathlineto{\pgfqpoint{5.111772in}{2.538018in}}%
\pgfpathlineto{\pgfqpoint{5.112846in}{2.542815in}}%
\pgfpathlineto{\pgfqpoint{5.118215in}{2.550284in}}%
\pgfpathlineto{\pgfqpoint{5.119289in}{2.545596in}}%
\pgfpathlineto{\pgfqpoint{5.120363in}{2.554645in}}%
\pgfpathlineto{\pgfqpoint{5.123584in}{2.558353in}}%
\pgfpathlineto{\pgfqpoint{5.124658in}{2.560806in}}%
\pgfpathlineto{\pgfqpoint{5.125732in}{2.561133in}}%
\pgfpathlineto{\pgfqpoint{5.127880in}{2.564567in}}%
\pgfpathlineto{\pgfqpoint{5.131101in}{2.565058in}}%
\pgfpathlineto{\pgfqpoint{5.132175in}{2.570946in}}%
\pgfpathlineto{\pgfqpoint{5.133249in}{2.572636in}}%
\pgfpathlineto{\pgfqpoint{5.134323in}{2.572200in}}%
\pgfpathlineto{\pgfqpoint{5.135397in}{2.573290in}}%
\pgfpathlineto{\pgfqpoint{5.139692in}{2.575798in}}%
\pgfpathlineto{\pgfqpoint{5.140766in}{2.574925in}}%
\pgfpathlineto{\pgfqpoint{5.142914in}{2.568493in}}%
\pgfpathlineto{\pgfqpoint{5.146135in}{2.567239in}}%
\pgfpathlineto{\pgfqpoint{5.147209in}{2.567729in}}%
\pgfpathlineto{\pgfqpoint{5.148283in}{2.571382in}}%
\pgfpathlineto{\pgfqpoint{5.149357in}{2.570183in}}%
\pgfpathlineto{\pgfqpoint{5.150430in}{2.570782in}}%
\pgfpathlineto{\pgfqpoint{5.153652in}{2.568493in}}%
\pgfpathlineto{\pgfqpoint{5.154726in}{2.571709in}}%
\pgfpathlineto{\pgfqpoint{5.155800in}{2.572091in}}%
\pgfpathlineto{\pgfqpoint{5.157947in}{2.579341in}}%
\pgfpathlineto{\pgfqpoint{5.161169in}{2.577706in}}%
\pgfpathlineto{\pgfqpoint{5.162243in}{2.582340in}}%
\pgfpathlineto{\pgfqpoint{5.163316in}{2.577324in}}%
\pgfpathlineto{\pgfqpoint{5.164390in}{2.580050in}}%
\pgfpathlineto{\pgfqpoint{5.165464in}{2.579123in}}%
\pgfpathlineto{\pgfqpoint{5.168686in}{2.580922in}}%
\pgfpathlineto{\pgfqpoint{5.169759in}{2.580650in}}%
\pgfpathlineto{\pgfqpoint{5.170833in}{2.577324in}}%
\pgfpathlineto{\pgfqpoint{5.171907in}{2.579287in}}%
\pgfpathlineto{\pgfqpoint{5.172981in}{2.575253in}}%
\pgfpathlineto{\pgfqpoint{5.176202in}{2.573617in}}%
\pgfpathlineto{\pgfqpoint{5.177276in}{2.574326in}}%
\pgfpathlineto{\pgfqpoint{5.179424in}{2.586701in}}%
\pgfpathlineto{\pgfqpoint{5.180498in}{2.586973in}}%
\pgfpathlineto{\pgfqpoint{5.183719in}{2.589536in}}%
\pgfpathlineto{\pgfqpoint{5.184793in}{2.592752in}}%
\pgfpathlineto{\pgfqpoint{5.185867in}{2.592043in}}%
\pgfpathlineto{\pgfqpoint{5.188015in}{2.593570in}}%
\pgfpathlineto{\pgfqpoint{5.192310in}{2.588609in}}%
\pgfpathlineto{\pgfqpoint{5.193384in}{2.582449in}}%
\pgfpathlineto{\pgfqpoint{5.195532in}{2.579396in}}%
\pgfpathlineto{\pgfqpoint{5.198753in}{2.577379in}}%
\pgfpathlineto{\pgfqpoint{5.199827in}{2.575471in}}%
\pgfpathlineto{\pgfqpoint{5.200901in}{2.577651in}}%
\pgfpathlineto{\pgfqpoint{5.201975in}{2.582503in}}%
\pgfpathlineto{\pgfqpoint{5.203048in}{2.578524in}}%
\pgfpathlineto{\pgfqpoint{5.206270in}{2.576943in}}%
\pgfpathlineto{\pgfqpoint{5.207344in}{2.578742in}}%
\pgfpathlineto{\pgfqpoint{5.209491in}{2.576670in}}%
\pgfpathlineto{\pgfqpoint{5.210565in}{2.583212in}}%
\pgfpathlineto{\pgfqpoint{5.214861in}{2.582939in}}%
\pgfpathlineto{\pgfqpoint{5.215935in}{2.583757in}}%
\pgfpathlineto{\pgfqpoint{5.217008in}{2.588009in}}%
\pgfpathlineto{\pgfqpoint{5.218082in}{2.580704in}}%
\pgfpathlineto{\pgfqpoint{5.221304in}{2.578360in}}%
\pgfpathlineto{\pgfqpoint{5.222378in}{2.563750in}}%
\pgfpathlineto{\pgfqpoint{5.223451in}{2.557426in}}%
\pgfpathlineto{\pgfqpoint{5.224525in}{2.559879in}}%
\pgfpathlineto{\pgfqpoint{5.225599in}{2.553392in}}%
\pgfpathlineto{\pgfqpoint{5.228821in}{2.557317in}}%
\pgfpathlineto{\pgfqpoint{5.229894in}{2.561242in}}%
\pgfpathlineto{\pgfqpoint{5.230968in}{2.560479in}}%
\pgfpathlineto{\pgfqpoint{5.232042in}{2.564785in}}%
\pgfpathlineto{\pgfqpoint{5.233116in}{2.559606in}}%
\pgfpathlineto{\pgfqpoint{5.236337in}{2.556990in}}%
\pgfpathlineto{\pgfqpoint{5.239559in}{2.561624in}}%
\pgfpathlineto{\pgfqpoint{5.240633in}{2.560969in}}%
\pgfpathlineto{\pgfqpoint{5.244928in}{2.559061in}}%
\pgfpathlineto{\pgfqpoint{5.246002in}{2.562714in}}%
\pgfpathlineto{\pgfqpoint{5.247076in}{2.557753in}}%
\pgfpathlineto{\pgfqpoint{5.248150in}{2.556117in}}%
\pgfpathlineto{\pgfqpoint{5.252445in}{2.559116in}}%
\pgfpathlineto{\pgfqpoint{5.253519in}{2.558898in}}%
\pgfpathlineto{\pgfqpoint{5.254593in}{2.557535in}}%
\pgfpathlineto{\pgfqpoint{5.255667in}{2.557371in}}%
\pgfpathlineto{\pgfqpoint{5.258888in}{2.558680in}}%
\pgfpathlineto{\pgfqpoint{5.259962in}{2.557480in}}%
\pgfpathlineto{\pgfqpoint{5.261036in}{2.553664in}}%
\pgfpathlineto{\pgfqpoint{5.262110in}{2.554973in}}%
\pgfpathlineto{\pgfqpoint{5.263183in}{2.545705in}}%
\pgfpathlineto{\pgfqpoint{5.266405in}{2.547722in}}%
\pgfpathlineto{\pgfqpoint{5.267479in}{2.540417in}}%
\pgfpathlineto{\pgfqpoint{5.268553in}{2.539708in}}%
\pgfpathlineto{\pgfqpoint{5.269626in}{2.543034in}}%
\pgfpathlineto{\pgfqpoint{5.270700in}{2.541780in}}%
\pgfpathlineto{\pgfqpoint{5.273922in}{2.549957in}}%
\pgfpathlineto{\pgfqpoint{5.274996in}{2.546577in}}%
\pgfpathlineto{\pgfqpoint{5.276069in}{2.550829in}}%
\pgfpathlineto{\pgfqpoint{5.277143in}{2.549085in}}%
\pgfpathlineto{\pgfqpoint{5.278217in}{2.555518in}}%
\pgfpathlineto{\pgfqpoint{5.281439in}{2.556063in}}%
\pgfpathlineto{\pgfqpoint{5.284660in}{2.543306in}}%
\pgfpathlineto{\pgfqpoint{5.288956in}{2.545977in}}%
\pgfpathlineto{\pgfqpoint{5.290029in}{2.542325in}}%
\pgfpathlineto{\pgfqpoint{5.291103in}{2.544069in}}%
\pgfpathlineto{\pgfqpoint{5.292177in}{2.544615in}}%
\pgfpathlineto{\pgfqpoint{5.296472in}{2.547504in}}%
\pgfpathlineto{\pgfqpoint{5.297546in}{2.544396in}}%
\pgfpathlineto{\pgfqpoint{5.298620in}{2.546195in}}%
\pgfpathlineto{\pgfqpoint{5.299694in}{2.546795in}}%
\pgfpathlineto{\pgfqpoint{5.300768in}{2.548976in}}%
\pgfpathlineto{\pgfqpoint{5.303989in}{2.549357in}}%
\pgfpathlineto{\pgfqpoint{5.305063in}{2.550121in}}%
\pgfpathlineto{\pgfqpoint{5.307211in}{2.549685in}}%
\pgfpathlineto{\pgfqpoint{5.308285in}{2.545105in}}%
\pgfpathlineto{\pgfqpoint{5.313654in}{2.547722in}}%
\pgfpathlineto{\pgfqpoint{5.314728in}{2.540689in}}%
\pgfpathlineto{\pgfqpoint{5.315802in}{2.540962in}}%
\pgfpathlineto{\pgfqpoint{5.320097in}{2.538182in}}%
\pgfpathlineto{\pgfqpoint{5.322245in}{2.533875in}}%
\pgfpathlineto{\pgfqpoint{5.323318in}{2.537582in}}%
\pgfpathlineto{\pgfqpoint{5.326540in}{2.537855in}}%
\pgfpathlineto{\pgfqpoint{5.327614in}{2.536492in}}%
\pgfpathlineto{\pgfqpoint{5.328688in}{2.538073in}}%
\pgfpathlineto{\pgfqpoint{5.329761in}{2.537146in}}%
\pgfpathlineto{\pgfqpoint{5.330835in}{2.540744in}}%
\pgfpathlineto{\pgfqpoint{5.336204in}{2.534747in}}%
\pgfpathlineto{\pgfqpoint{5.338352in}{2.541180in}}%
\pgfpathlineto{\pgfqpoint{5.341574in}{2.539708in}}%
\pgfpathlineto{\pgfqpoint{5.342647in}{2.540144in}}%
\pgfpathlineto{\pgfqpoint{5.343721in}{2.538454in}}%
\pgfpathlineto{\pgfqpoint{5.344795in}{2.538127in}}%
\pgfpathlineto{\pgfqpoint{5.345869in}{2.535946in}}%
\pgfpathlineto{\pgfqpoint{5.350164in}{2.532076in}}%
\pgfpathlineto{\pgfqpoint{5.351238in}{2.533275in}}%
\pgfpathlineto{\pgfqpoint{5.352312in}{2.533003in}}%
\pgfpathlineto{\pgfqpoint{5.353386in}{2.528532in}}%
\pgfpathlineto{\pgfqpoint{5.356607in}{2.530713in}}%
\pgfpathlineto{\pgfqpoint{5.357681in}{2.529241in}}%
\pgfpathlineto{\pgfqpoint{5.358755in}{2.529350in}}%
\pgfpathlineto{\pgfqpoint{5.359829in}{2.527387in}}%
\pgfpathlineto{\pgfqpoint{5.360903in}{2.523953in}}%
\pgfpathlineto{\pgfqpoint{5.364124in}{2.525261in}}%
\pgfpathlineto{\pgfqpoint{5.366272in}{2.533984in}}%
\pgfpathlineto{\pgfqpoint{5.367346in}{2.533384in}}%
\pgfpathlineto{\pgfqpoint{5.368420in}{2.530822in}}%
\pgfpathlineto{\pgfqpoint{5.371641in}{2.527224in}}%
\pgfpathlineto{\pgfqpoint{5.374863in}{2.540090in}}%
\pgfpathlineto{\pgfqpoint{5.375936in}{2.538727in}}%
\pgfpathlineto{\pgfqpoint{5.379158in}{2.538345in}}%
\pgfpathlineto{\pgfqpoint{5.380232in}{2.535183in}}%
\pgfpathlineto{\pgfqpoint{5.382379in}{2.533275in}}%
\pgfpathlineto{\pgfqpoint{5.383453in}{2.533057in}}%
\pgfpathlineto{\pgfqpoint{5.386675in}{2.528205in}}%
\pgfpathlineto{\pgfqpoint{5.387749in}{2.527824in}}%
\pgfpathlineto{\pgfqpoint{5.388823in}{2.534856in}}%
\pgfpathlineto{\pgfqpoint{5.389896in}{2.535837in}}%
\pgfpathlineto{\pgfqpoint{5.394192in}{2.536437in}}%
\pgfpathlineto{\pgfqpoint{5.395266in}{2.544342in}}%
\pgfpathlineto{\pgfqpoint{5.397413in}{2.539326in}}%
\pgfpathlineto{\pgfqpoint{5.402782in}{2.545868in}}%
\pgfpathlineto{\pgfqpoint{5.404930in}{2.547122in}}%
\pgfpathlineto{\pgfqpoint{5.409225in}{2.546577in}}%
\pgfpathlineto{\pgfqpoint{5.410299in}{2.543960in}}%
\pgfpathlineto{\pgfqpoint{5.412447in}{2.542652in}}%
\pgfpathlineto{\pgfqpoint{5.413521in}{2.540689in}}%
\pgfpathlineto{\pgfqpoint{5.416742in}{2.539163in}}%
\pgfpathlineto{\pgfqpoint{5.418890in}{2.542270in}}%
\pgfpathlineto{\pgfqpoint{5.419964in}{2.526788in}}%
\pgfpathlineto{\pgfqpoint{5.421038in}{2.523462in}}%
\pgfpathlineto{\pgfqpoint{5.424259in}{2.522045in}}%
\pgfpathlineto{\pgfqpoint{5.425333in}{2.519701in}}%
\pgfpathlineto{\pgfqpoint{5.428555in}{2.517629in}}%
\pgfpathlineto{\pgfqpoint{5.431776in}{2.521936in}}%
\pgfpathlineto{\pgfqpoint{5.432850in}{2.521173in}}%
\pgfpathlineto{\pgfqpoint{5.433924in}{2.521990in}}%
\pgfpathlineto{\pgfqpoint{5.434998in}{2.519046in}}%
\pgfpathlineto{\pgfqpoint{5.436071in}{2.518283in}}%
\pgfpathlineto{\pgfqpoint{5.439293in}{2.517847in}}%
\pgfpathlineto{\pgfqpoint{5.440367in}{2.515939in}}%
\pgfpathlineto{\pgfqpoint{5.441441in}{2.511087in}}%
\pgfpathlineto{\pgfqpoint{5.442514in}{2.510106in}}%
\pgfpathlineto{\pgfqpoint{5.443588in}{2.500184in}}%
\pgfpathlineto{\pgfqpoint{5.447884in}{2.483775in}}%
\pgfpathlineto{\pgfqpoint{5.448957in}{2.495659in}}%
\pgfpathlineto{\pgfqpoint{5.450031in}{2.498439in}}%
\pgfpathlineto{\pgfqpoint{5.451105in}{2.497131in}}%
\pgfpathlineto{\pgfqpoint{5.454327in}{2.494569in}}%
\pgfpathlineto{\pgfqpoint{5.455401in}{2.486064in}}%
\pgfpathlineto{\pgfqpoint{5.456474in}{2.490426in}}%
\pgfpathlineto{\pgfqpoint{5.457548in}{2.490971in}}%
\pgfpathlineto{\pgfqpoint{5.458622in}{2.485410in}}%
\pgfpathlineto{\pgfqpoint{5.462917in}{2.491243in}}%
\pgfpathlineto{\pgfqpoint{5.463991in}{2.484047in}}%
\pgfpathlineto{\pgfqpoint{5.465065in}{2.483284in}}%
\pgfpathlineto{\pgfqpoint{5.466139in}{2.483775in}}%
\pgfpathlineto{\pgfqpoint{5.469360in}{2.482030in}}%
\pgfpathlineto{\pgfqpoint{5.470434in}{2.488681in}}%
\pgfpathlineto{\pgfqpoint{5.471508in}{2.491788in}}%
\pgfpathlineto{\pgfqpoint{5.472582in}{2.492497in}}%
\pgfpathlineto{\pgfqpoint{5.473656in}{2.491025in}}%
\pgfpathlineto{\pgfqpoint{5.476877in}{2.494460in}}%
\pgfpathlineto{\pgfqpoint{5.477951in}{2.492225in}}%
\pgfpathlineto{\pgfqpoint{5.479025in}{2.492606in}}%
\pgfpathlineto{\pgfqpoint{5.481173in}{2.504164in}}%
\pgfpathlineto{\pgfqpoint{5.484394in}{2.499802in}}%
\pgfpathlineto{\pgfqpoint{5.485468in}{2.502256in}}%
\pgfpathlineto{\pgfqpoint{5.486542in}{2.500620in}}%
\pgfpathlineto{\pgfqpoint{5.487616in}{2.500675in}}%
\pgfpathlineto{\pgfqpoint{5.488690in}{2.502964in}}%
\pgfpathlineto{\pgfqpoint{5.494059in}{2.509179in}}%
\pgfpathlineto{\pgfqpoint{5.495133in}{2.512450in}}%
\pgfpathlineto{\pgfqpoint{5.496206in}{2.512832in}}%
\pgfpathlineto{\pgfqpoint{5.499428in}{2.512123in}}%
\pgfpathlineto{\pgfqpoint{5.500502in}{2.511033in}}%
\pgfpathlineto{\pgfqpoint{5.502649in}{2.511796in}}%
\pgfpathlineto{\pgfqpoint{5.503723in}{2.514849in}}%
\pgfpathlineto{\pgfqpoint{5.506945in}{2.516103in}}%
\pgfpathlineto{\pgfqpoint{5.508019in}{2.512559in}}%
\pgfpathlineto{\pgfqpoint{5.509092in}{2.511741in}}%
\pgfpathlineto{\pgfqpoint{5.510166in}{2.517847in}}%
\pgfpathlineto{\pgfqpoint{5.511240in}{2.528369in}}%
\pgfpathlineto{\pgfqpoint{5.514462in}{2.530604in}}%
\pgfpathlineto{\pgfqpoint{5.515535in}{2.529623in}}%
\pgfpathlineto{\pgfqpoint{5.516609in}{2.525861in}}%
\pgfpathlineto{\pgfqpoint{5.517683in}{2.528314in}}%
\pgfpathlineto{\pgfqpoint{5.518757in}{2.525207in}}%
\pgfpathlineto{\pgfqpoint{5.521979in}{2.526297in}}%
\pgfpathlineto{\pgfqpoint{5.523052in}{2.528478in}}%
\pgfpathlineto{\pgfqpoint{5.524126in}{2.528532in}}%
\pgfpathlineto{\pgfqpoint{5.526274in}{2.521282in}}%
\pgfpathlineto{\pgfqpoint{5.529495in}{2.520464in}}%
\pgfpathlineto{\pgfqpoint{5.531643in}{2.523190in}}%
\pgfpathlineto{\pgfqpoint{5.533791in}{2.513540in}}%
\pgfpathlineto{\pgfqpoint{5.537012in}{2.519973in}}%
\pgfpathlineto{\pgfqpoint{5.538086in}{2.518883in}}%
\pgfpathlineto{\pgfqpoint{5.539160in}{2.522917in}}%
\pgfpathlineto{\pgfqpoint{5.540234in}{2.524444in}}%
\pgfpathlineto{\pgfqpoint{5.541308in}{2.522536in}}%
\pgfpathlineto{\pgfqpoint{5.544529in}{2.523244in}}%
\pgfpathlineto{\pgfqpoint{5.545603in}{2.525588in}}%
\pgfpathlineto{\pgfqpoint{5.546677in}{2.522917in}}%
\pgfpathlineto{\pgfqpoint{5.548824in}{2.521936in}}%
\pgfpathlineto{\pgfqpoint{5.552046in}{2.517793in}}%
\pgfpathlineto{\pgfqpoint{5.553120in}{2.523081in}}%
\pgfpathlineto{\pgfqpoint{5.555267in}{2.522154in}}%
\pgfpathlineto{\pgfqpoint{5.556341in}{2.532239in}}%
\pgfpathlineto{\pgfqpoint{5.559563in}{2.534856in}}%
\pgfpathlineto{\pgfqpoint{5.560637in}{2.531858in}}%
\pgfpathlineto{\pgfqpoint{5.561711in}{2.531640in}}%
\pgfpathlineto{\pgfqpoint{5.563858in}{2.532021in}}%
\pgfpathlineto{\pgfqpoint{5.567080in}{2.534475in}}%
\pgfpathlineto{\pgfqpoint{5.569227in}{2.547558in}}%
\pgfpathlineto{\pgfqpoint{5.570301in}{2.544124in}}%
\pgfpathlineto{\pgfqpoint{5.571375in}{2.533711in}}%
\pgfpathlineto{\pgfqpoint{5.574597in}{2.537636in}}%
\pgfpathlineto{\pgfqpoint{5.576744in}{2.542379in}}%
\pgfpathlineto{\pgfqpoint{5.577818in}{2.541725in}}%
\pgfpathlineto{\pgfqpoint{5.582113in}{2.542379in}}%
\pgfpathlineto{\pgfqpoint{5.583187in}{2.544505in}}%
\pgfpathlineto{\pgfqpoint{5.584261in}{2.543088in}}%
\pgfpathlineto{\pgfqpoint{5.585335in}{2.539872in}}%
\pgfpathlineto{\pgfqpoint{5.589630in}{2.534856in}}%
\pgfpathlineto{\pgfqpoint{5.590704in}{2.536055in}}%
\pgfpathlineto{\pgfqpoint{5.593926in}{2.523244in}}%
\pgfpathlineto{\pgfqpoint{5.597147in}{2.526624in}}%
\pgfpathlineto{\pgfqpoint{5.598221in}{2.525861in}}%
\pgfpathlineto{\pgfqpoint{5.599295in}{2.522645in}}%
\pgfpathlineto{\pgfqpoint{5.600369in}{2.524116in}}%
\pgfpathlineto{\pgfqpoint{5.601443in}{2.518447in}}%
\pgfpathlineto{\pgfqpoint{5.605738in}{2.526897in}}%
\pgfpathlineto{\pgfqpoint{5.606812in}{2.525752in}}%
\pgfpathlineto{\pgfqpoint{5.608959in}{2.533221in}}%
\pgfpathlineto{\pgfqpoint{5.612181in}{2.530767in}}%
\pgfpathlineto{\pgfqpoint{5.613255in}{2.540308in}}%
\pgfpathlineto{\pgfqpoint{5.614329in}{2.540253in}}%
\pgfpathlineto{\pgfqpoint{5.615402in}{2.545214in}}%
\pgfpathlineto{\pgfqpoint{5.616476in}{2.554373in}}%
\pgfpathlineto{\pgfqpoint{5.619698in}{2.551593in}}%
\pgfpathlineto{\pgfqpoint{5.620772in}{2.547177in}}%
\pgfpathlineto{\pgfqpoint{5.621845in}{2.551484in}}%
\pgfpathlineto{\pgfqpoint{5.622919in}{2.549521in}}%
\pgfpathlineto{\pgfqpoint{5.627215in}{2.558898in}}%
\pgfpathlineto{\pgfqpoint{5.628289in}{2.559007in}}%
\pgfpathlineto{\pgfqpoint{5.629362in}{2.554046in}}%
\pgfpathlineto{\pgfqpoint{5.630436in}{2.545650in}}%
\pgfpathlineto{\pgfqpoint{5.631510in}{2.550938in}}%
\pgfpathlineto{\pgfqpoint{5.635805in}{2.553283in}}%
\pgfpathlineto{\pgfqpoint{5.636879in}{2.558080in}}%
\pgfpathlineto{\pgfqpoint{5.637953in}{2.555790in}}%
\pgfpathlineto{\pgfqpoint{5.639027in}{2.554864in}}%
\pgfpathlineto{\pgfqpoint{5.642248in}{2.556499in}}%
\pgfpathlineto{\pgfqpoint{5.644396in}{2.553719in}}%
\pgfpathlineto{\pgfqpoint{5.645470in}{2.557535in}}%
\pgfpathlineto{\pgfqpoint{5.646544in}{2.551484in}}%
\pgfpathlineto{\pgfqpoint{5.649765in}{2.547558in}}%
\pgfpathlineto{\pgfqpoint{5.652987in}{2.559988in}}%
\pgfpathlineto{\pgfqpoint{5.654061in}{2.563150in}}%
\pgfpathlineto{\pgfqpoint{5.659430in}{2.560588in}}%
\pgfpathlineto{\pgfqpoint{5.661578in}{2.554645in}}%
\pgfpathlineto{\pgfqpoint{5.664799in}{2.551811in}}%
\pgfpathlineto{\pgfqpoint{5.666947in}{2.552683in}}%
\pgfpathlineto{\pgfqpoint{5.668021in}{2.559552in}}%
\pgfpathlineto{\pgfqpoint{5.669094in}{2.561514in}}%
\pgfpathlineto{\pgfqpoint{5.672316in}{2.562332in}}%
\pgfpathlineto{\pgfqpoint{5.673390in}{2.559497in}}%
\pgfpathlineto{\pgfqpoint{5.675537in}{2.560206in}}%
\pgfpathlineto{\pgfqpoint{5.679833in}{2.558898in}}%
\pgfpathlineto{\pgfqpoint{5.680907in}{2.559770in}}%
\pgfpathlineto{\pgfqpoint{5.681980in}{2.559225in}}%
\pgfpathlineto{\pgfqpoint{5.683054in}{2.557371in}}%
\pgfpathlineto{\pgfqpoint{5.684128in}{2.563368in}}%
\pgfpathlineto{\pgfqpoint{5.688423in}{2.561569in}}%
\pgfpathlineto{\pgfqpoint{5.689497in}{2.564731in}}%
\pgfpathlineto{\pgfqpoint{5.690571in}{2.561951in}}%
\pgfpathlineto{\pgfqpoint{5.691645in}{2.561733in}}%
\pgfpathlineto{\pgfqpoint{5.694867in}{2.559443in}}%
\pgfpathlineto{\pgfqpoint{5.695940in}{2.559934in}}%
\pgfpathlineto{\pgfqpoint{5.697014in}{2.558134in}}%
\pgfpathlineto{\pgfqpoint{5.703457in}{2.565440in}}%
\pgfpathlineto{\pgfqpoint{5.704531in}{2.556935in}}%
\pgfpathlineto{\pgfqpoint{5.705605in}{2.553228in}}%
\pgfpathlineto{\pgfqpoint{5.709900in}{2.556281in}}%
\pgfpathlineto{\pgfqpoint{5.710974in}{2.547122in}}%
\pgfpathlineto{\pgfqpoint{5.712048in}{2.548758in}}%
\pgfpathlineto{\pgfqpoint{5.713122in}{2.548158in}}%
\pgfpathlineto{\pgfqpoint{5.717417in}{2.554100in}}%
\pgfpathlineto{\pgfqpoint{5.718491in}{2.554755in}}%
\pgfpathlineto{\pgfqpoint{5.719565in}{2.557208in}}%
\pgfpathlineto{\pgfqpoint{5.720639in}{2.555736in}}%
\pgfpathlineto{\pgfqpoint{5.721712in}{2.559824in}}%
\pgfpathlineto{\pgfqpoint{5.724934in}{2.559770in}}%
\pgfpathlineto{\pgfqpoint{5.726008in}{2.561514in}}%
\pgfpathlineto{\pgfqpoint{5.727082in}{2.559879in}}%
\pgfpathlineto{\pgfqpoint{5.728155in}{2.561187in}}%
\pgfpathlineto{\pgfqpoint{5.729229in}{2.555354in}}%
\pgfpathlineto{\pgfqpoint{5.732451in}{2.557317in}}%
\pgfpathlineto{\pgfqpoint{5.734599in}{2.548594in}}%
\pgfpathlineto{\pgfqpoint{5.735672in}{2.550230in}}%
\pgfpathlineto{\pgfqpoint{5.736746in}{2.549412in}}%
\pgfpathlineto{\pgfqpoint{5.739968in}{2.550284in}}%
\pgfpathlineto{\pgfqpoint{5.742115in}{2.556608in}}%
\pgfpathlineto{\pgfqpoint{5.743189in}{2.555300in}}%
\pgfpathlineto{\pgfqpoint{5.744263in}{2.556335in}}%
\pgfpathlineto{\pgfqpoint{5.748558in}{2.554427in}}%
\pgfpathlineto{\pgfqpoint{5.749632in}{2.558134in}}%
\pgfpathlineto{\pgfqpoint{5.750706in}{2.558898in}}%
\pgfpathlineto{\pgfqpoint{5.751780in}{2.561460in}}%
\pgfpathlineto{\pgfqpoint{5.755001in}{2.562932in}}%
\pgfpathlineto{\pgfqpoint{5.756075in}{2.560751in}}%
\pgfpathlineto{\pgfqpoint{5.759297in}{2.565058in}}%
\pgfpathlineto{\pgfqpoint{5.762518in}{2.561951in}}%
\pgfpathlineto{\pgfqpoint{5.763592in}{2.565821in}}%
\pgfpathlineto{\pgfqpoint{5.764666in}{2.563859in}}%
\pgfpathlineto{\pgfqpoint{5.765740in}{2.566094in}}%
\pgfpathlineto{\pgfqpoint{5.766814in}{2.564731in}}%
\pgfpathlineto{\pgfqpoint{5.770035in}{2.564295in}}%
\pgfpathlineto{\pgfqpoint{5.773257in}{2.570019in}}%
\pgfpathlineto{\pgfqpoint{5.774331in}{2.560424in}}%
\pgfpathlineto{\pgfqpoint{5.777552in}{2.555354in}}%
\pgfpathlineto{\pgfqpoint{5.780774in}{2.572309in}}%
\pgfpathlineto{\pgfqpoint{5.781847in}{2.572854in}}%
\pgfpathlineto{\pgfqpoint{5.786143in}{2.576070in}}%
\pgfpathlineto{\pgfqpoint{5.788290in}{2.573072in}}%
\pgfpathlineto{\pgfqpoint{5.789364in}{2.577706in}}%
\pgfpathlineto{\pgfqpoint{5.793660in}{2.577597in}}%
\pgfpathlineto{\pgfqpoint{5.794733in}{2.578305in}}%
\pgfpathlineto{\pgfqpoint{5.795807in}{2.578196in}}%
\pgfpathlineto{\pgfqpoint{5.796881in}{2.578905in}}%
\pgfpathlineto{\pgfqpoint{5.800103in}{2.578524in}}%
\pgfpathlineto{\pgfqpoint{5.801177in}{2.579668in}}%
\pgfpathlineto{\pgfqpoint{5.803324in}{2.578469in}}%
\pgfpathlineto{\pgfqpoint{5.804398in}{2.580759in}}%
\pgfpathlineto{\pgfqpoint{5.807620in}{2.581140in}}%
\pgfpathlineto{\pgfqpoint{5.809767in}{2.574544in}}%
\pgfpathlineto{\pgfqpoint{5.810841in}{2.576234in}}%
\pgfpathlineto{\pgfqpoint{5.811915in}{2.580104in}}%
\pgfpathlineto{\pgfqpoint{5.816210in}{2.585938in}}%
\pgfpathlineto{\pgfqpoint{5.817284in}{2.582013in}}%
\pgfpathlineto{\pgfqpoint{5.818358in}{2.582394in}}%
\pgfpathlineto{\pgfqpoint{5.819432in}{2.581086in}}%
\pgfpathlineto{\pgfqpoint{5.822653in}{2.580977in}}%
\pgfpathlineto{\pgfqpoint{5.824801in}{2.583703in}}%
\pgfpathlineto{\pgfqpoint{5.826949in}{2.587301in}}%
\pgfpathlineto{\pgfqpoint{5.830170in}{2.587192in}}%
\pgfpathlineto{\pgfqpoint{5.831244in}{2.585011in}}%
\pgfpathlineto{\pgfqpoint{5.833392in}{2.589318in}}%
\pgfpathlineto{\pgfqpoint{5.837687in}{2.586374in}}%
\pgfpathlineto{\pgfqpoint{5.838761in}{2.589100in}}%
\pgfpathlineto{\pgfqpoint{5.839835in}{2.588663in}}%
\pgfpathlineto{\pgfqpoint{5.840909in}{2.591607in}}%
\pgfpathlineto{\pgfqpoint{5.841982in}{2.589972in}}%
\pgfpathlineto{\pgfqpoint{5.845204in}{2.593570in}}%
\pgfpathlineto{\pgfqpoint{5.846278in}{2.589808in}}%
\pgfpathlineto{\pgfqpoint{5.847352in}{2.588663in}}%
\pgfpathlineto{\pgfqpoint{5.848425in}{2.593624in}}%
\pgfpathlineto{\pgfqpoint{5.849499in}{2.593079in}}%
\pgfpathlineto{\pgfqpoint{5.853795in}{2.595260in}}%
\pgfpathlineto{\pgfqpoint{5.854868in}{2.591825in}}%
\pgfpathlineto{\pgfqpoint{5.855942in}{2.591008in}}%
\pgfpathlineto{\pgfqpoint{5.857016in}{2.583375in}}%
\pgfpathlineto{\pgfqpoint{5.860238in}{2.593297in}}%
\pgfpathlineto{\pgfqpoint{5.861311in}{2.587355in}}%
\pgfpathlineto{\pgfqpoint{5.862385in}{2.587192in}}%
\pgfpathlineto{\pgfqpoint{5.863459in}{2.592371in}}%
\pgfpathlineto{\pgfqpoint{5.864533in}{2.592316in}}%
\pgfpathlineto{\pgfqpoint{5.868828in}{2.594933in}}%
\pgfpathlineto{\pgfqpoint{5.869902in}{2.591062in}}%
\pgfpathlineto{\pgfqpoint{5.870976in}{2.597004in}}%
\pgfpathlineto{\pgfqpoint{5.872050in}{2.590899in}}%
\pgfpathlineto{\pgfqpoint{5.875271in}{2.591335in}}%
\pgfpathlineto{\pgfqpoint{5.876345in}{2.593843in}}%
\pgfpathlineto{\pgfqpoint{5.877419in}{2.599294in}}%
\pgfpathlineto{\pgfqpoint{5.878493in}{2.593243in}}%
\pgfpathlineto{\pgfqpoint{5.879567in}{2.600766in}}%
\pgfpathlineto{\pgfqpoint{5.883862in}{2.593843in}}%
\pgfpathlineto{\pgfqpoint{5.887084in}{2.602020in}}%
\pgfpathlineto{\pgfqpoint{5.890305in}{2.597332in}}%
\pgfpathlineto{\pgfqpoint{5.891379in}{2.594769in}}%
\pgfpathlineto{\pgfqpoint{5.892453in}{2.594933in}}%
\pgfpathlineto{\pgfqpoint{5.893527in}{2.593243in}}%
\pgfpathlineto{\pgfqpoint{5.894600in}{2.594224in}}%
\pgfpathlineto{\pgfqpoint{5.897822in}{2.591226in}}%
\pgfpathlineto{\pgfqpoint{5.898896in}{2.589372in}}%
\pgfpathlineto{\pgfqpoint{5.901044in}{2.580104in}}%
\pgfpathlineto{\pgfqpoint{5.902117in}{2.577106in}}%
\pgfpathlineto{\pgfqpoint{5.905339in}{2.575961in}}%
\pgfpathlineto{\pgfqpoint{5.906413in}{2.590299in}}%
\pgfpathlineto{\pgfqpoint{5.907487in}{2.592425in}}%
\pgfpathlineto{\pgfqpoint{5.908560in}{2.588336in}}%
\pgfpathlineto{\pgfqpoint{5.909634in}{2.589645in}}%
\pgfpathlineto{\pgfqpoint{5.915003in}{2.589154in}}%
\pgfpathlineto{\pgfqpoint{5.916077in}{2.588445in}}%
\pgfpathlineto{\pgfqpoint{5.917151in}{2.580868in}}%
\pgfpathlineto{\pgfqpoint{5.920373in}{2.588227in}}%
\pgfpathlineto{\pgfqpoint{5.921446in}{2.592752in}}%
\pgfpathlineto{\pgfqpoint{5.922520in}{2.585120in}}%
\pgfpathlineto{\pgfqpoint{5.923594in}{2.570237in}}%
\pgfpathlineto{\pgfqpoint{5.924668in}{2.573344in}}%
\pgfpathlineto{\pgfqpoint{5.927889in}{2.570455in}}%
\pgfpathlineto{\pgfqpoint{5.928963in}{2.573563in}}%
\pgfpathlineto{\pgfqpoint{5.930037in}{2.571382in}}%
\pgfpathlineto{\pgfqpoint{5.931111in}{2.570782in}}%
\pgfpathlineto{\pgfqpoint{5.932185in}{2.565440in}}%
\pgfpathlineto{\pgfqpoint{5.936480in}{2.569256in}}%
\pgfpathlineto{\pgfqpoint{5.937554in}{2.568874in}}%
\pgfpathlineto{\pgfqpoint{5.939702in}{2.572745in}}%
\pgfpathlineto{\pgfqpoint{5.943997in}{2.569910in}}%
\pgfpathlineto{\pgfqpoint{5.946145in}{2.564731in}}%
\pgfpathlineto{\pgfqpoint{5.947219in}{2.567457in}}%
\pgfpathlineto{\pgfqpoint{5.950440in}{2.570401in}}%
\pgfpathlineto{\pgfqpoint{5.951514in}{2.570019in}}%
\pgfpathlineto{\pgfqpoint{5.952588in}{2.576343in}}%
\pgfpathlineto{\pgfqpoint{5.953662in}{2.572963in}}%
\pgfpathlineto{\pgfqpoint{5.954735in}{2.577324in}}%
\pgfpathlineto{\pgfqpoint{5.957957in}{2.581086in}}%
\pgfpathlineto{\pgfqpoint{5.959031in}{2.581358in}}%
\pgfpathlineto{\pgfqpoint{5.960105in}{2.577324in}}%
\pgfpathlineto{\pgfqpoint{5.961178in}{2.578851in}}%
\pgfpathlineto{\pgfqpoint{5.966548in}{2.578305in}}%
\pgfpathlineto{\pgfqpoint{5.967621in}{2.576834in}}%
\pgfpathlineto{\pgfqpoint{5.968695in}{2.577815in}}%
\pgfpathlineto{\pgfqpoint{5.969769in}{2.580268in}}%
\pgfpathlineto{\pgfqpoint{5.974065in}{2.578469in}}%
\pgfpathlineto{\pgfqpoint{5.975138in}{2.575798in}}%
\pgfpathlineto{\pgfqpoint{5.976212in}{2.577215in}}%
\pgfpathlineto{\pgfqpoint{5.977286in}{2.575852in}}%
\pgfpathlineto{\pgfqpoint{5.981581in}{2.576452in}}%
\pgfpathlineto{\pgfqpoint{5.982655in}{2.577978in}}%
\pgfpathlineto{\pgfqpoint{5.983729in}{2.580759in}}%
\pgfpathlineto{\pgfqpoint{5.984803in}{2.580595in}}%
\pgfpathlineto{\pgfqpoint{5.988024in}{2.577433in}}%
\pgfpathlineto{\pgfqpoint{5.989098in}{2.572908in}}%
\pgfpathlineto{\pgfqpoint{5.991246in}{2.574653in}}%
\pgfpathlineto{\pgfqpoint{5.992320in}{2.575525in}}%
\pgfpathlineto{\pgfqpoint{5.997689in}{2.583484in}}%
\pgfpathlineto{\pgfqpoint{5.998763in}{2.582285in}}%
\pgfpathlineto{\pgfqpoint{5.999837in}{2.596132in}}%
\pgfpathlineto{\pgfqpoint{6.003058in}{2.593679in}}%
\pgfpathlineto{\pgfqpoint{6.004132in}{2.598204in}}%
\pgfpathlineto{\pgfqpoint{6.006280in}{2.591880in}}%
\pgfpathlineto{\pgfqpoint{6.007354in}{2.592480in}}%
\pgfpathlineto{\pgfqpoint{6.010575in}{2.592643in}}%
\pgfpathlineto{\pgfqpoint{6.011649in}{2.596895in}}%
\pgfpathlineto{\pgfqpoint{6.012723in}{2.595533in}}%
\pgfpathlineto{\pgfqpoint{6.013797in}{2.597713in}}%
\pgfpathlineto{\pgfqpoint{6.014870in}{2.595969in}}%
\pgfpathlineto{\pgfqpoint{6.018092in}{2.595914in}}%
\pgfpathlineto{\pgfqpoint{6.020240in}{2.600603in}}%
\pgfpathlineto{\pgfqpoint{6.021313in}{2.602293in}}%
\pgfpathlineto{\pgfqpoint{6.022387in}{2.598803in}}%
\pgfpathlineto{\pgfqpoint{6.025609in}{2.600493in}}%
\pgfpathlineto{\pgfqpoint{6.026683in}{2.598204in}}%
\pgfpathlineto{\pgfqpoint{6.027756in}{2.614668in}}%
\pgfpathlineto{\pgfqpoint{6.028830in}{2.612978in}}%
\pgfpathlineto{\pgfqpoint{6.029904in}{2.614504in}}%
\pgfpathlineto{\pgfqpoint{6.034199in}{2.617393in}}%
\pgfpathlineto{\pgfqpoint{6.037421in}{2.614286in}}%
\pgfpathlineto{\pgfqpoint{6.040643in}{2.613468in}}%
\pgfpathlineto{\pgfqpoint{6.041716in}{2.614395in}}%
\pgfpathlineto{\pgfqpoint{6.042790in}{2.617393in}}%
\pgfpathlineto{\pgfqpoint{6.044938in}{2.611506in}}%
\pgfpathlineto{\pgfqpoint{6.049233in}{2.610470in}}%
\pgfpathlineto{\pgfqpoint{6.050307in}{2.609707in}}%
\pgfpathlineto{\pgfqpoint{6.051381in}{2.610743in}}%
\pgfpathlineto{\pgfqpoint{6.052455in}{2.614395in}}%
\pgfpathlineto{\pgfqpoint{6.055676in}{2.615594in}}%
\pgfpathlineto{\pgfqpoint{6.056750in}{2.614068in}}%
\pgfpathlineto{\pgfqpoint{6.057824in}{2.616085in}}%
\pgfpathlineto{\pgfqpoint{6.058898in}{2.616249in}}%
\pgfpathlineto{\pgfqpoint{6.059972in}{2.614068in}}%
\pgfpathlineto{\pgfqpoint{6.064267in}{2.614995in}}%
\pgfpathlineto{\pgfqpoint{6.067488in}{2.611887in}}%
\pgfpathlineto{\pgfqpoint{6.070710in}{2.611451in}}%
\pgfpathlineto{\pgfqpoint{6.071784in}{2.612814in}}%
\pgfpathlineto{\pgfqpoint{6.072858in}{2.612051in}}%
\pgfpathlineto{\pgfqpoint{6.075005in}{2.608235in}}%
\pgfpathlineto{\pgfqpoint{6.078227in}{2.607417in}}%
\pgfpathlineto{\pgfqpoint{6.079301in}{2.608562in}}%
\pgfpathlineto{\pgfqpoint{6.080375in}{2.608834in}}%
\pgfpathlineto{\pgfqpoint{6.081448in}{2.606000in}}%
\pgfpathlineto{\pgfqpoint{6.082522in}{2.605127in}}%
\pgfpathlineto{\pgfqpoint{6.085744in}{2.606436in}}%
\pgfpathlineto{\pgfqpoint{6.087891in}{2.610579in}}%
\pgfpathlineto{\pgfqpoint{6.088965in}{2.609162in}}%
\pgfpathlineto{\pgfqpoint{6.093261in}{2.610961in}}%
\pgfpathlineto{\pgfqpoint{6.094334in}{2.613032in}}%
\pgfpathlineto{\pgfqpoint{6.097556in}{2.605454in}}%
\pgfpathlineto{\pgfqpoint{6.101851in}{2.612487in}}%
\pgfpathlineto{\pgfqpoint{6.102925in}{2.600984in}}%
\pgfpathlineto{\pgfqpoint{6.103999in}{2.600766in}}%
\pgfpathlineto{\pgfqpoint{6.105073in}{2.598913in}}%
\pgfpathlineto{\pgfqpoint{6.108294in}{2.597713in}}%
\pgfpathlineto{\pgfqpoint{6.109368in}{2.593297in}}%
\pgfpathlineto{\pgfqpoint{6.110442in}{2.594333in}}%
\pgfpathlineto{\pgfqpoint{6.115811in}{2.594987in}}%
\pgfpathlineto{\pgfqpoint{6.116885in}{2.594224in}}%
\pgfpathlineto{\pgfqpoint{6.117959in}{2.594660in}}%
\pgfpathlineto{\pgfqpoint{6.119033in}{2.593025in}}%
\pgfpathlineto{\pgfqpoint{6.125476in}{2.593515in}}%
\pgfpathlineto{\pgfqpoint{6.126550in}{2.591498in}}%
\pgfpathlineto{\pgfqpoint{6.127623in}{2.593406in}}%
\pgfpathlineto{\pgfqpoint{6.130845in}{2.593243in}}%
\pgfpathlineto{\pgfqpoint{6.131919in}{2.592589in}}%
\pgfpathlineto{\pgfqpoint{6.135140in}{2.598531in}}%
\pgfpathlineto{\pgfqpoint{6.139436in}{2.599294in}}%
\pgfpathlineto{\pgfqpoint{6.140509in}{2.602783in}}%
\pgfpathlineto{\pgfqpoint{6.141583in}{2.603001in}}%
\pgfpathlineto{\pgfqpoint{6.142657in}{2.605291in}}%
\pgfpathlineto{\pgfqpoint{6.148026in}{2.606218in}}%
\pgfpathlineto{\pgfqpoint{6.149100in}{2.601584in}}%
\pgfpathlineto{\pgfqpoint{6.150174in}{2.603110in}}%
\pgfpathlineto{\pgfqpoint{6.153396in}{2.603546in}}%
\pgfpathlineto{\pgfqpoint{6.154469in}{2.602620in}}%
\pgfpathlineto{\pgfqpoint{6.155543in}{2.604582in}}%
\pgfpathlineto{\pgfqpoint{6.156617in}{2.609325in}}%
\pgfpathlineto{\pgfqpoint{6.157691in}{2.610743in}}%
\pgfpathlineto{\pgfqpoint{6.160912in}{2.611778in}}%
\pgfpathlineto{\pgfqpoint{6.164134in}{2.607526in}}%
\pgfpathlineto{\pgfqpoint{6.165208in}{2.609543in}}%
\pgfpathlineto{\pgfqpoint{6.168429in}{2.609216in}}%
\pgfpathlineto{\pgfqpoint{6.169503in}{2.605400in}}%
\pgfpathlineto{\pgfqpoint{6.170577in}{2.604201in}}%
\pgfpathlineto{\pgfqpoint{6.171651in}{2.597223in}}%
\pgfpathlineto{\pgfqpoint{6.172725in}{2.597986in}}%
\pgfpathlineto{\pgfqpoint{6.175946in}{2.600984in}}%
\pgfpathlineto{\pgfqpoint{6.178094in}{2.600657in}}%
\pgfpathlineto{\pgfqpoint{6.179168in}{2.599131in}}%
\pgfpathlineto{\pgfqpoint{6.180242in}{2.600548in}}%
\pgfpathlineto{\pgfqpoint{6.184537in}{2.596078in}}%
\pgfpathlineto{\pgfqpoint{6.185611in}{2.597059in}}%
\pgfpathlineto{\pgfqpoint{6.186685in}{2.595751in}}%
\pgfpathlineto{\pgfqpoint{6.187758in}{2.597768in}}%
\pgfpathlineto{\pgfqpoint{6.190980in}{2.600057in}}%
\pgfpathlineto{\pgfqpoint{6.192054in}{2.605563in}}%
\pgfpathlineto{\pgfqpoint{6.194201in}{2.608889in}}%
\pgfpathlineto{\pgfqpoint{6.195275in}{2.608943in}}%
\pgfpathlineto{\pgfqpoint{6.198497in}{2.606763in}}%
\pgfpathlineto{\pgfqpoint{6.199571in}{2.611669in}}%
\pgfpathlineto{\pgfqpoint{6.200644in}{2.612487in}}%
\pgfpathlineto{\pgfqpoint{6.201718in}{2.619520in}}%
\pgfpathlineto{\pgfqpoint{6.202792in}{2.617121in}}%
\pgfpathlineto{\pgfqpoint{6.207087in}{2.621700in}}%
\pgfpathlineto{\pgfqpoint{6.208161in}{2.621373in}}%
\pgfpathlineto{\pgfqpoint{6.210309in}{2.619465in}}%
\pgfpathlineto{\pgfqpoint{6.215678in}{2.625898in}}%
\pgfpathlineto{\pgfqpoint{6.216752in}{2.624971in}}%
\pgfpathlineto{\pgfqpoint{6.217826in}{2.622900in}}%
\pgfpathlineto{\pgfqpoint{6.221047in}{2.624481in}}%
\pgfpathlineto{\pgfqpoint{6.222121in}{2.627315in}}%
\pgfpathlineto{\pgfqpoint{6.223195in}{2.628569in}}%
\pgfpathlineto{\pgfqpoint{6.224269in}{2.626661in}}%
\pgfpathlineto{\pgfqpoint{6.225343in}{2.628678in}}%
\pgfpathlineto{\pgfqpoint{6.228564in}{2.630695in}}%
\pgfpathlineto{\pgfqpoint{6.229638in}{2.630314in}}%
\pgfpathlineto{\pgfqpoint{6.231786in}{2.627697in}}%
\pgfpathlineto{\pgfqpoint{6.232860in}{2.628896in}}%
\pgfpathlineto{\pgfqpoint{6.236081in}{2.628678in}}%
\pgfpathlineto{\pgfqpoint{6.237155in}{2.627915in}}%
\pgfpathlineto{\pgfqpoint{6.238229in}{2.625625in}}%
\pgfpathlineto{\pgfqpoint{6.240376in}{2.629005in}}%
\pgfpathlineto{\pgfqpoint{6.245746in}{2.629987in}}%
\pgfpathlineto{\pgfqpoint{6.246820in}{2.631241in}}%
\pgfpathlineto{\pgfqpoint{6.247893in}{2.630586in}}%
\pgfpathlineto{\pgfqpoint{6.251115in}{2.636474in}}%
\pgfpathlineto{\pgfqpoint{6.252189in}{2.634021in}}%
\pgfpathlineto{\pgfqpoint{6.254336in}{2.634239in}}%
\pgfpathlineto{\pgfqpoint{6.255410in}{2.632767in}}%
\pgfpathlineto{\pgfqpoint{6.258632in}{2.632167in}}%
\pgfpathlineto{\pgfqpoint{6.259706in}{2.637401in}}%
\pgfpathlineto{\pgfqpoint{6.260779in}{2.638546in}}%
\pgfpathlineto{\pgfqpoint{6.261853in}{2.629551in}}%
\pgfpathlineto{\pgfqpoint{6.262927in}{2.627533in}}%
\pgfpathlineto{\pgfqpoint{6.266149in}{2.629987in}}%
\pgfpathlineto{\pgfqpoint{6.267222in}{2.629605in}}%
\pgfpathlineto{\pgfqpoint{6.268296in}{2.620501in}}%
\pgfpathlineto{\pgfqpoint{6.270444in}{2.621046in}}%
\pgfpathlineto{\pgfqpoint{6.275813in}{2.628460in}}%
\pgfpathlineto{\pgfqpoint{6.276887in}{2.626443in}}%
\pgfpathlineto{\pgfqpoint{6.277961in}{2.627970in}}%
\pgfpathlineto{\pgfqpoint{6.281182in}{2.626879in}}%
\pgfpathlineto{\pgfqpoint{6.282256in}{2.624372in}}%
\pgfpathlineto{\pgfqpoint{6.283330in}{2.623554in}}%
\pgfpathlineto{\pgfqpoint{6.285478in}{2.631622in}}%
\pgfpathlineto{\pgfqpoint{6.288699in}{2.632113in}}%
\pgfpathlineto{\pgfqpoint{6.289773in}{2.630368in}}%
\pgfpathlineto{\pgfqpoint{6.290847in}{2.630205in}}%
\pgfpathlineto{\pgfqpoint{6.291921in}{2.627697in}}%
\pgfpathlineto{\pgfqpoint{6.292995in}{2.610470in}}%
\pgfpathlineto{\pgfqpoint{6.297290in}{2.603928in}}%
\pgfpathlineto{\pgfqpoint{6.298364in}{2.603328in}}%
\pgfpathlineto{\pgfqpoint{6.299438in}{2.606599in}}%
\pgfpathlineto{\pgfqpoint{6.300511in}{2.604255in}}%
\pgfpathlineto{\pgfqpoint{6.303733in}{2.600275in}}%
\pgfpathlineto{\pgfqpoint{6.304807in}{2.600657in}}%
\pgfpathlineto{\pgfqpoint{6.305881in}{2.603492in}}%
\pgfpathlineto{\pgfqpoint{6.306954in}{2.601529in}}%
\pgfpathlineto{\pgfqpoint{6.308028in}{2.601856in}}%
\pgfpathlineto{\pgfqpoint{6.311250in}{2.599131in}}%
\pgfpathlineto{\pgfqpoint{6.313397in}{2.607035in}}%
\pgfpathlineto{\pgfqpoint{6.314471in}{2.608071in}}%
\pgfpathlineto{\pgfqpoint{6.315545in}{2.610034in}}%
\pgfpathlineto{\pgfqpoint{6.318767in}{2.614341in}}%
\pgfpathlineto{\pgfqpoint{6.319841in}{2.613686in}}%
\pgfpathlineto{\pgfqpoint{6.320914in}{2.610361in}}%
\pgfpathlineto{\pgfqpoint{6.321988in}{2.615649in}}%
\pgfpathlineto{\pgfqpoint{6.323062in}{2.611397in}}%
\pgfpathlineto{\pgfqpoint{6.326284in}{2.610579in}}%
\pgfpathlineto{\pgfqpoint{6.327357in}{2.612923in}}%
\pgfpathlineto{\pgfqpoint{6.328431in}{2.610906in}}%
\pgfpathlineto{\pgfqpoint{6.330579in}{2.611506in}}%
\pgfpathlineto{\pgfqpoint{6.333800in}{2.614123in}}%
\pgfpathlineto{\pgfqpoint{6.334874in}{2.616412in}}%
\pgfpathlineto{\pgfqpoint{6.335948in}{2.616303in}}%
\pgfpathlineto{\pgfqpoint{6.338096in}{2.621373in}}%
\pgfpathlineto{\pgfqpoint{6.341317in}{2.626770in}}%
\pgfpathlineto{\pgfqpoint{6.342391in}{2.626716in}}%
\pgfpathlineto{\pgfqpoint{6.343465in}{2.625952in}}%
\pgfpathlineto{\pgfqpoint{6.344539in}{2.620010in}}%
\pgfpathlineto{\pgfqpoint{6.345613in}{2.621428in}}%
\pgfpathlineto{\pgfqpoint{6.348834in}{2.620719in}}%
\pgfpathlineto{\pgfqpoint{6.349908in}{2.618756in}}%
\pgfpathlineto{\pgfqpoint{6.350982in}{2.624044in}}%
\pgfpathlineto{\pgfqpoint{6.352056in}{2.624644in}}%
\pgfpathlineto{\pgfqpoint{6.353130in}{2.629223in}}%
\pgfpathlineto{\pgfqpoint{6.356351in}{2.629223in}}%
\pgfpathlineto{\pgfqpoint{6.358499in}{2.627370in}}%
\pgfpathlineto{\pgfqpoint{6.359573in}{2.628133in}}%
\pgfpathlineto{\pgfqpoint{6.360646in}{2.630477in}}%
\pgfpathlineto{\pgfqpoint{6.364942in}{2.632276in}}%
\pgfpathlineto{\pgfqpoint{6.366016in}{2.630314in}}%
\pgfpathlineto{\pgfqpoint{6.367089in}{2.630150in}}%
\pgfpathlineto{\pgfqpoint{6.368163in}{2.629223in}}%
\pgfpathlineto{\pgfqpoint{6.368163in}{2.629223in}}%
\pgfusepath{stroke}%
\end{pgfscope}%
\begin{pgfscope}%
\pgfpathrectangle{\pgfqpoint{3.902254in}{2.255883in}}{\pgfqpoint{2.583333in}{0.400885in}}%
\pgfusepath{clip}%
\pgfsetroundcap%
\pgfsetroundjoin%
\pgfsetlinewidth{1.505625pt}%
\definecolor{currentstroke}{rgb}{0.549020,0.337255,0.294118}%
\pgfsetstrokecolor{currentstroke}%
\pgfsetdash{}{0pt}%
\pgfpathmoveto{\pgfqpoint{4.019678in}{2.440434in}}%
\pgfpathlineto{\pgfqpoint{4.020752in}{2.440271in}}%
\pgfpathlineto{\pgfqpoint{4.022900in}{2.438417in}}%
\pgfpathlineto{\pgfqpoint{4.026121in}{2.437218in}}%
\pgfpathlineto{\pgfqpoint{4.027195in}{2.435866in}}%
\pgfpathlineto{\pgfqpoint{4.028269in}{2.433164in}}%
\pgfpathlineto{\pgfqpoint{4.030417in}{2.433704in}}%
\pgfpathlineto{\pgfqpoint{4.034712in}{2.435596in}}%
\pgfpathlineto{\pgfqpoint{4.035786in}{2.434353in}}%
\pgfpathlineto{\pgfqpoint{4.037934in}{2.433497in}}%
\pgfpathlineto{\pgfqpoint{4.042229in}{2.426255in}}%
\pgfpathlineto{\pgfqpoint{4.043303in}{2.428278in}}%
\pgfpathlineto{\pgfqpoint{4.044377in}{2.427479in}}%
\pgfpathlineto{\pgfqpoint{4.045451in}{2.425403in}}%
\pgfpathlineto{\pgfqpoint{4.049746in}{2.420131in}}%
\pgfpathlineto{\pgfqpoint{4.051894in}{2.421303in}}%
\pgfpathlineto{\pgfqpoint{4.052967in}{2.418960in}}%
\pgfpathlineto{\pgfqpoint{4.057263in}{2.422900in}}%
\pgfpathlineto{\pgfqpoint{4.058337in}{2.422634in}}%
\pgfpathlineto{\pgfqpoint{4.059410in}{2.424284in}}%
\pgfpathlineto{\pgfqpoint{4.060484in}{2.423646in}}%
\pgfpathlineto{\pgfqpoint{4.065853in}{2.426468in}}%
\pgfpathlineto{\pgfqpoint{4.066927in}{2.423752in}}%
\pgfpathlineto{\pgfqpoint{4.068001in}{2.424952in}}%
\pgfpathlineto{\pgfqpoint{4.072296in}{2.422899in}}%
\pgfpathlineto{\pgfqpoint{4.073370in}{2.423004in}}%
\pgfpathlineto{\pgfqpoint{4.074444in}{2.414789in}}%
\pgfpathlineto{\pgfqpoint{4.075518in}{2.413650in}}%
\pgfpathlineto{\pgfqpoint{4.078740in}{2.413699in}}%
\pgfpathlineto{\pgfqpoint{4.079813in}{2.411047in}}%
\pgfpathlineto{\pgfqpoint{4.080887in}{2.410133in}}%
\pgfpathlineto{\pgfqpoint{4.081961in}{2.413763in}}%
\pgfpathlineto{\pgfqpoint{4.083035in}{2.413714in}}%
\pgfpathlineto{\pgfqpoint{4.086256in}{2.412633in}}%
\pgfpathlineto{\pgfqpoint{4.088404in}{2.414049in}}%
\pgfpathlineto{\pgfqpoint{4.089478in}{2.412818in}}%
\pgfpathlineto{\pgfqpoint{4.090552in}{2.412720in}}%
\pgfpathlineto{\pgfqpoint{4.094847in}{2.408983in}}%
\pgfpathlineto{\pgfqpoint{4.096995in}{2.409837in}}%
\pgfpathlineto{\pgfqpoint{4.098069in}{2.411457in}}%
\pgfpathlineto{\pgfqpoint{4.103438in}{2.411651in}}%
\pgfpathlineto{\pgfqpoint{4.104512in}{2.410393in}}%
\pgfpathlineto{\pgfqpoint{4.105585in}{2.410729in}}%
\pgfpathlineto{\pgfqpoint{4.108807in}{2.410632in}}%
\pgfpathlineto{\pgfqpoint{4.109881in}{2.411784in}}%
\pgfpathlineto{\pgfqpoint{4.110955in}{2.411639in}}%
\pgfpathlineto{\pgfqpoint{4.112029in}{2.412364in}}%
\pgfpathlineto{\pgfqpoint{4.113102in}{2.411586in}}%
\pgfpathlineto{\pgfqpoint{4.116324in}{2.410232in}}%
\pgfpathlineto{\pgfqpoint{4.117398in}{2.412002in}}%
\pgfpathlineto{\pgfqpoint{4.119545in}{2.411179in}}%
\pgfpathlineto{\pgfqpoint{4.123841in}{2.413058in}}%
\pgfpathlineto{\pgfqpoint{4.124915in}{2.414867in}}%
\pgfpathlineto{\pgfqpoint{4.125988in}{2.414471in}}%
\pgfpathlineto{\pgfqpoint{4.127062in}{2.415855in}}%
\pgfpathlineto{\pgfqpoint{4.132431in}{2.410035in}}%
\pgfpathlineto{\pgfqpoint{4.134579in}{2.411727in}}%
\pgfpathlineto{\pgfqpoint{4.135653in}{2.408114in}}%
\pgfpathlineto{\pgfqpoint{4.138874in}{2.411342in}}%
\pgfpathlineto{\pgfqpoint{4.141022in}{2.408244in}}%
\pgfpathlineto{\pgfqpoint{4.142096in}{2.408339in}}%
\pgfpathlineto{\pgfqpoint{4.143170in}{2.399159in}}%
\pgfpathlineto{\pgfqpoint{4.146391in}{2.396115in}}%
\pgfpathlineto{\pgfqpoint{4.147465in}{2.395877in}}%
\pgfpathlineto{\pgfqpoint{4.149613in}{2.399397in}}%
\pgfpathlineto{\pgfqpoint{4.150687in}{2.398541in}}%
\pgfpathlineto{\pgfqpoint{4.154982in}{2.398113in}}%
\pgfpathlineto{\pgfqpoint{4.156056in}{2.396258in}}%
\pgfpathlineto{\pgfqpoint{4.157130in}{2.398018in}}%
\pgfpathlineto{\pgfqpoint{4.158204in}{2.396305in}}%
\pgfpathlineto{\pgfqpoint{4.161425in}{2.395925in}}%
\pgfpathlineto{\pgfqpoint{4.162499in}{2.396448in}}%
\pgfpathlineto{\pgfqpoint{4.163573in}{2.398588in}}%
\pgfpathlineto{\pgfqpoint{4.165720in}{2.395687in}}%
\pgfpathlineto{\pgfqpoint{4.168942in}{2.395164in}}%
\pgfpathlineto{\pgfqpoint{4.170016in}{2.394260in}}%
\pgfpathlineto{\pgfqpoint{4.171090in}{2.391406in}}%
\pgfpathlineto{\pgfqpoint{4.172163in}{2.392072in}}%
\pgfpathlineto{\pgfqpoint{4.173237in}{2.391787in}}%
\pgfpathlineto{\pgfqpoint{4.177533in}{2.393547in}}%
\pgfpathlineto{\pgfqpoint{4.178607in}{2.391121in}}%
\pgfpathlineto{\pgfqpoint{4.179680in}{2.391026in}}%
\pgfpathlineto{\pgfqpoint{4.180754in}{2.388220in}}%
\pgfpathlineto{\pgfqpoint{4.183976in}{2.387649in}}%
\pgfpathlineto{\pgfqpoint{4.185050in}{2.386793in}}%
\pgfpathlineto{\pgfqpoint{4.188271in}{2.392786in}}%
\pgfpathlineto{\pgfqpoint{4.191493in}{2.391977in}}%
\pgfpathlineto{\pgfqpoint{4.192566in}{2.392786in}}%
\pgfpathlineto{\pgfqpoint{4.193640in}{2.392072in}}%
\pgfpathlineto{\pgfqpoint{4.194714in}{2.394403in}}%
\pgfpathlineto{\pgfqpoint{4.195788in}{2.393261in}}%
\pgfpathlineto{\pgfqpoint{4.200083in}{2.390741in}}%
\pgfpathlineto{\pgfqpoint{4.201157in}{2.383844in}}%
\pgfpathlineto{\pgfqpoint{4.202231in}{2.381418in}}%
\pgfpathlineto{\pgfqpoint{4.203305in}{2.381751in}}%
\pgfpathlineto{\pgfqpoint{4.207600in}{2.379611in}}%
\pgfpathlineto{\pgfqpoint{4.208674in}{2.382322in}}%
\pgfpathlineto{\pgfqpoint{4.209748in}{2.383463in}}%
\pgfpathlineto{\pgfqpoint{4.210822in}{2.387078in}}%
\pgfpathlineto{\pgfqpoint{4.214043in}{2.386888in}}%
\pgfpathlineto{\pgfqpoint{4.215117in}{2.387506in}}%
\pgfpathlineto{\pgfqpoint{4.218339in}{2.387221in}}%
\pgfpathlineto{\pgfqpoint{4.222634in}{2.388933in}}%
\pgfpathlineto{\pgfqpoint{4.223708in}{2.387649in}}%
\pgfpathlineto{\pgfqpoint{4.224782in}{2.396353in}}%
\pgfpathlineto{\pgfqpoint{4.225855in}{2.391073in}}%
\pgfpathlineto{\pgfqpoint{4.229077in}{2.392120in}}%
\pgfpathlineto{\pgfqpoint{4.230151in}{2.390146in}}%
\pgfpathlineto{\pgfqpoint{4.232298in}{2.389649in}}%
\pgfpathlineto{\pgfqpoint{4.233372in}{2.390324in}}%
\pgfpathlineto{\pgfqpoint{4.236594in}{2.391590in}}%
\pgfpathlineto{\pgfqpoint{4.237668in}{2.392917in}}%
\pgfpathlineto{\pgfqpoint{4.238741in}{2.392917in}}%
\pgfpathlineto{\pgfqpoint{4.240889in}{2.389018in}}%
\pgfpathlineto{\pgfqpoint{4.244111in}{2.388973in}}%
\pgfpathlineto{\pgfqpoint{4.246258in}{2.392897in}}%
\pgfpathlineto{\pgfqpoint{4.247332in}{2.391047in}}%
\pgfpathlineto{\pgfqpoint{4.248406in}{2.383693in}}%
\pgfpathlineto{\pgfqpoint{4.251628in}{2.382607in}}%
\pgfpathlineto{\pgfqpoint{4.253775in}{2.379487in}}%
\pgfpathlineto{\pgfqpoint{4.255923in}{2.379319in}}%
\pgfpathlineto{\pgfqpoint{4.259144in}{2.380281in}}%
\pgfpathlineto{\pgfqpoint{4.260218in}{2.379438in}}%
\pgfpathlineto{\pgfqpoint{4.261292in}{2.379731in}}%
\pgfpathlineto{\pgfqpoint{4.262366in}{2.378514in}}%
\pgfpathlineto{\pgfqpoint{4.263440in}{2.378514in}}%
\pgfpathlineto{\pgfqpoint{4.267735in}{2.379302in}}%
\pgfpathlineto{\pgfqpoint{4.268809in}{2.379010in}}%
\pgfpathlineto{\pgfqpoint{4.269883in}{2.379593in}}%
\pgfpathlineto{\pgfqpoint{4.270957in}{2.378462in}}%
\pgfpathlineto{\pgfqpoint{4.275252in}{2.378544in}}%
\pgfpathlineto{\pgfqpoint{4.277400in}{2.378917in}}%
\pgfpathlineto{\pgfqpoint{4.278473in}{2.377876in}}%
\pgfpathlineto{\pgfqpoint{4.283843in}{2.377543in}}%
\pgfpathlineto{\pgfqpoint{4.284917in}{2.374416in}}%
\pgfpathlineto{\pgfqpoint{4.285990in}{2.373536in}}%
\pgfpathlineto{\pgfqpoint{4.289212in}{2.373536in}}%
\pgfpathlineto{\pgfqpoint{4.291360in}{2.374849in}}%
\pgfpathlineto{\pgfqpoint{4.292433in}{2.372238in}}%
\pgfpathlineto{\pgfqpoint{4.293507in}{2.371454in}}%
\pgfpathlineto{\pgfqpoint{4.298876in}{2.371143in}}%
\pgfpathlineto{\pgfqpoint{4.299950in}{2.370211in}}%
\pgfpathlineto{\pgfqpoint{4.301024in}{2.370634in}}%
\pgfpathlineto{\pgfqpoint{4.304246in}{2.369591in}}%
\pgfpathlineto{\pgfqpoint{4.307467in}{2.371011in}}%
\pgfpathlineto{\pgfqpoint{4.311762in}{2.370546in}}%
\pgfpathlineto{\pgfqpoint{4.312836in}{2.372593in}}%
\pgfpathlineto{\pgfqpoint{4.314984in}{2.370750in}}%
\pgfpathlineto{\pgfqpoint{4.316058in}{2.369977in}}%
\pgfpathlineto{\pgfqpoint{4.319279in}{2.371590in}}%
\pgfpathlineto{\pgfqpoint{4.322501in}{2.375054in}}%
\pgfpathlineto{\pgfqpoint{4.323575in}{2.375255in}}%
\pgfpathlineto{\pgfqpoint{4.327870in}{2.371848in}}%
\pgfpathlineto{\pgfqpoint{4.328944in}{2.368603in}}%
\pgfpathlineto{\pgfqpoint{4.330018in}{2.368603in}}%
\pgfpathlineto{\pgfqpoint{4.331092in}{2.371334in}}%
\pgfpathlineto{\pgfqpoint{4.334313in}{2.371568in}}%
\pgfpathlineto{\pgfqpoint{4.335387in}{2.374923in}}%
\pgfpathlineto{\pgfqpoint{4.336461in}{2.372830in}}%
\pgfpathlineto{\pgfqpoint{4.337535in}{2.366474in}}%
\pgfpathlineto{\pgfqpoint{4.338608in}{2.368373in}}%
\pgfpathlineto{\pgfqpoint{4.346125in}{2.369132in}}%
\pgfpathlineto{\pgfqpoint{4.349347in}{2.370736in}}%
\pgfpathlineto{\pgfqpoint{4.350421in}{2.369922in}}%
\pgfpathlineto{\pgfqpoint{4.352568in}{2.376324in}}%
\pgfpathlineto{\pgfqpoint{4.353642in}{2.375997in}}%
\pgfpathlineto{\pgfqpoint{4.356864in}{2.375753in}}%
\pgfpathlineto{\pgfqpoint{4.360085in}{2.378249in}}%
\pgfpathlineto{\pgfqpoint{4.361159in}{2.376587in}}%
\pgfpathlineto{\pgfqpoint{4.368676in}{2.367662in}}%
\pgfpathlineto{\pgfqpoint{4.371897in}{2.368001in}}%
\pgfpathlineto{\pgfqpoint{4.372971in}{2.369514in}}%
\pgfpathlineto{\pgfqpoint{4.374045in}{2.368095in}}%
\pgfpathlineto{\pgfqpoint{4.375119in}{2.367905in}}%
\pgfpathlineto{\pgfqpoint{4.376193in}{2.366922in}}%
\pgfpathlineto{\pgfqpoint{4.379414in}{2.367671in}}%
\pgfpathlineto{\pgfqpoint{4.380488in}{2.368501in}}%
\pgfpathlineto{\pgfqpoint{4.381562in}{2.368197in}}%
\pgfpathlineto{\pgfqpoint{4.383710in}{2.365484in}}%
\pgfpathlineto{\pgfqpoint{4.386931in}{2.365706in}}%
\pgfpathlineto{\pgfqpoint{4.388005in}{2.364410in}}%
\pgfpathlineto{\pgfqpoint{4.389079in}{2.364081in}}%
\pgfpathlineto{\pgfqpoint{4.390153in}{2.365940in}}%
\pgfpathlineto{\pgfqpoint{4.391227in}{2.366571in}}%
\pgfpathlineto{\pgfqpoint{4.395522in}{2.366422in}}%
\pgfpathlineto{\pgfqpoint{4.396596in}{2.368323in}}%
\pgfpathlineto{\pgfqpoint{4.397670in}{2.366843in}}%
\pgfpathlineto{\pgfqpoint{4.398743in}{2.370175in}}%
\pgfpathlineto{\pgfqpoint{4.401965in}{2.370832in}}%
\pgfpathlineto{\pgfqpoint{4.404113in}{2.372464in}}%
\pgfpathlineto{\pgfqpoint{4.405186in}{2.372543in}}%
\pgfpathlineto{\pgfqpoint{4.406260in}{2.375150in}}%
\pgfpathlineto{\pgfqpoint{4.409482in}{2.372723in}}%
\pgfpathlineto{\pgfqpoint{4.411629in}{2.367934in}}%
\pgfpathlineto{\pgfqpoint{4.412703in}{2.369297in}}%
\pgfpathlineto{\pgfqpoint{4.413777in}{2.368837in}}%
\pgfpathlineto{\pgfqpoint{4.418073in}{2.370637in}}%
\pgfpathlineto{\pgfqpoint{4.420220in}{2.368281in}}%
\pgfpathlineto{\pgfqpoint{4.421294in}{2.368433in}}%
\pgfpathlineto{\pgfqpoint{4.427737in}{2.365343in}}%
\pgfpathlineto{\pgfqpoint{4.428811in}{2.364530in}}%
\pgfpathlineto{\pgfqpoint{4.433106in}{2.364493in}}%
\pgfpathlineto{\pgfqpoint{4.434180in}{2.362294in}}%
\pgfpathlineto{\pgfqpoint{4.435254in}{2.363083in}}%
\pgfpathlineto{\pgfqpoint{4.436328in}{2.354729in}}%
\pgfpathlineto{\pgfqpoint{4.439549in}{2.353327in}}%
\pgfpathlineto{\pgfqpoint{4.440623in}{2.350037in}}%
\pgfpathlineto{\pgfqpoint{4.442771in}{2.349624in}}%
\pgfpathlineto{\pgfqpoint{4.443845in}{2.347659in}}%
\pgfpathlineto{\pgfqpoint{4.447066in}{2.349336in}}%
\pgfpathlineto{\pgfqpoint{4.449214in}{2.347043in}}%
\pgfpathlineto{\pgfqpoint{4.450288in}{2.347043in}}%
\pgfpathlineto{\pgfqpoint{4.451361in}{2.348061in}}%
\pgfpathlineto{\pgfqpoint{4.455657in}{2.347469in}}%
\pgfpathlineto{\pgfqpoint{4.457805in}{2.345463in}}%
\pgfpathlineto{\pgfqpoint{4.458878in}{2.346039in}}%
\pgfpathlineto{\pgfqpoint{4.463174in}{2.343963in}}%
\pgfpathlineto{\pgfqpoint{4.465321in}{2.344770in}}%
\pgfpathlineto{\pgfqpoint{4.466395in}{2.344921in}}%
\pgfpathlineto{\pgfqpoint{4.469617in}{2.347547in}}%
\pgfpathlineto{\pgfqpoint{4.470691in}{2.347144in}}%
\pgfpathlineto{\pgfqpoint{4.471764in}{2.345444in}}%
\pgfpathlineto{\pgfqpoint{4.472838in}{2.346871in}}%
\pgfpathlineto{\pgfqpoint{4.473912in}{2.346069in}}%
\pgfpathlineto{\pgfqpoint{4.477134in}{2.345611in}}%
\pgfpathlineto{\pgfqpoint{4.479281in}{2.344307in}}%
\pgfpathlineto{\pgfqpoint{4.480355in}{2.345057in}}%
\pgfpathlineto{\pgfqpoint{4.481429in}{2.344361in}}%
\pgfpathlineto{\pgfqpoint{4.484650in}{2.343941in}}%
\pgfpathlineto{\pgfqpoint{4.486798in}{2.345290in}}%
\pgfpathlineto{\pgfqpoint{4.487872in}{2.343835in}}%
\pgfpathlineto{\pgfqpoint{4.488946in}{2.346403in}}%
\pgfpathlineto{\pgfqpoint{4.492167in}{2.346832in}}%
\pgfpathlineto{\pgfqpoint{4.494315in}{2.343286in}}%
\pgfpathlineto{\pgfqpoint{4.495389in}{2.344207in}}%
\pgfpathlineto{\pgfqpoint{4.496463in}{2.344057in}}%
\pgfpathlineto{\pgfqpoint{4.499684in}{2.345494in}}%
\pgfpathlineto{\pgfqpoint{4.500758in}{2.343700in}}%
\pgfpathlineto{\pgfqpoint{4.501832in}{2.344536in}}%
\pgfpathlineto{\pgfqpoint{4.502906in}{2.344536in}}%
\pgfpathlineto{\pgfqpoint{4.507201in}{2.342971in}}%
\pgfpathlineto{\pgfqpoint{4.508275in}{2.339922in}}%
\pgfpathlineto{\pgfqpoint{4.509349in}{2.341870in}}%
\pgfpathlineto{\pgfqpoint{4.510423in}{2.340875in}}%
\pgfpathlineto{\pgfqpoint{4.511496in}{2.341628in}}%
\pgfpathlineto{\pgfqpoint{4.514718in}{2.340285in}}%
\pgfpathlineto{\pgfqpoint{4.515792in}{2.341522in}}%
\pgfpathlineto{\pgfqpoint{4.516866in}{2.339189in}}%
\pgfpathlineto{\pgfqpoint{4.519013in}{2.337266in}}%
\pgfpathlineto{\pgfqpoint{4.522235in}{2.338240in}}%
\pgfpathlineto{\pgfqpoint{4.523309in}{2.337199in}}%
\pgfpathlineto{\pgfqpoint{4.524383in}{2.339564in}}%
\pgfpathlineto{\pgfqpoint{4.525456in}{2.337679in}}%
\pgfpathlineto{\pgfqpoint{4.526530in}{2.334125in}}%
\pgfpathlineto{\pgfqpoint{4.529752in}{2.334179in}}%
\pgfpathlineto{\pgfqpoint{4.530826in}{2.331675in}}%
\pgfpathlineto{\pgfqpoint{4.531899in}{2.342044in}}%
\pgfpathlineto{\pgfqpoint{4.532973in}{2.343340in}}%
\pgfpathlineto{\pgfqpoint{4.534047in}{2.342055in}}%
\pgfpathlineto{\pgfqpoint{4.537269in}{2.340670in}}%
\pgfpathlineto{\pgfqpoint{4.538342in}{2.342848in}}%
\pgfpathlineto{\pgfqpoint{4.539416in}{2.342313in}}%
\pgfpathlineto{\pgfqpoint{4.541564in}{2.339408in}}%
\pgfpathlineto{\pgfqpoint{4.544785in}{2.340439in}}%
\pgfpathlineto{\pgfqpoint{4.545859in}{2.340004in}}%
\pgfpathlineto{\pgfqpoint{4.546933in}{2.338822in}}%
\pgfpathlineto{\pgfqpoint{4.548007in}{2.339335in}}%
\pgfpathlineto{\pgfqpoint{4.549081in}{2.338046in}}%
\pgfpathlineto{\pgfqpoint{4.552302in}{2.338442in}}%
\pgfpathlineto{\pgfqpoint{4.554450in}{2.333656in}}%
\pgfpathlineto{\pgfqpoint{4.556598in}{2.335113in}}%
\pgfpathlineto{\pgfqpoint{4.560893in}{2.337862in}}%
\pgfpathlineto{\pgfqpoint{4.563041in}{2.338087in}}%
\pgfpathlineto{\pgfqpoint{4.564115in}{2.330713in}}%
\pgfpathlineto{\pgfqpoint{4.568410in}{2.332902in}}%
\pgfpathlineto{\pgfqpoint{4.569484in}{2.337204in}}%
\pgfpathlineto{\pgfqpoint{4.570558in}{2.336755in}}%
\pgfpathlineto{\pgfqpoint{4.571631in}{2.342086in}}%
\pgfpathlineto{\pgfqpoint{4.574853in}{2.339925in}}%
\pgfpathlineto{\pgfqpoint{4.575927in}{2.340590in}}%
\pgfpathlineto{\pgfqpoint{4.577001in}{2.342309in}}%
\pgfpathlineto{\pgfqpoint{4.578074in}{2.341893in}}%
\pgfpathlineto{\pgfqpoint{4.579148in}{2.339648in}}%
\pgfpathlineto{\pgfqpoint{4.583444in}{2.338783in}}%
\pgfpathlineto{\pgfqpoint{4.584517in}{2.339983in}}%
\pgfpathlineto{\pgfqpoint{4.585591in}{2.338015in}}%
\pgfpathlineto{\pgfqpoint{4.586665in}{2.338950in}}%
\pgfpathlineto{\pgfqpoint{4.590961in}{2.336578in}}%
\pgfpathlineto{\pgfqpoint{4.592034in}{2.339870in}}%
\pgfpathlineto{\pgfqpoint{4.593108in}{2.345455in}}%
\pgfpathlineto{\pgfqpoint{4.594182in}{2.339952in}}%
\pgfpathlineto{\pgfqpoint{4.597404in}{2.341983in}}%
\pgfpathlineto{\pgfqpoint{4.598477in}{2.341716in}}%
\pgfpathlineto{\pgfqpoint{4.599551in}{2.340000in}}%
\pgfpathlineto{\pgfqpoint{4.600625in}{2.339362in}}%
\pgfpathlineto{\pgfqpoint{4.601699in}{2.340976in}}%
\pgfpathlineto{\pgfqpoint{4.607068in}{2.337200in}}%
\pgfpathlineto{\pgfqpoint{4.609216in}{2.337735in}}%
\pgfpathlineto{\pgfqpoint{4.612437in}{2.336744in}}%
\pgfpathlineto{\pgfqpoint{4.613511in}{2.334894in}}%
\pgfpathlineto{\pgfqpoint{4.614585in}{2.334372in}}%
\pgfpathlineto{\pgfqpoint{4.616733in}{2.330483in}}%
\pgfpathlineto{\pgfqpoint{4.619954in}{2.330588in}}%
\pgfpathlineto{\pgfqpoint{4.622102in}{2.332349in}}%
\pgfpathlineto{\pgfqpoint{4.623176in}{2.331895in}}%
\pgfpathlineto{\pgfqpoint{4.624249in}{2.329531in}}%
\pgfpathlineto{\pgfqpoint{4.628545in}{2.330438in}}%
\pgfpathlineto{\pgfqpoint{4.629619in}{2.331901in}}%
\pgfpathlineto{\pgfqpoint{4.631766in}{2.331636in}}%
\pgfpathlineto{\pgfqpoint{4.634988in}{2.332298in}}%
\pgfpathlineto{\pgfqpoint{4.636062in}{2.331524in}}%
\pgfpathlineto{\pgfqpoint{4.637136in}{2.331841in}}%
\pgfpathlineto{\pgfqpoint{4.638209in}{2.328894in}}%
\pgfpathlineto{\pgfqpoint{4.639283in}{2.329639in}}%
\pgfpathlineto{\pgfqpoint{4.642505in}{2.329406in}}%
\pgfpathlineto{\pgfqpoint{4.645726in}{2.327763in}}%
\pgfpathlineto{\pgfqpoint{4.646800in}{2.328878in}}%
\pgfpathlineto{\pgfqpoint{4.651095in}{2.328827in}}%
\pgfpathlineto{\pgfqpoint{4.652169in}{2.329699in}}%
\pgfpathlineto{\pgfqpoint{4.654317in}{2.332626in}}%
\pgfpathlineto{\pgfqpoint{4.659686in}{2.333782in}}%
\pgfpathlineto{\pgfqpoint{4.661834in}{2.332376in}}%
\pgfpathlineto{\pgfqpoint{4.666129in}{2.336912in}}%
\pgfpathlineto{\pgfqpoint{4.667203in}{2.339528in}}%
\pgfpathlineto{\pgfqpoint{4.669351in}{2.337055in}}%
\pgfpathlineto{\pgfqpoint{4.674720in}{2.337987in}}%
\pgfpathlineto{\pgfqpoint{4.675794in}{2.338812in}}%
\pgfpathlineto{\pgfqpoint{4.676868in}{2.338784in}}%
\pgfpathlineto{\pgfqpoint{4.680089in}{2.336371in}}%
\pgfpathlineto{\pgfqpoint{4.681163in}{2.336875in}}%
\pgfpathlineto{\pgfqpoint{4.682237in}{2.336115in}}%
\pgfpathlineto{\pgfqpoint{4.683311in}{2.336143in}}%
\pgfpathlineto{\pgfqpoint{4.684384in}{2.334330in}}%
\pgfpathlineto{\pgfqpoint{4.687606in}{2.331814in}}%
\pgfpathlineto{\pgfqpoint{4.688680in}{2.332532in}}%
\pgfpathlineto{\pgfqpoint{4.689754in}{2.331513in}}%
\pgfpathlineto{\pgfqpoint{4.690827in}{2.331884in}}%
\pgfpathlineto{\pgfqpoint{4.691901in}{2.333507in}}%
\pgfpathlineto{\pgfqpoint{4.695123in}{2.333751in}}%
\pgfpathlineto{\pgfqpoint{4.697271in}{2.337291in}}%
\pgfpathlineto{\pgfqpoint{4.698344in}{2.336528in}}%
\pgfpathlineto{\pgfqpoint{4.699418in}{2.338462in}}%
\pgfpathlineto{\pgfqpoint{4.702640in}{2.342301in}}%
\pgfpathlineto{\pgfqpoint{4.703714in}{2.340897in}}%
\pgfpathlineto{\pgfqpoint{4.705861in}{2.341663in}}%
\pgfpathlineto{\pgfqpoint{4.706935in}{2.341218in}}%
\pgfpathlineto{\pgfqpoint{4.710157in}{2.342134in}}%
\pgfpathlineto{\pgfqpoint{4.714452in}{2.335343in}}%
\pgfpathlineto{\pgfqpoint{4.717673in}{2.334761in}}%
\pgfpathlineto{\pgfqpoint{4.718747in}{2.337348in}}%
\pgfpathlineto{\pgfqpoint{4.720895in}{2.331713in}}%
\pgfpathlineto{\pgfqpoint{4.721969in}{2.331740in}}%
\pgfpathlineto{\pgfqpoint{4.725190in}{2.332698in}}%
\pgfpathlineto{\pgfqpoint{4.726264in}{2.329550in}}%
\pgfpathlineto{\pgfqpoint{4.727338in}{2.328407in}}%
\pgfpathlineto{\pgfqpoint{4.728412in}{2.329048in}}%
\pgfpathlineto{\pgfqpoint{4.729486in}{2.330365in}}%
\pgfpathlineto{\pgfqpoint{4.732707in}{2.327507in}}%
\pgfpathlineto{\pgfqpoint{4.733781in}{2.325071in}}%
\pgfpathlineto{\pgfqpoint{4.735929in}{2.328610in}}%
\pgfpathlineto{\pgfqpoint{4.737003in}{2.327761in}}%
\pgfpathlineto{\pgfqpoint{4.741298in}{2.327176in}}%
\pgfpathlineto{\pgfqpoint{4.742372in}{2.324268in}}%
\pgfpathlineto{\pgfqpoint{4.743446in}{2.325246in}}%
\pgfpathlineto{\pgfqpoint{4.744519in}{2.324875in}}%
\pgfpathlineto{\pgfqpoint{4.747741in}{2.325367in}}%
\pgfpathlineto{\pgfqpoint{4.749889in}{2.322835in}}%
\pgfpathlineto{\pgfqpoint{4.752036in}{2.320164in}}%
\pgfpathlineto{\pgfqpoint{4.756332in}{2.321144in}}%
\pgfpathlineto{\pgfqpoint{4.757405in}{2.320461in}}%
\pgfpathlineto{\pgfqpoint{4.758479in}{2.320508in}}%
\pgfpathlineto{\pgfqpoint{4.759553in}{2.319947in}}%
\pgfpathlineto{\pgfqpoint{4.762775in}{2.319064in}}%
\pgfpathlineto{\pgfqpoint{4.764922in}{2.321236in}}%
\pgfpathlineto{\pgfqpoint{4.767070in}{2.321354in}}%
\pgfpathlineto{\pgfqpoint{4.770292in}{2.323078in}}%
\pgfpathlineto{\pgfqpoint{4.771365in}{2.322114in}}%
\pgfpathlineto{\pgfqpoint{4.773513in}{2.324394in}}%
\pgfpathlineto{\pgfqpoint{4.774587in}{2.320645in}}%
\pgfpathlineto{\pgfqpoint{4.777808in}{2.320129in}}%
\pgfpathlineto{\pgfqpoint{4.778882in}{2.322342in}}%
\pgfpathlineto{\pgfqpoint{4.779956in}{2.321600in}}%
\pgfpathlineto{\pgfqpoint{4.781030in}{2.324991in}}%
\pgfpathlineto{\pgfqpoint{4.782104in}{2.324843in}}%
\pgfpathlineto{\pgfqpoint{4.785325in}{2.326249in}}%
\pgfpathlineto{\pgfqpoint{4.786399in}{2.327878in}}%
\pgfpathlineto{\pgfqpoint{4.787473in}{2.324735in}}%
\pgfpathlineto{\pgfqpoint{4.788547in}{2.325720in}}%
\pgfpathlineto{\pgfqpoint{4.789621in}{2.325845in}}%
\pgfpathlineto{\pgfqpoint{4.793916in}{2.326969in}}%
\pgfpathlineto{\pgfqpoint{4.797137in}{2.325482in}}%
\pgfpathlineto{\pgfqpoint{4.800359in}{2.325507in}}%
\pgfpathlineto{\pgfqpoint{4.803581in}{2.328567in}}%
\pgfpathlineto{\pgfqpoint{4.804654in}{2.328747in}}%
\pgfpathlineto{\pgfqpoint{4.807876in}{2.328334in}}%
\pgfpathlineto{\pgfqpoint{4.808950in}{2.326663in}}%
\pgfpathlineto{\pgfqpoint{4.810024in}{2.329159in}}%
\pgfpathlineto{\pgfqpoint{4.811097in}{2.328770in}}%
\pgfpathlineto{\pgfqpoint{4.815393in}{2.329650in}}%
\pgfpathlineto{\pgfqpoint{4.816467in}{2.327771in}}%
\pgfpathlineto{\pgfqpoint{4.818614in}{2.328436in}}%
\pgfpathlineto{\pgfqpoint{4.819688in}{2.329904in}}%
\pgfpathlineto{\pgfqpoint{4.823983in}{2.329249in}}%
\pgfpathlineto{\pgfqpoint{4.825057in}{2.330003in}}%
\pgfpathlineto{\pgfqpoint{4.826131in}{2.332179in}}%
\pgfpathlineto{\pgfqpoint{4.827205in}{2.330055in}}%
\pgfpathlineto{\pgfqpoint{4.830426in}{2.331630in}}%
\pgfpathlineto{\pgfqpoint{4.831500in}{2.330187in}}%
\pgfpathlineto{\pgfqpoint{4.833648in}{2.335193in}}%
\pgfpathlineto{\pgfqpoint{4.837943in}{2.337935in}}%
\pgfpathlineto{\pgfqpoint{4.842239in}{2.334122in}}%
\pgfpathlineto{\pgfqpoint{4.845460in}{2.332472in}}%
\pgfpathlineto{\pgfqpoint{4.846534in}{2.330636in}}%
\pgfpathlineto{\pgfqpoint{4.847608in}{2.333624in}}%
\pgfpathlineto{\pgfqpoint{4.848682in}{2.332913in}}%
\pgfpathlineto{\pgfqpoint{4.849756in}{2.329275in}}%
\pgfpathlineto{\pgfqpoint{4.854051in}{2.332402in}}%
\pgfpathlineto{\pgfqpoint{4.855125in}{2.331997in}}%
\pgfpathlineto{\pgfqpoint{4.856199in}{2.332507in}}%
\pgfpathlineto{\pgfqpoint{4.857272in}{2.332399in}}%
\pgfpathlineto{\pgfqpoint{4.860494in}{2.332642in}}%
\pgfpathlineto{\pgfqpoint{4.861568in}{2.332046in}}%
\pgfpathlineto{\pgfqpoint{4.862642in}{2.332638in}}%
\pgfpathlineto{\pgfqpoint{4.864789in}{2.330833in}}%
\pgfpathlineto{\pgfqpoint{4.868011in}{2.333460in}}%
\pgfpathlineto{\pgfqpoint{4.869085in}{2.331218in}}%
\pgfpathlineto{\pgfqpoint{4.870159in}{2.332631in}}%
\pgfpathlineto{\pgfqpoint{4.872306in}{2.331362in}}%
\pgfpathlineto{\pgfqpoint{4.875528in}{2.331041in}}%
\pgfpathlineto{\pgfqpoint{4.877675in}{2.329451in}}%
\pgfpathlineto{\pgfqpoint{4.879823in}{2.330027in}}%
\pgfpathlineto{\pgfqpoint{4.883045in}{2.328133in}}%
\pgfpathlineto{\pgfqpoint{4.884118in}{2.328287in}}%
\pgfpathlineto{\pgfqpoint{4.886266in}{2.331432in}}%
\pgfpathlineto{\pgfqpoint{4.887340in}{2.332422in}}%
\pgfpathlineto{\pgfqpoint{4.891635in}{2.328082in}}%
\pgfpathlineto{\pgfqpoint{4.892709in}{2.328751in}}%
\pgfpathlineto{\pgfqpoint{4.894857in}{2.328182in}}%
\pgfpathlineto{\pgfqpoint{4.898078in}{2.326378in}}%
\pgfpathlineto{\pgfqpoint{4.902374in}{2.328157in}}%
\pgfpathlineto{\pgfqpoint{4.905595in}{2.326585in}}%
\pgfpathlineto{\pgfqpoint{4.906669in}{2.324763in}}%
\pgfpathlineto{\pgfqpoint{4.907743in}{2.324466in}}%
\pgfpathlineto{\pgfqpoint{4.909891in}{2.325978in}}%
\pgfpathlineto{\pgfqpoint{4.914186in}{2.325828in}}%
\pgfpathlineto{\pgfqpoint{4.915260in}{2.324122in}}%
\pgfpathlineto{\pgfqpoint{4.916334in}{2.323901in}}%
\pgfpathlineto{\pgfqpoint{4.921703in}{2.324933in}}%
\pgfpathlineto{\pgfqpoint{4.922777in}{2.325455in}}%
\pgfpathlineto{\pgfqpoint{4.923850in}{2.323781in}}%
\pgfpathlineto{\pgfqpoint{4.924924in}{2.323243in}}%
\pgfpathlineto{\pgfqpoint{4.928146in}{2.320104in}}%
\pgfpathlineto{\pgfqpoint{4.929220in}{2.321088in}}%
\pgfpathlineto{\pgfqpoint{4.930293in}{2.320875in}}%
\pgfpathlineto{\pgfqpoint{4.936737in}{2.323749in}}%
\pgfpathlineto{\pgfqpoint{4.937810in}{2.321765in}}%
\pgfpathlineto{\pgfqpoint{4.939958in}{2.321144in}}%
\pgfpathlineto{\pgfqpoint{4.945327in}{2.323634in}}%
\pgfpathlineto{\pgfqpoint{4.946401in}{2.324955in}}%
\pgfpathlineto{\pgfqpoint{4.950696in}{2.326227in}}%
\pgfpathlineto{\pgfqpoint{4.953918in}{2.324697in}}%
\pgfpathlineto{\pgfqpoint{4.960361in}{2.325839in}}%
\pgfpathlineto{\pgfqpoint{4.962509in}{2.324388in}}%
\pgfpathlineto{\pgfqpoint{4.965730in}{2.325277in}}%
\pgfpathlineto{\pgfqpoint{4.966804in}{2.326200in}}%
\pgfpathlineto{\pgfqpoint{4.967878in}{2.326326in}}%
\pgfpathlineto{\pgfqpoint{4.968952in}{2.325796in}}%
\pgfpathlineto{\pgfqpoint{4.970026in}{2.325972in}}%
\pgfpathlineto{\pgfqpoint{4.975395in}{2.325921in}}%
\pgfpathlineto{\pgfqpoint{4.977542in}{2.326803in}}%
\pgfpathlineto{\pgfqpoint{4.982912in}{2.326471in}}%
\pgfpathlineto{\pgfqpoint{4.983985in}{2.325510in}}%
\pgfpathlineto{\pgfqpoint{4.985059in}{2.326185in}}%
\pgfpathlineto{\pgfqpoint{4.988281in}{2.327043in}}%
\pgfpathlineto{\pgfqpoint{4.989355in}{2.328138in}}%
\pgfpathlineto{\pgfqpoint{4.990428in}{2.327468in}}%
\pgfpathlineto{\pgfqpoint{4.991502in}{2.328978in}}%
\pgfpathlineto{\pgfqpoint{4.992576in}{2.328118in}}%
\pgfpathlineto{\pgfqpoint{4.995798in}{2.329047in}}%
\pgfpathlineto{\pgfqpoint{4.997945in}{2.326919in}}%
\pgfpathlineto{\pgfqpoint{4.999019in}{2.326003in}}%
\pgfpathlineto{\pgfqpoint{5.003314in}{2.325575in}}%
\pgfpathlineto{\pgfqpoint{5.004388in}{2.324774in}}%
\pgfpathlineto{\pgfqpoint{5.005462in}{2.322441in}}%
\pgfpathlineto{\pgfqpoint{5.006536in}{2.322562in}}%
\pgfpathlineto{\pgfqpoint{5.007610in}{2.323480in}}%
\pgfpathlineto{\pgfqpoint{5.012979in}{2.322602in}}%
\pgfpathlineto{\pgfqpoint{5.014053in}{2.323714in}}%
\pgfpathlineto{\pgfqpoint{5.015127in}{2.323420in}}%
\pgfpathlineto{\pgfqpoint{5.021570in}{2.324004in}}%
\pgfpathlineto{\pgfqpoint{5.022644in}{2.325479in}}%
\pgfpathlineto{\pgfqpoint{5.025865in}{2.326130in}}%
\pgfpathlineto{\pgfqpoint{5.029087in}{2.330359in}}%
\pgfpathlineto{\pgfqpoint{5.030160in}{2.325086in}}%
\pgfpathlineto{\pgfqpoint{5.033382in}{2.326009in}}%
\pgfpathlineto{\pgfqpoint{5.034456in}{2.325606in}}%
\pgfpathlineto{\pgfqpoint{5.035530in}{2.322017in}}%
\pgfpathlineto{\pgfqpoint{5.036603in}{2.323966in}}%
\pgfpathlineto{\pgfqpoint{5.037677in}{2.322266in}}%
\pgfpathlineto{\pgfqpoint{5.041973in}{2.321275in}}%
\pgfpathlineto{\pgfqpoint{5.043047in}{2.321156in}}%
\pgfpathlineto{\pgfqpoint{5.044120in}{2.320203in}}%
\pgfpathlineto{\pgfqpoint{5.045194in}{2.320556in}}%
\pgfpathlineto{\pgfqpoint{5.052711in}{2.317338in}}%
\pgfpathlineto{\pgfqpoint{5.057006in}{2.317360in}}%
\pgfpathlineto{\pgfqpoint{5.065597in}{2.318295in}}%
\pgfpathlineto{\pgfqpoint{5.066671in}{2.316730in}}%
\pgfpathlineto{\pgfqpoint{5.067745in}{2.316595in}}%
\pgfpathlineto{\pgfqpoint{5.072040in}{2.318112in}}%
\pgfpathlineto{\pgfqpoint{5.073114in}{2.316826in}}%
\pgfpathlineto{\pgfqpoint{5.075262in}{2.317573in}}%
\pgfpathlineto{\pgfqpoint{5.080631in}{2.315826in}}%
\pgfpathlineto{\pgfqpoint{5.082779in}{2.315224in}}%
\pgfpathlineto{\pgfqpoint{5.086000in}{2.314581in}}%
\pgfpathlineto{\pgfqpoint{5.087074in}{2.315284in}}%
\pgfpathlineto{\pgfqpoint{5.088148in}{2.313777in}}%
\pgfpathlineto{\pgfqpoint{5.089222in}{2.315452in}}%
\pgfpathlineto{\pgfqpoint{5.090295in}{2.314985in}}%
\pgfpathlineto{\pgfqpoint{5.093517in}{2.315250in}}%
\pgfpathlineto{\pgfqpoint{5.095665in}{2.317730in}}%
\pgfpathlineto{\pgfqpoint{5.096738in}{2.317890in}}%
\pgfpathlineto{\pgfqpoint{5.097812in}{2.316447in}}%
\pgfpathlineto{\pgfqpoint{5.101034in}{2.316874in}}%
\pgfpathlineto{\pgfqpoint{5.102108in}{2.317666in}}%
\pgfpathlineto{\pgfqpoint{5.103181in}{2.315678in}}%
\pgfpathlineto{\pgfqpoint{5.104255in}{2.316659in}}%
\pgfpathlineto{\pgfqpoint{5.105329in}{2.314696in}}%
\pgfpathlineto{\pgfqpoint{5.108551in}{2.317163in}}%
\pgfpathlineto{\pgfqpoint{5.109625in}{2.316822in}}%
\pgfpathlineto{\pgfqpoint{5.111772in}{2.319376in}}%
\pgfpathlineto{\pgfqpoint{5.112846in}{2.317321in}}%
\pgfpathlineto{\pgfqpoint{5.118215in}{2.314236in}}%
\pgfpathlineto{\pgfqpoint{5.119289in}{2.316120in}}%
\pgfpathlineto{\pgfqpoint{5.120363in}{2.312395in}}%
\pgfpathlineto{\pgfqpoint{5.127880in}{2.308567in}}%
\pgfpathlineto{\pgfqpoint{5.131101in}{2.308383in}}%
\pgfpathlineto{\pgfqpoint{5.132175in}{2.306186in}}%
\pgfpathlineto{\pgfqpoint{5.133249in}{2.305573in}}%
\pgfpathlineto{\pgfqpoint{5.135397in}{2.305337in}}%
\pgfpathlineto{\pgfqpoint{5.140766in}{2.304750in}}%
\pgfpathlineto{\pgfqpoint{5.142914in}{2.307054in}}%
\pgfpathlineto{\pgfqpoint{5.147209in}{2.307333in}}%
\pgfpathlineto{\pgfqpoint{5.148283in}{2.305988in}}%
\pgfpathlineto{\pgfqpoint{5.149357in}{2.306422in}}%
\pgfpathlineto{\pgfqpoint{5.150430in}{2.306204in}}%
\pgfpathlineto{\pgfqpoint{5.153652in}{2.307035in}}%
\pgfpathlineto{\pgfqpoint{5.154726in}{2.305855in}}%
\pgfpathlineto{\pgfqpoint{5.155800in}{2.305717in}}%
\pgfpathlineto{\pgfqpoint{5.157947in}{2.303103in}}%
\pgfpathlineto{\pgfqpoint{5.161169in}{2.303673in}}%
\pgfpathlineto{\pgfqpoint{5.162243in}{2.302047in}}%
\pgfpathlineto{\pgfqpoint{5.163316in}{2.303768in}}%
\pgfpathlineto{\pgfqpoint{5.164390in}{2.302811in}}%
\pgfpathlineto{\pgfqpoint{5.165464in}{2.303132in}}%
\pgfpathlineto{\pgfqpoint{5.169759in}{2.302600in}}%
\pgfpathlineto{\pgfqpoint{5.170833in}{2.303750in}}%
\pgfpathlineto{\pgfqpoint{5.171907in}{2.303060in}}%
\pgfpathlineto{\pgfqpoint{5.172981in}{2.304464in}}%
\pgfpathlineto{\pgfqpoint{5.177276in}{2.304791in}}%
\pgfpathlineto{\pgfqpoint{5.179424in}{2.300446in}}%
\pgfpathlineto{\pgfqpoint{5.180498in}{2.300354in}}%
\pgfpathlineto{\pgfqpoint{5.183719in}{2.299495in}}%
\pgfpathlineto{\pgfqpoint{5.184793in}{2.298428in}}%
\pgfpathlineto{\pgfqpoint{5.185867in}{2.298659in}}%
\pgfpathlineto{\pgfqpoint{5.188015in}{2.298159in}}%
\pgfpathlineto{\pgfqpoint{5.192310in}{2.299780in}}%
\pgfpathlineto{\pgfqpoint{5.193384in}{2.301832in}}%
\pgfpathlineto{\pgfqpoint{5.195532in}{2.302878in}}%
\pgfpathlineto{\pgfqpoint{5.199827in}{2.304248in}}%
\pgfpathlineto{\pgfqpoint{5.200901in}{2.303476in}}%
\pgfpathlineto{\pgfqpoint{5.201975in}{2.301776in}}%
\pgfpathlineto{\pgfqpoint{5.203048in}{2.303139in}}%
\pgfpathlineto{\pgfqpoint{5.206270in}{2.303690in}}%
\pgfpathlineto{\pgfqpoint{5.207344in}{2.303058in}}%
\pgfpathlineto{\pgfqpoint{5.209491in}{2.303781in}}%
\pgfpathlineto{\pgfqpoint{5.210565in}{2.301479in}}%
\pgfpathlineto{\pgfqpoint{5.215935in}{2.301293in}}%
\pgfpathlineto{\pgfqpoint{5.217008in}{2.299846in}}%
\pgfpathlineto{\pgfqpoint{5.218082in}{2.302283in}}%
\pgfpathlineto{\pgfqpoint{5.221304in}{2.303092in}}%
\pgfpathlineto{\pgfqpoint{5.222378in}{2.308189in}}%
\pgfpathlineto{\pgfqpoint{5.223451in}{2.310553in}}%
\pgfpathlineto{\pgfqpoint{5.224525in}{2.309607in}}%
\pgfpathlineto{\pgfqpoint{5.225599in}{2.312079in}}%
\pgfpathlineto{\pgfqpoint{5.228821in}{2.310535in}}%
\pgfpathlineto{\pgfqpoint{5.229894in}{2.309021in}}%
\pgfpathlineto{\pgfqpoint{5.230968in}{2.309309in}}%
\pgfpathlineto{\pgfqpoint{5.232042in}{2.307674in}}%
\pgfpathlineto{\pgfqpoint{5.233116in}{2.309599in}}%
\pgfpathlineto{\pgfqpoint{5.236337in}{2.310597in}}%
\pgfpathlineto{\pgfqpoint{5.239559in}{2.308821in}}%
\pgfpathlineto{\pgfqpoint{5.244928in}{2.309790in}}%
\pgfpathlineto{\pgfqpoint{5.246002in}{2.308394in}}%
\pgfpathlineto{\pgfqpoint{5.247076in}{2.310256in}}%
\pgfpathlineto{\pgfqpoint{5.248150in}{2.310885in}}%
\pgfpathlineto{\pgfqpoint{5.253519in}{2.309808in}}%
\pgfpathlineto{\pgfqpoint{5.255667in}{2.310392in}}%
\pgfpathlineto{\pgfqpoint{5.258888in}{2.309888in}}%
\pgfpathlineto{\pgfqpoint{5.259962in}{2.310347in}}%
\pgfpathlineto{\pgfqpoint{5.261036in}{2.311817in}}%
\pgfpathlineto{\pgfqpoint{5.262110in}{2.311303in}}%
\pgfpathlineto{\pgfqpoint{5.263183in}{2.314917in}}%
\pgfpathlineto{\pgfqpoint{5.266405in}{2.314093in}}%
\pgfpathlineto{\pgfqpoint{5.267479in}{2.317046in}}%
\pgfpathlineto{\pgfqpoint{5.268553in}{2.317343in}}%
\pgfpathlineto{\pgfqpoint{5.269626in}{2.315943in}}%
\pgfpathlineto{\pgfqpoint{5.270700in}{2.316462in}}%
\pgfpathlineto{\pgfqpoint{5.273922in}{2.313055in}}%
\pgfpathlineto{\pgfqpoint{5.274996in}{2.314405in}}%
\pgfpathlineto{\pgfqpoint{5.276069in}{2.312678in}}%
\pgfpathlineto{\pgfqpoint{5.277143in}{2.313371in}}%
\pgfpathlineto{\pgfqpoint{5.278217in}{2.310792in}}%
\pgfpathlineto{\pgfqpoint{5.281439in}{2.310580in}}%
\pgfpathlineto{\pgfqpoint{5.284660in}{2.315624in}}%
\pgfpathlineto{\pgfqpoint{5.288956in}{2.314525in}}%
\pgfpathlineto{\pgfqpoint{5.290029in}{2.316012in}}%
\pgfpathlineto{\pgfqpoint{5.292177in}{2.315064in}}%
\pgfpathlineto{\pgfqpoint{5.296472in}{2.313880in}}%
\pgfpathlineto{\pgfqpoint{5.297546in}{2.315135in}}%
\pgfpathlineto{\pgfqpoint{5.300768in}{2.313269in}}%
\pgfpathlineto{\pgfqpoint{5.307211in}{2.312984in}}%
\pgfpathlineto{\pgfqpoint{5.308285in}{2.314813in}}%
\pgfpathlineto{\pgfqpoint{5.313654in}{2.313747in}}%
\pgfpathlineto{\pgfqpoint{5.314728in}{2.316583in}}%
\pgfpathlineto{\pgfqpoint{5.315802in}{2.316469in}}%
\pgfpathlineto{\pgfqpoint{5.320097in}{2.317634in}}%
\pgfpathlineto{\pgfqpoint{5.322245in}{2.319467in}}%
\pgfpathlineto{\pgfqpoint{5.323318in}{2.317862in}}%
\pgfpathlineto{\pgfqpoint{5.326540in}{2.317746in}}%
\pgfpathlineto{\pgfqpoint{5.327614in}{2.318324in}}%
\pgfpathlineto{\pgfqpoint{5.328688in}{2.317648in}}%
\pgfpathlineto{\pgfqpoint{5.329761in}{2.318041in}}%
\pgfpathlineto{\pgfqpoint{5.330835in}{2.316509in}}%
\pgfpathlineto{\pgfqpoint{5.336204in}{2.319038in}}%
\pgfpathlineto{\pgfqpoint{5.338352in}{2.316286in}}%
\pgfpathlineto{\pgfqpoint{5.344795in}{2.317563in}}%
\pgfpathlineto{\pgfqpoint{5.345869in}{2.318486in}}%
\pgfpathlineto{\pgfqpoint{5.350164in}{2.320144in}}%
\pgfpathlineto{\pgfqpoint{5.351238in}{2.319620in}}%
\pgfpathlineto{\pgfqpoint{5.352312in}{2.319738in}}%
\pgfpathlineto{\pgfqpoint{5.353386in}{2.321683in}}%
\pgfpathlineto{\pgfqpoint{5.356607in}{2.320711in}}%
\pgfpathlineto{\pgfqpoint{5.357681in}{2.321359in}}%
\pgfpathlineto{\pgfqpoint{5.358755in}{2.321311in}}%
\pgfpathlineto{\pgfqpoint{5.360903in}{2.323720in}}%
\pgfpathlineto{\pgfqpoint{5.364124in}{2.323123in}}%
\pgfpathlineto{\pgfqpoint{5.366272in}{2.319212in}}%
\pgfpathlineto{\pgfqpoint{5.367346in}{2.319471in}}%
\pgfpathlineto{\pgfqpoint{5.368420in}{2.320582in}}%
\pgfpathlineto{\pgfqpoint{5.371641in}{2.322165in}}%
\pgfpathlineto{\pgfqpoint{5.374863in}{2.316527in}}%
\pgfpathlineto{\pgfqpoint{5.375936in}{2.317097in}}%
\pgfpathlineto{\pgfqpoint{5.379158in}{2.317258in}}%
\pgfpathlineto{\pgfqpoint{5.381306in}{2.319156in}}%
\pgfpathlineto{\pgfqpoint{5.383453in}{2.319509in}}%
\pgfpathlineto{\pgfqpoint{5.387749in}{2.321786in}}%
\pgfpathlineto{\pgfqpoint{5.388823in}{2.318647in}}%
\pgfpathlineto{\pgfqpoint{5.389896in}{2.318225in}}%
\pgfpathlineto{\pgfqpoint{5.394192in}{2.317968in}}%
\pgfpathlineto{\pgfqpoint{5.395266in}{2.314598in}}%
\pgfpathlineto{\pgfqpoint{5.397413in}{2.316661in}}%
\pgfpathlineto{\pgfqpoint{5.403856in}{2.313715in}}%
\pgfpathlineto{\pgfqpoint{5.406004in}{2.313538in}}%
\pgfpathlineto{\pgfqpoint{5.409225in}{2.313648in}}%
\pgfpathlineto{\pgfqpoint{5.411373in}{2.314996in}}%
\pgfpathlineto{\pgfqpoint{5.417816in}{2.316162in}}%
\pgfpathlineto{\pgfqpoint{5.418890in}{2.315388in}}%
\pgfpathlineto{\pgfqpoint{5.419964in}{2.321787in}}%
\pgfpathlineto{\pgfqpoint{5.421038in}{2.323276in}}%
\pgfpathlineto{\pgfqpoint{5.424259in}{2.323922in}}%
\pgfpathlineto{\pgfqpoint{5.426407in}{2.325328in}}%
\pgfpathlineto{\pgfqpoint{5.428555in}{2.325965in}}%
\pgfpathlineto{\pgfqpoint{5.431776in}{2.323939in}}%
\pgfpathlineto{\pgfqpoint{5.432850in}{2.324289in}}%
\pgfpathlineto{\pgfqpoint{5.433924in}{2.323912in}}%
\pgfpathlineto{\pgfqpoint{5.434998in}{2.325264in}}%
\pgfpathlineto{\pgfqpoint{5.436071in}{2.325621in}}%
\pgfpathlineto{\pgfqpoint{5.439293in}{2.325825in}}%
\pgfpathlineto{\pgfqpoint{5.440367in}{2.326722in}}%
\pgfpathlineto{\pgfqpoint{5.441441in}{2.329026in}}%
\pgfpathlineto{\pgfqpoint{5.442514in}{2.329505in}}%
\pgfpathlineto{\pgfqpoint{5.443588in}{2.324637in}}%
\pgfpathlineto{\pgfqpoint{5.447884in}{2.316587in}}%
\pgfpathlineto{\pgfqpoint{5.448957in}{2.322417in}}%
\pgfpathlineto{\pgfqpoint{5.450031in}{2.323781in}}%
\pgfpathlineto{\pgfqpoint{5.451105in}{2.323140in}}%
\pgfpathlineto{\pgfqpoint{5.454327in}{2.321883in}}%
\pgfpathlineto{\pgfqpoint{5.455401in}{2.317711in}}%
\pgfpathlineto{\pgfqpoint{5.456474in}{2.319850in}}%
\pgfpathlineto{\pgfqpoint{5.457548in}{2.320118in}}%
\pgfpathlineto{\pgfqpoint{5.458622in}{2.317390in}}%
\pgfpathlineto{\pgfqpoint{5.462917in}{2.320251in}}%
\pgfpathlineto{\pgfqpoint{5.463991in}{2.316721in}}%
\pgfpathlineto{\pgfqpoint{5.465065in}{2.316347in}}%
\pgfpathlineto{\pgfqpoint{5.466139in}{2.316587in}}%
\pgfpathlineto{\pgfqpoint{5.469360in}{2.315731in}}%
\pgfpathlineto{\pgfqpoint{5.471508in}{2.320519in}}%
\pgfpathlineto{\pgfqpoint{5.472582in}{2.320866in}}%
\pgfpathlineto{\pgfqpoint{5.473656in}{2.320144in}}%
\pgfpathlineto{\pgfqpoint{5.476877in}{2.321829in}}%
\pgfpathlineto{\pgfqpoint{5.477951in}{2.320733in}}%
\pgfpathlineto{\pgfqpoint{5.479025in}{2.320920in}}%
\pgfpathlineto{\pgfqpoint{5.481173in}{2.326590in}}%
\pgfpathlineto{\pgfqpoint{5.484394in}{2.324450in}}%
\pgfpathlineto{\pgfqpoint{5.485468in}{2.325654in}}%
\pgfpathlineto{\pgfqpoint{5.486542in}{2.324851in}}%
\pgfpathlineto{\pgfqpoint{5.487616in}{2.324878in}}%
\pgfpathlineto{\pgfqpoint{5.488690in}{2.326001in}}%
\pgfpathlineto{\pgfqpoint{5.494059in}{2.322982in}}%
\pgfpathlineto{\pgfqpoint{5.495133in}{2.321433in}}%
\pgfpathlineto{\pgfqpoint{5.496206in}{2.321256in}}%
\pgfpathlineto{\pgfqpoint{5.502649in}{2.321735in}}%
\pgfpathlineto{\pgfqpoint{5.503723in}{2.320311in}}%
\pgfpathlineto{\pgfqpoint{5.506945in}{2.319736in}}%
\pgfpathlineto{\pgfqpoint{5.508019in}{2.321349in}}%
\pgfpathlineto{\pgfqpoint{5.509092in}{2.321729in}}%
\pgfpathlineto{\pgfqpoint{5.510166in}{2.318881in}}%
\pgfpathlineto{\pgfqpoint{5.511240in}{2.314138in}}%
\pgfpathlineto{\pgfqpoint{5.514462in}{2.313187in}}%
\pgfpathlineto{\pgfqpoint{5.515535in}{2.313600in}}%
\pgfpathlineto{\pgfqpoint{5.516609in}{2.315189in}}%
\pgfpathlineto{\pgfqpoint{5.517683in}{2.314131in}}%
\pgfpathlineto{\pgfqpoint{5.518757in}{2.315453in}}%
\pgfpathlineto{\pgfqpoint{5.521979in}{2.314982in}}%
\pgfpathlineto{\pgfqpoint{5.523052in}{2.314044in}}%
\pgfpathlineto{\pgfqpoint{5.524126in}{2.314021in}}%
\pgfpathlineto{\pgfqpoint{5.526274in}{2.317131in}}%
\pgfpathlineto{\pgfqpoint{5.529495in}{2.317492in}}%
\pgfpathlineto{\pgfqpoint{5.531643in}{2.316287in}}%
\pgfpathlineto{\pgfqpoint{5.533791in}{2.320556in}}%
\pgfpathlineto{\pgfqpoint{5.537012in}{2.317592in}}%
\pgfpathlineto{\pgfqpoint{5.538086in}{2.318076in}}%
\pgfpathlineto{\pgfqpoint{5.539160in}{2.316272in}}%
\pgfpathlineto{\pgfqpoint{5.540234in}{2.315604in}}%
\pgfpathlineto{\pgfqpoint{5.541308in}{2.316432in}}%
\pgfpathlineto{\pgfqpoint{5.544529in}{2.316121in}}%
\pgfpathlineto{\pgfqpoint{5.545603in}{2.315098in}}%
\pgfpathlineto{\pgfqpoint{5.546677in}{2.316249in}}%
\pgfpathlineto{\pgfqpoint{5.548824in}{2.316679in}}%
\pgfpathlineto{\pgfqpoint{5.552046in}{2.318500in}}%
\pgfpathlineto{\pgfqpoint{5.553120in}{2.316122in}}%
\pgfpathlineto{\pgfqpoint{5.555267in}{2.316527in}}%
\pgfpathlineto{\pgfqpoint{5.556341in}{2.312099in}}%
\pgfpathlineto{\pgfqpoint{5.559563in}{2.311012in}}%
\pgfpathlineto{\pgfqpoint{5.560637in}{2.312241in}}%
\pgfpathlineto{\pgfqpoint{5.563858in}{2.312172in}}%
\pgfpathlineto{\pgfqpoint{5.567080in}{2.311152in}}%
\pgfpathlineto{\pgfqpoint{5.569227in}{2.305871in}}%
\pgfpathlineto{\pgfqpoint{5.570301in}{2.307188in}}%
\pgfpathlineto{\pgfqpoint{5.571375in}{2.311251in}}%
\pgfpathlineto{\pgfqpoint{5.574597in}{2.309635in}}%
\pgfpathlineto{\pgfqpoint{5.576744in}{2.307733in}}%
\pgfpathlineto{\pgfqpoint{5.577818in}{2.307990in}}%
\pgfpathlineto{\pgfqpoint{5.582113in}{2.307732in}}%
\pgfpathlineto{\pgfqpoint{5.583187in}{2.306896in}}%
\pgfpathlineto{\pgfqpoint{5.584261in}{2.307447in}}%
\pgfpathlineto{\pgfqpoint{5.585335in}{2.308708in}}%
\pgfpathlineto{\pgfqpoint{5.589630in}{2.310706in}}%
\pgfpathlineto{\pgfqpoint{5.590704in}{2.310215in}}%
\pgfpathlineto{\pgfqpoint{5.593926in}{2.315537in}}%
\pgfpathlineto{\pgfqpoint{5.597147in}{2.314067in}}%
\pgfpathlineto{\pgfqpoint{5.598221in}{2.314393in}}%
\pgfpathlineto{\pgfqpoint{5.599295in}{2.315772in}}%
\pgfpathlineto{\pgfqpoint{5.600369in}{2.315130in}}%
\pgfpathlineto{\pgfqpoint{5.601443in}{2.317584in}}%
\pgfpathlineto{\pgfqpoint{5.605738in}{2.313812in}}%
\pgfpathlineto{\pgfqpoint{5.606812in}{2.314300in}}%
\pgfpathlineto{\pgfqpoint{5.608959in}{2.311129in}}%
\pgfpathlineto{\pgfqpoint{5.612181in}{2.312139in}}%
\pgfpathlineto{\pgfqpoint{5.613255in}{2.308158in}}%
\pgfpathlineto{\pgfqpoint{5.614329in}{2.308180in}}%
\pgfpathlineto{\pgfqpoint{5.615402in}{2.306212in}}%
\pgfpathlineto{\pgfqpoint{5.616476in}{2.302671in}}%
\pgfpathlineto{\pgfqpoint{5.619698in}{2.303697in}}%
\pgfpathlineto{\pgfqpoint{5.620772in}{2.305348in}}%
\pgfpathlineto{\pgfqpoint{5.621845in}{2.303701in}}%
\pgfpathlineto{\pgfqpoint{5.622919in}{2.304435in}}%
\pgfpathlineto{\pgfqpoint{5.627215in}{2.300925in}}%
\pgfpathlineto{\pgfqpoint{5.628289in}{2.300885in}}%
\pgfpathlineto{\pgfqpoint{5.629362in}{2.302672in}}%
\pgfpathlineto{\pgfqpoint{5.630436in}{2.305771in}}%
\pgfpathlineto{\pgfqpoint{5.631510in}{2.303735in}}%
\pgfpathlineto{\pgfqpoint{5.635805in}{2.302857in}}%
\pgfpathlineto{\pgfqpoint{5.636879in}{2.301080in}}%
\pgfpathlineto{\pgfqpoint{5.639027in}{2.302247in}}%
\pgfpathlineto{\pgfqpoint{5.642248in}{2.301646in}}%
\pgfpathlineto{\pgfqpoint{5.644396in}{2.302662in}}%
\pgfpathlineto{\pgfqpoint{5.645470in}{2.301253in}}%
\pgfpathlineto{\pgfqpoint{5.646544in}{2.303446in}}%
\pgfpathlineto{\pgfqpoint{5.649765in}{2.304912in}}%
\pgfpathlineto{\pgfqpoint{5.652987in}{2.300262in}}%
\pgfpathlineto{\pgfqpoint{5.654061in}{2.299131in}}%
\pgfpathlineto{\pgfqpoint{5.659430in}{2.300036in}}%
\pgfpathlineto{\pgfqpoint{5.661578in}{2.302171in}}%
\pgfpathlineto{\pgfqpoint{5.664799in}{2.303212in}}%
\pgfpathlineto{\pgfqpoint{5.666947in}{2.302887in}}%
\pgfpathlineto{\pgfqpoint{5.668021in}{2.300339in}}%
\pgfpathlineto{\pgfqpoint{5.669094in}{2.299636in}}%
\pgfpathlineto{\pgfqpoint{5.672316in}{2.299346in}}%
\pgfpathlineto{\pgfqpoint{5.673390in}{2.300348in}}%
\pgfpathlineto{\pgfqpoint{5.675537in}{2.300094in}}%
\pgfpathlineto{\pgfqpoint{5.683054in}{2.301108in}}%
\pgfpathlineto{\pgfqpoint{5.684128in}{2.298936in}}%
\pgfpathlineto{\pgfqpoint{5.688423in}{2.299569in}}%
\pgfpathlineto{\pgfqpoint{5.689497in}{2.298448in}}%
\pgfpathlineto{\pgfqpoint{5.690571in}{2.299419in}}%
\pgfpathlineto{\pgfqpoint{5.691645in}{2.299496in}}%
\pgfpathlineto{\pgfqpoint{5.697014in}{2.300774in}}%
\pgfpathlineto{\pgfqpoint{5.703457in}{2.298174in}}%
\pgfpathlineto{\pgfqpoint{5.705605in}{2.302477in}}%
\pgfpathlineto{\pgfqpoint{5.709900in}{2.301352in}}%
\pgfpathlineto{\pgfqpoint{5.710974in}{2.304684in}}%
\pgfpathlineto{\pgfqpoint{5.712048in}{2.304061in}}%
\pgfpathlineto{\pgfqpoint{5.713122in}{2.304288in}}%
\pgfpathlineto{\pgfqpoint{5.717417in}{2.302051in}}%
\pgfpathlineto{\pgfqpoint{5.718491in}{2.301810in}}%
\pgfpathlineto{\pgfqpoint{5.719565in}{2.300912in}}%
\pgfpathlineto{\pgfqpoint{5.720639in}{2.301444in}}%
\pgfpathlineto{\pgfqpoint{5.721712in}{2.299954in}}%
\pgfpathlineto{\pgfqpoint{5.724934in}{2.299974in}}%
\pgfpathlineto{\pgfqpoint{5.726008in}{2.299350in}}%
\pgfpathlineto{\pgfqpoint{5.727082in}{2.299930in}}%
\pgfpathlineto{\pgfqpoint{5.728155in}{2.299462in}}%
\pgfpathlineto{\pgfqpoint{5.729229in}{2.301532in}}%
\pgfpathlineto{\pgfqpoint{5.732451in}{2.300815in}}%
\pgfpathlineto{\pgfqpoint{5.734599in}{2.304002in}}%
\pgfpathlineto{\pgfqpoint{5.735672in}{2.303385in}}%
\pgfpathlineto{\pgfqpoint{5.736746in}{2.303691in}}%
\pgfpathlineto{\pgfqpoint{5.739968in}{2.303363in}}%
\pgfpathlineto{\pgfqpoint{5.742115in}{2.301013in}}%
\pgfpathlineto{\pgfqpoint{5.743189in}{2.301488in}}%
\pgfpathlineto{\pgfqpoint{5.744263in}{2.301110in}}%
\pgfpathlineto{\pgfqpoint{5.748558in}{2.301803in}}%
\pgfpathlineto{\pgfqpoint{5.749632in}{2.300444in}}%
\pgfpathlineto{\pgfqpoint{5.750706in}{2.300169in}}%
\pgfpathlineto{\pgfqpoint{5.751780in}{2.299250in}}%
\pgfpathlineto{\pgfqpoint{5.755001in}{2.298729in}}%
\pgfpathlineto{\pgfqpoint{5.756075in}{2.299496in}}%
\pgfpathlineto{\pgfqpoint{5.759297in}{2.297974in}}%
\pgfpathlineto{\pgfqpoint{5.762518in}{2.299055in}}%
\pgfpathlineto{\pgfqpoint{5.763592in}{2.297688in}}%
\pgfpathlineto{\pgfqpoint{5.764666in}{2.298368in}}%
\pgfpathlineto{\pgfqpoint{5.765740in}{2.297586in}}%
\pgfpathlineto{\pgfqpoint{5.766814in}{2.298058in}}%
\pgfpathlineto{\pgfqpoint{5.770035in}{2.298210in}}%
\pgfpathlineto{\pgfqpoint{5.773257in}{2.296228in}}%
\pgfpathlineto{\pgfqpoint{5.774331in}{2.299484in}}%
\pgfpathlineto{\pgfqpoint{5.777552in}{2.301287in}}%
\pgfpathlineto{\pgfqpoint{5.780774in}{2.295272in}}%
\pgfpathlineto{\pgfqpoint{5.787217in}{2.294678in}}%
\pgfpathlineto{\pgfqpoint{5.788290in}{2.295004in}}%
\pgfpathlineto{\pgfqpoint{5.789364in}{2.293457in}}%
\pgfpathlineto{\pgfqpoint{5.800103in}{2.293189in}}%
\pgfpathlineto{\pgfqpoint{5.801177in}{2.292817in}}%
\pgfpathlineto{\pgfqpoint{5.803324in}{2.293205in}}%
\pgfpathlineto{\pgfqpoint{5.804398in}{2.292460in}}%
\pgfpathlineto{\pgfqpoint{5.807620in}{2.292337in}}%
\pgfpathlineto{\pgfqpoint{5.809767in}{2.294473in}}%
\pgfpathlineto{\pgfqpoint{5.810841in}{2.293912in}}%
\pgfpathlineto{\pgfqpoint{5.811915in}{2.292640in}}%
\pgfpathlineto{\pgfqpoint{5.816210in}{2.290767in}}%
\pgfpathlineto{\pgfqpoint{5.817284in}{2.292000in}}%
\pgfpathlineto{\pgfqpoint{5.819432in}{2.292296in}}%
\pgfpathlineto{\pgfqpoint{5.823727in}{2.291963in}}%
\pgfpathlineto{\pgfqpoint{5.826949in}{2.290320in}}%
\pgfpathlineto{\pgfqpoint{5.830170in}{2.290354in}}%
\pgfpathlineto{\pgfqpoint{5.831244in}{2.291035in}}%
\pgfpathlineto{\pgfqpoint{5.833392in}{2.289683in}}%
\pgfpathlineto{\pgfqpoint{5.837687in}{2.290596in}}%
\pgfpathlineto{\pgfqpoint{5.838761in}{2.289741in}}%
\pgfpathlineto{\pgfqpoint{5.839835in}{2.289876in}}%
\pgfpathlineto{\pgfqpoint{5.840909in}{2.288964in}}%
\pgfpathlineto{\pgfqpoint{5.841982in}{2.289464in}}%
\pgfpathlineto{\pgfqpoint{5.845204in}{2.288355in}}%
\pgfpathlineto{\pgfqpoint{5.847352in}{2.289848in}}%
\pgfpathlineto{\pgfqpoint{5.848425in}{2.288310in}}%
\pgfpathlineto{\pgfqpoint{5.849499in}{2.288475in}}%
\pgfpathlineto{\pgfqpoint{5.853795in}{2.287813in}}%
\pgfpathlineto{\pgfqpoint{5.854868in}{2.288846in}}%
\pgfpathlineto{\pgfqpoint{5.855942in}{2.289096in}}%
\pgfpathlineto{\pgfqpoint{5.857016in}{2.291436in}}%
\pgfpathlineto{\pgfqpoint{5.860238in}{2.288285in}}%
\pgfpathlineto{\pgfqpoint{5.861311in}{2.290087in}}%
\pgfpathlineto{\pgfqpoint{5.862385in}{2.290138in}}%
\pgfpathlineto{\pgfqpoint{5.863459in}{2.288523in}}%
\pgfpathlineto{\pgfqpoint{5.867755in}{2.288058in}}%
\pgfpathlineto{\pgfqpoint{5.868828in}{2.287745in}}%
\pgfpathlineto{\pgfqpoint{5.869902in}{2.288910in}}%
\pgfpathlineto{\pgfqpoint{5.870976in}{2.287090in}}%
\pgfpathlineto{\pgfqpoint{5.872050in}{2.288909in}}%
\pgfpathlineto{\pgfqpoint{5.875271in}{2.288776in}}%
\pgfpathlineto{\pgfqpoint{5.876345in}{2.288009in}}%
\pgfpathlineto{\pgfqpoint{5.877419in}{2.286362in}}%
\pgfpathlineto{\pgfqpoint{5.878493in}{2.288146in}}%
\pgfpathlineto{\pgfqpoint{5.879567in}{2.285867in}}%
\pgfpathlineto{\pgfqpoint{5.883862in}{2.287904in}}%
\pgfpathlineto{\pgfqpoint{5.887084in}{2.285464in}}%
\pgfpathlineto{\pgfqpoint{5.893527in}{2.288048in}}%
\pgfpathlineto{\pgfqpoint{5.894600in}{2.287751in}}%
\pgfpathlineto{\pgfqpoint{5.898896in}{2.289220in}}%
\pgfpathlineto{\pgfqpoint{5.901044in}{2.292103in}}%
\pgfpathlineto{\pgfqpoint{5.902117in}{2.293067in}}%
\pgfpathlineto{\pgfqpoint{5.905339in}{2.293440in}}%
\pgfpathlineto{\pgfqpoint{5.906413in}{2.288738in}}%
\pgfpathlineto{\pgfqpoint{5.907487in}{2.288087in}}%
\pgfpathlineto{\pgfqpoint{5.908560in}{2.289327in}}%
\pgfpathlineto{\pgfqpoint{5.909634in}{2.288923in}}%
\pgfpathlineto{\pgfqpoint{5.916077in}{2.289292in}}%
\pgfpathlineto{\pgfqpoint{5.917151in}{2.291632in}}%
\pgfpathlineto{\pgfqpoint{5.921446in}{2.287880in}}%
\pgfpathlineto{\pgfqpoint{5.922520in}{2.290190in}}%
\pgfpathlineto{\pgfqpoint{5.923594in}{2.294853in}}%
\pgfpathlineto{\pgfqpoint{5.924668in}{2.293810in}}%
\pgfpathlineto{\pgfqpoint{5.927889in}{2.294766in}}%
\pgfpathlineto{\pgfqpoint{5.928963in}{2.293723in}}%
\pgfpathlineto{\pgfqpoint{5.932185in}{2.296433in}}%
\pgfpathlineto{\pgfqpoint{5.936480in}{2.295125in}}%
\pgfpathlineto{\pgfqpoint{5.937554in}{2.295254in}}%
\pgfpathlineto{\pgfqpoint{5.939702in}{2.293946in}}%
\pgfpathlineto{\pgfqpoint{5.943997in}{2.294889in}}%
\pgfpathlineto{\pgfqpoint{5.946145in}{2.296641in}}%
\pgfpathlineto{\pgfqpoint{5.947219in}{2.295701in}}%
\pgfpathlineto{\pgfqpoint{5.951514in}{2.294828in}}%
\pgfpathlineto{\pgfqpoint{5.952588in}{2.292703in}}%
\pgfpathlineto{\pgfqpoint{5.953662in}{2.293804in}}%
\pgfpathlineto{\pgfqpoint{5.954735in}{2.292360in}}%
\pgfpathlineto{\pgfqpoint{5.959031in}{2.291053in}}%
\pgfpathlineto{\pgfqpoint{5.960105in}{2.292337in}}%
\pgfpathlineto{\pgfqpoint{5.962252in}{2.291842in}}%
\pgfpathlineto{\pgfqpoint{5.966548in}{2.292017in}}%
\pgfpathlineto{\pgfqpoint{5.967621in}{2.292492in}}%
\pgfpathlineto{\pgfqpoint{5.969769in}{2.291380in}}%
\pgfpathlineto{\pgfqpoint{5.974065in}{2.291955in}}%
\pgfpathlineto{\pgfqpoint{5.975138in}{2.292817in}}%
\pgfpathlineto{\pgfqpoint{5.976212in}{2.292354in}}%
\pgfpathlineto{\pgfqpoint{5.977286in}{2.292796in}}%
\pgfpathlineto{\pgfqpoint{5.982655in}{2.292103in}}%
\pgfpathlineto{\pgfqpoint{5.983729in}{2.291204in}}%
\pgfpathlineto{\pgfqpoint{5.984803in}{2.291257in}}%
\pgfpathlineto{\pgfqpoint{5.988024in}{2.292266in}}%
\pgfpathlineto{\pgfqpoint{5.989098in}{2.293732in}}%
\pgfpathlineto{\pgfqpoint{5.997689in}{2.290285in}}%
\pgfpathlineto{\pgfqpoint{5.998763in}{2.290663in}}%
\pgfpathlineto{\pgfqpoint{5.999837in}{2.286279in}}%
\pgfpathlineto{\pgfqpoint{6.003058in}{2.287007in}}%
\pgfpathlineto{\pgfqpoint{6.004132in}{2.285650in}}%
\pgfpathlineto{\pgfqpoint{6.006280in}{2.287521in}}%
\pgfpathlineto{\pgfqpoint{6.010575in}{2.287290in}}%
\pgfpathlineto{\pgfqpoint{6.011649in}{2.286009in}}%
\pgfpathlineto{\pgfqpoint{6.012723in}{2.286412in}}%
\pgfpathlineto{\pgfqpoint{6.013797in}{2.285763in}}%
\pgfpathlineto{\pgfqpoint{6.014870in}{2.286277in}}%
\pgfpathlineto{\pgfqpoint{6.018092in}{2.286293in}}%
\pgfpathlineto{\pgfqpoint{6.020240in}{2.284908in}}%
\pgfpathlineto{\pgfqpoint{6.021313in}{2.284417in}}%
\pgfpathlineto{\pgfqpoint{6.022387in}{2.285423in}}%
\pgfpathlineto{\pgfqpoint{6.025609in}{2.284928in}}%
\pgfpathlineto{\pgfqpoint{6.026683in}{2.285594in}}%
\pgfpathlineto{\pgfqpoint{6.027756in}{2.280758in}}%
\pgfpathlineto{\pgfqpoint{6.028830in}{2.281219in}}%
\pgfpathlineto{\pgfqpoint{6.029904in}{2.280800in}}%
\pgfpathlineto{\pgfqpoint{6.035273in}{2.280320in}}%
\pgfpathlineto{\pgfqpoint{6.037421in}{2.280852in}}%
\pgfpathlineto{\pgfqpoint{6.041716in}{2.280821in}}%
\pgfpathlineto{\pgfqpoint{6.042790in}{2.280003in}}%
\pgfpathlineto{\pgfqpoint{6.044938in}{2.281598in}}%
\pgfpathlineto{\pgfqpoint{6.051381in}{2.281808in}}%
\pgfpathlineto{\pgfqpoint{6.052455in}{2.280795in}}%
\pgfpathlineto{\pgfqpoint{6.058898in}{2.280287in}}%
\pgfpathlineto{\pgfqpoint{6.059972in}{2.280877in}}%
\pgfpathlineto{\pgfqpoint{6.065341in}{2.280891in}}%
\pgfpathlineto{\pgfqpoint{6.067488in}{2.281473in}}%
\pgfpathlineto{\pgfqpoint{6.073932in}{2.281983in}}%
\pgfpathlineto{\pgfqpoint{6.075005in}{2.282483in}}%
\pgfpathlineto{\pgfqpoint{6.085744in}{2.282982in}}%
\pgfpathlineto{\pgfqpoint{6.088965in}{2.282209in}}%
\pgfpathlineto{\pgfqpoint{6.095408in}{2.281836in}}%
\pgfpathlineto{\pgfqpoint{6.096482in}{2.282230in}}%
\pgfpathlineto{\pgfqpoint{6.097556in}{2.283235in}}%
\pgfpathlineto{\pgfqpoint{6.101851in}{2.281253in}}%
\pgfpathlineto{\pgfqpoint{6.102925in}{2.284417in}}%
\pgfpathlineto{\pgfqpoint{6.105073in}{2.285016in}}%
\pgfpathlineto{\pgfqpoint{6.108294in}{2.285366in}}%
\pgfpathlineto{\pgfqpoint{6.109368in}{2.286663in}}%
\pgfpathlineto{\pgfqpoint{6.112590in}{2.286239in}}%
\pgfpathlineto{\pgfqpoint{6.117959in}{2.286254in}}%
\pgfpathlineto{\pgfqpoint{6.120107in}{2.286708in}}%
\pgfpathlineto{\pgfqpoint{6.125476in}{2.286594in}}%
\pgfpathlineto{\pgfqpoint{6.126550in}{2.287197in}}%
\pgfpathlineto{\pgfqpoint{6.127623in}{2.286621in}}%
\pgfpathlineto{\pgfqpoint{6.132993in}{2.286227in}}%
\pgfpathlineto{\pgfqpoint{6.135140in}{2.285097in}}%
\pgfpathlineto{\pgfqpoint{6.139436in}{2.284874in}}%
\pgfpathlineto{\pgfqpoint{6.140509in}{2.283857in}}%
\pgfpathlineto{\pgfqpoint{6.141583in}{2.283794in}}%
\pgfpathlineto{\pgfqpoint{6.142657in}{2.283138in}}%
\pgfpathlineto{\pgfqpoint{6.148026in}{2.282876in}}%
\pgfpathlineto{\pgfqpoint{6.149100in}{2.284184in}}%
\pgfpathlineto{\pgfqpoint{6.150174in}{2.283744in}}%
\pgfpathlineto{\pgfqpoint{6.155543in}{2.283321in}}%
\pgfpathlineto{\pgfqpoint{6.156617in}{2.281972in}}%
\pgfpathlineto{\pgfqpoint{6.157691in}{2.281577in}}%
\pgfpathlineto{\pgfqpoint{6.161986in}{2.281621in}}%
\pgfpathlineto{\pgfqpoint{6.164134in}{2.282469in}}%
\pgfpathlineto{\pgfqpoint{6.165208in}{2.281903in}}%
\pgfpathlineto{\pgfqpoint{6.168429in}{2.281994in}}%
\pgfpathlineto{\pgfqpoint{6.171651in}{2.285385in}}%
\pgfpathlineto{\pgfqpoint{6.178094in}{2.284377in}}%
\pgfpathlineto{\pgfqpoint{6.179168in}{2.284819in}}%
\pgfpathlineto{\pgfqpoint{6.180242in}{2.284406in}}%
\pgfpathlineto{\pgfqpoint{6.186685in}{2.285801in}}%
\pgfpathlineto{\pgfqpoint{6.187758in}{2.285204in}}%
\pgfpathlineto{\pgfqpoint{6.190980in}{2.284533in}}%
\pgfpathlineto{\pgfqpoint{6.192054in}{2.282936in}}%
\pgfpathlineto{\pgfqpoint{6.194201in}{2.281998in}}%
\pgfpathlineto{\pgfqpoint{6.195275in}{2.281983in}}%
\pgfpathlineto{\pgfqpoint{6.198497in}{2.282591in}}%
\pgfpathlineto{\pgfqpoint{6.199571in}{2.281209in}}%
\pgfpathlineto{\pgfqpoint{6.200644in}{2.280984in}}%
\pgfpathlineto{\pgfqpoint{6.201718in}{2.279054in}}%
\pgfpathlineto{\pgfqpoint{6.202792in}{2.279692in}}%
\pgfpathlineto{\pgfqpoint{6.208161in}{2.278553in}}%
\pgfpathlineto{\pgfqpoint{6.210309in}{2.279057in}}%
\pgfpathlineto{\pgfqpoint{6.215678in}{2.277358in}}%
\pgfpathlineto{\pgfqpoint{6.217826in}{2.278136in}}%
\pgfpathlineto{\pgfqpoint{6.221047in}{2.277721in}}%
\pgfpathlineto{\pgfqpoint{6.223195in}{2.276660in}}%
\pgfpathlineto{\pgfqpoint{6.224269in}{2.277149in}}%
\pgfpathlineto{\pgfqpoint{6.225343in}{2.276628in}}%
\pgfpathlineto{\pgfqpoint{6.229638in}{2.276209in}}%
\pgfpathlineto{\pgfqpoint{6.231786in}{2.276876in}}%
\pgfpathlineto{\pgfqpoint{6.232860in}{2.276567in}}%
\pgfpathlineto{\pgfqpoint{6.237155in}{2.276818in}}%
\pgfpathlineto{\pgfqpoint{6.238229in}{2.277406in}}%
\pgfpathlineto{\pgfqpoint{6.240376in}{2.276533in}}%
\pgfpathlineto{\pgfqpoint{6.247893in}{2.276129in}}%
\pgfpathlineto{\pgfqpoint{6.251115in}{2.274635in}}%
\pgfpathlineto{\pgfqpoint{6.252189in}{2.275242in}}%
\pgfpathlineto{\pgfqpoint{6.255410in}{2.275555in}}%
\pgfpathlineto{\pgfqpoint{6.258632in}{2.275706in}}%
\pgfpathlineto{\pgfqpoint{6.259706in}{2.274387in}}%
\pgfpathlineto{\pgfqpoint{6.260779in}{2.274105in}}%
\pgfpathlineto{\pgfqpoint{6.261853in}{2.276312in}}%
\pgfpathlineto{\pgfqpoint{6.262927in}{2.276826in}}%
\pgfpathlineto{\pgfqpoint{6.267222in}{2.276293in}}%
\pgfpathlineto{\pgfqpoint{6.268296in}{2.278611in}}%
\pgfpathlineto{\pgfqpoint{6.270444in}{2.278467in}}%
\pgfpathlineto{\pgfqpoint{6.275813in}{2.276528in}}%
\pgfpathlineto{\pgfqpoint{6.276887in}{2.277044in}}%
\pgfpathlineto{\pgfqpoint{6.277961in}{2.276650in}}%
\pgfpathlineto{\pgfqpoint{6.282256in}{2.277575in}}%
\pgfpathlineto{\pgfqpoint{6.283330in}{2.277788in}}%
\pgfpathlineto{\pgfqpoint{6.285478in}{2.275699in}}%
\pgfpathlineto{\pgfqpoint{6.289773in}{2.276014in}}%
\pgfpathlineto{\pgfqpoint{6.290847in}{2.276056in}}%
\pgfpathlineto{\pgfqpoint{6.291921in}{2.276692in}}%
\pgfpathlineto{\pgfqpoint{6.292995in}{2.281111in}}%
\pgfpathlineto{\pgfqpoint{6.298364in}{2.283097in}}%
\pgfpathlineto{\pgfqpoint{6.299438in}{2.282165in}}%
\pgfpathlineto{\pgfqpoint{6.300511in}{2.282823in}}%
\pgfpathlineto{\pgfqpoint{6.304807in}{2.283841in}}%
\pgfpathlineto{\pgfqpoint{6.305881in}{2.283025in}}%
\pgfpathlineto{\pgfqpoint{6.306954in}{2.283583in}}%
\pgfpathlineto{\pgfqpoint{6.308028in}{2.283489in}}%
\pgfpathlineto{\pgfqpoint{6.311250in}{2.284270in}}%
\pgfpathlineto{\pgfqpoint{6.313397in}{2.281996in}}%
\pgfpathlineto{\pgfqpoint{6.315545in}{2.281159in}}%
\pgfpathlineto{\pgfqpoint{6.319841in}{2.280146in}}%
\pgfpathlineto{\pgfqpoint{6.320914in}{2.281051in}}%
\pgfpathlineto{\pgfqpoint{6.321988in}{2.279592in}}%
\pgfpathlineto{\pgfqpoint{6.323062in}{2.280738in}}%
\pgfpathlineto{\pgfqpoint{6.326284in}{2.280962in}}%
\pgfpathlineto{\pgfqpoint{6.327357in}{2.280317in}}%
\pgfpathlineto{\pgfqpoint{6.328431in}{2.280867in}}%
\pgfpathlineto{\pgfqpoint{6.333800in}{2.279984in}}%
\pgfpathlineto{\pgfqpoint{6.334874in}{2.279362in}}%
\pgfpathlineto{\pgfqpoint{6.335948in}{2.279392in}}%
\pgfpathlineto{\pgfqpoint{6.338096in}{2.278036in}}%
\pgfpathlineto{\pgfqpoint{6.342391in}{2.276631in}}%
\pgfpathlineto{\pgfqpoint{6.343465in}{2.276828in}}%
\pgfpathlineto{\pgfqpoint{6.344539in}{2.278359in}}%
\pgfpathlineto{\pgfqpoint{6.345613in}{2.277984in}}%
\pgfpathlineto{\pgfqpoint{6.349908in}{2.278688in}}%
\pgfpathlineto{\pgfqpoint{6.350982in}{2.277282in}}%
\pgfpathlineto{\pgfqpoint{6.352056in}{2.277127in}}%
\pgfpathlineto{\pgfqpoint{6.353130in}{2.275940in}}%
\pgfpathlineto{\pgfqpoint{6.360646in}{2.275618in}}%
\pgfpathlineto{\pgfqpoint{6.364942in}{2.275163in}}%
\pgfpathlineto{\pgfqpoint{6.367089in}{2.275697in}}%
\pgfpathlineto{\pgfqpoint{6.368163in}{2.275932in}}%
\pgfpathlineto{\pgfqpoint{6.368163in}{2.275932in}}%
\pgfusepath{stroke}%
\end{pgfscope}%
\begin{pgfscope}%
\pgfsetrectcap%
\pgfsetmiterjoin%
\pgfsetlinewidth{0.803000pt}%
\definecolor{currentstroke}{rgb}{1.000000,1.000000,1.000000}%
\pgfsetstrokecolor{currentstroke}%
\pgfsetdash{}{0pt}%
\pgfpathmoveto{\pgfqpoint{3.902254in}{2.255883in}}%
\pgfpathlineto{\pgfqpoint{3.902254in}{2.656768in}}%
\pgfusepath{stroke}%
\end{pgfscope}%
\begin{pgfscope}%
\pgfsetrectcap%
\pgfsetmiterjoin%
\pgfsetlinewidth{0.803000pt}%
\definecolor{currentstroke}{rgb}{1.000000,1.000000,1.000000}%
\pgfsetstrokecolor{currentstroke}%
\pgfsetdash{}{0pt}%
\pgfpathmoveto{\pgfqpoint{6.485588in}{2.255883in}}%
\pgfpathlineto{\pgfqpoint{6.485588in}{2.656768in}}%
\pgfusepath{stroke}%
\end{pgfscope}%
\begin{pgfscope}%
\pgfsetrectcap%
\pgfsetmiterjoin%
\pgfsetlinewidth{0.803000pt}%
\definecolor{currentstroke}{rgb}{1.000000,1.000000,1.000000}%
\pgfsetstrokecolor{currentstroke}%
\pgfsetdash{}{0pt}%
\pgfpathmoveto{\pgfqpoint{3.902254in}{2.255883in}}%
\pgfpathlineto{\pgfqpoint{6.485588in}{2.255883in}}%
\pgfusepath{stroke}%
\end{pgfscope}%
\begin{pgfscope}%
\pgfsetrectcap%
\pgfsetmiterjoin%
\pgfsetlinewidth{0.803000pt}%
\definecolor{currentstroke}{rgb}{1.000000,1.000000,1.000000}%
\pgfsetstrokecolor{currentstroke}%
\pgfsetdash{}{0pt}%
\pgfpathmoveto{\pgfqpoint{3.902254in}{2.656768in}}%
\pgfpathlineto{\pgfqpoint{6.485588in}{2.656768in}}%
\pgfusepath{stroke}%
\end{pgfscope}%
\begin{pgfscope}%
\definecolor{textcolor}{rgb}{0.150000,0.150000,0.150000}%
\pgfsetstrokecolor{textcolor}%
\pgfsetfillcolor{textcolor}%
\pgftext[x=5.193921in,y=2.740101in,,base]{\color{textcolor}\rmfamily\fontsize{12.000000}{14.400000}\selectfont PG}%
\end{pgfscope}%
\begin{pgfscope}%
\pgfsetbuttcap%
\pgfsetmiterjoin%
\definecolor{currentfill}{rgb}{0.917647,0.917647,0.949020}%
\pgfsetfillcolor{currentfill}%
\pgfsetlinewidth{0.000000pt}%
\definecolor{currentstroke}{rgb}{0.000000,0.000000,0.000000}%
\pgfsetstrokecolor{currentstroke}%
\pgfsetstrokeopacity{0.000000}%
\pgfsetdash{}{0pt}%
\pgfpathmoveto{\pgfqpoint{0.285588in}{1.293759in}}%
\pgfpathlineto{\pgfqpoint{2.868921in}{1.293759in}}%
\pgfpathlineto{\pgfqpoint{2.868921in}{1.694644in}}%
\pgfpathlineto{\pgfqpoint{0.285588in}{1.694644in}}%
\pgfpathclose%
\pgfusepath{fill}%
\end{pgfscope}%
\begin{pgfscope}%
\pgfpathrectangle{\pgfqpoint{0.285588in}{1.293759in}}{\pgfqpoint{2.583333in}{0.400885in}}%
\pgfusepath{clip}%
\pgfsetroundcap%
\pgfsetroundjoin%
\pgfsetlinewidth{0.803000pt}%
\definecolor{currentstroke}{rgb}{1.000000,1.000000,1.000000}%
\pgfsetstrokecolor{currentstroke}%
\pgfsetdash{}{0pt}%
\pgfpathmoveto{\pgfqpoint{0.400864in}{1.293759in}}%
\pgfpathlineto{\pgfqpoint{0.400864in}{1.694644in}}%
\pgfusepath{stroke}%
\end{pgfscope}%
\begin{pgfscope}%
\definecolor{textcolor}{rgb}{0.150000,0.150000,0.150000}%
\pgfsetstrokecolor{textcolor}%
\pgfsetfillcolor{textcolor}%
\pgftext[x=0.400864in,y=1.196537in,,top]{\color{textcolor}\rmfamily\fontsize{10.000000}{12.000000}\selectfont 2012}%
\end{pgfscope}%
\begin{pgfscope}%
\pgfpathrectangle{\pgfqpoint{0.285588in}{1.293759in}}{\pgfqpoint{2.583333in}{0.400885in}}%
\pgfusepath{clip}%
\pgfsetroundcap%
\pgfsetroundjoin%
\pgfsetlinewidth{0.803000pt}%
\definecolor{currentstroke}{rgb}{1.000000,1.000000,1.000000}%
\pgfsetstrokecolor{currentstroke}%
\pgfsetdash{}{0pt}%
\pgfpathmoveto{\pgfqpoint{0.793889in}{1.293759in}}%
\pgfpathlineto{\pgfqpoint{0.793889in}{1.694644in}}%
\pgfusepath{stroke}%
\end{pgfscope}%
\begin{pgfscope}%
\definecolor{textcolor}{rgb}{0.150000,0.150000,0.150000}%
\pgfsetstrokecolor{textcolor}%
\pgfsetfillcolor{textcolor}%
\pgftext[x=0.793889in,y=1.196537in,,top]{\color{textcolor}\rmfamily\fontsize{10.000000}{12.000000}\selectfont 2013}%
\end{pgfscope}%
\begin{pgfscope}%
\pgfpathrectangle{\pgfqpoint{0.285588in}{1.293759in}}{\pgfqpoint{2.583333in}{0.400885in}}%
\pgfusepath{clip}%
\pgfsetroundcap%
\pgfsetroundjoin%
\pgfsetlinewidth{0.803000pt}%
\definecolor{currentstroke}{rgb}{1.000000,1.000000,1.000000}%
\pgfsetstrokecolor{currentstroke}%
\pgfsetdash{}{0pt}%
\pgfpathmoveto{\pgfqpoint{1.185840in}{1.293759in}}%
\pgfpathlineto{\pgfqpoint{1.185840in}{1.694644in}}%
\pgfusepath{stroke}%
\end{pgfscope}%
\begin{pgfscope}%
\definecolor{textcolor}{rgb}{0.150000,0.150000,0.150000}%
\pgfsetstrokecolor{textcolor}%
\pgfsetfillcolor{textcolor}%
\pgftext[x=1.185840in,y=1.196537in,,top]{\color{textcolor}\rmfamily\fontsize{10.000000}{12.000000}\selectfont 2014}%
\end{pgfscope}%
\begin{pgfscope}%
\pgfpathrectangle{\pgfqpoint{0.285588in}{1.293759in}}{\pgfqpoint{2.583333in}{0.400885in}}%
\pgfusepath{clip}%
\pgfsetroundcap%
\pgfsetroundjoin%
\pgfsetlinewidth{0.803000pt}%
\definecolor{currentstroke}{rgb}{1.000000,1.000000,1.000000}%
\pgfsetstrokecolor{currentstroke}%
\pgfsetdash{}{0pt}%
\pgfpathmoveto{\pgfqpoint{1.577791in}{1.293759in}}%
\pgfpathlineto{\pgfqpoint{1.577791in}{1.694644in}}%
\pgfusepath{stroke}%
\end{pgfscope}%
\begin{pgfscope}%
\definecolor{textcolor}{rgb}{0.150000,0.150000,0.150000}%
\pgfsetstrokecolor{textcolor}%
\pgfsetfillcolor{textcolor}%
\pgftext[x=1.577791in,y=1.196537in,,top]{\color{textcolor}\rmfamily\fontsize{10.000000}{12.000000}\selectfont 2015}%
\end{pgfscope}%
\begin{pgfscope}%
\pgfpathrectangle{\pgfqpoint{0.285588in}{1.293759in}}{\pgfqpoint{2.583333in}{0.400885in}}%
\pgfusepath{clip}%
\pgfsetroundcap%
\pgfsetroundjoin%
\pgfsetlinewidth{0.803000pt}%
\definecolor{currentstroke}{rgb}{1.000000,1.000000,1.000000}%
\pgfsetstrokecolor{currentstroke}%
\pgfsetdash{}{0pt}%
\pgfpathmoveto{\pgfqpoint{1.969742in}{1.293759in}}%
\pgfpathlineto{\pgfqpoint{1.969742in}{1.694644in}}%
\pgfusepath{stroke}%
\end{pgfscope}%
\begin{pgfscope}%
\definecolor{textcolor}{rgb}{0.150000,0.150000,0.150000}%
\pgfsetstrokecolor{textcolor}%
\pgfsetfillcolor{textcolor}%
\pgftext[x=1.969742in,y=1.196537in,,top]{\color{textcolor}\rmfamily\fontsize{10.000000}{12.000000}\selectfont 2016}%
\end{pgfscope}%
\begin{pgfscope}%
\pgfpathrectangle{\pgfqpoint{0.285588in}{1.293759in}}{\pgfqpoint{2.583333in}{0.400885in}}%
\pgfusepath{clip}%
\pgfsetroundcap%
\pgfsetroundjoin%
\pgfsetlinewidth{0.803000pt}%
\definecolor{currentstroke}{rgb}{1.000000,1.000000,1.000000}%
\pgfsetstrokecolor{currentstroke}%
\pgfsetdash{}{0pt}%
\pgfpathmoveto{\pgfqpoint{2.362767in}{1.293759in}}%
\pgfpathlineto{\pgfqpoint{2.362767in}{1.694644in}}%
\pgfusepath{stroke}%
\end{pgfscope}%
\begin{pgfscope}%
\definecolor{textcolor}{rgb}{0.150000,0.150000,0.150000}%
\pgfsetstrokecolor{textcolor}%
\pgfsetfillcolor{textcolor}%
\pgftext[x=2.362767in,y=1.196537in,,top]{\color{textcolor}\rmfamily\fontsize{10.000000}{12.000000}\selectfont 2017}%
\end{pgfscope}%
\begin{pgfscope}%
\pgfpathrectangle{\pgfqpoint{0.285588in}{1.293759in}}{\pgfqpoint{2.583333in}{0.400885in}}%
\pgfusepath{clip}%
\pgfsetroundcap%
\pgfsetroundjoin%
\pgfsetlinewidth{0.803000pt}%
\definecolor{currentstroke}{rgb}{1.000000,1.000000,1.000000}%
\pgfsetstrokecolor{currentstroke}%
\pgfsetdash{}{0pt}%
\pgfpathmoveto{\pgfqpoint{2.754718in}{1.293759in}}%
\pgfpathlineto{\pgfqpoint{2.754718in}{1.694644in}}%
\pgfusepath{stroke}%
\end{pgfscope}%
\begin{pgfscope}%
\definecolor{textcolor}{rgb}{0.150000,0.150000,0.150000}%
\pgfsetstrokecolor{textcolor}%
\pgfsetfillcolor{textcolor}%
\pgftext[x=2.754718in,y=1.196537in,,top]{\color{textcolor}\rmfamily\fontsize{10.000000}{12.000000}\selectfont 2018}%
\end{pgfscope}%
\begin{pgfscope}%
\pgfpathrectangle{\pgfqpoint{0.285588in}{1.293759in}}{\pgfqpoint{2.583333in}{0.400885in}}%
\pgfusepath{clip}%
\pgfsetroundcap%
\pgfsetroundjoin%
\pgfsetlinewidth{0.803000pt}%
\definecolor{currentstroke}{rgb}{1.000000,1.000000,1.000000}%
\pgfsetstrokecolor{currentstroke}%
\pgfsetdash{}{0pt}%
\pgfpathmoveto{\pgfqpoint{0.285588in}{1.469626in}}%
\pgfpathlineto{\pgfqpoint{2.868921in}{1.469626in}}%
\pgfusepath{stroke}%
\end{pgfscope}%
\begin{pgfscope}%
\definecolor{textcolor}{rgb}{0.150000,0.150000,0.150000}%
\pgfsetstrokecolor{textcolor}%
\pgfsetfillcolor{textcolor}%
\pgftext[x=0.100000in,y=1.416864in,left,base]{\color{textcolor}\rmfamily\fontsize{10.000000}{12.000000}\selectfont 1}%
\end{pgfscope}%
\begin{pgfscope}%
\pgfpathrectangle{\pgfqpoint{0.285588in}{1.293759in}}{\pgfqpoint{2.583333in}{0.400885in}}%
\pgfusepath{clip}%
\pgfsetroundcap%
\pgfsetroundjoin%
\pgfsetlinewidth{0.803000pt}%
\definecolor{currentstroke}{rgb}{1.000000,1.000000,1.000000}%
\pgfsetstrokecolor{currentstroke}%
\pgfsetdash{}{0pt}%
\pgfpathmoveto{\pgfqpoint{0.285588in}{1.680800in}}%
\pgfpathlineto{\pgfqpoint{2.868921in}{1.680800in}}%
\pgfusepath{stroke}%
\end{pgfscope}%
\begin{pgfscope}%
\definecolor{textcolor}{rgb}{0.150000,0.150000,0.150000}%
\pgfsetstrokecolor{textcolor}%
\pgfsetfillcolor{textcolor}%
\pgftext[x=0.100000in,y=1.628039in,left,base]{\color{textcolor}\rmfamily\fontsize{10.000000}{12.000000}\selectfont 2}%
\end{pgfscope}%
\begin{pgfscope}%
\pgfpathrectangle{\pgfqpoint{0.285588in}{1.293759in}}{\pgfqpoint{2.583333in}{0.400885in}}%
\pgfusepath{clip}%
\pgfsetroundcap%
\pgfsetroundjoin%
\pgfsetlinewidth{1.505625pt}%
\definecolor{currentstroke}{rgb}{0.890196,0.466667,0.760784}%
\pgfsetstrokecolor{currentstroke}%
\pgfsetstrokeopacity{0.300000}%
\pgfsetdash{}{0pt}%
\pgfpathmoveto{\pgfqpoint{0.403012in}{1.469626in}}%
\pgfpathlineto{\pgfqpoint{0.404086in}{1.470712in}}%
\pgfpathlineto{\pgfqpoint{0.406233in}{1.467454in}}%
\pgfpathlineto{\pgfqpoint{0.409455in}{1.468031in}}%
\pgfpathlineto{\pgfqpoint{0.410529in}{1.473495in}}%
\pgfpathlineto{\pgfqpoint{0.412676in}{1.476923in}}%
\pgfpathlineto{\pgfqpoint{0.413750in}{1.473631in}}%
\pgfpathlineto{\pgfqpoint{0.418045in}{1.476346in}}%
\pgfpathlineto{\pgfqpoint{0.419119in}{1.477941in}}%
\pgfpathlineto{\pgfqpoint{0.421267in}{1.475362in}}%
\pgfpathlineto{\pgfqpoint{0.424489in}{1.475837in}}%
\pgfpathlineto{\pgfqpoint{0.425562in}{1.478450in}}%
\pgfpathlineto{\pgfqpoint{0.427710in}{1.477398in}}%
\pgfpathlineto{\pgfqpoint{0.428784in}{1.477975in}}%
\pgfpathlineto{\pgfqpoint{0.432005in}{1.477941in}}%
\pgfpathlineto{\pgfqpoint{0.433079in}{1.480045in}}%
\pgfpathlineto{\pgfqpoint{0.434153in}{1.485340in}}%
\pgfpathlineto{\pgfqpoint{0.435227in}{1.484763in}}%
\pgfpathlineto{\pgfqpoint{0.436301in}{1.487682in}}%
\pgfpathlineto{\pgfqpoint{0.440596in}{1.485510in}}%
\pgfpathlineto{\pgfqpoint{0.442744in}{1.495420in}}%
\pgfpathlineto{\pgfqpoint{0.443818in}{1.494606in}}%
\pgfpathlineto{\pgfqpoint{0.447039in}{1.498509in}}%
\pgfpathlineto{\pgfqpoint{0.448113in}{1.497830in}}%
\pgfpathlineto{\pgfqpoint{0.449187in}{1.493418in}}%
\pgfpathlineto{\pgfqpoint{0.451334in}{1.496744in}}%
\pgfpathlineto{\pgfqpoint{0.456704in}{1.497253in}}%
\pgfpathlineto{\pgfqpoint{0.457778in}{1.495929in}}%
\pgfpathlineto{\pgfqpoint{0.458851in}{1.497287in}}%
\pgfpathlineto{\pgfqpoint{0.463147in}{1.495861in}}%
\pgfpathlineto{\pgfqpoint{0.466368in}{1.498916in}}%
\pgfpathlineto{\pgfqpoint{0.469590in}{1.495352in}}%
\pgfpathlineto{\pgfqpoint{0.470664in}{1.489956in}}%
\pgfpathlineto{\pgfqpoint{0.472811in}{1.496065in}}%
\pgfpathlineto{\pgfqpoint{0.473885in}{1.496099in}}%
\pgfpathlineto{\pgfqpoint{0.477107in}{1.497355in}}%
\pgfpathlineto{\pgfqpoint{0.478180in}{1.504686in}}%
\pgfpathlineto{\pgfqpoint{0.480328in}{1.505602in}}%
\pgfpathlineto{\pgfqpoint{0.481402in}{1.501597in}}%
\pgfpathlineto{\pgfqpoint{0.484623in}{1.499153in}}%
\pgfpathlineto{\pgfqpoint{0.485697in}{1.495284in}}%
\pgfpathlineto{\pgfqpoint{0.488919in}{1.491144in}}%
\pgfpathlineto{\pgfqpoint{0.492140in}{1.495963in}}%
\pgfpathlineto{\pgfqpoint{0.493214in}{1.494843in}}%
\pgfpathlineto{\pgfqpoint{0.494288in}{1.490329in}}%
\pgfpathlineto{\pgfqpoint{0.496436in}{1.494368in}}%
\pgfpathlineto{\pgfqpoint{0.499657in}{1.493757in}}%
\pgfpathlineto{\pgfqpoint{0.501805in}{1.491483in}}%
\pgfpathlineto{\pgfqpoint{0.507174in}{1.486664in}}%
\pgfpathlineto{\pgfqpoint{0.508248in}{1.481674in}}%
\pgfpathlineto{\pgfqpoint{0.510396in}{1.489345in}}%
\pgfpathlineto{\pgfqpoint{0.511469in}{1.485442in}}%
\pgfpathlineto{\pgfqpoint{0.514691in}{1.485679in}}%
\pgfpathlineto{\pgfqpoint{0.515765in}{1.489548in}}%
\pgfpathlineto{\pgfqpoint{0.516839in}{1.489413in}}%
\pgfpathlineto{\pgfqpoint{0.517912in}{1.487139in}}%
\pgfpathlineto{\pgfqpoint{0.518986in}{1.488836in}}%
\pgfpathlineto{\pgfqpoint{0.522208in}{1.485306in}}%
\pgfpathlineto{\pgfqpoint{0.524355in}{1.485476in}}%
\pgfpathlineto{\pgfqpoint{0.526503in}{1.491687in}}%
\pgfpathlineto{\pgfqpoint{0.531872in}{1.490431in}}%
\pgfpathlineto{\pgfqpoint{0.532946in}{1.488598in}}%
\pgfpathlineto{\pgfqpoint{0.534020in}{1.484525in}}%
\pgfpathlineto{\pgfqpoint{0.538315in}{1.482659in}}%
\pgfpathlineto{\pgfqpoint{0.539389in}{1.477500in}}%
\pgfpathlineto{\pgfqpoint{0.540463in}{1.478314in}}%
\pgfpathlineto{\pgfqpoint{0.541537in}{1.477975in}}%
\pgfpathlineto{\pgfqpoint{0.544758in}{1.474479in}}%
\pgfpathlineto{\pgfqpoint{0.545832in}{1.475124in}}%
\pgfpathlineto{\pgfqpoint{0.549054in}{1.465655in}}%
\pgfpathlineto{\pgfqpoint{0.553349in}{1.470101in}}%
\pgfpathlineto{\pgfqpoint{0.554423in}{1.470440in}}%
\pgfpathlineto{\pgfqpoint{0.556571in}{1.467454in}}%
\pgfpathlineto{\pgfqpoint{0.560866in}{1.473393in}}%
\pgfpathlineto{\pgfqpoint{0.561940in}{1.469524in}}%
\pgfpathlineto{\pgfqpoint{0.563014in}{1.470576in}}%
\pgfpathlineto{\pgfqpoint{0.564088in}{1.464603in}}%
\pgfpathlineto{\pgfqpoint{0.567309in}{1.462973in}}%
\pgfpathlineto{\pgfqpoint{0.568383in}{1.461344in}}%
\pgfpathlineto{\pgfqpoint{0.570531in}{1.474275in}}%
\pgfpathlineto{\pgfqpoint{0.571604in}{1.474581in}}%
\pgfpathlineto{\pgfqpoint{0.575900in}{1.471289in}}%
\pgfpathlineto{\pgfqpoint{0.576974in}{1.468947in}}%
\pgfpathlineto{\pgfqpoint{0.579121in}{1.471662in}}%
\pgfpathlineto{\pgfqpoint{0.582343in}{1.473122in}}%
\pgfpathlineto{\pgfqpoint{0.583417in}{1.477160in}}%
\pgfpathlineto{\pgfqpoint{0.584490in}{1.475735in}}%
\pgfpathlineto{\pgfqpoint{0.585564in}{1.472612in}}%
\pgfpathlineto{\pgfqpoint{0.586638in}{1.473698in}}%
\pgfpathlineto{\pgfqpoint{0.589860in}{1.469558in}}%
\pgfpathlineto{\pgfqpoint{0.590933in}{1.469218in}}%
\pgfpathlineto{\pgfqpoint{0.592007in}{1.470746in}}%
\pgfpathlineto{\pgfqpoint{0.593081in}{1.466266in}}%
\pgfpathlineto{\pgfqpoint{0.594155in}{1.474649in}}%
\pgfpathlineto{\pgfqpoint{0.597377in}{1.473257in}}%
\pgfpathlineto{\pgfqpoint{0.598450in}{1.475294in}}%
\pgfpathlineto{\pgfqpoint{0.600598in}{1.474242in}}%
\pgfpathlineto{\pgfqpoint{0.601672in}{1.470542in}}%
\pgfpathlineto{\pgfqpoint{0.604893in}{1.471221in}}%
\pgfpathlineto{\pgfqpoint{0.605967in}{1.470712in}}%
\pgfpathlineto{\pgfqpoint{0.607041in}{1.465994in}}%
\pgfpathlineto{\pgfqpoint{0.608115in}{1.464093in}}%
\pgfpathlineto{\pgfqpoint{0.609189in}{1.469117in}}%
\pgfpathlineto{\pgfqpoint{0.612410in}{1.468031in}}%
\pgfpathlineto{\pgfqpoint{0.613484in}{1.469117in}}%
\pgfpathlineto{\pgfqpoint{0.615632in}{1.475497in}}%
\pgfpathlineto{\pgfqpoint{0.616706in}{1.470949in}}%
\pgfpathlineto{\pgfqpoint{0.619927in}{1.468200in}}%
\pgfpathlineto{\pgfqpoint{0.621001in}{1.464365in}}%
\pgfpathlineto{\pgfqpoint{0.624222in}{1.471085in}}%
\pgfpathlineto{\pgfqpoint{0.627444in}{1.473054in}}%
\pgfpathlineto{\pgfqpoint{0.628518in}{1.471526in}}%
\pgfpathlineto{\pgfqpoint{0.629592in}{1.472545in}}%
\pgfpathlineto{\pgfqpoint{0.630666in}{1.471526in}}%
\pgfpathlineto{\pgfqpoint{0.631739in}{1.478314in}}%
\pgfpathlineto{\pgfqpoint{0.634961in}{1.477873in}}%
\pgfpathlineto{\pgfqpoint{0.636035in}{1.482082in}}%
\pgfpathlineto{\pgfqpoint{0.638182in}{1.478959in}}%
\pgfpathlineto{\pgfqpoint{0.639256in}{1.481403in}}%
\pgfpathlineto{\pgfqpoint{0.642478in}{1.480215in}}%
\pgfpathlineto{\pgfqpoint{0.643552in}{1.481097in}}%
\pgfpathlineto{\pgfqpoint{0.645699in}{1.485476in}}%
\pgfpathlineto{\pgfqpoint{0.646773in}{1.490091in}}%
\pgfpathlineto{\pgfqpoint{0.649995in}{1.489277in}}%
\pgfpathlineto{\pgfqpoint{0.651068in}{1.486935in}}%
\pgfpathlineto{\pgfqpoint{0.652142in}{1.488055in}}%
\pgfpathlineto{\pgfqpoint{0.653216in}{1.486731in}}%
\pgfpathlineto{\pgfqpoint{0.654290in}{1.489277in}}%
\pgfpathlineto{\pgfqpoint{0.658585in}{1.490872in}}%
\pgfpathlineto{\pgfqpoint{0.659659in}{1.489515in}}%
\pgfpathlineto{\pgfqpoint{0.660733in}{1.486324in}}%
\pgfpathlineto{\pgfqpoint{0.661807in}{1.488598in}}%
\pgfpathlineto{\pgfqpoint{0.667176in}{1.483371in}}%
\pgfpathlineto{\pgfqpoint{0.668250in}{1.487308in}}%
\pgfpathlineto{\pgfqpoint{0.669324in}{1.487342in}}%
\pgfpathlineto{\pgfqpoint{0.672545in}{1.485034in}}%
\pgfpathlineto{\pgfqpoint{0.674693in}{1.485679in}}%
\pgfpathlineto{\pgfqpoint{0.676841in}{1.496099in}}%
\pgfpathlineto{\pgfqpoint{0.680062in}{1.495250in}}%
\pgfpathlineto{\pgfqpoint{0.681136in}{1.493316in}}%
\pgfpathlineto{\pgfqpoint{0.682210in}{1.493995in}}%
\pgfpathlineto{\pgfqpoint{0.683284in}{1.491687in}}%
\pgfpathlineto{\pgfqpoint{0.684357in}{1.491212in}}%
\pgfpathlineto{\pgfqpoint{0.687579in}{1.489107in}}%
\pgfpathlineto{\pgfqpoint{0.688653in}{1.485476in}}%
\pgfpathlineto{\pgfqpoint{0.690800in}{1.484288in}}%
\pgfpathlineto{\pgfqpoint{0.691874in}{1.484118in}}%
\pgfpathlineto{\pgfqpoint{0.702613in}{1.484322in}}%
\pgfpathlineto{\pgfqpoint{0.704760in}{1.477568in}}%
\pgfpathlineto{\pgfqpoint{0.710130in}{1.477941in}}%
\pgfpathlineto{\pgfqpoint{0.712277in}{1.484831in}}%
\pgfpathlineto{\pgfqpoint{0.713351in}{1.486833in}}%
\pgfpathlineto{\pgfqpoint{0.714425in}{1.483236in}}%
\pgfpathlineto{\pgfqpoint{0.717646in}{1.482794in}}%
\pgfpathlineto{\pgfqpoint{0.718720in}{1.480588in}}%
\pgfpathlineto{\pgfqpoint{0.719794in}{1.482998in}}%
\pgfpathlineto{\pgfqpoint{0.720868in}{1.481233in}}%
\pgfpathlineto{\pgfqpoint{0.721942in}{1.483847in}}%
\pgfpathlineto{\pgfqpoint{0.727311in}{1.483745in}}%
\pgfpathlineto{\pgfqpoint{0.728385in}{1.486324in}}%
\pgfpathlineto{\pgfqpoint{0.729459in}{1.483473in}}%
\pgfpathlineto{\pgfqpoint{0.732680in}{1.482998in}}%
\pgfpathlineto{\pgfqpoint{0.733754in}{1.488938in}}%
\pgfpathlineto{\pgfqpoint{0.735902in}{1.477975in}}%
\pgfpathlineto{\pgfqpoint{0.736976in}{1.477059in}}%
\pgfpathlineto{\pgfqpoint{0.740197in}{1.480249in}}%
\pgfpathlineto{\pgfqpoint{0.741271in}{1.480385in}}%
\pgfpathlineto{\pgfqpoint{0.742345in}{1.475124in}}%
\pgfpathlineto{\pgfqpoint{0.743419in}{1.475667in}}%
\pgfpathlineto{\pgfqpoint{0.749862in}{1.482896in}}%
\pgfpathlineto{\pgfqpoint{0.752009in}{1.486630in}}%
\pgfpathlineto{\pgfqpoint{0.756305in}{1.487241in}}%
\pgfpathlineto{\pgfqpoint{0.757378in}{1.490058in}}%
\pgfpathlineto{\pgfqpoint{0.758452in}{1.489956in}}%
\pgfpathlineto{\pgfqpoint{0.759526in}{1.490974in}}%
\pgfpathlineto{\pgfqpoint{0.762748in}{1.490058in}}%
\pgfpathlineto{\pgfqpoint{0.767043in}{1.493519in}}%
\pgfpathlineto{\pgfqpoint{0.772412in}{1.493553in}}%
\pgfpathlineto{\pgfqpoint{0.774560in}{1.490601in}}%
\pgfpathlineto{\pgfqpoint{0.777781in}{1.490635in}}%
\pgfpathlineto{\pgfqpoint{0.778855in}{1.497117in}}%
\pgfpathlineto{\pgfqpoint{0.779929in}{1.499459in}}%
\pgfpathlineto{\pgfqpoint{0.781003in}{1.499866in}}%
\pgfpathlineto{\pgfqpoint{0.782077in}{1.498033in}}%
\pgfpathlineto{\pgfqpoint{0.788520in}{1.496642in}}%
\pgfpathlineto{\pgfqpoint{0.789594in}{1.493010in}}%
\pgfpathlineto{\pgfqpoint{0.792815in}{1.496472in}}%
\pgfpathlineto{\pgfqpoint{0.794963in}{1.502276in}}%
\pgfpathlineto{\pgfqpoint{0.796037in}{1.503158in}}%
\pgfpathlineto{\pgfqpoint{0.797110in}{1.505093in}}%
\pgfpathlineto{\pgfqpoint{0.800332in}{1.503905in}}%
\pgfpathlineto{\pgfqpoint{0.801406in}{1.500952in}}%
\pgfpathlineto{\pgfqpoint{0.802480in}{1.503871in}}%
\pgfpathlineto{\pgfqpoint{0.804627in}{1.505670in}}%
\pgfpathlineto{\pgfqpoint{0.808923in}{1.507944in}}%
\pgfpathlineto{\pgfqpoint{0.809997in}{1.506824in}}%
\pgfpathlineto{\pgfqpoint{0.812144in}{1.510829in}}%
\pgfpathlineto{\pgfqpoint{0.816440in}{1.512322in}}%
\pgfpathlineto{\pgfqpoint{0.818587in}{1.516531in}}%
\pgfpathlineto{\pgfqpoint{0.819661in}{1.519076in}}%
\pgfpathlineto{\pgfqpoint{0.823956in}{1.518974in}}%
\pgfpathlineto{\pgfqpoint{0.825030in}{1.517074in}}%
\pgfpathlineto{\pgfqpoint{0.826104in}{1.512628in}}%
\pgfpathlineto{\pgfqpoint{0.827178in}{1.519212in}}%
\pgfpathlineto{\pgfqpoint{0.832547in}{1.517854in}}%
\pgfpathlineto{\pgfqpoint{0.833621in}{1.519517in}}%
\pgfpathlineto{\pgfqpoint{0.834695in}{1.519925in}}%
\pgfpathlineto{\pgfqpoint{0.837916in}{1.518533in}}%
\pgfpathlineto{\pgfqpoint{0.838990in}{1.519619in}}%
\pgfpathlineto{\pgfqpoint{0.841138in}{1.520332in}}%
\pgfpathlineto{\pgfqpoint{0.842212in}{1.523522in}}%
\pgfpathlineto{\pgfqpoint{0.846507in}{1.524201in}}%
\pgfpathlineto{\pgfqpoint{0.848655in}{1.519517in}}%
\pgfpathlineto{\pgfqpoint{0.849729in}{1.522674in}}%
\pgfpathlineto{\pgfqpoint{0.852950in}{1.516463in}}%
\pgfpathlineto{\pgfqpoint{0.854024in}{1.518907in}}%
\pgfpathlineto{\pgfqpoint{0.855098in}{1.522911in}}%
\pgfpathlineto{\pgfqpoint{0.856172in}{1.522844in}}%
\pgfpathlineto{\pgfqpoint{0.860467in}{1.518703in}}%
\pgfpathlineto{\pgfqpoint{0.861541in}{1.524201in}}%
\pgfpathlineto{\pgfqpoint{0.862615in}{1.524371in}}%
\pgfpathlineto{\pgfqpoint{0.864762in}{1.527290in}}%
\pgfpathlineto{\pgfqpoint{0.869058in}{1.530412in}}%
\pgfpathlineto{\pgfqpoint{0.870131in}{1.530242in}}%
\pgfpathlineto{\pgfqpoint{0.871205in}{1.531295in}}%
\pgfpathlineto{\pgfqpoint{0.875501in}{1.529326in}}%
\pgfpathlineto{\pgfqpoint{0.877648in}{1.531295in}}%
\pgfpathlineto{\pgfqpoint{0.878722in}{1.528342in}}%
\pgfpathlineto{\pgfqpoint{0.879796in}{1.531736in}}%
\pgfpathlineto{\pgfqpoint{0.884091in}{1.528987in}}%
\pgfpathlineto{\pgfqpoint{0.885165in}{1.528851in}}%
\pgfpathlineto{\pgfqpoint{0.886239in}{1.531261in}}%
\pgfpathlineto{\pgfqpoint{0.891608in}{1.529971in}}%
\pgfpathlineto{\pgfqpoint{0.892682in}{1.530582in}}%
\pgfpathlineto{\pgfqpoint{0.893756in}{1.530480in}}%
\pgfpathlineto{\pgfqpoint{0.894830in}{1.528953in}}%
\pgfpathlineto{\pgfqpoint{0.898051in}{1.532075in}}%
\pgfpathlineto{\pgfqpoint{0.901273in}{1.538218in}}%
\pgfpathlineto{\pgfqpoint{0.902347in}{1.537845in}}%
\pgfpathlineto{\pgfqpoint{0.905568in}{1.531498in}}%
\pgfpathlineto{\pgfqpoint{0.906642in}{1.534417in}}%
\pgfpathlineto{\pgfqpoint{0.908790in}{1.525762in}}%
\pgfpathlineto{\pgfqpoint{0.909864in}{1.530650in}}%
\pgfpathlineto{\pgfqpoint{0.913085in}{1.531838in}}%
\pgfpathlineto{\pgfqpoint{0.915233in}{1.527086in}}%
\pgfpathlineto{\pgfqpoint{0.916307in}{1.527391in}}%
\pgfpathlineto{\pgfqpoint{0.917380in}{1.524608in}}%
\pgfpathlineto{\pgfqpoint{0.920602in}{1.525966in}}%
\pgfpathlineto{\pgfqpoint{0.922750in}{1.524303in}}%
\pgfpathlineto{\pgfqpoint{0.923823in}{1.526339in}}%
\pgfpathlineto{\pgfqpoint{0.924897in}{1.530310in}}%
\pgfpathlineto{\pgfqpoint{0.928119in}{1.531430in}}%
\pgfpathlineto{\pgfqpoint{0.932414in}{1.536555in}}%
\pgfpathlineto{\pgfqpoint{0.935636in}{1.535707in}}%
\pgfpathlineto{\pgfqpoint{0.936709in}{1.538558in}}%
\pgfpathlineto{\pgfqpoint{0.937783in}{1.539780in}}%
\pgfpathlineto{\pgfqpoint{0.938857in}{1.537879in}}%
\pgfpathlineto{\pgfqpoint{0.939931in}{1.544294in}}%
\pgfpathlineto{\pgfqpoint{0.943153in}{1.543954in}}%
\pgfpathlineto{\pgfqpoint{0.944226in}{1.544871in}}%
\pgfpathlineto{\pgfqpoint{0.946374in}{1.538829in}}%
\pgfpathlineto{\pgfqpoint{0.947448in}{1.537506in}}%
\pgfpathlineto{\pgfqpoint{0.951743in}{1.540255in}}%
\pgfpathlineto{\pgfqpoint{0.952817in}{1.537743in}}%
\pgfpathlineto{\pgfqpoint{0.953891in}{1.540221in}}%
\pgfpathlineto{\pgfqpoint{0.954965in}{1.537098in}}%
\pgfpathlineto{\pgfqpoint{0.958186in}{1.538083in}}%
\pgfpathlineto{\pgfqpoint{0.961408in}{1.530785in}}%
\pgfpathlineto{\pgfqpoint{0.962482in}{1.535910in}}%
\pgfpathlineto{\pgfqpoint{0.965703in}{1.534858in}}%
\pgfpathlineto{\pgfqpoint{0.966777in}{1.533501in}}%
\pgfpathlineto{\pgfqpoint{0.967851in}{1.530582in}}%
\pgfpathlineto{\pgfqpoint{0.968925in}{1.535300in}}%
\pgfpathlineto{\pgfqpoint{0.969998in}{1.534519in}}%
\pgfpathlineto{\pgfqpoint{0.973220in}{1.537336in}}%
\pgfpathlineto{\pgfqpoint{0.974294in}{1.540832in}}%
\pgfpathlineto{\pgfqpoint{0.976442in}{1.529598in}}%
\pgfpathlineto{\pgfqpoint{0.977515in}{1.529122in}}%
\pgfpathlineto{\pgfqpoint{0.980737in}{1.527086in}}%
\pgfpathlineto{\pgfqpoint{0.981811in}{1.527935in}}%
\pgfpathlineto{\pgfqpoint{0.982885in}{1.531464in}}%
\pgfpathlineto{\pgfqpoint{0.983958in}{1.533026in}}%
\pgfpathlineto{\pgfqpoint{0.985032in}{1.531329in}}%
\pgfpathlineto{\pgfqpoint{0.988254in}{1.536623in}}%
\pgfpathlineto{\pgfqpoint{0.989328in}{1.533874in}}%
\pgfpathlineto{\pgfqpoint{0.992549in}{1.541884in}}%
\pgfpathlineto{\pgfqpoint{0.995771in}{1.543208in}}%
\pgfpathlineto{\pgfqpoint{0.996844in}{1.546262in}}%
\pgfpathlineto{\pgfqpoint{0.997918in}{1.545583in}}%
\pgfpathlineto{\pgfqpoint{0.998992in}{1.551149in}}%
\pgfpathlineto{\pgfqpoint{1.003287in}{1.552779in}}%
\pgfpathlineto{\pgfqpoint{1.004361in}{1.552032in}}%
\pgfpathlineto{\pgfqpoint{1.007583in}{1.559363in}}%
\pgfpathlineto{\pgfqpoint{1.010804in}{1.558277in}}%
\pgfpathlineto{\pgfqpoint{1.011878in}{1.567101in}}%
\pgfpathlineto{\pgfqpoint{1.014026in}{1.566151in}}%
\pgfpathlineto{\pgfqpoint{1.015100in}{1.566660in}}%
\pgfpathlineto{\pgfqpoint{1.018321in}{1.566965in}}%
\pgfpathlineto{\pgfqpoint{1.019395in}{1.568425in}}%
\pgfpathlineto{\pgfqpoint{1.020469in}{1.568425in}}%
\pgfpathlineto{\pgfqpoint{1.021543in}{1.573143in}}%
\pgfpathlineto{\pgfqpoint{1.022617in}{1.574873in}}%
\pgfpathlineto{\pgfqpoint{1.025838in}{1.571581in}}%
\pgfpathlineto{\pgfqpoint{1.026912in}{1.567339in}}%
\pgfpathlineto{\pgfqpoint{1.027986in}{1.569782in}}%
\pgfpathlineto{\pgfqpoint{1.029060in}{1.570461in}}%
\pgfpathlineto{\pgfqpoint{1.030133in}{1.568561in}}%
\pgfpathlineto{\pgfqpoint{1.033355in}{1.568323in}}%
\pgfpathlineto{\pgfqpoint{1.034429in}{1.572023in}}%
\pgfpathlineto{\pgfqpoint{1.035503in}{1.568493in}}%
\pgfpathlineto{\pgfqpoint{1.036576in}{1.562384in}}%
\pgfpathlineto{\pgfqpoint{1.037650in}{1.562655in}}%
\pgfpathlineto{\pgfqpoint{1.041946in}{1.560619in}}%
\pgfpathlineto{\pgfqpoint{1.043020in}{1.558582in}}%
\pgfpathlineto{\pgfqpoint{1.044093in}{1.562282in}}%
\pgfpathlineto{\pgfqpoint{1.048389in}{1.560347in}}%
\pgfpathlineto{\pgfqpoint{1.049463in}{1.553389in}}%
\pgfpathlineto{\pgfqpoint{1.050536in}{1.553491in}}%
\pgfpathlineto{\pgfqpoint{1.051610in}{1.554917in}}%
\pgfpathlineto{\pgfqpoint{1.052684in}{1.553831in}}%
\pgfpathlineto{\pgfqpoint{1.059127in}{1.564420in}}%
\pgfpathlineto{\pgfqpoint{1.060201in}{1.563096in}}%
\pgfpathlineto{\pgfqpoint{1.063422in}{1.566762in}}%
\pgfpathlineto{\pgfqpoint{1.065570in}{1.576672in}}%
\pgfpathlineto{\pgfqpoint{1.066644in}{1.576672in}}%
\pgfpathlineto{\pgfqpoint{1.067718in}{1.578301in}}%
\pgfpathlineto{\pgfqpoint{1.072013in}{1.584207in}}%
\pgfpathlineto{\pgfqpoint{1.074161in}{1.588959in}}%
\pgfpathlineto{\pgfqpoint{1.075235in}{1.581831in}}%
\pgfpathlineto{\pgfqpoint{1.078456in}{1.581356in}}%
\pgfpathlineto{\pgfqpoint{1.079530in}{1.583053in}}%
\pgfpathlineto{\pgfqpoint{1.080604in}{1.580881in}}%
\pgfpathlineto{\pgfqpoint{1.081678in}{1.582069in}}%
\pgfpathlineto{\pgfqpoint{1.082752in}{1.581186in}}%
\pgfpathlineto{\pgfqpoint{1.087047in}{1.575315in}}%
\pgfpathlineto{\pgfqpoint{1.089195in}{1.564454in}}%
\pgfpathlineto{\pgfqpoint{1.090268in}{1.566151in}}%
\pgfpathlineto{\pgfqpoint{1.093490in}{1.565438in}}%
\pgfpathlineto{\pgfqpoint{1.094564in}{1.561705in}}%
\pgfpathlineto{\pgfqpoint{1.095638in}{1.561942in}}%
\pgfpathlineto{\pgfqpoint{1.096711in}{1.571004in}}%
\pgfpathlineto{\pgfqpoint{1.097785in}{1.574229in}}%
\pgfpathlineto{\pgfqpoint{1.101007in}{1.573923in}}%
\pgfpathlineto{\pgfqpoint{1.102081in}{1.570665in}}%
\pgfpathlineto{\pgfqpoint{1.103154in}{1.572532in}}%
\pgfpathlineto{\pgfqpoint{1.104228in}{1.577215in}}%
\pgfpathlineto{\pgfqpoint{1.105302in}{1.576401in}}%
\pgfpathlineto{\pgfqpoint{1.108524in}{1.576027in}}%
\pgfpathlineto{\pgfqpoint{1.109597in}{1.571649in}}%
\pgfpathlineto{\pgfqpoint{1.110671in}{1.572396in}}%
\pgfpathlineto{\pgfqpoint{1.112819in}{1.575756in}}%
\pgfpathlineto{\pgfqpoint{1.116041in}{1.571242in}}%
\pgfpathlineto{\pgfqpoint{1.117114in}{1.572498in}}%
\pgfpathlineto{\pgfqpoint{1.118188in}{1.571106in}}%
\pgfpathlineto{\pgfqpoint{1.119262in}{1.571989in}}%
\pgfpathlineto{\pgfqpoint{1.120336in}{1.575722in}}%
\pgfpathlineto{\pgfqpoint{1.123557in}{1.577114in}}%
\pgfpathlineto{\pgfqpoint{1.124631in}{1.576231in}}%
\pgfpathlineto{\pgfqpoint{1.125705in}{1.579150in}}%
\pgfpathlineto{\pgfqpoint{1.126779in}{1.575009in}}%
\pgfpathlineto{\pgfqpoint{1.127853in}{1.578777in}}%
\pgfpathlineto{\pgfqpoint{1.131074in}{1.577385in}}%
\pgfpathlineto{\pgfqpoint{1.132148in}{1.575552in}}%
\pgfpathlineto{\pgfqpoint{1.133222in}{1.577317in}}%
\pgfpathlineto{\pgfqpoint{1.134296in}{1.580983in}}%
\pgfpathlineto{\pgfqpoint{1.135370in}{1.580677in}}%
\pgfpathlineto{\pgfqpoint{1.138591in}{1.582272in}}%
\pgfpathlineto{\pgfqpoint{1.139665in}{1.582171in}}%
\pgfpathlineto{\pgfqpoint{1.140739in}{1.581390in}}%
\pgfpathlineto{\pgfqpoint{1.142886in}{1.585463in}}%
\pgfpathlineto{\pgfqpoint{1.146108in}{1.585768in}}%
\pgfpathlineto{\pgfqpoint{1.148256in}{1.589332in}}%
\pgfpathlineto{\pgfqpoint{1.150403in}{1.587397in}}%
\pgfpathlineto{\pgfqpoint{1.153625in}{1.585768in}}%
\pgfpathlineto{\pgfqpoint{1.155773in}{1.581695in}}%
\pgfpathlineto{\pgfqpoint{1.156846in}{1.582035in}}%
\pgfpathlineto{\pgfqpoint{1.157920in}{1.588178in}}%
\pgfpathlineto{\pgfqpoint{1.161142in}{1.588348in}}%
\pgfpathlineto{\pgfqpoint{1.162216in}{1.587771in}}%
\pgfpathlineto{\pgfqpoint{1.163289in}{1.580881in}}%
\pgfpathlineto{\pgfqpoint{1.165437in}{1.576978in}}%
\pgfpathlineto{\pgfqpoint{1.168659in}{1.580508in}}%
\pgfpathlineto{\pgfqpoint{1.169732in}{1.577792in}}%
\pgfpathlineto{\pgfqpoint{1.170806in}{1.584275in}}%
\pgfpathlineto{\pgfqpoint{1.171880in}{1.583358in}}%
\pgfpathlineto{\pgfqpoint{1.172954in}{1.586820in}}%
\pgfpathlineto{\pgfqpoint{1.176175in}{1.587296in}}%
\pgfpathlineto{\pgfqpoint{1.179397in}{1.592828in}}%
\pgfpathlineto{\pgfqpoint{1.180471in}{1.593167in}}%
\pgfpathlineto{\pgfqpoint{1.183692in}{1.592963in}}%
\pgfpathlineto{\pgfqpoint{1.184766in}{1.596120in}}%
\pgfpathlineto{\pgfqpoint{1.186914in}{1.592251in}}%
\pgfpathlineto{\pgfqpoint{1.187988in}{1.593473in}}%
\pgfpathlineto{\pgfqpoint{1.191209in}{1.593133in}}%
\pgfpathlineto{\pgfqpoint{1.192283in}{1.595271in}}%
\pgfpathlineto{\pgfqpoint{1.195505in}{1.596222in}}%
\pgfpathlineto{\pgfqpoint{1.199800in}{1.592658in}}%
\pgfpathlineto{\pgfqpoint{1.200874in}{1.596934in}}%
\pgfpathlineto{\pgfqpoint{1.207317in}{1.599650in}}%
\pgfpathlineto{\pgfqpoint{1.208391in}{1.603010in}}%
\pgfpathlineto{\pgfqpoint{1.209464in}{1.599208in}}%
\pgfpathlineto{\pgfqpoint{1.210538in}{1.590180in}}%
\pgfpathlineto{\pgfqpoint{1.213760in}{1.596086in}}%
\pgfpathlineto{\pgfqpoint{1.214834in}{1.596324in}}%
\pgfpathlineto{\pgfqpoint{1.215908in}{1.594559in}}%
\pgfpathlineto{\pgfqpoint{1.216981in}{1.598733in}}%
\pgfpathlineto{\pgfqpoint{1.218055in}{1.596799in}}%
\pgfpathlineto{\pgfqpoint{1.223424in}{1.578641in}}%
\pgfpathlineto{\pgfqpoint{1.225572in}{1.587024in}}%
\pgfpathlineto{\pgfqpoint{1.228794in}{1.589603in}}%
\pgfpathlineto{\pgfqpoint{1.229867in}{1.593642in}}%
\pgfpathlineto{\pgfqpoint{1.233089in}{1.598122in}}%
\pgfpathlineto{\pgfqpoint{1.237384in}{1.597715in}}%
\pgfpathlineto{\pgfqpoint{1.238458in}{1.598631in}}%
\pgfpathlineto{\pgfqpoint{1.239532in}{1.601618in}}%
\pgfpathlineto{\pgfqpoint{1.243827in}{1.605929in}}%
\pgfpathlineto{\pgfqpoint{1.244901in}{1.604130in}}%
\pgfpathlineto{\pgfqpoint{1.245975in}{1.604809in}}%
\pgfpathlineto{\pgfqpoint{1.248123in}{1.607524in}}%
\pgfpathlineto{\pgfqpoint{1.251344in}{1.606743in}}%
\pgfpathlineto{\pgfqpoint{1.252418in}{1.609221in}}%
\pgfpathlineto{\pgfqpoint{1.253492in}{1.608678in}}%
\pgfpathlineto{\pgfqpoint{1.255640in}{1.611359in}}%
\pgfpathlineto{\pgfqpoint{1.258861in}{1.609730in}}%
\pgfpathlineto{\pgfqpoint{1.259935in}{1.603383in}}%
\pgfpathlineto{\pgfqpoint{1.261009in}{1.603892in}}%
\pgfpathlineto{\pgfqpoint{1.262083in}{1.595204in}}%
\pgfpathlineto{\pgfqpoint{1.263156in}{1.594321in}}%
\pgfpathlineto{\pgfqpoint{1.267452in}{1.600532in}}%
\pgfpathlineto{\pgfqpoint{1.268526in}{1.598530in}}%
\pgfpathlineto{\pgfqpoint{1.269599in}{1.597817in}}%
\pgfpathlineto{\pgfqpoint{1.270673in}{1.600193in}}%
\pgfpathlineto{\pgfqpoint{1.273895in}{1.597749in}}%
\pgfpathlineto{\pgfqpoint{1.274969in}{1.602093in}}%
\pgfpathlineto{\pgfqpoint{1.277116in}{1.597919in}}%
\pgfpathlineto{\pgfqpoint{1.278190in}{1.600905in}}%
\pgfpathlineto{\pgfqpoint{1.281412in}{1.606981in}}%
\pgfpathlineto{\pgfqpoint{1.282485in}{1.610205in}}%
\pgfpathlineto{\pgfqpoint{1.283559in}{1.616009in}}%
\pgfpathlineto{\pgfqpoint{1.284633in}{1.615703in}}%
\pgfpathlineto{\pgfqpoint{1.285707in}{1.610952in}}%
\pgfpathlineto{\pgfqpoint{1.290002in}{1.603790in}}%
\pgfpathlineto{\pgfqpoint{1.291076in}{1.607558in}}%
\pgfpathlineto{\pgfqpoint{1.292150in}{1.600159in}}%
\pgfpathlineto{\pgfqpoint{1.293224in}{1.598292in}}%
\pgfpathlineto{\pgfqpoint{1.296445in}{1.601313in}}%
\pgfpathlineto{\pgfqpoint{1.297519in}{1.603994in}}%
\pgfpathlineto{\pgfqpoint{1.298593in}{1.610646in}}%
\pgfpathlineto{\pgfqpoint{1.299667in}{1.612140in}}%
\pgfpathlineto{\pgfqpoint{1.303962in}{1.611325in}}%
\pgfpathlineto{\pgfqpoint{1.306110in}{1.615364in}}%
\pgfpathlineto{\pgfqpoint{1.307184in}{1.613361in}}%
\pgfpathlineto{\pgfqpoint{1.308258in}{1.608067in}}%
\pgfpathlineto{\pgfqpoint{1.311479in}{1.609526in}}%
\pgfpathlineto{\pgfqpoint{1.312553in}{1.609119in}}%
\pgfpathlineto{\pgfqpoint{1.313627in}{1.611427in}}%
\pgfpathlineto{\pgfqpoint{1.314701in}{1.606845in}}%
\pgfpathlineto{\pgfqpoint{1.315774in}{1.606030in}}%
\pgfpathlineto{\pgfqpoint{1.318996in}{1.606913in}}%
\pgfpathlineto{\pgfqpoint{1.320070in}{1.604503in}}%
\pgfpathlineto{\pgfqpoint{1.321144in}{1.607184in}}%
\pgfpathlineto{\pgfqpoint{1.323291in}{1.607422in}}%
\pgfpathlineto{\pgfqpoint{1.326513in}{1.612309in}}%
\pgfpathlineto{\pgfqpoint{1.327587in}{1.612649in}}%
\pgfpathlineto{\pgfqpoint{1.328661in}{1.610137in}}%
\pgfpathlineto{\pgfqpoint{1.330808in}{1.601924in}}%
\pgfpathlineto{\pgfqpoint{1.334030in}{1.603247in}}%
\pgfpathlineto{\pgfqpoint{1.335104in}{1.597511in}}%
\pgfpathlineto{\pgfqpoint{1.336177in}{1.602704in}}%
\pgfpathlineto{\pgfqpoint{1.337251in}{1.603315in}}%
\pgfpathlineto{\pgfqpoint{1.338325in}{1.604809in}}%
\pgfpathlineto{\pgfqpoint{1.343694in}{1.606030in}}%
\pgfpathlineto{\pgfqpoint{1.344768in}{1.607252in}}%
\pgfpathlineto{\pgfqpoint{1.345842in}{1.606845in}}%
\pgfpathlineto{\pgfqpoint{1.350137in}{1.611664in}}%
\pgfpathlineto{\pgfqpoint{1.351211in}{1.609594in}}%
\pgfpathlineto{\pgfqpoint{1.353359in}{1.614889in}}%
\pgfpathlineto{\pgfqpoint{1.356580in}{1.618452in}}%
\pgfpathlineto{\pgfqpoint{1.359802in}{1.608746in}}%
\pgfpathlineto{\pgfqpoint{1.360876in}{1.608576in}}%
\pgfpathlineto{\pgfqpoint{1.365171in}{1.609221in}}%
\pgfpathlineto{\pgfqpoint{1.367319in}{1.610918in}}%
\pgfpathlineto{\pgfqpoint{1.368393in}{1.612173in}}%
\pgfpathlineto{\pgfqpoint{1.371614in}{1.609594in}}%
\pgfpathlineto{\pgfqpoint{1.372688in}{1.605250in}}%
\pgfpathlineto{\pgfqpoint{1.373762in}{1.606539in}}%
\pgfpathlineto{\pgfqpoint{1.374836in}{1.605453in}}%
\pgfpathlineto{\pgfqpoint{1.375909in}{1.607931in}}%
\pgfpathlineto{\pgfqpoint{1.379131in}{1.604537in}}%
\pgfpathlineto{\pgfqpoint{1.380205in}{1.605996in}}%
\pgfpathlineto{\pgfqpoint{1.381279in}{1.603621in}}%
\pgfpathlineto{\pgfqpoint{1.382352in}{1.604775in}}%
\pgfpathlineto{\pgfqpoint{1.386648in}{1.603485in}}%
\pgfpathlineto{\pgfqpoint{1.387722in}{1.600396in}}%
\pgfpathlineto{\pgfqpoint{1.389869in}{1.598733in}}%
\pgfpathlineto{\pgfqpoint{1.390943in}{1.600600in}}%
\pgfpathlineto{\pgfqpoint{1.394165in}{1.602840in}}%
\pgfpathlineto{\pgfqpoint{1.395239in}{1.602704in}}%
\pgfpathlineto{\pgfqpoint{1.396312in}{1.601279in}}%
\pgfpathlineto{\pgfqpoint{1.397386in}{1.596493in}}%
\pgfpathlineto{\pgfqpoint{1.398460in}{1.598903in}}%
\pgfpathlineto{\pgfqpoint{1.401682in}{1.597138in}}%
\pgfpathlineto{\pgfqpoint{1.403829in}{1.587397in}}%
\pgfpathlineto{\pgfqpoint{1.404903in}{1.585055in}}%
\pgfpathlineto{\pgfqpoint{1.405977in}{1.584852in}}%
\pgfpathlineto{\pgfqpoint{1.409198in}{1.585191in}}%
\pgfpathlineto{\pgfqpoint{1.411346in}{1.577351in}}%
\pgfpathlineto{\pgfqpoint{1.412420in}{1.573652in}}%
\pgfpathlineto{\pgfqpoint{1.413494in}{1.572464in}}%
\pgfpathlineto{\pgfqpoint{1.417789in}{1.573278in}}%
\pgfpathlineto{\pgfqpoint{1.418863in}{1.569579in}}%
\pgfpathlineto{\pgfqpoint{1.419937in}{1.570869in}}%
\pgfpathlineto{\pgfqpoint{1.421011in}{1.576095in}}%
\pgfpathlineto{\pgfqpoint{1.424232in}{1.575383in}}%
\pgfpathlineto{\pgfqpoint{1.425306in}{1.572939in}}%
\pgfpathlineto{\pgfqpoint{1.426380in}{1.576672in}}%
\pgfpathlineto{\pgfqpoint{1.427454in}{1.577317in}}%
\pgfpathlineto{\pgfqpoint{1.428528in}{1.576910in}}%
\pgfpathlineto{\pgfqpoint{1.433897in}{1.589196in}}%
\pgfpathlineto{\pgfqpoint{1.434971in}{1.590248in}}%
\pgfpathlineto{\pgfqpoint{1.436044in}{1.588246in}}%
\pgfpathlineto{\pgfqpoint{1.439266in}{1.589536in}}%
\pgfpathlineto{\pgfqpoint{1.440340in}{1.589094in}}%
\pgfpathlineto{\pgfqpoint{1.441414in}{1.587329in}}%
\pgfpathlineto{\pgfqpoint{1.442487in}{1.587397in}}%
\pgfpathlineto{\pgfqpoint{1.443561in}{1.583969in}}%
\pgfpathlineto{\pgfqpoint{1.448930in}{1.587669in}}%
\pgfpathlineto{\pgfqpoint{1.450004in}{1.587703in}}%
\pgfpathlineto{\pgfqpoint{1.451078in}{1.586142in}}%
\pgfpathlineto{\pgfqpoint{1.461817in}{1.585022in}}%
\pgfpathlineto{\pgfqpoint{1.462890in}{1.585666in}}%
\pgfpathlineto{\pgfqpoint{1.463964in}{1.584139in}}%
\pgfpathlineto{\pgfqpoint{1.465038in}{1.585599in}}%
\pgfpathlineto{\pgfqpoint{1.466112in}{1.585395in}}%
\pgfpathlineto{\pgfqpoint{1.470407in}{1.576265in}}%
\pgfpathlineto{\pgfqpoint{1.471481in}{1.578234in}}%
\pgfpathlineto{\pgfqpoint{1.472555in}{1.573448in}}%
\pgfpathlineto{\pgfqpoint{1.473629in}{1.575722in}}%
\pgfpathlineto{\pgfqpoint{1.476850in}{1.575247in}}%
\pgfpathlineto{\pgfqpoint{1.477924in}{1.576808in}}%
\pgfpathlineto{\pgfqpoint{1.478998in}{1.571513in}}%
\pgfpathlineto{\pgfqpoint{1.480072in}{1.569681in}}%
\pgfpathlineto{\pgfqpoint{1.481146in}{1.573244in}}%
\pgfpathlineto{\pgfqpoint{1.484367in}{1.572633in}}%
\pgfpathlineto{\pgfqpoint{1.485441in}{1.564216in}}%
\pgfpathlineto{\pgfqpoint{1.486515in}{1.567848in}}%
\pgfpathlineto{\pgfqpoint{1.487589in}{1.559736in}}%
\pgfpathlineto{\pgfqpoint{1.488662in}{1.559736in}}%
\pgfpathlineto{\pgfqpoint{1.491884in}{1.557836in}}%
\pgfpathlineto{\pgfqpoint{1.492958in}{1.560279in}}%
\pgfpathlineto{\pgfqpoint{1.494032in}{1.557428in}}%
\pgfpathlineto{\pgfqpoint{1.495106in}{1.557598in}}%
\pgfpathlineto{\pgfqpoint{1.496179in}{1.564522in}}%
\pgfpathlineto{\pgfqpoint{1.499401in}{1.564386in}}%
\pgfpathlineto{\pgfqpoint{1.500475in}{1.565879in}}%
\pgfpathlineto{\pgfqpoint{1.501549in}{1.563470in}}%
\pgfpathlineto{\pgfqpoint{1.502622in}{1.569511in}}%
\pgfpathlineto{\pgfqpoint{1.503696in}{1.571446in}}%
\pgfpathlineto{\pgfqpoint{1.506918in}{1.572566in}}%
\pgfpathlineto{\pgfqpoint{1.507992in}{1.578811in}}%
\pgfpathlineto{\pgfqpoint{1.509065in}{1.577555in}}%
\pgfpathlineto{\pgfqpoint{1.511213in}{1.581017in}}%
\pgfpathlineto{\pgfqpoint{1.514435in}{1.578912in}}%
\pgfpathlineto{\pgfqpoint{1.515508in}{1.580677in}}%
\pgfpathlineto{\pgfqpoint{1.517656in}{1.585802in}}%
\pgfpathlineto{\pgfqpoint{1.518730in}{1.587296in}}%
\pgfpathlineto{\pgfqpoint{1.521951in}{1.587058in}}%
\pgfpathlineto{\pgfqpoint{1.523025in}{1.584818in}}%
\pgfpathlineto{\pgfqpoint{1.524099in}{1.586243in}}%
\pgfpathlineto{\pgfqpoint{1.525173in}{1.586243in}}%
\pgfpathlineto{\pgfqpoint{1.526247in}{1.584173in}}%
\pgfpathlineto{\pgfqpoint{1.529468in}{1.583902in}}%
\pgfpathlineto{\pgfqpoint{1.530542in}{1.588246in}}%
\pgfpathlineto{\pgfqpoint{1.531616in}{1.587839in}}%
\pgfpathlineto{\pgfqpoint{1.532690in}{1.588314in}}%
\pgfpathlineto{\pgfqpoint{1.533764in}{1.592794in}}%
\pgfpathlineto{\pgfqpoint{1.536985in}{1.588212in}}%
\pgfpathlineto{\pgfqpoint{1.538059in}{1.597070in}}%
\pgfpathlineto{\pgfqpoint{1.539133in}{1.592386in}}%
\pgfpathlineto{\pgfqpoint{1.541281in}{1.592115in}}%
\pgfpathlineto{\pgfqpoint{1.545576in}{1.590927in}}%
\pgfpathlineto{\pgfqpoint{1.546650in}{1.594898in}}%
\pgfpathlineto{\pgfqpoint{1.548797in}{1.595814in}}%
\pgfpathlineto{\pgfqpoint{1.552019in}{1.601143in}}%
\pgfpathlineto{\pgfqpoint{1.553093in}{1.607015in}}%
\pgfpathlineto{\pgfqpoint{1.554167in}{1.602501in}}%
\pgfpathlineto{\pgfqpoint{1.555240in}{1.604130in}}%
\pgfpathlineto{\pgfqpoint{1.556314in}{1.598394in}}%
\pgfpathlineto{\pgfqpoint{1.559536in}{1.597681in}}%
\pgfpathlineto{\pgfqpoint{1.561684in}{1.603722in}}%
\pgfpathlineto{\pgfqpoint{1.562757in}{1.612988in}}%
\pgfpathlineto{\pgfqpoint{1.563831in}{1.608847in}}%
\pgfpathlineto{\pgfqpoint{1.567053in}{1.613531in}}%
\pgfpathlineto{\pgfqpoint{1.568127in}{1.613735in}}%
\pgfpathlineto{\pgfqpoint{1.569200in}{1.612818in}}%
\pgfpathlineto{\pgfqpoint{1.571348in}{1.613837in}}%
\pgfpathlineto{\pgfqpoint{1.574570in}{1.612683in}}%
\pgfpathlineto{\pgfqpoint{1.575643in}{1.610680in}}%
\pgfpathlineto{\pgfqpoint{1.576717in}{1.607049in}}%
\pgfpathlineto{\pgfqpoint{1.578865in}{1.607150in}}%
\pgfpathlineto{\pgfqpoint{1.582086in}{1.601347in}}%
\pgfpathlineto{\pgfqpoint{1.583160in}{1.596493in}}%
\pgfpathlineto{\pgfqpoint{1.584234in}{1.600159in}}%
\pgfpathlineto{\pgfqpoint{1.585308in}{1.605996in}}%
\pgfpathlineto{\pgfqpoint{1.586382in}{1.604062in}}%
\pgfpathlineto{\pgfqpoint{1.589603in}{1.605385in}}%
\pgfpathlineto{\pgfqpoint{1.590677in}{1.605046in}}%
\pgfpathlineto{\pgfqpoint{1.591751in}{1.602568in}}%
\pgfpathlineto{\pgfqpoint{1.592825in}{1.602568in}}%
\pgfpathlineto{\pgfqpoint{1.593899in}{1.610544in}}%
\pgfpathlineto{\pgfqpoint{1.598194in}{1.614719in}}%
\pgfpathlineto{\pgfqpoint{1.600342in}{1.623577in}}%
\pgfpathlineto{\pgfqpoint{1.604637in}{1.618418in}}%
\pgfpathlineto{\pgfqpoint{1.605711in}{1.619640in}}%
\pgfpathlineto{\pgfqpoint{1.606785in}{1.612852in}}%
\pgfpathlineto{\pgfqpoint{1.607859in}{1.611427in}}%
\pgfpathlineto{\pgfqpoint{1.608932in}{1.606370in}}%
\pgfpathlineto{\pgfqpoint{1.612154in}{1.611834in}}%
\pgfpathlineto{\pgfqpoint{1.613228in}{1.618826in}}%
\pgfpathlineto{\pgfqpoint{1.614302in}{1.615500in}}%
\pgfpathlineto{\pgfqpoint{1.615375in}{1.622525in}}%
\pgfpathlineto{\pgfqpoint{1.616449in}{1.621609in}}%
\pgfpathlineto{\pgfqpoint{1.619671in}{1.620149in}}%
\pgfpathlineto{\pgfqpoint{1.621818in}{1.620183in}}%
\pgfpathlineto{\pgfqpoint{1.622892in}{1.622899in}}%
\pgfpathlineto{\pgfqpoint{1.623966in}{1.627956in}}%
\pgfpathlineto{\pgfqpoint{1.628261in}{1.628193in}}%
\pgfpathlineto{\pgfqpoint{1.630409in}{1.633013in}}%
\pgfpathlineto{\pgfqpoint{1.631483in}{1.636678in}}%
\pgfpathlineto{\pgfqpoint{1.634705in}{1.635490in}}%
\pgfpathlineto{\pgfqpoint{1.635778in}{1.635728in}}%
\pgfpathlineto{\pgfqpoint{1.637926in}{1.632300in}}%
\pgfpathlineto{\pgfqpoint{1.639000in}{1.629992in}}%
\pgfpathlineto{\pgfqpoint{1.642221in}{1.633895in}}%
\pgfpathlineto{\pgfqpoint{1.643295in}{1.629347in}}%
\pgfpathlineto{\pgfqpoint{1.644369in}{1.627345in}}%
\pgfpathlineto{\pgfqpoint{1.645443in}{1.626564in}}%
\pgfpathlineto{\pgfqpoint{1.646517in}{1.622423in}}%
\pgfpathlineto{\pgfqpoint{1.649738in}{1.629483in}}%
\pgfpathlineto{\pgfqpoint{1.650812in}{1.616246in}}%
\pgfpathlineto{\pgfqpoint{1.651886in}{1.619097in}}%
\pgfpathlineto{\pgfqpoint{1.652960in}{1.627956in}}%
\pgfpathlineto{\pgfqpoint{1.654034in}{1.620319in}}%
\pgfpathlineto{\pgfqpoint{1.657255in}{1.624358in}}%
\pgfpathlineto{\pgfqpoint{1.658329in}{1.623747in}}%
\pgfpathlineto{\pgfqpoint{1.659403in}{1.625105in}}%
\pgfpathlineto{\pgfqpoint{1.660477in}{1.622288in}}%
\pgfpathlineto{\pgfqpoint{1.661550in}{1.622525in}}%
\pgfpathlineto{\pgfqpoint{1.664772in}{1.620149in}}%
\pgfpathlineto{\pgfqpoint{1.665846in}{1.620862in}}%
\pgfpathlineto{\pgfqpoint{1.666920in}{1.613361in}}%
\pgfpathlineto{\pgfqpoint{1.667994in}{1.612106in}}%
\pgfpathlineto{\pgfqpoint{1.669067in}{1.614719in}}%
\pgfpathlineto{\pgfqpoint{1.672289in}{1.620591in}}%
\pgfpathlineto{\pgfqpoint{1.674437in}{1.611732in}}%
\pgfpathlineto{\pgfqpoint{1.675510in}{1.615398in}}%
\pgfpathlineto{\pgfqpoint{1.679806in}{1.617638in}}%
\pgfpathlineto{\pgfqpoint{1.680880in}{1.616518in}}%
\pgfpathlineto{\pgfqpoint{1.681953in}{1.617570in}}%
\pgfpathlineto{\pgfqpoint{1.683027in}{1.617706in}}%
\pgfpathlineto{\pgfqpoint{1.684101in}{1.619471in}}%
\pgfpathlineto{\pgfqpoint{1.687323in}{1.616280in}}%
\pgfpathlineto{\pgfqpoint{1.689470in}{1.617672in}}%
\pgfpathlineto{\pgfqpoint{1.690544in}{1.616484in}}%
\pgfpathlineto{\pgfqpoint{1.691618in}{1.609255in}}%
\pgfpathlineto{\pgfqpoint{1.695913in}{1.614855in}}%
\pgfpathlineto{\pgfqpoint{1.696987in}{1.614889in}}%
\pgfpathlineto{\pgfqpoint{1.698061in}{1.615805in}}%
\pgfpathlineto{\pgfqpoint{1.699135in}{1.612445in}}%
\pgfpathlineto{\pgfqpoint{1.702356in}{1.611189in}}%
\pgfpathlineto{\pgfqpoint{1.703430in}{1.612207in}}%
\pgfpathlineto{\pgfqpoint{1.704504in}{1.610069in}}%
\pgfpathlineto{\pgfqpoint{1.705578in}{1.605114in}}%
\pgfpathlineto{\pgfqpoint{1.706652in}{1.610273in}}%
\pgfpathlineto{\pgfqpoint{1.709873in}{1.613327in}}%
\pgfpathlineto{\pgfqpoint{1.710947in}{1.609153in}}%
\pgfpathlineto{\pgfqpoint{1.712021in}{1.609153in}}%
\pgfpathlineto{\pgfqpoint{1.713095in}{1.612106in}}%
\pgfpathlineto{\pgfqpoint{1.714169in}{1.619369in}}%
\pgfpathlineto{\pgfqpoint{1.718464in}{1.616111in}}%
\pgfpathlineto{\pgfqpoint{1.719538in}{1.618113in}}%
\pgfpathlineto{\pgfqpoint{1.720612in}{1.623543in}}%
\pgfpathlineto{\pgfqpoint{1.721685in}{1.621541in}}%
\pgfpathlineto{\pgfqpoint{1.724907in}{1.621609in}}%
\pgfpathlineto{\pgfqpoint{1.725981in}{1.623306in}}%
\pgfpathlineto{\pgfqpoint{1.727055in}{1.622695in}}%
\pgfpathlineto{\pgfqpoint{1.728128in}{1.623475in}}%
\pgfpathlineto{\pgfqpoint{1.733498in}{1.616586in}}%
\pgfpathlineto{\pgfqpoint{1.734572in}{1.618928in}}%
\pgfpathlineto{\pgfqpoint{1.735645in}{1.619097in}}%
\pgfpathlineto{\pgfqpoint{1.736719in}{1.617502in}}%
\pgfpathlineto{\pgfqpoint{1.739941in}{1.616959in}}%
\pgfpathlineto{\pgfqpoint{1.741015in}{1.618011in}}%
\pgfpathlineto{\pgfqpoint{1.742088in}{1.621609in}}%
\pgfpathlineto{\pgfqpoint{1.743162in}{1.617400in}}%
\pgfpathlineto{\pgfqpoint{1.744236in}{1.616959in}}%
\pgfpathlineto{\pgfqpoint{1.747458in}{1.614447in}}%
\pgfpathlineto{\pgfqpoint{1.748531in}{1.614821in}}%
\pgfpathlineto{\pgfqpoint{1.750679in}{1.621032in}}%
\pgfpathlineto{\pgfqpoint{1.751753in}{1.618826in}}%
\pgfpathlineto{\pgfqpoint{1.754974in}{1.609662in}}%
\pgfpathlineto{\pgfqpoint{1.757122in}{1.611053in}}%
\pgfpathlineto{\pgfqpoint{1.758196in}{1.613599in}}%
\pgfpathlineto{\pgfqpoint{1.759270in}{1.610510in}}%
\pgfpathlineto{\pgfqpoint{1.763565in}{1.611563in}}%
\pgfpathlineto{\pgfqpoint{1.765713in}{1.605623in}}%
\pgfpathlineto{\pgfqpoint{1.766787in}{1.606302in}}%
\pgfpathlineto{\pgfqpoint{1.771082in}{1.598360in}}%
\pgfpathlineto{\pgfqpoint{1.772156in}{1.598021in}}%
\pgfpathlineto{\pgfqpoint{1.773230in}{1.594185in}}%
\pgfpathlineto{\pgfqpoint{1.777525in}{1.593574in}}%
\pgfpathlineto{\pgfqpoint{1.778599in}{1.595814in}}%
\pgfpathlineto{\pgfqpoint{1.779673in}{1.591436in}}%
\pgfpathlineto{\pgfqpoint{1.780747in}{1.592217in}}%
\pgfpathlineto{\pgfqpoint{1.781820in}{1.596086in}}%
\pgfpathlineto{\pgfqpoint{1.785042in}{1.600362in}}%
\pgfpathlineto{\pgfqpoint{1.786116in}{1.600159in}}%
\pgfpathlineto{\pgfqpoint{1.787190in}{1.599344in}}%
\pgfpathlineto{\pgfqpoint{1.788263in}{1.599378in}}%
\pgfpathlineto{\pgfqpoint{1.789337in}{1.597783in}}%
\pgfpathlineto{\pgfqpoint{1.792559in}{1.597002in}}%
\pgfpathlineto{\pgfqpoint{1.793633in}{1.573176in}}%
\pgfpathlineto{\pgfqpoint{1.794706in}{1.569647in}}%
\pgfpathlineto{\pgfqpoint{1.795780in}{1.568357in}}%
\pgfpathlineto{\pgfqpoint{1.796854in}{1.562757in}}%
\pgfpathlineto{\pgfqpoint{1.800076in}{1.561433in}}%
\pgfpathlineto{\pgfqpoint{1.801149in}{1.561739in}}%
\pgfpathlineto{\pgfqpoint{1.802223in}{1.562927in}}%
\pgfpathlineto{\pgfqpoint{1.803297in}{1.567101in}}%
\pgfpathlineto{\pgfqpoint{1.804371in}{1.565845in}}%
\pgfpathlineto{\pgfqpoint{1.809740in}{1.561433in}}%
\pgfpathlineto{\pgfqpoint{1.810814in}{1.561739in}}%
\pgfpathlineto{\pgfqpoint{1.811888in}{1.559634in}}%
\pgfpathlineto{\pgfqpoint{1.815109in}{1.563504in}}%
\pgfpathlineto{\pgfqpoint{1.816183in}{1.560619in}}%
\pgfpathlineto{\pgfqpoint{1.817257in}{1.562825in}}%
\pgfpathlineto{\pgfqpoint{1.818331in}{1.561874in}}%
\pgfpathlineto{\pgfqpoint{1.819405in}{1.562859in}}%
\pgfpathlineto{\pgfqpoint{1.823700in}{1.565065in}}%
\pgfpathlineto{\pgfqpoint{1.824774in}{1.561365in}}%
\pgfpathlineto{\pgfqpoint{1.826922in}{1.545312in}}%
\pgfpathlineto{\pgfqpoint{1.830143in}{1.538727in}}%
\pgfpathlineto{\pgfqpoint{1.831217in}{1.532347in}}%
\pgfpathlineto{\pgfqpoint{1.833365in}{1.546126in}}%
\pgfpathlineto{\pgfqpoint{1.834438in}{1.546025in}}%
\pgfpathlineto{\pgfqpoint{1.837660in}{1.541001in}}%
\pgfpathlineto{\pgfqpoint{1.838734in}{1.535164in}}%
\pgfpathlineto{\pgfqpoint{1.839808in}{1.539847in}}%
\pgfpathlineto{\pgfqpoint{1.840882in}{1.541918in}}%
\pgfpathlineto{\pgfqpoint{1.841955in}{1.538150in}}%
\pgfpathlineto{\pgfqpoint{1.846251in}{1.544735in}}%
\pgfpathlineto{\pgfqpoint{1.848398in}{1.540357in}}%
\pgfpathlineto{\pgfqpoint{1.849472in}{1.543241in}}%
\pgfpathlineto{\pgfqpoint{1.852694in}{1.541680in}}%
\pgfpathlineto{\pgfqpoint{1.854841in}{1.548197in}}%
\pgfpathlineto{\pgfqpoint{1.855915in}{1.546262in}}%
\pgfpathlineto{\pgfqpoint{1.856989in}{1.539338in}}%
\pgfpathlineto{\pgfqpoint{1.860211in}{1.540832in}}%
\pgfpathlineto{\pgfqpoint{1.861284in}{1.530480in}}%
\pgfpathlineto{\pgfqpoint{1.862358in}{1.526679in}}%
\pgfpathlineto{\pgfqpoint{1.863432in}{1.526238in}}%
\pgfpathlineto{\pgfqpoint{1.864506in}{1.527697in}}%
\pgfpathlineto{\pgfqpoint{1.867727in}{1.526271in}}%
\pgfpathlineto{\pgfqpoint{1.869875in}{1.532924in}}%
\pgfpathlineto{\pgfqpoint{1.870949in}{1.530989in}}%
\pgfpathlineto{\pgfqpoint{1.872023in}{1.535333in}}%
\pgfpathlineto{\pgfqpoint{1.875244in}{1.543004in}}%
\pgfpathlineto{\pgfqpoint{1.876318in}{1.543920in}}%
\pgfpathlineto{\pgfqpoint{1.879540in}{1.552609in}}%
\pgfpathlineto{\pgfqpoint{1.882761in}{1.552779in}}%
\pgfpathlineto{\pgfqpoint{1.883835in}{1.549249in}}%
\pgfpathlineto{\pgfqpoint{1.884909in}{1.542732in}}%
\pgfpathlineto{\pgfqpoint{1.885983in}{1.545855in}}%
\pgfpathlineto{\pgfqpoint{1.887057in}{1.545278in}}%
\pgfpathlineto{\pgfqpoint{1.890278in}{1.542359in}}%
\pgfpathlineto{\pgfqpoint{1.892426in}{1.560619in}}%
\pgfpathlineto{\pgfqpoint{1.893500in}{1.566219in}}%
\pgfpathlineto{\pgfqpoint{1.894573in}{1.568798in}}%
\pgfpathlineto{\pgfqpoint{1.897795in}{1.567576in}}%
\pgfpathlineto{\pgfqpoint{1.898869in}{1.563436in}}%
\pgfpathlineto{\pgfqpoint{1.899943in}{1.564793in}}%
\pgfpathlineto{\pgfqpoint{1.901016in}{1.563945in}}%
\pgfpathlineto{\pgfqpoint{1.902090in}{1.561976in}}%
\pgfpathlineto{\pgfqpoint{1.905312in}{1.564997in}}%
\pgfpathlineto{\pgfqpoint{1.907460in}{1.567916in}}%
\pgfpathlineto{\pgfqpoint{1.908533in}{1.569341in}}%
\pgfpathlineto{\pgfqpoint{1.909607in}{1.569341in}}%
\pgfpathlineto{\pgfqpoint{1.914976in}{1.563470in}}%
\pgfpathlineto{\pgfqpoint{1.916050in}{1.566456in}}%
\pgfpathlineto{\pgfqpoint{1.917124in}{1.557937in}}%
\pgfpathlineto{\pgfqpoint{1.920346in}{1.562146in}}%
\pgfpathlineto{\pgfqpoint{1.921419in}{1.561264in}}%
\pgfpathlineto{\pgfqpoint{1.922493in}{1.561773in}}%
\pgfpathlineto{\pgfqpoint{1.923567in}{1.563605in}}%
\pgfpathlineto{\pgfqpoint{1.927862in}{1.562961in}}%
\pgfpathlineto{\pgfqpoint{1.928936in}{1.560687in}}%
\pgfpathlineto{\pgfqpoint{1.930010in}{1.560415in}}%
\pgfpathlineto{\pgfqpoint{1.935379in}{1.556614in}}%
\pgfpathlineto{\pgfqpoint{1.936453in}{1.558616in}}%
\pgfpathlineto{\pgfqpoint{1.937527in}{1.554136in}}%
\pgfpathlineto{\pgfqpoint{1.938601in}{1.552439in}}%
\pgfpathlineto{\pgfqpoint{1.939675in}{1.555630in}}%
\pgfpathlineto{\pgfqpoint{1.942896in}{1.556173in}}%
\pgfpathlineto{\pgfqpoint{1.943970in}{1.550980in}}%
\pgfpathlineto{\pgfqpoint{1.945044in}{1.550742in}}%
\pgfpathlineto{\pgfqpoint{1.946118in}{1.549860in}}%
\pgfpathlineto{\pgfqpoint{1.947192in}{1.548061in}}%
\pgfpathlineto{\pgfqpoint{1.950413in}{1.547246in}}%
\pgfpathlineto{\pgfqpoint{1.951487in}{1.547959in}}%
\pgfpathlineto{\pgfqpoint{1.952561in}{1.553763in}}%
\pgfpathlineto{\pgfqpoint{1.954708in}{1.545108in}}%
\pgfpathlineto{\pgfqpoint{1.957930in}{1.549045in}}%
\pgfpathlineto{\pgfqpoint{1.960078in}{1.557394in}}%
\pgfpathlineto{\pgfqpoint{1.961151in}{1.557394in}}%
\pgfpathlineto{\pgfqpoint{1.965447in}{1.556750in}}%
\pgfpathlineto{\pgfqpoint{1.966521in}{1.560415in}}%
\pgfpathlineto{\pgfqpoint{1.967594in}{1.559261in}}%
\pgfpathlineto{\pgfqpoint{1.968668in}{1.556682in}}%
\pgfpathlineto{\pgfqpoint{1.972964in}{1.555120in}}%
\pgfpathlineto{\pgfqpoint{1.974038in}{1.555596in}}%
\pgfpathlineto{\pgfqpoint{1.975111in}{1.547518in}}%
\pgfpathlineto{\pgfqpoint{1.977259in}{1.539067in}}%
\pgfpathlineto{\pgfqpoint{1.981554in}{1.539372in}}%
\pgfpathlineto{\pgfqpoint{1.982628in}{1.534213in}}%
\pgfpathlineto{\pgfqpoint{1.983702in}{1.534756in}}%
\pgfpathlineto{\pgfqpoint{1.984776in}{1.524269in}}%
\pgfpathlineto{\pgfqpoint{1.989071in}{1.523047in}}%
\pgfpathlineto{\pgfqpoint{1.990145in}{1.521791in}}%
\pgfpathlineto{\pgfqpoint{1.992293in}{1.526509in}}%
\pgfpathlineto{\pgfqpoint{1.995514in}{1.521995in}}%
\pgfpathlineto{\pgfqpoint{1.996588in}{1.524337in}}%
\pgfpathlineto{\pgfqpoint{1.997662in}{1.524812in}}%
\pgfpathlineto{\pgfqpoint{1.998736in}{1.526815in}}%
\pgfpathlineto{\pgfqpoint{1.999810in}{1.530650in}}%
\pgfpathlineto{\pgfqpoint{2.003031in}{1.530242in}}%
\pgfpathlineto{\pgfqpoint{2.004105in}{1.523624in}}%
\pgfpathlineto{\pgfqpoint{2.005179in}{1.525287in}}%
\pgfpathlineto{\pgfqpoint{2.006253in}{1.532007in}}%
\pgfpathlineto{\pgfqpoint{2.010548in}{1.527833in}}%
\pgfpathlineto{\pgfqpoint{2.011622in}{1.529258in}}%
\pgfpathlineto{\pgfqpoint{2.012696in}{1.528410in}}%
\pgfpathlineto{\pgfqpoint{2.013770in}{1.521248in}}%
\pgfpathlineto{\pgfqpoint{2.014843in}{1.525253in}}%
\pgfpathlineto{\pgfqpoint{2.019139in}{1.526916in}}%
\pgfpathlineto{\pgfqpoint{2.020213in}{1.534044in}}%
\pgfpathlineto{\pgfqpoint{2.021286in}{1.534790in}}%
\pgfpathlineto{\pgfqpoint{2.022360in}{1.534383in}}%
\pgfpathlineto{\pgfqpoint{2.025582in}{1.547314in}}%
\pgfpathlineto{\pgfqpoint{2.026656in}{1.544905in}}%
\pgfpathlineto{\pgfqpoint{2.027729in}{1.551217in}}%
\pgfpathlineto{\pgfqpoint{2.028803in}{1.565167in}}%
\pgfpathlineto{\pgfqpoint{2.033099in}{1.560619in}}%
\pgfpathlineto{\pgfqpoint{2.034172in}{1.555697in}}%
\pgfpathlineto{\pgfqpoint{2.036320in}{1.559024in}}%
\pgfpathlineto{\pgfqpoint{2.037394in}{1.561807in}}%
\pgfpathlineto{\pgfqpoint{2.041689in}{1.561501in}}%
\pgfpathlineto{\pgfqpoint{2.043837in}{1.559227in}}%
\pgfpathlineto{\pgfqpoint{2.044911in}{1.561026in}}%
\pgfpathlineto{\pgfqpoint{2.048132in}{1.561264in}}%
\pgfpathlineto{\pgfqpoint{2.049206in}{1.559601in}}%
\pgfpathlineto{\pgfqpoint{2.051354in}{1.567576in}}%
\pgfpathlineto{\pgfqpoint{2.052428in}{1.568221in}}%
\pgfpathlineto{\pgfqpoint{2.055649in}{1.568595in}}%
\pgfpathlineto{\pgfqpoint{2.056723in}{1.566999in}}%
\pgfpathlineto{\pgfqpoint{2.057797in}{1.568391in}}%
\pgfpathlineto{\pgfqpoint{2.063166in}{1.567780in}}%
\pgfpathlineto{\pgfqpoint{2.064240in}{1.571581in}}%
\pgfpathlineto{\pgfqpoint{2.065314in}{1.571989in}}%
\pgfpathlineto{\pgfqpoint{2.067461in}{1.571106in}}%
\pgfpathlineto{\pgfqpoint{2.070683in}{1.572023in}}%
\pgfpathlineto{\pgfqpoint{2.071757in}{1.570767in}}%
\pgfpathlineto{\pgfqpoint{2.074978in}{1.575315in}}%
\pgfpathlineto{\pgfqpoint{2.078200in}{1.577894in}}%
\pgfpathlineto{\pgfqpoint{2.079274in}{1.580813in}}%
\pgfpathlineto{\pgfqpoint{2.080348in}{1.585599in}}%
\pgfpathlineto{\pgfqpoint{2.081421in}{1.585836in}}%
\pgfpathlineto{\pgfqpoint{2.082495in}{1.585497in}}%
\pgfpathlineto{\pgfqpoint{2.085717in}{1.587228in}}%
\pgfpathlineto{\pgfqpoint{2.086791in}{1.586820in}}%
\pgfpathlineto{\pgfqpoint{2.087864in}{1.588483in}}%
\pgfpathlineto{\pgfqpoint{2.088938in}{1.588178in}}%
\pgfpathlineto{\pgfqpoint{2.090012in}{1.589026in}}%
\pgfpathlineto{\pgfqpoint{2.093234in}{1.587465in}}%
\pgfpathlineto{\pgfqpoint{2.094307in}{1.586209in}}%
\pgfpathlineto{\pgfqpoint{2.095381in}{1.589603in}}%
\pgfpathlineto{\pgfqpoint{2.096455in}{1.584411in}}%
\pgfpathlineto{\pgfqpoint{2.097529in}{1.584852in}}%
\pgfpathlineto{\pgfqpoint{2.100750in}{1.584852in}}%
\pgfpathlineto{\pgfqpoint{2.102898in}{1.573278in}}%
\pgfpathlineto{\pgfqpoint{2.103972in}{1.572090in}}%
\pgfpathlineto{\pgfqpoint{2.105046in}{1.574704in}}%
\pgfpathlineto{\pgfqpoint{2.108267in}{1.571479in}}%
\pgfpathlineto{\pgfqpoint{2.109341in}{1.577962in}}%
\pgfpathlineto{\pgfqpoint{2.110415in}{1.576197in}}%
\pgfpathlineto{\pgfqpoint{2.111489in}{1.575756in}}%
\pgfpathlineto{\pgfqpoint{2.112563in}{1.572023in}}%
\pgfpathlineto{\pgfqpoint{2.115784in}{1.576876in}}%
\pgfpathlineto{\pgfqpoint{2.116858in}{1.571378in}}%
\pgfpathlineto{\pgfqpoint{2.117932in}{1.571038in}}%
\pgfpathlineto{\pgfqpoint{2.119006in}{1.568561in}}%
\pgfpathlineto{\pgfqpoint{2.120080in}{1.570427in}}%
\pgfpathlineto{\pgfqpoint{2.123301in}{1.569816in}}%
\pgfpathlineto{\pgfqpoint{2.124375in}{1.572973in}}%
\pgfpathlineto{\pgfqpoint{2.125449in}{1.574263in}}%
\pgfpathlineto{\pgfqpoint{2.126523in}{1.574568in}}%
\pgfpathlineto{\pgfqpoint{2.127596in}{1.575654in}}%
\pgfpathlineto{\pgfqpoint{2.131892in}{1.575077in}}%
\pgfpathlineto{\pgfqpoint{2.132966in}{1.574364in}}%
\pgfpathlineto{\pgfqpoint{2.134039in}{1.575756in}}%
\pgfpathlineto{\pgfqpoint{2.135113in}{1.574636in}}%
\pgfpathlineto{\pgfqpoint{2.141556in}{1.580575in}}%
\pgfpathlineto{\pgfqpoint{2.145852in}{1.577080in}}%
\pgfpathlineto{\pgfqpoint{2.146926in}{1.577012in}}%
\pgfpathlineto{\pgfqpoint{2.147999in}{1.574975in}}%
\pgfpathlineto{\pgfqpoint{2.149073in}{1.576808in}}%
\pgfpathlineto{\pgfqpoint{2.150147in}{1.577046in}}%
\pgfpathlineto{\pgfqpoint{2.153369in}{1.578607in}}%
\pgfpathlineto{\pgfqpoint{2.155516in}{1.577521in}}%
\pgfpathlineto{\pgfqpoint{2.156590in}{1.580609in}}%
\pgfpathlineto{\pgfqpoint{2.157664in}{1.569782in}}%
\pgfpathlineto{\pgfqpoint{2.160885in}{1.564488in}}%
\pgfpathlineto{\pgfqpoint{2.164107in}{1.581288in}}%
\pgfpathlineto{\pgfqpoint{2.165181in}{1.581865in}}%
\pgfpathlineto{\pgfqpoint{2.169476in}{1.575213in}}%
\pgfpathlineto{\pgfqpoint{2.171624in}{1.579455in}}%
\pgfpathlineto{\pgfqpoint{2.172698in}{1.584784in}}%
\pgfpathlineto{\pgfqpoint{2.175919in}{1.585734in}}%
\pgfpathlineto{\pgfqpoint{2.178067in}{1.589298in}}%
\pgfpathlineto{\pgfqpoint{2.179141in}{1.589434in}}%
\pgfpathlineto{\pgfqpoint{2.180214in}{1.590588in}}%
\pgfpathlineto{\pgfqpoint{2.184510in}{1.590961in}}%
\pgfpathlineto{\pgfqpoint{2.185584in}{1.592285in}}%
\pgfpathlineto{\pgfqpoint{2.186658in}{1.591674in}}%
\pgfpathlineto{\pgfqpoint{2.187731in}{1.589400in}}%
\pgfpathlineto{\pgfqpoint{2.190953in}{1.587906in}}%
\pgfpathlineto{\pgfqpoint{2.192027in}{1.598088in}}%
\pgfpathlineto{\pgfqpoint{2.194174in}{1.597138in}}%
\pgfpathlineto{\pgfqpoint{2.195248in}{1.597342in}}%
\pgfpathlineto{\pgfqpoint{2.198470in}{1.595102in}}%
\pgfpathlineto{\pgfqpoint{2.199544in}{1.592997in}}%
\pgfpathlineto{\pgfqpoint{2.201691in}{1.593303in}}%
\pgfpathlineto{\pgfqpoint{2.202765in}{1.597647in}}%
\pgfpathlineto{\pgfqpoint{2.205987in}{1.597783in}}%
\pgfpathlineto{\pgfqpoint{2.207060in}{1.599514in}}%
\pgfpathlineto{\pgfqpoint{2.208134in}{1.598835in}}%
\pgfpathlineto{\pgfqpoint{2.209208in}{1.602161in}}%
\pgfpathlineto{\pgfqpoint{2.210282in}{1.601177in}}%
\pgfpathlineto{\pgfqpoint{2.213503in}{1.603756in}}%
\pgfpathlineto{\pgfqpoint{2.214577in}{1.602093in}}%
\pgfpathlineto{\pgfqpoint{2.216725in}{1.604741in}}%
\pgfpathlineto{\pgfqpoint{2.221020in}{1.602093in}}%
\pgfpathlineto{\pgfqpoint{2.222094in}{1.600532in}}%
\pgfpathlineto{\pgfqpoint{2.223168in}{1.600396in}}%
\pgfpathlineto{\pgfqpoint{2.225316in}{1.598326in}}%
\pgfpathlineto{\pgfqpoint{2.228537in}{1.600430in}}%
\pgfpathlineto{\pgfqpoint{2.230685in}{1.595543in}}%
\pgfpathlineto{\pgfqpoint{2.232833in}{1.597138in}}%
\pgfpathlineto{\pgfqpoint{2.239276in}{1.593405in}}%
\pgfpathlineto{\pgfqpoint{2.240349in}{1.583664in}}%
\pgfpathlineto{\pgfqpoint{2.243571in}{1.587940in}}%
\pgfpathlineto{\pgfqpoint{2.244645in}{1.582544in}}%
\pgfpathlineto{\pgfqpoint{2.245719in}{1.580440in}}%
\pgfpathlineto{\pgfqpoint{2.246792in}{1.583766in}}%
\pgfpathlineto{\pgfqpoint{2.247866in}{1.575484in}}%
\pgfpathlineto{\pgfqpoint{2.251088in}{1.576570in}}%
\pgfpathlineto{\pgfqpoint{2.252162in}{1.575926in}}%
\pgfpathlineto{\pgfqpoint{2.254309in}{1.584682in}}%
\pgfpathlineto{\pgfqpoint{2.255383in}{1.583291in}}%
\pgfpathlineto{\pgfqpoint{2.258605in}{1.582238in}}%
\pgfpathlineto{\pgfqpoint{2.260752in}{1.582612in}}%
\pgfpathlineto{\pgfqpoint{2.261826in}{1.578539in}}%
\pgfpathlineto{\pgfqpoint{2.262900in}{1.580236in}}%
\pgfpathlineto{\pgfqpoint{2.266122in}{1.582883in}}%
\pgfpathlineto{\pgfqpoint{2.267195in}{1.579727in}}%
\pgfpathlineto{\pgfqpoint{2.268269in}{1.582306in}}%
\pgfpathlineto{\pgfqpoint{2.269343in}{1.581763in}}%
\pgfpathlineto{\pgfqpoint{2.270417in}{1.577012in}}%
\pgfpathlineto{\pgfqpoint{2.273638in}{1.575111in}}%
\pgfpathlineto{\pgfqpoint{2.274712in}{1.571276in}}%
\pgfpathlineto{\pgfqpoint{2.275786in}{1.571785in}}%
\pgfpathlineto{\pgfqpoint{2.276860in}{1.574704in}}%
\pgfpathlineto{\pgfqpoint{2.277934in}{1.575688in}}%
\pgfpathlineto{\pgfqpoint{2.281155in}{1.574263in}}%
\pgfpathlineto{\pgfqpoint{2.282229in}{1.575145in}}%
\pgfpathlineto{\pgfqpoint{2.283303in}{1.574432in}}%
\pgfpathlineto{\pgfqpoint{2.285451in}{1.570970in}}%
\pgfpathlineto{\pgfqpoint{2.288672in}{1.573652in}}%
\pgfpathlineto{\pgfqpoint{2.289746in}{1.579489in}}%
\pgfpathlineto{\pgfqpoint{2.290820in}{1.578369in}}%
\pgfpathlineto{\pgfqpoint{2.291894in}{1.575417in}}%
\pgfpathlineto{\pgfqpoint{2.292968in}{1.581017in}}%
\pgfpathlineto{\pgfqpoint{2.296189in}{1.582137in}}%
\pgfpathlineto{\pgfqpoint{2.297263in}{1.581560in}}%
\pgfpathlineto{\pgfqpoint{2.299411in}{1.578505in}}%
\pgfpathlineto{\pgfqpoint{2.300484in}{1.579387in}}%
\pgfpathlineto{\pgfqpoint{2.304780in}{1.585870in}}%
\pgfpathlineto{\pgfqpoint{2.305854in}{1.590418in}}%
\pgfpathlineto{\pgfqpoint{2.306927in}{1.601822in}}%
\pgfpathlineto{\pgfqpoint{2.308001in}{1.603247in}}%
\pgfpathlineto{\pgfqpoint{2.312297in}{1.599514in}}%
\pgfpathlineto{\pgfqpoint{2.313370in}{1.599310in}}%
\pgfpathlineto{\pgfqpoint{2.315518in}{1.597851in}}%
\pgfpathlineto{\pgfqpoint{2.319814in}{1.599242in}}%
\pgfpathlineto{\pgfqpoint{2.320887in}{1.602976in}}%
\pgfpathlineto{\pgfqpoint{2.323035in}{1.605114in}}%
\pgfpathlineto{\pgfqpoint{2.326257in}{1.604062in}}%
\pgfpathlineto{\pgfqpoint{2.327330in}{1.605318in}}%
\pgfpathlineto{\pgfqpoint{2.328404in}{1.601754in}}%
\pgfpathlineto{\pgfqpoint{2.329478in}{1.600973in}}%
\pgfpathlineto{\pgfqpoint{2.330552in}{1.603315in}}%
\pgfpathlineto{\pgfqpoint{2.333773in}{1.600770in}}%
\pgfpathlineto{\pgfqpoint{2.334847in}{1.600804in}}%
\pgfpathlineto{\pgfqpoint{2.335921in}{1.607999in}}%
\pgfpathlineto{\pgfqpoint{2.336995in}{1.603994in}}%
\pgfpathlineto{\pgfqpoint{2.338069in}{1.608338in}}%
\pgfpathlineto{\pgfqpoint{2.341290in}{1.610239in}}%
\pgfpathlineto{\pgfqpoint{2.342364in}{1.610001in}}%
\pgfpathlineto{\pgfqpoint{2.344512in}{1.603044in}}%
\pgfpathlineto{\pgfqpoint{2.345586in}{1.604299in}}%
\pgfpathlineto{\pgfqpoint{2.348807in}{1.611630in}}%
\pgfpathlineto{\pgfqpoint{2.352029in}{1.610476in}}%
\pgfpathlineto{\pgfqpoint{2.353102in}{1.611087in}}%
\pgfpathlineto{\pgfqpoint{2.357398in}{1.612140in}}%
\pgfpathlineto{\pgfqpoint{2.358472in}{1.609594in}}%
\pgfpathlineto{\pgfqpoint{2.359546in}{1.610748in}}%
\pgfpathlineto{\pgfqpoint{2.360619in}{1.607795in}}%
\pgfpathlineto{\pgfqpoint{2.364915in}{1.611664in}}%
\pgfpathlineto{\pgfqpoint{2.365989in}{1.611868in}}%
\pgfpathlineto{\pgfqpoint{2.367062in}{1.613294in}}%
\pgfpathlineto{\pgfqpoint{2.368136in}{1.617129in}}%
\pgfpathlineto{\pgfqpoint{2.373505in}{1.612173in}}%
\pgfpathlineto{\pgfqpoint{2.374579in}{1.611630in}}%
\pgfpathlineto{\pgfqpoint{2.375653in}{1.609696in}}%
\pgfpathlineto{\pgfqpoint{2.379948in}{1.608915in}}%
\pgfpathlineto{\pgfqpoint{2.382096in}{1.611325in}}%
\pgfpathlineto{\pgfqpoint{2.383170in}{1.611529in}}%
\pgfpathlineto{\pgfqpoint{2.386391in}{1.610103in}}%
\pgfpathlineto{\pgfqpoint{2.387465in}{1.614142in}}%
\pgfpathlineto{\pgfqpoint{2.390687in}{1.608033in}}%
\pgfpathlineto{\pgfqpoint{2.393908in}{1.606370in}}%
\pgfpathlineto{\pgfqpoint{2.394982in}{1.607965in}}%
\pgfpathlineto{\pgfqpoint{2.396056in}{1.603213in}}%
\pgfpathlineto{\pgfqpoint{2.397130in}{1.603790in}}%
\pgfpathlineto{\pgfqpoint{2.398204in}{1.607897in}}%
\pgfpathlineto{\pgfqpoint{2.402499in}{1.612988in}}%
\pgfpathlineto{\pgfqpoint{2.403573in}{1.610273in}}%
\pgfpathlineto{\pgfqpoint{2.404647in}{1.609289in}}%
\pgfpathlineto{\pgfqpoint{2.405721in}{1.612343in}}%
\pgfpathlineto{\pgfqpoint{2.408942in}{1.615432in}}%
\pgfpathlineto{\pgfqpoint{2.410016in}{1.614006in}}%
\pgfpathlineto{\pgfqpoint{2.411090in}{1.617197in}}%
\pgfpathlineto{\pgfqpoint{2.413237in}{1.617977in}}%
\pgfpathlineto{\pgfqpoint{2.418607in}{1.619606in}}%
\pgfpathlineto{\pgfqpoint{2.419680in}{1.617774in}}%
\pgfpathlineto{\pgfqpoint{2.420754in}{1.618961in}}%
\pgfpathlineto{\pgfqpoint{2.423976in}{1.620081in}}%
\pgfpathlineto{\pgfqpoint{2.425050in}{1.619267in}}%
\pgfpathlineto{\pgfqpoint{2.426124in}{1.622899in}}%
\pgfpathlineto{\pgfqpoint{2.427197in}{1.619708in}}%
\pgfpathlineto{\pgfqpoint{2.428271in}{1.618588in}}%
\pgfpathlineto{\pgfqpoint{2.431493in}{1.616416in}}%
\pgfpathlineto{\pgfqpoint{2.432567in}{1.618384in}}%
\pgfpathlineto{\pgfqpoint{2.433640in}{1.616687in}}%
\pgfpathlineto{\pgfqpoint{2.435788in}{1.617943in}}%
\pgfpathlineto{\pgfqpoint{2.439010in}{1.618486in}}%
\pgfpathlineto{\pgfqpoint{2.440083in}{1.616959in}}%
\pgfpathlineto{\pgfqpoint{2.441157in}{1.620964in}}%
\pgfpathlineto{\pgfqpoint{2.442231in}{1.618588in}}%
\pgfpathlineto{\pgfqpoint{2.443305in}{1.622152in}}%
\pgfpathlineto{\pgfqpoint{2.446526in}{1.622355in}}%
\pgfpathlineto{\pgfqpoint{2.447600in}{1.618011in}}%
\pgfpathlineto{\pgfqpoint{2.449748in}{1.616993in}}%
\pgfpathlineto{\pgfqpoint{2.454043in}{1.616959in}}%
\pgfpathlineto{\pgfqpoint{2.455117in}{1.619878in}}%
\pgfpathlineto{\pgfqpoint{2.456191in}{1.617706in}}%
\pgfpathlineto{\pgfqpoint{2.457265in}{1.618928in}}%
\pgfpathlineto{\pgfqpoint{2.458339in}{1.618181in}}%
\pgfpathlineto{\pgfqpoint{2.461560in}{1.617264in}}%
\pgfpathlineto{\pgfqpoint{2.462634in}{1.620658in}}%
\pgfpathlineto{\pgfqpoint{2.463708in}{1.619097in}}%
\pgfpathlineto{\pgfqpoint{2.464782in}{1.618826in}}%
\pgfpathlineto{\pgfqpoint{2.465856in}{1.620658in}}%
\pgfpathlineto{\pgfqpoint{2.469077in}{1.620319in}}%
\pgfpathlineto{\pgfqpoint{2.470151in}{1.621168in}}%
\pgfpathlineto{\pgfqpoint{2.471225in}{1.618283in}}%
\pgfpathlineto{\pgfqpoint{2.472299in}{1.617774in}}%
\pgfpathlineto{\pgfqpoint{2.477668in}{1.622491in}}%
\pgfpathlineto{\pgfqpoint{2.478742in}{1.620455in}}%
\pgfpathlineto{\pgfqpoint{2.480889in}{1.627073in}}%
\pgfpathlineto{\pgfqpoint{2.485185in}{1.633114in}}%
\pgfpathlineto{\pgfqpoint{2.486258in}{1.637391in}}%
\pgfpathlineto{\pgfqpoint{2.487332in}{1.639224in}}%
\pgfpathlineto{\pgfqpoint{2.488406in}{1.639902in}}%
\pgfpathlineto{\pgfqpoint{2.491628in}{1.639699in}}%
\pgfpathlineto{\pgfqpoint{2.492702in}{1.640955in}}%
\pgfpathlineto{\pgfqpoint{2.494849in}{1.646351in}}%
\pgfpathlineto{\pgfqpoint{2.495923in}{1.647471in}}%
\pgfpathlineto{\pgfqpoint{2.499145in}{1.646623in}}%
\pgfpathlineto{\pgfqpoint{2.500218in}{1.647844in}}%
\pgfpathlineto{\pgfqpoint{2.501292in}{1.646419in}}%
\pgfpathlineto{\pgfqpoint{2.502366in}{1.647267in}}%
\pgfpathlineto{\pgfqpoint{2.503440in}{1.645808in}}%
\pgfpathlineto{\pgfqpoint{2.506661in}{1.646079in}}%
\pgfpathlineto{\pgfqpoint{2.507735in}{1.647641in}}%
\pgfpathlineto{\pgfqpoint{2.508809in}{1.643907in}}%
\pgfpathlineto{\pgfqpoint{2.509883in}{1.643229in}}%
\pgfpathlineto{\pgfqpoint{2.510957in}{1.648998in}}%
\pgfpathlineto{\pgfqpoint{2.514178in}{1.650594in}}%
\pgfpathlineto{\pgfqpoint{2.515252in}{1.651917in}}%
\pgfpathlineto{\pgfqpoint{2.517400in}{1.652494in}}%
\pgfpathlineto{\pgfqpoint{2.518474in}{1.651204in}}%
\pgfpathlineto{\pgfqpoint{2.523843in}{1.649372in}}%
\pgfpathlineto{\pgfqpoint{2.525991in}{1.652087in}}%
\pgfpathlineto{\pgfqpoint{2.529212in}{1.648421in}}%
\pgfpathlineto{\pgfqpoint{2.530286in}{1.645638in}}%
\pgfpathlineto{\pgfqpoint{2.531360in}{1.644552in}}%
\pgfpathlineto{\pgfqpoint{2.532434in}{1.645061in}}%
\pgfpathlineto{\pgfqpoint{2.533507in}{1.647301in}}%
\pgfpathlineto{\pgfqpoint{2.536729in}{1.644858in}}%
\pgfpathlineto{\pgfqpoint{2.537803in}{1.645435in}}%
\pgfpathlineto{\pgfqpoint{2.538877in}{1.645299in}}%
\pgfpathlineto{\pgfqpoint{2.539950in}{1.647675in}}%
\pgfpathlineto{\pgfqpoint{2.541024in}{1.646623in}}%
\pgfpathlineto{\pgfqpoint{2.544246in}{1.651204in}}%
\pgfpathlineto{\pgfqpoint{2.545320in}{1.650118in}}%
\pgfpathlineto{\pgfqpoint{2.548541in}{1.652392in}}%
\pgfpathlineto{\pgfqpoint{2.552836in}{1.650390in}}%
\pgfpathlineto{\pgfqpoint{2.553910in}{1.653309in}}%
\pgfpathlineto{\pgfqpoint{2.554984in}{1.650560in}}%
\pgfpathlineto{\pgfqpoint{2.556058in}{1.652053in}}%
\pgfpathlineto{\pgfqpoint{2.559279in}{1.651951in}}%
\pgfpathlineto{\pgfqpoint{2.561427in}{1.654259in}}%
\pgfpathlineto{\pgfqpoint{2.562501in}{1.651340in}}%
\pgfpathlineto{\pgfqpoint{2.563575in}{1.653648in}}%
\pgfpathlineto{\pgfqpoint{2.566796in}{1.655108in}}%
\pgfpathlineto{\pgfqpoint{2.567870in}{1.656669in}}%
\pgfpathlineto{\pgfqpoint{2.568944in}{1.657212in}}%
\pgfpathlineto{\pgfqpoint{2.570018in}{1.655141in}}%
\pgfpathlineto{\pgfqpoint{2.571092in}{1.656228in}}%
\pgfpathlineto{\pgfqpoint{2.574313in}{1.655277in}}%
\pgfpathlineto{\pgfqpoint{2.575387in}{1.653716in}}%
\pgfpathlineto{\pgfqpoint{2.576461in}{1.655175in}}%
\pgfpathlineto{\pgfqpoint{2.577535in}{1.653139in}}%
\pgfpathlineto{\pgfqpoint{2.578609in}{1.656499in}}%
\pgfpathlineto{\pgfqpoint{2.581830in}{1.655345in}}%
\pgfpathlineto{\pgfqpoint{2.582904in}{1.646589in}}%
\pgfpathlineto{\pgfqpoint{2.585052in}{1.641022in}}%
\pgfpathlineto{\pgfqpoint{2.586125in}{1.641599in}}%
\pgfpathlineto{\pgfqpoint{2.589347in}{1.640649in}}%
\pgfpathlineto{\pgfqpoint{2.590421in}{1.641633in}}%
\pgfpathlineto{\pgfqpoint{2.592568in}{1.648795in}}%
\pgfpathlineto{\pgfqpoint{2.593642in}{1.650050in}}%
\pgfpathlineto{\pgfqpoint{2.596864in}{1.640479in}}%
\pgfpathlineto{\pgfqpoint{2.597938in}{1.639427in}}%
\pgfpathlineto{\pgfqpoint{2.599012in}{1.636237in}}%
\pgfpathlineto{\pgfqpoint{2.600085in}{1.634811in}}%
\pgfpathlineto{\pgfqpoint{2.604381in}{1.636067in}}%
\pgfpathlineto{\pgfqpoint{2.605455in}{1.632232in}}%
\pgfpathlineto{\pgfqpoint{2.606528in}{1.641124in}}%
\pgfpathlineto{\pgfqpoint{2.607602in}{1.634947in}}%
\pgfpathlineto{\pgfqpoint{2.608676in}{1.632945in}}%
\pgfpathlineto{\pgfqpoint{2.611898in}{1.632300in}}%
\pgfpathlineto{\pgfqpoint{2.612971in}{1.633624in}}%
\pgfpathlineto{\pgfqpoint{2.614045in}{1.637968in}}%
\pgfpathlineto{\pgfqpoint{2.615119in}{1.632368in}}%
\pgfpathlineto{\pgfqpoint{2.616193in}{1.631621in}}%
\pgfpathlineto{\pgfqpoint{2.619414in}{1.632470in}}%
\pgfpathlineto{\pgfqpoint{2.620488in}{1.643398in}}%
\pgfpathlineto{\pgfqpoint{2.621562in}{1.646317in}}%
\pgfpathlineto{\pgfqpoint{2.622636in}{1.646690in}}%
\pgfpathlineto{\pgfqpoint{2.629079in}{1.613904in}}%
\pgfpathlineto{\pgfqpoint{2.630153in}{1.615092in}}%
\pgfpathlineto{\pgfqpoint{2.631227in}{1.613701in}}%
\pgfpathlineto{\pgfqpoint{2.634448in}{1.614006in}}%
\pgfpathlineto{\pgfqpoint{2.636596in}{1.616077in}}%
\pgfpathlineto{\pgfqpoint{2.637670in}{1.625342in}}%
\pgfpathlineto{\pgfqpoint{2.641965in}{1.624154in}}%
\pgfpathlineto{\pgfqpoint{2.644113in}{1.628465in}}%
\pgfpathlineto{\pgfqpoint{2.646260in}{1.631417in}}%
\pgfpathlineto{\pgfqpoint{2.649482in}{1.629076in}}%
\pgfpathlineto{\pgfqpoint{2.650556in}{1.630467in}}%
\pgfpathlineto{\pgfqpoint{2.651630in}{1.638443in}}%
\pgfpathlineto{\pgfqpoint{2.652703in}{1.634099in}}%
\pgfpathlineto{\pgfqpoint{2.653777in}{1.634879in}}%
\pgfpathlineto{\pgfqpoint{2.656999in}{1.639767in}}%
\pgfpathlineto{\pgfqpoint{2.658073in}{1.640174in}}%
\pgfpathlineto{\pgfqpoint{2.659146in}{1.639902in}}%
\pgfpathlineto{\pgfqpoint{2.660220in}{1.641667in}}%
\pgfpathlineto{\pgfqpoint{2.661294in}{1.641871in}}%
\pgfpathlineto{\pgfqpoint{2.664516in}{1.643161in}}%
\pgfpathlineto{\pgfqpoint{2.666663in}{1.640310in}}%
\pgfpathlineto{\pgfqpoint{2.667737in}{1.643772in}}%
\pgfpathlineto{\pgfqpoint{2.668811in}{1.643432in}}%
\pgfpathlineto{\pgfqpoint{2.672033in}{1.644382in}}%
\pgfpathlineto{\pgfqpoint{2.673106in}{1.645536in}}%
\pgfpathlineto{\pgfqpoint{2.674180in}{1.644926in}}%
\pgfpathlineto{\pgfqpoint{2.675254in}{1.645944in}}%
\pgfpathlineto{\pgfqpoint{2.676328in}{1.650627in}}%
\pgfpathlineto{\pgfqpoint{2.679549in}{1.650492in}}%
\pgfpathlineto{\pgfqpoint{2.681697in}{1.644315in}}%
\pgfpathlineto{\pgfqpoint{2.682771in}{1.647369in}}%
\pgfpathlineto{\pgfqpoint{2.683845in}{1.644586in}}%
\pgfpathlineto{\pgfqpoint{2.687066in}{1.647064in}}%
\pgfpathlineto{\pgfqpoint{2.688140in}{1.646826in}}%
\pgfpathlineto{\pgfqpoint{2.689214in}{1.647980in}}%
\pgfpathlineto{\pgfqpoint{2.690288in}{1.652324in}}%
\pgfpathlineto{\pgfqpoint{2.691362in}{1.651069in}}%
\pgfpathlineto{\pgfqpoint{2.694583in}{1.648863in}}%
\pgfpathlineto{\pgfqpoint{2.695657in}{1.650017in}}%
\pgfpathlineto{\pgfqpoint{2.696731in}{1.648523in}}%
\pgfpathlineto{\pgfqpoint{2.697805in}{1.642312in}}%
\pgfpathlineto{\pgfqpoint{2.698879in}{1.641464in}}%
\pgfpathlineto{\pgfqpoint{2.702100in}{1.638104in}}%
\pgfpathlineto{\pgfqpoint{2.703174in}{1.643704in}}%
\pgfpathlineto{\pgfqpoint{2.704248in}{1.639733in}}%
\pgfpathlineto{\pgfqpoint{2.705322in}{1.643059in}}%
\pgfpathlineto{\pgfqpoint{2.706395in}{1.638613in}}%
\pgfpathlineto{\pgfqpoint{2.709617in}{1.638104in}}%
\pgfpathlineto{\pgfqpoint{2.710691in}{1.640276in}}%
\pgfpathlineto{\pgfqpoint{2.711765in}{1.639258in}}%
\pgfpathlineto{\pgfqpoint{2.717134in}{1.640547in}}%
\pgfpathlineto{\pgfqpoint{2.719281in}{1.643941in}}%
\pgfpathlineto{\pgfqpoint{2.720355in}{1.654666in}}%
\pgfpathlineto{\pgfqpoint{2.721429in}{1.650322in}}%
\pgfpathlineto{\pgfqpoint{2.724651in}{1.650050in}}%
\pgfpathlineto{\pgfqpoint{2.725724in}{1.650865in}}%
\pgfpathlineto{\pgfqpoint{2.728946in}{1.659112in}}%
\pgfpathlineto{\pgfqpoint{2.733241in}{1.661285in}}%
\pgfpathlineto{\pgfqpoint{2.734315in}{1.663966in}}%
\pgfpathlineto{\pgfqpoint{2.735389in}{1.662201in}}%
\pgfpathlineto{\pgfqpoint{2.736463in}{1.670075in}}%
\pgfpathlineto{\pgfqpoint{2.743980in}{1.673503in}}%
\pgfpathlineto{\pgfqpoint{2.748275in}{1.673231in}}%
\pgfpathlineto{\pgfqpoint{2.750423in}{1.676422in}}%
\pgfpathlineto{\pgfqpoint{2.751497in}{1.674623in}}%
\pgfpathlineto{\pgfqpoint{2.751497in}{1.674623in}}%
\pgfusepath{stroke}%
\end{pgfscope}%
\begin{pgfscope}%
\pgfpathrectangle{\pgfqpoint{0.285588in}{1.293759in}}{\pgfqpoint{2.583333in}{0.400885in}}%
\pgfusepath{clip}%
\pgfsetroundcap%
\pgfsetroundjoin%
\pgfsetlinewidth{1.505625pt}%
\definecolor{currentstroke}{rgb}{0.890196,0.466667,0.760784}%
\pgfsetstrokecolor{currentstroke}%
\pgfsetdash{}{0pt}%
\pgfpathmoveto{\pgfqpoint{0.403012in}{1.469626in}}%
\pgfpathlineto{\pgfqpoint{0.404086in}{1.468540in}}%
\pgfpathlineto{\pgfqpoint{0.406233in}{1.465315in}}%
\pgfpathlineto{\pgfqpoint{0.409455in}{1.465886in}}%
\pgfpathlineto{\pgfqpoint{0.410529in}{1.460477in}}%
\pgfpathlineto{\pgfqpoint{0.412676in}{1.457282in}}%
\pgfpathlineto{\pgfqpoint{0.413750in}{1.460278in}}%
\pgfpathlineto{\pgfqpoint{0.418045in}{1.457732in}}%
\pgfpathlineto{\pgfqpoint{0.419119in}{1.456273in}}%
\pgfpathlineto{\pgfqpoint{0.421267in}{1.458611in}}%
\pgfpathlineto{\pgfqpoint{0.424489in}{1.458173in}}%
\pgfpathlineto{\pgfqpoint{0.425562in}{1.455772in}}%
\pgfpathlineto{\pgfqpoint{0.428784in}{1.456195in}}%
\pgfpathlineto{\pgfqpoint{0.432005in}{1.456225in}}%
\pgfpathlineto{\pgfqpoint{0.433079in}{1.454329in}}%
\pgfpathlineto{\pgfqpoint{0.434153in}{1.449649in}}%
\pgfpathlineto{\pgfqpoint{0.435227in}{1.450135in}}%
\pgfpathlineto{\pgfqpoint{0.436301in}{1.447663in}}%
\pgfpathlineto{\pgfqpoint{0.440596in}{1.449464in}}%
\pgfpathlineto{\pgfqpoint{0.442744in}{1.441301in}}%
\pgfpathlineto{\pgfqpoint{0.443818in}{1.441929in}}%
\pgfpathlineto{\pgfqpoint{0.447039in}{1.438897in}}%
\pgfpathlineto{\pgfqpoint{0.448113in}{1.439407in}}%
\pgfpathlineto{\pgfqpoint{0.449187in}{1.442742in}}%
\pgfpathlineto{\pgfqpoint{0.451334in}{1.440152in}}%
\pgfpathlineto{\pgfqpoint{0.456704in}{1.439764in}}%
\pgfpathlineto{\pgfqpoint{0.457778in}{1.440769in}}%
\pgfpathlineto{\pgfqpoint{0.458851in}{1.439726in}}%
\pgfpathlineto{\pgfqpoint{0.463147in}{1.440811in}}%
\pgfpathlineto{\pgfqpoint{0.466368in}{1.438485in}}%
\pgfpathlineto{\pgfqpoint{0.469590in}{1.441153in}}%
\pgfpathlineto{\pgfqpoint{0.470664in}{1.445315in}}%
\pgfpathlineto{\pgfqpoint{0.472811in}{1.440447in}}%
\pgfpathlineto{\pgfqpoint{0.473885in}{1.440421in}}%
\pgfpathlineto{\pgfqpoint{0.477107in}{1.439459in}}%
\pgfpathlineto{\pgfqpoint{0.478180in}{1.433905in}}%
\pgfpathlineto{\pgfqpoint{0.480328in}{1.433253in}}%
\pgfpathlineto{\pgfqpoint{0.481402in}{1.436085in}}%
\pgfpathlineto{\pgfqpoint{0.484623in}{1.437871in}}%
\pgfpathlineto{\pgfqpoint{0.485697in}{1.440755in}}%
\pgfpathlineto{\pgfqpoint{0.488919in}{1.443977in}}%
\pgfpathlineto{\pgfqpoint{0.492140in}{1.440135in}}%
\pgfpathlineto{\pgfqpoint{0.493214in}{1.440992in}}%
\pgfpathlineto{\pgfqpoint{0.494288in}{1.444477in}}%
\pgfpathlineto{\pgfqpoint{0.496436in}{1.441265in}}%
\pgfpathlineto{\pgfqpoint{0.499657in}{1.441738in}}%
\pgfpathlineto{\pgfqpoint{0.501805in}{1.443517in}}%
\pgfpathlineto{\pgfqpoint{0.502879in}{1.444245in}}%
\pgfpathlineto{\pgfqpoint{0.507174in}{1.441121in}}%
\pgfpathlineto{\pgfqpoint{0.508248in}{1.437127in}}%
\pgfpathlineto{\pgfqpoint{0.509322in}{1.439762in}}%
\pgfpathlineto{\pgfqpoint{0.511469in}{1.433252in}}%
\pgfpathlineto{\pgfqpoint{0.514691in}{1.433435in}}%
\pgfpathlineto{\pgfqpoint{0.515765in}{1.430456in}}%
\pgfpathlineto{\pgfqpoint{0.516839in}{1.430557in}}%
\pgfpathlineto{\pgfqpoint{0.518986in}{1.427598in}}%
\pgfpathlineto{\pgfqpoint{0.522208in}{1.425006in}}%
\pgfpathlineto{\pgfqpoint{0.524355in}{1.425131in}}%
\pgfpathlineto{\pgfqpoint{0.526503in}{1.420626in}}%
\pgfpathlineto{\pgfqpoint{0.531872in}{1.421499in}}%
\pgfpathlineto{\pgfqpoint{0.532946in}{1.422787in}}%
\pgfpathlineto{\pgfqpoint{0.534020in}{1.419879in}}%
\pgfpathlineto{\pgfqpoint{0.538315in}{1.418546in}}%
\pgfpathlineto{\pgfqpoint{0.539389in}{1.414863in}}%
\pgfpathlineto{\pgfqpoint{0.540463in}{1.415444in}}%
\pgfpathlineto{\pgfqpoint{0.541537in}{1.415202in}}%
\pgfpathlineto{\pgfqpoint{0.544758in}{1.412706in}}%
\pgfpathlineto{\pgfqpoint{0.545832in}{1.413166in}}%
\pgfpathlineto{\pgfqpoint{0.549054in}{1.406405in}}%
\pgfpathlineto{\pgfqpoint{0.553349in}{1.409580in}}%
\pgfpathlineto{\pgfqpoint{0.554423in}{1.409822in}}%
\pgfpathlineto{\pgfqpoint{0.556571in}{1.407689in}}%
\pgfpathlineto{\pgfqpoint{0.560866in}{1.411930in}}%
\pgfpathlineto{\pgfqpoint{0.561940in}{1.409168in}}%
\pgfpathlineto{\pgfqpoint{0.563014in}{1.409919in}}%
\pgfpathlineto{\pgfqpoint{0.564088in}{1.405653in}}%
\pgfpathlineto{\pgfqpoint{0.567309in}{1.404490in}}%
\pgfpathlineto{\pgfqpoint{0.568383in}{1.403327in}}%
\pgfpathlineto{\pgfqpoint{0.570531in}{1.412560in}}%
\pgfpathlineto{\pgfqpoint{0.571604in}{1.412778in}}%
\pgfpathlineto{\pgfqpoint{0.575900in}{1.410428in}}%
\pgfpathlineto{\pgfqpoint{0.576974in}{1.408756in}}%
\pgfpathlineto{\pgfqpoint{0.579121in}{1.410694in}}%
\pgfpathlineto{\pgfqpoint{0.582343in}{1.411736in}}%
\pgfpathlineto{\pgfqpoint{0.583417in}{1.414620in}}%
\pgfpathlineto{\pgfqpoint{0.584490in}{1.413602in}}%
\pgfpathlineto{\pgfqpoint{0.585564in}{1.411373in}}%
\pgfpathlineto{\pgfqpoint{0.586638in}{1.412148in}}%
\pgfpathlineto{\pgfqpoint{0.590933in}{1.408949in}}%
\pgfpathlineto{\pgfqpoint{0.592007in}{1.410040in}}%
\pgfpathlineto{\pgfqpoint{0.593081in}{1.406841in}}%
\pgfpathlineto{\pgfqpoint{0.594155in}{1.412827in}}%
\pgfpathlineto{\pgfqpoint{0.597377in}{1.411833in}}%
\pgfpathlineto{\pgfqpoint{0.598450in}{1.413287in}}%
\pgfpathlineto{\pgfqpoint{0.600598in}{1.412536in}}%
\pgfpathlineto{\pgfqpoint{0.601672in}{1.409895in}}%
\pgfpathlineto{\pgfqpoint{0.605967in}{1.410016in}}%
\pgfpathlineto{\pgfqpoint{0.607041in}{1.406647in}}%
\pgfpathlineto{\pgfqpoint{0.608115in}{1.405290in}}%
\pgfpathlineto{\pgfqpoint{0.609189in}{1.408877in}}%
\pgfpathlineto{\pgfqpoint{0.612410in}{1.408101in}}%
\pgfpathlineto{\pgfqpoint{0.613484in}{1.408877in}}%
\pgfpathlineto{\pgfqpoint{0.615632in}{1.413433in}}%
\pgfpathlineto{\pgfqpoint{0.616706in}{1.410185in}}%
\pgfpathlineto{\pgfqpoint{0.619927in}{1.408222in}}%
\pgfpathlineto{\pgfqpoint{0.621001in}{1.405484in}}%
\pgfpathlineto{\pgfqpoint{0.624222in}{1.410282in}}%
\pgfpathlineto{\pgfqpoint{0.627444in}{1.411688in}}%
\pgfpathlineto{\pgfqpoint{0.628518in}{1.410597in}}%
\pgfpathlineto{\pgfqpoint{0.629592in}{1.411324in}}%
\pgfpathlineto{\pgfqpoint{0.630666in}{1.410597in}}%
\pgfpathlineto{\pgfqpoint{0.631739in}{1.415444in}}%
\pgfpathlineto{\pgfqpoint{0.634961in}{1.415129in}}%
\pgfpathlineto{\pgfqpoint{0.636035in}{1.412124in}}%
\pgfpathlineto{\pgfqpoint{0.639256in}{1.408299in}}%
\pgfpathlineto{\pgfqpoint{0.642478in}{1.407501in}}%
\pgfpathlineto{\pgfqpoint{0.643552in}{1.408094in}}%
\pgfpathlineto{\pgfqpoint{0.645699in}{1.405180in}}%
\pgfpathlineto{\pgfqpoint{0.646773in}{1.402197in}}%
\pgfpathlineto{\pgfqpoint{0.649995in}{1.402702in}}%
\pgfpathlineto{\pgfqpoint{0.651068in}{1.404166in}}%
\pgfpathlineto{\pgfqpoint{0.652142in}{1.403451in}}%
\pgfpathlineto{\pgfqpoint{0.653216in}{1.404287in}}%
\pgfpathlineto{\pgfqpoint{0.654290in}{1.402661in}}%
\pgfpathlineto{\pgfqpoint{0.658585in}{1.401666in}}%
\pgfpathlineto{\pgfqpoint{0.659659in}{1.402503in}}%
\pgfpathlineto{\pgfqpoint{0.660733in}{1.404492in}}%
\pgfpathlineto{\pgfqpoint{0.661807in}{1.403034in}}%
\pgfpathlineto{\pgfqpoint{0.667176in}{1.406340in}}%
\pgfpathlineto{\pgfqpoint{0.668250in}{1.403751in}}%
\pgfpathlineto{\pgfqpoint{0.669324in}{1.403730in}}%
\pgfpathlineto{\pgfqpoint{0.672545in}{1.405195in}}%
\pgfpathlineto{\pgfqpoint{0.674693in}{1.404777in}}%
\pgfpathlineto{\pgfqpoint{0.676841in}{1.398218in}}%
\pgfpathlineto{\pgfqpoint{0.680062in}{1.398717in}}%
\pgfpathlineto{\pgfqpoint{0.681136in}{1.399863in}}%
\pgfpathlineto{\pgfqpoint{0.682210in}{1.399454in}}%
\pgfpathlineto{\pgfqpoint{0.683284in}{1.400835in}}%
\pgfpathlineto{\pgfqpoint{0.687579in}{1.402415in}}%
\pgfpathlineto{\pgfqpoint{0.688653in}{1.404682in}}%
\pgfpathlineto{\pgfqpoint{0.691874in}{1.405560in}}%
\pgfpathlineto{\pgfqpoint{0.702613in}{1.405426in}}%
\pgfpathlineto{\pgfqpoint{0.704760in}{1.401031in}}%
\pgfpathlineto{\pgfqpoint{0.710130in}{1.401274in}}%
\pgfpathlineto{\pgfqpoint{0.711203in}{1.403990in}}%
\pgfpathlineto{\pgfqpoint{0.713351in}{1.400952in}}%
\pgfpathlineto{\pgfqpoint{0.714425in}{1.403196in}}%
\pgfpathlineto{\pgfqpoint{0.717646in}{1.403480in}}%
\pgfpathlineto{\pgfqpoint{0.721942in}{1.397727in}}%
\pgfpathlineto{\pgfqpoint{0.727311in}{1.397790in}}%
\pgfpathlineto{\pgfqpoint{0.728385in}{1.396194in}}%
\pgfpathlineto{\pgfqpoint{0.729459in}{1.397918in}}%
\pgfpathlineto{\pgfqpoint{0.732680in}{1.398212in}}%
\pgfpathlineto{\pgfqpoint{0.735902in}{1.388044in}}%
\pgfpathlineto{\pgfqpoint{0.736976in}{1.387503in}}%
\pgfpathlineto{\pgfqpoint{0.741271in}{1.389466in}}%
\pgfpathlineto{\pgfqpoint{0.742345in}{1.386361in}}%
\pgfpathlineto{\pgfqpoint{0.744492in}{1.387383in}}%
\pgfpathlineto{\pgfqpoint{0.748788in}{1.389847in}}%
\pgfpathlineto{\pgfqpoint{0.752009in}{1.386578in}}%
\pgfpathlineto{\pgfqpoint{0.756305in}{1.386235in}}%
\pgfpathlineto{\pgfqpoint{0.757378in}{1.384662in}}%
\pgfpathlineto{\pgfqpoint{0.758452in}{1.384717in}}%
\pgfpathlineto{\pgfqpoint{0.759526in}{1.384162in}}%
\pgfpathlineto{\pgfqpoint{0.762748in}{1.384657in}}%
\pgfpathlineto{\pgfqpoint{0.767043in}{1.382788in}}%
\pgfpathlineto{\pgfqpoint{0.772412in}{1.382769in}}%
\pgfpathlineto{\pgfqpoint{0.774560in}{1.384340in}}%
\pgfpathlineto{\pgfqpoint{0.777781in}{1.384321in}}%
\pgfpathlineto{\pgfqpoint{0.778855in}{1.380807in}}%
\pgfpathlineto{\pgfqpoint{0.779929in}{1.379606in}}%
\pgfpathlineto{\pgfqpoint{0.781003in}{1.379402in}}%
\pgfpathlineto{\pgfqpoint{0.782077in}{1.380320in}}%
\pgfpathlineto{\pgfqpoint{0.788520in}{1.381030in}}%
\pgfpathlineto{\pgfqpoint{0.789594in}{1.382899in}}%
\pgfpathlineto{\pgfqpoint{0.792815in}{1.381062in}}%
\pgfpathlineto{\pgfqpoint{0.794963in}{1.378072in}}%
\pgfpathlineto{\pgfqpoint{0.797110in}{1.376697in}}%
\pgfpathlineto{\pgfqpoint{0.800332in}{1.377267in}}%
\pgfpathlineto{\pgfqpoint{0.801406in}{1.378696in}}%
\pgfpathlineto{\pgfqpoint{0.803554in}{1.376674in}}%
\pgfpathlineto{\pgfqpoint{0.808923in}{1.375299in}}%
\pgfpathlineto{\pgfqpoint{0.809997in}{1.375824in}}%
\pgfpathlineto{\pgfqpoint{0.812144in}{1.373946in}}%
\pgfpathlineto{\pgfqpoint{0.816440in}{1.373262in}}%
\pgfpathlineto{\pgfqpoint{0.819661in}{1.370260in}}%
\pgfpathlineto{\pgfqpoint{0.823956in}{1.370304in}}%
\pgfpathlineto{\pgfqpoint{0.825030in}{1.371120in}}%
\pgfpathlineto{\pgfqpoint{0.826104in}{1.373057in}}%
\pgfpathlineto{\pgfqpoint{0.827178in}{1.370088in}}%
\pgfpathlineto{\pgfqpoint{0.832547in}{1.370670in}}%
\pgfpathlineto{\pgfqpoint{0.834695in}{1.369777in}}%
\pgfpathlineto{\pgfqpoint{0.837916in}{1.370369in}}%
\pgfpathlineto{\pgfqpoint{0.840064in}{1.369829in}}%
\pgfpathlineto{\pgfqpoint{0.841138in}{1.369598in}}%
\pgfpathlineto{\pgfqpoint{0.842212in}{1.368244in}}%
\pgfpathlineto{\pgfqpoint{0.846507in}{1.367963in}}%
\pgfpathlineto{\pgfqpoint{0.848655in}{1.369910in}}%
\pgfpathlineto{\pgfqpoint{0.849729in}{1.368562in}}%
\pgfpathlineto{\pgfqpoint{0.852950in}{1.371151in}}%
\pgfpathlineto{\pgfqpoint{0.856172in}{1.368395in}}%
\pgfpathlineto{\pgfqpoint{0.860467in}{1.370128in}}%
\pgfpathlineto{\pgfqpoint{0.861541in}{1.367769in}}%
\pgfpathlineto{\pgfqpoint{0.862615in}{1.367699in}}%
\pgfpathlineto{\pgfqpoint{0.864762in}{1.366506in}}%
\pgfpathlineto{\pgfqpoint{0.871205in}{1.364909in}}%
\pgfpathlineto{\pgfqpoint{0.876575in}{1.365289in}}%
\pgfpathlineto{\pgfqpoint{0.877648in}{1.364902in}}%
\pgfpathlineto{\pgfqpoint{0.878722in}{1.366054in}}%
\pgfpathlineto{\pgfqpoint{0.879796in}{1.364701in}}%
\pgfpathlineto{\pgfqpoint{0.885165in}{1.365827in}}%
\pgfpathlineto{\pgfqpoint{0.886239in}{1.364870in}}%
\pgfpathlineto{\pgfqpoint{0.891608in}{1.365372in}}%
\pgfpathlineto{\pgfqpoint{0.893756in}{1.365171in}}%
\pgfpathlineto{\pgfqpoint{0.894830in}{1.365771in}}%
\pgfpathlineto{\pgfqpoint{0.899125in}{1.363690in}}%
\pgfpathlineto{\pgfqpoint{0.901273in}{1.362185in}}%
\pgfpathlineto{\pgfqpoint{0.902347in}{1.362324in}}%
\pgfpathlineto{\pgfqpoint{0.905568in}{1.364683in}}%
\pgfpathlineto{\pgfqpoint{0.906642in}{1.363548in}}%
\pgfpathlineto{\pgfqpoint{0.908790in}{1.366895in}}%
\pgfpathlineto{\pgfqpoint{0.909864in}{1.364913in}}%
\pgfpathlineto{\pgfqpoint{0.913085in}{1.364448in}}%
\pgfpathlineto{\pgfqpoint{0.915233in}{1.366306in}}%
\pgfpathlineto{\pgfqpoint{0.916307in}{1.366184in}}%
\pgfpathlineto{\pgfqpoint{0.917380in}{1.367299in}}%
\pgfpathlineto{\pgfqpoint{0.920602in}{1.366743in}}%
\pgfpathlineto{\pgfqpoint{0.922750in}{1.367419in}}%
\pgfpathlineto{\pgfqpoint{0.923823in}{1.366584in}}%
\pgfpathlineto{\pgfqpoint{0.924897in}{1.364981in}}%
\pgfpathlineto{\pgfqpoint{0.929193in}{1.364107in}}%
\pgfpathlineto{\pgfqpoint{0.932414in}{1.362578in}}%
\pgfpathlineto{\pgfqpoint{0.935636in}{1.362896in}}%
\pgfpathlineto{\pgfqpoint{0.937783in}{1.361371in}}%
\pgfpathlineto{\pgfqpoint{0.938857in}{1.362066in}}%
\pgfpathlineto{\pgfqpoint{0.939931in}{1.359687in}}%
\pgfpathlineto{\pgfqpoint{0.944226in}{1.359482in}}%
\pgfpathlineto{\pgfqpoint{0.946374in}{1.361635in}}%
\pgfpathlineto{\pgfqpoint{0.947448in}{1.362122in}}%
\pgfpathlineto{\pgfqpoint{0.951743in}{1.361100in}}%
\pgfpathlineto{\pgfqpoint{0.952817in}{1.362015in}}%
\pgfpathlineto{\pgfqpoint{0.953891in}{1.361096in}}%
\pgfpathlineto{\pgfqpoint{0.954965in}{1.362234in}}%
\pgfpathlineto{\pgfqpoint{0.958186in}{1.361867in}}%
\pgfpathlineto{\pgfqpoint{0.961408in}{1.364607in}}%
\pgfpathlineto{\pgfqpoint{0.962482in}{1.362610in}}%
\pgfpathlineto{\pgfqpoint{0.966777in}{1.363518in}}%
\pgfpathlineto{\pgfqpoint{0.967851in}{1.364633in}}%
\pgfpathlineto{\pgfqpoint{0.968925in}{1.362792in}}%
\pgfpathlineto{\pgfqpoint{0.969998in}{1.363087in}}%
\pgfpathlineto{\pgfqpoint{0.973220in}{1.362019in}}%
\pgfpathlineto{\pgfqpoint{0.974294in}{1.360721in}}%
\pgfpathlineto{\pgfqpoint{0.976442in}{1.364872in}}%
\pgfpathlineto{\pgfqpoint{0.981811in}{1.365521in}}%
\pgfpathlineto{\pgfqpoint{0.983958in}{1.363515in}}%
\pgfpathlineto{\pgfqpoint{0.985032in}{1.364164in}}%
\pgfpathlineto{\pgfqpoint{0.988254in}{1.362113in}}%
\pgfpathlineto{\pgfqpoint{0.989328in}{1.363137in}}%
\pgfpathlineto{\pgfqpoint{0.992549in}{1.360132in}}%
\pgfpathlineto{\pgfqpoint{0.995771in}{1.359657in}}%
\pgfpathlineto{\pgfqpoint{0.996844in}{1.358572in}}%
\pgfpathlineto{\pgfqpoint{0.997918in}{1.358808in}}%
\pgfpathlineto{\pgfqpoint{0.998992in}{1.356862in}}%
\pgfpathlineto{\pgfqpoint{1.005435in}{1.355702in}}%
\pgfpathlineto{\pgfqpoint{1.006509in}{1.355245in}}%
\pgfpathlineto{\pgfqpoint{1.007583in}{1.354152in}}%
\pgfpathlineto{\pgfqpoint{1.010804in}{1.354497in}}%
\pgfpathlineto{\pgfqpoint{1.011878in}{1.351671in}}%
\pgfpathlineto{\pgfqpoint{1.015100in}{1.351803in}}%
\pgfpathlineto{\pgfqpoint{1.020469in}{1.351270in}}%
\pgfpathlineto{\pgfqpoint{1.022617in}{1.349354in}}%
\pgfpathlineto{\pgfqpoint{1.025838in}{1.350300in}}%
\pgfpathlineto{\pgfqpoint{1.026912in}{1.351544in}}%
\pgfpathlineto{\pgfqpoint{1.029060in}{1.350607in}}%
\pgfpathlineto{\pgfqpoint{1.030133in}{1.351168in}}%
\pgfpathlineto{\pgfqpoint{1.033355in}{1.351239in}}%
\pgfpathlineto{\pgfqpoint{1.034429in}{1.350131in}}%
\pgfpathlineto{\pgfqpoint{1.035503in}{1.351163in}}%
\pgfpathlineto{\pgfqpoint{1.036576in}{1.352990in}}%
\pgfpathlineto{\pgfqpoint{1.037650in}{1.352906in}}%
\pgfpathlineto{\pgfqpoint{1.041946in}{1.353540in}}%
\pgfpathlineto{\pgfqpoint{1.043020in}{1.354181in}}%
\pgfpathlineto{\pgfqpoint{1.044093in}{1.353001in}}%
\pgfpathlineto{\pgfqpoint{1.048389in}{1.353604in}}%
\pgfpathlineto{\pgfqpoint{1.049463in}{1.355797in}}%
\pgfpathlineto{\pgfqpoint{1.052684in}{1.355648in}}%
\pgfpathlineto{\pgfqpoint{1.059127in}{1.352217in}}%
\pgfpathlineto{\pgfqpoint{1.060201in}{1.352623in}}%
\pgfpathlineto{\pgfqpoint{1.063422in}{1.351490in}}%
\pgfpathlineto{\pgfqpoint{1.065570in}{1.348546in}}%
\pgfpathlineto{\pgfqpoint{1.067718in}{1.348085in}}%
\pgfpathlineto{\pgfqpoint{1.073087in}{1.345949in}}%
\pgfpathlineto{\pgfqpoint{1.074161in}{1.345169in}}%
\pgfpathlineto{\pgfqpoint{1.075235in}{1.347039in}}%
\pgfpathlineto{\pgfqpoint{1.082752in}{1.347208in}}%
\pgfpathlineto{\pgfqpoint{1.087047in}{1.348833in}}%
\pgfpathlineto{\pgfqpoint{1.089195in}{1.351980in}}%
\pgfpathlineto{\pgfqpoint{1.090268in}{1.351461in}}%
\pgfpathlineto{\pgfqpoint{1.093490in}{1.351677in}}%
\pgfpathlineto{\pgfqpoint{1.094564in}{1.352811in}}%
\pgfpathlineto{\pgfqpoint{1.095638in}{1.352737in}}%
\pgfpathlineto{\pgfqpoint{1.096711in}{1.349921in}}%
\pgfpathlineto{\pgfqpoint{1.097785in}{1.348978in}}%
\pgfpathlineto{\pgfqpoint{1.101007in}{1.349065in}}%
\pgfpathlineto{\pgfqpoint{1.102081in}{1.350001in}}%
\pgfpathlineto{\pgfqpoint{1.103154in}{1.349454in}}%
\pgfpathlineto{\pgfqpoint{1.104228in}{1.348097in}}%
\pgfpathlineto{\pgfqpoint{1.105302in}{1.348326in}}%
\pgfpathlineto{\pgfqpoint{1.108524in}{1.348431in}}%
\pgfpathlineto{\pgfqpoint{1.109597in}{1.349672in}}%
\pgfpathlineto{\pgfqpoint{1.112819in}{1.348485in}}%
\pgfpathlineto{\pgfqpoint{1.116041in}{1.349766in}}%
\pgfpathlineto{\pgfqpoint{1.117114in}{1.349400in}}%
\pgfpathlineto{\pgfqpoint{1.118188in}{1.349803in}}%
\pgfpathlineto{\pgfqpoint{1.119262in}{1.349545in}}%
\pgfpathlineto{\pgfqpoint{1.120336in}{1.348460in}}%
\pgfpathlineto{\pgfqpoint{1.125705in}{1.347488in}}%
\pgfpathlineto{\pgfqpoint{1.126779in}{1.348638in}}%
\pgfpathlineto{\pgfqpoint{1.127853in}{1.347564in}}%
\pgfpathlineto{\pgfqpoint{1.133222in}{1.347965in}}%
\pgfpathlineto{\pgfqpoint{1.134296in}{1.346936in}}%
\pgfpathlineto{\pgfqpoint{1.135370in}{1.347020in}}%
\pgfpathlineto{\pgfqpoint{1.147182in}{1.345134in}}%
\pgfpathlineto{\pgfqpoint{1.148256in}{1.344688in}}%
\pgfpathlineto{\pgfqpoint{1.156846in}{1.346620in}}%
\pgfpathlineto{\pgfqpoint{1.157920in}{1.344946in}}%
\pgfpathlineto{\pgfqpoint{1.162216in}{1.345053in}}%
\pgfpathlineto{\pgfqpoint{1.163289in}{1.346865in}}%
\pgfpathlineto{\pgfqpoint{1.165437in}{1.347941in}}%
\pgfpathlineto{\pgfqpoint{1.168659in}{1.346950in}}%
\pgfpathlineto{\pgfqpoint{1.169732in}{1.347696in}}%
\pgfpathlineto{\pgfqpoint{1.170806in}{1.345884in}}%
\pgfpathlineto{\pgfqpoint{1.171880in}{1.346130in}}%
\pgfpathlineto{\pgfqpoint{1.172954in}{1.345196in}}%
\pgfpathlineto{\pgfqpoint{1.177249in}{1.344561in}}%
\pgfpathlineto{\pgfqpoint{1.180471in}{1.343538in}}%
\pgfpathlineto{\pgfqpoint{1.183692in}{1.343589in}}%
\pgfpathlineto{\pgfqpoint{1.184766in}{1.342786in}}%
\pgfpathlineto{\pgfqpoint{1.186914in}{1.343752in}}%
\pgfpathlineto{\pgfqpoint{1.187988in}{1.343440in}}%
\pgfpathlineto{\pgfqpoint{1.191209in}{1.343526in}}%
\pgfpathlineto{\pgfqpoint{1.193357in}{1.342881in}}%
\pgfpathlineto{\pgfqpoint{1.195505in}{1.342745in}}%
\pgfpathlineto{\pgfqpoint{1.199800in}{1.343636in}}%
\pgfpathlineto{\pgfqpoint{1.200874in}{1.342546in}}%
\pgfpathlineto{\pgfqpoint{1.207317in}{1.341873in}}%
\pgfpathlineto{\pgfqpoint{1.208391in}{1.341052in}}%
\pgfpathlineto{\pgfqpoint{1.209464in}{1.341963in}}%
\pgfpathlineto{\pgfqpoint{1.210538in}{1.344175in}}%
\pgfpathlineto{\pgfqpoint{1.214834in}{1.342590in}}%
\pgfpathlineto{\pgfqpoint{1.215908in}{1.343030in}}%
\pgfpathlineto{\pgfqpoint{1.216981in}{1.341979in}}%
\pgfpathlineto{\pgfqpoint{1.218055in}{1.342454in}}%
\pgfpathlineto{\pgfqpoint{1.223424in}{1.347097in}}%
\pgfpathlineto{\pgfqpoint{1.225572in}{1.344806in}}%
\pgfpathlineto{\pgfqpoint{1.228794in}{1.344128in}}%
\pgfpathlineto{\pgfqpoint{1.230941in}{1.342774in}}%
\pgfpathlineto{\pgfqpoint{1.233089in}{1.341962in}}%
\pgfpathlineto{\pgfqpoint{1.238458in}{1.341836in}}%
\pgfpathlineto{\pgfqpoint{1.240606in}{1.340867in}}%
\pgfpathlineto{\pgfqpoint{1.243827in}{1.340070in}}%
\pgfpathlineto{\pgfqpoint{1.245975in}{1.340332in}}%
\pgfpathlineto{\pgfqpoint{1.248123in}{1.339692in}}%
\pgfpathlineto{\pgfqpoint{1.251344in}{1.339874in}}%
\pgfpathlineto{\pgfqpoint{1.252418in}{1.339295in}}%
\pgfpathlineto{\pgfqpoint{1.254566in}{1.339177in}}%
\pgfpathlineto{\pgfqpoint{1.255640in}{1.338802in}}%
\pgfpathlineto{\pgfqpoint{1.258861in}{1.339173in}}%
\pgfpathlineto{\pgfqpoint{1.259935in}{1.340632in}}%
\pgfpathlineto{\pgfqpoint{1.261009in}{1.340510in}}%
\pgfpathlineto{\pgfqpoint{1.262083in}{1.342574in}}%
\pgfpathlineto{\pgfqpoint{1.263156in}{1.342795in}}%
\pgfpathlineto{\pgfqpoint{1.267452in}{1.341241in}}%
\pgfpathlineto{\pgfqpoint{1.269599in}{1.341901in}}%
\pgfpathlineto{\pgfqpoint{1.270673in}{1.341316in}}%
\pgfpathlineto{\pgfqpoint{1.273895in}{1.341909in}}%
\pgfpathlineto{\pgfqpoint{1.274969in}{1.340840in}}%
\pgfpathlineto{\pgfqpoint{1.277116in}{1.341847in}}%
\pgfpathlineto{\pgfqpoint{1.278190in}{1.341114in}}%
\pgfpathlineto{\pgfqpoint{1.282485in}{1.338896in}}%
\pgfpathlineto{\pgfqpoint{1.283559in}{1.337569in}}%
\pgfpathlineto{\pgfqpoint{1.284633in}{1.337636in}}%
\pgfpathlineto{\pgfqpoint{1.285707in}{1.338689in}}%
\pgfpathlineto{\pgfqpoint{1.290002in}{1.340329in}}%
\pgfpathlineto{\pgfqpoint{1.291076in}{1.339436in}}%
\pgfpathlineto{\pgfqpoint{1.292150in}{1.341152in}}%
\pgfpathlineto{\pgfqpoint{1.293224in}{1.341604in}}%
\pgfpathlineto{\pgfqpoint{1.297519in}{1.340220in}}%
\pgfpathlineto{\pgfqpoint{1.298593in}{1.338646in}}%
\pgfpathlineto{\pgfqpoint{1.299667in}{1.338306in}}%
\pgfpathlineto{\pgfqpoint{1.303962in}{1.338490in}}%
\pgfpathlineto{\pgfqpoint{1.306110in}{1.337579in}}%
\pgfpathlineto{\pgfqpoint{1.307184in}{1.338023in}}%
\pgfpathlineto{\pgfqpoint{1.308258in}{1.339210in}}%
\pgfpathlineto{\pgfqpoint{1.313627in}{1.338436in}}%
\pgfpathlineto{\pgfqpoint{1.314701in}{1.339474in}}%
\pgfpathlineto{\pgfqpoint{1.315774in}{1.339664in}}%
\pgfpathlineto{\pgfqpoint{1.318996in}{1.339458in}}%
\pgfpathlineto{\pgfqpoint{1.320070in}{1.340018in}}%
\pgfpathlineto{\pgfqpoint{1.322218in}{1.339307in}}%
\pgfpathlineto{\pgfqpoint{1.323291in}{1.339331in}}%
\pgfpathlineto{\pgfqpoint{1.327587in}{1.338121in}}%
\pgfpathlineto{\pgfqpoint{1.329734in}{1.339794in}}%
\pgfpathlineto{\pgfqpoint{1.330808in}{1.340582in}}%
\pgfpathlineto{\pgfqpoint{1.334030in}{1.340265in}}%
\pgfpathlineto{\pgfqpoint{1.335104in}{1.341626in}}%
\pgfpathlineto{\pgfqpoint{1.336177in}{1.340352in}}%
\pgfpathlineto{\pgfqpoint{1.353359in}{1.337528in}}%
\pgfpathlineto{\pgfqpoint{1.356580in}{1.336737in}}%
\pgfpathlineto{\pgfqpoint{1.360876in}{1.338925in}}%
\pgfpathlineto{\pgfqpoint{1.367319in}{1.338389in}}%
\pgfpathlineto{\pgfqpoint{1.368393in}{1.338104in}}%
\pgfpathlineto{\pgfqpoint{1.371614in}{1.338685in}}%
\pgfpathlineto{\pgfqpoint{1.372688in}{1.339677in}}%
\pgfpathlineto{\pgfqpoint{1.375909in}{1.339048in}}%
\pgfpathlineto{\pgfqpoint{1.379131in}{1.339831in}}%
\pgfpathlineto{\pgfqpoint{1.380205in}{1.339488in}}%
\pgfpathlineto{\pgfqpoint{1.381279in}{1.340042in}}%
\pgfpathlineto{\pgfqpoint{1.382352in}{1.339769in}}%
\pgfpathlineto{\pgfqpoint{1.386648in}{1.340072in}}%
\pgfpathlineto{\pgfqpoint{1.388796in}{1.340966in}}%
\pgfpathlineto{\pgfqpoint{1.389869in}{1.341204in}}%
\pgfpathlineto{\pgfqpoint{1.390943in}{1.340750in}}%
\pgfpathlineto{\pgfqpoint{1.395239in}{1.340243in}}%
\pgfpathlineto{\pgfqpoint{1.396312in}{1.340582in}}%
\pgfpathlineto{\pgfqpoint{1.397386in}{1.341729in}}%
\pgfpathlineto{\pgfqpoint{1.398460in}{1.341135in}}%
\pgfpathlineto{\pgfqpoint{1.401682in}{1.341564in}}%
\pgfpathlineto{\pgfqpoint{1.403829in}{1.343986in}}%
\pgfpathlineto{\pgfqpoint{1.405977in}{1.344648in}}%
\pgfpathlineto{\pgfqpoint{1.409198in}{1.344559in}}%
\pgfpathlineto{\pgfqpoint{1.412420in}{1.347671in}}%
\pgfpathlineto{\pgfqpoint{1.413494in}{1.348008in}}%
\pgfpathlineto{\pgfqpoint{1.417789in}{1.347775in}}%
\pgfpathlineto{\pgfqpoint{1.418863in}{1.348825in}}%
\pgfpathlineto{\pgfqpoint{1.419937in}{1.348450in}}%
\pgfpathlineto{\pgfqpoint{1.421011in}{1.346944in}}%
\pgfpathlineto{\pgfqpoint{1.424232in}{1.347143in}}%
\pgfpathlineto{\pgfqpoint{1.425306in}{1.347827in}}%
\pgfpathlineto{\pgfqpoint{1.426380in}{1.346766in}}%
\pgfpathlineto{\pgfqpoint{1.428528in}{1.346699in}}%
\pgfpathlineto{\pgfqpoint{1.434971in}{1.343094in}}%
\pgfpathlineto{\pgfqpoint{1.436044in}{1.343605in}}%
\pgfpathlineto{\pgfqpoint{1.440340in}{1.343385in}}%
\pgfpathlineto{\pgfqpoint{1.443561in}{1.344710in}}%
\pgfpathlineto{\pgfqpoint{1.450004in}{1.343724in}}%
\pgfpathlineto{\pgfqpoint{1.451078in}{1.344128in}}%
\pgfpathlineto{\pgfqpoint{1.466112in}{1.344319in}}%
\pgfpathlineto{\pgfqpoint{1.470407in}{1.346748in}}%
\pgfpathlineto{\pgfqpoint{1.471481in}{1.346201in}}%
\pgfpathlineto{\pgfqpoint{1.472555in}{1.347514in}}%
\pgfpathlineto{\pgfqpoint{1.473629in}{1.346871in}}%
\pgfpathlineto{\pgfqpoint{1.477924in}{1.346567in}}%
\pgfpathlineto{\pgfqpoint{1.478998in}{1.348033in}}%
\pgfpathlineto{\pgfqpoint{1.480072in}{1.348557in}}%
\pgfpathlineto{\pgfqpoint{1.481146in}{1.347526in}}%
\pgfpathlineto{\pgfqpoint{1.484367in}{1.347698in}}%
\pgfpathlineto{\pgfqpoint{1.485441in}{1.350089in}}%
\pgfpathlineto{\pgfqpoint{1.486515in}{1.349001in}}%
\pgfpathlineto{\pgfqpoint{1.487589in}{1.351375in}}%
\pgfpathlineto{\pgfqpoint{1.488662in}{1.351375in}}%
\pgfpathlineto{\pgfqpoint{1.491884in}{1.351961in}}%
\pgfpathlineto{\pgfqpoint{1.492958in}{1.351198in}}%
\pgfpathlineto{\pgfqpoint{1.494032in}{1.352074in}}%
\pgfpathlineto{\pgfqpoint{1.495106in}{1.352021in}}%
\pgfpathlineto{\pgfqpoint{1.496179in}{1.349855in}}%
\pgfpathlineto{\pgfqpoint{1.501549in}{1.350163in}}%
\pgfpathlineto{\pgfqpoint{1.502622in}{1.348346in}}%
\pgfpathlineto{\pgfqpoint{1.503696in}{1.347787in}}%
\pgfpathlineto{\pgfqpoint{1.506918in}{1.347467in}}%
\pgfpathlineto{\pgfqpoint{1.507992in}{1.345698in}}%
\pgfpathlineto{\pgfqpoint{1.509065in}{1.346040in}}%
\pgfpathlineto{\pgfqpoint{1.511213in}{1.345094in}}%
\pgfpathlineto{\pgfqpoint{1.514435in}{1.345660in}}%
\pgfpathlineto{\pgfqpoint{1.518730in}{1.343421in}}%
\pgfpathlineto{\pgfqpoint{1.521951in}{1.343482in}}%
\pgfpathlineto{\pgfqpoint{1.523025in}{1.344062in}}%
\pgfpathlineto{\pgfqpoint{1.525173in}{1.343688in}}%
\pgfpathlineto{\pgfqpoint{1.526247in}{1.344226in}}%
\pgfpathlineto{\pgfqpoint{1.529468in}{1.344298in}}%
\pgfpathlineto{\pgfqpoint{1.530542in}{1.343152in}}%
\pgfpathlineto{\pgfqpoint{1.532690in}{1.343134in}}%
\pgfpathlineto{\pgfqpoint{1.533764in}{1.341984in}}%
\pgfpathlineto{\pgfqpoint{1.536985in}{1.343129in}}%
\pgfpathlineto{\pgfqpoint{1.538059in}{1.340854in}}%
\pgfpathlineto{\pgfqpoint{1.539133in}{1.341994in}}%
\pgfpathlineto{\pgfqpoint{1.545576in}{1.342360in}}%
\pgfpathlineto{\pgfqpoint{1.546650in}{1.341357in}}%
\pgfpathlineto{\pgfqpoint{1.548797in}{1.341132in}}%
\pgfpathlineto{\pgfqpoint{1.552019in}{1.339826in}}%
\pgfpathlineto{\pgfqpoint{1.553093in}{1.338432in}}%
\pgfpathlineto{\pgfqpoint{1.554167in}{1.339467in}}%
\pgfpathlineto{\pgfqpoint{1.555240in}{1.339084in}}%
\pgfpathlineto{\pgfqpoint{1.556314in}{1.340422in}}%
\pgfpathlineto{\pgfqpoint{1.559536in}{1.340594in}}%
\pgfpathlineto{\pgfqpoint{1.562757in}{1.336977in}}%
\pgfpathlineto{\pgfqpoint{1.563831in}{1.337894in}}%
\pgfpathlineto{\pgfqpoint{1.568127in}{1.336787in}}%
\pgfpathlineto{\pgfqpoint{1.569200in}{1.336989in}}%
\pgfpathlineto{\pgfqpoint{1.571348in}{1.336764in}}%
\pgfpathlineto{\pgfqpoint{1.575643in}{1.337462in}}%
\pgfpathlineto{\pgfqpoint{1.576717in}{1.338277in}}%
\pgfpathlineto{\pgfqpoint{1.578865in}{1.338253in}}%
\pgfpathlineto{\pgfqpoint{1.582086in}{1.339581in}}%
\pgfpathlineto{\pgfqpoint{1.583160in}{1.340730in}}%
\pgfpathlineto{\pgfqpoint{1.585308in}{1.338447in}}%
\pgfpathlineto{\pgfqpoint{1.586382in}{1.338893in}}%
\pgfpathlineto{\pgfqpoint{1.590677in}{1.338663in}}%
\pgfpathlineto{\pgfqpoint{1.592825in}{1.339236in}}%
\pgfpathlineto{\pgfqpoint{1.593899in}{1.337364in}}%
\pgfpathlineto{\pgfqpoint{1.598194in}{1.336428in}}%
\pgfpathlineto{\pgfqpoint{1.600342in}{1.334513in}}%
\pgfpathlineto{\pgfqpoint{1.604637in}{1.335594in}}%
\pgfpathlineto{\pgfqpoint{1.605711in}{1.335332in}}%
\pgfpathlineto{\pgfqpoint{1.606785in}{1.336777in}}%
\pgfpathlineto{\pgfqpoint{1.607859in}{1.337092in}}%
\pgfpathlineto{\pgfqpoint{1.608932in}{1.338219in}}%
\pgfpathlineto{\pgfqpoint{1.612154in}{1.336966in}}%
\pgfpathlineto{\pgfqpoint{1.613228in}{1.335412in}}%
\pgfpathlineto{\pgfqpoint{1.614302in}{1.336123in}}%
\pgfpathlineto{\pgfqpoint{1.615375in}{1.334594in}}%
\pgfpathlineto{\pgfqpoint{1.620745in}{1.335042in}}%
\pgfpathlineto{\pgfqpoint{1.621818in}{1.335086in}}%
\pgfpathlineto{\pgfqpoint{1.623966in}{1.333455in}}%
\pgfpathlineto{\pgfqpoint{1.629335in}{1.332994in}}%
\pgfpathlineto{\pgfqpoint{1.631483in}{1.331712in}}%
\pgfpathlineto{\pgfqpoint{1.637926in}{1.332566in}}%
\pgfpathlineto{\pgfqpoint{1.639000in}{1.333023in}}%
\pgfpathlineto{\pgfqpoint{1.642221in}{1.332240in}}%
\pgfpathlineto{\pgfqpoint{1.644369in}{1.333537in}}%
\pgfpathlineto{\pgfqpoint{1.645443in}{1.333696in}}%
\pgfpathlineto{\pgfqpoint{1.646517in}{1.334542in}}%
\pgfpathlineto{\pgfqpoint{1.649738in}{1.333066in}}%
\pgfpathlineto{\pgfqpoint{1.650812in}{1.335728in}}%
\pgfpathlineto{\pgfqpoint{1.651886in}{1.335113in}}%
\pgfpathlineto{\pgfqpoint{1.652960in}{1.333230in}}%
\pgfpathlineto{\pgfqpoint{1.654034in}{1.334775in}}%
\pgfpathlineto{\pgfqpoint{1.659403in}{1.333768in}}%
\pgfpathlineto{\pgfqpoint{1.661550in}{1.334297in}}%
\pgfpathlineto{\pgfqpoint{1.665846in}{1.334642in}}%
\pgfpathlineto{\pgfqpoint{1.666920in}{1.336219in}}%
\pgfpathlineto{\pgfqpoint{1.667994in}{1.336494in}}%
\pgfpathlineto{\pgfqpoint{1.672289in}{1.334640in}}%
\pgfpathlineto{\pgfqpoint{1.674437in}{1.336527in}}%
\pgfpathlineto{\pgfqpoint{1.675510in}{1.335717in}}%
\pgfpathlineto{\pgfqpoint{1.684101in}{1.334839in}}%
\pgfpathlineto{\pgfqpoint{1.688396in}{1.335331in}}%
\pgfpathlineto{\pgfqpoint{1.690544in}{1.335469in}}%
\pgfpathlineto{\pgfqpoint{1.691618in}{1.337024in}}%
\pgfpathlineto{\pgfqpoint{1.698061in}{1.335570in}}%
\pgfpathlineto{\pgfqpoint{1.699135in}{1.336296in}}%
\pgfpathlineto{\pgfqpoint{1.704504in}{1.336817in}}%
\pgfpathlineto{\pgfqpoint{1.705578in}{1.337921in}}%
\pgfpathlineto{\pgfqpoint{1.706652in}{1.336739in}}%
\pgfpathlineto{\pgfqpoint{1.709873in}{1.336059in}}%
\pgfpathlineto{\pgfqpoint{1.710947in}{1.336972in}}%
\pgfpathlineto{\pgfqpoint{1.712021in}{1.336972in}}%
\pgfpathlineto{\pgfqpoint{1.713095in}{1.336311in}}%
\pgfpathlineto{\pgfqpoint{1.714169in}{1.334712in}}%
\pgfpathlineto{\pgfqpoint{1.719538in}{1.334972in}}%
\pgfpathlineto{\pgfqpoint{1.720612in}{1.333817in}}%
\pgfpathlineto{\pgfqpoint{1.721685in}{1.334230in}}%
\pgfpathlineto{\pgfqpoint{1.729202in}{1.334142in}}%
\pgfpathlineto{\pgfqpoint{1.733498in}{1.335258in}}%
\pgfpathlineto{\pgfqpoint{1.735645in}{1.334720in}}%
\pgfpathlineto{\pgfqpoint{1.736719in}{1.335058in}}%
\pgfpathlineto{\pgfqpoint{1.741015in}{1.334948in}}%
\pgfpathlineto{\pgfqpoint{1.742088in}{1.334183in}}%
\pgfpathlineto{\pgfqpoint{1.743162in}{1.335061in}}%
\pgfpathlineto{\pgfqpoint{1.748531in}{1.335611in}}%
\pgfpathlineto{\pgfqpoint{1.750679in}{1.334277in}}%
\pgfpathlineto{\pgfqpoint{1.756048in}{1.336459in}}%
\pgfpathlineto{\pgfqpoint{1.763565in}{1.336246in}}%
\pgfpathlineto{\pgfqpoint{1.765713in}{1.337565in}}%
\pgfpathlineto{\pgfqpoint{1.766787in}{1.337410in}}%
\pgfpathlineto{\pgfqpoint{1.773230in}{1.340222in}}%
\pgfpathlineto{\pgfqpoint{1.777525in}{1.340371in}}%
\pgfpathlineto{\pgfqpoint{1.778599in}{1.339823in}}%
\pgfpathlineto{\pgfqpoint{1.779673in}{1.340879in}}%
\pgfpathlineto{\pgfqpoint{1.780747in}{1.340686in}}%
\pgfpathlineto{\pgfqpoint{1.781820in}{1.339732in}}%
\pgfpathlineto{\pgfqpoint{1.786116in}{1.338751in}}%
\pgfpathlineto{\pgfqpoint{1.792559in}{1.339497in}}%
\pgfpathlineto{\pgfqpoint{1.793633in}{1.345200in}}%
\pgfpathlineto{\pgfqpoint{1.796854in}{1.348129in}}%
\pgfpathlineto{\pgfqpoint{1.801149in}{1.348428in}}%
\pgfpathlineto{\pgfqpoint{1.802223in}{1.348075in}}%
\pgfpathlineto{\pgfqpoint{1.803297in}{1.346847in}}%
\pgfpathlineto{\pgfqpoint{1.809740in}{1.348490in}}%
\pgfpathlineto{\pgfqpoint{1.810814in}{1.348399in}}%
\pgfpathlineto{\pgfqpoint{1.811888in}{1.349023in}}%
\pgfpathlineto{\pgfqpoint{1.815109in}{1.347860in}}%
\pgfpathlineto{\pgfqpoint{1.816183in}{1.348705in}}%
\pgfpathlineto{\pgfqpoint{1.817257in}{1.348046in}}%
\pgfpathlineto{\pgfqpoint{1.819405in}{1.348035in}}%
\pgfpathlineto{\pgfqpoint{1.823700in}{1.347386in}}%
\pgfpathlineto{\pgfqpoint{1.824774in}{1.348459in}}%
\pgfpathlineto{\pgfqpoint{1.826922in}{1.343689in}}%
\pgfpathlineto{\pgfqpoint{1.830143in}{1.341732in}}%
\pgfpathlineto{\pgfqpoint{1.831217in}{1.339836in}}%
\pgfpathlineto{\pgfqpoint{1.833365in}{1.343931in}}%
\pgfpathlineto{\pgfqpoint{1.834438in}{1.343901in}}%
\pgfpathlineto{\pgfqpoint{1.837660in}{1.342408in}}%
\pgfpathlineto{\pgfqpoint{1.838734in}{1.340674in}}%
\pgfpathlineto{\pgfqpoint{1.840882in}{1.342680in}}%
\pgfpathlineto{\pgfqpoint{1.841955in}{1.341561in}}%
\pgfpathlineto{\pgfqpoint{1.846251in}{1.343517in}}%
\pgfpathlineto{\pgfqpoint{1.848398in}{1.342217in}}%
\pgfpathlineto{\pgfqpoint{1.849472in}{1.343074in}}%
\pgfpathlineto{\pgfqpoint{1.852694in}{1.342610in}}%
\pgfpathlineto{\pgfqpoint{1.854841in}{1.344546in}}%
\pgfpathlineto{\pgfqpoint{1.855915in}{1.343971in}}%
\pgfpathlineto{\pgfqpoint{1.856989in}{1.341914in}}%
\pgfpathlineto{\pgfqpoint{1.860211in}{1.342358in}}%
\pgfpathlineto{\pgfqpoint{1.861284in}{1.339282in}}%
\pgfpathlineto{\pgfqpoint{1.862358in}{1.338152in}}%
\pgfpathlineto{\pgfqpoint{1.863432in}{1.338021in}}%
\pgfpathlineto{\pgfqpoint{1.864506in}{1.338455in}}%
\pgfpathlineto{\pgfqpoint{1.867727in}{1.338031in}}%
\pgfpathlineto{\pgfqpoint{1.869875in}{1.340008in}}%
\pgfpathlineto{\pgfqpoint{1.870949in}{1.339433in}}%
\pgfpathlineto{\pgfqpoint{1.872023in}{1.340724in}}%
\pgfpathlineto{\pgfqpoint{1.879540in}{1.345857in}}%
\pgfpathlineto{\pgfqpoint{1.882761in}{1.345908in}}%
\pgfpathlineto{\pgfqpoint{1.883835in}{1.344859in}}%
\pgfpathlineto{\pgfqpoint{1.884909in}{1.342922in}}%
\pgfpathlineto{\pgfqpoint{1.885983in}{1.343850in}}%
\pgfpathlineto{\pgfqpoint{1.890278in}{1.342812in}}%
\pgfpathlineto{\pgfqpoint{1.891352in}{1.346089in}}%
\pgfpathlineto{\pgfqpoint{1.893500in}{1.342357in}}%
\pgfpathlineto{\pgfqpoint{1.894573in}{1.341653in}}%
\pgfpathlineto{\pgfqpoint{1.897795in}{1.341981in}}%
\pgfpathlineto{\pgfqpoint{1.898869in}{1.343100in}}%
\pgfpathlineto{\pgfqpoint{1.899943in}{1.342723in}}%
\pgfpathlineto{\pgfqpoint{1.902090in}{1.343501in}}%
\pgfpathlineto{\pgfqpoint{1.909607in}{1.341472in}}%
\pgfpathlineto{\pgfqpoint{1.914976in}{1.343057in}}%
\pgfpathlineto{\pgfqpoint{1.916050in}{1.342229in}}%
\pgfpathlineto{\pgfqpoint{1.917124in}{1.344546in}}%
\pgfpathlineto{\pgfqpoint{1.920346in}{1.343336in}}%
\pgfpathlineto{\pgfqpoint{1.922493in}{1.343439in}}%
\pgfpathlineto{\pgfqpoint{1.924641in}{1.343001in}}%
\pgfpathlineto{\pgfqpoint{1.927862in}{1.343105in}}%
\pgfpathlineto{\pgfqpoint{1.930010in}{1.343813in}}%
\pgfpathlineto{\pgfqpoint{1.935379in}{1.344894in}}%
\pgfpathlineto{\pgfqpoint{1.936453in}{1.344314in}}%
\pgfpathlineto{\pgfqpoint{1.937527in}{1.345595in}}%
\pgfpathlineto{\pgfqpoint{1.939675in}{1.344155in}}%
\pgfpathlineto{\pgfqpoint{1.942896in}{1.343998in}}%
\pgfpathlineto{\pgfqpoint{1.943970in}{1.342506in}}%
\pgfpathlineto{\pgfqpoint{1.946118in}{1.342184in}}%
\pgfpathlineto{\pgfqpoint{1.947192in}{1.341667in}}%
\pgfpathlineto{\pgfqpoint{1.951487in}{1.341638in}}%
\pgfpathlineto{\pgfqpoint{1.954708in}{1.337582in}}%
\pgfpathlineto{\pgfqpoint{1.957930in}{1.338668in}}%
\pgfpathlineto{\pgfqpoint{1.959004in}{1.339746in}}%
\pgfpathlineto{\pgfqpoint{1.960078in}{1.338518in}}%
\pgfpathlineto{\pgfqpoint{1.965447in}{1.338691in}}%
\pgfpathlineto{\pgfqpoint{1.966521in}{1.337705in}}%
\pgfpathlineto{\pgfqpoint{1.972964in}{1.339110in}}%
\pgfpathlineto{\pgfqpoint{1.974038in}{1.338981in}}%
\pgfpathlineto{\pgfqpoint{1.976185in}{1.335762in}}%
\pgfpathlineto{\pgfqpoint{1.977259in}{1.334502in}}%
\pgfpathlineto{\pgfqpoint{1.981554in}{1.334584in}}%
\pgfpathlineto{\pgfqpoint{1.982628in}{1.333186in}}%
\pgfpathlineto{\pgfqpoint{1.983702in}{1.333333in}}%
\pgfpathlineto{\pgfqpoint{1.984776in}{1.330491in}}%
\pgfpathlineto{\pgfqpoint{1.991219in}{1.330436in}}%
\pgfpathlineto{\pgfqpoint{1.992293in}{1.331098in}}%
\pgfpathlineto{\pgfqpoint{1.995514in}{1.329875in}}%
\pgfpathlineto{\pgfqpoint{1.999810in}{1.332220in}}%
\pgfpathlineto{\pgfqpoint{2.003031in}{1.332110in}}%
\pgfpathlineto{\pgfqpoint{2.004105in}{1.330316in}}%
\pgfpathlineto{\pgfqpoint{2.005179in}{1.330767in}}%
\pgfpathlineto{\pgfqpoint{2.006253in}{1.332588in}}%
\pgfpathlineto{\pgfqpoint{2.012696in}{1.331613in}}%
\pgfpathlineto{\pgfqpoint{2.013770in}{1.329673in}}%
\pgfpathlineto{\pgfqpoint{2.014843in}{1.330758in}}%
\pgfpathlineto{\pgfqpoint{2.019139in}{1.331209in}}%
\pgfpathlineto{\pgfqpoint{2.020213in}{1.333140in}}%
\pgfpathlineto{\pgfqpoint{2.022360in}{1.333232in}}%
\pgfpathlineto{\pgfqpoint{2.025582in}{1.336737in}}%
\pgfpathlineto{\pgfqpoint{2.026656in}{1.336084in}}%
\pgfpathlineto{\pgfqpoint{2.027729in}{1.337794in}}%
\pgfpathlineto{\pgfqpoint{2.028803in}{1.334014in}}%
\pgfpathlineto{\pgfqpoint{2.033099in}{1.335141in}}%
\pgfpathlineto{\pgfqpoint{2.034172in}{1.336390in}}%
\pgfpathlineto{\pgfqpoint{2.037394in}{1.334809in}}%
\pgfpathlineto{\pgfqpoint{2.049206in}{1.335361in}}%
\pgfpathlineto{\pgfqpoint{2.052428in}{1.333194in}}%
\pgfpathlineto{\pgfqpoint{2.072831in}{1.332189in}}%
\pgfpathlineto{\pgfqpoint{2.074978in}{1.331499in}}%
\pgfpathlineto{\pgfqpoint{2.079274in}{1.330242in}}%
\pgfpathlineto{\pgfqpoint{2.080348in}{1.329177in}}%
\pgfpathlineto{\pgfqpoint{2.095381in}{1.328307in}}%
\pgfpathlineto{\pgfqpoint{2.096455in}{1.329402in}}%
\pgfpathlineto{\pgfqpoint{2.100750in}{1.329306in}}%
\pgfpathlineto{\pgfqpoint{2.102898in}{1.331862in}}%
\pgfpathlineto{\pgfqpoint{2.103972in}{1.332139in}}%
\pgfpathlineto{\pgfqpoint{2.105046in}{1.331525in}}%
\pgfpathlineto{\pgfqpoint{2.108267in}{1.332270in}}%
\pgfpathlineto{\pgfqpoint{2.109341in}{1.330741in}}%
\pgfpathlineto{\pgfqpoint{2.112563in}{1.332098in}}%
\pgfpathlineto{\pgfqpoint{2.115784in}{1.330958in}}%
\pgfpathlineto{\pgfqpoint{2.116858in}{1.332210in}}%
\pgfpathlineto{\pgfqpoint{2.123301in}{1.332572in}}%
\pgfpathlineto{\pgfqpoint{2.125449in}{1.331520in}}%
\pgfpathlineto{\pgfqpoint{2.142630in}{1.330301in}}%
\pgfpathlineto{\pgfqpoint{2.149073in}{1.330912in}}%
\pgfpathlineto{\pgfqpoint{2.156590in}{1.330048in}}%
\pgfpathlineto{\pgfqpoint{2.157664in}{1.332454in}}%
\pgfpathlineto{\pgfqpoint{2.160885in}{1.333713in}}%
\pgfpathlineto{\pgfqpoint{2.164107in}{1.329726in}}%
\pgfpathlineto{\pgfqpoint{2.165181in}{1.329598in}}%
\pgfpathlineto{\pgfqpoint{2.169476in}{1.331062in}}%
\pgfpathlineto{\pgfqpoint{2.172698in}{1.328906in}}%
\pgfpathlineto{\pgfqpoint{2.176993in}{1.328337in}}%
\pgfpathlineto{\pgfqpoint{2.180214in}{1.327670in}}%
\pgfpathlineto{\pgfqpoint{2.190953in}{1.328227in}}%
\pgfpathlineto{\pgfqpoint{2.192027in}{1.326070in}}%
\pgfpathlineto{\pgfqpoint{2.201691in}{1.327031in}}%
\pgfpathlineto{\pgfqpoint{2.202765in}{1.326141in}}%
\pgfpathlineto{\pgfqpoint{2.208134in}{1.325903in}}%
\pgfpathlineto{\pgfqpoint{2.209208in}{1.325244in}}%
\pgfpathlineto{\pgfqpoint{2.210282in}{1.325435in}}%
\pgfpathlineto{\pgfqpoint{2.216725in}{1.324738in}}%
\pgfpathlineto{\pgfqpoint{2.239276in}{1.326960in}}%
\pgfpathlineto{\pgfqpoint{2.240349in}{1.328953in}}%
\pgfpathlineto{\pgfqpoint{2.243571in}{1.328025in}}%
\pgfpathlineto{\pgfqpoint{2.245719in}{1.329624in}}%
\pgfpathlineto{\pgfqpoint{2.246792in}{1.328889in}}%
\pgfpathlineto{\pgfqpoint{2.247866in}{1.330682in}}%
\pgfpathlineto{\pgfqpoint{2.252162in}{1.330580in}}%
\pgfpathlineto{\pgfqpoint{2.254309in}{1.328616in}}%
\pgfpathlineto{\pgfqpoint{2.255383in}{1.328916in}}%
\pgfpathlineto{\pgfqpoint{2.260752in}{1.329062in}}%
\pgfpathlineto{\pgfqpoint{2.261826in}{1.329949in}}%
\pgfpathlineto{\pgfqpoint{2.266122in}{1.328985in}}%
\pgfpathlineto{\pgfqpoint{2.267195in}{1.329672in}}%
\pgfpathlineto{\pgfqpoint{2.268269in}{1.329100in}}%
\pgfpathlineto{\pgfqpoint{2.269343in}{1.329218in}}%
\pgfpathlineto{\pgfqpoint{2.270417in}{1.330258in}}%
\pgfpathlineto{\pgfqpoint{2.273638in}{1.330687in}}%
\pgfpathlineto{\pgfqpoint{2.274712in}{1.331562in}}%
\pgfpathlineto{\pgfqpoint{2.275786in}{1.331443in}}%
\pgfpathlineto{\pgfqpoint{2.277934in}{1.330538in}}%
\pgfpathlineto{\pgfqpoint{2.285451in}{1.331619in}}%
\pgfpathlineto{\pgfqpoint{2.288672in}{1.330991in}}%
\pgfpathlineto{\pgfqpoint{2.289746in}{1.329648in}}%
\pgfpathlineto{\pgfqpoint{2.291894in}{1.330555in}}%
\pgfpathlineto{\pgfqpoint{2.292968in}{1.329281in}}%
\pgfpathlineto{\pgfqpoint{2.297263in}{1.329161in}}%
\pgfpathlineto{\pgfqpoint{2.300484in}{1.329636in}}%
\pgfpathlineto{\pgfqpoint{2.305854in}{1.327237in}}%
\pgfpathlineto{\pgfqpoint{2.306927in}{1.324874in}}%
\pgfpathlineto{\pgfqpoint{2.308001in}{1.324598in}}%
\pgfpathlineto{\pgfqpoint{2.315518in}{1.325644in}}%
\pgfpathlineto{\pgfqpoint{2.319814in}{1.325368in}}%
\pgfpathlineto{\pgfqpoint{2.320887in}{1.324635in}}%
\pgfpathlineto{\pgfqpoint{2.323035in}{1.324225in}}%
\pgfpathlineto{\pgfqpoint{2.334847in}{1.325037in}}%
\pgfpathlineto{\pgfqpoint{2.335921in}{1.323638in}}%
\pgfpathlineto{\pgfqpoint{2.336995in}{1.324385in}}%
\pgfpathlineto{\pgfqpoint{2.338069in}{1.323556in}}%
\pgfpathlineto{\pgfqpoint{2.342364in}{1.323246in}}%
\pgfpathlineto{\pgfqpoint{2.344512in}{1.324541in}}%
\pgfpathlineto{\pgfqpoint{2.345586in}{1.324300in}}%
\pgfpathlineto{\pgfqpoint{2.349881in}{1.323003in}}%
\pgfpathlineto{\pgfqpoint{2.353102in}{1.323003in}}%
\pgfpathlineto{\pgfqpoint{2.359546in}{1.323061in}}%
\pgfpathlineto{\pgfqpoint{2.360619in}{1.323602in}}%
\pgfpathlineto{\pgfqpoint{2.368136in}{1.321890in}}%
\pgfpathlineto{\pgfqpoint{2.375653in}{1.323224in}}%
\pgfpathlineto{\pgfqpoint{2.382096in}{1.322924in}}%
\pgfpathlineto{\pgfqpoint{2.389613in}{1.323128in}}%
\pgfpathlineto{\pgfqpoint{2.390687in}{1.323515in}}%
\pgfpathlineto{\pgfqpoint{2.397130in}{1.324299in}}%
\pgfpathlineto{\pgfqpoint{2.398204in}{1.323516in}}%
\pgfpathlineto{\pgfqpoint{2.403573in}{1.323065in}}%
\pgfpathlineto{\pgfqpoint{2.404647in}{1.323246in}}%
\pgfpathlineto{\pgfqpoint{2.405721in}{1.322682in}}%
\pgfpathlineto{\pgfqpoint{2.413237in}{1.321664in}}%
\pgfpathlineto{\pgfqpoint{2.427197in}{1.321346in}}%
\pgfpathlineto{\pgfqpoint{2.431493in}{1.321922in}}%
\pgfpathlineto{\pgfqpoint{2.432567in}{1.321572in}}%
\pgfpathlineto{\pgfqpoint{2.434714in}{1.321768in}}%
\pgfpathlineto{\pgfqpoint{2.439010in}{1.321553in}}%
\pgfpathlineto{\pgfqpoint{2.440083in}{1.321820in}}%
\pgfpathlineto{\pgfqpoint{2.441157in}{1.321112in}}%
\pgfpathlineto{\pgfqpoint{2.442231in}{1.321523in}}%
\pgfpathlineto{\pgfqpoint{2.443305in}{1.320899in}}%
\pgfpathlineto{\pgfqpoint{2.446526in}{1.320864in}}%
\pgfpathlineto{\pgfqpoint{2.448674in}{1.321740in}}%
\pgfpathlineto{\pgfqpoint{2.454043in}{1.321794in}}%
\pgfpathlineto{\pgfqpoint{2.455117in}{1.321278in}}%
\pgfpathlineto{\pgfqpoint{2.456191in}{1.321656in}}%
\pgfpathlineto{\pgfqpoint{2.458339in}{1.321572in}}%
\pgfpathlineto{\pgfqpoint{2.461560in}{1.321732in}}%
\pgfpathlineto{\pgfqpoint{2.462634in}{1.321134in}}%
\pgfpathlineto{\pgfqpoint{2.464782in}{1.321451in}}%
\pgfpathlineto{\pgfqpoint{2.465856in}{1.321131in}}%
\pgfpathlineto{\pgfqpoint{2.478742in}{1.321152in}}%
\pgfpathlineto{\pgfqpoint{2.480889in}{1.320016in}}%
\pgfpathlineto{\pgfqpoint{2.495923in}{1.316763in}}%
\pgfpathlineto{\pgfqpoint{2.509883in}{1.317396in}}%
\pgfpathlineto{\pgfqpoint{2.510957in}{1.316513in}}%
\pgfpathlineto{\pgfqpoint{2.518474in}{1.316184in}}%
\pgfpathlineto{\pgfqpoint{2.524917in}{1.316207in}}%
\pgfpathlineto{\pgfqpoint{2.525991in}{1.316052in}}%
\pgfpathlineto{\pgfqpoint{2.533507in}{1.316750in}}%
\pgfpathlineto{\pgfqpoint{2.538877in}{1.317050in}}%
\pgfpathlineto{\pgfqpoint{2.541024in}{1.316847in}}%
\pgfpathlineto{\pgfqpoint{2.547467in}{1.316002in}}%
\pgfpathlineto{\pgfqpoint{2.581830in}{1.315528in}}%
\pgfpathlineto{\pgfqpoint{2.583978in}{1.317211in}}%
\pgfpathlineto{\pgfqpoint{2.586125in}{1.317541in}}%
\pgfpathlineto{\pgfqpoint{2.590421in}{1.317535in}}%
\pgfpathlineto{\pgfqpoint{2.592568in}{1.316440in}}%
\pgfpathlineto{\pgfqpoint{2.593642in}{1.316253in}}%
\pgfpathlineto{\pgfqpoint{2.600085in}{1.318552in}}%
\pgfpathlineto{\pgfqpoint{2.604381in}{1.318352in}}%
\pgfpathlineto{\pgfqpoint{2.605455in}{1.318960in}}%
\pgfpathlineto{\pgfqpoint{2.606528in}{1.317521in}}%
\pgfpathlineto{\pgfqpoint{2.608676in}{1.318793in}}%
\pgfpathlineto{\pgfqpoint{2.612971in}{1.318683in}}%
\pgfpathlineto{\pgfqpoint{2.614045in}{1.317986in}}%
\pgfpathlineto{\pgfqpoint{2.615119in}{1.318864in}}%
\pgfpathlineto{\pgfqpoint{2.619414in}{1.318847in}}%
\pgfpathlineto{\pgfqpoint{2.620488in}{1.317083in}}%
\pgfpathlineto{\pgfqpoint{2.622636in}{1.316582in}}%
\pgfpathlineto{\pgfqpoint{2.629079in}{1.321711in}}%
\pgfpathlineto{\pgfqpoint{2.631227in}{1.321745in}}%
\pgfpathlineto{\pgfqpoint{2.636596in}{1.321324in}}%
\pgfpathlineto{\pgfqpoint{2.637670in}{1.319695in}}%
\pgfpathlineto{\pgfqpoint{2.643039in}{1.319414in}}%
\pgfpathlineto{\pgfqpoint{2.646260in}{1.318690in}}%
\pgfpathlineto{\pgfqpoint{2.650556in}{1.318841in}}%
\pgfpathlineto{\pgfqpoint{2.651630in}{1.317546in}}%
\pgfpathlineto{\pgfqpoint{2.652703in}{1.318222in}}%
\pgfpathlineto{\pgfqpoint{2.665590in}{1.317054in}}%
\pgfpathlineto{\pgfqpoint{2.666663in}{1.317236in}}%
\pgfpathlineto{\pgfqpoint{2.668811in}{1.316754in}}%
\pgfpathlineto{\pgfqpoint{2.675254in}{1.316375in}}%
\pgfpathlineto{\pgfqpoint{2.676328in}{1.315675in}}%
\pgfpathlineto{\pgfqpoint{2.680623in}{1.316240in}}%
\pgfpathlineto{\pgfqpoint{2.681697in}{1.316603in}}%
\pgfpathlineto{\pgfqpoint{2.682771in}{1.316143in}}%
\pgfpathlineto{\pgfqpoint{2.683845in}{1.316556in}}%
\pgfpathlineto{\pgfqpoint{2.691362in}{1.315586in}}%
\pgfpathlineto{\pgfqpoint{2.696731in}{1.315955in}}%
\pgfpathlineto{\pgfqpoint{2.697805in}{1.316871in}}%
\pgfpathlineto{\pgfqpoint{2.702100in}{1.317514in}}%
\pgfpathlineto{\pgfqpoint{2.703174in}{1.316643in}}%
\pgfpathlineto{\pgfqpoint{2.704248in}{1.317242in}}%
\pgfpathlineto{\pgfqpoint{2.705322in}{1.316730in}}%
\pgfpathlineto{\pgfqpoint{2.706395in}{1.317403in}}%
\pgfpathlineto{\pgfqpoint{2.713912in}{1.317212in}}%
\pgfpathlineto{\pgfqpoint{2.719281in}{1.316583in}}%
\pgfpathlineto{\pgfqpoint{2.720355in}{1.314966in}}%
\pgfpathlineto{\pgfqpoint{2.721429in}{1.315586in}}%
\pgfpathlineto{\pgfqpoint{2.725724in}{1.315506in}}%
\pgfpathlineto{\pgfqpoint{2.728946in}{1.314323in}}%
\pgfpathlineto{\pgfqpoint{2.735389in}{1.313891in}}%
\pgfpathlineto{\pgfqpoint{2.736463in}{1.312809in}}%
\pgfpathlineto{\pgfqpoint{2.751497in}{1.312211in}}%
\pgfpathlineto{\pgfqpoint{2.751497in}{1.312211in}}%
\pgfusepath{stroke}%
\end{pgfscope}%
\begin{pgfscope}%
\pgfsetrectcap%
\pgfsetmiterjoin%
\pgfsetlinewidth{0.803000pt}%
\definecolor{currentstroke}{rgb}{1.000000,1.000000,1.000000}%
\pgfsetstrokecolor{currentstroke}%
\pgfsetdash{}{0pt}%
\pgfpathmoveto{\pgfqpoint{0.285588in}{1.293759in}}%
\pgfpathlineto{\pgfqpoint{0.285588in}{1.694644in}}%
\pgfusepath{stroke}%
\end{pgfscope}%
\begin{pgfscope}%
\pgfsetrectcap%
\pgfsetmiterjoin%
\pgfsetlinewidth{0.803000pt}%
\definecolor{currentstroke}{rgb}{1.000000,1.000000,1.000000}%
\pgfsetstrokecolor{currentstroke}%
\pgfsetdash{}{0pt}%
\pgfpathmoveto{\pgfqpoint{2.868921in}{1.293759in}}%
\pgfpathlineto{\pgfqpoint{2.868921in}{1.694644in}}%
\pgfusepath{stroke}%
\end{pgfscope}%
\begin{pgfscope}%
\pgfsetrectcap%
\pgfsetmiterjoin%
\pgfsetlinewidth{0.803000pt}%
\definecolor{currentstroke}{rgb}{1.000000,1.000000,1.000000}%
\pgfsetstrokecolor{currentstroke}%
\pgfsetdash{}{0pt}%
\pgfpathmoveto{\pgfqpoint{0.285587in}{1.293759in}}%
\pgfpathlineto{\pgfqpoint{2.868921in}{1.293759in}}%
\pgfusepath{stroke}%
\end{pgfscope}%
\begin{pgfscope}%
\pgfsetrectcap%
\pgfsetmiterjoin%
\pgfsetlinewidth{0.803000pt}%
\definecolor{currentstroke}{rgb}{1.000000,1.000000,1.000000}%
\pgfsetstrokecolor{currentstroke}%
\pgfsetdash{}{0pt}%
\pgfpathmoveto{\pgfqpoint{0.285587in}{1.694644in}}%
\pgfpathlineto{\pgfqpoint{2.868921in}{1.694644in}}%
\pgfusepath{stroke}%
\end{pgfscope}%
\begin{pgfscope}%
\definecolor{textcolor}{rgb}{0.150000,0.150000,0.150000}%
\pgfsetstrokecolor{textcolor}%
\pgfsetfillcolor{textcolor}%
\pgftext[x=1.577254in,y=1.777977in,,base]{\color{textcolor}\rmfamily\fontsize{12.000000}{14.400000}\selectfont UTX}%
\end{pgfscope}%
\begin{pgfscope}%
\pgfsetbuttcap%
\pgfsetmiterjoin%
\definecolor{currentfill}{rgb}{0.917647,0.917647,0.949020}%
\pgfsetfillcolor{currentfill}%
\pgfsetlinewidth{0.000000pt}%
\definecolor{currentstroke}{rgb}{0.000000,0.000000,0.000000}%
\pgfsetstrokecolor{currentstroke}%
\pgfsetstrokeopacity{0.000000}%
\pgfsetdash{}{0pt}%
\pgfpathmoveto{\pgfqpoint{3.902254in}{1.293759in}}%
\pgfpathlineto{\pgfqpoint{6.485588in}{1.293759in}}%
\pgfpathlineto{\pgfqpoint{6.485588in}{1.694644in}}%
\pgfpathlineto{\pgfqpoint{3.902254in}{1.694644in}}%
\pgfpathclose%
\pgfusepath{fill}%
\end{pgfscope}%
\begin{pgfscope}%
\pgfpathrectangle{\pgfqpoint{3.902254in}{1.293759in}}{\pgfqpoint{2.583333in}{0.400885in}}%
\pgfusepath{clip}%
\pgfsetroundcap%
\pgfsetroundjoin%
\pgfsetlinewidth{0.803000pt}%
\definecolor{currentstroke}{rgb}{1.000000,1.000000,1.000000}%
\pgfsetstrokecolor{currentstroke}%
\pgfsetdash{}{0pt}%
\pgfpathmoveto{\pgfqpoint{4.017531in}{1.293759in}}%
\pgfpathlineto{\pgfqpoint{4.017531in}{1.694644in}}%
\pgfusepath{stroke}%
\end{pgfscope}%
\begin{pgfscope}%
\definecolor{textcolor}{rgb}{0.150000,0.150000,0.150000}%
\pgfsetstrokecolor{textcolor}%
\pgfsetfillcolor{textcolor}%
\pgftext[x=4.017531in,y=1.196537in,,top]{\color{textcolor}\rmfamily\fontsize{10.000000}{12.000000}\selectfont 2012}%
\end{pgfscope}%
\begin{pgfscope}%
\pgfpathrectangle{\pgfqpoint{3.902254in}{1.293759in}}{\pgfqpoint{2.583333in}{0.400885in}}%
\pgfusepath{clip}%
\pgfsetroundcap%
\pgfsetroundjoin%
\pgfsetlinewidth{0.803000pt}%
\definecolor{currentstroke}{rgb}{1.000000,1.000000,1.000000}%
\pgfsetstrokecolor{currentstroke}%
\pgfsetdash{}{0pt}%
\pgfpathmoveto{\pgfqpoint{4.410556in}{1.293759in}}%
\pgfpathlineto{\pgfqpoint{4.410556in}{1.694644in}}%
\pgfusepath{stroke}%
\end{pgfscope}%
\begin{pgfscope}%
\definecolor{textcolor}{rgb}{0.150000,0.150000,0.150000}%
\pgfsetstrokecolor{textcolor}%
\pgfsetfillcolor{textcolor}%
\pgftext[x=4.410556in,y=1.196537in,,top]{\color{textcolor}\rmfamily\fontsize{10.000000}{12.000000}\selectfont 2013}%
\end{pgfscope}%
\begin{pgfscope}%
\pgfpathrectangle{\pgfqpoint{3.902254in}{1.293759in}}{\pgfqpoint{2.583333in}{0.400885in}}%
\pgfusepath{clip}%
\pgfsetroundcap%
\pgfsetroundjoin%
\pgfsetlinewidth{0.803000pt}%
\definecolor{currentstroke}{rgb}{1.000000,1.000000,1.000000}%
\pgfsetstrokecolor{currentstroke}%
\pgfsetdash{}{0pt}%
\pgfpathmoveto{\pgfqpoint{4.802507in}{1.293759in}}%
\pgfpathlineto{\pgfqpoint{4.802507in}{1.694644in}}%
\pgfusepath{stroke}%
\end{pgfscope}%
\begin{pgfscope}%
\definecolor{textcolor}{rgb}{0.150000,0.150000,0.150000}%
\pgfsetstrokecolor{textcolor}%
\pgfsetfillcolor{textcolor}%
\pgftext[x=4.802507in,y=1.196537in,,top]{\color{textcolor}\rmfamily\fontsize{10.000000}{12.000000}\selectfont 2014}%
\end{pgfscope}%
\begin{pgfscope}%
\pgfpathrectangle{\pgfqpoint{3.902254in}{1.293759in}}{\pgfqpoint{2.583333in}{0.400885in}}%
\pgfusepath{clip}%
\pgfsetroundcap%
\pgfsetroundjoin%
\pgfsetlinewidth{0.803000pt}%
\definecolor{currentstroke}{rgb}{1.000000,1.000000,1.000000}%
\pgfsetstrokecolor{currentstroke}%
\pgfsetdash{}{0pt}%
\pgfpathmoveto{\pgfqpoint{5.194458in}{1.293759in}}%
\pgfpathlineto{\pgfqpoint{5.194458in}{1.694644in}}%
\pgfusepath{stroke}%
\end{pgfscope}%
\begin{pgfscope}%
\definecolor{textcolor}{rgb}{0.150000,0.150000,0.150000}%
\pgfsetstrokecolor{textcolor}%
\pgfsetfillcolor{textcolor}%
\pgftext[x=5.194458in,y=1.196537in,,top]{\color{textcolor}\rmfamily\fontsize{10.000000}{12.000000}\selectfont 2015}%
\end{pgfscope}%
\begin{pgfscope}%
\pgfpathrectangle{\pgfqpoint{3.902254in}{1.293759in}}{\pgfqpoint{2.583333in}{0.400885in}}%
\pgfusepath{clip}%
\pgfsetroundcap%
\pgfsetroundjoin%
\pgfsetlinewidth{0.803000pt}%
\definecolor{currentstroke}{rgb}{1.000000,1.000000,1.000000}%
\pgfsetstrokecolor{currentstroke}%
\pgfsetdash{}{0pt}%
\pgfpathmoveto{\pgfqpoint{5.586409in}{1.293759in}}%
\pgfpathlineto{\pgfqpoint{5.586409in}{1.694644in}}%
\pgfusepath{stroke}%
\end{pgfscope}%
\begin{pgfscope}%
\definecolor{textcolor}{rgb}{0.150000,0.150000,0.150000}%
\pgfsetstrokecolor{textcolor}%
\pgfsetfillcolor{textcolor}%
\pgftext[x=5.586409in,y=1.196537in,,top]{\color{textcolor}\rmfamily\fontsize{10.000000}{12.000000}\selectfont 2016}%
\end{pgfscope}%
\begin{pgfscope}%
\pgfpathrectangle{\pgfqpoint{3.902254in}{1.293759in}}{\pgfqpoint{2.583333in}{0.400885in}}%
\pgfusepath{clip}%
\pgfsetroundcap%
\pgfsetroundjoin%
\pgfsetlinewidth{0.803000pt}%
\definecolor{currentstroke}{rgb}{1.000000,1.000000,1.000000}%
\pgfsetstrokecolor{currentstroke}%
\pgfsetdash{}{0pt}%
\pgfpathmoveto{\pgfqpoint{5.979434in}{1.293759in}}%
\pgfpathlineto{\pgfqpoint{5.979434in}{1.694644in}}%
\pgfusepath{stroke}%
\end{pgfscope}%
\begin{pgfscope}%
\definecolor{textcolor}{rgb}{0.150000,0.150000,0.150000}%
\pgfsetstrokecolor{textcolor}%
\pgfsetfillcolor{textcolor}%
\pgftext[x=5.979434in,y=1.196537in,,top]{\color{textcolor}\rmfamily\fontsize{10.000000}{12.000000}\selectfont 2017}%
\end{pgfscope}%
\begin{pgfscope}%
\pgfpathrectangle{\pgfqpoint{3.902254in}{1.293759in}}{\pgfqpoint{2.583333in}{0.400885in}}%
\pgfusepath{clip}%
\pgfsetroundcap%
\pgfsetroundjoin%
\pgfsetlinewidth{0.803000pt}%
\definecolor{currentstroke}{rgb}{1.000000,1.000000,1.000000}%
\pgfsetstrokecolor{currentstroke}%
\pgfsetdash{}{0pt}%
\pgfpathmoveto{\pgfqpoint{6.371385in}{1.293759in}}%
\pgfpathlineto{\pgfqpoint{6.371385in}{1.694644in}}%
\pgfusepath{stroke}%
\end{pgfscope}%
\begin{pgfscope}%
\definecolor{textcolor}{rgb}{0.150000,0.150000,0.150000}%
\pgfsetstrokecolor{textcolor}%
\pgfsetfillcolor{textcolor}%
\pgftext[x=6.371385in,y=1.196537in,,top]{\color{textcolor}\rmfamily\fontsize{10.000000}{12.000000}\selectfont 2018}%
\end{pgfscope}%
\begin{pgfscope}%
\pgfpathrectangle{\pgfqpoint{3.902254in}{1.293759in}}{\pgfqpoint{2.583333in}{0.400885in}}%
\pgfusepath{clip}%
\pgfsetroundcap%
\pgfsetroundjoin%
\pgfsetlinewidth{0.803000pt}%
\definecolor{currentstroke}{rgb}{1.000000,1.000000,1.000000}%
\pgfsetstrokecolor{currentstroke}%
\pgfsetdash{}{0pt}%
\pgfpathmoveto{\pgfqpoint{3.902254in}{1.334573in}}%
\pgfpathlineto{\pgfqpoint{6.485588in}{1.334573in}}%
\pgfusepath{stroke}%
\end{pgfscope}%
\begin{pgfscope}%
\definecolor{textcolor}{rgb}{0.150000,0.150000,0.150000}%
\pgfsetstrokecolor{textcolor}%
\pgfsetfillcolor{textcolor}%
\pgftext[x=3.584153in,y=1.281812in,left,base]{\color{textcolor}\rmfamily\fontsize{10.000000}{12.000000}\selectfont 0.5}%
\end{pgfscope}%
\begin{pgfscope}%
\pgfpathrectangle{\pgfqpoint{3.902254in}{1.293759in}}{\pgfqpoint{2.583333in}{0.400885in}}%
\pgfusepath{clip}%
\pgfsetroundcap%
\pgfsetroundjoin%
\pgfsetlinewidth{0.803000pt}%
\definecolor{currentstroke}{rgb}{1.000000,1.000000,1.000000}%
\pgfsetstrokecolor{currentstroke}%
\pgfsetdash{}{0pt}%
\pgfpathmoveto{\pgfqpoint{3.902254in}{1.469174in}}%
\pgfpathlineto{\pgfqpoint{6.485588in}{1.469174in}}%
\pgfusepath{stroke}%
\end{pgfscope}%
\begin{pgfscope}%
\definecolor{textcolor}{rgb}{0.150000,0.150000,0.150000}%
\pgfsetstrokecolor{textcolor}%
\pgfsetfillcolor{textcolor}%
\pgftext[x=3.584153in,y=1.416412in,left,base]{\color{textcolor}\rmfamily\fontsize{10.000000}{12.000000}\selectfont 1.0}%
\end{pgfscope}%
\begin{pgfscope}%
\pgfpathrectangle{\pgfqpoint{3.902254in}{1.293759in}}{\pgfqpoint{2.583333in}{0.400885in}}%
\pgfusepath{clip}%
\pgfsetroundcap%
\pgfsetroundjoin%
\pgfsetlinewidth{0.803000pt}%
\definecolor{currentstroke}{rgb}{1.000000,1.000000,1.000000}%
\pgfsetstrokecolor{currentstroke}%
\pgfsetdash{}{0pt}%
\pgfpathmoveto{\pgfqpoint{3.902254in}{1.603774in}}%
\pgfpathlineto{\pgfqpoint{6.485588in}{1.603774in}}%
\pgfusepath{stroke}%
\end{pgfscope}%
\begin{pgfscope}%
\definecolor{textcolor}{rgb}{0.150000,0.150000,0.150000}%
\pgfsetstrokecolor{textcolor}%
\pgfsetfillcolor{textcolor}%
\pgftext[x=3.584153in,y=1.551013in,left,base]{\color{textcolor}\rmfamily\fontsize{10.000000}{12.000000}\selectfont 1.5}%
\end{pgfscope}%
\begin{pgfscope}%
\pgfpathrectangle{\pgfqpoint{3.902254in}{1.293759in}}{\pgfqpoint{2.583333in}{0.400885in}}%
\pgfusepath{clip}%
\pgfsetroundcap%
\pgfsetroundjoin%
\pgfsetlinewidth{1.505625pt}%
\definecolor{currentstroke}{rgb}{0.498039,0.498039,0.498039}%
\pgfsetstrokecolor{currentstroke}%
\pgfsetstrokeopacity{0.300000}%
\pgfsetdash{}{0pt}%
\pgfpathmoveto{\pgfqpoint{4.019678in}{1.469174in}}%
\pgfpathlineto{\pgfqpoint{4.021826in}{1.463875in}}%
\pgfpathlineto{\pgfqpoint{4.022900in}{1.463104in}}%
\pgfpathlineto{\pgfqpoint{4.026121in}{1.463393in}}%
\pgfpathlineto{\pgfqpoint{4.027195in}{1.464742in}}%
\pgfpathlineto{\pgfqpoint{4.028269in}{1.467054in}}%
\pgfpathlineto{\pgfqpoint{4.037934in}{1.467536in}}%
\pgfpathlineto{\pgfqpoint{4.041155in}{1.463586in}}%
\pgfpathlineto{\pgfqpoint{4.042229in}{1.459443in}}%
\pgfpathlineto{\pgfqpoint{4.043303in}{1.458672in}}%
\pgfpathlineto{\pgfqpoint{4.044377in}{1.456263in}}%
\pgfpathlineto{\pgfqpoint{4.045451in}{1.455396in}}%
\pgfpathlineto{\pgfqpoint{4.050820in}{1.459443in}}%
\pgfpathlineto{\pgfqpoint{4.051894in}{1.457805in}}%
\pgfpathlineto{\pgfqpoint{4.052967in}{1.459732in}}%
\pgfpathlineto{\pgfqpoint{4.056189in}{1.461755in}}%
\pgfpathlineto{\pgfqpoint{4.057263in}{1.460310in}}%
\pgfpathlineto{\pgfqpoint{4.059410in}{1.460310in}}%
\pgfpathlineto{\pgfqpoint{4.060484in}{1.458672in}}%
\pgfpathlineto{\pgfqpoint{4.063706in}{1.461755in}}%
\pgfpathlineto{\pgfqpoint{4.064780in}{1.461080in}}%
\pgfpathlineto{\pgfqpoint{4.065853in}{1.459635in}}%
\pgfpathlineto{\pgfqpoint{4.066927in}{1.461177in}}%
\pgfpathlineto{\pgfqpoint{4.068001in}{1.463971in}}%
\pgfpathlineto{\pgfqpoint{4.072296in}{1.464164in}}%
\pgfpathlineto{\pgfqpoint{4.073370in}{1.462237in}}%
\pgfpathlineto{\pgfqpoint{4.075518in}{1.461755in}}%
\pgfpathlineto{\pgfqpoint{4.080887in}{1.461562in}}%
\pgfpathlineto{\pgfqpoint{4.083035in}{1.465416in}}%
\pgfpathlineto{\pgfqpoint{4.086256in}{1.467729in}}%
\pgfpathlineto{\pgfqpoint{4.087330in}{1.465609in}}%
\pgfpathlineto{\pgfqpoint{4.088404in}{1.466861in}}%
\pgfpathlineto{\pgfqpoint{4.089478in}{1.469174in}}%
\pgfpathlineto{\pgfqpoint{4.090552in}{1.468403in}}%
\pgfpathlineto{\pgfqpoint{4.098069in}{1.471583in}}%
\pgfpathlineto{\pgfqpoint{4.104512in}{1.472257in}}%
\pgfpathlineto{\pgfqpoint{4.105585in}{1.470619in}}%
\pgfpathlineto{\pgfqpoint{4.108807in}{1.469945in}}%
\pgfpathlineto{\pgfqpoint{4.110955in}{1.462622in}}%
\pgfpathlineto{\pgfqpoint{4.112029in}{1.461370in}}%
\pgfpathlineto{\pgfqpoint{4.113102in}{1.462429in}}%
\pgfpathlineto{\pgfqpoint{4.116324in}{1.464453in}}%
\pgfpathlineto{\pgfqpoint{4.118472in}{1.463489in}}%
\pgfpathlineto{\pgfqpoint{4.119545in}{1.461948in}}%
\pgfpathlineto{\pgfqpoint{4.123841in}{1.460502in}}%
\pgfpathlineto{\pgfqpoint{4.124915in}{1.455974in}}%
\pgfpathlineto{\pgfqpoint{4.125988in}{1.459924in}}%
\pgfpathlineto{\pgfqpoint{4.127062in}{1.461177in}}%
\pgfpathlineto{\pgfqpoint{4.128136in}{1.459153in}}%
\pgfpathlineto{\pgfqpoint{4.131358in}{1.460310in}}%
\pgfpathlineto{\pgfqpoint{4.132431in}{1.462429in}}%
\pgfpathlineto{\pgfqpoint{4.133505in}{1.461948in}}%
\pgfpathlineto{\pgfqpoint{4.135653in}{1.469367in}}%
\pgfpathlineto{\pgfqpoint{4.138874in}{1.468210in}}%
\pgfpathlineto{\pgfqpoint{4.139948in}{1.474762in}}%
\pgfpathlineto{\pgfqpoint{4.141022in}{1.474569in}}%
\pgfpathlineto{\pgfqpoint{4.142096in}{1.479194in}}%
\pgfpathlineto{\pgfqpoint{4.143170in}{1.479772in}}%
\pgfpathlineto{\pgfqpoint{4.146391in}{1.480832in}}%
\pgfpathlineto{\pgfqpoint{4.147465in}{1.482085in}}%
\pgfpathlineto{\pgfqpoint{4.149613in}{1.482663in}}%
\pgfpathlineto{\pgfqpoint{4.150687in}{1.479965in}}%
\pgfpathlineto{\pgfqpoint{4.154982in}{1.481988in}}%
\pgfpathlineto{\pgfqpoint{4.156056in}{1.479965in}}%
\pgfpathlineto{\pgfqpoint{4.157130in}{1.481988in}}%
\pgfpathlineto{\pgfqpoint{4.158204in}{1.486228in}}%
\pgfpathlineto{\pgfqpoint{4.161425in}{1.484397in}}%
\pgfpathlineto{\pgfqpoint{4.162499in}{1.485457in}}%
\pgfpathlineto{\pgfqpoint{4.163573in}{1.484301in}}%
\pgfpathlineto{\pgfqpoint{4.164647in}{1.487769in}}%
\pgfpathlineto{\pgfqpoint{4.165720in}{1.488829in}}%
\pgfpathlineto{\pgfqpoint{4.168942in}{1.487480in}}%
\pgfpathlineto{\pgfqpoint{4.170016in}{1.487866in}}%
\pgfpathlineto{\pgfqpoint{4.171090in}{1.487095in}}%
\pgfpathlineto{\pgfqpoint{4.173237in}{1.488251in}}%
\pgfpathlineto{\pgfqpoint{4.177533in}{1.490371in}}%
\pgfpathlineto{\pgfqpoint{4.178607in}{1.488058in}}%
\pgfpathlineto{\pgfqpoint{4.179680in}{1.489600in}}%
\pgfpathlineto{\pgfqpoint{4.180754in}{1.485361in}}%
\pgfpathlineto{\pgfqpoint{4.183976in}{1.487480in}}%
\pgfpathlineto{\pgfqpoint{4.185050in}{1.486517in}}%
\pgfpathlineto{\pgfqpoint{4.186123in}{1.490467in}}%
\pgfpathlineto{\pgfqpoint{4.187197in}{1.489600in}}%
\pgfpathlineto{\pgfqpoint{4.188271in}{1.495188in}}%
\pgfpathlineto{\pgfqpoint{4.191493in}{1.495959in}}%
\pgfpathlineto{\pgfqpoint{4.192566in}{1.498657in}}%
\pgfpathlineto{\pgfqpoint{4.193640in}{1.498946in}}%
\pgfpathlineto{\pgfqpoint{4.194714in}{1.504245in}}%
\pgfpathlineto{\pgfqpoint{4.195788in}{1.502896in}}%
\pgfpathlineto{\pgfqpoint{4.199009in}{1.504727in}}%
\pgfpathlineto{\pgfqpoint{4.200083in}{1.504149in}}%
\pgfpathlineto{\pgfqpoint{4.201157in}{1.501162in}}%
\pgfpathlineto{\pgfqpoint{4.202231in}{1.501355in}}%
\pgfpathlineto{\pgfqpoint{4.203305in}{1.505690in}}%
\pgfpathlineto{\pgfqpoint{4.206526in}{1.503571in}}%
\pgfpathlineto{\pgfqpoint{4.207600in}{1.504920in}}%
\pgfpathlineto{\pgfqpoint{4.208674in}{1.504534in}}%
\pgfpathlineto{\pgfqpoint{4.209748in}{1.505787in}}%
\pgfpathlineto{\pgfqpoint{4.210822in}{1.509063in}}%
\pgfpathlineto{\pgfqpoint{4.214043in}{1.512531in}}%
\pgfpathlineto{\pgfqpoint{4.215117in}{1.512628in}}%
\pgfpathlineto{\pgfqpoint{4.218339in}{1.512435in}}%
\pgfpathlineto{\pgfqpoint{4.221560in}{1.514651in}}%
\pgfpathlineto{\pgfqpoint{4.222634in}{1.514362in}}%
\pgfpathlineto{\pgfqpoint{4.223708in}{1.515807in}}%
\pgfpathlineto{\pgfqpoint{4.224782in}{1.514169in}}%
\pgfpathlineto{\pgfqpoint{4.225855in}{1.518023in}}%
\pgfpathlineto{\pgfqpoint{4.229077in}{1.518408in}}%
\pgfpathlineto{\pgfqpoint{4.231225in}{1.522744in}}%
\pgfpathlineto{\pgfqpoint{4.232298in}{1.513302in}}%
\pgfpathlineto{\pgfqpoint{4.236594in}{1.511953in}}%
\pgfpathlineto{\pgfqpoint{4.237668in}{1.507810in}}%
\pgfpathlineto{\pgfqpoint{4.238741in}{1.507521in}}%
\pgfpathlineto{\pgfqpoint{4.240889in}{1.515807in}}%
\pgfpathlineto{\pgfqpoint{4.244111in}{1.516192in}}%
\pgfpathlineto{\pgfqpoint{4.246258in}{1.518023in}}%
\pgfpathlineto{\pgfqpoint{4.247332in}{1.513784in}}%
\pgfpathlineto{\pgfqpoint{4.248406in}{1.512724in}}%
\pgfpathlineto{\pgfqpoint{4.251628in}{1.514362in}}%
\pgfpathlineto{\pgfqpoint{4.253775in}{1.510701in}}%
\pgfpathlineto{\pgfqpoint{4.255923in}{1.513687in}}%
\pgfpathlineto{\pgfqpoint{4.259144in}{1.511375in}}%
\pgfpathlineto{\pgfqpoint{4.260218in}{1.511760in}}%
\pgfpathlineto{\pgfqpoint{4.262366in}{1.510315in}}%
\pgfpathlineto{\pgfqpoint{4.263440in}{1.509930in}}%
\pgfpathlineto{\pgfqpoint{4.266661in}{1.507328in}}%
\pgfpathlineto{\pgfqpoint{4.267735in}{1.501644in}}%
\pgfpathlineto{\pgfqpoint{4.269883in}{1.497115in}}%
\pgfpathlineto{\pgfqpoint{4.270957in}{1.503667in}}%
\pgfpathlineto{\pgfqpoint{4.275252in}{1.500102in}}%
\pgfpathlineto{\pgfqpoint{4.276326in}{1.503089in}}%
\pgfpathlineto{\pgfqpoint{4.277400in}{1.500777in}}%
\pgfpathlineto{\pgfqpoint{4.283843in}{1.508099in}}%
\pgfpathlineto{\pgfqpoint{4.284917in}{1.510508in}}%
\pgfpathlineto{\pgfqpoint{4.285990in}{1.507521in}}%
\pgfpathlineto{\pgfqpoint{4.290286in}{1.511182in}}%
\pgfpathlineto{\pgfqpoint{4.292433in}{1.520625in}}%
\pgfpathlineto{\pgfqpoint{4.293507in}{1.513206in}}%
\pgfpathlineto{\pgfqpoint{4.296729in}{1.513591in}}%
\pgfpathlineto{\pgfqpoint{4.299950in}{1.519950in}}%
\pgfpathlineto{\pgfqpoint{4.301024in}{1.521010in}}%
\pgfpathlineto{\pgfqpoint{4.304246in}{1.521299in}}%
\pgfpathlineto{\pgfqpoint{4.306393in}{1.520625in}}%
\pgfpathlineto{\pgfqpoint{4.307467in}{1.521877in}}%
\pgfpathlineto{\pgfqpoint{4.308541in}{1.520528in}}%
\pgfpathlineto{\pgfqpoint{4.312836in}{1.522552in}}%
\pgfpathlineto{\pgfqpoint{4.313910in}{1.525538in}}%
\pgfpathlineto{\pgfqpoint{4.314984in}{1.532379in}}%
\pgfpathlineto{\pgfqpoint{4.316058in}{1.534595in}}%
\pgfpathlineto{\pgfqpoint{4.319279in}{1.531127in}}%
\pgfpathlineto{\pgfqpoint{4.323575in}{1.517252in}}%
\pgfpathlineto{\pgfqpoint{4.326796in}{1.516385in}}%
\pgfpathlineto{\pgfqpoint{4.327870in}{1.513398in}}%
\pgfpathlineto{\pgfqpoint{4.328944in}{1.518023in}}%
\pgfpathlineto{\pgfqpoint{4.330018in}{1.525538in}}%
\pgfpathlineto{\pgfqpoint{4.331092in}{1.521106in}}%
\pgfpathlineto{\pgfqpoint{4.334313in}{1.518312in}}%
\pgfpathlineto{\pgfqpoint{4.335387in}{1.513398in}}%
\pgfpathlineto{\pgfqpoint{4.336461in}{1.514554in}}%
\pgfpathlineto{\pgfqpoint{4.337535in}{1.514651in}}%
\pgfpathlineto{\pgfqpoint{4.338608in}{1.518023in}}%
\pgfpathlineto{\pgfqpoint{4.343978in}{1.517445in}}%
\pgfpathlineto{\pgfqpoint{4.345051in}{1.521010in}}%
\pgfpathlineto{\pgfqpoint{4.346125in}{1.516578in}}%
\pgfpathlineto{\pgfqpoint{4.349347in}{1.514265in}}%
\pgfpathlineto{\pgfqpoint{4.350421in}{1.515036in}}%
\pgfpathlineto{\pgfqpoint{4.352568in}{1.502993in}}%
\pgfpathlineto{\pgfqpoint{4.353642in}{1.503185in}}%
\pgfpathlineto{\pgfqpoint{4.357938in}{1.502511in}}%
\pgfpathlineto{\pgfqpoint{4.359011in}{1.500391in}}%
\pgfpathlineto{\pgfqpoint{4.361159in}{1.494417in}}%
\pgfpathlineto{\pgfqpoint{4.364381in}{1.504438in}}%
\pgfpathlineto{\pgfqpoint{4.365454in}{1.504438in}}%
\pgfpathlineto{\pgfqpoint{4.368676in}{1.511182in}}%
\pgfpathlineto{\pgfqpoint{4.371897in}{1.507906in}}%
\pgfpathlineto{\pgfqpoint{4.372971in}{1.505594in}}%
\pgfpathlineto{\pgfqpoint{4.375119in}{1.512917in}}%
\pgfpathlineto{\pgfqpoint{4.376193in}{1.513687in}}%
\pgfpathlineto{\pgfqpoint{4.379414in}{1.513591in}}%
\pgfpathlineto{\pgfqpoint{4.380488in}{1.510508in}}%
\pgfpathlineto{\pgfqpoint{4.382636in}{1.516096in}}%
\pgfpathlineto{\pgfqpoint{4.383710in}{1.515807in}}%
\pgfpathlineto{\pgfqpoint{4.386931in}{1.513109in}}%
\pgfpathlineto{\pgfqpoint{4.389079in}{1.518505in}}%
\pgfpathlineto{\pgfqpoint{4.391227in}{1.514362in}}%
\pgfpathlineto{\pgfqpoint{4.394448in}{1.513687in}}%
\pgfpathlineto{\pgfqpoint{4.395522in}{1.511953in}}%
\pgfpathlineto{\pgfqpoint{4.396596in}{1.508484in}}%
\pgfpathlineto{\pgfqpoint{4.397670in}{1.511568in}}%
\pgfpathlineto{\pgfqpoint{4.398743in}{1.509833in}}%
\pgfpathlineto{\pgfqpoint{4.405186in}{1.509159in}}%
\pgfpathlineto{\pgfqpoint{4.406260in}{1.505016in}}%
\pgfpathlineto{\pgfqpoint{4.409482in}{1.507714in}}%
\pgfpathlineto{\pgfqpoint{4.411629in}{1.514747in}}%
\pgfpathlineto{\pgfqpoint{4.412703in}{1.513302in}}%
\pgfpathlineto{\pgfqpoint{4.413777in}{1.515036in}}%
\pgfpathlineto{\pgfqpoint{4.416999in}{1.517734in}}%
\pgfpathlineto{\pgfqpoint{4.418073in}{1.510026in}}%
\pgfpathlineto{\pgfqpoint{4.419146in}{1.509352in}}%
\pgfpathlineto{\pgfqpoint{4.420220in}{1.513591in}}%
\pgfpathlineto{\pgfqpoint{4.421294in}{1.511471in}}%
\pgfpathlineto{\pgfqpoint{4.424516in}{1.506365in}}%
\pgfpathlineto{\pgfqpoint{4.426663in}{1.498560in}}%
\pgfpathlineto{\pgfqpoint{4.428811in}{1.505979in}}%
\pgfpathlineto{\pgfqpoint{4.433106in}{1.508870in}}%
\pgfpathlineto{\pgfqpoint{4.435254in}{1.506365in}}%
\pgfpathlineto{\pgfqpoint{4.436328in}{1.506943in}}%
\pgfpathlineto{\pgfqpoint{4.439549in}{1.507714in}}%
\pgfpathlineto{\pgfqpoint{4.440623in}{1.512917in}}%
\pgfpathlineto{\pgfqpoint{4.441697in}{1.513687in}}%
\pgfpathlineto{\pgfqpoint{4.442771in}{1.513687in}}%
\pgfpathlineto{\pgfqpoint{4.443845in}{1.520528in}}%
\pgfpathlineto{\pgfqpoint{4.448140in}{1.520528in}}%
\pgfpathlineto{\pgfqpoint{4.449214in}{1.521877in}}%
\pgfpathlineto{\pgfqpoint{4.450288in}{1.519854in}}%
\pgfpathlineto{\pgfqpoint{4.451361in}{1.519083in}}%
\pgfpathlineto{\pgfqpoint{4.454583in}{1.518794in}}%
\pgfpathlineto{\pgfqpoint{4.456731in}{1.520239in}}%
\pgfpathlineto{\pgfqpoint{4.457805in}{1.518890in}}%
\pgfpathlineto{\pgfqpoint{4.458878in}{1.519372in}}%
\pgfpathlineto{\pgfqpoint{4.463174in}{1.520143in}}%
\pgfpathlineto{\pgfqpoint{4.465321in}{1.524575in}}%
\pgfpathlineto{\pgfqpoint{4.466395in}{1.526598in}}%
\pgfpathlineto{\pgfqpoint{4.469617in}{1.528911in}}%
\pgfpathlineto{\pgfqpoint{4.471764in}{1.533439in}}%
\pgfpathlineto{\pgfqpoint{4.477134in}{1.538931in}}%
\pgfpathlineto{\pgfqpoint{4.478207in}{1.543074in}}%
\pgfpathlineto{\pgfqpoint{4.479281in}{1.540087in}}%
\pgfpathlineto{\pgfqpoint{4.480355in}{1.541532in}}%
\pgfpathlineto{\pgfqpoint{4.481429in}{1.545001in}}%
\pgfpathlineto{\pgfqpoint{4.484650in}{1.543941in}}%
\pgfpathlineto{\pgfqpoint{4.485724in}{1.547891in}}%
\pgfpathlineto{\pgfqpoint{4.486798in}{1.544905in}}%
\pgfpathlineto{\pgfqpoint{4.487872in}{1.548759in}}%
\pgfpathlineto{\pgfqpoint{4.488946in}{1.545483in}}%
\pgfpathlineto{\pgfqpoint{4.493241in}{1.552131in}}%
\pgfpathlineto{\pgfqpoint{4.494315in}{1.549626in}}%
\pgfpathlineto{\pgfqpoint{4.496463in}{1.552613in}}%
\pgfpathlineto{\pgfqpoint{4.499684in}{1.553672in}}%
\pgfpathlineto{\pgfqpoint{4.500758in}{1.555985in}}%
\pgfpathlineto{\pgfqpoint{4.501832in}{1.552034in}}%
\pgfpathlineto{\pgfqpoint{4.502906in}{1.553576in}}%
\pgfpathlineto{\pgfqpoint{4.507201in}{1.554058in}}%
\pgfpathlineto{\pgfqpoint{4.508275in}{1.556081in}}%
\pgfpathlineto{\pgfqpoint{4.509349in}{1.552420in}}%
\pgfpathlineto{\pgfqpoint{4.511496in}{1.556563in}}%
\pgfpathlineto{\pgfqpoint{4.514718in}{1.559357in}}%
\pgfpathlineto{\pgfqpoint{4.515792in}{1.558779in}}%
\pgfpathlineto{\pgfqpoint{4.519013in}{1.569763in}}%
\pgfpathlineto{\pgfqpoint{4.522235in}{1.568125in}}%
\pgfpathlineto{\pgfqpoint{4.523309in}{1.566776in}}%
\pgfpathlineto{\pgfqpoint{4.524383in}{1.560128in}}%
\pgfpathlineto{\pgfqpoint{4.526530in}{1.579783in}}%
\pgfpathlineto{\pgfqpoint{4.529752in}{1.579398in}}%
\pgfpathlineto{\pgfqpoint{4.530826in}{1.580361in}}%
\pgfpathlineto{\pgfqpoint{4.531899in}{1.576507in}}%
\pgfpathlineto{\pgfqpoint{4.532973in}{1.586913in}}%
\pgfpathlineto{\pgfqpoint{4.534047in}{1.589900in}}%
\pgfpathlineto{\pgfqpoint{4.537269in}{1.588647in}}%
\pgfpathlineto{\pgfqpoint{4.538342in}{1.591923in}}%
\pgfpathlineto{\pgfqpoint{4.539416in}{1.580939in}}%
\pgfpathlineto{\pgfqpoint{4.541564in}{1.582963in}}%
\pgfpathlineto{\pgfqpoint{4.544785in}{1.578338in}}%
\pgfpathlineto{\pgfqpoint{4.545859in}{1.584697in}}%
\pgfpathlineto{\pgfqpoint{4.546933in}{1.586046in}}%
\pgfpathlineto{\pgfqpoint{4.548007in}{1.583155in}}%
\pgfpathlineto{\pgfqpoint{4.549081in}{1.584504in}}%
\pgfpathlineto{\pgfqpoint{4.552302in}{1.581999in}}%
\pgfpathlineto{\pgfqpoint{4.554450in}{1.589611in}}%
\pgfpathlineto{\pgfqpoint{4.555524in}{1.586720in}}%
\pgfpathlineto{\pgfqpoint{4.556598in}{1.587780in}}%
\pgfpathlineto{\pgfqpoint{4.559819in}{1.583348in}}%
\pgfpathlineto{\pgfqpoint{4.561967in}{1.574195in}}%
\pgfpathlineto{\pgfqpoint{4.563041in}{1.577182in}}%
\pgfpathlineto{\pgfqpoint{4.564115in}{1.573617in}}%
\pgfpathlineto{\pgfqpoint{4.568410in}{1.569474in}}%
\pgfpathlineto{\pgfqpoint{4.569484in}{1.560321in}}%
\pgfpathlineto{\pgfqpoint{4.571631in}{1.552420in}}%
\pgfpathlineto{\pgfqpoint{4.574853in}{1.553769in}}%
\pgfpathlineto{\pgfqpoint{4.575927in}{1.555021in}}%
\pgfpathlineto{\pgfqpoint{4.577001in}{1.551071in}}%
\pgfpathlineto{\pgfqpoint{4.578074in}{1.563211in}}%
\pgfpathlineto{\pgfqpoint{4.579148in}{1.565234in}}%
\pgfpathlineto{\pgfqpoint{4.582370in}{1.567354in}}%
\pgfpathlineto{\pgfqpoint{4.584517in}{1.562729in}}%
\pgfpathlineto{\pgfqpoint{4.586665in}{1.571208in}}%
\pgfpathlineto{\pgfqpoint{4.589887in}{1.568607in}}%
\pgfpathlineto{\pgfqpoint{4.590961in}{1.574773in}}%
\pgfpathlineto{\pgfqpoint{4.593108in}{1.555888in}}%
\pgfpathlineto{\pgfqpoint{4.594182in}{1.559935in}}%
\pgfpathlineto{\pgfqpoint{4.597404in}{1.557045in}}%
\pgfpathlineto{\pgfqpoint{4.598477in}{1.566680in}}%
\pgfpathlineto{\pgfqpoint{4.600625in}{1.570726in}}%
\pgfpathlineto{\pgfqpoint{4.601699in}{1.565909in}}%
\pgfpathlineto{\pgfqpoint{4.604920in}{1.566102in}}%
\pgfpathlineto{\pgfqpoint{4.609216in}{1.572942in}}%
\pgfpathlineto{\pgfqpoint{4.612437in}{1.575736in}}%
\pgfpathlineto{\pgfqpoint{4.613511in}{1.574195in}}%
\pgfpathlineto{\pgfqpoint{4.614585in}{1.571112in}}%
\pgfpathlineto{\pgfqpoint{4.615659in}{1.576026in}}%
\pgfpathlineto{\pgfqpoint{4.616733in}{1.570148in}}%
\pgfpathlineto{\pgfqpoint{4.619954in}{1.566872in}}%
\pgfpathlineto{\pgfqpoint{4.622102in}{1.572557in}}%
\pgfpathlineto{\pgfqpoint{4.623176in}{1.566969in}}%
\pgfpathlineto{\pgfqpoint{4.624249in}{1.566776in}}%
\pgfpathlineto{\pgfqpoint{4.629619in}{1.569956in}}%
\pgfpathlineto{\pgfqpoint{4.631766in}{1.574677in}}%
\pgfpathlineto{\pgfqpoint{4.634988in}{1.578049in}}%
\pgfpathlineto{\pgfqpoint{4.637136in}{1.563307in}}%
\pgfpathlineto{\pgfqpoint{4.638209in}{1.567258in}}%
\pgfpathlineto{\pgfqpoint{4.639283in}{1.568992in}}%
\pgfpathlineto{\pgfqpoint{4.642505in}{1.568703in}}%
\pgfpathlineto{\pgfqpoint{4.644652in}{1.566680in}}%
\pgfpathlineto{\pgfqpoint{4.646800in}{1.562151in}}%
\pgfpathlineto{\pgfqpoint{4.650022in}{1.564560in}}%
\pgfpathlineto{\pgfqpoint{4.652169in}{1.558972in}}%
\pgfpathlineto{\pgfqpoint{4.653243in}{1.556467in}}%
\pgfpathlineto{\pgfqpoint{4.654317in}{1.550300in}}%
\pgfpathlineto{\pgfqpoint{4.657538in}{1.548759in}}%
\pgfpathlineto{\pgfqpoint{4.658612in}{1.551842in}}%
\pgfpathlineto{\pgfqpoint{4.659686in}{1.547121in}}%
\pgfpathlineto{\pgfqpoint{4.660760in}{1.545290in}}%
\pgfpathlineto{\pgfqpoint{4.661834in}{1.549626in}}%
\pgfpathlineto{\pgfqpoint{4.665055in}{1.544712in}}%
\pgfpathlineto{\pgfqpoint{4.666129in}{1.544712in}}%
\pgfpathlineto{\pgfqpoint{4.667203in}{1.541918in}}%
\pgfpathlineto{\pgfqpoint{4.668277in}{1.551167in}}%
\pgfpathlineto{\pgfqpoint{4.669351in}{1.547891in}}%
\pgfpathlineto{\pgfqpoint{4.673646in}{1.537871in}}%
\pgfpathlineto{\pgfqpoint{4.674720in}{1.543459in}}%
\pgfpathlineto{\pgfqpoint{4.675794in}{1.542496in}}%
\pgfpathlineto{\pgfqpoint{4.676868in}{1.540280in}}%
\pgfpathlineto{\pgfqpoint{4.680089in}{1.537100in}}%
\pgfpathlineto{\pgfqpoint{4.681163in}{1.541243in}}%
\pgfpathlineto{\pgfqpoint{4.682237in}{1.541629in}}%
\pgfpathlineto{\pgfqpoint{4.684384in}{1.550686in}}%
\pgfpathlineto{\pgfqpoint{4.689754in}{1.557719in}}%
\pgfpathlineto{\pgfqpoint{4.690827in}{1.556178in}}%
\pgfpathlineto{\pgfqpoint{4.691901in}{1.550878in}}%
\pgfpathlineto{\pgfqpoint{4.695123in}{1.552324in}}%
\pgfpathlineto{\pgfqpoint{4.696197in}{1.547121in}}%
\pgfpathlineto{\pgfqpoint{4.697271in}{1.544712in}}%
\pgfpathlineto{\pgfqpoint{4.698344in}{1.550011in}}%
\pgfpathlineto{\pgfqpoint{4.699418in}{1.545097in}}%
\pgfpathlineto{\pgfqpoint{4.702640in}{1.542689in}}%
\pgfpathlineto{\pgfqpoint{4.703714in}{1.545001in}}%
\pgfpathlineto{\pgfqpoint{4.704787in}{1.543556in}}%
\pgfpathlineto{\pgfqpoint{4.706935in}{1.545868in}}%
\pgfpathlineto{\pgfqpoint{4.710157in}{1.547217in}}%
\pgfpathlineto{\pgfqpoint{4.711230in}{1.542014in}}%
\pgfpathlineto{\pgfqpoint{4.712304in}{1.543074in}}%
\pgfpathlineto{\pgfqpoint{4.713378in}{1.547988in}}%
\pgfpathlineto{\pgfqpoint{4.714452in}{1.549722in}}%
\pgfpathlineto{\pgfqpoint{4.717673in}{1.547602in}}%
\pgfpathlineto{\pgfqpoint{4.718747in}{1.543941in}}%
\pgfpathlineto{\pgfqpoint{4.719821in}{1.550878in}}%
\pgfpathlineto{\pgfqpoint{4.721969in}{1.571401in}}%
\pgfpathlineto{\pgfqpoint{4.725190in}{1.575640in}}%
\pgfpathlineto{\pgfqpoint{4.726264in}{1.579783in}}%
\pgfpathlineto{\pgfqpoint{4.728412in}{1.574388in}}%
\pgfpathlineto{\pgfqpoint{4.729486in}{1.576604in}}%
\pgfpathlineto{\pgfqpoint{4.732707in}{1.575544in}}%
\pgfpathlineto{\pgfqpoint{4.733781in}{1.579398in}}%
\pgfpathlineto{\pgfqpoint{4.734855in}{1.575255in}}%
\pgfpathlineto{\pgfqpoint{4.737003in}{1.574966in}}%
\pgfpathlineto{\pgfqpoint{4.740224in}{1.579301in}}%
\pgfpathlineto{\pgfqpoint{4.741298in}{1.572075in}}%
\pgfpathlineto{\pgfqpoint{4.742372in}{1.575833in}}%
\pgfpathlineto{\pgfqpoint{4.743446in}{1.572557in}}%
\pgfpathlineto{\pgfqpoint{4.744519in}{1.572846in}}%
\pgfpathlineto{\pgfqpoint{4.747741in}{1.571015in}}%
\pgfpathlineto{\pgfqpoint{4.748815in}{1.572461in}}%
\pgfpathlineto{\pgfqpoint{4.749889in}{1.571208in}}%
\pgfpathlineto{\pgfqpoint{4.750962in}{1.573328in}}%
\pgfpathlineto{\pgfqpoint{4.752036in}{1.573617in}}%
\pgfpathlineto{\pgfqpoint{4.755258in}{1.576989in}}%
\pgfpathlineto{\pgfqpoint{4.756332in}{1.577085in}}%
\pgfpathlineto{\pgfqpoint{4.757405in}{1.574291in}}%
\pgfpathlineto{\pgfqpoint{4.758479in}{1.574099in}}%
\pgfpathlineto{\pgfqpoint{4.759553in}{1.572942in}}%
\pgfpathlineto{\pgfqpoint{4.762775in}{1.571401in}}%
\pgfpathlineto{\pgfqpoint{4.763849in}{1.571690in}}%
\pgfpathlineto{\pgfqpoint{4.770292in}{1.565812in}}%
\pgfpathlineto{\pgfqpoint{4.771365in}{1.568318in}}%
\pgfpathlineto{\pgfqpoint{4.772439in}{1.566680in}}%
\pgfpathlineto{\pgfqpoint{4.773513in}{1.563211in}}%
\pgfpathlineto{\pgfqpoint{4.774587in}{1.567450in}}%
\pgfpathlineto{\pgfqpoint{4.777808in}{1.568125in}}%
\pgfpathlineto{\pgfqpoint{4.781030in}{1.557430in}}%
\pgfpathlineto{\pgfqpoint{4.782104in}{1.555310in}}%
\pgfpathlineto{\pgfqpoint{4.785325in}{1.558394in}}%
\pgfpathlineto{\pgfqpoint{4.786399in}{1.553191in}}%
\pgfpathlineto{\pgfqpoint{4.787473in}{1.559935in}}%
\pgfpathlineto{\pgfqpoint{4.788547in}{1.559646in}}%
\pgfpathlineto{\pgfqpoint{4.789621in}{1.557045in}}%
\pgfpathlineto{\pgfqpoint{4.792842in}{1.560802in}}%
\pgfpathlineto{\pgfqpoint{4.793916in}{1.563693in}}%
\pgfpathlineto{\pgfqpoint{4.796064in}{1.565234in}}%
\pgfpathlineto{\pgfqpoint{4.801433in}{1.564945in}}%
\pgfpathlineto{\pgfqpoint{4.803581in}{1.563885in}}%
\pgfpathlineto{\pgfqpoint{4.804654in}{1.559550in}}%
\pgfpathlineto{\pgfqpoint{4.807876in}{1.561573in}}%
\pgfpathlineto{\pgfqpoint{4.808950in}{1.566102in}}%
\pgfpathlineto{\pgfqpoint{4.810024in}{1.564078in}}%
\pgfpathlineto{\pgfqpoint{4.811097in}{1.556563in}}%
\pgfpathlineto{\pgfqpoint{4.812171in}{1.558490in}}%
\pgfpathlineto{\pgfqpoint{4.815393in}{1.553094in}}%
\pgfpathlineto{\pgfqpoint{4.816467in}{1.553383in}}%
\pgfpathlineto{\pgfqpoint{4.817540in}{1.562344in}}%
\pgfpathlineto{\pgfqpoint{4.818614in}{1.564271in}}%
\pgfpathlineto{\pgfqpoint{4.825057in}{1.555310in}}%
\pgfpathlineto{\pgfqpoint{4.826131in}{1.559261in}}%
\pgfpathlineto{\pgfqpoint{4.827205in}{1.557526in}}%
\pgfpathlineto{\pgfqpoint{4.830426in}{1.558008in}}%
\pgfpathlineto{\pgfqpoint{4.831500in}{1.555503in}}%
\pgfpathlineto{\pgfqpoint{4.832574in}{1.558008in}}%
\pgfpathlineto{\pgfqpoint{4.833648in}{1.557526in}}%
\pgfpathlineto{\pgfqpoint{4.834722in}{1.560513in}}%
\pgfpathlineto{\pgfqpoint{4.837943in}{1.548373in}}%
\pgfpathlineto{\pgfqpoint{4.839017in}{1.551456in}}%
\pgfpathlineto{\pgfqpoint{4.840091in}{1.550493in}}%
\pgfpathlineto{\pgfqpoint{4.841165in}{1.550397in}}%
\pgfpathlineto{\pgfqpoint{4.842239in}{1.551360in}}%
\pgfpathlineto{\pgfqpoint{4.845460in}{1.552131in}}%
\pgfpathlineto{\pgfqpoint{4.847608in}{1.555599in}}%
\pgfpathlineto{\pgfqpoint{4.848682in}{1.555118in}}%
\pgfpathlineto{\pgfqpoint{4.849756in}{1.549144in}}%
\pgfpathlineto{\pgfqpoint{4.854051in}{1.545194in}}%
\pgfpathlineto{\pgfqpoint{4.855125in}{1.549337in}}%
\pgfpathlineto{\pgfqpoint{4.856199in}{1.561188in}}%
\pgfpathlineto{\pgfqpoint{4.857272in}{1.554829in}}%
\pgfpathlineto{\pgfqpoint{4.860494in}{1.547024in}}%
\pgfpathlineto{\pgfqpoint{4.862642in}{1.547988in}}%
\pgfpathlineto{\pgfqpoint{4.863715in}{1.556563in}}%
\pgfpathlineto{\pgfqpoint{4.864789in}{1.557141in}}%
\pgfpathlineto{\pgfqpoint{4.868011in}{1.555118in}}%
\pgfpathlineto{\pgfqpoint{4.869085in}{1.559550in}}%
\pgfpathlineto{\pgfqpoint{4.870159in}{1.555696in}}%
\pgfpathlineto{\pgfqpoint{4.871232in}{1.556274in}}%
\pgfpathlineto{\pgfqpoint{4.872306in}{1.553961in}}%
\pgfpathlineto{\pgfqpoint{4.875528in}{1.553094in}}%
\pgfpathlineto{\pgfqpoint{4.878749in}{1.545579in}}%
\pgfpathlineto{\pgfqpoint{4.883045in}{1.547602in}}%
\pgfpathlineto{\pgfqpoint{4.884118in}{1.550589in}}%
\pgfpathlineto{\pgfqpoint{4.885192in}{1.547988in}}%
\pgfpathlineto{\pgfqpoint{4.886266in}{1.554443in}}%
\pgfpathlineto{\pgfqpoint{4.887340in}{1.552131in}}%
\pgfpathlineto{\pgfqpoint{4.890561in}{1.552902in}}%
\pgfpathlineto{\pgfqpoint{4.891635in}{1.554443in}}%
\pgfpathlineto{\pgfqpoint{4.892709in}{1.552902in}}%
\pgfpathlineto{\pgfqpoint{4.893783in}{1.558008in}}%
\pgfpathlineto{\pgfqpoint{4.894857in}{1.555985in}}%
\pgfpathlineto{\pgfqpoint{4.898078in}{1.557141in}}%
\pgfpathlineto{\pgfqpoint{4.901300in}{1.561188in}}%
\pgfpathlineto{\pgfqpoint{4.902374in}{1.560610in}}%
\pgfpathlineto{\pgfqpoint{4.905595in}{1.561188in}}%
\pgfpathlineto{\pgfqpoint{4.906669in}{1.565909in}}%
\pgfpathlineto{\pgfqpoint{4.907743in}{1.564175in}}%
\pgfpathlineto{\pgfqpoint{4.909891in}{1.557334in}}%
\pgfpathlineto{\pgfqpoint{4.913112in}{1.558779in}}%
\pgfpathlineto{\pgfqpoint{4.914186in}{1.556178in}}%
\pgfpathlineto{\pgfqpoint{4.915260in}{1.557526in}}%
\pgfpathlineto{\pgfqpoint{4.916334in}{1.561284in}}%
\pgfpathlineto{\pgfqpoint{4.920629in}{1.564175in}}%
\pgfpathlineto{\pgfqpoint{4.921703in}{1.563693in}}%
\pgfpathlineto{\pgfqpoint{4.922777in}{1.560032in}}%
\pgfpathlineto{\pgfqpoint{4.923850in}{1.551264in}}%
\pgfpathlineto{\pgfqpoint{4.924924in}{1.548662in}}%
\pgfpathlineto{\pgfqpoint{4.929220in}{1.555021in}}%
\pgfpathlineto{\pgfqpoint{4.930293in}{1.554732in}}%
\pgfpathlineto{\pgfqpoint{4.931367in}{1.558394in}}%
\pgfpathlineto{\pgfqpoint{4.932441in}{1.557623in}}%
\pgfpathlineto{\pgfqpoint{4.936737in}{1.560321in}}%
\pgfpathlineto{\pgfqpoint{4.938884in}{1.568029in}}%
\pgfpathlineto{\pgfqpoint{4.939958in}{1.568029in}}%
\pgfpathlineto{\pgfqpoint{4.943180in}{1.565523in}}%
\pgfpathlineto{\pgfqpoint{4.944253in}{1.563211in}}%
\pgfpathlineto{\pgfqpoint{4.945327in}{1.564464in}}%
\pgfpathlineto{\pgfqpoint{4.946401in}{1.564078in}}%
\pgfpathlineto{\pgfqpoint{4.947475in}{1.572461in}}%
\pgfpathlineto{\pgfqpoint{4.950696in}{1.573039in}}%
\pgfpathlineto{\pgfqpoint{4.951770in}{1.569281in}}%
\pgfpathlineto{\pgfqpoint{4.953918in}{1.575351in}}%
\pgfpathlineto{\pgfqpoint{4.954992in}{1.577567in}}%
\pgfpathlineto{\pgfqpoint{4.959287in}{1.576604in}}%
\pgfpathlineto{\pgfqpoint{4.960361in}{1.577567in}}%
\pgfpathlineto{\pgfqpoint{4.961435in}{1.577374in}}%
\pgfpathlineto{\pgfqpoint{4.962509in}{1.579205in}}%
\pgfpathlineto{\pgfqpoint{4.965730in}{1.579880in}}%
\pgfpathlineto{\pgfqpoint{4.966804in}{1.574099in}}%
\pgfpathlineto{\pgfqpoint{4.967878in}{1.573039in}}%
\pgfpathlineto{\pgfqpoint{4.970026in}{1.575158in}}%
\pgfpathlineto{\pgfqpoint{4.973247in}{1.576218in}}%
\pgfpathlineto{\pgfqpoint{4.974321in}{1.575929in}}%
\pgfpathlineto{\pgfqpoint{4.975395in}{1.574773in}}%
\pgfpathlineto{\pgfqpoint{4.976469in}{1.572268in}}%
\pgfpathlineto{\pgfqpoint{4.977542in}{1.573328in}}%
\pgfpathlineto{\pgfqpoint{4.980764in}{1.574291in}}%
\pgfpathlineto{\pgfqpoint{4.981838in}{1.573520in}}%
\pgfpathlineto{\pgfqpoint{4.982912in}{1.575158in}}%
\pgfpathlineto{\pgfqpoint{4.983985in}{1.575544in}}%
\pgfpathlineto{\pgfqpoint{4.985059in}{1.574869in}}%
\pgfpathlineto{\pgfqpoint{4.988281in}{1.577471in}}%
\pgfpathlineto{\pgfqpoint{4.989355in}{1.574099in}}%
\pgfpathlineto{\pgfqpoint{4.990428in}{1.575062in}}%
\pgfpathlineto{\pgfqpoint{4.991502in}{1.573520in}}%
\pgfpathlineto{\pgfqpoint{4.992576in}{1.574388in}}%
\pgfpathlineto{\pgfqpoint{4.995798in}{1.571401in}}%
\pgfpathlineto{\pgfqpoint{4.997945in}{1.576989in}}%
\pgfpathlineto{\pgfqpoint{4.999019in}{1.577374in}}%
\pgfpathlineto{\pgfqpoint{5.003314in}{1.577663in}}%
\pgfpathlineto{\pgfqpoint{5.004388in}{1.574099in}}%
\pgfpathlineto{\pgfqpoint{5.005462in}{1.575158in}}%
\pgfpathlineto{\pgfqpoint{5.007610in}{1.586046in}}%
\pgfpathlineto{\pgfqpoint{5.010831in}{1.587684in}}%
\pgfpathlineto{\pgfqpoint{5.012979in}{1.590960in}}%
\pgfpathlineto{\pgfqpoint{5.014053in}{1.586046in}}%
\pgfpathlineto{\pgfqpoint{5.015127in}{1.589418in}}%
\pgfpathlineto{\pgfqpoint{5.018348in}{1.589033in}}%
\pgfpathlineto{\pgfqpoint{5.019422in}{1.591152in}}%
\pgfpathlineto{\pgfqpoint{5.020496in}{1.590574in}}%
\pgfpathlineto{\pgfqpoint{5.025865in}{1.595777in}}%
\pgfpathlineto{\pgfqpoint{5.026939in}{1.598764in}}%
\pgfpathlineto{\pgfqpoint{5.028013in}{1.597126in}}%
\pgfpathlineto{\pgfqpoint{5.029087in}{1.586817in}}%
\pgfpathlineto{\pgfqpoint{5.030160in}{1.582288in}}%
\pgfpathlineto{\pgfqpoint{5.033382in}{1.585275in}}%
\pgfpathlineto{\pgfqpoint{5.036603in}{1.573231in}}%
\pgfpathlineto{\pgfqpoint{5.037677in}{1.573617in}}%
\pgfpathlineto{\pgfqpoint{5.040899in}{1.573424in}}%
\pgfpathlineto{\pgfqpoint{5.043047in}{1.575351in}}%
\pgfpathlineto{\pgfqpoint{5.044120in}{1.575929in}}%
\pgfpathlineto{\pgfqpoint{5.045194in}{1.574388in}}%
\pgfpathlineto{\pgfqpoint{5.048416in}{1.574291in}}%
\pgfpathlineto{\pgfqpoint{5.049490in}{1.573617in}}%
\pgfpathlineto{\pgfqpoint{5.051637in}{1.574966in}}%
\pgfpathlineto{\pgfqpoint{5.052711in}{1.573135in}}%
\pgfpathlineto{\pgfqpoint{5.058080in}{1.579205in}}%
\pgfpathlineto{\pgfqpoint{5.059154in}{1.579109in}}%
\pgfpathlineto{\pgfqpoint{5.060228in}{1.582192in}}%
\pgfpathlineto{\pgfqpoint{5.064523in}{1.581807in}}%
\pgfpathlineto{\pgfqpoint{5.065597in}{1.582674in}}%
\pgfpathlineto{\pgfqpoint{5.066671in}{1.581421in}}%
\pgfpathlineto{\pgfqpoint{5.067745in}{1.583155in}}%
\pgfpathlineto{\pgfqpoint{5.070966in}{1.579976in}}%
\pgfpathlineto{\pgfqpoint{5.072040in}{1.575158in}}%
\pgfpathlineto{\pgfqpoint{5.073114in}{1.574002in}}%
\pgfpathlineto{\pgfqpoint{5.074188in}{1.576026in}}%
\pgfpathlineto{\pgfqpoint{5.075262in}{1.571304in}}%
\pgfpathlineto{\pgfqpoint{5.078483in}{1.572557in}}%
\pgfpathlineto{\pgfqpoint{5.082779in}{1.586335in}}%
\pgfpathlineto{\pgfqpoint{5.086000in}{1.584986in}}%
\pgfpathlineto{\pgfqpoint{5.087074in}{1.582963in}}%
\pgfpathlineto{\pgfqpoint{5.088148in}{1.584119in}}%
\pgfpathlineto{\pgfqpoint{5.089222in}{1.580650in}}%
\pgfpathlineto{\pgfqpoint{5.090295in}{1.581807in}}%
\pgfpathlineto{\pgfqpoint{5.093517in}{1.581710in}}%
\pgfpathlineto{\pgfqpoint{5.094591in}{1.583541in}}%
\pgfpathlineto{\pgfqpoint{5.095665in}{1.579205in}}%
\pgfpathlineto{\pgfqpoint{5.096738in}{1.578145in}}%
\pgfpathlineto{\pgfqpoint{5.097812in}{1.581421in}}%
\pgfpathlineto{\pgfqpoint{5.101034in}{1.584215in}}%
\pgfpathlineto{\pgfqpoint{5.102108in}{1.581132in}}%
\pgfpathlineto{\pgfqpoint{5.103181in}{1.586913in}}%
\pgfpathlineto{\pgfqpoint{5.104255in}{1.579590in}}%
\pgfpathlineto{\pgfqpoint{5.105329in}{1.579687in}}%
\pgfpathlineto{\pgfqpoint{5.110698in}{1.571786in}}%
\pgfpathlineto{\pgfqpoint{5.111772in}{1.569859in}}%
\pgfpathlineto{\pgfqpoint{5.112846in}{1.572942in}}%
\pgfpathlineto{\pgfqpoint{5.117141in}{1.577760in}}%
\pgfpathlineto{\pgfqpoint{5.118215in}{1.574773in}}%
\pgfpathlineto{\pgfqpoint{5.119289in}{1.574099in}}%
\pgfpathlineto{\pgfqpoint{5.120363in}{1.578338in}}%
\pgfpathlineto{\pgfqpoint{5.123584in}{1.583444in}}%
\pgfpathlineto{\pgfqpoint{5.124658in}{1.587587in}}%
\pgfpathlineto{\pgfqpoint{5.125732in}{1.586624in}}%
\pgfpathlineto{\pgfqpoint{5.126806in}{1.587106in}}%
\pgfpathlineto{\pgfqpoint{5.127880in}{1.589804in}}%
\pgfpathlineto{\pgfqpoint{5.131101in}{1.590960in}}%
\pgfpathlineto{\pgfqpoint{5.134323in}{1.590093in}}%
\pgfpathlineto{\pgfqpoint{5.135397in}{1.594621in}}%
\pgfpathlineto{\pgfqpoint{5.139692in}{1.592598in}}%
\pgfpathlineto{\pgfqpoint{5.140766in}{1.594236in}}%
\pgfpathlineto{\pgfqpoint{5.142914in}{1.599535in}}%
\pgfpathlineto{\pgfqpoint{5.146135in}{1.598764in}}%
\pgfpathlineto{\pgfqpoint{5.147209in}{1.597415in}}%
\pgfpathlineto{\pgfqpoint{5.148283in}{1.591731in}}%
\pgfpathlineto{\pgfqpoint{5.149357in}{1.589418in}}%
\pgfpathlineto{\pgfqpoint{5.150430in}{1.589514in}}%
\pgfpathlineto{\pgfqpoint{5.154726in}{1.582770in}}%
\pgfpathlineto{\pgfqpoint{5.155800in}{1.588262in}}%
\pgfpathlineto{\pgfqpoint{5.157947in}{1.592501in}}%
\pgfpathlineto{\pgfqpoint{5.161169in}{1.588166in}}%
\pgfpathlineto{\pgfqpoint{5.162243in}{1.581036in}}%
\pgfpathlineto{\pgfqpoint{5.163316in}{1.578531in}}%
\pgfpathlineto{\pgfqpoint{5.164390in}{1.578434in}}%
\pgfpathlineto{\pgfqpoint{5.165464in}{1.577085in}}%
\pgfpathlineto{\pgfqpoint{5.168686in}{1.579398in}}%
\pgfpathlineto{\pgfqpoint{5.169759in}{1.563982in}}%
\pgfpathlineto{\pgfqpoint{5.170833in}{1.558297in}}%
\pgfpathlineto{\pgfqpoint{5.171907in}{1.559646in}}%
\pgfpathlineto{\pgfqpoint{5.172981in}{1.553576in}}%
\pgfpathlineto{\pgfqpoint{5.176202in}{1.552420in}}%
\pgfpathlineto{\pgfqpoint{5.177276in}{1.553191in}}%
\pgfpathlineto{\pgfqpoint{5.179424in}{1.565042in}}%
\pgfpathlineto{\pgfqpoint{5.180498in}{1.564753in}}%
\pgfpathlineto{\pgfqpoint{5.184793in}{1.569859in}}%
\pgfpathlineto{\pgfqpoint{5.185867in}{1.569859in}}%
\pgfpathlineto{\pgfqpoint{5.188015in}{1.571304in}}%
\pgfpathlineto{\pgfqpoint{5.191236in}{1.568896in}}%
\pgfpathlineto{\pgfqpoint{5.192310in}{1.567161in}}%
\pgfpathlineto{\pgfqpoint{5.193384in}{1.562922in}}%
\pgfpathlineto{\pgfqpoint{5.195532in}{1.564367in}}%
\pgfpathlineto{\pgfqpoint{5.198753in}{1.561284in}}%
\pgfpathlineto{\pgfqpoint{5.199827in}{1.564945in}}%
\pgfpathlineto{\pgfqpoint{5.200901in}{1.562633in}}%
\pgfpathlineto{\pgfqpoint{5.201975in}{1.570341in}}%
\pgfpathlineto{\pgfqpoint{5.203048in}{1.567065in}}%
\pgfpathlineto{\pgfqpoint{5.207344in}{1.570341in}}%
\pgfpathlineto{\pgfqpoint{5.208418in}{1.568703in}}%
\pgfpathlineto{\pgfqpoint{5.209491in}{1.569763in}}%
\pgfpathlineto{\pgfqpoint{5.210565in}{1.576700in}}%
\pgfpathlineto{\pgfqpoint{5.215935in}{1.578723in}}%
\pgfpathlineto{\pgfqpoint{5.218082in}{1.570148in}}%
\pgfpathlineto{\pgfqpoint{5.221304in}{1.568607in}}%
\pgfpathlineto{\pgfqpoint{5.223451in}{1.561477in}}%
\pgfpathlineto{\pgfqpoint{5.224525in}{1.561958in}}%
\pgfpathlineto{\pgfqpoint{5.225599in}{1.558779in}}%
\pgfpathlineto{\pgfqpoint{5.228821in}{1.568799in}}%
\pgfpathlineto{\pgfqpoint{5.229894in}{1.575447in}}%
\pgfpathlineto{\pgfqpoint{5.230968in}{1.575255in}}%
\pgfpathlineto{\pgfqpoint{5.232042in}{1.575736in}}%
\pgfpathlineto{\pgfqpoint{5.233116in}{1.587202in}}%
\pgfpathlineto{\pgfqpoint{5.236337in}{1.585371in}}%
\pgfpathlineto{\pgfqpoint{5.238485in}{1.591056in}}%
\pgfpathlineto{\pgfqpoint{5.240633in}{1.587106in}}%
\pgfpathlineto{\pgfqpoint{5.244928in}{1.586046in}}%
\pgfpathlineto{\pgfqpoint{5.246002in}{1.584215in}}%
\pgfpathlineto{\pgfqpoint{5.247076in}{1.583926in}}%
\pgfpathlineto{\pgfqpoint{5.248150in}{1.584408in}}%
\pgfpathlineto{\pgfqpoint{5.251371in}{1.582866in}}%
\pgfpathlineto{\pgfqpoint{5.252445in}{1.586335in}}%
\pgfpathlineto{\pgfqpoint{5.253519in}{1.586239in}}%
\pgfpathlineto{\pgfqpoint{5.255667in}{1.588166in}}%
\pgfpathlineto{\pgfqpoint{5.259962in}{1.588936in}}%
\pgfpathlineto{\pgfqpoint{5.261036in}{1.585179in}}%
\pgfpathlineto{\pgfqpoint{5.262110in}{1.584023in}}%
\pgfpathlineto{\pgfqpoint{5.263183in}{1.579109in}}%
\pgfpathlineto{\pgfqpoint{5.266405in}{1.578627in}}%
\pgfpathlineto{\pgfqpoint{5.267479in}{1.572942in}}%
\pgfpathlineto{\pgfqpoint{5.268553in}{1.574291in}}%
\pgfpathlineto{\pgfqpoint{5.269626in}{1.582577in}}%
\pgfpathlineto{\pgfqpoint{5.270700in}{1.583348in}}%
\pgfpathlineto{\pgfqpoint{5.273922in}{1.586817in}}%
\pgfpathlineto{\pgfqpoint{5.274996in}{1.584215in}}%
\pgfpathlineto{\pgfqpoint{5.276069in}{1.588936in}}%
\pgfpathlineto{\pgfqpoint{5.277143in}{1.587009in}}%
\pgfpathlineto{\pgfqpoint{5.278217in}{1.589033in}}%
\pgfpathlineto{\pgfqpoint{5.281439in}{1.589707in}}%
\pgfpathlineto{\pgfqpoint{5.282513in}{1.587877in}}%
\pgfpathlineto{\pgfqpoint{5.284660in}{1.580072in}}%
\pgfpathlineto{\pgfqpoint{5.285734in}{1.581228in}}%
\pgfpathlineto{\pgfqpoint{5.288956in}{1.585564in}}%
\pgfpathlineto{\pgfqpoint{5.290029in}{1.581710in}}%
\pgfpathlineto{\pgfqpoint{5.291103in}{1.584023in}}%
\pgfpathlineto{\pgfqpoint{5.292177in}{1.588358in}}%
\pgfpathlineto{\pgfqpoint{5.296472in}{1.589611in}}%
\pgfpathlineto{\pgfqpoint{5.297546in}{1.586817in}}%
\pgfpathlineto{\pgfqpoint{5.298620in}{1.589996in}}%
\pgfpathlineto{\pgfqpoint{5.299694in}{1.589033in}}%
\pgfpathlineto{\pgfqpoint{5.300768in}{1.590767in}}%
\pgfpathlineto{\pgfqpoint{5.303989in}{1.589322in}}%
\pgfpathlineto{\pgfqpoint{5.306137in}{1.592116in}}%
\pgfpathlineto{\pgfqpoint{5.307211in}{1.591152in}}%
\pgfpathlineto{\pgfqpoint{5.308285in}{1.588166in}}%
\pgfpathlineto{\pgfqpoint{5.311506in}{1.592020in}}%
\pgfpathlineto{\pgfqpoint{5.312580in}{1.590382in}}%
\pgfpathlineto{\pgfqpoint{5.314728in}{1.597319in}}%
\pgfpathlineto{\pgfqpoint{5.315802in}{1.597126in}}%
\pgfpathlineto{\pgfqpoint{5.319023in}{1.597608in}}%
\pgfpathlineto{\pgfqpoint{5.320097in}{1.601269in}}%
\pgfpathlineto{\pgfqpoint{5.322245in}{1.600402in}}%
\pgfpathlineto{\pgfqpoint{5.323318in}{1.600209in}}%
\pgfpathlineto{\pgfqpoint{5.326540in}{1.601076in}}%
\pgfpathlineto{\pgfqpoint{5.328688in}{1.594043in}}%
\pgfpathlineto{\pgfqpoint{5.329761in}{1.594814in}}%
\pgfpathlineto{\pgfqpoint{5.330835in}{1.598090in}}%
\pgfpathlineto{\pgfqpoint{5.335131in}{1.593947in}}%
\pgfpathlineto{\pgfqpoint{5.336204in}{1.594814in}}%
\pgfpathlineto{\pgfqpoint{5.337278in}{1.596741in}}%
\pgfpathlineto{\pgfqpoint{5.338352in}{1.595295in}}%
\pgfpathlineto{\pgfqpoint{5.342647in}{1.593368in}}%
\pgfpathlineto{\pgfqpoint{5.344795in}{1.596163in}}%
\pgfpathlineto{\pgfqpoint{5.345869in}{1.593850in}}%
\pgfpathlineto{\pgfqpoint{5.350164in}{1.592309in}}%
\pgfpathlineto{\pgfqpoint{5.351238in}{1.593561in}}%
\pgfpathlineto{\pgfqpoint{5.353386in}{1.592501in}}%
\pgfpathlineto{\pgfqpoint{5.358755in}{1.589611in}}%
\pgfpathlineto{\pgfqpoint{5.360903in}{1.574966in}}%
\pgfpathlineto{\pgfqpoint{5.364124in}{1.576604in}}%
\pgfpathlineto{\pgfqpoint{5.365198in}{1.575833in}}%
\pgfpathlineto{\pgfqpoint{5.366272in}{1.576893in}}%
\pgfpathlineto{\pgfqpoint{5.367346in}{1.579109in}}%
\pgfpathlineto{\pgfqpoint{5.368420in}{1.575062in}}%
\pgfpathlineto{\pgfqpoint{5.371641in}{1.573135in}}%
\pgfpathlineto{\pgfqpoint{5.372715in}{1.576411in}}%
\pgfpathlineto{\pgfqpoint{5.373789in}{1.575255in}}%
\pgfpathlineto{\pgfqpoint{5.374863in}{1.579205in}}%
\pgfpathlineto{\pgfqpoint{5.375936in}{1.576796in}}%
\pgfpathlineto{\pgfqpoint{5.379158in}{1.577278in}}%
\pgfpathlineto{\pgfqpoint{5.380232in}{1.579205in}}%
\pgfpathlineto{\pgfqpoint{5.381306in}{1.575447in}}%
\pgfpathlineto{\pgfqpoint{5.383453in}{1.578049in}}%
\pgfpathlineto{\pgfqpoint{5.387749in}{1.570052in}}%
\pgfpathlineto{\pgfqpoint{5.389896in}{1.574677in}}%
\pgfpathlineto{\pgfqpoint{5.394192in}{1.573424in}}%
\pgfpathlineto{\pgfqpoint{5.395266in}{1.575062in}}%
\pgfpathlineto{\pgfqpoint{5.396339in}{1.574099in}}%
\pgfpathlineto{\pgfqpoint{5.397413in}{1.571593in}}%
\pgfpathlineto{\pgfqpoint{5.398487in}{1.577471in}}%
\pgfpathlineto{\pgfqpoint{5.401709in}{1.579109in}}%
\pgfpathlineto{\pgfqpoint{5.402782in}{1.580843in}}%
\pgfpathlineto{\pgfqpoint{5.403856in}{1.580169in}}%
\pgfpathlineto{\pgfqpoint{5.404930in}{1.584215in}}%
\pgfpathlineto{\pgfqpoint{5.406004in}{1.582288in}}%
\pgfpathlineto{\pgfqpoint{5.409225in}{1.586335in}}%
\pgfpathlineto{\pgfqpoint{5.410299in}{1.577278in}}%
\pgfpathlineto{\pgfqpoint{5.411373in}{1.573135in}}%
\pgfpathlineto{\pgfqpoint{5.412447in}{1.572268in}}%
\pgfpathlineto{\pgfqpoint{5.413521in}{1.569763in}}%
\pgfpathlineto{\pgfqpoint{5.416742in}{1.568125in}}%
\pgfpathlineto{\pgfqpoint{5.417816in}{1.568607in}}%
\pgfpathlineto{\pgfqpoint{5.418890in}{1.574002in}}%
\pgfpathlineto{\pgfqpoint{5.421038in}{1.575833in}}%
\pgfpathlineto{\pgfqpoint{5.424259in}{1.577278in}}%
\pgfpathlineto{\pgfqpoint{5.425333in}{1.574869in}}%
\pgfpathlineto{\pgfqpoint{5.427481in}{1.574580in}}%
\pgfpathlineto{\pgfqpoint{5.428555in}{1.572364in}}%
\pgfpathlineto{\pgfqpoint{5.433924in}{1.584312in}}%
\pgfpathlineto{\pgfqpoint{5.436071in}{1.581421in}}%
\pgfpathlineto{\pgfqpoint{5.439293in}{1.581710in}}%
\pgfpathlineto{\pgfqpoint{5.441441in}{1.581036in}}%
\pgfpathlineto{\pgfqpoint{5.446810in}{1.559357in}}%
\pgfpathlineto{\pgfqpoint{5.447884in}{1.549433in}}%
\pgfpathlineto{\pgfqpoint{5.450031in}{1.571015in}}%
\pgfpathlineto{\pgfqpoint{5.451105in}{1.570052in}}%
\pgfpathlineto{\pgfqpoint{5.454327in}{1.569570in}}%
\pgfpathlineto{\pgfqpoint{5.455401in}{1.560610in}}%
\pgfpathlineto{\pgfqpoint{5.457548in}{1.567258in}}%
\pgfpathlineto{\pgfqpoint{5.458622in}{1.560032in}}%
\pgfpathlineto{\pgfqpoint{5.462917in}{1.568510in}}%
\pgfpathlineto{\pgfqpoint{5.463991in}{1.564560in}}%
\pgfpathlineto{\pgfqpoint{5.465065in}{1.565138in}}%
\pgfpathlineto{\pgfqpoint{5.466139in}{1.567354in}}%
\pgfpathlineto{\pgfqpoint{5.469360in}{1.566680in}}%
\pgfpathlineto{\pgfqpoint{5.470434in}{1.572461in}}%
\pgfpathlineto{\pgfqpoint{5.471508in}{1.571015in}}%
\pgfpathlineto{\pgfqpoint{5.473656in}{1.558008in}}%
\pgfpathlineto{\pgfqpoint{5.476877in}{1.559839in}}%
\pgfpathlineto{\pgfqpoint{5.479025in}{1.553287in}}%
\pgfpathlineto{\pgfqpoint{5.481173in}{1.555214in}}%
\pgfpathlineto{\pgfqpoint{5.486542in}{1.549433in}}%
\pgfpathlineto{\pgfqpoint{5.487616in}{1.545097in}}%
\pgfpathlineto{\pgfqpoint{5.488690in}{1.544134in}}%
\pgfpathlineto{\pgfqpoint{5.491911in}{1.553383in}}%
\pgfpathlineto{\pgfqpoint{5.492985in}{1.553865in}}%
\pgfpathlineto{\pgfqpoint{5.495133in}{1.559839in}}%
\pgfpathlineto{\pgfqpoint{5.496206in}{1.559261in}}%
\pgfpathlineto{\pgfqpoint{5.500502in}{1.560899in}}%
\pgfpathlineto{\pgfqpoint{5.501576in}{1.557912in}}%
\pgfpathlineto{\pgfqpoint{5.502649in}{1.563500in}}%
\pgfpathlineto{\pgfqpoint{5.503723in}{1.563693in}}%
\pgfpathlineto{\pgfqpoint{5.506945in}{1.563693in}}%
\pgfpathlineto{\pgfqpoint{5.508019in}{1.568125in}}%
\pgfpathlineto{\pgfqpoint{5.509092in}{1.565042in}}%
\pgfpathlineto{\pgfqpoint{5.510166in}{1.573424in}}%
\pgfpathlineto{\pgfqpoint{5.511240in}{1.575544in}}%
\pgfpathlineto{\pgfqpoint{5.514462in}{1.577182in}}%
\pgfpathlineto{\pgfqpoint{5.515535in}{1.575640in}}%
\pgfpathlineto{\pgfqpoint{5.516609in}{1.578145in}}%
\pgfpathlineto{\pgfqpoint{5.517683in}{1.577567in}}%
\pgfpathlineto{\pgfqpoint{5.518757in}{1.581421in}}%
\pgfpathlineto{\pgfqpoint{5.521979in}{1.580650in}}%
\pgfpathlineto{\pgfqpoint{5.524126in}{1.575447in}}%
\pgfpathlineto{\pgfqpoint{5.525200in}{1.575929in}}%
\pgfpathlineto{\pgfqpoint{5.526274in}{1.572461in}}%
\pgfpathlineto{\pgfqpoint{5.530569in}{1.566969in}}%
\pgfpathlineto{\pgfqpoint{5.531643in}{1.568703in}}%
\pgfpathlineto{\pgfqpoint{5.533791in}{1.559839in}}%
\pgfpathlineto{\pgfqpoint{5.537012in}{1.566487in}}%
\pgfpathlineto{\pgfqpoint{5.538086in}{1.566776in}}%
\pgfpathlineto{\pgfqpoint{5.540234in}{1.572364in}}%
\pgfpathlineto{\pgfqpoint{5.541308in}{1.569281in}}%
\pgfpathlineto{\pgfqpoint{5.544529in}{1.566102in}}%
\pgfpathlineto{\pgfqpoint{5.545603in}{1.567643in}}%
\pgfpathlineto{\pgfqpoint{5.546677in}{1.565523in}}%
\pgfpathlineto{\pgfqpoint{5.548824in}{1.568029in}}%
\pgfpathlineto{\pgfqpoint{5.553120in}{1.570823in}}%
\pgfpathlineto{\pgfqpoint{5.555267in}{1.562537in}}%
\pgfpathlineto{\pgfqpoint{5.556341in}{1.571883in}}%
\pgfpathlineto{\pgfqpoint{5.559563in}{1.574773in}}%
\pgfpathlineto{\pgfqpoint{5.561711in}{1.569088in}}%
\pgfpathlineto{\pgfqpoint{5.562784in}{1.568703in}}%
\pgfpathlineto{\pgfqpoint{5.563858in}{1.564656in}}%
\pgfpathlineto{\pgfqpoint{5.568154in}{1.570630in}}%
\pgfpathlineto{\pgfqpoint{5.569227in}{1.578434in}}%
\pgfpathlineto{\pgfqpoint{5.571375in}{1.570726in}}%
\pgfpathlineto{\pgfqpoint{5.574597in}{1.573424in}}%
\pgfpathlineto{\pgfqpoint{5.576744in}{1.581999in}}%
\pgfpathlineto{\pgfqpoint{5.577818in}{1.580072in}}%
\pgfpathlineto{\pgfqpoint{5.582113in}{1.580361in}}%
\pgfpathlineto{\pgfqpoint{5.583187in}{1.584119in}}%
\pgfpathlineto{\pgfqpoint{5.585335in}{1.576026in}}%
\pgfpathlineto{\pgfqpoint{5.589630in}{1.573231in}}%
\pgfpathlineto{\pgfqpoint{5.590704in}{1.578338in}}%
\pgfpathlineto{\pgfqpoint{5.593926in}{1.569281in}}%
\pgfpathlineto{\pgfqpoint{5.597147in}{1.571401in}}%
\pgfpathlineto{\pgfqpoint{5.598221in}{1.570052in}}%
\pgfpathlineto{\pgfqpoint{5.599295in}{1.563596in}}%
\pgfpathlineto{\pgfqpoint{5.600369in}{1.569570in}}%
\pgfpathlineto{\pgfqpoint{5.601443in}{1.565909in}}%
\pgfpathlineto{\pgfqpoint{5.605738in}{1.569570in}}%
\pgfpathlineto{\pgfqpoint{5.606812in}{1.565909in}}%
\pgfpathlineto{\pgfqpoint{5.608959in}{1.587491in}}%
\pgfpathlineto{\pgfqpoint{5.612181in}{1.587395in}}%
\pgfpathlineto{\pgfqpoint{5.614329in}{1.603871in}}%
\pgfpathlineto{\pgfqpoint{5.615402in}{1.603678in}}%
\pgfpathlineto{\pgfqpoint{5.616476in}{1.611579in}}%
\pgfpathlineto{\pgfqpoint{5.619698in}{1.618130in}}%
\pgfpathlineto{\pgfqpoint{5.620772in}{1.611097in}}%
\pgfpathlineto{\pgfqpoint{5.621845in}{1.616974in}}%
\pgfpathlineto{\pgfqpoint{5.622919in}{1.615336in}}%
\pgfpathlineto{\pgfqpoint{5.623993in}{1.619865in}}%
\pgfpathlineto{\pgfqpoint{5.627215in}{1.617938in}}%
\pgfpathlineto{\pgfqpoint{5.628289in}{1.613024in}}%
\pgfpathlineto{\pgfqpoint{5.629362in}{1.611675in}}%
\pgfpathlineto{\pgfqpoint{5.630436in}{1.606761in}}%
\pgfpathlineto{\pgfqpoint{5.631510in}{1.612735in}}%
\pgfpathlineto{\pgfqpoint{5.636879in}{1.614469in}}%
\pgfpathlineto{\pgfqpoint{5.637953in}{1.619576in}}%
\pgfpathlineto{\pgfqpoint{5.639027in}{1.618901in}}%
\pgfpathlineto{\pgfqpoint{5.642248in}{1.620635in}}%
\pgfpathlineto{\pgfqpoint{5.643322in}{1.616974in}}%
\pgfpathlineto{\pgfqpoint{5.645470in}{1.620924in}}%
\pgfpathlineto{\pgfqpoint{5.649765in}{1.617841in}}%
\pgfpathlineto{\pgfqpoint{5.651913in}{1.629307in}}%
\pgfpathlineto{\pgfqpoint{5.652987in}{1.627380in}}%
\pgfpathlineto{\pgfqpoint{5.654061in}{1.626705in}}%
\pgfpathlineto{\pgfqpoint{5.657282in}{1.630078in}}%
\pgfpathlineto{\pgfqpoint{5.658356in}{1.632101in}}%
\pgfpathlineto{\pgfqpoint{5.659430in}{1.631137in}}%
\pgfpathlineto{\pgfqpoint{5.660504in}{1.630945in}}%
\pgfpathlineto{\pgfqpoint{5.661578in}{1.632679in}}%
\pgfpathlineto{\pgfqpoint{5.664799in}{1.632775in}}%
\pgfpathlineto{\pgfqpoint{5.665873in}{1.633835in}}%
\pgfpathlineto{\pgfqpoint{5.668021in}{1.641736in}}%
\pgfpathlineto{\pgfqpoint{5.669094in}{1.638556in}}%
\pgfpathlineto{\pgfqpoint{5.672316in}{1.640194in}}%
\pgfpathlineto{\pgfqpoint{5.674464in}{1.635762in}}%
\pgfpathlineto{\pgfqpoint{5.675537in}{1.641158in}}%
\pgfpathlineto{\pgfqpoint{5.679833in}{1.639809in}}%
\pgfpathlineto{\pgfqpoint{5.680907in}{1.645205in}}%
\pgfpathlineto{\pgfqpoint{5.684128in}{1.644819in}}%
\pgfpathlineto{\pgfqpoint{5.687350in}{1.648191in}}%
\pgfpathlineto{\pgfqpoint{5.688423in}{1.645494in}}%
\pgfpathlineto{\pgfqpoint{5.689497in}{1.645494in}}%
\pgfpathlineto{\pgfqpoint{5.690571in}{1.632872in}}%
\pgfpathlineto{\pgfqpoint{5.691645in}{1.634317in}}%
\pgfpathlineto{\pgfqpoint{5.694867in}{1.629596in}}%
\pgfpathlineto{\pgfqpoint{5.695940in}{1.632390in}}%
\pgfpathlineto{\pgfqpoint{5.697014in}{1.626898in}}%
\pgfpathlineto{\pgfqpoint{5.698088in}{1.627476in}}%
\pgfpathlineto{\pgfqpoint{5.699162in}{1.627380in}}%
\pgfpathlineto{\pgfqpoint{5.702383in}{1.630559in}}%
\pgfpathlineto{\pgfqpoint{5.703457in}{1.633450in}}%
\pgfpathlineto{\pgfqpoint{5.704531in}{1.630752in}}%
\pgfpathlineto{\pgfqpoint{5.705605in}{1.616396in}}%
\pgfpathlineto{\pgfqpoint{5.706679in}{1.620732in}}%
\pgfpathlineto{\pgfqpoint{5.709900in}{1.622466in}}%
\pgfpathlineto{\pgfqpoint{5.710974in}{1.619865in}}%
\pgfpathlineto{\pgfqpoint{5.712048in}{1.630270in}}%
\pgfpathlineto{\pgfqpoint{5.713122in}{1.624682in}}%
\pgfpathlineto{\pgfqpoint{5.714196in}{1.624008in}}%
\pgfpathlineto{\pgfqpoint{5.717417in}{1.627187in}}%
\pgfpathlineto{\pgfqpoint{5.718491in}{1.621792in}}%
\pgfpathlineto{\pgfqpoint{5.719565in}{1.623140in}}%
\pgfpathlineto{\pgfqpoint{5.720639in}{1.623140in}}%
\pgfpathlineto{\pgfqpoint{5.721712in}{1.625453in}}%
\pgfpathlineto{\pgfqpoint{5.724934in}{1.625164in}}%
\pgfpathlineto{\pgfqpoint{5.726008in}{1.629018in}}%
\pgfpathlineto{\pgfqpoint{5.727082in}{1.625742in}}%
\pgfpathlineto{\pgfqpoint{5.728155in}{1.628440in}}%
\pgfpathlineto{\pgfqpoint{5.729229in}{1.624008in}}%
\pgfpathlineto{\pgfqpoint{5.732451in}{1.626416in}}%
\pgfpathlineto{\pgfqpoint{5.734599in}{1.619383in}}%
\pgfpathlineto{\pgfqpoint{5.735672in}{1.613120in}}%
\pgfpathlineto{\pgfqpoint{5.736746in}{1.613313in}}%
\pgfpathlineto{\pgfqpoint{5.739968in}{1.608977in}}%
\pgfpathlineto{\pgfqpoint{5.742115in}{1.614951in}}%
\pgfpathlineto{\pgfqpoint{5.744263in}{1.621310in}}%
\pgfpathlineto{\pgfqpoint{5.748558in}{1.623622in}}%
\pgfpathlineto{\pgfqpoint{5.749632in}{1.619768in}}%
\pgfpathlineto{\pgfqpoint{5.751780in}{1.623815in}}%
\pgfpathlineto{\pgfqpoint{5.755001in}{1.622081in}}%
\pgfpathlineto{\pgfqpoint{5.756075in}{1.630752in}}%
\pgfpathlineto{\pgfqpoint{5.757149in}{1.628825in}}%
\pgfpathlineto{\pgfqpoint{5.758223in}{1.632390in}}%
\pgfpathlineto{\pgfqpoint{5.759297in}{1.638364in}}%
\pgfpathlineto{\pgfqpoint{5.762518in}{1.637593in}}%
\pgfpathlineto{\pgfqpoint{5.763592in}{1.641062in}}%
\pgfpathlineto{\pgfqpoint{5.764666in}{1.639809in}}%
\pgfpathlineto{\pgfqpoint{5.766814in}{1.647613in}}%
\pgfpathlineto{\pgfqpoint{5.770035in}{1.647517in}}%
\pgfpathlineto{\pgfqpoint{5.771109in}{1.650311in}}%
\pgfpathlineto{\pgfqpoint{5.772183in}{1.649733in}}%
\pgfpathlineto{\pgfqpoint{5.773257in}{1.655032in}}%
\pgfpathlineto{\pgfqpoint{5.774331in}{1.653009in}}%
\pgfpathlineto{\pgfqpoint{5.778626in}{1.656285in}}%
\pgfpathlineto{\pgfqpoint{5.779700in}{1.658308in}}%
\pgfpathlineto{\pgfqpoint{5.781847in}{1.668039in}}%
\pgfpathlineto{\pgfqpoint{5.786143in}{1.670544in}}%
\pgfpathlineto{\pgfqpoint{5.787217in}{1.673050in}}%
\pgfpathlineto{\pgfqpoint{5.788290in}{1.665631in}}%
\pgfpathlineto{\pgfqpoint{5.789364in}{1.669966in}}%
\pgfpathlineto{\pgfqpoint{5.792586in}{1.670255in}}%
\pgfpathlineto{\pgfqpoint{5.793660in}{1.666401in}}%
\pgfpathlineto{\pgfqpoint{5.794733in}{1.670834in}}%
\pgfpathlineto{\pgfqpoint{5.795807in}{1.669485in}}%
\pgfpathlineto{\pgfqpoint{5.796881in}{1.669485in}}%
\pgfpathlineto{\pgfqpoint{5.800103in}{1.670255in}}%
\pgfpathlineto{\pgfqpoint{5.801177in}{1.668328in}}%
\pgfpathlineto{\pgfqpoint{5.802250in}{1.667654in}}%
\pgfpathlineto{\pgfqpoint{5.803324in}{1.665534in}}%
\pgfpathlineto{\pgfqpoint{5.804398in}{1.671701in}}%
\pgfpathlineto{\pgfqpoint{5.807620in}{1.669774in}}%
\pgfpathlineto{\pgfqpoint{5.808693in}{1.660813in}}%
\pgfpathlineto{\pgfqpoint{5.809767in}{1.665149in}}%
\pgfpathlineto{\pgfqpoint{5.810841in}{1.661199in}}%
\pgfpathlineto{\pgfqpoint{5.811915in}{1.665823in}}%
\pgfpathlineto{\pgfqpoint{5.815136in}{1.658212in}}%
\pgfpathlineto{\pgfqpoint{5.816210in}{1.653972in}}%
\pgfpathlineto{\pgfqpoint{5.817284in}{1.653202in}}%
\pgfpathlineto{\pgfqpoint{5.818358in}{1.653394in}}%
\pgfpathlineto{\pgfqpoint{5.819432in}{1.650986in}}%
\pgfpathlineto{\pgfqpoint{5.822653in}{1.650600in}}%
\pgfpathlineto{\pgfqpoint{5.823727in}{1.651178in}}%
\pgfpathlineto{\pgfqpoint{5.824801in}{1.652431in}}%
\pgfpathlineto{\pgfqpoint{5.825875in}{1.652816in}}%
\pgfpathlineto{\pgfqpoint{5.826949in}{1.651082in}}%
\pgfpathlineto{\pgfqpoint{5.830170in}{1.650696in}}%
\pgfpathlineto{\pgfqpoint{5.831244in}{1.643567in}}%
\pgfpathlineto{\pgfqpoint{5.832318in}{1.647035in}}%
\pgfpathlineto{\pgfqpoint{5.834466in}{1.640965in}}%
\pgfpathlineto{\pgfqpoint{5.838761in}{1.642507in}}%
\pgfpathlineto{\pgfqpoint{5.839835in}{1.641640in}}%
\pgfpathlineto{\pgfqpoint{5.840909in}{1.643470in}}%
\pgfpathlineto{\pgfqpoint{5.841982in}{1.637786in}}%
\pgfpathlineto{\pgfqpoint{5.845204in}{1.641351in}}%
\pgfpathlineto{\pgfqpoint{5.846278in}{1.639424in}}%
\pgfpathlineto{\pgfqpoint{5.847352in}{1.640002in}}%
\pgfpathlineto{\pgfqpoint{5.849499in}{1.644626in}}%
\pgfpathlineto{\pgfqpoint{5.854868in}{1.651564in}}%
\pgfpathlineto{\pgfqpoint{5.855942in}{1.650600in}}%
\pgfpathlineto{\pgfqpoint{5.857016in}{1.635666in}}%
\pgfpathlineto{\pgfqpoint{5.860238in}{1.642025in}}%
\pgfpathlineto{\pgfqpoint{5.861311in}{1.632583in}}%
\pgfpathlineto{\pgfqpoint{5.862385in}{1.632872in}}%
\pgfpathlineto{\pgfqpoint{5.863459in}{1.637015in}}%
\pgfpathlineto{\pgfqpoint{5.864533in}{1.636148in}}%
\pgfpathlineto{\pgfqpoint{5.867755in}{1.630463in}}%
\pgfpathlineto{\pgfqpoint{5.868828in}{1.631041in}}%
\pgfpathlineto{\pgfqpoint{5.870976in}{1.640098in}}%
\pgfpathlineto{\pgfqpoint{5.872050in}{1.641929in}}%
\pgfpathlineto{\pgfqpoint{5.875271in}{1.638460in}}%
\pgfpathlineto{\pgfqpoint{5.876345in}{1.641351in}}%
\pgfpathlineto{\pgfqpoint{5.877419in}{1.637689in}}%
\pgfpathlineto{\pgfqpoint{5.878493in}{1.638171in}}%
\pgfpathlineto{\pgfqpoint{5.879567in}{1.637015in}}%
\pgfpathlineto{\pgfqpoint{5.882788in}{1.636148in}}%
\pgfpathlineto{\pgfqpoint{5.884936in}{1.627476in}}%
\pgfpathlineto{\pgfqpoint{5.886010in}{1.627380in}}%
\pgfpathlineto{\pgfqpoint{5.887084in}{1.624489in}}%
\pgfpathlineto{\pgfqpoint{5.890305in}{1.626802in}}%
\pgfpathlineto{\pgfqpoint{5.891379in}{1.624297in}}%
\pgfpathlineto{\pgfqpoint{5.892453in}{1.627765in}}%
\pgfpathlineto{\pgfqpoint{5.894600in}{1.627573in}}%
\pgfpathlineto{\pgfqpoint{5.897822in}{1.628825in}}%
\pgfpathlineto{\pgfqpoint{5.898896in}{1.627476in}}%
\pgfpathlineto{\pgfqpoint{5.899970in}{1.628440in}}%
\pgfpathlineto{\pgfqpoint{5.902117in}{1.609844in}}%
\pgfpathlineto{\pgfqpoint{5.905339in}{1.609941in}}%
\pgfpathlineto{\pgfqpoint{5.907487in}{1.605027in}}%
\pgfpathlineto{\pgfqpoint{5.908560in}{1.612735in}}%
\pgfpathlineto{\pgfqpoint{5.909634in}{1.609941in}}%
\pgfpathlineto{\pgfqpoint{5.912856in}{1.608977in}}%
\pgfpathlineto{\pgfqpoint{5.913930in}{1.605316in}}%
\pgfpathlineto{\pgfqpoint{5.915003in}{1.599149in}}%
\pgfpathlineto{\pgfqpoint{5.916077in}{1.598571in}}%
\pgfpathlineto{\pgfqpoint{5.917151in}{1.600306in}}%
\pgfpathlineto{\pgfqpoint{5.921446in}{1.605219in}}%
\pgfpathlineto{\pgfqpoint{5.922520in}{1.606954in}}%
\pgfpathlineto{\pgfqpoint{5.923594in}{1.597030in}}%
\pgfpathlineto{\pgfqpoint{5.924668in}{1.597030in}}%
\pgfpathlineto{\pgfqpoint{5.927889in}{1.592694in}}%
\pgfpathlineto{\pgfqpoint{5.928963in}{1.602811in}}%
\pgfpathlineto{\pgfqpoint{5.930037in}{1.607532in}}%
\pgfpathlineto{\pgfqpoint{5.931111in}{1.606761in}}%
\pgfpathlineto{\pgfqpoint{5.932185in}{1.608784in}}%
\pgfpathlineto{\pgfqpoint{5.935406in}{1.610808in}}%
\pgfpathlineto{\pgfqpoint{5.937554in}{1.627091in}}%
\pgfpathlineto{\pgfqpoint{5.939702in}{1.630848in}}%
\pgfpathlineto{\pgfqpoint{5.942923in}{1.634702in}}%
\pgfpathlineto{\pgfqpoint{5.943997in}{1.633354in}}%
\pgfpathlineto{\pgfqpoint{5.945071in}{1.624297in}}%
\pgfpathlineto{\pgfqpoint{5.950440in}{1.623044in}}%
\pgfpathlineto{\pgfqpoint{5.951514in}{1.628247in}}%
\pgfpathlineto{\pgfqpoint{5.952588in}{1.636918in}}%
\pgfpathlineto{\pgfqpoint{5.953662in}{1.634799in}}%
\pgfpathlineto{\pgfqpoint{5.954735in}{1.637882in}}%
\pgfpathlineto{\pgfqpoint{5.957957in}{1.640098in}}%
\pgfpathlineto{\pgfqpoint{5.959031in}{1.645205in}}%
\pgfpathlineto{\pgfqpoint{5.960105in}{1.639038in}}%
\pgfpathlineto{\pgfqpoint{5.961178in}{1.640580in}}%
\pgfpathlineto{\pgfqpoint{5.962252in}{1.644434in}}%
\pgfpathlineto{\pgfqpoint{5.966548in}{1.651660in}}%
\pgfpathlineto{\pgfqpoint{5.967621in}{1.650407in}}%
\pgfpathlineto{\pgfqpoint{5.968695in}{1.656188in}}%
\pgfpathlineto{\pgfqpoint{5.969769in}{1.656477in}}%
\pgfpathlineto{\pgfqpoint{5.974065in}{1.656092in}}%
\pgfpathlineto{\pgfqpoint{5.975138in}{1.654454in}}%
\pgfpathlineto{\pgfqpoint{5.976212in}{1.656959in}}%
\pgfpathlineto{\pgfqpoint{5.977286in}{1.653876in}}%
\pgfpathlineto{\pgfqpoint{5.981581in}{1.664089in}}%
\pgfpathlineto{\pgfqpoint{5.982655in}{1.663607in}}%
\pgfpathlineto{\pgfqpoint{5.983729in}{1.664667in}}%
\pgfpathlineto{\pgfqpoint{5.984803in}{1.657730in}}%
\pgfpathlineto{\pgfqpoint{5.988024in}{1.652720in}}%
\pgfpathlineto{\pgfqpoint{5.989098in}{1.653394in}}%
\pgfpathlineto{\pgfqpoint{5.990172in}{1.650889in}}%
\pgfpathlineto{\pgfqpoint{5.991246in}{1.652720in}}%
\pgfpathlineto{\pgfqpoint{5.992320in}{1.651660in}}%
\pgfpathlineto{\pgfqpoint{5.996615in}{1.653298in}}%
\pgfpathlineto{\pgfqpoint{5.997689in}{1.649059in}}%
\pgfpathlineto{\pgfqpoint{5.998763in}{1.650022in}}%
\pgfpathlineto{\pgfqpoint{5.999837in}{1.653105in}}%
\pgfpathlineto{\pgfqpoint{6.003058in}{1.650407in}}%
\pgfpathlineto{\pgfqpoint{6.004132in}{1.630752in}}%
\pgfpathlineto{\pgfqpoint{6.005206in}{1.627765in}}%
\pgfpathlineto{\pgfqpoint{6.006280in}{1.622177in}}%
\pgfpathlineto{\pgfqpoint{6.007354in}{1.626224in}}%
\pgfpathlineto{\pgfqpoint{6.010575in}{1.624297in}}%
\pgfpathlineto{\pgfqpoint{6.011649in}{1.621213in}}%
\pgfpathlineto{\pgfqpoint{6.012723in}{1.615914in}}%
\pgfpathlineto{\pgfqpoint{6.013797in}{1.614951in}}%
\pgfpathlineto{\pgfqpoint{6.014870in}{1.617552in}}%
\pgfpathlineto{\pgfqpoint{6.018092in}{1.612735in}}%
\pgfpathlineto{\pgfqpoint{6.019166in}{1.612831in}}%
\pgfpathlineto{\pgfqpoint{6.022387in}{1.620924in}}%
\pgfpathlineto{\pgfqpoint{6.025609in}{1.617263in}}%
\pgfpathlineto{\pgfqpoint{6.027756in}{1.613216in}}%
\pgfpathlineto{\pgfqpoint{6.028830in}{1.616492in}}%
\pgfpathlineto{\pgfqpoint{6.029904in}{1.622755in}}%
\pgfpathlineto{\pgfqpoint{6.034199in}{1.624778in}}%
\pgfpathlineto{\pgfqpoint{6.035273in}{1.626898in}}%
\pgfpathlineto{\pgfqpoint{6.036347in}{1.632390in}}%
\pgfpathlineto{\pgfqpoint{6.037421in}{1.634895in}}%
\pgfpathlineto{\pgfqpoint{6.041716in}{1.626513in}}%
\pgfpathlineto{\pgfqpoint{6.044938in}{1.630463in}}%
\pgfpathlineto{\pgfqpoint{6.048159in}{1.629981in}}%
\pgfpathlineto{\pgfqpoint{6.050307in}{1.622466in}}%
\pgfpathlineto{\pgfqpoint{6.052455in}{1.624104in}}%
\pgfpathlineto{\pgfqpoint{6.055676in}{1.625164in}}%
\pgfpathlineto{\pgfqpoint{6.056750in}{1.624200in}}%
\pgfpathlineto{\pgfqpoint{6.057824in}{1.630945in}}%
\pgfpathlineto{\pgfqpoint{6.058898in}{1.630078in}}%
\pgfpathlineto{\pgfqpoint{6.059972in}{1.633064in}}%
\pgfpathlineto{\pgfqpoint{6.064267in}{1.631041in}}%
\pgfpathlineto{\pgfqpoint{6.065341in}{1.627187in}}%
\pgfpathlineto{\pgfqpoint{6.066415in}{1.626609in}}%
\pgfpathlineto{\pgfqpoint{6.067488in}{1.626994in}}%
\pgfpathlineto{\pgfqpoint{6.070710in}{1.622273in}}%
\pgfpathlineto{\pgfqpoint{6.071784in}{1.623719in}}%
\pgfpathlineto{\pgfqpoint{6.075005in}{1.618997in}}%
\pgfpathlineto{\pgfqpoint{6.079301in}{1.623815in}}%
\pgfpathlineto{\pgfqpoint{6.080375in}{1.621310in}}%
\pgfpathlineto{\pgfqpoint{6.081448in}{1.621213in}}%
\pgfpathlineto{\pgfqpoint{6.082522in}{1.623140in}}%
\pgfpathlineto{\pgfqpoint{6.085744in}{1.622177in}}%
\pgfpathlineto{\pgfqpoint{6.087891in}{1.625453in}}%
\pgfpathlineto{\pgfqpoint{6.088965in}{1.622851in}}%
\pgfpathlineto{\pgfqpoint{6.093261in}{1.624489in}}%
\pgfpathlineto{\pgfqpoint{6.094334in}{1.628054in}}%
\pgfpathlineto{\pgfqpoint{6.095408in}{1.625646in}}%
\pgfpathlineto{\pgfqpoint{6.096482in}{1.621021in}}%
\pgfpathlineto{\pgfqpoint{6.097556in}{1.610904in}}%
\pgfpathlineto{\pgfqpoint{6.100777in}{1.609170in}}%
\pgfpathlineto{\pgfqpoint{6.101851in}{1.606087in}}%
\pgfpathlineto{\pgfqpoint{6.102925in}{1.611868in}}%
\pgfpathlineto{\pgfqpoint{6.105073in}{1.599246in}}%
\pgfpathlineto{\pgfqpoint{6.109368in}{1.599246in}}%
\pgfpathlineto{\pgfqpoint{6.110442in}{1.601365in}}%
\pgfpathlineto{\pgfqpoint{6.111516in}{1.598957in}}%
\pgfpathlineto{\pgfqpoint{6.112590in}{1.606087in}}%
\pgfpathlineto{\pgfqpoint{6.115811in}{1.605509in}}%
\pgfpathlineto{\pgfqpoint{6.116885in}{1.603678in}}%
\pgfpathlineto{\pgfqpoint{6.117959in}{1.603389in}}%
\pgfpathlineto{\pgfqpoint{6.120107in}{1.598668in}}%
\pgfpathlineto{\pgfqpoint{6.124402in}{1.594043in}}%
\pgfpathlineto{\pgfqpoint{6.125476in}{1.586817in}}%
\pgfpathlineto{\pgfqpoint{6.127623in}{1.595006in}}%
\pgfpathlineto{\pgfqpoint{6.131919in}{1.595488in}}%
\pgfpathlineto{\pgfqpoint{6.132993in}{1.591731in}}%
\pgfpathlineto{\pgfqpoint{6.134066in}{1.594043in}}%
\pgfpathlineto{\pgfqpoint{6.135140in}{1.594139in}}%
\pgfpathlineto{\pgfqpoint{6.139436in}{1.601751in}}%
\pgfpathlineto{\pgfqpoint{6.140509in}{1.605605in}}%
\pgfpathlineto{\pgfqpoint{6.142657in}{1.603871in}}%
\pgfpathlineto{\pgfqpoint{6.145879in}{1.603292in}}%
\pgfpathlineto{\pgfqpoint{6.148026in}{1.604352in}}%
\pgfpathlineto{\pgfqpoint{6.149100in}{1.601655in}}%
\pgfpathlineto{\pgfqpoint{6.150174in}{1.606279in}}%
\pgfpathlineto{\pgfqpoint{6.153396in}{1.610422in}}%
\pgfpathlineto{\pgfqpoint{6.154469in}{1.604063in}}%
\pgfpathlineto{\pgfqpoint{6.155543in}{1.606087in}}%
\pgfpathlineto{\pgfqpoint{6.157691in}{1.605509in}}%
\pgfpathlineto{\pgfqpoint{6.160912in}{1.605027in}}%
\pgfpathlineto{\pgfqpoint{6.163060in}{1.594910in}}%
\pgfpathlineto{\pgfqpoint{6.165208in}{1.594717in}}%
\pgfpathlineto{\pgfqpoint{6.168429in}{1.597897in}}%
\pgfpathlineto{\pgfqpoint{6.169503in}{1.589996in}}%
\pgfpathlineto{\pgfqpoint{6.170577in}{1.589996in}}%
\pgfpathlineto{\pgfqpoint{6.171651in}{1.586239in}}%
\pgfpathlineto{\pgfqpoint{6.172725in}{1.588358in}}%
\pgfpathlineto{\pgfqpoint{6.175946in}{1.591345in}}%
\pgfpathlineto{\pgfqpoint{6.178094in}{1.588262in}}%
\pgfpathlineto{\pgfqpoint{6.179168in}{1.583444in}}%
\pgfpathlineto{\pgfqpoint{6.180242in}{1.583059in}}%
\pgfpathlineto{\pgfqpoint{6.183463in}{1.580650in}}%
\pgfpathlineto{\pgfqpoint{6.184537in}{1.577856in}}%
\pgfpathlineto{\pgfqpoint{6.186685in}{1.583155in}}%
\pgfpathlineto{\pgfqpoint{6.187758in}{1.583830in}}%
\pgfpathlineto{\pgfqpoint{6.190980in}{1.584697in}}%
\pgfpathlineto{\pgfqpoint{6.192054in}{1.581999in}}%
\pgfpathlineto{\pgfqpoint{6.193128in}{1.582770in}}%
\pgfpathlineto{\pgfqpoint{6.194201in}{1.589611in}}%
\pgfpathlineto{\pgfqpoint{6.195275in}{1.589707in}}%
\pgfpathlineto{\pgfqpoint{6.198497in}{1.585082in}}%
\pgfpathlineto{\pgfqpoint{6.200644in}{1.591152in}}%
\pgfpathlineto{\pgfqpoint{6.201718in}{1.621213in}}%
\pgfpathlineto{\pgfqpoint{6.206014in}{1.626416in}}%
\pgfpathlineto{\pgfqpoint{6.207087in}{1.630752in}}%
\pgfpathlineto{\pgfqpoint{6.208161in}{1.624778in}}%
\pgfpathlineto{\pgfqpoint{6.210309in}{1.630945in}}%
\pgfpathlineto{\pgfqpoint{6.213531in}{1.630463in}}%
\pgfpathlineto{\pgfqpoint{6.216752in}{1.623044in}}%
\pgfpathlineto{\pgfqpoint{6.217826in}{1.623430in}}%
\pgfpathlineto{\pgfqpoint{6.221047in}{1.629789in}}%
\pgfpathlineto{\pgfqpoint{6.222121in}{1.627091in}}%
\pgfpathlineto{\pgfqpoint{6.223195in}{1.626513in}}%
\pgfpathlineto{\pgfqpoint{6.224269in}{1.621984in}}%
\pgfpathlineto{\pgfqpoint{6.225343in}{1.620154in}}%
\pgfpathlineto{\pgfqpoint{6.229638in}{1.626031in}}%
\pgfpathlineto{\pgfqpoint{6.230712in}{1.625357in}}%
\pgfpathlineto{\pgfqpoint{6.231786in}{1.625646in}}%
\pgfpathlineto{\pgfqpoint{6.232860in}{1.628921in}}%
\pgfpathlineto{\pgfqpoint{6.236081in}{1.628247in}}%
\pgfpathlineto{\pgfqpoint{6.237155in}{1.627380in}}%
\pgfpathlineto{\pgfqpoint{6.238229in}{1.624008in}}%
\pgfpathlineto{\pgfqpoint{6.240376in}{1.622177in}}%
\pgfpathlineto{\pgfqpoint{6.244672in}{1.617263in}}%
\pgfpathlineto{\pgfqpoint{6.245746in}{1.613313in}}%
\pgfpathlineto{\pgfqpoint{6.246820in}{1.607050in}}%
\pgfpathlineto{\pgfqpoint{6.247893in}{1.606279in}}%
\pgfpathlineto{\pgfqpoint{6.251115in}{1.607917in}}%
\pgfpathlineto{\pgfqpoint{6.253263in}{1.616300in}}%
\pgfpathlineto{\pgfqpoint{6.254336in}{1.615722in}}%
\pgfpathlineto{\pgfqpoint{6.255410in}{1.621695in}}%
\pgfpathlineto{\pgfqpoint{6.258632in}{1.623719in}}%
\pgfpathlineto{\pgfqpoint{6.259706in}{1.634702in}}%
\pgfpathlineto{\pgfqpoint{6.260779in}{1.635955in}}%
\pgfpathlineto{\pgfqpoint{6.261853in}{1.631234in}}%
\pgfpathlineto{\pgfqpoint{6.262927in}{1.639616in}}%
\pgfpathlineto{\pgfqpoint{6.266149in}{1.639616in}}%
\pgfpathlineto{\pgfqpoint{6.267222in}{1.636244in}}%
\pgfpathlineto{\pgfqpoint{6.268296in}{1.636244in}}%
\pgfpathlineto{\pgfqpoint{6.269370in}{1.635377in}}%
\pgfpathlineto{\pgfqpoint{6.270444in}{1.636051in}}%
\pgfpathlineto{\pgfqpoint{6.273665in}{1.634895in}}%
\pgfpathlineto{\pgfqpoint{6.274739in}{1.639231in}}%
\pgfpathlineto{\pgfqpoint{6.275813in}{1.639616in}}%
\pgfpathlineto{\pgfqpoint{6.276887in}{1.638460in}}%
\pgfpathlineto{\pgfqpoint{6.277961in}{1.635184in}}%
\pgfpathlineto{\pgfqpoint{6.282256in}{1.638171in}}%
\pgfpathlineto{\pgfqpoint{6.283330in}{1.635666in}}%
\pgfpathlineto{\pgfqpoint{6.285478in}{1.626705in}}%
\pgfpathlineto{\pgfqpoint{6.288699in}{1.628825in}}%
\pgfpathlineto{\pgfqpoint{6.290847in}{1.633739in}}%
\pgfpathlineto{\pgfqpoint{6.292995in}{1.641640in}}%
\pgfpathlineto{\pgfqpoint{6.296216in}{1.636822in}}%
\pgfpathlineto{\pgfqpoint{6.297290in}{1.636340in}}%
\pgfpathlineto{\pgfqpoint{6.298364in}{1.633643in}}%
\pgfpathlineto{\pgfqpoint{6.299438in}{1.635955in}}%
\pgfpathlineto{\pgfqpoint{6.300511in}{1.635762in}}%
\pgfpathlineto{\pgfqpoint{6.303733in}{1.626513in}}%
\pgfpathlineto{\pgfqpoint{6.304807in}{1.626802in}}%
\pgfpathlineto{\pgfqpoint{6.305881in}{1.626513in}}%
\pgfpathlineto{\pgfqpoint{6.306954in}{1.623140in}}%
\pgfpathlineto{\pgfqpoint{6.308028in}{1.622851in}}%
\pgfpathlineto{\pgfqpoint{6.311250in}{1.605990in}}%
\pgfpathlineto{\pgfqpoint{6.312324in}{1.606279in}}%
\pgfpathlineto{\pgfqpoint{6.313397in}{1.605316in}}%
\pgfpathlineto{\pgfqpoint{6.315545in}{1.600209in}}%
\pgfpathlineto{\pgfqpoint{6.318767in}{1.598957in}}%
\pgfpathlineto{\pgfqpoint{6.319841in}{1.594236in}}%
\pgfpathlineto{\pgfqpoint{6.320914in}{1.593272in}}%
\pgfpathlineto{\pgfqpoint{6.323062in}{1.604930in}}%
\pgfpathlineto{\pgfqpoint{6.326284in}{1.611964in}}%
\pgfpathlineto{\pgfqpoint{6.327357in}{1.611771in}}%
\pgfpathlineto{\pgfqpoint{6.328431in}{1.619961in}}%
\pgfpathlineto{\pgfqpoint{6.330579in}{1.619190in}}%
\pgfpathlineto{\pgfqpoint{6.333800in}{1.624971in}}%
\pgfpathlineto{\pgfqpoint{6.337022in}{1.653780in}}%
\pgfpathlineto{\pgfqpoint{6.338096in}{1.656959in}}%
\pgfpathlineto{\pgfqpoint{6.341317in}{1.661199in}}%
\pgfpathlineto{\pgfqpoint{6.342391in}{1.653972in}}%
\pgfpathlineto{\pgfqpoint{6.344539in}{1.649540in}}%
\pgfpathlineto{\pgfqpoint{6.345613in}{1.655514in}}%
\pgfpathlineto{\pgfqpoint{6.348834in}{1.662258in}}%
\pgfpathlineto{\pgfqpoint{6.349908in}{1.674302in}}%
\pgfpathlineto{\pgfqpoint{6.350982in}{1.671604in}}%
\pgfpathlineto{\pgfqpoint{6.352056in}{1.666690in}}%
\pgfpathlineto{\pgfqpoint{6.353130in}{1.669581in}}%
\pgfpathlineto{\pgfqpoint{6.356351in}{1.674784in}}%
\pgfpathlineto{\pgfqpoint{6.357425in}{1.671026in}}%
\pgfpathlineto{\pgfqpoint{6.358499in}{1.670544in}}%
\pgfpathlineto{\pgfqpoint{6.360646in}{1.674302in}}%
\pgfpathlineto{\pgfqpoint{6.366016in}{1.675073in}}%
\pgfpathlineto{\pgfqpoint{6.367089in}{1.676422in}}%
\pgfpathlineto{\pgfqpoint{6.368163in}{1.671990in}}%
\pgfpathlineto{\pgfqpoint{6.368163in}{1.671990in}}%
\pgfusepath{stroke}%
\end{pgfscope}%
\begin{pgfscope}%
\pgfpathrectangle{\pgfqpoint{3.902254in}{1.293759in}}{\pgfqpoint{2.583333in}{0.400885in}}%
\pgfusepath{clip}%
\pgfsetroundcap%
\pgfsetroundjoin%
\pgfsetlinewidth{1.505625pt}%
\definecolor{currentstroke}{rgb}{0.498039,0.498039,0.498039}%
\pgfsetstrokecolor{currentstroke}%
\pgfsetdash{}{0pt}%
\pgfpathmoveto{\pgfqpoint{4.019678in}{1.469174in}}%
\pgfpathlineto{\pgfqpoint{4.021826in}{1.463875in}}%
\pgfpathlineto{\pgfqpoint{4.022900in}{1.463104in}}%
\pgfpathlineto{\pgfqpoint{4.026121in}{1.463393in}}%
\pgfpathlineto{\pgfqpoint{4.027195in}{1.464742in}}%
\pgfpathlineto{\pgfqpoint{4.028269in}{1.462429in}}%
\pgfpathlineto{\pgfqpoint{4.037934in}{1.461955in}}%
\pgfpathlineto{\pgfqpoint{4.041155in}{1.458087in}}%
\pgfpathlineto{\pgfqpoint{4.042229in}{1.454030in}}%
\pgfpathlineto{\pgfqpoint{4.043303in}{1.453276in}}%
\pgfpathlineto{\pgfqpoint{4.044377in}{1.450917in}}%
\pgfpathlineto{\pgfqpoint{4.045451in}{1.450068in}}%
\pgfpathlineto{\pgfqpoint{4.050820in}{1.454030in}}%
\pgfpathlineto{\pgfqpoint{4.051894in}{1.452427in}}%
\pgfpathlineto{\pgfqpoint{4.052967in}{1.454313in}}%
\pgfpathlineto{\pgfqpoint{4.056189in}{1.456295in}}%
\pgfpathlineto{\pgfqpoint{4.057263in}{1.454879in}}%
\pgfpathlineto{\pgfqpoint{4.059410in}{1.454879in}}%
\pgfpathlineto{\pgfqpoint{4.060484in}{1.453276in}}%
\pgfpathlineto{\pgfqpoint{4.063706in}{1.456295in}}%
\pgfpathlineto{\pgfqpoint{4.064780in}{1.455634in}}%
\pgfpathlineto{\pgfqpoint{4.065853in}{1.454219in}}%
\pgfpathlineto{\pgfqpoint{4.066927in}{1.455729in}}%
\pgfpathlineto{\pgfqpoint{4.068001in}{1.452993in}}%
\pgfpathlineto{\pgfqpoint{4.072296in}{1.452808in}}%
\pgfpathlineto{\pgfqpoint{4.073370in}{1.450964in}}%
\pgfpathlineto{\pgfqpoint{4.075518in}{1.450503in}}%
\pgfpathlineto{\pgfqpoint{4.080887in}{1.450318in}}%
\pgfpathlineto{\pgfqpoint{4.083035in}{1.446656in}}%
\pgfpathlineto{\pgfqpoint{4.086256in}{1.444507in}}%
\pgfpathlineto{\pgfqpoint{4.087330in}{1.446443in}}%
\pgfpathlineto{\pgfqpoint{4.088404in}{1.445281in}}%
\pgfpathlineto{\pgfqpoint{4.089478in}{1.443156in}}%
\pgfpathlineto{\pgfqpoint{4.090552in}{1.443852in}}%
\pgfpathlineto{\pgfqpoint{4.098069in}{1.440984in}}%
\pgfpathlineto{\pgfqpoint{4.104512in}{1.440381in}}%
\pgfpathlineto{\pgfqpoint{4.105585in}{1.441828in}}%
\pgfpathlineto{\pgfqpoint{4.108807in}{1.442430in}}%
\pgfpathlineto{\pgfqpoint{4.109881in}{1.446497in}}%
\pgfpathlineto{\pgfqpoint{4.112029in}{1.442739in}}%
\pgfpathlineto{\pgfqpoint{4.113102in}{1.443723in}}%
\pgfpathlineto{\pgfqpoint{4.116324in}{1.445602in}}%
\pgfpathlineto{\pgfqpoint{4.118472in}{1.444708in}}%
\pgfpathlineto{\pgfqpoint{4.119545in}{1.443276in}}%
\pgfpathlineto{\pgfqpoint{4.123841in}{1.441934in}}%
\pgfpathlineto{\pgfqpoint{4.124915in}{1.437728in}}%
\pgfpathlineto{\pgfqpoint{4.125988in}{1.441397in}}%
\pgfpathlineto{\pgfqpoint{4.127062in}{1.442560in}}%
\pgfpathlineto{\pgfqpoint{4.128136in}{1.440681in}}%
\pgfpathlineto{\pgfqpoint{4.131358in}{1.441755in}}%
\pgfpathlineto{\pgfqpoint{4.132431in}{1.443723in}}%
\pgfpathlineto{\pgfqpoint{4.133505in}{1.443276in}}%
\pgfpathlineto{\pgfqpoint{4.135653in}{1.436481in}}%
\pgfpathlineto{\pgfqpoint{4.138874in}{1.437496in}}%
\pgfpathlineto{\pgfqpoint{4.139948in}{1.431695in}}%
\pgfpathlineto{\pgfqpoint{4.141022in}{1.431857in}}%
\pgfpathlineto{\pgfqpoint{4.142096in}{1.427952in}}%
\pgfpathlineto{\pgfqpoint{4.149613in}{1.425145in}}%
\pgfpathlineto{\pgfqpoint{4.150687in}{1.427294in}}%
\pgfpathlineto{\pgfqpoint{4.154982in}{1.425653in}}%
\pgfpathlineto{\pgfqpoint{4.156056in}{1.427272in}}%
\pgfpathlineto{\pgfqpoint{4.157130in}{1.425629in}}%
\pgfpathlineto{\pgfqpoint{4.158204in}{1.422237in}}%
\pgfpathlineto{\pgfqpoint{4.161425in}{1.423658in}}%
\pgfpathlineto{\pgfqpoint{4.162499in}{1.422825in}}%
\pgfpathlineto{\pgfqpoint{4.163573in}{1.423727in}}%
\pgfpathlineto{\pgfqpoint{4.164647in}{1.420998in}}%
\pgfpathlineto{\pgfqpoint{4.165720in}{1.420184in}}%
\pgfpathlineto{\pgfqpoint{4.168942in}{1.421212in}}%
\pgfpathlineto{\pgfqpoint{4.170016in}{1.420916in}}%
\pgfpathlineto{\pgfqpoint{4.171090in}{1.421507in}}%
\pgfpathlineto{\pgfqpoint{4.173237in}{1.420617in}}%
\pgfpathlineto{\pgfqpoint{4.177533in}{1.418994in}}%
\pgfpathlineto{\pgfqpoint{4.178607in}{1.420738in}}%
\pgfpathlineto{\pgfqpoint{4.179680in}{1.419557in}}%
\pgfpathlineto{\pgfqpoint{4.180754in}{1.422771in}}%
\pgfpathlineto{\pgfqpoint{4.183976in}{1.421116in}}%
\pgfpathlineto{\pgfqpoint{4.185050in}{1.421857in}}%
\pgfpathlineto{\pgfqpoint{4.186123in}{1.418799in}}%
\pgfpathlineto{\pgfqpoint{4.187197in}{1.419452in}}%
\pgfpathlineto{\pgfqpoint{4.188271in}{1.415217in}}%
\pgfpathlineto{\pgfqpoint{4.191493in}{1.414655in}}%
\pgfpathlineto{\pgfqpoint{4.192566in}{1.412698in}}%
\pgfpathlineto{\pgfqpoint{4.193640in}{1.412492in}}%
\pgfpathlineto{\pgfqpoint{4.194714in}{1.408725in}}%
\pgfpathlineto{\pgfqpoint{4.195788in}{1.409651in}}%
\pgfpathlineto{\pgfqpoint{4.199009in}{1.408384in}}%
\pgfpathlineto{\pgfqpoint{4.200083in}{1.408779in}}%
\pgfpathlineto{\pgfqpoint{4.201157in}{1.410829in}}%
\pgfpathlineto{\pgfqpoint{4.202231in}{1.410695in}}%
\pgfpathlineto{\pgfqpoint{4.203305in}{1.407663in}}%
\pgfpathlineto{\pgfqpoint{4.206526in}{1.409103in}}%
\pgfpathlineto{\pgfqpoint{4.207600in}{1.408174in}}%
\pgfpathlineto{\pgfqpoint{4.208674in}{1.408437in}}%
\pgfpathlineto{\pgfqpoint{4.209748in}{1.407580in}}%
\pgfpathlineto{\pgfqpoint{4.210822in}{1.405356in}}%
\pgfpathlineto{\pgfqpoint{4.214043in}{1.403051in}}%
\pgfpathlineto{\pgfqpoint{4.218339in}{1.403113in}}%
\pgfpathlineto{\pgfqpoint{4.221560in}{1.401673in}}%
\pgfpathlineto{\pgfqpoint{4.222634in}{1.401858in}}%
\pgfpathlineto{\pgfqpoint{4.223708in}{1.400930in}}%
\pgfpathlineto{\pgfqpoint{4.224782in}{1.401972in}}%
\pgfpathlineto{\pgfqpoint{4.225855in}{1.399494in}}%
\pgfpathlineto{\pgfqpoint{4.229077in}{1.399252in}}%
\pgfpathlineto{\pgfqpoint{4.231225in}{1.396556in}}%
\pgfpathlineto{\pgfqpoint{4.232298in}{1.402307in}}%
\pgfpathlineto{\pgfqpoint{4.236594in}{1.403180in}}%
\pgfpathlineto{\pgfqpoint{4.237668in}{1.405878in}}%
\pgfpathlineto{\pgfqpoint{4.238741in}{1.406072in}}%
\pgfpathlineto{\pgfqpoint{4.240889in}{1.400588in}}%
\pgfpathlineto{\pgfqpoint{4.244111in}{1.400343in}}%
\pgfpathlineto{\pgfqpoint{4.246258in}{1.399186in}}%
\pgfpathlineto{\pgfqpoint{4.247332in}{1.401841in}}%
\pgfpathlineto{\pgfqpoint{4.248406in}{1.402523in}}%
\pgfpathlineto{\pgfqpoint{4.251628in}{1.401462in}}%
\pgfpathlineto{\pgfqpoint{4.253775in}{1.403822in}}%
\pgfpathlineto{\pgfqpoint{4.255923in}{1.401872in}}%
\pgfpathlineto{\pgfqpoint{4.259144in}{1.403360in}}%
\pgfpathlineto{\pgfqpoint{4.260218in}{1.403108in}}%
\pgfpathlineto{\pgfqpoint{4.262366in}{1.404052in}}%
\pgfpathlineto{\pgfqpoint{4.263440in}{1.404305in}}%
\pgfpathlineto{\pgfqpoint{4.266661in}{1.406020in}}%
\pgfpathlineto{\pgfqpoint{4.267735in}{1.409831in}}%
\pgfpathlineto{\pgfqpoint{4.269883in}{1.413004in}}%
\pgfpathlineto{\pgfqpoint{4.270957in}{1.408307in}}%
\pgfpathlineto{\pgfqpoint{4.275252in}{1.410762in}}%
\pgfpathlineto{\pgfqpoint{4.276326in}{1.408664in}}%
\pgfpathlineto{\pgfqpoint{4.277400in}{1.410256in}}%
\pgfpathlineto{\pgfqpoint{4.284917in}{1.403586in}}%
\pgfpathlineto{\pgfqpoint{4.285990in}{1.405545in}}%
\pgfpathlineto{\pgfqpoint{4.290286in}{1.403110in}}%
\pgfpathlineto{\pgfqpoint{4.292433in}{1.397039in}}%
\pgfpathlineto{\pgfqpoint{4.293507in}{1.401599in}}%
\pgfpathlineto{\pgfqpoint{4.296729in}{1.401351in}}%
\pgfpathlineto{\pgfqpoint{4.299950in}{1.397321in}}%
\pgfpathlineto{\pgfqpoint{4.301024in}{1.396667in}}%
\pgfpathlineto{\pgfqpoint{4.307467in}{1.396133in}}%
\pgfpathlineto{\pgfqpoint{4.308541in}{1.396955in}}%
\pgfpathlineto{\pgfqpoint{4.312836in}{1.395715in}}%
\pgfpathlineto{\pgfqpoint{4.313910in}{1.393902in}}%
\pgfpathlineto{\pgfqpoint{4.314984in}{1.389827in}}%
\pgfpathlineto{\pgfqpoint{4.316058in}{1.388562in}}%
\pgfpathlineto{\pgfqpoint{4.319279in}{1.390516in}}%
\pgfpathlineto{\pgfqpoint{4.323575in}{1.398754in}}%
\pgfpathlineto{\pgfqpoint{4.326796in}{1.399297in}}%
\pgfpathlineto{\pgfqpoint{4.327870in}{1.401179in}}%
\pgfpathlineto{\pgfqpoint{4.328944in}{1.398210in}}%
\pgfpathlineto{\pgfqpoint{4.330018in}{1.393526in}}%
\pgfpathlineto{\pgfqpoint{4.331092in}{1.396161in}}%
\pgfpathlineto{\pgfqpoint{4.334313in}{1.397868in}}%
\pgfpathlineto{\pgfqpoint{4.335387in}{1.400922in}}%
\pgfpathlineto{\pgfqpoint{4.336461in}{1.400181in}}%
\pgfpathlineto{\pgfqpoint{4.337535in}{1.400120in}}%
\pgfpathlineto{\pgfqpoint{4.338608in}{1.397975in}}%
\pgfpathlineto{\pgfqpoint{4.343978in}{1.398335in}}%
\pgfpathlineto{\pgfqpoint{4.345051in}{1.396107in}}%
\pgfpathlineto{\pgfqpoint{4.346125in}{1.398815in}}%
\pgfpathlineto{\pgfqpoint{4.349347in}{1.400267in}}%
\pgfpathlineto{\pgfqpoint{4.350421in}{1.399776in}}%
\pgfpathlineto{\pgfqpoint{4.352568in}{1.407549in}}%
\pgfpathlineto{\pgfqpoint{4.353642in}{1.407417in}}%
\pgfpathlineto{\pgfqpoint{4.357938in}{1.407879in}}%
\pgfpathlineto{\pgfqpoint{4.359011in}{1.409335in}}%
\pgfpathlineto{\pgfqpoint{4.361159in}{1.413537in}}%
\pgfpathlineto{\pgfqpoint{4.364381in}{1.406269in}}%
\pgfpathlineto{\pgfqpoint{4.365454in}{1.406269in}}%
\pgfpathlineto{\pgfqpoint{4.368676in}{1.401746in}}%
\pgfpathlineto{\pgfqpoint{4.371897in}{1.403870in}}%
\pgfpathlineto{\pgfqpoint{4.372971in}{1.405401in}}%
\pgfpathlineto{\pgfqpoint{4.375119in}{1.400536in}}%
\pgfpathlineto{\pgfqpoint{4.376193in}{1.400042in}}%
\pgfpathlineto{\pgfqpoint{4.379414in}{1.400104in}}%
\pgfpathlineto{\pgfqpoint{4.380488in}{1.402071in}}%
\pgfpathlineto{\pgfqpoint{4.382636in}{1.398466in}}%
\pgfpathlineto{\pgfqpoint{4.383710in}{1.398648in}}%
\pgfpathlineto{\pgfqpoint{4.386931in}{1.400345in}}%
\pgfpathlineto{\pgfqpoint{4.389079in}{1.396922in}}%
\pgfpathlineto{\pgfqpoint{4.391227in}{1.399500in}}%
\pgfpathlineto{\pgfqpoint{4.394448in}{1.399928in}}%
\pgfpathlineto{\pgfqpoint{4.395522in}{1.401034in}}%
\pgfpathlineto{\pgfqpoint{4.396596in}{1.403269in}}%
\pgfpathlineto{\pgfqpoint{4.397670in}{1.401237in}}%
\pgfpathlineto{\pgfqpoint{4.398743in}{1.402357in}}%
\pgfpathlineto{\pgfqpoint{4.405186in}{1.402798in}}%
\pgfpathlineto{\pgfqpoint{4.406260in}{1.405516in}}%
\pgfpathlineto{\pgfqpoint{4.409482in}{1.403698in}}%
\pgfpathlineto{\pgfqpoint{4.411629in}{1.399042in}}%
\pgfpathlineto{\pgfqpoint{4.412703in}{1.399956in}}%
\pgfpathlineto{\pgfqpoint{4.413777in}{1.398849in}}%
\pgfpathlineto{\pgfqpoint{4.416999in}{1.397146in}}%
\pgfpathlineto{\pgfqpoint{4.418073in}{1.401929in}}%
\pgfpathlineto{\pgfqpoint{4.419146in}{1.402368in}}%
\pgfpathlineto{\pgfqpoint{4.420220in}{1.399595in}}%
\pgfpathlineto{\pgfqpoint{4.425589in}{1.407193in}}%
\pgfpathlineto{\pgfqpoint{4.426663in}{1.409507in}}%
\pgfpathlineto{\pgfqpoint{4.428811in}{1.404362in}}%
\pgfpathlineto{\pgfqpoint{4.433106in}{1.402431in}}%
\pgfpathlineto{\pgfqpoint{4.435254in}{1.404079in}}%
\pgfpathlineto{\pgfqpoint{4.436328in}{1.403694in}}%
\pgfpathlineto{\pgfqpoint{4.439549in}{1.403183in}}%
\pgfpathlineto{\pgfqpoint{4.440623in}{1.399747in}}%
\pgfpathlineto{\pgfqpoint{4.442771in}{1.399255in}}%
\pgfpathlineto{\pgfqpoint{4.443845in}{1.394910in}}%
\pgfpathlineto{\pgfqpoint{4.448140in}{1.394909in}}%
\pgfpathlineto{\pgfqpoint{4.449214in}{1.394089in}}%
\pgfpathlineto{\pgfqpoint{4.451361in}{1.395780in}}%
\pgfpathlineto{\pgfqpoint{4.454583in}{1.395957in}}%
\pgfpathlineto{\pgfqpoint{4.456731in}{1.395071in}}%
\pgfpathlineto{\pgfqpoint{4.457805in}{1.395892in}}%
\pgfpathlineto{\pgfqpoint{4.458878in}{1.395596in}}%
\pgfpathlineto{\pgfqpoint{4.463174in}{1.395124in}}%
\pgfpathlineto{\pgfqpoint{4.465321in}{1.392439in}}%
\pgfpathlineto{\pgfqpoint{4.466395in}{1.391239in}}%
\pgfpathlineto{\pgfqpoint{4.469617in}{1.389885in}}%
\pgfpathlineto{\pgfqpoint{4.471764in}{1.387287in}}%
\pgfpathlineto{\pgfqpoint{4.477134in}{1.384234in}}%
\pgfpathlineto{\pgfqpoint{4.478207in}{1.381981in}}%
\pgfpathlineto{\pgfqpoint{4.479281in}{1.383566in}}%
\pgfpathlineto{\pgfqpoint{4.480355in}{1.382786in}}%
\pgfpathlineto{\pgfqpoint{4.481429in}{1.380929in}}%
\pgfpathlineto{\pgfqpoint{4.484650in}{1.381485in}}%
\pgfpathlineto{\pgfqpoint{4.485724in}{1.379401in}}%
\pgfpathlineto{\pgfqpoint{4.486798in}{1.380941in}}%
\pgfpathlineto{\pgfqpoint{4.487872in}{1.378919in}}%
\pgfpathlineto{\pgfqpoint{4.488946in}{1.380600in}}%
\pgfpathlineto{\pgfqpoint{4.493241in}{1.377147in}}%
\pgfpathlineto{\pgfqpoint{4.494315in}{1.378407in}}%
\pgfpathlineto{\pgfqpoint{4.496463in}{1.376889in}}%
\pgfpathlineto{\pgfqpoint{4.499684in}{1.376357in}}%
\pgfpathlineto{\pgfqpoint{4.500758in}{1.375204in}}%
\pgfpathlineto{\pgfqpoint{4.501832in}{1.377149in}}%
\pgfpathlineto{\pgfqpoint{4.502906in}{1.376373in}}%
\pgfpathlineto{\pgfqpoint{4.507201in}{1.376132in}}%
\pgfpathlineto{\pgfqpoint{4.508275in}{1.375126in}}%
\pgfpathlineto{\pgfqpoint{4.509349in}{1.376927in}}%
\pgfpathlineto{\pgfqpoint{4.511496in}{1.374859in}}%
\pgfpathlineto{\pgfqpoint{4.514718in}{1.373488in}}%
\pgfpathlineto{\pgfqpoint{4.515792in}{1.373767in}}%
\pgfpathlineto{\pgfqpoint{4.519013in}{1.368551in}}%
\pgfpathlineto{\pgfqpoint{4.523309in}{1.369918in}}%
\pgfpathlineto{\pgfqpoint{4.524383in}{1.372998in}}%
\pgfpathlineto{\pgfqpoint{4.526530in}{1.363806in}}%
\pgfpathlineto{\pgfqpoint{4.530826in}{1.363556in}}%
\pgfpathlineto{\pgfqpoint{4.531899in}{1.365213in}}%
\pgfpathlineto{\pgfqpoint{4.532973in}{1.360646in}}%
\pgfpathlineto{\pgfqpoint{4.534047in}{1.359406in}}%
\pgfpathlineto{\pgfqpoint{4.537269in}{1.359918in}}%
\pgfpathlineto{\pgfqpoint{4.538342in}{1.358570in}}%
\pgfpathlineto{\pgfqpoint{4.539416in}{1.363015in}}%
\pgfpathlineto{\pgfqpoint{4.541564in}{1.362151in}}%
\pgfpathlineto{\pgfqpoint{4.544785in}{1.364109in}}%
\pgfpathlineto{\pgfqpoint{4.545859in}{1.361351in}}%
\pgfpathlineto{\pgfqpoint{4.546933in}{1.360785in}}%
\pgfpathlineto{\pgfqpoint{4.548007in}{1.361989in}}%
\pgfpathlineto{\pgfqpoint{4.549081in}{1.361419in}}%
\pgfpathlineto{\pgfqpoint{4.552302in}{1.362470in}}%
\pgfpathlineto{\pgfqpoint{4.554450in}{1.359264in}}%
\pgfpathlineto{\pgfqpoint{4.555524in}{1.360445in}}%
\pgfpathlineto{\pgfqpoint{4.556598in}{1.360005in}}%
\pgfpathlineto{\pgfqpoint{4.559819in}{1.361834in}}%
\pgfpathlineto{\pgfqpoint{4.561967in}{1.365746in}}%
\pgfpathlineto{\pgfqpoint{4.563041in}{1.364422in}}%
\pgfpathlineto{\pgfqpoint{4.564115in}{1.365977in}}%
\pgfpathlineto{\pgfqpoint{4.568410in}{1.367817in}}%
\pgfpathlineto{\pgfqpoint{4.569484in}{1.371975in}}%
\pgfpathlineto{\pgfqpoint{4.571631in}{1.375788in}}%
\pgfpathlineto{\pgfqpoint{4.575927in}{1.374495in}}%
\pgfpathlineto{\pgfqpoint{4.577001in}{1.376436in}}%
\pgfpathlineto{\pgfqpoint{4.578074in}{1.370335in}}%
\pgfpathlineto{\pgfqpoint{4.579148in}{1.369386in}}%
\pgfpathlineto{\pgfqpoint{4.582370in}{1.368403in}}%
\pgfpathlineto{\pgfqpoint{4.584517in}{1.370536in}}%
\pgfpathlineto{\pgfqpoint{4.586665in}{1.366592in}}%
\pgfpathlineto{\pgfqpoint{4.589887in}{1.367760in}}%
\pgfpathlineto{\pgfqpoint{4.590961in}{1.364953in}}%
\pgfpathlineto{\pgfqpoint{4.593108in}{1.373476in}}%
\pgfpathlineto{\pgfqpoint{4.594182in}{1.371503in}}%
\pgfpathlineto{\pgfqpoint{4.597404in}{1.372880in}}%
\pgfpathlineto{\pgfqpoint{4.598477in}{1.368215in}}%
\pgfpathlineto{\pgfqpoint{4.600625in}{1.366368in}}%
\pgfpathlineto{\pgfqpoint{4.601699in}{1.368530in}}%
\pgfpathlineto{\pgfqpoint{4.604920in}{1.368441in}}%
\pgfpathlineto{\pgfqpoint{4.609216in}{1.365332in}}%
\pgfpathlineto{\pgfqpoint{4.612437in}{1.364093in}}%
\pgfpathlineto{\pgfqpoint{4.613511in}{1.364766in}}%
\pgfpathlineto{\pgfqpoint{4.614585in}{1.366124in}}%
\pgfpathlineto{\pgfqpoint{4.615659in}{1.363924in}}%
\pgfpathlineto{\pgfqpoint{4.616733in}{1.366486in}}%
\pgfpathlineto{\pgfqpoint{4.619954in}{1.367960in}}%
\pgfpathlineto{\pgfqpoint{4.622102in}{1.365377in}}%
\pgfpathlineto{\pgfqpoint{4.623176in}{1.367857in}}%
\pgfpathlineto{\pgfqpoint{4.624249in}{1.367946in}}%
\pgfpathlineto{\pgfqpoint{4.630693in}{1.365411in}}%
\pgfpathlineto{\pgfqpoint{4.631766in}{1.364383in}}%
\pgfpathlineto{\pgfqpoint{4.634988in}{1.362904in}}%
\pgfpathlineto{\pgfqpoint{4.637136in}{1.369383in}}%
\pgfpathlineto{\pgfqpoint{4.639283in}{1.366749in}}%
\pgfpathlineto{\pgfqpoint{4.643579in}{1.367273in}}%
\pgfpathlineto{\pgfqpoint{4.645726in}{1.368857in}}%
\pgfpathlineto{\pgfqpoint{4.646800in}{1.369884in}}%
\pgfpathlineto{\pgfqpoint{4.650022in}{1.368754in}}%
\pgfpathlineto{\pgfqpoint{4.652169in}{1.371359in}}%
\pgfpathlineto{\pgfqpoint{4.653243in}{1.372555in}}%
\pgfpathlineto{\pgfqpoint{4.654317in}{1.375540in}}%
\pgfpathlineto{\pgfqpoint{4.657538in}{1.376313in}}%
\pgfpathlineto{\pgfqpoint{4.658612in}{1.374754in}}%
\pgfpathlineto{\pgfqpoint{4.659686in}{1.377099in}}%
\pgfpathlineto{\pgfqpoint{4.660760in}{1.378033in}}%
\pgfpathlineto{\pgfqpoint{4.661834in}{1.375797in}}%
\pgfpathlineto{\pgfqpoint{4.665055in}{1.378268in}}%
\pgfpathlineto{\pgfqpoint{4.666129in}{1.378268in}}%
\pgfpathlineto{\pgfqpoint{4.667203in}{1.379713in}}%
\pgfpathlineto{\pgfqpoint{4.668277in}{1.374851in}}%
\pgfpathlineto{\pgfqpoint{4.669351in}{1.376483in}}%
\pgfpathlineto{\pgfqpoint{4.673646in}{1.381566in}}%
\pgfpathlineto{\pgfqpoint{4.674720in}{1.378563in}}%
\pgfpathlineto{\pgfqpoint{4.675794in}{1.379064in}}%
\pgfpathlineto{\pgfqpoint{4.676868in}{1.380223in}}%
\pgfpathlineto{\pgfqpoint{4.680089in}{1.381907in}}%
\pgfpathlineto{\pgfqpoint{4.681163in}{1.379671in}}%
\pgfpathlineto{\pgfqpoint{4.682237in}{1.379468in}}%
\pgfpathlineto{\pgfqpoint{4.684384in}{1.374764in}}%
\pgfpathlineto{\pgfqpoint{4.689754in}{1.371299in}}%
\pgfpathlineto{\pgfqpoint{4.690827in}{1.372037in}}%
\pgfpathlineto{\pgfqpoint{4.691901in}{1.374597in}}%
\pgfpathlineto{\pgfqpoint{4.695123in}{1.373878in}}%
\pgfpathlineto{\pgfqpoint{4.697271in}{1.377670in}}%
\pgfpathlineto{\pgfqpoint{4.698344in}{1.374939in}}%
\pgfpathlineto{\pgfqpoint{4.699418in}{1.377395in}}%
\pgfpathlineto{\pgfqpoint{4.702640in}{1.378633in}}%
\pgfpathlineto{\pgfqpoint{4.703714in}{1.377428in}}%
\pgfpathlineto{\pgfqpoint{4.704787in}{1.378171in}}%
\pgfpathlineto{\pgfqpoint{4.706935in}{1.376975in}}%
\pgfpathlineto{\pgfqpoint{4.710157in}{1.376285in}}%
\pgfpathlineto{\pgfqpoint{4.711230in}{1.378927in}}%
\pgfpathlineto{\pgfqpoint{4.712304in}{1.378372in}}%
\pgfpathlineto{\pgfqpoint{4.713378in}{1.375817in}}%
\pgfpathlineto{\pgfqpoint{4.714452in}{1.374941in}}%
\pgfpathlineto{\pgfqpoint{4.717673in}{1.376001in}}%
\pgfpathlineto{\pgfqpoint{4.718747in}{1.377855in}}%
\pgfpathlineto{\pgfqpoint{4.719821in}{1.374268in}}%
\pgfpathlineto{\pgfqpoint{4.721969in}{1.364352in}}%
\pgfpathlineto{\pgfqpoint{4.725190in}{1.362475in}}%
\pgfpathlineto{\pgfqpoint{4.726264in}{1.360683in}}%
\pgfpathlineto{\pgfqpoint{4.728412in}{1.362982in}}%
\pgfpathlineto{\pgfqpoint{4.729486in}{1.362018in}}%
\pgfpathlineto{\pgfqpoint{4.732707in}{1.362474in}}%
\pgfpathlineto{\pgfqpoint{4.733781in}{1.360806in}}%
\pgfpathlineto{\pgfqpoint{4.734855in}{1.362562in}}%
\pgfpathlineto{\pgfqpoint{4.737003in}{1.362688in}}%
\pgfpathlineto{\pgfqpoint{4.740224in}{1.360806in}}%
\pgfpathlineto{\pgfqpoint{4.741298in}{1.363870in}}%
\pgfpathlineto{\pgfqpoint{4.742372in}{1.362215in}}%
\pgfpathlineto{\pgfqpoint{4.743446in}{1.363629in}}%
\pgfpathlineto{\pgfqpoint{4.744519in}{1.363502in}}%
\pgfpathlineto{\pgfqpoint{4.747741in}{1.364305in}}%
\pgfpathlineto{\pgfqpoint{4.748815in}{1.363665in}}%
\pgfpathlineto{\pgfqpoint{4.749889in}{1.364215in}}%
\pgfpathlineto{\pgfqpoint{4.750962in}{1.363277in}}%
\pgfpathlineto{\pgfqpoint{4.752036in}{1.363151in}}%
\pgfpathlineto{\pgfqpoint{4.756332in}{1.361637in}}%
\pgfpathlineto{\pgfqpoint{4.757405in}{1.362835in}}%
\pgfpathlineto{\pgfqpoint{4.759553in}{1.363422in}}%
\pgfpathlineto{\pgfqpoint{4.767070in}{1.365378in}}%
\pgfpathlineto{\pgfqpoint{4.770292in}{1.366588in}}%
\pgfpathlineto{\pgfqpoint{4.771365in}{1.365448in}}%
\pgfpathlineto{\pgfqpoint{4.772439in}{1.366183in}}%
\pgfpathlineto{\pgfqpoint{4.773513in}{1.367755in}}%
\pgfpathlineto{\pgfqpoint{4.774587in}{1.365797in}}%
\pgfpathlineto{\pgfqpoint{4.777808in}{1.365493in}}%
\pgfpathlineto{\pgfqpoint{4.782104in}{1.371405in}}%
\pgfpathlineto{\pgfqpoint{4.785325in}{1.369918in}}%
\pgfpathlineto{\pgfqpoint{4.786399in}{1.372385in}}%
\pgfpathlineto{\pgfqpoint{4.787473in}{1.369093in}}%
\pgfpathlineto{\pgfqpoint{4.788547in}{1.369229in}}%
\pgfpathlineto{\pgfqpoint{4.789621in}{1.370453in}}%
\pgfpathlineto{\pgfqpoint{4.792842in}{1.368659in}}%
\pgfpathlineto{\pgfqpoint{4.793916in}{1.367307in}}%
\pgfpathlineto{\pgfqpoint{4.797137in}{1.366642in}}%
\pgfpathlineto{\pgfqpoint{4.803581in}{1.367214in}}%
\pgfpathlineto{\pgfqpoint{4.804654in}{1.369207in}}%
\pgfpathlineto{\pgfqpoint{4.807876in}{1.368255in}}%
\pgfpathlineto{\pgfqpoint{4.808950in}{1.366147in}}%
\pgfpathlineto{\pgfqpoint{4.810024in}{1.367066in}}%
\pgfpathlineto{\pgfqpoint{4.811097in}{1.370514in}}%
\pgfpathlineto{\pgfqpoint{4.812171in}{1.369593in}}%
\pgfpathlineto{\pgfqpoint{4.815393in}{1.372146in}}%
\pgfpathlineto{\pgfqpoint{4.816467in}{1.372005in}}%
\pgfpathlineto{\pgfqpoint{4.817540in}{1.367643in}}%
\pgfpathlineto{\pgfqpoint{4.818614in}{1.366751in}}%
\pgfpathlineto{\pgfqpoint{4.825057in}{1.370914in}}%
\pgfpathlineto{\pgfqpoint{4.826131in}{1.369014in}}%
\pgfpathlineto{\pgfqpoint{4.827205in}{1.369830in}}%
\pgfpathlineto{\pgfqpoint{4.830426in}{1.369601in}}%
\pgfpathlineto{\pgfqpoint{4.831500in}{1.370788in}}%
\pgfpathlineto{\pgfqpoint{4.832574in}{1.369584in}}%
\pgfpathlineto{\pgfqpoint{4.833648in}{1.369812in}}%
\pgfpathlineto{\pgfqpoint{4.834722in}{1.368393in}}%
\pgfpathlineto{\pgfqpoint{4.837943in}{1.374064in}}%
\pgfpathlineto{\pgfqpoint{4.839017in}{1.372524in}}%
\pgfpathlineto{\pgfqpoint{4.841165in}{1.373044in}}%
\pgfpathlineto{\pgfqpoint{4.842239in}{1.372569in}}%
\pgfpathlineto{\pgfqpoint{4.845460in}{1.372190in}}%
\pgfpathlineto{\pgfqpoint{4.847608in}{1.370501in}}%
\pgfpathlineto{\pgfqpoint{4.848682in}{1.370732in}}%
\pgfpathlineto{\pgfqpoint{4.849756in}{1.373604in}}%
\pgfpathlineto{\pgfqpoint{4.854051in}{1.375569in}}%
\pgfpathlineto{\pgfqpoint{4.855125in}{1.373461in}}%
\pgfpathlineto{\pgfqpoint{4.856199in}{1.367576in}}%
\pgfpathlineto{\pgfqpoint{4.857272in}{1.370527in}}%
\pgfpathlineto{\pgfqpoint{4.860494in}{1.374278in}}%
\pgfpathlineto{\pgfqpoint{4.862642in}{1.373795in}}%
\pgfpathlineto{\pgfqpoint{4.863715in}{1.369512in}}%
\pgfpathlineto{\pgfqpoint{4.864789in}{1.369237in}}%
\pgfpathlineto{\pgfqpoint{4.868011in}{1.370196in}}%
\pgfpathlineto{\pgfqpoint{4.869085in}{1.368071in}}%
\pgfpathlineto{\pgfqpoint{4.870159in}{1.369873in}}%
\pgfpathlineto{\pgfqpoint{4.871232in}{1.369597in}}%
\pgfpathlineto{\pgfqpoint{4.872306in}{1.370698in}}%
\pgfpathlineto{\pgfqpoint{4.875528in}{1.371116in}}%
\pgfpathlineto{\pgfqpoint{4.878749in}{1.374811in}}%
\pgfpathlineto{\pgfqpoint{4.883045in}{1.373789in}}%
\pgfpathlineto{\pgfqpoint{4.884118in}{1.372296in}}%
\pgfpathlineto{\pgfqpoint{4.885192in}{1.373574in}}%
\pgfpathlineto{\pgfqpoint{4.886266in}{1.370354in}}%
\pgfpathlineto{\pgfqpoint{4.887340in}{1.371465in}}%
\pgfpathlineto{\pgfqpoint{4.890561in}{1.371090in}}%
\pgfpathlineto{\pgfqpoint{4.891635in}{1.370343in}}%
\pgfpathlineto{\pgfqpoint{4.892709in}{1.371084in}}%
\pgfpathlineto{\pgfqpoint{4.893783in}{1.368608in}}%
\pgfpathlineto{\pgfqpoint{4.894857in}{1.369561in}}%
\pgfpathlineto{\pgfqpoint{4.900226in}{1.368145in}}%
\pgfpathlineto{\pgfqpoint{4.901300in}{1.367107in}}%
\pgfpathlineto{\pgfqpoint{4.902374in}{1.367375in}}%
\pgfpathlineto{\pgfqpoint{4.905595in}{1.367106in}}%
\pgfpathlineto{\pgfqpoint{4.906669in}{1.364922in}}%
\pgfpathlineto{\pgfqpoint{4.907743in}{1.365704in}}%
\pgfpathlineto{\pgfqpoint{4.909891in}{1.368846in}}%
\pgfpathlineto{\pgfqpoint{4.913112in}{1.368163in}}%
\pgfpathlineto{\pgfqpoint{4.914186in}{1.369382in}}%
\pgfpathlineto{\pgfqpoint{4.915260in}{1.368741in}}%
\pgfpathlineto{\pgfqpoint{4.916334in}{1.366967in}}%
\pgfpathlineto{\pgfqpoint{4.921703in}{1.365850in}}%
\pgfpathlineto{\pgfqpoint{4.922777in}{1.367520in}}%
\pgfpathlineto{\pgfqpoint{4.923850in}{1.371600in}}%
\pgfpathlineto{\pgfqpoint{4.924924in}{1.372871in}}%
\pgfpathlineto{\pgfqpoint{4.929220in}{1.369737in}}%
\pgfpathlineto{\pgfqpoint{4.930293in}{1.369875in}}%
\pgfpathlineto{\pgfqpoint{4.931367in}{1.368121in}}%
\pgfpathlineto{\pgfqpoint{4.932441in}{1.368483in}}%
\pgfpathlineto{\pgfqpoint{4.936737in}{1.367215in}}%
\pgfpathlineto{\pgfqpoint{4.938884in}{1.363673in}}%
\pgfpathlineto{\pgfqpoint{4.939958in}{1.363673in}}%
\pgfpathlineto{\pgfqpoint{4.943180in}{1.364787in}}%
\pgfpathlineto{\pgfqpoint{4.944253in}{1.365830in}}%
\pgfpathlineto{\pgfqpoint{4.945327in}{1.365258in}}%
\pgfpathlineto{\pgfqpoint{4.946401in}{1.365433in}}%
\pgfpathlineto{\pgfqpoint{4.947475in}{1.361624in}}%
\pgfpathlineto{\pgfqpoint{4.950696in}{1.361373in}}%
\pgfpathlineto{\pgfqpoint{4.951770in}{1.362998in}}%
\pgfpathlineto{\pgfqpoint{4.953918in}{1.360339in}}%
\pgfpathlineto{\pgfqpoint{4.954992in}{1.359392in}}%
\pgfpathlineto{\pgfqpoint{4.961435in}{1.359471in}}%
\pgfpathlineto{\pgfqpoint{4.962509in}{1.358698in}}%
\pgfpathlineto{\pgfqpoint{4.965730in}{1.358415in}}%
\pgfpathlineto{\pgfqpoint{4.966804in}{1.360826in}}%
\pgfpathlineto{\pgfqpoint{4.967878in}{1.361282in}}%
\pgfpathlineto{\pgfqpoint{4.970026in}{1.360368in}}%
\pgfpathlineto{\pgfqpoint{4.974321in}{1.360038in}}%
\pgfpathlineto{\pgfqpoint{4.977542in}{1.361143in}}%
\pgfpathlineto{\pgfqpoint{4.988281in}{1.359361in}}%
\pgfpathlineto{\pgfqpoint{4.989355in}{1.360785in}}%
\pgfpathlineto{\pgfqpoint{4.990428in}{1.360371in}}%
\pgfpathlineto{\pgfqpoint{4.991502in}{1.361030in}}%
\pgfpathlineto{\pgfqpoint{4.992576in}{1.360656in}}%
\pgfpathlineto{\pgfqpoint{4.995798in}{1.361938in}}%
\pgfpathlineto{\pgfqpoint{4.997945in}{1.359518in}}%
\pgfpathlineto{\pgfqpoint{5.003314in}{1.359233in}}%
\pgfpathlineto{\pgfqpoint{5.004388in}{1.360737in}}%
\pgfpathlineto{\pgfqpoint{5.005462in}{1.360281in}}%
\pgfpathlineto{\pgfqpoint{5.007610in}{1.355695in}}%
\pgfpathlineto{\pgfqpoint{5.011905in}{1.354495in}}%
\pgfpathlineto{\pgfqpoint{5.012979in}{1.353730in}}%
\pgfpathlineto{\pgfqpoint{5.014053in}{1.355662in}}%
\pgfpathlineto{\pgfqpoint{5.015127in}{1.354302in}}%
\pgfpathlineto{\pgfqpoint{5.018348in}{1.354455in}}%
\pgfpathlineto{\pgfqpoint{5.019422in}{1.353614in}}%
\pgfpathlineto{\pgfqpoint{5.020496in}{1.353841in}}%
\pgfpathlineto{\pgfqpoint{5.025865in}{1.351808in}}%
\pgfpathlineto{\pgfqpoint{5.026939in}{1.350663in}}%
\pgfpathlineto{\pgfqpoint{5.028013in}{1.351281in}}%
\pgfpathlineto{\pgfqpoint{5.029087in}{1.355209in}}%
\pgfpathlineto{\pgfqpoint{5.030160in}{1.357026in}}%
\pgfpathlineto{\pgfqpoint{5.033382in}{1.355799in}}%
\pgfpathlineto{\pgfqpoint{5.036603in}{1.360774in}}%
\pgfpathlineto{\pgfqpoint{5.037677in}{1.360608in}}%
\pgfpathlineto{\pgfqpoint{5.040899in}{1.360690in}}%
\pgfpathlineto{\pgfqpoint{5.044120in}{1.359617in}}%
\pgfpathlineto{\pgfqpoint{5.045194in}{1.360272in}}%
\pgfpathlineto{\pgfqpoint{5.052711in}{1.360804in}}%
\pgfpathlineto{\pgfqpoint{5.058080in}{1.358209in}}%
\pgfpathlineto{\pgfqpoint{5.059154in}{1.358249in}}%
\pgfpathlineto{\pgfqpoint{5.060228in}{1.356962in}}%
\pgfpathlineto{\pgfqpoint{5.066671in}{1.357277in}}%
\pgfpathlineto{\pgfqpoint{5.067745in}{1.356562in}}%
\pgfpathlineto{\pgfqpoint{5.070966in}{1.357861in}}%
\pgfpathlineto{\pgfqpoint{5.072040in}{1.359863in}}%
\pgfpathlineto{\pgfqpoint{5.073114in}{1.360355in}}%
\pgfpathlineto{\pgfqpoint{5.074188in}{1.359488in}}%
\pgfpathlineto{\pgfqpoint{5.075262in}{1.361490in}}%
\pgfpathlineto{\pgfqpoint{5.078483in}{1.360946in}}%
\pgfpathlineto{\pgfqpoint{5.082779in}{1.355148in}}%
\pgfpathlineto{\pgfqpoint{5.086000in}{1.355690in}}%
\pgfpathlineto{\pgfqpoint{5.087074in}{1.356508in}}%
\pgfpathlineto{\pgfqpoint{5.088148in}{1.356036in}}%
\pgfpathlineto{\pgfqpoint{5.089222in}{1.357445in}}%
\pgfpathlineto{\pgfqpoint{5.090295in}{1.356966in}}%
\pgfpathlineto{\pgfqpoint{5.093517in}{1.357006in}}%
\pgfpathlineto{\pgfqpoint{5.094591in}{1.356253in}}%
\pgfpathlineto{\pgfqpoint{5.095665in}{1.358020in}}%
\pgfpathlineto{\pgfqpoint{5.096738in}{1.358461in}}%
\pgfpathlineto{\pgfqpoint{5.097812in}{1.357088in}}%
\pgfpathlineto{\pgfqpoint{5.101034in}{1.355937in}}%
\pgfpathlineto{\pgfqpoint{5.102108in}{1.357189in}}%
\pgfpathlineto{\pgfqpoint{5.103181in}{1.354804in}}%
\pgfpathlineto{\pgfqpoint{5.104255in}{1.357735in}}%
\pgfpathlineto{\pgfqpoint{5.105329in}{1.357694in}}%
\pgfpathlineto{\pgfqpoint{5.111772in}{1.361851in}}%
\pgfpathlineto{\pgfqpoint{5.112846in}{1.360502in}}%
\pgfpathlineto{\pgfqpoint{5.117141in}{1.358440in}}%
\pgfpathlineto{\pgfqpoint{5.118215in}{1.359693in}}%
\pgfpathlineto{\pgfqpoint{5.119289in}{1.359980in}}%
\pgfpathlineto{\pgfqpoint{5.120363in}{1.358167in}}%
\pgfpathlineto{\pgfqpoint{5.123584in}{1.356032in}}%
\pgfpathlineto{\pgfqpoint{5.124658in}{1.354346in}}%
\pgfpathlineto{\pgfqpoint{5.126806in}{1.354537in}}%
\pgfpathlineto{\pgfqpoint{5.127880in}{1.353460in}}%
\pgfpathlineto{\pgfqpoint{5.133249in}{1.353155in}}%
\pgfpathlineto{\pgfqpoint{5.134323in}{1.353344in}}%
\pgfpathlineto{\pgfqpoint{5.135397in}{1.351564in}}%
\pgfpathlineto{\pgfqpoint{5.139692in}{1.352343in}}%
\pgfpathlineto{\pgfqpoint{5.141840in}{1.350558in}}%
\pgfpathlineto{\pgfqpoint{5.142914in}{1.349681in}}%
\pgfpathlineto{\pgfqpoint{5.147209in}{1.350478in}}%
\pgfpathlineto{\pgfqpoint{5.148283in}{1.352630in}}%
\pgfpathlineto{\pgfqpoint{5.149357in}{1.353531in}}%
\pgfpathlineto{\pgfqpoint{5.150430in}{1.353493in}}%
\pgfpathlineto{\pgfqpoint{5.154726in}{1.356166in}}%
\pgfpathlineto{\pgfqpoint{5.155800in}{1.353925in}}%
\pgfpathlineto{\pgfqpoint{5.157947in}{1.352244in}}%
\pgfpathlineto{\pgfqpoint{5.161169in}{1.353926in}}%
\pgfpathlineto{\pgfqpoint{5.162243in}{1.356753in}}%
\pgfpathlineto{\pgfqpoint{5.163316in}{1.357784in}}%
\pgfpathlineto{\pgfqpoint{5.165464in}{1.358387in}}%
\pgfpathlineto{\pgfqpoint{5.168686in}{1.357415in}}%
\pgfpathlineto{\pgfqpoint{5.169759in}{1.363812in}}%
\pgfpathlineto{\pgfqpoint{5.170833in}{1.366371in}}%
\pgfpathlineto{\pgfqpoint{5.171907in}{1.365745in}}%
\pgfpathlineto{\pgfqpoint{5.172981in}{1.368542in}}%
\pgfpathlineto{\pgfqpoint{5.177276in}{1.368723in}}%
\pgfpathlineto{\pgfqpoint{5.179424in}{1.363151in}}%
\pgfpathlineto{\pgfqpoint{5.180498in}{1.363280in}}%
\pgfpathlineto{\pgfqpoint{5.185867in}{1.361006in}}%
\pgfpathlineto{\pgfqpoint{5.188015in}{1.360377in}}%
\pgfpathlineto{\pgfqpoint{5.192310in}{1.362176in}}%
\pgfpathlineto{\pgfqpoint{5.193384in}{1.364049in}}%
\pgfpathlineto{\pgfqpoint{5.195532in}{1.363395in}}%
\pgfpathlineto{\pgfqpoint{5.198753in}{1.364778in}}%
\pgfpathlineto{\pgfqpoint{5.199827in}{1.363108in}}%
\pgfpathlineto{\pgfqpoint{5.200901in}{1.364142in}}%
\pgfpathlineto{\pgfqpoint{5.201975in}{1.360652in}}%
\pgfpathlineto{\pgfqpoint{5.203048in}{1.362074in}}%
\pgfpathlineto{\pgfqpoint{5.207344in}{1.360632in}}%
\pgfpathlineto{\pgfqpoint{5.208418in}{1.361342in}}%
\pgfpathlineto{\pgfqpoint{5.209491in}{1.360878in}}%
\pgfpathlineto{\pgfqpoint{5.210565in}{1.357860in}}%
\pgfpathlineto{\pgfqpoint{5.215935in}{1.357013in}}%
\pgfpathlineto{\pgfqpoint{5.218082in}{1.360608in}}%
\pgfpathlineto{\pgfqpoint{5.221304in}{1.361277in}}%
\pgfpathlineto{\pgfqpoint{5.223451in}{1.364424in}}%
\pgfpathlineto{\pgfqpoint{5.224525in}{1.364205in}}%
\pgfpathlineto{\pgfqpoint{5.225599in}{1.365648in}}%
\pgfpathlineto{\pgfqpoint{5.228821in}{1.361021in}}%
\pgfpathlineto{\pgfqpoint{5.229894in}{1.358118in}}%
\pgfpathlineto{\pgfqpoint{5.232042in}{1.357996in}}%
\pgfpathlineto{\pgfqpoint{5.233116in}{1.353174in}}%
\pgfpathlineto{\pgfqpoint{5.236337in}{1.353899in}}%
\pgfpathlineto{\pgfqpoint{5.238485in}{1.351644in}}%
\pgfpathlineto{\pgfqpoint{5.240633in}{1.353184in}}%
\pgfpathlineto{\pgfqpoint{5.244928in}{1.353603in}}%
\pgfpathlineto{\pgfqpoint{5.247076in}{1.354448in}}%
\pgfpathlineto{\pgfqpoint{5.248150in}{1.354254in}}%
\pgfpathlineto{\pgfqpoint{5.251371in}{1.354873in}}%
\pgfpathlineto{\pgfqpoint{5.252445in}{1.353470in}}%
\pgfpathlineto{\pgfqpoint{5.253519in}{1.353508in}}%
\pgfpathlineto{\pgfqpoint{5.255667in}{1.352744in}}%
\pgfpathlineto{\pgfqpoint{5.259962in}{1.352440in}}%
\pgfpathlineto{\pgfqpoint{5.261036in}{1.353913in}}%
\pgfpathlineto{\pgfqpoint{5.262110in}{1.354375in}}%
\pgfpathlineto{\pgfqpoint{5.263183in}{1.356351in}}%
\pgfpathlineto{\pgfqpoint{5.266405in}{1.356550in}}%
\pgfpathlineto{\pgfqpoint{5.267479in}{1.358900in}}%
\pgfpathlineto{\pgfqpoint{5.268553in}{1.358325in}}%
\pgfpathlineto{\pgfqpoint{5.269626in}{1.354820in}}%
\pgfpathlineto{\pgfqpoint{5.273922in}{1.353110in}}%
\pgfpathlineto{\pgfqpoint{5.274996in}{1.354140in}}%
\pgfpathlineto{\pgfqpoint{5.276069in}{1.352246in}}%
\pgfpathlineto{\pgfqpoint{5.277143in}{1.353000in}}%
\pgfpathlineto{\pgfqpoint{5.278217in}{1.352200in}}%
\pgfpathlineto{\pgfqpoint{5.281439in}{1.351936in}}%
\pgfpathlineto{\pgfqpoint{5.282513in}{1.352650in}}%
\pgfpathlineto{\pgfqpoint{5.284660in}{1.355749in}}%
\pgfpathlineto{\pgfqpoint{5.288956in}{1.353509in}}%
\pgfpathlineto{\pgfqpoint{5.290029in}{1.355044in}}%
\pgfpathlineto{\pgfqpoint{5.291103in}{1.354104in}}%
\pgfpathlineto{\pgfqpoint{5.292177in}{1.352364in}}%
\pgfpathlineto{\pgfqpoint{5.296472in}{1.351873in}}%
\pgfpathlineto{\pgfqpoint{5.297546in}{1.352962in}}%
\pgfpathlineto{\pgfqpoint{5.298620in}{1.351705in}}%
\pgfpathlineto{\pgfqpoint{5.299694in}{1.352079in}}%
\pgfpathlineto{\pgfqpoint{5.300768in}{1.351401in}}%
\pgfpathlineto{\pgfqpoint{5.303989in}{1.351961in}}%
\pgfpathlineto{\pgfqpoint{5.307211in}{1.351245in}}%
\pgfpathlineto{\pgfqpoint{5.308285in}{1.352400in}}%
\pgfpathlineto{\pgfqpoint{5.311506in}{1.350887in}}%
\pgfpathlineto{\pgfqpoint{5.312580in}{1.351517in}}%
\pgfpathlineto{\pgfqpoint{5.314728in}{1.348848in}}%
\pgfpathlineto{\pgfqpoint{5.319023in}{1.348740in}}%
\pgfpathlineto{\pgfqpoint{5.320097in}{1.347370in}}%
\pgfpathlineto{\pgfqpoint{5.323318in}{1.347760in}}%
\pgfpathlineto{\pgfqpoint{5.326540in}{1.347440in}}%
\pgfpathlineto{\pgfqpoint{5.328688in}{1.350048in}}%
\pgfpathlineto{\pgfqpoint{5.329761in}{1.349755in}}%
\pgfpathlineto{\pgfqpoint{5.330835in}{1.348512in}}%
\pgfpathlineto{\pgfqpoint{5.336204in}{1.349735in}}%
\pgfpathlineto{\pgfqpoint{5.337278in}{1.349004in}}%
\pgfpathlineto{\pgfqpoint{5.338352in}{1.349547in}}%
\pgfpathlineto{\pgfqpoint{5.342647in}{1.350277in}}%
\pgfpathlineto{\pgfqpoint{5.344795in}{1.349213in}}%
\pgfpathlineto{\pgfqpoint{5.345869in}{1.350084in}}%
\pgfpathlineto{\pgfqpoint{5.350164in}{1.350672in}}%
\pgfpathlineto{\pgfqpoint{5.352312in}{1.350301in}}%
\pgfpathlineto{\pgfqpoint{5.358755in}{1.351709in}}%
\pgfpathlineto{\pgfqpoint{5.360903in}{1.357521in}}%
\pgfpathlineto{\pgfqpoint{5.367346in}{1.355789in}}%
\pgfpathlineto{\pgfqpoint{5.368420in}{1.357452in}}%
\pgfpathlineto{\pgfqpoint{5.371641in}{1.358262in}}%
\pgfpathlineto{\pgfqpoint{5.372715in}{1.356872in}}%
\pgfpathlineto{\pgfqpoint{5.373789in}{1.357354in}}%
\pgfpathlineto{\pgfqpoint{5.374863in}{1.355697in}}%
\pgfpathlineto{\pgfqpoint{5.375936in}{1.356686in}}%
\pgfpathlineto{\pgfqpoint{5.379158in}{1.356486in}}%
\pgfpathlineto{\pgfqpoint{5.380232in}{1.355687in}}%
\pgfpathlineto{\pgfqpoint{5.381306in}{1.357230in}}%
\pgfpathlineto{\pgfqpoint{5.383453in}{1.356144in}}%
\pgfpathlineto{\pgfqpoint{5.387749in}{1.359466in}}%
\pgfpathlineto{\pgfqpoint{5.389896in}{1.357484in}}%
\pgfpathlineto{\pgfqpoint{5.394192in}{1.358010in}}%
\pgfpathlineto{\pgfqpoint{5.395266in}{1.357317in}}%
\pgfpathlineto{\pgfqpoint{5.396339in}{1.357721in}}%
\pgfpathlineto{\pgfqpoint{5.397413in}{1.358777in}}%
\pgfpathlineto{\pgfqpoint{5.398487in}{1.356266in}}%
\pgfpathlineto{\pgfqpoint{5.403856in}{1.355150in}}%
\pgfpathlineto{\pgfqpoint{5.404930in}{1.353499in}}%
\pgfpathlineto{\pgfqpoint{5.406004in}{1.354268in}}%
\pgfpathlineto{\pgfqpoint{5.409225in}{1.352635in}}%
\pgfpathlineto{\pgfqpoint{5.411373in}{1.357930in}}%
\pgfpathlineto{\pgfqpoint{5.412447in}{1.358297in}}%
\pgfpathlineto{\pgfqpoint{5.413521in}{1.359362in}}%
\pgfpathlineto{\pgfqpoint{5.417816in}{1.359858in}}%
\pgfpathlineto{\pgfqpoint{5.418890in}{1.357518in}}%
\pgfpathlineto{\pgfqpoint{5.424259in}{1.356146in}}%
\pgfpathlineto{\pgfqpoint{5.425333in}{1.357143in}}%
\pgfpathlineto{\pgfqpoint{5.427481in}{1.357264in}}%
\pgfpathlineto{\pgfqpoint{5.428555in}{1.358195in}}%
\pgfpathlineto{\pgfqpoint{5.433924in}{1.353209in}}%
\pgfpathlineto{\pgfqpoint{5.436071in}{1.354365in}}%
\pgfpathlineto{\pgfqpoint{5.441441in}{1.354521in}}%
\pgfpathlineto{\pgfqpoint{5.446810in}{1.363634in}}%
\pgfpathlineto{\pgfqpoint{5.447884in}{1.368153in}}%
\pgfpathlineto{\pgfqpoint{5.450031in}{1.358059in}}%
\pgfpathlineto{\pgfqpoint{5.451105in}{1.358469in}}%
\pgfpathlineto{\pgfqpoint{5.454327in}{1.358675in}}%
\pgfpathlineto{\pgfqpoint{5.455401in}{1.362523in}}%
\pgfpathlineto{\pgfqpoint{5.457548in}{1.359554in}}%
\pgfpathlineto{\pgfqpoint{5.458622in}{1.362693in}}%
\pgfpathlineto{\pgfqpoint{5.462917in}{1.358862in}}%
\pgfpathlineto{\pgfqpoint{5.463991in}{1.360565in}}%
\pgfpathlineto{\pgfqpoint{5.465065in}{1.360310in}}%
\pgfpathlineto{\pgfqpoint{5.466139in}{1.359337in}}%
\pgfpathlineto{\pgfqpoint{5.469360in}{1.359630in}}%
\pgfpathlineto{\pgfqpoint{5.470434in}{1.357113in}}%
\pgfpathlineto{\pgfqpoint{5.471508in}{1.357722in}}%
\pgfpathlineto{\pgfqpoint{5.473656in}{1.363348in}}%
\pgfpathlineto{\pgfqpoint{5.476877in}{1.362513in}}%
\pgfpathlineto{\pgfqpoint{5.479025in}{1.365499in}}%
\pgfpathlineto{\pgfqpoint{5.481173in}{1.364598in}}%
\pgfpathlineto{\pgfqpoint{5.486542in}{1.367298in}}%
\pgfpathlineto{\pgfqpoint{5.487616in}{1.365222in}}%
\pgfpathlineto{\pgfqpoint{5.488690in}{1.364761in}}%
\pgfpathlineto{\pgfqpoint{5.491911in}{1.360332in}}%
\pgfpathlineto{\pgfqpoint{5.492985in}{1.360113in}}%
\pgfpathlineto{\pgfqpoint{5.495133in}{1.357433in}}%
\pgfpathlineto{\pgfqpoint{5.496206in}{1.357686in}}%
\pgfpathlineto{\pgfqpoint{5.500502in}{1.356968in}}%
\pgfpathlineto{\pgfqpoint{5.501576in}{1.358267in}}%
\pgfpathlineto{\pgfqpoint{5.502649in}{1.355796in}}%
\pgfpathlineto{\pgfqpoint{5.506945in}{1.355713in}}%
\pgfpathlineto{\pgfqpoint{5.508019in}{1.353816in}}%
\pgfpathlineto{\pgfqpoint{5.509092in}{1.355104in}}%
\pgfpathlineto{\pgfqpoint{5.510166in}{1.351542in}}%
\pgfpathlineto{\pgfqpoint{5.511240in}{1.350682in}}%
\pgfpathlineto{\pgfqpoint{5.514462in}{1.350024in}}%
\pgfpathlineto{\pgfqpoint{5.515535in}{1.350638in}}%
\pgfpathlineto{\pgfqpoint{5.516609in}{1.349633in}}%
\pgfpathlineto{\pgfqpoint{5.517683in}{1.349862in}}%
\pgfpathlineto{\pgfqpoint{5.518757in}{1.348332in}}%
\pgfpathlineto{\pgfqpoint{5.521979in}{1.348632in}}%
\pgfpathlineto{\pgfqpoint{5.524126in}{1.350677in}}%
\pgfpathlineto{\pgfqpoint{5.525200in}{1.350484in}}%
\pgfpathlineto{\pgfqpoint{5.526274in}{1.351873in}}%
\pgfpathlineto{\pgfqpoint{5.530569in}{1.354126in}}%
\pgfpathlineto{\pgfqpoint{5.531643in}{1.353398in}}%
\pgfpathlineto{\pgfqpoint{5.533791in}{1.357130in}}%
\pgfpathlineto{\pgfqpoint{5.537012in}{1.354227in}}%
\pgfpathlineto{\pgfqpoint{5.538086in}{1.354105in}}%
\pgfpathlineto{\pgfqpoint{5.540234in}{1.351775in}}%
\pgfpathlineto{\pgfqpoint{5.541308in}{1.353031in}}%
\pgfpathlineto{\pgfqpoint{5.544529in}{1.354349in}}%
\pgfpathlineto{\pgfqpoint{5.545603in}{1.353699in}}%
\pgfpathlineto{\pgfqpoint{5.546677in}{1.354585in}}%
\pgfpathlineto{\pgfqpoint{5.548824in}{1.353526in}}%
\pgfpathlineto{\pgfqpoint{5.553120in}{1.352364in}}%
\pgfpathlineto{\pgfqpoint{5.555267in}{1.355805in}}%
\pgfpathlineto{\pgfqpoint{5.556341in}{1.351788in}}%
\pgfpathlineto{\pgfqpoint{5.559563in}{1.350608in}}%
\pgfpathlineto{\pgfqpoint{5.561711in}{1.352910in}}%
\pgfpathlineto{\pgfqpoint{5.562784in}{1.353070in}}%
\pgfpathlineto{\pgfqpoint{5.563858in}{1.354750in}}%
\pgfpathlineto{\pgfqpoint{5.568154in}{1.352225in}}%
\pgfpathlineto{\pgfqpoint{5.569227in}{1.349019in}}%
\pgfpathlineto{\pgfqpoint{5.571375in}{1.352085in}}%
\pgfpathlineto{\pgfqpoint{5.574597in}{1.350979in}}%
\pgfpathlineto{\pgfqpoint{5.576744in}{1.347550in}}%
\pgfpathlineto{\pgfqpoint{5.577818in}{1.348294in}}%
\pgfpathlineto{\pgfqpoint{5.582113in}{1.348181in}}%
\pgfpathlineto{\pgfqpoint{5.583187in}{1.346717in}}%
\pgfpathlineto{\pgfqpoint{5.585335in}{1.349841in}}%
\pgfpathlineto{\pgfqpoint{5.589630in}{1.350955in}}%
\pgfpathlineto{\pgfqpoint{5.590704in}{1.348889in}}%
\pgfpathlineto{\pgfqpoint{5.593926in}{1.352510in}}%
\pgfpathlineto{\pgfqpoint{5.597147in}{1.351635in}}%
\pgfpathlineto{\pgfqpoint{5.598221in}{1.352186in}}%
\pgfpathlineto{\pgfqpoint{5.599295in}{1.354841in}}%
\pgfpathlineto{\pgfqpoint{5.600369in}{1.352297in}}%
\pgfpathlineto{\pgfqpoint{5.601443in}{1.353806in}}%
\pgfpathlineto{\pgfqpoint{5.605738in}{1.352266in}}%
\pgfpathlineto{\pgfqpoint{5.606812in}{1.353775in}}%
\pgfpathlineto{\pgfqpoint{5.608959in}{1.344960in}}%
\pgfpathlineto{\pgfqpoint{5.612181in}{1.344996in}}%
\pgfpathlineto{\pgfqpoint{5.614329in}{1.338951in}}%
\pgfpathlineto{\pgfqpoint{5.615402in}{1.339017in}}%
\pgfpathlineto{\pgfqpoint{5.616476in}{1.336296in}}%
\pgfpathlineto{\pgfqpoint{5.619698in}{1.334126in}}%
\pgfpathlineto{\pgfqpoint{5.620772in}{1.336382in}}%
\pgfpathlineto{\pgfqpoint{5.621845in}{1.334432in}}%
\pgfpathlineto{\pgfqpoint{5.622919in}{1.334960in}}%
\pgfpathlineto{\pgfqpoint{5.623993in}{1.333489in}}%
\pgfpathlineto{\pgfqpoint{5.627215in}{1.334101in}}%
\pgfpathlineto{\pgfqpoint{5.628289in}{1.335678in}}%
\pgfpathlineto{\pgfqpoint{5.629362in}{1.336122in}}%
\pgfpathlineto{\pgfqpoint{5.630436in}{1.337747in}}%
\pgfpathlineto{\pgfqpoint{5.631510in}{1.335723in}}%
\pgfpathlineto{\pgfqpoint{5.636879in}{1.335154in}}%
\pgfpathlineto{\pgfqpoint{5.637953in}{1.333489in}}%
\pgfpathlineto{\pgfqpoint{5.639027in}{1.333703in}}%
\pgfpathlineto{\pgfqpoint{5.642248in}{1.333150in}}%
\pgfpathlineto{\pgfqpoint{5.643322in}{1.334309in}}%
\pgfpathlineto{\pgfqpoint{5.646544in}{1.333255in}}%
\pgfpathlineto{\pgfqpoint{5.649765in}{1.334019in}}%
\pgfpathlineto{\pgfqpoint{5.651913in}{1.330391in}}%
\pgfpathlineto{\pgfqpoint{5.654061in}{1.331183in}}%
\pgfpathlineto{\pgfqpoint{5.659430in}{1.329822in}}%
\pgfpathlineto{\pgfqpoint{5.660504in}{1.329881in}}%
\pgfpathlineto{\pgfqpoint{5.661578in}{1.329358in}}%
\pgfpathlineto{\pgfqpoint{5.665873in}{1.329012in}}%
\pgfpathlineto{\pgfqpoint{5.668021in}{1.326683in}}%
\pgfpathlineto{\pgfqpoint{5.669094in}{1.327595in}}%
\pgfpathlineto{\pgfqpoint{5.672316in}{1.327119in}}%
\pgfpathlineto{\pgfqpoint{5.674464in}{1.328405in}}%
\pgfpathlineto{\pgfqpoint{5.675537in}{1.326815in}}%
\pgfpathlineto{\pgfqpoint{5.679833in}{1.327203in}}%
\pgfpathlineto{\pgfqpoint{5.680907in}{1.325642in}}%
\pgfpathlineto{\pgfqpoint{5.684128in}{1.325751in}}%
\pgfpathlineto{\pgfqpoint{5.687350in}{1.324797in}}%
\pgfpathlineto{\pgfqpoint{5.688423in}{1.325548in}}%
\pgfpathlineto{\pgfqpoint{5.689497in}{1.325548in}}%
\pgfpathlineto{\pgfqpoint{5.690571in}{1.329106in}}%
\pgfpathlineto{\pgfqpoint{5.691645in}{1.328675in}}%
\pgfpathlineto{\pgfqpoint{5.694867in}{1.330074in}}%
\pgfpathlineto{\pgfqpoint{5.695940in}{1.329228in}}%
\pgfpathlineto{\pgfqpoint{5.697014in}{1.330869in}}%
\pgfpathlineto{\pgfqpoint{5.702383in}{1.329749in}}%
\pgfpathlineto{\pgfqpoint{5.703457in}{1.328878in}}%
\pgfpathlineto{\pgfqpoint{5.704531in}{1.329680in}}%
\pgfpathlineto{\pgfqpoint{5.705605in}{1.334002in}}%
\pgfpathlineto{\pgfqpoint{5.706679in}{1.332607in}}%
\pgfpathlineto{\pgfqpoint{5.709900in}{1.332060in}}%
\pgfpathlineto{\pgfqpoint{5.710974in}{1.332874in}}%
\pgfpathlineto{\pgfqpoint{5.712048in}{1.329580in}}%
\pgfpathlineto{\pgfqpoint{5.713122in}{1.331263in}}%
\pgfpathlineto{\pgfqpoint{5.714196in}{1.331472in}}%
\pgfpathlineto{\pgfqpoint{5.717417in}{1.330486in}}%
\pgfpathlineto{\pgfqpoint{5.718491in}{1.332134in}}%
\pgfpathlineto{\pgfqpoint{5.721712in}{1.330992in}}%
\pgfpathlineto{\pgfqpoint{5.724934in}{1.331081in}}%
\pgfpathlineto{\pgfqpoint{5.726008in}{1.329892in}}%
\pgfpathlineto{\pgfqpoint{5.727082in}{1.330884in}}%
\pgfpathlineto{\pgfqpoint{5.728155in}{1.330055in}}%
\pgfpathlineto{\pgfqpoint{5.729229in}{1.331400in}}%
\pgfpathlineto{\pgfqpoint{5.732451in}{1.330654in}}%
\pgfpathlineto{\pgfqpoint{5.734599in}{1.332826in}}%
\pgfpathlineto{\pgfqpoint{5.735672in}{1.334810in}}%
\pgfpathlineto{\pgfqpoint{5.736746in}{1.334747in}}%
\pgfpathlineto{\pgfqpoint{5.739968in}{1.336161in}}%
\pgfpathlineto{\pgfqpoint{5.742115in}{1.334185in}}%
\pgfpathlineto{\pgfqpoint{5.744263in}{1.332144in}}%
\pgfpathlineto{\pgfqpoint{5.748558in}{1.331418in}}%
\pgfpathlineto{\pgfqpoint{5.749632in}{1.332614in}}%
\pgfpathlineto{\pgfqpoint{5.751780in}{1.331341in}}%
\pgfpathlineto{\pgfqpoint{5.755001in}{1.331878in}}%
\pgfpathlineto{\pgfqpoint{5.756075in}{1.329169in}}%
\pgfpathlineto{\pgfqpoint{5.757149in}{1.329747in}}%
\pgfpathlineto{\pgfqpoint{5.759297in}{1.326890in}}%
\pgfpathlineto{\pgfqpoint{5.762518in}{1.327113in}}%
\pgfpathlineto{\pgfqpoint{5.763592in}{1.326105in}}%
\pgfpathlineto{\pgfqpoint{5.764666in}{1.326464in}}%
\pgfpathlineto{\pgfqpoint{5.766814in}{1.324237in}}%
\pgfpathlineto{\pgfqpoint{5.770035in}{1.324263in}}%
\pgfpathlineto{\pgfqpoint{5.771109in}{1.323487in}}%
\pgfpathlineto{\pgfqpoint{5.772183in}{1.323646in}}%
\pgfpathlineto{\pgfqpoint{5.773257in}{1.322189in}}%
\pgfpathlineto{\pgfqpoint{5.774331in}{1.322732in}}%
\pgfpathlineto{\pgfqpoint{5.779700in}{1.321306in}}%
\pgfpathlineto{\pgfqpoint{5.781847in}{1.318754in}}%
\pgfpathlineto{\pgfqpoint{5.787217in}{1.317490in}}%
\pgfpathlineto{\pgfqpoint{5.788290in}{1.319332in}}%
\pgfpathlineto{\pgfqpoint{5.789364in}{1.318221in}}%
\pgfpathlineto{\pgfqpoint{5.792586in}{1.318148in}}%
\pgfpathlineto{\pgfqpoint{5.793660in}{1.319117in}}%
\pgfpathlineto{\pgfqpoint{5.794733in}{1.317985in}}%
\pgfpathlineto{\pgfqpoint{5.796881in}{1.318323in}}%
\pgfpathlineto{\pgfqpoint{5.801177in}{1.318613in}}%
\pgfpathlineto{\pgfqpoint{5.803324in}{1.319322in}}%
\pgfpathlineto{\pgfqpoint{5.804398in}{1.317741in}}%
\pgfpathlineto{\pgfqpoint{5.807620in}{1.318222in}}%
\pgfpathlineto{\pgfqpoint{5.808693in}{1.320478in}}%
\pgfpathlineto{\pgfqpoint{5.809767in}{1.319344in}}%
\pgfpathlineto{\pgfqpoint{5.810841in}{1.320358in}}%
\pgfpathlineto{\pgfqpoint{5.811915in}{1.319150in}}%
\pgfpathlineto{\pgfqpoint{5.817284in}{1.322426in}}%
\pgfpathlineto{\pgfqpoint{5.818358in}{1.322374in}}%
\pgfpathlineto{\pgfqpoint{5.819432in}{1.323024in}}%
\pgfpathlineto{\pgfqpoint{5.823727in}{1.322971in}}%
\pgfpathlineto{\pgfqpoint{5.825875in}{1.322525in}}%
\pgfpathlineto{\pgfqpoint{5.826949in}{1.322995in}}%
\pgfpathlineto{\pgfqpoint{5.830170in}{1.323100in}}%
\pgfpathlineto{\pgfqpoint{5.831244in}{1.325047in}}%
\pgfpathlineto{\pgfqpoint{5.832318in}{1.324069in}}%
\pgfpathlineto{\pgfqpoint{5.834466in}{1.325766in}}%
\pgfpathlineto{\pgfqpoint{5.840909in}{1.325052in}}%
\pgfpathlineto{\pgfqpoint{5.841982in}{1.326655in}}%
\pgfpathlineto{\pgfqpoint{5.845204in}{1.325623in}}%
\pgfpathlineto{\pgfqpoint{5.846278in}{1.326172in}}%
\pgfpathlineto{\pgfqpoint{5.848425in}{1.325454in}}%
\pgfpathlineto{\pgfqpoint{5.849499in}{1.324688in}}%
\pgfpathlineto{\pgfqpoint{5.855942in}{1.323015in}}%
\pgfpathlineto{\pgfqpoint{5.857016in}{1.327093in}}%
\pgfpathlineto{\pgfqpoint{5.860238in}{1.325237in}}%
\pgfpathlineto{\pgfqpoint{5.861311in}{1.327913in}}%
\pgfpathlineto{\pgfqpoint{5.862385in}{1.327828in}}%
\pgfpathlineto{\pgfqpoint{5.863459in}{1.326604in}}%
\pgfpathlineto{\pgfqpoint{5.864533in}{1.326855in}}%
\pgfpathlineto{\pgfqpoint{5.867755in}{1.328509in}}%
\pgfpathlineto{\pgfqpoint{5.868828in}{1.328336in}}%
\pgfpathlineto{\pgfqpoint{5.870976in}{1.325667in}}%
\pgfpathlineto{\pgfqpoint{5.872050in}{1.325144in}}%
\pgfpathlineto{\pgfqpoint{5.875271in}{1.326126in}}%
\pgfpathlineto{\pgfqpoint{5.876345in}{1.325295in}}%
\pgfpathlineto{\pgfqpoint{5.877419in}{1.326334in}}%
\pgfpathlineto{\pgfqpoint{5.879567in}{1.326528in}}%
\pgfpathlineto{\pgfqpoint{5.882788in}{1.326779in}}%
\pgfpathlineto{\pgfqpoint{5.884936in}{1.329325in}}%
\pgfpathlineto{\pgfqpoint{5.886010in}{1.329354in}}%
\pgfpathlineto{\pgfqpoint{5.887084in}{1.330229in}}%
\pgfpathlineto{\pgfqpoint{5.890305in}{1.329519in}}%
\pgfpathlineto{\pgfqpoint{5.891379in}{1.330280in}}%
\pgfpathlineto{\pgfqpoint{5.892453in}{1.329215in}}%
\pgfpathlineto{\pgfqpoint{5.899970in}{1.329008in}}%
\pgfpathlineto{\pgfqpoint{5.902117in}{1.334730in}}%
\pgfpathlineto{\pgfqpoint{5.905339in}{1.334699in}}%
\pgfpathlineto{\pgfqpoint{5.907487in}{1.336322in}}%
\pgfpathlineto{\pgfqpoint{5.908560in}{1.333728in}}%
\pgfpathlineto{\pgfqpoint{5.909634in}{1.334633in}}%
\pgfpathlineto{\pgfqpoint{5.912856in}{1.334950in}}%
\pgfpathlineto{\pgfqpoint{5.913930in}{1.336158in}}%
\pgfpathlineto{\pgfqpoint{5.915003in}{1.338230in}}%
\pgfpathlineto{\pgfqpoint{5.916077in}{1.338430in}}%
\pgfpathlineto{\pgfqpoint{5.917151in}{1.337827in}}%
\pgfpathlineto{\pgfqpoint{5.922520in}{1.335562in}}%
\pgfpathlineto{\pgfqpoint{5.923594in}{1.338868in}}%
\pgfpathlineto{\pgfqpoint{5.924668in}{1.338868in}}%
\pgfpathlineto{\pgfqpoint{5.927889in}{1.340385in}}%
\pgfpathlineto{\pgfqpoint{5.928963in}{1.336768in}}%
\pgfpathlineto{\pgfqpoint{5.930037in}{1.335164in}}%
\pgfpathlineto{\pgfqpoint{5.931111in}{1.335420in}}%
\pgfpathlineto{\pgfqpoint{5.932185in}{1.334746in}}%
\pgfpathlineto{\pgfqpoint{5.935406in}{1.334079in}}%
\pgfpathlineto{\pgfqpoint{5.937554in}{1.328862in}}%
\pgfpathlineto{\pgfqpoint{5.942923in}{1.326585in}}%
\pgfpathlineto{\pgfqpoint{5.943997in}{1.326978in}}%
\pgfpathlineto{\pgfqpoint{5.945071in}{1.329632in}}%
\pgfpathlineto{\pgfqpoint{5.950440in}{1.330015in}}%
\pgfpathlineto{\pgfqpoint{5.951514in}{1.328416in}}%
\pgfpathlineto{\pgfqpoint{5.952588in}{1.325815in}}%
\pgfpathlineto{\pgfqpoint{5.953662in}{1.326426in}}%
\pgfpathlineto{\pgfqpoint{5.954735in}{1.325529in}}%
\pgfpathlineto{\pgfqpoint{5.957957in}{1.324894in}}%
\pgfpathlineto{\pgfqpoint{5.959031in}{1.323445in}}%
\pgfpathlineto{\pgfqpoint{5.960105in}{1.325155in}}%
\pgfpathlineto{\pgfqpoint{5.961178in}{1.324715in}}%
\pgfpathlineto{\pgfqpoint{5.962252in}{1.323624in}}%
\pgfpathlineto{\pgfqpoint{5.966548in}{1.321625in}}%
\pgfpathlineto{\pgfqpoint{5.967621in}{1.321963in}}%
\pgfpathlineto{\pgfqpoint{5.968695in}{1.320397in}}%
\pgfpathlineto{\pgfqpoint{5.974065in}{1.320422in}}%
\pgfpathlineto{\pgfqpoint{5.975138in}{1.320855in}}%
\pgfpathlineto{\pgfqpoint{5.976212in}{1.320189in}}%
\pgfpathlineto{\pgfqpoint{5.977286in}{1.321000in}}%
\pgfpathlineto{\pgfqpoint{5.981581in}{1.318277in}}%
\pgfpathlineto{\pgfqpoint{5.983729in}{1.318129in}}%
\pgfpathlineto{\pgfqpoint{5.984803in}{1.319893in}}%
\pgfpathlineto{\pgfqpoint{5.988024in}{1.321205in}}%
\pgfpathlineto{\pgfqpoint{5.989098in}{1.321024in}}%
\pgfpathlineto{\pgfqpoint{5.990172in}{1.321693in}}%
\pgfpathlineto{\pgfqpoint{5.991246in}{1.321199in}}%
\pgfpathlineto{\pgfqpoint{5.992320in}{1.321483in}}%
\pgfpathlineto{\pgfqpoint{5.996615in}{1.321042in}}%
\pgfpathlineto{\pgfqpoint{5.997689in}{1.322174in}}%
\pgfpathlineto{\pgfqpoint{5.998763in}{1.321912in}}%
\pgfpathlineto{\pgfqpoint{5.999837in}{1.321077in}}%
\pgfpathlineto{\pgfqpoint{6.003058in}{1.321798in}}%
\pgfpathlineto{\pgfqpoint{6.004132in}{1.327114in}}%
\pgfpathlineto{\pgfqpoint{6.005206in}{1.327995in}}%
\pgfpathlineto{\pgfqpoint{6.006280in}{1.329668in}}%
\pgfpathlineto{\pgfqpoint{6.007354in}{1.328425in}}%
\pgfpathlineto{\pgfqpoint{6.010575in}{1.329005in}}%
\pgfpathlineto{\pgfqpoint{6.013797in}{1.331883in}}%
\pgfpathlineto{\pgfqpoint{6.014870in}{1.331056in}}%
\pgfpathlineto{\pgfqpoint{6.019166in}{1.332537in}}%
\pgfpathlineto{\pgfqpoint{6.022387in}{1.329970in}}%
\pgfpathlineto{\pgfqpoint{6.027756in}{1.332378in}}%
\pgfpathlineto{\pgfqpoint{6.028830in}{1.331329in}}%
\pgfpathlineto{\pgfqpoint{6.029904in}{1.329354in}}%
\pgfpathlineto{\pgfqpoint{6.035273in}{1.328092in}}%
\pgfpathlineto{\pgfqpoint{6.037421in}{1.325711in}}%
\pgfpathlineto{\pgfqpoint{6.041716in}{1.328155in}}%
\pgfpathlineto{\pgfqpoint{6.044938in}{1.326975in}}%
\pgfpathlineto{\pgfqpoint{6.048159in}{1.327117in}}%
\pgfpathlineto{\pgfqpoint{6.050307in}{1.329357in}}%
\pgfpathlineto{\pgfqpoint{6.052455in}{1.328856in}}%
\pgfpathlineto{\pgfqpoint{6.056750in}{1.328825in}}%
\pgfpathlineto{\pgfqpoint{6.057824in}{1.326777in}}%
\pgfpathlineto{\pgfqpoint{6.058898in}{1.327032in}}%
\pgfpathlineto{\pgfqpoint{6.059972in}{1.326149in}}%
\pgfpathlineto{\pgfqpoint{6.064267in}{1.326740in}}%
\pgfpathlineto{\pgfqpoint{6.065341in}{1.327873in}}%
\pgfpathlineto{\pgfqpoint{6.067488in}{1.327931in}}%
\pgfpathlineto{\pgfqpoint{6.070710in}{1.329346in}}%
\pgfpathlineto{\pgfqpoint{6.071784in}{1.328903in}}%
\pgfpathlineto{\pgfqpoint{6.075005in}{1.330349in}}%
\pgfpathlineto{\pgfqpoint{6.079301in}{1.328856in}}%
\pgfpathlineto{\pgfqpoint{6.080375in}{1.329618in}}%
\pgfpathlineto{\pgfqpoint{6.081448in}{1.329647in}}%
\pgfpathlineto{\pgfqpoint{6.082522in}{1.329054in}}%
\pgfpathlineto{\pgfqpoint{6.086818in}{1.328935in}}%
\pgfpathlineto{\pgfqpoint{6.087891in}{1.328348in}}%
\pgfpathlineto{\pgfqpoint{6.088965in}{1.329133in}}%
\pgfpathlineto{\pgfqpoint{6.093261in}{1.328633in}}%
\pgfpathlineto{\pgfqpoint{6.094334in}{1.327552in}}%
\pgfpathlineto{\pgfqpoint{6.095408in}{1.328270in}}%
\pgfpathlineto{\pgfqpoint{6.096482in}{1.329664in}}%
\pgfpathlineto{\pgfqpoint{6.097556in}{1.332780in}}%
\pgfpathlineto{\pgfqpoint{6.100777in}{1.333341in}}%
\pgfpathlineto{\pgfqpoint{6.101851in}{1.334345in}}%
\pgfpathlineto{\pgfqpoint{6.102925in}{1.332433in}}%
\pgfpathlineto{\pgfqpoint{6.105073in}{1.336555in}}%
\pgfpathlineto{\pgfqpoint{6.109368in}{1.336554in}}%
\pgfpathlineto{\pgfqpoint{6.110442in}{1.335829in}}%
\pgfpathlineto{\pgfqpoint{6.111516in}{1.336645in}}%
\pgfpathlineto{\pgfqpoint{6.112590in}{1.334202in}}%
\pgfpathlineto{\pgfqpoint{6.116885in}{1.335000in}}%
\pgfpathlineto{\pgfqpoint{6.117959in}{1.335097in}}%
\pgfpathlineto{\pgfqpoint{6.120107in}{1.336686in}}%
\pgfpathlineto{\pgfqpoint{6.124402in}{1.338276in}}%
\pgfpathlineto{\pgfqpoint{6.125476in}{1.340812in}}%
\pgfpathlineto{\pgfqpoint{6.127623in}{1.337861in}}%
\pgfpathlineto{\pgfqpoint{6.131919in}{1.337693in}}%
\pgfpathlineto{\pgfqpoint{6.132993in}{1.339001in}}%
\pgfpathlineto{\pgfqpoint{6.134066in}{1.338180in}}%
\pgfpathlineto{\pgfqpoint{6.135140in}{1.338147in}}%
\pgfpathlineto{\pgfqpoint{6.141583in}{1.334561in}}%
\pgfpathlineto{\pgfqpoint{6.145879in}{1.334946in}}%
\pgfpathlineto{\pgfqpoint{6.148026in}{1.334592in}}%
\pgfpathlineto{\pgfqpoint{6.149100in}{1.335490in}}%
\pgfpathlineto{\pgfqpoint{6.150174in}{1.333930in}}%
\pgfpathlineto{\pgfqpoint{6.153396in}{1.332564in}}%
\pgfpathlineto{\pgfqpoint{6.154469in}{1.334618in}}%
\pgfpathlineto{\pgfqpoint{6.155543in}{1.333944in}}%
\pgfpathlineto{\pgfqpoint{6.160912in}{1.334294in}}%
\pgfpathlineto{\pgfqpoint{6.163060in}{1.337691in}}%
\pgfpathlineto{\pgfqpoint{6.165208in}{1.337758in}}%
\pgfpathlineto{\pgfqpoint{6.168429in}{1.336648in}}%
\pgfpathlineto{\pgfqpoint{6.169503in}{1.339362in}}%
\pgfpathlineto{\pgfqpoint{6.170577in}{1.339362in}}%
\pgfpathlineto{\pgfqpoint{6.171651in}{1.340705in}}%
\pgfpathlineto{\pgfqpoint{6.172725in}{1.339933in}}%
\pgfpathlineto{\pgfqpoint{6.175946in}{1.338856in}}%
\pgfpathlineto{\pgfqpoint{6.178094in}{1.339951in}}%
\pgfpathlineto{\pgfqpoint{6.179168in}{1.341687in}}%
\pgfpathlineto{\pgfqpoint{6.183463in}{1.342722in}}%
\pgfpathlineto{\pgfqpoint{6.184537in}{1.343769in}}%
\pgfpathlineto{\pgfqpoint{6.186685in}{1.341767in}}%
\pgfpathlineto{\pgfqpoint{6.190980in}{1.341197in}}%
\pgfpathlineto{\pgfqpoint{6.192054in}{1.342188in}}%
\pgfpathlineto{\pgfqpoint{6.193128in}{1.341901in}}%
\pgfpathlineto{\pgfqpoint{6.194201in}{1.339364in}}%
\pgfpathlineto{\pgfqpoint{6.195275in}{1.339330in}}%
\pgfpathlineto{\pgfqpoint{6.198497in}{1.340984in}}%
\pgfpathlineto{\pgfqpoint{6.200644in}{1.338778in}}%
\pgfpathlineto{\pgfqpoint{6.201718in}{1.328111in}}%
\pgfpathlineto{\pgfqpoint{6.206014in}{1.326535in}}%
\pgfpathlineto{\pgfqpoint{6.207087in}{1.325248in}}%
\pgfpathlineto{\pgfqpoint{6.208161in}{1.326985in}}%
\pgfpathlineto{\pgfqpoint{6.210309in}{1.325155in}}%
\pgfpathlineto{\pgfqpoint{6.213531in}{1.325295in}}%
\pgfpathlineto{\pgfqpoint{6.217826in}{1.327363in}}%
\pgfpathlineto{\pgfqpoint{6.221047in}{1.325450in}}%
\pgfpathlineto{\pgfqpoint{6.225343in}{1.328305in}}%
\pgfpathlineto{\pgfqpoint{6.230712in}{1.326721in}}%
\pgfpathlineto{\pgfqpoint{6.231786in}{1.326635in}}%
\pgfpathlineto{\pgfqpoint{6.232860in}{1.325660in}}%
\pgfpathlineto{\pgfqpoint{6.237155in}{1.326113in}}%
\pgfpathlineto{\pgfqpoint{6.239303in}{1.327512in}}%
\pgfpathlineto{\pgfqpoint{6.240376in}{1.327658in}}%
\pgfpathlineto{\pgfqpoint{6.244672in}{1.329144in}}%
\pgfpathlineto{\pgfqpoint{6.247893in}{1.332593in}}%
\pgfpathlineto{\pgfqpoint{6.251115in}{1.332058in}}%
\pgfpathlineto{\pgfqpoint{6.253263in}{1.329372in}}%
\pgfpathlineto{\pgfqpoint{6.254336in}{1.329551in}}%
\pgfpathlineto{\pgfqpoint{6.255410in}{1.327690in}}%
\pgfpathlineto{\pgfqpoint{6.258632in}{1.327077in}}%
\pgfpathlineto{\pgfqpoint{6.259706in}{1.323782in}}%
\pgfpathlineto{\pgfqpoint{6.260779in}{1.323425in}}%
\pgfpathlineto{\pgfqpoint{6.261853in}{1.324762in}}%
\pgfpathlineto{\pgfqpoint{6.262927in}{1.322337in}}%
\pgfpathlineto{\pgfqpoint{6.266149in}{1.322337in}}%
\pgfpathlineto{\pgfqpoint{6.267222in}{1.323275in}}%
\pgfpathlineto{\pgfqpoint{6.273665in}{1.323656in}}%
\pgfpathlineto{\pgfqpoint{6.274739in}{1.322423in}}%
\pgfpathlineto{\pgfqpoint{6.276887in}{1.322637in}}%
\pgfpathlineto{\pgfqpoint{6.277961in}{1.323554in}}%
\pgfpathlineto{\pgfqpoint{6.282256in}{1.322708in}}%
\pgfpathlineto{\pgfqpoint{6.284404in}{1.324693in}}%
\pgfpathlineto{\pgfqpoint{6.285478in}{1.325975in}}%
\pgfpathlineto{\pgfqpoint{6.288699in}{1.325349in}}%
\pgfpathlineto{\pgfqpoint{6.292995in}{1.321682in}}%
\pgfpathlineto{\pgfqpoint{6.298364in}{1.323906in}}%
\pgfpathlineto{\pgfqpoint{6.299438in}{1.323246in}}%
\pgfpathlineto{\pgfqpoint{6.300511in}{1.323300in}}%
\pgfpathlineto{\pgfqpoint{6.303733in}{1.325918in}}%
\pgfpathlineto{\pgfqpoint{6.305881in}{1.325917in}}%
\pgfpathlineto{\pgfqpoint{6.306954in}{1.326913in}}%
\pgfpathlineto{\pgfqpoint{6.308028in}{1.327000in}}%
\pgfpathlineto{\pgfqpoint{6.311250in}{1.332065in}}%
\pgfpathlineto{\pgfqpoint{6.313397in}{1.332284in}}%
\pgfpathlineto{\pgfqpoint{6.315545in}{1.333960in}}%
\pgfpathlineto{\pgfqpoint{6.318767in}{1.334379in}}%
\pgfpathlineto{\pgfqpoint{6.319841in}{1.335970in}}%
\pgfpathlineto{\pgfqpoint{6.320914in}{1.336302in}}%
\pgfpathlineto{\pgfqpoint{6.323062in}{1.332320in}}%
\pgfpathlineto{\pgfqpoint{6.326284in}{1.330021in}}%
\pgfpathlineto{\pgfqpoint{6.327357in}{1.330082in}}%
\pgfpathlineto{\pgfqpoint{6.328431in}{1.327494in}}%
\pgfpathlineto{\pgfqpoint{6.330579in}{1.327728in}}%
\pgfpathlineto{\pgfqpoint{6.333800in}{1.325967in}}%
\pgfpathlineto{\pgfqpoint{6.337022in}{1.317794in}}%
\pgfpathlineto{\pgfqpoint{6.338096in}{1.316969in}}%
\pgfpathlineto{\pgfqpoint{6.341317in}{1.315884in}}%
\pgfpathlineto{\pgfqpoint{6.342391in}{1.317700in}}%
\pgfpathlineto{\pgfqpoint{6.344539in}{1.318854in}}%
\pgfpathlineto{\pgfqpoint{6.345613in}{1.317275in}}%
\pgfpathlineto{\pgfqpoint{6.348834in}{1.315538in}}%
\pgfpathlineto{\pgfqpoint{6.349908in}{1.312527in}}%
\pgfpathlineto{\pgfqpoint{6.352056in}{1.314347in}}%
\pgfpathlineto{\pgfqpoint{6.353130in}{1.313639in}}%
\pgfpathlineto{\pgfqpoint{6.356351in}{1.312379in}}%
\pgfpathlineto{\pgfqpoint{6.357425in}{1.313269in}}%
\pgfpathlineto{\pgfqpoint{6.358499in}{1.313385in}}%
\pgfpathlineto{\pgfqpoint{6.360646in}{1.312483in}}%
\pgfpathlineto{\pgfqpoint{6.367089in}{1.311981in}}%
\pgfpathlineto{\pgfqpoint{6.368163in}{1.313023in}}%
\pgfpathlineto{\pgfqpoint{6.368163in}{1.313023in}}%
\pgfusepath{stroke}%
\end{pgfscope}%
\begin{pgfscope}%
\pgfsetrectcap%
\pgfsetmiterjoin%
\pgfsetlinewidth{0.803000pt}%
\definecolor{currentstroke}{rgb}{1.000000,1.000000,1.000000}%
\pgfsetstrokecolor{currentstroke}%
\pgfsetdash{}{0pt}%
\pgfpathmoveto{\pgfqpoint{3.902254in}{1.293759in}}%
\pgfpathlineto{\pgfqpoint{3.902254in}{1.694644in}}%
\pgfusepath{stroke}%
\end{pgfscope}%
\begin{pgfscope}%
\pgfsetrectcap%
\pgfsetmiterjoin%
\pgfsetlinewidth{0.803000pt}%
\definecolor{currentstroke}{rgb}{1.000000,1.000000,1.000000}%
\pgfsetstrokecolor{currentstroke}%
\pgfsetdash{}{0pt}%
\pgfpathmoveto{\pgfqpoint{6.485588in}{1.293759in}}%
\pgfpathlineto{\pgfqpoint{6.485588in}{1.694644in}}%
\pgfusepath{stroke}%
\end{pgfscope}%
\begin{pgfscope}%
\pgfsetrectcap%
\pgfsetmiterjoin%
\pgfsetlinewidth{0.803000pt}%
\definecolor{currentstroke}{rgb}{1.000000,1.000000,1.000000}%
\pgfsetstrokecolor{currentstroke}%
\pgfsetdash{}{0pt}%
\pgfpathmoveto{\pgfqpoint{3.902254in}{1.293759in}}%
\pgfpathlineto{\pgfqpoint{6.485588in}{1.293759in}}%
\pgfusepath{stroke}%
\end{pgfscope}%
\begin{pgfscope}%
\pgfsetrectcap%
\pgfsetmiterjoin%
\pgfsetlinewidth{0.803000pt}%
\definecolor{currentstroke}{rgb}{1.000000,1.000000,1.000000}%
\pgfsetstrokecolor{currentstroke}%
\pgfsetdash{}{0pt}%
\pgfpathmoveto{\pgfqpoint{3.902254in}{1.694644in}}%
\pgfpathlineto{\pgfqpoint{6.485588in}{1.694644in}}%
\pgfusepath{stroke}%
\end{pgfscope}%
\begin{pgfscope}%
\definecolor{textcolor}{rgb}{0.150000,0.150000,0.150000}%
\pgfsetstrokecolor{textcolor}%
\pgfsetfillcolor{textcolor}%
\pgftext[x=5.193921in,y=1.777977in,,base]{\color{textcolor}\rmfamily\fontsize{12.000000}{14.400000}\selectfont VZ}%
\end{pgfscope}%
\begin{pgfscope}%
\pgfsetbuttcap%
\pgfsetmiterjoin%
\definecolor{currentfill}{rgb}{0.917647,0.917647,0.949020}%
\pgfsetfillcolor{currentfill}%
\pgfsetlinewidth{0.000000pt}%
\definecolor{currentstroke}{rgb}{0.000000,0.000000,0.000000}%
\pgfsetstrokecolor{currentstroke}%
\pgfsetstrokeopacity{0.000000}%
\pgfsetdash{}{0pt}%
\pgfpathmoveto{\pgfqpoint{0.285588in}{0.331635in}}%
\pgfpathlineto{\pgfqpoint{2.868921in}{0.331635in}}%
\pgfpathlineto{\pgfqpoint{2.868921in}{0.732520in}}%
\pgfpathlineto{\pgfqpoint{0.285588in}{0.732520in}}%
\pgfpathclose%
\pgfusepath{fill}%
\end{pgfscope}%
\begin{pgfscope}%
\pgfpathrectangle{\pgfqpoint{0.285588in}{0.331635in}}{\pgfqpoint{2.583333in}{0.400885in}}%
\pgfusepath{clip}%
\pgfsetroundcap%
\pgfsetroundjoin%
\pgfsetlinewidth{0.803000pt}%
\definecolor{currentstroke}{rgb}{1.000000,1.000000,1.000000}%
\pgfsetstrokecolor{currentstroke}%
\pgfsetdash{}{0pt}%
\pgfpathmoveto{\pgfqpoint{0.400864in}{0.331635in}}%
\pgfpathlineto{\pgfqpoint{0.400864in}{0.732520in}}%
\pgfusepath{stroke}%
\end{pgfscope}%
\begin{pgfscope}%
\definecolor{textcolor}{rgb}{0.150000,0.150000,0.150000}%
\pgfsetstrokecolor{textcolor}%
\pgfsetfillcolor{textcolor}%
\pgftext[x=0.400864in,y=0.234413in,,top]{\color{textcolor}\rmfamily\fontsize{10.000000}{12.000000}\selectfont 2012}%
\end{pgfscope}%
\begin{pgfscope}%
\pgfpathrectangle{\pgfqpoint{0.285588in}{0.331635in}}{\pgfqpoint{2.583333in}{0.400885in}}%
\pgfusepath{clip}%
\pgfsetroundcap%
\pgfsetroundjoin%
\pgfsetlinewidth{0.803000pt}%
\definecolor{currentstroke}{rgb}{1.000000,1.000000,1.000000}%
\pgfsetstrokecolor{currentstroke}%
\pgfsetdash{}{0pt}%
\pgfpathmoveto{\pgfqpoint{0.793889in}{0.331635in}}%
\pgfpathlineto{\pgfqpoint{0.793889in}{0.732520in}}%
\pgfusepath{stroke}%
\end{pgfscope}%
\begin{pgfscope}%
\definecolor{textcolor}{rgb}{0.150000,0.150000,0.150000}%
\pgfsetstrokecolor{textcolor}%
\pgfsetfillcolor{textcolor}%
\pgftext[x=0.793889in,y=0.234413in,,top]{\color{textcolor}\rmfamily\fontsize{10.000000}{12.000000}\selectfont 2013}%
\end{pgfscope}%
\begin{pgfscope}%
\pgfpathrectangle{\pgfqpoint{0.285588in}{0.331635in}}{\pgfqpoint{2.583333in}{0.400885in}}%
\pgfusepath{clip}%
\pgfsetroundcap%
\pgfsetroundjoin%
\pgfsetlinewidth{0.803000pt}%
\definecolor{currentstroke}{rgb}{1.000000,1.000000,1.000000}%
\pgfsetstrokecolor{currentstroke}%
\pgfsetdash{}{0pt}%
\pgfpathmoveto{\pgfqpoint{1.185840in}{0.331635in}}%
\pgfpathlineto{\pgfqpoint{1.185840in}{0.732520in}}%
\pgfusepath{stroke}%
\end{pgfscope}%
\begin{pgfscope}%
\definecolor{textcolor}{rgb}{0.150000,0.150000,0.150000}%
\pgfsetstrokecolor{textcolor}%
\pgfsetfillcolor{textcolor}%
\pgftext[x=1.185840in,y=0.234413in,,top]{\color{textcolor}\rmfamily\fontsize{10.000000}{12.000000}\selectfont 2014}%
\end{pgfscope}%
\begin{pgfscope}%
\pgfpathrectangle{\pgfqpoint{0.285588in}{0.331635in}}{\pgfqpoint{2.583333in}{0.400885in}}%
\pgfusepath{clip}%
\pgfsetroundcap%
\pgfsetroundjoin%
\pgfsetlinewidth{0.803000pt}%
\definecolor{currentstroke}{rgb}{1.000000,1.000000,1.000000}%
\pgfsetstrokecolor{currentstroke}%
\pgfsetdash{}{0pt}%
\pgfpathmoveto{\pgfqpoint{1.577791in}{0.331635in}}%
\pgfpathlineto{\pgfqpoint{1.577791in}{0.732520in}}%
\pgfusepath{stroke}%
\end{pgfscope}%
\begin{pgfscope}%
\definecolor{textcolor}{rgb}{0.150000,0.150000,0.150000}%
\pgfsetstrokecolor{textcolor}%
\pgfsetfillcolor{textcolor}%
\pgftext[x=1.577791in,y=0.234413in,,top]{\color{textcolor}\rmfamily\fontsize{10.000000}{12.000000}\selectfont 2015}%
\end{pgfscope}%
\begin{pgfscope}%
\pgfpathrectangle{\pgfqpoint{0.285588in}{0.331635in}}{\pgfqpoint{2.583333in}{0.400885in}}%
\pgfusepath{clip}%
\pgfsetroundcap%
\pgfsetroundjoin%
\pgfsetlinewidth{0.803000pt}%
\definecolor{currentstroke}{rgb}{1.000000,1.000000,1.000000}%
\pgfsetstrokecolor{currentstroke}%
\pgfsetdash{}{0pt}%
\pgfpathmoveto{\pgfqpoint{1.969742in}{0.331635in}}%
\pgfpathlineto{\pgfqpoint{1.969742in}{0.732520in}}%
\pgfusepath{stroke}%
\end{pgfscope}%
\begin{pgfscope}%
\definecolor{textcolor}{rgb}{0.150000,0.150000,0.150000}%
\pgfsetstrokecolor{textcolor}%
\pgfsetfillcolor{textcolor}%
\pgftext[x=1.969742in,y=0.234413in,,top]{\color{textcolor}\rmfamily\fontsize{10.000000}{12.000000}\selectfont 2016}%
\end{pgfscope}%
\begin{pgfscope}%
\pgfpathrectangle{\pgfqpoint{0.285588in}{0.331635in}}{\pgfqpoint{2.583333in}{0.400885in}}%
\pgfusepath{clip}%
\pgfsetroundcap%
\pgfsetroundjoin%
\pgfsetlinewidth{0.803000pt}%
\definecolor{currentstroke}{rgb}{1.000000,1.000000,1.000000}%
\pgfsetstrokecolor{currentstroke}%
\pgfsetdash{}{0pt}%
\pgfpathmoveto{\pgfqpoint{2.362767in}{0.331635in}}%
\pgfpathlineto{\pgfqpoint{2.362767in}{0.732520in}}%
\pgfusepath{stroke}%
\end{pgfscope}%
\begin{pgfscope}%
\definecolor{textcolor}{rgb}{0.150000,0.150000,0.150000}%
\pgfsetstrokecolor{textcolor}%
\pgfsetfillcolor{textcolor}%
\pgftext[x=2.362767in,y=0.234413in,,top]{\color{textcolor}\rmfamily\fontsize{10.000000}{12.000000}\selectfont 2017}%
\end{pgfscope}%
\begin{pgfscope}%
\pgfpathrectangle{\pgfqpoint{0.285588in}{0.331635in}}{\pgfqpoint{2.583333in}{0.400885in}}%
\pgfusepath{clip}%
\pgfsetroundcap%
\pgfsetroundjoin%
\pgfsetlinewidth{0.803000pt}%
\definecolor{currentstroke}{rgb}{1.000000,1.000000,1.000000}%
\pgfsetstrokecolor{currentstroke}%
\pgfsetdash{}{0pt}%
\pgfpathmoveto{\pgfqpoint{2.754718in}{0.331635in}}%
\pgfpathlineto{\pgfqpoint{2.754718in}{0.732520in}}%
\pgfusepath{stroke}%
\end{pgfscope}%
\begin{pgfscope}%
\definecolor{textcolor}{rgb}{0.150000,0.150000,0.150000}%
\pgfsetstrokecolor{textcolor}%
\pgfsetfillcolor{textcolor}%
\pgftext[x=2.754718in,y=0.234413in,,top]{\color{textcolor}\rmfamily\fontsize{10.000000}{12.000000}\selectfont 2018}%
\end{pgfscope}%
\begin{pgfscope}%
\pgfpathrectangle{\pgfqpoint{0.285588in}{0.331635in}}{\pgfqpoint{2.583333in}{0.400885in}}%
\pgfusepath{clip}%
\pgfsetroundcap%
\pgfsetroundjoin%
\pgfsetlinewidth{0.803000pt}%
\definecolor{currentstroke}{rgb}{1.000000,1.000000,1.000000}%
\pgfsetstrokecolor{currentstroke}%
\pgfsetdash{}{0pt}%
\pgfpathmoveto{\pgfqpoint{0.285588in}{0.340496in}}%
\pgfpathlineto{\pgfqpoint{2.868921in}{0.340496in}}%
\pgfusepath{stroke}%
\end{pgfscope}%
\begin{pgfscope}%
\definecolor{textcolor}{rgb}{0.150000,0.150000,0.150000}%
\pgfsetstrokecolor{textcolor}%
\pgfsetfillcolor{textcolor}%
\pgftext[x=0.100000in,y=0.287734in,left,base]{\color{textcolor}\rmfamily\fontsize{10.000000}{12.000000}\selectfont 0}%
\end{pgfscope}%
\begin{pgfscope}%
\pgfpathrectangle{\pgfqpoint{0.285588in}{0.331635in}}{\pgfqpoint{2.583333in}{0.400885in}}%
\pgfusepath{clip}%
\pgfsetroundcap%
\pgfsetroundjoin%
\pgfsetlinewidth{0.803000pt}%
\definecolor{currentstroke}{rgb}{1.000000,1.000000,1.000000}%
\pgfsetstrokecolor{currentstroke}%
\pgfsetdash{}{0pt}%
\pgfpathmoveto{\pgfqpoint{0.285588in}{0.712942in}}%
\pgfpathlineto{\pgfqpoint{2.868921in}{0.712942in}}%
\pgfusepath{stroke}%
\end{pgfscope}%
\begin{pgfscope}%
\definecolor{textcolor}{rgb}{0.150000,0.150000,0.150000}%
\pgfsetstrokecolor{textcolor}%
\pgfsetfillcolor{textcolor}%
\pgftext[x=0.100000in,y=0.660181in,left,base]{\color{textcolor}\rmfamily\fontsize{10.000000}{12.000000}\selectfont 5}%
\end{pgfscope}%
\begin{pgfscope}%
\pgfpathrectangle{\pgfqpoint{0.285588in}{0.331635in}}{\pgfqpoint{2.583333in}{0.400885in}}%
\pgfusepath{clip}%
\pgfsetroundcap%
\pgfsetroundjoin%
\pgfsetlinewidth{1.505625pt}%
\definecolor{currentstroke}{rgb}{0.737255,0.741176,0.133333}%
\pgfsetstrokecolor{currentstroke}%
\pgfsetstrokeopacity{0.300000}%
\pgfsetdash{}{0pt}%
\pgfpathmoveto{\pgfqpoint{0.403012in}{0.414985in}}%
\pgfpathlineto{\pgfqpoint{0.404086in}{0.413663in}}%
\pgfpathlineto{\pgfqpoint{0.405159in}{0.414225in}}%
\pgfpathlineto{\pgfqpoint{0.406233in}{0.413365in}}%
\pgfpathlineto{\pgfqpoint{0.411602in}{0.412208in}}%
\pgfpathlineto{\pgfqpoint{0.412676in}{0.413795in}}%
\pgfpathlineto{\pgfqpoint{0.413750in}{0.413365in}}%
\pgfpathlineto{\pgfqpoint{0.418045in}{0.414655in}}%
\pgfpathlineto{\pgfqpoint{0.419119in}{0.415580in}}%
\pgfpathlineto{\pgfqpoint{0.421267in}{0.413266in}}%
\pgfpathlineto{\pgfqpoint{0.424489in}{0.412539in}}%
\pgfpathlineto{\pgfqpoint{0.425562in}{0.413564in}}%
\pgfpathlineto{\pgfqpoint{0.427710in}{0.413464in}}%
\pgfpathlineto{\pgfqpoint{0.428784in}{0.413597in}}%
\pgfpathlineto{\pgfqpoint{0.432005in}{0.412737in}}%
\pgfpathlineto{\pgfqpoint{0.433079in}{0.413299in}}%
\pgfpathlineto{\pgfqpoint{0.434153in}{0.414655in}}%
\pgfpathlineto{\pgfqpoint{0.435227in}{0.417233in}}%
\pgfpathlineto{\pgfqpoint{0.436301in}{0.417928in}}%
\pgfpathlineto{\pgfqpoint{0.440596in}{0.417895in}}%
\pgfpathlineto{\pgfqpoint{0.441670in}{0.418887in}}%
\pgfpathlineto{\pgfqpoint{0.442744in}{0.421829in}}%
\pgfpathlineto{\pgfqpoint{0.443818in}{0.422887in}}%
\pgfpathlineto{\pgfqpoint{0.447039in}{0.422027in}}%
\pgfpathlineto{\pgfqpoint{0.448113in}{0.423846in}}%
\pgfpathlineto{\pgfqpoint{0.449187in}{0.424408in}}%
\pgfpathlineto{\pgfqpoint{0.450261in}{0.423449in}}%
\pgfpathlineto{\pgfqpoint{0.451334in}{0.424342in}}%
\pgfpathlineto{\pgfqpoint{0.455630in}{0.423747in}}%
\pgfpathlineto{\pgfqpoint{0.456704in}{0.425334in}}%
\pgfpathlineto{\pgfqpoint{0.457778in}{0.425400in}}%
\pgfpathlineto{\pgfqpoint{0.458851in}{0.426193in}}%
\pgfpathlineto{\pgfqpoint{0.462073in}{0.425697in}}%
\pgfpathlineto{\pgfqpoint{0.463147in}{0.427218in}}%
\pgfpathlineto{\pgfqpoint{0.464221in}{0.425334in}}%
\pgfpathlineto{\pgfqpoint{0.465294in}{0.425929in}}%
\pgfpathlineto{\pgfqpoint{0.466368in}{0.425168in}}%
\pgfpathlineto{\pgfqpoint{0.469590in}{0.425234in}}%
\pgfpathlineto{\pgfqpoint{0.470664in}{0.424210in}}%
\pgfpathlineto{\pgfqpoint{0.471737in}{0.424805in}}%
\pgfpathlineto{\pgfqpoint{0.472811in}{0.426491in}}%
\pgfpathlineto{\pgfqpoint{0.473885in}{0.425896in}}%
\pgfpathlineto{\pgfqpoint{0.477107in}{0.425433in}}%
\pgfpathlineto{\pgfqpoint{0.478180in}{0.425995in}}%
\pgfpathlineto{\pgfqpoint{0.479254in}{0.425598in}}%
\pgfpathlineto{\pgfqpoint{0.484623in}{0.427119in}}%
\pgfpathlineto{\pgfqpoint{0.485697in}{0.425433in}}%
\pgfpathlineto{\pgfqpoint{0.487845in}{0.425995in}}%
\pgfpathlineto{\pgfqpoint{0.488919in}{0.427086in}}%
\pgfpathlineto{\pgfqpoint{0.493214in}{0.427846in}}%
\pgfpathlineto{\pgfqpoint{0.496436in}{0.426524in}}%
\pgfpathlineto{\pgfqpoint{0.499657in}{0.427218in}}%
\pgfpathlineto{\pgfqpoint{0.500731in}{0.428243in}}%
\pgfpathlineto{\pgfqpoint{0.501805in}{0.427218in}}%
\pgfpathlineto{\pgfqpoint{0.502879in}{0.428706in}}%
\pgfpathlineto{\pgfqpoint{0.507174in}{0.427615in}}%
\pgfpathlineto{\pgfqpoint{0.508248in}{0.425598in}}%
\pgfpathlineto{\pgfqpoint{0.509322in}{0.426061in}}%
\pgfpathlineto{\pgfqpoint{0.511469in}{0.430260in}}%
\pgfpathlineto{\pgfqpoint{0.514691in}{0.428574in}}%
\pgfpathlineto{\pgfqpoint{0.515765in}{0.429500in}}%
\pgfpathlineto{\pgfqpoint{0.518986in}{0.428706in}}%
\pgfpathlineto{\pgfqpoint{0.522208in}{0.426590in}}%
\pgfpathlineto{\pgfqpoint{0.523282in}{0.427185in}}%
\pgfpathlineto{\pgfqpoint{0.525429in}{0.430293in}}%
\pgfpathlineto{\pgfqpoint{0.526503in}{0.430524in}}%
\pgfpathlineto{\pgfqpoint{0.531872in}{0.429566in}}%
\pgfpathlineto{\pgfqpoint{0.532946in}{0.425367in}}%
\pgfpathlineto{\pgfqpoint{0.534020in}{0.426359in}}%
\pgfpathlineto{\pgfqpoint{0.538315in}{0.426623in}}%
\pgfpathlineto{\pgfqpoint{0.539389in}{0.426094in}}%
\pgfpathlineto{\pgfqpoint{0.540463in}{0.426623in}}%
\pgfpathlineto{\pgfqpoint{0.545832in}{0.425499in}}%
\pgfpathlineto{\pgfqpoint{0.546906in}{0.426821in}}%
\pgfpathlineto{\pgfqpoint{0.549054in}{0.423218in}}%
\pgfpathlineto{\pgfqpoint{0.555497in}{0.428475in}}%
\pgfpathlineto{\pgfqpoint{0.556571in}{0.428177in}}%
\pgfpathlineto{\pgfqpoint{0.560866in}{0.428838in}}%
\pgfpathlineto{\pgfqpoint{0.564088in}{0.422953in}}%
\pgfpathlineto{\pgfqpoint{0.567309in}{0.424342in}}%
\pgfpathlineto{\pgfqpoint{0.568383in}{0.424077in}}%
\pgfpathlineto{\pgfqpoint{0.569457in}{0.425962in}}%
\pgfpathlineto{\pgfqpoint{0.571604in}{0.426127in}}%
\pgfpathlineto{\pgfqpoint{0.574826in}{0.425962in}}%
\pgfpathlineto{\pgfqpoint{0.575900in}{0.426557in}}%
\pgfpathlineto{\pgfqpoint{0.576974in}{0.424904in}}%
\pgfpathlineto{\pgfqpoint{0.579121in}{0.427450in}}%
\pgfpathlineto{\pgfqpoint{0.583417in}{0.429632in}}%
\pgfpathlineto{\pgfqpoint{0.584490in}{0.430524in}}%
\pgfpathlineto{\pgfqpoint{0.585564in}{0.428243in}}%
\pgfpathlineto{\pgfqpoint{0.586638in}{0.432277in}}%
\pgfpathlineto{\pgfqpoint{0.589860in}{0.429566in}}%
\pgfpathlineto{\pgfqpoint{0.590933in}{0.430954in}}%
\pgfpathlineto{\pgfqpoint{0.592007in}{0.431153in}}%
\pgfpathlineto{\pgfqpoint{0.593081in}{0.429764in}}%
\pgfpathlineto{\pgfqpoint{0.594155in}{0.431318in}}%
\pgfpathlineto{\pgfqpoint{0.597377in}{0.433500in}}%
\pgfpathlineto{\pgfqpoint{0.598450in}{0.433302in}}%
\pgfpathlineto{\pgfqpoint{0.600598in}{0.433798in}}%
\pgfpathlineto{\pgfqpoint{0.601672in}{0.432508in}}%
\pgfpathlineto{\pgfqpoint{0.604893in}{0.431318in}}%
\pgfpathlineto{\pgfqpoint{0.607041in}{0.429169in}}%
\pgfpathlineto{\pgfqpoint{0.609189in}{0.431649in}}%
\pgfpathlineto{\pgfqpoint{0.613484in}{0.434690in}}%
\pgfpathlineto{\pgfqpoint{0.614558in}{0.434393in}}%
\pgfpathlineto{\pgfqpoint{0.615632in}{0.432607in}}%
\pgfpathlineto{\pgfqpoint{0.616706in}{0.432872in}}%
\pgfpathlineto{\pgfqpoint{0.619927in}{0.431913in}}%
\pgfpathlineto{\pgfqpoint{0.621001in}{0.430591in}}%
\pgfpathlineto{\pgfqpoint{0.622075in}{0.430260in}}%
\pgfpathlineto{\pgfqpoint{0.624222in}{0.435352in}}%
\pgfpathlineto{\pgfqpoint{0.627444in}{0.436608in}}%
\pgfpathlineto{\pgfqpoint{0.629592in}{0.434128in}}%
\pgfpathlineto{\pgfqpoint{0.631739in}{0.436707in}}%
\pgfpathlineto{\pgfqpoint{0.637109in}{0.436839in}}%
\pgfpathlineto{\pgfqpoint{0.638182in}{0.434889in}}%
\pgfpathlineto{\pgfqpoint{0.639256in}{0.435318in}}%
\pgfpathlineto{\pgfqpoint{0.642478in}{0.434756in}}%
\pgfpathlineto{\pgfqpoint{0.643552in}{0.435914in}}%
\pgfpathlineto{\pgfqpoint{0.646773in}{0.435649in}}%
\pgfpathlineto{\pgfqpoint{0.653216in}{0.434525in}}%
\pgfpathlineto{\pgfqpoint{0.654290in}{0.434161in}}%
\pgfpathlineto{\pgfqpoint{0.659659in}{0.435153in}}%
\pgfpathlineto{\pgfqpoint{0.660733in}{0.434194in}}%
\pgfpathlineto{\pgfqpoint{0.661807in}{0.435352in}}%
\pgfpathlineto{\pgfqpoint{0.666102in}{0.435550in}}%
\pgfpathlineto{\pgfqpoint{0.667176in}{0.434856in}}%
\pgfpathlineto{\pgfqpoint{0.668250in}{0.436277in}}%
\pgfpathlineto{\pgfqpoint{0.669324in}{0.436410in}}%
\pgfpathlineto{\pgfqpoint{0.672545in}{0.435649in}}%
\pgfpathlineto{\pgfqpoint{0.674693in}{0.439187in}}%
\pgfpathlineto{\pgfqpoint{0.675767in}{0.440311in}}%
\pgfpathlineto{\pgfqpoint{0.676841in}{0.439782in}}%
\pgfpathlineto{\pgfqpoint{0.681136in}{0.439352in}}%
\pgfpathlineto{\pgfqpoint{0.682210in}{0.440179in}}%
\pgfpathlineto{\pgfqpoint{0.687579in}{0.439385in}}%
\pgfpathlineto{\pgfqpoint{0.688653in}{0.440013in}}%
\pgfpathlineto{\pgfqpoint{0.689727in}{0.438492in}}%
\pgfpathlineto{\pgfqpoint{0.691874in}{0.439815in}}%
\pgfpathlineto{\pgfqpoint{0.695096in}{0.441600in}}%
\pgfpathlineto{\pgfqpoint{0.696170in}{0.441071in}}%
\pgfpathlineto{\pgfqpoint{0.698317in}{0.443782in}}%
\pgfpathlineto{\pgfqpoint{0.699391in}{0.444212in}}%
\pgfpathlineto{\pgfqpoint{0.702613in}{0.443088in}}%
\pgfpathlineto{\pgfqpoint{0.703687in}{0.441633in}}%
\pgfpathlineto{\pgfqpoint{0.704760in}{0.442129in}}%
\pgfpathlineto{\pgfqpoint{0.705834in}{0.443320in}}%
\pgfpathlineto{\pgfqpoint{0.710130in}{0.443948in}}%
\pgfpathlineto{\pgfqpoint{0.712277in}{0.445931in}}%
\pgfpathlineto{\pgfqpoint{0.713351in}{0.445436in}}%
\pgfpathlineto{\pgfqpoint{0.714425in}{0.444014in}}%
\pgfpathlineto{\pgfqpoint{0.717646in}{0.443286in}}%
\pgfpathlineto{\pgfqpoint{0.718720in}{0.441534in}}%
\pgfpathlineto{\pgfqpoint{0.719794in}{0.441435in}}%
\pgfpathlineto{\pgfqpoint{0.721942in}{0.442791in}}%
\pgfpathlineto{\pgfqpoint{0.727311in}{0.443121in}}%
\pgfpathlineto{\pgfqpoint{0.728385in}{0.446890in}}%
\pgfpathlineto{\pgfqpoint{0.732680in}{0.445138in}}%
\pgfpathlineto{\pgfqpoint{0.733754in}{0.446890in}}%
\pgfpathlineto{\pgfqpoint{0.735902in}{0.445568in}}%
\pgfpathlineto{\pgfqpoint{0.736976in}{0.446196in}}%
\pgfpathlineto{\pgfqpoint{0.740197in}{0.446427in}}%
\pgfpathlineto{\pgfqpoint{0.743419in}{0.445006in}}%
\pgfpathlineto{\pgfqpoint{0.744492in}{0.447023in}}%
\pgfpathlineto{\pgfqpoint{0.748788in}{0.449734in}}%
\pgfpathlineto{\pgfqpoint{0.749862in}{0.449965in}}%
\pgfpathlineto{\pgfqpoint{0.752009in}{0.451056in}}%
\pgfpathlineto{\pgfqpoint{0.757378in}{0.450428in}}%
\pgfpathlineto{\pgfqpoint{0.759526in}{0.452246in}}%
\pgfpathlineto{\pgfqpoint{0.764895in}{0.451023in}}%
\pgfpathlineto{\pgfqpoint{0.767043in}{0.451387in}}%
\pgfpathlineto{\pgfqpoint{0.771338in}{0.451651in}}%
\pgfpathlineto{\pgfqpoint{0.773486in}{0.450362in}}%
\pgfpathlineto{\pgfqpoint{0.774560in}{0.450097in}}%
\pgfpathlineto{\pgfqpoint{0.778855in}{0.453007in}}%
\pgfpathlineto{\pgfqpoint{0.779929in}{0.451850in}}%
\pgfpathlineto{\pgfqpoint{0.781003in}{0.454296in}}%
\pgfpathlineto{\pgfqpoint{0.782077in}{0.453040in}}%
\pgfpathlineto{\pgfqpoint{0.785298in}{0.453337in}}%
\pgfpathlineto{\pgfqpoint{0.789594in}{0.451453in}}%
\pgfpathlineto{\pgfqpoint{0.792815in}{0.453635in}}%
\pgfpathlineto{\pgfqpoint{0.794963in}{0.456478in}}%
\pgfpathlineto{\pgfqpoint{0.796037in}{0.456578in}}%
\pgfpathlineto{\pgfqpoint{0.797110in}{0.457503in}}%
\pgfpathlineto{\pgfqpoint{0.800332in}{0.458363in}}%
\pgfpathlineto{\pgfqpoint{0.802480in}{0.461272in}}%
\pgfpathlineto{\pgfqpoint{0.803554in}{0.460314in}}%
\pgfpathlineto{\pgfqpoint{0.804627in}{0.460776in}}%
\pgfpathlineto{\pgfqpoint{0.811070in}{0.459851in}}%
\pgfpathlineto{\pgfqpoint{0.812144in}{0.458627in}}%
\pgfpathlineto{\pgfqpoint{0.819661in}{0.459818in}}%
\pgfpathlineto{\pgfqpoint{0.822883in}{0.457206in}}%
\pgfpathlineto{\pgfqpoint{0.823956in}{0.457437in}}%
\pgfpathlineto{\pgfqpoint{0.825030in}{0.456214in}}%
\pgfpathlineto{\pgfqpoint{0.826104in}{0.458363in}}%
\pgfpathlineto{\pgfqpoint{0.827178in}{0.458859in}}%
\pgfpathlineto{\pgfqpoint{0.830399in}{0.457437in}}%
\pgfpathlineto{\pgfqpoint{0.832547in}{0.460545in}}%
\pgfpathlineto{\pgfqpoint{0.833621in}{0.457735in}}%
\pgfpathlineto{\pgfqpoint{0.834695in}{0.458297in}}%
\pgfpathlineto{\pgfqpoint{0.837916in}{0.457173in}}%
\pgfpathlineto{\pgfqpoint{0.838990in}{0.457536in}}%
\pgfpathlineto{\pgfqpoint{0.840064in}{0.457007in}}%
\pgfpathlineto{\pgfqpoint{0.842212in}{0.459421in}}%
\pgfpathlineto{\pgfqpoint{0.846507in}{0.459156in}}%
\pgfpathlineto{\pgfqpoint{0.847581in}{0.457503in}}%
\pgfpathlineto{\pgfqpoint{0.849729in}{0.460512in}}%
\pgfpathlineto{\pgfqpoint{0.852950in}{0.457834in}}%
\pgfpathlineto{\pgfqpoint{0.855098in}{0.460843in}}%
\pgfpathlineto{\pgfqpoint{0.857245in}{0.459487in}}%
\pgfpathlineto{\pgfqpoint{0.863688in}{0.461338in}}%
\pgfpathlineto{\pgfqpoint{0.867984in}{0.461834in}}%
\pgfpathlineto{\pgfqpoint{0.870131in}{0.460479in}}%
\pgfpathlineto{\pgfqpoint{0.871205in}{0.461438in}}%
\pgfpathlineto{\pgfqpoint{0.872279in}{0.459851in}}%
\pgfpathlineto{\pgfqpoint{0.875501in}{0.459487in}}%
\pgfpathlineto{\pgfqpoint{0.876575in}{0.457933in}}%
\pgfpathlineto{\pgfqpoint{0.877648in}{0.460446in}}%
\pgfpathlineto{\pgfqpoint{0.878722in}{0.459289in}}%
\pgfpathlineto{\pgfqpoint{0.879796in}{0.461008in}}%
\pgfpathlineto{\pgfqpoint{0.883018in}{0.463950in}}%
\pgfpathlineto{\pgfqpoint{0.884091in}{0.466562in}}%
\pgfpathlineto{\pgfqpoint{0.886239in}{0.468348in}}%
\pgfpathlineto{\pgfqpoint{0.890534in}{0.466463in}}%
\pgfpathlineto{\pgfqpoint{0.891608in}{0.466926in}}%
\pgfpathlineto{\pgfqpoint{0.892682in}{0.464512in}}%
\pgfpathlineto{\pgfqpoint{0.893756in}{0.465802in}}%
\pgfpathlineto{\pgfqpoint{0.894830in}{0.464810in}}%
\pgfpathlineto{\pgfqpoint{0.898051in}{0.465670in}}%
\pgfpathlineto{\pgfqpoint{0.899125in}{0.464512in}}%
\pgfpathlineto{\pgfqpoint{0.900199in}{0.466199in}}%
\pgfpathlineto{\pgfqpoint{0.901273in}{0.466662in}}%
\pgfpathlineto{\pgfqpoint{0.905568in}{0.461901in}}%
\pgfpathlineto{\pgfqpoint{0.906642in}{0.464479in}}%
\pgfpathlineto{\pgfqpoint{0.907716in}{0.462628in}}%
\pgfpathlineto{\pgfqpoint{0.908790in}{0.462000in}}%
\pgfpathlineto{\pgfqpoint{0.909864in}{0.463917in}}%
\pgfpathlineto{\pgfqpoint{0.913085in}{0.463587in}}%
\pgfpathlineto{\pgfqpoint{0.916307in}{0.467521in}}%
\pgfpathlineto{\pgfqpoint{0.917380in}{0.466397in}}%
\pgfpathlineto{\pgfqpoint{0.921676in}{0.467323in}}%
\pgfpathlineto{\pgfqpoint{0.922750in}{0.465471in}}%
\pgfpathlineto{\pgfqpoint{0.923823in}{0.472547in}}%
\pgfpathlineto{\pgfqpoint{0.924897in}{0.475654in}}%
\pgfpathlineto{\pgfqpoint{0.928119in}{0.475159in}}%
\pgfpathlineto{\pgfqpoint{0.929193in}{0.475853in}}%
\pgfpathlineto{\pgfqpoint{0.932414in}{0.475059in}}%
\pgfpathlineto{\pgfqpoint{0.935636in}{0.475192in}}%
\pgfpathlineto{\pgfqpoint{0.936709in}{0.476250in}}%
\pgfpathlineto{\pgfqpoint{0.937783in}{0.478333in}}%
\pgfpathlineto{\pgfqpoint{0.938857in}{0.476845in}}%
\pgfpathlineto{\pgfqpoint{0.939931in}{0.480448in}}%
\pgfpathlineto{\pgfqpoint{0.943153in}{0.478200in}}%
\pgfpathlineto{\pgfqpoint{0.944226in}{0.478167in}}%
\pgfpathlineto{\pgfqpoint{0.946374in}{0.475555in}}%
\pgfpathlineto{\pgfqpoint{0.947448in}{0.477341in}}%
\pgfpathlineto{\pgfqpoint{0.951743in}{0.477010in}}%
\pgfpathlineto{\pgfqpoint{0.952817in}{0.475588in}}%
\pgfpathlineto{\pgfqpoint{0.953891in}{0.477770in}}%
\pgfpathlineto{\pgfqpoint{0.954965in}{0.475588in}}%
\pgfpathlineto{\pgfqpoint{0.959260in}{0.477142in}}%
\pgfpathlineto{\pgfqpoint{0.960334in}{0.474894in}}%
\pgfpathlineto{\pgfqpoint{0.961408in}{0.476514in}}%
\pgfpathlineto{\pgfqpoint{0.965703in}{0.478762in}}%
\pgfpathlineto{\pgfqpoint{0.966777in}{0.476746in}}%
\pgfpathlineto{\pgfqpoint{0.967851in}{0.476349in}}%
\pgfpathlineto{\pgfqpoint{0.968925in}{0.478696in}}%
\pgfpathlineto{\pgfqpoint{0.969998in}{0.477704in}}%
\pgfpathlineto{\pgfqpoint{0.973220in}{0.478762in}}%
\pgfpathlineto{\pgfqpoint{0.974294in}{0.480052in}}%
\pgfpathlineto{\pgfqpoint{0.975368in}{0.478994in}}%
\pgfpathlineto{\pgfqpoint{0.976442in}{0.475820in}}%
\pgfpathlineto{\pgfqpoint{0.977515in}{0.476613in}}%
\pgfpathlineto{\pgfqpoint{0.980737in}{0.475688in}}%
\pgfpathlineto{\pgfqpoint{0.983958in}{0.480217in}}%
\pgfpathlineto{\pgfqpoint{0.985032in}{0.479093in}}%
\pgfpathlineto{\pgfqpoint{0.990401in}{0.482333in}}%
\pgfpathlineto{\pgfqpoint{0.992549in}{0.485176in}}%
\pgfpathlineto{\pgfqpoint{0.997918in}{0.482069in}}%
\pgfpathlineto{\pgfqpoint{0.998992in}{0.484548in}}%
\pgfpathlineto{\pgfqpoint{1.000066in}{0.485110in}}%
\pgfpathlineto{\pgfqpoint{1.005435in}{0.484317in}}%
\pgfpathlineto{\pgfqpoint{1.006509in}{0.485342in}}%
\pgfpathlineto{\pgfqpoint{1.007583in}{0.484515in}}%
\pgfpathlineto{\pgfqpoint{1.010804in}{0.485507in}}%
\pgfpathlineto{\pgfqpoint{1.012952in}{0.482102in}}%
\pgfpathlineto{\pgfqpoint{1.014026in}{0.488086in}}%
\pgfpathlineto{\pgfqpoint{1.015100in}{0.486995in}}%
\pgfpathlineto{\pgfqpoint{1.019395in}{0.485672in}}%
\pgfpathlineto{\pgfqpoint{1.020469in}{0.474729in}}%
\pgfpathlineto{\pgfqpoint{1.021543in}{0.476349in}}%
\pgfpathlineto{\pgfqpoint{1.022617in}{0.480019in}}%
\pgfpathlineto{\pgfqpoint{1.025838in}{0.480448in}}%
\pgfpathlineto{\pgfqpoint{1.027986in}{0.478035in}}%
\pgfpathlineto{\pgfqpoint{1.030133in}{0.476845in}}%
\pgfpathlineto{\pgfqpoint{1.035503in}{0.476746in}}%
\pgfpathlineto{\pgfqpoint{1.036576in}{0.473373in}}%
\pgfpathlineto{\pgfqpoint{1.037650in}{0.472745in}}%
\pgfpathlineto{\pgfqpoint{1.040872in}{0.474167in}}%
\pgfpathlineto{\pgfqpoint{1.041946in}{0.472811in}}%
\pgfpathlineto{\pgfqpoint{1.043020in}{0.476779in}}%
\pgfpathlineto{\pgfqpoint{1.045167in}{0.477308in}}%
\pgfpathlineto{\pgfqpoint{1.049463in}{0.473538in}}%
\pgfpathlineto{\pgfqpoint{1.051610in}{0.474464in}}%
\pgfpathlineto{\pgfqpoint{1.052684in}{0.473737in}}%
\pgfpathlineto{\pgfqpoint{1.056979in}{0.475721in}}%
\pgfpathlineto{\pgfqpoint{1.058053in}{0.475092in}}%
\pgfpathlineto{\pgfqpoint{1.060201in}{0.475456in}}%
\pgfpathlineto{\pgfqpoint{1.063422in}{0.476911in}}%
\pgfpathlineto{\pgfqpoint{1.064496in}{0.481506in}}%
\pgfpathlineto{\pgfqpoint{1.065570in}{0.482829in}}%
\pgfpathlineto{\pgfqpoint{1.066644in}{0.481870in}}%
\pgfpathlineto{\pgfqpoint{1.067718in}{0.484879in}}%
\pgfpathlineto{\pgfqpoint{1.070939in}{0.485176in}}%
\pgfpathlineto{\pgfqpoint{1.075235in}{0.492384in}}%
\pgfpathlineto{\pgfqpoint{1.078456in}{0.490400in}}%
\pgfpathlineto{\pgfqpoint{1.080604in}{0.486830in}}%
\pgfpathlineto{\pgfqpoint{1.081678in}{0.488350in}}%
\pgfpathlineto{\pgfqpoint{1.085973in}{0.486499in}}%
\pgfpathlineto{\pgfqpoint{1.087047in}{0.488119in}}%
\pgfpathlineto{\pgfqpoint{1.088121in}{0.487028in}}%
\pgfpathlineto{\pgfqpoint{1.089195in}{0.484614in}}%
\pgfpathlineto{\pgfqpoint{1.090268in}{0.486003in}}%
\pgfpathlineto{\pgfqpoint{1.093490in}{0.482829in}}%
\pgfpathlineto{\pgfqpoint{1.094564in}{0.479953in}}%
\pgfpathlineto{\pgfqpoint{1.095638in}{0.480944in}}%
\pgfpathlineto{\pgfqpoint{1.097785in}{0.487325in}}%
\pgfpathlineto{\pgfqpoint{1.101007in}{0.488284in}}%
\pgfpathlineto{\pgfqpoint{1.102081in}{0.486697in}}%
\pgfpathlineto{\pgfqpoint{1.104228in}{0.491987in}}%
\pgfpathlineto{\pgfqpoint{1.105302in}{0.493640in}}%
\pgfpathlineto{\pgfqpoint{1.109597in}{0.493277in}}%
\pgfpathlineto{\pgfqpoint{1.110671in}{0.492450in}}%
\pgfpathlineto{\pgfqpoint{1.111745in}{0.495525in}}%
\pgfpathlineto{\pgfqpoint{1.118188in}{0.496219in}}%
\pgfpathlineto{\pgfqpoint{1.119262in}{0.490731in}}%
\pgfpathlineto{\pgfqpoint{1.120336in}{0.492648in}}%
\pgfpathlineto{\pgfqpoint{1.123557in}{0.490566in}}%
\pgfpathlineto{\pgfqpoint{1.125705in}{0.492549in}}%
\pgfpathlineto{\pgfqpoint{1.126779in}{0.490301in}}%
\pgfpathlineto{\pgfqpoint{1.127853in}{0.491987in}}%
\pgfpathlineto{\pgfqpoint{1.131074in}{0.492715in}}%
\pgfpathlineto{\pgfqpoint{1.132148in}{0.491954in}}%
\pgfpathlineto{\pgfqpoint{1.133222in}{0.494368in}}%
\pgfpathlineto{\pgfqpoint{1.134296in}{0.494632in}}%
\pgfpathlineto{\pgfqpoint{1.135370in}{0.496054in}}%
\pgfpathlineto{\pgfqpoint{1.138591in}{0.494467in}}%
\pgfpathlineto{\pgfqpoint{1.139665in}{0.492516in}}%
\pgfpathlineto{\pgfqpoint{1.140739in}{0.493078in}}%
\pgfpathlineto{\pgfqpoint{1.141813in}{0.495756in}}%
\pgfpathlineto{\pgfqpoint{1.142886in}{0.496153in}}%
\pgfpathlineto{\pgfqpoint{1.146108in}{0.496021in}}%
\pgfpathlineto{\pgfqpoint{1.147182in}{0.497244in}}%
\pgfpathlineto{\pgfqpoint{1.148256in}{0.497608in}}%
\pgfpathlineto{\pgfqpoint{1.150403in}{0.497178in}}%
\pgfpathlineto{\pgfqpoint{1.153625in}{0.498467in}}%
\pgfpathlineto{\pgfqpoint{1.154699in}{0.495889in}}%
\pgfpathlineto{\pgfqpoint{1.155773in}{0.496649in}}%
\pgfpathlineto{\pgfqpoint{1.156846in}{0.495856in}}%
\pgfpathlineto{\pgfqpoint{1.161142in}{0.495756in}}%
\pgfpathlineto{\pgfqpoint{1.162216in}{0.494103in}}%
\pgfpathlineto{\pgfqpoint{1.163289in}{0.498897in}}%
\pgfpathlineto{\pgfqpoint{1.164363in}{0.497178in}}%
\pgfpathlineto{\pgfqpoint{1.165437in}{0.500187in}}%
\pgfpathlineto{\pgfqpoint{1.168659in}{0.500484in}}%
\pgfpathlineto{\pgfqpoint{1.169732in}{0.504749in}}%
\pgfpathlineto{\pgfqpoint{1.170806in}{0.506336in}}%
\pgfpathlineto{\pgfqpoint{1.171880in}{0.506898in}}%
\pgfpathlineto{\pgfqpoint{1.172954in}{0.506832in}}%
\pgfpathlineto{\pgfqpoint{1.179397in}{0.510072in}}%
\pgfpathlineto{\pgfqpoint{1.180471in}{0.509676in}}%
\pgfpathlineto{\pgfqpoint{1.183692in}{0.510634in}}%
\pgfpathlineto{\pgfqpoint{1.184766in}{0.511990in}}%
\pgfpathlineto{\pgfqpoint{1.186914in}{0.510700in}}%
\pgfpathlineto{\pgfqpoint{1.187988in}{0.510833in}}%
\pgfpathlineto{\pgfqpoint{1.191209in}{0.509808in}}%
\pgfpathlineto{\pgfqpoint{1.193357in}{0.511626in}}%
\pgfpathlineto{\pgfqpoint{1.195505in}{0.510800in}}%
\pgfpathlineto{\pgfqpoint{1.198726in}{0.509080in}}%
\pgfpathlineto{\pgfqpoint{1.199800in}{0.511990in}}%
\pgfpathlineto{\pgfqpoint{1.200874in}{0.512816in}}%
\pgfpathlineto{\pgfqpoint{1.201948in}{0.511296in}}%
\pgfpathlineto{\pgfqpoint{1.203021in}{0.519330in}}%
\pgfpathlineto{\pgfqpoint{1.207317in}{0.519131in}}%
\pgfpathlineto{\pgfqpoint{1.208391in}{0.519925in}}%
\pgfpathlineto{\pgfqpoint{1.210538in}{0.510899in}}%
\pgfpathlineto{\pgfqpoint{1.213760in}{0.507031in}}%
\pgfpathlineto{\pgfqpoint{1.214834in}{0.510667in}}%
\pgfpathlineto{\pgfqpoint{1.215908in}{0.507725in}}%
\pgfpathlineto{\pgfqpoint{1.216981in}{0.510601in}}%
\pgfpathlineto{\pgfqpoint{1.218055in}{0.506402in}}%
\pgfpathlineto{\pgfqpoint{1.221277in}{0.504915in}}%
\pgfpathlineto{\pgfqpoint{1.223424in}{0.506568in}}%
\pgfpathlineto{\pgfqpoint{1.225572in}{0.511296in}}%
\pgfpathlineto{\pgfqpoint{1.228794in}{0.510370in}}%
\pgfpathlineto{\pgfqpoint{1.229867in}{0.511725in}}%
\pgfpathlineto{\pgfqpoint{1.230941in}{0.514403in}}%
\pgfpathlineto{\pgfqpoint{1.232015in}{0.514304in}}%
\pgfpathlineto{\pgfqpoint{1.233089in}{0.515825in}}%
\pgfpathlineto{\pgfqpoint{1.237384in}{0.515858in}}%
\pgfpathlineto{\pgfqpoint{1.238458in}{0.514172in}}%
\pgfpathlineto{\pgfqpoint{1.240606in}{0.513775in}}%
\pgfpathlineto{\pgfqpoint{1.244901in}{0.516685in}}%
\pgfpathlineto{\pgfqpoint{1.245975in}{0.515891in}}%
\pgfpathlineto{\pgfqpoint{1.248123in}{0.515759in}}%
\pgfpathlineto{\pgfqpoint{1.251344in}{0.512287in}}%
\pgfpathlineto{\pgfqpoint{1.252418in}{0.515428in}}%
\pgfpathlineto{\pgfqpoint{1.253492in}{0.513345in}}%
\pgfpathlineto{\pgfqpoint{1.255640in}{0.515461in}}%
\pgfpathlineto{\pgfqpoint{1.258861in}{0.515428in}}%
\pgfpathlineto{\pgfqpoint{1.259935in}{0.516718in}}%
\pgfpathlineto{\pgfqpoint{1.261009in}{0.515891in}}%
\pgfpathlineto{\pgfqpoint{1.262083in}{0.511758in}}%
\pgfpathlineto{\pgfqpoint{1.263156in}{0.511758in}}%
\pgfpathlineto{\pgfqpoint{1.266378in}{0.514106in}}%
\pgfpathlineto{\pgfqpoint{1.267452in}{0.516123in}}%
\pgfpathlineto{\pgfqpoint{1.269599in}{0.512585in}}%
\pgfpathlineto{\pgfqpoint{1.270673in}{0.513775in}}%
\pgfpathlineto{\pgfqpoint{1.273895in}{0.511758in}}%
\pgfpathlineto{\pgfqpoint{1.276042in}{0.507824in}}%
\pgfpathlineto{\pgfqpoint{1.277116in}{0.507890in}}%
\pgfpathlineto{\pgfqpoint{1.278190in}{0.505014in}}%
\pgfpathlineto{\pgfqpoint{1.281412in}{0.507956in}}%
\pgfpathlineto{\pgfqpoint{1.282485in}{0.507064in}}%
\pgfpathlineto{\pgfqpoint{1.284633in}{0.507295in}}%
\pgfpathlineto{\pgfqpoint{1.285707in}{0.501608in}}%
\pgfpathlineto{\pgfqpoint{1.290002in}{0.497641in}}%
\pgfpathlineto{\pgfqpoint{1.291076in}{0.501509in}}%
\pgfpathlineto{\pgfqpoint{1.293224in}{0.493045in}}%
\pgfpathlineto{\pgfqpoint{1.296445in}{0.496418in}}%
\pgfpathlineto{\pgfqpoint{1.297519in}{0.498798in}}%
\pgfpathlineto{\pgfqpoint{1.298593in}{0.502898in}}%
\pgfpathlineto{\pgfqpoint{1.299667in}{0.501807in}}%
\pgfpathlineto{\pgfqpoint{1.305036in}{0.503394in}}%
\pgfpathlineto{\pgfqpoint{1.306110in}{0.502501in}}%
\pgfpathlineto{\pgfqpoint{1.307184in}{0.502931in}}%
\pgfpathlineto{\pgfqpoint{1.308258in}{0.494831in}}%
\pgfpathlineto{\pgfqpoint{1.313627in}{0.497674in}}%
\pgfpathlineto{\pgfqpoint{1.314701in}{0.500385in}}%
\pgfpathlineto{\pgfqpoint{1.315774in}{0.499063in}}%
\pgfpathlineto{\pgfqpoint{1.318996in}{0.501179in}}%
\pgfpathlineto{\pgfqpoint{1.320070in}{0.499790in}}%
\pgfpathlineto{\pgfqpoint{1.322218in}{0.504088in}}%
\pgfpathlineto{\pgfqpoint{1.323291in}{0.504022in}}%
\pgfpathlineto{\pgfqpoint{1.327587in}{0.505014in}}%
\pgfpathlineto{\pgfqpoint{1.328661in}{0.504518in}}%
\pgfpathlineto{\pgfqpoint{1.329734in}{0.502633in}}%
\pgfpathlineto{\pgfqpoint{1.330808in}{0.504485in}}%
\pgfpathlineto{\pgfqpoint{1.334030in}{0.504915in}}%
\pgfpathlineto{\pgfqpoint{1.335104in}{0.503030in}}%
\pgfpathlineto{\pgfqpoint{1.336177in}{0.504716in}}%
\pgfpathlineto{\pgfqpoint{1.337251in}{0.504154in}}%
\pgfpathlineto{\pgfqpoint{1.338325in}{0.506270in}}%
\pgfpathlineto{\pgfqpoint{1.342620in}{0.508155in}}%
\pgfpathlineto{\pgfqpoint{1.343694in}{0.507626in}}%
\pgfpathlineto{\pgfqpoint{1.345842in}{0.508419in}}%
\pgfpathlineto{\pgfqpoint{1.349063in}{0.507394in}}%
\pgfpathlineto{\pgfqpoint{1.350137in}{0.505675in}}%
\pgfpathlineto{\pgfqpoint{1.352285in}{0.506369in}}%
\pgfpathlineto{\pgfqpoint{1.353359in}{0.506997in}}%
\pgfpathlineto{\pgfqpoint{1.356580in}{0.506667in}}%
\pgfpathlineto{\pgfqpoint{1.357654in}{0.507956in}}%
\pgfpathlineto{\pgfqpoint{1.359802in}{0.506039in}}%
\pgfpathlineto{\pgfqpoint{1.364097in}{0.504815in}}%
\pgfpathlineto{\pgfqpoint{1.366245in}{0.505609in}}%
\pgfpathlineto{\pgfqpoint{1.368393in}{0.504253in}}%
\pgfpathlineto{\pgfqpoint{1.371614in}{0.504220in}}%
\pgfpathlineto{\pgfqpoint{1.372688in}{0.502931in}}%
\pgfpathlineto{\pgfqpoint{1.373762in}{0.503923in}}%
\pgfpathlineto{\pgfqpoint{1.375909in}{0.504088in}}%
\pgfpathlineto{\pgfqpoint{1.379131in}{0.505212in}}%
\pgfpathlineto{\pgfqpoint{1.380205in}{0.507956in}}%
\pgfpathlineto{\pgfqpoint{1.381279in}{0.508419in}}%
\pgfpathlineto{\pgfqpoint{1.382352in}{0.509742in}}%
\pgfpathlineto{\pgfqpoint{1.386648in}{0.509907in}}%
\pgfpathlineto{\pgfqpoint{1.387722in}{0.508750in}}%
\pgfpathlineto{\pgfqpoint{1.388796in}{0.509444in}}%
\pgfpathlineto{\pgfqpoint{1.389869in}{0.508981in}}%
\pgfpathlineto{\pgfqpoint{1.395239in}{0.513841in}}%
\pgfpathlineto{\pgfqpoint{1.396312in}{0.514536in}}%
\pgfpathlineto{\pgfqpoint{1.397386in}{0.510800in}}%
\pgfpathlineto{\pgfqpoint{1.398460in}{0.512618in}}%
\pgfpathlineto{\pgfqpoint{1.401682in}{0.511858in}}%
\pgfpathlineto{\pgfqpoint{1.402755in}{0.513445in}}%
\pgfpathlineto{\pgfqpoint{1.403829in}{0.513412in}}%
\pgfpathlineto{\pgfqpoint{1.404903in}{0.514602in}}%
\pgfpathlineto{\pgfqpoint{1.405977in}{0.508386in}}%
\pgfpathlineto{\pgfqpoint{1.411346in}{0.507824in}}%
\pgfpathlineto{\pgfqpoint{1.412420in}{0.505444in}}%
\pgfpathlineto{\pgfqpoint{1.413494in}{0.506072in}}%
\pgfpathlineto{\pgfqpoint{1.416715in}{0.506336in}}%
\pgfpathlineto{\pgfqpoint{1.417789in}{0.505080in}}%
\pgfpathlineto{\pgfqpoint{1.418863in}{0.505179in}}%
\pgfpathlineto{\pgfqpoint{1.419937in}{0.503890in}}%
\pgfpathlineto{\pgfqpoint{1.421011in}{0.505014in}}%
\pgfpathlineto{\pgfqpoint{1.425306in}{0.504882in}}%
\pgfpathlineto{\pgfqpoint{1.426380in}{0.507031in}}%
\pgfpathlineto{\pgfqpoint{1.427454in}{0.507890in}}%
\pgfpathlineto{\pgfqpoint{1.428528in}{0.506039in}}%
\pgfpathlineto{\pgfqpoint{1.433897in}{0.510800in}}%
\pgfpathlineto{\pgfqpoint{1.434971in}{0.510436in}}%
\pgfpathlineto{\pgfqpoint{1.436044in}{0.510700in}}%
\pgfpathlineto{\pgfqpoint{1.439266in}{0.510634in}}%
\pgfpathlineto{\pgfqpoint{1.441414in}{0.511527in}}%
\pgfpathlineto{\pgfqpoint{1.443561in}{0.507890in}}%
\pgfpathlineto{\pgfqpoint{1.448930in}{0.509775in}}%
\pgfpathlineto{\pgfqpoint{1.451078in}{0.509213in}}%
\pgfpathlineto{\pgfqpoint{1.454300in}{0.510436in}}%
\pgfpathlineto{\pgfqpoint{1.455373in}{0.509279in}}%
\pgfpathlineto{\pgfqpoint{1.456447in}{0.511329in}}%
\pgfpathlineto{\pgfqpoint{1.458595in}{0.509080in}}%
\pgfpathlineto{\pgfqpoint{1.461817in}{0.509543in}}%
\pgfpathlineto{\pgfqpoint{1.462890in}{0.511659in}}%
\pgfpathlineto{\pgfqpoint{1.463964in}{0.510271in}}%
\pgfpathlineto{\pgfqpoint{1.465038in}{0.510965in}}%
\pgfpathlineto{\pgfqpoint{1.466112in}{0.510833in}}%
\pgfpathlineto{\pgfqpoint{1.470407in}{0.507824in}}%
\pgfpathlineto{\pgfqpoint{1.471481in}{0.509378in}}%
\pgfpathlineto{\pgfqpoint{1.472555in}{0.506468in}}%
\pgfpathlineto{\pgfqpoint{1.473629in}{0.507427in}}%
\pgfpathlineto{\pgfqpoint{1.476850in}{0.506634in}}%
\pgfpathlineto{\pgfqpoint{1.477924in}{0.508551in}}%
\pgfpathlineto{\pgfqpoint{1.478998in}{0.505906in}}%
\pgfpathlineto{\pgfqpoint{1.480072in}{0.505675in}}%
\pgfpathlineto{\pgfqpoint{1.481146in}{0.507460in}}%
\pgfpathlineto{\pgfqpoint{1.484367in}{0.507295in}}%
\pgfpathlineto{\pgfqpoint{1.485441in}{0.504353in}}%
\pgfpathlineto{\pgfqpoint{1.486515in}{0.507758in}}%
\pgfpathlineto{\pgfqpoint{1.488662in}{0.501939in}}%
\pgfpathlineto{\pgfqpoint{1.491884in}{0.501410in}}%
\pgfpathlineto{\pgfqpoint{1.494032in}{0.498236in}}%
\pgfpathlineto{\pgfqpoint{1.496179in}{0.502766in}}%
\pgfpathlineto{\pgfqpoint{1.499401in}{0.504220in}}%
\pgfpathlineto{\pgfqpoint{1.500475in}{0.508518in}}%
\pgfpathlineto{\pgfqpoint{1.501549in}{0.506634in}}%
\pgfpathlineto{\pgfqpoint{1.502622in}{0.509279in}}%
\pgfpathlineto{\pgfqpoint{1.503696in}{0.508651in}}%
\pgfpathlineto{\pgfqpoint{1.506918in}{0.508584in}}%
\pgfpathlineto{\pgfqpoint{1.507992in}{0.511196in}}%
\pgfpathlineto{\pgfqpoint{1.509065in}{0.509576in}}%
\pgfpathlineto{\pgfqpoint{1.510139in}{0.526901in}}%
\pgfpathlineto{\pgfqpoint{1.511213in}{0.530670in}}%
\pgfpathlineto{\pgfqpoint{1.514435in}{0.530736in}}%
\pgfpathlineto{\pgfqpoint{1.515508in}{0.531860in}}%
\pgfpathlineto{\pgfqpoint{1.516582in}{0.537051in}}%
\pgfpathlineto{\pgfqpoint{1.517656in}{0.537481in}}%
\pgfpathlineto{\pgfqpoint{1.518730in}{0.539332in}}%
\pgfpathlineto{\pgfqpoint{1.523025in}{0.537183in}}%
\pgfpathlineto{\pgfqpoint{1.524099in}{0.540490in}}%
\pgfpathlineto{\pgfqpoint{1.525173in}{0.539663in}}%
\pgfpathlineto{\pgfqpoint{1.526247in}{0.538010in}}%
\pgfpathlineto{\pgfqpoint{1.531616in}{0.538836in}}%
\pgfpathlineto{\pgfqpoint{1.533764in}{0.542275in}}%
\pgfpathlineto{\pgfqpoint{1.536985in}{0.542606in}}%
\pgfpathlineto{\pgfqpoint{1.538059in}{0.544688in}}%
\pgfpathlineto{\pgfqpoint{1.539133in}{0.544688in}}%
\pgfpathlineto{\pgfqpoint{1.541281in}{0.545416in}}%
\pgfpathlineto{\pgfqpoint{1.544502in}{0.545416in}}%
\pgfpathlineto{\pgfqpoint{1.546650in}{0.548127in}}%
\pgfpathlineto{\pgfqpoint{1.547724in}{0.547763in}}%
\pgfpathlineto{\pgfqpoint{1.548797in}{0.549516in}}%
\pgfpathlineto{\pgfqpoint{1.552019in}{0.549317in}}%
\pgfpathlineto{\pgfqpoint{1.553093in}{0.550144in}}%
\pgfpathlineto{\pgfqpoint{1.554167in}{0.548226in}}%
\pgfpathlineto{\pgfqpoint{1.555240in}{0.549350in}}%
\pgfpathlineto{\pgfqpoint{1.556314in}{0.544325in}}%
\pgfpathlineto{\pgfqpoint{1.559536in}{0.544259in}}%
\pgfpathlineto{\pgfqpoint{1.560610in}{0.541647in}}%
\pgfpathlineto{\pgfqpoint{1.562757in}{0.550177in}}%
\pgfpathlineto{\pgfqpoint{1.563831in}{0.548193in}}%
\pgfpathlineto{\pgfqpoint{1.568127in}{0.551036in}}%
\pgfpathlineto{\pgfqpoint{1.569200in}{0.552921in}}%
\pgfpathlineto{\pgfqpoint{1.575643in}{0.550474in}}%
\pgfpathlineto{\pgfqpoint{1.576717in}{0.548623in}}%
\pgfpathlineto{\pgfqpoint{1.578865in}{0.550838in}}%
\pgfpathlineto{\pgfqpoint{1.583160in}{0.544887in}}%
\pgfpathlineto{\pgfqpoint{1.585308in}{0.550408in}}%
\pgfpathlineto{\pgfqpoint{1.586382in}{0.547300in}}%
\pgfpathlineto{\pgfqpoint{1.589603in}{0.546871in}}%
\pgfpathlineto{\pgfqpoint{1.590677in}{0.547499in}}%
\pgfpathlineto{\pgfqpoint{1.591751in}{0.543333in}}%
\pgfpathlineto{\pgfqpoint{1.592825in}{0.541415in}}%
\pgfpathlineto{\pgfqpoint{1.593899in}{0.542870in}}%
\pgfpathlineto{\pgfqpoint{1.601416in}{0.545515in}}%
\pgfpathlineto{\pgfqpoint{1.604637in}{0.544126in}}%
\pgfpathlineto{\pgfqpoint{1.606785in}{0.536026in}}%
\pgfpathlineto{\pgfqpoint{1.607859in}{0.537349in}}%
\pgfpathlineto{\pgfqpoint{1.608932in}{0.542837in}}%
\pgfpathlineto{\pgfqpoint{1.612154in}{0.543168in}}%
\pgfpathlineto{\pgfqpoint{1.614302in}{0.550739in}}%
\pgfpathlineto{\pgfqpoint{1.615375in}{0.556227in}}%
\pgfpathlineto{\pgfqpoint{1.616449in}{0.552756in}}%
\pgfpathlineto{\pgfqpoint{1.620745in}{0.550474in}}%
\pgfpathlineto{\pgfqpoint{1.622892in}{0.557087in}}%
\pgfpathlineto{\pgfqpoint{1.623966in}{0.556062in}}%
\pgfpathlineto{\pgfqpoint{1.628261in}{0.557054in}}%
\pgfpathlineto{\pgfqpoint{1.629335in}{0.555665in}}%
\pgfpathlineto{\pgfqpoint{1.630409in}{0.555665in}}%
\pgfpathlineto{\pgfqpoint{1.631483in}{0.558773in}}%
\pgfpathlineto{\pgfqpoint{1.636852in}{0.558773in}}%
\pgfpathlineto{\pgfqpoint{1.637926in}{0.559368in}}%
\pgfpathlineto{\pgfqpoint{1.639000in}{0.557417in}}%
\pgfpathlineto{\pgfqpoint{1.642221in}{0.563005in}}%
\pgfpathlineto{\pgfqpoint{1.644369in}{0.559368in}}%
\pgfpathlineto{\pgfqpoint{1.645443in}{0.559666in}}%
\pgfpathlineto{\pgfqpoint{1.646517in}{0.555830in}}%
\pgfpathlineto{\pgfqpoint{1.649738in}{0.557517in}}%
\pgfpathlineto{\pgfqpoint{1.650812in}{0.552524in}}%
\pgfpathlineto{\pgfqpoint{1.651886in}{0.552161in}}%
\pgfpathlineto{\pgfqpoint{1.652960in}{0.556029in}}%
\pgfpathlineto{\pgfqpoint{1.654034in}{0.552392in}}%
\pgfpathlineto{\pgfqpoint{1.657255in}{0.555599in}}%
\pgfpathlineto{\pgfqpoint{1.658329in}{0.551962in}}%
\pgfpathlineto{\pgfqpoint{1.659403in}{0.554508in}}%
\pgfpathlineto{\pgfqpoint{1.660477in}{0.554177in}}%
\pgfpathlineto{\pgfqpoint{1.661550in}{0.556095in}}%
\pgfpathlineto{\pgfqpoint{1.665846in}{0.555103in}}%
\pgfpathlineto{\pgfqpoint{1.666920in}{0.550706in}}%
\pgfpathlineto{\pgfqpoint{1.669067in}{0.550111in}}%
\pgfpathlineto{\pgfqpoint{1.672289in}{0.550474in}}%
\pgfpathlineto{\pgfqpoint{1.674437in}{0.548954in}}%
\pgfpathlineto{\pgfqpoint{1.675510in}{0.549317in}}%
\pgfpathlineto{\pgfqpoint{1.679806in}{0.549020in}}%
\pgfpathlineto{\pgfqpoint{1.681953in}{0.553185in}}%
\pgfpathlineto{\pgfqpoint{1.684101in}{0.552656in}}%
\pgfpathlineto{\pgfqpoint{1.688396in}{0.550078in}}%
\pgfpathlineto{\pgfqpoint{1.690544in}{0.550441in}}%
\pgfpathlineto{\pgfqpoint{1.691618in}{0.546838in}}%
\pgfpathlineto{\pgfqpoint{1.694839in}{0.547466in}}%
\pgfpathlineto{\pgfqpoint{1.695913in}{0.549483in}}%
\pgfpathlineto{\pgfqpoint{1.696987in}{0.558013in}}%
\pgfpathlineto{\pgfqpoint{1.699135in}{0.556293in}}%
\pgfpathlineto{\pgfqpoint{1.702356in}{0.555103in}}%
\pgfpathlineto{\pgfqpoint{1.703430in}{0.554045in}}%
\pgfpathlineto{\pgfqpoint{1.704504in}{0.555864in}}%
\pgfpathlineto{\pgfqpoint{1.705578in}{0.551731in}}%
\pgfpathlineto{\pgfqpoint{1.706652in}{0.550838in}}%
\pgfpathlineto{\pgfqpoint{1.709873in}{0.550243in}}%
\pgfpathlineto{\pgfqpoint{1.710947in}{0.551433in}}%
\pgfpathlineto{\pgfqpoint{1.712021in}{0.550507in}}%
\pgfpathlineto{\pgfqpoint{1.713095in}{0.553417in}}%
\pgfpathlineto{\pgfqpoint{1.714169in}{0.562674in}}%
\pgfpathlineto{\pgfqpoint{1.719538in}{0.560426in}}%
\pgfpathlineto{\pgfqpoint{1.720612in}{0.564757in}}%
\pgfpathlineto{\pgfqpoint{1.721685in}{0.563369in}}%
\pgfpathlineto{\pgfqpoint{1.725981in}{0.565286in}}%
\pgfpathlineto{\pgfqpoint{1.728128in}{0.562740in}}%
\pgfpathlineto{\pgfqpoint{1.729202in}{0.563534in}}%
\pgfpathlineto{\pgfqpoint{1.733498in}{0.560095in}}%
\pgfpathlineto{\pgfqpoint{1.734572in}{0.563137in}}%
\pgfpathlineto{\pgfqpoint{1.735645in}{0.563336in}}%
\pgfpathlineto{\pgfqpoint{1.736719in}{0.560525in}}%
\pgfpathlineto{\pgfqpoint{1.739941in}{0.561914in}}%
\pgfpathlineto{\pgfqpoint{1.742088in}{0.561385in}}%
\pgfpathlineto{\pgfqpoint{1.743162in}{0.559037in}}%
\pgfpathlineto{\pgfqpoint{1.744236in}{0.559533in}}%
\pgfpathlineto{\pgfqpoint{1.747458in}{0.557318in}}%
\pgfpathlineto{\pgfqpoint{1.748531in}{0.558178in}}%
\pgfpathlineto{\pgfqpoint{1.749605in}{0.563567in}}%
\pgfpathlineto{\pgfqpoint{1.750679in}{0.563600in}}%
\pgfpathlineto{\pgfqpoint{1.754974in}{0.560195in}}%
\pgfpathlineto{\pgfqpoint{1.756048in}{0.561583in}}%
\pgfpathlineto{\pgfqpoint{1.757122in}{0.560757in}}%
\pgfpathlineto{\pgfqpoint{1.758196in}{0.563170in}}%
\pgfpathlineto{\pgfqpoint{1.759270in}{0.560558in}}%
\pgfpathlineto{\pgfqpoint{1.762491in}{0.561848in}}%
\pgfpathlineto{\pgfqpoint{1.763565in}{0.562906in}}%
\pgfpathlineto{\pgfqpoint{1.765713in}{0.560327in}}%
\pgfpathlineto{\pgfqpoint{1.766787in}{0.560757in}}%
\pgfpathlineto{\pgfqpoint{1.770008in}{0.554243in}}%
\pgfpathlineto{\pgfqpoint{1.773230in}{0.559137in}}%
\pgfpathlineto{\pgfqpoint{1.777525in}{0.558740in}}%
\pgfpathlineto{\pgfqpoint{1.778599in}{0.557616in}}%
\pgfpathlineto{\pgfqpoint{1.779673in}{0.554277in}}%
\pgfpathlineto{\pgfqpoint{1.780747in}{0.555301in}}%
\pgfpathlineto{\pgfqpoint{1.781820in}{0.559699in}}%
\pgfpathlineto{\pgfqpoint{1.789337in}{0.567568in}}%
\pgfpathlineto{\pgfqpoint{1.792559in}{0.573420in}}%
\pgfpathlineto{\pgfqpoint{1.793633in}{0.571237in}}%
\pgfpathlineto{\pgfqpoint{1.795780in}{0.570378in}}%
\pgfpathlineto{\pgfqpoint{1.796854in}{0.580131in}}%
\pgfpathlineto{\pgfqpoint{1.800076in}{0.577123in}}%
\pgfpathlineto{\pgfqpoint{1.803297in}{0.585190in}}%
\pgfpathlineto{\pgfqpoint{1.804371in}{0.581884in}}%
\pgfpathlineto{\pgfqpoint{1.807593in}{0.583239in}}%
\pgfpathlineto{\pgfqpoint{1.809740in}{0.580726in}}%
\pgfpathlineto{\pgfqpoint{1.810814in}{0.576230in}}%
\pgfpathlineto{\pgfqpoint{1.811888in}{0.578247in}}%
\pgfpathlineto{\pgfqpoint{1.815109in}{0.578809in}}%
\pgfpathlineto{\pgfqpoint{1.816183in}{0.575536in}}%
\pgfpathlineto{\pgfqpoint{1.819405in}{0.578676in}}%
\pgfpathlineto{\pgfqpoint{1.824774in}{0.579239in}}%
\pgfpathlineto{\pgfqpoint{1.825848in}{0.577817in}}%
\pgfpathlineto{\pgfqpoint{1.826922in}{0.568956in}}%
\pgfpathlineto{\pgfqpoint{1.831217in}{0.555401in}}%
\pgfpathlineto{\pgfqpoint{1.832291in}{0.567336in}}%
\pgfpathlineto{\pgfqpoint{1.833365in}{0.572824in}}%
\pgfpathlineto{\pgfqpoint{1.834438in}{0.573023in}}%
\pgfpathlineto{\pgfqpoint{1.837660in}{0.569287in}}%
\pgfpathlineto{\pgfqpoint{1.838734in}{0.561782in}}%
\pgfpathlineto{\pgfqpoint{1.840882in}{0.566410in}}%
\pgfpathlineto{\pgfqpoint{1.841955in}{0.562443in}}%
\pgfpathlineto{\pgfqpoint{1.846251in}{0.566840in}}%
\pgfpathlineto{\pgfqpoint{1.847325in}{0.563832in}}%
\pgfpathlineto{\pgfqpoint{1.849472in}{0.567568in}}%
\pgfpathlineto{\pgfqpoint{1.852694in}{0.565154in}}%
\pgfpathlineto{\pgfqpoint{1.854841in}{0.568295in}}%
\pgfpathlineto{\pgfqpoint{1.855915in}{0.568196in}}%
\pgfpathlineto{\pgfqpoint{1.856989in}{0.564460in}}%
\pgfpathlineto{\pgfqpoint{1.860211in}{0.567634in}}%
\pgfpathlineto{\pgfqpoint{1.861284in}{0.565914in}}%
\pgfpathlineto{\pgfqpoint{1.862358in}{0.568163in}}%
\pgfpathlineto{\pgfqpoint{1.863432in}{0.565881in}}%
\pgfpathlineto{\pgfqpoint{1.864506in}{0.567336in}}%
\pgfpathlineto{\pgfqpoint{1.867727in}{0.556128in}}%
\pgfpathlineto{\pgfqpoint{1.869875in}{0.564030in}}%
\pgfpathlineto{\pgfqpoint{1.870949in}{0.565088in}}%
\pgfpathlineto{\pgfqpoint{1.872023in}{0.567270in}}%
\pgfpathlineto{\pgfqpoint{1.875244in}{0.572295in}}%
\pgfpathlineto{\pgfqpoint{1.876318in}{0.571800in}}%
\pgfpathlineto{\pgfqpoint{1.878466in}{0.577652in}}%
\pgfpathlineto{\pgfqpoint{1.879540in}{0.577916in}}%
\pgfpathlineto{\pgfqpoint{1.882761in}{0.581156in}}%
\pgfpathlineto{\pgfqpoint{1.883835in}{0.581189in}}%
\pgfpathlineto{\pgfqpoint{1.884909in}{0.578610in}}%
\pgfpathlineto{\pgfqpoint{1.887057in}{0.584396in}}%
\pgfpathlineto{\pgfqpoint{1.890278in}{0.587570in}}%
\pgfpathlineto{\pgfqpoint{1.892426in}{0.582644in}}%
\pgfpathlineto{\pgfqpoint{1.894573in}{0.587802in}}%
\pgfpathlineto{\pgfqpoint{1.897795in}{0.591372in}}%
\pgfpathlineto{\pgfqpoint{1.898869in}{0.589256in}}%
\pgfpathlineto{\pgfqpoint{1.899943in}{0.593588in}}%
\pgfpathlineto{\pgfqpoint{1.901016in}{0.592430in}}%
\pgfpathlineto{\pgfqpoint{1.905312in}{0.581884in}}%
\pgfpathlineto{\pgfqpoint{1.906386in}{0.590480in}}%
\pgfpathlineto{\pgfqpoint{1.907460in}{0.591934in}}%
\pgfpathlineto{\pgfqpoint{1.908533in}{0.594844in}}%
\pgfpathlineto{\pgfqpoint{1.909607in}{0.593191in}}%
\pgfpathlineto{\pgfqpoint{1.912829in}{0.590943in}}%
\pgfpathlineto{\pgfqpoint{1.913903in}{0.596001in}}%
\pgfpathlineto{\pgfqpoint{1.914976in}{0.595042in}}%
\pgfpathlineto{\pgfqpoint{1.916050in}{0.592232in}}%
\pgfpathlineto{\pgfqpoint{1.917124in}{0.591604in}}%
\pgfpathlineto{\pgfqpoint{1.920346in}{0.594150in}}%
\pgfpathlineto{\pgfqpoint{1.921419in}{0.593918in}}%
\pgfpathlineto{\pgfqpoint{1.922493in}{0.599142in}}%
\pgfpathlineto{\pgfqpoint{1.923567in}{0.598150in}}%
\pgfpathlineto{\pgfqpoint{1.924641in}{0.598282in}}%
\pgfpathlineto{\pgfqpoint{1.927862in}{0.598018in}}%
\pgfpathlineto{\pgfqpoint{1.930010in}{0.596266in}}%
\pgfpathlineto{\pgfqpoint{1.932158in}{0.597158in}}%
\pgfpathlineto{\pgfqpoint{1.935379in}{0.594480in}}%
\pgfpathlineto{\pgfqpoint{1.936453in}{0.597390in}}%
\pgfpathlineto{\pgfqpoint{1.938601in}{0.592166in}}%
\pgfpathlineto{\pgfqpoint{1.939675in}{0.598977in}}%
\pgfpathlineto{\pgfqpoint{1.943970in}{0.594679in}}%
\pgfpathlineto{\pgfqpoint{1.945044in}{0.590876in}}%
\pgfpathlineto{\pgfqpoint{1.946118in}{0.591637in}}%
\pgfpathlineto{\pgfqpoint{1.947192in}{0.585190in}}%
\pgfpathlineto{\pgfqpoint{1.950413in}{0.587603in}}%
\pgfpathlineto{\pgfqpoint{1.952561in}{0.597224in}}%
\pgfpathlineto{\pgfqpoint{1.953635in}{0.593455in}}%
\pgfpathlineto{\pgfqpoint{1.954708in}{0.585851in}}%
\pgfpathlineto{\pgfqpoint{1.959004in}{0.589256in}}%
\pgfpathlineto{\pgfqpoint{1.960078in}{0.593059in}}%
\pgfpathlineto{\pgfqpoint{1.961151in}{0.592100in}}%
\pgfpathlineto{\pgfqpoint{1.965447in}{0.592959in}}%
\pgfpathlineto{\pgfqpoint{1.966521in}{0.595141in}}%
\pgfpathlineto{\pgfqpoint{1.968668in}{0.589818in}}%
\pgfpathlineto{\pgfqpoint{1.972964in}{0.583867in}}%
\pgfpathlineto{\pgfqpoint{1.974038in}{0.585686in}}%
\pgfpathlineto{\pgfqpoint{1.977259in}{0.574775in}}%
\pgfpathlineto{\pgfqpoint{1.980481in}{0.578147in}}%
\pgfpathlineto{\pgfqpoint{1.981554in}{0.580826in}}%
\pgfpathlineto{\pgfqpoint{1.982628in}{0.575503in}}%
\pgfpathlineto{\pgfqpoint{1.983702in}{0.577751in}}%
\pgfpathlineto{\pgfqpoint{1.984776in}{0.571403in}}%
\pgfpathlineto{\pgfqpoint{1.989071in}{0.570014in}}%
\pgfpathlineto{\pgfqpoint{1.990145in}{0.567733in}}%
\pgfpathlineto{\pgfqpoint{1.992293in}{0.574213in}}%
\pgfpathlineto{\pgfqpoint{1.995514in}{0.571138in}}%
\pgfpathlineto{\pgfqpoint{1.996588in}{0.571568in}}%
\pgfpathlineto{\pgfqpoint{1.997662in}{0.568460in}}%
\pgfpathlineto{\pgfqpoint{1.998736in}{0.563369in}}%
\pgfpathlineto{\pgfqpoint{1.999810in}{0.579966in}}%
\pgfpathlineto{\pgfqpoint{2.003031in}{0.579602in}}%
\pgfpathlineto{\pgfqpoint{2.004105in}{0.576461in}}%
\pgfpathlineto{\pgfqpoint{2.005179in}{0.579602in}}%
\pgfpathlineto{\pgfqpoint{2.006253in}{0.577354in}}%
\pgfpathlineto{\pgfqpoint{2.007326in}{0.570477in}}%
\pgfpathlineto{\pgfqpoint{2.010548in}{0.558376in}}%
\pgfpathlineto{\pgfqpoint{2.011622in}{0.560162in}}%
\pgfpathlineto{\pgfqpoint{2.012696in}{0.566477in}}%
\pgfpathlineto{\pgfqpoint{2.013770in}{0.561054in}}%
\pgfpathlineto{\pgfqpoint{2.014843in}{0.567336in}}%
\pgfpathlineto{\pgfqpoint{2.019139in}{0.569551in}}%
\pgfpathlineto{\pgfqpoint{2.020213in}{0.572990in}}%
\pgfpathlineto{\pgfqpoint{2.021286in}{0.570444in}}%
\pgfpathlineto{\pgfqpoint{2.022360in}{0.571370in}}%
\pgfpathlineto{\pgfqpoint{2.025582in}{0.576329in}}%
\pgfpathlineto{\pgfqpoint{2.026656in}{0.573387in}}%
\pgfpathlineto{\pgfqpoint{2.027729in}{0.572395in}}%
\pgfpathlineto{\pgfqpoint{2.028803in}{0.576990in}}%
\pgfpathlineto{\pgfqpoint{2.029877in}{0.575238in}}%
\pgfpathlineto{\pgfqpoint{2.033099in}{0.574147in}}%
\pgfpathlineto{\pgfqpoint{2.034172in}{0.581421in}}%
\pgfpathlineto{\pgfqpoint{2.036320in}{0.579073in}}%
\pgfpathlineto{\pgfqpoint{2.037394in}{0.579040in}}%
\pgfpathlineto{\pgfqpoint{2.040615in}{0.572692in}}%
\pgfpathlineto{\pgfqpoint{2.041689in}{0.568427in}}%
\pgfpathlineto{\pgfqpoint{2.042763in}{0.568626in}}%
\pgfpathlineto{\pgfqpoint{2.043837in}{0.567138in}}%
\pgfpathlineto{\pgfqpoint{2.044911in}{0.571700in}}%
\pgfpathlineto{\pgfqpoint{2.048132in}{0.571237in}}%
\pgfpathlineto{\pgfqpoint{2.050280in}{0.574081in}}%
\pgfpathlineto{\pgfqpoint{2.052428in}{0.578643in}}%
\pgfpathlineto{\pgfqpoint{2.055649in}{0.578610in}}%
\pgfpathlineto{\pgfqpoint{2.056723in}{0.575998in}}%
\pgfpathlineto{\pgfqpoint{2.057797in}{0.579073in}}%
\pgfpathlineto{\pgfqpoint{2.058871in}{0.579801in}}%
\pgfpathlineto{\pgfqpoint{2.063166in}{0.579602in}}%
\pgfpathlineto{\pgfqpoint{2.065314in}{0.588331in}}%
\pgfpathlineto{\pgfqpoint{2.066388in}{0.587339in}}%
\pgfpathlineto{\pgfqpoint{2.067461in}{0.590943in}}%
\pgfpathlineto{\pgfqpoint{2.070683in}{0.591703in}}%
\pgfpathlineto{\pgfqpoint{2.071757in}{0.588926in}}%
\pgfpathlineto{\pgfqpoint{2.072831in}{0.592926in}}%
\pgfpathlineto{\pgfqpoint{2.073904in}{0.590910in}}%
\pgfpathlineto{\pgfqpoint{2.074978in}{0.592364in}}%
\pgfpathlineto{\pgfqpoint{2.078200in}{0.591670in}}%
\pgfpathlineto{\pgfqpoint{2.079274in}{0.593951in}}%
\pgfpathlineto{\pgfqpoint{2.081421in}{0.599770in}}%
\pgfpathlineto{\pgfqpoint{2.082495in}{0.598977in}}%
\pgfpathlineto{\pgfqpoint{2.085717in}{0.603407in}}%
\pgfpathlineto{\pgfqpoint{2.086791in}{0.601126in}}%
\pgfpathlineto{\pgfqpoint{2.087864in}{0.602415in}}%
\pgfpathlineto{\pgfqpoint{2.088938in}{0.601258in}}%
\pgfpathlineto{\pgfqpoint{2.090012in}{0.595836in}}%
\pgfpathlineto{\pgfqpoint{2.093234in}{0.592728in}}%
\pgfpathlineto{\pgfqpoint{2.095381in}{0.594712in}}%
\pgfpathlineto{\pgfqpoint{2.096455in}{0.591240in}}%
\pgfpathlineto{\pgfqpoint{2.097529in}{0.589818in}}%
\pgfpathlineto{\pgfqpoint{2.100750in}{0.593753in}}%
\pgfpathlineto{\pgfqpoint{2.101824in}{0.589653in}}%
\pgfpathlineto{\pgfqpoint{2.102898in}{0.589256in}}%
\pgfpathlineto{\pgfqpoint{2.105046in}{0.591372in}}%
\pgfpathlineto{\pgfqpoint{2.108267in}{0.592959in}}%
\pgfpathlineto{\pgfqpoint{2.109341in}{0.596133in}}%
\pgfpathlineto{\pgfqpoint{2.110415in}{0.590447in}}%
\pgfpathlineto{\pgfqpoint{2.111489in}{0.592364in}}%
\pgfpathlineto{\pgfqpoint{2.112563in}{0.588926in}}%
\pgfpathlineto{\pgfqpoint{2.115784in}{0.592067in}}%
\pgfpathlineto{\pgfqpoint{2.116858in}{0.588794in}}%
\pgfpathlineto{\pgfqpoint{2.117932in}{0.590876in}}%
\pgfpathlineto{\pgfqpoint{2.119006in}{0.589091in}}%
\pgfpathlineto{\pgfqpoint{2.120080in}{0.591637in}}%
\pgfpathlineto{\pgfqpoint{2.123301in}{0.590149in}}%
\pgfpathlineto{\pgfqpoint{2.124375in}{0.597158in}}%
\pgfpathlineto{\pgfqpoint{2.125449in}{0.596133in}}%
\pgfpathlineto{\pgfqpoint{2.126523in}{0.595935in}}%
\pgfpathlineto{\pgfqpoint{2.127596in}{0.598084in}}%
\pgfpathlineto{\pgfqpoint{2.131892in}{0.595737in}}%
\pgfpathlineto{\pgfqpoint{2.132966in}{0.596662in}}%
\pgfpathlineto{\pgfqpoint{2.134039in}{0.599010in}}%
\pgfpathlineto{\pgfqpoint{2.135113in}{0.598977in}}%
\pgfpathlineto{\pgfqpoint{2.139409in}{0.601126in}}%
\pgfpathlineto{\pgfqpoint{2.140482in}{0.604498in}}%
\pgfpathlineto{\pgfqpoint{2.141556in}{0.603242in}}%
\pgfpathlineto{\pgfqpoint{2.142630in}{0.599770in}}%
\pgfpathlineto{\pgfqpoint{2.145852in}{0.593984in}}%
\pgfpathlineto{\pgfqpoint{2.146926in}{0.594546in}}%
\pgfpathlineto{\pgfqpoint{2.147999in}{0.593257in}}%
\pgfpathlineto{\pgfqpoint{2.149073in}{0.593852in}}%
\pgfpathlineto{\pgfqpoint{2.150147in}{0.589455in}}%
\pgfpathlineto{\pgfqpoint{2.154442in}{0.590546in}}%
\pgfpathlineto{\pgfqpoint{2.155516in}{0.587967in}}%
\pgfpathlineto{\pgfqpoint{2.156590in}{0.593455in}}%
\pgfpathlineto{\pgfqpoint{2.157664in}{0.583173in}}%
\pgfpathlineto{\pgfqpoint{2.160885in}{0.577652in}}%
\pgfpathlineto{\pgfqpoint{2.163033in}{0.588628in}}%
\pgfpathlineto{\pgfqpoint{2.164107in}{0.580330in}}%
\pgfpathlineto{\pgfqpoint{2.165181in}{0.581321in}}%
\pgfpathlineto{\pgfqpoint{2.169476in}{0.581950in}}%
\pgfpathlineto{\pgfqpoint{2.170550in}{0.579966in}}%
\pgfpathlineto{\pgfqpoint{2.171624in}{0.581421in}}%
\pgfpathlineto{\pgfqpoint{2.172698in}{0.587603in}}%
\pgfpathlineto{\pgfqpoint{2.175919in}{0.587934in}}%
\pgfpathlineto{\pgfqpoint{2.176993in}{0.591009in}}%
\pgfpathlineto{\pgfqpoint{2.178067in}{0.590976in}}%
\pgfpathlineto{\pgfqpoint{2.179141in}{0.593158in}}%
\pgfpathlineto{\pgfqpoint{2.180214in}{0.593687in}}%
\pgfpathlineto{\pgfqpoint{2.183436in}{0.593720in}}%
\pgfpathlineto{\pgfqpoint{2.185584in}{0.597092in}}%
\pgfpathlineto{\pgfqpoint{2.186658in}{0.595274in}}%
\pgfpathlineto{\pgfqpoint{2.187731in}{0.598878in}}%
\pgfpathlineto{\pgfqpoint{2.193101in}{0.594414in}}%
\pgfpathlineto{\pgfqpoint{2.194174in}{0.596563in}}%
\pgfpathlineto{\pgfqpoint{2.195248in}{0.592860in}}%
\pgfpathlineto{\pgfqpoint{2.199544in}{0.593621in}}%
\pgfpathlineto{\pgfqpoint{2.200617in}{0.595009in}}%
\pgfpathlineto{\pgfqpoint{2.202765in}{0.599638in}}%
\pgfpathlineto{\pgfqpoint{2.205987in}{0.599010in}}%
\pgfpathlineto{\pgfqpoint{2.207060in}{0.599373in}}%
\pgfpathlineto{\pgfqpoint{2.208134in}{0.598150in}}%
\pgfpathlineto{\pgfqpoint{2.209208in}{0.599572in}}%
\pgfpathlineto{\pgfqpoint{2.210282in}{0.599241in}}%
\pgfpathlineto{\pgfqpoint{2.213503in}{0.602118in}}%
\pgfpathlineto{\pgfqpoint{2.214577in}{0.601787in}}%
\pgfpathlineto{\pgfqpoint{2.215651in}{0.602349in}}%
\pgfpathlineto{\pgfqpoint{2.216725in}{0.600498in}}%
\pgfpathlineto{\pgfqpoint{2.221020in}{0.603043in}}%
\pgfpathlineto{\pgfqpoint{2.222094in}{0.602217in}}%
\pgfpathlineto{\pgfqpoint{2.223168in}{0.600564in}}%
\pgfpathlineto{\pgfqpoint{2.224242in}{0.600663in}}%
\pgfpathlineto{\pgfqpoint{2.225316in}{0.601456in}}%
\pgfpathlineto{\pgfqpoint{2.228537in}{0.602448in}}%
\pgfpathlineto{\pgfqpoint{2.229611in}{0.603407in}}%
\pgfpathlineto{\pgfqpoint{2.230685in}{0.602547in}}%
\pgfpathlineto{\pgfqpoint{2.231759in}{0.603837in}}%
\pgfpathlineto{\pgfqpoint{2.232833in}{0.606085in}}%
\pgfpathlineto{\pgfqpoint{2.237128in}{0.607804in}}%
\pgfpathlineto{\pgfqpoint{2.238202in}{0.610119in}}%
\pgfpathlineto{\pgfqpoint{2.239276in}{0.609226in}}%
\pgfpathlineto{\pgfqpoint{2.240349in}{0.603936in}}%
\pgfpathlineto{\pgfqpoint{2.243571in}{0.609226in}}%
\pgfpathlineto{\pgfqpoint{2.244645in}{0.605754in}}%
\pgfpathlineto{\pgfqpoint{2.245719in}{0.604432in}}%
\pgfpathlineto{\pgfqpoint{2.246792in}{0.606118in}}%
\pgfpathlineto{\pgfqpoint{2.247866in}{0.606317in}}%
\pgfpathlineto{\pgfqpoint{2.252162in}{0.607639in}}%
\pgfpathlineto{\pgfqpoint{2.253236in}{0.610053in}}%
\pgfpathlineto{\pgfqpoint{2.254309in}{0.610515in}}%
\pgfpathlineto{\pgfqpoint{2.255383in}{0.607837in}}%
\pgfpathlineto{\pgfqpoint{2.258605in}{0.605490in}}%
\pgfpathlineto{\pgfqpoint{2.259679in}{0.606614in}}%
\pgfpathlineto{\pgfqpoint{2.260752in}{0.609226in}}%
\pgfpathlineto{\pgfqpoint{2.261826in}{0.605854in}}%
\pgfpathlineto{\pgfqpoint{2.262900in}{0.608366in}}%
\pgfpathlineto{\pgfqpoint{2.266122in}{0.608895in}}%
\pgfpathlineto{\pgfqpoint{2.267195in}{0.608466in}}%
\pgfpathlineto{\pgfqpoint{2.268269in}{0.610482in}}%
\pgfpathlineto{\pgfqpoint{2.269343in}{0.610515in}}%
\pgfpathlineto{\pgfqpoint{2.270417in}{0.608962in}}%
\pgfpathlineto{\pgfqpoint{2.273638in}{0.609722in}}%
\pgfpathlineto{\pgfqpoint{2.274712in}{0.606217in}}%
\pgfpathlineto{\pgfqpoint{2.275786in}{0.606912in}}%
\pgfpathlineto{\pgfqpoint{2.276860in}{0.605688in}}%
\pgfpathlineto{\pgfqpoint{2.277934in}{0.607573in}}%
\pgfpathlineto{\pgfqpoint{2.281155in}{0.606581in}}%
\pgfpathlineto{\pgfqpoint{2.282229in}{0.604730in}}%
\pgfpathlineto{\pgfqpoint{2.283303in}{0.608730in}}%
\pgfpathlineto{\pgfqpoint{2.285451in}{0.607242in}}%
\pgfpathlineto{\pgfqpoint{2.288672in}{0.609887in}}%
\pgfpathlineto{\pgfqpoint{2.289746in}{0.606184in}}%
\pgfpathlineto{\pgfqpoint{2.290820in}{0.605292in}}%
\pgfpathlineto{\pgfqpoint{2.296189in}{0.607738in}}%
\pgfpathlineto{\pgfqpoint{2.298337in}{0.601489in}}%
\pgfpathlineto{\pgfqpoint{2.299411in}{0.601721in}}%
\pgfpathlineto{\pgfqpoint{2.300484in}{0.600795in}}%
\pgfpathlineto{\pgfqpoint{2.304780in}{0.608962in}}%
\pgfpathlineto{\pgfqpoint{2.305854in}{0.610119in}}%
\pgfpathlineto{\pgfqpoint{2.306927in}{0.605688in}}%
\pgfpathlineto{\pgfqpoint{2.308001in}{0.605721in}}%
\pgfpathlineto{\pgfqpoint{2.311223in}{0.594381in}}%
\pgfpathlineto{\pgfqpoint{2.312297in}{0.594976in}}%
\pgfpathlineto{\pgfqpoint{2.314444in}{0.603638in}}%
\pgfpathlineto{\pgfqpoint{2.315518in}{0.602845in}}%
\pgfpathlineto{\pgfqpoint{2.318740in}{0.605655in}}%
\pgfpathlineto{\pgfqpoint{2.319814in}{0.599936in}}%
\pgfpathlineto{\pgfqpoint{2.320887in}{0.598778in}}%
\pgfpathlineto{\pgfqpoint{2.323035in}{0.600597in}}%
\pgfpathlineto{\pgfqpoint{2.326257in}{0.597191in}}%
\pgfpathlineto{\pgfqpoint{2.327330in}{0.597423in}}%
\pgfpathlineto{\pgfqpoint{2.329478in}{0.585322in}}%
\pgfpathlineto{\pgfqpoint{2.330552in}{0.586281in}}%
\pgfpathlineto{\pgfqpoint{2.333773in}{0.591439in}}%
\pgfpathlineto{\pgfqpoint{2.334847in}{0.590777in}}%
\pgfpathlineto{\pgfqpoint{2.335921in}{0.597919in}}%
\pgfpathlineto{\pgfqpoint{2.338069in}{0.597390in}}%
\pgfpathlineto{\pgfqpoint{2.341290in}{0.595307in}}%
\pgfpathlineto{\pgfqpoint{2.342364in}{0.597555in}}%
\pgfpathlineto{\pgfqpoint{2.343438in}{0.597357in}}%
\pgfpathlineto{\pgfqpoint{2.344512in}{0.598547in}}%
\pgfpathlineto{\pgfqpoint{2.345586in}{0.594811in}}%
\pgfpathlineto{\pgfqpoint{2.348807in}{0.593984in}}%
\pgfpathlineto{\pgfqpoint{2.349881in}{0.594844in}}%
\pgfpathlineto{\pgfqpoint{2.352029in}{0.593356in}}%
\pgfpathlineto{\pgfqpoint{2.353102in}{0.594150in}}%
\pgfpathlineto{\pgfqpoint{2.359546in}{0.594745in}}%
\pgfpathlineto{\pgfqpoint{2.360619in}{0.593753in}}%
\pgfpathlineto{\pgfqpoint{2.364915in}{0.598547in}}%
\pgfpathlineto{\pgfqpoint{2.367062in}{0.603705in}}%
\pgfpathlineto{\pgfqpoint{2.368136in}{0.607341in}}%
\pgfpathlineto{\pgfqpoint{2.371358in}{0.605854in}}%
\pgfpathlineto{\pgfqpoint{2.372432in}{0.604432in}}%
\pgfpathlineto{\pgfqpoint{2.373505in}{0.606019in}}%
\pgfpathlineto{\pgfqpoint{2.375653in}{0.603969in}}%
\pgfpathlineto{\pgfqpoint{2.379948in}{0.604267in}}%
\pgfpathlineto{\pgfqpoint{2.382096in}{0.605788in}}%
\pgfpathlineto{\pgfqpoint{2.386391in}{0.607143in}}%
\pgfpathlineto{\pgfqpoint{2.388539in}{0.612830in}}%
\pgfpathlineto{\pgfqpoint{2.389613in}{0.610681in}}%
\pgfpathlineto{\pgfqpoint{2.390687in}{0.612400in}}%
\pgfpathlineto{\pgfqpoint{2.393908in}{0.612169in}}%
\pgfpathlineto{\pgfqpoint{2.394982in}{0.608962in}}%
\pgfpathlineto{\pgfqpoint{2.397130in}{0.607639in}}%
\pgfpathlineto{\pgfqpoint{2.398204in}{0.619905in}}%
\pgfpathlineto{\pgfqpoint{2.402499in}{0.618913in}}%
\pgfpathlineto{\pgfqpoint{2.403573in}{0.616698in}}%
\pgfpathlineto{\pgfqpoint{2.405721in}{0.619310in}}%
\pgfpathlineto{\pgfqpoint{2.408942in}{0.621062in}}%
\pgfpathlineto{\pgfqpoint{2.410016in}{0.622418in}}%
\pgfpathlineto{\pgfqpoint{2.411090in}{0.625195in}}%
\pgfpathlineto{\pgfqpoint{2.413237in}{0.624931in}}%
\pgfpathlineto{\pgfqpoint{2.417533in}{0.626418in}}%
\pgfpathlineto{\pgfqpoint{2.418607in}{0.626022in}}%
\pgfpathlineto{\pgfqpoint{2.420754in}{0.628072in}}%
\pgfpathlineto{\pgfqpoint{2.425050in}{0.626485in}}%
\pgfpathlineto{\pgfqpoint{2.426124in}{0.629890in}}%
\pgfpathlineto{\pgfqpoint{2.427197in}{0.628402in}}%
\pgfpathlineto{\pgfqpoint{2.428271in}{0.629262in}}%
\pgfpathlineto{\pgfqpoint{2.434714in}{0.630287in}}%
\pgfpathlineto{\pgfqpoint{2.435788in}{0.632303in}}%
\pgfpathlineto{\pgfqpoint{2.439010in}{0.633527in}}%
\pgfpathlineto{\pgfqpoint{2.440083in}{0.631708in}}%
\pgfpathlineto{\pgfqpoint{2.442231in}{0.633328in}}%
\pgfpathlineto{\pgfqpoint{2.443305in}{0.633957in}}%
\pgfpathlineto{\pgfqpoint{2.446526in}{0.630518in}}%
\pgfpathlineto{\pgfqpoint{2.447600in}{0.626881in}}%
\pgfpathlineto{\pgfqpoint{2.450822in}{0.630551in}}%
\pgfpathlineto{\pgfqpoint{2.454043in}{0.629758in}}%
\pgfpathlineto{\pgfqpoint{2.456191in}{0.630518in}}%
\pgfpathlineto{\pgfqpoint{2.458339in}{0.629493in}}%
\pgfpathlineto{\pgfqpoint{2.461560in}{0.631113in}}%
\pgfpathlineto{\pgfqpoint{2.462634in}{0.629229in}}%
\pgfpathlineto{\pgfqpoint{2.464782in}{0.630221in}}%
\pgfpathlineto{\pgfqpoint{2.465856in}{0.629096in}}%
\pgfpathlineto{\pgfqpoint{2.470151in}{0.629427in}}%
\pgfpathlineto{\pgfqpoint{2.471225in}{0.628898in}}%
\pgfpathlineto{\pgfqpoint{2.472299in}{0.629493in}}%
\pgfpathlineto{\pgfqpoint{2.476594in}{0.632568in}}%
\pgfpathlineto{\pgfqpoint{2.478742in}{0.632370in}}%
\pgfpathlineto{\pgfqpoint{2.479815in}{0.636932in}}%
\pgfpathlineto{\pgfqpoint{2.480889in}{0.636932in}}%
\pgfpathlineto{\pgfqpoint{2.485185in}{0.640040in}}%
\pgfpathlineto{\pgfqpoint{2.488406in}{0.637164in}}%
\pgfpathlineto{\pgfqpoint{2.491628in}{0.637296in}}%
\pgfpathlineto{\pgfqpoint{2.492702in}{0.641429in}}%
\pgfpathlineto{\pgfqpoint{2.493775in}{0.641131in}}%
\pgfpathlineto{\pgfqpoint{2.494849in}{0.641693in}}%
\pgfpathlineto{\pgfqpoint{2.495923in}{0.639974in}}%
\pgfpathlineto{\pgfqpoint{2.500218in}{0.639676in}}%
\pgfpathlineto{\pgfqpoint{2.501292in}{0.640503in}}%
\pgfpathlineto{\pgfqpoint{2.502366in}{0.640007in}}%
\pgfpathlineto{\pgfqpoint{2.503440in}{0.642057in}}%
\pgfpathlineto{\pgfqpoint{2.506661in}{0.643677in}}%
\pgfpathlineto{\pgfqpoint{2.507735in}{0.643412in}}%
\pgfpathlineto{\pgfqpoint{2.508809in}{0.639445in}}%
\pgfpathlineto{\pgfqpoint{2.509883in}{0.639280in}}%
\pgfpathlineto{\pgfqpoint{2.510957in}{0.641792in}}%
\pgfpathlineto{\pgfqpoint{2.514178in}{0.644470in}}%
\pgfpathlineto{\pgfqpoint{2.515252in}{0.646289in}}%
\pgfpathlineto{\pgfqpoint{2.516326in}{0.649364in}}%
\pgfpathlineto{\pgfqpoint{2.517400in}{0.650124in}}%
\pgfpathlineto{\pgfqpoint{2.518474in}{0.648901in}}%
\pgfpathlineto{\pgfqpoint{2.522769in}{0.649066in}}%
\pgfpathlineto{\pgfqpoint{2.523843in}{0.650752in}}%
\pgfpathlineto{\pgfqpoint{2.524917in}{0.651281in}}%
\pgfpathlineto{\pgfqpoint{2.525991in}{0.653728in}}%
\pgfpathlineto{\pgfqpoint{2.529212in}{0.655050in}}%
\pgfpathlineto{\pgfqpoint{2.530286in}{0.652571in}}%
\pgfpathlineto{\pgfqpoint{2.531360in}{0.653529in}}%
\pgfpathlineto{\pgfqpoint{2.532434in}{0.653529in}}%
\pgfpathlineto{\pgfqpoint{2.533507in}{0.648570in}}%
\pgfpathlineto{\pgfqpoint{2.536729in}{0.645099in}}%
\pgfpathlineto{\pgfqpoint{2.537803in}{0.650256in}}%
\pgfpathlineto{\pgfqpoint{2.538877in}{0.651050in}}%
\pgfpathlineto{\pgfqpoint{2.539950in}{0.647281in}}%
\pgfpathlineto{\pgfqpoint{2.541024in}{0.647281in}}%
\pgfpathlineto{\pgfqpoint{2.544246in}{0.649297in}}%
\pgfpathlineto{\pgfqpoint{2.545320in}{0.647975in}}%
\pgfpathlineto{\pgfqpoint{2.546393in}{0.648504in}}%
\pgfpathlineto{\pgfqpoint{2.547467in}{0.646586in}}%
\pgfpathlineto{\pgfqpoint{2.548541in}{0.651876in}}%
\pgfpathlineto{\pgfqpoint{2.551763in}{0.650719in}}%
\pgfpathlineto{\pgfqpoint{2.552836in}{0.649661in}}%
\pgfpathlineto{\pgfqpoint{2.553910in}{0.653992in}}%
\pgfpathlineto{\pgfqpoint{2.554984in}{0.648107in}}%
\pgfpathlineto{\pgfqpoint{2.556058in}{0.646024in}}%
\pgfpathlineto{\pgfqpoint{2.559279in}{0.644603in}}%
\pgfpathlineto{\pgfqpoint{2.561427in}{0.646752in}}%
\pgfpathlineto{\pgfqpoint{2.562501in}{0.644305in}}%
\pgfpathlineto{\pgfqpoint{2.563575in}{0.646487in}}%
\pgfpathlineto{\pgfqpoint{2.567870in}{0.651314in}}%
\pgfpathlineto{\pgfqpoint{2.568944in}{0.653827in}}%
\pgfpathlineto{\pgfqpoint{2.570018in}{0.653067in}}%
\pgfpathlineto{\pgfqpoint{2.571092in}{0.656274in}}%
\pgfpathlineto{\pgfqpoint{2.574313in}{0.655943in}}%
\pgfpathlineto{\pgfqpoint{2.576461in}{0.660572in}}%
\pgfpathlineto{\pgfqpoint{2.577535in}{0.660109in}}%
\pgfpathlineto{\pgfqpoint{2.578609in}{0.664969in}}%
\pgfpathlineto{\pgfqpoint{2.581830in}{0.667482in}}%
\pgfpathlineto{\pgfqpoint{2.582904in}{0.666258in}}%
\pgfpathlineto{\pgfqpoint{2.583978in}{0.669036in}}%
\pgfpathlineto{\pgfqpoint{2.585052in}{0.664870in}}%
\pgfpathlineto{\pgfqpoint{2.586125in}{0.663514in}}%
\pgfpathlineto{\pgfqpoint{2.589347in}{0.664837in}}%
\pgfpathlineto{\pgfqpoint{2.590421in}{0.669102in}}%
\pgfpathlineto{\pgfqpoint{2.591495in}{0.670457in}}%
\pgfpathlineto{\pgfqpoint{2.592568in}{0.668209in}}%
\pgfpathlineto{\pgfqpoint{2.593642in}{0.669168in}}%
\pgfpathlineto{\pgfqpoint{2.596864in}{0.671119in}}%
\pgfpathlineto{\pgfqpoint{2.599012in}{0.669432in}}%
\pgfpathlineto{\pgfqpoint{2.600085in}{0.664605in}}%
\pgfpathlineto{\pgfqpoint{2.606528in}{0.677632in}}%
\pgfpathlineto{\pgfqpoint{2.607602in}{0.673863in}}%
\pgfpathlineto{\pgfqpoint{2.608676in}{0.674987in}}%
\pgfpathlineto{\pgfqpoint{2.611898in}{0.677301in}}%
\pgfpathlineto{\pgfqpoint{2.612971in}{0.680078in}}%
\pgfpathlineto{\pgfqpoint{2.614045in}{0.677169in}}%
\pgfpathlineto{\pgfqpoint{2.616193in}{0.677731in}}%
\pgfpathlineto{\pgfqpoint{2.620488in}{0.679120in}}%
\pgfpathlineto{\pgfqpoint{2.621562in}{0.678987in}}%
\pgfpathlineto{\pgfqpoint{2.622636in}{0.678293in}}%
\pgfpathlineto{\pgfqpoint{2.623710in}{0.679516in}}%
\pgfpathlineto{\pgfqpoint{2.628005in}{0.676640in}}%
\pgfpathlineto{\pgfqpoint{2.629079in}{0.677169in}}%
\pgfpathlineto{\pgfqpoint{2.630153in}{0.681699in}}%
\pgfpathlineto{\pgfqpoint{2.631227in}{0.681269in}}%
\pgfpathlineto{\pgfqpoint{2.634448in}{0.686856in}}%
\pgfpathlineto{\pgfqpoint{2.635522in}{0.687055in}}%
\pgfpathlineto{\pgfqpoint{2.636596in}{0.685831in}}%
\pgfpathlineto{\pgfqpoint{2.637670in}{0.686658in}}%
\pgfpathlineto{\pgfqpoint{2.638744in}{0.684112in}}%
\pgfpathlineto{\pgfqpoint{2.641965in}{0.682558in}}%
\pgfpathlineto{\pgfqpoint{2.643039in}{0.684443in}}%
\pgfpathlineto{\pgfqpoint{2.644113in}{0.683021in}}%
\pgfpathlineto{\pgfqpoint{2.646260in}{0.684939in}}%
\pgfpathlineto{\pgfqpoint{2.649482in}{0.676673in}}%
\pgfpathlineto{\pgfqpoint{2.650556in}{0.676409in}}%
\pgfpathlineto{\pgfqpoint{2.652703in}{0.681765in}}%
\pgfpathlineto{\pgfqpoint{2.653777in}{0.683914in}}%
\pgfpathlineto{\pgfqpoint{2.658073in}{0.685038in}}%
\pgfpathlineto{\pgfqpoint{2.659146in}{0.684145in}}%
\pgfpathlineto{\pgfqpoint{2.661294in}{0.688774in}}%
\pgfpathlineto{\pgfqpoint{2.664516in}{0.689667in}}%
\pgfpathlineto{\pgfqpoint{2.665590in}{0.690658in}}%
\pgfpathlineto{\pgfqpoint{2.666663in}{0.694361in}}%
\pgfpathlineto{\pgfqpoint{2.667737in}{0.693270in}}%
\pgfpathlineto{\pgfqpoint{2.668811in}{0.695056in}}%
\pgfpathlineto{\pgfqpoint{2.672033in}{0.693898in}}%
\pgfpathlineto{\pgfqpoint{2.673106in}{0.691419in}}%
\pgfpathlineto{\pgfqpoint{2.674180in}{0.692245in}}%
\pgfpathlineto{\pgfqpoint{2.675254in}{0.689700in}}%
\pgfpathlineto{\pgfqpoint{2.676328in}{0.691452in}}%
\pgfpathlineto{\pgfqpoint{2.679549in}{0.691386in}}%
\pgfpathlineto{\pgfqpoint{2.682771in}{0.698792in}}%
\pgfpathlineto{\pgfqpoint{2.683845in}{0.698494in}}%
\pgfpathlineto{\pgfqpoint{2.688140in}{0.699354in}}%
\pgfpathlineto{\pgfqpoint{2.689214in}{0.702924in}}%
\pgfpathlineto{\pgfqpoint{2.690288in}{0.702627in}}%
\pgfpathlineto{\pgfqpoint{2.691362in}{0.703883in}}%
\pgfpathlineto{\pgfqpoint{2.696731in}{0.707487in}}%
\pgfpathlineto{\pgfqpoint{2.698879in}{0.705569in}}%
\pgfpathlineto{\pgfqpoint{2.702100in}{0.704016in}}%
\pgfpathlineto{\pgfqpoint{2.703174in}{0.705900in}}%
\pgfpathlineto{\pgfqpoint{2.704248in}{0.700246in}}%
\pgfpathlineto{\pgfqpoint{2.705322in}{0.703420in}}%
\pgfpathlineto{\pgfqpoint{2.706395in}{0.699486in}}%
\pgfpathlineto{\pgfqpoint{2.709617in}{0.699916in}}%
\pgfpathlineto{\pgfqpoint{2.710691in}{0.704809in}}%
\pgfpathlineto{\pgfqpoint{2.711765in}{0.702759in}}%
\pgfpathlineto{\pgfqpoint{2.713912in}{0.706495in}}%
\pgfpathlineto{\pgfqpoint{2.717134in}{0.707851in}}%
\pgfpathlineto{\pgfqpoint{2.718208in}{0.711058in}}%
\pgfpathlineto{\pgfqpoint{2.719281in}{0.699618in}}%
\pgfpathlineto{\pgfqpoint{2.720355in}{0.708545in}}%
\pgfpathlineto{\pgfqpoint{2.721429in}{0.702462in}}%
\pgfpathlineto{\pgfqpoint{2.724651in}{0.691683in}}%
\pgfpathlineto{\pgfqpoint{2.728946in}{0.708578in}}%
\pgfpathlineto{\pgfqpoint{2.732167in}{0.707818in}}%
\pgfpathlineto{\pgfqpoint{2.733241in}{0.711388in}}%
\pgfpathlineto{\pgfqpoint{2.734315in}{0.710892in}}%
\pgfpathlineto{\pgfqpoint{2.735389in}{0.709603in}}%
\pgfpathlineto{\pgfqpoint{2.736463in}{0.712546in}}%
\pgfpathlineto{\pgfqpoint{2.739684in}{0.711686in}}%
\pgfpathlineto{\pgfqpoint{2.740758in}{0.707057in}}%
\pgfpathlineto{\pgfqpoint{2.741832in}{0.706991in}}%
\pgfpathlineto{\pgfqpoint{2.743980in}{0.708876in}}%
\pgfpathlineto{\pgfqpoint{2.748275in}{0.709835in}}%
\pgfpathlineto{\pgfqpoint{2.749349in}{0.713207in}}%
\pgfpathlineto{\pgfqpoint{2.750423in}{0.714298in}}%
\pgfpathlineto{\pgfqpoint{2.751497in}{0.713207in}}%
\pgfpathlineto{\pgfqpoint{2.751497in}{0.713207in}}%
\pgfusepath{stroke}%
\end{pgfscope}%
\begin{pgfscope}%
\pgfpathrectangle{\pgfqpoint{0.285588in}{0.331635in}}{\pgfqpoint{2.583333in}{0.400885in}}%
\pgfusepath{clip}%
\pgfsetroundcap%
\pgfsetroundjoin%
\pgfsetlinewidth{1.505625pt}%
\definecolor{currentstroke}{rgb}{0.737255,0.741176,0.133333}%
\pgfsetstrokecolor{currentstroke}%
\pgfsetdash{}{0pt}%
\pgfpathmoveto{\pgfqpoint{0.403012in}{0.414985in}}%
\pgfpathlineto{\pgfqpoint{0.404086in}{0.413663in}}%
\pgfpathlineto{\pgfqpoint{0.405159in}{0.414225in}}%
\pgfpathlineto{\pgfqpoint{0.406233in}{0.413365in}}%
\pgfpathlineto{\pgfqpoint{0.411602in}{0.412208in}}%
\pgfpathlineto{\pgfqpoint{0.412676in}{0.410621in}}%
\pgfpathlineto{\pgfqpoint{0.419119in}{0.408121in}}%
\pgfpathlineto{\pgfqpoint{0.420193in}{0.409044in}}%
\pgfpathlineto{\pgfqpoint{0.421267in}{0.407851in}}%
\pgfpathlineto{\pgfqpoint{0.424489in}{0.407178in}}%
\pgfpathlineto{\pgfqpoint{0.425562in}{0.408126in}}%
\pgfpathlineto{\pgfqpoint{0.427710in}{0.408035in}}%
\pgfpathlineto{\pgfqpoint{0.428784in}{0.408157in}}%
\pgfpathlineto{\pgfqpoint{0.432005in}{0.407361in}}%
\pgfpathlineto{\pgfqpoint{0.433079in}{0.407882in}}%
\pgfpathlineto{\pgfqpoint{0.434153in}{0.406627in}}%
\pgfpathlineto{\pgfqpoint{0.435227in}{0.404327in}}%
\pgfpathlineto{\pgfqpoint{0.436301in}{0.403750in}}%
\pgfpathlineto{\pgfqpoint{0.441670in}{0.402964in}}%
\pgfpathlineto{\pgfqpoint{0.442744in}{0.400620in}}%
\pgfpathlineto{\pgfqpoint{0.443818in}{0.399837in}}%
\pgfpathlineto{\pgfqpoint{0.447039in}{0.400457in}}%
\pgfpathlineto{\pgfqpoint{0.448113in}{0.399119in}}%
\pgfpathlineto{\pgfqpoint{0.449187in}{0.398724in}}%
\pgfpathlineto{\pgfqpoint{0.450261in}{0.399389in}}%
\pgfpathlineto{\pgfqpoint{0.451334in}{0.398756in}}%
\pgfpathlineto{\pgfqpoint{0.455630in}{0.399169in}}%
\pgfpathlineto{\pgfqpoint{0.456704in}{0.398051in}}%
\pgfpathlineto{\pgfqpoint{0.462073in}{0.397798in}}%
\pgfpathlineto{\pgfqpoint{0.463147in}{0.396775in}}%
\pgfpathlineto{\pgfqpoint{0.464221in}{0.397998in}}%
\pgfpathlineto{\pgfqpoint{0.465294in}{0.397595in}}%
\pgfpathlineto{\pgfqpoint{0.466368in}{0.398103in}}%
\pgfpathlineto{\pgfqpoint{0.469590in}{0.398058in}}%
\pgfpathlineto{\pgfqpoint{0.470664in}{0.398754in}}%
\pgfpathlineto{\pgfqpoint{0.471737in}{0.398340in}}%
\pgfpathlineto{\pgfqpoint{0.472811in}{0.397183in}}%
\pgfpathlineto{\pgfqpoint{0.473885in}{0.397575in}}%
\pgfpathlineto{\pgfqpoint{0.481402in}{0.397791in}}%
\pgfpathlineto{\pgfqpoint{0.484623in}{0.396745in}}%
\pgfpathlineto{\pgfqpoint{0.485697in}{0.397839in}}%
\pgfpathlineto{\pgfqpoint{0.487845in}{0.397461in}}%
\pgfpathlineto{\pgfqpoint{0.488919in}{0.396734in}}%
\pgfpathlineto{\pgfqpoint{0.493214in}{0.396238in}}%
\pgfpathlineto{\pgfqpoint{0.496436in}{0.397089in}}%
\pgfpathlineto{\pgfqpoint{0.502879in}{0.395654in}}%
\pgfpathlineto{\pgfqpoint{0.507174in}{0.396337in}}%
\pgfpathlineto{\pgfqpoint{0.508248in}{0.397629in}}%
\pgfpathlineto{\pgfqpoint{0.509322in}{0.397319in}}%
\pgfpathlineto{\pgfqpoint{0.511469in}{0.394592in}}%
\pgfpathlineto{\pgfqpoint{0.514691in}{0.395608in}}%
\pgfpathlineto{\pgfqpoint{0.515765in}{0.395029in}}%
\pgfpathlineto{\pgfqpoint{0.518986in}{0.395518in}}%
\pgfpathlineto{\pgfqpoint{0.522208in}{0.396838in}}%
\pgfpathlineto{\pgfqpoint{0.523282in}{0.396448in}}%
\pgfpathlineto{\pgfqpoint{0.525429in}{0.394473in}}%
\pgfpathlineto{\pgfqpoint{0.526503in}{0.394334in}}%
\pgfpathlineto{\pgfqpoint{0.531872in}{0.394910in}}%
\pgfpathlineto{\pgfqpoint{0.532946in}{0.397475in}}%
\pgfpathlineto{\pgfqpoint{0.534020in}{0.396809in}}%
\pgfpathlineto{\pgfqpoint{0.538315in}{0.396633in}}%
\pgfpathlineto{\pgfqpoint{0.539389in}{0.396978in}}%
\pgfpathlineto{\pgfqpoint{0.541537in}{0.396737in}}%
\pgfpathlineto{\pgfqpoint{0.545832in}{0.397365in}}%
\pgfpathlineto{\pgfqpoint{0.546906in}{0.396480in}}%
\pgfpathlineto{\pgfqpoint{0.547980in}{0.397681in}}%
\pgfpathlineto{\pgfqpoint{0.549054in}{0.396495in}}%
\pgfpathlineto{\pgfqpoint{0.555497in}{0.393073in}}%
\pgfpathlineto{\pgfqpoint{0.556571in}{0.393251in}}%
\pgfpathlineto{\pgfqpoint{0.560866in}{0.392853in}}%
\pgfpathlineto{\pgfqpoint{0.563014in}{0.395115in}}%
\pgfpathlineto{\pgfqpoint{0.564088in}{0.393727in}}%
\pgfpathlineto{\pgfqpoint{0.571604in}{0.391712in}}%
\pgfpathlineto{\pgfqpoint{0.578047in}{0.391721in}}%
\pgfpathlineto{\pgfqpoint{0.579121in}{0.390889in}}%
\pgfpathlineto{\pgfqpoint{0.584490in}{0.389148in}}%
\pgfpathlineto{\pgfqpoint{0.585564in}{0.390381in}}%
\pgfpathlineto{\pgfqpoint{0.586638in}{0.388087in}}%
\pgfpathlineto{\pgfqpoint{0.589860in}{0.389493in}}%
\pgfpathlineto{\pgfqpoint{0.592007in}{0.388624in}}%
\pgfpathlineto{\pgfqpoint{0.593081in}{0.389361in}}%
\pgfpathlineto{\pgfqpoint{0.594155in}{0.388510in}}%
\pgfpathlineto{\pgfqpoint{0.598450in}{0.387457in}}%
\pgfpathlineto{\pgfqpoint{0.600598in}{0.387206in}}%
\pgfpathlineto{\pgfqpoint{0.601672in}{0.387851in}}%
\pgfpathlineto{\pgfqpoint{0.605967in}{0.388988in}}%
\pgfpathlineto{\pgfqpoint{0.607041in}{0.389612in}}%
\pgfpathlineto{\pgfqpoint{0.609189in}{0.388257in}}%
\pgfpathlineto{\pgfqpoint{0.614558in}{0.386829in}}%
\pgfpathlineto{\pgfqpoint{0.615632in}{0.387710in}}%
\pgfpathlineto{\pgfqpoint{0.616706in}{0.387575in}}%
\pgfpathlineto{\pgfqpoint{0.622075in}{0.388929in}}%
\pgfpathlineto{\pgfqpoint{0.624222in}{0.386249in}}%
\pgfpathlineto{\pgfqpoint{0.627444in}{0.385643in}}%
\pgfpathlineto{\pgfqpoint{0.629592in}{0.386823in}}%
\pgfpathlineto{\pgfqpoint{0.631739in}{0.385565in}}%
\pgfpathlineto{\pgfqpoint{0.637109in}{0.385500in}}%
\pgfpathlineto{\pgfqpoint{0.638182in}{0.386411in}}%
\pgfpathlineto{\pgfqpoint{0.639256in}{0.386202in}}%
\pgfpathlineto{\pgfqpoint{0.642478in}{0.386473in}}%
\pgfpathlineto{\pgfqpoint{0.644625in}{0.385893in}}%
\pgfpathlineto{\pgfqpoint{0.645699in}{0.385673in}}%
\pgfpathlineto{\pgfqpoint{0.646773in}{0.386031in}}%
\pgfpathlineto{\pgfqpoint{0.660733in}{0.386723in}}%
\pgfpathlineto{\pgfqpoint{0.661807in}{0.386152in}}%
\pgfpathlineto{\pgfqpoint{0.672545in}{0.385994in}}%
\pgfpathlineto{\pgfqpoint{0.674693in}{0.384332in}}%
\pgfpathlineto{\pgfqpoint{0.675767in}{0.383833in}}%
\pgfpathlineto{\pgfqpoint{0.680062in}{0.384193in}}%
\pgfpathlineto{\pgfqpoint{0.683284in}{0.383943in}}%
\pgfpathlineto{\pgfqpoint{0.684357in}{0.383813in}}%
\pgfpathlineto{\pgfqpoint{0.690800in}{0.384154in}}%
\pgfpathlineto{\pgfqpoint{0.696170in}{0.383464in}}%
\pgfpathlineto{\pgfqpoint{0.698317in}{0.382319in}}%
\pgfpathlineto{\pgfqpoint{0.699391in}{0.382145in}}%
\pgfpathlineto{\pgfqpoint{0.704760in}{0.382984in}}%
\pgfpathlineto{\pgfqpoint{0.706908in}{0.382459in}}%
\pgfpathlineto{\pgfqpoint{0.711203in}{0.381857in}}%
\pgfpathlineto{\pgfqpoint{0.712277in}{0.381437in}}%
\pgfpathlineto{\pgfqpoint{0.717646in}{0.382480in}}%
\pgfpathlineto{\pgfqpoint{0.718720in}{0.383196in}}%
\pgfpathlineto{\pgfqpoint{0.720868in}{0.382902in}}%
\pgfpathlineto{\pgfqpoint{0.721942in}{0.382668in}}%
\pgfpathlineto{\pgfqpoint{0.727311in}{0.382531in}}%
\pgfpathlineto{\pgfqpoint{0.728385in}{0.380987in}}%
\pgfpathlineto{\pgfqpoint{0.732680in}{0.381658in}}%
\pgfpathlineto{\pgfqpoint{0.733754in}{0.380969in}}%
\pgfpathlineto{\pgfqpoint{0.735902in}{0.381474in}}%
\pgfpathlineto{\pgfqpoint{0.736976in}{0.381229in}}%
\pgfpathlineto{\pgfqpoint{0.741271in}{0.381368in}}%
\pgfpathlineto{\pgfqpoint{0.743419in}{0.381689in}}%
\pgfpathlineto{\pgfqpoint{0.744492in}{0.380894in}}%
\pgfpathlineto{\pgfqpoint{0.752009in}{0.379399in}}%
\pgfpathlineto{\pgfqpoint{0.758452in}{0.379384in}}%
\pgfpathlineto{\pgfqpoint{0.759526in}{0.378977in}}%
\pgfpathlineto{\pgfqpoint{0.767043in}{0.379270in}}%
\pgfpathlineto{\pgfqpoint{0.779929in}{0.379088in}}%
\pgfpathlineto{\pgfqpoint{0.781003in}{0.378240in}}%
\pgfpathlineto{\pgfqpoint{0.782077in}{0.378657in}}%
\pgfpathlineto{\pgfqpoint{0.788520in}{0.379004in}}%
\pgfpathlineto{\pgfqpoint{0.789594in}{0.379198in}}%
\pgfpathlineto{\pgfqpoint{0.804627in}{0.376143in}}%
\pgfpathlineto{\pgfqpoint{0.823956in}{0.377137in}}%
\pgfpathlineto{\pgfqpoint{0.825030in}{0.377521in}}%
\pgfpathlineto{\pgfqpoint{0.827178in}{0.376680in}}%
\pgfpathlineto{\pgfqpoint{0.830399in}{0.377115in}}%
\pgfpathlineto{\pgfqpoint{0.832547in}{0.376153in}}%
\pgfpathlineto{\pgfqpoint{0.833621in}{0.376987in}}%
\pgfpathlineto{\pgfqpoint{0.834695in}{0.376812in}}%
\pgfpathlineto{\pgfqpoint{0.840064in}{0.377210in}}%
\pgfpathlineto{\pgfqpoint{0.842212in}{0.376457in}}%
\pgfpathlineto{\pgfqpoint{0.854024in}{0.376444in}}%
\pgfpathlineto{\pgfqpoint{0.855098in}{0.375984in}}%
\pgfpathlineto{\pgfqpoint{0.857245in}{0.376386in}}%
\pgfpathlineto{\pgfqpoint{0.867984in}{0.375686in}}%
\pgfpathlineto{\pgfqpoint{0.870131in}{0.376080in}}%
\pgfpathlineto{\pgfqpoint{0.871205in}{0.375796in}}%
\pgfpathlineto{\pgfqpoint{0.872279in}{0.376259in}}%
\pgfpathlineto{\pgfqpoint{0.878722in}{0.376402in}}%
\pgfpathlineto{\pgfqpoint{0.879796in}{0.375883in}}%
\pgfpathlineto{\pgfqpoint{0.886239in}{0.373813in}}%
\pgfpathlineto{\pgfqpoint{0.894830in}{0.374735in}}%
\pgfpathlineto{\pgfqpoint{0.902347in}{0.374592in}}%
\pgfpathlineto{\pgfqpoint{0.905568in}{0.375514in}}%
\pgfpathlineto{\pgfqpoint{0.906642in}{0.374770in}}%
\pgfpathlineto{\pgfqpoint{0.908790in}{0.375461in}}%
\pgfpathlineto{\pgfqpoint{0.909864in}{0.374909in}}%
\pgfpathlineto{\pgfqpoint{0.914159in}{0.374547in}}%
\pgfpathlineto{\pgfqpoint{0.917380in}{0.374217in}}%
\pgfpathlineto{\pgfqpoint{0.922750in}{0.374458in}}%
\pgfpathlineto{\pgfqpoint{0.923823in}{0.372536in}}%
\pgfpathlineto{\pgfqpoint{0.924897in}{0.371782in}}%
\pgfpathlineto{\pgfqpoint{0.935636in}{0.371887in}}%
\pgfpathlineto{\pgfqpoint{0.939931in}{0.370674in}}%
\pgfpathlineto{\pgfqpoint{0.947448in}{0.371340in}}%
\pgfpathlineto{\pgfqpoint{0.962482in}{0.371393in}}%
\pgfpathlineto{\pgfqpoint{0.965703in}{0.370982in}}%
\pgfpathlineto{\pgfqpoint{0.967851in}{0.371516in}}%
\pgfpathlineto{\pgfqpoint{0.968925in}{0.370980in}}%
\pgfpathlineto{\pgfqpoint{0.969998in}{0.371199in}}%
\pgfpathlineto{\pgfqpoint{0.975368in}{0.370907in}}%
\pgfpathlineto{\pgfqpoint{0.976442in}{0.371604in}}%
\pgfpathlineto{\pgfqpoint{0.977515in}{0.371422in}}%
\pgfpathlineto{\pgfqpoint{0.981811in}{0.371251in}}%
\pgfpathlineto{\pgfqpoint{0.983958in}{0.370611in}}%
\pgfpathlineto{\pgfqpoint{0.985032in}{0.370854in}}%
\pgfpathlineto{\pgfqpoint{0.995771in}{0.369969in}}%
\pgfpathlineto{\pgfqpoint{0.997918in}{0.370189in}}%
\pgfpathlineto{\pgfqpoint{1.000066in}{0.369555in}}%
\pgfpathlineto{\pgfqpoint{1.012952in}{0.370159in}}%
\pgfpathlineto{\pgfqpoint{1.014026in}{0.368905in}}%
\pgfpathlineto{\pgfqpoint{1.018321in}{0.369296in}}%
\pgfpathlineto{\pgfqpoint{1.019395in}{0.369375in}}%
\pgfpathlineto{\pgfqpoint{1.020469in}{0.371552in}}%
\pgfpathlineto{\pgfqpoint{1.025838in}{0.370256in}}%
\pgfpathlineto{\pgfqpoint{1.030133in}{0.371037in}}%
\pgfpathlineto{\pgfqpoint{1.035503in}{0.371059in}}%
\pgfpathlineto{\pgfqpoint{1.037650in}{0.371963in}}%
\pgfpathlineto{\pgfqpoint{1.045167in}{0.370880in}}%
\pgfpathlineto{\pgfqpoint{1.050536in}{0.371537in}}%
\pgfpathlineto{\pgfqpoint{1.063422in}{0.370938in}}%
\pgfpathlineto{\pgfqpoint{1.065570in}{0.369637in}}%
\pgfpathlineto{\pgfqpoint{1.066644in}{0.369833in}}%
\pgfpathlineto{\pgfqpoint{1.067718in}{0.369209in}}%
\pgfpathlineto{\pgfqpoint{1.072013in}{0.368816in}}%
\pgfpathlineto{\pgfqpoint{1.075235in}{0.367770in}}%
\pgfpathlineto{\pgfqpoint{1.095638in}{0.369928in}}%
\pgfpathlineto{\pgfqpoint{1.097785in}{0.368618in}}%
\pgfpathlineto{\pgfqpoint{1.104228in}{0.367730in}}%
\pgfpathlineto{\pgfqpoint{1.105302in}{0.367433in}}%
\pgfpathlineto{\pgfqpoint{1.116041in}{0.367077in}}%
\pgfpathlineto{\pgfqpoint{1.118188in}{0.366974in}}%
\pgfpathlineto{\pgfqpoint{1.119262in}{0.367907in}}%
\pgfpathlineto{\pgfqpoint{1.120336in}{0.367557in}}%
\pgfpathlineto{\pgfqpoint{1.124631in}{0.367795in}}%
\pgfpathlineto{\pgfqpoint{1.125705in}{0.367568in}}%
\pgfpathlineto{\pgfqpoint{1.126779in}{0.367968in}}%
\pgfpathlineto{\pgfqpoint{1.127853in}{0.367659in}}%
\pgfpathlineto{\pgfqpoint{1.157920in}{0.366927in}}%
\pgfpathlineto{\pgfqpoint{1.162216in}{0.367243in}}%
\pgfpathlineto{\pgfqpoint{1.163289in}{0.366408in}}%
\pgfpathlineto{\pgfqpoint{1.164363in}{0.366689in}}%
\pgfpathlineto{\pgfqpoint{1.165437in}{0.366186in}}%
\pgfpathlineto{\pgfqpoint{1.168659in}{0.366138in}}%
\pgfpathlineto{\pgfqpoint{1.170806in}{0.365214in}}%
\pgfpathlineto{\pgfqpoint{1.184766in}{0.364393in}}%
\pgfpathlineto{\pgfqpoint{1.192283in}{0.364514in}}%
\pgfpathlineto{\pgfqpoint{1.195505in}{0.364555in}}%
\pgfpathlineto{\pgfqpoint{1.199800in}{0.364379in}}%
\pgfpathlineto{\pgfqpoint{1.201948in}{0.364474in}}%
\pgfpathlineto{\pgfqpoint{1.203021in}{0.363346in}}%
\pgfpathlineto{\pgfqpoint{1.209464in}{0.363731in}}%
\pgfpathlineto{\pgfqpoint{1.210538in}{0.364443in}}%
\pgfpathlineto{\pgfqpoint{1.213760in}{0.364987in}}%
\pgfpathlineto{\pgfqpoint{1.214834in}{0.364452in}}%
\pgfpathlineto{\pgfqpoint{1.215908in}{0.364866in}}%
\pgfpathlineto{\pgfqpoint{1.216981in}{0.364447in}}%
\pgfpathlineto{\pgfqpoint{1.218055in}{0.365038in}}%
\pgfpathlineto{\pgfqpoint{1.223424in}{0.365011in}}%
\pgfpathlineto{\pgfqpoint{1.225572in}{0.364322in}}%
\pgfpathlineto{\pgfqpoint{1.229867in}{0.364260in}}%
\pgfpathlineto{\pgfqpoint{1.233089in}{0.363697in}}%
\pgfpathlineto{\pgfqpoint{1.273895in}{0.364187in}}%
\pgfpathlineto{\pgfqpoint{1.278190in}{0.365144in}}%
\pgfpathlineto{\pgfqpoint{1.284633in}{0.364799in}}%
\pgfpathlineto{\pgfqpoint{1.285707in}{0.365627in}}%
\pgfpathlineto{\pgfqpoint{1.290002in}{0.366250in}}%
\pgfpathlineto{\pgfqpoint{1.291076in}{0.365616in}}%
\pgfpathlineto{\pgfqpoint{1.293224in}{0.366972in}}%
\pgfpathlineto{\pgfqpoint{1.297519in}{0.365992in}}%
\pgfpathlineto{\pgfqpoint{1.298593in}{0.365331in}}%
\pgfpathlineto{\pgfqpoint{1.299667in}{0.365498in}}%
\pgfpathlineto{\pgfqpoint{1.307184in}{0.365323in}}%
\pgfpathlineto{\pgfqpoint{1.308258in}{0.366561in}}%
\pgfpathlineto{\pgfqpoint{1.328661in}{0.364992in}}%
\pgfpathlineto{\pgfqpoint{1.330808in}{0.364990in}}%
\pgfpathlineto{\pgfqpoint{1.360876in}{0.364797in}}%
\pgfpathlineto{\pgfqpoint{1.375909in}{0.365024in}}%
\pgfpathlineto{\pgfqpoint{1.404903in}{0.363505in}}%
\pgfpathlineto{\pgfqpoint{1.405977in}{0.364327in}}%
\pgfpathlineto{\pgfqpoint{1.428528in}{0.364647in}}%
\pgfpathlineto{\pgfqpoint{1.436044in}{0.363975in}}%
\pgfpathlineto{\pgfqpoint{1.484367in}{0.364405in}}%
\pgfpathlineto{\pgfqpoint{1.485441in}{0.364827in}}%
\pgfpathlineto{\pgfqpoint{1.486515in}{0.364321in}}%
\pgfpathlineto{\pgfqpoint{1.488662in}{0.365164in}}%
\pgfpathlineto{\pgfqpoint{1.496179in}{0.365023in}}%
\pgfpathlineto{\pgfqpoint{1.499401in}{0.364803in}}%
\pgfpathlineto{\pgfqpoint{1.500475in}{0.364165in}}%
\pgfpathlineto{\pgfqpoint{1.501549in}{0.364430in}}%
\pgfpathlineto{\pgfqpoint{1.503696in}{0.364137in}}%
\pgfpathlineto{\pgfqpoint{1.509065in}{0.364000in}}%
\pgfpathlineto{\pgfqpoint{1.510139in}{0.361591in}}%
\pgfpathlineto{\pgfqpoint{1.511213in}{0.361165in}}%
\pgfpathlineto{\pgfqpoint{1.515508in}{0.361036in}}%
\pgfpathlineto{\pgfqpoint{1.517656in}{0.360435in}}%
\pgfpathlineto{\pgfqpoint{1.518730in}{0.360247in}}%
\pgfpathlineto{\pgfqpoint{1.529468in}{0.360294in}}%
\pgfpathlineto{\pgfqpoint{1.536985in}{0.359915in}}%
\pgfpathlineto{\pgfqpoint{1.541281in}{0.359647in}}%
\pgfpathlineto{\pgfqpoint{1.568127in}{0.359096in}}%
\pgfpathlineto{\pgfqpoint{1.571348in}{0.358998in}}%
\pgfpathlineto{\pgfqpoint{1.612154in}{0.359746in}}%
\pgfpathlineto{\pgfqpoint{1.616449in}{0.358847in}}%
\pgfpathlineto{\pgfqpoint{1.621818in}{0.358808in}}%
\pgfpathlineto{\pgfqpoint{1.623966in}{0.358555in}}%
\pgfpathlineto{\pgfqpoint{1.649738in}{0.358406in}}%
\pgfpathlineto{\pgfqpoint{1.651886in}{0.358849in}}%
\pgfpathlineto{\pgfqpoint{1.652960in}{0.358514in}}%
\pgfpathlineto{\pgfqpoint{1.654034in}{0.358818in}}%
\pgfpathlineto{\pgfqpoint{1.661550in}{0.358490in}}%
\pgfpathlineto{\pgfqpoint{1.695913in}{0.359037in}}%
\pgfpathlineto{\pgfqpoint{1.696987in}{0.358280in}}%
\pgfpathlineto{\pgfqpoint{1.712021in}{0.358904in}}%
\pgfpathlineto{\pgfqpoint{1.717390in}{0.357974in}}%
\pgfpathlineto{\pgfqpoint{1.721685in}{0.357797in}}%
\pgfpathlineto{\pgfqpoint{1.729202in}{0.357783in}}%
\pgfpathlineto{\pgfqpoint{1.739941in}{0.357898in}}%
\pgfpathlineto{\pgfqpoint{1.780747in}{0.358384in}}%
\pgfpathlineto{\pgfqpoint{1.781820in}{0.358018in}}%
\pgfpathlineto{\pgfqpoint{1.807593in}{0.356258in}}%
\pgfpathlineto{\pgfqpoint{1.817257in}{0.356698in}}%
\pgfpathlineto{\pgfqpoint{1.824774in}{0.356511in}}%
\pgfpathlineto{\pgfqpoint{1.826922in}{0.357208in}}%
\pgfpathlineto{\pgfqpoint{1.831217in}{0.358226in}}%
\pgfpathlineto{\pgfqpoint{1.833365in}{0.356836in}}%
\pgfpathlineto{\pgfqpoint{1.837660in}{0.357085in}}%
\pgfpathlineto{\pgfqpoint{1.838734in}{0.357629in}}%
\pgfpathlineto{\pgfqpoint{1.841955in}{0.357569in}}%
\pgfpathlineto{\pgfqpoint{1.849472in}{0.357172in}}%
\pgfpathlineto{\pgfqpoint{1.856989in}{0.357395in}}%
\pgfpathlineto{\pgfqpoint{1.864506in}{0.357172in}}%
\pgfpathlineto{\pgfqpoint{1.867727in}{0.357996in}}%
\pgfpathlineto{\pgfqpoint{1.872023in}{0.357123in}}%
\pgfpathlineto{\pgfqpoint{1.883835in}{0.356146in}}%
\pgfpathlineto{\pgfqpoint{1.885983in}{0.356070in}}%
\pgfpathlineto{\pgfqpoint{1.891352in}{0.355873in}}%
\pgfpathlineto{\pgfqpoint{1.893500in}{0.355842in}}%
\pgfpathlineto{\pgfqpoint{1.901016in}{0.355423in}}%
\pgfpathlineto{\pgfqpoint{1.905312in}{0.356059in}}%
\pgfpathlineto{\pgfqpoint{1.907460in}{0.355417in}}%
\pgfpathlineto{\pgfqpoint{1.909607in}{0.355340in}}%
\pgfpathlineto{\pgfqpoint{1.921419in}{0.355287in}}%
\pgfpathlineto{\pgfqpoint{1.923567in}{0.355038in}}%
\pgfpathlineto{\pgfqpoint{1.946118in}{0.355391in}}%
\pgfpathlineto{\pgfqpoint{1.947192in}{0.355773in}}%
\pgfpathlineto{\pgfqpoint{1.967594in}{0.355300in}}%
\pgfpathlineto{\pgfqpoint{1.977259in}{0.356388in}}%
\pgfpathlineto{\pgfqpoint{1.983702in}{0.356175in}}%
\pgfpathlineto{\pgfqpoint{1.984776in}{0.356594in}}%
\pgfpathlineto{\pgfqpoint{1.998736in}{0.357151in}}%
\pgfpathlineto{\pgfqpoint{1.999810in}{0.355911in}}%
\pgfpathlineto{\pgfqpoint{2.006253in}{0.356074in}}%
\pgfpathlineto{\pgfqpoint{2.010548in}{0.357370in}}%
\pgfpathlineto{\pgfqpoint{2.022360in}{0.356377in}}%
\pgfpathlineto{\pgfqpoint{2.029877in}{0.356098in}}%
\pgfpathlineto{\pgfqpoint{2.050280in}{0.356134in}}%
\pgfpathlineto{\pgfqpoint{2.055649in}{0.355834in}}%
\pgfpathlineto{\pgfqpoint{2.057797in}{0.355800in}}%
\pgfpathlineto{\pgfqpoint{2.090012in}{0.354758in}}%
\pgfpathlineto{\pgfqpoint{2.103972in}{0.355067in}}%
\pgfpathlineto{\pgfqpoint{2.111489in}{0.354930in}}%
\pgfpathlineto{\pgfqpoint{2.112563in}{0.355127in}}%
\pgfpathlineto{\pgfqpoint{2.127596in}{0.354586in}}%
\pgfpathlineto{\pgfqpoint{2.139409in}{0.354418in}}%
\pgfpathlineto{\pgfqpoint{2.141556in}{0.354303in}}%
\pgfpathlineto{\pgfqpoint{2.147999in}{0.354839in}}%
\pgfpathlineto{\pgfqpoint{2.156590in}{0.354814in}}%
\pgfpathlineto{\pgfqpoint{2.157664in}{0.355396in}}%
\pgfpathlineto{\pgfqpoint{2.161959in}{0.355357in}}%
\pgfpathlineto{\pgfqpoint{2.163033in}{0.355046in}}%
\pgfpathlineto{\pgfqpoint{2.165181in}{0.355470in}}%
\pgfpathlineto{\pgfqpoint{2.178067in}{0.354880in}}%
\pgfpathlineto{\pgfqpoint{2.185584in}{0.354534in}}%
\pgfpathlineto{\pgfqpoint{2.193101in}{0.354676in}}%
\pgfpathlineto{\pgfqpoint{2.195248in}{0.354760in}}%
\pgfpathlineto{\pgfqpoint{2.297263in}{0.354063in}}%
\pgfpathlineto{\pgfqpoint{2.300484in}{0.354273in}}%
\pgfpathlineto{\pgfqpoint{2.306927in}{0.354005in}}%
\pgfpathlineto{\pgfqpoint{2.308001in}{0.354003in}}%
\pgfpathlineto{\pgfqpoint{2.312297in}{0.354548in}}%
\pgfpathlineto{\pgfqpoint{2.315518in}{0.354118in}}%
\pgfpathlineto{\pgfqpoint{2.323035in}{0.354226in}}%
\pgfpathlineto{\pgfqpoint{2.334847in}{0.354740in}}%
\pgfpathlineto{\pgfqpoint{2.336995in}{0.354339in}}%
\pgfpathlineto{\pgfqpoint{2.360619in}{0.354555in}}%
\pgfpathlineto{\pgfqpoint{2.375653in}{0.353998in}}%
\pgfpathlineto{\pgfqpoint{2.397130in}{0.353805in}}%
\pgfpathlineto{\pgfqpoint{2.398204in}{0.353194in}}%
\pgfpathlineto{\pgfqpoint{2.418607in}{0.352918in}}%
\pgfpathlineto{\pgfqpoint{2.432567in}{0.352739in}}%
\pgfpathlineto{\pgfqpoint{2.472299in}{0.352759in}}%
\pgfpathlineto{\pgfqpoint{2.507735in}{0.352186in}}%
\pgfpathlineto{\pgfqpoint{2.510957in}{0.352246in}}%
\pgfpathlineto{\pgfqpoint{2.533507in}{0.351980in}}%
\pgfpathlineto{\pgfqpoint{2.547467in}{0.352046in}}%
\pgfpathlineto{\pgfqpoint{2.548541in}{0.351847in}}%
\pgfpathlineto{\pgfqpoint{2.574313in}{0.351688in}}%
\pgfpathlineto{\pgfqpoint{2.585052in}{0.351373in}}%
\pgfpathlineto{\pgfqpoint{2.593642in}{0.351227in}}%
\pgfpathlineto{\pgfqpoint{2.646260in}{0.350718in}}%
\pgfpathlineto{\pgfqpoint{2.653777in}{0.350741in}}%
\pgfpathlineto{\pgfqpoint{2.751497in}{0.349884in}}%
\pgfpathlineto{\pgfqpoint{2.751497in}{0.349884in}}%
\pgfusepath{stroke}%
\end{pgfscope}%
\begin{pgfscope}%
\pgfsetrectcap%
\pgfsetmiterjoin%
\pgfsetlinewidth{0.803000pt}%
\definecolor{currentstroke}{rgb}{1.000000,1.000000,1.000000}%
\pgfsetstrokecolor{currentstroke}%
\pgfsetdash{}{0pt}%
\pgfpathmoveto{\pgfqpoint{0.285588in}{0.331635in}}%
\pgfpathlineto{\pgfqpoint{0.285588in}{0.732520in}}%
\pgfusepath{stroke}%
\end{pgfscope}%
\begin{pgfscope}%
\pgfsetrectcap%
\pgfsetmiterjoin%
\pgfsetlinewidth{0.803000pt}%
\definecolor{currentstroke}{rgb}{1.000000,1.000000,1.000000}%
\pgfsetstrokecolor{currentstroke}%
\pgfsetdash{}{0pt}%
\pgfpathmoveto{\pgfqpoint{2.868921in}{0.331635in}}%
\pgfpathlineto{\pgfqpoint{2.868921in}{0.732520in}}%
\pgfusepath{stroke}%
\end{pgfscope}%
\begin{pgfscope}%
\pgfsetrectcap%
\pgfsetmiterjoin%
\pgfsetlinewidth{0.803000pt}%
\definecolor{currentstroke}{rgb}{1.000000,1.000000,1.000000}%
\pgfsetstrokecolor{currentstroke}%
\pgfsetdash{}{0pt}%
\pgfpathmoveto{\pgfqpoint{0.285587in}{0.331635in}}%
\pgfpathlineto{\pgfqpoint{2.868921in}{0.331635in}}%
\pgfusepath{stroke}%
\end{pgfscope}%
\begin{pgfscope}%
\pgfsetrectcap%
\pgfsetmiterjoin%
\pgfsetlinewidth{0.803000pt}%
\definecolor{currentstroke}{rgb}{1.000000,1.000000,1.000000}%
\pgfsetstrokecolor{currentstroke}%
\pgfsetdash{}{0pt}%
\pgfpathmoveto{\pgfqpoint{0.285587in}{0.732520in}}%
\pgfpathlineto{\pgfqpoint{2.868921in}{0.732520in}}%
\pgfusepath{stroke}%
\end{pgfscope}%
\begin{pgfscope}%
\definecolor{textcolor}{rgb}{0.150000,0.150000,0.150000}%
\pgfsetstrokecolor{textcolor}%
\pgfsetfillcolor{textcolor}%
\pgftext[x=1.577254in,y=0.815853in,,base]{\color{textcolor}\rmfamily\fontsize{12.000000}{14.400000}\selectfont V}%
\end{pgfscope}%
\begin{pgfscope}%
\pgfsetbuttcap%
\pgfsetmiterjoin%
\definecolor{currentfill}{rgb}{0.917647,0.917647,0.949020}%
\pgfsetfillcolor{currentfill}%
\pgfsetlinewidth{0.000000pt}%
\definecolor{currentstroke}{rgb}{0.000000,0.000000,0.000000}%
\pgfsetstrokecolor{currentstroke}%
\pgfsetstrokeopacity{0.000000}%
\pgfsetdash{}{0pt}%
\pgfpathmoveto{\pgfqpoint{3.902254in}{0.331635in}}%
\pgfpathlineto{\pgfqpoint{6.485588in}{0.331635in}}%
\pgfpathlineto{\pgfqpoint{6.485588in}{0.732520in}}%
\pgfpathlineto{\pgfqpoint{3.902254in}{0.732520in}}%
\pgfpathclose%
\pgfusepath{fill}%
\end{pgfscope}%
\begin{pgfscope}%
\pgfpathrectangle{\pgfqpoint{3.902254in}{0.331635in}}{\pgfqpoint{2.583333in}{0.400885in}}%
\pgfusepath{clip}%
\pgfsetroundcap%
\pgfsetroundjoin%
\pgfsetlinewidth{0.803000pt}%
\definecolor{currentstroke}{rgb}{1.000000,1.000000,1.000000}%
\pgfsetstrokecolor{currentstroke}%
\pgfsetdash{}{0pt}%
\pgfpathmoveto{\pgfqpoint{4.017531in}{0.331635in}}%
\pgfpathlineto{\pgfqpoint{4.017531in}{0.732520in}}%
\pgfusepath{stroke}%
\end{pgfscope}%
\begin{pgfscope}%
\definecolor{textcolor}{rgb}{0.150000,0.150000,0.150000}%
\pgfsetstrokecolor{textcolor}%
\pgfsetfillcolor{textcolor}%
\pgftext[x=4.017531in,y=0.234413in,,top]{\color{textcolor}\rmfamily\fontsize{10.000000}{12.000000}\selectfont 2012}%
\end{pgfscope}%
\begin{pgfscope}%
\pgfpathrectangle{\pgfqpoint{3.902254in}{0.331635in}}{\pgfqpoint{2.583333in}{0.400885in}}%
\pgfusepath{clip}%
\pgfsetroundcap%
\pgfsetroundjoin%
\pgfsetlinewidth{0.803000pt}%
\definecolor{currentstroke}{rgb}{1.000000,1.000000,1.000000}%
\pgfsetstrokecolor{currentstroke}%
\pgfsetdash{}{0pt}%
\pgfpathmoveto{\pgfqpoint{4.410556in}{0.331635in}}%
\pgfpathlineto{\pgfqpoint{4.410556in}{0.732520in}}%
\pgfusepath{stroke}%
\end{pgfscope}%
\begin{pgfscope}%
\definecolor{textcolor}{rgb}{0.150000,0.150000,0.150000}%
\pgfsetstrokecolor{textcolor}%
\pgfsetfillcolor{textcolor}%
\pgftext[x=4.410556in,y=0.234413in,,top]{\color{textcolor}\rmfamily\fontsize{10.000000}{12.000000}\selectfont 2013}%
\end{pgfscope}%
\begin{pgfscope}%
\pgfpathrectangle{\pgfqpoint{3.902254in}{0.331635in}}{\pgfqpoint{2.583333in}{0.400885in}}%
\pgfusepath{clip}%
\pgfsetroundcap%
\pgfsetroundjoin%
\pgfsetlinewidth{0.803000pt}%
\definecolor{currentstroke}{rgb}{1.000000,1.000000,1.000000}%
\pgfsetstrokecolor{currentstroke}%
\pgfsetdash{}{0pt}%
\pgfpathmoveto{\pgfqpoint{4.802507in}{0.331635in}}%
\pgfpathlineto{\pgfqpoint{4.802507in}{0.732520in}}%
\pgfusepath{stroke}%
\end{pgfscope}%
\begin{pgfscope}%
\definecolor{textcolor}{rgb}{0.150000,0.150000,0.150000}%
\pgfsetstrokecolor{textcolor}%
\pgfsetfillcolor{textcolor}%
\pgftext[x=4.802507in,y=0.234413in,,top]{\color{textcolor}\rmfamily\fontsize{10.000000}{12.000000}\selectfont 2014}%
\end{pgfscope}%
\begin{pgfscope}%
\pgfpathrectangle{\pgfqpoint{3.902254in}{0.331635in}}{\pgfqpoint{2.583333in}{0.400885in}}%
\pgfusepath{clip}%
\pgfsetroundcap%
\pgfsetroundjoin%
\pgfsetlinewidth{0.803000pt}%
\definecolor{currentstroke}{rgb}{1.000000,1.000000,1.000000}%
\pgfsetstrokecolor{currentstroke}%
\pgfsetdash{}{0pt}%
\pgfpathmoveto{\pgfqpoint{5.194458in}{0.331635in}}%
\pgfpathlineto{\pgfqpoint{5.194458in}{0.732520in}}%
\pgfusepath{stroke}%
\end{pgfscope}%
\begin{pgfscope}%
\definecolor{textcolor}{rgb}{0.150000,0.150000,0.150000}%
\pgfsetstrokecolor{textcolor}%
\pgfsetfillcolor{textcolor}%
\pgftext[x=5.194458in,y=0.234413in,,top]{\color{textcolor}\rmfamily\fontsize{10.000000}{12.000000}\selectfont 2015}%
\end{pgfscope}%
\begin{pgfscope}%
\pgfpathrectangle{\pgfqpoint{3.902254in}{0.331635in}}{\pgfqpoint{2.583333in}{0.400885in}}%
\pgfusepath{clip}%
\pgfsetroundcap%
\pgfsetroundjoin%
\pgfsetlinewidth{0.803000pt}%
\definecolor{currentstroke}{rgb}{1.000000,1.000000,1.000000}%
\pgfsetstrokecolor{currentstroke}%
\pgfsetdash{}{0pt}%
\pgfpathmoveto{\pgfqpoint{5.586409in}{0.331635in}}%
\pgfpathlineto{\pgfqpoint{5.586409in}{0.732520in}}%
\pgfusepath{stroke}%
\end{pgfscope}%
\begin{pgfscope}%
\definecolor{textcolor}{rgb}{0.150000,0.150000,0.150000}%
\pgfsetstrokecolor{textcolor}%
\pgfsetfillcolor{textcolor}%
\pgftext[x=5.586409in,y=0.234413in,,top]{\color{textcolor}\rmfamily\fontsize{10.000000}{12.000000}\selectfont 2016}%
\end{pgfscope}%
\begin{pgfscope}%
\pgfpathrectangle{\pgfqpoint{3.902254in}{0.331635in}}{\pgfqpoint{2.583333in}{0.400885in}}%
\pgfusepath{clip}%
\pgfsetroundcap%
\pgfsetroundjoin%
\pgfsetlinewidth{0.803000pt}%
\definecolor{currentstroke}{rgb}{1.000000,1.000000,1.000000}%
\pgfsetstrokecolor{currentstroke}%
\pgfsetdash{}{0pt}%
\pgfpathmoveto{\pgfqpoint{5.979434in}{0.331635in}}%
\pgfpathlineto{\pgfqpoint{5.979434in}{0.732520in}}%
\pgfusepath{stroke}%
\end{pgfscope}%
\begin{pgfscope}%
\definecolor{textcolor}{rgb}{0.150000,0.150000,0.150000}%
\pgfsetstrokecolor{textcolor}%
\pgfsetfillcolor{textcolor}%
\pgftext[x=5.979434in,y=0.234413in,,top]{\color{textcolor}\rmfamily\fontsize{10.000000}{12.000000}\selectfont 2017}%
\end{pgfscope}%
\begin{pgfscope}%
\pgfpathrectangle{\pgfqpoint{3.902254in}{0.331635in}}{\pgfqpoint{2.583333in}{0.400885in}}%
\pgfusepath{clip}%
\pgfsetroundcap%
\pgfsetroundjoin%
\pgfsetlinewidth{0.803000pt}%
\definecolor{currentstroke}{rgb}{1.000000,1.000000,1.000000}%
\pgfsetstrokecolor{currentstroke}%
\pgfsetdash{}{0pt}%
\pgfpathmoveto{\pgfqpoint{6.371385in}{0.331635in}}%
\pgfpathlineto{\pgfqpoint{6.371385in}{0.732520in}}%
\pgfusepath{stroke}%
\end{pgfscope}%
\begin{pgfscope}%
\definecolor{textcolor}{rgb}{0.150000,0.150000,0.150000}%
\pgfsetstrokecolor{textcolor}%
\pgfsetfillcolor{textcolor}%
\pgftext[x=6.371385in,y=0.234413in,,top]{\color{textcolor}\rmfamily\fontsize{10.000000}{12.000000}\selectfont 2018}%
\end{pgfscope}%
\begin{pgfscope}%
\pgfpathrectangle{\pgfqpoint{3.902254in}{0.331635in}}{\pgfqpoint{2.583333in}{0.400885in}}%
\pgfusepath{clip}%
\pgfsetroundcap%
\pgfsetroundjoin%
\pgfsetlinewidth{0.803000pt}%
\definecolor{currentstroke}{rgb}{1.000000,1.000000,1.000000}%
\pgfsetstrokecolor{currentstroke}%
\pgfsetdash{}{0pt}%
\pgfpathmoveto{\pgfqpoint{3.902254in}{0.440624in}}%
\pgfpathlineto{\pgfqpoint{6.485588in}{0.440624in}}%
\pgfusepath{stroke}%
\end{pgfscope}%
\begin{pgfscope}%
\definecolor{textcolor}{rgb}{0.150000,0.150000,0.150000}%
\pgfsetstrokecolor{textcolor}%
\pgfsetfillcolor{textcolor}%
\pgftext[x=3.716667in,y=0.387862in,left,base]{\color{textcolor}\rmfamily\fontsize{10.000000}{12.000000}\selectfont 1}%
\end{pgfscope}%
\begin{pgfscope}%
\pgfpathrectangle{\pgfqpoint{3.902254in}{0.331635in}}{\pgfqpoint{2.583333in}{0.400885in}}%
\pgfusepath{clip}%
\pgfsetroundcap%
\pgfsetroundjoin%
\pgfsetlinewidth{0.803000pt}%
\definecolor{currentstroke}{rgb}{1.000000,1.000000,1.000000}%
\pgfsetstrokecolor{currentstroke}%
\pgfsetdash{}{0pt}%
\pgfpathmoveto{\pgfqpoint{3.902254in}{0.558312in}}%
\pgfpathlineto{\pgfqpoint{6.485588in}{0.558312in}}%
\pgfusepath{stroke}%
\end{pgfscope}%
\begin{pgfscope}%
\definecolor{textcolor}{rgb}{0.150000,0.150000,0.150000}%
\pgfsetstrokecolor{textcolor}%
\pgfsetfillcolor{textcolor}%
\pgftext[x=3.716667in,y=0.505551in,left,base]{\color{textcolor}\rmfamily\fontsize{10.000000}{12.000000}\selectfont 2}%
\end{pgfscope}%
\begin{pgfscope}%
\pgfpathrectangle{\pgfqpoint{3.902254in}{0.331635in}}{\pgfqpoint{2.583333in}{0.400885in}}%
\pgfusepath{clip}%
\pgfsetroundcap%
\pgfsetroundjoin%
\pgfsetlinewidth{0.803000pt}%
\definecolor{currentstroke}{rgb}{1.000000,1.000000,1.000000}%
\pgfsetstrokecolor{currentstroke}%
\pgfsetdash{}{0pt}%
\pgfpathmoveto{\pgfqpoint{3.902254in}{0.676001in}}%
\pgfpathlineto{\pgfqpoint{6.485588in}{0.676001in}}%
\pgfusepath{stroke}%
\end{pgfscope}%
\begin{pgfscope}%
\definecolor{textcolor}{rgb}{0.150000,0.150000,0.150000}%
\pgfsetstrokecolor{textcolor}%
\pgfsetfillcolor{textcolor}%
\pgftext[x=3.716667in,y=0.623239in,left,base]{\color{textcolor}\rmfamily\fontsize{10.000000}{12.000000}\selectfont 3}%
\end{pgfscope}%
\begin{pgfscope}%
\pgfpathrectangle{\pgfqpoint{3.902254in}{0.331635in}}{\pgfqpoint{2.583333in}{0.400885in}}%
\pgfusepath{clip}%
\pgfsetroundcap%
\pgfsetroundjoin%
\pgfsetlinewidth{1.505625pt}%
\definecolor{currentstroke}{rgb}{0.090196,0.745098,0.811765}%
\pgfsetstrokecolor{currentstroke}%
\pgfsetstrokeopacity{0.300000}%
\pgfsetdash{}{0pt}%
\pgfpathmoveto{\pgfqpoint{4.019678in}{0.440624in}}%
\pgfpathlineto{\pgfqpoint{4.022900in}{0.445535in}}%
\pgfpathlineto{\pgfqpoint{4.027195in}{0.444682in}}%
\pgfpathlineto{\pgfqpoint{4.028269in}{0.441817in}}%
\pgfpathlineto{\pgfqpoint{4.029343in}{0.441886in}}%
\pgfpathlineto{\pgfqpoint{4.030417in}{0.440897in}}%
\pgfpathlineto{\pgfqpoint{4.034712in}{0.441135in}}%
\pgfpathlineto{\pgfqpoint{4.036860in}{0.444068in}}%
\pgfpathlineto{\pgfqpoint{4.037934in}{0.443693in}}%
\pgfpathlineto{\pgfqpoint{4.042229in}{0.443489in}}%
\pgfpathlineto{\pgfqpoint{4.043303in}{0.444443in}}%
\pgfpathlineto{\pgfqpoint{4.045451in}{0.443489in}}%
\pgfpathlineto{\pgfqpoint{4.049746in}{0.442431in}}%
\pgfpathlineto{\pgfqpoint{4.050820in}{0.443727in}}%
\pgfpathlineto{\pgfqpoint{4.051894in}{0.442465in}}%
\pgfpathlineto{\pgfqpoint{4.052967in}{0.445807in}}%
\pgfpathlineto{\pgfqpoint{4.056189in}{0.447206in}}%
\pgfpathlineto{\pgfqpoint{4.058337in}{0.449695in}}%
\pgfpathlineto{\pgfqpoint{4.059410in}{0.450514in}}%
\pgfpathlineto{\pgfqpoint{4.060484in}{0.450241in}}%
\pgfpathlineto{\pgfqpoint{4.063706in}{0.451298in}}%
\pgfpathlineto{\pgfqpoint{4.065853in}{0.449627in}}%
\pgfpathlineto{\pgfqpoint{4.068001in}{0.451162in}}%
\pgfpathlineto{\pgfqpoint{4.072296in}{0.450616in}}%
\pgfpathlineto{\pgfqpoint{4.073370in}{0.449695in}}%
\pgfpathlineto{\pgfqpoint{4.074444in}{0.450343in}}%
\pgfpathlineto{\pgfqpoint{4.075518in}{0.449832in}}%
\pgfpathlineto{\pgfqpoint{4.080887in}{0.451912in}}%
\pgfpathlineto{\pgfqpoint{4.081961in}{0.453140in}}%
\pgfpathlineto{\pgfqpoint{4.083035in}{0.453037in}}%
\pgfpathlineto{\pgfqpoint{4.086256in}{0.454094in}}%
\pgfpathlineto{\pgfqpoint{4.087330in}{0.451946in}}%
\pgfpathlineto{\pgfqpoint{4.088404in}{0.451162in}}%
\pgfpathlineto{\pgfqpoint{4.090552in}{0.452696in}}%
\pgfpathlineto{\pgfqpoint{4.093773in}{0.453003in}}%
\pgfpathlineto{\pgfqpoint{4.094847in}{0.458119in}}%
\pgfpathlineto{\pgfqpoint{4.095921in}{0.456482in}}%
\pgfpathlineto{\pgfqpoint{4.096995in}{0.456448in}}%
\pgfpathlineto{\pgfqpoint{4.098069in}{0.455595in}}%
\pgfpathlineto{\pgfqpoint{4.101290in}{0.456379in}}%
\pgfpathlineto{\pgfqpoint{4.102364in}{0.455765in}}%
\pgfpathlineto{\pgfqpoint{4.104512in}{0.455902in}}%
\pgfpathlineto{\pgfqpoint{4.105585in}{0.457027in}}%
\pgfpathlineto{\pgfqpoint{4.108807in}{0.459244in}}%
\pgfpathlineto{\pgfqpoint{4.109881in}{0.458562in}}%
\pgfpathlineto{\pgfqpoint{4.112029in}{0.455049in}}%
\pgfpathlineto{\pgfqpoint{4.113102in}{0.457402in}}%
\pgfpathlineto{\pgfqpoint{4.116324in}{0.457607in}}%
\pgfpathlineto{\pgfqpoint{4.118472in}{0.454811in}}%
\pgfpathlineto{\pgfqpoint{4.119545in}{0.455254in}}%
\pgfpathlineto{\pgfqpoint{4.123841in}{0.452287in}}%
\pgfpathlineto{\pgfqpoint{4.124915in}{0.448843in}}%
\pgfpathlineto{\pgfqpoint{4.125988in}{0.450002in}}%
\pgfpathlineto{\pgfqpoint{4.127062in}{0.452389in}}%
\pgfpathlineto{\pgfqpoint{4.128136in}{0.451469in}}%
\pgfpathlineto{\pgfqpoint{4.131358in}{0.450889in}}%
\pgfpathlineto{\pgfqpoint{4.132431in}{0.454026in}}%
\pgfpathlineto{\pgfqpoint{4.133505in}{0.453446in}}%
\pgfpathlineto{\pgfqpoint{4.134579in}{0.452185in}}%
\pgfpathlineto{\pgfqpoint{4.135653in}{0.453003in}}%
\pgfpathlineto{\pgfqpoint{4.138874in}{0.451980in}}%
\pgfpathlineto{\pgfqpoint{4.139948in}{0.452492in}}%
\pgfpathlineto{\pgfqpoint{4.142096in}{0.456107in}}%
\pgfpathlineto{\pgfqpoint{4.143170in}{0.456107in}}%
\pgfpathlineto{\pgfqpoint{4.146391in}{0.455356in}}%
\pgfpathlineto{\pgfqpoint{4.147465in}{0.457437in}}%
\pgfpathlineto{\pgfqpoint{4.148539in}{0.456686in}}%
\pgfpathlineto{\pgfqpoint{4.149613in}{0.457505in}}%
\pgfpathlineto{\pgfqpoint{4.150687in}{0.454811in}}%
\pgfpathlineto{\pgfqpoint{4.154982in}{0.459005in}}%
\pgfpathlineto{\pgfqpoint{4.156056in}{0.461222in}}%
\pgfpathlineto{\pgfqpoint{4.158204in}{0.462893in}}%
\pgfpathlineto{\pgfqpoint{4.163573in}{0.461392in}}%
\pgfpathlineto{\pgfqpoint{4.165720in}{0.457505in}}%
\pgfpathlineto{\pgfqpoint{4.170016in}{0.459278in}}%
\pgfpathlineto{\pgfqpoint{4.171090in}{0.458732in}}%
\pgfpathlineto{\pgfqpoint{4.177533in}{0.462620in}}%
\pgfpathlineto{\pgfqpoint{4.178607in}{0.461768in}}%
\pgfpathlineto{\pgfqpoint{4.179680in}{0.463336in}}%
\pgfpathlineto{\pgfqpoint{4.180754in}{0.459312in}}%
\pgfpathlineto{\pgfqpoint{4.183976in}{0.459346in}}%
\pgfpathlineto{\pgfqpoint{4.188271in}{0.464973in}}%
\pgfpathlineto{\pgfqpoint{4.191493in}{0.463609in}}%
\pgfpathlineto{\pgfqpoint{4.192566in}{0.465382in}}%
\pgfpathlineto{\pgfqpoint{4.193640in}{0.464939in}}%
\pgfpathlineto{\pgfqpoint{4.194714in}{0.467872in}}%
\pgfpathlineto{\pgfqpoint{4.195788in}{0.467565in}}%
\pgfpathlineto{\pgfqpoint{4.199009in}{0.467599in}}%
\pgfpathlineto{\pgfqpoint{4.201157in}{0.469543in}}%
\pgfpathlineto{\pgfqpoint{4.202231in}{0.468520in}}%
\pgfpathlineto{\pgfqpoint{4.203305in}{0.468759in}}%
\pgfpathlineto{\pgfqpoint{4.206526in}{0.466371in}}%
\pgfpathlineto{\pgfqpoint{4.208674in}{0.469986in}}%
\pgfpathlineto{\pgfqpoint{4.209748in}{0.469816in}}%
\pgfpathlineto{\pgfqpoint{4.210822in}{0.471896in}}%
\pgfpathlineto{\pgfqpoint{4.214043in}{0.472578in}}%
\pgfpathlineto{\pgfqpoint{4.215117in}{0.472203in}}%
\pgfpathlineto{\pgfqpoint{4.218339in}{0.470498in}}%
\pgfpathlineto{\pgfqpoint{4.221560in}{0.470361in}}%
\pgfpathlineto{\pgfqpoint{4.222634in}{0.468418in}}%
\pgfpathlineto{\pgfqpoint{4.223708in}{0.468145in}}%
\pgfpathlineto{\pgfqpoint{4.224782in}{0.468554in}}%
\pgfpathlineto{\pgfqpoint{4.225855in}{0.470975in}}%
\pgfpathlineto{\pgfqpoint{4.229077in}{0.469952in}}%
\pgfpathlineto{\pgfqpoint{4.230151in}{0.474522in}}%
\pgfpathlineto{\pgfqpoint{4.231225in}{0.474522in}}%
\pgfpathlineto{\pgfqpoint{4.233372in}{0.472203in}}%
\pgfpathlineto{\pgfqpoint{4.236594in}{0.470327in}}%
\pgfpathlineto{\pgfqpoint{4.238741in}{0.471282in}}%
\pgfpathlineto{\pgfqpoint{4.239815in}{0.475613in}}%
\pgfpathlineto{\pgfqpoint{4.240889in}{0.476329in}}%
\pgfpathlineto{\pgfqpoint{4.244111in}{0.475920in}}%
\pgfpathlineto{\pgfqpoint{4.246258in}{0.472885in}}%
\pgfpathlineto{\pgfqpoint{4.247332in}{0.473397in}}%
\pgfpathlineto{\pgfqpoint{4.248406in}{0.475818in}}%
\pgfpathlineto{\pgfqpoint{4.251628in}{0.475443in}}%
\pgfpathlineto{\pgfqpoint{4.252701in}{0.475920in}}%
\pgfpathlineto{\pgfqpoint{4.253775in}{0.478035in}}%
\pgfpathlineto{\pgfqpoint{4.255923in}{0.475443in}}%
\pgfpathlineto{\pgfqpoint{4.259144in}{0.476091in}}%
\pgfpathlineto{\pgfqpoint{4.260218in}{0.475579in}}%
\pgfpathlineto{\pgfqpoint{4.263440in}{0.477932in}}%
\pgfpathlineto{\pgfqpoint{4.266661in}{0.477898in}}%
\pgfpathlineto{\pgfqpoint{4.267735in}{0.475409in}}%
\pgfpathlineto{\pgfqpoint{4.268809in}{0.475477in}}%
\pgfpathlineto{\pgfqpoint{4.269883in}{0.473976in}}%
\pgfpathlineto{\pgfqpoint{4.270957in}{0.475170in}}%
\pgfpathlineto{\pgfqpoint{4.275252in}{0.475375in}}%
\pgfpathlineto{\pgfqpoint{4.276326in}{0.476636in}}%
\pgfpathlineto{\pgfqpoint{4.277400in}{0.474727in}}%
\pgfpathlineto{\pgfqpoint{4.282769in}{0.475477in}}%
\pgfpathlineto{\pgfqpoint{4.284917in}{0.482229in}}%
\pgfpathlineto{\pgfqpoint{4.289212in}{0.481172in}}%
\pgfpathlineto{\pgfqpoint{4.291360in}{0.481752in}}%
\pgfpathlineto{\pgfqpoint{4.292433in}{0.484514in}}%
\pgfpathlineto{\pgfqpoint{4.293507in}{0.483730in}}%
\pgfpathlineto{\pgfqpoint{4.297803in}{0.482366in}}%
\pgfpathlineto{\pgfqpoint{4.298876in}{0.484821in}}%
\pgfpathlineto{\pgfqpoint{4.301024in}{0.484923in}}%
\pgfpathlineto{\pgfqpoint{4.304246in}{0.485503in}}%
\pgfpathlineto{\pgfqpoint{4.306393in}{0.482468in}}%
\pgfpathlineto{\pgfqpoint{4.307467in}{0.484446in}}%
\pgfpathlineto{\pgfqpoint{4.308541in}{0.483525in}}%
\pgfpathlineto{\pgfqpoint{4.311762in}{0.482877in}}%
\pgfpathlineto{\pgfqpoint{4.312836in}{0.481547in}}%
\pgfpathlineto{\pgfqpoint{4.313910in}{0.484105in}}%
\pgfpathlineto{\pgfqpoint{4.316058in}{0.485639in}}%
\pgfpathlineto{\pgfqpoint{4.319279in}{0.483661in}}%
\pgfpathlineto{\pgfqpoint{4.320353in}{0.481172in}}%
\pgfpathlineto{\pgfqpoint{4.321427in}{0.480251in}}%
\pgfpathlineto{\pgfqpoint{4.322501in}{0.477557in}}%
\pgfpathlineto{\pgfqpoint{4.323575in}{0.478341in}}%
\pgfpathlineto{\pgfqpoint{4.326796in}{0.478955in}}%
\pgfpathlineto{\pgfqpoint{4.327870in}{0.480319in}}%
\pgfpathlineto{\pgfqpoint{4.328944in}{0.483559in}}%
\pgfpathlineto{\pgfqpoint{4.330018in}{0.483968in}}%
\pgfpathlineto{\pgfqpoint{4.331092in}{0.482366in}}%
\pgfpathlineto{\pgfqpoint{4.334313in}{0.482025in}}%
\pgfpathlineto{\pgfqpoint{4.335387in}{0.478853in}}%
\pgfpathlineto{\pgfqpoint{4.336461in}{0.478512in}}%
\pgfpathlineto{\pgfqpoint{4.338608in}{0.476773in}}%
\pgfpathlineto{\pgfqpoint{4.343978in}{0.473806in}}%
\pgfpathlineto{\pgfqpoint{4.345051in}{0.475852in}}%
\pgfpathlineto{\pgfqpoint{4.346125in}{0.476091in}}%
\pgfpathlineto{\pgfqpoint{4.350421in}{0.477966in}}%
\pgfpathlineto{\pgfqpoint{4.351495in}{0.476773in}}%
\pgfpathlineto{\pgfqpoint{4.352568in}{0.476636in}}%
\pgfpathlineto{\pgfqpoint{4.353642in}{0.467497in}}%
\pgfpathlineto{\pgfqpoint{4.356864in}{0.468690in}}%
\pgfpathlineto{\pgfqpoint{4.357938in}{0.470259in}}%
\pgfpathlineto{\pgfqpoint{4.359011in}{0.467838in}}%
\pgfpathlineto{\pgfqpoint{4.360085in}{0.468759in}}%
\pgfpathlineto{\pgfqpoint{4.361159in}{0.468588in}}%
\pgfpathlineto{\pgfqpoint{4.364381in}{0.470089in}}%
\pgfpathlineto{\pgfqpoint{4.366528in}{0.472476in}}%
\pgfpathlineto{\pgfqpoint{4.368676in}{0.474249in}}%
\pgfpathlineto{\pgfqpoint{4.371897in}{0.473533in}}%
\pgfpathlineto{\pgfqpoint{4.372971in}{0.472203in}}%
\pgfpathlineto{\pgfqpoint{4.375119in}{0.475647in}}%
\pgfpathlineto{\pgfqpoint{4.376193in}{0.475477in}}%
\pgfpathlineto{\pgfqpoint{4.380488in}{0.474351in}}%
\pgfpathlineto{\pgfqpoint{4.383710in}{0.476500in}}%
\pgfpathlineto{\pgfqpoint{4.388005in}{0.477250in}}%
\pgfpathlineto{\pgfqpoint{4.389079in}{0.477762in}}%
\pgfpathlineto{\pgfqpoint{4.391227in}{0.474727in}}%
\pgfpathlineto{\pgfqpoint{4.394448in}{0.476636in}}%
\pgfpathlineto{\pgfqpoint{4.395522in}{0.479569in}}%
\pgfpathlineto{\pgfqpoint{4.396596in}{0.478682in}}%
\pgfpathlineto{\pgfqpoint{4.397670in}{0.481786in}}%
\pgfpathlineto{\pgfqpoint{4.398743in}{0.478887in}}%
\pgfpathlineto{\pgfqpoint{4.404113in}{0.478410in}}%
\pgfpathlineto{\pgfqpoint{4.406260in}{0.476227in}}%
\pgfpathlineto{\pgfqpoint{4.409482in}{0.478205in}}%
\pgfpathlineto{\pgfqpoint{4.411629in}{0.482297in}}%
\pgfpathlineto{\pgfqpoint{4.412703in}{0.482638in}}%
\pgfpathlineto{\pgfqpoint{4.413777in}{0.485708in}}%
\pgfpathlineto{\pgfqpoint{4.418073in}{0.481240in}}%
\pgfpathlineto{\pgfqpoint{4.420220in}{0.481342in}}%
\pgfpathlineto{\pgfqpoint{4.421294in}{0.480695in}}%
\pgfpathlineto{\pgfqpoint{4.424516in}{0.480729in}}%
\pgfpathlineto{\pgfqpoint{4.426663in}{0.483661in}}%
\pgfpathlineto{\pgfqpoint{4.427737in}{0.486390in}}%
\pgfpathlineto{\pgfqpoint{4.428811in}{0.486185in}}%
\pgfpathlineto{\pgfqpoint{4.433106in}{0.487379in}}%
\pgfpathlineto{\pgfqpoint{4.434180in}{0.491198in}}%
\pgfpathlineto{\pgfqpoint{4.435254in}{0.491198in}}%
\pgfpathlineto{\pgfqpoint{4.436328in}{0.492528in}}%
\pgfpathlineto{\pgfqpoint{4.439549in}{0.492460in}}%
\pgfpathlineto{\pgfqpoint{4.441697in}{0.490687in}}%
\pgfpathlineto{\pgfqpoint{4.442771in}{0.490994in}}%
\pgfpathlineto{\pgfqpoint{4.443845in}{0.493176in}}%
\pgfpathlineto{\pgfqpoint{4.447066in}{0.491028in}}%
\pgfpathlineto{\pgfqpoint{4.449214in}{0.492972in}}%
\pgfpathlineto{\pgfqpoint{4.450288in}{0.492460in}}%
\pgfpathlineto{\pgfqpoint{4.451361in}{0.493415in}}%
\pgfpathlineto{\pgfqpoint{4.457805in}{0.494097in}}%
\pgfpathlineto{\pgfqpoint{4.458878in}{0.496382in}}%
\pgfpathlineto{\pgfqpoint{4.463174in}{0.496757in}}%
\pgfpathlineto{\pgfqpoint{4.464248in}{0.493210in}}%
\pgfpathlineto{\pgfqpoint{4.465321in}{0.491880in}}%
\pgfpathlineto{\pgfqpoint{4.466395in}{0.492119in}}%
\pgfpathlineto{\pgfqpoint{4.469617in}{0.490073in}}%
\pgfpathlineto{\pgfqpoint{4.470691in}{0.491028in}}%
\pgfpathlineto{\pgfqpoint{4.471764in}{0.492835in}}%
\pgfpathlineto{\pgfqpoint{4.472838in}{0.493176in}}%
\pgfpathlineto{\pgfqpoint{4.473912in}{0.495495in}}%
\pgfpathlineto{\pgfqpoint{4.477134in}{0.496962in}}%
\pgfpathlineto{\pgfqpoint{4.478207in}{0.499076in}}%
\pgfpathlineto{\pgfqpoint{4.480355in}{0.498598in}}%
\pgfpathlineto{\pgfqpoint{4.481429in}{0.501941in}}%
\pgfpathlineto{\pgfqpoint{4.484650in}{0.502759in}}%
\pgfpathlineto{\pgfqpoint{4.485724in}{0.501054in}}%
\pgfpathlineto{\pgfqpoint{4.487872in}{0.503032in}}%
\pgfpathlineto{\pgfqpoint{4.488946in}{0.502520in}}%
\pgfpathlineto{\pgfqpoint{4.492167in}{0.500167in}}%
\pgfpathlineto{\pgfqpoint{4.493241in}{0.498564in}}%
\pgfpathlineto{\pgfqpoint{4.494315in}{0.500508in}}%
\pgfpathlineto{\pgfqpoint{4.495389in}{0.498564in}}%
\pgfpathlineto{\pgfqpoint{4.496463in}{0.500031in}}%
\pgfpathlineto{\pgfqpoint{4.499684in}{0.498257in}}%
\pgfpathlineto{\pgfqpoint{4.500758in}{0.499553in}}%
\pgfpathlineto{\pgfqpoint{4.501832in}{0.499042in}}%
\pgfpathlineto{\pgfqpoint{4.502906in}{0.500099in}}%
\pgfpathlineto{\pgfqpoint{4.507201in}{0.499758in}}%
\pgfpathlineto{\pgfqpoint{4.508275in}{0.502145in}}%
\pgfpathlineto{\pgfqpoint{4.509349in}{0.501497in}}%
\pgfpathlineto{\pgfqpoint{4.515792in}{0.507397in}}%
\pgfpathlineto{\pgfqpoint{4.517939in}{0.511796in}}%
\pgfpathlineto{\pgfqpoint{4.519013in}{0.511796in}}%
\pgfpathlineto{\pgfqpoint{4.522235in}{0.506578in}}%
\pgfpathlineto{\pgfqpoint{4.523309in}{0.512410in}}%
\pgfpathlineto{\pgfqpoint{4.524383in}{0.512171in}}%
\pgfpathlineto{\pgfqpoint{4.525456in}{0.510023in}}%
\pgfpathlineto{\pgfqpoint{4.526530in}{0.514934in}}%
\pgfpathlineto{\pgfqpoint{4.529752in}{0.516332in}}%
\pgfpathlineto{\pgfqpoint{4.530826in}{0.518139in}}%
\pgfpathlineto{\pgfqpoint{4.531899in}{0.516127in}}%
\pgfpathlineto{\pgfqpoint{4.532973in}{0.516298in}}%
\pgfpathlineto{\pgfqpoint{4.534047in}{0.515889in}}%
\pgfpathlineto{\pgfqpoint{4.537269in}{0.519435in}}%
\pgfpathlineto{\pgfqpoint{4.538342in}{0.518924in}}%
\pgfpathlineto{\pgfqpoint{4.539416in}{0.520083in}}%
\pgfpathlineto{\pgfqpoint{4.541564in}{0.525028in}}%
\pgfpathlineto{\pgfqpoint{4.544785in}{0.525846in}}%
\pgfpathlineto{\pgfqpoint{4.545859in}{0.528984in}}%
\pgfpathlineto{\pgfqpoint{4.546933in}{0.528745in}}%
\pgfpathlineto{\pgfqpoint{4.549081in}{0.532531in}}%
\pgfpathlineto{\pgfqpoint{4.553376in}{0.533349in}}%
\pgfpathlineto{\pgfqpoint{4.554450in}{0.533997in}}%
\pgfpathlineto{\pgfqpoint{4.555524in}{0.530246in}}%
\pgfpathlineto{\pgfqpoint{4.556598in}{0.530587in}}%
\pgfpathlineto{\pgfqpoint{4.560893in}{0.528234in}}%
\pgfpathlineto{\pgfqpoint{4.563041in}{0.526392in}}%
\pgfpathlineto{\pgfqpoint{4.568410in}{0.530928in}}%
\pgfpathlineto{\pgfqpoint{4.569484in}{0.529598in}}%
\pgfpathlineto{\pgfqpoint{4.571631in}{0.519674in}}%
\pgfpathlineto{\pgfqpoint{4.574853in}{0.521925in}}%
\pgfpathlineto{\pgfqpoint{4.575927in}{0.523630in}}%
\pgfpathlineto{\pgfqpoint{4.577001in}{0.519810in}}%
\pgfpathlineto{\pgfqpoint{4.578074in}{0.519844in}}%
\pgfpathlineto{\pgfqpoint{4.579148in}{0.525199in}}%
\pgfpathlineto{\pgfqpoint{4.582370in}{0.522027in}}%
\pgfpathlineto{\pgfqpoint{4.583444in}{0.521925in}}%
\pgfpathlineto{\pgfqpoint{4.584517in}{0.519401in}}%
\pgfpathlineto{\pgfqpoint{4.585591in}{0.523459in}}%
\pgfpathlineto{\pgfqpoint{4.586665in}{0.521925in}}%
\pgfpathlineto{\pgfqpoint{4.589887in}{0.524073in}}%
\pgfpathlineto{\pgfqpoint{4.590961in}{0.526563in}}%
\pgfpathlineto{\pgfqpoint{4.592034in}{0.523562in}}%
\pgfpathlineto{\pgfqpoint{4.593108in}{0.516230in}}%
\pgfpathlineto{\pgfqpoint{4.594182in}{0.518583in}}%
\pgfpathlineto{\pgfqpoint{4.597404in}{0.517662in}}%
\pgfpathlineto{\pgfqpoint{4.598477in}{0.518071in}}%
\pgfpathlineto{\pgfqpoint{4.600625in}{0.521652in}}%
\pgfpathlineto{\pgfqpoint{4.601699in}{0.519879in}}%
\pgfpathlineto{\pgfqpoint{4.604920in}{0.522334in}}%
\pgfpathlineto{\pgfqpoint{4.605994in}{0.520220in}}%
\pgfpathlineto{\pgfqpoint{4.607068in}{0.521311in}}%
\pgfpathlineto{\pgfqpoint{4.609216in}{0.521993in}}%
\pgfpathlineto{\pgfqpoint{4.613511in}{0.525471in}}%
\pgfpathlineto{\pgfqpoint{4.614585in}{0.525369in}}%
\pgfpathlineto{\pgfqpoint{4.615659in}{0.530587in}}%
\pgfpathlineto{\pgfqpoint{4.616733in}{0.531849in}}%
\pgfpathlineto{\pgfqpoint{4.619954in}{0.528575in}}%
\pgfpathlineto{\pgfqpoint{4.621028in}{0.525744in}}%
\pgfpathlineto{\pgfqpoint{4.623176in}{0.528200in}}%
\pgfpathlineto{\pgfqpoint{4.624249in}{0.526153in}}%
\pgfpathlineto{\pgfqpoint{4.627471in}{0.523800in}}%
\pgfpathlineto{\pgfqpoint{4.630693in}{0.524244in}}%
\pgfpathlineto{\pgfqpoint{4.631766in}{0.525608in}}%
\pgfpathlineto{\pgfqpoint{4.634988in}{0.524448in}}%
\pgfpathlineto{\pgfqpoint{4.636062in}{0.523118in}}%
\pgfpathlineto{\pgfqpoint{4.638209in}{0.526767in}}%
\pgfpathlineto{\pgfqpoint{4.639283in}{0.530382in}}%
\pgfpathlineto{\pgfqpoint{4.642505in}{0.528848in}}%
\pgfpathlineto{\pgfqpoint{4.643579in}{0.532053in}}%
\pgfpathlineto{\pgfqpoint{4.644652in}{0.528506in}}%
\pgfpathlineto{\pgfqpoint{4.645726in}{0.527995in}}%
\pgfpathlineto{\pgfqpoint{4.646800in}{0.524823in}}%
\pgfpathlineto{\pgfqpoint{4.650022in}{0.522334in}}%
\pgfpathlineto{\pgfqpoint{4.652169in}{0.522402in}}%
\pgfpathlineto{\pgfqpoint{4.653243in}{0.517491in}}%
\pgfpathlineto{\pgfqpoint{4.654317in}{0.516843in}}%
\pgfpathlineto{\pgfqpoint{4.658612in}{0.515923in}}%
\pgfpathlineto{\pgfqpoint{4.659686in}{0.513604in}}%
\pgfpathlineto{\pgfqpoint{4.660760in}{0.515172in}}%
\pgfpathlineto{\pgfqpoint{4.661834in}{0.515445in}}%
\pgfpathlineto{\pgfqpoint{4.665055in}{0.514286in}}%
\pgfpathlineto{\pgfqpoint{4.666129in}{0.512205in}}%
\pgfpathlineto{\pgfqpoint{4.668277in}{0.513433in}}%
\pgfpathlineto{\pgfqpoint{4.669351in}{0.512649in}}%
\pgfpathlineto{\pgfqpoint{4.674720in}{0.513535in}}%
\pgfpathlineto{\pgfqpoint{4.676868in}{0.514388in}}%
\pgfpathlineto{\pgfqpoint{4.680089in}{0.515036in}}%
\pgfpathlineto{\pgfqpoint{4.684384in}{0.530928in}}%
\pgfpathlineto{\pgfqpoint{4.689754in}{0.532224in}}%
\pgfpathlineto{\pgfqpoint{4.691901in}{0.525676in}}%
\pgfpathlineto{\pgfqpoint{4.695123in}{0.524892in}}%
\pgfpathlineto{\pgfqpoint{4.696197in}{0.523528in}}%
\pgfpathlineto{\pgfqpoint{4.697271in}{0.523937in}}%
\pgfpathlineto{\pgfqpoint{4.698344in}{0.526392in}}%
\pgfpathlineto{\pgfqpoint{4.699418in}{0.526256in}}%
\pgfpathlineto{\pgfqpoint{4.702640in}{0.524073in}}%
\pgfpathlineto{\pgfqpoint{4.703714in}{0.525130in}}%
\pgfpathlineto{\pgfqpoint{4.704787in}{0.525301in}}%
\pgfpathlineto{\pgfqpoint{4.705861in}{0.522607in}}%
\pgfpathlineto{\pgfqpoint{4.706935in}{0.526597in}}%
\pgfpathlineto{\pgfqpoint{4.710157in}{0.524380in}}%
\pgfpathlineto{\pgfqpoint{4.712304in}{0.521277in}}%
\pgfpathlineto{\pgfqpoint{4.713378in}{0.527483in}}%
\pgfpathlineto{\pgfqpoint{4.714452in}{0.529427in}}%
\pgfpathlineto{\pgfqpoint{4.717673in}{0.531371in}}%
\pgfpathlineto{\pgfqpoint{4.718747in}{0.530143in}}%
\pgfpathlineto{\pgfqpoint{4.720895in}{0.530075in}}%
\pgfpathlineto{\pgfqpoint{4.721969in}{0.532360in}}%
\pgfpathlineto{\pgfqpoint{4.725190in}{0.533792in}}%
\pgfpathlineto{\pgfqpoint{4.726264in}{0.538123in}}%
\pgfpathlineto{\pgfqpoint{4.727338in}{0.535395in}}%
\pgfpathlineto{\pgfqpoint{4.728412in}{0.538294in}}%
\pgfpathlineto{\pgfqpoint{4.729486in}{0.538942in}}%
\pgfpathlineto{\pgfqpoint{4.733781in}{0.537885in}}%
\pgfpathlineto{\pgfqpoint{4.734855in}{0.536487in}}%
\pgfpathlineto{\pgfqpoint{4.735929in}{0.536862in}}%
\pgfpathlineto{\pgfqpoint{4.737003in}{0.538158in}}%
\pgfpathlineto{\pgfqpoint{4.741298in}{0.537680in}}%
\pgfpathlineto{\pgfqpoint{4.742372in}{0.538123in}}%
\pgfpathlineto{\pgfqpoint{4.743446in}{0.532360in}}%
\pgfpathlineto{\pgfqpoint{4.744519in}{0.536828in}}%
\pgfpathlineto{\pgfqpoint{4.747741in}{0.536077in}}%
\pgfpathlineto{\pgfqpoint{4.748815in}{0.534304in}}%
\pgfpathlineto{\pgfqpoint{4.750962in}{0.541329in}}%
\pgfpathlineto{\pgfqpoint{4.752036in}{0.541261in}}%
\pgfpathlineto{\pgfqpoint{4.755258in}{0.539692in}}%
\pgfpathlineto{\pgfqpoint{4.756332in}{0.538499in}}%
\pgfpathlineto{\pgfqpoint{4.757405in}{0.538840in}}%
\pgfpathlineto{\pgfqpoint{4.758479in}{0.541056in}}%
\pgfpathlineto{\pgfqpoint{4.759553in}{0.541875in}}%
\pgfpathlineto{\pgfqpoint{4.762775in}{0.540442in}}%
\pgfpathlineto{\pgfqpoint{4.763849in}{0.544944in}}%
\pgfpathlineto{\pgfqpoint{4.764922in}{0.543648in}}%
\pgfpathlineto{\pgfqpoint{4.767070in}{0.542932in}}%
\pgfpathlineto{\pgfqpoint{4.770292in}{0.544091in}}%
\pgfpathlineto{\pgfqpoint{4.771365in}{0.540954in}}%
\pgfpathlineto{\pgfqpoint{4.772439in}{0.541159in}}%
\pgfpathlineto{\pgfqpoint{4.773513in}{0.541977in}}%
\pgfpathlineto{\pgfqpoint{4.774587in}{0.545797in}}%
\pgfpathlineto{\pgfqpoint{4.777808in}{0.544705in}}%
\pgfpathlineto{\pgfqpoint{4.778882in}{0.546138in}}%
\pgfpathlineto{\pgfqpoint{4.779956in}{0.542864in}}%
\pgfpathlineto{\pgfqpoint{4.782104in}{0.542761in}}%
\pgfpathlineto{\pgfqpoint{4.786399in}{0.546035in}}%
\pgfpathlineto{\pgfqpoint{4.788547in}{0.553333in}}%
\pgfpathlineto{\pgfqpoint{4.789621in}{0.551526in}}%
\pgfpathlineto{\pgfqpoint{4.792842in}{0.554288in}}%
\pgfpathlineto{\pgfqpoint{4.793916in}{0.556096in}}%
\pgfpathlineto{\pgfqpoint{4.796064in}{0.558551in}}%
\pgfpathlineto{\pgfqpoint{4.797137in}{0.557698in}}%
\pgfpathlineto{\pgfqpoint{4.800359in}{0.563632in}}%
\pgfpathlineto{\pgfqpoint{4.801433in}{0.564144in}}%
\pgfpathlineto{\pgfqpoint{4.804654in}{0.563223in}}%
\pgfpathlineto{\pgfqpoint{4.807876in}{0.562336in}}%
\pgfpathlineto{\pgfqpoint{4.808950in}{0.563973in}}%
\pgfpathlineto{\pgfqpoint{4.810024in}{0.560427in}}%
\pgfpathlineto{\pgfqpoint{4.811097in}{0.559404in}}%
\pgfpathlineto{\pgfqpoint{4.812171in}{0.560972in}}%
\pgfpathlineto{\pgfqpoint{4.815393in}{0.554288in}}%
\pgfpathlineto{\pgfqpoint{4.816467in}{0.558005in}}%
\pgfpathlineto{\pgfqpoint{4.819688in}{0.556505in}}%
\pgfpathlineto{\pgfqpoint{4.823983in}{0.557221in}}%
\pgfpathlineto{\pgfqpoint{4.825057in}{0.560699in}}%
\pgfpathlineto{\pgfqpoint{4.826131in}{0.559063in}}%
\pgfpathlineto{\pgfqpoint{4.827205in}{0.552549in}}%
\pgfpathlineto{\pgfqpoint{4.830426in}{0.551048in}}%
\pgfpathlineto{\pgfqpoint{4.831500in}{0.553026in}}%
\pgfpathlineto{\pgfqpoint{4.832574in}{0.548150in}}%
\pgfpathlineto{\pgfqpoint{4.833648in}{0.554118in}}%
\pgfpathlineto{\pgfqpoint{4.834722in}{0.552174in}}%
\pgfpathlineto{\pgfqpoint{4.837943in}{0.543921in}}%
\pgfpathlineto{\pgfqpoint{4.840091in}{0.549514in}}%
\pgfpathlineto{\pgfqpoint{4.841165in}{0.561518in}}%
\pgfpathlineto{\pgfqpoint{4.842239in}{0.561859in}}%
\pgfpathlineto{\pgfqpoint{4.847608in}{0.568918in}}%
\pgfpathlineto{\pgfqpoint{4.848682in}{0.568884in}}%
\pgfpathlineto{\pgfqpoint{4.849756in}{0.573079in}}%
\pgfpathlineto{\pgfqpoint{4.854051in}{0.574204in}}%
\pgfpathlineto{\pgfqpoint{4.855125in}{0.571953in}}%
\pgfpathlineto{\pgfqpoint{4.856199in}{0.572976in}}%
\pgfpathlineto{\pgfqpoint{4.857272in}{0.575943in}}%
\pgfpathlineto{\pgfqpoint{4.860494in}{0.577819in}}%
\pgfpathlineto{\pgfqpoint{4.861568in}{0.576182in}}%
\pgfpathlineto{\pgfqpoint{4.862642in}{0.575773in}}%
\pgfpathlineto{\pgfqpoint{4.864789in}{0.578092in}}%
\pgfpathlineto{\pgfqpoint{4.868011in}{0.573829in}}%
\pgfpathlineto{\pgfqpoint{4.869085in}{0.580922in}}%
\pgfpathlineto{\pgfqpoint{4.871232in}{0.586072in}}%
\pgfpathlineto{\pgfqpoint{4.872306in}{0.582491in}}%
\pgfpathlineto{\pgfqpoint{4.875528in}{0.581741in}}%
\pgfpathlineto{\pgfqpoint{4.876602in}{0.578842in}}%
\pgfpathlineto{\pgfqpoint{4.877675in}{0.579865in}}%
\pgfpathlineto{\pgfqpoint{4.878749in}{0.575295in}}%
\pgfpathlineto{\pgfqpoint{4.879823in}{0.575739in}}%
\pgfpathlineto{\pgfqpoint{4.884118in}{0.581809in}}%
\pgfpathlineto{\pgfqpoint{4.885192in}{0.577171in}}%
\pgfpathlineto{\pgfqpoint{4.886266in}{0.578092in}}%
\pgfpathlineto{\pgfqpoint{4.887340in}{0.576625in}}%
\pgfpathlineto{\pgfqpoint{4.890561in}{0.573897in}}%
\pgfpathlineto{\pgfqpoint{4.891635in}{0.574102in}}%
\pgfpathlineto{\pgfqpoint{4.892709in}{0.571169in}}%
\pgfpathlineto{\pgfqpoint{4.893783in}{0.570726in}}%
\pgfpathlineto{\pgfqpoint{4.894857in}{0.572328in}}%
\pgfpathlineto{\pgfqpoint{4.898078in}{0.575739in}}%
\pgfpathlineto{\pgfqpoint{4.899152in}{0.580479in}}%
\pgfpathlineto{\pgfqpoint{4.901300in}{0.580854in}}%
\pgfpathlineto{\pgfqpoint{4.902374in}{0.576864in}}%
\pgfpathlineto{\pgfqpoint{4.905595in}{0.572772in}}%
\pgfpathlineto{\pgfqpoint{4.906669in}{0.574170in}}%
\pgfpathlineto{\pgfqpoint{4.907743in}{0.577001in}}%
\pgfpathlineto{\pgfqpoint{4.908817in}{0.567656in}}%
\pgfpathlineto{\pgfqpoint{4.909891in}{0.566088in}}%
\pgfpathlineto{\pgfqpoint{4.914186in}{0.568134in}}%
\pgfpathlineto{\pgfqpoint{4.916334in}{0.575500in}}%
\pgfpathlineto{\pgfqpoint{4.920629in}{0.572704in}}%
\pgfpathlineto{\pgfqpoint{4.921703in}{0.573795in}}%
\pgfpathlineto{\pgfqpoint{4.923850in}{0.574238in}}%
\pgfpathlineto{\pgfqpoint{4.924924in}{0.569941in}}%
\pgfpathlineto{\pgfqpoint{4.928146in}{0.568509in}}%
\pgfpathlineto{\pgfqpoint{4.930293in}{0.573454in}}%
\pgfpathlineto{\pgfqpoint{4.931367in}{0.574136in}}%
\pgfpathlineto{\pgfqpoint{4.932441in}{0.576489in}}%
\pgfpathlineto{\pgfqpoint{4.935663in}{0.579388in}}%
\pgfpathlineto{\pgfqpoint{4.936737in}{0.578774in}}%
\pgfpathlineto{\pgfqpoint{4.937810in}{0.576421in}}%
\pgfpathlineto{\pgfqpoint{4.938884in}{0.580581in}}%
\pgfpathlineto{\pgfqpoint{4.939958in}{0.581673in}}%
\pgfpathlineto{\pgfqpoint{4.943180in}{0.583173in}}%
\pgfpathlineto{\pgfqpoint{4.944253in}{0.582082in}}%
\pgfpathlineto{\pgfqpoint{4.946401in}{0.576012in}}%
\pgfpathlineto{\pgfqpoint{4.950696in}{0.578842in}}%
\pgfpathlineto{\pgfqpoint{4.951770in}{0.578978in}}%
\pgfpathlineto{\pgfqpoint{4.952844in}{0.582423in}}%
\pgfpathlineto{\pgfqpoint{4.953918in}{0.582934in}}%
\pgfpathlineto{\pgfqpoint{4.954992in}{0.586004in}}%
\pgfpathlineto{\pgfqpoint{4.959287in}{0.587334in}}%
\pgfpathlineto{\pgfqpoint{4.960361in}{0.586890in}}%
\pgfpathlineto{\pgfqpoint{4.961435in}{0.588254in}}%
\pgfpathlineto{\pgfqpoint{4.962509in}{0.588186in}}%
\pgfpathlineto{\pgfqpoint{4.965730in}{0.589005in}}%
\pgfpathlineto{\pgfqpoint{4.966804in}{0.587777in}}%
\pgfpathlineto{\pgfqpoint{4.968952in}{0.590607in}}%
\pgfpathlineto{\pgfqpoint{4.970026in}{0.590062in}}%
\pgfpathlineto{\pgfqpoint{4.973247in}{0.592824in}}%
\pgfpathlineto{\pgfqpoint{4.975395in}{0.589141in}}%
\pgfpathlineto{\pgfqpoint{4.976469in}{0.584367in}}%
\pgfpathlineto{\pgfqpoint{4.977542in}{0.584367in}}%
\pgfpathlineto{\pgfqpoint{4.980764in}{0.585935in}}%
\pgfpathlineto{\pgfqpoint{4.981838in}{0.585492in}}%
\pgfpathlineto{\pgfqpoint{4.983985in}{0.587436in}}%
\pgfpathlineto{\pgfqpoint{4.985059in}{0.584435in}}%
\pgfpathlineto{\pgfqpoint{4.989355in}{0.583992in}}%
\pgfpathlineto{\pgfqpoint{4.990428in}{0.587845in}}%
\pgfpathlineto{\pgfqpoint{4.992576in}{0.592244in}}%
\pgfpathlineto{\pgfqpoint{4.995798in}{0.593643in}}%
\pgfpathlineto{\pgfqpoint{4.996871in}{0.595962in}}%
\pgfpathlineto{\pgfqpoint{4.997945in}{0.595859in}}%
\pgfpathlineto{\pgfqpoint{4.999019in}{0.597121in}}%
\pgfpathlineto{\pgfqpoint{5.003314in}{0.596337in}}%
\pgfpathlineto{\pgfqpoint{5.004388in}{0.594018in}}%
\pgfpathlineto{\pgfqpoint{5.005462in}{0.598315in}}%
\pgfpathlineto{\pgfqpoint{5.006536in}{0.597189in}}%
\pgfpathlineto{\pgfqpoint{5.007610in}{0.597292in}}%
\pgfpathlineto{\pgfqpoint{5.010831in}{0.596882in}}%
\pgfpathlineto{\pgfqpoint{5.014053in}{0.591392in}}%
\pgfpathlineto{\pgfqpoint{5.015127in}{0.593881in}}%
\pgfpathlineto{\pgfqpoint{5.018348in}{0.593643in}}%
\pgfpathlineto{\pgfqpoint{5.019422in}{0.595143in}}%
\pgfpathlineto{\pgfqpoint{5.020496in}{0.594597in}}%
\pgfpathlineto{\pgfqpoint{5.021570in}{0.596985in}}%
\pgfpathlineto{\pgfqpoint{5.022644in}{0.595177in}}%
\pgfpathlineto{\pgfqpoint{5.025865in}{0.598110in}}%
\pgfpathlineto{\pgfqpoint{5.026939in}{0.595109in}}%
\pgfpathlineto{\pgfqpoint{5.028013in}{0.598281in}}%
\pgfpathlineto{\pgfqpoint{5.029087in}{0.594086in}}%
\pgfpathlineto{\pgfqpoint{5.030160in}{0.592517in}}%
\pgfpathlineto{\pgfqpoint{5.033382in}{0.598383in}}%
\pgfpathlineto{\pgfqpoint{5.034456in}{0.596848in}}%
\pgfpathlineto{\pgfqpoint{5.035530in}{0.596337in}}%
\pgfpathlineto{\pgfqpoint{5.036603in}{0.592926in}}%
\pgfpathlineto{\pgfqpoint{5.037677in}{0.597155in}}%
\pgfpathlineto{\pgfqpoint{5.040899in}{0.599201in}}%
\pgfpathlineto{\pgfqpoint{5.041973in}{0.598281in}}%
\pgfpathlineto{\pgfqpoint{5.043047in}{0.599508in}}%
\pgfpathlineto{\pgfqpoint{5.045194in}{0.604828in}}%
\pgfpathlineto{\pgfqpoint{5.049490in}{0.607386in}}%
\pgfpathlineto{\pgfqpoint{5.050563in}{0.606602in}}%
\pgfpathlineto{\pgfqpoint{5.051637in}{0.608307in}}%
\pgfpathlineto{\pgfqpoint{5.052711in}{0.608648in}}%
\pgfpathlineto{\pgfqpoint{5.055933in}{0.608307in}}%
\pgfpathlineto{\pgfqpoint{5.057006in}{0.607147in}}%
\pgfpathlineto{\pgfqpoint{5.058080in}{0.608273in}}%
\pgfpathlineto{\pgfqpoint{5.059154in}{0.607829in}}%
\pgfpathlineto{\pgfqpoint{5.060228in}{0.606704in}}%
\pgfpathlineto{\pgfqpoint{5.065597in}{0.610046in}}%
\pgfpathlineto{\pgfqpoint{5.066671in}{0.607522in}}%
\pgfpathlineto{\pgfqpoint{5.067745in}{0.610046in}}%
\pgfpathlineto{\pgfqpoint{5.070966in}{0.608852in}}%
\pgfpathlineto{\pgfqpoint{5.072040in}{0.605817in}}%
\pgfpathlineto{\pgfqpoint{5.073114in}{0.605579in}}%
\pgfpathlineto{\pgfqpoint{5.074188in}{0.607011in}}%
\pgfpathlineto{\pgfqpoint{5.075262in}{0.606056in}}%
\pgfpathlineto{\pgfqpoint{5.082779in}{0.608648in}}%
\pgfpathlineto{\pgfqpoint{5.086000in}{0.604862in}}%
\pgfpathlineto{\pgfqpoint{5.087074in}{0.601759in}}%
\pgfpathlineto{\pgfqpoint{5.088148in}{0.605374in}}%
\pgfpathlineto{\pgfqpoint{5.089222in}{0.601009in}}%
\pgfpathlineto{\pgfqpoint{5.090295in}{0.603123in}}%
\pgfpathlineto{\pgfqpoint{5.094591in}{0.604044in}}%
\pgfpathlineto{\pgfqpoint{5.095665in}{0.599167in}}%
\pgfpathlineto{\pgfqpoint{5.096738in}{0.596951in}}%
\pgfpathlineto{\pgfqpoint{5.097812in}{0.602202in}}%
\pgfpathlineto{\pgfqpoint{5.101034in}{0.602543in}}%
\pgfpathlineto{\pgfqpoint{5.102108in}{0.597974in}}%
\pgfpathlineto{\pgfqpoint{5.103181in}{0.601111in}}%
\pgfpathlineto{\pgfqpoint{5.104255in}{0.593540in}}%
\pgfpathlineto{\pgfqpoint{5.105329in}{0.595314in}}%
\pgfpathlineto{\pgfqpoint{5.108551in}{0.587879in}}%
\pgfpathlineto{\pgfqpoint{5.109625in}{0.588595in}}%
\pgfpathlineto{\pgfqpoint{5.110698in}{0.582082in}}%
\pgfpathlineto{\pgfqpoint{5.111772in}{0.581025in}}%
\pgfpathlineto{\pgfqpoint{5.112846in}{0.587606in}}%
\pgfpathlineto{\pgfqpoint{5.116068in}{0.592961in}}%
\pgfpathlineto{\pgfqpoint{5.117141in}{0.599338in}}%
\pgfpathlineto{\pgfqpoint{5.118215in}{0.597940in}}%
\pgfpathlineto{\pgfqpoint{5.120363in}{0.602714in}}%
\pgfpathlineto{\pgfqpoint{5.123584in}{0.602202in}}%
\pgfpathlineto{\pgfqpoint{5.124658in}{0.606874in}}%
\pgfpathlineto{\pgfqpoint{5.125732in}{0.605613in}}%
\pgfpathlineto{\pgfqpoint{5.126806in}{0.607795in}}%
\pgfpathlineto{\pgfqpoint{5.127880in}{0.611444in}}%
\pgfpathlineto{\pgfqpoint{5.131101in}{0.612501in}}%
\pgfpathlineto{\pgfqpoint{5.132175in}{0.608204in}}%
\pgfpathlineto{\pgfqpoint{5.134323in}{0.613422in}}%
\pgfpathlineto{\pgfqpoint{5.135397in}{0.607079in}}%
\pgfpathlineto{\pgfqpoint{5.138618in}{0.606397in}}%
\pgfpathlineto{\pgfqpoint{5.139692in}{0.607045in}}%
\pgfpathlineto{\pgfqpoint{5.140766in}{0.606772in}}%
\pgfpathlineto{\pgfqpoint{5.142914in}{0.609637in}}%
\pgfpathlineto{\pgfqpoint{5.147209in}{0.607966in}}%
\pgfpathlineto{\pgfqpoint{5.148283in}{0.606533in}}%
\pgfpathlineto{\pgfqpoint{5.149357in}{0.603635in}}%
\pgfpathlineto{\pgfqpoint{5.150430in}{0.603805in}}%
\pgfpathlineto{\pgfqpoint{5.153652in}{0.608921in}}%
\pgfpathlineto{\pgfqpoint{5.154726in}{0.612297in}}%
\pgfpathlineto{\pgfqpoint{5.157947in}{0.615025in}}%
\pgfpathlineto{\pgfqpoint{5.161169in}{0.615605in}}%
\pgfpathlineto{\pgfqpoint{5.162243in}{0.618060in}}%
\pgfpathlineto{\pgfqpoint{5.163316in}{0.616901in}}%
\pgfpathlineto{\pgfqpoint{5.164390in}{0.617310in}}%
\pgfpathlineto{\pgfqpoint{5.165464in}{0.618981in}}%
\pgfpathlineto{\pgfqpoint{5.168686in}{0.619083in}}%
\pgfpathlineto{\pgfqpoint{5.170833in}{0.612229in}}%
\pgfpathlineto{\pgfqpoint{5.171907in}{0.616287in}}%
\pgfpathlineto{\pgfqpoint{5.176202in}{0.613593in}}%
\pgfpathlineto{\pgfqpoint{5.177276in}{0.611240in}}%
\pgfpathlineto{\pgfqpoint{5.179424in}{0.619049in}}%
\pgfpathlineto{\pgfqpoint{5.180498in}{0.619936in}}%
\pgfpathlineto{\pgfqpoint{5.184793in}{0.625699in}}%
\pgfpathlineto{\pgfqpoint{5.185867in}{0.624847in}}%
\pgfpathlineto{\pgfqpoint{5.188015in}{0.626790in}}%
\pgfpathlineto{\pgfqpoint{5.191236in}{0.628291in}}%
\pgfpathlineto{\pgfqpoint{5.193384in}{0.624096in}}%
\pgfpathlineto{\pgfqpoint{5.195532in}{0.622698in}}%
\pgfpathlineto{\pgfqpoint{5.199827in}{0.616764in}}%
\pgfpathlineto{\pgfqpoint{5.203048in}{0.624301in}}%
\pgfpathlineto{\pgfqpoint{5.206270in}{0.624983in}}%
\pgfpathlineto{\pgfqpoint{5.207344in}{0.627302in}}%
\pgfpathlineto{\pgfqpoint{5.208418in}{0.624233in}}%
\pgfpathlineto{\pgfqpoint{5.209491in}{0.624608in}}%
\pgfpathlineto{\pgfqpoint{5.210565in}{0.627268in}}%
\pgfpathlineto{\pgfqpoint{5.214861in}{0.625870in}}%
\pgfpathlineto{\pgfqpoint{5.215935in}{0.624028in}}%
\pgfpathlineto{\pgfqpoint{5.217008in}{0.627166in}}%
\pgfpathlineto{\pgfqpoint{5.218082in}{0.625801in}}%
\pgfpathlineto{\pgfqpoint{5.221304in}{0.626586in}}%
\pgfpathlineto{\pgfqpoint{5.223451in}{0.619254in}}%
\pgfpathlineto{\pgfqpoint{5.224525in}{0.620993in}}%
\pgfpathlineto{\pgfqpoint{5.225599in}{0.613763in}}%
\pgfpathlineto{\pgfqpoint{5.228821in}{0.616867in}}%
\pgfpathlineto{\pgfqpoint{5.229894in}{0.623823in}}%
\pgfpathlineto{\pgfqpoint{5.230968in}{0.646775in}}%
\pgfpathlineto{\pgfqpoint{5.232042in}{0.651106in}}%
\pgfpathlineto{\pgfqpoint{5.233116in}{0.649128in}}%
\pgfpathlineto{\pgfqpoint{5.236337in}{0.648207in}}%
\pgfpathlineto{\pgfqpoint{5.237411in}{0.648821in}}%
\pgfpathlineto{\pgfqpoint{5.238485in}{0.648650in}}%
\pgfpathlineto{\pgfqpoint{5.239559in}{0.654141in}}%
\pgfpathlineto{\pgfqpoint{5.240633in}{0.656016in}}%
\pgfpathlineto{\pgfqpoint{5.244928in}{0.655846in}}%
\pgfpathlineto{\pgfqpoint{5.246002in}{0.655027in}}%
\pgfpathlineto{\pgfqpoint{5.247076in}{0.655130in}}%
\pgfpathlineto{\pgfqpoint{5.248150in}{0.657244in}}%
\pgfpathlineto{\pgfqpoint{5.251371in}{0.658642in}}%
\pgfpathlineto{\pgfqpoint{5.252445in}{0.657619in}}%
\pgfpathlineto{\pgfqpoint{5.253519in}{0.660484in}}%
\pgfpathlineto{\pgfqpoint{5.255667in}{0.655744in}}%
\pgfpathlineto{\pgfqpoint{5.259962in}{0.662973in}}%
\pgfpathlineto{\pgfqpoint{5.263183in}{0.654891in}}%
\pgfpathlineto{\pgfqpoint{5.266405in}{0.659495in}}%
\pgfpathlineto{\pgfqpoint{5.267479in}{0.652572in}}%
\pgfpathlineto{\pgfqpoint{5.268553in}{0.651924in}}%
\pgfpathlineto{\pgfqpoint{5.269626in}{0.665599in}}%
\pgfpathlineto{\pgfqpoint{5.270700in}{0.663280in}}%
\pgfpathlineto{\pgfqpoint{5.273922in}{0.666247in}}%
\pgfpathlineto{\pgfqpoint{5.274996in}{0.664951in}}%
\pgfpathlineto{\pgfqpoint{5.276069in}{0.668157in}}%
\pgfpathlineto{\pgfqpoint{5.277143in}{0.666247in}}%
\pgfpathlineto{\pgfqpoint{5.278217in}{0.669623in}}%
\pgfpathlineto{\pgfqpoint{5.281439in}{0.668975in}}%
\pgfpathlineto{\pgfqpoint{5.282513in}{0.665429in}}%
\pgfpathlineto{\pgfqpoint{5.283586in}{0.658676in}}%
\pgfpathlineto{\pgfqpoint{5.288956in}{0.662257in}}%
\pgfpathlineto{\pgfqpoint{5.290029in}{0.658301in}}%
\pgfpathlineto{\pgfqpoint{5.292177in}{0.661882in}}%
\pgfpathlineto{\pgfqpoint{5.297546in}{0.660041in}}%
\pgfpathlineto{\pgfqpoint{5.298620in}{0.663007in}}%
\pgfpathlineto{\pgfqpoint{5.300768in}{0.664917in}}%
\pgfpathlineto{\pgfqpoint{5.303989in}{0.663485in}}%
\pgfpathlineto{\pgfqpoint{5.306137in}{0.664985in}}%
\pgfpathlineto{\pgfqpoint{5.307211in}{0.668566in}}%
\pgfpathlineto{\pgfqpoint{5.308285in}{0.664065in}}%
\pgfpathlineto{\pgfqpoint{5.311506in}{0.668975in}}%
\pgfpathlineto{\pgfqpoint{5.312580in}{0.667236in}}%
\pgfpathlineto{\pgfqpoint{5.313654in}{0.668055in}}%
\pgfpathlineto{\pgfqpoint{5.315802in}{0.673170in}}%
\pgfpathlineto{\pgfqpoint{5.319023in}{0.675182in}}%
\pgfpathlineto{\pgfqpoint{5.321171in}{0.674057in}}%
\pgfpathlineto{\pgfqpoint{5.322245in}{0.670578in}}%
\pgfpathlineto{\pgfqpoint{5.323318in}{0.676308in}}%
\pgfpathlineto{\pgfqpoint{5.326540in}{0.677944in}}%
\pgfpathlineto{\pgfqpoint{5.327614in}{0.677262in}}%
\pgfpathlineto{\pgfqpoint{5.328688in}{0.673750in}}%
\pgfpathlineto{\pgfqpoint{5.329761in}{0.672283in}}%
\pgfpathlineto{\pgfqpoint{5.330835in}{0.675012in}}%
\pgfpathlineto{\pgfqpoint{5.334057in}{0.670169in}}%
\pgfpathlineto{\pgfqpoint{5.335131in}{0.672215in}}%
\pgfpathlineto{\pgfqpoint{5.336204in}{0.672079in}}%
\pgfpathlineto{\pgfqpoint{5.338352in}{0.675625in}}%
\pgfpathlineto{\pgfqpoint{5.341574in}{0.675728in}}%
\pgfpathlineto{\pgfqpoint{5.342647in}{0.676444in}}%
\pgfpathlineto{\pgfqpoint{5.343721in}{0.675284in}}%
\pgfpathlineto{\pgfqpoint{5.344795in}{0.675898in}}%
\pgfpathlineto{\pgfqpoint{5.345869in}{0.675489in}}%
\pgfpathlineto{\pgfqpoint{5.350164in}{0.672863in}}%
\pgfpathlineto{\pgfqpoint{5.351238in}{0.675830in}}%
\pgfpathlineto{\pgfqpoint{5.352312in}{0.676342in}}%
\pgfpathlineto{\pgfqpoint{5.353386in}{0.675830in}}%
\pgfpathlineto{\pgfqpoint{5.356607in}{0.677740in}}%
\pgfpathlineto{\pgfqpoint{5.357681in}{0.677058in}}%
\pgfpathlineto{\pgfqpoint{5.358755in}{0.678388in}}%
\pgfpathlineto{\pgfqpoint{5.359829in}{0.675625in}}%
\pgfpathlineto{\pgfqpoint{5.360903in}{0.675625in}}%
\pgfpathlineto{\pgfqpoint{5.364124in}{0.672386in}}%
\pgfpathlineto{\pgfqpoint{5.365198in}{0.669930in}}%
\pgfpathlineto{\pgfqpoint{5.366272in}{0.674671in}}%
\pgfpathlineto{\pgfqpoint{5.367346in}{0.676649in}}%
\pgfpathlineto{\pgfqpoint{5.368420in}{0.674500in}}%
\pgfpathlineto{\pgfqpoint{5.371641in}{0.675216in}}%
\pgfpathlineto{\pgfqpoint{5.374863in}{0.684936in}}%
\pgfpathlineto{\pgfqpoint{5.375936in}{0.683026in}}%
\pgfpathlineto{\pgfqpoint{5.379158in}{0.685959in}}%
\pgfpathlineto{\pgfqpoint{5.380232in}{0.688755in}}%
\pgfpathlineto{\pgfqpoint{5.381306in}{0.686709in}}%
\pgfpathlineto{\pgfqpoint{5.383453in}{0.690597in}}%
\pgfpathlineto{\pgfqpoint{5.386675in}{0.684424in}}%
\pgfpathlineto{\pgfqpoint{5.388823in}{0.693188in}}%
\pgfpathlineto{\pgfqpoint{5.389896in}{0.692677in}}%
\pgfpathlineto{\pgfqpoint{5.394192in}{0.695030in}}%
\pgfpathlineto{\pgfqpoint{5.395266in}{0.699531in}}%
\pgfpathlineto{\pgfqpoint{5.396339in}{0.693393in}}%
\pgfpathlineto{\pgfqpoint{5.397413in}{0.694723in}}%
\pgfpathlineto{\pgfqpoint{5.398487in}{0.697417in}}%
\pgfpathlineto{\pgfqpoint{5.401709in}{0.702601in}}%
\pgfpathlineto{\pgfqpoint{5.402782in}{0.701953in}}%
\pgfpathlineto{\pgfqpoint{5.403856in}{0.703385in}}%
\pgfpathlineto{\pgfqpoint{5.404930in}{0.705875in}}%
\pgfpathlineto{\pgfqpoint{5.406004in}{0.705192in}}%
\pgfpathlineto{\pgfqpoint{5.409225in}{0.707511in}}%
\pgfpathlineto{\pgfqpoint{5.410299in}{0.706659in}}%
\pgfpathlineto{\pgfqpoint{5.411373in}{0.706693in}}%
\pgfpathlineto{\pgfqpoint{5.412447in}{0.704988in}}%
\pgfpathlineto{\pgfqpoint{5.413521in}{0.705363in}}%
\pgfpathlineto{\pgfqpoint{5.416742in}{0.703249in}}%
\pgfpathlineto{\pgfqpoint{5.417816in}{0.703897in}}%
\pgfpathlineto{\pgfqpoint{5.418890in}{0.708364in}}%
\pgfpathlineto{\pgfqpoint{5.419964in}{0.708944in}}%
\pgfpathlineto{\pgfqpoint{5.421038in}{0.708876in}}%
\pgfpathlineto{\pgfqpoint{5.425333in}{0.714298in}}%
\pgfpathlineto{\pgfqpoint{5.426407in}{0.678422in}}%
\pgfpathlineto{\pgfqpoint{5.427481in}{0.672045in}}%
\pgfpathlineto{\pgfqpoint{5.428555in}{0.674602in}}%
\pgfpathlineto{\pgfqpoint{5.431776in}{0.679922in}}%
\pgfpathlineto{\pgfqpoint{5.432850in}{0.670271in}}%
\pgfpathlineto{\pgfqpoint{5.433924in}{0.667032in}}%
\pgfpathlineto{\pgfqpoint{5.434998in}{0.668737in}}%
\pgfpathlineto{\pgfqpoint{5.436071in}{0.667577in}}%
\pgfpathlineto{\pgfqpoint{5.439293in}{0.673648in}}%
\pgfpathlineto{\pgfqpoint{5.440367in}{0.666861in}}%
\pgfpathlineto{\pgfqpoint{5.441441in}{0.665292in}}%
\pgfpathlineto{\pgfqpoint{5.442514in}{0.644592in}}%
\pgfpathlineto{\pgfqpoint{5.446810in}{0.629621in}}%
\pgfpathlineto{\pgfqpoint{5.447884in}{0.631326in}}%
\pgfpathlineto{\pgfqpoint{5.450031in}{0.651515in}}%
\pgfpathlineto{\pgfqpoint{5.451105in}{0.652504in}}%
\pgfpathlineto{\pgfqpoint{5.454327in}{0.650594in}}%
\pgfpathlineto{\pgfqpoint{5.455401in}{0.642955in}}%
\pgfpathlineto{\pgfqpoint{5.456474in}{0.650628in}}%
\pgfpathlineto{\pgfqpoint{5.457548in}{0.650935in}}%
\pgfpathlineto{\pgfqpoint{5.458622in}{0.647661in}}%
\pgfpathlineto{\pgfqpoint{5.462917in}{0.657449in}}%
\pgfpathlineto{\pgfqpoint{5.463991in}{0.650696in}}%
\pgfpathlineto{\pgfqpoint{5.465065in}{0.652913in}}%
\pgfpathlineto{\pgfqpoint{5.466139in}{0.658949in}}%
\pgfpathlineto{\pgfqpoint{5.470434in}{0.655573in}}%
\pgfpathlineto{\pgfqpoint{5.472582in}{0.658063in}}%
\pgfpathlineto{\pgfqpoint{5.473656in}{0.653663in}}%
\pgfpathlineto{\pgfqpoint{5.476877in}{0.655505in}}%
\pgfpathlineto{\pgfqpoint{5.480099in}{0.646536in}}%
\pgfpathlineto{\pgfqpoint{5.481173in}{0.645513in}}%
\pgfpathlineto{\pgfqpoint{5.484394in}{0.639681in}}%
\pgfpathlineto{\pgfqpoint{5.485468in}{0.642682in}}%
\pgfpathlineto{\pgfqpoint{5.486542in}{0.651617in}}%
\pgfpathlineto{\pgfqpoint{5.488690in}{0.654175in}}%
\pgfpathlineto{\pgfqpoint{5.491911in}{0.656937in}}%
\pgfpathlineto{\pgfqpoint{5.492985in}{0.656664in}}%
\pgfpathlineto{\pgfqpoint{5.494059in}{0.655437in}}%
\pgfpathlineto{\pgfqpoint{5.496206in}{0.662428in}}%
\pgfpathlineto{\pgfqpoint{5.500502in}{0.665736in}}%
\pgfpathlineto{\pgfqpoint{5.501576in}{0.662973in}}%
\pgfpathlineto{\pgfqpoint{5.502649in}{0.669930in}}%
\pgfpathlineto{\pgfqpoint{5.509092in}{0.676990in}}%
\pgfpathlineto{\pgfqpoint{5.510166in}{0.687152in}}%
\pgfpathlineto{\pgfqpoint{5.511240in}{0.686641in}}%
\pgfpathlineto{\pgfqpoint{5.515535in}{0.688823in}}%
\pgfpathlineto{\pgfqpoint{5.517683in}{0.692916in}}%
\pgfpathlineto{\pgfqpoint{5.518757in}{0.688721in}}%
\pgfpathlineto{\pgfqpoint{5.523052in}{0.694518in}}%
\pgfpathlineto{\pgfqpoint{5.524126in}{0.687152in}}%
\pgfpathlineto{\pgfqpoint{5.525200in}{0.686334in}}%
\pgfpathlineto{\pgfqpoint{5.526274in}{0.694928in}}%
\pgfpathlineto{\pgfqpoint{5.529495in}{0.697349in}}%
\pgfpathlineto{\pgfqpoint{5.530569in}{0.700555in}}%
\pgfpathlineto{\pgfqpoint{5.531643in}{0.697656in}}%
\pgfpathlineto{\pgfqpoint{5.532717in}{0.696667in}}%
\pgfpathlineto{\pgfqpoint{5.533791in}{0.692268in}}%
\pgfpathlineto{\pgfqpoint{5.538086in}{0.696428in}}%
\pgfpathlineto{\pgfqpoint{5.539160in}{0.702874in}}%
\pgfpathlineto{\pgfqpoint{5.540234in}{0.704715in}}%
\pgfpathlineto{\pgfqpoint{5.541308in}{0.709080in}}%
\pgfpathlineto{\pgfqpoint{5.544529in}{0.707000in}}%
\pgfpathlineto{\pgfqpoint{5.545603in}{0.702260in}}%
\pgfpathlineto{\pgfqpoint{5.546677in}{0.704579in}}%
\pgfpathlineto{\pgfqpoint{5.548824in}{0.693188in}}%
\pgfpathlineto{\pgfqpoint{5.552046in}{0.687868in}}%
\pgfpathlineto{\pgfqpoint{5.553120in}{0.694041in}}%
\pgfpathlineto{\pgfqpoint{5.555267in}{0.682787in}}%
\pgfpathlineto{\pgfqpoint{5.556341in}{0.690324in}}%
\pgfpathlineto{\pgfqpoint{5.559563in}{0.689028in}}%
\pgfpathlineto{\pgfqpoint{5.561711in}{0.681423in}}%
\pgfpathlineto{\pgfqpoint{5.562784in}{0.681423in}}%
\pgfpathlineto{\pgfqpoint{5.563858in}{0.672624in}}%
\pgfpathlineto{\pgfqpoint{5.567080in}{0.676853in}}%
\pgfpathlineto{\pgfqpoint{5.569227in}{0.691244in}}%
\pgfpathlineto{\pgfqpoint{5.570301in}{0.685481in}}%
\pgfpathlineto{\pgfqpoint{5.571375in}{0.671601in}}%
\pgfpathlineto{\pgfqpoint{5.574597in}{0.667918in}}%
\pgfpathlineto{\pgfqpoint{5.575670in}{0.668430in}}%
\pgfpathlineto{\pgfqpoint{5.576744in}{0.664610in}}%
\pgfpathlineto{\pgfqpoint{5.582113in}{0.670067in}}%
\pgfpathlineto{\pgfqpoint{5.583187in}{0.669521in}}%
\pgfpathlineto{\pgfqpoint{5.584261in}{0.667134in}}%
\pgfpathlineto{\pgfqpoint{5.585335in}{0.663042in}}%
\pgfpathlineto{\pgfqpoint{5.589630in}{0.656255in}}%
\pgfpathlineto{\pgfqpoint{5.590704in}{0.649503in}}%
\pgfpathlineto{\pgfqpoint{5.593926in}{0.644183in}}%
\pgfpathlineto{\pgfqpoint{5.597147in}{0.646331in}}%
\pgfpathlineto{\pgfqpoint{5.598221in}{0.651344in}}%
\pgfpathlineto{\pgfqpoint{5.599295in}{0.641693in}}%
\pgfpathlineto{\pgfqpoint{5.600369in}{0.643705in}}%
\pgfpathlineto{\pgfqpoint{5.601443in}{0.626859in}}%
\pgfpathlineto{\pgfqpoint{5.605738in}{0.627097in}}%
\pgfpathlineto{\pgfqpoint{5.606812in}{0.622459in}}%
\pgfpathlineto{\pgfqpoint{5.607886in}{0.627234in}}%
\pgfpathlineto{\pgfqpoint{5.608959in}{0.636578in}}%
\pgfpathlineto{\pgfqpoint{5.612181in}{0.631360in}}%
\pgfpathlineto{\pgfqpoint{5.613255in}{0.634532in}}%
\pgfpathlineto{\pgfqpoint{5.614329in}{0.628223in}}%
\pgfpathlineto{\pgfqpoint{5.615402in}{0.625665in}}%
\pgfpathlineto{\pgfqpoint{5.616476in}{0.633065in}}%
\pgfpathlineto{\pgfqpoint{5.619698in}{0.630917in}}%
\pgfpathlineto{\pgfqpoint{5.620772in}{0.624335in}}%
\pgfpathlineto{\pgfqpoint{5.621845in}{0.630883in}}%
\pgfpathlineto{\pgfqpoint{5.622919in}{0.631804in}}%
\pgfpathlineto{\pgfqpoint{5.623993in}{0.626859in}}%
\pgfpathlineto{\pgfqpoint{5.627215in}{0.621095in}}%
\pgfpathlineto{\pgfqpoint{5.628289in}{0.621743in}}%
\pgfpathlineto{\pgfqpoint{5.629362in}{0.610523in}}%
\pgfpathlineto{\pgfqpoint{5.631510in}{0.617958in}}%
\pgfpathlineto{\pgfqpoint{5.635805in}{0.623653in}}%
\pgfpathlineto{\pgfqpoint{5.636879in}{0.632042in}}%
\pgfpathlineto{\pgfqpoint{5.639027in}{0.630439in}}%
\pgfpathlineto{\pgfqpoint{5.642248in}{0.634839in}}%
\pgfpathlineto{\pgfqpoint{5.643322in}{0.631633in}}%
\pgfpathlineto{\pgfqpoint{5.645470in}{0.632520in}}%
\pgfpathlineto{\pgfqpoint{5.646544in}{0.631428in}}%
\pgfpathlineto{\pgfqpoint{5.649765in}{0.632110in}}%
\pgfpathlineto{\pgfqpoint{5.650839in}{0.638999in}}%
\pgfpathlineto{\pgfqpoint{5.651913in}{0.636885in}}%
\pgfpathlineto{\pgfqpoint{5.652987in}{0.642785in}}%
\pgfpathlineto{\pgfqpoint{5.654061in}{0.641693in}}%
\pgfpathlineto{\pgfqpoint{5.657282in}{0.644626in}}%
\pgfpathlineto{\pgfqpoint{5.658356in}{0.639545in}}%
\pgfpathlineto{\pgfqpoint{5.659430in}{0.639033in}}%
\pgfpathlineto{\pgfqpoint{5.660504in}{0.637021in}}%
\pgfpathlineto{\pgfqpoint{5.661578in}{0.639920in}}%
\pgfpathlineto{\pgfqpoint{5.664799in}{0.642750in}}%
\pgfpathlineto{\pgfqpoint{5.665873in}{0.640909in}}%
\pgfpathlineto{\pgfqpoint{5.666947in}{0.641557in}}%
\pgfpathlineto{\pgfqpoint{5.668021in}{0.645308in}}%
\pgfpathlineto{\pgfqpoint{5.669094in}{0.644012in}}%
\pgfpathlineto{\pgfqpoint{5.672316in}{0.641625in}}%
\pgfpathlineto{\pgfqpoint{5.674464in}{0.636339in}}%
\pgfpathlineto{\pgfqpoint{5.675537in}{0.637601in}}%
\pgfpathlineto{\pgfqpoint{5.680907in}{0.640636in}}%
\pgfpathlineto{\pgfqpoint{5.683054in}{0.644353in}}%
\pgfpathlineto{\pgfqpoint{5.684128in}{0.643603in}}%
\pgfpathlineto{\pgfqpoint{5.687350in}{0.642341in}}%
\pgfpathlineto{\pgfqpoint{5.688423in}{0.636885in}}%
\pgfpathlineto{\pgfqpoint{5.689497in}{0.638454in}}%
\pgfpathlineto{\pgfqpoint{5.690571in}{0.634191in}}%
\pgfpathlineto{\pgfqpoint{5.691645in}{0.635009in}}%
\pgfpathlineto{\pgfqpoint{5.694867in}{0.634532in}}%
\pgfpathlineto{\pgfqpoint{5.695940in}{0.638010in}}%
\pgfpathlineto{\pgfqpoint{5.697014in}{0.644933in}}%
\pgfpathlineto{\pgfqpoint{5.698088in}{0.642171in}}%
\pgfpathlineto{\pgfqpoint{5.699162in}{0.642034in}}%
\pgfpathlineto{\pgfqpoint{5.704531in}{0.657176in}}%
\pgfpathlineto{\pgfqpoint{5.705605in}{0.656016in}}%
\pgfpathlineto{\pgfqpoint{5.706679in}{0.658813in}}%
\pgfpathlineto{\pgfqpoint{5.712048in}{0.663690in}}%
\pgfpathlineto{\pgfqpoint{5.714196in}{0.657142in}}%
\pgfpathlineto{\pgfqpoint{5.717417in}{0.660723in}}%
\pgfpathlineto{\pgfqpoint{5.718491in}{0.658847in}}%
\pgfpathlineto{\pgfqpoint{5.719565in}{0.658472in}}%
\pgfpathlineto{\pgfqpoint{5.721712in}{0.664542in}}%
\pgfpathlineto{\pgfqpoint{5.724934in}{0.663894in}}%
\pgfpathlineto{\pgfqpoint{5.726008in}{0.667952in}}%
\pgfpathlineto{\pgfqpoint{5.727082in}{0.654004in}}%
\pgfpathlineto{\pgfqpoint{5.728155in}{0.652129in}}%
\pgfpathlineto{\pgfqpoint{5.729229in}{0.648275in}}%
\pgfpathlineto{\pgfqpoint{5.732451in}{0.647764in}}%
\pgfpathlineto{\pgfqpoint{5.733525in}{0.646399in}}%
\pgfpathlineto{\pgfqpoint{5.735672in}{0.641455in}}%
\pgfpathlineto{\pgfqpoint{5.736746in}{0.645888in}}%
\pgfpathlineto{\pgfqpoint{5.739968in}{0.643944in}}%
\pgfpathlineto{\pgfqpoint{5.742115in}{0.646161in}}%
\pgfpathlineto{\pgfqpoint{5.743189in}{0.645990in}}%
\pgfpathlineto{\pgfqpoint{5.744263in}{0.647525in}}%
\pgfpathlineto{\pgfqpoint{5.748558in}{0.644080in}}%
\pgfpathlineto{\pgfqpoint{5.749632in}{0.641796in}}%
\pgfpathlineto{\pgfqpoint{5.751780in}{0.642546in}}%
\pgfpathlineto{\pgfqpoint{5.755001in}{0.642648in}}%
\pgfpathlineto{\pgfqpoint{5.757149in}{0.640261in}}%
\pgfpathlineto{\pgfqpoint{5.758223in}{0.639613in}}%
\pgfpathlineto{\pgfqpoint{5.759297in}{0.637976in}}%
\pgfpathlineto{\pgfqpoint{5.762518in}{0.638726in}}%
\pgfpathlineto{\pgfqpoint{5.763592in}{0.641420in}}%
\pgfpathlineto{\pgfqpoint{5.764666in}{0.641011in}}%
\pgfpathlineto{\pgfqpoint{5.765740in}{0.641352in}}%
\pgfpathlineto{\pgfqpoint{5.766814in}{0.643364in}}%
\pgfpathlineto{\pgfqpoint{5.770035in}{0.645206in}}%
\pgfpathlineto{\pgfqpoint{5.771109in}{0.642785in}}%
\pgfpathlineto{\pgfqpoint{5.772183in}{0.642682in}}%
\pgfpathlineto{\pgfqpoint{5.773257in}{0.643433in}}%
\pgfpathlineto{\pgfqpoint{5.774331in}{0.632758in}}%
\pgfpathlineto{\pgfqpoint{5.777552in}{0.628427in}}%
\pgfpathlineto{\pgfqpoint{5.779700in}{0.636817in}}%
\pgfpathlineto{\pgfqpoint{5.780774in}{0.639545in}}%
\pgfpathlineto{\pgfqpoint{5.781847in}{0.640227in}}%
\pgfpathlineto{\pgfqpoint{5.786143in}{0.639033in}}%
\pgfpathlineto{\pgfqpoint{5.789364in}{0.647729in}}%
\pgfpathlineto{\pgfqpoint{5.793660in}{0.649605in}}%
\pgfpathlineto{\pgfqpoint{5.794733in}{0.648548in}}%
\pgfpathlineto{\pgfqpoint{5.795807in}{0.648855in}}%
\pgfpathlineto{\pgfqpoint{5.796881in}{0.648309in}}%
\pgfpathlineto{\pgfqpoint{5.800103in}{0.649435in}}%
\pgfpathlineto{\pgfqpoint{5.801177in}{0.647218in}}%
\pgfpathlineto{\pgfqpoint{5.802250in}{0.643160in}}%
\pgfpathlineto{\pgfqpoint{5.804398in}{0.641489in}}%
\pgfpathlineto{\pgfqpoint{5.807620in}{0.640432in}}%
\pgfpathlineto{\pgfqpoint{5.809767in}{0.636987in}}%
\pgfpathlineto{\pgfqpoint{5.810841in}{0.635623in}}%
\pgfpathlineto{\pgfqpoint{5.811915in}{0.635759in}}%
\pgfpathlineto{\pgfqpoint{5.815136in}{0.634429in}}%
\pgfpathlineto{\pgfqpoint{5.816210in}{0.632690in}}%
\pgfpathlineto{\pgfqpoint{5.817284in}{0.636203in}}%
\pgfpathlineto{\pgfqpoint{5.818358in}{0.633168in}}%
\pgfpathlineto{\pgfqpoint{5.819432in}{0.635350in}}%
\pgfpathlineto{\pgfqpoint{5.822653in}{0.635111in}}%
\pgfpathlineto{\pgfqpoint{5.824801in}{0.641966in}}%
\pgfpathlineto{\pgfqpoint{5.825875in}{0.641693in}}%
\pgfpathlineto{\pgfqpoint{5.826949in}{0.638658in}}%
\pgfpathlineto{\pgfqpoint{5.830170in}{0.639511in}}%
\pgfpathlineto{\pgfqpoint{5.831244in}{0.638795in}}%
\pgfpathlineto{\pgfqpoint{5.832318in}{0.638760in}}%
\pgfpathlineto{\pgfqpoint{5.837687in}{0.635487in}}%
\pgfpathlineto{\pgfqpoint{5.838761in}{0.635828in}}%
\pgfpathlineto{\pgfqpoint{5.840909in}{0.634429in}}%
\pgfpathlineto{\pgfqpoint{5.841982in}{0.633338in}}%
\pgfpathlineto{\pgfqpoint{5.847352in}{0.630883in}}%
\pgfpathlineto{\pgfqpoint{5.848425in}{0.630235in}}%
\pgfpathlineto{\pgfqpoint{5.849499in}{0.630746in}}%
\pgfpathlineto{\pgfqpoint{5.854868in}{0.628461in}}%
\pgfpathlineto{\pgfqpoint{5.855942in}{0.629689in}}%
\pgfpathlineto{\pgfqpoint{5.857016in}{0.624233in}}%
\pgfpathlineto{\pgfqpoint{5.860238in}{0.628223in}}%
\pgfpathlineto{\pgfqpoint{5.862385in}{0.623721in}}%
\pgfpathlineto{\pgfqpoint{5.864533in}{0.624710in}}%
\pgfpathlineto{\pgfqpoint{5.867755in}{0.624915in}}%
\pgfpathlineto{\pgfqpoint{5.868828in}{0.625972in}}%
\pgfpathlineto{\pgfqpoint{5.869902in}{0.624130in}}%
\pgfpathlineto{\pgfqpoint{5.870976in}{0.627472in}}%
\pgfpathlineto{\pgfqpoint{5.872050in}{0.627029in}}%
\pgfpathlineto{\pgfqpoint{5.876345in}{0.621948in}}%
\pgfpathlineto{\pgfqpoint{5.877419in}{0.623517in}}%
\pgfpathlineto{\pgfqpoint{5.878493in}{0.622221in}}%
\pgfpathlineto{\pgfqpoint{5.879567in}{0.625665in}}%
\pgfpathlineto{\pgfqpoint{5.882788in}{0.624471in}}%
\pgfpathlineto{\pgfqpoint{5.883862in}{0.624812in}}%
\pgfpathlineto{\pgfqpoint{5.884936in}{0.624335in}}%
\pgfpathlineto{\pgfqpoint{5.886010in}{0.625563in}}%
\pgfpathlineto{\pgfqpoint{5.887084in}{0.624471in}}%
\pgfpathlineto{\pgfqpoint{5.890305in}{0.624471in}}%
\pgfpathlineto{\pgfqpoint{5.892453in}{0.620959in}}%
\pgfpathlineto{\pgfqpoint{5.893527in}{0.620004in}}%
\pgfpathlineto{\pgfqpoint{5.894600in}{0.620584in}}%
\pgfpathlineto{\pgfqpoint{5.897822in}{0.619049in}}%
\pgfpathlineto{\pgfqpoint{5.898896in}{0.620175in}}%
\pgfpathlineto{\pgfqpoint{5.899970in}{0.622630in}}%
\pgfpathlineto{\pgfqpoint{5.901044in}{0.622971in}}%
\pgfpathlineto{\pgfqpoint{5.902117in}{0.626245in}}%
\pgfpathlineto{\pgfqpoint{5.905339in}{0.627336in}}%
\pgfpathlineto{\pgfqpoint{5.906413in}{0.625324in}}%
\pgfpathlineto{\pgfqpoint{5.908560in}{0.629450in}}%
\pgfpathlineto{\pgfqpoint{5.909634in}{0.628905in}}%
\pgfpathlineto{\pgfqpoint{5.915003in}{0.622596in}}%
\pgfpathlineto{\pgfqpoint{5.916077in}{0.627336in}}%
\pgfpathlineto{\pgfqpoint{5.917151in}{0.624335in}}%
\pgfpathlineto{\pgfqpoint{5.920373in}{0.630780in}}%
\pgfpathlineto{\pgfqpoint{5.921446in}{0.630644in}}%
\pgfpathlineto{\pgfqpoint{5.923594in}{0.632520in}}%
\pgfpathlineto{\pgfqpoint{5.924668in}{0.641386in}}%
\pgfpathlineto{\pgfqpoint{5.927889in}{0.642171in}}%
\pgfpathlineto{\pgfqpoint{5.928963in}{0.641455in}}%
\pgfpathlineto{\pgfqpoint{5.930037in}{0.646093in}}%
\pgfpathlineto{\pgfqpoint{5.931111in}{0.646911in}}%
\pgfpathlineto{\pgfqpoint{5.932185in}{0.643228in}}%
\pgfpathlineto{\pgfqpoint{5.935406in}{0.641216in}}%
\pgfpathlineto{\pgfqpoint{5.936480in}{0.641489in}}%
\pgfpathlineto{\pgfqpoint{5.937554in}{0.643296in}}%
\pgfpathlineto{\pgfqpoint{5.939702in}{0.645104in}}%
\pgfpathlineto{\pgfqpoint{5.942923in}{0.645581in}}%
\pgfpathlineto{\pgfqpoint{5.943997in}{0.647866in}}%
\pgfpathlineto{\pgfqpoint{5.945071in}{0.646093in}}%
\pgfpathlineto{\pgfqpoint{5.946145in}{0.645513in}}%
\pgfpathlineto{\pgfqpoint{5.947219in}{0.644080in}}%
\pgfpathlineto{\pgfqpoint{5.951514in}{0.651106in}}%
\pgfpathlineto{\pgfqpoint{5.952588in}{0.655437in}}%
\pgfpathlineto{\pgfqpoint{5.954735in}{0.667441in}}%
\pgfpathlineto{\pgfqpoint{5.959031in}{0.664099in}}%
\pgfpathlineto{\pgfqpoint{5.961178in}{0.665872in}}%
\pgfpathlineto{\pgfqpoint{5.962252in}{0.664303in}}%
\pgfpathlineto{\pgfqpoint{5.965474in}{0.668873in}}%
\pgfpathlineto{\pgfqpoint{5.967621in}{0.669726in}}%
\pgfpathlineto{\pgfqpoint{5.969769in}{0.668396in}}%
\pgfpathlineto{\pgfqpoint{5.974065in}{0.668464in}}%
\pgfpathlineto{\pgfqpoint{5.975138in}{0.665599in}}%
\pgfpathlineto{\pgfqpoint{5.976212in}{0.666452in}}%
\pgfpathlineto{\pgfqpoint{5.977286in}{0.665326in}}%
\pgfpathlineto{\pgfqpoint{5.981581in}{0.671431in}}%
\pgfpathlineto{\pgfqpoint{5.982655in}{0.675898in}}%
\pgfpathlineto{\pgfqpoint{5.983729in}{0.675728in}}%
\pgfpathlineto{\pgfqpoint{5.984803in}{0.680980in}}%
\pgfpathlineto{\pgfqpoint{5.988024in}{0.678933in}}%
\pgfpathlineto{\pgfqpoint{5.989098in}{0.679002in}}%
\pgfpathlineto{\pgfqpoint{5.990172in}{0.682480in}}%
\pgfpathlineto{\pgfqpoint{5.991246in}{0.676205in}}%
\pgfpathlineto{\pgfqpoint{5.992320in}{0.677944in}}%
\pgfpathlineto{\pgfqpoint{5.996615in}{0.677638in}}%
\pgfpathlineto{\pgfqpoint{5.997689in}{0.678285in}}%
\pgfpathlineto{\pgfqpoint{5.998763in}{0.675387in}}%
\pgfpathlineto{\pgfqpoint{5.999837in}{0.676649in}}%
\pgfpathlineto{\pgfqpoint{6.003058in}{0.674841in}}%
\pgfpathlineto{\pgfqpoint{6.004132in}{0.677433in}}%
\pgfpathlineto{\pgfqpoint{6.006280in}{0.677944in}}%
\pgfpathlineto{\pgfqpoint{6.007354in}{0.682003in}}%
\pgfpathlineto{\pgfqpoint{6.010575in}{0.687391in}}%
\pgfpathlineto{\pgfqpoint{6.011649in}{0.686470in}}%
\pgfpathlineto{\pgfqpoint{6.012723in}{0.688584in}}%
\pgfpathlineto{\pgfqpoint{6.014870in}{0.685311in}}%
\pgfpathlineto{\pgfqpoint{6.018092in}{0.682923in}}%
\pgfpathlineto{\pgfqpoint{6.019166in}{0.681048in}}%
\pgfpathlineto{\pgfqpoint{6.020240in}{0.681048in}}%
\pgfpathlineto{\pgfqpoint{6.021313in}{0.682651in}}%
\pgfpathlineto{\pgfqpoint{6.022387in}{0.681900in}}%
\pgfpathlineto{\pgfqpoint{6.025609in}{0.683162in}}%
\pgfpathlineto{\pgfqpoint{6.026683in}{0.685413in}}%
\pgfpathlineto{\pgfqpoint{6.027756in}{0.684901in}}%
\pgfpathlineto{\pgfqpoint{6.028830in}{0.686641in}}%
\pgfpathlineto{\pgfqpoint{6.029904in}{0.684526in}}%
\pgfpathlineto{\pgfqpoint{6.035273in}{0.684697in}}%
\pgfpathlineto{\pgfqpoint{6.036347in}{0.683435in}}%
\pgfpathlineto{\pgfqpoint{6.037421in}{0.685379in}}%
\pgfpathlineto{\pgfqpoint{6.041716in}{0.684629in}}%
\pgfpathlineto{\pgfqpoint{6.042790in}{0.687732in}}%
\pgfpathlineto{\pgfqpoint{6.043864in}{0.686266in}}%
\pgfpathlineto{\pgfqpoint{6.044938in}{0.688380in}}%
\pgfpathlineto{\pgfqpoint{6.048159in}{0.686504in}}%
\pgfpathlineto{\pgfqpoint{6.049233in}{0.687152in}}%
\pgfpathlineto{\pgfqpoint{6.050307in}{0.687084in}}%
\pgfpathlineto{\pgfqpoint{6.051381in}{0.687698in}}%
\pgfpathlineto{\pgfqpoint{6.052455in}{0.687357in}}%
\pgfpathlineto{\pgfqpoint{6.055676in}{0.689301in}}%
\pgfpathlineto{\pgfqpoint{6.056750in}{0.691892in}}%
\pgfpathlineto{\pgfqpoint{6.058898in}{0.689949in}}%
\pgfpathlineto{\pgfqpoint{6.059972in}{0.690085in}}%
\pgfpathlineto{\pgfqpoint{6.063193in}{0.693222in}}%
\pgfpathlineto{\pgfqpoint{6.064267in}{0.690085in}}%
\pgfpathlineto{\pgfqpoint{6.066415in}{0.691688in}}%
\pgfpathlineto{\pgfqpoint{6.067488in}{0.691347in}}%
\pgfpathlineto{\pgfqpoint{6.070710in}{0.692131in}}%
\pgfpathlineto{\pgfqpoint{6.071784in}{0.694348in}}%
\pgfpathlineto{\pgfqpoint{6.072858in}{0.692643in}}%
\pgfpathlineto{\pgfqpoint{6.073932in}{0.694757in}}%
\pgfpathlineto{\pgfqpoint{6.075005in}{0.695473in}}%
\pgfpathlineto{\pgfqpoint{6.081448in}{0.694348in}}%
\pgfpathlineto{\pgfqpoint{6.082522in}{0.692779in}}%
\pgfpathlineto{\pgfqpoint{6.085744in}{0.692302in}}%
\pgfpathlineto{\pgfqpoint{6.086818in}{0.694416in}}%
\pgfpathlineto{\pgfqpoint{6.087891in}{0.694314in}}%
\pgfpathlineto{\pgfqpoint{6.093261in}{0.696735in}}%
\pgfpathlineto{\pgfqpoint{6.094334in}{0.698099in}}%
\pgfpathlineto{\pgfqpoint{6.095408in}{0.696565in}}%
\pgfpathlineto{\pgfqpoint{6.096482in}{0.700043in}}%
\pgfpathlineto{\pgfqpoint{6.097556in}{0.698918in}}%
\pgfpathlineto{\pgfqpoint{6.100777in}{0.696462in}}%
\pgfpathlineto{\pgfqpoint{6.101851in}{0.701339in}}%
\pgfpathlineto{\pgfqpoint{6.103999in}{0.703521in}}%
\pgfpathlineto{\pgfqpoint{6.108294in}{0.699872in}}%
\pgfpathlineto{\pgfqpoint{6.109368in}{0.698679in}}%
\pgfpathlineto{\pgfqpoint{6.110442in}{0.689642in}}%
\pgfpathlineto{\pgfqpoint{6.111516in}{0.688175in}}%
\pgfpathlineto{\pgfqpoint{6.112590in}{0.690869in}}%
\pgfpathlineto{\pgfqpoint{6.115811in}{0.688994in}}%
\pgfpathlineto{\pgfqpoint{6.116885in}{0.691108in}}%
\pgfpathlineto{\pgfqpoint{6.117959in}{0.683196in}}%
\pgfpathlineto{\pgfqpoint{6.119033in}{0.682923in}}%
\pgfpathlineto{\pgfqpoint{6.120107in}{0.683299in}}%
\pgfpathlineto{\pgfqpoint{6.123328in}{0.681457in}}%
\pgfpathlineto{\pgfqpoint{6.125476in}{0.672181in}}%
\pgfpathlineto{\pgfqpoint{6.126550in}{0.673409in}}%
\pgfpathlineto{\pgfqpoint{6.127623in}{0.676171in}}%
\pgfpathlineto{\pgfqpoint{6.130845in}{0.676546in}}%
\pgfpathlineto{\pgfqpoint{6.131919in}{0.674534in}}%
\pgfpathlineto{\pgfqpoint{6.132993in}{0.676785in}}%
\pgfpathlineto{\pgfqpoint{6.134066in}{0.675353in}}%
\pgfpathlineto{\pgfqpoint{6.135140in}{0.679104in}}%
\pgfpathlineto{\pgfqpoint{6.139436in}{0.678865in}}%
\pgfpathlineto{\pgfqpoint{6.140509in}{0.677535in}}%
\pgfpathlineto{\pgfqpoint{6.141583in}{0.678422in}}%
\pgfpathlineto{\pgfqpoint{6.142657in}{0.675046in}}%
\pgfpathlineto{\pgfqpoint{6.145879in}{0.672897in}}%
\pgfpathlineto{\pgfqpoint{6.146953in}{0.669521in}}%
\pgfpathlineto{\pgfqpoint{6.148026in}{0.670919in}}%
\pgfpathlineto{\pgfqpoint{6.149100in}{0.665667in}}%
\pgfpathlineto{\pgfqpoint{6.150174in}{0.669930in}}%
\pgfpathlineto{\pgfqpoint{6.153396in}{0.674602in}}%
\pgfpathlineto{\pgfqpoint{6.155543in}{0.671635in}}%
\pgfpathlineto{\pgfqpoint{6.156617in}{0.671124in}}%
\pgfpathlineto{\pgfqpoint{6.157691in}{0.669555in}}%
\pgfpathlineto{\pgfqpoint{6.160912in}{0.669112in}}%
\pgfpathlineto{\pgfqpoint{6.161986in}{0.664406in}}%
\pgfpathlineto{\pgfqpoint{6.163060in}{0.667236in}}%
\pgfpathlineto{\pgfqpoint{6.164134in}{0.665326in}}%
\pgfpathlineto{\pgfqpoint{6.165208in}{0.665804in}}%
\pgfpathlineto{\pgfqpoint{6.168429in}{0.669760in}}%
\pgfpathlineto{\pgfqpoint{6.169503in}{0.669248in}}%
\pgfpathlineto{\pgfqpoint{6.170577in}{0.673989in}}%
\pgfpathlineto{\pgfqpoint{6.171651in}{0.670237in}}%
\pgfpathlineto{\pgfqpoint{6.172725in}{0.672011in}}%
\pgfpathlineto{\pgfqpoint{6.175946in}{0.675966in}}%
\pgfpathlineto{\pgfqpoint{6.178094in}{0.669964in}}%
\pgfpathlineto{\pgfqpoint{6.179168in}{0.664985in}}%
\pgfpathlineto{\pgfqpoint{6.180242in}{0.664883in}}%
\pgfpathlineto{\pgfqpoint{6.184537in}{0.666622in}}%
\pgfpathlineto{\pgfqpoint{6.185611in}{0.668396in}}%
\pgfpathlineto{\pgfqpoint{6.186685in}{0.668123in}}%
\pgfpathlineto{\pgfqpoint{6.187758in}{0.670749in}}%
\pgfpathlineto{\pgfqpoint{6.190980in}{0.669760in}}%
\pgfpathlineto{\pgfqpoint{6.194201in}{0.678285in}}%
\pgfpathlineto{\pgfqpoint{6.195275in}{0.677365in}}%
\pgfpathlineto{\pgfqpoint{6.198497in}{0.677092in}}%
\pgfpathlineto{\pgfqpoint{6.199571in}{0.675148in}}%
\pgfpathlineto{\pgfqpoint{6.200644in}{0.676887in}}%
\pgfpathlineto{\pgfqpoint{6.201718in}{0.687016in}}%
\pgfpathlineto{\pgfqpoint{6.206014in}{0.686777in}}%
\pgfpathlineto{\pgfqpoint{6.207087in}{0.689028in}}%
\pgfpathlineto{\pgfqpoint{6.208161in}{0.682617in}}%
\pgfpathlineto{\pgfqpoint{6.209235in}{0.684083in}}%
\pgfpathlineto{\pgfqpoint{6.210309in}{0.679377in}}%
\pgfpathlineto{\pgfqpoint{6.213531in}{0.674943in}}%
\pgfpathlineto{\pgfqpoint{6.214604in}{0.677024in}}%
\pgfpathlineto{\pgfqpoint{6.215678in}{0.663280in}}%
\pgfpathlineto{\pgfqpoint{6.216752in}{0.658370in}}%
\pgfpathlineto{\pgfqpoint{6.217826in}{0.660484in}}%
\pgfpathlineto{\pgfqpoint{6.221047in}{0.658540in}}%
\pgfpathlineto{\pgfqpoint{6.222121in}{0.658915in}}%
\pgfpathlineto{\pgfqpoint{6.223195in}{0.661200in}}%
\pgfpathlineto{\pgfqpoint{6.225343in}{0.656221in}}%
\pgfpathlineto{\pgfqpoint{6.228564in}{0.657790in}}%
\pgfpathlineto{\pgfqpoint{6.229638in}{0.663076in}}%
\pgfpathlineto{\pgfqpoint{6.230712in}{0.658881in}}%
\pgfpathlineto{\pgfqpoint{6.231786in}{0.658949in}}%
\pgfpathlineto{\pgfqpoint{6.232860in}{0.661882in}}%
\pgfpathlineto{\pgfqpoint{6.237155in}{0.662428in}}%
\pgfpathlineto{\pgfqpoint{6.238229in}{0.663417in}}%
\pgfpathlineto{\pgfqpoint{6.239303in}{0.657892in}}%
\pgfpathlineto{\pgfqpoint{6.240376in}{0.658881in}}%
\pgfpathlineto{\pgfqpoint{6.245746in}{0.658881in}}%
\pgfpathlineto{\pgfqpoint{6.246820in}{0.644183in}}%
\pgfpathlineto{\pgfqpoint{6.251115in}{0.644285in}}%
\pgfpathlineto{\pgfqpoint{6.253263in}{0.650014in}}%
\pgfpathlineto{\pgfqpoint{6.254336in}{0.646945in}}%
\pgfpathlineto{\pgfqpoint{6.255410in}{0.649025in}}%
\pgfpathlineto{\pgfqpoint{6.258632in}{0.647627in}}%
\pgfpathlineto{\pgfqpoint{6.259706in}{0.648718in}}%
\pgfpathlineto{\pgfqpoint{6.260779in}{0.651310in}}%
\pgfpathlineto{\pgfqpoint{6.262927in}{0.649264in}}%
\pgfpathlineto{\pgfqpoint{6.266149in}{0.652504in}}%
\pgfpathlineto{\pgfqpoint{6.267222in}{0.649366in}}%
\pgfpathlineto{\pgfqpoint{6.268296in}{0.651378in}}%
\pgfpathlineto{\pgfqpoint{6.269370in}{0.647457in}}%
\pgfpathlineto{\pgfqpoint{6.274739in}{0.656528in}}%
\pgfpathlineto{\pgfqpoint{6.281182in}{0.652504in}}%
\pgfpathlineto{\pgfqpoint{6.282256in}{0.652504in}}%
\pgfpathlineto{\pgfqpoint{6.283330in}{0.649128in}}%
\pgfpathlineto{\pgfqpoint{6.284404in}{0.643739in}}%
\pgfpathlineto{\pgfqpoint{6.285478in}{0.645240in}}%
\pgfpathlineto{\pgfqpoint{6.289773in}{0.648480in}}%
\pgfpathlineto{\pgfqpoint{6.290847in}{0.648105in}}%
\pgfpathlineto{\pgfqpoint{6.292995in}{0.651924in}}%
\pgfpathlineto{\pgfqpoint{6.297290in}{0.648241in}}%
\pgfpathlineto{\pgfqpoint{6.298364in}{0.646638in}}%
\pgfpathlineto{\pgfqpoint{6.299438in}{0.649128in}}%
\pgfpathlineto{\pgfqpoint{6.300511in}{0.648309in}}%
\pgfpathlineto{\pgfqpoint{6.304807in}{0.646672in}}%
\pgfpathlineto{\pgfqpoint{6.305881in}{0.650696in}}%
\pgfpathlineto{\pgfqpoint{6.306954in}{0.648446in}}%
\pgfpathlineto{\pgfqpoint{6.308028in}{0.649401in}}%
\pgfpathlineto{\pgfqpoint{6.312324in}{0.659222in}}%
\pgfpathlineto{\pgfqpoint{6.313397in}{0.657824in}}%
\pgfpathlineto{\pgfqpoint{6.315545in}{0.669726in}}%
\pgfpathlineto{\pgfqpoint{6.318767in}{0.669589in}}%
\pgfpathlineto{\pgfqpoint{6.319841in}{0.664406in}}%
\pgfpathlineto{\pgfqpoint{6.320914in}{0.666111in}}%
\pgfpathlineto{\pgfqpoint{6.323062in}{0.665292in}}%
\pgfpathlineto{\pgfqpoint{6.326284in}{0.663007in}}%
\pgfpathlineto{\pgfqpoint{6.327357in}{0.663826in}}%
\pgfpathlineto{\pgfqpoint{6.328431in}{0.662973in}}%
\pgfpathlineto{\pgfqpoint{6.330579in}{0.662632in}}%
\pgfpathlineto{\pgfqpoint{6.333800in}{0.663144in}}%
\pgfpathlineto{\pgfqpoint{6.334874in}{0.665190in}}%
\pgfpathlineto{\pgfqpoint{6.335948in}{0.671260in}}%
\pgfpathlineto{\pgfqpoint{6.337022in}{0.669862in}}%
\pgfpathlineto{\pgfqpoint{6.338096in}{0.671294in}}%
\pgfpathlineto{\pgfqpoint{6.341317in}{0.687732in}}%
\pgfpathlineto{\pgfqpoint{6.343465in}{0.671976in}}%
\pgfpathlineto{\pgfqpoint{6.345613in}{0.670681in}}%
\pgfpathlineto{\pgfqpoint{6.349908in}{0.681355in}}%
\pgfpathlineto{\pgfqpoint{6.350982in}{0.681969in}}%
\pgfpathlineto{\pgfqpoint{6.352056in}{0.691824in}}%
\pgfpathlineto{\pgfqpoint{6.353130in}{0.694177in}}%
\pgfpathlineto{\pgfqpoint{6.356351in}{0.693359in}}%
\pgfpathlineto{\pgfqpoint{6.357425in}{0.695985in}}%
\pgfpathlineto{\pgfqpoint{6.358499in}{0.688891in}}%
\pgfpathlineto{\pgfqpoint{6.359573in}{0.688516in}}%
\pgfpathlineto{\pgfqpoint{6.360646in}{0.685515in}}%
\pgfpathlineto{\pgfqpoint{6.364942in}{0.683674in}}%
\pgfpathlineto{\pgfqpoint{6.366016in}{0.682071in}}%
\pgfpathlineto{\pgfqpoint{6.367089in}{0.682480in}}%
\pgfpathlineto{\pgfqpoint{6.368163in}{0.681628in}}%
\pgfpathlineto{\pgfqpoint{6.368163in}{0.681628in}}%
\pgfusepath{stroke}%
\end{pgfscope}%
\begin{pgfscope}%
\pgfpathrectangle{\pgfqpoint{3.902254in}{0.331635in}}{\pgfqpoint{2.583333in}{0.400885in}}%
\pgfusepath{clip}%
\pgfsetroundcap%
\pgfsetroundjoin%
\pgfsetlinewidth{1.505625pt}%
\definecolor{currentstroke}{rgb}{0.090196,0.745098,0.811765}%
\pgfsetstrokecolor{currentstroke}%
\pgfsetdash{}{0pt}%
\pgfpathmoveto{\pgfqpoint{4.019678in}{0.440624in}}%
\pgfpathlineto{\pgfqpoint{4.022900in}{0.435844in}}%
\pgfpathlineto{\pgfqpoint{4.027195in}{0.436631in}}%
\pgfpathlineto{\pgfqpoint{4.028269in}{0.433956in}}%
\pgfpathlineto{\pgfqpoint{4.029343in}{0.434020in}}%
\pgfpathlineto{\pgfqpoint{4.030417in}{0.433096in}}%
\pgfpathlineto{\pgfqpoint{4.034712in}{0.433319in}}%
\pgfpathlineto{\pgfqpoint{4.035786in}{0.434880in}}%
\pgfpathlineto{\pgfqpoint{4.036860in}{0.433701in}}%
\pgfpathlineto{\pgfqpoint{4.037934in}{0.434044in}}%
\pgfpathlineto{\pgfqpoint{4.042229in}{0.434233in}}%
\pgfpathlineto{\pgfqpoint{4.043303in}{0.433351in}}%
\pgfpathlineto{\pgfqpoint{4.045451in}{0.434222in}}%
\pgfpathlineto{\pgfqpoint{4.049746in}{0.433246in}}%
\pgfpathlineto{\pgfqpoint{4.051894in}{0.430910in}}%
\pgfpathlineto{\pgfqpoint{4.052967in}{0.427891in}}%
\pgfpathlineto{\pgfqpoint{4.056189in}{0.426697in}}%
\pgfpathlineto{\pgfqpoint{4.058337in}{0.424637in}}%
\pgfpathlineto{\pgfqpoint{4.059410in}{0.423980in}}%
\pgfpathlineto{\pgfqpoint{4.060484in}{0.424196in}}%
\pgfpathlineto{\pgfqpoint{4.063706in}{0.423355in}}%
\pgfpathlineto{\pgfqpoint{4.065853in}{0.424670in}}%
\pgfpathlineto{\pgfqpoint{4.068001in}{0.423445in}}%
\pgfpathlineto{\pgfqpoint{4.072296in}{0.423873in}}%
\pgfpathlineto{\pgfqpoint{4.073370in}{0.424601in}}%
\pgfpathlineto{\pgfqpoint{4.074444in}{0.424081in}}%
\pgfpathlineto{\pgfqpoint{4.075518in}{0.424487in}}%
\pgfpathlineto{\pgfqpoint{4.080887in}{0.422838in}}%
\pgfpathlineto{\pgfqpoint{4.081961in}{0.421887in}}%
\pgfpathlineto{\pgfqpoint{4.083035in}{0.421965in}}%
\pgfpathlineto{\pgfqpoint{4.086256in}{0.421160in}}%
\pgfpathlineto{\pgfqpoint{4.087330in}{0.422769in}}%
\pgfpathlineto{\pgfqpoint{4.088404in}{0.423376in}}%
\pgfpathlineto{\pgfqpoint{4.090552in}{0.422181in}}%
\pgfpathlineto{\pgfqpoint{4.093773in}{0.421946in}}%
\pgfpathlineto{\pgfqpoint{4.094847in}{0.418052in}}%
\pgfpathlineto{\pgfqpoint{4.095921in}{0.419204in}}%
\pgfpathlineto{\pgfqpoint{4.096995in}{0.419228in}}%
\pgfpathlineto{\pgfqpoint{4.098069in}{0.419843in}}%
\pgfpathlineto{\pgfqpoint{4.101290in}{0.419270in}}%
\pgfpathlineto{\pgfqpoint{4.103438in}{0.419664in}}%
\pgfpathlineto{\pgfqpoint{4.104512in}{0.419614in}}%
\pgfpathlineto{\pgfqpoint{4.105585in}{0.418796in}}%
\pgfpathlineto{\pgfqpoint{4.108807in}{0.417211in}}%
\pgfpathlineto{\pgfqpoint{4.109881in}{0.417683in}}%
\pgfpathlineto{\pgfqpoint{4.112029in}{0.420169in}}%
\pgfpathlineto{\pgfqpoint{4.113102in}{0.418437in}}%
\pgfpathlineto{\pgfqpoint{4.116324in}{0.418291in}}%
\pgfpathlineto{\pgfqpoint{4.118472in}{0.420291in}}%
\pgfpathlineto{\pgfqpoint{4.119545in}{0.419964in}}%
\pgfpathlineto{\pgfqpoint{4.123841in}{0.422140in}}%
\pgfpathlineto{\pgfqpoint{4.124915in}{0.419498in}}%
\pgfpathlineto{\pgfqpoint{4.125988in}{0.418609in}}%
\pgfpathlineto{\pgfqpoint{4.127062in}{0.416811in}}%
\pgfpathlineto{\pgfqpoint{4.128136in}{0.417479in}}%
\pgfpathlineto{\pgfqpoint{4.131358in}{0.417905in}}%
\pgfpathlineto{\pgfqpoint{4.132431in}{0.415577in}}%
\pgfpathlineto{\pgfqpoint{4.135653in}{0.416291in}}%
\pgfpathlineto{\pgfqpoint{4.138874in}{0.417026in}}%
\pgfpathlineto{\pgfqpoint{4.139948in}{0.416653in}}%
\pgfpathlineto{\pgfqpoint{4.142096in}{0.414073in}}%
\pgfpathlineto{\pgfqpoint{4.146391in}{0.414587in}}%
\pgfpathlineto{\pgfqpoint{4.147465in}{0.413147in}}%
\pgfpathlineto{\pgfqpoint{4.148539in}{0.413650in}}%
\pgfpathlineto{\pgfqpoint{4.149613in}{0.413095in}}%
\pgfpathlineto{\pgfqpoint{4.150687in}{0.414900in}}%
\pgfpathlineto{\pgfqpoint{4.154982in}{0.412016in}}%
\pgfpathlineto{\pgfqpoint{4.156056in}{0.410565in}}%
\pgfpathlineto{\pgfqpoint{4.158204in}{0.409513in}}%
\pgfpathlineto{\pgfqpoint{4.163573in}{0.410443in}}%
\pgfpathlineto{\pgfqpoint{4.165720in}{0.412934in}}%
\pgfpathlineto{\pgfqpoint{4.170016in}{0.411748in}}%
\pgfpathlineto{\pgfqpoint{4.171090in}{0.412103in}}%
\pgfpathlineto{\pgfqpoint{4.177533in}{0.409577in}}%
\pgfpathlineto{\pgfqpoint{4.178607in}{0.410106in}}%
\pgfpathlineto{\pgfqpoint{4.179680in}{0.409121in}}%
\pgfpathlineto{\pgfqpoint{4.180754in}{0.411591in}}%
\pgfpathlineto{\pgfqpoint{4.183976in}{0.411569in}}%
\pgfpathlineto{\pgfqpoint{4.188271in}{0.408014in}}%
\pgfpathlineto{\pgfqpoint{4.191493in}{0.408831in}}%
\pgfpathlineto{\pgfqpoint{4.192566in}{0.407748in}}%
\pgfpathlineto{\pgfqpoint{4.193640in}{0.408012in}}%
\pgfpathlineto{\pgfqpoint{4.194714in}{0.406255in}}%
\pgfpathlineto{\pgfqpoint{4.199009in}{0.406412in}}%
\pgfpathlineto{\pgfqpoint{4.201157in}{0.405297in}}%
\pgfpathlineto{\pgfqpoint{4.202231in}{0.405872in}}%
\pgfpathlineto{\pgfqpoint{4.203305in}{0.405736in}}%
\pgfpathlineto{\pgfqpoint{4.206526in}{0.407091in}}%
\pgfpathlineto{\pgfqpoint{4.208674in}{0.404997in}}%
\pgfpathlineto{\pgfqpoint{4.209748in}{0.405092in}}%
\pgfpathlineto{\pgfqpoint{4.210822in}{0.403928in}}%
\pgfpathlineto{\pgfqpoint{4.215117in}{0.403760in}}%
\pgfpathlineto{\pgfqpoint{4.218339in}{0.404685in}}%
\pgfpathlineto{\pgfqpoint{4.221560in}{0.404761in}}%
\pgfpathlineto{\pgfqpoint{4.222634in}{0.405840in}}%
\pgfpathlineto{\pgfqpoint{4.224782in}{0.405761in}}%
\pgfpathlineto{\pgfqpoint{4.225855in}{0.404384in}}%
\pgfpathlineto{\pgfqpoint{4.229077in}{0.404947in}}%
\pgfpathlineto{\pgfqpoint{4.230151in}{0.402398in}}%
\pgfpathlineto{\pgfqpoint{4.231225in}{0.402398in}}%
\pgfpathlineto{\pgfqpoint{4.236594in}{0.404636in}}%
\pgfpathlineto{\pgfqpoint{4.238741in}{0.404109in}}%
\pgfpathlineto{\pgfqpoint{4.239815in}{0.401739in}}%
\pgfpathlineto{\pgfqpoint{4.240889in}{0.401369in}}%
\pgfpathlineto{\pgfqpoint{4.244111in}{0.401578in}}%
\pgfpathlineto{\pgfqpoint{4.246258in}{0.403152in}}%
\pgfpathlineto{\pgfqpoint{4.247332in}{0.402879in}}%
\pgfpathlineto{\pgfqpoint{4.248406in}{0.401592in}}%
\pgfpathlineto{\pgfqpoint{4.252701in}{0.401539in}}%
\pgfpathlineto{\pgfqpoint{4.253775in}{0.400452in}}%
\pgfpathlineto{\pgfqpoint{4.255923in}{0.401758in}}%
\pgfpathlineto{\pgfqpoint{4.266661in}{0.400500in}}%
\pgfpathlineto{\pgfqpoint{4.267735in}{0.401746in}}%
\pgfpathlineto{\pgfqpoint{4.268809in}{0.401711in}}%
\pgfpathlineto{\pgfqpoint{4.269883in}{0.402486in}}%
\pgfpathlineto{\pgfqpoint{4.270957in}{0.401857in}}%
\pgfpathlineto{\pgfqpoint{4.275252in}{0.401751in}}%
\pgfpathlineto{\pgfqpoint{4.276326in}{0.401099in}}%
\pgfpathlineto{\pgfqpoint{4.277400in}{0.402070in}}%
\pgfpathlineto{\pgfqpoint{4.282769in}{0.401679in}}%
\pgfpathlineto{\pgfqpoint{4.284917in}{0.398269in}}%
\pgfpathlineto{\pgfqpoint{4.290286in}{0.398705in}}%
\pgfpathlineto{\pgfqpoint{4.291360in}{0.398493in}}%
\pgfpathlineto{\pgfqpoint{4.292433in}{0.397179in}}%
\pgfpathlineto{\pgfqpoint{4.293507in}{0.397539in}}%
\pgfpathlineto{\pgfqpoint{4.297803in}{0.398175in}}%
\pgfpathlineto{\pgfqpoint{4.298876in}{0.397016in}}%
\pgfpathlineto{\pgfqpoint{4.301024in}{0.396969in}}%
\pgfpathlineto{\pgfqpoint{4.304246in}{0.396704in}}%
\pgfpathlineto{\pgfqpoint{4.306393in}{0.398094in}}%
\pgfpathlineto{\pgfqpoint{4.307467in}{0.397162in}}%
\pgfpathlineto{\pgfqpoint{4.308541in}{0.397585in}}%
\pgfpathlineto{\pgfqpoint{4.312836in}{0.398509in}}%
\pgfpathlineto{\pgfqpoint{4.313910in}{0.397291in}}%
\pgfpathlineto{\pgfqpoint{4.316058in}{0.396586in}}%
\pgfpathlineto{\pgfqpoint{4.319279in}{0.397481in}}%
\pgfpathlineto{\pgfqpoint{4.322501in}{0.400380in}}%
\pgfpathlineto{\pgfqpoint{4.323575in}{0.399987in}}%
\pgfpathlineto{\pgfqpoint{4.326796in}{0.399683in}}%
\pgfpathlineto{\pgfqpoint{4.327870in}{0.399012in}}%
\pgfpathlineto{\pgfqpoint{4.328944in}{0.397446in}}%
\pgfpathlineto{\pgfqpoint{4.330018in}{0.397256in}}%
\pgfpathlineto{\pgfqpoint{4.331092in}{0.397996in}}%
\pgfpathlineto{\pgfqpoint{4.334313in}{0.398156in}}%
\pgfpathlineto{\pgfqpoint{4.335387in}{0.399656in}}%
\pgfpathlineto{\pgfqpoint{4.338608in}{0.400687in}}%
\pgfpathlineto{\pgfqpoint{4.343978in}{0.402187in}}%
\pgfpathlineto{\pgfqpoint{4.345051in}{0.401112in}}%
\pgfpathlineto{\pgfqpoint{4.352568in}{0.400701in}}%
\pgfpathlineto{\pgfqpoint{4.353642in}{0.405325in}}%
\pgfpathlineto{\pgfqpoint{4.356864in}{0.404645in}}%
\pgfpathlineto{\pgfqpoint{4.357938in}{0.403766in}}%
\pgfpathlineto{\pgfqpoint{4.359011in}{0.405094in}}%
\pgfpathlineto{\pgfqpoint{4.361159in}{0.404667in}}%
\pgfpathlineto{\pgfqpoint{4.364381in}{0.403825in}}%
\pgfpathlineto{\pgfqpoint{4.366528in}{0.402522in}}%
\pgfpathlineto{\pgfqpoint{4.368676in}{0.401578in}}%
\pgfpathlineto{\pgfqpoint{4.371897in}{0.401951in}}%
\pgfpathlineto{\pgfqpoint{4.372971in}{0.402649in}}%
\pgfpathlineto{\pgfqpoint{4.375119in}{0.400830in}}%
\pgfpathlineto{\pgfqpoint{4.381562in}{0.401015in}}%
\pgfpathlineto{\pgfqpoint{4.383710in}{0.400388in}}%
\pgfpathlineto{\pgfqpoint{4.390153in}{0.400584in}}%
\pgfpathlineto{\pgfqpoint{4.391227in}{0.401275in}}%
\pgfpathlineto{\pgfqpoint{4.394448in}{0.400290in}}%
\pgfpathlineto{\pgfqpoint{4.395522in}{0.398814in}}%
\pgfpathlineto{\pgfqpoint{4.396596in}{0.399243in}}%
\pgfpathlineto{\pgfqpoint{4.397670in}{0.397723in}}%
\pgfpathlineto{\pgfqpoint{4.398743in}{0.399087in}}%
\pgfpathlineto{\pgfqpoint{4.405186in}{0.399807in}}%
\pgfpathlineto{\pgfqpoint{4.406260in}{0.400401in}}%
\pgfpathlineto{\pgfqpoint{4.409482in}{0.399401in}}%
\pgfpathlineto{\pgfqpoint{4.411629in}{0.397386in}}%
\pgfpathlineto{\pgfqpoint{4.412703in}{0.397226in}}%
\pgfpathlineto{\pgfqpoint{4.413777in}{0.395799in}}%
\pgfpathlineto{\pgfqpoint{4.418073in}{0.397812in}}%
\pgfpathlineto{\pgfqpoint{4.425589in}{0.397323in}}%
\pgfpathlineto{\pgfqpoint{4.428811in}{0.395510in}}%
\pgfpathlineto{\pgfqpoint{4.433106in}{0.394979in}}%
\pgfpathlineto{\pgfqpoint{4.434180in}{0.393306in}}%
\pgfpathlineto{\pgfqpoint{4.435254in}{0.393306in}}%
\pgfpathlineto{\pgfqpoint{4.436328in}{0.392749in}}%
\pgfpathlineto{\pgfqpoint{4.440623in}{0.393241in}}%
\pgfpathlineto{\pgfqpoint{4.442771in}{0.393383in}}%
\pgfpathlineto{\pgfqpoint{4.443845in}{0.392468in}}%
\pgfpathlineto{\pgfqpoint{4.447066in}{0.393345in}}%
\pgfpathlineto{\pgfqpoint{4.449214in}{0.392535in}}%
\pgfpathlineto{\pgfqpoint{4.450288in}{0.392745in}}%
\pgfpathlineto{\pgfqpoint{4.451361in}{0.392351in}}%
\pgfpathlineto{\pgfqpoint{4.457805in}{0.392074in}}%
\pgfpathlineto{\pgfqpoint{4.458878in}{0.391151in}}%
\pgfpathlineto{\pgfqpoint{4.463174in}{0.391003in}}%
\pgfpathlineto{\pgfqpoint{4.465321in}{0.392935in}}%
\pgfpathlineto{\pgfqpoint{4.466395in}{0.392836in}}%
\pgfpathlineto{\pgfqpoint{4.469617in}{0.393681in}}%
\pgfpathlineto{\pgfqpoint{4.472838in}{0.392381in}}%
\pgfpathlineto{\pgfqpoint{4.473912in}{0.391435in}}%
\pgfpathlineto{\pgfqpoint{4.477134in}{0.390853in}}%
\pgfpathlineto{\pgfqpoint{4.478207in}{0.390028in}}%
\pgfpathlineto{\pgfqpoint{4.480355in}{0.390210in}}%
\pgfpathlineto{\pgfqpoint{4.481429in}{0.388930in}}%
\pgfpathlineto{\pgfqpoint{4.484650in}{0.388628in}}%
\pgfpathlineto{\pgfqpoint{4.485724in}{0.389251in}}%
\pgfpathlineto{\pgfqpoint{4.488946in}{0.388704in}}%
\pgfpathlineto{\pgfqpoint{4.493241in}{0.390169in}}%
\pgfpathlineto{\pgfqpoint{4.494315in}{0.389425in}}%
\pgfpathlineto{\pgfqpoint{4.495389in}{0.390153in}}%
\pgfpathlineto{\pgfqpoint{4.496463in}{0.389591in}}%
\pgfpathlineto{\pgfqpoint{4.499684in}{0.390259in}}%
\pgfpathlineto{\pgfqpoint{4.500758in}{0.389761in}}%
\pgfpathlineto{\pgfqpoint{4.501832in}{0.389955in}}%
\pgfpathlineto{\pgfqpoint{4.502906in}{0.389552in}}%
\pgfpathlineto{\pgfqpoint{4.507201in}{0.389681in}}%
\pgfpathlineto{\pgfqpoint{4.508275in}{0.388780in}}%
\pgfpathlineto{\pgfqpoint{4.509349in}{0.389018in}}%
\pgfpathlineto{\pgfqpoint{4.519013in}{0.385366in}}%
\pgfpathlineto{\pgfqpoint{4.522235in}{0.387091in}}%
\pgfpathlineto{\pgfqpoint{4.523309in}{0.385054in}}%
\pgfpathlineto{\pgfqpoint{4.524383in}{0.385132in}}%
\pgfpathlineto{\pgfqpoint{4.525456in}{0.385838in}}%
\pgfpathlineto{\pgfqpoint{4.526530in}{0.384187in}}%
\pgfpathlineto{\pgfqpoint{4.532973in}{0.383740in}}%
\pgfpathlineto{\pgfqpoint{4.534047in}{0.383869in}}%
\pgfpathlineto{\pgfqpoint{4.541564in}{0.381072in}}%
\pgfpathlineto{\pgfqpoint{4.544785in}{0.380837in}}%
\pgfpathlineto{\pgfqpoint{4.545859in}{0.379942in}}%
\pgfpathlineto{\pgfqpoint{4.546933in}{0.380008in}}%
\pgfpathlineto{\pgfqpoint{4.549081in}{0.378967in}}%
\pgfpathlineto{\pgfqpoint{4.554450in}{0.378577in}}%
\pgfpathlineto{\pgfqpoint{4.555524in}{0.379566in}}%
\pgfpathlineto{\pgfqpoint{4.556598in}{0.379473in}}%
\pgfpathlineto{\pgfqpoint{4.564115in}{0.380410in}}%
\pgfpathlineto{\pgfqpoint{4.568410in}{0.379354in}}%
\pgfpathlineto{\pgfqpoint{4.569484in}{0.379715in}}%
\pgfpathlineto{\pgfqpoint{4.571631in}{0.382509in}}%
\pgfpathlineto{\pgfqpoint{4.575927in}{0.381322in}}%
\pgfpathlineto{\pgfqpoint{4.577001in}{0.382434in}}%
\pgfpathlineto{\pgfqpoint{4.578074in}{0.382423in}}%
\pgfpathlineto{\pgfqpoint{4.579148in}{0.380806in}}%
\pgfpathlineto{\pgfqpoint{4.584517in}{0.382489in}}%
\pgfpathlineto{\pgfqpoint{4.585591in}{0.381259in}}%
\pgfpathlineto{\pgfqpoint{4.586665in}{0.381705in}}%
\pgfpathlineto{\pgfqpoint{4.592034in}{0.381198in}}%
\pgfpathlineto{\pgfqpoint{4.593108in}{0.383327in}}%
\pgfpathlineto{\pgfqpoint{4.594182in}{0.382592in}}%
\pgfpathlineto{\pgfqpoint{4.598477in}{0.382746in}}%
\pgfpathlineto{\pgfqpoint{4.600625in}{0.381659in}}%
\pgfpathlineto{\pgfqpoint{4.601699in}{0.382183in}}%
\pgfpathlineto{\pgfqpoint{4.604920in}{0.381444in}}%
\pgfpathlineto{\pgfqpoint{4.605994in}{0.382064in}}%
\pgfpathlineto{\pgfqpoint{4.609216in}{0.381535in}}%
\pgfpathlineto{\pgfqpoint{4.614585in}{0.380546in}}%
\pgfpathlineto{\pgfqpoint{4.615659in}{0.379061in}}%
\pgfpathlineto{\pgfqpoint{4.616733in}{0.378720in}}%
\pgfpathlineto{\pgfqpoint{4.622102in}{0.380084in}}%
\pgfpathlineto{\pgfqpoint{4.623176in}{0.379683in}}%
\pgfpathlineto{\pgfqpoint{4.624249in}{0.380249in}}%
\pgfpathlineto{\pgfqpoint{4.628545in}{0.380873in}}%
\pgfpathlineto{\pgfqpoint{4.631766in}{0.380392in}}%
\pgfpathlineto{\pgfqpoint{4.637136in}{0.380686in}}%
\pgfpathlineto{\pgfqpoint{4.639283in}{0.379038in}}%
\pgfpathlineto{\pgfqpoint{4.642505in}{0.379453in}}%
\pgfpathlineto{\pgfqpoint{4.643579in}{0.378573in}}%
\pgfpathlineto{\pgfqpoint{4.644652in}{0.379517in}}%
\pgfpathlineto{\pgfqpoint{4.645726in}{0.379657in}}%
\pgfpathlineto{\pgfqpoint{4.646800in}{0.380535in}}%
\pgfpathlineto{\pgfqpoint{4.652169in}{0.381225in}}%
\pgfpathlineto{\pgfqpoint{4.653243in}{0.382660in}}%
\pgfpathlineto{\pgfqpoint{4.659686in}{0.383867in}}%
\pgfpathlineto{\pgfqpoint{4.661834in}{0.383279in}}%
\pgfpathlineto{\pgfqpoint{4.673646in}{0.384058in}}%
\pgfpathlineto{\pgfqpoint{4.680089in}{0.383394in}}%
\pgfpathlineto{\pgfqpoint{4.684384in}{0.378683in}}%
\pgfpathlineto{\pgfqpoint{4.689754in}{0.378337in}}%
\pgfpathlineto{\pgfqpoint{4.691901in}{0.380095in}}%
\pgfpathlineto{\pgfqpoint{4.697271in}{0.380586in}}%
\pgfpathlineto{\pgfqpoint{4.698344in}{0.379882in}}%
\pgfpathlineto{\pgfqpoint{4.704787in}{0.380181in}}%
\pgfpathlineto{\pgfqpoint{4.705861in}{0.380943in}}%
\pgfpathlineto{\pgfqpoint{4.706935in}{0.379784in}}%
\pgfpathlineto{\pgfqpoint{4.712304in}{0.381294in}}%
\pgfpathlineto{\pgfqpoint{4.713378in}{0.379468in}}%
\pgfpathlineto{\pgfqpoint{4.714452in}{0.378931in}}%
\pgfpathlineto{\pgfqpoint{4.718747in}{0.378730in}}%
\pgfpathlineto{\pgfqpoint{4.720895in}{0.378749in}}%
\pgfpathlineto{\pgfqpoint{4.721969in}{0.378133in}}%
\pgfpathlineto{\pgfqpoint{4.725190in}{0.377756in}}%
\pgfpathlineto{\pgfqpoint{4.726264in}{0.376630in}}%
\pgfpathlineto{\pgfqpoint{4.727338in}{0.377310in}}%
\pgfpathlineto{\pgfqpoint{4.729486in}{0.376407in}}%
\pgfpathlineto{\pgfqpoint{4.741298in}{0.376716in}}%
\pgfpathlineto{\pgfqpoint{4.742372in}{0.376604in}}%
\pgfpathlineto{\pgfqpoint{4.743446in}{0.378042in}}%
\pgfpathlineto{\pgfqpoint{4.744519in}{0.376866in}}%
\pgfpathlineto{\pgfqpoint{4.748815in}{0.377506in}}%
\pgfpathlineto{\pgfqpoint{4.750962in}{0.375722in}}%
\pgfpathlineto{\pgfqpoint{4.755258in}{0.376117in}}%
\pgfpathlineto{\pgfqpoint{4.757405in}{0.376326in}}%
\pgfpathlineto{\pgfqpoint{4.759553in}{0.375579in}}%
\pgfpathlineto{\pgfqpoint{4.762775in}{0.375924in}}%
\pgfpathlineto{\pgfqpoint{4.763849in}{0.374827in}}%
\pgfpathlineto{\pgfqpoint{4.767070in}{0.375299in}}%
\pgfpathlineto{\pgfqpoint{4.770292in}{0.375023in}}%
\pgfpathlineto{\pgfqpoint{4.771365in}{0.375762in}}%
\pgfpathlineto{\pgfqpoint{4.773513in}{0.375515in}}%
\pgfpathlineto{\pgfqpoint{4.774587in}{0.374598in}}%
\pgfpathlineto{\pgfqpoint{4.782104in}{0.375296in}}%
\pgfpathlineto{\pgfqpoint{4.786399in}{0.374520in}}%
\pgfpathlineto{\pgfqpoint{4.788547in}{0.372856in}}%
\pgfpathlineto{\pgfqpoint{4.789621in}{0.373248in}}%
\pgfpathlineto{\pgfqpoint{4.804654in}{0.370752in}}%
\pgfpathlineto{\pgfqpoint{4.812171in}{0.371187in}}%
\pgfpathlineto{\pgfqpoint{4.815393in}{0.372542in}}%
\pgfpathlineto{\pgfqpoint{4.816467in}{0.371745in}}%
\pgfpathlineto{\pgfqpoint{4.819688in}{0.372058in}}%
\pgfpathlineto{\pgfqpoint{4.823983in}{0.371907in}}%
\pgfpathlineto{\pgfqpoint{4.825057in}{0.371180in}}%
\pgfpathlineto{\pgfqpoint{4.826131in}{0.371513in}}%
\pgfpathlineto{\pgfqpoint{4.827205in}{0.372853in}}%
\pgfpathlineto{\pgfqpoint{4.832574in}{0.373799in}}%
\pgfpathlineto{\pgfqpoint{4.833648in}{0.372451in}}%
\pgfpathlineto{\pgfqpoint{4.837943in}{0.374665in}}%
\pgfpathlineto{\pgfqpoint{4.840091in}{0.373371in}}%
\pgfpathlineto{\pgfqpoint{4.841165in}{0.370699in}}%
\pgfpathlineto{\pgfqpoint{4.842239in}{0.370631in}}%
\pgfpathlineto{\pgfqpoint{4.849756in}{0.368459in}}%
\pgfpathlineto{\pgfqpoint{4.864789in}{0.367547in}}%
\pgfpathlineto{\pgfqpoint{4.868011in}{0.368292in}}%
\pgfpathlineto{\pgfqpoint{4.870159in}{0.366485in}}%
\pgfpathlineto{\pgfqpoint{4.871232in}{0.366138in}}%
\pgfpathlineto{\pgfqpoint{4.872306in}{0.366726in}}%
\pgfpathlineto{\pgfqpoint{4.877675in}{0.367167in}}%
\pgfpathlineto{\pgfqpoint{4.878749in}{0.367954in}}%
\pgfpathlineto{\pgfqpoint{4.883045in}{0.367135in}}%
\pgfpathlineto{\pgfqpoint{4.884118in}{0.366807in}}%
\pgfpathlineto{\pgfqpoint{4.885192in}{0.367593in}}%
\pgfpathlineto{\pgfqpoint{4.887340in}{0.367687in}}%
\pgfpathlineto{\pgfqpoint{4.894857in}{0.368444in}}%
\pgfpathlineto{\pgfqpoint{4.898078in}{0.367822in}}%
\pgfpathlineto{\pgfqpoint{4.899152in}{0.366980in}}%
\pgfpathlineto{\pgfqpoint{4.901300in}{0.366916in}}%
\pgfpathlineto{\pgfqpoint{4.902374in}{0.367596in}}%
\pgfpathlineto{\pgfqpoint{4.906669in}{0.368062in}}%
\pgfpathlineto{\pgfqpoint{4.907743in}{0.367554in}}%
\pgfpathlineto{\pgfqpoint{4.908817in}{0.369195in}}%
\pgfpathlineto{\pgfqpoint{4.909891in}{0.369491in}}%
\pgfpathlineto{\pgfqpoint{4.915260in}{0.368336in}}%
\pgfpathlineto{\pgfqpoint{4.916334in}{0.367733in}}%
\pgfpathlineto{\pgfqpoint{4.922777in}{0.367995in}}%
\pgfpathlineto{\pgfqpoint{4.923850in}{0.367952in}}%
\pgfpathlineto{\pgfqpoint{4.924924in}{0.368722in}}%
\pgfpathlineto{\pgfqpoint{4.928146in}{0.368987in}}%
\pgfpathlineto{\pgfqpoint{4.932441in}{0.367524in}}%
\pgfpathlineto{\pgfqpoint{4.936737in}{0.367120in}}%
\pgfpathlineto{\pgfqpoint{4.937810in}{0.367527in}}%
\pgfpathlineto{\pgfqpoint{4.939958in}{0.366609in}}%
\pgfpathlineto{\pgfqpoint{4.944253in}{0.366538in}}%
\pgfpathlineto{\pgfqpoint{4.946401in}{0.367571in}}%
\pgfpathlineto{\pgfqpoint{4.953918in}{0.366371in}}%
\pgfpathlineto{\pgfqpoint{4.954992in}{0.365858in}}%
\pgfpathlineto{\pgfqpoint{4.975395in}{0.365337in}}%
\pgfpathlineto{\pgfqpoint{4.976469in}{0.366098in}}%
\pgfpathlineto{\pgfqpoint{4.989355in}{0.366151in}}%
\pgfpathlineto{\pgfqpoint{4.992576in}{0.364812in}}%
\pgfpathlineto{\pgfqpoint{5.010831in}{0.364083in}}%
\pgfpathlineto{\pgfqpoint{5.014053in}{0.364918in}}%
\pgfpathlineto{\pgfqpoint{5.015127in}{0.364529in}}%
\pgfpathlineto{\pgfqpoint{5.022644in}{0.364325in}}%
\pgfpathlineto{\pgfqpoint{5.025865in}{0.363879in}}%
\pgfpathlineto{\pgfqpoint{5.026939in}{0.364325in}}%
\pgfpathlineto{\pgfqpoint{5.028013in}{0.363843in}}%
\pgfpathlineto{\pgfqpoint{5.030160in}{0.364706in}}%
\pgfpathlineto{\pgfqpoint{5.034456in}{0.364025in}}%
\pgfpathlineto{\pgfqpoint{5.037677in}{0.363963in}}%
\pgfpathlineto{\pgfqpoint{5.059154in}{0.362409in}}%
\pgfpathlineto{\pgfqpoint{5.060228in}{0.362565in}}%
\pgfpathlineto{\pgfqpoint{5.067745in}{0.362093in}}%
\pgfpathlineto{\pgfqpoint{5.081705in}{0.362343in}}%
\pgfpathlineto{\pgfqpoint{5.082779in}{0.362277in}}%
\pgfpathlineto{\pgfqpoint{5.090295in}{0.363022in}}%
\pgfpathlineto{\pgfqpoint{5.094591in}{0.362891in}}%
\pgfpathlineto{\pgfqpoint{5.096738in}{0.363910in}}%
\pgfpathlineto{\pgfqpoint{5.097812in}{0.363125in}}%
\pgfpathlineto{\pgfqpoint{5.101034in}{0.363076in}}%
\pgfpathlineto{\pgfqpoint{5.102108in}{0.363732in}}%
\pgfpathlineto{\pgfqpoint{5.103181in}{0.363266in}}%
\pgfpathlineto{\pgfqpoint{5.104255in}{0.364364in}}%
\pgfpathlineto{\pgfqpoint{5.105329in}{0.364092in}}%
\pgfpathlineto{\pgfqpoint{5.111772in}{0.366312in}}%
\pgfpathlineto{\pgfqpoint{5.112846in}{0.365205in}}%
\pgfpathlineto{\pgfqpoint{5.116068in}{0.364350in}}%
\pgfpathlineto{\pgfqpoint{5.117141in}{0.363372in}}%
\pgfpathlineto{\pgfqpoint{5.118215in}{0.363577in}}%
\pgfpathlineto{\pgfqpoint{5.120363in}{0.362877in}}%
\pgfpathlineto{\pgfqpoint{5.123584in}{0.362950in}}%
\pgfpathlineto{\pgfqpoint{5.124658in}{0.362281in}}%
\pgfpathlineto{\pgfqpoint{5.126806in}{0.362150in}}%
\pgfpathlineto{\pgfqpoint{5.127880in}{0.361648in}}%
\pgfpathlineto{\pgfqpoint{5.131101in}{0.361506in}}%
\pgfpathlineto{\pgfqpoint{5.132175in}{0.362079in}}%
\pgfpathlineto{\pgfqpoint{5.134323in}{0.361369in}}%
\pgfpathlineto{\pgfqpoint{5.135397in}{0.362208in}}%
\pgfpathlineto{\pgfqpoint{5.141840in}{0.361990in}}%
\pgfpathlineto{\pgfqpoint{5.142914in}{0.361855in}}%
\pgfpathlineto{\pgfqpoint{5.153652in}{0.361934in}}%
\pgfpathlineto{\pgfqpoint{5.155800in}{0.361360in}}%
\pgfpathlineto{\pgfqpoint{5.165464in}{0.360597in}}%
\pgfpathlineto{\pgfqpoint{5.169759in}{0.360926in}}%
\pgfpathlineto{\pgfqpoint{5.170833in}{0.361465in}}%
\pgfpathlineto{\pgfqpoint{5.171907in}{0.360924in}}%
\pgfpathlineto{\pgfqpoint{5.178350in}{0.361064in}}%
\pgfpathlineto{\pgfqpoint{5.180498in}{0.360439in}}%
\pgfpathlineto{\pgfqpoint{5.188015in}{0.359583in}}%
\pgfpathlineto{\pgfqpoint{5.192310in}{0.359708in}}%
\pgfpathlineto{\pgfqpoint{5.199827in}{0.360819in}}%
\pgfpathlineto{\pgfqpoint{5.203048in}{0.359863in}}%
\pgfpathlineto{\pgfqpoint{5.221304in}{0.359569in}}%
\pgfpathlineto{\pgfqpoint{5.225599in}{0.361149in}}%
\pgfpathlineto{\pgfqpoint{5.228821in}{0.360741in}}%
\pgfpathlineto{\pgfqpoint{5.229894in}{0.359846in}}%
\pgfpathlineto{\pgfqpoint{5.230968in}{0.357031in}}%
\pgfpathlineto{\pgfqpoint{5.232042in}{0.356575in}}%
\pgfpathlineto{\pgfqpoint{5.236337in}{0.356873in}}%
\pgfpathlineto{\pgfqpoint{5.262110in}{0.355773in}}%
\pgfpathlineto{\pgfqpoint{5.263183in}{0.356153in}}%
\pgfpathlineto{\pgfqpoint{5.266405in}{0.355692in}}%
\pgfpathlineto{\pgfqpoint{5.268553in}{0.356432in}}%
\pgfpathlineto{\pgfqpoint{5.269626in}{0.355039in}}%
\pgfpathlineto{\pgfqpoint{5.270700in}{0.355257in}}%
\pgfpathlineto{\pgfqpoint{5.281439in}{0.354715in}}%
\pgfpathlineto{\pgfqpoint{5.285734in}{0.355524in}}%
\pgfpathlineto{\pgfqpoint{5.292177in}{0.355356in}}%
\pgfpathlineto{\pgfqpoint{5.305063in}{0.355154in}}%
\pgfpathlineto{\pgfqpoint{5.321171in}{0.354206in}}%
\pgfpathlineto{\pgfqpoint{5.322245in}{0.354516in}}%
\pgfpathlineto{\pgfqpoint{5.323318in}{0.353996in}}%
\pgfpathlineto{\pgfqpoint{5.328688in}{0.354218in}}%
\pgfpathlineto{\pgfqpoint{5.334057in}{0.354532in}}%
\pgfpathlineto{\pgfqpoint{5.338352in}{0.354041in}}%
\pgfpathlineto{\pgfqpoint{5.364124in}{0.354318in}}%
\pgfpathlineto{\pgfqpoint{5.365198in}{0.354539in}}%
\pgfpathlineto{\pgfqpoint{5.367346in}{0.353932in}}%
\pgfpathlineto{\pgfqpoint{5.368420in}{0.354120in}}%
\pgfpathlineto{\pgfqpoint{5.373789in}{0.353688in}}%
\pgfpathlineto{\pgfqpoint{5.374863in}{0.353211in}}%
\pgfpathlineto{\pgfqpoint{5.375936in}{0.353371in}}%
\pgfpathlineto{\pgfqpoint{5.383453in}{0.352738in}}%
\pgfpathlineto{\pgfqpoint{5.387749in}{0.352947in}}%
\pgfpathlineto{\pgfqpoint{5.389896in}{0.352553in}}%
\pgfpathlineto{\pgfqpoint{5.425333in}{0.350885in}}%
\pgfpathlineto{\pgfqpoint{5.426407in}{0.353448in}}%
\pgfpathlineto{\pgfqpoint{5.427481in}{0.353995in}}%
\pgfpathlineto{\pgfqpoint{5.431776in}{0.353301in}}%
\pgfpathlineto{\pgfqpoint{5.433924in}{0.354413in}}%
\pgfpathlineto{\pgfqpoint{5.439293in}{0.353808in}}%
\pgfpathlineto{\pgfqpoint{5.441441in}{0.354549in}}%
\pgfpathlineto{\pgfqpoint{5.442514in}{0.356461in}}%
\pgfpathlineto{\pgfqpoint{5.447884in}{0.357854in}}%
\pgfpathlineto{\pgfqpoint{5.450031in}{0.355640in}}%
\pgfpathlineto{\pgfqpoint{5.454327in}{0.355730in}}%
\pgfpathlineto{\pgfqpoint{5.455401in}{0.356495in}}%
\pgfpathlineto{\pgfqpoint{5.456474in}{0.355690in}}%
\pgfpathlineto{\pgfqpoint{5.465065in}{0.355416in}}%
\pgfpathlineto{\pgfqpoint{5.466139in}{0.354822in}}%
\pgfpathlineto{\pgfqpoint{5.481173in}{0.356132in}}%
\pgfpathlineto{\pgfqpoint{5.485468in}{0.356412in}}%
\pgfpathlineto{\pgfqpoint{5.486542in}{0.355476in}}%
\pgfpathlineto{\pgfqpoint{5.492985in}{0.354981in}}%
\pgfpathlineto{\pgfqpoint{5.495133in}{0.354719in}}%
\pgfpathlineto{\pgfqpoint{5.496206in}{0.354429in}}%
\pgfpathlineto{\pgfqpoint{5.509092in}{0.353113in}}%
\pgfpathlineto{\pgfqpoint{5.510166in}{0.352247in}}%
\pgfpathlineto{\pgfqpoint{5.516609in}{0.351965in}}%
\pgfpathlineto{\pgfqpoint{5.518757in}{0.352115in}}%
\pgfpathlineto{\pgfqpoint{5.523052in}{0.351655in}}%
\pgfpathlineto{\pgfqpoint{5.525200in}{0.352290in}}%
\pgfpathlineto{\pgfqpoint{5.526274in}{0.351596in}}%
\pgfpathlineto{\pgfqpoint{5.532717in}{0.351458in}}%
\pgfpathlineto{\pgfqpoint{5.533791in}{0.351793in}}%
\pgfpathlineto{\pgfqpoint{5.540234in}{0.350841in}}%
\pgfpathlineto{\pgfqpoint{5.541308in}{0.350522in}}%
\pgfpathlineto{\pgfqpoint{5.546677in}{0.350841in}}%
\pgfpathlineto{\pgfqpoint{5.548824in}{0.351674in}}%
\pgfpathlineto{\pgfqpoint{5.552046in}{0.352087in}}%
\pgfpathlineto{\pgfqpoint{5.553120in}{0.351594in}}%
\pgfpathlineto{\pgfqpoint{5.555267in}{0.352476in}}%
\pgfpathlineto{\pgfqpoint{5.556341in}{0.351857in}}%
\pgfpathlineto{\pgfqpoint{5.562784in}{0.352568in}}%
\pgfpathlineto{\pgfqpoint{5.563858in}{0.353295in}}%
\pgfpathlineto{\pgfqpoint{5.567080in}{0.352928in}}%
\pgfpathlineto{\pgfqpoint{5.569227in}{0.351731in}}%
\pgfpathlineto{\pgfqpoint{5.570301in}{0.352182in}}%
\pgfpathlineto{\pgfqpoint{5.571375in}{0.353301in}}%
\pgfpathlineto{\pgfqpoint{5.577818in}{0.353829in}}%
\pgfpathlineto{\pgfqpoint{5.584261in}{0.353681in}}%
\pgfpathlineto{\pgfqpoint{5.585335in}{0.354047in}}%
\pgfpathlineto{\pgfqpoint{5.591778in}{0.355483in}}%
\pgfpathlineto{\pgfqpoint{5.593926in}{0.355843in}}%
\pgfpathlineto{\pgfqpoint{5.598221in}{0.355116in}}%
\pgfpathlineto{\pgfqpoint{5.599295in}{0.356062in}}%
\pgfpathlineto{\pgfqpoint{5.600369in}{0.355853in}}%
\pgfpathlineto{\pgfqpoint{5.601443in}{0.357582in}}%
\pgfpathlineto{\pgfqpoint{5.607886in}{0.357522in}}%
\pgfpathlineto{\pgfqpoint{5.608959in}{0.356460in}}%
\pgfpathlineto{\pgfqpoint{5.615402in}{0.357639in}}%
\pgfpathlineto{\pgfqpoint{5.616476in}{0.356790in}}%
\pgfpathlineto{\pgfqpoint{5.619698in}{0.357025in}}%
\pgfpathlineto{\pgfqpoint{5.620772in}{0.357753in}}%
\pgfpathlineto{\pgfqpoint{5.622919in}{0.356895in}}%
\pgfpathlineto{\pgfqpoint{5.623993in}{0.357439in}}%
\pgfpathlineto{\pgfqpoint{5.628289in}{0.358017in}}%
\pgfpathlineto{\pgfqpoint{5.629362in}{0.359334in}}%
\pgfpathlineto{\pgfqpoint{5.631510in}{0.358404in}}%
\pgfpathlineto{\pgfqpoint{5.635805in}{0.357719in}}%
\pgfpathlineto{\pgfqpoint{5.636879in}{0.356749in}}%
\pgfpathlineto{\pgfqpoint{5.639027in}{0.356925in}}%
\pgfpathlineto{\pgfqpoint{5.643322in}{0.356783in}}%
\pgfpathlineto{\pgfqpoint{5.649765in}{0.356730in}}%
\pgfpathlineto{\pgfqpoint{5.650839in}{0.355977in}}%
\pgfpathlineto{\pgfqpoint{5.651913in}{0.356198in}}%
\pgfpathlineto{\pgfqpoint{5.652987in}{0.355573in}}%
\pgfpathlineto{\pgfqpoint{5.654061in}{0.355684in}}%
\pgfpathlineto{\pgfqpoint{5.657282in}{0.355383in}}%
\pgfpathlineto{\pgfqpoint{5.659430in}{0.355949in}}%
\pgfpathlineto{\pgfqpoint{5.661578in}{0.355852in}}%
\pgfpathlineto{\pgfqpoint{5.669094in}{0.355424in}}%
\pgfpathlineto{\pgfqpoint{5.680907in}{0.355759in}}%
\pgfpathlineto{\pgfqpoint{5.684128in}{0.355453in}}%
\pgfpathlineto{\pgfqpoint{5.689497in}{0.355973in}}%
\pgfpathlineto{\pgfqpoint{5.691645in}{0.356331in}}%
\pgfpathlineto{\pgfqpoint{5.695940in}{0.356009in}}%
\pgfpathlineto{\pgfqpoint{5.697014in}{0.355282in}}%
\pgfpathlineto{\pgfqpoint{5.699162in}{0.355574in}}%
\pgfpathlineto{\pgfqpoint{5.706679in}{0.353909in}}%
\pgfpathlineto{\pgfqpoint{5.713122in}{0.353826in}}%
\pgfpathlineto{\pgfqpoint{5.714196in}{0.354055in}}%
\pgfpathlineto{\pgfqpoint{5.726008in}{0.353066in}}%
\pgfpathlineto{\pgfqpoint{5.727082in}{0.354284in}}%
\pgfpathlineto{\pgfqpoint{5.735672in}{0.355509in}}%
\pgfpathlineto{\pgfqpoint{5.736746in}{0.355055in}}%
\pgfpathlineto{\pgfqpoint{5.743189in}{0.355043in}}%
\pgfpathlineto{\pgfqpoint{5.744263in}{0.354890in}}%
\pgfpathlineto{\pgfqpoint{5.759297in}{0.355852in}}%
\pgfpathlineto{\pgfqpoint{5.773257in}{0.355284in}}%
\pgfpathlineto{\pgfqpoint{5.774331in}{0.356362in}}%
\pgfpathlineto{\pgfqpoint{5.777552in}{0.356829in}}%
\pgfpathlineto{\pgfqpoint{5.780774in}{0.355623in}}%
\pgfpathlineto{\pgfqpoint{5.801177in}{0.354838in}}%
\pgfpathlineto{\pgfqpoint{5.803324in}{0.355306in}}%
\pgfpathlineto{\pgfqpoint{5.816210in}{0.356323in}}%
\pgfpathlineto{\pgfqpoint{5.817284in}{0.355945in}}%
\pgfpathlineto{\pgfqpoint{5.819432in}{0.356030in}}%
\pgfpathlineto{\pgfqpoint{5.823727in}{0.355737in}}%
\pgfpathlineto{\pgfqpoint{5.825875in}{0.355364in}}%
\pgfpathlineto{\pgfqpoint{5.826949in}{0.355672in}}%
\pgfpathlineto{\pgfqpoint{5.834466in}{0.355824in}}%
\pgfpathlineto{\pgfqpoint{5.867755in}{0.357135in}}%
\pgfpathlineto{\pgfqpoint{5.872050in}{0.356891in}}%
\pgfpathlineto{\pgfqpoint{5.878493in}{0.357428in}}%
\pgfpathlineto{\pgfqpoint{5.879567in}{0.357031in}}%
\pgfpathlineto{\pgfqpoint{5.899970in}{0.357368in}}%
\pgfpathlineto{\pgfqpoint{5.923594in}{0.356225in}}%
\pgfpathlineto{\pgfqpoint{5.924668in}{0.355272in}}%
\pgfpathlineto{\pgfqpoint{5.951514in}{0.354288in}}%
\pgfpathlineto{\pgfqpoint{5.954735in}{0.352776in}}%
\pgfpathlineto{\pgfqpoint{5.962252in}{0.353047in}}%
\pgfpathlineto{\pgfqpoint{5.968695in}{0.352609in}}%
\pgfpathlineto{\pgfqpoint{5.977286in}{0.352949in}}%
\pgfpathlineto{\pgfqpoint{5.990172in}{0.351493in}}%
\pgfpathlineto{\pgfqpoint{5.991246in}{0.351992in}}%
\pgfpathlineto{\pgfqpoint{5.992320in}{0.351849in}}%
\pgfpathlineto{\pgfqpoint{6.007354in}{0.351514in}}%
\pgfpathlineto{\pgfqpoint{6.013797in}{0.351164in}}%
\pgfpathlineto{\pgfqpoint{6.022387in}{0.351510in}}%
\pgfpathlineto{\pgfqpoint{6.063193in}{0.350623in}}%
\pgfpathlineto{\pgfqpoint{6.065341in}{0.350777in}}%
\pgfpathlineto{\pgfqpoint{6.082522in}{0.350651in}}%
\pgfpathlineto{\pgfqpoint{6.095408in}{0.350367in}}%
\pgfpathlineto{\pgfqpoint{6.097556in}{0.350193in}}%
\pgfpathlineto{\pgfqpoint{6.101851in}{0.350013in}}%
\pgfpathlineto{\pgfqpoint{6.105073in}{0.349915in}}%
\pgfpathlineto{\pgfqpoint{6.109368in}{0.350202in}}%
\pgfpathlineto{\pgfqpoint{6.111516in}{0.350970in}}%
\pgfpathlineto{\pgfqpoint{6.112590in}{0.350763in}}%
\pgfpathlineto{\pgfqpoint{6.124402in}{0.351779in}}%
\pgfpathlineto{\pgfqpoint{6.126550in}{0.352122in}}%
\pgfpathlineto{\pgfqpoint{6.127623in}{0.351892in}}%
\pgfpathlineto{\pgfqpoint{6.145879in}{0.352153in}}%
\pgfpathlineto{\pgfqpoint{6.149100in}{0.352759in}}%
\pgfpathlineto{\pgfqpoint{6.150174in}{0.352388in}}%
\pgfpathlineto{\pgfqpoint{6.155543in}{0.352238in}}%
\pgfpathlineto{\pgfqpoint{6.169503in}{0.352427in}}%
\pgfpathlineto{\pgfqpoint{6.170577in}{0.352023in}}%
\pgfpathlineto{\pgfqpoint{6.172725in}{0.352184in}}%
\pgfpathlineto{\pgfqpoint{6.175946in}{0.351852in}}%
\pgfpathlineto{\pgfqpoint{6.180242in}{0.352775in}}%
\pgfpathlineto{\pgfqpoint{6.192054in}{0.352083in}}%
\pgfpathlineto{\pgfqpoint{6.195275in}{0.351715in}}%
\pgfpathlineto{\pgfqpoint{6.200644in}{0.351752in}}%
\pgfpathlineto{\pgfqpoint{6.201718in}{0.350927in}}%
\pgfpathlineto{\pgfqpoint{6.210309in}{0.351512in}}%
\pgfpathlineto{\pgfqpoint{6.214604in}{0.351696in}}%
\pgfpathlineto{\pgfqpoint{6.216752in}{0.353244in}}%
\pgfpathlineto{\pgfqpoint{6.217826in}{0.353053in}}%
\pgfpathlineto{\pgfqpoint{6.228564in}{0.353289in}}%
\pgfpathlineto{\pgfqpoint{6.229638in}{0.352810in}}%
\pgfpathlineto{\pgfqpoint{6.231786in}{0.353172in}}%
\pgfpathlineto{\pgfqpoint{6.232860in}{0.352908in}}%
\pgfpathlineto{\pgfqpoint{6.245746in}{0.353167in}}%
\pgfpathlineto{\pgfqpoint{6.246820in}{0.354490in}}%
\pgfpathlineto{\pgfqpoint{6.259706in}{0.354041in}}%
\pgfpathlineto{\pgfqpoint{6.261853in}{0.353892in}}%
\pgfpathlineto{\pgfqpoint{6.262927in}{0.353986in}}%
\pgfpathlineto{\pgfqpoint{6.268296in}{0.353779in}}%
\pgfpathlineto{\pgfqpoint{6.270444in}{0.353983in}}%
\pgfpathlineto{\pgfqpoint{6.277961in}{0.353508in}}%
\pgfpathlineto{\pgfqpoint{6.283330in}{0.353974in}}%
\pgfpathlineto{\pgfqpoint{6.284404in}{0.354486in}}%
\pgfpathlineto{\pgfqpoint{6.292995in}{0.353696in}}%
\pgfpathlineto{\pgfqpoint{6.304807in}{0.354190in}}%
\pgfpathlineto{\pgfqpoint{6.305881in}{0.353802in}}%
\pgfpathlineto{\pgfqpoint{6.308028in}{0.353922in}}%
\pgfpathlineto{\pgfqpoint{6.318767in}{0.352084in}}%
\pgfpathlineto{\pgfqpoint{6.320914in}{0.352372in}}%
\pgfpathlineto{\pgfqpoint{6.334874in}{0.352448in}}%
\pgfpathlineto{\pgfqpoint{6.335948in}{0.351925in}}%
\pgfpathlineto{\pgfqpoint{6.338096in}{0.351921in}}%
\pgfpathlineto{\pgfqpoint{6.341317in}{0.350554in}}%
\pgfpathlineto{\pgfqpoint{6.343465in}{0.351771in}}%
\pgfpathlineto{\pgfqpoint{6.345613in}{0.351878in}}%
\pgfpathlineto{\pgfqpoint{6.353130in}{0.350007in}}%
\pgfpathlineto{\pgfqpoint{6.364942in}{0.350781in}}%
\pgfpathlineto{\pgfqpoint{6.368163in}{0.350939in}}%
\pgfpathlineto{\pgfqpoint{6.368163in}{0.350939in}}%
\pgfusepath{stroke}%
\end{pgfscope}%
\begin{pgfscope}%
\pgfsetrectcap%
\pgfsetmiterjoin%
\pgfsetlinewidth{0.803000pt}%
\definecolor{currentstroke}{rgb}{1.000000,1.000000,1.000000}%
\pgfsetstrokecolor{currentstroke}%
\pgfsetdash{}{0pt}%
\pgfpathmoveto{\pgfqpoint{3.902254in}{0.331635in}}%
\pgfpathlineto{\pgfqpoint{3.902254in}{0.732520in}}%
\pgfusepath{stroke}%
\end{pgfscope}%
\begin{pgfscope}%
\pgfsetrectcap%
\pgfsetmiterjoin%
\pgfsetlinewidth{0.803000pt}%
\definecolor{currentstroke}{rgb}{1.000000,1.000000,1.000000}%
\pgfsetstrokecolor{currentstroke}%
\pgfsetdash{}{0pt}%
\pgfpathmoveto{\pgfqpoint{6.485588in}{0.331635in}}%
\pgfpathlineto{\pgfqpoint{6.485588in}{0.732520in}}%
\pgfusepath{stroke}%
\end{pgfscope}%
\begin{pgfscope}%
\pgfsetrectcap%
\pgfsetmiterjoin%
\pgfsetlinewidth{0.803000pt}%
\definecolor{currentstroke}{rgb}{1.000000,1.000000,1.000000}%
\pgfsetstrokecolor{currentstroke}%
\pgfsetdash{}{0pt}%
\pgfpathmoveto{\pgfqpoint{3.902254in}{0.331635in}}%
\pgfpathlineto{\pgfqpoint{6.485588in}{0.331635in}}%
\pgfusepath{stroke}%
\end{pgfscope}%
\begin{pgfscope}%
\pgfsetrectcap%
\pgfsetmiterjoin%
\pgfsetlinewidth{0.803000pt}%
\definecolor{currentstroke}{rgb}{1.000000,1.000000,1.000000}%
\pgfsetstrokecolor{currentstroke}%
\pgfsetdash{}{0pt}%
\pgfpathmoveto{\pgfqpoint{3.902254in}{0.732520in}}%
\pgfpathlineto{\pgfqpoint{6.485588in}{0.732520in}}%
\pgfusepath{stroke}%
\end{pgfscope}%
\begin{pgfscope}%
\definecolor{textcolor}{rgb}{0.150000,0.150000,0.150000}%
\pgfsetstrokecolor{textcolor}%
\pgfsetfillcolor{textcolor}%
\pgftext[x=5.193921in,y=0.815853in,,base]{\color{textcolor}\rmfamily\fontsize{12.000000}{14.400000}\selectfont DIS}%
\end{pgfscope}%
\end{pgfpicture}%
\makeatother%
\endgroup%

        \end{adjustbox}
        %\caption{Flower two.}
    \end{minipage}
    \caption{Caption}
    \label{fig:mean_reversion_cum_mean}
\end{figure}{}

\subsubsection{Cumulative Mean - Returns}

\begin{figure}
    \centering
    \begin{adjustbox}{width=.7\textwidth,center}
        %% Creator: Matplotlib, PGF backend
%%
%% To include the figure in your LaTeX document, write
%%   \input{<filename>.pgf}
%%
%% Make sure the required packages are loaded in your preamble
%%   \usepackage{pgf}
%%
%% Figures using additional raster images can only be included by \input if
%% they are in the same directory as the main LaTeX file. For loading figures
%% from other directories you can use the `import` package
%%   \usepackage{import}
%% and then include the figures with
%%   \import{<path to file>}{<filename>.pgf}
%%
%% Matplotlib used the following preamble
%%   \usepackage{fontspec}
%%   \setmainfont{DejaVuSerif.ttf}[Path=/opt/tljh/user/lib/python3.6/site-packages/matplotlib/mpl-data/fonts/ttf/]
%%   \setsansfont{DejaVuSans.ttf}[Path=/opt/tljh/user/lib/python3.6/site-packages/matplotlib/mpl-data/fonts/ttf/]
%%   \setmonofont{DejaVuSansMono.ttf}[Path=/opt/tljh/user/lib/python3.6/site-packages/matplotlib/mpl-data/fonts/ttf/]
%%
\begingroup%
\makeatletter%
\begin{pgfpicture}%
\pgfpathrectangle{\pgfpointorigin}{\pgfqpoint{6.780630in}{5.881016in}}%
\pgfusepath{use as bounding box, clip}%
\begin{pgfscope}%
\pgfsetbuttcap%
\pgfsetmiterjoin%
\definecolor{currentfill}{rgb}{1.000000,1.000000,1.000000}%
\pgfsetfillcolor{currentfill}%
\pgfsetlinewidth{0.000000pt}%
\definecolor{currentstroke}{rgb}{1.000000,1.000000,1.000000}%
\pgfsetstrokecolor{currentstroke}%
\pgfsetdash{}{0pt}%
\pgfpathmoveto{\pgfqpoint{0.000000in}{0.000000in}}%
\pgfpathlineto{\pgfqpoint{6.780630in}{0.000000in}}%
\pgfpathlineto{\pgfqpoint{6.780630in}{5.881016in}}%
\pgfpathlineto{\pgfqpoint{0.000000in}{5.881016in}}%
\pgfpathclose%
\pgfusepath{fill}%
\end{pgfscope}%
\begin{pgfscope}%
\pgfsetbuttcap%
\pgfsetmiterjoin%
\definecolor{currentfill}{rgb}{0.917647,0.917647,0.949020}%
\pgfsetfillcolor{currentfill}%
\pgfsetlinewidth{0.000000pt}%
\definecolor{currentstroke}{rgb}{0.000000,0.000000,0.000000}%
\pgfsetstrokecolor{currentstroke}%
\pgfsetstrokeopacity{0.000000}%
\pgfsetdash{}{0pt}%
\pgfpathmoveto{\pgfqpoint{0.418102in}{4.180131in}}%
\pgfpathlineto{\pgfqpoint{3.001435in}{4.180131in}}%
\pgfpathlineto{\pgfqpoint{3.001435in}{4.581016in}}%
\pgfpathlineto{\pgfqpoint{0.418102in}{4.581016in}}%
\pgfpathclose%
\pgfusepath{fill}%
\end{pgfscope}%
\begin{pgfscope}%
\pgfpathrectangle{\pgfqpoint{0.418102in}{4.180131in}}{\pgfqpoint{2.583333in}{0.400885in}}%
\pgfusepath{clip}%
\pgfsetroundcap%
\pgfsetroundjoin%
\pgfsetlinewidth{0.803000pt}%
\definecolor{currentstroke}{rgb}{1.000000,1.000000,1.000000}%
\pgfsetstrokecolor{currentstroke}%
\pgfsetdash{}{0pt}%
\pgfpathmoveto{\pgfqpoint{0.533378in}{4.180131in}}%
\pgfpathlineto{\pgfqpoint{0.533378in}{4.581016in}}%
\pgfusepath{stroke}%
\end{pgfscope}%
\begin{pgfscope}%
\definecolor{textcolor}{rgb}{0.150000,0.150000,0.150000}%
\pgfsetstrokecolor{textcolor}%
\pgfsetfillcolor{textcolor}%
\pgftext[x=0.533378in,y=4.082908in,,top]{\color{textcolor}\rmfamily\fontsize{10.000000}{12.000000}\selectfont 2012}%
\end{pgfscope}%
\begin{pgfscope}%
\pgfpathrectangle{\pgfqpoint{0.418102in}{4.180131in}}{\pgfqpoint{2.583333in}{0.400885in}}%
\pgfusepath{clip}%
\pgfsetroundcap%
\pgfsetroundjoin%
\pgfsetlinewidth{0.803000pt}%
\definecolor{currentstroke}{rgb}{1.000000,1.000000,1.000000}%
\pgfsetstrokecolor{currentstroke}%
\pgfsetdash{}{0pt}%
\pgfpathmoveto{\pgfqpoint{0.926403in}{4.180131in}}%
\pgfpathlineto{\pgfqpoint{0.926403in}{4.581016in}}%
\pgfusepath{stroke}%
\end{pgfscope}%
\begin{pgfscope}%
\definecolor{textcolor}{rgb}{0.150000,0.150000,0.150000}%
\pgfsetstrokecolor{textcolor}%
\pgfsetfillcolor{textcolor}%
\pgftext[x=0.926403in,y=4.082908in,,top]{\color{textcolor}\rmfamily\fontsize{10.000000}{12.000000}\selectfont 2013}%
\end{pgfscope}%
\begin{pgfscope}%
\pgfpathrectangle{\pgfqpoint{0.418102in}{4.180131in}}{\pgfqpoint{2.583333in}{0.400885in}}%
\pgfusepath{clip}%
\pgfsetroundcap%
\pgfsetroundjoin%
\pgfsetlinewidth{0.803000pt}%
\definecolor{currentstroke}{rgb}{1.000000,1.000000,1.000000}%
\pgfsetstrokecolor{currentstroke}%
\pgfsetdash{}{0pt}%
\pgfpathmoveto{\pgfqpoint{1.318354in}{4.180131in}}%
\pgfpathlineto{\pgfqpoint{1.318354in}{4.581016in}}%
\pgfusepath{stroke}%
\end{pgfscope}%
\begin{pgfscope}%
\definecolor{textcolor}{rgb}{0.150000,0.150000,0.150000}%
\pgfsetstrokecolor{textcolor}%
\pgfsetfillcolor{textcolor}%
\pgftext[x=1.318354in,y=4.082908in,,top]{\color{textcolor}\rmfamily\fontsize{10.000000}{12.000000}\selectfont 2014}%
\end{pgfscope}%
\begin{pgfscope}%
\pgfpathrectangle{\pgfqpoint{0.418102in}{4.180131in}}{\pgfqpoint{2.583333in}{0.400885in}}%
\pgfusepath{clip}%
\pgfsetroundcap%
\pgfsetroundjoin%
\pgfsetlinewidth{0.803000pt}%
\definecolor{currentstroke}{rgb}{1.000000,1.000000,1.000000}%
\pgfsetstrokecolor{currentstroke}%
\pgfsetdash{}{0pt}%
\pgfpathmoveto{\pgfqpoint{1.710305in}{4.180131in}}%
\pgfpathlineto{\pgfqpoint{1.710305in}{4.581016in}}%
\pgfusepath{stroke}%
\end{pgfscope}%
\begin{pgfscope}%
\definecolor{textcolor}{rgb}{0.150000,0.150000,0.150000}%
\pgfsetstrokecolor{textcolor}%
\pgfsetfillcolor{textcolor}%
\pgftext[x=1.710305in,y=4.082908in,,top]{\color{textcolor}\rmfamily\fontsize{10.000000}{12.000000}\selectfont 2015}%
\end{pgfscope}%
\begin{pgfscope}%
\pgfpathrectangle{\pgfqpoint{0.418102in}{4.180131in}}{\pgfqpoint{2.583333in}{0.400885in}}%
\pgfusepath{clip}%
\pgfsetroundcap%
\pgfsetroundjoin%
\pgfsetlinewidth{0.803000pt}%
\definecolor{currentstroke}{rgb}{1.000000,1.000000,1.000000}%
\pgfsetstrokecolor{currentstroke}%
\pgfsetdash{}{0pt}%
\pgfpathmoveto{\pgfqpoint{2.102256in}{4.180131in}}%
\pgfpathlineto{\pgfqpoint{2.102256in}{4.581016in}}%
\pgfusepath{stroke}%
\end{pgfscope}%
\begin{pgfscope}%
\definecolor{textcolor}{rgb}{0.150000,0.150000,0.150000}%
\pgfsetstrokecolor{textcolor}%
\pgfsetfillcolor{textcolor}%
\pgftext[x=2.102256in,y=4.082908in,,top]{\color{textcolor}\rmfamily\fontsize{10.000000}{12.000000}\selectfont 2016}%
\end{pgfscope}%
\begin{pgfscope}%
\pgfpathrectangle{\pgfqpoint{0.418102in}{4.180131in}}{\pgfqpoint{2.583333in}{0.400885in}}%
\pgfusepath{clip}%
\pgfsetroundcap%
\pgfsetroundjoin%
\pgfsetlinewidth{0.803000pt}%
\definecolor{currentstroke}{rgb}{1.000000,1.000000,1.000000}%
\pgfsetstrokecolor{currentstroke}%
\pgfsetdash{}{0pt}%
\pgfpathmoveto{\pgfqpoint{2.495281in}{4.180131in}}%
\pgfpathlineto{\pgfqpoint{2.495281in}{4.581016in}}%
\pgfusepath{stroke}%
\end{pgfscope}%
\begin{pgfscope}%
\definecolor{textcolor}{rgb}{0.150000,0.150000,0.150000}%
\pgfsetstrokecolor{textcolor}%
\pgfsetfillcolor{textcolor}%
\pgftext[x=2.495281in,y=4.082908in,,top]{\color{textcolor}\rmfamily\fontsize{10.000000}{12.000000}\selectfont 2017}%
\end{pgfscope}%
\begin{pgfscope}%
\pgfpathrectangle{\pgfqpoint{0.418102in}{4.180131in}}{\pgfqpoint{2.583333in}{0.400885in}}%
\pgfusepath{clip}%
\pgfsetroundcap%
\pgfsetroundjoin%
\pgfsetlinewidth{0.803000pt}%
\definecolor{currentstroke}{rgb}{1.000000,1.000000,1.000000}%
\pgfsetstrokecolor{currentstroke}%
\pgfsetdash{}{0pt}%
\pgfpathmoveto{\pgfqpoint{2.887232in}{4.180131in}}%
\pgfpathlineto{\pgfqpoint{2.887232in}{4.581016in}}%
\pgfusepath{stroke}%
\end{pgfscope}%
\begin{pgfscope}%
\definecolor{textcolor}{rgb}{0.150000,0.150000,0.150000}%
\pgfsetstrokecolor{textcolor}%
\pgfsetfillcolor{textcolor}%
\pgftext[x=2.887232in,y=4.082908in,,top]{\color{textcolor}\rmfamily\fontsize{10.000000}{12.000000}\selectfont 2018}%
\end{pgfscope}%
\begin{pgfscope}%
\pgfpathrectangle{\pgfqpoint{0.418102in}{4.180131in}}{\pgfqpoint{2.583333in}{0.400885in}}%
\pgfusepath{clip}%
\pgfsetroundcap%
\pgfsetroundjoin%
\pgfsetlinewidth{0.803000pt}%
\definecolor{currentstroke}{rgb}{1.000000,1.000000,1.000000}%
\pgfsetstrokecolor{currentstroke}%
\pgfsetdash{}{0pt}%
\pgfpathmoveto{\pgfqpoint{0.418102in}{4.230473in}}%
\pgfpathlineto{\pgfqpoint{3.001435in}{4.230473in}}%
\pgfusepath{stroke}%
\end{pgfscope}%
\begin{pgfscope}%
\definecolor{textcolor}{rgb}{0.150000,0.150000,0.150000}%
\pgfsetstrokecolor{textcolor}%
\pgfsetfillcolor{textcolor}%
\pgftext[x=0.232514in,y=4.177711in,left,base]{\color{textcolor}\rmfamily\fontsize{10.000000}{12.000000}\selectfont 1}%
\end{pgfscope}%
\begin{pgfscope}%
\pgfpathrectangle{\pgfqpoint{0.418102in}{4.180131in}}{\pgfqpoint{2.583333in}{0.400885in}}%
\pgfusepath{clip}%
\pgfsetroundcap%
\pgfsetroundjoin%
\pgfsetlinewidth{0.803000pt}%
\definecolor{currentstroke}{rgb}{1.000000,1.000000,1.000000}%
\pgfsetstrokecolor{currentstroke}%
\pgfsetdash{}{0pt}%
\pgfpathmoveto{\pgfqpoint{0.418102in}{4.370194in}}%
\pgfpathlineto{\pgfqpoint{3.001435in}{4.370194in}}%
\pgfusepath{stroke}%
\end{pgfscope}%
\begin{pgfscope}%
\definecolor{textcolor}{rgb}{0.150000,0.150000,0.150000}%
\pgfsetstrokecolor{textcolor}%
\pgfsetfillcolor{textcolor}%
\pgftext[x=0.232514in,y=4.317433in,left,base]{\color{textcolor}\rmfamily\fontsize{10.000000}{12.000000}\selectfont 2}%
\end{pgfscope}%
\begin{pgfscope}%
\pgfpathrectangle{\pgfqpoint{0.418102in}{4.180131in}}{\pgfqpoint{2.583333in}{0.400885in}}%
\pgfusepath{clip}%
\pgfsetroundcap%
\pgfsetroundjoin%
\pgfsetlinewidth{0.803000pt}%
\definecolor{currentstroke}{rgb}{1.000000,1.000000,1.000000}%
\pgfsetstrokecolor{currentstroke}%
\pgfsetdash{}{0pt}%
\pgfpathmoveto{\pgfqpoint{0.418102in}{4.509915in}}%
\pgfpathlineto{\pgfqpoint{3.001435in}{4.509915in}}%
\pgfusepath{stroke}%
\end{pgfscope}%
\begin{pgfscope}%
\definecolor{textcolor}{rgb}{0.150000,0.150000,0.150000}%
\pgfsetstrokecolor{textcolor}%
\pgfsetfillcolor{textcolor}%
\pgftext[x=0.232514in,y=4.457154in,left,base]{\color{textcolor}\rmfamily\fontsize{10.000000}{12.000000}\selectfont 3}%
\end{pgfscope}%
\begin{pgfscope}%
\pgfpathrectangle{\pgfqpoint{0.418102in}{4.180131in}}{\pgfqpoint{2.583333in}{0.400885in}}%
\pgfusepath{clip}%
\pgfsetroundcap%
\pgfsetroundjoin%
\pgfsetlinewidth{1.505625pt}%
\definecolor{currentstroke}{rgb}{0.000000,0.000000,0.000000}%
\pgfsetstrokecolor{currentstroke}%
\pgfsetdash{}{0pt}%
\pgfpathmoveto{\pgfqpoint{0.535526in}{4.230473in}}%
\pgfpathlineto{\pgfqpoint{0.536600in}{4.231637in}}%
\pgfpathlineto{\pgfqpoint{0.538747in}{4.230289in}}%
\pgfpathlineto{\pgfqpoint{0.543043in}{4.231841in}}%
\pgfpathlineto{\pgfqpoint{0.544117in}{4.230942in}}%
\pgfpathlineto{\pgfqpoint{0.545190in}{4.231800in}}%
\pgfpathlineto{\pgfqpoint{0.546264in}{4.230657in}}%
\pgfpathlineto{\pgfqpoint{0.550560in}{4.231719in}}%
\pgfpathlineto{\pgfqpoint{0.552707in}{4.234353in}}%
\pgfpathlineto{\pgfqpoint{0.553781in}{4.234088in}}%
\pgfpathlineto{\pgfqpoint{0.558077in}{4.234558in}}%
\pgfpathlineto{\pgfqpoint{0.559150in}{4.235477in}}%
\pgfpathlineto{\pgfqpoint{0.560224in}{4.237335in}}%
\pgfpathlineto{\pgfqpoint{0.565593in}{4.235865in}}%
\pgfpathlineto{\pgfqpoint{0.566667in}{4.236947in}}%
\pgfpathlineto{\pgfqpoint{0.575258in}{4.238070in}}%
\pgfpathlineto{\pgfqpoint{0.576332in}{4.236600in}}%
\pgfpathlineto{\pgfqpoint{0.580627in}{4.238009in}}%
\pgfpathlineto{\pgfqpoint{0.581701in}{4.237356in}}%
\pgfpathlineto{\pgfqpoint{0.582775in}{4.238438in}}%
\pgfpathlineto{\pgfqpoint{0.583849in}{4.238275in}}%
\pgfpathlineto{\pgfqpoint{0.590292in}{4.238847in}}%
\pgfpathlineto{\pgfqpoint{0.591366in}{4.239357in}}%
\pgfpathlineto{\pgfqpoint{0.595661in}{4.238642in}}%
\pgfpathlineto{\pgfqpoint{0.602104in}{4.237437in}}%
\pgfpathlineto{\pgfqpoint{0.603178in}{4.233884in}}%
\pgfpathlineto{\pgfqpoint{0.604252in}{4.234762in}}%
\pgfpathlineto{\pgfqpoint{0.605325in}{4.236845in}}%
\pgfpathlineto{\pgfqpoint{0.606399in}{4.237008in}}%
\pgfpathlineto{\pgfqpoint{0.609621in}{4.238254in}}%
\pgfpathlineto{\pgfqpoint{0.610695in}{4.240297in}}%
\pgfpathlineto{\pgfqpoint{0.611768in}{4.240481in}}%
\pgfpathlineto{\pgfqpoint{0.612842in}{4.242400in}}%
\pgfpathlineto{\pgfqpoint{0.613916in}{4.241645in}}%
\pgfpathlineto{\pgfqpoint{0.617138in}{4.241951in}}%
\pgfpathlineto{\pgfqpoint{0.621433in}{4.239807in}}%
\pgfpathlineto{\pgfqpoint{0.625728in}{4.240950in}}%
\pgfpathlineto{\pgfqpoint{0.626802in}{4.239786in}}%
\pgfpathlineto{\pgfqpoint{0.628950in}{4.241052in}}%
\pgfpathlineto{\pgfqpoint{0.632171in}{4.241093in}}%
\pgfpathlineto{\pgfqpoint{0.633245in}{4.240358in}}%
\pgfpathlineto{\pgfqpoint{0.635393in}{4.237785in}}%
\pgfpathlineto{\pgfqpoint{0.639688in}{4.236212in}}%
\pgfpathlineto{\pgfqpoint{0.640762in}{4.233271in}}%
\pgfpathlineto{\pgfqpoint{0.641836in}{4.234558in}}%
\pgfpathlineto{\pgfqpoint{0.642910in}{4.237111in}}%
\pgfpathlineto{\pgfqpoint{0.643984in}{4.235129in}}%
\pgfpathlineto{\pgfqpoint{0.647205in}{4.236334in}}%
\pgfpathlineto{\pgfqpoint{0.648279in}{4.238091in}}%
\pgfpathlineto{\pgfqpoint{0.650427in}{4.237008in}}%
\pgfpathlineto{\pgfqpoint{0.651500in}{4.238152in}}%
\pgfpathlineto{\pgfqpoint{0.654722in}{4.237560in}}%
\pgfpathlineto{\pgfqpoint{0.655796in}{4.239847in}}%
\pgfpathlineto{\pgfqpoint{0.659017in}{4.241318in}}%
\pgfpathlineto{\pgfqpoint{0.665460in}{4.241359in}}%
\pgfpathlineto{\pgfqpoint{0.666534in}{4.240154in}}%
\pgfpathlineto{\pgfqpoint{0.672977in}{4.237560in}}%
\pgfpathlineto{\pgfqpoint{0.677273in}{4.235374in}}%
\pgfpathlineto{\pgfqpoint{0.679420in}{4.235558in}}%
\pgfpathlineto{\pgfqpoint{0.681568in}{4.232433in}}%
\pgfpathlineto{\pgfqpoint{0.684789in}{4.234047in}}%
\pgfpathlineto{\pgfqpoint{0.685863in}{4.233475in}}%
\pgfpathlineto{\pgfqpoint{0.686937in}{4.234619in}}%
\pgfpathlineto{\pgfqpoint{0.688011in}{4.234925in}}%
\pgfpathlineto{\pgfqpoint{0.689085in}{4.234578in}}%
\pgfpathlineto{\pgfqpoint{0.693380in}{4.236191in}}%
\pgfpathlineto{\pgfqpoint{0.694454in}{4.234026in}}%
\pgfpathlineto{\pgfqpoint{0.695528in}{4.233965in}}%
\pgfpathlineto{\pgfqpoint{0.696602in}{4.231310in}}%
\pgfpathlineto{\pgfqpoint{0.700897in}{4.230738in}}%
\pgfpathlineto{\pgfqpoint{0.701971in}{4.234333in}}%
\pgfpathlineto{\pgfqpoint{0.704119in}{4.236661in}}%
\pgfpathlineto{\pgfqpoint{0.707340in}{4.235272in}}%
\pgfpathlineto{\pgfqpoint{0.708414in}{4.237887in}}%
\pgfpathlineto{\pgfqpoint{0.709488in}{4.236886in}}%
\pgfpathlineto{\pgfqpoint{0.711635in}{4.239092in}}%
\pgfpathlineto{\pgfqpoint{0.714857in}{4.238887in}}%
\pgfpathlineto{\pgfqpoint{0.715931in}{4.239745in}}%
\pgfpathlineto{\pgfqpoint{0.717005in}{4.239275in}}%
\pgfpathlineto{\pgfqpoint{0.718078in}{4.237887in}}%
\pgfpathlineto{\pgfqpoint{0.719152in}{4.238070in}}%
\pgfpathlineto{\pgfqpoint{0.722374in}{4.236396in}}%
\pgfpathlineto{\pgfqpoint{0.723448in}{4.236968in}}%
\pgfpathlineto{\pgfqpoint{0.724521in}{4.238622in}}%
\pgfpathlineto{\pgfqpoint{0.725595in}{4.238622in}}%
\pgfpathlineto{\pgfqpoint{0.726669in}{4.242768in}}%
\pgfpathlineto{\pgfqpoint{0.729891in}{4.242217in}}%
\pgfpathlineto{\pgfqpoint{0.730965in}{4.242931in}}%
\pgfpathlineto{\pgfqpoint{0.733112in}{4.242707in}}%
\pgfpathlineto{\pgfqpoint{0.734186in}{4.241726in}}%
\pgfpathlineto{\pgfqpoint{0.737408in}{4.241686in}}%
\pgfpathlineto{\pgfqpoint{0.740629in}{4.237356in}}%
\pgfpathlineto{\pgfqpoint{0.741703in}{4.239357in}}%
\pgfpathlineto{\pgfqpoint{0.744924in}{4.240215in}}%
\pgfpathlineto{\pgfqpoint{0.745998in}{4.241706in}}%
\pgfpathlineto{\pgfqpoint{0.747072in}{4.244913in}}%
\pgfpathlineto{\pgfqpoint{0.748146in}{4.244851in}}%
\pgfpathlineto{\pgfqpoint{0.749220in}{4.243422in}}%
\pgfpathlineto{\pgfqpoint{0.752441in}{4.242360in}}%
\pgfpathlineto{\pgfqpoint{0.753515in}{4.240460in}}%
\pgfpathlineto{\pgfqpoint{0.754589in}{4.241318in}}%
\pgfpathlineto{\pgfqpoint{0.756737in}{4.246342in}}%
\pgfpathlineto{\pgfqpoint{0.762106in}{4.245484in}}%
\pgfpathlineto{\pgfqpoint{0.763180in}{4.243585in}}%
\pgfpathlineto{\pgfqpoint{0.764254in}{4.246322in}}%
\pgfpathlineto{\pgfqpoint{0.767475in}{4.245832in}}%
\pgfpathlineto{\pgfqpoint{0.768549in}{4.246322in}}%
\pgfpathlineto{\pgfqpoint{0.770697in}{4.246138in}}%
\pgfpathlineto{\pgfqpoint{0.771770in}{4.247323in}}%
\pgfpathlineto{\pgfqpoint{0.777140in}{4.247752in}}%
\pgfpathlineto{\pgfqpoint{0.778213in}{4.249794in}}%
\pgfpathlineto{\pgfqpoint{0.779287in}{4.250631in}}%
\pgfpathlineto{\pgfqpoint{0.782509in}{4.250039in}}%
\pgfpathlineto{\pgfqpoint{0.783583in}{4.248895in}}%
\pgfpathlineto{\pgfqpoint{0.784656in}{4.248997in}}%
\pgfpathlineto{\pgfqpoint{0.785730in}{4.247792in}}%
\pgfpathlineto{\pgfqpoint{0.786804in}{4.249242in}}%
\pgfpathlineto{\pgfqpoint{0.792173in}{4.248568in}}%
\pgfpathlineto{\pgfqpoint{0.793247in}{4.247425in}}%
\pgfpathlineto{\pgfqpoint{0.794321in}{4.248854in}}%
\pgfpathlineto{\pgfqpoint{0.799690in}{4.247404in}}%
\pgfpathlineto{\pgfqpoint{0.800764in}{4.250019in}}%
\pgfpathlineto{\pgfqpoint{0.801838in}{4.249222in}}%
\pgfpathlineto{\pgfqpoint{0.805059in}{4.245566in}}%
\pgfpathlineto{\pgfqpoint{0.806133in}{4.246424in}}%
\pgfpathlineto{\pgfqpoint{0.807207in}{4.245791in}}%
\pgfpathlineto{\pgfqpoint{0.809355in}{4.251203in}}%
\pgfpathlineto{\pgfqpoint{0.816872in}{4.249896in}}%
\pgfpathlineto{\pgfqpoint{0.820093in}{4.250795in}}%
\pgfpathlineto{\pgfqpoint{0.821167in}{4.249242in}}%
\pgfpathlineto{\pgfqpoint{0.822241in}{4.248834in}}%
\pgfpathlineto{\pgfqpoint{0.823315in}{4.249222in}}%
\pgfpathlineto{\pgfqpoint{0.824388in}{4.248548in}}%
\pgfpathlineto{\pgfqpoint{0.830832in}{4.251979in}}%
\pgfpathlineto{\pgfqpoint{0.831905in}{4.252878in}}%
\pgfpathlineto{\pgfqpoint{0.835127in}{4.253593in}}%
\pgfpathlineto{\pgfqpoint{0.837275in}{4.250019in}}%
\pgfpathlineto{\pgfqpoint{0.839422in}{4.249120in}}%
\pgfpathlineto{\pgfqpoint{0.842644in}{4.249181in}}%
\pgfpathlineto{\pgfqpoint{0.843718in}{4.251632in}}%
\pgfpathlineto{\pgfqpoint{0.844791in}{4.252633in}}%
\pgfpathlineto{\pgfqpoint{0.845865in}{4.252510in}}%
\pgfpathlineto{\pgfqpoint{0.846939in}{4.249426in}}%
\pgfpathlineto{\pgfqpoint{0.850161in}{4.248732in}}%
\pgfpathlineto{\pgfqpoint{0.851234in}{4.242257in}}%
\pgfpathlineto{\pgfqpoint{0.854456in}{4.241052in}}%
\pgfpathlineto{\pgfqpoint{0.859825in}{4.240317in}}%
\pgfpathlineto{\pgfqpoint{0.860899in}{4.243136in}}%
\pgfpathlineto{\pgfqpoint{0.861973in}{4.242666in}}%
\pgfpathlineto{\pgfqpoint{0.865194in}{4.243728in}}%
\pgfpathlineto{\pgfqpoint{0.866268in}{4.245893in}}%
\pgfpathlineto{\pgfqpoint{0.868416in}{4.241931in}}%
\pgfpathlineto{\pgfqpoint{0.869490in}{4.242380in}}%
\pgfpathlineto{\pgfqpoint{0.873785in}{4.242850in}}%
\pgfpathlineto{\pgfqpoint{0.874859in}{4.239827in}}%
\pgfpathlineto{\pgfqpoint{0.877007in}{4.241849in}}%
\pgfpathlineto{\pgfqpoint{0.881302in}{4.243748in}}%
\pgfpathlineto{\pgfqpoint{0.882376in}{4.243605in}}%
\pgfpathlineto{\pgfqpoint{0.884523in}{4.245913in}}%
\pgfpathlineto{\pgfqpoint{0.888819in}{4.245975in}}%
\pgfpathlineto{\pgfqpoint{0.889893in}{4.247037in}}%
\pgfpathlineto{\pgfqpoint{0.890966in}{4.246546in}}%
\pgfpathlineto{\pgfqpoint{0.892040in}{4.247078in}}%
\pgfpathlineto{\pgfqpoint{0.896336in}{4.245648in}}%
\pgfpathlineto{\pgfqpoint{0.898483in}{4.247078in}}%
\pgfpathlineto{\pgfqpoint{0.899557in}{4.248037in}}%
\pgfpathlineto{\pgfqpoint{0.902779in}{4.248650in}}%
\pgfpathlineto{\pgfqpoint{0.903853in}{4.251755in}}%
\pgfpathlineto{\pgfqpoint{0.907074in}{4.249345in}}%
\pgfpathlineto{\pgfqpoint{0.910296in}{4.250652in}}%
\pgfpathlineto{\pgfqpoint{0.911369in}{4.252061in}}%
\pgfpathlineto{\pgfqpoint{0.912443in}{4.250631in}}%
\pgfpathlineto{\pgfqpoint{0.913517in}{4.252531in}}%
\pgfpathlineto{\pgfqpoint{0.914591in}{4.250754in}}%
\pgfpathlineto{\pgfqpoint{0.919960in}{4.250713in}}%
\pgfpathlineto{\pgfqpoint{0.921034in}{4.249978in}}%
\pgfpathlineto{\pgfqpoint{0.922108in}{4.248487in}}%
\pgfpathlineto{\pgfqpoint{0.925329in}{4.250325in}}%
\pgfpathlineto{\pgfqpoint{0.927477in}{4.253654in}}%
\pgfpathlineto{\pgfqpoint{0.928551in}{4.253470in}}%
\pgfpathlineto{\pgfqpoint{0.929625in}{4.254655in}}%
\pgfpathlineto{\pgfqpoint{0.933920in}{4.254880in}}%
\pgfpathlineto{\pgfqpoint{0.936068in}{4.257269in}}%
\pgfpathlineto{\pgfqpoint{0.937142in}{4.256228in}}%
\pgfpathlineto{\pgfqpoint{0.943585in}{4.259332in}}%
\pgfpathlineto{\pgfqpoint{0.944658in}{4.260455in}}%
\pgfpathlineto{\pgfqpoint{0.951101in}{4.262048in}}%
\pgfpathlineto{\pgfqpoint{0.952175in}{4.263641in}}%
\pgfpathlineto{\pgfqpoint{0.955397in}{4.263744in}}%
\pgfpathlineto{\pgfqpoint{0.956471in}{4.265725in}}%
\pgfpathlineto{\pgfqpoint{0.957544in}{4.263989in}}%
\pgfpathlineto{\pgfqpoint{0.958618in}{4.263560in}}%
\pgfpathlineto{\pgfqpoint{0.959692in}{4.265296in}}%
\pgfpathlineto{\pgfqpoint{0.962914in}{4.263948in}}%
\pgfpathlineto{\pgfqpoint{0.965061in}{4.267236in}}%
\pgfpathlineto{\pgfqpoint{0.966135in}{4.266440in}}%
\pgfpathlineto{\pgfqpoint{0.967209in}{4.267195in}}%
\pgfpathlineto{\pgfqpoint{0.970431in}{4.267134in}}%
\pgfpathlineto{\pgfqpoint{0.971504in}{4.268564in}}%
\pgfpathlineto{\pgfqpoint{0.974726in}{4.269279in}}%
\pgfpathlineto{\pgfqpoint{0.979021in}{4.270912in}}%
\pgfpathlineto{\pgfqpoint{0.981169in}{4.268380in}}%
\pgfpathlineto{\pgfqpoint{0.982243in}{4.269810in}}%
\pgfpathlineto{\pgfqpoint{0.985464in}{4.266705in}}%
\pgfpathlineto{\pgfqpoint{0.986538in}{4.267685in}}%
\pgfpathlineto{\pgfqpoint{0.987612in}{4.269850in}}%
\pgfpathlineto{\pgfqpoint{0.988686in}{4.270606in}}%
\pgfpathlineto{\pgfqpoint{0.992981in}{4.269360in}}%
\pgfpathlineto{\pgfqpoint{0.994055in}{4.271382in}}%
\pgfpathlineto{\pgfqpoint{0.995129in}{4.271750in}}%
\pgfpathlineto{\pgfqpoint{0.996203in}{4.271525in}}%
\pgfpathlineto{\pgfqpoint{0.997276in}{4.273568in}}%
\pgfpathlineto{\pgfqpoint{1.000498in}{4.273731in}}%
\pgfpathlineto{\pgfqpoint{1.001572in}{4.272546in}}%
\pgfpathlineto{\pgfqpoint{1.002646in}{4.272485in}}%
\pgfpathlineto{\pgfqpoint{1.004793in}{4.274752in}}%
\pgfpathlineto{\pgfqpoint{1.009089in}{4.272648in}}%
\pgfpathlineto{\pgfqpoint{1.010163in}{4.273465in}}%
\pgfpathlineto{\pgfqpoint{1.011236in}{4.272220in}}%
\pgfpathlineto{\pgfqpoint{1.012310in}{4.274793in}}%
\pgfpathlineto{\pgfqpoint{1.015532in}{4.272628in}}%
\pgfpathlineto{\pgfqpoint{1.016606in}{4.274180in}}%
\pgfpathlineto{\pgfqpoint{1.017679in}{4.272832in}}%
\pgfpathlineto{\pgfqpoint{1.018753in}{4.274589in}}%
\pgfpathlineto{\pgfqpoint{1.023049in}{4.273445in}}%
\pgfpathlineto{\pgfqpoint{1.024122in}{4.274956in}}%
\pgfpathlineto{\pgfqpoint{1.025196in}{4.273506in}}%
\pgfpathlineto{\pgfqpoint{1.027344in}{4.273670in}}%
\pgfpathlineto{\pgfqpoint{1.031639in}{4.273976in}}%
\pgfpathlineto{\pgfqpoint{1.032713in}{4.276978in}}%
\pgfpathlineto{\pgfqpoint{1.033787in}{4.277959in}}%
\pgfpathlineto{\pgfqpoint{1.038082in}{4.273649in}}%
\pgfpathlineto{\pgfqpoint{1.039156in}{4.274323in}}%
\pgfpathlineto{\pgfqpoint{1.041304in}{4.272301in}}%
\pgfpathlineto{\pgfqpoint{1.042378in}{4.273568in}}%
\pgfpathlineto{\pgfqpoint{1.045599in}{4.273711in}}%
\pgfpathlineto{\pgfqpoint{1.046673in}{4.276468in}}%
\pgfpathlineto{\pgfqpoint{1.047747in}{4.277285in}}%
\pgfpathlineto{\pgfqpoint{1.048821in}{4.272117in}}%
\pgfpathlineto{\pgfqpoint{1.049895in}{4.270238in}}%
\pgfpathlineto{\pgfqpoint{1.053116in}{4.270300in}}%
\pgfpathlineto{\pgfqpoint{1.054190in}{4.271832in}}%
\pgfpathlineto{\pgfqpoint{1.055264in}{4.271546in}}%
\pgfpathlineto{\pgfqpoint{1.057411in}{4.277224in}}%
\pgfpathlineto{\pgfqpoint{1.061707in}{4.277530in}}%
\pgfpathlineto{\pgfqpoint{1.062781in}{4.277857in}}%
\pgfpathlineto{\pgfqpoint{1.063854in}{4.280839in}}%
\pgfpathlineto{\pgfqpoint{1.064928in}{4.281799in}}%
\pgfpathlineto{\pgfqpoint{1.069224in}{4.282003in}}%
\pgfpathlineto{\pgfqpoint{1.070297in}{4.283575in}}%
\pgfpathlineto{\pgfqpoint{1.071371in}{4.282820in}}%
\pgfpathlineto{\pgfqpoint{1.072445in}{4.283371in}}%
\pgfpathlineto{\pgfqpoint{1.076741in}{4.284454in}}%
\pgfpathlineto{\pgfqpoint{1.078888in}{4.282779in}}%
\pgfpathlineto{\pgfqpoint{1.079962in}{4.282534in}}%
\pgfpathlineto{\pgfqpoint{1.084257in}{4.284821in}}%
\pgfpathlineto{\pgfqpoint{1.085331in}{4.284025in}}%
\pgfpathlineto{\pgfqpoint{1.086405in}{4.284535in}}%
\pgfpathlineto{\pgfqpoint{1.087479in}{4.282534in}}%
\pgfpathlineto{\pgfqpoint{1.090700in}{4.283126in}}%
\pgfpathlineto{\pgfqpoint{1.091774in}{4.282105in}}%
\pgfpathlineto{\pgfqpoint{1.092848in}{4.279613in}}%
\pgfpathlineto{\pgfqpoint{1.093922in}{4.279756in}}%
\pgfpathlineto{\pgfqpoint{1.094996in}{4.283984in}}%
\pgfpathlineto{\pgfqpoint{1.098217in}{4.283473in}}%
\pgfpathlineto{\pgfqpoint{1.099291in}{4.282452in}}%
\pgfpathlineto{\pgfqpoint{1.100365in}{4.280369in}}%
\pgfpathlineto{\pgfqpoint{1.101439in}{4.284147in}}%
\pgfpathlineto{\pgfqpoint{1.102513in}{4.283861in}}%
\pgfpathlineto{\pgfqpoint{1.105734in}{4.285393in}}%
\pgfpathlineto{\pgfqpoint{1.106808in}{4.287231in}}%
\pgfpathlineto{\pgfqpoint{1.107882in}{4.284801in}}%
\pgfpathlineto{\pgfqpoint{1.108956in}{4.279940in}}%
\pgfpathlineto{\pgfqpoint{1.110030in}{4.281349in}}%
\pgfpathlineto{\pgfqpoint{1.113251in}{4.277714in}}%
\pgfpathlineto{\pgfqpoint{1.114325in}{4.279000in}}%
\pgfpathlineto{\pgfqpoint{1.115399in}{4.281513in}}%
\pgfpathlineto{\pgfqpoint{1.116473in}{4.282472in}}%
\pgfpathlineto{\pgfqpoint{1.117546in}{4.280941in}}%
\pgfpathlineto{\pgfqpoint{1.120768in}{4.280859in}}%
\pgfpathlineto{\pgfqpoint{1.121842in}{4.279858in}}%
\pgfpathlineto{\pgfqpoint{1.122916in}{4.281104in}}%
\pgfpathlineto{\pgfqpoint{1.125063in}{4.284740in}}%
\pgfpathlineto{\pgfqpoint{1.128285in}{4.285761in}}%
\pgfpathlineto{\pgfqpoint{1.129359in}{4.287885in}}%
\pgfpathlineto{\pgfqpoint{1.130432in}{4.288028in}}%
\pgfpathlineto{\pgfqpoint{1.132580in}{4.291010in}}%
\pgfpathlineto{\pgfqpoint{1.135802in}{4.290499in}}%
\pgfpathlineto{\pgfqpoint{1.136875in}{4.289601in}}%
\pgfpathlineto{\pgfqpoint{1.137949in}{4.290132in}}%
\pgfpathlineto{\pgfqpoint{1.140097in}{4.292828in}}%
\pgfpathlineto{\pgfqpoint{1.143319in}{4.292991in}}%
\pgfpathlineto{\pgfqpoint{1.144392in}{4.293808in}}%
\pgfpathlineto{\pgfqpoint{1.145466in}{4.293073in}}%
\pgfpathlineto{\pgfqpoint{1.147614in}{4.294073in}}%
\pgfpathlineto{\pgfqpoint{1.151909in}{4.293951in}}%
\pgfpathlineto{\pgfqpoint{1.155131in}{4.296422in}}%
\pgfpathlineto{\pgfqpoint{1.160500in}{4.295687in}}%
\pgfpathlineto{\pgfqpoint{1.161574in}{4.297157in}}%
\pgfpathlineto{\pgfqpoint{1.162648in}{4.296524in}}%
\pgfpathlineto{\pgfqpoint{1.166943in}{4.297137in}}%
\pgfpathlineto{\pgfqpoint{1.168017in}{4.295319in}}%
\pgfpathlineto{\pgfqpoint{1.169091in}{4.292235in}}%
\pgfpathlineto{\pgfqpoint{1.170164in}{4.292317in}}%
\pgfpathlineto{\pgfqpoint{1.174460in}{4.291480in}}%
\pgfpathlineto{\pgfqpoint{1.175534in}{4.289110in}}%
\pgfpathlineto{\pgfqpoint{1.176608in}{4.291296in}}%
\pgfpathlineto{\pgfqpoint{1.177681in}{4.290806in}}%
\pgfpathlineto{\pgfqpoint{1.180903in}{4.290683in}}%
\pgfpathlineto{\pgfqpoint{1.181977in}{4.287885in}}%
\pgfpathlineto{\pgfqpoint{1.185198in}{4.289376in}}%
\pgfpathlineto{\pgfqpoint{1.189494in}{4.288743in}}%
\pgfpathlineto{\pgfqpoint{1.190567in}{4.291071in}}%
\pgfpathlineto{\pgfqpoint{1.192715in}{4.291929in}}%
\pgfpathlineto{\pgfqpoint{1.197010in}{4.296402in}}%
\pgfpathlineto{\pgfqpoint{1.198084in}{4.298342in}}%
\pgfpathlineto{\pgfqpoint{1.199158in}{4.297464in}}%
\pgfpathlineto{\pgfqpoint{1.200232in}{4.298158in}}%
\pgfpathlineto{\pgfqpoint{1.203453in}{4.299261in}}%
\pgfpathlineto{\pgfqpoint{1.204527in}{4.300487in}}%
\pgfpathlineto{\pgfqpoint{1.205601in}{4.302856in}}%
\pgfpathlineto{\pgfqpoint{1.206675in}{4.303346in}}%
\pgfpathlineto{\pgfqpoint{1.207749in}{4.300630in}}%
\pgfpathlineto{\pgfqpoint{1.210970in}{4.302549in}}%
\pgfpathlineto{\pgfqpoint{1.213118in}{4.300956in}}%
\pgfpathlineto{\pgfqpoint{1.214192in}{4.301753in}}%
\pgfpathlineto{\pgfqpoint{1.215266in}{4.300997in}}%
\pgfpathlineto{\pgfqpoint{1.218487in}{4.299567in}}%
\pgfpathlineto{\pgfqpoint{1.219561in}{4.299935in}}%
\pgfpathlineto{\pgfqpoint{1.221709in}{4.298077in}}%
\pgfpathlineto{\pgfqpoint{1.222783in}{4.299567in}}%
\pgfpathlineto{\pgfqpoint{1.226004in}{4.298322in}}%
\pgfpathlineto{\pgfqpoint{1.227078in}{4.295646in}}%
\pgfpathlineto{\pgfqpoint{1.228152in}{4.296361in}}%
\pgfpathlineto{\pgfqpoint{1.230299in}{4.301875in}}%
\pgfpathlineto{\pgfqpoint{1.233521in}{4.303060in}}%
\pgfpathlineto{\pgfqpoint{1.234595in}{4.300303in}}%
\pgfpathlineto{\pgfqpoint{1.237816in}{4.305572in}}%
\pgfpathlineto{\pgfqpoint{1.241038in}{4.306287in}}%
\pgfpathlineto{\pgfqpoint{1.242112in}{4.307247in}}%
\pgfpathlineto{\pgfqpoint{1.243185in}{4.306205in}}%
\pgfpathlineto{\pgfqpoint{1.244259in}{4.306716in}}%
\pgfpathlineto{\pgfqpoint{1.245333in}{4.308329in}}%
\pgfpathlineto{\pgfqpoint{1.249629in}{4.309841in}}%
\pgfpathlineto{\pgfqpoint{1.250702in}{4.308983in}}%
\pgfpathlineto{\pgfqpoint{1.251776in}{4.310842in}}%
\pgfpathlineto{\pgfqpoint{1.257145in}{4.311311in}}%
\pgfpathlineto{\pgfqpoint{1.258219in}{4.313068in}}%
\pgfpathlineto{\pgfqpoint{1.259293in}{4.311822in}}%
\pgfpathlineto{\pgfqpoint{1.260367in}{4.314579in}}%
\pgfpathlineto{\pgfqpoint{1.263588in}{4.314518in}}%
\pgfpathlineto{\pgfqpoint{1.265736in}{4.315621in}}%
\pgfpathlineto{\pgfqpoint{1.266810in}{4.317725in}}%
\pgfpathlineto{\pgfqpoint{1.272179in}{4.318174in}}%
\pgfpathlineto{\pgfqpoint{1.273253in}{4.317765in}}%
\pgfpathlineto{\pgfqpoint{1.275401in}{4.321013in}}%
\pgfpathlineto{\pgfqpoint{1.278622in}{4.321380in}}%
\pgfpathlineto{\pgfqpoint{1.280770in}{4.325363in}}%
\pgfpathlineto{\pgfqpoint{1.282918in}{4.325384in}}%
\pgfpathlineto{\pgfqpoint{1.287213in}{4.313231in}}%
\pgfpathlineto{\pgfqpoint{1.288287in}{4.312986in}}%
\pgfpathlineto{\pgfqpoint{1.289361in}{4.313640in}}%
\pgfpathlineto{\pgfqpoint{1.290434in}{4.316765in}}%
\pgfpathlineto{\pgfqpoint{1.293656in}{4.316703in}}%
\pgfpathlineto{\pgfqpoint{1.295804in}{4.313538in}}%
\pgfpathlineto{\pgfqpoint{1.297951in}{4.312945in}}%
\pgfpathlineto{\pgfqpoint{1.301173in}{4.315090in}}%
\pgfpathlineto{\pgfqpoint{1.303320in}{4.329407in}}%
\pgfpathlineto{\pgfqpoint{1.305468in}{4.331021in}}%
\pgfpathlineto{\pgfqpoint{1.309763in}{4.331490in}}%
\pgfpathlineto{\pgfqpoint{1.312985in}{4.335636in}}%
\pgfpathlineto{\pgfqpoint{1.316207in}{4.335759in}}%
\pgfpathlineto{\pgfqpoint{1.317280in}{4.337230in}}%
\pgfpathlineto{\pgfqpoint{1.319428in}{4.333492in}}%
\pgfpathlineto{\pgfqpoint{1.320502in}{4.334064in}}%
\pgfpathlineto{\pgfqpoint{1.324797in}{4.332655in}}%
\pgfpathlineto{\pgfqpoint{1.325871in}{4.330857in}}%
\pgfpathlineto{\pgfqpoint{1.328019in}{4.330081in}}%
\pgfpathlineto{\pgfqpoint{1.331240in}{4.327446in}}%
\pgfpathlineto{\pgfqpoint{1.332314in}{4.332226in}}%
\pgfpathlineto{\pgfqpoint{1.333388in}{4.334043in}}%
\pgfpathlineto{\pgfqpoint{1.334462in}{4.333553in}}%
\pgfpathlineto{\pgfqpoint{1.335536in}{4.332062in}}%
\pgfpathlineto{\pgfqpoint{1.339831in}{4.331490in}}%
\pgfpathlineto{\pgfqpoint{1.340905in}{4.330592in}}%
\pgfpathlineto{\pgfqpoint{1.341979in}{4.327487in}}%
\pgfpathlineto{\pgfqpoint{1.343052in}{4.319604in}}%
\pgfpathlineto{\pgfqpoint{1.346274in}{4.317377in}}%
\pgfpathlineto{\pgfqpoint{1.348422in}{4.319644in}}%
\pgfpathlineto{\pgfqpoint{1.349496in}{4.315784in}}%
\pgfpathlineto{\pgfqpoint{1.350569in}{4.316029in}}%
\pgfpathlineto{\pgfqpoint{1.353791in}{4.308493in}}%
\pgfpathlineto{\pgfqpoint{1.354865in}{4.313456in}}%
\pgfpathlineto{\pgfqpoint{1.355939in}{4.314579in}}%
\pgfpathlineto{\pgfqpoint{1.358086in}{4.319787in}}%
\pgfpathlineto{\pgfqpoint{1.361308in}{4.318684in}}%
\pgfpathlineto{\pgfqpoint{1.362382in}{4.320931in}}%
\pgfpathlineto{\pgfqpoint{1.363455in}{4.321483in}}%
\pgfpathlineto{\pgfqpoint{1.364529in}{4.320952in}}%
\pgfpathlineto{\pgfqpoint{1.365603in}{4.324464in}}%
\pgfpathlineto{\pgfqpoint{1.369898in}{4.323893in}}%
\pgfpathlineto{\pgfqpoint{1.370972in}{4.321707in}}%
\pgfpathlineto{\pgfqpoint{1.372046in}{4.323464in}}%
\pgfpathlineto{\pgfqpoint{1.373120in}{4.323484in}}%
\pgfpathlineto{\pgfqpoint{1.376341in}{4.324607in}}%
\pgfpathlineto{\pgfqpoint{1.377415in}{4.325894in}}%
\pgfpathlineto{\pgfqpoint{1.378489in}{4.325772in}}%
\pgfpathlineto{\pgfqpoint{1.379563in}{4.328386in}}%
\pgfpathlineto{\pgfqpoint{1.380637in}{4.329080in}}%
\pgfpathlineto{\pgfqpoint{1.383858in}{4.324628in}}%
\pgfpathlineto{\pgfqpoint{1.384932in}{4.325445in}}%
\pgfpathlineto{\pgfqpoint{1.386006in}{4.327528in}}%
\pgfpathlineto{\pgfqpoint{1.388154in}{4.327977in}}%
\pgfpathlineto{\pgfqpoint{1.391375in}{4.327017in}}%
\pgfpathlineto{\pgfqpoint{1.392449in}{4.325159in}}%
\pgfpathlineto{\pgfqpoint{1.393523in}{4.325261in}}%
\pgfpathlineto{\pgfqpoint{1.395671in}{4.320421in}}%
\pgfpathlineto{\pgfqpoint{1.399966in}{4.325567in}}%
\pgfpathlineto{\pgfqpoint{1.401040in}{4.322892in}}%
\pgfpathlineto{\pgfqpoint{1.403187in}{4.326241in}}%
\pgfpathlineto{\pgfqpoint{1.406409in}{4.324996in}}%
\pgfpathlineto{\pgfqpoint{1.407483in}{4.327896in}}%
\pgfpathlineto{\pgfqpoint{1.408557in}{4.326180in}}%
\pgfpathlineto{\pgfqpoint{1.409630in}{4.325731in}}%
\pgfpathlineto{\pgfqpoint{1.410704in}{4.328141in}}%
\pgfpathlineto{\pgfqpoint{1.415000in}{4.332266in}}%
\pgfpathlineto{\pgfqpoint{1.416073in}{4.331327in}}%
\pgfpathlineto{\pgfqpoint{1.417147in}{4.331572in}}%
\pgfpathlineto{\pgfqpoint{1.418221in}{4.331082in}}%
\pgfpathlineto{\pgfqpoint{1.421443in}{4.328488in}}%
\pgfpathlineto{\pgfqpoint{1.422517in}{4.329285in}}%
\pgfpathlineto{\pgfqpoint{1.423590in}{4.331041in}}%
\pgfpathlineto{\pgfqpoint{1.425738in}{4.324934in}}%
\pgfpathlineto{\pgfqpoint{1.428960in}{4.326282in}}%
\pgfpathlineto{\pgfqpoint{1.430033in}{4.327957in}}%
\pgfpathlineto{\pgfqpoint{1.431107in}{4.332695in}}%
\pgfpathlineto{\pgfqpoint{1.432181in}{4.334391in}}%
\pgfpathlineto{\pgfqpoint{1.437550in}{4.336433in}}%
\pgfpathlineto{\pgfqpoint{1.440772in}{4.332307in}}%
\pgfpathlineto{\pgfqpoint{1.445067in}{4.334064in}}%
\pgfpathlineto{\pgfqpoint{1.447215in}{4.339844in}}%
\pgfpathlineto{\pgfqpoint{1.448289in}{4.338618in}}%
\pgfpathlineto{\pgfqpoint{1.451510in}{4.339476in}}%
\pgfpathlineto{\pgfqpoint{1.452584in}{4.337148in}}%
\pgfpathlineto{\pgfqpoint{1.453658in}{4.340416in}}%
\pgfpathlineto{\pgfqpoint{1.454732in}{4.339864in}}%
\pgfpathlineto{\pgfqpoint{1.459027in}{4.343398in}}%
\pgfpathlineto{\pgfqpoint{1.460101in}{4.342703in}}%
\pgfpathlineto{\pgfqpoint{1.462249in}{4.340130in}}%
\pgfpathlineto{\pgfqpoint{1.466544in}{4.341355in}}%
\pgfpathlineto{\pgfqpoint{1.467618in}{4.338802in}}%
\pgfpathlineto{\pgfqpoint{1.468692in}{4.341090in}}%
\pgfpathlineto{\pgfqpoint{1.469765in}{4.340477in}}%
\pgfpathlineto{\pgfqpoint{1.470839in}{4.341948in}}%
\pgfpathlineto{\pgfqpoint{1.476208in}{4.342458in}}%
\pgfpathlineto{\pgfqpoint{1.477282in}{4.344174in}}%
\pgfpathlineto{\pgfqpoint{1.478356in}{4.344460in}}%
\pgfpathlineto{\pgfqpoint{1.481578in}{4.344051in}}%
\pgfpathlineto{\pgfqpoint{1.482651in}{4.345072in}}%
\pgfpathlineto{\pgfqpoint{1.483725in}{4.343949in}}%
\pgfpathlineto{\pgfqpoint{1.485873in}{4.348177in}}%
\pgfpathlineto{\pgfqpoint{1.489095in}{4.349382in}}%
\pgfpathlineto{\pgfqpoint{1.491242in}{4.347768in}}%
\pgfpathlineto{\pgfqpoint{1.492316in}{4.345379in}}%
\pgfpathlineto{\pgfqpoint{1.493390in}{4.345910in}}%
\pgfpathlineto{\pgfqpoint{1.496611in}{4.345828in}}%
\pgfpathlineto{\pgfqpoint{1.499833in}{4.347911in}}%
\pgfpathlineto{\pgfqpoint{1.500907in}{4.349096in}}%
\pgfpathlineto{\pgfqpoint{1.504128in}{4.347197in}}%
\pgfpathlineto{\pgfqpoint{1.505202in}{4.345420in}}%
\pgfpathlineto{\pgfqpoint{1.506276in}{4.346420in}}%
\pgfpathlineto{\pgfqpoint{1.508424in}{4.346482in}}%
\pgfpathlineto{\pgfqpoint{1.511645in}{4.345685in}}%
\pgfpathlineto{\pgfqpoint{1.513793in}{4.349239in}}%
\pgfpathlineto{\pgfqpoint{1.514867in}{4.349545in}}%
\pgfpathlineto{\pgfqpoint{1.521310in}{4.348259in}}%
\pgfpathlineto{\pgfqpoint{1.522384in}{4.346849in}}%
\pgfpathlineto{\pgfqpoint{1.523457in}{4.347585in}}%
\pgfpathlineto{\pgfqpoint{1.527753in}{4.348933in}}%
\pgfpathlineto{\pgfqpoint{1.528827in}{4.350893in}}%
\pgfpathlineto{\pgfqpoint{1.529900in}{4.346216in}}%
\pgfpathlineto{\pgfqpoint{1.530974in}{4.348524in}}%
\pgfpathlineto{\pgfqpoint{1.534196in}{4.347585in}}%
\pgfpathlineto{\pgfqpoint{1.535270in}{4.349035in}}%
\pgfpathlineto{\pgfqpoint{1.536343in}{4.348259in}}%
\pgfpathlineto{\pgfqpoint{1.537417in}{4.349055in}}%
\pgfpathlineto{\pgfqpoint{1.538491in}{4.349035in}}%
\pgfpathlineto{\pgfqpoint{1.541713in}{4.349770in}}%
\pgfpathlineto{\pgfqpoint{1.542786in}{4.347074in}}%
\pgfpathlineto{\pgfqpoint{1.543860in}{4.346584in}}%
\pgfpathlineto{\pgfqpoint{1.544934in}{4.341498in}}%
\pgfpathlineto{\pgfqpoint{1.546008in}{4.340109in}}%
\pgfpathlineto{\pgfqpoint{1.549229in}{4.341253in}}%
\pgfpathlineto{\pgfqpoint{1.550303in}{4.339537in}}%
\pgfpathlineto{\pgfqpoint{1.552451in}{4.338373in}}%
\pgfpathlineto{\pgfqpoint{1.553525in}{4.341437in}}%
\pgfpathlineto{\pgfqpoint{1.556746in}{4.340947in}}%
\pgfpathlineto{\pgfqpoint{1.557820in}{4.341478in}}%
\pgfpathlineto{\pgfqpoint{1.559968in}{4.344092in}}%
\pgfpathlineto{\pgfqpoint{1.561042in}{4.343336in}}%
\pgfpathlineto{\pgfqpoint{1.564263in}{4.347258in}}%
\pgfpathlineto{\pgfqpoint{1.565337in}{4.347482in}}%
\pgfpathlineto{\pgfqpoint{1.566411in}{4.349647in}}%
\pgfpathlineto{\pgfqpoint{1.568559in}{4.348810in}}%
\pgfpathlineto{\pgfqpoint{1.572854in}{4.349647in}}%
\pgfpathlineto{\pgfqpoint{1.573928in}{4.348463in}}%
\pgfpathlineto{\pgfqpoint{1.580371in}{4.348953in}}%
\pgfpathlineto{\pgfqpoint{1.582518in}{4.347871in}}%
\pgfpathlineto{\pgfqpoint{1.583592in}{4.348933in}}%
\pgfpathlineto{\pgfqpoint{1.586814in}{4.350158in}}%
\pgfpathlineto{\pgfqpoint{1.587888in}{4.349402in}}%
\pgfpathlineto{\pgfqpoint{1.588961in}{4.349709in}}%
\pgfpathlineto{\pgfqpoint{1.591109in}{4.348463in}}%
\pgfpathlineto{\pgfqpoint{1.595405in}{4.350097in}}%
\pgfpathlineto{\pgfqpoint{1.596478in}{4.351016in}}%
\pgfpathlineto{\pgfqpoint{1.597552in}{4.353651in}}%
\pgfpathlineto{\pgfqpoint{1.598626in}{4.353385in}}%
\pgfpathlineto{\pgfqpoint{1.601848in}{4.351629in}}%
\pgfpathlineto{\pgfqpoint{1.602921in}{4.349239in}}%
\pgfpathlineto{\pgfqpoint{1.603995in}{4.350138in}}%
\pgfpathlineto{\pgfqpoint{1.605069in}{4.345849in}}%
\pgfpathlineto{\pgfqpoint{1.609364in}{4.345338in}}%
\pgfpathlineto{\pgfqpoint{1.610438in}{4.344419in}}%
\pgfpathlineto{\pgfqpoint{1.611512in}{4.339946in}}%
\pgfpathlineto{\pgfqpoint{1.612586in}{4.339027in}}%
\pgfpathlineto{\pgfqpoint{1.613660in}{4.341702in}}%
\pgfpathlineto{\pgfqpoint{1.616881in}{4.342009in}}%
\pgfpathlineto{\pgfqpoint{1.617955in}{4.337250in}}%
\pgfpathlineto{\pgfqpoint{1.619029in}{4.343929in}}%
\pgfpathlineto{\pgfqpoint{1.620103in}{4.338966in}}%
\pgfpathlineto{\pgfqpoint{1.621177in}{4.330367in}}%
\pgfpathlineto{\pgfqpoint{1.624398in}{4.328692in}}%
\pgfpathlineto{\pgfqpoint{1.625472in}{4.330980in}}%
\pgfpathlineto{\pgfqpoint{1.626546in}{4.331061in}}%
\pgfpathlineto{\pgfqpoint{1.627620in}{4.332552in}}%
\pgfpathlineto{\pgfqpoint{1.628694in}{4.336760in}}%
\pgfpathlineto{\pgfqpoint{1.631915in}{4.337107in}}%
\pgfpathlineto{\pgfqpoint{1.632989in}{4.343071in}}%
\pgfpathlineto{\pgfqpoint{1.634063in}{4.339517in}}%
\pgfpathlineto{\pgfqpoint{1.636210in}{4.356796in}}%
\pgfpathlineto{\pgfqpoint{1.639432in}{4.358532in}}%
\pgfpathlineto{\pgfqpoint{1.640506in}{4.361207in}}%
\pgfpathlineto{\pgfqpoint{1.641580in}{4.361126in}}%
\pgfpathlineto{\pgfqpoint{1.643727in}{4.366068in}}%
\pgfpathlineto{\pgfqpoint{1.646949in}{4.365211in}}%
\pgfpathlineto{\pgfqpoint{1.648023in}{4.368356in}}%
\pgfpathlineto{\pgfqpoint{1.651244in}{4.371031in}}%
\pgfpathlineto{\pgfqpoint{1.654466in}{4.373033in}}%
\pgfpathlineto{\pgfqpoint{1.655539in}{4.372032in}}%
\pgfpathlineto{\pgfqpoint{1.658761in}{4.375157in}}%
\pgfpathlineto{\pgfqpoint{1.661983in}{4.374871in}}%
\pgfpathlineto{\pgfqpoint{1.663056in}{4.377485in}}%
\pgfpathlineto{\pgfqpoint{1.664130in}{4.376648in}}%
\pgfpathlineto{\pgfqpoint{1.666278in}{4.379038in}}%
\pgfpathlineto{\pgfqpoint{1.669499in}{4.378527in}}%
\pgfpathlineto{\pgfqpoint{1.670573in}{4.375259in}}%
\pgfpathlineto{\pgfqpoint{1.671647in}{4.375709in}}%
\pgfpathlineto{\pgfqpoint{1.673795in}{4.378915in}}%
\pgfpathlineto{\pgfqpoint{1.677016in}{4.375443in}}%
\pgfpathlineto{\pgfqpoint{1.679164in}{4.382816in}}%
\pgfpathlineto{\pgfqpoint{1.681312in}{4.382837in}}%
\pgfpathlineto{\pgfqpoint{1.685607in}{4.380243in}}%
\pgfpathlineto{\pgfqpoint{1.686681in}{4.375586in}}%
\pgfpathlineto{\pgfqpoint{1.687755in}{4.377220in}}%
\pgfpathlineto{\pgfqpoint{1.688828in}{4.373564in}}%
\pgfpathlineto{\pgfqpoint{1.692050in}{4.373074in}}%
\pgfpathlineto{\pgfqpoint{1.694198in}{4.379834in}}%
\pgfpathlineto{\pgfqpoint{1.695272in}{4.388290in}}%
\pgfpathlineto{\pgfqpoint{1.696345in}{4.388617in}}%
\pgfpathlineto{\pgfqpoint{1.699567in}{4.391844in}}%
\pgfpathlineto{\pgfqpoint{1.700641in}{4.391129in}}%
\pgfpathlineto{\pgfqpoint{1.701715in}{4.391272in}}%
\pgfpathlineto{\pgfqpoint{1.703862in}{4.390026in}}%
\pgfpathlineto{\pgfqpoint{1.707084in}{4.390822in}}%
\pgfpathlineto{\pgfqpoint{1.708158in}{4.389270in}}%
\pgfpathlineto{\pgfqpoint{1.709231in}{4.386533in}}%
\pgfpathlineto{\pgfqpoint{1.711379in}{4.386064in}}%
\pgfpathlineto{\pgfqpoint{1.715674in}{4.376321in}}%
\pgfpathlineto{\pgfqpoint{1.716748in}{4.378384in}}%
\pgfpathlineto{\pgfqpoint{1.717822in}{4.385287in}}%
\pgfpathlineto{\pgfqpoint{1.718896in}{4.381672in}}%
\pgfpathlineto{\pgfqpoint{1.725339in}{4.378139in}}%
\pgfpathlineto{\pgfqpoint{1.726413in}{4.382346in}}%
\pgfpathlineto{\pgfqpoint{1.730708in}{4.382244in}}%
\pgfpathlineto{\pgfqpoint{1.731782in}{4.383327in}}%
\pgfpathlineto{\pgfqpoint{1.732856in}{4.389352in}}%
\pgfpathlineto{\pgfqpoint{1.733930in}{4.385982in}}%
\pgfpathlineto{\pgfqpoint{1.737151in}{4.386390in}}%
\pgfpathlineto{\pgfqpoint{1.738225in}{4.385287in}}%
\pgfpathlineto{\pgfqpoint{1.739299in}{4.385839in}}%
\pgfpathlineto{\pgfqpoint{1.740373in}{4.389781in}}%
\pgfpathlineto{\pgfqpoint{1.741447in}{4.382898in}}%
\pgfpathlineto{\pgfqpoint{1.744668in}{4.386717in}}%
\pgfpathlineto{\pgfqpoint{1.745742in}{4.389454in}}%
\pgfpathlineto{\pgfqpoint{1.746816in}{4.387412in}}%
\pgfpathlineto{\pgfqpoint{1.747890in}{4.390536in}}%
\pgfpathlineto{\pgfqpoint{1.752185in}{4.387473in}}%
\pgfpathlineto{\pgfqpoint{1.753259in}{4.389229in}}%
\pgfpathlineto{\pgfqpoint{1.754333in}{4.388657in}}%
\pgfpathlineto{\pgfqpoint{1.755406in}{4.391251in}}%
\pgfpathlineto{\pgfqpoint{1.756480in}{4.391313in}}%
\pgfpathlineto{\pgfqpoint{1.762923in}{4.393375in}}%
\pgfpathlineto{\pgfqpoint{1.763997in}{4.395254in}}%
\pgfpathlineto{\pgfqpoint{1.768293in}{4.397072in}}%
\pgfpathlineto{\pgfqpoint{1.769366in}{4.396643in}}%
\pgfpathlineto{\pgfqpoint{1.770440in}{4.397950in}}%
\pgfpathlineto{\pgfqpoint{1.771514in}{4.396214in}}%
\pgfpathlineto{\pgfqpoint{1.774736in}{4.399564in}}%
\pgfpathlineto{\pgfqpoint{1.776883in}{4.393518in}}%
\pgfpathlineto{\pgfqpoint{1.777957in}{4.394254in}}%
\pgfpathlineto{\pgfqpoint{1.779031in}{4.388433in}}%
\pgfpathlineto{\pgfqpoint{1.782252in}{4.392089in}}%
\pgfpathlineto{\pgfqpoint{1.783326in}{4.384634in}}%
\pgfpathlineto{\pgfqpoint{1.784400in}{4.383674in}}%
\pgfpathlineto{\pgfqpoint{1.785474in}{4.388637in}}%
\pgfpathlineto{\pgfqpoint{1.786548in}{4.385512in}}%
\pgfpathlineto{\pgfqpoint{1.789769in}{4.391782in}}%
\pgfpathlineto{\pgfqpoint{1.790843in}{4.388208in}}%
\pgfpathlineto{\pgfqpoint{1.791917in}{4.392273in}}%
\pgfpathlineto{\pgfqpoint{1.792991in}{4.390802in}}%
\pgfpathlineto{\pgfqpoint{1.794065in}{4.392293in}}%
\pgfpathlineto{\pgfqpoint{1.798360in}{4.391987in}}%
\pgfpathlineto{\pgfqpoint{1.799434in}{4.385430in}}%
\pgfpathlineto{\pgfqpoint{1.800508in}{4.385226in}}%
\pgfpathlineto{\pgfqpoint{1.804803in}{4.391496in}}%
\pgfpathlineto{\pgfqpoint{1.805877in}{4.389515in}}%
\pgfpathlineto{\pgfqpoint{1.806951in}{4.385104in}}%
\pgfpathlineto{\pgfqpoint{1.808025in}{4.385614in}}%
\pgfpathlineto{\pgfqpoint{1.813394in}{4.391782in}}%
\pgfpathlineto{\pgfqpoint{1.814468in}{4.391905in}}%
\pgfpathlineto{\pgfqpoint{1.816615in}{4.393355in}}%
\pgfpathlineto{\pgfqpoint{1.819837in}{4.391129in}}%
\pgfpathlineto{\pgfqpoint{1.821984in}{4.392211in}}%
\pgfpathlineto{\pgfqpoint{1.823058in}{4.391210in}}%
\pgfpathlineto{\pgfqpoint{1.824132in}{4.383633in}}%
\pgfpathlineto{\pgfqpoint{1.827354in}{4.388719in}}%
\pgfpathlineto{\pgfqpoint{1.828427in}{4.387779in}}%
\pgfpathlineto{\pgfqpoint{1.829501in}{4.389005in}}%
\pgfpathlineto{\pgfqpoint{1.830575in}{4.379936in}}%
\pgfpathlineto{\pgfqpoint{1.831649in}{4.378711in}}%
\pgfpathlineto{\pgfqpoint{1.834871in}{4.376893in}}%
\pgfpathlineto{\pgfqpoint{1.835944in}{4.377526in}}%
\pgfpathlineto{\pgfqpoint{1.837018in}{4.375055in}}%
\pgfpathlineto{\pgfqpoint{1.838092in}{4.374013in}}%
\pgfpathlineto{\pgfqpoint{1.839166in}{4.376342in}}%
\pgfpathlineto{\pgfqpoint{1.842387in}{4.378772in}}%
\pgfpathlineto{\pgfqpoint{1.843461in}{4.376750in}}%
\pgfpathlineto{\pgfqpoint{1.844535in}{4.376260in}}%
\pgfpathlineto{\pgfqpoint{1.845609in}{4.378017in}}%
\pgfpathlineto{\pgfqpoint{1.846683in}{4.381632in}}%
\pgfpathlineto{\pgfqpoint{1.849904in}{4.380365in}}%
\pgfpathlineto{\pgfqpoint{1.850978in}{4.380672in}}%
\pgfpathlineto{\pgfqpoint{1.854200in}{4.386513in}}%
\pgfpathlineto{\pgfqpoint{1.857421in}{4.385737in}}%
\pgfpathlineto{\pgfqpoint{1.858495in}{4.386186in}}%
\pgfpathlineto{\pgfqpoint{1.859569in}{4.385839in}}%
\pgfpathlineto{\pgfqpoint{1.860643in}{4.386125in}}%
\pgfpathlineto{\pgfqpoint{1.861716in}{4.384164in}}%
\pgfpathlineto{\pgfqpoint{1.866012in}{4.381632in}}%
\pgfpathlineto{\pgfqpoint{1.867086in}{4.384225in}}%
\pgfpathlineto{\pgfqpoint{1.868160in}{4.383960in}}%
\pgfpathlineto{\pgfqpoint{1.869233in}{4.380692in}}%
\pgfpathlineto{\pgfqpoint{1.873529in}{4.380692in}}%
\pgfpathlineto{\pgfqpoint{1.874603in}{4.382673in}}%
\pgfpathlineto{\pgfqpoint{1.876750in}{4.377057in}}%
\pgfpathlineto{\pgfqpoint{1.879972in}{4.376158in}}%
\pgfpathlineto{\pgfqpoint{1.881046in}{4.376873in}}%
\pgfpathlineto{\pgfqpoint{1.882119in}{4.380631in}}%
\pgfpathlineto{\pgfqpoint{1.883193in}{4.382122in}}%
\pgfpathlineto{\pgfqpoint{1.884267in}{4.378813in}}%
\pgfpathlineto{\pgfqpoint{1.887489in}{4.374851in}}%
\pgfpathlineto{\pgfqpoint{1.889636in}{4.376811in}}%
\pgfpathlineto{\pgfqpoint{1.890710in}{4.381693in}}%
\pgfpathlineto{\pgfqpoint{1.891784in}{4.380467in}}%
\pgfpathlineto{\pgfqpoint{1.896079in}{4.382081in}}%
\pgfpathlineto{\pgfqpoint{1.898227in}{4.374932in}}%
\pgfpathlineto{\pgfqpoint{1.899301in}{4.377077in}}%
\pgfpathlineto{\pgfqpoint{1.902522in}{4.371358in}}%
\pgfpathlineto{\pgfqpoint{1.903596in}{4.371991in}}%
\pgfpathlineto{\pgfqpoint{1.904670in}{4.374442in}}%
\pgfpathlineto{\pgfqpoint{1.905744in}{4.373952in}}%
\pgfpathlineto{\pgfqpoint{1.910039in}{4.373401in}}%
\pgfpathlineto{\pgfqpoint{1.911113in}{4.374054in}}%
\pgfpathlineto{\pgfqpoint{1.912187in}{4.369091in}}%
\pgfpathlineto{\pgfqpoint{1.914335in}{4.373176in}}%
\pgfpathlineto{\pgfqpoint{1.918630in}{4.376505in}}%
\pgfpathlineto{\pgfqpoint{1.919704in}{4.375096in}}%
\pgfpathlineto{\pgfqpoint{1.920778in}{4.377220in}}%
\pgfpathlineto{\pgfqpoint{1.921851in}{4.376342in}}%
\pgfpathlineto{\pgfqpoint{1.925073in}{4.377118in}}%
\pgfpathlineto{\pgfqpoint{1.926147in}{4.374647in}}%
\pgfpathlineto{\pgfqpoint{1.927221in}{4.374013in}}%
\pgfpathlineto{\pgfqpoint{1.928294in}{4.363229in}}%
\pgfpathlineto{\pgfqpoint{1.932590in}{4.361800in}}%
\pgfpathlineto{\pgfqpoint{1.933664in}{4.366171in}}%
\pgfpathlineto{\pgfqpoint{1.935811in}{4.367008in}}%
\pgfpathlineto{\pgfqpoint{1.936885in}{4.366599in}}%
\pgfpathlineto{\pgfqpoint{1.940107in}{4.364373in}}%
\pgfpathlineto{\pgfqpoint{1.942254in}{4.365987in}}%
\pgfpathlineto{\pgfqpoint{1.943328in}{4.362801in}}%
\pgfpathlineto{\pgfqpoint{1.944402in}{4.362127in}}%
\pgfpathlineto{\pgfqpoint{1.947624in}{4.366416in}}%
\pgfpathlineto{\pgfqpoint{1.948697in}{4.361371in}}%
\pgfpathlineto{\pgfqpoint{1.949771in}{4.361453in}}%
\pgfpathlineto{\pgfqpoint{1.950845in}{4.359410in}}%
\pgfpathlineto{\pgfqpoint{1.951919in}{4.361024in}}%
\pgfpathlineto{\pgfqpoint{1.955140in}{4.362760in}}%
\pgfpathlineto{\pgfqpoint{1.957288in}{4.358471in}}%
\pgfpathlineto{\pgfqpoint{1.959436in}{4.351526in}}%
\pgfpathlineto{\pgfqpoint{1.963731in}{4.343377in}}%
\pgfpathlineto{\pgfqpoint{1.964805in}{4.352691in}}%
\pgfpathlineto{\pgfqpoint{1.965879in}{4.354856in}}%
\pgfpathlineto{\pgfqpoint{1.966953in}{4.355427in}}%
\pgfpathlineto{\pgfqpoint{1.970174in}{4.351629in}}%
\pgfpathlineto{\pgfqpoint{1.971248in}{4.344950in}}%
\pgfpathlineto{\pgfqpoint{1.972322in}{4.349974in}}%
\pgfpathlineto{\pgfqpoint{1.973396in}{4.350893in}}%
\pgfpathlineto{\pgfqpoint{1.974470in}{4.347401in}}%
\pgfpathlineto{\pgfqpoint{1.978765in}{4.353998in}}%
\pgfpathlineto{\pgfqpoint{1.979839in}{4.349280in}}%
\pgfpathlineto{\pgfqpoint{1.980913in}{4.349137in}}%
\pgfpathlineto{\pgfqpoint{1.981986in}{4.350015in}}%
\pgfpathlineto{\pgfqpoint{1.985208in}{4.349178in}}%
\pgfpathlineto{\pgfqpoint{1.986282in}{4.354304in}}%
\pgfpathlineto{\pgfqpoint{1.987356in}{4.355407in}}%
\pgfpathlineto{\pgfqpoint{1.988429in}{4.353120in}}%
\pgfpathlineto{\pgfqpoint{1.989503in}{4.346992in}}%
\pgfpathlineto{\pgfqpoint{1.992725in}{4.347768in}}%
\pgfpathlineto{\pgfqpoint{1.993799in}{4.344072in}}%
\pgfpathlineto{\pgfqpoint{1.995946in}{4.343255in}}%
\pgfpathlineto{\pgfqpoint{1.997020in}{4.346890in}}%
\pgfpathlineto{\pgfqpoint{2.000242in}{4.344725in}}%
\pgfpathlineto{\pgfqpoint{2.001315in}{4.350546in}}%
\pgfpathlineto{\pgfqpoint{2.002389in}{4.350955in}}%
\pgfpathlineto{\pgfqpoint{2.003463in}{4.349178in}}%
\pgfpathlineto{\pgfqpoint{2.004537in}{4.353569in}}%
\pgfpathlineto{\pgfqpoint{2.007759in}{4.359308in}}%
\pgfpathlineto{\pgfqpoint{2.008832in}{4.358328in}}%
\pgfpathlineto{\pgfqpoint{2.010980in}{4.365088in}}%
\pgfpathlineto{\pgfqpoint{2.012054in}{4.365864in}}%
\pgfpathlineto{\pgfqpoint{2.015275in}{4.366171in}}%
\pgfpathlineto{\pgfqpoint{2.017423in}{4.362964in}}%
\pgfpathlineto{\pgfqpoint{2.018497in}{4.364557in}}%
\pgfpathlineto{\pgfqpoint{2.019571in}{4.363658in}}%
\pgfpathlineto{\pgfqpoint{2.022792in}{4.362392in}}%
\pgfpathlineto{\pgfqpoint{2.024940in}{4.365721in}}%
\pgfpathlineto{\pgfqpoint{2.026014in}{4.377057in}}%
\pgfpathlineto{\pgfqpoint{2.027088in}{4.376689in}}%
\pgfpathlineto{\pgfqpoint{2.031383in}{4.378405in}}%
\pgfpathlineto{\pgfqpoint{2.032457in}{4.380876in}}%
\pgfpathlineto{\pgfqpoint{2.034604in}{4.379283in}}%
\pgfpathlineto{\pgfqpoint{2.037826in}{4.384266in}}%
\pgfpathlineto{\pgfqpoint{2.038900in}{4.382122in}}%
\pgfpathlineto{\pgfqpoint{2.042121in}{4.383041in}}%
\pgfpathlineto{\pgfqpoint{2.045343in}{4.379753in}}%
\pgfpathlineto{\pgfqpoint{2.046417in}{4.379977in}}%
\pgfpathlineto{\pgfqpoint{2.047491in}{4.382653in}}%
\pgfpathlineto{\pgfqpoint{2.048564in}{4.377608in}}%
\pgfpathlineto{\pgfqpoint{2.049638in}{4.376423in}}%
\pgfpathlineto{\pgfqpoint{2.052860in}{4.381121in}}%
\pgfpathlineto{\pgfqpoint{2.053934in}{4.378956in}}%
\pgfpathlineto{\pgfqpoint{2.056081in}{4.383347in}}%
\pgfpathlineto{\pgfqpoint{2.057155in}{4.384389in}}%
\pgfpathlineto{\pgfqpoint{2.060377in}{4.383837in}}%
\pgfpathlineto{\pgfqpoint{2.061450in}{4.382183in}}%
\pgfpathlineto{\pgfqpoint{2.062524in}{4.381958in}}%
\pgfpathlineto{\pgfqpoint{2.064672in}{4.382612in}}%
\pgfpathlineto{\pgfqpoint{2.067893in}{4.380018in}}%
\pgfpathlineto{\pgfqpoint{2.068967in}{4.380610in}}%
\pgfpathlineto{\pgfqpoint{2.071115in}{4.375954in}}%
\pgfpathlineto{\pgfqpoint{2.072189in}{4.383041in}}%
\pgfpathlineto{\pgfqpoint{2.075410in}{4.382183in}}%
\pgfpathlineto{\pgfqpoint{2.077558in}{4.379038in}}%
\pgfpathlineto{\pgfqpoint{2.078632in}{4.381754in}}%
\pgfpathlineto{\pgfqpoint{2.079706in}{4.376791in}}%
\pgfpathlineto{\pgfqpoint{2.082927in}{4.381958in}}%
\pgfpathlineto{\pgfqpoint{2.084001in}{4.364414in}}%
\pgfpathlineto{\pgfqpoint{2.085075in}{4.367764in}}%
\pgfpathlineto{\pgfqpoint{2.086149in}{4.365742in}}%
\pgfpathlineto{\pgfqpoint{2.087223in}{4.362167in}}%
\pgfpathlineto{\pgfqpoint{2.090444in}{4.363209in}}%
\pgfpathlineto{\pgfqpoint{2.093666in}{4.370051in}}%
\pgfpathlineto{\pgfqpoint{2.097961in}{4.370112in}}%
\pgfpathlineto{\pgfqpoint{2.099035in}{4.373196in}}%
\pgfpathlineto{\pgfqpoint{2.105478in}{4.361984in}}%
\pgfpathlineto{\pgfqpoint{2.106552in}{4.363168in}}%
\pgfpathlineto{\pgfqpoint{2.108699in}{4.351179in}}%
\pgfpathlineto{\pgfqpoint{2.109773in}{4.350301in}}%
\pgfpathlineto{\pgfqpoint{2.112995in}{4.350240in}}%
\pgfpathlineto{\pgfqpoint{2.114069in}{4.350975in}}%
\pgfpathlineto{\pgfqpoint{2.115142in}{4.347013in}}%
\pgfpathlineto{\pgfqpoint{2.116216in}{4.351567in}}%
\pgfpathlineto{\pgfqpoint{2.117290in}{4.346972in}}%
\pgfpathlineto{\pgfqpoint{2.121585in}{4.346420in}}%
\pgfpathlineto{\pgfqpoint{2.122659in}{4.343765in}}%
\pgfpathlineto{\pgfqpoint{2.123733in}{4.345256in}}%
\pgfpathlineto{\pgfqpoint{2.124807in}{4.348504in}}%
\pgfpathlineto{\pgfqpoint{2.128028in}{4.344889in}}%
\pgfpathlineto{\pgfqpoint{2.129102in}{4.358226in}}%
\pgfpathlineto{\pgfqpoint{2.130176in}{4.359635in}}%
\pgfpathlineto{\pgfqpoint{2.131250in}{4.362903in}}%
\pgfpathlineto{\pgfqpoint{2.132324in}{4.369704in}}%
\pgfpathlineto{\pgfqpoint{2.136619in}{4.363924in}}%
\pgfpathlineto{\pgfqpoint{2.137693in}{4.372522in}}%
\pgfpathlineto{\pgfqpoint{2.138767in}{4.374218in}}%
\pgfpathlineto{\pgfqpoint{2.139841in}{4.374279in}}%
\pgfpathlineto{\pgfqpoint{2.143062in}{4.375075in}}%
\pgfpathlineto{\pgfqpoint{2.144136in}{4.376628in}}%
\pgfpathlineto{\pgfqpoint{2.146284in}{4.371113in}}%
\pgfpathlineto{\pgfqpoint{2.147358in}{4.377240in}}%
\pgfpathlineto{\pgfqpoint{2.151653in}{4.380161in}}%
\pgfpathlineto{\pgfqpoint{2.152727in}{4.382183in}}%
\pgfpathlineto{\pgfqpoint{2.153801in}{4.382489in}}%
\pgfpathlineto{\pgfqpoint{2.154874in}{4.381856in}}%
\pgfpathlineto{\pgfqpoint{2.158096in}{4.384185in}}%
\pgfpathlineto{\pgfqpoint{2.159170in}{4.381346in}}%
\pgfpathlineto{\pgfqpoint{2.160244in}{4.383306in}}%
\pgfpathlineto{\pgfqpoint{2.161317in}{4.386595in}}%
\pgfpathlineto{\pgfqpoint{2.162391in}{4.385247in}}%
\pgfpathlineto{\pgfqpoint{2.165613in}{4.382653in}}%
\pgfpathlineto{\pgfqpoint{2.166687in}{4.387718in}}%
\pgfpathlineto{\pgfqpoint{2.168834in}{4.387309in}}%
\pgfpathlineto{\pgfqpoint{2.169908in}{4.388596in}}%
\pgfpathlineto{\pgfqpoint{2.173130in}{4.389536in}}%
\pgfpathlineto{\pgfqpoint{2.175277in}{4.388637in}}%
\pgfpathlineto{\pgfqpoint{2.176351in}{4.388269in}}%
\pgfpathlineto{\pgfqpoint{2.177425in}{4.391966in}}%
\pgfpathlineto{\pgfqpoint{2.180647in}{4.391844in}}%
\pgfpathlineto{\pgfqpoint{2.183868in}{4.395459in}}%
\pgfpathlineto{\pgfqpoint{2.184942in}{4.398379in}}%
\pgfpathlineto{\pgfqpoint{2.191385in}{4.396766in}}%
\pgfpathlineto{\pgfqpoint{2.195680in}{4.400156in}}%
\pgfpathlineto{\pgfqpoint{2.196754in}{4.396643in}}%
\pgfpathlineto{\pgfqpoint{2.197828in}{4.401034in}}%
\pgfpathlineto{\pgfqpoint{2.198902in}{4.400810in}}%
\pgfpathlineto{\pgfqpoint{2.199976in}{4.402485in}}%
\pgfpathlineto{\pgfqpoint{2.204271in}{4.399278in}}%
\pgfpathlineto{\pgfqpoint{2.205345in}{4.401137in}}%
\pgfpathlineto{\pgfqpoint{2.206419in}{4.401790in}}%
\pgfpathlineto{\pgfqpoint{2.207492in}{4.400810in}}%
\pgfpathlineto{\pgfqpoint{2.210714in}{4.400728in}}%
\pgfpathlineto{\pgfqpoint{2.211788in}{4.403404in}}%
\pgfpathlineto{\pgfqpoint{2.212862in}{4.404445in}}%
\pgfpathlineto{\pgfqpoint{2.213936in}{4.403649in}}%
\pgfpathlineto{\pgfqpoint{2.215009in}{4.404813in}}%
\pgfpathlineto{\pgfqpoint{2.219305in}{4.406426in}}%
\pgfpathlineto{\pgfqpoint{2.221452in}{4.404588in}}%
\pgfpathlineto{\pgfqpoint{2.225748in}{4.404057in}}%
\pgfpathlineto{\pgfqpoint{2.226822in}{4.399972in}}%
\pgfpathlineto{\pgfqpoint{2.227895in}{4.402914in}}%
\pgfpathlineto{\pgfqpoint{2.228969in}{4.401504in}}%
\pgfpathlineto{\pgfqpoint{2.233265in}{4.403996in}}%
\pgfpathlineto{\pgfqpoint{2.234338in}{4.403302in}}%
\pgfpathlineto{\pgfqpoint{2.235412in}{4.401790in}}%
\pgfpathlineto{\pgfqpoint{2.236486in}{4.402934in}}%
\pgfpathlineto{\pgfqpoint{2.237560in}{4.405017in}}%
\pgfpathlineto{\pgfqpoint{2.240781in}{4.404323in}}%
\pgfpathlineto{\pgfqpoint{2.241855in}{4.407570in}}%
\pgfpathlineto{\pgfqpoint{2.242929in}{4.406651in}}%
\pgfpathlineto{\pgfqpoint{2.244003in}{4.407366in}}%
\pgfpathlineto{\pgfqpoint{2.245077in}{4.403955in}}%
\pgfpathlineto{\pgfqpoint{2.248298in}{4.406243in}}%
\pgfpathlineto{\pgfqpoint{2.249372in}{4.402995in}}%
\pgfpathlineto{\pgfqpoint{2.250446in}{4.403220in}}%
\pgfpathlineto{\pgfqpoint{2.251520in}{4.400034in}}%
\pgfpathlineto{\pgfqpoint{2.252594in}{4.399829in}}%
\pgfpathlineto{\pgfqpoint{2.255815in}{4.401892in}}%
\pgfpathlineto{\pgfqpoint{2.257963in}{4.409163in}}%
\pgfpathlineto{\pgfqpoint{2.259037in}{4.407141in}}%
\pgfpathlineto{\pgfqpoint{2.260111in}{4.407100in}}%
\pgfpathlineto{\pgfqpoint{2.264406in}{4.406038in}}%
\pgfpathlineto{\pgfqpoint{2.265480in}{4.406733in}}%
\pgfpathlineto{\pgfqpoint{2.266554in}{4.405507in}}%
\pgfpathlineto{\pgfqpoint{2.267627in}{4.406141in}}%
\pgfpathlineto{\pgfqpoint{2.272997in}{4.411839in}}%
\pgfpathlineto{\pgfqpoint{2.275144in}{4.406488in}}%
\pgfpathlineto{\pgfqpoint{2.278366in}{4.403812in}}%
\pgfpathlineto{\pgfqpoint{2.280514in}{4.405058in}}%
\pgfpathlineto{\pgfqpoint{2.281587in}{4.409061in}}%
\pgfpathlineto{\pgfqpoint{2.282661in}{4.407203in}}%
\pgfpathlineto{\pgfqpoint{2.285883in}{4.411492in}}%
\pgfpathlineto{\pgfqpoint{2.288030in}{4.411492in}}%
\pgfpathlineto{\pgfqpoint{2.289104in}{4.416884in}}%
\pgfpathlineto{\pgfqpoint{2.290178in}{4.407529in}}%
\pgfpathlineto{\pgfqpoint{2.293400in}{4.403914in}}%
\pgfpathlineto{\pgfqpoint{2.297695in}{4.419559in}}%
\pgfpathlineto{\pgfqpoint{2.301990in}{4.419723in}}%
\pgfpathlineto{\pgfqpoint{2.304138in}{4.418313in}}%
\pgfpathlineto{\pgfqpoint{2.305212in}{4.422521in}}%
\pgfpathlineto{\pgfqpoint{2.308433in}{4.424195in}}%
\pgfpathlineto{\pgfqpoint{2.309507in}{4.426340in}}%
\pgfpathlineto{\pgfqpoint{2.310581in}{4.426442in}}%
\pgfpathlineto{\pgfqpoint{2.311655in}{4.429608in}}%
\pgfpathlineto{\pgfqpoint{2.312729in}{4.430547in}}%
\pgfpathlineto{\pgfqpoint{2.317024in}{4.430241in}}%
\pgfpathlineto{\pgfqpoint{2.318098in}{4.430568in}}%
\pgfpathlineto{\pgfqpoint{2.319172in}{4.428403in}}%
\pgfpathlineto{\pgfqpoint{2.320246in}{4.428750in}}%
\pgfpathlineto{\pgfqpoint{2.323467in}{4.427218in}}%
\pgfpathlineto{\pgfqpoint{2.324541in}{4.423542in}}%
\pgfpathlineto{\pgfqpoint{2.325615in}{4.424686in}}%
\pgfpathlineto{\pgfqpoint{2.326689in}{4.424134in}}%
\pgfpathlineto{\pgfqpoint{2.327762in}{4.424849in}}%
\pgfpathlineto{\pgfqpoint{2.333132in}{4.424890in}}%
\pgfpathlineto{\pgfqpoint{2.334205in}{4.423807in}}%
\pgfpathlineto{\pgfqpoint{2.335279in}{4.425237in}}%
\pgfpathlineto{\pgfqpoint{2.340648in}{4.425707in}}%
\pgfpathlineto{\pgfqpoint{2.341722in}{4.429792in}}%
\pgfpathlineto{\pgfqpoint{2.342796in}{4.428423in}}%
\pgfpathlineto{\pgfqpoint{2.346018in}{4.428975in}}%
\pgfpathlineto{\pgfqpoint{2.347091in}{4.426503in}}%
\pgfpathlineto{\pgfqpoint{2.348165in}{4.429771in}}%
\pgfpathlineto{\pgfqpoint{2.349239in}{4.428464in}}%
\pgfpathlineto{\pgfqpoint{2.350313in}{4.429281in}}%
\pgfpathlineto{\pgfqpoint{2.353535in}{4.428260in}}%
\pgfpathlineto{\pgfqpoint{2.354608in}{4.429587in}}%
\pgfpathlineto{\pgfqpoint{2.355682in}{4.428913in}}%
\pgfpathlineto{\pgfqpoint{2.356756in}{4.429220in}}%
\pgfpathlineto{\pgfqpoint{2.357830in}{4.428913in}}%
\pgfpathlineto{\pgfqpoint{2.361051in}{4.430956in}}%
\pgfpathlineto{\pgfqpoint{2.362125in}{4.430302in}}%
\pgfpathlineto{\pgfqpoint{2.363199in}{4.428587in}}%
\pgfpathlineto{\pgfqpoint{2.365347in}{4.431548in}}%
\pgfpathlineto{\pgfqpoint{2.369642in}{4.430874in}}%
\pgfpathlineto{\pgfqpoint{2.370716in}{4.429547in}}%
\pgfpathlineto{\pgfqpoint{2.371790in}{4.430200in}}%
\pgfpathlineto{\pgfqpoint{2.372864in}{4.421806in}}%
\pgfpathlineto{\pgfqpoint{2.376085in}{4.426442in}}%
\pgfpathlineto{\pgfqpoint{2.377159in}{4.422602in}}%
\pgfpathlineto{\pgfqpoint{2.378233in}{4.421765in}}%
\pgfpathlineto{\pgfqpoint{2.379307in}{4.423583in}}%
\pgfpathlineto{\pgfqpoint{2.380380in}{4.420703in}}%
\pgfpathlineto{\pgfqpoint{2.384676in}{4.425503in}}%
\pgfpathlineto{\pgfqpoint{2.385750in}{4.429240in}}%
\pgfpathlineto{\pgfqpoint{2.386824in}{4.429730in}}%
\pgfpathlineto{\pgfqpoint{2.387897in}{4.425094in}}%
\pgfpathlineto{\pgfqpoint{2.391119in}{4.422378in}}%
\pgfpathlineto{\pgfqpoint{2.392193in}{4.423052in}}%
\pgfpathlineto{\pgfqpoint{2.393267in}{4.425482in}}%
\pgfpathlineto{\pgfqpoint{2.394340in}{4.421295in}}%
\pgfpathlineto{\pgfqpoint{2.395414in}{4.422909in}}%
\pgfpathlineto{\pgfqpoint{2.398636in}{4.420662in}}%
\pgfpathlineto{\pgfqpoint{2.399710in}{4.414331in}}%
\pgfpathlineto{\pgfqpoint{2.400783in}{4.415679in}}%
\pgfpathlineto{\pgfqpoint{2.402931in}{4.413677in}}%
\pgfpathlineto{\pgfqpoint{2.406153in}{4.413309in}}%
\pgfpathlineto{\pgfqpoint{2.407226in}{4.410573in}}%
\pgfpathlineto{\pgfqpoint{2.409374in}{4.411206in}}%
\pgfpathlineto{\pgfqpoint{2.410448in}{4.411798in}}%
\pgfpathlineto{\pgfqpoint{2.416891in}{4.410899in}}%
\pgfpathlineto{\pgfqpoint{2.417965in}{4.410225in}}%
\pgfpathlineto{\pgfqpoint{2.421186in}{4.413554in}}%
\pgfpathlineto{\pgfqpoint{2.422260in}{4.404057in}}%
\pgfpathlineto{\pgfqpoint{2.423334in}{4.404588in}}%
\pgfpathlineto{\pgfqpoint{2.424408in}{4.403179in}}%
\pgfpathlineto{\pgfqpoint{2.425482in}{4.403199in}}%
\pgfpathlineto{\pgfqpoint{2.428703in}{4.402301in}}%
\pgfpathlineto{\pgfqpoint{2.429777in}{4.400340in}}%
\pgfpathlineto{\pgfqpoint{2.431925in}{4.405201in}}%
\pgfpathlineto{\pgfqpoint{2.432999in}{4.404527in}}%
\pgfpathlineto{\pgfqpoint{2.437294in}{4.413105in}}%
\pgfpathlineto{\pgfqpoint{2.438368in}{4.411900in}}%
\pgfpathlineto{\pgfqpoint{2.439442in}{4.419232in}}%
\pgfpathlineto{\pgfqpoint{2.440515in}{4.420744in}}%
\pgfpathlineto{\pgfqpoint{2.443737in}{4.416659in}}%
\pgfpathlineto{\pgfqpoint{2.444811in}{4.419151in}}%
\pgfpathlineto{\pgfqpoint{2.445885in}{4.417027in}}%
\pgfpathlineto{\pgfqpoint{2.446958in}{4.418477in}}%
\pgfpathlineto{\pgfqpoint{2.448032in}{4.418844in}}%
\pgfpathlineto{\pgfqpoint{2.451254in}{4.416107in}}%
\pgfpathlineto{\pgfqpoint{2.453402in}{4.417496in}}%
\pgfpathlineto{\pgfqpoint{2.455549in}{4.419845in}}%
\pgfpathlineto{\pgfqpoint{2.458771in}{4.417966in}}%
\pgfpathlineto{\pgfqpoint{2.459845in}{4.418477in}}%
\pgfpathlineto{\pgfqpoint{2.460918in}{4.416516in}}%
\pgfpathlineto{\pgfqpoint{2.461992in}{4.418211in}}%
\pgfpathlineto{\pgfqpoint{2.467361in}{4.416639in}}%
\pgfpathlineto{\pgfqpoint{2.468435in}{4.424706in}}%
\pgfpathlineto{\pgfqpoint{2.469509in}{4.424379in}}%
\pgfpathlineto{\pgfqpoint{2.470583in}{4.429322in}}%
\pgfpathlineto{\pgfqpoint{2.473804in}{4.431507in}}%
\pgfpathlineto{\pgfqpoint{2.474878in}{4.429975in}}%
\pgfpathlineto{\pgfqpoint{2.475952in}{4.425748in}}%
\pgfpathlineto{\pgfqpoint{2.477026in}{4.424645in}}%
\pgfpathlineto{\pgfqpoint{2.478100in}{4.427361in}}%
\pgfpathlineto{\pgfqpoint{2.483469in}{4.429220in}}%
\pgfpathlineto{\pgfqpoint{2.484543in}{4.430670in}}%
\pgfpathlineto{\pgfqpoint{2.485617in}{4.429812in}}%
\pgfpathlineto{\pgfqpoint{2.489912in}{4.430139in}}%
\pgfpathlineto{\pgfqpoint{2.490986in}{4.428546in}}%
\pgfpathlineto{\pgfqpoint{2.493134in}{4.429485in}}%
\pgfpathlineto{\pgfqpoint{2.497429in}{4.428485in}}%
\pgfpathlineto{\pgfqpoint{2.498503in}{4.429016in}}%
\pgfpathlineto{\pgfqpoint{2.499577in}{4.427851in}}%
\pgfpathlineto{\pgfqpoint{2.500650in}{4.428832in}}%
\pgfpathlineto{\pgfqpoint{2.503872in}{4.427014in}}%
\pgfpathlineto{\pgfqpoint{2.504946in}{4.425707in}}%
\pgfpathlineto{\pgfqpoint{2.506020in}{4.428199in}}%
\pgfpathlineto{\pgfqpoint{2.507093in}{4.427341in}}%
\pgfpathlineto{\pgfqpoint{2.512463in}{4.426994in}}%
\pgfpathlineto{\pgfqpoint{2.513536in}{4.429322in}}%
\pgfpathlineto{\pgfqpoint{2.514610in}{4.429690in}}%
\pgfpathlineto{\pgfqpoint{2.515684in}{4.429322in}}%
\pgfpathlineto{\pgfqpoint{2.518906in}{4.429363in}}%
\pgfpathlineto{\pgfqpoint{2.519979in}{4.424543in}}%
\pgfpathlineto{\pgfqpoint{2.521053in}{4.425993in}}%
\pgfpathlineto{\pgfqpoint{2.522127in}{4.426156in}}%
\pgfpathlineto{\pgfqpoint{2.523201in}{4.427402in}}%
\pgfpathlineto{\pgfqpoint{2.527496in}{4.422357in}}%
\pgfpathlineto{\pgfqpoint{2.528570in}{4.423031in}}%
\pgfpathlineto{\pgfqpoint{2.529644in}{4.421152in}}%
\pgfpathlineto{\pgfqpoint{2.530718in}{4.422786in}}%
\pgfpathlineto{\pgfqpoint{2.533939in}{4.422888in}}%
\pgfpathlineto{\pgfqpoint{2.535013in}{4.424155in}}%
\pgfpathlineto{\pgfqpoint{2.538235in}{4.430302in}}%
\pgfpathlineto{\pgfqpoint{2.542530in}{4.435224in}}%
\pgfpathlineto{\pgfqpoint{2.544678in}{4.440923in}}%
\pgfpathlineto{\pgfqpoint{2.545752in}{4.440045in}}%
\pgfpathlineto{\pgfqpoint{2.550047in}{4.440841in}}%
\pgfpathlineto{\pgfqpoint{2.551121in}{4.445947in}}%
\pgfpathlineto{\pgfqpoint{2.552195in}{4.448153in}}%
\pgfpathlineto{\pgfqpoint{2.553268in}{4.448561in}}%
\pgfpathlineto{\pgfqpoint{2.556490in}{4.447601in}}%
\pgfpathlineto{\pgfqpoint{2.557564in}{4.446539in}}%
\pgfpathlineto{\pgfqpoint{2.558638in}{4.453239in}}%
\pgfpathlineto{\pgfqpoint{2.559712in}{4.453300in}}%
\pgfpathlineto{\pgfqpoint{2.560785in}{4.452197in}}%
\pgfpathlineto{\pgfqpoint{2.564007in}{4.451400in}}%
\pgfpathlineto{\pgfqpoint{2.566155in}{4.452565in}}%
\pgfpathlineto{\pgfqpoint{2.567228in}{4.453320in}}%
\pgfpathlineto{\pgfqpoint{2.568302in}{4.455812in}}%
\pgfpathlineto{\pgfqpoint{2.571524in}{4.456404in}}%
\pgfpathlineto{\pgfqpoint{2.572598in}{4.454137in}}%
\pgfpathlineto{\pgfqpoint{2.573671in}{4.455792in}}%
\pgfpathlineto{\pgfqpoint{2.574745in}{4.454096in}}%
\pgfpathlineto{\pgfqpoint{2.575819in}{4.458018in}}%
\pgfpathlineto{\pgfqpoint{2.579041in}{4.459264in}}%
\pgfpathlineto{\pgfqpoint{2.580114in}{4.457568in}}%
\pgfpathlineto{\pgfqpoint{2.582262in}{4.457589in}}%
\pgfpathlineto{\pgfqpoint{2.583336in}{4.456384in}}%
\pgfpathlineto{\pgfqpoint{2.586557in}{4.454321in}}%
\pgfpathlineto{\pgfqpoint{2.587631in}{4.455403in}}%
\pgfpathlineto{\pgfqpoint{2.588705in}{4.454811in}}%
\pgfpathlineto{\pgfqpoint{2.589779in}{4.455955in}}%
\pgfpathlineto{\pgfqpoint{2.590853in}{4.456057in}}%
\pgfpathlineto{\pgfqpoint{2.594074in}{4.454893in}}%
\pgfpathlineto{\pgfqpoint{2.595148in}{4.453872in}}%
\pgfpathlineto{\pgfqpoint{2.596222in}{4.453974in}}%
\pgfpathlineto{\pgfqpoint{2.597296in}{4.453300in}}%
\pgfpathlineto{\pgfqpoint{2.598370in}{4.453484in}}%
\pgfpathlineto{\pgfqpoint{2.601591in}{4.452953in}}%
\pgfpathlineto{\pgfqpoint{2.602665in}{4.453647in}}%
\pgfpathlineto{\pgfqpoint{2.603739in}{4.452932in}}%
\pgfpathlineto{\pgfqpoint{2.604813in}{4.450931in}}%
\pgfpathlineto{\pgfqpoint{2.609108in}{4.454198in}}%
\pgfpathlineto{\pgfqpoint{2.611256in}{4.453177in}}%
\pgfpathlineto{\pgfqpoint{2.612330in}{4.455730in}}%
\pgfpathlineto{\pgfqpoint{2.613403in}{4.456363in}}%
\pgfpathlineto{\pgfqpoint{2.617699in}{4.463308in}}%
\pgfpathlineto{\pgfqpoint{2.618773in}{4.463063in}}%
\pgfpathlineto{\pgfqpoint{2.619846in}{4.465166in}}%
\pgfpathlineto{\pgfqpoint{2.624142in}{4.462695in}}%
\pgfpathlineto{\pgfqpoint{2.627363in}{4.471661in}}%
\pgfpathlineto{\pgfqpoint{2.628437in}{4.471457in}}%
\pgfpathlineto{\pgfqpoint{2.631659in}{4.470027in}}%
\pgfpathlineto{\pgfqpoint{2.632733in}{4.468700in}}%
\pgfpathlineto{\pgfqpoint{2.633806in}{4.466187in}}%
\pgfpathlineto{\pgfqpoint{2.635954in}{4.466044in}}%
\pgfpathlineto{\pgfqpoint{2.640249in}{4.468618in}}%
\pgfpathlineto{\pgfqpoint{2.641323in}{4.464962in}}%
\pgfpathlineto{\pgfqpoint{2.643471in}{4.466821in}}%
\pgfpathlineto{\pgfqpoint{2.646692in}{4.472029in}}%
\pgfpathlineto{\pgfqpoint{2.647766in}{4.470517in}}%
\pgfpathlineto{\pgfqpoint{2.648840in}{4.470068in}}%
\pgfpathlineto{\pgfqpoint{2.650988in}{4.476175in}}%
\pgfpathlineto{\pgfqpoint{2.655283in}{4.479565in}}%
\pgfpathlineto{\pgfqpoint{2.656357in}{4.483466in}}%
\pgfpathlineto{\pgfqpoint{2.657431in}{4.483242in}}%
\pgfpathlineto{\pgfqpoint{2.658505in}{4.487755in}}%
\pgfpathlineto{\pgfqpoint{2.661726in}{4.486836in}}%
\pgfpathlineto{\pgfqpoint{2.663874in}{4.484508in}}%
\pgfpathlineto{\pgfqpoint{2.666022in}{4.488184in}}%
\pgfpathlineto{\pgfqpoint{2.669243in}{4.489083in}}%
\pgfpathlineto{\pgfqpoint{2.671391in}{4.494107in}}%
\pgfpathlineto{\pgfqpoint{2.673538in}{4.500316in}}%
\pgfpathlineto{\pgfqpoint{2.677834in}{4.500541in}}%
\pgfpathlineto{\pgfqpoint{2.679981in}{4.498355in}}%
\pgfpathlineto{\pgfqpoint{2.681055in}{4.499663in}}%
\pgfpathlineto{\pgfqpoint{2.684277in}{4.499091in}}%
\pgfpathlineto{\pgfqpoint{2.685351in}{4.493658in}}%
\pgfpathlineto{\pgfqpoint{2.686424in}{4.495271in}}%
\pgfpathlineto{\pgfqpoint{2.687498in}{4.489961in}}%
\pgfpathlineto{\pgfqpoint{2.688572in}{4.490615in}}%
\pgfpathlineto{\pgfqpoint{2.691794in}{4.493760in}}%
\pgfpathlineto{\pgfqpoint{2.693941in}{4.493617in}}%
\pgfpathlineto{\pgfqpoint{2.695015in}{4.490288in}}%
\pgfpathlineto{\pgfqpoint{2.696089in}{4.493290in}}%
\pgfpathlineto{\pgfqpoint{2.699311in}{4.495026in}}%
\pgfpathlineto{\pgfqpoint{2.700384in}{4.493433in}}%
\pgfpathlineto{\pgfqpoint{2.701458in}{4.496579in}}%
\pgfpathlineto{\pgfqpoint{2.702532in}{4.496190in}}%
\pgfpathlineto{\pgfqpoint{2.703606in}{4.497498in}}%
\pgfpathlineto{\pgfqpoint{2.706827in}{4.497314in}}%
\pgfpathlineto{\pgfqpoint{2.707901in}{4.496599in}}%
\pgfpathlineto{\pgfqpoint{2.710049in}{4.498784in}}%
\pgfpathlineto{\pgfqpoint{2.711123in}{4.496313in}}%
\pgfpathlineto{\pgfqpoint{2.714344in}{4.494087in}}%
\pgfpathlineto{\pgfqpoint{2.715418in}{4.473703in}}%
\pgfpathlineto{\pgfqpoint{2.716492in}{4.473009in}}%
\pgfpathlineto{\pgfqpoint{2.717566in}{4.474970in}}%
\pgfpathlineto{\pgfqpoint{2.718640in}{4.474337in}}%
\pgfpathlineto{\pgfqpoint{2.721861in}{4.477135in}}%
\pgfpathlineto{\pgfqpoint{2.725083in}{4.489512in}}%
\pgfpathlineto{\pgfqpoint{2.726156in}{4.489573in}}%
\pgfpathlineto{\pgfqpoint{2.729378in}{4.489165in}}%
\pgfpathlineto{\pgfqpoint{2.730452in}{4.487224in}}%
\pgfpathlineto{\pgfqpoint{2.731526in}{4.487326in}}%
\pgfpathlineto{\pgfqpoint{2.733673in}{4.486366in}}%
\pgfpathlineto{\pgfqpoint{2.736895in}{4.489042in}}%
\pgfpathlineto{\pgfqpoint{2.737969in}{4.488674in}}%
\pgfpathlineto{\pgfqpoint{2.739043in}{4.490084in}}%
\pgfpathlineto{\pgfqpoint{2.741190in}{4.481669in}}%
\pgfpathlineto{\pgfqpoint{2.744412in}{4.483630in}}%
\pgfpathlineto{\pgfqpoint{2.745486in}{4.485366in}}%
\pgfpathlineto{\pgfqpoint{2.746559in}{4.482241in}}%
\pgfpathlineto{\pgfqpoint{2.747633in}{4.481179in}}%
\pgfpathlineto{\pgfqpoint{2.748707in}{4.481199in}}%
\pgfpathlineto{\pgfqpoint{2.751929in}{4.481853in}}%
\pgfpathlineto{\pgfqpoint{2.754076in}{4.484222in}}%
\pgfpathlineto{\pgfqpoint{2.755150in}{4.485427in}}%
\pgfpathlineto{\pgfqpoint{2.760519in}{4.478993in}}%
\pgfpathlineto{\pgfqpoint{2.761593in}{4.481056in}}%
\pgfpathlineto{\pgfqpoint{2.763741in}{4.488082in}}%
\pgfpathlineto{\pgfqpoint{2.766962in}{4.495557in}}%
\pgfpathlineto{\pgfqpoint{2.769110in}{4.495414in}}%
\pgfpathlineto{\pgfqpoint{2.771258in}{4.502869in}}%
\pgfpathlineto{\pgfqpoint{2.774479in}{4.503666in}}%
\pgfpathlineto{\pgfqpoint{2.775553in}{4.503278in}}%
\pgfpathlineto{\pgfqpoint{2.776627in}{4.497395in}}%
\pgfpathlineto{\pgfqpoint{2.777701in}{4.497252in}}%
\pgfpathlineto{\pgfqpoint{2.778775in}{4.497886in}}%
\pgfpathlineto{\pgfqpoint{2.781996in}{4.497824in}}%
\pgfpathlineto{\pgfqpoint{2.783070in}{4.498396in}}%
\pgfpathlineto{\pgfqpoint{2.784144in}{4.495394in}}%
\pgfpathlineto{\pgfqpoint{2.786291in}{4.496211in}}%
\pgfpathlineto{\pgfqpoint{2.789513in}{4.501746in}}%
\pgfpathlineto{\pgfqpoint{2.791661in}{4.508996in}}%
\pgfpathlineto{\pgfqpoint{2.793808in}{4.508996in}}%
\pgfpathlineto{\pgfqpoint{2.799178in}{4.508976in}}%
\pgfpathlineto{\pgfqpoint{2.800251in}{4.511059in}}%
\pgfpathlineto{\pgfqpoint{2.801325in}{4.511325in}}%
\pgfpathlineto{\pgfqpoint{2.804547in}{4.513245in}}%
\pgfpathlineto{\pgfqpoint{2.805621in}{4.511386in}}%
\pgfpathlineto{\pgfqpoint{2.806694in}{4.512387in}}%
\pgfpathlineto{\pgfqpoint{2.807768in}{4.514266in}}%
\pgfpathlineto{\pgfqpoint{2.808842in}{4.518269in}}%
\pgfpathlineto{\pgfqpoint{2.812064in}{4.518718in}}%
\pgfpathlineto{\pgfqpoint{2.813137in}{4.544024in}}%
\pgfpathlineto{\pgfqpoint{2.814211in}{4.549885in}}%
\pgfpathlineto{\pgfqpoint{2.815285in}{4.540715in}}%
\pgfpathlineto{\pgfqpoint{2.816359in}{4.544208in}}%
\pgfpathlineto{\pgfqpoint{2.820654in}{4.535405in}}%
\pgfpathlineto{\pgfqpoint{2.821728in}{4.535384in}}%
\pgfpathlineto{\pgfqpoint{2.822802in}{4.539347in}}%
\pgfpathlineto{\pgfqpoint{2.823876in}{4.539326in}}%
\pgfpathlineto{\pgfqpoint{2.828171in}{4.535139in}}%
\pgfpathlineto{\pgfqpoint{2.829245in}{4.534710in}}%
\pgfpathlineto{\pgfqpoint{2.831393in}{4.530115in}}%
\pgfpathlineto{\pgfqpoint{2.834614in}{4.531606in}}%
\pgfpathlineto{\pgfqpoint{2.835688in}{4.533750in}}%
\pgfpathlineto{\pgfqpoint{2.836762in}{4.530013in}}%
\pgfpathlineto{\pgfqpoint{2.837836in}{4.533914in}}%
\pgfpathlineto{\pgfqpoint{2.838910in}{4.533812in}}%
\pgfpathlineto{\pgfqpoint{2.842131in}{4.537917in}}%
\pgfpathlineto{\pgfqpoint{2.843205in}{4.542941in}}%
\pgfpathlineto{\pgfqpoint{2.844279in}{4.540347in}}%
\pgfpathlineto{\pgfqpoint{2.846426in}{4.539959in}}%
\pgfpathlineto{\pgfqpoint{2.849648in}{4.545045in}}%
\pgfpathlineto{\pgfqpoint{2.851796in}{4.552827in}}%
\pgfpathlineto{\pgfqpoint{2.852869in}{4.562793in}}%
\pgfpathlineto{\pgfqpoint{2.853943in}{4.558933in}}%
\pgfpathlineto{\pgfqpoint{2.858239in}{4.553317in}}%
\pgfpathlineto{\pgfqpoint{2.859312in}{4.554113in}}%
\pgfpathlineto{\pgfqpoint{2.860386in}{4.557279in}}%
\pgfpathlineto{\pgfqpoint{2.861460in}{4.553072in}}%
\pgfpathlineto{\pgfqpoint{2.864682in}{4.555339in}}%
\pgfpathlineto{\pgfqpoint{2.865756in}{4.550049in}}%
\pgfpathlineto{\pgfqpoint{2.866829in}{4.554971in}}%
\pgfpathlineto{\pgfqpoint{2.867903in}{4.552969in}}%
\pgfpathlineto{\pgfqpoint{2.868977in}{4.552806in}}%
\pgfpathlineto{\pgfqpoint{2.873272in}{4.553562in}}%
\pgfpathlineto{\pgfqpoint{2.876494in}{4.546475in}}%
\pgfpathlineto{\pgfqpoint{2.880789in}{4.547863in}}%
\pgfpathlineto{\pgfqpoint{2.881863in}{4.549314in}}%
\pgfpathlineto{\pgfqpoint{2.884011in}{4.547700in}}%
\pgfpathlineto{\pgfqpoint{2.884011in}{4.547700in}}%
\pgfusepath{stroke}%
\end{pgfscope}%
\begin{pgfscope}%
\pgfpathrectangle{\pgfqpoint{0.418102in}{4.180131in}}{\pgfqpoint{2.583333in}{0.400885in}}%
\pgfusepath{clip}%
\pgfsetroundcap%
\pgfsetroundjoin%
\pgfsetlinewidth{1.505625pt}%
\definecolor{currentstroke}{rgb}{0.121569,0.466667,0.705882}%
\pgfsetstrokecolor{currentstroke}%
\pgfsetdash{}{0pt}%
\pgfpathmoveto{\pgfqpoint{0.535526in}{4.230473in}}%
\pgfpathlineto{\pgfqpoint{0.536600in}{4.230473in}}%
\pgfpathlineto{\pgfqpoint{0.537674in}{4.231101in}}%
\pgfpathlineto{\pgfqpoint{0.538747in}{4.230385in}}%
\pgfpathlineto{\pgfqpoint{0.541969in}{4.231223in}}%
\pgfpathlineto{\pgfqpoint{0.543043in}{4.230508in}}%
\pgfpathlineto{\pgfqpoint{0.546264in}{4.233406in}}%
\pgfpathlineto{\pgfqpoint{0.550560in}{4.234489in}}%
\pgfpathlineto{\pgfqpoint{0.552707in}{4.231827in}}%
\pgfpathlineto{\pgfqpoint{0.553781in}{4.232088in}}%
\pgfpathlineto{\pgfqpoint{0.559150in}{4.231645in}}%
\pgfpathlineto{\pgfqpoint{0.560224in}{4.229836in}}%
\pgfpathlineto{\pgfqpoint{0.561298in}{4.230029in}}%
\pgfpathlineto{\pgfqpoint{0.564520in}{4.229835in}}%
\pgfpathlineto{\pgfqpoint{0.565593in}{4.228825in}}%
\pgfpathlineto{\pgfqpoint{0.566667in}{4.229855in}}%
\pgfpathlineto{\pgfqpoint{0.567741in}{4.229738in}}%
\pgfpathlineto{\pgfqpoint{0.568815in}{4.230223in}}%
\pgfpathlineto{\pgfqpoint{0.575258in}{4.230962in}}%
\pgfpathlineto{\pgfqpoint{0.576332in}{4.229562in}}%
\pgfpathlineto{\pgfqpoint{0.580627in}{4.231020in}}%
\pgfpathlineto{\pgfqpoint{0.581701in}{4.230398in}}%
\pgfpathlineto{\pgfqpoint{0.582775in}{4.231429in}}%
\pgfpathlineto{\pgfqpoint{0.589218in}{4.231916in}}%
\pgfpathlineto{\pgfqpoint{0.590292in}{4.231701in}}%
\pgfpathlineto{\pgfqpoint{0.591366in}{4.232187in}}%
\pgfpathlineto{\pgfqpoint{0.595661in}{4.231933in}}%
\pgfpathlineto{\pgfqpoint{0.602104in}{4.230783in}}%
\pgfpathlineto{\pgfqpoint{0.603178in}{4.227390in}}%
\pgfpathlineto{\pgfqpoint{0.604252in}{4.228229in}}%
\pgfpathlineto{\pgfqpoint{0.605325in}{4.226240in}}%
\pgfpathlineto{\pgfqpoint{0.606399in}{4.226088in}}%
\pgfpathlineto{\pgfqpoint{0.609621in}{4.224936in}}%
\pgfpathlineto{\pgfqpoint{0.610695in}{4.223078in}}%
\pgfpathlineto{\pgfqpoint{0.611768in}{4.222915in}}%
\pgfpathlineto{\pgfqpoint{0.613916in}{4.225277in}}%
\pgfpathlineto{\pgfqpoint{0.619285in}{4.225471in}}%
\pgfpathlineto{\pgfqpoint{0.621433in}{4.224792in}}%
\pgfpathlineto{\pgfqpoint{0.625728in}{4.225747in}}%
\pgfpathlineto{\pgfqpoint{0.626802in}{4.224700in}}%
\pgfpathlineto{\pgfqpoint{0.627876in}{4.225178in}}%
\pgfpathlineto{\pgfqpoint{0.628950in}{4.224517in}}%
\pgfpathlineto{\pgfqpoint{0.632171in}{4.224480in}}%
\pgfpathlineto{\pgfqpoint{0.633245in}{4.223826in}}%
\pgfpathlineto{\pgfqpoint{0.635393in}{4.221537in}}%
\pgfpathlineto{\pgfqpoint{0.639688in}{4.220138in}}%
\pgfpathlineto{\pgfqpoint{0.640762in}{4.217522in}}%
\pgfpathlineto{\pgfqpoint{0.641836in}{4.218667in}}%
\pgfpathlineto{\pgfqpoint{0.642910in}{4.216396in}}%
\pgfpathlineto{\pgfqpoint{0.643984in}{4.218097in}}%
\pgfpathlineto{\pgfqpoint{0.647205in}{4.219160in}}%
\pgfpathlineto{\pgfqpoint{0.648279in}{4.217610in}}%
\pgfpathlineto{\pgfqpoint{0.649353in}{4.218068in}}%
\pgfpathlineto{\pgfqpoint{0.650427in}{4.217589in}}%
\pgfpathlineto{\pgfqpoint{0.651500in}{4.218581in}}%
\pgfpathlineto{\pgfqpoint{0.654722in}{4.219095in}}%
\pgfpathlineto{\pgfqpoint{0.655796in}{4.221095in}}%
\pgfpathlineto{\pgfqpoint{0.659017in}{4.219816in}}%
\pgfpathlineto{\pgfqpoint{0.665460in}{4.220095in}}%
\pgfpathlineto{\pgfqpoint{0.666534in}{4.219060in}}%
\pgfpathlineto{\pgfqpoint{0.672977in}{4.216833in}}%
\pgfpathlineto{\pgfqpoint{0.677273in}{4.214956in}}%
\pgfpathlineto{\pgfqpoint{0.679420in}{4.215114in}}%
\pgfpathlineto{\pgfqpoint{0.680494in}{4.216587in}}%
\pgfpathlineto{\pgfqpoint{0.681568in}{4.215348in}}%
\pgfpathlineto{\pgfqpoint{0.689085in}{4.218319in}}%
\pgfpathlineto{\pgfqpoint{0.693380in}{4.219750in}}%
\pgfpathlineto{\pgfqpoint{0.694454in}{4.221670in}}%
\pgfpathlineto{\pgfqpoint{0.695528in}{4.221614in}}%
\pgfpathlineto{\pgfqpoint{0.696602in}{4.219188in}}%
\pgfpathlineto{\pgfqpoint{0.700897in}{4.218665in}}%
\pgfpathlineto{\pgfqpoint{0.701971in}{4.221950in}}%
\pgfpathlineto{\pgfqpoint{0.704119in}{4.219838in}}%
\pgfpathlineto{\pgfqpoint{0.707340in}{4.221067in}}%
\pgfpathlineto{\pgfqpoint{0.708414in}{4.223424in}}%
\pgfpathlineto{\pgfqpoint{0.710562in}{4.225447in}}%
\pgfpathlineto{\pgfqpoint{0.711635in}{4.224551in}}%
\pgfpathlineto{\pgfqpoint{0.714857in}{4.224735in}}%
\pgfpathlineto{\pgfqpoint{0.717005in}{4.225936in}}%
\pgfpathlineto{\pgfqpoint{0.718078in}{4.224671in}}%
\pgfpathlineto{\pgfqpoint{0.719152in}{4.224839in}}%
\pgfpathlineto{\pgfqpoint{0.723448in}{4.226896in}}%
\pgfpathlineto{\pgfqpoint{0.724521in}{4.225355in}}%
\pgfpathlineto{\pgfqpoint{0.725595in}{4.225355in}}%
\pgfpathlineto{\pgfqpoint{0.726669in}{4.229129in}}%
\pgfpathlineto{\pgfqpoint{0.733112in}{4.230493in}}%
\pgfpathlineto{\pgfqpoint{0.734186in}{4.229591in}}%
\pgfpathlineto{\pgfqpoint{0.737408in}{4.229554in}}%
\pgfpathlineto{\pgfqpoint{0.740629in}{4.225572in}}%
\pgfpathlineto{\pgfqpoint{0.741703in}{4.227412in}}%
\pgfpathlineto{\pgfqpoint{0.744924in}{4.226624in}}%
\pgfpathlineto{\pgfqpoint{0.745998in}{4.225268in}}%
\pgfpathlineto{\pgfqpoint{0.747072in}{4.222411in}}%
\pgfpathlineto{\pgfqpoint{0.748146in}{4.222463in}}%
\pgfpathlineto{\pgfqpoint{0.749220in}{4.221241in}}%
\pgfpathlineto{\pgfqpoint{0.752441in}{4.220333in}}%
\pgfpathlineto{\pgfqpoint{0.753515in}{4.218710in}}%
\pgfpathlineto{\pgfqpoint{0.754589in}{4.219443in}}%
\pgfpathlineto{\pgfqpoint{0.756737in}{4.215215in}}%
\pgfpathlineto{\pgfqpoint{0.762106in}{4.215704in}}%
\pgfpathlineto{\pgfqpoint{0.763180in}{4.214170in}}%
\pgfpathlineto{\pgfqpoint{0.764254in}{4.216380in}}%
\pgfpathlineto{\pgfqpoint{0.770697in}{4.217224in}}%
\pgfpathlineto{\pgfqpoint{0.771770in}{4.218188in}}%
\pgfpathlineto{\pgfqpoint{0.776066in}{4.218171in}}%
\pgfpathlineto{\pgfqpoint{0.777140in}{4.218504in}}%
\pgfpathlineto{\pgfqpoint{0.779287in}{4.216178in}}%
\pgfpathlineto{\pgfqpoint{0.782509in}{4.216642in}}%
\pgfpathlineto{\pgfqpoint{0.783583in}{4.215739in}}%
\pgfpathlineto{\pgfqpoint{0.784656in}{4.215819in}}%
\pgfpathlineto{\pgfqpoint{0.785730in}{4.214867in}}%
\pgfpathlineto{\pgfqpoint{0.786804in}{4.216013in}}%
\pgfpathlineto{\pgfqpoint{0.793247in}{4.217033in}}%
\pgfpathlineto{\pgfqpoint{0.794321in}{4.218186in}}%
\pgfpathlineto{\pgfqpoint{0.799690in}{4.219554in}}%
\pgfpathlineto{\pgfqpoint{0.800764in}{4.217405in}}%
\pgfpathlineto{\pgfqpoint{0.801838in}{4.218038in}}%
\pgfpathlineto{\pgfqpoint{0.805059in}{4.215102in}}%
\pgfpathlineto{\pgfqpoint{0.807207in}{4.216299in}}%
\pgfpathlineto{\pgfqpoint{0.808281in}{4.218036in}}%
\pgfpathlineto{\pgfqpoint{0.809355in}{4.215390in}}%
\pgfpathlineto{\pgfqpoint{0.812576in}{4.215643in}}%
\pgfpathlineto{\pgfqpoint{0.813650in}{4.215165in}}%
\pgfpathlineto{\pgfqpoint{0.815798in}{4.215500in}}%
\pgfpathlineto{\pgfqpoint{0.816872in}{4.215006in}}%
\pgfpathlineto{\pgfqpoint{0.820093in}{4.215707in}}%
\pgfpathlineto{\pgfqpoint{0.821167in}{4.216919in}}%
\pgfpathlineto{\pgfqpoint{0.822241in}{4.216594in}}%
\pgfpathlineto{\pgfqpoint{0.828684in}{4.218292in}}%
\pgfpathlineto{\pgfqpoint{0.830832in}{4.217090in}}%
\pgfpathlineto{\pgfqpoint{0.831905in}{4.216386in}}%
\pgfpathlineto{\pgfqpoint{0.835127in}{4.215832in}}%
\pgfpathlineto{\pgfqpoint{0.836201in}{4.217714in}}%
\pgfpathlineto{\pgfqpoint{0.838348in}{4.216211in}}%
\pgfpathlineto{\pgfqpoint{0.842644in}{4.216162in}}%
\pgfpathlineto{\pgfqpoint{0.843718in}{4.218102in}}%
\pgfpathlineto{\pgfqpoint{0.844791in}{4.217310in}}%
\pgfpathlineto{\pgfqpoint{0.845865in}{4.217406in}}%
\pgfpathlineto{\pgfqpoint{0.846939in}{4.214991in}}%
\pgfpathlineto{\pgfqpoint{0.850161in}{4.214447in}}%
\pgfpathlineto{\pgfqpoint{0.851234in}{4.209378in}}%
\pgfpathlineto{\pgfqpoint{0.854456in}{4.208434in}}%
\pgfpathlineto{\pgfqpoint{0.859825in}{4.209010in}}%
\pgfpathlineto{\pgfqpoint{0.860899in}{4.211239in}}%
\pgfpathlineto{\pgfqpoint{0.865194in}{4.212455in}}%
\pgfpathlineto{\pgfqpoint{0.866268in}{4.210733in}}%
\pgfpathlineto{\pgfqpoint{0.867342in}{4.212691in}}%
\pgfpathlineto{\pgfqpoint{0.868416in}{4.211549in}}%
\pgfpathlineto{\pgfqpoint{0.869490in}{4.211908in}}%
\pgfpathlineto{\pgfqpoint{0.873785in}{4.211532in}}%
\pgfpathlineto{\pgfqpoint{0.874859in}{4.209131in}}%
\pgfpathlineto{\pgfqpoint{0.875933in}{4.210088in}}%
\pgfpathlineto{\pgfqpoint{0.880228in}{4.207995in}}%
\pgfpathlineto{\pgfqpoint{0.882376in}{4.207839in}}%
\pgfpathlineto{\pgfqpoint{0.884523in}{4.209607in}}%
\pgfpathlineto{\pgfqpoint{0.888819in}{4.210156in}}%
\pgfpathlineto{\pgfqpoint{0.889893in}{4.209339in}}%
\pgfpathlineto{\pgfqpoint{0.895262in}{4.210974in}}%
\pgfpathlineto{\pgfqpoint{0.896336in}{4.210737in}}%
\pgfpathlineto{\pgfqpoint{0.897409in}{4.211591in}}%
\pgfpathlineto{\pgfqpoint{0.902779in}{4.210130in}}%
\pgfpathlineto{\pgfqpoint{0.903853in}{4.207783in}}%
\pgfpathlineto{\pgfqpoint{0.904926in}{4.208481in}}%
\pgfpathlineto{\pgfqpoint{0.907074in}{4.207414in}}%
\pgfpathlineto{\pgfqpoint{0.910296in}{4.208376in}}%
\pgfpathlineto{\pgfqpoint{0.911369in}{4.207339in}}%
\pgfpathlineto{\pgfqpoint{0.914591in}{4.211077in}}%
\pgfpathlineto{\pgfqpoint{0.921034in}{4.210799in}}%
\pgfpathlineto{\pgfqpoint{0.922108in}{4.209675in}}%
\pgfpathlineto{\pgfqpoint{0.925329in}{4.211061in}}%
\pgfpathlineto{\pgfqpoint{0.927477in}{4.208551in}}%
\pgfpathlineto{\pgfqpoint{0.928551in}{4.208684in}}%
\pgfpathlineto{\pgfqpoint{0.929625in}{4.209543in}}%
\pgfpathlineto{\pgfqpoint{0.933920in}{4.209380in}}%
\pgfpathlineto{\pgfqpoint{0.934994in}{4.210516in}}%
\pgfpathlineto{\pgfqpoint{0.936068in}{4.209926in}}%
\pgfpathlineto{\pgfqpoint{0.937142in}{4.210671in}}%
\pgfpathlineto{\pgfqpoint{0.940363in}{4.211663in}}%
\pgfpathlineto{\pgfqpoint{0.943585in}{4.210413in}}%
\pgfpathlineto{\pgfqpoint{0.944658in}{4.209616in}}%
\pgfpathlineto{\pgfqpoint{0.951101in}{4.208505in}}%
\pgfpathlineto{\pgfqpoint{0.952175in}{4.207410in}}%
\pgfpathlineto{\pgfqpoint{0.955397in}{4.207341in}}%
\pgfpathlineto{\pgfqpoint{0.957544in}{4.209847in}}%
\pgfpathlineto{\pgfqpoint{0.958618in}{4.209552in}}%
\pgfpathlineto{\pgfqpoint{0.959692in}{4.210745in}}%
\pgfpathlineto{\pgfqpoint{0.963987in}{4.212527in}}%
\pgfpathlineto{\pgfqpoint{0.965061in}{4.211087in}}%
\pgfpathlineto{\pgfqpoint{0.967209in}{4.212150in}}%
\pgfpathlineto{\pgfqpoint{0.970431in}{4.212193in}}%
\pgfpathlineto{\pgfqpoint{0.971504in}{4.213177in}}%
\pgfpathlineto{\pgfqpoint{0.979021in}{4.212460in}}%
\pgfpathlineto{\pgfqpoint{0.980095in}{4.213661in}}%
\pgfpathlineto{\pgfqpoint{0.981169in}{4.213140in}}%
\pgfpathlineto{\pgfqpoint{0.982243in}{4.214125in}}%
\pgfpathlineto{\pgfqpoint{0.986538in}{4.216963in}}%
\pgfpathlineto{\pgfqpoint{0.987612in}{4.215419in}}%
\pgfpathlineto{\pgfqpoint{0.988686in}{4.214893in}}%
\pgfpathlineto{\pgfqpoint{0.989760in}{4.215175in}}%
\pgfpathlineto{\pgfqpoint{0.992981in}{4.214594in}}%
\pgfpathlineto{\pgfqpoint{0.994055in}{4.215996in}}%
\pgfpathlineto{\pgfqpoint{0.996203in}{4.215896in}}%
\pgfpathlineto{\pgfqpoint{0.997276in}{4.217310in}}%
\pgfpathlineto{\pgfqpoint{1.000498in}{4.217197in}}%
\pgfpathlineto{\pgfqpoint{1.001572in}{4.216379in}}%
\pgfpathlineto{\pgfqpoint{1.002646in}{4.216336in}}%
\pgfpathlineto{\pgfqpoint{1.003720in}{4.217451in}}%
\pgfpathlineto{\pgfqpoint{1.004793in}{4.217000in}}%
\pgfpathlineto{\pgfqpoint{1.008015in}{4.218177in}}%
\pgfpathlineto{\pgfqpoint{1.009089in}{4.217906in}}%
\pgfpathlineto{\pgfqpoint{1.011236in}{4.219348in}}%
\pgfpathlineto{\pgfqpoint{1.012310in}{4.221171in}}%
\pgfpathlineto{\pgfqpoint{1.016606in}{4.223832in}}%
\pgfpathlineto{\pgfqpoint{1.018753in}{4.226103in}}%
\pgfpathlineto{\pgfqpoint{1.023049in}{4.226945in}}%
\pgfpathlineto{\pgfqpoint{1.025196in}{4.229153in}}%
\pgfpathlineto{\pgfqpoint{1.027344in}{4.229400in}}%
\pgfpathlineto{\pgfqpoint{1.031639in}{4.229632in}}%
\pgfpathlineto{\pgfqpoint{1.032713in}{4.227357in}}%
\pgfpathlineto{\pgfqpoint{1.033787in}{4.226637in}}%
\pgfpathlineto{\pgfqpoint{1.034861in}{4.227364in}}%
\pgfpathlineto{\pgfqpoint{1.038082in}{4.224936in}}%
\pgfpathlineto{\pgfqpoint{1.040230in}{4.226285in}}%
\pgfpathlineto{\pgfqpoint{1.041304in}{4.225648in}}%
\pgfpathlineto{\pgfqpoint{1.042378in}{4.226588in}}%
\pgfpathlineto{\pgfqpoint{1.045599in}{4.226482in}}%
\pgfpathlineto{\pgfqpoint{1.046673in}{4.228528in}}%
\pgfpathlineto{\pgfqpoint{1.047747in}{4.227922in}}%
\pgfpathlineto{\pgfqpoint{1.048821in}{4.231722in}}%
\pgfpathlineto{\pgfqpoint{1.049895in}{4.230261in}}%
\pgfpathlineto{\pgfqpoint{1.053116in}{4.230309in}}%
\pgfpathlineto{\pgfqpoint{1.054190in}{4.231499in}}%
\pgfpathlineto{\pgfqpoint{1.055264in}{4.231722in}}%
\pgfpathlineto{\pgfqpoint{1.056338in}{4.233728in}}%
\pgfpathlineto{\pgfqpoint{1.057411in}{4.231307in}}%
\pgfpathlineto{\pgfqpoint{1.062781in}{4.231261in}}%
\pgfpathlineto{\pgfqpoint{1.063854in}{4.229022in}}%
\pgfpathlineto{\pgfqpoint{1.064928in}{4.228324in}}%
\pgfpathlineto{\pgfqpoint{1.069224in}{4.228412in}}%
\pgfpathlineto{\pgfqpoint{1.071371in}{4.230088in}}%
\pgfpathlineto{\pgfqpoint{1.072445in}{4.230488in}}%
\pgfpathlineto{\pgfqpoint{1.076741in}{4.229705in}}%
\pgfpathlineto{\pgfqpoint{1.077814in}{4.230437in}}%
\pgfpathlineto{\pgfqpoint{1.079962in}{4.229786in}}%
\pgfpathlineto{\pgfqpoint{1.086405in}{4.232395in}}%
\pgfpathlineto{\pgfqpoint{1.087479in}{4.233858in}}%
\pgfpathlineto{\pgfqpoint{1.090700in}{4.234300in}}%
\pgfpathlineto{\pgfqpoint{1.091774in}{4.235062in}}%
\pgfpathlineto{\pgfqpoint{1.092848in}{4.233183in}}%
\pgfpathlineto{\pgfqpoint{1.093922in}{4.233291in}}%
\pgfpathlineto{\pgfqpoint{1.094996in}{4.236479in}}%
\pgfpathlineto{\pgfqpoint{1.098217in}{4.236864in}}%
\pgfpathlineto{\pgfqpoint{1.099291in}{4.236090in}}%
\pgfpathlineto{\pgfqpoint{1.100365in}{4.234510in}}%
\pgfpathlineto{\pgfqpoint{1.101439in}{4.237375in}}%
\pgfpathlineto{\pgfqpoint{1.105734in}{4.238757in}}%
\pgfpathlineto{\pgfqpoint{1.106808in}{4.237359in}}%
\pgfpathlineto{\pgfqpoint{1.107882in}{4.239172in}}%
\pgfpathlineto{\pgfqpoint{1.108956in}{4.235454in}}%
\pgfpathlineto{\pgfqpoint{1.114325in}{4.240335in}}%
\pgfpathlineto{\pgfqpoint{1.115399in}{4.238339in}}%
\pgfpathlineto{\pgfqpoint{1.116473in}{4.237597in}}%
\pgfpathlineto{\pgfqpoint{1.117546in}{4.238770in}}%
\pgfpathlineto{\pgfqpoint{1.120768in}{4.238706in}}%
\pgfpathlineto{\pgfqpoint{1.121842in}{4.237927in}}%
\pgfpathlineto{\pgfqpoint{1.122916in}{4.238897in}}%
\pgfpathlineto{\pgfqpoint{1.125063in}{4.236068in}}%
\pgfpathlineto{\pgfqpoint{1.128285in}{4.235303in}}%
\pgfpathlineto{\pgfqpoint{1.129359in}{4.233728in}}%
\pgfpathlineto{\pgfqpoint{1.130432in}{4.233624in}}%
\pgfpathlineto{\pgfqpoint{1.131506in}{4.235015in}}%
\pgfpathlineto{\pgfqpoint{1.132580in}{4.234246in}}%
\pgfpathlineto{\pgfqpoint{1.135802in}{4.234612in}}%
\pgfpathlineto{\pgfqpoint{1.136875in}{4.233964in}}%
\pgfpathlineto{\pgfqpoint{1.137949in}{4.234347in}}%
\pgfpathlineto{\pgfqpoint{1.140097in}{4.232418in}}%
\pgfpathlineto{\pgfqpoint{1.143319in}{4.232303in}}%
\pgfpathlineto{\pgfqpoint{1.146540in}{4.233664in}}%
\pgfpathlineto{\pgfqpoint{1.147614in}{4.233232in}}%
\pgfpathlineto{\pgfqpoint{1.151909in}{4.233863in}}%
\pgfpathlineto{\pgfqpoint{1.155131in}{4.232129in}}%
\pgfpathlineto{\pgfqpoint{1.160500in}{4.231624in}}%
\pgfpathlineto{\pgfqpoint{1.161574in}{4.230613in}}%
\pgfpathlineto{\pgfqpoint{1.162648in}{4.231042in}}%
\pgfpathlineto{\pgfqpoint{1.166943in}{4.231460in}}%
\pgfpathlineto{\pgfqpoint{1.168017in}{4.232699in}}%
\pgfpathlineto{\pgfqpoint{1.169091in}{4.230559in}}%
\pgfpathlineto{\pgfqpoint{1.173386in}{4.230276in}}%
\pgfpathlineto{\pgfqpoint{1.174460in}{4.230035in}}%
\pgfpathlineto{\pgfqpoint{1.175534in}{4.228391in}}%
\pgfpathlineto{\pgfqpoint{1.176608in}{4.229907in}}%
\pgfpathlineto{\pgfqpoint{1.177681in}{4.230247in}}%
\pgfpathlineto{\pgfqpoint{1.180903in}{4.230162in}}%
\pgfpathlineto{\pgfqpoint{1.181977in}{4.228211in}}%
\pgfpathlineto{\pgfqpoint{1.183051in}{4.228553in}}%
\pgfpathlineto{\pgfqpoint{1.185198in}{4.227857in}}%
\pgfpathlineto{\pgfqpoint{1.189494in}{4.228294in}}%
\pgfpathlineto{\pgfqpoint{1.190567in}{4.229911in}}%
\pgfpathlineto{\pgfqpoint{1.192715in}{4.229316in}}%
\pgfpathlineto{\pgfqpoint{1.197010in}{4.226265in}}%
\pgfpathlineto{\pgfqpoint{1.198084in}{4.224987in}}%
\pgfpathlineto{\pgfqpoint{1.200232in}{4.226008in}}%
\pgfpathlineto{\pgfqpoint{1.204527in}{4.224498in}}%
\pgfpathlineto{\pgfqpoint{1.205601in}{4.222987in}}%
\pgfpathlineto{\pgfqpoint{1.206675in}{4.222681in}}%
\pgfpathlineto{\pgfqpoint{1.207749in}{4.224367in}}%
\pgfpathlineto{\pgfqpoint{1.212044in}{4.225940in}}%
\pgfpathlineto{\pgfqpoint{1.213118in}{4.225274in}}%
\pgfpathlineto{\pgfqpoint{1.215266in}{4.226267in}}%
\pgfpathlineto{\pgfqpoint{1.219561in}{4.225583in}}%
\pgfpathlineto{\pgfqpoint{1.220635in}{4.226056in}}%
\pgfpathlineto{\pgfqpoint{1.221709in}{4.225327in}}%
\pgfpathlineto{\pgfqpoint{1.222783in}{4.226295in}}%
\pgfpathlineto{\pgfqpoint{1.226004in}{4.227104in}}%
\pgfpathlineto{\pgfqpoint{1.227078in}{4.225346in}}%
\pgfpathlineto{\pgfqpoint{1.228152in}{4.225816in}}%
\pgfpathlineto{\pgfqpoint{1.230299in}{4.222241in}}%
\pgfpathlineto{\pgfqpoint{1.233521in}{4.221503in}}%
\pgfpathlineto{\pgfqpoint{1.235669in}{4.224427in}}%
\pgfpathlineto{\pgfqpoint{1.236742in}{4.222568in}}%
\pgfpathlineto{\pgfqpoint{1.242112in}{4.221308in}}%
\pgfpathlineto{\pgfqpoint{1.244259in}{4.222247in}}%
\pgfpathlineto{\pgfqpoint{1.245333in}{4.221264in}}%
\pgfpathlineto{\pgfqpoint{1.250702in}{4.220868in}}%
\pgfpathlineto{\pgfqpoint{1.251776in}{4.221977in}}%
\pgfpathlineto{\pgfqpoint{1.252850in}{4.221928in}}%
\pgfpathlineto{\pgfqpoint{1.257145in}{4.222573in}}%
\pgfpathlineto{\pgfqpoint{1.260367in}{4.226034in}}%
\pgfpathlineto{\pgfqpoint{1.265736in}{4.226268in}}%
\pgfpathlineto{\pgfqpoint{1.266810in}{4.225001in}}%
\pgfpathlineto{\pgfqpoint{1.271105in}{4.225230in}}%
\pgfpathlineto{\pgfqpoint{1.273253in}{4.225073in}}%
\pgfpathlineto{\pgfqpoint{1.274327in}{4.226293in}}%
\pgfpathlineto{\pgfqpoint{1.275401in}{4.225592in}}%
\pgfpathlineto{\pgfqpoint{1.278622in}{4.225377in}}%
\pgfpathlineto{\pgfqpoint{1.280770in}{4.223071in}}%
\pgfpathlineto{\pgfqpoint{1.282918in}{4.223060in}}%
\pgfpathlineto{\pgfqpoint{1.287213in}{4.216207in}}%
\pgfpathlineto{\pgfqpoint{1.289361in}{4.216437in}}%
\pgfpathlineto{\pgfqpoint{1.290434in}{4.214675in}}%
\pgfpathlineto{\pgfqpoint{1.293656in}{4.214709in}}%
\pgfpathlineto{\pgfqpoint{1.295804in}{4.212972in}}%
\pgfpathlineto{\pgfqpoint{1.297951in}{4.212647in}}%
\pgfpathlineto{\pgfqpoint{1.301173in}{4.213824in}}%
\pgfpathlineto{\pgfqpoint{1.303320in}{4.206211in}}%
\pgfpathlineto{\pgfqpoint{1.305468in}{4.205433in}}%
\pgfpathlineto{\pgfqpoint{1.309763in}{4.205520in}}%
\pgfpathlineto{\pgfqpoint{1.312985in}{4.203560in}}%
\pgfpathlineto{\pgfqpoint{1.316207in}{4.203504in}}%
\pgfpathlineto{\pgfqpoint{1.320502in}{4.206172in}}%
\pgfpathlineto{\pgfqpoint{1.324797in}{4.206880in}}%
\pgfpathlineto{\pgfqpoint{1.325871in}{4.206017in}}%
\pgfpathlineto{\pgfqpoint{1.328019in}{4.205644in}}%
\pgfpathlineto{\pgfqpoint{1.331240in}{4.204379in}}%
\pgfpathlineto{\pgfqpoint{1.332314in}{4.206674in}}%
\pgfpathlineto{\pgfqpoint{1.333388in}{4.205801in}}%
\pgfpathlineto{\pgfqpoint{1.334462in}{4.206033in}}%
\pgfpathlineto{\pgfqpoint{1.335536in}{4.205325in}}%
\pgfpathlineto{\pgfqpoint{1.340905in}{4.204627in}}%
\pgfpathlineto{\pgfqpoint{1.341979in}{4.203153in}}%
\pgfpathlineto{\pgfqpoint{1.343052in}{4.199410in}}%
\pgfpathlineto{\pgfqpoint{1.346274in}{4.198353in}}%
\pgfpathlineto{\pgfqpoint{1.347348in}{4.199061in}}%
\pgfpathlineto{\pgfqpoint{1.348422in}{4.198692in}}%
\pgfpathlineto{\pgfqpoint{1.349496in}{4.200512in}}%
\pgfpathlineto{\pgfqpoint{1.350569in}{4.200632in}}%
\pgfpathlineto{\pgfqpoint{1.353791in}{4.204308in}}%
\pgfpathlineto{\pgfqpoint{1.354865in}{4.206896in}}%
\pgfpathlineto{\pgfqpoint{1.355939in}{4.206310in}}%
\pgfpathlineto{\pgfqpoint{1.358086in}{4.203652in}}%
\pgfpathlineto{\pgfqpoint{1.361308in}{4.204196in}}%
\pgfpathlineto{\pgfqpoint{1.362382in}{4.205314in}}%
\pgfpathlineto{\pgfqpoint{1.364529in}{4.205303in}}%
\pgfpathlineto{\pgfqpoint{1.365603in}{4.207051in}}%
\pgfpathlineto{\pgfqpoint{1.369898in}{4.207335in}}%
\pgfpathlineto{\pgfqpoint{1.370972in}{4.206243in}}%
\pgfpathlineto{\pgfqpoint{1.372046in}{4.207121in}}%
\pgfpathlineto{\pgfqpoint{1.373120in}{4.207111in}}%
\pgfpathlineto{\pgfqpoint{1.376341in}{4.207672in}}%
\pgfpathlineto{\pgfqpoint{1.377415in}{4.207029in}}%
\pgfpathlineto{\pgfqpoint{1.378489in}{4.207090in}}%
\pgfpathlineto{\pgfqpoint{1.379563in}{4.208384in}}%
\pgfpathlineto{\pgfqpoint{1.380637in}{4.208040in}}%
\pgfpathlineto{\pgfqpoint{1.384932in}{4.210649in}}%
\pgfpathlineto{\pgfqpoint{1.386006in}{4.209584in}}%
\pgfpathlineto{\pgfqpoint{1.393523in}{4.208001in}}%
\pgfpathlineto{\pgfqpoint{1.395671in}{4.205581in}}%
\pgfpathlineto{\pgfqpoint{1.398892in}{4.207735in}}%
\pgfpathlineto{\pgfqpoint{1.399966in}{4.207317in}}%
\pgfpathlineto{\pgfqpoint{1.402114in}{4.209723in}}%
\pgfpathlineto{\pgfqpoint{1.403187in}{4.209101in}}%
\pgfpathlineto{\pgfqpoint{1.406409in}{4.209727in}}%
\pgfpathlineto{\pgfqpoint{1.408557in}{4.212072in}}%
\pgfpathlineto{\pgfqpoint{1.409630in}{4.211840in}}%
\pgfpathlineto{\pgfqpoint{1.410704in}{4.213082in}}%
\pgfpathlineto{\pgfqpoint{1.416073in}{4.211441in}}%
\pgfpathlineto{\pgfqpoint{1.418221in}{4.211810in}}%
\pgfpathlineto{\pgfqpoint{1.421443in}{4.210503in}}%
\pgfpathlineto{\pgfqpoint{1.422517in}{4.210904in}}%
\pgfpathlineto{\pgfqpoint{1.423590in}{4.210020in}}%
\pgfpathlineto{\pgfqpoint{1.424664in}{4.211662in}}%
\pgfpathlineto{\pgfqpoint{1.425738in}{4.210234in}}%
\pgfpathlineto{\pgfqpoint{1.428960in}{4.210922in}}%
\pgfpathlineto{\pgfqpoint{1.430033in}{4.210068in}}%
\pgfpathlineto{\pgfqpoint{1.431107in}{4.207684in}}%
\pgfpathlineto{\pgfqpoint{1.432181in}{4.206865in}}%
\pgfpathlineto{\pgfqpoint{1.437550in}{4.205894in}}%
\pgfpathlineto{\pgfqpoint{1.438624in}{4.206641in}}%
\pgfpathlineto{\pgfqpoint{1.439698in}{4.205516in}}%
\pgfpathlineto{\pgfqpoint{1.440772in}{4.205438in}}%
\pgfpathlineto{\pgfqpoint{1.445067in}{4.205749in}}%
\pgfpathlineto{\pgfqpoint{1.447215in}{4.203049in}}%
\pgfpathlineto{\pgfqpoint{1.448289in}{4.203601in}}%
\pgfpathlineto{\pgfqpoint{1.451510in}{4.203992in}}%
\pgfpathlineto{\pgfqpoint{1.454732in}{4.206824in}}%
\pgfpathlineto{\pgfqpoint{1.455806in}{4.207242in}}%
\pgfpathlineto{\pgfqpoint{1.459027in}{4.206015in}}%
\pgfpathlineto{\pgfqpoint{1.460101in}{4.206332in}}%
\pgfpathlineto{\pgfqpoint{1.462249in}{4.205151in}}%
\pgfpathlineto{\pgfqpoint{1.463322in}{4.205273in}}%
\pgfpathlineto{\pgfqpoint{1.466544in}{4.204833in}}%
\pgfpathlineto{\pgfqpoint{1.468692in}{4.207058in}}%
\pgfpathlineto{\pgfqpoint{1.470839in}{4.208029in}}%
\pgfpathlineto{\pgfqpoint{1.476208in}{4.207790in}}%
\pgfpathlineto{\pgfqpoint{1.477282in}{4.208588in}}%
\pgfpathlineto{\pgfqpoint{1.478356in}{4.208455in}}%
\pgfpathlineto{\pgfqpoint{1.482651in}{4.209120in}}%
\pgfpathlineto{\pgfqpoint{1.483725in}{4.209643in}}%
\pgfpathlineto{\pgfqpoint{1.484799in}{4.210851in}}%
\pgfpathlineto{\pgfqpoint{1.485873in}{4.210074in}}%
\pgfpathlineto{\pgfqpoint{1.491242in}{4.209335in}}%
\pgfpathlineto{\pgfqpoint{1.492316in}{4.208233in}}%
\pgfpathlineto{\pgfqpoint{1.493390in}{4.208478in}}%
\pgfpathlineto{\pgfqpoint{1.499833in}{4.208158in}}%
\pgfpathlineto{\pgfqpoint{1.500907in}{4.207618in}}%
\pgfpathlineto{\pgfqpoint{1.504128in}{4.208477in}}%
\pgfpathlineto{\pgfqpoint{1.505202in}{4.207661in}}%
\pgfpathlineto{\pgfqpoint{1.507350in}{4.208055in}}%
\pgfpathlineto{\pgfqpoint{1.511645in}{4.207652in}}%
\pgfpathlineto{\pgfqpoint{1.512719in}{4.208729in}}%
\pgfpathlineto{\pgfqpoint{1.514867in}{4.208037in}}%
\pgfpathlineto{\pgfqpoint{1.521310in}{4.208249in}}%
\pgfpathlineto{\pgfqpoint{1.522384in}{4.207606in}}%
\pgfpathlineto{\pgfqpoint{1.523457in}{4.207942in}}%
\pgfpathlineto{\pgfqpoint{1.527753in}{4.207327in}}%
\pgfpathlineto{\pgfqpoint{1.528827in}{4.208212in}}%
\pgfpathlineto{\pgfqpoint{1.530974in}{4.211404in}}%
\pgfpathlineto{\pgfqpoint{1.535270in}{4.212528in}}%
\pgfpathlineto{\pgfqpoint{1.538491in}{4.213281in}}%
\pgfpathlineto{\pgfqpoint{1.541713in}{4.213630in}}%
\pgfpathlineto{\pgfqpoint{1.542786in}{4.214909in}}%
\pgfpathlineto{\pgfqpoint{1.543860in}{4.214672in}}%
\pgfpathlineto{\pgfqpoint{1.544934in}{4.212208in}}%
\pgfpathlineto{\pgfqpoint{1.546008in}{4.211535in}}%
\pgfpathlineto{\pgfqpoint{1.549229in}{4.212089in}}%
\pgfpathlineto{\pgfqpoint{1.550303in}{4.212920in}}%
\pgfpathlineto{\pgfqpoint{1.552451in}{4.212349in}}%
\pgfpathlineto{\pgfqpoint{1.553525in}{4.213853in}}%
\pgfpathlineto{\pgfqpoint{1.557820in}{4.214356in}}%
\pgfpathlineto{\pgfqpoint{1.559968in}{4.213073in}}%
\pgfpathlineto{\pgfqpoint{1.561042in}{4.213438in}}%
\pgfpathlineto{\pgfqpoint{1.564263in}{4.215343in}}%
\pgfpathlineto{\pgfqpoint{1.565337in}{4.215234in}}%
\pgfpathlineto{\pgfqpoint{1.566411in}{4.216283in}}%
\pgfpathlineto{\pgfqpoint{1.572854in}{4.216640in}}%
\pgfpathlineto{\pgfqpoint{1.575002in}{4.216094in}}%
\pgfpathlineto{\pgfqpoint{1.582518in}{4.216470in}}%
\pgfpathlineto{\pgfqpoint{1.583592in}{4.216990in}}%
\pgfpathlineto{\pgfqpoint{1.586814in}{4.216391in}}%
\pgfpathlineto{\pgfqpoint{1.590035in}{4.217155in}}%
\pgfpathlineto{\pgfqpoint{1.591109in}{4.216795in}}%
\pgfpathlineto{\pgfqpoint{1.595405in}{4.216935in}}%
\pgfpathlineto{\pgfqpoint{1.596478in}{4.216488in}}%
\pgfpathlineto{\pgfqpoint{1.597552in}{4.215215in}}%
\pgfpathlineto{\pgfqpoint{1.598626in}{4.215340in}}%
\pgfpathlineto{\pgfqpoint{1.601848in}{4.214507in}}%
\pgfpathlineto{\pgfqpoint{1.602921in}{4.213374in}}%
\pgfpathlineto{\pgfqpoint{1.603995in}{4.213800in}}%
\pgfpathlineto{\pgfqpoint{1.605069in}{4.215835in}}%
\pgfpathlineto{\pgfqpoint{1.610438in}{4.215134in}}%
\pgfpathlineto{\pgfqpoint{1.611512in}{4.212940in}}%
\pgfpathlineto{\pgfqpoint{1.612586in}{4.212490in}}%
\pgfpathlineto{\pgfqpoint{1.613660in}{4.213802in}}%
\pgfpathlineto{\pgfqpoint{1.616881in}{4.213651in}}%
\pgfpathlineto{\pgfqpoint{1.620103in}{4.221893in}}%
\pgfpathlineto{\pgfqpoint{1.621177in}{4.217350in}}%
\pgfpathlineto{\pgfqpoint{1.624398in}{4.216466in}}%
\pgfpathlineto{\pgfqpoint{1.625472in}{4.217674in}}%
\pgfpathlineto{\pgfqpoint{1.626546in}{4.217631in}}%
\pgfpathlineto{\pgfqpoint{1.627620in}{4.218418in}}%
\pgfpathlineto{\pgfqpoint{1.628694in}{4.216197in}}%
\pgfpathlineto{\pgfqpoint{1.631915in}{4.216020in}}%
\pgfpathlineto{\pgfqpoint{1.632989in}{4.212987in}}%
\pgfpathlineto{\pgfqpoint{1.634063in}{4.214709in}}%
\pgfpathlineto{\pgfqpoint{1.635137in}{4.220154in}}%
\pgfpathlineto{\pgfqpoint{1.636210in}{4.216989in}}%
\pgfpathlineto{\pgfqpoint{1.639432in}{4.216165in}}%
\pgfpathlineto{\pgfqpoint{1.640506in}{4.214912in}}%
\pgfpathlineto{\pgfqpoint{1.641580in}{4.214949in}}%
\pgfpathlineto{\pgfqpoint{1.642653in}{4.215840in}}%
\pgfpathlineto{\pgfqpoint{1.643727in}{4.214461in}}%
\pgfpathlineto{\pgfqpoint{1.646949in}{4.214847in}}%
\pgfpathlineto{\pgfqpoint{1.648023in}{4.216269in}}%
\pgfpathlineto{\pgfqpoint{1.651244in}{4.215067in}}%
\pgfpathlineto{\pgfqpoint{1.654466in}{4.214179in}}%
\pgfpathlineto{\pgfqpoint{1.656613in}{4.215066in}}%
\pgfpathlineto{\pgfqpoint{1.658761in}{4.214143in}}%
\pgfpathlineto{\pgfqpoint{1.661983in}{4.214267in}}%
\pgfpathlineto{\pgfqpoint{1.664130in}{4.215768in}}%
\pgfpathlineto{\pgfqpoint{1.665204in}{4.216188in}}%
\pgfpathlineto{\pgfqpoint{1.666278in}{4.215562in}}%
\pgfpathlineto{\pgfqpoint{1.669499in}{4.215783in}}%
\pgfpathlineto{\pgfqpoint{1.670573in}{4.214364in}}%
\pgfpathlineto{\pgfqpoint{1.671647in}{4.214559in}}%
\pgfpathlineto{\pgfqpoint{1.673795in}{4.213166in}}%
\pgfpathlineto{\pgfqpoint{1.677016in}{4.214641in}}%
\pgfpathlineto{\pgfqpoint{1.678090in}{4.216552in}}%
\pgfpathlineto{\pgfqpoint{1.679164in}{4.215254in}}%
\pgfpathlineto{\pgfqpoint{1.681312in}{4.215263in}}%
\pgfpathlineto{\pgfqpoint{1.685607in}{4.214157in}}%
\pgfpathlineto{\pgfqpoint{1.686681in}{4.212172in}}%
\pgfpathlineto{\pgfqpoint{1.687755in}{4.212868in}}%
\pgfpathlineto{\pgfqpoint{1.688828in}{4.214427in}}%
\pgfpathlineto{\pgfqpoint{1.692050in}{4.214212in}}%
\pgfpathlineto{\pgfqpoint{1.693124in}{4.215945in}}%
\pgfpathlineto{\pgfqpoint{1.694198in}{4.214722in}}%
\pgfpathlineto{\pgfqpoint{1.695272in}{4.211095in}}%
\pgfpathlineto{\pgfqpoint{1.696345in}{4.210963in}}%
\pgfpathlineto{\pgfqpoint{1.699567in}{4.209661in}}%
\pgfpathlineto{\pgfqpoint{1.701715in}{4.210000in}}%
\pgfpathlineto{\pgfqpoint{1.703862in}{4.209506in}}%
\pgfpathlineto{\pgfqpoint{1.708158in}{4.210438in}}%
\pgfpathlineto{\pgfqpoint{1.709231in}{4.209340in}}%
\pgfpathlineto{\pgfqpoint{1.711379in}{4.209152in}}%
\pgfpathlineto{\pgfqpoint{1.715674in}{4.205246in}}%
\pgfpathlineto{\pgfqpoint{1.716748in}{4.206073in}}%
\pgfpathlineto{\pgfqpoint{1.717822in}{4.203305in}}%
\pgfpathlineto{\pgfqpoint{1.718896in}{4.204687in}}%
\pgfpathlineto{\pgfqpoint{1.725339in}{4.203303in}}%
\pgfpathlineto{\pgfqpoint{1.726413in}{4.204951in}}%
\pgfpathlineto{\pgfqpoint{1.731782in}{4.205415in}}%
\pgfpathlineto{\pgfqpoint{1.732856in}{4.203054in}}%
\pgfpathlineto{\pgfqpoint{1.733930in}{4.204321in}}%
\pgfpathlineto{\pgfqpoint{1.739299in}{4.205116in}}%
\pgfpathlineto{\pgfqpoint{1.740373in}{4.203588in}}%
\pgfpathlineto{\pgfqpoint{1.741447in}{4.206186in}}%
\pgfpathlineto{\pgfqpoint{1.744668in}{4.207695in}}%
\pgfpathlineto{\pgfqpoint{1.745742in}{4.206613in}}%
\pgfpathlineto{\pgfqpoint{1.748963in}{4.208972in}}%
\pgfpathlineto{\pgfqpoint{1.752185in}{4.208099in}}%
\pgfpathlineto{\pgfqpoint{1.755406in}{4.210050in}}%
\pgfpathlineto{\pgfqpoint{1.756480in}{4.210026in}}%
\pgfpathlineto{\pgfqpoint{1.763997in}{4.211344in}}%
\pgfpathlineto{\pgfqpoint{1.769366in}{4.210794in}}%
\pgfpathlineto{\pgfqpoint{1.774736in}{4.213317in}}%
\pgfpathlineto{\pgfqpoint{1.775809in}{4.214922in}}%
\pgfpathlineto{\pgfqpoint{1.776883in}{4.214107in}}%
\pgfpathlineto{\pgfqpoint{1.777957in}{4.214406in}}%
\pgfpathlineto{\pgfqpoint{1.779031in}{4.216778in}}%
\pgfpathlineto{\pgfqpoint{1.782252in}{4.218326in}}%
\pgfpathlineto{\pgfqpoint{1.783326in}{4.221482in}}%
\pgfpathlineto{\pgfqpoint{1.784400in}{4.221055in}}%
\pgfpathlineto{\pgfqpoint{1.786548in}{4.224653in}}%
\pgfpathlineto{\pgfqpoint{1.790843in}{4.229125in}}%
\pgfpathlineto{\pgfqpoint{1.791917in}{4.231015in}}%
\pgfpathlineto{\pgfqpoint{1.794065in}{4.232400in}}%
\pgfpathlineto{\pgfqpoint{1.798360in}{4.232871in}}%
\pgfpathlineto{\pgfqpoint{1.799434in}{4.235964in}}%
\pgfpathlineto{\pgfqpoint{1.800508in}{4.235863in}}%
\pgfpathlineto{\pgfqpoint{1.801582in}{4.236678in}}%
\pgfpathlineto{\pgfqpoint{1.804803in}{4.234404in}}%
\pgfpathlineto{\pgfqpoint{1.805877in}{4.235350in}}%
\pgfpathlineto{\pgfqpoint{1.806951in}{4.233215in}}%
\pgfpathlineto{\pgfqpoint{1.808025in}{4.233462in}}%
\pgfpathlineto{\pgfqpoint{1.814468in}{4.230440in}}%
\pgfpathlineto{\pgfqpoint{1.815541in}{4.230876in}}%
\pgfpathlineto{\pgfqpoint{1.816615in}{4.230639in}}%
\pgfpathlineto{\pgfqpoint{1.823058in}{4.232646in}}%
\pgfpathlineto{\pgfqpoint{1.824132in}{4.229067in}}%
\pgfpathlineto{\pgfqpoint{1.829501in}{4.232495in}}%
\pgfpathlineto{\pgfqpoint{1.830575in}{4.236805in}}%
\pgfpathlineto{\pgfqpoint{1.831649in}{4.236186in}}%
\pgfpathlineto{\pgfqpoint{1.834871in}{4.235268in}}%
\pgfpathlineto{\pgfqpoint{1.835944in}{4.235588in}}%
\pgfpathlineto{\pgfqpoint{1.837018in}{4.236836in}}%
\pgfpathlineto{\pgfqpoint{1.838092in}{4.236301in}}%
\pgfpathlineto{\pgfqpoint{1.839166in}{4.237497in}}%
\pgfpathlineto{\pgfqpoint{1.842387in}{4.236248in}}%
\pgfpathlineto{\pgfqpoint{1.843461in}{4.237270in}}%
\pgfpathlineto{\pgfqpoint{1.844535in}{4.237018in}}%
\pgfpathlineto{\pgfqpoint{1.845609in}{4.237918in}}%
\pgfpathlineto{\pgfqpoint{1.846683in}{4.236066in}}%
\pgfpathlineto{\pgfqpoint{1.850978in}{4.236853in}}%
\pgfpathlineto{\pgfqpoint{1.854200in}{4.233940in}}%
\pgfpathlineto{\pgfqpoint{1.858495in}{4.233782in}}%
\pgfpathlineto{\pgfqpoint{1.860643in}{4.234089in}}%
\pgfpathlineto{\pgfqpoint{1.861716in}{4.235041in}}%
\pgfpathlineto{\pgfqpoint{1.866012in}{4.233795in}}%
\pgfpathlineto{\pgfqpoint{1.867086in}{4.235071in}}%
\pgfpathlineto{\pgfqpoint{1.868160in}{4.235201in}}%
\pgfpathlineto{\pgfqpoint{1.869233in}{4.233591in}}%
\pgfpathlineto{\pgfqpoint{1.873529in}{4.233591in}}%
\pgfpathlineto{\pgfqpoint{1.875676in}{4.236197in}}%
\pgfpathlineto{\pgfqpoint{1.876750in}{4.235034in}}%
\pgfpathlineto{\pgfqpoint{1.881046in}{4.234942in}}%
\pgfpathlineto{\pgfqpoint{1.882119in}{4.233048in}}%
\pgfpathlineto{\pgfqpoint{1.883193in}{4.232316in}}%
\pgfpathlineto{\pgfqpoint{1.884267in}{4.233923in}}%
\pgfpathlineto{\pgfqpoint{1.887489in}{4.231954in}}%
\pgfpathlineto{\pgfqpoint{1.888562in}{4.232512in}}%
\pgfpathlineto{\pgfqpoint{1.889636in}{4.232096in}}%
\pgfpathlineto{\pgfqpoint{1.890710in}{4.229684in}}%
\pgfpathlineto{\pgfqpoint{1.891784in}{4.230270in}}%
\pgfpathlineto{\pgfqpoint{1.896079in}{4.230771in}}%
\pgfpathlineto{\pgfqpoint{1.897153in}{4.232784in}}%
\pgfpathlineto{\pgfqpoint{1.898227in}{4.231319in}}%
\pgfpathlineto{\pgfqpoint{1.902522in}{4.235208in}}%
\pgfpathlineto{\pgfqpoint{1.903596in}{4.235534in}}%
\pgfpathlineto{\pgfqpoint{1.904670in}{4.234272in}}%
\pgfpathlineto{\pgfqpoint{1.905744in}{4.234520in}}%
\pgfpathlineto{\pgfqpoint{1.911113in}{4.234572in}}%
\pgfpathlineto{\pgfqpoint{1.912187in}{4.237092in}}%
\pgfpathlineto{\pgfqpoint{1.913261in}{4.238090in}}%
\pgfpathlineto{\pgfqpoint{1.914335in}{4.236941in}}%
\pgfpathlineto{\pgfqpoint{1.918630in}{4.235224in}}%
\pgfpathlineto{\pgfqpoint{1.921851in}{4.237469in}}%
\pgfpathlineto{\pgfqpoint{1.925073in}{4.237868in}}%
\pgfpathlineto{\pgfqpoint{1.926147in}{4.239138in}}%
\pgfpathlineto{\pgfqpoint{1.927221in}{4.238807in}}%
\pgfpathlineto{\pgfqpoint{1.928294in}{4.233170in}}%
\pgfpathlineto{\pgfqpoint{1.932590in}{4.232423in}}%
\pgfpathlineto{\pgfqpoint{1.933664in}{4.234707in}}%
\pgfpathlineto{\pgfqpoint{1.935811in}{4.234270in}}%
\pgfpathlineto{\pgfqpoint{1.936885in}{4.234482in}}%
\pgfpathlineto{\pgfqpoint{1.940107in}{4.233322in}}%
\pgfpathlineto{\pgfqpoint{1.941181in}{4.233589in}}%
\pgfpathlineto{\pgfqpoint{1.942254in}{4.233014in}}%
\pgfpathlineto{\pgfqpoint{1.943328in}{4.234661in}}%
\pgfpathlineto{\pgfqpoint{1.944402in}{4.234304in}}%
\pgfpathlineto{\pgfqpoint{1.947624in}{4.236573in}}%
\pgfpathlineto{\pgfqpoint{1.948697in}{4.239242in}}%
\pgfpathlineto{\pgfqpoint{1.949771in}{4.239286in}}%
\pgfpathlineto{\pgfqpoint{1.950845in}{4.238166in}}%
\pgfpathlineto{\pgfqpoint{1.951919in}{4.239051in}}%
\pgfpathlineto{\pgfqpoint{1.955140in}{4.238099in}}%
\pgfpathlineto{\pgfqpoint{1.956214in}{4.239017in}}%
\pgfpathlineto{\pgfqpoint{1.957288in}{4.237594in}}%
\pgfpathlineto{\pgfqpoint{1.959436in}{4.233785in}}%
\pgfpathlineto{\pgfqpoint{1.963731in}{4.229315in}}%
\pgfpathlineto{\pgfqpoint{1.964805in}{4.234424in}}%
\pgfpathlineto{\pgfqpoint{1.965879in}{4.233236in}}%
\pgfpathlineto{\pgfqpoint{1.966953in}{4.232928in}}%
\pgfpathlineto{\pgfqpoint{1.970174in}{4.234968in}}%
\pgfpathlineto{\pgfqpoint{1.971248in}{4.231276in}}%
\pgfpathlineto{\pgfqpoint{1.972322in}{4.234054in}}%
\pgfpathlineto{\pgfqpoint{1.973396in}{4.233546in}}%
\pgfpathlineto{\pgfqpoint{1.974470in}{4.235463in}}%
\pgfpathlineto{\pgfqpoint{1.978765in}{4.239183in}}%
\pgfpathlineto{\pgfqpoint{1.979839in}{4.241843in}}%
\pgfpathlineto{\pgfqpoint{1.980913in}{4.241759in}}%
\pgfpathlineto{\pgfqpoint{1.981986in}{4.242273in}}%
\pgfpathlineto{\pgfqpoint{1.985208in}{4.242762in}}%
\pgfpathlineto{\pgfqpoint{1.986282in}{4.245777in}}%
\pgfpathlineto{\pgfqpoint{1.987356in}{4.245129in}}%
\pgfpathlineto{\pgfqpoint{1.988429in}{4.246463in}}%
\pgfpathlineto{\pgfqpoint{1.989503in}{4.242827in}}%
\pgfpathlineto{\pgfqpoint{1.992725in}{4.243287in}}%
\pgfpathlineto{\pgfqpoint{1.993799in}{4.245481in}}%
\pgfpathlineto{\pgfqpoint{1.995946in}{4.244982in}}%
\pgfpathlineto{\pgfqpoint{1.997020in}{4.247203in}}%
\pgfpathlineto{\pgfqpoint{2.000242in}{4.248525in}}%
\pgfpathlineto{\pgfqpoint{2.001315in}{4.252141in}}%
\pgfpathlineto{\pgfqpoint{2.002389in}{4.251887in}}%
\pgfpathlineto{\pgfqpoint{2.003463in}{4.252988in}}%
\pgfpathlineto{\pgfqpoint{2.004537in}{4.255744in}}%
\pgfpathlineto{\pgfqpoint{2.007759in}{4.252141in}}%
\pgfpathlineto{\pgfqpoint{2.008832in}{4.252731in}}%
\pgfpathlineto{\pgfqpoint{2.009906in}{4.254672in}}%
\pgfpathlineto{\pgfqpoint{2.010980in}{4.252520in}}%
\pgfpathlineto{\pgfqpoint{2.012054in}{4.252063in}}%
\pgfpathlineto{\pgfqpoint{2.015275in}{4.251883in}}%
\pgfpathlineto{\pgfqpoint{2.016349in}{4.252911in}}%
\pgfpathlineto{\pgfqpoint{2.017423in}{4.252052in}}%
\pgfpathlineto{\pgfqpoint{2.019571in}{4.253528in}}%
\pgfpathlineto{\pgfqpoint{2.022792in}{4.252773in}}%
\pgfpathlineto{\pgfqpoint{2.023866in}{4.253979in}}%
\pgfpathlineto{\pgfqpoint{2.024940in}{4.253199in}}%
\pgfpathlineto{\pgfqpoint{2.026014in}{4.246502in}}%
\pgfpathlineto{\pgfqpoint{2.031383in}{4.247259in}}%
\pgfpathlineto{\pgfqpoint{2.032457in}{4.245915in}}%
\pgfpathlineto{\pgfqpoint{2.033531in}{4.246373in}}%
\pgfpathlineto{\pgfqpoint{2.034604in}{4.245978in}}%
\pgfpathlineto{\pgfqpoint{2.039974in}{4.249969in}}%
\pgfpathlineto{\pgfqpoint{2.042121in}{4.250158in}}%
\pgfpathlineto{\pgfqpoint{2.047491in}{4.253569in}}%
\pgfpathlineto{\pgfqpoint{2.048564in}{4.256383in}}%
\pgfpathlineto{\pgfqpoint{2.049638in}{4.255699in}}%
\pgfpathlineto{\pgfqpoint{2.053934in}{4.259661in}}%
\pgfpathlineto{\pgfqpoint{2.055007in}{4.261158in}}%
\pgfpathlineto{\pgfqpoint{2.057155in}{4.259478in}}%
\pgfpathlineto{\pgfqpoint{2.060377in}{4.259794in}}%
\pgfpathlineto{\pgfqpoint{2.061450in}{4.258840in}}%
\pgfpathlineto{\pgfqpoint{2.062524in}{4.258711in}}%
\pgfpathlineto{\pgfqpoint{2.064672in}{4.259088in}}%
\pgfpathlineto{\pgfqpoint{2.068967in}{4.260931in}}%
\pgfpathlineto{\pgfqpoint{2.070041in}{4.262238in}}%
\pgfpathlineto{\pgfqpoint{2.071115in}{4.260789in}}%
\pgfpathlineto{\pgfqpoint{2.072189in}{4.265015in}}%
\pgfpathlineto{\pgfqpoint{2.075410in}{4.265526in}}%
\pgfpathlineto{\pgfqpoint{2.077558in}{4.263640in}}%
\pgfpathlineto{\pgfqpoint{2.078632in}{4.265269in}}%
\pgfpathlineto{\pgfqpoint{2.079706in}{4.268245in}}%
\pgfpathlineto{\pgfqpoint{2.082927in}{4.271452in}}%
\pgfpathlineto{\pgfqpoint{2.084001in}{4.282338in}}%
\pgfpathlineto{\pgfqpoint{2.086149in}{4.286099in}}%
\pgfpathlineto{\pgfqpoint{2.087223in}{4.283560in}}%
\pgfpathlineto{\pgfqpoint{2.090444in}{4.284300in}}%
\pgfpathlineto{\pgfqpoint{2.093666in}{4.279498in}}%
\pgfpathlineto{\pgfqpoint{2.097961in}{4.279540in}}%
\pgfpathlineto{\pgfqpoint{2.100109in}{4.282839in}}%
\pgfpathlineto{\pgfqpoint{2.101182in}{4.281231in}}%
\pgfpathlineto{\pgfqpoint{2.105478in}{4.276394in}}%
\pgfpathlineto{\pgfqpoint{2.106552in}{4.277205in}}%
\pgfpathlineto{\pgfqpoint{2.107626in}{4.280965in}}%
\pgfpathlineto{\pgfqpoint{2.108699in}{4.276337in}}%
\pgfpathlineto{\pgfqpoint{2.109773in}{4.275711in}}%
\pgfpathlineto{\pgfqpoint{2.114069in}{4.276191in}}%
\pgfpathlineto{\pgfqpoint{2.117290in}{4.285737in}}%
\pgfpathlineto{\pgfqpoint{2.121585in}{4.285317in}}%
\pgfpathlineto{\pgfqpoint{2.122659in}{4.283297in}}%
\pgfpathlineto{\pgfqpoint{2.123733in}{4.284432in}}%
\pgfpathlineto{\pgfqpoint{2.124807in}{4.281960in}}%
\pgfpathlineto{\pgfqpoint{2.128028in}{4.284642in}}%
\pgfpathlineto{\pgfqpoint{2.129102in}{4.294817in}}%
\pgfpathlineto{\pgfqpoint{2.130176in}{4.293742in}}%
\pgfpathlineto{\pgfqpoint{2.131250in}{4.291275in}}%
\pgfpathlineto{\pgfqpoint{2.132324in}{4.286264in}}%
\pgfpathlineto{\pgfqpoint{2.135545in}{4.289198in}}%
\pgfpathlineto{\pgfqpoint{2.136619in}{4.288048in}}%
\pgfpathlineto{\pgfqpoint{2.137693in}{4.294258in}}%
\pgfpathlineto{\pgfqpoint{2.138767in}{4.293034in}}%
\pgfpathlineto{\pgfqpoint{2.139841in}{4.292990in}}%
\pgfpathlineto{\pgfqpoint{2.143062in}{4.293558in}}%
\pgfpathlineto{\pgfqpoint{2.144136in}{4.292451in}}%
\pgfpathlineto{\pgfqpoint{2.145210in}{4.294007in}}%
\pgfpathlineto{\pgfqpoint{2.146284in}{4.291636in}}%
\pgfpathlineto{\pgfqpoint{2.147358in}{4.296027in}}%
\pgfpathlineto{\pgfqpoint{2.151653in}{4.293934in}}%
\pgfpathlineto{\pgfqpoint{2.152727in}{4.292514in}}%
\pgfpathlineto{\pgfqpoint{2.153801in}{4.292302in}}%
\pgfpathlineto{\pgfqpoint{2.158096in}{4.294355in}}%
\pgfpathlineto{\pgfqpoint{2.160244in}{4.297712in}}%
\pgfpathlineto{\pgfqpoint{2.161317in}{4.295386in}}%
\pgfpathlineto{\pgfqpoint{2.162391in}{4.296318in}}%
\pgfpathlineto{\pgfqpoint{2.165613in}{4.294508in}}%
\pgfpathlineto{\pgfqpoint{2.166687in}{4.298044in}}%
\pgfpathlineto{\pgfqpoint{2.169908in}{4.299143in}}%
\pgfpathlineto{\pgfqpoint{2.173130in}{4.298486in}}%
\pgfpathlineto{\pgfqpoint{2.174203in}{4.298969in}}%
\pgfpathlineto{\pgfqpoint{2.176351in}{4.298569in}}%
\pgfpathlineto{\pgfqpoint{2.177425in}{4.301152in}}%
\pgfpathlineto{\pgfqpoint{2.180647in}{4.301237in}}%
\pgfpathlineto{\pgfqpoint{2.181720in}{4.302008in}}%
\pgfpathlineto{\pgfqpoint{2.183868in}{4.300259in}}%
\pgfpathlineto{\pgfqpoint{2.184942in}{4.298250in}}%
\pgfpathlineto{\pgfqpoint{2.189237in}{4.298940in}}%
\pgfpathlineto{\pgfqpoint{2.190311in}{4.298136in}}%
\pgfpathlineto{\pgfqpoint{2.191385in}{4.298372in}}%
\pgfpathlineto{\pgfqpoint{2.195680in}{4.296071in}}%
\pgfpathlineto{\pgfqpoint{2.197828in}{4.301383in}}%
\pgfpathlineto{\pgfqpoint{2.198902in}{4.301536in}}%
\pgfpathlineto{\pgfqpoint{2.199976in}{4.302675in}}%
\pgfpathlineto{\pgfqpoint{2.203197in}{4.304105in}}%
\pgfpathlineto{\pgfqpoint{2.204271in}{4.303345in}}%
\pgfpathlineto{\pgfqpoint{2.205345in}{4.304625in}}%
\pgfpathlineto{\pgfqpoint{2.206419in}{4.304175in}}%
\pgfpathlineto{\pgfqpoint{2.207492in}{4.304848in}}%
\pgfpathlineto{\pgfqpoint{2.210714in}{4.304791in}}%
\pgfpathlineto{\pgfqpoint{2.211788in}{4.306639in}}%
\pgfpathlineto{\pgfqpoint{2.212862in}{4.305920in}}%
\pgfpathlineto{\pgfqpoint{2.215009in}{4.307268in}}%
\pgfpathlineto{\pgfqpoint{2.225748in}{4.304541in}}%
\pgfpathlineto{\pgfqpoint{2.226822in}{4.301753in}}%
\pgfpathlineto{\pgfqpoint{2.228969in}{4.304722in}}%
\pgfpathlineto{\pgfqpoint{2.230043in}{4.305200in}}%
\pgfpathlineto{\pgfqpoint{2.233265in}{4.303962in}}%
\pgfpathlineto{\pgfqpoint{2.234338in}{4.304435in}}%
\pgfpathlineto{\pgfqpoint{2.235412in}{4.303402in}}%
\pgfpathlineto{\pgfqpoint{2.236486in}{4.304184in}}%
\pgfpathlineto{\pgfqpoint{2.237560in}{4.302759in}}%
\pgfpathlineto{\pgfqpoint{2.240781in}{4.303228in}}%
\pgfpathlineto{\pgfqpoint{2.241855in}{4.305428in}}%
\pgfpathlineto{\pgfqpoint{2.244003in}{4.306538in}}%
\pgfpathlineto{\pgfqpoint{2.245077in}{4.308863in}}%
\pgfpathlineto{\pgfqpoint{2.248298in}{4.310456in}}%
\pgfpathlineto{\pgfqpoint{2.249372in}{4.312717in}}%
\pgfpathlineto{\pgfqpoint{2.250446in}{4.312877in}}%
\pgfpathlineto{\pgfqpoint{2.251520in}{4.310612in}}%
\pgfpathlineto{\pgfqpoint{2.252594in}{4.310467in}}%
\pgfpathlineto{\pgfqpoint{2.255815in}{4.311933in}}%
\pgfpathlineto{\pgfqpoint{2.257963in}{4.306819in}}%
\pgfpathlineto{\pgfqpoint{2.259037in}{4.308191in}}%
\pgfpathlineto{\pgfqpoint{2.260111in}{4.308163in}}%
\pgfpathlineto{\pgfqpoint{2.264406in}{4.307433in}}%
\pgfpathlineto{\pgfqpoint{2.267627in}{4.309191in}}%
\pgfpathlineto{\pgfqpoint{2.272997in}{4.305285in}}%
\pgfpathlineto{\pgfqpoint{2.274070in}{4.306896in}}%
\pgfpathlineto{\pgfqpoint{2.275144in}{4.304901in}}%
\pgfpathlineto{\pgfqpoint{2.278366in}{4.303086in}}%
\pgfpathlineto{\pgfqpoint{2.279440in}{4.303696in}}%
\pgfpathlineto{\pgfqpoint{2.280514in}{4.303460in}}%
\pgfpathlineto{\pgfqpoint{2.281587in}{4.300751in}}%
\pgfpathlineto{\pgfqpoint{2.285883in}{4.304840in}}%
\pgfpathlineto{\pgfqpoint{2.288030in}{4.304840in}}%
\pgfpathlineto{\pgfqpoint{2.289104in}{4.308439in}}%
\pgfpathlineto{\pgfqpoint{2.290178in}{4.314683in}}%
\pgfpathlineto{\pgfqpoint{2.293400in}{4.312127in}}%
\pgfpathlineto{\pgfqpoint{2.294473in}{4.315303in}}%
\pgfpathlineto{\pgfqpoint{2.295547in}{4.312748in}}%
\pgfpathlineto{\pgfqpoint{2.296621in}{4.308076in}}%
\pgfpathlineto{\pgfqpoint{2.297695in}{4.307562in}}%
\pgfpathlineto{\pgfqpoint{2.303064in}{4.306876in}}%
\pgfpathlineto{\pgfqpoint{2.304138in}{4.306526in}}%
\pgfpathlineto{\pgfqpoint{2.305212in}{4.309298in}}%
\pgfpathlineto{\pgfqpoint{2.308433in}{4.308195in}}%
\pgfpathlineto{\pgfqpoint{2.309507in}{4.306796in}}%
\pgfpathlineto{\pgfqpoint{2.310581in}{4.306730in}}%
\pgfpathlineto{\pgfqpoint{2.311655in}{4.308767in}}%
\pgfpathlineto{\pgfqpoint{2.312729in}{4.308163in}}%
\pgfpathlineto{\pgfqpoint{2.318098in}{4.308883in}}%
\pgfpathlineto{\pgfqpoint{2.319172in}{4.310272in}}%
\pgfpathlineto{\pgfqpoint{2.323467in}{4.311494in}}%
\pgfpathlineto{\pgfqpoint{2.324541in}{4.309082in}}%
\pgfpathlineto{\pgfqpoint{2.327762in}{4.310665in}}%
\pgfpathlineto{\pgfqpoint{2.333132in}{4.310880in}}%
\pgfpathlineto{\pgfqpoint{2.334205in}{4.310167in}}%
\pgfpathlineto{\pgfqpoint{2.335279in}{4.311109in}}%
\pgfpathlineto{\pgfqpoint{2.340648in}{4.311391in}}%
\pgfpathlineto{\pgfqpoint{2.341722in}{4.308700in}}%
\pgfpathlineto{\pgfqpoint{2.342796in}{4.309580in}}%
\pgfpathlineto{\pgfqpoint{2.346018in}{4.309937in}}%
\pgfpathlineto{\pgfqpoint{2.349239in}{4.314547in}}%
\pgfpathlineto{\pgfqpoint{2.350313in}{4.315089in}}%
\pgfpathlineto{\pgfqpoint{2.353535in}{4.315766in}}%
\pgfpathlineto{\pgfqpoint{2.355682in}{4.317100in}}%
\pgfpathlineto{\pgfqpoint{2.357830in}{4.317510in}}%
\pgfpathlineto{\pgfqpoint{2.362125in}{4.319318in}}%
\pgfpathlineto{\pgfqpoint{2.363199in}{4.318163in}}%
\pgfpathlineto{\pgfqpoint{2.364273in}{4.319098in}}%
\pgfpathlineto{\pgfqpoint{2.365347in}{4.318039in}}%
\pgfpathlineto{\pgfqpoint{2.369642in}{4.318489in}}%
\pgfpathlineto{\pgfqpoint{2.370716in}{4.317600in}}%
\pgfpathlineto{\pgfqpoint{2.371790in}{4.318038in}}%
\pgfpathlineto{\pgfqpoint{2.372864in}{4.323658in}}%
\pgfpathlineto{\pgfqpoint{2.376085in}{4.326920in}}%
\pgfpathlineto{\pgfqpoint{2.377159in}{4.329621in}}%
\pgfpathlineto{\pgfqpoint{2.378233in}{4.329019in}}%
\pgfpathlineto{\pgfqpoint{2.383602in}{4.334689in}}%
\pgfpathlineto{\pgfqpoint{2.384676in}{4.333462in}}%
\pgfpathlineto{\pgfqpoint{2.385750in}{4.330752in}}%
\pgfpathlineto{\pgfqpoint{2.386824in}{4.330404in}}%
\pgfpathlineto{\pgfqpoint{2.387897in}{4.333682in}}%
\pgfpathlineto{\pgfqpoint{2.391119in}{4.331709in}}%
\pgfpathlineto{\pgfqpoint{2.392193in}{4.332198in}}%
\pgfpathlineto{\pgfqpoint{2.393267in}{4.330432in}}%
\pgfpathlineto{\pgfqpoint{2.394340in}{4.333430in}}%
\pgfpathlineto{\pgfqpoint{2.395414in}{4.334615in}}%
\pgfpathlineto{\pgfqpoint{2.398636in}{4.336264in}}%
\pgfpathlineto{\pgfqpoint{2.399710in}{4.331553in}}%
\pgfpathlineto{\pgfqpoint{2.401857in}{4.333604in}}%
\pgfpathlineto{\pgfqpoint{2.402931in}{4.333160in}}%
\pgfpathlineto{\pgfqpoint{2.406153in}{4.332884in}}%
\pgfpathlineto{\pgfqpoint{2.407226in}{4.330829in}}%
\pgfpathlineto{\pgfqpoint{2.414743in}{4.330661in}}%
\pgfpathlineto{\pgfqpoint{2.417965in}{4.330019in}}%
\pgfpathlineto{\pgfqpoint{2.421186in}{4.332512in}}%
\pgfpathlineto{\pgfqpoint{2.422260in}{4.339625in}}%
\pgfpathlineto{\pgfqpoint{2.423334in}{4.340047in}}%
\pgfpathlineto{\pgfqpoint{2.424408in}{4.341166in}}%
\pgfpathlineto{\pgfqpoint{2.425482in}{4.341182in}}%
\pgfpathlineto{\pgfqpoint{2.428703in}{4.340462in}}%
\pgfpathlineto{\pgfqpoint{2.429777in}{4.338891in}}%
\pgfpathlineto{\pgfqpoint{2.430851in}{4.341117in}}%
\pgfpathlineto{\pgfqpoint{2.431925in}{4.339447in}}%
\pgfpathlineto{\pgfqpoint{2.432999in}{4.339980in}}%
\pgfpathlineto{\pgfqpoint{2.436220in}{4.344863in}}%
\pgfpathlineto{\pgfqpoint{2.437294in}{4.342933in}}%
\pgfpathlineto{\pgfqpoint{2.438368in}{4.343875in}}%
\pgfpathlineto{\pgfqpoint{2.439442in}{4.349655in}}%
\pgfpathlineto{\pgfqpoint{2.440515in}{4.348463in}}%
\pgfpathlineto{\pgfqpoint{2.443737in}{4.351653in}}%
\pgfpathlineto{\pgfqpoint{2.446958in}{4.356525in}}%
\pgfpathlineto{\pgfqpoint{2.448032in}{4.356226in}}%
\pgfpathlineto{\pgfqpoint{2.452328in}{4.359046in}}%
\pgfpathlineto{\pgfqpoint{2.453402in}{4.358508in}}%
\pgfpathlineto{\pgfqpoint{2.455549in}{4.356583in}}%
\pgfpathlineto{\pgfqpoint{2.459845in}{4.358518in}}%
\pgfpathlineto{\pgfqpoint{2.461992in}{4.361522in}}%
\pgfpathlineto{\pgfqpoint{2.463066in}{4.361843in}}%
\pgfpathlineto{\pgfqpoint{2.466288in}{4.360573in}}%
\pgfpathlineto{\pgfqpoint{2.467361in}{4.360861in}}%
\pgfpathlineto{\pgfqpoint{2.468435in}{4.354174in}}%
\pgfpathlineto{\pgfqpoint{2.469509in}{4.354432in}}%
\pgfpathlineto{\pgfqpoint{2.470583in}{4.358339in}}%
\pgfpathlineto{\pgfqpoint{2.473804in}{4.356611in}}%
\pgfpathlineto{\pgfqpoint{2.474878in}{4.357807in}}%
\pgfpathlineto{\pgfqpoint{2.475952in}{4.354478in}}%
\pgfpathlineto{\pgfqpoint{2.477026in}{4.353610in}}%
\pgfpathlineto{\pgfqpoint{2.478100in}{4.355748in}}%
\pgfpathlineto{\pgfqpoint{2.483469in}{4.354287in}}%
\pgfpathlineto{\pgfqpoint{2.485617in}{4.356084in}}%
\pgfpathlineto{\pgfqpoint{2.489912in}{4.356339in}}%
\pgfpathlineto{\pgfqpoint{2.492060in}{4.358086in}}%
\pgfpathlineto{\pgfqpoint{2.493134in}{4.357844in}}%
\pgfpathlineto{\pgfqpoint{2.498503in}{4.359055in}}%
\pgfpathlineto{\pgfqpoint{2.500650in}{4.360761in}}%
\pgfpathlineto{\pgfqpoint{2.503872in}{4.362213in}}%
\pgfpathlineto{\pgfqpoint{2.504946in}{4.361157in}}%
\pgfpathlineto{\pgfqpoint{2.506020in}{4.363169in}}%
\pgfpathlineto{\pgfqpoint{2.507093in}{4.363862in}}%
\pgfpathlineto{\pgfqpoint{2.512463in}{4.363580in}}%
\pgfpathlineto{\pgfqpoint{2.513536in}{4.365469in}}%
\pgfpathlineto{\pgfqpoint{2.514610in}{4.365171in}}%
\pgfpathlineto{\pgfqpoint{2.515684in}{4.365468in}}%
\pgfpathlineto{\pgfqpoint{2.518906in}{4.365502in}}%
\pgfpathlineto{\pgfqpoint{2.519979in}{4.361590in}}%
\pgfpathlineto{\pgfqpoint{2.521053in}{4.362767in}}%
\pgfpathlineto{\pgfqpoint{2.522127in}{4.362635in}}%
\pgfpathlineto{\pgfqpoint{2.526423in}{4.366807in}}%
\pgfpathlineto{\pgfqpoint{2.527496in}{4.365858in}}%
\pgfpathlineto{\pgfqpoint{2.528570in}{4.366417in}}%
\pgfpathlineto{\pgfqpoint{2.530718in}{4.369347in}}%
\pgfpathlineto{\pgfqpoint{2.533939in}{4.369261in}}%
\pgfpathlineto{\pgfqpoint{2.535013in}{4.370323in}}%
\pgfpathlineto{\pgfqpoint{2.538235in}{4.365229in}}%
\pgfpathlineto{\pgfqpoint{2.543604in}{4.359362in}}%
\pgfpathlineto{\pgfqpoint{2.544678in}{4.356831in}}%
\pgfpathlineto{\pgfqpoint{2.545752in}{4.357499in}}%
\pgfpathlineto{\pgfqpoint{2.550047in}{4.358107in}}%
\pgfpathlineto{\pgfqpoint{2.551121in}{4.354208in}}%
\pgfpathlineto{\pgfqpoint{2.552195in}{4.352571in}}%
\pgfpathlineto{\pgfqpoint{2.553268in}{4.352272in}}%
\pgfpathlineto{\pgfqpoint{2.556490in}{4.352974in}}%
\pgfpathlineto{\pgfqpoint{2.557564in}{4.352193in}}%
\pgfpathlineto{\pgfqpoint{2.558638in}{4.357116in}}%
\pgfpathlineto{\pgfqpoint{2.559712in}{4.357071in}}%
\pgfpathlineto{\pgfqpoint{2.560785in}{4.356261in}}%
\pgfpathlineto{\pgfqpoint{2.567228in}{4.354808in}}%
\pgfpathlineto{\pgfqpoint{2.568302in}{4.352993in}}%
\pgfpathlineto{\pgfqpoint{2.571524in}{4.352568in}}%
\pgfpathlineto{\pgfqpoint{2.575819in}{4.359489in}}%
\pgfpathlineto{\pgfqpoint{2.579041in}{4.358577in}}%
\pgfpathlineto{\pgfqpoint{2.580114in}{4.359809in}}%
\pgfpathlineto{\pgfqpoint{2.582262in}{4.359824in}}%
\pgfpathlineto{\pgfqpoint{2.583336in}{4.358940in}}%
\pgfpathlineto{\pgfqpoint{2.586557in}{4.357427in}}%
\pgfpathlineto{\pgfqpoint{2.589779in}{4.359497in}}%
\pgfpathlineto{\pgfqpoint{2.590853in}{4.359422in}}%
\pgfpathlineto{\pgfqpoint{2.601591in}{4.357139in}}%
\pgfpathlineto{\pgfqpoint{2.603739in}{4.358175in}}%
\pgfpathlineto{\pgfqpoint{2.604813in}{4.356698in}}%
\pgfpathlineto{\pgfqpoint{2.610182in}{4.359306in}}%
\pgfpathlineto{\pgfqpoint{2.611256in}{4.358748in}}%
\pgfpathlineto{\pgfqpoint{2.612330in}{4.360635in}}%
\pgfpathlineto{\pgfqpoint{2.613403in}{4.360167in}}%
\pgfpathlineto{\pgfqpoint{2.617699in}{4.355086in}}%
\pgfpathlineto{\pgfqpoint{2.618773in}{4.355260in}}%
\pgfpathlineto{\pgfqpoint{2.619846in}{4.356754in}}%
\pgfpathlineto{\pgfqpoint{2.620920in}{4.357132in}}%
\pgfpathlineto{\pgfqpoint{2.624142in}{4.355749in}}%
\pgfpathlineto{\pgfqpoint{2.625216in}{4.358092in}}%
\pgfpathlineto{\pgfqpoint{2.626290in}{4.356608in}}%
\pgfpathlineto{\pgfqpoint{2.627363in}{4.354075in}}%
\pgfpathlineto{\pgfqpoint{2.628437in}{4.354216in}}%
\pgfpathlineto{\pgfqpoint{2.631659in}{4.353227in}}%
\pgfpathlineto{\pgfqpoint{2.632733in}{4.352308in}}%
\pgfpathlineto{\pgfqpoint{2.633806in}{4.350569in}}%
\pgfpathlineto{\pgfqpoint{2.635954in}{4.350471in}}%
\pgfpathlineto{\pgfqpoint{2.639176in}{4.351870in}}%
\pgfpathlineto{\pgfqpoint{2.640249in}{4.351488in}}%
\pgfpathlineto{\pgfqpoint{2.641323in}{4.354011in}}%
\pgfpathlineto{\pgfqpoint{2.642397in}{4.354557in}}%
\pgfpathlineto{\pgfqpoint{2.643471in}{4.353795in}}%
\pgfpathlineto{\pgfqpoint{2.646692in}{4.350152in}}%
\pgfpathlineto{\pgfqpoint{2.647766in}{4.351181in}}%
\pgfpathlineto{\pgfqpoint{2.648840in}{4.350873in}}%
\pgfpathlineto{\pgfqpoint{2.649914in}{4.353562in}}%
\pgfpathlineto{\pgfqpoint{2.650988in}{4.352063in}}%
\pgfpathlineto{\pgfqpoint{2.655283in}{4.349764in}}%
\pgfpathlineto{\pgfqpoint{2.656357in}{4.347166in}}%
\pgfpathlineto{\pgfqpoint{2.657431in}{4.347312in}}%
\pgfpathlineto{\pgfqpoint{2.658505in}{4.350263in}}%
\pgfpathlineto{\pgfqpoint{2.661726in}{4.350864in}}%
\pgfpathlineto{\pgfqpoint{2.663874in}{4.349335in}}%
\pgfpathlineto{\pgfqpoint{2.664948in}{4.350502in}}%
\pgfpathlineto{\pgfqpoint{2.666022in}{4.349254in}}%
\pgfpathlineto{\pgfqpoint{2.669243in}{4.348670in}}%
\pgfpathlineto{\pgfqpoint{2.671391in}{4.345435in}}%
\pgfpathlineto{\pgfqpoint{2.673538in}{4.341543in}}%
\pgfpathlineto{\pgfqpoint{2.678908in}{4.340818in}}%
\pgfpathlineto{\pgfqpoint{2.679981in}{4.340069in}}%
\pgfpathlineto{\pgfqpoint{2.681055in}{4.340868in}}%
\pgfpathlineto{\pgfqpoint{2.684277in}{4.341218in}}%
\pgfpathlineto{\pgfqpoint{2.685351in}{4.337886in}}%
\pgfpathlineto{\pgfqpoint{2.686424in}{4.338875in}}%
\pgfpathlineto{\pgfqpoint{2.687498in}{4.342133in}}%
\pgfpathlineto{\pgfqpoint{2.688572in}{4.342544in}}%
\pgfpathlineto{\pgfqpoint{2.691794in}{4.340564in}}%
\pgfpathlineto{\pgfqpoint{2.693941in}{4.340652in}}%
\pgfpathlineto{\pgfqpoint{2.695015in}{4.338587in}}%
\pgfpathlineto{\pgfqpoint{2.696089in}{4.340450in}}%
\pgfpathlineto{\pgfqpoint{2.699311in}{4.339373in}}%
\pgfpathlineto{\pgfqpoint{2.700384in}{4.340352in}}%
\pgfpathlineto{\pgfqpoint{2.701458in}{4.342302in}}%
\pgfpathlineto{\pgfqpoint{2.702532in}{4.342543in}}%
\pgfpathlineto{\pgfqpoint{2.703606in}{4.343354in}}%
\pgfpathlineto{\pgfqpoint{2.710049in}{4.343570in}}%
\pgfpathlineto{\pgfqpoint{2.711123in}{4.345101in}}%
\pgfpathlineto{\pgfqpoint{2.714344in}{4.343705in}}%
\pgfpathlineto{\pgfqpoint{2.715418in}{4.330922in}}%
\pgfpathlineto{\pgfqpoint{2.716492in}{4.330486in}}%
\pgfpathlineto{\pgfqpoint{2.718640in}{4.332113in}}%
\pgfpathlineto{\pgfqpoint{2.721861in}{4.333874in}}%
\pgfpathlineto{\pgfqpoint{2.725083in}{4.326248in}}%
\pgfpathlineto{\pgfqpoint{2.729378in}{4.325971in}}%
\pgfpathlineto{\pgfqpoint{2.730452in}{4.324825in}}%
\pgfpathlineto{\pgfqpoint{2.732600in}{4.324608in}}%
\pgfpathlineto{\pgfqpoint{2.733673in}{4.324319in}}%
\pgfpathlineto{\pgfqpoint{2.739043in}{4.326949in}}%
\pgfpathlineto{\pgfqpoint{2.740116in}{4.329861in}}%
\pgfpathlineto{\pgfqpoint{2.741190in}{4.327743in}}%
\pgfpathlineto{\pgfqpoint{2.744412in}{4.328932in}}%
\pgfpathlineto{\pgfqpoint{2.745486in}{4.327879in}}%
\pgfpathlineto{\pgfqpoint{2.746559in}{4.329757in}}%
\pgfpathlineto{\pgfqpoint{2.747633in}{4.329109in}}%
\pgfpathlineto{\pgfqpoint{2.753002in}{4.328984in}}%
\pgfpathlineto{\pgfqpoint{2.755150in}{4.327351in}}%
\pgfpathlineto{\pgfqpoint{2.756224in}{4.328221in}}%
\pgfpathlineto{\pgfqpoint{2.760519in}{4.325211in}}%
\pgfpathlineto{\pgfqpoint{2.761593in}{4.326457in}}%
\pgfpathlineto{\pgfqpoint{2.763741in}{4.322248in}}%
\pgfpathlineto{\pgfqpoint{2.766962in}{4.317892in}}%
\pgfpathlineto{\pgfqpoint{2.769110in}{4.317629in}}%
\pgfpathlineto{\pgfqpoint{2.770184in}{4.319736in}}%
\pgfpathlineto{\pgfqpoint{2.771258in}{4.317663in}}%
\pgfpathlineto{\pgfqpoint{2.775553in}{4.317437in}}%
\pgfpathlineto{\pgfqpoint{2.776627in}{4.314205in}}%
\pgfpathlineto{\pgfqpoint{2.778775in}{4.314474in}}%
\pgfpathlineto{\pgfqpoint{2.783070in}{4.314823in}}%
\pgfpathlineto{\pgfqpoint{2.784144in}{4.316473in}}%
\pgfpathlineto{\pgfqpoint{2.786291in}{4.316929in}}%
\pgfpathlineto{\pgfqpoint{2.789513in}{4.313841in}}%
\pgfpathlineto{\pgfqpoint{2.791661in}{4.309940in}}%
\pgfpathlineto{\pgfqpoint{2.798104in}{4.310475in}}%
\pgfpathlineto{\pgfqpoint{2.799178in}{4.310229in}}%
\pgfpathlineto{\pgfqpoint{2.800251in}{4.311322in}}%
\pgfpathlineto{\pgfqpoint{2.801325in}{4.311183in}}%
\pgfpathlineto{\pgfqpoint{2.804547in}{4.312189in}}%
\pgfpathlineto{\pgfqpoint{2.806694in}{4.313692in}}%
\pgfpathlineto{\pgfqpoint{2.807768in}{4.312699in}}%
\pgfpathlineto{\pgfqpoint{2.808842in}{4.310601in}}%
\pgfpathlineto{\pgfqpoint{2.812064in}{4.310370in}}%
\pgfpathlineto{\pgfqpoint{2.813137in}{4.297384in}}%
\pgfpathlineto{\pgfqpoint{2.814211in}{4.294712in}}%
\pgfpathlineto{\pgfqpoint{2.815285in}{4.298786in}}%
\pgfpathlineto{\pgfqpoint{2.816359in}{4.300400in}}%
\pgfpathlineto{\pgfqpoint{2.819580in}{4.303724in}}%
\pgfpathlineto{\pgfqpoint{2.820654in}{4.302954in}}%
\pgfpathlineto{\pgfqpoint{2.821728in}{4.302944in}}%
\pgfpathlineto{\pgfqpoint{2.822802in}{4.304835in}}%
\pgfpathlineto{\pgfqpoint{2.823876in}{4.304845in}}%
\pgfpathlineto{\pgfqpoint{2.828171in}{4.302847in}}%
\pgfpathlineto{\pgfqpoint{2.829245in}{4.302642in}}%
\pgfpathlineto{\pgfqpoint{2.831393in}{4.300449in}}%
\pgfpathlineto{\pgfqpoint{2.834614in}{4.301160in}}%
\pgfpathlineto{\pgfqpoint{2.835688in}{4.300137in}}%
\pgfpathlineto{\pgfqpoint{2.837836in}{4.303779in}}%
\pgfpathlineto{\pgfqpoint{2.838910in}{4.303828in}}%
\pgfpathlineto{\pgfqpoint{2.842131in}{4.305802in}}%
\pgfpathlineto{\pgfqpoint{2.843205in}{4.303386in}}%
\pgfpathlineto{\pgfqpoint{2.844279in}{4.304605in}}%
\pgfpathlineto{\pgfqpoint{2.846426in}{4.304421in}}%
\pgfpathlineto{\pgfqpoint{2.849648in}{4.306840in}}%
\pgfpathlineto{\pgfqpoint{2.851796in}{4.303169in}}%
\pgfpathlineto{\pgfqpoint{2.852869in}{4.298587in}}%
\pgfpathlineto{\pgfqpoint{2.853943in}{4.300287in}}%
\pgfpathlineto{\pgfqpoint{2.860386in}{4.296713in}}%
\pgfpathlineto{\pgfqpoint{2.861460in}{4.298570in}}%
\pgfpathlineto{\pgfqpoint{2.864682in}{4.299589in}}%
\pgfpathlineto{\pgfqpoint{2.866829in}{4.304230in}}%
\pgfpathlineto{\pgfqpoint{2.867903in}{4.305151in}}%
\pgfpathlineto{\pgfqpoint{2.868977in}{4.305075in}}%
\pgfpathlineto{\pgfqpoint{2.873272in}{4.305482in}}%
\pgfpathlineto{\pgfqpoint{2.876494in}{4.302194in}}%
\pgfpathlineto{\pgfqpoint{2.880789in}{4.302839in}}%
\pgfpathlineto{\pgfqpoint{2.881863in}{4.302166in}}%
\pgfpathlineto{\pgfqpoint{2.882937in}{4.302590in}}%
\pgfpathlineto{\pgfqpoint{2.884011in}{4.302268in}}%
\pgfpathlineto{\pgfqpoint{2.884011in}{4.302268in}}%
\pgfusepath{stroke}%
\end{pgfscope}%
\begin{pgfscope}%
\pgfsetrectcap%
\pgfsetmiterjoin%
\pgfsetlinewidth{0.803000pt}%
\definecolor{currentstroke}{rgb}{1.000000,1.000000,1.000000}%
\pgfsetstrokecolor{currentstroke}%
\pgfsetdash{}{0pt}%
\pgfpathmoveto{\pgfqpoint{0.418102in}{4.180131in}}%
\pgfpathlineto{\pgfqpoint{0.418102in}{4.581016in}}%
\pgfusepath{stroke}%
\end{pgfscope}%
\begin{pgfscope}%
\pgfsetrectcap%
\pgfsetmiterjoin%
\pgfsetlinewidth{0.803000pt}%
\definecolor{currentstroke}{rgb}{1.000000,1.000000,1.000000}%
\pgfsetstrokecolor{currentstroke}%
\pgfsetdash{}{0pt}%
\pgfpathmoveto{\pgfqpoint{3.001435in}{4.180131in}}%
\pgfpathlineto{\pgfqpoint{3.001435in}{4.581016in}}%
\pgfusepath{stroke}%
\end{pgfscope}%
\begin{pgfscope}%
\pgfsetrectcap%
\pgfsetmiterjoin%
\pgfsetlinewidth{0.803000pt}%
\definecolor{currentstroke}{rgb}{1.000000,1.000000,1.000000}%
\pgfsetstrokecolor{currentstroke}%
\pgfsetdash{}{0pt}%
\pgfpathmoveto{\pgfqpoint{0.418102in}{4.180131in}}%
\pgfpathlineto{\pgfqpoint{3.001435in}{4.180131in}}%
\pgfusepath{stroke}%
\end{pgfscope}%
\begin{pgfscope}%
\pgfsetrectcap%
\pgfsetmiterjoin%
\pgfsetlinewidth{0.803000pt}%
\definecolor{currentstroke}{rgb}{1.000000,1.000000,1.000000}%
\pgfsetstrokecolor{currentstroke}%
\pgfsetdash{}{0pt}%
\pgfpathmoveto{\pgfqpoint{0.418102in}{4.581016in}}%
\pgfpathlineto{\pgfqpoint{3.001435in}{4.581016in}}%
\pgfusepath{stroke}%
\end{pgfscope}%
\begin{pgfscope}%
\definecolor{textcolor}{rgb}{0.150000,0.150000,0.150000}%
\pgfsetstrokecolor{textcolor}%
\pgfsetfillcolor{textcolor}%
\pgftext[x=1.709768in,y=4.664349in,,base]{\color{textcolor}\rmfamily\fontsize{12.000000}{14.400000}\selectfont MMM}%
\end{pgfscope}%
\begin{pgfscope}%
\pgfsetbuttcap%
\pgfsetmiterjoin%
\definecolor{currentfill}{rgb}{0.917647,0.917647,0.949020}%
\pgfsetfillcolor{currentfill}%
\pgfsetlinewidth{0.000000pt}%
\definecolor{currentstroke}{rgb}{0.000000,0.000000,0.000000}%
\pgfsetstrokecolor{currentstroke}%
\pgfsetstrokeopacity{0.000000}%
\pgfsetdash{}{0pt}%
\pgfpathmoveto{\pgfqpoint{4.034768in}{4.180131in}}%
\pgfpathlineto{\pgfqpoint{6.618102in}{4.180131in}}%
\pgfpathlineto{\pgfqpoint{6.618102in}{4.581016in}}%
\pgfpathlineto{\pgfqpoint{4.034768in}{4.581016in}}%
\pgfpathclose%
\pgfusepath{fill}%
\end{pgfscope}%
\begin{pgfscope}%
\pgfpathrectangle{\pgfqpoint{4.034768in}{4.180131in}}{\pgfqpoint{2.583333in}{0.400885in}}%
\pgfusepath{clip}%
\pgfsetroundcap%
\pgfsetroundjoin%
\pgfsetlinewidth{0.803000pt}%
\definecolor{currentstroke}{rgb}{1.000000,1.000000,1.000000}%
\pgfsetstrokecolor{currentstroke}%
\pgfsetdash{}{0pt}%
\pgfpathmoveto{\pgfqpoint{4.150045in}{4.180131in}}%
\pgfpathlineto{\pgfqpoint{4.150045in}{4.581016in}}%
\pgfusepath{stroke}%
\end{pgfscope}%
\begin{pgfscope}%
\definecolor{textcolor}{rgb}{0.150000,0.150000,0.150000}%
\pgfsetstrokecolor{textcolor}%
\pgfsetfillcolor{textcolor}%
\pgftext[x=4.150045in,y=4.082908in,,top]{\color{textcolor}\rmfamily\fontsize{10.000000}{12.000000}\selectfont 2012}%
\end{pgfscope}%
\begin{pgfscope}%
\pgfpathrectangle{\pgfqpoint{4.034768in}{4.180131in}}{\pgfqpoint{2.583333in}{0.400885in}}%
\pgfusepath{clip}%
\pgfsetroundcap%
\pgfsetroundjoin%
\pgfsetlinewidth{0.803000pt}%
\definecolor{currentstroke}{rgb}{1.000000,1.000000,1.000000}%
\pgfsetstrokecolor{currentstroke}%
\pgfsetdash{}{0pt}%
\pgfpathmoveto{\pgfqpoint{4.543070in}{4.180131in}}%
\pgfpathlineto{\pgfqpoint{4.543070in}{4.581016in}}%
\pgfusepath{stroke}%
\end{pgfscope}%
\begin{pgfscope}%
\definecolor{textcolor}{rgb}{0.150000,0.150000,0.150000}%
\pgfsetstrokecolor{textcolor}%
\pgfsetfillcolor{textcolor}%
\pgftext[x=4.543070in,y=4.082908in,,top]{\color{textcolor}\rmfamily\fontsize{10.000000}{12.000000}\selectfont 2013}%
\end{pgfscope}%
\begin{pgfscope}%
\pgfpathrectangle{\pgfqpoint{4.034768in}{4.180131in}}{\pgfqpoint{2.583333in}{0.400885in}}%
\pgfusepath{clip}%
\pgfsetroundcap%
\pgfsetroundjoin%
\pgfsetlinewidth{0.803000pt}%
\definecolor{currentstroke}{rgb}{1.000000,1.000000,1.000000}%
\pgfsetstrokecolor{currentstroke}%
\pgfsetdash{}{0pt}%
\pgfpathmoveto{\pgfqpoint{4.935021in}{4.180131in}}%
\pgfpathlineto{\pgfqpoint{4.935021in}{4.581016in}}%
\pgfusepath{stroke}%
\end{pgfscope}%
\begin{pgfscope}%
\definecolor{textcolor}{rgb}{0.150000,0.150000,0.150000}%
\pgfsetstrokecolor{textcolor}%
\pgfsetfillcolor{textcolor}%
\pgftext[x=4.935021in,y=4.082908in,,top]{\color{textcolor}\rmfamily\fontsize{10.000000}{12.000000}\selectfont 2014}%
\end{pgfscope}%
\begin{pgfscope}%
\pgfpathrectangle{\pgfqpoint{4.034768in}{4.180131in}}{\pgfqpoint{2.583333in}{0.400885in}}%
\pgfusepath{clip}%
\pgfsetroundcap%
\pgfsetroundjoin%
\pgfsetlinewidth{0.803000pt}%
\definecolor{currentstroke}{rgb}{1.000000,1.000000,1.000000}%
\pgfsetstrokecolor{currentstroke}%
\pgfsetdash{}{0pt}%
\pgfpathmoveto{\pgfqpoint{5.326972in}{4.180131in}}%
\pgfpathlineto{\pgfqpoint{5.326972in}{4.581016in}}%
\pgfusepath{stroke}%
\end{pgfscope}%
\begin{pgfscope}%
\definecolor{textcolor}{rgb}{0.150000,0.150000,0.150000}%
\pgfsetstrokecolor{textcolor}%
\pgfsetfillcolor{textcolor}%
\pgftext[x=5.326972in,y=4.082908in,,top]{\color{textcolor}\rmfamily\fontsize{10.000000}{12.000000}\selectfont 2015}%
\end{pgfscope}%
\begin{pgfscope}%
\pgfpathrectangle{\pgfqpoint{4.034768in}{4.180131in}}{\pgfqpoint{2.583333in}{0.400885in}}%
\pgfusepath{clip}%
\pgfsetroundcap%
\pgfsetroundjoin%
\pgfsetlinewidth{0.803000pt}%
\definecolor{currentstroke}{rgb}{1.000000,1.000000,1.000000}%
\pgfsetstrokecolor{currentstroke}%
\pgfsetdash{}{0pt}%
\pgfpathmoveto{\pgfqpoint{5.718923in}{4.180131in}}%
\pgfpathlineto{\pgfqpoint{5.718923in}{4.581016in}}%
\pgfusepath{stroke}%
\end{pgfscope}%
\begin{pgfscope}%
\definecolor{textcolor}{rgb}{0.150000,0.150000,0.150000}%
\pgfsetstrokecolor{textcolor}%
\pgfsetfillcolor{textcolor}%
\pgftext[x=5.718923in,y=4.082908in,,top]{\color{textcolor}\rmfamily\fontsize{10.000000}{12.000000}\selectfont 2016}%
\end{pgfscope}%
\begin{pgfscope}%
\pgfpathrectangle{\pgfqpoint{4.034768in}{4.180131in}}{\pgfqpoint{2.583333in}{0.400885in}}%
\pgfusepath{clip}%
\pgfsetroundcap%
\pgfsetroundjoin%
\pgfsetlinewidth{0.803000pt}%
\definecolor{currentstroke}{rgb}{1.000000,1.000000,1.000000}%
\pgfsetstrokecolor{currentstroke}%
\pgfsetdash{}{0pt}%
\pgfpathmoveto{\pgfqpoint{6.111948in}{4.180131in}}%
\pgfpathlineto{\pgfqpoint{6.111948in}{4.581016in}}%
\pgfusepath{stroke}%
\end{pgfscope}%
\begin{pgfscope}%
\definecolor{textcolor}{rgb}{0.150000,0.150000,0.150000}%
\pgfsetstrokecolor{textcolor}%
\pgfsetfillcolor{textcolor}%
\pgftext[x=6.111948in,y=4.082908in,,top]{\color{textcolor}\rmfamily\fontsize{10.000000}{12.000000}\selectfont 2017}%
\end{pgfscope}%
\begin{pgfscope}%
\pgfpathrectangle{\pgfqpoint{4.034768in}{4.180131in}}{\pgfqpoint{2.583333in}{0.400885in}}%
\pgfusepath{clip}%
\pgfsetroundcap%
\pgfsetroundjoin%
\pgfsetlinewidth{0.803000pt}%
\definecolor{currentstroke}{rgb}{1.000000,1.000000,1.000000}%
\pgfsetstrokecolor{currentstroke}%
\pgfsetdash{}{0pt}%
\pgfpathmoveto{\pgfqpoint{6.503899in}{4.180131in}}%
\pgfpathlineto{\pgfqpoint{6.503899in}{4.581016in}}%
\pgfusepath{stroke}%
\end{pgfscope}%
\begin{pgfscope}%
\definecolor{textcolor}{rgb}{0.150000,0.150000,0.150000}%
\pgfsetstrokecolor{textcolor}%
\pgfsetfillcolor{textcolor}%
\pgftext[x=6.503899in,y=4.082908in,,top]{\color{textcolor}\rmfamily\fontsize{10.000000}{12.000000}\selectfont 2018}%
\end{pgfscope}%
\begin{pgfscope}%
\pgfpathrectangle{\pgfqpoint{4.034768in}{4.180131in}}{\pgfqpoint{2.583333in}{0.400885in}}%
\pgfusepath{clip}%
\pgfsetroundcap%
\pgfsetroundjoin%
\pgfsetlinewidth{0.803000pt}%
\definecolor{currentstroke}{rgb}{1.000000,1.000000,1.000000}%
\pgfsetstrokecolor{currentstroke}%
\pgfsetdash{}{0pt}%
\pgfpathmoveto{\pgfqpoint{4.034768in}{4.243467in}}%
\pgfpathlineto{\pgfqpoint{6.618102in}{4.243467in}}%
\pgfusepath{stroke}%
\end{pgfscope}%
\begin{pgfscope}%
\definecolor{textcolor}{rgb}{0.150000,0.150000,0.150000}%
\pgfsetstrokecolor{textcolor}%
\pgfsetfillcolor{textcolor}%
\pgftext[x=3.849181in,y=4.190706in,left,base]{\color{textcolor}\rmfamily\fontsize{10.000000}{12.000000}\selectfont 1}%
\end{pgfscope}%
\begin{pgfscope}%
\pgfpathrectangle{\pgfqpoint{4.034768in}{4.180131in}}{\pgfqpoint{2.583333in}{0.400885in}}%
\pgfusepath{clip}%
\pgfsetroundcap%
\pgfsetroundjoin%
\pgfsetlinewidth{0.803000pt}%
\definecolor{currentstroke}{rgb}{1.000000,1.000000,1.000000}%
\pgfsetstrokecolor{currentstroke}%
\pgfsetdash{}{0pt}%
\pgfpathmoveto{\pgfqpoint{4.034768in}{4.499686in}}%
\pgfpathlineto{\pgfqpoint{6.618102in}{4.499686in}}%
\pgfusepath{stroke}%
\end{pgfscope}%
\begin{pgfscope}%
\definecolor{textcolor}{rgb}{0.150000,0.150000,0.150000}%
\pgfsetstrokecolor{textcolor}%
\pgfsetfillcolor{textcolor}%
\pgftext[x=3.849181in,y=4.446924in,left,base]{\color{textcolor}\rmfamily\fontsize{10.000000}{12.000000}\selectfont 2}%
\end{pgfscope}%
\begin{pgfscope}%
\pgfpathrectangle{\pgfqpoint{4.034768in}{4.180131in}}{\pgfqpoint{2.583333in}{0.400885in}}%
\pgfusepath{clip}%
\pgfsetroundcap%
\pgfsetroundjoin%
\pgfsetlinewidth{1.505625pt}%
\definecolor{currentstroke}{rgb}{0.000000,0.000000,0.000000}%
\pgfsetstrokecolor{currentstroke}%
\pgfsetdash{}{0pt}%
\pgfpathmoveto{\pgfqpoint{4.152193in}{4.243467in}}%
\pgfpathlineto{\pgfqpoint{4.153266in}{4.243645in}}%
\pgfpathlineto{\pgfqpoint{4.154340in}{4.246605in}}%
\pgfpathlineto{\pgfqpoint{4.155414in}{4.243763in}}%
\pgfpathlineto{\pgfqpoint{4.158636in}{4.244414in}}%
\pgfpathlineto{\pgfqpoint{4.160783in}{4.247374in}}%
\pgfpathlineto{\pgfqpoint{4.161857in}{4.251104in}}%
\pgfpathlineto{\pgfqpoint{4.167226in}{4.254123in}}%
\pgfpathlineto{\pgfqpoint{4.169374in}{4.258031in}}%
\pgfpathlineto{\pgfqpoint{4.170448in}{4.253176in}}%
\pgfpathlineto{\pgfqpoint{4.174743in}{4.248854in}}%
\pgfpathlineto{\pgfqpoint{4.175817in}{4.253887in}}%
\pgfpathlineto{\pgfqpoint{4.181186in}{4.248322in}}%
\pgfpathlineto{\pgfqpoint{4.183334in}{4.256255in}}%
\pgfpathlineto{\pgfqpoint{4.184408in}{4.259155in}}%
\pgfpathlineto{\pgfqpoint{4.185482in}{4.264957in}}%
\pgfpathlineto{\pgfqpoint{4.188703in}{4.262589in}}%
\pgfpathlineto{\pgfqpoint{4.189777in}{4.264306in}}%
\pgfpathlineto{\pgfqpoint{4.190851in}{4.261701in}}%
\pgfpathlineto{\pgfqpoint{4.191925in}{4.265253in}}%
\pgfpathlineto{\pgfqpoint{4.192998in}{4.262589in}}%
\pgfpathlineto{\pgfqpoint{4.196220in}{4.263951in}}%
\pgfpathlineto{\pgfqpoint{4.197294in}{4.263418in}}%
\pgfpathlineto{\pgfqpoint{4.198368in}{4.261050in}}%
\pgfpathlineto{\pgfqpoint{4.199441in}{4.268213in}}%
\pgfpathlineto{\pgfqpoint{4.205885in}{4.267799in}}%
\pgfpathlineto{\pgfqpoint{4.206958in}{4.267147in}}%
\pgfpathlineto{\pgfqpoint{4.208032in}{4.270699in}}%
\pgfpathlineto{\pgfqpoint{4.211254in}{4.275199in}}%
\pgfpathlineto{\pgfqpoint{4.212328in}{4.272949in}}%
\pgfpathlineto{\pgfqpoint{4.213401in}{4.268331in}}%
\pgfpathlineto{\pgfqpoint{4.214475in}{4.271943in}}%
\pgfpathlineto{\pgfqpoint{4.215549in}{4.268864in}}%
\pgfpathlineto{\pgfqpoint{4.218771in}{4.268746in}}%
\pgfpathlineto{\pgfqpoint{4.219844in}{4.262234in}}%
\pgfpathlineto{\pgfqpoint{4.223066in}{4.269989in}}%
\pgfpathlineto{\pgfqpoint{4.226287in}{4.267680in}}%
\pgfpathlineto{\pgfqpoint{4.229509in}{4.288696in}}%
\pgfpathlineto{\pgfqpoint{4.230583in}{4.287808in}}%
\pgfpathlineto{\pgfqpoint{4.233804in}{4.291597in}}%
\pgfpathlineto{\pgfqpoint{4.234878in}{4.289703in}}%
\pgfpathlineto{\pgfqpoint{4.238100in}{4.291479in}}%
\pgfpathlineto{\pgfqpoint{4.241321in}{4.298997in}}%
\pgfpathlineto{\pgfqpoint{4.242395in}{4.296688in}}%
\pgfpathlineto{\pgfqpoint{4.243469in}{4.301128in}}%
\pgfpathlineto{\pgfqpoint{4.244543in}{4.294912in}}%
\pgfpathlineto{\pgfqpoint{4.245617in}{4.294735in}}%
\pgfpathlineto{\pgfqpoint{4.248838in}{4.295564in}}%
\pgfpathlineto{\pgfqpoint{4.249912in}{4.298642in}}%
\pgfpathlineto{\pgfqpoint{4.250986in}{4.293195in}}%
\pgfpathlineto{\pgfqpoint{4.252060in}{4.297103in}}%
\pgfpathlineto{\pgfqpoint{4.256355in}{4.292071in}}%
\pgfpathlineto{\pgfqpoint{4.257429in}{4.286861in}}%
\pgfpathlineto{\pgfqpoint{4.259576in}{4.296748in}}%
\pgfpathlineto{\pgfqpoint{4.260650in}{4.292722in}}%
\pgfpathlineto{\pgfqpoint{4.263872in}{4.295623in}}%
\pgfpathlineto{\pgfqpoint{4.264946in}{4.297517in}}%
\pgfpathlineto{\pgfqpoint{4.266019in}{4.296748in}}%
\pgfpathlineto{\pgfqpoint{4.267093in}{4.294261in}}%
\pgfpathlineto{\pgfqpoint{4.268167in}{4.293610in}}%
\pgfpathlineto{\pgfqpoint{4.271389in}{4.292959in}}%
\pgfpathlineto{\pgfqpoint{4.272462in}{4.294557in}}%
\pgfpathlineto{\pgfqpoint{4.274610in}{4.305036in}}%
\pgfpathlineto{\pgfqpoint{4.275684in}{4.308114in}}%
\pgfpathlineto{\pgfqpoint{4.278906in}{4.308351in}}%
\pgfpathlineto{\pgfqpoint{4.279979in}{4.312376in}}%
\pgfpathlineto{\pgfqpoint{4.281053in}{4.312850in}}%
\pgfpathlineto{\pgfqpoint{4.282127in}{4.311784in}}%
\pgfpathlineto{\pgfqpoint{4.283201in}{4.307759in}}%
\pgfpathlineto{\pgfqpoint{4.286422in}{4.307759in}}%
\pgfpathlineto{\pgfqpoint{4.287496in}{4.306634in}}%
\pgfpathlineto{\pgfqpoint{4.288570in}{4.304325in}}%
\pgfpathlineto{\pgfqpoint{4.289644in}{4.304148in}}%
\pgfpathlineto{\pgfqpoint{4.290718in}{4.305332in}}%
\pgfpathlineto{\pgfqpoint{4.296087in}{4.293314in}}%
\pgfpathlineto{\pgfqpoint{4.297161in}{4.284019in}}%
\pgfpathlineto{\pgfqpoint{4.298235in}{4.282717in}}%
\pgfpathlineto{\pgfqpoint{4.301456in}{4.287631in}}%
\pgfpathlineto{\pgfqpoint{4.302530in}{4.287749in}}%
\pgfpathlineto{\pgfqpoint{4.303604in}{4.285736in}}%
\pgfpathlineto{\pgfqpoint{4.304678in}{4.287690in}}%
\pgfpathlineto{\pgfqpoint{4.305751in}{4.284848in}}%
\pgfpathlineto{\pgfqpoint{4.310047in}{4.288874in}}%
\pgfpathlineto{\pgfqpoint{4.311121in}{4.283013in}}%
\pgfpathlineto{\pgfqpoint{4.312195in}{4.284967in}}%
\pgfpathlineto{\pgfqpoint{4.313268in}{4.272179in}}%
\pgfpathlineto{\pgfqpoint{4.316490in}{4.272594in}}%
\pgfpathlineto{\pgfqpoint{4.317564in}{4.275139in}}%
\pgfpathlineto{\pgfqpoint{4.318638in}{4.282599in}}%
\pgfpathlineto{\pgfqpoint{4.319711in}{4.281829in}}%
\pgfpathlineto{\pgfqpoint{4.320785in}{4.285144in}}%
\pgfpathlineto{\pgfqpoint{4.324007in}{4.281119in}}%
\pgfpathlineto{\pgfqpoint{4.325081in}{4.288459in}}%
\pgfpathlineto{\pgfqpoint{4.326154in}{4.281119in}}%
\pgfpathlineto{\pgfqpoint{4.327228in}{4.280823in}}%
\pgfpathlineto{\pgfqpoint{4.328302in}{4.287394in}}%
\pgfpathlineto{\pgfqpoint{4.331524in}{4.285085in}}%
\pgfpathlineto{\pgfqpoint{4.332597in}{4.290887in}}%
\pgfpathlineto{\pgfqpoint{4.333671in}{4.293551in}}%
\pgfpathlineto{\pgfqpoint{4.334745in}{4.287571in}}%
\pgfpathlineto{\pgfqpoint{4.335819in}{4.290117in}}%
\pgfpathlineto{\pgfqpoint{4.339040in}{4.286151in}}%
\pgfpathlineto{\pgfqpoint{4.340114in}{4.286506in}}%
\pgfpathlineto{\pgfqpoint{4.341188in}{4.290591in}}%
\pgfpathlineto{\pgfqpoint{4.342262in}{4.289584in}}%
\pgfpathlineto{\pgfqpoint{4.343336in}{4.297695in}}%
\pgfpathlineto{\pgfqpoint{4.346557in}{4.302135in}}%
\pgfpathlineto{\pgfqpoint{4.347631in}{4.305154in}}%
\pgfpathlineto{\pgfqpoint{4.349779in}{4.304325in}}%
\pgfpathlineto{\pgfqpoint{4.350853in}{4.300951in}}%
\pgfpathlineto{\pgfqpoint{4.355148in}{4.299708in}}%
\pgfpathlineto{\pgfqpoint{4.356222in}{4.298050in}}%
\pgfpathlineto{\pgfqpoint{4.357296in}{4.291893in}}%
\pgfpathlineto{\pgfqpoint{4.358370in}{4.297221in}}%
\pgfpathlineto{\pgfqpoint{4.361591in}{4.301010in}}%
\pgfpathlineto{\pgfqpoint{4.362665in}{4.301247in}}%
\pgfpathlineto{\pgfqpoint{4.363739in}{4.299175in}}%
\pgfpathlineto{\pgfqpoint{4.364813in}{4.288163in}}%
\pgfpathlineto{\pgfqpoint{4.365886in}{4.285914in}}%
\pgfpathlineto{\pgfqpoint{4.370182in}{4.284907in}}%
\pgfpathlineto{\pgfqpoint{4.371256in}{4.287157in}}%
\pgfpathlineto{\pgfqpoint{4.372329in}{4.296333in}}%
\pgfpathlineto{\pgfqpoint{4.373403in}{4.300418in}}%
\pgfpathlineto{\pgfqpoint{4.376625in}{4.299175in}}%
\pgfpathlineto{\pgfqpoint{4.377699in}{4.296037in}}%
\pgfpathlineto{\pgfqpoint{4.378773in}{4.291183in}}%
\pgfpathlineto{\pgfqpoint{4.379846in}{4.289525in}}%
\pgfpathlineto{\pgfqpoint{4.380920in}{4.295504in}}%
\pgfpathlineto{\pgfqpoint{4.384142in}{4.292781in}}%
\pgfpathlineto{\pgfqpoint{4.386289in}{4.297103in}}%
\pgfpathlineto{\pgfqpoint{4.387363in}{4.289407in}}%
\pgfpathlineto{\pgfqpoint{4.388437in}{4.286091in}}%
\pgfpathlineto{\pgfqpoint{4.391659in}{4.287571in}}%
\pgfpathlineto{\pgfqpoint{4.392732in}{4.287335in}}%
\pgfpathlineto{\pgfqpoint{4.395954in}{4.295386in}}%
\pgfpathlineto{\pgfqpoint{4.400249in}{4.290117in}}%
\pgfpathlineto{\pgfqpoint{4.401323in}{4.291301in}}%
\pgfpathlineto{\pgfqpoint{4.402397in}{4.289170in}}%
\pgfpathlineto{\pgfqpoint{4.403471in}{4.294853in}}%
\pgfpathlineto{\pgfqpoint{4.406692in}{4.294498in}}%
\pgfpathlineto{\pgfqpoint{4.407766in}{4.295268in}}%
\pgfpathlineto{\pgfqpoint{4.409914in}{4.293195in}}%
\pgfpathlineto{\pgfqpoint{4.410988in}{4.299234in}}%
\pgfpathlineto{\pgfqpoint{4.415283in}{4.300892in}}%
\pgfpathlineto{\pgfqpoint{4.416357in}{4.293255in}}%
\pgfpathlineto{\pgfqpoint{4.418505in}{4.296156in}}%
\pgfpathlineto{\pgfqpoint{4.421726in}{4.295031in}}%
\pgfpathlineto{\pgfqpoint{4.422800in}{4.293491in}}%
\pgfpathlineto{\pgfqpoint{4.423874in}{4.293728in}}%
\pgfpathlineto{\pgfqpoint{4.424948in}{4.303200in}}%
\pgfpathlineto{\pgfqpoint{4.426021in}{4.304384in}}%
\pgfpathlineto{\pgfqpoint{4.429243in}{4.303911in}}%
\pgfpathlineto{\pgfqpoint{4.430317in}{4.301188in}}%
\pgfpathlineto{\pgfqpoint{4.431391in}{4.301188in}}%
\pgfpathlineto{\pgfqpoint{4.433538in}{4.296866in}}%
\pgfpathlineto{\pgfqpoint{4.436760in}{4.295860in}}%
\pgfpathlineto{\pgfqpoint{4.437834in}{4.292959in}}%
\pgfpathlineto{\pgfqpoint{4.438907in}{4.287631in}}%
\pgfpathlineto{\pgfqpoint{4.441055in}{4.291479in}}%
\pgfpathlineto{\pgfqpoint{4.444277in}{4.296156in}}%
\pgfpathlineto{\pgfqpoint{4.445350in}{4.293195in}}%
\pgfpathlineto{\pgfqpoint{4.446424in}{4.295504in}}%
\pgfpathlineto{\pgfqpoint{4.447498in}{4.300655in}}%
\pgfpathlineto{\pgfqpoint{4.448572in}{4.301720in}}%
\pgfpathlineto{\pgfqpoint{4.451794in}{4.303082in}}%
\pgfpathlineto{\pgfqpoint{4.453941in}{4.298524in}}%
\pgfpathlineto{\pgfqpoint{4.455015in}{4.301247in}}%
\pgfpathlineto{\pgfqpoint{4.456089in}{4.298109in}}%
\pgfpathlineto{\pgfqpoint{4.459310in}{4.296511in}}%
\pgfpathlineto{\pgfqpoint{4.461458in}{4.306042in}}%
\pgfpathlineto{\pgfqpoint{4.462532in}{4.296629in}}%
\pgfpathlineto{\pgfqpoint{4.463606in}{4.292603in}}%
\pgfpathlineto{\pgfqpoint{4.466827in}{4.291952in}}%
\pgfpathlineto{\pgfqpoint{4.467901in}{4.284611in}}%
\pgfpathlineto{\pgfqpoint{4.468975in}{4.283842in}}%
\pgfpathlineto{\pgfqpoint{4.471123in}{4.286624in}}%
\pgfpathlineto{\pgfqpoint{4.476492in}{4.287808in}}%
\pgfpathlineto{\pgfqpoint{4.477566in}{4.292544in}}%
\pgfpathlineto{\pgfqpoint{4.481861in}{4.289880in}}%
\pgfpathlineto{\pgfqpoint{4.482935in}{4.294320in}}%
\pgfpathlineto{\pgfqpoint{4.484009in}{4.285677in}}%
\pgfpathlineto{\pgfqpoint{4.485083in}{4.285618in}}%
\pgfpathlineto{\pgfqpoint{4.486156in}{4.287039in}}%
\pgfpathlineto{\pgfqpoint{4.489378in}{4.285381in}}%
\pgfpathlineto{\pgfqpoint{4.491526in}{4.275317in}}%
\pgfpathlineto{\pgfqpoint{4.492599in}{4.275258in}}%
\pgfpathlineto{\pgfqpoint{4.493673in}{4.278810in}}%
\pgfpathlineto{\pgfqpoint{4.496895in}{4.283842in}}%
\pgfpathlineto{\pgfqpoint{4.497969in}{4.287216in}}%
\pgfpathlineto{\pgfqpoint{4.499042in}{4.287867in}}%
\pgfpathlineto{\pgfqpoint{4.501190in}{4.290709in}}%
\pgfpathlineto{\pgfqpoint{4.504412in}{4.286269in}}%
\pgfpathlineto{\pgfqpoint{4.505485in}{4.279579in}}%
\pgfpathlineto{\pgfqpoint{4.506559in}{4.285322in}}%
\pgfpathlineto{\pgfqpoint{4.507633in}{4.287512in}}%
\pgfpathlineto{\pgfqpoint{4.508707in}{4.287394in}}%
\pgfpathlineto{\pgfqpoint{4.511928in}{4.287927in}}%
\pgfpathlineto{\pgfqpoint{4.513002in}{4.287098in}}%
\pgfpathlineto{\pgfqpoint{4.514076in}{4.290117in}}%
\pgfpathlineto{\pgfqpoint{4.515150in}{4.288578in}}%
\pgfpathlineto{\pgfqpoint{4.516224in}{4.291242in}}%
\pgfpathlineto{\pgfqpoint{4.519445in}{4.292011in}}%
\pgfpathlineto{\pgfqpoint{4.520519in}{4.293669in}}%
\pgfpathlineto{\pgfqpoint{4.521593in}{4.296925in}}%
\pgfpathlineto{\pgfqpoint{4.522667in}{4.297221in}}%
\pgfpathlineto{\pgfqpoint{4.523741in}{4.291419in}}%
\pgfpathlineto{\pgfqpoint{4.526962in}{4.294794in}}%
\pgfpathlineto{\pgfqpoint{4.528036in}{4.297754in}}%
\pgfpathlineto{\pgfqpoint{4.529110in}{4.292189in}}%
\pgfpathlineto{\pgfqpoint{4.530184in}{4.295504in}}%
\pgfpathlineto{\pgfqpoint{4.531258in}{4.296807in}}%
\pgfpathlineto{\pgfqpoint{4.534479in}{4.296156in}}%
\pgfpathlineto{\pgfqpoint{4.536627in}{4.294024in}}%
\pgfpathlineto{\pgfqpoint{4.537701in}{4.291656in}}%
\pgfpathlineto{\pgfqpoint{4.538774in}{4.291538in}}%
\pgfpathlineto{\pgfqpoint{4.541996in}{4.295919in}}%
\pgfpathlineto{\pgfqpoint{4.544144in}{4.303852in}}%
\pgfpathlineto{\pgfqpoint{4.545217in}{4.305154in}}%
\pgfpathlineto{\pgfqpoint{4.546291in}{4.308469in}}%
\pgfpathlineto{\pgfqpoint{4.549513in}{4.309831in}}%
\pgfpathlineto{\pgfqpoint{4.550587in}{4.311666in}}%
\pgfpathlineto{\pgfqpoint{4.551661in}{4.311962in}}%
\pgfpathlineto{\pgfqpoint{4.553808in}{4.317231in}}%
\pgfpathlineto{\pgfqpoint{4.557030in}{4.317053in}}%
\pgfpathlineto{\pgfqpoint{4.558104in}{4.314685in}}%
\pgfpathlineto{\pgfqpoint{4.559177in}{4.313916in}}%
\pgfpathlineto{\pgfqpoint{4.560251in}{4.314567in}}%
\pgfpathlineto{\pgfqpoint{4.561325in}{4.309357in}}%
\pgfpathlineto{\pgfqpoint{4.565620in}{4.307344in}}%
\pgfpathlineto{\pgfqpoint{4.566694in}{4.304799in}}%
\pgfpathlineto{\pgfqpoint{4.568842in}{4.307877in}}%
\pgfpathlineto{\pgfqpoint{4.572063in}{4.305568in}}%
\pgfpathlineto{\pgfqpoint{4.573137in}{4.307581in}}%
\pgfpathlineto{\pgfqpoint{4.574211in}{4.306930in}}%
\pgfpathlineto{\pgfqpoint{4.575285in}{4.304148in}}%
\pgfpathlineto{\pgfqpoint{4.576359in}{4.310068in}}%
\pgfpathlineto{\pgfqpoint{4.579580in}{4.307404in}}%
\pgfpathlineto{\pgfqpoint{4.580654in}{4.314093in}}%
\pgfpathlineto{\pgfqpoint{4.581728in}{4.313560in}}%
\pgfpathlineto{\pgfqpoint{4.582802in}{4.322085in}}%
\pgfpathlineto{\pgfqpoint{4.583876in}{4.320250in}}%
\pgfpathlineto{\pgfqpoint{4.587097in}{4.321256in}}%
\pgfpathlineto{\pgfqpoint{4.588171in}{4.322440in}}%
\pgfpathlineto{\pgfqpoint{4.589245in}{4.321848in}}%
\pgfpathlineto{\pgfqpoint{4.590319in}{4.323151in}}%
\pgfpathlineto{\pgfqpoint{4.591393in}{4.319658in}}%
\pgfpathlineto{\pgfqpoint{4.595688in}{4.322855in}}%
\pgfpathlineto{\pgfqpoint{4.597836in}{4.318888in}}%
\pgfpathlineto{\pgfqpoint{4.598909in}{4.324394in}}%
\pgfpathlineto{\pgfqpoint{4.603205in}{4.320901in}}%
\pgfpathlineto{\pgfqpoint{4.604279in}{4.324039in}}%
\pgfpathlineto{\pgfqpoint{4.605352in}{4.322144in}}%
\pgfpathlineto{\pgfqpoint{4.609648in}{4.326111in}}%
\pgfpathlineto{\pgfqpoint{4.610722in}{4.332741in}}%
\pgfpathlineto{\pgfqpoint{4.611795in}{4.335642in}}%
\pgfpathlineto{\pgfqpoint{4.612869in}{4.335109in}}%
\pgfpathlineto{\pgfqpoint{4.613943in}{4.335879in}}%
\pgfpathlineto{\pgfqpoint{4.617165in}{4.340201in}}%
\pgfpathlineto{\pgfqpoint{4.618238in}{4.339076in}}%
\pgfpathlineto{\pgfqpoint{4.620386in}{4.339549in}}%
\pgfpathlineto{\pgfqpoint{4.621460in}{4.343397in}}%
\pgfpathlineto{\pgfqpoint{4.624682in}{4.341681in}}%
\pgfpathlineto{\pgfqpoint{4.625755in}{4.338188in}}%
\pgfpathlineto{\pgfqpoint{4.626829in}{4.342865in}}%
\pgfpathlineto{\pgfqpoint{4.627903in}{4.339727in}}%
\pgfpathlineto{\pgfqpoint{4.628977in}{4.344049in}}%
\pgfpathlineto{\pgfqpoint{4.632198in}{4.343457in}}%
\pgfpathlineto{\pgfqpoint{4.633272in}{4.349199in}}%
\pgfpathlineto{\pgfqpoint{4.634346in}{4.349081in}}%
\pgfpathlineto{\pgfqpoint{4.635420in}{4.350738in}}%
\pgfpathlineto{\pgfqpoint{4.639715in}{4.349673in}}%
\pgfpathlineto{\pgfqpoint{4.640789in}{4.351745in}}%
\pgfpathlineto{\pgfqpoint{4.641863in}{4.345292in}}%
\pgfpathlineto{\pgfqpoint{4.642937in}{4.347897in}}%
\pgfpathlineto{\pgfqpoint{4.644011in}{4.340141in}}%
\pgfpathlineto{\pgfqpoint{4.647232in}{4.341799in}}%
\pgfpathlineto{\pgfqpoint{4.648306in}{4.339727in}}%
\pgfpathlineto{\pgfqpoint{4.649380in}{4.340733in}}%
\pgfpathlineto{\pgfqpoint{4.650454in}{4.342569in}}%
\pgfpathlineto{\pgfqpoint{4.651527in}{4.342213in}}%
\pgfpathlineto{\pgfqpoint{4.654749in}{4.333688in}}%
\pgfpathlineto{\pgfqpoint{4.655823in}{4.336352in}}%
\pgfpathlineto{\pgfqpoint{4.656897in}{4.333866in}}%
\pgfpathlineto{\pgfqpoint{4.657971in}{4.338780in}}%
\pgfpathlineto{\pgfqpoint{4.659044in}{4.350620in}}%
\pgfpathlineto{\pgfqpoint{4.662266in}{4.347541in}}%
\pgfpathlineto{\pgfqpoint{4.663340in}{4.351804in}}%
\pgfpathlineto{\pgfqpoint{4.664414in}{4.351626in}}%
\pgfpathlineto{\pgfqpoint{4.665487in}{4.355474in}}%
\pgfpathlineto{\pgfqpoint{4.666561in}{4.353402in}}%
\pgfpathlineto{\pgfqpoint{4.669783in}{4.352751in}}%
\pgfpathlineto{\pgfqpoint{4.670857in}{4.356954in}}%
\pgfpathlineto{\pgfqpoint{4.671930in}{4.356244in}}%
\pgfpathlineto{\pgfqpoint{4.674078in}{4.366781in}}%
\pgfpathlineto{\pgfqpoint{4.677300in}{4.365893in}}%
\pgfpathlineto{\pgfqpoint{4.679447in}{4.367255in}}%
\pgfpathlineto{\pgfqpoint{4.684816in}{4.364413in}}%
\pgfpathlineto{\pgfqpoint{4.686964in}{4.380575in}}%
\pgfpathlineto{\pgfqpoint{4.688038in}{4.377615in}}%
\pgfpathlineto{\pgfqpoint{4.689112in}{4.383535in}}%
\pgfpathlineto{\pgfqpoint{4.692333in}{4.389337in}}%
\pgfpathlineto{\pgfqpoint{4.693407in}{4.393185in}}%
\pgfpathlineto{\pgfqpoint{4.694481in}{4.389574in}}%
\pgfpathlineto{\pgfqpoint{4.695555in}{4.390935in}}%
\pgfpathlineto{\pgfqpoint{4.696629in}{4.394073in}}%
\pgfpathlineto{\pgfqpoint{4.700924in}{4.398868in}}%
\pgfpathlineto{\pgfqpoint{4.701998in}{4.397092in}}%
\pgfpathlineto{\pgfqpoint{4.703072in}{4.398750in}}%
\pgfpathlineto{\pgfqpoint{4.704146in}{4.396441in}}%
\pgfpathlineto{\pgfqpoint{4.707367in}{4.400526in}}%
\pgfpathlineto{\pgfqpoint{4.708441in}{4.398335in}}%
\pgfpathlineto{\pgfqpoint{4.709515in}{4.391290in}}%
\pgfpathlineto{\pgfqpoint{4.711662in}{4.408991in}}%
\pgfpathlineto{\pgfqpoint{4.714884in}{4.410294in}}%
\pgfpathlineto{\pgfqpoint{4.717032in}{4.391054in}}%
\pgfpathlineto{\pgfqpoint{4.718105in}{4.393718in}}%
\pgfpathlineto{\pgfqpoint{4.719179in}{4.381641in}}%
\pgfpathlineto{\pgfqpoint{4.722401in}{4.386318in}}%
\pgfpathlineto{\pgfqpoint{4.723475in}{4.392474in}}%
\pgfpathlineto{\pgfqpoint{4.724549in}{4.388449in}}%
\pgfpathlineto{\pgfqpoint{4.725622in}{4.381345in}}%
\pgfpathlineto{\pgfqpoint{4.726696in}{4.383476in}}%
\pgfpathlineto{\pgfqpoint{4.729918in}{4.376431in}}%
\pgfpathlineto{\pgfqpoint{4.733139in}{4.393244in}}%
\pgfpathlineto{\pgfqpoint{4.734213in}{4.391290in}}%
\pgfpathlineto{\pgfqpoint{4.737435in}{4.396026in}}%
\pgfpathlineto{\pgfqpoint{4.738508in}{4.391764in}}%
\pgfpathlineto{\pgfqpoint{4.739582in}{4.391527in}}%
\pgfpathlineto{\pgfqpoint{4.741730in}{4.400940in}}%
\pgfpathlineto{\pgfqpoint{4.744951in}{4.404906in}}%
\pgfpathlineto{\pgfqpoint{4.746025in}{4.407985in}}%
\pgfpathlineto{\pgfqpoint{4.747099in}{4.401058in}}%
\pgfpathlineto{\pgfqpoint{4.748173in}{4.404314in}}%
\pgfpathlineto{\pgfqpoint{4.749247in}{4.411892in}}%
\pgfpathlineto{\pgfqpoint{4.752468in}{4.409287in}}%
\pgfpathlineto{\pgfqpoint{4.753542in}{4.411537in}}%
\pgfpathlineto{\pgfqpoint{4.754616in}{4.403604in}}%
\pgfpathlineto{\pgfqpoint{4.755690in}{4.388449in}}%
\pgfpathlineto{\pgfqpoint{4.756764in}{4.388745in}}%
\pgfpathlineto{\pgfqpoint{4.759985in}{4.392297in}}%
\pgfpathlineto{\pgfqpoint{4.761059in}{4.390639in}}%
\pgfpathlineto{\pgfqpoint{4.762133in}{4.395790in}}%
\pgfpathlineto{\pgfqpoint{4.763207in}{4.397980in}}%
\pgfpathlineto{\pgfqpoint{4.764281in}{4.395671in}}%
\pgfpathlineto{\pgfqpoint{4.767502in}{4.394132in}}%
\pgfpathlineto{\pgfqpoint{4.768576in}{4.394842in}}%
\pgfpathlineto{\pgfqpoint{4.769650in}{4.387146in}}%
\pgfpathlineto{\pgfqpoint{4.770724in}{4.397210in}}%
\pgfpathlineto{\pgfqpoint{4.775019in}{4.398868in}}%
\pgfpathlineto{\pgfqpoint{4.776093in}{4.398217in}}%
\pgfpathlineto{\pgfqpoint{4.777167in}{4.394606in}}%
\pgfpathlineto{\pgfqpoint{4.778240in}{4.400289in}}%
\pgfpathlineto{\pgfqpoint{4.779314in}{4.396559in}}%
\pgfpathlineto{\pgfqpoint{4.782536in}{4.396500in}}%
\pgfpathlineto{\pgfqpoint{4.783610in}{4.400466in}}%
\pgfpathlineto{\pgfqpoint{4.784683in}{4.398572in}}%
\pgfpathlineto{\pgfqpoint{4.785757in}{4.393185in}}%
\pgfpathlineto{\pgfqpoint{4.786831in}{4.394724in}}%
\pgfpathlineto{\pgfqpoint{4.790053in}{4.390225in}}%
\pgfpathlineto{\pgfqpoint{4.791126in}{4.389870in}}%
\pgfpathlineto{\pgfqpoint{4.792200in}{4.385370in}}%
\pgfpathlineto{\pgfqpoint{4.793274in}{4.387620in}}%
\pgfpathlineto{\pgfqpoint{4.794348in}{4.386495in}}%
\pgfpathlineto{\pgfqpoint{4.797570in}{4.386318in}}%
\pgfpathlineto{\pgfqpoint{4.798643in}{4.377082in}}%
\pgfpathlineto{\pgfqpoint{4.800791in}{4.378740in}}%
\pgfpathlineto{\pgfqpoint{4.801865in}{4.377082in}}%
\pgfpathlineto{\pgfqpoint{4.806160in}{4.379865in}}%
\pgfpathlineto{\pgfqpoint{4.808308in}{4.387324in}}%
\pgfpathlineto{\pgfqpoint{4.809382in}{4.384897in}}%
\pgfpathlineto{\pgfqpoint{4.812603in}{4.386554in}}%
\pgfpathlineto{\pgfqpoint{4.814751in}{4.394783in}}%
\pgfpathlineto{\pgfqpoint{4.816899in}{4.395434in}}%
\pgfpathlineto{\pgfqpoint{4.820120in}{4.397092in}}%
\pgfpathlineto{\pgfqpoint{4.822268in}{4.408162in}}%
\pgfpathlineto{\pgfqpoint{4.823342in}{4.407748in}}%
\pgfpathlineto{\pgfqpoint{4.827637in}{4.401650in}}%
\pgfpathlineto{\pgfqpoint{4.828711in}{4.399638in}}%
\pgfpathlineto{\pgfqpoint{4.829785in}{4.399164in}}%
\pgfpathlineto{\pgfqpoint{4.830859in}{4.400999in}}%
\pgfpathlineto{\pgfqpoint{4.831932in}{4.398631in}}%
\pgfpathlineto{\pgfqpoint{4.835154in}{4.396618in}}%
\pgfpathlineto{\pgfqpoint{4.836228in}{4.398868in}}%
\pgfpathlineto{\pgfqpoint{4.838375in}{4.389751in}}%
\pgfpathlineto{\pgfqpoint{4.839449in}{4.391290in}}%
\pgfpathlineto{\pgfqpoint{4.842671in}{4.383831in}}%
\pgfpathlineto{\pgfqpoint{4.843745in}{4.379924in}}%
\pgfpathlineto{\pgfqpoint{4.844818in}{4.379924in}}%
\pgfpathlineto{\pgfqpoint{4.845892in}{4.393185in}}%
\pgfpathlineto{\pgfqpoint{4.846966in}{4.397210in}}%
\pgfpathlineto{\pgfqpoint{4.850188in}{4.400881in}}%
\pgfpathlineto{\pgfqpoint{4.851261in}{4.396441in}}%
\pgfpathlineto{\pgfqpoint{4.852335in}{4.402242in}}%
\pgfpathlineto{\pgfqpoint{4.853409in}{4.423495in}}%
\pgfpathlineto{\pgfqpoint{4.854483in}{4.425094in}}%
\pgfpathlineto{\pgfqpoint{4.857704in}{4.424442in}}%
\pgfpathlineto{\pgfqpoint{4.858778in}{4.426988in}}%
\pgfpathlineto{\pgfqpoint{4.859852in}{4.425686in}}%
\pgfpathlineto{\pgfqpoint{4.860926in}{4.427166in}}%
\pgfpathlineto{\pgfqpoint{4.862000in}{4.436460in}}%
\pgfpathlineto{\pgfqpoint{4.865221in}{4.437230in}}%
\pgfpathlineto{\pgfqpoint{4.866295in}{4.442143in}}%
\pgfpathlineto{\pgfqpoint{4.867369in}{4.439124in}}%
\pgfpathlineto{\pgfqpoint{4.868443in}{4.432020in}}%
\pgfpathlineto{\pgfqpoint{4.869517in}{4.433974in}}%
\pgfpathlineto{\pgfqpoint{4.873812in}{4.432553in}}%
\pgfpathlineto{\pgfqpoint{4.874886in}{4.433796in}}%
\pgfpathlineto{\pgfqpoint{4.875960in}{4.427225in}}%
\pgfpathlineto{\pgfqpoint{4.877034in}{4.431902in}}%
\pgfpathlineto{\pgfqpoint{4.880255in}{4.429889in}}%
\pgfpathlineto{\pgfqpoint{4.881329in}{4.427876in}}%
\pgfpathlineto{\pgfqpoint{4.883477in}{4.432553in}}%
\pgfpathlineto{\pgfqpoint{4.884550in}{4.437467in}}%
\pgfpathlineto{\pgfqpoint{4.887772in}{4.434980in}}%
\pgfpathlineto{\pgfqpoint{4.888846in}{4.435631in}}%
\pgfpathlineto{\pgfqpoint{4.889920in}{4.434507in}}%
\pgfpathlineto{\pgfqpoint{4.890993in}{4.443742in}}%
\pgfpathlineto{\pgfqpoint{4.892067in}{4.443209in}}%
\pgfpathlineto{\pgfqpoint{4.895289in}{4.447235in}}%
\pgfpathlineto{\pgfqpoint{4.897437in}{4.452622in}}%
\pgfpathlineto{\pgfqpoint{4.899584in}{4.453806in}}%
\pgfpathlineto{\pgfqpoint{4.902806in}{4.451023in}}%
\pgfpathlineto{\pgfqpoint{4.903880in}{4.447294in}}%
\pgfpathlineto{\pgfqpoint{4.904953in}{4.446702in}}%
\pgfpathlineto{\pgfqpoint{4.906027in}{4.446939in}}%
\pgfpathlineto{\pgfqpoint{4.907101in}{4.454516in}}%
\pgfpathlineto{\pgfqpoint{4.910323in}{4.453510in}}%
\pgfpathlineto{\pgfqpoint{4.911396in}{4.451023in}}%
\pgfpathlineto{\pgfqpoint{4.912470in}{4.443446in}}%
\pgfpathlineto{\pgfqpoint{4.913544in}{4.440190in}}%
\pgfpathlineto{\pgfqpoint{4.914618in}{4.442262in}}%
\pgfpathlineto{\pgfqpoint{4.917839in}{4.446820in}}%
\pgfpathlineto{\pgfqpoint{4.918913in}{4.444571in}}%
\pgfpathlineto{\pgfqpoint{4.919987in}{4.454812in}}%
\pgfpathlineto{\pgfqpoint{4.921061in}{4.457121in}}%
\pgfpathlineto{\pgfqpoint{4.922135in}{4.463396in}}%
\pgfpathlineto{\pgfqpoint{4.929652in}{4.472217in}}%
\pgfpathlineto{\pgfqpoint{4.932873in}{4.474407in}}%
\pgfpathlineto{\pgfqpoint{4.933947in}{4.480564in}}%
\pgfpathlineto{\pgfqpoint{4.936095in}{4.473638in}}%
\pgfpathlineto{\pgfqpoint{4.937169in}{4.475236in}}%
\pgfpathlineto{\pgfqpoint{4.940390in}{4.474999in}}%
\pgfpathlineto{\pgfqpoint{4.941464in}{4.473164in}}%
\pgfpathlineto{\pgfqpoint{4.942538in}{4.474644in}}%
\pgfpathlineto{\pgfqpoint{4.947907in}{4.461502in}}%
\pgfpathlineto{\pgfqpoint{4.948981in}{4.462212in}}%
\pgfpathlineto{\pgfqpoint{4.950055in}{4.468369in}}%
\pgfpathlineto{\pgfqpoint{4.951128in}{4.465764in}}%
\pgfpathlineto{\pgfqpoint{4.952202in}{4.483169in}}%
\pgfpathlineto{\pgfqpoint{4.956498in}{4.481097in}}%
\pgfpathlineto{\pgfqpoint{4.957571in}{4.484057in}}%
\pgfpathlineto{\pgfqpoint{4.959719in}{4.461265in}}%
\pgfpathlineto{\pgfqpoint{4.962941in}{4.454575in}}%
\pgfpathlineto{\pgfqpoint{4.964014in}{4.459548in}}%
\pgfpathlineto{\pgfqpoint{4.965088in}{4.453628in}}%
\pgfpathlineto{\pgfqpoint{4.966162in}{4.459430in}}%
\pgfpathlineto{\pgfqpoint{4.967236in}{4.450727in}}%
\pgfpathlineto{\pgfqpoint{4.970458in}{4.438947in}}%
\pgfpathlineto{\pgfqpoint{4.971531in}{4.445163in}}%
\pgfpathlineto{\pgfqpoint{4.972605in}{4.443623in}}%
\pgfpathlineto{\pgfqpoint{4.974753in}{4.461502in}}%
\pgfpathlineto{\pgfqpoint{4.980122in}{4.472513in}}%
\pgfpathlineto{\pgfqpoint{4.981196in}{4.471862in}}%
\pgfpathlineto{\pgfqpoint{4.982270in}{4.472454in}}%
\pgfpathlineto{\pgfqpoint{4.986565in}{4.472513in}}%
\pgfpathlineto{\pgfqpoint{4.987639in}{4.471566in}}%
\pgfpathlineto{\pgfqpoint{4.988713in}{4.472513in}}%
\pgfpathlineto{\pgfqpoint{4.989787in}{4.471092in}}%
\pgfpathlineto{\pgfqpoint{4.993008in}{4.477427in}}%
\pgfpathlineto{\pgfqpoint{4.994082in}{4.477368in}}%
\pgfpathlineto{\pgfqpoint{4.995156in}{4.476420in}}%
\pgfpathlineto{\pgfqpoint{4.996230in}{4.479440in}}%
\pgfpathlineto{\pgfqpoint{4.997303in}{4.484886in}}%
\pgfpathlineto{\pgfqpoint{5.000525in}{4.477723in}}%
\pgfpathlineto{\pgfqpoint{5.001599in}{4.492108in}}%
\pgfpathlineto{\pgfqpoint{5.002673in}{4.489563in}}%
\pgfpathlineto{\pgfqpoint{5.003747in}{4.497081in}}%
\pgfpathlineto{\pgfqpoint{5.004820in}{4.498916in}}%
\pgfpathlineto{\pgfqpoint{5.008042in}{4.498028in}}%
\pgfpathlineto{\pgfqpoint{5.010190in}{4.493707in}}%
\pgfpathlineto{\pgfqpoint{5.011263in}{4.481630in}}%
\pgfpathlineto{\pgfqpoint{5.012337in}{4.478788in}}%
\pgfpathlineto{\pgfqpoint{5.016633in}{4.486544in}}%
\pgfpathlineto{\pgfqpoint{5.017706in}{4.481867in}}%
\pgfpathlineto{\pgfqpoint{5.018780in}{4.487076in}}%
\pgfpathlineto{\pgfqpoint{5.024149in}{4.482459in}}%
\pgfpathlineto{\pgfqpoint{5.025223in}{4.476006in}}%
\pgfpathlineto{\pgfqpoint{5.027371in}{4.480387in}}%
\pgfpathlineto{\pgfqpoint{5.030592in}{4.478019in}}%
\pgfpathlineto{\pgfqpoint{5.031666in}{4.484235in}}%
\pgfpathlineto{\pgfqpoint{5.032740in}{4.481334in}}%
\pgfpathlineto{\pgfqpoint{5.033814in}{4.484472in}}%
\pgfpathlineto{\pgfqpoint{5.034888in}{4.474585in}}%
\pgfpathlineto{\pgfqpoint{5.038109in}{4.460555in}}%
\pgfpathlineto{\pgfqpoint{5.039183in}{4.459963in}}%
\pgfpathlineto{\pgfqpoint{5.040257in}{4.472158in}}%
\pgfpathlineto{\pgfqpoint{5.041331in}{4.453747in}}%
\pgfpathlineto{\pgfqpoint{5.042405in}{4.449307in}}%
\pgfpathlineto{\pgfqpoint{5.045626in}{4.454516in}}%
\pgfpathlineto{\pgfqpoint{5.046700in}{4.457476in}}%
\pgfpathlineto{\pgfqpoint{5.047774in}{4.464935in}}%
\pgfpathlineto{\pgfqpoint{5.048848in}{4.458483in}}%
\pgfpathlineto{\pgfqpoint{5.053143in}{4.460910in}}%
\pgfpathlineto{\pgfqpoint{5.054217in}{4.463100in}}%
\pgfpathlineto{\pgfqpoint{5.055291in}{4.463455in}}%
\pgfpathlineto{\pgfqpoint{5.056365in}{4.464995in}}%
\pgfpathlineto{\pgfqpoint{5.057438in}{4.462923in}}%
\pgfpathlineto{\pgfqpoint{5.060660in}{4.463041in}}%
\pgfpathlineto{\pgfqpoint{5.061734in}{4.467067in}}%
\pgfpathlineto{\pgfqpoint{5.062808in}{4.465054in}}%
\pgfpathlineto{\pgfqpoint{5.063881in}{4.461739in}}%
\pgfpathlineto{\pgfqpoint{5.068177in}{4.464699in}}%
\pgfpathlineto{\pgfqpoint{5.069251in}{4.458364in}}%
\pgfpathlineto{\pgfqpoint{5.070325in}{4.468073in}}%
\pgfpathlineto{\pgfqpoint{5.071398in}{4.471566in}}%
\pgfpathlineto{\pgfqpoint{5.075694in}{4.477249in}}%
\pgfpathlineto{\pgfqpoint{5.077841in}{4.470737in}}%
\pgfpathlineto{\pgfqpoint{5.078915in}{4.466001in}}%
\pgfpathlineto{\pgfqpoint{5.079989in}{4.465468in}}%
\pgfpathlineto{\pgfqpoint{5.083211in}{4.468843in}}%
\pgfpathlineto{\pgfqpoint{5.084284in}{4.463219in}}%
\pgfpathlineto{\pgfqpoint{5.085358in}{4.467422in}}%
\pgfpathlineto{\pgfqpoint{5.086432in}{4.469020in}}%
\pgfpathlineto{\pgfqpoint{5.091801in}{4.486721in}}%
\pgfpathlineto{\pgfqpoint{5.092875in}{4.484945in}}%
\pgfpathlineto{\pgfqpoint{5.095023in}{4.487313in}}%
\pgfpathlineto{\pgfqpoint{5.098244in}{4.489444in}}%
\pgfpathlineto{\pgfqpoint{5.099318in}{4.488556in}}%
\pgfpathlineto{\pgfqpoint{5.100392in}{4.489030in}}%
\pgfpathlineto{\pgfqpoint{5.101466in}{4.494417in}}%
\pgfpathlineto{\pgfqpoint{5.102540in}{4.505961in}}%
\pgfpathlineto{\pgfqpoint{5.105761in}{4.509572in}}%
\pgfpathlineto{\pgfqpoint{5.108983in}{4.505132in}}%
\pgfpathlineto{\pgfqpoint{5.110057in}{4.505606in}}%
\pgfpathlineto{\pgfqpoint{5.113278in}{4.503001in}}%
\pgfpathlineto{\pgfqpoint{5.114352in}{4.504600in}}%
\pgfpathlineto{\pgfqpoint{5.115426in}{4.509513in}}%
\pgfpathlineto{\pgfqpoint{5.116500in}{4.506849in}}%
\pgfpathlineto{\pgfqpoint{5.117573in}{4.509395in}}%
\pgfpathlineto{\pgfqpoint{5.120795in}{4.509217in}}%
\pgfpathlineto{\pgfqpoint{5.121869in}{4.503356in}}%
\pgfpathlineto{\pgfqpoint{5.122943in}{4.504481in}}%
\pgfpathlineto{\pgfqpoint{5.124016in}{4.502646in}}%
\pgfpathlineto{\pgfqpoint{5.125090in}{4.506080in}}%
\pgfpathlineto{\pgfqpoint{5.128312in}{4.505724in}}%
\pgfpathlineto{\pgfqpoint{5.129386in}{4.508388in}}%
\pgfpathlineto{\pgfqpoint{5.130459in}{4.507619in}}%
\pgfpathlineto{\pgfqpoint{5.131533in}{4.511052in}}%
\pgfpathlineto{\pgfqpoint{5.135829in}{4.508625in}}%
\pgfpathlineto{\pgfqpoint{5.136903in}{4.504126in}}%
\pgfpathlineto{\pgfqpoint{5.137976in}{4.506435in}}%
\pgfpathlineto{\pgfqpoint{5.139050in}{4.504896in}}%
\pgfpathlineto{\pgfqpoint{5.145493in}{4.504955in}}%
\pgfpathlineto{\pgfqpoint{5.146567in}{4.496844in}}%
\pgfpathlineto{\pgfqpoint{5.147641in}{4.499804in}}%
\pgfpathlineto{\pgfqpoint{5.150862in}{4.496252in}}%
\pgfpathlineto{\pgfqpoint{5.151936in}{4.498976in}}%
\pgfpathlineto{\pgfqpoint{5.154084in}{4.497732in}}%
\pgfpathlineto{\pgfqpoint{5.155158in}{4.491043in}}%
\pgfpathlineto{\pgfqpoint{5.158379in}{4.490688in}}%
\pgfpathlineto{\pgfqpoint{5.159453in}{4.489859in}}%
\pgfpathlineto{\pgfqpoint{5.160527in}{4.485478in}}%
\pgfpathlineto{\pgfqpoint{5.162675in}{4.461147in}}%
\pgfpathlineto{\pgfqpoint{5.165896in}{4.463633in}}%
\pgfpathlineto{\pgfqpoint{5.166970in}{4.460555in}}%
\pgfpathlineto{\pgfqpoint{5.168044in}{4.460732in}}%
\pgfpathlineto{\pgfqpoint{5.169118in}{4.458660in}}%
\pgfpathlineto{\pgfqpoint{5.170191in}{4.466593in}}%
\pgfpathlineto{\pgfqpoint{5.173413in}{4.463751in}}%
\pgfpathlineto{\pgfqpoint{5.174487in}{4.464462in}}%
\pgfpathlineto{\pgfqpoint{5.175561in}{4.466238in}}%
\pgfpathlineto{\pgfqpoint{5.176635in}{4.465527in}}%
\pgfpathlineto{\pgfqpoint{5.177708in}{4.461857in}}%
\pgfpathlineto{\pgfqpoint{5.180930in}{4.464935in}}%
\pgfpathlineto{\pgfqpoint{5.182004in}{4.470204in}}%
\pgfpathlineto{\pgfqpoint{5.183078in}{4.472217in}}%
\pgfpathlineto{\pgfqpoint{5.184151in}{4.475769in}}%
\pgfpathlineto{\pgfqpoint{5.185225in}{4.474348in}}%
\pgfpathlineto{\pgfqpoint{5.188447in}{4.478256in}}%
\pgfpathlineto{\pgfqpoint{5.189521in}{4.475828in}}%
\pgfpathlineto{\pgfqpoint{5.190594in}{4.476302in}}%
\pgfpathlineto{\pgfqpoint{5.191668in}{4.475118in}}%
\pgfpathlineto{\pgfqpoint{5.192742in}{4.478019in}}%
\pgfpathlineto{\pgfqpoint{5.197037in}{4.478966in}}%
\pgfpathlineto{\pgfqpoint{5.198111in}{4.481275in}}%
\pgfpathlineto{\pgfqpoint{5.199185in}{4.478552in}}%
\pgfpathlineto{\pgfqpoint{5.200259in}{4.478315in}}%
\pgfpathlineto{\pgfqpoint{5.203480in}{4.474644in}}%
\pgfpathlineto{\pgfqpoint{5.204554in}{4.468961in}}%
\pgfpathlineto{\pgfqpoint{5.205628in}{4.471743in}}%
\pgfpathlineto{\pgfqpoint{5.206702in}{4.471803in}}%
\pgfpathlineto{\pgfqpoint{5.207776in}{4.467540in}}%
\pgfpathlineto{\pgfqpoint{5.210997in}{4.466119in}}%
\pgfpathlineto{\pgfqpoint{5.214219in}{4.481038in}}%
\pgfpathlineto{\pgfqpoint{5.215293in}{4.478848in}}%
\pgfpathlineto{\pgfqpoint{5.218514in}{4.474940in}}%
\pgfpathlineto{\pgfqpoint{5.219588in}{4.471211in}}%
\pgfpathlineto{\pgfqpoint{5.220662in}{4.472395in}}%
\pgfpathlineto{\pgfqpoint{5.221736in}{4.462745in}}%
\pgfpathlineto{\pgfqpoint{5.222810in}{4.471566in}}%
\pgfpathlineto{\pgfqpoint{5.226031in}{4.469257in}}%
\pgfpathlineto{\pgfqpoint{5.227105in}{4.467007in}}%
\pgfpathlineto{\pgfqpoint{5.228179in}{4.458897in}}%
\pgfpathlineto{\pgfqpoint{5.229253in}{4.459252in}}%
\pgfpathlineto{\pgfqpoint{5.230326in}{4.466356in}}%
\pgfpathlineto{\pgfqpoint{5.233548in}{4.465646in}}%
\pgfpathlineto{\pgfqpoint{5.234622in}{4.456411in}}%
\pgfpathlineto{\pgfqpoint{5.235696in}{4.467718in}}%
\pgfpathlineto{\pgfqpoint{5.237843in}{4.454398in}}%
\pgfpathlineto{\pgfqpoint{5.241065in}{4.442262in}}%
\pgfpathlineto{\pgfqpoint{5.242139in}{4.442025in}}%
\pgfpathlineto{\pgfqpoint{5.243213in}{4.432079in}}%
\pgfpathlineto{\pgfqpoint{5.244286in}{4.428290in}}%
\pgfpathlineto{\pgfqpoint{5.245360in}{4.441137in}}%
\pgfpathlineto{\pgfqpoint{5.248582in}{4.449011in}}%
\pgfpathlineto{\pgfqpoint{5.249656in}{4.458009in}}%
\pgfpathlineto{\pgfqpoint{5.250729in}{4.448774in}}%
\pgfpathlineto{\pgfqpoint{5.251803in}{4.457831in}}%
\pgfpathlineto{\pgfqpoint{5.252877in}{4.462153in}}%
\pgfpathlineto{\pgfqpoint{5.256099in}{4.463455in}}%
\pgfpathlineto{\pgfqpoint{5.257172in}{4.471033in}}%
\pgfpathlineto{\pgfqpoint{5.259320in}{4.474999in}}%
\pgfpathlineto{\pgfqpoint{5.260394in}{4.481689in}}%
\pgfpathlineto{\pgfqpoint{5.263615in}{4.486603in}}%
\pgfpathlineto{\pgfqpoint{5.264689in}{4.489563in}}%
\pgfpathlineto{\pgfqpoint{5.265763in}{4.495246in}}%
\pgfpathlineto{\pgfqpoint{5.266837in}{4.490628in}}%
\pgfpathlineto{\pgfqpoint{5.267911in}{4.494358in}}%
\pgfpathlineto{\pgfqpoint{5.271132in}{4.495128in}}%
\pgfpathlineto{\pgfqpoint{5.272206in}{4.491516in}}%
\pgfpathlineto{\pgfqpoint{5.273280in}{4.490451in}}%
\pgfpathlineto{\pgfqpoint{5.275428in}{4.485656in}}%
\pgfpathlineto{\pgfqpoint{5.278649in}{4.482636in}}%
\pgfpathlineto{\pgfqpoint{5.279723in}{4.485123in}}%
\pgfpathlineto{\pgfqpoint{5.280797in}{4.484708in}}%
\pgfpathlineto{\pgfqpoint{5.281871in}{4.485360in}}%
\pgfpathlineto{\pgfqpoint{5.282945in}{4.484116in}}%
\pgfpathlineto{\pgfqpoint{5.288314in}{4.489918in}}%
\pgfpathlineto{\pgfqpoint{5.290461in}{4.495246in}}%
\pgfpathlineto{\pgfqpoint{5.293683in}{4.493707in}}%
\pgfpathlineto{\pgfqpoint{5.294757in}{4.498443in}}%
\pgfpathlineto{\pgfqpoint{5.295831in}{4.488793in}}%
\pgfpathlineto{\pgfqpoint{5.297978in}{4.496489in}}%
\pgfpathlineto{\pgfqpoint{5.301200in}{4.501521in}}%
\pgfpathlineto{\pgfqpoint{5.302274in}{4.500396in}}%
\pgfpathlineto{\pgfqpoint{5.303347in}{4.497081in}}%
\pgfpathlineto{\pgfqpoint{5.304421in}{4.499153in}}%
\pgfpathlineto{\pgfqpoint{5.305495in}{4.487432in}}%
\pgfpathlineto{\pgfqpoint{5.308717in}{4.482163in}}%
\pgfpathlineto{\pgfqpoint{5.309791in}{4.472276in}}%
\pgfpathlineto{\pgfqpoint{5.311938in}{4.499390in}}%
\pgfpathlineto{\pgfqpoint{5.313012in}{4.497910in}}%
\pgfpathlineto{\pgfqpoint{5.318381in}{4.504363in}}%
\pgfpathlineto{\pgfqpoint{5.320529in}{4.505547in}}%
\pgfpathlineto{\pgfqpoint{5.324824in}{4.505428in}}%
\pgfpathlineto{\pgfqpoint{5.325898in}{4.498680in}}%
\pgfpathlineto{\pgfqpoint{5.328046in}{4.498561in}}%
\pgfpathlineto{\pgfqpoint{5.331267in}{4.485004in}}%
\pgfpathlineto{\pgfqpoint{5.332341in}{4.474407in}}%
\pgfpathlineto{\pgfqpoint{5.334489in}{4.492108in}}%
\pgfpathlineto{\pgfqpoint{5.335563in}{4.485715in}}%
\pgfpathlineto{\pgfqpoint{5.339858in}{4.479144in}}%
\pgfpathlineto{\pgfqpoint{5.342006in}{4.460673in}}%
\pgfpathlineto{\pgfqpoint{5.343079in}{4.461561in}}%
\pgfpathlineto{\pgfqpoint{5.348449in}{4.470559in}}%
\pgfpathlineto{\pgfqpoint{5.349523in}{4.452385in}}%
\pgfpathlineto{\pgfqpoint{5.353818in}{4.446465in}}%
\pgfpathlineto{\pgfqpoint{5.355966in}{4.437822in}}%
\pgfpathlineto{\pgfqpoint{5.357039in}{4.439302in}}%
\pgfpathlineto{\pgfqpoint{5.358113in}{4.432079in}}%
\pgfpathlineto{\pgfqpoint{5.361335in}{4.440012in}}%
\pgfpathlineto{\pgfqpoint{5.362409in}{4.448833in}}%
\pgfpathlineto{\pgfqpoint{5.363482in}{4.448182in}}%
\pgfpathlineto{\pgfqpoint{5.364556in}{4.454339in}}%
\pgfpathlineto{\pgfqpoint{5.365630in}{4.455878in}}%
\pgfpathlineto{\pgfqpoint{5.368852in}{4.455759in}}%
\pgfpathlineto{\pgfqpoint{5.369925in}{4.460495in}}%
\pgfpathlineto{\pgfqpoint{5.370999in}{4.461383in}}%
\pgfpathlineto{\pgfqpoint{5.372073in}{4.430895in}}%
\pgfpathlineto{\pgfqpoint{5.373147in}{4.417694in}}%
\pgfpathlineto{\pgfqpoint{5.377442in}{4.423199in}}%
\pgfpathlineto{\pgfqpoint{5.378516in}{4.427047in}}%
\pgfpathlineto{\pgfqpoint{5.379590in}{4.419470in}}%
\pgfpathlineto{\pgfqpoint{5.380664in}{4.427343in}}%
\pgfpathlineto{\pgfqpoint{5.383885in}{4.429948in}}%
\pgfpathlineto{\pgfqpoint{5.384959in}{4.433027in}}%
\pgfpathlineto{\pgfqpoint{5.387107in}{4.446169in}}%
\pgfpathlineto{\pgfqpoint{5.388181in}{4.437052in}}%
\pgfpathlineto{\pgfqpoint{5.391402in}{4.439479in}}%
\pgfpathlineto{\pgfqpoint{5.392476in}{4.438828in}}%
\pgfpathlineto{\pgfqpoint{5.393550in}{4.431724in}}%
\pgfpathlineto{\pgfqpoint{5.394624in}{4.434625in}}%
\pgfpathlineto{\pgfqpoint{5.395698in}{4.430007in}}%
\pgfpathlineto{\pgfqpoint{5.398919in}{4.431073in}}%
\pgfpathlineto{\pgfqpoint{5.399993in}{4.423377in}}%
\pgfpathlineto{\pgfqpoint{5.401067in}{4.425271in}}%
\pgfpathlineto{\pgfqpoint{5.402141in}{4.436875in}}%
\pgfpathlineto{\pgfqpoint{5.403214in}{4.431606in}}%
\pgfpathlineto{\pgfqpoint{5.406436in}{4.436519in}}%
\pgfpathlineto{\pgfqpoint{5.407510in}{4.434092in}}%
\pgfpathlineto{\pgfqpoint{5.408584in}{4.438532in}}%
\pgfpathlineto{\pgfqpoint{5.409657in}{4.436756in}}%
\pgfpathlineto{\pgfqpoint{5.410731in}{4.443150in}}%
\pgfpathlineto{\pgfqpoint{5.413953in}{4.440427in}}%
\pgfpathlineto{\pgfqpoint{5.416101in}{4.428882in}}%
\pgfpathlineto{\pgfqpoint{5.417174in}{4.419884in}}%
\pgfpathlineto{\pgfqpoint{5.418248in}{4.417102in}}%
\pgfpathlineto{\pgfqpoint{5.421470in}{4.417516in}}%
\pgfpathlineto{\pgfqpoint{5.422544in}{4.419351in}}%
\pgfpathlineto{\pgfqpoint{5.424691in}{4.428113in}}%
\pgfpathlineto{\pgfqpoint{5.428987in}{4.427698in}}%
\pgfpathlineto{\pgfqpoint{5.430060in}{4.420535in}}%
\pgfpathlineto{\pgfqpoint{5.433282in}{4.427462in}}%
\pgfpathlineto{\pgfqpoint{5.436503in}{4.425686in}}%
\pgfpathlineto{\pgfqpoint{5.438651in}{4.428350in}}%
\pgfpathlineto{\pgfqpoint{5.439725in}{4.434803in}}%
\pgfpathlineto{\pgfqpoint{5.440799in}{4.414911in}}%
\pgfpathlineto{\pgfqpoint{5.445094in}{4.414734in}}%
\pgfpathlineto{\pgfqpoint{5.446168in}{4.421127in}}%
\pgfpathlineto{\pgfqpoint{5.448316in}{4.418641in}}%
\pgfpathlineto{\pgfqpoint{5.451537in}{4.415858in}}%
\pgfpathlineto{\pgfqpoint{5.452611in}{4.415858in}}%
\pgfpathlineto{\pgfqpoint{5.453685in}{4.414023in}}%
\pgfpathlineto{\pgfqpoint{5.455833in}{4.416924in}}%
\pgfpathlineto{\pgfqpoint{5.459054in}{4.420239in}}%
\pgfpathlineto{\pgfqpoint{5.460128in}{4.417812in}}%
\pgfpathlineto{\pgfqpoint{5.461202in}{4.417871in}}%
\pgfpathlineto{\pgfqpoint{5.463349in}{4.424087in}}%
\pgfpathlineto{\pgfqpoint{5.466571in}{4.428113in}}%
\pgfpathlineto{\pgfqpoint{5.467645in}{4.424679in}}%
\pgfpathlineto{\pgfqpoint{5.469792in}{4.433915in}}%
\pgfpathlineto{\pgfqpoint{5.470866in}{4.430955in}}%
\pgfpathlineto{\pgfqpoint{5.474088in}{4.430718in}}%
\pgfpathlineto{\pgfqpoint{5.475162in}{4.437289in}}%
\pgfpathlineto{\pgfqpoint{5.477309in}{4.433974in}}%
\pgfpathlineto{\pgfqpoint{5.478383in}{4.436638in}}%
\pgfpathlineto{\pgfqpoint{5.482679in}{4.430777in}}%
\pgfpathlineto{\pgfqpoint{5.484826in}{4.430244in}}%
\pgfpathlineto{\pgfqpoint{5.485900in}{4.428231in}}%
\pgfpathlineto{\pgfqpoint{5.489122in}{4.426810in}}%
\pgfpathlineto{\pgfqpoint{5.491269in}{4.433086in}}%
\pgfpathlineto{\pgfqpoint{5.492343in}{4.426218in}}%
\pgfpathlineto{\pgfqpoint{5.493417in}{4.426396in}}%
\pgfpathlineto{\pgfqpoint{5.496638in}{4.423140in}}%
\pgfpathlineto{\pgfqpoint{5.497712in}{4.425153in}}%
\pgfpathlineto{\pgfqpoint{5.498786in}{4.430777in}}%
\pgfpathlineto{\pgfqpoint{5.499860in}{4.431428in}}%
\pgfpathlineto{\pgfqpoint{5.500934in}{4.427166in}}%
\pgfpathlineto{\pgfqpoint{5.504155in}{4.425626in}}%
\pgfpathlineto{\pgfqpoint{5.505229in}{4.426218in}}%
\pgfpathlineto{\pgfqpoint{5.506303in}{4.431547in}}%
\pgfpathlineto{\pgfqpoint{5.507377in}{4.434033in}}%
\pgfpathlineto{\pgfqpoint{5.508451in}{4.430777in}}%
\pgfpathlineto{\pgfqpoint{5.511672in}{4.436638in}}%
\pgfpathlineto{\pgfqpoint{5.512746in}{4.437289in}}%
\pgfpathlineto{\pgfqpoint{5.514894in}{4.429060in}}%
\pgfpathlineto{\pgfqpoint{5.515967in}{4.429060in}}%
\pgfpathlineto{\pgfqpoint{5.519189in}{4.417575in}}%
\pgfpathlineto{\pgfqpoint{5.520263in}{4.418759in}}%
\pgfpathlineto{\pgfqpoint{5.521337in}{4.422548in}}%
\pgfpathlineto{\pgfqpoint{5.522411in}{4.421482in}}%
\pgfpathlineto{\pgfqpoint{5.527780in}{4.417812in}}%
\pgfpathlineto{\pgfqpoint{5.528854in}{4.408814in}}%
\pgfpathlineto{\pgfqpoint{5.529927in}{4.411063in}}%
\pgfpathlineto{\pgfqpoint{5.531001in}{4.416450in}}%
\pgfpathlineto{\pgfqpoint{5.535297in}{4.425686in}}%
\pgfpathlineto{\pgfqpoint{5.536370in}{4.423318in}}%
\pgfpathlineto{\pgfqpoint{5.538518in}{4.427047in}}%
\pgfpathlineto{\pgfqpoint{5.541740in}{4.427521in}}%
\pgfpathlineto{\pgfqpoint{5.542813in}{4.425567in}}%
\pgfpathlineto{\pgfqpoint{5.543887in}{4.425804in}}%
\pgfpathlineto{\pgfqpoint{5.546035in}{4.408636in}}%
\pgfpathlineto{\pgfqpoint{5.549256in}{4.403190in}}%
\pgfpathlineto{\pgfqpoint{5.550330in}{4.404255in}}%
\pgfpathlineto{\pgfqpoint{5.552478in}{4.409879in}}%
\pgfpathlineto{\pgfqpoint{5.553552in}{4.409524in}}%
\pgfpathlineto{\pgfqpoint{5.556773in}{4.409110in}}%
\pgfpathlineto{\pgfqpoint{5.558921in}{4.406919in}}%
\pgfpathlineto{\pgfqpoint{5.559995in}{4.403663in}}%
\pgfpathlineto{\pgfqpoint{5.561069in}{4.429830in}}%
\pgfpathlineto{\pgfqpoint{5.564290in}{4.438414in}}%
\pgfpathlineto{\pgfqpoint{5.565364in}{4.438828in}}%
\pgfpathlineto{\pgfqpoint{5.567512in}{4.435631in}}%
\pgfpathlineto{\pgfqpoint{5.568586in}{4.436460in}}%
\pgfpathlineto{\pgfqpoint{5.571807in}{4.436934in}}%
\pgfpathlineto{\pgfqpoint{5.572881in}{4.438295in}}%
\pgfpathlineto{\pgfqpoint{5.573955in}{4.436697in}}%
\pgfpathlineto{\pgfqpoint{5.576102in}{4.414911in}}%
\pgfpathlineto{\pgfqpoint{5.580398in}{4.395434in}}%
\pgfpathlineto{\pgfqpoint{5.582545in}{4.414082in}}%
\pgfpathlineto{\pgfqpoint{5.583619in}{4.412780in}}%
\pgfpathlineto{\pgfqpoint{5.586841in}{4.413194in}}%
\pgfpathlineto{\pgfqpoint{5.587915in}{4.398098in}}%
\pgfpathlineto{\pgfqpoint{5.588989in}{4.403308in}}%
\pgfpathlineto{\pgfqpoint{5.590062in}{4.405084in}}%
\pgfpathlineto{\pgfqpoint{5.591136in}{4.398513in}}%
\pgfpathlineto{\pgfqpoint{5.595432in}{4.406386in}}%
\pgfpathlineto{\pgfqpoint{5.596505in}{4.404314in}}%
\pgfpathlineto{\pgfqpoint{5.598653in}{4.406446in}}%
\pgfpathlineto{\pgfqpoint{5.601875in}{4.404433in}}%
\pgfpathlineto{\pgfqpoint{5.604022in}{4.416510in}}%
\pgfpathlineto{\pgfqpoint{5.605096in}{4.415089in}}%
\pgfpathlineto{\pgfqpoint{5.606170in}{4.408932in}}%
\pgfpathlineto{\pgfqpoint{5.609391in}{4.413313in}}%
\pgfpathlineto{\pgfqpoint{5.610465in}{4.407570in}}%
\pgfpathlineto{\pgfqpoint{5.611539in}{4.407156in}}%
\pgfpathlineto{\pgfqpoint{5.612613in}{4.401946in}}%
\pgfpathlineto{\pgfqpoint{5.613687in}{4.404196in}}%
\pgfpathlineto{\pgfqpoint{5.616908in}{4.394428in}}%
\pgfpathlineto{\pgfqpoint{5.617982in}{4.393126in}}%
\pgfpathlineto{\pgfqpoint{5.619056in}{4.398809in}}%
\pgfpathlineto{\pgfqpoint{5.620130in}{4.397506in}}%
\pgfpathlineto{\pgfqpoint{5.621204in}{4.400348in}}%
\pgfpathlineto{\pgfqpoint{5.624425in}{4.416391in}}%
\pgfpathlineto{\pgfqpoint{5.625499in}{4.414260in}}%
\pgfpathlineto{\pgfqpoint{5.626573in}{4.417398in}}%
\pgfpathlineto{\pgfqpoint{5.627647in}{4.417398in}}%
\pgfpathlineto{\pgfqpoint{5.628721in}{4.418226in}}%
\pgfpathlineto{\pgfqpoint{5.631942in}{4.418108in}}%
\pgfpathlineto{\pgfqpoint{5.634090in}{4.411714in}}%
\pgfpathlineto{\pgfqpoint{5.636237in}{4.417516in}}%
\pgfpathlineto{\pgfqpoint{5.640533in}{4.416095in}}%
\pgfpathlineto{\pgfqpoint{5.641607in}{4.413668in}}%
\pgfpathlineto{\pgfqpoint{5.642680in}{4.391290in}}%
\pgfpathlineto{\pgfqpoint{5.643754in}{4.402953in}}%
\pgfpathlineto{\pgfqpoint{5.648050in}{4.399697in}}%
\pgfpathlineto{\pgfqpoint{5.649123in}{4.402183in}}%
\pgfpathlineto{\pgfqpoint{5.650197in}{4.400881in}}%
\pgfpathlineto{\pgfqpoint{5.651271in}{4.395553in}}%
\pgfpathlineto{\pgfqpoint{5.655567in}{4.399815in}}%
\pgfpathlineto{\pgfqpoint{5.656640in}{4.400052in}}%
\pgfpathlineto{\pgfqpoint{5.657714in}{4.399282in}}%
\pgfpathlineto{\pgfqpoint{5.658788in}{4.401295in}}%
\pgfpathlineto{\pgfqpoint{5.662010in}{4.396441in}}%
\pgfpathlineto{\pgfqpoint{5.663083in}{4.396145in}}%
\pgfpathlineto{\pgfqpoint{5.664157in}{4.393599in}}%
\pgfpathlineto{\pgfqpoint{5.666305in}{4.384068in}}%
\pgfpathlineto{\pgfqpoint{5.669526in}{4.386614in}}%
\pgfpathlineto{\pgfqpoint{5.670600in}{4.383535in}}%
\pgfpathlineto{\pgfqpoint{5.672748in}{4.392652in}}%
\pgfpathlineto{\pgfqpoint{5.673822in}{4.390817in}}%
\pgfpathlineto{\pgfqpoint{5.677043in}{4.389870in}}%
\pgfpathlineto{\pgfqpoint{5.678117in}{4.386436in}}%
\pgfpathlineto{\pgfqpoint{5.681339in}{4.387679in}}%
\pgfpathlineto{\pgfqpoint{5.684560in}{4.386495in}}%
\pgfpathlineto{\pgfqpoint{5.685634in}{4.389633in}}%
\pgfpathlineto{\pgfqpoint{5.687782in}{4.379687in}}%
\pgfpathlineto{\pgfqpoint{5.688856in}{4.383476in}}%
\pgfpathlineto{\pgfqpoint{5.692077in}{4.380753in}}%
\pgfpathlineto{\pgfqpoint{5.693151in}{4.376845in}}%
\pgfpathlineto{\pgfqpoint{5.694225in}{4.376549in}}%
\pgfpathlineto{\pgfqpoint{5.695299in}{4.377970in}}%
\pgfpathlineto{\pgfqpoint{5.696372in}{4.370985in}}%
\pgfpathlineto{\pgfqpoint{5.699594in}{4.370866in}}%
\pgfpathlineto{\pgfqpoint{5.700668in}{4.378207in}}%
\pgfpathlineto{\pgfqpoint{5.701742in}{4.381345in}}%
\pgfpathlineto{\pgfqpoint{5.703889in}{4.365479in}}%
\pgfpathlineto{\pgfqpoint{5.707111in}{4.368439in}}%
\pgfpathlineto{\pgfqpoint{5.708185in}{4.370925in}}%
\pgfpathlineto{\pgfqpoint{5.709258in}{4.377201in}}%
\pgfpathlineto{\pgfqpoint{5.710332in}{4.378266in}}%
\pgfpathlineto{\pgfqpoint{5.714628in}{4.376017in}}%
\pgfpathlineto{\pgfqpoint{5.715701in}{4.380397in}}%
\pgfpathlineto{\pgfqpoint{5.716775in}{4.378266in}}%
\pgfpathlineto{\pgfqpoint{5.717849in}{4.374833in}}%
\pgfpathlineto{\pgfqpoint{5.722144in}{4.363940in}}%
\pgfpathlineto{\pgfqpoint{5.723218in}{4.358138in}}%
\pgfpathlineto{\pgfqpoint{5.724292in}{4.347837in}}%
\pgfpathlineto{\pgfqpoint{5.725366in}{4.344581in}}%
\pgfpathlineto{\pgfqpoint{5.726440in}{4.343397in}}%
\pgfpathlineto{\pgfqpoint{5.729661in}{4.345765in}}%
\pgfpathlineto{\pgfqpoint{5.730735in}{4.347719in}}%
\pgfpathlineto{\pgfqpoint{5.731809in}{4.339016in}}%
\pgfpathlineto{\pgfqpoint{5.732883in}{4.341503in}}%
\pgfpathlineto{\pgfqpoint{5.733957in}{4.339372in}}%
\pgfpathlineto{\pgfqpoint{5.738252in}{4.337892in}}%
\pgfpathlineto{\pgfqpoint{5.739326in}{4.340023in}}%
\pgfpathlineto{\pgfqpoint{5.740400in}{4.337892in}}%
\pgfpathlineto{\pgfqpoint{5.741474in}{4.295445in}}%
\pgfpathlineto{\pgfqpoint{5.745769in}{4.295623in}}%
\pgfpathlineto{\pgfqpoint{5.746843in}{4.292426in}}%
\pgfpathlineto{\pgfqpoint{5.747917in}{4.283250in}}%
\pgfpathlineto{\pgfqpoint{5.748990in}{4.286683in}}%
\pgfpathlineto{\pgfqpoint{5.752212in}{4.293432in}}%
\pgfpathlineto{\pgfqpoint{5.753286in}{4.287631in}}%
\pgfpathlineto{\pgfqpoint{5.755433in}{4.291656in}}%
\pgfpathlineto{\pgfqpoint{5.756507in}{4.289407in}}%
\pgfpathlineto{\pgfqpoint{5.759729in}{4.280527in}}%
\pgfpathlineto{\pgfqpoint{5.760803in}{4.281829in}}%
\pgfpathlineto{\pgfqpoint{5.761877in}{4.279935in}}%
\pgfpathlineto{\pgfqpoint{5.762950in}{4.273304in}}%
\pgfpathlineto{\pgfqpoint{5.764024in}{4.282007in}}%
\pgfpathlineto{\pgfqpoint{5.768320in}{4.284907in}}%
\pgfpathlineto{\pgfqpoint{5.774763in}{4.298642in}}%
\pgfpathlineto{\pgfqpoint{5.776910in}{4.293077in}}%
\pgfpathlineto{\pgfqpoint{5.777984in}{4.297280in}}%
\pgfpathlineto{\pgfqpoint{5.779058in}{4.297221in}}%
\pgfpathlineto{\pgfqpoint{5.782279in}{4.298346in}}%
\pgfpathlineto{\pgfqpoint{5.783353in}{4.305154in}}%
\pgfpathlineto{\pgfqpoint{5.784427in}{4.306989in}}%
\pgfpathlineto{\pgfqpoint{5.785501in}{4.312376in}}%
\pgfpathlineto{\pgfqpoint{5.789796in}{4.317468in}}%
\pgfpathlineto{\pgfqpoint{5.790870in}{4.319895in}}%
\pgfpathlineto{\pgfqpoint{5.793018in}{4.316106in}}%
\pgfpathlineto{\pgfqpoint{5.794092in}{4.320072in}}%
\pgfpathlineto{\pgfqpoint{5.797313in}{4.320664in}}%
\pgfpathlineto{\pgfqpoint{5.798387in}{4.318770in}}%
\pgfpathlineto{\pgfqpoint{5.800535in}{4.323565in}}%
\pgfpathlineto{\pgfqpoint{5.801609in}{4.329900in}}%
\pgfpathlineto{\pgfqpoint{5.804830in}{4.329840in}}%
\pgfpathlineto{\pgfqpoint{5.805904in}{4.326525in}}%
\pgfpathlineto{\pgfqpoint{5.806978in}{4.326644in}}%
\pgfpathlineto{\pgfqpoint{5.808052in}{4.325696in}}%
\pgfpathlineto{\pgfqpoint{5.812347in}{4.324631in}}%
\pgfpathlineto{\pgfqpoint{5.813421in}{4.326466in}}%
\pgfpathlineto{\pgfqpoint{5.814495in}{4.324690in}}%
\pgfpathlineto{\pgfqpoint{5.815568in}{4.330906in}}%
\pgfpathlineto{\pgfqpoint{5.816642in}{4.329248in}}%
\pgfpathlineto{\pgfqpoint{5.819864in}{4.326584in}}%
\pgfpathlineto{\pgfqpoint{5.820938in}{4.324157in}}%
\pgfpathlineto{\pgfqpoint{5.822011in}{4.324039in}}%
\pgfpathlineto{\pgfqpoint{5.823085in}{4.318296in}}%
\pgfpathlineto{\pgfqpoint{5.824159in}{4.321908in}}%
\pgfpathlineto{\pgfqpoint{5.827381in}{4.323565in}}%
\pgfpathlineto{\pgfqpoint{5.828455in}{4.328538in}}%
\pgfpathlineto{\pgfqpoint{5.829528in}{4.336885in}}%
\pgfpathlineto{\pgfqpoint{5.830602in}{4.338839in}}%
\pgfpathlineto{\pgfqpoint{5.831676in}{4.336767in}}%
\pgfpathlineto{\pgfqpoint{5.834898in}{4.339312in}}%
\pgfpathlineto{\pgfqpoint{5.838119in}{4.356244in}}%
\pgfpathlineto{\pgfqpoint{5.839193in}{4.358079in}}%
\pgfpathlineto{\pgfqpoint{5.842414in}{4.356777in}}%
\pgfpathlineto{\pgfqpoint{5.843488in}{4.359618in}}%
\pgfpathlineto{\pgfqpoint{5.844562in}{4.359737in}}%
\pgfpathlineto{\pgfqpoint{5.846710in}{4.355237in}}%
\pgfpathlineto{\pgfqpoint{5.849931in}{4.356658in}}%
\pgfpathlineto{\pgfqpoint{5.852079in}{4.348666in}}%
\pgfpathlineto{\pgfqpoint{5.853153in}{4.346772in}}%
\pgfpathlineto{\pgfqpoint{5.854227in}{4.350087in}}%
\pgfpathlineto{\pgfqpoint{5.857448in}{4.347127in}}%
\pgfpathlineto{\pgfqpoint{5.858522in}{4.351922in}}%
\pgfpathlineto{\pgfqpoint{5.861744in}{4.347897in}}%
\pgfpathlineto{\pgfqpoint{5.864965in}{4.347601in}}%
\pgfpathlineto{\pgfqpoint{5.866039in}{4.340260in}}%
\pgfpathlineto{\pgfqpoint{5.867113in}{4.344522in}}%
\pgfpathlineto{\pgfqpoint{5.868187in}{4.340141in}}%
\pgfpathlineto{\pgfqpoint{5.869260in}{4.346772in}}%
\pgfpathlineto{\pgfqpoint{5.872482in}{4.344937in}}%
\pgfpathlineto{\pgfqpoint{5.873556in}{4.352100in}}%
\pgfpathlineto{\pgfqpoint{5.874630in}{4.354586in}}%
\pgfpathlineto{\pgfqpoint{5.875703in}{4.354113in}}%
\pgfpathlineto{\pgfqpoint{5.876777in}{4.355770in}}%
\pgfpathlineto{\pgfqpoint{5.882146in}{4.357961in}}%
\pgfpathlineto{\pgfqpoint{5.883220in}{4.360743in}}%
\pgfpathlineto{\pgfqpoint{5.884294in}{4.355593in}}%
\pgfpathlineto{\pgfqpoint{5.887516in}{4.358138in}}%
\pgfpathlineto{\pgfqpoint{5.888589in}{4.357842in}}%
\pgfpathlineto{\pgfqpoint{5.889663in}{4.359914in}}%
\pgfpathlineto{\pgfqpoint{5.895032in}{4.345351in}}%
\pgfpathlineto{\pgfqpoint{5.896106in}{4.330728in}}%
\pgfpathlineto{\pgfqpoint{5.898254in}{4.335583in}}%
\pgfpathlineto{\pgfqpoint{5.899328in}{4.335168in}}%
\pgfpathlineto{\pgfqpoint{5.902549in}{4.337655in}}%
\pgfpathlineto{\pgfqpoint{5.903623in}{4.337596in}}%
\pgfpathlineto{\pgfqpoint{5.904697in}{4.335701in}}%
\pgfpathlineto{\pgfqpoint{5.905771in}{4.342983in}}%
\pgfpathlineto{\pgfqpoint{5.906845in}{4.325045in}}%
\pgfpathlineto{\pgfqpoint{5.910066in}{4.311607in}}%
\pgfpathlineto{\pgfqpoint{5.911140in}{4.312909in}}%
\pgfpathlineto{\pgfqpoint{5.913288in}{4.330728in}}%
\pgfpathlineto{\pgfqpoint{5.914362in}{4.330314in}}%
\pgfpathlineto{\pgfqpoint{5.918657in}{4.321612in}}%
\pgfpathlineto{\pgfqpoint{5.920805in}{4.325341in}}%
\pgfpathlineto{\pgfqpoint{5.921878in}{4.334813in}}%
\pgfpathlineto{\pgfqpoint{5.925100in}{4.338602in}}%
\pgfpathlineto{\pgfqpoint{5.926174in}{4.343397in}}%
\pgfpathlineto{\pgfqpoint{5.927248in}{4.343930in}}%
\pgfpathlineto{\pgfqpoint{5.928321in}{4.346831in}}%
\pgfpathlineto{\pgfqpoint{5.929395in}{4.347778in}}%
\pgfpathlineto{\pgfqpoint{5.933691in}{4.349909in}}%
\pgfpathlineto{\pgfqpoint{5.934765in}{4.351745in}}%
\pgfpathlineto{\pgfqpoint{5.935838in}{4.345825in}}%
\pgfpathlineto{\pgfqpoint{5.936912in}{4.350620in}}%
\pgfpathlineto{\pgfqpoint{5.941208in}{4.351093in}}%
\pgfpathlineto{\pgfqpoint{5.943355in}{4.353521in}}%
\pgfpathlineto{\pgfqpoint{5.944429in}{4.351626in}}%
\pgfpathlineto{\pgfqpoint{5.947651in}{4.349791in}}%
\pgfpathlineto{\pgfqpoint{5.948724in}{4.346061in}}%
\pgfpathlineto{\pgfqpoint{5.949798in}{4.348074in}}%
\pgfpathlineto{\pgfqpoint{5.950872in}{4.348666in}}%
\pgfpathlineto{\pgfqpoint{5.951946in}{4.357605in}}%
\pgfpathlineto{\pgfqpoint{5.955167in}{4.359381in}}%
\pgfpathlineto{\pgfqpoint{5.957315in}{4.353225in}}%
\pgfpathlineto{\pgfqpoint{5.958389in}{4.357369in}}%
\pgfpathlineto{\pgfqpoint{5.959463in}{4.356836in}}%
\pgfpathlineto{\pgfqpoint{5.962684in}{4.358257in}}%
\pgfpathlineto{\pgfqpoint{5.963758in}{4.356362in}}%
\pgfpathlineto{\pgfqpoint{5.964832in}{4.358553in}}%
\pgfpathlineto{\pgfqpoint{5.970201in}{4.356717in}}%
\pgfpathlineto{\pgfqpoint{5.971275in}{4.358553in}}%
\pgfpathlineto{\pgfqpoint{5.972349in}{4.355593in}}%
\pgfpathlineto{\pgfqpoint{5.974497in}{4.353521in}}%
\pgfpathlineto{\pgfqpoint{5.977718in}{4.357605in}}%
\pgfpathlineto{\pgfqpoint{5.978792in}{4.357250in}}%
\pgfpathlineto{\pgfqpoint{5.979866in}{4.357961in}}%
\pgfpathlineto{\pgfqpoint{5.980940in}{4.353876in}}%
\pgfpathlineto{\pgfqpoint{5.982013in}{4.355770in}}%
\pgfpathlineto{\pgfqpoint{5.986309in}{4.358967in}}%
\pgfpathlineto{\pgfqpoint{5.987383in}{4.361394in}}%
\pgfpathlineto{\pgfqpoint{5.988456in}{4.361690in}}%
\pgfpathlineto{\pgfqpoint{5.989530in}{4.355356in}}%
\pgfpathlineto{\pgfqpoint{5.992752in}{4.359796in}}%
\pgfpathlineto{\pgfqpoint{5.994899in}{4.346061in}}%
\pgfpathlineto{\pgfqpoint{5.995973in}{4.348074in}}%
\pgfpathlineto{\pgfqpoint{5.997047in}{4.347127in}}%
\pgfpathlineto{\pgfqpoint{6.000269in}{4.349377in}}%
\pgfpathlineto{\pgfqpoint{6.001343in}{4.347423in}}%
\pgfpathlineto{\pgfqpoint{6.003490in}{4.352573in}}%
\pgfpathlineto{\pgfqpoint{6.004564in}{4.348193in}}%
\pgfpathlineto{\pgfqpoint{6.007786in}{4.345765in}}%
\pgfpathlineto{\pgfqpoint{6.008859in}{4.350620in}}%
\pgfpathlineto{\pgfqpoint{6.009933in}{4.350265in}}%
\pgfpathlineto{\pgfqpoint{6.011007in}{4.345469in}}%
\pgfpathlineto{\pgfqpoint{6.012081in}{4.349258in}}%
\pgfpathlineto{\pgfqpoint{6.015302in}{4.347956in}}%
\pgfpathlineto{\pgfqpoint{6.016376in}{4.348548in}}%
\pgfpathlineto{\pgfqpoint{6.017450in}{4.352869in}}%
\pgfpathlineto{\pgfqpoint{6.018524in}{4.339135in}}%
\pgfpathlineto{\pgfqpoint{6.019598in}{4.338128in}}%
\pgfpathlineto{\pgfqpoint{6.022819in}{4.338898in}}%
\pgfpathlineto{\pgfqpoint{6.023893in}{4.333037in}}%
\pgfpathlineto{\pgfqpoint{6.026041in}{4.330432in}}%
\pgfpathlineto{\pgfqpoint{6.027115in}{4.328952in}}%
\pgfpathlineto{\pgfqpoint{6.030336in}{4.327532in}}%
\pgfpathlineto{\pgfqpoint{6.031410in}{4.328597in}}%
\pgfpathlineto{\pgfqpoint{6.032484in}{4.335228in}}%
\pgfpathlineto{\pgfqpoint{6.033558in}{4.366663in}}%
\pgfpathlineto{\pgfqpoint{6.034632in}{4.369919in}}%
\pgfpathlineto{\pgfqpoint{6.037853in}{4.368380in}}%
\pgfpathlineto{\pgfqpoint{6.038927in}{4.366426in}}%
\pgfpathlineto{\pgfqpoint{6.041075in}{4.367492in}}%
\pgfpathlineto{\pgfqpoint{6.042148in}{4.364769in}}%
\pgfpathlineto{\pgfqpoint{6.045370in}{4.364591in}}%
\pgfpathlineto{\pgfqpoint{6.046444in}{4.363644in}}%
\pgfpathlineto{\pgfqpoint{6.047518in}{4.359085in}}%
\pgfpathlineto{\pgfqpoint{6.048591in}{4.358434in}}%
\pgfpathlineto{\pgfqpoint{6.049665in}{4.359441in}}%
\pgfpathlineto{\pgfqpoint{6.052887in}{4.367906in}}%
\pgfpathlineto{\pgfqpoint{6.053961in}{4.368261in}}%
\pgfpathlineto{\pgfqpoint{6.056108in}{4.385548in}}%
\pgfpathlineto{\pgfqpoint{6.057182in}{4.387798in}}%
\pgfpathlineto{\pgfqpoint{6.060404in}{4.398690in}}%
\pgfpathlineto{\pgfqpoint{6.061477in}{4.398986in}}%
\pgfpathlineto{\pgfqpoint{6.062551in}{4.394487in}}%
\pgfpathlineto{\pgfqpoint{6.063625in}{4.395020in}}%
\pgfpathlineto{\pgfqpoint{6.064699in}{4.390639in}}%
\pgfpathlineto{\pgfqpoint{6.068994in}{4.394724in}}%
\pgfpathlineto{\pgfqpoint{6.070068in}{4.401295in}}%
\pgfpathlineto{\pgfqpoint{6.072216in}{4.401177in}}%
\pgfpathlineto{\pgfqpoint{6.075437in}{4.397033in}}%
\pgfpathlineto{\pgfqpoint{6.076511in}{4.393422in}}%
\pgfpathlineto{\pgfqpoint{6.078659in}{4.399282in}}%
\pgfpathlineto{\pgfqpoint{6.079733in}{4.395494in}}%
\pgfpathlineto{\pgfqpoint{6.082954in}{4.396441in}}%
\pgfpathlineto{\pgfqpoint{6.084028in}{4.398039in}}%
\pgfpathlineto{\pgfqpoint{6.085102in}{4.409287in}}%
\pgfpathlineto{\pgfqpoint{6.086176in}{4.412839in}}%
\pgfpathlineto{\pgfqpoint{6.087250in}{4.412010in}}%
\pgfpathlineto{\pgfqpoint{6.090471in}{4.405262in}}%
\pgfpathlineto{\pgfqpoint{6.092619in}{4.408044in}}%
\pgfpathlineto{\pgfqpoint{6.093693in}{4.413017in}}%
\pgfpathlineto{\pgfqpoint{6.094766in}{4.413313in}}%
\pgfpathlineto{\pgfqpoint{6.097988in}{4.410767in}}%
\pgfpathlineto{\pgfqpoint{6.100136in}{4.415148in}}%
\pgfpathlineto{\pgfqpoint{6.101209in}{4.410945in}}%
\pgfpathlineto{\pgfqpoint{6.102283in}{4.413194in}}%
\pgfpathlineto{\pgfqpoint{6.106579in}{4.413254in}}%
\pgfpathlineto{\pgfqpoint{6.108726in}{4.407215in}}%
\pgfpathlineto{\pgfqpoint{6.109800in}{4.408103in}}%
\pgfpathlineto{\pgfqpoint{6.114096in}{4.415326in}}%
\pgfpathlineto{\pgfqpoint{6.115169in}{4.422370in}}%
\pgfpathlineto{\pgfqpoint{6.116243in}{4.416983in}}%
\pgfpathlineto{\pgfqpoint{6.120539in}{4.420062in}}%
\pgfpathlineto{\pgfqpoint{6.121612in}{4.424561in}}%
\pgfpathlineto{\pgfqpoint{6.122686in}{4.426041in}}%
\pgfpathlineto{\pgfqpoint{6.123760in}{4.425863in}}%
\pgfpathlineto{\pgfqpoint{6.124834in}{4.424383in}}%
\pgfpathlineto{\pgfqpoint{6.129129in}{4.424265in}}%
\pgfpathlineto{\pgfqpoint{6.130203in}{4.429356in}}%
\pgfpathlineto{\pgfqpoint{6.132351in}{4.422015in}}%
\pgfpathlineto{\pgfqpoint{6.135572in}{4.420713in}}%
\pgfpathlineto{\pgfqpoint{6.136646in}{4.429001in}}%
\pgfpathlineto{\pgfqpoint{6.137720in}{4.425922in}}%
\pgfpathlineto{\pgfqpoint{6.138794in}{4.426159in}}%
\pgfpathlineto{\pgfqpoint{6.139868in}{4.425686in}}%
\pgfpathlineto{\pgfqpoint{6.143089in}{4.428172in}}%
\pgfpathlineto{\pgfqpoint{6.144163in}{4.423022in}}%
\pgfpathlineto{\pgfqpoint{6.145237in}{4.425212in}}%
\pgfpathlineto{\pgfqpoint{6.146311in}{4.423791in}}%
\pgfpathlineto{\pgfqpoint{6.147385in}{4.432494in}}%
\pgfpathlineto{\pgfqpoint{6.151680in}{4.430659in}}%
\pgfpathlineto{\pgfqpoint{6.152754in}{4.431132in}}%
\pgfpathlineto{\pgfqpoint{6.154901in}{4.435039in}}%
\pgfpathlineto{\pgfqpoint{6.158123in}{4.437467in}}%
\pgfpathlineto{\pgfqpoint{6.159197in}{4.440308in}}%
\pgfpathlineto{\pgfqpoint{6.160271in}{4.441433in}}%
\pgfpathlineto{\pgfqpoint{6.161344in}{4.440900in}}%
\pgfpathlineto{\pgfqpoint{6.162418in}{4.442025in}}%
\pgfpathlineto{\pgfqpoint{6.166714in}{4.443505in}}%
\pgfpathlineto{\pgfqpoint{6.167787in}{4.442913in}}%
\pgfpathlineto{\pgfqpoint{6.168861in}{4.443979in}}%
\pgfpathlineto{\pgfqpoint{6.169935in}{4.442321in}}%
\pgfpathlineto{\pgfqpoint{6.173157in}{4.444630in}}%
\pgfpathlineto{\pgfqpoint{6.174231in}{4.444038in}}%
\pgfpathlineto{\pgfqpoint{6.175304in}{4.454635in}}%
\pgfpathlineto{\pgfqpoint{6.176378in}{4.444275in}}%
\pgfpathlineto{\pgfqpoint{6.177452in}{4.443031in}}%
\pgfpathlineto{\pgfqpoint{6.180674in}{4.440841in}}%
\pgfpathlineto{\pgfqpoint{6.181747in}{4.441315in}}%
\pgfpathlineto{\pgfqpoint{6.182821in}{4.438236in}}%
\pgfpathlineto{\pgfqpoint{6.183895in}{4.439716in}}%
\pgfpathlineto{\pgfqpoint{6.184969in}{4.440131in}}%
\pgfpathlineto{\pgfqpoint{6.188190in}{4.439183in}}%
\pgfpathlineto{\pgfqpoint{6.189264in}{4.441907in}}%
\pgfpathlineto{\pgfqpoint{6.190338in}{4.439302in}}%
\pgfpathlineto{\pgfqpoint{6.191412in}{4.442380in}}%
\pgfpathlineto{\pgfqpoint{6.192486in}{4.439420in}}%
\pgfpathlineto{\pgfqpoint{6.195707in}{4.437111in}}%
\pgfpathlineto{\pgfqpoint{6.196781in}{4.429593in}}%
\pgfpathlineto{\pgfqpoint{6.198929in}{4.431369in}}%
\pgfpathlineto{\pgfqpoint{6.200003in}{4.433441in}}%
\pgfpathlineto{\pgfqpoint{6.203224in}{4.430007in}}%
\pgfpathlineto{\pgfqpoint{6.204298in}{4.435927in}}%
\pgfpathlineto{\pgfqpoint{6.205372in}{4.433678in}}%
\pgfpathlineto{\pgfqpoint{6.206446in}{4.439183in}}%
\pgfpathlineto{\pgfqpoint{6.207520in}{4.438591in}}%
\pgfpathlineto{\pgfqpoint{6.210741in}{4.435631in}}%
\pgfpathlineto{\pgfqpoint{6.212889in}{4.432731in}}%
\pgfpathlineto{\pgfqpoint{6.213963in}{4.433619in}}%
\pgfpathlineto{\pgfqpoint{6.215036in}{4.432790in}}%
\pgfpathlineto{\pgfqpoint{6.218258in}{4.431191in}}%
\pgfpathlineto{\pgfqpoint{6.219332in}{4.429889in}}%
\pgfpathlineto{\pgfqpoint{6.221479in}{4.421482in}}%
\pgfpathlineto{\pgfqpoint{6.225775in}{4.426455in}}%
\pgfpathlineto{\pgfqpoint{6.226849in}{4.421423in}}%
\pgfpathlineto{\pgfqpoint{6.227922in}{4.420062in}}%
\pgfpathlineto{\pgfqpoint{6.228996in}{4.445695in}}%
\pgfpathlineto{\pgfqpoint{6.230070in}{4.443209in}}%
\pgfpathlineto{\pgfqpoint{6.234365in}{4.449188in}}%
\pgfpathlineto{\pgfqpoint{6.236513in}{4.447471in}}%
\pgfpathlineto{\pgfqpoint{6.237587in}{4.441255in}}%
\pgfpathlineto{\pgfqpoint{6.240809in}{4.441137in}}%
\pgfpathlineto{\pgfqpoint{6.241882in}{4.442913in}}%
\pgfpathlineto{\pgfqpoint{6.244030in}{4.435987in}}%
\pgfpathlineto{\pgfqpoint{6.245104in}{4.435927in}}%
\pgfpathlineto{\pgfqpoint{6.248325in}{4.435039in}}%
\pgfpathlineto{\pgfqpoint{6.250473in}{4.437822in}}%
\pgfpathlineto{\pgfqpoint{6.252621in}{4.431191in}}%
\pgfpathlineto{\pgfqpoint{6.255842in}{4.435987in}}%
\pgfpathlineto{\pgfqpoint{6.256916in}{4.434862in}}%
\pgfpathlineto{\pgfqpoint{6.257990in}{4.424738in}}%
\pgfpathlineto{\pgfqpoint{6.259064in}{4.424798in}}%
\pgfpathlineto{\pgfqpoint{6.260138in}{4.427225in}}%
\pgfpathlineto{\pgfqpoint{6.263359in}{4.428231in}}%
\pgfpathlineto{\pgfqpoint{6.264433in}{4.429534in}}%
\pgfpathlineto{\pgfqpoint{6.265507in}{4.429060in}}%
\pgfpathlineto{\pgfqpoint{6.266581in}{4.430895in}}%
\pgfpathlineto{\pgfqpoint{6.267654in}{4.431014in}}%
\pgfpathlineto{\pgfqpoint{6.273024in}{4.428054in}}%
\pgfpathlineto{\pgfqpoint{6.274097in}{4.435691in}}%
\pgfpathlineto{\pgfqpoint{6.275171in}{4.436934in}}%
\pgfpathlineto{\pgfqpoint{6.278393in}{4.439657in}}%
\pgfpathlineto{\pgfqpoint{6.279467in}{4.438947in}}%
\pgfpathlineto{\pgfqpoint{6.280541in}{4.444452in}}%
\pgfpathlineto{\pgfqpoint{6.281614in}{4.445281in}}%
\pgfpathlineto{\pgfqpoint{6.282688in}{4.447353in}}%
\pgfpathlineto{\pgfqpoint{6.285910in}{4.446524in}}%
\pgfpathlineto{\pgfqpoint{6.288057in}{4.450372in}}%
\pgfpathlineto{\pgfqpoint{6.289131in}{4.449543in}}%
\pgfpathlineto{\pgfqpoint{6.290205in}{4.453865in}}%
\pgfpathlineto{\pgfqpoint{6.293427in}{4.456351in}}%
\pgfpathlineto{\pgfqpoint{6.294500in}{4.459963in}}%
\pgfpathlineto{\pgfqpoint{6.295574in}{4.458246in}}%
\pgfpathlineto{\pgfqpoint{6.297722in}{4.458305in}}%
\pgfpathlineto{\pgfqpoint{6.302017in}{4.463219in}}%
\pgfpathlineto{\pgfqpoint{6.303091in}{4.468310in}}%
\pgfpathlineto{\pgfqpoint{6.304165in}{4.466356in}}%
\pgfpathlineto{\pgfqpoint{6.305239in}{4.469849in}}%
\pgfpathlineto{\pgfqpoint{6.308460in}{4.474763in}}%
\pgfpathlineto{\pgfqpoint{6.310608in}{4.475414in}}%
\pgfpathlineto{\pgfqpoint{6.311682in}{4.468783in}}%
\pgfpathlineto{\pgfqpoint{6.312756in}{4.472217in}}%
\pgfpathlineto{\pgfqpoint{6.315977in}{4.472039in}}%
\pgfpathlineto{\pgfqpoint{6.317051in}{4.471211in}}%
\pgfpathlineto{\pgfqpoint{6.319199in}{4.478196in}}%
\pgfpathlineto{\pgfqpoint{6.320273in}{4.477664in}}%
\pgfpathlineto{\pgfqpoint{6.323494in}{4.477308in}}%
\pgfpathlineto{\pgfqpoint{6.325642in}{4.481393in}}%
\pgfpathlineto{\pgfqpoint{6.326716in}{4.478078in}}%
\pgfpathlineto{\pgfqpoint{6.327789in}{4.479440in}}%
\pgfpathlineto{\pgfqpoint{6.331011in}{4.476065in}}%
\pgfpathlineto{\pgfqpoint{6.332085in}{4.478374in}}%
\pgfpathlineto{\pgfqpoint{6.333159in}{4.477782in}}%
\pgfpathlineto{\pgfqpoint{6.334232in}{4.469435in}}%
\pgfpathlineto{\pgfqpoint{6.335306in}{4.474881in}}%
\pgfpathlineto{\pgfqpoint{6.338528in}{4.477368in}}%
\pgfpathlineto{\pgfqpoint{6.340675in}{4.477782in}}%
\pgfpathlineto{\pgfqpoint{6.341749in}{4.479144in}}%
\pgfpathlineto{\pgfqpoint{6.342823in}{4.481630in}}%
\pgfpathlineto{\pgfqpoint{6.346045in}{4.480920in}}%
\pgfpathlineto{\pgfqpoint{6.347119in}{4.481452in}}%
\pgfpathlineto{\pgfqpoint{6.348192in}{4.480032in}}%
\pgfpathlineto{\pgfqpoint{6.349266in}{4.473519in}}%
\pgfpathlineto{\pgfqpoint{6.350340in}{4.471980in}}%
\pgfpathlineto{\pgfqpoint{6.353562in}{4.478729in}}%
\pgfpathlineto{\pgfqpoint{6.354635in}{4.486366in}}%
\pgfpathlineto{\pgfqpoint{6.355709in}{4.489800in}}%
\pgfpathlineto{\pgfqpoint{6.357857in}{4.478256in}}%
\pgfpathlineto{\pgfqpoint{6.363226in}{4.477664in}}%
\pgfpathlineto{\pgfqpoint{6.365374in}{4.478729in}}%
\pgfpathlineto{\pgfqpoint{6.369669in}{4.478433in}}%
\pgfpathlineto{\pgfqpoint{6.372891in}{4.482577in}}%
\pgfpathlineto{\pgfqpoint{6.378260in}{4.477368in}}%
\pgfpathlineto{\pgfqpoint{6.379334in}{4.472631in}}%
\pgfpathlineto{\pgfqpoint{6.380408in}{4.471743in}}%
\pgfpathlineto{\pgfqpoint{6.383629in}{4.480032in}}%
\pgfpathlineto{\pgfqpoint{6.384703in}{4.484945in}}%
\pgfpathlineto{\pgfqpoint{6.385777in}{4.485478in}}%
\pgfpathlineto{\pgfqpoint{6.386851in}{4.482873in}}%
\pgfpathlineto{\pgfqpoint{6.387924in}{4.487491in}}%
\pgfpathlineto{\pgfqpoint{6.391146in}{4.492523in}}%
\pgfpathlineto{\pgfqpoint{6.392220in}{4.499035in}}%
\pgfpathlineto{\pgfqpoint{6.393294in}{4.495779in}}%
\pgfpathlineto{\pgfqpoint{6.395441in}{4.495601in}}%
\pgfpathlineto{\pgfqpoint{6.398663in}{4.494476in}}%
\pgfpathlineto{\pgfqpoint{6.399737in}{4.497259in}}%
\pgfpathlineto{\pgfqpoint{6.401884in}{4.505606in}}%
\pgfpathlineto{\pgfqpoint{6.402958in}{4.507441in}}%
\pgfpathlineto{\pgfqpoint{6.406180in}{4.507915in}}%
\pgfpathlineto{\pgfqpoint{6.407253in}{4.513006in}}%
\pgfpathlineto{\pgfqpoint{6.408327in}{4.510579in}}%
\pgfpathlineto{\pgfqpoint{6.410475in}{4.515729in}}%
\pgfpathlineto{\pgfqpoint{6.413697in}{4.516558in}}%
\pgfpathlineto{\pgfqpoint{6.415844in}{4.518097in}}%
\pgfpathlineto{\pgfqpoint{6.416918in}{4.516084in}}%
\pgfpathlineto{\pgfqpoint{6.417992in}{4.523307in}}%
\pgfpathlineto{\pgfqpoint{6.422287in}{4.516558in}}%
\pgfpathlineto{\pgfqpoint{6.423361in}{4.518808in}}%
\pgfpathlineto{\pgfqpoint{6.424435in}{4.517742in}}%
\pgfpathlineto{\pgfqpoint{6.425509in}{4.518867in}}%
\pgfpathlineto{\pgfqpoint{6.428730in}{4.520525in}}%
\pgfpathlineto{\pgfqpoint{6.429804in}{4.529049in}}%
\pgfpathlineto{\pgfqpoint{6.430878in}{4.527155in}}%
\pgfpathlineto{\pgfqpoint{6.431952in}{4.539646in}}%
\pgfpathlineto{\pgfqpoint{6.433026in}{4.540238in}}%
\pgfpathlineto{\pgfqpoint{6.436247in}{4.536035in}}%
\pgfpathlineto{\pgfqpoint{6.438395in}{4.540238in}}%
\pgfpathlineto{\pgfqpoint{6.439469in}{4.541304in}}%
\pgfpathlineto{\pgfqpoint{6.440542in}{4.543909in}}%
\pgfpathlineto{\pgfqpoint{6.443764in}{4.543080in}}%
\pgfpathlineto{\pgfqpoint{6.444838in}{4.537811in}}%
\pgfpathlineto{\pgfqpoint{6.445912in}{4.536390in}}%
\pgfpathlineto{\pgfqpoint{6.446985in}{4.528457in}}%
\pgfpathlineto{\pgfqpoint{6.448059in}{4.527096in}}%
\pgfpathlineto{\pgfqpoint{6.451281in}{4.529286in}}%
\pgfpathlineto{\pgfqpoint{6.452355in}{4.528517in}}%
\pgfpathlineto{\pgfqpoint{6.453429in}{4.525616in}}%
\pgfpathlineto{\pgfqpoint{6.455576in}{4.528102in}}%
\pgfpathlineto{\pgfqpoint{6.458798in}{4.529582in}}%
\pgfpathlineto{\pgfqpoint{6.459872in}{4.532246in}}%
\pgfpathlineto{\pgfqpoint{6.460945in}{4.528872in}}%
\pgfpathlineto{\pgfqpoint{6.463093in}{4.526859in}}%
\pgfpathlineto{\pgfqpoint{6.466315in}{4.526800in}}%
\pgfpathlineto{\pgfqpoint{6.468462in}{4.544915in}}%
\pgfpathlineto{\pgfqpoint{6.469536in}{4.551309in}}%
\pgfpathlineto{\pgfqpoint{6.470610in}{4.552197in}}%
\pgfpathlineto{\pgfqpoint{6.474905in}{4.557051in}}%
\pgfpathlineto{\pgfqpoint{6.475979in}{4.554209in}}%
\pgfpathlineto{\pgfqpoint{6.477053in}{4.556341in}}%
\pgfpathlineto{\pgfqpoint{6.478127in}{4.556163in}}%
\pgfpathlineto{\pgfqpoint{6.481348in}{4.558768in}}%
\pgfpathlineto{\pgfqpoint{6.482422in}{4.560899in}}%
\pgfpathlineto{\pgfqpoint{6.483496in}{4.551723in}}%
\pgfpathlineto{\pgfqpoint{6.484570in}{4.548053in}}%
\pgfpathlineto{\pgfqpoint{6.485644in}{4.555985in}}%
\pgfpathlineto{\pgfqpoint{6.488865in}{4.562675in}}%
\pgfpathlineto{\pgfqpoint{6.491013in}{4.555926in}}%
\pgfpathlineto{\pgfqpoint{6.492087in}{4.555867in}}%
\pgfpathlineto{\pgfqpoint{6.493161in}{4.557229in}}%
\pgfpathlineto{\pgfqpoint{6.497456in}{4.556281in}}%
\pgfpathlineto{\pgfqpoint{6.499604in}{4.562793in}}%
\pgfpathlineto{\pgfqpoint{6.500677in}{4.560544in}}%
\pgfpathlineto{\pgfqpoint{6.500677in}{4.560544in}}%
\pgfusepath{stroke}%
\end{pgfscope}%
\begin{pgfscope}%
\pgfpathrectangle{\pgfqpoint{4.034768in}{4.180131in}}{\pgfqpoint{2.583333in}{0.400885in}}%
\pgfusepath{clip}%
\pgfsetroundcap%
\pgfsetroundjoin%
\pgfsetlinewidth{1.505625pt}%
\definecolor{currentstroke}{rgb}{1.000000,0.498039,0.054902}%
\pgfsetstrokecolor{currentstroke}%
\pgfsetdash{}{0pt}%
\pgfpathmoveto{\pgfqpoint{4.152193in}{4.243467in}}%
\pgfpathlineto{\pgfqpoint{4.153266in}{4.243467in}}%
\pgfpathlineto{\pgfqpoint{4.154340in}{4.240509in}}%
\pgfpathlineto{\pgfqpoint{4.155414in}{4.243284in}}%
\pgfpathlineto{\pgfqpoint{4.158636in}{4.243934in}}%
\pgfpathlineto{\pgfqpoint{4.160783in}{4.240997in}}%
\pgfpathlineto{\pgfqpoint{4.161857in}{4.237358in}}%
\pgfpathlineto{\pgfqpoint{4.162931in}{4.236797in}}%
\pgfpathlineto{\pgfqpoint{4.167226in}{4.239088in}}%
\pgfpathlineto{\pgfqpoint{4.169374in}{4.235427in}}%
\pgfpathlineto{\pgfqpoint{4.170448in}{4.239877in}}%
\pgfpathlineto{\pgfqpoint{4.174743in}{4.235771in}}%
\pgfpathlineto{\pgfqpoint{4.175817in}{4.240551in}}%
\pgfpathlineto{\pgfqpoint{4.176891in}{4.241507in}}%
\pgfpathlineto{\pgfqpoint{4.177965in}{4.240827in}}%
\pgfpathlineto{\pgfqpoint{4.181186in}{4.237144in}}%
\pgfpathlineto{\pgfqpoint{4.182260in}{4.242301in}}%
\pgfpathlineto{\pgfqpoint{4.184408in}{4.237140in}}%
\pgfpathlineto{\pgfqpoint{4.185482in}{4.231808in}}%
\pgfpathlineto{\pgfqpoint{4.188703in}{4.233894in}}%
\pgfpathlineto{\pgfqpoint{4.190851in}{4.237765in}}%
\pgfpathlineto{\pgfqpoint{4.192998in}{4.243439in}}%
\pgfpathlineto{\pgfqpoint{4.197294in}{4.245201in}}%
\pgfpathlineto{\pgfqpoint{4.198368in}{4.242990in}}%
\pgfpathlineto{\pgfqpoint{4.199441in}{4.249680in}}%
\pgfpathlineto{\pgfqpoint{4.204811in}{4.249680in}}%
\pgfpathlineto{\pgfqpoint{4.206958in}{4.248795in}}%
\pgfpathlineto{\pgfqpoint{4.208032in}{4.252114in}}%
\pgfpathlineto{\pgfqpoint{4.211254in}{4.247910in}}%
\pgfpathlineto{\pgfqpoint{4.212328in}{4.249946in}}%
\pgfpathlineto{\pgfqpoint{4.213401in}{4.245701in}}%
\pgfpathlineto{\pgfqpoint{4.215549in}{4.251852in}}%
\pgfpathlineto{\pgfqpoint{4.218771in}{4.251740in}}%
\pgfpathlineto{\pgfqpoint{4.219844in}{4.245622in}}%
\pgfpathlineto{\pgfqpoint{4.220918in}{4.248236in}}%
\pgfpathlineto{\pgfqpoint{4.221992in}{4.244843in}}%
\pgfpathlineto{\pgfqpoint{4.223066in}{4.243596in}}%
\pgfpathlineto{\pgfqpoint{4.226287in}{4.245690in}}%
\pgfpathlineto{\pgfqpoint{4.227361in}{4.252946in}}%
\pgfpathlineto{\pgfqpoint{4.228435in}{4.243617in}}%
\pgfpathlineto{\pgfqpoint{4.229509in}{4.241023in}}%
\pgfpathlineto{\pgfqpoint{4.230583in}{4.241770in}}%
\pgfpathlineto{\pgfqpoint{4.238100in}{4.248512in}}%
\pgfpathlineto{\pgfqpoint{4.242395in}{4.256951in}}%
\pgfpathlineto{\pgfqpoint{4.244543in}{4.266239in}}%
\pgfpathlineto{\pgfqpoint{4.245617in}{4.266078in}}%
\pgfpathlineto{\pgfqpoint{4.248838in}{4.266830in}}%
\pgfpathlineto{\pgfqpoint{4.249912in}{4.269621in}}%
\pgfpathlineto{\pgfqpoint{4.252060in}{4.278229in}}%
\pgfpathlineto{\pgfqpoint{4.256355in}{4.282955in}}%
\pgfpathlineto{\pgfqpoint{4.257429in}{4.277901in}}%
\pgfpathlineto{\pgfqpoint{4.258503in}{4.281863in}}%
\pgfpathlineto{\pgfqpoint{4.259576in}{4.276235in}}%
\pgfpathlineto{\pgfqpoint{4.260650in}{4.279994in}}%
\pgfpathlineto{\pgfqpoint{4.263872in}{4.282774in}}%
\pgfpathlineto{\pgfqpoint{4.264946in}{4.280959in}}%
\pgfpathlineto{\pgfqpoint{4.266019in}{4.281687in}}%
\pgfpathlineto{\pgfqpoint{4.267093in}{4.279322in}}%
\pgfpathlineto{\pgfqpoint{4.268167in}{4.278702in}}%
\pgfpathlineto{\pgfqpoint{4.271389in}{4.278083in}}%
\pgfpathlineto{\pgfqpoint{4.272462in}{4.279603in}}%
\pgfpathlineto{\pgfqpoint{4.274610in}{4.269785in}}%
\pgfpathlineto{\pgfqpoint{4.275684in}{4.267048in}}%
\pgfpathlineto{\pgfqpoint{4.278906in}{4.266842in}}%
\pgfpathlineto{\pgfqpoint{4.279979in}{4.270347in}}%
\pgfpathlineto{\pgfqpoint{4.282127in}{4.269009in}}%
\pgfpathlineto{\pgfqpoint{4.283201in}{4.265514in}}%
\pgfpathlineto{\pgfqpoint{4.286422in}{4.265514in}}%
\pgfpathlineto{\pgfqpoint{4.287496in}{4.264538in}}%
\pgfpathlineto{\pgfqpoint{4.288570in}{4.262533in}}%
\pgfpathlineto{\pgfqpoint{4.289644in}{4.262379in}}%
\pgfpathlineto{\pgfqpoint{4.290718in}{4.263407in}}%
\pgfpathlineto{\pgfqpoint{4.293939in}{4.269061in}}%
\pgfpathlineto{\pgfqpoint{4.296087in}{4.264081in}}%
\pgfpathlineto{\pgfqpoint{4.297161in}{4.255674in}}%
\pgfpathlineto{\pgfqpoint{4.298235in}{4.254496in}}%
\pgfpathlineto{\pgfqpoint{4.301456in}{4.258940in}}%
\pgfpathlineto{\pgfqpoint{4.302530in}{4.258833in}}%
\pgfpathlineto{\pgfqpoint{4.303604in}{4.257014in}}%
\pgfpathlineto{\pgfqpoint{4.305751in}{4.261348in}}%
\pgfpathlineto{\pgfqpoint{4.310047in}{4.265056in}}%
\pgfpathlineto{\pgfqpoint{4.311121in}{4.270454in}}%
\pgfpathlineto{\pgfqpoint{4.312195in}{4.272324in}}%
\pgfpathlineto{\pgfqpoint{4.313268in}{4.284569in}}%
\pgfpathlineto{\pgfqpoint{4.316490in}{4.285001in}}%
\pgfpathlineto{\pgfqpoint{4.317564in}{4.282345in}}%
\pgfpathlineto{\pgfqpoint{4.318638in}{4.274699in}}%
\pgfpathlineto{\pgfqpoint{4.319711in}{4.275448in}}%
\pgfpathlineto{\pgfqpoint{4.320785in}{4.278691in}}%
\pgfpathlineto{\pgfqpoint{4.324007in}{4.282630in}}%
\pgfpathlineto{\pgfqpoint{4.326154in}{4.297387in}}%
\pgfpathlineto{\pgfqpoint{4.327228in}{4.297075in}}%
\pgfpathlineto{\pgfqpoint{4.328302in}{4.304010in}}%
\pgfpathlineto{\pgfqpoint{4.331524in}{4.306446in}}%
\pgfpathlineto{\pgfqpoint{4.332597in}{4.312664in}}%
\pgfpathlineto{\pgfqpoint{4.333671in}{4.309809in}}%
\pgfpathlineto{\pgfqpoint{4.334745in}{4.316105in}}%
\pgfpathlineto{\pgfqpoint{4.335819in}{4.318893in}}%
\pgfpathlineto{\pgfqpoint{4.339040in}{4.323236in}}%
\pgfpathlineto{\pgfqpoint{4.340114in}{4.323635in}}%
\pgfpathlineto{\pgfqpoint{4.341188in}{4.328227in}}%
\pgfpathlineto{\pgfqpoint{4.342262in}{4.329358in}}%
\pgfpathlineto{\pgfqpoint{4.343336in}{4.338536in}}%
\pgfpathlineto{\pgfqpoint{4.346557in}{4.333512in}}%
\pgfpathlineto{\pgfqpoint{4.347631in}{4.330192in}}%
\pgfpathlineto{\pgfqpoint{4.349779in}{4.331086in}}%
\pgfpathlineto{\pgfqpoint{4.350853in}{4.327426in}}%
\pgfpathlineto{\pgfqpoint{4.355148in}{4.326078in}}%
\pgfpathlineto{\pgfqpoint{4.356222in}{4.324281in}}%
\pgfpathlineto{\pgfqpoint{4.357296in}{4.317604in}}%
\pgfpathlineto{\pgfqpoint{4.358370in}{4.323382in}}%
\pgfpathlineto{\pgfqpoint{4.361591in}{4.319273in}}%
\pgfpathlineto{\pgfqpoint{4.362665in}{4.319023in}}%
\pgfpathlineto{\pgfqpoint{4.363739in}{4.316834in}}%
\pgfpathlineto{\pgfqpoint{4.364813in}{4.305199in}}%
\pgfpathlineto{\pgfqpoint{4.365886in}{4.302822in}}%
\pgfpathlineto{\pgfqpoint{4.370182in}{4.301759in}}%
\pgfpathlineto{\pgfqpoint{4.371256in}{4.304136in}}%
\pgfpathlineto{\pgfqpoint{4.372329in}{4.294440in}}%
\pgfpathlineto{\pgfqpoint{4.373403in}{4.290380in}}%
\pgfpathlineto{\pgfqpoint{4.376625in}{4.291584in}}%
\pgfpathlineto{\pgfqpoint{4.379846in}{4.282169in}}%
\pgfpathlineto{\pgfqpoint{4.380920in}{4.288002in}}%
\pgfpathlineto{\pgfqpoint{4.384142in}{4.290659in}}%
\pgfpathlineto{\pgfqpoint{4.385216in}{4.293540in}}%
\pgfpathlineto{\pgfqpoint{4.386289in}{4.292129in}}%
\pgfpathlineto{\pgfqpoint{4.387363in}{4.299702in}}%
\pgfpathlineto{\pgfqpoint{4.388437in}{4.296273in}}%
\pgfpathlineto{\pgfqpoint{4.392732in}{4.298049in}}%
\pgfpathlineto{\pgfqpoint{4.393806in}{4.301237in}}%
\pgfpathlineto{\pgfqpoint{4.394880in}{4.297252in}}%
\pgfpathlineto{\pgfqpoint{4.395954in}{4.296116in}}%
\pgfpathlineto{\pgfqpoint{4.399175in}{4.299795in}}%
\pgfpathlineto{\pgfqpoint{4.400249in}{4.298154in}}%
\pgfpathlineto{\pgfqpoint{4.401323in}{4.299370in}}%
\pgfpathlineto{\pgfqpoint{4.402397in}{4.301557in}}%
\pgfpathlineto{\pgfqpoint{4.403471in}{4.307474in}}%
\pgfpathlineto{\pgfqpoint{4.406692in}{4.307844in}}%
\pgfpathlineto{\pgfqpoint{4.408840in}{4.309450in}}%
\pgfpathlineto{\pgfqpoint{4.409914in}{4.308084in}}%
\pgfpathlineto{\pgfqpoint{4.410988in}{4.314416in}}%
\pgfpathlineto{\pgfqpoint{4.415283in}{4.312678in}}%
\pgfpathlineto{\pgfqpoint{4.416357in}{4.320602in}}%
\pgfpathlineto{\pgfqpoint{4.417431in}{4.321892in}}%
\pgfpathlineto{\pgfqpoint{4.418505in}{4.320021in}}%
\pgfpathlineto{\pgfqpoint{4.421726in}{4.321233in}}%
\pgfpathlineto{\pgfqpoint{4.422800in}{4.319563in}}%
\pgfpathlineto{\pgfqpoint{4.423874in}{4.319820in}}%
\pgfpathlineto{\pgfqpoint{4.424948in}{4.330098in}}%
\pgfpathlineto{\pgfqpoint{4.426021in}{4.328813in}}%
\pgfpathlineto{\pgfqpoint{4.429243in}{4.329323in}}%
\pgfpathlineto{\pgfqpoint{4.430317in}{4.326382in}}%
\pgfpathlineto{\pgfqpoint{4.431391in}{4.326382in}}%
\pgfpathlineto{\pgfqpoint{4.433538in}{4.321713in}}%
\pgfpathlineto{\pgfqpoint{4.436760in}{4.320626in}}%
\pgfpathlineto{\pgfqpoint{4.437834in}{4.317492in}}%
\pgfpathlineto{\pgfqpoint{4.438907in}{4.311737in}}%
\pgfpathlineto{\pgfqpoint{4.439981in}{4.314231in}}%
\pgfpathlineto{\pgfqpoint{4.444277in}{4.307567in}}%
\pgfpathlineto{\pgfqpoint{4.446424in}{4.313077in}}%
\pgfpathlineto{\pgfqpoint{4.447498in}{4.307633in}}%
\pgfpathlineto{\pgfqpoint{4.448572in}{4.306544in}}%
\pgfpathlineto{\pgfqpoint{4.451794in}{4.305161in}}%
\pgfpathlineto{\pgfqpoint{4.452867in}{4.308260in}}%
\pgfpathlineto{\pgfqpoint{4.453941in}{4.306741in}}%
\pgfpathlineto{\pgfqpoint{4.456089in}{4.312756in}}%
\pgfpathlineto{\pgfqpoint{4.459310in}{4.311083in}}%
\pgfpathlineto{\pgfqpoint{4.460384in}{4.316910in}}%
\pgfpathlineto{\pgfqpoint{4.461458in}{4.312756in}}%
\pgfpathlineto{\pgfqpoint{4.462532in}{4.322367in}}%
\pgfpathlineto{\pgfqpoint{4.463606in}{4.318007in}}%
\pgfpathlineto{\pgfqpoint{4.466827in}{4.317301in}}%
\pgfpathlineto{\pgfqpoint{4.467901in}{4.309350in}}%
\pgfpathlineto{\pgfqpoint{4.468975in}{4.308516in}}%
\pgfpathlineto{\pgfqpoint{4.470049in}{4.309735in}}%
\pgfpathlineto{\pgfqpoint{4.471123in}{4.307939in}}%
\pgfpathlineto{\pgfqpoint{4.476492in}{4.306671in}}%
\pgfpathlineto{\pgfqpoint{4.477566in}{4.301638in}}%
\pgfpathlineto{\pgfqpoint{4.478639in}{4.302430in}}%
\pgfpathlineto{\pgfqpoint{4.481861in}{4.300469in}}%
\pgfpathlineto{\pgfqpoint{4.482935in}{4.305065in}}%
\pgfpathlineto{\pgfqpoint{4.484009in}{4.314010in}}%
\pgfpathlineto{\pgfqpoint{4.485083in}{4.313946in}}%
\pgfpathlineto{\pgfqpoint{4.486156in}{4.315501in}}%
\pgfpathlineto{\pgfqpoint{4.489378in}{4.317316in}}%
\pgfpathlineto{\pgfqpoint{4.491526in}{4.306174in}}%
\pgfpathlineto{\pgfqpoint{4.492599in}{4.306109in}}%
\pgfpathlineto{\pgfqpoint{4.493673in}{4.310041in}}%
\pgfpathlineto{\pgfqpoint{4.496895in}{4.304470in}}%
\pgfpathlineto{\pgfqpoint{4.497969in}{4.300861in}}%
\pgfpathlineto{\pgfqpoint{4.499042in}{4.300180in}}%
\pgfpathlineto{\pgfqpoint{4.501190in}{4.297222in}}%
\pgfpathlineto{\pgfqpoint{4.504412in}{4.301758in}}%
\pgfpathlineto{\pgfqpoint{4.505485in}{4.294721in}}%
\pgfpathlineto{\pgfqpoint{4.506559in}{4.300761in}}%
\pgfpathlineto{\pgfqpoint{4.507633in}{4.298457in}}%
\pgfpathlineto{\pgfqpoint{4.511928in}{4.299133in}}%
\pgfpathlineto{\pgfqpoint{4.513002in}{4.299992in}}%
\pgfpathlineto{\pgfqpoint{4.515150in}{4.304747in}}%
\pgfpathlineto{\pgfqpoint{4.516224in}{4.307554in}}%
\pgfpathlineto{\pgfqpoint{4.519445in}{4.306743in}}%
\pgfpathlineto{\pgfqpoint{4.520519in}{4.305005in}}%
\pgfpathlineto{\pgfqpoint{4.521593in}{4.301629in}}%
\pgfpathlineto{\pgfqpoint{4.522667in}{4.301328in}}%
\pgfpathlineto{\pgfqpoint{4.523741in}{4.307207in}}%
\pgfpathlineto{\pgfqpoint{4.526962in}{4.310756in}}%
\pgfpathlineto{\pgfqpoint{4.528036in}{4.307643in}}%
\pgfpathlineto{\pgfqpoint{4.530184in}{4.316930in}}%
\pgfpathlineto{\pgfqpoint{4.531258in}{4.315537in}}%
\pgfpathlineto{\pgfqpoint{4.534479in}{4.316228in}}%
\pgfpathlineto{\pgfqpoint{4.536627in}{4.313958in}}%
\pgfpathlineto{\pgfqpoint{4.537701in}{4.311437in}}%
\pgfpathlineto{\pgfqpoint{4.538774in}{4.311310in}}%
\pgfpathlineto{\pgfqpoint{4.541996in}{4.315976in}}%
\pgfpathlineto{\pgfqpoint{4.544144in}{4.307528in}}%
\pgfpathlineto{\pgfqpoint{4.545217in}{4.306210in}}%
\pgfpathlineto{\pgfqpoint{4.546291in}{4.302884in}}%
\pgfpathlineto{\pgfqpoint{4.549513in}{4.301546in}}%
\pgfpathlineto{\pgfqpoint{4.550587in}{4.299758in}}%
\pgfpathlineto{\pgfqpoint{4.551661in}{4.299473in}}%
\pgfpathlineto{\pgfqpoint{4.552734in}{4.302205in}}%
\pgfpathlineto{\pgfqpoint{4.553808in}{4.299871in}}%
\pgfpathlineto{\pgfqpoint{4.557030in}{4.300039in}}%
\pgfpathlineto{\pgfqpoint{4.558104in}{4.297794in}}%
\pgfpathlineto{\pgfqpoint{4.559177in}{4.297064in}}%
\pgfpathlineto{\pgfqpoint{4.560251in}{4.297681in}}%
\pgfpathlineto{\pgfqpoint{4.561325in}{4.302622in}}%
\pgfpathlineto{\pgfqpoint{4.565620in}{4.300651in}}%
\pgfpathlineto{\pgfqpoint{4.566694in}{4.298159in}}%
\pgfpathlineto{\pgfqpoint{4.567768in}{4.300130in}}%
\pgfpathlineto{\pgfqpoint{4.568842in}{4.299086in}}%
\pgfpathlineto{\pgfqpoint{4.572063in}{4.301332in}}%
\pgfpathlineto{\pgfqpoint{4.573137in}{4.303318in}}%
\pgfpathlineto{\pgfqpoint{4.574211in}{4.303960in}}%
\pgfpathlineto{\pgfqpoint{4.575285in}{4.301204in}}%
\pgfpathlineto{\pgfqpoint{4.576359in}{4.307069in}}%
\pgfpathlineto{\pgfqpoint{4.579580in}{4.309708in}}%
\pgfpathlineto{\pgfqpoint{4.580654in}{4.316446in}}%
\pgfpathlineto{\pgfqpoint{4.581728in}{4.316983in}}%
\pgfpathlineto{\pgfqpoint{4.582802in}{4.325597in}}%
\pgfpathlineto{\pgfqpoint{4.583876in}{4.327451in}}%
\pgfpathlineto{\pgfqpoint{4.587097in}{4.328480in}}%
\pgfpathlineto{\pgfqpoint{4.588171in}{4.327270in}}%
\pgfpathlineto{\pgfqpoint{4.589245in}{4.327870in}}%
\pgfpathlineto{\pgfqpoint{4.590319in}{4.329196in}}%
\pgfpathlineto{\pgfqpoint{4.591393in}{4.332752in}}%
\pgfpathlineto{\pgfqpoint{4.595688in}{4.336075in}}%
\pgfpathlineto{\pgfqpoint{4.596762in}{4.338167in}}%
\pgfpathlineto{\pgfqpoint{4.597836in}{4.336112in}}%
\pgfpathlineto{\pgfqpoint{4.598909in}{4.341903in}}%
\pgfpathlineto{\pgfqpoint{4.602131in}{4.345017in}}%
\pgfpathlineto{\pgfqpoint{4.603205in}{4.344447in}}%
\pgfpathlineto{\pgfqpoint{4.605352in}{4.349834in}}%
\pgfpathlineto{\pgfqpoint{4.606426in}{4.351116in}}%
\pgfpathlineto{\pgfqpoint{4.609648in}{4.348103in}}%
\pgfpathlineto{\pgfqpoint{4.610722in}{4.341042in}}%
\pgfpathlineto{\pgfqpoint{4.611795in}{4.338072in}}%
\pgfpathlineto{\pgfqpoint{4.613943in}{4.339386in}}%
\pgfpathlineto{\pgfqpoint{4.617165in}{4.335021in}}%
\pgfpathlineto{\pgfqpoint{4.618238in}{4.336129in}}%
\pgfpathlineto{\pgfqpoint{4.620386in}{4.336599in}}%
\pgfpathlineto{\pgfqpoint{4.621460in}{4.332783in}}%
\pgfpathlineto{\pgfqpoint{4.624682in}{4.334448in}}%
\pgfpathlineto{\pgfqpoint{4.625755in}{4.331027in}}%
\pgfpathlineto{\pgfqpoint{4.628977in}{4.342991in}}%
\pgfpathlineto{\pgfqpoint{4.632198in}{4.343581in}}%
\pgfpathlineto{\pgfqpoint{4.633272in}{4.349325in}}%
\pgfpathlineto{\pgfqpoint{4.634346in}{4.349444in}}%
\pgfpathlineto{\pgfqpoint{4.635420in}{4.351103in}}%
\pgfpathlineto{\pgfqpoint{4.639715in}{4.352170in}}%
\pgfpathlineto{\pgfqpoint{4.640789in}{4.354256in}}%
\pgfpathlineto{\pgfqpoint{4.641863in}{4.360753in}}%
\pgfpathlineto{\pgfqpoint{4.642937in}{4.363471in}}%
\pgfpathlineto{\pgfqpoint{4.644011in}{4.371561in}}%
\pgfpathlineto{\pgfqpoint{4.647232in}{4.373366in}}%
\pgfpathlineto{\pgfqpoint{4.649380in}{4.376731in}}%
\pgfpathlineto{\pgfqpoint{4.650454in}{4.374709in}}%
\pgfpathlineto{\pgfqpoint{4.651527in}{4.375097in}}%
\pgfpathlineto{\pgfqpoint{4.654749in}{4.365782in}}%
\pgfpathlineto{\pgfqpoint{4.656897in}{4.371410in}}%
\pgfpathlineto{\pgfqpoint{4.657971in}{4.376855in}}%
\pgfpathlineto{\pgfqpoint{4.659044in}{4.363733in}}%
\pgfpathlineto{\pgfqpoint{4.662266in}{4.366922in}}%
\pgfpathlineto{\pgfqpoint{4.663340in}{4.371414in}}%
\pgfpathlineto{\pgfqpoint{4.664414in}{4.371601in}}%
\pgfpathlineto{\pgfqpoint{4.666561in}{4.377846in}}%
\pgfpathlineto{\pgfqpoint{4.669783in}{4.377151in}}%
\pgfpathlineto{\pgfqpoint{4.670857in}{4.381635in}}%
\pgfpathlineto{\pgfqpoint{4.671930in}{4.382393in}}%
\pgfpathlineto{\pgfqpoint{4.673004in}{4.388796in}}%
\pgfpathlineto{\pgfqpoint{4.674078in}{4.383914in}}%
\pgfpathlineto{\pgfqpoint{4.677300in}{4.384842in}}%
\pgfpathlineto{\pgfqpoint{4.678373in}{4.385837in}}%
\pgfpathlineto{\pgfqpoint{4.681595in}{4.384099in}}%
\pgfpathlineto{\pgfqpoint{4.684816in}{4.382425in}}%
\pgfpathlineto{\pgfqpoint{4.685890in}{4.392163in}}%
\pgfpathlineto{\pgfqpoint{4.686964in}{4.384968in}}%
\pgfpathlineto{\pgfqpoint{4.688038in}{4.387961in}}%
\pgfpathlineto{\pgfqpoint{4.689112in}{4.394038in}}%
\pgfpathlineto{\pgfqpoint{4.692333in}{4.388082in}}%
\pgfpathlineto{\pgfqpoint{4.693407in}{4.384246in}}%
\pgfpathlineto{\pgfqpoint{4.694481in}{4.387778in}}%
\pgfpathlineto{\pgfqpoint{4.695555in}{4.389134in}}%
\pgfpathlineto{\pgfqpoint{4.696629in}{4.386010in}}%
\pgfpathlineto{\pgfqpoint{4.700924in}{4.381310in}}%
\pgfpathlineto{\pgfqpoint{4.708441in}{4.392968in}}%
\pgfpathlineto{\pgfqpoint{4.709515in}{4.386015in}}%
\pgfpathlineto{\pgfqpoint{4.710589in}{4.393903in}}%
\pgfpathlineto{\pgfqpoint{4.711662in}{4.384321in}}%
\pgfpathlineto{\pgfqpoint{4.714884in}{4.383095in}}%
\pgfpathlineto{\pgfqpoint{4.715958in}{4.391958in}}%
\pgfpathlineto{\pgfqpoint{4.717032in}{4.382399in}}%
\pgfpathlineto{\pgfqpoint{4.718105in}{4.385006in}}%
\pgfpathlineto{\pgfqpoint{4.719179in}{4.396824in}}%
\pgfpathlineto{\pgfqpoint{4.722401in}{4.401681in}}%
\pgfpathlineto{\pgfqpoint{4.723475in}{4.395287in}}%
\pgfpathlineto{\pgfqpoint{4.724549in}{4.399341in}}%
\pgfpathlineto{\pgfqpoint{4.725622in}{4.392044in}}%
\pgfpathlineto{\pgfqpoint{4.729918in}{4.401469in}}%
\pgfpathlineto{\pgfqpoint{4.730992in}{4.408463in}}%
\pgfpathlineto{\pgfqpoint{4.732065in}{4.404619in}}%
\pgfpathlineto{\pgfqpoint{4.733139in}{4.397690in}}%
\pgfpathlineto{\pgfqpoint{4.737435in}{4.404499in}}%
\pgfpathlineto{\pgfqpoint{4.738508in}{4.408850in}}%
\pgfpathlineto{\pgfqpoint{4.739582in}{4.408603in}}%
\pgfpathlineto{\pgfqpoint{4.741730in}{4.418413in}}%
\pgfpathlineto{\pgfqpoint{4.744951in}{4.414279in}}%
\pgfpathlineto{\pgfqpoint{4.746025in}{4.411132in}}%
\pgfpathlineto{\pgfqpoint{4.748173in}{4.421500in}}%
\pgfpathlineto{\pgfqpoint{4.749247in}{4.413610in}}%
\pgfpathlineto{\pgfqpoint{4.752468in}{4.416226in}}%
\pgfpathlineto{\pgfqpoint{4.753542in}{4.418512in}}%
\pgfpathlineto{\pgfqpoint{4.754616in}{4.426576in}}%
\pgfpathlineto{\pgfqpoint{4.755690in}{4.410584in}}%
\pgfpathlineto{\pgfqpoint{4.756764in}{4.410897in}}%
\pgfpathlineto{\pgfqpoint{4.761059in}{4.416394in}}%
\pgfpathlineto{\pgfqpoint{4.762133in}{4.421873in}}%
\pgfpathlineto{\pgfqpoint{4.763207in}{4.419543in}}%
\pgfpathlineto{\pgfqpoint{4.764281in}{4.421973in}}%
\pgfpathlineto{\pgfqpoint{4.767502in}{4.420334in}}%
\pgfpathlineto{\pgfqpoint{4.768576in}{4.421090in}}%
\pgfpathlineto{\pgfqpoint{4.770724in}{4.440407in}}%
\pgfpathlineto{\pgfqpoint{4.771797in}{4.440014in}}%
\pgfpathlineto{\pgfqpoint{4.776093in}{4.442170in}}%
\pgfpathlineto{\pgfqpoint{4.777167in}{4.438172in}}%
\pgfpathlineto{\pgfqpoint{4.779314in}{4.448592in}}%
\pgfpathlineto{\pgfqpoint{4.782536in}{4.448525in}}%
\pgfpathlineto{\pgfqpoint{4.784683in}{4.455131in}}%
\pgfpathlineto{\pgfqpoint{4.785757in}{4.449003in}}%
\pgfpathlineto{\pgfqpoint{4.790053in}{4.455872in}}%
\pgfpathlineto{\pgfqpoint{4.791126in}{4.455459in}}%
\pgfpathlineto{\pgfqpoint{4.792200in}{4.450226in}}%
\pgfpathlineto{\pgfqpoint{4.794348in}{4.454151in}}%
\pgfpathlineto{\pgfqpoint{4.797570in}{4.453943in}}%
\pgfpathlineto{\pgfqpoint{4.798643in}{4.443143in}}%
\pgfpathlineto{\pgfqpoint{4.799717in}{4.443766in}}%
\pgfpathlineto{\pgfqpoint{4.800791in}{4.442450in}}%
\pgfpathlineto{\pgfqpoint{4.801865in}{4.444378in}}%
\pgfpathlineto{\pgfqpoint{4.806160in}{4.447640in}}%
\pgfpathlineto{\pgfqpoint{4.808308in}{4.438976in}}%
\pgfpathlineto{\pgfqpoint{4.809382in}{4.441716in}}%
\pgfpathlineto{\pgfqpoint{4.812603in}{4.443611in}}%
\pgfpathlineto{\pgfqpoint{4.814751in}{4.434296in}}%
\pgfpathlineto{\pgfqpoint{4.816899in}{4.433582in}}%
\pgfpathlineto{\pgfqpoint{4.820120in}{4.435395in}}%
\pgfpathlineto{\pgfqpoint{4.822268in}{4.423436in}}%
\pgfpathlineto{\pgfqpoint{4.823342in}{4.423866in}}%
\pgfpathlineto{\pgfqpoint{4.827637in}{4.417535in}}%
\pgfpathlineto{\pgfqpoint{4.828711in}{4.415445in}}%
\pgfpathlineto{\pgfqpoint{4.829785in}{4.414953in}}%
\pgfpathlineto{\pgfqpoint{4.831932in}{4.419317in}}%
\pgfpathlineto{\pgfqpoint{4.835154in}{4.417203in}}%
\pgfpathlineto{\pgfqpoint{4.836228in}{4.419566in}}%
\pgfpathlineto{\pgfqpoint{4.837302in}{4.425970in}}%
\pgfpathlineto{\pgfqpoint{4.838375in}{4.422704in}}%
\pgfpathlineto{\pgfqpoint{4.839449in}{4.424369in}}%
\pgfpathlineto{\pgfqpoint{4.842671in}{4.432439in}}%
\pgfpathlineto{\pgfqpoint{4.843745in}{4.428053in}}%
\pgfpathlineto{\pgfqpoint{4.844818in}{4.428053in}}%
\pgfpathlineto{\pgfqpoint{4.845892in}{4.442939in}}%
\pgfpathlineto{\pgfqpoint{4.846966in}{4.438420in}}%
\pgfpathlineto{\pgfqpoint{4.850188in}{4.434381in}}%
\pgfpathlineto{\pgfqpoint{4.852335in}{4.445588in}}%
\pgfpathlineto{\pgfqpoint{4.853409in}{4.422115in}}%
\pgfpathlineto{\pgfqpoint{4.854483in}{4.420522in}}%
\pgfpathlineto{\pgfqpoint{4.857704in}{4.421166in}}%
\pgfpathlineto{\pgfqpoint{4.859852in}{4.424985in}}%
\pgfpathlineto{\pgfqpoint{4.860926in}{4.426463in}}%
\pgfpathlineto{\pgfqpoint{4.862000in}{4.417183in}}%
\pgfpathlineto{\pgfqpoint{4.865221in}{4.416447in}}%
\pgfpathlineto{\pgfqpoint{4.866295in}{4.411760in}}%
\pgfpathlineto{\pgfqpoint{4.867369in}{4.414578in}}%
\pgfpathlineto{\pgfqpoint{4.868443in}{4.407860in}}%
\pgfpathlineto{\pgfqpoint{4.869517in}{4.409707in}}%
\pgfpathlineto{\pgfqpoint{4.874886in}{4.411896in}}%
\pgfpathlineto{\pgfqpoint{4.877034in}{4.422725in}}%
\pgfpathlineto{\pgfqpoint{4.880255in}{4.424697in}}%
\pgfpathlineto{\pgfqpoint{4.881329in}{4.422707in}}%
\pgfpathlineto{\pgfqpoint{4.882403in}{4.424872in}}%
\pgfpathlineto{\pgfqpoint{4.883477in}{4.422415in}}%
\pgfpathlineto{\pgfqpoint{4.884550in}{4.417613in}}%
\pgfpathlineto{\pgfqpoint{4.889920in}{4.421706in}}%
\pgfpathlineto{\pgfqpoint{4.890993in}{4.430677in}}%
\pgfpathlineto{\pgfqpoint{4.892067in}{4.431195in}}%
\pgfpathlineto{\pgfqpoint{4.895289in}{4.435114in}}%
\pgfpathlineto{\pgfqpoint{4.897437in}{4.429900in}}%
\pgfpathlineto{\pgfqpoint{4.899584in}{4.428773in}}%
\pgfpathlineto{\pgfqpoint{4.902806in}{4.431407in}}%
\pgfpathlineto{\pgfqpoint{4.903880in}{4.427835in}}%
\pgfpathlineto{\pgfqpoint{4.904953in}{4.427268in}}%
\pgfpathlineto{\pgfqpoint{4.906027in}{4.427495in}}%
\pgfpathlineto{\pgfqpoint{4.907101in}{4.434752in}}%
\pgfpathlineto{\pgfqpoint{4.910323in}{4.435716in}}%
\pgfpathlineto{\pgfqpoint{4.911396in}{4.433324in}}%
\pgfpathlineto{\pgfqpoint{4.912470in}{4.426036in}}%
\pgfpathlineto{\pgfqpoint{4.913544in}{4.422904in}}%
\pgfpathlineto{\pgfqpoint{4.914618in}{4.424897in}}%
\pgfpathlineto{\pgfqpoint{4.917839in}{4.420512in}}%
\pgfpathlineto{\pgfqpoint{4.918913in}{4.422633in}}%
\pgfpathlineto{\pgfqpoint{4.919987in}{4.432383in}}%
\pgfpathlineto{\pgfqpoint{4.921061in}{4.430185in}}%
\pgfpathlineto{\pgfqpoint{4.922135in}{4.424270in}}%
\pgfpathlineto{\pgfqpoint{4.929652in}{4.416275in}}%
\pgfpathlineto{\pgfqpoint{4.932873in}{4.414337in}}%
\pgfpathlineto{\pgfqpoint{4.933947in}{4.408940in}}%
\pgfpathlineto{\pgfqpoint{4.937169in}{4.416266in}}%
\pgfpathlineto{\pgfqpoint{4.940390in}{4.416474in}}%
\pgfpathlineto{\pgfqpoint{4.941464in}{4.414859in}}%
\pgfpathlineto{\pgfqpoint{4.942538in}{4.416161in}}%
\pgfpathlineto{\pgfqpoint{4.943612in}{4.418662in}}%
\pgfpathlineto{\pgfqpoint{4.944685in}{4.417028in}}%
\pgfpathlineto{\pgfqpoint{4.947907in}{4.409491in}}%
\pgfpathlineto{\pgfqpoint{4.948981in}{4.410123in}}%
\pgfpathlineto{\pgfqpoint{4.950055in}{4.404642in}}%
\pgfpathlineto{\pgfqpoint{4.951128in}{4.406902in}}%
\pgfpathlineto{\pgfqpoint{4.952202in}{4.422166in}}%
\pgfpathlineto{\pgfqpoint{4.956498in}{4.423983in}}%
\pgfpathlineto{\pgfqpoint{4.957571in}{4.426600in}}%
\pgfpathlineto{\pgfqpoint{4.958645in}{4.436076in}}%
\pgfpathlineto{\pgfqpoint{4.959719in}{4.424925in}}%
\pgfpathlineto{\pgfqpoint{4.962941in}{4.418749in}}%
\pgfpathlineto{\pgfqpoint{4.966162in}{4.434299in}}%
\pgfpathlineto{\pgfqpoint{4.967236in}{4.442538in}}%
\pgfpathlineto{\pgfqpoint{4.970458in}{4.430966in}}%
\pgfpathlineto{\pgfqpoint{4.971531in}{4.437072in}}%
\pgfpathlineto{\pgfqpoint{4.972605in}{4.438584in}}%
\pgfpathlineto{\pgfqpoint{4.973679in}{4.449239in}}%
\pgfpathlineto{\pgfqpoint{4.974753in}{4.442214in}}%
\pgfpathlineto{\pgfqpoint{4.980122in}{4.431770in}}%
\pgfpathlineto{\pgfqpoint{4.982270in}{4.432910in}}%
\pgfpathlineto{\pgfqpoint{4.986565in}{4.432964in}}%
\pgfpathlineto{\pgfqpoint{4.987639in}{4.432094in}}%
\pgfpathlineto{\pgfqpoint{4.989787in}{4.434269in}}%
\pgfpathlineto{\pgfqpoint{4.993008in}{4.440122in}}%
\pgfpathlineto{\pgfqpoint{4.994082in}{4.440176in}}%
\pgfpathlineto{\pgfqpoint{4.995156in}{4.439301in}}%
\pgfpathlineto{\pgfqpoint{4.996230in}{4.442091in}}%
\pgfpathlineto{\pgfqpoint{4.997303in}{4.437058in}}%
\pgfpathlineto{\pgfqpoint{5.000525in}{4.443533in}}%
\pgfpathlineto{\pgfqpoint{5.001599in}{4.456916in}}%
\pgfpathlineto{\pgfqpoint{5.002673in}{4.459284in}}%
\pgfpathlineto{\pgfqpoint{5.003747in}{4.466349in}}%
\pgfpathlineto{\pgfqpoint{5.004820in}{4.464624in}}%
\pgfpathlineto{\pgfqpoint{5.008042in}{4.465453in}}%
\pgfpathlineto{\pgfqpoint{5.010190in}{4.461407in}}%
\pgfpathlineto{\pgfqpoint{5.011263in}{4.450100in}}%
\pgfpathlineto{\pgfqpoint{5.012337in}{4.447440in}}%
\pgfpathlineto{\pgfqpoint{5.015559in}{4.453038in}}%
\pgfpathlineto{\pgfqpoint{5.016633in}{4.451375in}}%
\pgfpathlineto{\pgfqpoint{5.018780in}{4.460657in}}%
\pgfpathlineto{\pgfqpoint{5.019854in}{4.461498in}}%
\pgfpathlineto{\pgfqpoint{5.024149in}{4.457953in}}%
\pgfpathlineto{\pgfqpoint{5.025223in}{4.451819in}}%
\pgfpathlineto{\pgfqpoint{5.026297in}{4.453564in}}%
\pgfpathlineto{\pgfqpoint{5.027371in}{4.451144in}}%
\pgfpathlineto{\pgfqpoint{5.030592in}{4.453372in}}%
\pgfpathlineto{\pgfqpoint{5.031666in}{4.459275in}}%
\pgfpathlineto{\pgfqpoint{5.033814in}{4.465046in}}%
\pgfpathlineto{\pgfqpoint{5.034888in}{4.474546in}}%
\pgfpathlineto{\pgfqpoint{5.038109in}{4.460516in}}%
\pgfpathlineto{\pgfqpoint{5.039183in}{4.459924in}}%
\pgfpathlineto{\pgfqpoint{5.041331in}{4.490529in}}%
\pgfpathlineto{\pgfqpoint{5.042405in}{4.485738in}}%
\pgfpathlineto{\pgfqpoint{5.045626in}{4.491359in}}%
\pgfpathlineto{\pgfqpoint{5.046700in}{4.488165in}}%
\pgfpathlineto{\pgfqpoint{5.047774in}{4.480219in}}%
\pgfpathlineto{\pgfqpoint{5.048848in}{4.486879in}}%
\pgfpathlineto{\pgfqpoint{5.053143in}{4.489452in}}%
\pgfpathlineto{\pgfqpoint{5.054217in}{4.487130in}}%
\pgfpathlineto{\pgfqpoint{5.055291in}{4.486757in}}%
\pgfpathlineto{\pgfqpoint{5.057438in}{4.490545in}}%
\pgfpathlineto{\pgfqpoint{5.060660in}{4.490670in}}%
\pgfpathlineto{\pgfqpoint{5.062808in}{4.497059in}}%
\pgfpathlineto{\pgfqpoint{5.063881in}{4.493522in}}%
\pgfpathlineto{\pgfqpoint{5.064955in}{4.494153in}}%
\pgfpathlineto{\pgfqpoint{5.068177in}{4.491627in}}%
\pgfpathlineto{\pgfqpoint{5.069251in}{4.498318in}}%
\pgfpathlineto{\pgfqpoint{5.070325in}{4.508851in}}%
\pgfpathlineto{\pgfqpoint{5.071398in}{4.505061in}}%
\pgfpathlineto{\pgfqpoint{5.075694in}{4.499010in}}%
\pgfpathlineto{\pgfqpoint{5.076768in}{4.502225in}}%
\pgfpathlineto{\pgfqpoint{5.079989in}{4.493021in}}%
\pgfpathlineto{\pgfqpoint{5.083211in}{4.496590in}}%
\pgfpathlineto{\pgfqpoint{5.085358in}{4.507088in}}%
\pgfpathlineto{\pgfqpoint{5.086432in}{4.505358in}}%
\pgfpathlineto{\pgfqpoint{5.091801in}{4.486539in}}%
\pgfpathlineto{\pgfqpoint{5.093949in}{4.489506in}}%
\pgfpathlineto{\pgfqpoint{5.095023in}{4.488314in}}%
\pgfpathlineto{\pgfqpoint{5.098244in}{4.486179in}}%
\pgfpathlineto{\pgfqpoint{5.100392in}{4.487533in}}%
\pgfpathlineto{\pgfqpoint{5.101466in}{4.492904in}}%
\pgfpathlineto{\pgfqpoint{5.102540in}{4.481395in}}%
\pgfpathlineto{\pgfqpoint{5.105761in}{4.477955in}}%
\pgfpathlineto{\pgfqpoint{5.106835in}{4.479456in}}%
\pgfpathlineto{\pgfqpoint{5.108983in}{4.476770in}}%
\pgfpathlineto{\pgfqpoint{5.110057in}{4.477218in}}%
\pgfpathlineto{\pgfqpoint{5.113278in}{4.474756in}}%
\pgfpathlineto{\pgfqpoint{5.114352in}{4.476267in}}%
\pgfpathlineto{\pgfqpoint{5.115426in}{4.471622in}}%
\pgfpathlineto{\pgfqpoint{5.117573in}{4.476478in}}%
\pgfpathlineto{\pgfqpoint{5.120795in}{4.476644in}}%
\pgfpathlineto{\pgfqpoint{5.121869in}{4.471149in}}%
\pgfpathlineto{\pgfqpoint{5.124016in}{4.473925in}}%
\pgfpathlineto{\pgfqpoint{5.125090in}{4.477167in}}%
\pgfpathlineto{\pgfqpoint{5.128312in}{4.477502in}}%
\pgfpathlineto{\pgfqpoint{5.129386in}{4.480021in}}%
\pgfpathlineto{\pgfqpoint{5.130459in}{4.480749in}}%
\pgfpathlineto{\pgfqpoint{5.131533in}{4.484005in}}%
\pgfpathlineto{\pgfqpoint{5.135829in}{4.486307in}}%
\pgfpathlineto{\pgfqpoint{5.136903in}{4.482000in}}%
\pgfpathlineto{\pgfqpoint{5.139050in}{4.485684in}}%
\pgfpathlineto{\pgfqpoint{5.145493in}{4.485741in}}%
\pgfpathlineto{\pgfqpoint{5.146567in}{4.477931in}}%
\pgfpathlineto{\pgfqpoint{5.147641in}{4.480782in}}%
\pgfpathlineto{\pgfqpoint{5.150862in}{4.484202in}}%
\pgfpathlineto{\pgfqpoint{5.151936in}{4.486860in}}%
\pgfpathlineto{\pgfqpoint{5.153010in}{4.487381in}}%
\pgfpathlineto{\pgfqpoint{5.154084in}{4.486686in}}%
\pgfpathlineto{\pgfqpoint{5.155158in}{4.480141in}}%
\pgfpathlineto{\pgfqpoint{5.158379in}{4.479793in}}%
\pgfpathlineto{\pgfqpoint{5.159453in}{4.478982in}}%
\pgfpathlineto{\pgfqpoint{5.160527in}{4.474696in}}%
\pgfpathlineto{\pgfqpoint{5.162675in}{4.450892in}}%
\pgfpathlineto{\pgfqpoint{5.165896in}{4.453324in}}%
\pgfpathlineto{\pgfqpoint{5.166970in}{4.456336in}}%
\pgfpathlineto{\pgfqpoint{5.168044in}{4.456512in}}%
\pgfpathlineto{\pgfqpoint{5.169118in}{4.454458in}}%
\pgfpathlineto{\pgfqpoint{5.170191in}{4.462321in}}%
\pgfpathlineto{\pgfqpoint{5.174487in}{4.465849in}}%
\pgfpathlineto{\pgfqpoint{5.175561in}{4.464068in}}%
\pgfpathlineto{\pgfqpoint{5.176635in}{4.464775in}}%
\pgfpathlineto{\pgfqpoint{5.177708in}{4.461111in}}%
\pgfpathlineto{\pgfqpoint{5.180930in}{4.464184in}}%
\pgfpathlineto{\pgfqpoint{5.182004in}{4.458924in}}%
\pgfpathlineto{\pgfqpoint{5.183078in}{4.456958in}}%
\pgfpathlineto{\pgfqpoint{5.184151in}{4.453518in}}%
\pgfpathlineto{\pgfqpoint{5.190594in}{4.461414in}}%
\pgfpathlineto{\pgfqpoint{5.191668in}{4.460266in}}%
\pgfpathlineto{\pgfqpoint{5.192742in}{4.463079in}}%
\pgfpathlineto{\pgfqpoint{5.197037in}{4.462160in}}%
\pgfpathlineto{\pgfqpoint{5.198111in}{4.459930in}}%
\pgfpathlineto{\pgfqpoint{5.199185in}{4.462536in}}%
\pgfpathlineto{\pgfqpoint{5.200259in}{4.462307in}}%
\pgfpathlineto{\pgfqpoint{5.203480in}{4.458756in}}%
\pgfpathlineto{\pgfqpoint{5.204554in}{4.453258in}}%
\pgfpathlineto{\pgfqpoint{5.205628in}{4.455950in}}%
\pgfpathlineto{\pgfqpoint{5.206702in}{4.455893in}}%
\pgfpathlineto{\pgfqpoint{5.207776in}{4.451770in}}%
\pgfpathlineto{\pgfqpoint{5.210997in}{4.450396in}}%
\pgfpathlineto{\pgfqpoint{5.212071in}{4.455606in}}%
\pgfpathlineto{\pgfqpoint{5.214219in}{4.446475in}}%
\pgfpathlineto{\pgfqpoint{5.215293in}{4.448512in}}%
\pgfpathlineto{\pgfqpoint{5.218514in}{4.444846in}}%
\pgfpathlineto{\pgfqpoint{5.219588in}{4.441347in}}%
\pgfpathlineto{\pgfqpoint{5.220662in}{4.442458in}}%
\pgfpathlineto{\pgfqpoint{5.222810in}{4.460124in}}%
\pgfpathlineto{\pgfqpoint{5.226031in}{4.462378in}}%
\pgfpathlineto{\pgfqpoint{5.227105in}{4.460161in}}%
\pgfpathlineto{\pgfqpoint{5.228179in}{4.452166in}}%
\pgfpathlineto{\pgfqpoint{5.229253in}{4.452516in}}%
\pgfpathlineto{\pgfqpoint{5.230326in}{4.459519in}}%
\pgfpathlineto{\pgfqpoint{5.233548in}{4.460219in}}%
\pgfpathlineto{\pgfqpoint{5.234622in}{4.451089in}}%
\pgfpathlineto{\pgfqpoint{5.236769in}{4.470520in}}%
\pgfpathlineto{\pgfqpoint{5.237843in}{4.465430in}}%
\pgfpathlineto{\pgfqpoint{5.241065in}{4.453007in}}%
\pgfpathlineto{\pgfqpoint{5.242139in}{4.452765in}}%
\pgfpathlineto{\pgfqpoint{5.243213in}{4.442584in}}%
\pgfpathlineto{\pgfqpoint{5.244286in}{4.438706in}}%
\pgfpathlineto{\pgfqpoint{5.245360in}{4.451856in}}%
\pgfpathlineto{\pgfqpoint{5.248582in}{4.443796in}}%
\pgfpathlineto{\pgfqpoint{5.249656in}{4.434900in}}%
\pgfpathlineto{\pgfqpoint{5.251803in}{4.452639in}}%
\pgfpathlineto{\pgfqpoint{5.252877in}{4.448365in}}%
\pgfpathlineto{\pgfqpoint{5.256099in}{4.447101in}}%
\pgfpathlineto{\pgfqpoint{5.257172in}{4.439783in}}%
\pgfpathlineto{\pgfqpoint{5.259320in}{4.436088in}}%
\pgfpathlineto{\pgfqpoint{5.260394in}{4.429932in}}%
\pgfpathlineto{\pgfqpoint{5.263615in}{4.425533in}}%
\pgfpathlineto{\pgfqpoint{5.264689in}{4.422935in}}%
\pgfpathlineto{\pgfqpoint{5.265763in}{4.418005in}}%
\pgfpathlineto{\pgfqpoint{5.267911in}{4.425142in}}%
\pgfpathlineto{\pgfqpoint{5.271132in}{4.424477in}}%
\pgfpathlineto{\pgfqpoint{5.272206in}{4.427586in}}%
\pgfpathlineto{\pgfqpoint{5.273280in}{4.426655in}}%
\pgfpathlineto{\pgfqpoint{5.275428in}{4.422468in}}%
\pgfpathlineto{\pgfqpoint{5.278649in}{4.419832in}}%
\pgfpathlineto{\pgfqpoint{5.279723in}{4.422003in}}%
\pgfpathlineto{\pgfqpoint{5.281871in}{4.422934in}}%
\pgfpathlineto{\pgfqpoint{5.286166in}{4.427144in}}%
\pgfpathlineto{\pgfqpoint{5.287240in}{4.425427in}}%
\pgfpathlineto{\pgfqpoint{5.288314in}{4.425169in}}%
\pgfpathlineto{\pgfqpoint{5.290461in}{4.429810in}}%
\pgfpathlineto{\pgfqpoint{5.293683in}{4.431151in}}%
\pgfpathlineto{\pgfqpoint{5.294757in}{4.435302in}}%
\pgfpathlineto{\pgfqpoint{5.295831in}{4.443760in}}%
\pgfpathlineto{\pgfqpoint{5.296904in}{4.446670in}}%
\pgfpathlineto{\pgfqpoint{5.297978in}{4.442575in}}%
\pgfpathlineto{\pgfqpoint{5.301200in}{4.438075in}}%
\pgfpathlineto{\pgfqpoint{5.302274in}{4.439061in}}%
\pgfpathlineto{\pgfqpoint{5.303347in}{4.436142in}}%
\pgfpathlineto{\pgfqpoint{5.304421in}{4.437967in}}%
\pgfpathlineto{\pgfqpoint{5.305495in}{4.448287in}}%
\pgfpathlineto{\pgfqpoint{5.308717in}{4.443431in}}%
\pgfpathlineto{\pgfqpoint{5.309791in}{4.434318in}}%
\pgfpathlineto{\pgfqpoint{5.310864in}{4.445450in}}%
\pgfpathlineto{\pgfqpoint{5.311938in}{4.431590in}}%
\pgfpathlineto{\pgfqpoint{5.316234in}{4.436335in}}%
\pgfpathlineto{\pgfqpoint{5.320529in}{4.433147in}}%
\pgfpathlineto{\pgfqpoint{5.324824in}{4.433249in}}%
\pgfpathlineto{\pgfqpoint{5.325898in}{4.427440in}}%
\pgfpathlineto{\pgfqpoint{5.328046in}{4.427338in}}%
\pgfpathlineto{\pgfqpoint{5.331267in}{4.415670in}}%
\pgfpathlineto{\pgfqpoint{5.332341in}{4.406549in}}%
\pgfpathlineto{\pgfqpoint{5.333415in}{4.415721in}}%
\pgfpathlineto{\pgfqpoint{5.334489in}{4.409657in}}%
\pgfpathlineto{\pgfqpoint{5.335563in}{4.415007in}}%
\pgfpathlineto{\pgfqpoint{5.339858in}{4.409368in}}%
\pgfpathlineto{\pgfqpoint{5.342006in}{4.393517in}}%
\pgfpathlineto{\pgfqpoint{5.343079in}{4.394279in}}%
\pgfpathlineto{\pgfqpoint{5.348449in}{4.386611in}}%
\pgfpathlineto{\pgfqpoint{5.349523in}{4.401629in}}%
\pgfpathlineto{\pgfqpoint{5.353818in}{4.396355in}}%
\pgfpathlineto{\pgfqpoint{5.355966in}{4.388655in}}%
\pgfpathlineto{\pgfqpoint{5.357039in}{4.389974in}}%
\pgfpathlineto{\pgfqpoint{5.358113in}{4.396408in}}%
\pgfpathlineto{\pgfqpoint{5.361335in}{4.403705in}}%
\pgfpathlineto{\pgfqpoint{5.362409in}{4.395591in}}%
\pgfpathlineto{\pgfqpoint{5.363482in}{4.396167in}}%
\pgfpathlineto{\pgfqpoint{5.364556in}{4.401629in}}%
\pgfpathlineto{\pgfqpoint{5.365630in}{4.400264in}}%
\pgfpathlineto{\pgfqpoint{5.368852in}{4.400368in}}%
\pgfpathlineto{\pgfqpoint{5.369925in}{4.404544in}}%
\pgfpathlineto{\pgfqpoint{5.370999in}{4.403761in}}%
\pgfpathlineto{\pgfqpoint{5.372073in}{4.430544in}}%
\pgfpathlineto{\pgfqpoint{5.373147in}{4.417353in}}%
\pgfpathlineto{\pgfqpoint{5.377442in}{4.422854in}}%
\pgfpathlineto{\pgfqpoint{5.378516in}{4.419009in}}%
\pgfpathlineto{\pgfqpoint{5.380664in}{4.434449in}}%
\pgfpathlineto{\pgfqpoint{5.383885in}{4.431802in}}%
\pgfpathlineto{\pgfqpoint{5.384959in}{4.428711in}}%
\pgfpathlineto{\pgfqpoint{5.387107in}{4.415882in}}%
\pgfpathlineto{\pgfqpoint{5.388181in}{4.424398in}}%
\pgfpathlineto{\pgfqpoint{5.392476in}{4.427389in}}%
\pgfpathlineto{\pgfqpoint{5.393550in}{4.420465in}}%
\pgfpathlineto{\pgfqpoint{5.394624in}{4.423293in}}%
\pgfpathlineto{\pgfqpoint{5.395698in}{4.427793in}}%
\pgfpathlineto{\pgfqpoint{5.398919in}{4.428854in}}%
\pgfpathlineto{\pgfqpoint{5.399993in}{4.436511in}}%
\pgfpathlineto{\pgfqpoint{5.401067in}{4.438463in}}%
\pgfpathlineto{\pgfqpoint{5.402141in}{4.426510in}}%
\pgfpathlineto{\pgfqpoint{5.403214in}{4.431657in}}%
\pgfpathlineto{\pgfqpoint{5.407510in}{4.438999in}}%
\pgfpathlineto{\pgfqpoint{5.408584in}{4.443488in}}%
\pgfpathlineto{\pgfqpoint{5.409657in}{4.445283in}}%
\pgfpathlineto{\pgfqpoint{5.410731in}{4.451798in}}%
\pgfpathlineto{\pgfqpoint{5.413953in}{4.454573in}}%
\pgfpathlineto{\pgfqpoint{5.416101in}{4.442669in}}%
\pgfpathlineto{\pgfqpoint{5.417174in}{4.433389in}}%
\pgfpathlineto{\pgfqpoint{5.418248in}{4.430520in}}%
\pgfpathlineto{\pgfqpoint{5.421470in}{4.430947in}}%
\pgfpathlineto{\pgfqpoint{5.422544in}{4.429055in}}%
\pgfpathlineto{\pgfqpoint{5.424691in}{4.420186in}}%
\pgfpathlineto{\pgfqpoint{5.428987in}{4.420593in}}%
\pgfpathlineto{\pgfqpoint{5.430060in}{4.413546in}}%
\pgfpathlineto{\pgfqpoint{5.431134in}{4.415468in}}%
\pgfpathlineto{\pgfqpoint{5.433282in}{4.410603in}}%
\pgfpathlineto{\pgfqpoint{5.436503in}{4.412311in}}%
\pgfpathlineto{\pgfqpoint{5.437577in}{4.413516in}}%
\pgfpathlineto{\pgfqpoint{5.438651in}{4.412139in}}%
\pgfpathlineto{\pgfqpoint{5.439725in}{4.405923in}}%
\pgfpathlineto{\pgfqpoint{5.440799in}{4.424531in}}%
\pgfpathlineto{\pgfqpoint{5.445094in}{4.424349in}}%
\pgfpathlineto{\pgfqpoint{5.446168in}{4.417812in}}%
\pgfpathlineto{\pgfqpoint{5.447242in}{4.419398in}}%
\pgfpathlineto{\pgfqpoint{5.451537in}{4.415729in}}%
\pgfpathlineto{\pgfqpoint{5.452611in}{4.415729in}}%
\pgfpathlineto{\pgfqpoint{5.453685in}{4.413894in}}%
\pgfpathlineto{\pgfqpoint{5.454759in}{4.415492in}}%
\pgfpathlineto{\pgfqpoint{5.459054in}{4.410896in}}%
\pgfpathlineto{\pgfqpoint{5.460128in}{4.413271in}}%
\pgfpathlineto{\pgfqpoint{5.461202in}{4.413329in}}%
\pgfpathlineto{\pgfqpoint{5.462276in}{4.416492in}}%
\pgfpathlineto{\pgfqpoint{5.463349in}{4.413505in}}%
\pgfpathlineto{\pgfqpoint{5.466571in}{4.409577in}}%
\pgfpathlineto{\pgfqpoint{5.468719in}{4.417071in}}%
\pgfpathlineto{\pgfqpoint{5.469792in}{4.412290in}}%
\pgfpathlineto{\pgfqpoint{5.470866in}{4.415107in}}%
\pgfpathlineto{\pgfqpoint{5.474088in}{4.414878in}}%
\pgfpathlineto{\pgfqpoint{5.475162in}{4.421215in}}%
\pgfpathlineto{\pgfqpoint{5.476235in}{4.423384in}}%
\pgfpathlineto{\pgfqpoint{5.477309in}{4.422346in}}%
\pgfpathlineto{\pgfqpoint{5.478383in}{4.424941in}}%
\pgfpathlineto{\pgfqpoint{5.482679in}{4.430649in}}%
\pgfpathlineto{\pgfqpoint{5.484826in}{4.430117in}}%
\pgfpathlineto{\pgfqpoint{5.485900in}{4.428104in}}%
\pgfpathlineto{\pgfqpoint{5.489122in}{4.426684in}}%
\pgfpathlineto{\pgfqpoint{5.490195in}{4.429584in}}%
\pgfpathlineto{\pgfqpoint{5.491269in}{4.426210in}}%
\pgfpathlineto{\pgfqpoint{5.492343in}{4.432972in}}%
\pgfpathlineto{\pgfqpoint{5.493417in}{4.433152in}}%
\pgfpathlineto{\pgfqpoint{5.496638in}{4.429846in}}%
\pgfpathlineto{\pgfqpoint{5.497712in}{4.431890in}}%
\pgfpathlineto{\pgfqpoint{5.498786in}{4.426179in}}%
\pgfpathlineto{\pgfqpoint{5.499860in}{4.425535in}}%
\pgfpathlineto{\pgfqpoint{5.500934in}{4.429741in}}%
\pgfpathlineto{\pgfqpoint{5.504155in}{4.428192in}}%
\pgfpathlineto{\pgfqpoint{5.505229in}{4.428788in}}%
\pgfpathlineto{\pgfqpoint{5.506303in}{4.423429in}}%
\pgfpathlineto{\pgfqpoint{5.507377in}{4.420988in}}%
\pgfpathlineto{\pgfqpoint{5.508451in}{4.424149in}}%
\pgfpathlineto{\pgfqpoint{5.511672in}{4.429922in}}%
\pgfpathlineto{\pgfqpoint{5.512746in}{4.429280in}}%
\pgfpathlineto{\pgfqpoint{5.513820in}{4.432653in}}%
\pgfpathlineto{\pgfqpoint{5.514894in}{4.427871in}}%
\pgfpathlineto{\pgfqpoint{5.515967in}{4.427871in}}%
\pgfpathlineto{\pgfqpoint{5.519189in}{4.416417in}}%
\pgfpathlineto{\pgfqpoint{5.520263in}{4.417598in}}%
\pgfpathlineto{\pgfqpoint{5.521337in}{4.413819in}}%
\pgfpathlineto{\pgfqpoint{5.522411in}{4.414863in}}%
\pgfpathlineto{\pgfqpoint{5.527780in}{4.411249in}}%
\pgfpathlineto{\pgfqpoint{5.528854in}{4.402387in}}%
\pgfpathlineto{\pgfqpoint{5.529927in}{4.404603in}}%
\pgfpathlineto{\pgfqpoint{5.531001in}{4.399298in}}%
\pgfpathlineto{\pgfqpoint{5.535297in}{4.390504in}}%
\pgfpathlineto{\pgfqpoint{5.537444in}{4.394609in}}%
\pgfpathlineto{\pgfqpoint{5.538518in}{4.393068in}}%
\pgfpathlineto{\pgfqpoint{5.541740in}{4.392631in}}%
\pgfpathlineto{\pgfqpoint{5.542813in}{4.394429in}}%
\pgfpathlineto{\pgfqpoint{5.543887in}{4.394649in}}%
\pgfpathlineto{\pgfqpoint{5.546035in}{4.378701in}}%
\pgfpathlineto{\pgfqpoint{5.549256in}{4.373641in}}%
\pgfpathlineto{\pgfqpoint{5.550330in}{4.374631in}}%
\pgfpathlineto{\pgfqpoint{5.552478in}{4.369441in}}%
\pgfpathlineto{\pgfqpoint{5.553552in}{4.369762in}}%
\pgfpathlineto{\pgfqpoint{5.556773in}{4.369387in}}%
\pgfpathlineto{\pgfqpoint{5.558921in}{4.367402in}}%
\pgfpathlineto{\pgfqpoint{5.559995in}{4.364453in}}%
\pgfpathlineto{\pgfqpoint{5.561069in}{4.388156in}}%
\pgfpathlineto{\pgfqpoint{5.564290in}{4.380380in}}%
\pgfpathlineto{\pgfqpoint{5.565364in}{4.380019in}}%
\pgfpathlineto{\pgfqpoint{5.566438in}{4.381667in}}%
\pgfpathlineto{\pgfqpoint{5.567512in}{4.380524in}}%
\pgfpathlineto{\pgfqpoint{5.568586in}{4.381251in}}%
\pgfpathlineto{\pgfqpoint{5.571807in}{4.380836in}}%
\pgfpathlineto{\pgfqpoint{5.572881in}{4.379644in}}%
\pgfpathlineto{\pgfqpoint{5.573955in}{4.381035in}}%
\pgfpathlineto{\pgfqpoint{5.576102in}{4.361947in}}%
\pgfpathlineto{\pgfqpoint{5.580398in}{4.344882in}}%
\pgfpathlineto{\pgfqpoint{5.581472in}{4.355152in}}%
\pgfpathlineto{\pgfqpoint{5.582545in}{4.349084in}}%
\pgfpathlineto{\pgfqpoint{5.583619in}{4.350188in}}%
\pgfpathlineto{\pgfqpoint{5.586841in}{4.350541in}}%
\pgfpathlineto{\pgfqpoint{5.587915in}{4.363417in}}%
\pgfpathlineto{\pgfqpoint{5.588989in}{4.368186in}}%
\pgfpathlineto{\pgfqpoint{5.590062in}{4.366560in}}%
\pgfpathlineto{\pgfqpoint{5.591136in}{4.372526in}}%
\pgfpathlineto{\pgfqpoint{5.597579in}{4.382851in}}%
\pgfpathlineto{\pgfqpoint{5.598653in}{4.381843in}}%
\pgfpathlineto{\pgfqpoint{5.601875in}{4.383738in}}%
\pgfpathlineto{\pgfqpoint{5.602948in}{4.390883in}}%
\pgfpathlineto{\pgfqpoint{5.604022in}{4.386551in}}%
\pgfpathlineto{\pgfqpoint{5.605096in}{4.387872in}}%
\pgfpathlineto{\pgfqpoint{5.606170in}{4.382107in}}%
\pgfpathlineto{\pgfqpoint{5.609391in}{4.386209in}}%
\pgfpathlineto{\pgfqpoint{5.610465in}{4.391587in}}%
\pgfpathlineto{\pgfqpoint{5.611539in}{4.391188in}}%
\pgfpathlineto{\pgfqpoint{5.612613in}{4.386176in}}%
\pgfpathlineto{\pgfqpoint{5.613687in}{4.388340in}}%
\pgfpathlineto{\pgfqpoint{5.616908in}{4.397737in}}%
\pgfpathlineto{\pgfqpoint{5.617982in}{4.396424in}}%
\pgfpathlineto{\pgfqpoint{5.619056in}{4.402153in}}%
\pgfpathlineto{\pgfqpoint{5.620130in}{4.403466in}}%
\pgfpathlineto{\pgfqpoint{5.621204in}{4.406349in}}%
\pgfpathlineto{\pgfqpoint{5.624425in}{4.390073in}}%
\pgfpathlineto{\pgfqpoint{5.627647in}{4.395048in}}%
\pgfpathlineto{\pgfqpoint{5.628721in}{4.395834in}}%
\pgfpathlineto{\pgfqpoint{5.631942in}{4.395946in}}%
\pgfpathlineto{\pgfqpoint{5.634090in}{4.389881in}}%
\pgfpathlineto{\pgfqpoint{5.635164in}{4.392914in}}%
\pgfpathlineto{\pgfqpoint{5.636237in}{4.390443in}}%
\pgfpathlineto{\pgfqpoint{5.639459in}{4.391497in}}%
\pgfpathlineto{\pgfqpoint{5.640533in}{4.391218in}}%
\pgfpathlineto{\pgfqpoint{5.641607in}{4.388932in}}%
\pgfpathlineto{\pgfqpoint{5.642680in}{4.367852in}}%
\pgfpathlineto{\pgfqpoint{5.643754in}{4.378838in}}%
\pgfpathlineto{\pgfqpoint{5.646976in}{4.380957in}}%
\pgfpathlineto{\pgfqpoint{5.648050in}{4.379999in}}%
\pgfpathlineto{\pgfqpoint{5.650197in}{4.383607in}}%
\pgfpathlineto{\pgfqpoint{5.651271in}{4.378501in}}%
\pgfpathlineto{\pgfqpoint{5.654493in}{4.382132in}}%
\pgfpathlineto{\pgfqpoint{5.656640in}{4.381452in}}%
\pgfpathlineto{\pgfqpoint{5.657714in}{4.382186in}}%
\pgfpathlineto{\pgfqpoint{5.662010in}{4.388769in}}%
\pgfpathlineto{\pgfqpoint{5.663083in}{4.388478in}}%
\pgfpathlineto{\pgfqpoint{5.664157in}{4.385980in}}%
\pgfpathlineto{\pgfqpoint{5.666305in}{4.376628in}}%
\pgfpathlineto{\pgfqpoint{5.669526in}{4.379126in}}%
\pgfpathlineto{\pgfqpoint{5.670600in}{4.382146in}}%
\pgfpathlineto{\pgfqpoint{5.671674in}{4.387928in}}%
\pgfpathlineto{\pgfqpoint{5.672748in}{4.384624in}}%
\pgfpathlineto{\pgfqpoint{5.673822in}{4.386423in}}%
\pgfpathlineto{\pgfqpoint{5.677043in}{4.385486in}}%
\pgfpathlineto{\pgfqpoint{5.678117in}{4.382090in}}%
\pgfpathlineto{\pgfqpoint{5.679191in}{4.382441in}}%
\pgfpathlineto{\pgfqpoint{5.681339in}{4.381563in}}%
\pgfpathlineto{\pgfqpoint{5.684560in}{4.382729in}}%
\pgfpathlineto{\pgfqpoint{5.685634in}{4.385837in}}%
\pgfpathlineto{\pgfqpoint{5.686708in}{4.391349in}}%
\pgfpathlineto{\pgfqpoint{5.687782in}{4.386888in}}%
\pgfpathlineto{\pgfqpoint{5.688856in}{4.390746in}}%
\pgfpathlineto{\pgfqpoint{5.692077in}{4.393519in}}%
\pgfpathlineto{\pgfqpoint{5.693151in}{4.389485in}}%
\pgfpathlineto{\pgfqpoint{5.694225in}{4.389180in}}%
\pgfpathlineto{\pgfqpoint{5.695299in}{4.390647in}}%
\pgfpathlineto{\pgfqpoint{5.696372in}{4.397859in}}%
\pgfpathlineto{\pgfqpoint{5.699594in}{4.397732in}}%
\pgfpathlineto{\pgfqpoint{5.700668in}{4.405587in}}%
\pgfpathlineto{\pgfqpoint{5.701742in}{4.402230in}}%
\pgfpathlineto{\pgfqpoint{5.702815in}{4.408962in}}%
\pgfpathlineto{\pgfqpoint{5.703889in}{4.398659in}}%
\pgfpathlineto{\pgfqpoint{5.707111in}{4.401879in}}%
\pgfpathlineto{\pgfqpoint{5.708185in}{4.399175in}}%
\pgfpathlineto{\pgfqpoint{5.709258in}{4.392437in}}%
\pgfpathlineto{\pgfqpoint{5.710332in}{4.391330in}}%
\pgfpathlineto{\pgfqpoint{5.714628in}{4.393655in}}%
\pgfpathlineto{\pgfqpoint{5.716775in}{4.400462in}}%
\pgfpathlineto{\pgfqpoint{5.717849in}{4.396834in}}%
\pgfpathlineto{\pgfqpoint{5.722144in}{4.385323in}}%
\pgfpathlineto{\pgfqpoint{5.723218in}{4.379192in}}%
\pgfpathlineto{\pgfqpoint{5.724292in}{4.368306in}}%
\pgfpathlineto{\pgfqpoint{5.725366in}{4.364865in}}%
\pgfpathlineto{\pgfqpoint{5.726440in}{4.363614in}}%
\pgfpathlineto{\pgfqpoint{5.729661in}{4.366116in}}%
\pgfpathlineto{\pgfqpoint{5.730735in}{4.364052in}}%
\pgfpathlineto{\pgfqpoint{5.731809in}{4.373149in}}%
\pgfpathlineto{\pgfqpoint{5.733957in}{4.378214in}}%
\pgfpathlineto{\pgfqpoint{5.738252in}{4.376571in}}%
\pgfpathlineto{\pgfqpoint{5.740400in}{4.381304in}}%
\pgfpathlineto{\pgfqpoint{5.741474in}{4.333602in}}%
\pgfpathlineto{\pgfqpoint{5.745769in}{4.333802in}}%
\pgfpathlineto{\pgfqpoint{5.746843in}{4.337394in}}%
\pgfpathlineto{\pgfqpoint{5.747917in}{4.326866in}}%
\pgfpathlineto{\pgfqpoint{5.748990in}{4.330806in}}%
\pgfpathlineto{\pgfqpoint{5.752212in}{4.323062in}}%
\pgfpathlineto{\pgfqpoint{5.753286in}{4.329425in}}%
\pgfpathlineto{\pgfqpoint{5.754360in}{4.332258in}}%
\pgfpathlineto{\pgfqpoint{5.755433in}{4.330504in}}%
\pgfpathlineto{\pgfqpoint{5.756507in}{4.333041in}}%
\pgfpathlineto{\pgfqpoint{5.759729in}{4.322879in}}%
\pgfpathlineto{\pgfqpoint{5.761877in}{4.326537in}}%
\pgfpathlineto{\pgfqpoint{5.762950in}{4.318851in}}%
\pgfpathlineto{\pgfqpoint{5.764024in}{4.328939in}}%
\pgfpathlineto{\pgfqpoint{5.768320in}{4.325576in}}%
\pgfpathlineto{\pgfqpoint{5.774763in}{4.310474in}}%
\pgfpathlineto{\pgfqpoint{5.775836in}{4.313485in}}%
\pgfpathlineto{\pgfqpoint{5.776910in}{4.310667in}}%
\pgfpathlineto{\pgfqpoint{5.777984in}{4.315112in}}%
\pgfpathlineto{\pgfqpoint{5.779058in}{4.315175in}}%
\pgfpathlineto{\pgfqpoint{5.782279in}{4.316365in}}%
\pgfpathlineto{\pgfqpoint{5.783353in}{4.309163in}}%
\pgfpathlineto{\pgfqpoint{5.784427in}{4.307304in}}%
\pgfpathlineto{\pgfqpoint{5.785501in}{4.301912in}}%
\pgfpathlineto{\pgfqpoint{5.789796in}{4.297011in}}%
\pgfpathlineto{\pgfqpoint{5.790870in}{4.294734in}}%
\pgfpathlineto{\pgfqpoint{5.791944in}{4.296704in}}%
\pgfpathlineto{\pgfqpoint{5.793018in}{4.295152in}}%
\pgfpathlineto{\pgfqpoint{5.794092in}{4.298866in}}%
\pgfpathlineto{\pgfqpoint{5.797313in}{4.298311in}}%
\pgfpathlineto{\pgfqpoint{5.799461in}{4.302593in}}%
\pgfpathlineto{\pgfqpoint{5.800535in}{4.300582in}}%
\pgfpathlineto{\pgfqpoint{5.801609in}{4.294680in}}%
\pgfpathlineto{\pgfqpoint{5.804830in}{4.294733in}}%
\pgfpathlineto{\pgfqpoint{5.805904in}{4.291758in}}%
\pgfpathlineto{\pgfqpoint{5.806978in}{4.291864in}}%
\pgfpathlineto{\pgfqpoint{5.808052in}{4.292714in}}%
\pgfpathlineto{\pgfqpoint{5.812347in}{4.291752in}}%
\pgfpathlineto{\pgfqpoint{5.814495in}{4.295012in}}%
\pgfpathlineto{\pgfqpoint{5.815568in}{4.300681in}}%
\pgfpathlineto{\pgfqpoint{5.816642in}{4.302193in}}%
\pgfpathlineto{\pgfqpoint{5.819864in}{4.299739in}}%
\pgfpathlineto{\pgfqpoint{5.820938in}{4.297504in}}%
\pgfpathlineto{\pgfqpoint{5.822011in}{4.297395in}}%
\pgfpathlineto{\pgfqpoint{5.823085in}{4.292107in}}%
\pgfpathlineto{\pgfqpoint{5.824159in}{4.295433in}}%
\pgfpathlineto{\pgfqpoint{5.827381in}{4.293906in}}%
\pgfpathlineto{\pgfqpoint{5.828455in}{4.289372in}}%
\pgfpathlineto{\pgfqpoint{5.829528in}{4.281983in}}%
\pgfpathlineto{\pgfqpoint{5.830602in}{4.280336in}}%
\pgfpathlineto{\pgfqpoint{5.831676in}{4.282063in}}%
\pgfpathlineto{\pgfqpoint{5.834898in}{4.284210in}}%
\pgfpathlineto{\pgfqpoint{5.838119in}{4.270346in}}%
\pgfpathlineto{\pgfqpoint{5.839193in}{4.268938in}}%
\pgfpathlineto{\pgfqpoint{5.842414in}{4.269927in}}%
\pgfpathlineto{\pgfqpoint{5.843488in}{4.272101in}}%
\pgfpathlineto{\pgfqpoint{5.844562in}{4.272010in}}%
\pgfpathlineto{\pgfqpoint{5.846710in}{4.268571in}}%
\pgfpathlineto{\pgfqpoint{5.849931in}{4.269657in}}%
\pgfpathlineto{\pgfqpoint{5.851005in}{4.273232in}}%
\pgfpathlineto{\pgfqpoint{5.853153in}{4.269147in}}%
\pgfpathlineto{\pgfqpoint{5.854227in}{4.271747in}}%
\pgfpathlineto{\pgfqpoint{5.857448in}{4.274068in}}%
\pgfpathlineto{\pgfqpoint{5.858522in}{4.277890in}}%
\pgfpathlineto{\pgfqpoint{5.859596in}{4.278975in}}%
\pgfpathlineto{\pgfqpoint{5.861744in}{4.276836in}}%
\pgfpathlineto{\pgfqpoint{5.864965in}{4.276598in}}%
\pgfpathlineto{\pgfqpoint{5.866039in}{4.270704in}}%
\pgfpathlineto{\pgfqpoint{5.868187in}{4.277644in}}%
\pgfpathlineto{\pgfqpoint{5.869260in}{4.283100in}}%
\pgfpathlineto{\pgfqpoint{5.872482in}{4.284610in}}%
\pgfpathlineto{\pgfqpoint{5.873556in}{4.290565in}}%
\pgfpathlineto{\pgfqpoint{5.874630in}{4.288498in}}%
\pgfpathlineto{\pgfqpoint{5.875703in}{4.288887in}}%
\pgfpathlineto{\pgfqpoint{5.876777in}{4.290249in}}%
\pgfpathlineto{\pgfqpoint{5.882146in}{4.288453in}}%
\pgfpathlineto{\pgfqpoint{5.883220in}{4.286193in}}%
\pgfpathlineto{\pgfqpoint{5.884294in}{4.290315in}}%
\pgfpathlineto{\pgfqpoint{5.889663in}{4.294361in}}%
\pgfpathlineto{\pgfqpoint{5.890737in}{4.296898in}}%
\pgfpathlineto{\pgfqpoint{5.891811in}{4.293426in}}%
\pgfpathlineto{\pgfqpoint{5.895032in}{4.287275in}}%
\pgfpathlineto{\pgfqpoint{5.896106in}{4.275024in}}%
\pgfpathlineto{\pgfqpoint{5.897180in}{4.276661in}}%
\pgfpathlineto{\pgfqpoint{5.898254in}{4.274231in}}%
\pgfpathlineto{\pgfqpoint{5.899328in}{4.274572in}}%
\pgfpathlineto{\pgfqpoint{5.903623in}{4.276674in}}%
\pgfpathlineto{\pgfqpoint{5.904697in}{4.275109in}}%
\pgfpathlineto{\pgfqpoint{5.905771in}{4.281125in}}%
\pgfpathlineto{\pgfqpoint{5.906845in}{4.295943in}}%
\pgfpathlineto{\pgfqpoint{5.910066in}{4.283663in}}%
\pgfpathlineto{\pgfqpoint{5.911140in}{4.284853in}}%
\pgfpathlineto{\pgfqpoint{5.913288in}{4.268965in}}%
\pgfpathlineto{\pgfqpoint{5.914362in}{4.269305in}}%
\pgfpathlineto{\pgfqpoint{5.918657in}{4.262150in}}%
\pgfpathlineto{\pgfqpoint{5.919731in}{4.263707in}}%
\pgfpathlineto{\pgfqpoint{5.920805in}{4.262199in}}%
\pgfpathlineto{\pgfqpoint{5.921878in}{4.254495in}}%
\pgfpathlineto{\pgfqpoint{5.925100in}{4.251582in}}%
\pgfpathlineto{\pgfqpoint{5.926174in}{4.247975in}}%
\pgfpathlineto{\pgfqpoint{5.927248in}{4.247585in}}%
\pgfpathlineto{\pgfqpoint{5.928321in}{4.245467in}}%
\pgfpathlineto{\pgfqpoint{5.929395in}{4.244787in}}%
\pgfpathlineto{\pgfqpoint{5.933691in}{4.243269in}}%
\pgfpathlineto{\pgfqpoint{5.934765in}{4.241974in}}%
\pgfpathlineto{\pgfqpoint{5.936912in}{4.249573in}}%
\pgfpathlineto{\pgfqpoint{5.941208in}{4.249231in}}%
\pgfpathlineto{\pgfqpoint{5.943355in}{4.247489in}}%
\pgfpathlineto{\pgfqpoint{5.944429in}{4.248835in}}%
\pgfpathlineto{\pgfqpoint{5.947651in}{4.247517in}}%
\pgfpathlineto{\pgfqpoint{5.948724in}{4.244840in}}%
\pgfpathlineto{\pgfqpoint{5.949798in}{4.246285in}}%
\pgfpathlineto{\pgfqpoint{5.950872in}{4.245860in}}%
\pgfpathlineto{\pgfqpoint{5.951946in}{4.239464in}}%
\pgfpathlineto{\pgfqpoint{5.955167in}{4.238254in}}%
\pgfpathlineto{\pgfqpoint{5.956241in}{4.239931in}}%
\pgfpathlineto{\pgfqpoint{5.957315in}{4.237422in}}%
\pgfpathlineto{\pgfqpoint{5.958389in}{4.240255in}}%
\pgfpathlineto{\pgfqpoint{5.962684in}{4.241593in}}%
\pgfpathlineto{\pgfqpoint{5.964832in}{4.244409in}}%
\pgfpathlineto{\pgfqpoint{5.965906in}{4.244614in}}%
\pgfpathlineto{\pgfqpoint{5.970201in}{4.243546in}}%
\pgfpathlineto{\pgfqpoint{5.972349in}{4.246873in}}%
\pgfpathlineto{\pgfqpoint{5.974497in}{4.245412in}}%
\pgfpathlineto{\pgfqpoint{5.978792in}{4.248542in}}%
\pgfpathlineto{\pgfqpoint{5.979866in}{4.249043in}}%
\pgfpathlineto{\pgfqpoint{5.982013in}{4.253296in}}%
\pgfpathlineto{\pgfqpoint{5.986309in}{4.250988in}}%
\pgfpathlineto{\pgfqpoint{5.987383in}{4.249266in}}%
\pgfpathlineto{\pgfqpoint{5.988456in}{4.249058in}}%
\pgfpathlineto{\pgfqpoint{5.989530in}{4.253487in}}%
\pgfpathlineto{\pgfqpoint{5.992752in}{4.256699in}}%
\pgfpathlineto{\pgfqpoint{5.993826in}{4.263036in}}%
\pgfpathlineto{\pgfqpoint{5.994899in}{4.259266in}}%
\pgfpathlineto{\pgfqpoint{5.997047in}{4.261510in}}%
\pgfpathlineto{\pgfqpoint{6.000269in}{4.263224in}}%
\pgfpathlineto{\pgfqpoint{6.001343in}{4.264713in}}%
\pgfpathlineto{\pgfqpoint{6.002416in}{4.267176in}}%
\pgfpathlineto{\pgfqpoint{6.003490in}{4.265671in}}%
\pgfpathlineto{\pgfqpoint{6.004564in}{4.269009in}}%
\pgfpathlineto{\pgfqpoint{6.007786in}{4.267115in}}%
\pgfpathlineto{\pgfqpoint{6.008859in}{4.270904in}}%
\pgfpathlineto{\pgfqpoint{6.009933in}{4.271181in}}%
\pgfpathlineto{\pgfqpoint{6.011007in}{4.267431in}}%
\pgfpathlineto{\pgfqpoint{6.012081in}{4.270394in}}%
\pgfpathlineto{\pgfqpoint{6.016376in}{4.271879in}}%
\pgfpathlineto{\pgfqpoint{6.017450in}{4.268475in}}%
\pgfpathlineto{\pgfqpoint{6.018524in}{4.279039in}}%
\pgfpathlineto{\pgfqpoint{6.019598in}{4.278204in}}%
\pgfpathlineto{\pgfqpoint{6.022819in}{4.278843in}}%
\pgfpathlineto{\pgfqpoint{6.023893in}{4.283703in}}%
\pgfpathlineto{\pgfqpoint{6.026041in}{4.281469in}}%
\pgfpathlineto{\pgfqpoint{6.027115in}{4.280200in}}%
\pgfpathlineto{\pgfqpoint{6.030336in}{4.278982in}}%
\pgfpathlineto{\pgfqpoint{6.031410in}{4.279896in}}%
\pgfpathlineto{\pgfqpoint{6.032484in}{4.274212in}}%
\pgfpathlineto{\pgfqpoint{6.033558in}{4.248288in}}%
\pgfpathlineto{\pgfqpoint{6.034632in}{4.246048in}}%
\pgfpathlineto{\pgfqpoint{6.037853in}{4.247089in}}%
\pgfpathlineto{\pgfqpoint{6.038927in}{4.245757in}}%
\pgfpathlineto{\pgfqpoint{6.040001in}{4.245999in}}%
\pgfpathlineto{\pgfqpoint{6.041075in}{4.245515in}}%
\pgfpathlineto{\pgfqpoint{6.042148in}{4.247365in}}%
\pgfpathlineto{\pgfqpoint{6.045370in}{4.247242in}}%
\pgfpathlineto{\pgfqpoint{6.046444in}{4.246590in}}%
\pgfpathlineto{\pgfqpoint{6.047518in}{4.243449in}}%
\pgfpathlineto{\pgfqpoint{6.048591in}{4.243000in}}%
\pgfpathlineto{\pgfqpoint{6.049665in}{4.243693in}}%
\pgfpathlineto{\pgfqpoint{6.052887in}{4.237861in}}%
\pgfpathlineto{\pgfqpoint{6.053961in}{4.237627in}}%
\pgfpathlineto{\pgfqpoint{6.056108in}{4.226516in}}%
\pgfpathlineto{\pgfqpoint{6.057182in}{4.225164in}}%
\pgfpathlineto{\pgfqpoint{6.060404in}{4.218694in}}%
\pgfpathlineto{\pgfqpoint{6.061477in}{4.218528in}}%
\pgfpathlineto{\pgfqpoint{6.062551in}{4.221055in}}%
\pgfpathlineto{\pgfqpoint{6.063625in}{4.221361in}}%
\pgfpathlineto{\pgfqpoint{6.064699in}{4.223876in}}%
\pgfpathlineto{\pgfqpoint{6.067920in}{4.225647in}}%
\pgfpathlineto{\pgfqpoint{6.068994in}{4.225022in}}%
\pgfpathlineto{\pgfqpoint{6.070068in}{4.221188in}}%
\pgfpathlineto{\pgfqpoint{6.072216in}{4.221255in}}%
\pgfpathlineto{\pgfqpoint{6.075437in}{4.218912in}}%
\pgfpathlineto{\pgfqpoint{6.076511in}{4.216870in}}%
\pgfpathlineto{\pgfqpoint{6.077585in}{4.218611in}}%
\pgfpathlineto{\pgfqpoint{6.078659in}{4.217038in}}%
\pgfpathlineto{\pgfqpoint{6.079733in}{4.219151in}}%
\pgfpathlineto{\pgfqpoint{6.082954in}{4.219689in}}%
\pgfpathlineto{\pgfqpoint{6.084028in}{4.218781in}}%
\pgfpathlineto{\pgfqpoint{6.085102in}{4.212441in}}%
\pgfpathlineto{\pgfqpoint{6.086176in}{4.210546in}}%
\pgfpathlineto{\pgfqpoint{6.087250in}{4.210981in}}%
\pgfpathlineto{\pgfqpoint{6.090471in}{4.207426in}}%
\pgfpathlineto{\pgfqpoint{6.091545in}{4.208205in}}%
\pgfpathlineto{\pgfqpoint{6.092619in}{4.207519in}}%
\pgfpathlineto{\pgfqpoint{6.093693in}{4.204916in}}%
\pgfpathlineto{\pgfqpoint{6.094766in}{4.204765in}}%
\pgfpathlineto{\pgfqpoint{6.097988in}{4.206065in}}%
\pgfpathlineto{\pgfqpoint{6.099062in}{4.207563in}}%
\pgfpathlineto{\pgfqpoint{6.100136in}{4.206799in}}%
\pgfpathlineto{\pgfqpoint{6.102283in}{4.210132in}}%
\pgfpathlineto{\pgfqpoint{6.106579in}{4.210101in}}%
\pgfpathlineto{\pgfqpoint{6.108726in}{4.206943in}}%
\pgfpathlineto{\pgfqpoint{6.109800in}{4.207407in}}%
\pgfpathlineto{\pgfqpoint{6.114096in}{4.203629in}}%
\pgfpathlineto{\pgfqpoint{6.115169in}{4.200068in}}%
\pgfpathlineto{\pgfqpoint{6.116243in}{4.202703in}}%
\pgfpathlineto{\pgfqpoint{6.117317in}{4.203118in}}%
\pgfpathlineto{\pgfqpoint{6.120539in}{4.201991in}}%
\pgfpathlineto{\pgfqpoint{6.121612in}{4.199758in}}%
\pgfpathlineto{\pgfqpoint{6.122686in}{4.199039in}}%
\pgfpathlineto{\pgfqpoint{6.123760in}{4.199125in}}%
\pgfpathlineto{\pgfqpoint{6.124834in}{4.198410in}}%
\pgfpathlineto{\pgfqpoint{6.129129in}{4.198353in}}%
\pgfpathlineto{\pgfqpoint{6.131277in}{4.203014in}}%
\pgfpathlineto{\pgfqpoint{6.132351in}{4.201642in}}%
\pgfpathlineto{\pgfqpoint{6.135572in}{4.201000in}}%
\pgfpathlineto{\pgfqpoint{6.136646in}{4.205087in}}%
\pgfpathlineto{\pgfqpoint{6.137720in}{4.206605in}}%
\pgfpathlineto{\pgfqpoint{6.139868in}{4.206960in}}%
\pgfpathlineto{\pgfqpoint{6.143089in}{4.208206in}}%
\pgfpathlineto{\pgfqpoint{6.144163in}{4.210787in}}%
\pgfpathlineto{\pgfqpoint{6.146311in}{4.212639in}}%
\pgfpathlineto{\pgfqpoint{6.147385in}{4.217133in}}%
\pgfpathlineto{\pgfqpoint{6.152754in}{4.217713in}}%
\pgfpathlineto{\pgfqpoint{6.154901in}{4.215693in}}%
\pgfpathlineto{\pgfqpoint{6.158123in}{4.214455in}}%
\pgfpathlineto{\pgfqpoint{6.160271in}{4.212460in}}%
\pgfpathlineto{\pgfqpoint{6.162418in}{4.213283in}}%
\pgfpathlineto{\pgfqpoint{6.167787in}{4.212840in}}%
\pgfpathlineto{\pgfqpoint{6.174231in}{4.215635in}}%
\pgfpathlineto{\pgfqpoint{6.176378in}{4.226113in}}%
\pgfpathlineto{\pgfqpoint{6.177452in}{4.225463in}}%
\pgfpathlineto{\pgfqpoint{6.180674in}{4.224319in}}%
\pgfpathlineto{\pgfqpoint{6.181747in}{4.224566in}}%
\pgfpathlineto{\pgfqpoint{6.183895in}{4.226959in}}%
\pgfpathlineto{\pgfqpoint{6.184969in}{4.226740in}}%
\pgfpathlineto{\pgfqpoint{6.188190in}{4.227240in}}%
\pgfpathlineto{\pgfqpoint{6.192486in}{4.233313in}}%
\pgfpathlineto{\pgfqpoint{6.195707in}{4.232057in}}%
\pgfpathlineto{\pgfqpoint{6.196781in}{4.227965in}}%
\pgfpathlineto{\pgfqpoint{6.197855in}{4.228417in}}%
\pgfpathlineto{\pgfqpoint{6.200003in}{4.226778in}}%
\pgfpathlineto{\pgfqpoint{6.203224in}{4.228622in}}%
\pgfpathlineto{\pgfqpoint{6.204298in}{4.231849in}}%
\pgfpathlineto{\pgfqpoint{6.205372in}{4.233075in}}%
\pgfpathlineto{\pgfqpoint{6.206446in}{4.236107in}}%
\pgfpathlineto{\pgfqpoint{6.207520in}{4.236433in}}%
\pgfpathlineto{\pgfqpoint{6.213963in}{4.233688in}}%
\pgfpathlineto{\pgfqpoint{6.215036in}{4.234145in}}%
\pgfpathlineto{\pgfqpoint{6.219332in}{4.232538in}}%
\pgfpathlineto{\pgfqpoint{6.221479in}{4.227879in}}%
\pgfpathlineto{\pgfqpoint{6.225775in}{4.230635in}}%
\pgfpathlineto{\pgfqpoint{6.226849in}{4.233423in}}%
\pgfpathlineto{\pgfqpoint{6.227922in}{4.232651in}}%
\pgfpathlineto{\pgfqpoint{6.228996in}{4.247186in}}%
\pgfpathlineto{\pgfqpoint{6.230070in}{4.248595in}}%
\pgfpathlineto{\pgfqpoint{6.233292in}{4.251412in}}%
\pgfpathlineto{\pgfqpoint{6.234365in}{4.250801in}}%
\pgfpathlineto{\pgfqpoint{6.235439in}{4.251172in}}%
\pgfpathlineto{\pgfqpoint{6.236513in}{4.250563in}}%
\pgfpathlineto{\pgfqpoint{6.237587in}{4.247006in}}%
\pgfpathlineto{\pgfqpoint{6.240809in}{4.246939in}}%
\pgfpathlineto{\pgfqpoint{6.241882in}{4.247955in}}%
\pgfpathlineto{\pgfqpoint{6.242956in}{4.250258in}}%
\pgfpathlineto{\pgfqpoint{6.244030in}{4.248569in}}%
\pgfpathlineto{\pgfqpoint{6.250473in}{4.248258in}}%
\pgfpathlineto{\pgfqpoint{6.251547in}{4.250693in}}%
\pgfpathlineto{\pgfqpoint{6.252621in}{4.249261in}}%
\pgfpathlineto{\pgfqpoint{6.256916in}{4.252755in}}%
\pgfpathlineto{\pgfqpoint{6.257990in}{4.246750in}}%
\pgfpathlineto{\pgfqpoint{6.259064in}{4.246785in}}%
\pgfpathlineto{\pgfqpoint{6.260138in}{4.248225in}}%
\pgfpathlineto{\pgfqpoint{6.263359in}{4.247628in}}%
\pgfpathlineto{\pgfqpoint{6.264433in}{4.246859in}}%
\pgfpathlineto{\pgfqpoint{6.265507in}{4.247137in}}%
\pgfpathlineto{\pgfqpoint{6.266581in}{4.248216in}}%
\pgfpathlineto{\pgfqpoint{6.267654in}{4.248147in}}%
\pgfpathlineto{\pgfqpoint{6.273024in}{4.246406in}}%
\pgfpathlineto{\pgfqpoint{6.274097in}{4.250896in}}%
\pgfpathlineto{\pgfqpoint{6.278393in}{4.248573in}}%
\pgfpathlineto{\pgfqpoint{6.279467in}{4.248984in}}%
\pgfpathlineto{\pgfqpoint{6.280541in}{4.252174in}}%
\pgfpathlineto{\pgfqpoint{6.281614in}{4.251694in}}%
\pgfpathlineto{\pgfqpoint{6.282688in}{4.250497in}}%
\pgfpathlineto{\pgfqpoint{6.285910in}{4.250971in}}%
\pgfpathlineto{\pgfqpoint{6.286984in}{4.252365in}}%
\pgfpathlineto{\pgfqpoint{6.288057in}{4.251549in}}%
\pgfpathlineto{\pgfqpoint{6.289131in}{4.252022in}}%
\pgfpathlineto{\pgfqpoint{6.290205in}{4.254497in}}%
\pgfpathlineto{\pgfqpoint{6.293427in}{4.253073in}}%
\pgfpathlineto{\pgfqpoint{6.294500in}{4.251027in}}%
\pgfpathlineto{\pgfqpoint{6.295574in}{4.251985in}}%
\pgfpathlineto{\pgfqpoint{6.297722in}{4.252018in}}%
\pgfpathlineto{\pgfqpoint{6.300943in}{4.254381in}}%
\pgfpathlineto{\pgfqpoint{6.302017in}{4.253982in}}%
\pgfpathlineto{\pgfqpoint{6.303091in}{4.251128in}}%
\pgfpathlineto{\pgfqpoint{6.304165in}{4.252200in}}%
\pgfpathlineto{\pgfqpoint{6.305239in}{4.254132in}}%
\pgfpathlineto{\pgfqpoint{6.308460in}{4.251414in}}%
\pgfpathlineto{\pgfqpoint{6.310608in}{4.251062in}}%
\pgfpathlineto{\pgfqpoint{6.312756in}{4.256551in}}%
\pgfpathlineto{\pgfqpoint{6.317051in}{4.256189in}}%
\pgfpathlineto{\pgfqpoint{6.318125in}{4.258295in}}%
\pgfpathlineto{\pgfqpoint{6.319199in}{4.256518in}}%
\pgfpathlineto{\pgfqpoint{6.320273in}{4.256811in}}%
\pgfpathlineto{\pgfqpoint{6.323494in}{4.256615in}}%
\pgfpathlineto{\pgfqpoint{6.324568in}{4.257559in}}%
\pgfpathlineto{\pgfqpoint{6.325642in}{4.256257in}}%
\pgfpathlineto{\pgfqpoint{6.327789in}{4.258814in}}%
\pgfpathlineto{\pgfqpoint{6.331011in}{4.260675in}}%
\pgfpathlineto{\pgfqpoint{6.332085in}{4.261967in}}%
\pgfpathlineto{\pgfqpoint{6.333159in}{4.262298in}}%
\pgfpathlineto{\pgfqpoint{6.334232in}{4.257618in}}%
\pgfpathlineto{\pgfqpoint{6.335306in}{4.260671in}}%
\pgfpathlineto{\pgfqpoint{6.339602in}{4.259244in}}%
\pgfpathlineto{\pgfqpoint{6.340675in}{4.259441in}}%
\pgfpathlineto{\pgfqpoint{6.342823in}{4.257314in}}%
\pgfpathlineto{\pgfqpoint{6.347119in}{4.257994in}}%
\pgfpathlineto{\pgfqpoint{6.348192in}{4.258772in}}%
\pgfpathlineto{\pgfqpoint{6.349266in}{4.255184in}}%
\pgfpathlineto{\pgfqpoint{6.350340in}{4.254336in}}%
\pgfpathlineto{\pgfqpoint{6.353562in}{4.258055in}}%
\pgfpathlineto{\pgfqpoint{6.354635in}{4.253847in}}%
\pgfpathlineto{\pgfqpoint{6.355709in}{4.252013in}}%
\pgfpathlineto{\pgfqpoint{6.356783in}{4.255943in}}%
\pgfpathlineto{\pgfqpoint{6.357857in}{4.253726in}}%
\pgfpathlineto{\pgfqpoint{6.365374in}{4.253340in}}%
\pgfpathlineto{\pgfqpoint{6.369669in}{4.253564in}}%
\pgfpathlineto{\pgfqpoint{6.370743in}{4.254431in}}%
\pgfpathlineto{\pgfqpoint{6.371817in}{4.253179in}}%
\pgfpathlineto{\pgfqpoint{6.372891in}{4.253052in}}%
\pgfpathlineto{\pgfqpoint{6.378260in}{4.250257in}}%
\pgfpathlineto{\pgfqpoint{6.379334in}{4.247715in}}%
\pgfpathlineto{\pgfqpoint{6.380408in}{4.247239in}}%
\pgfpathlineto{\pgfqpoint{6.383629in}{4.251686in}}%
\pgfpathlineto{\pgfqpoint{6.384703in}{4.249049in}}%
\pgfpathlineto{\pgfqpoint{6.385777in}{4.248769in}}%
\pgfpathlineto{\pgfqpoint{6.386851in}{4.250136in}}%
\pgfpathlineto{\pgfqpoint{6.387924in}{4.252586in}}%
\pgfpathlineto{\pgfqpoint{6.391146in}{4.249917in}}%
\pgfpathlineto{\pgfqpoint{6.392220in}{4.246531in}}%
\pgfpathlineto{\pgfqpoint{6.393294in}{4.248181in}}%
\pgfpathlineto{\pgfqpoint{6.398663in}{4.247513in}}%
\pgfpathlineto{\pgfqpoint{6.399737in}{4.248940in}}%
\pgfpathlineto{\pgfqpoint{6.401884in}{4.244691in}}%
\pgfpathlineto{\pgfqpoint{6.402958in}{4.243780in}}%
\pgfpathlineto{\pgfqpoint{6.406180in}{4.243546in}}%
\pgfpathlineto{\pgfqpoint{6.407253in}{4.241040in}}%
\pgfpathlineto{\pgfqpoint{6.409401in}{4.243625in}}%
\pgfpathlineto{\pgfqpoint{6.410475in}{4.242529in}}%
\pgfpathlineto{\pgfqpoint{6.415844in}{4.241389in}}%
\pgfpathlineto{\pgfqpoint{6.416918in}{4.242352in}}%
\pgfpathlineto{\pgfqpoint{6.417992in}{4.245836in}}%
\pgfpathlineto{\pgfqpoint{6.421213in}{4.248349in}}%
\pgfpathlineto{\pgfqpoint{6.422287in}{4.247592in}}%
\pgfpathlineto{\pgfqpoint{6.424435in}{4.249223in}}%
\pgfpathlineto{\pgfqpoint{6.425509in}{4.249778in}}%
\pgfpathlineto{\pgfqpoint{6.428730in}{4.248960in}}%
\pgfpathlineto{\pgfqpoint{6.429804in}{4.244776in}}%
\pgfpathlineto{\pgfqpoint{6.430878in}{4.245676in}}%
\pgfpathlineto{\pgfqpoint{6.431952in}{4.251655in}}%
\pgfpathlineto{\pgfqpoint{6.433026in}{4.251372in}}%
\pgfpathlineto{\pgfqpoint{6.437321in}{4.254643in}}%
\pgfpathlineto{\pgfqpoint{6.440542in}{4.252103in}}%
\pgfpathlineto{\pgfqpoint{6.443764in}{4.252497in}}%
\pgfpathlineto{\pgfqpoint{6.444838in}{4.249983in}}%
\pgfpathlineto{\pgfqpoint{6.445912in}{4.249305in}}%
\pgfpathlineto{\pgfqpoint{6.446985in}{4.245519in}}%
\pgfpathlineto{\pgfqpoint{6.448059in}{4.244869in}}%
\pgfpathlineto{\pgfqpoint{6.452355in}{4.246282in}}%
\pgfpathlineto{\pgfqpoint{6.453429in}{4.244894in}}%
\pgfpathlineto{\pgfqpoint{6.454502in}{4.245715in}}%
\pgfpathlineto{\pgfqpoint{6.459872in}{4.243376in}}%
\pgfpathlineto{\pgfqpoint{6.460945in}{4.244962in}}%
\pgfpathlineto{\pgfqpoint{6.463093in}{4.244005in}}%
\pgfpathlineto{\pgfqpoint{6.466315in}{4.243976in}}%
\pgfpathlineto{\pgfqpoint{6.467388in}{4.248962in}}%
\pgfpathlineto{\pgfqpoint{6.469536in}{4.242370in}}%
\pgfpathlineto{\pgfqpoint{6.470610in}{4.241968in}}%
\pgfpathlineto{\pgfqpoint{6.474905in}{4.239784in}}%
\pgfpathlineto{\pgfqpoint{6.477053in}{4.241997in}}%
\pgfpathlineto{\pgfqpoint{6.478127in}{4.242076in}}%
\pgfpathlineto{\pgfqpoint{6.481348in}{4.243243in}}%
\pgfpathlineto{\pgfqpoint{6.482422in}{4.242289in}}%
\pgfpathlineto{\pgfqpoint{6.483496in}{4.246368in}}%
\pgfpathlineto{\pgfqpoint{6.484570in}{4.244683in}}%
\pgfpathlineto{\pgfqpoint{6.485644in}{4.248325in}}%
\pgfpathlineto{\pgfqpoint{6.488865in}{4.245254in}}%
\pgfpathlineto{\pgfqpoint{6.489939in}{4.246634in}}%
\pgfpathlineto{\pgfqpoint{6.491013in}{4.244971in}}%
\pgfpathlineto{\pgfqpoint{6.492087in}{4.244944in}}%
\pgfpathlineto{\pgfqpoint{6.493161in}{4.245561in}}%
\pgfpathlineto{\pgfqpoint{6.497456in}{4.245990in}}%
\pgfpathlineto{\pgfqpoint{6.498530in}{4.247444in}}%
\pgfpathlineto{\pgfqpoint{6.499604in}{4.245937in}}%
\pgfpathlineto{\pgfqpoint{6.500677in}{4.246948in}}%
\pgfpathlineto{\pgfqpoint{6.500677in}{4.246948in}}%
\pgfusepath{stroke}%
\end{pgfscope}%
\begin{pgfscope}%
\pgfsetrectcap%
\pgfsetmiterjoin%
\pgfsetlinewidth{0.803000pt}%
\definecolor{currentstroke}{rgb}{1.000000,1.000000,1.000000}%
\pgfsetstrokecolor{currentstroke}%
\pgfsetdash{}{0pt}%
\pgfpathmoveto{\pgfqpoint{4.034768in}{4.180131in}}%
\pgfpathlineto{\pgfqpoint{4.034768in}{4.581016in}}%
\pgfusepath{stroke}%
\end{pgfscope}%
\begin{pgfscope}%
\pgfsetrectcap%
\pgfsetmiterjoin%
\pgfsetlinewidth{0.803000pt}%
\definecolor{currentstroke}{rgb}{1.000000,1.000000,1.000000}%
\pgfsetstrokecolor{currentstroke}%
\pgfsetdash{}{0pt}%
\pgfpathmoveto{\pgfqpoint{6.618102in}{4.180131in}}%
\pgfpathlineto{\pgfqpoint{6.618102in}{4.581016in}}%
\pgfusepath{stroke}%
\end{pgfscope}%
\begin{pgfscope}%
\pgfsetrectcap%
\pgfsetmiterjoin%
\pgfsetlinewidth{0.803000pt}%
\definecolor{currentstroke}{rgb}{1.000000,1.000000,1.000000}%
\pgfsetstrokecolor{currentstroke}%
\pgfsetdash{}{0pt}%
\pgfpathmoveto{\pgfqpoint{4.034768in}{4.180131in}}%
\pgfpathlineto{\pgfqpoint{6.618102in}{4.180131in}}%
\pgfusepath{stroke}%
\end{pgfscope}%
\begin{pgfscope}%
\pgfsetrectcap%
\pgfsetmiterjoin%
\pgfsetlinewidth{0.803000pt}%
\definecolor{currentstroke}{rgb}{1.000000,1.000000,1.000000}%
\pgfsetstrokecolor{currentstroke}%
\pgfsetdash{}{0pt}%
\pgfpathmoveto{\pgfqpoint{4.034768in}{4.581016in}}%
\pgfpathlineto{\pgfqpoint{6.618102in}{4.581016in}}%
\pgfusepath{stroke}%
\end{pgfscope}%
\begin{pgfscope}%
\definecolor{textcolor}{rgb}{0.150000,0.150000,0.150000}%
\pgfsetstrokecolor{textcolor}%
\pgfsetfillcolor{textcolor}%
\pgftext[x=5.326435in,y=4.664349in,,base]{\color{textcolor}\rmfamily\fontsize{12.000000}{14.400000}\selectfont AXP}%
\end{pgfscope}%
\begin{pgfscope}%
\pgfsetbuttcap%
\pgfsetmiterjoin%
\definecolor{currentfill}{rgb}{0.917647,0.917647,0.949020}%
\pgfsetfillcolor{currentfill}%
\pgfsetlinewidth{0.000000pt}%
\definecolor{currentstroke}{rgb}{0.000000,0.000000,0.000000}%
\pgfsetstrokecolor{currentstroke}%
\pgfsetstrokeopacity{0.000000}%
\pgfsetdash{}{0pt}%
\pgfpathmoveto{\pgfqpoint{0.418102in}{3.218007in}}%
\pgfpathlineto{\pgfqpoint{3.001435in}{3.218007in}}%
\pgfpathlineto{\pgfqpoint{3.001435in}{3.618892in}}%
\pgfpathlineto{\pgfqpoint{0.418102in}{3.618892in}}%
\pgfpathclose%
\pgfusepath{fill}%
\end{pgfscope}%
\begin{pgfscope}%
\pgfpathrectangle{\pgfqpoint{0.418102in}{3.218007in}}{\pgfqpoint{2.583333in}{0.400885in}}%
\pgfusepath{clip}%
\pgfsetroundcap%
\pgfsetroundjoin%
\pgfsetlinewidth{0.803000pt}%
\definecolor{currentstroke}{rgb}{1.000000,1.000000,1.000000}%
\pgfsetstrokecolor{currentstroke}%
\pgfsetdash{}{0pt}%
\pgfpathmoveto{\pgfqpoint{0.533378in}{3.218007in}}%
\pgfpathlineto{\pgfqpoint{0.533378in}{3.618892in}}%
\pgfusepath{stroke}%
\end{pgfscope}%
\begin{pgfscope}%
\definecolor{textcolor}{rgb}{0.150000,0.150000,0.150000}%
\pgfsetstrokecolor{textcolor}%
\pgfsetfillcolor{textcolor}%
\pgftext[x=0.533378in,y=3.120784in,,top]{\color{textcolor}\rmfamily\fontsize{10.000000}{12.000000}\selectfont 2012}%
\end{pgfscope}%
\begin{pgfscope}%
\pgfpathrectangle{\pgfqpoint{0.418102in}{3.218007in}}{\pgfqpoint{2.583333in}{0.400885in}}%
\pgfusepath{clip}%
\pgfsetroundcap%
\pgfsetroundjoin%
\pgfsetlinewidth{0.803000pt}%
\definecolor{currentstroke}{rgb}{1.000000,1.000000,1.000000}%
\pgfsetstrokecolor{currentstroke}%
\pgfsetdash{}{0pt}%
\pgfpathmoveto{\pgfqpoint{0.926403in}{3.218007in}}%
\pgfpathlineto{\pgfqpoint{0.926403in}{3.618892in}}%
\pgfusepath{stroke}%
\end{pgfscope}%
\begin{pgfscope}%
\definecolor{textcolor}{rgb}{0.150000,0.150000,0.150000}%
\pgfsetstrokecolor{textcolor}%
\pgfsetfillcolor{textcolor}%
\pgftext[x=0.926403in,y=3.120784in,,top]{\color{textcolor}\rmfamily\fontsize{10.000000}{12.000000}\selectfont 2013}%
\end{pgfscope}%
\begin{pgfscope}%
\pgfpathrectangle{\pgfqpoint{0.418102in}{3.218007in}}{\pgfqpoint{2.583333in}{0.400885in}}%
\pgfusepath{clip}%
\pgfsetroundcap%
\pgfsetroundjoin%
\pgfsetlinewidth{0.803000pt}%
\definecolor{currentstroke}{rgb}{1.000000,1.000000,1.000000}%
\pgfsetstrokecolor{currentstroke}%
\pgfsetdash{}{0pt}%
\pgfpathmoveto{\pgfqpoint{1.318354in}{3.218007in}}%
\pgfpathlineto{\pgfqpoint{1.318354in}{3.618892in}}%
\pgfusepath{stroke}%
\end{pgfscope}%
\begin{pgfscope}%
\definecolor{textcolor}{rgb}{0.150000,0.150000,0.150000}%
\pgfsetstrokecolor{textcolor}%
\pgfsetfillcolor{textcolor}%
\pgftext[x=1.318354in,y=3.120784in,,top]{\color{textcolor}\rmfamily\fontsize{10.000000}{12.000000}\selectfont 2014}%
\end{pgfscope}%
\begin{pgfscope}%
\pgfpathrectangle{\pgfqpoint{0.418102in}{3.218007in}}{\pgfqpoint{2.583333in}{0.400885in}}%
\pgfusepath{clip}%
\pgfsetroundcap%
\pgfsetroundjoin%
\pgfsetlinewidth{0.803000pt}%
\definecolor{currentstroke}{rgb}{1.000000,1.000000,1.000000}%
\pgfsetstrokecolor{currentstroke}%
\pgfsetdash{}{0pt}%
\pgfpathmoveto{\pgfqpoint{1.710305in}{3.218007in}}%
\pgfpathlineto{\pgfqpoint{1.710305in}{3.618892in}}%
\pgfusepath{stroke}%
\end{pgfscope}%
\begin{pgfscope}%
\definecolor{textcolor}{rgb}{0.150000,0.150000,0.150000}%
\pgfsetstrokecolor{textcolor}%
\pgfsetfillcolor{textcolor}%
\pgftext[x=1.710305in,y=3.120784in,,top]{\color{textcolor}\rmfamily\fontsize{10.000000}{12.000000}\selectfont 2015}%
\end{pgfscope}%
\begin{pgfscope}%
\pgfpathrectangle{\pgfqpoint{0.418102in}{3.218007in}}{\pgfqpoint{2.583333in}{0.400885in}}%
\pgfusepath{clip}%
\pgfsetroundcap%
\pgfsetroundjoin%
\pgfsetlinewidth{0.803000pt}%
\definecolor{currentstroke}{rgb}{1.000000,1.000000,1.000000}%
\pgfsetstrokecolor{currentstroke}%
\pgfsetdash{}{0pt}%
\pgfpathmoveto{\pgfqpoint{2.102256in}{3.218007in}}%
\pgfpathlineto{\pgfqpoint{2.102256in}{3.618892in}}%
\pgfusepath{stroke}%
\end{pgfscope}%
\begin{pgfscope}%
\definecolor{textcolor}{rgb}{0.150000,0.150000,0.150000}%
\pgfsetstrokecolor{textcolor}%
\pgfsetfillcolor{textcolor}%
\pgftext[x=2.102256in,y=3.120784in,,top]{\color{textcolor}\rmfamily\fontsize{10.000000}{12.000000}\selectfont 2016}%
\end{pgfscope}%
\begin{pgfscope}%
\pgfpathrectangle{\pgfqpoint{0.418102in}{3.218007in}}{\pgfqpoint{2.583333in}{0.400885in}}%
\pgfusepath{clip}%
\pgfsetroundcap%
\pgfsetroundjoin%
\pgfsetlinewidth{0.803000pt}%
\definecolor{currentstroke}{rgb}{1.000000,1.000000,1.000000}%
\pgfsetstrokecolor{currentstroke}%
\pgfsetdash{}{0pt}%
\pgfpathmoveto{\pgfqpoint{2.495281in}{3.218007in}}%
\pgfpathlineto{\pgfqpoint{2.495281in}{3.618892in}}%
\pgfusepath{stroke}%
\end{pgfscope}%
\begin{pgfscope}%
\definecolor{textcolor}{rgb}{0.150000,0.150000,0.150000}%
\pgfsetstrokecolor{textcolor}%
\pgfsetfillcolor{textcolor}%
\pgftext[x=2.495281in,y=3.120784in,,top]{\color{textcolor}\rmfamily\fontsize{10.000000}{12.000000}\selectfont 2017}%
\end{pgfscope}%
\begin{pgfscope}%
\pgfpathrectangle{\pgfqpoint{0.418102in}{3.218007in}}{\pgfqpoint{2.583333in}{0.400885in}}%
\pgfusepath{clip}%
\pgfsetroundcap%
\pgfsetroundjoin%
\pgfsetlinewidth{0.803000pt}%
\definecolor{currentstroke}{rgb}{1.000000,1.000000,1.000000}%
\pgfsetstrokecolor{currentstroke}%
\pgfsetdash{}{0pt}%
\pgfpathmoveto{\pgfqpoint{2.887232in}{3.218007in}}%
\pgfpathlineto{\pgfqpoint{2.887232in}{3.618892in}}%
\pgfusepath{stroke}%
\end{pgfscope}%
\begin{pgfscope}%
\definecolor{textcolor}{rgb}{0.150000,0.150000,0.150000}%
\pgfsetstrokecolor{textcolor}%
\pgfsetfillcolor{textcolor}%
\pgftext[x=2.887232in,y=3.120784in,,top]{\color{textcolor}\rmfamily\fontsize{10.000000}{12.000000}\selectfont 2018}%
\end{pgfscope}%
\begin{pgfscope}%
\pgfpathrectangle{\pgfqpoint{0.418102in}{3.218007in}}{\pgfqpoint{2.583333in}{0.400885in}}%
\pgfusepath{clip}%
\pgfsetroundcap%
\pgfsetroundjoin%
\pgfsetlinewidth{0.803000pt}%
\definecolor{currentstroke}{rgb}{1.000000,1.000000,1.000000}%
\pgfsetstrokecolor{currentstroke}%
\pgfsetdash{}{0pt}%
\pgfpathmoveto{\pgfqpoint{0.418102in}{3.367609in}}%
\pgfpathlineto{\pgfqpoint{3.001435in}{3.367609in}}%
\pgfusepath{stroke}%
\end{pgfscope}%
\begin{pgfscope}%
\definecolor{textcolor}{rgb}{0.150000,0.150000,0.150000}%
\pgfsetstrokecolor{textcolor}%
\pgfsetfillcolor{textcolor}%
\pgftext[x=0.232514in,y=3.314848in,left,base]{\color{textcolor}\rmfamily\fontsize{10.000000}{12.000000}\selectfont 1}%
\end{pgfscope}%
\begin{pgfscope}%
\pgfpathrectangle{\pgfqpoint{0.418102in}{3.218007in}}{\pgfqpoint{2.583333in}{0.400885in}}%
\pgfusepath{clip}%
\pgfsetroundcap%
\pgfsetroundjoin%
\pgfsetlinewidth{0.803000pt}%
\definecolor{currentstroke}{rgb}{1.000000,1.000000,1.000000}%
\pgfsetstrokecolor{currentstroke}%
\pgfsetdash{}{0pt}%
\pgfpathmoveto{\pgfqpoint{0.418102in}{3.581442in}}%
\pgfpathlineto{\pgfqpoint{3.001435in}{3.581442in}}%
\pgfusepath{stroke}%
\end{pgfscope}%
\begin{pgfscope}%
\definecolor{textcolor}{rgb}{0.150000,0.150000,0.150000}%
\pgfsetstrokecolor{textcolor}%
\pgfsetfillcolor{textcolor}%
\pgftext[x=0.232514in,y=3.528680in,left,base]{\color{textcolor}\rmfamily\fontsize{10.000000}{12.000000}\selectfont 2}%
\end{pgfscope}%
\begin{pgfscope}%
\pgfpathrectangle{\pgfqpoint{0.418102in}{3.218007in}}{\pgfqpoint{2.583333in}{0.400885in}}%
\pgfusepath{clip}%
\pgfsetroundcap%
\pgfsetroundjoin%
\pgfsetlinewidth{1.505625pt}%
\definecolor{currentstroke}{rgb}{0.000000,0.000000,0.000000}%
\pgfsetstrokecolor{currentstroke}%
\pgfsetdash{}{0pt}%
\pgfpathmoveto{\pgfqpoint{0.535526in}{3.367609in}}%
\pgfpathlineto{\pgfqpoint{0.536600in}{3.369935in}}%
\pgfpathlineto{\pgfqpoint{0.537674in}{3.369780in}}%
\pgfpathlineto{\pgfqpoint{0.538747in}{3.371020in}}%
\pgfpathlineto{\pgfqpoint{0.541969in}{3.373501in}}%
\pgfpathlineto{\pgfqpoint{0.543043in}{3.371796in}}%
\pgfpathlineto{\pgfqpoint{0.544117in}{3.373657in}}%
\pgfpathlineto{\pgfqpoint{0.545190in}{3.374277in}}%
\pgfpathlineto{\pgfqpoint{0.546264in}{3.373191in}}%
\pgfpathlineto{\pgfqpoint{0.550560in}{3.372106in}}%
\pgfpathlineto{\pgfqpoint{0.551633in}{3.375362in}}%
\pgfpathlineto{\pgfqpoint{0.552707in}{3.376758in}}%
\pgfpathlineto{\pgfqpoint{0.553781in}{3.376758in}}%
\pgfpathlineto{\pgfqpoint{0.558077in}{3.373191in}}%
\pgfpathlineto{\pgfqpoint{0.559150in}{3.376603in}}%
\pgfpathlineto{\pgfqpoint{0.561298in}{3.375362in}}%
\pgfpathlineto{\pgfqpoint{0.564520in}{3.373967in}}%
\pgfpathlineto{\pgfqpoint{0.565593in}{3.371641in}}%
\pgfpathlineto{\pgfqpoint{0.566667in}{3.372416in}}%
\pgfpathlineto{\pgfqpoint{0.567741in}{3.372106in}}%
\pgfpathlineto{\pgfqpoint{0.568815in}{3.375362in}}%
\pgfpathlineto{\pgfqpoint{0.572036in}{3.375672in}}%
\pgfpathlineto{\pgfqpoint{0.574184in}{3.377843in}}%
\pgfpathlineto{\pgfqpoint{0.575258in}{3.376603in}}%
\pgfpathlineto{\pgfqpoint{0.576332in}{3.373657in}}%
\pgfpathlineto{\pgfqpoint{0.579553in}{3.375827in}}%
\pgfpathlineto{\pgfqpoint{0.581701in}{3.372261in}}%
\pgfpathlineto{\pgfqpoint{0.583849in}{3.378308in}}%
\pgfpathlineto{\pgfqpoint{0.588144in}{3.379859in}}%
\pgfpathlineto{\pgfqpoint{0.589218in}{3.379549in}}%
\pgfpathlineto{\pgfqpoint{0.590292in}{3.380634in}}%
\pgfpathlineto{\pgfqpoint{0.594587in}{3.377843in}}%
\pgfpathlineto{\pgfqpoint{0.595661in}{3.378929in}}%
\pgfpathlineto{\pgfqpoint{0.596735in}{3.377688in}}%
\pgfpathlineto{\pgfqpoint{0.597809in}{3.378464in}}%
\pgfpathlineto{\pgfqpoint{0.598882in}{3.376758in}}%
\pgfpathlineto{\pgfqpoint{0.602104in}{3.375207in}}%
\pgfpathlineto{\pgfqpoint{0.603178in}{3.370245in}}%
\pgfpathlineto{\pgfqpoint{0.605325in}{3.377378in}}%
\pgfpathlineto{\pgfqpoint{0.606399in}{3.377533in}}%
\pgfpathlineto{\pgfqpoint{0.609621in}{3.378619in}}%
\pgfpathlineto{\pgfqpoint{0.610695in}{3.384046in}}%
\pgfpathlineto{\pgfqpoint{0.611768in}{3.386372in}}%
\pgfpathlineto{\pgfqpoint{0.612842in}{3.390714in}}%
\pgfpathlineto{\pgfqpoint{0.613916in}{3.391179in}}%
\pgfpathlineto{\pgfqpoint{0.617138in}{3.391334in}}%
\pgfpathlineto{\pgfqpoint{0.618211in}{3.389628in}}%
\pgfpathlineto{\pgfqpoint{0.619285in}{3.389628in}}%
\pgfpathlineto{\pgfqpoint{0.620359in}{3.386992in}}%
\pgfpathlineto{\pgfqpoint{0.621433in}{3.386217in}}%
\pgfpathlineto{\pgfqpoint{0.624655in}{3.389318in}}%
\pgfpathlineto{\pgfqpoint{0.626802in}{3.388853in}}%
\pgfpathlineto{\pgfqpoint{0.627876in}{3.388233in}}%
\pgfpathlineto{\pgfqpoint{0.628950in}{3.389628in}}%
\pgfpathlineto{\pgfqpoint{0.633245in}{3.388388in}}%
\pgfpathlineto{\pgfqpoint{0.635393in}{3.382805in}}%
\pgfpathlineto{\pgfqpoint{0.639688in}{3.379394in}}%
\pgfpathlineto{\pgfqpoint{0.640762in}{3.373967in}}%
\pgfpathlineto{\pgfqpoint{0.642910in}{3.380634in}}%
\pgfpathlineto{\pgfqpoint{0.643984in}{3.375672in}}%
\pgfpathlineto{\pgfqpoint{0.647205in}{3.375827in}}%
\pgfpathlineto{\pgfqpoint{0.648279in}{3.381100in}}%
\pgfpathlineto{\pgfqpoint{0.649353in}{3.378153in}}%
\pgfpathlineto{\pgfqpoint{0.650427in}{3.378619in}}%
\pgfpathlineto{\pgfqpoint{0.651500in}{3.381255in}}%
\pgfpathlineto{\pgfqpoint{0.654722in}{3.377843in}}%
\pgfpathlineto{\pgfqpoint{0.655796in}{3.383426in}}%
\pgfpathlineto{\pgfqpoint{0.656870in}{3.382340in}}%
\pgfpathlineto{\pgfqpoint{0.659017in}{3.386217in}}%
\pgfpathlineto{\pgfqpoint{0.662239in}{3.383891in}}%
\pgfpathlineto{\pgfqpoint{0.663313in}{3.386372in}}%
\pgfpathlineto{\pgfqpoint{0.664387in}{3.386062in}}%
\pgfpathlineto{\pgfqpoint{0.665460in}{3.384201in}}%
\pgfpathlineto{\pgfqpoint{0.666534in}{3.381100in}}%
\pgfpathlineto{\pgfqpoint{0.669756in}{3.380789in}}%
\pgfpathlineto{\pgfqpoint{0.670830in}{3.380014in}}%
\pgfpathlineto{\pgfqpoint{0.671903in}{3.375982in}}%
\pgfpathlineto{\pgfqpoint{0.672977in}{3.378153in}}%
\pgfpathlineto{\pgfqpoint{0.674051in}{3.377223in}}%
\pgfpathlineto{\pgfqpoint{0.678346in}{3.369935in}}%
\pgfpathlineto{\pgfqpoint{0.679420in}{3.377068in}}%
\pgfpathlineto{\pgfqpoint{0.680494in}{3.375672in}}%
\pgfpathlineto{\pgfqpoint{0.685863in}{3.379084in}}%
\pgfpathlineto{\pgfqpoint{0.686937in}{3.379084in}}%
\pgfpathlineto{\pgfqpoint{0.688011in}{3.380014in}}%
\pgfpathlineto{\pgfqpoint{0.689085in}{3.379394in}}%
\pgfpathlineto{\pgfqpoint{0.693380in}{3.381100in}}%
\pgfpathlineto{\pgfqpoint{0.694454in}{3.377533in}}%
\pgfpathlineto{\pgfqpoint{0.695528in}{3.378153in}}%
\pgfpathlineto{\pgfqpoint{0.696602in}{3.371641in}}%
\pgfpathlineto{\pgfqpoint{0.699823in}{3.366989in}}%
\pgfpathlineto{\pgfqpoint{0.700897in}{3.368074in}}%
\pgfpathlineto{\pgfqpoint{0.701971in}{3.375672in}}%
\pgfpathlineto{\pgfqpoint{0.703045in}{3.377068in}}%
\pgfpathlineto{\pgfqpoint{0.704119in}{3.379394in}}%
\pgfpathlineto{\pgfqpoint{0.707340in}{3.378308in}}%
\pgfpathlineto{\pgfqpoint{0.708414in}{3.382650in}}%
\pgfpathlineto{\pgfqpoint{0.709488in}{3.381410in}}%
\pgfpathlineto{\pgfqpoint{0.711635in}{3.389473in}}%
\pgfpathlineto{\pgfqpoint{0.714857in}{3.386527in}}%
\pgfpathlineto{\pgfqpoint{0.715931in}{3.389473in}}%
\pgfpathlineto{\pgfqpoint{0.717005in}{3.390558in}}%
\pgfpathlineto{\pgfqpoint{0.718078in}{3.385907in}}%
\pgfpathlineto{\pgfqpoint{0.719152in}{3.389163in}}%
\pgfpathlineto{\pgfqpoint{0.722374in}{3.385751in}}%
\pgfpathlineto{\pgfqpoint{0.725595in}{3.393815in}}%
\pgfpathlineto{\pgfqpoint{0.726669in}{3.401413in}}%
\pgfpathlineto{\pgfqpoint{0.730965in}{3.396606in}}%
\pgfpathlineto{\pgfqpoint{0.733112in}{3.395365in}}%
\pgfpathlineto{\pgfqpoint{0.734186in}{3.391489in}}%
\pgfpathlineto{\pgfqpoint{0.737408in}{3.391954in}}%
\pgfpathlineto{\pgfqpoint{0.738481in}{3.386992in}}%
\pgfpathlineto{\pgfqpoint{0.739555in}{3.387612in}}%
\pgfpathlineto{\pgfqpoint{0.740629in}{3.384821in}}%
\pgfpathlineto{\pgfqpoint{0.741703in}{3.388698in}}%
\pgfpathlineto{\pgfqpoint{0.744924in}{3.386527in}}%
\pgfpathlineto{\pgfqpoint{0.747072in}{3.389628in}}%
\pgfpathlineto{\pgfqpoint{0.748146in}{3.389008in}}%
\pgfpathlineto{\pgfqpoint{0.752441in}{3.392574in}}%
\pgfpathlineto{\pgfqpoint{0.753515in}{3.391024in}}%
\pgfpathlineto{\pgfqpoint{0.754589in}{3.391489in}}%
\pgfpathlineto{\pgfqpoint{0.756737in}{3.402343in}}%
\pgfpathlineto{\pgfqpoint{0.762106in}{3.400172in}}%
\pgfpathlineto{\pgfqpoint{0.763180in}{3.397691in}}%
\pgfpathlineto{\pgfqpoint{0.764254in}{3.402808in}}%
\pgfpathlineto{\pgfqpoint{0.767475in}{3.402964in}}%
\pgfpathlineto{\pgfqpoint{0.768549in}{3.404824in}}%
\pgfpathlineto{\pgfqpoint{0.769623in}{3.403429in}}%
\pgfpathlineto{\pgfqpoint{0.771770in}{3.404514in}}%
\pgfpathlineto{\pgfqpoint{0.777140in}{3.402808in}}%
\pgfpathlineto{\pgfqpoint{0.778213in}{3.403894in}}%
\pgfpathlineto{\pgfqpoint{0.779287in}{3.403274in}}%
\pgfpathlineto{\pgfqpoint{0.782509in}{3.402498in}}%
\pgfpathlineto{\pgfqpoint{0.785730in}{3.399087in}}%
\pgfpathlineto{\pgfqpoint{0.786804in}{3.400948in}}%
\pgfpathlineto{\pgfqpoint{0.790026in}{3.401568in}}%
\pgfpathlineto{\pgfqpoint{0.791099in}{3.401103in}}%
\pgfpathlineto{\pgfqpoint{0.792173in}{3.401258in}}%
\pgfpathlineto{\pgfqpoint{0.793247in}{3.399087in}}%
\pgfpathlineto{\pgfqpoint{0.794321in}{3.399862in}}%
\pgfpathlineto{\pgfqpoint{0.798616in}{3.397536in}}%
\pgfpathlineto{\pgfqpoint{0.799690in}{3.399242in}}%
\pgfpathlineto{\pgfqpoint{0.800764in}{3.406995in}}%
\pgfpathlineto{\pgfqpoint{0.801838in}{3.410407in}}%
\pgfpathlineto{\pgfqpoint{0.805059in}{3.409011in}}%
\pgfpathlineto{\pgfqpoint{0.806133in}{3.410407in}}%
\pgfpathlineto{\pgfqpoint{0.807207in}{3.413973in}}%
\pgfpathlineto{\pgfqpoint{0.809355in}{3.416454in}}%
\pgfpathlineto{\pgfqpoint{0.812576in}{3.415834in}}%
\pgfpathlineto{\pgfqpoint{0.816872in}{3.423587in}}%
\pgfpathlineto{\pgfqpoint{0.821167in}{3.420951in}}%
\pgfpathlineto{\pgfqpoint{0.822241in}{3.418470in}}%
\pgfpathlineto{\pgfqpoint{0.823315in}{3.425913in}}%
\pgfpathlineto{\pgfqpoint{0.824388in}{3.425758in}}%
\pgfpathlineto{\pgfqpoint{0.829758in}{3.428084in}}%
\pgfpathlineto{\pgfqpoint{0.830832in}{3.428549in}}%
\pgfpathlineto{\pgfqpoint{0.831905in}{3.430565in}}%
\pgfpathlineto{\pgfqpoint{0.835127in}{3.428239in}}%
\pgfpathlineto{\pgfqpoint{0.837275in}{3.422346in}}%
\pgfpathlineto{\pgfqpoint{0.838348in}{3.423277in}}%
\pgfpathlineto{\pgfqpoint{0.839422in}{3.422967in}}%
\pgfpathlineto{\pgfqpoint{0.843718in}{3.424828in}}%
\pgfpathlineto{\pgfqpoint{0.844791in}{3.428084in}}%
\pgfpathlineto{\pgfqpoint{0.845865in}{3.426843in}}%
\pgfpathlineto{\pgfqpoint{0.846939in}{3.417540in}}%
\pgfpathlineto{\pgfqpoint{0.850161in}{3.413663in}}%
\pgfpathlineto{\pgfqpoint{0.851234in}{3.408546in}}%
\pgfpathlineto{\pgfqpoint{0.853382in}{3.408391in}}%
\pgfpathlineto{\pgfqpoint{0.854456in}{3.406530in}}%
\pgfpathlineto{\pgfqpoint{0.859825in}{3.405910in}}%
\pgfpathlineto{\pgfqpoint{0.860899in}{3.409321in}}%
\pgfpathlineto{\pgfqpoint{0.861973in}{3.409011in}}%
\pgfpathlineto{\pgfqpoint{0.865194in}{3.410096in}}%
\pgfpathlineto{\pgfqpoint{0.866268in}{3.412267in}}%
\pgfpathlineto{\pgfqpoint{0.868416in}{3.403894in}}%
\pgfpathlineto{\pgfqpoint{0.869490in}{3.405290in}}%
\pgfpathlineto{\pgfqpoint{0.872711in}{3.403894in}}%
\pgfpathlineto{\pgfqpoint{0.873785in}{3.401413in}}%
\pgfpathlineto{\pgfqpoint{0.874859in}{3.393350in}}%
\pgfpathlineto{\pgfqpoint{0.877007in}{3.395055in}}%
\pgfpathlineto{\pgfqpoint{0.880228in}{3.401103in}}%
\pgfpathlineto{\pgfqpoint{0.881302in}{3.400638in}}%
\pgfpathlineto{\pgfqpoint{0.882376in}{3.401413in}}%
\pgfpathlineto{\pgfqpoint{0.884523in}{3.405755in}}%
\pgfpathlineto{\pgfqpoint{0.887745in}{3.405910in}}%
\pgfpathlineto{\pgfqpoint{0.888819in}{3.403739in}}%
\pgfpathlineto{\pgfqpoint{0.889893in}{3.406995in}}%
\pgfpathlineto{\pgfqpoint{0.892040in}{3.406840in}}%
\pgfpathlineto{\pgfqpoint{0.895262in}{3.403119in}}%
\pgfpathlineto{\pgfqpoint{0.896336in}{3.403584in}}%
\pgfpathlineto{\pgfqpoint{0.897409in}{3.407926in}}%
\pgfpathlineto{\pgfqpoint{0.899557in}{3.410717in}}%
\pgfpathlineto{\pgfqpoint{0.902779in}{3.409941in}}%
\pgfpathlineto{\pgfqpoint{0.903853in}{3.411337in}}%
\pgfpathlineto{\pgfqpoint{0.904926in}{3.414593in}}%
\pgfpathlineto{\pgfqpoint{0.906000in}{3.412733in}}%
\pgfpathlineto{\pgfqpoint{0.907074in}{3.412733in}}%
\pgfpathlineto{\pgfqpoint{0.910296in}{3.416454in}}%
\pgfpathlineto{\pgfqpoint{0.911369in}{3.413508in}}%
\pgfpathlineto{\pgfqpoint{0.912443in}{3.405290in}}%
\pgfpathlineto{\pgfqpoint{0.913517in}{3.408081in}}%
\pgfpathlineto{\pgfqpoint{0.914591in}{3.406065in}}%
\pgfpathlineto{\pgfqpoint{0.919960in}{3.404824in}}%
\pgfpathlineto{\pgfqpoint{0.921034in}{3.403739in}}%
\pgfpathlineto{\pgfqpoint{0.922108in}{3.400793in}}%
\pgfpathlineto{\pgfqpoint{0.927477in}{3.411647in}}%
\pgfpathlineto{\pgfqpoint{0.928551in}{3.408701in}}%
\pgfpathlineto{\pgfqpoint{0.929625in}{3.409941in}}%
\pgfpathlineto{\pgfqpoint{0.932846in}{3.409166in}}%
\pgfpathlineto{\pgfqpoint{0.933920in}{3.406375in}}%
\pgfpathlineto{\pgfqpoint{0.934994in}{3.406995in}}%
\pgfpathlineto{\pgfqpoint{0.936068in}{3.409631in}}%
\pgfpathlineto{\pgfqpoint{0.937142in}{3.409166in}}%
\pgfpathlineto{\pgfqpoint{0.940363in}{3.409011in}}%
\pgfpathlineto{\pgfqpoint{0.941437in}{3.409941in}}%
\pgfpathlineto{\pgfqpoint{0.942511in}{3.409011in}}%
\pgfpathlineto{\pgfqpoint{0.943585in}{3.411182in}}%
\pgfpathlineto{\pgfqpoint{0.944658in}{3.420176in}}%
\pgfpathlineto{\pgfqpoint{0.948954in}{3.419710in}}%
\pgfpathlineto{\pgfqpoint{0.950028in}{3.418935in}}%
\pgfpathlineto{\pgfqpoint{0.951101in}{3.420176in}}%
\pgfpathlineto{\pgfqpoint{0.952175in}{3.423122in}}%
\pgfpathlineto{\pgfqpoint{0.955397in}{3.425603in}}%
\pgfpathlineto{\pgfqpoint{0.956471in}{3.425603in}}%
\pgfpathlineto{\pgfqpoint{0.957544in}{3.422346in}}%
\pgfpathlineto{\pgfqpoint{0.958618in}{3.422967in}}%
\pgfpathlineto{\pgfqpoint{0.959692in}{3.427153in}}%
\pgfpathlineto{\pgfqpoint{0.962914in}{3.423432in}}%
\pgfpathlineto{\pgfqpoint{0.963987in}{3.426223in}}%
\pgfpathlineto{\pgfqpoint{0.965061in}{3.424983in}}%
\pgfpathlineto{\pgfqpoint{0.967209in}{3.425603in}}%
\pgfpathlineto{\pgfqpoint{0.970431in}{3.425138in}}%
\pgfpathlineto{\pgfqpoint{0.971504in}{3.426688in}}%
\pgfpathlineto{\pgfqpoint{0.972578in}{3.436457in}}%
\pgfpathlineto{\pgfqpoint{0.973652in}{3.436612in}}%
\pgfpathlineto{\pgfqpoint{0.974726in}{3.435217in}}%
\pgfpathlineto{\pgfqpoint{0.979021in}{3.440799in}}%
\pgfpathlineto{\pgfqpoint{0.980095in}{3.436612in}}%
\pgfpathlineto{\pgfqpoint{0.981169in}{3.437078in}}%
\pgfpathlineto{\pgfqpoint{0.982243in}{3.438783in}}%
\pgfpathlineto{\pgfqpoint{0.985464in}{3.431650in}}%
\pgfpathlineto{\pgfqpoint{0.987612in}{3.438473in}}%
\pgfpathlineto{\pgfqpoint{0.988686in}{3.436612in}}%
\pgfpathlineto{\pgfqpoint{0.989760in}{3.436302in}}%
\pgfpathlineto{\pgfqpoint{0.992981in}{3.437233in}}%
\pgfpathlineto{\pgfqpoint{0.994055in}{3.441109in}}%
\pgfpathlineto{\pgfqpoint{0.995129in}{3.442195in}}%
\pgfpathlineto{\pgfqpoint{0.996203in}{3.442195in}}%
\pgfpathlineto{\pgfqpoint{0.997276in}{3.443435in}}%
\pgfpathlineto{\pgfqpoint{1.000498in}{3.441574in}}%
\pgfpathlineto{\pgfqpoint{1.001572in}{3.438938in}}%
\pgfpathlineto{\pgfqpoint{1.002646in}{3.440024in}}%
\pgfpathlineto{\pgfqpoint{1.003720in}{3.442350in}}%
\pgfpathlineto{\pgfqpoint{1.004793in}{3.439403in}}%
\pgfpathlineto{\pgfqpoint{1.008015in}{3.437078in}}%
\pgfpathlineto{\pgfqpoint{1.009089in}{3.437853in}}%
\pgfpathlineto{\pgfqpoint{1.010163in}{3.439559in}}%
\pgfpathlineto{\pgfqpoint{1.011236in}{3.437543in}}%
\pgfpathlineto{\pgfqpoint{1.012310in}{3.438473in}}%
\pgfpathlineto{\pgfqpoint{1.015532in}{3.436922in}}%
\pgfpathlineto{\pgfqpoint{1.016606in}{3.435372in}}%
\pgfpathlineto{\pgfqpoint{1.023049in}{3.434907in}}%
\pgfpathlineto{\pgfqpoint{1.024122in}{3.438163in}}%
\pgfpathlineto{\pgfqpoint{1.025196in}{3.433976in}}%
\pgfpathlineto{\pgfqpoint{1.026270in}{3.434907in}}%
\pgfpathlineto{\pgfqpoint{1.027344in}{3.433201in}}%
\pgfpathlineto{\pgfqpoint{1.030565in}{3.435372in}}%
\pgfpathlineto{\pgfqpoint{1.031639in}{3.434752in}}%
\pgfpathlineto{\pgfqpoint{1.032713in}{3.441109in}}%
\pgfpathlineto{\pgfqpoint{1.033787in}{3.441109in}}%
\pgfpathlineto{\pgfqpoint{1.034861in}{3.439559in}}%
\pgfpathlineto{\pgfqpoint{1.038082in}{3.431650in}}%
\pgfpathlineto{\pgfqpoint{1.039156in}{3.435217in}}%
\pgfpathlineto{\pgfqpoint{1.040230in}{3.431030in}}%
\pgfpathlineto{\pgfqpoint{1.041304in}{3.429945in}}%
\pgfpathlineto{\pgfqpoint{1.042378in}{3.418780in}}%
\pgfpathlineto{\pgfqpoint{1.045599in}{3.413818in}}%
\pgfpathlineto{\pgfqpoint{1.046673in}{3.415679in}}%
\pgfpathlineto{\pgfqpoint{1.047747in}{3.421261in}}%
\pgfpathlineto{\pgfqpoint{1.048821in}{3.421261in}}%
\pgfpathlineto{\pgfqpoint{1.049895in}{3.424362in}}%
\pgfpathlineto{\pgfqpoint{1.054190in}{3.425293in}}%
\pgfpathlineto{\pgfqpoint{1.055264in}{3.423587in}}%
\pgfpathlineto{\pgfqpoint{1.057411in}{3.428704in}}%
\pgfpathlineto{\pgfqpoint{1.060633in}{3.428859in}}%
\pgfpathlineto{\pgfqpoint{1.061707in}{3.430100in}}%
\pgfpathlineto{\pgfqpoint{1.062781in}{3.434131in}}%
\pgfpathlineto{\pgfqpoint{1.063854in}{3.431340in}}%
\pgfpathlineto{\pgfqpoint{1.064928in}{3.432736in}}%
\pgfpathlineto{\pgfqpoint{1.068150in}{3.432115in}}%
\pgfpathlineto{\pgfqpoint{1.072445in}{3.439559in}}%
\pgfpathlineto{\pgfqpoint{1.075667in}{3.440954in}}%
\pgfpathlineto{\pgfqpoint{1.076741in}{3.442040in}}%
\pgfpathlineto{\pgfqpoint{1.077814in}{3.444521in}}%
\pgfpathlineto{\pgfqpoint{1.079962in}{3.440489in}}%
\pgfpathlineto{\pgfqpoint{1.086405in}{3.441264in}}%
\pgfpathlineto{\pgfqpoint{1.087479in}{3.437853in}}%
\pgfpathlineto{\pgfqpoint{1.090700in}{3.441729in}}%
\pgfpathlineto{\pgfqpoint{1.091774in}{3.442040in}}%
\pgfpathlineto{\pgfqpoint{1.092848in}{3.437853in}}%
\pgfpathlineto{\pgfqpoint{1.093922in}{3.438628in}}%
\pgfpathlineto{\pgfqpoint{1.094996in}{3.444521in}}%
\pgfpathlineto{\pgfqpoint{1.098217in}{3.443435in}}%
\pgfpathlineto{\pgfqpoint{1.100365in}{3.440024in}}%
\pgfpathlineto{\pgfqpoint{1.101439in}{3.442195in}}%
\pgfpathlineto{\pgfqpoint{1.102513in}{3.440334in}}%
\pgfpathlineto{\pgfqpoint{1.105734in}{3.443435in}}%
\pgfpathlineto{\pgfqpoint{1.106808in}{3.450258in}}%
\pgfpathlineto{\pgfqpoint{1.107882in}{3.445916in}}%
\pgfpathlineto{\pgfqpoint{1.108956in}{3.439248in}}%
\pgfpathlineto{\pgfqpoint{1.110030in}{3.440644in}}%
\pgfpathlineto{\pgfqpoint{1.113251in}{3.435372in}}%
\pgfpathlineto{\pgfqpoint{1.115399in}{3.439248in}}%
\pgfpathlineto{\pgfqpoint{1.116473in}{3.440179in}}%
\pgfpathlineto{\pgfqpoint{1.117546in}{3.438628in}}%
\pgfpathlineto{\pgfqpoint{1.120768in}{3.440334in}}%
\pgfpathlineto{\pgfqpoint{1.121842in}{3.435062in}}%
\pgfpathlineto{\pgfqpoint{1.122916in}{3.435062in}}%
\pgfpathlineto{\pgfqpoint{1.125063in}{3.439093in}}%
\pgfpathlineto{\pgfqpoint{1.128285in}{3.440179in}}%
\pgfpathlineto{\pgfqpoint{1.129359in}{3.443900in}}%
\pgfpathlineto{\pgfqpoint{1.130432in}{3.442815in}}%
\pgfpathlineto{\pgfqpoint{1.131506in}{3.447777in}}%
\pgfpathlineto{\pgfqpoint{1.132580in}{3.445606in}}%
\pgfpathlineto{\pgfqpoint{1.135802in}{3.443900in}}%
\pgfpathlineto{\pgfqpoint{1.136875in}{3.441574in}}%
\pgfpathlineto{\pgfqpoint{1.139023in}{3.443900in}}%
\pgfpathlineto{\pgfqpoint{1.140097in}{3.457391in}}%
\pgfpathlineto{\pgfqpoint{1.143319in}{3.459097in}}%
\pgfpathlineto{\pgfqpoint{1.145466in}{3.456150in}}%
\pgfpathlineto{\pgfqpoint{1.146540in}{3.456926in}}%
\pgfpathlineto{\pgfqpoint{1.147614in}{3.456460in}}%
\pgfpathlineto{\pgfqpoint{1.150835in}{3.454445in}}%
\pgfpathlineto{\pgfqpoint{1.151909in}{3.454445in}}%
\pgfpathlineto{\pgfqpoint{1.152983in}{3.453049in}}%
\pgfpathlineto{\pgfqpoint{1.154057in}{3.456150in}}%
\pgfpathlineto{\pgfqpoint{1.155131in}{3.457081in}}%
\pgfpathlineto{\pgfqpoint{1.158352in}{3.454910in}}%
\pgfpathlineto{\pgfqpoint{1.159426in}{3.452274in}}%
\pgfpathlineto{\pgfqpoint{1.160500in}{3.452739in}}%
\pgfpathlineto{\pgfqpoint{1.161574in}{3.452584in}}%
\pgfpathlineto{\pgfqpoint{1.162648in}{3.451498in}}%
\pgfpathlineto{\pgfqpoint{1.165869in}{3.451809in}}%
\pgfpathlineto{\pgfqpoint{1.166943in}{3.451033in}}%
\pgfpathlineto{\pgfqpoint{1.169091in}{3.448552in}}%
\pgfpathlineto{\pgfqpoint{1.174460in}{3.444986in}}%
\pgfpathlineto{\pgfqpoint{1.175534in}{3.443745in}}%
\pgfpathlineto{\pgfqpoint{1.176608in}{3.445761in}}%
\pgfpathlineto{\pgfqpoint{1.177681in}{3.445761in}}%
\pgfpathlineto{\pgfqpoint{1.180903in}{3.443745in}}%
\pgfpathlineto{\pgfqpoint{1.181977in}{3.438473in}}%
\pgfpathlineto{\pgfqpoint{1.183051in}{3.438628in}}%
\pgfpathlineto{\pgfqpoint{1.184124in}{3.437543in}}%
\pgfpathlineto{\pgfqpoint{1.185198in}{3.438008in}}%
\pgfpathlineto{\pgfqpoint{1.189494in}{3.436922in}}%
\pgfpathlineto{\pgfqpoint{1.190567in}{3.438318in}}%
\pgfpathlineto{\pgfqpoint{1.192715in}{3.438163in}}%
\pgfpathlineto{\pgfqpoint{1.195937in}{3.440954in}}%
\pgfpathlineto{\pgfqpoint{1.197010in}{3.446847in}}%
\pgfpathlineto{\pgfqpoint{1.198084in}{3.449638in}}%
\pgfpathlineto{\pgfqpoint{1.199158in}{3.446691in}}%
\pgfpathlineto{\pgfqpoint{1.200232in}{3.445761in}}%
\pgfpathlineto{\pgfqpoint{1.203453in}{3.450258in}}%
\pgfpathlineto{\pgfqpoint{1.205601in}{3.459097in}}%
\pgfpathlineto{\pgfqpoint{1.206675in}{3.456460in}}%
\pgfpathlineto{\pgfqpoint{1.207749in}{3.450878in}}%
\pgfpathlineto{\pgfqpoint{1.212044in}{3.454755in}}%
\pgfpathlineto{\pgfqpoint{1.213118in}{3.453669in}}%
\pgfpathlineto{\pgfqpoint{1.214192in}{3.453824in}}%
\pgfpathlineto{\pgfqpoint{1.215266in}{3.451343in}}%
\pgfpathlineto{\pgfqpoint{1.218487in}{3.449483in}}%
\pgfpathlineto{\pgfqpoint{1.220635in}{3.454910in}}%
\pgfpathlineto{\pgfqpoint{1.221709in}{3.451964in}}%
\pgfpathlineto{\pgfqpoint{1.226004in}{3.450103in}}%
\pgfpathlineto{\pgfqpoint{1.227078in}{3.446691in}}%
\pgfpathlineto{\pgfqpoint{1.228152in}{3.445451in}}%
\pgfpathlineto{\pgfqpoint{1.229226in}{3.453824in}}%
\pgfpathlineto{\pgfqpoint{1.230299in}{3.455685in}}%
\pgfpathlineto{\pgfqpoint{1.233521in}{3.455530in}}%
\pgfpathlineto{\pgfqpoint{1.234595in}{3.453049in}}%
\pgfpathlineto{\pgfqpoint{1.235669in}{3.455220in}}%
\pgfpathlineto{\pgfqpoint{1.236742in}{3.459252in}}%
\pgfpathlineto{\pgfqpoint{1.237816in}{3.469951in}}%
\pgfpathlineto{\pgfqpoint{1.241038in}{3.477239in}}%
\pgfpathlineto{\pgfqpoint{1.242112in}{3.475843in}}%
\pgfpathlineto{\pgfqpoint{1.243185in}{3.471812in}}%
\pgfpathlineto{\pgfqpoint{1.244259in}{3.474758in}}%
\pgfpathlineto{\pgfqpoint{1.245333in}{3.473983in}}%
\pgfpathlineto{\pgfqpoint{1.249629in}{3.478169in}}%
\pgfpathlineto{\pgfqpoint{1.250702in}{3.480030in}}%
\pgfpathlineto{\pgfqpoint{1.251776in}{3.477239in}}%
\pgfpathlineto{\pgfqpoint{1.252850in}{3.482201in}}%
\pgfpathlineto{\pgfqpoint{1.257145in}{3.480650in}}%
\pgfpathlineto{\pgfqpoint{1.258219in}{3.486698in}}%
\pgfpathlineto{\pgfqpoint{1.259293in}{3.482976in}}%
\pgfpathlineto{\pgfqpoint{1.260367in}{3.488559in}}%
\pgfpathlineto{\pgfqpoint{1.263588in}{3.488093in}}%
\pgfpathlineto{\pgfqpoint{1.264662in}{3.488559in}}%
\pgfpathlineto{\pgfqpoint{1.265736in}{3.489799in}}%
\pgfpathlineto{\pgfqpoint{1.266810in}{3.487783in}}%
\pgfpathlineto{\pgfqpoint{1.267884in}{3.490419in}}%
\pgfpathlineto{\pgfqpoint{1.271105in}{3.490574in}}%
\pgfpathlineto{\pgfqpoint{1.272179in}{3.488249in}}%
\pgfpathlineto{\pgfqpoint{1.274327in}{3.486853in}}%
\pgfpathlineto{\pgfqpoint{1.275401in}{3.488869in}}%
\pgfpathlineto{\pgfqpoint{1.278622in}{3.484527in}}%
\pgfpathlineto{\pgfqpoint{1.280770in}{3.485767in}}%
\pgfpathlineto{\pgfqpoint{1.282918in}{3.483752in}}%
\pgfpathlineto{\pgfqpoint{1.286139in}{3.483752in}}%
\pgfpathlineto{\pgfqpoint{1.287213in}{3.482511in}}%
\pgfpathlineto{\pgfqpoint{1.288287in}{3.483442in}}%
\pgfpathlineto{\pgfqpoint{1.289361in}{3.481116in}}%
\pgfpathlineto{\pgfqpoint{1.290434in}{3.487163in}}%
\pgfpathlineto{\pgfqpoint{1.293656in}{3.490264in}}%
\pgfpathlineto{\pgfqpoint{1.294730in}{3.489644in}}%
\pgfpathlineto{\pgfqpoint{1.295804in}{3.482666in}}%
\pgfpathlineto{\pgfqpoint{1.296877in}{3.482201in}}%
\pgfpathlineto{\pgfqpoint{1.297951in}{3.485923in}}%
\pgfpathlineto{\pgfqpoint{1.302247in}{3.488249in}}%
\pgfpathlineto{\pgfqpoint{1.303320in}{3.492900in}}%
\pgfpathlineto{\pgfqpoint{1.304394in}{3.494606in}}%
\pgfpathlineto{\pgfqpoint{1.305468in}{3.495071in}}%
\pgfpathlineto{\pgfqpoint{1.308690in}{3.495536in}}%
\pgfpathlineto{\pgfqpoint{1.309763in}{3.498173in}}%
\pgfpathlineto{\pgfqpoint{1.311911in}{3.500964in}}%
\pgfpathlineto{\pgfqpoint{1.312985in}{3.500964in}}%
\pgfpathlineto{\pgfqpoint{1.316207in}{3.501739in}}%
\pgfpathlineto{\pgfqpoint{1.317280in}{3.503445in}}%
\pgfpathlineto{\pgfqpoint{1.319428in}{3.496777in}}%
\pgfpathlineto{\pgfqpoint{1.320502in}{3.496622in}}%
\pgfpathlineto{\pgfqpoint{1.323723in}{3.493831in}}%
\pgfpathlineto{\pgfqpoint{1.324797in}{3.494141in}}%
\pgfpathlineto{\pgfqpoint{1.325871in}{3.493211in}}%
\pgfpathlineto{\pgfqpoint{1.326945in}{3.493366in}}%
\pgfpathlineto{\pgfqpoint{1.328019in}{3.490109in}}%
\pgfpathlineto{\pgfqpoint{1.331240in}{3.487163in}}%
\pgfpathlineto{\pgfqpoint{1.333388in}{3.494916in}}%
\pgfpathlineto{\pgfqpoint{1.334462in}{3.493055in}}%
\pgfpathlineto{\pgfqpoint{1.335536in}{3.485302in}}%
\pgfpathlineto{\pgfqpoint{1.339831in}{3.481736in}}%
\pgfpathlineto{\pgfqpoint{1.341979in}{3.475843in}}%
\pgfpathlineto{\pgfqpoint{1.343052in}{3.464989in}}%
\pgfpathlineto{\pgfqpoint{1.346274in}{3.466540in}}%
\pgfpathlineto{\pgfqpoint{1.347348in}{3.471347in}}%
\pgfpathlineto{\pgfqpoint{1.348422in}{3.469331in}}%
\pgfpathlineto{\pgfqpoint{1.349496in}{3.471812in}}%
\pgfpathlineto{\pgfqpoint{1.350569in}{3.467315in}}%
\pgfpathlineto{\pgfqpoint{1.353791in}{3.457546in}}%
\pgfpathlineto{\pgfqpoint{1.354865in}{3.460337in}}%
\pgfpathlineto{\pgfqpoint{1.355939in}{3.459717in}}%
\pgfpathlineto{\pgfqpoint{1.358086in}{3.468090in}}%
\pgfpathlineto{\pgfqpoint{1.361308in}{3.466229in}}%
\pgfpathlineto{\pgfqpoint{1.362382in}{3.471036in}}%
\pgfpathlineto{\pgfqpoint{1.363455in}{3.470571in}}%
\pgfpathlineto{\pgfqpoint{1.364529in}{3.471192in}}%
\pgfpathlineto{\pgfqpoint{1.365603in}{3.474913in}}%
\pgfpathlineto{\pgfqpoint{1.369898in}{3.473828in}}%
\pgfpathlineto{\pgfqpoint{1.370972in}{3.470571in}}%
\pgfpathlineto{\pgfqpoint{1.372046in}{3.469951in}}%
\pgfpathlineto{\pgfqpoint{1.373120in}{3.467625in}}%
\pgfpathlineto{\pgfqpoint{1.376341in}{3.471967in}}%
\pgfpathlineto{\pgfqpoint{1.378489in}{3.472122in}}%
\pgfpathlineto{\pgfqpoint{1.379563in}{3.474603in}}%
\pgfpathlineto{\pgfqpoint{1.380637in}{3.474293in}}%
\pgfpathlineto{\pgfqpoint{1.383858in}{3.469951in}}%
\pgfpathlineto{\pgfqpoint{1.386006in}{3.480030in}}%
\pgfpathlineto{\pgfqpoint{1.387080in}{3.483752in}}%
\pgfpathlineto{\pgfqpoint{1.388154in}{3.482666in}}%
\pgfpathlineto{\pgfqpoint{1.391375in}{3.481426in}}%
\pgfpathlineto{\pgfqpoint{1.393523in}{3.478014in}}%
\pgfpathlineto{\pgfqpoint{1.395671in}{3.469796in}}%
\pgfpathlineto{\pgfqpoint{1.398892in}{3.473828in}}%
\pgfpathlineto{\pgfqpoint{1.399966in}{3.476619in}}%
\pgfpathlineto{\pgfqpoint{1.401040in}{3.471967in}}%
\pgfpathlineto{\pgfqpoint{1.402114in}{3.471812in}}%
\pgfpathlineto{\pgfqpoint{1.403187in}{3.473362in}}%
\pgfpathlineto{\pgfqpoint{1.406409in}{3.473517in}}%
\pgfpathlineto{\pgfqpoint{1.407483in}{3.477239in}}%
\pgfpathlineto{\pgfqpoint{1.408557in}{3.476154in}}%
\pgfpathlineto{\pgfqpoint{1.409630in}{3.478635in}}%
\pgfpathlineto{\pgfqpoint{1.410704in}{3.479410in}}%
\pgfpathlineto{\pgfqpoint{1.415000in}{3.479255in}}%
\pgfpathlineto{\pgfqpoint{1.417147in}{3.483907in}}%
\pgfpathlineto{\pgfqpoint{1.418221in}{3.481271in}}%
\pgfpathlineto{\pgfqpoint{1.422517in}{3.477859in}}%
\pgfpathlineto{\pgfqpoint{1.423590in}{3.480340in}}%
\pgfpathlineto{\pgfqpoint{1.424664in}{3.475688in}}%
\pgfpathlineto{\pgfqpoint{1.425738in}{3.473828in}}%
\pgfpathlineto{\pgfqpoint{1.430033in}{3.478635in}}%
\pgfpathlineto{\pgfqpoint{1.432181in}{3.487938in}}%
\pgfpathlineto{\pgfqpoint{1.437550in}{3.488249in}}%
\pgfpathlineto{\pgfqpoint{1.438624in}{3.486233in}}%
\pgfpathlineto{\pgfqpoint{1.439698in}{3.486698in}}%
\pgfpathlineto{\pgfqpoint{1.440772in}{3.488559in}}%
\pgfpathlineto{\pgfqpoint{1.443993in}{3.490730in}}%
\pgfpathlineto{\pgfqpoint{1.445067in}{3.490574in}}%
\pgfpathlineto{\pgfqpoint{1.446141in}{3.492125in}}%
\pgfpathlineto{\pgfqpoint{1.448289in}{3.489489in}}%
\pgfpathlineto{\pgfqpoint{1.451510in}{3.488249in}}%
\pgfpathlineto{\pgfqpoint{1.452584in}{3.483286in}}%
\pgfpathlineto{\pgfqpoint{1.453658in}{3.487628in}}%
\pgfpathlineto{\pgfqpoint{1.455806in}{3.486233in}}%
\pgfpathlineto{\pgfqpoint{1.460101in}{3.492590in}}%
\pgfpathlineto{\pgfqpoint{1.462249in}{3.488559in}}%
\pgfpathlineto{\pgfqpoint{1.463322in}{3.489334in}}%
\pgfpathlineto{\pgfqpoint{1.466544in}{3.488559in}}%
\pgfpathlineto{\pgfqpoint{1.467618in}{3.484682in}}%
\pgfpathlineto{\pgfqpoint{1.468692in}{3.487163in}}%
\pgfpathlineto{\pgfqpoint{1.475135in}{3.488093in}}%
\pgfpathlineto{\pgfqpoint{1.477282in}{3.490264in}}%
\pgfpathlineto{\pgfqpoint{1.478356in}{3.490885in}}%
\pgfpathlineto{\pgfqpoint{1.481578in}{3.491350in}}%
\pgfpathlineto{\pgfqpoint{1.482651in}{3.490885in}}%
\pgfpathlineto{\pgfqpoint{1.483725in}{3.487938in}}%
\pgfpathlineto{\pgfqpoint{1.484799in}{3.490574in}}%
\pgfpathlineto{\pgfqpoint{1.485873in}{3.495847in}}%
\pgfpathlineto{\pgfqpoint{1.489095in}{3.499103in}}%
\pgfpathlineto{\pgfqpoint{1.490168in}{3.498638in}}%
\pgfpathlineto{\pgfqpoint{1.492316in}{3.493055in}}%
\pgfpathlineto{\pgfqpoint{1.493390in}{3.493986in}}%
\pgfpathlineto{\pgfqpoint{1.496611in}{3.491350in}}%
\pgfpathlineto{\pgfqpoint{1.498759in}{3.492125in}}%
\pgfpathlineto{\pgfqpoint{1.499833in}{3.495381in}}%
\pgfpathlineto{\pgfqpoint{1.500907in}{3.496002in}}%
\pgfpathlineto{\pgfqpoint{1.506276in}{3.489024in}}%
\pgfpathlineto{\pgfqpoint{1.507350in}{3.487318in}}%
\pgfpathlineto{\pgfqpoint{1.508424in}{3.489179in}}%
\pgfpathlineto{\pgfqpoint{1.511645in}{3.487163in}}%
\pgfpathlineto{\pgfqpoint{1.512719in}{3.488714in}}%
\pgfpathlineto{\pgfqpoint{1.514867in}{3.494606in}}%
\pgfpathlineto{\pgfqpoint{1.519162in}{3.493211in}}%
\pgfpathlineto{\pgfqpoint{1.520236in}{3.488404in}}%
\pgfpathlineto{\pgfqpoint{1.521310in}{3.487783in}}%
\pgfpathlineto{\pgfqpoint{1.522384in}{3.486233in}}%
\pgfpathlineto{\pgfqpoint{1.523457in}{3.490574in}}%
\pgfpathlineto{\pgfqpoint{1.526679in}{3.491970in}}%
\pgfpathlineto{\pgfqpoint{1.527753in}{3.491350in}}%
\pgfpathlineto{\pgfqpoint{1.528827in}{3.496622in}}%
\pgfpathlineto{\pgfqpoint{1.529900in}{3.491350in}}%
\pgfpathlineto{\pgfqpoint{1.534196in}{3.483442in}}%
\pgfpathlineto{\pgfqpoint{1.535270in}{3.483907in}}%
\pgfpathlineto{\pgfqpoint{1.536343in}{3.482511in}}%
\pgfpathlineto{\pgfqpoint{1.537417in}{3.482821in}}%
\pgfpathlineto{\pgfqpoint{1.538491in}{3.480961in}}%
\pgfpathlineto{\pgfqpoint{1.541713in}{3.478480in}}%
\pgfpathlineto{\pgfqpoint{1.542786in}{3.476619in}}%
\pgfpathlineto{\pgfqpoint{1.543860in}{3.479100in}}%
\pgfpathlineto{\pgfqpoint{1.544934in}{3.472897in}}%
\pgfpathlineto{\pgfqpoint{1.546008in}{3.475378in}}%
\pgfpathlineto{\pgfqpoint{1.549229in}{3.474448in}}%
\pgfpathlineto{\pgfqpoint{1.550303in}{3.471192in}}%
\pgfpathlineto{\pgfqpoint{1.551377in}{3.476619in}}%
\pgfpathlineto{\pgfqpoint{1.552451in}{3.477239in}}%
\pgfpathlineto{\pgfqpoint{1.553525in}{3.479410in}}%
\pgfpathlineto{\pgfqpoint{1.556746in}{3.480961in}}%
\pgfpathlineto{\pgfqpoint{1.557820in}{3.478635in}}%
\pgfpathlineto{\pgfqpoint{1.558894in}{3.481426in}}%
\pgfpathlineto{\pgfqpoint{1.559968in}{3.482201in}}%
\pgfpathlineto{\pgfqpoint{1.561042in}{3.479100in}}%
\pgfpathlineto{\pgfqpoint{1.564263in}{3.484527in}}%
\pgfpathlineto{\pgfqpoint{1.565337in}{3.484217in}}%
\pgfpathlineto{\pgfqpoint{1.566411in}{3.488249in}}%
\pgfpathlineto{\pgfqpoint{1.567485in}{3.489179in}}%
\pgfpathlineto{\pgfqpoint{1.568559in}{3.485612in}}%
\pgfpathlineto{\pgfqpoint{1.571780in}{3.486233in}}%
\pgfpathlineto{\pgfqpoint{1.572854in}{3.483752in}}%
\pgfpathlineto{\pgfqpoint{1.573928in}{3.485302in}}%
\pgfpathlineto{\pgfqpoint{1.575002in}{3.483752in}}%
\pgfpathlineto{\pgfqpoint{1.580371in}{3.481736in}}%
\pgfpathlineto{\pgfqpoint{1.581445in}{3.482976in}}%
\pgfpathlineto{\pgfqpoint{1.582518in}{3.483131in}}%
\pgfpathlineto{\pgfqpoint{1.583592in}{3.484992in}}%
\pgfpathlineto{\pgfqpoint{1.586814in}{3.484682in}}%
\pgfpathlineto{\pgfqpoint{1.587888in}{3.482356in}}%
\pgfpathlineto{\pgfqpoint{1.590035in}{3.483907in}}%
\pgfpathlineto{\pgfqpoint{1.591109in}{3.482046in}}%
\pgfpathlineto{\pgfqpoint{1.594331in}{3.482666in}}%
\pgfpathlineto{\pgfqpoint{1.595405in}{3.486388in}}%
\pgfpathlineto{\pgfqpoint{1.596478in}{3.487008in}}%
\pgfpathlineto{\pgfqpoint{1.598626in}{3.490109in}}%
\pgfpathlineto{\pgfqpoint{1.603995in}{3.485612in}}%
\pgfpathlineto{\pgfqpoint{1.605069in}{3.480650in}}%
\pgfpathlineto{\pgfqpoint{1.606143in}{3.481736in}}%
\pgfpathlineto{\pgfqpoint{1.609364in}{3.478945in}}%
\pgfpathlineto{\pgfqpoint{1.610438in}{3.481581in}}%
\pgfpathlineto{\pgfqpoint{1.611512in}{3.475688in}}%
\pgfpathlineto{\pgfqpoint{1.612586in}{3.475223in}}%
\pgfpathlineto{\pgfqpoint{1.613660in}{3.478790in}}%
\pgfpathlineto{\pgfqpoint{1.616881in}{3.476464in}}%
\pgfpathlineto{\pgfqpoint{1.617955in}{3.471192in}}%
\pgfpathlineto{\pgfqpoint{1.619029in}{3.476774in}}%
\pgfpathlineto{\pgfqpoint{1.621177in}{3.464369in}}%
\pgfpathlineto{\pgfqpoint{1.624398in}{3.460182in}}%
\pgfpathlineto{\pgfqpoint{1.626546in}{3.464369in}}%
\pgfpathlineto{\pgfqpoint{1.627620in}{3.464059in}}%
\pgfpathlineto{\pgfqpoint{1.628694in}{3.471347in}}%
\pgfpathlineto{\pgfqpoint{1.631915in}{3.473983in}}%
\pgfpathlineto{\pgfqpoint{1.632989in}{3.479410in}}%
\pgfpathlineto{\pgfqpoint{1.634063in}{3.476154in}}%
\pgfpathlineto{\pgfqpoint{1.636210in}{3.481891in}}%
\pgfpathlineto{\pgfqpoint{1.639432in}{3.480340in}}%
\pgfpathlineto{\pgfqpoint{1.640506in}{3.484837in}}%
\pgfpathlineto{\pgfqpoint{1.641580in}{3.482046in}}%
\pgfpathlineto{\pgfqpoint{1.642653in}{3.482201in}}%
\pgfpathlineto{\pgfqpoint{1.643727in}{3.484062in}}%
\pgfpathlineto{\pgfqpoint{1.648023in}{3.482666in}}%
\pgfpathlineto{\pgfqpoint{1.649096in}{3.484217in}}%
\pgfpathlineto{\pgfqpoint{1.650170in}{3.491040in}}%
\pgfpathlineto{\pgfqpoint{1.651244in}{3.491660in}}%
\pgfpathlineto{\pgfqpoint{1.654466in}{3.492435in}}%
\pgfpathlineto{\pgfqpoint{1.655539in}{3.491350in}}%
\pgfpathlineto{\pgfqpoint{1.656613in}{3.493055in}}%
\pgfpathlineto{\pgfqpoint{1.657687in}{3.491815in}}%
\pgfpathlineto{\pgfqpoint{1.658761in}{3.492280in}}%
\pgfpathlineto{\pgfqpoint{1.661983in}{3.494296in}}%
\pgfpathlineto{\pgfqpoint{1.663056in}{3.499413in}}%
\pgfpathlineto{\pgfqpoint{1.665204in}{3.497242in}}%
\pgfpathlineto{\pgfqpoint{1.666278in}{3.499103in}}%
\pgfpathlineto{\pgfqpoint{1.669499in}{3.499258in}}%
\pgfpathlineto{\pgfqpoint{1.670573in}{3.497397in}}%
\pgfpathlineto{\pgfqpoint{1.671647in}{3.497552in}}%
\pgfpathlineto{\pgfqpoint{1.677016in}{3.486698in}}%
\pgfpathlineto{\pgfqpoint{1.678090in}{3.487008in}}%
\pgfpathlineto{\pgfqpoint{1.679164in}{3.491350in}}%
\pgfpathlineto{\pgfqpoint{1.680238in}{3.487628in}}%
\pgfpathlineto{\pgfqpoint{1.684533in}{3.482511in}}%
\pgfpathlineto{\pgfqpoint{1.685607in}{3.481116in}}%
\pgfpathlineto{\pgfqpoint{1.686681in}{3.477084in}}%
\pgfpathlineto{\pgfqpoint{1.687755in}{3.478945in}}%
\pgfpathlineto{\pgfqpoint{1.688828in}{3.472277in}}%
\pgfpathlineto{\pgfqpoint{1.693124in}{3.467160in}}%
\pgfpathlineto{\pgfqpoint{1.694198in}{3.469331in}}%
\pgfpathlineto{\pgfqpoint{1.696345in}{3.484682in}}%
\pgfpathlineto{\pgfqpoint{1.699567in}{3.485767in}}%
\pgfpathlineto{\pgfqpoint{1.700641in}{3.488093in}}%
\pgfpathlineto{\pgfqpoint{1.701715in}{3.487318in}}%
\pgfpathlineto{\pgfqpoint{1.707084in}{3.485767in}}%
\pgfpathlineto{\pgfqpoint{1.708158in}{3.484062in}}%
\pgfpathlineto{\pgfqpoint{1.709231in}{3.480185in}}%
\pgfpathlineto{\pgfqpoint{1.711379in}{3.477394in}}%
\pgfpathlineto{\pgfqpoint{1.714601in}{3.471502in}}%
\pgfpathlineto{\pgfqpoint{1.715674in}{3.464679in}}%
\pgfpathlineto{\pgfqpoint{1.716748in}{3.464834in}}%
\pgfpathlineto{\pgfqpoint{1.717822in}{3.468555in}}%
\pgfpathlineto{\pgfqpoint{1.718896in}{3.464059in}}%
\pgfpathlineto{\pgfqpoint{1.722117in}{3.463438in}}%
\pgfpathlineto{\pgfqpoint{1.725339in}{3.458321in}}%
\pgfpathlineto{\pgfqpoint{1.726413in}{3.458476in}}%
\pgfpathlineto{\pgfqpoint{1.730708in}{3.461733in}}%
\pgfpathlineto{\pgfqpoint{1.733930in}{3.469951in}}%
\pgfpathlineto{\pgfqpoint{1.737151in}{3.471347in}}%
\pgfpathlineto{\pgfqpoint{1.738225in}{3.468711in}}%
\pgfpathlineto{\pgfqpoint{1.739299in}{3.461733in}}%
\pgfpathlineto{\pgfqpoint{1.740373in}{3.464834in}}%
\pgfpathlineto{\pgfqpoint{1.741447in}{3.462353in}}%
\pgfpathlineto{\pgfqpoint{1.744668in}{3.466385in}}%
\pgfpathlineto{\pgfqpoint{1.745742in}{3.469796in}}%
\pgfpathlineto{\pgfqpoint{1.746816in}{3.465764in}}%
\pgfpathlineto{\pgfqpoint{1.747890in}{3.470261in}}%
\pgfpathlineto{\pgfqpoint{1.748963in}{3.470416in}}%
\pgfpathlineto{\pgfqpoint{1.753259in}{3.473052in}}%
\pgfpathlineto{\pgfqpoint{1.754333in}{3.473673in}}%
\pgfpathlineto{\pgfqpoint{1.755406in}{3.475223in}}%
\pgfpathlineto{\pgfqpoint{1.756480in}{3.478635in}}%
\pgfpathlineto{\pgfqpoint{1.760776in}{3.478790in}}%
\pgfpathlineto{\pgfqpoint{1.761850in}{3.479875in}}%
\pgfpathlineto{\pgfqpoint{1.762923in}{3.479720in}}%
\pgfpathlineto{\pgfqpoint{1.763997in}{3.482356in}}%
\pgfpathlineto{\pgfqpoint{1.767219in}{3.481891in}}%
\pgfpathlineto{\pgfqpoint{1.768293in}{3.484682in}}%
\pgfpathlineto{\pgfqpoint{1.769366in}{3.491505in}}%
\pgfpathlineto{\pgfqpoint{1.770440in}{3.491195in}}%
\pgfpathlineto{\pgfqpoint{1.771514in}{3.492590in}}%
\pgfpathlineto{\pgfqpoint{1.774736in}{3.494141in}}%
\pgfpathlineto{\pgfqpoint{1.776883in}{3.488249in}}%
\pgfpathlineto{\pgfqpoint{1.777957in}{3.490264in}}%
\pgfpathlineto{\pgfqpoint{1.779031in}{3.485147in}}%
\pgfpathlineto{\pgfqpoint{1.782252in}{3.487938in}}%
\pgfpathlineto{\pgfqpoint{1.783326in}{3.481891in}}%
\pgfpathlineto{\pgfqpoint{1.784400in}{3.482046in}}%
\pgfpathlineto{\pgfqpoint{1.785474in}{3.484837in}}%
\pgfpathlineto{\pgfqpoint{1.786548in}{3.480185in}}%
\pgfpathlineto{\pgfqpoint{1.789769in}{3.485457in}}%
\pgfpathlineto{\pgfqpoint{1.790843in}{3.483597in}}%
\pgfpathlineto{\pgfqpoint{1.791917in}{3.487938in}}%
\pgfpathlineto{\pgfqpoint{1.792991in}{3.483907in}}%
\pgfpathlineto{\pgfqpoint{1.794065in}{3.484837in}}%
\pgfpathlineto{\pgfqpoint{1.797286in}{3.485767in}}%
\pgfpathlineto{\pgfqpoint{1.798360in}{3.483131in}}%
\pgfpathlineto{\pgfqpoint{1.799434in}{3.478480in}}%
\pgfpathlineto{\pgfqpoint{1.800508in}{3.477084in}}%
\pgfpathlineto{\pgfqpoint{1.801582in}{3.477859in}}%
\pgfpathlineto{\pgfqpoint{1.804803in}{3.481116in}}%
\pgfpathlineto{\pgfqpoint{1.805877in}{3.477084in}}%
\pgfpathlineto{\pgfqpoint{1.806951in}{3.477549in}}%
\pgfpathlineto{\pgfqpoint{1.808025in}{3.478790in}}%
\pgfpathlineto{\pgfqpoint{1.812320in}{3.482046in}}%
\pgfpathlineto{\pgfqpoint{1.813394in}{3.479875in}}%
\pgfpathlineto{\pgfqpoint{1.814468in}{3.479720in}}%
\pgfpathlineto{\pgfqpoint{1.815541in}{3.489179in}}%
\pgfpathlineto{\pgfqpoint{1.816615in}{3.525309in}}%
\pgfpathlineto{\pgfqpoint{1.819837in}{3.513834in}}%
\pgfpathlineto{\pgfqpoint{1.820911in}{3.515230in}}%
\pgfpathlineto{\pgfqpoint{1.823058in}{3.509337in}}%
\pgfpathlineto{\pgfqpoint{1.824132in}{3.509027in}}%
\pgfpathlineto{\pgfqpoint{1.827354in}{3.505926in}}%
\pgfpathlineto{\pgfqpoint{1.828427in}{3.500809in}}%
\pgfpathlineto{\pgfqpoint{1.829501in}{3.504530in}}%
\pgfpathlineto{\pgfqpoint{1.831649in}{3.503135in}}%
\pgfpathlineto{\pgfqpoint{1.834871in}{3.504065in}}%
\pgfpathlineto{\pgfqpoint{1.835944in}{3.507321in}}%
\pgfpathlineto{\pgfqpoint{1.838092in}{3.506701in}}%
\pgfpathlineto{\pgfqpoint{1.839166in}{3.509802in}}%
\pgfpathlineto{\pgfqpoint{1.842387in}{3.509182in}}%
\pgfpathlineto{\pgfqpoint{1.843461in}{3.504685in}}%
\pgfpathlineto{\pgfqpoint{1.844535in}{3.503290in}}%
\pgfpathlineto{\pgfqpoint{1.846683in}{3.510423in}}%
\pgfpathlineto{\pgfqpoint{1.849904in}{3.504685in}}%
\pgfpathlineto{\pgfqpoint{1.850978in}{3.506081in}}%
\pgfpathlineto{\pgfqpoint{1.853126in}{3.511043in}}%
\pgfpathlineto{\pgfqpoint{1.854200in}{3.509182in}}%
\pgfpathlineto{\pgfqpoint{1.858495in}{3.510268in}}%
\pgfpathlineto{\pgfqpoint{1.859569in}{3.513989in}}%
\pgfpathlineto{\pgfqpoint{1.860643in}{3.515075in}}%
\pgfpathlineto{\pgfqpoint{1.867086in}{3.512438in}}%
\pgfpathlineto{\pgfqpoint{1.868160in}{3.513834in}}%
\pgfpathlineto{\pgfqpoint{1.869233in}{3.509182in}}%
\pgfpathlineto{\pgfqpoint{1.872455in}{3.509337in}}%
\pgfpathlineto{\pgfqpoint{1.873529in}{3.509957in}}%
\pgfpathlineto{\pgfqpoint{1.874603in}{3.512593in}}%
\pgfpathlineto{\pgfqpoint{1.875676in}{3.509027in}}%
\pgfpathlineto{\pgfqpoint{1.876750in}{3.509492in}}%
\pgfpathlineto{\pgfqpoint{1.879972in}{3.508872in}}%
\pgfpathlineto{\pgfqpoint{1.881046in}{3.509957in}}%
\pgfpathlineto{\pgfqpoint{1.882119in}{3.513834in}}%
\pgfpathlineto{\pgfqpoint{1.884267in}{3.510733in}}%
\pgfpathlineto{\pgfqpoint{1.887489in}{3.508407in}}%
\pgfpathlineto{\pgfqpoint{1.889636in}{3.509182in}}%
\pgfpathlineto{\pgfqpoint{1.890710in}{3.513524in}}%
\pgfpathlineto{\pgfqpoint{1.891784in}{3.511818in}}%
\pgfpathlineto{\pgfqpoint{1.895005in}{3.514144in}}%
\pgfpathlineto{\pgfqpoint{1.896079in}{3.515850in}}%
\pgfpathlineto{\pgfqpoint{1.898227in}{3.509182in}}%
\pgfpathlineto{\pgfqpoint{1.899301in}{3.509802in}}%
\pgfpathlineto{\pgfqpoint{1.903596in}{3.502980in}}%
\pgfpathlineto{\pgfqpoint{1.905744in}{3.505771in}}%
\pgfpathlineto{\pgfqpoint{1.910039in}{3.499568in}}%
\pgfpathlineto{\pgfqpoint{1.911113in}{3.501739in}}%
\pgfpathlineto{\pgfqpoint{1.912187in}{3.494141in}}%
\pgfpathlineto{\pgfqpoint{1.913261in}{3.495847in}}%
\pgfpathlineto{\pgfqpoint{1.914335in}{3.499103in}}%
\pgfpathlineto{\pgfqpoint{1.917556in}{3.501739in}}%
\pgfpathlineto{\pgfqpoint{1.921851in}{3.511818in}}%
\pgfpathlineto{\pgfqpoint{1.925073in}{3.510578in}}%
\pgfpathlineto{\pgfqpoint{1.928294in}{3.498948in}}%
\pgfpathlineto{\pgfqpoint{1.929368in}{3.492280in}}%
\pgfpathlineto{\pgfqpoint{1.932590in}{3.494916in}}%
\pgfpathlineto{\pgfqpoint{1.934738in}{3.498948in}}%
\pgfpathlineto{\pgfqpoint{1.935811in}{3.497087in}}%
\pgfpathlineto{\pgfqpoint{1.936885in}{3.496932in}}%
\pgfpathlineto{\pgfqpoint{1.940107in}{3.493831in}}%
\pgfpathlineto{\pgfqpoint{1.941181in}{3.494296in}}%
\pgfpathlineto{\pgfqpoint{1.942254in}{3.496932in}}%
\pgfpathlineto{\pgfqpoint{1.943328in}{3.496002in}}%
\pgfpathlineto{\pgfqpoint{1.944402in}{3.492745in}}%
\pgfpathlineto{\pgfqpoint{1.947624in}{3.498638in}}%
\pgfpathlineto{\pgfqpoint{1.948697in}{3.491660in}}%
\pgfpathlineto{\pgfqpoint{1.949771in}{3.493676in}}%
\pgfpathlineto{\pgfqpoint{1.950845in}{3.492745in}}%
\pgfpathlineto{\pgfqpoint{1.951919in}{3.496622in}}%
\pgfpathlineto{\pgfqpoint{1.955140in}{3.498328in}}%
\pgfpathlineto{\pgfqpoint{1.956214in}{3.496467in}}%
\pgfpathlineto{\pgfqpoint{1.957288in}{3.491970in}}%
\pgfpathlineto{\pgfqpoint{1.959436in}{3.477084in}}%
\pgfpathlineto{\pgfqpoint{1.962657in}{3.467470in}}%
\pgfpathlineto{\pgfqpoint{1.963731in}{3.459717in}}%
\pgfpathlineto{\pgfqpoint{1.966953in}{3.484527in}}%
\pgfpathlineto{\pgfqpoint{1.970174in}{3.480030in}}%
\pgfpathlineto{\pgfqpoint{1.971248in}{3.467625in}}%
\pgfpathlineto{\pgfqpoint{1.972322in}{3.476774in}}%
\pgfpathlineto{\pgfqpoint{1.973396in}{3.475998in}}%
\pgfpathlineto{\pgfqpoint{1.974470in}{3.469331in}}%
\pgfpathlineto{\pgfqpoint{1.978765in}{3.481891in}}%
\pgfpathlineto{\pgfqpoint{1.979839in}{3.476464in}}%
\pgfpathlineto{\pgfqpoint{1.980913in}{3.478169in}}%
\pgfpathlineto{\pgfqpoint{1.981986in}{3.481736in}}%
\pgfpathlineto{\pgfqpoint{1.985208in}{3.479410in}}%
\pgfpathlineto{\pgfqpoint{1.987356in}{3.494606in}}%
\pgfpathlineto{\pgfqpoint{1.988429in}{3.489954in}}%
\pgfpathlineto{\pgfqpoint{1.989503in}{3.482666in}}%
\pgfpathlineto{\pgfqpoint{1.992725in}{3.486543in}}%
\pgfpathlineto{\pgfqpoint{1.994872in}{3.487163in}}%
\pgfpathlineto{\pgfqpoint{1.995946in}{3.484217in}}%
\pgfpathlineto{\pgfqpoint{1.997020in}{3.484217in}}%
\pgfpathlineto{\pgfqpoint{2.000242in}{3.476154in}}%
\pgfpathlineto{\pgfqpoint{2.001315in}{3.479565in}}%
\pgfpathlineto{\pgfqpoint{2.002389in}{3.488249in}}%
\pgfpathlineto{\pgfqpoint{2.003463in}{3.487783in}}%
\pgfpathlineto{\pgfqpoint{2.004537in}{3.491505in}}%
\pgfpathlineto{\pgfqpoint{2.010980in}{3.525464in}}%
\pgfpathlineto{\pgfqpoint{2.012054in}{3.526084in}}%
\pgfpathlineto{\pgfqpoint{2.015275in}{3.526394in}}%
\pgfpathlineto{\pgfqpoint{2.017423in}{3.519881in}}%
\pgfpathlineto{\pgfqpoint{2.018497in}{3.525464in}}%
\pgfpathlineto{\pgfqpoint{2.019571in}{3.538179in}}%
\pgfpathlineto{\pgfqpoint{2.022792in}{3.538334in}}%
\pgfpathlineto{\pgfqpoint{2.023866in}{3.535543in}}%
\pgfpathlineto{\pgfqpoint{2.024940in}{3.536473in}}%
\pgfpathlineto{\pgfqpoint{2.026014in}{3.546087in}}%
\pgfpathlineto{\pgfqpoint{2.027088in}{3.545157in}}%
\pgfpathlineto{\pgfqpoint{2.030309in}{3.545622in}}%
\pgfpathlineto{\pgfqpoint{2.033531in}{3.542831in}}%
\pgfpathlineto{\pgfqpoint{2.034604in}{3.537404in}}%
\pgfpathlineto{\pgfqpoint{2.038900in}{3.546242in}}%
\pgfpathlineto{\pgfqpoint{2.039974in}{3.545622in}}%
\pgfpathlineto{\pgfqpoint{2.041048in}{3.546863in}}%
\pgfpathlineto{\pgfqpoint{2.042121in}{3.550584in}}%
\pgfpathlineto{\pgfqpoint{2.045343in}{3.548413in}}%
\pgfpathlineto{\pgfqpoint{2.047491in}{3.560508in}}%
\pgfpathlineto{\pgfqpoint{2.048564in}{3.553840in}}%
\pgfpathlineto{\pgfqpoint{2.049638in}{3.555391in}}%
\pgfpathlineto{\pgfqpoint{2.052860in}{3.556476in}}%
\pgfpathlineto{\pgfqpoint{2.053934in}{3.555856in}}%
\pgfpathlineto{\pgfqpoint{2.055007in}{3.558492in}}%
\pgfpathlineto{\pgfqpoint{2.056081in}{3.555236in}}%
\pgfpathlineto{\pgfqpoint{2.057155in}{3.560353in}}%
\pgfpathlineto{\pgfqpoint{2.060377in}{3.559423in}}%
\pgfpathlineto{\pgfqpoint{2.061450in}{3.560353in}}%
\pgfpathlineto{\pgfqpoint{2.062524in}{3.556476in}}%
\pgfpathlineto{\pgfqpoint{2.064672in}{3.556476in}}%
\pgfpathlineto{\pgfqpoint{2.067893in}{3.550894in}}%
\pgfpathlineto{\pgfqpoint{2.068967in}{3.553840in}}%
\pgfpathlineto{\pgfqpoint{2.070041in}{3.551204in}}%
\pgfpathlineto{\pgfqpoint{2.071115in}{3.551980in}}%
\pgfpathlineto{\pgfqpoint{2.072189in}{3.558182in}}%
\pgfpathlineto{\pgfqpoint{2.075410in}{3.556632in}}%
\pgfpathlineto{\pgfqpoint{2.076484in}{3.554151in}}%
\pgfpathlineto{\pgfqpoint{2.078632in}{3.560198in}}%
\pgfpathlineto{\pgfqpoint{2.079706in}{3.555081in}}%
\pgfpathlineto{\pgfqpoint{2.082927in}{3.555081in}}%
\pgfpathlineto{\pgfqpoint{2.084001in}{3.555856in}}%
\pgfpathlineto{\pgfqpoint{2.085075in}{3.564695in}}%
\pgfpathlineto{\pgfqpoint{2.087223in}{3.558337in}}%
\pgfpathlineto{\pgfqpoint{2.091518in}{3.561128in}}%
\pgfpathlineto{\pgfqpoint{2.092592in}{3.567331in}}%
\pgfpathlineto{\pgfqpoint{2.093666in}{3.565780in}}%
\pgfpathlineto{\pgfqpoint{2.097961in}{3.566711in}}%
\pgfpathlineto{\pgfqpoint{2.099035in}{3.571673in}}%
\pgfpathlineto{\pgfqpoint{2.100109in}{3.568726in}}%
\pgfpathlineto{\pgfqpoint{2.101182in}{3.569967in}}%
\pgfpathlineto{\pgfqpoint{2.105478in}{3.564075in}}%
\pgfpathlineto{\pgfqpoint{2.106552in}{3.564540in}}%
\pgfpathlineto{\pgfqpoint{2.107626in}{3.558027in}}%
\pgfpathlineto{\pgfqpoint{2.108699in}{3.540815in}}%
\pgfpathlineto{\pgfqpoint{2.109773in}{3.533992in}}%
\pgfpathlineto{\pgfqpoint{2.114069in}{3.536473in}}%
\pgfpathlineto{\pgfqpoint{2.115142in}{3.531046in}}%
\pgfpathlineto{\pgfqpoint{2.116216in}{3.542056in}}%
\pgfpathlineto{\pgfqpoint{2.117290in}{3.534457in}}%
\pgfpathlineto{\pgfqpoint{2.121585in}{3.534457in}}%
\pgfpathlineto{\pgfqpoint{2.122659in}{3.527945in}}%
\pgfpathlineto{\pgfqpoint{2.123733in}{3.535853in}}%
\pgfpathlineto{\pgfqpoint{2.124807in}{3.531046in}}%
\pgfpathlineto{\pgfqpoint{2.128028in}{3.528410in}}%
\pgfpathlineto{\pgfqpoint{2.129102in}{3.531976in}}%
\pgfpathlineto{\pgfqpoint{2.130176in}{3.527945in}}%
\pgfpathlineto{\pgfqpoint{2.131250in}{3.530736in}}%
\pgfpathlineto{\pgfqpoint{2.132324in}{3.542676in}}%
\pgfpathlineto{\pgfqpoint{2.135545in}{3.536473in}}%
\pgfpathlineto{\pgfqpoint{2.136619in}{3.531046in}}%
\pgfpathlineto{\pgfqpoint{2.138767in}{3.543606in}}%
\pgfpathlineto{\pgfqpoint{2.139841in}{3.535078in}}%
\pgfpathlineto{\pgfqpoint{2.143062in}{3.530116in}}%
\pgfpathlineto{\pgfqpoint{2.144136in}{3.531666in}}%
\pgfpathlineto{\pgfqpoint{2.145210in}{3.531976in}}%
\pgfpathlineto{\pgfqpoint{2.146284in}{3.520502in}}%
\pgfpathlineto{\pgfqpoint{2.147358in}{3.531356in}}%
\pgfpathlineto{\pgfqpoint{2.151653in}{3.539419in}}%
\pgfpathlineto{\pgfqpoint{2.152727in}{3.545777in}}%
\pgfpathlineto{\pgfqpoint{2.153801in}{3.542366in}}%
\pgfpathlineto{\pgfqpoint{2.154874in}{3.541590in}}%
\pgfpathlineto{\pgfqpoint{2.158096in}{3.546707in}}%
\pgfpathlineto{\pgfqpoint{2.160244in}{3.540660in}}%
\pgfpathlineto{\pgfqpoint{2.161317in}{3.547483in}}%
\pgfpathlineto{\pgfqpoint{2.162391in}{3.549809in}}%
\pgfpathlineto{\pgfqpoint{2.165613in}{3.546242in}}%
\pgfpathlineto{\pgfqpoint{2.166687in}{3.556166in}}%
\pgfpathlineto{\pgfqpoint{2.167760in}{3.560198in}}%
\pgfpathlineto{\pgfqpoint{2.168834in}{3.560818in}}%
\pgfpathlineto{\pgfqpoint{2.169908in}{3.564075in}}%
\pgfpathlineto{\pgfqpoint{2.173130in}{3.561749in}}%
\pgfpathlineto{\pgfqpoint{2.174203in}{3.558647in}}%
\pgfpathlineto{\pgfqpoint{2.175277in}{3.558492in}}%
\pgfpathlineto{\pgfqpoint{2.176351in}{3.557097in}}%
\pgfpathlineto{\pgfqpoint{2.177425in}{3.562369in}}%
\pgfpathlineto{\pgfqpoint{2.181720in}{3.561594in}}%
\pgfpathlineto{\pgfqpoint{2.182794in}{3.560198in}}%
\pgfpathlineto{\pgfqpoint{2.183868in}{3.570742in}}%
\pgfpathlineto{\pgfqpoint{2.184942in}{3.570277in}}%
\pgfpathlineto{\pgfqpoint{2.188163in}{3.572448in}}%
\pgfpathlineto{\pgfqpoint{2.190311in}{3.572293in}}%
\pgfpathlineto{\pgfqpoint{2.191385in}{3.572758in}}%
\pgfpathlineto{\pgfqpoint{2.195680in}{3.577875in}}%
\pgfpathlineto{\pgfqpoint{2.196754in}{3.577720in}}%
\pgfpathlineto{\pgfqpoint{2.197828in}{3.582527in}}%
\pgfpathlineto{\pgfqpoint{2.198902in}{3.581907in}}%
\pgfpathlineto{\pgfqpoint{2.199976in}{3.583768in}}%
\pgfpathlineto{\pgfqpoint{2.204271in}{3.571052in}}%
\pgfpathlineto{\pgfqpoint{2.205345in}{3.569967in}}%
\pgfpathlineto{\pgfqpoint{2.206419in}{3.566401in}}%
\pgfpathlineto{\pgfqpoint{2.207492in}{3.568416in}}%
\pgfpathlineto{\pgfqpoint{2.210714in}{3.567331in}}%
\pgfpathlineto{\pgfqpoint{2.211788in}{3.568726in}}%
\pgfpathlineto{\pgfqpoint{2.212862in}{3.571052in}}%
\pgfpathlineto{\pgfqpoint{2.215009in}{3.571673in}}%
\pgfpathlineto{\pgfqpoint{2.218231in}{3.572138in}}%
\pgfpathlineto{\pgfqpoint{2.219305in}{3.573378in}}%
\pgfpathlineto{\pgfqpoint{2.220379in}{3.573378in}}%
\pgfpathlineto{\pgfqpoint{2.222526in}{3.568106in}}%
\pgfpathlineto{\pgfqpoint{2.225748in}{3.567021in}}%
\pgfpathlineto{\pgfqpoint{2.226822in}{3.569967in}}%
\pgfpathlineto{\pgfqpoint{2.227895in}{3.570432in}}%
\pgfpathlineto{\pgfqpoint{2.228969in}{3.569967in}}%
\pgfpathlineto{\pgfqpoint{2.230043in}{3.567951in}}%
\pgfpathlineto{\pgfqpoint{2.233265in}{3.569812in}}%
\pgfpathlineto{\pgfqpoint{2.234338in}{3.566401in}}%
\pgfpathlineto{\pgfqpoint{2.235412in}{3.558802in}}%
\pgfpathlineto{\pgfqpoint{2.236486in}{3.556321in}}%
\pgfpathlineto{\pgfqpoint{2.237560in}{3.559423in}}%
\pgfpathlineto{\pgfqpoint{2.240781in}{3.556011in}}%
\pgfpathlineto{\pgfqpoint{2.241855in}{3.564230in}}%
\pgfpathlineto{\pgfqpoint{2.242929in}{3.562369in}}%
\pgfpathlineto{\pgfqpoint{2.244003in}{3.559113in}}%
\pgfpathlineto{\pgfqpoint{2.245077in}{3.553065in}}%
\pgfpathlineto{\pgfqpoint{2.248298in}{3.557252in}}%
\pgfpathlineto{\pgfqpoint{2.249372in}{3.553995in}}%
\pgfpathlineto{\pgfqpoint{2.250446in}{3.552600in}}%
\pgfpathlineto{\pgfqpoint{2.251520in}{3.549188in}}%
\pgfpathlineto{\pgfqpoint{2.252594in}{3.551980in}}%
\pgfpathlineto{\pgfqpoint{2.255815in}{3.550894in}}%
\pgfpathlineto{\pgfqpoint{2.257963in}{3.559113in}}%
\pgfpathlineto{\pgfqpoint{2.259037in}{3.558182in}}%
\pgfpathlineto{\pgfqpoint{2.260111in}{3.559423in}}%
\pgfpathlineto{\pgfqpoint{2.264406in}{3.560973in}}%
\pgfpathlineto{\pgfqpoint{2.266554in}{3.558492in}}%
\pgfpathlineto{\pgfqpoint{2.267627in}{3.557097in}}%
\pgfpathlineto{\pgfqpoint{2.271923in}{3.559733in}}%
\pgfpathlineto{\pgfqpoint{2.272997in}{3.562059in}}%
\pgfpathlineto{\pgfqpoint{2.274070in}{3.561128in}}%
\pgfpathlineto{\pgfqpoint{2.275144in}{3.558337in}}%
\pgfpathlineto{\pgfqpoint{2.278366in}{3.555546in}}%
\pgfpathlineto{\pgfqpoint{2.279440in}{3.563764in}}%
\pgfpathlineto{\pgfqpoint{2.280514in}{3.565780in}}%
\pgfpathlineto{\pgfqpoint{2.281587in}{3.569657in}}%
\pgfpathlineto{\pgfqpoint{2.282661in}{3.569037in}}%
\pgfpathlineto{\pgfqpoint{2.286957in}{3.573689in}}%
\pgfpathlineto{\pgfqpoint{2.288030in}{3.571518in}}%
\pgfpathlineto{\pgfqpoint{2.289104in}{3.577100in}}%
\pgfpathlineto{\pgfqpoint{2.290178in}{3.558492in}}%
\pgfpathlineto{\pgfqpoint{2.293400in}{3.551670in}}%
\pgfpathlineto{\pgfqpoint{2.297695in}{3.581132in}}%
\pgfpathlineto{\pgfqpoint{2.301990in}{3.580511in}}%
\pgfpathlineto{\pgfqpoint{2.303064in}{3.584543in}}%
\pgfpathlineto{\pgfqpoint{2.304138in}{3.585628in}}%
\pgfpathlineto{\pgfqpoint{2.305212in}{3.590746in}}%
\pgfpathlineto{\pgfqpoint{2.308433in}{3.590901in}}%
\pgfpathlineto{\pgfqpoint{2.309507in}{3.591521in}}%
\pgfpathlineto{\pgfqpoint{2.310581in}{3.592916in}}%
\pgfpathlineto{\pgfqpoint{2.312729in}{3.600049in}}%
\pgfpathlineto{\pgfqpoint{2.317024in}{3.600670in}}%
\pgfpathlineto{\pgfqpoint{2.319172in}{3.596018in}}%
\pgfpathlineto{\pgfqpoint{2.320246in}{3.588885in}}%
\pgfpathlineto{\pgfqpoint{2.327762in}{3.576325in}}%
\pgfpathlineto{\pgfqpoint{2.330984in}{3.576480in}}%
\pgfpathlineto{\pgfqpoint{2.332058in}{3.575084in}}%
\pgfpathlineto{\pgfqpoint{2.335279in}{3.578340in}}%
\pgfpathlineto{\pgfqpoint{2.346018in}{3.577720in}}%
\pgfpathlineto{\pgfqpoint{2.347091in}{3.577100in}}%
\pgfpathlineto{\pgfqpoint{2.349239in}{3.580356in}}%
\pgfpathlineto{\pgfqpoint{2.350313in}{3.577875in}}%
\pgfpathlineto{\pgfqpoint{2.353535in}{3.578806in}}%
\pgfpathlineto{\pgfqpoint{2.354608in}{3.577565in}}%
\pgfpathlineto{\pgfqpoint{2.357830in}{3.577565in}}%
\pgfpathlineto{\pgfqpoint{2.362125in}{3.579426in}}%
\pgfpathlineto{\pgfqpoint{2.363199in}{3.577720in}}%
\pgfpathlineto{\pgfqpoint{2.364273in}{3.577255in}}%
\pgfpathlineto{\pgfqpoint{2.365347in}{3.578340in}}%
\pgfpathlineto{\pgfqpoint{2.369642in}{3.575084in}}%
\pgfpathlineto{\pgfqpoint{2.371790in}{3.575084in}}%
\pgfpathlineto{\pgfqpoint{2.372864in}{3.562369in}}%
\pgfpathlineto{\pgfqpoint{2.376085in}{3.567486in}}%
\pgfpathlineto{\pgfqpoint{2.377159in}{3.558802in}}%
\pgfpathlineto{\pgfqpoint{2.378233in}{3.556787in}}%
\pgfpathlineto{\pgfqpoint{2.379307in}{3.560663in}}%
\pgfpathlineto{\pgfqpoint{2.383602in}{3.556321in}}%
\pgfpathlineto{\pgfqpoint{2.386824in}{3.564695in}}%
\pgfpathlineto{\pgfqpoint{2.387897in}{3.562524in}}%
\pgfpathlineto{\pgfqpoint{2.391119in}{3.557717in}}%
\pgfpathlineto{\pgfqpoint{2.392193in}{3.562369in}}%
\pgfpathlineto{\pgfqpoint{2.393267in}{3.562679in}}%
\pgfpathlineto{\pgfqpoint{2.394340in}{3.557717in}}%
\pgfpathlineto{\pgfqpoint{2.395414in}{3.558802in}}%
\pgfpathlineto{\pgfqpoint{2.398636in}{3.559113in}}%
\pgfpathlineto{\pgfqpoint{2.399710in}{3.557252in}}%
\pgfpathlineto{\pgfqpoint{2.400783in}{3.557252in}}%
\pgfpathlineto{\pgfqpoint{2.402931in}{3.551514in}}%
\pgfpathlineto{\pgfqpoint{2.406153in}{3.548413in}}%
\pgfpathlineto{\pgfqpoint{2.407226in}{3.549344in}}%
\pgfpathlineto{\pgfqpoint{2.408300in}{3.549033in}}%
\pgfpathlineto{\pgfqpoint{2.409374in}{3.547328in}}%
\pgfpathlineto{\pgfqpoint{2.410448in}{3.548878in}}%
\pgfpathlineto{\pgfqpoint{2.413669in}{3.548413in}}%
\pgfpathlineto{\pgfqpoint{2.415817in}{3.551204in}}%
\pgfpathlineto{\pgfqpoint{2.416891in}{3.551359in}}%
\pgfpathlineto{\pgfqpoint{2.417965in}{3.550119in}}%
\pgfpathlineto{\pgfqpoint{2.421186in}{3.549344in}}%
\pgfpathlineto{\pgfqpoint{2.422260in}{3.545622in}}%
\pgfpathlineto{\pgfqpoint{2.423334in}{3.548568in}}%
\pgfpathlineto{\pgfqpoint{2.424408in}{3.545312in}}%
\pgfpathlineto{\pgfqpoint{2.425482in}{3.553375in}}%
\pgfpathlineto{\pgfqpoint{2.428703in}{3.551825in}}%
\pgfpathlineto{\pgfqpoint{2.429777in}{3.548723in}}%
\pgfpathlineto{\pgfqpoint{2.431925in}{3.540505in}}%
\pgfpathlineto{\pgfqpoint{2.432999in}{3.542676in}}%
\pgfpathlineto{\pgfqpoint{2.436220in}{3.554616in}}%
\pgfpathlineto{\pgfqpoint{2.437294in}{3.556166in}}%
\pgfpathlineto{\pgfqpoint{2.438368in}{3.558957in}}%
\pgfpathlineto{\pgfqpoint{2.439442in}{3.569657in}}%
\pgfpathlineto{\pgfqpoint{2.440515in}{3.573844in}}%
\pgfpathlineto{\pgfqpoint{2.443737in}{3.571052in}}%
\pgfpathlineto{\pgfqpoint{2.444811in}{3.574309in}}%
\pgfpathlineto{\pgfqpoint{2.445885in}{3.574154in}}%
\pgfpathlineto{\pgfqpoint{2.446958in}{3.574929in}}%
\pgfpathlineto{\pgfqpoint{2.448032in}{3.573223in}}%
\pgfpathlineto{\pgfqpoint{2.451254in}{3.576014in}}%
\pgfpathlineto{\pgfqpoint{2.453402in}{3.582372in}}%
\pgfpathlineto{\pgfqpoint{2.455549in}{3.583768in}}%
\pgfpathlineto{\pgfqpoint{2.458771in}{3.581132in}}%
\pgfpathlineto{\pgfqpoint{2.460918in}{3.574464in}}%
\pgfpathlineto{\pgfqpoint{2.461992in}{3.583147in}}%
\pgfpathlineto{\pgfqpoint{2.463066in}{3.582372in}}%
\pgfpathlineto{\pgfqpoint{2.466288in}{3.579271in}}%
\pgfpathlineto{\pgfqpoint{2.467361in}{3.580046in}}%
\pgfpathlineto{\pgfqpoint{2.468435in}{3.585939in}}%
\pgfpathlineto{\pgfqpoint{2.469509in}{3.585008in}}%
\pgfpathlineto{\pgfqpoint{2.470583in}{3.588420in}}%
\pgfpathlineto{\pgfqpoint{2.473804in}{3.589505in}}%
\pgfpathlineto{\pgfqpoint{2.474878in}{3.587799in}}%
\pgfpathlineto{\pgfqpoint{2.477026in}{3.581287in}}%
\pgfpathlineto{\pgfqpoint{2.478100in}{3.587954in}}%
\pgfpathlineto{\pgfqpoint{2.481321in}{3.590280in}}%
\pgfpathlineto{\pgfqpoint{2.482395in}{3.594777in}}%
\pgfpathlineto{\pgfqpoint{2.484543in}{3.592296in}}%
\pgfpathlineto{\pgfqpoint{2.485617in}{3.593071in}}%
\pgfpathlineto{\pgfqpoint{2.489912in}{3.593382in}}%
\pgfpathlineto{\pgfqpoint{2.490986in}{3.590590in}}%
\pgfpathlineto{\pgfqpoint{2.492060in}{3.590746in}}%
\pgfpathlineto{\pgfqpoint{2.493134in}{3.589195in}}%
\pgfpathlineto{\pgfqpoint{2.498503in}{3.590590in}}%
\pgfpathlineto{\pgfqpoint{2.499577in}{3.588109in}}%
\pgfpathlineto{\pgfqpoint{2.500650in}{3.589350in}}%
\pgfpathlineto{\pgfqpoint{2.504946in}{3.586094in}}%
\pgfpathlineto{\pgfqpoint{2.506020in}{3.587489in}}%
\pgfpathlineto{\pgfqpoint{2.507093in}{3.586249in}}%
\pgfpathlineto{\pgfqpoint{2.514610in}{3.583768in}}%
\pgfpathlineto{\pgfqpoint{2.515684in}{3.574464in}}%
\pgfpathlineto{\pgfqpoint{2.518906in}{3.563764in}}%
\pgfpathlineto{\pgfqpoint{2.521053in}{3.572293in}}%
\pgfpathlineto{\pgfqpoint{2.522127in}{3.571518in}}%
\pgfpathlineto{\pgfqpoint{2.523201in}{3.567331in}}%
\pgfpathlineto{\pgfqpoint{2.526423in}{3.566556in}}%
\pgfpathlineto{\pgfqpoint{2.527496in}{3.562989in}}%
\pgfpathlineto{\pgfqpoint{2.533939in}{3.562524in}}%
\pgfpathlineto{\pgfqpoint{2.536087in}{3.559268in}}%
\pgfpathlineto{\pgfqpoint{2.538235in}{3.563299in}}%
\pgfpathlineto{\pgfqpoint{2.541456in}{3.567641in}}%
\pgfpathlineto{\pgfqpoint{2.542530in}{3.571052in}}%
\pgfpathlineto{\pgfqpoint{2.544678in}{3.573378in}}%
\pgfpathlineto{\pgfqpoint{2.545752in}{3.572293in}}%
\pgfpathlineto{\pgfqpoint{2.550047in}{3.574309in}}%
\pgfpathlineto{\pgfqpoint{2.551121in}{3.571673in}}%
\pgfpathlineto{\pgfqpoint{2.552195in}{3.570742in}}%
\pgfpathlineto{\pgfqpoint{2.553268in}{3.573068in}}%
\pgfpathlineto{\pgfqpoint{2.557564in}{3.567796in}}%
\pgfpathlineto{\pgfqpoint{2.558638in}{3.573068in}}%
\pgfpathlineto{\pgfqpoint{2.559712in}{3.573068in}}%
\pgfpathlineto{\pgfqpoint{2.560785in}{3.572138in}}%
\pgfpathlineto{\pgfqpoint{2.564007in}{3.570432in}}%
\pgfpathlineto{\pgfqpoint{2.567228in}{3.565780in}}%
\pgfpathlineto{\pgfqpoint{2.568302in}{3.574309in}}%
\pgfpathlineto{\pgfqpoint{2.571524in}{3.568571in}}%
\pgfpathlineto{\pgfqpoint{2.572598in}{3.564075in}}%
\pgfpathlineto{\pgfqpoint{2.573671in}{3.567176in}}%
\pgfpathlineto{\pgfqpoint{2.574745in}{3.567021in}}%
\pgfpathlineto{\pgfqpoint{2.575819in}{3.568726in}}%
\pgfpathlineto{\pgfqpoint{2.579041in}{3.566866in}}%
\pgfpathlineto{\pgfqpoint{2.580114in}{3.562059in}}%
\pgfpathlineto{\pgfqpoint{2.583336in}{3.566556in}}%
\pgfpathlineto{\pgfqpoint{2.586557in}{3.562679in}}%
\pgfpathlineto{\pgfqpoint{2.587631in}{3.565160in}}%
\pgfpathlineto{\pgfqpoint{2.588705in}{3.566090in}}%
\pgfpathlineto{\pgfqpoint{2.589779in}{3.568726in}}%
\pgfpathlineto{\pgfqpoint{2.590853in}{3.567641in}}%
\pgfpathlineto{\pgfqpoint{2.594074in}{3.568726in}}%
\pgfpathlineto{\pgfqpoint{2.595148in}{3.570742in}}%
\pgfpathlineto{\pgfqpoint{2.597296in}{3.569502in}}%
\pgfpathlineto{\pgfqpoint{2.598370in}{3.570277in}}%
\pgfpathlineto{\pgfqpoint{2.602665in}{3.571052in}}%
\pgfpathlineto{\pgfqpoint{2.604813in}{3.564385in}}%
\pgfpathlineto{\pgfqpoint{2.609108in}{3.565470in}}%
\pgfpathlineto{\pgfqpoint{2.612330in}{3.574154in}}%
\pgfpathlineto{\pgfqpoint{2.613403in}{3.564230in}}%
\pgfpathlineto{\pgfqpoint{2.616625in}{3.564230in}}%
\pgfpathlineto{\pgfqpoint{2.617699in}{3.562834in}}%
\pgfpathlineto{\pgfqpoint{2.619846in}{3.557717in}}%
\pgfpathlineto{\pgfqpoint{2.620920in}{3.556476in}}%
\pgfpathlineto{\pgfqpoint{2.624142in}{3.555701in}}%
\pgfpathlineto{\pgfqpoint{2.625216in}{3.556476in}}%
\pgfpathlineto{\pgfqpoint{2.626290in}{3.559733in}}%
\pgfpathlineto{\pgfqpoint{2.628437in}{3.559578in}}%
\pgfpathlineto{\pgfqpoint{2.631659in}{3.557562in}}%
\pgfpathlineto{\pgfqpoint{2.633806in}{3.552445in}}%
\pgfpathlineto{\pgfqpoint{2.634880in}{3.554771in}}%
\pgfpathlineto{\pgfqpoint{2.635954in}{3.546397in}}%
\pgfpathlineto{\pgfqpoint{2.639176in}{3.545157in}}%
\pgfpathlineto{\pgfqpoint{2.640249in}{3.543296in}}%
\pgfpathlineto{\pgfqpoint{2.641323in}{3.534457in}}%
\pgfpathlineto{\pgfqpoint{2.642397in}{3.535388in}}%
\pgfpathlineto{\pgfqpoint{2.643471in}{3.543451in}}%
\pgfpathlineto{\pgfqpoint{2.646692in}{3.545157in}}%
\pgfpathlineto{\pgfqpoint{2.647766in}{3.546552in}}%
\pgfpathlineto{\pgfqpoint{2.649914in}{3.535543in}}%
\pgfpathlineto{\pgfqpoint{2.650988in}{3.535078in}}%
\pgfpathlineto{\pgfqpoint{2.655283in}{3.533837in}}%
\pgfpathlineto{\pgfqpoint{2.656357in}{3.534147in}}%
\pgfpathlineto{\pgfqpoint{2.657431in}{3.538799in}}%
\pgfpathlineto{\pgfqpoint{2.658505in}{3.540970in}}%
\pgfpathlineto{\pgfqpoint{2.661726in}{3.542366in}}%
\pgfpathlineto{\pgfqpoint{2.662800in}{3.541745in}}%
\pgfpathlineto{\pgfqpoint{2.663874in}{3.538179in}}%
\pgfpathlineto{\pgfqpoint{2.664948in}{3.536938in}}%
\pgfpathlineto{\pgfqpoint{2.669243in}{3.555701in}}%
\pgfpathlineto{\pgfqpoint{2.670317in}{3.548878in}}%
\pgfpathlineto{\pgfqpoint{2.671391in}{3.552290in}}%
\pgfpathlineto{\pgfqpoint{2.673538in}{3.563454in}}%
\pgfpathlineto{\pgfqpoint{2.676760in}{3.560663in}}%
\pgfpathlineto{\pgfqpoint{2.678908in}{3.546242in}}%
\pgfpathlineto{\pgfqpoint{2.679981in}{3.542986in}}%
\pgfpathlineto{\pgfqpoint{2.684277in}{3.543916in}}%
\pgfpathlineto{\pgfqpoint{2.685351in}{3.538179in}}%
\pgfpathlineto{\pgfqpoint{2.687498in}{3.535543in}}%
\pgfpathlineto{\pgfqpoint{2.688572in}{3.535388in}}%
\pgfpathlineto{\pgfqpoint{2.691794in}{3.541590in}}%
\pgfpathlineto{\pgfqpoint{2.693941in}{3.540195in}}%
\pgfpathlineto{\pgfqpoint{2.695015in}{3.525464in}}%
\pgfpathlineto{\pgfqpoint{2.696089in}{3.523293in}}%
\pgfpathlineto{\pgfqpoint{2.699311in}{3.521742in}}%
\pgfpathlineto{\pgfqpoint{2.701458in}{3.529340in}}%
\pgfpathlineto{\pgfqpoint{2.702532in}{3.532287in}}%
\pgfpathlineto{\pgfqpoint{2.703606in}{3.532131in}}%
\pgfpathlineto{\pgfqpoint{2.706827in}{3.532752in}}%
\pgfpathlineto{\pgfqpoint{2.708975in}{3.534457in}}%
\pgfpathlineto{\pgfqpoint{2.710049in}{3.530891in}}%
\pgfpathlineto{\pgfqpoint{2.711123in}{3.519881in}}%
\pgfpathlineto{\pgfqpoint{2.714344in}{3.513059in}}%
\pgfpathlineto{\pgfqpoint{2.715418in}{3.513214in}}%
\pgfpathlineto{\pgfqpoint{2.717566in}{3.518176in}}%
\pgfpathlineto{\pgfqpoint{2.718640in}{3.514454in}}%
\pgfpathlineto{\pgfqpoint{2.721861in}{3.515540in}}%
\pgfpathlineto{\pgfqpoint{2.722935in}{3.513214in}}%
\pgfpathlineto{\pgfqpoint{2.724009in}{3.514299in}}%
\pgfpathlineto{\pgfqpoint{2.725083in}{3.517711in}}%
\pgfpathlineto{\pgfqpoint{2.726156in}{3.518021in}}%
\pgfpathlineto{\pgfqpoint{2.730452in}{3.514919in}}%
\pgfpathlineto{\pgfqpoint{2.731526in}{3.517090in}}%
\pgfpathlineto{\pgfqpoint{2.732600in}{3.511198in}}%
\pgfpathlineto{\pgfqpoint{2.733673in}{3.509802in}}%
\pgfpathlineto{\pgfqpoint{2.736895in}{3.512128in}}%
\pgfpathlineto{\pgfqpoint{2.737969in}{3.509027in}}%
\pgfpathlineto{\pgfqpoint{2.739043in}{3.508407in}}%
\pgfpathlineto{\pgfqpoint{2.741190in}{3.500654in}}%
\pgfpathlineto{\pgfqpoint{2.744412in}{3.499723in}}%
\pgfpathlineto{\pgfqpoint{2.745486in}{3.501274in}}%
\pgfpathlineto{\pgfqpoint{2.746559in}{3.498328in}}%
\pgfpathlineto{\pgfqpoint{2.747633in}{3.497087in}}%
\pgfpathlineto{\pgfqpoint{2.748707in}{3.499723in}}%
\pgfpathlineto{\pgfqpoint{2.753002in}{3.499103in}}%
\pgfpathlineto{\pgfqpoint{2.754076in}{3.496777in}}%
\pgfpathlineto{\pgfqpoint{2.755150in}{3.500654in}}%
\pgfpathlineto{\pgfqpoint{2.756224in}{3.509027in}}%
\pgfpathlineto{\pgfqpoint{2.760519in}{3.503445in}}%
\pgfpathlineto{\pgfqpoint{2.761593in}{3.505926in}}%
\pgfpathlineto{\pgfqpoint{2.762667in}{3.493211in}}%
\pgfpathlineto{\pgfqpoint{2.763741in}{3.490264in}}%
\pgfpathlineto{\pgfqpoint{2.766962in}{3.488869in}}%
\pgfpathlineto{\pgfqpoint{2.770184in}{3.496467in}}%
\pgfpathlineto{\pgfqpoint{2.771258in}{3.495226in}}%
\pgfpathlineto{\pgfqpoint{2.774479in}{3.502824in}}%
\pgfpathlineto{\pgfqpoint{2.775553in}{3.499103in}}%
\pgfpathlineto{\pgfqpoint{2.776627in}{3.500809in}}%
\pgfpathlineto{\pgfqpoint{2.777701in}{3.507011in}}%
\pgfpathlineto{\pgfqpoint{2.778775in}{3.508717in}}%
\pgfpathlineto{\pgfqpoint{2.781996in}{3.512128in}}%
\pgfpathlineto{\pgfqpoint{2.783070in}{3.509492in}}%
\pgfpathlineto{\pgfqpoint{2.784144in}{3.501584in}}%
\pgfpathlineto{\pgfqpoint{2.786291in}{3.498793in}}%
\pgfpathlineto{\pgfqpoint{2.789513in}{3.504375in}}%
\pgfpathlineto{\pgfqpoint{2.790587in}{3.507631in}}%
\pgfpathlineto{\pgfqpoint{2.791661in}{3.503135in}}%
\pgfpathlineto{\pgfqpoint{2.792734in}{3.503910in}}%
\pgfpathlineto{\pgfqpoint{2.793808in}{3.501739in}}%
\pgfpathlineto{\pgfqpoint{2.797030in}{3.488093in}}%
\pgfpathlineto{\pgfqpoint{2.798104in}{3.487163in}}%
\pgfpathlineto{\pgfqpoint{2.799178in}{3.482976in}}%
\pgfpathlineto{\pgfqpoint{2.800251in}{3.482666in}}%
\pgfpathlineto{\pgfqpoint{2.801325in}{3.481736in}}%
\pgfpathlineto{\pgfqpoint{2.804547in}{3.487163in}}%
\pgfpathlineto{\pgfqpoint{2.805621in}{3.484682in}}%
\pgfpathlineto{\pgfqpoint{2.806694in}{3.483752in}}%
\pgfpathlineto{\pgfqpoint{2.808842in}{3.493831in}}%
\pgfpathlineto{\pgfqpoint{2.814211in}{3.460647in}}%
\pgfpathlineto{\pgfqpoint{2.815285in}{3.458011in}}%
\pgfpathlineto{\pgfqpoint{2.816359in}{3.450413in}}%
\pgfpathlineto{\pgfqpoint{2.819580in}{3.444986in}}%
\pgfpathlineto{\pgfqpoint{2.821728in}{3.439403in}}%
\pgfpathlineto{\pgfqpoint{2.822802in}{3.438318in}}%
\pgfpathlineto{\pgfqpoint{2.823876in}{3.441109in}}%
\pgfpathlineto{\pgfqpoint{2.827097in}{3.440954in}}%
\pgfpathlineto{\pgfqpoint{2.828171in}{3.442195in}}%
\pgfpathlineto{\pgfqpoint{2.830319in}{3.439093in}}%
\pgfpathlineto{\pgfqpoint{2.831393in}{3.446226in}}%
\pgfpathlineto{\pgfqpoint{2.834614in}{3.425138in}}%
\pgfpathlineto{\pgfqpoint{2.835688in}{3.409166in}}%
\pgfpathlineto{\pgfqpoint{2.836762in}{3.414283in}}%
\pgfpathlineto{\pgfqpoint{2.838910in}{3.413663in}}%
\pgfpathlineto{\pgfqpoint{2.842131in}{3.410407in}}%
\pgfpathlineto{\pgfqpoint{2.843205in}{3.408236in}}%
\pgfpathlineto{\pgfqpoint{2.844279in}{3.412733in}}%
\pgfpathlineto{\pgfqpoint{2.846426in}{3.413353in}}%
\pgfpathlineto{\pgfqpoint{2.849648in}{3.412267in}}%
\pgfpathlineto{\pgfqpoint{2.850722in}{3.416454in}}%
\pgfpathlineto{\pgfqpoint{2.851796in}{3.417540in}}%
\pgfpathlineto{\pgfqpoint{2.852869in}{3.414748in}}%
\pgfpathlineto{\pgfqpoint{2.853943in}{3.408856in}}%
\pgfpathlineto{\pgfqpoint{2.857165in}{3.409941in}}%
\pgfpathlineto{\pgfqpoint{2.858239in}{3.407150in}}%
\pgfpathlineto{\pgfqpoint{2.861460in}{3.408236in}}%
\pgfpathlineto{\pgfqpoint{2.864682in}{3.407460in}}%
\pgfpathlineto{\pgfqpoint{2.865756in}{3.411182in}}%
\pgfpathlineto{\pgfqpoint{2.867903in}{3.407305in}}%
\pgfpathlineto{\pgfqpoint{2.868977in}{3.409786in}}%
\pgfpathlineto{\pgfqpoint{2.872199in}{3.409011in}}%
\pgfpathlineto{\pgfqpoint{2.874346in}{3.404514in}}%
\pgfpathlineto{\pgfqpoint{2.880789in}{3.405910in}}%
\pgfpathlineto{\pgfqpoint{2.882937in}{3.404979in}}%
\pgfpathlineto{\pgfqpoint{2.884011in}{3.406220in}}%
\pgfpathlineto{\pgfqpoint{2.884011in}{3.406220in}}%
\pgfusepath{stroke}%
\end{pgfscope}%
\begin{pgfscope}%
\pgfpathrectangle{\pgfqpoint{0.418102in}{3.218007in}}{\pgfqpoint{2.583333in}{0.400885in}}%
\pgfusepath{clip}%
\pgfsetroundcap%
\pgfsetroundjoin%
\pgfsetlinewidth{1.505625pt}%
\definecolor{currentstroke}{rgb}{0.172549,0.627451,0.172549}%
\pgfsetstrokecolor{currentstroke}%
\pgfsetdash{}{0pt}%
\pgfpathmoveto{\pgfqpoint{0.535526in}{3.367609in}}%
\pgfpathlineto{\pgfqpoint{0.537674in}{3.367762in}}%
\pgfpathlineto{\pgfqpoint{0.538747in}{3.368991in}}%
\pgfpathlineto{\pgfqpoint{0.541969in}{3.366534in}}%
\pgfpathlineto{\pgfqpoint{0.544117in}{3.370015in}}%
\pgfpathlineto{\pgfqpoint{0.546264in}{3.368344in}}%
\pgfpathlineto{\pgfqpoint{0.550560in}{3.367282in}}%
\pgfpathlineto{\pgfqpoint{0.551633in}{3.370467in}}%
\pgfpathlineto{\pgfqpoint{0.552707in}{3.369102in}}%
\pgfpathlineto{\pgfqpoint{0.553781in}{3.369102in}}%
\pgfpathlineto{\pgfqpoint{0.558077in}{3.365658in}}%
\pgfpathlineto{\pgfqpoint{0.559150in}{3.368952in}}%
\pgfpathlineto{\pgfqpoint{0.560224in}{3.369701in}}%
\pgfpathlineto{\pgfqpoint{0.564520in}{3.367891in}}%
\pgfpathlineto{\pgfqpoint{0.565593in}{3.365630in}}%
\pgfpathlineto{\pgfqpoint{0.567741in}{3.366685in}}%
\pgfpathlineto{\pgfqpoint{0.568815in}{3.369861in}}%
\pgfpathlineto{\pgfqpoint{0.572036in}{3.369558in}}%
\pgfpathlineto{\pgfqpoint{0.573110in}{3.371066in}}%
\pgfpathlineto{\pgfqpoint{0.574184in}{3.370463in}}%
\pgfpathlineto{\pgfqpoint{0.575258in}{3.371663in}}%
\pgfpathlineto{\pgfqpoint{0.576332in}{3.368782in}}%
\pgfpathlineto{\pgfqpoint{0.580627in}{3.372269in}}%
\pgfpathlineto{\pgfqpoint{0.581701in}{3.370119in}}%
\pgfpathlineto{\pgfqpoint{0.582775in}{3.373037in}}%
\pgfpathlineto{\pgfqpoint{0.583849in}{3.369966in}}%
\pgfpathlineto{\pgfqpoint{0.588144in}{3.368473in}}%
\pgfpathlineto{\pgfqpoint{0.589218in}{3.368767in}}%
\pgfpathlineto{\pgfqpoint{0.591366in}{3.370539in}}%
\pgfpathlineto{\pgfqpoint{0.594587in}{3.368607in}}%
\pgfpathlineto{\pgfqpoint{0.598882in}{3.373242in}}%
\pgfpathlineto{\pgfqpoint{0.602104in}{3.371716in}}%
\pgfpathlineto{\pgfqpoint{0.603178in}{3.366832in}}%
\pgfpathlineto{\pgfqpoint{0.604252in}{3.370800in}}%
\pgfpathlineto{\pgfqpoint{0.605325in}{3.367747in}}%
\pgfpathlineto{\pgfqpoint{0.606399in}{3.367599in}}%
\pgfpathlineto{\pgfqpoint{0.609621in}{3.368636in}}%
\pgfpathlineto{\pgfqpoint{0.610695in}{3.363450in}}%
\pgfpathlineto{\pgfqpoint{0.611768in}{3.361332in}}%
\pgfpathlineto{\pgfqpoint{0.612842in}{3.357458in}}%
\pgfpathlineto{\pgfqpoint{0.613916in}{3.357058in}}%
\pgfpathlineto{\pgfqpoint{0.617138in}{3.357191in}}%
\pgfpathlineto{\pgfqpoint{0.618211in}{3.355730in}}%
\pgfpathlineto{\pgfqpoint{0.619285in}{3.355730in}}%
\pgfpathlineto{\pgfqpoint{0.620359in}{3.353473in}}%
\pgfpathlineto{\pgfqpoint{0.621433in}{3.352809in}}%
\pgfpathlineto{\pgfqpoint{0.624655in}{3.355465in}}%
\pgfpathlineto{\pgfqpoint{0.626802in}{3.355066in}}%
\pgfpathlineto{\pgfqpoint{0.627876in}{3.354535in}}%
\pgfpathlineto{\pgfqpoint{0.628950in}{3.355730in}}%
\pgfpathlineto{\pgfqpoint{0.632171in}{3.356261in}}%
\pgfpathlineto{\pgfqpoint{0.633245in}{3.355727in}}%
\pgfpathlineto{\pgfqpoint{0.635393in}{3.350922in}}%
\pgfpathlineto{\pgfqpoint{0.639688in}{3.347986in}}%
\pgfpathlineto{\pgfqpoint{0.640762in}{3.343314in}}%
\pgfpathlineto{\pgfqpoint{0.641836in}{3.346117in}}%
\pgfpathlineto{\pgfqpoint{0.642910in}{3.343180in}}%
\pgfpathlineto{\pgfqpoint{0.643984in}{3.347323in}}%
\pgfpathlineto{\pgfqpoint{0.647205in}{3.347458in}}%
\pgfpathlineto{\pgfqpoint{0.648279in}{3.342860in}}%
\pgfpathlineto{\pgfqpoint{0.649353in}{3.345311in}}%
\pgfpathlineto{\pgfqpoint{0.650427in}{3.345708in}}%
\pgfpathlineto{\pgfqpoint{0.651500in}{3.343457in}}%
\pgfpathlineto{\pgfqpoint{0.654722in}{3.346302in}}%
\pgfpathlineto{\pgfqpoint{0.655796in}{3.351098in}}%
\pgfpathlineto{\pgfqpoint{0.656870in}{3.352031in}}%
\pgfpathlineto{\pgfqpoint{0.657944in}{3.353780in}}%
\pgfpathlineto{\pgfqpoint{0.659017in}{3.352166in}}%
\pgfpathlineto{\pgfqpoint{0.662239in}{3.354151in}}%
\pgfpathlineto{\pgfqpoint{0.663313in}{3.356311in}}%
\pgfpathlineto{\pgfqpoint{0.664387in}{3.356581in}}%
\pgfpathlineto{\pgfqpoint{0.665460in}{3.354957in}}%
\pgfpathlineto{\pgfqpoint{0.666534in}{3.352249in}}%
\pgfpathlineto{\pgfqpoint{0.669756in}{3.351978in}}%
\pgfpathlineto{\pgfqpoint{0.670830in}{3.351301in}}%
\pgfpathlineto{\pgfqpoint{0.671903in}{3.347781in}}%
\pgfpathlineto{\pgfqpoint{0.674051in}{3.350489in}}%
\pgfpathlineto{\pgfqpoint{0.678346in}{3.344073in}}%
\pgfpathlineto{\pgfqpoint{0.679420in}{3.350352in}}%
\pgfpathlineto{\pgfqpoint{0.681568in}{3.352272in}}%
\pgfpathlineto{\pgfqpoint{0.685863in}{3.349932in}}%
\pgfpathlineto{\pgfqpoint{0.686937in}{3.349932in}}%
\pgfpathlineto{\pgfqpoint{0.689085in}{3.351282in}}%
\pgfpathlineto{\pgfqpoint{0.693380in}{3.352775in}}%
\pgfpathlineto{\pgfqpoint{0.694454in}{3.355897in}}%
\pgfpathlineto{\pgfqpoint{0.695528in}{3.356458in}}%
\pgfpathlineto{\pgfqpoint{0.696602in}{3.362341in}}%
\pgfpathlineto{\pgfqpoint{0.699823in}{3.357887in}}%
\pgfpathlineto{\pgfqpoint{0.700897in}{3.358926in}}%
\pgfpathlineto{\pgfqpoint{0.701971in}{3.351653in}}%
\pgfpathlineto{\pgfqpoint{0.704119in}{3.348360in}}%
\pgfpathlineto{\pgfqpoint{0.707340in}{3.349296in}}%
\pgfpathlineto{\pgfqpoint{0.708414in}{3.353077in}}%
\pgfpathlineto{\pgfqpoint{0.709488in}{3.354157in}}%
\pgfpathlineto{\pgfqpoint{0.710562in}{3.357706in}}%
\pgfpathlineto{\pgfqpoint{0.711635in}{3.354157in}}%
\pgfpathlineto{\pgfqpoint{0.714857in}{3.356662in}}%
\pgfpathlineto{\pgfqpoint{0.715931in}{3.359230in}}%
\pgfpathlineto{\pgfqpoint{0.717005in}{3.358284in}}%
\pgfpathlineto{\pgfqpoint{0.719152in}{3.365227in}}%
\pgfpathlineto{\pgfqpoint{0.722374in}{3.368291in}}%
\pgfpathlineto{\pgfqpoint{0.723448in}{3.371303in}}%
\pgfpathlineto{\pgfqpoint{0.724521in}{3.367574in}}%
\pgfpathlineto{\pgfqpoint{0.725595in}{3.366882in}}%
\pgfpathlineto{\pgfqpoint{0.726669in}{3.360136in}}%
\pgfpathlineto{\pgfqpoint{0.729891in}{3.363625in}}%
\pgfpathlineto{\pgfqpoint{0.733112in}{3.362021in}}%
\pgfpathlineto{\pgfqpoint{0.734186in}{3.358679in}}%
\pgfpathlineto{\pgfqpoint{0.737408in}{3.359080in}}%
\pgfpathlineto{\pgfqpoint{0.738481in}{3.363358in}}%
\pgfpathlineto{\pgfqpoint{0.739555in}{3.363915in}}%
\pgfpathlineto{\pgfqpoint{0.741703in}{3.369991in}}%
\pgfpathlineto{\pgfqpoint{0.744924in}{3.371989in}}%
\pgfpathlineto{\pgfqpoint{0.745998in}{3.373443in}}%
\pgfpathlineto{\pgfqpoint{0.747072in}{3.371989in}}%
\pgfpathlineto{\pgfqpoint{0.749220in}{3.373428in}}%
\pgfpathlineto{\pgfqpoint{0.752441in}{3.370977in}}%
\pgfpathlineto{\pgfqpoint{0.753515in}{3.372387in}}%
\pgfpathlineto{\pgfqpoint{0.754589in}{3.372816in}}%
\pgfpathlineto{\pgfqpoint{0.756737in}{3.363024in}}%
\pgfpathlineto{\pgfqpoint{0.759958in}{3.364199in}}%
\pgfpathlineto{\pgfqpoint{0.762106in}{3.363539in}}%
\pgfpathlineto{\pgfqpoint{0.763180in}{3.361427in}}%
\pgfpathlineto{\pgfqpoint{0.764254in}{3.365783in}}%
\pgfpathlineto{\pgfqpoint{0.767475in}{3.365651in}}%
\pgfpathlineto{\pgfqpoint{0.769623in}{3.368420in}}%
\pgfpathlineto{\pgfqpoint{0.770697in}{3.368953in}}%
\pgfpathlineto{\pgfqpoint{0.771770in}{3.368553in}}%
\pgfpathlineto{\pgfqpoint{0.774992in}{3.369616in}}%
\pgfpathlineto{\pgfqpoint{0.776066in}{3.369079in}}%
\pgfpathlineto{\pgfqpoint{0.777140in}{3.369213in}}%
\pgfpathlineto{\pgfqpoint{0.779287in}{3.370689in}}%
\pgfpathlineto{\pgfqpoint{0.782509in}{3.370015in}}%
\pgfpathlineto{\pgfqpoint{0.785730in}{3.367049in}}%
\pgfpathlineto{\pgfqpoint{0.786804in}{3.368667in}}%
\pgfpathlineto{\pgfqpoint{0.790026in}{3.368128in}}%
\pgfpathlineto{\pgfqpoint{0.792173in}{3.368665in}}%
\pgfpathlineto{\pgfqpoint{0.793247in}{3.366780in}}%
\pgfpathlineto{\pgfqpoint{0.794321in}{3.367453in}}%
\pgfpathlineto{\pgfqpoint{0.798616in}{3.369473in}}%
\pgfpathlineto{\pgfqpoint{0.799690in}{3.370982in}}%
\pgfpathlineto{\pgfqpoint{0.800764in}{3.364121in}}%
\pgfpathlineto{\pgfqpoint{0.801838in}{3.361288in}}%
\pgfpathlineto{\pgfqpoint{0.805059in}{3.362416in}}%
\pgfpathlineto{\pgfqpoint{0.806133in}{3.363557in}}%
\pgfpathlineto{\pgfqpoint{0.807207in}{3.360641in}}%
\pgfpathlineto{\pgfqpoint{0.809355in}{3.358678in}}%
\pgfpathlineto{\pgfqpoint{0.812576in}{3.359162in}}%
\pgfpathlineto{\pgfqpoint{0.813650in}{3.360863in}}%
\pgfpathlineto{\pgfqpoint{0.815798in}{3.357488in}}%
\pgfpathlineto{\pgfqpoint{0.816872in}{3.356547in}}%
\pgfpathlineto{\pgfqpoint{0.820093in}{3.358062in}}%
\pgfpathlineto{\pgfqpoint{0.821167in}{3.357589in}}%
\pgfpathlineto{\pgfqpoint{0.822241in}{3.355696in}}%
\pgfpathlineto{\pgfqpoint{0.823315in}{3.361374in}}%
\pgfpathlineto{\pgfqpoint{0.828684in}{3.362439in}}%
\pgfpathlineto{\pgfqpoint{0.829758in}{3.363506in}}%
\pgfpathlineto{\pgfqpoint{0.830832in}{3.363151in}}%
\pgfpathlineto{\pgfqpoint{0.831905in}{3.361615in}}%
\pgfpathlineto{\pgfqpoint{0.835127in}{3.363361in}}%
\pgfpathlineto{\pgfqpoint{0.837275in}{3.358862in}}%
\pgfpathlineto{\pgfqpoint{0.839422in}{3.359809in}}%
\pgfpathlineto{\pgfqpoint{0.843718in}{3.361233in}}%
\pgfpathlineto{\pgfqpoint{0.844791in}{3.363726in}}%
\pgfpathlineto{\pgfqpoint{0.845865in}{3.364675in}}%
\pgfpathlineto{\pgfqpoint{0.846939in}{3.357489in}}%
\pgfpathlineto{\pgfqpoint{0.850161in}{3.354495in}}%
\pgfpathlineto{\pgfqpoint{0.851234in}{3.350543in}}%
\pgfpathlineto{\pgfqpoint{0.853382in}{3.350423in}}%
\pgfpathlineto{\pgfqpoint{0.854456in}{3.348986in}}%
\pgfpathlineto{\pgfqpoint{0.859825in}{3.348507in}}%
\pgfpathlineto{\pgfqpoint{0.860899in}{3.351142in}}%
\pgfpathlineto{\pgfqpoint{0.865194in}{3.352222in}}%
\pgfpathlineto{\pgfqpoint{0.866268in}{3.350541in}}%
\pgfpathlineto{\pgfqpoint{0.867342in}{3.354672in}}%
\pgfpathlineto{\pgfqpoint{0.868416in}{3.352334in}}%
\pgfpathlineto{\pgfqpoint{0.869490in}{3.353441in}}%
\pgfpathlineto{\pgfqpoint{0.872711in}{3.354549in}}%
\pgfpathlineto{\pgfqpoint{0.873785in}{3.352558in}}%
\pgfpathlineto{\pgfqpoint{0.874859in}{3.346085in}}%
\pgfpathlineto{\pgfqpoint{0.875933in}{3.346583in}}%
\pgfpathlineto{\pgfqpoint{0.877007in}{3.345712in}}%
\pgfpathlineto{\pgfqpoint{0.880228in}{3.340901in}}%
\pgfpathlineto{\pgfqpoint{0.882376in}{3.341842in}}%
\pgfpathlineto{\pgfqpoint{0.884523in}{3.338545in}}%
\pgfpathlineto{\pgfqpoint{0.887745in}{3.338431in}}%
\pgfpathlineto{\pgfqpoint{0.888819in}{3.336841in}}%
\pgfpathlineto{\pgfqpoint{0.889893in}{3.339226in}}%
\pgfpathlineto{\pgfqpoint{0.892040in}{3.339339in}}%
\pgfpathlineto{\pgfqpoint{0.895262in}{3.336611in}}%
\pgfpathlineto{\pgfqpoint{0.896336in}{3.336952in}}%
\pgfpathlineto{\pgfqpoint{0.897409in}{3.333768in}}%
\pgfpathlineto{\pgfqpoint{0.899557in}{3.331801in}}%
\pgfpathlineto{\pgfqpoint{0.902779in}{3.332339in}}%
\pgfpathlineto{\pgfqpoint{0.903853in}{3.333311in}}%
\pgfpathlineto{\pgfqpoint{0.904926in}{3.331042in}}%
\pgfpathlineto{\pgfqpoint{0.906000in}{3.332306in}}%
\pgfpathlineto{\pgfqpoint{0.907074in}{3.332306in}}%
\pgfpathlineto{\pgfqpoint{0.910296in}{3.334872in}}%
\pgfpathlineto{\pgfqpoint{0.911369in}{3.336903in}}%
\pgfpathlineto{\pgfqpoint{0.912443in}{3.331109in}}%
\pgfpathlineto{\pgfqpoint{0.914591in}{3.334498in}}%
\pgfpathlineto{\pgfqpoint{0.921034in}{3.332832in}}%
\pgfpathlineto{\pgfqpoint{0.922108in}{3.330721in}}%
\pgfpathlineto{\pgfqpoint{0.925329in}{3.335498in}}%
\pgfpathlineto{\pgfqpoint{0.927477in}{3.332499in}}%
\pgfpathlineto{\pgfqpoint{0.929625in}{3.335420in}}%
\pgfpathlineto{\pgfqpoint{0.932846in}{3.335970in}}%
\pgfpathlineto{\pgfqpoint{0.933920in}{3.333979in}}%
\pgfpathlineto{\pgfqpoint{0.934994in}{3.334421in}}%
\pgfpathlineto{\pgfqpoint{0.936068in}{3.332541in}}%
\pgfpathlineto{\pgfqpoint{0.937142in}{3.332866in}}%
\pgfpathlineto{\pgfqpoint{0.940363in}{3.332757in}}%
\pgfpathlineto{\pgfqpoint{0.942511in}{3.334062in}}%
\pgfpathlineto{\pgfqpoint{0.943585in}{3.335595in}}%
\pgfpathlineto{\pgfqpoint{0.944658in}{3.329242in}}%
\pgfpathlineto{\pgfqpoint{0.948954in}{3.329549in}}%
\pgfpathlineto{\pgfqpoint{0.950028in}{3.329036in}}%
\pgfpathlineto{\pgfqpoint{0.951101in}{3.329856in}}%
\pgfpathlineto{\pgfqpoint{0.952175in}{3.327909in}}%
\pgfpathlineto{\pgfqpoint{0.956471in}{3.326305in}}%
\pgfpathlineto{\pgfqpoint{0.957544in}{3.324238in}}%
\pgfpathlineto{\pgfqpoint{0.958618in}{3.324632in}}%
\pgfpathlineto{\pgfqpoint{0.959692in}{3.321975in}}%
\pgfpathlineto{\pgfqpoint{0.962914in}{3.324264in}}%
\pgfpathlineto{\pgfqpoint{0.965061in}{3.326813in}}%
\pgfpathlineto{\pgfqpoint{0.967209in}{3.327011in}}%
\pgfpathlineto{\pgfqpoint{0.970431in}{3.326715in}}%
\pgfpathlineto{\pgfqpoint{0.971504in}{3.327703in}}%
\pgfpathlineto{\pgfqpoint{0.972578in}{3.321477in}}%
\pgfpathlineto{\pgfqpoint{0.973652in}{3.321385in}}%
\pgfpathlineto{\pgfqpoint{0.974726in}{3.320558in}}%
\pgfpathlineto{\pgfqpoint{0.979021in}{3.323866in}}%
\pgfpathlineto{\pgfqpoint{0.980095in}{3.326347in}}%
\pgfpathlineto{\pgfqpoint{0.981169in}{3.326631in}}%
\pgfpathlineto{\pgfqpoint{0.982243in}{3.325590in}}%
\pgfpathlineto{\pgfqpoint{0.986538in}{3.331758in}}%
\pgfpathlineto{\pgfqpoint{0.987612in}{3.329301in}}%
\pgfpathlineto{\pgfqpoint{0.988686in}{3.330448in}}%
\pgfpathlineto{\pgfqpoint{0.989760in}{3.330254in}}%
\pgfpathlineto{\pgfqpoint{0.992981in}{3.330835in}}%
\pgfpathlineto{\pgfqpoint{0.994055in}{3.328414in}}%
\pgfpathlineto{\pgfqpoint{0.995129in}{3.327754in}}%
\pgfpathlineto{\pgfqpoint{0.996203in}{3.327754in}}%
\pgfpathlineto{\pgfqpoint{0.997276in}{3.328502in}}%
\pgfpathlineto{\pgfqpoint{1.000498in}{3.329625in}}%
\pgfpathlineto{\pgfqpoint{1.001572in}{3.328014in}}%
\pgfpathlineto{\pgfqpoint{1.002646in}{3.328677in}}%
\pgfpathlineto{\pgfqpoint{1.003720in}{3.327256in}}%
\pgfpathlineto{\pgfqpoint{1.004793in}{3.329027in}}%
\pgfpathlineto{\pgfqpoint{1.008015in}{3.327600in}}%
\pgfpathlineto{\pgfqpoint{1.009089in}{3.328076in}}%
\pgfpathlineto{\pgfqpoint{1.010163in}{3.327029in}}%
\pgfpathlineto{\pgfqpoint{1.012310in}{3.328823in}}%
\pgfpathlineto{\pgfqpoint{1.015532in}{3.329777in}}%
\pgfpathlineto{\pgfqpoint{1.016606in}{3.328813in}}%
\pgfpathlineto{\pgfqpoint{1.023049in}{3.328524in}}%
\pgfpathlineto{\pgfqpoint{1.026270in}{3.333746in}}%
\pgfpathlineto{\pgfqpoint{1.027344in}{3.334838in}}%
\pgfpathlineto{\pgfqpoint{1.031639in}{3.336647in}}%
\pgfpathlineto{\pgfqpoint{1.032713in}{3.340784in}}%
\pgfpathlineto{\pgfqpoint{1.033787in}{3.340784in}}%
\pgfpathlineto{\pgfqpoint{1.034861in}{3.339775in}}%
\pgfpathlineto{\pgfqpoint{1.038082in}{3.334628in}}%
\pgfpathlineto{\pgfqpoint{1.040230in}{3.339674in}}%
\pgfpathlineto{\pgfqpoint{1.041304in}{3.338946in}}%
\pgfpathlineto{\pgfqpoint{1.042378in}{3.331461in}}%
\pgfpathlineto{\pgfqpoint{1.045599in}{3.328134in}}%
\pgfpathlineto{\pgfqpoint{1.046673in}{3.329381in}}%
\pgfpathlineto{\pgfqpoint{1.047747in}{3.325638in}}%
\pgfpathlineto{\pgfqpoint{1.048821in}{3.325638in}}%
\pgfpathlineto{\pgfqpoint{1.049895in}{3.327631in}}%
\pgfpathlineto{\pgfqpoint{1.054190in}{3.327034in}}%
\pgfpathlineto{\pgfqpoint{1.055264in}{3.325945in}}%
\pgfpathlineto{\pgfqpoint{1.056338in}{3.327331in}}%
\pgfpathlineto{\pgfqpoint{1.057411in}{3.325451in}}%
\pgfpathlineto{\pgfqpoint{1.060633in}{3.325354in}}%
\pgfpathlineto{\pgfqpoint{1.061707in}{3.326127in}}%
\pgfpathlineto{\pgfqpoint{1.062781in}{3.323613in}}%
\pgfpathlineto{\pgfqpoint{1.064928in}{3.326166in}}%
\pgfpathlineto{\pgfqpoint{1.068150in}{3.326549in}}%
\pgfpathlineto{\pgfqpoint{1.069224in}{3.327801in}}%
\pgfpathlineto{\pgfqpoint{1.070297in}{3.326068in}}%
\pgfpathlineto{\pgfqpoint{1.071371in}{3.325879in}}%
\pgfpathlineto{\pgfqpoint{1.072445in}{3.324467in}}%
\pgfpathlineto{\pgfqpoint{1.076741in}{3.322992in}}%
\pgfpathlineto{\pgfqpoint{1.077814in}{3.321535in}}%
\pgfpathlineto{\pgfqpoint{1.078888in}{3.322967in}}%
\pgfpathlineto{\pgfqpoint{1.079962in}{3.322057in}}%
\pgfpathlineto{\pgfqpoint{1.086405in}{3.322511in}}%
\pgfpathlineto{\pgfqpoint{1.087479in}{3.320509in}}%
\pgfpathlineto{\pgfqpoint{1.090700in}{3.322784in}}%
\pgfpathlineto{\pgfqpoint{1.091774in}{3.322602in}}%
\pgfpathlineto{\pgfqpoint{1.092848in}{3.325054in}}%
\pgfpathlineto{\pgfqpoint{1.093922in}{3.325521in}}%
\pgfpathlineto{\pgfqpoint{1.094996in}{3.321969in}}%
\pgfpathlineto{\pgfqpoint{1.098217in}{3.322596in}}%
\pgfpathlineto{\pgfqpoint{1.100365in}{3.320608in}}%
\pgfpathlineto{\pgfqpoint{1.102513in}{3.322958in}}%
\pgfpathlineto{\pgfqpoint{1.105734in}{3.324789in}}%
\pgfpathlineto{\pgfqpoint{1.106808in}{3.320761in}}%
\pgfpathlineto{\pgfqpoint{1.107882in}{3.323206in}}%
\pgfpathlineto{\pgfqpoint{1.108956in}{3.319339in}}%
\pgfpathlineto{\pgfqpoint{1.113251in}{3.323206in}}%
\pgfpathlineto{\pgfqpoint{1.114325in}{3.324512in}}%
\pgfpathlineto{\pgfqpoint{1.116473in}{3.322933in}}%
\pgfpathlineto{\pgfqpoint{1.117546in}{3.323849in}}%
\pgfpathlineto{\pgfqpoint{1.120768in}{3.324867in}}%
\pgfpathlineto{\pgfqpoint{1.121842in}{3.328015in}}%
\pgfpathlineto{\pgfqpoint{1.122916in}{3.328015in}}%
\pgfpathlineto{\pgfqpoint{1.125063in}{3.330512in}}%
\pgfpathlineto{\pgfqpoint{1.128285in}{3.329840in}}%
\pgfpathlineto{\pgfqpoint{1.129359in}{3.327552in}}%
\pgfpathlineto{\pgfqpoint{1.130432in}{3.328202in}}%
\pgfpathlineto{\pgfqpoint{1.131506in}{3.331197in}}%
\pgfpathlineto{\pgfqpoint{1.132580in}{3.332507in}}%
\pgfpathlineto{\pgfqpoint{1.135802in}{3.331462in}}%
\pgfpathlineto{\pgfqpoint{1.136875in}{3.330038in}}%
\pgfpathlineto{\pgfqpoint{1.137949in}{3.330797in}}%
\pgfpathlineto{\pgfqpoint{1.139023in}{3.330133in}}%
\pgfpathlineto{\pgfqpoint{1.140097in}{3.321932in}}%
\pgfpathlineto{\pgfqpoint{1.143319in}{3.320988in}}%
\pgfpathlineto{\pgfqpoint{1.144392in}{3.322007in}}%
\pgfpathlineto{\pgfqpoint{1.145466in}{3.321405in}}%
\pgfpathlineto{\pgfqpoint{1.147614in}{3.322093in}}%
\pgfpathlineto{\pgfqpoint{1.152983in}{3.320196in}}%
\pgfpathlineto{\pgfqpoint{1.154057in}{3.321920in}}%
\pgfpathlineto{\pgfqpoint{1.155131in}{3.321403in}}%
\pgfpathlineto{\pgfqpoint{1.158352in}{3.322603in}}%
\pgfpathlineto{\pgfqpoint{1.159426in}{3.321125in}}%
\pgfpathlineto{\pgfqpoint{1.161574in}{3.321472in}}%
\pgfpathlineto{\pgfqpoint{1.162648in}{3.320863in}}%
\pgfpathlineto{\pgfqpoint{1.166943in}{3.321472in}}%
\pgfpathlineto{\pgfqpoint{1.169091in}{3.320073in}}%
\pgfpathlineto{\pgfqpoint{1.174460in}{3.318061in}}%
\pgfpathlineto{\pgfqpoint{1.175534in}{3.317361in}}%
\pgfpathlineto{\pgfqpoint{1.176608in}{3.318498in}}%
\pgfpathlineto{\pgfqpoint{1.177681in}{3.318498in}}%
\pgfpathlineto{\pgfqpoint{1.180903in}{3.317361in}}%
\pgfpathlineto{\pgfqpoint{1.181977in}{3.314387in}}%
\pgfpathlineto{\pgfqpoint{1.183051in}{3.314474in}}%
\pgfpathlineto{\pgfqpoint{1.184124in}{3.313862in}}%
\pgfpathlineto{\pgfqpoint{1.195937in}{3.317206in}}%
\pgfpathlineto{\pgfqpoint{1.197010in}{3.313853in}}%
\pgfpathlineto{\pgfqpoint{1.198084in}{3.312329in}}%
\pgfpathlineto{\pgfqpoint{1.199158in}{3.313907in}}%
\pgfpathlineto{\pgfqpoint{1.200232in}{3.313399in}}%
\pgfpathlineto{\pgfqpoint{1.203453in}{3.315857in}}%
\pgfpathlineto{\pgfqpoint{1.205601in}{3.311095in}}%
\pgfpathlineto{\pgfqpoint{1.206675in}{3.312453in}}%
\pgfpathlineto{\pgfqpoint{1.207749in}{3.309527in}}%
\pgfpathlineto{\pgfqpoint{1.210970in}{3.311315in}}%
\pgfpathlineto{\pgfqpoint{1.212044in}{3.311071in}}%
\pgfpathlineto{\pgfqpoint{1.214192in}{3.311720in}}%
\pgfpathlineto{\pgfqpoint{1.215266in}{3.310414in}}%
\pgfpathlineto{\pgfqpoint{1.218487in}{3.309434in}}%
\pgfpathlineto{\pgfqpoint{1.219561in}{3.311230in}}%
\pgfpathlineto{\pgfqpoint{1.220635in}{3.310169in}}%
\pgfpathlineto{\pgfqpoint{1.221709in}{3.311699in}}%
\pgfpathlineto{\pgfqpoint{1.226004in}{3.310714in}}%
\pgfpathlineto{\pgfqpoint{1.227078in}{3.308907in}}%
\pgfpathlineto{\pgfqpoint{1.228152in}{3.308250in}}%
\pgfpathlineto{\pgfqpoint{1.229226in}{3.312685in}}%
\pgfpathlineto{\pgfqpoint{1.230299in}{3.311699in}}%
\pgfpathlineto{\pgfqpoint{1.233521in}{3.311780in}}%
\pgfpathlineto{\pgfqpoint{1.234595in}{3.310481in}}%
\pgfpathlineto{\pgfqpoint{1.235669in}{3.311618in}}%
\pgfpathlineto{\pgfqpoint{1.236742in}{3.309507in}}%
\pgfpathlineto{\pgfqpoint{1.237816in}{3.304052in}}%
\pgfpathlineto{\pgfqpoint{1.241038in}{3.300588in}}%
\pgfpathlineto{\pgfqpoint{1.242112in}{3.301222in}}%
\pgfpathlineto{\pgfqpoint{1.243185in}{3.299376in}}%
\pgfpathlineto{\pgfqpoint{1.244259in}{3.300725in}}%
\pgfpathlineto{\pgfqpoint{1.248555in}{3.302292in}}%
\pgfpathlineto{\pgfqpoint{1.250702in}{3.300731in}}%
\pgfpathlineto{\pgfqpoint{1.251776in}{3.301989in}}%
\pgfpathlineto{\pgfqpoint{1.252850in}{3.304262in}}%
\pgfpathlineto{\pgfqpoint{1.257145in}{3.304830in}}%
\pgfpathlineto{\pgfqpoint{1.260367in}{3.311982in}}%
\pgfpathlineto{\pgfqpoint{1.266810in}{3.312783in}}%
\pgfpathlineto{\pgfqpoint{1.267884in}{3.314038in}}%
\pgfpathlineto{\pgfqpoint{1.271105in}{3.313964in}}%
\pgfpathlineto{\pgfqpoint{1.273253in}{3.312489in}}%
\pgfpathlineto{\pgfqpoint{1.274327in}{3.312194in}}%
\pgfpathlineto{\pgfqpoint{1.275401in}{3.313153in}}%
\pgfpathlineto{\pgfqpoint{1.279696in}{3.315521in}}%
\pgfpathlineto{\pgfqpoint{1.280770in}{3.315218in}}%
\pgfpathlineto{\pgfqpoint{1.282918in}{3.316198in}}%
\pgfpathlineto{\pgfqpoint{1.286139in}{3.316198in}}%
\pgfpathlineto{\pgfqpoint{1.287213in}{3.315588in}}%
\pgfpathlineto{\pgfqpoint{1.288287in}{3.316045in}}%
\pgfpathlineto{\pgfqpoint{1.289361in}{3.317190in}}%
\pgfpathlineto{\pgfqpoint{1.290434in}{3.320209in}}%
\pgfpathlineto{\pgfqpoint{1.293656in}{3.318661in}}%
\pgfpathlineto{\pgfqpoint{1.294730in}{3.318965in}}%
\pgfpathlineto{\pgfqpoint{1.295804in}{3.315533in}}%
\pgfpathlineto{\pgfqpoint{1.296877in}{3.315304in}}%
\pgfpathlineto{\pgfqpoint{1.297951in}{3.317135in}}%
\pgfpathlineto{\pgfqpoint{1.302247in}{3.315994in}}%
\pgfpathlineto{\pgfqpoint{1.303320in}{3.313738in}}%
\pgfpathlineto{\pgfqpoint{1.305468in}{3.312716in}}%
\pgfpathlineto{\pgfqpoint{1.308690in}{3.312499in}}%
\pgfpathlineto{\pgfqpoint{1.309763in}{3.311275in}}%
\pgfpathlineto{\pgfqpoint{1.312985in}{3.309999in}}%
\pgfpathlineto{\pgfqpoint{1.316207in}{3.310347in}}%
\pgfpathlineto{\pgfqpoint{1.317280in}{3.309580in}}%
\pgfpathlineto{\pgfqpoint{1.319428in}{3.312551in}}%
\pgfpathlineto{\pgfqpoint{1.320502in}{3.312479in}}%
\pgfpathlineto{\pgfqpoint{1.325871in}{3.310900in}}%
\pgfpathlineto{\pgfqpoint{1.326945in}{3.310972in}}%
\pgfpathlineto{\pgfqpoint{1.328019in}{3.309464in}}%
\pgfpathlineto{\pgfqpoint{1.331240in}{3.308101in}}%
\pgfpathlineto{\pgfqpoint{1.332314in}{3.309536in}}%
\pgfpathlineto{\pgfqpoint{1.333388in}{3.307383in}}%
\pgfpathlineto{\pgfqpoint{1.334462in}{3.308221in}}%
\pgfpathlineto{\pgfqpoint{1.335536in}{3.304691in}}%
\pgfpathlineto{\pgfqpoint{1.339831in}{3.303068in}}%
\pgfpathlineto{\pgfqpoint{1.341979in}{3.300386in}}%
\pgfpathlineto{\pgfqpoint{1.343052in}{3.295444in}}%
\pgfpathlineto{\pgfqpoint{1.346274in}{3.296150in}}%
\pgfpathlineto{\pgfqpoint{1.347348in}{3.293962in}}%
\pgfpathlineto{\pgfqpoint{1.349496in}{3.295961in}}%
\pgfpathlineto{\pgfqpoint{1.350569in}{3.297972in}}%
\pgfpathlineto{\pgfqpoint{1.353791in}{3.293479in}}%
\pgfpathlineto{\pgfqpoint{1.354865in}{3.294763in}}%
\pgfpathlineto{\pgfqpoint{1.355939in}{3.295048in}}%
\pgfpathlineto{\pgfqpoint{1.357012in}{3.297482in}}%
\pgfpathlineto{\pgfqpoint{1.358086in}{3.296050in}}%
\pgfpathlineto{\pgfqpoint{1.361308in}{3.296892in}}%
\pgfpathlineto{\pgfqpoint{1.362382in}{3.299094in}}%
\pgfpathlineto{\pgfqpoint{1.364529in}{3.299592in}}%
\pgfpathlineto{\pgfqpoint{1.365603in}{3.297883in}}%
\pgfpathlineto{\pgfqpoint{1.369898in}{3.298370in}}%
\pgfpathlineto{\pgfqpoint{1.370972in}{3.296899in}}%
\pgfpathlineto{\pgfqpoint{1.372046in}{3.296618in}}%
\pgfpathlineto{\pgfqpoint{1.373120in}{3.295568in}}%
\pgfpathlineto{\pgfqpoint{1.377415in}{3.297599in}}%
\pgfpathlineto{\pgfqpoint{1.378489in}{3.297739in}}%
\pgfpathlineto{\pgfqpoint{1.379563in}{3.296617in}}%
\pgfpathlineto{\pgfqpoint{1.380637in}{3.296755in}}%
\pgfpathlineto{\pgfqpoint{1.383858in}{3.294819in}}%
\pgfpathlineto{\pgfqpoint{1.384932in}{3.297793in}}%
\pgfpathlineto{\pgfqpoint{1.387080in}{3.294646in}}%
\pgfpathlineto{\pgfqpoint{1.388154in}{3.295109in}}%
\pgfpathlineto{\pgfqpoint{1.392449in}{3.293843in}}%
\pgfpathlineto{\pgfqpoint{1.393523in}{3.293110in}}%
\pgfpathlineto{\pgfqpoint{1.395671in}{3.289579in}}%
\pgfpathlineto{\pgfqpoint{1.398892in}{3.291311in}}%
\pgfpathlineto{\pgfqpoint{1.399966in}{3.290112in}}%
\pgfpathlineto{\pgfqpoint{1.401040in}{3.292076in}}%
\pgfpathlineto{\pgfqpoint{1.402114in}{3.292009in}}%
\pgfpathlineto{\pgfqpoint{1.403187in}{3.292683in}}%
\pgfpathlineto{\pgfqpoint{1.406409in}{3.292615in}}%
\pgfpathlineto{\pgfqpoint{1.407483in}{3.294231in}}%
\pgfpathlineto{\pgfqpoint{1.408557in}{3.294703in}}%
\pgfpathlineto{\pgfqpoint{1.409630in}{3.295787in}}%
\pgfpathlineto{\pgfqpoint{1.410704in}{3.295448in}}%
\pgfpathlineto{\pgfqpoint{1.415000in}{3.295246in}}%
\pgfpathlineto{\pgfqpoint{1.416073in}{3.296190in}}%
\pgfpathlineto{\pgfqpoint{1.417147in}{3.295111in}}%
\pgfpathlineto{\pgfqpoint{1.418221in}{3.296240in}}%
\pgfpathlineto{\pgfqpoint{1.422517in}{3.294756in}}%
\pgfpathlineto{\pgfqpoint{1.423590in}{3.295835in}}%
\pgfpathlineto{\pgfqpoint{1.424664in}{3.297859in}}%
\pgfpathlineto{\pgfqpoint{1.425738in}{3.297026in}}%
\pgfpathlineto{\pgfqpoint{1.428960in}{3.298553in}}%
\pgfpathlineto{\pgfqpoint{1.430033in}{3.297928in}}%
\pgfpathlineto{\pgfqpoint{1.432181in}{3.293856in}}%
\pgfpathlineto{\pgfqpoint{1.437550in}{3.293726in}}%
\pgfpathlineto{\pgfqpoint{1.438624in}{3.292883in}}%
\pgfpathlineto{\pgfqpoint{1.439698in}{3.293077in}}%
\pgfpathlineto{\pgfqpoint{1.440772in}{3.292299in}}%
\pgfpathlineto{\pgfqpoint{1.445067in}{3.291464in}}%
\pgfpathlineto{\pgfqpoint{1.447215in}{3.292732in}}%
\pgfpathlineto{\pgfqpoint{1.448289in}{3.292284in}}%
\pgfpathlineto{\pgfqpoint{1.451510in}{3.291772in}}%
\pgfpathlineto{\pgfqpoint{1.452584in}{3.289725in}}%
\pgfpathlineto{\pgfqpoint{1.453658in}{3.291516in}}%
\pgfpathlineto{\pgfqpoint{1.461175in}{3.294535in}}%
\pgfpathlineto{\pgfqpoint{1.462249in}{3.293693in}}%
\pgfpathlineto{\pgfqpoint{1.463322in}{3.294017in}}%
\pgfpathlineto{\pgfqpoint{1.466544in}{3.294341in}}%
\pgfpathlineto{\pgfqpoint{1.467618in}{3.292713in}}%
\pgfpathlineto{\pgfqpoint{1.468692in}{3.293755in}}%
\pgfpathlineto{\pgfqpoint{1.470839in}{3.293690in}}%
\pgfpathlineto{\pgfqpoint{1.475135in}{3.294015in}}%
\pgfpathlineto{\pgfqpoint{1.478356in}{3.292850in}}%
\pgfpathlineto{\pgfqpoint{1.482651in}{3.292850in}}%
\pgfpathlineto{\pgfqpoint{1.483725in}{3.291634in}}%
\pgfpathlineto{\pgfqpoint{1.484799in}{3.292722in}}%
\pgfpathlineto{\pgfqpoint{1.485873in}{3.290547in}}%
\pgfpathlineto{\pgfqpoint{1.489095in}{3.289245in}}%
\pgfpathlineto{\pgfqpoint{1.490168in}{3.289427in}}%
\pgfpathlineto{\pgfqpoint{1.492316in}{3.287232in}}%
\pgfpathlineto{\pgfqpoint{1.497685in}{3.288882in}}%
\pgfpathlineto{\pgfqpoint{1.498759in}{3.288820in}}%
\pgfpathlineto{\pgfqpoint{1.499833in}{3.290120in}}%
\pgfpathlineto{\pgfqpoint{1.500907in}{3.289872in}}%
\pgfpathlineto{\pgfqpoint{1.504128in}{3.291352in}}%
\pgfpathlineto{\pgfqpoint{1.507350in}{3.289336in}}%
\pgfpathlineto{\pgfqpoint{1.508424in}{3.290092in}}%
\pgfpathlineto{\pgfqpoint{1.512719in}{3.291549in}}%
\pgfpathlineto{\pgfqpoint{1.514867in}{3.289146in}}%
\pgfpathlineto{\pgfqpoint{1.519162in}{3.289700in}}%
\pgfpathlineto{\pgfqpoint{1.520236in}{3.287776in}}%
\pgfpathlineto{\pgfqpoint{1.522384in}{3.286906in}}%
\pgfpathlineto{\pgfqpoint{1.523457in}{3.288645in}}%
\pgfpathlineto{\pgfqpoint{1.527753in}{3.288332in}}%
\pgfpathlineto{\pgfqpoint{1.529900in}{3.292535in}}%
\pgfpathlineto{\pgfqpoint{1.534196in}{3.289285in}}%
\pgfpathlineto{\pgfqpoint{1.537417in}{3.290178in}}%
\pgfpathlineto{\pgfqpoint{1.538491in}{3.290949in}}%
\pgfpathlineto{\pgfqpoint{1.542786in}{3.289129in}}%
\pgfpathlineto{\pgfqpoint{1.543860in}{3.290169in}}%
\pgfpathlineto{\pgfqpoint{1.544934in}{3.292770in}}%
\pgfpathlineto{\pgfqpoint{1.546008in}{3.293850in}}%
\pgfpathlineto{\pgfqpoint{1.549229in}{3.294256in}}%
\pgfpathlineto{\pgfqpoint{1.550303in}{3.292829in}}%
\pgfpathlineto{\pgfqpoint{1.551377in}{3.295207in}}%
\pgfpathlineto{\pgfqpoint{1.552451in}{3.294935in}}%
\pgfpathlineto{\pgfqpoint{1.553525in}{3.293987in}}%
\pgfpathlineto{\pgfqpoint{1.556746in}{3.293320in}}%
\pgfpathlineto{\pgfqpoint{1.558894in}{3.295519in}}%
\pgfpathlineto{\pgfqpoint{1.559968in}{3.295184in}}%
\pgfpathlineto{\pgfqpoint{1.561042in}{3.296519in}}%
\pgfpathlineto{\pgfqpoint{1.566411in}{3.300809in}}%
\pgfpathlineto{\pgfqpoint{1.567485in}{3.300400in}}%
\pgfpathlineto{\pgfqpoint{1.568559in}{3.301959in}}%
\pgfpathlineto{\pgfqpoint{1.571780in}{3.302236in}}%
\pgfpathlineto{\pgfqpoint{1.573928in}{3.304047in}}%
\pgfpathlineto{\pgfqpoint{1.575002in}{3.304750in}}%
\pgfpathlineto{\pgfqpoint{1.581445in}{3.304395in}}%
\pgfpathlineto{\pgfqpoint{1.582518in}{3.304324in}}%
\pgfpathlineto{\pgfqpoint{1.583592in}{3.305174in}}%
\pgfpathlineto{\pgfqpoint{1.586814in}{3.305316in}}%
\pgfpathlineto{\pgfqpoint{1.587888in}{3.304251in}}%
\pgfpathlineto{\pgfqpoint{1.588961in}{3.304535in}}%
\pgfpathlineto{\pgfqpoint{1.590035in}{3.304109in}}%
\pgfpathlineto{\pgfqpoint{1.591109in}{3.304956in}}%
\pgfpathlineto{\pgfqpoint{1.594331in}{3.305242in}}%
\pgfpathlineto{\pgfqpoint{1.595405in}{3.303528in}}%
\pgfpathlineto{\pgfqpoint{1.596478in}{3.303249in}}%
\pgfpathlineto{\pgfqpoint{1.598626in}{3.301863in}}%
\pgfpathlineto{\pgfqpoint{1.601848in}{3.303024in}}%
\pgfpathlineto{\pgfqpoint{1.603995in}{3.302192in}}%
\pgfpathlineto{\pgfqpoint{1.605069in}{3.299972in}}%
\pgfpathlineto{\pgfqpoint{1.610438in}{3.302905in}}%
\pgfpathlineto{\pgfqpoint{1.611512in}{3.305586in}}%
\pgfpathlineto{\pgfqpoint{1.612586in}{3.305367in}}%
\pgfpathlineto{\pgfqpoint{1.613660in}{3.307048in}}%
\pgfpathlineto{\pgfqpoint{1.616881in}{3.308145in}}%
\pgfpathlineto{\pgfqpoint{1.617955in}{3.305623in}}%
\pgfpathlineto{\pgfqpoint{1.620103in}{3.311113in}}%
\pgfpathlineto{\pgfqpoint{1.621177in}{3.307881in}}%
\pgfpathlineto{\pgfqpoint{1.624398in}{3.305804in}}%
\pgfpathlineto{\pgfqpoint{1.625472in}{3.306804in}}%
\pgfpathlineto{\pgfqpoint{1.626546in}{3.305727in}}%
\pgfpathlineto{\pgfqpoint{1.627620in}{3.305879in}}%
\pgfpathlineto{\pgfqpoint{1.628694in}{3.309451in}}%
\pgfpathlineto{\pgfqpoint{1.631915in}{3.308159in}}%
\pgfpathlineto{\pgfqpoint{1.632989in}{3.305542in}}%
\pgfpathlineto{\pgfqpoint{1.635137in}{3.308535in}}%
\pgfpathlineto{\pgfqpoint{1.636210in}{3.307281in}}%
\pgfpathlineto{\pgfqpoint{1.639432in}{3.308007in}}%
\pgfpathlineto{\pgfqpoint{1.641580in}{3.311449in}}%
\pgfpathlineto{\pgfqpoint{1.642653in}{3.311523in}}%
\pgfpathlineto{\pgfqpoint{1.643727in}{3.312417in}}%
\pgfpathlineto{\pgfqpoint{1.649096in}{3.313838in}}%
\pgfpathlineto{\pgfqpoint{1.650170in}{3.310534in}}%
\pgfpathlineto{\pgfqpoint{1.654466in}{3.309886in}}%
\pgfpathlineto{\pgfqpoint{1.658761in}{3.311971in}}%
\pgfpathlineto{\pgfqpoint{1.661983in}{3.311029in}}%
\pgfpathlineto{\pgfqpoint{1.663056in}{3.308666in}}%
\pgfpathlineto{\pgfqpoint{1.664130in}{3.309222in}}%
\pgfpathlineto{\pgfqpoint{1.665204in}{3.308802in}}%
\pgfpathlineto{\pgfqpoint{1.666278in}{3.309641in}}%
\pgfpathlineto{\pgfqpoint{1.669499in}{3.309571in}}%
\pgfpathlineto{\pgfqpoint{1.670573in}{3.308732in}}%
\pgfpathlineto{\pgfqpoint{1.671647in}{3.308802in}}%
\pgfpathlineto{\pgfqpoint{1.677016in}{3.303907in}}%
\pgfpathlineto{\pgfqpoint{1.678090in}{3.304047in}}%
\pgfpathlineto{\pgfqpoint{1.679164in}{3.302089in}}%
\pgfpathlineto{\pgfqpoint{1.680238in}{3.303724in}}%
\pgfpathlineto{\pgfqpoint{1.685607in}{3.300799in}}%
\pgfpathlineto{\pgfqpoint{1.686681in}{3.298989in}}%
\pgfpathlineto{\pgfqpoint{1.687755in}{3.299824in}}%
\pgfpathlineto{\pgfqpoint{1.688828in}{3.302819in}}%
\pgfpathlineto{\pgfqpoint{1.693124in}{3.300424in}}%
\pgfpathlineto{\pgfqpoint{1.694198in}{3.301440in}}%
\pgfpathlineto{\pgfqpoint{1.696345in}{3.294420in}}%
\pgfpathlineto{\pgfqpoint{1.699567in}{3.293959in}}%
\pgfpathlineto{\pgfqpoint{1.700641in}{3.292977in}}%
\pgfpathlineto{\pgfqpoint{1.701715in}{3.293300in}}%
\pgfpathlineto{\pgfqpoint{1.707084in}{3.292651in}}%
\pgfpathlineto{\pgfqpoint{1.708158in}{3.291937in}}%
\pgfpathlineto{\pgfqpoint{1.709231in}{3.290316in}}%
\pgfpathlineto{\pgfqpoint{1.711379in}{3.289148in}}%
\pgfpathlineto{\pgfqpoint{1.714601in}{3.286683in}}%
\pgfpathlineto{\pgfqpoint{1.715674in}{3.283829in}}%
\pgfpathlineto{\pgfqpoint{1.716748in}{3.283894in}}%
\pgfpathlineto{\pgfqpoint{1.718896in}{3.287332in}}%
\pgfpathlineto{\pgfqpoint{1.722117in}{3.287065in}}%
\pgfpathlineto{\pgfqpoint{1.726413in}{3.284929in}}%
\pgfpathlineto{\pgfqpoint{1.730708in}{3.286331in}}%
\pgfpathlineto{\pgfqpoint{1.733930in}{3.282855in}}%
\pgfpathlineto{\pgfqpoint{1.737151in}{3.282285in}}%
\pgfpathlineto{\pgfqpoint{1.738225in}{3.283352in}}%
\pgfpathlineto{\pgfqpoint{1.739299in}{3.280481in}}%
\pgfpathlineto{\pgfqpoint{1.741447in}{3.282778in}}%
\pgfpathlineto{\pgfqpoint{1.744668in}{3.284463in}}%
\pgfpathlineto{\pgfqpoint{1.745742in}{3.283037in}}%
\pgfpathlineto{\pgfqpoint{1.747890in}{3.286573in}}%
\pgfpathlineto{\pgfqpoint{1.748963in}{3.286508in}}%
\pgfpathlineto{\pgfqpoint{1.752185in}{3.287158in}}%
\pgfpathlineto{\pgfqpoint{1.755406in}{3.285802in}}%
\pgfpathlineto{\pgfqpoint{1.756480in}{3.284401in}}%
\pgfpathlineto{\pgfqpoint{1.762923in}{3.284837in}}%
\pgfpathlineto{\pgfqpoint{1.763997in}{3.285896in}}%
\pgfpathlineto{\pgfqpoint{1.767219in}{3.286084in}}%
\pgfpathlineto{\pgfqpoint{1.768293in}{3.287209in}}%
\pgfpathlineto{\pgfqpoint{1.769366in}{3.284458in}}%
\pgfpathlineto{\pgfqpoint{1.774736in}{3.284518in}}%
\pgfpathlineto{\pgfqpoint{1.775809in}{3.285769in}}%
\pgfpathlineto{\pgfqpoint{1.776883in}{3.284736in}}%
\pgfpathlineto{\pgfqpoint{1.777957in}{3.285526in}}%
\pgfpathlineto{\pgfqpoint{1.779031in}{3.287529in}}%
\pgfpathlineto{\pgfqpoint{1.782252in}{3.288656in}}%
\pgfpathlineto{\pgfqpoint{1.783326in}{3.291097in}}%
\pgfpathlineto{\pgfqpoint{1.784400in}{3.291162in}}%
\pgfpathlineto{\pgfqpoint{1.789769in}{3.296546in}}%
\pgfpathlineto{\pgfqpoint{1.790843in}{3.297347in}}%
\pgfpathlineto{\pgfqpoint{1.792991in}{3.300992in}}%
\pgfpathlineto{\pgfqpoint{1.794065in}{3.301407in}}%
\pgfpathlineto{\pgfqpoint{1.797286in}{3.300992in}}%
\pgfpathlineto{\pgfqpoint{1.798360in}{3.302161in}}%
\pgfpathlineto{\pgfqpoint{1.799434in}{3.300065in}}%
\pgfpathlineto{\pgfqpoint{1.800508in}{3.299436in}}%
\pgfpathlineto{\pgfqpoint{1.801582in}{3.299786in}}%
\pgfpathlineto{\pgfqpoint{1.804803in}{3.298319in}}%
\pgfpathlineto{\pgfqpoint{1.805877in}{3.300099in}}%
\pgfpathlineto{\pgfqpoint{1.806951in}{3.300309in}}%
\pgfpathlineto{\pgfqpoint{1.808025in}{3.299748in}}%
\pgfpathlineto{\pgfqpoint{1.812320in}{3.298285in}}%
\pgfpathlineto{\pgfqpoint{1.813394in}{3.299241in}}%
\pgfpathlineto{\pgfqpoint{1.814468in}{3.299172in}}%
\pgfpathlineto{\pgfqpoint{1.815541in}{3.303391in}}%
\pgfpathlineto{\pgfqpoint{1.816615in}{3.287275in}}%
\pgfpathlineto{\pgfqpoint{1.821984in}{3.293294in}}%
\pgfpathlineto{\pgfqpoint{1.823058in}{3.292388in}}%
\pgfpathlineto{\pgfqpoint{1.824132in}{3.292267in}}%
\pgfpathlineto{\pgfqpoint{1.827354in}{3.291058in}}%
\pgfpathlineto{\pgfqpoint{1.828427in}{3.289063in}}%
\pgfpathlineto{\pgfqpoint{1.829501in}{3.290514in}}%
\pgfpathlineto{\pgfqpoint{1.831649in}{3.290573in}}%
\pgfpathlineto{\pgfqpoint{1.834871in}{3.290937in}}%
\pgfpathlineto{\pgfqpoint{1.835944in}{3.289662in}}%
\pgfpathlineto{\pgfqpoint{1.838092in}{3.289781in}}%
\pgfpathlineto{\pgfqpoint{1.839166in}{3.290976in}}%
\pgfpathlineto{\pgfqpoint{1.842387in}{3.291215in}}%
\pgfpathlineto{\pgfqpoint{1.843461in}{3.289476in}}%
\pgfpathlineto{\pgfqpoint{1.844535in}{3.288937in}}%
\pgfpathlineto{\pgfqpoint{1.845609in}{3.290076in}}%
\pgfpathlineto{\pgfqpoint{1.846683in}{3.288457in}}%
\pgfpathlineto{\pgfqpoint{1.850978in}{3.291168in}}%
\pgfpathlineto{\pgfqpoint{1.853126in}{3.289246in}}%
\pgfpathlineto{\pgfqpoint{1.854200in}{3.289952in}}%
\pgfpathlineto{\pgfqpoint{1.858495in}{3.290011in}}%
\pgfpathlineto{\pgfqpoint{1.859569in}{3.288589in}}%
\pgfpathlineto{\pgfqpoint{1.861716in}{3.288356in}}%
\pgfpathlineto{\pgfqpoint{1.867086in}{3.287546in}}%
\pgfpathlineto{\pgfqpoint{1.868160in}{3.288067in}}%
\pgfpathlineto{\pgfqpoint{1.869233in}{3.289802in}}%
\pgfpathlineto{\pgfqpoint{1.873529in}{3.290098in}}%
\pgfpathlineto{\pgfqpoint{1.874603in}{3.289090in}}%
\pgfpathlineto{\pgfqpoint{1.875676in}{3.290435in}}%
\pgfpathlineto{\pgfqpoint{1.881046in}{3.291271in}}%
\pgfpathlineto{\pgfqpoint{1.882119in}{3.289775in}}%
\pgfpathlineto{\pgfqpoint{1.883193in}{3.290360in}}%
\pgfpathlineto{\pgfqpoint{1.884267in}{3.289770in}}%
\pgfpathlineto{\pgfqpoint{1.888562in}{3.288942in}}%
\pgfpathlineto{\pgfqpoint{1.889636in}{3.289179in}}%
\pgfpathlineto{\pgfqpoint{1.890710in}{3.287525in}}%
\pgfpathlineto{\pgfqpoint{1.891784in}{3.288159in}}%
\pgfpathlineto{\pgfqpoint{1.895005in}{3.289032in}}%
\pgfpathlineto{\pgfqpoint{1.896079in}{3.288392in}}%
\pgfpathlineto{\pgfqpoint{1.897153in}{3.289775in}}%
\pgfpathlineto{\pgfqpoint{1.898227in}{3.288657in}}%
\pgfpathlineto{\pgfqpoint{1.899301in}{3.288893in}}%
\pgfpathlineto{\pgfqpoint{1.902522in}{3.291129in}}%
\pgfpathlineto{\pgfqpoint{1.903596in}{3.290764in}}%
\pgfpathlineto{\pgfqpoint{1.904670in}{3.291250in}}%
\pgfpathlineto{\pgfqpoint{1.905744in}{3.290642in}}%
\pgfpathlineto{\pgfqpoint{1.911113in}{3.293928in}}%
\pgfpathlineto{\pgfqpoint{1.912187in}{3.296989in}}%
\pgfpathlineto{\pgfqpoint{1.913261in}{3.297706in}}%
\pgfpathlineto{\pgfqpoint{1.914335in}{3.296336in}}%
\pgfpathlineto{\pgfqpoint{1.917556in}{3.295248in}}%
\pgfpathlineto{\pgfqpoint{1.921851in}{3.291235in}}%
\pgfpathlineto{\pgfqpoint{1.925073in}{3.291711in}}%
\pgfpathlineto{\pgfqpoint{1.928294in}{3.287215in}}%
\pgfpathlineto{\pgfqpoint{1.929368in}{3.284637in}}%
\pgfpathlineto{\pgfqpoint{1.932590in}{3.285656in}}%
\pgfpathlineto{\pgfqpoint{1.934738in}{3.284107in}}%
\pgfpathlineto{\pgfqpoint{1.935811in}{3.284809in}}%
\pgfpathlineto{\pgfqpoint{1.936885in}{3.284750in}}%
\pgfpathlineto{\pgfqpoint{1.942254in}{3.282738in}}%
\pgfpathlineto{\pgfqpoint{1.943328in}{3.283088in}}%
\pgfpathlineto{\pgfqpoint{1.944402in}{3.281857in}}%
\pgfpathlineto{\pgfqpoint{1.947624in}{3.284084in}}%
\pgfpathlineto{\pgfqpoint{1.948697in}{3.286720in}}%
\pgfpathlineto{\pgfqpoint{1.951919in}{3.289413in}}%
\pgfpathlineto{\pgfqpoint{1.955140in}{3.288738in}}%
\pgfpathlineto{\pgfqpoint{1.956214in}{3.289467in}}%
\pgfpathlineto{\pgfqpoint{1.958362in}{3.284862in}}%
\pgfpathlineto{\pgfqpoint{1.959436in}{3.281792in}}%
\pgfpathlineto{\pgfqpoint{1.962657in}{3.277986in}}%
\pgfpathlineto{\pgfqpoint{1.963731in}{3.274916in}}%
\pgfpathlineto{\pgfqpoint{1.964805in}{3.278723in}}%
\pgfpathlineto{\pgfqpoint{1.965879in}{3.273504in}}%
\pgfpathlineto{\pgfqpoint{1.966953in}{3.272769in}}%
\pgfpathlineto{\pgfqpoint{1.970174in}{3.274387in}}%
\pgfpathlineto{\pgfqpoint{1.971248in}{3.269801in}}%
\pgfpathlineto{\pgfqpoint{1.972322in}{3.273183in}}%
\pgfpathlineto{\pgfqpoint{1.973396in}{3.273470in}}%
\pgfpathlineto{\pgfqpoint{1.974470in}{3.270993in}}%
\pgfpathlineto{\pgfqpoint{1.978765in}{3.275659in}}%
\pgfpathlineto{\pgfqpoint{1.979839in}{3.277675in}}%
\pgfpathlineto{\pgfqpoint{1.980913in}{3.278330in}}%
\pgfpathlineto{\pgfqpoint{1.981986in}{3.276960in}}%
\pgfpathlineto{\pgfqpoint{1.985208in}{3.277834in}}%
\pgfpathlineto{\pgfqpoint{1.986282in}{3.280492in}}%
\pgfpathlineto{\pgfqpoint{1.987356in}{3.277361in}}%
\pgfpathlineto{\pgfqpoint{1.988429in}{3.279048in}}%
\pgfpathlineto{\pgfqpoint{1.989503in}{3.276333in}}%
\pgfpathlineto{\pgfqpoint{1.993799in}{3.277662in}}%
\pgfpathlineto{\pgfqpoint{1.994872in}{3.277546in}}%
\pgfpathlineto{\pgfqpoint{1.995946in}{3.278640in}}%
\pgfpathlineto{\pgfqpoint{1.997020in}{3.278640in}}%
\pgfpathlineto{\pgfqpoint{2.000242in}{3.275593in}}%
\pgfpathlineto{\pgfqpoint{2.001315in}{3.276882in}}%
\pgfpathlineto{\pgfqpoint{2.002389in}{3.273601in}}%
\pgfpathlineto{\pgfqpoint{2.003463in}{3.273768in}}%
\pgfpathlineto{\pgfqpoint{2.004537in}{3.275104in}}%
\pgfpathlineto{\pgfqpoint{2.009906in}{3.264657in}}%
\pgfpathlineto{\pgfqpoint{2.010980in}{3.263630in}}%
\pgfpathlineto{\pgfqpoint{2.015275in}{3.263355in}}%
\pgfpathlineto{\pgfqpoint{2.016349in}{3.264222in}}%
\pgfpathlineto{\pgfqpoint{2.017423in}{3.263156in}}%
\pgfpathlineto{\pgfqpoint{2.018497in}{3.264824in}}%
\pgfpathlineto{\pgfqpoint{2.019571in}{3.261025in}}%
\pgfpathlineto{\pgfqpoint{2.022792in}{3.260982in}}%
\pgfpathlineto{\pgfqpoint{2.023866in}{3.260204in}}%
\pgfpathlineto{\pgfqpoint{2.024940in}{3.260463in}}%
\pgfpathlineto{\pgfqpoint{2.026014in}{3.257783in}}%
\pgfpathlineto{\pgfqpoint{2.027088in}{3.258030in}}%
\pgfpathlineto{\pgfqpoint{2.033531in}{3.257986in}}%
\pgfpathlineto{\pgfqpoint{2.034604in}{3.256532in}}%
\pgfpathlineto{\pgfqpoint{2.037826in}{3.258235in}}%
\pgfpathlineto{\pgfqpoint{2.038900in}{3.257570in}}%
\pgfpathlineto{\pgfqpoint{2.041048in}{3.258064in}}%
\pgfpathlineto{\pgfqpoint{2.042121in}{3.257076in}}%
\pgfpathlineto{\pgfqpoint{2.045343in}{3.257641in}}%
\pgfpathlineto{\pgfqpoint{2.046417in}{3.258907in}}%
\pgfpathlineto{\pgfqpoint{2.047491in}{3.256988in}}%
\pgfpathlineto{\pgfqpoint{2.048564in}{3.258680in}}%
\pgfpathlineto{\pgfqpoint{2.049638in}{3.259087in}}%
\pgfpathlineto{\pgfqpoint{2.053934in}{3.258964in}}%
\pgfpathlineto{\pgfqpoint{2.056081in}{3.260506in}}%
\pgfpathlineto{\pgfqpoint{2.057155in}{3.261866in}}%
\pgfpathlineto{\pgfqpoint{2.061450in}{3.262362in}}%
\pgfpathlineto{\pgfqpoint{2.062524in}{3.263397in}}%
\pgfpathlineto{\pgfqpoint{2.064672in}{3.263397in}}%
\pgfpathlineto{\pgfqpoint{2.067893in}{3.261878in}}%
\pgfpathlineto{\pgfqpoint{2.071115in}{3.263611in}}%
\pgfpathlineto{\pgfqpoint{2.072189in}{3.261900in}}%
\pgfpathlineto{\pgfqpoint{2.075410in}{3.262315in}}%
\pgfpathlineto{\pgfqpoint{2.076484in}{3.261646in}}%
\pgfpathlineto{\pgfqpoint{2.077558in}{3.262649in}}%
\pgfpathlineto{\pgfqpoint{2.078632in}{3.262022in}}%
\pgfpathlineto{\pgfqpoint{2.079706in}{3.263385in}}%
\pgfpathlineto{\pgfqpoint{2.084001in}{3.263597in}}%
\pgfpathlineto{\pgfqpoint{2.085075in}{3.261183in}}%
\pgfpathlineto{\pgfqpoint{2.086149in}{3.261912in}}%
\pgfpathlineto{\pgfqpoint{2.087223in}{3.260968in}}%
\pgfpathlineto{\pgfqpoint{2.091518in}{3.261132in}}%
\pgfpathlineto{\pgfqpoint{2.092592in}{3.259497in}}%
\pgfpathlineto{\pgfqpoint{2.093666in}{3.259894in}}%
\pgfpathlineto{\pgfqpoint{2.097961in}{3.260133in}}%
\pgfpathlineto{\pgfqpoint{2.099035in}{3.258855in}}%
\pgfpathlineto{\pgfqpoint{2.101182in}{3.259912in}}%
\pgfpathlineto{\pgfqpoint{2.106552in}{3.261537in}}%
\pgfpathlineto{\pgfqpoint{2.107626in}{3.263246in}}%
\pgfpathlineto{\pgfqpoint{2.108699in}{3.258585in}}%
\pgfpathlineto{\pgfqpoint{2.109773in}{3.256737in}}%
\pgfpathlineto{\pgfqpoint{2.114069in}{3.256989in}}%
\pgfpathlineto{\pgfqpoint{2.115142in}{3.258453in}}%
\pgfpathlineto{\pgfqpoint{2.117290in}{3.263616in}}%
\pgfpathlineto{\pgfqpoint{2.121585in}{3.263616in}}%
\pgfpathlineto{\pgfqpoint{2.122659in}{3.261736in}}%
\pgfpathlineto{\pgfqpoint{2.124807in}{3.265405in}}%
\pgfpathlineto{\pgfqpoint{2.128028in}{3.264625in}}%
\pgfpathlineto{\pgfqpoint{2.131250in}{3.267717in}}%
\pgfpathlineto{\pgfqpoint{2.132324in}{3.264108in}}%
\pgfpathlineto{\pgfqpoint{2.135545in}{3.265868in}}%
\pgfpathlineto{\pgfqpoint{2.136619in}{3.264278in}}%
\pgfpathlineto{\pgfqpoint{2.137693in}{3.265959in}}%
\pgfpathlineto{\pgfqpoint{2.138767in}{3.263960in}}%
\pgfpathlineto{\pgfqpoint{2.139841in}{3.266371in}}%
\pgfpathlineto{\pgfqpoint{2.143062in}{3.264906in}}%
\pgfpathlineto{\pgfqpoint{2.145210in}{3.265272in}}%
\pgfpathlineto{\pgfqpoint{2.147358in}{3.272055in}}%
\pgfpathlineto{\pgfqpoint{2.151653in}{3.269529in}}%
\pgfpathlineto{\pgfqpoint{2.152727in}{3.267621in}}%
\pgfpathlineto{\pgfqpoint{2.153801in}{3.268611in}}%
\pgfpathlineto{\pgfqpoint{2.154874in}{3.268382in}}%
\pgfpathlineto{\pgfqpoint{2.159170in}{3.270628in}}%
\pgfpathlineto{\pgfqpoint{2.160244in}{3.269560in}}%
\pgfpathlineto{\pgfqpoint{2.161317in}{3.271602in}}%
\pgfpathlineto{\pgfqpoint{2.162391in}{3.270906in}}%
\pgfpathlineto{\pgfqpoint{2.165613in}{3.271961in}}%
\pgfpathlineto{\pgfqpoint{2.166687in}{3.274949in}}%
\pgfpathlineto{\pgfqpoint{2.167760in}{3.273735in}}%
\pgfpathlineto{\pgfqpoint{2.168834in}{3.273552in}}%
\pgfpathlineto{\pgfqpoint{2.169908in}{3.272594in}}%
\pgfpathlineto{\pgfqpoint{2.173130in}{3.273268in}}%
\pgfpathlineto{\pgfqpoint{2.174203in}{3.272359in}}%
\pgfpathlineto{\pgfqpoint{2.176351in}{3.271905in}}%
\pgfpathlineto{\pgfqpoint{2.177425in}{3.273449in}}%
\pgfpathlineto{\pgfqpoint{2.182794in}{3.273357in}}%
\pgfpathlineto{\pgfqpoint{2.183868in}{3.276459in}}%
\pgfpathlineto{\pgfqpoint{2.191385in}{3.277511in}}%
\pgfpathlineto{\pgfqpoint{2.196754in}{3.276044in}}%
\pgfpathlineto{\pgfqpoint{2.197828in}{3.277431in}}%
\pgfpathlineto{\pgfqpoint{2.199976in}{3.278148in}}%
\pgfpathlineto{\pgfqpoint{2.203197in}{3.280839in}}%
\pgfpathlineto{\pgfqpoint{2.206419in}{3.278404in}}%
\pgfpathlineto{\pgfqpoint{2.207492in}{3.279012in}}%
\pgfpathlineto{\pgfqpoint{2.212862in}{3.279058in}}%
\pgfpathlineto{\pgfqpoint{2.215009in}{3.278872in}}%
\pgfpathlineto{\pgfqpoint{2.220379in}{3.278640in}}%
\pgfpathlineto{\pgfqpoint{2.222526in}{3.277071in}}%
\pgfpathlineto{\pgfqpoint{2.225748in}{3.276748in}}%
\pgfpathlineto{\pgfqpoint{2.226822in}{3.277624in}}%
\pgfpathlineto{\pgfqpoint{2.230043in}{3.277024in}}%
\pgfpathlineto{\pgfqpoint{2.233265in}{3.277578in}}%
\pgfpathlineto{\pgfqpoint{2.234338in}{3.278593in}}%
\pgfpathlineto{\pgfqpoint{2.235412in}{3.276295in}}%
\pgfpathlineto{\pgfqpoint{2.236486in}{3.275544in}}%
\pgfpathlineto{\pgfqpoint{2.237560in}{3.276482in}}%
\pgfpathlineto{\pgfqpoint{2.240781in}{3.277514in}}%
\pgfpathlineto{\pgfqpoint{2.241855in}{3.280042in}}%
\pgfpathlineto{\pgfqpoint{2.242929in}{3.280615in}}%
\pgfpathlineto{\pgfqpoint{2.244003in}{3.279604in}}%
\pgfpathlineto{\pgfqpoint{2.245077in}{3.277727in}}%
\pgfpathlineto{\pgfqpoint{2.250446in}{3.279597in}}%
\pgfpathlineto{\pgfqpoint{2.251520in}{3.278521in}}%
\pgfpathlineto{\pgfqpoint{2.252594in}{3.279401in}}%
\pgfpathlineto{\pgfqpoint{2.255815in}{3.279744in}}%
\pgfpathlineto{\pgfqpoint{2.256889in}{3.281318in}}%
\pgfpathlineto{\pgfqpoint{2.257963in}{3.280285in}}%
\pgfpathlineto{\pgfqpoint{2.260111in}{3.280964in}}%
\pgfpathlineto{\pgfqpoint{2.264406in}{3.280478in}}%
\pgfpathlineto{\pgfqpoint{2.265480in}{3.281009in}}%
\pgfpathlineto{\pgfqpoint{2.267627in}{3.280327in}}%
\pgfpathlineto{\pgfqpoint{2.271923in}{3.280960in}}%
\pgfpathlineto{\pgfqpoint{2.272997in}{3.280231in}}%
\pgfpathlineto{\pgfqpoint{2.274070in}{3.280519in}}%
\pgfpathlineto{\pgfqpoint{2.275144in}{3.279651in}}%
\pgfpathlineto{\pgfqpoint{2.278366in}{3.278783in}}%
\pgfpathlineto{\pgfqpoint{2.279440in}{3.281340in}}%
\pgfpathlineto{\pgfqpoint{2.282661in}{3.279706in}}%
\pgfpathlineto{\pgfqpoint{2.285883in}{3.280646in}}%
\pgfpathlineto{\pgfqpoint{2.286957in}{3.280176in}}%
\pgfpathlineto{\pgfqpoint{2.288030in}{3.280829in}}%
\pgfpathlineto{\pgfqpoint{2.289104in}{3.282527in}}%
\pgfpathlineto{\pgfqpoint{2.290178in}{3.288186in}}%
\pgfpathlineto{\pgfqpoint{2.293400in}{3.285921in}}%
\pgfpathlineto{\pgfqpoint{2.294473in}{3.288701in}}%
\pgfpathlineto{\pgfqpoint{2.297695in}{3.281871in}}%
\pgfpathlineto{\pgfqpoint{2.301990in}{3.281685in}}%
\pgfpathlineto{\pgfqpoint{2.303064in}{3.282894in}}%
\pgfpathlineto{\pgfqpoint{2.304138in}{3.282568in}}%
\pgfpathlineto{\pgfqpoint{2.305212in}{3.281042in}}%
\pgfpathlineto{\pgfqpoint{2.310581in}{3.280771in}}%
\pgfpathlineto{\pgfqpoint{2.312729in}{3.278725in}}%
\pgfpathlineto{\pgfqpoint{2.318098in}{3.278162in}}%
\pgfpathlineto{\pgfqpoint{2.319172in}{3.277425in}}%
\pgfpathlineto{\pgfqpoint{2.320246in}{3.275430in}}%
\pgfpathlineto{\pgfqpoint{2.327762in}{3.271918in}}%
\pgfpathlineto{\pgfqpoint{2.334205in}{3.271702in}}%
\pgfpathlineto{\pgfqpoint{2.335279in}{3.271269in}}%
\pgfpathlineto{\pgfqpoint{2.353535in}{3.271908in}}%
\pgfpathlineto{\pgfqpoint{2.355682in}{3.272210in}}%
\pgfpathlineto{\pgfqpoint{2.362125in}{3.271733in}}%
\pgfpathlineto{\pgfqpoint{2.364273in}{3.271131in}}%
\pgfpathlineto{\pgfqpoint{2.370716in}{3.272378in}}%
\pgfpathlineto{\pgfqpoint{2.371790in}{3.272334in}}%
\pgfpathlineto{\pgfqpoint{2.372864in}{3.268756in}}%
\pgfpathlineto{\pgfqpoint{2.376085in}{3.270196in}}%
\pgfpathlineto{\pgfqpoint{2.377159in}{3.272640in}}%
\pgfpathlineto{\pgfqpoint{2.378233in}{3.272048in}}%
\pgfpathlineto{\pgfqpoint{2.379307in}{3.273186in}}%
\pgfpathlineto{\pgfqpoint{2.380380in}{3.273459in}}%
\pgfpathlineto{\pgfqpoint{2.383602in}{3.272453in}}%
\pgfpathlineto{\pgfqpoint{2.384676in}{3.273413in}}%
\pgfpathlineto{\pgfqpoint{2.386824in}{3.271914in}}%
\pgfpathlineto{\pgfqpoint{2.387897in}{3.272538in}}%
\pgfpathlineto{\pgfqpoint{2.391119in}{3.271142in}}%
\pgfpathlineto{\pgfqpoint{2.392193in}{3.272493in}}%
\pgfpathlineto{\pgfqpoint{2.393267in}{3.272403in}}%
\pgfpathlineto{\pgfqpoint{2.394340in}{3.273843in}}%
\pgfpathlineto{\pgfqpoint{2.395414in}{3.274165in}}%
\pgfpathlineto{\pgfqpoint{2.398636in}{3.274073in}}%
\pgfpathlineto{\pgfqpoint{2.400783in}{3.274625in}}%
\pgfpathlineto{\pgfqpoint{2.402931in}{3.272907in}}%
\pgfpathlineto{\pgfqpoint{2.406153in}{3.271978in}}%
\pgfpathlineto{\pgfqpoint{2.408300in}{3.272349in}}%
\pgfpathlineto{\pgfqpoint{2.409374in}{3.271838in}}%
\pgfpathlineto{\pgfqpoint{2.410448in}{3.272303in}}%
\pgfpathlineto{\pgfqpoint{2.417965in}{3.272212in}}%
\pgfpathlineto{\pgfqpoint{2.421186in}{3.271980in}}%
\pgfpathlineto{\pgfqpoint{2.422260in}{3.270868in}}%
\pgfpathlineto{\pgfqpoint{2.424408in}{3.272722in}}%
\pgfpathlineto{\pgfqpoint{2.425482in}{3.275171in}}%
\pgfpathlineto{\pgfqpoint{2.428703in}{3.275642in}}%
\pgfpathlineto{\pgfqpoint{2.431925in}{3.272177in}}%
\pgfpathlineto{\pgfqpoint{2.432999in}{3.272841in}}%
\pgfpathlineto{\pgfqpoint{2.437294in}{3.268739in}}%
\pgfpathlineto{\pgfqpoint{2.438368in}{3.267942in}}%
\pgfpathlineto{\pgfqpoint{2.439442in}{3.264927in}}%
\pgfpathlineto{\pgfqpoint{2.440515in}{3.263808in}}%
\pgfpathlineto{\pgfqpoint{2.443737in}{3.264539in}}%
\pgfpathlineto{\pgfqpoint{2.444811in}{3.265404in}}%
\pgfpathlineto{\pgfqpoint{2.446958in}{3.265651in}}%
\pgfpathlineto{\pgfqpoint{2.448032in}{3.266104in}}%
\pgfpathlineto{\pgfqpoint{2.451254in}{3.266851in}}%
\pgfpathlineto{\pgfqpoint{2.453402in}{3.265160in}}%
\pgfpathlineto{\pgfqpoint{2.455549in}{3.264798in}}%
\pgfpathlineto{\pgfqpoint{2.458771in}{3.265478in}}%
\pgfpathlineto{\pgfqpoint{2.460918in}{3.263735in}}%
\pgfpathlineto{\pgfqpoint{2.461992in}{3.266005in}}%
\pgfpathlineto{\pgfqpoint{2.463066in}{3.266208in}}%
\pgfpathlineto{\pgfqpoint{2.467361in}{3.265598in}}%
\pgfpathlineto{\pgfqpoint{2.468435in}{3.264052in}}%
\pgfpathlineto{\pgfqpoint{2.469509in}{3.264289in}}%
\pgfpathlineto{\pgfqpoint{2.470583in}{3.265164in}}%
\pgfpathlineto{\pgfqpoint{2.475952in}{3.264483in}}%
\pgfpathlineto{\pgfqpoint{2.477026in}{3.263647in}}%
\pgfpathlineto{\pgfqpoint{2.478100in}{3.265360in}}%
\pgfpathlineto{\pgfqpoint{2.481321in}{3.264762in}}%
\pgfpathlineto{\pgfqpoint{2.482395in}{3.263619in}}%
\pgfpathlineto{\pgfqpoint{2.483469in}{3.264005in}}%
\pgfpathlineto{\pgfqpoint{2.485617in}{3.263966in}}%
\pgfpathlineto{\pgfqpoint{2.489912in}{3.263889in}}%
\pgfpathlineto{\pgfqpoint{2.492060in}{3.264627in}}%
\pgfpathlineto{\pgfqpoint{2.493134in}{3.264234in}}%
\pgfpathlineto{\pgfqpoint{2.498503in}{3.264509in}}%
\pgfpathlineto{\pgfqpoint{2.499577in}{3.263880in}}%
\pgfpathlineto{\pgfqpoint{2.503872in}{3.264706in}}%
\pgfpathlineto{\pgfqpoint{2.504946in}{3.264388in}}%
\pgfpathlineto{\pgfqpoint{2.508167in}{3.264983in}}%
\pgfpathlineto{\pgfqpoint{2.514610in}{3.264424in}}%
\pgfpathlineto{\pgfqpoint{2.515684in}{3.262030in}}%
\pgfpathlineto{\pgfqpoint{2.518906in}{3.259277in}}%
\pgfpathlineto{\pgfqpoint{2.519979in}{3.260155in}}%
\pgfpathlineto{\pgfqpoint{2.521053in}{3.258838in}}%
\pgfpathlineto{\pgfqpoint{2.522127in}{3.259033in}}%
\pgfpathlineto{\pgfqpoint{2.523201in}{3.257978in}}%
\pgfpathlineto{\pgfqpoint{2.526423in}{3.257782in}}%
\pgfpathlineto{\pgfqpoint{2.527496in}{3.256884in}}%
\pgfpathlineto{\pgfqpoint{2.538235in}{3.256257in}}%
\pgfpathlineto{\pgfqpoint{2.545752in}{3.254034in}}%
\pgfpathlineto{\pgfqpoint{2.552195in}{3.254923in}}%
\pgfpathlineto{\pgfqpoint{2.553268in}{3.255487in}}%
\pgfpathlineto{\pgfqpoint{2.556490in}{3.256315in}}%
\pgfpathlineto{\pgfqpoint{2.557564in}{3.255856in}}%
\pgfpathlineto{\pgfqpoint{2.558638in}{3.257156in}}%
\pgfpathlineto{\pgfqpoint{2.560785in}{3.256926in}}%
\pgfpathlineto{\pgfqpoint{2.566155in}{3.255818in}}%
\pgfpathlineto{\pgfqpoint{2.567228in}{3.255359in}}%
\pgfpathlineto{\pgfqpoint{2.568302in}{3.257462in}}%
\pgfpathlineto{\pgfqpoint{2.571524in}{3.258876in}}%
\pgfpathlineto{\pgfqpoint{2.572598in}{3.257737in}}%
\pgfpathlineto{\pgfqpoint{2.573671in}{3.258523in}}%
\pgfpathlineto{\pgfqpoint{2.575819in}{3.258994in}}%
\pgfpathlineto{\pgfqpoint{2.579041in}{3.259466in}}%
\pgfpathlineto{\pgfqpoint{2.580114in}{3.258236in}}%
\pgfpathlineto{\pgfqpoint{2.581188in}{3.258712in}}%
\pgfpathlineto{\pgfqpoint{2.583336in}{3.258040in}}%
\pgfpathlineto{\pgfqpoint{2.588705in}{3.259418in}}%
\pgfpathlineto{\pgfqpoint{2.589779in}{3.258743in}}%
\pgfpathlineto{\pgfqpoint{2.594074in}{3.259294in}}%
\pgfpathlineto{\pgfqpoint{2.595148in}{3.258781in}}%
\pgfpathlineto{\pgfqpoint{2.598370in}{3.259055in}}%
\pgfpathlineto{\pgfqpoint{2.602665in}{3.258859in}}%
\pgfpathlineto{\pgfqpoint{2.603739in}{3.259796in}}%
\pgfpathlineto{\pgfqpoint{2.604813in}{3.259041in}}%
\pgfpathlineto{\pgfqpoint{2.609108in}{3.259319in}}%
\pgfpathlineto{\pgfqpoint{2.612330in}{3.257123in}}%
\pgfpathlineto{\pgfqpoint{2.613403in}{3.259563in}}%
\pgfpathlineto{\pgfqpoint{2.617699in}{3.259203in}}%
\pgfpathlineto{\pgfqpoint{2.620920in}{3.257565in}}%
\pgfpathlineto{\pgfqpoint{2.628437in}{3.256843in}}%
\pgfpathlineto{\pgfqpoint{2.632733in}{3.255819in}}%
\pgfpathlineto{\pgfqpoint{2.633806in}{3.255032in}}%
\pgfpathlineto{\pgfqpoint{2.634880in}{3.255622in}}%
\pgfpathlineto{\pgfqpoint{2.635954in}{3.257749in}}%
\pgfpathlineto{\pgfqpoint{2.640249in}{3.256928in}}%
\pgfpathlineto{\pgfqpoint{2.641323in}{3.254587in}}%
\pgfpathlineto{\pgfqpoint{2.642397in}{3.254834in}}%
\pgfpathlineto{\pgfqpoint{2.643471in}{3.252698in}}%
\pgfpathlineto{\pgfqpoint{2.647766in}{3.251914in}}%
\pgfpathlineto{\pgfqpoint{2.648840in}{3.253464in}}%
\pgfpathlineto{\pgfqpoint{2.649914in}{3.252224in}}%
\pgfpathlineto{\pgfqpoint{2.656357in}{3.251864in}}%
\pgfpathlineto{\pgfqpoint{2.658505in}{3.250119in}}%
\pgfpathlineto{\pgfqpoint{2.663874in}{3.249041in}}%
\pgfpathlineto{\pgfqpoint{2.664948in}{3.248733in}}%
\pgfpathlineto{\pgfqpoint{2.666022in}{3.249963in}}%
\pgfpathlineto{\pgfqpoint{2.669243in}{3.246543in}}%
\pgfpathlineto{\pgfqpoint{2.671391in}{3.248932in}}%
\pgfpathlineto{\pgfqpoint{2.673538in}{3.246301in}}%
\pgfpathlineto{\pgfqpoint{2.676760in}{3.246932in}}%
\pgfpathlineto{\pgfqpoint{2.678908in}{3.243630in}}%
\pgfpathlineto{\pgfqpoint{2.679981in}{3.242884in}}%
\pgfpathlineto{\pgfqpoint{2.684277in}{3.242813in}}%
\pgfpathlineto{\pgfqpoint{2.685351in}{3.244123in}}%
\pgfpathlineto{\pgfqpoint{2.687498in}{3.243503in}}%
\pgfpathlineto{\pgfqpoint{2.688572in}{3.243467in}}%
\pgfpathlineto{\pgfqpoint{2.691794in}{3.244925in}}%
\pgfpathlineto{\pgfqpoint{2.693941in}{3.245253in}}%
\pgfpathlineto{\pgfqpoint{2.695015in}{3.241765in}}%
\pgfpathlineto{\pgfqpoint{2.696089in}{3.241251in}}%
\pgfpathlineto{\pgfqpoint{2.699311in}{3.240884in}}%
\pgfpathlineto{\pgfqpoint{2.700384in}{3.242022in}}%
\pgfpathlineto{\pgfqpoint{2.702532in}{3.240674in}}%
\pgfpathlineto{\pgfqpoint{2.710049in}{3.241274in}}%
\pgfpathlineto{\pgfqpoint{2.711123in}{3.238719in}}%
\pgfpathlineto{\pgfqpoint{2.715418in}{3.237172in}}%
\pgfpathlineto{\pgfqpoint{2.716492in}{3.237676in}}%
\pgfpathlineto{\pgfqpoint{2.717566in}{3.237028in}}%
\pgfpathlineto{\pgfqpoint{2.718640in}{3.237879in}}%
\pgfpathlineto{\pgfqpoint{2.722935in}{3.238674in}}%
\pgfpathlineto{\pgfqpoint{2.724009in}{3.238930in}}%
\pgfpathlineto{\pgfqpoint{2.725083in}{3.238125in}}%
\pgfpathlineto{\pgfqpoint{2.726156in}{3.238053in}}%
\pgfpathlineto{\pgfqpoint{2.731526in}{3.238845in}}%
\pgfpathlineto{\pgfqpoint{2.732600in}{3.240225in}}%
\pgfpathlineto{\pgfqpoint{2.733673in}{3.239888in}}%
\pgfpathlineto{\pgfqpoint{2.739043in}{3.241048in}}%
\pgfpathlineto{\pgfqpoint{2.741190in}{3.239140in}}%
\pgfpathlineto{\pgfqpoint{2.745486in}{3.239292in}}%
\pgfpathlineto{\pgfqpoint{2.746559in}{3.240017in}}%
\pgfpathlineto{\pgfqpoint{2.747633in}{3.239707in}}%
\pgfpathlineto{\pgfqpoint{2.748707in}{3.240367in}}%
\pgfpathlineto{\pgfqpoint{2.753002in}{3.240289in}}%
\pgfpathlineto{\pgfqpoint{2.754076in}{3.239706in}}%
\pgfpathlineto{\pgfqpoint{2.755150in}{3.240677in}}%
\pgfpathlineto{\pgfqpoint{2.756224in}{3.238580in}}%
\pgfpathlineto{\pgfqpoint{2.761593in}{3.240523in}}%
\pgfpathlineto{\pgfqpoint{2.762667in}{3.243656in}}%
\pgfpathlineto{\pgfqpoint{2.763741in}{3.242875in}}%
\pgfpathlineto{\pgfqpoint{2.766962in}{3.242506in}}%
\pgfpathlineto{\pgfqpoint{2.768036in}{3.243245in}}%
\pgfpathlineto{\pgfqpoint{2.770184in}{3.241981in}}%
\pgfpathlineto{\pgfqpoint{2.771258in}{3.242300in}}%
\pgfpathlineto{\pgfqpoint{2.776627in}{3.245687in}}%
\pgfpathlineto{\pgfqpoint{2.777701in}{3.244044in}}%
\pgfpathlineto{\pgfqpoint{2.781996in}{3.242745in}}%
\pgfpathlineto{\pgfqpoint{2.783070in}{3.243399in}}%
\pgfpathlineto{\pgfqpoint{2.784144in}{3.241407in}}%
\pgfpathlineto{\pgfqpoint{2.786291in}{3.240703in}}%
\pgfpathlineto{\pgfqpoint{2.789513in}{3.242110in}}%
\pgfpathlineto{\pgfqpoint{2.790587in}{3.241289in}}%
\pgfpathlineto{\pgfqpoint{2.791661in}{3.242402in}}%
\pgfpathlineto{\pgfqpoint{2.793808in}{3.243149in}}%
\pgfpathlineto{\pgfqpoint{2.797030in}{3.239644in}}%
\pgfpathlineto{\pgfqpoint{2.798104in}{3.239405in}}%
\pgfpathlineto{\pgfqpoint{2.799178in}{3.238330in}}%
\pgfpathlineto{\pgfqpoint{2.801325in}{3.238011in}}%
\pgfpathlineto{\pgfqpoint{2.806694in}{3.239800in}}%
\pgfpathlineto{\pgfqpoint{2.807768in}{3.241498in}}%
\pgfpathlineto{\pgfqpoint{2.808842in}{3.240568in}}%
\pgfpathlineto{\pgfqpoint{2.812064in}{3.246069in}}%
\pgfpathlineto{\pgfqpoint{2.814211in}{3.242699in}}%
\pgfpathlineto{\pgfqpoint{2.815285in}{3.241935in}}%
\pgfpathlineto{\pgfqpoint{2.816359in}{3.239734in}}%
\pgfpathlineto{\pgfqpoint{2.822802in}{3.236229in}}%
\pgfpathlineto{\pgfqpoint{2.823876in}{3.237038in}}%
\pgfpathlineto{\pgfqpoint{2.828171in}{3.237442in}}%
\pgfpathlineto{\pgfqpoint{2.829245in}{3.237802in}}%
\pgfpathlineto{\pgfqpoint{2.830319in}{3.237258in}}%
\pgfpathlineto{\pgfqpoint{2.834614in}{3.245515in}}%
\pgfpathlineto{\pgfqpoint{2.835688in}{3.240116in}}%
\pgfpathlineto{\pgfqpoint{2.836762in}{3.241846in}}%
\pgfpathlineto{\pgfqpoint{2.838910in}{3.241741in}}%
\pgfpathlineto{\pgfqpoint{2.843205in}{3.239904in}}%
\pgfpathlineto{\pgfqpoint{2.844279in}{3.241426in}}%
\pgfpathlineto{\pgfqpoint{2.846426in}{3.241216in}}%
\pgfpathlineto{\pgfqpoint{2.849648in}{3.241581in}}%
\pgfpathlineto{\pgfqpoint{2.850722in}{3.243003in}}%
\pgfpathlineto{\pgfqpoint{2.851796in}{3.242635in}}%
\pgfpathlineto{\pgfqpoint{2.852869in}{3.243575in}}%
\pgfpathlineto{\pgfqpoint{2.853943in}{3.241548in}}%
\pgfpathlineto{\pgfqpoint{2.857165in}{3.241921in}}%
\pgfpathlineto{\pgfqpoint{2.858239in}{3.242881in}}%
\pgfpathlineto{\pgfqpoint{2.861460in}{3.242827in}}%
\pgfpathlineto{\pgfqpoint{2.864682in}{3.242556in}}%
\pgfpathlineto{\pgfqpoint{2.866829in}{3.244618in}}%
\pgfpathlineto{\pgfqpoint{2.867903in}{3.244011in}}%
\pgfpathlineto{\pgfqpoint{2.868977in}{3.244894in}}%
\pgfpathlineto{\pgfqpoint{2.872199in}{3.245170in}}%
\pgfpathlineto{\pgfqpoint{2.874346in}{3.243559in}}%
\pgfpathlineto{\pgfqpoint{2.876494in}{3.243504in}}%
\pgfpathlineto{\pgfqpoint{2.882937in}{3.243392in}}%
\pgfpathlineto{\pgfqpoint{2.884011in}{3.243835in}}%
\pgfpathlineto{\pgfqpoint{2.884011in}{3.243835in}}%
\pgfusepath{stroke}%
\end{pgfscope}%
\begin{pgfscope}%
\pgfsetrectcap%
\pgfsetmiterjoin%
\pgfsetlinewidth{0.803000pt}%
\definecolor{currentstroke}{rgb}{1.000000,1.000000,1.000000}%
\pgfsetstrokecolor{currentstroke}%
\pgfsetdash{}{0pt}%
\pgfpathmoveto{\pgfqpoint{0.418102in}{3.218007in}}%
\pgfpathlineto{\pgfqpoint{0.418102in}{3.618892in}}%
\pgfusepath{stroke}%
\end{pgfscope}%
\begin{pgfscope}%
\pgfsetrectcap%
\pgfsetmiterjoin%
\pgfsetlinewidth{0.803000pt}%
\definecolor{currentstroke}{rgb}{1.000000,1.000000,1.000000}%
\pgfsetstrokecolor{currentstroke}%
\pgfsetdash{}{0pt}%
\pgfpathmoveto{\pgfqpoint{3.001435in}{3.218007in}}%
\pgfpathlineto{\pgfqpoint{3.001435in}{3.618892in}}%
\pgfusepath{stroke}%
\end{pgfscope}%
\begin{pgfscope}%
\pgfsetrectcap%
\pgfsetmiterjoin%
\pgfsetlinewidth{0.803000pt}%
\definecolor{currentstroke}{rgb}{1.000000,1.000000,1.000000}%
\pgfsetstrokecolor{currentstroke}%
\pgfsetdash{}{0pt}%
\pgfpathmoveto{\pgfqpoint{0.418102in}{3.218007in}}%
\pgfpathlineto{\pgfqpoint{3.001435in}{3.218007in}}%
\pgfusepath{stroke}%
\end{pgfscope}%
\begin{pgfscope}%
\pgfsetrectcap%
\pgfsetmiterjoin%
\pgfsetlinewidth{0.803000pt}%
\definecolor{currentstroke}{rgb}{1.000000,1.000000,1.000000}%
\pgfsetstrokecolor{currentstroke}%
\pgfsetdash{}{0pt}%
\pgfpathmoveto{\pgfqpoint{0.418102in}{3.618892in}}%
\pgfpathlineto{\pgfqpoint{3.001435in}{3.618892in}}%
\pgfusepath{stroke}%
\end{pgfscope}%
\begin{pgfscope}%
\definecolor{textcolor}{rgb}{0.150000,0.150000,0.150000}%
\pgfsetstrokecolor{textcolor}%
\pgfsetfillcolor{textcolor}%
\pgftext[x=1.709768in,y=3.702225in,,base]{\color{textcolor}\rmfamily\fontsize{12.000000}{14.400000}\selectfont GE}%
\end{pgfscope}%
\begin{pgfscope}%
\pgfsetbuttcap%
\pgfsetmiterjoin%
\definecolor{currentfill}{rgb}{0.917647,0.917647,0.949020}%
\pgfsetfillcolor{currentfill}%
\pgfsetlinewidth{0.000000pt}%
\definecolor{currentstroke}{rgb}{0.000000,0.000000,0.000000}%
\pgfsetstrokecolor{currentstroke}%
\pgfsetstrokeopacity{0.000000}%
\pgfsetdash{}{0pt}%
\pgfpathmoveto{\pgfqpoint{4.034768in}{3.218007in}}%
\pgfpathlineto{\pgfqpoint{6.618102in}{3.218007in}}%
\pgfpathlineto{\pgfqpoint{6.618102in}{3.618892in}}%
\pgfpathlineto{\pgfqpoint{4.034768in}{3.618892in}}%
\pgfpathclose%
\pgfusepath{fill}%
\end{pgfscope}%
\begin{pgfscope}%
\pgfpathrectangle{\pgfqpoint{4.034768in}{3.218007in}}{\pgfqpoint{2.583333in}{0.400885in}}%
\pgfusepath{clip}%
\pgfsetroundcap%
\pgfsetroundjoin%
\pgfsetlinewidth{0.803000pt}%
\definecolor{currentstroke}{rgb}{1.000000,1.000000,1.000000}%
\pgfsetstrokecolor{currentstroke}%
\pgfsetdash{}{0pt}%
\pgfpathmoveto{\pgfqpoint{4.150045in}{3.218007in}}%
\pgfpathlineto{\pgfqpoint{4.150045in}{3.618892in}}%
\pgfusepath{stroke}%
\end{pgfscope}%
\begin{pgfscope}%
\definecolor{textcolor}{rgb}{0.150000,0.150000,0.150000}%
\pgfsetstrokecolor{textcolor}%
\pgfsetfillcolor{textcolor}%
\pgftext[x=4.150045in,y=3.120784in,,top]{\color{textcolor}\rmfamily\fontsize{10.000000}{12.000000}\selectfont 2012}%
\end{pgfscope}%
\begin{pgfscope}%
\pgfpathrectangle{\pgfqpoint{4.034768in}{3.218007in}}{\pgfqpoint{2.583333in}{0.400885in}}%
\pgfusepath{clip}%
\pgfsetroundcap%
\pgfsetroundjoin%
\pgfsetlinewidth{0.803000pt}%
\definecolor{currentstroke}{rgb}{1.000000,1.000000,1.000000}%
\pgfsetstrokecolor{currentstroke}%
\pgfsetdash{}{0pt}%
\pgfpathmoveto{\pgfqpoint{4.543070in}{3.218007in}}%
\pgfpathlineto{\pgfqpoint{4.543070in}{3.618892in}}%
\pgfusepath{stroke}%
\end{pgfscope}%
\begin{pgfscope}%
\definecolor{textcolor}{rgb}{0.150000,0.150000,0.150000}%
\pgfsetstrokecolor{textcolor}%
\pgfsetfillcolor{textcolor}%
\pgftext[x=4.543070in,y=3.120784in,,top]{\color{textcolor}\rmfamily\fontsize{10.000000}{12.000000}\selectfont 2013}%
\end{pgfscope}%
\begin{pgfscope}%
\pgfpathrectangle{\pgfqpoint{4.034768in}{3.218007in}}{\pgfqpoint{2.583333in}{0.400885in}}%
\pgfusepath{clip}%
\pgfsetroundcap%
\pgfsetroundjoin%
\pgfsetlinewidth{0.803000pt}%
\definecolor{currentstroke}{rgb}{1.000000,1.000000,1.000000}%
\pgfsetstrokecolor{currentstroke}%
\pgfsetdash{}{0pt}%
\pgfpathmoveto{\pgfqpoint{4.935021in}{3.218007in}}%
\pgfpathlineto{\pgfqpoint{4.935021in}{3.618892in}}%
\pgfusepath{stroke}%
\end{pgfscope}%
\begin{pgfscope}%
\definecolor{textcolor}{rgb}{0.150000,0.150000,0.150000}%
\pgfsetstrokecolor{textcolor}%
\pgfsetfillcolor{textcolor}%
\pgftext[x=4.935021in,y=3.120784in,,top]{\color{textcolor}\rmfamily\fontsize{10.000000}{12.000000}\selectfont 2014}%
\end{pgfscope}%
\begin{pgfscope}%
\pgfpathrectangle{\pgfqpoint{4.034768in}{3.218007in}}{\pgfqpoint{2.583333in}{0.400885in}}%
\pgfusepath{clip}%
\pgfsetroundcap%
\pgfsetroundjoin%
\pgfsetlinewidth{0.803000pt}%
\definecolor{currentstroke}{rgb}{1.000000,1.000000,1.000000}%
\pgfsetstrokecolor{currentstroke}%
\pgfsetdash{}{0pt}%
\pgfpathmoveto{\pgfqpoint{5.326972in}{3.218007in}}%
\pgfpathlineto{\pgfqpoint{5.326972in}{3.618892in}}%
\pgfusepath{stroke}%
\end{pgfscope}%
\begin{pgfscope}%
\definecolor{textcolor}{rgb}{0.150000,0.150000,0.150000}%
\pgfsetstrokecolor{textcolor}%
\pgfsetfillcolor{textcolor}%
\pgftext[x=5.326972in,y=3.120784in,,top]{\color{textcolor}\rmfamily\fontsize{10.000000}{12.000000}\selectfont 2015}%
\end{pgfscope}%
\begin{pgfscope}%
\pgfpathrectangle{\pgfqpoint{4.034768in}{3.218007in}}{\pgfqpoint{2.583333in}{0.400885in}}%
\pgfusepath{clip}%
\pgfsetroundcap%
\pgfsetroundjoin%
\pgfsetlinewidth{0.803000pt}%
\definecolor{currentstroke}{rgb}{1.000000,1.000000,1.000000}%
\pgfsetstrokecolor{currentstroke}%
\pgfsetdash{}{0pt}%
\pgfpathmoveto{\pgfqpoint{5.718923in}{3.218007in}}%
\pgfpathlineto{\pgfqpoint{5.718923in}{3.618892in}}%
\pgfusepath{stroke}%
\end{pgfscope}%
\begin{pgfscope}%
\definecolor{textcolor}{rgb}{0.150000,0.150000,0.150000}%
\pgfsetstrokecolor{textcolor}%
\pgfsetfillcolor{textcolor}%
\pgftext[x=5.718923in,y=3.120784in,,top]{\color{textcolor}\rmfamily\fontsize{10.000000}{12.000000}\selectfont 2016}%
\end{pgfscope}%
\begin{pgfscope}%
\pgfpathrectangle{\pgfqpoint{4.034768in}{3.218007in}}{\pgfqpoint{2.583333in}{0.400885in}}%
\pgfusepath{clip}%
\pgfsetroundcap%
\pgfsetroundjoin%
\pgfsetlinewidth{0.803000pt}%
\definecolor{currentstroke}{rgb}{1.000000,1.000000,1.000000}%
\pgfsetstrokecolor{currentstroke}%
\pgfsetdash{}{0pt}%
\pgfpathmoveto{\pgfqpoint{6.111948in}{3.218007in}}%
\pgfpathlineto{\pgfqpoint{6.111948in}{3.618892in}}%
\pgfusepath{stroke}%
\end{pgfscope}%
\begin{pgfscope}%
\definecolor{textcolor}{rgb}{0.150000,0.150000,0.150000}%
\pgfsetstrokecolor{textcolor}%
\pgfsetfillcolor{textcolor}%
\pgftext[x=6.111948in,y=3.120784in,,top]{\color{textcolor}\rmfamily\fontsize{10.000000}{12.000000}\selectfont 2017}%
\end{pgfscope}%
\begin{pgfscope}%
\pgfpathrectangle{\pgfqpoint{4.034768in}{3.218007in}}{\pgfqpoint{2.583333in}{0.400885in}}%
\pgfusepath{clip}%
\pgfsetroundcap%
\pgfsetroundjoin%
\pgfsetlinewidth{0.803000pt}%
\definecolor{currentstroke}{rgb}{1.000000,1.000000,1.000000}%
\pgfsetstrokecolor{currentstroke}%
\pgfsetdash{}{0pt}%
\pgfpathmoveto{\pgfqpoint{6.503899in}{3.218007in}}%
\pgfpathlineto{\pgfqpoint{6.503899in}{3.618892in}}%
\pgfusepath{stroke}%
\end{pgfscope}%
\begin{pgfscope}%
\definecolor{textcolor}{rgb}{0.150000,0.150000,0.150000}%
\pgfsetstrokecolor{textcolor}%
\pgfsetfillcolor{textcolor}%
\pgftext[x=6.503899in,y=3.120784in,,top]{\color{textcolor}\rmfamily\fontsize{10.000000}{12.000000}\selectfont 2018}%
\end{pgfscope}%
\begin{pgfscope}%
\pgfpathrectangle{\pgfqpoint{4.034768in}{3.218007in}}{\pgfqpoint{2.583333in}{0.400885in}}%
\pgfusepath{clip}%
\pgfsetroundcap%
\pgfsetroundjoin%
\pgfsetlinewidth{0.803000pt}%
\definecolor{currentstroke}{rgb}{1.000000,1.000000,1.000000}%
\pgfsetstrokecolor{currentstroke}%
\pgfsetdash{}{0pt}%
\pgfpathmoveto{\pgfqpoint{4.034768in}{3.285139in}}%
\pgfpathlineto{\pgfqpoint{6.618102in}{3.285139in}}%
\pgfusepath{stroke}%
\end{pgfscope}%
\begin{pgfscope}%
\definecolor{textcolor}{rgb}{0.150000,0.150000,0.150000}%
\pgfsetstrokecolor{textcolor}%
\pgfsetfillcolor{textcolor}%
\pgftext[x=3.849181in,y=3.232377in,left,base]{\color{textcolor}\rmfamily\fontsize{10.000000}{12.000000}\selectfont 1}%
\end{pgfscope}%
\begin{pgfscope}%
\pgfpathrectangle{\pgfqpoint{4.034768in}{3.218007in}}{\pgfqpoint{2.583333in}{0.400885in}}%
\pgfusepath{clip}%
\pgfsetroundcap%
\pgfsetroundjoin%
\pgfsetlinewidth{0.803000pt}%
\definecolor{currentstroke}{rgb}{1.000000,1.000000,1.000000}%
\pgfsetstrokecolor{currentstroke}%
\pgfsetdash{}{0pt}%
\pgfpathmoveto{\pgfqpoint{4.034768in}{3.518111in}}%
\pgfpathlineto{\pgfqpoint{6.618102in}{3.518111in}}%
\pgfusepath{stroke}%
\end{pgfscope}%
\begin{pgfscope}%
\definecolor{textcolor}{rgb}{0.150000,0.150000,0.150000}%
\pgfsetstrokecolor{textcolor}%
\pgfsetfillcolor{textcolor}%
\pgftext[x=3.849181in,y=3.465349in,left,base]{\color{textcolor}\rmfamily\fontsize{10.000000}{12.000000}\selectfont 2}%
\end{pgfscope}%
\begin{pgfscope}%
\pgfpathrectangle{\pgfqpoint{4.034768in}{3.218007in}}{\pgfqpoint{2.583333in}{0.400885in}}%
\pgfusepath{clip}%
\pgfsetroundcap%
\pgfsetroundjoin%
\pgfsetlinewidth{1.505625pt}%
\definecolor{currentstroke}{rgb}{0.000000,0.000000,0.000000}%
\pgfsetstrokecolor{currentstroke}%
\pgfsetdash{}{0pt}%
\pgfpathmoveto{\pgfqpoint{4.152193in}{3.285139in}}%
\pgfpathlineto{\pgfqpoint{4.154340in}{3.293334in}}%
\pgfpathlineto{\pgfqpoint{4.155414in}{3.291888in}}%
\pgfpathlineto{\pgfqpoint{4.159709in}{3.295142in}}%
\pgfpathlineto{\pgfqpoint{4.160783in}{3.297191in}}%
\pgfpathlineto{\pgfqpoint{4.161857in}{3.296709in}}%
\pgfpathlineto{\pgfqpoint{4.162931in}{3.290924in}}%
\pgfpathlineto{\pgfqpoint{4.167226in}{3.289960in}}%
\pgfpathlineto{\pgfqpoint{4.169374in}{3.295504in}}%
\pgfpathlineto{\pgfqpoint{4.170448in}{3.302615in}}%
\pgfpathlineto{\pgfqpoint{4.175817in}{3.307556in}}%
\pgfpathlineto{\pgfqpoint{4.176891in}{3.306110in}}%
\pgfpathlineto{\pgfqpoint{4.181186in}{3.306110in}}%
\pgfpathlineto{\pgfqpoint{4.182260in}{3.302976in}}%
\pgfpathlineto{\pgfqpoint{4.183334in}{3.304302in}}%
\pgfpathlineto{\pgfqpoint{4.184408in}{3.303699in}}%
\pgfpathlineto{\pgfqpoint{4.185482in}{3.308038in}}%
\pgfpathlineto{\pgfqpoint{4.188703in}{3.307918in}}%
\pgfpathlineto{\pgfqpoint{4.189777in}{3.307074in}}%
\pgfpathlineto{\pgfqpoint{4.190851in}{3.309123in}}%
\pgfpathlineto{\pgfqpoint{4.191925in}{3.309244in}}%
\pgfpathlineto{\pgfqpoint{4.192998in}{3.307677in}}%
\pgfpathlineto{\pgfqpoint{4.196220in}{3.307677in}}%
\pgfpathlineto{\pgfqpoint{4.197294in}{3.308520in}}%
\pgfpathlineto{\pgfqpoint{4.198368in}{3.306592in}}%
\pgfpathlineto{\pgfqpoint{4.199441in}{3.309002in}}%
\pgfpathlineto{\pgfqpoint{4.200515in}{3.314064in}}%
\pgfpathlineto{\pgfqpoint{4.204811in}{3.312136in}}%
\pgfpathlineto{\pgfqpoint{4.205885in}{3.308038in}}%
\pgfpathlineto{\pgfqpoint{4.206958in}{3.307315in}}%
\pgfpathlineto{\pgfqpoint{4.208032in}{3.307677in}}%
\pgfpathlineto{\pgfqpoint{4.211254in}{3.309485in}}%
\pgfpathlineto{\pgfqpoint{4.212328in}{3.312859in}}%
\pgfpathlineto{\pgfqpoint{4.213401in}{3.309485in}}%
\pgfpathlineto{\pgfqpoint{4.214475in}{3.309244in}}%
\pgfpathlineto{\pgfqpoint{4.215549in}{3.309846in}}%
\pgfpathlineto{\pgfqpoint{4.218771in}{3.306230in}}%
\pgfpathlineto{\pgfqpoint{4.219844in}{3.306833in}}%
\pgfpathlineto{\pgfqpoint{4.220918in}{3.309726in}}%
\pgfpathlineto{\pgfqpoint{4.221992in}{3.309002in}}%
\pgfpathlineto{\pgfqpoint{4.223066in}{3.311292in}}%
\pgfpathlineto{\pgfqpoint{4.226287in}{3.310449in}}%
\pgfpathlineto{\pgfqpoint{4.227361in}{3.315270in}}%
\pgfpathlineto{\pgfqpoint{4.228435in}{3.315029in}}%
\pgfpathlineto{\pgfqpoint{4.229509in}{3.317801in}}%
\pgfpathlineto{\pgfqpoint{4.230583in}{3.317560in}}%
\pgfpathlineto{\pgfqpoint{4.235952in}{3.318042in}}%
\pgfpathlineto{\pgfqpoint{4.237026in}{3.319247in}}%
\pgfpathlineto{\pgfqpoint{4.238100in}{3.319006in}}%
\pgfpathlineto{\pgfqpoint{4.241321in}{3.322019in}}%
\pgfpathlineto{\pgfqpoint{4.242395in}{3.322019in}}%
\pgfpathlineto{\pgfqpoint{4.243469in}{3.318283in}}%
\pgfpathlineto{\pgfqpoint{4.244543in}{3.321657in}}%
\pgfpathlineto{\pgfqpoint{4.245617in}{3.321296in}}%
\pgfpathlineto{\pgfqpoint{4.248838in}{3.323827in}}%
\pgfpathlineto{\pgfqpoint{4.250986in}{3.319488in}}%
\pgfpathlineto{\pgfqpoint{4.252060in}{3.320814in}}%
\pgfpathlineto{\pgfqpoint{4.256355in}{3.317801in}}%
\pgfpathlineto{\pgfqpoint{4.257429in}{3.314908in}}%
\pgfpathlineto{\pgfqpoint{4.258503in}{3.318765in}}%
\pgfpathlineto{\pgfqpoint{4.259576in}{3.324791in}}%
\pgfpathlineto{\pgfqpoint{4.260650in}{3.321055in}}%
\pgfpathlineto{\pgfqpoint{4.264946in}{3.324671in}}%
\pgfpathlineto{\pgfqpoint{4.267093in}{3.317198in}}%
\pgfpathlineto{\pgfqpoint{4.268167in}{3.316354in}}%
\pgfpathlineto{\pgfqpoint{4.271389in}{3.314908in}}%
\pgfpathlineto{\pgfqpoint{4.272462in}{3.313582in}}%
\pgfpathlineto{\pgfqpoint{4.274610in}{3.322260in}}%
\pgfpathlineto{\pgfqpoint{4.275684in}{3.323827in}}%
\pgfpathlineto{\pgfqpoint{4.278906in}{3.323947in}}%
\pgfpathlineto{\pgfqpoint{4.279979in}{3.329250in}}%
\pgfpathlineto{\pgfqpoint{4.281053in}{3.331420in}}%
\pgfpathlineto{\pgfqpoint{4.282127in}{3.327443in}}%
\pgfpathlineto{\pgfqpoint{4.283201in}{3.321175in}}%
\pgfpathlineto{\pgfqpoint{4.286422in}{3.319729in}}%
\pgfpathlineto{\pgfqpoint{4.287496in}{3.315993in}}%
\pgfpathlineto{\pgfqpoint{4.288570in}{3.314306in}}%
\pgfpathlineto{\pgfqpoint{4.289644in}{3.314788in}}%
\pgfpathlineto{\pgfqpoint{4.290718in}{3.318524in}}%
\pgfpathlineto{\pgfqpoint{4.296087in}{3.307677in}}%
\pgfpathlineto{\pgfqpoint{4.297161in}{3.304664in}}%
\pgfpathlineto{\pgfqpoint{4.298235in}{3.303458in}}%
\pgfpathlineto{\pgfqpoint{4.301456in}{3.304302in}}%
\pgfpathlineto{\pgfqpoint{4.302530in}{3.303097in}}%
\pgfpathlineto{\pgfqpoint{4.303604in}{3.297432in}}%
\pgfpathlineto{\pgfqpoint{4.304678in}{3.299481in}}%
\pgfpathlineto{\pgfqpoint{4.311121in}{3.304061in}}%
\pgfpathlineto{\pgfqpoint{4.312195in}{3.301289in}}%
\pgfpathlineto{\pgfqpoint{4.313268in}{3.294540in}}%
\pgfpathlineto{\pgfqpoint{4.316490in}{3.293575in}}%
\pgfpathlineto{\pgfqpoint{4.317564in}{3.297312in}}%
\pgfpathlineto{\pgfqpoint{4.318638in}{3.303458in}}%
\pgfpathlineto{\pgfqpoint{4.319711in}{3.302253in}}%
\pgfpathlineto{\pgfqpoint{4.320785in}{3.306713in}}%
\pgfpathlineto{\pgfqpoint{4.324007in}{3.302735in}}%
\pgfpathlineto{\pgfqpoint{4.325081in}{3.307797in}}%
\pgfpathlineto{\pgfqpoint{4.326154in}{3.308038in}}%
\pgfpathlineto{\pgfqpoint{4.328302in}{3.315752in}}%
\pgfpathlineto{\pgfqpoint{4.331524in}{3.316475in}}%
\pgfpathlineto{\pgfqpoint{4.333671in}{3.318644in}}%
\pgfpathlineto{\pgfqpoint{4.334745in}{3.309605in}}%
\pgfpathlineto{\pgfqpoint{4.335819in}{3.311895in}}%
\pgfpathlineto{\pgfqpoint{4.339040in}{3.303338in}}%
\pgfpathlineto{\pgfqpoint{4.340114in}{3.302856in}}%
\pgfpathlineto{\pgfqpoint{4.341188in}{3.304905in}}%
\pgfpathlineto{\pgfqpoint{4.342262in}{3.301168in}}%
\pgfpathlineto{\pgfqpoint{4.343336in}{3.309123in}}%
\pgfpathlineto{\pgfqpoint{4.346557in}{3.309244in}}%
\pgfpathlineto{\pgfqpoint{4.347631in}{3.311051in}}%
\pgfpathlineto{\pgfqpoint{4.349779in}{3.308159in}}%
\pgfpathlineto{\pgfqpoint{4.350853in}{3.304302in}}%
\pgfpathlineto{\pgfqpoint{4.354074in}{3.304423in}}%
\pgfpathlineto{\pgfqpoint{4.355148in}{3.298517in}}%
\pgfpathlineto{\pgfqpoint{4.356222in}{3.296950in}}%
\pgfpathlineto{\pgfqpoint{4.357296in}{3.290683in}}%
\pgfpathlineto{\pgfqpoint{4.358370in}{3.295624in}}%
\pgfpathlineto{\pgfqpoint{4.361591in}{3.294419in}}%
\pgfpathlineto{\pgfqpoint{4.362665in}{3.296830in}}%
\pgfpathlineto{\pgfqpoint{4.363739in}{3.304784in}}%
\pgfpathlineto{\pgfqpoint{4.364813in}{3.303338in}}%
\pgfpathlineto{\pgfqpoint{4.365886in}{3.298155in}}%
\pgfpathlineto{\pgfqpoint{4.369108in}{3.295624in}}%
\pgfpathlineto{\pgfqpoint{4.370182in}{3.293214in}}%
\pgfpathlineto{\pgfqpoint{4.371256in}{3.294419in}}%
\pgfpathlineto{\pgfqpoint{4.373403in}{3.302976in}}%
\pgfpathlineto{\pgfqpoint{4.377699in}{3.299963in}}%
\pgfpathlineto{\pgfqpoint{4.378773in}{3.302133in}}%
\pgfpathlineto{\pgfqpoint{4.379846in}{3.301892in}}%
\pgfpathlineto{\pgfqpoint{4.380920in}{3.307195in}}%
\pgfpathlineto{\pgfqpoint{4.384142in}{3.308038in}}%
\pgfpathlineto{\pgfqpoint{4.386289in}{3.310810in}}%
\pgfpathlineto{\pgfqpoint{4.387363in}{3.311775in}}%
\pgfpathlineto{\pgfqpoint{4.388437in}{3.313582in}}%
\pgfpathlineto{\pgfqpoint{4.391659in}{3.311775in}}%
\pgfpathlineto{\pgfqpoint{4.393806in}{3.307677in}}%
\pgfpathlineto{\pgfqpoint{4.394880in}{3.310690in}}%
\pgfpathlineto{\pgfqpoint{4.395954in}{3.308159in}}%
\pgfpathlineto{\pgfqpoint{4.399175in}{3.307195in}}%
\pgfpathlineto{\pgfqpoint{4.400249in}{3.306110in}}%
\pgfpathlineto{\pgfqpoint{4.401323in}{3.302374in}}%
\pgfpathlineto{\pgfqpoint{4.402397in}{3.295624in}}%
\pgfpathlineto{\pgfqpoint{4.403471in}{3.294419in}}%
\pgfpathlineto{\pgfqpoint{4.406692in}{3.293696in}}%
\pgfpathlineto{\pgfqpoint{4.407766in}{3.295263in}}%
\pgfpathlineto{\pgfqpoint{4.409914in}{3.288152in}}%
\pgfpathlineto{\pgfqpoint{4.410988in}{3.293575in}}%
\pgfpathlineto{\pgfqpoint{4.416357in}{3.289357in}}%
\pgfpathlineto{\pgfqpoint{4.417431in}{3.296227in}}%
\pgfpathlineto{\pgfqpoint{4.418505in}{3.287429in}}%
\pgfpathlineto{\pgfqpoint{4.421726in}{3.278389in}}%
\pgfpathlineto{\pgfqpoint{4.422800in}{3.279113in}}%
\pgfpathlineto{\pgfqpoint{4.423874in}{3.277666in}}%
\pgfpathlineto{\pgfqpoint{4.424948in}{3.279354in}}%
\pgfpathlineto{\pgfqpoint{4.426021in}{3.279474in}}%
\pgfpathlineto{\pgfqpoint{4.429243in}{3.278872in}}%
\pgfpathlineto{\pgfqpoint{4.430317in}{3.279474in}}%
\pgfpathlineto{\pgfqpoint{4.431391in}{3.277305in}}%
\pgfpathlineto{\pgfqpoint{4.432464in}{3.277546in}}%
\pgfpathlineto{\pgfqpoint{4.433538in}{3.277064in}}%
\pgfpathlineto{\pgfqpoint{4.436760in}{3.273930in}}%
\pgfpathlineto{\pgfqpoint{4.437834in}{3.271399in}}%
\pgfpathlineto{\pgfqpoint{4.438907in}{3.272484in}}%
\pgfpathlineto{\pgfqpoint{4.439981in}{3.276702in}}%
\pgfpathlineto{\pgfqpoint{4.441055in}{3.272484in}}%
\pgfpathlineto{\pgfqpoint{4.445350in}{3.274292in}}%
\pgfpathlineto{\pgfqpoint{4.446424in}{3.271399in}}%
\pgfpathlineto{\pgfqpoint{4.447498in}{3.270676in}}%
\pgfpathlineto{\pgfqpoint{4.448572in}{3.272725in}}%
\pgfpathlineto{\pgfqpoint{4.451794in}{3.271038in}}%
\pgfpathlineto{\pgfqpoint{4.452867in}{3.265132in}}%
\pgfpathlineto{\pgfqpoint{4.456089in}{3.261034in}}%
\pgfpathlineto{\pgfqpoint{4.459310in}{3.263445in}}%
\pgfpathlineto{\pgfqpoint{4.460384in}{3.269471in}}%
\pgfpathlineto{\pgfqpoint{4.461458in}{3.264047in}}%
\pgfpathlineto{\pgfqpoint{4.462532in}{3.262842in}}%
\pgfpathlineto{\pgfqpoint{4.463606in}{3.258985in}}%
\pgfpathlineto{\pgfqpoint{4.467901in}{3.262119in}}%
\pgfpathlineto{\pgfqpoint{4.468975in}{3.260914in}}%
\pgfpathlineto{\pgfqpoint{4.471123in}{3.265614in}}%
\pgfpathlineto{\pgfqpoint{4.476492in}{3.262480in}}%
\pgfpathlineto{\pgfqpoint{4.477566in}{3.268627in}}%
\pgfpathlineto{\pgfqpoint{4.478639in}{3.266699in}}%
\pgfpathlineto{\pgfqpoint{4.481861in}{3.266699in}}%
\pgfpathlineto{\pgfqpoint{4.482935in}{3.265614in}}%
\pgfpathlineto{\pgfqpoint{4.484009in}{3.257659in}}%
\pgfpathlineto{\pgfqpoint{4.486156in}{3.256575in}}%
\pgfpathlineto{\pgfqpoint{4.489378in}{3.256213in}}%
\pgfpathlineto{\pgfqpoint{4.491526in}{3.248259in}}%
\pgfpathlineto{\pgfqpoint{4.492599in}{3.248982in}}%
\pgfpathlineto{\pgfqpoint{4.493673in}{3.250549in}}%
\pgfpathlineto{\pgfqpoint{4.496895in}{3.251151in}}%
\pgfpathlineto{\pgfqpoint{4.497969in}{3.243799in}}%
\pgfpathlineto{\pgfqpoint{4.499042in}{3.242353in}}%
\pgfpathlineto{\pgfqpoint{4.501190in}{3.245969in}}%
\pgfpathlineto{\pgfqpoint{4.505485in}{3.248018in}}%
\pgfpathlineto{\pgfqpoint{4.506559in}{3.249584in}}%
\pgfpathlineto{\pgfqpoint{4.507633in}{3.244040in}}%
\pgfpathlineto{\pgfqpoint{4.508707in}{3.244402in}}%
\pgfpathlineto{\pgfqpoint{4.511928in}{3.244161in}}%
\pgfpathlineto{\pgfqpoint{4.513002in}{3.248379in}}%
\pgfpathlineto{\pgfqpoint{4.514076in}{3.247174in}}%
\pgfpathlineto{\pgfqpoint{4.515150in}{3.250187in}}%
\pgfpathlineto{\pgfqpoint{4.516224in}{3.250187in}}%
\pgfpathlineto{\pgfqpoint{4.519445in}{3.249464in}}%
\pgfpathlineto{\pgfqpoint{4.520519in}{3.255008in}}%
\pgfpathlineto{\pgfqpoint{4.521593in}{3.255249in}}%
\pgfpathlineto{\pgfqpoint{4.522667in}{3.253441in}}%
\pgfpathlineto{\pgfqpoint{4.523741in}{3.253923in}}%
\pgfpathlineto{\pgfqpoint{4.526962in}{3.254285in}}%
\pgfpathlineto{\pgfqpoint{4.528036in}{3.258142in}}%
\pgfpathlineto{\pgfqpoint{4.529110in}{3.259467in}}%
\pgfpathlineto{\pgfqpoint{4.530184in}{3.258744in}}%
\pgfpathlineto{\pgfqpoint{4.531258in}{3.256213in}}%
\pgfpathlineto{\pgfqpoint{4.534479in}{3.255008in}}%
\pgfpathlineto{\pgfqpoint{4.536627in}{3.255008in}}%
\pgfpathlineto{\pgfqpoint{4.537701in}{3.253682in}}%
\pgfpathlineto{\pgfqpoint{4.538774in}{3.250910in}}%
\pgfpathlineto{\pgfqpoint{4.541996in}{3.254767in}}%
\pgfpathlineto{\pgfqpoint{4.544144in}{3.262239in}}%
\pgfpathlineto{\pgfqpoint{4.545217in}{3.261637in}}%
\pgfpathlineto{\pgfqpoint{4.546291in}{3.260070in}}%
\pgfpathlineto{\pgfqpoint{4.549513in}{3.260914in}}%
\pgfpathlineto{\pgfqpoint{4.550587in}{3.259347in}}%
\pgfpathlineto{\pgfqpoint{4.553808in}{3.268266in}}%
\pgfpathlineto{\pgfqpoint{4.557030in}{3.268266in}}%
\pgfpathlineto{\pgfqpoint{4.558104in}{3.267181in}}%
\pgfpathlineto{\pgfqpoint{4.559177in}{3.269350in}}%
\pgfpathlineto{\pgfqpoint{4.560251in}{3.275015in}}%
\pgfpathlineto{\pgfqpoint{4.561325in}{3.260914in}}%
\pgfpathlineto{\pgfqpoint{4.566694in}{3.259588in}}%
\pgfpathlineto{\pgfqpoint{4.567768in}{3.258021in}}%
\pgfpathlineto{\pgfqpoint{4.572063in}{3.258985in}}%
\pgfpathlineto{\pgfqpoint{4.573137in}{3.261275in}}%
\pgfpathlineto{\pgfqpoint{4.574211in}{3.262119in}}%
\pgfpathlineto{\pgfqpoint{4.575285in}{3.258865in}}%
\pgfpathlineto{\pgfqpoint{4.576359in}{3.261998in}}%
\pgfpathlineto{\pgfqpoint{4.579580in}{3.260070in}}%
\pgfpathlineto{\pgfqpoint{4.580654in}{3.262480in}}%
\pgfpathlineto{\pgfqpoint{4.582802in}{3.258865in}}%
\pgfpathlineto{\pgfqpoint{4.583876in}{3.260673in}}%
\pgfpathlineto{\pgfqpoint{4.587097in}{3.261034in}}%
\pgfpathlineto{\pgfqpoint{4.589245in}{3.263204in}}%
\pgfpathlineto{\pgfqpoint{4.590319in}{3.262962in}}%
\pgfpathlineto{\pgfqpoint{4.591393in}{3.261878in}}%
\pgfpathlineto{\pgfqpoint{4.595688in}{3.261637in}}%
\pgfpathlineto{\pgfqpoint{4.597836in}{3.253200in}}%
\pgfpathlineto{\pgfqpoint{4.598909in}{3.254887in}}%
\pgfpathlineto{\pgfqpoint{4.602131in}{3.253080in}}%
\pgfpathlineto{\pgfqpoint{4.604279in}{3.259949in}}%
\pgfpathlineto{\pgfqpoint{4.605352in}{3.259467in}}%
\pgfpathlineto{\pgfqpoint{4.606426in}{3.261034in}}%
\pgfpathlineto{\pgfqpoint{4.609648in}{3.263445in}}%
\pgfpathlineto{\pgfqpoint{4.612869in}{3.269591in}}%
\pgfpathlineto{\pgfqpoint{4.613943in}{3.266458in}}%
\pgfpathlineto{\pgfqpoint{4.617165in}{3.267542in}}%
\pgfpathlineto{\pgfqpoint{4.618238in}{3.267060in}}%
\pgfpathlineto{\pgfqpoint{4.620386in}{3.267181in}}%
\pgfpathlineto{\pgfqpoint{4.621460in}{3.264529in}}%
\pgfpathlineto{\pgfqpoint{4.624682in}{3.263324in}}%
\pgfpathlineto{\pgfqpoint{4.625755in}{3.262119in}}%
\pgfpathlineto{\pgfqpoint{4.626829in}{3.262480in}}%
\pgfpathlineto{\pgfqpoint{4.627903in}{3.261155in}}%
\pgfpathlineto{\pgfqpoint{4.628977in}{3.263927in}}%
\pgfpathlineto{\pgfqpoint{4.632198in}{3.262239in}}%
\pgfpathlineto{\pgfqpoint{4.633272in}{3.268386in}}%
\pgfpathlineto{\pgfqpoint{4.635420in}{3.268989in}}%
\pgfpathlineto{\pgfqpoint{4.639715in}{3.265011in}}%
\pgfpathlineto{\pgfqpoint{4.640789in}{3.265252in}}%
\pgfpathlineto{\pgfqpoint{4.641863in}{3.261155in}}%
\pgfpathlineto{\pgfqpoint{4.642937in}{3.262119in}}%
\pgfpathlineto{\pgfqpoint{4.644011in}{3.260070in}}%
\pgfpathlineto{\pgfqpoint{4.647232in}{3.261637in}}%
\pgfpathlineto{\pgfqpoint{4.649380in}{3.273207in}}%
\pgfpathlineto{\pgfqpoint{4.650454in}{3.268989in}}%
\pgfpathlineto{\pgfqpoint{4.651527in}{3.267422in}}%
\pgfpathlineto{\pgfqpoint{4.654749in}{3.264529in}}%
\pgfpathlineto{\pgfqpoint{4.655823in}{3.269832in}}%
\pgfpathlineto{\pgfqpoint{4.656897in}{3.269953in}}%
\pgfpathlineto{\pgfqpoint{4.659044in}{3.275015in}}%
\pgfpathlineto{\pgfqpoint{4.662266in}{3.279354in}}%
\pgfpathlineto{\pgfqpoint{4.664414in}{3.287067in}}%
\pgfpathlineto{\pgfqpoint{4.665487in}{3.284295in}}%
\pgfpathlineto{\pgfqpoint{4.666561in}{3.284536in}}%
\pgfpathlineto{\pgfqpoint{4.674078in}{3.292370in}}%
\pgfpathlineto{\pgfqpoint{4.677300in}{3.291888in}}%
\pgfpathlineto{\pgfqpoint{4.678373in}{3.294299in}}%
\pgfpathlineto{\pgfqpoint{4.681595in}{3.297794in}}%
\pgfpathlineto{\pgfqpoint{4.684816in}{3.293575in}}%
\pgfpathlineto{\pgfqpoint{4.685890in}{3.291165in}}%
\pgfpathlineto{\pgfqpoint{4.686964in}{3.294781in}}%
\pgfpathlineto{\pgfqpoint{4.688038in}{3.292129in}}%
\pgfpathlineto{\pgfqpoint{4.689112in}{3.293093in}}%
\pgfpathlineto{\pgfqpoint{4.692333in}{3.293575in}}%
\pgfpathlineto{\pgfqpoint{4.693407in}{3.294299in}}%
\pgfpathlineto{\pgfqpoint{4.696629in}{3.291888in}}%
\pgfpathlineto{\pgfqpoint{4.700924in}{3.293575in}}%
\pgfpathlineto{\pgfqpoint{4.701998in}{3.295504in}}%
\pgfpathlineto{\pgfqpoint{4.703072in}{3.294901in}}%
\pgfpathlineto{\pgfqpoint{4.704146in}{3.295504in}}%
\pgfpathlineto{\pgfqpoint{4.707367in}{3.305146in}}%
\pgfpathlineto{\pgfqpoint{4.708441in}{3.306351in}}%
\pgfpathlineto{\pgfqpoint{4.709515in}{3.299722in}}%
\pgfpathlineto{\pgfqpoint{4.711662in}{3.298637in}}%
\pgfpathlineto{\pgfqpoint{4.714884in}{3.302856in}}%
\pgfpathlineto{\pgfqpoint{4.717032in}{3.297312in}}%
\pgfpathlineto{\pgfqpoint{4.718105in}{3.302615in}}%
\pgfpathlineto{\pgfqpoint{4.719179in}{3.302012in}}%
\pgfpathlineto{\pgfqpoint{4.722401in}{3.303820in}}%
\pgfpathlineto{\pgfqpoint{4.723475in}{3.307436in}}%
\pgfpathlineto{\pgfqpoint{4.724549in}{3.302735in}}%
\pgfpathlineto{\pgfqpoint{4.725622in}{3.294660in}}%
\pgfpathlineto{\pgfqpoint{4.726696in}{3.294781in}}%
\pgfpathlineto{\pgfqpoint{4.729918in}{3.288513in}}%
\pgfpathlineto{\pgfqpoint{4.730992in}{3.291527in}}%
\pgfpathlineto{\pgfqpoint{4.732065in}{3.292852in}}%
\pgfpathlineto{\pgfqpoint{4.733139in}{3.293214in}}%
\pgfpathlineto{\pgfqpoint{4.734213in}{3.295022in}}%
\pgfpathlineto{\pgfqpoint{4.738508in}{3.289960in}}%
\pgfpathlineto{\pgfqpoint{4.739582in}{3.290321in}}%
\pgfpathlineto{\pgfqpoint{4.741730in}{3.293334in}}%
\pgfpathlineto{\pgfqpoint{4.744951in}{3.284657in}}%
\pgfpathlineto{\pgfqpoint{4.746025in}{3.284175in}}%
\pgfpathlineto{\pgfqpoint{4.747099in}{3.285259in}}%
\pgfpathlineto{\pgfqpoint{4.748173in}{3.292611in}}%
\pgfpathlineto{\pgfqpoint{4.749247in}{3.291768in}}%
\pgfpathlineto{\pgfqpoint{4.752468in}{3.292129in}}%
\pgfpathlineto{\pgfqpoint{4.753542in}{3.295263in}}%
\pgfpathlineto{\pgfqpoint{4.754616in}{3.294299in}}%
\pgfpathlineto{\pgfqpoint{4.755690in}{3.285139in}}%
\pgfpathlineto{\pgfqpoint{4.756764in}{3.283090in}}%
\pgfpathlineto{\pgfqpoint{4.761059in}{3.280197in}}%
\pgfpathlineto{\pgfqpoint{4.764281in}{3.285380in}}%
\pgfpathlineto{\pgfqpoint{4.767502in}{3.285139in}}%
\pgfpathlineto{\pgfqpoint{4.768576in}{3.286585in}}%
\pgfpathlineto{\pgfqpoint{4.769650in}{3.286103in}}%
\pgfpathlineto{\pgfqpoint{4.770724in}{3.284777in}}%
\pgfpathlineto{\pgfqpoint{4.771797in}{3.284898in}}%
\pgfpathlineto{\pgfqpoint{4.775019in}{3.284175in}}%
\pgfpathlineto{\pgfqpoint{4.777167in}{3.281885in}}%
\pgfpathlineto{\pgfqpoint{4.778240in}{3.279354in}}%
\pgfpathlineto{\pgfqpoint{4.782536in}{3.281282in}}%
\pgfpathlineto{\pgfqpoint{4.783610in}{3.280077in}}%
\pgfpathlineto{\pgfqpoint{4.784683in}{3.280679in}}%
\pgfpathlineto{\pgfqpoint{4.785757in}{3.275135in}}%
\pgfpathlineto{\pgfqpoint{4.786831in}{3.273930in}}%
\pgfpathlineto{\pgfqpoint{4.790053in}{3.277666in}}%
\pgfpathlineto{\pgfqpoint{4.791126in}{3.280077in}}%
\pgfpathlineto{\pgfqpoint{4.792200in}{3.276582in}}%
\pgfpathlineto{\pgfqpoint{4.793274in}{3.277425in}}%
\pgfpathlineto{\pgfqpoint{4.794348in}{3.279354in}}%
\pgfpathlineto{\pgfqpoint{4.798643in}{3.276823in}}%
\pgfpathlineto{\pgfqpoint{4.799717in}{3.277787in}}%
\pgfpathlineto{\pgfqpoint{4.800791in}{3.275497in}}%
\pgfpathlineto{\pgfqpoint{4.801865in}{3.274653in}}%
\pgfpathlineto{\pgfqpoint{4.806160in}{3.275617in}}%
\pgfpathlineto{\pgfqpoint{4.807234in}{3.281282in}}%
\pgfpathlineto{\pgfqpoint{4.808308in}{3.280920in}}%
\pgfpathlineto{\pgfqpoint{4.813677in}{3.284777in}}%
\pgfpathlineto{\pgfqpoint{4.815825in}{3.281282in}}%
\pgfpathlineto{\pgfqpoint{4.816899in}{3.289478in}}%
\pgfpathlineto{\pgfqpoint{4.820120in}{3.288875in}}%
\pgfpathlineto{\pgfqpoint{4.821194in}{3.292491in}}%
\pgfpathlineto{\pgfqpoint{4.822268in}{3.294058in}}%
\pgfpathlineto{\pgfqpoint{4.823342in}{3.294178in}}%
\pgfpathlineto{\pgfqpoint{4.824415in}{3.292732in}}%
\pgfpathlineto{\pgfqpoint{4.827637in}{3.291286in}}%
\pgfpathlineto{\pgfqpoint{4.828711in}{3.292009in}}%
\pgfpathlineto{\pgfqpoint{4.829785in}{3.292009in}}%
\pgfpathlineto{\pgfqpoint{4.831932in}{3.284777in}}%
\pgfpathlineto{\pgfqpoint{4.835154in}{3.284175in}}%
\pgfpathlineto{\pgfqpoint{4.836228in}{3.283210in}}%
\pgfpathlineto{\pgfqpoint{4.837302in}{3.283813in}}%
\pgfpathlineto{\pgfqpoint{4.838375in}{3.280920in}}%
\pgfpathlineto{\pgfqpoint{4.839449in}{3.283090in}}%
\pgfpathlineto{\pgfqpoint{4.842671in}{3.283210in}}%
\pgfpathlineto{\pgfqpoint{4.843745in}{3.279715in}}%
\pgfpathlineto{\pgfqpoint{4.844818in}{3.280800in}}%
\pgfpathlineto{\pgfqpoint{4.845892in}{3.285982in}}%
\pgfpathlineto{\pgfqpoint{4.846966in}{3.287549in}}%
\pgfpathlineto{\pgfqpoint{4.850188in}{3.289478in}}%
\pgfpathlineto{\pgfqpoint{4.851261in}{3.288875in}}%
\pgfpathlineto{\pgfqpoint{4.853409in}{3.294299in}}%
\pgfpathlineto{\pgfqpoint{4.854483in}{3.293937in}}%
\pgfpathlineto{\pgfqpoint{4.857704in}{3.296468in}}%
\pgfpathlineto{\pgfqpoint{4.858778in}{3.295745in}}%
\pgfpathlineto{\pgfqpoint{4.859852in}{3.292491in}}%
\pgfpathlineto{\pgfqpoint{4.860926in}{3.292852in}}%
\pgfpathlineto{\pgfqpoint{4.862000in}{3.297553in}}%
\pgfpathlineto{\pgfqpoint{4.865221in}{3.298758in}}%
\pgfpathlineto{\pgfqpoint{4.866295in}{3.300325in}}%
\pgfpathlineto{\pgfqpoint{4.868443in}{3.299843in}}%
\pgfpathlineto{\pgfqpoint{4.869517in}{3.298396in}}%
\pgfpathlineto{\pgfqpoint{4.873812in}{3.297673in}}%
\pgfpathlineto{\pgfqpoint{4.874886in}{3.299963in}}%
\pgfpathlineto{\pgfqpoint{4.875960in}{3.298035in}}%
\pgfpathlineto{\pgfqpoint{4.880255in}{3.299120in}}%
\pgfpathlineto{\pgfqpoint{4.882403in}{3.303458in}}%
\pgfpathlineto{\pgfqpoint{4.883477in}{3.301409in}}%
\pgfpathlineto{\pgfqpoint{4.884550in}{3.302735in}}%
\pgfpathlineto{\pgfqpoint{4.887772in}{3.303458in}}%
\pgfpathlineto{\pgfqpoint{4.888846in}{3.304543in}}%
\pgfpathlineto{\pgfqpoint{4.889920in}{3.303097in}}%
\pgfpathlineto{\pgfqpoint{4.890993in}{3.309967in}}%
\pgfpathlineto{\pgfqpoint{4.892067in}{3.295986in}}%
\pgfpathlineto{\pgfqpoint{4.896363in}{3.293817in}}%
\pgfpathlineto{\pgfqpoint{4.897437in}{3.296348in}}%
\pgfpathlineto{\pgfqpoint{4.902806in}{3.294299in}}%
\pgfpathlineto{\pgfqpoint{4.903880in}{3.292732in}}%
\pgfpathlineto{\pgfqpoint{4.904953in}{3.294660in}}%
\pgfpathlineto{\pgfqpoint{4.907101in}{3.305748in}}%
\pgfpathlineto{\pgfqpoint{4.910323in}{3.306833in}}%
\pgfpathlineto{\pgfqpoint{4.911396in}{3.305748in}}%
\pgfpathlineto{\pgfqpoint{4.912470in}{3.301651in}}%
\pgfpathlineto{\pgfqpoint{4.913544in}{3.302133in}}%
\pgfpathlineto{\pgfqpoint{4.914618in}{3.300325in}}%
\pgfpathlineto{\pgfqpoint{4.917839in}{3.302012in}}%
\pgfpathlineto{\pgfqpoint{4.918913in}{3.304061in}}%
\pgfpathlineto{\pgfqpoint{4.919987in}{3.309123in}}%
\pgfpathlineto{\pgfqpoint{4.921061in}{3.309002in}}%
\pgfpathlineto{\pgfqpoint{4.922135in}{3.308159in}}%
\pgfpathlineto{\pgfqpoint{4.926430in}{3.312016in}}%
\pgfpathlineto{\pgfqpoint{4.928578in}{3.314788in}}%
\pgfpathlineto{\pgfqpoint{4.929652in}{3.313703in}}%
\pgfpathlineto{\pgfqpoint{4.933947in}{3.317439in}}%
\pgfpathlineto{\pgfqpoint{4.936095in}{3.315631in}}%
\pgfpathlineto{\pgfqpoint{4.937169in}{3.315511in}}%
\pgfpathlineto{\pgfqpoint{4.940390in}{3.312257in}}%
\pgfpathlineto{\pgfqpoint{4.941464in}{3.313582in}}%
\pgfpathlineto{\pgfqpoint{4.943612in}{3.310810in}}%
\pgfpathlineto{\pgfqpoint{4.944685in}{3.312980in}}%
\pgfpathlineto{\pgfqpoint{4.947907in}{3.312739in}}%
\pgfpathlineto{\pgfqpoint{4.948981in}{3.322983in}}%
\pgfpathlineto{\pgfqpoint{4.950055in}{3.324671in}}%
\pgfpathlineto{\pgfqpoint{4.951128in}{3.323345in}}%
\pgfpathlineto{\pgfqpoint{4.952202in}{3.316234in}}%
\pgfpathlineto{\pgfqpoint{4.956498in}{3.313582in}}%
\pgfpathlineto{\pgfqpoint{4.959719in}{3.305628in}}%
\pgfpathlineto{\pgfqpoint{4.962941in}{3.304784in}}%
\pgfpathlineto{\pgfqpoint{4.964014in}{3.306592in}}%
\pgfpathlineto{\pgfqpoint{4.965088in}{3.304302in}}%
\pgfpathlineto{\pgfqpoint{4.966162in}{3.304905in}}%
\pgfpathlineto{\pgfqpoint{4.971531in}{3.295504in}}%
\pgfpathlineto{\pgfqpoint{4.972605in}{3.294781in}}%
\pgfpathlineto{\pgfqpoint{4.974753in}{3.301892in}}%
\pgfpathlineto{\pgfqpoint{4.977974in}{3.302735in}}%
\pgfpathlineto{\pgfqpoint{4.980122in}{3.305387in}}%
\pgfpathlineto{\pgfqpoint{4.982270in}{3.307556in}}%
\pgfpathlineto{\pgfqpoint{4.986565in}{3.307556in}}%
\pgfpathlineto{\pgfqpoint{4.987639in}{3.304905in}}%
\pgfpathlineto{\pgfqpoint{4.988713in}{3.307315in}}%
\pgfpathlineto{\pgfqpoint{4.989787in}{3.304061in}}%
\pgfpathlineto{\pgfqpoint{4.993008in}{3.306230in}}%
\pgfpathlineto{\pgfqpoint{4.994082in}{3.306110in}}%
\pgfpathlineto{\pgfqpoint{4.995156in}{3.307918in}}%
\pgfpathlineto{\pgfqpoint{4.997303in}{3.307556in}}%
\pgfpathlineto{\pgfqpoint{5.000525in}{3.304905in}}%
\pgfpathlineto{\pgfqpoint{5.001599in}{3.305989in}}%
\pgfpathlineto{\pgfqpoint{5.002673in}{3.304905in}}%
\pgfpathlineto{\pgfqpoint{5.003747in}{3.306230in}}%
\pgfpathlineto{\pgfqpoint{5.004820in}{3.306351in}}%
\pgfpathlineto{\pgfqpoint{5.008042in}{3.308400in}}%
\pgfpathlineto{\pgfqpoint{5.009116in}{3.307195in}}%
\pgfpathlineto{\pgfqpoint{5.010190in}{3.307556in}}%
\pgfpathlineto{\pgfqpoint{5.011263in}{3.305628in}}%
\pgfpathlineto{\pgfqpoint{5.012337in}{3.304905in}}%
\pgfpathlineto{\pgfqpoint{5.016633in}{3.308159in}}%
\pgfpathlineto{\pgfqpoint{5.017706in}{3.310208in}}%
\pgfpathlineto{\pgfqpoint{5.018780in}{3.314426in}}%
\pgfpathlineto{\pgfqpoint{5.019854in}{3.311775in}}%
\pgfpathlineto{\pgfqpoint{5.023076in}{3.311292in}}%
\pgfpathlineto{\pgfqpoint{5.024149in}{3.314788in}}%
\pgfpathlineto{\pgfqpoint{5.026297in}{3.313221in}}%
\pgfpathlineto{\pgfqpoint{5.027371in}{3.316475in}}%
\pgfpathlineto{\pgfqpoint{5.030592in}{3.318403in}}%
\pgfpathlineto{\pgfqpoint{5.031666in}{3.320211in}}%
\pgfpathlineto{\pgfqpoint{5.032740in}{3.319247in}}%
\pgfpathlineto{\pgfqpoint{5.033814in}{3.324550in}}%
\pgfpathlineto{\pgfqpoint{5.034888in}{3.322019in}}%
\pgfpathlineto{\pgfqpoint{5.038109in}{3.325394in}}%
\pgfpathlineto{\pgfqpoint{5.039183in}{3.329733in}}%
\pgfpathlineto{\pgfqpoint{5.040257in}{3.330456in}}%
\pgfpathlineto{\pgfqpoint{5.041331in}{3.324791in}}%
\pgfpathlineto{\pgfqpoint{5.042405in}{3.322140in}}%
\pgfpathlineto{\pgfqpoint{5.046700in}{3.328286in}}%
\pgfpathlineto{\pgfqpoint{5.048848in}{3.331058in}}%
\pgfpathlineto{\pgfqpoint{5.053143in}{3.330094in}}%
\pgfpathlineto{\pgfqpoint{5.055291in}{3.328045in}}%
\pgfpathlineto{\pgfqpoint{5.056365in}{3.328045in}}%
\pgfpathlineto{\pgfqpoint{5.057438in}{3.322983in}}%
\pgfpathlineto{\pgfqpoint{5.060660in}{3.323706in}}%
\pgfpathlineto{\pgfqpoint{5.062808in}{3.327443in}}%
\pgfpathlineto{\pgfqpoint{5.063881in}{3.325032in}}%
\pgfpathlineto{\pgfqpoint{5.064955in}{3.324550in}}%
\pgfpathlineto{\pgfqpoint{5.069251in}{3.324671in}}%
\pgfpathlineto{\pgfqpoint{5.070325in}{3.326478in}}%
\pgfpathlineto{\pgfqpoint{5.072472in}{3.325755in}}%
\pgfpathlineto{\pgfqpoint{5.075694in}{3.326478in}}%
\pgfpathlineto{\pgfqpoint{5.076768in}{3.327322in}}%
\pgfpathlineto{\pgfqpoint{5.077841in}{3.326117in}}%
\pgfpathlineto{\pgfqpoint{5.079989in}{3.320814in}}%
\pgfpathlineto{\pgfqpoint{5.083211in}{3.323104in}}%
\pgfpathlineto{\pgfqpoint{5.084284in}{3.323104in}}%
\pgfpathlineto{\pgfqpoint{5.085358in}{3.324671in}}%
\pgfpathlineto{\pgfqpoint{5.086432in}{3.324188in}}%
\pgfpathlineto{\pgfqpoint{5.087506in}{3.325635in}}%
\pgfpathlineto{\pgfqpoint{5.093949in}{3.332625in}}%
\pgfpathlineto{\pgfqpoint{5.095023in}{3.336361in}}%
\pgfpathlineto{\pgfqpoint{5.098244in}{3.335759in}}%
\pgfpathlineto{\pgfqpoint{5.099318in}{3.339857in}}%
\pgfpathlineto{\pgfqpoint{5.100392in}{3.339254in}}%
\pgfpathlineto{\pgfqpoint{5.101466in}{3.339857in}}%
\pgfpathlineto{\pgfqpoint{5.102540in}{3.345160in}}%
\pgfpathlineto{\pgfqpoint{5.105761in}{3.342508in}}%
\pgfpathlineto{\pgfqpoint{5.106835in}{3.345883in}}%
\pgfpathlineto{\pgfqpoint{5.107909in}{3.342749in}}%
\pgfpathlineto{\pgfqpoint{5.108983in}{3.342990in}}%
\pgfpathlineto{\pgfqpoint{5.110057in}{3.362877in}}%
\pgfpathlineto{\pgfqpoint{5.113278in}{3.364323in}}%
\pgfpathlineto{\pgfqpoint{5.115426in}{3.363479in}}%
\pgfpathlineto{\pgfqpoint{5.117573in}{3.366372in}}%
\pgfpathlineto{\pgfqpoint{5.120795in}{3.366613in}}%
\pgfpathlineto{\pgfqpoint{5.122943in}{3.373362in}}%
\pgfpathlineto{\pgfqpoint{5.124016in}{3.372398in}}%
\pgfpathlineto{\pgfqpoint{5.125090in}{3.373965in}}%
\pgfpathlineto{\pgfqpoint{5.128312in}{3.373603in}}%
\pgfpathlineto{\pgfqpoint{5.129386in}{3.374447in}}%
\pgfpathlineto{\pgfqpoint{5.130459in}{3.374447in}}%
\pgfpathlineto{\pgfqpoint{5.131533in}{3.376134in}}%
\pgfpathlineto{\pgfqpoint{5.135829in}{3.374929in}}%
\pgfpathlineto{\pgfqpoint{5.136903in}{3.372518in}}%
\pgfpathlineto{\pgfqpoint{5.137976in}{3.373483in}}%
\pgfpathlineto{\pgfqpoint{5.139050in}{3.377339in}}%
\pgfpathlineto{\pgfqpoint{5.140124in}{3.377219in}}%
\pgfpathlineto{\pgfqpoint{5.143346in}{3.379750in}}%
\pgfpathlineto{\pgfqpoint{5.144419in}{3.382040in}}%
\pgfpathlineto{\pgfqpoint{5.145493in}{3.412653in}}%
\pgfpathlineto{\pgfqpoint{5.146567in}{3.402770in}}%
\pgfpathlineto{\pgfqpoint{5.147641in}{3.402770in}}%
\pgfpathlineto{\pgfqpoint{5.150862in}{3.406506in}}%
\pgfpathlineto{\pgfqpoint{5.151936in}{3.414099in}}%
\pgfpathlineto{\pgfqpoint{5.154084in}{3.408434in}}%
\pgfpathlineto{\pgfqpoint{5.158379in}{3.408314in}}%
\pgfpathlineto{\pgfqpoint{5.159453in}{3.407832in}}%
\pgfpathlineto{\pgfqpoint{5.160527in}{3.409519in}}%
\pgfpathlineto{\pgfqpoint{5.161601in}{3.404698in}}%
\pgfpathlineto{\pgfqpoint{5.162675in}{3.403131in}}%
\pgfpathlineto{\pgfqpoint{5.165896in}{3.406386in}}%
\pgfpathlineto{\pgfqpoint{5.166970in}{3.395900in}}%
\pgfpathlineto{\pgfqpoint{5.168044in}{3.396141in}}%
\pgfpathlineto{\pgfqpoint{5.170191in}{3.393610in}}%
\pgfpathlineto{\pgfqpoint{5.174487in}{3.399154in}}%
\pgfpathlineto{\pgfqpoint{5.175561in}{3.409278in}}%
\pgfpathlineto{\pgfqpoint{5.176635in}{3.407591in}}%
\pgfpathlineto{\pgfqpoint{5.177708in}{3.410001in}}%
\pgfpathlineto{\pgfqpoint{5.180930in}{3.412532in}}%
\pgfpathlineto{\pgfqpoint{5.182004in}{3.411809in}}%
\pgfpathlineto{\pgfqpoint{5.183078in}{3.413496in}}%
\pgfpathlineto{\pgfqpoint{5.184151in}{3.420246in}}%
\pgfpathlineto{\pgfqpoint{5.185225in}{3.418076in}}%
\pgfpathlineto{\pgfqpoint{5.189521in}{3.416630in}}%
\pgfpathlineto{\pgfqpoint{5.190594in}{3.416510in}}%
\pgfpathlineto{\pgfqpoint{5.191668in}{3.415063in}}%
\pgfpathlineto{\pgfqpoint{5.192742in}{3.417835in}}%
\pgfpathlineto{\pgfqpoint{5.197037in}{3.414220in}}%
\pgfpathlineto{\pgfqpoint{5.198111in}{3.414220in}}%
\pgfpathlineto{\pgfqpoint{5.199185in}{3.417715in}}%
\pgfpathlineto{\pgfqpoint{5.203480in}{3.422174in}}%
\pgfpathlineto{\pgfqpoint{5.204554in}{3.417715in}}%
\pgfpathlineto{\pgfqpoint{5.205628in}{3.418920in}}%
\pgfpathlineto{\pgfqpoint{5.206702in}{3.418920in}}%
\pgfpathlineto{\pgfqpoint{5.207776in}{3.414702in}}%
\pgfpathlineto{\pgfqpoint{5.210997in}{3.413858in}}%
\pgfpathlineto{\pgfqpoint{5.212071in}{3.417956in}}%
\pgfpathlineto{\pgfqpoint{5.213145in}{3.418438in}}%
\pgfpathlineto{\pgfqpoint{5.214219in}{3.420487in}}%
\pgfpathlineto{\pgfqpoint{5.215293in}{3.416751in}}%
\pgfpathlineto{\pgfqpoint{5.218514in}{3.415666in}}%
\pgfpathlineto{\pgfqpoint{5.219588in}{3.412653in}}%
\pgfpathlineto{\pgfqpoint{5.220662in}{3.416027in}}%
\pgfpathlineto{\pgfqpoint{5.221736in}{3.409640in}}%
\pgfpathlineto{\pgfqpoint{5.222810in}{3.410965in}}%
\pgfpathlineto{\pgfqpoint{5.226031in}{3.417594in}}%
\pgfpathlineto{\pgfqpoint{5.227105in}{3.416751in}}%
\pgfpathlineto{\pgfqpoint{5.229253in}{3.403131in}}%
\pgfpathlineto{\pgfqpoint{5.230326in}{3.408555in}}%
\pgfpathlineto{\pgfqpoint{5.233548in}{3.409399in}}%
\pgfpathlineto{\pgfqpoint{5.234622in}{3.402649in}}%
\pgfpathlineto{\pgfqpoint{5.235696in}{3.411086in}}%
\pgfpathlineto{\pgfqpoint{5.236769in}{3.404216in}}%
\pgfpathlineto{\pgfqpoint{5.237843in}{3.386379in}}%
\pgfpathlineto{\pgfqpoint{5.241065in}{3.381678in}}%
\pgfpathlineto{\pgfqpoint{5.242139in}{3.388789in}}%
\pgfpathlineto{\pgfqpoint{5.244286in}{3.375170in}}%
\pgfpathlineto{\pgfqpoint{5.245360in}{3.380835in}}%
\pgfpathlineto{\pgfqpoint{5.248582in}{3.382883in}}%
\pgfpathlineto{\pgfqpoint{5.249656in}{3.393610in}}%
\pgfpathlineto{\pgfqpoint{5.250729in}{3.390115in}}%
\pgfpathlineto{\pgfqpoint{5.252877in}{3.399636in}}%
\pgfpathlineto{\pgfqpoint{5.256099in}{3.399877in}}%
\pgfpathlineto{\pgfqpoint{5.257172in}{3.405542in}}%
\pgfpathlineto{\pgfqpoint{5.258246in}{3.407350in}}%
\pgfpathlineto{\pgfqpoint{5.259320in}{3.393369in}}%
\pgfpathlineto{\pgfqpoint{5.260394in}{3.408314in}}%
\pgfpathlineto{\pgfqpoint{5.263615in}{3.411448in}}%
\pgfpathlineto{\pgfqpoint{5.264689in}{3.413858in}}%
\pgfpathlineto{\pgfqpoint{5.265763in}{3.408073in}}%
\pgfpathlineto{\pgfqpoint{5.266837in}{3.408675in}}%
\pgfpathlineto{\pgfqpoint{5.267911in}{3.406145in}}%
\pgfpathlineto{\pgfqpoint{5.271132in}{3.402770in}}%
\pgfpathlineto{\pgfqpoint{5.273280in}{3.403975in}}%
\pgfpathlineto{\pgfqpoint{5.275428in}{3.410001in}}%
\pgfpathlineto{\pgfqpoint{5.278649in}{3.413135in}}%
\pgfpathlineto{\pgfqpoint{5.279723in}{3.418076in}}%
\pgfpathlineto{\pgfqpoint{5.280797in}{3.414220in}}%
\pgfpathlineto{\pgfqpoint{5.281871in}{3.431093in}}%
\pgfpathlineto{\pgfqpoint{5.282945in}{3.427357in}}%
\pgfpathlineto{\pgfqpoint{5.286166in}{3.434227in}}%
\pgfpathlineto{\pgfqpoint{5.287240in}{3.434950in}}%
\pgfpathlineto{\pgfqpoint{5.288314in}{3.441096in}}%
\pgfpathlineto{\pgfqpoint{5.290461in}{3.444833in}}%
\pgfpathlineto{\pgfqpoint{5.293683in}{3.443989in}}%
\pgfpathlineto{\pgfqpoint{5.294757in}{3.448448in}}%
\pgfpathlineto{\pgfqpoint{5.295831in}{3.446761in}}%
\pgfpathlineto{\pgfqpoint{5.296904in}{3.447002in}}%
\pgfpathlineto{\pgfqpoint{5.297978in}{3.449292in}}%
\pgfpathlineto{\pgfqpoint{5.301200in}{3.444230in}}%
\pgfpathlineto{\pgfqpoint{5.303347in}{3.436034in}}%
\pgfpathlineto{\pgfqpoint{5.304421in}{3.439047in}}%
\pgfpathlineto{\pgfqpoint{5.305495in}{3.434106in}}%
\pgfpathlineto{\pgfqpoint{5.308717in}{3.430731in}}%
\pgfpathlineto{\pgfqpoint{5.309791in}{3.426995in}}%
\pgfpathlineto{\pgfqpoint{5.311938in}{3.442422in}}%
\pgfpathlineto{\pgfqpoint{5.313012in}{3.435552in}}%
\pgfpathlineto{\pgfqpoint{5.317307in}{3.446761in}}%
\pgfpathlineto{\pgfqpoint{5.318381in}{3.446761in}}%
\pgfpathlineto{\pgfqpoint{5.320529in}{3.447966in}}%
\pgfpathlineto{\pgfqpoint{5.323750in}{3.444109in}}%
\pgfpathlineto{\pgfqpoint{5.325898in}{3.434709in}}%
\pgfpathlineto{\pgfqpoint{5.328046in}{3.435432in}}%
\pgfpathlineto{\pgfqpoint{5.331267in}{3.431093in}}%
\pgfpathlineto{\pgfqpoint{5.332341in}{3.423982in}}%
\pgfpathlineto{\pgfqpoint{5.334489in}{3.438927in}}%
\pgfpathlineto{\pgfqpoint{5.335563in}{3.439650in}}%
\pgfpathlineto{\pgfqpoint{5.339858in}{3.436878in}}%
\pgfpathlineto{\pgfqpoint{5.342006in}{3.433624in}}%
\pgfpathlineto{\pgfqpoint{5.343079in}{3.436396in}}%
\pgfpathlineto{\pgfqpoint{5.347375in}{3.432539in}}%
\pgfpathlineto{\pgfqpoint{5.349523in}{3.441217in}}%
\pgfpathlineto{\pgfqpoint{5.350596in}{3.436396in}}%
\pgfpathlineto{\pgfqpoint{5.353818in}{3.429647in}}%
\pgfpathlineto{\pgfqpoint{5.354892in}{3.412532in}}%
\pgfpathlineto{\pgfqpoint{5.355966in}{3.408193in}}%
\pgfpathlineto{\pgfqpoint{5.357039in}{3.412773in}}%
\pgfpathlineto{\pgfqpoint{5.358113in}{3.400480in}}%
\pgfpathlineto{\pgfqpoint{5.361335in}{3.406868in}}%
\pgfpathlineto{\pgfqpoint{5.362409in}{3.407350in}}%
\pgfpathlineto{\pgfqpoint{5.363482in}{3.408917in}}%
\pgfpathlineto{\pgfqpoint{5.364556in}{3.412532in}}%
\pgfpathlineto{\pgfqpoint{5.365630in}{3.405662in}}%
\pgfpathlineto{\pgfqpoint{5.368852in}{3.401806in}}%
\pgfpathlineto{\pgfqpoint{5.369925in}{3.409881in}}%
\pgfpathlineto{\pgfqpoint{5.370999in}{3.408314in}}%
\pgfpathlineto{\pgfqpoint{5.372073in}{3.414461in}}%
\pgfpathlineto{\pgfqpoint{5.373147in}{3.416992in}}%
\pgfpathlineto{\pgfqpoint{5.377442in}{3.420969in}}%
\pgfpathlineto{\pgfqpoint{5.378516in}{3.416027in}}%
\pgfpathlineto{\pgfqpoint{5.379590in}{3.415304in}}%
\pgfpathlineto{\pgfqpoint{5.380664in}{3.417474in}}%
\pgfpathlineto{\pgfqpoint{5.383885in}{3.410604in}}%
\pgfpathlineto{\pgfqpoint{5.384959in}{3.417474in}}%
\pgfpathlineto{\pgfqpoint{5.388181in}{3.405180in}}%
\pgfpathlineto{\pgfqpoint{5.391402in}{3.413737in}}%
\pgfpathlineto{\pgfqpoint{5.393550in}{3.414340in}}%
\pgfpathlineto{\pgfqpoint{5.395698in}{3.404578in}}%
\pgfpathlineto{\pgfqpoint{5.398919in}{3.399516in}}%
\pgfpathlineto{\pgfqpoint{5.399993in}{3.388669in}}%
\pgfpathlineto{\pgfqpoint{5.401067in}{3.395418in}}%
\pgfpathlineto{\pgfqpoint{5.402141in}{3.379147in}}%
\pgfpathlineto{\pgfqpoint{5.403214in}{3.380473in}}%
\pgfpathlineto{\pgfqpoint{5.406436in}{3.379509in}}%
\pgfpathlineto{\pgfqpoint{5.407510in}{3.376978in}}%
\pgfpathlineto{\pgfqpoint{5.408584in}{3.380111in}}%
\pgfpathlineto{\pgfqpoint{5.409657in}{3.378545in}}%
\pgfpathlineto{\pgfqpoint{5.410731in}{3.384571in}}%
\pgfpathlineto{\pgfqpoint{5.413953in}{3.383366in}}%
\pgfpathlineto{\pgfqpoint{5.415027in}{3.379027in}}%
\pgfpathlineto{\pgfqpoint{5.416101in}{3.369505in}}%
\pgfpathlineto{\pgfqpoint{5.417174in}{3.371554in}}%
\pgfpathlineto{\pgfqpoint{5.418248in}{3.391923in}}%
\pgfpathlineto{\pgfqpoint{5.422544in}{3.384089in}}%
\pgfpathlineto{\pgfqpoint{5.423617in}{3.379268in}}%
\pgfpathlineto{\pgfqpoint{5.424691in}{3.379268in}}%
\pgfpathlineto{\pgfqpoint{5.428987in}{3.381678in}}%
\pgfpathlineto{\pgfqpoint{5.430060in}{3.384089in}}%
\pgfpathlineto{\pgfqpoint{5.431134in}{3.384571in}}%
\pgfpathlineto{\pgfqpoint{5.432208in}{3.383848in}}%
\pgfpathlineto{\pgfqpoint{5.433282in}{3.391200in}}%
\pgfpathlineto{\pgfqpoint{5.436503in}{3.389030in}}%
\pgfpathlineto{\pgfqpoint{5.437577in}{3.386499in}}%
\pgfpathlineto{\pgfqpoint{5.438651in}{3.400721in}}%
\pgfpathlineto{\pgfqpoint{5.439725in}{3.401083in}}%
\pgfpathlineto{\pgfqpoint{5.440799in}{3.396864in}}%
\pgfpathlineto{\pgfqpoint{5.444020in}{3.399636in}}%
\pgfpathlineto{\pgfqpoint{5.445094in}{3.396503in}}%
\pgfpathlineto{\pgfqpoint{5.446168in}{3.399275in}}%
\pgfpathlineto{\pgfqpoint{5.448316in}{3.392766in}}%
\pgfpathlineto{\pgfqpoint{5.451537in}{3.397226in}}%
\pgfpathlineto{\pgfqpoint{5.452611in}{3.402770in}}%
\pgfpathlineto{\pgfqpoint{5.453685in}{3.401324in}}%
\pgfpathlineto{\pgfqpoint{5.454759in}{3.397708in}}%
\pgfpathlineto{\pgfqpoint{5.455833in}{3.406988in}}%
\pgfpathlineto{\pgfqpoint{5.459054in}{3.407109in}}%
\pgfpathlineto{\pgfqpoint{5.461202in}{3.396744in}}%
\pgfpathlineto{\pgfqpoint{5.462276in}{3.396864in}}%
\pgfpathlineto{\pgfqpoint{5.463349in}{3.402890in}}%
\pgfpathlineto{\pgfqpoint{5.466571in}{3.401685in}}%
\pgfpathlineto{\pgfqpoint{5.467645in}{3.396985in}}%
\pgfpathlineto{\pgfqpoint{5.469792in}{3.404698in}}%
\pgfpathlineto{\pgfqpoint{5.470866in}{3.404939in}}%
\pgfpathlineto{\pgfqpoint{5.474088in}{3.409399in}}%
\pgfpathlineto{\pgfqpoint{5.475162in}{3.406627in}}%
\pgfpathlineto{\pgfqpoint{5.477309in}{3.410965in}}%
\pgfpathlineto{\pgfqpoint{5.482679in}{3.406145in}}%
\pgfpathlineto{\pgfqpoint{5.484826in}{3.415786in}}%
\pgfpathlineto{\pgfqpoint{5.485900in}{3.420607in}}%
\pgfpathlineto{\pgfqpoint{5.489122in}{3.414822in}}%
\pgfpathlineto{\pgfqpoint{5.492343in}{3.397708in}}%
\pgfpathlineto{\pgfqpoint{5.493417in}{3.392646in}}%
\pgfpathlineto{\pgfqpoint{5.496638in}{3.386861in}}%
\pgfpathlineto{\pgfqpoint{5.497712in}{3.386379in}}%
\pgfpathlineto{\pgfqpoint{5.498786in}{3.392405in}}%
\pgfpathlineto{\pgfqpoint{5.499860in}{3.392766in}}%
\pgfpathlineto{\pgfqpoint{5.500934in}{3.387102in}}%
\pgfpathlineto{\pgfqpoint{5.504155in}{3.387825in}}%
\pgfpathlineto{\pgfqpoint{5.506303in}{3.393851in}}%
\pgfpathlineto{\pgfqpoint{5.507377in}{3.398431in}}%
\pgfpathlineto{\pgfqpoint{5.508451in}{3.395056in}}%
\pgfpathlineto{\pgfqpoint{5.511672in}{3.397105in}}%
\pgfpathlineto{\pgfqpoint{5.513820in}{3.393369in}}%
\pgfpathlineto{\pgfqpoint{5.514894in}{3.394213in}}%
\pgfpathlineto{\pgfqpoint{5.515967in}{3.383848in}}%
\pgfpathlineto{\pgfqpoint{5.519189in}{3.377098in}}%
\pgfpathlineto{\pgfqpoint{5.520263in}{3.377460in}}%
\pgfpathlineto{\pgfqpoint{5.521337in}{3.374929in}}%
\pgfpathlineto{\pgfqpoint{5.522411in}{3.378906in}}%
\pgfpathlineto{\pgfqpoint{5.527780in}{3.371916in}}%
\pgfpathlineto{\pgfqpoint{5.529927in}{3.361551in}}%
\pgfpathlineto{\pgfqpoint{5.531001in}{3.364082in}}%
\pgfpathlineto{\pgfqpoint{5.534223in}{3.370108in}}%
\pgfpathlineto{\pgfqpoint{5.535297in}{3.369264in}}%
\pgfpathlineto{\pgfqpoint{5.536370in}{3.369626in}}%
\pgfpathlineto{\pgfqpoint{5.537444in}{3.371916in}}%
\pgfpathlineto{\pgfqpoint{5.538518in}{3.367336in}}%
\pgfpathlineto{\pgfqpoint{5.541740in}{3.363359in}}%
\pgfpathlineto{\pgfqpoint{5.542813in}{3.359261in}}%
\pgfpathlineto{\pgfqpoint{5.543887in}{3.358056in}}%
\pgfpathlineto{\pgfqpoint{5.544961in}{3.357935in}}%
\pgfpathlineto{\pgfqpoint{5.546035in}{3.352270in}}%
\pgfpathlineto{\pgfqpoint{5.549256in}{3.355284in}}%
\pgfpathlineto{\pgfqpoint{5.550330in}{3.361792in}}%
\pgfpathlineto{\pgfqpoint{5.551404in}{3.362394in}}%
\pgfpathlineto{\pgfqpoint{5.552478in}{3.361310in}}%
\pgfpathlineto{\pgfqpoint{5.557847in}{3.363600in}}%
\pgfpathlineto{\pgfqpoint{5.558921in}{3.366131in}}%
\pgfpathlineto{\pgfqpoint{5.561069in}{3.363600in}}%
\pgfpathlineto{\pgfqpoint{5.564290in}{3.371795in}}%
\pgfpathlineto{\pgfqpoint{5.565364in}{3.364564in}}%
\pgfpathlineto{\pgfqpoint{5.566438in}{3.369746in}}%
\pgfpathlineto{\pgfqpoint{5.567512in}{3.363479in}}%
\pgfpathlineto{\pgfqpoint{5.568586in}{3.365046in}}%
\pgfpathlineto{\pgfqpoint{5.571807in}{3.365649in}}%
\pgfpathlineto{\pgfqpoint{5.572881in}{3.363841in}}%
\pgfpathlineto{\pgfqpoint{5.575029in}{3.349016in}}%
\pgfpathlineto{\pgfqpoint{5.576102in}{3.338531in}}%
\pgfpathlineto{\pgfqpoint{5.579324in}{3.335156in}}%
\pgfpathlineto{\pgfqpoint{5.580398in}{3.331058in}}%
\pgfpathlineto{\pgfqpoint{5.581472in}{3.346485in}}%
\pgfpathlineto{\pgfqpoint{5.582545in}{3.351065in}}%
\pgfpathlineto{\pgfqpoint{5.583619in}{3.358538in}}%
\pgfpathlineto{\pgfqpoint{5.586841in}{3.359863in}}%
\pgfpathlineto{\pgfqpoint{5.587915in}{3.352150in}}%
\pgfpathlineto{\pgfqpoint{5.590062in}{3.365649in}}%
\pgfpathlineto{\pgfqpoint{5.591136in}{3.359622in}}%
\pgfpathlineto{\pgfqpoint{5.595432in}{3.370228in}}%
\pgfpathlineto{\pgfqpoint{5.596505in}{3.367456in}}%
\pgfpathlineto{\pgfqpoint{5.597579in}{3.367697in}}%
\pgfpathlineto{\pgfqpoint{5.598653in}{3.369867in}}%
\pgfpathlineto{\pgfqpoint{5.601875in}{3.369023in}}%
\pgfpathlineto{\pgfqpoint{5.602948in}{3.372759in}}%
\pgfpathlineto{\pgfqpoint{5.604022in}{3.373121in}}%
\pgfpathlineto{\pgfqpoint{5.605096in}{3.372518in}}%
\pgfpathlineto{\pgfqpoint{5.606170in}{3.365046in}}%
\pgfpathlineto{\pgfqpoint{5.609391in}{3.366613in}}%
\pgfpathlineto{\pgfqpoint{5.610465in}{3.361310in}}%
\pgfpathlineto{\pgfqpoint{5.611539in}{3.362033in}}%
\pgfpathlineto{\pgfqpoint{5.612613in}{3.359261in}}%
\pgfpathlineto{\pgfqpoint{5.613687in}{3.362756in}}%
\pgfpathlineto{\pgfqpoint{5.616908in}{3.362274in}}%
\pgfpathlineto{\pgfqpoint{5.617982in}{3.367456in}}%
\pgfpathlineto{\pgfqpoint{5.619056in}{3.377098in}}%
\pgfpathlineto{\pgfqpoint{5.620130in}{3.375652in}}%
\pgfpathlineto{\pgfqpoint{5.621204in}{3.381076in}}%
\pgfpathlineto{\pgfqpoint{5.624425in}{3.388669in}}%
\pgfpathlineto{\pgfqpoint{5.627647in}{3.402770in}}%
\pgfpathlineto{\pgfqpoint{5.628721in}{3.398672in}}%
\pgfpathlineto{\pgfqpoint{5.631942in}{3.399395in}}%
\pgfpathlineto{\pgfqpoint{5.633016in}{3.397587in}}%
\pgfpathlineto{\pgfqpoint{5.634090in}{3.405783in}}%
\pgfpathlineto{\pgfqpoint{5.635164in}{3.405301in}}%
\pgfpathlineto{\pgfqpoint{5.636237in}{3.408434in}}%
\pgfpathlineto{\pgfqpoint{5.639459in}{3.414340in}}%
\pgfpathlineto{\pgfqpoint{5.641607in}{3.411930in}}%
\pgfpathlineto{\pgfqpoint{5.643754in}{3.428441in}}%
\pgfpathlineto{\pgfqpoint{5.648050in}{3.423861in}}%
\pgfpathlineto{\pgfqpoint{5.649123in}{3.426392in}}%
\pgfpathlineto{\pgfqpoint{5.650197in}{3.419041in}}%
\pgfpathlineto{\pgfqpoint{5.651271in}{3.417233in}}%
\pgfpathlineto{\pgfqpoint{5.654493in}{3.419884in}}%
\pgfpathlineto{\pgfqpoint{5.656640in}{3.423018in}}%
\pgfpathlineto{\pgfqpoint{5.663083in}{3.412773in}}%
\pgfpathlineto{\pgfqpoint{5.666305in}{3.400841in}}%
\pgfpathlineto{\pgfqpoint{5.669526in}{3.400721in}}%
\pgfpathlineto{\pgfqpoint{5.671674in}{3.412171in}}%
\pgfpathlineto{\pgfqpoint{5.672748in}{3.424585in}}%
\pgfpathlineto{\pgfqpoint{5.673822in}{3.428441in}}%
\pgfpathlineto{\pgfqpoint{5.677043in}{3.426513in}}%
\pgfpathlineto{\pgfqpoint{5.678117in}{3.425187in}}%
\pgfpathlineto{\pgfqpoint{5.679191in}{3.426272in}}%
\pgfpathlineto{\pgfqpoint{5.681339in}{3.426272in}}%
\pgfpathlineto{\pgfqpoint{5.684560in}{3.429647in}}%
\pgfpathlineto{\pgfqpoint{5.685634in}{3.433142in}}%
\pgfpathlineto{\pgfqpoint{5.686708in}{3.430370in}}%
\pgfpathlineto{\pgfqpoint{5.687782in}{3.421813in}}%
\pgfpathlineto{\pgfqpoint{5.688856in}{3.431575in}}%
\pgfpathlineto{\pgfqpoint{5.692077in}{3.432057in}}%
\pgfpathlineto{\pgfqpoint{5.693151in}{3.429526in}}%
\pgfpathlineto{\pgfqpoint{5.694225in}{3.430129in}}%
\pgfpathlineto{\pgfqpoint{5.695299in}{3.429647in}}%
\pgfpathlineto{\pgfqpoint{5.696372in}{3.424223in}}%
\pgfpathlineto{\pgfqpoint{5.699594in}{3.426392in}}%
\pgfpathlineto{\pgfqpoint{5.700668in}{3.434106in}}%
\pgfpathlineto{\pgfqpoint{5.701742in}{3.435432in}}%
\pgfpathlineto{\pgfqpoint{5.702815in}{3.431213in}}%
\pgfpathlineto{\pgfqpoint{5.703889in}{3.419884in}}%
\pgfpathlineto{\pgfqpoint{5.707111in}{3.423982in}}%
\pgfpathlineto{\pgfqpoint{5.709258in}{3.432178in}}%
\pgfpathlineto{\pgfqpoint{5.714628in}{3.431454in}}%
\pgfpathlineto{\pgfqpoint{5.715701in}{3.436999in}}%
\pgfpathlineto{\pgfqpoint{5.717849in}{3.426272in}}%
\pgfpathlineto{\pgfqpoint{5.723218in}{3.419523in}}%
\pgfpathlineto{\pgfqpoint{5.724292in}{3.411327in}}%
\pgfpathlineto{\pgfqpoint{5.725366in}{3.397828in}}%
\pgfpathlineto{\pgfqpoint{5.726440in}{3.394333in}}%
\pgfpathlineto{\pgfqpoint{5.729661in}{3.400239in}}%
\pgfpathlineto{\pgfqpoint{5.730735in}{3.406988in}}%
\pgfpathlineto{\pgfqpoint{5.731809in}{3.398672in}}%
\pgfpathlineto{\pgfqpoint{5.732883in}{3.407591in}}%
\pgfpathlineto{\pgfqpoint{5.733957in}{3.375290in}}%
\pgfpathlineto{\pgfqpoint{5.738252in}{3.375773in}}%
\pgfpathlineto{\pgfqpoint{5.739326in}{3.373483in}}%
\pgfpathlineto{\pgfqpoint{5.740400in}{3.374206in}}%
\pgfpathlineto{\pgfqpoint{5.741474in}{3.377098in}}%
\pgfpathlineto{\pgfqpoint{5.744695in}{3.373603in}}%
\pgfpathlineto{\pgfqpoint{5.745769in}{3.377219in}}%
\pgfpathlineto{\pgfqpoint{5.746843in}{3.375893in}}%
\pgfpathlineto{\pgfqpoint{5.747917in}{3.377580in}}%
\pgfpathlineto{\pgfqpoint{5.748990in}{3.389030in}}%
\pgfpathlineto{\pgfqpoint{5.752212in}{3.386861in}}%
\pgfpathlineto{\pgfqpoint{5.753286in}{3.375773in}}%
\pgfpathlineto{\pgfqpoint{5.754360in}{3.373483in}}%
\pgfpathlineto{\pgfqpoint{5.755433in}{3.378304in}}%
\pgfpathlineto{\pgfqpoint{5.756507in}{3.370228in}}%
\pgfpathlineto{\pgfqpoint{5.760803in}{3.367697in}}%
\pgfpathlineto{\pgfqpoint{5.761877in}{3.361430in}}%
\pgfpathlineto{\pgfqpoint{5.762950in}{3.361310in}}%
\pgfpathlineto{\pgfqpoint{5.764024in}{3.365890in}}%
\pgfpathlineto{\pgfqpoint{5.768320in}{3.367456in}}%
\pgfpathlineto{\pgfqpoint{5.769393in}{3.374929in}}%
\pgfpathlineto{\pgfqpoint{5.770467in}{3.374447in}}%
\pgfpathlineto{\pgfqpoint{5.771541in}{3.366613in}}%
\pgfpathlineto{\pgfqpoint{5.774763in}{3.373603in}}%
\pgfpathlineto{\pgfqpoint{5.775836in}{3.367577in}}%
\pgfpathlineto{\pgfqpoint{5.779058in}{3.378545in}}%
\pgfpathlineto{\pgfqpoint{5.782279in}{3.376255in}}%
\pgfpathlineto{\pgfqpoint{5.783353in}{3.384812in}}%
\pgfpathlineto{\pgfqpoint{5.784427in}{3.386740in}}%
\pgfpathlineto{\pgfqpoint{5.786575in}{3.387704in}}%
\pgfpathlineto{\pgfqpoint{5.789796in}{3.391079in}}%
\pgfpathlineto{\pgfqpoint{5.790870in}{3.386861in}}%
\pgfpathlineto{\pgfqpoint{5.793018in}{3.394454in}}%
\pgfpathlineto{\pgfqpoint{5.794092in}{3.399998in}}%
\pgfpathlineto{\pgfqpoint{5.797313in}{3.396382in}}%
\pgfpathlineto{\pgfqpoint{5.798387in}{3.398793in}}%
\pgfpathlineto{\pgfqpoint{5.799461in}{3.399275in}}%
\pgfpathlineto{\pgfqpoint{5.800535in}{3.402408in}}%
\pgfpathlineto{\pgfqpoint{5.801609in}{3.410122in}}%
\pgfpathlineto{\pgfqpoint{5.804830in}{3.406386in}}%
\pgfpathlineto{\pgfqpoint{5.805904in}{3.406145in}}%
\pgfpathlineto{\pgfqpoint{5.806978in}{3.402649in}}%
\pgfpathlineto{\pgfqpoint{5.808052in}{3.401324in}}%
\pgfpathlineto{\pgfqpoint{5.812347in}{3.401565in}}%
\pgfpathlineto{\pgfqpoint{5.814495in}{3.410483in}}%
\pgfpathlineto{\pgfqpoint{5.815568in}{3.406506in}}%
\pgfpathlineto{\pgfqpoint{5.816642in}{3.407591in}}%
\pgfpathlineto{\pgfqpoint{5.820938in}{3.401565in}}%
\pgfpathlineto{\pgfqpoint{5.822011in}{3.403493in}}%
\pgfpathlineto{\pgfqpoint{5.823085in}{3.397708in}}%
\pgfpathlineto{\pgfqpoint{5.824159in}{3.398672in}}%
\pgfpathlineto{\pgfqpoint{5.827381in}{3.399034in}}%
\pgfpathlineto{\pgfqpoint{5.829528in}{3.404096in}}%
\pgfpathlineto{\pgfqpoint{5.831676in}{3.396744in}}%
\pgfpathlineto{\pgfqpoint{5.834898in}{3.398793in}}%
\pgfpathlineto{\pgfqpoint{5.835971in}{3.398310in}}%
\pgfpathlineto{\pgfqpoint{5.837045in}{3.402649in}}%
\pgfpathlineto{\pgfqpoint{5.838119in}{3.402288in}}%
\pgfpathlineto{\pgfqpoint{5.839193in}{3.398672in}}%
\pgfpathlineto{\pgfqpoint{5.842414in}{3.396021in}}%
\pgfpathlineto{\pgfqpoint{5.843488in}{3.396141in}}%
\pgfpathlineto{\pgfqpoint{5.844562in}{3.399877in}}%
\pgfpathlineto{\pgfqpoint{5.846710in}{3.383848in}}%
\pgfpathlineto{\pgfqpoint{5.849931in}{3.387463in}}%
\pgfpathlineto{\pgfqpoint{5.852079in}{3.381919in}}%
\pgfpathlineto{\pgfqpoint{5.853153in}{3.382522in}}%
\pgfpathlineto{\pgfqpoint{5.854227in}{3.384089in}}%
\pgfpathlineto{\pgfqpoint{5.857448in}{3.381437in}}%
\pgfpathlineto{\pgfqpoint{5.858522in}{3.385173in}}%
\pgfpathlineto{\pgfqpoint{5.859596in}{3.384209in}}%
\pgfpathlineto{\pgfqpoint{5.860670in}{3.380955in}}%
\pgfpathlineto{\pgfqpoint{5.864965in}{3.387945in}}%
\pgfpathlineto{\pgfqpoint{5.866039in}{3.383366in}}%
\pgfpathlineto{\pgfqpoint{5.867113in}{3.383486in}}%
\pgfpathlineto{\pgfqpoint{5.868187in}{3.379509in}}%
\pgfpathlineto{\pgfqpoint{5.869260in}{3.385294in}}%
\pgfpathlineto{\pgfqpoint{5.872482in}{3.386138in}}%
\pgfpathlineto{\pgfqpoint{5.873556in}{3.395297in}}%
\pgfpathlineto{\pgfqpoint{5.874630in}{3.398913in}}%
\pgfpathlineto{\pgfqpoint{5.876777in}{3.400962in}}%
\pgfpathlineto{\pgfqpoint{5.881073in}{3.401203in}}%
\pgfpathlineto{\pgfqpoint{5.883220in}{3.403011in}}%
\pgfpathlineto{\pgfqpoint{5.884294in}{3.401444in}}%
\pgfpathlineto{\pgfqpoint{5.887516in}{3.402167in}}%
\pgfpathlineto{\pgfqpoint{5.888589in}{3.404337in}}%
\pgfpathlineto{\pgfqpoint{5.890737in}{3.405060in}}%
\pgfpathlineto{\pgfqpoint{5.891811in}{3.406145in}}%
\pgfpathlineto{\pgfqpoint{5.895032in}{3.407591in}}%
\pgfpathlineto{\pgfqpoint{5.896106in}{3.407229in}}%
\pgfpathlineto{\pgfqpoint{5.897180in}{3.401444in}}%
\pgfpathlineto{\pgfqpoint{5.899328in}{3.403011in}}%
\pgfpathlineto{\pgfqpoint{5.903623in}{3.409278in}}%
\pgfpathlineto{\pgfqpoint{5.904697in}{3.408917in}}%
\pgfpathlineto{\pgfqpoint{5.905771in}{3.416630in}}%
\pgfpathlineto{\pgfqpoint{5.906845in}{3.400721in}}%
\pgfpathlineto{\pgfqpoint{5.910066in}{3.391561in}}%
\pgfpathlineto{\pgfqpoint{5.911140in}{3.396744in}}%
\pgfpathlineto{\pgfqpoint{5.913288in}{3.414581in}}%
\pgfpathlineto{\pgfqpoint{5.914362in}{3.413979in}}%
\pgfpathlineto{\pgfqpoint{5.918657in}{3.413255in}}%
\pgfpathlineto{\pgfqpoint{5.920805in}{3.418920in}}%
\pgfpathlineto{\pgfqpoint{5.921878in}{3.427839in}}%
\pgfpathlineto{\pgfqpoint{5.925100in}{3.431937in}}%
\pgfpathlineto{\pgfqpoint{5.926174in}{3.438204in}}%
\pgfpathlineto{\pgfqpoint{5.927248in}{3.438927in}}%
\pgfpathlineto{\pgfqpoint{5.928321in}{3.441096in}}%
\pgfpathlineto{\pgfqpoint{5.929395in}{3.439650in}}%
\pgfpathlineto{\pgfqpoint{5.932617in}{3.439409in}}%
\pgfpathlineto{\pgfqpoint{5.933691in}{3.440494in}}%
\pgfpathlineto{\pgfqpoint{5.934765in}{3.446520in}}%
\pgfpathlineto{\pgfqpoint{5.935838in}{3.430731in}}%
\pgfpathlineto{\pgfqpoint{5.936912in}{3.435070in}}%
\pgfpathlineto{\pgfqpoint{5.940134in}{3.435432in}}%
\pgfpathlineto{\pgfqpoint{5.941208in}{3.439891in}}%
\pgfpathlineto{\pgfqpoint{5.942281in}{3.436999in}}%
\pgfpathlineto{\pgfqpoint{5.943355in}{3.436275in}}%
\pgfpathlineto{\pgfqpoint{5.944429in}{3.437240in}}%
\pgfpathlineto{\pgfqpoint{5.947651in}{3.437240in}}%
\pgfpathlineto{\pgfqpoint{5.948724in}{3.433985in}}%
\pgfpathlineto{\pgfqpoint{5.949798in}{3.433383in}}%
\pgfpathlineto{\pgfqpoint{5.951946in}{3.441578in}}%
\pgfpathlineto{\pgfqpoint{5.955167in}{3.442181in}}%
\pgfpathlineto{\pgfqpoint{5.956241in}{3.440855in}}%
\pgfpathlineto{\pgfqpoint{5.957315in}{3.436516in}}%
\pgfpathlineto{\pgfqpoint{5.958389in}{3.438204in}}%
\pgfpathlineto{\pgfqpoint{5.959463in}{3.436999in}}%
\pgfpathlineto{\pgfqpoint{5.962684in}{3.440735in}}%
\pgfpathlineto{\pgfqpoint{5.963758in}{3.444109in}}%
\pgfpathlineto{\pgfqpoint{5.964832in}{3.441940in}}%
\pgfpathlineto{\pgfqpoint{5.965906in}{3.441458in}}%
\pgfpathlineto{\pgfqpoint{5.966980in}{3.444471in}}%
\pgfpathlineto{\pgfqpoint{5.971275in}{3.446279in}}%
\pgfpathlineto{\pgfqpoint{5.972349in}{3.443507in}}%
\pgfpathlineto{\pgfqpoint{5.973423in}{3.442784in}}%
\pgfpathlineto{\pgfqpoint{5.974497in}{3.444712in}}%
\pgfpathlineto{\pgfqpoint{5.978792in}{3.449533in}}%
\pgfpathlineto{\pgfqpoint{5.980940in}{3.453149in}}%
\pgfpathlineto{\pgfqpoint{5.982013in}{3.453751in}}%
\pgfpathlineto{\pgfqpoint{5.986309in}{3.459295in}}%
\pgfpathlineto{\pgfqpoint{5.987383in}{3.457970in}}%
\pgfpathlineto{\pgfqpoint{5.988456in}{3.457849in}}%
\pgfpathlineto{\pgfqpoint{5.989530in}{3.446640in}}%
\pgfpathlineto{\pgfqpoint{5.992752in}{3.453751in}}%
\pgfpathlineto{\pgfqpoint{5.993826in}{3.448569in}}%
\pgfpathlineto{\pgfqpoint{5.994899in}{3.448689in}}%
\pgfpathlineto{\pgfqpoint{5.997047in}{3.471468in}}%
\pgfpathlineto{\pgfqpoint{6.000269in}{3.465804in}}%
\pgfpathlineto{\pgfqpoint{6.001343in}{3.465563in}}%
\pgfpathlineto{\pgfqpoint{6.002416in}{3.469058in}}%
\pgfpathlineto{\pgfqpoint{6.003490in}{3.470143in}}%
\pgfpathlineto{\pgfqpoint{6.004564in}{3.466165in}}%
\pgfpathlineto{\pgfqpoint{6.007786in}{3.460139in}}%
\pgfpathlineto{\pgfqpoint{6.009933in}{3.468937in}}%
\pgfpathlineto{\pgfqpoint{6.011007in}{3.467612in}}%
\pgfpathlineto{\pgfqpoint{6.012081in}{3.472432in}}%
\pgfpathlineto{\pgfqpoint{6.015302in}{3.471348in}}%
\pgfpathlineto{\pgfqpoint{6.016376in}{3.470022in}}%
\pgfpathlineto{\pgfqpoint{6.017450in}{3.475084in}}%
\pgfpathlineto{\pgfqpoint{6.019598in}{3.476289in}}%
\pgfpathlineto{\pgfqpoint{6.022819in}{3.475446in}}%
\pgfpathlineto{\pgfqpoint{6.023893in}{3.467009in}}%
\pgfpathlineto{\pgfqpoint{6.026041in}{3.463755in}}%
\pgfpathlineto{\pgfqpoint{6.027115in}{3.469058in}}%
\pgfpathlineto{\pgfqpoint{6.030336in}{3.467250in}}%
\pgfpathlineto{\pgfqpoint{6.031410in}{3.472432in}}%
\pgfpathlineto{\pgfqpoint{6.032484in}{3.447484in}}%
\pgfpathlineto{\pgfqpoint{6.033558in}{3.446520in}}%
\pgfpathlineto{\pgfqpoint{6.034632in}{3.443507in}}%
\pgfpathlineto{\pgfqpoint{6.037853in}{3.444712in}}%
\pgfpathlineto{\pgfqpoint{6.041075in}{3.439650in}}%
\pgfpathlineto{\pgfqpoint{6.042148in}{3.438927in}}%
\pgfpathlineto{\pgfqpoint{6.045370in}{3.440373in}}%
\pgfpathlineto{\pgfqpoint{6.046444in}{3.436396in}}%
\pgfpathlineto{\pgfqpoint{6.047518in}{3.437360in}}%
\pgfpathlineto{\pgfqpoint{6.049665in}{3.429165in}}%
\pgfpathlineto{\pgfqpoint{6.052887in}{3.441217in}}%
\pgfpathlineto{\pgfqpoint{6.055034in}{3.441940in}}%
\pgfpathlineto{\pgfqpoint{6.056108in}{3.439168in}}%
\pgfpathlineto{\pgfqpoint{6.057182in}{3.440373in}}%
\pgfpathlineto{\pgfqpoint{6.060404in}{3.438927in}}%
\pgfpathlineto{\pgfqpoint{6.061477in}{3.443748in}}%
\pgfpathlineto{\pgfqpoint{6.062551in}{3.442904in}}%
\pgfpathlineto{\pgfqpoint{6.063625in}{3.444953in}}%
\pgfpathlineto{\pgfqpoint{6.064699in}{3.444109in}}%
\pgfpathlineto{\pgfqpoint{6.067920in}{3.444471in}}%
\pgfpathlineto{\pgfqpoint{6.068994in}{3.450136in}}%
\pgfpathlineto{\pgfqpoint{6.070068in}{3.447002in}}%
\pgfpathlineto{\pgfqpoint{6.072216in}{3.449654in}}%
\pgfpathlineto{\pgfqpoint{6.075437in}{3.450497in}}%
\pgfpathlineto{\pgfqpoint{6.076511in}{3.448207in}}%
\pgfpathlineto{\pgfqpoint{6.077585in}{3.441337in}}%
\pgfpathlineto{\pgfqpoint{6.078659in}{3.430852in}}%
\pgfpathlineto{\pgfqpoint{6.079733in}{3.435311in}}%
\pgfpathlineto{\pgfqpoint{6.082954in}{3.437842in}}%
\pgfpathlineto{\pgfqpoint{6.084028in}{3.441578in}}%
\pgfpathlineto{\pgfqpoint{6.085102in}{3.450377in}}%
\pgfpathlineto{\pgfqpoint{6.086176in}{3.452546in}}%
\pgfpathlineto{\pgfqpoint{6.090471in}{3.455559in}}%
\pgfpathlineto{\pgfqpoint{6.091545in}{3.464960in}}%
\pgfpathlineto{\pgfqpoint{6.092619in}{3.462067in}}%
\pgfpathlineto{\pgfqpoint{6.093693in}{3.464839in}}%
\pgfpathlineto{\pgfqpoint{6.094766in}{3.459416in}}%
\pgfpathlineto{\pgfqpoint{6.097988in}{3.465924in}}%
\pgfpathlineto{\pgfqpoint{6.099062in}{3.469540in}}%
\pgfpathlineto{\pgfqpoint{6.100136in}{3.466888in}}%
\pgfpathlineto{\pgfqpoint{6.101209in}{3.466406in}}%
\pgfpathlineto{\pgfqpoint{6.106579in}{3.467973in}}%
\pgfpathlineto{\pgfqpoint{6.107653in}{3.463032in}}%
\pgfpathlineto{\pgfqpoint{6.108726in}{3.463393in}}%
\pgfpathlineto{\pgfqpoint{6.109800in}{3.458934in}}%
\pgfpathlineto{\pgfqpoint{6.114096in}{3.462670in}}%
\pgfpathlineto{\pgfqpoint{6.115169in}{3.460501in}}%
\pgfpathlineto{\pgfqpoint{6.116243in}{3.459898in}}%
\pgfpathlineto{\pgfqpoint{6.117317in}{3.461344in}}%
\pgfpathlineto{\pgfqpoint{6.120539in}{3.462791in}}%
\pgfpathlineto{\pgfqpoint{6.121612in}{3.461947in}}%
\pgfpathlineto{\pgfqpoint{6.122686in}{3.466647in}}%
\pgfpathlineto{\pgfqpoint{6.123760in}{3.463875in}}%
\pgfpathlineto{\pgfqpoint{6.124834in}{3.464839in}}%
\pgfpathlineto{\pgfqpoint{6.130203in}{3.464478in}}%
\pgfpathlineto{\pgfqpoint{6.131277in}{3.462309in}}%
\pgfpathlineto{\pgfqpoint{6.132351in}{3.466527in}}%
\pgfpathlineto{\pgfqpoint{6.135572in}{3.464598in}}%
\pgfpathlineto{\pgfqpoint{6.136646in}{3.474120in}}%
\pgfpathlineto{\pgfqpoint{6.137720in}{3.476169in}}%
\pgfpathlineto{\pgfqpoint{6.138794in}{3.473397in}}%
\pgfpathlineto{\pgfqpoint{6.139868in}{3.478097in}}%
\pgfpathlineto{\pgfqpoint{6.143089in}{3.471830in}}%
\pgfpathlineto{\pgfqpoint{6.145237in}{3.461826in}}%
\pgfpathlineto{\pgfqpoint{6.147385in}{3.464719in}}%
\pgfpathlineto{\pgfqpoint{6.150606in}{3.461826in}}%
\pgfpathlineto{\pgfqpoint{6.152754in}{3.463152in}}%
\pgfpathlineto{\pgfqpoint{6.153828in}{3.452667in}}%
\pgfpathlineto{\pgfqpoint{6.154901in}{3.451341in}}%
\pgfpathlineto{\pgfqpoint{6.160271in}{3.459416in}}%
\pgfpathlineto{\pgfqpoint{6.161344in}{3.463514in}}%
\pgfpathlineto{\pgfqpoint{6.162418in}{3.464237in}}%
\pgfpathlineto{\pgfqpoint{6.166714in}{3.464719in}}%
\pgfpathlineto{\pgfqpoint{6.167787in}{3.459657in}}%
\pgfpathlineto{\pgfqpoint{6.168861in}{3.460862in}}%
\pgfpathlineto{\pgfqpoint{6.169935in}{3.464839in}}%
\pgfpathlineto{\pgfqpoint{6.173157in}{3.464598in}}%
\pgfpathlineto{\pgfqpoint{6.175304in}{3.458090in}}%
\pgfpathlineto{\pgfqpoint{6.177452in}{3.457729in}}%
\pgfpathlineto{\pgfqpoint{6.180674in}{3.453992in}}%
\pgfpathlineto{\pgfqpoint{6.181747in}{3.456523in}}%
\pgfpathlineto{\pgfqpoint{6.182821in}{3.454474in}}%
\pgfpathlineto{\pgfqpoint{6.184969in}{3.457849in}}%
\pgfpathlineto{\pgfqpoint{6.188190in}{3.449292in}}%
\pgfpathlineto{\pgfqpoint{6.189264in}{3.449533in}}%
\pgfpathlineto{\pgfqpoint{6.190338in}{3.448689in}}%
\pgfpathlineto{\pgfqpoint{6.191412in}{3.449051in}}%
\pgfpathlineto{\pgfqpoint{6.192486in}{3.450618in}}%
\pgfpathlineto{\pgfqpoint{6.195707in}{3.452426in}}%
\pgfpathlineto{\pgfqpoint{6.196781in}{3.447966in}}%
\pgfpathlineto{\pgfqpoint{6.197855in}{3.451702in}}%
\pgfpathlineto{\pgfqpoint{6.200003in}{3.449292in}}%
\pgfpathlineto{\pgfqpoint{6.203224in}{3.451943in}}%
\pgfpathlineto{\pgfqpoint{6.204298in}{3.454354in}}%
\pgfpathlineto{\pgfqpoint{6.205372in}{3.453992in}}%
\pgfpathlineto{\pgfqpoint{6.206446in}{3.456041in}}%
\pgfpathlineto{\pgfqpoint{6.207520in}{3.459657in}}%
\pgfpathlineto{\pgfqpoint{6.210741in}{3.460621in}}%
\pgfpathlineto{\pgfqpoint{6.211815in}{3.461947in}}%
\pgfpathlineto{\pgfqpoint{6.212889in}{3.461344in}}%
\pgfpathlineto{\pgfqpoint{6.213963in}{3.459175in}}%
\pgfpathlineto{\pgfqpoint{6.215036in}{3.459175in}}%
\pgfpathlineto{\pgfqpoint{6.220406in}{3.454595in}}%
\pgfpathlineto{\pgfqpoint{6.221479in}{3.450377in}}%
\pgfpathlineto{\pgfqpoint{6.225775in}{3.452908in}}%
\pgfpathlineto{\pgfqpoint{6.227922in}{3.457849in}}%
\pgfpathlineto{\pgfqpoint{6.230070in}{3.462429in}}%
\pgfpathlineto{\pgfqpoint{6.234365in}{3.468696in}}%
\pgfpathlineto{\pgfqpoint{6.235439in}{3.469299in}}%
\pgfpathlineto{\pgfqpoint{6.236513in}{3.474963in}}%
\pgfpathlineto{\pgfqpoint{6.237587in}{3.460501in}}%
\pgfpathlineto{\pgfqpoint{6.240809in}{3.462309in}}%
\pgfpathlineto{\pgfqpoint{6.241882in}{3.469781in}}%
\pgfpathlineto{\pgfqpoint{6.242956in}{3.473035in}}%
\pgfpathlineto{\pgfqpoint{6.244030in}{3.471468in}}%
\pgfpathlineto{\pgfqpoint{6.245104in}{3.471227in}}%
\pgfpathlineto{\pgfqpoint{6.248325in}{3.467973in}}%
\pgfpathlineto{\pgfqpoint{6.249399in}{3.466045in}}%
\pgfpathlineto{\pgfqpoint{6.251547in}{3.458331in}}%
\pgfpathlineto{\pgfqpoint{6.252621in}{3.456523in}}%
\pgfpathlineto{\pgfqpoint{6.255842in}{3.457608in}}%
\pgfpathlineto{\pgfqpoint{6.256916in}{3.459778in}}%
\pgfpathlineto{\pgfqpoint{6.257990in}{3.450979in}}%
\pgfpathlineto{\pgfqpoint{6.260138in}{3.455077in}}%
\pgfpathlineto{\pgfqpoint{6.264433in}{3.460260in}}%
\pgfpathlineto{\pgfqpoint{6.266581in}{3.464839in}}%
\pgfpathlineto{\pgfqpoint{6.267654in}{3.464839in}}%
\pgfpathlineto{\pgfqpoint{6.274097in}{3.463273in}}%
\pgfpathlineto{\pgfqpoint{6.275171in}{3.465442in}}%
\pgfpathlineto{\pgfqpoint{6.278393in}{3.465683in}}%
\pgfpathlineto{\pgfqpoint{6.279467in}{3.463273in}}%
\pgfpathlineto{\pgfqpoint{6.281614in}{3.467370in}}%
\pgfpathlineto{\pgfqpoint{6.282688in}{3.458572in}}%
\pgfpathlineto{\pgfqpoint{6.285910in}{3.458813in}}%
\pgfpathlineto{\pgfqpoint{6.286984in}{3.460501in}}%
\pgfpathlineto{\pgfqpoint{6.289131in}{3.453992in}}%
\pgfpathlineto{\pgfqpoint{6.290205in}{3.452908in}}%
\pgfpathlineto{\pgfqpoint{6.293427in}{3.456282in}}%
\pgfpathlineto{\pgfqpoint{6.294500in}{3.448930in}}%
\pgfpathlineto{\pgfqpoint{6.296648in}{3.443145in}}%
\pgfpathlineto{\pgfqpoint{6.297722in}{3.441217in}}%
\pgfpathlineto{\pgfqpoint{6.300943in}{3.439891in}}%
\pgfpathlineto{\pgfqpoint{6.302017in}{3.435070in}}%
\pgfpathlineto{\pgfqpoint{6.303091in}{3.441337in}}%
\pgfpathlineto{\pgfqpoint{6.304165in}{3.433865in}}%
\pgfpathlineto{\pgfqpoint{6.305239in}{3.436155in}}%
\pgfpathlineto{\pgfqpoint{6.308460in}{3.432901in}}%
\pgfpathlineto{\pgfqpoint{6.310608in}{3.442904in}}%
\pgfpathlineto{\pgfqpoint{6.311682in}{3.434829in}}%
\pgfpathlineto{\pgfqpoint{6.312756in}{3.437722in}}%
\pgfpathlineto{\pgfqpoint{6.315977in}{3.435070in}}%
\pgfpathlineto{\pgfqpoint{6.318125in}{3.441940in}}%
\pgfpathlineto{\pgfqpoint{6.319199in}{3.441819in}}%
\pgfpathlineto{\pgfqpoint{6.320273in}{3.446881in}}%
\pgfpathlineto{\pgfqpoint{6.323494in}{3.444471in}}%
\pgfpathlineto{\pgfqpoint{6.325642in}{3.445435in}}%
\pgfpathlineto{\pgfqpoint{6.326716in}{3.447605in}}%
\pgfpathlineto{\pgfqpoint{6.327789in}{3.447364in}}%
\pgfpathlineto{\pgfqpoint{6.331011in}{3.444833in}}%
\pgfpathlineto{\pgfqpoint{6.334232in}{3.450136in}}%
\pgfpathlineto{\pgfqpoint{6.335306in}{3.453992in}}%
\pgfpathlineto{\pgfqpoint{6.338528in}{3.455800in}}%
\pgfpathlineto{\pgfqpoint{6.339602in}{3.465804in}}%
\pgfpathlineto{\pgfqpoint{6.340675in}{3.469178in}}%
\pgfpathlineto{\pgfqpoint{6.341749in}{3.470504in}}%
\pgfpathlineto{\pgfqpoint{6.342823in}{3.468335in}}%
\pgfpathlineto{\pgfqpoint{6.346045in}{3.469901in}}%
\pgfpathlineto{\pgfqpoint{6.347119in}{3.469660in}}%
\pgfpathlineto{\pgfqpoint{6.348192in}{3.471709in}}%
\pgfpathlineto{\pgfqpoint{6.350340in}{3.463393in}}%
\pgfpathlineto{\pgfqpoint{6.353562in}{3.468817in}}%
\pgfpathlineto{\pgfqpoint{6.357857in}{3.453631in}}%
\pgfpathlineto{\pgfqpoint{6.361078in}{3.452546in}}%
\pgfpathlineto{\pgfqpoint{6.362152in}{3.449412in}}%
\pgfpathlineto{\pgfqpoint{6.365374in}{3.449654in}}%
\pgfpathlineto{\pgfqpoint{6.368595in}{3.449412in}}%
\pgfpathlineto{\pgfqpoint{6.369669in}{3.450377in}}%
\pgfpathlineto{\pgfqpoint{6.371817in}{3.454233in}}%
\pgfpathlineto{\pgfqpoint{6.372891in}{3.454474in}}%
\pgfpathlineto{\pgfqpoint{6.377186in}{3.453631in}}%
\pgfpathlineto{\pgfqpoint{6.378260in}{3.462188in}}%
\pgfpathlineto{\pgfqpoint{6.379334in}{3.459657in}}%
\pgfpathlineto{\pgfqpoint{6.380408in}{3.455680in}}%
\pgfpathlineto{\pgfqpoint{6.383629in}{3.462309in}}%
\pgfpathlineto{\pgfqpoint{6.385777in}{3.468696in}}%
\pgfpathlineto{\pgfqpoint{6.386851in}{3.470384in}}%
\pgfpathlineto{\pgfqpoint{6.387924in}{3.476410in}}%
\pgfpathlineto{\pgfqpoint{6.391146in}{3.476410in}}%
\pgfpathlineto{\pgfqpoint{6.392220in}{3.479061in}}%
\pgfpathlineto{\pgfqpoint{6.393294in}{3.477253in}}%
\pgfpathlineto{\pgfqpoint{6.394367in}{3.478700in}}%
\pgfpathlineto{\pgfqpoint{6.398663in}{3.478218in}}%
\pgfpathlineto{\pgfqpoint{6.399737in}{3.481833in}}%
\pgfpathlineto{\pgfqpoint{6.400810in}{3.482556in}}%
\pgfpathlineto{\pgfqpoint{6.407253in}{3.503648in}}%
\pgfpathlineto{\pgfqpoint{6.408327in}{3.503166in}}%
\pgfpathlineto{\pgfqpoint{6.410475in}{3.506541in}}%
\pgfpathlineto{\pgfqpoint{6.413697in}{3.509192in}}%
\pgfpathlineto{\pgfqpoint{6.414770in}{3.506782in}}%
\pgfpathlineto{\pgfqpoint{6.415844in}{3.502804in}}%
\pgfpathlineto{\pgfqpoint{6.416918in}{3.501479in}}%
\pgfpathlineto{\pgfqpoint{6.417992in}{3.507023in}}%
\pgfpathlineto{\pgfqpoint{6.422287in}{3.508349in}}%
\pgfpathlineto{\pgfqpoint{6.423361in}{3.513652in}}%
\pgfpathlineto{\pgfqpoint{6.424435in}{3.511844in}}%
\pgfpathlineto{\pgfqpoint{6.425509in}{3.515700in}}%
\pgfpathlineto{\pgfqpoint{6.429804in}{3.521727in}}%
\pgfpathlineto{\pgfqpoint{6.430878in}{3.519678in}}%
\pgfpathlineto{\pgfqpoint{6.431952in}{3.526307in}}%
\pgfpathlineto{\pgfqpoint{6.433026in}{3.561258in}}%
\pgfpathlineto{\pgfqpoint{6.436247in}{3.560897in}}%
\pgfpathlineto{\pgfqpoint{6.438395in}{3.587774in}}%
\pgfpathlineto{\pgfqpoint{6.439469in}{3.592233in}}%
\pgfpathlineto{\pgfqpoint{6.440542in}{3.583435in}}%
\pgfpathlineto{\pgfqpoint{6.443764in}{3.590787in}}%
\pgfpathlineto{\pgfqpoint{6.444838in}{3.591751in}}%
\pgfpathlineto{\pgfqpoint{6.445912in}{3.590787in}}%
\pgfpathlineto{\pgfqpoint{6.446985in}{3.586207in}}%
\pgfpathlineto{\pgfqpoint{6.448059in}{3.577891in}}%
\pgfpathlineto{\pgfqpoint{6.451281in}{3.579819in}}%
\pgfpathlineto{\pgfqpoint{6.452355in}{3.581145in}}%
\pgfpathlineto{\pgfqpoint{6.453429in}{3.576444in}}%
\pgfpathlineto{\pgfqpoint{6.454502in}{3.578614in}}%
\pgfpathlineto{\pgfqpoint{6.455576in}{3.566923in}}%
\pgfpathlineto{\pgfqpoint{6.458798in}{3.566802in}}%
\pgfpathlineto{\pgfqpoint{6.459872in}{3.570539in}}%
\pgfpathlineto{\pgfqpoint{6.460945in}{3.567164in}}%
\pgfpathlineto{\pgfqpoint{6.463093in}{3.568249in}}%
\pgfpathlineto{\pgfqpoint{6.466315in}{3.565236in}}%
\pgfpathlineto{\pgfqpoint{6.467388in}{3.568008in}}%
\pgfpathlineto{\pgfqpoint{6.468462in}{3.559089in}}%
\pgfpathlineto{\pgfqpoint{6.469536in}{3.569333in}}%
\pgfpathlineto{\pgfqpoint{6.470610in}{3.567526in}}%
\pgfpathlineto{\pgfqpoint{6.473831in}{3.565236in}}%
\pgfpathlineto{\pgfqpoint{6.474905in}{3.553183in}}%
\pgfpathlineto{\pgfqpoint{6.475979in}{3.553304in}}%
\pgfpathlineto{\pgfqpoint{6.477053in}{3.549085in}}%
\pgfpathlineto{\pgfqpoint{6.478127in}{3.552099in}}%
\pgfpathlineto{\pgfqpoint{6.481348in}{3.555714in}}%
\pgfpathlineto{\pgfqpoint{6.482422in}{3.551858in}}%
\pgfpathlineto{\pgfqpoint{6.483496in}{3.551978in}}%
\pgfpathlineto{\pgfqpoint{6.484570in}{3.551134in}}%
\pgfpathlineto{\pgfqpoint{6.485644in}{3.566079in}}%
\pgfpathlineto{\pgfqpoint{6.491013in}{3.600670in}}%
\pgfpathlineto{\pgfqpoint{6.492087in}{3.591510in}}%
\pgfpathlineto{\pgfqpoint{6.493161in}{3.590787in}}%
\pgfpathlineto{\pgfqpoint{6.497456in}{3.583676in}}%
\pgfpathlineto{\pgfqpoint{6.498530in}{3.583917in}}%
\pgfpathlineto{\pgfqpoint{6.499604in}{3.585243in}}%
\pgfpathlineto{\pgfqpoint{6.500677in}{3.584519in}}%
\pgfpathlineto{\pgfqpoint{6.500677in}{3.584519in}}%
\pgfusepath{stroke}%
\end{pgfscope}%
\begin{pgfscope}%
\pgfpathrectangle{\pgfqpoint{4.034768in}{3.218007in}}{\pgfqpoint{2.583333in}{0.400885in}}%
\pgfusepath{clip}%
\pgfsetroundcap%
\pgfsetroundjoin%
\pgfsetlinewidth{1.505625pt}%
\definecolor{currentstroke}{rgb}{0.839216,0.152941,0.156863}%
\pgfsetstrokecolor{currentstroke}%
\pgfsetdash{}{0pt}%
\pgfpathmoveto{\pgfqpoint{4.152193in}{3.285139in}}%
\pgfpathlineto{\pgfqpoint{4.153266in}{3.285139in}}%
\pgfpathlineto{\pgfqpoint{4.155414in}{3.281049in}}%
\pgfpathlineto{\pgfqpoint{4.158636in}{3.283120in}}%
\pgfpathlineto{\pgfqpoint{4.159709in}{3.282085in}}%
\pgfpathlineto{\pgfqpoint{4.160783in}{3.284023in}}%
\pgfpathlineto{\pgfqpoint{4.161857in}{3.284480in}}%
\pgfpathlineto{\pgfqpoint{4.162931in}{3.278984in}}%
\pgfpathlineto{\pgfqpoint{4.167226in}{3.278068in}}%
\pgfpathlineto{\pgfqpoint{4.168300in}{3.281159in}}%
\pgfpathlineto{\pgfqpoint{4.169374in}{3.278984in}}%
\pgfpathlineto{\pgfqpoint{4.170448in}{3.272356in}}%
\pgfpathlineto{\pgfqpoint{4.176891in}{3.266828in}}%
\pgfpathlineto{\pgfqpoint{4.181186in}{3.266828in}}%
\pgfpathlineto{\pgfqpoint{4.182260in}{3.264179in}}%
\pgfpathlineto{\pgfqpoint{4.184408in}{3.265809in}}%
\pgfpathlineto{\pgfqpoint{4.185482in}{3.269494in}}%
\pgfpathlineto{\pgfqpoint{4.188703in}{3.269597in}}%
\pgfpathlineto{\pgfqpoint{4.189777in}{3.268879in}}%
\pgfpathlineto{\pgfqpoint{4.190851in}{3.270621in}}%
\pgfpathlineto{\pgfqpoint{4.191925in}{3.270519in}}%
\pgfpathlineto{\pgfqpoint{4.192998in}{3.269188in}}%
\pgfpathlineto{\pgfqpoint{4.196220in}{3.269188in}}%
\pgfpathlineto{\pgfqpoint{4.197294in}{3.269905in}}%
\pgfpathlineto{\pgfqpoint{4.199441in}{3.273621in}}%
\pgfpathlineto{\pgfqpoint{4.200515in}{3.269256in}}%
\pgfpathlineto{\pgfqpoint{4.204811in}{3.270855in}}%
\pgfpathlineto{\pgfqpoint{4.205885in}{3.267408in}}%
\pgfpathlineto{\pgfqpoint{4.206958in}{3.266799in}}%
\pgfpathlineto{\pgfqpoint{4.208032in}{3.267104in}}%
\pgfpathlineto{\pgfqpoint{4.211254in}{3.268624in}}%
\pgfpathlineto{\pgfqpoint{4.212328in}{3.265786in}}%
\pgfpathlineto{\pgfqpoint{4.213401in}{3.268551in}}%
\pgfpathlineto{\pgfqpoint{4.214475in}{3.268348in}}%
\pgfpathlineto{\pgfqpoint{4.215549in}{3.268855in}}%
\pgfpathlineto{\pgfqpoint{4.218771in}{3.265814in}}%
\pgfpathlineto{\pgfqpoint{4.219844in}{3.266321in}}%
\pgfpathlineto{\pgfqpoint{4.220918in}{3.263889in}}%
\pgfpathlineto{\pgfqpoint{4.221992in}{3.264483in}}%
\pgfpathlineto{\pgfqpoint{4.223066in}{3.266376in}}%
\pgfpathlineto{\pgfqpoint{4.226287in}{3.267074in}}%
\pgfpathlineto{\pgfqpoint{4.227361in}{3.271085in}}%
\pgfpathlineto{\pgfqpoint{4.228435in}{3.271285in}}%
\pgfpathlineto{\pgfqpoint{4.229509in}{3.273596in}}%
\pgfpathlineto{\pgfqpoint{4.233804in}{3.273898in}}%
\pgfpathlineto{\pgfqpoint{4.235952in}{3.274200in}}%
\pgfpathlineto{\pgfqpoint{4.237026in}{3.275206in}}%
\pgfpathlineto{\pgfqpoint{4.238100in}{3.275408in}}%
\pgfpathlineto{\pgfqpoint{4.241321in}{3.277928in}}%
\pgfpathlineto{\pgfqpoint{4.242395in}{3.277928in}}%
\pgfpathlineto{\pgfqpoint{4.243469in}{3.274803in}}%
\pgfpathlineto{\pgfqpoint{4.244543in}{3.277626in}}%
\pgfpathlineto{\pgfqpoint{4.245617in}{3.277928in}}%
\pgfpathlineto{\pgfqpoint{4.248838in}{3.280051in}}%
\pgfpathlineto{\pgfqpoint{4.249912in}{3.282276in}}%
\pgfpathlineto{\pgfqpoint{4.250986in}{3.280832in}}%
\pgfpathlineto{\pgfqpoint{4.252060in}{3.281966in}}%
\pgfpathlineto{\pgfqpoint{4.256355in}{3.284544in}}%
\pgfpathlineto{\pgfqpoint{4.257429in}{3.282013in}}%
\pgfpathlineto{\pgfqpoint{4.258503in}{3.285387in}}%
\pgfpathlineto{\pgfqpoint{4.259576in}{3.280116in}}%
\pgfpathlineto{\pgfqpoint{4.260650in}{3.283240in}}%
\pgfpathlineto{\pgfqpoint{4.263872in}{3.285829in}}%
\pgfpathlineto{\pgfqpoint{4.264946in}{3.285311in}}%
\pgfpathlineto{\pgfqpoint{4.266019in}{3.289642in}}%
\pgfpathlineto{\pgfqpoint{4.267093in}{3.287502in}}%
\pgfpathlineto{\pgfqpoint{4.268167in}{3.286752in}}%
\pgfpathlineto{\pgfqpoint{4.271389in}{3.285468in}}%
\pgfpathlineto{\pgfqpoint{4.272462in}{3.284291in}}%
\pgfpathlineto{\pgfqpoint{4.273536in}{3.288893in}}%
\pgfpathlineto{\pgfqpoint{4.274610in}{3.285789in}}%
\pgfpathlineto{\pgfqpoint{4.275684in}{3.284434in}}%
\pgfpathlineto{\pgfqpoint{4.278906in}{3.284331in}}%
\pgfpathlineto{\pgfqpoint{4.279979in}{3.288861in}}%
\pgfpathlineto{\pgfqpoint{4.281053in}{3.287008in}}%
\pgfpathlineto{\pgfqpoint{4.282127in}{3.290352in}}%
\pgfpathlineto{\pgfqpoint{4.283201in}{3.284930in}}%
\pgfpathlineto{\pgfqpoint{4.286422in}{3.283678in}}%
\pgfpathlineto{\pgfqpoint{4.287496in}{3.280445in}}%
\pgfpathlineto{\pgfqpoint{4.288570in}{3.278985in}}%
\pgfpathlineto{\pgfqpoint{4.289644in}{3.279403in}}%
\pgfpathlineto{\pgfqpoint{4.290718in}{3.276170in}}%
\pgfpathlineto{\pgfqpoint{4.293939in}{3.281136in}}%
\pgfpathlineto{\pgfqpoint{4.295013in}{3.279971in}}%
\pgfpathlineto{\pgfqpoint{4.297161in}{3.274143in}}%
\pgfpathlineto{\pgfqpoint{4.298235in}{3.273084in}}%
\pgfpathlineto{\pgfqpoint{4.301456in}{3.273825in}}%
\pgfpathlineto{\pgfqpoint{4.302530in}{3.274885in}}%
\pgfpathlineto{\pgfqpoint{4.303604in}{3.269857in}}%
\pgfpathlineto{\pgfqpoint{4.304678in}{3.271676in}}%
\pgfpathlineto{\pgfqpoint{4.311121in}{3.267642in}}%
\pgfpathlineto{\pgfqpoint{4.312195in}{3.270013in}}%
\pgfpathlineto{\pgfqpoint{4.313268in}{3.264111in}}%
\pgfpathlineto{\pgfqpoint{4.316490in}{3.263268in}}%
\pgfpathlineto{\pgfqpoint{4.317564in}{3.266535in}}%
\pgfpathlineto{\pgfqpoint{4.318638in}{3.261160in}}%
\pgfpathlineto{\pgfqpoint{4.319711in}{3.262163in}}%
\pgfpathlineto{\pgfqpoint{4.320785in}{3.265907in}}%
\pgfpathlineto{\pgfqpoint{4.324007in}{3.269247in}}%
\pgfpathlineto{\pgfqpoint{4.325081in}{3.273632in}}%
\pgfpathlineto{\pgfqpoint{4.326154in}{3.273423in}}%
\pgfpathlineto{\pgfqpoint{4.328302in}{3.266851in}}%
\pgfpathlineto{\pgfqpoint{4.332597in}{3.265579in}}%
\pgfpathlineto{\pgfqpoint{4.333671in}{3.264512in}}%
\pgfpathlineto{\pgfqpoint{4.334745in}{3.271715in}}%
\pgfpathlineto{\pgfqpoint{4.335819in}{3.273668in}}%
\pgfpathlineto{\pgfqpoint{4.339040in}{3.280966in}}%
\pgfpathlineto{\pgfqpoint{4.340114in}{3.280527in}}%
\pgfpathlineto{\pgfqpoint{4.341188in}{3.282393in}}%
\pgfpathlineto{\pgfqpoint{4.342262in}{3.285796in}}%
\pgfpathlineto{\pgfqpoint{4.343336in}{3.293260in}}%
\pgfpathlineto{\pgfqpoint{4.346557in}{3.293147in}}%
\pgfpathlineto{\pgfqpoint{4.349779in}{3.297553in}}%
\pgfpathlineto{\pgfqpoint{4.350853in}{3.293856in}}%
\pgfpathlineto{\pgfqpoint{4.354074in}{3.293971in}}%
\pgfpathlineto{\pgfqpoint{4.355148in}{3.288310in}}%
\pgfpathlineto{\pgfqpoint{4.356222in}{3.286809in}}%
\pgfpathlineto{\pgfqpoint{4.357296in}{3.280801in}}%
\pgfpathlineto{\pgfqpoint{4.358370in}{3.285538in}}%
\pgfpathlineto{\pgfqpoint{4.361591in}{3.286693in}}%
\pgfpathlineto{\pgfqpoint{4.362665in}{3.289027in}}%
\pgfpathlineto{\pgfqpoint{4.363739in}{3.281326in}}%
\pgfpathlineto{\pgfqpoint{4.364813in}{3.282638in}}%
\pgfpathlineto{\pgfqpoint{4.365886in}{3.277882in}}%
\pgfpathlineto{\pgfqpoint{4.369108in}{3.275560in}}%
\pgfpathlineto{\pgfqpoint{4.370182in}{3.273348in}}%
\pgfpathlineto{\pgfqpoint{4.371256in}{3.274454in}}%
\pgfpathlineto{\pgfqpoint{4.373403in}{3.266735in}}%
\pgfpathlineto{\pgfqpoint{4.376625in}{3.268901in}}%
\pgfpathlineto{\pgfqpoint{4.377699in}{3.268480in}}%
\pgfpathlineto{\pgfqpoint{4.378773in}{3.270374in}}%
\pgfpathlineto{\pgfqpoint{4.379846in}{3.270584in}}%
\pgfpathlineto{\pgfqpoint{4.380920in}{3.275222in}}%
\pgfpathlineto{\pgfqpoint{4.384142in}{3.274484in}}%
\pgfpathlineto{\pgfqpoint{4.386289in}{3.272088in}}%
\pgfpathlineto{\pgfqpoint{4.387363in}{3.271268in}}%
\pgfpathlineto{\pgfqpoint{4.388437in}{3.269742in}}%
\pgfpathlineto{\pgfqpoint{4.391659in}{3.271247in}}%
\pgfpathlineto{\pgfqpoint{4.393806in}{3.267789in}}%
\pgfpathlineto{\pgfqpoint{4.395954in}{3.272467in}}%
\pgfpathlineto{\pgfqpoint{4.399175in}{3.271637in}}%
\pgfpathlineto{\pgfqpoint{4.400249in}{3.270704in}}%
\pgfpathlineto{\pgfqpoint{4.401323in}{3.267489in}}%
\pgfpathlineto{\pgfqpoint{4.402397in}{3.261680in}}%
\pgfpathlineto{\pgfqpoint{4.403471in}{3.260643in}}%
\pgfpathlineto{\pgfqpoint{4.406692in}{3.260021in}}%
\pgfpathlineto{\pgfqpoint{4.407766in}{3.261369in}}%
\pgfpathlineto{\pgfqpoint{4.408840in}{3.264066in}}%
\pgfpathlineto{\pgfqpoint{4.409914in}{3.260554in}}%
\pgfpathlineto{\pgfqpoint{4.410988in}{3.265343in}}%
\pgfpathlineto{\pgfqpoint{4.415283in}{3.268855in}}%
\pgfpathlineto{\pgfqpoint{4.416357in}{3.268635in}}%
\pgfpathlineto{\pgfqpoint{4.418505in}{3.282934in}}%
\pgfpathlineto{\pgfqpoint{4.421726in}{3.274068in}}%
\pgfpathlineto{\pgfqpoint{4.422800in}{3.274777in}}%
\pgfpathlineto{\pgfqpoint{4.424948in}{3.277872in}}%
\pgfpathlineto{\pgfqpoint{4.426021in}{3.277752in}}%
\pgfpathlineto{\pgfqpoint{4.430317in}{3.278952in}}%
\pgfpathlineto{\pgfqpoint{4.431391in}{3.281116in}}%
\pgfpathlineto{\pgfqpoint{4.433538in}{3.281852in}}%
\pgfpathlineto{\pgfqpoint{4.436760in}{3.278651in}}%
\pgfpathlineto{\pgfqpoint{4.437834in}{3.276066in}}%
\pgfpathlineto{\pgfqpoint{4.438907in}{3.277174in}}%
\pgfpathlineto{\pgfqpoint{4.439981in}{3.272866in}}%
\pgfpathlineto{\pgfqpoint{4.441055in}{3.277012in}}%
\pgfpathlineto{\pgfqpoint{4.444277in}{3.277996in}}%
\pgfpathlineto{\pgfqpoint{4.445350in}{3.277135in}}%
\pgfpathlineto{\pgfqpoint{4.446424in}{3.280065in}}%
\pgfpathlineto{\pgfqpoint{4.447498in}{3.279313in}}%
\pgfpathlineto{\pgfqpoint{4.448572in}{3.281443in}}%
\pgfpathlineto{\pgfqpoint{4.451794in}{3.283197in}}%
\pgfpathlineto{\pgfqpoint{4.452867in}{3.276963in}}%
\pgfpathlineto{\pgfqpoint{4.456089in}{3.272638in}}%
\pgfpathlineto{\pgfqpoint{4.459310in}{3.275182in}}%
\pgfpathlineto{\pgfqpoint{4.460384in}{3.268821in}}%
\pgfpathlineto{\pgfqpoint{4.461458in}{3.274229in}}%
\pgfpathlineto{\pgfqpoint{4.462532in}{3.272965in}}%
\pgfpathlineto{\pgfqpoint{4.463606in}{3.268923in}}%
\pgfpathlineto{\pgfqpoint{4.466827in}{3.270944in}}%
\pgfpathlineto{\pgfqpoint{4.467901in}{3.269681in}}%
\pgfpathlineto{\pgfqpoint{4.468975in}{3.270930in}}%
\pgfpathlineto{\pgfqpoint{4.470049in}{3.273203in}}%
\pgfpathlineto{\pgfqpoint{4.471123in}{3.270551in}}%
\pgfpathlineto{\pgfqpoint{4.476492in}{3.273757in}}%
\pgfpathlineto{\pgfqpoint{4.477566in}{3.280233in}}%
\pgfpathlineto{\pgfqpoint{4.478639in}{3.282265in}}%
\pgfpathlineto{\pgfqpoint{4.481861in}{3.282265in}}%
\pgfpathlineto{\pgfqpoint{4.482935in}{3.283429in}}%
\pgfpathlineto{\pgfqpoint{4.484009in}{3.274810in}}%
\pgfpathlineto{\pgfqpoint{4.486156in}{3.273635in}}%
\pgfpathlineto{\pgfqpoint{4.489378in}{3.273243in}}%
\pgfpathlineto{\pgfqpoint{4.491526in}{3.264625in}}%
\pgfpathlineto{\pgfqpoint{4.492599in}{3.265408in}}%
\pgfpathlineto{\pgfqpoint{4.493673in}{3.263711in}}%
\pgfpathlineto{\pgfqpoint{4.496895in}{3.263068in}}%
\pgfpathlineto{\pgfqpoint{4.497969in}{3.270860in}}%
\pgfpathlineto{\pgfqpoint{4.499042in}{3.269210in}}%
\pgfpathlineto{\pgfqpoint{4.501190in}{3.273336in}}%
\pgfpathlineto{\pgfqpoint{4.505485in}{3.271007in}}%
\pgfpathlineto{\pgfqpoint{4.506559in}{3.269256in}}%
\pgfpathlineto{\pgfqpoint{4.507633in}{3.275352in}}%
\pgfpathlineto{\pgfqpoint{4.508707in}{3.275773in}}%
\pgfpathlineto{\pgfqpoint{4.511928in}{3.276053in}}%
\pgfpathlineto{\pgfqpoint{4.513002in}{3.280972in}}%
\pgfpathlineto{\pgfqpoint{4.514076in}{3.282378in}}%
\pgfpathlineto{\pgfqpoint{4.515150in}{3.285935in}}%
\pgfpathlineto{\pgfqpoint{4.516224in}{3.285935in}}%
\pgfpathlineto{\pgfqpoint{4.519445in}{3.286789in}}%
\pgfpathlineto{\pgfqpoint{4.520519in}{3.293381in}}%
\pgfpathlineto{\pgfqpoint{4.521593in}{3.293095in}}%
\pgfpathlineto{\pgfqpoint{4.522667in}{3.295240in}}%
\pgfpathlineto{\pgfqpoint{4.523741in}{3.295822in}}%
\pgfpathlineto{\pgfqpoint{4.526962in}{3.295385in}}%
\pgfpathlineto{\pgfqpoint{4.528036in}{3.290744in}}%
\pgfpathlineto{\pgfqpoint{4.529110in}{3.289208in}}%
\pgfpathlineto{\pgfqpoint{4.530184in}{3.290035in}}%
\pgfpathlineto{\pgfqpoint{4.531258in}{3.287121in}}%
\pgfpathlineto{\pgfqpoint{4.534479in}{3.285733in}}%
\pgfpathlineto{\pgfqpoint{4.536627in}{3.285733in}}%
\pgfpathlineto{\pgfqpoint{4.537701in}{3.287260in}}%
\pgfpathlineto{\pgfqpoint{4.538774in}{3.284026in}}%
\pgfpathlineto{\pgfqpoint{4.541996in}{3.288525in}}%
\pgfpathlineto{\pgfqpoint{4.544144in}{3.279808in}}%
\pgfpathlineto{\pgfqpoint{4.545217in}{3.280461in}}%
\pgfpathlineto{\pgfqpoint{4.546291in}{3.278753in}}%
\pgfpathlineto{\pgfqpoint{4.549513in}{3.279673in}}%
\pgfpathlineto{\pgfqpoint{4.550587in}{3.281380in}}%
\pgfpathlineto{\pgfqpoint{4.551661in}{3.285380in}}%
\pgfpathlineto{\pgfqpoint{4.553808in}{3.279581in}}%
\pgfpathlineto{\pgfqpoint{4.557030in}{3.279581in}}%
\pgfpathlineto{\pgfqpoint{4.558104in}{3.280722in}}%
\pgfpathlineto{\pgfqpoint{4.559177in}{3.283028in}}%
\pgfpathlineto{\pgfqpoint{4.560251in}{3.277007in}}%
\pgfpathlineto{\pgfqpoint{4.561325in}{3.291234in}}%
\pgfpathlineto{\pgfqpoint{4.566694in}{3.289716in}}%
\pgfpathlineto{\pgfqpoint{4.567768in}{3.287921in}}%
\pgfpathlineto{\pgfqpoint{4.568842in}{3.288059in}}%
\pgfpathlineto{\pgfqpoint{4.572063in}{3.287093in}}%
\pgfpathlineto{\pgfqpoint{4.573137in}{3.284492in}}%
\pgfpathlineto{\pgfqpoint{4.574211in}{3.283555in}}%
\pgfpathlineto{\pgfqpoint{4.576359in}{3.290703in}}%
\pgfpathlineto{\pgfqpoint{4.579580in}{3.292896in}}%
\pgfpathlineto{\pgfqpoint{4.581728in}{3.297920in}}%
\pgfpathlineto{\pgfqpoint{4.582802in}{3.295930in}}%
\pgfpathlineto{\pgfqpoint{4.583876in}{3.298062in}}%
\pgfpathlineto{\pgfqpoint{4.587097in}{3.297635in}}%
\pgfpathlineto{\pgfqpoint{4.588171in}{3.295794in}}%
\pgfpathlineto{\pgfqpoint{4.589245in}{3.295096in}}%
\pgfpathlineto{\pgfqpoint{4.590319in}{3.295374in}}%
\pgfpathlineto{\pgfqpoint{4.591393in}{3.294122in}}%
\pgfpathlineto{\pgfqpoint{4.595688in}{3.293844in}}%
\pgfpathlineto{\pgfqpoint{4.597836in}{3.284110in}}%
\pgfpathlineto{\pgfqpoint{4.598909in}{3.286057in}}%
\pgfpathlineto{\pgfqpoint{4.602131in}{3.288143in}}%
\pgfpathlineto{\pgfqpoint{4.603205in}{3.292248in}}%
\pgfpathlineto{\pgfqpoint{4.604279in}{3.288284in}}%
\pgfpathlineto{\pgfqpoint{4.605352in}{3.288832in}}%
\pgfpathlineto{\pgfqpoint{4.606426in}{3.290621in}}%
\pgfpathlineto{\pgfqpoint{4.609648in}{3.287869in}}%
\pgfpathlineto{\pgfqpoint{4.612869in}{3.281139in}}%
\pgfpathlineto{\pgfqpoint{4.613943in}{3.284439in}}%
\pgfpathlineto{\pgfqpoint{4.620386in}{3.286531in}}%
\pgfpathlineto{\pgfqpoint{4.621460in}{3.283641in}}%
\pgfpathlineto{\pgfqpoint{4.624682in}{3.282327in}}%
\pgfpathlineto{\pgfqpoint{4.625755in}{3.281014in}}%
\pgfpathlineto{\pgfqpoint{4.626829in}{3.281408in}}%
\pgfpathlineto{\pgfqpoint{4.627903in}{3.282853in}}%
\pgfpathlineto{\pgfqpoint{4.628977in}{3.285913in}}%
\pgfpathlineto{\pgfqpoint{4.632198in}{3.287775in}}%
\pgfpathlineto{\pgfqpoint{4.633272in}{3.294669in}}%
\pgfpathlineto{\pgfqpoint{4.634346in}{3.293993in}}%
\pgfpathlineto{\pgfqpoint{4.635420in}{3.293993in}}%
\pgfpathlineto{\pgfqpoint{4.639715in}{3.298429in}}%
\pgfpathlineto{\pgfqpoint{4.640789in}{3.298708in}}%
\pgfpathlineto{\pgfqpoint{4.641863in}{3.303449in}}%
\pgfpathlineto{\pgfqpoint{4.642937in}{3.304609in}}%
\pgfpathlineto{\pgfqpoint{4.644011in}{3.307072in}}%
\pgfpathlineto{\pgfqpoint{4.647232in}{3.308993in}}%
\pgfpathlineto{\pgfqpoint{4.649380in}{3.295181in}}%
\pgfpathlineto{\pgfqpoint{4.650454in}{3.299819in}}%
\pgfpathlineto{\pgfqpoint{4.651527in}{3.298029in}}%
\pgfpathlineto{\pgfqpoint{4.654749in}{3.294725in}}%
\pgfpathlineto{\pgfqpoint{4.655823in}{3.300783in}}%
\pgfpathlineto{\pgfqpoint{4.656897in}{3.300645in}}%
\pgfpathlineto{\pgfqpoint{4.659044in}{3.294933in}}%
\pgfpathlineto{\pgfqpoint{4.662266in}{3.290207in}}%
\pgfpathlineto{\pgfqpoint{4.664414in}{3.282248in}}%
\pgfpathlineto{\pgfqpoint{4.665487in}{3.284963in}}%
\pgfpathlineto{\pgfqpoint{4.666561in}{3.285205in}}%
\pgfpathlineto{\pgfqpoint{4.674078in}{3.277528in}}%
\pgfpathlineto{\pgfqpoint{4.677300in}{3.277981in}}%
\pgfpathlineto{\pgfqpoint{4.678373in}{3.280251in}}%
\pgfpathlineto{\pgfqpoint{4.681595in}{3.276990in}}%
\pgfpathlineto{\pgfqpoint{4.684816in}{3.280851in}}%
\pgfpathlineto{\pgfqpoint{4.685890in}{3.278567in}}%
\pgfpathlineto{\pgfqpoint{4.688038in}{3.284504in}}%
\pgfpathlineto{\pgfqpoint{4.689112in}{3.285438in}}%
\pgfpathlineto{\pgfqpoint{4.695555in}{3.284850in}}%
\pgfpathlineto{\pgfqpoint{4.696629in}{3.283570in}}%
\pgfpathlineto{\pgfqpoint{4.700924in}{3.285199in}}%
\pgfpathlineto{\pgfqpoint{4.701998in}{3.283337in}}%
\pgfpathlineto{\pgfqpoint{4.704146in}{3.284485in}}%
\pgfpathlineto{\pgfqpoint{4.707367in}{3.275280in}}%
\pgfpathlineto{\pgfqpoint{4.708441in}{3.274217in}}%
\pgfpathlineto{\pgfqpoint{4.709515in}{3.280008in}}%
\pgfpathlineto{\pgfqpoint{4.711662in}{3.279009in}}%
\pgfpathlineto{\pgfqpoint{4.714884in}{3.282892in}}%
\pgfpathlineto{\pgfqpoint{4.715958in}{3.285665in}}%
\pgfpathlineto{\pgfqpoint{4.717032in}{3.283279in}}%
\pgfpathlineto{\pgfqpoint{4.718105in}{3.288278in}}%
\pgfpathlineto{\pgfqpoint{4.722401in}{3.290559in}}%
\pgfpathlineto{\pgfqpoint{4.723475in}{3.287134in}}%
\pgfpathlineto{\pgfqpoint{4.724549in}{3.291460in}}%
\pgfpathlineto{\pgfqpoint{4.725622in}{3.283749in}}%
\pgfpathlineto{\pgfqpoint{4.726696in}{3.283864in}}%
\pgfpathlineto{\pgfqpoint{4.730992in}{3.292879in}}%
\pgfpathlineto{\pgfqpoint{4.732065in}{3.291546in}}%
\pgfpathlineto{\pgfqpoint{4.733139in}{3.291186in}}%
\pgfpathlineto{\pgfqpoint{4.734213in}{3.289394in}}%
\pgfpathlineto{\pgfqpoint{4.737435in}{3.292690in}}%
\pgfpathlineto{\pgfqpoint{4.738508in}{3.290995in}}%
\pgfpathlineto{\pgfqpoint{4.739582in}{3.291358in}}%
\pgfpathlineto{\pgfqpoint{4.741730in}{3.288332in}}%
\pgfpathlineto{\pgfqpoint{4.744951in}{3.296830in}}%
\pgfpathlineto{\pgfqpoint{4.746025in}{3.296323in}}%
\pgfpathlineto{\pgfqpoint{4.747099in}{3.297464in}}%
\pgfpathlineto{\pgfqpoint{4.748173in}{3.289727in}}%
\pgfpathlineto{\pgfqpoint{4.749247in}{3.290561in}}%
\pgfpathlineto{\pgfqpoint{4.752468in}{3.290921in}}%
\pgfpathlineto{\pgfqpoint{4.753542in}{3.287803in}}%
\pgfpathlineto{\pgfqpoint{4.754616in}{3.288737in}}%
\pgfpathlineto{\pgfqpoint{4.755690in}{3.279788in}}%
\pgfpathlineto{\pgfqpoint{4.756764in}{3.277786in}}%
\pgfpathlineto{\pgfqpoint{4.761059in}{3.274960in}}%
\pgfpathlineto{\pgfqpoint{4.762133in}{3.276726in}}%
\pgfpathlineto{\pgfqpoint{4.764281in}{3.273452in}}%
\pgfpathlineto{\pgfqpoint{4.767502in}{3.273681in}}%
\pgfpathlineto{\pgfqpoint{4.768576in}{3.275056in}}%
\pgfpathlineto{\pgfqpoint{4.769650in}{3.275514in}}%
\pgfpathlineto{\pgfqpoint{4.770724in}{3.274249in}}%
\pgfpathlineto{\pgfqpoint{4.775019in}{3.275054in}}%
\pgfpathlineto{\pgfqpoint{4.777167in}{3.272854in}}%
\pgfpathlineto{\pgfqpoint{4.778240in}{3.270423in}}%
\pgfpathlineto{\pgfqpoint{4.779314in}{3.271002in}}%
\pgfpathlineto{\pgfqpoint{4.782536in}{3.269728in}}%
\pgfpathlineto{\pgfqpoint{4.784683in}{3.271451in}}%
\pgfpathlineto{\pgfqpoint{4.785757in}{3.276771in}}%
\pgfpathlineto{\pgfqpoint{4.786831in}{3.275557in}}%
\pgfpathlineto{\pgfqpoint{4.790053in}{3.279320in}}%
\pgfpathlineto{\pgfqpoint{4.791126in}{3.276892in}}%
\pgfpathlineto{\pgfqpoint{4.792200in}{3.280339in}}%
\pgfpathlineto{\pgfqpoint{4.793274in}{3.281196in}}%
\pgfpathlineto{\pgfqpoint{4.794348in}{3.279236in}}%
\pgfpathlineto{\pgfqpoint{4.797570in}{3.281043in}}%
\pgfpathlineto{\pgfqpoint{4.798643in}{3.280308in}}%
\pgfpathlineto{\pgfqpoint{4.799717in}{3.281287in}}%
\pgfpathlineto{\pgfqpoint{4.800791in}{3.283613in}}%
\pgfpathlineto{\pgfqpoint{4.801865in}{3.282739in}}%
\pgfpathlineto{\pgfqpoint{4.806160in}{3.283738in}}%
\pgfpathlineto{\pgfqpoint{4.807234in}{3.277867in}}%
\pgfpathlineto{\pgfqpoint{4.809382in}{3.278938in}}%
\pgfpathlineto{\pgfqpoint{4.813677in}{3.275856in}}%
\pgfpathlineto{\pgfqpoint{4.814751in}{3.277479in}}%
\pgfpathlineto{\pgfqpoint{4.815825in}{3.275715in}}%
\pgfpathlineto{\pgfqpoint{4.816899in}{3.283711in}}%
\pgfpathlineto{\pgfqpoint{4.820120in}{3.284299in}}%
\pgfpathlineto{\pgfqpoint{4.821194in}{3.287845in}}%
\pgfpathlineto{\pgfqpoint{4.822268in}{3.286309in}}%
\pgfpathlineto{\pgfqpoint{4.823342in}{3.286192in}}%
\pgfpathlineto{\pgfqpoint{4.824415in}{3.287591in}}%
\pgfpathlineto{\pgfqpoint{4.827637in}{3.286175in}}%
\pgfpathlineto{\pgfqpoint{4.828711in}{3.286883in}}%
\pgfpathlineto{\pgfqpoint{4.829785in}{3.286883in}}%
\pgfpathlineto{\pgfqpoint{4.831932in}{3.279806in}}%
\pgfpathlineto{\pgfqpoint{4.835154in}{3.279216in}}%
\pgfpathlineto{\pgfqpoint{4.836228in}{3.278273in}}%
\pgfpathlineto{\pgfqpoint{4.837302in}{3.278862in}}%
\pgfpathlineto{\pgfqpoint{4.839449in}{3.283870in}}%
\pgfpathlineto{\pgfqpoint{4.842671in}{3.283749in}}%
\pgfpathlineto{\pgfqpoint{4.843745in}{3.287252in}}%
\pgfpathlineto{\pgfqpoint{4.844818in}{3.288373in}}%
\pgfpathlineto{\pgfqpoint{4.845892in}{3.283019in}}%
\pgfpathlineto{\pgfqpoint{4.846966in}{3.281472in}}%
\pgfpathlineto{\pgfqpoint{4.850188in}{3.279593in}}%
\pgfpathlineto{\pgfqpoint{4.851261in}{3.280171in}}%
\pgfpathlineto{\pgfqpoint{4.852335in}{3.283189in}}%
\pgfpathlineto{\pgfqpoint{4.853409in}{3.280983in}}%
\pgfpathlineto{\pgfqpoint{4.854483in}{3.281325in}}%
\pgfpathlineto{\pgfqpoint{4.858778in}{3.284410in}}%
\pgfpathlineto{\pgfqpoint{4.859852in}{3.281307in}}%
\pgfpathlineto{\pgfqpoint{4.860926in}{3.281652in}}%
\pgfpathlineto{\pgfqpoint{4.862000in}{3.277170in}}%
\pgfpathlineto{\pgfqpoint{4.865221in}{3.276065in}}%
\pgfpathlineto{\pgfqpoint{4.866295in}{3.274642in}}%
\pgfpathlineto{\pgfqpoint{4.868443in}{3.274426in}}%
\pgfpathlineto{\pgfqpoint{4.869517in}{3.273128in}}%
\pgfpathlineto{\pgfqpoint{4.873812in}{3.272479in}}%
\pgfpathlineto{\pgfqpoint{4.875960in}{3.276264in}}%
\pgfpathlineto{\pgfqpoint{4.877034in}{3.276484in}}%
\pgfpathlineto{\pgfqpoint{4.880255in}{3.275715in}}%
\pgfpathlineto{\pgfqpoint{4.882403in}{3.271820in}}%
\pgfpathlineto{\pgfqpoint{4.884550in}{3.274789in}}%
\pgfpathlineto{\pgfqpoint{4.887772in}{3.274146in}}%
\pgfpathlineto{\pgfqpoint{4.888846in}{3.273188in}}%
\pgfpathlineto{\pgfqpoint{4.889920in}{3.274455in}}%
\pgfpathlineto{\pgfqpoint{4.890993in}{3.280540in}}%
\pgfpathlineto{\pgfqpoint{4.892067in}{3.292925in}}%
\pgfpathlineto{\pgfqpoint{4.896363in}{3.290783in}}%
\pgfpathlineto{\pgfqpoint{4.897437in}{3.293282in}}%
\pgfpathlineto{\pgfqpoint{4.899584in}{3.293877in}}%
\pgfpathlineto{\pgfqpoint{4.902806in}{3.292442in}}%
\pgfpathlineto{\pgfqpoint{4.903880in}{3.290887in}}%
\pgfpathlineto{\pgfqpoint{4.904953in}{3.292801in}}%
\pgfpathlineto{\pgfqpoint{4.907101in}{3.282044in}}%
\pgfpathlineto{\pgfqpoint{4.910323in}{3.281060in}}%
\pgfpathlineto{\pgfqpoint{4.911396in}{3.282035in}}%
\pgfpathlineto{\pgfqpoint{4.912470in}{3.278321in}}%
\pgfpathlineto{\pgfqpoint{4.913544in}{3.278758in}}%
\pgfpathlineto{\pgfqpoint{4.914618in}{3.280397in}}%
\pgfpathlineto{\pgfqpoint{4.917839in}{3.281948in}}%
\pgfpathlineto{\pgfqpoint{4.918913in}{3.280064in}}%
\pgfpathlineto{\pgfqpoint{4.919987in}{3.275484in}}%
\pgfpathlineto{\pgfqpoint{4.921061in}{3.275589in}}%
\pgfpathlineto{\pgfqpoint{4.922135in}{3.274855in}}%
\pgfpathlineto{\pgfqpoint{4.925356in}{3.277162in}}%
\pgfpathlineto{\pgfqpoint{4.928578in}{3.273724in}}%
\pgfpathlineto{\pgfqpoint{4.932873in}{3.276792in}}%
\pgfpathlineto{\pgfqpoint{4.933947in}{3.275767in}}%
\pgfpathlineto{\pgfqpoint{4.936095in}{3.277291in}}%
\pgfpathlineto{\pgfqpoint{4.937169in}{3.277188in}}%
\pgfpathlineto{\pgfqpoint{4.940390in}{3.274407in}}%
\pgfpathlineto{\pgfqpoint{4.942538in}{3.276879in}}%
\pgfpathlineto{\pgfqpoint{4.943612in}{3.275837in}}%
\pgfpathlineto{\pgfqpoint{4.944685in}{3.277713in}}%
\pgfpathlineto{\pgfqpoint{4.947907in}{3.277921in}}%
\pgfpathlineto{\pgfqpoint{4.948981in}{3.286797in}}%
\pgfpathlineto{\pgfqpoint{4.950055in}{3.285335in}}%
\pgfpathlineto{\pgfqpoint{4.951128in}{3.286469in}}%
\pgfpathlineto{\pgfqpoint{4.952202in}{3.280326in}}%
\pgfpathlineto{\pgfqpoint{4.956498in}{3.278035in}}%
\pgfpathlineto{\pgfqpoint{4.959719in}{3.271162in}}%
\pgfpathlineto{\pgfqpoint{4.962941in}{3.270433in}}%
\pgfpathlineto{\pgfqpoint{4.967236in}{3.276306in}}%
\pgfpathlineto{\pgfqpoint{4.971531in}{3.269733in}}%
\pgfpathlineto{\pgfqpoint{4.972605in}{3.269086in}}%
\pgfpathlineto{\pgfqpoint{4.973679in}{3.273396in}}%
\pgfpathlineto{\pgfqpoint{4.974753in}{3.271349in}}%
\pgfpathlineto{\pgfqpoint{4.977974in}{3.270608in}}%
\pgfpathlineto{\pgfqpoint{4.980122in}{3.268307in}}%
\pgfpathlineto{\pgfqpoint{4.982270in}{3.266462in}}%
\pgfpathlineto{\pgfqpoint{4.986565in}{3.266462in}}%
\pgfpathlineto{\pgfqpoint{4.987639in}{3.264237in}}%
\pgfpathlineto{\pgfqpoint{4.989787in}{3.268990in}}%
\pgfpathlineto{\pgfqpoint{4.997303in}{3.272831in}}%
\pgfpathlineto{\pgfqpoint{5.000525in}{3.270540in}}%
\pgfpathlineto{\pgfqpoint{5.003747in}{3.273569in}}%
\pgfpathlineto{\pgfqpoint{5.004820in}{3.273464in}}%
\pgfpathlineto{\pgfqpoint{5.008042in}{3.271681in}}%
\pgfpathlineto{\pgfqpoint{5.011263in}{3.274693in}}%
\pgfpathlineto{\pgfqpoint{5.012337in}{3.274058in}}%
\pgfpathlineto{\pgfqpoint{5.015559in}{3.275857in}}%
\pgfpathlineto{\pgfqpoint{5.017706in}{3.273017in}}%
\pgfpathlineto{\pgfqpoint{5.018780in}{3.269407in}}%
\pgfpathlineto{\pgfqpoint{5.019854in}{3.271603in}}%
\pgfpathlineto{\pgfqpoint{5.023076in}{3.271196in}}%
\pgfpathlineto{\pgfqpoint{5.024149in}{3.274150in}}%
\pgfpathlineto{\pgfqpoint{5.025223in}{3.274863in}}%
\pgfpathlineto{\pgfqpoint{5.026297in}{3.274248in}}%
\pgfpathlineto{\pgfqpoint{5.027371in}{3.277016in}}%
\pgfpathlineto{\pgfqpoint{5.030592in}{3.275376in}}%
\pgfpathlineto{\pgfqpoint{5.031666in}{3.273860in}}%
\pgfpathlineto{\pgfqpoint{5.032740in}{3.274658in}}%
\pgfpathlineto{\pgfqpoint{5.034888in}{3.281184in}}%
\pgfpathlineto{\pgfqpoint{5.038109in}{3.284048in}}%
\pgfpathlineto{\pgfqpoint{5.039183in}{3.280366in}}%
\pgfpathlineto{\pgfqpoint{5.040257in}{3.279771in}}%
\pgfpathlineto{\pgfqpoint{5.041331in}{3.284404in}}%
\pgfpathlineto{\pgfqpoint{5.042405in}{3.282145in}}%
\pgfpathlineto{\pgfqpoint{5.045626in}{3.285533in}}%
\pgfpathlineto{\pgfqpoint{5.047774in}{3.282270in}}%
\pgfpathlineto{\pgfqpoint{5.048848in}{3.281372in}}%
\pgfpathlineto{\pgfqpoint{5.053143in}{3.282164in}}%
\pgfpathlineto{\pgfqpoint{5.055291in}{3.280469in}}%
\pgfpathlineto{\pgfqpoint{5.056365in}{3.280469in}}%
\pgfpathlineto{\pgfqpoint{5.057438in}{3.276280in}}%
\pgfpathlineto{\pgfqpoint{5.060660in}{3.276878in}}%
\pgfpathlineto{\pgfqpoint{5.062808in}{3.273807in}}%
\pgfpathlineto{\pgfqpoint{5.063881in}{3.275748in}}%
\pgfpathlineto{\pgfqpoint{5.064955in}{3.275353in}}%
\pgfpathlineto{\pgfqpoint{5.069251in}{3.275451in}}%
\pgfpathlineto{\pgfqpoint{5.070325in}{3.273970in}}%
\pgfpathlineto{\pgfqpoint{5.071398in}{3.274263in}}%
\pgfpathlineto{\pgfqpoint{5.072472in}{3.273969in}}%
\pgfpathlineto{\pgfqpoint{5.075694in}{3.274556in}}%
\pgfpathlineto{\pgfqpoint{5.076768in}{3.273872in}}%
\pgfpathlineto{\pgfqpoint{5.077841in}{3.274843in}}%
\pgfpathlineto{\pgfqpoint{5.079989in}{3.270532in}}%
\pgfpathlineto{\pgfqpoint{5.087506in}{3.275239in}}%
\pgfpathlineto{\pgfqpoint{5.093949in}{3.269613in}}%
\pgfpathlineto{\pgfqpoint{5.095023in}{3.266716in}}%
\pgfpathlineto{\pgfqpoint{5.098244in}{3.267171in}}%
\pgfpathlineto{\pgfqpoint{5.099318in}{3.270278in}}%
\pgfpathlineto{\pgfqpoint{5.101466in}{3.271193in}}%
\pgfpathlineto{\pgfqpoint{5.102540in}{3.267156in}}%
\pgfpathlineto{\pgfqpoint{5.105761in}{3.269102in}}%
\pgfpathlineto{\pgfqpoint{5.107909in}{3.273964in}}%
\pgfpathlineto{\pgfqpoint{5.108983in}{3.274148in}}%
\pgfpathlineto{\pgfqpoint{5.110057in}{3.258969in}}%
\pgfpathlineto{\pgfqpoint{5.113278in}{3.258007in}}%
\pgfpathlineto{\pgfqpoint{5.116500in}{3.259362in}}%
\pgfpathlineto{\pgfqpoint{5.117573in}{3.258564in}}%
\pgfpathlineto{\pgfqpoint{5.120795in}{3.258405in}}%
\pgfpathlineto{\pgfqpoint{5.122943in}{3.254025in}}%
\pgfpathlineto{\pgfqpoint{5.125090in}{3.255621in}}%
\pgfpathlineto{\pgfqpoint{5.130459in}{3.256385in}}%
\pgfpathlineto{\pgfqpoint{5.131533in}{3.257454in}}%
\pgfpathlineto{\pgfqpoint{5.135829in}{3.258217in}}%
\pgfpathlineto{\pgfqpoint{5.136903in}{3.256679in}}%
\pgfpathlineto{\pgfqpoint{5.137976in}{3.257294in}}%
\pgfpathlineto{\pgfqpoint{5.139050in}{3.254832in}}%
\pgfpathlineto{\pgfqpoint{5.140124in}{3.254907in}}%
\pgfpathlineto{\pgfqpoint{5.143346in}{3.256486in}}%
\pgfpathlineto{\pgfqpoint{5.144419in}{3.255057in}}%
\pgfpathlineto{\pgfqpoint{5.145493in}{3.236229in}}%
\pgfpathlineto{\pgfqpoint{5.146567in}{3.241275in}}%
\pgfpathlineto{\pgfqpoint{5.147641in}{3.241275in}}%
\pgfpathlineto{\pgfqpoint{5.150862in}{3.243290in}}%
\pgfpathlineto{\pgfqpoint{5.151936in}{3.239195in}}%
\pgfpathlineto{\pgfqpoint{5.153010in}{3.240752in}}%
\pgfpathlineto{\pgfqpoint{5.154084in}{3.239358in}}%
\pgfpathlineto{\pgfqpoint{5.159453in}{3.239042in}}%
\pgfpathlineto{\pgfqpoint{5.160527in}{3.239928in}}%
\pgfpathlineto{\pgfqpoint{5.161601in}{3.242461in}}%
\pgfpathlineto{\pgfqpoint{5.162675in}{3.241616in}}%
\pgfpathlineto{\pgfqpoint{5.165896in}{3.243372in}}%
\pgfpathlineto{\pgfqpoint{5.166970in}{3.249032in}}%
\pgfpathlineto{\pgfqpoint{5.168044in}{3.249170in}}%
\pgfpathlineto{\pgfqpoint{5.169118in}{3.250206in}}%
\pgfpathlineto{\pgfqpoint{5.170191in}{3.249787in}}%
\pgfpathlineto{\pgfqpoint{5.173413in}{3.252298in}}%
\pgfpathlineto{\pgfqpoint{5.174487in}{3.251601in}}%
\pgfpathlineto{\pgfqpoint{5.175561in}{3.245782in}}%
\pgfpathlineto{\pgfqpoint{5.177708in}{3.248016in}}%
\pgfpathlineto{\pgfqpoint{5.180930in}{3.246631in}}%
\pgfpathlineto{\pgfqpoint{5.183078in}{3.247935in}}%
\pgfpathlineto{\pgfqpoint{5.184151in}{3.244279in}}%
\pgfpathlineto{\pgfqpoint{5.185225in}{3.245411in}}%
\pgfpathlineto{\pgfqpoint{5.191668in}{3.243820in}}%
\pgfpathlineto{\pgfqpoint{5.192742in}{3.245284in}}%
\pgfpathlineto{\pgfqpoint{5.198111in}{3.247193in}}%
\pgfpathlineto{\pgfqpoint{5.199185in}{3.249076in}}%
\pgfpathlineto{\pgfqpoint{5.203480in}{3.246684in}}%
\pgfpathlineto{\pgfqpoint{5.204554in}{3.249028in}}%
\pgfpathlineto{\pgfqpoint{5.206702in}{3.249677in}}%
\pgfpathlineto{\pgfqpoint{5.207776in}{3.247405in}}%
\pgfpathlineto{\pgfqpoint{5.210997in}{3.246951in}}%
\pgfpathlineto{\pgfqpoint{5.212071in}{3.249158in}}%
\pgfpathlineto{\pgfqpoint{5.213145in}{3.248898in}}%
\pgfpathlineto{\pgfqpoint{5.214219in}{3.247798in}}%
\pgfpathlineto{\pgfqpoint{5.215293in}{3.249782in}}%
\pgfpathlineto{\pgfqpoint{5.218514in}{3.249194in}}%
\pgfpathlineto{\pgfqpoint{5.219588in}{3.247561in}}%
\pgfpathlineto{\pgfqpoint{5.220662in}{3.249390in}}%
\pgfpathlineto{\pgfqpoint{5.221736in}{3.252852in}}%
\pgfpathlineto{\pgfqpoint{5.222810in}{3.253597in}}%
\pgfpathlineto{\pgfqpoint{5.226031in}{3.249875in}}%
\pgfpathlineto{\pgfqpoint{5.227105in}{3.250332in}}%
\pgfpathlineto{\pgfqpoint{5.229253in}{3.242929in}}%
\pgfpathlineto{\pgfqpoint{5.230326in}{3.245877in}}%
\pgfpathlineto{\pgfqpoint{5.233548in}{3.245419in}}%
\pgfpathlineto{\pgfqpoint{5.236769in}{3.257669in}}%
\pgfpathlineto{\pgfqpoint{5.237843in}{3.247257in}}%
\pgfpathlineto{\pgfqpoint{5.241065in}{3.244513in}}%
\pgfpathlineto{\pgfqpoint{5.243213in}{3.253940in}}%
\pgfpathlineto{\pgfqpoint{5.244286in}{3.251119in}}%
\pgfpathlineto{\pgfqpoint{5.245360in}{3.254608in}}%
\pgfpathlineto{\pgfqpoint{5.248582in}{3.253346in}}%
\pgfpathlineto{\pgfqpoint{5.249656in}{3.246821in}}%
\pgfpathlineto{\pgfqpoint{5.251803in}{3.251339in}}%
\pgfpathlineto{\pgfqpoint{5.252877in}{3.248323in}}%
\pgfpathlineto{\pgfqpoint{5.256099in}{3.248187in}}%
\pgfpathlineto{\pgfqpoint{5.257172in}{3.244993in}}%
\pgfpathlineto{\pgfqpoint{5.258246in}{3.244007in}}%
\pgfpathlineto{\pgfqpoint{5.260394in}{3.260292in}}%
\pgfpathlineto{\pgfqpoint{5.263615in}{3.258461in}}%
\pgfpathlineto{\pgfqpoint{5.264689in}{3.257076in}}%
\pgfpathlineto{\pgfqpoint{5.265763in}{3.260354in}}%
\pgfpathlineto{\pgfqpoint{5.266837in}{3.260706in}}%
\pgfpathlineto{\pgfqpoint{5.267911in}{3.262187in}}%
\pgfpathlineto{\pgfqpoint{5.271132in}{3.260185in}}%
\pgfpathlineto{\pgfqpoint{5.272206in}{3.260471in}}%
\pgfpathlineto{\pgfqpoint{5.273280in}{3.260042in}}%
\pgfpathlineto{\pgfqpoint{5.275428in}{3.256511in}}%
\pgfpathlineto{\pgfqpoint{5.278649in}{3.254722in}}%
\pgfpathlineto{\pgfqpoint{5.279723in}{3.251949in}}%
\pgfpathlineto{\pgfqpoint{5.280797in}{3.254054in}}%
\pgfpathlineto{\pgfqpoint{5.281871in}{3.263463in}}%
\pgfpathlineto{\pgfqpoint{5.282945in}{3.265547in}}%
\pgfpathlineto{\pgfqpoint{5.286166in}{3.269454in}}%
\pgfpathlineto{\pgfqpoint{5.287240in}{3.269042in}}%
\pgfpathlineto{\pgfqpoint{5.288314in}{3.265560in}}%
\pgfpathlineto{\pgfqpoint{5.290461in}{3.263510in}}%
\pgfpathlineto{\pgfqpoint{5.293683in}{3.263964in}}%
\pgfpathlineto{\pgfqpoint{5.294757in}{3.266374in}}%
\pgfpathlineto{\pgfqpoint{5.295831in}{3.267287in}}%
\pgfpathlineto{\pgfqpoint{5.296904in}{3.267418in}}%
\pgfpathlineto{\pgfqpoint{5.297978in}{3.268666in}}%
\pgfpathlineto{\pgfqpoint{5.301200in}{3.271426in}}%
\pgfpathlineto{\pgfqpoint{5.303347in}{3.266843in}}%
\pgfpathlineto{\pgfqpoint{5.304421in}{3.268528in}}%
\pgfpathlineto{\pgfqpoint{5.305495in}{3.271291in}}%
\pgfpathlineto{\pgfqpoint{5.308717in}{3.269355in}}%
\pgfpathlineto{\pgfqpoint{5.309791in}{3.267212in}}%
\pgfpathlineto{\pgfqpoint{5.310864in}{3.271291in}}%
\pgfpathlineto{\pgfqpoint{5.311938in}{3.266520in}}%
\pgfpathlineto{\pgfqpoint{5.313012in}{3.270293in}}%
\pgfpathlineto{\pgfqpoint{5.316234in}{3.275299in}}%
\pgfpathlineto{\pgfqpoint{5.317307in}{3.273928in}}%
\pgfpathlineto{\pgfqpoint{5.318381in}{3.273928in}}%
\pgfpathlineto{\pgfqpoint{5.320529in}{3.274605in}}%
\pgfpathlineto{\pgfqpoint{5.323750in}{3.276773in}}%
\pgfpathlineto{\pgfqpoint{5.325898in}{3.271385in}}%
\pgfpathlineto{\pgfqpoint{5.328046in}{3.271800in}}%
\pgfpathlineto{\pgfqpoint{5.331267in}{3.274286in}}%
\pgfpathlineto{\pgfqpoint{5.332341in}{3.270118in}}%
\pgfpathlineto{\pgfqpoint{5.333415in}{3.274710in}}%
\pgfpathlineto{\pgfqpoint{5.334489in}{3.270542in}}%
\pgfpathlineto{\pgfqpoint{5.335563in}{3.270134in}}%
\pgfpathlineto{\pgfqpoint{5.338784in}{3.271083in}}%
\pgfpathlineto{\pgfqpoint{5.342006in}{3.268621in}}%
\pgfpathlineto{\pgfqpoint{5.343079in}{3.270194in}}%
\pgfpathlineto{\pgfqpoint{5.347375in}{3.272382in}}%
\pgfpathlineto{\pgfqpoint{5.348449in}{3.274754in}}%
\pgfpathlineto{\pgfqpoint{5.349523in}{3.272103in}}%
\pgfpathlineto{\pgfqpoint{5.350596in}{3.274828in}}%
\pgfpathlineto{\pgfqpoint{5.353818in}{3.270917in}}%
\pgfpathlineto{\pgfqpoint{5.354892in}{3.260999in}}%
\pgfpathlineto{\pgfqpoint{5.355966in}{3.258485in}}%
\pgfpathlineto{\pgfqpoint{5.357039in}{3.261139in}}%
\pgfpathlineto{\pgfqpoint{5.358113in}{3.268263in}}%
\pgfpathlineto{\pgfqpoint{5.361335in}{3.272226in}}%
\pgfpathlineto{\pgfqpoint{5.362409in}{3.271927in}}%
\pgfpathlineto{\pgfqpoint{5.363482in}{3.270958in}}%
\pgfpathlineto{\pgfqpoint{5.364556in}{3.268740in}}%
\pgfpathlineto{\pgfqpoint{5.365630in}{3.272869in}}%
\pgfpathlineto{\pgfqpoint{5.368852in}{3.270461in}}%
\pgfpathlineto{\pgfqpoint{5.369925in}{3.275502in}}%
\pgfpathlineto{\pgfqpoint{5.370999in}{3.276481in}}%
\pgfpathlineto{\pgfqpoint{5.372073in}{3.280352in}}%
\pgfpathlineto{\pgfqpoint{5.373147in}{3.278758in}}%
\pgfpathlineto{\pgfqpoint{5.377442in}{3.276288in}}%
\pgfpathlineto{\pgfqpoint{5.378516in}{3.279291in}}%
\pgfpathlineto{\pgfqpoint{5.379590in}{3.278839in}}%
\pgfpathlineto{\pgfqpoint{5.383885in}{3.284482in}}%
\pgfpathlineto{\pgfqpoint{5.386033in}{3.292137in}}%
\pgfpathlineto{\pgfqpoint{5.388181in}{3.287241in}}%
\pgfpathlineto{\pgfqpoint{5.391402in}{3.292939in}}%
\pgfpathlineto{\pgfqpoint{5.393550in}{3.292538in}}%
\pgfpathlineto{\pgfqpoint{5.395698in}{3.286059in}}%
\pgfpathlineto{\pgfqpoint{5.398919in}{3.282699in}}%
\pgfpathlineto{\pgfqpoint{5.399993in}{3.275500in}}%
\pgfpathlineto{\pgfqpoint{5.401067in}{3.279980in}}%
\pgfpathlineto{\pgfqpoint{5.402141in}{3.290779in}}%
\pgfpathlineto{\pgfqpoint{5.403214in}{3.291746in}}%
\pgfpathlineto{\pgfqpoint{5.406436in}{3.292450in}}%
\pgfpathlineto{\pgfqpoint{5.407510in}{3.290592in}}%
\pgfpathlineto{\pgfqpoint{5.410731in}{3.298508in}}%
\pgfpathlineto{\pgfqpoint{5.413953in}{3.299401in}}%
\pgfpathlineto{\pgfqpoint{5.415027in}{3.296162in}}%
\pgfpathlineto{\pgfqpoint{5.416101in}{3.289055in}}%
\pgfpathlineto{\pgfqpoint{5.417174in}{3.290584in}}%
\pgfpathlineto{\pgfqpoint{5.418248in}{3.275379in}}%
\pgfpathlineto{\pgfqpoint{5.421470in}{3.279180in}}%
\pgfpathlineto{\pgfqpoint{5.422544in}{3.277787in}}%
\pgfpathlineto{\pgfqpoint{5.423617in}{3.274510in}}%
\pgfpathlineto{\pgfqpoint{5.424691in}{3.274510in}}%
\pgfpathlineto{\pgfqpoint{5.428987in}{3.276149in}}%
\pgfpathlineto{\pgfqpoint{5.430060in}{3.274510in}}%
\pgfpathlineto{\pgfqpoint{5.431134in}{3.274188in}}%
\pgfpathlineto{\pgfqpoint{5.432208in}{3.274671in}}%
\pgfpathlineto{\pgfqpoint{5.433282in}{3.279603in}}%
\pgfpathlineto{\pgfqpoint{5.436503in}{3.281058in}}%
\pgfpathlineto{\pgfqpoint{5.437577in}{3.279338in}}%
\pgfpathlineto{\pgfqpoint{5.438651in}{3.289001in}}%
\pgfpathlineto{\pgfqpoint{5.439725in}{3.288756in}}%
\pgfpathlineto{\pgfqpoint{5.440799in}{3.291616in}}%
\pgfpathlineto{\pgfqpoint{5.444020in}{3.293542in}}%
\pgfpathlineto{\pgfqpoint{5.447242in}{3.300237in}}%
\pgfpathlineto{\pgfqpoint{5.448316in}{3.298148in}}%
\pgfpathlineto{\pgfqpoint{5.451537in}{3.301368in}}%
\pgfpathlineto{\pgfqpoint{5.452611in}{3.297364in}}%
\pgfpathlineto{\pgfqpoint{5.453685in}{3.298376in}}%
\pgfpathlineto{\pgfqpoint{5.454759in}{3.295826in}}%
\pgfpathlineto{\pgfqpoint{5.455833in}{3.302370in}}%
\pgfpathlineto{\pgfqpoint{5.459054in}{3.302285in}}%
\pgfpathlineto{\pgfqpoint{5.461202in}{3.294981in}}%
\pgfpathlineto{\pgfqpoint{5.462276in}{3.295066in}}%
\pgfpathlineto{\pgfqpoint{5.463349in}{3.299313in}}%
\pgfpathlineto{\pgfqpoint{5.466571in}{3.300162in}}%
\pgfpathlineto{\pgfqpoint{5.467645in}{3.296827in}}%
\pgfpathlineto{\pgfqpoint{5.468719in}{3.299820in}}%
\pgfpathlineto{\pgfqpoint{5.469792in}{3.297340in}}%
\pgfpathlineto{\pgfqpoint{5.470866in}{3.297172in}}%
\pgfpathlineto{\pgfqpoint{5.474088in}{3.294075in}}%
\pgfpathlineto{\pgfqpoint{5.476235in}{3.297610in}}%
\pgfpathlineto{\pgfqpoint{5.477309in}{3.296284in}}%
\pgfpathlineto{\pgfqpoint{5.478383in}{3.297022in}}%
\pgfpathlineto{\pgfqpoint{5.482679in}{3.294464in}}%
\pgfpathlineto{\pgfqpoint{5.483752in}{3.298919in}}%
\pgfpathlineto{\pgfqpoint{5.485900in}{3.293531in}}%
\pgfpathlineto{\pgfqpoint{5.489122in}{3.297321in}}%
\pgfpathlineto{\pgfqpoint{5.492343in}{3.285752in}}%
\pgfpathlineto{\pgfqpoint{5.493417in}{3.282330in}}%
\pgfpathlineto{\pgfqpoint{5.496638in}{3.278419in}}%
\pgfpathlineto{\pgfqpoint{5.497712in}{3.278093in}}%
\pgfpathlineto{\pgfqpoint{5.498786in}{3.282167in}}%
\pgfpathlineto{\pgfqpoint{5.499860in}{3.281923in}}%
\pgfpathlineto{\pgfqpoint{5.500934in}{3.285744in}}%
\pgfpathlineto{\pgfqpoint{5.504155in}{3.286248in}}%
\pgfpathlineto{\pgfqpoint{5.506303in}{3.282082in}}%
\pgfpathlineto{\pgfqpoint{5.507377in}{3.279001in}}%
\pgfpathlineto{\pgfqpoint{5.508451in}{3.281211in}}%
\pgfpathlineto{\pgfqpoint{5.511672in}{3.282580in}}%
\pgfpathlineto{\pgfqpoint{5.512746in}{3.283788in}}%
\pgfpathlineto{\pgfqpoint{5.513820in}{3.282486in}}%
\pgfpathlineto{\pgfqpoint{5.514894in}{3.283055in}}%
\pgfpathlineto{\pgfqpoint{5.515967in}{3.290052in}}%
\pgfpathlineto{\pgfqpoint{5.519189in}{3.285211in}}%
\pgfpathlineto{\pgfqpoint{5.520263in}{3.285471in}}%
\pgfpathlineto{\pgfqpoint{5.522411in}{3.290183in}}%
\pgfpathlineto{\pgfqpoint{5.526706in}{3.294222in}}%
\pgfpathlineto{\pgfqpoint{5.527780in}{3.293132in}}%
\pgfpathlineto{\pgfqpoint{5.529927in}{3.285321in}}%
\pgfpathlineto{\pgfqpoint{5.531001in}{3.287228in}}%
\pgfpathlineto{\pgfqpoint{5.534223in}{3.282687in}}%
\pgfpathlineto{\pgfqpoint{5.536370in}{3.283562in}}%
\pgfpathlineto{\pgfqpoint{5.537444in}{3.281893in}}%
\pgfpathlineto{\pgfqpoint{5.538518in}{3.285183in}}%
\pgfpathlineto{\pgfqpoint{5.541740in}{3.282243in}}%
\pgfpathlineto{\pgfqpoint{5.542813in}{3.279213in}}%
\pgfpathlineto{\pgfqpoint{5.543887in}{3.278322in}}%
\pgfpathlineto{\pgfqpoint{5.544961in}{3.278233in}}%
\pgfpathlineto{\pgfqpoint{5.546035in}{3.274045in}}%
\pgfpathlineto{\pgfqpoint{5.549256in}{3.276273in}}%
\pgfpathlineto{\pgfqpoint{5.550330in}{3.271461in}}%
\pgfpathlineto{\pgfqpoint{5.551404in}{3.271034in}}%
\pgfpathlineto{\pgfqpoint{5.553552in}{3.272056in}}%
\pgfpathlineto{\pgfqpoint{5.557847in}{3.270690in}}%
\pgfpathlineto{\pgfqpoint{5.558921in}{3.268914in}}%
\pgfpathlineto{\pgfqpoint{5.559995in}{3.269746in}}%
\pgfpathlineto{\pgfqpoint{5.561069in}{3.268824in}}%
\pgfpathlineto{\pgfqpoint{5.564290in}{3.274526in}}%
\pgfpathlineto{\pgfqpoint{5.567512in}{3.287890in}}%
\pgfpathlineto{\pgfqpoint{5.568586in}{3.289077in}}%
\pgfpathlineto{\pgfqpoint{5.571807in}{3.288621in}}%
\pgfpathlineto{\pgfqpoint{5.572881in}{3.289984in}}%
\pgfpathlineto{\pgfqpoint{5.575029in}{3.278673in}}%
\pgfpathlineto{\pgfqpoint{5.576102in}{3.270672in}}%
\pgfpathlineto{\pgfqpoint{5.579324in}{3.268097in}}%
\pgfpathlineto{\pgfqpoint{5.580398in}{3.264970in}}%
\pgfpathlineto{\pgfqpoint{5.581472in}{3.276741in}}%
\pgfpathlineto{\pgfqpoint{5.582545in}{3.273247in}}%
\pgfpathlineto{\pgfqpoint{5.583619in}{3.267720in}}%
\pgfpathlineto{\pgfqpoint{5.586841in}{3.266787in}}%
\pgfpathlineto{\pgfqpoint{5.588989in}{3.278355in}}%
\pgfpathlineto{\pgfqpoint{5.590062in}{3.274642in}}%
\pgfpathlineto{\pgfqpoint{5.591136in}{3.278919in}}%
\pgfpathlineto{\pgfqpoint{5.596505in}{3.288785in}}%
\pgfpathlineto{\pgfqpoint{5.597579in}{3.288966in}}%
\pgfpathlineto{\pgfqpoint{5.598653in}{3.287338in}}%
\pgfpathlineto{\pgfqpoint{5.601875in}{3.287963in}}%
\pgfpathlineto{\pgfqpoint{5.602948in}{3.290743in}}%
\pgfpathlineto{\pgfqpoint{5.604022in}{3.290474in}}%
\pgfpathlineto{\pgfqpoint{5.605096in}{3.290921in}}%
\pgfpathlineto{\pgfqpoint{5.606170in}{3.285352in}}%
\pgfpathlineto{\pgfqpoint{5.609391in}{3.286520in}}%
\pgfpathlineto{\pgfqpoint{5.610465in}{3.290472in}}%
\pgfpathlineto{\pgfqpoint{5.611539in}{3.291030in}}%
\pgfpathlineto{\pgfqpoint{5.613687in}{3.295910in}}%
\pgfpathlineto{\pgfqpoint{5.616908in}{3.296288in}}%
\pgfpathlineto{\pgfqpoint{5.617982in}{3.300368in}}%
\pgfpathlineto{\pgfqpoint{5.619056in}{3.292777in}}%
\pgfpathlineto{\pgfqpoint{5.620130in}{3.293848in}}%
\pgfpathlineto{\pgfqpoint{5.621204in}{3.297900in}}%
\pgfpathlineto{\pgfqpoint{5.624425in}{3.292228in}}%
\pgfpathlineto{\pgfqpoint{5.626573in}{3.283863in}}%
\pgfpathlineto{\pgfqpoint{5.627647in}{3.282421in}}%
\pgfpathlineto{\pgfqpoint{5.628721in}{3.285112in}}%
\pgfpathlineto{\pgfqpoint{5.631942in}{3.285598in}}%
\pgfpathlineto{\pgfqpoint{5.633016in}{3.286814in}}%
\pgfpathlineto{\pgfqpoint{5.634090in}{3.292381in}}%
\pgfpathlineto{\pgfqpoint{5.635164in}{3.292708in}}%
\pgfpathlineto{\pgfqpoint{5.636237in}{3.294843in}}%
\pgfpathlineto{\pgfqpoint{5.639459in}{3.290820in}}%
\pgfpathlineto{\pgfqpoint{5.640533in}{3.291932in}}%
\pgfpathlineto{\pgfqpoint{5.641607in}{3.291451in}}%
\pgfpathlineto{\pgfqpoint{5.642680in}{3.298826in}}%
\pgfpathlineto{\pgfqpoint{5.643754in}{3.295219in}}%
\pgfpathlineto{\pgfqpoint{5.646976in}{3.297399in}}%
\pgfpathlineto{\pgfqpoint{5.648050in}{3.296606in}}%
\pgfpathlineto{\pgfqpoint{5.649123in}{3.298270in}}%
\pgfpathlineto{\pgfqpoint{5.650197in}{3.303105in}}%
\pgfpathlineto{\pgfqpoint{5.651271in}{3.301869in}}%
\pgfpathlineto{\pgfqpoint{5.654493in}{3.303682in}}%
\pgfpathlineto{\pgfqpoint{5.656640in}{3.301547in}}%
\pgfpathlineto{\pgfqpoint{5.657714in}{3.302681in}}%
\pgfpathlineto{\pgfqpoint{5.663083in}{3.296874in}}%
\pgfpathlineto{\pgfqpoint{5.666305in}{3.288777in}}%
\pgfpathlineto{\pgfqpoint{5.669526in}{3.288696in}}%
\pgfpathlineto{\pgfqpoint{5.670600in}{3.292621in}}%
\pgfpathlineto{\pgfqpoint{5.671674in}{3.288777in}}%
\pgfpathlineto{\pgfqpoint{5.672748in}{3.280618in}}%
\pgfpathlineto{\pgfqpoint{5.673822in}{3.278253in}}%
\pgfpathlineto{\pgfqpoint{5.677043in}{3.279411in}}%
\pgfpathlineto{\pgfqpoint{5.678117in}{3.278607in}}%
\pgfpathlineto{\pgfqpoint{5.679191in}{3.279265in}}%
\pgfpathlineto{\pgfqpoint{5.681339in}{3.279265in}}%
\pgfpathlineto{\pgfqpoint{5.684560in}{3.281314in}}%
\pgfpathlineto{\pgfqpoint{5.685634in}{3.279192in}}%
\pgfpathlineto{\pgfqpoint{5.686708in}{3.280844in}}%
\pgfpathlineto{\pgfqpoint{5.687782in}{3.275670in}}%
\pgfpathlineto{\pgfqpoint{5.688856in}{3.281572in}}%
\pgfpathlineto{\pgfqpoint{5.692077in}{3.281281in}}%
\pgfpathlineto{\pgfqpoint{5.693151in}{3.282807in}}%
\pgfpathlineto{\pgfqpoint{5.695299in}{3.283470in}}%
\pgfpathlineto{\pgfqpoint{5.696372in}{3.280147in}}%
\pgfpathlineto{\pgfqpoint{5.699594in}{3.281476in}}%
\pgfpathlineto{\pgfqpoint{5.700668in}{3.276750in}}%
\pgfpathlineto{\pgfqpoint{5.701742in}{3.275970in}}%
\pgfpathlineto{\pgfqpoint{5.702815in}{3.278434in}}%
\pgfpathlineto{\pgfqpoint{5.703889in}{3.271671in}}%
\pgfpathlineto{\pgfqpoint{5.707111in}{3.274117in}}%
\pgfpathlineto{\pgfqpoint{5.709258in}{3.269273in}}%
\pgfpathlineto{\pgfqpoint{5.710332in}{3.269411in}}%
\pgfpathlineto{\pgfqpoint{5.714628in}{3.269135in}}%
\pgfpathlineto{\pgfqpoint{5.716775in}{3.275133in}}%
\pgfpathlineto{\pgfqpoint{5.717849in}{3.271738in}}%
\pgfpathlineto{\pgfqpoint{5.723218in}{3.267777in}}%
\pgfpathlineto{\pgfqpoint{5.724292in}{3.262966in}}%
\pgfpathlineto{\pgfqpoint{5.725366in}{3.255044in}}%
\pgfpathlineto{\pgfqpoint{5.726440in}{3.252992in}}%
\pgfpathlineto{\pgfqpoint{5.729661in}{3.256458in}}%
\pgfpathlineto{\pgfqpoint{5.730735in}{3.252497in}}%
\pgfpathlineto{\pgfqpoint{5.732883in}{3.262470in}}%
\pgfpathlineto{\pgfqpoint{5.733957in}{3.281581in}}%
\pgfpathlineto{\pgfqpoint{5.738252in}{3.281924in}}%
\pgfpathlineto{\pgfqpoint{5.739326in}{3.283550in}}%
\pgfpathlineto{\pgfqpoint{5.740400in}{3.284070in}}%
\pgfpathlineto{\pgfqpoint{5.741474in}{3.281987in}}%
\pgfpathlineto{\pgfqpoint{5.744695in}{3.284459in}}%
\pgfpathlineto{\pgfqpoint{5.745769in}{3.287072in}}%
\pgfpathlineto{\pgfqpoint{5.747917in}{3.289260in}}%
\pgfpathlineto{\pgfqpoint{5.748990in}{3.280918in}}%
\pgfpathlineto{\pgfqpoint{5.752212in}{3.282391in}}%
\pgfpathlineto{\pgfqpoint{5.753286in}{3.274764in}}%
\pgfpathlineto{\pgfqpoint{5.754360in}{3.273189in}}%
\pgfpathlineto{\pgfqpoint{5.755433in}{3.276505in}}%
\pgfpathlineto{\pgfqpoint{5.756507in}{3.282059in}}%
\pgfpathlineto{\pgfqpoint{5.760803in}{3.280230in}}%
\pgfpathlineto{\pgfqpoint{5.761877in}{3.275700in}}%
\pgfpathlineto{\pgfqpoint{5.762950in}{3.275613in}}%
\pgfpathlineto{\pgfqpoint{5.764024in}{3.278923in}}%
\pgfpathlineto{\pgfqpoint{5.768320in}{3.277791in}}%
\pgfpathlineto{\pgfqpoint{5.769393in}{3.272443in}}%
\pgfpathlineto{\pgfqpoint{5.770467in}{3.272772in}}%
\pgfpathlineto{\pgfqpoint{5.771541in}{3.267410in}}%
\pgfpathlineto{\pgfqpoint{5.774763in}{3.272195in}}%
\pgfpathlineto{\pgfqpoint{5.776910in}{3.279403in}}%
\pgfpathlineto{\pgfqpoint{5.777984in}{3.276063in}}%
\pgfpathlineto{\pgfqpoint{5.779058in}{3.274732in}}%
\pgfpathlineto{\pgfqpoint{5.782279in}{3.276294in}}%
\pgfpathlineto{\pgfqpoint{5.783353in}{3.282212in}}%
\pgfpathlineto{\pgfqpoint{5.784427in}{3.280878in}}%
\pgfpathlineto{\pgfqpoint{5.786575in}{3.280220in}}%
\pgfpathlineto{\pgfqpoint{5.789796in}{3.277926in}}%
\pgfpathlineto{\pgfqpoint{5.791944in}{3.283123in}}%
\pgfpathlineto{\pgfqpoint{5.794092in}{3.276629in}}%
\pgfpathlineto{\pgfqpoint{5.797313in}{3.278962in}}%
\pgfpathlineto{\pgfqpoint{5.798387in}{3.280551in}}%
\pgfpathlineto{\pgfqpoint{5.799461in}{3.280233in}}%
\pgfpathlineto{\pgfqpoint{5.800535in}{3.278174in}}%
\pgfpathlineto{\pgfqpoint{5.801609in}{3.273197in}}%
\pgfpathlineto{\pgfqpoint{5.804830in}{3.275504in}}%
\pgfpathlineto{\pgfqpoint{5.805904in}{3.275352in}}%
\pgfpathlineto{\pgfqpoint{5.806978in}{3.273148in}}%
\pgfpathlineto{\pgfqpoint{5.808052in}{3.272312in}}%
\pgfpathlineto{\pgfqpoint{5.812347in}{3.272464in}}%
\pgfpathlineto{\pgfqpoint{5.814495in}{3.266909in}}%
\pgfpathlineto{\pgfqpoint{5.815568in}{3.269292in}}%
\pgfpathlineto{\pgfqpoint{5.816642in}{3.269957in}}%
\pgfpathlineto{\pgfqpoint{5.819864in}{3.272985in}}%
\pgfpathlineto{\pgfqpoint{5.820938in}{3.272302in}}%
\pgfpathlineto{\pgfqpoint{5.822011in}{3.273517in}}%
\pgfpathlineto{\pgfqpoint{5.823085in}{3.277161in}}%
\pgfpathlineto{\pgfqpoint{5.824159in}{3.277789in}}%
\pgfpathlineto{\pgfqpoint{5.827381in}{3.277554in}}%
\pgfpathlineto{\pgfqpoint{5.829528in}{3.274288in}}%
\pgfpathlineto{\pgfqpoint{5.830602in}{3.276570in}}%
\pgfpathlineto{\pgfqpoint{5.831676in}{3.274163in}}%
\pgfpathlineto{\pgfqpoint{5.835971in}{3.275793in}}%
\pgfpathlineto{\pgfqpoint{5.837045in}{3.278596in}}%
\pgfpathlineto{\pgfqpoint{5.838119in}{3.278830in}}%
\pgfpathlineto{\pgfqpoint{5.839193in}{3.276489in}}%
\pgfpathlineto{\pgfqpoint{5.842414in}{3.274773in}}%
\pgfpathlineto{\pgfqpoint{5.843488in}{3.274851in}}%
\pgfpathlineto{\pgfqpoint{5.844562in}{3.277269in}}%
\pgfpathlineto{\pgfqpoint{5.845636in}{3.281795in}}%
\pgfpathlineto{\pgfqpoint{5.846710in}{3.275703in}}%
\pgfpathlineto{\pgfqpoint{5.849931in}{3.278140in}}%
\pgfpathlineto{\pgfqpoint{5.851005in}{3.280008in}}%
\pgfpathlineto{\pgfqpoint{5.852079in}{3.278109in}}%
\pgfpathlineto{\pgfqpoint{5.853153in}{3.278521in}}%
\pgfpathlineto{\pgfqpoint{5.854227in}{3.277448in}}%
\pgfpathlineto{\pgfqpoint{5.857448in}{3.279247in}}%
\pgfpathlineto{\pgfqpoint{5.858522in}{3.281824in}}%
\pgfpathlineto{\pgfqpoint{5.859596in}{3.282489in}}%
\pgfpathlineto{\pgfqpoint{5.860670in}{3.280232in}}%
\pgfpathlineto{\pgfqpoint{5.861744in}{3.281402in}}%
\pgfpathlineto{\pgfqpoint{5.864965in}{3.277724in}}%
\pgfpathlineto{\pgfqpoint{5.866039in}{3.280800in}}%
\pgfpathlineto{\pgfqpoint{5.867113in}{3.280884in}}%
\pgfpathlineto{\pgfqpoint{5.868187in}{3.278138in}}%
\pgfpathlineto{\pgfqpoint{5.869260in}{3.282132in}}%
\pgfpathlineto{\pgfqpoint{5.872482in}{3.281549in}}%
\pgfpathlineto{\pgfqpoint{5.873556in}{3.275258in}}%
\pgfpathlineto{\pgfqpoint{5.874630in}{3.272907in}}%
\pgfpathlineto{\pgfqpoint{5.876777in}{3.271606in}}%
\pgfpathlineto{\pgfqpoint{5.882146in}{3.271000in}}%
\pgfpathlineto{\pgfqpoint{5.883220in}{3.270322in}}%
\pgfpathlineto{\pgfqpoint{5.884294in}{3.271296in}}%
\pgfpathlineto{\pgfqpoint{5.887516in}{3.271750in}}%
\pgfpathlineto{\pgfqpoint{5.888589in}{3.270389in}}%
\pgfpathlineto{\pgfqpoint{5.891811in}{3.270015in}}%
\pgfpathlineto{\pgfqpoint{5.896106in}{3.269346in}}%
\pgfpathlineto{\pgfqpoint{5.897180in}{3.265808in}}%
\pgfpathlineto{\pgfqpoint{5.898254in}{3.266324in}}%
\pgfpathlineto{\pgfqpoint{5.899328in}{3.265881in}}%
\pgfpathlineto{\pgfqpoint{5.903623in}{3.262090in}}%
\pgfpathlineto{\pgfqpoint{5.904697in}{3.262303in}}%
\pgfpathlineto{\pgfqpoint{5.905771in}{3.266846in}}%
\pgfpathlineto{\pgfqpoint{5.906845in}{3.276217in}}%
\pgfpathlineto{\pgfqpoint{5.910066in}{3.270329in}}%
\pgfpathlineto{\pgfqpoint{5.911140in}{3.273660in}}%
\pgfpathlineto{\pgfqpoint{5.913288in}{3.262483in}}%
\pgfpathlineto{\pgfqpoint{5.914362in}{3.262832in}}%
\pgfpathlineto{\pgfqpoint{5.918657in}{3.262411in}}%
\pgfpathlineto{\pgfqpoint{5.919731in}{3.264236in}}%
\pgfpathlineto{\pgfqpoint{5.920805in}{3.262762in}}%
\pgfpathlineto{\pgfqpoint{5.921878in}{3.257641in}}%
\pgfpathlineto{\pgfqpoint{5.925100in}{3.255399in}}%
\pgfpathlineto{\pgfqpoint{5.926174in}{3.252046in}}%
\pgfpathlineto{\pgfqpoint{5.927248in}{3.251671in}}%
\pgfpathlineto{\pgfqpoint{5.928321in}{3.250552in}}%
\pgfpathlineto{\pgfqpoint{5.929395in}{3.251290in}}%
\pgfpathlineto{\pgfqpoint{5.932617in}{3.251166in}}%
\pgfpathlineto{\pgfqpoint{5.933691in}{3.251723in}}%
\pgfpathlineto{\pgfqpoint{5.934765in}{3.248627in}}%
\pgfpathlineto{\pgfqpoint{5.935838in}{3.256492in}}%
\pgfpathlineto{\pgfqpoint{5.936912in}{3.258834in}}%
\pgfpathlineto{\pgfqpoint{5.940134in}{3.258639in}}%
\pgfpathlineto{\pgfqpoint{5.941208in}{3.256237in}}%
\pgfpathlineto{\pgfqpoint{5.942281in}{3.257759in}}%
\pgfpathlineto{\pgfqpoint{5.943355in}{3.257373in}}%
\pgfpathlineto{\pgfqpoint{5.944429in}{3.257888in}}%
\pgfpathlineto{\pgfqpoint{5.947651in}{3.257888in}}%
\pgfpathlineto{\pgfqpoint{5.948724in}{3.256149in}}%
\pgfpathlineto{\pgfqpoint{5.949798in}{3.255827in}}%
\pgfpathlineto{\pgfqpoint{5.950872in}{3.257823in}}%
\pgfpathlineto{\pgfqpoint{5.951946in}{3.255441in}}%
\pgfpathlineto{\pgfqpoint{5.955167in}{3.255126in}}%
\pgfpathlineto{\pgfqpoint{5.956241in}{3.255816in}}%
\pgfpathlineto{\pgfqpoint{5.957315in}{3.253543in}}%
\pgfpathlineto{\pgfqpoint{5.959463in}{3.255059in}}%
\pgfpathlineto{\pgfqpoint{5.962684in}{3.257028in}}%
\pgfpathlineto{\pgfqpoint{5.963758in}{3.255249in}}%
\pgfpathlineto{\pgfqpoint{5.964832in}{3.256373in}}%
\pgfpathlineto{\pgfqpoint{5.965906in}{3.256121in}}%
\pgfpathlineto{\pgfqpoint{5.966980in}{3.257699in}}%
\pgfpathlineto{\pgfqpoint{5.971275in}{3.256754in}}%
\pgfpathlineto{\pgfqpoint{5.972349in}{3.258193in}}%
\pgfpathlineto{\pgfqpoint{5.973423in}{3.257812in}}%
\pgfpathlineto{\pgfqpoint{5.974497in}{3.258827in}}%
\pgfpathlineto{\pgfqpoint{5.978792in}{3.256303in}}%
\pgfpathlineto{\pgfqpoint{5.980940in}{3.254454in}}%
\pgfpathlineto{\pgfqpoint{5.982013in}{3.254150in}}%
\pgfpathlineto{\pgfqpoint{5.986309in}{3.251361in}}%
\pgfpathlineto{\pgfqpoint{5.987383in}{3.252010in}}%
\pgfpathlineto{\pgfqpoint{5.988456in}{3.251951in}}%
\pgfpathlineto{\pgfqpoint{5.989530in}{3.246431in}}%
\pgfpathlineto{\pgfqpoint{5.992752in}{3.249933in}}%
\pgfpathlineto{\pgfqpoint{5.993826in}{3.252485in}}%
\pgfpathlineto{\pgfqpoint{5.994899in}{3.252546in}}%
\pgfpathlineto{\pgfqpoint{5.995973in}{3.257845in}}%
\pgfpathlineto{\pgfqpoint{5.997047in}{3.251632in}}%
\pgfpathlineto{\pgfqpoint{6.000269in}{3.254327in}}%
\pgfpathlineto{\pgfqpoint{6.001343in}{3.254209in}}%
\pgfpathlineto{\pgfqpoint{6.002416in}{3.255917in}}%
\pgfpathlineto{\pgfqpoint{6.003490in}{3.255387in}}%
\pgfpathlineto{\pgfqpoint{6.004564in}{3.257321in}}%
\pgfpathlineto{\pgfqpoint{6.007786in}{3.254335in}}%
\pgfpathlineto{\pgfqpoint{6.008859in}{3.257261in}}%
\pgfpathlineto{\pgfqpoint{6.009933in}{3.255828in}}%
\pgfpathlineto{\pgfqpoint{6.011007in}{3.256476in}}%
\pgfpathlineto{\pgfqpoint{6.012081in}{3.258847in}}%
\pgfpathlineto{\pgfqpoint{6.015302in}{3.259380in}}%
\pgfpathlineto{\pgfqpoint{6.016376in}{3.258725in}}%
\pgfpathlineto{\pgfqpoint{6.017450in}{3.261227in}}%
\pgfpathlineto{\pgfqpoint{6.019598in}{3.260632in}}%
\pgfpathlineto{\pgfqpoint{6.022819in}{3.261047in}}%
\pgfpathlineto{\pgfqpoint{6.023893in}{3.256883in}}%
\pgfpathlineto{\pgfqpoint{6.026041in}{3.255277in}}%
\pgfpathlineto{\pgfqpoint{6.027115in}{3.257894in}}%
\pgfpathlineto{\pgfqpoint{6.030336in}{3.258786in}}%
\pgfpathlineto{\pgfqpoint{6.031410in}{3.261366in}}%
\pgfpathlineto{\pgfqpoint{6.032484in}{3.273785in}}%
\pgfpathlineto{\pgfqpoint{6.033558in}{3.273244in}}%
\pgfpathlineto{\pgfqpoint{6.034632in}{3.271555in}}%
\pgfpathlineto{\pgfqpoint{6.037853in}{3.272231in}}%
\pgfpathlineto{\pgfqpoint{6.038927in}{3.273244in}}%
\pgfpathlineto{\pgfqpoint{6.041075in}{3.271403in}}%
\pgfpathlineto{\pgfqpoint{6.042148in}{3.270994in}}%
\pgfpathlineto{\pgfqpoint{6.045370in}{3.271812in}}%
\pgfpathlineto{\pgfqpoint{6.046444in}{3.274063in}}%
\pgfpathlineto{\pgfqpoint{6.047518in}{3.274620in}}%
\pgfpathlineto{\pgfqpoint{6.048591in}{3.277265in}}%
\pgfpathlineto{\pgfqpoint{6.049665in}{3.275126in}}%
\pgfpathlineto{\pgfqpoint{6.052887in}{3.282254in}}%
\pgfpathlineto{\pgfqpoint{6.055034in}{3.281827in}}%
\pgfpathlineto{\pgfqpoint{6.056108in}{3.280193in}}%
\pgfpathlineto{\pgfqpoint{6.057182in}{3.280903in}}%
\pgfpathlineto{\pgfqpoint{6.060404in}{3.281756in}}%
\pgfpathlineto{\pgfqpoint{6.061477in}{3.284617in}}%
\pgfpathlineto{\pgfqpoint{6.062551in}{3.285118in}}%
\pgfpathlineto{\pgfqpoint{6.064699in}{3.286843in}}%
\pgfpathlineto{\pgfqpoint{6.067920in}{3.287059in}}%
\pgfpathlineto{\pgfqpoint{6.068994in}{3.283668in}}%
\pgfpathlineto{\pgfqpoint{6.070068in}{3.285490in}}%
\pgfpathlineto{\pgfqpoint{6.072216in}{3.287057in}}%
\pgfpathlineto{\pgfqpoint{6.075437in}{3.286559in}}%
\pgfpathlineto{\pgfqpoint{6.076511in}{3.287906in}}%
\pgfpathlineto{\pgfqpoint{6.078659in}{3.277576in}}%
\pgfpathlineto{\pgfqpoint{6.079733in}{3.280230in}}%
\pgfpathlineto{\pgfqpoint{6.082954in}{3.278723in}}%
\pgfpathlineto{\pgfqpoint{6.084028in}{3.276529in}}%
\pgfpathlineto{\pgfqpoint{6.085102in}{3.271459in}}%
\pgfpathlineto{\pgfqpoint{6.087250in}{3.269871in}}%
\pgfpathlineto{\pgfqpoint{6.090471in}{3.268628in}}%
\pgfpathlineto{\pgfqpoint{6.091545in}{3.263583in}}%
\pgfpathlineto{\pgfqpoint{6.093693in}{3.266505in}}%
\pgfpathlineto{\pgfqpoint{6.094766in}{3.269322in}}%
\pgfpathlineto{\pgfqpoint{6.097988in}{3.272792in}}%
\pgfpathlineto{\pgfqpoint{6.099062in}{3.270864in}}%
\pgfpathlineto{\pgfqpoint{6.100136in}{3.272253in}}%
\pgfpathlineto{\pgfqpoint{6.102283in}{3.272189in}}%
\pgfpathlineto{\pgfqpoint{6.106579in}{3.271550in}}%
\pgfpathlineto{\pgfqpoint{6.107653in}{3.274157in}}%
\pgfpathlineto{\pgfqpoint{6.108726in}{3.274352in}}%
\pgfpathlineto{\pgfqpoint{6.109800in}{3.276762in}}%
\pgfpathlineto{\pgfqpoint{6.114096in}{3.278825in}}%
\pgfpathlineto{\pgfqpoint{6.115169in}{3.280023in}}%
\pgfpathlineto{\pgfqpoint{6.116243in}{3.279686in}}%
\pgfpathlineto{\pgfqpoint{6.117317in}{3.280493in}}%
\pgfpathlineto{\pgfqpoint{6.120539in}{3.279686in}}%
\pgfpathlineto{\pgfqpoint{6.121612in}{3.280154in}}%
\pgfpathlineto{\pgfqpoint{6.123760in}{3.284311in}}%
\pgfpathlineto{\pgfqpoint{6.124834in}{3.284855in}}%
\pgfpathlineto{\pgfqpoint{6.130203in}{3.284515in}}%
\pgfpathlineto{\pgfqpoint{6.131277in}{3.283293in}}%
\pgfpathlineto{\pgfqpoint{6.132351in}{3.285670in}}%
\pgfpathlineto{\pgfqpoint{6.135572in}{3.286756in}}%
\pgfpathlineto{\pgfqpoint{6.136646in}{3.292172in}}%
\pgfpathlineto{\pgfqpoint{6.137720in}{3.291007in}}%
\pgfpathlineto{\pgfqpoint{6.138794in}{3.292568in}}%
\pgfpathlineto{\pgfqpoint{6.139868in}{3.295251in}}%
\pgfpathlineto{\pgfqpoint{6.143089in}{3.298828in}}%
\pgfpathlineto{\pgfqpoint{6.145237in}{3.292948in}}%
\pgfpathlineto{\pgfqpoint{6.146311in}{3.293940in}}%
\pgfpathlineto{\pgfqpoint{6.147385in}{3.293231in}}%
\pgfpathlineto{\pgfqpoint{6.151680in}{3.295493in}}%
\pgfpathlineto{\pgfqpoint{6.152754in}{3.295279in}}%
\pgfpathlineto{\pgfqpoint{6.153828in}{3.301481in}}%
\pgfpathlineto{\pgfqpoint{6.154901in}{3.300656in}}%
\pgfpathlineto{\pgfqpoint{6.158123in}{3.303882in}}%
\pgfpathlineto{\pgfqpoint{6.160271in}{3.302088in}}%
\pgfpathlineto{\pgfqpoint{6.161344in}{3.299573in}}%
\pgfpathlineto{\pgfqpoint{6.162418in}{3.299138in}}%
\pgfpathlineto{\pgfqpoint{6.166714in}{3.298849in}}%
\pgfpathlineto{\pgfqpoint{6.167787in}{3.301876in}}%
\pgfpathlineto{\pgfqpoint{6.168861in}{3.302615in}}%
\pgfpathlineto{\pgfqpoint{6.169935in}{3.300177in}}%
\pgfpathlineto{\pgfqpoint{6.173157in}{3.300322in}}%
\pgfpathlineto{\pgfqpoint{6.175304in}{3.296406in}}%
\pgfpathlineto{\pgfqpoint{6.177452in}{3.296189in}}%
\pgfpathlineto{\pgfqpoint{6.180674in}{3.293941in}}%
\pgfpathlineto{\pgfqpoint{6.183895in}{3.298088in}}%
\pgfpathlineto{\pgfqpoint{6.184969in}{3.297429in}}%
\pgfpathlineto{\pgfqpoint{6.188190in}{3.302602in}}%
\pgfpathlineto{\pgfqpoint{6.191412in}{3.303515in}}%
\pgfpathlineto{\pgfqpoint{6.192486in}{3.302523in}}%
\pgfpathlineto{\pgfqpoint{6.195707in}{3.301387in}}%
\pgfpathlineto{\pgfqpoint{6.197855in}{3.306542in}}%
\pgfpathlineto{\pgfqpoint{6.198929in}{3.307233in}}%
\pgfpathlineto{\pgfqpoint{6.200003in}{3.306384in}}%
\pgfpathlineto{\pgfqpoint{6.203224in}{3.308082in}}%
\pgfpathlineto{\pgfqpoint{6.204298in}{3.306539in}}%
\pgfpathlineto{\pgfqpoint{6.205372in}{3.306767in}}%
\pgfpathlineto{\pgfqpoint{6.206446in}{3.308066in}}%
\pgfpathlineto{\pgfqpoint{6.207520in}{3.305775in}}%
\pgfpathlineto{\pgfqpoint{6.210741in}{3.305175in}}%
\pgfpathlineto{\pgfqpoint{6.211815in}{3.304353in}}%
\pgfpathlineto{\pgfqpoint{6.212889in}{3.304724in}}%
\pgfpathlineto{\pgfqpoint{6.213963in}{3.303385in}}%
\pgfpathlineto{\pgfqpoint{6.215036in}{3.303385in}}%
\pgfpathlineto{\pgfqpoint{6.220406in}{3.300558in}}%
\pgfpathlineto{\pgfqpoint{6.221479in}{3.297955in}}%
\pgfpathlineto{\pgfqpoint{6.225775in}{3.299517in}}%
\pgfpathlineto{\pgfqpoint{6.227922in}{3.296483in}}%
\pgfpathlineto{\pgfqpoint{6.230070in}{3.293739in}}%
\pgfpathlineto{\pgfqpoint{6.235439in}{3.289724in}}%
\pgfpathlineto{\pgfqpoint{6.236513in}{3.286498in}}%
\pgfpathlineto{\pgfqpoint{6.237587in}{3.294514in}}%
\pgfpathlineto{\pgfqpoint{6.240809in}{3.295587in}}%
\pgfpathlineto{\pgfqpoint{6.241882in}{3.291152in}}%
\pgfpathlineto{\pgfqpoint{6.242956in}{3.289290in}}%
\pgfpathlineto{\pgfqpoint{6.244030in}{3.290172in}}%
\pgfpathlineto{\pgfqpoint{6.245104in}{3.290036in}}%
\pgfpathlineto{\pgfqpoint{6.249399in}{3.287094in}}%
\pgfpathlineto{\pgfqpoint{6.251547in}{3.282716in}}%
\pgfpathlineto{\pgfqpoint{6.252621in}{3.281689in}}%
\pgfpathlineto{\pgfqpoint{6.255842in}{3.282305in}}%
\pgfpathlineto{\pgfqpoint{6.256916in}{3.281074in}}%
\pgfpathlineto{\pgfqpoint{6.257990in}{3.286015in}}%
\pgfpathlineto{\pgfqpoint{6.259064in}{3.287216in}}%
\pgfpathlineto{\pgfqpoint{6.260138in}{3.286015in}}%
\pgfpathlineto{\pgfqpoint{6.264433in}{3.283019in}}%
\pgfpathlineto{\pgfqpoint{6.266581in}{3.280442in}}%
\pgfpathlineto{\pgfqpoint{6.271950in}{3.279908in}}%
\pgfpathlineto{\pgfqpoint{6.274097in}{3.279575in}}%
\pgfpathlineto{\pgfqpoint{6.275171in}{3.280775in}}%
\pgfpathlineto{\pgfqpoint{6.278393in}{3.280642in}}%
\pgfpathlineto{\pgfqpoint{6.280541in}{3.282849in}}%
\pgfpathlineto{\pgfqpoint{6.281614in}{3.281434in}}%
\pgfpathlineto{\pgfqpoint{6.282688in}{3.286293in}}%
\pgfpathlineto{\pgfqpoint{6.285910in}{3.286431in}}%
\pgfpathlineto{\pgfqpoint{6.286984in}{3.285459in}}%
\pgfpathlineto{\pgfqpoint{6.288057in}{3.287732in}}%
\pgfpathlineto{\pgfqpoint{6.290205in}{3.285625in}}%
\pgfpathlineto{\pgfqpoint{6.293427in}{3.287591in}}%
\pgfpathlineto{\pgfqpoint{6.294500in}{3.291874in}}%
\pgfpathlineto{\pgfqpoint{6.296648in}{3.288379in}}%
\pgfpathlineto{\pgfqpoint{6.297722in}{3.287214in}}%
\pgfpathlineto{\pgfqpoint{6.300943in}{3.286413in}}%
\pgfpathlineto{\pgfqpoint{6.302017in}{3.283501in}}%
\pgfpathlineto{\pgfqpoint{6.305239in}{3.293239in}}%
\pgfpathlineto{\pgfqpoint{6.308460in}{3.295282in}}%
\pgfpathlineto{\pgfqpoint{6.310608in}{3.301670in}}%
\pgfpathlineto{\pgfqpoint{6.311682in}{3.306826in}}%
\pgfpathlineto{\pgfqpoint{6.312756in}{3.308751in}}%
\pgfpathlineto{\pgfqpoint{6.315977in}{3.310516in}}%
\pgfpathlineto{\pgfqpoint{6.317051in}{3.312630in}}%
\pgfpathlineto{\pgfqpoint{6.318125in}{3.310109in}}%
\pgfpathlineto{\pgfqpoint{6.319199in}{3.310189in}}%
\pgfpathlineto{\pgfqpoint{6.320273in}{3.313541in}}%
\pgfpathlineto{\pgfqpoint{6.324568in}{3.315541in}}%
\pgfpathlineto{\pgfqpoint{6.325642in}{3.315299in}}%
\pgfpathlineto{\pgfqpoint{6.326716in}{3.313847in}}%
\pgfpathlineto{\pgfqpoint{6.327789in}{3.314007in}}%
\pgfpathlineto{\pgfqpoint{6.331011in}{3.312330in}}%
\pgfpathlineto{\pgfqpoint{6.332085in}{3.313607in}}%
\pgfpathlineto{\pgfqpoint{6.333159in}{3.313048in}}%
\pgfpathlineto{\pgfqpoint{6.335306in}{3.308866in}}%
\pgfpathlineto{\pgfqpoint{6.338528in}{3.307712in}}%
\pgfpathlineto{\pgfqpoint{6.339602in}{3.301378in}}%
\pgfpathlineto{\pgfqpoint{6.340675in}{3.299345in}}%
\pgfpathlineto{\pgfqpoint{6.341749in}{3.298559in}}%
\pgfpathlineto{\pgfqpoint{6.342823in}{3.299837in}}%
\pgfpathlineto{\pgfqpoint{6.347119in}{3.300913in}}%
\pgfpathlineto{\pgfqpoint{6.348192in}{3.302134in}}%
\pgfpathlineto{\pgfqpoint{6.349266in}{3.305221in}}%
\pgfpathlineto{\pgfqpoint{6.350340in}{3.303308in}}%
\pgfpathlineto{\pgfqpoint{6.353562in}{3.306620in}}%
\pgfpathlineto{\pgfqpoint{6.354635in}{3.308975in}}%
\pgfpathlineto{\pgfqpoint{6.355709in}{3.307626in}}%
\pgfpathlineto{\pgfqpoint{6.356783in}{3.303052in}}%
\pgfpathlineto{\pgfqpoint{6.357857in}{3.301927in}}%
\pgfpathlineto{\pgfqpoint{6.361078in}{3.301252in}}%
\pgfpathlineto{\pgfqpoint{6.362152in}{3.299303in}}%
\pgfpathlineto{\pgfqpoint{6.368595in}{3.299902in}}%
\pgfpathlineto{\pgfqpoint{6.369669in}{3.300504in}}%
\pgfpathlineto{\pgfqpoint{6.371817in}{3.298110in}}%
\pgfpathlineto{\pgfqpoint{6.372891in}{3.297962in}}%
\pgfpathlineto{\pgfqpoint{6.377186in}{3.298478in}}%
\pgfpathlineto{\pgfqpoint{6.378260in}{3.303728in}}%
\pgfpathlineto{\pgfqpoint{6.379334in}{3.305281in}}%
\pgfpathlineto{\pgfqpoint{6.380408in}{3.302810in}}%
\pgfpathlineto{\pgfqpoint{6.383629in}{3.306928in}}%
\pgfpathlineto{\pgfqpoint{6.385777in}{3.302990in}}%
\pgfpathlineto{\pgfqpoint{6.386851in}{3.301974in}}%
\pgfpathlineto{\pgfqpoint{6.387924in}{3.298375in}}%
\pgfpathlineto{\pgfqpoint{6.391146in}{3.298375in}}%
\pgfpathlineto{\pgfqpoint{6.393294in}{3.300963in}}%
\pgfpathlineto{\pgfqpoint{6.395441in}{3.301950in}}%
\pgfpathlineto{\pgfqpoint{6.398663in}{3.301809in}}%
\pgfpathlineto{\pgfqpoint{6.399737in}{3.303927in}}%
\pgfpathlineto{\pgfqpoint{6.400810in}{3.303504in}}%
\pgfpathlineto{\pgfqpoint{6.407253in}{3.291568in}}%
\pgfpathlineto{\pgfqpoint{6.408327in}{3.291824in}}%
\pgfpathlineto{\pgfqpoint{6.409401in}{3.293040in}}%
\pgfpathlineto{\pgfqpoint{6.413697in}{3.291062in}}%
\pgfpathlineto{\pgfqpoint{6.414770in}{3.292322in}}%
\pgfpathlineto{\pgfqpoint{6.415844in}{3.290221in}}%
\pgfpathlineto{\pgfqpoint{6.416918in}{3.289520in}}%
\pgfpathlineto{\pgfqpoint{6.417992in}{3.292449in}}%
\pgfpathlineto{\pgfqpoint{6.422287in}{3.291749in}}%
\pgfpathlineto{\pgfqpoint{6.423361in}{3.288964in}}%
\pgfpathlineto{\pgfqpoint{6.424435in}{3.289892in}}%
\pgfpathlineto{\pgfqpoint{6.425509in}{3.291887in}}%
\pgfpathlineto{\pgfqpoint{6.429804in}{3.288785in}}%
\pgfpathlineto{\pgfqpoint{6.430878in}{3.289817in}}%
\pgfpathlineto{\pgfqpoint{6.431952in}{3.293187in}}%
\pgfpathlineto{\pgfqpoint{6.433026in}{3.275420in}}%
\pgfpathlineto{\pgfqpoint{6.436247in}{3.275578in}}%
\pgfpathlineto{\pgfqpoint{6.437321in}{3.281242in}}%
\pgfpathlineto{\pgfqpoint{6.438395in}{3.275102in}}%
\pgfpathlineto{\pgfqpoint{6.439469in}{3.273246in}}%
\pgfpathlineto{\pgfqpoint{6.440542in}{3.276847in}}%
\pgfpathlineto{\pgfqpoint{6.443764in}{3.279957in}}%
\pgfpathlineto{\pgfqpoint{6.444838in}{3.279549in}}%
\pgfpathlineto{\pgfqpoint{6.445912in}{3.279955in}}%
\pgfpathlineto{\pgfqpoint{6.446985in}{3.278018in}}%
\pgfpathlineto{\pgfqpoint{6.448059in}{3.274501in}}%
\pgfpathlineto{\pgfqpoint{6.451281in}{3.275317in}}%
\pgfpathlineto{\pgfqpoint{6.452355in}{3.274756in}}%
\pgfpathlineto{\pgfqpoint{6.454502in}{3.277663in}}%
\pgfpathlineto{\pgfqpoint{6.455576in}{3.282671in}}%
\pgfpathlineto{\pgfqpoint{6.458798in}{3.282617in}}%
\pgfpathlineto{\pgfqpoint{6.460945in}{3.285801in}}%
\pgfpathlineto{\pgfqpoint{6.463093in}{3.286293in}}%
\pgfpathlineto{\pgfqpoint{6.466315in}{3.287660in}}%
\pgfpathlineto{\pgfqpoint{6.467388in}{3.288932in}}%
\pgfpathlineto{\pgfqpoint{6.469536in}{3.297894in}}%
\pgfpathlineto{\pgfqpoint{6.470610in}{3.298753in}}%
\pgfpathlineto{\pgfqpoint{6.473831in}{3.297657in}}%
\pgfpathlineto{\pgfqpoint{6.474905in}{3.291890in}}%
\pgfpathlineto{\pgfqpoint{6.475979in}{3.291948in}}%
\pgfpathlineto{\pgfqpoint{6.477053in}{3.289930in}}%
\pgfpathlineto{\pgfqpoint{6.478127in}{3.291371in}}%
\pgfpathlineto{\pgfqpoint{6.481348in}{3.289641in}}%
\pgfpathlineto{\pgfqpoint{6.482422in}{3.291460in}}%
\pgfpathlineto{\pgfqpoint{6.484570in}{3.291114in}}%
\pgfpathlineto{\pgfqpoint{6.485644in}{3.298271in}}%
\pgfpathlineto{\pgfqpoint{6.491013in}{3.282321in}}%
\pgfpathlineto{\pgfqpoint{6.492087in}{3.286164in}}%
\pgfpathlineto{\pgfqpoint{6.493161in}{3.285851in}}%
\pgfpathlineto{\pgfqpoint{6.497456in}{3.282766in}}%
\pgfpathlineto{\pgfqpoint{6.499604in}{3.283445in}}%
\pgfpathlineto{\pgfqpoint{6.500677in}{3.283759in}}%
\pgfpathlineto{\pgfqpoint{6.500677in}{3.283759in}}%
\pgfusepath{stroke}%
\end{pgfscope}%
\begin{pgfscope}%
\pgfsetrectcap%
\pgfsetmiterjoin%
\pgfsetlinewidth{0.803000pt}%
\definecolor{currentstroke}{rgb}{1.000000,1.000000,1.000000}%
\pgfsetstrokecolor{currentstroke}%
\pgfsetdash{}{0pt}%
\pgfpathmoveto{\pgfqpoint{4.034768in}{3.218007in}}%
\pgfpathlineto{\pgfqpoint{4.034768in}{3.618892in}}%
\pgfusepath{stroke}%
\end{pgfscope}%
\begin{pgfscope}%
\pgfsetrectcap%
\pgfsetmiterjoin%
\pgfsetlinewidth{0.803000pt}%
\definecolor{currentstroke}{rgb}{1.000000,1.000000,1.000000}%
\pgfsetstrokecolor{currentstroke}%
\pgfsetdash{}{0pt}%
\pgfpathmoveto{\pgfqpoint{6.618102in}{3.218007in}}%
\pgfpathlineto{\pgfqpoint{6.618102in}{3.618892in}}%
\pgfusepath{stroke}%
\end{pgfscope}%
\begin{pgfscope}%
\pgfsetrectcap%
\pgfsetmiterjoin%
\pgfsetlinewidth{0.803000pt}%
\definecolor{currentstroke}{rgb}{1.000000,1.000000,1.000000}%
\pgfsetstrokecolor{currentstroke}%
\pgfsetdash{}{0pt}%
\pgfpathmoveto{\pgfqpoint{4.034768in}{3.218007in}}%
\pgfpathlineto{\pgfqpoint{6.618102in}{3.218007in}}%
\pgfusepath{stroke}%
\end{pgfscope}%
\begin{pgfscope}%
\pgfsetrectcap%
\pgfsetmiterjoin%
\pgfsetlinewidth{0.803000pt}%
\definecolor{currentstroke}{rgb}{1.000000,1.000000,1.000000}%
\pgfsetstrokecolor{currentstroke}%
\pgfsetdash{}{0pt}%
\pgfpathmoveto{\pgfqpoint{4.034768in}{3.618892in}}%
\pgfpathlineto{\pgfqpoint{6.618102in}{3.618892in}}%
\pgfusepath{stroke}%
\end{pgfscope}%
\begin{pgfscope}%
\definecolor{textcolor}{rgb}{0.150000,0.150000,0.150000}%
\pgfsetstrokecolor{textcolor}%
\pgfsetfillcolor{textcolor}%
\pgftext[x=5.326435in,y=3.702225in,,base]{\color{textcolor}\rmfamily\fontsize{12.000000}{14.400000}\selectfont INTC}%
\end{pgfscope}%
\begin{pgfscope}%
\pgfsetbuttcap%
\pgfsetmiterjoin%
\definecolor{currentfill}{rgb}{0.917647,0.917647,0.949020}%
\pgfsetfillcolor{currentfill}%
\pgfsetlinewidth{0.000000pt}%
\definecolor{currentstroke}{rgb}{0.000000,0.000000,0.000000}%
\pgfsetstrokecolor{currentstroke}%
\pgfsetstrokeopacity{0.000000}%
\pgfsetdash{}{0pt}%
\pgfpathmoveto{\pgfqpoint{0.418102in}{2.255883in}}%
\pgfpathlineto{\pgfqpoint{3.001435in}{2.255883in}}%
\pgfpathlineto{\pgfqpoint{3.001435in}{2.656768in}}%
\pgfpathlineto{\pgfqpoint{0.418102in}{2.656768in}}%
\pgfpathclose%
\pgfusepath{fill}%
\end{pgfscope}%
\begin{pgfscope}%
\pgfpathrectangle{\pgfqpoint{0.418102in}{2.255883in}}{\pgfqpoint{2.583333in}{0.400885in}}%
\pgfusepath{clip}%
\pgfsetroundcap%
\pgfsetroundjoin%
\pgfsetlinewidth{0.803000pt}%
\definecolor{currentstroke}{rgb}{1.000000,1.000000,1.000000}%
\pgfsetstrokecolor{currentstroke}%
\pgfsetdash{}{0pt}%
\pgfpathmoveto{\pgfqpoint{0.533378in}{2.255883in}}%
\pgfpathlineto{\pgfqpoint{0.533378in}{2.656768in}}%
\pgfusepath{stroke}%
\end{pgfscope}%
\begin{pgfscope}%
\definecolor{textcolor}{rgb}{0.150000,0.150000,0.150000}%
\pgfsetstrokecolor{textcolor}%
\pgfsetfillcolor{textcolor}%
\pgftext[x=0.533378in,y=2.158661in,,top]{\color{textcolor}\rmfamily\fontsize{10.000000}{12.000000}\selectfont 2012}%
\end{pgfscope}%
\begin{pgfscope}%
\pgfpathrectangle{\pgfqpoint{0.418102in}{2.255883in}}{\pgfqpoint{2.583333in}{0.400885in}}%
\pgfusepath{clip}%
\pgfsetroundcap%
\pgfsetroundjoin%
\pgfsetlinewidth{0.803000pt}%
\definecolor{currentstroke}{rgb}{1.000000,1.000000,1.000000}%
\pgfsetstrokecolor{currentstroke}%
\pgfsetdash{}{0pt}%
\pgfpathmoveto{\pgfqpoint{0.926403in}{2.255883in}}%
\pgfpathlineto{\pgfqpoint{0.926403in}{2.656768in}}%
\pgfusepath{stroke}%
\end{pgfscope}%
\begin{pgfscope}%
\definecolor{textcolor}{rgb}{0.150000,0.150000,0.150000}%
\pgfsetstrokecolor{textcolor}%
\pgfsetfillcolor{textcolor}%
\pgftext[x=0.926403in,y=2.158661in,,top]{\color{textcolor}\rmfamily\fontsize{10.000000}{12.000000}\selectfont 2013}%
\end{pgfscope}%
\begin{pgfscope}%
\pgfpathrectangle{\pgfqpoint{0.418102in}{2.255883in}}{\pgfqpoint{2.583333in}{0.400885in}}%
\pgfusepath{clip}%
\pgfsetroundcap%
\pgfsetroundjoin%
\pgfsetlinewidth{0.803000pt}%
\definecolor{currentstroke}{rgb}{1.000000,1.000000,1.000000}%
\pgfsetstrokecolor{currentstroke}%
\pgfsetdash{}{0pt}%
\pgfpathmoveto{\pgfqpoint{1.318354in}{2.255883in}}%
\pgfpathlineto{\pgfqpoint{1.318354in}{2.656768in}}%
\pgfusepath{stroke}%
\end{pgfscope}%
\begin{pgfscope}%
\definecolor{textcolor}{rgb}{0.150000,0.150000,0.150000}%
\pgfsetstrokecolor{textcolor}%
\pgfsetfillcolor{textcolor}%
\pgftext[x=1.318354in,y=2.158661in,,top]{\color{textcolor}\rmfamily\fontsize{10.000000}{12.000000}\selectfont 2014}%
\end{pgfscope}%
\begin{pgfscope}%
\pgfpathrectangle{\pgfqpoint{0.418102in}{2.255883in}}{\pgfqpoint{2.583333in}{0.400885in}}%
\pgfusepath{clip}%
\pgfsetroundcap%
\pgfsetroundjoin%
\pgfsetlinewidth{0.803000pt}%
\definecolor{currentstroke}{rgb}{1.000000,1.000000,1.000000}%
\pgfsetstrokecolor{currentstroke}%
\pgfsetdash{}{0pt}%
\pgfpathmoveto{\pgfqpoint{1.710305in}{2.255883in}}%
\pgfpathlineto{\pgfqpoint{1.710305in}{2.656768in}}%
\pgfusepath{stroke}%
\end{pgfscope}%
\begin{pgfscope}%
\definecolor{textcolor}{rgb}{0.150000,0.150000,0.150000}%
\pgfsetstrokecolor{textcolor}%
\pgfsetfillcolor{textcolor}%
\pgftext[x=1.710305in,y=2.158661in,,top]{\color{textcolor}\rmfamily\fontsize{10.000000}{12.000000}\selectfont 2015}%
\end{pgfscope}%
\begin{pgfscope}%
\pgfpathrectangle{\pgfqpoint{0.418102in}{2.255883in}}{\pgfqpoint{2.583333in}{0.400885in}}%
\pgfusepath{clip}%
\pgfsetroundcap%
\pgfsetroundjoin%
\pgfsetlinewidth{0.803000pt}%
\definecolor{currentstroke}{rgb}{1.000000,1.000000,1.000000}%
\pgfsetstrokecolor{currentstroke}%
\pgfsetdash{}{0pt}%
\pgfpathmoveto{\pgfqpoint{2.102256in}{2.255883in}}%
\pgfpathlineto{\pgfqpoint{2.102256in}{2.656768in}}%
\pgfusepath{stroke}%
\end{pgfscope}%
\begin{pgfscope}%
\definecolor{textcolor}{rgb}{0.150000,0.150000,0.150000}%
\pgfsetstrokecolor{textcolor}%
\pgfsetfillcolor{textcolor}%
\pgftext[x=2.102256in,y=2.158661in,,top]{\color{textcolor}\rmfamily\fontsize{10.000000}{12.000000}\selectfont 2016}%
\end{pgfscope}%
\begin{pgfscope}%
\pgfpathrectangle{\pgfqpoint{0.418102in}{2.255883in}}{\pgfqpoint{2.583333in}{0.400885in}}%
\pgfusepath{clip}%
\pgfsetroundcap%
\pgfsetroundjoin%
\pgfsetlinewidth{0.803000pt}%
\definecolor{currentstroke}{rgb}{1.000000,1.000000,1.000000}%
\pgfsetstrokecolor{currentstroke}%
\pgfsetdash{}{0pt}%
\pgfpathmoveto{\pgfqpoint{2.495281in}{2.255883in}}%
\pgfpathlineto{\pgfqpoint{2.495281in}{2.656768in}}%
\pgfusepath{stroke}%
\end{pgfscope}%
\begin{pgfscope}%
\definecolor{textcolor}{rgb}{0.150000,0.150000,0.150000}%
\pgfsetstrokecolor{textcolor}%
\pgfsetfillcolor{textcolor}%
\pgftext[x=2.495281in,y=2.158661in,,top]{\color{textcolor}\rmfamily\fontsize{10.000000}{12.000000}\selectfont 2017}%
\end{pgfscope}%
\begin{pgfscope}%
\pgfpathrectangle{\pgfqpoint{0.418102in}{2.255883in}}{\pgfqpoint{2.583333in}{0.400885in}}%
\pgfusepath{clip}%
\pgfsetroundcap%
\pgfsetroundjoin%
\pgfsetlinewidth{0.803000pt}%
\definecolor{currentstroke}{rgb}{1.000000,1.000000,1.000000}%
\pgfsetstrokecolor{currentstroke}%
\pgfsetdash{}{0pt}%
\pgfpathmoveto{\pgfqpoint{2.887232in}{2.255883in}}%
\pgfpathlineto{\pgfqpoint{2.887232in}{2.656768in}}%
\pgfusepath{stroke}%
\end{pgfscope}%
\begin{pgfscope}%
\definecolor{textcolor}{rgb}{0.150000,0.150000,0.150000}%
\pgfsetstrokecolor{textcolor}%
\pgfsetfillcolor{textcolor}%
\pgftext[x=2.887232in,y=2.158661in,,top]{\color{textcolor}\rmfamily\fontsize{10.000000}{12.000000}\selectfont 2018}%
\end{pgfscope}%
\begin{pgfscope}%
\pgfpathrectangle{\pgfqpoint{0.418102in}{2.255883in}}{\pgfqpoint{2.583333in}{0.400885in}}%
\pgfusepath{clip}%
\pgfsetroundcap%
\pgfsetroundjoin%
\pgfsetlinewidth{0.803000pt}%
\definecolor{currentstroke}{rgb}{1.000000,1.000000,1.000000}%
\pgfsetstrokecolor{currentstroke}%
\pgfsetdash{}{0pt}%
\pgfpathmoveto{\pgfqpoint{0.418102in}{2.284081in}}%
\pgfpathlineto{\pgfqpoint{3.001435in}{2.284081in}}%
\pgfusepath{stroke}%
\end{pgfscope}%
\begin{pgfscope}%
\definecolor{textcolor}{rgb}{0.150000,0.150000,0.150000}%
\pgfsetstrokecolor{textcolor}%
\pgfsetfillcolor{textcolor}%
\pgftext[x=0.232514in,y=2.231319in,left,base]{\color{textcolor}\rmfamily\fontsize{10.000000}{12.000000}\selectfont 1}%
\end{pgfscope}%
\begin{pgfscope}%
\pgfpathrectangle{\pgfqpoint{0.418102in}{2.255883in}}{\pgfqpoint{2.583333in}{0.400885in}}%
\pgfusepath{clip}%
\pgfsetroundcap%
\pgfsetroundjoin%
\pgfsetlinewidth{0.803000pt}%
\definecolor{currentstroke}{rgb}{1.000000,1.000000,1.000000}%
\pgfsetstrokecolor{currentstroke}%
\pgfsetdash{}{0pt}%
\pgfpathmoveto{\pgfqpoint{0.418102in}{2.507384in}}%
\pgfpathlineto{\pgfqpoint{3.001435in}{2.507384in}}%
\pgfusepath{stroke}%
\end{pgfscope}%
\begin{pgfscope}%
\definecolor{textcolor}{rgb}{0.150000,0.150000,0.150000}%
\pgfsetstrokecolor{textcolor}%
\pgfsetfillcolor{textcolor}%
\pgftext[x=0.232514in,y=2.454623in,left,base]{\color{textcolor}\rmfamily\fontsize{10.000000}{12.000000}\selectfont 2}%
\end{pgfscope}%
\begin{pgfscope}%
\pgfpathrectangle{\pgfqpoint{0.418102in}{2.255883in}}{\pgfqpoint{2.583333in}{0.400885in}}%
\pgfusepath{clip}%
\pgfsetroundcap%
\pgfsetroundjoin%
\pgfsetlinewidth{1.505625pt}%
\definecolor{currentstroke}{rgb}{0.000000,0.000000,0.000000}%
\pgfsetstrokecolor{currentstroke}%
\pgfsetdash{}{0pt}%
\pgfpathmoveto{\pgfqpoint{0.535526in}{2.284081in}}%
\pgfpathlineto{\pgfqpoint{0.536600in}{2.282734in}}%
\pgfpathlineto{\pgfqpoint{0.537674in}{2.282481in}}%
\pgfpathlineto{\pgfqpoint{0.538747in}{2.280545in}}%
\pgfpathlineto{\pgfqpoint{0.541969in}{2.280882in}}%
\pgfpathlineto{\pgfqpoint{0.543043in}{2.281808in}}%
\pgfpathlineto{\pgfqpoint{0.544117in}{2.281555in}}%
\pgfpathlineto{\pgfqpoint{0.546264in}{2.281976in}}%
\pgfpathlineto{\pgfqpoint{0.550560in}{2.281513in}}%
\pgfpathlineto{\pgfqpoint{0.551633in}{2.282060in}}%
\pgfpathlineto{\pgfqpoint{0.557003in}{2.281134in}}%
\pgfpathlineto{\pgfqpoint{0.558077in}{2.281134in}}%
\pgfpathlineto{\pgfqpoint{0.559150in}{2.281850in}}%
\pgfpathlineto{\pgfqpoint{0.560224in}{2.283492in}}%
\pgfpathlineto{\pgfqpoint{0.561298in}{2.283029in}}%
\pgfpathlineto{\pgfqpoint{0.566667in}{2.283449in}}%
\pgfpathlineto{\pgfqpoint{0.573110in}{2.281976in}}%
\pgfpathlineto{\pgfqpoint{0.574184in}{2.281934in}}%
\pgfpathlineto{\pgfqpoint{0.576332in}{2.279745in}}%
\pgfpathlineto{\pgfqpoint{0.582775in}{2.280840in}}%
\pgfpathlineto{\pgfqpoint{0.583849in}{2.281092in}}%
\pgfpathlineto{\pgfqpoint{0.594587in}{2.281176in}}%
\pgfpathlineto{\pgfqpoint{0.595661in}{2.283660in}}%
\pgfpathlineto{\pgfqpoint{0.598882in}{2.282271in}}%
\pgfpathlineto{\pgfqpoint{0.602104in}{2.282734in}}%
\pgfpathlineto{\pgfqpoint{0.603178in}{2.280840in}}%
\pgfpathlineto{\pgfqpoint{0.604252in}{2.280671in}}%
\pgfpathlineto{\pgfqpoint{0.605325in}{2.282566in}}%
\pgfpathlineto{\pgfqpoint{0.606399in}{2.282187in}}%
\pgfpathlineto{\pgfqpoint{0.610695in}{2.284207in}}%
\pgfpathlineto{\pgfqpoint{0.611768in}{2.283323in}}%
\pgfpathlineto{\pgfqpoint{0.613916in}{2.283492in}}%
\pgfpathlineto{\pgfqpoint{0.617138in}{2.283786in}}%
\pgfpathlineto{\pgfqpoint{0.620359in}{2.281219in}}%
\pgfpathlineto{\pgfqpoint{0.621433in}{2.281513in}}%
\pgfpathlineto{\pgfqpoint{0.626802in}{2.285175in}}%
\pgfpathlineto{\pgfqpoint{0.627876in}{2.284923in}}%
\pgfpathlineto{\pgfqpoint{0.628950in}{2.286354in}}%
\pgfpathlineto{\pgfqpoint{0.632171in}{2.287196in}}%
\pgfpathlineto{\pgfqpoint{0.635393in}{2.284207in}}%
\pgfpathlineto{\pgfqpoint{0.639688in}{2.282776in}}%
\pgfpathlineto{\pgfqpoint{0.640762in}{2.280335in}}%
\pgfpathlineto{\pgfqpoint{0.642910in}{2.280166in}}%
\pgfpathlineto{\pgfqpoint{0.643984in}{2.278062in}}%
\pgfpathlineto{\pgfqpoint{0.648279in}{2.280377in}}%
\pgfpathlineto{\pgfqpoint{0.649353in}{2.277093in}}%
\pgfpathlineto{\pgfqpoint{0.650427in}{2.276336in}}%
\pgfpathlineto{\pgfqpoint{0.651500in}{2.278651in}}%
\pgfpathlineto{\pgfqpoint{0.654722in}{2.277472in}}%
\pgfpathlineto{\pgfqpoint{0.659017in}{2.282523in}}%
\pgfpathlineto{\pgfqpoint{0.665460in}{2.284207in}}%
\pgfpathlineto{\pgfqpoint{0.666534in}{2.282187in}}%
\pgfpathlineto{\pgfqpoint{0.670830in}{2.282986in}}%
\pgfpathlineto{\pgfqpoint{0.671903in}{2.280587in}}%
\pgfpathlineto{\pgfqpoint{0.672977in}{2.281597in}}%
\pgfpathlineto{\pgfqpoint{0.674051in}{2.280798in}}%
\pgfpathlineto{\pgfqpoint{0.677273in}{2.279409in}}%
\pgfpathlineto{\pgfqpoint{0.678346in}{2.278314in}}%
\pgfpathlineto{\pgfqpoint{0.679420in}{2.278651in}}%
\pgfpathlineto{\pgfqpoint{0.681568in}{2.277430in}}%
\pgfpathlineto{\pgfqpoint{0.685863in}{2.278019in}}%
\pgfpathlineto{\pgfqpoint{0.686937in}{2.277136in}}%
\pgfpathlineto{\pgfqpoint{0.688011in}{2.278651in}}%
\pgfpathlineto{\pgfqpoint{0.689085in}{2.276630in}}%
\pgfpathlineto{\pgfqpoint{0.693380in}{2.276925in}}%
\pgfpathlineto{\pgfqpoint{0.694454in}{2.275578in}}%
\pgfpathlineto{\pgfqpoint{0.695528in}{2.276336in}}%
\pgfpathlineto{\pgfqpoint{0.696602in}{2.274105in}}%
\pgfpathlineto{\pgfqpoint{0.699823in}{2.276041in}}%
\pgfpathlineto{\pgfqpoint{0.700897in}{2.275578in}}%
\pgfpathlineto{\pgfqpoint{0.701971in}{2.277641in}}%
\pgfpathlineto{\pgfqpoint{0.703045in}{2.277641in}}%
\pgfpathlineto{\pgfqpoint{0.704119in}{2.278230in}}%
\pgfpathlineto{\pgfqpoint{0.707340in}{2.275283in}}%
\pgfpathlineto{\pgfqpoint{0.711635in}{2.288711in}}%
\pgfpathlineto{\pgfqpoint{0.714857in}{2.289721in}}%
\pgfpathlineto{\pgfqpoint{0.717005in}{2.292121in}}%
\pgfpathlineto{\pgfqpoint{0.718078in}{2.290016in}}%
\pgfpathlineto{\pgfqpoint{0.719152in}{2.290858in}}%
\pgfpathlineto{\pgfqpoint{0.723448in}{2.290142in}}%
\pgfpathlineto{\pgfqpoint{0.724521in}{2.291573in}}%
\pgfpathlineto{\pgfqpoint{0.725595in}{2.291910in}}%
\pgfpathlineto{\pgfqpoint{0.726669in}{2.294057in}}%
\pgfpathlineto{\pgfqpoint{0.730965in}{2.295741in}}%
\pgfpathlineto{\pgfqpoint{0.734186in}{2.294352in}}%
\pgfpathlineto{\pgfqpoint{0.739555in}{2.295235in}}%
\pgfpathlineto{\pgfqpoint{0.740629in}{2.294604in}}%
\pgfpathlineto{\pgfqpoint{0.741703in}{2.297677in}}%
\pgfpathlineto{\pgfqpoint{0.744924in}{2.297130in}}%
\pgfpathlineto{\pgfqpoint{0.747072in}{2.300329in}}%
\pgfpathlineto{\pgfqpoint{0.748146in}{2.300876in}}%
\pgfpathlineto{\pgfqpoint{0.749220in}{2.297761in}}%
\pgfpathlineto{\pgfqpoint{0.752441in}{2.295951in}}%
\pgfpathlineto{\pgfqpoint{0.753515in}{2.293341in}}%
\pgfpathlineto{\pgfqpoint{0.754589in}{2.293973in}}%
\pgfpathlineto{\pgfqpoint{0.756737in}{2.300834in}}%
\pgfpathlineto{\pgfqpoint{0.759958in}{2.300581in}}%
\pgfpathlineto{\pgfqpoint{0.761032in}{2.299782in}}%
\pgfpathlineto{\pgfqpoint{0.762106in}{2.300371in}}%
\pgfpathlineto{\pgfqpoint{0.763180in}{2.297130in}}%
\pgfpathlineto{\pgfqpoint{0.764254in}{2.299445in}}%
\pgfpathlineto{\pgfqpoint{0.767475in}{2.298477in}}%
\pgfpathlineto{\pgfqpoint{0.768549in}{2.296582in}}%
\pgfpathlineto{\pgfqpoint{0.770697in}{2.296709in}}%
\pgfpathlineto{\pgfqpoint{0.771770in}{2.297803in}}%
\pgfpathlineto{\pgfqpoint{0.774992in}{2.297172in}}%
\pgfpathlineto{\pgfqpoint{0.776066in}{2.297803in}}%
\pgfpathlineto{\pgfqpoint{0.779287in}{2.294899in}}%
\pgfpathlineto{\pgfqpoint{0.785730in}{2.294688in}}%
\pgfpathlineto{\pgfqpoint{0.786804in}{2.296330in}}%
\pgfpathlineto{\pgfqpoint{0.792173in}{2.295530in}}%
\pgfpathlineto{\pgfqpoint{0.793247in}{2.294983in}}%
\pgfpathlineto{\pgfqpoint{0.794321in}{2.295741in}}%
\pgfpathlineto{\pgfqpoint{0.799690in}{2.295193in}}%
\pgfpathlineto{\pgfqpoint{0.800764in}{2.297172in}}%
\pgfpathlineto{\pgfqpoint{0.801838in}{2.297298in}}%
\pgfpathlineto{\pgfqpoint{0.806133in}{2.298435in}}%
\pgfpathlineto{\pgfqpoint{0.807207in}{2.298224in}}%
\pgfpathlineto{\pgfqpoint{0.808281in}{2.301171in}}%
\pgfpathlineto{\pgfqpoint{0.809355in}{2.299361in}}%
\pgfpathlineto{\pgfqpoint{0.812576in}{2.298603in}}%
\pgfpathlineto{\pgfqpoint{0.813650in}{2.299655in}}%
\pgfpathlineto{\pgfqpoint{0.814724in}{2.299824in}}%
\pgfpathlineto{\pgfqpoint{0.816872in}{2.301423in}}%
\pgfpathlineto{\pgfqpoint{0.820093in}{2.301213in}}%
\pgfpathlineto{\pgfqpoint{0.821167in}{2.302307in}}%
\pgfpathlineto{\pgfqpoint{0.822241in}{2.301213in}}%
\pgfpathlineto{\pgfqpoint{0.829758in}{2.301213in}}%
\pgfpathlineto{\pgfqpoint{0.831905in}{2.303486in}}%
\pgfpathlineto{\pgfqpoint{0.835127in}{2.302728in}}%
\pgfpathlineto{\pgfqpoint{0.836201in}{2.299150in}}%
\pgfpathlineto{\pgfqpoint{0.839422in}{2.297635in}}%
\pgfpathlineto{\pgfqpoint{0.842644in}{2.299824in}}%
\pgfpathlineto{\pgfqpoint{0.844791in}{2.308158in}}%
\pgfpathlineto{\pgfqpoint{0.845865in}{2.313462in}}%
\pgfpathlineto{\pgfqpoint{0.846939in}{2.311189in}}%
\pgfpathlineto{\pgfqpoint{0.850161in}{2.310894in}}%
\pgfpathlineto{\pgfqpoint{0.851234in}{2.307779in}}%
\pgfpathlineto{\pgfqpoint{0.852308in}{2.307274in}}%
\pgfpathlineto{\pgfqpoint{0.853382in}{2.308663in}}%
\pgfpathlineto{\pgfqpoint{0.854456in}{2.307821in}}%
\pgfpathlineto{\pgfqpoint{0.859825in}{2.307527in}}%
\pgfpathlineto{\pgfqpoint{0.860899in}{2.309926in}}%
\pgfpathlineto{\pgfqpoint{0.861973in}{2.307821in}}%
\pgfpathlineto{\pgfqpoint{0.865194in}{2.307442in}}%
\pgfpathlineto{\pgfqpoint{0.866268in}{2.308200in}}%
\pgfpathlineto{\pgfqpoint{0.868416in}{2.303486in}}%
\pgfpathlineto{\pgfqpoint{0.869490in}{2.304243in}}%
\pgfpathlineto{\pgfqpoint{0.873785in}{2.302981in}}%
\pgfpathlineto{\pgfqpoint{0.875933in}{2.301465in}}%
\pgfpathlineto{\pgfqpoint{0.877007in}{2.301886in}}%
\pgfpathlineto{\pgfqpoint{0.880228in}{2.302055in}}%
\pgfpathlineto{\pgfqpoint{0.881302in}{2.303528in}}%
\pgfpathlineto{\pgfqpoint{0.882376in}{2.303275in}}%
\pgfpathlineto{\pgfqpoint{0.884523in}{2.305296in}}%
\pgfpathlineto{\pgfqpoint{0.888819in}{2.302644in}}%
\pgfpathlineto{\pgfqpoint{0.889893in}{2.304370in}}%
\pgfpathlineto{\pgfqpoint{0.890966in}{2.304117in}}%
\pgfpathlineto{\pgfqpoint{0.892040in}{2.305885in}}%
\pgfpathlineto{\pgfqpoint{0.895262in}{2.305632in}}%
\pgfpathlineto{\pgfqpoint{0.897409in}{2.306727in}}%
\pgfpathlineto{\pgfqpoint{0.898483in}{2.307022in}}%
\pgfpathlineto{\pgfqpoint{0.899557in}{2.308411in}}%
\pgfpathlineto{\pgfqpoint{0.902779in}{2.308958in}}%
\pgfpathlineto{\pgfqpoint{0.903853in}{2.310726in}}%
\pgfpathlineto{\pgfqpoint{0.907074in}{2.309252in}}%
\pgfpathlineto{\pgfqpoint{0.911369in}{2.310178in}}%
\pgfpathlineto{\pgfqpoint{0.912443in}{2.309042in}}%
\pgfpathlineto{\pgfqpoint{0.913517in}{2.309463in}}%
\pgfpathlineto{\pgfqpoint{0.914591in}{2.307779in}}%
\pgfpathlineto{\pgfqpoint{0.917812in}{2.306895in}}%
\pgfpathlineto{\pgfqpoint{0.921034in}{2.307148in}}%
\pgfpathlineto{\pgfqpoint{0.922108in}{2.305001in}}%
\pgfpathlineto{\pgfqpoint{0.925329in}{2.307190in}}%
\pgfpathlineto{\pgfqpoint{0.927477in}{2.309800in}}%
\pgfpathlineto{\pgfqpoint{0.928551in}{2.309463in}}%
\pgfpathlineto{\pgfqpoint{0.929625in}{2.312283in}}%
\pgfpathlineto{\pgfqpoint{0.933920in}{2.311820in}}%
\pgfpathlineto{\pgfqpoint{0.937142in}{2.315103in}}%
\pgfpathlineto{\pgfqpoint{0.940363in}{2.315861in}}%
\pgfpathlineto{\pgfqpoint{0.941437in}{2.315188in}}%
\pgfpathlineto{\pgfqpoint{0.943585in}{2.317040in}}%
\pgfpathlineto{\pgfqpoint{0.944658in}{2.318218in}}%
\pgfpathlineto{\pgfqpoint{0.948954in}{2.316282in}}%
\pgfpathlineto{\pgfqpoint{0.951101in}{2.317755in}}%
\pgfpathlineto{\pgfqpoint{0.952175in}{2.320618in}}%
\pgfpathlineto{\pgfqpoint{0.955397in}{2.319565in}}%
\pgfpathlineto{\pgfqpoint{0.956471in}{2.322343in}}%
\pgfpathlineto{\pgfqpoint{0.958618in}{2.320618in}}%
\pgfpathlineto{\pgfqpoint{0.959692in}{2.321544in}}%
\pgfpathlineto{\pgfqpoint{0.962914in}{2.321291in}}%
\pgfpathlineto{\pgfqpoint{0.965061in}{2.325795in}}%
\pgfpathlineto{\pgfqpoint{0.966135in}{2.324616in}}%
\pgfpathlineto{\pgfqpoint{0.967209in}{2.326090in}}%
\pgfpathlineto{\pgfqpoint{0.970431in}{2.325879in}}%
\pgfpathlineto{\pgfqpoint{0.971504in}{2.327226in}}%
\pgfpathlineto{\pgfqpoint{0.972578in}{2.326721in}}%
\pgfpathlineto{\pgfqpoint{0.973652in}{2.327268in}}%
\pgfpathlineto{\pgfqpoint{0.974726in}{2.328489in}}%
\pgfpathlineto{\pgfqpoint{0.979021in}{2.331309in}}%
\pgfpathlineto{\pgfqpoint{0.980095in}{2.330215in}}%
\pgfpathlineto{\pgfqpoint{0.981169in}{2.331015in}}%
\pgfpathlineto{\pgfqpoint{0.982243in}{2.330972in}}%
\pgfpathlineto{\pgfqpoint{0.985464in}{2.328531in}}%
\pgfpathlineto{\pgfqpoint{0.986538in}{2.329205in}}%
\pgfpathlineto{\pgfqpoint{0.987612in}{2.331225in}}%
\pgfpathlineto{\pgfqpoint{0.988686in}{2.330467in}}%
\pgfpathlineto{\pgfqpoint{0.989760in}{2.332572in}}%
\pgfpathlineto{\pgfqpoint{0.992981in}{2.334340in}}%
\pgfpathlineto{\pgfqpoint{0.994055in}{2.335939in}}%
\pgfpathlineto{\pgfqpoint{0.995129in}{2.335013in}}%
\pgfpathlineto{\pgfqpoint{0.997276in}{2.337834in}}%
\pgfpathlineto{\pgfqpoint{1.002646in}{2.339096in}}%
\pgfpathlineto{\pgfqpoint{1.003720in}{2.341075in}}%
\pgfpathlineto{\pgfqpoint{1.004793in}{2.341369in}}%
\pgfpathlineto{\pgfqpoint{1.008015in}{2.340022in}}%
\pgfpathlineto{\pgfqpoint{1.009089in}{2.340191in}}%
\pgfpathlineto{\pgfqpoint{1.010163in}{2.342295in}}%
\pgfpathlineto{\pgfqpoint{1.011236in}{2.340738in}}%
\pgfpathlineto{\pgfqpoint{1.012310in}{2.343306in}}%
\pgfpathlineto{\pgfqpoint{1.015532in}{2.343095in}}%
\pgfpathlineto{\pgfqpoint{1.016606in}{2.347262in}}%
\pgfpathlineto{\pgfqpoint{1.018753in}{2.349662in}}%
\pgfpathlineto{\pgfqpoint{1.023049in}{2.351093in}}%
\pgfpathlineto{\pgfqpoint{1.024122in}{2.353787in}}%
\pgfpathlineto{\pgfqpoint{1.025196in}{2.351598in}}%
\pgfpathlineto{\pgfqpoint{1.026270in}{2.352777in}}%
\pgfpathlineto{\pgfqpoint{1.030565in}{2.348188in}}%
\pgfpathlineto{\pgfqpoint{1.034861in}{2.353955in}}%
\pgfpathlineto{\pgfqpoint{1.038082in}{2.350293in}}%
\pgfpathlineto{\pgfqpoint{1.039156in}{2.356439in}}%
\pgfpathlineto{\pgfqpoint{1.040230in}{2.358080in}}%
\pgfpathlineto{\pgfqpoint{1.041304in}{2.355513in}}%
\pgfpathlineto{\pgfqpoint{1.042378in}{2.360143in}}%
\pgfpathlineto{\pgfqpoint{1.045599in}{2.361364in}}%
\pgfpathlineto{\pgfqpoint{1.046673in}{2.363552in}}%
\pgfpathlineto{\pgfqpoint{1.047747in}{2.359806in}}%
\pgfpathlineto{\pgfqpoint{1.048821in}{2.362753in}}%
\pgfpathlineto{\pgfqpoint{1.049895in}{2.362374in}}%
\pgfpathlineto{\pgfqpoint{1.053116in}{2.364015in}}%
\pgfpathlineto{\pgfqpoint{1.054190in}{2.362795in}}%
\pgfpathlineto{\pgfqpoint{1.055264in}{2.359217in}}%
\pgfpathlineto{\pgfqpoint{1.057411in}{2.364605in}}%
\pgfpathlineto{\pgfqpoint{1.060633in}{2.360816in}}%
\pgfpathlineto{\pgfqpoint{1.061707in}{2.363847in}}%
\pgfpathlineto{\pgfqpoint{1.062781in}{2.363594in}}%
\pgfpathlineto{\pgfqpoint{1.063854in}{2.362500in}}%
\pgfpathlineto{\pgfqpoint{1.064928in}{2.364647in}}%
\pgfpathlineto{\pgfqpoint{1.068150in}{2.364984in}}%
\pgfpathlineto{\pgfqpoint{1.070297in}{2.371297in}}%
\pgfpathlineto{\pgfqpoint{1.071371in}{2.370624in}}%
\pgfpathlineto{\pgfqpoint{1.072445in}{2.372897in}}%
\pgfpathlineto{\pgfqpoint{1.075667in}{2.372644in}}%
\pgfpathlineto{\pgfqpoint{1.076741in}{2.374665in}}%
\pgfpathlineto{\pgfqpoint{1.077814in}{2.374244in}}%
\pgfpathlineto{\pgfqpoint{1.079962in}{2.370708in}}%
\pgfpathlineto{\pgfqpoint{1.084257in}{2.373528in}}%
\pgfpathlineto{\pgfqpoint{1.085331in}{2.366541in}}%
\pgfpathlineto{\pgfqpoint{1.086405in}{2.367762in}}%
\pgfpathlineto{\pgfqpoint{1.087479in}{2.361279in}}%
\pgfpathlineto{\pgfqpoint{1.090700in}{2.363174in}}%
\pgfpathlineto{\pgfqpoint{1.092848in}{2.359511in}}%
\pgfpathlineto{\pgfqpoint{1.094996in}{2.363889in}}%
\pgfpathlineto{\pgfqpoint{1.098217in}{2.364731in}}%
\pgfpathlineto{\pgfqpoint{1.099291in}{2.363131in}}%
\pgfpathlineto{\pgfqpoint{1.100365in}{2.359722in}}%
\pgfpathlineto{\pgfqpoint{1.101439in}{2.363889in}}%
\pgfpathlineto{\pgfqpoint{1.102513in}{2.363889in}}%
\pgfpathlineto{\pgfqpoint{1.105734in}{2.366457in}}%
\pgfpathlineto{\pgfqpoint{1.106808in}{2.369067in}}%
\pgfpathlineto{\pgfqpoint{1.108956in}{2.355723in}}%
\pgfpathlineto{\pgfqpoint{1.114325in}{2.365489in}}%
\pgfpathlineto{\pgfqpoint{1.115399in}{2.371340in}}%
\pgfpathlineto{\pgfqpoint{1.116473in}{2.370329in}}%
\pgfpathlineto{\pgfqpoint{1.117546in}{2.367299in}}%
\pgfpathlineto{\pgfqpoint{1.120768in}{2.370035in}}%
\pgfpathlineto{\pgfqpoint{1.121842in}{2.369824in}}%
\pgfpathlineto{\pgfqpoint{1.122916in}{2.370582in}}%
\pgfpathlineto{\pgfqpoint{1.125063in}{2.374454in}}%
\pgfpathlineto{\pgfqpoint{1.130432in}{2.379379in}}%
\pgfpathlineto{\pgfqpoint{1.132580in}{2.382031in}}%
\pgfpathlineto{\pgfqpoint{1.136875in}{2.383504in}}%
\pgfpathlineto{\pgfqpoint{1.137949in}{2.382536in}}%
\pgfpathlineto{\pgfqpoint{1.139023in}{2.382705in}}%
\pgfpathlineto{\pgfqpoint{1.140097in}{2.390029in}}%
\pgfpathlineto{\pgfqpoint{1.145466in}{2.390492in}}%
\pgfpathlineto{\pgfqpoint{1.147614in}{2.392176in}}%
\pgfpathlineto{\pgfqpoint{1.152983in}{2.394575in}}%
\pgfpathlineto{\pgfqpoint{1.154057in}{2.395543in}}%
\pgfpathlineto{\pgfqpoint{1.155131in}{2.397732in}}%
\pgfpathlineto{\pgfqpoint{1.158352in}{2.395627in}}%
\pgfpathlineto{\pgfqpoint{1.160500in}{2.395164in}}%
\pgfpathlineto{\pgfqpoint{1.161574in}{2.393986in}}%
\pgfpathlineto{\pgfqpoint{1.162648in}{2.390492in}}%
\pgfpathlineto{\pgfqpoint{1.165869in}{2.389229in}}%
\pgfpathlineto{\pgfqpoint{1.166943in}{2.392807in}}%
\pgfpathlineto{\pgfqpoint{1.169091in}{2.380474in}}%
\pgfpathlineto{\pgfqpoint{1.170164in}{2.379842in}}%
\pgfpathlineto{\pgfqpoint{1.173386in}{2.383673in}}%
\pgfpathlineto{\pgfqpoint{1.176608in}{2.375886in}}%
\pgfpathlineto{\pgfqpoint{1.177681in}{2.378790in}}%
\pgfpathlineto{\pgfqpoint{1.180903in}{2.375633in}}%
\pgfpathlineto{\pgfqpoint{1.181977in}{2.370708in}}%
\pgfpathlineto{\pgfqpoint{1.183051in}{2.372013in}}%
\pgfpathlineto{\pgfqpoint{1.184124in}{2.372181in}}%
\pgfpathlineto{\pgfqpoint{1.185198in}{2.371592in}}%
\pgfpathlineto{\pgfqpoint{1.189494in}{2.371634in}}%
\pgfpathlineto{\pgfqpoint{1.190567in}{2.373360in}}%
\pgfpathlineto{\pgfqpoint{1.195937in}{2.375717in}}%
\pgfpathlineto{\pgfqpoint{1.198084in}{2.381737in}}%
\pgfpathlineto{\pgfqpoint{1.199158in}{2.380937in}}%
\pgfpathlineto{\pgfqpoint{1.200232in}{2.379337in}}%
\pgfpathlineto{\pgfqpoint{1.204527in}{2.381105in}}%
\pgfpathlineto{\pgfqpoint{1.205601in}{2.384178in}}%
\pgfpathlineto{\pgfqpoint{1.206675in}{2.384767in}}%
\pgfpathlineto{\pgfqpoint{1.207749in}{2.383336in}}%
\pgfpathlineto{\pgfqpoint{1.210970in}{2.381231in}}%
\pgfpathlineto{\pgfqpoint{1.213118in}{2.373991in}}%
\pgfpathlineto{\pgfqpoint{1.214192in}{2.373949in}}%
\pgfpathlineto{\pgfqpoint{1.215266in}{2.372729in}}%
\pgfpathlineto{\pgfqpoint{1.218487in}{2.372602in}}%
\pgfpathlineto{\pgfqpoint{1.219561in}{2.375380in}}%
\pgfpathlineto{\pgfqpoint{1.220635in}{2.374749in}}%
\pgfpathlineto{\pgfqpoint{1.221709in}{2.372181in}}%
\pgfpathlineto{\pgfqpoint{1.222783in}{2.374833in}}%
\pgfpathlineto{\pgfqpoint{1.226004in}{2.372224in}}%
\pgfpathlineto{\pgfqpoint{1.227078in}{2.368688in}}%
\pgfpathlineto{\pgfqpoint{1.228152in}{2.369951in}}%
\pgfpathlineto{\pgfqpoint{1.230299in}{2.382536in}}%
\pgfpathlineto{\pgfqpoint{1.234595in}{2.384262in}}%
\pgfpathlineto{\pgfqpoint{1.236742in}{2.391586in}}%
\pgfpathlineto{\pgfqpoint{1.237816in}{2.390366in}}%
\pgfpathlineto{\pgfqpoint{1.241038in}{2.388808in}}%
\pgfpathlineto{\pgfqpoint{1.242112in}{2.392975in}}%
\pgfpathlineto{\pgfqpoint{1.243185in}{2.392049in}}%
\pgfpathlineto{\pgfqpoint{1.244259in}{2.392933in}}%
\pgfpathlineto{\pgfqpoint{1.245333in}{2.392007in}}%
\pgfpathlineto{\pgfqpoint{1.248555in}{2.393102in}}%
\pgfpathlineto{\pgfqpoint{1.249629in}{2.395796in}}%
\pgfpathlineto{\pgfqpoint{1.251776in}{2.393901in}}%
\pgfpathlineto{\pgfqpoint{1.252850in}{2.396637in}}%
\pgfpathlineto{\pgfqpoint{1.257145in}{2.394617in}}%
\pgfpathlineto{\pgfqpoint{1.258219in}{2.395417in}}%
\pgfpathlineto{\pgfqpoint{1.259293in}{2.394154in}}%
\pgfpathlineto{\pgfqpoint{1.260367in}{2.399079in}}%
\pgfpathlineto{\pgfqpoint{1.263588in}{2.399921in}}%
\pgfpathlineto{\pgfqpoint{1.264662in}{2.397311in}}%
\pgfpathlineto{\pgfqpoint{1.265736in}{2.396511in}}%
\pgfpathlineto{\pgfqpoint{1.267884in}{2.400300in}}%
\pgfpathlineto{\pgfqpoint{1.271105in}{2.399963in}}%
\pgfpathlineto{\pgfqpoint{1.273253in}{2.403036in}}%
\pgfpathlineto{\pgfqpoint{1.274327in}{2.403204in}}%
\pgfpathlineto{\pgfqpoint{1.275401in}{2.405772in}}%
\pgfpathlineto{\pgfqpoint{1.278622in}{2.407161in}}%
\pgfpathlineto{\pgfqpoint{1.279696in}{2.405098in}}%
\pgfpathlineto{\pgfqpoint{1.280770in}{2.404803in}}%
\pgfpathlineto{\pgfqpoint{1.287213in}{2.401141in}}%
\pgfpathlineto{\pgfqpoint{1.288287in}{2.399921in}}%
\pgfpathlineto{\pgfqpoint{1.289361in}{2.397521in}}%
\pgfpathlineto{\pgfqpoint{1.290434in}{2.402825in}}%
\pgfpathlineto{\pgfqpoint{1.293656in}{2.402825in}}%
\pgfpathlineto{\pgfqpoint{1.294730in}{2.401773in}}%
\pgfpathlineto{\pgfqpoint{1.295804in}{2.398027in}}%
\pgfpathlineto{\pgfqpoint{1.296877in}{2.390955in}}%
\pgfpathlineto{\pgfqpoint{1.297951in}{2.391628in}}%
\pgfpathlineto{\pgfqpoint{1.301173in}{2.391713in}}%
\pgfpathlineto{\pgfqpoint{1.302247in}{2.389145in}}%
\pgfpathlineto{\pgfqpoint{1.303320in}{2.396301in}}%
\pgfpathlineto{\pgfqpoint{1.304394in}{2.393943in}}%
\pgfpathlineto{\pgfqpoint{1.305468in}{2.394322in}}%
\pgfpathlineto{\pgfqpoint{1.309763in}{2.394238in}}%
\pgfpathlineto{\pgfqpoint{1.311911in}{2.395964in}}%
\pgfpathlineto{\pgfqpoint{1.312985in}{2.395290in}}%
\pgfpathlineto{\pgfqpoint{1.316207in}{2.395080in}}%
\pgfpathlineto{\pgfqpoint{1.317280in}{2.392512in}}%
\pgfpathlineto{\pgfqpoint{1.319428in}{2.390492in}}%
\pgfpathlineto{\pgfqpoint{1.320502in}{2.393438in}}%
\pgfpathlineto{\pgfqpoint{1.323723in}{2.395206in}}%
\pgfpathlineto{\pgfqpoint{1.324797in}{2.402278in}}%
\pgfpathlineto{\pgfqpoint{1.325871in}{2.401815in}}%
\pgfpathlineto{\pgfqpoint{1.326945in}{2.403877in}}%
\pgfpathlineto{\pgfqpoint{1.328019in}{2.403920in}}%
\pgfpathlineto{\pgfqpoint{1.331240in}{2.403078in}}%
\pgfpathlineto{\pgfqpoint{1.333388in}{2.404130in}}%
\pgfpathlineto{\pgfqpoint{1.334462in}{2.403583in}}%
\pgfpathlineto{\pgfqpoint{1.335536in}{2.405098in}}%
\pgfpathlineto{\pgfqpoint{1.339831in}{2.401352in}}%
\pgfpathlineto{\pgfqpoint{1.340905in}{2.402404in}}%
\pgfpathlineto{\pgfqpoint{1.343052in}{2.388977in}}%
\pgfpathlineto{\pgfqpoint{1.346274in}{2.386535in}}%
\pgfpathlineto{\pgfqpoint{1.347348in}{2.387124in}}%
\pgfpathlineto{\pgfqpoint{1.348422in}{2.382789in}}%
\pgfpathlineto{\pgfqpoint{1.349496in}{2.384936in}}%
\pgfpathlineto{\pgfqpoint{1.350569in}{2.381231in}}%
\pgfpathlineto{\pgfqpoint{1.353791in}{2.375086in}}%
\pgfpathlineto{\pgfqpoint{1.354865in}{2.374497in}}%
\pgfpathlineto{\pgfqpoint{1.355939in}{2.376896in}}%
\pgfpathlineto{\pgfqpoint{1.358086in}{2.386914in}}%
\pgfpathlineto{\pgfqpoint{1.361308in}{2.390618in}}%
\pgfpathlineto{\pgfqpoint{1.362382in}{2.397521in}}%
\pgfpathlineto{\pgfqpoint{1.363455in}{2.395543in}}%
\pgfpathlineto{\pgfqpoint{1.365603in}{2.396764in}}%
\pgfpathlineto{\pgfqpoint{1.369898in}{2.394617in}}%
\pgfpathlineto{\pgfqpoint{1.370972in}{2.392681in}}%
\pgfpathlineto{\pgfqpoint{1.372046in}{2.395375in}}%
\pgfpathlineto{\pgfqpoint{1.376341in}{2.393144in}}%
\pgfpathlineto{\pgfqpoint{1.378489in}{2.393144in}}%
\pgfpathlineto{\pgfqpoint{1.379563in}{2.394070in}}%
\pgfpathlineto{\pgfqpoint{1.380637in}{2.396848in}}%
\pgfpathlineto{\pgfqpoint{1.383858in}{2.394785in}}%
\pgfpathlineto{\pgfqpoint{1.384932in}{2.401268in}}%
\pgfpathlineto{\pgfqpoint{1.386006in}{2.398532in}}%
\pgfpathlineto{\pgfqpoint{1.388154in}{2.401226in}}%
\pgfpathlineto{\pgfqpoint{1.393523in}{2.402236in}}%
\pgfpathlineto{\pgfqpoint{1.394597in}{2.400047in}}%
\pgfpathlineto{\pgfqpoint{1.395671in}{2.399373in}}%
\pgfpathlineto{\pgfqpoint{1.398892in}{2.403457in}}%
\pgfpathlineto{\pgfqpoint{1.399966in}{2.403457in}}%
\pgfpathlineto{\pgfqpoint{1.401040in}{2.402194in}}%
\pgfpathlineto{\pgfqpoint{1.402114in}{2.404130in}}%
\pgfpathlineto{\pgfqpoint{1.403187in}{2.410739in}}%
\pgfpathlineto{\pgfqpoint{1.406409in}{2.408087in}}%
\pgfpathlineto{\pgfqpoint{1.407483in}{2.416042in}}%
\pgfpathlineto{\pgfqpoint{1.408557in}{2.414822in}}%
\pgfpathlineto{\pgfqpoint{1.413926in}{2.419115in}}%
\pgfpathlineto{\pgfqpoint{1.415000in}{2.418063in}}%
\pgfpathlineto{\pgfqpoint{1.416073in}{2.419115in}}%
\pgfpathlineto{\pgfqpoint{1.418221in}{2.419831in}}%
\pgfpathlineto{\pgfqpoint{1.421443in}{2.417894in}}%
\pgfpathlineto{\pgfqpoint{1.422517in}{2.418526in}}%
\pgfpathlineto{\pgfqpoint{1.423590in}{2.421767in}}%
\pgfpathlineto{\pgfqpoint{1.424664in}{2.412970in}}%
\pgfpathlineto{\pgfqpoint{1.425738in}{2.414148in}}%
\pgfpathlineto{\pgfqpoint{1.428960in}{2.415158in}}%
\pgfpathlineto{\pgfqpoint{1.430033in}{2.422651in}}%
\pgfpathlineto{\pgfqpoint{1.431107in}{2.421009in}}%
\pgfpathlineto{\pgfqpoint{1.438624in}{2.426397in}}%
\pgfpathlineto{\pgfqpoint{1.440772in}{2.424798in}}%
\pgfpathlineto{\pgfqpoint{1.443993in}{2.430480in}}%
\pgfpathlineto{\pgfqpoint{1.445067in}{2.429344in}}%
\pgfpathlineto{\pgfqpoint{1.446141in}{2.430270in}}%
\pgfpathlineto{\pgfqpoint{1.447215in}{2.427534in}}%
\pgfpathlineto{\pgfqpoint{1.448289in}{2.423072in}}%
\pgfpathlineto{\pgfqpoint{1.451510in}{2.425597in}}%
\pgfpathlineto{\pgfqpoint{1.452584in}{2.423787in}}%
\pgfpathlineto{\pgfqpoint{1.453658in}{2.428923in}}%
\pgfpathlineto{\pgfqpoint{1.454732in}{2.427407in}}%
\pgfpathlineto{\pgfqpoint{1.455806in}{2.428923in}}%
\pgfpathlineto{\pgfqpoint{1.459027in}{2.427492in}}%
\pgfpathlineto{\pgfqpoint{1.460101in}{2.429302in}}%
\pgfpathlineto{\pgfqpoint{1.463322in}{2.427702in}}%
\pgfpathlineto{\pgfqpoint{1.466544in}{2.427913in}}%
\pgfpathlineto{\pgfqpoint{1.467618in}{2.426481in}}%
\pgfpathlineto{\pgfqpoint{1.469765in}{2.431659in}}%
\pgfpathlineto{\pgfqpoint{1.470839in}{2.431743in}}%
\pgfpathlineto{\pgfqpoint{1.475135in}{2.431112in}}%
\pgfpathlineto{\pgfqpoint{1.476208in}{2.429217in}}%
\pgfpathlineto{\pgfqpoint{1.478356in}{2.433511in}}%
\pgfpathlineto{\pgfqpoint{1.483725in}{2.437973in}}%
\pgfpathlineto{\pgfqpoint{1.484799in}{2.439951in}}%
\pgfpathlineto{\pgfqpoint{1.489095in}{2.439951in}}%
\pgfpathlineto{\pgfqpoint{1.490168in}{2.443192in}}%
\pgfpathlineto{\pgfqpoint{1.492316in}{2.437426in}}%
\pgfpathlineto{\pgfqpoint{1.496611in}{2.437131in}}%
\pgfpathlineto{\pgfqpoint{1.497685in}{2.435237in}}%
\pgfpathlineto{\pgfqpoint{1.499833in}{2.442140in}}%
\pgfpathlineto{\pgfqpoint{1.500907in}{2.447486in}}%
\pgfpathlineto{\pgfqpoint{1.505202in}{2.445002in}}%
\pgfpathlineto{\pgfqpoint{1.506276in}{2.449296in}}%
\pgfpathlineto{\pgfqpoint{1.507350in}{2.448875in}}%
\pgfpathlineto{\pgfqpoint{1.508424in}{2.446476in}}%
\pgfpathlineto{\pgfqpoint{1.511645in}{2.445086in}}%
\pgfpathlineto{\pgfqpoint{1.512719in}{2.449675in}}%
\pgfpathlineto{\pgfqpoint{1.513793in}{2.449632in}}%
\pgfpathlineto{\pgfqpoint{1.514867in}{2.448033in}}%
\pgfpathlineto{\pgfqpoint{1.519162in}{2.451906in}}%
\pgfpathlineto{\pgfqpoint{1.520236in}{2.449127in}}%
\pgfpathlineto{\pgfqpoint{1.521310in}{2.450306in}}%
\pgfpathlineto{\pgfqpoint{1.522384in}{2.449422in}}%
\pgfpathlineto{\pgfqpoint{1.523457in}{2.446854in}}%
\pgfpathlineto{\pgfqpoint{1.526679in}{2.447907in}}%
\pgfpathlineto{\pgfqpoint{1.528827in}{2.436289in}}%
\pgfpathlineto{\pgfqpoint{1.529900in}{2.429470in}}%
\pgfpathlineto{\pgfqpoint{1.530974in}{2.434732in}}%
\pgfpathlineto{\pgfqpoint{1.534196in}{2.432795in}}%
\pgfpathlineto{\pgfqpoint{1.535270in}{2.437215in}}%
\pgfpathlineto{\pgfqpoint{1.536343in}{2.436163in}}%
\pgfpathlineto{\pgfqpoint{1.541713in}{2.435868in}}%
\pgfpathlineto{\pgfqpoint{1.542786in}{2.435321in}}%
\pgfpathlineto{\pgfqpoint{1.543860in}{2.436584in}}%
\pgfpathlineto{\pgfqpoint{1.544934in}{2.428460in}}%
\pgfpathlineto{\pgfqpoint{1.546008in}{2.427744in}}%
\pgfpathlineto{\pgfqpoint{1.549229in}{2.428712in}}%
\pgfpathlineto{\pgfqpoint{1.550303in}{2.427449in}}%
\pgfpathlineto{\pgfqpoint{1.551377in}{2.430733in}}%
\pgfpathlineto{\pgfqpoint{1.552451in}{2.427870in}}%
\pgfpathlineto{\pgfqpoint{1.553525in}{2.432080in}}%
\pgfpathlineto{\pgfqpoint{1.556746in}{2.432374in}}%
\pgfpathlineto{\pgfqpoint{1.557820in}{2.430438in}}%
\pgfpathlineto{\pgfqpoint{1.558894in}{2.434521in}}%
\pgfpathlineto{\pgfqpoint{1.559968in}{2.435531in}}%
\pgfpathlineto{\pgfqpoint{1.561042in}{2.432416in}}%
\pgfpathlineto{\pgfqpoint{1.565337in}{2.438983in}}%
\pgfpathlineto{\pgfqpoint{1.566411in}{2.439909in}}%
\pgfpathlineto{\pgfqpoint{1.567485in}{2.443529in}}%
\pgfpathlineto{\pgfqpoint{1.568559in}{2.442056in}}%
\pgfpathlineto{\pgfqpoint{1.571780in}{2.442561in}}%
\pgfpathlineto{\pgfqpoint{1.572854in}{2.443319in}}%
\pgfpathlineto{\pgfqpoint{1.575002in}{2.441509in}}%
\pgfpathlineto{\pgfqpoint{1.576075in}{2.444413in}}%
\pgfpathlineto{\pgfqpoint{1.580371in}{2.443024in}}%
\pgfpathlineto{\pgfqpoint{1.581445in}{2.444497in}}%
\pgfpathlineto{\pgfqpoint{1.582518in}{2.444834in}}%
\pgfpathlineto{\pgfqpoint{1.583592in}{2.446939in}}%
\pgfpathlineto{\pgfqpoint{1.587888in}{2.444666in}}%
\pgfpathlineto{\pgfqpoint{1.588961in}{2.449085in}}%
\pgfpathlineto{\pgfqpoint{1.590035in}{2.447444in}}%
\pgfpathlineto{\pgfqpoint{1.594331in}{2.448075in}}%
\pgfpathlineto{\pgfqpoint{1.595405in}{2.452369in}}%
\pgfpathlineto{\pgfqpoint{1.596478in}{2.453505in}}%
\pgfpathlineto{\pgfqpoint{1.598626in}{2.460156in}}%
\pgfpathlineto{\pgfqpoint{1.601848in}{2.459735in}}%
\pgfpathlineto{\pgfqpoint{1.602921in}{2.458219in}}%
\pgfpathlineto{\pgfqpoint{1.603995in}{2.462555in}}%
\pgfpathlineto{\pgfqpoint{1.605069in}{2.456872in}}%
\pgfpathlineto{\pgfqpoint{1.606143in}{2.456872in}}%
\pgfpathlineto{\pgfqpoint{1.609364in}{2.454810in}}%
\pgfpathlineto{\pgfqpoint{1.610438in}{2.454978in}}%
\pgfpathlineto{\pgfqpoint{1.611512in}{2.446518in}}%
\pgfpathlineto{\pgfqpoint{1.612586in}{2.444834in}}%
\pgfpathlineto{\pgfqpoint{1.613660in}{2.449590in}}%
\pgfpathlineto{\pgfqpoint{1.616881in}{2.448580in}}%
\pgfpathlineto{\pgfqpoint{1.617955in}{2.439446in}}%
\pgfpathlineto{\pgfqpoint{1.619029in}{2.448791in}}%
\pgfpathlineto{\pgfqpoint{1.620103in}{2.438309in}}%
\pgfpathlineto{\pgfqpoint{1.624398in}{2.427365in}}%
\pgfpathlineto{\pgfqpoint{1.625472in}{2.419536in}}%
\pgfpathlineto{\pgfqpoint{1.626546in}{2.423998in}}%
\pgfpathlineto{\pgfqpoint{1.627620in}{2.418694in}}%
\pgfpathlineto{\pgfqpoint{1.628694in}{2.425808in}}%
\pgfpathlineto{\pgfqpoint{1.631915in}{2.427660in}}%
\pgfpathlineto{\pgfqpoint{1.635137in}{2.440330in}}%
\pgfpathlineto{\pgfqpoint{1.636210in}{2.442182in}}%
\pgfpathlineto{\pgfqpoint{1.639432in}{2.445676in}}%
\pgfpathlineto{\pgfqpoint{1.641580in}{2.451190in}}%
\pgfpathlineto{\pgfqpoint{1.643727in}{2.459398in}}%
\pgfpathlineto{\pgfqpoint{1.646949in}{2.458219in}}%
\pgfpathlineto{\pgfqpoint{1.648023in}{2.462471in}}%
\pgfpathlineto{\pgfqpoint{1.650170in}{2.463944in}}%
\pgfpathlineto{\pgfqpoint{1.651244in}{2.460955in}}%
\pgfpathlineto{\pgfqpoint{1.655539in}{2.463565in}}%
\pgfpathlineto{\pgfqpoint{1.656613in}{2.462976in}}%
\pgfpathlineto{\pgfqpoint{1.657687in}{2.464155in}}%
\pgfpathlineto{\pgfqpoint{1.658761in}{2.460787in}}%
\pgfpathlineto{\pgfqpoint{1.661983in}{2.461292in}}%
\pgfpathlineto{\pgfqpoint{1.663056in}{2.463271in}}%
\pgfpathlineto{\pgfqpoint{1.664130in}{2.463018in}}%
\pgfpathlineto{\pgfqpoint{1.665204in}{2.460871in}}%
\pgfpathlineto{\pgfqpoint{1.666278in}{2.462260in}}%
\pgfpathlineto{\pgfqpoint{1.670573in}{2.457967in}}%
\pgfpathlineto{\pgfqpoint{1.673795in}{2.463734in}}%
\pgfpathlineto{\pgfqpoint{1.677016in}{2.462892in}}%
\pgfpathlineto{\pgfqpoint{1.678090in}{2.464702in}}%
\pgfpathlineto{\pgfqpoint{1.679164in}{2.461755in}}%
\pgfpathlineto{\pgfqpoint{1.680238in}{2.461166in}}%
\pgfpathlineto{\pgfqpoint{1.681312in}{2.464702in}}%
\pgfpathlineto{\pgfqpoint{1.684533in}{2.464744in}}%
\pgfpathlineto{\pgfqpoint{1.685607in}{2.462976in}}%
\pgfpathlineto{\pgfqpoint{1.686681in}{2.456241in}}%
\pgfpathlineto{\pgfqpoint{1.687755in}{2.458051in}}%
\pgfpathlineto{\pgfqpoint{1.688828in}{2.449506in}}%
\pgfpathlineto{\pgfqpoint{1.692050in}{2.447780in}}%
\pgfpathlineto{\pgfqpoint{1.693124in}{2.443276in}}%
\pgfpathlineto{\pgfqpoint{1.694198in}{2.448159in}}%
\pgfpathlineto{\pgfqpoint{1.695272in}{2.458388in}}%
\pgfpathlineto{\pgfqpoint{1.696345in}{2.453673in}}%
\pgfpathlineto{\pgfqpoint{1.699567in}{2.458093in}}%
\pgfpathlineto{\pgfqpoint{1.700641in}{2.448959in}}%
\pgfpathlineto{\pgfqpoint{1.703862in}{2.451863in}}%
\pgfpathlineto{\pgfqpoint{1.708158in}{2.452958in}}%
\pgfpathlineto{\pgfqpoint{1.709231in}{2.450011in}}%
\pgfpathlineto{\pgfqpoint{1.711379in}{2.449843in}}%
\pgfpathlineto{\pgfqpoint{1.714601in}{2.447149in}}%
\pgfpathlineto{\pgfqpoint{1.715674in}{2.445213in}}%
\pgfpathlineto{\pgfqpoint{1.716748in}{2.453716in}}%
\pgfpathlineto{\pgfqpoint{1.717822in}{2.456788in}}%
\pgfpathlineto{\pgfqpoint{1.718896in}{2.451400in}}%
\pgfpathlineto{\pgfqpoint{1.722117in}{2.450053in}}%
\pgfpathlineto{\pgfqpoint{1.723191in}{2.450727in}}%
\pgfpathlineto{\pgfqpoint{1.724265in}{2.447907in}}%
\pgfpathlineto{\pgfqpoint{1.725339in}{2.442308in}}%
\pgfpathlineto{\pgfqpoint{1.726413in}{2.448075in}}%
\pgfpathlineto{\pgfqpoint{1.730708in}{2.437804in}}%
\pgfpathlineto{\pgfqpoint{1.731782in}{2.440077in}}%
\pgfpathlineto{\pgfqpoint{1.732856in}{2.447023in}}%
\pgfpathlineto{\pgfqpoint{1.733930in}{2.441214in}}%
\pgfpathlineto{\pgfqpoint{1.737151in}{2.441424in}}%
\pgfpathlineto{\pgfqpoint{1.738225in}{2.440793in}}%
\pgfpathlineto{\pgfqpoint{1.739299in}{2.438520in}}%
\pgfpathlineto{\pgfqpoint{1.740373in}{2.441887in}}%
\pgfpathlineto{\pgfqpoint{1.741447in}{2.433553in}}%
\pgfpathlineto{\pgfqpoint{1.744668in}{2.436121in}}%
\pgfpathlineto{\pgfqpoint{1.745742in}{2.442182in}}%
\pgfpathlineto{\pgfqpoint{1.746816in}{2.438099in}}%
\pgfpathlineto{\pgfqpoint{1.747890in}{2.442182in}}%
\pgfpathlineto{\pgfqpoint{1.748963in}{2.437131in}}%
\pgfpathlineto{\pgfqpoint{1.752185in}{2.432206in}}%
\pgfpathlineto{\pgfqpoint{1.753259in}{2.434311in}}%
\pgfpathlineto{\pgfqpoint{1.754333in}{2.434437in}}%
\pgfpathlineto{\pgfqpoint{1.755406in}{2.427197in}}%
\pgfpathlineto{\pgfqpoint{1.756480in}{2.431617in}}%
\pgfpathlineto{\pgfqpoint{1.760776in}{2.434647in}}%
\pgfpathlineto{\pgfqpoint{1.761850in}{2.432879in}}%
\pgfpathlineto{\pgfqpoint{1.762923in}{2.435616in}}%
\pgfpathlineto{\pgfqpoint{1.763997in}{2.436584in}}%
\pgfpathlineto{\pgfqpoint{1.767219in}{2.436289in}}%
\pgfpathlineto{\pgfqpoint{1.769366in}{2.440162in}}%
\pgfpathlineto{\pgfqpoint{1.770440in}{2.446139in}}%
\pgfpathlineto{\pgfqpoint{1.771514in}{2.445044in}}%
\pgfpathlineto{\pgfqpoint{1.774736in}{2.447696in}}%
\pgfpathlineto{\pgfqpoint{1.776883in}{2.441803in}}%
\pgfpathlineto{\pgfqpoint{1.777957in}{2.445086in}}%
\pgfpathlineto{\pgfqpoint{1.779031in}{2.436036in}}%
\pgfpathlineto{\pgfqpoint{1.782252in}{2.438099in}}%
\pgfpathlineto{\pgfqpoint{1.784400in}{2.429344in}}%
\pgfpathlineto{\pgfqpoint{1.785474in}{2.434984in}}%
\pgfpathlineto{\pgfqpoint{1.786548in}{2.432669in}}%
\pgfpathlineto{\pgfqpoint{1.789769in}{2.439614in}}%
\pgfpathlineto{\pgfqpoint{1.790843in}{2.435195in}}%
\pgfpathlineto{\pgfqpoint{1.791917in}{2.441088in}}%
\pgfpathlineto{\pgfqpoint{1.792991in}{2.441972in}}%
\pgfpathlineto{\pgfqpoint{1.794065in}{2.444623in}}%
\pgfpathlineto{\pgfqpoint{1.797286in}{2.446812in}}%
\pgfpathlineto{\pgfqpoint{1.798360in}{2.442982in}}%
\pgfpathlineto{\pgfqpoint{1.799434in}{2.436920in}}%
\pgfpathlineto{\pgfqpoint{1.800508in}{2.436163in}}%
\pgfpathlineto{\pgfqpoint{1.801582in}{2.436920in}}%
\pgfpathlineto{\pgfqpoint{1.804803in}{2.441424in}}%
\pgfpathlineto{\pgfqpoint{1.806951in}{2.432459in}}%
\pgfpathlineto{\pgfqpoint{1.808025in}{2.434269in}}%
\pgfpathlineto{\pgfqpoint{1.812320in}{2.432459in}}%
\pgfpathlineto{\pgfqpoint{1.813394in}{2.435994in}}%
\pgfpathlineto{\pgfqpoint{1.814468in}{2.436205in}}%
\pgfpathlineto{\pgfqpoint{1.816615in}{2.443361in}}%
\pgfpathlineto{\pgfqpoint{1.819837in}{2.437678in}}%
\pgfpathlineto{\pgfqpoint{1.821984in}{2.437889in}}%
\pgfpathlineto{\pgfqpoint{1.823058in}{2.434858in}}%
\pgfpathlineto{\pgfqpoint{1.824132in}{2.434058in}}%
\pgfpathlineto{\pgfqpoint{1.828427in}{2.436752in}}%
\pgfpathlineto{\pgfqpoint{1.830575in}{2.437341in}}%
\pgfpathlineto{\pgfqpoint{1.831649in}{2.439656in}}%
\pgfpathlineto{\pgfqpoint{1.834871in}{2.437804in}}%
\pgfpathlineto{\pgfqpoint{1.835944in}{2.438394in}}%
\pgfpathlineto{\pgfqpoint{1.837018in}{2.437089in}}%
\pgfpathlineto{\pgfqpoint{1.838092in}{2.432627in}}%
\pgfpathlineto{\pgfqpoint{1.839166in}{2.436121in}}%
\pgfpathlineto{\pgfqpoint{1.842387in}{2.436920in}}%
\pgfpathlineto{\pgfqpoint{1.843461in}{2.433763in}}%
\pgfpathlineto{\pgfqpoint{1.844535in}{2.432501in}}%
\pgfpathlineto{\pgfqpoint{1.845609in}{2.434353in}}%
\pgfpathlineto{\pgfqpoint{1.846683in}{2.441130in}}%
\pgfpathlineto{\pgfqpoint{1.849904in}{2.439488in}}%
\pgfpathlineto{\pgfqpoint{1.850978in}{2.437383in}}%
\pgfpathlineto{\pgfqpoint{1.852052in}{2.437678in}}%
\pgfpathlineto{\pgfqpoint{1.853126in}{2.442477in}}%
\pgfpathlineto{\pgfqpoint{1.854200in}{2.444245in}}%
\pgfpathlineto{\pgfqpoint{1.858495in}{2.450474in}}%
\pgfpathlineto{\pgfqpoint{1.860643in}{2.447486in}}%
\pgfpathlineto{\pgfqpoint{1.861716in}{2.443445in}}%
\pgfpathlineto{\pgfqpoint{1.866012in}{2.441466in}}%
\pgfpathlineto{\pgfqpoint{1.867086in}{2.442645in}}%
\pgfpathlineto{\pgfqpoint{1.868160in}{2.442687in}}%
\pgfpathlineto{\pgfqpoint{1.869233in}{2.438899in}}%
\pgfpathlineto{\pgfqpoint{1.874603in}{2.438688in}}%
\pgfpathlineto{\pgfqpoint{1.876750in}{2.433048in}}%
\pgfpathlineto{\pgfqpoint{1.879972in}{2.430649in}}%
\pgfpathlineto{\pgfqpoint{1.881046in}{2.431617in}}%
\pgfpathlineto{\pgfqpoint{1.883193in}{2.435489in}}%
\pgfpathlineto{\pgfqpoint{1.884267in}{2.432122in}}%
\pgfpathlineto{\pgfqpoint{1.887489in}{2.428881in}}%
\pgfpathlineto{\pgfqpoint{1.888562in}{2.432164in}}%
\pgfpathlineto{\pgfqpoint{1.889636in}{2.433553in}}%
\pgfpathlineto{\pgfqpoint{1.890710in}{2.439699in}}%
\pgfpathlineto{\pgfqpoint{1.891784in}{2.437846in}}%
\pgfpathlineto{\pgfqpoint{1.895005in}{2.438688in}}%
\pgfpathlineto{\pgfqpoint{1.898227in}{2.435026in}}%
\pgfpathlineto{\pgfqpoint{1.899301in}{2.437005in}}%
\pgfpathlineto{\pgfqpoint{1.902522in}{2.429596in}}%
\pgfpathlineto{\pgfqpoint{1.903596in}{2.428754in}}%
\pgfpathlineto{\pgfqpoint{1.904670in}{2.432585in}}%
\pgfpathlineto{\pgfqpoint{1.910039in}{2.431575in}}%
\pgfpathlineto{\pgfqpoint{1.911113in}{2.434269in}}%
\pgfpathlineto{\pgfqpoint{1.912187in}{2.429933in}}%
\pgfpathlineto{\pgfqpoint{1.913261in}{2.432332in}}%
\pgfpathlineto{\pgfqpoint{1.914335in}{2.436584in}}%
\pgfpathlineto{\pgfqpoint{1.917556in}{2.439362in}}%
\pgfpathlineto{\pgfqpoint{1.918630in}{2.437510in}}%
\pgfpathlineto{\pgfqpoint{1.920778in}{2.442561in}}%
\pgfpathlineto{\pgfqpoint{1.921851in}{2.438646in}}%
\pgfpathlineto{\pgfqpoint{1.926147in}{2.439656in}}%
\pgfpathlineto{\pgfqpoint{1.927221in}{2.439025in}}%
\pgfpathlineto{\pgfqpoint{1.928294in}{2.439109in}}%
\pgfpathlineto{\pgfqpoint{1.929368in}{2.435153in}}%
\pgfpathlineto{\pgfqpoint{1.932590in}{2.431869in}}%
\pgfpathlineto{\pgfqpoint{1.934738in}{2.437341in}}%
\pgfpathlineto{\pgfqpoint{1.935811in}{2.437762in}}%
\pgfpathlineto{\pgfqpoint{1.936885in}{2.439151in}}%
\pgfpathlineto{\pgfqpoint{1.940107in}{2.438436in}}%
\pgfpathlineto{\pgfqpoint{1.941181in}{2.437594in}}%
\pgfpathlineto{\pgfqpoint{1.942254in}{2.440330in}}%
\pgfpathlineto{\pgfqpoint{1.943328in}{2.434858in}}%
\pgfpathlineto{\pgfqpoint{1.944402in}{2.434016in}}%
\pgfpathlineto{\pgfqpoint{1.947624in}{2.437552in}}%
\pgfpathlineto{\pgfqpoint{1.948697in}{2.434605in}}%
\pgfpathlineto{\pgfqpoint{1.950845in}{2.432627in}}%
\pgfpathlineto{\pgfqpoint{1.955140in}{2.437889in}}%
\pgfpathlineto{\pgfqpoint{1.956214in}{2.435994in}}%
\pgfpathlineto{\pgfqpoint{1.957288in}{2.435742in}}%
\pgfpathlineto{\pgfqpoint{1.958362in}{2.433806in}}%
\pgfpathlineto{\pgfqpoint{1.959436in}{2.424335in}}%
\pgfpathlineto{\pgfqpoint{1.962657in}{2.413938in}}%
\pgfpathlineto{\pgfqpoint{1.963731in}{2.405982in}}%
\pgfpathlineto{\pgfqpoint{1.964805in}{2.422651in}}%
\pgfpathlineto{\pgfqpoint{1.965879in}{2.426860in}}%
\pgfpathlineto{\pgfqpoint{1.966953in}{2.422861in}}%
\pgfpathlineto{\pgfqpoint{1.970174in}{2.418357in}}%
\pgfpathlineto{\pgfqpoint{1.971248in}{2.411160in}}%
\pgfpathlineto{\pgfqpoint{1.972322in}{2.415958in}}%
\pgfpathlineto{\pgfqpoint{1.973396in}{2.413264in}}%
\pgfpathlineto{\pgfqpoint{1.974470in}{2.408171in}}%
\pgfpathlineto{\pgfqpoint{1.978765in}{2.418189in}}%
\pgfpathlineto{\pgfqpoint{1.979839in}{2.411665in}}%
\pgfpathlineto{\pgfqpoint{1.981986in}{2.414359in}}%
\pgfpathlineto{\pgfqpoint{1.985208in}{2.415832in}}%
\pgfpathlineto{\pgfqpoint{1.986282in}{2.419957in}}%
\pgfpathlineto{\pgfqpoint{1.988429in}{2.421599in}}%
\pgfpathlineto{\pgfqpoint{1.989503in}{2.416084in}}%
\pgfpathlineto{\pgfqpoint{1.992725in}{2.415116in}}%
\pgfpathlineto{\pgfqpoint{1.993799in}{2.415537in}}%
\pgfpathlineto{\pgfqpoint{1.994872in}{2.414569in}}%
\pgfpathlineto{\pgfqpoint{1.995946in}{2.412633in}}%
\pgfpathlineto{\pgfqpoint{1.997020in}{2.406992in}}%
\pgfpathlineto{\pgfqpoint{2.000242in}{2.408423in}}%
\pgfpathlineto{\pgfqpoint{2.001315in}{2.414780in}}%
\pgfpathlineto{\pgfqpoint{2.002389in}{2.415958in}}%
\pgfpathlineto{\pgfqpoint{2.003463in}{2.415243in}}%
\pgfpathlineto{\pgfqpoint{2.004537in}{2.418147in}}%
\pgfpathlineto{\pgfqpoint{2.007759in}{2.421304in}}%
\pgfpathlineto{\pgfqpoint{2.008832in}{2.416211in}}%
\pgfpathlineto{\pgfqpoint{2.009906in}{2.422146in}}%
\pgfpathlineto{\pgfqpoint{2.010980in}{2.422525in}}%
\pgfpathlineto{\pgfqpoint{2.012054in}{2.423619in}}%
\pgfpathlineto{\pgfqpoint{2.015275in}{2.425976in}}%
\pgfpathlineto{\pgfqpoint{2.016349in}{2.423956in}}%
\pgfpathlineto{\pgfqpoint{2.017423in}{2.420420in}}%
\pgfpathlineto{\pgfqpoint{2.018497in}{2.430396in}}%
\pgfpathlineto{\pgfqpoint{2.019571in}{2.434563in}}%
\pgfpathlineto{\pgfqpoint{2.022792in}{2.433427in}}%
\pgfpathlineto{\pgfqpoint{2.023866in}{2.432080in}}%
\pgfpathlineto{\pgfqpoint{2.024940in}{2.432248in}}%
\pgfpathlineto{\pgfqpoint{2.026014in}{2.439446in}}%
\pgfpathlineto{\pgfqpoint{2.027088in}{2.442435in}}%
\pgfpathlineto{\pgfqpoint{2.030309in}{2.441003in}}%
\pgfpathlineto{\pgfqpoint{2.032457in}{2.443066in}}%
\pgfpathlineto{\pgfqpoint{2.033531in}{2.446476in}}%
\pgfpathlineto{\pgfqpoint{2.034604in}{2.445171in}}%
\pgfpathlineto{\pgfqpoint{2.037826in}{2.449506in}}%
\pgfpathlineto{\pgfqpoint{2.038900in}{2.448706in}}%
\pgfpathlineto{\pgfqpoint{2.039974in}{2.448622in}}%
\pgfpathlineto{\pgfqpoint{2.041048in}{2.450096in}}%
\pgfpathlineto{\pgfqpoint{2.045343in}{2.444455in}}%
\pgfpathlineto{\pgfqpoint{2.047491in}{2.448328in}}%
\pgfpathlineto{\pgfqpoint{2.048564in}{2.442266in}}%
\pgfpathlineto{\pgfqpoint{2.049638in}{2.440793in}}%
\pgfpathlineto{\pgfqpoint{2.053934in}{2.446939in}}%
\pgfpathlineto{\pgfqpoint{2.055007in}{2.451400in}}%
\pgfpathlineto{\pgfqpoint{2.056081in}{2.450727in}}%
\pgfpathlineto{\pgfqpoint{2.057155in}{2.453547in}}%
\pgfpathlineto{\pgfqpoint{2.060377in}{2.454684in}}%
\pgfpathlineto{\pgfqpoint{2.061450in}{2.451863in}}%
\pgfpathlineto{\pgfqpoint{2.062524in}{2.451569in}}%
\pgfpathlineto{\pgfqpoint{2.064672in}{2.453126in}}%
\pgfpathlineto{\pgfqpoint{2.067893in}{2.448791in}}%
\pgfpathlineto{\pgfqpoint{2.068967in}{2.453084in}}%
\pgfpathlineto{\pgfqpoint{2.070041in}{2.451906in}}%
\pgfpathlineto{\pgfqpoint{2.071115in}{2.447065in}}%
\pgfpathlineto{\pgfqpoint{2.072189in}{2.455357in}}%
\pgfpathlineto{\pgfqpoint{2.075410in}{2.456788in}}%
\pgfpathlineto{\pgfqpoint{2.076484in}{2.453337in}}%
\pgfpathlineto{\pgfqpoint{2.077558in}{2.452284in}}%
\pgfpathlineto{\pgfqpoint{2.078632in}{2.454179in}}%
\pgfpathlineto{\pgfqpoint{2.079706in}{2.450474in}}%
\pgfpathlineto{\pgfqpoint{2.082927in}{2.452284in}}%
\pgfpathlineto{\pgfqpoint{2.085075in}{2.464155in}}%
\pgfpathlineto{\pgfqpoint{2.087223in}{2.451527in}}%
\pgfpathlineto{\pgfqpoint{2.090444in}{2.450096in}}%
\pgfpathlineto{\pgfqpoint{2.092592in}{2.457420in}}%
\pgfpathlineto{\pgfqpoint{2.093666in}{2.458304in}}%
\pgfpathlineto{\pgfqpoint{2.097961in}{2.456409in}}%
\pgfpathlineto{\pgfqpoint{2.099035in}{2.459482in}}%
\pgfpathlineto{\pgfqpoint{2.100109in}{2.458556in}}%
\pgfpathlineto{\pgfqpoint{2.101182in}{2.454473in}}%
\pgfpathlineto{\pgfqpoint{2.105478in}{2.445886in}}%
\pgfpathlineto{\pgfqpoint{2.106552in}{2.447486in}}%
\pgfpathlineto{\pgfqpoint{2.107626in}{2.445549in}}%
\pgfpathlineto{\pgfqpoint{2.109773in}{2.437005in}}%
\pgfpathlineto{\pgfqpoint{2.112995in}{2.434732in}}%
\pgfpathlineto{\pgfqpoint{2.114069in}{2.437299in}}%
\pgfpathlineto{\pgfqpoint{2.115142in}{2.432627in}}%
\pgfpathlineto{\pgfqpoint{2.116216in}{2.439783in}}%
\pgfpathlineto{\pgfqpoint{2.117290in}{2.432543in}}%
\pgfpathlineto{\pgfqpoint{2.121585in}{2.434479in}}%
\pgfpathlineto{\pgfqpoint{2.122659in}{2.427744in}}%
\pgfpathlineto{\pgfqpoint{2.123733in}{2.428502in}}%
\pgfpathlineto{\pgfqpoint{2.124807in}{2.431617in}}%
\pgfpathlineto{\pgfqpoint{2.128028in}{2.430270in}}%
\pgfpathlineto{\pgfqpoint{2.129102in}{2.448580in}}%
\pgfpathlineto{\pgfqpoint{2.130176in}{2.452326in}}%
\pgfpathlineto{\pgfqpoint{2.131250in}{2.452747in}}%
\pgfpathlineto{\pgfqpoint{2.132324in}{2.461082in}}%
\pgfpathlineto{\pgfqpoint{2.135545in}{2.460787in}}%
\pgfpathlineto{\pgfqpoint{2.136619in}{2.457125in}}%
\pgfpathlineto{\pgfqpoint{2.137693in}{2.459903in}}%
\pgfpathlineto{\pgfqpoint{2.138767in}{2.459019in}}%
\pgfpathlineto{\pgfqpoint{2.139841in}{2.446139in}}%
\pgfpathlineto{\pgfqpoint{2.143062in}{2.451737in}}%
\pgfpathlineto{\pgfqpoint{2.144136in}{2.451611in}}%
\pgfpathlineto{\pgfqpoint{2.145210in}{2.450685in}}%
\pgfpathlineto{\pgfqpoint{2.146284in}{2.450559in}}%
\pgfpathlineto{\pgfqpoint{2.152727in}{2.453631in}}%
\pgfpathlineto{\pgfqpoint{2.153801in}{2.460324in}}%
\pgfpathlineto{\pgfqpoint{2.154874in}{2.462892in}}%
\pgfpathlineto{\pgfqpoint{2.158096in}{2.465165in}}%
\pgfpathlineto{\pgfqpoint{2.159170in}{2.462597in}}%
\pgfpathlineto{\pgfqpoint{2.160244in}{2.465965in}}%
\pgfpathlineto{\pgfqpoint{2.161317in}{2.471479in}}%
\pgfpathlineto{\pgfqpoint{2.162391in}{2.469164in}}%
\pgfpathlineto{\pgfqpoint{2.165613in}{2.466933in}}%
\pgfpathlineto{\pgfqpoint{2.166687in}{2.474720in}}%
\pgfpathlineto{\pgfqpoint{2.167760in}{2.474004in}}%
\pgfpathlineto{\pgfqpoint{2.168834in}{2.472489in}}%
\pgfpathlineto{\pgfqpoint{2.169908in}{2.471942in}}%
\pgfpathlineto{\pgfqpoint{2.173130in}{2.472868in}}%
\pgfpathlineto{\pgfqpoint{2.174203in}{2.470847in}}%
\pgfpathlineto{\pgfqpoint{2.176351in}{2.474383in}}%
\pgfpathlineto{\pgfqpoint{2.177425in}{2.476614in}}%
\pgfpathlineto{\pgfqpoint{2.181720in}{2.476782in}}%
\pgfpathlineto{\pgfqpoint{2.182794in}{2.475435in}}%
\pgfpathlineto{\pgfqpoint{2.183868in}{2.472868in}}%
\pgfpathlineto{\pgfqpoint{2.184942in}{2.475772in}}%
\pgfpathlineto{\pgfqpoint{2.188163in}{2.474930in}}%
\pgfpathlineto{\pgfqpoint{2.189237in}{2.475309in}}%
\pgfpathlineto{\pgfqpoint{2.190311in}{2.479645in}}%
\pgfpathlineto{\pgfqpoint{2.191385in}{2.478929in}}%
\pgfpathlineto{\pgfqpoint{2.195680in}{2.478592in}}%
\pgfpathlineto{\pgfqpoint{2.196754in}{2.482128in}}%
\pgfpathlineto{\pgfqpoint{2.197828in}{2.481497in}}%
\pgfpathlineto{\pgfqpoint{2.198902in}{2.478466in}}%
\pgfpathlineto{\pgfqpoint{2.199976in}{2.482297in}}%
\pgfpathlineto{\pgfqpoint{2.203197in}{2.479982in}}%
\pgfpathlineto{\pgfqpoint{2.205345in}{2.483181in}}%
\pgfpathlineto{\pgfqpoint{2.207492in}{2.481960in}}%
\pgfpathlineto{\pgfqpoint{2.210714in}{2.481455in}}%
\pgfpathlineto{\pgfqpoint{2.211788in}{2.483896in}}%
\pgfpathlineto{\pgfqpoint{2.212862in}{2.484948in}}%
\pgfpathlineto{\pgfqpoint{2.213936in}{2.484822in}}%
\pgfpathlineto{\pgfqpoint{2.215009in}{2.486127in}}%
\pgfpathlineto{\pgfqpoint{2.218231in}{2.489031in}}%
\pgfpathlineto{\pgfqpoint{2.220379in}{2.499260in}}%
\pgfpathlineto{\pgfqpoint{2.221452in}{2.499218in}}%
\pgfpathlineto{\pgfqpoint{2.222526in}{2.498250in}}%
\pgfpathlineto{\pgfqpoint{2.225748in}{2.498965in}}%
\pgfpathlineto{\pgfqpoint{2.226822in}{2.496650in}}%
\pgfpathlineto{\pgfqpoint{2.227895in}{2.496145in}}%
\pgfpathlineto{\pgfqpoint{2.230043in}{2.493451in}}%
\pgfpathlineto{\pgfqpoint{2.233265in}{2.496061in}}%
\pgfpathlineto{\pgfqpoint{2.234338in}{2.495808in}}%
\pgfpathlineto{\pgfqpoint{2.235412in}{2.493998in}}%
\pgfpathlineto{\pgfqpoint{2.236486in}{2.496566in}}%
\pgfpathlineto{\pgfqpoint{2.237560in}{2.496019in}}%
\pgfpathlineto{\pgfqpoint{2.240781in}{2.499807in}}%
\pgfpathlineto{\pgfqpoint{2.241855in}{2.503469in}}%
\pgfpathlineto{\pgfqpoint{2.245077in}{2.499176in}}%
\pgfpathlineto{\pgfqpoint{2.248298in}{2.502585in}}%
\pgfpathlineto{\pgfqpoint{2.249372in}{2.500228in}}%
\pgfpathlineto{\pgfqpoint{2.250446in}{2.499302in}}%
\pgfpathlineto{\pgfqpoint{2.251520in}{2.496440in}}%
\pgfpathlineto{\pgfqpoint{2.252594in}{2.498713in}}%
\pgfpathlineto{\pgfqpoint{2.255815in}{2.496861in}}%
\pgfpathlineto{\pgfqpoint{2.257963in}{2.501491in}}%
\pgfpathlineto{\pgfqpoint{2.259037in}{2.499723in}}%
\pgfpathlineto{\pgfqpoint{2.260111in}{2.500355in}}%
\pgfpathlineto{\pgfqpoint{2.264406in}{2.498923in}}%
\pgfpathlineto{\pgfqpoint{2.265480in}{2.499260in}}%
\pgfpathlineto{\pgfqpoint{2.266554in}{2.505911in}}%
\pgfpathlineto{\pgfqpoint{2.267627in}{2.506963in}}%
\pgfpathlineto{\pgfqpoint{2.270849in}{2.510920in}}%
\pgfpathlineto{\pgfqpoint{2.272997in}{2.511004in}}%
\pgfpathlineto{\pgfqpoint{2.274070in}{2.515676in}}%
\pgfpathlineto{\pgfqpoint{2.275144in}{2.515718in}}%
\pgfpathlineto{\pgfqpoint{2.278366in}{2.514877in}}%
\pgfpathlineto{\pgfqpoint{2.279440in}{2.516139in}}%
\pgfpathlineto{\pgfqpoint{2.280514in}{2.513403in}}%
\pgfpathlineto{\pgfqpoint{2.281587in}{2.514287in}}%
\pgfpathlineto{\pgfqpoint{2.282661in}{2.509783in}}%
\pgfpathlineto{\pgfqpoint{2.285883in}{2.513908in}}%
\pgfpathlineto{\pgfqpoint{2.286957in}{2.512477in}}%
\pgfpathlineto{\pgfqpoint{2.288030in}{2.513572in}}%
\pgfpathlineto{\pgfqpoint{2.289104in}{2.517150in}}%
\pgfpathlineto{\pgfqpoint{2.290178in}{2.510373in}}%
\pgfpathlineto{\pgfqpoint{2.293400in}{2.513951in}}%
\pgfpathlineto{\pgfqpoint{2.296621in}{2.532387in}}%
\pgfpathlineto{\pgfqpoint{2.297695in}{2.532345in}}%
\pgfpathlineto{\pgfqpoint{2.303064in}{2.537607in}}%
\pgfpathlineto{\pgfqpoint{2.304138in}{2.537102in}}%
\pgfpathlineto{\pgfqpoint{2.305212in}{2.538449in}}%
\pgfpathlineto{\pgfqpoint{2.310581in}{2.538996in}}%
\pgfpathlineto{\pgfqpoint{2.311655in}{2.539711in}}%
\pgfpathlineto{\pgfqpoint{2.312729in}{2.538996in}}%
\pgfpathlineto{\pgfqpoint{2.315950in}{2.539543in}}%
\pgfpathlineto{\pgfqpoint{2.317024in}{2.547751in}}%
\pgfpathlineto{\pgfqpoint{2.320246in}{2.546909in}}%
\pgfpathlineto{\pgfqpoint{2.323467in}{2.546362in}}%
\pgfpathlineto{\pgfqpoint{2.324541in}{2.547372in}}%
\pgfpathlineto{\pgfqpoint{2.326689in}{2.544678in}}%
\pgfpathlineto{\pgfqpoint{2.327762in}{2.547667in}}%
\pgfpathlineto{\pgfqpoint{2.330984in}{2.548340in}}%
\pgfpathlineto{\pgfqpoint{2.332058in}{2.546194in}}%
\pgfpathlineto{\pgfqpoint{2.333132in}{2.542532in}}%
\pgfpathlineto{\pgfqpoint{2.334205in}{2.542363in}}%
\pgfpathlineto{\pgfqpoint{2.335279in}{2.543837in}}%
\pgfpathlineto{\pgfqpoint{2.340648in}{2.540427in}}%
\pgfpathlineto{\pgfqpoint{2.341722in}{2.542027in}}%
\pgfpathlineto{\pgfqpoint{2.342796in}{2.539880in}}%
\pgfpathlineto{\pgfqpoint{2.346018in}{2.536344in}}%
\pgfpathlineto{\pgfqpoint{2.347091in}{2.528641in}}%
\pgfpathlineto{\pgfqpoint{2.348165in}{2.532429in}}%
\pgfpathlineto{\pgfqpoint{2.349239in}{2.530156in}}%
\pgfpathlineto{\pgfqpoint{2.350313in}{2.530156in}}%
\pgfpathlineto{\pgfqpoint{2.353535in}{2.527041in}}%
\pgfpathlineto{\pgfqpoint{2.354608in}{2.528262in}}%
\pgfpathlineto{\pgfqpoint{2.355682in}{2.525400in}}%
\pgfpathlineto{\pgfqpoint{2.356756in}{2.524853in}}%
\pgfpathlineto{\pgfqpoint{2.357830in}{2.526705in}}%
\pgfpathlineto{\pgfqpoint{2.361051in}{2.530156in}}%
\pgfpathlineto{\pgfqpoint{2.362125in}{2.528388in}}%
\pgfpathlineto{\pgfqpoint{2.364273in}{2.526873in}}%
\pgfpathlineto{\pgfqpoint{2.365347in}{2.527799in}}%
\pgfpathlineto{\pgfqpoint{2.369642in}{2.529483in}}%
\pgfpathlineto{\pgfqpoint{2.371790in}{2.528388in}}%
\pgfpathlineto{\pgfqpoint{2.372864in}{2.523548in}}%
\pgfpathlineto{\pgfqpoint{2.376085in}{2.527168in}}%
\pgfpathlineto{\pgfqpoint{2.377159in}{2.521106in}}%
\pgfpathlineto{\pgfqpoint{2.378233in}{2.522074in}}%
\pgfpathlineto{\pgfqpoint{2.379307in}{2.525105in}}%
\pgfpathlineto{\pgfqpoint{2.380380in}{2.523632in}}%
\pgfpathlineto{\pgfqpoint{2.383602in}{2.521317in}}%
\pgfpathlineto{\pgfqpoint{2.384676in}{2.522453in}}%
\pgfpathlineto{\pgfqpoint{2.386824in}{2.528346in}}%
\pgfpathlineto{\pgfqpoint{2.387897in}{2.525821in}}%
\pgfpathlineto{\pgfqpoint{2.391119in}{2.521780in}}%
\pgfpathlineto{\pgfqpoint{2.392193in}{2.527420in}}%
\pgfpathlineto{\pgfqpoint{2.393267in}{2.528094in}}%
\pgfpathlineto{\pgfqpoint{2.394340in}{2.519759in}}%
\pgfpathlineto{\pgfqpoint{2.395414in}{2.523127in}}%
\pgfpathlineto{\pgfqpoint{2.398636in}{2.525821in}}%
\pgfpathlineto{\pgfqpoint{2.399710in}{2.525863in}}%
\pgfpathlineto{\pgfqpoint{2.400783in}{2.527252in}}%
\pgfpathlineto{\pgfqpoint{2.401857in}{2.525568in}}%
\pgfpathlineto{\pgfqpoint{2.402931in}{2.527504in}}%
\pgfpathlineto{\pgfqpoint{2.406153in}{2.529693in}}%
\pgfpathlineto{\pgfqpoint{2.407226in}{2.521233in}}%
\pgfpathlineto{\pgfqpoint{2.409374in}{2.523632in}}%
\pgfpathlineto{\pgfqpoint{2.410448in}{2.520896in}}%
\pgfpathlineto{\pgfqpoint{2.413669in}{2.524558in}}%
\pgfpathlineto{\pgfqpoint{2.414743in}{2.512477in}}%
\pgfpathlineto{\pgfqpoint{2.415817in}{2.509278in}}%
\pgfpathlineto{\pgfqpoint{2.416891in}{2.510373in}}%
\pgfpathlineto{\pgfqpoint{2.417965in}{2.504774in}}%
\pgfpathlineto{\pgfqpoint{2.421186in}{2.505448in}}%
\pgfpathlineto{\pgfqpoint{2.422260in}{2.506837in}}%
\pgfpathlineto{\pgfqpoint{2.423334in}{2.509152in}}%
\pgfpathlineto{\pgfqpoint{2.424408in}{2.513614in}}%
\pgfpathlineto{\pgfqpoint{2.425482in}{2.512183in}}%
\pgfpathlineto{\pgfqpoint{2.428703in}{2.514750in}}%
\pgfpathlineto{\pgfqpoint{2.430851in}{2.510331in}}%
\pgfpathlineto{\pgfqpoint{2.432999in}{2.511341in}}%
\pgfpathlineto{\pgfqpoint{2.437294in}{2.518918in}}%
\pgfpathlineto{\pgfqpoint{2.438368in}{2.531672in}}%
\pgfpathlineto{\pgfqpoint{2.443737in}{2.517150in}}%
\pgfpathlineto{\pgfqpoint{2.444811in}{2.516055in}}%
\pgfpathlineto{\pgfqpoint{2.445885in}{2.516224in}}%
\pgfpathlineto{\pgfqpoint{2.446958in}{2.517023in}}%
\pgfpathlineto{\pgfqpoint{2.448032in}{2.515424in}}%
\pgfpathlineto{\pgfqpoint{2.451254in}{2.513993in}}%
\pgfpathlineto{\pgfqpoint{2.452328in}{2.505111in}}%
\pgfpathlineto{\pgfqpoint{2.453402in}{2.506374in}}%
\pgfpathlineto{\pgfqpoint{2.455549in}{2.510583in}}%
\pgfpathlineto{\pgfqpoint{2.458771in}{2.506626in}}%
\pgfpathlineto{\pgfqpoint{2.459845in}{2.504059in}}%
\pgfpathlineto{\pgfqpoint{2.460918in}{2.499428in}}%
\pgfpathlineto{\pgfqpoint{2.461992in}{2.499723in}}%
\pgfpathlineto{\pgfqpoint{2.463066in}{2.502038in}}%
\pgfpathlineto{\pgfqpoint{2.467361in}{2.502417in}}%
\pgfpathlineto{\pgfqpoint{2.468435in}{2.498629in}}%
\pgfpathlineto{\pgfqpoint{2.469509in}{2.498208in}}%
\pgfpathlineto{\pgfqpoint{2.470583in}{2.503217in}}%
\pgfpathlineto{\pgfqpoint{2.474878in}{2.517528in}}%
\pgfpathlineto{\pgfqpoint{2.475952in}{2.513951in}}%
\pgfpathlineto{\pgfqpoint{2.477026in}{2.517528in}}%
\pgfpathlineto{\pgfqpoint{2.478100in}{2.517486in}}%
\pgfpathlineto{\pgfqpoint{2.481321in}{2.518034in}}%
\pgfpathlineto{\pgfqpoint{2.483469in}{2.515213in}}%
\pgfpathlineto{\pgfqpoint{2.484543in}{2.515718in}}%
\pgfpathlineto{\pgfqpoint{2.485617in}{2.517781in}}%
\pgfpathlineto{\pgfqpoint{2.489912in}{2.517571in}}%
\pgfpathlineto{\pgfqpoint{2.490986in}{2.514414in}}%
\pgfpathlineto{\pgfqpoint{2.492060in}{2.515929in}}%
\pgfpathlineto{\pgfqpoint{2.493134in}{2.514834in}}%
\pgfpathlineto{\pgfqpoint{2.497429in}{2.517318in}}%
\pgfpathlineto{\pgfqpoint{2.498503in}{2.516560in}}%
\pgfpathlineto{\pgfqpoint{2.499577in}{2.521317in}}%
\pgfpathlineto{\pgfqpoint{2.500650in}{2.519128in}}%
\pgfpathlineto{\pgfqpoint{2.504946in}{2.518581in}}%
\pgfpathlineto{\pgfqpoint{2.506020in}{2.512940in}}%
\pgfpathlineto{\pgfqpoint{2.508167in}{2.512435in}}%
\pgfpathlineto{\pgfqpoint{2.512463in}{2.513488in}}%
\pgfpathlineto{\pgfqpoint{2.513536in}{2.512814in}}%
\pgfpathlineto{\pgfqpoint{2.514610in}{2.510836in}}%
\pgfpathlineto{\pgfqpoint{2.518906in}{2.509699in}}%
\pgfpathlineto{\pgfqpoint{2.519979in}{2.501238in}}%
\pgfpathlineto{\pgfqpoint{2.521053in}{2.505321in}}%
\pgfpathlineto{\pgfqpoint{2.522127in}{2.501533in}}%
\pgfpathlineto{\pgfqpoint{2.523201in}{2.507637in}}%
\pgfpathlineto{\pgfqpoint{2.526423in}{2.506626in}}%
\pgfpathlineto{\pgfqpoint{2.530718in}{2.508647in}}%
\pgfpathlineto{\pgfqpoint{2.533939in}{2.507679in}}%
\pgfpathlineto{\pgfqpoint{2.535013in}{2.508015in}}%
\pgfpathlineto{\pgfqpoint{2.536087in}{2.507679in}}%
\pgfpathlineto{\pgfqpoint{2.537161in}{2.510373in}}%
\pgfpathlineto{\pgfqpoint{2.538235in}{2.514961in}}%
\pgfpathlineto{\pgfqpoint{2.541456in}{2.517486in}}%
\pgfpathlineto{\pgfqpoint{2.542530in}{2.519381in}}%
\pgfpathlineto{\pgfqpoint{2.545752in}{2.529230in}}%
\pgfpathlineto{\pgfqpoint{2.550047in}{2.532387in}}%
\pgfpathlineto{\pgfqpoint{2.551121in}{2.531798in}}%
\pgfpathlineto{\pgfqpoint{2.553268in}{2.547667in}}%
\pgfpathlineto{\pgfqpoint{2.557564in}{2.545604in}}%
\pgfpathlineto{\pgfqpoint{2.558638in}{2.552129in}}%
\pgfpathlineto{\pgfqpoint{2.559712in}{2.551245in}}%
\pgfpathlineto{\pgfqpoint{2.560785in}{2.551876in}}%
\pgfpathlineto{\pgfqpoint{2.564007in}{2.551540in}}%
\pgfpathlineto{\pgfqpoint{2.566155in}{2.553097in}}%
\pgfpathlineto{\pgfqpoint{2.567228in}{2.560421in}}%
\pgfpathlineto{\pgfqpoint{2.568302in}{2.561473in}}%
\pgfpathlineto{\pgfqpoint{2.571524in}{2.563326in}}%
\pgfpathlineto{\pgfqpoint{2.572598in}{2.564799in}}%
\pgfpathlineto{\pgfqpoint{2.573671in}{2.572376in}}%
\pgfpathlineto{\pgfqpoint{2.575819in}{2.568798in}}%
\pgfpathlineto{\pgfqpoint{2.579041in}{2.568840in}}%
\pgfpathlineto{\pgfqpoint{2.582262in}{2.560253in}}%
\pgfpathlineto{\pgfqpoint{2.583336in}{2.558569in}}%
\pgfpathlineto{\pgfqpoint{2.586557in}{2.559832in}}%
\pgfpathlineto{\pgfqpoint{2.587631in}{2.559285in}}%
\pgfpathlineto{\pgfqpoint{2.588705in}{2.556338in}}%
\pgfpathlineto{\pgfqpoint{2.590853in}{2.554907in}}%
\pgfpathlineto{\pgfqpoint{2.596222in}{2.555875in}}%
\pgfpathlineto{\pgfqpoint{2.597296in}{2.556885in}}%
\pgfpathlineto{\pgfqpoint{2.598370in}{2.556338in}}%
\pgfpathlineto{\pgfqpoint{2.602665in}{2.553560in}}%
\pgfpathlineto{\pgfqpoint{2.603739in}{2.558274in}}%
\pgfpathlineto{\pgfqpoint{2.604813in}{2.556633in}}%
\pgfpathlineto{\pgfqpoint{2.609108in}{2.559537in}}%
\pgfpathlineto{\pgfqpoint{2.610182in}{2.544047in}}%
\pgfpathlineto{\pgfqpoint{2.611256in}{2.542279in}}%
\pgfpathlineto{\pgfqpoint{2.612330in}{2.544257in}}%
\pgfpathlineto{\pgfqpoint{2.613403in}{2.543837in}}%
\pgfpathlineto{\pgfqpoint{2.618773in}{2.550740in}}%
\pgfpathlineto{\pgfqpoint{2.619846in}{2.551666in}}%
\pgfpathlineto{\pgfqpoint{2.620920in}{2.550614in}}%
\pgfpathlineto{\pgfqpoint{2.624142in}{2.550066in}}%
\pgfpathlineto{\pgfqpoint{2.625216in}{2.551497in}}%
\pgfpathlineto{\pgfqpoint{2.626290in}{2.550066in}}%
\pgfpathlineto{\pgfqpoint{2.627363in}{2.552508in}}%
\pgfpathlineto{\pgfqpoint{2.628437in}{2.550740in}}%
\pgfpathlineto{\pgfqpoint{2.632733in}{2.549561in}}%
\pgfpathlineto{\pgfqpoint{2.633806in}{2.548004in}}%
\pgfpathlineto{\pgfqpoint{2.635954in}{2.551287in}}%
\pgfpathlineto{\pgfqpoint{2.640249in}{2.567661in}}%
\pgfpathlineto{\pgfqpoint{2.641323in}{2.563283in}}%
\pgfpathlineto{\pgfqpoint{2.642397in}{2.564504in}}%
\pgfpathlineto{\pgfqpoint{2.643471in}{2.564588in}}%
\pgfpathlineto{\pgfqpoint{2.646692in}{2.565641in}}%
\pgfpathlineto{\pgfqpoint{2.647766in}{2.566651in}}%
\pgfpathlineto{\pgfqpoint{2.648840in}{2.566609in}}%
\pgfpathlineto{\pgfqpoint{2.649914in}{2.570060in}}%
\pgfpathlineto{\pgfqpoint{2.650988in}{2.567619in}}%
\pgfpathlineto{\pgfqpoint{2.655283in}{2.568377in}}%
\pgfpathlineto{\pgfqpoint{2.658505in}{2.580247in}}%
\pgfpathlineto{\pgfqpoint{2.661726in}{2.581383in}}%
\pgfpathlineto{\pgfqpoint{2.662800in}{2.583236in}}%
\pgfpathlineto{\pgfqpoint{2.664948in}{2.582141in}}%
\pgfpathlineto{\pgfqpoint{2.666022in}{2.586056in}}%
\pgfpathlineto{\pgfqpoint{2.670317in}{2.587992in}}%
\pgfpathlineto{\pgfqpoint{2.671391in}{2.590981in}}%
\pgfpathlineto{\pgfqpoint{2.672465in}{2.592075in}}%
\pgfpathlineto{\pgfqpoint{2.673538in}{2.597295in}}%
\pgfpathlineto{\pgfqpoint{2.676760in}{2.596200in}}%
\pgfpathlineto{\pgfqpoint{2.677834in}{2.596789in}}%
\pgfpathlineto{\pgfqpoint{2.678908in}{2.599526in}}%
\pgfpathlineto{\pgfqpoint{2.679981in}{2.604072in}}%
\pgfpathlineto{\pgfqpoint{2.681055in}{2.605587in}}%
\pgfpathlineto{\pgfqpoint{2.684277in}{2.605250in}}%
\pgfpathlineto{\pgfqpoint{2.687498in}{2.590476in}}%
\pgfpathlineto{\pgfqpoint{2.688572in}{2.589086in}}%
\pgfpathlineto{\pgfqpoint{2.691794in}{2.591528in}}%
\pgfpathlineto{\pgfqpoint{2.693941in}{2.594474in}}%
\pgfpathlineto{\pgfqpoint{2.695015in}{2.589970in}}%
\pgfpathlineto{\pgfqpoint{2.696089in}{2.590055in}}%
\pgfpathlineto{\pgfqpoint{2.700384in}{2.584835in}}%
\pgfpathlineto{\pgfqpoint{2.701458in}{2.588834in}}%
\pgfpathlineto{\pgfqpoint{2.702532in}{2.587361in}}%
\pgfpathlineto{\pgfqpoint{2.703606in}{2.590307in}}%
\pgfpathlineto{\pgfqpoint{2.706827in}{2.588497in}}%
\pgfpathlineto{\pgfqpoint{2.707901in}{2.597758in}}%
\pgfpathlineto{\pgfqpoint{2.708975in}{2.600746in}}%
\pgfpathlineto{\pgfqpoint{2.710049in}{2.606176in}}%
\pgfpathlineto{\pgfqpoint{2.711123in}{2.601125in}}%
\pgfpathlineto{\pgfqpoint{2.715418in}{2.587445in}}%
\pgfpathlineto{\pgfqpoint{2.716492in}{2.583741in}}%
\pgfpathlineto{\pgfqpoint{2.717566in}{2.583236in}}%
\pgfpathlineto{\pgfqpoint{2.718640in}{2.587319in}}%
\pgfpathlineto{\pgfqpoint{2.721861in}{2.590770in}}%
\pgfpathlineto{\pgfqpoint{2.724009in}{2.588539in}}%
\pgfpathlineto{\pgfqpoint{2.725083in}{2.593296in}}%
\pgfpathlineto{\pgfqpoint{2.726156in}{2.592622in}}%
\pgfpathlineto{\pgfqpoint{2.729378in}{2.591359in}}%
\pgfpathlineto{\pgfqpoint{2.730452in}{2.589002in}}%
\pgfpathlineto{\pgfqpoint{2.731526in}{2.592833in}}%
\pgfpathlineto{\pgfqpoint{2.733673in}{2.592328in}}%
\pgfpathlineto{\pgfqpoint{2.736895in}{2.593843in}}%
\pgfpathlineto{\pgfqpoint{2.737969in}{2.593422in}}%
\pgfpathlineto{\pgfqpoint{2.739043in}{2.596579in}}%
\pgfpathlineto{\pgfqpoint{2.740116in}{2.592075in}}%
\pgfpathlineto{\pgfqpoint{2.741190in}{2.590433in}}%
\pgfpathlineto{\pgfqpoint{2.744412in}{2.593717in}}%
\pgfpathlineto{\pgfqpoint{2.745486in}{2.598684in}}%
\pgfpathlineto{\pgfqpoint{2.746559in}{2.591107in}}%
\pgfpathlineto{\pgfqpoint{2.747633in}{2.591486in}}%
\pgfpathlineto{\pgfqpoint{2.748707in}{2.589970in}}%
\pgfpathlineto{\pgfqpoint{2.751929in}{2.590223in}}%
\pgfpathlineto{\pgfqpoint{2.753002in}{2.592117in}}%
\pgfpathlineto{\pgfqpoint{2.754076in}{2.587529in}}%
\pgfpathlineto{\pgfqpoint{2.755150in}{2.592749in}}%
\pgfpathlineto{\pgfqpoint{2.756224in}{2.587361in}}%
\pgfpathlineto{\pgfqpoint{2.760519in}{2.582815in}}%
\pgfpathlineto{\pgfqpoint{2.761593in}{2.585929in}}%
\pgfpathlineto{\pgfqpoint{2.762667in}{2.592033in}}%
\pgfpathlineto{\pgfqpoint{2.763741in}{2.587150in}}%
\pgfpathlineto{\pgfqpoint{2.766962in}{2.596116in}}%
\pgfpathlineto{\pgfqpoint{2.768036in}{2.593801in}}%
\pgfpathlineto{\pgfqpoint{2.769110in}{2.593085in}}%
\pgfpathlineto{\pgfqpoint{2.770184in}{2.600031in}}%
\pgfpathlineto{\pgfqpoint{2.774479in}{2.604871in}}%
\pgfpathlineto{\pgfqpoint{2.775553in}{2.604198in}}%
\pgfpathlineto{\pgfqpoint{2.777701in}{2.590265in}}%
\pgfpathlineto{\pgfqpoint{2.778775in}{2.588834in}}%
\pgfpathlineto{\pgfqpoint{2.781996in}{2.587950in}}%
\pgfpathlineto{\pgfqpoint{2.783070in}{2.587024in}}%
\pgfpathlineto{\pgfqpoint{2.784144in}{2.582225in}}%
\pgfpathlineto{\pgfqpoint{2.785218in}{2.581089in}}%
\pgfpathlineto{\pgfqpoint{2.786291in}{2.583278in}}%
\pgfpathlineto{\pgfqpoint{2.789513in}{2.588118in}}%
\pgfpathlineto{\pgfqpoint{2.791661in}{2.594853in}}%
\pgfpathlineto{\pgfqpoint{2.792734in}{2.596032in}}%
\pgfpathlineto{\pgfqpoint{2.797030in}{2.597084in}}%
\pgfpathlineto{\pgfqpoint{2.798104in}{2.598894in}}%
\pgfpathlineto{\pgfqpoint{2.799178in}{2.609965in}}%
\pgfpathlineto{\pgfqpoint{2.800251in}{2.610680in}}%
\pgfpathlineto{\pgfqpoint{2.801325in}{2.609081in}}%
\pgfpathlineto{\pgfqpoint{2.804547in}{2.607818in}}%
\pgfpathlineto{\pgfqpoint{2.805621in}{2.626591in}}%
\pgfpathlineto{\pgfqpoint{2.806694in}{2.626170in}}%
\pgfpathlineto{\pgfqpoint{2.807768in}{2.631600in}}%
\pgfpathlineto{\pgfqpoint{2.812064in}{2.637956in}}%
\pgfpathlineto{\pgfqpoint{2.813137in}{2.630001in}}%
\pgfpathlineto{\pgfqpoint{2.814211in}{2.632905in}}%
\pgfpathlineto{\pgfqpoint{2.815285in}{2.630674in}}%
\pgfpathlineto{\pgfqpoint{2.816359in}{2.630590in}}%
\pgfpathlineto{\pgfqpoint{2.820654in}{2.621035in}}%
\pgfpathlineto{\pgfqpoint{2.821728in}{2.623350in}}%
\pgfpathlineto{\pgfqpoint{2.822802in}{2.623140in}}%
\pgfpathlineto{\pgfqpoint{2.823876in}{2.623729in}}%
\pgfpathlineto{\pgfqpoint{2.828171in}{2.622508in}}%
\pgfpathlineto{\pgfqpoint{2.829245in}{2.628738in}}%
\pgfpathlineto{\pgfqpoint{2.831393in}{2.621666in}}%
\pgfpathlineto{\pgfqpoint{2.834614in}{2.622466in}}%
\pgfpathlineto{\pgfqpoint{2.837836in}{2.618888in}}%
\pgfpathlineto{\pgfqpoint{2.838910in}{2.615395in}}%
\pgfpathlineto{\pgfqpoint{2.842131in}{2.615100in}}%
\pgfpathlineto{\pgfqpoint{2.843205in}{2.616742in}}%
\pgfpathlineto{\pgfqpoint{2.844279in}{2.612532in}}%
\pgfpathlineto{\pgfqpoint{2.849648in}{2.619183in}}%
\pgfpathlineto{\pgfqpoint{2.850722in}{2.626928in}}%
\pgfpathlineto{\pgfqpoint{2.851796in}{2.626086in}}%
\pgfpathlineto{\pgfqpoint{2.852869in}{2.624150in}}%
\pgfpathlineto{\pgfqpoint{2.853943in}{2.626802in}}%
\pgfpathlineto{\pgfqpoint{2.857165in}{2.622845in}}%
\pgfpathlineto{\pgfqpoint{2.858239in}{2.625539in}}%
\pgfpathlineto{\pgfqpoint{2.859312in}{2.631137in}}%
\pgfpathlineto{\pgfqpoint{2.860386in}{2.626886in}}%
\pgfpathlineto{\pgfqpoint{2.861460in}{2.629243in}}%
\pgfpathlineto{\pgfqpoint{2.864682in}{2.631474in}}%
\pgfpathlineto{\pgfqpoint{2.865756in}{2.637367in}}%
\pgfpathlineto{\pgfqpoint{2.866829in}{2.638546in}}%
\pgfpathlineto{\pgfqpoint{2.867903in}{2.633537in}}%
\pgfpathlineto{\pgfqpoint{2.868977in}{2.636820in}}%
\pgfpathlineto{\pgfqpoint{2.872199in}{2.634126in}}%
\pgfpathlineto{\pgfqpoint{2.873272in}{2.634042in}}%
\pgfpathlineto{\pgfqpoint{2.874346in}{2.631558in}}%
\pgfpathlineto{\pgfqpoint{2.875420in}{2.631137in}}%
\pgfpathlineto{\pgfqpoint{2.876494in}{2.627349in}}%
\pgfpathlineto{\pgfqpoint{2.880789in}{2.627223in}}%
\pgfpathlineto{\pgfqpoint{2.881863in}{2.629159in}}%
\pgfpathlineto{\pgfqpoint{2.882937in}{2.629117in}}%
\pgfpathlineto{\pgfqpoint{2.884011in}{2.625749in}}%
\pgfpathlineto{\pgfqpoint{2.884011in}{2.625749in}}%
\pgfusepath{stroke}%
\end{pgfscope}%
\begin{pgfscope}%
\pgfpathrectangle{\pgfqpoint{0.418102in}{2.255883in}}{\pgfqpoint{2.583333in}{0.400885in}}%
\pgfusepath{clip}%
\pgfsetroundcap%
\pgfsetroundjoin%
\pgfsetlinewidth{1.505625pt}%
\definecolor{currentstroke}{rgb}{0.580392,0.403922,0.741176}%
\pgfsetstrokecolor{currentstroke}%
\pgfsetdash{}{0pt}%
\pgfpathmoveto{\pgfqpoint{0.535526in}{2.284081in}}%
\pgfpathlineto{\pgfqpoint{0.537674in}{2.283827in}}%
\pgfpathlineto{\pgfqpoint{0.538747in}{2.285775in}}%
\pgfpathlineto{\pgfqpoint{0.541969in}{2.286120in}}%
\pgfpathlineto{\pgfqpoint{0.543043in}{2.285171in}}%
\pgfpathlineto{\pgfqpoint{0.544117in}{2.285428in}}%
\pgfpathlineto{\pgfqpoint{0.546264in}{2.285000in}}%
\pgfpathlineto{\pgfqpoint{0.551633in}{2.286026in}}%
\pgfpathlineto{\pgfqpoint{0.558077in}{2.287486in}}%
\pgfpathlineto{\pgfqpoint{0.559150in}{2.286750in}}%
\pgfpathlineto{\pgfqpoint{0.560224in}{2.285072in}}%
\pgfpathlineto{\pgfqpoint{0.561298in}{2.285538in}}%
\pgfpathlineto{\pgfqpoint{0.564520in}{2.286049in}}%
\pgfpathlineto{\pgfqpoint{0.565593in}{2.285368in}}%
\pgfpathlineto{\pgfqpoint{0.566667in}{2.286130in}}%
\pgfpathlineto{\pgfqpoint{0.568815in}{2.285959in}}%
\pgfpathlineto{\pgfqpoint{0.573110in}{2.287709in}}%
\pgfpathlineto{\pgfqpoint{0.574184in}{2.287752in}}%
\pgfpathlineto{\pgfqpoint{0.575258in}{2.288961in}}%
\pgfpathlineto{\pgfqpoint{0.576332in}{2.287913in}}%
\pgfpathlineto{\pgfqpoint{0.581701in}{2.288612in}}%
\pgfpathlineto{\pgfqpoint{0.583849in}{2.287389in}}%
\pgfpathlineto{\pgfqpoint{0.589218in}{2.287346in}}%
\pgfpathlineto{\pgfqpoint{0.591366in}{2.287952in}}%
\pgfpathlineto{\pgfqpoint{0.594587in}{2.287908in}}%
\pgfpathlineto{\pgfqpoint{0.595661in}{2.285349in}}%
\pgfpathlineto{\pgfqpoint{0.596735in}{2.285688in}}%
\pgfpathlineto{\pgfqpoint{0.598882in}{2.284625in}}%
\pgfpathlineto{\pgfqpoint{0.602104in}{2.285093in}}%
\pgfpathlineto{\pgfqpoint{0.603178in}{2.287007in}}%
\pgfpathlineto{\pgfqpoint{0.604252in}{2.286834in}}%
\pgfpathlineto{\pgfqpoint{0.605325in}{2.288781in}}%
\pgfpathlineto{\pgfqpoint{0.609621in}{2.290430in}}%
\pgfpathlineto{\pgfqpoint{0.610695in}{2.289605in}}%
\pgfpathlineto{\pgfqpoint{0.611768in}{2.290510in}}%
\pgfpathlineto{\pgfqpoint{0.617138in}{2.290380in}}%
\pgfpathlineto{\pgfqpoint{0.618211in}{2.291247in}}%
\pgfpathlineto{\pgfqpoint{0.620359in}{2.289456in}}%
\pgfpathlineto{\pgfqpoint{0.621433in}{2.289762in}}%
\pgfpathlineto{\pgfqpoint{0.626802in}{2.285998in}}%
\pgfpathlineto{\pgfqpoint{0.627876in}{2.286252in}}%
\pgfpathlineto{\pgfqpoint{0.628950in}{2.287692in}}%
\pgfpathlineto{\pgfqpoint{0.632171in}{2.286845in}}%
\pgfpathlineto{\pgfqpoint{0.633245in}{2.287979in}}%
\pgfpathlineto{\pgfqpoint{0.634319in}{2.286281in}}%
\pgfpathlineto{\pgfqpoint{0.635393in}{2.286112in}}%
\pgfpathlineto{\pgfqpoint{0.639688in}{2.284668in}}%
\pgfpathlineto{\pgfqpoint{0.640762in}{2.282206in}}%
\pgfpathlineto{\pgfqpoint{0.642910in}{2.282036in}}%
\pgfpathlineto{\pgfqpoint{0.643984in}{2.284159in}}%
\pgfpathlineto{\pgfqpoint{0.647205in}{2.285717in}}%
\pgfpathlineto{\pgfqpoint{0.648279in}{2.284894in}}%
\pgfpathlineto{\pgfqpoint{0.649353in}{2.288245in}}%
\pgfpathlineto{\pgfqpoint{0.650427in}{2.287449in}}%
\pgfpathlineto{\pgfqpoint{0.651500in}{2.289883in}}%
\pgfpathlineto{\pgfqpoint{0.654722in}{2.291122in}}%
\pgfpathlineto{\pgfqpoint{0.655796in}{2.292599in}}%
\pgfpathlineto{\pgfqpoint{0.657944in}{2.289088in}}%
\pgfpathlineto{\pgfqpoint{0.665460in}{2.287017in}}%
\pgfpathlineto{\pgfqpoint{0.666534in}{2.284972in}}%
\pgfpathlineto{\pgfqpoint{0.669756in}{2.285185in}}%
\pgfpathlineto{\pgfqpoint{0.670830in}{2.284588in}}%
\pgfpathlineto{\pgfqpoint{0.671903in}{2.287005in}}%
\pgfpathlineto{\pgfqpoint{0.674051in}{2.288867in}}%
\pgfpathlineto{\pgfqpoint{0.677273in}{2.287427in}}%
\pgfpathlineto{\pgfqpoint{0.678346in}{2.286293in}}%
\pgfpathlineto{\pgfqpoint{0.680494in}{2.287209in}}%
\pgfpathlineto{\pgfqpoint{0.681568in}{2.286508in}}%
\pgfpathlineto{\pgfqpoint{0.686937in}{2.287602in}}%
\pgfpathlineto{\pgfqpoint{0.689085in}{2.291309in}}%
\pgfpathlineto{\pgfqpoint{0.693380in}{2.291624in}}%
\pgfpathlineto{\pgfqpoint{0.696602in}{2.296294in}}%
\pgfpathlineto{\pgfqpoint{0.700897in}{2.298943in}}%
\pgfpathlineto{\pgfqpoint{0.701971in}{2.301230in}}%
\pgfpathlineto{\pgfqpoint{0.703045in}{2.301230in}}%
\pgfpathlineto{\pgfqpoint{0.704119in}{2.300577in}}%
\pgfpathlineto{\pgfqpoint{0.707340in}{2.303826in}}%
\pgfpathlineto{\pgfqpoint{0.708414in}{2.307594in}}%
\pgfpathlineto{\pgfqpoint{0.710562in}{2.298507in}}%
\pgfpathlineto{\pgfqpoint{0.711635in}{2.296470in}}%
\pgfpathlineto{\pgfqpoint{0.714857in}{2.295425in}}%
\pgfpathlineto{\pgfqpoint{0.717005in}{2.292979in}}%
\pgfpathlineto{\pgfqpoint{0.718078in}{2.295091in}}%
\pgfpathlineto{\pgfqpoint{0.719152in}{2.295952in}}%
\pgfpathlineto{\pgfqpoint{0.723448in}{2.296166in}}%
\pgfpathlineto{\pgfqpoint{0.724521in}{2.297634in}}%
\pgfpathlineto{\pgfqpoint{0.725595in}{2.297289in}}%
\pgfpathlineto{\pgfqpoint{0.726669in}{2.295092in}}%
\pgfpathlineto{\pgfqpoint{0.730965in}{2.293403in}}%
\pgfpathlineto{\pgfqpoint{0.733112in}{2.294320in}}%
\pgfpathlineto{\pgfqpoint{0.734186in}{2.293858in}}%
\pgfpathlineto{\pgfqpoint{0.738481in}{2.293984in}}%
\pgfpathlineto{\pgfqpoint{0.740629in}{2.293272in}}%
\pgfpathlineto{\pgfqpoint{0.741703in}{2.296328in}}%
\pgfpathlineto{\pgfqpoint{0.744924in}{2.296872in}}%
\pgfpathlineto{\pgfqpoint{0.745998in}{2.298764in}}%
\pgfpathlineto{\pgfqpoint{0.748146in}{2.296920in}}%
\pgfpathlineto{\pgfqpoint{0.749220in}{2.299983in}}%
\pgfpathlineto{\pgfqpoint{0.752441in}{2.298156in}}%
\pgfpathlineto{\pgfqpoint{0.753515in}{2.295522in}}%
\pgfpathlineto{\pgfqpoint{0.754589in}{2.296159in}}%
\pgfpathlineto{\pgfqpoint{0.756737in}{2.289329in}}%
\pgfpathlineto{\pgfqpoint{0.759958in}{2.289570in}}%
\pgfpathlineto{\pgfqpoint{0.761032in}{2.288807in}}%
\pgfpathlineto{\pgfqpoint{0.762106in}{2.289369in}}%
\pgfpathlineto{\pgfqpoint{0.764254in}{2.294731in}}%
\pgfpathlineto{\pgfqpoint{0.767475in}{2.295680in}}%
\pgfpathlineto{\pgfqpoint{0.768549in}{2.293808in}}%
\pgfpathlineto{\pgfqpoint{0.770697in}{2.294099in}}%
\pgfpathlineto{\pgfqpoint{0.771770in}{2.295181in}}%
\pgfpathlineto{\pgfqpoint{0.776066in}{2.296433in}}%
\pgfpathlineto{\pgfqpoint{0.777140in}{2.297438in}}%
\pgfpathlineto{\pgfqpoint{0.778213in}{2.296931in}}%
\pgfpathlineto{\pgfqpoint{0.779287in}{2.295539in}}%
\pgfpathlineto{\pgfqpoint{0.783583in}{2.295454in}}%
\pgfpathlineto{\pgfqpoint{0.785730in}{2.295496in}}%
\pgfpathlineto{\pgfqpoint{0.786804in}{2.297144in}}%
\pgfpathlineto{\pgfqpoint{0.791099in}{2.297566in}}%
\pgfpathlineto{\pgfqpoint{0.793247in}{2.296549in}}%
\pgfpathlineto{\pgfqpoint{0.794321in}{2.297312in}}%
\pgfpathlineto{\pgfqpoint{0.799690in}{2.297948in}}%
\pgfpathlineto{\pgfqpoint{0.800764in}{2.299949in}}%
\pgfpathlineto{\pgfqpoint{0.801838in}{2.299822in}}%
\pgfpathlineto{\pgfqpoint{0.807207in}{2.298463in}}%
\pgfpathlineto{\pgfqpoint{0.809355in}{2.303224in}}%
\pgfpathlineto{\pgfqpoint{0.812576in}{2.302454in}}%
\pgfpathlineto{\pgfqpoint{0.813650in}{2.303524in}}%
\pgfpathlineto{\pgfqpoint{0.814724in}{2.303353in}}%
\pgfpathlineto{\pgfqpoint{0.816872in}{2.301734in}}%
\pgfpathlineto{\pgfqpoint{0.820093in}{2.301945in}}%
\pgfpathlineto{\pgfqpoint{0.822241in}{2.304141in}}%
\pgfpathlineto{\pgfqpoint{0.827610in}{2.304609in}}%
\pgfpathlineto{\pgfqpoint{0.829758in}{2.305420in}}%
\pgfpathlineto{\pgfqpoint{0.831905in}{2.303117in}}%
\pgfpathlineto{\pgfqpoint{0.835127in}{2.303874in}}%
\pgfpathlineto{\pgfqpoint{0.836201in}{2.300279in}}%
\pgfpathlineto{\pgfqpoint{0.839422in}{2.298757in}}%
\pgfpathlineto{\pgfqpoint{0.842644in}{2.300956in}}%
\pgfpathlineto{\pgfqpoint{0.844791in}{2.292719in}}%
\pgfpathlineto{\pgfqpoint{0.845865in}{2.287747in}}%
\pgfpathlineto{\pgfqpoint{0.846939in}{2.289788in}}%
\pgfpathlineto{\pgfqpoint{0.850161in}{2.289519in}}%
\pgfpathlineto{\pgfqpoint{0.851234in}{2.286670in}}%
\pgfpathlineto{\pgfqpoint{0.852308in}{2.286208in}}%
\pgfpathlineto{\pgfqpoint{0.854456in}{2.288249in}}%
\pgfpathlineto{\pgfqpoint{0.859825in}{2.287977in}}%
\pgfpathlineto{\pgfqpoint{0.861973in}{2.292124in}}%
\pgfpathlineto{\pgfqpoint{0.865194in}{2.291770in}}%
\pgfpathlineto{\pgfqpoint{0.866268in}{2.292479in}}%
\pgfpathlineto{\pgfqpoint{0.867342in}{2.294647in}}%
\pgfpathlineto{\pgfqpoint{0.868416in}{2.292358in}}%
\pgfpathlineto{\pgfqpoint{0.869490in}{2.293081in}}%
\pgfpathlineto{\pgfqpoint{0.872711in}{2.293723in}}%
\pgfpathlineto{\pgfqpoint{0.875933in}{2.291704in}}%
\pgfpathlineto{\pgfqpoint{0.877007in}{2.292108in}}%
\pgfpathlineto{\pgfqpoint{0.880228in}{2.291946in}}%
\pgfpathlineto{\pgfqpoint{0.881302in}{2.290535in}}%
\pgfpathlineto{\pgfqpoint{0.882376in}{2.290774in}}%
\pgfpathlineto{\pgfqpoint{0.884523in}{2.292690in}}%
\pgfpathlineto{\pgfqpoint{0.887745in}{2.294247in}}%
\pgfpathlineto{\pgfqpoint{0.888819in}{2.293276in}}%
\pgfpathlineto{\pgfqpoint{0.889893in}{2.294935in}}%
\pgfpathlineto{\pgfqpoint{0.890966in}{2.295178in}}%
\pgfpathlineto{\pgfqpoint{0.892040in}{2.296881in}}%
\pgfpathlineto{\pgfqpoint{0.895262in}{2.297124in}}%
\pgfpathlineto{\pgfqpoint{0.896336in}{2.297815in}}%
\pgfpathlineto{\pgfqpoint{0.898483in}{2.297166in}}%
\pgfpathlineto{\pgfqpoint{0.899557in}{2.295832in}}%
\pgfpathlineto{\pgfqpoint{0.902779in}{2.295313in}}%
\pgfpathlineto{\pgfqpoint{0.903853in}{2.293642in}}%
\pgfpathlineto{\pgfqpoint{0.904926in}{2.294113in}}%
\pgfpathlineto{\pgfqpoint{0.907074in}{2.293207in}}%
\pgfpathlineto{\pgfqpoint{0.911369in}{2.293995in}}%
\pgfpathlineto{\pgfqpoint{0.912443in}{2.292932in}}%
\pgfpathlineto{\pgfqpoint{0.913517in}{2.293325in}}%
\pgfpathlineto{\pgfqpoint{0.914591in}{2.294900in}}%
\pgfpathlineto{\pgfqpoint{0.917812in}{2.294062in}}%
\pgfpathlineto{\pgfqpoint{0.921034in}{2.294860in}}%
\pgfpathlineto{\pgfqpoint{0.922108in}{2.292820in}}%
\pgfpathlineto{\pgfqpoint{0.925329in}{2.294900in}}%
\pgfpathlineto{\pgfqpoint{0.927477in}{2.292420in}}%
\pgfpathlineto{\pgfqpoint{0.928551in}{2.292734in}}%
\pgfpathlineto{\pgfqpoint{0.929625in}{2.295364in}}%
\pgfpathlineto{\pgfqpoint{0.933920in}{2.295875in}}%
\pgfpathlineto{\pgfqpoint{0.934994in}{2.296900in}}%
\pgfpathlineto{\pgfqpoint{0.936068in}{2.295362in}}%
\pgfpathlineto{\pgfqpoint{0.937142in}{2.294856in}}%
\pgfpathlineto{\pgfqpoint{0.940363in}{2.294159in}}%
\pgfpathlineto{\pgfqpoint{0.942511in}{2.295433in}}%
\pgfpathlineto{\pgfqpoint{0.944658in}{2.293314in}}%
\pgfpathlineto{\pgfqpoint{0.950028in}{2.295603in}}%
\pgfpathlineto{\pgfqpoint{0.951101in}{2.294792in}}%
\pgfpathlineto{\pgfqpoint{0.952175in}{2.292186in}}%
\pgfpathlineto{\pgfqpoint{0.955397in}{2.293123in}}%
\pgfpathlineto{\pgfqpoint{0.956471in}{2.295617in}}%
\pgfpathlineto{\pgfqpoint{0.957544in}{2.296638in}}%
\pgfpathlineto{\pgfqpoint{0.958618in}{2.296104in}}%
\pgfpathlineto{\pgfqpoint{0.959692in}{2.296943in}}%
\pgfpathlineto{\pgfqpoint{0.962914in}{2.297171in}}%
\pgfpathlineto{\pgfqpoint{0.963987in}{2.298928in}}%
\pgfpathlineto{\pgfqpoint{0.965061in}{2.296598in}}%
\pgfpathlineto{\pgfqpoint{0.967209in}{2.298970in}}%
\pgfpathlineto{\pgfqpoint{0.970431in}{2.299159in}}%
\pgfpathlineto{\pgfqpoint{0.972578in}{2.300824in}}%
\pgfpathlineto{\pgfqpoint{0.973652in}{2.301318in}}%
\pgfpathlineto{\pgfqpoint{0.974726in}{2.300216in}}%
\pgfpathlineto{\pgfqpoint{0.979021in}{2.297694in}}%
\pgfpathlineto{\pgfqpoint{0.981169in}{2.299358in}}%
\pgfpathlineto{\pgfqpoint{0.982243in}{2.299396in}}%
\pgfpathlineto{\pgfqpoint{0.985464in}{2.297240in}}%
\pgfpathlineto{\pgfqpoint{0.986538in}{2.297834in}}%
\pgfpathlineto{\pgfqpoint{0.987612in}{2.296050in}}%
\pgfpathlineto{\pgfqpoint{0.988686in}{2.296709in}}%
\pgfpathlineto{\pgfqpoint{0.989760in}{2.298550in}}%
\pgfpathlineto{\pgfqpoint{0.992981in}{2.297004in}}%
\pgfpathlineto{\pgfqpoint{0.994055in}{2.295623in}}%
\pgfpathlineto{\pgfqpoint{0.996203in}{2.297498in}}%
\pgfpathlineto{\pgfqpoint{0.997276in}{2.296160in}}%
\pgfpathlineto{\pgfqpoint{1.002646in}{2.295089in}}%
\pgfpathlineto{\pgfqpoint{1.003720in}{2.296754in}}%
\pgfpathlineto{\pgfqpoint{1.004793in}{2.296506in}}%
\pgfpathlineto{\pgfqpoint{1.009089in}{2.297781in}}%
\pgfpathlineto{\pgfqpoint{1.012310in}{2.303089in}}%
\pgfpathlineto{\pgfqpoint{1.015532in}{2.303270in}}%
\pgfpathlineto{\pgfqpoint{1.016606in}{2.306849in}}%
\pgfpathlineto{\pgfqpoint{1.018753in}{2.304796in}}%
\pgfpathlineto{\pgfqpoint{1.023049in}{2.303587in}}%
\pgfpathlineto{\pgfqpoint{1.024122in}{2.301334in}}%
\pgfpathlineto{\pgfqpoint{1.026270in}{2.304113in}}%
\pgfpathlineto{\pgfqpoint{1.027344in}{2.305201in}}%
\pgfpathlineto{\pgfqpoint{1.030565in}{2.302440in}}%
\pgfpathlineto{\pgfqpoint{1.031639in}{2.303644in}}%
\pgfpathlineto{\pgfqpoint{1.034861in}{2.300034in}}%
\pgfpathlineto{\pgfqpoint{1.038082in}{2.303022in}}%
\pgfpathlineto{\pgfqpoint{1.039156in}{2.308164in}}%
\pgfpathlineto{\pgfqpoint{1.040230in}{2.306791in}}%
\pgfpathlineto{\pgfqpoint{1.041304in}{2.308915in}}%
\pgfpathlineto{\pgfqpoint{1.042378in}{2.312814in}}%
\pgfpathlineto{\pgfqpoint{1.045599in}{2.311786in}}%
\pgfpathlineto{\pgfqpoint{1.046673in}{2.309958in}}%
\pgfpathlineto{\pgfqpoint{1.048821in}{2.315527in}}%
\pgfpathlineto{\pgfqpoint{1.053116in}{2.317235in}}%
\pgfpathlineto{\pgfqpoint{1.054190in}{2.318267in}}%
\pgfpathlineto{\pgfqpoint{1.055264in}{2.315217in}}%
\pgfpathlineto{\pgfqpoint{1.056338in}{2.318052in}}%
\pgfpathlineto{\pgfqpoint{1.057411in}{2.316293in}}%
\pgfpathlineto{\pgfqpoint{1.060633in}{2.319479in}}%
\pgfpathlineto{\pgfqpoint{1.061707in}{2.322093in}}%
\pgfpathlineto{\pgfqpoint{1.062781in}{2.322310in}}%
\pgfpathlineto{\pgfqpoint{1.063854in}{2.321365in}}%
\pgfpathlineto{\pgfqpoint{1.064928in}{2.323219in}}%
\pgfpathlineto{\pgfqpoint{1.068150in}{2.322928in}}%
\pgfpathlineto{\pgfqpoint{1.070297in}{2.317539in}}%
\pgfpathlineto{\pgfqpoint{1.071371in}{2.318096in}}%
\pgfpathlineto{\pgfqpoint{1.072445in}{2.319984in}}%
\pgfpathlineto{\pgfqpoint{1.075667in}{2.320194in}}%
\pgfpathlineto{\pgfqpoint{1.076741in}{2.321874in}}%
\pgfpathlineto{\pgfqpoint{1.077814in}{2.322225in}}%
\pgfpathlineto{\pgfqpoint{1.079962in}{2.319276in}}%
\pgfpathlineto{\pgfqpoint{1.084257in}{2.321628in}}%
\pgfpathlineto{\pgfqpoint{1.085331in}{2.327456in}}%
\pgfpathlineto{\pgfqpoint{1.086405in}{2.328520in}}%
\pgfpathlineto{\pgfqpoint{1.087479in}{2.334174in}}%
\pgfpathlineto{\pgfqpoint{1.090700in}{2.335897in}}%
\pgfpathlineto{\pgfqpoint{1.091774in}{2.337850in}}%
\pgfpathlineto{\pgfqpoint{1.092848in}{2.336452in}}%
\pgfpathlineto{\pgfqpoint{1.093922in}{2.339016in}}%
\pgfpathlineto{\pgfqpoint{1.094996in}{2.337540in}}%
\pgfpathlineto{\pgfqpoint{1.098217in}{2.336771in}}%
\pgfpathlineto{\pgfqpoint{1.099291in}{2.338223in}}%
\pgfpathlineto{\pgfqpoint{1.100365in}{2.335095in}}%
\pgfpathlineto{\pgfqpoint{1.101439in}{2.338919in}}%
\pgfpathlineto{\pgfqpoint{1.102513in}{2.338919in}}%
\pgfpathlineto{\pgfqpoint{1.105734in}{2.341275in}}%
\pgfpathlineto{\pgfqpoint{1.106808in}{2.338880in}}%
\pgfpathlineto{\pgfqpoint{1.107882in}{2.343702in}}%
\pgfpathlineto{\pgfqpoint{1.108956in}{2.336233in}}%
\pgfpathlineto{\pgfqpoint{1.110030in}{2.338159in}}%
\pgfpathlineto{\pgfqpoint{1.114325in}{2.331048in}}%
\pgfpathlineto{\pgfqpoint{1.115399in}{2.325859in}}%
\pgfpathlineto{\pgfqpoint{1.116473in}{2.326721in}}%
\pgfpathlineto{\pgfqpoint{1.117546in}{2.324117in}}%
\pgfpathlineto{\pgfqpoint{1.121842in}{2.326649in}}%
\pgfpathlineto{\pgfqpoint{1.122916in}{2.327300in}}%
\pgfpathlineto{\pgfqpoint{1.125063in}{2.323969in}}%
\pgfpathlineto{\pgfqpoint{1.130432in}{2.319876in}}%
\pgfpathlineto{\pgfqpoint{1.132580in}{2.317728in}}%
\pgfpathlineto{\pgfqpoint{1.139023in}{2.315916in}}%
\pgfpathlineto{\pgfqpoint{1.140097in}{2.321720in}}%
\pgfpathlineto{\pgfqpoint{1.144392in}{2.321987in}}%
\pgfpathlineto{\pgfqpoint{1.147614in}{2.322020in}}%
\pgfpathlineto{\pgfqpoint{1.151909in}{2.321057in}}%
\pgfpathlineto{\pgfqpoint{1.152983in}{2.321979in}}%
\pgfpathlineto{\pgfqpoint{1.154057in}{2.321221in}}%
\pgfpathlineto{\pgfqpoint{1.155131in}{2.319519in}}%
\pgfpathlineto{\pgfqpoint{1.159426in}{2.321167in}}%
\pgfpathlineto{\pgfqpoint{1.161574in}{2.319858in}}%
\pgfpathlineto{\pgfqpoint{1.162648in}{2.317142in}}%
\pgfpathlineto{\pgfqpoint{1.165869in}{2.316160in}}%
\pgfpathlineto{\pgfqpoint{1.166943in}{2.318942in}}%
\pgfpathlineto{\pgfqpoint{1.168017in}{2.325357in}}%
\pgfpathlineto{\pgfqpoint{1.169091in}{2.322020in}}%
\pgfpathlineto{\pgfqpoint{1.170164in}{2.321504in}}%
\pgfpathlineto{\pgfqpoint{1.173386in}{2.324634in}}%
\pgfpathlineto{\pgfqpoint{1.174460in}{2.326733in}}%
\pgfpathlineto{\pgfqpoint{1.176608in}{2.322399in}}%
\pgfpathlineto{\pgfqpoint{1.177681in}{2.324810in}}%
\pgfpathlineto{\pgfqpoint{1.180903in}{2.327432in}}%
\pgfpathlineto{\pgfqpoint{1.181977in}{2.323261in}}%
\pgfpathlineto{\pgfqpoint{1.183051in}{2.324366in}}%
\pgfpathlineto{\pgfqpoint{1.189494in}{2.323760in}}%
\pgfpathlineto{\pgfqpoint{1.190567in}{2.325220in}}%
\pgfpathlineto{\pgfqpoint{1.195937in}{2.323234in}}%
\pgfpathlineto{\pgfqpoint{1.198084in}{2.318264in}}%
\pgfpathlineto{\pgfqpoint{1.199158in}{2.318906in}}%
\pgfpathlineto{\pgfqpoint{1.200232in}{2.317616in}}%
\pgfpathlineto{\pgfqpoint{1.204527in}{2.318906in}}%
\pgfpathlineto{\pgfqpoint{1.205601in}{2.321382in}}%
\pgfpathlineto{\pgfqpoint{1.206675in}{2.320907in}}%
\pgfpathlineto{\pgfqpoint{1.207749in}{2.322056in}}%
\pgfpathlineto{\pgfqpoint{1.210970in}{2.320351in}}%
\pgfpathlineto{\pgfqpoint{1.213118in}{2.314487in}}%
\pgfpathlineto{\pgfqpoint{1.214192in}{2.314453in}}%
\pgfpathlineto{\pgfqpoint{1.215266in}{2.313464in}}%
\pgfpathlineto{\pgfqpoint{1.218487in}{2.313362in}}%
\pgfpathlineto{\pgfqpoint{1.219561in}{2.315612in}}%
\pgfpathlineto{\pgfqpoint{1.220635in}{2.316123in}}%
\pgfpathlineto{\pgfqpoint{1.221709in}{2.314035in}}%
\pgfpathlineto{\pgfqpoint{1.222783in}{2.316192in}}%
\pgfpathlineto{\pgfqpoint{1.226004in}{2.318314in}}%
\pgfpathlineto{\pgfqpoint{1.227078in}{2.315391in}}%
\pgfpathlineto{\pgfqpoint{1.228152in}{2.316435in}}%
\pgfpathlineto{\pgfqpoint{1.230299in}{2.306235in}}%
\pgfpathlineto{\pgfqpoint{1.234595in}{2.304921in}}%
\pgfpathlineto{\pgfqpoint{1.236742in}{2.299453in}}%
\pgfpathlineto{\pgfqpoint{1.237816in}{2.300334in}}%
\pgfpathlineto{\pgfqpoint{1.241038in}{2.299202in}}%
\pgfpathlineto{\pgfqpoint{1.242112in}{2.302231in}}%
\pgfpathlineto{\pgfqpoint{1.245333in}{2.304227in}}%
\pgfpathlineto{\pgfqpoint{1.248555in}{2.305031in}}%
\pgfpathlineto{\pgfqpoint{1.249629in}{2.303051in}}%
\pgfpathlineto{\pgfqpoint{1.250702in}{2.303965in}}%
\pgfpathlineto{\pgfqpoint{1.251776in}{2.303505in}}%
\pgfpathlineto{\pgfqpoint{1.252850in}{2.305498in}}%
\pgfpathlineto{\pgfqpoint{1.256072in}{2.306388in}}%
\pgfpathlineto{\pgfqpoint{1.257145in}{2.305801in}}%
\pgfpathlineto{\pgfqpoint{1.259293in}{2.307314in}}%
\pgfpathlineto{\pgfqpoint{1.260367in}{2.310956in}}%
\pgfpathlineto{\pgfqpoint{1.263588in}{2.310334in}}%
\pgfpathlineto{\pgfqpoint{1.264662in}{2.312254in}}%
\pgfpathlineto{\pgfqpoint{1.265736in}{2.311657in}}%
\pgfpathlineto{\pgfqpoint{1.266810in}{2.313229in}}%
\pgfpathlineto{\pgfqpoint{1.267884in}{2.311971in}}%
\pgfpathlineto{\pgfqpoint{1.271105in}{2.312220in}}%
\pgfpathlineto{\pgfqpoint{1.272179in}{2.313718in}}%
\pgfpathlineto{\pgfqpoint{1.274327in}{2.312814in}}%
\pgfpathlineto{\pgfqpoint{1.275401in}{2.314704in}}%
\pgfpathlineto{\pgfqpoint{1.278622in}{2.313681in}}%
\pgfpathlineto{\pgfqpoint{1.279696in}{2.315187in}}%
\pgfpathlineto{\pgfqpoint{1.282918in}{2.314099in}}%
\pgfpathlineto{\pgfqpoint{1.287213in}{2.312264in}}%
\pgfpathlineto{\pgfqpoint{1.288287in}{2.311362in}}%
\pgfpathlineto{\pgfqpoint{1.289361in}{2.309589in}}%
\pgfpathlineto{\pgfqpoint{1.290434in}{2.313508in}}%
\pgfpathlineto{\pgfqpoint{1.293656in}{2.313508in}}%
\pgfpathlineto{\pgfqpoint{1.294730in}{2.312730in}}%
\pgfpathlineto{\pgfqpoint{1.295804in}{2.309962in}}%
\pgfpathlineto{\pgfqpoint{1.296877in}{2.304737in}}%
\pgfpathlineto{\pgfqpoint{1.297951in}{2.305235in}}%
\pgfpathlineto{\pgfqpoint{1.301173in}{2.305173in}}%
\pgfpathlineto{\pgfqpoint{1.302247in}{2.303276in}}%
\pgfpathlineto{\pgfqpoint{1.303320in}{2.308561in}}%
\pgfpathlineto{\pgfqpoint{1.304394in}{2.310302in}}%
\pgfpathlineto{\pgfqpoint{1.305468in}{2.310586in}}%
\pgfpathlineto{\pgfqpoint{1.309763in}{2.310838in}}%
\pgfpathlineto{\pgfqpoint{1.312985in}{2.312637in}}%
\pgfpathlineto{\pgfqpoint{1.316207in}{2.312479in}}%
\pgfpathlineto{\pgfqpoint{1.317280in}{2.310545in}}%
\pgfpathlineto{\pgfqpoint{1.319428in}{2.309024in}}%
\pgfpathlineto{\pgfqpoint{1.320502in}{2.311243in}}%
\pgfpathlineto{\pgfqpoint{1.323723in}{2.309912in}}%
\pgfpathlineto{\pgfqpoint{1.324797in}{2.304644in}}%
\pgfpathlineto{\pgfqpoint{1.325871in}{2.304974in}}%
\pgfpathlineto{\pgfqpoint{1.326945in}{2.306451in}}%
\pgfpathlineto{\pgfqpoint{1.334462in}{2.305759in}}%
\pgfpathlineto{\pgfqpoint{1.335536in}{2.306842in}}%
\pgfpathlineto{\pgfqpoint{1.340905in}{2.310287in}}%
\pgfpathlineto{\pgfqpoint{1.341979in}{2.314468in}}%
\pgfpathlineto{\pgfqpoint{1.343052in}{2.308651in}}%
\pgfpathlineto{\pgfqpoint{1.346274in}{2.306807in}}%
\pgfpathlineto{\pgfqpoint{1.347348in}{2.307252in}}%
\pgfpathlineto{\pgfqpoint{1.349496in}{2.312191in}}%
\pgfpathlineto{\pgfqpoint{1.350569in}{2.315064in}}%
\pgfpathlineto{\pgfqpoint{1.353791in}{2.310188in}}%
\pgfpathlineto{\pgfqpoint{1.354865in}{2.309720in}}%
\pgfpathlineto{\pgfqpoint{1.355939in}{2.311624in}}%
\pgfpathlineto{\pgfqpoint{1.358086in}{2.303798in}}%
\pgfpathlineto{\pgfqpoint{1.361308in}{2.301037in}}%
\pgfpathlineto{\pgfqpoint{1.362382in}{2.296009in}}%
\pgfpathlineto{\pgfqpoint{1.363455in}{2.297391in}}%
\pgfpathlineto{\pgfqpoint{1.364529in}{2.297748in}}%
\pgfpathlineto{\pgfqpoint{1.365603in}{2.297242in}}%
\pgfpathlineto{\pgfqpoint{1.369898in}{2.298753in}}%
\pgfpathlineto{\pgfqpoint{1.370972in}{2.297373in}}%
\pgfpathlineto{\pgfqpoint{1.372046in}{2.299293in}}%
\pgfpathlineto{\pgfqpoint{1.373120in}{2.299803in}}%
\pgfpathlineto{\pgfqpoint{1.377415in}{2.298718in}}%
\pgfpathlineto{\pgfqpoint{1.378489in}{2.298718in}}%
\pgfpathlineto{\pgfqpoint{1.379563in}{2.299381in}}%
\pgfpathlineto{\pgfqpoint{1.380637in}{2.297393in}}%
\pgfpathlineto{\pgfqpoint{1.383858in}{2.298845in}}%
\pgfpathlineto{\pgfqpoint{1.384932in}{2.303465in}}%
\pgfpathlineto{\pgfqpoint{1.386006in}{2.305415in}}%
\pgfpathlineto{\pgfqpoint{1.387080in}{2.306208in}}%
\pgfpathlineto{\pgfqpoint{1.388154in}{2.305049in}}%
\pgfpathlineto{\pgfqpoint{1.393523in}{2.304928in}}%
\pgfpathlineto{\pgfqpoint{1.394597in}{2.306493in}}%
\pgfpathlineto{\pgfqpoint{1.395671in}{2.306005in}}%
\pgfpathlineto{\pgfqpoint{1.398892in}{2.308962in}}%
\pgfpathlineto{\pgfqpoint{1.399966in}{2.308962in}}%
\pgfpathlineto{\pgfqpoint{1.401040in}{2.308048in}}%
\pgfpathlineto{\pgfqpoint{1.402114in}{2.309450in}}%
\pgfpathlineto{\pgfqpoint{1.403187in}{2.304664in}}%
\pgfpathlineto{\pgfqpoint{1.406409in}{2.306512in}}%
\pgfpathlineto{\pgfqpoint{1.407483in}{2.312141in}}%
\pgfpathlineto{\pgfqpoint{1.409630in}{2.313604in}}%
\pgfpathlineto{\pgfqpoint{1.410704in}{2.313184in}}%
\pgfpathlineto{\pgfqpoint{1.413926in}{2.311152in}}%
\pgfpathlineto{\pgfqpoint{1.416073in}{2.312627in}}%
\pgfpathlineto{\pgfqpoint{1.418221in}{2.312952in}}%
\pgfpathlineto{\pgfqpoint{1.422517in}{2.314760in}}%
\pgfpathlineto{\pgfqpoint{1.423590in}{2.312459in}}%
\pgfpathlineto{\pgfqpoint{1.424664in}{2.318592in}}%
\pgfpathlineto{\pgfqpoint{1.425738in}{2.319455in}}%
\pgfpathlineto{\pgfqpoint{1.428960in}{2.318716in}}%
\pgfpathlineto{\pgfqpoint{1.430033in}{2.313262in}}%
\pgfpathlineto{\pgfqpoint{1.432181in}{2.314971in}}%
\pgfpathlineto{\pgfqpoint{1.437550in}{2.311868in}}%
\pgfpathlineto{\pgfqpoint{1.439698in}{2.311088in}}%
\pgfpathlineto{\pgfqpoint{1.440772in}{2.310654in}}%
\pgfpathlineto{\pgfqpoint{1.445067in}{2.315335in}}%
\pgfpathlineto{\pgfqpoint{1.446141in}{2.315975in}}%
\pgfpathlineto{\pgfqpoint{1.447215in}{2.317865in}}%
\pgfpathlineto{\pgfqpoint{1.448289in}{2.314737in}}%
\pgfpathlineto{\pgfqpoint{1.451510in}{2.316507in}}%
\pgfpathlineto{\pgfqpoint{1.452584in}{2.317776in}}%
\pgfpathlineto{\pgfqpoint{1.453658in}{2.321412in}}%
\pgfpathlineto{\pgfqpoint{1.455806in}{2.323566in}}%
\pgfpathlineto{\pgfqpoint{1.459027in}{2.324588in}}%
\pgfpathlineto{\pgfqpoint{1.460101in}{2.325890in}}%
\pgfpathlineto{\pgfqpoint{1.461175in}{2.326284in}}%
\pgfpathlineto{\pgfqpoint{1.463322in}{2.325524in}}%
\pgfpathlineto{\pgfqpoint{1.466544in}{2.325676in}}%
\pgfpathlineto{\pgfqpoint{1.467618in}{2.324644in}}%
\pgfpathlineto{\pgfqpoint{1.468692in}{2.326982in}}%
\pgfpathlineto{\pgfqpoint{1.469765in}{2.325585in}}%
\pgfpathlineto{\pgfqpoint{1.475135in}{2.325074in}}%
\pgfpathlineto{\pgfqpoint{1.476208in}{2.323722in}}%
\pgfpathlineto{\pgfqpoint{1.477282in}{2.324924in}}%
\pgfpathlineto{\pgfqpoint{1.478356in}{2.323062in}}%
\pgfpathlineto{\pgfqpoint{1.483725in}{2.319943in}}%
\pgfpathlineto{\pgfqpoint{1.484799in}{2.318584in}}%
\pgfpathlineto{\pgfqpoint{1.489095in}{2.318756in}}%
\pgfpathlineto{\pgfqpoint{1.491242in}{2.322908in}}%
\pgfpathlineto{\pgfqpoint{1.492316in}{2.320903in}}%
\pgfpathlineto{\pgfqpoint{1.496611in}{2.320699in}}%
\pgfpathlineto{\pgfqpoint{1.497685in}{2.319391in}}%
\pgfpathlineto{\pgfqpoint{1.498759in}{2.321600in}}%
\pgfpathlineto{\pgfqpoint{1.500907in}{2.315422in}}%
\pgfpathlineto{\pgfqpoint{1.504128in}{2.316669in}}%
\pgfpathlineto{\pgfqpoint{1.505202in}{2.316277in}}%
\pgfpathlineto{\pgfqpoint{1.506276in}{2.319132in}}%
\pgfpathlineto{\pgfqpoint{1.507350in}{2.319412in}}%
\pgfpathlineto{\pgfqpoint{1.508424in}{2.317813in}}%
\pgfpathlineto{\pgfqpoint{1.511645in}{2.316888in}}%
\pgfpathlineto{\pgfqpoint{1.512719in}{2.319945in}}%
\pgfpathlineto{\pgfqpoint{1.513793in}{2.319973in}}%
\pgfpathlineto{\pgfqpoint{1.514867in}{2.318907in}}%
\pgfpathlineto{\pgfqpoint{1.519162in}{2.321488in}}%
\pgfpathlineto{\pgfqpoint{1.521310in}{2.324137in}}%
\pgfpathlineto{\pgfqpoint{1.522384in}{2.324735in}}%
\pgfpathlineto{\pgfqpoint{1.523457in}{2.322991in}}%
\pgfpathlineto{\pgfqpoint{1.526679in}{2.323705in}}%
\pgfpathlineto{\pgfqpoint{1.527753in}{2.328966in}}%
\pgfpathlineto{\pgfqpoint{1.528827in}{2.326228in}}%
\pgfpathlineto{\pgfqpoint{1.529900in}{2.321408in}}%
\pgfpathlineto{\pgfqpoint{1.530974in}{2.325127in}}%
\pgfpathlineto{\pgfqpoint{1.534196in}{2.326496in}}%
\pgfpathlineto{\pgfqpoint{1.535270in}{2.329653in}}%
\pgfpathlineto{\pgfqpoint{1.536343in}{2.330404in}}%
\pgfpathlineto{\pgfqpoint{1.541713in}{2.330193in}}%
\pgfpathlineto{\pgfqpoint{1.542786in}{2.329800in}}%
\pgfpathlineto{\pgfqpoint{1.543860in}{2.330707in}}%
\pgfpathlineto{\pgfqpoint{1.544934in}{2.336542in}}%
\pgfpathlineto{\pgfqpoint{1.546008in}{2.336005in}}%
\pgfpathlineto{\pgfqpoint{1.549229in}{2.336731in}}%
\pgfpathlineto{\pgfqpoint{1.550303in}{2.337678in}}%
\pgfpathlineto{\pgfqpoint{1.553525in}{2.345548in}}%
\pgfpathlineto{\pgfqpoint{1.556746in}{2.345322in}}%
\pgfpathlineto{\pgfqpoint{1.557820in}{2.343839in}}%
\pgfpathlineto{\pgfqpoint{1.558894in}{2.346966in}}%
\pgfpathlineto{\pgfqpoint{1.559968in}{2.346192in}}%
\pgfpathlineto{\pgfqpoint{1.561042in}{2.348564in}}%
\pgfpathlineto{\pgfqpoint{1.564263in}{2.352932in}}%
\pgfpathlineto{\pgfqpoint{1.566411in}{2.351501in}}%
\pgfpathlineto{\pgfqpoint{1.567485in}{2.348726in}}%
\pgfpathlineto{\pgfqpoint{1.568559in}{2.349834in}}%
\pgfpathlineto{\pgfqpoint{1.571780in}{2.350217in}}%
\pgfpathlineto{\pgfqpoint{1.572854in}{2.349642in}}%
\pgfpathlineto{\pgfqpoint{1.573928in}{2.350246in}}%
\pgfpathlineto{\pgfqpoint{1.575002in}{2.349480in}}%
\pgfpathlineto{\pgfqpoint{1.576075in}{2.351683in}}%
\pgfpathlineto{\pgfqpoint{1.580371in}{2.352736in}}%
\pgfpathlineto{\pgfqpoint{1.581445in}{2.353861in}}%
\pgfpathlineto{\pgfqpoint{1.582518in}{2.353604in}}%
\pgfpathlineto{\pgfqpoint{1.583592in}{2.351999in}}%
\pgfpathlineto{\pgfqpoint{1.586814in}{2.353015in}}%
\pgfpathlineto{\pgfqpoint{1.587888in}{2.352312in}}%
\pgfpathlineto{\pgfqpoint{1.588961in}{2.355668in}}%
\pgfpathlineto{\pgfqpoint{1.590035in}{2.356915in}}%
\pgfpathlineto{\pgfqpoint{1.594331in}{2.357399in}}%
\pgfpathlineto{\pgfqpoint{1.595405in}{2.354110in}}%
\pgfpathlineto{\pgfqpoint{1.596478in}{2.353259in}}%
\pgfpathlineto{\pgfqpoint{1.598626in}{2.348344in}}%
\pgfpathlineto{\pgfqpoint{1.601848in}{2.348647in}}%
\pgfpathlineto{\pgfqpoint{1.602921in}{2.347554in}}%
\pgfpathlineto{\pgfqpoint{1.605069in}{2.354782in}}%
\pgfpathlineto{\pgfqpoint{1.606143in}{2.354782in}}%
\pgfpathlineto{\pgfqpoint{1.609364in}{2.353251in}}%
\pgfpathlineto{\pgfqpoint{1.610438in}{2.353376in}}%
\pgfpathlineto{\pgfqpoint{1.611512in}{2.347096in}}%
\pgfpathlineto{\pgfqpoint{1.612586in}{2.345847in}}%
\pgfpathlineto{\pgfqpoint{1.613660in}{2.349377in}}%
\pgfpathlineto{\pgfqpoint{1.616881in}{2.350127in}}%
\pgfpathlineto{\pgfqpoint{1.617955in}{2.343312in}}%
\pgfpathlineto{\pgfqpoint{1.620103in}{2.358104in}}%
\pgfpathlineto{\pgfqpoint{1.624398in}{2.349485in}}%
\pgfpathlineto{\pgfqpoint{1.625472in}{2.343319in}}%
\pgfpathlineto{\pgfqpoint{1.627620in}{2.351010in}}%
\pgfpathlineto{\pgfqpoint{1.628694in}{2.356779in}}%
\pgfpathlineto{\pgfqpoint{1.631915in}{2.355277in}}%
\pgfpathlineto{\pgfqpoint{1.635137in}{2.345331in}}%
\pgfpathlineto{\pgfqpoint{1.636210in}{2.343942in}}%
\pgfpathlineto{\pgfqpoint{1.639432in}{2.341349in}}%
\pgfpathlineto{\pgfqpoint{1.641580in}{2.337358in}}%
\pgfpathlineto{\pgfqpoint{1.643727in}{2.331596in}}%
\pgfpathlineto{\pgfqpoint{1.646949in}{2.332397in}}%
\pgfpathlineto{\pgfqpoint{1.648023in}{2.335303in}}%
\pgfpathlineto{\pgfqpoint{1.650170in}{2.334298in}}%
\pgfpathlineto{\pgfqpoint{1.651244in}{2.336325in}}%
\pgfpathlineto{\pgfqpoint{1.654466in}{2.337890in}}%
\pgfpathlineto{\pgfqpoint{1.656613in}{2.337253in}}%
\pgfpathlineto{\pgfqpoint{1.657687in}{2.338063in}}%
\pgfpathlineto{\pgfqpoint{1.658761in}{2.340378in}}%
\pgfpathlineto{\pgfqpoint{1.661983in}{2.340731in}}%
\pgfpathlineto{\pgfqpoint{1.663056in}{2.339348in}}%
\pgfpathlineto{\pgfqpoint{1.664130in}{2.339523in}}%
\pgfpathlineto{\pgfqpoint{1.665204in}{2.338036in}}%
\pgfpathlineto{\pgfqpoint{1.669499in}{2.341507in}}%
\pgfpathlineto{\pgfqpoint{1.670573in}{2.341032in}}%
\pgfpathlineto{\pgfqpoint{1.671647in}{2.342368in}}%
\pgfpathlineto{\pgfqpoint{1.673795in}{2.339636in}}%
\pgfpathlineto{\pgfqpoint{1.677016in}{2.340218in}}%
\pgfpathlineto{\pgfqpoint{1.679164in}{2.343524in}}%
\pgfpathlineto{\pgfqpoint{1.680238in}{2.343108in}}%
\pgfpathlineto{\pgfqpoint{1.681312in}{2.345601in}}%
\pgfpathlineto{\pgfqpoint{1.684533in}{2.345572in}}%
\pgfpathlineto{\pgfqpoint{1.685607in}{2.344325in}}%
\pgfpathlineto{\pgfqpoint{1.686681in}{2.339577in}}%
\pgfpathlineto{\pgfqpoint{1.687755in}{2.340853in}}%
\pgfpathlineto{\pgfqpoint{1.688828in}{2.346877in}}%
\pgfpathlineto{\pgfqpoint{1.692050in}{2.345607in}}%
\pgfpathlineto{\pgfqpoint{1.693124in}{2.342292in}}%
\pgfpathlineto{\pgfqpoint{1.694198in}{2.345886in}}%
\pgfpathlineto{\pgfqpoint{1.695272in}{2.338358in}}%
\pgfpathlineto{\pgfqpoint{1.696345in}{2.341649in}}%
\pgfpathlineto{\pgfqpoint{1.699567in}{2.344809in}}%
\pgfpathlineto{\pgfqpoint{1.700641in}{2.351339in}}%
\pgfpathlineto{\pgfqpoint{1.701715in}{2.352189in}}%
\pgfpathlineto{\pgfqpoint{1.703862in}{2.350866in}}%
\pgfpathlineto{\pgfqpoint{1.708158in}{2.350054in}}%
\pgfpathlineto{\pgfqpoint{1.709231in}{2.347881in}}%
\pgfpathlineto{\pgfqpoint{1.711379in}{2.347757in}}%
\pgfpathlineto{\pgfqpoint{1.714601in}{2.345770in}}%
\pgfpathlineto{\pgfqpoint{1.715674in}{2.344342in}}%
\pgfpathlineto{\pgfqpoint{1.716748in}{2.350613in}}%
\pgfpathlineto{\pgfqpoint{1.717822in}{2.348347in}}%
\pgfpathlineto{\pgfqpoint{1.718896in}{2.352259in}}%
\pgfpathlineto{\pgfqpoint{1.722117in}{2.351254in}}%
\pgfpathlineto{\pgfqpoint{1.723191in}{2.351757in}}%
\pgfpathlineto{\pgfqpoint{1.724265in}{2.353861in}}%
\pgfpathlineto{\pgfqpoint{1.725339in}{2.349623in}}%
\pgfpathlineto{\pgfqpoint{1.726413in}{2.353989in}}%
\pgfpathlineto{\pgfqpoint{1.731782in}{2.363579in}}%
\pgfpathlineto{\pgfqpoint{1.732856in}{2.358034in}}%
\pgfpathlineto{\pgfqpoint{1.733930in}{2.362505in}}%
\pgfpathlineto{\pgfqpoint{1.737151in}{2.362672in}}%
\pgfpathlineto{\pgfqpoint{1.738225in}{2.362171in}}%
\pgfpathlineto{\pgfqpoint{1.739299in}{2.360368in}}%
\pgfpathlineto{\pgfqpoint{1.740373in}{2.363039in}}%
\pgfpathlineto{\pgfqpoint{1.741447in}{2.369649in}}%
\pgfpathlineto{\pgfqpoint{1.744668in}{2.371777in}}%
\pgfpathlineto{\pgfqpoint{1.745742in}{2.366754in}}%
\pgfpathlineto{\pgfqpoint{1.748963in}{2.377516in}}%
\pgfpathlineto{\pgfqpoint{1.752185in}{2.373371in}}%
\pgfpathlineto{\pgfqpoint{1.753259in}{2.375143in}}%
\pgfpathlineto{\pgfqpoint{1.754333in}{2.375036in}}%
\pgfpathlineto{\pgfqpoint{1.755406in}{2.368947in}}%
\pgfpathlineto{\pgfqpoint{1.756480in}{2.372665in}}%
\pgfpathlineto{\pgfqpoint{1.760776in}{2.370116in}}%
\pgfpathlineto{\pgfqpoint{1.762923in}{2.373864in}}%
\pgfpathlineto{\pgfqpoint{1.763997in}{2.373055in}}%
\pgfpathlineto{\pgfqpoint{1.767219in}{2.373300in}}%
\pgfpathlineto{\pgfqpoint{1.768293in}{2.374946in}}%
\pgfpathlineto{\pgfqpoint{1.769366in}{2.373370in}}%
\pgfpathlineto{\pgfqpoint{1.770440in}{2.368445in}}%
\pgfpathlineto{\pgfqpoint{1.774736in}{2.371448in}}%
\pgfpathlineto{\pgfqpoint{1.775809in}{2.374084in}}%
\pgfpathlineto{\pgfqpoint{1.776883in}{2.371953in}}%
\pgfpathlineto{\pgfqpoint{1.777957in}{2.374634in}}%
\pgfpathlineto{\pgfqpoint{1.779031in}{2.382025in}}%
\pgfpathlineto{\pgfqpoint{1.782252in}{2.383791in}}%
\pgfpathlineto{\pgfqpoint{1.783326in}{2.387430in}}%
\pgfpathlineto{\pgfqpoint{1.784400in}{2.383487in}}%
\pgfpathlineto{\pgfqpoint{1.785474in}{2.388426in}}%
\pgfpathlineto{\pgfqpoint{1.789769in}{2.396610in}}%
\pgfpathlineto{\pgfqpoint{1.791917in}{2.405875in}}%
\pgfpathlineto{\pgfqpoint{1.792991in}{2.405073in}}%
\pgfpathlineto{\pgfqpoint{1.794065in}{2.402678in}}%
\pgfpathlineto{\pgfqpoint{1.797286in}{2.400728in}}%
\pgfpathlineto{\pgfqpoint{1.798360in}{2.404101in}}%
\pgfpathlineto{\pgfqpoint{1.799434in}{2.398656in}}%
\pgfpathlineto{\pgfqpoint{1.800508in}{2.397976in}}%
\pgfpathlineto{\pgfqpoint{1.801582in}{2.398656in}}%
\pgfpathlineto{\pgfqpoint{1.804803in}{2.394611in}}%
\pgfpathlineto{\pgfqpoint{1.805877in}{2.397712in}}%
\pgfpathlineto{\pgfqpoint{1.806951in}{2.392860in}}%
\pgfpathlineto{\pgfqpoint{1.808025in}{2.394477in}}%
\pgfpathlineto{\pgfqpoint{1.812320in}{2.396094in}}%
\pgfpathlineto{\pgfqpoint{1.813394in}{2.399284in}}%
\pgfpathlineto{\pgfqpoint{1.814468in}{2.399094in}}%
\pgfpathlineto{\pgfqpoint{1.815541in}{2.403077in}}%
\pgfpathlineto{\pgfqpoint{1.816615in}{2.400612in}}%
\pgfpathlineto{\pgfqpoint{1.819837in}{2.405659in}}%
\pgfpathlineto{\pgfqpoint{1.821984in}{2.405852in}}%
\pgfpathlineto{\pgfqpoint{1.823058in}{2.408625in}}%
\pgfpathlineto{\pgfqpoint{1.824132in}{2.407881in}}%
\pgfpathlineto{\pgfqpoint{1.827354in}{2.410073in}}%
\pgfpathlineto{\pgfqpoint{1.830575in}{2.409213in}}%
\pgfpathlineto{\pgfqpoint{1.831649in}{2.411356in}}%
\pgfpathlineto{\pgfqpoint{1.837018in}{2.414839in}}%
\pgfpathlineto{\pgfqpoint{1.838092in}{2.410641in}}%
\pgfpathlineto{\pgfqpoint{1.839166in}{2.413928in}}%
\pgfpathlineto{\pgfqpoint{1.842387in}{2.413176in}}%
\pgfpathlineto{\pgfqpoint{1.843461in}{2.416134in}}%
\pgfpathlineto{\pgfqpoint{1.844535in}{2.414931in}}%
\pgfpathlineto{\pgfqpoint{1.845609in}{2.416695in}}%
\pgfpathlineto{\pgfqpoint{1.846683in}{2.410238in}}%
\pgfpathlineto{\pgfqpoint{1.849904in}{2.411747in}}%
\pgfpathlineto{\pgfqpoint{1.850978in}{2.409796in}}%
\pgfpathlineto{\pgfqpoint{1.852052in}{2.410069in}}%
\pgfpathlineto{\pgfqpoint{1.853126in}{2.405622in}}%
\pgfpathlineto{\pgfqpoint{1.854200in}{2.404025in}}%
\pgfpathlineto{\pgfqpoint{1.858495in}{2.398488in}}%
\pgfpathlineto{\pgfqpoint{1.859569in}{2.399655in}}%
\pgfpathlineto{\pgfqpoint{1.860643in}{2.398223in}}%
\pgfpathlineto{\pgfqpoint{1.861716in}{2.394697in}}%
\pgfpathlineto{\pgfqpoint{1.866012in}{2.392970in}}%
\pgfpathlineto{\pgfqpoint{1.867086in}{2.393999in}}%
\pgfpathlineto{\pgfqpoint{1.868160in}{2.393962in}}%
\pgfpathlineto{\pgfqpoint{1.869233in}{2.390657in}}%
\pgfpathlineto{\pgfqpoint{1.874603in}{2.390473in}}%
\pgfpathlineto{\pgfqpoint{1.875676in}{2.393301in}}%
\pgfpathlineto{\pgfqpoint{1.876750in}{2.391172in}}%
\pgfpathlineto{\pgfqpoint{1.879972in}{2.389042in}}%
\pgfpathlineto{\pgfqpoint{1.881046in}{2.389901in}}%
\pgfpathlineto{\pgfqpoint{1.883193in}{2.386481in}}%
\pgfpathlineto{\pgfqpoint{1.884267in}{2.389408in}}%
\pgfpathlineto{\pgfqpoint{1.887489in}{2.386539in}}%
\pgfpathlineto{\pgfqpoint{1.888562in}{2.389445in}}%
\pgfpathlineto{\pgfqpoint{1.889636in}{2.388216in}}%
\pgfpathlineto{\pgfqpoint{1.890710in}{2.382817in}}%
\pgfpathlineto{\pgfqpoint{1.891784in}{2.384391in}}%
\pgfpathlineto{\pgfqpoint{1.895005in}{2.385114in}}%
\pgfpathlineto{\pgfqpoint{1.896079in}{2.386126in}}%
\pgfpathlineto{\pgfqpoint{1.898227in}{2.383981in}}%
\pgfpathlineto{\pgfqpoint{1.899301in}{2.385689in}}%
\pgfpathlineto{\pgfqpoint{1.902522in}{2.392087in}}%
\pgfpathlineto{\pgfqpoint{1.903596in}{2.391331in}}%
\pgfpathlineto{\pgfqpoint{1.904670in}{2.394772in}}%
\pgfpathlineto{\pgfqpoint{1.905744in}{2.394885in}}%
\pgfpathlineto{\pgfqpoint{1.910039in}{2.394091in}}%
\pgfpathlineto{\pgfqpoint{1.911113in}{2.396512in}}%
\pgfpathlineto{\pgfqpoint{1.913261in}{2.402617in}}%
\pgfpathlineto{\pgfqpoint{1.914335in}{2.398706in}}%
\pgfpathlineto{\pgfqpoint{1.917556in}{2.396208in}}%
\pgfpathlineto{\pgfqpoint{1.919704in}{2.400033in}}%
\pgfpathlineto{\pgfqpoint{1.920778in}{2.397698in}}%
\pgfpathlineto{\pgfqpoint{1.921851in}{2.401152in}}%
\pgfpathlineto{\pgfqpoint{1.926147in}{2.402214in}}%
\pgfpathlineto{\pgfqpoint{1.927221in}{2.401645in}}%
\pgfpathlineto{\pgfqpoint{1.928294in}{2.401721in}}%
\pgfpathlineto{\pgfqpoint{1.929368in}{2.398155in}}%
\pgfpathlineto{\pgfqpoint{1.932590in}{2.395197in}}%
\pgfpathlineto{\pgfqpoint{1.933664in}{2.397700in}}%
\pgfpathlineto{\pgfqpoint{1.934738in}{2.395272in}}%
\pgfpathlineto{\pgfqpoint{1.935811in}{2.394898in}}%
\pgfpathlineto{\pgfqpoint{1.936885in}{2.393667in}}%
\pgfpathlineto{\pgfqpoint{1.940107in}{2.394297in}}%
\pgfpathlineto{\pgfqpoint{1.941181in}{2.393553in}}%
\pgfpathlineto{\pgfqpoint{1.942254in}{2.395970in}}%
\pgfpathlineto{\pgfqpoint{1.943328in}{2.400802in}}%
\pgfpathlineto{\pgfqpoint{1.944402in}{2.400037in}}%
\pgfpathlineto{\pgfqpoint{1.947624in}{2.403251in}}%
\pgfpathlineto{\pgfqpoint{1.948697in}{2.405929in}}%
\pgfpathlineto{\pgfqpoint{1.950845in}{2.404103in}}%
\pgfpathlineto{\pgfqpoint{1.951919in}{2.405230in}}%
\pgfpathlineto{\pgfqpoint{1.955140in}{2.401499in}}%
\pgfpathlineto{\pgfqpoint{1.956214in}{2.403210in}}%
\pgfpathlineto{\pgfqpoint{1.957288in}{2.402980in}}%
\pgfpathlineto{\pgfqpoint{1.958362in}{2.401213in}}%
\pgfpathlineto{\pgfqpoint{1.959436in}{2.392569in}}%
\pgfpathlineto{\pgfqpoint{1.962657in}{2.383081in}}%
\pgfpathlineto{\pgfqpoint{1.963731in}{2.375820in}}%
\pgfpathlineto{\pgfqpoint{1.964805in}{2.391033in}}%
\pgfpathlineto{\pgfqpoint{1.965879in}{2.387191in}}%
\pgfpathlineto{\pgfqpoint{1.966953in}{2.390757in}}%
\pgfpathlineto{\pgfqpoint{1.970174in}{2.386652in}}%
\pgfpathlineto{\pgfqpoint{1.971248in}{2.380092in}}%
\pgfpathlineto{\pgfqpoint{1.973396in}{2.386920in}}%
\pgfpathlineto{\pgfqpoint{1.974470in}{2.382208in}}%
\pgfpathlineto{\pgfqpoint{1.978765in}{2.391477in}}%
\pgfpathlineto{\pgfqpoint{1.979839in}{2.397514in}}%
\pgfpathlineto{\pgfqpoint{1.980913in}{2.399332in}}%
\pgfpathlineto{\pgfqpoint{1.981986in}{2.398564in}}%
\pgfpathlineto{\pgfqpoint{1.985208in}{2.397157in}}%
\pgfpathlineto{\pgfqpoint{1.986282in}{2.393249in}}%
\pgfpathlineto{\pgfqpoint{1.988429in}{2.391733in}}%
\pgfpathlineto{\pgfqpoint{1.989503in}{2.396790in}}%
\pgfpathlineto{\pgfqpoint{1.992725in}{2.395875in}}%
\pgfpathlineto{\pgfqpoint{1.994872in}{2.397188in}}%
\pgfpathlineto{\pgfqpoint{1.995946in}{2.395347in}}%
\pgfpathlineto{\pgfqpoint{1.997020in}{2.389984in}}%
\pgfpathlineto{\pgfqpoint{2.000242in}{2.391345in}}%
\pgfpathlineto{\pgfqpoint{2.001315in}{2.385301in}}%
\pgfpathlineto{\pgfqpoint{2.002389in}{2.384221in}}%
\pgfpathlineto{\pgfqpoint{2.003463in}{2.384872in}}%
\pgfpathlineto{\pgfqpoint{2.004537in}{2.387528in}}%
\pgfpathlineto{\pgfqpoint{2.007759in}{2.384641in}}%
\pgfpathlineto{\pgfqpoint{2.009906in}{2.394701in}}%
\pgfpathlineto{\pgfqpoint{2.010980in}{2.394351in}}%
\pgfpathlineto{\pgfqpoint{2.012054in}{2.393342in}}%
\pgfpathlineto{\pgfqpoint{2.015275in}{2.391181in}}%
\pgfpathlineto{\pgfqpoint{2.016349in}{2.393009in}}%
\pgfpathlineto{\pgfqpoint{2.017423in}{2.389775in}}%
\pgfpathlineto{\pgfqpoint{2.018497in}{2.398901in}}%
\pgfpathlineto{\pgfqpoint{2.019571in}{2.395088in}}%
\pgfpathlineto{\pgfqpoint{2.022792in}{2.396105in}}%
\pgfpathlineto{\pgfqpoint{2.023866in}{2.394893in}}%
\pgfpathlineto{\pgfqpoint{2.024940in}{2.395044in}}%
\pgfpathlineto{\pgfqpoint{2.026014in}{2.401521in}}%
\pgfpathlineto{\pgfqpoint{2.027088in}{2.398832in}}%
\pgfpathlineto{\pgfqpoint{2.031383in}{2.400964in}}%
\pgfpathlineto{\pgfqpoint{2.032457in}{2.399987in}}%
\pgfpathlineto{\pgfqpoint{2.033531in}{2.396962in}}%
\pgfpathlineto{\pgfqpoint{2.038900in}{2.402606in}}%
\pgfpathlineto{\pgfqpoint{2.039974in}{2.402531in}}%
\pgfpathlineto{\pgfqpoint{2.042121in}{2.405202in}}%
\pgfpathlineto{\pgfqpoint{2.045343in}{2.401575in}}%
\pgfpathlineto{\pgfqpoint{2.046417in}{2.403632in}}%
\pgfpathlineto{\pgfqpoint{2.047491in}{2.402248in}}%
\pgfpathlineto{\pgfqpoint{2.048564in}{2.407589in}}%
\pgfpathlineto{\pgfqpoint{2.049638in}{2.406250in}}%
\pgfpathlineto{\pgfqpoint{2.052860in}{2.410880in}}%
\pgfpathlineto{\pgfqpoint{2.053934in}{2.409923in}}%
\pgfpathlineto{\pgfqpoint{2.055007in}{2.405889in}}%
\pgfpathlineto{\pgfqpoint{2.056081in}{2.406484in}}%
\pgfpathlineto{\pgfqpoint{2.057155in}{2.408984in}}%
\pgfpathlineto{\pgfqpoint{2.060377in}{2.407977in}}%
\pgfpathlineto{\pgfqpoint{2.061450in}{2.410463in}}%
\pgfpathlineto{\pgfqpoint{2.062524in}{2.410199in}}%
\pgfpathlineto{\pgfqpoint{2.064672in}{2.411592in}}%
\pgfpathlineto{\pgfqpoint{2.067893in}{2.415468in}}%
\pgfpathlineto{\pgfqpoint{2.068967in}{2.419393in}}%
\pgfpathlineto{\pgfqpoint{2.070041in}{2.420470in}}%
\pgfpathlineto{\pgfqpoint{2.071115in}{2.416019in}}%
\pgfpathlineto{\pgfqpoint{2.072189in}{2.423645in}}%
\pgfpathlineto{\pgfqpoint{2.075410in}{2.422329in}}%
\pgfpathlineto{\pgfqpoint{2.076484in}{2.425480in}}%
\pgfpathlineto{\pgfqpoint{2.077558in}{2.424502in}}%
\pgfpathlineto{\pgfqpoint{2.078632in}{2.426262in}}%
\pgfpathlineto{\pgfqpoint{2.079706in}{2.429703in}}%
\pgfpathlineto{\pgfqpoint{2.082927in}{2.431417in}}%
\pgfpathlineto{\pgfqpoint{2.085075in}{2.420334in}}%
\pgfpathlineto{\pgfqpoint{2.086149in}{2.425812in}}%
\pgfpathlineto{\pgfqpoint{2.087223in}{2.419855in}}%
\pgfpathlineto{\pgfqpoint{2.090444in}{2.418539in}}%
\pgfpathlineto{\pgfqpoint{2.091518in}{2.422524in}}%
\pgfpathlineto{\pgfqpoint{2.092592in}{2.419777in}}%
\pgfpathlineto{\pgfqpoint{2.093666in}{2.418977in}}%
\pgfpathlineto{\pgfqpoint{2.097961in}{2.420684in}}%
\pgfpathlineto{\pgfqpoint{2.099035in}{2.423479in}}%
\pgfpathlineto{\pgfqpoint{2.100109in}{2.424322in}}%
\pgfpathlineto{\pgfqpoint{2.101182in}{2.420590in}}%
\pgfpathlineto{\pgfqpoint{2.105478in}{2.412742in}}%
\pgfpathlineto{\pgfqpoint{2.107626in}{2.415974in}}%
\pgfpathlineto{\pgfqpoint{2.109773in}{2.408086in}}%
\pgfpathlineto{\pgfqpoint{2.112995in}{2.405987in}}%
\pgfpathlineto{\pgfqpoint{2.114069in}{2.408358in}}%
\pgfpathlineto{\pgfqpoint{2.115142in}{2.412671in}}%
\pgfpathlineto{\pgfqpoint{2.117290in}{2.426294in}}%
\pgfpathlineto{\pgfqpoint{2.121585in}{2.428198in}}%
\pgfpathlineto{\pgfqpoint{2.122659in}{2.434819in}}%
\pgfpathlineto{\pgfqpoint{2.123733in}{2.435592in}}%
\pgfpathlineto{\pgfqpoint{2.124807in}{2.432417in}}%
\pgfpathlineto{\pgfqpoint{2.128028in}{2.433767in}}%
\pgfpathlineto{\pgfqpoint{2.129102in}{2.452250in}}%
\pgfpathlineto{\pgfqpoint{2.130176in}{2.448469in}}%
\pgfpathlineto{\pgfqpoint{2.131250in}{2.448052in}}%
\pgfpathlineto{\pgfqpoint{2.132324in}{2.439817in}}%
\pgfpathlineto{\pgfqpoint{2.135545in}{2.440096in}}%
\pgfpathlineto{\pgfqpoint{2.136619in}{2.436624in}}%
\pgfpathlineto{\pgfqpoint{2.137693in}{2.439258in}}%
\pgfpathlineto{\pgfqpoint{2.138767in}{2.440096in}}%
\pgfpathlineto{\pgfqpoint{2.139841in}{2.427828in}}%
\pgfpathlineto{\pgfqpoint{2.143062in}{2.433160in}}%
\pgfpathlineto{\pgfqpoint{2.144136in}{2.433281in}}%
\pgfpathlineto{\pgfqpoint{2.145210in}{2.432398in}}%
\pgfpathlineto{\pgfqpoint{2.146284in}{2.432278in}}%
\pgfpathlineto{\pgfqpoint{2.147358in}{2.432719in}}%
\pgfpathlineto{\pgfqpoint{2.152727in}{2.430238in}}%
\pgfpathlineto{\pgfqpoint{2.153801in}{2.423944in}}%
\pgfpathlineto{\pgfqpoint{2.154874in}{2.421610in}}%
\pgfpathlineto{\pgfqpoint{2.158096in}{2.419570in}}%
\pgfpathlineto{\pgfqpoint{2.160244in}{2.424874in}}%
\pgfpathlineto{\pgfqpoint{2.161317in}{2.419919in}}%
\pgfpathlineto{\pgfqpoint{2.162391in}{2.421944in}}%
\pgfpathlineto{\pgfqpoint{2.165613in}{2.419971in}}%
\pgfpathlineto{\pgfqpoint{2.166687in}{2.426857in}}%
\pgfpathlineto{\pgfqpoint{2.167760in}{2.427490in}}%
\pgfpathlineto{\pgfqpoint{2.169908in}{2.425660in}}%
\pgfpathlineto{\pgfqpoint{2.173130in}{2.426482in}}%
\pgfpathlineto{\pgfqpoint{2.175277in}{2.430425in}}%
\pgfpathlineto{\pgfqpoint{2.176351in}{2.429406in}}%
\pgfpathlineto{\pgfqpoint{2.177425in}{2.427418in}}%
\pgfpathlineto{\pgfqpoint{2.181720in}{2.427938in}}%
\pgfpathlineto{\pgfqpoint{2.182794in}{2.429127in}}%
\pgfpathlineto{\pgfqpoint{2.183868in}{2.426846in}}%
\pgfpathlineto{\pgfqpoint{2.184942in}{2.429426in}}%
\pgfpathlineto{\pgfqpoint{2.189237in}{2.430512in}}%
\pgfpathlineto{\pgfqpoint{2.190311in}{2.426645in}}%
\pgfpathlineto{\pgfqpoint{2.191385in}{2.427270in}}%
\pgfpathlineto{\pgfqpoint{2.195680in}{2.426975in}}%
\pgfpathlineto{\pgfqpoint{2.196754in}{2.430074in}}%
\pgfpathlineto{\pgfqpoint{2.197828in}{2.430627in}}%
\pgfpathlineto{\pgfqpoint{2.198902in}{2.427963in}}%
\pgfpathlineto{\pgfqpoint{2.199976in}{2.431330in}}%
\pgfpathlineto{\pgfqpoint{2.204271in}{2.434600in}}%
\pgfpathlineto{\pgfqpoint{2.205345in}{2.432991in}}%
\pgfpathlineto{\pgfqpoint{2.206419in}{2.433473in}}%
\pgfpathlineto{\pgfqpoint{2.207492in}{2.432878in}}%
\pgfpathlineto{\pgfqpoint{2.210714in}{2.432432in}}%
\pgfpathlineto{\pgfqpoint{2.211788in}{2.434589in}}%
\pgfpathlineto{\pgfqpoint{2.212862in}{2.433659in}}%
\pgfpathlineto{\pgfqpoint{2.213936in}{2.433770in}}%
\pgfpathlineto{\pgfqpoint{2.215009in}{2.434918in}}%
\pgfpathlineto{\pgfqpoint{2.218231in}{2.432363in}}%
\pgfpathlineto{\pgfqpoint{2.220379in}{2.423582in}}%
\pgfpathlineto{\pgfqpoint{2.221452in}{2.423617in}}%
\pgfpathlineto{\pgfqpoint{2.222526in}{2.422816in}}%
\pgfpathlineto{\pgfqpoint{2.225748in}{2.423408in}}%
\pgfpathlineto{\pgfqpoint{2.226822in}{2.425324in}}%
\pgfpathlineto{\pgfqpoint{2.228969in}{2.423951in}}%
\pgfpathlineto{\pgfqpoint{2.230043in}{2.422648in}}%
\pgfpathlineto{\pgfqpoint{2.234338in}{2.425042in}}%
\pgfpathlineto{\pgfqpoint{2.235412in}{2.423527in}}%
\pgfpathlineto{\pgfqpoint{2.236486in}{2.425677in}}%
\pgfpathlineto{\pgfqpoint{2.237560in}{2.426135in}}%
\pgfpathlineto{\pgfqpoint{2.240781in}{2.429315in}}%
\pgfpathlineto{\pgfqpoint{2.241855in}{2.426241in}}%
\pgfpathlineto{\pgfqpoint{2.242929in}{2.427214in}}%
\pgfpathlineto{\pgfqpoint{2.244003in}{2.426795in}}%
\pgfpathlineto{\pgfqpoint{2.245077in}{2.424629in}}%
\pgfpathlineto{\pgfqpoint{2.248298in}{2.427458in}}%
\pgfpathlineto{\pgfqpoint{2.249372in}{2.429415in}}%
\pgfpathlineto{\pgfqpoint{2.250446in}{2.428638in}}%
\pgfpathlineto{\pgfqpoint{2.251520in}{2.426237in}}%
\pgfpathlineto{\pgfqpoint{2.252594in}{2.428144in}}%
\pgfpathlineto{\pgfqpoint{2.255815in}{2.429697in}}%
\pgfpathlineto{\pgfqpoint{2.256889in}{2.431442in}}%
\pgfpathlineto{\pgfqpoint{2.257963in}{2.429270in}}%
\pgfpathlineto{\pgfqpoint{2.259037in}{2.430748in}}%
\pgfpathlineto{\pgfqpoint{2.260111in}{2.431280in}}%
\pgfpathlineto{\pgfqpoint{2.265480in}{2.432772in}}%
\pgfpathlineto{\pgfqpoint{2.266554in}{2.427130in}}%
\pgfpathlineto{\pgfqpoint{2.271923in}{2.423158in}}%
\pgfpathlineto{\pgfqpoint{2.272997in}{2.423362in}}%
\pgfpathlineto{\pgfqpoint{2.274070in}{2.427124in}}%
\pgfpathlineto{\pgfqpoint{2.275144in}{2.427091in}}%
\pgfpathlineto{\pgfqpoint{2.278366in}{2.426413in}}%
\pgfpathlineto{\pgfqpoint{2.279440in}{2.427429in}}%
\pgfpathlineto{\pgfqpoint{2.280514in}{2.429632in}}%
\pgfpathlineto{\pgfqpoint{2.281587in}{2.430353in}}%
\pgfpathlineto{\pgfqpoint{2.282661in}{2.434023in}}%
\pgfpathlineto{\pgfqpoint{2.288030in}{2.439558in}}%
\pgfpathlineto{\pgfqpoint{2.289104in}{2.436564in}}%
\pgfpathlineto{\pgfqpoint{2.290178in}{2.442145in}}%
\pgfpathlineto{\pgfqpoint{2.293400in}{2.445180in}}%
\pgfpathlineto{\pgfqpoint{2.296621in}{2.429943in}}%
\pgfpathlineto{\pgfqpoint{2.297695in}{2.429976in}}%
\pgfpathlineto{\pgfqpoint{2.301990in}{2.432975in}}%
\pgfpathlineto{\pgfqpoint{2.303064in}{2.431855in}}%
\pgfpathlineto{\pgfqpoint{2.304138in}{2.432248in}}%
\pgfpathlineto{\pgfqpoint{2.305212in}{2.433298in}}%
\pgfpathlineto{\pgfqpoint{2.310581in}{2.433265in}}%
\pgfpathlineto{\pgfqpoint{2.312729in}{2.434380in}}%
\pgfpathlineto{\pgfqpoint{2.315950in}{2.434807in}}%
\pgfpathlineto{\pgfqpoint{2.317024in}{2.428395in}}%
\pgfpathlineto{\pgfqpoint{2.319172in}{2.428745in}}%
\pgfpathlineto{\pgfqpoint{2.320246in}{2.428394in}}%
\pgfpathlineto{\pgfqpoint{2.323467in}{2.427981in}}%
\pgfpathlineto{\pgfqpoint{2.325615in}{2.429859in}}%
\pgfpathlineto{\pgfqpoint{2.326689in}{2.428930in}}%
\pgfpathlineto{\pgfqpoint{2.327762in}{2.431204in}}%
\pgfpathlineto{\pgfqpoint{2.330984in}{2.430691in}}%
\pgfpathlineto{\pgfqpoint{2.332058in}{2.432320in}}%
\pgfpathlineto{\pgfqpoint{2.333132in}{2.429517in}}%
\pgfpathlineto{\pgfqpoint{2.334205in}{2.429388in}}%
\pgfpathlineto{\pgfqpoint{2.335279in}{2.430516in}}%
\pgfpathlineto{\pgfqpoint{2.338501in}{2.432288in}}%
\pgfpathlineto{\pgfqpoint{2.340648in}{2.431442in}}%
\pgfpathlineto{\pgfqpoint{2.342796in}{2.434337in}}%
\pgfpathlineto{\pgfqpoint{2.346018in}{2.431580in}}%
\pgfpathlineto{\pgfqpoint{2.347091in}{2.425574in}}%
\pgfpathlineto{\pgfqpoint{2.349239in}{2.430300in}}%
\pgfpathlineto{\pgfqpoint{2.350313in}{2.430300in}}%
\pgfpathlineto{\pgfqpoint{2.353535in}{2.427848in}}%
\pgfpathlineto{\pgfqpoint{2.354608in}{2.428809in}}%
\pgfpathlineto{\pgfqpoint{2.355682in}{2.431062in}}%
\pgfpathlineto{\pgfqpoint{2.356756in}{2.430626in}}%
\pgfpathlineto{\pgfqpoint{2.357830in}{2.432102in}}%
\pgfpathlineto{\pgfqpoint{2.361051in}{2.429352in}}%
\pgfpathlineto{\pgfqpoint{2.362125in}{2.430740in}}%
\pgfpathlineto{\pgfqpoint{2.364273in}{2.429541in}}%
\pgfpathlineto{\pgfqpoint{2.365347in}{2.430274in}}%
\pgfpathlineto{\pgfqpoint{2.369642in}{2.428941in}}%
\pgfpathlineto{\pgfqpoint{2.370716in}{2.429371in}}%
\pgfpathlineto{\pgfqpoint{2.371790in}{2.428940in}}%
\pgfpathlineto{\pgfqpoint{2.372864in}{2.425129in}}%
\pgfpathlineto{\pgfqpoint{2.376085in}{2.427979in}}%
\pgfpathlineto{\pgfqpoint{2.377159in}{2.432752in}}%
\pgfpathlineto{\pgfqpoint{2.378233in}{2.433534in}}%
\pgfpathlineto{\pgfqpoint{2.379307in}{2.431085in}}%
\pgfpathlineto{\pgfqpoint{2.380380in}{2.432260in}}%
\pgfpathlineto{\pgfqpoint{2.383602in}{2.430402in}}%
\pgfpathlineto{\pgfqpoint{2.384676in}{2.431314in}}%
\pgfpathlineto{\pgfqpoint{2.386824in}{2.426612in}}%
\pgfpathlineto{\pgfqpoint{2.387897in}{2.428588in}}%
\pgfpathlineto{\pgfqpoint{2.391119in}{2.425392in}}%
\pgfpathlineto{\pgfqpoint{2.392193in}{2.429853in}}%
\pgfpathlineto{\pgfqpoint{2.393267in}{2.429321in}}%
\pgfpathlineto{\pgfqpoint{2.394340in}{2.435893in}}%
\pgfpathlineto{\pgfqpoint{2.395414in}{2.438645in}}%
\pgfpathlineto{\pgfqpoint{2.398636in}{2.436444in}}%
\pgfpathlineto{\pgfqpoint{2.399710in}{2.436410in}}%
\pgfpathlineto{\pgfqpoint{2.402931in}{2.440467in}}%
\pgfpathlineto{\pgfqpoint{2.406153in}{2.438686in}}%
\pgfpathlineto{\pgfqpoint{2.407226in}{2.445505in}}%
\pgfpathlineto{\pgfqpoint{2.408300in}{2.446560in}}%
\pgfpathlineto{\pgfqpoint{2.409374in}{2.445610in}}%
\pgfpathlineto{\pgfqpoint{2.410448in}{2.447885in}}%
\pgfpathlineto{\pgfqpoint{2.413669in}{2.450966in}}%
\pgfpathlineto{\pgfqpoint{2.414743in}{2.461130in}}%
\pgfpathlineto{\pgfqpoint{2.415817in}{2.458294in}}%
\pgfpathlineto{\pgfqpoint{2.416891in}{2.459264in}}%
\pgfpathlineto{\pgfqpoint{2.417965in}{2.464226in}}%
\pgfpathlineto{\pgfqpoint{2.421186in}{2.464838in}}%
\pgfpathlineto{\pgfqpoint{2.423334in}{2.461486in}}%
\pgfpathlineto{\pgfqpoint{2.424408in}{2.457498in}}%
\pgfpathlineto{\pgfqpoint{2.425482in}{2.458752in}}%
\pgfpathlineto{\pgfqpoint{2.428703in}{2.461016in}}%
\pgfpathlineto{\pgfqpoint{2.429777in}{2.463242in}}%
\pgfpathlineto{\pgfqpoint{2.430851in}{2.461554in}}%
\pgfpathlineto{\pgfqpoint{2.431925in}{2.462154in}}%
\pgfpathlineto{\pgfqpoint{2.432999in}{2.461854in}}%
\pgfpathlineto{\pgfqpoint{2.437294in}{2.455145in}}%
\pgfpathlineto{\pgfqpoint{2.438368in}{2.444166in}}%
\pgfpathlineto{\pgfqpoint{2.439442in}{2.446634in}}%
\pgfpathlineto{\pgfqpoint{2.440515in}{2.443197in}}%
\pgfpathlineto{\pgfqpoint{2.444811in}{2.436254in}}%
\pgfpathlineto{\pgfqpoint{2.446958in}{2.437052in}}%
\pgfpathlineto{\pgfqpoint{2.448032in}{2.438372in}}%
\pgfpathlineto{\pgfqpoint{2.451254in}{2.437183in}}%
\pgfpathlineto{\pgfqpoint{2.452328in}{2.429807in}}%
\pgfpathlineto{\pgfqpoint{2.453402in}{2.430855in}}%
\pgfpathlineto{\pgfqpoint{2.455549in}{2.427359in}}%
\pgfpathlineto{\pgfqpoint{2.458771in}{2.430584in}}%
\pgfpathlineto{\pgfqpoint{2.459845in}{2.428454in}}%
\pgfpathlineto{\pgfqpoint{2.460918in}{2.424614in}}%
\pgfpathlineto{\pgfqpoint{2.461992in}{2.424858in}}%
\pgfpathlineto{\pgfqpoint{2.463066in}{2.422938in}}%
\pgfpathlineto{\pgfqpoint{2.467361in}{2.423387in}}%
\pgfpathlineto{\pgfqpoint{2.468435in}{2.426498in}}%
\pgfpathlineto{\pgfqpoint{2.469509in}{2.426146in}}%
\pgfpathlineto{\pgfqpoint{2.470583in}{2.430330in}}%
\pgfpathlineto{\pgfqpoint{2.474878in}{2.418470in}}%
\pgfpathlineto{\pgfqpoint{2.477026in}{2.424119in}}%
\pgfpathlineto{\pgfqpoint{2.481321in}{2.424587in}}%
\pgfpathlineto{\pgfqpoint{2.482395in}{2.425726in}}%
\pgfpathlineto{\pgfqpoint{2.483469in}{2.424614in}}%
\pgfpathlineto{\pgfqpoint{2.484543in}{2.425018in}}%
\pgfpathlineto{\pgfqpoint{2.485617in}{2.423367in}}%
\pgfpathlineto{\pgfqpoint{2.489912in}{2.423534in}}%
\pgfpathlineto{\pgfqpoint{2.490986in}{2.421027in}}%
\pgfpathlineto{\pgfqpoint{2.493134in}{2.423099in}}%
\pgfpathlineto{\pgfqpoint{2.498503in}{2.425686in}}%
\pgfpathlineto{\pgfqpoint{2.499577in}{2.429494in}}%
\pgfpathlineto{\pgfqpoint{2.500650in}{2.431246in}}%
\pgfpathlineto{\pgfqpoint{2.504946in}{2.430804in}}%
\pgfpathlineto{\pgfqpoint{2.506020in}{2.426245in}}%
\pgfpathlineto{\pgfqpoint{2.508167in}{2.425837in}}%
\pgfpathlineto{\pgfqpoint{2.513536in}{2.427232in}}%
\pgfpathlineto{\pgfqpoint{2.514610in}{2.425628in}}%
\pgfpathlineto{\pgfqpoint{2.518906in}{2.424707in}}%
\pgfpathlineto{\pgfqpoint{2.519979in}{2.417848in}}%
\pgfpathlineto{\pgfqpoint{2.523201in}{2.429262in}}%
\pgfpathlineto{\pgfqpoint{2.528570in}{2.430548in}}%
\pgfpathlineto{\pgfqpoint{2.529644in}{2.431664in}}%
\pgfpathlineto{\pgfqpoint{2.530718in}{2.431420in}}%
\pgfpathlineto{\pgfqpoint{2.536087in}{2.432781in}}%
\pgfpathlineto{\pgfqpoint{2.537161in}{2.435024in}}%
\pgfpathlineto{\pgfqpoint{2.538235in}{2.431204in}}%
\pgfpathlineto{\pgfqpoint{2.541456in}{2.429145in}}%
\pgfpathlineto{\pgfqpoint{2.542530in}{2.427617in}}%
\pgfpathlineto{\pgfqpoint{2.545752in}{2.419849in}}%
\pgfpathlineto{\pgfqpoint{2.550047in}{2.417429in}}%
\pgfpathlineto{\pgfqpoint{2.551121in}{2.417875in}}%
\pgfpathlineto{\pgfqpoint{2.552195in}{2.424417in}}%
\pgfpathlineto{\pgfqpoint{2.553268in}{2.418928in}}%
\pgfpathlineto{\pgfqpoint{2.556490in}{2.419888in}}%
\pgfpathlineto{\pgfqpoint{2.557564in}{2.419327in}}%
\pgfpathlineto{\pgfqpoint{2.558638in}{2.424153in}}%
\pgfpathlineto{\pgfqpoint{2.560785in}{2.425275in}}%
\pgfpathlineto{\pgfqpoint{2.565081in}{2.425900in}}%
\pgfpathlineto{\pgfqpoint{2.566155in}{2.425118in}}%
\pgfpathlineto{\pgfqpoint{2.567228in}{2.419698in}}%
\pgfpathlineto{\pgfqpoint{2.568302in}{2.418942in}}%
\pgfpathlineto{\pgfqpoint{2.571524in}{2.417617in}}%
\pgfpathlineto{\pgfqpoint{2.572598in}{2.416571in}}%
\pgfpathlineto{\pgfqpoint{2.573671in}{2.411222in}}%
\pgfpathlineto{\pgfqpoint{2.574745in}{2.412578in}}%
\pgfpathlineto{\pgfqpoint{2.575819in}{2.411473in}}%
\pgfpathlineto{\pgfqpoint{2.579041in}{2.411503in}}%
\pgfpathlineto{\pgfqpoint{2.582262in}{2.405575in}}%
\pgfpathlineto{\pgfqpoint{2.583336in}{2.404412in}}%
\pgfpathlineto{\pgfqpoint{2.587631in}{2.405662in}}%
\pgfpathlineto{\pgfqpoint{2.588705in}{2.403623in}}%
\pgfpathlineto{\pgfqpoint{2.590853in}{2.402633in}}%
\pgfpathlineto{\pgfqpoint{2.596222in}{2.403361in}}%
\pgfpathlineto{\pgfqpoint{2.597296in}{2.402662in}}%
\pgfpathlineto{\pgfqpoint{2.598370in}{2.403039in}}%
\pgfpathlineto{\pgfqpoint{2.602665in}{2.401121in}}%
\pgfpathlineto{\pgfqpoint{2.603739in}{2.404377in}}%
\pgfpathlineto{\pgfqpoint{2.604813in}{2.405511in}}%
\pgfpathlineto{\pgfqpoint{2.609108in}{2.407530in}}%
\pgfpathlineto{\pgfqpoint{2.610182in}{2.418299in}}%
\pgfpathlineto{\pgfqpoint{2.611256in}{2.416991in}}%
\pgfpathlineto{\pgfqpoint{2.612330in}{2.418455in}}%
\pgfpathlineto{\pgfqpoint{2.613403in}{2.418766in}}%
\pgfpathlineto{\pgfqpoint{2.616625in}{2.422073in}}%
\pgfpathlineto{\pgfqpoint{2.617699in}{2.420482in}}%
\pgfpathlineto{\pgfqpoint{2.618773in}{2.420265in}}%
\pgfpathlineto{\pgfqpoint{2.620920in}{2.421717in}}%
\pgfpathlineto{\pgfqpoint{2.624142in}{2.421314in}}%
\pgfpathlineto{\pgfqpoint{2.627363in}{2.425232in}}%
\pgfpathlineto{\pgfqpoint{2.628437in}{2.426543in}}%
\pgfpathlineto{\pgfqpoint{2.632733in}{2.425663in}}%
\pgfpathlineto{\pgfqpoint{2.633806in}{2.424500in}}%
\pgfpathlineto{\pgfqpoint{2.634880in}{2.425631in}}%
\pgfpathlineto{\pgfqpoint{2.635954in}{2.424312in}}%
\pgfpathlineto{\pgfqpoint{2.640249in}{2.412298in}}%
\pgfpathlineto{\pgfqpoint{2.641323in}{2.415334in}}%
\pgfpathlineto{\pgfqpoint{2.642397in}{2.416195in}}%
\pgfpathlineto{\pgfqpoint{2.643471in}{2.416135in}}%
\pgfpathlineto{\pgfqpoint{2.646692in}{2.416878in}}%
\pgfpathlineto{\pgfqpoint{2.647766in}{2.416165in}}%
\pgfpathlineto{\pgfqpoint{2.648840in}{2.416195in}}%
\pgfpathlineto{\pgfqpoint{2.650988in}{2.420335in}}%
\pgfpathlineto{\pgfqpoint{2.655283in}{2.420873in}}%
\pgfpathlineto{\pgfqpoint{2.658505in}{2.412574in}}%
\pgfpathlineto{\pgfqpoint{2.661726in}{2.411805in}}%
\pgfpathlineto{\pgfqpoint{2.662800in}{2.410556in}}%
\pgfpathlineto{\pgfqpoint{2.663874in}{2.410753in}}%
\pgfpathlineto{\pgfqpoint{2.664948in}{2.410217in}}%
\pgfpathlineto{\pgfqpoint{2.666022in}{2.412841in}}%
\pgfpathlineto{\pgfqpoint{2.670317in}{2.411545in}}%
\pgfpathlineto{\pgfqpoint{2.671391in}{2.409557in}}%
\pgfpathlineto{\pgfqpoint{2.672465in}{2.408837in}}%
\pgfpathlineto{\pgfqpoint{2.673538in}{2.405418in}}%
\pgfpathlineto{\pgfqpoint{2.677834in}{2.406501in}}%
\pgfpathlineto{\pgfqpoint{2.678908in}{2.404736in}}%
\pgfpathlineto{\pgfqpoint{2.679981in}{2.401834in}}%
\pgfpathlineto{\pgfqpoint{2.681055in}{2.400882in}}%
\pgfpathlineto{\pgfqpoint{2.684277in}{2.401093in}}%
\pgfpathlineto{\pgfqpoint{2.687498in}{2.391858in}}%
\pgfpathlineto{\pgfqpoint{2.688572in}{2.390990in}}%
\pgfpathlineto{\pgfqpoint{2.691794in}{2.392516in}}%
\pgfpathlineto{\pgfqpoint{2.693941in}{2.390674in}}%
\pgfpathlineto{\pgfqpoint{2.695015in}{2.393458in}}%
\pgfpathlineto{\pgfqpoint{2.696089in}{2.393511in}}%
\pgfpathlineto{\pgfqpoint{2.700384in}{2.390230in}}%
\pgfpathlineto{\pgfqpoint{2.701458in}{2.392744in}}%
\pgfpathlineto{\pgfqpoint{2.702532in}{2.393670in}}%
\pgfpathlineto{\pgfqpoint{2.703606in}{2.395533in}}%
\pgfpathlineto{\pgfqpoint{2.706827in}{2.396677in}}%
\pgfpathlineto{\pgfqpoint{2.707901in}{2.402571in}}%
\pgfpathlineto{\pgfqpoint{2.708975in}{2.400669in}}%
\pgfpathlineto{\pgfqpoint{2.710049in}{2.397251in}}%
\pgfpathlineto{\pgfqpoint{2.711123in}{2.400367in}}%
\pgfpathlineto{\pgfqpoint{2.715418in}{2.391770in}}%
\pgfpathlineto{\pgfqpoint{2.716492in}{2.389442in}}%
\pgfpathlineto{\pgfqpoint{2.717566in}{2.389124in}}%
\pgfpathlineto{\pgfqpoint{2.718640in}{2.391690in}}%
\pgfpathlineto{\pgfqpoint{2.721861in}{2.389521in}}%
\pgfpathlineto{\pgfqpoint{2.722935in}{2.390017in}}%
\pgfpathlineto{\pgfqpoint{2.724009in}{2.389127in}}%
\pgfpathlineto{\pgfqpoint{2.725083in}{2.392086in}}%
\pgfpathlineto{\pgfqpoint{2.726156in}{2.392505in}}%
\pgfpathlineto{\pgfqpoint{2.729378in}{2.391717in}}%
\pgfpathlineto{\pgfqpoint{2.730452in}{2.390247in}}%
\pgfpathlineto{\pgfqpoint{2.731526in}{2.392636in}}%
\pgfpathlineto{\pgfqpoint{2.737969in}{2.394214in}}%
\pgfpathlineto{\pgfqpoint{2.740116in}{2.399010in}}%
\pgfpathlineto{\pgfqpoint{2.741190in}{2.397965in}}%
\pgfpathlineto{\pgfqpoint{2.744412in}{2.400055in}}%
\pgfpathlineto{\pgfqpoint{2.745486in}{2.396893in}}%
\pgfpathlineto{\pgfqpoint{2.746559in}{2.401627in}}%
\pgfpathlineto{\pgfqpoint{2.747633in}{2.401871in}}%
\pgfpathlineto{\pgfqpoint{2.748707in}{2.402845in}}%
\pgfpathlineto{\pgfqpoint{2.751929in}{2.403008in}}%
\pgfpathlineto{\pgfqpoint{2.753002in}{2.404232in}}%
\pgfpathlineto{\pgfqpoint{2.756224in}{2.414174in}}%
\pgfpathlineto{\pgfqpoint{2.760519in}{2.411123in}}%
\pgfpathlineto{\pgfqpoint{2.761593in}{2.413214in}}%
\pgfpathlineto{\pgfqpoint{2.762667in}{2.409117in}}%
\pgfpathlineto{\pgfqpoint{2.763741in}{2.412319in}}%
\pgfpathlineto{\pgfqpoint{2.768036in}{2.419853in}}%
\pgfpathlineto{\pgfqpoint{2.769110in}{2.419371in}}%
\pgfpathlineto{\pgfqpoint{2.770184in}{2.424050in}}%
\pgfpathlineto{\pgfqpoint{2.774479in}{2.420799in}}%
\pgfpathlineto{\pgfqpoint{2.775553in}{2.421245in}}%
\pgfpathlineto{\pgfqpoint{2.777701in}{2.412003in}}%
\pgfpathlineto{\pgfqpoint{2.778775in}{2.411053in}}%
\pgfpathlineto{\pgfqpoint{2.783070in}{2.409853in}}%
\pgfpathlineto{\pgfqpoint{2.784144in}{2.406670in}}%
\pgfpathlineto{\pgfqpoint{2.785218in}{2.405916in}}%
\pgfpathlineto{\pgfqpoint{2.786291in}{2.407368in}}%
\pgfpathlineto{\pgfqpoint{2.789513in}{2.404157in}}%
\pgfpathlineto{\pgfqpoint{2.791661in}{2.399799in}}%
\pgfpathlineto{\pgfqpoint{2.793808in}{2.398971in}}%
\pgfpathlineto{\pgfqpoint{2.797030in}{2.399556in}}%
\pgfpathlineto{\pgfqpoint{2.798104in}{2.398413in}}%
\pgfpathlineto{\pgfqpoint{2.799178in}{2.391467in}}%
\pgfpathlineto{\pgfqpoint{2.800251in}{2.391036in}}%
\pgfpathlineto{\pgfqpoint{2.801325in}{2.391996in}}%
\pgfpathlineto{\pgfqpoint{2.804547in}{2.391234in}}%
\pgfpathlineto{\pgfqpoint{2.805621in}{2.402574in}}%
\pgfpathlineto{\pgfqpoint{2.806694in}{2.402829in}}%
\pgfpathlineto{\pgfqpoint{2.807768in}{2.406114in}}%
\pgfpathlineto{\pgfqpoint{2.812064in}{2.402284in}}%
\pgfpathlineto{\pgfqpoint{2.813137in}{2.406991in}}%
\pgfpathlineto{\pgfqpoint{2.815285in}{2.410114in}}%
\pgfpathlineto{\pgfqpoint{2.816359in}{2.410062in}}%
\pgfpathlineto{\pgfqpoint{2.820654in}{2.404205in}}%
\pgfpathlineto{\pgfqpoint{2.821728in}{2.405625in}}%
\pgfpathlineto{\pgfqpoint{2.823876in}{2.406115in}}%
\pgfpathlineto{\pgfqpoint{2.828171in}{2.406916in}}%
\pgfpathlineto{\pgfqpoint{2.830319in}{2.413167in}}%
\pgfpathlineto{\pgfqpoint{2.831393in}{2.411194in}}%
\pgfpathlineto{\pgfqpoint{2.835688in}{2.412378in}}%
\pgfpathlineto{\pgfqpoint{2.837836in}{2.410820in}}%
\pgfpathlineto{\pgfqpoint{2.838910in}{2.408629in}}%
\pgfpathlineto{\pgfqpoint{2.842131in}{2.408444in}}%
\pgfpathlineto{\pgfqpoint{2.843205in}{2.409474in}}%
\pgfpathlineto{\pgfqpoint{2.844279in}{2.412114in}}%
\pgfpathlineto{\pgfqpoint{2.846426in}{2.413963in}}%
\pgfpathlineto{\pgfqpoint{2.849648in}{2.411578in}}%
\pgfpathlineto{\pgfqpoint{2.850722in}{2.406712in}}%
\pgfpathlineto{\pgfqpoint{2.851796in}{2.407226in}}%
\pgfpathlineto{\pgfqpoint{2.852869in}{2.406040in}}%
\pgfpathlineto{\pgfqpoint{2.853943in}{2.407665in}}%
\pgfpathlineto{\pgfqpoint{2.857165in}{2.410090in}}%
\pgfpathlineto{\pgfqpoint{2.858239in}{2.411764in}}%
\pgfpathlineto{\pgfqpoint{2.859312in}{2.408285in}}%
\pgfpathlineto{\pgfqpoint{2.861460in}{2.412333in}}%
\pgfpathlineto{\pgfqpoint{2.864682in}{2.410953in}}%
\pgfpathlineto{\pgfqpoint{2.865756in}{2.407337in}}%
\pgfpathlineto{\pgfqpoint{2.866829in}{2.406629in}}%
\pgfpathlineto{\pgfqpoint{2.868977in}{2.411627in}}%
\pgfpathlineto{\pgfqpoint{2.873272in}{2.413216in}}%
\pgfpathlineto{\pgfqpoint{2.874346in}{2.411689in}}%
\pgfpathlineto{\pgfqpoint{2.875420in}{2.411430in}}%
\pgfpathlineto{\pgfqpoint{2.876494in}{2.409101in}}%
\pgfpathlineto{\pgfqpoint{2.880789in}{2.409024in}}%
\pgfpathlineto{\pgfqpoint{2.881863in}{2.410214in}}%
\pgfpathlineto{\pgfqpoint{2.882937in}{2.410240in}}%
\pgfpathlineto{\pgfqpoint{2.884011in}{2.408169in}}%
\pgfpathlineto{\pgfqpoint{2.884011in}{2.408169in}}%
\pgfusepath{stroke}%
\end{pgfscope}%
\begin{pgfscope}%
\pgfsetrectcap%
\pgfsetmiterjoin%
\pgfsetlinewidth{0.803000pt}%
\definecolor{currentstroke}{rgb}{1.000000,1.000000,1.000000}%
\pgfsetstrokecolor{currentstroke}%
\pgfsetdash{}{0pt}%
\pgfpathmoveto{\pgfqpoint{0.418102in}{2.255883in}}%
\pgfpathlineto{\pgfqpoint{0.418102in}{2.656768in}}%
\pgfusepath{stroke}%
\end{pgfscope}%
\begin{pgfscope}%
\pgfsetrectcap%
\pgfsetmiterjoin%
\pgfsetlinewidth{0.803000pt}%
\definecolor{currentstroke}{rgb}{1.000000,1.000000,1.000000}%
\pgfsetstrokecolor{currentstroke}%
\pgfsetdash{}{0pt}%
\pgfpathmoveto{\pgfqpoint{3.001435in}{2.255883in}}%
\pgfpathlineto{\pgfqpoint{3.001435in}{2.656768in}}%
\pgfusepath{stroke}%
\end{pgfscope}%
\begin{pgfscope}%
\pgfsetrectcap%
\pgfsetmiterjoin%
\pgfsetlinewidth{0.803000pt}%
\definecolor{currentstroke}{rgb}{1.000000,1.000000,1.000000}%
\pgfsetstrokecolor{currentstroke}%
\pgfsetdash{}{0pt}%
\pgfpathmoveto{\pgfqpoint{0.418102in}{2.255883in}}%
\pgfpathlineto{\pgfqpoint{3.001435in}{2.255883in}}%
\pgfusepath{stroke}%
\end{pgfscope}%
\begin{pgfscope}%
\pgfsetrectcap%
\pgfsetmiterjoin%
\pgfsetlinewidth{0.803000pt}%
\definecolor{currentstroke}{rgb}{1.000000,1.000000,1.000000}%
\pgfsetstrokecolor{currentstroke}%
\pgfsetdash{}{0pt}%
\pgfpathmoveto{\pgfqpoint{0.418102in}{2.656768in}}%
\pgfpathlineto{\pgfqpoint{3.001435in}{2.656768in}}%
\pgfusepath{stroke}%
\end{pgfscope}%
\begin{pgfscope}%
\definecolor{textcolor}{rgb}{0.150000,0.150000,0.150000}%
\pgfsetstrokecolor{textcolor}%
\pgfsetfillcolor{textcolor}%
\pgftext[x=1.709768in,y=2.740101in,,base]{\color{textcolor}\rmfamily\fontsize{12.000000}{14.400000}\selectfont JNJ}%
\end{pgfscope}%
\begin{pgfscope}%
\pgfsetbuttcap%
\pgfsetmiterjoin%
\definecolor{currentfill}{rgb}{0.917647,0.917647,0.949020}%
\pgfsetfillcolor{currentfill}%
\pgfsetlinewidth{0.000000pt}%
\definecolor{currentstroke}{rgb}{0.000000,0.000000,0.000000}%
\pgfsetstrokecolor{currentstroke}%
\pgfsetstrokeopacity{0.000000}%
\pgfsetdash{}{0pt}%
\pgfpathmoveto{\pgfqpoint{4.034768in}{2.255883in}}%
\pgfpathlineto{\pgfqpoint{6.618102in}{2.255883in}}%
\pgfpathlineto{\pgfqpoint{6.618102in}{2.656768in}}%
\pgfpathlineto{\pgfqpoint{4.034768in}{2.656768in}}%
\pgfpathclose%
\pgfusepath{fill}%
\end{pgfscope}%
\begin{pgfscope}%
\pgfpathrectangle{\pgfqpoint{4.034768in}{2.255883in}}{\pgfqpoint{2.583333in}{0.400885in}}%
\pgfusepath{clip}%
\pgfsetroundcap%
\pgfsetroundjoin%
\pgfsetlinewidth{0.803000pt}%
\definecolor{currentstroke}{rgb}{1.000000,1.000000,1.000000}%
\pgfsetstrokecolor{currentstroke}%
\pgfsetdash{}{0pt}%
\pgfpathmoveto{\pgfqpoint{4.150045in}{2.255883in}}%
\pgfpathlineto{\pgfqpoint{4.150045in}{2.656768in}}%
\pgfusepath{stroke}%
\end{pgfscope}%
\begin{pgfscope}%
\definecolor{textcolor}{rgb}{0.150000,0.150000,0.150000}%
\pgfsetstrokecolor{textcolor}%
\pgfsetfillcolor{textcolor}%
\pgftext[x=4.150045in,y=2.158661in,,top]{\color{textcolor}\rmfamily\fontsize{10.000000}{12.000000}\selectfont 2012}%
\end{pgfscope}%
\begin{pgfscope}%
\pgfpathrectangle{\pgfqpoint{4.034768in}{2.255883in}}{\pgfqpoint{2.583333in}{0.400885in}}%
\pgfusepath{clip}%
\pgfsetroundcap%
\pgfsetroundjoin%
\pgfsetlinewidth{0.803000pt}%
\definecolor{currentstroke}{rgb}{1.000000,1.000000,1.000000}%
\pgfsetstrokecolor{currentstroke}%
\pgfsetdash{}{0pt}%
\pgfpathmoveto{\pgfqpoint{4.543070in}{2.255883in}}%
\pgfpathlineto{\pgfqpoint{4.543070in}{2.656768in}}%
\pgfusepath{stroke}%
\end{pgfscope}%
\begin{pgfscope}%
\definecolor{textcolor}{rgb}{0.150000,0.150000,0.150000}%
\pgfsetstrokecolor{textcolor}%
\pgfsetfillcolor{textcolor}%
\pgftext[x=4.543070in,y=2.158661in,,top]{\color{textcolor}\rmfamily\fontsize{10.000000}{12.000000}\selectfont 2013}%
\end{pgfscope}%
\begin{pgfscope}%
\pgfpathrectangle{\pgfqpoint{4.034768in}{2.255883in}}{\pgfqpoint{2.583333in}{0.400885in}}%
\pgfusepath{clip}%
\pgfsetroundcap%
\pgfsetroundjoin%
\pgfsetlinewidth{0.803000pt}%
\definecolor{currentstroke}{rgb}{1.000000,1.000000,1.000000}%
\pgfsetstrokecolor{currentstroke}%
\pgfsetdash{}{0pt}%
\pgfpathmoveto{\pgfqpoint{4.935021in}{2.255883in}}%
\pgfpathlineto{\pgfqpoint{4.935021in}{2.656768in}}%
\pgfusepath{stroke}%
\end{pgfscope}%
\begin{pgfscope}%
\definecolor{textcolor}{rgb}{0.150000,0.150000,0.150000}%
\pgfsetstrokecolor{textcolor}%
\pgfsetfillcolor{textcolor}%
\pgftext[x=4.935021in,y=2.158661in,,top]{\color{textcolor}\rmfamily\fontsize{10.000000}{12.000000}\selectfont 2014}%
\end{pgfscope}%
\begin{pgfscope}%
\pgfpathrectangle{\pgfqpoint{4.034768in}{2.255883in}}{\pgfqpoint{2.583333in}{0.400885in}}%
\pgfusepath{clip}%
\pgfsetroundcap%
\pgfsetroundjoin%
\pgfsetlinewidth{0.803000pt}%
\definecolor{currentstroke}{rgb}{1.000000,1.000000,1.000000}%
\pgfsetstrokecolor{currentstroke}%
\pgfsetdash{}{0pt}%
\pgfpathmoveto{\pgfqpoint{5.326972in}{2.255883in}}%
\pgfpathlineto{\pgfqpoint{5.326972in}{2.656768in}}%
\pgfusepath{stroke}%
\end{pgfscope}%
\begin{pgfscope}%
\definecolor{textcolor}{rgb}{0.150000,0.150000,0.150000}%
\pgfsetstrokecolor{textcolor}%
\pgfsetfillcolor{textcolor}%
\pgftext[x=5.326972in,y=2.158661in,,top]{\color{textcolor}\rmfamily\fontsize{10.000000}{12.000000}\selectfont 2015}%
\end{pgfscope}%
\begin{pgfscope}%
\pgfpathrectangle{\pgfqpoint{4.034768in}{2.255883in}}{\pgfqpoint{2.583333in}{0.400885in}}%
\pgfusepath{clip}%
\pgfsetroundcap%
\pgfsetroundjoin%
\pgfsetlinewidth{0.803000pt}%
\definecolor{currentstroke}{rgb}{1.000000,1.000000,1.000000}%
\pgfsetstrokecolor{currentstroke}%
\pgfsetdash{}{0pt}%
\pgfpathmoveto{\pgfqpoint{5.718923in}{2.255883in}}%
\pgfpathlineto{\pgfqpoint{5.718923in}{2.656768in}}%
\pgfusepath{stroke}%
\end{pgfscope}%
\begin{pgfscope}%
\definecolor{textcolor}{rgb}{0.150000,0.150000,0.150000}%
\pgfsetstrokecolor{textcolor}%
\pgfsetfillcolor{textcolor}%
\pgftext[x=5.718923in,y=2.158661in,,top]{\color{textcolor}\rmfamily\fontsize{10.000000}{12.000000}\selectfont 2016}%
\end{pgfscope}%
\begin{pgfscope}%
\pgfpathrectangle{\pgfqpoint{4.034768in}{2.255883in}}{\pgfqpoint{2.583333in}{0.400885in}}%
\pgfusepath{clip}%
\pgfsetroundcap%
\pgfsetroundjoin%
\pgfsetlinewidth{0.803000pt}%
\definecolor{currentstroke}{rgb}{1.000000,1.000000,1.000000}%
\pgfsetstrokecolor{currentstroke}%
\pgfsetdash{}{0pt}%
\pgfpathmoveto{\pgfqpoint{6.111948in}{2.255883in}}%
\pgfpathlineto{\pgfqpoint{6.111948in}{2.656768in}}%
\pgfusepath{stroke}%
\end{pgfscope}%
\begin{pgfscope}%
\definecolor{textcolor}{rgb}{0.150000,0.150000,0.150000}%
\pgfsetstrokecolor{textcolor}%
\pgfsetfillcolor{textcolor}%
\pgftext[x=6.111948in,y=2.158661in,,top]{\color{textcolor}\rmfamily\fontsize{10.000000}{12.000000}\selectfont 2017}%
\end{pgfscope}%
\begin{pgfscope}%
\pgfpathrectangle{\pgfqpoint{4.034768in}{2.255883in}}{\pgfqpoint{2.583333in}{0.400885in}}%
\pgfusepath{clip}%
\pgfsetroundcap%
\pgfsetroundjoin%
\pgfsetlinewidth{0.803000pt}%
\definecolor{currentstroke}{rgb}{1.000000,1.000000,1.000000}%
\pgfsetstrokecolor{currentstroke}%
\pgfsetdash{}{0pt}%
\pgfpathmoveto{\pgfqpoint{6.503899in}{2.255883in}}%
\pgfpathlineto{\pgfqpoint{6.503899in}{2.656768in}}%
\pgfusepath{stroke}%
\end{pgfscope}%
\begin{pgfscope}%
\definecolor{textcolor}{rgb}{0.150000,0.150000,0.150000}%
\pgfsetstrokecolor{textcolor}%
\pgfsetfillcolor{textcolor}%
\pgftext[x=6.503899in,y=2.158661in,,top]{\color{textcolor}\rmfamily\fontsize{10.000000}{12.000000}\selectfont 2018}%
\end{pgfscope}%
\begin{pgfscope}%
\pgfpathrectangle{\pgfqpoint{4.034768in}{2.255883in}}{\pgfqpoint{2.583333in}{0.400885in}}%
\pgfusepath{clip}%
\pgfsetroundcap%
\pgfsetroundjoin%
\pgfsetlinewidth{0.803000pt}%
\definecolor{currentstroke}{rgb}{1.000000,1.000000,1.000000}%
\pgfsetstrokecolor{currentstroke}%
\pgfsetdash{}{0pt}%
\pgfpathmoveto{\pgfqpoint{4.034768in}{2.340231in}}%
\pgfpathlineto{\pgfqpoint{6.618102in}{2.340231in}}%
\pgfusepath{stroke}%
\end{pgfscope}%
\begin{pgfscope}%
\definecolor{textcolor}{rgb}{0.150000,0.150000,0.150000}%
\pgfsetstrokecolor{textcolor}%
\pgfsetfillcolor{textcolor}%
\pgftext[x=3.716667in,y=2.287470in,left,base]{\color{textcolor}\rmfamily\fontsize{10.000000}{12.000000}\selectfont 1.0}%
\end{pgfscope}%
\begin{pgfscope}%
\pgfpathrectangle{\pgfqpoint{4.034768in}{2.255883in}}{\pgfqpoint{2.583333in}{0.400885in}}%
\pgfusepath{clip}%
\pgfsetroundcap%
\pgfsetroundjoin%
\pgfsetlinewidth{0.803000pt}%
\definecolor{currentstroke}{rgb}{1.000000,1.000000,1.000000}%
\pgfsetstrokecolor{currentstroke}%
\pgfsetdash{}{0pt}%
\pgfpathmoveto{\pgfqpoint{4.034768in}{2.554691in}}%
\pgfpathlineto{\pgfqpoint{6.618102in}{2.554691in}}%
\pgfusepath{stroke}%
\end{pgfscope}%
\begin{pgfscope}%
\definecolor{textcolor}{rgb}{0.150000,0.150000,0.150000}%
\pgfsetstrokecolor{textcolor}%
\pgfsetfillcolor{textcolor}%
\pgftext[x=3.716667in,y=2.501929in,left,base]{\color{textcolor}\rmfamily\fontsize{10.000000}{12.000000}\selectfont 1.5}%
\end{pgfscope}%
\begin{pgfscope}%
\pgfpathrectangle{\pgfqpoint{4.034768in}{2.255883in}}{\pgfqpoint{2.583333in}{0.400885in}}%
\pgfusepath{clip}%
\pgfsetroundcap%
\pgfsetroundjoin%
\pgfsetlinewidth{1.505625pt}%
\definecolor{currentstroke}{rgb}{0.000000,0.000000,0.000000}%
\pgfsetstrokecolor{currentstroke}%
\pgfsetdash{}{0pt}%
\pgfpathmoveto{\pgfqpoint{4.152193in}{2.340231in}}%
\pgfpathlineto{\pgfqpoint{4.153266in}{2.339985in}}%
\pgfpathlineto{\pgfqpoint{4.155414in}{2.337194in}}%
\pgfpathlineto{\pgfqpoint{4.158636in}{2.339000in}}%
\pgfpathlineto{\pgfqpoint{4.159709in}{2.336947in}}%
\pgfpathlineto{\pgfqpoint{4.160783in}{2.332843in}}%
\pgfpathlineto{\pgfqpoint{4.161857in}{2.333664in}}%
\pgfpathlineto{\pgfqpoint{4.162931in}{2.333664in}}%
\pgfpathlineto{\pgfqpoint{4.167226in}{2.336537in}}%
\pgfpathlineto{\pgfqpoint{4.168300in}{2.338425in}}%
\pgfpathlineto{\pgfqpoint{4.169374in}{2.338753in}}%
\pgfpathlineto{\pgfqpoint{4.170448in}{2.339738in}}%
\pgfpathlineto{\pgfqpoint{4.174743in}{2.328574in}}%
\pgfpathlineto{\pgfqpoint{4.175817in}{2.331694in}}%
\pgfpathlineto{\pgfqpoint{4.176891in}{2.330462in}}%
\pgfpathlineto{\pgfqpoint{4.177965in}{2.327261in}}%
\pgfpathlineto{\pgfqpoint{4.182260in}{2.319134in}}%
\pgfpathlineto{\pgfqpoint{4.184408in}{2.320940in}}%
\pgfpathlineto{\pgfqpoint{4.185482in}{2.317328in}}%
\pgfpathlineto{\pgfqpoint{4.189777in}{2.323403in}}%
\pgfpathlineto{\pgfqpoint{4.190851in}{2.322992in}}%
\pgfpathlineto{\pgfqpoint{4.191925in}{2.325537in}}%
\pgfpathlineto{\pgfqpoint{4.192998in}{2.324552in}}%
\pgfpathlineto{\pgfqpoint{4.196220in}{2.326768in}}%
\pgfpathlineto{\pgfqpoint{4.197294in}{2.328410in}}%
\pgfpathlineto{\pgfqpoint{4.198368in}{2.328903in}}%
\pgfpathlineto{\pgfqpoint{4.199441in}{2.333089in}}%
\pgfpathlineto{\pgfqpoint{4.200515in}{2.331201in}}%
\pgfpathlineto{\pgfqpoint{4.204811in}{2.328000in}}%
\pgfpathlineto{\pgfqpoint{4.205885in}{2.328164in}}%
\pgfpathlineto{\pgfqpoint{4.206958in}{2.340970in}}%
\pgfpathlineto{\pgfqpoint{4.208032in}{2.342858in}}%
\pgfpathlineto{\pgfqpoint{4.211254in}{2.342776in}}%
\pgfpathlineto{\pgfqpoint{4.212328in}{2.347209in}}%
\pgfpathlineto{\pgfqpoint{4.213401in}{2.348768in}}%
\pgfpathlineto{\pgfqpoint{4.214475in}{2.342530in}}%
\pgfpathlineto{\pgfqpoint{4.215549in}{2.342612in}}%
\pgfpathlineto{\pgfqpoint{4.218771in}{2.344418in}}%
\pgfpathlineto{\pgfqpoint{4.219844in}{2.343679in}}%
\pgfpathlineto{\pgfqpoint{4.220918in}{2.342037in}}%
\pgfpathlineto{\pgfqpoint{4.221992in}{2.344089in}}%
\pgfpathlineto{\pgfqpoint{4.223066in}{2.344253in}}%
\pgfpathlineto{\pgfqpoint{4.227361in}{2.350574in}}%
\pgfpathlineto{\pgfqpoint{4.228435in}{2.350246in}}%
\pgfpathlineto{\pgfqpoint{4.229509in}{2.349097in}}%
\pgfpathlineto{\pgfqpoint{4.230583in}{2.346306in}}%
\pgfpathlineto{\pgfqpoint{4.235952in}{2.345977in}}%
\pgfpathlineto{\pgfqpoint{4.237026in}{2.348112in}}%
\pgfpathlineto{\pgfqpoint{4.238100in}{2.347537in}}%
\pgfpathlineto{\pgfqpoint{4.241321in}{2.347701in}}%
\pgfpathlineto{\pgfqpoint{4.242395in}{2.345731in}}%
\pgfpathlineto{\pgfqpoint{4.243469in}{2.345977in}}%
\pgfpathlineto{\pgfqpoint{4.244543in}{2.344746in}}%
\pgfpathlineto{\pgfqpoint{4.245617in}{2.346059in}}%
\pgfpathlineto{\pgfqpoint{4.248838in}{2.348358in}}%
\pgfpathlineto{\pgfqpoint{4.249912in}{2.345321in}}%
\pgfpathlineto{\pgfqpoint{4.252060in}{2.346716in}}%
\pgfpathlineto{\pgfqpoint{4.256355in}{2.343515in}}%
\pgfpathlineto{\pgfqpoint{4.257429in}{2.340477in}}%
\pgfpathlineto{\pgfqpoint{4.258503in}{2.341134in}}%
\pgfpathlineto{\pgfqpoint{4.260650in}{2.337030in}}%
\pgfpathlineto{\pgfqpoint{4.264946in}{2.344828in}}%
\pgfpathlineto{\pgfqpoint{4.267093in}{2.341955in}}%
\pgfpathlineto{\pgfqpoint{4.268167in}{2.348030in}}%
\pgfpathlineto{\pgfqpoint{4.271389in}{2.342447in}}%
\pgfpathlineto{\pgfqpoint{4.273536in}{2.347701in}}%
\pgfpathlineto{\pgfqpoint{4.274610in}{2.347537in}}%
\pgfpathlineto{\pgfqpoint{4.275684in}{2.331694in}}%
\pgfpathlineto{\pgfqpoint{4.278906in}{2.326440in}}%
\pgfpathlineto{\pgfqpoint{4.279979in}{2.326029in}}%
\pgfpathlineto{\pgfqpoint{4.282127in}{2.332104in}}%
\pgfpathlineto{\pgfqpoint{4.283201in}{2.330627in}}%
\pgfpathlineto{\pgfqpoint{4.287496in}{2.329888in}}%
\pgfpathlineto{\pgfqpoint{4.288570in}{2.326686in}}%
\pgfpathlineto{\pgfqpoint{4.289644in}{2.329724in}}%
\pgfpathlineto{\pgfqpoint{4.290718in}{2.326768in}}%
\pgfpathlineto{\pgfqpoint{4.293939in}{2.326112in}}%
\pgfpathlineto{\pgfqpoint{4.295013in}{2.327015in}}%
\pgfpathlineto{\pgfqpoint{4.296087in}{2.330709in}}%
\pgfpathlineto{\pgfqpoint{4.298235in}{2.325701in}}%
\pgfpathlineto{\pgfqpoint{4.301456in}{2.324798in}}%
\pgfpathlineto{\pgfqpoint{4.302530in}{2.323238in}}%
\pgfpathlineto{\pgfqpoint{4.303604in}{2.318313in}}%
\pgfpathlineto{\pgfqpoint{4.304678in}{2.319462in}}%
\pgfpathlineto{\pgfqpoint{4.305751in}{2.318970in}}%
\pgfpathlineto{\pgfqpoint{4.310047in}{2.322007in}}%
\pgfpathlineto{\pgfqpoint{4.311121in}{2.317820in}}%
\pgfpathlineto{\pgfqpoint{4.312195in}{2.317656in}}%
\pgfpathlineto{\pgfqpoint{4.313268in}{2.312813in}}%
\pgfpathlineto{\pgfqpoint{4.316490in}{2.311828in}}%
\pgfpathlineto{\pgfqpoint{4.317564in}{2.310350in}}%
\pgfpathlineto{\pgfqpoint{4.320785in}{2.320694in}}%
\pgfpathlineto{\pgfqpoint{4.324007in}{2.319298in}}%
\pgfpathlineto{\pgfqpoint{4.325081in}{2.320694in}}%
\pgfpathlineto{\pgfqpoint{4.326154in}{2.319462in}}%
\pgfpathlineto{\pgfqpoint{4.327228in}{2.323485in}}%
\pgfpathlineto{\pgfqpoint{4.328302in}{2.321515in}}%
\pgfpathlineto{\pgfqpoint{4.332597in}{2.317164in}}%
\pgfpathlineto{\pgfqpoint{4.333671in}{2.305261in}}%
\pgfpathlineto{\pgfqpoint{4.334745in}{2.301074in}}%
\pgfpathlineto{\pgfqpoint{4.335819in}{2.301649in}}%
\pgfpathlineto{\pgfqpoint{4.339040in}{2.298201in}}%
\pgfpathlineto{\pgfqpoint{4.340114in}{2.297955in}}%
\pgfpathlineto{\pgfqpoint{4.341188in}{2.302634in}}%
\pgfpathlineto{\pgfqpoint{4.342262in}{2.304604in}}%
\pgfpathlineto{\pgfqpoint{4.343336in}{2.310843in}}%
\pgfpathlineto{\pgfqpoint{4.346557in}{2.310514in}}%
\pgfpathlineto{\pgfqpoint{4.347631in}{2.311582in}}%
\pgfpathlineto{\pgfqpoint{4.350853in}{2.311089in}}%
\pgfpathlineto{\pgfqpoint{4.354074in}{2.312813in}}%
\pgfpathlineto{\pgfqpoint{4.355148in}{2.314044in}}%
\pgfpathlineto{\pgfqpoint{4.356222in}{2.311828in}}%
\pgfpathlineto{\pgfqpoint{4.358370in}{2.335962in}}%
\pgfpathlineto{\pgfqpoint{4.361591in}{2.334074in}}%
\pgfpathlineto{\pgfqpoint{4.362665in}{2.337604in}}%
\pgfpathlineto{\pgfqpoint{4.364813in}{2.338507in}}%
\pgfpathlineto{\pgfqpoint{4.365886in}{2.337276in}}%
\pgfpathlineto{\pgfqpoint{4.369108in}{2.334977in}}%
\pgfpathlineto{\pgfqpoint{4.370182in}{2.332597in}}%
\pgfpathlineto{\pgfqpoint{4.371256in}{2.332597in}}%
\pgfpathlineto{\pgfqpoint{4.373403in}{2.339574in}}%
\pgfpathlineto{\pgfqpoint{4.376625in}{2.339656in}}%
\pgfpathlineto{\pgfqpoint{4.379846in}{2.329231in}}%
\pgfpathlineto{\pgfqpoint{4.380920in}{2.342283in}}%
\pgfpathlineto{\pgfqpoint{4.384142in}{2.344336in}}%
\pgfpathlineto{\pgfqpoint{4.386289in}{2.350328in}}%
\pgfpathlineto{\pgfqpoint{4.388437in}{2.350656in}}%
\pgfpathlineto{\pgfqpoint{4.391659in}{2.348768in}}%
\pgfpathlineto{\pgfqpoint{4.392732in}{2.350410in}}%
\pgfpathlineto{\pgfqpoint{4.393806in}{2.349836in}}%
\pgfpathlineto{\pgfqpoint{4.394880in}{2.352216in}}%
\pgfpathlineto{\pgfqpoint{4.395954in}{2.352216in}}%
\pgfpathlineto{\pgfqpoint{4.399175in}{2.350574in}}%
\pgfpathlineto{\pgfqpoint{4.401323in}{2.351231in}}%
\pgfpathlineto{\pgfqpoint{4.402397in}{2.350082in}}%
\pgfpathlineto{\pgfqpoint{4.403471in}{2.352298in}}%
\pgfpathlineto{\pgfqpoint{4.406692in}{2.352873in}}%
\pgfpathlineto{\pgfqpoint{4.408840in}{2.351313in}}%
\pgfpathlineto{\pgfqpoint{4.409914in}{2.351395in}}%
\pgfpathlineto{\pgfqpoint{4.410988in}{2.353448in}}%
\pgfpathlineto{\pgfqpoint{4.415283in}{2.354843in}}%
\pgfpathlineto{\pgfqpoint{4.416357in}{2.354104in}}%
\pgfpathlineto{\pgfqpoint{4.417431in}{2.360343in}}%
\pgfpathlineto{\pgfqpoint{4.418505in}{2.362149in}}%
\pgfpathlineto{\pgfqpoint{4.421726in}{2.362149in}}%
\pgfpathlineto{\pgfqpoint{4.423874in}{2.359440in}}%
\pgfpathlineto{\pgfqpoint{4.424948in}{2.364776in}}%
\pgfpathlineto{\pgfqpoint{4.426021in}{2.366418in}}%
\pgfpathlineto{\pgfqpoint{4.431391in}{2.367074in}}%
\pgfpathlineto{\pgfqpoint{4.432464in}{2.369045in}}%
\pgfpathlineto{\pgfqpoint{4.433538in}{2.368142in}}%
\pgfpathlineto{\pgfqpoint{4.436760in}{2.370358in}}%
\pgfpathlineto{\pgfqpoint{4.439981in}{2.367321in}}%
\pgfpathlineto{\pgfqpoint{4.441055in}{2.367731in}}%
\pgfpathlineto{\pgfqpoint{4.444277in}{2.368306in}}%
\pgfpathlineto{\pgfqpoint{4.445350in}{2.363955in}}%
\pgfpathlineto{\pgfqpoint{4.447498in}{2.367813in}}%
\pgfpathlineto{\pgfqpoint{4.448572in}{2.369455in}}%
\pgfpathlineto{\pgfqpoint{4.451794in}{2.366007in}}%
\pgfpathlineto{\pgfqpoint{4.455015in}{2.358783in}}%
\pgfpathlineto{\pgfqpoint{4.456089in}{2.358373in}}%
\pgfpathlineto{\pgfqpoint{4.460384in}{2.365351in}}%
\pgfpathlineto{\pgfqpoint{4.461458in}{2.372164in}}%
\pgfpathlineto{\pgfqpoint{4.462532in}{2.372164in}}%
\pgfpathlineto{\pgfqpoint{4.463606in}{2.366254in}}%
\pgfpathlineto{\pgfqpoint{4.466827in}{2.365761in}}%
\pgfpathlineto{\pgfqpoint{4.467901in}{2.358701in}}%
\pgfpathlineto{\pgfqpoint{4.468975in}{2.362970in}}%
\pgfpathlineto{\pgfqpoint{4.470049in}{2.376186in}}%
\pgfpathlineto{\pgfqpoint{4.471123in}{2.372000in}}%
\pgfpathlineto{\pgfqpoint{4.478639in}{2.370358in}}%
\pgfpathlineto{\pgfqpoint{4.481861in}{2.366910in}}%
\pgfpathlineto{\pgfqpoint{4.482935in}{2.368634in}}%
\pgfpathlineto{\pgfqpoint{4.485083in}{2.355171in}}%
\pgfpathlineto{\pgfqpoint{4.486156in}{2.355828in}}%
\pgfpathlineto{\pgfqpoint{4.489378in}{2.356321in}}%
\pgfpathlineto{\pgfqpoint{4.492599in}{2.351313in}}%
\pgfpathlineto{\pgfqpoint{4.493673in}{2.354597in}}%
\pgfpathlineto{\pgfqpoint{4.501190in}{2.372985in}}%
\pgfpathlineto{\pgfqpoint{4.504412in}{2.372246in}}%
\pgfpathlineto{\pgfqpoint{4.505485in}{2.368962in}}%
\pgfpathlineto{\pgfqpoint{4.506559in}{2.372000in}}%
\pgfpathlineto{\pgfqpoint{4.507633in}{2.372410in}}%
\pgfpathlineto{\pgfqpoint{4.508707in}{2.374545in}}%
\pgfpathlineto{\pgfqpoint{4.511928in}{2.372903in}}%
\pgfpathlineto{\pgfqpoint{4.513002in}{2.371097in}}%
\pgfpathlineto{\pgfqpoint{4.514076in}{2.371754in}}%
\pgfpathlineto{\pgfqpoint{4.516224in}{2.377664in}}%
\pgfpathlineto{\pgfqpoint{4.519445in}{2.377171in}}%
\pgfpathlineto{\pgfqpoint{4.520519in}{2.380045in}}%
\pgfpathlineto{\pgfqpoint{4.521593in}{2.380783in}}%
\pgfpathlineto{\pgfqpoint{4.522667in}{2.376597in}}%
\pgfpathlineto{\pgfqpoint{4.523741in}{2.375201in}}%
\pgfpathlineto{\pgfqpoint{4.528036in}{2.375530in}}%
\pgfpathlineto{\pgfqpoint{4.529110in}{2.371343in}}%
\pgfpathlineto{\pgfqpoint{4.530184in}{2.374545in}}%
\pgfpathlineto{\pgfqpoint{4.531258in}{2.367239in}}%
\pgfpathlineto{\pgfqpoint{4.534479in}{2.365843in}}%
\pgfpathlineto{\pgfqpoint{4.536627in}{2.362395in}}%
\pgfpathlineto{\pgfqpoint{4.537701in}{2.362231in}}%
\pgfpathlineto{\pgfqpoint{4.538774in}{2.356813in}}%
\pgfpathlineto{\pgfqpoint{4.541996in}{2.361739in}}%
\pgfpathlineto{\pgfqpoint{4.544144in}{2.371671in}}%
\pgfpathlineto{\pgfqpoint{4.545217in}{2.368716in}}%
\pgfpathlineto{\pgfqpoint{4.546291in}{2.369701in}}%
\pgfpathlineto{\pgfqpoint{4.550587in}{2.365843in}}%
\pgfpathlineto{\pgfqpoint{4.552734in}{2.370851in}}%
\pgfpathlineto{\pgfqpoint{4.553808in}{2.370522in}}%
\pgfpathlineto{\pgfqpoint{4.557030in}{2.373231in}}%
\pgfpathlineto{\pgfqpoint{4.558104in}{2.374873in}}%
\pgfpathlineto{\pgfqpoint{4.559177in}{2.375037in}}%
\pgfpathlineto{\pgfqpoint{4.561325in}{2.379060in}}%
\pgfpathlineto{\pgfqpoint{4.565620in}{2.379142in}}%
\pgfpathlineto{\pgfqpoint{4.566694in}{2.384067in}}%
\pgfpathlineto{\pgfqpoint{4.567768in}{2.382261in}}%
\pgfpathlineto{\pgfqpoint{4.568842in}{2.401224in}}%
\pgfpathlineto{\pgfqpoint{4.572063in}{2.404672in}}%
\pgfpathlineto{\pgfqpoint{4.573137in}{2.412881in}}%
\pgfpathlineto{\pgfqpoint{4.575285in}{2.413948in}}%
\pgfpathlineto{\pgfqpoint{4.576359in}{2.419037in}}%
\pgfpathlineto{\pgfqpoint{4.579580in}{2.414604in}}%
\pgfpathlineto{\pgfqpoint{4.581728in}{2.420597in}}%
\pgfpathlineto{\pgfqpoint{4.582802in}{2.420597in}}%
\pgfpathlineto{\pgfqpoint{4.583876in}{2.417888in}}%
\pgfpathlineto{\pgfqpoint{4.587097in}{2.418299in}}%
\pgfpathlineto{\pgfqpoint{4.588171in}{2.419448in}}%
\pgfpathlineto{\pgfqpoint{4.589245in}{2.423306in}}%
\pgfpathlineto{\pgfqpoint{4.590319in}{2.424784in}}%
\pgfpathlineto{\pgfqpoint{4.591393in}{2.423224in}}%
\pgfpathlineto{\pgfqpoint{4.595688in}{2.428806in}}%
\pgfpathlineto{\pgfqpoint{4.596762in}{2.426836in}}%
\pgfpathlineto{\pgfqpoint{4.598909in}{2.426179in}}%
\pgfpathlineto{\pgfqpoint{4.602131in}{2.419037in}}%
\pgfpathlineto{\pgfqpoint{4.603205in}{2.420105in}}%
\pgfpathlineto{\pgfqpoint{4.604279in}{2.424619in}}%
\pgfpathlineto{\pgfqpoint{4.605352in}{2.420761in}}%
\pgfpathlineto{\pgfqpoint{4.606426in}{2.422896in}}%
\pgfpathlineto{\pgfqpoint{4.609648in}{2.424127in}}%
\pgfpathlineto{\pgfqpoint{4.610722in}{2.426590in}}%
\pgfpathlineto{\pgfqpoint{4.611795in}{2.427657in}}%
\pgfpathlineto{\pgfqpoint{4.612869in}{2.425605in}}%
\pgfpathlineto{\pgfqpoint{4.613943in}{2.427493in}}%
\pgfpathlineto{\pgfqpoint{4.617165in}{2.428642in}}%
\pgfpathlineto{\pgfqpoint{4.618238in}{2.427411in}}%
\pgfpathlineto{\pgfqpoint{4.619312in}{2.424948in}}%
\pgfpathlineto{\pgfqpoint{4.620386in}{2.428888in}}%
\pgfpathlineto{\pgfqpoint{4.621460in}{2.421828in}}%
\pgfpathlineto{\pgfqpoint{4.624682in}{2.420679in}}%
\pgfpathlineto{\pgfqpoint{4.626829in}{2.430202in}}%
\pgfpathlineto{\pgfqpoint{4.627903in}{2.427657in}}%
\pgfpathlineto{\pgfqpoint{4.628977in}{2.428067in}}%
\pgfpathlineto{\pgfqpoint{4.632198in}{2.424127in}}%
\pgfpathlineto{\pgfqpoint{4.633272in}{2.428970in}}%
\pgfpathlineto{\pgfqpoint{4.634346in}{2.426672in}}%
\pgfpathlineto{\pgfqpoint{4.635420in}{2.426672in}}%
\pgfpathlineto{\pgfqpoint{4.639715in}{2.430940in}}%
\pgfpathlineto{\pgfqpoint{4.640789in}{2.439396in}}%
\pgfpathlineto{\pgfqpoint{4.641863in}{2.433814in}}%
\pgfpathlineto{\pgfqpoint{4.642937in}{2.436605in}}%
\pgfpathlineto{\pgfqpoint{4.644011in}{2.434470in}}%
\pgfpathlineto{\pgfqpoint{4.647232in}{2.438246in}}%
\pgfpathlineto{\pgfqpoint{4.648306in}{2.434717in}}%
\pgfpathlineto{\pgfqpoint{4.649380in}{2.441284in}}%
\pgfpathlineto{\pgfqpoint{4.651527in}{2.446866in}}%
\pgfpathlineto{\pgfqpoint{4.654749in}{2.443993in}}%
\pgfpathlineto{\pgfqpoint{4.655823in}{2.447030in}}%
\pgfpathlineto{\pgfqpoint{4.656897in}{2.440052in}}%
\pgfpathlineto{\pgfqpoint{4.657971in}{2.445470in}}%
\pgfpathlineto{\pgfqpoint{4.659044in}{2.455896in}}%
\pgfpathlineto{\pgfqpoint{4.662266in}{2.455732in}}%
\pgfpathlineto{\pgfqpoint{4.663340in}{2.463366in}}%
\pgfpathlineto{\pgfqpoint{4.664414in}{2.430858in}}%
\pgfpathlineto{\pgfqpoint{4.665487in}{2.427246in}}%
\pgfpathlineto{\pgfqpoint{4.666561in}{2.430776in}}%
\pgfpathlineto{\pgfqpoint{4.669783in}{2.434634in}}%
\pgfpathlineto{\pgfqpoint{4.670857in}{2.428478in}}%
\pgfpathlineto{\pgfqpoint{4.671930in}{2.429955in}}%
\pgfpathlineto{\pgfqpoint{4.674078in}{2.438082in}}%
\pgfpathlineto{\pgfqpoint{4.677300in}{2.435127in}}%
\pgfpathlineto{\pgfqpoint{4.678373in}{2.436358in}}%
\pgfpathlineto{\pgfqpoint{4.679447in}{2.439724in}}%
\pgfpathlineto{\pgfqpoint{4.680521in}{2.438246in}}%
\pgfpathlineto{\pgfqpoint{4.681595in}{2.441940in}}%
\pgfpathlineto{\pgfqpoint{4.684816in}{2.440791in}}%
\pgfpathlineto{\pgfqpoint{4.686964in}{2.454829in}}%
\pgfpathlineto{\pgfqpoint{4.688038in}{2.451627in}}%
\pgfpathlineto{\pgfqpoint{4.689112in}{2.450396in}}%
\pgfpathlineto{\pgfqpoint{4.693407in}{2.442187in}}%
\pgfpathlineto{\pgfqpoint{4.694481in}{2.442351in}}%
\pgfpathlineto{\pgfqpoint{4.695555in}{2.441530in}}%
\pgfpathlineto{\pgfqpoint{4.696629in}{2.462955in}}%
\pgfpathlineto{\pgfqpoint{4.700924in}{2.456060in}}%
\pgfpathlineto{\pgfqpoint{4.701998in}{2.442843in}}%
\pgfpathlineto{\pgfqpoint{4.703072in}{2.444157in}}%
\pgfpathlineto{\pgfqpoint{4.704146in}{2.428478in}}%
\pgfpathlineto{\pgfqpoint{4.707367in}{2.434470in}}%
\pgfpathlineto{\pgfqpoint{4.708441in}{2.432582in}}%
\pgfpathlineto{\pgfqpoint{4.709515in}{2.427739in}}%
\pgfpathlineto{\pgfqpoint{4.710589in}{2.428888in}}%
\pgfpathlineto{\pgfqpoint{4.711662in}{2.435127in}}%
\pgfpathlineto{\pgfqpoint{4.715958in}{2.437590in}}%
\pgfpathlineto{\pgfqpoint{4.717032in}{2.434142in}}%
\pgfpathlineto{\pgfqpoint{4.718105in}{2.439724in}}%
\pgfpathlineto{\pgfqpoint{4.719179in}{2.437015in}}%
\pgfpathlineto{\pgfqpoint{4.722401in}{2.443254in}}%
\pgfpathlineto{\pgfqpoint{4.723475in}{2.443828in}}%
\pgfpathlineto{\pgfqpoint{4.724549in}{2.434142in}}%
\pgfpathlineto{\pgfqpoint{4.725622in}{2.418299in}}%
\pgfpathlineto{\pgfqpoint{4.726696in}{2.432993in}}%
\pgfpathlineto{\pgfqpoint{4.729918in}{2.427246in}}%
\pgfpathlineto{\pgfqpoint{4.730992in}{2.427985in}}%
\pgfpathlineto{\pgfqpoint{4.732065in}{2.432746in}}%
\pgfpathlineto{\pgfqpoint{4.733139in}{2.434552in}}%
\pgfpathlineto{\pgfqpoint{4.734213in}{2.429955in}}%
\pgfpathlineto{\pgfqpoint{4.739582in}{2.440627in}}%
\pgfpathlineto{\pgfqpoint{4.741730in}{2.439067in}}%
\pgfpathlineto{\pgfqpoint{4.744951in}{2.441940in}}%
\pgfpathlineto{\pgfqpoint{4.746025in}{2.447358in}}%
\pgfpathlineto{\pgfqpoint{4.747099in}{2.448918in}}%
\pgfpathlineto{\pgfqpoint{4.749247in}{2.460739in}}%
\pgfpathlineto{\pgfqpoint{4.752468in}{2.460411in}}%
\pgfpathlineto{\pgfqpoint{4.754616in}{2.454911in}}%
\pgfpathlineto{\pgfqpoint{4.755690in}{2.456306in}}%
\pgfpathlineto{\pgfqpoint{4.756764in}{2.463612in}}%
\pgfpathlineto{\pgfqpoint{4.759985in}{2.462627in}}%
\pgfpathlineto{\pgfqpoint{4.761059in}{2.460739in}}%
\pgfpathlineto{\pgfqpoint{4.762133in}{2.456142in}}%
\pgfpathlineto{\pgfqpoint{4.763207in}{2.457127in}}%
\pgfpathlineto{\pgfqpoint{4.764281in}{2.456963in}}%
\pgfpathlineto{\pgfqpoint{4.767502in}{2.454911in}}%
\pgfpathlineto{\pgfqpoint{4.768576in}{2.457291in}}%
\pgfpathlineto{\pgfqpoint{4.769650in}{2.456306in}}%
\pgfpathlineto{\pgfqpoint{4.770724in}{2.465418in}}%
\pgfpathlineto{\pgfqpoint{4.771797in}{2.463038in}}%
\pgfpathlineto{\pgfqpoint{4.775019in}{2.463776in}}%
\pgfpathlineto{\pgfqpoint{4.777167in}{2.467635in}}%
\pgfpathlineto{\pgfqpoint{4.778240in}{2.469030in}}%
\pgfpathlineto{\pgfqpoint{4.779314in}{2.465418in}}%
\pgfpathlineto{\pgfqpoint{4.783610in}{2.465582in}}%
\pgfpathlineto{\pgfqpoint{4.784683in}{2.462791in}}%
\pgfpathlineto{\pgfqpoint{4.786831in}{2.453597in}}%
\pgfpathlineto{\pgfqpoint{4.792200in}{2.450067in}}%
\pgfpathlineto{\pgfqpoint{4.794348in}{2.454336in}}%
\pgfpathlineto{\pgfqpoint{4.798643in}{2.440545in}}%
\pgfpathlineto{\pgfqpoint{4.799717in}{2.432911in}}%
\pgfpathlineto{\pgfqpoint{4.801865in}{2.439970in}}%
\pgfpathlineto{\pgfqpoint{4.806160in}{2.439067in}}%
\pgfpathlineto{\pgfqpoint{4.808308in}{2.434881in}}%
\pgfpathlineto{\pgfqpoint{4.809382in}{2.434963in}}%
\pgfpathlineto{\pgfqpoint{4.812603in}{2.441858in}}%
\pgfpathlineto{\pgfqpoint{4.813677in}{2.440381in}}%
\pgfpathlineto{\pgfqpoint{4.814751in}{2.442597in}}%
\pgfpathlineto{\pgfqpoint{4.815825in}{2.442515in}}%
\pgfpathlineto{\pgfqpoint{4.816899in}{2.447851in}}%
\pgfpathlineto{\pgfqpoint{4.820120in}{2.455403in}}%
\pgfpathlineto{\pgfqpoint{4.821194in}{2.453187in}}%
\pgfpathlineto{\pgfqpoint{4.822268in}{2.456306in}}%
\pgfpathlineto{\pgfqpoint{4.823342in}{2.455157in}}%
\pgfpathlineto{\pgfqpoint{4.824415in}{2.450149in}}%
\pgfpathlineto{\pgfqpoint{4.827637in}{2.449411in}}%
\pgfpathlineto{\pgfqpoint{4.829785in}{2.438821in}}%
\pgfpathlineto{\pgfqpoint{4.830859in}{2.441037in}}%
\pgfpathlineto{\pgfqpoint{4.831932in}{2.435373in}}%
\pgfpathlineto{\pgfqpoint{4.835154in}{2.424373in}}%
\pgfpathlineto{\pgfqpoint{4.836228in}{2.428231in}}%
\pgfpathlineto{\pgfqpoint{4.837302in}{2.426672in}}%
\pgfpathlineto{\pgfqpoint{4.838375in}{2.426097in}}%
\pgfpathlineto{\pgfqpoint{4.839449in}{2.427328in}}%
\pgfpathlineto{\pgfqpoint{4.842671in}{2.424784in}}%
\pgfpathlineto{\pgfqpoint{4.846966in}{2.443993in}}%
\pgfpathlineto{\pgfqpoint{4.850188in}{2.445717in}}%
\pgfpathlineto{\pgfqpoint{4.851261in}{2.438000in}}%
\pgfpathlineto{\pgfqpoint{4.853409in}{2.454582in}}%
\pgfpathlineto{\pgfqpoint{4.854483in}{2.454500in}}%
\pgfpathlineto{\pgfqpoint{4.857704in}{2.451545in}}%
\pgfpathlineto{\pgfqpoint{4.858778in}{2.461149in}}%
\pgfpathlineto{\pgfqpoint{4.859852in}{2.464761in}}%
\pgfpathlineto{\pgfqpoint{4.860926in}{2.462709in}}%
\pgfpathlineto{\pgfqpoint{4.862000in}{2.458523in}}%
\pgfpathlineto{\pgfqpoint{4.865221in}{2.467470in}}%
\pgfpathlineto{\pgfqpoint{4.866295in}{2.475351in}}%
\pgfpathlineto{\pgfqpoint{4.868443in}{2.463694in}}%
\pgfpathlineto{\pgfqpoint{4.869517in}{2.466403in}}%
\pgfpathlineto{\pgfqpoint{4.873812in}{2.468291in}}%
\pgfpathlineto{\pgfqpoint{4.874886in}{2.477732in}}%
\pgfpathlineto{\pgfqpoint{4.875960in}{2.474448in}}%
\pgfpathlineto{\pgfqpoint{4.877034in}{2.475679in}}%
\pgfpathlineto{\pgfqpoint{4.880255in}{2.474038in}}%
\pgfpathlineto{\pgfqpoint{4.882403in}{2.482493in}}%
\pgfpathlineto{\pgfqpoint{4.884550in}{2.491687in}}%
\pgfpathlineto{\pgfqpoint{4.887772in}{2.489799in}}%
\pgfpathlineto{\pgfqpoint{4.888846in}{2.488239in}}%
\pgfpathlineto{\pgfqpoint{4.889920in}{2.490620in}}%
\pgfpathlineto{\pgfqpoint{4.890993in}{2.490456in}}%
\pgfpathlineto{\pgfqpoint{4.892067in}{2.492426in}}%
\pgfpathlineto{\pgfqpoint{4.895289in}{2.495545in}}%
\pgfpathlineto{\pgfqpoint{4.896363in}{2.490291in}}%
\pgfpathlineto{\pgfqpoint{4.897437in}{2.487829in}}%
\pgfpathlineto{\pgfqpoint{4.899584in}{2.487418in}}%
\pgfpathlineto{\pgfqpoint{4.902806in}{2.481426in}}%
\pgfpathlineto{\pgfqpoint{4.903880in}{2.484709in}}%
\pgfpathlineto{\pgfqpoint{4.906027in}{2.476911in}}%
\pgfpathlineto{\pgfqpoint{4.907101in}{2.489470in}}%
\pgfpathlineto{\pgfqpoint{4.910323in}{2.491276in}}%
\pgfpathlineto{\pgfqpoint{4.911396in}{2.483478in}}%
\pgfpathlineto{\pgfqpoint{4.912470in}{2.486023in}}%
\pgfpathlineto{\pgfqpoint{4.913544in}{2.474284in}}%
\pgfpathlineto{\pgfqpoint{4.914618in}{2.474776in}}%
\pgfpathlineto{\pgfqpoint{4.917839in}{2.470097in}}%
\pgfpathlineto{\pgfqpoint{4.918913in}{2.464761in}}%
\pgfpathlineto{\pgfqpoint{4.919987in}{2.474858in}}%
\pgfpathlineto{\pgfqpoint{4.921061in}{2.471575in}}%
\pgfpathlineto{\pgfqpoint{4.922135in}{2.471164in}}%
\pgfpathlineto{\pgfqpoint{4.925356in}{2.467470in}}%
\pgfpathlineto{\pgfqpoint{4.926430in}{2.467470in}}%
\pgfpathlineto{\pgfqpoint{4.929652in}{2.472314in}}%
\pgfpathlineto{\pgfqpoint{4.932873in}{2.472232in}}%
\pgfpathlineto{\pgfqpoint{4.936095in}{2.462217in}}%
\pgfpathlineto{\pgfqpoint{4.937169in}{2.461642in}}%
\pgfpathlineto{\pgfqpoint{4.940390in}{2.462955in}}%
\pgfpathlineto{\pgfqpoint{4.941464in}{2.468291in}}%
\pgfpathlineto{\pgfqpoint{4.942538in}{2.460164in}}%
\pgfpathlineto{\pgfqpoint{4.943612in}{2.461396in}}%
\pgfpathlineto{\pgfqpoint{4.947907in}{2.458605in}}%
\pgfpathlineto{\pgfqpoint{4.948981in}{2.464515in}}%
\pgfpathlineto{\pgfqpoint{4.950055in}{2.463941in}}%
\pgfpathlineto{\pgfqpoint{4.951128in}{2.462381in}}%
\pgfpathlineto{\pgfqpoint{4.952202in}{2.457702in}}%
\pgfpathlineto{\pgfqpoint{4.956498in}{2.459754in}}%
\pgfpathlineto{\pgfqpoint{4.957571in}{2.457373in}}%
\pgfpathlineto{\pgfqpoint{4.958645in}{2.450560in}}%
\pgfpathlineto{\pgfqpoint{4.959719in}{2.457045in}}%
\pgfpathlineto{\pgfqpoint{4.962941in}{2.452120in}}%
\pgfpathlineto{\pgfqpoint{4.964014in}{2.456552in}}%
\pgfpathlineto{\pgfqpoint{4.966162in}{2.441120in}}%
\pgfpathlineto{\pgfqpoint{4.967236in}{2.439396in}}%
\pgfpathlineto{\pgfqpoint{4.970458in}{2.433075in}}%
\pgfpathlineto{\pgfqpoint{4.974753in}{2.444157in}}%
\pgfpathlineto{\pgfqpoint{4.977974in}{2.449082in}}%
\pgfpathlineto{\pgfqpoint{4.979048in}{2.454664in}}%
\pgfpathlineto{\pgfqpoint{4.980122in}{2.445388in}}%
\pgfpathlineto{\pgfqpoint{4.981196in}{2.447523in}}%
\pgfpathlineto{\pgfqpoint{4.982270in}{2.458523in}}%
\pgfpathlineto{\pgfqpoint{4.986565in}{2.448672in}}%
\pgfpathlineto{\pgfqpoint{4.987639in}{2.449903in}}%
\pgfpathlineto{\pgfqpoint{4.988713in}{2.448343in}}%
\pgfpathlineto{\pgfqpoint{4.989787in}{2.448672in}}%
\pgfpathlineto{\pgfqpoint{4.993008in}{2.447933in}}%
\pgfpathlineto{\pgfqpoint{4.994082in}{2.449739in}}%
\pgfpathlineto{\pgfqpoint{4.995156in}{2.447933in}}%
\pgfpathlineto{\pgfqpoint{4.997303in}{2.453433in}}%
\pgfpathlineto{\pgfqpoint{5.000525in}{2.445306in}}%
\pgfpathlineto{\pgfqpoint{5.001599in}{2.452037in}}%
\pgfpathlineto{\pgfqpoint{5.002673in}{2.447687in}}%
\pgfpathlineto{\pgfqpoint{5.004820in}{2.451545in}}%
\pgfpathlineto{\pgfqpoint{5.008042in}{2.452530in}}%
\pgfpathlineto{\pgfqpoint{5.010190in}{2.457455in}}%
\pgfpathlineto{\pgfqpoint{5.011263in}{2.457127in}}%
\pgfpathlineto{\pgfqpoint{5.012337in}{2.455649in}}%
\pgfpathlineto{\pgfqpoint{5.015559in}{2.461560in}}%
\pgfpathlineto{\pgfqpoint{5.016633in}{2.461067in}}%
\pgfpathlineto{\pgfqpoint{5.017706in}{2.454254in}}%
\pgfpathlineto{\pgfqpoint{5.019854in}{2.448097in}}%
\pgfpathlineto{\pgfqpoint{5.024149in}{2.461396in}}%
\pgfpathlineto{\pgfqpoint{5.025223in}{2.459261in}}%
\pgfpathlineto{\pgfqpoint{5.027371in}{2.461067in}}%
\pgfpathlineto{\pgfqpoint{5.030592in}{2.466814in}}%
\pgfpathlineto{\pgfqpoint{5.032740in}{2.463612in}}%
\pgfpathlineto{\pgfqpoint{5.033814in}{2.463366in}}%
\pgfpathlineto{\pgfqpoint{5.034888in}{2.461067in}}%
\pgfpathlineto{\pgfqpoint{5.038109in}{2.466075in}}%
\pgfpathlineto{\pgfqpoint{5.039183in}{2.471985in}}%
\pgfpathlineto{\pgfqpoint{5.040257in}{2.472970in}}%
\pgfpathlineto{\pgfqpoint{5.042405in}{2.467963in}}%
\pgfpathlineto{\pgfqpoint{5.046700in}{2.468455in}}%
\pgfpathlineto{\pgfqpoint{5.047774in}{2.474038in}}%
\pgfpathlineto{\pgfqpoint{5.048848in}{2.474776in}}%
\pgfpathlineto{\pgfqpoint{5.053143in}{2.473463in}}%
\pgfpathlineto{\pgfqpoint{5.055291in}{2.469605in}}%
\pgfpathlineto{\pgfqpoint{5.056365in}{2.475105in}}%
\pgfpathlineto{\pgfqpoint{5.057438in}{2.476911in}}%
\pgfpathlineto{\pgfqpoint{5.060660in}{2.487500in}}%
\pgfpathlineto{\pgfqpoint{5.061734in}{2.484053in}}%
\pgfpathlineto{\pgfqpoint{5.062808in}{2.484791in}}%
\pgfpathlineto{\pgfqpoint{5.063881in}{2.483314in}}%
\pgfpathlineto{\pgfqpoint{5.064955in}{2.480441in}}%
\pgfpathlineto{\pgfqpoint{5.068177in}{2.478881in}}%
\pgfpathlineto{\pgfqpoint{5.069251in}{2.474941in}}%
\pgfpathlineto{\pgfqpoint{5.070325in}{2.481590in}}%
\pgfpathlineto{\pgfqpoint{5.071398in}{2.482082in}}%
\pgfpathlineto{\pgfqpoint{5.072472in}{2.483724in}}%
\pgfpathlineto{\pgfqpoint{5.077841in}{2.475187in}}%
\pgfpathlineto{\pgfqpoint{5.078915in}{2.470754in}}%
\pgfpathlineto{\pgfqpoint{5.079989in}{2.469358in}}%
\pgfpathlineto{\pgfqpoint{5.083211in}{2.466567in}}%
\pgfpathlineto{\pgfqpoint{5.085358in}{2.470508in}}%
\pgfpathlineto{\pgfqpoint{5.086432in}{2.471575in}}%
\pgfpathlineto{\pgfqpoint{5.091801in}{2.467635in}}%
\pgfpathlineto{\pgfqpoint{5.092875in}{2.467799in}}%
\pgfpathlineto{\pgfqpoint{5.095023in}{2.472560in}}%
\pgfpathlineto{\pgfqpoint{5.098244in}{2.469605in}}%
\pgfpathlineto{\pgfqpoint{5.099318in}{2.466567in}}%
\pgfpathlineto{\pgfqpoint{5.100392in}{2.466157in}}%
\pgfpathlineto{\pgfqpoint{5.101466in}{2.467881in}}%
\pgfpathlineto{\pgfqpoint{5.102540in}{2.467306in}}%
\pgfpathlineto{\pgfqpoint{5.107909in}{2.467470in}}%
\pgfpathlineto{\pgfqpoint{5.108983in}{2.465418in}}%
\pgfpathlineto{\pgfqpoint{5.110057in}{2.464597in}}%
\pgfpathlineto{\pgfqpoint{5.113278in}{2.464926in}}%
\pgfpathlineto{\pgfqpoint{5.114352in}{2.464187in}}%
\pgfpathlineto{\pgfqpoint{5.115426in}{2.465664in}}%
\pgfpathlineto{\pgfqpoint{5.116500in}{2.468784in}}%
\pgfpathlineto{\pgfqpoint{5.117573in}{2.466567in}}%
\pgfpathlineto{\pgfqpoint{5.120795in}{2.463776in}}%
\pgfpathlineto{\pgfqpoint{5.121869in}{2.460246in}}%
\pgfpathlineto{\pgfqpoint{5.122943in}{2.462381in}}%
\pgfpathlineto{\pgfqpoint{5.124016in}{2.457537in}}%
\pgfpathlineto{\pgfqpoint{5.125090in}{2.460246in}}%
\pgfpathlineto{\pgfqpoint{5.128312in}{2.457291in}}%
\pgfpathlineto{\pgfqpoint{5.129386in}{2.462052in}}%
\pgfpathlineto{\pgfqpoint{5.131533in}{2.466978in}}%
\pgfpathlineto{\pgfqpoint{5.135829in}{2.468373in}}%
\pgfpathlineto{\pgfqpoint{5.136903in}{2.471000in}}%
\pgfpathlineto{\pgfqpoint{5.137976in}{2.478717in}}%
\pgfpathlineto{\pgfqpoint{5.139050in}{2.478306in}}%
\pgfpathlineto{\pgfqpoint{5.140124in}{2.475187in}}%
\pgfpathlineto{\pgfqpoint{5.143346in}{2.476254in}}%
\pgfpathlineto{\pgfqpoint{5.144419in}{2.475844in}}%
\pgfpathlineto{\pgfqpoint{5.145493in}{2.478142in}}%
\pgfpathlineto{\pgfqpoint{5.146567in}{2.474366in}}%
\pgfpathlineto{\pgfqpoint{5.147641in}{2.475351in}}%
\pgfpathlineto{\pgfqpoint{5.151936in}{2.472232in}}%
\pgfpathlineto{\pgfqpoint{5.153010in}{2.471493in}}%
\pgfpathlineto{\pgfqpoint{5.154084in}{2.473381in}}%
\pgfpathlineto{\pgfqpoint{5.155158in}{2.468455in}}%
\pgfpathlineto{\pgfqpoint{5.158379in}{2.466321in}}%
\pgfpathlineto{\pgfqpoint{5.161601in}{2.452776in}}%
\pgfpathlineto{\pgfqpoint{5.162675in}{2.469112in}}%
\pgfpathlineto{\pgfqpoint{5.165896in}{2.466075in}}%
\pgfpathlineto{\pgfqpoint{5.166970in}{2.467388in}}%
\pgfpathlineto{\pgfqpoint{5.168044in}{2.479127in}}%
\pgfpathlineto{\pgfqpoint{5.169118in}{2.472478in}}%
\pgfpathlineto{\pgfqpoint{5.170191in}{2.478142in}}%
\pgfpathlineto{\pgfqpoint{5.173413in}{2.481918in}}%
\pgfpathlineto{\pgfqpoint{5.174487in}{2.481508in}}%
\pgfpathlineto{\pgfqpoint{5.175561in}{2.481918in}}%
\pgfpathlineto{\pgfqpoint{5.176635in}{2.485202in}}%
\pgfpathlineto{\pgfqpoint{5.177708in}{2.483970in}}%
\pgfpathlineto{\pgfqpoint{5.183078in}{2.491194in}}%
\pgfpathlineto{\pgfqpoint{5.184151in}{2.494478in}}%
\pgfpathlineto{\pgfqpoint{5.185225in}{2.495299in}}%
\pgfpathlineto{\pgfqpoint{5.188447in}{2.496284in}}%
\pgfpathlineto{\pgfqpoint{5.191668in}{2.492754in}}%
\pgfpathlineto{\pgfqpoint{5.192742in}{2.493329in}}%
\pgfpathlineto{\pgfqpoint{5.198111in}{2.491851in}}%
\pgfpathlineto{\pgfqpoint{5.199185in}{2.497433in}}%
\pgfpathlineto{\pgfqpoint{5.200259in}{2.497926in}}%
\pgfpathlineto{\pgfqpoint{5.203480in}{2.494806in}}%
\pgfpathlineto{\pgfqpoint{5.204554in}{2.492426in}}%
\pgfpathlineto{\pgfqpoint{5.205628in}{2.497023in}}%
\pgfpathlineto{\pgfqpoint{5.207776in}{2.494314in}}%
\pgfpathlineto{\pgfqpoint{5.213145in}{2.500635in}}%
\pgfpathlineto{\pgfqpoint{5.214219in}{2.500881in}}%
\pgfpathlineto{\pgfqpoint{5.215293in}{2.502851in}}%
\pgfpathlineto{\pgfqpoint{5.218514in}{2.505232in}}%
\pgfpathlineto{\pgfqpoint{5.219588in}{2.502605in}}%
\pgfpathlineto{\pgfqpoint{5.220662in}{2.508187in}}%
\pgfpathlineto{\pgfqpoint{5.221736in}{2.501866in}}%
\pgfpathlineto{\pgfqpoint{5.222810in}{2.503590in}}%
\pgfpathlineto{\pgfqpoint{5.226031in}{2.502605in}}%
\pgfpathlineto{\pgfqpoint{5.228179in}{2.493493in}}%
\pgfpathlineto{\pgfqpoint{5.229253in}{2.492918in}}%
\pgfpathlineto{\pgfqpoint{5.230326in}{2.498090in}}%
\pgfpathlineto{\pgfqpoint{5.233548in}{2.496530in}}%
\pgfpathlineto{\pgfqpoint{5.234622in}{2.493657in}}%
\pgfpathlineto{\pgfqpoint{5.235696in}{2.500799in}}%
\pgfpathlineto{\pgfqpoint{5.236769in}{2.497187in}}%
\pgfpathlineto{\pgfqpoint{5.237843in}{2.504329in}}%
\pgfpathlineto{\pgfqpoint{5.241065in}{2.495135in}}%
\pgfpathlineto{\pgfqpoint{5.242139in}{2.496366in}}%
\pgfpathlineto{\pgfqpoint{5.244286in}{2.487172in}}%
\pgfpathlineto{\pgfqpoint{5.245360in}{2.494396in}}%
\pgfpathlineto{\pgfqpoint{5.250729in}{2.505642in}}%
\pgfpathlineto{\pgfqpoint{5.251803in}{2.498582in}}%
\pgfpathlineto{\pgfqpoint{5.252877in}{2.512209in}}%
\pgfpathlineto{\pgfqpoint{5.256099in}{2.517791in}}%
\pgfpathlineto{\pgfqpoint{5.257172in}{2.521486in}}%
\pgfpathlineto{\pgfqpoint{5.258246in}{2.521978in}}%
\pgfpathlineto{\pgfqpoint{5.260394in}{2.527150in}}%
\pgfpathlineto{\pgfqpoint{5.263615in}{2.527889in}}%
\pgfpathlineto{\pgfqpoint{5.264689in}{2.536754in}}%
\pgfpathlineto{\pgfqpoint{5.265763in}{2.539299in}}%
\pgfpathlineto{\pgfqpoint{5.266837in}{2.538642in}}%
\pgfpathlineto{\pgfqpoint{5.267911in}{2.540284in}}%
\pgfpathlineto{\pgfqpoint{5.271132in}{2.542501in}}%
\pgfpathlineto{\pgfqpoint{5.272206in}{2.544060in}}%
\pgfpathlineto{\pgfqpoint{5.273280in}{2.542747in}}%
\pgfpathlineto{\pgfqpoint{5.275428in}{2.533060in}}%
\pgfpathlineto{\pgfqpoint{5.278649in}{2.531172in}}%
\pgfpathlineto{\pgfqpoint{5.279723in}{2.531911in}}%
\pgfpathlineto{\pgfqpoint{5.280797in}{2.537411in}}%
\pgfpathlineto{\pgfqpoint{5.281871in}{2.535605in}}%
\pgfpathlineto{\pgfqpoint{5.282945in}{2.536508in}}%
\pgfpathlineto{\pgfqpoint{5.286166in}{2.533060in}}%
\pgfpathlineto{\pgfqpoint{5.287240in}{2.537903in}}%
\pgfpathlineto{\pgfqpoint{5.288314in}{2.538478in}}%
\pgfpathlineto{\pgfqpoint{5.290461in}{2.549396in}}%
\pgfpathlineto{\pgfqpoint{5.293683in}{2.546933in}}%
\pgfpathlineto{\pgfqpoint{5.294757in}{2.553911in}}%
\pgfpathlineto{\pgfqpoint{5.295831in}{2.546359in}}%
\pgfpathlineto{\pgfqpoint{5.296904in}{2.550463in}}%
\pgfpathlineto{\pgfqpoint{5.297978in}{2.549068in}}%
\pgfpathlineto{\pgfqpoint{5.301200in}{2.551777in}}%
\pgfpathlineto{\pgfqpoint{5.302274in}{2.551366in}}%
\pgfpathlineto{\pgfqpoint{5.303347in}{2.546359in}}%
\pgfpathlineto{\pgfqpoint{5.304421in}{2.549314in}}%
\pgfpathlineto{\pgfqpoint{5.305495in}{2.543239in}}%
\pgfpathlineto{\pgfqpoint{5.308717in}{2.540777in}}%
\pgfpathlineto{\pgfqpoint{5.309791in}{2.541844in}}%
\pgfpathlineto{\pgfqpoint{5.311938in}{2.560478in}}%
\pgfpathlineto{\pgfqpoint{5.313012in}{2.560889in}}%
\pgfpathlineto{\pgfqpoint{5.316234in}{2.564747in}}%
\pgfpathlineto{\pgfqpoint{5.317307in}{2.569590in}}%
\pgfpathlineto{\pgfqpoint{5.318381in}{2.568523in}}%
\pgfpathlineto{\pgfqpoint{5.320529in}{2.570822in}}%
\pgfpathlineto{\pgfqpoint{5.324824in}{2.563351in}}%
\pgfpathlineto{\pgfqpoint{5.325898in}{2.554075in}}%
\pgfpathlineto{\pgfqpoint{5.328046in}{2.549478in}}%
\pgfpathlineto{\pgfqpoint{5.331267in}{2.546441in}}%
\pgfpathlineto{\pgfqpoint{5.332341in}{2.543568in}}%
\pgfpathlineto{\pgfqpoint{5.333415in}{2.546851in}}%
\pgfpathlineto{\pgfqpoint{5.334489in}{2.554157in}}%
\pgfpathlineto{\pgfqpoint{5.335563in}{2.548165in}}%
\pgfpathlineto{\pgfqpoint{5.338784in}{2.545784in}}%
\pgfpathlineto{\pgfqpoint{5.339858in}{2.548493in}}%
\pgfpathlineto{\pgfqpoint{5.340932in}{2.546277in}}%
\pgfpathlineto{\pgfqpoint{5.342006in}{2.545374in}}%
\pgfpathlineto{\pgfqpoint{5.343079in}{2.555224in}}%
\pgfpathlineto{\pgfqpoint{5.347375in}{2.554814in}}%
\pgfpathlineto{\pgfqpoint{5.348449in}{2.556045in}}%
\pgfpathlineto{\pgfqpoint{5.349523in}{2.562448in}}%
\pgfpathlineto{\pgfqpoint{5.350596in}{2.551448in}}%
\pgfpathlineto{\pgfqpoint{5.353818in}{2.547918in}}%
\pgfpathlineto{\pgfqpoint{5.354892in}{2.525918in}}%
\pgfpathlineto{\pgfqpoint{5.355966in}{2.516396in}}%
\pgfpathlineto{\pgfqpoint{5.357039in}{2.520090in}}%
\pgfpathlineto{\pgfqpoint{5.358113in}{2.510321in}}%
\pgfpathlineto{\pgfqpoint{5.361335in}{2.516232in}}%
\pgfpathlineto{\pgfqpoint{5.362409in}{2.522142in}}%
\pgfpathlineto{\pgfqpoint{5.363482in}{2.520993in}}%
\pgfpathlineto{\pgfqpoint{5.364556in}{2.527478in}}%
\pgfpathlineto{\pgfqpoint{5.365630in}{2.519680in}}%
\pgfpathlineto{\pgfqpoint{5.368852in}{2.515739in}}%
\pgfpathlineto{\pgfqpoint{5.372073in}{2.522717in}}%
\pgfpathlineto{\pgfqpoint{5.373147in}{2.521732in}}%
\pgfpathlineto{\pgfqpoint{5.377442in}{2.518859in}}%
\pgfpathlineto{\pgfqpoint{5.378516in}{2.524359in}}%
\pgfpathlineto{\pgfqpoint{5.379590in}{2.516888in}}%
\pgfpathlineto{\pgfqpoint{5.380664in}{2.514426in}}%
\pgfpathlineto{\pgfqpoint{5.384959in}{2.518941in}}%
\pgfpathlineto{\pgfqpoint{5.386033in}{2.518612in}}%
\pgfpathlineto{\pgfqpoint{5.387107in}{2.516560in}}%
\pgfpathlineto{\pgfqpoint{5.388181in}{2.516314in}}%
\pgfpathlineto{\pgfqpoint{5.391402in}{2.518284in}}%
\pgfpathlineto{\pgfqpoint{5.392476in}{2.516478in}}%
\pgfpathlineto{\pgfqpoint{5.393550in}{2.510732in}}%
\pgfpathlineto{\pgfqpoint{5.394624in}{2.512702in}}%
\pgfpathlineto{\pgfqpoint{5.395698in}{2.498747in}}%
\pgfpathlineto{\pgfqpoint{5.398919in}{2.501784in}}%
\pgfpathlineto{\pgfqpoint{5.399993in}{2.490784in}}%
\pgfpathlineto{\pgfqpoint{5.401067in}{2.489717in}}%
\pgfpathlineto{\pgfqpoint{5.402141in}{2.494724in}}%
\pgfpathlineto{\pgfqpoint{5.403214in}{2.492836in}}%
\pgfpathlineto{\pgfqpoint{5.406436in}{2.505150in}}%
\pgfpathlineto{\pgfqpoint{5.407510in}{2.500060in}}%
\pgfpathlineto{\pgfqpoint{5.408584in}{2.506463in}}%
\pgfpathlineto{\pgfqpoint{5.409657in}{2.503836in}}%
\pgfpathlineto{\pgfqpoint{5.410731in}{2.513523in}}%
\pgfpathlineto{\pgfqpoint{5.413953in}{2.514344in}}%
\pgfpathlineto{\pgfqpoint{5.417174in}{2.495135in}}%
\pgfpathlineto{\pgfqpoint{5.421470in}{2.499157in}}%
\pgfpathlineto{\pgfqpoint{5.422544in}{2.493657in}}%
\pgfpathlineto{\pgfqpoint{5.423617in}{2.496284in}}%
\pgfpathlineto{\pgfqpoint{5.424691in}{2.497105in}}%
\pgfpathlineto{\pgfqpoint{5.428987in}{2.501456in}}%
\pgfpathlineto{\pgfqpoint{5.430060in}{2.496776in}}%
\pgfpathlineto{\pgfqpoint{5.431134in}{2.499485in}}%
\pgfpathlineto{\pgfqpoint{5.432208in}{2.500388in}}%
\pgfpathlineto{\pgfqpoint{5.433282in}{2.503672in}}%
\pgfpathlineto{\pgfqpoint{5.436503in}{2.504247in}}%
\pgfpathlineto{\pgfqpoint{5.437577in}{2.505396in}}%
\pgfpathlineto{\pgfqpoint{5.438651in}{2.504739in}}%
\pgfpathlineto{\pgfqpoint{5.439725in}{2.504739in}}%
\pgfpathlineto{\pgfqpoint{5.440799in}{2.497844in}}%
\pgfpathlineto{\pgfqpoint{5.446168in}{2.501784in}}%
\pgfpathlineto{\pgfqpoint{5.447242in}{2.491194in}}%
\pgfpathlineto{\pgfqpoint{5.448316in}{2.491605in}}%
\pgfpathlineto{\pgfqpoint{5.452611in}{2.487418in}}%
\pgfpathlineto{\pgfqpoint{5.454759in}{2.480933in}}%
\pgfpathlineto{\pgfqpoint{5.455833in}{2.486515in}}%
\pgfpathlineto{\pgfqpoint{5.459054in}{2.486926in}}%
\pgfpathlineto{\pgfqpoint{5.460128in}{2.484873in}}%
\pgfpathlineto{\pgfqpoint{5.461202in}{2.487254in}}%
\pgfpathlineto{\pgfqpoint{5.462276in}{2.485859in}}%
\pgfpathlineto{\pgfqpoint{5.463349in}{2.491276in}}%
\pgfpathlineto{\pgfqpoint{5.468719in}{2.482247in}}%
\pgfpathlineto{\pgfqpoint{5.470866in}{2.491933in}}%
\pgfpathlineto{\pgfqpoint{5.474088in}{2.489717in}}%
\pgfpathlineto{\pgfqpoint{5.475162in}{2.490373in}}%
\pgfpathlineto{\pgfqpoint{5.476235in}{2.487829in}}%
\pgfpathlineto{\pgfqpoint{5.477309in}{2.487336in}}%
\pgfpathlineto{\pgfqpoint{5.478383in}{2.484053in}}%
\pgfpathlineto{\pgfqpoint{5.482679in}{2.478224in}}%
\pgfpathlineto{\pgfqpoint{5.483752in}{2.480030in}}%
\pgfpathlineto{\pgfqpoint{5.484826in}{2.479620in}}%
\pgfpathlineto{\pgfqpoint{5.485900in}{2.472888in}}%
\pgfpathlineto{\pgfqpoint{5.489122in}{2.476172in}}%
\pgfpathlineto{\pgfqpoint{5.490195in}{2.473955in}}%
\pgfpathlineto{\pgfqpoint{5.491269in}{2.474120in}}%
\pgfpathlineto{\pgfqpoint{5.492343in}{2.471164in}}%
\pgfpathlineto{\pgfqpoint{5.493417in}{2.465993in}}%
\pgfpathlineto{\pgfqpoint{5.496638in}{2.467963in}}%
\pgfpathlineto{\pgfqpoint{5.498786in}{2.481097in}}%
\pgfpathlineto{\pgfqpoint{5.499860in}{2.480194in}}%
\pgfpathlineto{\pgfqpoint{5.500934in}{2.476336in}}%
\pgfpathlineto{\pgfqpoint{5.504155in}{2.470918in}}%
\pgfpathlineto{\pgfqpoint{5.507377in}{2.490291in}}%
\pgfpathlineto{\pgfqpoint{5.508451in}{2.488239in}}%
\pgfpathlineto{\pgfqpoint{5.511672in}{2.487664in}}%
\pgfpathlineto{\pgfqpoint{5.512746in}{2.482903in}}%
\pgfpathlineto{\pgfqpoint{5.514894in}{2.480030in}}%
\pgfpathlineto{\pgfqpoint{5.515967in}{2.479702in}}%
\pgfpathlineto{\pgfqpoint{5.519189in}{2.472396in}}%
\pgfpathlineto{\pgfqpoint{5.520263in}{2.471821in}}%
\pgfpathlineto{\pgfqpoint{5.521337in}{2.482411in}}%
\pgfpathlineto{\pgfqpoint{5.522411in}{2.483888in}}%
\pgfpathlineto{\pgfqpoint{5.526706in}{2.484791in}}%
\pgfpathlineto{\pgfqpoint{5.527780in}{2.496694in}}%
\pgfpathlineto{\pgfqpoint{5.528854in}{2.491523in}}%
\pgfpathlineto{\pgfqpoint{5.529927in}{2.489142in}}%
\pgfpathlineto{\pgfqpoint{5.535297in}{2.498993in}}%
\pgfpathlineto{\pgfqpoint{5.537444in}{2.500881in}}%
\pgfpathlineto{\pgfqpoint{5.538518in}{2.500471in}}%
\pgfpathlineto{\pgfqpoint{5.541740in}{2.500060in}}%
\pgfpathlineto{\pgfqpoint{5.542813in}{2.496120in}}%
\pgfpathlineto{\pgfqpoint{5.544961in}{2.494150in}}%
\pgfpathlineto{\pgfqpoint{5.546035in}{2.491194in}}%
\pgfpathlineto{\pgfqpoint{5.549256in}{2.488896in}}%
\pgfpathlineto{\pgfqpoint{5.551404in}{2.493575in}}%
\pgfpathlineto{\pgfqpoint{5.552478in}{2.470261in}}%
\pgfpathlineto{\pgfqpoint{5.553552in}{2.465254in}}%
\pgfpathlineto{\pgfqpoint{5.556773in}{2.463120in}}%
\pgfpathlineto{\pgfqpoint{5.557847in}{2.459590in}}%
\pgfpathlineto{\pgfqpoint{5.558921in}{2.458523in}}%
\pgfpathlineto{\pgfqpoint{5.559995in}{2.458358in}}%
\pgfpathlineto{\pgfqpoint{5.561069in}{2.456470in}}%
\pgfpathlineto{\pgfqpoint{5.564290in}{2.462955in}}%
\pgfpathlineto{\pgfqpoint{5.565364in}{2.461806in}}%
\pgfpathlineto{\pgfqpoint{5.566438in}{2.463038in}}%
\pgfpathlineto{\pgfqpoint{5.567512in}{2.458605in}}%
\pgfpathlineto{\pgfqpoint{5.568586in}{2.457455in}}%
\pgfpathlineto{\pgfqpoint{5.571807in}{2.456799in}}%
\pgfpathlineto{\pgfqpoint{5.572881in}{2.453926in}}%
\pgfpathlineto{\pgfqpoint{5.573955in}{2.446620in}}%
\pgfpathlineto{\pgfqpoint{5.575029in}{2.445142in}}%
\pgfpathlineto{\pgfqpoint{5.576102in}{2.430202in}}%
\pgfpathlineto{\pgfqpoint{5.580398in}{2.405493in}}%
\pgfpathlineto{\pgfqpoint{5.581472in}{2.423388in}}%
\pgfpathlineto{\pgfqpoint{5.582545in}{2.427575in}}%
\pgfpathlineto{\pgfqpoint{5.583619in}{2.425605in}}%
\pgfpathlineto{\pgfqpoint{5.586841in}{2.421746in}}%
\pgfpathlineto{\pgfqpoint{5.587915in}{2.408940in}}%
\pgfpathlineto{\pgfqpoint{5.588989in}{2.415507in}}%
\pgfpathlineto{\pgfqpoint{5.590062in}{2.416328in}}%
\pgfpathlineto{\pgfqpoint{5.591136in}{2.407955in}}%
\pgfpathlineto{\pgfqpoint{5.595432in}{2.416739in}}%
\pgfpathlineto{\pgfqpoint{5.596505in}{2.405903in}}%
\pgfpathlineto{\pgfqpoint{5.597579in}{2.404754in}}%
\pgfpathlineto{\pgfqpoint{5.598653in}{2.405493in}}%
\pgfpathlineto{\pgfqpoint{5.601875in}{2.402866in}}%
\pgfpathlineto{\pgfqpoint{5.602948in}{2.412881in}}%
\pgfpathlineto{\pgfqpoint{5.604022in}{2.417560in}}%
\pgfpathlineto{\pgfqpoint{5.605096in}{2.418627in}}%
\pgfpathlineto{\pgfqpoint{5.606170in}{2.416410in}}%
\pgfpathlineto{\pgfqpoint{5.609391in}{2.421582in}}%
\pgfpathlineto{\pgfqpoint{5.610465in}{2.418216in}}%
\pgfpathlineto{\pgfqpoint{5.611539in}{2.418791in}}%
\pgfpathlineto{\pgfqpoint{5.613687in}{2.436194in}}%
\pgfpathlineto{\pgfqpoint{5.616908in}{2.429627in}}%
\pgfpathlineto{\pgfqpoint{5.617982in}{2.433321in}}%
\pgfpathlineto{\pgfqpoint{5.619056in}{2.430858in}}%
\pgfpathlineto{\pgfqpoint{5.620130in}{2.430940in}}%
\pgfpathlineto{\pgfqpoint{5.621204in}{2.434388in}}%
\pgfpathlineto{\pgfqpoint{5.626573in}{2.443746in}}%
\pgfpathlineto{\pgfqpoint{5.627647in}{2.448672in}}%
\pgfpathlineto{\pgfqpoint{5.628721in}{2.449246in}}%
\pgfpathlineto{\pgfqpoint{5.631942in}{2.448179in}}%
\pgfpathlineto{\pgfqpoint{5.633016in}{2.446537in}}%
\pgfpathlineto{\pgfqpoint{5.635164in}{2.447687in}}%
\pgfpathlineto{\pgfqpoint{5.636237in}{2.452284in}}%
\pgfpathlineto{\pgfqpoint{5.639459in}{2.454172in}}%
\pgfpathlineto{\pgfqpoint{5.640533in}{2.448836in}}%
\pgfpathlineto{\pgfqpoint{5.641607in}{2.447605in}}%
\pgfpathlineto{\pgfqpoint{5.642680in}{2.456799in}}%
\pgfpathlineto{\pgfqpoint{5.643754in}{2.472642in}}%
\pgfpathlineto{\pgfqpoint{5.646976in}{2.476008in}}%
\pgfpathlineto{\pgfqpoint{5.648050in}{2.474530in}}%
\pgfpathlineto{\pgfqpoint{5.649123in}{2.468866in}}%
\pgfpathlineto{\pgfqpoint{5.650197in}{2.472560in}}%
\pgfpathlineto{\pgfqpoint{5.651271in}{2.467881in}}%
\pgfpathlineto{\pgfqpoint{5.654493in}{2.469523in}}%
\pgfpathlineto{\pgfqpoint{5.655567in}{2.472806in}}%
\pgfpathlineto{\pgfqpoint{5.656640in}{2.472888in}}%
\pgfpathlineto{\pgfqpoint{5.658788in}{2.461970in}}%
\pgfpathlineto{\pgfqpoint{5.662010in}{2.460739in}}%
\pgfpathlineto{\pgfqpoint{5.664157in}{2.464843in}}%
\pgfpathlineto{\pgfqpoint{5.666305in}{2.450314in}}%
\pgfpathlineto{\pgfqpoint{5.669526in}{2.460000in}}%
\pgfpathlineto{\pgfqpoint{5.670600in}{2.458358in}}%
\pgfpathlineto{\pgfqpoint{5.671674in}{2.464433in}}%
\pgfpathlineto{\pgfqpoint{5.672748in}{2.466732in}}%
\pgfpathlineto{\pgfqpoint{5.673822in}{2.463858in}}%
\pgfpathlineto{\pgfqpoint{5.677043in}{2.464926in}}%
\pgfpathlineto{\pgfqpoint{5.678117in}{2.468455in}}%
\pgfpathlineto{\pgfqpoint{5.679191in}{2.464433in}}%
\pgfpathlineto{\pgfqpoint{5.681339in}{2.462955in}}%
\pgfpathlineto{\pgfqpoint{5.684560in}{2.456717in}}%
\pgfpathlineto{\pgfqpoint{5.685634in}{2.464679in}}%
\pgfpathlineto{\pgfqpoint{5.687782in}{2.463284in}}%
\pgfpathlineto{\pgfqpoint{5.688856in}{2.478470in}}%
\pgfpathlineto{\pgfqpoint{5.692077in}{2.482411in}}%
\pgfpathlineto{\pgfqpoint{5.693151in}{2.477896in}}%
\pgfpathlineto{\pgfqpoint{5.694225in}{2.477567in}}%
\pgfpathlineto{\pgfqpoint{5.695299in}{2.478224in}}%
\pgfpathlineto{\pgfqpoint{5.696372in}{2.478142in}}%
\pgfpathlineto{\pgfqpoint{5.699594in}{2.481836in}}%
\pgfpathlineto{\pgfqpoint{5.701742in}{2.501538in}}%
\pgfpathlineto{\pgfqpoint{5.702815in}{2.496366in}}%
\pgfpathlineto{\pgfqpoint{5.703889in}{2.480687in}}%
\pgfpathlineto{\pgfqpoint{5.707111in}{2.486597in}}%
\pgfpathlineto{\pgfqpoint{5.709258in}{2.493739in}}%
\pgfpathlineto{\pgfqpoint{5.710332in}{2.492754in}}%
\pgfpathlineto{\pgfqpoint{5.714628in}{2.493739in}}%
\pgfpathlineto{\pgfqpoint{5.715701in}{2.496941in}}%
\pgfpathlineto{\pgfqpoint{5.716775in}{2.494806in}}%
\pgfpathlineto{\pgfqpoint{5.717849in}{2.489963in}}%
\pgfpathlineto{\pgfqpoint{5.722144in}{2.482411in}}%
\pgfpathlineto{\pgfqpoint{5.723218in}{2.484217in}}%
\pgfpathlineto{\pgfqpoint{5.726440in}{2.464926in}}%
\pgfpathlineto{\pgfqpoint{5.729661in}{2.470015in}}%
\pgfpathlineto{\pgfqpoint{5.730735in}{2.468866in}}%
\pgfpathlineto{\pgfqpoint{5.731809in}{2.464023in}}%
\pgfpathlineto{\pgfqpoint{5.732883in}{2.466239in}}%
\pgfpathlineto{\pgfqpoint{5.733957in}{2.457702in}}%
\pgfpathlineto{\pgfqpoint{5.738252in}{2.470426in}}%
\pgfpathlineto{\pgfqpoint{5.739326in}{2.468702in}}%
\pgfpathlineto{\pgfqpoint{5.741474in}{2.479948in}}%
\pgfpathlineto{\pgfqpoint{5.744695in}{2.476254in}}%
\pgfpathlineto{\pgfqpoint{5.745769in}{2.490620in}}%
\pgfpathlineto{\pgfqpoint{5.746843in}{2.490538in}}%
\pgfpathlineto{\pgfqpoint{5.747917in}{2.498008in}}%
\pgfpathlineto{\pgfqpoint{5.748990in}{2.511799in}}%
\pgfpathlineto{\pgfqpoint{5.752212in}{2.507612in}}%
\pgfpathlineto{\pgfqpoint{5.753286in}{2.500963in}}%
\pgfpathlineto{\pgfqpoint{5.754360in}{2.507448in}}%
\pgfpathlineto{\pgfqpoint{5.755433in}{2.504493in}}%
\pgfpathlineto{\pgfqpoint{5.759729in}{2.518612in}}%
\pgfpathlineto{\pgfqpoint{5.760803in}{2.518777in}}%
\pgfpathlineto{\pgfqpoint{5.761877in}{2.511306in}}%
\pgfpathlineto{\pgfqpoint{5.762950in}{2.498665in}}%
\pgfpathlineto{\pgfqpoint{5.764024in}{2.506627in}}%
\pgfpathlineto{\pgfqpoint{5.768320in}{2.510157in}}%
\pgfpathlineto{\pgfqpoint{5.769393in}{2.517381in}}%
\pgfpathlineto{\pgfqpoint{5.770467in}{2.513933in}}%
\pgfpathlineto{\pgfqpoint{5.771541in}{2.512538in}}%
\pgfpathlineto{\pgfqpoint{5.774763in}{2.515000in}}%
\pgfpathlineto{\pgfqpoint{5.776910in}{2.510814in}}%
\pgfpathlineto{\pgfqpoint{5.777984in}{2.516560in}}%
\pgfpathlineto{\pgfqpoint{5.779058in}{2.507448in}}%
\pgfpathlineto{\pgfqpoint{5.782279in}{2.501538in}}%
\pgfpathlineto{\pgfqpoint{5.785501in}{2.520254in}}%
\pgfpathlineto{\pgfqpoint{5.786575in}{2.525015in}}%
\pgfpathlineto{\pgfqpoint{5.790870in}{2.521896in}}%
\pgfpathlineto{\pgfqpoint{5.791944in}{2.521157in}}%
\pgfpathlineto{\pgfqpoint{5.794092in}{2.512209in}}%
\pgfpathlineto{\pgfqpoint{5.797313in}{2.507941in}}%
\pgfpathlineto{\pgfqpoint{5.798387in}{2.509008in}}%
\pgfpathlineto{\pgfqpoint{5.799461in}{2.509254in}}%
\pgfpathlineto{\pgfqpoint{5.800535in}{2.519597in}}%
\pgfpathlineto{\pgfqpoint{5.801609in}{2.522553in}}%
\pgfpathlineto{\pgfqpoint{5.804830in}{2.523784in}}%
\pgfpathlineto{\pgfqpoint{5.805904in}{2.519515in}}%
\pgfpathlineto{\pgfqpoint{5.808052in}{2.520583in}}%
\pgfpathlineto{\pgfqpoint{5.812347in}{2.518612in}}%
\pgfpathlineto{\pgfqpoint{5.813421in}{2.519926in}}%
\pgfpathlineto{\pgfqpoint{5.814495in}{2.519105in}}%
\pgfpathlineto{\pgfqpoint{5.815568in}{2.516314in}}%
\pgfpathlineto{\pgfqpoint{5.816642in}{2.525344in}}%
\pgfpathlineto{\pgfqpoint{5.820938in}{2.522635in}}%
\pgfpathlineto{\pgfqpoint{5.822011in}{2.527396in}}%
\pgfpathlineto{\pgfqpoint{5.823085in}{2.523209in}}%
\pgfpathlineto{\pgfqpoint{5.824159in}{2.522881in}}%
\pgfpathlineto{\pgfqpoint{5.827381in}{2.519433in}}%
\pgfpathlineto{\pgfqpoint{5.828455in}{2.520172in}}%
\pgfpathlineto{\pgfqpoint{5.829528in}{2.517463in}}%
\pgfpathlineto{\pgfqpoint{5.834898in}{2.525180in}}%
\pgfpathlineto{\pgfqpoint{5.835971in}{2.528463in}}%
\pgfpathlineto{\pgfqpoint{5.837045in}{2.515657in}}%
\pgfpathlineto{\pgfqpoint{5.838119in}{2.510075in}}%
\pgfpathlineto{\pgfqpoint{5.842414in}{2.514672in}}%
\pgfpathlineto{\pgfqpoint{5.843488in}{2.500881in}}%
\pgfpathlineto{\pgfqpoint{5.844562in}{2.503344in}}%
\pgfpathlineto{\pgfqpoint{5.845636in}{2.502441in}}%
\pgfpathlineto{\pgfqpoint{5.846710in}{2.505068in}}%
\pgfpathlineto{\pgfqpoint{5.852079in}{2.516068in}}%
\pgfpathlineto{\pgfqpoint{5.853153in}{2.513851in}}%
\pgfpathlineto{\pgfqpoint{5.854227in}{2.520008in}}%
\pgfpathlineto{\pgfqpoint{5.857448in}{2.519926in}}%
\pgfpathlineto{\pgfqpoint{5.858522in}{2.522553in}}%
\pgfpathlineto{\pgfqpoint{5.859596in}{2.520090in}}%
\pgfpathlineto{\pgfqpoint{5.860670in}{2.522060in}}%
\pgfpathlineto{\pgfqpoint{5.861744in}{2.513277in}}%
\pgfpathlineto{\pgfqpoint{5.864965in}{2.516232in}}%
\pgfpathlineto{\pgfqpoint{5.867113in}{2.503097in}}%
\pgfpathlineto{\pgfqpoint{5.868187in}{2.505560in}}%
\pgfpathlineto{\pgfqpoint{5.869260in}{2.504329in}}%
\pgfpathlineto{\pgfqpoint{5.872482in}{2.505642in}}%
\pgfpathlineto{\pgfqpoint{5.874630in}{2.515165in}}%
\pgfpathlineto{\pgfqpoint{5.875703in}{2.513194in}}%
\pgfpathlineto{\pgfqpoint{5.876777in}{2.514754in}}%
\pgfpathlineto{\pgfqpoint{5.881073in}{2.511881in}}%
\pgfpathlineto{\pgfqpoint{5.882146in}{2.517463in}}%
\pgfpathlineto{\pgfqpoint{5.883220in}{2.518612in}}%
\pgfpathlineto{\pgfqpoint{5.884294in}{2.522471in}}%
\pgfpathlineto{\pgfqpoint{5.887516in}{2.524687in}}%
\pgfpathlineto{\pgfqpoint{5.888589in}{2.521403in}}%
\pgfpathlineto{\pgfqpoint{5.889663in}{2.523784in}}%
\pgfpathlineto{\pgfqpoint{5.890737in}{2.527642in}}%
\pgfpathlineto{\pgfqpoint{5.891811in}{2.527889in}}%
\pgfpathlineto{\pgfqpoint{5.895032in}{2.523209in}}%
\pgfpathlineto{\pgfqpoint{5.896106in}{2.529038in}}%
\pgfpathlineto{\pgfqpoint{5.897180in}{2.526083in}}%
\pgfpathlineto{\pgfqpoint{5.898254in}{2.529448in}}%
\pgfpathlineto{\pgfqpoint{5.899328in}{2.527396in}}%
\pgfpathlineto{\pgfqpoint{5.902549in}{2.526739in}}%
\pgfpathlineto{\pgfqpoint{5.905771in}{2.535359in}}%
\pgfpathlineto{\pgfqpoint{5.906845in}{2.520911in}}%
\pgfpathlineto{\pgfqpoint{5.910066in}{2.513277in}}%
\pgfpathlineto{\pgfqpoint{5.913288in}{2.538806in}}%
\pgfpathlineto{\pgfqpoint{5.914362in}{2.539627in}}%
\pgfpathlineto{\pgfqpoint{5.918657in}{2.544471in}}%
\pgfpathlineto{\pgfqpoint{5.920805in}{2.539956in}}%
\pgfpathlineto{\pgfqpoint{5.921878in}{2.546933in}}%
\pgfpathlineto{\pgfqpoint{5.926174in}{2.546769in}}%
\pgfpathlineto{\pgfqpoint{5.927248in}{2.547836in}}%
\pgfpathlineto{\pgfqpoint{5.928321in}{2.547672in}}%
\pgfpathlineto{\pgfqpoint{5.929395in}{2.548739in}}%
\pgfpathlineto{\pgfqpoint{5.932617in}{2.548165in}}%
\pgfpathlineto{\pgfqpoint{5.933691in}{2.549889in}}%
\pgfpathlineto{\pgfqpoint{5.935838in}{2.548083in}}%
\pgfpathlineto{\pgfqpoint{5.936912in}{2.551530in}}%
\pgfpathlineto{\pgfqpoint{5.940134in}{2.552105in}}%
\pgfpathlineto{\pgfqpoint{5.942281in}{2.542172in}}%
\pgfpathlineto{\pgfqpoint{5.943355in}{2.544717in}}%
\pgfpathlineto{\pgfqpoint{5.944429in}{2.550545in}}%
\pgfpathlineto{\pgfqpoint{5.948724in}{2.559329in}}%
\pgfpathlineto{\pgfqpoint{5.949798in}{2.553418in}}%
\pgfpathlineto{\pgfqpoint{5.950872in}{2.553993in}}%
\pgfpathlineto{\pgfqpoint{5.951946in}{2.552023in}}%
\pgfpathlineto{\pgfqpoint{5.955167in}{2.551859in}}%
\pgfpathlineto{\pgfqpoint{5.957315in}{2.555963in}}%
\pgfpathlineto{\pgfqpoint{5.959463in}{2.561381in}}%
\pgfpathlineto{\pgfqpoint{5.962684in}{2.561217in}}%
\pgfpathlineto{\pgfqpoint{5.963758in}{2.557933in}}%
\pgfpathlineto{\pgfqpoint{5.965906in}{2.564419in}}%
\pgfpathlineto{\pgfqpoint{5.970201in}{2.559986in}}%
\pgfpathlineto{\pgfqpoint{5.971275in}{2.564090in}}%
\pgfpathlineto{\pgfqpoint{5.972349in}{2.563433in}}%
\pgfpathlineto{\pgfqpoint{5.973423in}{2.567866in}}%
\pgfpathlineto{\pgfqpoint{5.974497in}{2.565404in}}%
\pgfpathlineto{\pgfqpoint{5.977718in}{2.570822in}}%
\pgfpathlineto{\pgfqpoint{5.978792in}{2.565157in}}%
\pgfpathlineto{\pgfqpoint{5.979866in}{2.563433in}}%
\pgfpathlineto{\pgfqpoint{5.980940in}{2.570904in}}%
\pgfpathlineto{\pgfqpoint{5.982013in}{2.570083in}}%
\pgfpathlineto{\pgfqpoint{5.986309in}{2.573366in}}%
\pgfpathlineto{\pgfqpoint{5.987383in}{2.568195in}}%
\pgfpathlineto{\pgfqpoint{5.988456in}{2.566963in}}%
\pgfpathlineto{\pgfqpoint{5.989530in}{2.555471in}}%
\pgfpathlineto{\pgfqpoint{5.992752in}{2.570411in}}%
\pgfpathlineto{\pgfqpoint{5.993826in}{2.561463in}}%
\pgfpathlineto{\pgfqpoint{5.994899in}{2.561217in}}%
\pgfpathlineto{\pgfqpoint{5.995973in}{2.569016in}}%
\pgfpathlineto{\pgfqpoint{5.997047in}{2.568933in}}%
\pgfpathlineto{\pgfqpoint{6.000269in}{2.571314in}}%
\pgfpathlineto{\pgfqpoint{6.001343in}{2.572874in}}%
\pgfpathlineto{\pgfqpoint{6.002416in}{2.567045in}}%
\pgfpathlineto{\pgfqpoint{6.003490in}{2.575993in}}%
\pgfpathlineto{\pgfqpoint{6.004564in}{2.566799in}}%
\pgfpathlineto{\pgfqpoint{6.007786in}{2.567456in}}%
\pgfpathlineto{\pgfqpoint{6.008859in}{2.571232in}}%
\pgfpathlineto{\pgfqpoint{6.009933in}{2.579441in}}%
\pgfpathlineto{\pgfqpoint{6.011007in}{2.570329in}}%
\pgfpathlineto{\pgfqpoint{6.012081in}{2.581657in}}%
\pgfpathlineto{\pgfqpoint{6.016376in}{2.571232in}}%
\pgfpathlineto{\pgfqpoint{6.019598in}{2.583545in}}%
\pgfpathlineto{\pgfqpoint{6.022819in}{2.576486in}}%
\pgfpathlineto{\pgfqpoint{6.023893in}{2.572628in}}%
\pgfpathlineto{\pgfqpoint{6.024967in}{2.572874in}}%
\pgfpathlineto{\pgfqpoint{6.026041in}{2.570329in}}%
\pgfpathlineto{\pgfqpoint{6.027115in}{2.571807in}}%
\pgfpathlineto{\pgfqpoint{6.030336in}{2.567292in}}%
\pgfpathlineto{\pgfqpoint{6.031410in}{2.564501in}}%
\pgfpathlineto{\pgfqpoint{6.033558in}{2.550545in}}%
\pgfpathlineto{\pgfqpoint{6.034632in}{2.546030in}}%
\pgfpathlineto{\pgfqpoint{6.037853in}{2.544307in}}%
\pgfpathlineto{\pgfqpoint{6.038927in}{2.565896in}}%
\pgfpathlineto{\pgfqpoint{6.040001in}{2.569098in}}%
\pgfpathlineto{\pgfqpoint{6.041075in}{2.562941in}}%
\pgfpathlineto{\pgfqpoint{6.042148in}{2.564911in}}%
\pgfpathlineto{\pgfqpoint{6.047518in}{2.564172in}}%
\pgfpathlineto{\pgfqpoint{6.048591in}{2.563105in}}%
\pgfpathlineto{\pgfqpoint{6.049665in}{2.551695in}}%
\pgfpathlineto{\pgfqpoint{6.052887in}{2.562777in}}%
\pgfpathlineto{\pgfqpoint{6.053961in}{2.569590in}}%
\pgfpathlineto{\pgfqpoint{6.055034in}{2.558098in}}%
\pgfpathlineto{\pgfqpoint{6.056108in}{2.535687in}}%
\pgfpathlineto{\pgfqpoint{6.057182in}{2.540366in}}%
\pgfpathlineto{\pgfqpoint{6.060404in}{2.536015in}}%
\pgfpathlineto{\pgfqpoint{6.061477in}{2.540695in}}%
\pgfpathlineto{\pgfqpoint{6.062551in}{2.537411in}}%
\pgfpathlineto{\pgfqpoint{6.063625in}{2.536508in}}%
\pgfpathlineto{\pgfqpoint{6.064699in}{2.528463in}}%
\pgfpathlineto{\pgfqpoint{6.068994in}{2.534209in}}%
\pgfpathlineto{\pgfqpoint{6.070068in}{2.533635in}}%
\pgfpathlineto{\pgfqpoint{6.072216in}{2.539463in}}%
\pgfpathlineto{\pgfqpoint{6.076511in}{2.535195in}}%
\pgfpathlineto{\pgfqpoint{6.078659in}{2.527396in}}%
\pgfpathlineto{\pgfqpoint{6.079733in}{2.531500in}}%
\pgfpathlineto{\pgfqpoint{6.082954in}{2.535933in}}%
\pgfpathlineto{\pgfqpoint{6.084028in}{2.535359in}}%
\pgfpathlineto{\pgfqpoint{6.085102in}{2.544881in}}%
\pgfpathlineto{\pgfqpoint{6.086176in}{2.539792in}}%
\pgfpathlineto{\pgfqpoint{6.087250in}{2.546359in}}%
\pgfpathlineto{\pgfqpoint{6.090471in}{2.552023in}}%
\pgfpathlineto{\pgfqpoint{6.091545in}{2.552433in}}%
\pgfpathlineto{\pgfqpoint{6.092619in}{2.546359in}}%
\pgfpathlineto{\pgfqpoint{6.093693in}{2.548657in}}%
\pgfpathlineto{\pgfqpoint{6.097988in}{2.548904in}}%
\pgfpathlineto{\pgfqpoint{6.099062in}{2.547836in}}%
\pgfpathlineto{\pgfqpoint{6.100136in}{2.545620in}}%
\pgfpathlineto{\pgfqpoint{6.101209in}{2.547098in}}%
\pgfpathlineto{\pgfqpoint{6.102283in}{2.550792in}}%
\pgfpathlineto{\pgfqpoint{6.106579in}{2.548083in}}%
\pgfpathlineto{\pgfqpoint{6.107653in}{2.544060in}}%
\pgfpathlineto{\pgfqpoint{6.108726in}{2.546195in}}%
\pgfpathlineto{\pgfqpoint{6.109800in}{2.544142in}}%
\pgfpathlineto{\pgfqpoint{6.114096in}{2.545045in}}%
\pgfpathlineto{\pgfqpoint{6.115169in}{2.547344in}}%
\pgfpathlineto{\pgfqpoint{6.116243in}{2.551530in}}%
\pgfpathlineto{\pgfqpoint{6.117317in}{2.551284in}}%
\pgfpathlineto{\pgfqpoint{6.120539in}{2.546523in}}%
\pgfpathlineto{\pgfqpoint{6.121612in}{2.539709in}}%
\pgfpathlineto{\pgfqpoint{6.122686in}{2.541680in}}%
\pgfpathlineto{\pgfqpoint{6.123760in}{2.542336in}}%
\pgfpathlineto{\pgfqpoint{6.124834in}{2.543650in}}%
\pgfpathlineto{\pgfqpoint{6.130203in}{2.555635in}}%
\pgfpathlineto{\pgfqpoint{6.131277in}{2.553829in}}%
\pgfpathlineto{\pgfqpoint{6.132351in}{2.574680in}}%
\pgfpathlineto{\pgfqpoint{6.135572in}{2.570986in}}%
\pgfpathlineto{\pgfqpoint{6.136646in}{2.577799in}}%
\pgfpathlineto{\pgfqpoint{6.138794in}{2.568277in}}%
\pgfpathlineto{\pgfqpoint{6.139868in}{2.569180in}}%
\pgfpathlineto{\pgfqpoint{6.143089in}{2.569426in}}%
\pgfpathlineto{\pgfqpoint{6.144163in}{2.575829in}}%
\pgfpathlineto{\pgfqpoint{6.145237in}{2.573777in}}%
\pgfpathlineto{\pgfqpoint{6.146311in}{2.577060in}}%
\pgfpathlineto{\pgfqpoint{6.147385in}{2.574433in}}%
\pgfpathlineto{\pgfqpoint{6.150606in}{2.574351in}}%
\pgfpathlineto{\pgfqpoint{6.152754in}{2.581411in}}%
\pgfpathlineto{\pgfqpoint{6.153828in}{2.583956in}}%
\pgfpathlineto{\pgfqpoint{6.154901in}{2.578702in}}%
\pgfpathlineto{\pgfqpoint{6.158123in}{2.581247in}}%
\pgfpathlineto{\pgfqpoint{6.159197in}{2.577799in}}%
\pgfpathlineto{\pgfqpoint{6.160271in}{2.602590in}}%
\pgfpathlineto{\pgfqpoint{6.161344in}{2.600046in}}%
\pgfpathlineto{\pgfqpoint{6.162418in}{2.602344in}}%
\pgfpathlineto{\pgfqpoint{6.166714in}{2.606695in}}%
\pgfpathlineto{\pgfqpoint{6.169935in}{2.602016in}}%
\pgfpathlineto{\pgfqpoint{6.173157in}{2.600784in}}%
\pgfpathlineto{\pgfqpoint{6.174231in}{2.602180in}}%
\pgfpathlineto{\pgfqpoint{6.175304in}{2.606695in}}%
\pgfpathlineto{\pgfqpoint{6.177452in}{2.597829in}}%
\pgfpathlineto{\pgfqpoint{6.181747in}{2.596269in}}%
\pgfpathlineto{\pgfqpoint{6.182821in}{2.595120in}}%
\pgfpathlineto{\pgfqpoint{6.183895in}{2.596680in}}%
\pgfpathlineto{\pgfqpoint{6.184969in}{2.602180in}}%
\pgfpathlineto{\pgfqpoint{6.188190in}{2.603986in}}%
\pgfpathlineto{\pgfqpoint{6.189264in}{2.601687in}}%
\pgfpathlineto{\pgfqpoint{6.190338in}{2.604725in}}%
\pgfpathlineto{\pgfqpoint{6.191412in}{2.604971in}}%
\pgfpathlineto{\pgfqpoint{6.192486in}{2.601687in}}%
\pgfpathlineto{\pgfqpoint{6.195707in}{2.603329in}}%
\pgfpathlineto{\pgfqpoint{6.196781in}{2.603083in}}%
\pgfpathlineto{\pgfqpoint{6.200003in}{2.598404in}}%
\pgfpathlineto{\pgfqpoint{6.203224in}{2.597747in}}%
\pgfpathlineto{\pgfqpoint{6.204298in}{2.599799in}}%
\pgfpathlineto{\pgfqpoint{6.205372in}{2.598650in}}%
\pgfpathlineto{\pgfqpoint{6.207520in}{2.592904in}}%
\pgfpathlineto{\pgfqpoint{6.210741in}{2.591672in}}%
\pgfpathlineto{\pgfqpoint{6.211815in}{2.593396in}}%
\pgfpathlineto{\pgfqpoint{6.212889in}{2.593807in}}%
\pgfpathlineto{\pgfqpoint{6.213963in}{2.589538in}}%
\pgfpathlineto{\pgfqpoint{6.215036in}{2.588225in}}%
\pgfpathlineto{\pgfqpoint{6.218258in}{2.590195in}}%
\pgfpathlineto{\pgfqpoint{6.219332in}{2.592575in}}%
\pgfpathlineto{\pgfqpoint{6.220406in}{2.596434in}}%
\pgfpathlineto{\pgfqpoint{6.221479in}{2.594299in}}%
\pgfpathlineto{\pgfqpoint{6.225775in}{2.597008in}}%
\pgfpathlineto{\pgfqpoint{6.226849in}{2.600128in}}%
\pgfpathlineto{\pgfqpoint{6.230070in}{2.588717in}}%
\pgfpathlineto{\pgfqpoint{6.234365in}{2.599307in}}%
\pgfpathlineto{\pgfqpoint{6.235439in}{2.581986in}}%
\pgfpathlineto{\pgfqpoint{6.236513in}{2.581657in}}%
\pgfpathlineto{\pgfqpoint{6.237587in}{2.578866in}}%
\pgfpathlineto{\pgfqpoint{6.240809in}{2.577060in}}%
\pgfpathlineto{\pgfqpoint{6.241882in}{2.570411in}}%
\pgfpathlineto{\pgfqpoint{6.242956in}{2.571971in}}%
\pgfpathlineto{\pgfqpoint{6.248325in}{2.572956in}}%
\pgfpathlineto{\pgfqpoint{6.249399in}{2.571807in}}%
\pgfpathlineto{\pgfqpoint{6.250473in}{2.572463in}}%
\pgfpathlineto{\pgfqpoint{6.251547in}{2.570001in}}%
\pgfpathlineto{\pgfqpoint{6.252621in}{2.570165in}}%
\pgfpathlineto{\pgfqpoint{6.255842in}{2.571232in}}%
\pgfpathlineto{\pgfqpoint{6.256916in}{2.570575in}}%
\pgfpathlineto{\pgfqpoint{6.257990in}{2.570739in}}%
\pgfpathlineto{\pgfqpoint{6.259064in}{2.567702in}}%
\pgfpathlineto{\pgfqpoint{6.260138in}{2.570575in}}%
\pgfpathlineto{\pgfqpoint{6.263359in}{2.570329in}}%
\pgfpathlineto{\pgfqpoint{6.264433in}{2.569344in}}%
\pgfpathlineto{\pgfqpoint{6.267654in}{2.578292in}}%
\pgfpathlineto{\pgfqpoint{6.271950in}{2.579441in}}%
\pgfpathlineto{\pgfqpoint{6.273024in}{2.584695in}}%
\pgfpathlineto{\pgfqpoint{6.274097in}{2.585023in}}%
\pgfpathlineto{\pgfqpoint{6.275171in}{2.588471in}}%
\pgfpathlineto{\pgfqpoint{6.280541in}{2.589866in}}%
\pgfpathlineto{\pgfqpoint{6.281614in}{2.582889in}}%
\pgfpathlineto{\pgfqpoint{6.282688in}{2.585187in}}%
\pgfpathlineto{\pgfqpoint{6.285910in}{2.585844in}}%
\pgfpathlineto{\pgfqpoint{6.286984in}{2.584448in}}%
\pgfpathlineto{\pgfqpoint{6.288057in}{2.587404in}}%
\pgfpathlineto{\pgfqpoint{6.289131in}{2.594546in}}%
\pgfpathlineto{\pgfqpoint{6.290205in}{2.596680in}}%
\pgfpathlineto{\pgfqpoint{6.293427in}{2.598240in}}%
\pgfpathlineto{\pgfqpoint{6.296648in}{2.591837in}}%
\pgfpathlineto{\pgfqpoint{6.297722in}{2.594874in}}%
\pgfpathlineto{\pgfqpoint{6.300943in}{2.594381in}}%
\pgfpathlineto{\pgfqpoint{6.302017in}{2.588635in}}%
\pgfpathlineto{\pgfqpoint{6.303091in}{2.586829in}}%
\pgfpathlineto{\pgfqpoint{6.304165in}{2.576322in}}%
\pgfpathlineto{\pgfqpoint{6.305239in}{2.577471in}}%
\pgfpathlineto{\pgfqpoint{6.308460in}{2.581986in}}%
\pgfpathlineto{\pgfqpoint{6.310608in}{2.581493in}}%
\pgfpathlineto{\pgfqpoint{6.311682in}{2.579195in}}%
\pgfpathlineto{\pgfqpoint{6.312756in}{2.581329in}}%
\pgfpathlineto{\pgfqpoint{6.315977in}{2.577060in}}%
\pgfpathlineto{\pgfqpoint{6.317051in}{2.574598in}}%
\pgfpathlineto{\pgfqpoint{6.318125in}{2.576075in}}%
\pgfpathlineto{\pgfqpoint{6.319199in}{2.574105in}}%
\pgfpathlineto{\pgfqpoint{6.320273in}{2.577142in}}%
\pgfpathlineto{\pgfqpoint{6.323494in}{2.580590in}}%
\pgfpathlineto{\pgfqpoint{6.324568in}{2.588881in}}%
\pgfpathlineto{\pgfqpoint{6.326716in}{2.593889in}}%
\pgfpathlineto{\pgfqpoint{6.327789in}{2.593971in}}%
\pgfpathlineto{\pgfqpoint{6.331011in}{2.590687in}}%
\pgfpathlineto{\pgfqpoint{6.332085in}{2.598075in}}%
\pgfpathlineto{\pgfqpoint{6.333159in}{2.599307in}}%
\pgfpathlineto{\pgfqpoint{6.334232in}{2.609896in}}%
\pgfpathlineto{\pgfqpoint{6.335306in}{2.606284in}}%
\pgfpathlineto{\pgfqpoint{6.339602in}{2.613180in}}%
\pgfpathlineto{\pgfqpoint{6.340675in}{2.612687in}}%
\pgfpathlineto{\pgfqpoint{6.342823in}{2.609814in}}%
\pgfpathlineto{\pgfqpoint{6.348192in}{2.619501in}}%
\pgfpathlineto{\pgfqpoint{6.349266in}{2.618105in}}%
\pgfpathlineto{\pgfqpoint{6.350340in}{2.614986in}}%
\pgfpathlineto{\pgfqpoint{6.353562in}{2.617367in}}%
\pgfpathlineto{\pgfqpoint{6.354635in}{2.621635in}}%
\pgfpathlineto{\pgfqpoint{6.355709in}{2.623523in}}%
\pgfpathlineto{\pgfqpoint{6.356783in}{2.620650in}}%
\pgfpathlineto{\pgfqpoint{6.357857in}{2.623687in}}%
\pgfpathlineto{\pgfqpoint{6.361078in}{2.626725in}}%
\pgfpathlineto{\pgfqpoint{6.362152in}{2.626150in}}%
\pgfpathlineto{\pgfqpoint{6.364300in}{2.622210in}}%
\pgfpathlineto{\pgfqpoint{6.365374in}{2.624016in}}%
\pgfpathlineto{\pgfqpoint{6.368595in}{2.623687in}}%
\pgfpathlineto{\pgfqpoint{6.369669in}{2.622538in}}%
\pgfpathlineto{\pgfqpoint{6.370743in}{2.619090in}}%
\pgfpathlineto{\pgfqpoint{6.372891in}{2.624180in}}%
\pgfpathlineto{\pgfqpoint{6.378260in}{2.625658in}}%
\pgfpathlineto{\pgfqpoint{6.379334in}{2.627546in}}%
\pgfpathlineto{\pgfqpoint{6.380408in}{2.626561in}}%
\pgfpathlineto{\pgfqpoint{6.383629in}{2.635426in}}%
\pgfpathlineto{\pgfqpoint{6.384703in}{2.631732in}}%
\pgfpathlineto{\pgfqpoint{6.386851in}{2.632061in}}%
\pgfpathlineto{\pgfqpoint{6.387924in}{2.629844in}}%
\pgfpathlineto{\pgfqpoint{6.391146in}{2.628941in}}%
\pgfpathlineto{\pgfqpoint{6.392220in}{2.636822in}}%
\pgfpathlineto{\pgfqpoint{6.393294in}{2.638546in}}%
\pgfpathlineto{\pgfqpoint{6.394367in}{2.625001in}}%
\pgfpathlineto{\pgfqpoint{6.395441in}{2.621964in}}%
\pgfpathlineto{\pgfqpoint{6.398663in}{2.625658in}}%
\pgfpathlineto{\pgfqpoint{6.399737in}{2.625083in}}%
\pgfpathlineto{\pgfqpoint{6.400810in}{2.611374in}}%
\pgfpathlineto{\pgfqpoint{6.402958in}{2.612195in}}%
\pgfpathlineto{\pgfqpoint{6.408327in}{2.623359in}}%
\pgfpathlineto{\pgfqpoint{6.409401in}{2.620322in}}%
\pgfpathlineto{\pgfqpoint{6.410475in}{2.622620in}}%
\pgfpathlineto{\pgfqpoint{6.413697in}{2.620978in}}%
\pgfpathlineto{\pgfqpoint{6.414770in}{2.617202in}}%
\pgfpathlineto{\pgfqpoint{6.415844in}{2.615971in}}%
\pgfpathlineto{\pgfqpoint{6.417992in}{2.628120in}}%
\pgfpathlineto{\pgfqpoint{6.421213in}{2.628859in}}%
\pgfpathlineto{\pgfqpoint{6.422287in}{2.626232in}}%
\pgfpathlineto{\pgfqpoint{6.423361in}{2.625986in}}%
\pgfpathlineto{\pgfqpoint{6.424435in}{2.622210in}}%
\pgfpathlineto{\pgfqpoint{6.425509in}{2.596269in}}%
\pgfpathlineto{\pgfqpoint{6.429804in}{2.586419in}}%
\pgfpathlineto{\pgfqpoint{6.430878in}{2.585516in}}%
\pgfpathlineto{\pgfqpoint{6.431952in}{2.590441in}}%
\pgfpathlineto{\pgfqpoint{6.433026in}{2.586911in}}%
\pgfpathlineto{\pgfqpoint{6.436247in}{2.580919in}}%
\pgfpathlineto{\pgfqpoint{6.437321in}{2.581493in}}%
\pgfpathlineto{\pgfqpoint{6.438395in}{2.585762in}}%
\pgfpathlineto{\pgfqpoint{6.439469in}{2.582807in}}%
\pgfpathlineto{\pgfqpoint{6.440542in}{2.583299in}}%
\pgfpathlineto{\pgfqpoint{6.443764in}{2.579195in}}%
\pgfpathlineto{\pgfqpoint{6.445912in}{2.591098in}}%
\pgfpathlineto{\pgfqpoint{6.446985in}{2.592657in}}%
\pgfpathlineto{\pgfqpoint{6.448059in}{2.595613in}}%
\pgfpathlineto{\pgfqpoint{6.451281in}{2.602098in}}%
\pgfpathlineto{\pgfqpoint{6.452355in}{2.601113in}}%
\pgfpathlineto{\pgfqpoint{6.453429in}{2.596105in}}%
\pgfpathlineto{\pgfqpoint{6.454502in}{2.604068in}}%
\pgfpathlineto{\pgfqpoint{6.455576in}{2.597665in}}%
\pgfpathlineto{\pgfqpoint{6.458798in}{2.596434in}}%
\pgfpathlineto{\pgfqpoint{6.459872in}{2.599963in}}%
\pgfpathlineto{\pgfqpoint{6.460945in}{2.596926in}}%
\pgfpathlineto{\pgfqpoint{6.463093in}{2.597829in}}%
\pgfpathlineto{\pgfqpoint{6.466315in}{2.601769in}}%
\pgfpathlineto{\pgfqpoint{6.467388in}{2.605217in}}%
\pgfpathlineto{\pgfqpoint{6.468462in}{2.605053in}}%
\pgfpathlineto{\pgfqpoint{6.470610in}{2.612687in}}%
\pgfpathlineto{\pgfqpoint{6.473831in}{2.620814in}}%
\pgfpathlineto{\pgfqpoint{6.474905in}{2.620732in}}%
\pgfpathlineto{\pgfqpoint{6.475979in}{2.619583in}}%
\pgfpathlineto{\pgfqpoint{6.477053in}{2.610635in}}%
\pgfpathlineto{\pgfqpoint{6.478127in}{2.612769in}}%
\pgfpathlineto{\pgfqpoint{6.481348in}{2.611702in}}%
\pgfpathlineto{\pgfqpoint{6.482422in}{2.608747in}}%
\pgfpathlineto{\pgfqpoint{6.483496in}{2.616710in}}%
\pgfpathlineto{\pgfqpoint{6.484570in}{2.617613in}}%
\pgfpathlineto{\pgfqpoint{6.485644in}{2.624508in}}%
\pgfpathlineto{\pgfqpoint{6.488865in}{2.624508in}}%
\pgfpathlineto{\pgfqpoint{6.491013in}{2.621717in}}%
\pgfpathlineto{\pgfqpoint{6.492087in}{2.622867in}}%
\pgfpathlineto{\pgfqpoint{6.493161in}{2.626396in}}%
\pgfpathlineto{\pgfqpoint{6.497456in}{2.629105in}}%
\pgfpathlineto{\pgfqpoint{6.498530in}{2.626150in}}%
\pgfpathlineto{\pgfqpoint{6.499604in}{2.625904in}}%
\pgfpathlineto{\pgfqpoint{6.500677in}{2.624508in}}%
\pgfpathlineto{\pgfqpoint{6.500677in}{2.624508in}}%
\pgfusepath{stroke}%
\end{pgfscope}%
\begin{pgfscope}%
\pgfpathrectangle{\pgfqpoint{4.034768in}{2.255883in}}{\pgfqpoint{2.583333in}{0.400885in}}%
\pgfusepath{clip}%
\pgfsetroundcap%
\pgfsetroundjoin%
\pgfsetlinewidth{1.505625pt}%
\definecolor{currentstroke}{rgb}{0.549020,0.337255,0.294118}%
\pgfsetstrokecolor{currentstroke}%
\pgfsetdash{}{0pt}%
\pgfpathmoveto{\pgfqpoint{4.152193in}{2.340231in}}%
\pgfpathlineto{\pgfqpoint{4.153266in}{2.340231in}}%
\pgfpathlineto{\pgfqpoint{4.155414in}{2.337438in}}%
\pgfpathlineto{\pgfqpoint{4.158636in}{2.339245in}}%
\pgfpathlineto{\pgfqpoint{4.159709in}{2.341299in}}%
\pgfpathlineto{\pgfqpoint{4.160783in}{2.337152in}}%
\pgfpathlineto{\pgfqpoint{4.161857in}{2.337982in}}%
\pgfpathlineto{\pgfqpoint{4.162931in}{2.337982in}}%
\pgfpathlineto{\pgfqpoint{4.167226in}{2.335079in}}%
\pgfpathlineto{\pgfqpoint{4.168300in}{2.333198in}}%
\pgfpathlineto{\pgfqpoint{4.169374in}{2.332873in}}%
\pgfpathlineto{\pgfqpoint{4.170448in}{2.331902in}}%
\pgfpathlineto{\pgfqpoint{4.173669in}{2.339719in}}%
\pgfpathlineto{\pgfqpoint{4.174743in}{2.336457in}}%
\pgfpathlineto{\pgfqpoint{4.175817in}{2.339635in}}%
\pgfpathlineto{\pgfqpoint{4.176891in}{2.340890in}}%
\pgfpathlineto{\pgfqpoint{4.177965in}{2.337609in}}%
\pgfpathlineto{\pgfqpoint{4.182260in}{2.329279in}}%
\pgfpathlineto{\pgfqpoint{4.183334in}{2.330373in}}%
\pgfpathlineto{\pgfqpoint{4.184408in}{2.329616in}}%
\pgfpathlineto{\pgfqpoint{4.185482in}{2.333304in}}%
\pgfpathlineto{\pgfqpoint{4.188703in}{2.338338in}}%
\pgfpathlineto{\pgfqpoint{4.189777in}{2.337059in}}%
\pgfpathlineto{\pgfqpoint{4.190851in}{2.337483in}}%
\pgfpathlineto{\pgfqpoint{4.191925in}{2.334848in}}%
\pgfpathlineto{\pgfqpoint{4.196220in}{2.338133in}}%
\pgfpathlineto{\pgfqpoint{4.197294in}{2.336446in}}%
\pgfpathlineto{\pgfqpoint{4.198368in}{2.335944in}}%
\pgfpathlineto{\pgfqpoint{4.199441in}{2.331687in}}%
\pgfpathlineto{\pgfqpoint{4.200515in}{2.333568in}}%
\pgfpathlineto{\pgfqpoint{4.204811in}{2.330349in}}%
\pgfpathlineto{\pgfqpoint{4.205885in}{2.330514in}}%
\pgfpathlineto{\pgfqpoint{4.206958in}{2.317636in}}%
\pgfpathlineto{\pgfqpoint{4.208032in}{2.315850in}}%
\pgfpathlineto{\pgfqpoint{4.211254in}{2.315927in}}%
\pgfpathlineto{\pgfqpoint{4.212328in}{2.320084in}}%
\pgfpathlineto{\pgfqpoint{4.213401in}{2.318622in}}%
\pgfpathlineto{\pgfqpoint{4.214475in}{2.324430in}}%
\pgfpathlineto{\pgfqpoint{4.215549in}{2.324509in}}%
\pgfpathlineto{\pgfqpoint{4.218771in}{2.322779in}}%
\pgfpathlineto{\pgfqpoint{4.219844in}{2.323481in}}%
\pgfpathlineto{\pgfqpoint{4.220918in}{2.321916in}}%
\pgfpathlineto{\pgfqpoint{4.221992in}{2.323872in}}%
\pgfpathlineto{\pgfqpoint{4.223066in}{2.323716in}}%
\pgfpathlineto{\pgfqpoint{4.227361in}{2.317722in}}%
\pgfpathlineto{\pgfqpoint{4.228435in}{2.318026in}}%
\pgfpathlineto{\pgfqpoint{4.229509in}{2.316961in}}%
\pgfpathlineto{\pgfqpoint{4.230583in}{2.314374in}}%
\pgfpathlineto{\pgfqpoint{4.235952in}{2.314070in}}%
\pgfpathlineto{\pgfqpoint{4.237026in}{2.316048in}}%
\pgfpathlineto{\pgfqpoint{4.238100in}{2.316580in}}%
\pgfpathlineto{\pgfqpoint{4.241321in}{2.316733in}}%
\pgfpathlineto{\pgfqpoint{4.242395in}{2.318563in}}%
\pgfpathlineto{\pgfqpoint{4.243469in}{2.318794in}}%
\pgfpathlineto{\pgfqpoint{4.245617in}{2.321187in}}%
\pgfpathlineto{\pgfqpoint{4.248838in}{2.319020in}}%
\pgfpathlineto{\pgfqpoint{4.249912in}{2.321853in}}%
\pgfpathlineto{\pgfqpoint{4.250986in}{2.322863in}}%
\pgfpathlineto{\pgfqpoint{4.252060in}{2.322552in}}%
\pgfpathlineto{\pgfqpoint{4.256355in}{2.325576in}}%
\pgfpathlineto{\pgfqpoint{4.257429in}{2.322665in}}%
\pgfpathlineto{\pgfqpoint{4.258503in}{2.323294in}}%
\pgfpathlineto{\pgfqpoint{4.259576in}{2.325497in}}%
\pgfpathlineto{\pgfqpoint{4.260650in}{2.323747in}}%
\pgfpathlineto{\pgfqpoint{4.263872in}{2.329792in}}%
\pgfpathlineto{\pgfqpoint{4.264946in}{2.328281in}}%
\pgfpathlineto{\pgfqpoint{4.266019in}{2.329939in}}%
\pgfpathlineto{\pgfqpoint{4.267093in}{2.328824in}}%
\pgfpathlineto{\pgfqpoint{4.268167in}{2.334714in}}%
\pgfpathlineto{\pgfqpoint{4.272462in}{2.342412in}}%
\pgfpathlineto{\pgfqpoint{4.273536in}{2.339473in}}%
\pgfpathlineto{\pgfqpoint{4.274610in}{2.339634in}}%
\pgfpathlineto{\pgfqpoint{4.275684in}{2.324077in}}%
\pgfpathlineto{\pgfqpoint{4.278906in}{2.318919in}}%
\pgfpathlineto{\pgfqpoint{4.279979in}{2.318516in}}%
\pgfpathlineto{\pgfqpoint{4.281053in}{2.321095in}}%
\pgfpathlineto{\pgfqpoint{4.282127in}{2.317710in}}%
\pgfpathlineto{\pgfqpoint{4.283201in}{2.319137in}}%
\pgfpathlineto{\pgfqpoint{4.287496in}{2.318418in}}%
\pgfpathlineto{\pgfqpoint{4.288570in}{2.315305in}}%
\pgfpathlineto{\pgfqpoint{4.290718in}{2.321133in}}%
\pgfpathlineto{\pgfqpoint{4.293939in}{2.320485in}}%
\pgfpathlineto{\pgfqpoint{4.295013in}{2.321376in}}%
\pgfpathlineto{\pgfqpoint{4.296087in}{2.317732in}}%
\pgfpathlineto{\pgfqpoint{4.297161in}{2.319800in}}%
\pgfpathlineto{\pgfqpoint{4.298235in}{2.316987in}}%
\pgfpathlineto{\pgfqpoint{4.301456in}{2.316103in}}%
\pgfpathlineto{\pgfqpoint{4.302530in}{2.314577in}}%
\pgfpathlineto{\pgfqpoint{4.303604in}{2.309755in}}%
\pgfpathlineto{\pgfqpoint{4.305751in}{2.311362in}}%
\pgfpathlineto{\pgfqpoint{4.310047in}{2.314343in}}%
\pgfpathlineto{\pgfqpoint{4.311121in}{2.318451in}}%
\pgfpathlineto{\pgfqpoint{4.312195in}{2.318287in}}%
\pgfpathlineto{\pgfqpoint{4.313268in}{2.323137in}}%
\pgfpathlineto{\pgfqpoint{4.316490in}{2.322127in}}%
\pgfpathlineto{\pgfqpoint{4.317564in}{2.320611in}}%
\pgfpathlineto{\pgfqpoint{4.318638in}{2.324821in}}%
\pgfpathlineto{\pgfqpoint{4.319711in}{2.318422in}}%
\pgfpathlineto{\pgfqpoint{4.320785in}{2.318422in}}%
\pgfpathlineto{\pgfqpoint{4.324007in}{2.319810in}}%
\pgfpathlineto{\pgfqpoint{4.326154in}{2.322440in}}%
\pgfpathlineto{\pgfqpoint{4.328302in}{2.328476in}}%
\pgfpathlineto{\pgfqpoint{4.332597in}{2.324052in}}%
\pgfpathlineto{\pgfqpoint{4.333671in}{2.311947in}}%
\pgfpathlineto{\pgfqpoint{4.334745in}{2.307689in}}%
\pgfpathlineto{\pgfqpoint{4.335819in}{2.308274in}}%
\pgfpathlineto{\pgfqpoint{4.339040in}{2.311780in}}%
\pgfpathlineto{\pgfqpoint{4.340114in}{2.311525in}}%
\pgfpathlineto{\pgfqpoint{4.341188in}{2.306682in}}%
\pgfpathlineto{\pgfqpoint{4.342262in}{2.304691in}}%
\pgfpathlineto{\pgfqpoint{4.343336in}{2.298451in}}%
\pgfpathlineto{\pgfqpoint{4.346557in}{2.298769in}}%
\pgfpathlineto{\pgfqpoint{4.347631in}{2.299805in}}%
\pgfpathlineto{\pgfqpoint{4.350853in}{2.300283in}}%
\pgfpathlineto{\pgfqpoint{4.354074in}{2.301960in}}%
\pgfpathlineto{\pgfqpoint{4.355148in}{2.300762in}}%
\pgfpathlineto{\pgfqpoint{4.356222in}{2.302906in}}%
\pgfpathlineto{\pgfqpoint{4.357296in}{2.317593in}}%
\pgfpathlineto{\pgfqpoint{4.358370in}{2.308684in}}%
\pgfpathlineto{\pgfqpoint{4.361591in}{2.310451in}}%
\pgfpathlineto{\pgfqpoint{4.362665in}{2.313784in}}%
\pgfpathlineto{\pgfqpoint{4.364813in}{2.312932in}}%
\pgfpathlineto{\pgfqpoint{4.365886in}{2.314090in}}%
\pgfpathlineto{\pgfqpoint{4.369108in}{2.311916in}}%
\pgfpathlineto{\pgfqpoint{4.370182in}{2.309665in}}%
\pgfpathlineto{\pgfqpoint{4.371256in}{2.309665in}}%
\pgfpathlineto{\pgfqpoint{4.373403in}{2.303121in}}%
\pgfpathlineto{\pgfqpoint{4.376625in}{2.303046in}}%
\pgfpathlineto{\pgfqpoint{4.377699in}{2.306425in}}%
\pgfpathlineto{\pgfqpoint{4.379846in}{2.300162in}}%
\pgfpathlineto{\pgfqpoint{4.380920in}{2.312306in}}%
\pgfpathlineto{\pgfqpoint{4.384142in}{2.310396in}}%
\pgfpathlineto{\pgfqpoint{4.386289in}{2.304911in}}%
\pgfpathlineto{\pgfqpoint{4.388437in}{2.305058in}}%
\pgfpathlineto{\pgfqpoint{4.391659in}{2.306750in}}%
\pgfpathlineto{\pgfqpoint{4.392732in}{2.308235in}}%
\pgfpathlineto{\pgfqpoint{4.393806in}{2.308754in}}%
\pgfpathlineto{\pgfqpoint{4.394880in}{2.310912in}}%
\pgfpathlineto{\pgfqpoint{4.395954in}{2.310912in}}%
\pgfpathlineto{\pgfqpoint{4.400249in}{2.309498in}}%
\pgfpathlineto{\pgfqpoint{4.402397in}{2.311060in}}%
\pgfpathlineto{\pgfqpoint{4.403471in}{2.313080in}}%
\pgfpathlineto{\pgfqpoint{4.406692in}{2.312556in}}%
\pgfpathlineto{\pgfqpoint{4.407766in}{2.313228in}}%
\pgfpathlineto{\pgfqpoint{4.408840in}{2.312479in}}%
\pgfpathlineto{\pgfqpoint{4.409914in}{2.312554in}}%
\pgfpathlineto{\pgfqpoint{4.410988in}{2.314425in}}%
\pgfpathlineto{\pgfqpoint{4.415283in}{2.313153in}}%
\pgfpathlineto{\pgfqpoint{4.416357in}{2.313822in}}%
\pgfpathlineto{\pgfqpoint{4.417431in}{2.319493in}}%
\pgfpathlineto{\pgfqpoint{4.418505in}{2.317852in}}%
\pgfpathlineto{\pgfqpoint{4.421726in}{2.317852in}}%
\pgfpathlineto{\pgfqpoint{4.423874in}{2.315409in}}%
\pgfpathlineto{\pgfqpoint{4.424948in}{2.320220in}}%
\pgfpathlineto{\pgfqpoint{4.426021in}{2.318740in}}%
\pgfpathlineto{\pgfqpoint{4.430317in}{2.318372in}}%
\pgfpathlineto{\pgfqpoint{4.431391in}{2.318592in}}%
\pgfpathlineto{\pgfqpoint{4.432464in}{2.316832in}}%
\pgfpathlineto{\pgfqpoint{4.437834in}{2.320625in}}%
\pgfpathlineto{\pgfqpoint{4.438907in}{2.318937in}}%
\pgfpathlineto{\pgfqpoint{4.444277in}{2.318791in}}%
\pgfpathlineto{\pgfqpoint{4.446424in}{2.324908in}}%
\pgfpathlineto{\pgfqpoint{4.448572in}{2.322157in}}%
\pgfpathlineto{\pgfqpoint{4.451794in}{2.325249in}}%
\pgfpathlineto{\pgfqpoint{4.455015in}{2.318672in}}%
\pgfpathlineto{\pgfqpoint{4.456089in}{2.318299in}}%
\pgfpathlineto{\pgfqpoint{4.459310in}{2.322932in}}%
\pgfpathlineto{\pgfqpoint{4.460384in}{2.321213in}}%
\pgfpathlineto{\pgfqpoint{4.461458in}{2.315062in}}%
\pgfpathlineto{\pgfqpoint{4.462532in}{2.315062in}}%
\pgfpathlineto{\pgfqpoint{4.463606in}{2.309884in}}%
\pgfpathlineto{\pgfqpoint{4.466827in}{2.309452in}}%
\pgfpathlineto{\pgfqpoint{4.467901in}{2.303267in}}%
\pgfpathlineto{\pgfqpoint{4.468975in}{2.307007in}}%
\pgfpathlineto{\pgfqpoint{4.470049in}{2.295428in}}%
\pgfpathlineto{\pgfqpoint{4.471123in}{2.298888in}}%
\pgfpathlineto{\pgfqpoint{4.478639in}{2.297506in}}%
\pgfpathlineto{\pgfqpoint{4.481861in}{2.294606in}}%
\pgfpathlineto{\pgfqpoint{4.482935in}{2.296056in}}%
\pgfpathlineto{\pgfqpoint{4.484009in}{2.300959in}}%
\pgfpathlineto{\pgfqpoint{4.485083in}{2.294371in}}%
\pgfpathlineto{\pgfqpoint{4.486156in}{2.294938in}}%
\pgfpathlineto{\pgfqpoint{4.489378in}{2.294513in}}%
\pgfpathlineto{\pgfqpoint{4.490452in}{2.295785in}}%
\pgfpathlineto{\pgfqpoint{4.492599in}{2.292725in}}%
\pgfpathlineto{\pgfqpoint{4.493673in}{2.295571in}}%
\pgfpathlineto{\pgfqpoint{4.501190in}{2.280053in}}%
\pgfpathlineto{\pgfqpoint{4.504412in}{2.280643in}}%
\pgfpathlineto{\pgfqpoint{4.505485in}{2.278012in}}%
\pgfpathlineto{\pgfqpoint{4.506559in}{2.280446in}}%
\pgfpathlineto{\pgfqpoint{4.507633in}{2.280117in}}%
\pgfpathlineto{\pgfqpoint{4.508707in}{2.278410in}}%
\pgfpathlineto{\pgfqpoint{4.511928in}{2.279711in}}%
\pgfpathlineto{\pgfqpoint{4.513002in}{2.278269in}}%
\pgfpathlineto{\pgfqpoint{4.514076in}{2.278793in}}%
\pgfpathlineto{\pgfqpoint{4.516224in}{2.274105in}}%
\pgfpathlineto{\pgfqpoint{4.519445in}{2.274488in}}%
\pgfpathlineto{\pgfqpoint{4.520519in}{2.276728in}}%
\pgfpathlineto{\pgfqpoint{4.521593in}{2.276152in}}%
\pgfpathlineto{\pgfqpoint{4.522667in}{2.279405in}}%
\pgfpathlineto{\pgfqpoint{4.523741in}{2.278301in}}%
\pgfpathlineto{\pgfqpoint{4.528036in}{2.278561in}}%
\pgfpathlineto{\pgfqpoint{4.530184in}{2.284452in}}%
\pgfpathlineto{\pgfqpoint{4.531258in}{2.290337in}}%
\pgfpathlineto{\pgfqpoint{4.534479in}{2.289177in}}%
\pgfpathlineto{\pgfqpoint{4.536627in}{2.286311in}}%
\pgfpathlineto{\pgfqpoint{4.537701in}{2.286174in}}%
\pgfpathlineto{\pgfqpoint{4.538774in}{2.281670in}}%
\pgfpathlineto{\pgfqpoint{4.541996in}{2.285765in}}%
\pgfpathlineto{\pgfqpoint{4.544144in}{2.277507in}}%
\pgfpathlineto{\pgfqpoint{4.545217in}{2.279858in}}%
\pgfpathlineto{\pgfqpoint{4.549513in}{2.283165in}}%
\pgfpathlineto{\pgfqpoint{4.550587in}{2.282562in}}%
\pgfpathlineto{\pgfqpoint{4.551661in}{2.284573in}}%
\pgfpathlineto{\pgfqpoint{4.552734in}{2.282495in}}%
\pgfpathlineto{\pgfqpoint{4.553808in}{2.282760in}}%
\pgfpathlineto{\pgfqpoint{4.557030in}{2.284951in}}%
\pgfpathlineto{\pgfqpoint{4.558104in}{2.283623in}}%
\pgfpathlineto{\pgfqpoint{4.559177in}{2.283491in}}%
\pgfpathlineto{\pgfqpoint{4.561325in}{2.280277in}}%
\pgfpathlineto{\pgfqpoint{4.565620in}{2.280212in}}%
\pgfpathlineto{\pgfqpoint{4.566694in}{2.284096in}}%
\pgfpathlineto{\pgfqpoint{4.567768in}{2.285520in}}%
\pgfpathlineto{\pgfqpoint{4.568842in}{2.300587in}}%
\pgfpathlineto{\pgfqpoint{4.572063in}{2.297848in}}%
\pgfpathlineto{\pgfqpoint{4.573137in}{2.291416in}}%
\pgfpathlineto{\pgfqpoint{4.575285in}{2.290608in}}%
\pgfpathlineto{\pgfqpoint{4.576359in}{2.286768in}}%
\pgfpathlineto{\pgfqpoint{4.579580in}{2.290046in}}%
\pgfpathlineto{\pgfqpoint{4.580654in}{2.292270in}}%
\pgfpathlineto{\pgfqpoint{4.581728in}{2.289984in}}%
\pgfpathlineto{\pgfqpoint{4.582802in}{2.289984in}}%
\pgfpathlineto{\pgfqpoint{4.583876in}{2.287970in}}%
\pgfpathlineto{\pgfqpoint{4.587097in}{2.288275in}}%
\pgfpathlineto{\pgfqpoint{4.588171in}{2.287421in}}%
\pgfpathlineto{\pgfqpoint{4.589245in}{2.284565in}}%
\pgfpathlineto{\pgfqpoint{4.590319in}{2.283488in}}%
\pgfpathlineto{\pgfqpoint{4.596762in}{2.290125in}}%
\pgfpathlineto{\pgfqpoint{4.598909in}{2.289643in}}%
\pgfpathlineto{\pgfqpoint{4.602131in}{2.284395in}}%
\pgfpathlineto{\pgfqpoint{4.603205in}{2.285179in}}%
\pgfpathlineto{\pgfqpoint{4.604279in}{2.281862in}}%
\pgfpathlineto{\pgfqpoint{4.606426in}{2.286211in}}%
\pgfpathlineto{\pgfqpoint{4.609648in}{2.285308in}}%
\pgfpathlineto{\pgfqpoint{4.611795in}{2.282742in}}%
\pgfpathlineto{\pgfqpoint{4.613943in}{2.285587in}}%
\pgfpathlineto{\pgfqpoint{4.617165in}{2.284754in}}%
\pgfpathlineto{\pgfqpoint{4.618238in}{2.285642in}}%
\pgfpathlineto{\pgfqpoint{4.619312in}{2.283856in}}%
\pgfpathlineto{\pgfqpoint{4.620386in}{2.286714in}}%
\pgfpathlineto{\pgfqpoint{4.621460in}{2.291835in}}%
\pgfpathlineto{\pgfqpoint{4.624682in}{2.290978in}}%
\pgfpathlineto{\pgfqpoint{4.625755in}{2.295689in}}%
\pgfpathlineto{\pgfqpoint{4.626829in}{2.293303in}}%
\pgfpathlineto{\pgfqpoint{4.627903in}{2.295176in}}%
\pgfpathlineto{\pgfqpoint{4.628977in}{2.295482in}}%
\pgfpathlineto{\pgfqpoint{4.632198in}{2.298411in}}%
\pgfpathlineto{\pgfqpoint{4.634346in}{2.303802in}}%
\pgfpathlineto{\pgfqpoint{4.635420in}{2.303802in}}%
\pgfpathlineto{\pgfqpoint{4.639715in}{2.307053in}}%
\pgfpathlineto{\pgfqpoint{4.640789in}{2.300614in}}%
\pgfpathlineto{\pgfqpoint{4.642937in}{2.306830in}}%
\pgfpathlineto{\pgfqpoint{4.644011in}{2.308437in}}%
\pgfpathlineto{\pgfqpoint{4.647232in}{2.311304in}}%
\pgfpathlineto{\pgfqpoint{4.648306in}{2.313983in}}%
\pgfpathlineto{\pgfqpoint{4.649380in}{2.319035in}}%
\pgfpathlineto{\pgfqpoint{4.651527in}{2.314763in}}%
\pgfpathlineto{\pgfqpoint{4.654749in}{2.316928in}}%
\pgfpathlineto{\pgfqpoint{4.655823in}{2.319241in}}%
\pgfpathlineto{\pgfqpoint{4.657971in}{2.328788in}}%
\pgfpathlineto{\pgfqpoint{4.659044in}{2.320640in}}%
\pgfpathlineto{\pgfqpoint{4.662266in}{2.320764in}}%
\pgfpathlineto{\pgfqpoint{4.663340in}{2.326505in}}%
\pgfpathlineto{\pgfqpoint{4.664414in}{2.350954in}}%
\pgfpathlineto{\pgfqpoint{4.665487in}{2.347898in}}%
\pgfpathlineto{\pgfqpoint{4.666561in}{2.350885in}}%
\pgfpathlineto{\pgfqpoint{4.669783in}{2.347620in}}%
\pgfpathlineto{\pgfqpoint{4.670857in}{2.352753in}}%
\pgfpathlineto{\pgfqpoint{4.671930in}{2.354014in}}%
\pgfpathlineto{\pgfqpoint{4.674078in}{2.347126in}}%
\pgfpathlineto{\pgfqpoint{4.678373in}{2.350601in}}%
\pgfpathlineto{\pgfqpoint{4.679447in}{2.347785in}}%
\pgfpathlineto{\pgfqpoint{4.680521in}{2.349006in}}%
\pgfpathlineto{\pgfqpoint{4.681595in}{2.352074in}}%
\pgfpathlineto{\pgfqpoint{4.684816in}{2.353029in}}%
\pgfpathlineto{\pgfqpoint{4.685890in}{2.357960in}}%
\pgfpathlineto{\pgfqpoint{4.686964in}{2.351180in}}%
\pgfpathlineto{\pgfqpoint{4.688038in}{2.353771in}}%
\pgfpathlineto{\pgfqpoint{4.689112in}{2.352763in}}%
\pgfpathlineto{\pgfqpoint{4.693407in}{2.346040in}}%
\pgfpathlineto{\pgfqpoint{4.694481in}{2.346175in}}%
\pgfpathlineto{\pgfqpoint{4.695555in}{2.345503in}}%
\pgfpathlineto{\pgfqpoint{4.696629in}{2.363048in}}%
\pgfpathlineto{\pgfqpoint{4.700924in}{2.368694in}}%
\pgfpathlineto{\pgfqpoint{4.701998in}{2.357598in}}%
\pgfpathlineto{\pgfqpoint{4.703072in}{2.358700in}}%
\pgfpathlineto{\pgfqpoint{4.704146in}{2.371865in}}%
\pgfpathlineto{\pgfqpoint{4.708441in}{2.378883in}}%
\pgfpathlineto{\pgfqpoint{4.709515in}{2.374539in}}%
\pgfpathlineto{\pgfqpoint{4.710589in}{2.375569in}}%
\pgfpathlineto{\pgfqpoint{4.711662in}{2.369973in}}%
\pgfpathlineto{\pgfqpoint{4.715958in}{2.367819in}}%
\pgfpathlineto{\pgfqpoint{4.717032in}{2.370810in}}%
\pgfpathlineto{\pgfqpoint{4.719179in}{2.378097in}}%
\pgfpathlineto{\pgfqpoint{4.722401in}{2.383636in}}%
\pgfpathlineto{\pgfqpoint{4.723475in}{2.383126in}}%
\pgfpathlineto{\pgfqpoint{4.724549in}{2.391708in}}%
\pgfpathlineto{\pgfqpoint{4.725622in}{2.377151in}}%
\pgfpathlineto{\pgfqpoint{4.726696in}{2.390653in}}%
\pgfpathlineto{\pgfqpoint{4.730992in}{2.396626in}}%
\pgfpathlineto{\pgfqpoint{4.732065in}{2.392154in}}%
\pgfpathlineto{\pgfqpoint{4.733139in}{2.390489in}}%
\pgfpathlineto{\pgfqpoint{4.734213in}{2.394699in}}%
\pgfpathlineto{\pgfqpoint{4.737435in}{2.401202in}}%
\pgfpathlineto{\pgfqpoint{4.738508in}{2.398601in}}%
\pgfpathlineto{\pgfqpoint{4.739582in}{2.397768in}}%
\pgfpathlineto{\pgfqpoint{4.744951in}{2.401857in}}%
\pgfpathlineto{\pgfqpoint{4.746025in}{2.396849in}}%
\pgfpathlineto{\pgfqpoint{4.747099in}{2.395436in}}%
\pgfpathlineto{\pgfqpoint{4.749247in}{2.384905in}}%
\pgfpathlineto{\pgfqpoint{4.752468in}{2.385188in}}%
\pgfpathlineto{\pgfqpoint{4.754616in}{2.380442in}}%
\pgfpathlineto{\pgfqpoint{4.755690in}{2.381646in}}%
\pgfpathlineto{\pgfqpoint{4.756764in}{2.375341in}}%
\pgfpathlineto{\pgfqpoint{4.759985in}{2.376169in}}%
\pgfpathlineto{\pgfqpoint{4.761059in}{2.374577in}}%
\pgfpathlineto{\pgfqpoint{4.762133in}{2.370701in}}%
\pgfpathlineto{\pgfqpoint{4.764281in}{2.371670in}}%
\pgfpathlineto{\pgfqpoint{4.767502in}{2.369938in}}%
\pgfpathlineto{\pgfqpoint{4.769650in}{2.372778in}}%
\pgfpathlineto{\pgfqpoint{4.770724in}{2.380493in}}%
\pgfpathlineto{\pgfqpoint{4.771797in}{2.382509in}}%
\pgfpathlineto{\pgfqpoint{4.775019in}{2.383140in}}%
\pgfpathlineto{\pgfqpoint{4.777167in}{2.379856in}}%
\pgfpathlineto{\pgfqpoint{4.778240in}{2.378680in}}%
\pgfpathlineto{\pgfqpoint{4.779314in}{2.381707in}}%
\pgfpathlineto{\pgfqpoint{4.783610in}{2.381846in}}%
\pgfpathlineto{\pgfqpoint{4.784683in}{2.379477in}}%
\pgfpathlineto{\pgfqpoint{4.786831in}{2.371672in}}%
\pgfpathlineto{\pgfqpoint{4.792200in}{2.368675in}}%
\pgfpathlineto{\pgfqpoint{4.793274in}{2.370975in}}%
\pgfpathlineto{\pgfqpoint{4.794348in}{2.369651in}}%
\pgfpathlineto{\pgfqpoint{4.797570in}{2.378035in}}%
\pgfpathlineto{\pgfqpoint{4.798643in}{2.374657in}}%
\pgfpathlineto{\pgfqpoint{4.799717in}{2.367973in}}%
\pgfpathlineto{\pgfqpoint{4.800791in}{2.370704in}}%
\pgfpathlineto{\pgfqpoint{4.801865in}{2.367254in}}%
\pgfpathlineto{\pgfqpoint{4.806160in}{2.368033in}}%
\pgfpathlineto{\pgfqpoint{4.808308in}{2.364410in}}%
\pgfpathlineto{\pgfqpoint{4.809382in}{2.364481in}}%
\pgfpathlineto{\pgfqpoint{4.814751in}{2.373656in}}%
\pgfpathlineto{\pgfqpoint{4.815825in}{2.373728in}}%
\pgfpathlineto{\pgfqpoint{4.816899in}{2.378372in}}%
\pgfpathlineto{\pgfqpoint{4.820120in}{2.371798in}}%
\pgfpathlineto{\pgfqpoint{4.823342in}{2.377316in}}%
\pgfpathlineto{\pgfqpoint{4.824415in}{2.373025in}}%
\pgfpathlineto{\pgfqpoint{4.827637in}{2.372392in}}%
\pgfpathlineto{\pgfqpoint{4.829785in}{2.363319in}}%
\pgfpathlineto{\pgfqpoint{4.830859in}{2.365218in}}%
\pgfpathlineto{\pgfqpoint{4.831932in}{2.370071in}}%
\pgfpathlineto{\pgfqpoint{4.835154in}{2.360442in}}%
\pgfpathlineto{\pgfqpoint{4.836228in}{2.363819in}}%
\pgfpathlineto{\pgfqpoint{4.837302in}{2.365185in}}%
\pgfpathlineto{\pgfqpoint{4.838375in}{2.364679in}}%
\pgfpathlineto{\pgfqpoint{4.839449in}{2.365763in}}%
\pgfpathlineto{\pgfqpoint{4.842671in}{2.368004in}}%
\pgfpathlineto{\pgfqpoint{4.843745in}{2.372093in}}%
\pgfpathlineto{\pgfqpoint{4.846966in}{2.359331in}}%
\pgfpathlineto{\pgfqpoint{4.850188in}{2.357881in}}%
\pgfpathlineto{\pgfqpoint{4.852335in}{2.372237in}}%
\pgfpathlineto{\pgfqpoint{4.853409in}{2.365882in}}%
\pgfpathlineto{\pgfqpoint{4.854483in}{2.365951in}}%
\pgfpathlineto{\pgfqpoint{4.857704in}{2.363477in}}%
\pgfpathlineto{\pgfqpoint{4.858778in}{2.371516in}}%
\pgfpathlineto{\pgfqpoint{4.859852in}{2.368493in}}%
\pgfpathlineto{\pgfqpoint{4.860926in}{2.370188in}}%
\pgfpathlineto{\pgfqpoint{4.862000in}{2.366704in}}%
\pgfpathlineto{\pgfqpoint{4.865221in}{2.374151in}}%
\pgfpathlineto{\pgfqpoint{4.866295in}{2.367592in}}%
\pgfpathlineto{\pgfqpoint{4.867369in}{2.372772in}}%
\pgfpathlineto{\pgfqpoint{4.868443in}{2.368424in}}%
\pgfpathlineto{\pgfqpoint{4.869517in}{2.370666in}}%
\pgfpathlineto{\pgfqpoint{4.873812in}{2.369106in}}%
\pgfpathlineto{\pgfqpoint{4.874886in}{2.361347in}}%
\pgfpathlineto{\pgfqpoint{4.875960in}{2.363956in}}%
\pgfpathlineto{\pgfqpoint{4.877034in}{2.364946in}}%
\pgfpathlineto{\pgfqpoint{4.880255in}{2.366265in}}%
\pgfpathlineto{\pgfqpoint{4.881329in}{2.369252in}}%
\pgfpathlineto{\pgfqpoint{4.884550in}{2.358149in}}%
\pgfpathlineto{\pgfqpoint{4.887772in}{2.359602in}}%
\pgfpathlineto{\pgfqpoint{4.888846in}{2.358394in}}%
\pgfpathlineto{\pgfqpoint{4.889920in}{2.360239in}}%
\pgfpathlineto{\pgfqpoint{4.890993in}{2.360366in}}%
\pgfpathlineto{\pgfqpoint{4.892067in}{2.361893in}}%
\pgfpathlineto{\pgfqpoint{4.895289in}{2.359475in}}%
\pgfpathlineto{\pgfqpoint{4.896363in}{2.363505in}}%
\pgfpathlineto{\pgfqpoint{4.897437in}{2.361581in}}%
\pgfpathlineto{\pgfqpoint{4.899584in}{2.361261in}}%
\pgfpathlineto{\pgfqpoint{4.902806in}{2.356581in}}%
\pgfpathlineto{\pgfqpoint{4.904953in}{2.361710in}}%
\pgfpathlineto{\pgfqpoint{4.906027in}{2.358143in}}%
\pgfpathlineto{\pgfqpoint{4.907101in}{2.368065in}}%
\pgfpathlineto{\pgfqpoint{4.910323in}{2.366638in}}%
\pgfpathlineto{\pgfqpoint{4.911396in}{2.372761in}}%
\pgfpathlineto{\pgfqpoint{4.912470in}{2.374813in}}%
\pgfpathlineto{\pgfqpoint{4.913544in}{2.384281in}}%
\pgfpathlineto{\pgfqpoint{4.914618in}{2.384694in}}%
\pgfpathlineto{\pgfqpoint{4.917839in}{2.388626in}}%
\pgfpathlineto{\pgfqpoint{4.918913in}{2.384068in}}%
\pgfpathlineto{\pgfqpoint{4.919987in}{2.392693in}}%
\pgfpathlineto{\pgfqpoint{4.921061in}{2.395497in}}%
\pgfpathlineto{\pgfqpoint{4.922135in}{2.395143in}}%
\pgfpathlineto{\pgfqpoint{4.925356in}{2.391950in}}%
\pgfpathlineto{\pgfqpoint{4.926430in}{2.391950in}}%
\pgfpathlineto{\pgfqpoint{4.928578in}{2.394646in}}%
\pgfpathlineto{\pgfqpoint{4.929652in}{2.393156in}}%
\pgfpathlineto{\pgfqpoint{4.932873in}{2.393227in}}%
\pgfpathlineto{\pgfqpoint{4.936095in}{2.384622in}}%
\pgfpathlineto{\pgfqpoint{4.937169in}{2.384129in}}%
\pgfpathlineto{\pgfqpoint{4.940390in}{2.385257in}}%
\pgfpathlineto{\pgfqpoint{4.941464in}{2.380673in}}%
\pgfpathlineto{\pgfqpoint{4.942538in}{2.387521in}}%
\pgfpathlineto{\pgfqpoint{4.944685in}{2.389302in}}%
\pgfpathlineto{\pgfqpoint{4.947907in}{2.387587in}}%
\pgfpathlineto{\pgfqpoint{4.948981in}{2.392731in}}%
\pgfpathlineto{\pgfqpoint{4.950055in}{2.393231in}}%
\pgfpathlineto{\pgfqpoint{4.951128in}{2.391871in}}%
\pgfpathlineto{\pgfqpoint{4.952202in}{2.387790in}}%
\pgfpathlineto{\pgfqpoint{4.956498in}{2.389580in}}%
\pgfpathlineto{\pgfqpoint{4.957571in}{2.391656in}}%
\pgfpathlineto{\pgfqpoint{4.958645in}{2.385663in}}%
\pgfpathlineto{\pgfqpoint{4.959719in}{2.391367in}}%
\pgfpathlineto{\pgfqpoint{4.962941in}{2.395700in}}%
\pgfpathlineto{\pgfqpoint{4.964014in}{2.399670in}}%
\pgfpathlineto{\pgfqpoint{4.965088in}{2.408714in}}%
\pgfpathlineto{\pgfqpoint{4.966162in}{2.403754in}}%
\pgfpathlineto{\pgfqpoint{4.967236in}{2.402152in}}%
\pgfpathlineto{\pgfqpoint{4.970458in}{2.396277in}}%
\pgfpathlineto{\pgfqpoint{4.971531in}{2.398719in}}%
\pgfpathlineto{\pgfqpoint{4.974753in}{2.390943in}}%
\pgfpathlineto{\pgfqpoint{4.977974in}{2.386510in}}%
\pgfpathlineto{\pgfqpoint{4.979048in}{2.381577in}}%
\pgfpathlineto{\pgfqpoint{4.980122in}{2.389605in}}%
\pgfpathlineto{\pgfqpoint{4.981196in}{2.391517in}}%
\pgfpathlineto{\pgfqpoint{4.982270in}{2.381666in}}%
\pgfpathlineto{\pgfqpoint{4.987639in}{2.391230in}}%
\pgfpathlineto{\pgfqpoint{4.988713in}{2.392620in}}%
\pgfpathlineto{\pgfqpoint{4.993008in}{2.393576in}}%
\pgfpathlineto{\pgfqpoint{4.996230in}{2.398902in}}%
\pgfpathlineto{\pgfqpoint{4.997303in}{2.396005in}}%
\pgfpathlineto{\pgfqpoint{5.000525in}{2.403271in}}%
\pgfpathlineto{\pgfqpoint{5.002673in}{2.413481in}}%
\pgfpathlineto{\pgfqpoint{5.003747in}{2.415018in}}%
\pgfpathlineto{\pgfqpoint{5.004820in}{2.412943in}}%
\pgfpathlineto{\pgfqpoint{5.008042in}{2.412029in}}%
\pgfpathlineto{\pgfqpoint{5.010190in}{2.407492in}}%
\pgfpathlineto{\pgfqpoint{5.011263in}{2.407790in}}%
\pgfpathlineto{\pgfqpoint{5.012337in}{2.406446in}}%
\pgfpathlineto{\pgfqpoint{5.015559in}{2.411822in}}%
\pgfpathlineto{\pgfqpoint{5.016633in}{2.412270in}}%
\pgfpathlineto{\pgfqpoint{5.017706in}{2.406061in}}%
\pgfpathlineto{\pgfqpoint{5.019854in}{2.400451in}}%
\pgfpathlineto{\pgfqpoint{5.023076in}{2.409353in}}%
\pgfpathlineto{\pgfqpoint{5.024149in}{2.406136in}}%
\pgfpathlineto{\pgfqpoint{5.026297in}{2.408949in}}%
\pgfpathlineto{\pgfqpoint{5.027371in}{2.408205in}}%
\pgfpathlineto{\pgfqpoint{5.030592in}{2.403011in}}%
\pgfpathlineto{\pgfqpoint{5.031666in}{2.404610in}}%
\pgfpathlineto{\pgfqpoint{5.032740in}{2.403366in}}%
\pgfpathlineto{\pgfqpoint{5.033814in}{2.403147in}}%
\pgfpathlineto{\pgfqpoint{5.034888in}{2.401099in}}%
\pgfpathlineto{\pgfqpoint{5.038109in}{2.405560in}}%
\pgfpathlineto{\pgfqpoint{5.039183in}{2.400295in}}%
\pgfpathlineto{\pgfqpoint{5.040257in}{2.399436in}}%
\pgfpathlineto{\pgfqpoint{5.041331in}{2.401861in}}%
\pgfpathlineto{\pgfqpoint{5.042405in}{2.399916in}}%
\pgfpathlineto{\pgfqpoint{5.046700in}{2.400060in}}%
\pgfpathlineto{\pgfqpoint{5.047774in}{2.404957in}}%
\pgfpathlineto{\pgfqpoint{5.048848in}{2.404309in}}%
\pgfpathlineto{\pgfqpoint{5.053143in}{2.405458in}}%
\pgfpathlineto{\pgfqpoint{5.055291in}{2.402066in}}%
\pgfpathlineto{\pgfqpoint{5.056365in}{2.406901in}}%
\pgfpathlineto{\pgfqpoint{5.057438in}{2.405313in}}%
\pgfpathlineto{\pgfqpoint{5.060660in}{2.396064in}}%
\pgfpathlineto{\pgfqpoint{5.061734in}{2.398965in}}%
\pgfpathlineto{\pgfqpoint{5.063881in}{2.400852in}}%
\pgfpathlineto{\pgfqpoint{5.064955in}{2.398393in}}%
\pgfpathlineto{\pgfqpoint{5.068177in}{2.397058in}}%
\pgfpathlineto{\pgfqpoint{5.069251in}{2.393686in}}%
\pgfpathlineto{\pgfqpoint{5.070325in}{2.399377in}}%
\pgfpathlineto{\pgfqpoint{5.071398in}{2.398955in}}%
\pgfpathlineto{\pgfqpoint{5.072472in}{2.397553in}}%
\pgfpathlineto{\pgfqpoint{5.075694in}{2.401458in}}%
\pgfpathlineto{\pgfqpoint{5.076768in}{2.400749in}}%
\pgfpathlineto{\pgfqpoint{5.079989in}{2.393025in}}%
\pgfpathlineto{\pgfqpoint{5.083211in}{2.390616in}}%
\pgfpathlineto{\pgfqpoint{5.084284in}{2.392458in}}%
\pgfpathlineto{\pgfqpoint{5.086432in}{2.389984in}}%
\pgfpathlineto{\pgfqpoint{5.087506in}{2.390755in}}%
\pgfpathlineto{\pgfqpoint{5.091801in}{2.388152in}}%
\pgfpathlineto{\pgfqpoint{5.092875in}{2.388293in}}%
\pgfpathlineto{\pgfqpoint{5.093949in}{2.390052in}}%
\pgfpathlineto{\pgfqpoint{5.095023in}{2.387730in}}%
\pgfpathlineto{\pgfqpoint{5.098244in}{2.390239in}}%
\pgfpathlineto{\pgfqpoint{5.099318in}{2.387633in}}%
\pgfpathlineto{\pgfqpoint{5.100392in}{2.387281in}}%
\pgfpathlineto{\pgfqpoint{5.101466in}{2.388760in}}%
\pgfpathlineto{\pgfqpoint{5.102540in}{2.389253in}}%
\pgfpathlineto{\pgfqpoint{5.107909in}{2.389816in}}%
\pgfpathlineto{\pgfqpoint{5.110057in}{2.387344in}}%
\pgfpathlineto{\pgfqpoint{5.114352in}{2.388262in}}%
\pgfpathlineto{\pgfqpoint{5.115426in}{2.389537in}}%
\pgfpathlineto{\pgfqpoint{5.116500in}{2.386846in}}%
\pgfpathlineto{\pgfqpoint{5.117573in}{2.388737in}}%
\pgfpathlineto{\pgfqpoint{5.120795in}{2.386337in}}%
\pgfpathlineto{\pgfqpoint{5.121869in}{2.383302in}}%
\pgfpathlineto{\pgfqpoint{5.122943in}{2.385137in}}%
\pgfpathlineto{\pgfqpoint{5.125090in}{2.391672in}}%
\pgfpathlineto{\pgfqpoint{5.128312in}{2.394258in}}%
\pgfpathlineto{\pgfqpoint{5.129386in}{2.398470in}}%
\pgfpathlineto{\pgfqpoint{5.131533in}{2.394132in}}%
\pgfpathlineto{\pgfqpoint{5.135829in}{2.392919in}}%
\pgfpathlineto{\pgfqpoint{5.136903in}{2.390648in}}%
\pgfpathlineto{\pgfqpoint{5.137976in}{2.384039in}}%
\pgfpathlineto{\pgfqpoint{5.139050in}{2.384381in}}%
\pgfpathlineto{\pgfqpoint{5.140124in}{2.381779in}}%
\pgfpathlineto{\pgfqpoint{5.144419in}{2.383011in}}%
\pgfpathlineto{\pgfqpoint{5.145493in}{2.384932in}}%
\pgfpathlineto{\pgfqpoint{5.146567in}{2.388087in}}%
\pgfpathlineto{\pgfqpoint{5.147641in}{2.388921in}}%
\pgfpathlineto{\pgfqpoint{5.150862in}{2.390520in}}%
\pgfpathlineto{\pgfqpoint{5.153010in}{2.388841in}}%
\pgfpathlineto{\pgfqpoint{5.154084in}{2.390450in}}%
\pgfpathlineto{\pgfqpoint{5.155158in}{2.394649in}}%
\pgfpathlineto{\pgfqpoint{5.158379in}{2.392797in}}%
\pgfpathlineto{\pgfqpoint{5.161601in}{2.381047in}}%
\pgfpathlineto{\pgfqpoint{5.162675in}{2.395218in}}%
\pgfpathlineto{\pgfqpoint{5.166970in}{2.399005in}}%
\pgfpathlineto{\pgfqpoint{5.168044in}{2.388710in}}%
\pgfpathlineto{\pgfqpoint{5.170191in}{2.399176in}}%
\pgfpathlineto{\pgfqpoint{5.173413in}{2.395926in}}%
\pgfpathlineto{\pgfqpoint{5.175561in}{2.396623in}}%
\pgfpathlineto{\pgfqpoint{5.176635in}{2.393830in}}%
\pgfpathlineto{\pgfqpoint{5.177708in}{2.394866in}}%
\pgfpathlineto{\pgfqpoint{5.180930in}{2.398817in}}%
\pgfpathlineto{\pgfqpoint{5.185225in}{2.393244in}}%
\pgfpathlineto{\pgfqpoint{5.188447in}{2.392431in}}%
\pgfpathlineto{\pgfqpoint{5.189521in}{2.393309in}}%
\pgfpathlineto{\pgfqpoint{5.190594in}{2.392903in}}%
\pgfpathlineto{\pgfqpoint{5.191668in}{2.391276in}}%
\pgfpathlineto{\pgfqpoint{5.192742in}{2.391751in}}%
\pgfpathlineto{\pgfqpoint{5.197037in}{2.392496in}}%
\pgfpathlineto{\pgfqpoint{5.198111in}{2.392020in}}%
\pgfpathlineto{\pgfqpoint{5.199185in}{2.396642in}}%
\pgfpathlineto{\pgfqpoint{5.200259in}{2.396235in}}%
\pgfpathlineto{\pgfqpoint{5.203480in}{2.398813in}}%
\pgfpathlineto{\pgfqpoint{5.204554in}{2.396824in}}%
\pgfpathlineto{\pgfqpoint{5.205628in}{2.400665in}}%
\pgfpathlineto{\pgfqpoint{5.206702in}{2.401557in}}%
\pgfpathlineto{\pgfqpoint{5.207776in}{2.400180in}}%
\pgfpathlineto{\pgfqpoint{5.210997in}{2.403828in}}%
\pgfpathlineto{\pgfqpoint{5.213145in}{2.402178in}}%
\pgfpathlineto{\pgfqpoint{5.214219in}{2.401973in}}%
\pgfpathlineto{\pgfqpoint{5.215293in}{2.403613in}}%
\pgfpathlineto{\pgfqpoint{5.218514in}{2.401632in}}%
\pgfpathlineto{\pgfqpoint{5.219588in}{2.403800in}}%
\pgfpathlineto{\pgfqpoint{5.221736in}{2.413714in}}%
\pgfpathlineto{\pgfqpoint{5.222810in}{2.415181in}}%
\pgfpathlineto{\pgfqpoint{5.226031in}{2.416019in}}%
\pgfpathlineto{\pgfqpoint{5.228179in}{2.408241in}}%
\pgfpathlineto{\pgfqpoint{5.229253in}{2.407751in}}%
\pgfpathlineto{\pgfqpoint{5.230326in}{2.412165in}}%
\pgfpathlineto{\pgfqpoint{5.233548in}{2.413497in}}%
\pgfpathlineto{\pgfqpoint{5.234622in}{2.411031in}}%
\pgfpathlineto{\pgfqpoint{5.237843in}{2.426463in}}%
\pgfpathlineto{\pgfqpoint{5.241065in}{2.434450in}}%
\pgfpathlineto{\pgfqpoint{5.242139in}{2.435553in}}%
\pgfpathlineto{\pgfqpoint{5.243213in}{2.439305in}}%
\pgfpathlineto{\pgfqpoint{5.244286in}{2.434753in}}%
\pgfpathlineto{\pgfqpoint{5.245360in}{2.441319in}}%
\pgfpathlineto{\pgfqpoint{5.250729in}{2.431209in}}%
\pgfpathlineto{\pgfqpoint{5.251803in}{2.437384in}}%
\pgfpathlineto{\pgfqpoint{5.252877in}{2.449591in}}%
\pgfpathlineto{\pgfqpoint{5.256099in}{2.444591in}}%
\pgfpathlineto{\pgfqpoint{5.257172in}{2.441343in}}%
\pgfpathlineto{\pgfqpoint{5.258246in}{2.440915in}}%
\pgfpathlineto{\pgfqpoint{5.260394in}{2.436449in}}%
\pgfpathlineto{\pgfqpoint{5.263615in}{2.435819in}}%
\pgfpathlineto{\pgfqpoint{5.264689in}{2.428277in}}%
\pgfpathlineto{\pgfqpoint{5.265763in}{2.426173in}}%
\pgfpathlineto{\pgfqpoint{5.266837in}{2.426712in}}%
\pgfpathlineto{\pgfqpoint{5.267911in}{2.428061in}}%
\pgfpathlineto{\pgfqpoint{5.271132in}{2.426240in}}%
\pgfpathlineto{\pgfqpoint{5.272206in}{2.424967in}}%
\pgfpathlineto{\pgfqpoint{5.273280in}{2.426033in}}%
\pgfpathlineto{\pgfqpoint{5.275428in}{2.418137in}}%
\pgfpathlineto{\pgfqpoint{5.278649in}{2.416598in}}%
\pgfpathlineto{\pgfqpoint{5.279723in}{2.417200in}}%
\pgfpathlineto{\pgfqpoint{5.280797in}{2.412717in}}%
\pgfpathlineto{\pgfqpoint{5.282945in}{2.414891in}}%
\pgfpathlineto{\pgfqpoint{5.286166in}{2.417668in}}%
\pgfpathlineto{\pgfqpoint{5.287240in}{2.421612in}}%
\pgfpathlineto{\pgfqpoint{5.288314in}{2.421144in}}%
\pgfpathlineto{\pgfqpoint{5.290461in}{2.412269in}}%
\pgfpathlineto{\pgfqpoint{5.293683in}{2.414202in}}%
\pgfpathlineto{\pgfqpoint{5.296904in}{2.429023in}}%
\pgfpathlineto{\pgfqpoint{5.297978in}{2.430153in}}%
\pgfpathlineto{\pgfqpoint{5.302274in}{2.432691in}}%
\pgfpathlineto{\pgfqpoint{5.303347in}{2.428612in}}%
\pgfpathlineto{\pgfqpoint{5.304421in}{2.431019in}}%
\pgfpathlineto{\pgfqpoint{5.305495in}{2.435967in}}%
\pgfpathlineto{\pgfqpoint{5.308717in}{2.433923in}}%
\pgfpathlineto{\pgfqpoint{5.309791in}{2.434809in}}%
\pgfpathlineto{\pgfqpoint{5.311938in}{2.419563in}}%
\pgfpathlineto{\pgfqpoint{5.313012in}{2.419241in}}%
\pgfpathlineto{\pgfqpoint{5.316234in}{2.416225in}}%
\pgfpathlineto{\pgfqpoint{5.317307in}{2.412482in}}%
\pgfpathlineto{\pgfqpoint{5.320529in}{2.415050in}}%
\pgfpathlineto{\pgfqpoint{5.323750in}{2.419502in}}%
\pgfpathlineto{\pgfqpoint{5.324824in}{2.418226in}}%
\pgfpathlineto{\pgfqpoint{5.325898in}{2.411014in}}%
\pgfpathlineto{\pgfqpoint{5.328046in}{2.407440in}}%
\pgfpathlineto{\pgfqpoint{5.331267in}{2.405079in}}%
\pgfpathlineto{\pgfqpoint{5.332341in}{2.402845in}}%
\pgfpathlineto{\pgfqpoint{5.333415in}{2.405398in}}%
\pgfpathlineto{\pgfqpoint{5.334489in}{2.399718in}}%
\pgfpathlineto{\pgfqpoint{5.335563in}{2.404271in}}%
\pgfpathlineto{\pgfqpoint{5.338784in}{2.402428in}}%
\pgfpathlineto{\pgfqpoint{5.340932in}{2.406241in}}%
\pgfpathlineto{\pgfqpoint{5.342006in}{2.405537in}}%
\pgfpathlineto{\pgfqpoint{5.343079in}{2.413215in}}%
\pgfpathlineto{\pgfqpoint{5.347375in}{2.413535in}}%
\pgfpathlineto{\pgfqpoint{5.348449in}{2.414496in}}%
\pgfpathlineto{\pgfqpoint{5.349523in}{2.409499in}}%
\pgfpathlineto{\pgfqpoint{5.350596in}{2.417915in}}%
\pgfpathlineto{\pgfqpoint{5.353818in}{2.415122in}}%
\pgfpathlineto{\pgfqpoint{5.354892in}{2.397711in}}%
\pgfpathlineto{\pgfqpoint{5.355966in}{2.390175in}}%
\pgfpathlineto{\pgfqpoint{5.357039in}{2.393098in}}%
\pgfpathlineto{\pgfqpoint{5.358113in}{2.400829in}}%
\pgfpathlineto{\pgfqpoint{5.361335in}{2.405659in}}%
\pgfpathlineto{\pgfqpoint{5.362409in}{2.400829in}}%
\pgfpathlineto{\pgfqpoint{5.363482in}{2.401750in}}%
\pgfpathlineto{\pgfqpoint{5.365630in}{2.413240in}}%
\pgfpathlineto{\pgfqpoint{5.368852in}{2.409989in}}%
\pgfpathlineto{\pgfqpoint{5.369925in}{2.411953in}}%
\pgfpathlineto{\pgfqpoint{5.372073in}{2.408174in}}%
\pgfpathlineto{\pgfqpoint{5.373147in}{2.408975in}}%
\pgfpathlineto{\pgfqpoint{5.377442in}{2.406632in}}%
\pgfpathlineto{\pgfqpoint{5.379590in}{2.417207in}}%
\pgfpathlineto{\pgfqpoint{5.380664in}{2.415149in}}%
\pgfpathlineto{\pgfqpoint{5.383885in}{2.418235in}}%
\pgfpathlineto{\pgfqpoint{5.384959in}{2.417550in}}%
\pgfpathlineto{\pgfqpoint{5.386033in}{2.417823in}}%
\pgfpathlineto{\pgfqpoint{5.387107in}{2.416111in}}%
\pgfpathlineto{\pgfqpoint{5.388181in}{2.415906in}}%
\pgfpathlineto{\pgfqpoint{5.391402in}{2.417549in}}%
\pgfpathlineto{\pgfqpoint{5.392476in}{2.419056in}}%
\pgfpathlineto{\pgfqpoint{5.393550in}{2.414234in}}%
\pgfpathlineto{\pgfqpoint{5.394624in}{2.415887in}}%
\pgfpathlineto{\pgfqpoint{5.395698in}{2.427596in}}%
\pgfpathlineto{\pgfqpoint{5.398919in}{2.430265in}}%
\pgfpathlineto{\pgfqpoint{5.399993in}{2.439933in}}%
\pgfpathlineto{\pgfqpoint{5.401067in}{2.438960in}}%
\pgfpathlineto{\pgfqpoint{5.402141in}{2.443528in}}%
\pgfpathlineto{\pgfqpoint{5.403214in}{2.445250in}}%
\pgfpathlineto{\pgfqpoint{5.407510in}{2.461229in}}%
\pgfpathlineto{\pgfqpoint{5.408584in}{2.467210in}}%
\pgfpathlineto{\pgfqpoint{5.409657in}{2.469663in}}%
\pgfpathlineto{\pgfqpoint{5.410731in}{2.478791in}}%
\pgfpathlineto{\pgfqpoint{5.413953in}{2.478018in}}%
\pgfpathlineto{\pgfqpoint{5.415027in}{2.484266in}}%
\pgfpathlineto{\pgfqpoint{5.417174in}{2.472200in}}%
\pgfpathlineto{\pgfqpoint{5.418248in}{2.473304in}}%
\pgfpathlineto{\pgfqpoint{5.421470in}{2.470544in}}%
\pgfpathlineto{\pgfqpoint{5.423617in}{2.478322in}}%
\pgfpathlineto{\pgfqpoint{5.424691in}{2.477527in}}%
\pgfpathlineto{\pgfqpoint{5.428987in}{2.473321in}}%
\pgfpathlineto{\pgfqpoint{5.431134in}{2.480399in}}%
\pgfpathlineto{\pgfqpoint{5.432208in}{2.479525in}}%
\pgfpathlineto{\pgfqpoint{5.433282in}{2.476358in}}%
\pgfpathlineto{\pgfqpoint{5.436503in}{2.475809in}}%
\pgfpathlineto{\pgfqpoint{5.437577in}{2.474715in}}%
\pgfpathlineto{\pgfqpoint{5.439725in}{2.475338in}}%
\pgfpathlineto{\pgfqpoint{5.440799in}{2.468784in}}%
\pgfpathlineto{\pgfqpoint{5.444020in}{2.471047in}}%
\pgfpathlineto{\pgfqpoint{5.445094in}{2.469642in}}%
\pgfpathlineto{\pgfqpoint{5.446168in}{2.469565in}}%
\pgfpathlineto{\pgfqpoint{5.447242in}{2.459553in}}%
\pgfpathlineto{\pgfqpoint{5.448316in}{2.459941in}}%
\pgfpathlineto{\pgfqpoint{5.451537in}{2.462657in}}%
\pgfpathlineto{\pgfqpoint{5.452611in}{2.461403in}}%
\pgfpathlineto{\pgfqpoint{5.454759in}{2.455211in}}%
\pgfpathlineto{\pgfqpoint{5.455833in}{2.460541in}}%
\pgfpathlineto{\pgfqpoint{5.459054in}{2.460149in}}%
\pgfpathlineto{\pgfqpoint{5.462276in}{2.465732in}}%
\pgfpathlineto{\pgfqpoint{5.463349in}{2.470960in}}%
\pgfpathlineto{\pgfqpoint{5.466571in}{2.475713in}}%
\pgfpathlineto{\pgfqpoint{5.468719in}{2.471685in}}%
\pgfpathlineto{\pgfqpoint{5.469792in}{2.477808in}}%
\pgfpathlineto{\pgfqpoint{5.470866in}{2.474424in}}%
\pgfpathlineto{\pgfqpoint{5.475162in}{2.477215in}}%
\pgfpathlineto{\pgfqpoint{5.476235in}{2.479702in}}%
\pgfpathlineto{\pgfqpoint{5.477309in}{2.479217in}}%
\pgfpathlineto{\pgfqpoint{5.478383in}{2.475979in}}%
\pgfpathlineto{\pgfqpoint{5.482679in}{2.470233in}}%
\pgfpathlineto{\pgfqpoint{5.483752in}{2.472014in}}%
\pgfpathlineto{\pgfqpoint{5.484826in}{2.472418in}}%
\pgfpathlineto{\pgfqpoint{5.485900in}{2.465772in}}%
\pgfpathlineto{\pgfqpoint{5.489122in}{2.469014in}}%
\pgfpathlineto{\pgfqpoint{5.490195in}{2.471203in}}%
\pgfpathlineto{\pgfqpoint{5.491269in}{2.471366in}}%
\pgfpathlineto{\pgfqpoint{5.492343in}{2.468425in}}%
\pgfpathlineto{\pgfqpoint{5.493417in}{2.463279in}}%
\pgfpathlineto{\pgfqpoint{5.496638in}{2.465239in}}%
\pgfpathlineto{\pgfqpoint{5.498786in}{2.452308in}}%
\pgfpathlineto{\pgfqpoint{5.499860in}{2.453165in}}%
\pgfpathlineto{\pgfqpoint{5.500934in}{2.449490in}}%
\pgfpathlineto{\pgfqpoint{5.504155in}{2.444329in}}%
\pgfpathlineto{\pgfqpoint{5.505229in}{2.451054in}}%
\pgfpathlineto{\pgfqpoint{5.507377in}{2.439449in}}%
\pgfpathlineto{\pgfqpoint{5.508451in}{2.441321in}}%
\pgfpathlineto{\pgfqpoint{5.511672in}{2.440793in}}%
\pgfpathlineto{\pgfqpoint{5.512746in}{2.436419in}}%
\pgfpathlineto{\pgfqpoint{5.514894in}{2.433780in}}%
\pgfpathlineto{\pgfqpoint{5.515967in}{2.433478in}}%
\pgfpathlineto{\pgfqpoint{5.519189in}{2.426766in}}%
\pgfpathlineto{\pgfqpoint{5.520263in}{2.426238in}}%
\pgfpathlineto{\pgfqpoint{5.521337in}{2.435967in}}%
\pgfpathlineto{\pgfqpoint{5.522411in}{2.434609in}}%
\pgfpathlineto{\pgfqpoint{5.526706in}{2.433784in}}%
\pgfpathlineto{\pgfqpoint{5.527780in}{2.422940in}}%
\pgfpathlineto{\pgfqpoint{5.528854in}{2.427460in}}%
\pgfpathlineto{\pgfqpoint{5.529927in}{2.425342in}}%
\pgfpathlineto{\pgfqpoint{5.531001in}{2.427168in}}%
\pgfpathlineto{\pgfqpoint{5.534223in}{2.421033in}}%
\pgfpathlineto{\pgfqpoint{5.537444in}{2.418616in}}%
\pgfpathlineto{\pgfqpoint{5.538518in}{2.418970in}}%
\pgfpathlineto{\pgfqpoint{5.541740in}{2.418616in}}%
\pgfpathlineto{\pgfqpoint{5.542813in}{2.415221in}}%
\pgfpathlineto{\pgfqpoint{5.544961in}{2.413523in}}%
\pgfpathlineto{\pgfqpoint{5.546035in}{2.410977in}}%
\pgfpathlineto{\pgfqpoint{5.549256in}{2.408996in}}%
\pgfpathlineto{\pgfqpoint{5.550330in}{2.410623in}}%
\pgfpathlineto{\pgfqpoint{5.551404in}{2.408218in}}%
\pgfpathlineto{\pgfqpoint{5.552478in}{2.428114in}}%
\pgfpathlineto{\pgfqpoint{5.553552in}{2.423484in}}%
\pgfpathlineto{\pgfqpoint{5.556773in}{2.421511in}}%
\pgfpathlineto{\pgfqpoint{5.557847in}{2.418247in}}%
\pgfpathlineto{\pgfqpoint{5.558921in}{2.417260in}}%
\pgfpathlineto{\pgfqpoint{5.559995in}{2.417108in}}%
\pgfpathlineto{\pgfqpoint{5.561069in}{2.415363in}}%
\pgfpathlineto{\pgfqpoint{5.565364in}{2.422421in}}%
\pgfpathlineto{\pgfqpoint{5.566438in}{2.423565in}}%
\pgfpathlineto{\pgfqpoint{5.567512in}{2.427680in}}%
\pgfpathlineto{\pgfqpoint{5.568586in}{2.426596in}}%
\pgfpathlineto{\pgfqpoint{5.571807in}{2.425976in}}%
\pgfpathlineto{\pgfqpoint{5.572881in}{2.423266in}}%
\pgfpathlineto{\pgfqpoint{5.573955in}{2.416372in}}%
\pgfpathlineto{\pgfqpoint{5.575029in}{2.414978in}}%
\pgfpathlineto{\pgfqpoint{5.576102in}{2.400882in}}%
\pgfpathlineto{\pgfqpoint{5.580398in}{2.377569in}}%
\pgfpathlineto{\pgfqpoint{5.581472in}{2.394454in}}%
\pgfpathlineto{\pgfqpoint{5.582545in}{2.390504in}}%
\pgfpathlineto{\pgfqpoint{5.583619in}{2.392332in}}%
\pgfpathlineto{\pgfqpoint{5.586841in}{2.388724in}}%
\pgfpathlineto{\pgfqpoint{5.587915in}{2.376746in}}%
\pgfpathlineto{\pgfqpoint{5.588989in}{2.382889in}}%
\pgfpathlineto{\pgfqpoint{5.590062in}{2.382121in}}%
\pgfpathlineto{\pgfqpoint{5.591136in}{2.389927in}}%
\pgfpathlineto{\pgfqpoint{5.595432in}{2.398392in}}%
\pgfpathlineto{\pgfqpoint{5.596505in}{2.408834in}}%
\pgfpathlineto{\pgfqpoint{5.597579in}{2.407678in}}%
\pgfpathlineto{\pgfqpoint{5.601875in}{2.411064in}}%
\pgfpathlineto{\pgfqpoint{5.602948in}{2.421246in}}%
\pgfpathlineto{\pgfqpoint{5.604022in}{2.416488in}}%
\pgfpathlineto{\pgfqpoint{5.605096in}{2.415424in}}%
\pgfpathlineto{\pgfqpoint{5.606170in}{2.417626in}}%
\pgfpathlineto{\pgfqpoint{5.609391in}{2.422810in}}%
\pgfpathlineto{\pgfqpoint{5.610465in}{2.426184in}}%
\pgfpathlineto{\pgfqpoint{5.611539in}{2.426768in}}%
\pgfpathlineto{\pgfqpoint{5.613687in}{2.409387in}}%
\pgfpathlineto{\pgfqpoint{5.616908in}{2.415618in}}%
\pgfpathlineto{\pgfqpoint{5.619056in}{2.421609in}}%
\pgfpathlineto{\pgfqpoint{5.620130in}{2.421689in}}%
\pgfpathlineto{\pgfqpoint{5.621204in}{2.425076in}}%
\pgfpathlineto{\pgfqpoint{5.626573in}{2.415972in}}%
\pgfpathlineto{\pgfqpoint{5.627647in}{2.411304in}}%
\pgfpathlineto{\pgfqpoint{5.628721in}{2.410769in}}%
\pgfpathlineto{\pgfqpoint{5.631942in}{2.411760in}}%
\pgfpathlineto{\pgfqpoint{5.633016in}{2.410230in}}%
\pgfpathlineto{\pgfqpoint{5.634090in}{2.410918in}}%
\pgfpathlineto{\pgfqpoint{5.635164in}{2.410536in}}%
\pgfpathlineto{\pgfqpoint{5.636237in}{2.406257in}}%
\pgfpathlineto{\pgfqpoint{5.639459in}{2.404530in}}%
\pgfpathlineto{\pgfqpoint{5.640533in}{2.409377in}}%
\pgfpathlineto{\pgfqpoint{5.641607in}{2.408237in}}%
\pgfpathlineto{\pgfqpoint{5.642680in}{2.416756in}}%
\pgfpathlineto{\pgfqpoint{5.643754in}{2.402075in}}%
\pgfpathlineto{\pgfqpoint{5.646976in}{2.399133in}}%
\pgfpathlineto{\pgfqpoint{5.648050in}{2.400409in}}%
\pgfpathlineto{\pgfqpoint{5.649123in}{2.395490in}}%
\pgfpathlineto{\pgfqpoint{5.651271in}{2.402762in}}%
\pgfpathlineto{\pgfqpoint{5.654493in}{2.404211in}}%
\pgfpathlineto{\pgfqpoint{5.655567in}{2.401312in}}%
\pgfpathlineto{\pgfqpoint{5.656640in}{2.401240in}}%
\pgfpathlineto{\pgfqpoint{5.658788in}{2.391715in}}%
\pgfpathlineto{\pgfqpoint{5.662010in}{2.390641in}}%
\pgfpathlineto{\pgfqpoint{5.663083in}{2.393004in}}%
\pgfpathlineto{\pgfqpoint{5.664157in}{2.391787in}}%
\pgfpathlineto{\pgfqpoint{5.665231in}{2.399981in}}%
\pgfpathlineto{\pgfqpoint{5.666305in}{2.395410in}}%
\pgfpathlineto{\pgfqpoint{5.671674in}{2.411073in}}%
\pgfpathlineto{\pgfqpoint{5.672748in}{2.408997in}}%
\pgfpathlineto{\pgfqpoint{5.673822in}{2.411571in}}%
\pgfpathlineto{\pgfqpoint{5.677043in}{2.412537in}}%
\pgfpathlineto{\pgfqpoint{5.678117in}{2.409341in}}%
\pgfpathlineto{\pgfqpoint{5.679191in}{2.412937in}}%
\pgfpathlineto{\pgfqpoint{5.681339in}{2.411597in}}%
\pgfpathlineto{\pgfqpoint{5.684560in}{2.405939in}}%
\pgfpathlineto{\pgfqpoint{5.685634in}{2.413160in}}%
\pgfpathlineto{\pgfqpoint{5.686708in}{2.414128in}}%
\pgfpathlineto{\pgfqpoint{5.687782in}{2.413829in}}%
\pgfpathlineto{\pgfqpoint{5.688856in}{2.427655in}}%
\pgfpathlineto{\pgfqpoint{5.692077in}{2.424068in}}%
\pgfpathlineto{\pgfqpoint{5.693151in}{2.428122in}}%
\pgfpathlineto{\pgfqpoint{5.694225in}{2.427822in}}%
\pgfpathlineto{\pgfqpoint{5.696372in}{2.428496in}}%
\pgfpathlineto{\pgfqpoint{5.699594in}{2.431866in}}%
\pgfpathlineto{\pgfqpoint{5.701742in}{2.414195in}}%
\pgfpathlineto{\pgfqpoint{5.702815in}{2.418602in}}%
\pgfpathlineto{\pgfqpoint{5.703889in}{2.405006in}}%
\pgfpathlineto{\pgfqpoint{5.707111in}{2.410131in}}%
\pgfpathlineto{\pgfqpoint{5.709258in}{2.403973in}}%
\pgfpathlineto{\pgfqpoint{5.710332in}{2.404806in}}%
\pgfpathlineto{\pgfqpoint{5.714628in}{2.405642in}}%
\pgfpathlineto{\pgfqpoint{5.715701in}{2.402925in}}%
\pgfpathlineto{\pgfqpoint{5.716775in}{2.404716in}}%
\pgfpathlineto{\pgfqpoint{5.717849in}{2.400621in}}%
\pgfpathlineto{\pgfqpoint{5.722144in}{2.394235in}}%
\pgfpathlineto{\pgfqpoint{5.723218in}{2.395762in}}%
\pgfpathlineto{\pgfqpoint{5.724292in}{2.400413in}}%
\pgfpathlineto{\pgfqpoint{5.725366in}{2.396096in}}%
\pgfpathlineto{\pgfqpoint{5.726440in}{2.388525in}}%
\pgfpathlineto{\pgfqpoint{5.730735in}{2.393903in}}%
\pgfpathlineto{\pgfqpoint{5.731809in}{2.389711in}}%
\pgfpathlineto{\pgfqpoint{5.732883in}{2.391629in}}%
\pgfpathlineto{\pgfqpoint{5.733957in}{2.399019in}}%
\pgfpathlineto{\pgfqpoint{5.739326in}{2.411915in}}%
\pgfpathlineto{\pgfqpoint{5.740400in}{2.417813in}}%
\pgfpathlineto{\pgfqpoint{5.741474in}{2.413610in}}%
\pgfpathlineto{\pgfqpoint{5.744695in}{2.416873in}}%
\pgfpathlineto{\pgfqpoint{5.745769in}{2.429729in}}%
\pgfpathlineto{\pgfqpoint{5.746843in}{2.429803in}}%
\pgfpathlineto{\pgfqpoint{5.747917in}{2.436490in}}%
\pgfpathlineto{\pgfqpoint{5.748990in}{2.424144in}}%
\pgfpathlineto{\pgfqpoint{5.752212in}{2.427720in}}%
\pgfpathlineto{\pgfqpoint{5.753286in}{2.421962in}}%
\pgfpathlineto{\pgfqpoint{5.754360in}{2.427578in}}%
\pgfpathlineto{\pgfqpoint{5.756507in}{2.433368in}}%
\pgfpathlineto{\pgfqpoint{5.759729in}{2.424249in}}%
\pgfpathlineto{\pgfqpoint{5.760803in}{2.424111in}}%
\pgfpathlineto{\pgfqpoint{5.761877in}{2.417805in}}%
\pgfpathlineto{\pgfqpoint{5.762950in}{2.407133in}}%
\pgfpathlineto{\pgfqpoint{5.764024in}{2.413855in}}%
\pgfpathlineto{\pgfqpoint{5.768320in}{2.410875in}}%
\pgfpathlineto{\pgfqpoint{5.769393in}{2.404849in}}%
\pgfpathlineto{\pgfqpoint{5.770467in}{2.407656in}}%
\pgfpathlineto{\pgfqpoint{5.771541in}{2.406507in}}%
\pgfpathlineto{\pgfqpoint{5.774763in}{2.408535in}}%
\pgfpathlineto{\pgfqpoint{5.775836in}{2.410429in}}%
\pgfpathlineto{\pgfqpoint{5.776910in}{2.408862in}}%
\pgfpathlineto{\pgfqpoint{5.777984in}{2.413631in}}%
\pgfpathlineto{\pgfqpoint{5.779058in}{2.421193in}}%
\pgfpathlineto{\pgfqpoint{5.782279in}{2.416138in}}%
\pgfpathlineto{\pgfqpoint{5.783353in}{2.422036in}}%
\pgfpathlineto{\pgfqpoint{5.784427in}{2.413750in}}%
\pgfpathlineto{\pgfqpoint{5.785501in}{2.411983in}}%
\pgfpathlineto{\pgfqpoint{5.786575in}{2.408069in}}%
\pgfpathlineto{\pgfqpoint{5.789796in}{2.410394in}}%
\pgfpathlineto{\pgfqpoint{5.791944in}{2.409589in}}%
\pgfpathlineto{\pgfqpoint{5.794092in}{2.402278in}}%
\pgfpathlineto{\pgfqpoint{5.797313in}{2.398791in}}%
\pgfpathlineto{\pgfqpoint{5.798387in}{2.399663in}}%
\pgfpathlineto{\pgfqpoint{5.799461in}{2.399461in}}%
\pgfpathlineto{\pgfqpoint{5.800535in}{2.391017in}}%
\pgfpathlineto{\pgfqpoint{5.801609in}{2.388687in}}%
\pgfpathlineto{\pgfqpoint{5.804830in}{2.387725in}}%
\pgfpathlineto{\pgfqpoint{5.805904in}{2.391046in}}%
\pgfpathlineto{\pgfqpoint{5.806978in}{2.391499in}}%
\pgfpathlineto{\pgfqpoint{5.808052in}{2.391110in}}%
\pgfpathlineto{\pgfqpoint{5.812347in}{2.392662in}}%
\pgfpathlineto{\pgfqpoint{5.814495in}{2.394353in}}%
\pgfpathlineto{\pgfqpoint{5.815568in}{2.392135in}}%
\pgfpathlineto{\pgfqpoint{5.816642in}{2.399312in}}%
\pgfpathlineto{\pgfqpoint{5.819864in}{2.401204in}}%
\pgfpathlineto{\pgfqpoint{5.820938in}{2.400941in}}%
\pgfpathlineto{\pgfqpoint{5.823085in}{2.408107in}}%
\pgfpathlineto{\pgfqpoint{5.824159in}{2.407841in}}%
\pgfpathlineto{\pgfqpoint{5.827381in}{2.405042in}}%
\pgfpathlineto{\pgfqpoint{5.828455in}{2.405641in}}%
\pgfpathlineto{\pgfqpoint{5.830602in}{2.409186in}}%
\pgfpathlineto{\pgfqpoint{5.831676in}{2.407437in}}%
\pgfpathlineto{\pgfqpoint{5.834898in}{2.404232in}}%
\pgfpathlineto{\pgfqpoint{5.835971in}{2.401596in}}%
\pgfpathlineto{\pgfqpoint{5.837045in}{2.411769in}}%
\pgfpathlineto{\pgfqpoint{5.838119in}{2.407146in}}%
\pgfpathlineto{\pgfqpoint{5.839193in}{2.408098in}}%
\pgfpathlineto{\pgfqpoint{5.842414in}{2.405243in}}%
\pgfpathlineto{\pgfqpoint{5.843488in}{2.416533in}}%
\pgfpathlineto{\pgfqpoint{5.844562in}{2.418643in}}%
\pgfpathlineto{\pgfqpoint{5.845636in}{2.419417in}}%
\pgfpathlineto{\pgfqpoint{5.846710in}{2.421675in}}%
\pgfpathlineto{\pgfqpoint{5.852079in}{2.412315in}}%
\pgfpathlineto{\pgfqpoint{5.853153in}{2.414151in}}%
\pgfpathlineto{\pgfqpoint{5.854227in}{2.419289in}}%
\pgfpathlineto{\pgfqpoint{5.857448in}{2.419358in}}%
\pgfpathlineto{\pgfqpoint{5.860670in}{2.425264in}}%
\pgfpathlineto{\pgfqpoint{5.861744in}{2.432656in}}%
\pgfpathlineto{\pgfqpoint{5.864965in}{2.435215in}}%
\pgfpathlineto{\pgfqpoint{5.866039in}{2.441685in}}%
\pgfpathlineto{\pgfqpoint{5.867113in}{2.436656in}}%
\pgfpathlineto{\pgfqpoint{5.869260in}{2.439936in}}%
\pgfpathlineto{\pgfqpoint{5.872482in}{2.441107in}}%
\pgfpathlineto{\pgfqpoint{5.874630in}{2.432683in}}%
\pgfpathlineto{\pgfqpoint{5.876777in}{2.435739in}}%
\pgfpathlineto{\pgfqpoint{5.881073in}{2.438236in}}%
\pgfpathlineto{\pgfqpoint{5.882146in}{2.443134in}}%
\pgfpathlineto{\pgfqpoint{5.883220in}{2.442125in}}%
\pgfpathlineto{\pgfqpoint{5.884294in}{2.438753in}}%
\pgfpathlineto{\pgfqpoint{5.887516in}{2.436840in}}%
\pgfpathlineto{\pgfqpoint{5.889663in}{2.441715in}}%
\pgfpathlineto{\pgfqpoint{5.890737in}{2.438374in}}%
\pgfpathlineto{\pgfqpoint{5.891811in}{2.438163in}}%
\pgfpathlineto{\pgfqpoint{5.895032in}{2.442162in}}%
\pgfpathlineto{\pgfqpoint{5.897180in}{2.449782in}}%
\pgfpathlineto{\pgfqpoint{5.899328in}{2.454527in}}%
\pgfpathlineto{\pgfqpoint{5.902549in}{2.453948in}}%
\pgfpathlineto{\pgfqpoint{5.903623in}{2.456337in}}%
\pgfpathlineto{\pgfqpoint{5.904697in}{2.455251in}}%
\pgfpathlineto{\pgfqpoint{5.905771in}{2.451142in}}%
\pgfpathlineto{\pgfqpoint{5.906845in}{2.463640in}}%
\pgfpathlineto{\pgfqpoint{5.910066in}{2.456723in}}%
\pgfpathlineto{\pgfqpoint{5.911140in}{2.464979in}}%
\pgfpathlineto{\pgfqpoint{5.913288in}{2.450281in}}%
\pgfpathlineto{\pgfqpoint{5.914362in}{2.449575in}}%
\pgfpathlineto{\pgfqpoint{5.918657in}{2.445426in}}%
\pgfpathlineto{\pgfqpoint{5.919731in}{2.447989in}}%
\pgfpathlineto{\pgfqpoint{5.920805in}{2.446730in}}%
\pgfpathlineto{\pgfqpoint{5.921878in}{2.452673in}}%
\pgfpathlineto{\pgfqpoint{5.926174in}{2.452813in}}%
\pgfpathlineto{\pgfqpoint{5.927248in}{2.453722in}}%
\pgfpathlineto{\pgfqpoint{5.928321in}{2.453862in}}%
\pgfpathlineto{\pgfqpoint{5.929395in}{2.454772in}}%
\pgfpathlineto{\pgfqpoint{5.932617in}{2.455262in}}%
\pgfpathlineto{\pgfqpoint{5.934765in}{2.457856in}}%
\pgfpathlineto{\pgfqpoint{5.935838in}{2.457434in}}%
\pgfpathlineto{\pgfqpoint{5.936912in}{2.460391in}}%
\pgfpathlineto{\pgfqpoint{5.940134in}{2.459898in}}%
\pgfpathlineto{\pgfqpoint{5.941208in}{2.463271in}}%
\pgfpathlineto{\pgfqpoint{5.942281in}{2.458077in}}%
\pgfpathlineto{\pgfqpoint{5.943355in}{2.460283in}}%
\pgfpathlineto{\pgfqpoint{5.944429in}{2.455231in}}%
\pgfpathlineto{\pgfqpoint{5.948724in}{2.447800in}}%
\pgfpathlineto{\pgfqpoint{5.949798in}{2.452693in}}%
\pgfpathlineto{\pgfqpoint{5.950872in}{2.453178in}}%
\pgfpathlineto{\pgfqpoint{5.951946in}{2.454839in}}%
\pgfpathlineto{\pgfqpoint{5.955167in}{2.454700in}}%
\pgfpathlineto{\pgfqpoint{5.956241in}{2.456162in}}%
\pgfpathlineto{\pgfqpoint{5.959463in}{2.449599in}}%
\pgfpathlineto{\pgfqpoint{5.962684in}{2.449735in}}%
\pgfpathlineto{\pgfqpoint{5.963758in}{2.447015in}}%
\pgfpathlineto{\pgfqpoint{5.964832in}{2.449395in}}%
\pgfpathlineto{\pgfqpoint{5.965906in}{2.446403in}}%
\pgfpathlineto{\pgfqpoint{5.966980in}{2.447210in}}%
\pgfpathlineto{\pgfqpoint{5.970201in}{2.444376in}}%
\pgfpathlineto{\pgfqpoint{5.971275in}{2.447749in}}%
\pgfpathlineto{\pgfqpoint{5.972349in}{2.448289in}}%
\pgfpathlineto{\pgfqpoint{5.974497in}{2.453967in}}%
\pgfpathlineto{\pgfqpoint{5.977718in}{2.458462in}}%
\pgfpathlineto{\pgfqpoint{5.978792in}{2.463161in}}%
\pgfpathlineto{\pgfqpoint{5.979866in}{2.461706in}}%
\pgfpathlineto{\pgfqpoint{5.980940in}{2.468011in}}%
\pgfpathlineto{\pgfqpoint{5.986309in}{2.471482in}}%
\pgfpathlineto{\pgfqpoint{5.987383in}{2.475858in}}%
\pgfpathlineto{\pgfqpoint{5.988456in}{2.474800in}}%
\pgfpathlineto{\pgfqpoint{5.989530in}{2.464923in}}%
\pgfpathlineto{\pgfqpoint{5.992752in}{2.477763in}}%
\pgfpathlineto{\pgfqpoint{5.993826in}{2.485453in}}%
\pgfpathlineto{\pgfqpoint{5.994899in}{2.485235in}}%
\pgfpathlineto{\pgfqpoint{5.995973in}{2.492122in}}%
\pgfpathlineto{\pgfqpoint{5.997047in}{2.492195in}}%
\pgfpathlineto{\pgfqpoint{6.000269in}{2.494297in}}%
\pgfpathlineto{\pgfqpoint{6.001343in}{2.492920in}}%
\pgfpathlineto{\pgfqpoint{6.002416in}{2.498044in}}%
\pgfpathlineto{\pgfqpoint{6.004564in}{2.514277in}}%
\pgfpathlineto{\pgfqpoint{6.007786in}{2.514881in}}%
\pgfpathlineto{\pgfqpoint{6.008859in}{2.511407in}}%
\pgfpathlineto{\pgfqpoint{6.009933in}{2.503942in}}%
\pgfpathlineto{\pgfqpoint{6.012081in}{2.522351in}}%
\pgfpathlineto{\pgfqpoint{6.015302in}{2.529759in}}%
\pgfpathlineto{\pgfqpoint{6.016376in}{2.527612in}}%
\pgfpathlineto{\pgfqpoint{6.017450in}{2.531062in}}%
\pgfpathlineto{\pgfqpoint{6.018524in}{2.528455in}}%
\pgfpathlineto{\pgfqpoint{6.019598in}{2.523058in}}%
\pgfpathlineto{\pgfqpoint{6.022819in}{2.529482in}}%
\pgfpathlineto{\pgfqpoint{6.023893in}{2.525897in}}%
\pgfpathlineto{\pgfqpoint{6.024967in}{2.526126in}}%
\pgfpathlineto{\pgfqpoint{6.026041in}{2.523761in}}%
\pgfpathlineto{\pgfqpoint{6.030336in}{2.529330in}}%
\pgfpathlineto{\pgfqpoint{6.031410in}{2.526700in}}%
\pgfpathlineto{\pgfqpoint{6.033558in}{2.513553in}}%
\pgfpathlineto{\pgfqpoint{6.034632in}{2.509299in}}%
\pgfpathlineto{\pgfqpoint{6.037853in}{2.507675in}}%
\pgfpathlineto{\pgfqpoint{6.038927in}{2.528015in}}%
\pgfpathlineto{\pgfqpoint{6.040001in}{2.524999in}}%
\pgfpathlineto{\pgfqpoint{6.041075in}{2.530743in}}%
\pgfpathlineto{\pgfqpoint{6.042148in}{2.532616in}}%
\pgfpathlineto{\pgfqpoint{6.046444in}{2.533318in}}%
\pgfpathlineto{\pgfqpoint{6.047518in}{2.534099in}}%
\pgfpathlineto{\pgfqpoint{6.048591in}{2.533081in}}%
\pgfpathlineto{\pgfqpoint{6.049665in}{2.522196in}}%
\pgfpathlineto{\pgfqpoint{6.052887in}{2.532768in}}%
\pgfpathlineto{\pgfqpoint{6.053961in}{2.526269in}}%
\pgfpathlineto{\pgfqpoint{6.055034in}{2.537005in}}%
\pgfpathlineto{\pgfqpoint{6.056108in}{2.515325in}}%
\pgfpathlineto{\pgfqpoint{6.057182in}{2.519852in}}%
\pgfpathlineto{\pgfqpoint{6.060404in}{2.524060in}}%
\pgfpathlineto{\pgfqpoint{6.062551in}{2.531871in}}%
\pgfpathlineto{\pgfqpoint{6.063625in}{2.530976in}}%
\pgfpathlineto{\pgfqpoint{6.064699in}{2.523002in}}%
\pgfpathlineto{\pgfqpoint{6.067920in}{2.527803in}}%
\pgfpathlineto{\pgfqpoint{6.068994in}{2.526908in}}%
\pgfpathlineto{\pgfqpoint{6.070068in}{2.527476in}}%
\pgfpathlineto{\pgfqpoint{6.072216in}{2.533246in}}%
\pgfpathlineto{\pgfqpoint{6.075437in}{2.536172in}}%
\pgfpathlineto{\pgfqpoint{6.076511in}{2.534860in}}%
\pgfpathlineto{\pgfqpoint{6.078659in}{2.527065in}}%
\pgfpathlineto{\pgfqpoint{6.079733in}{2.531167in}}%
\pgfpathlineto{\pgfqpoint{6.082954in}{2.526737in}}%
\pgfpathlineto{\pgfqpoint{6.084028in}{2.527303in}}%
\pgfpathlineto{\pgfqpoint{6.086176in}{2.541727in}}%
\pgfpathlineto{\pgfqpoint{6.087250in}{2.548314in}}%
\pgfpathlineto{\pgfqpoint{6.090471in}{2.542632in}}%
\pgfpathlineto{\pgfqpoint{6.091545in}{2.542228in}}%
\pgfpathlineto{\pgfqpoint{6.092619in}{2.548206in}}%
\pgfpathlineto{\pgfqpoint{6.093693in}{2.550511in}}%
\pgfpathlineto{\pgfqpoint{6.097988in}{2.550758in}}%
\pgfpathlineto{\pgfqpoint{6.099062in}{2.551828in}}%
\pgfpathlineto{\pgfqpoint{6.100136in}{2.549598in}}%
\pgfpathlineto{\pgfqpoint{6.101209in}{2.551085in}}%
\pgfpathlineto{\pgfqpoint{6.102283in}{2.547368in}}%
\pgfpathlineto{\pgfqpoint{6.106579in}{2.550062in}}%
\pgfpathlineto{\pgfqpoint{6.107653in}{2.546027in}}%
\pgfpathlineto{\pgfqpoint{6.109800in}{2.550227in}}%
\pgfpathlineto{\pgfqpoint{6.114096in}{2.551138in}}%
\pgfpathlineto{\pgfqpoint{6.115169in}{2.548818in}}%
\pgfpathlineto{\pgfqpoint{6.116243in}{2.544622in}}%
\pgfpathlineto{\pgfqpoint{6.117317in}{2.544865in}}%
\pgfpathlineto{\pgfqpoint{6.120539in}{2.540152in}}%
\pgfpathlineto{\pgfqpoint{6.121612in}{2.533407in}}%
\pgfpathlineto{\pgfqpoint{6.122686in}{2.535357in}}%
\pgfpathlineto{\pgfqpoint{6.124834in}{2.533409in}}%
\pgfpathlineto{\pgfqpoint{6.130203in}{2.521700in}}%
\pgfpathlineto{\pgfqpoint{6.131277in}{2.523411in}}%
\pgfpathlineto{\pgfqpoint{6.132351in}{2.543275in}}%
\pgfpathlineto{\pgfqpoint{6.135572in}{2.546794in}}%
\pgfpathlineto{\pgfqpoint{6.137720in}{2.558419in}}%
\pgfpathlineto{\pgfqpoint{6.138794in}{2.554241in}}%
\pgfpathlineto{\pgfqpoint{6.139868in}{2.555125in}}%
\pgfpathlineto{\pgfqpoint{6.143089in}{2.554884in}}%
\pgfpathlineto{\pgfqpoint{6.144163in}{2.561145in}}%
\pgfpathlineto{\pgfqpoint{6.145237in}{2.563152in}}%
\pgfpathlineto{\pgfqpoint{6.147385in}{2.568968in}}%
\pgfpathlineto{\pgfqpoint{6.150606in}{2.568886in}}%
\pgfpathlineto{\pgfqpoint{6.151680in}{2.573446in}}%
\pgfpathlineto{\pgfqpoint{6.153828in}{2.568498in}}%
\pgfpathlineto{\pgfqpoint{6.154901in}{2.573631in}}%
\pgfpathlineto{\pgfqpoint{6.158123in}{2.576156in}}%
\pgfpathlineto{\pgfqpoint{6.159197in}{2.579578in}}%
\pgfpathlineto{\pgfqpoint{6.160271in}{2.604435in}}%
\pgfpathlineto{\pgfqpoint{6.162418in}{2.609309in}}%
\pgfpathlineto{\pgfqpoint{6.166714in}{2.604914in}}%
\pgfpathlineto{\pgfqpoint{6.167787in}{2.606633in}}%
\pgfpathlineto{\pgfqpoint{6.168861in}{2.604329in}}%
\pgfpathlineto{\pgfqpoint{6.169935in}{2.603671in}}%
\pgfpathlineto{\pgfqpoint{6.173157in}{2.602437in}}%
\pgfpathlineto{\pgfqpoint{6.174231in}{2.603836in}}%
\pgfpathlineto{\pgfqpoint{6.175304in}{2.599310in}}%
\pgfpathlineto{\pgfqpoint{6.176378in}{2.604995in}}%
\pgfpathlineto{\pgfqpoint{6.177452in}{2.601857in}}%
\pgfpathlineto{\pgfqpoint{6.181747in}{2.600289in}}%
\pgfpathlineto{\pgfqpoint{6.182821in}{2.599133in}}%
\pgfpathlineto{\pgfqpoint{6.183895in}{2.600701in}}%
\pgfpathlineto{\pgfqpoint{6.184969in}{2.595169in}}%
\pgfpathlineto{\pgfqpoint{6.188190in}{2.593381in}}%
\pgfpathlineto{\pgfqpoint{6.190338in}{2.598656in}}%
\pgfpathlineto{\pgfqpoint{6.191412in}{2.598411in}}%
\pgfpathlineto{\pgfqpoint{6.192486in}{2.595159in}}%
\pgfpathlineto{\pgfqpoint{6.196781in}{2.597029in}}%
\pgfpathlineto{\pgfqpoint{6.200003in}{2.592391in}}%
\pgfpathlineto{\pgfqpoint{6.203224in}{2.591740in}}%
\pgfpathlineto{\pgfqpoint{6.205372in}{2.594913in}}%
\pgfpathlineto{\pgfqpoint{6.207520in}{2.589198in}}%
\pgfpathlineto{\pgfqpoint{6.210741in}{2.587974in}}%
\pgfpathlineto{\pgfqpoint{6.211815in}{2.589688in}}%
\pgfpathlineto{\pgfqpoint{6.212889in}{2.589280in}}%
\pgfpathlineto{\pgfqpoint{6.213963in}{2.593520in}}%
\pgfpathlineto{\pgfqpoint{6.215036in}{2.592199in}}%
\pgfpathlineto{\pgfqpoint{6.218258in}{2.594181in}}%
\pgfpathlineto{\pgfqpoint{6.219332in}{2.591786in}}%
\pgfpathlineto{\pgfqpoint{6.220406in}{2.587933in}}%
\pgfpathlineto{\pgfqpoint{6.221479in}{2.590040in}}%
\pgfpathlineto{\pgfqpoint{6.225775in}{2.592733in}}%
\pgfpathlineto{\pgfqpoint{6.226849in}{2.589633in}}%
\pgfpathlineto{\pgfqpoint{6.227922in}{2.593432in}}%
\pgfpathlineto{\pgfqpoint{6.228996in}{2.591307in}}%
\pgfpathlineto{\pgfqpoint{6.230070in}{2.585911in}}%
\pgfpathlineto{\pgfqpoint{6.233292in}{2.593023in}}%
\pgfpathlineto{\pgfqpoint{6.234365in}{2.589590in}}%
\pgfpathlineto{\pgfqpoint{6.235439in}{2.606666in}}%
\pgfpathlineto{\pgfqpoint{6.236513in}{2.606326in}}%
\pgfpathlineto{\pgfqpoint{6.237587in}{2.603432in}}%
\pgfpathlineto{\pgfqpoint{6.240809in}{2.601559in}}%
\pgfpathlineto{\pgfqpoint{6.241882in}{2.594666in}}%
\pgfpathlineto{\pgfqpoint{6.242956in}{2.596283in}}%
\pgfpathlineto{\pgfqpoint{6.244030in}{2.596112in}}%
\pgfpathlineto{\pgfqpoint{6.245104in}{2.596538in}}%
\pgfpathlineto{\pgfqpoint{6.248325in}{2.596112in}}%
\pgfpathlineto{\pgfqpoint{6.250473in}{2.597984in}}%
\pgfpathlineto{\pgfqpoint{6.251547in}{2.600542in}}%
\pgfpathlineto{\pgfqpoint{6.252621in}{2.600713in}}%
\pgfpathlineto{\pgfqpoint{6.257990in}{2.602689in}}%
\pgfpathlineto{\pgfqpoint{6.259064in}{2.599505in}}%
\pgfpathlineto{\pgfqpoint{6.260138in}{2.602517in}}%
\pgfpathlineto{\pgfqpoint{6.263359in}{2.602776in}}%
\pgfpathlineto{\pgfqpoint{6.264433in}{2.601742in}}%
\pgfpathlineto{\pgfqpoint{6.265507in}{2.605101in}}%
\pgfpathlineto{\pgfqpoint{6.267654in}{2.599098in}}%
\pgfpathlineto{\pgfqpoint{6.271950in}{2.597913in}}%
\pgfpathlineto{\pgfqpoint{6.273024in}{2.592514in}}%
\pgfpathlineto{\pgfqpoint{6.274097in}{2.592182in}}%
\pgfpathlineto{\pgfqpoint{6.275171in}{2.588697in}}%
\pgfpathlineto{\pgfqpoint{6.280541in}{2.587302in}}%
\pgfpathlineto{\pgfqpoint{6.281614in}{2.580350in}}%
\pgfpathlineto{\pgfqpoint{6.282688in}{2.582640in}}%
\pgfpathlineto{\pgfqpoint{6.285910in}{2.581986in}}%
\pgfpathlineto{\pgfqpoint{6.286984in}{2.583374in}}%
\pgfpathlineto{\pgfqpoint{6.288057in}{2.586324in}}%
\pgfpathlineto{\pgfqpoint{6.289131in}{2.579194in}}%
\pgfpathlineto{\pgfqpoint{6.290205in}{2.577107in}}%
\pgfpathlineto{\pgfqpoint{6.293427in}{2.575592in}}%
\pgfpathlineto{\pgfqpoint{6.294500in}{2.577339in}}%
\pgfpathlineto{\pgfqpoint{6.295574in}{2.575663in}}%
\pgfpathlineto{\pgfqpoint{6.296648in}{2.572870in}}%
\pgfpathlineto{\pgfqpoint{6.297722in}{2.575822in}}%
\pgfpathlineto{\pgfqpoint{6.300943in}{2.576301in}}%
\pgfpathlineto{\pgfqpoint{6.302017in}{2.570707in}}%
\pgfpathlineto{\pgfqpoint{6.303091in}{2.568949in}}%
\pgfpathlineto{\pgfqpoint{6.304165in}{2.558719in}}%
\pgfpathlineto{\pgfqpoint{6.305239in}{2.559838in}}%
\pgfpathlineto{\pgfqpoint{6.308460in}{2.555443in}}%
\pgfpathlineto{\pgfqpoint{6.310608in}{2.555916in}}%
\pgfpathlineto{\pgfqpoint{6.311682in}{2.553705in}}%
\pgfpathlineto{\pgfqpoint{6.312756in}{2.555758in}}%
\pgfpathlineto{\pgfqpoint{6.315977in}{2.559864in}}%
\pgfpathlineto{\pgfqpoint{6.317051in}{2.557465in}}%
\pgfpathlineto{\pgfqpoint{6.319199in}{2.560823in}}%
\pgfpathlineto{\pgfqpoint{6.320273in}{2.563800in}}%
\pgfpathlineto{\pgfqpoint{6.323494in}{2.560421in}}%
\pgfpathlineto{\pgfqpoint{6.324568in}{2.552380in}}%
\pgfpathlineto{\pgfqpoint{6.326716in}{2.547658in}}%
\pgfpathlineto{\pgfqpoint{6.327789in}{2.547582in}}%
\pgfpathlineto{\pgfqpoint{6.331011in}{2.544521in}}%
\pgfpathlineto{\pgfqpoint{6.332085in}{2.551408in}}%
\pgfpathlineto{\pgfqpoint{6.333159in}{2.550260in}}%
\pgfpathlineto{\pgfqpoint{6.334232in}{2.540425in}}%
\pgfpathlineto{\pgfqpoint{6.335306in}{2.543678in}}%
\pgfpathlineto{\pgfqpoint{6.338528in}{2.547936in}}%
\pgfpathlineto{\pgfqpoint{6.339602in}{2.545919in}}%
\pgfpathlineto{\pgfqpoint{6.340675in}{2.546364in}}%
\pgfpathlineto{\pgfqpoint{6.342823in}{2.543763in}}%
\pgfpathlineto{\pgfqpoint{6.346045in}{2.549189in}}%
\pgfpathlineto{\pgfqpoint{6.347119in}{2.548222in}}%
\pgfpathlineto{\pgfqpoint{6.348192in}{2.545851in}}%
\pgfpathlineto{\pgfqpoint{6.349266in}{2.547101in}}%
\pgfpathlineto{\pgfqpoint{6.350340in}{2.544295in}}%
\pgfpathlineto{\pgfqpoint{6.353562in}{2.546437in}}%
\pgfpathlineto{\pgfqpoint{6.354635in}{2.542597in}}%
\pgfpathlineto{\pgfqpoint{6.355709in}{2.540919in}}%
\pgfpathlineto{\pgfqpoint{6.357857in}{2.546166in}}%
\pgfpathlineto{\pgfqpoint{6.361078in}{2.543459in}}%
\pgfpathlineto{\pgfqpoint{6.362152in}{2.543967in}}%
\pgfpathlineto{\pgfqpoint{6.364300in}{2.540479in}}%
\pgfpathlineto{\pgfqpoint{6.365374in}{2.542078in}}%
\pgfpathlineto{\pgfqpoint{6.368595in}{2.542368in}}%
\pgfpathlineto{\pgfqpoint{6.369669in}{2.541350in}}%
\pgfpathlineto{\pgfqpoint{6.370743in}{2.538296in}}%
\pgfpathlineto{\pgfqpoint{6.371817in}{2.541059in}}%
\pgfpathlineto{\pgfqpoint{6.372891in}{2.539314in}}%
\pgfpathlineto{\pgfqpoint{6.378260in}{2.538012in}}%
\pgfpathlineto{\pgfqpoint{6.380408in}{2.540533in}}%
\pgfpathlineto{\pgfqpoint{6.384703in}{2.551582in}}%
\pgfpathlineto{\pgfqpoint{6.386851in}{2.551874in}}%
\pgfpathlineto{\pgfqpoint{6.387924in}{2.549904in}}%
\pgfpathlineto{\pgfqpoint{6.391146in}{2.549102in}}%
\pgfpathlineto{\pgfqpoint{6.392220in}{2.556105in}}%
\pgfpathlineto{\pgfqpoint{6.393294in}{2.554573in}}%
\pgfpathlineto{\pgfqpoint{6.394367in}{2.566554in}}%
\pgfpathlineto{\pgfqpoint{6.395441in}{2.563766in}}%
\pgfpathlineto{\pgfqpoint{6.399737in}{2.567685in}}%
\pgfpathlineto{\pgfqpoint{6.400810in}{2.555078in}}%
\pgfpathlineto{\pgfqpoint{6.402958in}{2.555833in}}%
\pgfpathlineto{\pgfqpoint{6.408327in}{2.545663in}}%
\pgfpathlineto{\pgfqpoint{6.410475in}{2.550434in}}%
\pgfpathlineto{\pgfqpoint{6.413697in}{2.551909in}}%
\pgfpathlineto{\pgfqpoint{6.414770in}{2.548500in}}%
\pgfpathlineto{\pgfqpoint{6.415844in}{2.547389in}}%
\pgfpathlineto{\pgfqpoint{6.416918in}{2.552131in}}%
\pgfpathlineto{\pgfqpoint{6.417992in}{2.545907in}}%
\pgfpathlineto{\pgfqpoint{6.421213in}{2.545253in}}%
\pgfpathlineto{\pgfqpoint{6.422287in}{2.547574in}}%
\pgfpathlineto{\pgfqpoint{6.423361in}{2.547354in}}%
\pgfpathlineto{\pgfqpoint{6.424435in}{2.543994in}}%
\pgfpathlineto{\pgfqpoint{6.425509in}{2.520907in}}%
\pgfpathlineto{\pgfqpoint{6.429804in}{2.512140in}}%
\pgfpathlineto{\pgfqpoint{6.430878in}{2.511337in}}%
\pgfpathlineto{\pgfqpoint{6.433026in}{2.518862in}}%
\pgfpathlineto{\pgfqpoint{6.436247in}{2.513473in}}%
\pgfpathlineto{\pgfqpoint{6.437321in}{2.513990in}}%
\pgfpathlineto{\pgfqpoint{6.438395in}{2.510151in}}%
\pgfpathlineto{\pgfqpoint{6.439469in}{2.512775in}}%
\pgfpathlineto{\pgfqpoint{6.440542in}{2.513216in}}%
\pgfpathlineto{\pgfqpoint{6.443764in}{2.516892in}}%
\pgfpathlineto{\pgfqpoint{6.444838in}{2.523442in}}%
\pgfpathlineto{\pgfqpoint{6.445912in}{2.519200in}}%
\pgfpathlineto{\pgfqpoint{6.446985in}{2.517805in}}%
\pgfpathlineto{\pgfqpoint{6.448059in}{2.515175in}}%
\pgfpathlineto{\pgfqpoint{6.451281in}{2.509452in}}%
\pgfpathlineto{\pgfqpoint{6.452355in}{2.510305in}}%
\pgfpathlineto{\pgfqpoint{6.453429in}{2.505956in}}%
\pgfpathlineto{\pgfqpoint{6.455576in}{2.518431in}}%
\pgfpathlineto{\pgfqpoint{6.458798in}{2.517342in}}%
\pgfpathlineto{\pgfqpoint{6.460945in}{2.523151in}}%
\pgfpathlineto{\pgfqpoint{6.463093in}{2.523957in}}%
\pgfpathlineto{\pgfqpoint{6.466315in}{2.520440in}}%
\pgfpathlineto{\pgfqpoint{6.467388in}{2.517399in}}%
\pgfpathlineto{\pgfqpoint{6.468462in}{2.517542in}}%
\pgfpathlineto{\pgfqpoint{6.469536in}{2.521703in}}%
\pgfpathlineto{\pgfqpoint{6.473831in}{2.512148in}}%
\pgfpathlineto{\pgfqpoint{6.474905in}{2.512218in}}%
\pgfpathlineto{\pgfqpoint{6.475979in}{2.511244in}}%
\pgfpathlineto{\pgfqpoint{6.477053in}{2.503665in}}%
\pgfpathlineto{\pgfqpoint{6.478127in}{2.505473in}}%
\pgfpathlineto{\pgfqpoint{6.481348in}{2.506377in}}%
\pgfpathlineto{\pgfqpoint{6.482422in}{2.503866in}}%
\pgfpathlineto{\pgfqpoint{6.483496in}{2.510632in}}%
\pgfpathlineto{\pgfqpoint{6.484570in}{2.509864in}}%
\pgfpathlineto{\pgfqpoint{6.485644in}{2.504021in}}%
\pgfpathlineto{\pgfqpoint{6.488865in}{2.504021in}}%
\pgfpathlineto{\pgfqpoint{6.491013in}{2.501701in}}%
\pgfpathlineto{\pgfqpoint{6.492087in}{2.502656in}}%
\pgfpathlineto{\pgfqpoint{6.493161in}{2.499723in}}%
\pgfpathlineto{\pgfqpoint{6.497456in}{2.497494in}}%
\pgfpathlineto{\pgfqpoint{6.498530in}{2.499907in}}%
\pgfpathlineto{\pgfqpoint{6.499604in}{2.499704in}}%
\pgfpathlineto{\pgfqpoint{6.500677in}{2.498555in}}%
\pgfpathlineto{\pgfqpoint{6.500677in}{2.498555in}}%
\pgfusepath{stroke}%
\end{pgfscope}%
\begin{pgfscope}%
\pgfsetrectcap%
\pgfsetmiterjoin%
\pgfsetlinewidth{0.803000pt}%
\definecolor{currentstroke}{rgb}{1.000000,1.000000,1.000000}%
\pgfsetstrokecolor{currentstroke}%
\pgfsetdash{}{0pt}%
\pgfpathmoveto{\pgfqpoint{4.034768in}{2.255883in}}%
\pgfpathlineto{\pgfqpoint{4.034768in}{2.656768in}}%
\pgfusepath{stroke}%
\end{pgfscope}%
\begin{pgfscope}%
\pgfsetrectcap%
\pgfsetmiterjoin%
\pgfsetlinewidth{0.803000pt}%
\definecolor{currentstroke}{rgb}{1.000000,1.000000,1.000000}%
\pgfsetstrokecolor{currentstroke}%
\pgfsetdash{}{0pt}%
\pgfpathmoveto{\pgfqpoint{6.618102in}{2.255883in}}%
\pgfpathlineto{\pgfqpoint{6.618102in}{2.656768in}}%
\pgfusepath{stroke}%
\end{pgfscope}%
\begin{pgfscope}%
\pgfsetrectcap%
\pgfsetmiterjoin%
\pgfsetlinewidth{0.803000pt}%
\definecolor{currentstroke}{rgb}{1.000000,1.000000,1.000000}%
\pgfsetstrokecolor{currentstroke}%
\pgfsetdash{}{0pt}%
\pgfpathmoveto{\pgfqpoint{4.034768in}{2.255883in}}%
\pgfpathlineto{\pgfqpoint{6.618102in}{2.255883in}}%
\pgfusepath{stroke}%
\end{pgfscope}%
\begin{pgfscope}%
\pgfsetrectcap%
\pgfsetmiterjoin%
\pgfsetlinewidth{0.803000pt}%
\definecolor{currentstroke}{rgb}{1.000000,1.000000,1.000000}%
\pgfsetstrokecolor{currentstroke}%
\pgfsetdash{}{0pt}%
\pgfpathmoveto{\pgfqpoint{4.034768in}{2.656768in}}%
\pgfpathlineto{\pgfqpoint{6.618102in}{2.656768in}}%
\pgfusepath{stroke}%
\end{pgfscope}%
\begin{pgfscope}%
\definecolor{textcolor}{rgb}{0.150000,0.150000,0.150000}%
\pgfsetstrokecolor{textcolor}%
\pgfsetfillcolor{textcolor}%
\pgftext[x=5.326435in,y=2.740101in,,base]{\color{textcolor}\rmfamily\fontsize{12.000000}{14.400000}\selectfont PG}%
\end{pgfscope}%
\begin{pgfscope}%
\pgfsetbuttcap%
\pgfsetmiterjoin%
\definecolor{currentfill}{rgb}{0.917647,0.917647,0.949020}%
\pgfsetfillcolor{currentfill}%
\pgfsetlinewidth{0.000000pt}%
\definecolor{currentstroke}{rgb}{0.000000,0.000000,0.000000}%
\pgfsetstrokecolor{currentstroke}%
\pgfsetstrokeopacity{0.000000}%
\pgfsetdash{}{0pt}%
\pgfpathmoveto{\pgfqpoint{0.418102in}{1.293759in}}%
\pgfpathlineto{\pgfqpoint{3.001435in}{1.293759in}}%
\pgfpathlineto{\pgfqpoint{3.001435in}{1.694644in}}%
\pgfpathlineto{\pgfqpoint{0.418102in}{1.694644in}}%
\pgfpathclose%
\pgfusepath{fill}%
\end{pgfscope}%
\begin{pgfscope}%
\pgfpathrectangle{\pgfqpoint{0.418102in}{1.293759in}}{\pgfqpoint{2.583333in}{0.400885in}}%
\pgfusepath{clip}%
\pgfsetroundcap%
\pgfsetroundjoin%
\pgfsetlinewidth{0.803000pt}%
\definecolor{currentstroke}{rgb}{1.000000,1.000000,1.000000}%
\pgfsetstrokecolor{currentstroke}%
\pgfsetdash{}{0pt}%
\pgfpathmoveto{\pgfqpoint{0.533378in}{1.293759in}}%
\pgfpathlineto{\pgfqpoint{0.533378in}{1.694644in}}%
\pgfusepath{stroke}%
\end{pgfscope}%
\begin{pgfscope}%
\definecolor{textcolor}{rgb}{0.150000,0.150000,0.150000}%
\pgfsetstrokecolor{textcolor}%
\pgfsetfillcolor{textcolor}%
\pgftext[x=0.533378in,y=1.196537in,,top]{\color{textcolor}\rmfamily\fontsize{10.000000}{12.000000}\selectfont 2012}%
\end{pgfscope}%
\begin{pgfscope}%
\pgfpathrectangle{\pgfqpoint{0.418102in}{1.293759in}}{\pgfqpoint{2.583333in}{0.400885in}}%
\pgfusepath{clip}%
\pgfsetroundcap%
\pgfsetroundjoin%
\pgfsetlinewidth{0.803000pt}%
\definecolor{currentstroke}{rgb}{1.000000,1.000000,1.000000}%
\pgfsetstrokecolor{currentstroke}%
\pgfsetdash{}{0pt}%
\pgfpathmoveto{\pgfqpoint{0.926403in}{1.293759in}}%
\pgfpathlineto{\pgfqpoint{0.926403in}{1.694644in}}%
\pgfusepath{stroke}%
\end{pgfscope}%
\begin{pgfscope}%
\definecolor{textcolor}{rgb}{0.150000,0.150000,0.150000}%
\pgfsetstrokecolor{textcolor}%
\pgfsetfillcolor{textcolor}%
\pgftext[x=0.926403in,y=1.196537in,,top]{\color{textcolor}\rmfamily\fontsize{10.000000}{12.000000}\selectfont 2013}%
\end{pgfscope}%
\begin{pgfscope}%
\pgfpathrectangle{\pgfqpoint{0.418102in}{1.293759in}}{\pgfqpoint{2.583333in}{0.400885in}}%
\pgfusepath{clip}%
\pgfsetroundcap%
\pgfsetroundjoin%
\pgfsetlinewidth{0.803000pt}%
\definecolor{currentstroke}{rgb}{1.000000,1.000000,1.000000}%
\pgfsetstrokecolor{currentstroke}%
\pgfsetdash{}{0pt}%
\pgfpathmoveto{\pgfqpoint{1.318354in}{1.293759in}}%
\pgfpathlineto{\pgfqpoint{1.318354in}{1.694644in}}%
\pgfusepath{stroke}%
\end{pgfscope}%
\begin{pgfscope}%
\definecolor{textcolor}{rgb}{0.150000,0.150000,0.150000}%
\pgfsetstrokecolor{textcolor}%
\pgfsetfillcolor{textcolor}%
\pgftext[x=1.318354in,y=1.196537in,,top]{\color{textcolor}\rmfamily\fontsize{10.000000}{12.000000}\selectfont 2014}%
\end{pgfscope}%
\begin{pgfscope}%
\pgfpathrectangle{\pgfqpoint{0.418102in}{1.293759in}}{\pgfqpoint{2.583333in}{0.400885in}}%
\pgfusepath{clip}%
\pgfsetroundcap%
\pgfsetroundjoin%
\pgfsetlinewidth{0.803000pt}%
\definecolor{currentstroke}{rgb}{1.000000,1.000000,1.000000}%
\pgfsetstrokecolor{currentstroke}%
\pgfsetdash{}{0pt}%
\pgfpathmoveto{\pgfqpoint{1.710305in}{1.293759in}}%
\pgfpathlineto{\pgfqpoint{1.710305in}{1.694644in}}%
\pgfusepath{stroke}%
\end{pgfscope}%
\begin{pgfscope}%
\definecolor{textcolor}{rgb}{0.150000,0.150000,0.150000}%
\pgfsetstrokecolor{textcolor}%
\pgfsetfillcolor{textcolor}%
\pgftext[x=1.710305in,y=1.196537in,,top]{\color{textcolor}\rmfamily\fontsize{10.000000}{12.000000}\selectfont 2015}%
\end{pgfscope}%
\begin{pgfscope}%
\pgfpathrectangle{\pgfqpoint{0.418102in}{1.293759in}}{\pgfqpoint{2.583333in}{0.400885in}}%
\pgfusepath{clip}%
\pgfsetroundcap%
\pgfsetroundjoin%
\pgfsetlinewidth{0.803000pt}%
\definecolor{currentstroke}{rgb}{1.000000,1.000000,1.000000}%
\pgfsetstrokecolor{currentstroke}%
\pgfsetdash{}{0pt}%
\pgfpathmoveto{\pgfqpoint{2.102256in}{1.293759in}}%
\pgfpathlineto{\pgfqpoint{2.102256in}{1.694644in}}%
\pgfusepath{stroke}%
\end{pgfscope}%
\begin{pgfscope}%
\definecolor{textcolor}{rgb}{0.150000,0.150000,0.150000}%
\pgfsetstrokecolor{textcolor}%
\pgfsetfillcolor{textcolor}%
\pgftext[x=2.102256in,y=1.196537in,,top]{\color{textcolor}\rmfamily\fontsize{10.000000}{12.000000}\selectfont 2016}%
\end{pgfscope}%
\begin{pgfscope}%
\pgfpathrectangle{\pgfqpoint{0.418102in}{1.293759in}}{\pgfqpoint{2.583333in}{0.400885in}}%
\pgfusepath{clip}%
\pgfsetroundcap%
\pgfsetroundjoin%
\pgfsetlinewidth{0.803000pt}%
\definecolor{currentstroke}{rgb}{1.000000,1.000000,1.000000}%
\pgfsetstrokecolor{currentstroke}%
\pgfsetdash{}{0pt}%
\pgfpathmoveto{\pgfqpoint{2.495281in}{1.293759in}}%
\pgfpathlineto{\pgfqpoint{2.495281in}{1.694644in}}%
\pgfusepath{stroke}%
\end{pgfscope}%
\begin{pgfscope}%
\definecolor{textcolor}{rgb}{0.150000,0.150000,0.150000}%
\pgfsetstrokecolor{textcolor}%
\pgfsetfillcolor{textcolor}%
\pgftext[x=2.495281in,y=1.196537in,,top]{\color{textcolor}\rmfamily\fontsize{10.000000}{12.000000}\selectfont 2017}%
\end{pgfscope}%
\begin{pgfscope}%
\pgfpathrectangle{\pgfqpoint{0.418102in}{1.293759in}}{\pgfqpoint{2.583333in}{0.400885in}}%
\pgfusepath{clip}%
\pgfsetroundcap%
\pgfsetroundjoin%
\pgfsetlinewidth{0.803000pt}%
\definecolor{currentstroke}{rgb}{1.000000,1.000000,1.000000}%
\pgfsetstrokecolor{currentstroke}%
\pgfsetdash{}{0pt}%
\pgfpathmoveto{\pgfqpoint{2.887232in}{1.293759in}}%
\pgfpathlineto{\pgfqpoint{2.887232in}{1.694644in}}%
\pgfusepath{stroke}%
\end{pgfscope}%
\begin{pgfscope}%
\definecolor{textcolor}{rgb}{0.150000,0.150000,0.150000}%
\pgfsetstrokecolor{textcolor}%
\pgfsetfillcolor{textcolor}%
\pgftext[x=2.887232in,y=1.196537in,,top]{\color{textcolor}\rmfamily\fontsize{10.000000}{12.000000}\selectfont 2018}%
\end{pgfscope}%
\begin{pgfscope}%
\pgfpathrectangle{\pgfqpoint{0.418102in}{1.293759in}}{\pgfqpoint{2.583333in}{0.400885in}}%
\pgfusepath{clip}%
\pgfsetroundcap%
\pgfsetroundjoin%
\pgfsetlinewidth{0.803000pt}%
\definecolor{currentstroke}{rgb}{1.000000,1.000000,1.000000}%
\pgfsetstrokecolor{currentstroke}%
\pgfsetdash{}{0pt}%
\pgfpathmoveto{\pgfqpoint{0.418102in}{1.384815in}}%
\pgfpathlineto{\pgfqpoint{3.001435in}{1.384815in}}%
\pgfusepath{stroke}%
\end{pgfscope}%
\begin{pgfscope}%
\definecolor{textcolor}{rgb}{0.150000,0.150000,0.150000}%
\pgfsetstrokecolor{textcolor}%
\pgfsetfillcolor{textcolor}%
\pgftext[x=0.232514in,y=1.332054in,left,base]{\color{textcolor}\rmfamily\fontsize{10.000000}{12.000000}\selectfont 1}%
\end{pgfscope}%
\begin{pgfscope}%
\pgfpathrectangle{\pgfqpoint{0.418102in}{1.293759in}}{\pgfqpoint{2.583333in}{0.400885in}}%
\pgfusepath{clip}%
\pgfsetroundcap%
\pgfsetroundjoin%
\pgfsetlinewidth{0.803000pt}%
\definecolor{currentstroke}{rgb}{1.000000,1.000000,1.000000}%
\pgfsetstrokecolor{currentstroke}%
\pgfsetdash{}{0pt}%
\pgfpathmoveto{\pgfqpoint{0.418102in}{1.682596in}}%
\pgfpathlineto{\pgfqpoint{3.001435in}{1.682596in}}%
\pgfusepath{stroke}%
\end{pgfscope}%
\begin{pgfscope}%
\definecolor{textcolor}{rgb}{0.150000,0.150000,0.150000}%
\pgfsetstrokecolor{textcolor}%
\pgfsetfillcolor{textcolor}%
\pgftext[x=0.232514in,y=1.629834in,left,base]{\color{textcolor}\rmfamily\fontsize{10.000000}{12.000000}\selectfont 2}%
\end{pgfscope}%
\begin{pgfscope}%
\pgfpathrectangle{\pgfqpoint{0.418102in}{1.293759in}}{\pgfqpoint{2.583333in}{0.400885in}}%
\pgfusepath{clip}%
\pgfsetroundcap%
\pgfsetroundjoin%
\pgfsetlinewidth{1.505625pt}%
\definecolor{currentstroke}{rgb}{0.000000,0.000000,0.000000}%
\pgfsetstrokecolor{currentstroke}%
\pgfsetdash{}{0pt}%
\pgfpathmoveto{\pgfqpoint{0.535526in}{1.384815in}}%
\pgfpathlineto{\pgfqpoint{0.536600in}{1.386347in}}%
\pgfpathlineto{\pgfqpoint{0.538747in}{1.381752in}}%
\pgfpathlineto{\pgfqpoint{0.541969in}{1.382566in}}%
\pgfpathlineto{\pgfqpoint{0.543043in}{1.390271in}}%
\pgfpathlineto{\pgfqpoint{0.545190in}{1.395105in}}%
\pgfpathlineto{\pgfqpoint{0.546264in}{1.390463in}}%
\pgfpathlineto{\pgfqpoint{0.550560in}{1.394291in}}%
\pgfpathlineto{\pgfqpoint{0.551633in}{1.396541in}}%
\pgfpathlineto{\pgfqpoint{0.553781in}{1.392903in}}%
\pgfpathlineto{\pgfqpoint{0.557003in}{1.393573in}}%
\pgfpathlineto{\pgfqpoint{0.558077in}{1.397259in}}%
\pgfpathlineto{\pgfqpoint{0.560224in}{1.395775in}}%
\pgfpathlineto{\pgfqpoint{0.561298in}{1.396589in}}%
\pgfpathlineto{\pgfqpoint{0.564520in}{1.396541in}}%
\pgfpathlineto{\pgfqpoint{0.565593in}{1.399508in}}%
\pgfpathlineto{\pgfqpoint{0.566667in}{1.406974in}}%
\pgfpathlineto{\pgfqpoint{0.567741in}{1.406160in}}%
\pgfpathlineto{\pgfqpoint{0.568815in}{1.410276in}}%
\pgfpathlineto{\pgfqpoint{0.573110in}{1.407213in}}%
\pgfpathlineto{\pgfqpoint{0.575258in}{1.421188in}}%
\pgfpathlineto{\pgfqpoint{0.576332in}{1.420040in}}%
\pgfpathlineto{\pgfqpoint{0.579553in}{1.425543in}}%
\pgfpathlineto{\pgfqpoint{0.580627in}{1.424586in}}%
\pgfpathlineto{\pgfqpoint{0.581701in}{1.418365in}}%
\pgfpathlineto{\pgfqpoint{0.583849in}{1.423055in}}%
\pgfpathlineto{\pgfqpoint{0.589218in}{1.423773in}}%
\pgfpathlineto{\pgfqpoint{0.590292in}{1.421906in}}%
\pgfpathlineto{\pgfqpoint{0.591366in}{1.423821in}}%
\pgfpathlineto{\pgfqpoint{0.595661in}{1.421810in}}%
\pgfpathlineto{\pgfqpoint{0.598882in}{1.426118in}}%
\pgfpathlineto{\pgfqpoint{0.602104in}{1.421093in}}%
\pgfpathlineto{\pgfqpoint{0.603178in}{1.413483in}}%
\pgfpathlineto{\pgfqpoint{0.605325in}{1.422098in}}%
\pgfpathlineto{\pgfqpoint{0.606399in}{1.422145in}}%
\pgfpathlineto{\pgfqpoint{0.609621in}{1.423916in}}%
\pgfpathlineto{\pgfqpoint{0.610695in}{1.434254in}}%
\pgfpathlineto{\pgfqpoint{0.612842in}{1.435546in}}%
\pgfpathlineto{\pgfqpoint{0.613916in}{1.429899in}}%
\pgfpathlineto{\pgfqpoint{0.617138in}{1.426453in}}%
\pgfpathlineto{\pgfqpoint{0.618211in}{1.420997in}}%
\pgfpathlineto{\pgfqpoint{0.621433in}{1.415158in}}%
\pgfpathlineto{\pgfqpoint{0.624655in}{1.421954in}}%
\pgfpathlineto{\pgfqpoint{0.625728in}{1.420375in}}%
\pgfpathlineto{\pgfqpoint{0.626802in}{1.414009in}}%
\pgfpathlineto{\pgfqpoint{0.628950in}{1.419705in}}%
\pgfpathlineto{\pgfqpoint{0.632171in}{1.418843in}}%
\pgfpathlineto{\pgfqpoint{0.634319in}{1.415637in}}%
\pgfpathlineto{\pgfqpoint{0.639688in}{1.408841in}}%
\pgfpathlineto{\pgfqpoint{0.640762in}{1.401805in}}%
\pgfpathlineto{\pgfqpoint{0.642910in}{1.412621in}}%
\pgfpathlineto{\pgfqpoint{0.643984in}{1.407118in}}%
\pgfpathlineto{\pgfqpoint{0.647205in}{1.407453in}}%
\pgfpathlineto{\pgfqpoint{0.648279in}{1.412909in}}%
\pgfpathlineto{\pgfqpoint{0.649353in}{1.412717in}}%
\pgfpathlineto{\pgfqpoint{0.650427in}{1.409511in}}%
\pgfpathlineto{\pgfqpoint{0.651500in}{1.411904in}}%
\pgfpathlineto{\pgfqpoint{0.654722in}{1.406926in}}%
\pgfpathlineto{\pgfqpoint{0.656870in}{1.407166in}}%
\pgfpathlineto{\pgfqpoint{0.659017in}{1.415924in}}%
\pgfpathlineto{\pgfqpoint{0.663313in}{1.413674in}}%
\pgfpathlineto{\pgfqpoint{0.664387in}{1.414153in}}%
\pgfpathlineto{\pgfqpoint{0.665460in}{1.411569in}}%
\pgfpathlineto{\pgfqpoint{0.666534in}{1.405825in}}%
\pgfpathlineto{\pgfqpoint{0.670830in}{1.403193in}}%
\pgfpathlineto{\pgfqpoint{0.671903in}{1.395919in}}%
\pgfpathlineto{\pgfqpoint{0.672977in}{1.397067in}}%
\pgfpathlineto{\pgfqpoint{0.674051in}{1.396589in}}%
\pgfpathlineto{\pgfqpoint{0.677273in}{1.391659in}}%
\pgfpathlineto{\pgfqpoint{0.678346in}{1.392568in}}%
\pgfpathlineto{\pgfqpoint{0.681568in}{1.379216in}}%
\pgfpathlineto{\pgfqpoint{0.685863in}{1.385485in}}%
\pgfpathlineto{\pgfqpoint{0.686937in}{1.385964in}}%
\pgfpathlineto{\pgfqpoint{0.689085in}{1.381752in}}%
\pgfpathlineto{\pgfqpoint{0.693380in}{1.390128in}}%
\pgfpathlineto{\pgfqpoint{0.694454in}{1.384672in}}%
\pgfpathlineto{\pgfqpoint{0.695528in}{1.386155in}}%
\pgfpathlineto{\pgfqpoint{0.696602in}{1.377732in}}%
\pgfpathlineto{\pgfqpoint{0.699823in}{1.375435in}}%
\pgfpathlineto{\pgfqpoint{0.700897in}{1.373138in}}%
\pgfpathlineto{\pgfqpoint{0.703045in}{1.391372in}}%
\pgfpathlineto{\pgfqpoint{0.704119in}{1.391803in}}%
\pgfpathlineto{\pgfqpoint{0.708414in}{1.387160in}}%
\pgfpathlineto{\pgfqpoint{0.709488in}{1.383858in}}%
\pgfpathlineto{\pgfqpoint{0.711635in}{1.387687in}}%
\pgfpathlineto{\pgfqpoint{0.714857in}{1.389745in}}%
\pgfpathlineto{\pgfqpoint{0.715931in}{1.395440in}}%
\pgfpathlineto{\pgfqpoint{0.717005in}{1.393430in}}%
\pgfpathlineto{\pgfqpoint{0.718078in}{1.389027in}}%
\pgfpathlineto{\pgfqpoint{0.719152in}{1.390558in}}%
\pgfpathlineto{\pgfqpoint{0.722374in}{1.384720in}}%
\pgfpathlineto{\pgfqpoint{0.723448in}{1.384241in}}%
\pgfpathlineto{\pgfqpoint{0.724521in}{1.386395in}}%
\pgfpathlineto{\pgfqpoint{0.725595in}{1.380077in}}%
\pgfpathlineto{\pgfqpoint{0.726669in}{1.391898in}}%
\pgfpathlineto{\pgfqpoint{0.729891in}{1.389936in}}%
\pgfpathlineto{\pgfqpoint{0.730965in}{1.392808in}}%
\pgfpathlineto{\pgfqpoint{0.733112in}{1.391324in}}%
\pgfpathlineto{\pgfqpoint{0.734186in}{1.386107in}}%
\pgfpathlineto{\pgfqpoint{0.737408in}{1.387065in}}%
\pgfpathlineto{\pgfqpoint{0.738481in}{1.386347in}}%
\pgfpathlineto{\pgfqpoint{0.739555in}{1.379694in}}%
\pgfpathlineto{\pgfqpoint{0.740629in}{1.377014in}}%
\pgfpathlineto{\pgfqpoint{0.741703in}{1.384097in}}%
\pgfpathlineto{\pgfqpoint{0.744924in}{1.382566in}}%
\pgfpathlineto{\pgfqpoint{0.745998in}{1.384097in}}%
\pgfpathlineto{\pgfqpoint{0.748146in}{1.393095in}}%
\pgfpathlineto{\pgfqpoint{0.749220in}{1.386682in}}%
\pgfpathlineto{\pgfqpoint{0.752441in}{1.382805in}}%
\pgfpathlineto{\pgfqpoint{0.753515in}{1.377397in}}%
\pgfpathlineto{\pgfqpoint{0.756737in}{1.386873in}}%
\pgfpathlineto{\pgfqpoint{0.759958in}{1.389649in}}%
\pgfpathlineto{\pgfqpoint{0.761032in}{1.387495in}}%
\pgfpathlineto{\pgfqpoint{0.762106in}{1.388931in}}%
\pgfpathlineto{\pgfqpoint{0.763180in}{1.387495in}}%
\pgfpathlineto{\pgfqpoint{0.764254in}{1.397067in}}%
\pgfpathlineto{\pgfqpoint{0.767475in}{1.396445in}}%
\pgfpathlineto{\pgfqpoint{0.768549in}{1.402380in}}%
\pgfpathlineto{\pgfqpoint{0.770697in}{1.397977in}}%
\pgfpathlineto{\pgfqpoint{0.771770in}{1.401422in}}%
\pgfpathlineto{\pgfqpoint{0.774992in}{1.399747in}}%
\pgfpathlineto{\pgfqpoint{0.776066in}{1.400992in}}%
\pgfpathlineto{\pgfqpoint{0.778213in}{1.407166in}}%
\pgfpathlineto{\pgfqpoint{0.779287in}{1.413674in}}%
\pgfpathlineto{\pgfqpoint{0.782509in}{1.412526in}}%
\pgfpathlineto{\pgfqpoint{0.783583in}{1.409223in}}%
\pgfpathlineto{\pgfqpoint{0.784656in}{1.410803in}}%
\pgfpathlineto{\pgfqpoint{0.785730in}{1.408936in}}%
\pgfpathlineto{\pgfqpoint{0.786804in}{1.412526in}}%
\pgfpathlineto{\pgfqpoint{0.791099in}{1.414775in}}%
\pgfpathlineto{\pgfqpoint{0.792173in}{1.412861in}}%
\pgfpathlineto{\pgfqpoint{0.793247in}{1.408362in}}%
\pgfpathlineto{\pgfqpoint{0.794321in}{1.411569in}}%
\pgfpathlineto{\pgfqpoint{0.799690in}{1.404198in}}%
\pgfpathlineto{\pgfqpoint{0.800764in}{1.409750in}}%
\pgfpathlineto{\pgfqpoint{0.801838in}{1.409798in}}%
\pgfpathlineto{\pgfqpoint{0.805059in}{1.406543in}}%
\pgfpathlineto{\pgfqpoint{0.807207in}{1.407453in}}%
\pgfpathlineto{\pgfqpoint{0.809355in}{1.422145in}}%
\pgfpathlineto{\pgfqpoint{0.812576in}{1.420949in}}%
\pgfpathlineto{\pgfqpoint{0.813650in}{1.418221in}}%
\pgfpathlineto{\pgfqpoint{0.814724in}{1.419178in}}%
\pgfpathlineto{\pgfqpoint{0.815798in}{1.415924in}}%
\pgfpathlineto{\pgfqpoint{0.816872in}{1.415254in}}%
\pgfpathlineto{\pgfqpoint{0.820093in}{1.412286in}}%
\pgfpathlineto{\pgfqpoint{0.821167in}{1.407166in}}%
\pgfpathlineto{\pgfqpoint{0.823315in}{1.405490in}}%
\pgfpathlineto{\pgfqpoint{0.824388in}{1.405251in}}%
\pgfpathlineto{\pgfqpoint{0.828684in}{1.405634in}}%
\pgfpathlineto{\pgfqpoint{0.829758in}{1.406352in}}%
\pgfpathlineto{\pgfqpoint{0.835127in}{1.405538in}}%
\pgfpathlineto{\pgfqpoint{0.837275in}{1.396014in}}%
\pgfpathlineto{\pgfqpoint{0.842644in}{1.396541in}}%
\pgfpathlineto{\pgfqpoint{0.844791in}{1.406256in}}%
\pgfpathlineto{\pgfqpoint{0.845865in}{1.409080in}}%
\pgfpathlineto{\pgfqpoint{0.846939in}{1.404007in}}%
\pgfpathlineto{\pgfqpoint{0.850161in}{1.403385in}}%
\pgfpathlineto{\pgfqpoint{0.851234in}{1.400274in}}%
\pgfpathlineto{\pgfqpoint{0.852308in}{1.403672in}}%
\pgfpathlineto{\pgfqpoint{0.853382in}{1.401183in}}%
\pgfpathlineto{\pgfqpoint{0.854456in}{1.404868in}}%
\pgfpathlineto{\pgfqpoint{0.859825in}{1.404725in}}%
\pgfpathlineto{\pgfqpoint{0.860899in}{1.408362in}}%
\pgfpathlineto{\pgfqpoint{0.861973in}{1.404342in}}%
\pgfpathlineto{\pgfqpoint{0.865194in}{1.403672in}}%
\pgfpathlineto{\pgfqpoint{0.866268in}{1.412047in}}%
\pgfpathlineto{\pgfqpoint{0.868416in}{1.396589in}}%
\pgfpathlineto{\pgfqpoint{0.869490in}{1.395296in}}%
\pgfpathlineto{\pgfqpoint{0.872711in}{1.399795in}}%
\pgfpathlineto{\pgfqpoint{0.873785in}{1.399987in}}%
\pgfpathlineto{\pgfqpoint{0.874859in}{1.392568in}}%
\pgfpathlineto{\pgfqpoint{0.875933in}{1.393334in}}%
\pgfpathlineto{\pgfqpoint{0.882376in}{1.403528in}}%
\pgfpathlineto{\pgfqpoint{0.884523in}{1.408793in}}%
\pgfpathlineto{\pgfqpoint{0.888819in}{1.409654in}}%
\pgfpathlineto{\pgfqpoint{0.889893in}{1.413627in}}%
\pgfpathlineto{\pgfqpoint{0.890966in}{1.413483in}}%
\pgfpathlineto{\pgfqpoint{0.892040in}{1.414919in}}%
\pgfpathlineto{\pgfqpoint{0.895262in}{1.413627in}}%
\pgfpathlineto{\pgfqpoint{0.896336in}{1.415014in}}%
\pgfpathlineto{\pgfqpoint{0.897409in}{1.415397in}}%
\pgfpathlineto{\pgfqpoint{0.898483in}{1.418030in}}%
\pgfpathlineto{\pgfqpoint{0.899557in}{1.418508in}}%
\pgfpathlineto{\pgfqpoint{0.902779in}{1.418604in}}%
\pgfpathlineto{\pgfqpoint{0.903853in}{1.419322in}}%
\pgfpathlineto{\pgfqpoint{0.904926in}{1.418556in}}%
\pgfpathlineto{\pgfqpoint{0.907074in}{1.414392in}}%
\pgfpathlineto{\pgfqpoint{0.910296in}{1.414440in}}%
\pgfpathlineto{\pgfqpoint{0.911369in}{1.423581in}}%
\pgfpathlineto{\pgfqpoint{0.912443in}{1.426884in}}%
\pgfpathlineto{\pgfqpoint{0.913517in}{1.427458in}}%
\pgfpathlineto{\pgfqpoint{0.914591in}{1.424873in}}%
\pgfpathlineto{\pgfqpoint{0.921034in}{1.422911in}}%
\pgfpathlineto{\pgfqpoint{0.922108in}{1.417790in}}%
\pgfpathlineto{\pgfqpoint{0.925329in}{1.422672in}}%
\pgfpathlineto{\pgfqpoint{0.927477in}{1.430856in}}%
\pgfpathlineto{\pgfqpoint{0.928551in}{1.432100in}}%
\pgfpathlineto{\pgfqpoint{0.929625in}{1.434828in}}%
\pgfpathlineto{\pgfqpoint{0.932846in}{1.433153in}}%
\pgfpathlineto{\pgfqpoint{0.933920in}{1.428989in}}%
\pgfpathlineto{\pgfqpoint{0.934994in}{1.433105in}}%
\pgfpathlineto{\pgfqpoint{0.937142in}{1.435642in}}%
\pgfpathlineto{\pgfqpoint{0.941437in}{1.438848in}}%
\pgfpathlineto{\pgfqpoint{0.942511in}{1.437269in}}%
\pgfpathlineto{\pgfqpoint{0.944658in}{1.442916in}}%
\pgfpathlineto{\pgfqpoint{0.948954in}{1.445022in}}%
\pgfpathlineto{\pgfqpoint{0.951101in}{1.450957in}}%
\pgfpathlineto{\pgfqpoint{0.952175in}{1.454546in}}%
\pgfpathlineto{\pgfqpoint{0.956471in}{1.454403in}}%
\pgfpathlineto{\pgfqpoint{0.957544in}{1.451722in}}%
\pgfpathlineto{\pgfqpoint{0.958618in}{1.445453in}}%
\pgfpathlineto{\pgfqpoint{0.959692in}{1.454738in}}%
\pgfpathlineto{\pgfqpoint{0.962914in}{1.453493in}}%
\pgfpathlineto{\pgfqpoint{0.963987in}{1.452345in}}%
\pgfpathlineto{\pgfqpoint{0.965061in}{1.452823in}}%
\pgfpathlineto{\pgfqpoint{0.966135in}{1.455168in}}%
\pgfpathlineto{\pgfqpoint{0.967209in}{1.455743in}}%
\pgfpathlineto{\pgfqpoint{0.970431in}{1.453780in}}%
\pgfpathlineto{\pgfqpoint{0.971504in}{1.455312in}}%
\pgfpathlineto{\pgfqpoint{0.973652in}{1.456317in}}%
\pgfpathlineto{\pgfqpoint{0.974726in}{1.460816in}}%
\pgfpathlineto{\pgfqpoint{0.979021in}{1.461773in}}%
\pgfpathlineto{\pgfqpoint{0.981169in}{1.455168in}}%
\pgfpathlineto{\pgfqpoint{0.982243in}{1.459619in}}%
\pgfpathlineto{\pgfqpoint{0.985464in}{1.450861in}}%
\pgfpathlineto{\pgfqpoint{0.986538in}{1.454307in}}%
\pgfpathlineto{\pgfqpoint{0.987612in}{1.459954in}}%
\pgfpathlineto{\pgfqpoint{0.988686in}{1.459859in}}%
\pgfpathlineto{\pgfqpoint{0.989760in}{1.458136in}}%
\pgfpathlineto{\pgfqpoint{0.992981in}{1.454020in}}%
\pgfpathlineto{\pgfqpoint{0.994055in}{1.461773in}}%
\pgfpathlineto{\pgfqpoint{0.995129in}{1.462012in}}%
\pgfpathlineto{\pgfqpoint{0.997276in}{1.466128in}}%
\pgfpathlineto{\pgfqpoint{1.001572in}{1.470531in}}%
\pgfpathlineto{\pgfqpoint{1.002646in}{1.470292in}}%
\pgfpathlineto{\pgfqpoint{1.003720in}{1.471776in}}%
\pgfpathlineto{\pgfqpoint{1.008015in}{1.469000in}}%
\pgfpathlineto{\pgfqpoint{1.010163in}{1.471776in}}%
\pgfpathlineto{\pgfqpoint{1.011236in}{1.467612in}}%
\pgfpathlineto{\pgfqpoint{1.012310in}{1.472398in}}%
\pgfpathlineto{\pgfqpoint{1.016606in}{1.468521in}}%
\pgfpathlineto{\pgfqpoint{1.017679in}{1.468330in}}%
\pgfpathlineto{\pgfqpoint{1.018753in}{1.471728in}}%
\pgfpathlineto{\pgfqpoint{1.023049in}{1.469622in}}%
\pgfpathlineto{\pgfqpoint{1.026270in}{1.470627in}}%
\pgfpathlineto{\pgfqpoint{1.027344in}{1.468473in}}%
\pgfpathlineto{\pgfqpoint{1.030565in}{1.472876in}}%
\pgfpathlineto{\pgfqpoint{1.033787in}{1.481539in}}%
\pgfpathlineto{\pgfqpoint{1.034861in}{1.481012in}}%
\pgfpathlineto{\pgfqpoint{1.038082in}{1.472063in}}%
\pgfpathlineto{\pgfqpoint{1.039156in}{1.476179in}}%
\pgfpathlineto{\pgfqpoint{1.041304in}{1.463974in}}%
\pgfpathlineto{\pgfqpoint{1.042378in}{1.470866in}}%
\pgfpathlineto{\pgfqpoint{1.045599in}{1.472541in}}%
\pgfpathlineto{\pgfqpoint{1.047747in}{1.465841in}}%
\pgfpathlineto{\pgfqpoint{1.048821in}{1.466272in}}%
\pgfpathlineto{\pgfqpoint{1.049895in}{1.462347in}}%
\pgfpathlineto{\pgfqpoint{1.053116in}{1.464262in}}%
\pgfpathlineto{\pgfqpoint{1.055264in}{1.461917in}}%
\pgfpathlineto{\pgfqpoint{1.056338in}{1.464788in}}%
\pgfpathlineto{\pgfqpoint{1.057411in}{1.470388in}}%
\pgfpathlineto{\pgfqpoint{1.060633in}{1.471967in}}%
\pgfpathlineto{\pgfqpoint{1.064928in}{1.479194in}}%
\pgfpathlineto{\pgfqpoint{1.068150in}{1.477997in}}%
\pgfpathlineto{\pgfqpoint{1.069224in}{1.482017in}}%
\pgfpathlineto{\pgfqpoint{1.070297in}{1.483740in}}%
\pgfpathlineto{\pgfqpoint{1.071371in}{1.481060in}}%
\pgfpathlineto{\pgfqpoint{1.072445in}{1.490106in}}%
\pgfpathlineto{\pgfqpoint{1.075667in}{1.489627in}}%
\pgfpathlineto{\pgfqpoint{1.076741in}{1.490919in}}%
\pgfpathlineto{\pgfqpoint{1.078888in}{1.482400in}}%
\pgfpathlineto{\pgfqpoint{1.079962in}{1.480534in}}%
\pgfpathlineto{\pgfqpoint{1.084257in}{1.484410in}}%
\pgfpathlineto{\pgfqpoint{1.085331in}{1.480869in}}%
\pgfpathlineto{\pgfqpoint{1.086405in}{1.484363in}}%
\pgfpathlineto{\pgfqpoint{1.087479in}{1.479959in}}%
\pgfpathlineto{\pgfqpoint{1.090700in}{1.481347in}}%
\pgfpathlineto{\pgfqpoint{1.093922in}{1.471058in}}%
\pgfpathlineto{\pgfqpoint{1.094996in}{1.478284in}}%
\pgfpathlineto{\pgfqpoint{1.098217in}{1.476801in}}%
\pgfpathlineto{\pgfqpoint{1.099291in}{1.474886in}}%
\pgfpathlineto{\pgfqpoint{1.100365in}{1.470770in}}%
\pgfpathlineto{\pgfqpoint{1.101439in}{1.477423in}}%
\pgfpathlineto{\pgfqpoint{1.102513in}{1.476322in}}%
\pgfpathlineto{\pgfqpoint{1.105734in}{1.480294in}}%
\pgfpathlineto{\pgfqpoint{1.106808in}{1.485224in}}%
\pgfpathlineto{\pgfqpoint{1.108956in}{1.469383in}}%
\pgfpathlineto{\pgfqpoint{1.110030in}{1.468713in}}%
\pgfpathlineto{\pgfqpoint{1.113251in}{1.465841in}}%
\pgfpathlineto{\pgfqpoint{1.114325in}{1.467037in}}%
\pgfpathlineto{\pgfqpoint{1.115399in}{1.472015in}}%
\pgfpathlineto{\pgfqpoint{1.116473in}{1.474216in}}%
\pgfpathlineto{\pgfqpoint{1.117546in}{1.471823in}}%
\pgfpathlineto{\pgfqpoint{1.120768in}{1.479289in}}%
\pgfpathlineto{\pgfqpoint{1.121842in}{1.475413in}}%
\pgfpathlineto{\pgfqpoint{1.125063in}{1.486708in}}%
\pgfpathlineto{\pgfqpoint{1.128285in}{1.488574in}}%
\pgfpathlineto{\pgfqpoint{1.129359in}{1.492881in}}%
\pgfpathlineto{\pgfqpoint{1.130432in}{1.491924in}}%
\pgfpathlineto{\pgfqpoint{1.131506in}{1.499773in}}%
\pgfpathlineto{\pgfqpoint{1.135802in}{1.502070in}}%
\pgfpathlineto{\pgfqpoint{1.136875in}{1.501018in}}%
\pgfpathlineto{\pgfqpoint{1.140097in}{1.511355in}}%
\pgfpathlineto{\pgfqpoint{1.143319in}{1.509824in}}%
\pgfpathlineto{\pgfqpoint{1.144392in}{1.522267in}}%
\pgfpathlineto{\pgfqpoint{1.146540in}{1.520927in}}%
\pgfpathlineto{\pgfqpoint{1.147614in}{1.521645in}}%
\pgfpathlineto{\pgfqpoint{1.150835in}{1.522076in}}%
\pgfpathlineto{\pgfqpoint{1.151909in}{1.524134in}}%
\pgfpathlineto{\pgfqpoint{1.152983in}{1.524134in}}%
\pgfpathlineto{\pgfqpoint{1.154057in}{1.530786in}}%
\pgfpathlineto{\pgfqpoint{1.155131in}{1.533227in}}%
\pgfpathlineto{\pgfqpoint{1.158352in}{1.528584in}}%
\pgfpathlineto{\pgfqpoint{1.159426in}{1.522602in}}%
\pgfpathlineto{\pgfqpoint{1.160500in}{1.526048in}}%
\pgfpathlineto{\pgfqpoint{1.161574in}{1.527005in}}%
\pgfpathlineto{\pgfqpoint{1.162648in}{1.524325in}}%
\pgfpathlineto{\pgfqpoint{1.165869in}{1.523990in}}%
\pgfpathlineto{\pgfqpoint{1.166943in}{1.529207in}}%
\pgfpathlineto{\pgfqpoint{1.168017in}{1.524229in}}%
\pgfpathlineto{\pgfqpoint{1.169091in}{1.515615in}}%
\pgfpathlineto{\pgfqpoint{1.170164in}{1.515998in}}%
\pgfpathlineto{\pgfqpoint{1.174460in}{1.513126in}}%
\pgfpathlineto{\pgfqpoint{1.175534in}{1.510254in}}%
\pgfpathlineto{\pgfqpoint{1.176608in}{1.515471in}}%
\pgfpathlineto{\pgfqpoint{1.180903in}{1.512743in}}%
\pgfpathlineto{\pgfqpoint{1.181977in}{1.502932in}}%
\pgfpathlineto{\pgfqpoint{1.183051in}{1.503075in}}%
\pgfpathlineto{\pgfqpoint{1.184124in}{1.505086in}}%
\pgfpathlineto{\pgfqpoint{1.185198in}{1.503554in}}%
\pgfpathlineto{\pgfqpoint{1.191641in}{1.518486in}}%
\pgfpathlineto{\pgfqpoint{1.192715in}{1.516620in}}%
\pgfpathlineto{\pgfqpoint{1.195937in}{1.521788in}}%
\pgfpathlineto{\pgfqpoint{1.198084in}{1.535763in}}%
\pgfpathlineto{\pgfqpoint{1.199158in}{1.535763in}}%
\pgfpathlineto{\pgfqpoint{1.200232in}{1.538061in}}%
\pgfpathlineto{\pgfqpoint{1.204527in}{1.546388in}}%
\pgfpathlineto{\pgfqpoint{1.205601in}{1.548973in}}%
\pgfpathlineto{\pgfqpoint{1.206675in}{1.553088in}}%
\pgfpathlineto{\pgfqpoint{1.207749in}{1.543038in}}%
\pgfpathlineto{\pgfqpoint{1.210970in}{1.542368in}}%
\pgfpathlineto{\pgfqpoint{1.212044in}{1.544761in}}%
\pgfpathlineto{\pgfqpoint{1.213118in}{1.541698in}}%
\pgfpathlineto{\pgfqpoint{1.214192in}{1.543373in}}%
\pgfpathlineto{\pgfqpoint{1.215266in}{1.542129in}}%
\pgfpathlineto{\pgfqpoint{1.219561in}{1.533849in}}%
\pgfpathlineto{\pgfqpoint{1.221709in}{1.518534in}}%
\pgfpathlineto{\pgfqpoint{1.222783in}{1.520927in}}%
\pgfpathlineto{\pgfqpoint{1.226004in}{1.519922in}}%
\pgfpathlineto{\pgfqpoint{1.227078in}{1.514657in}}%
\pgfpathlineto{\pgfqpoint{1.228152in}{1.514992in}}%
\pgfpathlineto{\pgfqpoint{1.229226in}{1.527771in}}%
\pgfpathlineto{\pgfqpoint{1.230299in}{1.532318in}}%
\pgfpathlineto{\pgfqpoint{1.233521in}{1.531887in}}%
\pgfpathlineto{\pgfqpoint{1.234595in}{1.527292in}}%
\pgfpathlineto{\pgfqpoint{1.235669in}{1.529925in}}%
\pgfpathlineto{\pgfqpoint{1.236742in}{1.536529in}}%
\pgfpathlineto{\pgfqpoint{1.237816in}{1.535381in}}%
\pgfpathlineto{\pgfqpoint{1.241038in}{1.534854in}}%
\pgfpathlineto{\pgfqpoint{1.242112in}{1.528680in}}%
\pgfpathlineto{\pgfqpoint{1.243185in}{1.529733in}}%
\pgfpathlineto{\pgfqpoint{1.245333in}{1.534471in}}%
\pgfpathlineto{\pgfqpoint{1.248555in}{1.528106in}}%
\pgfpathlineto{\pgfqpoint{1.249629in}{1.529877in}}%
\pgfpathlineto{\pgfqpoint{1.250702in}{1.527914in}}%
\pgfpathlineto{\pgfqpoint{1.251776in}{1.529159in}}%
\pgfpathlineto{\pgfqpoint{1.252850in}{1.534423in}}%
\pgfpathlineto{\pgfqpoint{1.256072in}{1.536386in}}%
\pgfpathlineto{\pgfqpoint{1.257145in}{1.535141in}}%
\pgfpathlineto{\pgfqpoint{1.258219in}{1.539257in}}%
\pgfpathlineto{\pgfqpoint{1.259293in}{1.533418in}}%
\pgfpathlineto{\pgfqpoint{1.260367in}{1.538731in}}%
\pgfpathlineto{\pgfqpoint{1.263588in}{1.536768in}}%
\pgfpathlineto{\pgfqpoint{1.264662in}{1.534184in}}%
\pgfpathlineto{\pgfqpoint{1.265736in}{1.536673in}}%
\pgfpathlineto{\pgfqpoint{1.266810in}{1.541842in}}%
\pgfpathlineto{\pgfqpoint{1.267884in}{1.541411in}}%
\pgfpathlineto{\pgfqpoint{1.271105in}{1.543660in}}%
\pgfpathlineto{\pgfqpoint{1.272179in}{1.543517in}}%
\pgfpathlineto{\pgfqpoint{1.273253in}{1.542416in}}%
\pgfpathlineto{\pgfqpoint{1.274327in}{1.546340in}}%
\pgfpathlineto{\pgfqpoint{1.275401in}{1.548159in}}%
\pgfpathlineto{\pgfqpoint{1.278622in}{1.548590in}}%
\pgfpathlineto{\pgfqpoint{1.280770in}{1.553615in}}%
\pgfpathlineto{\pgfqpoint{1.282918in}{1.550887in}}%
\pgfpathlineto{\pgfqpoint{1.286139in}{1.548590in}}%
\pgfpathlineto{\pgfqpoint{1.288287in}{1.542847in}}%
\pgfpathlineto{\pgfqpoint{1.289361in}{1.543325in}}%
\pgfpathlineto{\pgfqpoint{1.290434in}{1.551988in}}%
\pgfpathlineto{\pgfqpoint{1.293656in}{1.552227in}}%
\pgfpathlineto{\pgfqpoint{1.294730in}{1.551413in}}%
\pgfpathlineto{\pgfqpoint{1.295804in}{1.541698in}}%
\pgfpathlineto{\pgfqpoint{1.297951in}{1.536194in}}%
\pgfpathlineto{\pgfqpoint{1.301173in}{1.541171in}}%
\pgfpathlineto{\pgfqpoint{1.302247in}{1.537343in}}%
\pgfpathlineto{\pgfqpoint{1.303320in}{1.546484in}}%
\pgfpathlineto{\pgfqpoint{1.304394in}{1.545192in}}%
\pgfpathlineto{\pgfqpoint{1.305468in}{1.550073in}}%
\pgfpathlineto{\pgfqpoint{1.308690in}{1.550743in}}%
\pgfpathlineto{\pgfqpoint{1.311911in}{1.558544in}}%
\pgfpathlineto{\pgfqpoint{1.312985in}{1.559023in}}%
\pgfpathlineto{\pgfqpoint{1.316207in}{1.558736in}}%
\pgfpathlineto{\pgfqpoint{1.317280in}{1.563187in}}%
\pgfpathlineto{\pgfqpoint{1.319428in}{1.557731in}}%
\pgfpathlineto{\pgfqpoint{1.320502in}{1.559454in}}%
\pgfpathlineto{\pgfqpoint{1.323723in}{1.558975in}}%
\pgfpathlineto{\pgfqpoint{1.324797in}{1.561990in}}%
\pgfpathlineto{\pgfqpoint{1.328019in}{1.563330in}}%
\pgfpathlineto{\pgfqpoint{1.332314in}{1.558305in}}%
\pgfpathlineto{\pgfqpoint{1.333388in}{1.564335in}}%
\pgfpathlineto{\pgfqpoint{1.334462in}{1.564958in}}%
\pgfpathlineto{\pgfqpoint{1.335536in}{1.564910in}}%
\pgfpathlineto{\pgfqpoint{1.339831in}{1.568164in}}%
\pgfpathlineto{\pgfqpoint{1.340905in}{1.572902in}}%
\pgfpathlineto{\pgfqpoint{1.341979in}{1.567542in}}%
\pgfpathlineto{\pgfqpoint{1.343052in}{1.554811in}}%
\pgfpathlineto{\pgfqpoint{1.346274in}{1.563139in}}%
\pgfpathlineto{\pgfqpoint{1.347348in}{1.563474in}}%
\pgfpathlineto{\pgfqpoint{1.348422in}{1.560985in}}%
\pgfpathlineto{\pgfqpoint{1.349496in}{1.566872in}}%
\pgfpathlineto{\pgfqpoint{1.350569in}{1.564144in}}%
\pgfpathlineto{\pgfqpoint{1.355939in}{1.538539in}}%
\pgfpathlineto{\pgfqpoint{1.358086in}{1.550360in}}%
\pgfpathlineto{\pgfqpoint{1.361308in}{1.553998in}}%
\pgfpathlineto{\pgfqpoint{1.362382in}{1.559693in}}%
\pgfpathlineto{\pgfqpoint{1.364529in}{1.563617in}}%
\pgfpathlineto{\pgfqpoint{1.365603in}{1.566010in}}%
\pgfpathlineto{\pgfqpoint{1.369898in}{1.565436in}}%
\pgfpathlineto{\pgfqpoint{1.370972in}{1.566728in}}%
\pgfpathlineto{\pgfqpoint{1.372046in}{1.570940in}}%
\pgfpathlineto{\pgfqpoint{1.376341in}{1.577018in}}%
\pgfpathlineto{\pgfqpoint{1.377415in}{1.574482in}}%
\pgfpathlineto{\pgfqpoint{1.378489in}{1.575439in}}%
\pgfpathlineto{\pgfqpoint{1.380637in}{1.579267in}}%
\pgfpathlineto{\pgfqpoint{1.383858in}{1.578167in}}%
\pgfpathlineto{\pgfqpoint{1.384932in}{1.581660in}}%
\pgfpathlineto{\pgfqpoint{1.386006in}{1.580895in}}%
\pgfpathlineto{\pgfqpoint{1.388154in}{1.584676in}}%
\pgfpathlineto{\pgfqpoint{1.391375in}{1.582378in}}%
\pgfpathlineto{\pgfqpoint{1.392449in}{1.573429in}}%
\pgfpathlineto{\pgfqpoint{1.393523in}{1.574147in}}%
\pgfpathlineto{\pgfqpoint{1.394597in}{1.561895in}}%
\pgfpathlineto{\pgfqpoint{1.395671in}{1.560650in}}%
\pgfpathlineto{\pgfqpoint{1.399966in}{1.569408in}}%
\pgfpathlineto{\pgfqpoint{1.401040in}{1.566585in}}%
\pgfpathlineto{\pgfqpoint{1.402114in}{1.565580in}}%
\pgfpathlineto{\pgfqpoint{1.403187in}{1.568930in}}%
\pgfpathlineto{\pgfqpoint{1.406409in}{1.565484in}}%
\pgfpathlineto{\pgfqpoint{1.407483in}{1.571610in}}%
\pgfpathlineto{\pgfqpoint{1.409630in}{1.565723in}}%
\pgfpathlineto{\pgfqpoint{1.410704in}{1.569935in}}%
\pgfpathlineto{\pgfqpoint{1.413926in}{1.578502in}}%
\pgfpathlineto{\pgfqpoint{1.415000in}{1.583048in}}%
\pgfpathlineto{\pgfqpoint{1.416073in}{1.591232in}}%
\pgfpathlineto{\pgfqpoint{1.417147in}{1.590802in}}%
\pgfpathlineto{\pgfqpoint{1.418221in}{1.584101in}}%
\pgfpathlineto{\pgfqpoint{1.422517in}{1.574003in}}%
\pgfpathlineto{\pgfqpoint{1.423590in}{1.579315in}}%
\pgfpathlineto{\pgfqpoint{1.424664in}{1.568882in}}%
\pgfpathlineto{\pgfqpoint{1.425738in}{1.566250in}}%
\pgfpathlineto{\pgfqpoint{1.428960in}{1.570509in}}%
\pgfpathlineto{\pgfqpoint{1.430033in}{1.574290in}}%
\pgfpathlineto{\pgfqpoint{1.431107in}{1.583671in}}%
\pgfpathlineto{\pgfqpoint{1.432181in}{1.585776in}}%
\pgfpathlineto{\pgfqpoint{1.436476in}{1.584628in}}%
\pgfpathlineto{\pgfqpoint{1.438624in}{1.590323in}}%
\pgfpathlineto{\pgfqpoint{1.439698in}{1.587499in}}%
\pgfpathlineto{\pgfqpoint{1.440772in}{1.580033in}}%
\pgfpathlineto{\pgfqpoint{1.443993in}{1.582091in}}%
\pgfpathlineto{\pgfqpoint{1.445067in}{1.581517in}}%
\pgfpathlineto{\pgfqpoint{1.446141in}{1.584771in}}%
\pgfpathlineto{\pgfqpoint{1.447215in}{1.578310in}}%
\pgfpathlineto{\pgfqpoint{1.448289in}{1.577162in}}%
\pgfpathlineto{\pgfqpoint{1.451510in}{1.578406in}}%
\pgfpathlineto{\pgfqpoint{1.452584in}{1.575008in}}%
\pgfpathlineto{\pgfqpoint{1.453658in}{1.578789in}}%
\pgfpathlineto{\pgfqpoint{1.454732in}{1.579267in}}%
\pgfpathlineto{\pgfqpoint{1.455806in}{1.579124in}}%
\pgfpathlineto{\pgfqpoint{1.459027in}{1.586016in}}%
\pgfpathlineto{\pgfqpoint{1.460101in}{1.586494in}}%
\pgfpathlineto{\pgfqpoint{1.461175in}{1.582953in}}%
\pgfpathlineto{\pgfqpoint{1.463322in}{1.571371in}}%
\pgfpathlineto{\pgfqpoint{1.466544in}{1.573237in}}%
\pgfpathlineto{\pgfqpoint{1.467618in}{1.565149in}}%
\pgfpathlineto{\pgfqpoint{1.468692in}{1.572471in}}%
\pgfpathlineto{\pgfqpoint{1.469765in}{1.573333in}}%
\pgfpathlineto{\pgfqpoint{1.470839in}{1.575439in}}%
\pgfpathlineto{\pgfqpoint{1.476208in}{1.577162in}}%
\pgfpathlineto{\pgfqpoint{1.477282in}{1.578885in}}%
\pgfpathlineto{\pgfqpoint{1.478356in}{1.578310in}}%
\pgfpathlineto{\pgfqpoint{1.482651in}{1.585106in}}%
\pgfpathlineto{\pgfqpoint{1.483725in}{1.582187in}}%
\pgfpathlineto{\pgfqpoint{1.485873in}{1.589653in}}%
\pgfpathlineto{\pgfqpoint{1.489095in}{1.594678in}}%
\pgfpathlineto{\pgfqpoint{1.492316in}{1.580990in}}%
\pgfpathlineto{\pgfqpoint{1.493390in}{1.580751in}}%
\pgfpathlineto{\pgfqpoint{1.497685in}{1.581660in}}%
\pgfpathlineto{\pgfqpoint{1.499833in}{1.584053in}}%
\pgfpathlineto{\pgfqpoint{1.500907in}{1.585824in}}%
\pgfpathlineto{\pgfqpoint{1.504128in}{1.582187in}}%
\pgfpathlineto{\pgfqpoint{1.505202in}{1.576061in}}%
\pgfpathlineto{\pgfqpoint{1.506276in}{1.577880in}}%
\pgfpathlineto{\pgfqpoint{1.507350in}{1.576348in}}%
\pgfpathlineto{\pgfqpoint{1.508424in}{1.579842in}}%
\pgfpathlineto{\pgfqpoint{1.511645in}{1.575056in}}%
\pgfpathlineto{\pgfqpoint{1.512719in}{1.577114in}}%
\pgfpathlineto{\pgfqpoint{1.513793in}{1.573764in}}%
\pgfpathlineto{\pgfqpoint{1.514867in}{1.575391in}}%
\pgfpathlineto{\pgfqpoint{1.519162in}{1.573572in}}%
\pgfpathlineto{\pgfqpoint{1.520236in}{1.569217in}}%
\pgfpathlineto{\pgfqpoint{1.522384in}{1.566872in}}%
\pgfpathlineto{\pgfqpoint{1.523457in}{1.569504in}}%
\pgfpathlineto{\pgfqpoint{1.526679in}{1.572663in}}%
\pgfpathlineto{\pgfqpoint{1.527753in}{1.572471in}}%
\pgfpathlineto{\pgfqpoint{1.528827in}{1.570461in}}%
\pgfpathlineto{\pgfqpoint{1.529900in}{1.563713in}}%
\pgfpathlineto{\pgfqpoint{1.530974in}{1.567111in}}%
\pgfpathlineto{\pgfqpoint{1.534196in}{1.564623in}}%
\pgfpathlineto{\pgfqpoint{1.536343in}{1.550887in}}%
\pgfpathlineto{\pgfqpoint{1.537417in}{1.547585in}}%
\pgfpathlineto{\pgfqpoint{1.538491in}{1.547297in}}%
\pgfpathlineto{\pgfqpoint{1.541713in}{1.547776in}}%
\pgfpathlineto{\pgfqpoint{1.543860in}{1.536721in}}%
\pgfpathlineto{\pgfqpoint{1.544934in}{1.531504in}}%
\pgfpathlineto{\pgfqpoint{1.546008in}{1.529829in}}%
\pgfpathlineto{\pgfqpoint{1.550303in}{1.530977in}}%
\pgfpathlineto{\pgfqpoint{1.551377in}{1.525761in}}%
\pgfpathlineto{\pgfqpoint{1.552451in}{1.527579in}}%
\pgfpathlineto{\pgfqpoint{1.553525in}{1.534950in}}%
\pgfpathlineto{\pgfqpoint{1.556746in}{1.533945in}}%
\pgfpathlineto{\pgfqpoint{1.557820in}{1.530499in}}%
\pgfpathlineto{\pgfqpoint{1.558894in}{1.535763in}}%
\pgfpathlineto{\pgfqpoint{1.559968in}{1.536673in}}%
\pgfpathlineto{\pgfqpoint{1.561042in}{1.536098in}}%
\pgfpathlineto{\pgfqpoint{1.566411in}{1.553423in}}%
\pgfpathlineto{\pgfqpoint{1.567485in}{1.554907in}}%
\pgfpathlineto{\pgfqpoint{1.568559in}{1.552083in}}%
\pgfpathlineto{\pgfqpoint{1.571780in}{1.553902in}}%
\pgfpathlineto{\pgfqpoint{1.572854in}{1.553280in}}%
\pgfpathlineto{\pgfqpoint{1.573928in}{1.550791in}}%
\pgfpathlineto{\pgfqpoint{1.575002in}{1.550887in}}%
\pgfpathlineto{\pgfqpoint{1.576075in}{1.546053in}}%
\pgfpathlineto{\pgfqpoint{1.581445in}{1.551270in}}%
\pgfpathlineto{\pgfqpoint{1.582518in}{1.551318in}}%
\pgfpathlineto{\pgfqpoint{1.583592in}{1.549116in}}%
\pgfpathlineto{\pgfqpoint{1.594331in}{1.547537in}}%
\pgfpathlineto{\pgfqpoint{1.595405in}{1.548446in}}%
\pgfpathlineto{\pgfqpoint{1.596478in}{1.546292in}}%
\pgfpathlineto{\pgfqpoint{1.597552in}{1.548350in}}%
\pgfpathlineto{\pgfqpoint{1.598626in}{1.548063in}}%
\pgfpathlineto{\pgfqpoint{1.601848in}{1.539640in}}%
\pgfpathlineto{\pgfqpoint{1.602921in}{1.535189in}}%
\pgfpathlineto{\pgfqpoint{1.603995in}{1.537965in}}%
\pgfpathlineto{\pgfqpoint{1.605069in}{1.531217in}}%
\pgfpathlineto{\pgfqpoint{1.606143in}{1.534423in}}%
\pgfpathlineto{\pgfqpoint{1.609364in}{1.533753in}}%
\pgfpathlineto{\pgfqpoint{1.610438in}{1.535955in}}%
\pgfpathlineto{\pgfqpoint{1.611512in}{1.528489in}}%
\pgfpathlineto{\pgfqpoint{1.612586in}{1.525904in}}%
\pgfpathlineto{\pgfqpoint{1.613660in}{1.530930in}}%
\pgfpathlineto{\pgfqpoint{1.616881in}{1.530068in}}%
\pgfpathlineto{\pgfqpoint{1.617955in}{1.518199in}}%
\pgfpathlineto{\pgfqpoint{1.619029in}{1.523320in}}%
\pgfpathlineto{\pgfqpoint{1.620103in}{1.511882in}}%
\pgfpathlineto{\pgfqpoint{1.621177in}{1.511882in}}%
\pgfpathlineto{\pgfqpoint{1.624398in}{1.509201in}}%
\pgfpathlineto{\pgfqpoint{1.625472in}{1.512647in}}%
\pgfpathlineto{\pgfqpoint{1.626546in}{1.508627in}}%
\pgfpathlineto{\pgfqpoint{1.627620in}{1.508866in}}%
\pgfpathlineto{\pgfqpoint{1.628694in}{1.518630in}}%
\pgfpathlineto{\pgfqpoint{1.631915in}{1.518438in}}%
\pgfpathlineto{\pgfqpoint{1.632989in}{1.520544in}}%
\pgfpathlineto{\pgfqpoint{1.634063in}{1.517146in}}%
\pgfpathlineto{\pgfqpoint{1.635137in}{1.525665in}}%
\pgfpathlineto{\pgfqpoint{1.636210in}{1.528393in}}%
\pgfpathlineto{\pgfqpoint{1.639432in}{1.529972in}}%
\pgfpathlineto{\pgfqpoint{1.640506in}{1.538779in}}%
\pgfpathlineto{\pgfqpoint{1.641580in}{1.537008in}}%
\pgfpathlineto{\pgfqpoint{1.643727in}{1.541889in}}%
\pgfpathlineto{\pgfqpoint{1.646949in}{1.538922in}}%
\pgfpathlineto{\pgfqpoint{1.648023in}{1.541411in}}%
\pgfpathlineto{\pgfqpoint{1.650170in}{1.548638in}}%
\pgfpathlineto{\pgfqpoint{1.651244in}{1.550743in}}%
\pgfpathlineto{\pgfqpoint{1.654466in}{1.550408in}}%
\pgfpathlineto{\pgfqpoint{1.655539in}{1.547250in}}%
\pgfpathlineto{\pgfqpoint{1.656613in}{1.549260in}}%
\pgfpathlineto{\pgfqpoint{1.657687in}{1.549260in}}%
\pgfpathlineto{\pgfqpoint{1.658761in}{1.546340in}}%
\pgfpathlineto{\pgfqpoint{1.661983in}{1.545957in}}%
\pgfpathlineto{\pgfqpoint{1.663056in}{1.552083in}}%
\pgfpathlineto{\pgfqpoint{1.664130in}{1.551509in}}%
\pgfpathlineto{\pgfqpoint{1.665204in}{1.552179in}}%
\pgfpathlineto{\pgfqpoint{1.666278in}{1.558497in}}%
\pgfpathlineto{\pgfqpoint{1.669499in}{1.552036in}}%
\pgfpathlineto{\pgfqpoint{1.670573in}{1.564527in}}%
\pgfpathlineto{\pgfqpoint{1.671647in}{1.557922in}}%
\pgfpathlineto{\pgfqpoint{1.673795in}{1.557539in}}%
\pgfpathlineto{\pgfqpoint{1.678090in}{1.555864in}}%
\pgfpathlineto{\pgfqpoint{1.679164in}{1.561464in}}%
\pgfpathlineto{\pgfqpoint{1.681312in}{1.562756in}}%
\pgfpathlineto{\pgfqpoint{1.684533in}{1.570270in}}%
\pgfpathlineto{\pgfqpoint{1.685607in}{1.578550in}}%
\pgfpathlineto{\pgfqpoint{1.686681in}{1.572184in}}%
\pgfpathlineto{\pgfqpoint{1.687755in}{1.574482in}}%
\pgfpathlineto{\pgfqpoint{1.688828in}{1.566393in}}%
\pgfpathlineto{\pgfqpoint{1.692050in}{1.565388in}}%
\pgfpathlineto{\pgfqpoint{1.694198in}{1.573907in}}%
\pgfpathlineto{\pgfqpoint{1.695272in}{1.586973in}}%
\pgfpathlineto{\pgfqpoint{1.696345in}{1.581134in}}%
\pgfpathlineto{\pgfqpoint{1.699567in}{1.587739in}}%
\pgfpathlineto{\pgfqpoint{1.700641in}{1.588026in}}%
\pgfpathlineto{\pgfqpoint{1.701715in}{1.586734in}}%
\pgfpathlineto{\pgfqpoint{1.703862in}{1.588169in}}%
\pgfpathlineto{\pgfqpoint{1.707084in}{1.586542in}}%
\pgfpathlineto{\pgfqpoint{1.708158in}{1.583718in}}%
\pgfpathlineto{\pgfqpoint{1.709231in}{1.578597in}}%
\pgfpathlineto{\pgfqpoint{1.711379in}{1.578741in}}%
\pgfpathlineto{\pgfqpoint{1.714601in}{1.570557in}}%
\pgfpathlineto{\pgfqpoint{1.715674in}{1.563713in}}%
\pgfpathlineto{\pgfqpoint{1.716748in}{1.568882in}}%
\pgfpathlineto{\pgfqpoint{1.717822in}{1.577114in}}%
\pgfpathlineto{\pgfqpoint{1.718896in}{1.574386in}}%
\pgfpathlineto{\pgfqpoint{1.722117in}{1.576252in}}%
\pgfpathlineto{\pgfqpoint{1.723191in}{1.575774in}}%
\pgfpathlineto{\pgfqpoint{1.724265in}{1.572280in}}%
\pgfpathlineto{\pgfqpoint{1.725339in}{1.572280in}}%
\pgfpathlineto{\pgfqpoint{1.726413in}{1.583527in}}%
\pgfpathlineto{\pgfqpoint{1.730708in}{1.589414in}}%
\pgfpathlineto{\pgfqpoint{1.732856in}{1.601905in}}%
\pgfpathlineto{\pgfqpoint{1.737151in}{1.594630in}}%
\pgfpathlineto{\pgfqpoint{1.738225in}{1.596353in}}%
\pgfpathlineto{\pgfqpoint{1.739299in}{1.586781in}}%
\pgfpathlineto{\pgfqpoint{1.740373in}{1.584771in}}%
\pgfpathlineto{\pgfqpoint{1.741447in}{1.577640in}}%
\pgfpathlineto{\pgfqpoint{1.744668in}{1.585346in}}%
\pgfpathlineto{\pgfqpoint{1.745742in}{1.595205in}}%
\pgfpathlineto{\pgfqpoint{1.746816in}{1.590514in}}%
\pgfpathlineto{\pgfqpoint{1.747890in}{1.600421in}}%
\pgfpathlineto{\pgfqpoint{1.748963in}{1.599129in}}%
\pgfpathlineto{\pgfqpoint{1.752185in}{1.597071in}}%
\pgfpathlineto{\pgfqpoint{1.753259in}{1.597406in}}%
\pgfpathlineto{\pgfqpoint{1.754333in}{1.597119in}}%
\pgfpathlineto{\pgfqpoint{1.755406in}{1.600948in}}%
\pgfpathlineto{\pgfqpoint{1.756480in}{1.608079in}}%
\pgfpathlineto{\pgfqpoint{1.760776in}{1.608414in}}%
\pgfpathlineto{\pgfqpoint{1.762923in}{1.615210in}}%
\pgfpathlineto{\pgfqpoint{1.763997in}{1.620379in}}%
\pgfpathlineto{\pgfqpoint{1.767219in}{1.618704in}}%
\pgfpathlineto{\pgfqpoint{1.768293in}{1.619039in}}%
\pgfpathlineto{\pgfqpoint{1.769366in}{1.617220in}}%
\pgfpathlineto{\pgfqpoint{1.771514in}{1.610950in}}%
\pgfpathlineto{\pgfqpoint{1.774736in}{1.616454in}}%
\pgfpathlineto{\pgfqpoint{1.775809in}{1.610041in}}%
\pgfpathlineto{\pgfqpoint{1.776883in}{1.607217in}}%
\pgfpathlineto{\pgfqpoint{1.777957in}{1.606117in}}%
\pgfpathlineto{\pgfqpoint{1.779031in}{1.600278in}}%
\pgfpathlineto{\pgfqpoint{1.782252in}{1.610232in}}%
\pgfpathlineto{\pgfqpoint{1.783326in}{1.591567in}}%
\pgfpathlineto{\pgfqpoint{1.784400in}{1.595587in}}%
\pgfpathlineto{\pgfqpoint{1.785474in}{1.608079in}}%
\pgfpathlineto{\pgfqpoint{1.786548in}{1.597310in}}%
\pgfpathlineto{\pgfqpoint{1.789769in}{1.603006in}}%
\pgfpathlineto{\pgfqpoint{1.790843in}{1.602144in}}%
\pgfpathlineto{\pgfqpoint{1.791917in}{1.604059in}}%
\pgfpathlineto{\pgfqpoint{1.792991in}{1.600086in}}%
\pgfpathlineto{\pgfqpoint{1.794065in}{1.600421in}}%
\pgfpathlineto{\pgfqpoint{1.797286in}{1.597071in}}%
\pgfpathlineto{\pgfqpoint{1.798360in}{1.598076in}}%
\pgfpathlineto{\pgfqpoint{1.799434in}{1.587499in}}%
\pgfpathlineto{\pgfqpoint{1.800508in}{1.585728in}}%
\pgfpathlineto{\pgfqpoint{1.801582in}{1.589414in}}%
\pgfpathlineto{\pgfqpoint{1.804803in}{1.597693in}}%
\pgfpathlineto{\pgfqpoint{1.806951in}{1.585202in}}%
\pgfpathlineto{\pgfqpoint{1.808025in}{1.590371in}}%
\pgfpathlineto{\pgfqpoint{1.812320in}{1.593530in}}%
\pgfpathlineto{\pgfqpoint{1.813394in}{1.591950in}}%
\pgfpathlineto{\pgfqpoint{1.814468in}{1.593434in}}%
\pgfpathlineto{\pgfqpoint{1.815541in}{1.593625in}}%
\pgfpathlineto{\pgfqpoint{1.816615in}{1.596114in}}%
\pgfpathlineto{\pgfqpoint{1.819837in}{1.591615in}}%
\pgfpathlineto{\pgfqpoint{1.821984in}{1.593577in}}%
\pgfpathlineto{\pgfqpoint{1.823058in}{1.591902in}}%
\pgfpathlineto{\pgfqpoint{1.824132in}{1.581708in}}%
\pgfpathlineto{\pgfqpoint{1.828427in}{1.589605in}}%
\pgfpathlineto{\pgfqpoint{1.829501in}{1.589653in}}%
\pgfpathlineto{\pgfqpoint{1.830575in}{1.590945in}}%
\pgfpathlineto{\pgfqpoint{1.831649in}{1.586207in}}%
\pgfpathlineto{\pgfqpoint{1.834871in}{1.584436in}}%
\pgfpathlineto{\pgfqpoint{1.835944in}{1.585872in}}%
\pgfpathlineto{\pgfqpoint{1.837018in}{1.582857in}}%
\pgfpathlineto{\pgfqpoint{1.838092in}{1.575869in}}%
\pgfpathlineto{\pgfqpoint{1.839166in}{1.583144in}}%
\pgfpathlineto{\pgfqpoint{1.842387in}{1.587451in}}%
\pgfpathlineto{\pgfqpoint{1.843461in}{1.581565in}}%
\pgfpathlineto{\pgfqpoint{1.844535in}{1.581565in}}%
\pgfpathlineto{\pgfqpoint{1.845609in}{1.585728in}}%
\pgfpathlineto{\pgfqpoint{1.846683in}{1.595970in}}%
\pgfpathlineto{\pgfqpoint{1.850978in}{1.591376in}}%
\pgfpathlineto{\pgfqpoint{1.852052in}{1.594200in}}%
\pgfpathlineto{\pgfqpoint{1.853126in}{1.601857in}}%
\pgfpathlineto{\pgfqpoint{1.854200in}{1.599033in}}%
\pgfpathlineto{\pgfqpoint{1.857421in}{1.599129in}}%
\pgfpathlineto{\pgfqpoint{1.858495in}{1.601522in}}%
\pgfpathlineto{\pgfqpoint{1.859569in}{1.600661in}}%
\pgfpathlineto{\pgfqpoint{1.860643in}{1.601761in}}%
\pgfpathlineto{\pgfqpoint{1.866012in}{1.592046in}}%
\pgfpathlineto{\pgfqpoint{1.867086in}{1.595348in}}%
\pgfpathlineto{\pgfqpoint{1.868160in}{1.595587in}}%
\pgfpathlineto{\pgfqpoint{1.869233in}{1.593338in}}%
\pgfpathlineto{\pgfqpoint{1.872455in}{1.592572in}}%
\pgfpathlineto{\pgfqpoint{1.873529in}{1.594056in}}%
\pgfpathlineto{\pgfqpoint{1.874603in}{1.599129in}}%
\pgfpathlineto{\pgfqpoint{1.875676in}{1.593195in}}%
\pgfpathlineto{\pgfqpoint{1.876750in}{1.592572in}}%
\pgfpathlineto{\pgfqpoint{1.879972in}{1.589031in}}%
\pgfpathlineto{\pgfqpoint{1.881046in}{1.589557in}}%
\pgfpathlineto{\pgfqpoint{1.883193in}{1.598315in}}%
\pgfpathlineto{\pgfqpoint{1.884267in}{1.595205in}}%
\pgfpathlineto{\pgfqpoint{1.887489in}{1.582283in}}%
\pgfpathlineto{\pgfqpoint{1.889636in}{1.584245in}}%
\pgfpathlineto{\pgfqpoint{1.890710in}{1.587834in}}%
\pgfpathlineto{\pgfqpoint{1.891784in}{1.583479in}}%
\pgfpathlineto{\pgfqpoint{1.895005in}{1.585154in}}%
\pgfpathlineto{\pgfqpoint{1.896079in}{1.584963in}}%
\pgfpathlineto{\pgfqpoint{1.898227in}{1.576587in}}%
\pgfpathlineto{\pgfqpoint{1.899301in}{1.577545in}}%
\pgfpathlineto{\pgfqpoint{1.903596in}{1.566345in}}%
\pgfpathlineto{\pgfqpoint{1.904670in}{1.565867in}}%
\pgfpathlineto{\pgfqpoint{1.905744in}{1.560459in}}%
\pgfpathlineto{\pgfqpoint{1.910039in}{1.559597in}}%
\pgfpathlineto{\pgfqpoint{1.911113in}{1.562756in}}%
\pgfpathlineto{\pgfqpoint{1.912187in}{1.556582in}}%
\pgfpathlineto{\pgfqpoint{1.913261in}{1.557683in}}%
\pgfpathlineto{\pgfqpoint{1.914335in}{1.563139in}}%
\pgfpathlineto{\pgfqpoint{1.917556in}{1.569169in}}%
\pgfpathlineto{\pgfqpoint{1.918630in}{1.568882in}}%
\pgfpathlineto{\pgfqpoint{1.919704in}{1.567733in}}%
\pgfpathlineto{\pgfqpoint{1.920778in}{1.567781in}}%
\pgfpathlineto{\pgfqpoint{1.921851in}{1.565532in}}%
\pgfpathlineto{\pgfqpoint{1.925073in}{1.564431in}}%
\pgfpathlineto{\pgfqpoint{1.926147in}{1.530834in}}%
\pgfpathlineto{\pgfqpoint{1.927221in}{1.525857in}}%
\pgfpathlineto{\pgfqpoint{1.928294in}{1.524038in}}%
\pgfpathlineto{\pgfqpoint{1.929368in}{1.516141in}}%
\pgfpathlineto{\pgfqpoint{1.932590in}{1.514275in}}%
\pgfpathlineto{\pgfqpoint{1.933664in}{1.514705in}}%
\pgfpathlineto{\pgfqpoint{1.934738in}{1.516380in}}%
\pgfpathlineto{\pgfqpoint{1.935811in}{1.522267in}}%
\pgfpathlineto{\pgfqpoint{1.936885in}{1.520496in}}%
\pgfpathlineto{\pgfqpoint{1.942254in}{1.514275in}}%
\pgfpathlineto{\pgfqpoint{1.943328in}{1.514705in}}%
\pgfpathlineto{\pgfqpoint{1.944402in}{1.511738in}}%
\pgfpathlineto{\pgfqpoint{1.947624in}{1.517194in}}%
\pgfpathlineto{\pgfqpoint{1.948697in}{1.513126in}}%
\pgfpathlineto{\pgfqpoint{1.949771in}{1.516237in}}%
\pgfpathlineto{\pgfqpoint{1.950845in}{1.514897in}}%
\pgfpathlineto{\pgfqpoint{1.951919in}{1.516285in}}%
\pgfpathlineto{\pgfqpoint{1.956214in}{1.519396in}}%
\pgfpathlineto{\pgfqpoint{1.957288in}{1.514179in}}%
\pgfpathlineto{\pgfqpoint{1.959436in}{1.491541in}}%
\pgfpathlineto{\pgfqpoint{1.962657in}{1.482257in}}%
\pgfpathlineto{\pgfqpoint{1.963731in}{1.473259in}}%
\pgfpathlineto{\pgfqpoint{1.965879in}{1.492690in}}%
\pgfpathlineto{\pgfqpoint{1.966953in}{1.492546in}}%
\pgfpathlineto{\pgfqpoint{1.970174in}{1.485463in}}%
\pgfpathlineto{\pgfqpoint{1.971248in}{1.477231in}}%
\pgfpathlineto{\pgfqpoint{1.972322in}{1.483836in}}%
\pgfpathlineto{\pgfqpoint{1.973396in}{1.486755in}}%
\pgfpathlineto{\pgfqpoint{1.974470in}{1.481443in}}%
\pgfpathlineto{\pgfqpoint{1.978765in}{1.490728in}}%
\pgfpathlineto{\pgfqpoint{1.979839in}{1.486516in}}%
\pgfpathlineto{\pgfqpoint{1.980913in}{1.484554in}}%
\pgfpathlineto{\pgfqpoint{1.981986in}{1.488622in}}%
\pgfpathlineto{\pgfqpoint{1.985208in}{1.486420in}}%
\pgfpathlineto{\pgfqpoint{1.987356in}{1.495609in}}%
\pgfpathlineto{\pgfqpoint{1.988429in}{1.492881in}}%
\pgfpathlineto{\pgfqpoint{1.989503in}{1.483118in}}%
\pgfpathlineto{\pgfqpoint{1.992725in}{1.485224in}}%
\pgfpathlineto{\pgfqpoint{1.993799in}{1.470627in}}%
\pgfpathlineto{\pgfqpoint{1.994872in}{1.465267in}}%
\pgfpathlineto{\pgfqpoint{1.995946in}{1.464644in}}%
\pgfpathlineto{\pgfqpoint{1.997020in}{1.466702in}}%
\pgfpathlineto{\pgfqpoint{2.000242in}{1.464692in}}%
\pgfpathlineto{\pgfqpoint{2.002389in}{1.474073in}}%
\pgfpathlineto{\pgfqpoint{2.003463in}{1.471345in}}%
\pgfpathlineto{\pgfqpoint{2.004537in}{1.477471in}}%
\pgfpathlineto{\pgfqpoint{2.007759in}{1.488287in}}%
\pgfpathlineto{\pgfqpoint{2.008832in}{1.489579in}}%
\pgfpathlineto{\pgfqpoint{2.012054in}{1.501831in}}%
\pgfpathlineto{\pgfqpoint{2.015275in}{1.502070in}}%
\pgfpathlineto{\pgfqpoint{2.016349in}{1.497093in}}%
\pgfpathlineto{\pgfqpoint{2.017423in}{1.487904in}}%
\pgfpathlineto{\pgfqpoint{2.018497in}{1.492307in}}%
\pgfpathlineto{\pgfqpoint{2.019571in}{1.491494in}}%
\pgfpathlineto{\pgfqpoint{2.022792in}{1.487378in}}%
\pgfpathlineto{\pgfqpoint{2.024940in}{1.513126in}}%
\pgfpathlineto{\pgfqpoint{2.026014in}{1.521023in}}%
\pgfpathlineto{\pgfqpoint{2.027088in}{1.524660in}}%
\pgfpathlineto{\pgfqpoint{2.030309in}{1.522937in}}%
\pgfpathlineto{\pgfqpoint{2.031383in}{1.517098in}}%
\pgfpathlineto{\pgfqpoint{2.032457in}{1.519013in}}%
\pgfpathlineto{\pgfqpoint{2.033531in}{1.517816in}}%
\pgfpathlineto{\pgfqpoint{2.034604in}{1.515040in}}%
\pgfpathlineto{\pgfqpoint{2.037826in}{1.519300in}}%
\pgfpathlineto{\pgfqpoint{2.039974in}{1.523416in}}%
\pgfpathlineto{\pgfqpoint{2.041048in}{1.525426in}}%
\pgfpathlineto{\pgfqpoint{2.042121in}{1.525426in}}%
\pgfpathlineto{\pgfqpoint{2.047491in}{1.517146in}}%
\pgfpathlineto{\pgfqpoint{2.048564in}{1.521358in}}%
\pgfpathlineto{\pgfqpoint{2.049638in}{1.509345in}}%
\pgfpathlineto{\pgfqpoint{2.052860in}{1.515280in}}%
\pgfpathlineto{\pgfqpoint{2.053934in}{1.514035in}}%
\pgfpathlineto{\pgfqpoint{2.055007in}{1.514753in}}%
\pgfpathlineto{\pgfqpoint{2.056081in}{1.517338in}}%
\pgfpathlineto{\pgfqpoint{2.057155in}{1.516955in}}%
\pgfpathlineto{\pgfqpoint{2.060377in}{1.516428in}}%
\pgfpathlineto{\pgfqpoint{2.061450in}{1.513222in}}%
\pgfpathlineto{\pgfqpoint{2.062524in}{1.512839in}}%
\pgfpathlineto{\pgfqpoint{2.067893in}{1.507479in}}%
\pgfpathlineto{\pgfqpoint{2.068967in}{1.510302in}}%
\pgfpathlineto{\pgfqpoint{2.070041in}{1.503985in}}%
\pgfpathlineto{\pgfqpoint{2.071115in}{1.501592in}}%
\pgfpathlineto{\pgfqpoint{2.072189in}{1.506091in}}%
\pgfpathlineto{\pgfqpoint{2.075410in}{1.506856in}}%
\pgfpathlineto{\pgfqpoint{2.076484in}{1.499534in}}%
\pgfpathlineto{\pgfqpoint{2.077558in}{1.499199in}}%
\pgfpathlineto{\pgfqpoint{2.078632in}{1.497955in}}%
\pgfpathlineto{\pgfqpoint{2.079706in}{1.495418in}}%
\pgfpathlineto{\pgfqpoint{2.082927in}{1.494269in}}%
\pgfpathlineto{\pgfqpoint{2.084001in}{1.495274in}}%
\pgfpathlineto{\pgfqpoint{2.085075in}{1.503458in}}%
\pgfpathlineto{\pgfqpoint{2.087223in}{1.491254in}}%
\pgfpathlineto{\pgfqpoint{2.090444in}{1.496806in}}%
\pgfpathlineto{\pgfqpoint{2.092592in}{1.508579in}}%
\pgfpathlineto{\pgfqpoint{2.093666in}{1.508579in}}%
\pgfpathlineto{\pgfqpoint{2.097961in}{1.507670in}}%
\pgfpathlineto{\pgfqpoint{2.099035in}{1.512839in}}%
\pgfpathlineto{\pgfqpoint{2.100109in}{1.511212in}}%
\pgfpathlineto{\pgfqpoint{2.101182in}{1.507574in}}%
\pgfpathlineto{\pgfqpoint{2.105478in}{1.505373in}}%
\pgfpathlineto{\pgfqpoint{2.106552in}{1.506043in}}%
\pgfpathlineto{\pgfqpoint{2.107626in}{1.494652in}}%
\pgfpathlineto{\pgfqpoint{2.109773in}{1.482735in}}%
\pgfpathlineto{\pgfqpoint{2.114069in}{1.483166in}}%
\pgfpathlineto{\pgfqpoint{2.115142in}{1.475891in}}%
\pgfpathlineto{\pgfqpoint{2.116216in}{1.476657in}}%
\pgfpathlineto{\pgfqpoint{2.117290in}{1.461869in}}%
\pgfpathlineto{\pgfqpoint{2.121585in}{1.460146in}}%
\pgfpathlineto{\pgfqpoint{2.122659in}{1.458375in}}%
\pgfpathlineto{\pgfqpoint{2.124807in}{1.465027in}}%
\pgfpathlineto{\pgfqpoint{2.128028in}{1.458662in}}%
\pgfpathlineto{\pgfqpoint{2.129102in}{1.461964in}}%
\pgfpathlineto{\pgfqpoint{2.130176in}{1.462634in}}%
\pgfpathlineto{\pgfqpoint{2.131250in}{1.465458in}}%
\pgfpathlineto{\pgfqpoint{2.132324in}{1.470866in}}%
\pgfpathlineto{\pgfqpoint{2.135545in}{1.470292in}}%
\pgfpathlineto{\pgfqpoint{2.136619in}{1.460959in}}%
\pgfpathlineto{\pgfqpoint{2.137693in}{1.463304in}}%
\pgfpathlineto{\pgfqpoint{2.138767in}{1.472781in}}%
\pgfpathlineto{\pgfqpoint{2.139841in}{1.471536in}}%
\pgfpathlineto{\pgfqpoint{2.143062in}{1.466894in}}%
\pgfpathlineto{\pgfqpoint{2.144136in}{1.468904in}}%
\pgfpathlineto{\pgfqpoint{2.145210in}{1.467707in}}%
\pgfpathlineto{\pgfqpoint{2.146284in}{1.457609in}}%
\pgfpathlineto{\pgfqpoint{2.147358in}{1.463257in}}%
\pgfpathlineto{\pgfqpoint{2.151653in}{1.465602in}}%
\pgfpathlineto{\pgfqpoint{2.152727in}{1.475652in}}%
\pgfpathlineto{\pgfqpoint{2.153801in}{1.476705in}}%
\pgfpathlineto{\pgfqpoint{2.154874in}{1.476131in}}%
\pgfpathlineto{\pgfqpoint{2.158096in}{1.494365in}}%
\pgfpathlineto{\pgfqpoint{2.159170in}{1.490967in}}%
\pgfpathlineto{\pgfqpoint{2.160244in}{1.499869in}}%
\pgfpathlineto{\pgfqpoint{2.161317in}{1.519539in}}%
\pgfpathlineto{\pgfqpoint{2.165613in}{1.513126in}}%
\pgfpathlineto{\pgfqpoint{2.166687in}{1.506186in}}%
\pgfpathlineto{\pgfqpoint{2.168834in}{1.510877in}}%
\pgfpathlineto{\pgfqpoint{2.169908in}{1.514801in}}%
\pgfpathlineto{\pgfqpoint{2.174203in}{1.514370in}}%
\pgfpathlineto{\pgfqpoint{2.176351in}{1.511164in}}%
\pgfpathlineto{\pgfqpoint{2.177425in}{1.513700in}}%
\pgfpathlineto{\pgfqpoint{2.180647in}{1.514035in}}%
\pgfpathlineto{\pgfqpoint{2.181720in}{1.511690in}}%
\pgfpathlineto{\pgfqpoint{2.183868in}{1.522937in}}%
\pgfpathlineto{\pgfqpoint{2.184942in}{1.523846in}}%
\pgfpathlineto{\pgfqpoint{2.188163in}{1.524373in}}%
\pgfpathlineto{\pgfqpoint{2.189237in}{1.522123in}}%
\pgfpathlineto{\pgfqpoint{2.190311in}{1.524086in}}%
\pgfpathlineto{\pgfqpoint{2.195680in}{1.523224in}}%
\pgfpathlineto{\pgfqpoint{2.196754in}{1.528584in}}%
\pgfpathlineto{\pgfqpoint{2.197828in}{1.529159in}}%
\pgfpathlineto{\pgfqpoint{2.199976in}{1.527914in}}%
\pgfpathlineto{\pgfqpoint{2.203197in}{1.529207in}}%
\pgfpathlineto{\pgfqpoint{2.204271in}{1.527436in}}%
\pgfpathlineto{\pgfqpoint{2.207492in}{1.533849in}}%
\pgfpathlineto{\pgfqpoint{2.210714in}{1.537486in}}%
\pgfpathlineto{\pgfqpoint{2.211788in}{1.541602in}}%
\pgfpathlineto{\pgfqpoint{2.212862in}{1.548350in}}%
\pgfpathlineto{\pgfqpoint{2.213936in}{1.548685in}}%
\pgfpathlineto{\pgfqpoint{2.215009in}{1.548207in}}%
\pgfpathlineto{\pgfqpoint{2.218231in}{1.550648in}}%
\pgfpathlineto{\pgfqpoint{2.219305in}{1.550073in}}%
\pgfpathlineto{\pgfqpoint{2.220379in}{1.552418in}}%
\pgfpathlineto{\pgfqpoint{2.221452in}{1.551988in}}%
\pgfpathlineto{\pgfqpoint{2.222526in}{1.553184in}}%
\pgfpathlineto{\pgfqpoint{2.225748in}{1.550983in}}%
\pgfpathlineto{\pgfqpoint{2.226822in}{1.549212in}}%
\pgfpathlineto{\pgfqpoint{2.227895in}{1.553998in}}%
\pgfpathlineto{\pgfqpoint{2.228969in}{1.546675in}}%
\pgfpathlineto{\pgfqpoint{2.230043in}{1.547297in}}%
\pgfpathlineto{\pgfqpoint{2.233265in}{1.547297in}}%
\pgfpathlineto{\pgfqpoint{2.235412in}{1.530977in}}%
\pgfpathlineto{\pgfqpoint{2.236486in}{1.529302in}}%
\pgfpathlineto{\pgfqpoint{2.237560in}{1.532988in}}%
\pgfpathlineto{\pgfqpoint{2.240781in}{1.528441in}}%
\pgfpathlineto{\pgfqpoint{2.241855in}{1.537582in}}%
\pgfpathlineto{\pgfqpoint{2.242929in}{1.535093in}}%
\pgfpathlineto{\pgfqpoint{2.244003in}{1.534471in}}%
\pgfpathlineto{\pgfqpoint{2.245077in}{1.529207in}}%
\pgfpathlineto{\pgfqpoint{2.248298in}{1.536051in}}%
\pgfpathlineto{\pgfqpoint{2.249372in}{1.528297in}}%
\pgfpathlineto{\pgfqpoint{2.250446in}{1.527819in}}%
\pgfpathlineto{\pgfqpoint{2.251520in}{1.524325in}}%
\pgfpathlineto{\pgfqpoint{2.252594in}{1.526957in}}%
\pgfpathlineto{\pgfqpoint{2.255815in}{1.526096in}}%
\pgfpathlineto{\pgfqpoint{2.256889in}{1.530547in}}%
\pgfpathlineto{\pgfqpoint{2.257963in}{1.532365in}}%
\pgfpathlineto{\pgfqpoint{2.259037in}{1.532796in}}%
\pgfpathlineto{\pgfqpoint{2.260111in}{1.534328in}}%
\pgfpathlineto{\pgfqpoint{2.264406in}{1.533514in}}%
\pgfpathlineto{\pgfqpoint{2.265480in}{1.532509in}}%
\pgfpathlineto{\pgfqpoint{2.266554in}{1.534471in}}%
\pgfpathlineto{\pgfqpoint{2.267627in}{1.532892in}}%
\pgfpathlineto{\pgfqpoint{2.270849in}{1.537008in}}%
\pgfpathlineto{\pgfqpoint{2.271923in}{1.537391in}}%
\pgfpathlineto{\pgfqpoint{2.272997in}{1.540166in}}%
\pgfpathlineto{\pgfqpoint{2.274070in}{1.541267in}}%
\pgfpathlineto{\pgfqpoint{2.278366in}{1.536338in}}%
\pgfpathlineto{\pgfqpoint{2.279440in}{1.536242in}}%
\pgfpathlineto{\pgfqpoint{2.280514in}{1.533370in}}%
\pgfpathlineto{\pgfqpoint{2.281587in}{1.535955in}}%
\pgfpathlineto{\pgfqpoint{2.282661in}{1.536290in}}%
\pgfpathlineto{\pgfqpoint{2.285883in}{1.538491in}}%
\pgfpathlineto{\pgfqpoint{2.288030in}{1.536960in}}%
\pgfpathlineto{\pgfqpoint{2.289104in}{1.541315in}}%
\pgfpathlineto{\pgfqpoint{2.290178in}{1.526048in}}%
\pgfpathlineto{\pgfqpoint{2.293400in}{1.518582in}}%
\pgfpathlineto{\pgfqpoint{2.296621in}{1.542272in}}%
\pgfpathlineto{\pgfqpoint{2.297695in}{1.543086in}}%
\pgfpathlineto{\pgfqpoint{2.301990in}{1.533705in}}%
\pgfpathlineto{\pgfqpoint{2.304138in}{1.539688in}}%
\pgfpathlineto{\pgfqpoint{2.305212in}{1.547202in}}%
\pgfpathlineto{\pgfqpoint{2.308433in}{1.548542in}}%
\pgfpathlineto{\pgfqpoint{2.310581in}{1.553567in}}%
\pgfpathlineto{\pgfqpoint{2.311655in}{1.553758in}}%
\pgfpathlineto{\pgfqpoint{2.312729in}{1.555386in}}%
\pgfpathlineto{\pgfqpoint{2.315950in}{1.555338in}}%
\pgfpathlineto{\pgfqpoint{2.317024in}{1.555912in}}%
\pgfpathlineto{\pgfqpoint{2.318098in}{1.557779in}}%
\pgfpathlineto{\pgfqpoint{2.319172in}{1.556917in}}%
\pgfpathlineto{\pgfqpoint{2.320246in}{1.553711in}}%
\pgfpathlineto{\pgfqpoint{2.323467in}{1.551605in}}%
\pgfpathlineto{\pgfqpoint{2.324541in}{1.565963in}}%
\pgfpathlineto{\pgfqpoint{2.326689in}{1.564623in}}%
\pgfpathlineto{\pgfqpoint{2.327762in}{1.564910in}}%
\pgfpathlineto{\pgfqpoint{2.330984in}{1.561751in}}%
\pgfpathlineto{\pgfqpoint{2.332058in}{1.558784in}}%
\pgfpathlineto{\pgfqpoint{2.334205in}{1.559214in}}%
\pgfpathlineto{\pgfqpoint{2.335279in}{1.565340in}}%
\pgfpathlineto{\pgfqpoint{2.338501in}{1.565532in}}%
\pgfpathlineto{\pgfqpoint{2.339575in}{1.567973in}}%
\pgfpathlineto{\pgfqpoint{2.340648in}{1.567015in}}%
\pgfpathlineto{\pgfqpoint{2.341722in}{1.571706in}}%
\pgfpathlineto{\pgfqpoint{2.342796in}{1.570318in}}%
\pgfpathlineto{\pgfqpoint{2.346018in}{1.573955in}}%
\pgfpathlineto{\pgfqpoint{2.347091in}{1.571610in}}%
\pgfpathlineto{\pgfqpoint{2.349239in}{1.575343in}}%
\pgfpathlineto{\pgfqpoint{2.353535in}{1.571610in}}%
\pgfpathlineto{\pgfqpoint{2.354608in}{1.569408in}}%
\pgfpathlineto{\pgfqpoint{2.355682in}{1.569217in}}%
\pgfpathlineto{\pgfqpoint{2.356756in}{1.568164in}}%
\pgfpathlineto{\pgfqpoint{2.357830in}{1.566298in}}%
\pgfpathlineto{\pgfqpoint{2.361051in}{1.569265in}}%
\pgfpathlineto{\pgfqpoint{2.363199in}{1.562373in}}%
\pgfpathlineto{\pgfqpoint{2.365347in}{1.564623in}}%
\pgfpathlineto{\pgfqpoint{2.370716in}{1.559789in}}%
\pgfpathlineto{\pgfqpoint{2.371790in}{1.559358in}}%
\pgfpathlineto{\pgfqpoint{2.372864in}{1.545622in}}%
\pgfpathlineto{\pgfqpoint{2.376085in}{1.551653in}}%
\pgfpathlineto{\pgfqpoint{2.377159in}{1.544043in}}%
\pgfpathlineto{\pgfqpoint{2.378233in}{1.541076in}}%
\pgfpathlineto{\pgfqpoint{2.379307in}{1.545766in}}%
\pgfpathlineto{\pgfqpoint{2.380380in}{1.534088in}}%
\pgfpathlineto{\pgfqpoint{2.383602in}{1.535620in}}%
\pgfpathlineto{\pgfqpoint{2.384676in}{1.534710in}}%
\pgfpathlineto{\pgfqpoint{2.386824in}{1.547058in}}%
\pgfpathlineto{\pgfqpoint{2.387897in}{1.545096in}}%
\pgfpathlineto{\pgfqpoint{2.391119in}{1.543612in}}%
\pgfpathlineto{\pgfqpoint{2.393267in}{1.544139in}}%
\pgfpathlineto{\pgfqpoint{2.394340in}{1.538396in}}%
\pgfpathlineto{\pgfqpoint{2.395414in}{1.540789in}}%
\pgfpathlineto{\pgfqpoint{2.398636in}{1.544522in}}%
\pgfpathlineto{\pgfqpoint{2.399710in}{1.540071in}}%
\pgfpathlineto{\pgfqpoint{2.400783in}{1.543708in}}%
\pgfpathlineto{\pgfqpoint{2.401857in}{1.542942in}}%
\pgfpathlineto{\pgfqpoint{2.402931in}{1.536242in}}%
\pgfpathlineto{\pgfqpoint{2.406153in}{1.533562in}}%
\pgfpathlineto{\pgfqpoint{2.407226in}{1.528154in}}%
\pgfpathlineto{\pgfqpoint{2.408300in}{1.528872in}}%
\pgfpathlineto{\pgfqpoint{2.409374in}{1.532988in}}%
\pgfpathlineto{\pgfqpoint{2.410448in}{1.534375in}}%
\pgfpathlineto{\pgfqpoint{2.413669in}{1.532365in}}%
\pgfpathlineto{\pgfqpoint{2.414743in}{1.533610in}}%
\pgfpathlineto{\pgfqpoint{2.415817in}{1.532605in}}%
\pgfpathlineto{\pgfqpoint{2.417965in}{1.527723in}}%
\pgfpathlineto{\pgfqpoint{2.421186in}{1.531504in}}%
\pgfpathlineto{\pgfqpoint{2.422260in}{1.539736in}}%
\pgfpathlineto{\pgfqpoint{2.423334in}{1.538156in}}%
\pgfpathlineto{\pgfqpoint{2.424408in}{1.533993in}}%
\pgfpathlineto{\pgfqpoint{2.425482in}{1.541889in}}%
\pgfpathlineto{\pgfqpoint{2.428703in}{1.543469in}}%
\pgfpathlineto{\pgfqpoint{2.429777in}{1.542655in}}%
\pgfpathlineto{\pgfqpoint{2.431925in}{1.538348in}}%
\pgfpathlineto{\pgfqpoint{2.432999in}{1.539592in}}%
\pgfpathlineto{\pgfqpoint{2.437294in}{1.548733in}}%
\pgfpathlineto{\pgfqpoint{2.438368in}{1.555146in}}%
\pgfpathlineto{\pgfqpoint{2.439442in}{1.571227in}}%
\pgfpathlineto{\pgfqpoint{2.440515in}{1.573237in}}%
\pgfpathlineto{\pgfqpoint{2.444811in}{1.567973in}}%
\pgfpathlineto{\pgfqpoint{2.445885in}{1.567686in}}%
\pgfpathlineto{\pgfqpoint{2.448032in}{1.565628in}}%
\pgfpathlineto{\pgfqpoint{2.452328in}{1.567590in}}%
\pgfpathlineto{\pgfqpoint{2.453402in}{1.572854in}}%
\pgfpathlineto{\pgfqpoint{2.455549in}{1.575869in}}%
\pgfpathlineto{\pgfqpoint{2.458771in}{1.574386in}}%
\pgfpathlineto{\pgfqpoint{2.459845in}{1.576157in}}%
\pgfpathlineto{\pgfqpoint{2.460918in}{1.571131in}}%
\pgfpathlineto{\pgfqpoint{2.461992in}{1.570031in}}%
\pgfpathlineto{\pgfqpoint{2.463066in}{1.573333in}}%
\pgfpathlineto{\pgfqpoint{2.466288in}{1.569743in}}%
\pgfpathlineto{\pgfqpoint{2.467361in}{1.569791in}}%
\pgfpathlineto{\pgfqpoint{2.468435in}{1.579938in}}%
\pgfpathlineto{\pgfqpoint{2.469509in}{1.574290in}}%
\pgfpathlineto{\pgfqpoint{2.470583in}{1.580416in}}%
\pgfpathlineto{\pgfqpoint{2.473804in}{1.583096in}}%
\pgfpathlineto{\pgfqpoint{2.474878in}{1.582761in}}%
\pgfpathlineto{\pgfqpoint{2.477026in}{1.572950in}}%
\pgfpathlineto{\pgfqpoint{2.478100in}{1.574721in}}%
\pgfpathlineto{\pgfqpoint{2.481321in}{1.585058in}}%
\pgfpathlineto{\pgfqpoint{2.484543in}{1.583431in}}%
\pgfpathlineto{\pgfqpoint{2.485617in}{1.584293in}}%
\pgfpathlineto{\pgfqpoint{2.489912in}{1.585776in}}%
\pgfpathlineto{\pgfqpoint{2.490986in}{1.582187in}}%
\pgfpathlineto{\pgfqpoint{2.492060in}{1.583814in}}%
\pgfpathlineto{\pgfqpoint{2.493134in}{1.579650in}}%
\pgfpathlineto{\pgfqpoint{2.497429in}{1.585106in}}%
\pgfpathlineto{\pgfqpoint{2.498503in}{1.585393in}}%
\pgfpathlineto{\pgfqpoint{2.499577in}{1.587404in}}%
\pgfpathlineto{\pgfqpoint{2.500650in}{1.592812in}}%
\pgfpathlineto{\pgfqpoint{2.506020in}{1.585824in}}%
\pgfpathlineto{\pgfqpoint{2.507093in}{1.585058in}}%
\pgfpathlineto{\pgfqpoint{2.508167in}{1.582330in}}%
\pgfpathlineto{\pgfqpoint{2.512463in}{1.581230in}}%
\pgfpathlineto{\pgfqpoint{2.514610in}{1.584628in}}%
\pgfpathlineto{\pgfqpoint{2.515684in}{1.584915in}}%
\pgfpathlineto{\pgfqpoint{2.518906in}{1.582905in}}%
\pgfpathlineto{\pgfqpoint{2.519979in}{1.588600in}}%
\pgfpathlineto{\pgfqpoint{2.523201in}{1.579985in}}%
\pgfpathlineto{\pgfqpoint{2.526423in}{1.577640in}}%
\pgfpathlineto{\pgfqpoint{2.527496in}{1.579890in}}%
\pgfpathlineto{\pgfqpoint{2.528570in}{1.573189in}}%
\pgfpathlineto{\pgfqpoint{2.529644in}{1.574003in}}%
\pgfpathlineto{\pgfqpoint{2.530718in}{1.579794in}}%
\pgfpathlineto{\pgfqpoint{2.533939in}{1.584388in}}%
\pgfpathlineto{\pgfqpoint{2.535013in}{1.586973in}}%
\pgfpathlineto{\pgfqpoint{2.536087in}{1.583144in}}%
\pgfpathlineto{\pgfqpoint{2.537161in}{1.581756in}}%
\pgfpathlineto{\pgfqpoint{2.538235in}{1.586063in}}%
\pgfpathlineto{\pgfqpoint{2.541456in}{1.590419in}}%
\pgfpathlineto{\pgfqpoint{2.542530in}{1.588409in}}%
\pgfpathlineto{\pgfqpoint{2.543604in}{1.592907in}}%
\pgfpathlineto{\pgfqpoint{2.545752in}{1.594008in}}%
\pgfpathlineto{\pgfqpoint{2.551121in}{1.596305in}}%
\pgfpathlineto{\pgfqpoint{2.552195in}{1.593721in}}%
\pgfpathlineto{\pgfqpoint{2.553268in}{1.595396in}}%
\pgfpathlineto{\pgfqpoint{2.556490in}{1.596975in}}%
\pgfpathlineto{\pgfqpoint{2.557564in}{1.595827in}}%
\pgfpathlineto{\pgfqpoint{2.558638in}{1.600948in}}%
\pgfpathlineto{\pgfqpoint{2.559712in}{1.596449in}}%
\pgfpathlineto{\pgfqpoint{2.560785in}{1.594870in}}%
\pgfpathlineto{\pgfqpoint{2.564007in}{1.591807in}}%
\pgfpathlineto{\pgfqpoint{2.565081in}{1.594582in}}%
\pgfpathlineto{\pgfqpoint{2.566155in}{1.592189in}}%
\pgfpathlineto{\pgfqpoint{2.568302in}{1.593960in}}%
\pgfpathlineto{\pgfqpoint{2.571524in}{1.594726in}}%
\pgfpathlineto{\pgfqpoint{2.572598in}{1.592572in}}%
\pgfpathlineto{\pgfqpoint{2.573671in}{1.598220in}}%
\pgfpathlineto{\pgfqpoint{2.574745in}{1.594870in}}%
\pgfpathlineto{\pgfqpoint{2.575819in}{1.599895in}}%
\pgfpathlineto{\pgfqpoint{2.579041in}{1.600182in}}%
\pgfpathlineto{\pgfqpoint{2.580114in}{1.594056in}}%
\pgfpathlineto{\pgfqpoint{2.582262in}{1.592620in}}%
\pgfpathlineto{\pgfqpoint{2.583336in}{1.592429in}}%
\pgfpathlineto{\pgfqpoint{2.586557in}{1.592572in}}%
\pgfpathlineto{\pgfqpoint{2.587631in}{1.596688in}}%
\pgfpathlineto{\pgfqpoint{2.588705in}{1.593625in}}%
\pgfpathlineto{\pgfqpoint{2.589779in}{1.595348in}}%
\pgfpathlineto{\pgfqpoint{2.590853in}{1.594295in}}%
\pgfpathlineto{\pgfqpoint{2.594074in}{1.593003in}}%
\pgfpathlineto{\pgfqpoint{2.595148in}{1.597789in}}%
\pgfpathlineto{\pgfqpoint{2.596222in}{1.595587in}}%
\pgfpathlineto{\pgfqpoint{2.597296in}{1.595205in}}%
\pgfpathlineto{\pgfqpoint{2.598370in}{1.597789in}}%
\pgfpathlineto{\pgfqpoint{2.601591in}{1.597310in}}%
\pgfpathlineto{\pgfqpoint{2.602665in}{1.598507in}}%
\pgfpathlineto{\pgfqpoint{2.603739in}{1.594439in}}%
\pgfpathlineto{\pgfqpoint{2.604813in}{1.593721in}}%
\pgfpathlineto{\pgfqpoint{2.610182in}{1.600373in}}%
\pgfpathlineto{\pgfqpoint{2.611256in}{1.597502in}}%
\pgfpathlineto{\pgfqpoint{2.613403in}{1.606834in}}%
\pgfpathlineto{\pgfqpoint{2.617699in}{1.615353in}}%
\pgfpathlineto{\pgfqpoint{2.618773in}{1.621384in}}%
\pgfpathlineto{\pgfqpoint{2.619846in}{1.623968in}}%
\pgfpathlineto{\pgfqpoint{2.620920in}{1.624925in}}%
\pgfpathlineto{\pgfqpoint{2.624142in}{1.624638in}}%
\pgfpathlineto{\pgfqpoint{2.625216in}{1.626409in}}%
\pgfpathlineto{\pgfqpoint{2.627363in}{1.634018in}}%
\pgfpathlineto{\pgfqpoint{2.628437in}{1.635598in}}%
\pgfpathlineto{\pgfqpoint{2.631659in}{1.634401in}}%
\pgfpathlineto{\pgfqpoint{2.632733in}{1.636124in}}%
\pgfpathlineto{\pgfqpoint{2.633806in}{1.634114in}}%
\pgfpathlineto{\pgfqpoint{2.634880in}{1.635311in}}%
\pgfpathlineto{\pgfqpoint{2.635954in}{1.633253in}}%
\pgfpathlineto{\pgfqpoint{2.639176in}{1.633636in}}%
\pgfpathlineto{\pgfqpoint{2.640249in}{1.635837in}}%
\pgfpathlineto{\pgfqpoint{2.641323in}{1.630573in}}%
\pgfpathlineto{\pgfqpoint{2.642397in}{1.629615in}}%
\pgfpathlineto{\pgfqpoint{2.643471in}{1.637752in}}%
\pgfpathlineto{\pgfqpoint{2.646692in}{1.640001in}}%
\pgfpathlineto{\pgfqpoint{2.647766in}{1.641867in}}%
\pgfpathlineto{\pgfqpoint{2.648840in}{1.641915in}}%
\pgfpathlineto{\pgfqpoint{2.649914in}{1.642681in}}%
\pgfpathlineto{\pgfqpoint{2.650988in}{1.640862in}}%
\pgfpathlineto{\pgfqpoint{2.655283in}{1.638230in}}%
\pgfpathlineto{\pgfqpoint{2.656357in}{1.638278in}}%
\pgfpathlineto{\pgfqpoint{2.658505in}{1.642107in}}%
\pgfpathlineto{\pgfqpoint{2.661726in}{1.636938in}}%
\pgfpathlineto{\pgfqpoint{2.662800in}{1.633013in}}%
\pgfpathlineto{\pgfqpoint{2.663874in}{1.631482in}}%
\pgfpathlineto{\pgfqpoint{2.664948in}{1.632200in}}%
\pgfpathlineto{\pgfqpoint{2.666022in}{1.635359in}}%
\pgfpathlineto{\pgfqpoint{2.669243in}{1.631913in}}%
\pgfpathlineto{\pgfqpoint{2.670317in}{1.632726in}}%
\pgfpathlineto{\pgfqpoint{2.671391in}{1.632535in}}%
\pgfpathlineto{\pgfqpoint{2.672465in}{1.635885in}}%
\pgfpathlineto{\pgfqpoint{2.673538in}{1.634401in}}%
\pgfpathlineto{\pgfqpoint{2.676760in}{1.640862in}}%
\pgfpathlineto{\pgfqpoint{2.677834in}{1.639331in}}%
\pgfpathlineto{\pgfqpoint{2.678908in}{1.640432in}}%
\pgfpathlineto{\pgfqpoint{2.679981in}{1.642346in}}%
\pgfpathlineto{\pgfqpoint{2.681055in}{1.642537in}}%
\pgfpathlineto{\pgfqpoint{2.684277in}{1.640958in}}%
\pgfpathlineto{\pgfqpoint{2.685351in}{1.639714in}}%
\pgfpathlineto{\pgfqpoint{2.686424in}{1.643830in}}%
\pgfpathlineto{\pgfqpoint{2.687498in}{1.639953in}}%
\pgfpathlineto{\pgfqpoint{2.688572in}{1.642059in}}%
\pgfpathlineto{\pgfqpoint{2.691794in}{1.641915in}}%
\pgfpathlineto{\pgfqpoint{2.693941in}{1.645170in}}%
\pgfpathlineto{\pgfqpoint{2.695015in}{1.641054in}}%
\pgfpathlineto{\pgfqpoint{2.696089in}{1.644308in}}%
\pgfpathlineto{\pgfqpoint{2.699311in}{1.646366in}}%
\pgfpathlineto{\pgfqpoint{2.700384in}{1.648568in}}%
\pgfpathlineto{\pgfqpoint{2.701458in}{1.649333in}}%
\pgfpathlineto{\pgfqpoint{2.702532in}{1.646414in}}%
\pgfpathlineto{\pgfqpoint{2.703606in}{1.647946in}}%
\pgfpathlineto{\pgfqpoint{2.706827in}{1.646605in}}%
\pgfpathlineto{\pgfqpoint{2.707901in}{1.644404in}}%
\pgfpathlineto{\pgfqpoint{2.708975in}{1.646462in}}%
\pgfpathlineto{\pgfqpoint{2.710049in}{1.643590in}}%
\pgfpathlineto{\pgfqpoint{2.711123in}{1.648328in}}%
\pgfpathlineto{\pgfqpoint{2.714344in}{1.646701in}}%
\pgfpathlineto{\pgfqpoint{2.715418in}{1.634354in}}%
\pgfpathlineto{\pgfqpoint{2.717566in}{1.626505in}}%
\pgfpathlineto{\pgfqpoint{2.718640in}{1.627318in}}%
\pgfpathlineto{\pgfqpoint{2.721861in}{1.625978in}}%
\pgfpathlineto{\pgfqpoint{2.722935in}{1.627366in}}%
\pgfpathlineto{\pgfqpoint{2.725083in}{1.637464in}}%
\pgfpathlineto{\pgfqpoint{2.726156in}{1.639235in}}%
\pgfpathlineto{\pgfqpoint{2.729378in}{1.625739in}}%
\pgfpathlineto{\pgfqpoint{2.730452in}{1.624255in}}%
\pgfpathlineto{\pgfqpoint{2.731526in}{1.619756in}}%
\pgfpathlineto{\pgfqpoint{2.732600in}{1.617746in}}%
\pgfpathlineto{\pgfqpoint{2.736895in}{1.619517in}}%
\pgfpathlineto{\pgfqpoint{2.737969in}{1.614109in}}%
\pgfpathlineto{\pgfqpoint{2.739043in}{1.626648in}}%
\pgfpathlineto{\pgfqpoint{2.740116in}{1.617938in}}%
\pgfpathlineto{\pgfqpoint{2.741190in}{1.615114in}}%
\pgfpathlineto{\pgfqpoint{2.744412in}{1.614205in}}%
\pgfpathlineto{\pgfqpoint{2.745486in}{1.616071in}}%
\pgfpathlineto{\pgfqpoint{2.746559in}{1.622197in}}%
\pgfpathlineto{\pgfqpoint{2.747633in}{1.614300in}}%
\pgfpathlineto{\pgfqpoint{2.748707in}{1.613248in}}%
\pgfpathlineto{\pgfqpoint{2.751929in}{1.614444in}}%
\pgfpathlineto{\pgfqpoint{2.753002in}{1.629855in}}%
\pgfpathlineto{\pgfqpoint{2.754076in}{1.633971in}}%
\pgfpathlineto{\pgfqpoint{2.755150in}{1.634497in}}%
\pgfpathlineto{\pgfqpoint{2.761593in}{1.588265in}}%
\pgfpathlineto{\pgfqpoint{2.762667in}{1.589940in}}%
\pgfpathlineto{\pgfqpoint{2.763741in}{1.587978in}}%
\pgfpathlineto{\pgfqpoint{2.766962in}{1.588409in}}%
\pgfpathlineto{\pgfqpoint{2.768036in}{1.589414in}}%
\pgfpathlineto{\pgfqpoint{2.769110in}{1.591328in}}%
\pgfpathlineto{\pgfqpoint{2.770184in}{1.604394in}}%
\pgfpathlineto{\pgfqpoint{2.771258in}{1.604154in}}%
\pgfpathlineto{\pgfqpoint{2.774479in}{1.602719in}}%
\pgfpathlineto{\pgfqpoint{2.776627in}{1.608797in}}%
\pgfpathlineto{\pgfqpoint{2.778775in}{1.612960in}}%
\pgfpathlineto{\pgfqpoint{2.781996in}{1.609658in}}%
\pgfpathlineto{\pgfqpoint{2.783070in}{1.611620in}}%
\pgfpathlineto{\pgfqpoint{2.784144in}{1.622867in}}%
\pgfpathlineto{\pgfqpoint{2.785218in}{1.616741in}}%
\pgfpathlineto{\pgfqpoint{2.786291in}{1.617842in}}%
\pgfpathlineto{\pgfqpoint{2.789513in}{1.624734in}}%
\pgfpathlineto{\pgfqpoint{2.790587in}{1.625308in}}%
\pgfpathlineto{\pgfqpoint{2.791661in}{1.624925in}}%
\pgfpathlineto{\pgfqpoint{2.792734in}{1.627414in}}%
\pgfpathlineto{\pgfqpoint{2.793808in}{1.627701in}}%
\pgfpathlineto{\pgfqpoint{2.797030in}{1.629520in}}%
\pgfpathlineto{\pgfqpoint{2.799178in}{1.625500in}}%
\pgfpathlineto{\pgfqpoint{2.800251in}{1.630381in}}%
\pgfpathlineto{\pgfqpoint{2.801325in}{1.629903in}}%
\pgfpathlineto{\pgfqpoint{2.804547in}{1.631243in}}%
\pgfpathlineto{\pgfqpoint{2.805621in}{1.632870in}}%
\pgfpathlineto{\pgfqpoint{2.806694in}{1.632008in}}%
\pgfpathlineto{\pgfqpoint{2.807768in}{1.633444in}}%
\pgfpathlineto{\pgfqpoint{2.808842in}{1.640049in}}%
\pgfpathlineto{\pgfqpoint{2.812064in}{1.639857in}}%
\pgfpathlineto{\pgfqpoint{2.814211in}{1.631147in}}%
\pgfpathlineto{\pgfqpoint{2.815285in}{1.635454in}}%
\pgfpathlineto{\pgfqpoint{2.816359in}{1.631530in}}%
\pgfpathlineto{\pgfqpoint{2.819580in}{1.635024in}}%
\pgfpathlineto{\pgfqpoint{2.820654in}{1.634689in}}%
\pgfpathlineto{\pgfqpoint{2.821728in}{1.636316in}}%
\pgfpathlineto{\pgfqpoint{2.822802in}{1.642442in}}%
\pgfpathlineto{\pgfqpoint{2.823876in}{1.640671in}}%
\pgfpathlineto{\pgfqpoint{2.827097in}{1.637560in}}%
\pgfpathlineto{\pgfqpoint{2.828171in}{1.639187in}}%
\pgfpathlineto{\pgfqpoint{2.829245in}{1.637081in}}%
\pgfpathlineto{\pgfqpoint{2.830319in}{1.628323in}}%
\pgfpathlineto{\pgfqpoint{2.831393in}{1.627127in}}%
\pgfpathlineto{\pgfqpoint{2.834614in}{1.622389in}}%
\pgfpathlineto{\pgfqpoint{2.835688in}{1.630285in}}%
\pgfpathlineto{\pgfqpoint{2.836762in}{1.624686in}}%
\pgfpathlineto{\pgfqpoint{2.837836in}{1.629376in}}%
\pgfpathlineto{\pgfqpoint{2.838910in}{1.623107in}}%
\pgfpathlineto{\pgfqpoint{2.842131in}{1.622389in}}%
\pgfpathlineto{\pgfqpoint{2.843205in}{1.625452in}}%
\pgfpathlineto{\pgfqpoint{2.844279in}{1.624016in}}%
\pgfpathlineto{\pgfqpoint{2.849648in}{1.625835in}}%
\pgfpathlineto{\pgfqpoint{2.851796in}{1.630620in}}%
\pgfpathlineto{\pgfqpoint{2.852869in}{1.645744in}}%
\pgfpathlineto{\pgfqpoint{2.853943in}{1.639618in}}%
\pgfpathlineto{\pgfqpoint{2.857165in}{1.639235in}}%
\pgfpathlineto{\pgfqpoint{2.858239in}{1.640384in}}%
\pgfpathlineto{\pgfqpoint{2.861460in}{1.652014in}}%
\pgfpathlineto{\pgfqpoint{2.865756in}{1.655077in}}%
\pgfpathlineto{\pgfqpoint{2.866829in}{1.658857in}}%
\pgfpathlineto{\pgfqpoint{2.867903in}{1.656369in}}%
\pgfpathlineto{\pgfqpoint{2.868977in}{1.667472in}}%
\pgfpathlineto{\pgfqpoint{2.875420in}{1.672689in}}%
\pgfpathlineto{\pgfqpoint{2.876494in}{1.672306in}}%
\pgfpathlineto{\pgfqpoint{2.880789in}{1.671923in}}%
\pgfpathlineto{\pgfqpoint{2.882937in}{1.676422in}}%
\pgfpathlineto{\pgfqpoint{2.884011in}{1.673885in}}%
\pgfpathlineto{\pgfqpoint{2.884011in}{1.673885in}}%
\pgfusepath{stroke}%
\end{pgfscope}%
\begin{pgfscope}%
\pgfpathrectangle{\pgfqpoint{0.418102in}{1.293759in}}{\pgfqpoint{2.583333in}{0.400885in}}%
\pgfusepath{clip}%
\pgfsetroundcap%
\pgfsetroundjoin%
\pgfsetlinewidth{1.505625pt}%
\definecolor{currentstroke}{rgb}{0.890196,0.466667,0.760784}%
\pgfsetstrokecolor{currentstroke}%
\pgfsetdash{}{0pt}%
\pgfpathmoveto{\pgfqpoint{0.535526in}{1.384815in}}%
\pgfpathlineto{\pgfqpoint{0.536600in}{1.384815in}}%
\pgfpathlineto{\pgfqpoint{0.537674in}{1.387672in}}%
\pgfpathlineto{\pgfqpoint{0.538747in}{1.385925in}}%
\pgfpathlineto{\pgfqpoint{0.541969in}{1.386750in}}%
\pgfpathlineto{\pgfqpoint{0.543043in}{1.378935in}}%
\pgfpathlineto{\pgfqpoint{0.545190in}{1.374319in}}%
\pgfpathlineto{\pgfqpoint{0.546264in}{1.378648in}}%
\pgfpathlineto{\pgfqpoint{0.550560in}{1.382327in}}%
\pgfpathlineto{\pgfqpoint{0.551633in}{1.380166in}}%
\pgfpathlineto{\pgfqpoint{0.552707in}{1.381752in}}%
\pgfpathlineto{\pgfqpoint{0.553781in}{1.379873in}}%
\pgfpathlineto{\pgfqpoint{0.557003in}{1.380515in}}%
\pgfpathlineto{\pgfqpoint{0.558077in}{1.376987in}}%
\pgfpathlineto{\pgfqpoint{0.559150in}{1.377479in}}%
\pgfpathlineto{\pgfqpoint{0.560224in}{1.376581in}}%
\pgfpathlineto{\pgfqpoint{0.561298in}{1.377344in}}%
\pgfpathlineto{\pgfqpoint{0.564520in}{1.377389in}}%
\pgfpathlineto{\pgfqpoint{0.565593in}{1.380173in}}%
\pgfpathlineto{\pgfqpoint{0.566667in}{1.373169in}}%
\pgfpathlineto{\pgfqpoint{0.567741in}{1.373896in}}%
\pgfpathlineto{\pgfqpoint{0.568815in}{1.377596in}}%
\pgfpathlineto{\pgfqpoint{0.572036in}{1.379317in}}%
\pgfpathlineto{\pgfqpoint{0.573110in}{1.378272in}}%
\pgfpathlineto{\pgfqpoint{0.574184in}{1.383583in}}%
\pgfpathlineto{\pgfqpoint{0.575258in}{1.376183in}}%
\pgfpathlineto{\pgfqpoint{0.576332in}{1.377176in}}%
\pgfpathlineto{\pgfqpoint{0.580627in}{1.382806in}}%
\pgfpathlineto{\pgfqpoint{0.581701in}{1.377354in}}%
\pgfpathlineto{\pgfqpoint{0.582775in}{1.379535in}}%
\pgfpathlineto{\pgfqpoint{0.583849in}{1.377606in}}%
\pgfpathlineto{\pgfqpoint{0.589218in}{1.376902in}}%
\pgfpathlineto{\pgfqpoint{0.590292in}{1.375296in}}%
\pgfpathlineto{\pgfqpoint{0.591366in}{1.376944in}}%
\pgfpathlineto{\pgfqpoint{0.594587in}{1.378097in}}%
\pgfpathlineto{\pgfqpoint{0.595661in}{1.377516in}}%
\pgfpathlineto{\pgfqpoint{0.596735in}{1.378928in}}%
\pgfpathlineto{\pgfqpoint{0.598882in}{1.376611in}}%
\pgfpathlineto{\pgfqpoint{0.602104in}{1.380903in}}%
\pgfpathlineto{\pgfqpoint{0.603178in}{1.374209in}}%
\pgfpathlineto{\pgfqpoint{0.604252in}{1.378377in}}%
\pgfpathlineto{\pgfqpoint{0.605325in}{1.374966in}}%
\pgfpathlineto{\pgfqpoint{0.606399in}{1.374925in}}%
\pgfpathlineto{\pgfqpoint{0.609621in}{1.376447in}}%
\pgfpathlineto{\pgfqpoint{0.610695in}{1.367566in}}%
\pgfpathlineto{\pgfqpoint{0.611768in}{1.366792in}}%
\pgfpathlineto{\pgfqpoint{0.612842in}{1.367062in}}%
\pgfpathlineto{\pgfqpoint{0.613916in}{1.362524in}}%
\pgfpathlineto{\pgfqpoint{0.617138in}{1.359755in}}%
\pgfpathlineto{\pgfqpoint{0.618211in}{1.355371in}}%
\pgfpathlineto{\pgfqpoint{0.621433in}{1.350680in}}%
\pgfpathlineto{\pgfqpoint{0.625728in}{1.357409in}}%
\pgfpathlineto{\pgfqpoint{0.626802in}{1.352246in}}%
\pgfpathlineto{\pgfqpoint{0.627876in}{1.354847in}}%
\pgfpathlineto{\pgfqpoint{0.628950in}{1.352829in}}%
\pgfpathlineto{\pgfqpoint{0.632171in}{1.353517in}}%
\pgfpathlineto{\pgfqpoint{0.634319in}{1.350942in}}%
\pgfpathlineto{\pgfqpoint{0.639688in}{1.345484in}}%
\pgfpathlineto{\pgfqpoint{0.640762in}{1.339834in}}%
\pgfpathlineto{\pgfqpoint{0.641836in}{1.343562in}}%
\pgfpathlineto{\pgfqpoint{0.642910in}{1.338604in}}%
\pgfpathlineto{\pgfqpoint{0.643984in}{1.342856in}}%
\pgfpathlineto{\pgfqpoint{0.647205in}{1.343124in}}%
\pgfpathlineto{\pgfqpoint{0.648279in}{1.347485in}}%
\pgfpathlineto{\pgfqpoint{0.649353in}{1.347638in}}%
\pgfpathlineto{\pgfqpoint{0.650427in}{1.345072in}}%
\pgfpathlineto{\pgfqpoint{0.651500in}{1.346986in}}%
\pgfpathlineto{\pgfqpoint{0.654722in}{1.350969in}}%
\pgfpathlineto{\pgfqpoint{0.656870in}{1.351404in}}%
\pgfpathlineto{\pgfqpoint{0.657944in}{1.356146in}}%
\pgfpathlineto{\pgfqpoint{0.659017in}{1.353656in}}%
\pgfpathlineto{\pgfqpoint{0.662239in}{1.354820in}}%
\pgfpathlineto{\pgfqpoint{0.663313in}{1.354155in}}%
\pgfpathlineto{\pgfqpoint{0.664387in}{1.354546in}}%
\pgfpathlineto{\pgfqpoint{0.665460in}{1.356660in}}%
\pgfpathlineto{\pgfqpoint{0.666534in}{1.351888in}}%
\pgfpathlineto{\pgfqpoint{0.670830in}{1.349702in}}%
\pgfpathlineto{\pgfqpoint{0.671903in}{1.343658in}}%
\pgfpathlineto{\pgfqpoint{0.674051in}{1.345010in}}%
\pgfpathlineto{\pgfqpoint{0.677273in}{1.340902in}}%
\pgfpathlineto{\pgfqpoint{0.678346in}{1.341659in}}%
\pgfpathlineto{\pgfqpoint{0.679420in}{1.345010in}}%
\pgfpathlineto{\pgfqpoint{0.681568in}{1.337025in}}%
\pgfpathlineto{\pgfqpoint{0.684789in}{1.341570in}}%
\pgfpathlineto{\pgfqpoint{0.686937in}{1.340344in}}%
\pgfpathlineto{\pgfqpoint{0.688011in}{1.342250in}}%
\pgfpathlineto{\pgfqpoint{0.689085in}{1.340562in}}%
\pgfpathlineto{\pgfqpoint{0.693380in}{1.347767in}}%
\pgfpathlineto{\pgfqpoint{0.694454in}{1.352461in}}%
\pgfpathlineto{\pgfqpoint{0.695528in}{1.353784in}}%
\pgfpathlineto{\pgfqpoint{0.696602in}{1.361295in}}%
\pgfpathlineto{\pgfqpoint{0.699823in}{1.359128in}}%
\pgfpathlineto{\pgfqpoint{0.700897in}{1.356961in}}%
\pgfpathlineto{\pgfqpoint{0.701971in}{1.367481in}}%
\pgfpathlineto{\pgfqpoint{0.703045in}{1.360799in}}%
\pgfpathlineto{\pgfqpoint{0.704119in}{1.360411in}}%
\pgfpathlineto{\pgfqpoint{0.707340in}{1.363631in}}%
\pgfpathlineto{\pgfqpoint{0.708414in}{1.362664in}}%
\pgfpathlineto{\pgfqpoint{0.709488in}{1.359631in}}%
\pgfpathlineto{\pgfqpoint{0.710562in}{1.361873in}}%
\pgfpathlineto{\pgfqpoint{0.711635in}{1.360598in}}%
\pgfpathlineto{\pgfqpoint{0.714857in}{1.358726in}}%
\pgfpathlineto{\pgfqpoint{0.715931in}{1.353614in}}%
\pgfpathlineto{\pgfqpoint{0.717005in}{1.355351in}}%
\pgfpathlineto{\pgfqpoint{0.718078in}{1.351496in}}%
\pgfpathlineto{\pgfqpoint{0.719152in}{1.352837in}}%
\pgfpathlineto{\pgfqpoint{0.722374in}{1.357950in}}%
\pgfpathlineto{\pgfqpoint{0.723448in}{1.357514in}}%
\pgfpathlineto{\pgfqpoint{0.724521in}{1.359474in}}%
\pgfpathlineto{\pgfqpoint{0.725595in}{1.365224in}}%
\pgfpathlineto{\pgfqpoint{0.726669in}{1.376446in}}%
\pgfpathlineto{\pgfqpoint{0.729891in}{1.378308in}}%
\pgfpathlineto{\pgfqpoint{0.730965in}{1.381070in}}%
\pgfpathlineto{\pgfqpoint{0.733112in}{1.382496in}}%
\pgfpathlineto{\pgfqpoint{0.734186in}{1.377431in}}%
\pgfpathlineto{\pgfqpoint{0.738481in}{1.379058in}}%
\pgfpathlineto{\pgfqpoint{0.739555in}{1.372567in}}%
\pgfpathlineto{\pgfqpoint{0.740629in}{1.369952in}}%
\pgfpathlineto{\pgfqpoint{0.741703in}{1.376863in}}%
\pgfpathlineto{\pgfqpoint{0.744924in}{1.378357in}}%
\pgfpathlineto{\pgfqpoint{0.745998in}{1.379867in}}%
\pgfpathlineto{\pgfqpoint{0.748146in}{1.371127in}}%
\pgfpathlineto{\pgfqpoint{0.749220in}{1.377080in}}%
\pgfpathlineto{\pgfqpoint{0.752441in}{1.373328in}}%
\pgfpathlineto{\pgfqpoint{0.753515in}{1.368093in}}%
\pgfpathlineto{\pgfqpoint{0.754589in}{1.370734in}}%
\pgfpathlineto{\pgfqpoint{0.755663in}{1.369483in}}%
\pgfpathlineto{\pgfqpoint{0.756737in}{1.364248in}}%
\pgfpathlineto{\pgfqpoint{0.759958in}{1.361682in}}%
\pgfpathlineto{\pgfqpoint{0.762106in}{1.364958in}}%
\pgfpathlineto{\pgfqpoint{0.763180in}{1.366280in}}%
\pgfpathlineto{\pgfqpoint{0.764254in}{1.375176in}}%
\pgfpathlineto{\pgfqpoint{0.767475in}{1.375754in}}%
\pgfpathlineto{\pgfqpoint{0.768549in}{1.381292in}}%
\pgfpathlineto{\pgfqpoint{0.769623in}{1.383703in}}%
\pgfpathlineto{\pgfqpoint{0.770697in}{1.381978in}}%
\pgfpathlineto{\pgfqpoint{0.771770in}{1.385247in}}%
\pgfpathlineto{\pgfqpoint{0.774992in}{1.386836in}}%
\pgfpathlineto{\pgfqpoint{0.776066in}{1.388029in}}%
\pgfpathlineto{\pgfqpoint{0.778213in}{1.382167in}}%
\pgfpathlineto{\pgfqpoint{0.779287in}{1.376167in}}%
\pgfpathlineto{\pgfqpoint{0.782509in}{1.377183in}}%
\pgfpathlineto{\pgfqpoint{0.783583in}{1.374240in}}%
\pgfpathlineto{\pgfqpoint{0.785730in}{1.377311in}}%
\pgfpathlineto{\pgfqpoint{0.786804in}{1.380548in}}%
\pgfpathlineto{\pgfqpoint{0.791099in}{1.378523in}}%
\pgfpathlineto{\pgfqpoint{0.792173in}{1.380226in}}%
\pgfpathlineto{\pgfqpoint{0.793247in}{1.376178in}}%
\pgfpathlineto{\pgfqpoint{0.794321in}{1.379063in}}%
\pgfpathlineto{\pgfqpoint{0.798616in}{1.384533in}}%
\pgfpathlineto{\pgfqpoint{0.799690in}{1.383325in}}%
\pgfpathlineto{\pgfqpoint{0.800764in}{1.388512in}}%
\pgfpathlineto{\pgfqpoint{0.801838in}{1.388467in}}%
\pgfpathlineto{\pgfqpoint{0.805059in}{1.385428in}}%
\pgfpathlineto{\pgfqpoint{0.806133in}{1.386009in}}%
\pgfpathlineto{\pgfqpoint{0.807207in}{1.385741in}}%
\pgfpathlineto{\pgfqpoint{0.809355in}{1.372350in}}%
\pgfpathlineto{\pgfqpoint{0.812576in}{1.373368in}}%
\pgfpathlineto{\pgfqpoint{0.813650in}{1.371029in}}%
\pgfpathlineto{\pgfqpoint{0.814724in}{1.371850in}}%
\pgfpathlineto{\pgfqpoint{0.815798in}{1.374641in}}%
\pgfpathlineto{\pgfqpoint{0.816872in}{1.374055in}}%
\pgfpathlineto{\pgfqpoint{0.820093in}{1.371460in}}%
\pgfpathlineto{\pgfqpoint{0.821167in}{1.366982in}}%
\pgfpathlineto{\pgfqpoint{0.823315in}{1.365517in}}%
\pgfpathlineto{\pgfqpoint{0.824388in}{1.365308in}}%
\pgfpathlineto{\pgfqpoint{0.828684in}{1.365391in}}%
\pgfpathlineto{\pgfqpoint{0.829758in}{1.364764in}}%
\pgfpathlineto{\pgfqpoint{0.831905in}{1.364806in}}%
\pgfpathlineto{\pgfqpoint{0.835127in}{1.364306in}}%
\pgfpathlineto{\pgfqpoint{0.837275in}{1.356015in}}%
\pgfpathlineto{\pgfqpoint{0.839422in}{1.356390in}}%
\pgfpathlineto{\pgfqpoint{0.842644in}{1.357058in}}%
\pgfpathlineto{\pgfqpoint{0.844791in}{1.348706in}}%
\pgfpathlineto{\pgfqpoint{0.845865in}{1.346392in}}%
\pgfpathlineto{\pgfqpoint{0.846939in}{1.350477in}}%
\pgfpathlineto{\pgfqpoint{0.850161in}{1.349960in}}%
\pgfpathlineto{\pgfqpoint{0.851234in}{1.347375in}}%
\pgfpathlineto{\pgfqpoint{0.854456in}{1.355379in}}%
\pgfpathlineto{\pgfqpoint{0.859825in}{1.355500in}}%
\pgfpathlineto{\pgfqpoint{0.861973in}{1.361971in}}%
\pgfpathlineto{\pgfqpoint{0.865194in}{1.361390in}}%
\pgfpathlineto{\pgfqpoint{0.867342in}{1.376692in}}%
\pgfpathlineto{\pgfqpoint{0.868416in}{1.371028in}}%
\pgfpathlineto{\pgfqpoint{0.869490in}{1.369843in}}%
\pgfpathlineto{\pgfqpoint{0.872711in}{1.373970in}}%
\pgfpathlineto{\pgfqpoint{0.873785in}{1.373794in}}%
\pgfpathlineto{\pgfqpoint{0.874859in}{1.380592in}}%
\pgfpathlineto{\pgfqpoint{0.875933in}{1.381327in}}%
\pgfpathlineto{\pgfqpoint{0.882376in}{1.371730in}}%
\pgfpathlineto{\pgfqpoint{0.884523in}{1.366994in}}%
\pgfpathlineto{\pgfqpoint{0.888819in}{1.366246in}}%
\pgfpathlineto{\pgfqpoint{0.889893in}{1.362808in}}%
\pgfpathlineto{\pgfqpoint{0.890966in}{1.362929in}}%
\pgfpathlineto{\pgfqpoint{0.892040in}{1.364142in}}%
\pgfpathlineto{\pgfqpoint{0.895262in}{1.365234in}}%
\pgfpathlineto{\pgfqpoint{0.896336in}{1.366417in}}%
\pgfpathlineto{\pgfqpoint{0.897409in}{1.366091in}}%
\pgfpathlineto{\pgfqpoint{0.898483in}{1.363854in}}%
\pgfpathlineto{\pgfqpoint{0.899557in}{1.363453in}}%
\pgfpathlineto{\pgfqpoint{0.902779in}{1.363374in}}%
\pgfpathlineto{\pgfqpoint{0.904926in}{1.364610in}}%
\pgfpathlineto{\pgfqpoint{0.907074in}{1.361124in}}%
\pgfpathlineto{\pgfqpoint{0.910296in}{1.361164in}}%
\pgfpathlineto{\pgfqpoint{0.911369in}{1.368818in}}%
\pgfpathlineto{\pgfqpoint{0.912443in}{1.366053in}}%
\pgfpathlineto{\pgfqpoint{0.913517in}{1.365581in}}%
\pgfpathlineto{\pgfqpoint{0.914591in}{1.367696in}}%
\pgfpathlineto{\pgfqpoint{0.921034in}{1.366066in}}%
\pgfpathlineto{\pgfqpoint{0.922108in}{1.361811in}}%
\pgfpathlineto{\pgfqpoint{0.925329in}{1.365867in}}%
\pgfpathlineto{\pgfqpoint{0.927477in}{1.359068in}}%
\pgfpathlineto{\pgfqpoint{0.928551in}{1.358083in}}%
\pgfpathlineto{\pgfqpoint{0.929625in}{1.355941in}}%
\pgfpathlineto{\pgfqpoint{0.932846in}{1.357236in}}%
\pgfpathlineto{\pgfqpoint{0.933920in}{1.353985in}}%
\pgfpathlineto{\pgfqpoint{0.934994in}{1.357198in}}%
\pgfpathlineto{\pgfqpoint{0.937142in}{1.355225in}}%
\pgfpathlineto{\pgfqpoint{0.941437in}{1.352765in}}%
\pgfpathlineto{\pgfqpoint{0.942511in}{1.353958in}}%
\pgfpathlineto{\pgfqpoint{0.943585in}{1.356548in}}%
\pgfpathlineto{\pgfqpoint{0.944658in}{1.354834in}}%
\pgfpathlineto{\pgfqpoint{0.948954in}{1.353249in}}%
\pgfpathlineto{\pgfqpoint{0.951101in}{1.348871in}}%
\pgfpathlineto{\pgfqpoint{0.952175in}{1.346289in}}%
\pgfpathlineto{\pgfqpoint{0.956471in}{1.346120in}}%
\pgfpathlineto{\pgfqpoint{0.957544in}{1.344230in}}%
\pgfpathlineto{\pgfqpoint{0.958618in}{1.339808in}}%
\pgfpathlineto{\pgfqpoint{0.959692in}{1.346356in}}%
\pgfpathlineto{\pgfqpoint{0.962914in}{1.347234in}}%
\pgfpathlineto{\pgfqpoint{0.963987in}{1.346418in}}%
\pgfpathlineto{\pgfqpoint{0.965061in}{1.346758in}}%
\pgfpathlineto{\pgfqpoint{0.966135in}{1.345093in}}%
\pgfpathlineto{\pgfqpoint{0.967209in}{1.344690in}}%
\pgfpathlineto{\pgfqpoint{0.970431in}{1.346061in}}%
\pgfpathlineto{\pgfqpoint{0.971504in}{1.347143in}}%
\pgfpathlineto{\pgfqpoint{0.972578in}{1.346974in}}%
\pgfpathlineto{\pgfqpoint{0.973652in}{1.347514in}}%
\pgfpathlineto{\pgfqpoint{0.974726in}{1.344341in}}%
\pgfpathlineto{\pgfqpoint{0.979021in}{1.343682in}}%
\pgfpathlineto{\pgfqpoint{0.980095in}{1.346108in}}%
\pgfpathlineto{\pgfqpoint{0.981169in}{1.343970in}}%
\pgfpathlineto{\pgfqpoint{0.982243in}{1.347076in}}%
\pgfpathlineto{\pgfqpoint{0.986538in}{1.355710in}}%
\pgfpathlineto{\pgfqpoint{0.987612in}{1.351578in}}%
\pgfpathlineto{\pgfqpoint{0.988686in}{1.351646in}}%
\pgfpathlineto{\pgfqpoint{0.992981in}{1.347502in}}%
\pgfpathlineto{\pgfqpoint{0.994055in}{1.353005in}}%
\pgfpathlineto{\pgfqpoint{0.995129in}{1.352835in}}%
\pgfpathlineto{\pgfqpoint{0.996203in}{1.354701in}}%
\pgfpathlineto{\pgfqpoint{1.000498in}{1.351226in}}%
\pgfpathlineto{\pgfqpoint{1.001572in}{1.350565in}}%
\pgfpathlineto{\pgfqpoint{1.002646in}{1.350729in}}%
\pgfpathlineto{\pgfqpoint{1.004793in}{1.352211in}}%
\pgfpathlineto{\pgfqpoint{1.008015in}{1.350757in}}%
\pgfpathlineto{\pgfqpoint{1.009089in}{1.351716in}}%
\pgfpathlineto{\pgfqpoint{1.010163in}{1.350757in}}%
\pgfpathlineto{\pgfqpoint{1.012310in}{1.356964in}}%
\pgfpathlineto{\pgfqpoint{1.015532in}{1.359243in}}%
\pgfpathlineto{\pgfqpoint{1.017679in}{1.358664in}}%
\pgfpathlineto{\pgfqpoint{1.018753in}{1.361084in}}%
\pgfpathlineto{\pgfqpoint{1.026270in}{1.363515in}}%
\pgfpathlineto{\pgfqpoint{1.027344in}{1.361963in}}%
\pgfpathlineto{\pgfqpoint{1.030565in}{1.365136in}}%
\pgfpathlineto{\pgfqpoint{1.033787in}{1.358985in}}%
\pgfpathlineto{\pgfqpoint{1.034861in}{1.359348in}}%
\pgfpathlineto{\pgfqpoint{1.038082in}{1.353162in}}%
\pgfpathlineto{\pgfqpoint{1.039156in}{1.356006in}}%
\pgfpathlineto{\pgfqpoint{1.040230in}{1.360737in}}%
\pgfpathlineto{\pgfqpoint{1.041304in}{1.356899in}}%
\pgfpathlineto{\pgfqpoint{1.042378in}{1.361833in}}%
\pgfpathlineto{\pgfqpoint{1.045599in}{1.360634in}}%
\pgfpathlineto{\pgfqpoint{1.046673in}{1.362808in}}%
\pgfpathlineto{\pgfqpoint{1.047747in}{1.360185in}}%
\pgfpathlineto{\pgfqpoint{1.048821in}{1.360496in}}%
\pgfpathlineto{\pgfqpoint{1.049895in}{1.363326in}}%
\pgfpathlineto{\pgfqpoint{1.054190in}{1.365721in}}%
\pgfpathlineto{\pgfqpoint{1.055264in}{1.364976in}}%
\pgfpathlineto{\pgfqpoint{1.056338in}{1.367105in}}%
\pgfpathlineto{\pgfqpoint{1.057411in}{1.362954in}}%
\pgfpathlineto{\pgfqpoint{1.060633in}{1.361817in}}%
\pgfpathlineto{\pgfqpoint{1.064928in}{1.356729in}}%
\pgfpathlineto{\pgfqpoint{1.068150in}{1.357552in}}%
\pgfpathlineto{\pgfqpoint{1.069224in}{1.360334in}}%
\pgfpathlineto{\pgfqpoint{1.070297in}{1.359142in}}%
\pgfpathlineto{\pgfqpoint{1.071371in}{1.360980in}}%
\pgfpathlineto{\pgfqpoint{1.072445in}{1.367269in}}%
\pgfpathlineto{\pgfqpoint{1.075667in}{1.367602in}}%
\pgfpathlineto{\pgfqpoint{1.076741in}{1.368502in}}%
\pgfpathlineto{\pgfqpoint{1.077814in}{1.372238in}}%
\pgfpathlineto{\pgfqpoint{1.079962in}{1.368641in}}%
\pgfpathlineto{\pgfqpoint{1.084257in}{1.371416in}}%
\pgfpathlineto{\pgfqpoint{1.087479in}{1.379703in}}%
\pgfpathlineto{\pgfqpoint{1.090700in}{1.380737in}}%
\pgfpathlineto{\pgfqpoint{1.091774in}{1.383696in}}%
\pgfpathlineto{\pgfqpoint{1.092848in}{1.379695in}}%
\pgfpathlineto{\pgfqpoint{1.093922in}{1.378894in}}%
\pgfpathlineto{\pgfqpoint{1.094996in}{1.384387in}}%
\pgfpathlineto{\pgfqpoint{1.098217in}{1.385514in}}%
\pgfpathlineto{\pgfqpoint{1.099291in}{1.384048in}}%
\pgfpathlineto{\pgfqpoint{1.100365in}{1.380897in}}%
\pgfpathlineto{\pgfqpoint{1.101439in}{1.385991in}}%
\pgfpathlineto{\pgfqpoint{1.105734in}{1.389893in}}%
\pgfpathlineto{\pgfqpoint{1.106808in}{1.386097in}}%
\pgfpathlineto{\pgfqpoint{1.107882in}{1.392279in}}%
\pgfpathlineto{\pgfqpoint{1.108956in}{1.386323in}}%
\pgfpathlineto{\pgfqpoint{1.110030in}{1.385798in}}%
\pgfpathlineto{\pgfqpoint{1.113251in}{1.383550in}}%
\pgfpathlineto{\pgfqpoint{1.114325in}{1.384487in}}%
\pgfpathlineto{\pgfqpoint{1.115399in}{1.380591in}}%
\pgfpathlineto{\pgfqpoint{1.116473in}{1.378912in}}%
\pgfpathlineto{\pgfqpoint{1.120768in}{1.386414in}}%
\pgfpathlineto{\pgfqpoint{1.122916in}{1.392354in}}%
\pgfpathlineto{\pgfqpoint{1.125063in}{1.386542in}}%
\pgfpathlineto{\pgfqpoint{1.128285in}{1.385143in}}%
\pgfpathlineto{\pgfqpoint{1.129359in}{1.381945in}}%
\pgfpathlineto{\pgfqpoint{1.130432in}{1.382641in}}%
\pgfpathlineto{\pgfqpoint{1.131506in}{1.388371in}}%
\pgfpathlineto{\pgfqpoint{1.135802in}{1.386698in}}%
\pgfpathlineto{\pgfqpoint{1.136875in}{1.387458in}}%
\pgfpathlineto{\pgfqpoint{1.137949in}{1.390097in}}%
\pgfpathlineto{\pgfqpoint{1.139023in}{1.388673in}}%
\pgfpathlineto{\pgfqpoint{1.140097in}{1.385267in}}%
\pgfpathlineto{\pgfqpoint{1.143319in}{1.386344in}}%
\pgfpathlineto{\pgfqpoint{1.144392in}{1.395153in}}%
\pgfpathlineto{\pgfqpoint{1.145466in}{1.395525in}}%
\pgfpathlineto{\pgfqpoint{1.146540in}{1.394948in}}%
\pgfpathlineto{\pgfqpoint{1.147614in}{1.395458in}}%
\pgfpathlineto{\pgfqpoint{1.150835in}{1.395152in}}%
\pgfpathlineto{\pgfqpoint{1.151909in}{1.396609in}}%
\pgfpathlineto{\pgfqpoint{1.152983in}{1.396609in}}%
\pgfpathlineto{\pgfqpoint{1.154057in}{1.401321in}}%
\pgfpathlineto{\pgfqpoint{1.155131in}{1.399592in}}%
\pgfpathlineto{\pgfqpoint{1.158352in}{1.402844in}}%
\pgfpathlineto{\pgfqpoint{1.159426in}{1.398565in}}%
\pgfpathlineto{\pgfqpoint{1.160500in}{1.401030in}}%
\pgfpathlineto{\pgfqpoint{1.161574in}{1.400345in}}%
\pgfpathlineto{\pgfqpoint{1.162648in}{1.402254in}}%
\pgfpathlineto{\pgfqpoint{1.165869in}{1.402012in}}%
\pgfpathlineto{\pgfqpoint{1.168017in}{1.409361in}}%
\pgfpathlineto{\pgfqpoint{1.169091in}{1.403010in}}%
\pgfpathlineto{\pgfqpoint{1.170164in}{1.403292in}}%
\pgfpathlineto{\pgfqpoint{1.174460in}{1.401175in}}%
\pgfpathlineto{\pgfqpoint{1.175534in}{1.399058in}}%
\pgfpathlineto{\pgfqpoint{1.176608in}{1.402904in}}%
\pgfpathlineto{\pgfqpoint{1.177681in}{1.403398in}}%
\pgfpathlineto{\pgfqpoint{1.180903in}{1.401876in}}%
\pgfpathlineto{\pgfqpoint{1.181977in}{1.394620in}}%
\pgfpathlineto{\pgfqpoint{1.183051in}{1.394726in}}%
\pgfpathlineto{\pgfqpoint{1.185198in}{1.397345in}}%
\pgfpathlineto{\pgfqpoint{1.189494in}{1.405296in}}%
\pgfpathlineto{\pgfqpoint{1.191641in}{1.402136in}}%
\pgfpathlineto{\pgfqpoint{1.195937in}{1.407307in}}%
\pgfpathlineto{\pgfqpoint{1.198084in}{1.397174in}}%
\pgfpathlineto{\pgfqpoint{1.199158in}{1.397174in}}%
\pgfpathlineto{\pgfqpoint{1.200232in}{1.398762in}}%
\pgfpathlineto{\pgfqpoint{1.204527in}{1.393054in}}%
\pgfpathlineto{\pgfqpoint{1.206675in}{1.388621in}}%
\pgfpathlineto{\pgfqpoint{1.207749in}{1.395125in}}%
\pgfpathlineto{\pgfqpoint{1.210970in}{1.394672in}}%
\pgfpathlineto{\pgfqpoint{1.215266in}{1.400357in}}%
\pgfpathlineto{\pgfqpoint{1.219561in}{1.394657in}}%
\pgfpathlineto{\pgfqpoint{1.221709in}{1.384113in}}%
\pgfpathlineto{\pgfqpoint{1.222783in}{1.385760in}}%
\pgfpathlineto{\pgfqpoint{1.226004in}{1.386452in}}%
\pgfpathlineto{\pgfqpoint{1.227078in}{1.382811in}}%
\pgfpathlineto{\pgfqpoint{1.228152in}{1.383043in}}%
\pgfpathlineto{\pgfqpoint{1.229226in}{1.391881in}}%
\pgfpathlineto{\pgfqpoint{1.230299in}{1.388736in}}%
\pgfpathlineto{\pgfqpoint{1.233521in}{1.389028in}}%
\pgfpathlineto{\pgfqpoint{1.234595in}{1.385909in}}%
\pgfpathlineto{\pgfqpoint{1.235669in}{1.387696in}}%
\pgfpathlineto{\pgfqpoint{1.236742in}{1.383213in}}%
\pgfpathlineto{\pgfqpoint{1.237816in}{1.383969in}}%
\pgfpathlineto{\pgfqpoint{1.241038in}{1.383621in}}%
\pgfpathlineto{\pgfqpoint{1.242112in}{1.379532in}}%
\pgfpathlineto{\pgfqpoint{1.243185in}{1.380229in}}%
\pgfpathlineto{\pgfqpoint{1.245333in}{1.377107in}}%
\pgfpathlineto{\pgfqpoint{1.251776in}{1.384561in}}%
\pgfpathlineto{\pgfqpoint{1.252850in}{1.381019in}}%
\pgfpathlineto{\pgfqpoint{1.256072in}{1.379729in}}%
\pgfpathlineto{\pgfqpoint{1.257145in}{1.380540in}}%
\pgfpathlineto{\pgfqpoint{1.259293in}{1.387060in}}%
\pgfpathlineto{\pgfqpoint{1.260367in}{1.390631in}}%
\pgfpathlineto{\pgfqpoint{1.263588in}{1.391949in}}%
\pgfpathlineto{\pgfqpoint{1.264662in}{1.390197in}}%
\pgfpathlineto{\pgfqpoint{1.265736in}{1.391884in}}%
\pgfpathlineto{\pgfqpoint{1.266810in}{1.388380in}}%
\pgfpathlineto{\pgfqpoint{1.267884in}{1.388665in}}%
\pgfpathlineto{\pgfqpoint{1.272179in}{1.390254in}}%
\pgfpathlineto{\pgfqpoint{1.273253in}{1.389523in}}%
\pgfpathlineto{\pgfqpoint{1.274327in}{1.392130in}}%
\pgfpathlineto{\pgfqpoint{1.275401in}{1.390922in}}%
\pgfpathlineto{\pgfqpoint{1.278622in}{1.390638in}}%
\pgfpathlineto{\pgfqpoint{1.279696in}{1.392369in}}%
\pgfpathlineto{\pgfqpoint{1.280770in}{1.390795in}}%
\pgfpathlineto{\pgfqpoint{1.282918in}{1.392571in}}%
\pgfpathlineto{\pgfqpoint{1.286139in}{1.391058in}}%
\pgfpathlineto{\pgfqpoint{1.288287in}{1.387275in}}%
\pgfpathlineto{\pgfqpoint{1.289361in}{1.387590in}}%
\pgfpathlineto{\pgfqpoint{1.290434in}{1.381884in}}%
\pgfpathlineto{\pgfqpoint{1.294730in}{1.381217in}}%
\pgfpathlineto{\pgfqpoint{1.295804in}{1.375062in}}%
\pgfpathlineto{\pgfqpoint{1.297951in}{1.371576in}}%
\pgfpathlineto{\pgfqpoint{1.301173in}{1.374729in}}%
\pgfpathlineto{\pgfqpoint{1.302247in}{1.377154in}}%
\pgfpathlineto{\pgfqpoint{1.303320in}{1.383044in}}%
\pgfpathlineto{\pgfqpoint{1.304394in}{1.383876in}}%
\pgfpathlineto{\pgfqpoint{1.305468in}{1.387039in}}%
\pgfpathlineto{\pgfqpoint{1.308690in}{1.386605in}}%
\pgfpathlineto{\pgfqpoint{1.311911in}{1.381604in}}%
\pgfpathlineto{\pgfqpoint{1.312985in}{1.381305in}}%
\pgfpathlineto{\pgfqpoint{1.316207in}{1.381484in}}%
\pgfpathlineto{\pgfqpoint{1.317280in}{1.384262in}}%
\pgfpathlineto{\pgfqpoint{1.320502in}{1.388768in}}%
\pgfpathlineto{\pgfqpoint{1.323723in}{1.389074in}}%
\pgfpathlineto{\pgfqpoint{1.324797in}{1.391004in}}%
\pgfpathlineto{\pgfqpoint{1.326945in}{1.390483in}}%
\pgfpathlineto{\pgfqpoint{1.328019in}{1.390819in}}%
\pgfpathlineto{\pgfqpoint{1.331240in}{1.393566in}}%
\pgfpathlineto{\pgfqpoint{1.332314in}{1.393100in}}%
\pgfpathlineto{\pgfqpoint{1.333388in}{1.397016in}}%
\pgfpathlineto{\pgfqpoint{1.335536in}{1.396643in}}%
\pgfpathlineto{\pgfqpoint{1.339831in}{1.398752in}}%
\pgfpathlineto{\pgfqpoint{1.340905in}{1.395682in}}%
\pgfpathlineto{\pgfqpoint{1.341979in}{1.399087in}}%
\pgfpathlineto{\pgfqpoint{1.343052in}{1.390820in}}%
\pgfpathlineto{\pgfqpoint{1.346274in}{1.396228in}}%
\pgfpathlineto{\pgfqpoint{1.347348in}{1.396010in}}%
\pgfpathlineto{\pgfqpoint{1.348422in}{1.394396in}}%
\pgfpathlineto{\pgfqpoint{1.350569in}{1.399983in}}%
\pgfpathlineto{\pgfqpoint{1.355939in}{1.383188in}}%
\pgfpathlineto{\pgfqpoint{1.357012in}{1.387426in}}%
\pgfpathlineto{\pgfqpoint{1.358086in}{1.383910in}}%
\pgfpathlineto{\pgfqpoint{1.361308in}{1.381580in}}%
\pgfpathlineto{\pgfqpoint{1.362382in}{1.377987in}}%
\pgfpathlineto{\pgfqpoint{1.365603in}{1.374132in}}%
\pgfpathlineto{\pgfqpoint{1.369898in}{1.374477in}}%
\pgfpathlineto{\pgfqpoint{1.370972in}{1.375253in}}%
\pgfpathlineto{\pgfqpoint{1.372046in}{1.372723in}}%
\pgfpathlineto{\pgfqpoint{1.376341in}{1.369150in}}%
\pgfpathlineto{\pgfqpoint{1.378489in}{1.371167in}}%
\pgfpathlineto{\pgfqpoint{1.380637in}{1.368949in}}%
\pgfpathlineto{\pgfqpoint{1.383858in}{1.369579in}}%
\pgfpathlineto{\pgfqpoint{1.384932in}{1.371589in}}%
\pgfpathlineto{\pgfqpoint{1.387080in}{1.372886in}}%
\pgfpathlineto{\pgfqpoint{1.388154in}{1.371560in}}%
\pgfpathlineto{\pgfqpoint{1.391375in}{1.372873in}}%
\pgfpathlineto{\pgfqpoint{1.392449in}{1.367709in}}%
\pgfpathlineto{\pgfqpoint{1.393523in}{1.368123in}}%
\pgfpathlineto{\pgfqpoint{1.394597in}{1.375193in}}%
\pgfpathlineto{\pgfqpoint{1.395671in}{1.374438in}}%
\pgfpathlineto{\pgfqpoint{1.398892in}{1.379056in}}%
\pgfpathlineto{\pgfqpoint{1.399966in}{1.378359in}}%
\pgfpathlineto{\pgfqpoint{1.401040in}{1.380064in}}%
\pgfpathlineto{\pgfqpoint{1.402114in}{1.379450in}}%
\pgfpathlineto{\pgfqpoint{1.403187in}{1.381497in}}%
\pgfpathlineto{\pgfqpoint{1.406409in}{1.383603in}}%
\pgfpathlineto{\pgfqpoint{1.407483in}{1.387400in}}%
\pgfpathlineto{\pgfqpoint{1.408557in}{1.389032in}}%
\pgfpathlineto{\pgfqpoint{1.409630in}{1.386992in}}%
\pgfpathlineto{\pgfqpoint{1.410704in}{1.389631in}}%
\pgfpathlineto{\pgfqpoint{1.415000in}{1.381514in}}%
\pgfpathlineto{\pgfqpoint{1.416073in}{1.376655in}}%
\pgfpathlineto{\pgfqpoint{1.417147in}{1.376902in}}%
\pgfpathlineto{\pgfqpoint{1.418221in}{1.373047in}}%
\pgfpathlineto{\pgfqpoint{1.422517in}{1.367236in}}%
\pgfpathlineto{\pgfqpoint{1.423590in}{1.370293in}}%
\pgfpathlineto{\pgfqpoint{1.424664in}{1.376296in}}%
\pgfpathlineto{\pgfqpoint{1.425738in}{1.374716in}}%
\pgfpathlineto{\pgfqpoint{1.428960in}{1.377273in}}%
\pgfpathlineto{\pgfqpoint{1.430033in}{1.375004in}}%
\pgfpathlineto{\pgfqpoint{1.431107in}{1.369460in}}%
\pgfpathlineto{\pgfqpoint{1.432181in}{1.368262in}}%
\pgfpathlineto{\pgfqpoint{1.436476in}{1.368910in}}%
\pgfpathlineto{\pgfqpoint{1.437550in}{1.371025in}}%
\pgfpathlineto{\pgfqpoint{1.438624in}{1.369913in}}%
\pgfpathlineto{\pgfqpoint{1.439698in}{1.371500in}}%
\pgfpathlineto{\pgfqpoint{1.440772in}{1.367256in}}%
\pgfpathlineto{\pgfqpoint{1.445067in}{1.368752in}}%
\pgfpathlineto{\pgfqpoint{1.446141in}{1.370607in}}%
\pgfpathlineto{\pgfqpoint{1.447215in}{1.374288in}}%
\pgfpathlineto{\pgfqpoint{1.448289in}{1.373616in}}%
\pgfpathlineto{\pgfqpoint{1.451510in}{1.374344in}}%
\pgfpathlineto{\pgfqpoint{1.453658in}{1.378572in}}%
\pgfpathlineto{\pgfqpoint{1.455806in}{1.378373in}}%
\pgfpathlineto{\pgfqpoint{1.459027in}{1.382453in}}%
\pgfpathlineto{\pgfqpoint{1.460101in}{1.382170in}}%
\pgfpathlineto{\pgfqpoint{1.461175in}{1.384263in}}%
\pgfpathlineto{\pgfqpoint{1.463322in}{1.377321in}}%
\pgfpathlineto{\pgfqpoint{1.466544in}{1.378440in}}%
\pgfpathlineto{\pgfqpoint{1.468692in}{1.387825in}}%
\pgfpathlineto{\pgfqpoint{1.469765in}{1.387291in}}%
\pgfpathlineto{\pgfqpoint{1.470839in}{1.385991in}}%
\pgfpathlineto{\pgfqpoint{1.476208in}{1.384937in}}%
\pgfpathlineto{\pgfqpoint{1.478356in}{1.386334in}}%
\pgfpathlineto{\pgfqpoint{1.481578in}{1.389366in}}%
\pgfpathlineto{\pgfqpoint{1.482651in}{1.388258in}}%
\pgfpathlineto{\pgfqpoint{1.484799in}{1.392689in}}%
\pgfpathlineto{\pgfqpoint{1.485873in}{1.390785in}}%
\pgfpathlineto{\pgfqpoint{1.489095in}{1.387748in}}%
\pgfpathlineto{\pgfqpoint{1.490168in}{1.390158in}}%
\pgfpathlineto{\pgfqpoint{1.491242in}{1.388055in}}%
\pgfpathlineto{\pgfqpoint{1.492316in}{1.384367in}}%
\pgfpathlineto{\pgfqpoint{1.493390in}{1.384223in}}%
\pgfpathlineto{\pgfqpoint{1.498759in}{1.384223in}}%
\pgfpathlineto{\pgfqpoint{1.500907in}{1.382278in}}%
\pgfpathlineto{\pgfqpoint{1.504128in}{1.384431in}}%
\pgfpathlineto{\pgfqpoint{1.505202in}{1.380752in}}%
\pgfpathlineto{\pgfqpoint{1.507350in}{1.382764in}}%
\pgfpathlineto{\pgfqpoint{1.508424in}{1.384875in}}%
\pgfpathlineto{\pgfqpoint{1.512719in}{1.389036in}}%
\pgfpathlineto{\pgfqpoint{1.514867in}{1.392117in}}%
\pgfpathlineto{\pgfqpoint{1.519162in}{1.393253in}}%
\pgfpathlineto{\pgfqpoint{1.520236in}{1.390512in}}%
\pgfpathlineto{\pgfqpoint{1.522384in}{1.389036in}}%
\pgfpathlineto{\pgfqpoint{1.523457in}{1.390693in}}%
\pgfpathlineto{\pgfqpoint{1.526679in}{1.388705in}}%
\pgfpathlineto{\pgfqpoint{1.527753in}{1.388824in}}%
\pgfpathlineto{\pgfqpoint{1.528827in}{1.387574in}}%
\pgfpathlineto{\pgfqpoint{1.529900in}{1.383379in}}%
\pgfpathlineto{\pgfqpoint{1.530974in}{1.385491in}}%
\pgfpathlineto{\pgfqpoint{1.534196in}{1.387038in}}%
\pgfpathlineto{\pgfqpoint{1.536343in}{1.378410in}}%
\pgfpathlineto{\pgfqpoint{1.537417in}{1.376336in}}%
\pgfpathlineto{\pgfqpoint{1.538491in}{1.376155in}}%
\pgfpathlineto{\pgfqpoint{1.541713in}{1.376456in}}%
\pgfpathlineto{\pgfqpoint{1.542786in}{1.380875in}}%
\pgfpathlineto{\pgfqpoint{1.546008in}{1.373808in}}%
\pgfpathlineto{\pgfqpoint{1.550303in}{1.374738in}}%
\pgfpathlineto{\pgfqpoint{1.551377in}{1.371357in}}%
\pgfpathlineto{\pgfqpoint{1.552451in}{1.372536in}}%
\pgfpathlineto{\pgfqpoint{1.553525in}{1.367760in}}%
\pgfpathlineto{\pgfqpoint{1.556746in}{1.368389in}}%
\pgfpathlineto{\pgfqpoint{1.557820in}{1.366220in}}%
\pgfpathlineto{\pgfqpoint{1.558894in}{1.369534in}}%
\pgfpathlineto{\pgfqpoint{1.559968in}{1.368962in}}%
\pgfpathlineto{\pgfqpoint{1.561042in}{1.369322in}}%
\pgfpathlineto{\pgfqpoint{1.564263in}{1.375881in}}%
\pgfpathlineto{\pgfqpoint{1.565337in}{1.374617in}}%
\pgfpathlineto{\pgfqpoint{1.566411in}{1.371575in}}%
\pgfpathlineto{\pgfqpoint{1.567485in}{1.370670in}}%
\pgfpathlineto{\pgfqpoint{1.568559in}{1.372382in}}%
\pgfpathlineto{\pgfqpoint{1.572854in}{1.373879in}}%
\pgfpathlineto{\pgfqpoint{1.573928in}{1.372348in}}%
\pgfpathlineto{\pgfqpoint{1.575002in}{1.372407in}}%
\pgfpathlineto{\pgfqpoint{1.576075in}{1.369433in}}%
\pgfpathlineto{\pgfqpoint{1.580371in}{1.372201in}}%
\pgfpathlineto{\pgfqpoint{1.586814in}{1.370116in}}%
\pgfpathlineto{\pgfqpoint{1.590035in}{1.370762in}}%
\pgfpathlineto{\pgfqpoint{1.591109in}{1.370349in}}%
\pgfpathlineto{\pgfqpoint{1.594331in}{1.370291in}}%
\pgfpathlineto{\pgfqpoint{1.595405in}{1.370850in}}%
\pgfpathlineto{\pgfqpoint{1.597552in}{1.373452in}}%
\pgfpathlineto{\pgfqpoint{1.598626in}{1.373631in}}%
\pgfpathlineto{\pgfqpoint{1.602921in}{1.365627in}}%
\pgfpathlineto{\pgfqpoint{1.603995in}{1.367353in}}%
\pgfpathlineto{\pgfqpoint{1.606143in}{1.373602in}}%
\pgfpathlineto{\pgfqpoint{1.609364in}{1.374031in}}%
\pgfpathlineto{\pgfqpoint{1.610438in}{1.375445in}}%
\pgfpathlineto{\pgfqpoint{1.611512in}{1.380242in}}%
\pgfpathlineto{\pgfqpoint{1.612586in}{1.378526in}}%
\pgfpathlineto{\pgfqpoint{1.613660in}{1.381863in}}%
\pgfpathlineto{\pgfqpoint{1.616881in}{1.382435in}}%
\pgfpathlineto{\pgfqpoint{1.617955in}{1.374521in}}%
\pgfpathlineto{\pgfqpoint{1.619029in}{1.377936in}}%
\pgfpathlineto{\pgfqpoint{1.620103in}{1.385563in}}%
\pgfpathlineto{\pgfqpoint{1.621177in}{1.385563in}}%
\pgfpathlineto{\pgfqpoint{1.624398in}{1.383679in}}%
\pgfpathlineto{\pgfqpoint{1.626546in}{1.388926in}}%
\pgfpathlineto{\pgfqpoint{1.627620in}{1.389097in}}%
\pgfpathlineto{\pgfqpoint{1.628694in}{1.382106in}}%
\pgfpathlineto{\pgfqpoint{1.631915in}{1.382237in}}%
\pgfpathlineto{\pgfqpoint{1.634063in}{1.386003in}}%
\pgfpathlineto{\pgfqpoint{1.635137in}{1.391924in}}%
\pgfpathlineto{\pgfqpoint{1.636210in}{1.390028in}}%
\pgfpathlineto{\pgfqpoint{1.639432in}{1.388944in}}%
\pgfpathlineto{\pgfqpoint{1.640506in}{1.382942in}}%
\pgfpathlineto{\pgfqpoint{1.642653in}{1.385492in}}%
\pgfpathlineto{\pgfqpoint{1.643727in}{1.383659in}}%
\pgfpathlineto{\pgfqpoint{1.646949in}{1.385594in}}%
\pgfpathlineto{\pgfqpoint{1.648023in}{1.387238in}}%
\pgfpathlineto{\pgfqpoint{1.650170in}{1.382499in}}%
\pgfpathlineto{\pgfqpoint{1.651244in}{1.381151in}}%
\pgfpathlineto{\pgfqpoint{1.654466in}{1.381364in}}%
\pgfpathlineto{\pgfqpoint{1.655539in}{1.379358in}}%
\pgfpathlineto{\pgfqpoint{1.656613in}{1.380634in}}%
\pgfpathlineto{\pgfqpoint{1.657687in}{1.380634in}}%
\pgfpathlineto{\pgfqpoint{1.658761in}{1.378780in}}%
\pgfpathlineto{\pgfqpoint{1.661983in}{1.378537in}}%
\pgfpathlineto{\pgfqpoint{1.663056in}{1.382428in}}%
\pgfpathlineto{\pgfqpoint{1.665204in}{1.383219in}}%
\pgfpathlineto{\pgfqpoint{1.666278in}{1.379197in}}%
\pgfpathlineto{\pgfqpoint{1.669499in}{1.383201in}}%
\pgfpathlineto{\pgfqpoint{1.671647in}{1.395363in}}%
\pgfpathlineto{\pgfqpoint{1.673795in}{1.395112in}}%
\pgfpathlineto{\pgfqpoint{1.678090in}{1.394015in}}%
\pgfpathlineto{\pgfqpoint{1.679164in}{1.397682in}}%
\pgfpathlineto{\pgfqpoint{1.681312in}{1.396837in}}%
\pgfpathlineto{\pgfqpoint{1.684533in}{1.391944in}}%
\pgfpathlineto{\pgfqpoint{1.685607in}{1.386719in}}%
\pgfpathlineto{\pgfqpoint{1.686681in}{1.390600in}}%
\pgfpathlineto{\pgfqpoint{1.687755in}{1.392038in}}%
\pgfpathlineto{\pgfqpoint{1.688828in}{1.397099in}}%
\pgfpathlineto{\pgfqpoint{1.692050in}{1.396449in}}%
\pgfpathlineto{\pgfqpoint{1.693124in}{1.400101in}}%
\pgfpathlineto{\pgfqpoint{1.694198in}{1.398244in}}%
\pgfpathlineto{\pgfqpoint{1.695272in}{1.389893in}}%
\pgfpathlineto{\pgfqpoint{1.696345in}{1.393430in}}%
\pgfpathlineto{\pgfqpoint{1.699567in}{1.397525in}}%
\pgfpathlineto{\pgfqpoint{1.700641in}{1.397347in}}%
\pgfpathlineto{\pgfqpoint{1.701715in}{1.396547in}}%
\pgfpathlineto{\pgfqpoint{1.707084in}{1.398444in}}%
\pgfpathlineto{\pgfqpoint{1.708158in}{1.396684in}}%
\pgfpathlineto{\pgfqpoint{1.709231in}{1.393491in}}%
\pgfpathlineto{\pgfqpoint{1.711379in}{1.393581in}}%
\pgfpathlineto{\pgfqpoint{1.714601in}{1.388478in}}%
\pgfpathlineto{\pgfqpoint{1.715674in}{1.384212in}}%
\pgfpathlineto{\pgfqpoint{1.716748in}{1.387434in}}%
\pgfpathlineto{\pgfqpoint{1.717822in}{1.382302in}}%
\pgfpathlineto{\pgfqpoint{1.718896in}{1.383946in}}%
\pgfpathlineto{\pgfqpoint{1.723191in}{1.385374in}}%
\pgfpathlineto{\pgfqpoint{1.724265in}{1.383242in}}%
\pgfpathlineto{\pgfqpoint{1.725339in}{1.383242in}}%
\pgfpathlineto{\pgfqpoint{1.726413in}{1.390107in}}%
\pgfpathlineto{\pgfqpoint{1.730708in}{1.386514in}}%
\pgfpathlineto{\pgfqpoint{1.732856in}{1.379159in}}%
\pgfpathlineto{\pgfqpoint{1.733930in}{1.380299in}}%
\pgfpathlineto{\pgfqpoint{1.737151in}{1.377289in}}%
\pgfpathlineto{\pgfqpoint{1.738225in}{1.378274in}}%
\pgfpathlineto{\pgfqpoint{1.739299in}{1.383748in}}%
\pgfpathlineto{\pgfqpoint{1.740373in}{1.382554in}}%
\pgfpathlineto{\pgfqpoint{1.741447in}{1.378320in}}%
\pgfpathlineto{\pgfqpoint{1.744668in}{1.382895in}}%
\pgfpathlineto{\pgfqpoint{1.745742in}{1.377041in}}%
\pgfpathlineto{\pgfqpoint{1.746816in}{1.379718in}}%
\pgfpathlineto{\pgfqpoint{1.747890in}{1.385477in}}%
\pgfpathlineto{\pgfqpoint{1.748963in}{1.386228in}}%
\pgfpathlineto{\pgfqpoint{1.752185in}{1.385026in}}%
\pgfpathlineto{\pgfqpoint{1.754333in}{1.385054in}}%
\pgfpathlineto{\pgfqpoint{1.755406in}{1.387291in}}%
\pgfpathlineto{\pgfqpoint{1.756480in}{1.383125in}}%
\pgfpathlineto{\pgfqpoint{1.760776in}{1.382934in}}%
\pgfpathlineto{\pgfqpoint{1.761850in}{1.384564in}}%
\pgfpathlineto{\pgfqpoint{1.763997in}{1.379447in}}%
\pgfpathlineto{\pgfqpoint{1.768293in}{1.380550in}}%
\pgfpathlineto{\pgfqpoint{1.770440in}{1.377883in}}%
\pgfpathlineto{\pgfqpoint{1.771514in}{1.376088in}}%
\pgfpathlineto{\pgfqpoint{1.774736in}{1.379124in}}%
\pgfpathlineto{\pgfqpoint{1.775809in}{1.382662in}}%
\pgfpathlineto{\pgfqpoint{1.779031in}{1.377144in}}%
\pgfpathlineto{\pgfqpoint{1.782252in}{1.382771in}}%
\pgfpathlineto{\pgfqpoint{1.783326in}{1.393321in}}%
\pgfpathlineto{\pgfqpoint{1.784400in}{1.395762in}}%
\pgfpathlineto{\pgfqpoint{1.785474in}{1.388178in}}%
\pgfpathlineto{\pgfqpoint{1.786548in}{1.394402in}}%
\pgfpathlineto{\pgfqpoint{1.794065in}{1.402115in}}%
\pgfpathlineto{\pgfqpoint{1.797286in}{1.400059in}}%
\pgfpathlineto{\pgfqpoint{1.798360in}{1.400676in}}%
\pgfpathlineto{\pgfqpoint{1.799434in}{1.407167in}}%
\pgfpathlineto{\pgfqpoint{1.800508in}{1.406034in}}%
\pgfpathlineto{\pgfqpoint{1.801582in}{1.408392in}}%
\pgfpathlineto{\pgfqpoint{1.804803in}{1.403095in}}%
\pgfpathlineto{\pgfqpoint{1.805877in}{1.407420in}}%
\pgfpathlineto{\pgfqpoint{1.806951in}{1.403919in}}%
\pgfpathlineto{\pgfqpoint{1.808025in}{1.407207in}}%
\pgfpathlineto{\pgfqpoint{1.812320in}{1.405198in}}%
\pgfpathlineto{\pgfqpoint{1.814468in}{1.407128in}}%
\pgfpathlineto{\pgfqpoint{1.815541in}{1.407007in}}%
\pgfpathlineto{\pgfqpoint{1.816615in}{1.408579in}}%
\pgfpathlineto{\pgfqpoint{1.820911in}{1.412189in}}%
\pgfpathlineto{\pgfqpoint{1.821984in}{1.411697in}}%
\pgfpathlineto{\pgfqpoint{1.823058in}{1.412771in}}%
\pgfpathlineto{\pgfqpoint{1.824132in}{1.406193in}}%
\pgfpathlineto{\pgfqpoint{1.827354in}{1.410084in}}%
\pgfpathlineto{\pgfqpoint{1.828427in}{1.408880in}}%
\pgfpathlineto{\pgfqpoint{1.829501in}{1.408849in}}%
\pgfpathlineto{\pgfqpoint{1.830575in}{1.409677in}}%
\pgfpathlineto{\pgfqpoint{1.831649in}{1.412710in}}%
\pgfpathlineto{\pgfqpoint{1.834871in}{1.411555in}}%
\pgfpathlineto{\pgfqpoint{1.835944in}{1.412492in}}%
\pgfpathlineto{\pgfqpoint{1.837018in}{1.414459in}}%
\pgfpathlineto{\pgfqpoint{1.838092in}{1.409845in}}%
\pgfpathlineto{\pgfqpoint{1.839166in}{1.414649in}}%
\pgfpathlineto{\pgfqpoint{1.842387in}{1.411804in}}%
\pgfpathlineto{\pgfqpoint{1.843461in}{1.415625in}}%
\pgfpathlineto{\pgfqpoint{1.844535in}{1.415625in}}%
\pgfpathlineto{\pgfqpoint{1.845609in}{1.418391in}}%
\pgfpathlineto{\pgfqpoint{1.846683in}{1.411586in}}%
\pgfpathlineto{\pgfqpoint{1.849904in}{1.413570in}}%
\pgfpathlineto{\pgfqpoint{1.850978in}{1.412612in}}%
\pgfpathlineto{\pgfqpoint{1.852052in}{1.414435in}}%
\pgfpathlineto{\pgfqpoint{1.853126in}{1.409492in}}%
\pgfpathlineto{\pgfqpoint{1.854200in}{1.411260in}}%
\pgfpathlineto{\pgfqpoint{1.857421in}{1.411321in}}%
\pgfpathlineto{\pgfqpoint{1.858495in}{1.412836in}}%
\pgfpathlineto{\pgfqpoint{1.860643in}{1.414081in}}%
\pgfpathlineto{\pgfqpoint{1.861716in}{1.415449in}}%
\pgfpathlineto{\pgfqpoint{1.866012in}{1.410604in}}%
\pgfpathlineto{\pgfqpoint{1.867086in}{1.412720in}}%
\pgfpathlineto{\pgfqpoint{1.868160in}{1.412567in}}%
\pgfpathlineto{\pgfqpoint{1.869233in}{1.411127in}}%
\pgfpathlineto{\pgfqpoint{1.872455in}{1.410637in}}%
\pgfpathlineto{\pgfqpoint{1.873529in}{1.411587in}}%
\pgfpathlineto{\pgfqpoint{1.874603in}{1.408339in}}%
\pgfpathlineto{\pgfqpoint{1.875676in}{1.412063in}}%
\pgfpathlineto{\pgfqpoint{1.876750in}{1.411663in}}%
\pgfpathlineto{\pgfqpoint{1.879972in}{1.409389in}}%
\pgfpathlineto{\pgfqpoint{1.881046in}{1.409727in}}%
\pgfpathlineto{\pgfqpoint{1.883193in}{1.404150in}}%
\pgfpathlineto{\pgfqpoint{1.884267in}{1.406079in}}%
\pgfpathlineto{\pgfqpoint{1.887489in}{1.397966in}}%
\pgfpathlineto{\pgfqpoint{1.888562in}{1.398838in}}%
\pgfpathlineto{\pgfqpoint{1.889636in}{1.398477in}}%
\pgfpathlineto{\pgfqpoint{1.890710in}{1.396229in}}%
\pgfpathlineto{\pgfqpoint{1.891784in}{1.398918in}}%
\pgfpathlineto{\pgfqpoint{1.896079in}{1.400090in}}%
\pgfpathlineto{\pgfqpoint{1.898227in}{1.394825in}}%
\pgfpathlineto{\pgfqpoint{1.899301in}{1.395426in}}%
\pgfpathlineto{\pgfqpoint{1.902522in}{1.400813in}}%
\pgfpathlineto{\pgfqpoint{1.903596in}{1.399099in}}%
\pgfpathlineto{\pgfqpoint{1.904670in}{1.398787in}}%
\pgfpathlineto{\pgfqpoint{1.905744in}{1.395266in}}%
\pgfpathlineto{\pgfqpoint{1.910039in}{1.394705in}}%
\pgfpathlineto{\pgfqpoint{1.911113in}{1.396762in}}%
\pgfpathlineto{\pgfqpoint{1.912187in}{1.400781in}}%
\pgfpathlineto{\pgfqpoint{1.913261in}{1.401517in}}%
\pgfpathlineto{\pgfqpoint{1.914335in}{1.397871in}}%
\pgfpathlineto{\pgfqpoint{1.917556in}{1.393934in}}%
\pgfpathlineto{\pgfqpoint{1.918630in}{1.394117in}}%
\pgfpathlineto{\pgfqpoint{1.919704in}{1.393385in}}%
\pgfpathlineto{\pgfqpoint{1.920778in}{1.393416in}}%
\pgfpathlineto{\pgfqpoint{1.921851in}{1.391982in}}%
\pgfpathlineto{\pgfqpoint{1.925073in}{1.391281in}}%
\pgfpathlineto{\pgfqpoint{1.926147in}{1.369869in}}%
\pgfpathlineto{\pgfqpoint{1.927221in}{1.366697in}}%
\pgfpathlineto{\pgfqpoint{1.928294in}{1.365538in}}%
\pgfpathlineto{\pgfqpoint{1.929368in}{1.360505in}}%
\pgfpathlineto{\pgfqpoint{1.932590in}{1.359316in}}%
\pgfpathlineto{\pgfqpoint{1.933664in}{1.359590in}}%
\pgfpathlineto{\pgfqpoint{1.934738in}{1.358523in}}%
\pgfpathlineto{\pgfqpoint{1.935811in}{1.354800in}}%
\pgfpathlineto{\pgfqpoint{1.936885in}{1.355890in}}%
\pgfpathlineto{\pgfqpoint{1.942254in}{1.352031in}}%
\pgfpathlineto{\pgfqpoint{1.943328in}{1.352298in}}%
\pgfpathlineto{\pgfqpoint{1.944402in}{1.354138in}}%
\pgfpathlineto{\pgfqpoint{1.947624in}{1.357570in}}%
\pgfpathlineto{\pgfqpoint{1.949771in}{1.362122in}}%
\pgfpathlineto{\pgfqpoint{1.951919in}{1.363876in}}%
\pgfpathlineto{\pgfqpoint{1.956214in}{1.361872in}}%
\pgfpathlineto{\pgfqpoint{1.957288in}{1.365188in}}%
\pgfpathlineto{\pgfqpoint{1.959436in}{1.350446in}}%
\pgfpathlineto{\pgfqpoint{1.962657in}{1.344400in}}%
\pgfpathlineto{\pgfqpoint{1.963731in}{1.338541in}}%
\pgfpathlineto{\pgfqpoint{1.964805in}{1.346364in}}%
\pgfpathlineto{\pgfqpoint{1.965879in}{1.341533in}}%
\pgfpathlineto{\pgfqpoint{1.966953in}{1.341623in}}%
\pgfpathlineto{\pgfqpoint{1.970174in}{1.337176in}}%
\pgfpathlineto{\pgfqpoint{1.971248in}{1.332008in}}%
\pgfpathlineto{\pgfqpoint{1.972322in}{1.336155in}}%
\pgfpathlineto{\pgfqpoint{1.973396in}{1.334322in}}%
\pgfpathlineto{\pgfqpoint{1.974470in}{1.337608in}}%
\pgfpathlineto{\pgfqpoint{1.978765in}{1.343507in}}%
\pgfpathlineto{\pgfqpoint{1.979839in}{1.346183in}}%
\pgfpathlineto{\pgfqpoint{1.980913in}{1.344910in}}%
\pgfpathlineto{\pgfqpoint{1.981986in}{1.347549in}}%
\pgfpathlineto{\pgfqpoint{1.985208in}{1.348977in}}%
\pgfpathlineto{\pgfqpoint{1.986282in}{1.351394in}}%
\pgfpathlineto{\pgfqpoint{1.987356in}{1.347784in}}%
\pgfpathlineto{\pgfqpoint{1.988429in}{1.349525in}}%
\pgfpathlineto{\pgfqpoint{1.989503in}{1.343210in}}%
\pgfpathlineto{\pgfqpoint{1.992725in}{1.344572in}}%
\pgfpathlineto{\pgfqpoint{1.993799in}{1.354013in}}%
\pgfpathlineto{\pgfqpoint{1.994872in}{1.350283in}}%
\pgfpathlineto{\pgfqpoint{1.995946in}{1.349850in}}%
\pgfpathlineto{\pgfqpoint{1.997020in}{1.351282in}}%
\pgfpathlineto{\pgfqpoint{2.000242in}{1.352681in}}%
\pgfpathlineto{\pgfqpoint{2.001315in}{1.356182in}}%
\pgfpathlineto{\pgfqpoint{2.002389in}{1.353085in}}%
\pgfpathlineto{\pgfqpoint{2.003463in}{1.354960in}}%
\pgfpathlineto{\pgfqpoint{2.004537in}{1.359231in}}%
\pgfpathlineto{\pgfqpoint{2.007759in}{1.351690in}}%
\pgfpathlineto{\pgfqpoint{2.008832in}{1.350838in}}%
\pgfpathlineto{\pgfqpoint{2.012054in}{1.342968in}}%
\pgfpathlineto{\pgfqpoint{2.015275in}{1.342820in}}%
\pgfpathlineto{\pgfqpoint{2.016349in}{1.345888in}}%
\pgfpathlineto{\pgfqpoint{2.017423in}{1.340087in}}%
\pgfpathlineto{\pgfqpoint{2.018497in}{1.342867in}}%
\pgfpathlineto{\pgfqpoint{2.019571in}{1.343380in}}%
\pgfpathlineto{\pgfqpoint{2.022792in}{1.340771in}}%
\pgfpathlineto{\pgfqpoint{2.023866in}{1.350630in}}%
\pgfpathlineto{\pgfqpoint{2.026014in}{1.339403in}}%
\pgfpathlineto{\pgfqpoint{2.027088in}{1.337288in}}%
\pgfpathlineto{\pgfqpoint{2.030309in}{1.338273in}}%
\pgfpathlineto{\pgfqpoint{2.031383in}{1.334908in}}%
\pgfpathlineto{\pgfqpoint{2.033531in}{1.336701in}}%
\pgfpathlineto{\pgfqpoint{2.034604in}{1.335092in}}%
\pgfpathlineto{\pgfqpoint{2.037826in}{1.337561in}}%
\pgfpathlineto{\pgfqpoint{2.039974in}{1.335186in}}%
\pgfpathlineto{\pgfqpoint{2.041048in}{1.334043in}}%
\pgfpathlineto{\pgfqpoint{2.042121in}{1.334043in}}%
\pgfpathlineto{\pgfqpoint{2.047491in}{1.329378in}}%
\pgfpathlineto{\pgfqpoint{2.048564in}{1.331751in}}%
\pgfpathlineto{\pgfqpoint{2.049638in}{1.338519in}}%
\pgfpathlineto{\pgfqpoint{2.053934in}{1.342794in}}%
\pgfpathlineto{\pgfqpoint{2.055007in}{1.343224in}}%
\pgfpathlineto{\pgfqpoint{2.056081in}{1.341676in}}%
\pgfpathlineto{\pgfqpoint{2.057155in}{1.341903in}}%
\pgfpathlineto{\pgfqpoint{2.060377in}{1.341591in}}%
\pgfpathlineto{\pgfqpoint{2.061450in}{1.339690in}}%
\pgfpathlineto{\pgfqpoint{2.062524in}{1.339463in}}%
\pgfpathlineto{\pgfqpoint{2.067893in}{1.336285in}}%
\pgfpathlineto{\pgfqpoint{2.068967in}{1.337959in}}%
\pgfpathlineto{\pgfqpoint{2.070041in}{1.341704in}}%
\pgfpathlineto{\pgfqpoint{2.071115in}{1.340242in}}%
\pgfpathlineto{\pgfqpoint{2.072189in}{1.342990in}}%
\pgfpathlineto{\pgfqpoint{2.075410in}{1.342523in}}%
\pgfpathlineto{\pgfqpoint{2.076484in}{1.346979in}}%
\pgfpathlineto{\pgfqpoint{2.077558in}{1.346768in}}%
\pgfpathlineto{\pgfqpoint{2.078632in}{1.345983in}}%
\pgfpathlineto{\pgfqpoint{2.079706in}{1.344385in}}%
\pgfpathlineto{\pgfqpoint{2.082927in}{1.343661in}}%
\pgfpathlineto{\pgfqpoint{2.084001in}{1.344295in}}%
\pgfpathlineto{\pgfqpoint{2.085075in}{1.339137in}}%
\pgfpathlineto{\pgfqpoint{2.086149in}{1.342469in}}%
\pgfpathlineto{\pgfqpoint{2.087223in}{1.338304in}}%
\pgfpathlineto{\pgfqpoint{2.090444in}{1.341755in}}%
\pgfpathlineto{\pgfqpoint{2.092592in}{1.334540in}}%
\pgfpathlineto{\pgfqpoint{2.097961in}{1.334006in}}%
\pgfpathlineto{\pgfqpoint{2.099035in}{1.337041in}}%
\pgfpathlineto{\pgfqpoint{2.100109in}{1.337996in}}%
\pgfpathlineto{\pgfqpoint{2.101182in}{1.335844in}}%
\pgfpathlineto{\pgfqpoint{2.105478in}{1.334542in}}%
\pgfpathlineto{\pgfqpoint{2.106552in}{1.334938in}}%
\pgfpathlineto{\pgfqpoint{2.107626in}{1.341677in}}%
\pgfpathlineto{\pgfqpoint{2.109773in}{1.334233in}}%
\pgfpathlineto{\pgfqpoint{2.114069in}{1.334502in}}%
\pgfpathlineto{\pgfqpoint{2.115142in}{1.329957in}}%
\pgfpathlineto{\pgfqpoint{2.116216in}{1.330436in}}%
\pgfpathlineto{\pgfqpoint{2.117290in}{1.339674in}}%
\pgfpathlineto{\pgfqpoint{2.121585in}{1.338513in}}%
\pgfpathlineto{\pgfqpoint{2.122659in}{1.337319in}}%
\pgfpathlineto{\pgfqpoint{2.123733in}{1.339481in}}%
\pgfpathlineto{\pgfqpoint{2.124807in}{1.337158in}}%
\pgfpathlineto{\pgfqpoint{2.129102in}{1.343630in}}%
\pgfpathlineto{\pgfqpoint{2.130176in}{1.343172in}}%
\pgfpathlineto{\pgfqpoint{2.131250in}{1.341246in}}%
\pgfpathlineto{\pgfqpoint{2.132324in}{1.337613in}}%
\pgfpathlineto{\pgfqpoint{2.135545in}{1.337988in}}%
\pgfpathlineto{\pgfqpoint{2.136619in}{1.331877in}}%
\pgfpathlineto{\pgfqpoint{2.137693in}{1.333413in}}%
\pgfpathlineto{\pgfqpoint{2.138767in}{1.327208in}}%
\pgfpathlineto{\pgfqpoint{2.139841in}{1.327983in}}%
\pgfpathlineto{\pgfqpoint{2.143062in}{1.325073in}}%
\pgfpathlineto{\pgfqpoint{2.145210in}{1.327083in}}%
\pgfpathlineto{\pgfqpoint{2.146284in}{1.320715in}}%
\pgfpathlineto{\pgfqpoint{2.147358in}{1.324276in}}%
\pgfpathlineto{\pgfqpoint{2.151653in}{1.322797in}}%
\pgfpathlineto{\pgfqpoint{2.152727in}{1.316538in}}%
\pgfpathlineto{\pgfqpoint{2.153801in}{1.315916in}}%
\pgfpathlineto{\pgfqpoint{2.154874in}{1.316254in}}%
\pgfpathlineto{\pgfqpoint{2.160244in}{1.334330in}}%
\pgfpathlineto{\pgfqpoint{2.161317in}{1.322547in}}%
\pgfpathlineto{\pgfqpoint{2.162391in}{1.323459in}}%
\pgfpathlineto{\pgfqpoint{2.165613in}{1.320859in}}%
\pgfpathlineto{\pgfqpoint{2.166687in}{1.317051in}}%
\pgfpathlineto{\pgfqpoint{2.167760in}{1.318233in}}%
\pgfpathlineto{\pgfqpoint{2.169908in}{1.314713in}}%
\pgfpathlineto{\pgfqpoint{2.174203in}{1.314840in}}%
\pgfpathlineto{\pgfqpoint{2.176351in}{1.313131in}}%
\pgfpathlineto{\pgfqpoint{2.177425in}{1.314483in}}%
\pgfpathlineto{\pgfqpoint{2.180647in}{1.314304in}}%
\pgfpathlineto{\pgfqpoint{2.181720in}{1.315553in}}%
\pgfpathlineto{\pgfqpoint{2.182794in}{1.318386in}}%
\pgfpathlineto{\pgfqpoint{2.183868in}{1.315166in}}%
\pgfpathlineto{\pgfqpoint{2.184942in}{1.314690in}}%
\pgfpathlineto{\pgfqpoint{2.188163in}{1.314416in}}%
\pgfpathlineto{\pgfqpoint{2.190311in}{1.316616in}}%
\pgfpathlineto{\pgfqpoint{2.191385in}{1.316717in}}%
\pgfpathlineto{\pgfqpoint{2.195680in}{1.316365in}}%
\pgfpathlineto{\pgfqpoint{2.196754in}{1.319183in}}%
\pgfpathlineto{\pgfqpoint{2.197828in}{1.318881in}}%
\pgfpathlineto{\pgfqpoint{2.198902in}{1.319232in}}%
\pgfpathlineto{\pgfqpoint{2.199976in}{1.318930in}}%
\pgfpathlineto{\pgfqpoint{2.203197in}{1.319610in}}%
\pgfpathlineto{\pgfqpoint{2.205345in}{1.321734in}}%
\pgfpathlineto{\pgfqpoint{2.207492in}{1.319536in}}%
\pgfpathlineto{\pgfqpoint{2.210714in}{1.317644in}}%
\pgfpathlineto{\pgfqpoint{2.211788in}{1.315537in}}%
\pgfpathlineto{\pgfqpoint{2.212862in}{1.312144in}}%
\pgfpathlineto{\pgfqpoint{2.215009in}{1.312214in}}%
\pgfpathlineto{\pgfqpoint{2.219305in}{1.313686in}}%
\pgfpathlineto{\pgfqpoint{2.220379in}{1.314834in}}%
\pgfpathlineto{\pgfqpoint{2.222526in}{1.315632in}}%
\pgfpathlineto{\pgfqpoint{2.225748in}{1.316711in}}%
\pgfpathlineto{\pgfqpoint{2.226822in}{1.315835in}}%
\pgfpathlineto{\pgfqpoint{2.230043in}{1.322147in}}%
\pgfpathlineto{\pgfqpoint{2.233265in}{1.322147in}}%
\pgfpathlineto{\pgfqpoint{2.235412in}{1.313810in}}%
\pgfpathlineto{\pgfqpoint{2.236486in}{1.312955in}}%
\pgfpathlineto{\pgfqpoint{2.237560in}{1.314837in}}%
\pgfpathlineto{\pgfqpoint{2.240781in}{1.317160in}}%
\pgfpathlineto{\pgfqpoint{2.241855in}{1.321925in}}%
\pgfpathlineto{\pgfqpoint{2.242929in}{1.323223in}}%
\pgfpathlineto{\pgfqpoint{2.244003in}{1.322895in}}%
\pgfpathlineto{\pgfqpoint{2.245077in}{1.320120in}}%
\pgfpathlineto{\pgfqpoint{2.248298in}{1.323727in}}%
\pgfpathlineto{\pgfqpoint{2.249372in}{1.327814in}}%
\pgfpathlineto{\pgfqpoint{2.250446in}{1.327553in}}%
\pgfpathlineto{\pgfqpoint{2.251520in}{1.325647in}}%
\pgfpathlineto{\pgfqpoint{2.252594in}{1.327083in}}%
\pgfpathlineto{\pgfqpoint{2.255815in}{1.327553in}}%
\pgfpathlineto{\pgfqpoint{2.256889in}{1.329991in}}%
\pgfpathlineto{\pgfqpoint{2.257963in}{1.328995in}}%
\pgfpathlineto{\pgfqpoint{2.259037in}{1.328761in}}%
\pgfpathlineto{\pgfqpoint{2.260111in}{1.327931in}}%
\pgfpathlineto{\pgfqpoint{2.264406in}{1.328369in}}%
\pgfpathlineto{\pgfqpoint{2.265480in}{1.327826in}}%
\pgfpathlineto{\pgfqpoint{2.267627in}{1.329740in}}%
\pgfpathlineto{\pgfqpoint{2.270849in}{1.331980in}}%
\pgfpathlineto{\pgfqpoint{2.271923in}{1.331772in}}%
\pgfpathlineto{\pgfqpoint{2.274070in}{1.329673in}}%
\pgfpathlineto{\pgfqpoint{2.275144in}{1.330440in}}%
\pgfpathlineto{\pgfqpoint{2.280514in}{1.326966in}}%
\pgfpathlineto{\pgfqpoint{2.281587in}{1.328356in}}%
\pgfpathlineto{\pgfqpoint{2.282661in}{1.328176in}}%
\pgfpathlineto{\pgfqpoint{2.285883in}{1.326994in}}%
\pgfpathlineto{\pgfqpoint{2.286957in}{1.327299in}}%
\pgfpathlineto{\pgfqpoint{2.288030in}{1.326789in}}%
\pgfpathlineto{\pgfqpoint{2.289104in}{1.329110in}}%
\pgfpathlineto{\pgfqpoint{2.290178in}{1.337245in}}%
\pgfpathlineto{\pgfqpoint{2.293400in}{1.332990in}}%
\pgfpathlineto{\pgfqpoint{2.294473in}{1.337600in}}%
\pgfpathlineto{\pgfqpoint{2.297695in}{1.328426in}}%
\pgfpathlineto{\pgfqpoint{2.303064in}{1.335265in}}%
\pgfpathlineto{\pgfqpoint{2.304138in}{1.333840in}}%
\pgfpathlineto{\pgfqpoint{2.305212in}{1.329743in}}%
\pgfpathlineto{\pgfqpoint{2.308433in}{1.329036in}}%
\pgfpathlineto{\pgfqpoint{2.310581in}{1.326415in}}%
\pgfpathlineto{\pgfqpoint{2.311655in}{1.326317in}}%
\pgfpathlineto{\pgfqpoint{2.312729in}{1.327151in}}%
\pgfpathlineto{\pgfqpoint{2.317024in}{1.327470in}}%
\pgfpathlineto{\pgfqpoint{2.318098in}{1.326513in}}%
\pgfpathlineto{\pgfqpoint{2.319172in}{1.326951in}}%
\pgfpathlineto{\pgfqpoint{2.320246in}{1.325314in}}%
\pgfpathlineto{\pgfqpoint{2.323467in}{1.324239in}}%
\pgfpathlineto{\pgfqpoint{2.324541in}{1.331570in}}%
\pgfpathlineto{\pgfqpoint{2.325615in}{1.332010in}}%
\pgfpathlineto{\pgfqpoint{2.327762in}{1.331911in}}%
\pgfpathlineto{\pgfqpoint{2.330984in}{1.333530in}}%
\pgfpathlineto{\pgfqpoint{2.332058in}{1.331989in}}%
\pgfpathlineto{\pgfqpoint{2.334205in}{1.332213in}}%
\pgfpathlineto{\pgfqpoint{2.335279in}{1.329032in}}%
\pgfpathlineto{\pgfqpoint{2.338501in}{1.328935in}}%
\pgfpathlineto{\pgfqpoint{2.342796in}{1.333738in}}%
\pgfpathlineto{\pgfqpoint{2.347091in}{1.336792in}}%
\pgfpathlineto{\pgfqpoint{2.348165in}{1.337803in}}%
\pgfpathlineto{\pgfqpoint{2.349239in}{1.336891in}}%
\pgfpathlineto{\pgfqpoint{2.350313in}{1.337331in}}%
\pgfpathlineto{\pgfqpoint{2.353535in}{1.335857in}}%
\pgfpathlineto{\pgfqpoint{2.354608in}{1.334726in}}%
\pgfpathlineto{\pgfqpoint{2.356756in}{1.334088in}}%
\pgfpathlineto{\pgfqpoint{2.357830in}{1.333129in}}%
\pgfpathlineto{\pgfqpoint{2.361051in}{1.334653in}}%
\pgfpathlineto{\pgfqpoint{2.362125in}{1.336078in}}%
\pgfpathlineto{\pgfqpoint{2.363199in}{1.333940in}}%
\pgfpathlineto{\pgfqpoint{2.364273in}{1.334611in}}%
\pgfpathlineto{\pgfqpoint{2.365347in}{1.334114in}}%
\pgfpathlineto{\pgfqpoint{2.369642in}{1.336244in}}%
\pgfpathlineto{\pgfqpoint{2.371790in}{1.335639in}}%
\pgfpathlineto{\pgfqpoint{2.372864in}{1.328409in}}%
\pgfpathlineto{\pgfqpoint{2.376085in}{1.331583in}}%
\pgfpathlineto{\pgfqpoint{2.377159in}{1.335589in}}%
\pgfpathlineto{\pgfqpoint{2.378233in}{1.333975in}}%
\pgfpathlineto{\pgfqpoint{2.379307in}{1.336526in}}%
\pgfpathlineto{\pgfqpoint{2.380380in}{1.342877in}}%
\pgfpathlineto{\pgfqpoint{2.384676in}{1.344274in}}%
\pgfpathlineto{\pgfqpoint{2.385750in}{1.348701in}}%
\pgfpathlineto{\pgfqpoint{2.386824in}{1.346034in}}%
\pgfpathlineto{\pgfqpoint{2.387897in}{1.347138in}}%
\pgfpathlineto{\pgfqpoint{2.391119in}{1.346296in}}%
\pgfpathlineto{\pgfqpoint{2.393267in}{1.346595in}}%
\pgfpathlineto{\pgfqpoint{2.394340in}{1.343334in}}%
\pgfpathlineto{\pgfqpoint{2.395414in}{1.344693in}}%
\pgfpathlineto{\pgfqpoint{2.398636in}{1.342573in}}%
\pgfpathlineto{\pgfqpoint{2.400783in}{1.347131in}}%
\pgfpathlineto{\pgfqpoint{2.401857in}{1.347567in}}%
\pgfpathlineto{\pgfqpoint{2.402931in}{1.343738in}}%
\pgfpathlineto{\pgfqpoint{2.406153in}{1.342206in}}%
\pgfpathlineto{\pgfqpoint{2.407226in}{1.339116in}}%
\pgfpathlineto{\pgfqpoint{2.408300in}{1.339526in}}%
\pgfpathlineto{\pgfqpoint{2.409374in}{1.337174in}}%
\pgfpathlineto{\pgfqpoint{2.410448in}{1.336395in}}%
\pgfpathlineto{\pgfqpoint{2.415817in}{1.338781in}}%
\pgfpathlineto{\pgfqpoint{2.417965in}{1.336023in}}%
\pgfpathlineto{\pgfqpoint{2.421186in}{1.338159in}}%
\pgfpathlineto{\pgfqpoint{2.422260in}{1.333508in}}%
\pgfpathlineto{\pgfqpoint{2.423334in}{1.334368in}}%
\pgfpathlineto{\pgfqpoint{2.424408in}{1.332085in}}%
\pgfpathlineto{\pgfqpoint{2.425482in}{1.336415in}}%
\pgfpathlineto{\pgfqpoint{2.428703in}{1.335549in}}%
\pgfpathlineto{\pgfqpoint{2.429777in}{1.335992in}}%
\pgfpathlineto{\pgfqpoint{2.431925in}{1.333638in}}%
\pgfpathlineto{\pgfqpoint{2.432999in}{1.334318in}}%
\pgfpathlineto{\pgfqpoint{2.437294in}{1.329350in}}%
\pgfpathlineto{\pgfqpoint{2.438368in}{1.325984in}}%
\pgfpathlineto{\pgfqpoint{2.439442in}{1.317775in}}%
\pgfpathlineto{\pgfqpoint{2.440515in}{1.316817in}}%
\pgfpathlineto{\pgfqpoint{2.443737in}{1.318921in}}%
\pgfpathlineto{\pgfqpoint{2.448032in}{1.317401in}}%
\pgfpathlineto{\pgfqpoint{2.452328in}{1.317976in}}%
\pgfpathlineto{\pgfqpoint{2.453402in}{1.315446in}}%
\pgfpathlineto{\pgfqpoint{2.455549in}{1.314029in}}%
\pgfpathlineto{\pgfqpoint{2.458771in}{1.314718in}}%
\pgfpathlineto{\pgfqpoint{2.459845in}{1.315545in}}%
\pgfpathlineto{\pgfqpoint{2.460918in}{1.317893in}}%
\pgfpathlineto{\pgfqpoint{2.461992in}{1.317368in}}%
\pgfpathlineto{\pgfqpoint{2.463066in}{1.318943in}}%
\pgfpathlineto{\pgfqpoint{2.467361in}{1.320678in}}%
\pgfpathlineto{\pgfqpoint{2.469509in}{1.328321in}}%
\pgfpathlineto{\pgfqpoint{2.470583in}{1.331355in}}%
\pgfpathlineto{\pgfqpoint{2.473804in}{1.330028in}}%
\pgfpathlineto{\pgfqpoint{2.474878in}{1.330192in}}%
\pgfpathlineto{\pgfqpoint{2.477026in}{1.325379in}}%
\pgfpathlineto{\pgfqpoint{2.478100in}{1.326248in}}%
\pgfpathlineto{\pgfqpoint{2.481321in}{1.321177in}}%
\pgfpathlineto{\pgfqpoint{2.482395in}{1.321537in}}%
\pgfpathlineto{\pgfqpoint{2.484543in}{1.321131in}}%
\pgfpathlineto{\pgfqpoint{2.485617in}{1.321537in}}%
\pgfpathlineto{\pgfqpoint{2.489912in}{1.320838in}}%
\pgfpathlineto{\pgfqpoint{2.493134in}{1.325274in}}%
\pgfpathlineto{\pgfqpoint{2.497429in}{1.327913in}}%
\pgfpathlineto{\pgfqpoint{2.498503in}{1.327774in}}%
\pgfpathlineto{\pgfqpoint{2.499577in}{1.326803in}}%
\pgfpathlineto{\pgfqpoint{2.500650in}{1.324212in}}%
\pgfpathlineto{\pgfqpoint{2.503872in}{1.326433in}}%
\pgfpathlineto{\pgfqpoint{2.507093in}{1.324993in}}%
\pgfpathlineto{\pgfqpoint{2.508167in}{1.323689in}}%
\pgfpathlineto{\pgfqpoint{2.512463in}{1.323163in}}%
\pgfpathlineto{\pgfqpoint{2.513536in}{1.324124in}}%
\pgfpathlineto{\pgfqpoint{2.515684in}{1.323324in}}%
\pgfpathlineto{\pgfqpoint{2.518906in}{1.324278in}}%
\pgfpathlineto{\pgfqpoint{2.521053in}{1.328400in}}%
\pgfpathlineto{\pgfqpoint{2.523201in}{1.325643in}}%
\pgfpathlineto{\pgfqpoint{2.526423in}{1.324508in}}%
\pgfpathlineto{\pgfqpoint{2.527496in}{1.325597in}}%
\pgfpathlineto{\pgfqpoint{2.528570in}{1.328840in}}%
\pgfpathlineto{\pgfqpoint{2.529644in}{1.329245in}}%
\pgfpathlineto{\pgfqpoint{2.530718in}{1.326364in}}%
\pgfpathlineto{\pgfqpoint{2.535013in}{1.322901in}}%
\pgfpathlineto{\pgfqpoint{2.536087in}{1.324707in}}%
\pgfpathlineto{\pgfqpoint{2.537161in}{1.324042in}}%
\pgfpathlineto{\pgfqpoint{2.538235in}{1.326106in}}%
\pgfpathlineto{\pgfqpoint{2.541456in}{1.324019in}}%
\pgfpathlineto{\pgfqpoint{2.542530in}{1.324966in}}%
\pgfpathlineto{\pgfqpoint{2.543604in}{1.327101in}}%
\pgfpathlineto{\pgfqpoint{2.545752in}{1.326579in}}%
\pgfpathlineto{\pgfqpoint{2.551121in}{1.325495in}}%
\pgfpathlineto{\pgfqpoint{2.553268in}{1.327498in}}%
\pgfpathlineto{\pgfqpoint{2.556490in}{1.326751in}}%
\pgfpathlineto{\pgfqpoint{2.557564in}{1.327291in}}%
\pgfpathlineto{\pgfqpoint{2.559712in}{1.331833in}}%
\pgfpathlineto{\pgfqpoint{2.560785in}{1.331074in}}%
\pgfpathlineto{\pgfqpoint{2.564007in}{1.329602in}}%
\pgfpathlineto{\pgfqpoint{2.566155in}{1.332086in}}%
\pgfpathlineto{\pgfqpoint{2.567228in}{1.332481in}}%
\pgfpathlineto{\pgfqpoint{2.568302in}{1.332017in}}%
\pgfpathlineto{\pgfqpoint{2.571524in}{1.331646in}}%
\pgfpathlineto{\pgfqpoint{2.572598in}{1.332684in}}%
\pgfpathlineto{\pgfqpoint{2.575819in}{1.339530in}}%
\pgfpathlineto{\pgfqpoint{2.579041in}{1.339389in}}%
\pgfpathlineto{\pgfqpoint{2.580114in}{1.342401in}}%
\pgfpathlineto{\pgfqpoint{2.582262in}{1.341678in}}%
\pgfpathlineto{\pgfqpoint{2.586557in}{1.341654in}}%
\pgfpathlineto{\pgfqpoint{2.588705in}{1.345270in}}%
\pgfpathlineto{\pgfqpoint{2.590853in}{1.346685in}}%
\pgfpathlineto{\pgfqpoint{2.594074in}{1.346024in}}%
\pgfpathlineto{\pgfqpoint{2.596222in}{1.349600in}}%
\pgfpathlineto{\pgfqpoint{2.597296in}{1.349402in}}%
\pgfpathlineto{\pgfqpoint{2.598370in}{1.350737in}}%
\pgfpathlineto{\pgfqpoint{2.602665in}{1.351603in}}%
\pgfpathlineto{\pgfqpoint{2.603739in}{1.353707in}}%
\pgfpathlineto{\pgfqpoint{2.604813in}{1.353330in}}%
\pgfpathlineto{\pgfqpoint{2.609108in}{1.356147in}}%
\pgfpathlineto{\pgfqpoint{2.610182in}{1.355468in}}%
\pgfpathlineto{\pgfqpoint{2.612330in}{1.359399in}}%
\pgfpathlineto{\pgfqpoint{2.613403in}{1.356893in}}%
\pgfpathlineto{\pgfqpoint{2.617699in}{1.352500in}}%
\pgfpathlineto{\pgfqpoint{2.618773in}{1.349470in}}%
\pgfpathlineto{\pgfqpoint{2.620920in}{1.347735in}}%
\pgfpathlineto{\pgfqpoint{2.624142in}{1.347875in}}%
\pgfpathlineto{\pgfqpoint{2.625216in}{1.348734in}}%
\pgfpathlineto{\pgfqpoint{2.627363in}{1.345067in}}%
\pgfpathlineto{\pgfqpoint{2.628437in}{1.344322in}}%
\pgfpathlineto{\pgfqpoint{2.631659in}{1.344883in}}%
\pgfpathlineto{\pgfqpoint{2.635954in}{1.348186in}}%
\pgfpathlineto{\pgfqpoint{2.639176in}{1.348369in}}%
\pgfpathlineto{\pgfqpoint{2.640249in}{1.347316in}}%
\pgfpathlineto{\pgfqpoint{2.641323in}{1.349813in}}%
\pgfpathlineto{\pgfqpoint{2.642397in}{1.349350in}}%
\pgfpathlineto{\pgfqpoint{2.643471in}{1.353284in}}%
\pgfpathlineto{\pgfqpoint{2.649914in}{1.351643in}}%
\pgfpathlineto{\pgfqpoint{2.650988in}{1.352509in}}%
\pgfpathlineto{\pgfqpoint{2.656357in}{1.351270in}}%
\pgfpathlineto{\pgfqpoint{2.657431in}{1.352394in}}%
\pgfpathlineto{\pgfqpoint{2.658505in}{1.351683in}}%
\pgfpathlineto{\pgfqpoint{2.661726in}{1.354148in}}%
\pgfpathlineto{\pgfqpoint{2.662800in}{1.352241in}}%
\pgfpathlineto{\pgfqpoint{2.663874in}{1.351498in}}%
\pgfpathlineto{\pgfqpoint{2.664948in}{1.351846in}}%
\pgfpathlineto{\pgfqpoint{2.666022in}{1.350312in}}%
\pgfpathlineto{\pgfqpoint{2.672465in}{1.354085in}}%
\pgfpathlineto{\pgfqpoint{2.673538in}{1.354807in}}%
\pgfpathlineto{\pgfqpoint{2.677834in}{1.358717in}}%
\pgfpathlineto{\pgfqpoint{2.678908in}{1.359258in}}%
\pgfpathlineto{\pgfqpoint{2.679981in}{1.358317in}}%
\pgfpathlineto{\pgfqpoint{2.684277in}{1.357452in}}%
\pgfpathlineto{\pgfqpoint{2.685351in}{1.356845in}}%
\pgfpathlineto{\pgfqpoint{2.688572in}{1.361789in}}%
\pgfpathlineto{\pgfqpoint{2.691794in}{1.361860in}}%
\pgfpathlineto{\pgfqpoint{2.693941in}{1.363472in}}%
\pgfpathlineto{\pgfqpoint{2.696089in}{1.367146in}}%
\pgfpathlineto{\pgfqpoint{2.699311in}{1.366112in}}%
\pgfpathlineto{\pgfqpoint{2.701458in}{1.364634in}}%
\pgfpathlineto{\pgfqpoint{2.703606in}{1.366840in}}%
\pgfpathlineto{\pgfqpoint{2.706827in}{1.367508in}}%
\pgfpathlineto{\pgfqpoint{2.707901in}{1.366405in}}%
\pgfpathlineto{\pgfqpoint{2.710049in}{1.368875in}}%
\pgfpathlineto{\pgfqpoint{2.711123in}{1.371275in}}%
\pgfpathlineto{\pgfqpoint{2.714344in}{1.372099in}}%
\pgfpathlineto{\pgfqpoint{2.715418in}{1.365810in}}%
\pgfpathlineto{\pgfqpoint{2.717566in}{1.361812in}}%
\pgfpathlineto{\pgfqpoint{2.722935in}{1.363619in}}%
\pgfpathlineto{\pgfqpoint{2.725083in}{1.358493in}}%
\pgfpathlineto{\pgfqpoint{2.726156in}{1.357619in}}%
\pgfpathlineto{\pgfqpoint{2.729378in}{1.364233in}}%
\pgfpathlineto{\pgfqpoint{2.730452in}{1.363469in}}%
\pgfpathlineto{\pgfqpoint{2.732600in}{1.360120in}}%
\pgfpathlineto{\pgfqpoint{2.733673in}{1.360416in}}%
\pgfpathlineto{\pgfqpoint{2.736895in}{1.359800in}}%
\pgfpathlineto{\pgfqpoint{2.737969in}{1.362570in}}%
\pgfpathlineto{\pgfqpoint{2.740116in}{1.373679in}}%
\pgfpathlineto{\pgfqpoint{2.741190in}{1.372154in}}%
\pgfpathlineto{\pgfqpoint{2.744412in}{1.371663in}}%
\pgfpathlineto{\pgfqpoint{2.745486in}{1.372671in}}%
\pgfpathlineto{\pgfqpoint{2.746559in}{1.369363in}}%
\pgfpathlineto{\pgfqpoint{2.747633in}{1.373529in}}%
\pgfpathlineto{\pgfqpoint{2.748707in}{1.372957in}}%
\pgfpathlineto{\pgfqpoint{2.751929in}{1.373607in}}%
\pgfpathlineto{\pgfqpoint{2.753002in}{1.365234in}}%
\pgfpathlineto{\pgfqpoint{2.754076in}{1.363124in}}%
\pgfpathlineto{\pgfqpoint{2.755150in}{1.362859in}}%
\pgfpathlineto{\pgfqpoint{2.756224in}{1.367006in}}%
\pgfpathlineto{\pgfqpoint{2.761593in}{1.347276in}}%
\pgfpathlineto{\pgfqpoint{2.763741in}{1.349165in}}%
\pgfpathlineto{\pgfqpoint{2.766962in}{1.349390in}}%
\pgfpathlineto{\pgfqpoint{2.769110in}{1.347866in}}%
\pgfpathlineto{\pgfqpoint{2.770184in}{1.341109in}}%
\pgfpathlineto{\pgfqpoint{2.771258in}{1.341226in}}%
\pgfpathlineto{\pgfqpoint{2.774479in}{1.340520in}}%
\pgfpathlineto{\pgfqpoint{2.775553in}{1.342496in}}%
\pgfpathlineto{\pgfqpoint{2.778775in}{1.339462in}}%
\pgfpathlineto{\pgfqpoint{2.783070in}{1.342001in}}%
\pgfpathlineto{\pgfqpoint{2.784144in}{1.336535in}}%
\pgfpathlineto{\pgfqpoint{2.785218in}{1.339387in}}%
\pgfpathlineto{\pgfqpoint{2.786291in}{1.339911in}}%
\pgfpathlineto{\pgfqpoint{2.789513in}{1.336628in}}%
\pgfpathlineto{\pgfqpoint{2.791661in}{1.336539in}}%
\pgfpathlineto{\pgfqpoint{2.792734in}{1.337693in}}%
\pgfpathlineto{\pgfqpoint{2.797030in}{1.336717in}}%
\pgfpathlineto{\pgfqpoint{2.798104in}{1.337797in}}%
\pgfpathlineto{\pgfqpoint{2.799178in}{1.337019in}}%
\pgfpathlineto{\pgfqpoint{2.800251in}{1.339285in}}%
\pgfpathlineto{\pgfqpoint{2.804547in}{1.340131in}}%
\pgfpathlineto{\pgfqpoint{2.805621in}{1.339374in}}%
\pgfpathlineto{\pgfqpoint{2.807768in}{1.340438in}}%
\pgfpathlineto{\pgfqpoint{2.808842in}{1.337375in}}%
\pgfpathlineto{\pgfqpoint{2.812064in}{1.337462in}}%
\pgfpathlineto{\pgfqpoint{2.814211in}{1.333516in}}%
\pgfpathlineto{\pgfqpoint{2.816359in}{1.337245in}}%
\pgfpathlineto{\pgfqpoint{2.821728in}{1.339753in}}%
\pgfpathlineto{\pgfqpoint{2.822802in}{1.336935in}}%
\pgfpathlineto{\pgfqpoint{2.823876in}{1.337731in}}%
\pgfpathlineto{\pgfqpoint{2.827097in}{1.336323in}}%
\pgfpathlineto{\pgfqpoint{2.829245in}{1.338013in}}%
\pgfpathlineto{\pgfqpoint{2.830319in}{1.334017in}}%
\pgfpathlineto{\pgfqpoint{2.834614in}{1.331309in}}%
\pgfpathlineto{\pgfqpoint{2.836762in}{1.337467in}}%
\pgfpathlineto{\pgfqpoint{2.838910in}{1.342572in}}%
\pgfpathlineto{\pgfqpoint{2.842131in}{1.342230in}}%
\pgfpathlineto{\pgfqpoint{2.844279in}{1.344375in}}%
\pgfpathlineto{\pgfqpoint{2.846426in}{1.344764in}}%
\pgfpathlineto{\pgfqpoint{2.849648in}{1.344283in}}%
\pgfpathlineto{\pgfqpoint{2.851796in}{1.342008in}}%
\pgfpathlineto{\pgfqpoint{2.852869in}{1.334914in}}%
\pgfpathlineto{\pgfqpoint{2.853943in}{1.337632in}}%
\pgfpathlineto{\pgfqpoint{2.857165in}{1.337458in}}%
\pgfpathlineto{\pgfqpoint{2.858239in}{1.337979in}}%
\pgfpathlineto{\pgfqpoint{2.861460in}{1.332772in}}%
\pgfpathlineto{\pgfqpoint{2.865756in}{1.331443in}}%
\pgfpathlineto{\pgfqpoint{2.866829in}{1.329816in}}%
\pgfpathlineto{\pgfqpoint{2.867903in}{1.330873in}}%
\pgfpathlineto{\pgfqpoint{2.868977in}{1.335628in}}%
\pgfpathlineto{\pgfqpoint{2.876494in}{1.333568in}}%
\pgfpathlineto{\pgfqpoint{2.880789in}{1.333407in}}%
\pgfpathlineto{\pgfqpoint{2.881863in}{1.334254in}}%
\pgfpathlineto{\pgfqpoint{2.882937in}{1.333205in}}%
\pgfpathlineto{\pgfqpoint{2.884011in}{1.334265in}}%
\pgfpathlineto{\pgfqpoint{2.884011in}{1.334265in}}%
\pgfusepath{stroke}%
\end{pgfscope}%
\begin{pgfscope}%
\pgfsetrectcap%
\pgfsetmiterjoin%
\pgfsetlinewidth{0.803000pt}%
\definecolor{currentstroke}{rgb}{1.000000,1.000000,1.000000}%
\pgfsetstrokecolor{currentstroke}%
\pgfsetdash{}{0pt}%
\pgfpathmoveto{\pgfqpoint{0.418102in}{1.293759in}}%
\pgfpathlineto{\pgfqpoint{0.418102in}{1.694644in}}%
\pgfusepath{stroke}%
\end{pgfscope}%
\begin{pgfscope}%
\pgfsetrectcap%
\pgfsetmiterjoin%
\pgfsetlinewidth{0.803000pt}%
\definecolor{currentstroke}{rgb}{1.000000,1.000000,1.000000}%
\pgfsetstrokecolor{currentstroke}%
\pgfsetdash{}{0pt}%
\pgfpathmoveto{\pgfqpoint{3.001435in}{1.293759in}}%
\pgfpathlineto{\pgfqpoint{3.001435in}{1.694644in}}%
\pgfusepath{stroke}%
\end{pgfscope}%
\begin{pgfscope}%
\pgfsetrectcap%
\pgfsetmiterjoin%
\pgfsetlinewidth{0.803000pt}%
\definecolor{currentstroke}{rgb}{1.000000,1.000000,1.000000}%
\pgfsetstrokecolor{currentstroke}%
\pgfsetdash{}{0pt}%
\pgfpathmoveto{\pgfqpoint{0.418102in}{1.293759in}}%
\pgfpathlineto{\pgfqpoint{3.001435in}{1.293759in}}%
\pgfusepath{stroke}%
\end{pgfscope}%
\begin{pgfscope}%
\pgfsetrectcap%
\pgfsetmiterjoin%
\pgfsetlinewidth{0.803000pt}%
\definecolor{currentstroke}{rgb}{1.000000,1.000000,1.000000}%
\pgfsetstrokecolor{currentstroke}%
\pgfsetdash{}{0pt}%
\pgfpathmoveto{\pgfqpoint{0.418102in}{1.694644in}}%
\pgfpathlineto{\pgfqpoint{3.001435in}{1.694644in}}%
\pgfusepath{stroke}%
\end{pgfscope}%
\begin{pgfscope}%
\definecolor{textcolor}{rgb}{0.150000,0.150000,0.150000}%
\pgfsetstrokecolor{textcolor}%
\pgfsetfillcolor{textcolor}%
\pgftext[x=1.709768in,y=1.777977in,,base]{\color{textcolor}\rmfamily\fontsize{12.000000}{14.400000}\selectfont UTX}%
\end{pgfscope}%
\begin{pgfscope}%
\pgfsetbuttcap%
\pgfsetmiterjoin%
\definecolor{currentfill}{rgb}{0.917647,0.917647,0.949020}%
\pgfsetfillcolor{currentfill}%
\pgfsetlinewidth{0.000000pt}%
\definecolor{currentstroke}{rgb}{0.000000,0.000000,0.000000}%
\pgfsetstrokecolor{currentstroke}%
\pgfsetstrokeopacity{0.000000}%
\pgfsetdash{}{0pt}%
\pgfpathmoveto{\pgfqpoint{4.034768in}{1.293759in}}%
\pgfpathlineto{\pgfqpoint{6.618102in}{1.293759in}}%
\pgfpathlineto{\pgfqpoint{6.618102in}{1.694644in}}%
\pgfpathlineto{\pgfqpoint{4.034768in}{1.694644in}}%
\pgfpathclose%
\pgfusepath{fill}%
\end{pgfscope}%
\begin{pgfscope}%
\pgfpathrectangle{\pgfqpoint{4.034768in}{1.293759in}}{\pgfqpoint{2.583333in}{0.400885in}}%
\pgfusepath{clip}%
\pgfsetroundcap%
\pgfsetroundjoin%
\pgfsetlinewidth{0.803000pt}%
\definecolor{currentstroke}{rgb}{1.000000,1.000000,1.000000}%
\pgfsetstrokecolor{currentstroke}%
\pgfsetdash{}{0pt}%
\pgfpathmoveto{\pgfqpoint{4.150045in}{1.293759in}}%
\pgfpathlineto{\pgfqpoint{4.150045in}{1.694644in}}%
\pgfusepath{stroke}%
\end{pgfscope}%
\begin{pgfscope}%
\definecolor{textcolor}{rgb}{0.150000,0.150000,0.150000}%
\pgfsetstrokecolor{textcolor}%
\pgfsetfillcolor{textcolor}%
\pgftext[x=4.150045in,y=1.196537in,,top]{\color{textcolor}\rmfamily\fontsize{10.000000}{12.000000}\selectfont 2012}%
\end{pgfscope}%
\begin{pgfscope}%
\pgfpathrectangle{\pgfqpoint{4.034768in}{1.293759in}}{\pgfqpoint{2.583333in}{0.400885in}}%
\pgfusepath{clip}%
\pgfsetroundcap%
\pgfsetroundjoin%
\pgfsetlinewidth{0.803000pt}%
\definecolor{currentstroke}{rgb}{1.000000,1.000000,1.000000}%
\pgfsetstrokecolor{currentstroke}%
\pgfsetdash{}{0pt}%
\pgfpathmoveto{\pgfqpoint{4.543070in}{1.293759in}}%
\pgfpathlineto{\pgfqpoint{4.543070in}{1.694644in}}%
\pgfusepath{stroke}%
\end{pgfscope}%
\begin{pgfscope}%
\definecolor{textcolor}{rgb}{0.150000,0.150000,0.150000}%
\pgfsetstrokecolor{textcolor}%
\pgfsetfillcolor{textcolor}%
\pgftext[x=4.543070in,y=1.196537in,,top]{\color{textcolor}\rmfamily\fontsize{10.000000}{12.000000}\selectfont 2013}%
\end{pgfscope}%
\begin{pgfscope}%
\pgfpathrectangle{\pgfqpoint{4.034768in}{1.293759in}}{\pgfqpoint{2.583333in}{0.400885in}}%
\pgfusepath{clip}%
\pgfsetroundcap%
\pgfsetroundjoin%
\pgfsetlinewidth{0.803000pt}%
\definecolor{currentstroke}{rgb}{1.000000,1.000000,1.000000}%
\pgfsetstrokecolor{currentstroke}%
\pgfsetdash{}{0pt}%
\pgfpathmoveto{\pgfqpoint{4.935021in}{1.293759in}}%
\pgfpathlineto{\pgfqpoint{4.935021in}{1.694644in}}%
\pgfusepath{stroke}%
\end{pgfscope}%
\begin{pgfscope}%
\definecolor{textcolor}{rgb}{0.150000,0.150000,0.150000}%
\pgfsetstrokecolor{textcolor}%
\pgfsetfillcolor{textcolor}%
\pgftext[x=4.935021in,y=1.196537in,,top]{\color{textcolor}\rmfamily\fontsize{10.000000}{12.000000}\selectfont 2014}%
\end{pgfscope}%
\begin{pgfscope}%
\pgfpathrectangle{\pgfqpoint{4.034768in}{1.293759in}}{\pgfqpoint{2.583333in}{0.400885in}}%
\pgfusepath{clip}%
\pgfsetroundcap%
\pgfsetroundjoin%
\pgfsetlinewidth{0.803000pt}%
\definecolor{currentstroke}{rgb}{1.000000,1.000000,1.000000}%
\pgfsetstrokecolor{currentstroke}%
\pgfsetdash{}{0pt}%
\pgfpathmoveto{\pgfqpoint{5.326972in}{1.293759in}}%
\pgfpathlineto{\pgfqpoint{5.326972in}{1.694644in}}%
\pgfusepath{stroke}%
\end{pgfscope}%
\begin{pgfscope}%
\definecolor{textcolor}{rgb}{0.150000,0.150000,0.150000}%
\pgfsetstrokecolor{textcolor}%
\pgfsetfillcolor{textcolor}%
\pgftext[x=5.326972in,y=1.196537in,,top]{\color{textcolor}\rmfamily\fontsize{10.000000}{12.000000}\selectfont 2015}%
\end{pgfscope}%
\begin{pgfscope}%
\pgfpathrectangle{\pgfqpoint{4.034768in}{1.293759in}}{\pgfqpoint{2.583333in}{0.400885in}}%
\pgfusepath{clip}%
\pgfsetroundcap%
\pgfsetroundjoin%
\pgfsetlinewidth{0.803000pt}%
\definecolor{currentstroke}{rgb}{1.000000,1.000000,1.000000}%
\pgfsetstrokecolor{currentstroke}%
\pgfsetdash{}{0pt}%
\pgfpathmoveto{\pgfqpoint{5.718923in}{1.293759in}}%
\pgfpathlineto{\pgfqpoint{5.718923in}{1.694644in}}%
\pgfusepath{stroke}%
\end{pgfscope}%
\begin{pgfscope}%
\definecolor{textcolor}{rgb}{0.150000,0.150000,0.150000}%
\pgfsetstrokecolor{textcolor}%
\pgfsetfillcolor{textcolor}%
\pgftext[x=5.718923in,y=1.196537in,,top]{\color{textcolor}\rmfamily\fontsize{10.000000}{12.000000}\selectfont 2016}%
\end{pgfscope}%
\begin{pgfscope}%
\pgfpathrectangle{\pgfqpoint{4.034768in}{1.293759in}}{\pgfqpoint{2.583333in}{0.400885in}}%
\pgfusepath{clip}%
\pgfsetroundcap%
\pgfsetroundjoin%
\pgfsetlinewidth{0.803000pt}%
\definecolor{currentstroke}{rgb}{1.000000,1.000000,1.000000}%
\pgfsetstrokecolor{currentstroke}%
\pgfsetdash{}{0pt}%
\pgfpathmoveto{\pgfqpoint{6.111948in}{1.293759in}}%
\pgfpathlineto{\pgfqpoint{6.111948in}{1.694644in}}%
\pgfusepath{stroke}%
\end{pgfscope}%
\begin{pgfscope}%
\definecolor{textcolor}{rgb}{0.150000,0.150000,0.150000}%
\pgfsetstrokecolor{textcolor}%
\pgfsetfillcolor{textcolor}%
\pgftext[x=6.111948in,y=1.196537in,,top]{\color{textcolor}\rmfamily\fontsize{10.000000}{12.000000}\selectfont 2017}%
\end{pgfscope}%
\begin{pgfscope}%
\pgfpathrectangle{\pgfqpoint{4.034768in}{1.293759in}}{\pgfqpoint{2.583333in}{0.400885in}}%
\pgfusepath{clip}%
\pgfsetroundcap%
\pgfsetroundjoin%
\pgfsetlinewidth{0.803000pt}%
\definecolor{currentstroke}{rgb}{1.000000,1.000000,1.000000}%
\pgfsetstrokecolor{currentstroke}%
\pgfsetdash{}{0pt}%
\pgfpathmoveto{\pgfqpoint{6.503899in}{1.293759in}}%
\pgfpathlineto{\pgfqpoint{6.503899in}{1.694644in}}%
\pgfusepath{stroke}%
\end{pgfscope}%
\begin{pgfscope}%
\definecolor{textcolor}{rgb}{0.150000,0.150000,0.150000}%
\pgfsetstrokecolor{textcolor}%
\pgfsetfillcolor{textcolor}%
\pgftext[x=6.503899in,y=1.196537in,,top]{\color{textcolor}\rmfamily\fontsize{10.000000}{12.000000}\selectfont 2018}%
\end{pgfscope}%
\begin{pgfscope}%
\pgfpathrectangle{\pgfqpoint{4.034768in}{1.293759in}}{\pgfqpoint{2.583333in}{0.400885in}}%
\pgfusepath{clip}%
\pgfsetroundcap%
\pgfsetroundjoin%
\pgfsetlinewidth{0.803000pt}%
\definecolor{currentstroke}{rgb}{1.000000,1.000000,1.000000}%
\pgfsetstrokecolor{currentstroke}%
\pgfsetdash{}{0pt}%
\pgfpathmoveto{\pgfqpoint{4.034768in}{1.353870in}}%
\pgfpathlineto{\pgfqpoint{6.618102in}{1.353870in}}%
\pgfusepath{stroke}%
\end{pgfscope}%
\begin{pgfscope}%
\definecolor{textcolor}{rgb}{0.150000,0.150000,0.150000}%
\pgfsetstrokecolor{textcolor}%
\pgfsetfillcolor{textcolor}%
\pgftext[x=3.716667in,y=1.301109in,left,base]{\color{textcolor}\rmfamily\fontsize{10.000000}{12.000000}\selectfont 1.0}%
\end{pgfscope}%
\begin{pgfscope}%
\pgfpathrectangle{\pgfqpoint{4.034768in}{1.293759in}}{\pgfqpoint{2.583333in}{0.400885in}}%
\pgfusepath{clip}%
\pgfsetroundcap%
\pgfsetroundjoin%
\pgfsetlinewidth{0.803000pt}%
\definecolor{currentstroke}{rgb}{1.000000,1.000000,1.000000}%
\pgfsetstrokecolor{currentstroke}%
\pgfsetdash{}{0pt}%
\pgfpathmoveto{\pgfqpoint{4.034768in}{1.563356in}}%
\pgfpathlineto{\pgfqpoint{6.618102in}{1.563356in}}%
\pgfusepath{stroke}%
\end{pgfscope}%
\begin{pgfscope}%
\definecolor{textcolor}{rgb}{0.150000,0.150000,0.150000}%
\pgfsetstrokecolor{textcolor}%
\pgfsetfillcolor{textcolor}%
\pgftext[x=3.716667in,y=1.510595in,left,base]{\color{textcolor}\rmfamily\fontsize{10.000000}{12.000000}\selectfont 1.5}%
\end{pgfscope}%
\begin{pgfscope}%
\pgfpathrectangle{\pgfqpoint{4.034768in}{1.293759in}}{\pgfqpoint{2.583333in}{0.400885in}}%
\pgfusepath{clip}%
\pgfsetroundcap%
\pgfsetroundjoin%
\pgfsetlinewidth{1.505625pt}%
\definecolor{currentstroke}{rgb}{0.000000,0.000000,0.000000}%
\pgfsetstrokecolor{currentstroke}%
\pgfsetdash{}{0pt}%
\pgfpathmoveto{\pgfqpoint{4.152193in}{1.353870in}}%
\pgfpathlineto{\pgfqpoint{4.154340in}{1.345623in}}%
\pgfpathlineto{\pgfqpoint{4.155414in}{1.344423in}}%
\pgfpathlineto{\pgfqpoint{4.158636in}{1.344873in}}%
\pgfpathlineto{\pgfqpoint{4.159709in}{1.346972in}}%
\pgfpathlineto{\pgfqpoint{4.160783in}{1.350571in}}%
\pgfpathlineto{\pgfqpoint{4.169374in}{1.351621in}}%
\pgfpathlineto{\pgfqpoint{4.170448in}{1.351321in}}%
\pgfpathlineto{\pgfqpoint{4.173669in}{1.345173in}}%
\pgfpathlineto{\pgfqpoint{4.174743in}{1.338725in}}%
\pgfpathlineto{\pgfqpoint{4.175817in}{1.337525in}}%
\pgfpathlineto{\pgfqpoint{4.176891in}{1.333776in}}%
\pgfpathlineto{\pgfqpoint{4.177965in}{1.332427in}}%
\pgfpathlineto{\pgfqpoint{4.181186in}{1.336776in}}%
\pgfpathlineto{\pgfqpoint{4.182260in}{1.337225in}}%
\pgfpathlineto{\pgfqpoint{4.183334in}{1.338725in}}%
\pgfpathlineto{\pgfqpoint{4.184408in}{1.336176in}}%
\pgfpathlineto{\pgfqpoint{4.185482in}{1.339175in}}%
\pgfpathlineto{\pgfqpoint{4.188703in}{1.342324in}}%
\pgfpathlineto{\pgfqpoint{4.189777in}{1.340074in}}%
\pgfpathlineto{\pgfqpoint{4.191925in}{1.340074in}}%
\pgfpathlineto{\pgfqpoint{4.192998in}{1.337525in}}%
\pgfpathlineto{\pgfqpoint{4.196220in}{1.342324in}}%
\pgfpathlineto{\pgfqpoint{4.197294in}{1.341274in}}%
\pgfpathlineto{\pgfqpoint{4.198368in}{1.339025in}}%
\pgfpathlineto{\pgfqpoint{4.199441in}{1.341424in}}%
\pgfpathlineto{\pgfqpoint{4.200515in}{1.345773in}}%
\pgfpathlineto{\pgfqpoint{4.204811in}{1.346073in}}%
\pgfpathlineto{\pgfqpoint{4.205885in}{1.343074in}}%
\pgfpathlineto{\pgfqpoint{4.206958in}{1.342324in}}%
\pgfpathlineto{\pgfqpoint{4.213401in}{1.342024in}}%
\pgfpathlineto{\pgfqpoint{4.215549in}{1.348022in}}%
\pgfpathlineto{\pgfqpoint{4.218771in}{1.351621in}}%
\pgfpathlineto{\pgfqpoint{4.219844in}{1.348322in}}%
\pgfpathlineto{\pgfqpoint{4.220918in}{1.350271in}}%
\pgfpathlineto{\pgfqpoint{4.221992in}{1.353870in}}%
\pgfpathlineto{\pgfqpoint{4.223066in}{1.352671in}}%
\pgfpathlineto{\pgfqpoint{4.226287in}{1.355070in}}%
\pgfpathlineto{\pgfqpoint{4.227361in}{1.356719in}}%
\pgfpathlineto{\pgfqpoint{4.228435in}{1.356569in}}%
\pgfpathlineto{\pgfqpoint{4.230583in}{1.357619in}}%
\pgfpathlineto{\pgfqpoint{4.234878in}{1.358369in}}%
\pgfpathlineto{\pgfqpoint{4.235952in}{1.359868in}}%
\pgfpathlineto{\pgfqpoint{4.237026in}{1.358669in}}%
\pgfpathlineto{\pgfqpoint{4.238100in}{1.356120in}}%
\pgfpathlineto{\pgfqpoint{4.241321in}{1.355070in}}%
\pgfpathlineto{\pgfqpoint{4.243469in}{1.343673in}}%
\pgfpathlineto{\pgfqpoint{4.244543in}{1.341724in}}%
\pgfpathlineto{\pgfqpoint{4.245617in}{1.343373in}}%
\pgfpathlineto{\pgfqpoint{4.248838in}{1.346523in}}%
\pgfpathlineto{\pgfqpoint{4.250986in}{1.345023in}}%
\pgfpathlineto{\pgfqpoint{4.252060in}{1.342624in}}%
\pgfpathlineto{\pgfqpoint{4.256355in}{1.340374in}}%
\pgfpathlineto{\pgfqpoint{4.257429in}{1.333327in}}%
\pgfpathlineto{\pgfqpoint{4.258503in}{1.339475in}}%
\pgfpathlineto{\pgfqpoint{4.259576in}{1.341424in}}%
\pgfpathlineto{\pgfqpoint{4.260650in}{1.338275in}}%
\pgfpathlineto{\pgfqpoint{4.263872in}{1.340074in}}%
\pgfpathlineto{\pgfqpoint{4.264946in}{1.343373in}}%
\pgfpathlineto{\pgfqpoint{4.266019in}{1.342624in}}%
\pgfpathlineto{\pgfqpoint{4.268167in}{1.354170in}}%
\pgfpathlineto{\pgfqpoint{4.271389in}{1.352371in}}%
\pgfpathlineto{\pgfqpoint{4.272462in}{1.362568in}}%
\pgfpathlineto{\pgfqpoint{4.273536in}{1.362268in}}%
\pgfpathlineto{\pgfqpoint{4.274610in}{1.369466in}}%
\pgfpathlineto{\pgfqpoint{4.275684in}{1.370365in}}%
\pgfpathlineto{\pgfqpoint{4.278906in}{1.372015in}}%
\pgfpathlineto{\pgfqpoint{4.279979in}{1.373964in}}%
\pgfpathlineto{\pgfqpoint{4.282127in}{1.374864in}}%
\pgfpathlineto{\pgfqpoint{4.283201in}{1.370665in}}%
\pgfpathlineto{\pgfqpoint{4.286422in}{1.373664in}}%
\pgfpathlineto{\pgfqpoint{4.287496in}{1.373814in}}%
\pgfpathlineto{\pgfqpoint{4.288570in}{1.370665in}}%
\pgfpathlineto{\pgfqpoint{4.289644in}{1.373814in}}%
\pgfpathlineto{\pgfqpoint{4.290718in}{1.380412in}}%
\pgfpathlineto{\pgfqpoint{4.293939in}{1.377563in}}%
\pgfpathlineto{\pgfqpoint{4.295013in}{1.379213in}}%
\pgfpathlineto{\pgfqpoint{4.296087in}{1.377413in}}%
\pgfpathlineto{\pgfqpoint{4.297161in}{1.382811in}}%
\pgfpathlineto{\pgfqpoint{4.298235in}{1.384461in}}%
\pgfpathlineto{\pgfqpoint{4.301456in}{1.382362in}}%
\pgfpathlineto{\pgfqpoint{4.302530in}{1.382961in}}%
\pgfpathlineto{\pgfqpoint{4.303604in}{1.381762in}}%
\pgfpathlineto{\pgfqpoint{4.305751in}{1.383561in}}%
\pgfpathlineto{\pgfqpoint{4.310047in}{1.386860in}}%
\pgfpathlineto{\pgfqpoint{4.311121in}{1.383261in}}%
\pgfpathlineto{\pgfqpoint{4.312195in}{1.385661in}}%
\pgfpathlineto{\pgfqpoint{4.313268in}{1.379063in}}%
\pgfpathlineto{\pgfqpoint{4.316490in}{1.382362in}}%
\pgfpathlineto{\pgfqpoint{4.317564in}{1.380862in}}%
\pgfpathlineto{\pgfqpoint{4.318638in}{1.387010in}}%
\pgfpathlineto{\pgfqpoint{4.319711in}{1.385661in}}%
\pgfpathlineto{\pgfqpoint{4.320785in}{1.394358in}}%
\pgfpathlineto{\pgfqpoint{4.324007in}{1.395558in}}%
\pgfpathlineto{\pgfqpoint{4.325081in}{1.399756in}}%
\pgfpathlineto{\pgfqpoint{4.326154in}{1.400206in}}%
\pgfpathlineto{\pgfqpoint{4.327228in}{1.408454in}}%
\pgfpathlineto{\pgfqpoint{4.328302in}{1.406354in}}%
\pgfpathlineto{\pgfqpoint{4.331524in}{1.409203in}}%
\pgfpathlineto{\pgfqpoint{4.332597in}{1.408304in}}%
\pgfpathlineto{\pgfqpoint{4.333671in}{1.403655in}}%
\pgfpathlineto{\pgfqpoint{4.334745in}{1.403955in}}%
\pgfpathlineto{\pgfqpoint{4.335819in}{1.410703in}}%
\pgfpathlineto{\pgfqpoint{4.339040in}{1.407404in}}%
\pgfpathlineto{\pgfqpoint{4.340114in}{1.409503in}}%
\pgfpathlineto{\pgfqpoint{4.341188in}{1.408903in}}%
\pgfpathlineto{\pgfqpoint{4.342262in}{1.410853in}}%
\pgfpathlineto{\pgfqpoint{4.343336in}{1.415951in}}%
\pgfpathlineto{\pgfqpoint{4.346557in}{1.421350in}}%
\pgfpathlineto{\pgfqpoint{4.347631in}{1.421500in}}%
\pgfpathlineto{\pgfqpoint{4.349779in}{1.420900in}}%
\pgfpathlineto{\pgfqpoint{4.350853in}{1.421200in}}%
\pgfpathlineto{\pgfqpoint{4.354074in}{1.424649in}}%
\pgfpathlineto{\pgfqpoint{4.355148in}{1.424199in}}%
\pgfpathlineto{\pgfqpoint{4.356222in}{1.426448in}}%
\pgfpathlineto{\pgfqpoint{4.357296in}{1.423899in}}%
\pgfpathlineto{\pgfqpoint{4.358370in}{1.429897in}}%
\pgfpathlineto{\pgfqpoint{4.361591in}{1.430497in}}%
\pgfpathlineto{\pgfqpoint{4.363739in}{1.437245in}}%
\pgfpathlineto{\pgfqpoint{4.364813in}{1.422549in}}%
\pgfpathlineto{\pgfqpoint{4.369108in}{1.420450in}}%
\pgfpathlineto{\pgfqpoint{4.370182in}{1.414002in}}%
\pgfpathlineto{\pgfqpoint{4.371256in}{1.413552in}}%
\pgfpathlineto{\pgfqpoint{4.373403in}{1.426448in}}%
\pgfpathlineto{\pgfqpoint{4.376625in}{1.427048in}}%
\pgfpathlineto{\pgfqpoint{4.378773in}{1.429897in}}%
\pgfpathlineto{\pgfqpoint{4.379846in}{1.423299in}}%
\pgfpathlineto{\pgfqpoint{4.380920in}{1.421650in}}%
\pgfpathlineto{\pgfqpoint{4.384142in}{1.424199in}}%
\pgfpathlineto{\pgfqpoint{4.386289in}{1.418501in}}%
\pgfpathlineto{\pgfqpoint{4.388437in}{1.423149in}}%
\pgfpathlineto{\pgfqpoint{4.391659in}{1.419550in}}%
\pgfpathlineto{\pgfqpoint{4.392732in}{1.420150in}}%
\pgfpathlineto{\pgfqpoint{4.394880in}{1.417901in}}%
\pgfpathlineto{\pgfqpoint{4.395954in}{1.417301in}}%
\pgfpathlineto{\pgfqpoint{4.399175in}{1.413252in}}%
\pgfpathlineto{\pgfqpoint{4.400249in}{1.404405in}}%
\pgfpathlineto{\pgfqpoint{4.402397in}{1.397357in}}%
\pgfpathlineto{\pgfqpoint{4.403471in}{1.407554in}}%
\pgfpathlineto{\pgfqpoint{4.407766in}{1.402006in}}%
\pgfpathlineto{\pgfqpoint{4.408840in}{1.406654in}}%
\pgfpathlineto{\pgfqpoint{4.409914in}{1.403055in}}%
\pgfpathlineto{\pgfqpoint{4.416357in}{1.414452in}}%
\pgfpathlineto{\pgfqpoint{4.417431in}{1.418201in}}%
\pgfpathlineto{\pgfqpoint{4.418505in}{1.413552in}}%
\pgfpathlineto{\pgfqpoint{4.422800in}{1.419250in}}%
\pgfpathlineto{\pgfqpoint{4.424948in}{1.433946in}}%
\pgfpathlineto{\pgfqpoint{4.426021in}{1.422399in}}%
\pgfpathlineto{\pgfqpoint{4.429243in}{1.422999in}}%
\pgfpathlineto{\pgfqpoint{4.432464in}{1.432896in}}%
\pgfpathlineto{\pgfqpoint{4.433538in}{1.434546in}}%
\pgfpathlineto{\pgfqpoint{4.436760in}{1.434996in}}%
\pgfpathlineto{\pgfqpoint{4.438907in}{1.433946in}}%
\pgfpathlineto{\pgfqpoint{4.439981in}{1.435895in}}%
\pgfpathlineto{\pgfqpoint{4.441055in}{1.433796in}}%
\pgfpathlineto{\pgfqpoint{4.445350in}{1.436945in}}%
\pgfpathlineto{\pgfqpoint{4.446424in}{1.441594in}}%
\pgfpathlineto{\pgfqpoint{4.447498in}{1.452240in}}%
\pgfpathlineto{\pgfqpoint{4.448572in}{1.455689in}}%
\pgfpathlineto{\pgfqpoint{4.451794in}{1.450291in}}%
\pgfpathlineto{\pgfqpoint{4.456089in}{1.428697in}}%
\pgfpathlineto{\pgfqpoint{4.459310in}{1.427348in}}%
\pgfpathlineto{\pgfqpoint{4.460384in}{1.422699in}}%
\pgfpathlineto{\pgfqpoint{4.461458in}{1.429897in}}%
\pgfpathlineto{\pgfqpoint{4.462532in}{1.441594in}}%
\pgfpathlineto{\pgfqpoint{4.463606in}{1.434696in}}%
\pgfpathlineto{\pgfqpoint{4.466827in}{1.430347in}}%
\pgfpathlineto{\pgfqpoint{4.467901in}{1.422699in}}%
\pgfpathlineto{\pgfqpoint{4.468975in}{1.424499in}}%
\pgfpathlineto{\pgfqpoint{4.470049in}{1.424649in}}%
\pgfpathlineto{\pgfqpoint{4.471123in}{1.429897in}}%
\pgfpathlineto{\pgfqpoint{4.476492in}{1.428997in}}%
\pgfpathlineto{\pgfqpoint{4.477566in}{1.434546in}}%
\pgfpathlineto{\pgfqpoint{4.478639in}{1.427648in}}%
\pgfpathlineto{\pgfqpoint{4.481861in}{1.424049in}}%
\pgfpathlineto{\pgfqpoint{4.482935in}{1.425248in}}%
\pgfpathlineto{\pgfqpoint{4.485083in}{1.406504in}}%
\pgfpathlineto{\pgfqpoint{4.486156in}{1.406804in}}%
\pgfpathlineto{\pgfqpoint{4.490452in}{1.405754in}}%
\pgfpathlineto{\pgfqpoint{4.491526in}{1.402455in}}%
\pgfpathlineto{\pgfqpoint{4.493673in}{1.393158in}}%
\pgfpathlineto{\pgfqpoint{4.496895in}{1.408754in}}%
\pgfpathlineto{\pgfqpoint{4.497969in}{1.408754in}}%
\pgfpathlineto{\pgfqpoint{4.501190in}{1.419250in}}%
\pgfpathlineto{\pgfqpoint{4.504412in}{1.414152in}}%
\pgfpathlineto{\pgfqpoint{4.505485in}{1.410553in}}%
\pgfpathlineto{\pgfqpoint{4.507633in}{1.421950in}}%
\pgfpathlineto{\pgfqpoint{4.508707in}{1.423149in}}%
\pgfpathlineto{\pgfqpoint{4.511928in}{1.422999in}}%
\pgfpathlineto{\pgfqpoint{4.513002in}{1.418201in}}%
\pgfpathlineto{\pgfqpoint{4.515150in}{1.426898in}}%
\pgfpathlineto{\pgfqpoint{4.516224in}{1.426448in}}%
\pgfpathlineto{\pgfqpoint{4.519445in}{1.422249in}}%
\pgfpathlineto{\pgfqpoint{4.521593in}{1.430647in}}%
\pgfpathlineto{\pgfqpoint{4.523741in}{1.424199in}}%
\pgfpathlineto{\pgfqpoint{4.526962in}{1.423149in}}%
\pgfpathlineto{\pgfqpoint{4.528036in}{1.420450in}}%
\pgfpathlineto{\pgfqpoint{4.529110in}{1.415052in}}%
\pgfpathlineto{\pgfqpoint{4.530184in}{1.419850in}}%
\pgfpathlineto{\pgfqpoint{4.531258in}{1.417151in}}%
\pgfpathlineto{\pgfqpoint{4.534479in}{1.416851in}}%
\pgfpathlineto{\pgfqpoint{4.536627in}{1.415801in}}%
\pgfpathlineto{\pgfqpoint{4.537701in}{1.416101in}}%
\pgfpathlineto{\pgfqpoint{4.538774in}{1.409653in}}%
\pgfpathlineto{\pgfqpoint{4.541996in}{1.413852in}}%
\pgfpathlineto{\pgfqpoint{4.544144in}{1.424799in}}%
\pgfpathlineto{\pgfqpoint{4.545217in}{1.422549in}}%
\pgfpathlineto{\pgfqpoint{4.546291in}{1.425248in}}%
\pgfpathlineto{\pgfqpoint{4.549513in}{1.429447in}}%
\pgfpathlineto{\pgfqpoint{4.550587in}{1.417451in}}%
\pgfpathlineto{\pgfqpoint{4.551661in}{1.416401in}}%
\pgfpathlineto{\pgfqpoint{4.552734in}{1.422999in}}%
\pgfpathlineto{\pgfqpoint{4.553808in}{1.419700in}}%
\pgfpathlineto{\pgfqpoint{4.557030in}{1.411753in}}%
\pgfpathlineto{\pgfqpoint{4.559177in}{1.399606in}}%
\pgfpathlineto{\pgfqpoint{4.561325in}{1.411153in}}%
\pgfpathlineto{\pgfqpoint{4.565620in}{1.415651in}}%
\pgfpathlineto{\pgfqpoint{4.567768in}{1.411753in}}%
\pgfpathlineto{\pgfqpoint{4.568842in}{1.412652in}}%
\pgfpathlineto{\pgfqpoint{4.572063in}{1.413852in}}%
\pgfpathlineto{\pgfqpoint{4.573137in}{1.421950in}}%
\pgfpathlineto{\pgfqpoint{4.574211in}{1.423149in}}%
\pgfpathlineto{\pgfqpoint{4.575285in}{1.423149in}}%
\pgfpathlineto{\pgfqpoint{4.576359in}{1.433796in}}%
\pgfpathlineto{\pgfqpoint{4.579580in}{1.433346in}}%
\pgfpathlineto{\pgfqpoint{4.580654in}{1.433796in}}%
\pgfpathlineto{\pgfqpoint{4.581728in}{1.435895in}}%
\pgfpathlineto{\pgfqpoint{4.582802in}{1.432746in}}%
\pgfpathlineto{\pgfqpoint{4.583876in}{1.431547in}}%
\pgfpathlineto{\pgfqpoint{4.587097in}{1.431097in}}%
\pgfpathlineto{\pgfqpoint{4.589245in}{1.433346in}}%
\pgfpathlineto{\pgfqpoint{4.590319in}{1.431247in}}%
\pgfpathlineto{\pgfqpoint{4.591393in}{1.431996in}}%
\pgfpathlineto{\pgfqpoint{4.595688in}{1.433196in}}%
\pgfpathlineto{\pgfqpoint{4.597836in}{1.440094in}}%
\pgfpathlineto{\pgfqpoint{4.598909in}{1.443243in}}%
\pgfpathlineto{\pgfqpoint{4.602131in}{1.446842in}}%
\pgfpathlineto{\pgfqpoint{4.604279in}{1.453890in}}%
\pgfpathlineto{\pgfqpoint{4.606426in}{1.458088in}}%
\pgfpathlineto{\pgfqpoint{4.609648in}{1.462437in}}%
\pgfpathlineto{\pgfqpoint{4.610722in}{1.468885in}}%
\pgfpathlineto{\pgfqpoint{4.611795in}{1.464237in}}%
\pgfpathlineto{\pgfqpoint{4.612869in}{1.466486in}}%
\pgfpathlineto{\pgfqpoint{4.613943in}{1.471884in}}%
\pgfpathlineto{\pgfqpoint{4.617165in}{1.470235in}}%
\pgfpathlineto{\pgfqpoint{4.618238in}{1.476383in}}%
\pgfpathlineto{\pgfqpoint{4.619312in}{1.471734in}}%
\pgfpathlineto{\pgfqpoint{4.620386in}{1.477732in}}%
\pgfpathlineto{\pgfqpoint{4.621460in}{1.472634in}}%
\pgfpathlineto{\pgfqpoint{4.625755in}{1.482981in}}%
\pgfpathlineto{\pgfqpoint{4.626829in}{1.479082in}}%
\pgfpathlineto{\pgfqpoint{4.628977in}{1.483731in}}%
\pgfpathlineto{\pgfqpoint{4.632198in}{1.485380in}}%
\pgfpathlineto{\pgfqpoint{4.633272in}{1.488979in}}%
\pgfpathlineto{\pgfqpoint{4.634346in}{1.482831in}}%
\pgfpathlineto{\pgfqpoint{4.635420in}{1.485230in}}%
\pgfpathlineto{\pgfqpoint{4.639715in}{1.485980in}}%
\pgfpathlineto{\pgfqpoint{4.640789in}{1.489129in}}%
\pgfpathlineto{\pgfqpoint{4.641863in}{1.483431in}}%
\pgfpathlineto{\pgfqpoint{4.644011in}{1.489879in}}%
\pgfpathlineto{\pgfqpoint{4.647232in}{1.494227in}}%
\pgfpathlineto{\pgfqpoint{4.648306in}{1.493328in}}%
\pgfpathlineto{\pgfqpoint{4.651527in}{1.510422in}}%
\pgfpathlineto{\pgfqpoint{4.654749in}{1.507873in}}%
\pgfpathlineto{\pgfqpoint{4.655823in}{1.505774in}}%
\pgfpathlineto{\pgfqpoint{4.656897in}{1.495427in}}%
\pgfpathlineto{\pgfqpoint{4.659044in}{1.526018in}}%
\pgfpathlineto{\pgfqpoint{4.662266in}{1.525418in}}%
\pgfpathlineto{\pgfqpoint{4.663340in}{1.526917in}}%
\pgfpathlineto{\pgfqpoint{4.664414in}{1.520919in}}%
\pgfpathlineto{\pgfqpoint{4.665487in}{1.537114in}}%
\pgfpathlineto{\pgfqpoint{4.666561in}{1.541763in}}%
\pgfpathlineto{\pgfqpoint{4.669783in}{1.539814in}}%
\pgfpathlineto{\pgfqpoint{4.670857in}{1.544912in}}%
\pgfpathlineto{\pgfqpoint{4.671930in}{1.527817in}}%
\pgfpathlineto{\pgfqpoint{4.674078in}{1.530966in}}%
\pgfpathlineto{\pgfqpoint{4.677300in}{1.523768in}}%
\pgfpathlineto{\pgfqpoint{4.678373in}{1.533665in}}%
\pgfpathlineto{\pgfqpoint{4.679447in}{1.535765in}}%
\pgfpathlineto{\pgfqpoint{4.680521in}{1.531266in}}%
\pgfpathlineto{\pgfqpoint{4.681595in}{1.533365in}}%
\pgfpathlineto{\pgfqpoint{4.684816in}{1.529467in}}%
\pgfpathlineto{\pgfqpoint{4.686964in}{1.541313in}}%
\pgfpathlineto{\pgfqpoint{4.688038in}{1.536814in}}%
\pgfpathlineto{\pgfqpoint{4.689112in}{1.538464in}}%
\pgfpathlineto{\pgfqpoint{4.692333in}{1.531566in}}%
\pgfpathlineto{\pgfqpoint{4.694481in}{1.517320in}}%
\pgfpathlineto{\pgfqpoint{4.695555in}{1.521969in}}%
\pgfpathlineto{\pgfqpoint{4.696629in}{1.516421in}}%
\pgfpathlineto{\pgfqpoint{4.700924in}{1.509973in}}%
\pgfpathlineto{\pgfqpoint{4.701998in}{1.495727in}}%
\pgfpathlineto{\pgfqpoint{4.704146in}{1.483431in}}%
\pgfpathlineto{\pgfqpoint{4.707367in}{1.485530in}}%
\pgfpathlineto{\pgfqpoint{4.708441in}{1.487479in}}%
\pgfpathlineto{\pgfqpoint{4.709515in}{1.481331in}}%
\pgfpathlineto{\pgfqpoint{4.710589in}{1.500226in}}%
\pgfpathlineto{\pgfqpoint{4.711662in}{1.503375in}}%
\pgfpathlineto{\pgfqpoint{4.714884in}{1.506674in}}%
\pgfpathlineto{\pgfqpoint{4.717032in}{1.499476in}}%
\pgfpathlineto{\pgfqpoint{4.719179in}{1.512672in}}%
\pgfpathlineto{\pgfqpoint{4.722401in}{1.508623in}}%
\pgfpathlineto{\pgfqpoint{4.723475in}{1.518220in}}%
\pgfpathlineto{\pgfqpoint{4.725622in}{1.488829in}}%
\pgfpathlineto{\pgfqpoint{4.726696in}{1.495127in}}%
\pgfpathlineto{\pgfqpoint{4.729918in}{1.490629in}}%
\pgfpathlineto{\pgfqpoint{4.730992in}{1.505624in}}%
\pgfpathlineto{\pgfqpoint{4.732065in}{1.508023in}}%
\pgfpathlineto{\pgfqpoint{4.733139in}{1.511922in}}%
\pgfpathlineto{\pgfqpoint{4.734213in}{1.504424in}}%
\pgfpathlineto{\pgfqpoint{4.737435in}{1.504724in}}%
\pgfpathlineto{\pgfqpoint{4.741730in}{1.515371in}}%
\pgfpathlineto{\pgfqpoint{4.744951in}{1.519720in}}%
\pgfpathlineto{\pgfqpoint{4.746025in}{1.517320in}}%
\pgfpathlineto{\pgfqpoint{4.747099in}{1.512522in}}%
\pgfpathlineto{\pgfqpoint{4.748173in}{1.520170in}}%
\pgfpathlineto{\pgfqpoint{4.749247in}{1.511022in}}%
\pgfpathlineto{\pgfqpoint{4.752468in}{1.505924in}}%
\pgfpathlineto{\pgfqpoint{4.754616in}{1.514771in}}%
\pgfpathlineto{\pgfqpoint{4.755690in}{1.506074in}}%
\pgfpathlineto{\pgfqpoint{4.756764in}{1.505774in}}%
\pgfpathlineto{\pgfqpoint{4.761059in}{1.510422in}}%
\pgfpathlineto{\pgfqpoint{4.762133in}{1.510722in}}%
\pgfpathlineto{\pgfqpoint{4.764281in}{1.518070in}}%
\pgfpathlineto{\pgfqpoint{4.767502in}{1.523319in}}%
\pgfpathlineto{\pgfqpoint{4.769650in}{1.500376in}}%
\pgfpathlineto{\pgfqpoint{4.770724in}{1.506524in}}%
\pgfpathlineto{\pgfqpoint{4.771797in}{1.509223in}}%
\pgfpathlineto{\pgfqpoint{4.775019in}{1.508773in}}%
\pgfpathlineto{\pgfqpoint{4.777167in}{1.505624in}}%
\pgfpathlineto{\pgfqpoint{4.779314in}{1.498576in}}%
\pgfpathlineto{\pgfqpoint{4.782536in}{1.502325in}}%
\pgfpathlineto{\pgfqpoint{4.784683in}{1.493628in}}%
\pgfpathlineto{\pgfqpoint{4.785757in}{1.489729in}}%
\pgfpathlineto{\pgfqpoint{4.786831in}{1.480132in}}%
\pgfpathlineto{\pgfqpoint{4.790053in}{1.477732in}}%
\pgfpathlineto{\pgfqpoint{4.791126in}{1.482531in}}%
\pgfpathlineto{\pgfqpoint{4.792200in}{1.475183in}}%
\pgfpathlineto{\pgfqpoint{4.793274in}{1.472334in}}%
\pgfpathlineto{\pgfqpoint{4.794348in}{1.479082in}}%
\pgfpathlineto{\pgfqpoint{4.797570in}{1.471434in}}%
\pgfpathlineto{\pgfqpoint{4.798643in}{1.471434in}}%
\pgfpathlineto{\pgfqpoint{4.799717in}{1.467086in}}%
\pgfpathlineto{\pgfqpoint{4.800791in}{1.481481in}}%
\pgfpathlineto{\pgfqpoint{4.801865in}{1.476383in}}%
\pgfpathlineto{\pgfqpoint{4.806160in}{1.460788in}}%
\pgfpathlineto{\pgfqpoint{4.807234in}{1.469485in}}%
\pgfpathlineto{\pgfqpoint{4.808308in}{1.467985in}}%
\pgfpathlineto{\pgfqpoint{4.809382in}{1.464536in}}%
\pgfpathlineto{\pgfqpoint{4.812603in}{1.459588in}}%
\pgfpathlineto{\pgfqpoint{4.813677in}{1.466036in}}%
\pgfpathlineto{\pgfqpoint{4.814751in}{1.466636in}}%
\pgfpathlineto{\pgfqpoint{4.816899in}{1.480732in}}%
\pgfpathlineto{\pgfqpoint{4.822268in}{1.491678in}}%
\pgfpathlineto{\pgfqpoint{4.823342in}{1.489279in}}%
\pgfpathlineto{\pgfqpoint{4.824415in}{1.481031in}}%
\pgfpathlineto{\pgfqpoint{4.827637in}{1.483281in}}%
\pgfpathlineto{\pgfqpoint{4.828711in}{1.475183in}}%
\pgfpathlineto{\pgfqpoint{4.829785in}{1.471434in}}%
\pgfpathlineto{\pgfqpoint{4.830859in}{1.479682in}}%
\pgfpathlineto{\pgfqpoint{4.831932in}{1.472034in}}%
\pgfpathlineto{\pgfqpoint{4.835154in}{1.468285in}}%
\pgfpathlineto{\pgfqpoint{4.836228in}{1.471884in}}%
\pgfpathlineto{\pgfqpoint{4.837302in}{1.469635in}}%
\pgfpathlineto{\pgfqpoint{4.838375in}{1.472184in}}%
\pgfpathlineto{\pgfqpoint{4.839449in}{1.473234in}}%
\pgfpathlineto{\pgfqpoint{4.842671in}{1.475333in}}%
\pgfpathlineto{\pgfqpoint{4.843745in}{1.467236in}}%
\pgfpathlineto{\pgfqpoint{4.844818in}{1.468885in}}%
\pgfpathlineto{\pgfqpoint{4.845892in}{1.476533in}}%
\pgfpathlineto{\pgfqpoint{4.846966in}{1.479232in}}%
\pgfpathlineto{\pgfqpoint{4.850188in}{1.475933in}}%
\pgfpathlineto{\pgfqpoint{4.851261in}{1.470235in}}%
\pgfpathlineto{\pgfqpoint{4.852335in}{1.481031in}}%
\pgfpathlineto{\pgfqpoint{4.854483in}{1.512972in}}%
\pgfpathlineto{\pgfqpoint{4.857704in}{1.519570in}}%
\pgfpathlineto{\pgfqpoint{4.858778in}{1.526018in}}%
\pgfpathlineto{\pgfqpoint{4.860926in}{1.517620in}}%
\pgfpathlineto{\pgfqpoint{4.862000in}{1.521069in}}%
\pgfpathlineto{\pgfqpoint{4.865221in}{1.519420in}}%
\pgfpathlineto{\pgfqpoint{4.866295in}{1.525418in}}%
\pgfpathlineto{\pgfqpoint{4.867369in}{1.518970in}}%
\pgfpathlineto{\pgfqpoint{4.869517in}{1.518520in}}%
\pgfpathlineto{\pgfqpoint{4.872738in}{1.525268in}}%
\pgfpathlineto{\pgfqpoint{4.873812in}{1.514021in}}%
\pgfpathlineto{\pgfqpoint{4.874886in}{1.519870in}}%
\pgfpathlineto{\pgfqpoint{4.875960in}{1.514771in}}%
\pgfpathlineto{\pgfqpoint{4.877034in}{1.515221in}}%
\pgfpathlineto{\pgfqpoint{4.880255in}{1.512372in}}%
\pgfpathlineto{\pgfqpoint{4.881329in}{1.514621in}}%
\pgfpathlineto{\pgfqpoint{4.882403in}{1.512672in}}%
\pgfpathlineto{\pgfqpoint{4.883477in}{1.515971in}}%
\pgfpathlineto{\pgfqpoint{4.884550in}{1.516421in}}%
\pgfpathlineto{\pgfqpoint{4.887772in}{1.521669in}}%
\pgfpathlineto{\pgfqpoint{4.888846in}{1.521819in}}%
\pgfpathlineto{\pgfqpoint{4.889920in}{1.517470in}}%
\pgfpathlineto{\pgfqpoint{4.890993in}{1.517170in}}%
\pgfpathlineto{\pgfqpoint{4.892067in}{1.515371in}}%
\pgfpathlineto{\pgfqpoint{4.895289in}{1.512972in}}%
\pgfpathlineto{\pgfqpoint{4.896363in}{1.513422in}}%
\pgfpathlineto{\pgfqpoint{4.897437in}{1.512072in}}%
\pgfpathlineto{\pgfqpoint{4.899584in}{1.508473in}}%
\pgfpathlineto{\pgfqpoint{4.902806in}{1.504274in}}%
\pgfpathlineto{\pgfqpoint{4.903880in}{1.508173in}}%
\pgfpathlineto{\pgfqpoint{4.904953in}{1.505624in}}%
\pgfpathlineto{\pgfqpoint{4.906027in}{1.500226in}}%
\pgfpathlineto{\pgfqpoint{4.907101in}{1.506824in}}%
\pgfpathlineto{\pgfqpoint{4.910323in}{1.507873in}}%
\pgfpathlineto{\pgfqpoint{4.913544in}{1.491228in}}%
\pgfpathlineto{\pgfqpoint{4.914618in}{1.487929in}}%
\pgfpathlineto{\pgfqpoint{4.917839in}{1.492728in}}%
\pgfpathlineto{\pgfqpoint{4.918913in}{1.484630in}}%
\pgfpathlineto{\pgfqpoint{4.919987in}{1.495127in}}%
\pgfpathlineto{\pgfqpoint{4.921061in}{1.494677in}}%
\pgfpathlineto{\pgfqpoint{4.922135in}{1.490629in}}%
\pgfpathlineto{\pgfqpoint{4.925356in}{1.496477in}}%
\pgfpathlineto{\pgfqpoint{4.926430in}{1.500975in}}%
\pgfpathlineto{\pgfqpoint{4.928578in}{1.503375in}}%
\pgfpathlineto{\pgfqpoint{4.933947in}{1.502925in}}%
\pgfpathlineto{\pgfqpoint{4.936095in}{1.501275in}}%
\pgfpathlineto{\pgfqpoint{4.937169in}{1.494527in}}%
\pgfpathlineto{\pgfqpoint{4.940390in}{1.497676in}}%
\pgfpathlineto{\pgfqpoint{4.941464in}{1.504724in}}%
\pgfpathlineto{\pgfqpoint{4.942538in}{1.501575in}}%
\pgfpathlineto{\pgfqpoint{4.943612in}{1.489879in}}%
\pgfpathlineto{\pgfqpoint{4.944685in}{1.492878in}}%
\pgfpathlineto{\pgfqpoint{4.947907in}{1.484480in}}%
\pgfpathlineto{\pgfqpoint{4.948981in}{1.484930in}}%
\pgfpathlineto{\pgfqpoint{4.950055in}{1.498876in}}%
\pgfpathlineto{\pgfqpoint{4.951128in}{1.501875in}}%
\pgfpathlineto{\pgfqpoint{4.956498in}{1.492278in}}%
\pgfpathlineto{\pgfqpoint{4.957571in}{1.487929in}}%
\pgfpathlineto{\pgfqpoint{4.958645in}{1.494077in}}%
\pgfpathlineto{\pgfqpoint{4.959719in}{1.491378in}}%
\pgfpathlineto{\pgfqpoint{4.962941in}{1.492128in}}%
\pgfpathlineto{\pgfqpoint{4.964014in}{1.488229in}}%
\pgfpathlineto{\pgfqpoint{4.965088in}{1.492128in}}%
\pgfpathlineto{\pgfqpoint{4.966162in}{1.491378in}}%
\pgfpathlineto{\pgfqpoint{4.967236in}{1.496027in}}%
\pgfpathlineto{\pgfqpoint{4.970458in}{1.477133in}}%
\pgfpathlineto{\pgfqpoint{4.971531in}{1.481931in}}%
\pgfpathlineto{\pgfqpoint{4.972605in}{1.480432in}}%
\pgfpathlineto{\pgfqpoint{4.973679in}{1.480282in}}%
\pgfpathlineto{\pgfqpoint{4.974753in}{1.481781in}}%
\pgfpathlineto{\pgfqpoint{4.977974in}{1.482981in}}%
\pgfpathlineto{\pgfqpoint{4.980122in}{1.488379in}}%
\pgfpathlineto{\pgfqpoint{4.981196in}{1.487629in}}%
\pgfpathlineto{\pgfqpoint{4.982270in}{1.478332in}}%
\pgfpathlineto{\pgfqpoint{4.986565in}{1.472184in}}%
\pgfpathlineto{\pgfqpoint{4.987639in}{1.478632in}}%
\pgfpathlineto{\pgfqpoint{4.988713in}{1.497077in}}%
\pgfpathlineto{\pgfqpoint{4.989787in}{1.487180in}}%
\pgfpathlineto{\pgfqpoint{4.993008in}{1.475033in}}%
\pgfpathlineto{\pgfqpoint{4.995156in}{1.476533in}}%
\pgfpathlineto{\pgfqpoint{4.996230in}{1.489879in}}%
\pgfpathlineto{\pgfqpoint{4.997303in}{1.490778in}}%
\pgfpathlineto{\pgfqpoint{5.000525in}{1.487629in}}%
\pgfpathlineto{\pgfqpoint{5.001599in}{1.494527in}}%
\pgfpathlineto{\pgfqpoint{5.002673in}{1.488529in}}%
\pgfpathlineto{\pgfqpoint{5.003747in}{1.489429in}}%
\pgfpathlineto{\pgfqpoint{5.004820in}{1.485830in}}%
\pgfpathlineto{\pgfqpoint{5.008042in}{1.484480in}}%
\pgfpathlineto{\pgfqpoint{5.011263in}{1.472784in}}%
\pgfpathlineto{\pgfqpoint{5.012337in}{1.473384in}}%
\pgfpathlineto{\pgfqpoint{5.015559in}{1.475933in}}%
\pgfpathlineto{\pgfqpoint{5.016633in}{1.480582in}}%
\pgfpathlineto{\pgfqpoint{5.017706in}{1.476533in}}%
\pgfpathlineto{\pgfqpoint{5.018780in}{1.486580in}}%
\pgfpathlineto{\pgfqpoint{5.019854in}{1.482981in}}%
\pgfpathlineto{\pgfqpoint{5.023076in}{1.484181in}}%
\pgfpathlineto{\pgfqpoint{5.024149in}{1.486580in}}%
\pgfpathlineto{\pgfqpoint{5.025223in}{1.484181in}}%
\pgfpathlineto{\pgfqpoint{5.026297in}{1.492128in}}%
\pgfpathlineto{\pgfqpoint{5.027371in}{1.488979in}}%
\pgfpathlineto{\pgfqpoint{5.030592in}{1.490778in}}%
\pgfpathlineto{\pgfqpoint{5.031666in}{1.492878in}}%
\pgfpathlineto{\pgfqpoint{5.032740in}{1.493628in}}%
\pgfpathlineto{\pgfqpoint{5.033814in}{1.497077in}}%
\pgfpathlineto{\pgfqpoint{5.034888in}{1.496177in}}%
\pgfpathlineto{\pgfqpoint{5.038109in}{1.497077in}}%
\pgfpathlineto{\pgfqpoint{5.039183in}{1.504424in}}%
\pgfpathlineto{\pgfqpoint{5.040257in}{1.501725in}}%
\pgfpathlineto{\pgfqpoint{5.042405in}{1.491078in}}%
\pgfpathlineto{\pgfqpoint{5.045626in}{1.493328in}}%
\pgfpathlineto{\pgfqpoint{5.046700in}{1.489279in}}%
\pgfpathlineto{\pgfqpoint{5.047774in}{1.491378in}}%
\pgfpathlineto{\pgfqpoint{5.048848in}{1.497227in}}%
\pgfpathlineto{\pgfqpoint{5.053143in}{1.501725in}}%
\pgfpathlineto{\pgfqpoint{5.054217in}{1.500975in}}%
\pgfpathlineto{\pgfqpoint{5.055291in}{1.495277in}}%
\pgfpathlineto{\pgfqpoint{5.056365in}{1.481631in}}%
\pgfpathlineto{\pgfqpoint{5.057438in}{1.477583in}}%
\pgfpathlineto{\pgfqpoint{5.061734in}{1.487479in}}%
\pgfpathlineto{\pgfqpoint{5.062808in}{1.487030in}}%
\pgfpathlineto{\pgfqpoint{5.063881in}{1.492728in}}%
\pgfpathlineto{\pgfqpoint{5.064955in}{1.491528in}}%
\pgfpathlineto{\pgfqpoint{5.069251in}{1.495727in}}%
\pgfpathlineto{\pgfqpoint{5.071398in}{1.507723in}}%
\pgfpathlineto{\pgfqpoint{5.072472in}{1.507723in}}%
\pgfpathlineto{\pgfqpoint{5.075694in}{1.503825in}}%
\pgfpathlineto{\pgfqpoint{5.076768in}{1.500226in}}%
\pgfpathlineto{\pgfqpoint{5.077841in}{1.502175in}}%
\pgfpathlineto{\pgfqpoint{5.078915in}{1.501575in}}%
\pgfpathlineto{\pgfqpoint{5.079989in}{1.514621in}}%
\pgfpathlineto{\pgfqpoint{5.083211in}{1.515521in}}%
\pgfpathlineto{\pgfqpoint{5.084284in}{1.509673in}}%
\pgfpathlineto{\pgfqpoint{5.086432in}{1.519120in}}%
\pgfpathlineto{\pgfqpoint{5.087506in}{1.522569in}}%
\pgfpathlineto{\pgfqpoint{5.091801in}{1.521069in}}%
\pgfpathlineto{\pgfqpoint{5.092875in}{1.522569in}}%
\pgfpathlineto{\pgfqpoint{5.093949in}{1.522269in}}%
\pgfpathlineto{\pgfqpoint{5.095023in}{1.525118in}}%
\pgfpathlineto{\pgfqpoint{5.098244in}{1.526168in}}%
\pgfpathlineto{\pgfqpoint{5.099318in}{1.517170in}}%
\pgfpathlineto{\pgfqpoint{5.100392in}{1.515521in}}%
\pgfpathlineto{\pgfqpoint{5.102540in}{1.518820in}}%
\pgfpathlineto{\pgfqpoint{5.105761in}{1.520469in}}%
\pgfpathlineto{\pgfqpoint{5.106835in}{1.520020in}}%
\pgfpathlineto{\pgfqpoint{5.107909in}{1.518220in}}%
\pgfpathlineto{\pgfqpoint{5.108983in}{1.514321in}}%
\pgfpathlineto{\pgfqpoint{5.110057in}{1.515971in}}%
\pgfpathlineto{\pgfqpoint{5.113278in}{1.517470in}}%
\pgfpathlineto{\pgfqpoint{5.114352in}{1.516271in}}%
\pgfpathlineto{\pgfqpoint{5.115426in}{1.518820in}}%
\pgfpathlineto{\pgfqpoint{5.116500in}{1.519420in}}%
\pgfpathlineto{\pgfqpoint{5.117573in}{1.518370in}}%
\pgfpathlineto{\pgfqpoint{5.120795in}{1.522419in}}%
\pgfpathlineto{\pgfqpoint{5.121869in}{1.517170in}}%
\pgfpathlineto{\pgfqpoint{5.122943in}{1.518670in}}%
\pgfpathlineto{\pgfqpoint{5.124016in}{1.516271in}}%
\pgfpathlineto{\pgfqpoint{5.125090in}{1.517620in}}%
\pgfpathlineto{\pgfqpoint{5.128312in}{1.512972in}}%
\pgfpathlineto{\pgfqpoint{5.130459in}{1.521669in}}%
\pgfpathlineto{\pgfqpoint{5.131533in}{1.522269in}}%
\pgfpathlineto{\pgfqpoint{5.135829in}{1.522719in}}%
\pgfpathlineto{\pgfqpoint{5.136903in}{1.517170in}}%
\pgfpathlineto{\pgfqpoint{5.137976in}{1.518820in}}%
\pgfpathlineto{\pgfqpoint{5.140124in}{1.535765in}}%
\pgfpathlineto{\pgfqpoint{5.143346in}{1.538314in}}%
\pgfpathlineto{\pgfqpoint{5.145493in}{1.543412in}}%
\pgfpathlineto{\pgfqpoint{5.146567in}{1.535765in}}%
\pgfpathlineto{\pgfqpoint{5.147641in}{1.541013in}}%
\pgfpathlineto{\pgfqpoint{5.150862in}{1.540413in}}%
\pgfpathlineto{\pgfqpoint{5.151936in}{1.543712in}}%
\pgfpathlineto{\pgfqpoint{5.153010in}{1.542813in}}%
\pgfpathlineto{\pgfqpoint{5.154084in}{1.544462in}}%
\pgfpathlineto{\pgfqpoint{5.155158in}{1.547311in}}%
\pgfpathlineto{\pgfqpoint{5.158379in}{1.550910in}}%
\pgfpathlineto{\pgfqpoint{5.159453in}{1.555559in}}%
\pgfpathlineto{\pgfqpoint{5.160527in}{1.553009in}}%
\pgfpathlineto{\pgfqpoint{5.161601in}{1.536964in}}%
\pgfpathlineto{\pgfqpoint{5.162675in}{1.529917in}}%
\pgfpathlineto{\pgfqpoint{5.165896in}{1.534565in}}%
\pgfpathlineto{\pgfqpoint{5.169118in}{1.515821in}}%
\pgfpathlineto{\pgfqpoint{5.170191in}{1.516421in}}%
\pgfpathlineto{\pgfqpoint{5.173413in}{1.516121in}}%
\pgfpathlineto{\pgfqpoint{5.175561in}{1.519120in}}%
\pgfpathlineto{\pgfqpoint{5.176635in}{1.520020in}}%
\pgfpathlineto{\pgfqpoint{5.177708in}{1.517620in}}%
\pgfpathlineto{\pgfqpoint{5.180930in}{1.517470in}}%
\pgfpathlineto{\pgfqpoint{5.182004in}{1.516421in}}%
\pgfpathlineto{\pgfqpoint{5.184151in}{1.518520in}}%
\pgfpathlineto{\pgfqpoint{5.185225in}{1.515671in}}%
\pgfpathlineto{\pgfqpoint{5.190594in}{1.525118in}}%
\pgfpathlineto{\pgfqpoint{5.191668in}{1.524968in}}%
\pgfpathlineto{\pgfqpoint{5.192742in}{1.529767in}}%
\pgfpathlineto{\pgfqpoint{5.197037in}{1.529167in}}%
\pgfpathlineto{\pgfqpoint{5.198111in}{1.530516in}}%
\pgfpathlineto{\pgfqpoint{5.199185in}{1.528567in}}%
\pgfpathlineto{\pgfqpoint{5.200259in}{1.531266in}}%
\pgfpathlineto{\pgfqpoint{5.203480in}{1.526318in}}%
\pgfpathlineto{\pgfqpoint{5.204554in}{1.518820in}}%
\pgfpathlineto{\pgfqpoint{5.205628in}{1.517020in}}%
\pgfpathlineto{\pgfqpoint{5.206702in}{1.520170in}}%
\pgfpathlineto{\pgfqpoint{5.207776in}{1.512822in}}%
\pgfpathlineto{\pgfqpoint{5.210997in}{1.514771in}}%
\pgfpathlineto{\pgfqpoint{5.215293in}{1.536215in}}%
\pgfpathlineto{\pgfqpoint{5.218514in}{1.534115in}}%
\pgfpathlineto{\pgfqpoint{5.219588in}{1.530966in}}%
\pgfpathlineto{\pgfqpoint{5.220662in}{1.532766in}}%
\pgfpathlineto{\pgfqpoint{5.221736in}{1.527367in}}%
\pgfpathlineto{\pgfqpoint{5.222810in}{1.529167in}}%
\pgfpathlineto{\pgfqpoint{5.226031in}{1.529017in}}%
\pgfpathlineto{\pgfqpoint{5.227105in}{1.531866in}}%
\pgfpathlineto{\pgfqpoint{5.228179in}{1.525118in}}%
\pgfpathlineto{\pgfqpoint{5.229253in}{1.523469in}}%
\pgfpathlineto{\pgfqpoint{5.230326in}{1.528567in}}%
\pgfpathlineto{\pgfqpoint{5.233548in}{1.532916in}}%
\pgfpathlineto{\pgfqpoint{5.234622in}{1.528117in}}%
\pgfpathlineto{\pgfqpoint{5.235696in}{1.537114in}}%
\pgfpathlineto{\pgfqpoint{5.236769in}{1.525718in}}%
\pgfpathlineto{\pgfqpoint{5.237843in}{1.525868in}}%
\pgfpathlineto{\pgfqpoint{5.243213in}{1.513572in}}%
\pgfpathlineto{\pgfqpoint{5.244286in}{1.510572in}}%
\pgfpathlineto{\pgfqpoint{5.245360in}{1.515371in}}%
\pgfpathlineto{\pgfqpoint{5.249656in}{1.522869in}}%
\pgfpathlineto{\pgfqpoint{5.250729in}{1.518220in}}%
\pgfpathlineto{\pgfqpoint{5.251803in}{1.517170in}}%
\pgfpathlineto{\pgfqpoint{5.252877in}{1.523768in}}%
\pgfpathlineto{\pgfqpoint{5.256099in}{1.531716in}}%
\pgfpathlineto{\pgfqpoint{5.257172in}{1.538164in}}%
\pgfpathlineto{\pgfqpoint{5.258246in}{1.536664in}}%
\pgfpathlineto{\pgfqpoint{5.259320in}{1.537414in}}%
\pgfpathlineto{\pgfqpoint{5.260394in}{1.541613in}}%
\pgfpathlineto{\pgfqpoint{5.263615in}{1.543412in}}%
\pgfpathlineto{\pgfqpoint{5.264689in}{1.542663in}}%
\pgfpathlineto{\pgfqpoint{5.265763in}{1.542813in}}%
\pgfpathlineto{\pgfqpoint{5.266837in}{1.542063in}}%
\pgfpathlineto{\pgfqpoint{5.267911in}{1.549111in}}%
\pgfpathlineto{\pgfqpoint{5.271132in}{1.547311in}}%
\pgfpathlineto{\pgfqpoint{5.272206in}{1.545962in}}%
\pgfpathlineto{\pgfqpoint{5.273280in}{1.548511in}}%
\pgfpathlineto{\pgfqpoint{5.275428in}{1.556758in}}%
\pgfpathlineto{\pgfqpoint{5.278649in}{1.555559in}}%
\pgfpathlineto{\pgfqpoint{5.279723in}{1.553459in}}%
\pgfpathlineto{\pgfqpoint{5.280797in}{1.544612in}}%
\pgfpathlineto{\pgfqpoint{5.281871in}{1.541013in}}%
\pgfpathlineto{\pgfqpoint{5.282945in}{1.541163in}}%
\pgfpathlineto{\pgfqpoint{5.287240in}{1.530666in}}%
\pgfpathlineto{\pgfqpoint{5.288314in}{1.539214in}}%
\pgfpathlineto{\pgfqpoint{5.290461in}{1.545812in}}%
\pgfpathlineto{\pgfqpoint{5.293683in}{1.539064in}}%
\pgfpathlineto{\pgfqpoint{5.294757in}{1.527967in}}%
\pgfpathlineto{\pgfqpoint{5.295831in}{1.524068in}}%
\pgfpathlineto{\pgfqpoint{5.296904in}{1.523918in}}%
\pgfpathlineto{\pgfqpoint{5.297978in}{1.521819in}}%
\pgfpathlineto{\pgfqpoint{5.301200in}{1.525418in}}%
\pgfpathlineto{\pgfqpoint{5.302274in}{1.501425in}}%
\pgfpathlineto{\pgfqpoint{5.303347in}{1.492578in}}%
\pgfpathlineto{\pgfqpoint{5.304421in}{1.494677in}}%
\pgfpathlineto{\pgfqpoint{5.305495in}{1.485230in}}%
\pgfpathlineto{\pgfqpoint{5.308717in}{1.483431in}}%
\pgfpathlineto{\pgfqpoint{5.309791in}{1.484630in}}%
\pgfpathlineto{\pgfqpoint{5.311938in}{1.503075in}}%
\pgfpathlineto{\pgfqpoint{5.313012in}{1.502625in}}%
\pgfpathlineto{\pgfqpoint{5.317307in}{1.510572in}}%
\pgfpathlineto{\pgfqpoint{5.318381in}{1.510572in}}%
\pgfpathlineto{\pgfqpoint{5.320529in}{1.512822in}}%
\pgfpathlineto{\pgfqpoint{5.323750in}{1.509073in}}%
\pgfpathlineto{\pgfqpoint{5.324824in}{1.506374in}}%
\pgfpathlineto{\pgfqpoint{5.325898in}{1.499776in}}%
\pgfpathlineto{\pgfqpoint{5.328046in}{1.502025in}}%
\pgfpathlineto{\pgfqpoint{5.331267in}{1.497227in}}%
\pgfpathlineto{\pgfqpoint{5.332341in}{1.502925in}}%
\pgfpathlineto{\pgfqpoint{5.333415in}{1.499326in}}%
\pgfpathlineto{\pgfqpoint{5.334489in}{1.511322in}}%
\pgfpathlineto{\pgfqpoint{5.335563in}{1.506224in}}%
\pgfpathlineto{\pgfqpoint{5.339858in}{1.511322in}}%
\pgfpathlineto{\pgfqpoint{5.340932in}{1.508773in}}%
\pgfpathlineto{\pgfqpoint{5.342006in}{1.510422in}}%
\pgfpathlineto{\pgfqpoint{5.343079in}{1.521219in}}%
\pgfpathlineto{\pgfqpoint{5.348449in}{1.524368in}}%
\pgfpathlineto{\pgfqpoint{5.350596in}{1.511022in}}%
\pgfpathlineto{\pgfqpoint{5.353818in}{1.508623in}}%
\pgfpathlineto{\pgfqpoint{5.355966in}{1.497526in}}%
\pgfpathlineto{\pgfqpoint{5.357039in}{1.498276in}}%
\pgfpathlineto{\pgfqpoint{5.358113in}{1.493328in}}%
\pgfpathlineto{\pgfqpoint{5.361335in}{1.508923in}}%
\pgfpathlineto{\pgfqpoint{5.362409in}{1.519270in}}%
\pgfpathlineto{\pgfqpoint{5.363482in}{1.518970in}}%
\pgfpathlineto{\pgfqpoint{5.364556in}{1.519720in}}%
\pgfpathlineto{\pgfqpoint{5.365630in}{1.537564in}}%
\pgfpathlineto{\pgfqpoint{5.368852in}{1.534715in}}%
\pgfpathlineto{\pgfqpoint{5.370999in}{1.543562in}}%
\pgfpathlineto{\pgfqpoint{5.373147in}{1.537414in}}%
\pgfpathlineto{\pgfqpoint{5.377442in}{1.535765in}}%
\pgfpathlineto{\pgfqpoint{5.378516in}{1.532916in}}%
\pgfpathlineto{\pgfqpoint{5.379590in}{1.532466in}}%
\pgfpathlineto{\pgfqpoint{5.380664in}{1.533216in}}%
\pgfpathlineto{\pgfqpoint{5.383885in}{1.530816in}}%
\pgfpathlineto{\pgfqpoint{5.384959in}{1.536215in}}%
\pgfpathlineto{\pgfqpoint{5.386033in}{1.536065in}}%
\pgfpathlineto{\pgfqpoint{5.388181in}{1.539064in}}%
\pgfpathlineto{\pgfqpoint{5.391402in}{1.539364in}}%
\pgfpathlineto{\pgfqpoint{5.392476in}{1.540263in}}%
\pgfpathlineto{\pgfqpoint{5.393550in}{1.534415in}}%
\pgfpathlineto{\pgfqpoint{5.394624in}{1.532616in}}%
\pgfpathlineto{\pgfqpoint{5.395698in}{1.524968in}}%
\pgfpathlineto{\pgfqpoint{5.398919in}{1.524218in}}%
\pgfpathlineto{\pgfqpoint{5.399993in}{1.515371in}}%
\pgfpathlineto{\pgfqpoint{5.401067in}{1.517470in}}%
\pgfpathlineto{\pgfqpoint{5.402141in}{1.530366in}}%
\pgfpathlineto{\pgfqpoint{5.403214in}{1.531566in}}%
\pgfpathlineto{\pgfqpoint{5.406436in}{1.536964in}}%
\pgfpathlineto{\pgfqpoint{5.407510in}{1.532916in}}%
\pgfpathlineto{\pgfqpoint{5.408584in}{1.540263in}}%
\pgfpathlineto{\pgfqpoint{5.409657in}{1.537264in}}%
\pgfpathlineto{\pgfqpoint{5.410731in}{1.540413in}}%
\pgfpathlineto{\pgfqpoint{5.413953in}{1.541463in}}%
\pgfpathlineto{\pgfqpoint{5.415027in}{1.538614in}}%
\pgfpathlineto{\pgfqpoint{5.416101in}{1.530366in}}%
\pgfpathlineto{\pgfqpoint{5.417174in}{1.526468in}}%
\pgfpathlineto{\pgfqpoint{5.418248in}{1.528267in}}%
\pgfpathlineto{\pgfqpoint{5.421470in}{1.535015in}}%
\pgfpathlineto{\pgfqpoint{5.422544in}{1.529017in}}%
\pgfpathlineto{\pgfqpoint{5.423617in}{1.532616in}}%
\pgfpathlineto{\pgfqpoint{5.424691in}{1.539364in}}%
\pgfpathlineto{\pgfqpoint{5.428987in}{1.541313in}}%
\pgfpathlineto{\pgfqpoint{5.430060in}{1.536964in}}%
\pgfpathlineto{\pgfqpoint{5.431134in}{1.541913in}}%
\pgfpathlineto{\pgfqpoint{5.432208in}{1.540413in}}%
\pgfpathlineto{\pgfqpoint{5.433282in}{1.543113in}}%
\pgfpathlineto{\pgfqpoint{5.436503in}{1.540863in}}%
\pgfpathlineto{\pgfqpoint{5.437577in}{1.542513in}}%
\pgfpathlineto{\pgfqpoint{5.438651in}{1.545212in}}%
\pgfpathlineto{\pgfqpoint{5.439725in}{1.543712in}}%
\pgfpathlineto{\pgfqpoint{5.440799in}{1.539064in}}%
\pgfpathlineto{\pgfqpoint{5.444020in}{1.545062in}}%
\pgfpathlineto{\pgfqpoint{5.445094in}{1.542513in}}%
\pgfpathlineto{\pgfqpoint{5.447242in}{1.553309in}}%
\pgfpathlineto{\pgfqpoint{5.448316in}{1.553009in}}%
\pgfpathlineto{\pgfqpoint{5.451537in}{1.553759in}}%
\pgfpathlineto{\pgfqpoint{5.452611in}{1.559458in}}%
\pgfpathlineto{\pgfqpoint{5.454759in}{1.558108in}}%
\pgfpathlineto{\pgfqpoint{5.455833in}{1.557808in}}%
\pgfpathlineto{\pgfqpoint{5.459054in}{1.559158in}}%
\pgfpathlineto{\pgfqpoint{5.461202in}{1.548211in}}%
\pgfpathlineto{\pgfqpoint{5.462276in}{1.549411in}}%
\pgfpathlineto{\pgfqpoint{5.463349in}{1.554509in}}%
\pgfpathlineto{\pgfqpoint{5.467645in}{1.548061in}}%
\pgfpathlineto{\pgfqpoint{5.468719in}{1.549411in}}%
\pgfpathlineto{\pgfqpoint{5.469792in}{1.552410in}}%
\pgfpathlineto{\pgfqpoint{5.470866in}{1.550160in}}%
\pgfpathlineto{\pgfqpoint{5.475162in}{1.547161in}}%
\pgfpathlineto{\pgfqpoint{5.476235in}{1.548811in}}%
\pgfpathlineto{\pgfqpoint{5.477309in}{1.551510in}}%
\pgfpathlineto{\pgfqpoint{5.478383in}{1.547911in}}%
\pgfpathlineto{\pgfqpoint{5.482679in}{1.545512in}}%
\pgfpathlineto{\pgfqpoint{5.483752in}{1.547461in}}%
\pgfpathlineto{\pgfqpoint{5.484826in}{1.547011in}}%
\pgfpathlineto{\pgfqpoint{5.485900in}{1.545812in}}%
\pgfpathlineto{\pgfqpoint{5.491269in}{1.541313in}}%
\pgfpathlineto{\pgfqpoint{5.493417in}{1.518520in}}%
\pgfpathlineto{\pgfqpoint{5.496638in}{1.521069in}}%
\pgfpathlineto{\pgfqpoint{5.497712in}{1.519870in}}%
\pgfpathlineto{\pgfqpoint{5.498786in}{1.521519in}}%
\pgfpathlineto{\pgfqpoint{5.499860in}{1.524968in}}%
\pgfpathlineto{\pgfqpoint{5.500934in}{1.518670in}}%
\pgfpathlineto{\pgfqpoint{5.504155in}{1.515671in}}%
\pgfpathlineto{\pgfqpoint{5.505229in}{1.520769in}}%
\pgfpathlineto{\pgfqpoint{5.506303in}{1.518970in}}%
\pgfpathlineto{\pgfqpoint{5.507377in}{1.525118in}}%
\pgfpathlineto{\pgfqpoint{5.508451in}{1.521369in}}%
\pgfpathlineto{\pgfqpoint{5.511672in}{1.522119in}}%
\pgfpathlineto{\pgfqpoint{5.512746in}{1.525118in}}%
\pgfpathlineto{\pgfqpoint{5.513820in}{1.519270in}}%
\pgfpathlineto{\pgfqpoint{5.515967in}{1.523319in}}%
\pgfpathlineto{\pgfqpoint{5.520263in}{1.510872in}}%
\pgfpathlineto{\pgfqpoint{5.522411in}{1.518070in}}%
\pgfpathlineto{\pgfqpoint{5.526706in}{1.516121in}}%
\pgfpathlineto{\pgfqpoint{5.527780in}{1.518670in}}%
\pgfpathlineto{\pgfqpoint{5.528854in}{1.517170in}}%
\pgfpathlineto{\pgfqpoint{5.529927in}{1.513272in}}%
\pgfpathlineto{\pgfqpoint{5.531001in}{1.522419in}}%
\pgfpathlineto{\pgfqpoint{5.534223in}{1.524968in}}%
\pgfpathlineto{\pgfqpoint{5.535297in}{1.527667in}}%
\pgfpathlineto{\pgfqpoint{5.536370in}{1.526618in}}%
\pgfpathlineto{\pgfqpoint{5.537444in}{1.532916in}}%
\pgfpathlineto{\pgfqpoint{5.538518in}{1.529917in}}%
\pgfpathlineto{\pgfqpoint{5.541740in}{1.536215in}}%
\pgfpathlineto{\pgfqpoint{5.542813in}{1.522119in}}%
\pgfpathlineto{\pgfqpoint{5.543887in}{1.515671in}}%
\pgfpathlineto{\pgfqpoint{5.544961in}{1.514321in}}%
\pgfpathlineto{\pgfqpoint{5.546035in}{1.510422in}}%
\pgfpathlineto{\pgfqpoint{5.549256in}{1.507873in}}%
\pgfpathlineto{\pgfqpoint{5.550330in}{1.508623in}}%
\pgfpathlineto{\pgfqpoint{5.551404in}{1.517020in}}%
\pgfpathlineto{\pgfqpoint{5.553552in}{1.519870in}}%
\pgfpathlineto{\pgfqpoint{5.556773in}{1.522119in}}%
\pgfpathlineto{\pgfqpoint{5.557847in}{1.518370in}}%
\pgfpathlineto{\pgfqpoint{5.559995in}{1.517920in}}%
\pgfpathlineto{\pgfqpoint{5.561069in}{1.514471in}}%
\pgfpathlineto{\pgfqpoint{5.566438in}{1.533066in}}%
\pgfpathlineto{\pgfqpoint{5.568586in}{1.528567in}}%
\pgfpathlineto{\pgfqpoint{5.571807in}{1.529017in}}%
\pgfpathlineto{\pgfqpoint{5.573955in}{1.527967in}}%
\pgfpathlineto{\pgfqpoint{5.579324in}{1.494227in}}%
\pgfpathlineto{\pgfqpoint{5.580398in}{1.478782in}}%
\pgfpathlineto{\pgfqpoint{5.582545in}{1.512372in}}%
\pgfpathlineto{\pgfqpoint{5.583619in}{1.510872in}}%
\pgfpathlineto{\pgfqpoint{5.586841in}{1.510123in}}%
\pgfpathlineto{\pgfqpoint{5.587915in}{1.496177in}}%
\pgfpathlineto{\pgfqpoint{5.590062in}{1.506524in}}%
\pgfpathlineto{\pgfqpoint{5.591136in}{1.495277in}}%
\pgfpathlineto{\pgfqpoint{5.595432in}{1.508473in}}%
\pgfpathlineto{\pgfqpoint{5.596505in}{1.502325in}}%
\pgfpathlineto{\pgfqpoint{5.597579in}{1.503225in}}%
\pgfpathlineto{\pgfqpoint{5.598653in}{1.506674in}}%
\pgfpathlineto{\pgfqpoint{5.601875in}{1.505624in}}%
\pgfpathlineto{\pgfqpoint{5.602948in}{1.514621in}}%
\pgfpathlineto{\pgfqpoint{5.604022in}{1.512372in}}%
\pgfpathlineto{\pgfqpoint{5.606170in}{1.492128in}}%
\pgfpathlineto{\pgfqpoint{5.609391in}{1.494977in}}%
\pgfpathlineto{\pgfqpoint{5.611539in}{1.484780in}}%
\pgfpathlineto{\pgfqpoint{5.613687in}{1.487779in}}%
\pgfpathlineto{\pgfqpoint{5.619056in}{1.478782in}}%
\pgfpathlineto{\pgfqpoint{5.620130in}{1.472034in}}%
\pgfpathlineto{\pgfqpoint{5.621204in}{1.470535in}}%
\pgfpathlineto{\pgfqpoint{5.624425in}{1.484930in}}%
\pgfpathlineto{\pgfqpoint{5.625499in}{1.485680in}}%
\pgfpathlineto{\pgfqpoint{5.627647in}{1.494977in}}%
\pgfpathlineto{\pgfqpoint{5.628721in}{1.494077in}}%
\pgfpathlineto{\pgfqpoint{5.633016in}{1.496627in}}%
\pgfpathlineto{\pgfqpoint{5.634090in}{1.491978in}}%
\pgfpathlineto{\pgfqpoint{5.635164in}{1.500675in}}%
\pgfpathlineto{\pgfqpoint{5.636237in}{1.500975in}}%
\pgfpathlineto{\pgfqpoint{5.639459in}{1.500975in}}%
\pgfpathlineto{\pgfqpoint{5.640533in}{1.507873in}}%
\pgfpathlineto{\pgfqpoint{5.641607in}{1.503075in}}%
\pgfpathlineto{\pgfqpoint{5.642680in}{1.516121in}}%
\pgfpathlineto{\pgfqpoint{5.643754in}{1.519420in}}%
\pgfpathlineto{\pgfqpoint{5.646976in}{1.521969in}}%
\pgfpathlineto{\pgfqpoint{5.648050in}{1.519570in}}%
\pgfpathlineto{\pgfqpoint{5.649123in}{1.523469in}}%
\pgfpathlineto{\pgfqpoint{5.650197in}{1.522569in}}%
\pgfpathlineto{\pgfqpoint{5.651271in}{1.528567in}}%
\pgfpathlineto{\pgfqpoint{5.654493in}{1.527367in}}%
\pgfpathlineto{\pgfqpoint{5.656640in}{1.519270in}}%
\pgfpathlineto{\pgfqpoint{5.657714in}{1.520020in}}%
\pgfpathlineto{\pgfqpoint{5.658788in}{1.514621in}}%
\pgfpathlineto{\pgfqpoint{5.663083in}{1.506074in}}%
\pgfpathlineto{\pgfqpoint{5.664157in}{1.508773in}}%
\pgfpathlineto{\pgfqpoint{5.666305in}{1.494977in}}%
\pgfpathlineto{\pgfqpoint{5.669526in}{1.505324in}}%
\pgfpathlineto{\pgfqpoint{5.670600in}{1.505774in}}%
\pgfpathlineto{\pgfqpoint{5.672748in}{1.514471in}}%
\pgfpathlineto{\pgfqpoint{5.673822in}{1.509673in}}%
\pgfpathlineto{\pgfqpoint{5.677043in}{1.504724in}}%
\pgfpathlineto{\pgfqpoint{5.678117in}{1.507123in}}%
\pgfpathlineto{\pgfqpoint{5.679191in}{1.503825in}}%
\pgfpathlineto{\pgfqpoint{5.681339in}{1.507723in}}%
\pgfpathlineto{\pgfqpoint{5.684560in}{1.510422in}}%
\pgfpathlineto{\pgfqpoint{5.685634in}{1.512072in}}%
\pgfpathlineto{\pgfqpoint{5.687782in}{1.499176in}}%
\pgfpathlineto{\pgfqpoint{5.688856in}{1.513721in}}%
\pgfpathlineto{\pgfqpoint{5.692077in}{1.518220in}}%
\pgfpathlineto{\pgfqpoint{5.694225in}{1.509373in}}%
\pgfpathlineto{\pgfqpoint{5.695299in}{1.508773in}}%
\pgfpathlineto{\pgfqpoint{5.696372in}{1.502475in}}%
\pgfpathlineto{\pgfqpoint{5.700668in}{1.511772in}}%
\pgfpathlineto{\pgfqpoint{5.701742in}{1.523918in}}%
\pgfpathlineto{\pgfqpoint{5.703889in}{1.511922in}}%
\pgfpathlineto{\pgfqpoint{5.707111in}{1.516121in}}%
\pgfpathlineto{\pgfqpoint{5.709258in}{1.529467in}}%
\pgfpathlineto{\pgfqpoint{5.710332in}{1.526468in}}%
\pgfpathlineto{\pgfqpoint{5.714628in}{1.526917in}}%
\pgfpathlineto{\pgfqpoint{5.715701in}{1.532766in}}%
\pgfpathlineto{\pgfqpoint{5.717849in}{1.520170in}}%
\pgfpathlineto{\pgfqpoint{5.722144in}{1.515821in}}%
\pgfpathlineto{\pgfqpoint{5.723218in}{1.523768in}}%
\pgfpathlineto{\pgfqpoint{5.726440in}{1.509673in}}%
\pgfpathlineto{\pgfqpoint{5.729661in}{1.512972in}}%
\pgfpathlineto{\pgfqpoint{5.730735in}{1.510872in}}%
\pgfpathlineto{\pgfqpoint{5.731809in}{1.500825in}}%
\pgfpathlineto{\pgfqpoint{5.732883in}{1.510123in}}%
\pgfpathlineto{\pgfqpoint{5.733957in}{1.504424in}}%
\pgfpathlineto{\pgfqpoint{5.738252in}{1.510123in}}%
\pgfpathlineto{\pgfqpoint{5.739326in}{1.504424in}}%
\pgfpathlineto{\pgfqpoint{5.741474in}{1.538014in}}%
\pgfpathlineto{\pgfqpoint{5.744695in}{1.537864in}}%
\pgfpathlineto{\pgfqpoint{5.746843in}{1.563506in}}%
\pgfpathlineto{\pgfqpoint{5.747917in}{1.563206in}}%
\pgfpathlineto{\pgfqpoint{5.748990in}{1.575503in}}%
\pgfpathlineto{\pgfqpoint{5.752212in}{1.585700in}}%
\pgfpathlineto{\pgfqpoint{5.753286in}{1.574753in}}%
\pgfpathlineto{\pgfqpoint{5.754360in}{1.583900in}}%
\pgfpathlineto{\pgfqpoint{5.755433in}{1.581351in}}%
\pgfpathlineto{\pgfqpoint{5.756507in}{1.588399in}}%
\pgfpathlineto{\pgfqpoint{5.759729in}{1.585400in}}%
\pgfpathlineto{\pgfqpoint{5.760803in}{1.577752in}}%
\pgfpathlineto{\pgfqpoint{5.761877in}{1.575653in}}%
\pgfpathlineto{\pgfqpoint{5.762950in}{1.568005in}}%
\pgfpathlineto{\pgfqpoint{5.764024in}{1.577302in}}%
\pgfpathlineto{\pgfqpoint{5.768320in}{1.578952in}}%
\pgfpathlineto{\pgfqpoint{5.769393in}{1.580001in}}%
\pgfpathlineto{\pgfqpoint{5.770467in}{1.587949in}}%
\pgfpathlineto{\pgfqpoint{5.771541in}{1.586899in}}%
\pgfpathlineto{\pgfqpoint{5.774763in}{1.589598in}}%
\pgfpathlineto{\pgfqpoint{5.775836in}{1.583900in}}%
\pgfpathlineto{\pgfqpoint{5.777984in}{1.590048in}}%
\pgfpathlineto{\pgfqpoint{5.782279in}{1.585250in}}%
\pgfpathlineto{\pgfqpoint{5.784427in}{1.603094in}}%
\pgfpathlineto{\pgfqpoint{5.785501in}{1.600095in}}%
\pgfpathlineto{\pgfqpoint{5.786575in}{1.599045in}}%
\pgfpathlineto{\pgfqpoint{5.789796in}{1.604294in}}%
\pgfpathlineto{\pgfqpoint{5.790870in}{1.607443in}}%
\pgfpathlineto{\pgfqpoint{5.791944in}{1.605943in}}%
\pgfpathlineto{\pgfqpoint{5.793018in}{1.605643in}}%
\pgfpathlineto{\pgfqpoint{5.794092in}{1.608343in}}%
\pgfpathlineto{\pgfqpoint{5.797313in}{1.608493in}}%
\pgfpathlineto{\pgfqpoint{5.798387in}{1.610142in}}%
\pgfpathlineto{\pgfqpoint{5.800535in}{1.622438in}}%
\pgfpathlineto{\pgfqpoint{5.801609in}{1.617490in}}%
\pgfpathlineto{\pgfqpoint{5.804830in}{1.620039in}}%
\pgfpathlineto{\pgfqpoint{5.806978in}{1.613141in}}%
\pgfpathlineto{\pgfqpoint{5.808052in}{1.621539in}}%
\pgfpathlineto{\pgfqpoint{5.812347in}{1.619439in}}%
\pgfpathlineto{\pgfqpoint{5.813421in}{1.627837in}}%
\pgfpathlineto{\pgfqpoint{5.814495in}{1.627687in}}%
\pgfpathlineto{\pgfqpoint{5.815568in}{1.628137in}}%
\pgfpathlineto{\pgfqpoint{5.816642in}{1.627237in}}%
\pgfpathlineto{\pgfqpoint{5.819864in}{1.632485in}}%
\pgfpathlineto{\pgfqpoint{5.820938in}{1.628287in}}%
\pgfpathlineto{\pgfqpoint{5.822011in}{1.628287in}}%
\pgfpathlineto{\pgfqpoint{5.823085in}{1.608643in}}%
\pgfpathlineto{\pgfqpoint{5.824159in}{1.610892in}}%
\pgfpathlineto{\pgfqpoint{5.827381in}{1.603544in}}%
\pgfpathlineto{\pgfqpoint{5.828455in}{1.607893in}}%
\pgfpathlineto{\pgfqpoint{5.829528in}{1.599345in}}%
\pgfpathlineto{\pgfqpoint{5.830602in}{1.600245in}}%
\pgfpathlineto{\pgfqpoint{5.831676in}{1.600095in}}%
\pgfpathlineto{\pgfqpoint{5.834898in}{1.605044in}}%
\pgfpathlineto{\pgfqpoint{5.835971in}{1.609542in}}%
\pgfpathlineto{\pgfqpoint{5.837045in}{1.605344in}}%
\pgfpathlineto{\pgfqpoint{5.838119in}{1.583000in}}%
\pgfpathlineto{\pgfqpoint{5.839193in}{1.589748in}}%
\pgfpathlineto{\pgfqpoint{5.842414in}{1.592447in}}%
\pgfpathlineto{\pgfqpoint{5.843488in}{1.588399in}}%
\pgfpathlineto{\pgfqpoint{5.844562in}{1.604594in}}%
\pgfpathlineto{\pgfqpoint{5.845636in}{1.595896in}}%
\pgfpathlineto{\pgfqpoint{5.846710in}{1.594847in}}%
\pgfpathlineto{\pgfqpoint{5.849931in}{1.599795in}}%
\pgfpathlineto{\pgfqpoint{5.851005in}{1.591398in}}%
\pgfpathlineto{\pgfqpoint{5.852079in}{1.593497in}}%
\pgfpathlineto{\pgfqpoint{5.853153in}{1.593497in}}%
\pgfpathlineto{\pgfqpoint{5.854227in}{1.597096in}}%
\pgfpathlineto{\pgfqpoint{5.857448in}{1.596646in}}%
\pgfpathlineto{\pgfqpoint{5.858522in}{1.602644in}}%
\pgfpathlineto{\pgfqpoint{5.859596in}{1.597546in}}%
\pgfpathlineto{\pgfqpoint{5.860670in}{1.601745in}}%
\pgfpathlineto{\pgfqpoint{5.861744in}{1.594847in}}%
\pgfpathlineto{\pgfqpoint{5.864965in}{1.598596in}}%
\pgfpathlineto{\pgfqpoint{5.867113in}{1.587649in}}%
\pgfpathlineto{\pgfqpoint{5.868187in}{1.577902in}}%
\pgfpathlineto{\pgfqpoint{5.869260in}{1.578202in}}%
\pgfpathlineto{\pgfqpoint{5.872482in}{1.571454in}}%
\pgfpathlineto{\pgfqpoint{5.874630in}{1.580751in}}%
\pgfpathlineto{\pgfqpoint{5.876777in}{1.590648in}}%
\pgfpathlineto{\pgfqpoint{5.881073in}{1.594247in}}%
\pgfpathlineto{\pgfqpoint{5.882146in}{1.588249in}}%
\pgfpathlineto{\pgfqpoint{5.883220in}{1.592597in}}%
\pgfpathlineto{\pgfqpoint{5.884294in}{1.594547in}}%
\pgfpathlineto{\pgfqpoint{5.887516in}{1.591848in}}%
\pgfpathlineto{\pgfqpoint{5.888589in}{1.605344in}}%
\pgfpathlineto{\pgfqpoint{5.889663in}{1.602344in}}%
\pgfpathlineto{\pgfqpoint{5.890737in}{1.607893in}}%
\pgfpathlineto{\pgfqpoint{5.891811in}{1.617190in}}%
\pgfpathlineto{\pgfqpoint{5.895032in}{1.615990in}}%
\pgfpathlineto{\pgfqpoint{5.896106in}{1.621389in}}%
\pgfpathlineto{\pgfqpoint{5.897180in}{1.619439in}}%
\pgfpathlineto{\pgfqpoint{5.899328in}{1.631585in}}%
\pgfpathlineto{\pgfqpoint{5.902549in}{1.631436in}}%
\pgfpathlineto{\pgfqpoint{5.903623in}{1.635784in}}%
\pgfpathlineto{\pgfqpoint{5.904697in}{1.634884in}}%
\pgfpathlineto{\pgfqpoint{5.905771in}{1.643132in}}%
\pgfpathlineto{\pgfqpoint{5.906845in}{1.639983in}}%
\pgfpathlineto{\pgfqpoint{5.911140in}{1.645081in}}%
\pgfpathlineto{\pgfqpoint{5.912214in}{1.648230in}}%
\pgfpathlineto{\pgfqpoint{5.914362in}{1.663376in}}%
\pgfpathlineto{\pgfqpoint{5.918657in}{1.667275in}}%
\pgfpathlineto{\pgfqpoint{5.919731in}{1.671173in}}%
\pgfpathlineto{\pgfqpoint{5.920805in}{1.659627in}}%
\pgfpathlineto{\pgfqpoint{5.921878in}{1.666375in}}%
\pgfpathlineto{\pgfqpoint{5.925100in}{1.666825in}}%
\pgfpathlineto{\pgfqpoint{5.926174in}{1.660827in}}%
\pgfpathlineto{\pgfqpoint{5.927248in}{1.667724in}}%
\pgfpathlineto{\pgfqpoint{5.928321in}{1.665625in}}%
\pgfpathlineto{\pgfqpoint{5.929395in}{1.665625in}}%
\pgfpathlineto{\pgfqpoint{5.932617in}{1.666825in}}%
\pgfpathlineto{\pgfqpoint{5.933691in}{1.663826in}}%
\pgfpathlineto{\pgfqpoint{5.934765in}{1.662776in}}%
\pgfpathlineto{\pgfqpoint{5.935838in}{1.659477in}}%
\pgfpathlineto{\pgfqpoint{5.936912in}{1.669074in}}%
\pgfpathlineto{\pgfqpoint{5.940134in}{1.666075in}}%
\pgfpathlineto{\pgfqpoint{5.941208in}{1.652129in}}%
\pgfpathlineto{\pgfqpoint{5.942281in}{1.658877in}}%
\pgfpathlineto{\pgfqpoint{5.943355in}{1.652729in}}%
\pgfpathlineto{\pgfqpoint{5.944429in}{1.659927in}}%
\pgfpathlineto{\pgfqpoint{5.947651in}{1.648080in}}%
\pgfpathlineto{\pgfqpoint{5.948724in}{1.641482in}}%
\pgfpathlineto{\pgfqpoint{5.949798in}{1.640283in}}%
\pgfpathlineto{\pgfqpoint{5.950872in}{1.640583in}}%
\pgfpathlineto{\pgfqpoint{5.951946in}{1.636834in}}%
\pgfpathlineto{\pgfqpoint{5.955167in}{1.636234in}}%
\pgfpathlineto{\pgfqpoint{5.956241in}{1.637134in}}%
\pgfpathlineto{\pgfqpoint{5.957315in}{1.639083in}}%
\pgfpathlineto{\pgfqpoint{5.958389in}{1.639683in}}%
\pgfpathlineto{\pgfqpoint{5.959463in}{1.636984in}}%
\pgfpathlineto{\pgfqpoint{5.962684in}{1.636384in}}%
\pgfpathlineto{\pgfqpoint{5.963758in}{1.625287in}}%
\pgfpathlineto{\pgfqpoint{5.964832in}{1.630686in}}%
\pgfpathlineto{\pgfqpoint{5.966980in}{1.621239in}}%
\pgfpathlineto{\pgfqpoint{5.971275in}{1.623638in}}%
\pgfpathlineto{\pgfqpoint{5.972349in}{1.622288in}}%
\pgfpathlineto{\pgfqpoint{5.973423in}{1.625137in}}%
\pgfpathlineto{\pgfqpoint{5.974497in}{1.616290in}}%
\pgfpathlineto{\pgfqpoint{5.977718in}{1.621838in}}%
\pgfpathlineto{\pgfqpoint{5.978792in}{1.618839in}}%
\pgfpathlineto{\pgfqpoint{5.979866in}{1.619739in}}%
\pgfpathlineto{\pgfqpoint{5.982013in}{1.626937in}}%
\pgfpathlineto{\pgfqpoint{5.987383in}{1.637734in}}%
\pgfpathlineto{\pgfqpoint{5.988456in}{1.636234in}}%
\pgfpathlineto{\pgfqpoint{5.989530in}{1.612991in}}%
\pgfpathlineto{\pgfqpoint{5.992752in}{1.622888in}}%
\pgfpathlineto{\pgfqpoint{5.993826in}{1.608193in}}%
\pgfpathlineto{\pgfqpoint{5.994899in}{1.608643in}}%
\pgfpathlineto{\pgfqpoint{5.995973in}{1.615091in}}%
\pgfpathlineto{\pgfqpoint{5.997047in}{1.613741in}}%
\pgfpathlineto{\pgfqpoint{6.000269in}{1.604894in}}%
\pgfpathlineto{\pgfqpoint{6.001343in}{1.605793in}}%
\pgfpathlineto{\pgfqpoint{6.003490in}{1.619889in}}%
\pgfpathlineto{\pgfqpoint{6.004564in}{1.622738in}}%
\pgfpathlineto{\pgfqpoint{6.007786in}{1.617340in}}%
\pgfpathlineto{\pgfqpoint{6.008859in}{1.621838in}}%
\pgfpathlineto{\pgfqpoint{6.009933in}{1.616140in}}%
\pgfpathlineto{\pgfqpoint{6.011007in}{1.616890in}}%
\pgfpathlineto{\pgfqpoint{6.012081in}{1.615091in}}%
\pgfpathlineto{\pgfqpoint{6.015302in}{1.613741in}}%
\pgfpathlineto{\pgfqpoint{6.017450in}{1.600245in}}%
\pgfpathlineto{\pgfqpoint{6.018524in}{1.600095in}}%
\pgfpathlineto{\pgfqpoint{6.019598in}{1.595596in}}%
\pgfpathlineto{\pgfqpoint{6.022819in}{1.599195in}}%
\pgfpathlineto{\pgfqpoint{6.023893in}{1.595297in}}%
\pgfpathlineto{\pgfqpoint{6.024967in}{1.600695in}}%
\pgfpathlineto{\pgfqpoint{6.027115in}{1.600395in}}%
\pgfpathlineto{\pgfqpoint{6.030336in}{1.602344in}}%
\pgfpathlineto{\pgfqpoint{6.031410in}{1.600245in}}%
\pgfpathlineto{\pgfqpoint{6.032484in}{1.601745in}}%
\pgfpathlineto{\pgfqpoint{6.034632in}{1.572803in}}%
\pgfpathlineto{\pgfqpoint{6.037853in}{1.572953in}}%
\pgfpathlineto{\pgfqpoint{6.040001in}{1.565306in}}%
\pgfpathlineto{\pgfqpoint{6.041075in}{1.577302in}}%
\pgfpathlineto{\pgfqpoint{6.042148in}{1.572953in}}%
\pgfpathlineto{\pgfqpoint{6.045370in}{1.571454in}}%
\pgfpathlineto{\pgfqpoint{6.046444in}{1.565756in}}%
\pgfpathlineto{\pgfqpoint{6.047518in}{1.556159in}}%
\pgfpathlineto{\pgfqpoint{6.048591in}{1.555259in}}%
\pgfpathlineto{\pgfqpoint{6.049665in}{1.557958in}}%
\pgfpathlineto{\pgfqpoint{6.053961in}{1.565606in}}%
\pgfpathlineto{\pgfqpoint{6.055034in}{1.568305in}}%
\pgfpathlineto{\pgfqpoint{6.056108in}{1.552860in}}%
\pgfpathlineto{\pgfqpoint{6.057182in}{1.552860in}}%
\pgfpathlineto{\pgfqpoint{6.060404in}{1.546112in}}%
\pgfpathlineto{\pgfqpoint{6.061477in}{1.561857in}}%
\pgfpathlineto{\pgfqpoint{6.062551in}{1.569205in}}%
\pgfpathlineto{\pgfqpoint{6.063625in}{1.568005in}}%
\pgfpathlineto{\pgfqpoint{6.064699in}{1.571154in}}%
\pgfpathlineto{\pgfqpoint{6.067920in}{1.574303in}}%
\pgfpathlineto{\pgfqpoint{6.070068in}{1.599645in}}%
\pgfpathlineto{\pgfqpoint{6.072216in}{1.605493in}}%
\pgfpathlineto{\pgfqpoint{6.075437in}{1.611492in}}%
\pgfpathlineto{\pgfqpoint{6.076511in}{1.609392in}}%
\pgfpathlineto{\pgfqpoint{6.077585in}{1.595297in}}%
\pgfpathlineto{\pgfqpoint{6.078659in}{1.594997in}}%
\pgfpathlineto{\pgfqpoint{6.079733in}{1.594097in}}%
\pgfpathlineto{\pgfqpoint{6.082954in}{1.593347in}}%
\pgfpathlineto{\pgfqpoint{6.084028in}{1.601445in}}%
\pgfpathlineto{\pgfqpoint{6.085102in}{1.614941in}}%
\pgfpathlineto{\pgfqpoint{6.086176in}{1.611642in}}%
\pgfpathlineto{\pgfqpoint{6.087250in}{1.616440in}}%
\pgfpathlineto{\pgfqpoint{6.090471in}{1.619889in}}%
\pgfpathlineto{\pgfqpoint{6.091545in}{1.627837in}}%
\pgfpathlineto{\pgfqpoint{6.092619in}{1.618240in}}%
\pgfpathlineto{\pgfqpoint{6.093693in}{1.620639in}}%
\pgfpathlineto{\pgfqpoint{6.094766in}{1.626637in}}%
\pgfpathlineto{\pgfqpoint{6.099062in}{1.637884in}}%
\pgfpathlineto{\pgfqpoint{6.100136in}{1.635934in}}%
\pgfpathlineto{\pgfqpoint{6.101209in}{1.644931in}}%
\pgfpathlineto{\pgfqpoint{6.102283in}{1.645381in}}%
\pgfpathlineto{\pgfqpoint{6.106579in}{1.644781in}}%
\pgfpathlineto{\pgfqpoint{6.107653in}{1.642232in}}%
\pgfpathlineto{\pgfqpoint{6.108726in}{1.646131in}}%
\pgfpathlineto{\pgfqpoint{6.109800in}{1.641333in}}%
\pgfpathlineto{\pgfqpoint{6.114096in}{1.657228in}}%
\pgfpathlineto{\pgfqpoint{6.115169in}{1.656478in}}%
\pgfpathlineto{\pgfqpoint{6.116243in}{1.658127in}}%
\pgfpathlineto{\pgfqpoint{6.117317in}{1.647331in}}%
\pgfpathlineto{\pgfqpoint{6.120539in}{1.639533in}}%
\pgfpathlineto{\pgfqpoint{6.121612in}{1.640583in}}%
\pgfpathlineto{\pgfqpoint{6.122686in}{1.636684in}}%
\pgfpathlineto{\pgfqpoint{6.123760in}{1.639533in}}%
\pgfpathlineto{\pgfqpoint{6.124834in}{1.637884in}}%
\pgfpathlineto{\pgfqpoint{6.129129in}{1.640433in}}%
\pgfpathlineto{\pgfqpoint{6.130203in}{1.633835in}}%
\pgfpathlineto{\pgfqpoint{6.131277in}{1.635334in}}%
\pgfpathlineto{\pgfqpoint{6.132351in}{1.640133in}}%
\pgfpathlineto{\pgfqpoint{6.135572in}{1.635934in}}%
\pgfpathlineto{\pgfqpoint{6.136646in}{1.605344in}}%
\pgfpathlineto{\pgfqpoint{6.137720in}{1.600695in}}%
\pgfpathlineto{\pgfqpoint{6.138794in}{1.591998in}}%
\pgfpathlineto{\pgfqpoint{6.139868in}{1.598296in}}%
\pgfpathlineto{\pgfqpoint{6.143089in}{1.595297in}}%
\pgfpathlineto{\pgfqpoint{6.144163in}{1.590498in}}%
\pgfpathlineto{\pgfqpoint{6.145237in}{1.582251in}}%
\pgfpathlineto{\pgfqpoint{6.146311in}{1.580751in}}%
\pgfpathlineto{\pgfqpoint{6.147385in}{1.584800in}}%
\pgfpathlineto{\pgfqpoint{6.150606in}{1.577302in}}%
\pgfpathlineto{\pgfqpoint{6.151680in}{1.577452in}}%
\pgfpathlineto{\pgfqpoint{6.154901in}{1.590048in}}%
\pgfpathlineto{\pgfqpoint{6.158123in}{1.584350in}}%
\pgfpathlineto{\pgfqpoint{6.160271in}{1.578052in}}%
\pgfpathlineto{\pgfqpoint{6.161344in}{1.583150in}}%
\pgfpathlineto{\pgfqpoint{6.162418in}{1.592897in}}%
\pgfpathlineto{\pgfqpoint{6.166714in}{1.596046in}}%
\pgfpathlineto{\pgfqpoint{6.167787in}{1.599345in}}%
\pgfpathlineto{\pgfqpoint{6.168861in}{1.607893in}}%
\pgfpathlineto{\pgfqpoint{6.169935in}{1.611792in}}%
\pgfpathlineto{\pgfqpoint{6.174231in}{1.598746in}}%
\pgfpathlineto{\pgfqpoint{6.177452in}{1.604894in}}%
\pgfpathlineto{\pgfqpoint{6.180674in}{1.604144in}}%
\pgfpathlineto{\pgfqpoint{6.181747in}{1.596196in}}%
\pgfpathlineto{\pgfqpoint{6.182821in}{1.592447in}}%
\pgfpathlineto{\pgfqpoint{6.184969in}{1.594997in}}%
\pgfpathlineto{\pgfqpoint{6.188190in}{1.596646in}}%
\pgfpathlineto{\pgfqpoint{6.189264in}{1.595147in}}%
\pgfpathlineto{\pgfqpoint{6.190338in}{1.605643in}}%
\pgfpathlineto{\pgfqpoint{6.191412in}{1.604294in}}%
\pgfpathlineto{\pgfqpoint{6.192486in}{1.608942in}}%
\pgfpathlineto{\pgfqpoint{6.196781in}{1.605793in}}%
\pgfpathlineto{\pgfqpoint{6.197855in}{1.599795in}}%
\pgfpathlineto{\pgfqpoint{6.198929in}{1.598895in}}%
\pgfpathlineto{\pgfqpoint{6.200003in}{1.599495in}}%
\pgfpathlineto{\pgfqpoint{6.203224in}{1.592148in}}%
\pgfpathlineto{\pgfqpoint{6.204298in}{1.594397in}}%
\pgfpathlineto{\pgfqpoint{6.207520in}{1.587049in}}%
\pgfpathlineto{\pgfqpoint{6.211815in}{1.594547in}}%
\pgfpathlineto{\pgfqpoint{6.212889in}{1.590648in}}%
\pgfpathlineto{\pgfqpoint{6.213963in}{1.590498in}}%
\pgfpathlineto{\pgfqpoint{6.215036in}{1.593497in}}%
\pgfpathlineto{\pgfqpoint{6.218258in}{1.591998in}}%
\pgfpathlineto{\pgfqpoint{6.220406in}{1.597096in}}%
\pgfpathlineto{\pgfqpoint{6.221479in}{1.593047in}}%
\pgfpathlineto{\pgfqpoint{6.225775in}{1.595596in}}%
\pgfpathlineto{\pgfqpoint{6.226849in}{1.601145in}}%
\pgfpathlineto{\pgfqpoint{6.227922in}{1.597396in}}%
\pgfpathlineto{\pgfqpoint{6.228996in}{1.590198in}}%
\pgfpathlineto{\pgfqpoint{6.230070in}{1.574453in}}%
\pgfpathlineto{\pgfqpoint{6.233292in}{1.571754in}}%
\pgfpathlineto{\pgfqpoint{6.234365in}{1.566955in}}%
\pgfpathlineto{\pgfqpoint{6.235439in}{1.575952in}}%
\pgfpathlineto{\pgfqpoint{6.237587in}{1.556308in}}%
\pgfpathlineto{\pgfqpoint{6.240809in}{1.555859in}}%
\pgfpathlineto{\pgfqpoint{6.241882in}{1.556308in}}%
\pgfpathlineto{\pgfqpoint{6.242956in}{1.559607in}}%
\pgfpathlineto{\pgfqpoint{6.244030in}{1.555859in}}%
\pgfpathlineto{\pgfqpoint{6.245104in}{1.566955in}}%
\pgfpathlineto{\pgfqpoint{6.248325in}{1.566056in}}%
\pgfpathlineto{\pgfqpoint{6.249399in}{1.563206in}}%
\pgfpathlineto{\pgfqpoint{6.250473in}{1.562757in}}%
\pgfpathlineto{\pgfqpoint{6.252621in}{1.555409in}}%
\pgfpathlineto{\pgfqpoint{6.256916in}{1.548211in}}%
\pgfpathlineto{\pgfqpoint{6.257990in}{1.536964in}}%
\pgfpathlineto{\pgfqpoint{6.260138in}{1.549710in}}%
\pgfpathlineto{\pgfqpoint{6.264433in}{1.550460in}}%
\pgfpathlineto{\pgfqpoint{6.265507in}{1.544612in}}%
\pgfpathlineto{\pgfqpoint{6.266581in}{1.548211in}}%
\pgfpathlineto{\pgfqpoint{6.267654in}{1.548361in}}%
\pgfpathlineto{\pgfqpoint{6.271950in}{1.560207in}}%
\pgfpathlineto{\pgfqpoint{6.273024in}{1.566205in}}%
\pgfpathlineto{\pgfqpoint{6.275171in}{1.563506in}}%
\pgfpathlineto{\pgfqpoint{6.278393in}{1.562607in}}%
\pgfpathlineto{\pgfqpoint{6.280541in}{1.564256in}}%
\pgfpathlineto{\pgfqpoint{6.281614in}{1.560057in}}%
\pgfpathlineto{\pgfqpoint{6.282688in}{1.567255in}}%
\pgfpathlineto{\pgfqpoint{6.285910in}{1.573703in}}%
\pgfpathlineto{\pgfqpoint{6.286984in}{1.563806in}}%
\pgfpathlineto{\pgfqpoint{6.288057in}{1.566955in}}%
\pgfpathlineto{\pgfqpoint{6.290205in}{1.566056in}}%
\pgfpathlineto{\pgfqpoint{6.293427in}{1.565306in}}%
\pgfpathlineto{\pgfqpoint{6.295574in}{1.549561in}}%
\pgfpathlineto{\pgfqpoint{6.296648in}{1.549710in}}%
\pgfpathlineto{\pgfqpoint{6.297722in}{1.549261in}}%
\pgfpathlineto{\pgfqpoint{6.300943in}{1.554209in}}%
\pgfpathlineto{\pgfqpoint{6.302017in}{1.541913in}}%
\pgfpathlineto{\pgfqpoint{6.303091in}{1.541913in}}%
\pgfpathlineto{\pgfqpoint{6.304165in}{1.536065in}}%
\pgfpathlineto{\pgfqpoint{6.305239in}{1.539364in}}%
\pgfpathlineto{\pgfqpoint{6.308460in}{1.544012in}}%
\pgfpathlineto{\pgfqpoint{6.310608in}{1.539214in}}%
\pgfpathlineto{\pgfqpoint{6.311682in}{1.531716in}}%
\pgfpathlineto{\pgfqpoint{6.312756in}{1.531116in}}%
\pgfpathlineto{\pgfqpoint{6.315977in}{1.527367in}}%
\pgfpathlineto{\pgfqpoint{6.317051in}{1.523019in}}%
\pgfpathlineto{\pgfqpoint{6.319199in}{1.531266in}}%
\pgfpathlineto{\pgfqpoint{6.320273in}{1.532316in}}%
\pgfpathlineto{\pgfqpoint{6.323494in}{1.533665in}}%
\pgfpathlineto{\pgfqpoint{6.324568in}{1.529467in}}%
\pgfpathlineto{\pgfqpoint{6.325642in}{1.530666in}}%
\pgfpathlineto{\pgfqpoint{6.326716in}{1.541313in}}%
\pgfpathlineto{\pgfqpoint{6.327789in}{1.541463in}}%
\pgfpathlineto{\pgfqpoint{6.331011in}{1.534265in}}%
\pgfpathlineto{\pgfqpoint{6.333159in}{1.543712in}}%
\pgfpathlineto{\pgfqpoint{6.334232in}{1.590498in}}%
\pgfpathlineto{\pgfqpoint{6.338528in}{1.598596in}}%
\pgfpathlineto{\pgfqpoint{6.339602in}{1.605344in}}%
\pgfpathlineto{\pgfqpoint{6.340675in}{1.596046in}}%
\pgfpathlineto{\pgfqpoint{6.342823in}{1.605643in}}%
\pgfpathlineto{\pgfqpoint{6.346045in}{1.604894in}}%
\pgfpathlineto{\pgfqpoint{6.349266in}{1.593347in}}%
\pgfpathlineto{\pgfqpoint{6.350340in}{1.593947in}}%
\pgfpathlineto{\pgfqpoint{6.353562in}{1.603844in}}%
\pgfpathlineto{\pgfqpoint{6.354635in}{1.599645in}}%
\pgfpathlineto{\pgfqpoint{6.355709in}{1.598746in}}%
\pgfpathlineto{\pgfqpoint{6.356783in}{1.591698in}}%
\pgfpathlineto{\pgfqpoint{6.357857in}{1.588849in}}%
\pgfpathlineto{\pgfqpoint{6.362152in}{1.597996in}}%
\pgfpathlineto{\pgfqpoint{6.363226in}{1.596946in}}%
\pgfpathlineto{\pgfqpoint{6.364300in}{1.597396in}}%
\pgfpathlineto{\pgfqpoint{6.365374in}{1.602494in}}%
\pgfpathlineto{\pgfqpoint{6.368595in}{1.601445in}}%
\pgfpathlineto{\pgfqpoint{6.369669in}{1.600095in}}%
\pgfpathlineto{\pgfqpoint{6.370743in}{1.594847in}}%
\pgfpathlineto{\pgfqpoint{6.371817in}{1.592747in}}%
\pgfpathlineto{\pgfqpoint{6.372891in}{1.591998in}}%
\pgfpathlineto{\pgfqpoint{6.377186in}{1.584350in}}%
\pgfpathlineto{\pgfqpoint{6.378260in}{1.578202in}}%
\pgfpathlineto{\pgfqpoint{6.379334in}{1.568455in}}%
\pgfpathlineto{\pgfqpoint{6.380408in}{1.567255in}}%
\pgfpathlineto{\pgfqpoint{6.383629in}{1.569804in}}%
\pgfpathlineto{\pgfqpoint{6.385777in}{1.582850in}}%
\pgfpathlineto{\pgfqpoint{6.386851in}{1.581951in}}%
\pgfpathlineto{\pgfqpoint{6.387924in}{1.591248in}}%
\pgfpathlineto{\pgfqpoint{6.391146in}{1.594397in}}%
\pgfpathlineto{\pgfqpoint{6.392220in}{1.611492in}}%
\pgfpathlineto{\pgfqpoint{6.393294in}{1.613441in}}%
\pgfpathlineto{\pgfqpoint{6.394367in}{1.606093in}}%
\pgfpathlineto{\pgfqpoint{6.395441in}{1.619139in}}%
\pgfpathlineto{\pgfqpoint{6.398663in}{1.619139in}}%
\pgfpathlineto{\pgfqpoint{6.399737in}{1.613891in}}%
\pgfpathlineto{\pgfqpoint{6.400810in}{1.613891in}}%
\pgfpathlineto{\pgfqpoint{6.401884in}{1.612541in}}%
\pgfpathlineto{\pgfqpoint{6.402958in}{1.613591in}}%
\pgfpathlineto{\pgfqpoint{6.406180in}{1.611792in}}%
\pgfpathlineto{\pgfqpoint{6.407253in}{1.618539in}}%
\pgfpathlineto{\pgfqpoint{6.408327in}{1.619139in}}%
\pgfpathlineto{\pgfqpoint{6.409401in}{1.617340in}}%
\pgfpathlineto{\pgfqpoint{6.410475in}{1.612241in}}%
\pgfpathlineto{\pgfqpoint{6.414770in}{1.616890in}}%
\pgfpathlineto{\pgfqpoint{6.415844in}{1.612991in}}%
\pgfpathlineto{\pgfqpoint{6.417992in}{1.599045in}}%
\pgfpathlineto{\pgfqpoint{6.421213in}{1.602344in}}%
\pgfpathlineto{\pgfqpoint{6.423361in}{1.609992in}}%
\pgfpathlineto{\pgfqpoint{6.425509in}{1.622288in}}%
\pgfpathlineto{\pgfqpoint{6.428730in}{1.614791in}}%
\pgfpathlineto{\pgfqpoint{6.429804in}{1.614041in}}%
\pgfpathlineto{\pgfqpoint{6.430878in}{1.609842in}}%
\pgfpathlineto{\pgfqpoint{6.431952in}{1.613441in}}%
\pgfpathlineto{\pgfqpoint{6.433026in}{1.613141in}}%
\pgfpathlineto{\pgfqpoint{6.436247in}{1.598746in}}%
\pgfpathlineto{\pgfqpoint{6.437321in}{1.599195in}}%
\pgfpathlineto{\pgfqpoint{6.438395in}{1.598746in}}%
\pgfpathlineto{\pgfqpoint{6.439469in}{1.593497in}}%
\pgfpathlineto{\pgfqpoint{6.440542in}{1.593047in}}%
\pgfpathlineto{\pgfqpoint{6.443764in}{1.566805in}}%
\pgfpathlineto{\pgfqpoint{6.444838in}{1.567255in}}%
\pgfpathlineto{\pgfqpoint{6.445912in}{1.565756in}}%
\pgfpathlineto{\pgfqpoint{6.446985in}{1.560357in}}%
\pgfpathlineto{\pgfqpoint{6.448059in}{1.557808in}}%
\pgfpathlineto{\pgfqpoint{6.451281in}{1.555859in}}%
\pgfpathlineto{\pgfqpoint{6.452355in}{1.548511in}}%
\pgfpathlineto{\pgfqpoint{6.453429in}{1.547011in}}%
\pgfpathlineto{\pgfqpoint{6.455576in}{1.565156in}}%
\pgfpathlineto{\pgfqpoint{6.458798in}{1.576102in}}%
\pgfpathlineto{\pgfqpoint{6.459872in}{1.575803in}}%
\pgfpathlineto{\pgfqpoint{6.460945in}{1.588549in}}%
\pgfpathlineto{\pgfqpoint{6.463093in}{1.587349in}}%
\pgfpathlineto{\pgfqpoint{6.466315in}{1.596346in}}%
\pgfpathlineto{\pgfqpoint{6.469536in}{1.641183in}}%
\pgfpathlineto{\pgfqpoint{6.470610in}{1.646131in}}%
\pgfpathlineto{\pgfqpoint{6.473831in}{1.652729in}}%
\pgfpathlineto{\pgfqpoint{6.474905in}{1.641482in}}%
\pgfpathlineto{\pgfqpoint{6.477053in}{1.634585in}}%
\pgfpathlineto{\pgfqpoint{6.478127in}{1.643882in}}%
\pgfpathlineto{\pgfqpoint{6.481348in}{1.654379in}}%
\pgfpathlineto{\pgfqpoint{6.482422in}{1.673123in}}%
\pgfpathlineto{\pgfqpoint{6.483496in}{1.668924in}}%
\pgfpathlineto{\pgfqpoint{6.484570in}{1.661276in}}%
\pgfpathlineto{\pgfqpoint{6.485644in}{1.665775in}}%
\pgfpathlineto{\pgfqpoint{6.488865in}{1.673873in}}%
\pgfpathlineto{\pgfqpoint{6.489939in}{1.668024in}}%
\pgfpathlineto{\pgfqpoint{6.491013in}{1.667275in}}%
\pgfpathlineto{\pgfqpoint{6.493161in}{1.673123in}}%
\pgfpathlineto{\pgfqpoint{6.497456in}{1.673423in}}%
\pgfpathlineto{\pgfqpoint{6.498530in}{1.674322in}}%
\pgfpathlineto{\pgfqpoint{6.499604in}{1.676422in}}%
\pgfpathlineto{\pgfqpoint{6.500677in}{1.669524in}}%
\pgfpathlineto{\pgfqpoint{6.500677in}{1.669524in}}%
\pgfusepath{stroke}%
\end{pgfscope}%
\begin{pgfscope}%
\pgfpathrectangle{\pgfqpoint{4.034768in}{1.293759in}}{\pgfqpoint{2.583333in}{0.400885in}}%
\pgfusepath{clip}%
\pgfsetroundcap%
\pgfsetroundjoin%
\pgfsetlinewidth{1.505625pt}%
\definecolor{currentstroke}{rgb}{0.498039,0.498039,0.498039}%
\pgfsetstrokecolor{currentstroke}%
\pgfsetdash{}{0pt}%
\pgfpathmoveto{\pgfqpoint{4.152193in}{1.353870in}}%
\pgfpathlineto{\pgfqpoint{4.153266in}{1.353870in}}%
\pgfpathlineto{\pgfqpoint{4.154340in}{1.350984in}}%
\pgfpathlineto{\pgfqpoint{4.155414in}{1.349769in}}%
\pgfpathlineto{\pgfqpoint{4.158636in}{1.349313in}}%
\pgfpathlineto{\pgfqpoint{4.159709in}{1.347191in}}%
\pgfpathlineto{\pgfqpoint{4.160783in}{1.343590in}}%
\pgfpathlineto{\pgfqpoint{4.169374in}{1.342558in}}%
\pgfpathlineto{\pgfqpoint{4.170448in}{1.342851in}}%
\pgfpathlineto{\pgfqpoint{4.173669in}{1.336828in}}%
\pgfpathlineto{\pgfqpoint{4.174743in}{1.330511in}}%
\pgfpathlineto{\pgfqpoint{4.175817in}{1.329336in}}%
\pgfpathlineto{\pgfqpoint{4.176891in}{1.325663in}}%
\pgfpathlineto{\pgfqpoint{4.177965in}{1.324341in}}%
\pgfpathlineto{\pgfqpoint{4.181186in}{1.328602in}}%
\pgfpathlineto{\pgfqpoint{4.182260in}{1.328161in}}%
\pgfpathlineto{\pgfqpoint{4.183334in}{1.326695in}}%
\pgfpathlineto{\pgfqpoint{4.185482in}{1.332115in}}%
\pgfpathlineto{\pgfqpoint{4.188703in}{1.329021in}}%
\pgfpathlineto{\pgfqpoint{4.189777in}{1.331197in}}%
\pgfpathlineto{\pgfqpoint{4.191925in}{1.331197in}}%
\pgfpathlineto{\pgfqpoint{4.192998in}{1.333690in}}%
\pgfpathlineto{\pgfqpoint{4.197294in}{1.339483in}}%
\pgfpathlineto{\pgfqpoint{4.198368in}{1.337243in}}%
\pgfpathlineto{\pgfqpoint{4.199441in}{1.339632in}}%
\pgfpathlineto{\pgfqpoint{4.200515in}{1.335303in}}%
\pgfpathlineto{\pgfqpoint{4.204811in}{1.335010in}}%
\pgfpathlineto{\pgfqpoint{4.205885in}{1.337929in}}%
\pgfpathlineto{\pgfqpoint{4.206958in}{1.337188in}}%
\pgfpathlineto{\pgfqpoint{4.213401in}{1.337484in}}%
\pgfpathlineto{\pgfqpoint{4.214475in}{1.340895in}}%
\pgfpathlineto{\pgfqpoint{4.215549in}{1.338374in}}%
\pgfpathlineto{\pgfqpoint{4.218771in}{1.334859in}}%
\pgfpathlineto{\pgfqpoint{4.220918in}{1.339926in}}%
\pgfpathlineto{\pgfqpoint{4.221992in}{1.336417in}}%
\pgfpathlineto{\pgfqpoint{4.223066in}{1.337567in}}%
\pgfpathlineto{\pgfqpoint{4.226287in}{1.339879in}}%
\pgfpathlineto{\pgfqpoint{4.227361in}{1.338289in}}%
\pgfpathlineto{\pgfqpoint{4.228435in}{1.338433in}}%
\pgfpathlineto{\pgfqpoint{4.229509in}{1.339150in}}%
\pgfpathlineto{\pgfqpoint{4.234878in}{1.338146in}}%
\pgfpathlineto{\pgfqpoint{4.237026in}{1.340716in}}%
\pgfpathlineto{\pgfqpoint{4.238100in}{1.338275in}}%
\pgfpathlineto{\pgfqpoint{4.241321in}{1.337270in}}%
\pgfpathlineto{\pgfqpoint{4.243469in}{1.326356in}}%
\pgfpathlineto{\pgfqpoint{4.244543in}{1.324489in}}%
\pgfpathlineto{\pgfqpoint{4.245617in}{1.326069in}}%
\pgfpathlineto{\pgfqpoint{4.248838in}{1.323053in}}%
\pgfpathlineto{\pgfqpoint{4.249912in}{1.323760in}}%
\pgfpathlineto{\pgfqpoint{4.250986in}{1.323051in}}%
\pgfpathlineto{\pgfqpoint{4.252060in}{1.320780in}}%
\pgfpathlineto{\pgfqpoint{4.256355in}{1.318651in}}%
\pgfpathlineto{\pgfqpoint{4.257429in}{1.311981in}}%
\pgfpathlineto{\pgfqpoint{4.258503in}{1.317800in}}%
\pgfpathlineto{\pgfqpoint{4.259576in}{1.315955in}}%
\pgfpathlineto{\pgfqpoint{4.260650in}{1.318906in}}%
\pgfpathlineto{\pgfqpoint{4.263872in}{1.320620in}}%
\pgfpathlineto{\pgfqpoint{4.264946in}{1.317479in}}%
\pgfpathlineto{\pgfqpoint{4.266019in}{1.318181in}}%
\pgfpathlineto{\pgfqpoint{4.267093in}{1.323115in}}%
\pgfpathlineto{\pgfqpoint{4.268167in}{1.317194in}}%
\pgfpathlineto{\pgfqpoint{4.271389in}{1.318835in}}%
\pgfpathlineto{\pgfqpoint{4.272462in}{1.328213in}}%
\pgfpathlineto{\pgfqpoint{4.273536in}{1.328489in}}%
\pgfpathlineto{\pgfqpoint{4.274610in}{1.335118in}}%
\pgfpathlineto{\pgfqpoint{4.275684in}{1.334289in}}%
\pgfpathlineto{\pgfqpoint{4.278906in}{1.332776in}}%
\pgfpathlineto{\pgfqpoint{4.279979in}{1.331002in}}%
\pgfpathlineto{\pgfqpoint{4.282127in}{1.330191in}}%
\pgfpathlineto{\pgfqpoint{4.283201in}{1.333963in}}%
\pgfpathlineto{\pgfqpoint{4.286422in}{1.336710in}}%
\pgfpathlineto{\pgfqpoint{4.287496in}{1.336572in}}%
\pgfpathlineto{\pgfqpoint{4.288570in}{1.333691in}}%
\pgfpathlineto{\pgfqpoint{4.289644in}{1.336572in}}%
\pgfpathlineto{\pgfqpoint{4.290718in}{1.330534in}}%
\pgfpathlineto{\pgfqpoint{4.295013in}{1.334548in}}%
\pgfpathlineto{\pgfqpoint{4.296087in}{1.336167in}}%
\pgfpathlineto{\pgfqpoint{4.297161in}{1.341062in}}%
\pgfpathlineto{\pgfqpoint{4.298235in}{1.339566in}}%
\pgfpathlineto{\pgfqpoint{4.303604in}{1.343091in}}%
\pgfpathlineto{\pgfqpoint{4.304678in}{1.344187in}}%
\pgfpathlineto{\pgfqpoint{4.305751in}{1.343639in}}%
\pgfpathlineto{\pgfqpoint{4.310047in}{1.340633in}}%
\pgfpathlineto{\pgfqpoint{4.313268in}{1.352071in}}%
\pgfpathlineto{\pgfqpoint{4.317564in}{1.356578in}}%
\pgfpathlineto{\pgfqpoint{4.318638in}{1.362391in}}%
\pgfpathlineto{\pgfqpoint{4.319711in}{1.363667in}}%
\pgfpathlineto{\pgfqpoint{4.320785in}{1.371940in}}%
\pgfpathlineto{\pgfqpoint{4.324007in}{1.370799in}}%
\pgfpathlineto{\pgfqpoint{4.325081in}{1.366826in}}%
\pgfpathlineto{\pgfqpoint{4.326154in}{1.366408in}}%
\pgfpathlineto{\pgfqpoint{4.327228in}{1.358760in}}%
\pgfpathlineto{\pgfqpoint{4.328302in}{1.360639in}}%
\pgfpathlineto{\pgfqpoint{4.332597in}{1.364024in}}%
\pgfpathlineto{\pgfqpoint{4.333671in}{1.359810in}}%
\pgfpathlineto{\pgfqpoint{4.334745in}{1.360082in}}%
\pgfpathlineto{\pgfqpoint{4.335819in}{1.366199in}}%
\pgfpathlineto{\pgfqpoint{4.339040in}{1.369189in}}%
\pgfpathlineto{\pgfqpoint{4.340114in}{1.371119in}}%
\pgfpathlineto{\pgfqpoint{4.341188in}{1.371670in}}%
\pgfpathlineto{\pgfqpoint{4.342262in}{1.373466in}}%
\pgfpathlineto{\pgfqpoint{4.343336in}{1.368768in}}%
\pgfpathlineto{\pgfqpoint{4.346557in}{1.363900in}}%
\pgfpathlineto{\pgfqpoint{4.350853in}{1.363503in}}%
\pgfpathlineto{\pgfqpoint{4.356222in}{1.368925in}}%
\pgfpathlineto{\pgfqpoint{4.357296in}{1.371176in}}%
\pgfpathlineto{\pgfqpoint{4.358370in}{1.376528in}}%
\pgfpathlineto{\pgfqpoint{4.361591in}{1.375993in}}%
\pgfpathlineto{\pgfqpoint{4.362665in}{1.379729in}}%
\pgfpathlineto{\pgfqpoint{4.363739in}{1.377461in}}%
\pgfpathlineto{\pgfqpoint{4.364813in}{1.390407in}}%
\pgfpathlineto{\pgfqpoint{4.369108in}{1.388446in}}%
\pgfpathlineto{\pgfqpoint{4.370182in}{1.382423in}}%
\pgfpathlineto{\pgfqpoint{4.371256in}{1.382003in}}%
\pgfpathlineto{\pgfqpoint{4.372329in}{1.389567in}}%
\pgfpathlineto{\pgfqpoint{4.373403in}{1.385085in}}%
\pgfpathlineto{\pgfqpoint{4.376625in}{1.384535in}}%
\pgfpathlineto{\pgfqpoint{4.378773in}{1.381939in}}%
\pgfpathlineto{\pgfqpoint{4.379846in}{1.387897in}}%
\pgfpathlineto{\pgfqpoint{4.380920in}{1.386367in}}%
\pgfpathlineto{\pgfqpoint{4.384142in}{1.388732in}}%
\pgfpathlineto{\pgfqpoint{4.385216in}{1.391096in}}%
\pgfpathlineto{\pgfqpoint{4.386289in}{1.388145in}}%
\pgfpathlineto{\pgfqpoint{4.387363in}{1.389831in}}%
\pgfpathlineto{\pgfqpoint{4.388437in}{1.387161in}}%
\pgfpathlineto{\pgfqpoint{4.393806in}{1.392468in}}%
\pgfpathlineto{\pgfqpoint{4.395954in}{1.391192in}}%
\pgfpathlineto{\pgfqpoint{4.399175in}{1.387362in}}%
\pgfpathlineto{\pgfqpoint{4.400249in}{1.378994in}}%
\pgfpathlineto{\pgfqpoint{4.402397in}{1.372327in}}%
\pgfpathlineto{\pgfqpoint{4.403471in}{1.381972in}}%
\pgfpathlineto{\pgfqpoint{4.406692in}{1.386227in}}%
\pgfpathlineto{\pgfqpoint{4.407766in}{1.385215in}}%
\pgfpathlineto{\pgfqpoint{4.409914in}{1.393167in}}%
\pgfpathlineto{\pgfqpoint{4.410988in}{1.395075in}}%
\pgfpathlineto{\pgfqpoint{4.417431in}{1.382342in}}%
\pgfpathlineto{\pgfqpoint{4.418505in}{1.386646in}}%
\pgfpathlineto{\pgfqpoint{4.421726in}{1.390184in}}%
\pgfpathlineto{\pgfqpoint{4.422800in}{1.388344in}}%
\pgfpathlineto{\pgfqpoint{4.424948in}{1.374792in}}%
\pgfpathlineto{\pgfqpoint{4.426021in}{1.384970in}}%
\pgfpathlineto{\pgfqpoint{4.429243in}{1.385523in}}%
\pgfpathlineto{\pgfqpoint{4.432464in}{1.376506in}}%
\pgfpathlineto{\pgfqpoint{4.433538in}{1.375043in}}%
\pgfpathlineto{\pgfqpoint{4.437834in}{1.373987in}}%
\pgfpathlineto{\pgfqpoint{4.438907in}{1.373724in}}%
\pgfpathlineto{\pgfqpoint{4.441055in}{1.377284in}}%
\pgfpathlineto{\pgfqpoint{4.444277in}{1.379544in}}%
\pgfpathlineto{\pgfqpoint{4.445350in}{1.379013in}}%
\pgfpathlineto{\pgfqpoint{4.446424in}{1.374900in}}%
\pgfpathlineto{\pgfqpoint{4.447498in}{1.365655in}}%
\pgfpathlineto{\pgfqpoint{4.448572in}{1.362783in}}%
\pgfpathlineto{\pgfqpoint{4.451794in}{1.367219in}}%
\pgfpathlineto{\pgfqpoint{4.456089in}{1.349106in}}%
\pgfpathlineto{\pgfqpoint{4.459310in}{1.347974in}}%
\pgfpathlineto{\pgfqpoint{4.460384in}{1.344074in}}%
\pgfpathlineto{\pgfqpoint{4.461458in}{1.350112in}}%
\pgfpathlineto{\pgfqpoint{4.462532in}{1.340301in}}%
\pgfpathlineto{\pgfqpoint{4.463606in}{1.345820in}}%
\pgfpathlineto{\pgfqpoint{4.466827in}{1.342244in}}%
\pgfpathlineto{\pgfqpoint{4.467901in}{1.335957in}}%
\pgfpathlineto{\pgfqpoint{4.468975in}{1.337436in}}%
\pgfpathlineto{\pgfqpoint{4.470049in}{1.337313in}}%
\pgfpathlineto{\pgfqpoint{4.471123in}{1.341625in}}%
\pgfpathlineto{\pgfqpoint{4.476492in}{1.342365in}}%
\pgfpathlineto{\pgfqpoint{4.478639in}{1.352628in}}%
\pgfpathlineto{\pgfqpoint{4.481861in}{1.349577in}}%
\pgfpathlineto{\pgfqpoint{4.482935in}{1.350594in}}%
\pgfpathlineto{\pgfqpoint{4.484009in}{1.361019in}}%
\pgfpathlineto{\pgfqpoint{4.485083in}{1.355271in}}%
\pgfpathlineto{\pgfqpoint{4.490452in}{1.356206in}}%
\pgfpathlineto{\pgfqpoint{4.491526in}{1.353254in}}%
\pgfpathlineto{\pgfqpoint{4.493673in}{1.344936in}}%
\pgfpathlineto{\pgfqpoint{4.496895in}{1.358890in}}%
\pgfpathlineto{\pgfqpoint{4.497969in}{1.358890in}}%
\pgfpathlineto{\pgfqpoint{4.499042in}{1.362378in}}%
\pgfpathlineto{\pgfqpoint{4.501190in}{1.356475in}}%
\pgfpathlineto{\pgfqpoint{4.504412in}{1.360912in}}%
\pgfpathlineto{\pgfqpoint{4.505485in}{1.357713in}}%
\pgfpathlineto{\pgfqpoint{4.506559in}{1.363445in}}%
\pgfpathlineto{\pgfqpoint{4.507633in}{1.359046in}}%
\pgfpathlineto{\pgfqpoint{4.508707in}{1.358001in}}%
\pgfpathlineto{\pgfqpoint{4.511928in}{1.358131in}}%
\pgfpathlineto{\pgfqpoint{4.513002in}{1.353971in}}%
\pgfpathlineto{\pgfqpoint{4.514076in}{1.358131in}}%
\pgfpathlineto{\pgfqpoint{4.515150in}{1.354751in}}%
\pgfpathlineto{\pgfqpoint{4.516224in}{1.355135in}}%
\pgfpathlineto{\pgfqpoint{4.519445in}{1.351545in}}%
\pgfpathlineto{\pgfqpoint{4.520519in}{1.355391in}}%
\pgfpathlineto{\pgfqpoint{4.521593in}{1.352058in}}%
\pgfpathlineto{\pgfqpoint{4.522667in}{1.354581in}}%
\pgfpathlineto{\pgfqpoint{4.523741in}{1.351644in}}%
\pgfpathlineto{\pgfqpoint{4.526962in}{1.350750in}}%
\pgfpathlineto{\pgfqpoint{4.528036in}{1.348451in}}%
\pgfpathlineto{\pgfqpoint{4.529110in}{1.343853in}}%
\pgfpathlineto{\pgfqpoint{4.531258in}{1.350239in}}%
\pgfpathlineto{\pgfqpoint{4.534479in}{1.349981in}}%
\pgfpathlineto{\pgfqpoint{4.536627in}{1.349077in}}%
\pgfpathlineto{\pgfqpoint{4.537701in}{1.349335in}}%
\pgfpathlineto{\pgfqpoint{4.538774in}{1.354888in}}%
\pgfpathlineto{\pgfqpoint{4.541996in}{1.358603in}}%
\pgfpathlineto{\pgfqpoint{4.544144in}{1.348919in}}%
\pgfpathlineto{\pgfqpoint{4.546291in}{1.353122in}}%
\pgfpathlineto{\pgfqpoint{4.549513in}{1.349541in}}%
\pgfpathlineto{\pgfqpoint{4.550587in}{1.359599in}}%
\pgfpathlineto{\pgfqpoint{4.551661in}{1.358675in}}%
\pgfpathlineto{\pgfqpoint{4.553808in}{1.367385in}}%
\pgfpathlineto{\pgfqpoint{4.557030in}{1.360295in}}%
\pgfpathlineto{\pgfqpoint{4.559177in}{1.349460in}}%
\pgfpathlineto{\pgfqpoint{4.560251in}{1.355747in}}%
\pgfpathlineto{\pgfqpoint{4.561325in}{1.351734in}}%
\pgfpathlineto{\pgfqpoint{4.565620in}{1.347797in}}%
\pgfpathlineto{\pgfqpoint{4.566694in}{1.349213in}}%
\pgfpathlineto{\pgfqpoint{4.567768in}{1.347268in}}%
\pgfpathlineto{\pgfqpoint{4.568842in}{1.348046in}}%
\pgfpathlineto{\pgfqpoint{4.572063in}{1.347009in}}%
\pgfpathlineto{\pgfqpoint{4.573137in}{1.340041in}}%
\pgfpathlineto{\pgfqpoint{4.574211in}{1.339044in}}%
\pgfpathlineto{\pgfqpoint{4.575285in}{1.339044in}}%
\pgfpathlineto{\pgfqpoint{4.576359in}{1.347856in}}%
\pgfpathlineto{\pgfqpoint{4.580654in}{1.348602in}}%
\pgfpathlineto{\pgfqpoint{4.581728in}{1.346861in}}%
\pgfpathlineto{\pgfqpoint{4.582802in}{1.349450in}}%
\pgfpathlineto{\pgfqpoint{4.583876in}{1.348451in}}%
\pgfpathlineto{\pgfqpoint{4.587097in}{1.348077in}}%
\pgfpathlineto{\pgfqpoint{4.588171in}{1.349200in}}%
\pgfpathlineto{\pgfqpoint{4.589245in}{1.348451in}}%
\pgfpathlineto{\pgfqpoint{4.590319in}{1.350193in}}%
\pgfpathlineto{\pgfqpoint{4.591393in}{1.350820in}}%
\pgfpathlineto{\pgfqpoint{4.595688in}{1.349817in}}%
\pgfpathlineto{\pgfqpoint{4.597836in}{1.344108in}}%
\pgfpathlineto{\pgfqpoint{4.598909in}{1.341557in}}%
\pgfpathlineto{\pgfqpoint{4.602131in}{1.338678in}}%
\pgfpathlineto{\pgfqpoint{4.604279in}{1.333154in}}%
\pgfpathlineto{\pgfqpoint{4.606426in}{1.329945in}}%
\pgfpathlineto{\pgfqpoint{4.609648in}{1.326662in}}%
\pgfpathlineto{\pgfqpoint{4.610722in}{1.321873in}}%
\pgfpathlineto{\pgfqpoint{4.612869in}{1.326901in}}%
\pgfpathlineto{\pgfqpoint{4.613943in}{1.322920in}}%
\pgfpathlineto{\pgfqpoint{4.617165in}{1.324112in}}%
\pgfpathlineto{\pgfqpoint{4.621460in}{1.340169in}}%
\pgfpathlineto{\pgfqpoint{4.624682in}{1.346272in}}%
\pgfpathlineto{\pgfqpoint{4.625755in}{1.344576in}}%
\pgfpathlineto{\pgfqpoint{4.627903in}{1.349196in}}%
\pgfpathlineto{\pgfqpoint{4.628977in}{1.347377in}}%
\pgfpathlineto{\pgfqpoint{4.632198in}{1.346137in}}%
\pgfpathlineto{\pgfqpoint{4.633272in}{1.343449in}}%
\pgfpathlineto{\pgfqpoint{4.634346in}{1.347982in}}%
\pgfpathlineto{\pgfqpoint{4.635420in}{1.349791in}}%
\pgfpathlineto{\pgfqpoint{4.639715in}{1.349226in}}%
\pgfpathlineto{\pgfqpoint{4.640789in}{1.346858in}}%
\pgfpathlineto{\pgfqpoint{4.642937in}{1.353710in}}%
\pgfpathlineto{\pgfqpoint{4.644011in}{1.351435in}}%
\pgfpathlineto{\pgfqpoint{4.647232in}{1.348171in}}%
\pgfpathlineto{\pgfqpoint{4.648306in}{1.348836in}}%
\pgfpathlineto{\pgfqpoint{4.649380in}{1.353060in}}%
\pgfpathlineto{\pgfqpoint{4.651527in}{1.344689in}}%
\pgfpathlineto{\pgfqpoint{4.654749in}{1.346504in}}%
\pgfpathlineto{\pgfqpoint{4.655823in}{1.344996in}}%
\pgfpathlineto{\pgfqpoint{4.656897in}{1.337563in}}%
\pgfpathlineto{\pgfqpoint{4.657971in}{1.348659in}}%
\pgfpathlineto{\pgfqpoint{4.659044in}{1.337779in}}%
\pgfpathlineto{\pgfqpoint{4.662266in}{1.338187in}}%
\pgfpathlineto{\pgfqpoint{4.663340in}{1.339212in}}%
\pgfpathlineto{\pgfqpoint{4.664414in}{1.343308in}}%
\pgfpathlineto{\pgfqpoint{4.665487in}{1.354595in}}%
\pgfpathlineto{\pgfqpoint{4.666561in}{1.351355in}}%
\pgfpathlineto{\pgfqpoint{4.669783in}{1.352693in}}%
\pgfpathlineto{\pgfqpoint{4.670857in}{1.356214in}}%
\pgfpathlineto{\pgfqpoint{4.671930in}{1.368021in}}%
\pgfpathlineto{\pgfqpoint{4.673004in}{1.369116in}}%
\pgfpathlineto{\pgfqpoint{4.674078in}{1.367911in}}%
\pgfpathlineto{\pgfqpoint{4.677300in}{1.373140in}}%
\pgfpathlineto{\pgfqpoint{4.678373in}{1.380505in}}%
\pgfpathlineto{\pgfqpoint{4.679447in}{1.378943in}}%
\pgfpathlineto{\pgfqpoint{4.681595in}{1.383842in}}%
\pgfpathlineto{\pgfqpoint{4.684816in}{1.386767in}}%
\pgfpathlineto{\pgfqpoint{4.685890in}{1.392124in}}%
\pgfpathlineto{\pgfqpoint{4.686964in}{1.388477in}}%
\pgfpathlineto{\pgfqpoint{4.688038in}{1.391842in}}%
\pgfpathlineto{\pgfqpoint{4.689112in}{1.393094in}}%
\pgfpathlineto{\pgfqpoint{4.692333in}{1.398330in}}%
\pgfpathlineto{\pgfqpoint{4.694481in}{1.387266in}}%
\pgfpathlineto{\pgfqpoint{4.696629in}{1.395186in}}%
\pgfpathlineto{\pgfqpoint{4.700924in}{1.390082in}}%
\pgfpathlineto{\pgfqpoint{4.701998in}{1.378806in}}%
\pgfpathlineto{\pgfqpoint{4.704146in}{1.369073in}}%
\pgfpathlineto{\pgfqpoint{4.707367in}{1.370735in}}%
\pgfpathlineto{\pgfqpoint{4.708441in}{1.369192in}}%
\pgfpathlineto{\pgfqpoint{4.709515in}{1.374024in}}%
\pgfpathlineto{\pgfqpoint{4.710589in}{1.389208in}}%
\pgfpathlineto{\pgfqpoint{4.711662in}{1.386677in}}%
\pgfpathlineto{\pgfqpoint{4.714884in}{1.384055in}}%
\pgfpathlineto{\pgfqpoint{4.715958in}{1.387236in}}%
\pgfpathlineto{\pgfqpoint{4.717032in}{1.384727in}}%
\pgfpathlineto{\pgfqpoint{4.718105in}{1.391418in}}%
\pgfpathlineto{\pgfqpoint{4.719179in}{1.387594in}}%
\pgfpathlineto{\pgfqpoint{4.722401in}{1.390767in}}%
\pgfpathlineto{\pgfqpoint{4.723475in}{1.398392in}}%
\pgfpathlineto{\pgfqpoint{4.724549in}{1.411975in}}%
\pgfpathlineto{\pgfqpoint{4.725622in}{1.401615in}}%
\pgfpathlineto{\pgfqpoint{4.726696in}{1.406922in}}%
\pgfpathlineto{\pgfqpoint{4.729918in}{1.410712in}}%
\pgfpathlineto{\pgfqpoint{4.730992in}{1.423551in}}%
\pgfpathlineto{\pgfqpoint{4.732065in}{1.421497in}}%
\pgfpathlineto{\pgfqpoint{4.733139in}{1.418187in}}%
\pgfpathlineto{\pgfqpoint{4.734213in}{1.424466in}}%
\pgfpathlineto{\pgfqpoint{4.737435in}{1.424724in}}%
\pgfpathlineto{\pgfqpoint{4.738508in}{1.427302in}}%
\pgfpathlineto{\pgfqpoint{4.739582in}{1.423564in}}%
\pgfpathlineto{\pgfqpoint{4.744951in}{1.417131in}}%
\pgfpathlineto{\pgfqpoint{4.746025in}{1.419109in}}%
\pgfpathlineto{\pgfqpoint{4.747099in}{1.415120in}}%
\pgfpathlineto{\pgfqpoint{4.749247in}{1.429083in}}%
\pgfpathlineto{\pgfqpoint{4.752468in}{1.424709in}}%
\pgfpathlineto{\pgfqpoint{4.753542in}{1.427796in}}%
\pgfpathlineto{\pgfqpoint{4.754616in}{1.423295in}}%
\pgfpathlineto{\pgfqpoint{4.755690in}{1.430620in}}%
\pgfpathlineto{\pgfqpoint{4.756764in}{1.430360in}}%
\pgfpathlineto{\pgfqpoint{4.759985in}{1.433483in}}%
\pgfpathlineto{\pgfqpoint{4.762133in}{1.432313in}}%
\pgfpathlineto{\pgfqpoint{4.763207in}{1.435551in}}%
\pgfpathlineto{\pgfqpoint{4.764281in}{1.432442in}}%
\pgfpathlineto{\pgfqpoint{4.767502in}{1.427964in}}%
\pgfpathlineto{\pgfqpoint{4.768576in}{1.438142in}}%
\pgfpathlineto{\pgfqpoint{4.769650in}{1.428714in}}%
\pgfpathlineto{\pgfqpoint{4.770724in}{1.434083in}}%
\pgfpathlineto{\pgfqpoint{4.771797in}{1.431726in}}%
\pgfpathlineto{\pgfqpoint{4.775019in}{1.432115in}}%
\pgfpathlineto{\pgfqpoint{4.777167in}{1.429387in}}%
\pgfpathlineto{\pgfqpoint{4.779314in}{1.423280in}}%
\pgfpathlineto{\pgfqpoint{4.782536in}{1.426528in}}%
\pgfpathlineto{\pgfqpoint{4.783610in}{1.431465in}}%
\pgfpathlineto{\pgfqpoint{4.785757in}{1.425368in}}%
\pgfpathlineto{\pgfqpoint{4.786831in}{1.416884in}}%
\pgfpathlineto{\pgfqpoint{4.790053in}{1.414763in}}%
\pgfpathlineto{\pgfqpoint{4.792200in}{1.425500in}}%
\pgfpathlineto{\pgfqpoint{4.793274in}{1.422913in}}%
\pgfpathlineto{\pgfqpoint{4.794348in}{1.429040in}}%
\pgfpathlineto{\pgfqpoint{4.797570in}{1.435985in}}%
\pgfpathlineto{\pgfqpoint{4.798643in}{1.435985in}}%
\pgfpathlineto{\pgfqpoint{4.799717in}{1.431924in}}%
\pgfpathlineto{\pgfqpoint{4.800791in}{1.445368in}}%
\pgfpathlineto{\pgfqpoint{4.801865in}{1.450130in}}%
\pgfpathlineto{\pgfqpoint{4.806160in}{1.435290in}}%
\pgfpathlineto{\pgfqpoint{4.807234in}{1.443566in}}%
\pgfpathlineto{\pgfqpoint{4.808308in}{1.444993in}}%
\pgfpathlineto{\pgfqpoint{4.809382in}{1.441693in}}%
\pgfpathlineto{\pgfqpoint{4.812603in}{1.436958in}}%
\pgfpathlineto{\pgfqpoint{4.813677in}{1.443128in}}%
\pgfpathlineto{\pgfqpoint{4.814751in}{1.442554in}}%
\pgfpathlineto{\pgfqpoint{4.815825in}{1.433534in}}%
\pgfpathlineto{\pgfqpoint{4.816899in}{1.429251in}}%
\pgfpathlineto{\pgfqpoint{4.822268in}{1.419450in}}%
\pgfpathlineto{\pgfqpoint{4.823342in}{1.421539in}}%
\pgfpathlineto{\pgfqpoint{4.824415in}{1.414299in}}%
\pgfpathlineto{\pgfqpoint{4.827637in}{1.416273in}}%
\pgfpathlineto{\pgfqpoint{4.828711in}{1.423381in}}%
\pgfpathlineto{\pgfqpoint{4.829785in}{1.419992in}}%
\pgfpathlineto{\pgfqpoint{4.831932in}{1.434363in}}%
\pgfpathlineto{\pgfqpoint{4.835154in}{1.430877in}}%
\pgfpathlineto{\pgfqpoint{4.837302in}{1.436315in}}%
\pgfpathlineto{\pgfqpoint{4.838375in}{1.438706in}}%
\pgfpathlineto{\pgfqpoint{4.839449in}{1.437721in}}%
\pgfpathlineto{\pgfqpoint{4.842671in}{1.435760in}}%
\pgfpathlineto{\pgfqpoint{4.843745in}{1.443265in}}%
\pgfpathlineto{\pgfqpoint{4.844818in}{1.444840in}}%
\pgfpathlineto{\pgfqpoint{4.845892in}{1.437537in}}%
\pgfpathlineto{\pgfqpoint{4.846966in}{1.435032in}}%
\pgfpathlineto{\pgfqpoint{4.850188in}{1.438063in}}%
\pgfpathlineto{\pgfqpoint{4.851261in}{1.432764in}}%
\pgfpathlineto{\pgfqpoint{4.852335in}{1.442805in}}%
\pgfpathlineto{\pgfqpoint{4.854483in}{1.413908in}}%
\pgfpathlineto{\pgfqpoint{4.857704in}{1.408441in}}%
\pgfpathlineto{\pgfqpoint{4.858778in}{1.403219in}}%
\pgfpathlineto{\pgfqpoint{4.859852in}{1.406189in}}%
\pgfpathlineto{\pgfqpoint{4.860926in}{1.402459in}}%
\pgfpathlineto{\pgfqpoint{4.862000in}{1.405226in}}%
\pgfpathlineto{\pgfqpoint{4.865221in}{1.406550in}}%
\pgfpathlineto{\pgfqpoint{4.867369in}{1.416593in}}%
\pgfpathlineto{\pgfqpoint{4.869517in}{1.416222in}}%
\pgfpathlineto{\pgfqpoint{4.872738in}{1.421787in}}%
\pgfpathlineto{\pgfqpoint{4.874886in}{1.436072in}}%
\pgfpathlineto{\pgfqpoint{4.875960in}{1.440441in}}%
\pgfpathlineto{\pgfqpoint{4.877034in}{1.440833in}}%
\pgfpathlineto{\pgfqpoint{4.880255in}{1.443317in}}%
\pgfpathlineto{\pgfqpoint{4.883477in}{1.449937in}}%
\pgfpathlineto{\pgfqpoint{4.884550in}{1.449539in}}%
\pgfpathlineto{\pgfqpoint{4.887772in}{1.444894in}}%
\pgfpathlineto{\pgfqpoint{4.888846in}{1.444764in}}%
\pgfpathlineto{\pgfqpoint{4.889920in}{1.440986in}}%
\pgfpathlineto{\pgfqpoint{4.890993in}{1.440725in}}%
\pgfpathlineto{\pgfqpoint{4.892067in}{1.439162in}}%
\pgfpathlineto{\pgfqpoint{4.895289in}{1.437078in}}%
\pgfpathlineto{\pgfqpoint{4.896363in}{1.437469in}}%
\pgfpathlineto{\pgfqpoint{4.897437in}{1.438641in}}%
\pgfpathlineto{\pgfqpoint{4.899584in}{1.435500in}}%
\pgfpathlineto{\pgfqpoint{4.902806in}{1.431835in}}%
\pgfpathlineto{\pgfqpoint{4.904953in}{1.437463in}}%
\pgfpathlineto{\pgfqpoint{4.906027in}{1.432709in}}%
\pgfpathlineto{\pgfqpoint{4.907101in}{1.438519in}}%
\pgfpathlineto{\pgfqpoint{4.910323in}{1.437595in}}%
\pgfpathlineto{\pgfqpoint{4.911396in}{1.442858in}}%
\pgfpathlineto{\pgfqpoint{4.913544in}{1.433319in}}%
\pgfpathlineto{\pgfqpoint{4.914618in}{1.430363in}}%
\pgfpathlineto{\pgfqpoint{4.917839in}{1.434663in}}%
\pgfpathlineto{\pgfqpoint{4.919987in}{1.451598in}}%
\pgfpathlineto{\pgfqpoint{4.921061in}{1.452013in}}%
\pgfpathlineto{\pgfqpoint{4.922135in}{1.448273in}}%
\pgfpathlineto{\pgfqpoint{4.925356in}{1.453676in}}%
\pgfpathlineto{\pgfqpoint{4.926430in}{1.449520in}}%
\pgfpathlineto{\pgfqpoint{4.928578in}{1.447339in}}%
\pgfpathlineto{\pgfqpoint{4.932873in}{1.447339in}}%
\pgfpathlineto{\pgfqpoint{4.933947in}{1.447203in}}%
\pgfpathlineto{\pgfqpoint{4.936095in}{1.445716in}}%
\pgfpathlineto{\pgfqpoint{4.937169in}{1.439630in}}%
\pgfpathlineto{\pgfqpoint{4.940390in}{1.442470in}}%
\pgfpathlineto{\pgfqpoint{4.941464in}{1.436113in}}%
\pgfpathlineto{\pgfqpoint{4.942538in}{1.438883in}}%
\pgfpathlineto{\pgfqpoint{4.943612in}{1.428481in}}%
\pgfpathlineto{\pgfqpoint{4.947907in}{1.438616in}}%
\pgfpathlineto{\pgfqpoint{4.948981in}{1.439029in}}%
\pgfpathlineto{\pgfqpoint{4.950055in}{1.426247in}}%
\pgfpathlineto{\pgfqpoint{4.951128in}{1.423634in}}%
\pgfpathlineto{\pgfqpoint{4.952202in}{1.425444in}}%
\pgfpathlineto{\pgfqpoint{4.956498in}{1.418933in}}%
\pgfpathlineto{\pgfqpoint{4.957571in}{1.415156in}}%
\pgfpathlineto{\pgfqpoint{4.958645in}{1.420495in}}%
\pgfpathlineto{\pgfqpoint{4.959719in}{1.422839in}}%
\pgfpathlineto{\pgfqpoint{4.962941in}{1.423497in}}%
\pgfpathlineto{\pgfqpoint{4.965088in}{1.430382in}}%
\pgfpathlineto{\pgfqpoint{4.966162in}{1.431049in}}%
\pgfpathlineto{\pgfqpoint{4.967236in}{1.435193in}}%
\pgfpathlineto{\pgfqpoint{4.971531in}{1.456616in}}%
\pgfpathlineto{\pgfqpoint{4.972605in}{1.458046in}}%
\pgfpathlineto{\pgfqpoint{4.973679in}{1.457902in}}%
\pgfpathlineto{\pgfqpoint{4.974753in}{1.459340in}}%
\pgfpathlineto{\pgfqpoint{4.977974in}{1.458190in}}%
\pgfpathlineto{\pgfqpoint{4.980122in}{1.453058in}}%
\pgfpathlineto{\pgfqpoint{4.981196in}{1.453760in}}%
\pgfpathlineto{\pgfqpoint{4.982270in}{1.445032in}}%
\pgfpathlineto{\pgfqpoint{4.986565in}{1.439261in}}%
\pgfpathlineto{\pgfqpoint{4.987639in}{1.445314in}}%
\pgfpathlineto{\pgfqpoint{4.988713in}{1.428000in}}%
\pgfpathlineto{\pgfqpoint{4.989787in}{1.436681in}}%
\pgfpathlineto{\pgfqpoint{4.993008in}{1.425645in}}%
\pgfpathlineto{\pgfqpoint{4.994082in}{1.426326in}}%
\pgfpathlineto{\pgfqpoint{4.995156in}{1.425645in}}%
\pgfpathlineto{\pgfqpoint{4.996230in}{1.413553in}}%
\pgfpathlineto{\pgfqpoint{4.997303in}{1.412777in}}%
\pgfpathlineto{\pgfqpoint{5.000525in}{1.415484in}}%
\pgfpathlineto{\pgfqpoint{5.002673in}{1.426697in}}%
\pgfpathlineto{\pgfqpoint{5.003747in}{1.427496in}}%
\pgfpathlineto{\pgfqpoint{5.004820in}{1.430693in}}%
\pgfpathlineto{\pgfqpoint{5.008042in}{1.429479in}}%
\pgfpathlineto{\pgfqpoint{5.011263in}{1.418953in}}%
\pgfpathlineto{\pgfqpoint{5.012337in}{1.419492in}}%
\pgfpathlineto{\pgfqpoint{5.015559in}{1.417198in}}%
\pgfpathlineto{\pgfqpoint{5.016633in}{1.413054in}}%
\pgfpathlineto{\pgfqpoint{5.017706in}{1.416602in}}%
\pgfpathlineto{\pgfqpoint{5.018780in}{1.425537in}}%
\pgfpathlineto{\pgfqpoint{5.019854in}{1.428738in}}%
\pgfpathlineto{\pgfqpoint{5.023076in}{1.429819in}}%
\pgfpathlineto{\pgfqpoint{5.024149in}{1.427657in}}%
\pgfpathlineto{\pgfqpoint{5.025223in}{1.429800in}}%
\pgfpathlineto{\pgfqpoint{5.026297in}{1.436961in}}%
\pgfpathlineto{\pgfqpoint{5.027371in}{1.439798in}}%
\pgfpathlineto{\pgfqpoint{5.030592in}{1.441438in}}%
\pgfpathlineto{\pgfqpoint{5.031666in}{1.439525in}}%
\pgfpathlineto{\pgfqpoint{5.032740in}{1.438847in}}%
\pgfpathlineto{\pgfqpoint{5.033814in}{1.435736in}}%
\pgfpathlineto{\pgfqpoint{5.034888in}{1.436538in}}%
\pgfpathlineto{\pgfqpoint{5.038109in}{1.437342in}}%
\pgfpathlineto{\pgfqpoint{5.039183in}{1.430775in}}%
\pgfpathlineto{\pgfqpoint{5.040257in}{1.433125in}}%
\pgfpathlineto{\pgfqpoint{5.042405in}{1.423767in}}%
\pgfpathlineto{\pgfqpoint{5.045626in}{1.425744in}}%
\pgfpathlineto{\pgfqpoint{5.047774in}{1.431175in}}%
\pgfpathlineto{\pgfqpoint{5.048848in}{1.425959in}}%
\pgfpathlineto{\pgfqpoint{5.053143in}{1.422031in}}%
\pgfpathlineto{\pgfqpoint{5.054217in}{1.422675in}}%
\pgfpathlineto{\pgfqpoint{5.055291in}{1.417765in}}%
\pgfpathlineto{\pgfqpoint{5.056365in}{1.406007in}}%
\pgfpathlineto{\pgfqpoint{5.057438in}{1.402518in}}%
\pgfpathlineto{\pgfqpoint{5.060660in}{1.409237in}}%
\pgfpathlineto{\pgfqpoint{5.061734in}{1.407428in}}%
\pgfpathlineto{\pgfqpoint{5.062808in}{1.407813in}}%
\pgfpathlineto{\pgfqpoint{5.063881in}{1.412693in}}%
\pgfpathlineto{\pgfqpoint{5.068177in}{1.416559in}}%
\pgfpathlineto{\pgfqpoint{5.069251in}{1.415785in}}%
\pgfpathlineto{\pgfqpoint{5.071398in}{1.405600in}}%
\pgfpathlineto{\pgfqpoint{5.072472in}{1.405600in}}%
\pgfpathlineto{\pgfqpoint{5.075694in}{1.402397in}}%
\pgfpathlineto{\pgfqpoint{5.076768in}{1.399439in}}%
\pgfpathlineto{\pgfqpoint{5.077841in}{1.401041in}}%
\pgfpathlineto{\pgfqpoint{5.078915in}{1.401534in}}%
\pgfpathlineto{\pgfqpoint{5.079989in}{1.412277in}}%
\pgfpathlineto{\pgfqpoint{5.083211in}{1.411536in}}%
\pgfpathlineto{\pgfqpoint{5.085358in}{1.421487in}}%
\pgfpathlineto{\pgfqpoint{5.087506in}{1.415867in}}%
\pgfpathlineto{\pgfqpoint{5.091801in}{1.417094in}}%
\pgfpathlineto{\pgfqpoint{5.092875in}{1.418328in}}%
\pgfpathlineto{\pgfqpoint{5.093949in}{1.418575in}}%
\pgfpathlineto{\pgfqpoint{5.095023in}{1.420921in}}%
\pgfpathlineto{\pgfqpoint{5.098244in}{1.420056in}}%
\pgfpathlineto{\pgfqpoint{5.099318in}{1.427439in}}%
\pgfpathlineto{\pgfqpoint{5.100392in}{1.426044in}}%
\pgfpathlineto{\pgfqpoint{5.101466in}{1.427439in}}%
\pgfpathlineto{\pgfqpoint{5.102540in}{1.426044in}}%
\pgfpathlineto{\pgfqpoint{5.105761in}{1.424656in}}%
\pgfpathlineto{\pgfqpoint{5.106835in}{1.425032in}}%
\pgfpathlineto{\pgfqpoint{5.107909in}{1.423525in}}%
\pgfpathlineto{\pgfqpoint{5.108983in}{1.420259in}}%
\pgfpathlineto{\pgfqpoint{5.110057in}{1.421641in}}%
\pgfpathlineto{\pgfqpoint{5.113278in}{1.420385in}}%
\pgfpathlineto{\pgfqpoint{5.114352in}{1.421385in}}%
\pgfpathlineto{\pgfqpoint{5.115426in}{1.423518in}}%
\pgfpathlineto{\pgfqpoint{5.116500in}{1.423016in}}%
\pgfpathlineto{\pgfqpoint{5.117573in}{1.423892in}}%
\pgfpathlineto{\pgfqpoint{5.120795in}{1.427286in}}%
\pgfpathlineto{\pgfqpoint{5.121869in}{1.431684in}}%
\pgfpathlineto{\pgfqpoint{5.128312in}{1.440170in}}%
\pgfpathlineto{\pgfqpoint{5.129386in}{1.444758in}}%
\pgfpathlineto{\pgfqpoint{5.130459in}{1.441743in}}%
\pgfpathlineto{\pgfqpoint{5.131533in}{1.441225in}}%
\pgfpathlineto{\pgfqpoint{5.135829in}{1.440837in}}%
\pgfpathlineto{\pgfqpoint{5.136903in}{1.445613in}}%
\pgfpathlineto{\pgfqpoint{5.137976in}{1.447059in}}%
\pgfpathlineto{\pgfqpoint{5.140124in}{1.432409in}}%
\pgfpathlineto{\pgfqpoint{5.143346in}{1.430298in}}%
\pgfpathlineto{\pgfqpoint{5.145493in}{1.426130in}}%
\pgfpathlineto{\pgfqpoint{5.147641in}{1.436648in}}%
\pgfpathlineto{\pgfqpoint{5.150862in}{1.437145in}}%
\pgfpathlineto{\pgfqpoint{5.151936in}{1.439881in}}%
\pgfpathlineto{\pgfqpoint{5.154084in}{1.441999in}}%
\pgfpathlineto{\pgfqpoint{5.155158in}{1.439629in}}%
\pgfpathlineto{\pgfqpoint{5.158379in}{1.436663in}}%
\pgfpathlineto{\pgfqpoint{5.159453in}{1.432877in}}%
\pgfpathlineto{\pgfqpoint{5.160527in}{1.434922in}}%
\pgfpathlineto{\pgfqpoint{5.161601in}{1.421942in}}%
\pgfpathlineto{\pgfqpoint{5.162675in}{1.416241in}}%
\pgfpathlineto{\pgfqpoint{5.165896in}{1.420001in}}%
\pgfpathlineto{\pgfqpoint{5.166970in}{1.425339in}}%
\pgfpathlineto{\pgfqpoint{5.169118in}{1.415294in}}%
\pgfpathlineto{\pgfqpoint{5.170191in}{1.415790in}}%
\pgfpathlineto{\pgfqpoint{5.173413in}{1.416038in}}%
\pgfpathlineto{\pgfqpoint{5.174487in}{1.417652in}}%
\pgfpathlineto{\pgfqpoint{5.176635in}{1.416041in}}%
\pgfpathlineto{\pgfqpoint{5.177708in}{1.418014in}}%
\pgfpathlineto{\pgfqpoint{5.180930in}{1.417890in}}%
\pgfpathlineto{\pgfqpoint{5.182004in}{1.417019in}}%
\pgfpathlineto{\pgfqpoint{5.183078in}{1.418263in}}%
\pgfpathlineto{\pgfqpoint{5.184151in}{1.417765in}}%
\pgfpathlineto{\pgfqpoint{5.185225in}{1.420123in}}%
\pgfpathlineto{\pgfqpoint{5.188447in}{1.425259in}}%
\pgfpathlineto{\pgfqpoint{5.191668in}{1.422634in}}%
\pgfpathlineto{\pgfqpoint{5.192742in}{1.426600in}}%
\pgfpathlineto{\pgfqpoint{5.197037in}{1.427096in}}%
\pgfpathlineto{\pgfqpoint{5.199185in}{1.429829in}}%
\pgfpathlineto{\pgfqpoint{5.200259in}{1.432079in}}%
\pgfpathlineto{\pgfqpoint{5.203480in}{1.436204in}}%
\pgfpathlineto{\pgfqpoint{5.204554in}{1.429849in}}%
\pgfpathlineto{\pgfqpoint{5.205628in}{1.428324in}}%
\pgfpathlineto{\pgfqpoint{5.206702in}{1.430993in}}%
\pgfpathlineto{\pgfqpoint{5.207776in}{1.437221in}}%
\pgfpathlineto{\pgfqpoint{5.210997in}{1.438915in}}%
\pgfpathlineto{\pgfqpoint{5.215293in}{1.420763in}}%
\pgfpathlineto{\pgfqpoint{5.218514in}{1.422459in}}%
\pgfpathlineto{\pgfqpoint{5.219588in}{1.419897in}}%
\pgfpathlineto{\pgfqpoint{5.220662in}{1.421361in}}%
\pgfpathlineto{\pgfqpoint{5.221736in}{1.425753in}}%
\pgfpathlineto{\pgfqpoint{5.222810in}{1.427244in}}%
\pgfpathlineto{\pgfqpoint{5.226031in}{1.427369in}}%
\pgfpathlineto{\pgfqpoint{5.227105in}{1.429730in}}%
\pgfpathlineto{\pgfqpoint{5.228179in}{1.435324in}}%
\pgfpathlineto{\pgfqpoint{5.229253in}{1.433925in}}%
\pgfpathlineto{\pgfqpoint{5.230326in}{1.438248in}}%
\pgfpathlineto{\pgfqpoint{5.233548in}{1.434561in}}%
\pgfpathlineto{\pgfqpoint{5.234622in}{1.438570in}}%
\pgfpathlineto{\pgfqpoint{5.236769in}{1.455885in}}%
\pgfpathlineto{\pgfqpoint{5.237843in}{1.456018in}}%
\pgfpathlineto{\pgfqpoint{5.243213in}{1.445175in}}%
\pgfpathlineto{\pgfqpoint{5.244286in}{1.442530in}}%
\pgfpathlineto{\pgfqpoint{5.245360in}{1.446762in}}%
\pgfpathlineto{\pgfqpoint{5.249656in}{1.440188in}}%
\pgfpathlineto{\pgfqpoint{5.250729in}{1.444183in}}%
\pgfpathlineto{\pgfqpoint{5.251803in}{1.443266in}}%
\pgfpathlineto{\pgfqpoint{5.252877in}{1.449027in}}%
\pgfpathlineto{\pgfqpoint{5.256099in}{1.442088in}}%
\pgfpathlineto{\pgfqpoint{5.257172in}{1.436608in}}%
\pgfpathlineto{\pgfqpoint{5.259320in}{1.438482in}}%
\pgfpathlineto{\pgfqpoint{5.260394in}{1.434973in}}%
\pgfpathlineto{\pgfqpoint{5.263615in}{1.433490in}}%
\pgfpathlineto{\pgfqpoint{5.265763in}{1.434227in}}%
\pgfpathlineto{\pgfqpoint{5.266837in}{1.433611in}}%
\pgfpathlineto{\pgfqpoint{5.267911in}{1.439400in}}%
\pgfpathlineto{\pgfqpoint{5.271132in}{1.440878in}}%
\pgfpathlineto{\pgfqpoint{5.272206in}{1.439763in}}%
\pgfpathlineto{\pgfqpoint{5.273280in}{1.441870in}}%
\pgfpathlineto{\pgfqpoint{5.275428in}{1.435100in}}%
\pgfpathlineto{\pgfqpoint{5.278649in}{1.436065in}}%
\pgfpathlineto{\pgfqpoint{5.279723in}{1.434370in}}%
\pgfpathlineto{\pgfqpoint{5.280797in}{1.427226in}}%
\pgfpathlineto{\pgfqpoint{5.281871in}{1.424320in}}%
\pgfpathlineto{\pgfqpoint{5.282945in}{1.424441in}}%
\pgfpathlineto{\pgfqpoint{5.287240in}{1.415965in}}%
\pgfpathlineto{\pgfqpoint{5.288314in}{1.422867in}}%
\pgfpathlineto{\pgfqpoint{5.290461in}{1.417539in}}%
\pgfpathlineto{\pgfqpoint{5.293683in}{1.422870in}}%
\pgfpathlineto{\pgfqpoint{5.294757in}{1.413908in}}%
\pgfpathlineto{\pgfqpoint{5.295831in}{1.410759in}}%
\pgfpathlineto{\pgfqpoint{5.296904in}{1.410638in}}%
\pgfpathlineto{\pgfqpoint{5.297978in}{1.408942in}}%
\pgfpathlineto{\pgfqpoint{5.301200in}{1.411849in}}%
\pgfpathlineto{\pgfqpoint{5.302274in}{1.431227in}}%
\pgfpathlineto{\pgfqpoint{5.303347in}{1.423476in}}%
\pgfpathlineto{\pgfqpoint{5.304421in}{1.425315in}}%
\pgfpathlineto{\pgfqpoint{5.305495in}{1.433592in}}%
\pgfpathlineto{\pgfqpoint{5.308717in}{1.431961in}}%
\pgfpathlineto{\pgfqpoint{5.309791in}{1.433048in}}%
\pgfpathlineto{\pgfqpoint{5.311938in}{1.416598in}}%
\pgfpathlineto{\pgfqpoint{5.313012in}{1.416980in}}%
\pgfpathlineto{\pgfqpoint{5.316234in}{1.422073in}}%
\pgfpathlineto{\pgfqpoint{5.317307in}{1.420418in}}%
\pgfpathlineto{\pgfqpoint{5.318381in}{1.420418in}}%
\pgfpathlineto{\pgfqpoint{5.323750in}{1.425476in}}%
\pgfpathlineto{\pgfqpoint{5.324824in}{1.423170in}}%
\pgfpathlineto{\pgfqpoint{5.325898in}{1.417533in}}%
\pgfpathlineto{\pgfqpoint{5.328046in}{1.419455in}}%
\pgfpathlineto{\pgfqpoint{5.331267in}{1.423555in}}%
\pgfpathlineto{\pgfqpoint{5.333415in}{1.431634in}}%
\pgfpathlineto{\pgfqpoint{5.334489in}{1.442191in}}%
\pgfpathlineto{\pgfqpoint{5.335563in}{1.446678in}}%
\pgfpathlineto{\pgfqpoint{5.338784in}{1.450171in}}%
\pgfpathlineto{\pgfqpoint{5.339858in}{1.449096in}}%
\pgfpathlineto{\pgfqpoint{5.342006in}{1.452855in}}%
\pgfpathlineto{\pgfqpoint{5.343079in}{1.443138in}}%
\pgfpathlineto{\pgfqpoint{5.348449in}{1.440414in}}%
\pgfpathlineto{\pgfqpoint{5.349523in}{1.445043in}}%
\pgfpathlineto{\pgfqpoint{5.350596in}{1.438101in}}%
\pgfpathlineto{\pgfqpoint{5.353818in}{1.436006in}}%
\pgfpathlineto{\pgfqpoint{5.355966in}{1.426314in}}%
\pgfpathlineto{\pgfqpoint{5.357039in}{1.426969in}}%
\pgfpathlineto{\pgfqpoint{5.361335in}{1.445153in}}%
\pgfpathlineto{\pgfqpoint{5.362409in}{1.435956in}}%
\pgfpathlineto{\pgfqpoint{5.364556in}{1.436857in}}%
\pgfpathlineto{\pgfqpoint{5.365630in}{1.421541in}}%
\pgfpathlineto{\pgfqpoint{5.368852in}{1.423841in}}%
\pgfpathlineto{\pgfqpoint{5.369925in}{1.427997in}}%
\pgfpathlineto{\pgfqpoint{5.370999in}{1.424941in}}%
\pgfpathlineto{\pgfqpoint{5.372073in}{1.427839in}}%
\pgfpathlineto{\pgfqpoint{5.373147in}{1.425762in}}%
\pgfpathlineto{\pgfqpoint{5.377442in}{1.424418in}}%
\pgfpathlineto{\pgfqpoint{5.378516in}{1.422097in}}%
\pgfpathlineto{\pgfqpoint{5.379590in}{1.421731in}}%
\pgfpathlineto{\pgfqpoint{5.383885in}{1.424296in}}%
\pgfpathlineto{\pgfqpoint{5.384959in}{1.428729in}}%
\pgfpathlineto{\pgfqpoint{5.386033in}{1.428853in}}%
\pgfpathlineto{\pgfqpoint{5.387107in}{1.430578in}}%
\pgfpathlineto{\pgfqpoint{5.388181in}{1.429838in}}%
\pgfpathlineto{\pgfqpoint{5.391402in}{1.429593in}}%
\pgfpathlineto{\pgfqpoint{5.392476in}{1.430329in}}%
\pgfpathlineto{\pgfqpoint{5.393550in}{1.435115in}}%
\pgfpathlineto{\pgfqpoint{5.394624in}{1.433614in}}%
\pgfpathlineto{\pgfqpoint{5.395698in}{1.427233in}}%
\pgfpathlineto{\pgfqpoint{5.398919in}{1.426607in}}%
\pgfpathlineto{\pgfqpoint{5.399993in}{1.419225in}}%
\pgfpathlineto{\pgfqpoint{5.401067in}{1.420977in}}%
\pgfpathlineto{\pgfqpoint{5.402141in}{1.410217in}}%
\pgfpathlineto{\pgfqpoint{5.403214in}{1.409259in}}%
\pgfpathlineto{\pgfqpoint{5.406436in}{1.404968in}}%
\pgfpathlineto{\pgfqpoint{5.407510in}{1.408129in}}%
\pgfpathlineto{\pgfqpoint{5.408584in}{1.413943in}}%
\pgfpathlineto{\pgfqpoint{5.410731in}{1.418833in}}%
\pgfpathlineto{\pgfqpoint{5.413953in}{1.417994in}}%
\pgfpathlineto{\pgfqpoint{5.415027in}{1.420263in}}%
\pgfpathlineto{\pgfqpoint{5.416101in}{1.413633in}}%
\pgfpathlineto{\pgfqpoint{5.417174in}{1.410498in}}%
\pgfpathlineto{\pgfqpoint{5.418248in}{1.411945in}}%
\pgfpathlineto{\pgfqpoint{5.421470in}{1.406520in}}%
\pgfpathlineto{\pgfqpoint{5.423617in}{1.414119in}}%
\pgfpathlineto{\pgfqpoint{5.424691in}{1.408709in}}%
\pgfpathlineto{\pgfqpoint{5.428987in}{1.407181in}}%
\pgfpathlineto{\pgfqpoint{5.432208in}{1.415662in}}%
\pgfpathlineto{\pgfqpoint{5.433282in}{1.417805in}}%
\pgfpathlineto{\pgfqpoint{5.436503in}{1.419591in}}%
\pgfpathlineto{\pgfqpoint{5.437577in}{1.420910in}}%
\pgfpathlineto{\pgfqpoint{5.438651in}{1.418751in}}%
\pgfpathlineto{\pgfqpoint{5.439725in}{1.419940in}}%
\pgfpathlineto{\pgfqpoint{5.440799in}{1.416237in}}%
\pgfpathlineto{\pgfqpoint{5.445094in}{1.423046in}}%
\pgfpathlineto{\pgfqpoint{5.446168in}{1.427022in}}%
\pgfpathlineto{\pgfqpoint{5.447242in}{1.422324in}}%
\pgfpathlineto{\pgfqpoint{5.451537in}{1.423152in}}%
\pgfpathlineto{\pgfqpoint{5.452611in}{1.418656in}}%
\pgfpathlineto{\pgfqpoint{5.453685in}{1.419353in}}%
\pgfpathlineto{\pgfqpoint{5.455833in}{1.418770in}}%
\pgfpathlineto{\pgfqpoint{5.459054in}{1.419819in}}%
\pgfpathlineto{\pgfqpoint{5.460128in}{1.424012in}}%
\pgfpathlineto{\pgfqpoint{5.461202in}{1.419627in}}%
\pgfpathlineto{\pgfqpoint{5.462276in}{1.420575in}}%
\pgfpathlineto{\pgfqpoint{5.463349in}{1.416546in}}%
\pgfpathlineto{\pgfqpoint{5.466571in}{1.419809in}}%
\pgfpathlineto{\pgfqpoint{5.467645in}{1.418037in}}%
\pgfpathlineto{\pgfqpoint{5.468719in}{1.419101in}}%
\pgfpathlineto{\pgfqpoint{5.469792in}{1.416737in}}%
\pgfpathlineto{\pgfqpoint{5.470866in}{1.418493in}}%
\pgfpathlineto{\pgfqpoint{5.475162in}{1.416135in}}%
\pgfpathlineto{\pgfqpoint{5.476235in}{1.417432in}}%
\pgfpathlineto{\pgfqpoint{5.477309in}{1.415310in}}%
\pgfpathlineto{\pgfqpoint{5.478383in}{1.418114in}}%
\pgfpathlineto{\pgfqpoint{5.482679in}{1.416223in}}%
\pgfpathlineto{\pgfqpoint{5.483752in}{1.417760in}}%
\pgfpathlineto{\pgfqpoint{5.484826in}{1.418114in}}%
\pgfpathlineto{\pgfqpoint{5.489122in}{1.415036in}}%
\pgfpathlineto{\pgfqpoint{5.491269in}{1.413616in}}%
\pgfpathlineto{\pgfqpoint{5.493417in}{1.395622in}}%
\pgfpathlineto{\pgfqpoint{5.497712in}{1.398582in}}%
\pgfpathlineto{\pgfqpoint{5.498786in}{1.399889in}}%
\pgfpathlineto{\pgfqpoint{5.499860in}{1.397156in}}%
\pgfpathlineto{\pgfqpoint{5.500934in}{1.402089in}}%
\pgfpathlineto{\pgfqpoint{5.504155in}{1.399689in}}%
\pgfpathlineto{\pgfqpoint{5.505229in}{1.403770in}}%
\pgfpathlineto{\pgfqpoint{5.506303in}{1.405210in}}%
\pgfpathlineto{\pgfqpoint{5.508451in}{1.413179in}}%
\pgfpathlineto{\pgfqpoint{5.511672in}{1.413790in}}%
\pgfpathlineto{\pgfqpoint{5.512746in}{1.411345in}}%
\pgfpathlineto{\pgfqpoint{5.513820in}{1.416066in}}%
\pgfpathlineto{\pgfqpoint{5.514894in}{1.417547in}}%
\pgfpathlineto{\pgfqpoint{5.515967in}{1.415695in}}%
\pgfpathlineto{\pgfqpoint{5.519189in}{1.424272in}}%
\pgfpathlineto{\pgfqpoint{5.520263in}{1.422621in}}%
\pgfpathlineto{\pgfqpoint{5.521337in}{1.426685in}}%
\pgfpathlineto{\pgfqpoint{5.522411in}{1.424653in}}%
\pgfpathlineto{\pgfqpoint{5.526706in}{1.426290in}}%
\pgfpathlineto{\pgfqpoint{5.528854in}{1.429713in}}%
\pgfpathlineto{\pgfqpoint{5.529927in}{1.426400in}}%
\pgfpathlineto{\pgfqpoint{5.531001in}{1.434173in}}%
\pgfpathlineto{\pgfqpoint{5.534223in}{1.432007in}}%
\pgfpathlineto{\pgfqpoint{5.535297in}{1.429733in}}%
\pgfpathlineto{\pgfqpoint{5.536370in}{1.430609in}}%
\pgfpathlineto{\pgfqpoint{5.538518in}{1.438398in}}%
\pgfpathlineto{\pgfqpoint{5.541740in}{1.443727in}}%
\pgfpathlineto{\pgfqpoint{5.542813in}{1.455655in}}%
\pgfpathlineto{\pgfqpoint{5.543887in}{1.449937in}}%
\pgfpathlineto{\pgfqpoint{5.544961in}{1.448740in}}%
\pgfpathlineto{\pgfqpoint{5.546035in}{1.445282in}}%
\pgfpathlineto{\pgfqpoint{5.549256in}{1.443022in}}%
\pgfpathlineto{\pgfqpoint{5.550330in}{1.443686in}}%
\pgfpathlineto{\pgfqpoint{5.551404in}{1.436240in}}%
\pgfpathlineto{\pgfqpoint{5.553552in}{1.433792in}}%
\pgfpathlineto{\pgfqpoint{5.556773in}{1.431873in}}%
\pgfpathlineto{\pgfqpoint{5.557847in}{1.435046in}}%
\pgfpathlineto{\pgfqpoint{5.559995in}{1.434661in}}%
\pgfpathlineto{\pgfqpoint{5.561069in}{1.431704in}}%
\pgfpathlineto{\pgfqpoint{5.564290in}{1.441087in}}%
\pgfpathlineto{\pgfqpoint{5.566438in}{1.434573in}}%
\pgfpathlineto{\pgfqpoint{5.567512in}{1.436577in}}%
\pgfpathlineto{\pgfqpoint{5.568586in}{1.434809in}}%
\pgfpathlineto{\pgfqpoint{5.573955in}{1.435566in}}%
\pgfpathlineto{\pgfqpoint{5.579324in}{1.407083in}}%
\pgfpathlineto{\pgfqpoint{5.580398in}{1.394044in}}%
\pgfpathlineto{\pgfqpoint{5.581472in}{1.411514in}}%
\pgfpathlineto{\pgfqpoint{5.582545in}{1.400627in}}%
\pgfpathlineto{\pgfqpoint{5.583619in}{1.401836in}}%
\pgfpathlineto{\pgfqpoint{5.586841in}{1.401229in}}%
\pgfpathlineto{\pgfqpoint{5.587915in}{1.389923in}}%
\pgfpathlineto{\pgfqpoint{5.588989in}{1.394542in}}%
\pgfpathlineto{\pgfqpoint{5.590062in}{1.390774in}}%
\pgfpathlineto{\pgfqpoint{5.591136in}{1.399743in}}%
\pgfpathlineto{\pgfqpoint{5.595432in}{1.410689in}}%
\pgfpathlineto{\pgfqpoint{5.596505in}{1.415789in}}%
\pgfpathlineto{\pgfqpoint{5.597579in}{1.416552in}}%
\pgfpathlineto{\pgfqpoint{5.598653in}{1.413629in}}%
\pgfpathlineto{\pgfqpoint{5.601875in}{1.414508in}}%
\pgfpathlineto{\pgfqpoint{5.602948in}{1.422069in}}%
\pgfpathlineto{\pgfqpoint{5.604022in}{1.423959in}}%
\pgfpathlineto{\pgfqpoint{5.606170in}{1.406814in}}%
\pgfpathlineto{\pgfqpoint{5.609391in}{1.409227in}}%
\pgfpathlineto{\pgfqpoint{5.610465in}{1.413164in}}%
\pgfpathlineto{\pgfqpoint{5.611539in}{1.408387in}}%
\pgfpathlineto{\pgfqpoint{5.612613in}{1.409549in}}%
\pgfpathlineto{\pgfqpoint{5.613687in}{1.408128in}}%
\pgfpathlineto{\pgfqpoint{5.616908in}{1.413262in}}%
\pgfpathlineto{\pgfqpoint{5.617982in}{1.410901in}}%
\pgfpathlineto{\pgfqpoint{5.619056in}{1.410639in}}%
\pgfpathlineto{\pgfqpoint{5.620130in}{1.404737in}}%
\pgfpathlineto{\pgfqpoint{5.621204in}{1.403425in}}%
\pgfpathlineto{\pgfqpoint{5.624425in}{1.416017in}}%
\pgfpathlineto{\pgfqpoint{5.625499in}{1.415361in}}%
\pgfpathlineto{\pgfqpoint{5.627647in}{1.407319in}}%
\pgfpathlineto{\pgfqpoint{5.628721in}{1.408078in}}%
\pgfpathlineto{\pgfqpoint{5.631942in}{1.409600in}}%
\pgfpathlineto{\pgfqpoint{5.633016in}{1.408966in}}%
\pgfpathlineto{\pgfqpoint{5.634090in}{1.412889in}}%
\pgfpathlineto{\pgfqpoint{5.635164in}{1.420352in}}%
\pgfpathlineto{\pgfqpoint{5.636237in}{1.420094in}}%
\pgfpathlineto{\pgfqpoint{5.639459in}{1.420094in}}%
\pgfpathlineto{\pgfqpoint{5.641607in}{1.430120in}}%
\pgfpathlineto{\pgfqpoint{5.642680in}{1.441490in}}%
\pgfpathlineto{\pgfqpoint{5.643754in}{1.438615in}}%
\pgfpathlineto{\pgfqpoint{5.646976in}{1.436418in}}%
\pgfpathlineto{\pgfqpoint{5.648050in}{1.438468in}}%
\pgfpathlineto{\pgfqpoint{5.649123in}{1.441826in}}%
\pgfpathlineto{\pgfqpoint{5.650197in}{1.442601in}}%
\pgfpathlineto{\pgfqpoint{5.651271in}{1.447783in}}%
\pgfpathlineto{\pgfqpoint{5.654493in}{1.448819in}}%
\pgfpathlineto{\pgfqpoint{5.656640in}{1.441795in}}%
\pgfpathlineto{\pgfqpoint{5.657714in}{1.442446in}}%
\pgfpathlineto{\pgfqpoint{5.658788in}{1.447128in}}%
\pgfpathlineto{\pgfqpoint{5.663083in}{1.439576in}}%
\pgfpathlineto{\pgfqpoint{5.664157in}{1.441961in}}%
\pgfpathlineto{\pgfqpoint{5.665231in}{1.447393in}}%
\pgfpathlineto{\pgfqpoint{5.666305in}{1.440489in}}%
\pgfpathlineto{\pgfqpoint{5.669526in}{1.449830in}}%
\pgfpathlineto{\pgfqpoint{5.670600in}{1.449424in}}%
\pgfpathlineto{\pgfqpoint{5.672748in}{1.441643in}}%
\pgfpathlineto{\pgfqpoint{5.673822in}{1.445839in}}%
\pgfpathlineto{\pgfqpoint{5.677043in}{1.441440in}}%
\pgfpathlineto{\pgfqpoint{5.681339in}{1.450011in}}%
\pgfpathlineto{\pgfqpoint{5.685634in}{1.446114in}}%
\pgfpathlineto{\pgfqpoint{5.686708in}{1.453419in}}%
\pgfpathlineto{\pgfqpoint{5.687782in}{1.449183in}}%
\pgfpathlineto{\pgfqpoint{5.688856in}{1.462440in}}%
\pgfpathlineto{\pgfqpoint{5.692077in}{1.458339in}}%
\pgfpathlineto{\pgfqpoint{5.693151in}{1.462376in}}%
\pgfpathlineto{\pgfqpoint{5.694225in}{1.458413in}}%
\pgfpathlineto{\pgfqpoint{5.695299in}{1.457867in}}%
\pgfpathlineto{\pgfqpoint{5.696372in}{1.452127in}}%
\pgfpathlineto{\pgfqpoint{5.699594in}{1.459370in}}%
\pgfpathlineto{\pgfqpoint{5.700668in}{1.458140in}}%
\pgfpathlineto{\pgfqpoint{5.701742in}{1.447123in}}%
\pgfpathlineto{\pgfqpoint{5.702815in}{1.451687in}}%
\pgfpathlineto{\pgfqpoint{5.703889in}{1.445713in}}%
\pgfpathlineto{\pgfqpoint{5.707111in}{1.449430in}}%
\pgfpathlineto{\pgfqpoint{5.709258in}{1.437747in}}%
\pgfpathlineto{\pgfqpoint{5.710332in}{1.440284in}}%
\pgfpathlineto{\pgfqpoint{5.714628in}{1.440668in}}%
\pgfpathlineto{\pgfqpoint{5.715701in}{1.435672in}}%
\pgfpathlineto{\pgfqpoint{5.716775in}{1.440319in}}%
\pgfpathlineto{\pgfqpoint{5.717849in}{1.434305in}}%
\pgfpathlineto{\pgfqpoint{5.722144in}{1.430595in}}%
\pgfpathlineto{\pgfqpoint{5.724292in}{1.441855in}}%
\pgfpathlineto{\pgfqpoint{5.725366in}{1.438989in}}%
\pgfpathlineto{\pgfqpoint{5.726440in}{1.434169in}}%
\pgfpathlineto{\pgfqpoint{5.729661in}{1.437035in}}%
\pgfpathlineto{\pgfqpoint{5.730735in}{1.438859in}}%
\pgfpathlineto{\pgfqpoint{5.731809in}{1.430068in}}%
\pgfpathlineto{\pgfqpoint{5.733957in}{1.443188in}}%
\pgfpathlineto{\pgfqpoint{5.738252in}{1.448274in}}%
\pgfpathlineto{\pgfqpoint{5.739326in}{1.453360in}}%
\pgfpathlineto{\pgfqpoint{5.740400in}{1.470287in}}%
\pgfpathlineto{\pgfqpoint{5.741474in}{1.456636in}}%
\pgfpathlineto{\pgfqpoint{5.744695in}{1.456766in}}%
\pgfpathlineto{\pgfqpoint{5.745769in}{1.470263in}}%
\pgfpathlineto{\pgfqpoint{5.746843in}{1.461568in}}%
\pgfpathlineto{\pgfqpoint{5.747917in}{1.461819in}}%
\pgfpathlineto{\pgfqpoint{5.748990in}{1.472131in}}%
\pgfpathlineto{\pgfqpoint{5.752212in}{1.463579in}}%
\pgfpathlineto{\pgfqpoint{5.754360in}{1.480157in}}%
\pgfpathlineto{\pgfqpoint{5.755433in}{1.482299in}}%
\pgfpathlineto{\pgfqpoint{5.756507in}{1.488267in}}%
\pgfpathlineto{\pgfqpoint{5.759729in}{1.490806in}}%
\pgfpathlineto{\pgfqpoint{5.760803in}{1.484271in}}%
\pgfpathlineto{\pgfqpoint{5.761877in}{1.482477in}}%
\pgfpathlineto{\pgfqpoint{5.762950in}{1.475941in}}%
\pgfpathlineto{\pgfqpoint{5.764024in}{1.483886in}}%
\pgfpathlineto{\pgfqpoint{5.769393in}{1.481584in}}%
\pgfpathlineto{\pgfqpoint{5.770467in}{1.474849in}}%
\pgfpathlineto{\pgfqpoint{5.774763in}{1.477956in}}%
\pgfpathlineto{\pgfqpoint{5.775836in}{1.482682in}}%
\pgfpathlineto{\pgfqpoint{5.776910in}{1.484834in}}%
\pgfpathlineto{\pgfqpoint{5.777984in}{1.481797in}}%
\pgfpathlineto{\pgfqpoint{5.779058in}{1.482673in}}%
\pgfpathlineto{\pgfqpoint{5.782279in}{1.479533in}}%
\pgfpathlineto{\pgfqpoint{5.783353in}{1.487319in}}%
\pgfpathlineto{\pgfqpoint{5.784427in}{1.480161in}}%
\pgfpathlineto{\pgfqpoint{5.785501in}{1.482608in}}%
\pgfpathlineto{\pgfqpoint{5.786575in}{1.481744in}}%
\pgfpathlineto{\pgfqpoint{5.789796in}{1.486066in}}%
\pgfpathlineto{\pgfqpoint{5.790870in}{1.483473in}}%
\pgfpathlineto{\pgfqpoint{5.791944in}{1.484696in}}%
\pgfpathlineto{\pgfqpoint{5.793018in}{1.484450in}}%
\pgfpathlineto{\pgfqpoint{5.794092in}{1.486662in}}%
\pgfpathlineto{\pgfqpoint{5.797313in}{1.486539in}}%
\pgfpathlineto{\pgfqpoint{5.798387in}{1.487890in}}%
\pgfpathlineto{\pgfqpoint{5.800535in}{1.477909in}}%
\pgfpathlineto{\pgfqpoint{5.801609in}{1.481817in}}%
\pgfpathlineto{\pgfqpoint{5.804830in}{1.483860in}}%
\pgfpathlineto{\pgfqpoint{5.805904in}{1.486263in}}%
\pgfpathlineto{\pgfqpoint{5.806978in}{1.483111in}}%
\pgfpathlineto{\pgfqpoint{5.808052in}{1.489899in}}%
\pgfpathlineto{\pgfqpoint{5.812347in}{1.491596in}}%
\pgfpathlineto{\pgfqpoint{5.813421in}{1.498425in}}%
\pgfpathlineto{\pgfqpoint{5.815568in}{1.498913in}}%
\pgfpathlineto{\pgfqpoint{5.816642in}{1.499645in}}%
\pgfpathlineto{\pgfqpoint{5.819864in}{1.503926in}}%
\pgfpathlineto{\pgfqpoint{5.820938in}{1.507351in}}%
\pgfpathlineto{\pgfqpoint{5.822011in}{1.507351in}}%
\pgfpathlineto{\pgfqpoint{5.823085in}{1.491133in}}%
\pgfpathlineto{\pgfqpoint{5.827381in}{1.499056in}}%
\pgfpathlineto{\pgfqpoint{5.828455in}{1.502726in}}%
\pgfpathlineto{\pgfqpoint{5.829528in}{1.509937in}}%
\pgfpathlineto{\pgfqpoint{5.831676in}{1.510846in}}%
\pgfpathlineto{\pgfqpoint{5.834898in}{1.515130in}}%
\pgfpathlineto{\pgfqpoint{5.835971in}{1.511235in}}%
\pgfpathlineto{\pgfqpoint{5.837045in}{1.514822in}}%
\pgfpathlineto{\pgfqpoint{5.838119in}{1.495496in}}%
\pgfpathlineto{\pgfqpoint{5.839193in}{1.501333in}}%
\pgfpathlineto{\pgfqpoint{5.842414in}{1.498998in}}%
\pgfpathlineto{\pgfqpoint{5.843488in}{1.502471in}}%
\pgfpathlineto{\pgfqpoint{5.845636in}{1.524090in}}%
\pgfpathlineto{\pgfqpoint{5.846710in}{1.523155in}}%
\pgfpathlineto{\pgfqpoint{5.849931in}{1.527566in}}%
\pgfpathlineto{\pgfqpoint{5.851005in}{1.535051in}}%
\pgfpathlineto{\pgfqpoint{5.852079in}{1.536970in}}%
\pgfpathlineto{\pgfqpoint{5.853153in}{1.536970in}}%
\pgfpathlineto{\pgfqpoint{5.854227in}{1.540260in}}%
\pgfpathlineto{\pgfqpoint{5.857448in}{1.540671in}}%
\pgfpathlineto{\pgfqpoint{5.861744in}{1.561144in}}%
\pgfpathlineto{\pgfqpoint{5.864965in}{1.564701in}}%
\pgfpathlineto{\pgfqpoint{5.866039in}{1.568828in}}%
\pgfpathlineto{\pgfqpoint{5.868187in}{1.553113in}}%
\pgfpathlineto{\pgfqpoint{5.869260in}{1.553401in}}%
\pgfpathlineto{\pgfqpoint{5.872482in}{1.559889in}}%
\pgfpathlineto{\pgfqpoint{5.873556in}{1.565484in}}%
\pgfpathlineto{\pgfqpoint{5.875703in}{1.558019in}}%
\pgfpathlineto{\pgfqpoint{5.876777in}{1.552412in}}%
\pgfpathlineto{\pgfqpoint{5.881073in}{1.549023in}}%
\pgfpathlineto{\pgfqpoint{5.883220in}{1.558735in}}%
\pgfpathlineto{\pgfqpoint{5.884294in}{1.556885in}}%
\pgfpathlineto{\pgfqpoint{5.887516in}{1.559431in}}%
\pgfpathlineto{\pgfqpoint{5.888589in}{1.572260in}}%
\pgfpathlineto{\pgfqpoint{5.889663in}{1.575112in}}%
\pgfpathlineto{\pgfqpoint{5.890737in}{1.580433in}}%
\pgfpathlineto{\pgfqpoint{5.891811in}{1.571516in}}%
\pgfpathlineto{\pgfqpoint{5.895032in}{1.572635in}}%
\pgfpathlineto{\pgfqpoint{5.896106in}{1.577690in}}%
\pgfpathlineto{\pgfqpoint{5.897180in}{1.579515in}}%
\pgfpathlineto{\pgfqpoint{5.898254in}{1.587140in}}%
\pgfpathlineto{\pgfqpoint{5.899328in}{1.583328in}}%
\pgfpathlineto{\pgfqpoint{5.902549in}{1.583467in}}%
\pgfpathlineto{\pgfqpoint{5.903623in}{1.587516in}}%
\pgfpathlineto{\pgfqpoint{5.904697in}{1.588354in}}%
\pgfpathlineto{\pgfqpoint{5.905771in}{1.596053in}}%
\pgfpathlineto{\pgfqpoint{5.906845in}{1.598993in}}%
\pgfpathlineto{\pgfqpoint{5.910066in}{1.602807in}}%
\pgfpathlineto{\pgfqpoint{5.911140in}{1.601818in}}%
\pgfpathlineto{\pgfqpoint{5.912214in}{1.598861in}}%
\pgfpathlineto{\pgfqpoint{5.914362in}{1.584895in}}%
\pgfpathlineto{\pgfqpoint{5.918657in}{1.581417in}}%
\pgfpathlineto{\pgfqpoint{5.919731in}{1.577975in}}%
\pgfpathlineto{\pgfqpoint{5.921878in}{1.594141in}}%
\pgfpathlineto{\pgfqpoint{5.925100in}{1.593736in}}%
\pgfpathlineto{\pgfqpoint{5.928321in}{1.607368in}}%
\pgfpathlineto{\pgfqpoint{5.929395in}{1.607368in}}%
\pgfpathlineto{\pgfqpoint{5.932617in}{1.608472in}}%
\pgfpathlineto{\pgfqpoint{5.933691in}{1.611232in}}%
\pgfpathlineto{\pgfqpoint{5.934765in}{1.610258in}}%
\pgfpathlineto{\pgfqpoint{5.935838in}{1.607197in}}%
\pgfpathlineto{\pgfqpoint{5.936912in}{1.616101in}}%
\pgfpathlineto{\pgfqpoint{5.940134in}{1.618884in}}%
\pgfpathlineto{\pgfqpoint{5.941208in}{1.605838in}}%
\pgfpathlineto{\pgfqpoint{5.944429in}{1.624751in}}%
\pgfpathlineto{\pgfqpoint{5.947651in}{1.636022in}}%
\pgfpathlineto{\pgfqpoint{5.948724in}{1.629536in}}%
\pgfpathlineto{\pgfqpoint{5.949798in}{1.628357in}}%
\pgfpathlineto{\pgfqpoint{5.950872in}{1.628651in}}%
\pgfpathlineto{\pgfqpoint{5.951946in}{1.624966in}}%
\pgfpathlineto{\pgfqpoint{5.955167in}{1.624376in}}%
\pgfpathlineto{\pgfqpoint{5.956241in}{1.625261in}}%
\pgfpathlineto{\pgfqpoint{5.957315in}{1.623344in}}%
\pgfpathlineto{\pgfqpoint{5.958389in}{1.622758in}}%
\pgfpathlineto{\pgfqpoint{5.959463in}{1.625392in}}%
\pgfpathlineto{\pgfqpoint{5.962684in}{1.624802in}}%
\pgfpathlineto{\pgfqpoint{5.963758in}{1.613889in}}%
\pgfpathlineto{\pgfqpoint{5.965906in}{1.623033in}}%
\pgfpathlineto{\pgfqpoint{5.966980in}{1.617514in}}%
\pgfpathlineto{\pgfqpoint{5.970201in}{1.618857in}}%
\pgfpathlineto{\pgfqpoint{5.971275in}{1.617813in}}%
\pgfpathlineto{\pgfqpoint{5.972349in}{1.619151in}}%
\pgfpathlineto{\pgfqpoint{5.973423in}{1.621987in}}%
\pgfpathlineto{\pgfqpoint{5.974497in}{1.630794in}}%
\pgfpathlineto{\pgfqpoint{5.977718in}{1.636460in}}%
\pgfpathlineto{\pgfqpoint{5.978792in}{1.639523in}}%
\pgfpathlineto{\pgfqpoint{5.979866in}{1.640450in}}%
\pgfpathlineto{\pgfqpoint{5.982013in}{1.633072in}}%
\pgfpathlineto{\pgfqpoint{5.987383in}{1.622241in}}%
\pgfpathlineto{\pgfqpoint{5.988456in}{1.623707in}}%
\pgfpathlineto{\pgfqpoint{5.989530in}{1.600879in}}%
\pgfpathlineto{\pgfqpoint{5.992752in}{1.610600in}}%
\pgfpathlineto{\pgfqpoint{5.993826in}{1.625033in}}%
\pgfpathlineto{\pgfqpoint{5.994899in}{1.625494in}}%
\pgfpathlineto{\pgfqpoint{5.995973in}{1.618884in}}%
\pgfpathlineto{\pgfqpoint{5.997047in}{1.620241in}}%
\pgfpathlineto{\pgfqpoint{6.000269in}{1.611309in}}%
\pgfpathlineto{\pgfqpoint{6.001343in}{1.612218in}}%
\pgfpathlineto{\pgfqpoint{6.003490in}{1.598132in}}%
\pgfpathlineto{\pgfqpoint{6.004564in}{1.595374in}}%
\pgfpathlineto{\pgfqpoint{6.007786in}{1.600557in}}%
\pgfpathlineto{\pgfqpoint{6.009933in}{1.610504in}}%
\pgfpathlineto{\pgfqpoint{6.011007in}{1.611247in}}%
\pgfpathlineto{\pgfqpoint{6.012081in}{1.613032in}}%
\pgfpathlineto{\pgfqpoint{6.015302in}{1.611686in}}%
\pgfpathlineto{\pgfqpoint{6.017450in}{1.598231in}}%
\pgfpathlineto{\pgfqpoint{6.018524in}{1.598082in}}%
\pgfpathlineto{\pgfqpoint{6.019598in}{1.593597in}}%
\pgfpathlineto{\pgfqpoint{6.022819in}{1.597185in}}%
\pgfpathlineto{\pgfqpoint{6.024967in}{1.606517in}}%
\pgfpathlineto{\pgfqpoint{6.027115in}{1.606517in}}%
\pgfpathlineto{\pgfqpoint{6.030336in}{1.608484in}}%
\pgfpathlineto{\pgfqpoint{6.032484in}{1.612126in}}%
\pgfpathlineto{\pgfqpoint{6.033558in}{1.628878in}}%
\pgfpathlineto{\pgfqpoint{6.034632in}{1.615597in}}%
\pgfpathlineto{\pgfqpoint{6.037853in}{1.615757in}}%
\pgfpathlineto{\pgfqpoint{6.040001in}{1.607596in}}%
\pgfpathlineto{\pgfqpoint{6.041075in}{1.620397in}}%
\pgfpathlineto{\pgfqpoint{6.042148in}{1.625037in}}%
\pgfpathlineto{\pgfqpoint{6.045370in}{1.623416in}}%
\pgfpathlineto{\pgfqpoint{6.046444in}{1.617252in}}%
\pgfpathlineto{\pgfqpoint{6.047518in}{1.606872in}}%
\pgfpathlineto{\pgfqpoint{6.048591in}{1.605898in}}%
\pgfpathlineto{\pgfqpoint{6.049665in}{1.608818in}}%
\pgfpathlineto{\pgfqpoint{6.053961in}{1.600591in}}%
\pgfpathlineto{\pgfqpoint{6.055034in}{1.597742in}}%
\pgfpathlineto{\pgfqpoint{6.056108in}{1.613905in}}%
\pgfpathlineto{\pgfqpoint{6.057182in}{1.613905in}}%
\pgfpathlineto{\pgfqpoint{6.060404in}{1.606490in}}%
\pgfpathlineto{\pgfqpoint{6.061477in}{1.623791in}}%
\pgfpathlineto{\pgfqpoint{6.062551in}{1.615717in}}%
\pgfpathlineto{\pgfqpoint{6.063625in}{1.617005in}}%
\pgfpathlineto{\pgfqpoint{6.064699in}{1.620398in}}%
\pgfpathlineto{\pgfqpoint{6.067920in}{1.617005in}}%
\pgfpathlineto{\pgfqpoint{6.070068in}{1.590465in}}%
\pgfpathlineto{\pgfqpoint{6.072216in}{1.584698in}}%
\pgfpathlineto{\pgfqpoint{6.075437in}{1.578886in}}%
\pgfpathlineto{\pgfqpoint{6.076511in}{1.580884in}}%
\pgfpathlineto{\pgfqpoint{6.077585in}{1.567384in}}%
\pgfpathlineto{\pgfqpoint{6.082954in}{1.565517in}}%
\pgfpathlineto{\pgfqpoint{6.084028in}{1.573272in}}%
\pgfpathlineto{\pgfqpoint{6.085102in}{1.560347in}}%
\pgfpathlineto{\pgfqpoint{6.087250in}{1.567837in}}%
\pgfpathlineto{\pgfqpoint{6.090471in}{1.564634in}}%
\pgfpathlineto{\pgfqpoint{6.091545in}{1.557328in}}%
\pgfpathlineto{\pgfqpoint{6.092619in}{1.565948in}}%
\pgfpathlineto{\pgfqpoint{6.093693in}{1.568164in}}%
\pgfpathlineto{\pgfqpoint{6.094766in}{1.562625in}}%
\pgfpathlineto{\pgfqpoint{6.099062in}{1.552479in}}%
\pgfpathlineto{\pgfqpoint{6.100136in}{1.554191in}}%
\pgfpathlineto{\pgfqpoint{6.101209in}{1.562140in}}%
\pgfpathlineto{\pgfqpoint{6.102283in}{1.561742in}}%
\pgfpathlineto{\pgfqpoint{6.106579in}{1.562271in}}%
\pgfpathlineto{\pgfqpoint{6.107653in}{1.560018in}}%
\pgfpathlineto{\pgfqpoint{6.115169in}{1.582615in}}%
\pgfpathlineto{\pgfqpoint{6.116243in}{1.584096in}}%
\pgfpathlineto{\pgfqpoint{6.117317in}{1.593787in}}%
\pgfpathlineto{\pgfqpoint{6.120539in}{1.586576in}}%
\pgfpathlineto{\pgfqpoint{6.121612in}{1.587546in}}%
\pgfpathlineto{\pgfqpoint{6.123760in}{1.593817in}}%
\pgfpathlineto{\pgfqpoint{6.124834in}{1.595359in}}%
\pgfpathlineto{\pgfqpoint{6.129129in}{1.597754in}}%
\pgfpathlineto{\pgfqpoint{6.130203in}{1.603953in}}%
\pgfpathlineto{\pgfqpoint{6.131277in}{1.605388in}}%
\pgfpathlineto{\pgfqpoint{6.132351in}{1.600795in}}%
\pgfpathlineto{\pgfqpoint{6.135572in}{1.604759in}}%
\pgfpathlineto{\pgfqpoint{6.136646in}{1.575529in}}%
\pgfpathlineto{\pgfqpoint{6.137720in}{1.571087in}}%
\pgfpathlineto{\pgfqpoint{6.138794in}{1.562777in}}%
\pgfpathlineto{\pgfqpoint{6.139868in}{1.568795in}}%
\pgfpathlineto{\pgfqpoint{6.143089in}{1.571660in}}%
\pgfpathlineto{\pgfqpoint{6.144163in}{1.567034in}}%
\pgfpathlineto{\pgfqpoint{6.145237in}{1.559081in}}%
\pgfpathlineto{\pgfqpoint{6.146311in}{1.557635in}}%
\pgfpathlineto{\pgfqpoint{6.147385in}{1.561539in}}%
\pgfpathlineto{\pgfqpoint{6.150606in}{1.568769in}}%
\pgfpathlineto{\pgfqpoint{6.151680in}{1.568917in}}%
\pgfpathlineto{\pgfqpoint{6.152754in}{1.573355in}}%
\pgfpathlineto{\pgfqpoint{6.153828in}{1.567585in}}%
\pgfpathlineto{\pgfqpoint{6.154901in}{1.565405in}}%
\pgfpathlineto{\pgfqpoint{6.158123in}{1.570889in}}%
\pgfpathlineto{\pgfqpoint{6.160271in}{1.564722in}}%
\pgfpathlineto{\pgfqpoint{6.161344in}{1.569714in}}%
\pgfpathlineto{\pgfqpoint{6.162418in}{1.560169in}}%
\pgfpathlineto{\pgfqpoint{6.166714in}{1.557177in}}%
\pgfpathlineto{\pgfqpoint{6.167787in}{1.554072in}}%
\pgfpathlineto{\pgfqpoint{6.168861in}{1.546107in}}%
\pgfpathlineto{\pgfqpoint{6.169935in}{1.542566in}}%
\pgfpathlineto{\pgfqpoint{6.173157in}{1.550509in}}%
\pgfpathlineto{\pgfqpoint{6.174231in}{1.546639in}}%
\pgfpathlineto{\pgfqpoint{6.175304in}{1.548850in}}%
\pgfpathlineto{\pgfqpoint{6.177452in}{1.545405in}}%
\pgfpathlineto{\pgfqpoint{6.180674in}{1.546088in}}%
\pgfpathlineto{\pgfqpoint{6.181747in}{1.538830in}}%
\pgfpathlineto{\pgfqpoint{6.182821in}{1.535407in}}%
\pgfpathlineto{\pgfqpoint{6.183895in}{1.536913in}}%
\pgfpathlineto{\pgfqpoint{6.184969in}{1.536091in}}%
\pgfpathlineto{\pgfqpoint{6.188190in}{1.534589in}}%
\pgfpathlineto{\pgfqpoint{6.189264in}{1.535948in}}%
\pgfpathlineto{\pgfqpoint{6.190338in}{1.545503in}}%
\pgfpathlineto{\pgfqpoint{6.191412in}{1.546732in}}%
\pgfpathlineto{\pgfqpoint{6.192486in}{1.550981in}}%
\pgfpathlineto{\pgfqpoint{6.195707in}{1.552763in}}%
\pgfpathlineto{\pgfqpoint{6.196781in}{1.551660in}}%
\pgfpathlineto{\pgfqpoint{6.197855in}{1.546146in}}%
\pgfpathlineto{\pgfqpoint{6.198929in}{1.545319in}}%
\pgfpathlineto{\pgfqpoint{6.200003in}{1.545870in}}%
\pgfpathlineto{\pgfqpoint{6.205372in}{1.556853in}}%
\pgfpathlineto{\pgfqpoint{6.206446in}{1.555860in}}%
\pgfpathlineto{\pgfqpoint{6.207520in}{1.552028in}}%
\pgfpathlineto{\pgfqpoint{6.210741in}{1.557421in}}%
\pgfpathlineto{\pgfqpoint{6.211815in}{1.555718in}}%
\pgfpathlineto{\pgfqpoint{6.212889in}{1.559387in}}%
\pgfpathlineto{\pgfqpoint{6.213963in}{1.559244in}}%
\pgfpathlineto{\pgfqpoint{6.215036in}{1.562100in}}%
\pgfpathlineto{\pgfqpoint{6.218258in}{1.563528in}}%
\pgfpathlineto{\pgfqpoint{6.219332in}{1.565537in}}%
\pgfpathlineto{\pgfqpoint{6.220406in}{1.562668in}}%
\pgfpathlineto{\pgfqpoint{6.221479in}{1.566506in}}%
\pgfpathlineto{\pgfqpoint{6.225775in}{1.568952in}}%
\pgfpathlineto{\pgfqpoint{6.226849in}{1.563628in}}%
\pgfpathlineto{\pgfqpoint{6.227922in}{1.567166in}}%
\pgfpathlineto{\pgfqpoint{6.228996in}{1.560296in}}%
\pgfpathlineto{\pgfqpoint{6.230070in}{1.545269in}}%
\pgfpathlineto{\pgfqpoint{6.233292in}{1.542693in}}%
\pgfpathlineto{\pgfqpoint{6.234365in}{1.538114in}}%
\pgfpathlineto{\pgfqpoint{6.236513in}{1.555574in}}%
\pgfpathlineto{\pgfqpoint{6.237587in}{1.545408in}}%
\pgfpathlineto{\pgfqpoint{6.240809in}{1.544966in}}%
\pgfpathlineto{\pgfqpoint{6.241882in}{1.545408in}}%
\pgfpathlineto{\pgfqpoint{6.242956in}{1.542167in}}%
\pgfpathlineto{\pgfqpoint{6.244030in}{1.545811in}}%
\pgfpathlineto{\pgfqpoint{6.245104in}{1.556728in}}%
\pgfpathlineto{\pgfqpoint{6.248325in}{1.557613in}}%
\pgfpathlineto{\pgfqpoint{6.249399in}{1.554802in}}%
\pgfpathlineto{\pgfqpoint{6.250473in}{1.554359in}}%
\pgfpathlineto{\pgfqpoint{6.252621in}{1.547109in}}%
\pgfpathlineto{\pgfqpoint{6.256916in}{1.540008in}}%
\pgfpathlineto{\pgfqpoint{6.257990in}{1.528911in}}%
\pgfpathlineto{\pgfqpoint{6.259064in}{1.536457in}}%
\pgfpathlineto{\pgfqpoint{6.260138in}{1.531427in}}%
\pgfpathlineto{\pgfqpoint{6.264433in}{1.530699in}}%
\pgfpathlineto{\pgfqpoint{6.265507in}{1.525039in}}%
\pgfpathlineto{\pgfqpoint{6.266581in}{1.528522in}}%
\pgfpathlineto{\pgfqpoint{6.267654in}{1.528377in}}%
\pgfpathlineto{\pgfqpoint{6.271950in}{1.539837in}}%
\pgfpathlineto{\pgfqpoint{6.273024in}{1.534035in}}%
\pgfpathlineto{\pgfqpoint{6.274097in}{1.535742in}}%
\pgfpathlineto{\pgfqpoint{6.275171in}{1.534884in}}%
\pgfpathlineto{\pgfqpoint{6.278393in}{1.534025in}}%
\pgfpathlineto{\pgfqpoint{6.279467in}{1.534884in}}%
\pgfpathlineto{\pgfqpoint{6.280541in}{1.534168in}}%
\pgfpathlineto{\pgfqpoint{6.281614in}{1.538166in}}%
\pgfpathlineto{\pgfqpoint{6.282688in}{1.545112in}}%
\pgfpathlineto{\pgfqpoint{6.285910in}{1.538889in}}%
\pgfpathlineto{\pgfqpoint{6.286984in}{1.548247in}}%
\pgfpathlineto{\pgfqpoint{6.288057in}{1.551318in}}%
\pgfpathlineto{\pgfqpoint{6.289131in}{1.552049in}}%
\pgfpathlineto{\pgfqpoint{6.293427in}{1.551170in}}%
\pgfpathlineto{\pgfqpoint{6.295574in}{1.535778in}}%
\pgfpathlineto{\pgfqpoint{6.296648in}{1.535924in}}%
\pgfpathlineto{\pgfqpoint{6.297722in}{1.535485in}}%
\pgfpathlineto{\pgfqpoint{6.300943in}{1.540322in}}%
\pgfpathlineto{\pgfqpoint{6.302017in}{1.552343in}}%
\pgfpathlineto{\pgfqpoint{6.303091in}{1.552343in}}%
\pgfpathlineto{\pgfqpoint{6.304165in}{1.546394in}}%
\pgfpathlineto{\pgfqpoint{6.305239in}{1.549750in}}%
\pgfpathlineto{\pgfqpoint{6.308460in}{1.545021in}}%
\pgfpathlineto{\pgfqpoint{6.310608in}{1.549828in}}%
\pgfpathlineto{\pgfqpoint{6.311682in}{1.542198in}}%
\pgfpathlineto{\pgfqpoint{6.312756in}{1.541588in}}%
\pgfpathlineto{\pgfqpoint{6.315977in}{1.537773in}}%
\pgfpathlineto{\pgfqpoint{6.317051in}{1.533348in}}%
\pgfpathlineto{\pgfqpoint{6.318125in}{1.537926in}}%
\pgfpathlineto{\pgfqpoint{6.319199in}{1.534111in}}%
\pgfpathlineto{\pgfqpoint{6.320273in}{1.533056in}}%
\pgfpathlineto{\pgfqpoint{6.323494in}{1.531705in}}%
\pgfpathlineto{\pgfqpoint{6.324568in}{1.535890in}}%
\pgfpathlineto{\pgfqpoint{6.325642in}{1.537103in}}%
\pgfpathlineto{\pgfqpoint{6.326716in}{1.526341in}}%
\pgfpathlineto{\pgfqpoint{6.327789in}{1.526195in}}%
\pgfpathlineto{\pgfqpoint{6.331011in}{1.519178in}}%
\pgfpathlineto{\pgfqpoint{6.332085in}{1.522832in}}%
\pgfpathlineto{\pgfqpoint{6.333159in}{1.517278in}}%
\pgfpathlineto{\pgfqpoint{6.334232in}{1.472523in}}%
\pgfpathlineto{\pgfqpoint{6.338528in}{1.465911in}}%
\pgfpathlineto{\pgfqpoint{6.339602in}{1.460512in}}%
\pgfpathlineto{\pgfqpoint{6.341749in}{1.472152in}}%
\pgfpathlineto{\pgfqpoint{6.342823in}{1.468768in}}%
\pgfpathlineto{\pgfqpoint{6.346045in}{1.469365in}}%
\pgfpathlineto{\pgfqpoint{6.349266in}{1.460154in}}%
\pgfpathlineto{\pgfqpoint{6.350340in}{1.460633in}}%
\pgfpathlineto{\pgfqpoint{6.353562in}{1.452738in}}%
\pgfpathlineto{\pgfqpoint{6.354635in}{1.455988in}}%
\pgfpathlineto{\pgfqpoint{6.355709in}{1.455283in}}%
\pgfpathlineto{\pgfqpoint{6.356783in}{1.449758in}}%
\pgfpathlineto{\pgfqpoint{6.357857in}{1.447524in}}%
\pgfpathlineto{\pgfqpoint{6.361078in}{1.452344in}}%
\pgfpathlineto{\pgfqpoint{6.362152in}{1.449993in}}%
\pgfpathlineto{\pgfqpoint{6.364300in}{1.451159in}}%
\pgfpathlineto{\pgfqpoint{6.365374in}{1.447186in}}%
\pgfpathlineto{\pgfqpoint{6.368595in}{1.447991in}}%
\pgfpathlineto{\pgfqpoint{6.369669in}{1.446952in}}%
\pgfpathlineto{\pgfqpoint{6.370743in}{1.442912in}}%
\pgfpathlineto{\pgfqpoint{6.371817in}{1.441296in}}%
\pgfpathlineto{\pgfqpoint{6.372891in}{1.440719in}}%
\pgfpathlineto{\pgfqpoint{6.377186in}{1.434832in}}%
\pgfpathlineto{\pgfqpoint{6.378260in}{1.430099in}}%
\pgfpathlineto{\pgfqpoint{6.379334in}{1.422596in}}%
\pgfpathlineto{\pgfqpoint{6.380408in}{1.421673in}}%
\pgfpathlineto{\pgfqpoint{6.383629in}{1.423635in}}%
\pgfpathlineto{\pgfqpoint{6.385777in}{1.413695in}}%
\pgfpathlineto{\pgfqpoint{6.386851in}{1.414359in}}%
\pgfpathlineto{\pgfqpoint{6.387924in}{1.421249in}}%
\pgfpathlineto{\pgfqpoint{6.391146in}{1.418915in}}%
\pgfpathlineto{\pgfqpoint{6.392220in}{1.406369in}}%
\pgfpathlineto{\pgfqpoint{6.393294in}{1.405011in}}%
\pgfpathlineto{\pgfqpoint{6.394367in}{1.410101in}}%
\pgfpathlineto{\pgfqpoint{6.395441in}{1.419338in}}%
\pgfpathlineto{\pgfqpoint{6.398663in}{1.419338in}}%
\pgfpathlineto{\pgfqpoint{6.399737in}{1.415622in}}%
\pgfpathlineto{\pgfqpoint{6.400810in}{1.415622in}}%
\pgfpathlineto{\pgfqpoint{6.401884in}{1.414667in}}%
\pgfpathlineto{\pgfqpoint{6.402958in}{1.415410in}}%
\pgfpathlineto{\pgfqpoint{6.406180in}{1.416684in}}%
\pgfpathlineto{\pgfqpoint{6.407253in}{1.421487in}}%
\pgfpathlineto{\pgfqpoint{6.408327in}{1.421060in}}%
\pgfpathlineto{\pgfqpoint{6.409401in}{1.422338in}}%
\pgfpathlineto{\pgfqpoint{6.410475in}{1.418697in}}%
\pgfpathlineto{\pgfqpoint{6.413697in}{1.421053in}}%
\pgfpathlineto{\pgfqpoint{6.414770in}{1.420089in}}%
\pgfpathlineto{\pgfqpoint{6.415844in}{1.422863in}}%
\pgfpathlineto{\pgfqpoint{6.417992in}{1.412827in}}%
\pgfpathlineto{\pgfqpoint{6.421213in}{1.415201in}}%
\pgfpathlineto{\pgfqpoint{6.423361in}{1.409729in}}%
\pgfpathlineto{\pgfqpoint{6.425509in}{1.401153in}}%
\pgfpathlineto{\pgfqpoint{6.428730in}{1.406238in}}%
\pgfpathlineto{\pgfqpoint{6.429804in}{1.405718in}}%
\pgfpathlineto{\pgfqpoint{6.430878in}{1.402808in}}%
\pgfpathlineto{\pgfqpoint{6.431952in}{1.405303in}}%
\pgfpathlineto{\pgfqpoint{6.433026in}{1.405510in}}%
\pgfpathlineto{\pgfqpoint{6.436247in}{1.395522in}}%
\pgfpathlineto{\pgfqpoint{6.438395in}{1.396146in}}%
\pgfpathlineto{\pgfqpoint{6.439469in}{1.392499in}}%
\pgfpathlineto{\pgfqpoint{6.440542in}{1.392187in}}%
\pgfpathlineto{\pgfqpoint{6.443764in}{1.373954in}}%
\pgfpathlineto{\pgfqpoint{6.444838in}{1.374266in}}%
\pgfpathlineto{\pgfqpoint{6.445912in}{1.375308in}}%
\pgfpathlineto{\pgfqpoint{6.448059in}{1.369760in}}%
\pgfpathlineto{\pgfqpoint{6.451281in}{1.368399in}}%
\pgfpathlineto{\pgfqpoint{6.452355in}{1.363269in}}%
\pgfpathlineto{\pgfqpoint{6.453429in}{1.362223in}}%
\pgfpathlineto{\pgfqpoint{6.454502in}{1.368608in}}%
\pgfpathlineto{\pgfqpoint{6.455576in}{1.362327in}}%
\pgfpathlineto{\pgfqpoint{6.458798in}{1.354903in}}%
\pgfpathlineto{\pgfqpoint{6.459872in}{1.355100in}}%
\pgfpathlineto{\pgfqpoint{6.460945in}{1.363457in}}%
\pgfpathlineto{\pgfqpoint{6.463093in}{1.364243in}}%
\pgfpathlineto{\pgfqpoint{6.466315in}{1.370164in}}%
\pgfpathlineto{\pgfqpoint{6.469536in}{1.341931in}}%
\pgfpathlineto{\pgfqpoint{6.470610in}{1.339079in}}%
\pgfpathlineto{\pgfqpoint{6.473831in}{1.335330in}}%
\pgfpathlineto{\pgfqpoint{6.474905in}{1.341604in}}%
\pgfpathlineto{\pgfqpoint{6.477053in}{1.337633in}}%
\pgfpathlineto{\pgfqpoint{6.478127in}{1.342985in}}%
\pgfpathlineto{\pgfqpoint{6.481348in}{1.336943in}}%
\pgfpathlineto{\pgfqpoint{6.482422in}{1.326469in}}%
\pgfpathlineto{\pgfqpoint{6.483496in}{1.328696in}}%
\pgfpathlineto{\pgfqpoint{6.484570in}{1.324593in}}%
\pgfpathlineto{\pgfqpoint{6.485644in}{1.327006in}}%
\pgfpathlineto{\pgfqpoint{6.488865in}{1.322662in}}%
\pgfpathlineto{\pgfqpoint{6.489939in}{1.325731in}}%
\pgfpathlineto{\pgfqpoint{6.491013in}{1.325331in}}%
\pgfpathlineto{\pgfqpoint{6.492087in}{1.327090in}}%
\pgfpathlineto{\pgfqpoint{6.493161in}{1.325731in}}%
\pgfpathlineto{\pgfqpoint{6.499604in}{1.324937in}}%
\pgfpathlineto{\pgfqpoint{6.500677in}{1.328566in}}%
\pgfpathlineto{\pgfqpoint{6.500677in}{1.328566in}}%
\pgfusepath{stroke}%
\end{pgfscope}%
\begin{pgfscope}%
\pgfsetrectcap%
\pgfsetmiterjoin%
\pgfsetlinewidth{0.803000pt}%
\definecolor{currentstroke}{rgb}{1.000000,1.000000,1.000000}%
\pgfsetstrokecolor{currentstroke}%
\pgfsetdash{}{0pt}%
\pgfpathmoveto{\pgfqpoint{4.034768in}{1.293759in}}%
\pgfpathlineto{\pgfqpoint{4.034768in}{1.694644in}}%
\pgfusepath{stroke}%
\end{pgfscope}%
\begin{pgfscope}%
\pgfsetrectcap%
\pgfsetmiterjoin%
\pgfsetlinewidth{0.803000pt}%
\definecolor{currentstroke}{rgb}{1.000000,1.000000,1.000000}%
\pgfsetstrokecolor{currentstroke}%
\pgfsetdash{}{0pt}%
\pgfpathmoveto{\pgfqpoint{6.618102in}{1.293759in}}%
\pgfpathlineto{\pgfqpoint{6.618102in}{1.694644in}}%
\pgfusepath{stroke}%
\end{pgfscope}%
\begin{pgfscope}%
\pgfsetrectcap%
\pgfsetmiterjoin%
\pgfsetlinewidth{0.803000pt}%
\definecolor{currentstroke}{rgb}{1.000000,1.000000,1.000000}%
\pgfsetstrokecolor{currentstroke}%
\pgfsetdash{}{0pt}%
\pgfpathmoveto{\pgfqpoint{4.034768in}{1.293759in}}%
\pgfpathlineto{\pgfqpoint{6.618102in}{1.293759in}}%
\pgfusepath{stroke}%
\end{pgfscope}%
\begin{pgfscope}%
\pgfsetrectcap%
\pgfsetmiterjoin%
\pgfsetlinewidth{0.803000pt}%
\definecolor{currentstroke}{rgb}{1.000000,1.000000,1.000000}%
\pgfsetstrokecolor{currentstroke}%
\pgfsetdash{}{0pt}%
\pgfpathmoveto{\pgfqpoint{4.034768in}{1.694644in}}%
\pgfpathlineto{\pgfqpoint{6.618102in}{1.694644in}}%
\pgfusepath{stroke}%
\end{pgfscope}%
\begin{pgfscope}%
\definecolor{textcolor}{rgb}{0.150000,0.150000,0.150000}%
\pgfsetstrokecolor{textcolor}%
\pgfsetfillcolor{textcolor}%
\pgftext[x=5.326435in,y=1.777977in,,base]{\color{textcolor}\rmfamily\fontsize{12.000000}{14.400000}\selectfont VZ}%
\end{pgfscope}%
\begin{pgfscope}%
\pgfsetbuttcap%
\pgfsetmiterjoin%
\definecolor{currentfill}{rgb}{0.917647,0.917647,0.949020}%
\pgfsetfillcolor{currentfill}%
\pgfsetlinewidth{0.000000pt}%
\definecolor{currentstroke}{rgb}{0.000000,0.000000,0.000000}%
\pgfsetstrokecolor{currentstroke}%
\pgfsetstrokeopacity{0.000000}%
\pgfsetdash{}{0pt}%
\pgfpathmoveto{\pgfqpoint{0.418102in}{0.331635in}}%
\pgfpathlineto{\pgfqpoint{3.001435in}{0.331635in}}%
\pgfpathlineto{\pgfqpoint{3.001435in}{0.732520in}}%
\pgfpathlineto{\pgfqpoint{0.418102in}{0.732520in}}%
\pgfpathclose%
\pgfusepath{fill}%
\end{pgfscope}%
\begin{pgfscope}%
\pgfpathrectangle{\pgfqpoint{0.418102in}{0.331635in}}{\pgfqpoint{2.583333in}{0.400885in}}%
\pgfusepath{clip}%
\pgfsetroundcap%
\pgfsetroundjoin%
\pgfsetlinewidth{0.803000pt}%
\definecolor{currentstroke}{rgb}{1.000000,1.000000,1.000000}%
\pgfsetstrokecolor{currentstroke}%
\pgfsetdash{}{0pt}%
\pgfpathmoveto{\pgfqpoint{0.533378in}{0.331635in}}%
\pgfpathlineto{\pgfqpoint{0.533378in}{0.732520in}}%
\pgfusepath{stroke}%
\end{pgfscope}%
\begin{pgfscope}%
\definecolor{textcolor}{rgb}{0.150000,0.150000,0.150000}%
\pgfsetstrokecolor{textcolor}%
\pgfsetfillcolor{textcolor}%
\pgftext[x=0.533378in,y=0.234413in,,top]{\color{textcolor}\rmfamily\fontsize{10.000000}{12.000000}\selectfont 2012}%
\end{pgfscope}%
\begin{pgfscope}%
\pgfpathrectangle{\pgfqpoint{0.418102in}{0.331635in}}{\pgfqpoint{2.583333in}{0.400885in}}%
\pgfusepath{clip}%
\pgfsetroundcap%
\pgfsetroundjoin%
\pgfsetlinewidth{0.803000pt}%
\definecolor{currentstroke}{rgb}{1.000000,1.000000,1.000000}%
\pgfsetstrokecolor{currentstroke}%
\pgfsetdash{}{0pt}%
\pgfpathmoveto{\pgfqpoint{0.926403in}{0.331635in}}%
\pgfpathlineto{\pgfqpoint{0.926403in}{0.732520in}}%
\pgfusepath{stroke}%
\end{pgfscope}%
\begin{pgfscope}%
\definecolor{textcolor}{rgb}{0.150000,0.150000,0.150000}%
\pgfsetstrokecolor{textcolor}%
\pgfsetfillcolor{textcolor}%
\pgftext[x=0.926403in,y=0.234413in,,top]{\color{textcolor}\rmfamily\fontsize{10.000000}{12.000000}\selectfont 2013}%
\end{pgfscope}%
\begin{pgfscope}%
\pgfpathrectangle{\pgfqpoint{0.418102in}{0.331635in}}{\pgfqpoint{2.583333in}{0.400885in}}%
\pgfusepath{clip}%
\pgfsetroundcap%
\pgfsetroundjoin%
\pgfsetlinewidth{0.803000pt}%
\definecolor{currentstroke}{rgb}{1.000000,1.000000,1.000000}%
\pgfsetstrokecolor{currentstroke}%
\pgfsetdash{}{0pt}%
\pgfpathmoveto{\pgfqpoint{1.318354in}{0.331635in}}%
\pgfpathlineto{\pgfqpoint{1.318354in}{0.732520in}}%
\pgfusepath{stroke}%
\end{pgfscope}%
\begin{pgfscope}%
\definecolor{textcolor}{rgb}{0.150000,0.150000,0.150000}%
\pgfsetstrokecolor{textcolor}%
\pgfsetfillcolor{textcolor}%
\pgftext[x=1.318354in,y=0.234413in,,top]{\color{textcolor}\rmfamily\fontsize{10.000000}{12.000000}\selectfont 2014}%
\end{pgfscope}%
\begin{pgfscope}%
\pgfpathrectangle{\pgfqpoint{0.418102in}{0.331635in}}{\pgfqpoint{2.583333in}{0.400885in}}%
\pgfusepath{clip}%
\pgfsetroundcap%
\pgfsetroundjoin%
\pgfsetlinewidth{0.803000pt}%
\definecolor{currentstroke}{rgb}{1.000000,1.000000,1.000000}%
\pgfsetstrokecolor{currentstroke}%
\pgfsetdash{}{0pt}%
\pgfpathmoveto{\pgfqpoint{1.710305in}{0.331635in}}%
\pgfpathlineto{\pgfqpoint{1.710305in}{0.732520in}}%
\pgfusepath{stroke}%
\end{pgfscope}%
\begin{pgfscope}%
\definecolor{textcolor}{rgb}{0.150000,0.150000,0.150000}%
\pgfsetstrokecolor{textcolor}%
\pgfsetfillcolor{textcolor}%
\pgftext[x=1.710305in,y=0.234413in,,top]{\color{textcolor}\rmfamily\fontsize{10.000000}{12.000000}\selectfont 2015}%
\end{pgfscope}%
\begin{pgfscope}%
\pgfpathrectangle{\pgfqpoint{0.418102in}{0.331635in}}{\pgfqpoint{2.583333in}{0.400885in}}%
\pgfusepath{clip}%
\pgfsetroundcap%
\pgfsetroundjoin%
\pgfsetlinewidth{0.803000pt}%
\definecolor{currentstroke}{rgb}{1.000000,1.000000,1.000000}%
\pgfsetstrokecolor{currentstroke}%
\pgfsetdash{}{0pt}%
\pgfpathmoveto{\pgfqpoint{2.102256in}{0.331635in}}%
\pgfpathlineto{\pgfqpoint{2.102256in}{0.732520in}}%
\pgfusepath{stroke}%
\end{pgfscope}%
\begin{pgfscope}%
\definecolor{textcolor}{rgb}{0.150000,0.150000,0.150000}%
\pgfsetstrokecolor{textcolor}%
\pgfsetfillcolor{textcolor}%
\pgftext[x=2.102256in,y=0.234413in,,top]{\color{textcolor}\rmfamily\fontsize{10.000000}{12.000000}\selectfont 2016}%
\end{pgfscope}%
\begin{pgfscope}%
\pgfpathrectangle{\pgfqpoint{0.418102in}{0.331635in}}{\pgfqpoint{2.583333in}{0.400885in}}%
\pgfusepath{clip}%
\pgfsetroundcap%
\pgfsetroundjoin%
\pgfsetlinewidth{0.803000pt}%
\definecolor{currentstroke}{rgb}{1.000000,1.000000,1.000000}%
\pgfsetstrokecolor{currentstroke}%
\pgfsetdash{}{0pt}%
\pgfpathmoveto{\pgfqpoint{2.495281in}{0.331635in}}%
\pgfpathlineto{\pgfqpoint{2.495281in}{0.732520in}}%
\pgfusepath{stroke}%
\end{pgfscope}%
\begin{pgfscope}%
\definecolor{textcolor}{rgb}{0.150000,0.150000,0.150000}%
\pgfsetstrokecolor{textcolor}%
\pgfsetfillcolor{textcolor}%
\pgftext[x=2.495281in,y=0.234413in,,top]{\color{textcolor}\rmfamily\fontsize{10.000000}{12.000000}\selectfont 2017}%
\end{pgfscope}%
\begin{pgfscope}%
\pgfpathrectangle{\pgfqpoint{0.418102in}{0.331635in}}{\pgfqpoint{2.583333in}{0.400885in}}%
\pgfusepath{clip}%
\pgfsetroundcap%
\pgfsetroundjoin%
\pgfsetlinewidth{0.803000pt}%
\definecolor{currentstroke}{rgb}{1.000000,1.000000,1.000000}%
\pgfsetstrokecolor{currentstroke}%
\pgfsetdash{}{0pt}%
\pgfpathmoveto{\pgfqpoint{2.887232in}{0.331635in}}%
\pgfpathlineto{\pgfqpoint{2.887232in}{0.732520in}}%
\pgfusepath{stroke}%
\end{pgfscope}%
\begin{pgfscope}%
\definecolor{textcolor}{rgb}{0.150000,0.150000,0.150000}%
\pgfsetstrokecolor{textcolor}%
\pgfsetfillcolor{textcolor}%
\pgftext[x=2.887232in,y=0.234413in,,top]{\color{textcolor}\rmfamily\fontsize{10.000000}{12.000000}\selectfont 2018}%
\end{pgfscope}%
\begin{pgfscope}%
\pgfpathrectangle{\pgfqpoint{0.418102in}{0.331635in}}{\pgfqpoint{2.583333in}{0.400885in}}%
\pgfusepath{clip}%
\pgfsetroundcap%
\pgfsetroundjoin%
\pgfsetlinewidth{0.803000pt}%
\definecolor{currentstroke}{rgb}{1.000000,1.000000,1.000000}%
\pgfsetstrokecolor{currentstroke}%
\pgfsetdash{}{0pt}%
\pgfpathmoveto{\pgfqpoint{0.418102in}{0.455222in}}%
\pgfpathlineto{\pgfqpoint{3.001435in}{0.455222in}}%
\pgfusepath{stroke}%
\end{pgfscope}%
\begin{pgfscope}%
\definecolor{textcolor}{rgb}{0.150000,0.150000,0.150000}%
\pgfsetstrokecolor{textcolor}%
\pgfsetfillcolor{textcolor}%
\pgftext[x=0.100000in,y=0.402461in,left,base]{\color{textcolor}\rmfamily\fontsize{10.000000}{12.000000}\selectfont 2.5}%
\end{pgfscope}%
\begin{pgfscope}%
\pgfpathrectangle{\pgfqpoint{0.418102in}{0.331635in}}{\pgfqpoint{2.583333in}{0.400885in}}%
\pgfusepath{clip}%
\pgfsetroundcap%
\pgfsetroundjoin%
\pgfsetlinewidth{0.803000pt}%
\definecolor{currentstroke}{rgb}{1.000000,1.000000,1.000000}%
\pgfsetstrokecolor{currentstroke}%
\pgfsetdash{}{0pt}%
\pgfpathmoveto{\pgfqpoint{0.418102in}{0.626572in}}%
\pgfpathlineto{\pgfqpoint{3.001435in}{0.626572in}}%
\pgfusepath{stroke}%
\end{pgfscope}%
\begin{pgfscope}%
\definecolor{textcolor}{rgb}{0.150000,0.150000,0.150000}%
\pgfsetstrokecolor{textcolor}%
\pgfsetfillcolor{textcolor}%
\pgftext[x=0.100000in,y=0.573811in,left,base]{\color{textcolor}\rmfamily\fontsize{10.000000}{12.000000}\selectfont 5.0}%
\end{pgfscope}%
\begin{pgfscope}%
\pgfpathrectangle{\pgfqpoint{0.418102in}{0.331635in}}{\pgfqpoint{2.583333in}{0.400885in}}%
\pgfusepath{clip}%
\pgfsetroundcap%
\pgfsetroundjoin%
\pgfsetlinewidth{1.505625pt}%
\definecolor{currentstroke}{rgb}{0.000000,0.000000,0.000000}%
\pgfsetstrokecolor{currentstroke}%
\pgfsetdash{}{0pt}%
\pgfpathmoveto{\pgfqpoint{0.535526in}{0.352412in}}%
\pgfpathlineto{\pgfqpoint{0.536600in}{0.351196in}}%
\pgfpathlineto{\pgfqpoint{0.537674in}{0.351713in}}%
\pgfpathlineto{\pgfqpoint{0.538747in}{0.350922in}}%
\pgfpathlineto{\pgfqpoint{0.544117in}{0.349857in}}%
\pgfpathlineto{\pgfqpoint{0.545190in}{0.351317in}}%
\pgfpathlineto{\pgfqpoint{0.546264in}{0.350922in}}%
\pgfpathlineto{\pgfqpoint{0.551633in}{0.352960in}}%
\pgfpathlineto{\pgfqpoint{0.553781in}{0.350831in}}%
\pgfpathlineto{\pgfqpoint{0.557003in}{0.350161in}}%
\pgfpathlineto{\pgfqpoint{0.558077in}{0.351104in}}%
\pgfpathlineto{\pgfqpoint{0.560224in}{0.351013in}}%
\pgfpathlineto{\pgfqpoint{0.561298in}{0.351135in}}%
\pgfpathlineto{\pgfqpoint{0.564520in}{0.350344in}}%
\pgfpathlineto{\pgfqpoint{0.565593in}{0.350861in}}%
\pgfpathlineto{\pgfqpoint{0.566667in}{0.352108in}}%
\pgfpathlineto{\pgfqpoint{0.567741in}{0.354481in}}%
\pgfpathlineto{\pgfqpoint{0.568815in}{0.355120in}}%
\pgfpathlineto{\pgfqpoint{0.574184in}{0.356002in}}%
\pgfpathlineto{\pgfqpoint{0.575258in}{0.358710in}}%
\pgfpathlineto{\pgfqpoint{0.576332in}{0.359683in}}%
\pgfpathlineto{\pgfqpoint{0.579553in}{0.358892in}}%
\pgfpathlineto{\pgfqpoint{0.580627in}{0.360565in}}%
\pgfpathlineto{\pgfqpoint{0.581701in}{0.361083in}}%
\pgfpathlineto{\pgfqpoint{0.582775in}{0.360200in}}%
\pgfpathlineto{\pgfqpoint{0.583849in}{0.361022in}}%
\pgfpathlineto{\pgfqpoint{0.588144in}{0.360474in}}%
\pgfpathlineto{\pgfqpoint{0.589218in}{0.361934in}}%
\pgfpathlineto{\pgfqpoint{0.590292in}{0.361995in}}%
\pgfpathlineto{\pgfqpoint{0.591366in}{0.362725in}}%
\pgfpathlineto{\pgfqpoint{0.594587in}{0.362269in}}%
\pgfpathlineto{\pgfqpoint{0.595661in}{0.363668in}}%
\pgfpathlineto{\pgfqpoint{0.596735in}{0.361934in}}%
\pgfpathlineto{\pgfqpoint{0.597809in}{0.362482in}}%
\pgfpathlineto{\pgfqpoint{0.598882in}{0.361782in}}%
\pgfpathlineto{\pgfqpoint{0.602104in}{0.361843in}}%
\pgfpathlineto{\pgfqpoint{0.603178in}{0.360900in}}%
\pgfpathlineto{\pgfqpoint{0.604252in}{0.361448in}}%
\pgfpathlineto{\pgfqpoint{0.605325in}{0.362999in}}%
\pgfpathlineto{\pgfqpoint{0.606399in}{0.362452in}}%
\pgfpathlineto{\pgfqpoint{0.609621in}{0.362026in}}%
\pgfpathlineto{\pgfqpoint{0.610695in}{0.362543in}}%
\pgfpathlineto{\pgfqpoint{0.612842in}{0.362360in}}%
\pgfpathlineto{\pgfqpoint{0.613916in}{0.362147in}}%
\pgfpathlineto{\pgfqpoint{0.617138in}{0.363577in}}%
\pgfpathlineto{\pgfqpoint{0.618211in}{0.362026in}}%
\pgfpathlineto{\pgfqpoint{0.620359in}{0.362543in}}%
\pgfpathlineto{\pgfqpoint{0.621433in}{0.363547in}}%
\pgfpathlineto{\pgfqpoint{0.625728in}{0.364246in}}%
\pgfpathlineto{\pgfqpoint{0.628950in}{0.363030in}}%
\pgfpathlineto{\pgfqpoint{0.632171in}{0.363668in}}%
\pgfpathlineto{\pgfqpoint{0.633245in}{0.364612in}}%
\pgfpathlineto{\pgfqpoint{0.634319in}{0.363668in}}%
\pgfpathlineto{\pgfqpoint{0.635393in}{0.365037in}}%
\pgfpathlineto{\pgfqpoint{0.639688in}{0.364034in}}%
\pgfpathlineto{\pgfqpoint{0.640762in}{0.362178in}}%
\pgfpathlineto{\pgfqpoint{0.641836in}{0.362604in}}%
\pgfpathlineto{\pgfqpoint{0.643984in}{0.366467in}}%
\pgfpathlineto{\pgfqpoint{0.647205in}{0.364916in}}%
\pgfpathlineto{\pgfqpoint{0.648279in}{0.365768in}}%
\pgfpathlineto{\pgfqpoint{0.651500in}{0.365037in}}%
\pgfpathlineto{\pgfqpoint{0.654722in}{0.363090in}}%
\pgfpathlineto{\pgfqpoint{0.655796in}{0.363638in}}%
\pgfpathlineto{\pgfqpoint{0.657944in}{0.366498in}}%
\pgfpathlineto{\pgfqpoint{0.659017in}{0.366711in}}%
\pgfpathlineto{\pgfqpoint{0.664387in}{0.365828in}}%
\pgfpathlineto{\pgfqpoint{0.665460in}{0.361965in}}%
\pgfpathlineto{\pgfqpoint{0.666534in}{0.362878in}}%
\pgfpathlineto{\pgfqpoint{0.670830in}{0.363121in}}%
\pgfpathlineto{\pgfqpoint{0.671903in}{0.362634in}}%
\pgfpathlineto{\pgfqpoint{0.672977in}{0.363121in}}%
\pgfpathlineto{\pgfqpoint{0.678346in}{0.362087in}}%
\pgfpathlineto{\pgfqpoint{0.679420in}{0.363303in}}%
\pgfpathlineto{\pgfqpoint{0.681568in}{0.359987in}}%
\pgfpathlineto{\pgfqpoint{0.688011in}{0.364825in}}%
\pgfpathlineto{\pgfqpoint{0.689085in}{0.364551in}}%
\pgfpathlineto{\pgfqpoint{0.693380in}{0.365159in}}%
\pgfpathlineto{\pgfqpoint{0.696602in}{0.359744in}}%
\pgfpathlineto{\pgfqpoint{0.699823in}{0.361022in}}%
\pgfpathlineto{\pgfqpoint{0.700897in}{0.360778in}}%
\pgfpathlineto{\pgfqpoint{0.701971in}{0.362512in}}%
\pgfpathlineto{\pgfqpoint{0.704119in}{0.362665in}}%
\pgfpathlineto{\pgfqpoint{0.707340in}{0.362512in}}%
\pgfpathlineto{\pgfqpoint{0.708414in}{0.363060in}}%
\pgfpathlineto{\pgfqpoint{0.709488in}{0.361539in}}%
\pgfpathlineto{\pgfqpoint{0.711635in}{0.363881in}}%
\pgfpathlineto{\pgfqpoint{0.715931in}{0.365889in}}%
\pgfpathlineto{\pgfqpoint{0.717005in}{0.366711in}}%
\pgfpathlineto{\pgfqpoint{0.718078in}{0.364612in}}%
\pgfpathlineto{\pgfqpoint{0.719152in}{0.368323in}}%
\pgfpathlineto{\pgfqpoint{0.722374in}{0.365828in}}%
\pgfpathlineto{\pgfqpoint{0.723448in}{0.367106in}}%
\pgfpathlineto{\pgfqpoint{0.724521in}{0.367289in}}%
\pgfpathlineto{\pgfqpoint{0.725595in}{0.366011in}}%
\pgfpathlineto{\pgfqpoint{0.726669in}{0.367441in}}%
\pgfpathlineto{\pgfqpoint{0.729891in}{0.369449in}}%
\pgfpathlineto{\pgfqpoint{0.730965in}{0.369266in}}%
\pgfpathlineto{\pgfqpoint{0.733112in}{0.369722in}}%
\pgfpathlineto{\pgfqpoint{0.734186in}{0.368536in}}%
\pgfpathlineto{\pgfqpoint{0.738481in}{0.366528in}}%
\pgfpathlineto{\pgfqpoint{0.739555in}{0.365463in}}%
\pgfpathlineto{\pgfqpoint{0.741703in}{0.367745in}}%
\pgfpathlineto{\pgfqpoint{0.745998in}{0.370544in}}%
\pgfpathlineto{\pgfqpoint{0.747072in}{0.370270in}}%
\pgfpathlineto{\pgfqpoint{0.748146in}{0.368627in}}%
\pgfpathlineto{\pgfqpoint{0.749220in}{0.368871in}}%
\pgfpathlineto{\pgfqpoint{0.752441in}{0.367988in}}%
\pgfpathlineto{\pgfqpoint{0.753515in}{0.366771in}}%
\pgfpathlineto{\pgfqpoint{0.754589in}{0.366467in}}%
\pgfpathlineto{\pgfqpoint{0.756737in}{0.371152in}}%
\pgfpathlineto{\pgfqpoint{0.759958in}{0.372308in}}%
\pgfpathlineto{\pgfqpoint{0.762106in}{0.370027in}}%
\pgfpathlineto{\pgfqpoint{0.764254in}{0.372399in}}%
\pgfpathlineto{\pgfqpoint{0.769623in}{0.372521in}}%
\pgfpathlineto{\pgfqpoint{0.770697in}{0.370726in}}%
\pgfpathlineto{\pgfqpoint{0.771770in}{0.371122in}}%
\pgfpathlineto{\pgfqpoint{0.774992in}{0.370605in}}%
\pgfpathlineto{\pgfqpoint{0.776066in}{0.371669in}}%
\pgfpathlineto{\pgfqpoint{0.779287in}{0.371426in}}%
\pgfpathlineto{\pgfqpoint{0.785730in}{0.370392in}}%
\pgfpathlineto{\pgfqpoint{0.786804in}{0.370057in}}%
\pgfpathlineto{\pgfqpoint{0.792173in}{0.370970in}}%
\pgfpathlineto{\pgfqpoint{0.793247in}{0.370087in}}%
\pgfpathlineto{\pgfqpoint{0.794321in}{0.371152in}}%
\pgfpathlineto{\pgfqpoint{0.798616in}{0.371335in}}%
\pgfpathlineto{\pgfqpoint{0.799690in}{0.370696in}}%
\pgfpathlineto{\pgfqpoint{0.800764in}{0.372004in}}%
\pgfpathlineto{\pgfqpoint{0.801838in}{0.372126in}}%
\pgfpathlineto{\pgfqpoint{0.805059in}{0.371426in}}%
\pgfpathlineto{\pgfqpoint{0.807207in}{0.374681in}}%
\pgfpathlineto{\pgfqpoint{0.808281in}{0.375715in}}%
\pgfpathlineto{\pgfqpoint{0.809355in}{0.375229in}}%
\pgfpathlineto{\pgfqpoint{0.813650in}{0.374833in}}%
\pgfpathlineto{\pgfqpoint{0.814724in}{0.375594in}}%
\pgfpathlineto{\pgfqpoint{0.820093in}{0.374864in}}%
\pgfpathlineto{\pgfqpoint{0.821167in}{0.375442in}}%
\pgfpathlineto{\pgfqpoint{0.822241in}{0.374042in}}%
\pgfpathlineto{\pgfqpoint{0.824388in}{0.375259in}}%
\pgfpathlineto{\pgfqpoint{0.827610in}{0.376902in}}%
\pgfpathlineto{\pgfqpoint{0.828684in}{0.376415in}}%
\pgfpathlineto{\pgfqpoint{0.830832in}{0.378910in}}%
\pgfpathlineto{\pgfqpoint{0.831905in}{0.379305in}}%
\pgfpathlineto{\pgfqpoint{0.835127in}{0.378271in}}%
\pgfpathlineto{\pgfqpoint{0.836201in}{0.376932in}}%
\pgfpathlineto{\pgfqpoint{0.837275in}{0.377389in}}%
\pgfpathlineto{\pgfqpoint{0.838348in}{0.378484in}}%
\pgfpathlineto{\pgfqpoint{0.842644in}{0.379062in}}%
\pgfpathlineto{\pgfqpoint{0.844791in}{0.380887in}}%
\pgfpathlineto{\pgfqpoint{0.845865in}{0.380431in}}%
\pgfpathlineto{\pgfqpoint{0.846939in}{0.379123in}}%
\pgfpathlineto{\pgfqpoint{0.850161in}{0.378453in}}%
\pgfpathlineto{\pgfqpoint{0.851234in}{0.376841in}}%
\pgfpathlineto{\pgfqpoint{0.852308in}{0.376750in}}%
\pgfpathlineto{\pgfqpoint{0.854456in}{0.377997in}}%
\pgfpathlineto{\pgfqpoint{0.859825in}{0.378301in}}%
\pgfpathlineto{\pgfqpoint{0.860899in}{0.381769in}}%
\pgfpathlineto{\pgfqpoint{0.865194in}{0.380157in}}%
\pgfpathlineto{\pgfqpoint{0.866268in}{0.381769in}}%
\pgfpathlineto{\pgfqpoint{0.868416in}{0.380553in}}%
\pgfpathlineto{\pgfqpoint{0.869490in}{0.381131in}}%
\pgfpathlineto{\pgfqpoint{0.872711in}{0.381343in}}%
\pgfpathlineto{\pgfqpoint{0.875933in}{0.380035in}}%
\pgfpathlineto{\pgfqpoint{0.877007in}{0.381891in}}%
\pgfpathlineto{\pgfqpoint{0.882376in}{0.384599in}}%
\pgfpathlineto{\pgfqpoint{0.884523in}{0.385602in}}%
\pgfpathlineto{\pgfqpoint{0.889893in}{0.385024in}}%
\pgfpathlineto{\pgfqpoint{0.892040in}{0.386698in}}%
\pgfpathlineto{\pgfqpoint{0.897409in}{0.385572in}}%
\pgfpathlineto{\pgfqpoint{0.899557in}{0.385907in}}%
\pgfpathlineto{\pgfqpoint{0.903853in}{0.386150in}}%
\pgfpathlineto{\pgfqpoint{0.906000in}{0.384964in}}%
\pgfpathlineto{\pgfqpoint{0.907074in}{0.384720in}}%
\pgfpathlineto{\pgfqpoint{0.911369in}{0.387397in}}%
\pgfpathlineto{\pgfqpoint{0.912443in}{0.386333in}}%
\pgfpathlineto{\pgfqpoint{0.913517in}{0.388584in}}%
\pgfpathlineto{\pgfqpoint{0.914591in}{0.387428in}}%
\pgfpathlineto{\pgfqpoint{0.917812in}{0.387702in}}%
\pgfpathlineto{\pgfqpoint{0.922108in}{0.385968in}}%
\pgfpathlineto{\pgfqpoint{0.925329in}{0.387975in}}%
\pgfpathlineto{\pgfqpoint{0.927477in}{0.390592in}}%
\pgfpathlineto{\pgfqpoint{0.928551in}{0.390683in}}%
\pgfpathlineto{\pgfqpoint{0.929625in}{0.391535in}}%
\pgfpathlineto{\pgfqpoint{0.932846in}{0.392326in}}%
\pgfpathlineto{\pgfqpoint{0.934994in}{0.395003in}}%
\pgfpathlineto{\pgfqpoint{0.936068in}{0.394121in}}%
\pgfpathlineto{\pgfqpoint{0.937142in}{0.394546in}}%
\pgfpathlineto{\pgfqpoint{0.943585in}{0.393695in}}%
\pgfpathlineto{\pgfqpoint{0.944658in}{0.392569in}}%
\pgfpathlineto{\pgfqpoint{0.952175in}{0.393664in}}%
\pgfpathlineto{\pgfqpoint{0.955397in}{0.391261in}}%
\pgfpathlineto{\pgfqpoint{0.956471in}{0.391474in}}%
\pgfpathlineto{\pgfqpoint{0.957544in}{0.390348in}}%
\pgfpathlineto{\pgfqpoint{0.958618in}{0.392326in}}%
\pgfpathlineto{\pgfqpoint{0.959692in}{0.392782in}}%
\pgfpathlineto{\pgfqpoint{0.962914in}{0.391474in}}%
\pgfpathlineto{\pgfqpoint{0.965061in}{0.394334in}}%
\pgfpathlineto{\pgfqpoint{0.966135in}{0.391748in}}%
\pgfpathlineto{\pgfqpoint{0.967209in}{0.392265in}}%
\pgfpathlineto{\pgfqpoint{0.970431in}{0.391231in}}%
\pgfpathlineto{\pgfqpoint{0.971504in}{0.391565in}}%
\pgfpathlineto{\pgfqpoint{0.972578in}{0.391078in}}%
\pgfpathlineto{\pgfqpoint{0.974726in}{0.393299in}}%
\pgfpathlineto{\pgfqpoint{0.979021in}{0.393056in}}%
\pgfpathlineto{\pgfqpoint{0.980095in}{0.391535in}}%
\pgfpathlineto{\pgfqpoint{0.982243in}{0.394303in}}%
\pgfpathlineto{\pgfqpoint{0.985464in}{0.391839in}}%
\pgfpathlineto{\pgfqpoint{0.987612in}{0.394607in}}%
\pgfpathlineto{\pgfqpoint{0.989760in}{0.393360in}}%
\pgfpathlineto{\pgfqpoint{0.997276in}{0.395155in}}%
\pgfpathlineto{\pgfqpoint{1.000498in}{0.395520in}}%
\pgfpathlineto{\pgfqpoint{1.002646in}{0.394273in}}%
\pgfpathlineto{\pgfqpoint{1.003720in}{0.395155in}}%
\pgfpathlineto{\pgfqpoint{1.004793in}{0.393695in}}%
\pgfpathlineto{\pgfqpoint{1.008015in}{0.393360in}}%
\pgfpathlineto{\pgfqpoint{1.009089in}{0.391930in}}%
\pgfpathlineto{\pgfqpoint{1.010163in}{0.394242in}}%
\pgfpathlineto{\pgfqpoint{1.011236in}{0.393177in}}%
\pgfpathlineto{\pgfqpoint{1.012310in}{0.394759in}}%
\pgfpathlineto{\pgfqpoint{1.015532in}{0.397467in}}%
\pgfpathlineto{\pgfqpoint{1.016606in}{0.399870in}}%
\pgfpathlineto{\pgfqpoint{1.018753in}{0.401513in}}%
\pgfpathlineto{\pgfqpoint{1.023049in}{0.399779in}}%
\pgfpathlineto{\pgfqpoint{1.024122in}{0.400205in}}%
\pgfpathlineto{\pgfqpoint{1.025196in}{0.397984in}}%
\pgfpathlineto{\pgfqpoint{1.026270in}{0.399171in}}%
\pgfpathlineto{\pgfqpoint{1.027344in}{0.398258in}}%
\pgfpathlineto{\pgfqpoint{1.030565in}{0.399049in}}%
\pgfpathlineto{\pgfqpoint{1.031639in}{0.397984in}}%
\pgfpathlineto{\pgfqpoint{1.032713in}{0.399536in}}%
\pgfpathlineto{\pgfqpoint{1.033787in}{0.399962in}}%
\pgfpathlineto{\pgfqpoint{1.038082in}{0.395581in}}%
\pgfpathlineto{\pgfqpoint{1.039156in}{0.397954in}}%
\pgfpathlineto{\pgfqpoint{1.040230in}{0.396250in}}%
\pgfpathlineto{\pgfqpoint{1.041304in}{0.395672in}}%
\pgfpathlineto{\pgfqpoint{1.042378in}{0.397437in}}%
\pgfpathlineto{\pgfqpoint{1.045599in}{0.397132in}}%
\pgfpathlineto{\pgfqpoint{1.048821in}{0.400752in}}%
\pgfpathlineto{\pgfqpoint{1.049895in}{0.399718in}}%
\pgfpathlineto{\pgfqpoint{1.054190in}{0.400570in}}%
\pgfpathlineto{\pgfqpoint{1.055264in}{0.398866in}}%
\pgfpathlineto{\pgfqpoint{1.056338in}{0.405377in}}%
\pgfpathlineto{\pgfqpoint{1.057411in}{0.408236in}}%
\pgfpathlineto{\pgfqpoint{1.060633in}{0.407780in}}%
\pgfpathlineto{\pgfqpoint{1.061707in}{0.408419in}}%
\pgfpathlineto{\pgfqpoint{1.064928in}{0.407689in}}%
\pgfpathlineto{\pgfqpoint{1.068150in}{0.407810in}}%
\pgfpathlineto{\pgfqpoint{1.069224in}{0.408784in}}%
\pgfpathlineto{\pgfqpoint{1.070297in}{0.410700in}}%
\pgfpathlineto{\pgfqpoint{1.071371in}{0.409331in}}%
\pgfpathlineto{\pgfqpoint{1.072445in}{0.412647in}}%
\pgfpathlineto{\pgfqpoint{1.078888in}{0.408145in}}%
\pgfpathlineto{\pgfqpoint{1.079962in}{0.409788in}}%
\pgfpathlineto{\pgfqpoint{1.084257in}{0.409484in}}%
\pgfpathlineto{\pgfqpoint{1.085331in}{0.408175in}}%
\pgfpathlineto{\pgfqpoint{1.086405in}{0.410183in}}%
\pgfpathlineto{\pgfqpoint{1.087479in}{0.408175in}}%
\pgfpathlineto{\pgfqpoint{1.091774in}{0.409605in}}%
\pgfpathlineto{\pgfqpoint{1.092848in}{0.407537in}}%
\pgfpathlineto{\pgfqpoint{1.093922in}{0.409027in}}%
\pgfpathlineto{\pgfqpoint{1.098217in}{0.411096in}}%
\pgfpathlineto{\pgfqpoint{1.099291in}{0.409240in}}%
\pgfpathlineto{\pgfqpoint{1.100365in}{0.408875in}}%
\pgfpathlineto{\pgfqpoint{1.101439in}{0.411035in}}%
\pgfpathlineto{\pgfqpoint{1.102513in}{0.410122in}}%
\pgfpathlineto{\pgfqpoint{1.105734in}{0.411096in}}%
\pgfpathlineto{\pgfqpoint{1.106808in}{0.412282in}}%
\pgfpathlineto{\pgfqpoint{1.107882in}{0.411309in}}%
\pgfpathlineto{\pgfqpoint{1.108956in}{0.408388in}}%
\pgfpathlineto{\pgfqpoint{1.110030in}{0.409118in}}%
\pgfpathlineto{\pgfqpoint{1.113251in}{0.408267in}}%
\pgfpathlineto{\pgfqpoint{1.116473in}{0.412434in}}%
\pgfpathlineto{\pgfqpoint{1.117546in}{0.411400in}}%
\pgfpathlineto{\pgfqpoint{1.122916in}{0.414381in}}%
\pgfpathlineto{\pgfqpoint{1.125063in}{0.416998in}}%
\pgfpathlineto{\pgfqpoint{1.130432in}{0.414138in}}%
\pgfpathlineto{\pgfqpoint{1.131506in}{0.416420in}}%
\pgfpathlineto{\pgfqpoint{1.132580in}{0.416937in}}%
\pgfpathlineto{\pgfqpoint{1.137949in}{0.416207in}}%
\pgfpathlineto{\pgfqpoint{1.139023in}{0.417150in}}%
\pgfpathlineto{\pgfqpoint{1.140097in}{0.416389in}}%
\pgfpathlineto{\pgfqpoint{1.143319in}{0.417302in}}%
\pgfpathlineto{\pgfqpoint{1.145466in}{0.414168in}}%
\pgfpathlineto{\pgfqpoint{1.146540in}{0.419675in}}%
\pgfpathlineto{\pgfqpoint{1.147614in}{0.418671in}}%
\pgfpathlineto{\pgfqpoint{1.151909in}{0.417454in}}%
\pgfpathlineto{\pgfqpoint{1.152983in}{0.407384in}}%
\pgfpathlineto{\pgfqpoint{1.154057in}{0.408875in}}%
\pgfpathlineto{\pgfqpoint{1.155131in}{0.412252in}}%
\pgfpathlineto{\pgfqpoint{1.158352in}{0.412647in}}%
\pgfpathlineto{\pgfqpoint{1.160500in}{0.410427in}}%
\pgfpathlineto{\pgfqpoint{1.162648in}{0.409331in}}%
\pgfpathlineto{\pgfqpoint{1.168017in}{0.409240in}}%
\pgfpathlineto{\pgfqpoint{1.169091in}{0.406137in}}%
\pgfpathlineto{\pgfqpoint{1.170164in}{0.405559in}}%
\pgfpathlineto{\pgfqpoint{1.173386in}{0.406867in}}%
\pgfpathlineto{\pgfqpoint{1.174460in}{0.405620in}}%
\pgfpathlineto{\pgfqpoint{1.175534in}{0.409271in}}%
\pgfpathlineto{\pgfqpoint{1.177681in}{0.409757in}}%
\pgfpathlineto{\pgfqpoint{1.181977in}{0.406289in}}%
\pgfpathlineto{\pgfqpoint{1.184124in}{0.407141in}}%
\pgfpathlineto{\pgfqpoint{1.185198in}{0.406472in}}%
\pgfpathlineto{\pgfqpoint{1.189494in}{0.408297in}}%
\pgfpathlineto{\pgfqpoint{1.191641in}{0.407749in}}%
\pgfpathlineto{\pgfqpoint{1.195937in}{0.409392in}}%
\pgfpathlineto{\pgfqpoint{1.197010in}{0.413621in}}%
\pgfpathlineto{\pgfqpoint{1.198084in}{0.414838in}}%
\pgfpathlineto{\pgfqpoint{1.199158in}{0.413955in}}%
\pgfpathlineto{\pgfqpoint{1.200232in}{0.416724in}}%
\pgfpathlineto{\pgfqpoint{1.203453in}{0.416998in}}%
\pgfpathlineto{\pgfqpoint{1.207749in}{0.423630in}}%
\pgfpathlineto{\pgfqpoint{1.210970in}{0.421804in}}%
\pgfpathlineto{\pgfqpoint{1.213118in}{0.418519in}}%
\pgfpathlineto{\pgfqpoint{1.214192in}{0.419918in}}%
\pgfpathlineto{\pgfqpoint{1.218487in}{0.418215in}}%
\pgfpathlineto{\pgfqpoint{1.219561in}{0.419705in}}%
\pgfpathlineto{\pgfqpoint{1.220635in}{0.418701in}}%
\pgfpathlineto{\pgfqpoint{1.221709in}{0.416480in}}%
\pgfpathlineto{\pgfqpoint{1.222783in}{0.417758in}}%
\pgfpathlineto{\pgfqpoint{1.226004in}{0.414838in}}%
\pgfpathlineto{\pgfqpoint{1.227078in}{0.412191in}}%
\pgfpathlineto{\pgfqpoint{1.228152in}{0.413104in}}%
\pgfpathlineto{\pgfqpoint{1.230299in}{0.418975in}}%
\pgfpathlineto{\pgfqpoint{1.233521in}{0.419857in}}%
\pgfpathlineto{\pgfqpoint{1.234595in}{0.418397in}}%
\pgfpathlineto{\pgfqpoint{1.236742in}{0.423265in}}%
\pgfpathlineto{\pgfqpoint{1.237816in}{0.424786in}}%
\pgfpathlineto{\pgfqpoint{1.242112in}{0.424451in}}%
\pgfpathlineto{\pgfqpoint{1.243185in}{0.423690in}}%
\pgfpathlineto{\pgfqpoint{1.244259in}{0.426520in}}%
\pgfpathlineto{\pgfqpoint{1.250702in}{0.427158in}}%
\pgfpathlineto{\pgfqpoint{1.251776in}{0.422108in}}%
\pgfpathlineto{\pgfqpoint{1.252850in}{0.423873in}}%
\pgfpathlineto{\pgfqpoint{1.256072in}{0.421956in}}%
\pgfpathlineto{\pgfqpoint{1.258219in}{0.423782in}}%
\pgfpathlineto{\pgfqpoint{1.259293in}{0.421713in}}%
\pgfpathlineto{\pgfqpoint{1.260367in}{0.423265in}}%
\pgfpathlineto{\pgfqpoint{1.263588in}{0.423934in}}%
\pgfpathlineto{\pgfqpoint{1.264662in}{0.423234in}}%
\pgfpathlineto{\pgfqpoint{1.265736in}{0.425455in}}%
\pgfpathlineto{\pgfqpoint{1.266810in}{0.425698in}}%
\pgfpathlineto{\pgfqpoint{1.267884in}{0.427006in}}%
\pgfpathlineto{\pgfqpoint{1.271105in}{0.425546in}}%
\pgfpathlineto{\pgfqpoint{1.272179in}{0.423751in}}%
\pgfpathlineto{\pgfqpoint{1.273253in}{0.424268in}}%
\pgfpathlineto{\pgfqpoint{1.274327in}{0.426733in}}%
\pgfpathlineto{\pgfqpoint{1.275401in}{0.427098in}}%
\pgfpathlineto{\pgfqpoint{1.278622in}{0.426976in}}%
\pgfpathlineto{\pgfqpoint{1.280770in}{0.428436in}}%
\pgfpathlineto{\pgfqpoint{1.282918in}{0.428041in}}%
\pgfpathlineto{\pgfqpoint{1.286139in}{0.429227in}}%
\pgfpathlineto{\pgfqpoint{1.287213in}{0.426854in}}%
\pgfpathlineto{\pgfqpoint{1.288287in}{0.427554in}}%
\pgfpathlineto{\pgfqpoint{1.289361in}{0.426824in}}%
\pgfpathlineto{\pgfqpoint{1.293656in}{0.426733in}}%
\pgfpathlineto{\pgfqpoint{1.294730in}{0.425211in}}%
\pgfpathlineto{\pgfqpoint{1.295804in}{0.429623in}}%
\pgfpathlineto{\pgfqpoint{1.296877in}{0.428041in}}%
\pgfpathlineto{\pgfqpoint{1.297951in}{0.430809in}}%
\pgfpathlineto{\pgfqpoint{1.301173in}{0.431083in}}%
\pgfpathlineto{\pgfqpoint{1.302247in}{0.435007in}}%
\pgfpathlineto{\pgfqpoint{1.303320in}{0.436468in}}%
\pgfpathlineto{\pgfqpoint{1.305468in}{0.436924in}}%
\pgfpathlineto{\pgfqpoint{1.311911in}{0.439905in}}%
\pgfpathlineto{\pgfqpoint{1.312985in}{0.439540in}}%
\pgfpathlineto{\pgfqpoint{1.316207in}{0.440422in}}%
\pgfpathlineto{\pgfqpoint{1.317280in}{0.441670in}}%
\pgfpathlineto{\pgfqpoint{1.319428in}{0.440483in}}%
\pgfpathlineto{\pgfqpoint{1.320502in}{0.440605in}}%
\pgfpathlineto{\pgfqpoint{1.323723in}{0.439662in}}%
\pgfpathlineto{\pgfqpoint{1.325871in}{0.441335in}}%
\pgfpathlineto{\pgfqpoint{1.328019in}{0.440574in}}%
\pgfpathlineto{\pgfqpoint{1.331240in}{0.438993in}}%
\pgfpathlineto{\pgfqpoint{1.332314in}{0.441670in}}%
\pgfpathlineto{\pgfqpoint{1.333388in}{0.442430in}}%
\pgfpathlineto{\pgfqpoint{1.334462in}{0.441031in}}%
\pgfpathlineto{\pgfqpoint{1.335536in}{0.448423in}}%
\pgfpathlineto{\pgfqpoint{1.339831in}{0.448241in}}%
\pgfpathlineto{\pgfqpoint{1.340905in}{0.448971in}}%
\pgfpathlineto{\pgfqpoint{1.343052in}{0.440666in}}%
\pgfpathlineto{\pgfqpoint{1.346274in}{0.437106in}}%
\pgfpathlineto{\pgfqpoint{1.347348in}{0.440453in}}%
\pgfpathlineto{\pgfqpoint{1.348422in}{0.437745in}}%
\pgfpathlineto{\pgfqpoint{1.349496in}{0.440392in}}%
\pgfpathlineto{\pgfqpoint{1.350569in}{0.436528in}}%
\pgfpathlineto{\pgfqpoint{1.353791in}{0.435159in}}%
\pgfpathlineto{\pgfqpoint{1.355939in}{0.436680in}}%
\pgfpathlineto{\pgfqpoint{1.358086in}{0.441031in}}%
\pgfpathlineto{\pgfqpoint{1.361308in}{0.440179in}}%
\pgfpathlineto{\pgfqpoint{1.362382in}{0.441426in}}%
\pgfpathlineto{\pgfqpoint{1.363455in}{0.443890in}}%
\pgfpathlineto{\pgfqpoint{1.364529in}{0.443799in}}%
\pgfpathlineto{\pgfqpoint{1.365603in}{0.445199in}}%
\pgfpathlineto{\pgfqpoint{1.369898in}{0.445229in}}%
\pgfpathlineto{\pgfqpoint{1.370972in}{0.443677in}}%
\pgfpathlineto{\pgfqpoint{1.373120in}{0.443312in}}%
\pgfpathlineto{\pgfqpoint{1.377415in}{0.445989in}}%
\pgfpathlineto{\pgfqpoint{1.378489in}{0.445259in}}%
\pgfpathlineto{\pgfqpoint{1.380637in}{0.445138in}}%
\pgfpathlineto{\pgfqpoint{1.383858in}{0.441943in}}%
\pgfpathlineto{\pgfqpoint{1.384932in}{0.444833in}}%
\pgfpathlineto{\pgfqpoint{1.386006in}{0.442917in}}%
\pgfpathlineto{\pgfqpoint{1.388154in}{0.444864in}}%
\pgfpathlineto{\pgfqpoint{1.391375in}{0.444833in}}%
\pgfpathlineto{\pgfqpoint{1.392449in}{0.446020in}}%
\pgfpathlineto{\pgfqpoint{1.393523in}{0.445259in}}%
\pgfpathlineto{\pgfqpoint{1.394597in}{0.441457in}}%
\pgfpathlineto{\pgfqpoint{1.395671in}{0.441457in}}%
\pgfpathlineto{\pgfqpoint{1.398892in}{0.443617in}}%
\pgfpathlineto{\pgfqpoint{1.399966in}{0.445472in}}%
\pgfpathlineto{\pgfqpoint{1.402114in}{0.442217in}}%
\pgfpathlineto{\pgfqpoint{1.403187in}{0.443312in}}%
\pgfpathlineto{\pgfqpoint{1.406409in}{0.441457in}}%
\pgfpathlineto{\pgfqpoint{1.408557in}{0.437836in}}%
\pgfpathlineto{\pgfqpoint{1.409630in}{0.437897in}}%
\pgfpathlineto{\pgfqpoint{1.410704in}{0.435251in}}%
\pgfpathlineto{\pgfqpoint{1.413926in}{0.437958in}}%
\pgfpathlineto{\pgfqpoint{1.415000in}{0.437137in}}%
\pgfpathlineto{\pgfqpoint{1.417147in}{0.437350in}}%
\pgfpathlineto{\pgfqpoint{1.418221in}{0.432117in}}%
\pgfpathlineto{\pgfqpoint{1.422517in}{0.428467in}}%
\pgfpathlineto{\pgfqpoint{1.423590in}{0.432026in}}%
\pgfpathlineto{\pgfqpoint{1.425738in}{0.424238in}}%
\pgfpathlineto{\pgfqpoint{1.428960in}{0.427341in}}%
\pgfpathlineto{\pgfqpoint{1.430033in}{0.429531in}}%
\pgfpathlineto{\pgfqpoint{1.431107in}{0.433304in}}%
\pgfpathlineto{\pgfqpoint{1.432181in}{0.432300in}}%
\pgfpathlineto{\pgfqpoint{1.437550in}{0.433760in}}%
\pgfpathlineto{\pgfqpoint{1.438624in}{0.432939in}}%
\pgfpathlineto{\pgfqpoint{1.439698in}{0.433334in}}%
\pgfpathlineto{\pgfqpoint{1.440772in}{0.425881in}}%
\pgfpathlineto{\pgfqpoint{1.446141in}{0.428497in}}%
\pgfpathlineto{\pgfqpoint{1.447215in}{0.430992in}}%
\pgfpathlineto{\pgfqpoint{1.448289in}{0.429775in}}%
\pgfpathlineto{\pgfqpoint{1.451510in}{0.431722in}}%
\pgfpathlineto{\pgfqpoint{1.452584in}{0.430444in}}%
\pgfpathlineto{\pgfqpoint{1.454732in}{0.434399in}}%
\pgfpathlineto{\pgfqpoint{1.455806in}{0.434338in}}%
\pgfpathlineto{\pgfqpoint{1.460101in}{0.435251in}}%
\pgfpathlineto{\pgfqpoint{1.461175in}{0.434794in}}%
\pgfpathlineto{\pgfqpoint{1.462249in}{0.433060in}}%
\pgfpathlineto{\pgfqpoint{1.463322in}{0.434764in}}%
\pgfpathlineto{\pgfqpoint{1.466544in}{0.435159in}}%
\pgfpathlineto{\pgfqpoint{1.467618in}{0.433425in}}%
\pgfpathlineto{\pgfqpoint{1.468692in}{0.434977in}}%
\pgfpathlineto{\pgfqpoint{1.469765in}{0.434460in}}%
\pgfpathlineto{\pgfqpoint{1.470839in}{0.436407in}}%
\pgfpathlineto{\pgfqpoint{1.475135in}{0.438141in}}%
\pgfpathlineto{\pgfqpoint{1.476208in}{0.437654in}}%
\pgfpathlineto{\pgfqpoint{1.478356in}{0.438384in}}%
\pgfpathlineto{\pgfqpoint{1.481578in}{0.437441in}}%
\pgfpathlineto{\pgfqpoint{1.482651in}{0.435859in}}%
\pgfpathlineto{\pgfqpoint{1.484799in}{0.436498in}}%
\pgfpathlineto{\pgfqpoint{1.485873in}{0.437076in}}%
\pgfpathlineto{\pgfqpoint{1.489095in}{0.436772in}}%
\pgfpathlineto{\pgfqpoint{1.490168in}{0.437958in}}%
\pgfpathlineto{\pgfqpoint{1.492316in}{0.436194in}}%
\pgfpathlineto{\pgfqpoint{1.496611in}{0.435068in}}%
\pgfpathlineto{\pgfqpoint{1.498759in}{0.435798in}}%
\pgfpathlineto{\pgfqpoint{1.500907in}{0.434551in}}%
\pgfpathlineto{\pgfqpoint{1.504128in}{0.434521in}}%
\pgfpathlineto{\pgfqpoint{1.505202in}{0.433334in}}%
\pgfpathlineto{\pgfqpoint{1.506276in}{0.434247in}}%
\pgfpathlineto{\pgfqpoint{1.508424in}{0.434399in}}%
\pgfpathlineto{\pgfqpoint{1.511645in}{0.435433in}}%
\pgfpathlineto{\pgfqpoint{1.512719in}{0.437958in}}%
\pgfpathlineto{\pgfqpoint{1.513793in}{0.438384in}}%
\pgfpathlineto{\pgfqpoint{1.514867in}{0.439601in}}%
\pgfpathlineto{\pgfqpoint{1.519162in}{0.439753in}}%
\pgfpathlineto{\pgfqpoint{1.520236in}{0.438688in}}%
\pgfpathlineto{\pgfqpoint{1.521310in}{0.439327in}}%
\pgfpathlineto{\pgfqpoint{1.522384in}{0.438901in}}%
\pgfpathlineto{\pgfqpoint{1.527753in}{0.443373in}}%
\pgfpathlineto{\pgfqpoint{1.528827in}{0.444012in}}%
\pgfpathlineto{\pgfqpoint{1.529900in}{0.440574in}}%
\pgfpathlineto{\pgfqpoint{1.530974in}{0.442248in}}%
\pgfpathlineto{\pgfqpoint{1.534196in}{0.441548in}}%
\pgfpathlineto{\pgfqpoint{1.535270in}{0.443008in}}%
\pgfpathlineto{\pgfqpoint{1.536343in}{0.442978in}}%
\pgfpathlineto{\pgfqpoint{1.537417in}{0.444073in}}%
\pgfpathlineto{\pgfqpoint{1.538491in}{0.438354in}}%
\pgfpathlineto{\pgfqpoint{1.543860in}{0.437836in}}%
\pgfpathlineto{\pgfqpoint{1.544934in}{0.435646in}}%
\pgfpathlineto{\pgfqpoint{1.546008in}{0.436224in}}%
\pgfpathlineto{\pgfqpoint{1.549229in}{0.436468in}}%
\pgfpathlineto{\pgfqpoint{1.550303in}{0.435311in}}%
\pgfpathlineto{\pgfqpoint{1.551377in}{0.435403in}}%
\pgfpathlineto{\pgfqpoint{1.552451in}{0.434216in}}%
\pgfpathlineto{\pgfqpoint{1.553525in}{0.435251in}}%
\pgfpathlineto{\pgfqpoint{1.557820in}{0.435129in}}%
\pgfpathlineto{\pgfqpoint{1.558894in}{0.437106in}}%
\pgfpathlineto{\pgfqpoint{1.559968in}{0.437897in}}%
\pgfpathlineto{\pgfqpoint{1.561042in}{0.436194in}}%
\pgfpathlineto{\pgfqpoint{1.566411in}{0.440574in}}%
\pgfpathlineto{\pgfqpoint{1.568559in}{0.440483in}}%
\pgfpathlineto{\pgfqpoint{1.572854in}{0.440939in}}%
\pgfpathlineto{\pgfqpoint{1.573928in}{0.441244in}}%
\pgfpathlineto{\pgfqpoint{1.576075in}{0.437897in}}%
\pgfpathlineto{\pgfqpoint{1.581445in}{0.439631in}}%
\pgfpathlineto{\pgfqpoint{1.583592in}{0.439114in}}%
\pgfpathlineto{\pgfqpoint{1.586814in}{0.440240in}}%
\pgfpathlineto{\pgfqpoint{1.587888in}{0.439175in}}%
\pgfpathlineto{\pgfqpoint{1.588961in}{0.441061in}}%
\pgfpathlineto{\pgfqpoint{1.591109in}{0.438993in}}%
\pgfpathlineto{\pgfqpoint{1.594331in}{0.439418in}}%
\pgfpathlineto{\pgfqpoint{1.595405in}{0.441365in}}%
\pgfpathlineto{\pgfqpoint{1.596478in}{0.440088in}}%
\pgfpathlineto{\pgfqpoint{1.597552in}{0.440727in}}%
\pgfpathlineto{\pgfqpoint{1.598626in}{0.440605in}}%
\pgfpathlineto{\pgfqpoint{1.602921in}{0.437836in}}%
\pgfpathlineto{\pgfqpoint{1.603995in}{0.439266in}}%
\pgfpathlineto{\pgfqpoint{1.605069in}{0.436589in}}%
\pgfpathlineto{\pgfqpoint{1.606143in}{0.437471in}}%
\pgfpathlineto{\pgfqpoint{1.609364in}{0.436741in}}%
\pgfpathlineto{\pgfqpoint{1.610438in}{0.438506in}}%
\pgfpathlineto{\pgfqpoint{1.611512in}{0.436072in}}%
\pgfpathlineto{\pgfqpoint{1.612586in}{0.435859in}}%
\pgfpathlineto{\pgfqpoint{1.613660in}{0.437502in}}%
\pgfpathlineto{\pgfqpoint{1.616881in}{0.437350in}}%
\pgfpathlineto{\pgfqpoint{1.617955in}{0.434642in}}%
\pgfpathlineto{\pgfqpoint{1.619029in}{0.437776in}}%
\pgfpathlineto{\pgfqpoint{1.621177in}{0.432421in}}%
\pgfpathlineto{\pgfqpoint{1.624398in}{0.431935in}}%
\pgfpathlineto{\pgfqpoint{1.626546in}{0.429014in}}%
\pgfpathlineto{\pgfqpoint{1.628694in}{0.433182in}}%
\pgfpathlineto{\pgfqpoint{1.631915in}{0.434521in}}%
\pgfpathlineto{\pgfqpoint{1.632989in}{0.438475in}}%
\pgfpathlineto{\pgfqpoint{1.634063in}{0.436741in}}%
\pgfpathlineto{\pgfqpoint{1.635137in}{0.439175in}}%
\pgfpathlineto{\pgfqpoint{1.636210in}{0.438597in}}%
\pgfpathlineto{\pgfqpoint{1.639432in}{0.438536in}}%
\pgfpathlineto{\pgfqpoint{1.640506in}{0.440939in}}%
\pgfpathlineto{\pgfqpoint{1.641580in}{0.439449in}}%
\pgfpathlineto{\pgfqpoint{1.642653in}{0.455390in}}%
\pgfpathlineto{\pgfqpoint{1.643727in}{0.458858in}}%
\pgfpathlineto{\pgfqpoint{1.646949in}{0.458919in}}%
\pgfpathlineto{\pgfqpoint{1.648023in}{0.459953in}}%
\pgfpathlineto{\pgfqpoint{1.649096in}{0.464729in}}%
\pgfpathlineto{\pgfqpoint{1.650170in}{0.465125in}}%
\pgfpathlineto{\pgfqpoint{1.651244in}{0.466828in}}%
\pgfpathlineto{\pgfqpoint{1.655539in}{0.464851in}}%
\pgfpathlineto{\pgfqpoint{1.656613in}{0.467893in}}%
\pgfpathlineto{\pgfqpoint{1.657687in}{0.467133in}}%
\pgfpathlineto{\pgfqpoint{1.658761in}{0.465611in}}%
\pgfpathlineto{\pgfqpoint{1.664130in}{0.466372in}}%
\pgfpathlineto{\pgfqpoint{1.666278in}{0.469536in}}%
\pgfpathlineto{\pgfqpoint{1.669499in}{0.469840in}}%
\pgfpathlineto{\pgfqpoint{1.670573in}{0.471757in}}%
\pgfpathlineto{\pgfqpoint{1.671647in}{0.471757in}}%
\pgfpathlineto{\pgfqpoint{1.673795in}{0.472426in}}%
\pgfpathlineto{\pgfqpoint{1.677016in}{0.472426in}}%
\pgfpathlineto{\pgfqpoint{1.679164in}{0.474920in}}%
\pgfpathlineto{\pgfqpoint{1.680238in}{0.474586in}}%
\pgfpathlineto{\pgfqpoint{1.681312in}{0.476198in}}%
\pgfpathlineto{\pgfqpoint{1.684533in}{0.476016in}}%
\pgfpathlineto{\pgfqpoint{1.685607in}{0.476776in}}%
\pgfpathlineto{\pgfqpoint{1.686681in}{0.475012in}}%
\pgfpathlineto{\pgfqpoint{1.687755in}{0.476046in}}%
\pgfpathlineto{\pgfqpoint{1.688828in}{0.471422in}}%
\pgfpathlineto{\pgfqpoint{1.692050in}{0.471361in}}%
\pgfpathlineto{\pgfqpoint{1.693124in}{0.468958in}}%
\pgfpathlineto{\pgfqpoint{1.695272in}{0.476807in}}%
\pgfpathlineto{\pgfqpoint{1.696345in}{0.474981in}}%
\pgfpathlineto{\pgfqpoint{1.700641in}{0.477598in}}%
\pgfpathlineto{\pgfqpoint{1.701715in}{0.479332in}}%
\pgfpathlineto{\pgfqpoint{1.708158in}{0.477080in}}%
\pgfpathlineto{\pgfqpoint{1.709231in}{0.475377in}}%
\pgfpathlineto{\pgfqpoint{1.711379in}{0.477415in}}%
\pgfpathlineto{\pgfqpoint{1.715674in}{0.471939in}}%
\pgfpathlineto{\pgfqpoint{1.717822in}{0.477020in}}%
\pgfpathlineto{\pgfqpoint{1.718896in}{0.474160in}}%
\pgfpathlineto{\pgfqpoint{1.722117in}{0.473764in}}%
\pgfpathlineto{\pgfqpoint{1.723191in}{0.474342in}}%
\pgfpathlineto{\pgfqpoint{1.724265in}{0.470509in}}%
\pgfpathlineto{\pgfqpoint{1.725339in}{0.468745in}}%
\pgfpathlineto{\pgfqpoint{1.726413in}{0.470083in}}%
\pgfpathlineto{\pgfqpoint{1.733930in}{0.472517in}}%
\pgfpathlineto{\pgfqpoint{1.737151in}{0.471239in}}%
\pgfpathlineto{\pgfqpoint{1.739299in}{0.463786in}}%
\pgfpathlineto{\pgfqpoint{1.740373in}{0.465003in}}%
\pgfpathlineto{\pgfqpoint{1.741447in}{0.470053in}}%
\pgfpathlineto{\pgfqpoint{1.744668in}{0.470357in}}%
\pgfpathlineto{\pgfqpoint{1.746816in}{0.477324in}}%
\pgfpathlineto{\pgfqpoint{1.747890in}{0.482374in}}%
\pgfpathlineto{\pgfqpoint{1.748963in}{0.479180in}}%
\pgfpathlineto{\pgfqpoint{1.753259in}{0.477080in}}%
\pgfpathlineto{\pgfqpoint{1.755406in}{0.483165in}}%
\pgfpathlineto{\pgfqpoint{1.756480in}{0.482222in}}%
\pgfpathlineto{\pgfqpoint{1.760776in}{0.483134in}}%
\pgfpathlineto{\pgfqpoint{1.761850in}{0.481857in}}%
\pgfpathlineto{\pgfqpoint{1.762923in}{0.481857in}}%
\pgfpathlineto{\pgfqpoint{1.763997in}{0.484716in}}%
\pgfpathlineto{\pgfqpoint{1.770440in}{0.485264in}}%
\pgfpathlineto{\pgfqpoint{1.771514in}{0.483469in}}%
\pgfpathlineto{\pgfqpoint{1.774736in}{0.488610in}}%
\pgfpathlineto{\pgfqpoint{1.776883in}{0.485264in}}%
\pgfpathlineto{\pgfqpoint{1.777957in}{0.485538in}}%
\pgfpathlineto{\pgfqpoint{1.779031in}{0.482009in}}%
\pgfpathlineto{\pgfqpoint{1.782252in}{0.483560in}}%
\pgfpathlineto{\pgfqpoint{1.783326in}{0.478967in}}%
\pgfpathlineto{\pgfqpoint{1.784400in}{0.478632in}}%
\pgfpathlineto{\pgfqpoint{1.785474in}{0.482191in}}%
\pgfpathlineto{\pgfqpoint{1.786548in}{0.478845in}}%
\pgfpathlineto{\pgfqpoint{1.789769in}{0.481796in}}%
\pgfpathlineto{\pgfqpoint{1.790843in}{0.478449in}}%
\pgfpathlineto{\pgfqpoint{1.791917in}{0.480792in}}%
\pgfpathlineto{\pgfqpoint{1.792991in}{0.480488in}}%
\pgfpathlineto{\pgfqpoint{1.794065in}{0.482252in}}%
\pgfpathlineto{\pgfqpoint{1.798360in}{0.481339in}}%
\pgfpathlineto{\pgfqpoint{1.799434in}{0.477293in}}%
\pgfpathlineto{\pgfqpoint{1.801582in}{0.476746in}}%
\pgfpathlineto{\pgfqpoint{1.804803in}{0.477080in}}%
\pgfpathlineto{\pgfqpoint{1.806951in}{0.475681in}}%
\pgfpathlineto{\pgfqpoint{1.808025in}{0.476016in}}%
\pgfpathlineto{\pgfqpoint{1.812320in}{0.475742in}}%
\pgfpathlineto{\pgfqpoint{1.814468in}{0.479575in}}%
\pgfpathlineto{\pgfqpoint{1.816615in}{0.479088in}}%
\pgfpathlineto{\pgfqpoint{1.820911in}{0.476715in}}%
\pgfpathlineto{\pgfqpoint{1.823058in}{0.477050in}}%
\pgfpathlineto{\pgfqpoint{1.824132in}{0.473734in}}%
\pgfpathlineto{\pgfqpoint{1.827354in}{0.474312in}}%
\pgfpathlineto{\pgfqpoint{1.828427in}{0.476168in}}%
\pgfpathlineto{\pgfqpoint{1.829501in}{0.484017in}}%
\pgfpathlineto{\pgfqpoint{1.831649in}{0.482435in}}%
\pgfpathlineto{\pgfqpoint{1.834871in}{0.481339in}}%
\pgfpathlineto{\pgfqpoint{1.835944in}{0.480366in}}%
\pgfpathlineto{\pgfqpoint{1.837018in}{0.482039in}}%
\pgfpathlineto{\pgfqpoint{1.838092in}{0.478236in}}%
\pgfpathlineto{\pgfqpoint{1.839166in}{0.477415in}}%
\pgfpathlineto{\pgfqpoint{1.842387in}{0.476867in}}%
\pgfpathlineto{\pgfqpoint{1.843461in}{0.477963in}}%
\pgfpathlineto{\pgfqpoint{1.844535in}{0.477111in}}%
\pgfpathlineto{\pgfqpoint{1.845609in}{0.479788in}}%
\pgfpathlineto{\pgfqpoint{1.846683in}{0.488306in}}%
\pgfpathlineto{\pgfqpoint{1.852052in}{0.486237in}}%
\pgfpathlineto{\pgfqpoint{1.853126in}{0.490223in}}%
\pgfpathlineto{\pgfqpoint{1.854200in}{0.488945in}}%
\pgfpathlineto{\pgfqpoint{1.858495in}{0.490709in}}%
\pgfpathlineto{\pgfqpoint{1.860643in}{0.488367in}}%
\pgfpathlineto{\pgfqpoint{1.861716in}{0.489097in}}%
\pgfpathlineto{\pgfqpoint{1.866012in}{0.485933in}}%
\pgfpathlineto{\pgfqpoint{1.867086in}{0.488732in}}%
\pgfpathlineto{\pgfqpoint{1.868160in}{0.488914in}}%
\pgfpathlineto{\pgfqpoint{1.869233in}{0.486329in}}%
\pgfpathlineto{\pgfqpoint{1.872455in}{0.487606in}}%
\pgfpathlineto{\pgfqpoint{1.874603in}{0.487120in}}%
\pgfpathlineto{\pgfqpoint{1.875676in}{0.484960in}}%
\pgfpathlineto{\pgfqpoint{1.876750in}{0.485416in}}%
\pgfpathlineto{\pgfqpoint{1.879972in}{0.483378in}}%
\pgfpathlineto{\pgfqpoint{1.881046in}{0.484169in}}%
\pgfpathlineto{\pgfqpoint{1.882119in}{0.489127in}}%
\pgfpathlineto{\pgfqpoint{1.883193in}{0.489158in}}%
\pgfpathlineto{\pgfqpoint{1.887489in}{0.486024in}}%
\pgfpathlineto{\pgfqpoint{1.888562in}{0.487302in}}%
\pgfpathlineto{\pgfqpoint{1.889636in}{0.486542in}}%
\pgfpathlineto{\pgfqpoint{1.890710in}{0.488762in}}%
\pgfpathlineto{\pgfqpoint{1.891784in}{0.486359in}}%
\pgfpathlineto{\pgfqpoint{1.896079in}{0.488519in}}%
\pgfpathlineto{\pgfqpoint{1.898227in}{0.486146in}}%
\pgfpathlineto{\pgfqpoint{1.899301in}{0.486542in}}%
\pgfpathlineto{\pgfqpoint{1.902522in}{0.480549in}}%
\pgfpathlineto{\pgfqpoint{1.905744in}{0.485051in}}%
\pgfpathlineto{\pgfqpoint{1.910039in}{0.484686in}}%
\pgfpathlineto{\pgfqpoint{1.911113in}{0.483652in}}%
\pgfpathlineto{\pgfqpoint{1.912187in}{0.480579in}}%
\pgfpathlineto{\pgfqpoint{1.913261in}{0.481522in}}%
\pgfpathlineto{\pgfqpoint{1.914335in}{0.485568in}}%
\pgfpathlineto{\pgfqpoint{1.921851in}{0.492808in}}%
\pgfpathlineto{\pgfqpoint{1.925073in}{0.498193in}}%
\pgfpathlineto{\pgfqpoint{1.926147in}{0.496185in}}%
\pgfpathlineto{\pgfqpoint{1.928294in}{0.495394in}}%
\pgfpathlineto{\pgfqpoint{1.929368in}{0.504369in}}%
\pgfpathlineto{\pgfqpoint{1.932590in}{0.501600in}}%
\pgfpathlineto{\pgfqpoint{1.935811in}{0.509023in}}%
\pgfpathlineto{\pgfqpoint{1.936885in}{0.505981in}}%
\pgfpathlineto{\pgfqpoint{1.940107in}{0.507228in}}%
\pgfpathlineto{\pgfqpoint{1.942254in}{0.504916in}}%
\pgfpathlineto{\pgfqpoint{1.943328in}{0.500779in}}%
\pgfpathlineto{\pgfqpoint{1.944402in}{0.502635in}}%
\pgfpathlineto{\pgfqpoint{1.947624in}{0.503152in}}%
\pgfpathlineto{\pgfqpoint{1.948697in}{0.500140in}}%
\pgfpathlineto{\pgfqpoint{1.951919in}{0.503030in}}%
\pgfpathlineto{\pgfqpoint{1.957288in}{0.503547in}}%
\pgfpathlineto{\pgfqpoint{1.958362in}{0.502239in}}%
\pgfpathlineto{\pgfqpoint{1.959436in}{0.494086in}}%
\pgfpathlineto{\pgfqpoint{1.963731in}{0.481613in}}%
\pgfpathlineto{\pgfqpoint{1.964805in}{0.492595in}}%
\pgfpathlineto{\pgfqpoint{1.965879in}{0.497645in}}%
\pgfpathlineto{\pgfqpoint{1.966953in}{0.497828in}}%
\pgfpathlineto{\pgfqpoint{1.970174in}{0.494390in}}%
\pgfpathlineto{\pgfqpoint{1.971248in}{0.487485in}}%
\pgfpathlineto{\pgfqpoint{1.973396in}{0.491744in}}%
\pgfpathlineto{\pgfqpoint{1.974470in}{0.488093in}}%
\pgfpathlineto{\pgfqpoint{1.978765in}{0.492139in}}%
\pgfpathlineto{\pgfqpoint{1.979839in}{0.489371in}}%
\pgfpathlineto{\pgfqpoint{1.981986in}{0.492808in}}%
\pgfpathlineto{\pgfqpoint{1.985208in}{0.490588in}}%
\pgfpathlineto{\pgfqpoint{1.987356in}{0.493478in}}%
\pgfpathlineto{\pgfqpoint{1.988429in}{0.493386in}}%
\pgfpathlineto{\pgfqpoint{1.989503in}{0.489949in}}%
\pgfpathlineto{\pgfqpoint{1.992725in}{0.492869in}}%
\pgfpathlineto{\pgfqpoint{1.993799in}{0.491287in}}%
\pgfpathlineto{\pgfqpoint{1.994872in}{0.493356in}}%
\pgfpathlineto{\pgfqpoint{1.995946in}{0.491257in}}%
\pgfpathlineto{\pgfqpoint{1.997020in}{0.492595in}}%
\pgfpathlineto{\pgfqpoint{2.000242in}{0.482283in}}%
\pgfpathlineto{\pgfqpoint{2.002389in}{0.489553in}}%
\pgfpathlineto{\pgfqpoint{2.003463in}{0.490527in}}%
\pgfpathlineto{\pgfqpoint{2.004537in}{0.492535in}}%
\pgfpathlineto{\pgfqpoint{2.007759in}{0.497159in}}%
\pgfpathlineto{\pgfqpoint{2.008832in}{0.496702in}}%
\pgfpathlineto{\pgfqpoint{2.010980in}{0.502087in}}%
\pgfpathlineto{\pgfqpoint{2.012054in}{0.502330in}}%
\pgfpathlineto{\pgfqpoint{2.015275in}{0.505312in}}%
\pgfpathlineto{\pgfqpoint{2.016349in}{0.505342in}}%
\pgfpathlineto{\pgfqpoint{2.017423in}{0.502969in}}%
\pgfpathlineto{\pgfqpoint{2.019571in}{0.508293in}}%
\pgfpathlineto{\pgfqpoint{2.022792in}{0.511214in}}%
\pgfpathlineto{\pgfqpoint{2.024940in}{0.506681in}}%
\pgfpathlineto{\pgfqpoint{2.027088in}{0.511426in}}%
\pgfpathlineto{\pgfqpoint{2.030309in}{0.514712in}}%
\pgfpathlineto{\pgfqpoint{2.031383in}{0.512765in}}%
\pgfpathlineto{\pgfqpoint{2.032457in}{0.516750in}}%
\pgfpathlineto{\pgfqpoint{2.033531in}{0.515686in}}%
\pgfpathlineto{\pgfqpoint{2.037826in}{0.505981in}}%
\pgfpathlineto{\pgfqpoint{2.038900in}{0.513891in}}%
\pgfpathlineto{\pgfqpoint{2.039974in}{0.515229in}}%
\pgfpathlineto{\pgfqpoint{2.041048in}{0.517906in}}%
\pgfpathlineto{\pgfqpoint{2.042121in}{0.516385in}}%
\pgfpathlineto{\pgfqpoint{2.045343in}{0.514317in}}%
\pgfpathlineto{\pgfqpoint{2.046417in}{0.518971in}}%
\pgfpathlineto{\pgfqpoint{2.047491in}{0.518089in}}%
\pgfpathlineto{\pgfqpoint{2.048564in}{0.515503in}}%
\pgfpathlineto{\pgfqpoint{2.049638in}{0.514925in}}%
\pgfpathlineto{\pgfqpoint{2.052860in}{0.517267in}}%
\pgfpathlineto{\pgfqpoint{2.053934in}{0.517054in}}%
\pgfpathlineto{\pgfqpoint{2.055007in}{0.521861in}}%
\pgfpathlineto{\pgfqpoint{2.056081in}{0.520948in}}%
\pgfpathlineto{\pgfqpoint{2.057155in}{0.521070in}}%
\pgfpathlineto{\pgfqpoint{2.060377in}{0.520827in}}%
\pgfpathlineto{\pgfqpoint{2.062524in}{0.519214in}}%
\pgfpathlineto{\pgfqpoint{2.064672in}{0.520036in}}%
\pgfpathlineto{\pgfqpoint{2.067893in}{0.517572in}}%
\pgfpathlineto{\pgfqpoint{2.068967in}{0.520249in}}%
\pgfpathlineto{\pgfqpoint{2.071115in}{0.515442in}}%
\pgfpathlineto{\pgfqpoint{2.072189in}{0.521709in}}%
\pgfpathlineto{\pgfqpoint{2.076484in}{0.517754in}}%
\pgfpathlineto{\pgfqpoint{2.077558in}{0.514256in}}%
\pgfpathlineto{\pgfqpoint{2.078632in}{0.514955in}}%
\pgfpathlineto{\pgfqpoint{2.079706in}{0.509023in}}%
\pgfpathlineto{\pgfqpoint{2.082927in}{0.511244in}}%
\pgfpathlineto{\pgfqpoint{2.085075in}{0.520097in}}%
\pgfpathlineto{\pgfqpoint{2.086149in}{0.516629in}}%
\pgfpathlineto{\pgfqpoint{2.087223in}{0.509632in}}%
\pgfpathlineto{\pgfqpoint{2.091518in}{0.512765in}}%
\pgfpathlineto{\pgfqpoint{2.092592in}{0.516264in}}%
\pgfpathlineto{\pgfqpoint{2.093666in}{0.515381in}}%
\pgfpathlineto{\pgfqpoint{2.097961in}{0.516172in}}%
\pgfpathlineto{\pgfqpoint{2.099035in}{0.518180in}}%
\pgfpathlineto{\pgfqpoint{2.101182in}{0.513282in}}%
\pgfpathlineto{\pgfqpoint{2.105478in}{0.507806in}}%
\pgfpathlineto{\pgfqpoint{2.106552in}{0.509480in}}%
\pgfpathlineto{\pgfqpoint{2.109773in}{0.499440in}}%
\pgfpathlineto{\pgfqpoint{2.112995in}{0.502543in}}%
\pgfpathlineto{\pgfqpoint{2.114069in}{0.505008in}}%
\pgfpathlineto{\pgfqpoint{2.115142in}{0.500110in}}%
\pgfpathlineto{\pgfqpoint{2.116216in}{0.502178in}}%
\pgfpathlineto{\pgfqpoint{2.117290in}{0.496337in}}%
\pgfpathlineto{\pgfqpoint{2.121585in}{0.495060in}}%
\pgfpathlineto{\pgfqpoint{2.122659in}{0.492961in}}%
\pgfpathlineto{\pgfqpoint{2.124807in}{0.498923in}}%
\pgfpathlineto{\pgfqpoint{2.128028in}{0.496094in}}%
\pgfpathlineto{\pgfqpoint{2.129102in}{0.496489in}}%
\pgfpathlineto{\pgfqpoint{2.130176in}{0.493630in}}%
\pgfpathlineto{\pgfqpoint{2.131250in}{0.488945in}}%
\pgfpathlineto{\pgfqpoint{2.132324in}{0.504217in}}%
\pgfpathlineto{\pgfqpoint{2.135545in}{0.503882in}}%
\pgfpathlineto{\pgfqpoint{2.136619in}{0.500992in}}%
\pgfpathlineto{\pgfqpoint{2.137693in}{0.503882in}}%
\pgfpathlineto{\pgfqpoint{2.138767in}{0.501813in}}%
\pgfpathlineto{\pgfqpoint{2.139841in}{0.495486in}}%
\pgfpathlineto{\pgfqpoint{2.143062in}{0.484351in}}%
\pgfpathlineto{\pgfqpoint{2.144136in}{0.485994in}}%
\pgfpathlineto{\pgfqpoint{2.145210in}{0.491805in}}%
\pgfpathlineto{\pgfqpoint{2.146284in}{0.486815in}}%
\pgfpathlineto{\pgfqpoint{2.147358in}{0.492595in}}%
\pgfpathlineto{\pgfqpoint{2.151653in}{0.494634in}}%
\pgfpathlineto{\pgfqpoint{2.152727in}{0.497798in}}%
\pgfpathlineto{\pgfqpoint{2.153801in}{0.495455in}}%
\pgfpathlineto{\pgfqpoint{2.154874in}{0.496307in}}%
\pgfpathlineto{\pgfqpoint{2.158096in}{0.500870in}}%
\pgfpathlineto{\pgfqpoint{2.159170in}{0.498163in}}%
\pgfpathlineto{\pgfqpoint{2.160244in}{0.497250in}}%
\pgfpathlineto{\pgfqpoint{2.161317in}{0.501479in}}%
\pgfpathlineto{\pgfqpoint{2.162391in}{0.499866in}}%
\pgfpathlineto{\pgfqpoint{2.165613in}{0.498862in}}%
\pgfpathlineto{\pgfqpoint{2.166687in}{0.505555in}}%
\pgfpathlineto{\pgfqpoint{2.168834in}{0.503395in}}%
\pgfpathlineto{\pgfqpoint{2.169908in}{0.503365in}}%
\pgfpathlineto{\pgfqpoint{2.173130in}{0.497524in}}%
\pgfpathlineto{\pgfqpoint{2.174203in}{0.493599in}}%
\pgfpathlineto{\pgfqpoint{2.175277in}{0.493782in}}%
\pgfpathlineto{\pgfqpoint{2.176351in}{0.492413in}}%
\pgfpathlineto{\pgfqpoint{2.177425in}{0.496611in}}%
\pgfpathlineto{\pgfqpoint{2.180647in}{0.496185in}}%
\pgfpathlineto{\pgfqpoint{2.182794in}{0.498801in}}%
\pgfpathlineto{\pgfqpoint{2.184942in}{0.503000in}}%
\pgfpathlineto{\pgfqpoint{2.188163in}{0.502969in}}%
\pgfpathlineto{\pgfqpoint{2.189237in}{0.500566in}}%
\pgfpathlineto{\pgfqpoint{2.190311in}{0.503395in}}%
\pgfpathlineto{\pgfqpoint{2.191385in}{0.504064in}}%
\pgfpathlineto{\pgfqpoint{2.195680in}{0.503882in}}%
\pgfpathlineto{\pgfqpoint{2.197828in}{0.511913in}}%
\pgfpathlineto{\pgfqpoint{2.198902in}{0.511001in}}%
\pgfpathlineto{\pgfqpoint{2.199976in}{0.514317in}}%
\pgfpathlineto{\pgfqpoint{2.203197in}{0.515016in}}%
\pgfpathlineto{\pgfqpoint{2.204271in}{0.512461in}}%
\pgfpathlineto{\pgfqpoint{2.205345in}{0.516142in}}%
\pgfpathlineto{\pgfqpoint{2.206419in}{0.514286in}}%
\pgfpathlineto{\pgfqpoint{2.207492in}{0.515625in}}%
\pgfpathlineto{\pgfqpoint{2.210714in}{0.514986in}}%
\pgfpathlineto{\pgfqpoint{2.211788in}{0.517085in}}%
\pgfpathlineto{\pgfqpoint{2.213936in}{0.522439in}}%
\pgfpathlineto{\pgfqpoint{2.215009in}{0.521709in}}%
\pgfpathlineto{\pgfqpoint{2.218231in}{0.525786in}}%
\pgfpathlineto{\pgfqpoint{2.219305in}{0.523686in}}%
\pgfpathlineto{\pgfqpoint{2.220379in}{0.524873in}}%
\pgfpathlineto{\pgfqpoint{2.221452in}{0.523808in}}%
\pgfpathlineto{\pgfqpoint{2.222526in}{0.518819in}}%
\pgfpathlineto{\pgfqpoint{2.225748in}{0.515959in}}%
\pgfpathlineto{\pgfqpoint{2.227895in}{0.517785in}}%
\pgfpathlineto{\pgfqpoint{2.228969in}{0.514590in}}%
\pgfpathlineto{\pgfqpoint{2.230043in}{0.513282in}}%
\pgfpathlineto{\pgfqpoint{2.233265in}{0.516902in}}%
\pgfpathlineto{\pgfqpoint{2.234338in}{0.513130in}}%
\pgfpathlineto{\pgfqpoint{2.235412in}{0.512765in}}%
\pgfpathlineto{\pgfqpoint{2.237560in}{0.514712in}}%
\pgfpathlineto{\pgfqpoint{2.240781in}{0.516172in}}%
\pgfpathlineto{\pgfqpoint{2.241855in}{0.519093in}}%
\pgfpathlineto{\pgfqpoint{2.242929in}{0.513860in}}%
\pgfpathlineto{\pgfqpoint{2.244003in}{0.515625in}}%
\pgfpathlineto{\pgfqpoint{2.245077in}{0.512461in}}%
\pgfpathlineto{\pgfqpoint{2.248298in}{0.515351in}}%
\pgfpathlineto{\pgfqpoint{2.249372in}{0.512339in}}%
\pgfpathlineto{\pgfqpoint{2.250446in}{0.514256in}}%
\pgfpathlineto{\pgfqpoint{2.251520in}{0.512613in}}%
\pgfpathlineto{\pgfqpoint{2.252594in}{0.514955in}}%
\pgfpathlineto{\pgfqpoint{2.255815in}{0.513586in}}%
\pgfpathlineto{\pgfqpoint{2.256889in}{0.520036in}}%
\pgfpathlineto{\pgfqpoint{2.257963in}{0.519093in}}%
\pgfpathlineto{\pgfqpoint{2.259037in}{0.518910in}}%
\pgfpathlineto{\pgfqpoint{2.260111in}{0.520888in}}%
\pgfpathlineto{\pgfqpoint{2.264406in}{0.518728in}}%
\pgfpathlineto{\pgfqpoint{2.265480in}{0.519579in}}%
\pgfpathlineto{\pgfqpoint{2.266554in}{0.521739in}}%
\pgfpathlineto{\pgfqpoint{2.267627in}{0.521709in}}%
\pgfpathlineto{\pgfqpoint{2.271923in}{0.523686in}}%
\pgfpathlineto{\pgfqpoint{2.272997in}{0.526789in}}%
\pgfpathlineto{\pgfqpoint{2.274070in}{0.525633in}}%
\pgfpathlineto{\pgfqpoint{2.275144in}{0.522439in}}%
\pgfpathlineto{\pgfqpoint{2.278366in}{0.517115in}}%
\pgfpathlineto{\pgfqpoint{2.279440in}{0.517633in}}%
\pgfpathlineto{\pgfqpoint{2.280514in}{0.516446in}}%
\pgfpathlineto{\pgfqpoint{2.281587in}{0.516994in}}%
\pgfpathlineto{\pgfqpoint{2.282661in}{0.512948in}}%
\pgfpathlineto{\pgfqpoint{2.286957in}{0.513951in}}%
\pgfpathlineto{\pgfqpoint{2.288030in}{0.511579in}}%
\pgfpathlineto{\pgfqpoint{2.289104in}{0.516629in}}%
\pgfpathlineto{\pgfqpoint{2.290178in}{0.507167in}}%
\pgfpathlineto{\pgfqpoint{2.293400in}{0.502087in}}%
\pgfpathlineto{\pgfqpoint{2.295547in}{0.512187in}}%
\pgfpathlineto{\pgfqpoint{2.296621in}{0.504551in}}%
\pgfpathlineto{\pgfqpoint{2.297695in}{0.505464in}}%
\pgfpathlineto{\pgfqpoint{2.301990in}{0.506042in}}%
\pgfpathlineto{\pgfqpoint{2.303064in}{0.504217in}}%
\pgfpathlineto{\pgfqpoint{2.304138in}{0.505555in}}%
\pgfpathlineto{\pgfqpoint{2.305212in}{0.511244in}}%
\pgfpathlineto{\pgfqpoint{2.308433in}{0.511548in}}%
\pgfpathlineto{\pgfqpoint{2.309507in}{0.514377in}}%
\pgfpathlineto{\pgfqpoint{2.310581in}{0.514347in}}%
\pgfpathlineto{\pgfqpoint{2.311655in}{0.516355in}}%
\pgfpathlineto{\pgfqpoint{2.312729in}{0.516842in}}%
\pgfpathlineto{\pgfqpoint{2.315950in}{0.516872in}}%
\pgfpathlineto{\pgfqpoint{2.318098in}{0.519975in}}%
\pgfpathlineto{\pgfqpoint{2.319172in}{0.518302in}}%
\pgfpathlineto{\pgfqpoint{2.320246in}{0.521618in}}%
\pgfpathlineto{\pgfqpoint{2.325615in}{0.517511in}}%
\pgfpathlineto{\pgfqpoint{2.326689in}{0.519488in}}%
\pgfpathlineto{\pgfqpoint{2.327762in}{0.516081in}}%
\pgfpathlineto{\pgfqpoint{2.332058in}{0.516781in}}%
\pgfpathlineto{\pgfqpoint{2.333132in}{0.518058in}}%
\pgfpathlineto{\pgfqpoint{2.335279in}{0.522317in}}%
\pgfpathlineto{\pgfqpoint{2.340648in}{0.520948in}}%
\pgfpathlineto{\pgfqpoint{2.341722in}{0.522257in}}%
\pgfpathlineto{\pgfqpoint{2.342796in}{0.521952in}}%
\pgfpathlineto{\pgfqpoint{2.346018in}{0.524599in}}%
\pgfpathlineto{\pgfqpoint{2.347091in}{0.524295in}}%
\pgfpathlineto{\pgfqpoint{2.348165in}{0.524812in}}%
\pgfpathlineto{\pgfqpoint{2.349239in}{0.523108in}}%
\pgfpathlineto{\pgfqpoint{2.353535in}{0.525451in}}%
\pgfpathlineto{\pgfqpoint{2.354608in}{0.524690in}}%
\pgfpathlineto{\pgfqpoint{2.355682in}{0.523169in}}%
\pgfpathlineto{\pgfqpoint{2.356756in}{0.523261in}}%
\pgfpathlineto{\pgfqpoint{2.357830in}{0.523991in}}%
\pgfpathlineto{\pgfqpoint{2.361051in}{0.524903in}}%
\pgfpathlineto{\pgfqpoint{2.362125in}{0.525786in}}%
\pgfpathlineto{\pgfqpoint{2.363199in}{0.524995in}}%
\pgfpathlineto{\pgfqpoint{2.364273in}{0.526181in}}%
\pgfpathlineto{\pgfqpoint{2.365347in}{0.528250in}}%
\pgfpathlineto{\pgfqpoint{2.369642in}{0.529832in}}%
\pgfpathlineto{\pgfqpoint{2.370716in}{0.531961in}}%
\pgfpathlineto{\pgfqpoint{2.371790in}{0.531140in}}%
\pgfpathlineto{\pgfqpoint{2.372864in}{0.526272in}}%
\pgfpathlineto{\pgfqpoint{2.376085in}{0.531140in}}%
\pgfpathlineto{\pgfqpoint{2.377159in}{0.527945in}}%
\pgfpathlineto{\pgfqpoint{2.378233in}{0.526729in}}%
\pgfpathlineto{\pgfqpoint{2.379307in}{0.528280in}}%
\pgfpathlineto{\pgfqpoint{2.380380in}{0.528463in}}%
\pgfpathlineto{\pgfqpoint{2.384676in}{0.529679in}}%
\pgfpathlineto{\pgfqpoint{2.385750in}{0.531900in}}%
\pgfpathlineto{\pgfqpoint{2.386824in}{0.532326in}}%
\pgfpathlineto{\pgfqpoint{2.387897in}{0.529862in}}%
\pgfpathlineto{\pgfqpoint{2.391119in}{0.527702in}}%
\pgfpathlineto{\pgfqpoint{2.392193in}{0.528736in}}%
\pgfpathlineto{\pgfqpoint{2.393267in}{0.531140in}}%
\pgfpathlineto{\pgfqpoint{2.394340in}{0.528037in}}%
\pgfpathlineto{\pgfqpoint{2.395414in}{0.530349in}}%
\pgfpathlineto{\pgfqpoint{2.399710in}{0.530440in}}%
\pgfpathlineto{\pgfqpoint{2.400783in}{0.532296in}}%
\pgfpathlineto{\pgfqpoint{2.401857in}{0.532326in}}%
\pgfpathlineto{\pgfqpoint{2.402931in}{0.530896in}}%
\pgfpathlineto{\pgfqpoint{2.406153in}{0.531596in}}%
\pgfpathlineto{\pgfqpoint{2.407226in}{0.528371in}}%
\pgfpathlineto{\pgfqpoint{2.408300in}{0.529010in}}%
\pgfpathlineto{\pgfqpoint{2.409374in}{0.527885in}}%
\pgfpathlineto{\pgfqpoint{2.410448in}{0.529619in}}%
\pgfpathlineto{\pgfqpoint{2.413669in}{0.528706in}}%
\pgfpathlineto{\pgfqpoint{2.414743in}{0.527002in}}%
\pgfpathlineto{\pgfqpoint{2.415817in}{0.530683in}}%
\pgfpathlineto{\pgfqpoint{2.417965in}{0.529314in}}%
\pgfpathlineto{\pgfqpoint{2.421186in}{0.531748in}}%
\pgfpathlineto{\pgfqpoint{2.422260in}{0.528341in}}%
\pgfpathlineto{\pgfqpoint{2.423334in}{0.527520in}}%
\pgfpathlineto{\pgfqpoint{2.428703in}{0.529771in}}%
\pgfpathlineto{\pgfqpoint{2.430851in}{0.524021in}}%
\pgfpathlineto{\pgfqpoint{2.431925in}{0.524234in}}%
\pgfpathlineto{\pgfqpoint{2.432999in}{0.523382in}}%
\pgfpathlineto{\pgfqpoint{2.437294in}{0.530896in}}%
\pgfpathlineto{\pgfqpoint{2.438368in}{0.531961in}}%
\pgfpathlineto{\pgfqpoint{2.439442in}{0.527885in}}%
\pgfpathlineto{\pgfqpoint{2.440515in}{0.527915in}}%
\pgfpathlineto{\pgfqpoint{2.443737in}{0.517480in}}%
\pgfpathlineto{\pgfqpoint{2.444811in}{0.518028in}}%
\pgfpathlineto{\pgfqpoint{2.446958in}{0.525998in}}%
\pgfpathlineto{\pgfqpoint{2.448032in}{0.525268in}}%
\pgfpathlineto{\pgfqpoint{2.451254in}{0.527854in}}%
\pgfpathlineto{\pgfqpoint{2.452328in}{0.522591in}}%
\pgfpathlineto{\pgfqpoint{2.453402in}{0.521526in}}%
\pgfpathlineto{\pgfqpoint{2.455549in}{0.523200in}}%
\pgfpathlineto{\pgfqpoint{2.458771in}{0.520066in}}%
\pgfpathlineto{\pgfqpoint{2.459845in}{0.520279in}}%
\pgfpathlineto{\pgfqpoint{2.461992in}{0.509145in}}%
\pgfpathlineto{\pgfqpoint{2.463066in}{0.510027in}}%
\pgfpathlineto{\pgfqpoint{2.466288in}{0.514773in}}%
\pgfpathlineto{\pgfqpoint{2.467361in}{0.514164in}}%
\pgfpathlineto{\pgfqpoint{2.468435in}{0.520736in}}%
\pgfpathlineto{\pgfqpoint{2.470583in}{0.520249in}}%
\pgfpathlineto{\pgfqpoint{2.473804in}{0.518332in}}%
\pgfpathlineto{\pgfqpoint{2.474878in}{0.520401in}}%
\pgfpathlineto{\pgfqpoint{2.475952in}{0.520218in}}%
\pgfpathlineto{\pgfqpoint{2.477026in}{0.521314in}}%
\pgfpathlineto{\pgfqpoint{2.478100in}{0.517876in}}%
\pgfpathlineto{\pgfqpoint{2.481321in}{0.517115in}}%
\pgfpathlineto{\pgfqpoint{2.482395in}{0.517906in}}%
\pgfpathlineto{\pgfqpoint{2.484543in}{0.516537in}}%
\pgfpathlineto{\pgfqpoint{2.485617in}{0.517267in}}%
\pgfpathlineto{\pgfqpoint{2.492060in}{0.517815in}}%
\pgfpathlineto{\pgfqpoint{2.493134in}{0.516902in}}%
\pgfpathlineto{\pgfqpoint{2.497429in}{0.521314in}}%
\pgfpathlineto{\pgfqpoint{2.499577in}{0.526059in}}%
\pgfpathlineto{\pgfqpoint{2.500650in}{0.529406in}}%
\pgfpathlineto{\pgfqpoint{2.503872in}{0.528037in}}%
\pgfpathlineto{\pgfqpoint{2.504946in}{0.526729in}}%
\pgfpathlineto{\pgfqpoint{2.506020in}{0.528189in}}%
\pgfpathlineto{\pgfqpoint{2.508167in}{0.526303in}}%
\pgfpathlineto{\pgfqpoint{2.512463in}{0.526576in}}%
\pgfpathlineto{\pgfqpoint{2.514610in}{0.527976in}}%
\pgfpathlineto{\pgfqpoint{2.518906in}{0.529223in}}%
\pgfpathlineto{\pgfqpoint{2.521053in}{0.534456in}}%
\pgfpathlineto{\pgfqpoint{2.522127in}{0.532478in}}%
\pgfpathlineto{\pgfqpoint{2.523201in}{0.534060in}}%
\pgfpathlineto{\pgfqpoint{2.526423in}{0.533847in}}%
\pgfpathlineto{\pgfqpoint{2.527496in}{0.530896in}}%
\pgfpathlineto{\pgfqpoint{2.529644in}{0.529679in}}%
\pgfpathlineto{\pgfqpoint{2.530718in}{0.540966in}}%
\pgfpathlineto{\pgfqpoint{2.535013in}{0.540053in}}%
\pgfpathlineto{\pgfqpoint{2.536087in}{0.538015in}}%
\pgfpathlineto{\pgfqpoint{2.538235in}{0.540418in}}%
\pgfpathlineto{\pgfqpoint{2.541456in}{0.542031in}}%
\pgfpathlineto{\pgfqpoint{2.542530in}{0.543278in}}%
\pgfpathlineto{\pgfqpoint{2.543604in}{0.545833in}}%
\pgfpathlineto{\pgfqpoint{2.545752in}{0.545590in}}%
\pgfpathlineto{\pgfqpoint{2.550047in}{0.546959in}}%
\pgfpathlineto{\pgfqpoint{2.551121in}{0.546594in}}%
\pgfpathlineto{\pgfqpoint{2.553268in}{0.548480in}}%
\pgfpathlineto{\pgfqpoint{2.557564in}{0.547020in}}%
\pgfpathlineto{\pgfqpoint{2.558638in}{0.550153in}}%
\pgfpathlineto{\pgfqpoint{2.559712in}{0.548784in}}%
\pgfpathlineto{\pgfqpoint{2.560785in}{0.549575in}}%
\pgfpathlineto{\pgfqpoint{2.567228in}{0.550518in}}%
\pgfpathlineto{\pgfqpoint{2.568302in}{0.552374in}}%
\pgfpathlineto{\pgfqpoint{2.571524in}{0.553500in}}%
\pgfpathlineto{\pgfqpoint{2.572598in}{0.551826in}}%
\pgfpathlineto{\pgfqpoint{2.574745in}{0.553317in}}%
\pgfpathlineto{\pgfqpoint{2.575819in}{0.553895in}}%
\pgfpathlineto{\pgfqpoint{2.579041in}{0.550731in}}%
\pgfpathlineto{\pgfqpoint{2.580114in}{0.547385in}}%
\pgfpathlineto{\pgfqpoint{2.583336in}{0.550762in}}%
\pgfpathlineto{\pgfqpoint{2.586557in}{0.550032in}}%
\pgfpathlineto{\pgfqpoint{2.588705in}{0.550731in}}%
\pgfpathlineto{\pgfqpoint{2.590853in}{0.549788in}}%
\pgfpathlineto{\pgfqpoint{2.594074in}{0.551279in}}%
\pgfpathlineto{\pgfqpoint{2.595148in}{0.549545in}}%
\pgfpathlineto{\pgfqpoint{2.597296in}{0.550457in}}%
\pgfpathlineto{\pgfqpoint{2.598370in}{0.549423in}}%
\pgfpathlineto{\pgfqpoint{2.602665in}{0.549727in}}%
\pgfpathlineto{\pgfqpoint{2.603739in}{0.549241in}}%
\pgfpathlineto{\pgfqpoint{2.604813in}{0.549788in}}%
\pgfpathlineto{\pgfqpoint{2.609108in}{0.552617in}}%
\pgfpathlineto{\pgfqpoint{2.611256in}{0.552435in}}%
\pgfpathlineto{\pgfqpoint{2.612330in}{0.556633in}}%
\pgfpathlineto{\pgfqpoint{2.613403in}{0.556633in}}%
\pgfpathlineto{\pgfqpoint{2.617699in}{0.559493in}}%
\pgfpathlineto{\pgfqpoint{2.620920in}{0.556846in}}%
\pgfpathlineto{\pgfqpoint{2.624142in}{0.556968in}}%
\pgfpathlineto{\pgfqpoint{2.625216in}{0.560770in}}%
\pgfpathlineto{\pgfqpoint{2.626290in}{0.560497in}}%
\pgfpathlineto{\pgfqpoint{2.627363in}{0.561014in}}%
\pgfpathlineto{\pgfqpoint{2.628437in}{0.559432in}}%
\pgfpathlineto{\pgfqpoint{2.632733in}{0.559158in}}%
\pgfpathlineto{\pgfqpoint{2.633806in}{0.559919in}}%
\pgfpathlineto{\pgfqpoint{2.634880in}{0.559462in}}%
\pgfpathlineto{\pgfqpoint{2.635954in}{0.561348in}}%
\pgfpathlineto{\pgfqpoint{2.639176in}{0.562839in}}%
\pgfpathlineto{\pgfqpoint{2.640249in}{0.562596in}}%
\pgfpathlineto{\pgfqpoint{2.641323in}{0.558945in}}%
\pgfpathlineto{\pgfqpoint{2.642397in}{0.558793in}}%
\pgfpathlineto{\pgfqpoint{2.643471in}{0.561105in}}%
\pgfpathlineto{\pgfqpoint{2.646692in}{0.563569in}}%
\pgfpathlineto{\pgfqpoint{2.647766in}{0.565242in}}%
\pgfpathlineto{\pgfqpoint{2.648840in}{0.568072in}}%
\pgfpathlineto{\pgfqpoint{2.649914in}{0.568771in}}%
\pgfpathlineto{\pgfqpoint{2.650988in}{0.567646in}}%
\pgfpathlineto{\pgfqpoint{2.655283in}{0.567798in}}%
\pgfpathlineto{\pgfqpoint{2.656357in}{0.569349in}}%
\pgfpathlineto{\pgfqpoint{2.657431in}{0.569836in}}%
\pgfpathlineto{\pgfqpoint{2.658505in}{0.572087in}}%
\pgfpathlineto{\pgfqpoint{2.661726in}{0.573304in}}%
\pgfpathlineto{\pgfqpoint{2.662800in}{0.571023in}}%
\pgfpathlineto{\pgfqpoint{2.663874in}{0.571905in}}%
\pgfpathlineto{\pgfqpoint{2.664948in}{0.571905in}}%
\pgfpathlineto{\pgfqpoint{2.666022in}{0.567342in}}%
\pgfpathlineto{\pgfqpoint{2.669243in}{0.564147in}}%
\pgfpathlineto{\pgfqpoint{2.670317in}{0.568893in}}%
\pgfpathlineto{\pgfqpoint{2.671391in}{0.569623in}}%
\pgfpathlineto{\pgfqpoint{2.672465in}{0.566155in}}%
\pgfpathlineto{\pgfqpoint{2.673538in}{0.566155in}}%
\pgfpathlineto{\pgfqpoint{2.676760in}{0.568011in}}%
\pgfpathlineto{\pgfqpoint{2.677834in}{0.566794in}}%
\pgfpathlineto{\pgfqpoint{2.678908in}{0.567281in}}%
\pgfpathlineto{\pgfqpoint{2.679981in}{0.565516in}}%
\pgfpathlineto{\pgfqpoint{2.681055in}{0.570384in}}%
\pgfpathlineto{\pgfqpoint{2.684277in}{0.569319in}}%
\pgfpathlineto{\pgfqpoint{2.685351in}{0.568345in}}%
\pgfpathlineto{\pgfqpoint{2.686424in}{0.572331in}}%
\pgfpathlineto{\pgfqpoint{2.687498in}{0.566916in}}%
\pgfpathlineto{\pgfqpoint{2.688572in}{0.564999in}}%
\pgfpathlineto{\pgfqpoint{2.691794in}{0.563691in}}%
\pgfpathlineto{\pgfqpoint{2.693941in}{0.565668in}}%
\pgfpathlineto{\pgfqpoint{2.695015in}{0.563417in}}%
\pgfpathlineto{\pgfqpoint{2.696089in}{0.565425in}}%
\pgfpathlineto{\pgfqpoint{2.700384in}{0.569867in}}%
\pgfpathlineto{\pgfqpoint{2.701458in}{0.572179in}}%
\pgfpathlineto{\pgfqpoint{2.702532in}{0.571479in}}%
\pgfpathlineto{\pgfqpoint{2.703606in}{0.574430in}}%
\pgfpathlineto{\pgfqpoint{2.706827in}{0.574126in}}%
\pgfpathlineto{\pgfqpoint{2.708975in}{0.578385in}}%
\pgfpathlineto{\pgfqpoint{2.710049in}{0.577959in}}%
\pgfpathlineto{\pgfqpoint{2.711123in}{0.582431in}}%
\pgfpathlineto{\pgfqpoint{2.714344in}{0.584743in}}%
\pgfpathlineto{\pgfqpoint{2.715418in}{0.583617in}}%
\pgfpathlineto{\pgfqpoint{2.716492in}{0.586173in}}%
\pgfpathlineto{\pgfqpoint{2.717566in}{0.582339in}}%
\pgfpathlineto{\pgfqpoint{2.718640in}{0.581092in}}%
\pgfpathlineto{\pgfqpoint{2.721861in}{0.582309in}}%
\pgfpathlineto{\pgfqpoint{2.722935in}{0.586233in}}%
\pgfpathlineto{\pgfqpoint{2.724009in}{0.587481in}}%
\pgfpathlineto{\pgfqpoint{2.725083in}{0.585412in}}%
\pgfpathlineto{\pgfqpoint{2.726156in}{0.586294in}}%
\pgfpathlineto{\pgfqpoint{2.729378in}{0.588089in}}%
\pgfpathlineto{\pgfqpoint{2.731526in}{0.586538in}}%
\pgfpathlineto{\pgfqpoint{2.732600in}{0.582096in}}%
\pgfpathlineto{\pgfqpoint{2.739043in}{0.594082in}}%
\pgfpathlineto{\pgfqpoint{2.740116in}{0.590614in}}%
\pgfpathlineto{\pgfqpoint{2.741190in}{0.591648in}}%
\pgfpathlineto{\pgfqpoint{2.744412in}{0.593778in}}%
\pgfpathlineto{\pgfqpoint{2.745486in}{0.596333in}}%
\pgfpathlineto{\pgfqpoint{2.746559in}{0.593656in}}%
\pgfpathlineto{\pgfqpoint{2.748707in}{0.594173in}}%
\pgfpathlineto{\pgfqpoint{2.753002in}{0.595451in}}%
\pgfpathlineto{\pgfqpoint{2.755150in}{0.594691in}}%
\pgfpathlineto{\pgfqpoint{2.756224in}{0.595816in}}%
\pgfpathlineto{\pgfqpoint{2.760519in}{0.593169in}}%
\pgfpathlineto{\pgfqpoint{2.761593in}{0.593656in}}%
\pgfpathlineto{\pgfqpoint{2.762667in}{0.597824in}}%
\pgfpathlineto{\pgfqpoint{2.763741in}{0.597429in}}%
\pgfpathlineto{\pgfqpoint{2.766962in}{0.602570in}}%
\pgfpathlineto{\pgfqpoint{2.768036in}{0.602752in}}%
\pgfpathlineto{\pgfqpoint{2.769110in}{0.601627in}}%
\pgfpathlineto{\pgfqpoint{2.770184in}{0.602387in}}%
\pgfpathlineto{\pgfqpoint{2.771258in}{0.600045in}}%
\pgfpathlineto{\pgfqpoint{2.774479in}{0.598615in}}%
\pgfpathlineto{\pgfqpoint{2.775553in}{0.600349in}}%
\pgfpathlineto{\pgfqpoint{2.776627in}{0.599041in}}%
\pgfpathlineto{\pgfqpoint{2.778775in}{0.600805in}}%
\pgfpathlineto{\pgfqpoint{2.781996in}{0.593200in}}%
\pgfpathlineto{\pgfqpoint{2.783070in}{0.592957in}}%
\pgfpathlineto{\pgfqpoint{2.785218in}{0.597885in}}%
\pgfpathlineto{\pgfqpoint{2.786291in}{0.599862in}}%
\pgfpathlineto{\pgfqpoint{2.790587in}{0.600897in}}%
\pgfpathlineto{\pgfqpoint{2.791661in}{0.600075in}}%
\pgfpathlineto{\pgfqpoint{2.793808in}{0.604334in}}%
\pgfpathlineto{\pgfqpoint{2.797030in}{0.605156in}}%
\pgfpathlineto{\pgfqpoint{2.798104in}{0.606068in}}%
\pgfpathlineto{\pgfqpoint{2.799178in}{0.609476in}}%
\pgfpathlineto{\pgfqpoint{2.800251in}{0.608472in}}%
\pgfpathlineto{\pgfqpoint{2.801325in}{0.610114in}}%
\pgfpathlineto{\pgfqpoint{2.804547in}{0.609050in}}%
\pgfpathlineto{\pgfqpoint{2.805621in}{0.606768in}}%
\pgfpathlineto{\pgfqpoint{2.806694in}{0.607529in}}%
\pgfpathlineto{\pgfqpoint{2.807768in}{0.605186in}}%
\pgfpathlineto{\pgfqpoint{2.808842in}{0.606798in}}%
\pgfpathlineto{\pgfqpoint{2.812064in}{0.606738in}}%
\pgfpathlineto{\pgfqpoint{2.815285in}{0.613552in}}%
\pgfpathlineto{\pgfqpoint{2.816359in}{0.613278in}}%
\pgfpathlineto{\pgfqpoint{2.820654in}{0.614069in}}%
\pgfpathlineto{\pgfqpoint{2.821728in}{0.617355in}}%
\pgfpathlineto{\pgfqpoint{2.822802in}{0.617081in}}%
\pgfpathlineto{\pgfqpoint{2.823876in}{0.618237in}}%
\pgfpathlineto{\pgfqpoint{2.829245in}{0.621553in}}%
\pgfpathlineto{\pgfqpoint{2.831393in}{0.619788in}}%
\pgfpathlineto{\pgfqpoint{2.834614in}{0.618359in}}%
\pgfpathlineto{\pgfqpoint{2.835688in}{0.620093in}}%
\pgfpathlineto{\pgfqpoint{2.836762in}{0.614891in}}%
\pgfpathlineto{\pgfqpoint{2.837836in}{0.617811in}}%
\pgfpathlineto{\pgfqpoint{2.838910in}{0.614191in}}%
\pgfpathlineto{\pgfqpoint{2.842131in}{0.614586in}}%
\pgfpathlineto{\pgfqpoint{2.843205in}{0.619089in}}%
\pgfpathlineto{\pgfqpoint{2.844279in}{0.617203in}}%
\pgfpathlineto{\pgfqpoint{2.846426in}{0.620640in}}%
\pgfpathlineto{\pgfqpoint{2.849648in}{0.621888in}}%
\pgfpathlineto{\pgfqpoint{2.850722in}{0.624838in}}%
\pgfpathlineto{\pgfqpoint{2.851796in}{0.614313in}}%
\pgfpathlineto{\pgfqpoint{2.852869in}{0.622526in}}%
\pgfpathlineto{\pgfqpoint{2.853943in}{0.616929in}}%
\pgfpathlineto{\pgfqpoint{2.857165in}{0.607011in}}%
\pgfpathlineto{\pgfqpoint{2.861460in}{0.622557in}}%
\pgfpathlineto{\pgfqpoint{2.864682in}{0.621857in}}%
\pgfpathlineto{\pgfqpoint{2.865756in}{0.625143in}}%
\pgfpathlineto{\pgfqpoint{2.866829in}{0.624686in}}%
\pgfpathlineto{\pgfqpoint{2.867903in}{0.623500in}}%
\pgfpathlineto{\pgfqpoint{2.868977in}{0.626207in}}%
\pgfpathlineto{\pgfqpoint{2.872199in}{0.625416in}}%
\pgfpathlineto{\pgfqpoint{2.873272in}{0.621157in}}%
\pgfpathlineto{\pgfqpoint{2.874346in}{0.621097in}}%
\pgfpathlineto{\pgfqpoint{2.876494in}{0.622831in}}%
\pgfpathlineto{\pgfqpoint{2.880789in}{0.623713in}}%
\pgfpathlineto{\pgfqpoint{2.881863in}{0.626816in}}%
\pgfpathlineto{\pgfqpoint{2.882937in}{0.627820in}}%
\pgfpathlineto{\pgfqpoint{2.884011in}{0.626816in}}%
\pgfpathlineto{\pgfqpoint{2.884011in}{0.626816in}}%
\pgfusepath{stroke}%
\end{pgfscope}%
\begin{pgfscope}%
\pgfpathrectangle{\pgfqpoint{0.418102in}{0.331635in}}{\pgfqpoint{2.583333in}{0.400885in}}%
\pgfusepath{clip}%
\pgfsetroundcap%
\pgfsetroundjoin%
\pgfsetlinewidth{1.505625pt}%
\definecolor{currentstroke}{rgb}{0.737255,0.741176,0.133333}%
\pgfsetstrokecolor{currentstroke}%
\pgfsetdash{}{0pt}%
\pgfpathmoveto{\pgfqpoint{0.535526in}{0.352412in}}%
\pgfpathlineto{\pgfqpoint{0.537674in}{0.352939in}}%
\pgfpathlineto{\pgfqpoint{0.538747in}{0.353744in}}%
\pgfpathlineto{\pgfqpoint{0.543043in}{0.353110in}}%
\pgfpathlineto{\pgfqpoint{0.544117in}{0.353586in}}%
\pgfpathlineto{\pgfqpoint{0.545190in}{0.355128in}}%
\pgfpathlineto{\pgfqpoint{0.550560in}{0.356815in}}%
\pgfpathlineto{\pgfqpoint{0.551633in}{0.355904in}}%
\pgfpathlineto{\pgfqpoint{0.552707in}{0.356887in}}%
\pgfpathlineto{\pgfqpoint{0.553781in}{0.355616in}}%
\pgfpathlineto{\pgfqpoint{0.557003in}{0.354899in}}%
\pgfpathlineto{\pgfqpoint{0.559150in}{0.356235in}}%
\pgfpathlineto{\pgfqpoint{0.565593in}{0.357757in}}%
\pgfpathlineto{\pgfqpoint{0.566667in}{0.356381in}}%
\pgfpathlineto{\pgfqpoint{0.567741in}{0.353859in}}%
\pgfpathlineto{\pgfqpoint{0.568815in}{0.353226in}}%
\pgfpathlineto{\pgfqpoint{0.573110in}{0.353254in}}%
\pgfpathlineto{\pgfqpoint{0.574184in}{0.354144in}}%
\pgfpathlineto{\pgfqpoint{0.575258in}{0.351506in}}%
\pgfpathlineto{\pgfqpoint{0.576332in}{0.350626in}}%
\pgfpathlineto{\pgfqpoint{0.579553in}{0.351322in}}%
\pgfpathlineto{\pgfqpoint{0.580627in}{0.352827in}}%
\pgfpathlineto{\pgfqpoint{0.581701in}{0.352362in}}%
\pgfpathlineto{\pgfqpoint{0.583849in}{0.353890in}}%
\pgfpathlineto{\pgfqpoint{0.588144in}{0.354387in}}%
\pgfpathlineto{\pgfqpoint{0.589218in}{0.355731in}}%
\pgfpathlineto{\pgfqpoint{0.590292in}{0.355675in}}%
\pgfpathlineto{\pgfqpoint{0.591366in}{0.356346in}}%
\pgfpathlineto{\pgfqpoint{0.594587in}{0.356765in}}%
\pgfpathlineto{\pgfqpoint{0.597809in}{0.360211in}}%
\pgfpathlineto{\pgfqpoint{0.598882in}{0.360890in}}%
\pgfpathlineto{\pgfqpoint{0.602104in}{0.360950in}}%
\pgfpathlineto{\pgfqpoint{0.603178in}{0.360018in}}%
\pgfpathlineto{\pgfqpoint{0.604252in}{0.360559in}}%
\pgfpathlineto{\pgfqpoint{0.605325in}{0.359026in}}%
\pgfpathlineto{\pgfqpoint{0.606399in}{0.359546in}}%
\pgfpathlineto{\pgfqpoint{0.609621in}{0.359136in}}%
\pgfpathlineto{\pgfqpoint{0.612842in}{0.360163in}}%
\pgfpathlineto{\pgfqpoint{0.613916in}{0.359956in}}%
\pgfpathlineto{\pgfqpoint{0.617138in}{0.361345in}}%
\pgfpathlineto{\pgfqpoint{0.618211in}{0.362853in}}%
\pgfpathlineto{\pgfqpoint{0.620359in}{0.363376in}}%
\pgfpathlineto{\pgfqpoint{0.621433in}{0.362362in}}%
\pgfpathlineto{\pgfqpoint{0.628950in}{0.360491in}}%
\pgfpathlineto{\pgfqpoint{0.632171in}{0.361110in}}%
\pgfpathlineto{\pgfqpoint{0.633245in}{0.360197in}}%
\pgfpathlineto{\pgfqpoint{0.635393in}{0.362413in}}%
\pgfpathlineto{\pgfqpoint{0.639688in}{0.363384in}}%
\pgfpathlineto{\pgfqpoint{0.640762in}{0.361544in}}%
\pgfpathlineto{\pgfqpoint{0.641836in}{0.361966in}}%
\pgfpathlineto{\pgfqpoint{0.643984in}{0.358219in}}%
\pgfpathlineto{\pgfqpoint{0.649353in}{0.360611in}}%
\pgfpathlineto{\pgfqpoint{0.655796in}{0.358810in}}%
\pgfpathlineto{\pgfqpoint{0.657944in}{0.356165in}}%
\pgfpathlineto{\pgfqpoint{0.659017in}{0.355979in}}%
\pgfpathlineto{\pgfqpoint{0.663313in}{0.356377in}}%
\pgfpathlineto{\pgfqpoint{0.664387in}{0.355842in}}%
\pgfpathlineto{\pgfqpoint{0.665460in}{0.352450in}}%
\pgfpathlineto{\pgfqpoint{0.666534in}{0.353251in}}%
\pgfpathlineto{\pgfqpoint{0.671903in}{0.352610in}}%
\pgfpathlineto{\pgfqpoint{0.674051in}{0.353167in}}%
\pgfpathlineto{\pgfqpoint{0.678346in}{0.352395in}}%
\pgfpathlineto{\pgfqpoint{0.680494in}{0.354953in}}%
\pgfpathlineto{\pgfqpoint{0.681568in}{0.353479in}}%
\pgfpathlineto{\pgfqpoint{0.684789in}{0.355704in}}%
\pgfpathlineto{\pgfqpoint{0.686937in}{0.353913in}}%
\pgfpathlineto{\pgfqpoint{0.688011in}{0.353543in}}%
\pgfpathlineto{\pgfqpoint{0.693380in}{0.354306in}}%
\pgfpathlineto{\pgfqpoint{0.694454in}{0.355940in}}%
\pgfpathlineto{\pgfqpoint{0.696602in}{0.352737in}}%
\pgfpathlineto{\pgfqpoint{0.700897in}{0.354118in}}%
\pgfpathlineto{\pgfqpoint{0.701971in}{0.355701in}}%
\pgfpathlineto{\pgfqpoint{0.704119in}{0.355561in}}%
\pgfpathlineto{\pgfqpoint{0.708414in}{0.355921in}}%
\pgfpathlineto{\pgfqpoint{0.710562in}{0.358312in}}%
\pgfpathlineto{\pgfqpoint{0.711635in}{0.357104in}}%
\pgfpathlineto{\pgfqpoint{0.715931in}{0.355288in}}%
\pgfpathlineto{\pgfqpoint{0.717005in}{0.354573in}}%
\pgfpathlineto{\pgfqpoint{0.718078in}{0.356365in}}%
\pgfpathlineto{\pgfqpoint{0.719152in}{0.359697in}}%
\pgfpathlineto{\pgfqpoint{0.726669in}{0.365590in}}%
\pgfpathlineto{\pgfqpoint{0.729891in}{0.363626in}}%
\pgfpathlineto{\pgfqpoint{0.733112in}{0.364224in}}%
\pgfpathlineto{\pgfqpoint{0.734186in}{0.365334in}}%
\pgfpathlineto{\pgfqpoint{0.738481in}{0.363402in}}%
\pgfpathlineto{\pgfqpoint{0.739555in}{0.362378in}}%
\pgfpathlineto{\pgfqpoint{0.740629in}{0.363636in}}%
\pgfpathlineto{\pgfqpoint{0.741703in}{0.362700in}}%
\pgfpathlineto{\pgfqpoint{0.745998in}{0.360102in}}%
\pgfpathlineto{\pgfqpoint{0.747072in}{0.360343in}}%
\pgfpathlineto{\pgfqpoint{0.748146in}{0.358889in}}%
\pgfpathlineto{\pgfqpoint{0.752441in}{0.359885in}}%
\pgfpathlineto{\pgfqpoint{0.753515in}{0.358786in}}%
\pgfpathlineto{\pgfqpoint{0.754589in}{0.358511in}}%
\pgfpathlineto{\pgfqpoint{0.755663in}{0.361287in}}%
\pgfpathlineto{\pgfqpoint{0.756737in}{0.359830in}}%
\pgfpathlineto{\pgfqpoint{0.759958in}{0.358824in}}%
\pgfpathlineto{\pgfqpoint{0.761032in}{0.359856in}}%
\pgfpathlineto{\pgfqpoint{0.762106in}{0.358928in}}%
\pgfpathlineto{\pgfqpoint{0.763180in}{0.359988in}}%
\pgfpathlineto{\pgfqpoint{0.764254in}{0.358981in}}%
\pgfpathlineto{\pgfqpoint{0.769623in}{0.359033in}}%
\pgfpathlineto{\pgfqpoint{0.770697in}{0.360554in}}%
\pgfpathlineto{\pgfqpoint{0.771770in}{0.360904in}}%
\pgfpathlineto{\pgfqpoint{0.774992in}{0.361360in}}%
\pgfpathlineto{\pgfqpoint{0.776066in}{0.362311in}}%
\pgfpathlineto{\pgfqpoint{0.778213in}{0.362664in}}%
\pgfpathlineto{\pgfqpoint{0.779287in}{0.363289in}}%
\pgfpathlineto{\pgfqpoint{0.784656in}{0.362930in}}%
\pgfpathlineto{\pgfqpoint{0.785730in}{0.363510in}}%
\pgfpathlineto{\pgfqpoint{0.786804in}{0.363202in}}%
\pgfpathlineto{\pgfqpoint{0.792173in}{0.364323in}}%
\pgfpathlineto{\pgfqpoint{0.794321in}{0.366142in}}%
\pgfpathlineto{\pgfqpoint{0.798616in}{0.365970in}}%
\pgfpathlineto{\pgfqpoint{0.799690in}{0.366569in}}%
\pgfpathlineto{\pgfqpoint{0.800764in}{0.367815in}}%
\pgfpathlineto{\pgfqpoint{0.805059in}{0.367035in}}%
\pgfpathlineto{\pgfqpoint{0.806133in}{0.368855in}}%
\pgfpathlineto{\pgfqpoint{0.808281in}{0.366630in}}%
\pgfpathlineto{\pgfqpoint{0.809355in}{0.367069in}}%
\pgfpathlineto{\pgfqpoint{0.813650in}{0.366709in}}%
\pgfpathlineto{\pgfqpoint{0.815798in}{0.367512in}}%
\pgfpathlineto{\pgfqpoint{0.821167in}{0.369106in}}%
\pgfpathlineto{\pgfqpoint{0.823315in}{0.371313in}}%
\pgfpathlineto{\pgfqpoint{0.824388in}{0.371051in}}%
\pgfpathlineto{\pgfqpoint{0.827610in}{0.369484in}}%
\pgfpathlineto{\pgfqpoint{0.828684in}{0.369932in}}%
\pgfpathlineto{\pgfqpoint{0.829758in}{0.371516in}}%
\pgfpathlineto{\pgfqpoint{0.831905in}{0.370419in}}%
\pgfpathlineto{\pgfqpoint{0.835127in}{0.371357in}}%
\pgfpathlineto{\pgfqpoint{0.836201in}{0.370116in}}%
\pgfpathlineto{\pgfqpoint{0.837275in}{0.370539in}}%
\pgfpathlineto{\pgfqpoint{0.838348in}{0.369524in}}%
\pgfpathlineto{\pgfqpoint{0.839422in}{0.369469in}}%
\pgfpathlineto{\pgfqpoint{0.842644in}{0.369937in}}%
\pgfpathlineto{\pgfqpoint{0.844791in}{0.368302in}}%
\pgfpathlineto{\pgfqpoint{0.845865in}{0.368699in}}%
\pgfpathlineto{\pgfqpoint{0.846939in}{0.367550in}}%
\pgfpathlineto{\pgfqpoint{0.850161in}{0.366962in}}%
\pgfpathlineto{\pgfqpoint{0.851234in}{0.365545in}}%
\pgfpathlineto{\pgfqpoint{0.852308in}{0.365465in}}%
\pgfpathlineto{\pgfqpoint{0.853382in}{0.366107in}}%
\pgfpathlineto{\pgfqpoint{0.854456in}{0.365652in}}%
\pgfpathlineto{\pgfqpoint{0.859825in}{0.365388in}}%
\pgfpathlineto{\pgfqpoint{0.860899in}{0.362394in}}%
\pgfpathlineto{\pgfqpoint{0.861973in}{0.362663in}}%
\pgfpathlineto{\pgfqpoint{0.865194in}{0.361631in}}%
\pgfpathlineto{\pgfqpoint{0.867342in}{0.363645in}}%
\pgfpathlineto{\pgfqpoint{0.868416in}{0.363370in}}%
\pgfpathlineto{\pgfqpoint{0.869490in}{0.363845in}}%
\pgfpathlineto{\pgfqpoint{0.875933in}{0.363489in}}%
\pgfpathlineto{\pgfqpoint{0.877007in}{0.365025in}}%
\pgfpathlineto{\pgfqpoint{0.882376in}{0.362809in}}%
\pgfpathlineto{\pgfqpoint{0.884523in}{0.362022in}}%
\pgfpathlineto{\pgfqpoint{0.889893in}{0.362513in}}%
\pgfpathlineto{\pgfqpoint{0.892040in}{0.361222in}}%
\pgfpathlineto{\pgfqpoint{0.895262in}{0.361771in}}%
\pgfpathlineto{\pgfqpoint{0.896336in}{0.361214in}}%
\pgfpathlineto{\pgfqpoint{0.898483in}{0.361261in}}%
\pgfpathlineto{\pgfqpoint{0.903853in}{0.361399in}}%
\pgfpathlineto{\pgfqpoint{0.904926in}{0.361929in}}%
\pgfpathlineto{\pgfqpoint{0.907074in}{0.361368in}}%
\pgfpathlineto{\pgfqpoint{0.910296in}{0.362584in}}%
\pgfpathlineto{\pgfqpoint{0.911369in}{0.361742in}}%
\pgfpathlineto{\pgfqpoint{0.912443in}{0.362543in}}%
\pgfpathlineto{\pgfqpoint{0.914591in}{0.365159in}}%
\pgfpathlineto{\pgfqpoint{0.917812in}{0.365374in}}%
\pgfpathlineto{\pgfqpoint{0.921034in}{0.365944in}}%
\pgfpathlineto{\pgfqpoint{0.922108in}{0.365530in}}%
\pgfpathlineto{\pgfqpoint{0.925329in}{0.367136in}}%
\pgfpathlineto{\pgfqpoint{0.927477in}{0.365043in}}%
\pgfpathlineto{\pgfqpoint{0.928551in}{0.364974in}}%
\pgfpathlineto{\pgfqpoint{0.929625in}{0.365621in}}%
\pgfpathlineto{\pgfqpoint{0.933920in}{0.364269in}}%
\pgfpathlineto{\pgfqpoint{0.934994in}{0.363040in}}%
\pgfpathlineto{\pgfqpoint{0.937142in}{0.363977in}}%
\pgfpathlineto{\pgfqpoint{0.942511in}{0.363997in}}%
\pgfpathlineto{\pgfqpoint{0.943585in}{0.363842in}}%
\pgfpathlineto{\pgfqpoint{0.944658in}{0.363023in}}%
\pgfpathlineto{\pgfqpoint{0.952175in}{0.363510in}}%
\pgfpathlineto{\pgfqpoint{0.957544in}{0.366268in}}%
\pgfpathlineto{\pgfqpoint{0.958618in}{0.367798in}}%
\pgfpathlineto{\pgfqpoint{0.959692in}{0.367445in}}%
\pgfpathlineto{\pgfqpoint{0.962914in}{0.368449in}}%
\pgfpathlineto{\pgfqpoint{0.963987in}{0.370003in}}%
\pgfpathlineto{\pgfqpoint{0.965061in}{0.369309in}}%
\pgfpathlineto{\pgfqpoint{0.966135in}{0.371309in}}%
\pgfpathlineto{\pgfqpoint{0.973652in}{0.373930in}}%
\pgfpathlineto{\pgfqpoint{0.974726in}{0.372764in}}%
\pgfpathlineto{\pgfqpoint{0.979021in}{0.372961in}}%
\pgfpathlineto{\pgfqpoint{0.980095in}{0.371720in}}%
\pgfpathlineto{\pgfqpoint{0.981169in}{0.372862in}}%
\pgfpathlineto{\pgfqpoint{0.982243in}{0.371745in}}%
\pgfpathlineto{\pgfqpoint{0.986538in}{0.374845in}}%
\pgfpathlineto{\pgfqpoint{0.987612in}{0.373681in}}%
\pgfpathlineto{\pgfqpoint{0.988686in}{0.374371in}}%
\pgfpathlineto{\pgfqpoint{0.989760in}{0.374046in}}%
\pgfpathlineto{\pgfqpoint{0.992981in}{0.374873in}}%
\pgfpathlineto{\pgfqpoint{0.995129in}{0.374000in}}%
\pgfpathlineto{\pgfqpoint{0.997276in}{0.374369in}}%
\pgfpathlineto{\pgfqpoint{1.000498in}{0.374666in}}%
\pgfpathlineto{\pgfqpoint{1.001572in}{0.375457in}}%
\pgfpathlineto{\pgfqpoint{1.002646in}{0.375231in}}%
\pgfpathlineto{\pgfqpoint{1.004793in}{0.377169in}}%
\pgfpathlineto{\pgfqpoint{1.008015in}{0.376885in}}%
\pgfpathlineto{\pgfqpoint{1.009089in}{0.375670in}}%
\pgfpathlineto{\pgfqpoint{1.011236in}{0.378539in}}%
\pgfpathlineto{\pgfqpoint{1.012310in}{0.379909in}}%
\pgfpathlineto{\pgfqpoint{1.015532in}{0.377564in}}%
\pgfpathlineto{\pgfqpoint{1.016606in}{0.375582in}}%
\pgfpathlineto{\pgfqpoint{1.018753in}{0.374292in}}%
\pgfpathlineto{\pgfqpoint{1.024122in}{0.375962in}}%
\pgfpathlineto{\pgfqpoint{1.026270in}{0.378696in}}%
\pgfpathlineto{\pgfqpoint{1.027344in}{0.379447in}}%
\pgfpathlineto{\pgfqpoint{1.030565in}{0.380108in}}%
\pgfpathlineto{\pgfqpoint{1.032713in}{0.382318in}}%
\pgfpathlineto{\pgfqpoint{1.033787in}{0.381955in}}%
\pgfpathlineto{\pgfqpoint{1.034861in}{0.383035in}}%
\pgfpathlineto{\pgfqpoint{1.038082in}{0.380355in}}%
\pgfpathlineto{\pgfqpoint{1.040230in}{0.383876in}}%
\pgfpathlineto{\pgfqpoint{1.041304in}{0.383361in}}%
\pgfpathlineto{\pgfqpoint{1.042378in}{0.384931in}}%
\pgfpathlineto{\pgfqpoint{1.045599in}{0.385202in}}%
\pgfpathlineto{\pgfqpoint{1.046673in}{0.386536in}}%
\pgfpathlineto{\pgfqpoint{1.048821in}{0.384648in}}%
\pgfpathlineto{\pgfqpoint{1.049895in}{0.385540in}}%
\pgfpathlineto{\pgfqpoint{1.054190in}{0.385593in}}%
\pgfpathlineto{\pgfqpoint{1.055264in}{0.387078in}}%
\pgfpathlineto{\pgfqpoint{1.056338in}{0.392921in}}%
\pgfpathlineto{\pgfqpoint{1.057411in}{0.390354in}}%
\pgfpathlineto{\pgfqpoint{1.061707in}{0.391296in}}%
\pgfpathlineto{\pgfqpoint{1.062781in}{0.391637in}}%
\pgfpathlineto{\pgfqpoint{1.064928in}{0.391347in}}%
\pgfpathlineto{\pgfqpoint{1.068150in}{0.391452in}}%
\pgfpathlineto{\pgfqpoint{1.069224in}{0.392297in}}%
\pgfpathlineto{\pgfqpoint{1.070297in}{0.390634in}}%
\pgfpathlineto{\pgfqpoint{1.071371in}{0.391786in}}%
\pgfpathlineto{\pgfqpoint{1.072445in}{0.394638in}}%
\pgfpathlineto{\pgfqpoint{1.076741in}{0.396391in}}%
\pgfpathlineto{\pgfqpoint{1.078888in}{0.394256in}}%
\pgfpathlineto{\pgfqpoint{1.079962in}{0.395715in}}%
\pgfpathlineto{\pgfqpoint{1.084257in}{0.395985in}}%
\pgfpathlineto{\pgfqpoint{1.085331in}{0.394818in}}%
\pgfpathlineto{\pgfqpoint{1.087479in}{0.398402in}}%
\pgfpathlineto{\pgfqpoint{1.091774in}{0.399719in}}%
\pgfpathlineto{\pgfqpoint{1.092848in}{0.397813in}}%
\pgfpathlineto{\pgfqpoint{1.093922in}{0.399187in}}%
\pgfpathlineto{\pgfqpoint{1.098217in}{0.397291in}}%
\pgfpathlineto{\pgfqpoint{1.099291in}{0.398945in}}%
\pgfpathlineto{\pgfqpoint{1.100365in}{0.398610in}}%
\pgfpathlineto{\pgfqpoint{1.102513in}{0.401430in}}%
\pgfpathlineto{\pgfqpoint{1.105734in}{0.402337in}}%
\pgfpathlineto{\pgfqpoint{1.106808in}{0.401232in}}%
\pgfpathlineto{\pgfqpoint{1.107882in}{0.402122in}}%
\pgfpathlineto{\pgfqpoint{1.108956in}{0.399412in}}%
\pgfpathlineto{\pgfqpoint{1.110030in}{0.400089in}}%
\pgfpathlineto{\pgfqpoint{1.113251in}{0.400879in}}%
\pgfpathlineto{\pgfqpoint{1.114325in}{0.402310in}}%
\pgfpathlineto{\pgfqpoint{1.116473in}{0.399846in}}%
\pgfpathlineto{\pgfqpoint{1.117546in}{0.400779in}}%
\pgfpathlineto{\pgfqpoint{1.122916in}{0.403456in}}%
\pgfpathlineto{\pgfqpoint{1.125063in}{0.401059in}}%
\pgfpathlineto{\pgfqpoint{1.128285in}{0.402719in}}%
\pgfpathlineto{\pgfqpoint{1.130432in}{0.401837in}}%
\pgfpathlineto{\pgfqpoint{1.131506in}{0.403903in}}%
\pgfpathlineto{\pgfqpoint{1.132580in}{0.403435in}}%
\pgfpathlineto{\pgfqpoint{1.135802in}{0.403708in}}%
\pgfpathlineto{\pgfqpoint{1.136875in}{0.403159in}}%
\pgfpathlineto{\pgfqpoint{1.137949in}{0.403324in}}%
\pgfpathlineto{\pgfqpoint{1.140097in}{0.404862in}}%
\pgfpathlineto{\pgfqpoint{1.143319in}{0.405695in}}%
\pgfpathlineto{\pgfqpoint{1.144392in}{0.407389in}}%
\pgfpathlineto{\pgfqpoint{1.145466in}{0.406190in}}%
\pgfpathlineto{\pgfqpoint{1.146540in}{0.411359in}}%
\pgfpathlineto{\pgfqpoint{1.147614in}{0.412301in}}%
\pgfpathlineto{\pgfqpoint{1.151909in}{0.411142in}}%
\pgfpathlineto{\pgfqpoint{1.152983in}{0.401548in}}%
\pgfpathlineto{\pgfqpoint{1.154057in}{0.402969in}}%
\pgfpathlineto{\pgfqpoint{1.155131in}{0.399751in}}%
\pgfpathlineto{\pgfqpoint{1.158352in}{0.399394in}}%
\pgfpathlineto{\pgfqpoint{1.159426in}{0.400431in}}%
\pgfpathlineto{\pgfqpoint{1.161574in}{0.399070in}}%
\pgfpathlineto{\pgfqpoint{1.162648in}{0.398459in}}%
\pgfpathlineto{\pgfqpoint{1.168017in}{0.398375in}}%
\pgfpathlineto{\pgfqpoint{1.169091in}{0.401209in}}%
\pgfpathlineto{\pgfqpoint{1.170164in}{0.400655in}}%
\pgfpathlineto{\pgfqpoint{1.173386in}{0.401910in}}%
\pgfpathlineto{\pgfqpoint{1.174460in}{0.403107in}}%
\pgfpathlineto{\pgfqpoint{1.175534in}{0.406682in}}%
\pgfpathlineto{\pgfqpoint{1.177681in}{0.406206in}}%
\pgfpathlineto{\pgfqpoint{1.181977in}{0.402836in}}%
\pgfpathlineto{\pgfqpoint{1.183051in}{0.403546in}}%
\pgfpathlineto{\pgfqpoint{1.185198in}{0.402778in}}%
\pgfpathlineto{\pgfqpoint{1.192715in}{0.405437in}}%
\pgfpathlineto{\pgfqpoint{1.195937in}{0.404126in}}%
\pgfpathlineto{\pgfqpoint{1.197010in}{0.400075in}}%
\pgfpathlineto{\pgfqpoint{1.198084in}{0.398985in}}%
\pgfpathlineto{\pgfqpoint{1.199158in}{0.399761in}}%
\pgfpathlineto{\pgfqpoint{1.200232in}{0.402227in}}%
\pgfpathlineto{\pgfqpoint{1.203453in}{0.401983in}}%
\pgfpathlineto{\pgfqpoint{1.207749in}{0.396297in}}%
\pgfpathlineto{\pgfqpoint{1.210970in}{0.397765in}}%
\pgfpathlineto{\pgfqpoint{1.213118in}{0.395052in}}%
\pgfpathlineto{\pgfqpoint{1.214192in}{0.396207in}}%
\pgfpathlineto{\pgfqpoint{1.215266in}{0.396484in}}%
\pgfpathlineto{\pgfqpoint{1.218487in}{0.395348in}}%
\pgfpathlineto{\pgfqpoint{1.220635in}{0.397418in}}%
\pgfpathlineto{\pgfqpoint{1.221709in}{0.395548in}}%
\pgfpathlineto{\pgfqpoint{1.226004in}{0.399083in}}%
\pgfpathlineto{\pgfqpoint{1.227078in}{0.396755in}}%
\pgfpathlineto{\pgfqpoint{1.228152in}{0.397558in}}%
\pgfpathlineto{\pgfqpoint{1.230299in}{0.392500in}}%
\pgfpathlineto{\pgfqpoint{1.233521in}{0.391791in}}%
\pgfpathlineto{\pgfqpoint{1.234595in}{0.392949in}}%
\pgfpathlineto{\pgfqpoint{1.235669in}{0.395367in}}%
\pgfpathlineto{\pgfqpoint{1.237816in}{0.392637in}}%
\pgfpathlineto{\pgfqpoint{1.242112in}{0.392849in}}%
\pgfpathlineto{\pgfqpoint{1.243185in}{0.392259in}}%
\pgfpathlineto{\pgfqpoint{1.244259in}{0.394452in}}%
\pgfpathlineto{\pgfqpoint{1.248555in}{0.394382in}}%
\pgfpathlineto{\pgfqpoint{1.250702in}{0.395229in}}%
\pgfpathlineto{\pgfqpoint{1.251776in}{0.391305in}}%
\pgfpathlineto{\pgfqpoint{1.252850in}{0.392676in}}%
\pgfpathlineto{\pgfqpoint{1.257145in}{0.394700in}}%
\pgfpathlineto{\pgfqpoint{1.258219in}{0.393777in}}%
\pgfpathlineto{\pgfqpoint{1.260367in}{0.396657in}}%
\pgfpathlineto{\pgfqpoint{1.263588in}{0.396116in}}%
\pgfpathlineto{\pgfqpoint{1.264662in}{0.396676in}}%
\pgfpathlineto{\pgfqpoint{1.265736in}{0.398474in}}%
\pgfpathlineto{\pgfqpoint{1.266810in}{0.398277in}}%
\pgfpathlineto{\pgfqpoint{1.267884in}{0.397222in}}%
\pgfpathlineto{\pgfqpoint{1.271105in}{0.398378in}}%
\pgfpathlineto{\pgfqpoint{1.272179in}{0.396927in}}%
\pgfpathlineto{\pgfqpoint{1.273253in}{0.397345in}}%
\pgfpathlineto{\pgfqpoint{1.274327in}{0.395354in}}%
\pgfpathlineto{\pgfqpoint{1.275401in}{0.395069in}}%
\pgfpathlineto{\pgfqpoint{1.278622in}{0.395163in}}%
\pgfpathlineto{\pgfqpoint{1.279696in}{0.396039in}}%
\pgfpathlineto{\pgfqpoint{1.280770in}{0.395778in}}%
\pgfpathlineto{\pgfqpoint{1.286139in}{0.397008in}}%
\pgfpathlineto{\pgfqpoint{1.287213in}{0.398855in}}%
\pgfpathlineto{\pgfqpoint{1.290434in}{0.400079in}}%
\pgfpathlineto{\pgfqpoint{1.293656in}{0.399931in}}%
\pgfpathlineto{\pgfqpoint{1.294730in}{0.398695in}}%
\pgfpathlineto{\pgfqpoint{1.295804in}{0.402279in}}%
\pgfpathlineto{\pgfqpoint{1.296877in}{0.403564in}}%
\pgfpathlineto{\pgfqpoint{1.297951in}{0.405862in}}%
\pgfpathlineto{\pgfqpoint{1.301173in}{0.405635in}}%
\pgfpathlineto{\pgfqpoint{1.302247in}{0.402389in}}%
\pgfpathlineto{\pgfqpoint{1.304394in}{0.400846in}}%
\pgfpathlineto{\pgfqpoint{1.305468in}{0.400892in}}%
\pgfpathlineto{\pgfqpoint{1.308690in}{0.402009in}}%
\pgfpathlineto{\pgfqpoint{1.311911in}{0.400851in}}%
\pgfpathlineto{\pgfqpoint{1.316207in}{0.401789in}}%
\pgfpathlineto{\pgfqpoint{1.317280in}{0.400849in}}%
\pgfpathlineto{\pgfqpoint{1.320502in}{0.401821in}}%
\pgfpathlineto{\pgfqpoint{1.323723in}{0.401111in}}%
\pgfpathlineto{\pgfqpoint{1.324797in}{0.402004in}}%
\pgfpathlineto{\pgfqpoint{1.331240in}{0.400200in}}%
\pgfpathlineto{\pgfqpoint{1.332314in}{0.402207in}}%
\pgfpathlineto{\pgfqpoint{1.333388in}{0.401637in}}%
\pgfpathlineto{\pgfqpoint{1.334462in}{0.402676in}}%
\pgfpathlineto{\pgfqpoint{1.335536in}{0.408265in}}%
\pgfpathlineto{\pgfqpoint{1.340905in}{0.408956in}}%
\pgfpathlineto{\pgfqpoint{1.341979in}{0.411491in}}%
\pgfpathlineto{\pgfqpoint{1.343052in}{0.407579in}}%
\pgfpathlineto{\pgfqpoint{1.346274in}{0.404771in}}%
\pgfpathlineto{\pgfqpoint{1.350569in}{0.414864in}}%
\pgfpathlineto{\pgfqpoint{1.353791in}{0.413690in}}%
\pgfpathlineto{\pgfqpoint{1.354865in}{0.414342in}}%
\pgfpathlineto{\pgfqpoint{1.355939in}{0.413690in}}%
\pgfpathlineto{\pgfqpoint{1.358086in}{0.410045in}}%
\pgfpathlineto{\pgfqpoint{1.361308in}{0.410729in}}%
\pgfpathlineto{\pgfqpoint{1.362382in}{0.411741in}}%
\pgfpathlineto{\pgfqpoint{1.363455in}{0.409741in}}%
\pgfpathlineto{\pgfqpoint{1.364529in}{0.409813in}}%
\pgfpathlineto{\pgfqpoint{1.365603in}{0.410915in}}%
\pgfpathlineto{\pgfqpoint{1.369898in}{0.410891in}}%
\pgfpathlineto{\pgfqpoint{1.370972in}{0.409670in}}%
\pgfpathlineto{\pgfqpoint{1.373120in}{0.409382in}}%
\pgfpathlineto{\pgfqpoint{1.376341in}{0.410963in}}%
\pgfpathlineto{\pgfqpoint{1.377415in}{0.410436in}}%
\pgfpathlineto{\pgfqpoint{1.379563in}{0.411102in}}%
\pgfpathlineto{\pgfqpoint{1.380637in}{0.410910in}}%
\pgfpathlineto{\pgfqpoint{1.383858in}{0.408394in}}%
\pgfpathlineto{\pgfqpoint{1.386006in}{0.412180in}}%
\pgfpathlineto{\pgfqpoint{1.387080in}{0.412769in}}%
\pgfpathlineto{\pgfqpoint{1.388154in}{0.411788in}}%
\pgfpathlineto{\pgfqpoint{1.391375in}{0.411812in}}%
\pgfpathlineto{\pgfqpoint{1.393523in}{0.413359in}}%
\pgfpathlineto{\pgfqpoint{1.394597in}{0.410308in}}%
\pgfpathlineto{\pgfqpoint{1.395671in}{0.410308in}}%
\pgfpathlineto{\pgfqpoint{1.398892in}{0.412041in}}%
\pgfpathlineto{\pgfqpoint{1.399966in}{0.410552in}}%
\pgfpathlineto{\pgfqpoint{1.401040in}{0.411983in}}%
\pgfpathlineto{\pgfqpoint{1.402114in}{0.410837in}}%
\pgfpathlineto{\pgfqpoint{1.403187in}{0.411715in}}%
\pgfpathlineto{\pgfqpoint{1.406409in}{0.413203in}}%
\pgfpathlineto{\pgfqpoint{1.408557in}{0.410232in}}%
\pgfpathlineto{\pgfqpoint{1.409630in}{0.410282in}}%
\pgfpathlineto{\pgfqpoint{1.410704in}{0.408110in}}%
\pgfpathlineto{\pgfqpoint{1.415000in}{0.411006in}}%
\pgfpathlineto{\pgfqpoint{1.417147in}{0.411182in}}%
\pgfpathlineto{\pgfqpoint{1.418221in}{0.415523in}}%
\pgfpathlineto{\pgfqpoint{1.422517in}{0.412281in}}%
\pgfpathlineto{\pgfqpoint{1.424664in}{0.419251in}}%
\pgfpathlineto{\pgfqpoint{1.425738in}{0.415959in}}%
\pgfpathlineto{\pgfqpoint{1.428960in}{0.418879in}}%
\pgfpathlineto{\pgfqpoint{1.430033in}{0.416818in}}%
\pgfpathlineto{\pgfqpoint{1.431107in}{0.413375in}}%
\pgfpathlineto{\pgfqpoint{1.432181in}{0.414245in}}%
\pgfpathlineto{\pgfqpoint{1.436476in}{0.414993in}}%
\pgfpathlineto{\pgfqpoint{1.437550in}{0.414459in}}%
\pgfpathlineto{\pgfqpoint{1.439698in}{0.415523in}}%
\pgfpathlineto{\pgfqpoint{1.440772in}{0.422088in}}%
\pgfpathlineto{\pgfqpoint{1.443993in}{0.423805in}}%
\pgfpathlineto{\pgfqpoint{1.445067in}{0.422946in}}%
\pgfpathlineto{\pgfqpoint{1.446141in}{0.422976in}}%
\pgfpathlineto{\pgfqpoint{1.448289in}{0.426545in}}%
\pgfpathlineto{\pgfqpoint{1.452584in}{0.429699in}}%
\pgfpathlineto{\pgfqpoint{1.453658in}{0.432090in}}%
\pgfpathlineto{\pgfqpoint{1.454732in}{0.430546in}}%
\pgfpathlineto{\pgfqpoint{1.461175in}{0.431405in}}%
\pgfpathlineto{\pgfqpoint{1.462249in}{0.429710in}}%
\pgfpathlineto{\pgfqpoint{1.463322in}{0.431375in}}%
\pgfpathlineto{\pgfqpoint{1.466544in}{0.430988in}}%
\pgfpathlineto{\pgfqpoint{1.468692in}{0.434218in}}%
\pgfpathlineto{\pgfqpoint{1.469765in}{0.434733in}}%
\pgfpathlineto{\pgfqpoint{1.470839in}{0.436683in}}%
\pgfpathlineto{\pgfqpoint{1.475135in}{0.434946in}}%
\pgfpathlineto{\pgfqpoint{1.478356in}{0.435903in}}%
\pgfpathlineto{\pgfqpoint{1.481578in}{0.434975in}}%
\pgfpathlineto{\pgfqpoint{1.482651in}{0.433418in}}%
\pgfpathlineto{\pgfqpoint{1.489095in}{0.433775in}}%
\pgfpathlineto{\pgfqpoint{1.491242in}{0.436042in}}%
\pgfpathlineto{\pgfqpoint{1.493390in}{0.435043in}}%
\pgfpathlineto{\pgfqpoint{1.496611in}{0.434286in}}%
\pgfpathlineto{\pgfqpoint{1.497685in}{0.434710in}}%
\pgfpathlineto{\pgfqpoint{1.498759in}{0.434408in}}%
\pgfpathlineto{\pgfqpoint{1.499833in}{0.435071in}}%
\pgfpathlineto{\pgfqpoint{1.500907in}{0.434493in}}%
\pgfpathlineto{\pgfqpoint{1.504128in}{0.434463in}}%
\pgfpathlineto{\pgfqpoint{1.505202in}{0.433277in}}%
\pgfpathlineto{\pgfqpoint{1.506276in}{0.434189in}}%
\pgfpathlineto{\pgfqpoint{1.508424in}{0.434524in}}%
\pgfpathlineto{\pgfqpoint{1.511645in}{0.433488in}}%
\pgfpathlineto{\pgfqpoint{1.512719in}{0.430996in}}%
\pgfpathlineto{\pgfqpoint{1.513793in}{0.430589in}}%
\pgfpathlineto{\pgfqpoint{1.514867in}{0.429434in}}%
\pgfpathlineto{\pgfqpoint{1.519162in}{0.429292in}}%
\pgfpathlineto{\pgfqpoint{1.520236in}{0.428298in}}%
\pgfpathlineto{\pgfqpoint{1.523457in}{0.430262in}}%
\pgfpathlineto{\pgfqpoint{1.527753in}{0.427055in}}%
\pgfpathlineto{\pgfqpoint{1.528827in}{0.426482in}}%
\pgfpathlineto{\pgfqpoint{1.530974in}{0.431098in}}%
\pgfpathlineto{\pgfqpoint{1.534196in}{0.431749in}}%
\pgfpathlineto{\pgfqpoint{1.535270in}{0.433118in}}%
\pgfpathlineto{\pgfqpoint{1.536343in}{0.433147in}}%
\pgfpathlineto{\pgfqpoint{1.537417in}{0.434174in}}%
\pgfpathlineto{\pgfqpoint{1.538491in}{0.439540in}}%
\pgfpathlineto{\pgfqpoint{1.543860in}{0.439019in}}%
\pgfpathlineto{\pgfqpoint{1.544934in}{0.441226in}}%
\pgfpathlineto{\pgfqpoint{1.546008in}{0.441825in}}%
\pgfpathlineto{\pgfqpoint{1.549229in}{0.441573in}}%
\pgfpathlineto{\pgfqpoint{1.550303in}{0.442768in}}%
\pgfpathlineto{\pgfqpoint{1.551377in}{0.442864in}}%
\pgfpathlineto{\pgfqpoint{1.552451in}{0.441619in}}%
\pgfpathlineto{\pgfqpoint{1.553525in}{0.442704in}}%
\pgfpathlineto{\pgfqpoint{1.557820in}{0.442449in}}%
\pgfpathlineto{\pgfqpoint{1.558894in}{0.444522in}}%
\pgfpathlineto{\pgfqpoint{1.559968in}{0.443693in}}%
\pgfpathlineto{\pgfqpoint{1.561042in}{0.445460in}}%
\pgfpathlineto{\pgfqpoint{1.564263in}{0.448655in}}%
\pgfpathlineto{\pgfqpoint{1.567485in}{0.447558in}}%
\pgfpathlineto{\pgfqpoint{1.568559in}{0.447812in}}%
\pgfpathlineto{\pgfqpoint{1.573928in}{0.448099in}}%
\pgfpathlineto{\pgfqpoint{1.575002in}{0.450036in}}%
\pgfpathlineto{\pgfqpoint{1.576075in}{0.448443in}}%
\pgfpathlineto{\pgfqpoint{1.581445in}{0.449971in}}%
\pgfpathlineto{\pgfqpoint{1.583592in}{0.449419in}}%
\pgfpathlineto{\pgfqpoint{1.586814in}{0.450620in}}%
\pgfpathlineto{\pgfqpoint{1.587888in}{0.451755in}}%
\pgfpathlineto{\pgfqpoint{1.590035in}{0.455307in}}%
\pgfpathlineto{\pgfqpoint{1.591109in}{0.454570in}}%
\pgfpathlineto{\pgfqpoint{1.594331in}{0.455039in}}%
\pgfpathlineto{\pgfqpoint{1.595405in}{0.452896in}}%
\pgfpathlineto{\pgfqpoint{1.597552in}{0.454964in}}%
\pgfpathlineto{\pgfqpoint{1.598626in}{0.455097in}}%
\pgfpathlineto{\pgfqpoint{1.602921in}{0.452073in}}%
\pgfpathlineto{\pgfqpoint{1.603995in}{0.453635in}}%
\pgfpathlineto{\pgfqpoint{1.605069in}{0.456560in}}%
\pgfpathlineto{\pgfqpoint{1.606143in}{0.457557in}}%
\pgfpathlineto{\pgfqpoint{1.609364in}{0.458383in}}%
\pgfpathlineto{\pgfqpoint{1.611512in}{0.463175in}}%
\pgfpathlineto{\pgfqpoint{1.612586in}{0.462924in}}%
\pgfpathlineto{\pgfqpoint{1.613660in}{0.464860in}}%
\pgfpathlineto{\pgfqpoint{1.616881in}{0.465039in}}%
\pgfpathlineto{\pgfqpoint{1.617955in}{0.461843in}}%
\pgfpathlineto{\pgfqpoint{1.620103in}{0.469528in}}%
\pgfpathlineto{\pgfqpoint{1.621177in}{0.467089in}}%
\pgfpathlineto{\pgfqpoint{1.624398in}{0.466488in}}%
\pgfpathlineto{\pgfqpoint{1.626546in}{0.462886in}}%
\pgfpathlineto{\pgfqpoint{1.627620in}{0.465325in}}%
\pgfpathlineto{\pgfqpoint{1.628694in}{0.462624in}}%
\pgfpathlineto{\pgfqpoint{1.631915in}{0.461021in}}%
\pgfpathlineto{\pgfqpoint{1.632989in}{0.456371in}}%
\pgfpathlineto{\pgfqpoint{1.636210in}{0.461742in}}%
\pgfpathlineto{\pgfqpoint{1.639432in}{0.461672in}}%
\pgfpathlineto{\pgfqpoint{1.641580in}{0.466149in}}%
\pgfpathlineto{\pgfqpoint{1.642653in}{0.484825in}}%
\pgfpathlineto{\pgfqpoint{1.643727in}{0.480762in}}%
\pgfpathlineto{\pgfqpoint{1.646949in}{0.480694in}}%
\pgfpathlineto{\pgfqpoint{1.648023in}{0.481857in}}%
\pgfpathlineto{\pgfqpoint{1.649096in}{0.476486in}}%
\pgfpathlineto{\pgfqpoint{1.650170in}{0.476065in}}%
\pgfpathlineto{\pgfqpoint{1.651244in}{0.474259in}}%
\pgfpathlineto{\pgfqpoint{1.654466in}{0.475936in}}%
\pgfpathlineto{\pgfqpoint{1.655539in}{0.475550in}}%
\pgfpathlineto{\pgfqpoint{1.656613in}{0.478772in}}%
\pgfpathlineto{\pgfqpoint{1.657687in}{0.479577in}}%
\pgfpathlineto{\pgfqpoint{1.658761in}{0.477953in}}%
\pgfpathlineto{\pgfqpoint{1.664130in}{0.478895in}}%
\pgfpathlineto{\pgfqpoint{1.665204in}{0.480293in}}%
\pgfpathlineto{\pgfqpoint{1.666278in}{0.478310in}}%
\pgfpathlineto{\pgfqpoint{1.669499in}{0.477991in}}%
\pgfpathlineto{\pgfqpoint{1.670573in}{0.475991in}}%
\pgfpathlineto{\pgfqpoint{1.671647in}{0.475991in}}%
\pgfpathlineto{\pgfqpoint{1.673795in}{0.476675in}}%
\pgfpathlineto{\pgfqpoint{1.677016in}{0.476675in}}%
\pgfpathlineto{\pgfqpoint{1.678090in}{0.478044in}}%
\pgfpathlineto{\pgfqpoint{1.679164in}{0.476862in}}%
\pgfpathlineto{\pgfqpoint{1.680238in}{0.477200in}}%
\pgfpathlineto{\pgfqpoint{1.681312in}{0.478834in}}%
\pgfpathlineto{\pgfqpoint{1.684533in}{0.479019in}}%
\pgfpathlineto{\pgfqpoint{1.685607in}{0.479792in}}%
\pgfpathlineto{\pgfqpoint{1.687755in}{0.482654in}}%
\pgfpathlineto{\pgfqpoint{1.688828in}{0.487437in}}%
\pgfpathlineto{\pgfqpoint{1.692050in}{0.487371in}}%
\pgfpathlineto{\pgfqpoint{1.693124in}{0.484762in}}%
\pgfpathlineto{\pgfqpoint{1.694198in}{0.489253in}}%
\pgfpathlineto{\pgfqpoint{1.695272in}{0.485225in}}%
\pgfpathlineto{\pgfqpoint{1.696345in}{0.487129in}}%
\pgfpathlineto{\pgfqpoint{1.699567in}{0.489103in}}%
\pgfpathlineto{\pgfqpoint{1.700641in}{0.488294in}}%
\pgfpathlineto{\pgfqpoint{1.701715in}{0.486464in}}%
\pgfpathlineto{\pgfqpoint{1.703862in}{0.487221in}}%
\pgfpathlineto{\pgfqpoint{1.708158in}{0.485633in}}%
\pgfpathlineto{\pgfqpoint{1.709231in}{0.483854in}}%
\pgfpathlineto{\pgfqpoint{1.711379in}{0.485982in}}%
\pgfpathlineto{\pgfqpoint{1.714601in}{0.490430in}}%
\pgfpathlineto{\pgfqpoint{1.715674in}{0.489102in}}%
\pgfpathlineto{\pgfqpoint{1.716748in}{0.491857in}}%
\pgfpathlineto{\pgfqpoint{1.717822in}{0.489069in}}%
\pgfpathlineto{\pgfqpoint{1.718896in}{0.492107in}}%
\pgfpathlineto{\pgfqpoint{1.722117in}{0.491674in}}%
\pgfpathlineto{\pgfqpoint{1.723191in}{0.492306in}}%
\pgfpathlineto{\pgfqpoint{1.724265in}{0.496501in}}%
\pgfpathlineto{\pgfqpoint{1.725339in}{0.494491in}}%
\pgfpathlineto{\pgfqpoint{1.726413in}{0.496016in}}%
\pgfpathlineto{\pgfqpoint{1.732856in}{0.493842in}}%
\pgfpathlineto{\pgfqpoint{1.737151in}{0.495846in}}%
\pgfpathlineto{\pgfqpoint{1.739299in}{0.487414in}}%
\pgfpathlineto{\pgfqpoint{1.740373in}{0.488790in}}%
\pgfpathlineto{\pgfqpoint{1.741447in}{0.483077in}}%
\pgfpathlineto{\pgfqpoint{1.744668in}{0.482752in}}%
\pgfpathlineto{\pgfqpoint{1.746816in}{0.475458in}}%
\pgfpathlineto{\pgfqpoint{1.747890in}{0.470456in}}%
\pgfpathlineto{\pgfqpoint{1.748963in}{0.473459in}}%
\pgfpathlineto{\pgfqpoint{1.753259in}{0.471421in}}%
\pgfpathlineto{\pgfqpoint{1.754333in}{0.473813in}}%
\pgfpathlineto{\pgfqpoint{1.755406in}{0.470299in}}%
\pgfpathlineto{\pgfqpoint{1.756480in}{0.471181in}}%
\pgfpathlineto{\pgfqpoint{1.760776in}{0.472043in}}%
\pgfpathlineto{\pgfqpoint{1.761850in}{0.473250in}}%
\pgfpathlineto{\pgfqpoint{1.762923in}{0.473250in}}%
\pgfpathlineto{\pgfqpoint{1.763997in}{0.475985in}}%
\pgfpathlineto{\pgfqpoint{1.770440in}{0.475461in}}%
\pgfpathlineto{\pgfqpoint{1.775809in}{0.484269in}}%
\pgfpathlineto{\pgfqpoint{1.776883in}{0.483125in}}%
\pgfpathlineto{\pgfqpoint{1.777957in}{0.483396in}}%
\pgfpathlineto{\pgfqpoint{1.779031in}{0.479905in}}%
\pgfpathlineto{\pgfqpoint{1.782252in}{0.481440in}}%
\pgfpathlineto{\pgfqpoint{1.783326in}{0.485985in}}%
\pgfpathlineto{\pgfqpoint{1.784400in}{0.485638in}}%
\pgfpathlineto{\pgfqpoint{1.786548in}{0.492792in}}%
\pgfpathlineto{\pgfqpoint{1.789769in}{0.495954in}}%
\pgfpathlineto{\pgfqpoint{1.791917in}{0.502136in}}%
\pgfpathlineto{\pgfqpoint{1.792991in}{0.502473in}}%
\pgfpathlineto{\pgfqpoint{1.794065in}{0.504435in}}%
\pgfpathlineto{\pgfqpoint{1.798360in}{0.505586in}}%
\pgfpathlineto{\pgfqpoint{1.799434in}{0.501043in}}%
\pgfpathlineto{\pgfqpoint{1.801582in}{0.500428in}}%
\pgfpathlineto{\pgfqpoint{1.804803in}{0.500804in}}%
\pgfpathlineto{\pgfqpoint{1.805877in}{0.501624in}}%
\pgfpathlineto{\pgfqpoint{1.806951in}{0.500867in}}%
\pgfpathlineto{\pgfqpoint{1.808025in}{0.501245in}}%
\pgfpathlineto{\pgfqpoint{1.812320in}{0.501555in}}%
\pgfpathlineto{\pgfqpoint{1.813394in}{0.503557in}}%
\pgfpathlineto{\pgfqpoint{1.814468in}{0.501210in}}%
\pgfpathlineto{\pgfqpoint{1.816615in}{0.501074in}}%
\pgfpathlineto{\pgfqpoint{1.820911in}{0.498434in}}%
\pgfpathlineto{\pgfqpoint{1.823058in}{0.499009in}}%
\pgfpathlineto{\pgfqpoint{1.824132in}{0.495316in}}%
\pgfpathlineto{\pgfqpoint{1.827354in}{0.495960in}}%
\pgfpathlineto{\pgfqpoint{1.828427in}{0.493893in}}%
\pgfpathlineto{\pgfqpoint{1.829501in}{0.485321in}}%
\pgfpathlineto{\pgfqpoint{1.830575in}{0.486178in}}%
\pgfpathlineto{\pgfqpoint{1.831649in}{0.485437in}}%
\pgfpathlineto{\pgfqpoint{1.834871in}{0.484326in}}%
\pgfpathlineto{\pgfqpoint{1.835944in}{0.483337in}}%
\pgfpathlineto{\pgfqpoint{1.837018in}{0.485036in}}%
\pgfpathlineto{\pgfqpoint{1.838092in}{0.488896in}}%
\pgfpathlineto{\pgfqpoint{1.839166in}{0.488030in}}%
\pgfpathlineto{\pgfqpoint{1.842387in}{0.487452in}}%
\pgfpathlineto{\pgfqpoint{1.844535in}{0.489506in}}%
\pgfpathlineto{\pgfqpoint{1.845609in}{0.492355in}}%
\pgfpathlineto{\pgfqpoint{1.846683in}{0.483290in}}%
\pgfpathlineto{\pgfqpoint{1.849904in}{0.484596in}}%
\pgfpathlineto{\pgfqpoint{1.850978in}{0.483664in}}%
\pgfpathlineto{\pgfqpoint{1.852052in}{0.483874in}}%
\pgfpathlineto{\pgfqpoint{1.853126in}{0.487813in}}%
\pgfpathlineto{\pgfqpoint{1.854200in}{0.489076in}}%
\pgfpathlineto{\pgfqpoint{1.857421in}{0.490354in}}%
\pgfpathlineto{\pgfqpoint{1.858495in}{0.489867in}}%
\pgfpathlineto{\pgfqpoint{1.859569in}{0.490988in}}%
\pgfpathlineto{\pgfqpoint{1.860643in}{0.489763in}}%
\pgfpathlineto{\pgfqpoint{1.866012in}{0.493684in}}%
\pgfpathlineto{\pgfqpoint{1.867086in}{0.496590in}}%
\pgfpathlineto{\pgfqpoint{1.868160in}{0.496400in}}%
\pgfpathlineto{\pgfqpoint{1.869233in}{0.493720in}}%
\pgfpathlineto{\pgfqpoint{1.873529in}{0.495265in}}%
\pgfpathlineto{\pgfqpoint{1.874603in}{0.494981in}}%
\pgfpathlineto{\pgfqpoint{1.875676in}{0.492737in}}%
\pgfpathlineto{\pgfqpoint{1.876750in}{0.493211in}}%
\pgfpathlineto{\pgfqpoint{1.881046in}{0.496167in}}%
\pgfpathlineto{\pgfqpoint{1.882119in}{0.490911in}}%
\pgfpathlineto{\pgfqpoint{1.883193in}{0.490880in}}%
\pgfpathlineto{\pgfqpoint{1.887489in}{0.487720in}}%
\pgfpathlineto{\pgfqpoint{1.890710in}{0.492032in}}%
\pgfpathlineto{\pgfqpoint{1.891784in}{0.494474in}}%
\pgfpathlineto{\pgfqpoint{1.895005in}{0.495708in}}%
\pgfpathlineto{\pgfqpoint{1.896079in}{0.494695in}}%
\pgfpathlineto{\pgfqpoint{1.897153in}{0.496388in}}%
\pgfpathlineto{\pgfqpoint{1.898227in}{0.495623in}}%
\pgfpathlineto{\pgfqpoint{1.899301in}{0.496037in}}%
\pgfpathlineto{\pgfqpoint{1.903596in}{0.503730in}}%
\pgfpathlineto{\pgfqpoint{1.905744in}{0.500175in}}%
\pgfpathlineto{\pgfqpoint{1.910039in}{0.500568in}}%
\pgfpathlineto{\pgfqpoint{1.911113in}{0.499452in}}%
\pgfpathlineto{\pgfqpoint{1.912187in}{0.496136in}}%
\pgfpathlineto{\pgfqpoint{1.913261in}{0.497154in}}%
\pgfpathlineto{\pgfqpoint{1.914335in}{0.492788in}}%
\pgfpathlineto{\pgfqpoint{1.919704in}{0.487947in}}%
\pgfpathlineto{\pgfqpoint{1.920778in}{0.489542in}}%
\pgfpathlineto{\pgfqpoint{1.921851in}{0.488639in}}%
\pgfpathlineto{\pgfqpoint{1.925073in}{0.483362in}}%
\pgfpathlineto{\pgfqpoint{1.926147in}{0.485231in}}%
\pgfpathlineto{\pgfqpoint{1.928294in}{0.484481in}}%
\pgfpathlineto{\pgfqpoint{1.929368in}{0.492992in}}%
\pgfpathlineto{\pgfqpoint{1.932590in}{0.495618in}}%
\pgfpathlineto{\pgfqpoint{1.933664in}{0.498132in}}%
\pgfpathlineto{\pgfqpoint{1.934738in}{0.494493in}}%
\pgfpathlineto{\pgfqpoint{1.935811in}{0.493464in}}%
\pgfpathlineto{\pgfqpoint{1.936885in}{0.496296in}}%
\pgfpathlineto{\pgfqpoint{1.940107in}{0.497489in}}%
\pgfpathlineto{\pgfqpoint{1.941181in}{0.498885in}}%
\pgfpathlineto{\pgfqpoint{1.942254in}{0.498060in}}%
\pgfpathlineto{\pgfqpoint{1.943328in}{0.494051in}}%
\pgfpathlineto{\pgfqpoint{1.944402in}{0.495849in}}%
\pgfpathlineto{\pgfqpoint{1.947624in}{0.495348in}}%
\pgfpathlineto{\pgfqpoint{1.948697in}{0.498252in}}%
\pgfpathlineto{\pgfqpoint{1.949771in}{0.499127in}}%
\pgfpathlineto{\pgfqpoint{1.951919in}{0.497146in}}%
\pgfpathlineto{\pgfqpoint{1.957288in}{0.496230in}}%
\pgfpathlineto{\pgfqpoint{1.958362in}{0.494966in}}%
\pgfpathlineto{\pgfqpoint{1.959436in}{0.487084in}}%
\pgfpathlineto{\pgfqpoint{1.963731in}{0.475027in}}%
\pgfpathlineto{\pgfqpoint{1.964805in}{0.485643in}}%
\pgfpathlineto{\pgfqpoint{1.965879in}{0.480761in}}%
\pgfpathlineto{\pgfqpoint{1.966953in}{0.480593in}}%
\pgfpathlineto{\pgfqpoint{1.970174in}{0.477433in}}%
\pgfpathlineto{\pgfqpoint{1.971248in}{0.471083in}}%
\pgfpathlineto{\pgfqpoint{1.972322in}{0.472873in}}%
\pgfpathlineto{\pgfqpoint{1.973396in}{0.470747in}}%
\pgfpathlineto{\pgfqpoint{1.974470in}{0.474029in}}%
\pgfpathlineto{\pgfqpoint{1.978765in}{0.477797in}}%
\pgfpathlineto{\pgfqpoint{1.980913in}{0.482178in}}%
\pgfpathlineto{\pgfqpoint{1.981986in}{0.480694in}}%
\pgfpathlineto{\pgfqpoint{1.985208in}{0.482786in}}%
\pgfpathlineto{\pgfqpoint{1.986282in}{0.484221in}}%
\pgfpathlineto{\pgfqpoint{1.987356in}{0.482874in}}%
\pgfpathlineto{\pgfqpoint{1.988429in}{0.482961in}}%
\pgfpathlineto{\pgfqpoint{1.989503in}{0.479694in}}%
\pgfpathlineto{\pgfqpoint{1.993799in}{0.483973in}}%
\pgfpathlineto{\pgfqpoint{1.997020in}{0.489311in}}%
\pgfpathlineto{\pgfqpoint{2.001315in}{0.503759in}}%
\pgfpathlineto{\pgfqpoint{2.002389in}{0.500156in}}%
\pgfpathlineto{\pgfqpoint{2.003463in}{0.499132in}}%
\pgfpathlineto{\pgfqpoint{2.004537in}{0.497041in}}%
\pgfpathlineto{\pgfqpoint{2.007759in}{0.492317in}}%
\pgfpathlineto{\pgfqpoint{2.008832in}{0.492763in}}%
\pgfpathlineto{\pgfqpoint{2.009906in}{0.496077in}}%
\pgfpathlineto{\pgfqpoint{2.010980in}{0.494106in}}%
\pgfpathlineto{\pgfqpoint{2.012054in}{0.493872in}}%
\pgfpathlineto{\pgfqpoint{2.015275in}{0.496738in}}%
\pgfpathlineto{\pgfqpoint{2.016349in}{0.496709in}}%
\pgfpathlineto{\pgfqpoint{2.017423in}{0.494428in}}%
\pgfpathlineto{\pgfqpoint{2.018497in}{0.497673in}}%
\pgfpathlineto{\pgfqpoint{2.019571in}{0.495802in}}%
\pgfpathlineto{\pgfqpoint{2.022792in}{0.493044in}}%
\pgfpathlineto{\pgfqpoint{2.023866in}{0.494976in}}%
\pgfpathlineto{\pgfqpoint{2.024940in}{0.492695in}}%
\pgfpathlineto{\pgfqpoint{2.026014in}{0.495346in}}%
\pgfpathlineto{\pgfqpoint{2.027088in}{0.493550in}}%
\pgfpathlineto{\pgfqpoint{2.030309in}{0.490523in}}%
\pgfpathlineto{\pgfqpoint{2.031383in}{0.492266in}}%
\pgfpathlineto{\pgfqpoint{2.032457in}{0.495894in}}%
\pgfpathlineto{\pgfqpoint{2.033531in}{0.496863in}}%
\pgfpathlineto{\pgfqpoint{2.037826in}{0.487947in}}%
\pgfpathlineto{\pgfqpoint{2.038900in}{0.495214in}}%
\pgfpathlineto{\pgfqpoint{2.039974in}{0.493984in}}%
\pgfpathlineto{\pgfqpoint{2.041048in}{0.491553in}}%
\pgfpathlineto{\pgfqpoint{2.042121in}{0.492903in}}%
\pgfpathlineto{\pgfqpoint{2.045343in}{0.491043in}}%
\pgfpathlineto{\pgfqpoint{2.046417in}{0.495227in}}%
\pgfpathlineto{\pgfqpoint{2.047491in}{0.496021in}}%
\pgfpathlineto{\pgfqpoint{2.048564in}{0.493678in}}%
\pgfpathlineto{\pgfqpoint{2.049638in}{0.493155in}}%
\pgfpathlineto{\pgfqpoint{2.053934in}{0.495469in}}%
\pgfpathlineto{\pgfqpoint{2.055007in}{0.499831in}}%
\pgfpathlineto{\pgfqpoint{2.056081in}{0.500659in}}%
\pgfpathlineto{\pgfqpoint{2.060377in}{0.500548in}}%
\pgfpathlineto{\pgfqpoint{2.062524in}{0.499074in}}%
\pgfpathlineto{\pgfqpoint{2.064672in}{0.499825in}}%
\pgfpathlineto{\pgfqpoint{2.067893in}{0.502078in}}%
\pgfpathlineto{\pgfqpoint{2.070041in}{0.506992in}}%
\pgfpathlineto{\pgfqpoint{2.071115in}{0.504873in}}%
\pgfpathlineto{\pgfqpoint{2.072189in}{0.510854in}}%
\pgfpathlineto{\pgfqpoint{2.075410in}{0.513263in}}%
\pgfpathlineto{\pgfqpoint{2.076484in}{0.511869in}}%
\pgfpathlineto{\pgfqpoint{2.077558in}{0.508459in}}%
\pgfpathlineto{\pgfqpoint{2.078632in}{0.509141in}}%
\pgfpathlineto{\pgfqpoint{2.079706in}{0.514924in}}%
\pgfpathlineto{\pgfqpoint{2.082927in}{0.517203in}}%
\pgfpathlineto{\pgfqpoint{2.085075in}{0.508286in}}%
\pgfpathlineto{\pgfqpoint{2.086149in}{0.511581in}}%
\pgfpathlineto{\pgfqpoint{2.087223in}{0.504735in}}%
\pgfpathlineto{\pgfqpoint{2.090444in}{0.507206in}}%
\pgfpathlineto{\pgfqpoint{2.091518in}{0.506610in}}%
\pgfpathlineto{\pgfqpoint{2.092592in}{0.503206in}}%
\pgfpathlineto{\pgfqpoint{2.093666in}{0.504039in}}%
\pgfpathlineto{\pgfqpoint{2.097961in}{0.504791in}}%
\pgfpathlineto{\pgfqpoint{2.099035in}{0.502881in}}%
\pgfpathlineto{\pgfqpoint{2.100109in}{0.505270in}}%
\pgfpathlineto{\pgfqpoint{2.101182in}{0.503032in}}%
\pgfpathlineto{\pgfqpoint{2.105478in}{0.497801in}}%
\pgfpathlineto{\pgfqpoint{2.106552in}{0.499399in}}%
\pgfpathlineto{\pgfqpoint{2.107626in}{0.502218in}}%
\pgfpathlineto{\pgfqpoint{2.109773in}{0.495267in}}%
\pgfpathlineto{\pgfqpoint{2.112995in}{0.498310in}}%
\pgfpathlineto{\pgfqpoint{2.114069in}{0.495894in}}%
\pgfpathlineto{\pgfqpoint{2.115142in}{0.500590in}}%
\pgfpathlineto{\pgfqpoint{2.116216in}{0.502663in}}%
\pgfpathlineto{\pgfqpoint{2.117290in}{0.508517in}}%
\pgfpathlineto{\pgfqpoint{2.121585in}{0.507166in}}%
\pgfpathlineto{\pgfqpoint{2.122659in}{0.504947in}}%
\pgfpathlineto{\pgfqpoint{2.123733in}{0.507681in}}%
\pgfpathlineto{\pgfqpoint{2.124807in}{0.504110in}}%
\pgfpathlineto{\pgfqpoint{2.129102in}{0.507424in}}%
\pgfpathlineto{\pgfqpoint{2.130176in}{0.510430in}}%
\pgfpathlineto{\pgfqpoint{2.131250in}{0.505370in}}%
\pgfpathlineto{\pgfqpoint{2.132324in}{0.521865in}}%
\pgfpathlineto{\pgfqpoint{2.135545in}{0.522226in}}%
\pgfpathlineto{\pgfqpoint{2.136619in}{0.519095in}}%
\pgfpathlineto{\pgfqpoint{2.138767in}{0.524468in}}%
\pgfpathlineto{\pgfqpoint{2.139841in}{0.517482in}}%
\pgfpathlineto{\pgfqpoint{2.143062in}{0.505190in}}%
\pgfpathlineto{\pgfqpoint{2.144136in}{0.507004in}}%
\pgfpathlineto{\pgfqpoint{2.145210in}{0.500589in}}%
\pgfpathlineto{\pgfqpoint{2.147358in}{0.512110in}}%
\pgfpathlineto{\pgfqpoint{2.151653in}{0.509881in}}%
\pgfpathlineto{\pgfqpoint{2.152727in}{0.506488in}}%
\pgfpathlineto{\pgfqpoint{2.153801in}{0.508926in}}%
\pgfpathlineto{\pgfqpoint{2.154874in}{0.509832in}}%
\pgfpathlineto{\pgfqpoint{2.158096in}{0.504978in}}%
\pgfpathlineto{\pgfqpoint{2.159170in}{0.507737in}}%
\pgfpathlineto{\pgfqpoint{2.160244in}{0.506784in}}%
\pgfpathlineto{\pgfqpoint{2.161317in}{0.511201in}}%
\pgfpathlineto{\pgfqpoint{2.162391in}{0.512885in}}%
\pgfpathlineto{\pgfqpoint{2.165613in}{0.511821in}}%
\pgfpathlineto{\pgfqpoint{2.166687in}{0.518917in}}%
\pgfpathlineto{\pgfqpoint{2.167760in}{0.519982in}}%
\pgfpathlineto{\pgfqpoint{2.168834in}{0.518745in}}%
\pgfpathlineto{\pgfqpoint{2.169908in}{0.518712in}}%
\pgfpathlineto{\pgfqpoint{2.173130in}{0.512463in}}%
\pgfpathlineto{\pgfqpoint{2.174203in}{0.508264in}}%
\pgfpathlineto{\pgfqpoint{2.175277in}{0.508459in}}%
\pgfpathlineto{\pgfqpoint{2.176351in}{0.506995in}}%
\pgfpathlineto{\pgfqpoint{2.177425in}{0.511486in}}%
\pgfpathlineto{\pgfqpoint{2.180647in}{0.511942in}}%
\pgfpathlineto{\pgfqpoint{2.181720in}{0.513282in}}%
\pgfpathlineto{\pgfqpoint{2.182794in}{0.511811in}}%
\pgfpathlineto{\pgfqpoint{2.184942in}{0.507402in}}%
\pgfpathlineto{\pgfqpoint{2.188163in}{0.507433in}}%
\pgfpathlineto{\pgfqpoint{2.189237in}{0.504981in}}%
\pgfpathlineto{\pgfqpoint{2.190311in}{0.507867in}}%
\pgfpathlineto{\pgfqpoint{2.191385in}{0.507184in}}%
\pgfpathlineto{\pgfqpoint{2.195680in}{0.507370in}}%
\pgfpathlineto{\pgfqpoint{2.196754in}{0.511294in}}%
\pgfpathlineto{\pgfqpoint{2.197828in}{0.507061in}}%
\pgfpathlineto{\pgfqpoint{2.198902in}{0.507954in}}%
\pgfpathlineto{\pgfqpoint{2.199976in}{0.511225in}}%
\pgfpathlineto{\pgfqpoint{2.203197in}{0.510535in}}%
\pgfpathlineto{\pgfqpoint{2.207492in}{0.519955in}}%
\pgfpathlineto{\pgfqpoint{2.210714in}{0.520606in}}%
\pgfpathlineto{\pgfqpoint{2.211788in}{0.522756in}}%
\pgfpathlineto{\pgfqpoint{2.213936in}{0.517327in}}%
\pgfpathlineto{\pgfqpoint{2.215009in}{0.518042in}}%
\pgfpathlineto{\pgfqpoint{2.221452in}{0.526378in}}%
\pgfpathlineto{\pgfqpoint{2.222526in}{0.521335in}}%
\pgfpathlineto{\pgfqpoint{2.225748in}{0.518445in}}%
\pgfpathlineto{\pgfqpoint{2.226822in}{0.519613in}}%
\pgfpathlineto{\pgfqpoint{2.227895in}{0.518937in}}%
\pgfpathlineto{\pgfqpoint{2.228969in}{0.522147in}}%
\pgfpathlineto{\pgfqpoint{2.230043in}{0.520796in}}%
\pgfpathlineto{\pgfqpoint{2.233265in}{0.524534in}}%
\pgfpathlineto{\pgfqpoint{2.234338in}{0.528430in}}%
\pgfpathlineto{\pgfqpoint{2.235412in}{0.528041in}}%
\pgfpathlineto{\pgfqpoint{2.236486in}{0.528950in}}%
\pgfpathlineto{\pgfqpoint{2.237560in}{0.527781in}}%
\pgfpathlineto{\pgfqpoint{2.240781in}{0.526238in}}%
\pgfpathlineto{\pgfqpoint{2.241855in}{0.523191in}}%
\pgfpathlineto{\pgfqpoint{2.242929in}{0.528515in}}%
\pgfpathlineto{\pgfqpoint{2.244003in}{0.530392in}}%
\pgfpathlineto{\pgfqpoint{2.245077in}{0.533757in}}%
\pgfpathlineto{\pgfqpoint{2.248298in}{0.536917in}}%
\pgfpathlineto{\pgfqpoint{2.250446in}{0.542359in}}%
\pgfpathlineto{\pgfqpoint{2.252594in}{0.546869in}}%
\pgfpathlineto{\pgfqpoint{2.255815in}{0.548427in}}%
\pgfpathlineto{\pgfqpoint{2.256889in}{0.555854in}}%
\pgfpathlineto{\pgfqpoint{2.257963in}{0.556940in}}%
\pgfpathlineto{\pgfqpoint{2.259037in}{0.556728in}}%
\pgfpathlineto{\pgfqpoint{2.260111in}{0.559024in}}%
\pgfpathlineto{\pgfqpoint{2.265480in}{0.562538in}}%
\pgfpathlineto{\pgfqpoint{2.266554in}{0.559985in}}%
\pgfpathlineto{\pgfqpoint{2.267627in}{0.560020in}}%
\pgfpathlineto{\pgfqpoint{2.270849in}{0.562104in}}%
\pgfpathlineto{\pgfqpoint{2.271923in}{0.561892in}}%
\pgfpathlineto{\pgfqpoint{2.272997in}{0.565490in}}%
\pgfpathlineto{\pgfqpoint{2.274070in}{0.566830in}}%
\pgfpathlineto{\pgfqpoint{2.275144in}{0.563091in}}%
\pgfpathlineto{\pgfqpoint{2.278366in}{0.556860in}}%
\pgfpathlineto{\pgfqpoint{2.279440in}{0.557466in}}%
\pgfpathlineto{\pgfqpoint{2.281587in}{0.559502in}}%
\pgfpathlineto{\pgfqpoint{2.282661in}{0.564285in}}%
\pgfpathlineto{\pgfqpoint{2.286957in}{0.565589in}}%
\pgfpathlineto{\pgfqpoint{2.288030in}{0.562683in}}%
\pgfpathlineto{\pgfqpoint{2.289104in}{0.568867in}}%
\pgfpathlineto{\pgfqpoint{2.290178in}{0.580451in}}%
\pgfpathlineto{\pgfqpoint{2.293400in}{0.573703in}}%
\pgfpathlineto{\pgfqpoint{2.294473in}{0.580896in}}%
\pgfpathlineto{\pgfqpoint{2.295547in}{0.574673in}}%
\pgfpathlineto{\pgfqpoint{2.296621in}{0.584399in}}%
\pgfpathlineto{\pgfqpoint{2.297695in}{0.585642in}}%
\pgfpathlineto{\pgfqpoint{2.301990in}{0.584855in}}%
\pgfpathlineto{\pgfqpoint{2.304138in}{0.589171in}}%
\pgfpathlineto{\pgfqpoint{2.305212in}{0.581336in}}%
\pgfpathlineto{\pgfqpoint{2.308433in}{0.580938in}}%
\pgfpathlineto{\pgfqpoint{2.309507in}{0.577247in}}%
\pgfpathlineto{\pgfqpoint{2.310581in}{0.577285in}}%
\pgfpathlineto{\pgfqpoint{2.311655in}{0.579842in}}%
\pgfpathlineto{\pgfqpoint{2.312729in}{0.579222in}}%
\pgfpathlineto{\pgfqpoint{2.315950in}{0.579183in}}%
\pgfpathlineto{\pgfqpoint{2.317024in}{0.580726in}}%
\pgfpathlineto{\pgfqpoint{2.318098in}{0.578335in}}%
\pgfpathlineto{\pgfqpoint{2.319172in}{0.580422in}}%
\pgfpathlineto{\pgfqpoint{2.320246in}{0.584617in}}%
\pgfpathlineto{\pgfqpoint{2.323467in}{0.588272in}}%
\pgfpathlineto{\pgfqpoint{2.324541in}{0.586537in}}%
\pgfpathlineto{\pgfqpoint{2.325615in}{0.586695in}}%
\pgfpathlineto{\pgfqpoint{2.326689in}{0.589258in}}%
\pgfpathlineto{\pgfqpoint{2.327762in}{0.593674in}}%
\pgfpathlineto{\pgfqpoint{2.332058in}{0.594445in}}%
\pgfpathlineto{\pgfqpoint{2.333132in}{0.596149in}}%
\pgfpathlineto{\pgfqpoint{2.335279in}{0.590521in}}%
\pgfpathlineto{\pgfqpoint{2.339575in}{0.591697in}}%
\pgfpathlineto{\pgfqpoint{2.342796in}{0.595255in}}%
\pgfpathlineto{\pgfqpoint{2.347091in}{0.599114in}}%
\pgfpathlineto{\pgfqpoint{2.348165in}{0.599792in}}%
\pgfpathlineto{\pgfqpoint{2.349239in}{0.602026in}}%
\pgfpathlineto{\pgfqpoint{2.350313in}{0.602835in}}%
\pgfpathlineto{\pgfqpoint{2.353535in}{0.600529in}}%
\pgfpathlineto{\pgfqpoint{2.354608in}{0.601526in}}%
\pgfpathlineto{\pgfqpoint{2.355682in}{0.599520in}}%
\pgfpathlineto{\pgfqpoint{2.356756in}{0.599640in}}%
\pgfpathlineto{\pgfqpoint{2.357830in}{0.600603in}}%
\pgfpathlineto{\pgfqpoint{2.361051in}{0.599399in}}%
\pgfpathlineto{\pgfqpoint{2.362125in}{0.598244in}}%
\pgfpathlineto{\pgfqpoint{2.364273in}{0.600824in}}%
\pgfpathlineto{\pgfqpoint{2.365347in}{0.598118in}}%
\pgfpathlineto{\pgfqpoint{2.369642in}{0.596084in}}%
\pgfpathlineto{\pgfqpoint{2.370716in}{0.593381in}}%
\pgfpathlineto{\pgfqpoint{2.371790in}{0.594406in}}%
\pgfpathlineto{\pgfqpoint{2.372864in}{0.588293in}}%
\pgfpathlineto{\pgfqpoint{2.376085in}{0.594406in}}%
\pgfpathlineto{\pgfqpoint{2.377159in}{0.598417in}}%
\pgfpathlineto{\pgfqpoint{2.378233in}{0.596849in}}%
\pgfpathlineto{\pgfqpoint{2.379307in}{0.598848in}}%
\pgfpathlineto{\pgfqpoint{2.380380in}{0.598613in}}%
\pgfpathlineto{\pgfqpoint{2.384676in}{0.600414in}}%
\pgfpathlineto{\pgfqpoint{2.385750in}{0.603274in}}%
\pgfpathlineto{\pgfqpoint{2.386824in}{0.602725in}}%
\pgfpathlineto{\pgfqpoint{2.387897in}{0.605888in}}%
\pgfpathlineto{\pgfqpoint{2.391119in}{0.603060in}}%
\pgfpathlineto{\pgfqpoint{2.392193in}{0.604414in}}%
\pgfpathlineto{\pgfqpoint{2.393267in}{0.601268in}}%
\pgfpathlineto{\pgfqpoint{2.395414in}{0.608294in}}%
\pgfpathlineto{\pgfqpoint{2.398636in}{0.607654in}}%
\pgfpathlineto{\pgfqpoint{2.399710in}{0.608172in}}%
\pgfpathlineto{\pgfqpoint{2.400783in}{0.610613in}}%
\pgfpathlineto{\pgfqpoint{2.401857in}{0.610573in}}%
\pgfpathlineto{\pgfqpoint{2.402931in}{0.608693in}}%
\pgfpathlineto{\pgfqpoint{2.406153in}{0.609613in}}%
\pgfpathlineto{\pgfqpoint{2.407226in}{0.613853in}}%
\pgfpathlineto{\pgfqpoint{2.409374in}{0.616235in}}%
\pgfpathlineto{\pgfqpoint{2.410448in}{0.618596in}}%
\pgfpathlineto{\pgfqpoint{2.413669in}{0.619839in}}%
\pgfpathlineto{\pgfqpoint{2.414743in}{0.617502in}}%
\pgfpathlineto{\pgfqpoint{2.415817in}{0.622553in}}%
\pgfpathlineto{\pgfqpoint{2.416891in}{0.623847in}}%
\pgfpathlineto{\pgfqpoint{2.417965in}{0.623258in}}%
\pgfpathlineto{\pgfqpoint{2.421186in}{0.626623in}}%
\pgfpathlineto{\pgfqpoint{2.422260in}{0.631335in}}%
\pgfpathlineto{\pgfqpoint{2.423334in}{0.630167in}}%
\pgfpathlineto{\pgfqpoint{2.424408in}{0.630902in}}%
\pgfpathlineto{\pgfqpoint{2.425482in}{0.629648in}}%
\pgfpathlineto{\pgfqpoint{2.428703in}{0.628446in}}%
\pgfpathlineto{\pgfqpoint{2.429777in}{0.632155in}}%
\pgfpathlineto{\pgfqpoint{2.430851in}{0.627712in}}%
\pgfpathlineto{\pgfqpoint{2.431925in}{0.628017in}}%
\pgfpathlineto{\pgfqpoint{2.432999in}{0.626798in}}%
\pgfpathlineto{\pgfqpoint{2.436220in}{0.635945in}}%
\pgfpathlineto{\pgfqpoint{2.438368in}{0.632822in}}%
\pgfpathlineto{\pgfqpoint{2.439442in}{0.638556in}}%
\pgfpathlineto{\pgfqpoint{2.440515in}{0.638601in}}%
\pgfpathlineto{\pgfqpoint{2.443737in}{0.623433in}}%
\pgfpathlineto{\pgfqpoint{2.444811in}{0.624229in}}%
\pgfpathlineto{\pgfqpoint{2.446958in}{0.612824in}}%
\pgfpathlineto{\pgfqpoint{2.451254in}{0.617350in}}%
\pgfpathlineto{\pgfqpoint{2.452328in}{0.624544in}}%
\pgfpathlineto{\pgfqpoint{2.453402in}{0.623024in}}%
\pgfpathlineto{\pgfqpoint{2.455549in}{0.625412in}}%
\pgfpathlineto{\pgfqpoint{2.458771in}{0.629883in}}%
\pgfpathlineto{\pgfqpoint{2.459845in}{0.630195in}}%
\pgfpathlineto{\pgfqpoint{2.461992in}{0.613884in}}%
\pgfpathlineto{\pgfqpoint{2.463066in}{0.615177in}}%
\pgfpathlineto{\pgfqpoint{2.466288in}{0.608224in}}%
\pgfpathlineto{\pgfqpoint{2.467361in}{0.609079in}}%
\pgfpathlineto{\pgfqpoint{2.468435in}{0.618358in}}%
\pgfpathlineto{\pgfqpoint{2.469509in}{0.618487in}}%
\pgfpathlineto{\pgfqpoint{2.470583in}{0.617928in}}%
\pgfpathlineto{\pgfqpoint{2.473804in}{0.615220in}}%
\pgfpathlineto{\pgfqpoint{2.474878in}{0.618143in}}%
\pgfpathlineto{\pgfqpoint{2.475952in}{0.618401in}}%
\pgfpathlineto{\pgfqpoint{2.477026in}{0.619951in}}%
\pgfpathlineto{\pgfqpoint{2.478100in}{0.624817in}}%
\pgfpathlineto{\pgfqpoint{2.481321in}{0.623709in}}%
\pgfpathlineto{\pgfqpoint{2.483469in}{0.625837in}}%
\pgfpathlineto{\pgfqpoint{2.484543in}{0.624811in}}%
\pgfpathlineto{\pgfqpoint{2.485617in}{0.625881in}}%
\pgfpathlineto{\pgfqpoint{2.490986in}{0.625211in}}%
\pgfpathlineto{\pgfqpoint{2.492060in}{0.625345in}}%
\pgfpathlineto{\pgfqpoint{2.493134in}{0.624012in}}%
\pgfpathlineto{\pgfqpoint{2.497429in}{0.630451in}}%
\pgfpathlineto{\pgfqpoint{2.499577in}{0.623590in}}%
\pgfpathlineto{\pgfqpoint{2.500650in}{0.618896in}}%
\pgfpathlineto{\pgfqpoint{2.503872in}{0.620764in}}%
\pgfpathlineto{\pgfqpoint{2.504946in}{0.618959in}}%
\pgfpathlineto{\pgfqpoint{2.507093in}{0.622737in}}%
\pgfpathlineto{\pgfqpoint{2.508167in}{0.621889in}}%
\pgfpathlineto{\pgfqpoint{2.512463in}{0.622271in}}%
\pgfpathlineto{\pgfqpoint{2.514610in}{0.620324in}}%
\pgfpathlineto{\pgfqpoint{2.518906in}{0.618608in}}%
\pgfpathlineto{\pgfqpoint{2.521053in}{0.611541in}}%
\pgfpathlineto{\pgfqpoint{2.523201in}{0.616228in}}%
\pgfpathlineto{\pgfqpoint{2.526423in}{0.616511in}}%
\pgfpathlineto{\pgfqpoint{2.527496in}{0.612584in}}%
\pgfpathlineto{\pgfqpoint{2.529644in}{0.610965in}}%
\pgfpathlineto{\pgfqpoint{2.530718in}{0.625984in}}%
\pgfpathlineto{\pgfqpoint{2.535013in}{0.626792in}}%
\pgfpathlineto{\pgfqpoint{2.536087in}{0.624064in}}%
\pgfpathlineto{\pgfqpoint{2.537161in}{0.626018in}}%
\pgfpathlineto{\pgfqpoint{2.538235in}{0.624756in}}%
\pgfpathlineto{\pgfqpoint{2.541456in}{0.622613in}}%
\pgfpathlineto{\pgfqpoint{2.542530in}{0.620977in}}%
\pgfpathlineto{\pgfqpoint{2.543604in}{0.617656in}}%
\pgfpathlineto{\pgfqpoint{2.552195in}{0.622052in}}%
\pgfpathlineto{\pgfqpoint{2.553268in}{0.621116in}}%
\pgfpathlineto{\pgfqpoint{2.556490in}{0.622745in}}%
\pgfpathlineto{\pgfqpoint{2.557564in}{0.622510in}}%
\pgfpathlineto{\pgfqpoint{2.558638in}{0.626542in}}%
\pgfpathlineto{\pgfqpoint{2.560785in}{0.629332in}}%
\pgfpathlineto{\pgfqpoint{2.566155in}{0.628699in}}%
\pgfpathlineto{\pgfqpoint{2.567228in}{0.629290in}}%
\pgfpathlineto{\pgfqpoint{2.568302in}{0.626886in}}%
\pgfpathlineto{\pgfqpoint{2.571524in}{0.625448in}}%
\pgfpathlineto{\pgfqpoint{2.573671in}{0.629011in}}%
\pgfpathlineto{\pgfqpoint{2.575819in}{0.627804in}}%
\pgfpathlineto{\pgfqpoint{2.579041in}{0.631833in}}%
\pgfpathlineto{\pgfqpoint{2.580114in}{0.627470in}}%
\pgfpathlineto{\pgfqpoint{2.581188in}{0.629255in}}%
\pgfpathlineto{\pgfqpoint{2.583336in}{0.626647in}}%
\pgfpathlineto{\pgfqpoint{2.588705in}{0.628017in}}%
\pgfpathlineto{\pgfqpoint{2.590853in}{0.626801in}}%
\pgfpathlineto{\pgfqpoint{2.594074in}{0.628723in}}%
\pgfpathlineto{\pgfqpoint{2.596222in}{0.631913in}}%
\pgfpathlineto{\pgfqpoint{2.597296in}{0.631675in}}%
\pgfpathlineto{\pgfqpoint{2.598370in}{0.630325in}}%
\pgfpathlineto{\pgfqpoint{2.602665in}{0.630801in}}%
\pgfpathlineto{\pgfqpoint{2.603739in}{0.630166in}}%
\pgfpathlineto{\pgfqpoint{2.604813in}{0.630881in}}%
\pgfpathlineto{\pgfqpoint{2.609108in}{0.627189in}}%
\pgfpathlineto{\pgfqpoint{2.611256in}{0.627578in}}%
\pgfpathlineto{\pgfqpoint{2.612330in}{0.632950in}}%
\pgfpathlineto{\pgfqpoint{2.613403in}{0.632950in}}%
\pgfpathlineto{\pgfqpoint{2.616625in}{0.635637in}}%
\pgfpathlineto{\pgfqpoint{2.617699in}{0.634663in}}%
\pgfpathlineto{\pgfqpoint{2.618773in}{0.635748in}}%
\pgfpathlineto{\pgfqpoint{2.620920in}{0.633449in}}%
\pgfpathlineto{\pgfqpoint{2.624142in}{0.633605in}}%
\pgfpathlineto{\pgfqpoint{2.625216in}{0.638475in}}%
\pgfpathlineto{\pgfqpoint{2.627363in}{0.639489in}}%
\pgfpathlineto{\pgfqpoint{2.628437in}{0.641519in}}%
\pgfpathlineto{\pgfqpoint{2.632733in}{0.641163in}}%
\pgfpathlineto{\pgfqpoint{2.634880in}{0.642743in}}%
\pgfpathlineto{\pgfqpoint{2.635954in}{0.645199in}}%
\pgfpathlineto{\pgfqpoint{2.639176in}{0.643258in}}%
\pgfpathlineto{\pgfqpoint{2.640249in}{0.643571in}}%
\pgfpathlineto{\pgfqpoint{2.641323in}{0.638860in}}%
\pgfpathlineto{\pgfqpoint{2.642397in}{0.638664in}}%
\pgfpathlineto{\pgfqpoint{2.643471in}{0.641648in}}%
\pgfpathlineto{\pgfqpoint{2.646692in}{0.638468in}}%
\pgfpathlineto{\pgfqpoint{2.647766in}{0.636346in}}%
\pgfpathlineto{\pgfqpoint{2.648840in}{0.632802in}}%
\pgfpathlineto{\pgfqpoint{2.649914in}{0.631943in}}%
\pgfpathlineto{\pgfqpoint{2.650988in}{0.633318in}}%
\pgfpathlineto{\pgfqpoint{2.655283in}{0.633506in}}%
\pgfpathlineto{\pgfqpoint{2.656357in}{0.635416in}}%
\pgfpathlineto{\pgfqpoint{2.657431in}{0.634817in}}%
\pgfpathlineto{\pgfqpoint{2.658505in}{0.632054in}}%
\pgfpathlineto{\pgfqpoint{2.661726in}{0.630584in}}%
\pgfpathlineto{\pgfqpoint{2.662800in}{0.633317in}}%
\pgfpathlineto{\pgfqpoint{2.663874in}{0.634391in}}%
\pgfpathlineto{\pgfqpoint{2.664948in}{0.634391in}}%
\pgfpathlineto{\pgfqpoint{2.666022in}{0.628838in}}%
\pgfpathlineto{\pgfqpoint{2.669243in}{0.624950in}}%
\pgfpathlineto{\pgfqpoint{2.670317in}{0.630726in}}%
\pgfpathlineto{\pgfqpoint{2.671391in}{0.629837in}}%
\pgfpathlineto{\pgfqpoint{2.672465in}{0.634036in}}%
\pgfpathlineto{\pgfqpoint{2.673538in}{0.634036in}}%
\pgfpathlineto{\pgfqpoint{2.676760in}{0.636338in}}%
\pgfpathlineto{\pgfqpoint{2.679981in}{0.640664in}}%
\pgfpathlineto{\pgfqpoint{2.681055in}{0.646830in}}%
\pgfpathlineto{\pgfqpoint{2.684277in}{0.648179in}}%
\pgfpathlineto{\pgfqpoint{2.685351in}{0.646937in}}%
\pgfpathlineto{\pgfqpoint{2.687498in}{0.658934in}}%
\pgfpathlineto{\pgfqpoint{2.688572in}{0.656394in}}%
\pgfpathlineto{\pgfqpoint{2.691794in}{0.654661in}}%
\pgfpathlineto{\pgfqpoint{2.693941in}{0.657281in}}%
\pgfpathlineto{\pgfqpoint{2.696089in}{0.662968in}}%
\pgfpathlineto{\pgfqpoint{2.700384in}{0.657018in}}%
\pgfpathlineto{\pgfqpoint{2.701458in}{0.654002in}}%
\pgfpathlineto{\pgfqpoint{2.702532in}{0.654900in}}%
\pgfpathlineto{\pgfqpoint{2.703606in}{0.658707in}}%
\pgfpathlineto{\pgfqpoint{2.706827in}{0.659099in}}%
\pgfpathlineto{\pgfqpoint{2.707901in}{0.662010in}}%
\pgfpathlineto{\pgfqpoint{2.708975in}{0.659414in}}%
\pgfpathlineto{\pgfqpoint{2.710049in}{0.659957in}}%
\pgfpathlineto{\pgfqpoint{2.711123in}{0.665676in}}%
\pgfpathlineto{\pgfqpoint{2.714344in}{0.662719in}}%
\pgfpathlineto{\pgfqpoint{2.715418in}{0.664137in}}%
\pgfpathlineto{\pgfqpoint{2.717566in}{0.672241in}}%
\pgfpathlineto{\pgfqpoint{2.718640in}{0.670618in}}%
\pgfpathlineto{\pgfqpoint{2.721861in}{0.672202in}}%
\pgfpathlineto{\pgfqpoint{2.722935in}{0.667095in}}%
\pgfpathlineto{\pgfqpoint{2.724009in}{0.665514in}}%
\pgfpathlineto{\pgfqpoint{2.725083in}{0.668115in}}%
\pgfpathlineto{\pgfqpoint{2.726156in}{0.669239in}}%
\pgfpathlineto{\pgfqpoint{2.729378in}{0.666952in}}%
\pgfpathlineto{\pgfqpoint{2.730452in}{0.667718in}}%
\pgfpathlineto{\pgfqpoint{2.731526in}{0.666526in}}%
\pgfpathlineto{\pgfqpoint{2.732600in}{0.660910in}}%
\pgfpathlineto{\pgfqpoint{2.733673in}{0.663218in}}%
\pgfpathlineto{\pgfqpoint{2.737969in}{0.653703in}}%
\pgfpathlineto{\pgfqpoint{2.739043in}{0.650631in}}%
\pgfpathlineto{\pgfqpoint{2.740116in}{0.654732in}}%
\pgfpathlineto{\pgfqpoint{2.741190in}{0.655982in}}%
\pgfpathlineto{\pgfqpoint{2.744412in}{0.653408in}}%
\pgfpathlineto{\pgfqpoint{2.745486in}{0.650360in}}%
\pgfpathlineto{\pgfqpoint{2.746559in}{0.653500in}}%
\pgfpathlineto{\pgfqpoint{2.748707in}{0.654117in}}%
\pgfpathlineto{\pgfqpoint{2.753002in}{0.652593in}}%
\pgfpathlineto{\pgfqpoint{2.754076in}{0.652449in}}%
\pgfpathlineto{\pgfqpoint{2.755150in}{0.651693in}}%
\pgfpathlineto{\pgfqpoint{2.756224in}{0.653025in}}%
\pgfpathlineto{\pgfqpoint{2.761593in}{0.656743in}}%
\pgfpathlineto{\pgfqpoint{2.762667in}{0.651726in}}%
\pgfpathlineto{\pgfqpoint{2.763741in}{0.652189in}}%
\pgfpathlineto{\pgfqpoint{2.766962in}{0.658229in}}%
\pgfpathlineto{\pgfqpoint{2.768036in}{0.658014in}}%
\pgfpathlineto{\pgfqpoint{2.769110in}{0.656694in}}%
\pgfpathlineto{\pgfqpoint{2.770184in}{0.657586in}}%
\pgfpathlineto{\pgfqpoint{2.771258in}{0.660334in}}%
\pgfpathlineto{\pgfqpoint{2.774479in}{0.658632in}}%
\pgfpathlineto{\pgfqpoint{2.776627in}{0.662254in}}%
\pgfpathlineto{\pgfqpoint{2.777701in}{0.663496in}}%
\pgfpathlineto{\pgfqpoint{2.778775in}{0.662619in}}%
\pgfpathlineto{\pgfqpoint{2.781996in}{0.671708in}}%
\pgfpathlineto{\pgfqpoint{2.783070in}{0.671403in}}%
\pgfpathlineto{\pgfqpoint{2.784144in}{0.674950in}}%
\pgfpathlineto{\pgfqpoint{2.786291in}{0.669872in}}%
\pgfpathlineto{\pgfqpoint{2.790587in}{0.668611in}}%
\pgfpathlineto{\pgfqpoint{2.791661in}{0.669608in}}%
\pgfpathlineto{\pgfqpoint{2.792734in}{0.672985in}}%
\pgfpathlineto{\pgfqpoint{2.793808in}{0.671166in}}%
\pgfpathlineto{\pgfqpoint{2.797030in}{0.670174in}}%
\pgfpathlineto{\pgfqpoint{2.798104in}{0.669076in}}%
\pgfpathlineto{\pgfqpoint{2.799178in}{0.665003in}}%
\pgfpathlineto{\pgfqpoint{2.801325in}{0.668113in}}%
\pgfpathlineto{\pgfqpoint{2.804547in}{0.669367in}}%
\pgfpathlineto{\pgfqpoint{2.805621in}{0.666662in}}%
\pgfpathlineto{\pgfqpoint{2.806694in}{0.667563in}}%
\pgfpathlineto{\pgfqpoint{2.808842in}{0.672280in}}%
\pgfpathlineto{\pgfqpoint{2.812064in}{0.672353in}}%
\pgfpathlineto{\pgfqpoint{2.813137in}{0.675537in}}%
\pgfpathlineto{\pgfqpoint{2.814211in}{0.671657in}}%
\pgfpathlineto{\pgfqpoint{2.815285in}{0.670545in}}%
\pgfpathlineto{\pgfqpoint{2.820654in}{0.672224in}}%
\pgfpathlineto{\pgfqpoint{2.821728in}{0.676088in}}%
\pgfpathlineto{\pgfqpoint{2.822802in}{0.676410in}}%
\pgfpathlineto{\pgfqpoint{2.823876in}{0.677772in}}%
\pgfpathlineto{\pgfqpoint{2.829245in}{0.673889in}}%
\pgfpathlineto{\pgfqpoint{2.830319in}{0.675014in}}%
\pgfpathlineto{\pgfqpoint{2.831393in}{0.674095in}}%
\pgfpathlineto{\pgfqpoint{2.834614in}{0.672434in}}%
\pgfpathlineto{\pgfqpoint{2.835688in}{0.674448in}}%
\pgfpathlineto{\pgfqpoint{2.837836in}{0.683991in}}%
\pgfpathlineto{\pgfqpoint{2.838910in}{0.688328in}}%
\pgfpathlineto{\pgfqpoint{2.842131in}{0.688812in}}%
\pgfpathlineto{\pgfqpoint{2.843205in}{0.683299in}}%
\pgfpathlineto{\pgfqpoint{2.846426in}{0.689689in}}%
\pgfpathlineto{\pgfqpoint{2.849648in}{0.688186in}}%
\pgfpathlineto{\pgfqpoint{2.850722in}{0.684657in}}%
\pgfpathlineto{\pgfqpoint{2.852869in}{0.707299in}}%
\pgfpathlineto{\pgfqpoint{2.853943in}{0.714298in}}%
\pgfpathlineto{\pgfqpoint{2.857165in}{0.701481in}}%
\pgfpathlineto{\pgfqpoint{2.858239in}{0.705924in}}%
\pgfpathlineto{\pgfqpoint{2.861460in}{0.690648in}}%
\pgfpathlineto{\pgfqpoint{2.864682in}{0.691489in}}%
\pgfpathlineto{\pgfqpoint{2.865756in}{0.695451in}}%
\pgfpathlineto{\pgfqpoint{2.866829in}{0.696002in}}%
\pgfpathlineto{\pgfqpoint{2.867903in}{0.694567in}}%
\pgfpathlineto{\pgfqpoint{2.868977in}{0.697841in}}%
\pgfpathlineto{\pgfqpoint{2.872199in}{0.698797in}}%
\pgfpathlineto{\pgfqpoint{2.873272in}{0.693623in}}%
\pgfpathlineto{\pgfqpoint{2.874346in}{0.693549in}}%
\pgfpathlineto{\pgfqpoint{2.875420in}{0.694621in}}%
\pgfpathlineto{\pgfqpoint{2.876494in}{0.693586in}}%
\pgfpathlineto{\pgfqpoint{2.880789in}{0.692520in}}%
\pgfpathlineto{\pgfqpoint{2.881863in}{0.688789in}}%
\pgfpathlineto{\pgfqpoint{2.882937in}{0.687603in}}%
\pgfpathlineto{\pgfqpoint{2.884011in}{0.688782in}}%
\pgfpathlineto{\pgfqpoint{2.884011in}{0.688782in}}%
\pgfusepath{stroke}%
\end{pgfscope}%
\begin{pgfscope}%
\pgfsetrectcap%
\pgfsetmiterjoin%
\pgfsetlinewidth{0.803000pt}%
\definecolor{currentstroke}{rgb}{1.000000,1.000000,1.000000}%
\pgfsetstrokecolor{currentstroke}%
\pgfsetdash{}{0pt}%
\pgfpathmoveto{\pgfqpoint{0.418102in}{0.331635in}}%
\pgfpathlineto{\pgfqpoint{0.418102in}{0.732520in}}%
\pgfusepath{stroke}%
\end{pgfscope}%
\begin{pgfscope}%
\pgfsetrectcap%
\pgfsetmiterjoin%
\pgfsetlinewidth{0.803000pt}%
\definecolor{currentstroke}{rgb}{1.000000,1.000000,1.000000}%
\pgfsetstrokecolor{currentstroke}%
\pgfsetdash{}{0pt}%
\pgfpathmoveto{\pgfqpoint{3.001435in}{0.331635in}}%
\pgfpathlineto{\pgfqpoint{3.001435in}{0.732520in}}%
\pgfusepath{stroke}%
\end{pgfscope}%
\begin{pgfscope}%
\pgfsetrectcap%
\pgfsetmiterjoin%
\pgfsetlinewidth{0.803000pt}%
\definecolor{currentstroke}{rgb}{1.000000,1.000000,1.000000}%
\pgfsetstrokecolor{currentstroke}%
\pgfsetdash{}{0pt}%
\pgfpathmoveto{\pgfqpoint{0.418102in}{0.331635in}}%
\pgfpathlineto{\pgfqpoint{3.001435in}{0.331635in}}%
\pgfusepath{stroke}%
\end{pgfscope}%
\begin{pgfscope}%
\pgfsetrectcap%
\pgfsetmiterjoin%
\pgfsetlinewidth{0.803000pt}%
\definecolor{currentstroke}{rgb}{1.000000,1.000000,1.000000}%
\pgfsetstrokecolor{currentstroke}%
\pgfsetdash{}{0pt}%
\pgfpathmoveto{\pgfqpoint{0.418102in}{0.732520in}}%
\pgfpathlineto{\pgfqpoint{3.001435in}{0.732520in}}%
\pgfusepath{stroke}%
\end{pgfscope}%
\begin{pgfscope}%
\definecolor{textcolor}{rgb}{0.150000,0.150000,0.150000}%
\pgfsetstrokecolor{textcolor}%
\pgfsetfillcolor{textcolor}%
\pgftext[x=1.709768in,y=0.815853in,,base]{\color{textcolor}\rmfamily\fontsize{12.000000}{14.400000}\selectfont V}%
\end{pgfscope}%
\begin{pgfscope}%
\pgfsetbuttcap%
\pgfsetmiterjoin%
\definecolor{currentfill}{rgb}{0.917647,0.917647,0.949020}%
\pgfsetfillcolor{currentfill}%
\pgfsetlinewidth{0.000000pt}%
\definecolor{currentstroke}{rgb}{0.000000,0.000000,0.000000}%
\pgfsetstrokecolor{currentstroke}%
\pgfsetstrokeopacity{0.000000}%
\pgfsetdash{}{0pt}%
\pgfpathmoveto{\pgfqpoint{4.034768in}{0.331635in}}%
\pgfpathlineto{\pgfqpoint{6.618102in}{0.331635in}}%
\pgfpathlineto{\pgfqpoint{6.618102in}{0.732520in}}%
\pgfpathlineto{\pgfqpoint{4.034768in}{0.732520in}}%
\pgfpathclose%
\pgfusepath{fill}%
\end{pgfscope}%
\begin{pgfscope}%
\pgfpathrectangle{\pgfqpoint{4.034768in}{0.331635in}}{\pgfqpoint{2.583333in}{0.400885in}}%
\pgfusepath{clip}%
\pgfsetroundcap%
\pgfsetroundjoin%
\pgfsetlinewidth{0.803000pt}%
\definecolor{currentstroke}{rgb}{1.000000,1.000000,1.000000}%
\pgfsetstrokecolor{currentstroke}%
\pgfsetdash{}{0pt}%
\pgfpathmoveto{\pgfqpoint{4.150045in}{0.331635in}}%
\pgfpathlineto{\pgfqpoint{4.150045in}{0.732520in}}%
\pgfusepath{stroke}%
\end{pgfscope}%
\begin{pgfscope}%
\definecolor{textcolor}{rgb}{0.150000,0.150000,0.150000}%
\pgfsetstrokecolor{textcolor}%
\pgfsetfillcolor{textcolor}%
\pgftext[x=4.150045in,y=0.234413in,,top]{\color{textcolor}\rmfamily\fontsize{10.000000}{12.000000}\selectfont 2012}%
\end{pgfscope}%
\begin{pgfscope}%
\pgfpathrectangle{\pgfqpoint{4.034768in}{0.331635in}}{\pgfqpoint{2.583333in}{0.400885in}}%
\pgfusepath{clip}%
\pgfsetroundcap%
\pgfsetroundjoin%
\pgfsetlinewidth{0.803000pt}%
\definecolor{currentstroke}{rgb}{1.000000,1.000000,1.000000}%
\pgfsetstrokecolor{currentstroke}%
\pgfsetdash{}{0pt}%
\pgfpathmoveto{\pgfqpoint{4.543070in}{0.331635in}}%
\pgfpathlineto{\pgfqpoint{4.543070in}{0.732520in}}%
\pgfusepath{stroke}%
\end{pgfscope}%
\begin{pgfscope}%
\definecolor{textcolor}{rgb}{0.150000,0.150000,0.150000}%
\pgfsetstrokecolor{textcolor}%
\pgfsetfillcolor{textcolor}%
\pgftext[x=4.543070in,y=0.234413in,,top]{\color{textcolor}\rmfamily\fontsize{10.000000}{12.000000}\selectfont 2013}%
\end{pgfscope}%
\begin{pgfscope}%
\pgfpathrectangle{\pgfqpoint{4.034768in}{0.331635in}}{\pgfqpoint{2.583333in}{0.400885in}}%
\pgfusepath{clip}%
\pgfsetroundcap%
\pgfsetroundjoin%
\pgfsetlinewidth{0.803000pt}%
\definecolor{currentstroke}{rgb}{1.000000,1.000000,1.000000}%
\pgfsetstrokecolor{currentstroke}%
\pgfsetdash{}{0pt}%
\pgfpathmoveto{\pgfqpoint{4.935021in}{0.331635in}}%
\pgfpathlineto{\pgfqpoint{4.935021in}{0.732520in}}%
\pgfusepath{stroke}%
\end{pgfscope}%
\begin{pgfscope}%
\definecolor{textcolor}{rgb}{0.150000,0.150000,0.150000}%
\pgfsetstrokecolor{textcolor}%
\pgfsetfillcolor{textcolor}%
\pgftext[x=4.935021in,y=0.234413in,,top]{\color{textcolor}\rmfamily\fontsize{10.000000}{12.000000}\selectfont 2014}%
\end{pgfscope}%
\begin{pgfscope}%
\pgfpathrectangle{\pgfqpoint{4.034768in}{0.331635in}}{\pgfqpoint{2.583333in}{0.400885in}}%
\pgfusepath{clip}%
\pgfsetroundcap%
\pgfsetroundjoin%
\pgfsetlinewidth{0.803000pt}%
\definecolor{currentstroke}{rgb}{1.000000,1.000000,1.000000}%
\pgfsetstrokecolor{currentstroke}%
\pgfsetdash{}{0pt}%
\pgfpathmoveto{\pgfqpoint{5.326972in}{0.331635in}}%
\pgfpathlineto{\pgfqpoint{5.326972in}{0.732520in}}%
\pgfusepath{stroke}%
\end{pgfscope}%
\begin{pgfscope}%
\definecolor{textcolor}{rgb}{0.150000,0.150000,0.150000}%
\pgfsetstrokecolor{textcolor}%
\pgfsetfillcolor{textcolor}%
\pgftext[x=5.326972in,y=0.234413in,,top]{\color{textcolor}\rmfamily\fontsize{10.000000}{12.000000}\selectfont 2015}%
\end{pgfscope}%
\begin{pgfscope}%
\pgfpathrectangle{\pgfqpoint{4.034768in}{0.331635in}}{\pgfqpoint{2.583333in}{0.400885in}}%
\pgfusepath{clip}%
\pgfsetroundcap%
\pgfsetroundjoin%
\pgfsetlinewidth{0.803000pt}%
\definecolor{currentstroke}{rgb}{1.000000,1.000000,1.000000}%
\pgfsetstrokecolor{currentstroke}%
\pgfsetdash{}{0pt}%
\pgfpathmoveto{\pgfqpoint{5.718923in}{0.331635in}}%
\pgfpathlineto{\pgfqpoint{5.718923in}{0.732520in}}%
\pgfusepath{stroke}%
\end{pgfscope}%
\begin{pgfscope}%
\definecolor{textcolor}{rgb}{0.150000,0.150000,0.150000}%
\pgfsetstrokecolor{textcolor}%
\pgfsetfillcolor{textcolor}%
\pgftext[x=5.718923in,y=0.234413in,,top]{\color{textcolor}\rmfamily\fontsize{10.000000}{12.000000}\selectfont 2016}%
\end{pgfscope}%
\begin{pgfscope}%
\pgfpathrectangle{\pgfqpoint{4.034768in}{0.331635in}}{\pgfqpoint{2.583333in}{0.400885in}}%
\pgfusepath{clip}%
\pgfsetroundcap%
\pgfsetroundjoin%
\pgfsetlinewidth{0.803000pt}%
\definecolor{currentstroke}{rgb}{1.000000,1.000000,1.000000}%
\pgfsetstrokecolor{currentstroke}%
\pgfsetdash{}{0pt}%
\pgfpathmoveto{\pgfqpoint{6.111948in}{0.331635in}}%
\pgfpathlineto{\pgfqpoint{6.111948in}{0.732520in}}%
\pgfusepath{stroke}%
\end{pgfscope}%
\begin{pgfscope}%
\definecolor{textcolor}{rgb}{0.150000,0.150000,0.150000}%
\pgfsetstrokecolor{textcolor}%
\pgfsetfillcolor{textcolor}%
\pgftext[x=6.111948in,y=0.234413in,,top]{\color{textcolor}\rmfamily\fontsize{10.000000}{12.000000}\selectfont 2017}%
\end{pgfscope}%
\begin{pgfscope}%
\pgfpathrectangle{\pgfqpoint{4.034768in}{0.331635in}}{\pgfqpoint{2.583333in}{0.400885in}}%
\pgfusepath{clip}%
\pgfsetroundcap%
\pgfsetroundjoin%
\pgfsetlinewidth{0.803000pt}%
\definecolor{currentstroke}{rgb}{1.000000,1.000000,1.000000}%
\pgfsetstrokecolor{currentstroke}%
\pgfsetdash{}{0pt}%
\pgfpathmoveto{\pgfqpoint{6.503899in}{0.331635in}}%
\pgfpathlineto{\pgfqpoint{6.503899in}{0.732520in}}%
\pgfusepath{stroke}%
\end{pgfscope}%
\begin{pgfscope}%
\definecolor{textcolor}{rgb}{0.150000,0.150000,0.150000}%
\pgfsetstrokecolor{textcolor}%
\pgfsetfillcolor{textcolor}%
\pgftext[x=6.503899in,y=0.234413in,,top]{\color{textcolor}\rmfamily\fontsize{10.000000}{12.000000}\selectfont 2018}%
\end{pgfscope}%
\begin{pgfscope}%
\pgfpathrectangle{\pgfqpoint{4.034768in}{0.331635in}}{\pgfqpoint{2.583333in}{0.400885in}}%
\pgfusepath{clip}%
\pgfsetroundcap%
\pgfsetroundjoin%
\pgfsetlinewidth{0.803000pt}%
\definecolor{currentstroke}{rgb}{1.000000,1.000000,1.000000}%
\pgfsetstrokecolor{currentstroke}%
\pgfsetdash{}{0pt}%
\pgfpathmoveto{\pgfqpoint{4.034768in}{0.395780in}}%
\pgfpathlineto{\pgfqpoint{6.618102in}{0.395780in}}%
\pgfusepath{stroke}%
\end{pgfscope}%
\begin{pgfscope}%
\definecolor{textcolor}{rgb}{0.150000,0.150000,0.150000}%
\pgfsetstrokecolor{textcolor}%
\pgfsetfillcolor{textcolor}%
\pgftext[x=3.849181in,y=0.343018in,left,base]{\color{textcolor}\rmfamily\fontsize{10.000000}{12.000000}\selectfont 1}%
\end{pgfscope}%
\begin{pgfscope}%
\pgfpathrectangle{\pgfqpoint{4.034768in}{0.331635in}}{\pgfqpoint{2.583333in}{0.400885in}}%
\pgfusepath{clip}%
\pgfsetroundcap%
\pgfsetroundjoin%
\pgfsetlinewidth{0.803000pt}%
\definecolor{currentstroke}{rgb}{1.000000,1.000000,1.000000}%
\pgfsetstrokecolor{currentstroke}%
\pgfsetdash{}{0pt}%
\pgfpathmoveto{\pgfqpoint{4.034768in}{0.532753in}}%
\pgfpathlineto{\pgfqpoint{6.618102in}{0.532753in}}%
\pgfusepath{stroke}%
\end{pgfscope}%
\begin{pgfscope}%
\definecolor{textcolor}{rgb}{0.150000,0.150000,0.150000}%
\pgfsetstrokecolor{textcolor}%
\pgfsetfillcolor{textcolor}%
\pgftext[x=3.849181in,y=0.479991in,left,base]{\color{textcolor}\rmfamily\fontsize{10.000000}{12.000000}\selectfont 2}%
\end{pgfscope}%
\begin{pgfscope}%
\pgfpathrectangle{\pgfqpoint{4.034768in}{0.331635in}}{\pgfqpoint{2.583333in}{0.400885in}}%
\pgfusepath{clip}%
\pgfsetroundcap%
\pgfsetroundjoin%
\pgfsetlinewidth{0.803000pt}%
\definecolor{currentstroke}{rgb}{1.000000,1.000000,1.000000}%
\pgfsetstrokecolor{currentstroke}%
\pgfsetdash{}{0pt}%
\pgfpathmoveto{\pgfqpoint{4.034768in}{0.669725in}}%
\pgfpathlineto{\pgfqpoint{6.618102in}{0.669725in}}%
\pgfusepath{stroke}%
\end{pgfscope}%
\begin{pgfscope}%
\definecolor{textcolor}{rgb}{0.150000,0.150000,0.150000}%
\pgfsetstrokecolor{textcolor}%
\pgfsetfillcolor{textcolor}%
\pgftext[x=3.849181in,y=0.616964in,left,base]{\color{textcolor}\rmfamily\fontsize{10.000000}{12.000000}\selectfont 3}%
\end{pgfscope}%
\begin{pgfscope}%
\pgfpathrectangle{\pgfqpoint{4.034768in}{0.331635in}}{\pgfqpoint{2.583333in}{0.400885in}}%
\pgfusepath{clip}%
\pgfsetroundcap%
\pgfsetroundjoin%
\pgfsetlinewidth{1.505625pt}%
\definecolor{currentstroke}{rgb}{0.000000,0.000000,0.000000}%
\pgfsetstrokecolor{currentstroke}%
\pgfsetdash{}{0pt}%
\pgfpathmoveto{\pgfqpoint{4.152193in}{0.395780in}}%
\pgfpathlineto{\pgfqpoint{4.155414in}{0.401495in}}%
\pgfpathlineto{\pgfqpoint{4.159709in}{0.400503in}}%
\pgfpathlineto{\pgfqpoint{4.160783in}{0.397169in}}%
\pgfpathlineto{\pgfqpoint{4.161857in}{0.397248in}}%
\pgfpathlineto{\pgfqpoint{4.162931in}{0.396097in}}%
\pgfpathlineto{\pgfqpoint{4.167226in}{0.396375in}}%
\pgfpathlineto{\pgfqpoint{4.169374in}{0.399789in}}%
\pgfpathlineto{\pgfqpoint{4.170448in}{0.399352in}}%
\pgfpathlineto{\pgfqpoint{4.174743in}{0.399114in}}%
\pgfpathlineto{\pgfqpoint{4.175817in}{0.400225in}}%
\pgfpathlineto{\pgfqpoint{4.177965in}{0.399114in}}%
\pgfpathlineto{\pgfqpoint{4.182260in}{0.397883in}}%
\pgfpathlineto{\pgfqpoint{4.183334in}{0.399392in}}%
\pgfpathlineto{\pgfqpoint{4.184408in}{0.397923in}}%
\pgfpathlineto{\pgfqpoint{4.185482in}{0.401813in}}%
\pgfpathlineto{\pgfqpoint{4.188703in}{0.403440in}}%
\pgfpathlineto{\pgfqpoint{4.190851in}{0.406338in}}%
\pgfpathlineto{\pgfqpoint{4.191925in}{0.407290in}}%
\pgfpathlineto{\pgfqpoint{4.192998in}{0.406973in}}%
\pgfpathlineto{\pgfqpoint{4.196220in}{0.408203in}}%
\pgfpathlineto{\pgfqpoint{4.198368in}{0.406258in}}%
\pgfpathlineto{\pgfqpoint{4.200515in}{0.408044in}}%
\pgfpathlineto{\pgfqpoint{4.204811in}{0.407409in}}%
\pgfpathlineto{\pgfqpoint{4.205885in}{0.406338in}}%
\pgfpathlineto{\pgfqpoint{4.206958in}{0.407092in}}%
\pgfpathlineto{\pgfqpoint{4.208032in}{0.406496in}}%
\pgfpathlineto{\pgfqpoint{4.211254in}{0.407687in}}%
\pgfpathlineto{\pgfqpoint{4.212328in}{0.408719in}}%
\pgfpathlineto{\pgfqpoint{4.213401in}{0.408917in}}%
\pgfpathlineto{\pgfqpoint{4.214475in}{0.410346in}}%
\pgfpathlineto{\pgfqpoint{4.215549in}{0.410227in}}%
\pgfpathlineto{\pgfqpoint{4.218771in}{0.411458in}}%
\pgfpathlineto{\pgfqpoint{4.219844in}{0.408957in}}%
\pgfpathlineto{\pgfqpoint{4.220918in}{0.408044in}}%
\pgfpathlineto{\pgfqpoint{4.223066in}{0.409830in}}%
\pgfpathlineto{\pgfqpoint{4.226287in}{0.410188in}}%
\pgfpathlineto{\pgfqpoint{4.227361in}{0.416141in}}%
\pgfpathlineto{\pgfqpoint{4.228435in}{0.414236in}}%
\pgfpathlineto{\pgfqpoint{4.229509in}{0.414196in}}%
\pgfpathlineto{\pgfqpoint{4.230583in}{0.413204in}}%
\pgfpathlineto{\pgfqpoint{4.233804in}{0.414117in}}%
\pgfpathlineto{\pgfqpoint{4.234878in}{0.413403in}}%
\pgfpathlineto{\pgfqpoint{4.237026in}{0.413561in}}%
\pgfpathlineto{\pgfqpoint{4.238100in}{0.414871in}}%
\pgfpathlineto{\pgfqpoint{4.241321in}{0.417451in}}%
\pgfpathlineto{\pgfqpoint{4.242395in}{0.416657in}}%
\pgfpathlineto{\pgfqpoint{4.244543in}{0.412569in}}%
\pgfpathlineto{\pgfqpoint{4.245617in}{0.415308in}}%
\pgfpathlineto{\pgfqpoint{4.248838in}{0.415546in}}%
\pgfpathlineto{\pgfqpoint{4.250986in}{0.412291in}}%
\pgfpathlineto{\pgfqpoint{4.252060in}{0.412807in}}%
\pgfpathlineto{\pgfqpoint{4.256355in}{0.409354in}}%
\pgfpathlineto{\pgfqpoint{4.257429in}{0.405345in}}%
\pgfpathlineto{\pgfqpoint{4.258503in}{0.406695in}}%
\pgfpathlineto{\pgfqpoint{4.259576in}{0.409473in}}%
\pgfpathlineto{\pgfqpoint{4.260650in}{0.408401in}}%
\pgfpathlineto{\pgfqpoint{4.263872in}{0.407727in}}%
\pgfpathlineto{\pgfqpoint{4.264946in}{0.411378in}}%
\pgfpathlineto{\pgfqpoint{4.266019in}{0.410704in}}%
\pgfpathlineto{\pgfqpoint{4.267093in}{0.409235in}}%
\pgfpathlineto{\pgfqpoint{4.268167in}{0.410188in}}%
\pgfpathlineto{\pgfqpoint{4.271389in}{0.408997in}}%
\pgfpathlineto{\pgfqpoint{4.272462in}{0.409592in}}%
\pgfpathlineto{\pgfqpoint{4.274610in}{0.413799in}}%
\pgfpathlineto{\pgfqpoint{4.275684in}{0.413799in}}%
\pgfpathlineto{\pgfqpoint{4.278906in}{0.412926in}}%
\pgfpathlineto{\pgfqpoint{4.279979in}{0.415347in}}%
\pgfpathlineto{\pgfqpoint{4.281053in}{0.414474in}}%
\pgfpathlineto{\pgfqpoint{4.282127in}{0.415427in}}%
\pgfpathlineto{\pgfqpoint{4.283201in}{0.412291in}}%
\pgfpathlineto{\pgfqpoint{4.287496in}{0.417173in}}%
\pgfpathlineto{\pgfqpoint{4.288570in}{0.419753in}}%
\pgfpathlineto{\pgfqpoint{4.290718in}{0.421698in}}%
\pgfpathlineto{\pgfqpoint{4.295013in}{0.419713in}}%
\pgfpathlineto{\pgfqpoint{4.296087in}{0.419951in}}%
\pgfpathlineto{\pgfqpoint{4.298235in}{0.415427in}}%
\pgfpathlineto{\pgfqpoint{4.301456in}{0.417491in}}%
\pgfpathlineto{\pgfqpoint{4.302530in}{0.417491in}}%
\pgfpathlineto{\pgfqpoint{4.303604in}{0.416856in}}%
\pgfpathlineto{\pgfqpoint{4.310047in}{0.421380in}}%
\pgfpathlineto{\pgfqpoint{4.311121in}{0.420388in}}%
\pgfpathlineto{\pgfqpoint{4.312195in}{0.422214in}}%
\pgfpathlineto{\pgfqpoint{4.313268in}{0.417530in}}%
\pgfpathlineto{\pgfqpoint{4.316490in}{0.417570in}}%
\pgfpathlineto{\pgfqpoint{4.320785in}{0.424119in}}%
\pgfpathlineto{\pgfqpoint{4.324007in}{0.422531in}}%
\pgfpathlineto{\pgfqpoint{4.325081in}{0.424595in}}%
\pgfpathlineto{\pgfqpoint{4.326154in}{0.424079in}}%
\pgfpathlineto{\pgfqpoint{4.327228in}{0.427493in}}%
\pgfpathlineto{\pgfqpoint{4.328302in}{0.427136in}}%
\pgfpathlineto{\pgfqpoint{4.331524in}{0.427175in}}%
\pgfpathlineto{\pgfqpoint{4.333671in}{0.429438in}}%
\pgfpathlineto{\pgfqpoint{4.334745in}{0.428247in}}%
\pgfpathlineto{\pgfqpoint{4.335819in}{0.428525in}}%
\pgfpathlineto{\pgfqpoint{4.339040in}{0.425746in}}%
\pgfpathlineto{\pgfqpoint{4.341188in}{0.429954in}}%
\pgfpathlineto{\pgfqpoint{4.342262in}{0.429755in}}%
\pgfpathlineto{\pgfqpoint{4.343336in}{0.432176in}}%
\pgfpathlineto{\pgfqpoint{4.346557in}{0.432970in}}%
\pgfpathlineto{\pgfqpoint{4.347631in}{0.432533in}}%
\pgfpathlineto{\pgfqpoint{4.350853in}{0.430549in}}%
\pgfpathlineto{\pgfqpoint{4.354074in}{0.430390in}}%
\pgfpathlineto{\pgfqpoint{4.355148in}{0.428128in}}%
\pgfpathlineto{\pgfqpoint{4.356222in}{0.427810in}}%
\pgfpathlineto{\pgfqpoint{4.357296in}{0.428287in}}%
\pgfpathlineto{\pgfqpoint{4.358370in}{0.431105in}}%
\pgfpathlineto{\pgfqpoint{4.361591in}{0.429914in}}%
\pgfpathlineto{\pgfqpoint{4.362665in}{0.435232in}}%
\pgfpathlineto{\pgfqpoint{4.363739in}{0.435232in}}%
\pgfpathlineto{\pgfqpoint{4.365886in}{0.432533in}}%
\pgfpathlineto{\pgfqpoint{4.369108in}{0.430350in}}%
\pgfpathlineto{\pgfqpoint{4.371256in}{0.431462in}}%
\pgfpathlineto{\pgfqpoint{4.372329in}{0.436503in}}%
\pgfpathlineto{\pgfqpoint{4.373403in}{0.437336in}}%
\pgfpathlineto{\pgfqpoint{4.376625in}{0.436860in}}%
\pgfpathlineto{\pgfqpoint{4.378773in}{0.433327in}}%
\pgfpathlineto{\pgfqpoint{4.379846in}{0.433923in}}%
\pgfpathlineto{\pgfqpoint{4.380920in}{0.436741in}}%
\pgfpathlineto{\pgfqpoint{4.384142in}{0.436304in}}%
\pgfpathlineto{\pgfqpoint{4.385216in}{0.436860in}}%
\pgfpathlineto{\pgfqpoint{4.386289in}{0.439321in}}%
\pgfpathlineto{\pgfqpoint{4.388437in}{0.436304in}}%
\pgfpathlineto{\pgfqpoint{4.391659in}{0.437058in}}%
\pgfpathlineto{\pgfqpoint{4.392732in}{0.436463in}}%
\pgfpathlineto{\pgfqpoint{4.395954in}{0.439201in}}%
\pgfpathlineto{\pgfqpoint{4.399175in}{0.439162in}}%
\pgfpathlineto{\pgfqpoint{4.400249in}{0.436264in}}%
\pgfpathlineto{\pgfqpoint{4.401323in}{0.436344in}}%
\pgfpathlineto{\pgfqpoint{4.402397in}{0.434597in}}%
\pgfpathlineto{\pgfqpoint{4.403471in}{0.435987in}}%
\pgfpathlineto{\pgfqpoint{4.407766in}{0.436225in}}%
\pgfpathlineto{\pgfqpoint{4.408840in}{0.437693in}}%
\pgfpathlineto{\pgfqpoint{4.409914in}{0.435471in}}%
\pgfpathlineto{\pgfqpoint{4.415283in}{0.436344in}}%
\pgfpathlineto{\pgfqpoint{4.417431in}{0.444203in}}%
\pgfpathlineto{\pgfqpoint{4.421726in}{0.442972in}}%
\pgfpathlineto{\pgfqpoint{4.423874in}{0.443647in}}%
\pgfpathlineto{\pgfqpoint{4.424948in}{0.446862in}}%
\pgfpathlineto{\pgfqpoint{4.426021in}{0.445949in}}%
\pgfpathlineto{\pgfqpoint{4.429243in}{0.445234in}}%
\pgfpathlineto{\pgfqpoint{4.430317in}{0.444361in}}%
\pgfpathlineto{\pgfqpoint{4.431391in}{0.447219in}}%
\pgfpathlineto{\pgfqpoint{4.433538in}{0.447338in}}%
\pgfpathlineto{\pgfqpoint{4.436760in}{0.448013in}}%
\pgfpathlineto{\pgfqpoint{4.438907in}{0.444480in}}%
\pgfpathlineto{\pgfqpoint{4.439981in}{0.446782in}}%
\pgfpathlineto{\pgfqpoint{4.441055in}{0.445711in}}%
\pgfpathlineto{\pgfqpoint{4.444277in}{0.444957in}}%
\pgfpathlineto{\pgfqpoint{4.445350in}{0.443409in}}%
\pgfpathlineto{\pgfqpoint{4.446424in}{0.446386in}}%
\pgfpathlineto{\pgfqpoint{4.447498in}{0.446941in}}%
\pgfpathlineto{\pgfqpoint{4.448572in}{0.448172in}}%
\pgfpathlineto{\pgfqpoint{4.451794in}{0.445870in}}%
\pgfpathlineto{\pgfqpoint{4.452867in}{0.442972in}}%
\pgfpathlineto{\pgfqpoint{4.453941in}{0.441900in}}%
\pgfpathlineto{\pgfqpoint{4.455015in}{0.438765in}}%
\pgfpathlineto{\pgfqpoint{4.456089in}{0.439678in}}%
\pgfpathlineto{\pgfqpoint{4.459310in}{0.440392in}}%
\pgfpathlineto{\pgfqpoint{4.460384in}{0.441980in}}%
\pgfpathlineto{\pgfqpoint{4.461458in}{0.445750in}}%
\pgfpathlineto{\pgfqpoint{4.462532in}{0.446227in}}%
\pgfpathlineto{\pgfqpoint{4.463606in}{0.444361in}}%
\pgfpathlineto{\pgfqpoint{4.466827in}{0.443964in}}%
\pgfpathlineto{\pgfqpoint{4.467901in}{0.440273in}}%
\pgfpathlineto{\pgfqpoint{4.468975in}{0.439876in}}%
\pgfpathlineto{\pgfqpoint{4.471123in}{0.437852in}}%
\pgfpathlineto{\pgfqpoint{4.476492in}{0.434399in}}%
\pgfpathlineto{\pgfqpoint{4.477566in}{0.436780in}}%
\pgfpathlineto{\pgfqpoint{4.478639in}{0.437058in}}%
\pgfpathlineto{\pgfqpoint{4.482935in}{0.439241in}}%
\pgfpathlineto{\pgfqpoint{4.484009in}{0.437852in}}%
\pgfpathlineto{\pgfqpoint{4.485083in}{0.437693in}}%
\pgfpathlineto{\pgfqpoint{4.486156in}{0.427056in}}%
\pgfpathlineto{\pgfqpoint{4.489378in}{0.428445in}}%
\pgfpathlineto{\pgfqpoint{4.490452in}{0.430271in}}%
\pgfpathlineto{\pgfqpoint{4.491526in}{0.427453in}}%
\pgfpathlineto{\pgfqpoint{4.492599in}{0.428525in}}%
\pgfpathlineto{\pgfqpoint{4.493673in}{0.428326in}}%
\pgfpathlineto{\pgfqpoint{4.496895in}{0.430073in}}%
\pgfpathlineto{\pgfqpoint{4.499042in}{0.432851in}}%
\pgfpathlineto{\pgfqpoint{4.501190in}{0.434915in}}%
\pgfpathlineto{\pgfqpoint{4.504412in}{0.434081in}}%
\pgfpathlineto{\pgfqpoint{4.505485in}{0.432533in}}%
\pgfpathlineto{\pgfqpoint{4.507633in}{0.436542in}}%
\pgfpathlineto{\pgfqpoint{4.508707in}{0.436344in}}%
\pgfpathlineto{\pgfqpoint{4.513002in}{0.435034in}}%
\pgfpathlineto{\pgfqpoint{4.516224in}{0.437534in}}%
\pgfpathlineto{\pgfqpoint{4.520519in}{0.438408in}}%
\pgfpathlineto{\pgfqpoint{4.521593in}{0.439003in}}%
\pgfpathlineto{\pgfqpoint{4.523741in}{0.435471in}}%
\pgfpathlineto{\pgfqpoint{4.526962in}{0.437693in}}%
\pgfpathlineto{\pgfqpoint{4.528036in}{0.441107in}}%
\pgfpathlineto{\pgfqpoint{4.529110in}{0.440075in}}%
\pgfpathlineto{\pgfqpoint{4.530184in}{0.443687in}}%
\pgfpathlineto{\pgfqpoint{4.531258in}{0.440313in}}%
\pgfpathlineto{\pgfqpoint{4.536627in}{0.439757in}}%
\pgfpathlineto{\pgfqpoint{4.538774in}{0.437217in}}%
\pgfpathlineto{\pgfqpoint{4.541996in}{0.439519in}}%
\pgfpathlineto{\pgfqpoint{4.544144in}{0.444282in}}%
\pgfpathlineto{\pgfqpoint{4.545217in}{0.444679in}}%
\pgfpathlineto{\pgfqpoint{4.546291in}{0.448251in}}%
\pgfpathlineto{\pgfqpoint{4.550587in}{0.443051in}}%
\pgfpathlineto{\pgfqpoint{4.552734in}{0.443171in}}%
\pgfpathlineto{\pgfqpoint{4.553808in}{0.442416in}}%
\pgfpathlineto{\pgfqpoint{4.557030in}{0.442456in}}%
\pgfpathlineto{\pgfqpoint{4.559177in}{0.445870in}}%
\pgfpathlineto{\pgfqpoint{4.560251in}{0.449045in}}%
\pgfpathlineto{\pgfqpoint{4.561325in}{0.448807in}}%
\pgfpathlineto{\pgfqpoint{4.565620in}{0.450196in}}%
\pgfpathlineto{\pgfqpoint{4.566694in}{0.454641in}}%
\pgfpathlineto{\pgfqpoint{4.567768in}{0.454641in}}%
\pgfpathlineto{\pgfqpoint{4.568842in}{0.456189in}}%
\pgfpathlineto{\pgfqpoint{4.572063in}{0.456110in}}%
\pgfpathlineto{\pgfqpoint{4.574211in}{0.454046in}}%
\pgfpathlineto{\pgfqpoint{4.575285in}{0.454403in}}%
\pgfpathlineto{\pgfqpoint{4.576359in}{0.456943in}}%
\pgfpathlineto{\pgfqpoint{4.579580in}{0.454443in}}%
\pgfpathlineto{\pgfqpoint{4.581728in}{0.456705in}}%
\pgfpathlineto{\pgfqpoint{4.582802in}{0.456110in}}%
\pgfpathlineto{\pgfqpoint{4.583876in}{0.457221in}}%
\pgfpathlineto{\pgfqpoint{4.587097in}{0.457539in}}%
\pgfpathlineto{\pgfqpoint{4.588171in}{0.458253in}}%
\pgfpathlineto{\pgfqpoint{4.590319in}{0.458015in}}%
\pgfpathlineto{\pgfqpoint{4.591393in}{0.460674in}}%
\pgfpathlineto{\pgfqpoint{4.595688in}{0.461111in}}%
\pgfpathlineto{\pgfqpoint{4.596762in}{0.456983in}}%
\pgfpathlineto{\pgfqpoint{4.597836in}{0.455435in}}%
\pgfpathlineto{\pgfqpoint{4.598909in}{0.455713in}}%
\pgfpathlineto{\pgfqpoint{4.602131in}{0.453331in}}%
\pgfpathlineto{\pgfqpoint{4.603205in}{0.454443in}}%
\pgfpathlineto{\pgfqpoint{4.604279in}{0.456546in}}%
\pgfpathlineto{\pgfqpoint{4.605352in}{0.456943in}}%
\pgfpathlineto{\pgfqpoint{4.606426in}{0.459642in}}%
\pgfpathlineto{\pgfqpoint{4.609648in}{0.461349in}}%
\pgfpathlineto{\pgfqpoint{4.610722in}{0.463810in}}%
\pgfpathlineto{\pgfqpoint{4.612869in}{0.463254in}}%
\pgfpathlineto{\pgfqpoint{4.613943in}{0.467144in}}%
\pgfpathlineto{\pgfqpoint{4.617165in}{0.468096in}}%
\pgfpathlineto{\pgfqpoint{4.618238in}{0.466112in}}%
\pgfpathlineto{\pgfqpoint{4.620386in}{0.468414in}}%
\pgfpathlineto{\pgfqpoint{4.621460in}{0.467819in}}%
\pgfpathlineto{\pgfqpoint{4.624682in}{0.465080in}}%
\pgfpathlineto{\pgfqpoint{4.625755in}{0.463214in}}%
\pgfpathlineto{\pgfqpoint{4.626829in}{0.465477in}}%
\pgfpathlineto{\pgfqpoint{4.627903in}{0.463214in}}%
\pgfpathlineto{\pgfqpoint{4.628977in}{0.464921in}}%
\pgfpathlineto{\pgfqpoint{4.632198in}{0.462857in}}%
\pgfpathlineto{\pgfqpoint{4.633272in}{0.464365in}}%
\pgfpathlineto{\pgfqpoint{4.634346in}{0.463770in}}%
\pgfpathlineto{\pgfqpoint{4.635420in}{0.465000in}}%
\pgfpathlineto{\pgfqpoint{4.639715in}{0.464604in}}%
\pgfpathlineto{\pgfqpoint{4.640789in}{0.467382in}}%
\pgfpathlineto{\pgfqpoint{4.641863in}{0.466628in}}%
\pgfpathlineto{\pgfqpoint{4.648306in}{0.473494in}}%
\pgfpathlineto{\pgfqpoint{4.650454in}{0.478614in}}%
\pgfpathlineto{\pgfqpoint{4.651527in}{0.478614in}}%
\pgfpathlineto{\pgfqpoint{4.654749in}{0.472542in}}%
\pgfpathlineto{\pgfqpoint{4.655823in}{0.479329in}}%
\pgfpathlineto{\pgfqpoint{4.656897in}{0.479051in}}%
\pgfpathlineto{\pgfqpoint{4.657971in}{0.476550in}}%
\pgfpathlineto{\pgfqpoint{4.659044in}{0.482266in}}%
\pgfpathlineto{\pgfqpoint{4.662266in}{0.483893in}}%
\pgfpathlineto{\pgfqpoint{4.663340in}{0.485997in}}%
\pgfpathlineto{\pgfqpoint{4.664414in}{0.483655in}}%
\pgfpathlineto{\pgfqpoint{4.665487in}{0.483854in}}%
\pgfpathlineto{\pgfqpoint{4.666561in}{0.483377in}}%
\pgfpathlineto{\pgfqpoint{4.669783in}{0.487505in}}%
\pgfpathlineto{\pgfqpoint{4.670857in}{0.486910in}}%
\pgfpathlineto{\pgfqpoint{4.671930in}{0.488259in}}%
\pgfpathlineto{\pgfqpoint{4.674078in}{0.494014in}}%
\pgfpathlineto{\pgfqpoint{4.677300in}{0.494967in}}%
\pgfpathlineto{\pgfqpoint{4.678373in}{0.498619in}}%
\pgfpathlineto{\pgfqpoint{4.679447in}{0.498341in}}%
\pgfpathlineto{\pgfqpoint{4.681595in}{0.502746in}}%
\pgfpathlineto{\pgfqpoint{4.685890in}{0.503699in}}%
\pgfpathlineto{\pgfqpoint{4.686964in}{0.504453in}}%
\pgfpathlineto{\pgfqpoint{4.688038in}{0.500087in}}%
\pgfpathlineto{\pgfqpoint{4.689112in}{0.500484in}}%
\pgfpathlineto{\pgfqpoint{4.693407in}{0.497745in}}%
\pgfpathlineto{\pgfqpoint{4.695555in}{0.495602in}}%
\pgfpathlineto{\pgfqpoint{4.700924in}{0.500881in}}%
\pgfpathlineto{\pgfqpoint{4.701998in}{0.499333in}}%
\pgfpathlineto{\pgfqpoint{4.704146in}{0.487783in}}%
\pgfpathlineto{\pgfqpoint{4.707367in}{0.490403in}}%
\pgfpathlineto{\pgfqpoint{4.708441in}{0.492387in}}%
\pgfpathlineto{\pgfqpoint{4.709515in}{0.487942in}}%
\pgfpathlineto{\pgfqpoint{4.710589in}{0.487981in}}%
\pgfpathlineto{\pgfqpoint{4.711662in}{0.494213in}}%
\pgfpathlineto{\pgfqpoint{4.714884in}{0.490522in}}%
\pgfpathlineto{\pgfqpoint{4.715958in}{0.490403in}}%
\pgfpathlineto{\pgfqpoint{4.717032in}{0.487465in}}%
\pgfpathlineto{\pgfqpoint{4.718105in}{0.492189in}}%
\pgfpathlineto{\pgfqpoint{4.719179in}{0.490403in}}%
\pgfpathlineto{\pgfqpoint{4.722401in}{0.492903in}}%
\pgfpathlineto{\pgfqpoint{4.723475in}{0.495800in}}%
\pgfpathlineto{\pgfqpoint{4.724549in}{0.492308in}}%
\pgfpathlineto{\pgfqpoint{4.725622in}{0.483774in}}%
\pgfpathlineto{\pgfqpoint{4.726696in}{0.486513in}}%
\pgfpathlineto{\pgfqpoint{4.729918in}{0.485441in}}%
\pgfpathlineto{\pgfqpoint{4.730992in}{0.485917in}}%
\pgfpathlineto{\pgfqpoint{4.733139in}{0.490085in}}%
\pgfpathlineto{\pgfqpoint{4.734213in}{0.488021in}}%
\pgfpathlineto{\pgfqpoint{4.737435in}{0.490879in}}%
\pgfpathlineto{\pgfqpoint{4.738508in}{0.488418in}}%
\pgfpathlineto{\pgfqpoint{4.739582in}{0.489688in}}%
\pgfpathlineto{\pgfqpoint{4.741730in}{0.490482in}}%
\pgfpathlineto{\pgfqpoint{4.746025in}{0.494530in}}%
\pgfpathlineto{\pgfqpoint{4.747099in}{0.494411in}}%
\pgfpathlineto{\pgfqpoint{4.748173in}{0.500484in}}%
\pgfpathlineto{\pgfqpoint{4.749247in}{0.501953in}}%
\pgfpathlineto{\pgfqpoint{4.752468in}{0.498142in}}%
\pgfpathlineto{\pgfqpoint{4.753542in}{0.494848in}}%
\pgfpathlineto{\pgfqpoint{4.755690in}{0.497706in}}%
\pgfpathlineto{\pgfqpoint{4.756764in}{0.495324in}}%
\pgfpathlineto{\pgfqpoint{4.759985in}{0.492586in}}%
\pgfpathlineto{\pgfqpoint{4.763207in}{0.493102in}}%
\pgfpathlineto{\pgfqpoint{4.764281in}{0.494689in}}%
\pgfpathlineto{\pgfqpoint{4.767502in}{0.493340in}}%
\pgfpathlineto{\pgfqpoint{4.768576in}{0.491792in}}%
\pgfpathlineto{\pgfqpoint{4.770724in}{0.496039in}}%
\pgfpathlineto{\pgfqpoint{4.771797in}{0.500246in}}%
\pgfpathlineto{\pgfqpoint{4.775019in}{0.498460in}}%
\pgfpathlineto{\pgfqpoint{4.776093in}{0.502191in}}%
\pgfpathlineto{\pgfqpoint{4.777167in}{0.498063in}}%
\pgfpathlineto{\pgfqpoint{4.778240in}{0.497467in}}%
\pgfpathlineto{\pgfqpoint{4.779314in}{0.493776in}}%
\pgfpathlineto{\pgfqpoint{4.782536in}{0.490879in}}%
\pgfpathlineto{\pgfqpoint{4.784683in}{0.490958in}}%
\pgfpathlineto{\pgfqpoint{4.785757in}{0.485243in}}%
\pgfpathlineto{\pgfqpoint{4.786831in}{0.484489in}}%
\pgfpathlineto{\pgfqpoint{4.791126in}{0.483417in}}%
\pgfpathlineto{\pgfqpoint{4.792200in}{0.480718in}}%
\pgfpathlineto{\pgfqpoint{4.793274in}{0.482544in}}%
\pgfpathlineto{\pgfqpoint{4.794348in}{0.482861in}}%
\pgfpathlineto{\pgfqpoint{4.797570in}{0.481512in}}%
\pgfpathlineto{\pgfqpoint{4.798643in}{0.479091in}}%
\pgfpathlineto{\pgfqpoint{4.800791in}{0.480520in}}%
\pgfpathlineto{\pgfqpoint{4.801865in}{0.479607in}}%
\pgfpathlineto{\pgfqpoint{4.806160in}{0.479964in}}%
\pgfpathlineto{\pgfqpoint{4.809382in}{0.481631in}}%
\pgfpathlineto{\pgfqpoint{4.812603in}{0.482385in}}%
\pgfpathlineto{\pgfqpoint{4.816899in}{0.500881in}}%
\pgfpathlineto{\pgfqpoint{4.822268in}{0.502389in}}%
\pgfpathlineto{\pgfqpoint{4.824415in}{0.494769in}}%
\pgfpathlineto{\pgfqpoint{4.827637in}{0.493856in}}%
\pgfpathlineto{\pgfqpoint{4.828711in}{0.492268in}}%
\pgfpathlineto{\pgfqpoint{4.829785in}{0.492744in}}%
\pgfpathlineto{\pgfqpoint{4.830859in}{0.495602in}}%
\pgfpathlineto{\pgfqpoint{4.831932in}{0.495443in}}%
\pgfpathlineto{\pgfqpoint{4.835154in}{0.492903in}}%
\pgfpathlineto{\pgfqpoint{4.836228in}{0.494133in}}%
\pgfpathlineto{\pgfqpoint{4.837302in}{0.494332in}}%
\pgfpathlineto{\pgfqpoint{4.838375in}{0.491196in}}%
\pgfpathlineto{\pgfqpoint{4.839449in}{0.495840in}}%
\pgfpathlineto{\pgfqpoint{4.842671in}{0.493260in}}%
\pgfpathlineto{\pgfqpoint{4.844818in}{0.489648in}}%
\pgfpathlineto{\pgfqpoint{4.845892in}{0.496872in}}%
\pgfpathlineto{\pgfqpoint{4.846966in}{0.499134in}}%
\pgfpathlineto{\pgfqpoint{4.850188in}{0.501397in}}%
\pgfpathlineto{\pgfqpoint{4.851261in}{0.499968in}}%
\pgfpathlineto{\pgfqpoint{4.853409in}{0.499889in}}%
\pgfpathlineto{\pgfqpoint{4.854483in}{0.502548in}}%
\pgfpathlineto{\pgfqpoint{4.857704in}{0.504215in}}%
\pgfpathlineto{\pgfqpoint{4.858778in}{0.509256in}}%
\pgfpathlineto{\pgfqpoint{4.859852in}{0.506080in}}%
\pgfpathlineto{\pgfqpoint{4.860926in}{0.509454in}}%
\pgfpathlineto{\pgfqpoint{4.862000in}{0.510208in}}%
\pgfpathlineto{\pgfqpoint{4.866295in}{0.508978in}}%
\pgfpathlineto{\pgfqpoint{4.867369in}{0.507350in}}%
\pgfpathlineto{\pgfqpoint{4.868443in}{0.507787in}}%
\pgfpathlineto{\pgfqpoint{4.869517in}{0.509295in}}%
\pgfpathlineto{\pgfqpoint{4.873812in}{0.508740in}}%
\pgfpathlineto{\pgfqpoint{4.874886in}{0.509256in}}%
\pgfpathlineto{\pgfqpoint{4.875960in}{0.502548in}}%
\pgfpathlineto{\pgfqpoint{4.877034in}{0.507747in}}%
\pgfpathlineto{\pgfqpoint{4.880255in}{0.506874in}}%
\pgfpathlineto{\pgfqpoint{4.881329in}{0.504810in}}%
\pgfpathlineto{\pgfqpoint{4.883477in}{0.512987in}}%
\pgfpathlineto{\pgfqpoint{4.884550in}{0.512907in}}%
\pgfpathlineto{\pgfqpoint{4.887772in}{0.511081in}}%
\pgfpathlineto{\pgfqpoint{4.888846in}{0.509692in}}%
\pgfpathlineto{\pgfqpoint{4.889920in}{0.510089in}}%
\pgfpathlineto{\pgfqpoint{4.890993in}{0.512669in}}%
\pgfpathlineto{\pgfqpoint{4.892067in}{0.513622in}}%
\pgfpathlineto{\pgfqpoint{4.895289in}{0.511955in}}%
\pgfpathlineto{\pgfqpoint{4.896363in}{0.517194in}}%
\pgfpathlineto{\pgfqpoint{4.897437in}{0.515686in}}%
\pgfpathlineto{\pgfqpoint{4.899584in}{0.514852in}}%
\pgfpathlineto{\pgfqpoint{4.902806in}{0.516202in}}%
\pgfpathlineto{\pgfqpoint{4.903880in}{0.512550in}}%
\pgfpathlineto{\pgfqpoint{4.904953in}{0.512788in}}%
\pgfpathlineto{\pgfqpoint{4.906027in}{0.513741in}}%
\pgfpathlineto{\pgfqpoint{4.907101in}{0.518186in}}%
\pgfpathlineto{\pgfqpoint{4.910323in}{0.516916in}}%
\pgfpathlineto{\pgfqpoint{4.911396in}{0.518583in}}%
\pgfpathlineto{\pgfqpoint{4.912470in}{0.514773in}}%
\pgfpathlineto{\pgfqpoint{4.914618in}{0.514654in}}%
\pgfpathlineto{\pgfqpoint{4.918913in}{0.518464in}}%
\pgfpathlineto{\pgfqpoint{4.921061in}{0.526958in}}%
\pgfpathlineto{\pgfqpoint{4.922135in}{0.524854in}}%
\pgfpathlineto{\pgfqpoint{4.925356in}{0.528069in}}%
\pgfpathlineto{\pgfqpoint{4.926430in}{0.530173in}}%
\pgfpathlineto{\pgfqpoint{4.928578in}{0.533030in}}%
\pgfpathlineto{\pgfqpoint{4.929652in}{0.532038in}}%
\pgfpathlineto{\pgfqpoint{4.932873in}{0.538944in}}%
\pgfpathlineto{\pgfqpoint{4.933947in}{0.539540in}}%
\pgfpathlineto{\pgfqpoint{4.937169in}{0.538468in}}%
\pgfpathlineto{\pgfqpoint{4.940390in}{0.537436in}}%
\pgfpathlineto{\pgfqpoint{4.941464in}{0.539341in}}%
\pgfpathlineto{\pgfqpoint{4.942538in}{0.535213in}}%
\pgfpathlineto{\pgfqpoint{4.943612in}{0.534023in}}%
\pgfpathlineto{\pgfqpoint{4.944685in}{0.535848in}}%
\pgfpathlineto{\pgfqpoint{4.947907in}{0.528069in}}%
\pgfpathlineto{\pgfqpoint{4.948981in}{0.532395in}}%
\pgfpathlineto{\pgfqpoint{4.952202in}{0.530649in}}%
\pgfpathlineto{\pgfqpoint{4.956498in}{0.531482in}}%
\pgfpathlineto{\pgfqpoint{4.957571in}{0.535531in}}%
\pgfpathlineto{\pgfqpoint{4.958645in}{0.533626in}}%
\pgfpathlineto{\pgfqpoint{4.959719in}{0.526045in}}%
\pgfpathlineto{\pgfqpoint{4.962941in}{0.524298in}}%
\pgfpathlineto{\pgfqpoint{4.964014in}{0.526600in}}%
\pgfpathlineto{\pgfqpoint{4.965088in}{0.520925in}}%
\pgfpathlineto{\pgfqpoint{4.966162in}{0.527871in}}%
\pgfpathlineto{\pgfqpoint{4.967236in}{0.525608in}}%
\pgfpathlineto{\pgfqpoint{4.970458in}{0.516003in}}%
\pgfpathlineto{\pgfqpoint{4.972605in}{0.522512in}}%
\pgfpathlineto{\pgfqpoint{4.973679in}{0.536483in}}%
\pgfpathlineto{\pgfqpoint{4.974753in}{0.536880in}}%
\pgfpathlineto{\pgfqpoint{4.980122in}{0.545096in}}%
\pgfpathlineto{\pgfqpoint{4.981196in}{0.545057in}}%
\pgfpathlineto{\pgfqpoint{4.982270in}{0.549939in}}%
\pgfpathlineto{\pgfqpoint{4.986565in}{0.551248in}}%
\pgfpathlineto{\pgfqpoint{4.987639in}{0.548629in}}%
\pgfpathlineto{\pgfqpoint{4.988713in}{0.549820in}}%
\pgfpathlineto{\pgfqpoint{4.989787in}{0.553273in}}%
\pgfpathlineto{\pgfqpoint{4.993008in}{0.555456in}}%
\pgfpathlineto{\pgfqpoint{4.994082in}{0.553550in}}%
\pgfpathlineto{\pgfqpoint{4.995156in}{0.553074in}}%
\pgfpathlineto{\pgfqpoint{4.997303in}{0.555773in}}%
\pgfpathlineto{\pgfqpoint{5.000525in}{0.550812in}}%
\pgfpathlineto{\pgfqpoint{5.001599in}{0.559067in}}%
\pgfpathlineto{\pgfqpoint{5.003747in}{0.565061in}}%
\pgfpathlineto{\pgfqpoint{5.004820in}{0.560893in}}%
\pgfpathlineto{\pgfqpoint{5.008042in}{0.560020in}}%
\pgfpathlineto{\pgfqpoint{5.009116in}{0.556646in}}%
\pgfpathlineto{\pgfqpoint{5.010190in}{0.557837in}}%
\pgfpathlineto{\pgfqpoint{5.011263in}{0.552519in}}%
\pgfpathlineto{\pgfqpoint{5.012337in}{0.553034in}}%
\pgfpathlineto{\pgfqpoint{5.016633in}{0.560099in}}%
\pgfpathlineto{\pgfqpoint{5.017706in}{0.554702in}}%
\pgfpathlineto{\pgfqpoint{5.018780in}{0.555773in}}%
\pgfpathlineto{\pgfqpoint{5.019854in}{0.554066in}}%
\pgfpathlineto{\pgfqpoint{5.023076in}{0.550891in}}%
\pgfpathlineto{\pgfqpoint{5.024149in}{0.551129in}}%
\pgfpathlineto{\pgfqpoint{5.025223in}{0.547716in}}%
\pgfpathlineto{\pgfqpoint{5.026297in}{0.547200in}}%
\pgfpathlineto{\pgfqpoint{5.027371in}{0.549065in}}%
\pgfpathlineto{\pgfqpoint{5.030592in}{0.553034in}}%
\pgfpathlineto{\pgfqpoint{5.031666in}{0.558552in}}%
\pgfpathlineto{\pgfqpoint{5.033814in}{0.558988in}}%
\pgfpathlineto{\pgfqpoint{5.034888in}{0.554344in}}%
\pgfpathlineto{\pgfqpoint{5.038109in}{0.549581in}}%
\pgfpathlineto{\pgfqpoint{5.039183in}{0.551209in}}%
\pgfpathlineto{\pgfqpoint{5.040257in}{0.554503in}}%
\pgfpathlineto{\pgfqpoint{5.041331in}{0.543628in}}%
\pgfpathlineto{\pgfqpoint{5.042405in}{0.541802in}}%
\pgfpathlineto{\pgfqpoint{5.046700in}{0.544183in}}%
\pgfpathlineto{\pgfqpoint{5.048848in}{0.552757in}}%
\pgfpathlineto{\pgfqpoint{5.053143in}{0.549502in}}%
\pgfpathlineto{\pgfqpoint{5.054217in}{0.550772in}}%
\pgfpathlineto{\pgfqpoint{5.056365in}{0.551288in}}%
\pgfpathlineto{\pgfqpoint{5.057438in}{0.546287in}}%
\pgfpathlineto{\pgfqpoint{5.060660in}{0.544620in}}%
\pgfpathlineto{\pgfqpoint{5.062808in}{0.550375in}}%
\pgfpathlineto{\pgfqpoint{5.063881in}{0.551169in}}%
\pgfpathlineto{\pgfqpoint{5.064955in}{0.553908in}}%
\pgfpathlineto{\pgfqpoint{5.068177in}{0.557281in}}%
\pgfpathlineto{\pgfqpoint{5.069251in}{0.556567in}}%
\pgfpathlineto{\pgfqpoint{5.070325in}{0.553828in}}%
\pgfpathlineto{\pgfqpoint{5.071398in}{0.558671in}}%
\pgfpathlineto{\pgfqpoint{5.072472in}{0.559941in}}%
\pgfpathlineto{\pgfqpoint{5.075694in}{0.561687in}}%
\pgfpathlineto{\pgfqpoint{5.076768in}{0.560417in}}%
\pgfpathlineto{\pgfqpoint{5.078915in}{0.553352in}}%
\pgfpathlineto{\pgfqpoint{5.083211in}{0.556646in}}%
\pgfpathlineto{\pgfqpoint{5.084284in}{0.556805in}}%
\pgfpathlineto{\pgfqpoint{5.085358in}{0.560814in}}%
\pgfpathlineto{\pgfqpoint{5.086432in}{0.561409in}}%
\pgfpathlineto{\pgfqpoint{5.087506in}{0.564981in}}%
\pgfpathlineto{\pgfqpoint{5.091801in}{0.566529in}}%
\pgfpathlineto{\pgfqpoint{5.092875in}{0.566013in}}%
\pgfpathlineto{\pgfqpoint{5.093949in}{0.567601in}}%
\pgfpathlineto{\pgfqpoint{5.095023in}{0.567522in}}%
\pgfpathlineto{\pgfqpoint{5.098244in}{0.568474in}}%
\pgfpathlineto{\pgfqpoint{5.099318in}{0.567045in}}%
\pgfpathlineto{\pgfqpoint{5.101466in}{0.570340in}}%
\pgfpathlineto{\pgfqpoint{5.102540in}{0.569705in}}%
\pgfpathlineto{\pgfqpoint{5.105761in}{0.572920in}}%
\pgfpathlineto{\pgfqpoint{5.107909in}{0.568633in}}%
\pgfpathlineto{\pgfqpoint{5.108983in}{0.563076in}}%
\pgfpathlineto{\pgfqpoint{5.110057in}{0.563076in}}%
\pgfpathlineto{\pgfqpoint{5.113278in}{0.564902in}}%
\pgfpathlineto{\pgfqpoint{5.114352in}{0.564386in}}%
\pgfpathlineto{\pgfqpoint{5.116500in}{0.566648in}}%
\pgfpathlineto{\pgfqpoint{5.117573in}{0.563156in}}%
\pgfpathlineto{\pgfqpoint{5.121869in}{0.562640in}}%
\pgfpathlineto{\pgfqpoint{5.122943in}{0.567125in}}%
\pgfpathlineto{\pgfqpoint{5.125090in}{0.572245in}}%
\pgfpathlineto{\pgfqpoint{5.128312in}{0.573872in}}%
\pgfpathlineto{\pgfqpoint{5.129386in}{0.576571in}}%
\pgfpathlineto{\pgfqpoint{5.130459in}{0.576452in}}%
\pgfpathlineto{\pgfqpoint{5.131533in}{0.577921in}}%
\pgfpathlineto{\pgfqpoint{5.135829in}{0.577008in}}%
\pgfpathlineto{\pgfqpoint{5.136903in}{0.574309in}}%
\pgfpathlineto{\pgfqpoint{5.137976in}{0.579310in}}%
\pgfpathlineto{\pgfqpoint{5.139050in}{0.578000in}}%
\pgfpathlineto{\pgfqpoint{5.140124in}{0.578119in}}%
\pgfpathlineto{\pgfqpoint{5.143346in}{0.577643in}}%
\pgfpathlineto{\pgfqpoint{5.146567in}{0.571253in}}%
\pgfpathlineto{\pgfqpoint{5.147641in}{0.574150in}}%
\pgfpathlineto{\pgfqpoint{5.150862in}{0.573872in}}%
\pgfpathlineto{\pgfqpoint{5.151936in}{0.575619in}}%
\pgfpathlineto{\pgfqpoint{5.153010in}{0.574983in}}%
\pgfpathlineto{\pgfqpoint{5.154084in}{0.577762in}}%
\pgfpathlineto{\pgfqpoint{5.155158in}{0.575658in}}%
\pgfpathlineto{\pgfqpoint{5.158379in}{0.579072in}}%
\pgfpathlineto{\pgfqpoint{5.159453in}{0.575579in}}%
\pgfpathlineto{\pgfqpoint{5.160527in}{0.579270in}}%
\pgfpathlineto{\pgfqpoint{5.161601in}{0.574388in}}%
\pgfpathlineto{\pgfqpoint{5.162675in}{0.572562in}}%
\pgfpathlineto{\pgfqpoint{5.165896in}{0.579389in}}%
\pgfpathlineto{\pgfqpoint{5.166970in}{0.577603in}}%
\pgfpathlineto{\pgfqpoint{5.168044in}{0.577008in}}%
\pgfpathlineto{\pgfqpoint{5.169118in}{0.573039in}}%
\pgfpathlineto{\pgfqpoint{5.170191in}{0.577960in}}%
\pgfpathlineto{\pgfqpoint{5.173413in}{0.580342in}}%
\pgfpathlineto{\pgfqpoint{5.174487in}{0.579270in}}%
\pgfpathlineto{\pgfqpoint{5.175561in}{0.580699in}}%
\pgfpathlineto{\pgfqpoint{5.177708in}{0.586891in}}%
\pgfpathlineto{\pgfqpoint{5.182004in}{0.589867in}}%
\pgfpathlineto{\pgfqpoint{5.183078in}{0.588955in}}%
\pgfpathlineto{\pgfqpoint{5.184151in}{0.590939in}}%
\pgfpathlineto{\pgfqpoint{5.185225in}{0.591336in}}%
\pgfpathlineto{\pgfqpoint{5.188447in}{0.590939in}}%
\pgfpathlineto{\pgfqpoint{5.189521in}{0.589590in}}%
\pgfpathlineto{\pgfqpoint{5.190594in}{0.590899in}}%
\pgfpathlineto{\pgfqpoint{5.191668in}{0.590383in}}%
\pgfpathlineto{\pgfqpoint{5.192742in}{0.589074in}}%
\pgfpathlineto{\pgfqpoint{5.198111in}{0.592963in}}%
\pgfpathlineto{\pgfqpoint{5.199185in}{0.590026in}}%
\pgfpathlineto{\pgfqpoint{5.200259in}{0.592963in}}%
\pgfpathlineto{\pgfqpoint{5.203480in}{0.591574in}}%
\pgfpathlineto{\pgfqpoint{5.204554in}{0.588042in}}%
\pgfpathlineto{\pgfqpoint{5.205628in}{0.587764in}}%
\pgfpathlineto{\pgfqpoint{5.206702in}{0.589431in}}%
\pgfpathlineto{\pgfqpoint{5.207776in}{0.588320in}}%
\pgfpathlineto{\pgfqpoint{5.215293in}{0.591336in}}%
\pgfpathlineto{\pgfqpoint{5.218514in}{0.586930in}}%
\pgfpathlineto{\pgfqpoint{5.219588in}{0.583319in}}%
\pgfpathlineto{\pgfqpoint{5.220662in}{0.587526in}}%
\pgfpathlineto{\pgfqpoint{5.221736in}{0.582445in}}%
\pgfpathlineto{\pgfqpoint{5.222810in}{0.584906in}}%
\pgfpathlineto{\pgfqpoint{5.226031in}{0.585224in}}%
\pgfpathlineto{\pgfqpoint{5.227105in}{0.585978in}}%
\pgfpathlineto{\pgfqpoint{5.228179in}{0.580302in}}%
\pgfpathlineto{\pgfqpoint{5.229253in}{0.577722in}}%
\pgfpathlineto{\pgfqpoint{5.230326in}{0.583834in}}%
\pgfpathlineto{\pgfqpoint{5.233548in}{0.584231in}}%
\pgfpathlineto{\pgfqpoint{5.234622in}{0.578913in}}%
\pgfpathlineto{\pgfqpoint{5.235696in}{0.582564in}}%
\pgfpathlineto{\pgfqpoint{5.236769in}{0.573753in}}%
\pgfpathlineto{\pgfqpoint{5.237843in}{0.575817in}}%
\pgfpathlineto{\pgfqpoint{5.241065in}{0.567164in}}%
\pgfpathlineto{\pgfqpoint{5.242139in}{0.567998in}}%
\pgfpathlineto{\pgfqpoint{5.243213in}{0.560417in}}%
\pgfpathlineto{\pgfqpoint{5.244286in}{0.559187in}}%
\pgfpathlineto{\pgfqpoint{5.245360in}{0.566847in}}%
\pgfpathlineto{\pgfqpoint{5.248582in}{0.573078in}}%
\pgfpathlineto{\pgfqpoint{5.249656in}{0.580500in}}%
\pgfpathlineto{\pgfqpoint{5.250729in}{0.578873in}}%
\pgfpathlineto{\pgfqpoint{5.252877in}{0.584430in}}%
\pgfpathlineto{\pgfqpoint{5.256099in}{0.583834in}}%
\pgfpathlineto{\pgfqpoint{5.257172in}{0.589272in}}%
\pgfpathlineto{\pgfqpoint{5.258246in}{0.587804in}}%
\pgfpathlineto{\pgfqpoint{5.259320in}{0.590344in}}%
\pgfpathlineto{\pgfqpoint{5.260394in}{0.594591in}}%
\pgfpathlineto{\pgfqpoint{5.263615in}{0.595821in}}%
\pgfpathlineto{\pgfqpoint{5.264689in}{0.590820in}}%
\pgfpathlineto{\pgfqpoint{5.266837in}{0.596893in}}%
\pgfpathlineto{\pgfqpoint{5.267911in}{0.589510in}}%
\pgfpathlineto{\pgfqpoint{5.271132in}{0.588716in}}%
\pgfpathlineto{\pgfqpoint{5.272206in}{0.589471in}}%
\pgfpathlineto{\pgfqpoint{5.273280in}{0.589153in}}%
\pgfpathlineto{\pgfqpoint{5.275428in}{0.592487in}}%
\pgfpathlineto{\pgfqpoint{5.279723in}{0.590542in}}%
\pgfpathlineto{\pgfqpoint{5.280797in}{0.588875in}}%
\pgfpathlineto{\pgfqpoint{5.281871in}{0.585502in}}%
\pgfpathlineto{\pgfqpoint{5.282945in}{0.585700in}}%
\pgfpathlineto{\pgfqpoint{5.286166in}{0.591654in}}%
\pgfpathlineto{\pgfqpoint{5.287240in}{0.595583in}}%
\pgfpathlineto{\pgfqpoint{5.290461in}{0.598758in}}%
\pgfpathlineto{\pgfqpoint{5.293683in}{0.599433in}}%
\pgfpathlineto{\pgfqpoint{5.294757in}{0.602291in}}%
\pgfpathlineto{\pgfqpoint{5.295831in}{0.600941in}}%
\pgfpathlineto{\pgfqpoint{5.296904in}{0.601417in}}%
\pgfpathlineto{\pgfqpoint{5.297978in}{0.603362in}}%
\pgfpathlineto{\pgfqpoint{5.301200in}{0.603481in}}%
\pgfpathlineto{\pgfqpoint{5.303347in}{0.595504in}}%
\pgfpathlineto{\pgfqpoint{5.304421in}{0.600227in}}%
\pgfpathlineto{\pgfqpoint{5.308717in}{0.597091in}}%
\pgfpathlineto{\pgfqpoint{5.309791in}{0.594353in}}%
\pgfpathlineto{\pgfqpoint{5.311938in}{0.603442in}}%
\pgfpathlineto{\pgfqpoint{5.313012in}{0.604474in}}%
\pgfpathlineto{\pgfqpoint{5.317307in}{0.611181in}}%
\pgfpathlineto{\pgfqpoint{5.318381in}{0.610189in}}%
\pgfpathlineto{\pgfqpoint{5.320529in}{0.612452in}}%
\pgfpathlineto{\pgfqpoint{5.323750in}{0.614198in}}%
\pgfpathlineto{\pgfqpoint{5.325898in}{0.609316in}}%
\pgfpathlineto{\pgfqpoint{5.328046in}{0.607689in}}%
\pgfpathlineto{\pgfqpoint{5.332341in}{0.600782in}}%
\pgfpathlineto{\pgfqpoint{5.335563in}{0.609554in}}%
\pgfpathlineto{\pgfqpoint{5.338784in}{0.610348in}}%
\pgfpathlineto{\pgfqpoint{5.339858in}{0.613047in}}%
\pgfpathlineto{\pgfqpoint{5.340932in}{0.609475in}}%
\pgfpathlineto{\pgfqpoint{5.342006in}{0.609911in}}%
\pgfpathlineto{\pgfqpoint{5.343079in}{0.613007in}}%
\pgfpathlineto{\pgfqpoint{5.347375in}{0.611380in}}%
\pgfpathlineto{\pgfqpoint{5.348449in}{0.609237in}}%
\pgfpathlineto{\pgfqpoint{5.349523in}{0.612888in}}%
\pgfpathlineto{\pgfqpoint{5.350596in}{0.611300in}}%
\pgfpathlineto{\pgfqpoint{5.353818in}{0.612213in}}%
\pgfpathlineto{\pgfqpoint{5.355966in}{0.603680in}}%
\pgfpathlineto{\pgfqpoint{5.357039in}{0.605704in}}%
\pgfpathlineto{\pgfqpoint{5.358113in}{0.597290in}}%
\pgfpathlineto{\pgfqpoint{5.361335in}{0.600902in}}%
\pgfpathlineto{\pgfqpoint{5.362409in}{0.608998in}}%
\pgfpathlineto{\pgfqpoint{5.363482in}{0.635710in}}%
\pgfpathlineto{\pgfqpoint{5.364556in}{0.640751in}}%
\pgfpathlineto{\pgfqpoint{5.365630in}{0.638449in}}%
\pgfpathlineto{\pgfqpoint{5.368852in}{0.637377in}}%
\pgfpathlineto{\pgfqpoint{5.369925in}{0.638092in}}%
\pgfpathlineto{\pgfqpoint{5.370999in}{0.637893in}}%
\pgfpathlineto{\pgfqpoint{5.372073in}{0.644283in}}%
\pgfpathlineto{\pgfqpoint{5.373147in}{0.646466in}}%
\pgfpathlineto{\pgfqpoint{5.377442in}{0.646268in}}%
\pgfpathlineto{\pgfqpoint{5.378516in}{0.645315in}}%
\pgfpathlineto{\pgfqpoint{5.379590in}{0.645435in}}%
\pgfpathlineto{\pgfqpoint{5.380664in}{0.647895in}}%
\pgfpathlineto{\pgfqpoint{5.383885in}{0.649523in}}%
\pgfpathlineto{\pgfqpoint{5.384959in}{0.648332in}}%
\pgfpathlineto{\pgfqpoint{5.386033in}{0.651666in}}%
\pgfpathlineto{\pgfqpoint{5.388181in}{0.646149in}}%
\pgfpathlineto{\pgfqpoint{5.392476in}{0.654563in}}%
\pgfpathlineto{\pgfqpoint{5.395698in}{0.645157in}}%
\pgfpathlineto{\pgfqpoint{5.398919in}{0.650515in}}%
\pgfpathlineto{\pgfqpoint{5.399993in}{0.642458in}}%
\pgfpathlineto{\pgfqpoint{5.401067in}{0.641704in}}%
\pgfpathlineto{\pgfqpoint{5.402141in}{0.657620in}}%
\pgfpathlineto{\pgfqpoint{5.403214in}{0.654921in}}%
\pgfpathlineto{\pgfqpoint{5.406436in}{0.658374in}}%
\pgfpathlineto{\pgfqpoint{5.407510in}{0.656865in}}%
\pgfpathlineto{\pgfqpoint{5.408584in}{0.660596in}}%
\pgfpathlineto{\pgfqpoint{5.409657in}{0.658374in}}%
\pgfpathlineto{\pgfqpoint{5.410731in}{0.662303in}}%
\pgfpathlineto{\pgfqpoint{5.413953in}{0.661549in}}%
\pgfpathlineto{\pgfqpoint{5.415027in}{0.657421in}}%
\pgfpathlineto{\pgfqpoint{5.416101in}{0.649562in}}%
\pgfpathlineto{\pgfqpoint{5.421470in}{0.653730in}}%
\pgfpathlineto{\pgfqpoint{5.422544in}{0.649126in}}%
\pgfpathlineto{\pgfqpoint{5.424691in}{0.653293in}}%
\pgfpathlineto{\pgfqpoint{5.430060in}{0.651150in}}%
\pgfpathlineto{\pgfqpoint{5.431134in}{0.654603in}}%
\pgfpathlineto{\pgfqpoint{5.433282in}{0.656826in}}%
\pgfpathlineto{\pgfqpoint{5.436503in}{0.655159in}}%
\pgfpathlineto{\pgfqpoint{5.437577in}{0.655714in}}%
\pgfpathlineto{\pgfqpoint{5.438651in}{0.656905in}}%
\pgfpathlineto{\pgfqpoint{5.439725in}{0.661073in}}%
\pgfpathlineto{\pgfqpoint{5.440799in}{0.655833in}}%
\pgfpathlineto{\pgfqpoint{5.444020in}{0.661549in}}%
\pgfpathlineto{\pgfqpoint{5.445094in}{0.659525in}}%
\pgfpathlineto{\pgfqpoint{5.446168in}{0.660477in}}%
\pgfpathlineto{\pgfqpoint{5.448316in}{0.666431in}}%
\pgfpathlineto{\pgfqpoint{5.451537in}{0.668773in}}%
\pgfpathlineto{\pgfqpoint{5.453685in}{0.667463in}}%
\pgfpathlineto{\pgfqpoint{5.454759in}{0.663414in}}%
\pgfpathlineto{\pgfqpoint{5.455833in}{0.670082in}}%
\pgfpathlineto{\pgfqpoint{5.459054in}{0.671988in}}%
\pgfpathlineto{\pgfqpoint{5.460128in}{0.671194in}}%
\pgfpathlineto{\pgfqpoint{5.461202in}{0.667106in}}%
\pgfpathlineto{\pgfqpoint{5.462276in}{0.665399in}}%
\pgfpathlineto{\pgfqpoint{5.463349in}{0.668574in}}%
\pgfpathlineto{\pgfqpoint{5.466571in}{0.662938in}}%
\pgfpathlineto{\pgfqpoint{5.467645in}{0.665320in}}%
\pgfpathlineto{\pgfqpoint{5.468719in}{0.665161in}}%
\pgfpathlineto{\pgfqpoint{5.470866in}{0.669289in}}%
\pgfpathlineto{\pgfqpoint{5.474088in}{0.669408in}}%
\pgfpathlineto{\pgfqpoint{5.475162in}{0.670241in}}%
\pgfpathlineto{\pgfqpoint{5.476235in}{0.668892in}}%
\pgfpathlineto{\pgfqpoint{5.477309in}{0.669606in}}%
\pgfpathlineto{\pgfqpoint{5.478383in}{0.669130in}}%
\pgfpathlineto{\pgfqpoint{5.482679in}{0.666074in}}%
\pgfpathlineto{\pgfqpoint{5.483752in}{0.669527in}}%
\pgfpathlineto{\pgfqpoint{5.484826in}{0.670122in}}%
\pgfpathlineto{\pgfqpoint{5.485900in}{0.669527in}}%
\pgfpathlineto{\pgfqpoint{5.489122in}{0.671749in}}%
\pgfpathlineto{\pgfqpoint{5.490195in}{0.670956in}}%
\pgfpathlineto{\pgfqpoint{5.491269in}{0.672504in}}%
\pgfpathlineto{\pgfqpoint{5.492343in}{0.669289in}}%
\pgfpathlineto{\pgfqpoint{5.493417in}{0.669289in}}%
\pgfpathlineto{\pgfqpoint{5.496638in}{0.665518in}}%
\pgfpathlineto{\pgfqpoint{5.497712in}{0.662660in}}%
\pgfpathlineto{\pgfqpoint{5.498786in}{0.668177in}}%
\pgfpathlineto{\pgfqpoint{5.499860in}{0.670479in}}%
\pgfpathlineto{\pgfqpoint{5.500934in}{0.667979in}}%
\pgfpathlineto{\pgfqpoint{5.504155in}{0.668812in}}%
\pgfpathlineto{\pgfqpoint{5.507377in}{0.680124in}}%
\pgfpathlineto{\pgfqpoint{5.508451in}{0.677902in}}%
\pgfpathlineto{\pgfqpoint{5.511672in}{0.681315in}}%
\pgfpathlineto{\pgfqpoint{5.512746in}{0.684570in}}%
\pgfpathlineto{\pgfqpoint{5.513820in}{0.682188in}}%
\pgfpathlineto{\pgfqpoint{5.515967in}{0.686713in}}%
\pgfpathlineto{\pgfqpoint{5.519189in}{0.679529in}}%
\pgfpathlineto{\pgfqpoint{5.521337in}{0.689729in}}%
\pgfpathlineto{\pgfqpoint{5.522411in}{0.689134in}}%
\pgfpathlineto{\pgfqpoint{5.526706in}{0.691873in}}%
\pgfpathlineto{\pgfqpoint{5.527780in}{0.697112in}}%
\pgfpathlineto{\pgfqpoint{5.528854in}{0.689967in}}%
\pgfpathlineto{\pgfqpoint{5.529927in}{0.691515in}}%
\pgfpathlineto{\pgfqpoint{5.531001in}{0.694651in}}%
\pgfpathlineto{\pgfqpoint{5.534223in}{0.700684in}}%
\pgfpathlineto{\pgfqpoint{5.535297in}{0.699930in}}%
\pgfpathlineto{\pgfqpoint{5.536370in}{0.701597in}}%
\pgfpathlineto{\pgfqpoint{5.537444in}{0.704494in}}%
\pgfpathlineto{\pgfqpoint{5.538518in}{0.703700in}}%
\pgfpathlineto{\pgfqpoint{5.541740in}{0.706399in}}%
\pgfpathlineto{\pgfqpoint{5.542813in}{0.705407in}}%
\pgfpathlineto{\pgfqpoint{5.543887in}{0.705447in}}%
\pgfpathlineto{\pgfqpoint{5.544961in}{0.703462in}}%
\pgfpathlineto{\pgfqpoint{5.546035in}{0.703899in}}%
\pgfpathlineto{\pgfqpoint{5.549256in}{0.701438in}}%
\pgfpathlineto{\pgfqpoint{5.550330in}{0.702192in}}%
\pgfpathlineto{\pgfqpoint{5.551404in}{0.707392in}}%
\pgfpathlineto{\pgfqpoint{5.552478in}{0.708066in}}%
\pgfpathlineto{\pgfqpoint{5.553552in}{0.707987in}}%
\pgfpathlineto{\pgfqpoint{5.557847in}{0.714298in}}%
\pgfpathlineto{\pgfqpoint{5.558921in}{0.672543in}}%
\pgfpathlineto{\pgfqpoint{5.559995in}{0.665121in}}%
\pgfpathlineto{\pgfqpoint{5.561069in}{0.668098in}}%
\pgfpathlineto{\pgfqpoint{5.564290in}{0.674290in}}%
\pgfpathlineto{\pgfqpoint{5.565364in}{0.663057in}}%
\pgfpathlineto{\pgfqpoint{5.566438in}{0.659287in}}%
\pgfpathlineto{\pgfqpoint{5.567512in}{0.661271in}}%
\pgfpathlineto{\pgfqpoint{5.568586in}{0.659922in}}%
\pgfpathlineto{\pgfqpoint{5.571807in}{0.666987in}}%
\pgfpathlineto{\pgfqpoint{5.572881in}{0.659088in}}%
\pgfpathlineto{\pgfqpoint{5.573955in}{0.657262in}}%
\pgfpathlineto{\pgfqpoint{5.575029in}{0.633170in}}%
\pgfpathlineto{\pgfqpoint{5.579324in}{0.615746in}}%
\pgfpathlineto{\pgfqpoint{5.580398in}{0.617730in}}%
\pgfpathlineto{\pgfqpoint{5.582545in}{0.641227in}}%
\pgfpathlineto{\pgfqpoint{5.583619in}{0.642378in}}%
\pgfpathlineto{\pgfqpoint{5.586841in}{0.640156in}}%
\pgfpathlineto{\pgfqpoint{5.587915in}{0.631265in}}%
\pgfpathlineto{\pgfqpoint{5.588989in}{0.640195in}}%
\pgfpathlineto{\pgfqpoint{5.590062in}{0.640553in}}%
\pgfpathlineto{\pgfqpoint{5.591136in}{0.636742in}}%
\pgfpathlineto{\pgfqpoint{5.595432in}{0.648133in}}%
\pgfpathlineto{\pgfqpoint{5.596505in}{0.640275in}}%
\pgfpathlineto{\pgfqpoint{5.597579in}{0.642855in}}%
\pgfpathlineto{\pgfqpoint{5.598653in}{0.649880in}}%
\pgfpathlineto{\pgfqpoint{5.602948in}{0.645950in}}%
\pgfpathlineto{\pgfqpoint{5.605096in}{0.648848in}}%
\pgfpathlineto{\pgfqpoint{5.606170in}{0.643728in}}%
\pgfpathlineto{\pgfqpoint{5.609391in}{0.645871in}}%
\pgfpathlineto{\pgfqpoint{5.612613in}{0.635432in}}%
\pgfpathlineto{\pgfqpoint{5.613687in}{0.634242in}}%
\pgfpathlineto{\pgfqpoint{5.616908in}{0.627455in}}%
\pgfpathlineto{\pgfqpoint{5.617982in}{0.630947in}}%
\pgfpathlineto{\pgfqpoint{5.619056in}{0.641346in}}%
\pgfpathlineto{\pgfqpoint{5.621204in}{0.644323in}}%
\pgfpathlineto{\pgfqpoint{5.624425in}{0.647538in}}%
\pgfpathlineto{\pgfqpoint{5.625499in}{0.647221in}}%
\pgfpathlineto{\pgfqpoint{5.626573in}{0.645792in}}%
\pgfpathlineto{\pgfqpoint{5.628721in}{0.653928in}}%
\pgfpathlineto{\pgfqpoint{5.633016in}{0.657778in}}%
\pgfpathlineto{\pgfqpoint{5.634090in}{0.654563in}}%
\pgfpathlineto{\pgfqpoint{5.635164in}{0.662660in}}%
\pgfpathlineto{\pgfqpoint{5.641607in}{0.670876in}}%
\pgfpathlineto{\pgfqpoint{5.642680in}{0.682704in}}%
\pgfpathlineto{\pgfqpoint{5.643754in}{0.682109in}}%
\pgfpathlineto{\pgfqpoint{5.648050in}{0.684649in}}%
\pgfpathlineto{\pgfqpoint{5.650197in}{0.689412in}}%
\pgfpathlineto{\pgfqpoint{5.651271in}{0.684530in}}%
\pgfpathlineto{\pgfqpoint{5.655567in}{0.691277in}}%
\pgfpathlineto{\pgfqpoint{5.656640in}{0.682704in}}%
\pgfpathlineto{\pgfqpoint{5.657714in}{0.681752in}}%
\pgfpathlineto{\pgfqpoint{5.658788in}{0.691754in}}%
\pgfpathlineto{\pgfqpoint{5.662010in}{0.694572in}}%
\pgfpathlineto{\pgfqpoint{5.663083in}{0.698303in}}%
\pgfpathlineto{\pgfqpoint{5.664157in}{0.694929in}}%
\pgfpathlineto{\pgfqpoint{5.665231in}{0.693778in}}%
\pgfpathlineto{\pgfqpoint{5.666305in}{0.688658in}}%
\pgfpathlineto{\pgfqpoint{5.670600in}{0.693500in}}%
\pgfpathlineto{\pgfqpoint{5.671674in}{0.701002in}}%
\pgfpathlineto{\pgfqpoint{5.672748in}{0.703145in}}%
\pgfpathlineto{\pgfqpoint{5.673822in}{0.708225in}}%
\pgfpathlineto{\pgfqpoint{5.677043in}{0.705804in}}%
\pgfpathlineto{\pgfqpoint{5.678117in}{0.700287in}}%
\pgfpathlineto{\pgfqpoint{5.679191in}{0.702986in}}%
\pgfpathlineto{\pgfqpoint{5.681339in}{0.689729in}}%
\pgfpathlineto{\pgfqpoint{5.684560in}{0.683538in}}%
\pgfpathlineto{\pgfqpoint{5.685634in}{0.690722in}}%
\pgfpathlineto{\pgfqpoint{5.687782in}{0.677624in}}%
\pgfpathlineto{\pgfqpoint{5.688856in}{0.686395in}}%
\pgfpathlineto{\pgfqpoint{5.692077in}{0.684887in}}%
\pgfpathlineto{\pgfqpoint{5.694225in}{0.676036in}}%
\pgfpathlineto{\pgfqpoint{5.695299in}{0.676036in}}%
\pgfpathlineto{\pgfqpoint{5.696372in}{0.665796in}}%
\pgfpathlineto{\pgfqpoint{5.699594in}{0.670717in}}%
\pgfpathlineto{\pgfqpoint{5.701742in}{0.687467in}}%
\pgfpathlineto{\pgfqpoint{5.702815in}{0.680759in}}%
\pgfpathlineto{\pgfqpoint{5.703889in}{0.664605in}}%
\pgfpathlineto{\pgfqpoint{5.707111in}{0.660319in}}%
\pgfpathlineto{\pgfqpoint{5.708185in}{0.660914in}}%
\pgfpathlineto{\pgfqpoint{5.709258in}{0.656469in}}%
\pgfpathlineto{\pgfqpoint{5.714628in}{0.662819in}}%
\pgfpathlineto{\pgfqpoint{5.715701in}{0.662184in}}%
\pgfpathlineto{\pgfqpoint{5.716775in}{0.659406in}}%
\pgfpathlineto{\pgfqpoint{5.717849in}{0.654643in}}%
\pgfpathlineto{\pgfqpoint{5.722144in}{0.646744in}}%
\pgfpathlineto{\pgfqpoint{5.723218in}{0.638886in}}%
\pgfpathlineto{\pgfqpoint{5.724292in}{0.636861in}}%
\pgfpathlineto{\pgfqpoint{5.725366in}{0.633646in}}%
\pgfpathlineto{\pgfqpoint{5.726440in}{0.632694in}}%
\pgfpathlineto{\pgfqpoint{5.729661in}{0.635194in}}%
\pgfpathlineto{\pgfqpoint{5.730735in}{0.641029in}}%
\pgfpathlineto{\pgfqpoint{5.731809in}{0.629796in}}%
\pgfpathlineto{\pgfqpoint{5.732883in}{0.632138in}}%
\pgfpathlineto{\pgfqpoint{5.733957in}{0.612531in}}%
\pgfpathlineto{\pgfqpoint{5.738252in}{0.612809in}}%
\pgfpathlineto{\pgfqpoint{5.739326in}{0.607411in}}%
\pgfpathlineto{\pgfqpoint{5.740400in}{0.612967in}}%
\pgfpathlineto{\pgfqpoint{5.741474in}{0.623843in}}%
\pgfpathlineto{\pgfqpoint{5.744695in}{0.617770in}}%
\pgfpathlineto{\pgfqpoint{5.745769in}{0.621461in}}%
\pgfpathlineto{\pgfqpoint{5.746843in}{0.614119in}}%
\pgfpathlineto{\pgfqpoint{5.747917in}{0.611142in}}%
\pgfpathlineto{\pgfqpoint{5.748990in}{0.619755in}}%
\pgfpathlineto{\pgfqpoint{5.752212in}{0.617254in}}%
\pgfpathlineto{\pgfqpoint{5.753286in}{0.609594in}}%
\pgfpathlineto{\pgfqpoint{5.754360in}{0.617214in}}%
\pgfpathlineto{\pgfqpoint{5.755433in}{0.618286in}}%
\pgfpathlineto{\pgfqpoint{5.756507in}{0.612531in}}%
\pgfpathlineto{\pgfqpoint{5.759729in}{0.605823in}}%
\pgfpathlineto{\pgfqpoint{5.760803in}{0.606577in}}%
\pgfpathlineto{\pgfqpoint{5.761877in}{0.593519in}}%
\pgfpathlineto{\pgfqpoint{5.764024in}{0.602172in}}%
\pgfpathlineto{\pgfqpoint{5.768320in}{0.608800in}}%
\pgfpathlineto{\pgfqpoint{5.769393in}{0.618564in}}%
\pgfpathlineto{\pgfqpoint{5.771541in}{0.616698in}}%
\pgfpathlineto{\pgfqpoint{5.774763in}{0.621819in}}%
\pgfpathlineto{\pgfqpoint{5.775836in}{0.618088in}}%
\pgfpathlineto{\pgfqpoint{5.776910in}{0.618286in}}%
\pgfpathlineto{\pgfqpoint{5.777984in}{0.619120in}}%
\pgfpathlineto{\pgfqpoint{5.779058in}{0.617849in}}%
\pgfpathlineto{\pgfqpoint{5.782279in}{0.618643in}}%
\pgfpathlineto{\pgfqpoint{5.783353in}{0.626661in}}%
\pgfpathlineto{\pgfqpoint{5.784427in}{0.624200in}}%
\pgfpathlineto{\pgfqpoint{5.785501in}{0.631066in}}%
\pgfpathlineto{\pgfqpoint{5.786575in}{0.629796in}}%
\pgfpathlineto{\pgfqpoint{5.789796in}{0.633210in}}%
\pgfpathlineto{\pgfqpoint{5.790870in}{0.627296in}}%
\pgfpathlineto{\pgfqpoint{5.791944in}{0.626700in}}%
\pgfpathlineto{\pgfqpoint{5.793018in}{0.624359in}}%
\pgfpathlineto{\pgfqpoint{5.794092in}{0.627732in}}%
\pgfpathlineto{\pgfqpoint{5.797313in}{0.631027in}}%
\pgfpathlineto{\pgfqpoint{5.798387in}{0.628883in}}%
\pgfpathlineto{\pgfqpoint{5.799461in}{0.629638in}}%
\pgfpathlineto{\pgfqpoint{5.800535in}{0.634004in}}%
\pgfpathlineto{\pgfqpoint{5.801609in}{0.632495in}}%
\pgfpathlineto{\pgfqpoint{5.804830in}{0.629717in}}%
\pgfpathlineto{\pgfqpoint{5.806978in}{0.623565in}}%
\pgfpathlineto{\pgfqpoint{5.808052in}{0.625033in}}%
\pgfpathlineto{\pgfqpoint{5.813421in}{0.628566in}}%
\pgfpathlineto{\pgfqpoint{5.815568in}{0.632892in}}%
\pgfpathlineto{\pgfqpoint{5.816642in}{0.632019in}}%
\pgfpathlineto{\pgfqpoint{5.819864in}{0.630550in}}%
\pgfpathlineto{\pgfqpoint{5.820938in}{0.624200in}}%
\pgfpathlineto{\pgfqpoint{5.822011in}{0.626026in}}%
\pgfpathlineto{\pgfqpoint{5.823085in}{0.621064in}}%
\pgfpathlineto{\pgfqpoint{5.824159in}{0.622017in}}%
\pgfpathlineto{\pgfqpoint{5.827381in}{0.621461in}}%
\pgfpathlineto{\pgfqpoint{5.828455in}{0.625510in}}%
\pgfpathlineto{\pgfqpoint{5.829528in}{0.633567in}}%
\pgfpathlineto{\pgfqpoint{5.830602in}{0.630352in}}%
\pgfpathlineto{\pgfqpoint{5.831676in}{0.630193in}}%
\pgfpathlineto{\pgfqpoint{5.837045in}{0.647816in}}%
\pgfpathlineto{\pgfqpoint{5.838119in}{0.646466in}}%
\pgfpathlineto{\pgfqpoint{5.839193in}{0.649721in}}%
\pgfpathlineto{\pgfqpoint{5.844562in}{0.655397in}}%
\pgfpathlineto{\pgfqpoint{5.846710in}{0.647776in}}%
\pgfpathlineto{\pgfqpoint{5.849931in}{0.651944in}}%
\pgfpathlineto{\pgfqpoint{5.851005in}{0.649761in}}%
\pgfpathlineto{\pgfqpoint{5.852079in}{0.649324in}}%
\pgfpathlineto{\pgfqpoint{5.854227in}{0.656389in}}%
\pgfpathlineto{\pgfqpoint{5.857448in}{0.655635in}}%
\pgfpathlineto{\pgfqpoint{5.858522in}{0.660358in}}%
\pgfpathlineto{\pgfqpoint{5.859596in}{0.644125in}}%
\pgfpathlineto{\pgfqpoint{5.860670in}{0.641942in}}%
\pgfpathlineto{\pgfqpoint{5.861744in}{0.637457in}}%
\pgfpathlineto{\pgfqpoint{5.864965in}{0.636861in}}%
\pgfpathlineto{\pgfqpoint{5.866039in}{0.635274in}}%
\pgfpathlineto{\pgfqpoint{5.868187in}{0.629519in}}%
\pgfpathlineto{\pgfqpoint{5.869260in}{0.634678in}}%
\pgfpathlineto{\pgfqpoint{5.872482in}{0.632416in}}%
\pgfpathlineto{\pgfqpoint{5.874630in}{0.634996in}}%
\pgfpathlineto{\pgfqpoint{5.875703in}{0.634797in}}%
\pgfpathlineto{\pgfqpoint{5.876777in}{0.636583in}}%
\pgfpathlineto{\pgfqpoint{5.881073in}{0.632575in}}%
\pgfpathlineto{\pgfqpoint{5.882146in}{0.629915in}}%
\pgfpathlineto{\pgfqpoint{5.884294in}{0.630789in}}%
\pgfpathlineto{\pgfqpoint{5.887516in}{0.630908in}}%
\pgfpathlineto{\pgfqpoint{5.889663in}{0.628129in}}%
\pgfpathlineto{\pgfqpoint{5.890737in}{0.627375in}}%
\pgfpathlineto{\pgfqpoint{5.891811in}{0.625470in}}%
\pgfpathlineto{\pgfqpoint{5.895032in}{0.626343in}}%
\pgfpathlineto{\pgfqpoint{5.896106in}{0.629479in}}%
\pgfpathlineto{\pgfqpoint{5.897180in}{0.629003in}}%
\pgfpathlineto{\pgfqpoint{5.898254in}{0.629399in}}%
\pgfpathlineto{\pgfqpoint{5.899328in}{0.631741in}}%
\pgfpathlineto{\pgfqpoint{5.902549in}{0.633885in}}%
\pgfpathlineto{\pgfqpoint{5.903623in}{0.631066in}}%
\pgfpathlineto{\pgfqpoint{5.904697in}{0.630947in}}%
\pgfpathlineto{\pgfqpoint{5.905771in}{0.631821in}}%
\pgfpathlineto{\pgfqpoint{5.906845in}{0.619397in}}%
\pgfpathlineto{\pgfqpoint{5.910066in}{0.614357in}}%
\pgfpathlineto{\pgfqpoint{5.912214in}{0.624121in}}%
\pgfpathlineto{\pgfqpoint{5.913288in}{0.627296in}}%
\pgfpathlineto{\pgfqpoint{5.914362in}{0.628090in}}%
\pgfpathlineto{\pgfqpoint{5.918657in}{0.626700in}}%
\pgfpathlineto{\pgfqpoint{5.921878in}{0.636822in}}%
\pgfpathlineto{\pgfqpoint{5.926174in}{0.639005in}}%
\pgfpathlineto{\pgfqpoint{5.927248in}{0.637774in}}%
\pgfpathlineto{\pgfqpoint{5.928321in}{0.638131in}}%
\pgfpathlineto{\pgfqpoint{5.929395in}{0.637496in}}%
\pgfpathlineto{\pgfqpoint{5.932617in}{0.638806in}}%
\pgfpathlineto{\pgfqpoint{5.933691in}{0.636226in}}%
\pgfpathlineto{\pgfqpoint{5.934765in}{0.631503in}}%
\pgfpathlineto{\pgfqpoint{5.936912in}{0.629558in}}%
\pgfpathlineto{\pgfqpoint{5.940134in}{0.628328in}}%
\pgfpathlineto{\pgfqpoint{5.942281in}{0.624319in}}%
\pgfpathlineto{\pgfqpoint{5.943355in}{0.622731in}}%
\pgfpathlineto{\pgfqpoint{5.944429in}{0.622890in}}%
\pgfpathlineto{\pgfqpoint{5.947651in}{0.621342in}}%
\pgfpathlineto{\pgfqpoint{5.948724in}{0.619318in}}%
\pgfpathlineto{\pgfqpoint{5.949798in}{0.623406in}}%
\pgfpathlineto{\pgfqpoint{5.950872in}{0.619874in}}%
\pgfpathlineto{\pgfqpoint{5.951946in}{0.622414in}}%
\pgfpathlineto{\pgfqpoint{5.955167in}{0.622136in}}%
\pgfpathlineto{\pgfqpoint{5.957315in}{0.630114in}}%
\pgfpathlineto{\pgfqpoint{5.958389in}{0.629796in}}%
\pgfpathlineto{\pgfqpoint{5.959463in}{0.626264in}}%
\pgfpathlineto{\pgfqpoint{5.962684in}{0.627256in}}%
\pgfpathlineto{\pgfqpoint{5.963758in}{0.626423in}}%
\pgfpathlineto{\pgfqpoint{5.964832in}{0.626383in}}%
\pgfpathlineto{\pgfqpoint{5.970201in}{0.622573in}}%
\pgfpathlineto{\pgfqpoint{5.971275in}{0.622970in}}%
\pgfpathlineto{\pgfqpoint{5.973423in}{0.621342in}}%
\pgfpathlineto{\pgfqpoint{5.974497in}{0.620072in}}%
\pgfpathlineto{\pgfqpoint{5.978792in}{0.618762in}}%
\pgfpathlineto{\pgfqpoint{5.980940in}{0.616460in}}%
\pgfpathlineto{\pgfqpoint{5.982013in}{0.617056in}}%
\pgfpathlineto{\pgfqpoint{5.987383in}{0.614396in}}%
\pgfpathlineto{\pgfqpoint{5.988456in}{0.615825in}}%
\pgfpathlineto{\pgfqpoint{5.989530in}{0.609475in}}%
\pgfpathlineto{\pgfqpoint{5.992752in}{0.614119in}}%
\pgfpathlineto{\pgfqpoint{5.994899in}{0.608879in}}%
\pgfpathlineto{\pgfqpoint{5.997047in}{0.610030in}}%
\pgfpathlineto{\pgfqpoint{6.000269in}{0.610269in}}%
\pgfpathlineto{\pgfqpoint{6.001343in}{0.611499in}}%
\pgfpathlineto{\pgfqpoint{6.002416in}{0.609356in}}%
\pgfpathlineto{\pgfqpoint{6.003490in}{0.613245in}}%
\pgfpathlineto{\pgfqpoint{6.004564in}{0.612729in}}%
\pgfpathlineto{\pgfqpoint{6.008859in}{0.606815in}}%
\pgfpathlineto{\pgfqpoint{6.009933in}{0.608641in}}%
\pgfpathlineto{\pgfqpoint{6.011007in}{0.607133in}}%
\pgfpathlineto{\pgfqpoint{6.012081in}{0.611142in}}%
\pgfpathlineto{\pgfqpoint{6.015302in}{0.609753in}}%
\pgfpathlineto{\pgfqpoint{6.016376in}{0.610149in}}%
\pgfpathlineto{\pgfqpoint{6.017450in}{0.609594in}}%
\pgfpathlineto{\pgfqpoint{6.018524in}{0.611023in}}%
\pgfpathlineto{\pgfqpoint{6.019598in}{0.609753in}}%
\pgfpathlineto{\pgfqpoint{6.022819in}{0.609753in}}%
\pgfpathlineto{\pgfqpoint{6.024967in}{0.605664in}}%
\pgfpathlineto{\pgfqpoint{6.026041in}{0.604553in}}%
\pgfpathlineto{\pgfqpoint{6.027115in}{0.605228in}}%
\pgfpathlineto{\pgfqpoint{6.030336in}{0.603442in}}%
\pgfpathlineto{\pgfqpoint{6.031410in}{0.604752in}}%
\pgfpathlineto{\pgfqpoint{6.032484in}{0.607609in}}%
\pgfpathlineto{\pgfqpoint{6.033558in}{0.608006in}}%
\pgfpathlineto{\pgfqpoint{6.034632in}{0.611816in}}%
\pgfpathlineto{\pgfqpoint{6.037853in}{0.613087in}}%
\pgfpathlineto{\pgfqpoint{6.038927in}{0.610745in}}%
\pgfpathlineto{\pgfqpoint{6.041075in}{0.615547in}}%
\pgfpathlineto{\pgfqpoint{6.042148in}{0.614912in}}%
\pgfpathlineto{\pgfqpoint{6.047518in}{0.607570in}}%
\pgfpathlineto{\pgfqpoint{6.048591in}{0.613087in}}%
\pgfpathlineto{\pgfqpoint{6.049665in}{0.609594in}}%
\pgfpathlineto{\pgfqpoint{6.052887in}{0.617095in}}%
\pgfpathlineto{\pgfqpoint{6.053961in}{0.616937in}}%
\pgfpathlineto{\pgfqpoint{6.056108in}{0.619120in}}%
\pgfpathlineto{\pgfqpoint{6.057182in}{0.629439in}}%
\pgfpathlineto{\pgfqpoint{6.060404in}{0.630352in}}%
\pgfpathlineto{\pgfqpoint{6.061477in}{0.629519in}}%
\pgfpathlineto{\pgfqpoint{6.062551in}{0.634916in}}%
\pgfpathlineto{\pgfqpoint{6.063625in}{0.635869in}}%
\pgfpathlineto{\pgfqpoint{6.064699in}{0.631582in}}%
\pgfpathlineto{\pgfqpoint{6.067920in}{0.629241in}}%
\pgfpathlineto{\pgfqpoint{6.068994in}{0.629558in}}%
\pgfpathlineto{\pgfqpoint{6.070068in}{0.631662in}}%
\pgfpathlineto{\pgfqpoint{6.072216in}{0.633765in}}%
\pgfpathlineto{\pgfqpoint{6.075437in}{0.634321in}}%
\pgfpathlineto{\pgfqpoint{6.076511in}{0.636980in}}%
\pgfpathlineto{\pgfqpoint{6.077585in}{0.634916in}}%
\pgfpathlineto{\pgfqpoint{6.078659in}{0.634242in}}%
\pgfpathlineto{\pgfqpoint{6.079733in}{0.632575in}}%
\pgfpathlineto{\pgfqpoint{6.084028in}{0.640751in}}%
\pgfpathlineto{\pgfqpoint{6.085102in}{0.645792in}}%
\pgfpathlineto{\pgfqpoint{6.087250in}{0.659763in}}%
\pgfpathlineto{\pgfqpoint{6.091545in}{0.655873in}}%
\pgfpathlineto{\pgfqpoint{6.093693in}{0.657937in}}%
\pgfpathlineto{\pgfqpoint{6.094766in}{0.656111in}}%
\pgfpathlineto{\pgfqpoint{6.097988in}{0.661430in}}%
\pgfpathlineto{\pgfqpoint{6.100136in}{0.662422in}}%
\pgfpathlineto{\pgfqpoint{6.102283in}{0.660874in}}%
\pgfpathlineto{\pgfqpoint{6.106579in}{0.660954in}}%
\pgfpathlineto{\pgfqpoint{6.107653in}{0.657620in}}%
\pgfpathlineto{\pgfqpoint{6.108726in}{0.658612in}}%
\pgfpathlineto{\pgfqpoint{6.109800in}{0.657302in}}%
\pgfpathlineto{\pgfqpoint{6.114096in}{0.664407in}}%
\pgfpathlineto{\pgfqpoint{6.115169in}{0.669606in}}%
\pgfpathlineto{\pgfqpoint{6.116243in}{0.669408in}}%
\pgfpathlineto{\pgfqpoint{6.117317in}{0.675520in}}%
\pgfpathlineto{\pgfqpoint{6.120539in}{0.673139in}}%
\pgfpathlineto{\pgfqpoint{6.121612in}{0.673218in}}%
\pgfpathlineto{\pgfqpoint{6.122686in}{0.677266in}}%
\pgfpathlineto{\pgfqpoint{6.123760in}{0.669963in}}%
\pgfpathlineto{\pgfqpoint{6.124834in}{0.671988in}}%
\pgfpathlineto{\pgfqpoint{6.129129in}{0.671630in}}%
\pgfpathlineto{\pgfqpoint{6.130203in}{0.672385in}}%
\pgfpathlineto{\pgfqpoint{6.131277in}{0.669011in}}%
\pgfpathlineto{\pgfqpoint{6.132351in}{0.670479in}}%
\pgfpathlineto{\pgfqpoint{6.135572in}{0.668376in}}%
\pgfpathlineto{\pgfqpoint{6.136646in}{0.671392in}}%
\pgfpathlineto{\pgfqpoint{6.138794in}{0.671988in}}%
\pgfpathlineto{\pgfqpoint{6.139868in}{0.676711in}}%
\pgfpathlineto{\pgfqpoint{6.143089in}{0.682982in}}%
\pgfpathlineto{\pgfqpoint{6.144163in}{0.681910in}}%
\pgfpathlineto{\pgfqpoint{6.145237in}{0.684371in}}%
\pgfpathlineto{\pgfqpoint{6.147385in}{0.680561in}}%
\pgfpathlineto{\pgfqpoint{6.150606in}{0.677782in}}%
\pgfpathlineto{\pgfqpoint{6.151680in}{0.675599in}}%
\pgfpathlineto{\pgfqpoint{6.152754in}{0.675599in}}%
\pgfpathlineto{\pgfqpoint{6.153828in}{0.677465in}}%
\pgfpathlineto{\pgfqpoint{6.154901in}{0.676592in}}%
\pgfpathlineto{\pgfqpoint{6.158123in}{0.678060in}}%
\pgfpathlineto{\pgfqpoint{6.159197in}{0.680680in}}%
\pgfpathlineto{\pgfqpoint{6.160271in}{0.680085in}}%
\pgfpathlineto{\pgfqpoint{6.161344in}{0.682109in}}%
\pgfpathlineto{\pgfqpoint{6.162418in}{0.679648in}}%
\pgfpathlineto{\pgfqpoint{6.167787in}{0.679846in}}%
\pgfpathlineto{\pgfqpoint{6.168861in}{0.678378in}}%
\pgfpathlineto{\pgfqpoint{6.169935in}{0.680640in}}%
\pgfpathlineto{\pgfqpoint{6.174231in}{0.679767in}}%
\pgfpathlineto{\pgfqpoint{6.175304in}{0.683379in}}%
\pgfpathlineto{\pgfqpoint{6.176378in}{0.681672in}}%
\pgfpathlineto{\pgfqpoint{6.177452in}{0.684133in}}%
\pgfpathlineto{\pgfqpoint{6.180674in}{0.681950in}}%
\pgfpathlineto{\pgfqpoint{6.181747in}{0.682704in}}%
\pgfpathlineto{\pgfqpoint{6.182821in}{0.682625in}}%
\pgfpathlineto{\pgfqpoint{6.183895in}{0.683339in}}%
\pgfpathlineto{\pgfqpoint{6.184969in}{0.682942in}}%
\pgfpathlineto{\pgfqpoint{6.188190in}{0.685205in}}%
\pgfpathlineto{\pgfqpoint{6.189264in}{0.688221in}}%
\pgfpathlineto{\pgfqpoint{6.190338in}{0.686554in}}%
\pgfpathlineto{\pgfqpoint{6.191412in}{0.685959in}}%
\pgfpathlineto{\pgfqpoint{6.192486in}{0.686117in}}%
\pgfpathlineto{\pgfqpoint{6.195707in}{0.689769in}}%
\pgfpathlineto{\pgfqpoint{6.196781in}{0.686117in}}%
\pgfpathlineto{\pgfqpoint{6.198929in}{0.687983in}}%
\pgfpathlineto{\pgfqpoint{6.200003in}{0.687586in}}%
\pgfpathlineto{\pgfqpoint{6.203224in}{0.688499in}}%
\pgfpathlineto{\pgfqpoint{6.204298in}{0.691079in}}%
\pgfpathlineto{\pgfqpoint{6.205372in}{0.689094in}}%
\pgfpathlineto{\pgfqpoint{6.206446in}{0.691555in}}%
\pgfpathlineto{\pgfqpoint{6.207520in}{0.692389in}}%
\pgfpathlineto{\pgfqpoint{6.213963in}{0.691079in}}%
\pgfpathlineto{\pgfqpoint{6.215036in}{0.689253in}}%
\pgfpathlineto{\pgfqpoint{6.218258in}{0.688697in}}%
\pgfpathlineto{\pgfqpoint{6.219332in}{0.691158in}}%
\pgfpathlineto{\pgfqpoint{6.220406in}{0.691039in}}%
\pgfpathlineto{\pgfqpoint{6.225775in}{0.693857in}}%
\pgfpathlineto{\pgfqpoint{6.226849in}{0.695445in}}%
\pgfpathlineto{\pgfqpoint{6.227922in}{0.693659in}}%
\pgfpathlineto{\pgfqpoint{6.228996in}{0.697707in}}%
\pgfpathlineto{\pgfqpoint{6.230070in}{0.696397in}}%
\pgfpathlineto{\pgfqpoint{6.233292in}{0.693540in}}%
\pgfpathlineto{\pgfqpoint{6.234365in}{0.699215in}}%
\pgfpathlineto{\pgfqpoint{6.236513in}{0.701756in}}%
\pgfpathlineto{\pgfqpoint{6.240809in}{0.697509in}}%
\pgfpathlineto{\pgfqpoint{6.241882in}{0.696120in}}%
\pgfpathlineto{\pgfqpoint{6.242956in}{0.685602in}}%
\pgfpathlineto{\pgfqpoint{6.244030in}{0.683895in}}%
\pgfpathlineto{\pgfqpoint{6.245104in}{0.687030in}}%
\pgfpathlineto{\pgfqpoint{6.248325in}{0.684847in}}%
\pgfpathlineto{\pgfqpoint{6.249399in}{0.687308in}}%
\pgfpathlineto{\pgfqpoint{6.250473in}{0.678100in}}%
\pgfpathlineto{\pgfqpoint{6.251547in}{0.677782in}}%
\pgfpathlineto{\pgfqpoint{6.252621in}{0.678219in}}%
\pgfpathlineto{\pgfqpoint{6.255842in}{0.676076in}}%
\pgfpathlineto{\pgfqpoint{6.257990in}{0.665280in}}%
\pgfpathlineto{\pgfqpoint{6.259064in}{0.666709in}}%
\pgfpathlineto{\pgfqpoint{6.260138in}{0.669924in}}%
\pgfpathlineto{\pgfqpoint{6.263359in}{0.670360in}}%
\pgfpathlineto{\pgfqpoint{6.264433in}{0.668019in}}%
\pgfpathlineto{\pgfqpoint{6.265507in}{0.670638in}}%
\pgfpathlineto{\pgfqpoint{6.266581in}{0.668971in}}%
\pgfpathlineto{\pgfqpoint{6.267654in}{0.673337in}}%
\pgfpathlineto{\pgfqpoint{6.271950in}{0.673059in}}%
\pgfpathlineto{\pgfqpoint{6.273024in}{0.671511in}}%
\pgfpathlineto{\pgfqpoint{6.274097in}{0.672543in}}%
\pgfpathlineto{\pgfqpoint{6.275171in}{0.668614in}}%
\pgfpathlineto{\pgfqpoint{6.278393in}{0.666113in}}%
\pgfpathlineto{\pgfqpoint{6.279467in}{0.662184in}}%
\pgfpathlineto{\pgfqpoint{6.280541in}{0.663811in}}%
\pgfpathlineto{\pgfqpoint{6.281614in}{0.657699in}}%
\pgfpathlineto{\pgfqpoint{6.282688in}{0.662660in}}%
\pgfpathlineto{\pgfqpoint{6.285910in}{0.668098in}}%
\pgfpathlineto{\pgfqpoint{6.288057in}{0.664645in}}%
\pgfpathlineto{\pgfqpoint{6.289131in}{0.664049in}}%
\pgfpathlineto{\pgfqpoint{6.290205in}{0.662224in}}%
\pgfpathlineto{\pgfqpoint{6.293427in}{0.661708in}}%
\pgfpathlineto{\pgfqpoint{6.294500in}{0.656230in}}%
\pgfpathlineto{\pgfqpoint{6.295574in}{0.659525in}}%
\pgfpathlineto{\pgfqpoint{6.296648in}{0.657302in}}%
\pgfpathlineto{\pgfqpoint{6.297722in}{0.657858in}}%
\pgfpathlineto{\pgfqpoint{6.300943in}{0.662462in}}%
\pgfpathlineto{\pgfqpoint{6.302017in}{0.661866in}}%
\pgfpathlineto{\pgfqpoint{6.303091in}{0.667383in}}%
\pgfpathlineto{\pgfqpoint{6.304165in}{0.663017in}}%
\pgfpathlineto{\pgfqpoint{6.305239in}{0.665081in}}%
\pgfpathlineto{\pgfqpoint{6.308460in}{0.669686in}}%
\pgfpathlineto{\pgfqpoint{6.310608in}{0.662700in}}%
\pgfpathlineto{\pgfqpoint{6.311682in}{0.656905in}}%
\pgfpathlineto{\pgfqpoint{6.312756in}{0.656786in}}%
\pgfpathlineto{\pgfqpoint{6.315977in}{0.657858in}}%
\pgfpathlineto{\pgfqpoint{6.317051in}{0.658810in}}%
\pgfpathlineto{\pgfqpoint{6.318125in}{0.660874in}}%
\pgfpathlineto{\pgfqpoint{6.319199in}{0.660557in}}%
\pgfpathlineto{\pgfqpoint{6.320273in}{0.663613in}}%
\pgfpathlineto{\pgfqpoint{6.323494in}{0.662462in}}%
\pgfpathlineto{\pgfqpoint{6.326716in}{0.672385in}}%
\pgfpathlineto{\pgfqpoint{6.327789in}{0.671313in}}%
\pgfpathlineto{\pgfqpoint{6.331011in}{0.670995in}}%
\pgfpathlineto{\pgfqpoint{6.332085in}{0.668733in}}%
\pgfpathlineto{\pgfqpoint{6.333159in}{0.670757in}}%
\pgfpathlineto{\pgfqpoint{6.334232in}{0.682545in}}%
\pgfpathlineto{\pgfqpoint{6.338528in}{0.682267in}}%
\pgfpathlineto{\pgfqpoint{6.339602in}{0.684887in}}%
\pgfpathlineto{\pgfqpoint{6.340675in}{0.677425in}}%
\pgfpathlineto{\pgfqpoint{6.341749in}{0.679132in}}%
\pgfpathlineto{\pgfqpoint{6.342823in}{0.673655in}}%
\pgfpathlineto{\pgfqpoint{6.346045in}{0.668495in}}%
\pgfpathlineto{\pgfqpoint{6.347119in}{0.670916in}}%
\pgfpathlineto{\pgfqpoint{6.348192in}{0.654921in}}%
\pgfpathlineto{\pgfqpoint{6.349266in}{0.649205in}}%
\pgfpathlineto{\pgfqpoint{6.350340in}{0.651666in}}%
\pgfpathlineto{\pgfqpoint{6.353562in}{0.649404in}}%
\pgfpathlineto{\pgfqpoint{6.354635in}{0.649840in}}%
\pgfpathlineto{\pgfqpoint{6.355709in}{0.652499in}}%
\pgfpathlineto{\pgfqpoint{6.357857in}{0.646705in}}%
\pgfpathlineto{\pgfqpoint{6.361078in}{0.648530in}}%
\pgfpathlineto{\pgfqpoint{6.362152in}{0.654682in}}%
\pgfpathlineto{\pgfqpoint{6.363226in}{0.649800in}}%
\pgfpathlineto{\pgfqpoint{6.364300in}{0.649880in}}%
\pgfpathlineto{\pgfqpoint{6.365374in}{0.653293in}}%
\pgfpathlineto{\pgfqpoint{6.369669in}{0.653928in}}%
\pgfpathlineto{\pgfqpoint{6.370743in}{0.655079in}}%
\pgfpathlineto{\pgfqpoint{6.371817in}{0.648649in}}%
\pgfpathlineto{\pgfqpoint{6.372891in}{0.649800in}}%
\pgfpathlineto{\pgfqpoint{6.378260in}{0.649800in}}%
\pgfpathlineto{\pgfqpoint{6.379334in}{0.632694in}}%
\pgfpathlineto{\pgfqpoint{6.383629in}{0.632813in}}%
\pgfpathlineto{\pgfqpoint{6.385777in}{0.639481in}}%
\pgfpathlineto{\pgfqpoint{6.386851in}{0.635909in}}%
\pgfpathlineto{\pgfqpoint{6.387924in}{0.638330in}}%
\pgfpathlineto{\pgfqpoint{6.391146in}{0.636703in}}%
\pgfpathlineto{\pgfqpoint{6.392220in}{0.637973in}}%
\pgfpathlineto{\pgfqpoint{6.393294in}{0.640989in}}%
\pgfpathlineto{\pgfqpoint{6.395441in}{0.638608in}}%
\pgfpathlineto{\pgfqpoint{6.398663in}{0.642378in}}%
\pgfpathlineto{\pgfqpoint{6.399737in}{0.638727in}}%
\pgfpathlineto{\pgfqpoint{6.400810in}{0.641069in}}%
\pgfpathlineto{\pgfqpoint{6.401884in}{0.636504in}}%
\pgfpathlineto{\pgfqpoint{6.407253in}{0.647062in}}%
\pgfpathlineto{\pgfqpoint{6.408327in}{0.646149in}}%
\pgfpathlineto{\pgfqpoint{6.409401in}{0.644442in}}%
\pgfpathlineto{\pgfqpoint{6.410475in}{0.644283in}}%
\pgfpathlineto{\pgfqpoint{6.414770in}{0.642378in}}%
\pgfpathlineto{\pgfqpoint{6.415844in}{0.638449in}}%
\pgfpathlineto{\pgfqpoint{6.416918in}{0.632178in}}%
\pgfpathlineto{\pgfqpoint{6.417992in}{0.633924in}}%
\pgfpathlineto{\pgfqpoint{6.422287in}{0.637695in}}%
\pgfpathlineto{\pgfqpoint{6.423361in}{0.637258in}}%
\pgfpathlineto{\pgfqpoint{6.425509in}{0.641704in}}%
\pgfpathlineto{\pgfqpoint{6.429804in}{0.637417in}}%
\pgfpathlineto{\pgfqpoint{6.430878in}{0.635552in}}%
\pgfpathlineto{\pgfqpoint{6.431952in}{0.638449in}}%
\pgfpathlineto{\pgfqpoint{6.433026in}{0.637496in}}%
\pgfpathlineto{\pgfqpoint{6.437321in}{0.635591in}}%
\pgfpathlineto{\pgfqpoint{6.438395in}{0.640275in}}%
\pgfpathlineto{\pgfqpoint{6.439469in}{0.637655in}}%
\pgfpathlineto{\pgfqpoint{6.440542in}{0.638766in}}%
\pgfpathlineto{\pgfqpoint{6.444838in}{0.650197in}}%
\pgfpathlineto{\pgfqpoint{6.445912in}{0.648570in}}%
\pgfpathlineto{\pgfqpoint{6.448059in}{0.662422in}}%
\pgfpathlineto{\pgfqpoint{6.451281in}{0.662263in}}%
\pgfpathlineto{\pgfqpoint{6.452355in}{0.656230in}}%
\pgfpathlineto{\pgfqpoint{6.453429in}{0.658215in}}%
\pgfpathlineto{\pgfqpoint{6.455576in}{0.657262in}}%
\pgfpathlineto{\pgfqpoint{6.458798in}{0.654603in}}%
\pgfpathlineto{\pgfqpoint{6.459872in}{0.655556in}}%
\pgfpathlineto{\pgfqpoint{6.460945in}{0.654563in}}%
\pgfpathlineto{\pgfqpoint{6.463093in}{0.654166in}}%
\pgfpathlineto{\pgfqpoint{6.466315in}{0.654762in}}%
\pgfpathlineto{\pgfqpoint{6.467388in}{0.657143in}}%
\pgfpathlineto{\pgfqpoint{6.468462in}{0.664208in}}%
\pgfpathlineto{\pgfqpoint{6.469536in}{0.662581in}}%
\pgfpathlineto{\pgfqpoint{6.470610in}{0.664248in}}%
\pgfpathlineto{\pgfqpoint{6.473831in}{0.683379in}}%
\pgfpathlineto{\pgfqpoint{6.475979in}{0.665042in}}%
\pgfpathlineto{\pgfqpoint{6.478127in}{0.663533in}}%
\pgfpathlineto{\pgfqpoint{6.482422in}{0.675957in}}%
\pgfpathlineto{\pgfqpoint{6.483496in}{0.676671in}}%
\pgfpathlineto{\pgfqpoint{6.484570in}{0.688142in}}%
\pgfpathlineto{\pgfqpoint{6.485644in}{0.690880in}}%
\pgfpathlineto{\pgfqpoint{6.488865in}{0.689928in}}%
\pgfpathlineto{\pgfqpoint{6.489939in}{0.692984in}}%
\pgfpathlineto{\pgfqpoint{6.491013in}{0.684728in}}%
\pgfpathlineto{\pgfqpoint{6.492087in}{0.684292in}}%
\pgfpathlineto{\pgfqpoint{6.493161in}{0.680799in}}%
\pgfpathlineto{\pgfqpoint{6.497456in}{0.678656in}}%
\pgfpathlineto{\pgfqpoint{6.498530in}{0.676790in}}%
\pgfpathlineto{\pgfqpoint{6.499604in}{0.677266in}}%
\pgfpathlineto{\pgfqpoint{6.500677in}{0.676274in}}%
\pgfpathlineto{\pgfqpoint{6.500677in}{0.676274in}}%
\pgfusepath{stroke}%
\end{pgfscope}%
\begin{pgfscope}%
\pgfpathrectangle{\pgfqpoint{4.034768in}{0.331635in}}{\pgfqpoint{2.583333in}{0.400885in}}%
\pgfusepath{clip}%
\pgfsetroundcap%
\pgfsetroundjoin%
\pgfsetlinewidth{1.505625pt}%
\definecolor{currentstroke}{rgb}{0.090196,0.745098,0.811765}%
\pgfsetstrokecolor{currentstroke}%
\pgfsetdash{}{0pt}%
\pgfpathmoveto{\pgfqpoint{4.152193in}{0.395780in}}%
\pgfpathlineto{\pgfqpoint{4.153266in}{0.395780in}}%
\pgfpathlineto{\pgfqpoint{4.155414in}{0.392070in}}%
\pgfpathlineto{\pgfqpoint{4.159709in}{0.392252in}}%
\pgfpathlineto{\pgfqpoint{4.160783in}{0.389112in}}%
\pgfpathlineto{\pgfqpoint{4.161857in}{0.389187in}}%
\pgfpathlineto{\pgfqpoint{4.162931in}{0.388103in}}%
\pgfpathlineto{\pgfqpoint{4.167226in}{0.388365in}}%
\pgfpathlineto{\pgfqpoint{4.169374in}{0.385189in}}%
\pgfpathlineto{\pgfqpoint{4.170448in}{0.385580in}}%
\pgfpathlineto{\pgfqpoint{4.174743in}{0.385365in}}%
\pgfpathlineto{\pgfqpoint{4.176891in}{0.387048in}}%
\pgfpathlineto{\pgfqpoint{4.182260in}{0.385600in}}%
\pgfpathlineto{\pgfqpoint{4.184408in}{0.388314in}}%
\pgfpathlineto{\pgfqpoint{4.185482in}{0.391935in}}%
\pgfpathlineto{\pgfqpoint{4.188703in}{0.390420in}}%
\pgfpathlineto{\pgfqpoint{4.190851in}{0.387808in}}%
\pgfpathlineto{\pgfqpoint{4.191925in}{0.386975in}}%
\pgfpathlineto{\pgfqpoint{4.197294in}{0.388900in}}%
\pgfpathlineto{\pgfqpoint{4.198368in}{0.387789in}}%
\pgfpathlineto{\pgfqpoint{4.199441in}{0.388727in}}%
\pgfpathlineto{\pgfqpoint{4.200515in}{0.388102in}}%
\pgfpathlineto{\pgfqpoint{4.204811in}{0.388652in}}%
\pgfpathlineto{\pgfqpoint{4.205885in}{0.387716in}}%
\pgfpathlineto{\pgfqpoint{4.208032in}{0.388895in}}%
\pgfpathlineto{\pgfqpoint{4.211254in}{0.389944in}}%
\pgfpathlineto{\pgfqpoint{4.212328in}{0.389035in}}%
\pgfpathlineto{\pgfqpoint{4.213401in}{0.388862in}}%
\pgfpathlineto{\pgfqpoint{4.214475in}{0.390100in}}%
\pgfpathlineto{\pgfqpoint{4.215549in}{0.390203in}}%
\pgfpathlineto{\pgfqpoint{4.218771in}{0.391271in}}%
\pgfpathlineto{\pgfqpoint{4.219844in}{0.393441in}}%
\pgfpathlineto{\pgfqpoint{4.220918in}{0.392622in}}%
\pgfpathlineto{\pgfqpoint{4.221992in}{0.393512in}}%
\pgfpathlineto{\pgfqpoint{4.223066in}{0.392800in}}%
\pgfpathlineto{\pgfqpoint{4.226287in}{0.392483in}}%
\pgfpathlineto{\pgfqpoint{4.227361in}{0.387226in}}%
\pgfpathlineto{\pgfqpoint{4.228435in}{0.388781in}}%
\pgfpathlineto{\pgfqpoint{4.229509in}{0.388748in}}%
\pgfpathlineto{\pgfqpoint{4.230583in}{0.387918in}}%
\pgfpathlineto{\pgfqpoint{4.237026in}{0.389413in}}%
\pgfpathlineto{\pgfqpoint{4.238100in}{0.390518in}}%
\pgfpathlineto{\pgfqpoint{4.241321in}{0.388341in}}%
\pgfpathlineto{\pgfqpoint{4.242395in}{0.388989in}}%
\pgfpathlineto{\pgfqpoint{4.244543in}{0.385618in}}%
\pgfpathlineto{\pgfqpoint{4.245617in}{0.387876in}}%
\pgfpathlineto{\pgfqpoint{4.248838in}{0.387680in}}%
\pgfpathlineto{\pgfqpoint{4.250986in}{0.385004in}}%
\pgfpathlineto{\pgfqpoint{4.252060in}{0.385428in}}%
\pgfpathlineto{\pgfqpoint{4.256355in}{0.388267in}}%
\pgfpathlineto{\pgfqpoint{4.257429in}{0.384820in}}%
\pgfpathlineto{\pgfqpoint{4.258503in}{0.385980in}}%
\pgfpathlineto{\pgfqpoint{4.259576in}{0.383591in}}%
\pgfpathlineto{\pgfqpoint{4.260650in}{0.384479in}}%
\pgfpathlineto{\pgfqpoint{4.263872in}{0.383912in}}%
\pgfpathlineto{\pgfqpoint{4.264946in}{0.386980in}}%
\pgfpathlineto{\pgfqpoint{4.266019in}{0.387546in}}%
\pgfpathlineto{\pgfqpoint{4.267093in}{0.386302in}}%
\pgfpathlineto{\pgfqpoint{4.268167in}{0.387109in}}%
\pgfpathlineto{\pgfqpoint{4.272462in}{0.388631in}}%
\pgfpathlineto{\pgfqpoint{4.274610in}{0.385058in}}%
\pgfpathlineto{\pgfqpoint{4.275684in}{0.385058in}}%
\pgfpathlineto{\pgfqpoint{4.278906in}{0.384347in}}%
\pgfpathlineto{\pgfqpoint{4.279979in}{0.386319in}}%
\pgfpathlineto{\pgfqpoint{4.282127in}{0.387815in}}%
\pgfpathlineto{\pgfqpoint{4.283201in}{0.390397in}}%
\pgfpathlineto{\pgfqpoint{4.286422in}{0.393120in}}%
\pgfpathlineto{\pgfqpoint{4.290718in}{0.387922in}}%
\pgfpathlineto{\pgfqpoint{4.293939in}{0.389055in}}%
\pgfpathlineto{\pgfqpoint{4.295013in}{0.388607in}}%
\pgfpathlineto{\pgfqpoint{4.296087in}{0.388799in}}%
\pgfpathlineto{\pgfqpoint{4.298235in}{0.385149in}}%
\pgfpathlineto{\pgfqpoint{4.302530in}{0.386814in}}%
\pgfpathlineto{\pgfqpoint{4.303604in}{0.386302in}}%
\pgfpathlineto{\pgfqpoint{4.304678in}{0.386974in}}%
\pgfpathlineto{\pgfqpoint{4.305751in}{0.386814in}}%
\pgfpathlineto{\pgfqpoint{4.311121in}{0.390423in}}%
\pgfpathlineto{\pgfqpoint{4.312195in}{0.391910in}}%
\pgfpathlineto{\pgfqpoint{4.313268in}{0.395725in}}%
\pgfpathlineto{\pgfqpoint{4.316490in}{0.395759in}}%
\pgfpathlineto{\pgfqpoint{4.317564in}{0.397094in}}%
\pgfpathlineto{\pgfqpoint{4.318638in}{0.394972in}}%
\pgfpathlineto{\pgfqpoint{4.319711in}{0.394673in}}%
\pgfpathlineto{\pgfqpoint{4.320785in}{0.392855in}}%
\pgfpathlineto{\pgfqpoint{4.324007in}{0.394142in}}%
\pgfpathlineto{\pgfqpoint{4.325081in}{0.395848in}}%
\pgfpathlineto{\pgfqpoint{4.326154in}{0.396275in}}%
\pgfpathlineto{\pgfqpoint{4.327228in}{0.399114in}}%
\pgfpathlineto{\pgfqpoint{4.328302in}{0.399411in}}%
\pgfpathlineto{\pgfqpoint{4.331524in}{0.399444in}}%
\pgfpathlineto{\pgfqpoint{4.332597in}{0.400671in}}%
\pgfpathlineto{\pgfqpoint{4.333671in}{0.400008in}}%
\pgfpathlineto{\pgfqpoint{4.335819in}{0.401226in}}%
\pgfpathlineto{\pgfqpoint{4.339040in}{0.398895in}}%
\pgfpathlineto{\pgfqpoint{4.340114in}{0.400693in}}%
\pgfpathlineto{\pgfqpoint{4.341188in}{0.398961in}}%
\pgfpathlineto{\pgfqpoint{4.342262in}{0.399124in}}%
\pgfpathlineto{\pgfqpoint{4.343336in}{0.401111in}}%
\pgfpathlineto{\pgfqpoint{4.346557in}{0.400460in}}%
\pgfpathlineto{\pgfqpoint{4.347631in}{0.400815in}}%
\pgfpathlineto{\pgfqpoint{4.350853in}{0.399192in}}%
\pgfpathlineto{\pgfqpoint{4.354074in}{0.399063in}}%
\pgfpathlineto{\pgfqpoint{4.355148in}{0.397213in}}%
\pgfpathlineto{\pgfqpoint{4.356222in}{0.396954in}}%
\pgfpathlineto{\pgfqpoint{4.357296in}{0.397343in}}%
\pgfpathlineto{\pgfqpoint{4.358370in}{0.395040in}}%
\pgfpathlineto{\pgfqpoint{4.361591in}{0.395981in}}%
\pgfpathlineto{\pgfqpoint{4.362665in}{0.400245in}}%
\pgfpathlineto{\pgfqpoint{4.363739in}{0.400245in}}%
\pgfpathlineto{\pgfqpoint{4.365886in}{0.398081in}}%
\pgfpathlineto{\pgfqpoint{4.369108in}{0.396331in}}%
\pgfpathlineto{\pgfqpoint{4.370182in}{0.396840in}}%
\pgfpathlineto{\pgfqpoint{4.371256in}{0.396458in}}%
\pgfpathlineto{\pgfqpoint{4.372329in}{0.392440in}}%
\pgfpathlineto{\pgfqpoint{4.373403in}{0.391813in}}%
\pgfpathlineto{\pgfqpoint{4.376625in}{0.392168in}}%
\pgfpathlineto{\pgfqpoint{4.378773in}{0.389522in}}%
\pgfpathlineto{\pgfqpoint{4.379846in}{0.389968in}}%
\pgfpathlineto{\pgfqpoint{4.380920in}{0.387857in}}%
\pgfpathlineto{\pgfqpoint{4.385216in}{0.388579in}}%
\pgfpathlineto{\pgfqpoint{4.386289in}{0.386785in}}%
\pgfpathlineto{\pgfqpoint{4.387363in}{0.388136in}}%
\pgfpathlineto{\pgfqpoint{4.388437in}{0.387331in}}%
\pgfpathlineto{\pgfqpoint{4.392732in}{0.388308in}}%
\pgfpathlineto{\pgfqpoint{4.393806in}{0.388829in}}%
\pgfpathlineto{\pgfqpoint{4.395954in}{0.387361in}}%
\pgfpathlineto{\pgfqpoint{4.399175in}{0.387390in}}%
\pgfpathlineto{\pgfqpoint{4.400249in}{0.385324in}}%
\pgfpathlineto{\pgfqpoint{4.401323in}{0.385380in}}%
\pgfpathlineto{\pgfqpoint{4.402397in}{0.384135in}}%
\pgfpathlineto{\pgfqpoint{4.403471in}{0.385126in}}%
\pgfpathlineto{\pgfqpoint{4.407766in}{0.384956in}}%
\pgfpathlineto{\pgfqpoint{4.410988in}{0.387725in}}%
\pgfpathlineto{\pgfqpoint{4.415283in}{0.388217in}}%
\pgfpathlineto{\pgfqpoint{4.417431in}{0.382613in}}%
\pgfpathlineto{\pgfqpoint{4.418505in}{0.382904in}}%
\pgfpathlineto{\pgfqpoint{4.422800in}{0.382478in}}%
\pgfpathlineto{\pgfqpoint{4.423874in}{0.382824in}}%
\pgfpathlineto{\pgfqpoint{4.424948in}{0.380667in}}%
\pgfpathlineto{\pgfqpoint{4.426021in}{0.381259in}}%
\pgfpathlineto{\pgfqpoint{4.430317in}{0.380220in}}%
\pgfpathlineto{\pgfqpoint{4.431391in}{0.382090in}}%
\pgfpathlineto{\pgfqpoint{4.436760in}{0.382818in}}%
\pgfpathlineto{\pgfqpoint{4.437834in}{0.383728in}}%
\pgfpathlineto{\pgfqpoint{4.438907in}{0.382303in}}%
\pgfpathlineto{\pgfqpoint{4.441055in}{0.384547in}}%
\pgfpathlineto{\pgfqpoint{4.444277in}{0.384040in}}%
\pgfpathlineto{\pgfqpoint{4.445350in}{0.382998in}}%
\pgfpathlineto{\pgfqpoint{4.446424in}{0.385001in}}%
\pgfpathlineto{\pgfqpoint{4.448572in}{0.383804in}}%
\pgfpathlineto{\pgfqpoint{4.451794in}{0.385324in}}%
\pgfpathlineto{\pgfqpoint{4.452867in}{0.383364in}}%
\pgfpathlineto{\pgfqpoint{4.453941in}{0.382639in}}%
\pgfpathlineto{\pgfqpoint{4.455015in}{0.380519in}}%
\pgfpathlineto{\pgfqpoint{4.456089in}{0.381136in}}%
\pgfpathlineto{\pgfqpoint{4.459310in}{0.380653in}}%
\pgfpathlineto{\pgfqpoint{4.460384in}{0.379587in}}%
\pgfpathlineto{\pgfqpoint{4.461458in}{0.377101in}}%
\pgfpathlineto{\pgfqpoint{4.462532in}{0.376800in}}%
\pgfpathlineto{\pgfqpoint{4.463606in}{0.377974in}}%
\pgfpathlineto{\pgfqpoint{4.466827in}{0.377719in}}%
\pgfpathlineto{\pgfqpoint{4.467901in}{0.375349in}}%
\pgfpathlineto{\pgfqpoint{4.468975in}{0.375094in}}%
\pgfpathlineto{\pgfqpoint{4.471123in}{0.373794in}}%
\pgfpathlineto{\pgfqpoint{4.476492in}{0.371576in}}%
\pgfpathlineto{\pgfqpoint{4.477566in}{0.373106in}}%
\pgfpathlineto{\pgfqpoint{4.481861in}{0.371885in}}%
\pgfpathlineto{\pgfqpoint{4.482935in}{0.371536in}}%
\pgfpathlineto{\pgfqpoint{4.484009in}{0.372404in}}%
\pgfpathlineto{\pgfqpoint{4.485083in}{0.372303in}}%
\pgfpathlineto{\pgfqpoint{4.486156in}{0.365554in}}%
\pgfpathlineto{\pgfqpoint{4.489378in}{0.366436in}}%
\pgfpathlineto{\pgfqpoint{4.490452in}{0.365277in}}%
\pgfpathlineto{\pgfqpoint{4.491526in}{0.367027in}}%
\pgfpathlineto{\pgfqpoint{4.493673in}{0.367842in}}%
\pgfpathlineto{\pgfqpoint{4.496895in}{0.368966in}}%
\pgfpathlineto{\pgfqpoint{4.499042in}{0.367191in}}%
\pgfpathlineto{\pgfqpoint{4.501190in}{0.365906in}}%
\pgfpathlineto{\pgfqpoint{4.504412in}{0.366412in}}%
\pgfpathlineto{\pgfqpoint{4.505485in}{0.365462in}}%
\pgfpathlineto{\pgfqpoint{4.506559in}{0.366778in}}%
\pgfpathlineto{\pgfqpoint{4.507633in}{0.365633in}}%
\pgfpathlineto{\pgfqpoint{4.508707in}{0.365752in}}%
\pgfpathlineto{\pgfqpoint{4.513002in}{0.364963in}}%
\pgfpathlineto{\pgfqpoint{4.514076in}{0.365609in}}%
\pgfpathlineto{\pgfqpoint{4.516224in}{0.364751in}}%
\pgfpathlineto{\pgfqpoint{4.520519in}{0.364235in}}%
\pgfpathlineto{\pgfqpoint{4.521593in}{0.363885in}}%
\pgfpathlineto{\pgfqpoint{4.522667in}{0.365019in}}%
\pgfpathlineto{\pgfqpoint{4.523741in}{0.364073in}}%
\pgfpathlineto{\pgfqpoint{4.526962in}{0.365398in}}%
\pgfpathlineto{\pgfqpoint{4.528036in}{0.363364in}}%
\pgfpathlineto{\pgfqpoint{4.529110in}{0.363956in}}%
\pgfpathlineto{\pgfqpoint{4.531258in}{0.368008in}}%
\pgfpathlineto{\pgfqpoint{4.537701in}{0.366981in}}%
\pgfpathlineto{\pgfqpoint{4.538774in}{0.366145in}}%
\pgfpathlineto{\pgfqpoint{4.541996in}{0.367530in}}%
\pgfpathlineto{\pgfqpoint{4.544144in}{0.364665in}}%
\pgfpathlineto{\pgfqpoint{4.545217in}{0.364438in}}%
\pgfpathlineto{\pgfqpoint{4.546291in}{0.362408in}}%
\pgfpathlineto{\pgfqpoint{4.549513in}{0.364839in}}%
\pgfpathlineto{\pgfqpoint{4.551661in}{0.364452in}}%
\pgfpathlineto{\pgfqpoint{4.559177in}{0.364157in}}%
\pgfpathlineto{\pgfqpoint{4.560251in}{0.362368in}}%
\pgfpathlineto{\pgfqpoint{4.565620in}{0.363256in}}%
\pgfpathlineto{\pgfqpoint{4.566694in}{0.360830in}}%
\pgfpathlineto{\pgfqpoint{4.567768in}{0.360830in}}%
\pgfpathlineto{\pgfqpoint{4.568842in}{0.361637in}}%
\pgfpathlineto{\pgfqpoint{4.572063in}{0.361678in}}%
\pgfpathlineto{\pgfqpoint{4.574211in}{0.360602in}}%
\pgfpathlineto{\pgfqpoint{4.575285in}{0.360788in}}%
\pgfpathlineto{\pgfqpoint{4.576359in}{0.359464in}}%
\pgfpathlineto{\pgfqpoint{4.581728in}{0.361044in}}%
\pgfpathlineto{\pgfqpoint{4.583876in}{0.361929in}}%
\pgfpathlineto{\pgfqpoint{4.590319in}{0.361231in}}%
\pgfpathlineto{\pgfqpoint{4.591393in}{0.362598in}}%
\pgfpathlineto{\pgfqpoint{4.595688in}{0.362374in}}%
\pgfpathlineto{\pgfqpoint{4.596762in}{0.364487in}}%
\pgfpathlineto{\pgfqpoint{4.597836in}{0.363661in}}%
\pgfpathlineto{\pgfqpoint{4.598909in}{0.363810in}}%
\pgfpathlineto{\pgfqpoint{4.603205in}{0.365687in}}%
\pgfpathlineto{\pgfqpoint{4.604279in}{0.364537in}}%
\pgfpathlineto{\pgfqpoint{4.605352in}{0.364325in}}%
\pgfpathlineto{\pgfqpoint{4.606426in}{0.362888in}}%
\pgfpathlineto{\pgfqpoint{4.609648in}{0.362003in}}%
\pgfpathlineto{\pgfqpoint{4.610722in}{0.360750in}}%
\pgfpathlineto{\pgfqpoint{4.612869in}{0.360907in}}%
\pgfpathlineto{\pgfqpoint{4.613943in}{0.362850in}}%
\pgfpathlineto{\pgfqpoint{4.617165in}{0.362374in}}%
\pgfpathlineto{\pgfqpoint{4.619312in}{0.363776in}}%
\pgfpathlineto{\pgfqpoint{4.620386in}{0.363036in}}%
\pgfpathlineto{\pgfqpoint{4.621460in}{0.363332in}}%
\pgfpathlineto{\pgfqpoint{4.625755in}{0.361029in}}%
\pgfpathlineto{\pgfqpoint{4.628977in}{0.364164in}}%
\pgfpathlineto{\pgfqpoint{4.634346in}{0.366317in}}%
\pgfpathlineto{\pgfqpoint{4.635420in}{0.366962in}}%
\pgfpathlineto{\pgfqpoint{4.639715in}{0.367170in}}%
\pgfpathlineto{\pgfqpoint{4.640789in}{0.368633in}}%
\pgfpathlineto{\pgfqpoint{4.644011in}{0.369472in}}%
\pgfpathlineto{\pgfqpoint{4.648306in}{0.366728in}}%
\pgfpathlineto{\pgfqpoint{4.650454in}{0.364180in}}%
\pgfpathlineto{\pgfqpoint{4.651527in}{0.364180in}}%
\pgfpathlineto{\pgfqpoint{4.654749in}{0.361269in}}%
\pgfpathlineto{\pgfqpoint{4.655823in}{0.364523in}}%
\pgfpathlineto{\pgfqpoint{4.656897in}{0.364656in}}%
\pgfpathlineto{\pgfqpoint{4.657971in}{0.363454in}}%
\pgfpathlineto{\pgfqpoint{4.659044in}{0.366201in}}%
\pgfpathlineto{\pgfqpoint{4.662266in}{0.365419in}}%
\pgfpathlineto{\pgfqpoint{4.663340in}{0.364423in}}%
\pgfpathlineto{\pgfqpoint{4.664414in}{0.365511in}}%
\pgfpathlineto{\pgfqpoint{4.666561in}{0.365379in}}%
\pgfpathlineto{\pgfqpoint{4.671930in}{0.368265in}}%
\pgfpathlineto{\pgfqpoint{4.674078in}{0.365552in}}%
\pgfpathlineto{\pgfqpoint{4.677300in}{0.365120in}}%
\pgfpathlineto{\pgfqpoint{4.678373in}{0.363476in}}%
\pgfpathlineto{\pgfqpoint{4.679447in}{0.363597in}}%
\pgfpathlineto{\pgfqpoint{4.680521in}{0.364674in}}%
\pgfpathlineto{\pgfqpoint{4.681595in}{0.363823in}}%
\pgfpathlineto{\pgfqpoint{4.686964in}{0.363092in}}%
\pgfpathlineto{\pgfqpoint{4.688038in}{0.364945in}}%
\pgfpathlineto{\pgfqpoint{4.689112in}{0.365120in}}%
\pgfpathlineto{\pgfqpoint{4.694481in}{0.363514in}}%
\pgfpathlineto{\pgfqpoint{4.695555in}{0.362972in}}%
\pgfpathlineto{\pgfqpoint{4.696629in}{0.363374in}}%
\pgfpathlineto{\pgfqpoint{4.700924in}{0.361453in}}%
\pgfpathlineto{\pgfqpoint{4.701998in}{0.362110in}}%
\pgfpathlineto{\pgfqpoint{4.704146in}{0.357149in}}%
\pgfpathlineto{\pgfqpoint{4.707367in}{0.358274in}}%
\pgfpathlineto{\pgfqpoint{4.708441in}{0.357422in}}%
\pgfpathlineto{\pgfqpoint{4.709515in}{0.359299in}}%
\pgfpathlineto{\pgfqpoint{4.710589in}{0.359316in}}%
\pgfpathlineto{\pgfqpoint{4.711662in}{0.362049in}}%
\pgfpathlineto{\pgfqpoint{4.715958in}{0.363614in}}%
\pgfpathlineto{\pgfqpoint{4.717032in}{0.362285in}}%
\pgfpathlineto{\pgfqpoint{4.718105in}{0.364422in}}%
\pgfpathlineto{\pgfqpoint{4.719179in}{0.365230in}}%
\pgfpathlineto{\pgfqpoint{4.722401in}{0.366380in}}%
\pgfpathlineto{\pgfqpoint{4.723475in}{0.365048in}}%
\pgfpathlineto{\pgfqpoint{4.724549in}{0.366614in}}%
\pgfpathlineto{\pgfqpoint{4.725622in}{0.362674in}}%
\pgfpathlineto{\pgfqpoint{4.726696in}{0.363938in}}%
\pgfpathlineto{\pgfqpoint{4.730992in}{0.364655in}}%
\pgfpathlineto{\pgfqpoint{4.733139in}{0.362730in}}%
\pgfpathlineto{\pgfqpoint{4.734213in}{0.363658in}}%
\pgfpathlineto{\pgfqpoint{4.737435in}{0.364965in}}%
\pgfpathlineto{\pgfqpoint{4.739582in}{0.366684in}}%
\pgfpathlineto{\pgfqpoint{4.741730in}{0.366313in}}%
\pgfpathlineto{\pgfqpoint{4.746025in}{0.364445in}}%
\pgfpathlineto{\pgfqpoint{4.747099in}{0.364498in}}%
\pgfpathlineto{\pgfqpoint{4.748173in}{0.367223in}}%
\pgfpathlineto{\pgfqpoint{4.749247in}{0.366564in}}%
\pgfpathlineto{\pgfqpoint{4.752468in}{0.368252in}}%
\pgfpathlineto{\pgfqpoint{4.753542in}{0.366746in}}%
\pgfpathlineto{\pgfqpoint{4.754616in}{0.367290in}}%
\pgfpathlineto{\pgfqpoint{4.755690in}{0.366528in}}%
\pgfpathlineto{\pgfqpoint{4.756764in}{0.367602in}}%
\pgfpathlineto{\pgfqpoint{4.761059in}{0.366415in}}%
\pgfpathlineto{\pgfqpoint{4.767502in}{0.368263in}}%
\pgfpathlineto{\pgfqpoint{4.768576in}{0.367541in}}%
\pgfpathlineto{\pgfqpoint{4.769650in}{0.368319in}}%
\pgfpathlineto{\pgfqpoint{4.771797in}{0.365194in}}%
\pgfpathlineto{\pgfqpoint{4.775019in}{0.365981in}}%
\pgfpathlineto{\pgfqpoint{4.777167in}{0.369496in}}%
\pgfpathlineto{\pgfqpoint{4.778240in}{0.369220in}}%
\pgfpathlineto{\pgfqpoint{4.779314in}{0.367512in}}%
\pgfpathlineto{\pgfqpoint{4.783610in}{0.366245in}}%
\pgfpathlineto{\pgfqpoint{4.784683in}{0.366209in}}%
\pgfpathlineto{\pgfqpoint{4.785757in}{0.363564in}}%
\pgfpathlineto{\pgfqpoint{4.790053in}{0.362646in}}%
\pgfpathlineto{\pgfqpoint{4.791126in}{0.362720in}}%
\pgfpathlineto{\pgfqpoint{4.792200in}{0.361471in}}%
\pgfpathlineto{\pgfqpoint{4.793274in}{0.362316in}}%
\pgfpathlineto{\pgfqpoint{4.794348in}{0.362169in}}%
\pgfpathlineto{\pgfqpoint{4.797570in}{0.362792in}}%
\pgfpathlineto{\pgfqpoint{4.798643in}{0.361661in}}%
\pgfpathlineto{\pgfqpoint{4.799717in}{0.361865in}}%
\pgfpathlineto{\pgfqpoint{4.800791in}{0.361402in}}%
\pgfpathlineto{\pgfqpoint{4.801865in}{0.361824in}}%
\pgfpathlineto{\pgfqpoint{4.807234in}{0.361676in}}%
\pgfpathlineto{\pgfqpoint{4.809382in}{0.361217in}}%
\pgfpathlineto{\pgfqpoint{4.812603in}{0.360870in}}%
\pgfpathlineto{\pgfqpoint{4.816899in}{0.352918in}}%
\pgfpathlineto{\pgfqpoint{4.822268in}{0.352670in}}%
\pgfpathlineto{\pgfqpoint{4.823342in}{0.354612in}}%
\pgfpathlineto{\pgfqpoint{4.824415in}{0.353576in}}%
\pgfpathlineto{\pgfqpoint{4.829785in}{0.352763in}}%
\pgfpathlineto{\pgfqpoint{4.830859in}{0.351615in}}%
\pgfpathlineto{\pgfqpoint{4.831932in}{0.351677in}}%
\pgfpathlineto{\pgfqpoint{4.835154in}{0.350681in}}%
\pgfpathlineto{\pgfqpoint{4.837302in}{0.351086in}}%
\pgfpathlineto{\pgfqpoint{4.838375in}{0.349857in}}%
\pgfpathlineto{\pgfqpoint{4.839449in}{0.351676in}}%
\pgfpathlineto{\pgfqpoint{4.842671in}{0.352687in}}%
\pgfpathlineto{\pgfqpoint{4.844818in}{0.351241in}}%
\pgfpathlineto{\pgfqpoint{4.845892in}{0.354134in}}%
\pgfpathlineto{\pgfqpoint{4.846966in}{0.353228in}}%
\pgfpathlineto{\pgfqpoint{4.850188in}{0.352339in}}%
\pgfpathlineto{\pgfqpoint{4.851261in}{0.352890in}}%
\pgfpathlineto{\pgfqpoint{4.853409in}{0.352859in}}%
\pgfpathlineto{\pgfqpoint{4.854483in}{0.353896in}}%
\pgfpathlineto{\pgfqpoint{4.857704in}{0.353246in}}%
\pgfpathlineto{\pgfqpoint{4.858778in}{0.351306in}}%
\pgfpathlineto{\pgfqpoint{4.860926in}{0.353757in}}%
\pgfpathlineto{\pgfqpoint{4.862000in}{0.353471in}}%
\pgfpathlineto{\pgfqpoint{4.866295in}{0.353724in}}%
\pgfpathlineto{\pgfqpoint{4.867369in}{0.353107in}}%
\pgfpathlineto{\pgfqpoint{4.868443in}{0.353273in}}%
\pgfpathlineto{\pgfqpoint{4.869517in}{0.352700in}}%
\pgfpathlineto{\pgfqpoint{4.874886in}{0.353223in}}%
\pgfpathlineto{\pgfqpoint{4.877034in}{0.357819in}}%
\pgfpathlineto{\pgfqpoint{4.880255in}{0.358167in}}%
\pgfpathlineto{\pgfqpoint{4.881329in}{0.357340in}}%
\pgfpathlineto{\pgfqpoint{4.882403in}{0.359089in}}%
\pgfpathlineto{\pgfqpoint{4.883477in}{0.357562in}}%
\pgfpathlineto{\pgfqpoint{4.884550in}{0.357593in}}%
\pgfpathlineto{\pgfqpoint{4.889920in}{0.356498in}}%
\pgfpathlineto{\pgfqpoint{4.892067in}{0.355132in}}%
\pgfpathlineto{\pgfqpoint{4.895289in}{0.355762in}}%
\pgfpathlineto{\pgfqpoint{4.896363in}{0.357769in}}%
\pgfpathlineto{\pgfqpoint{4.897437in}{0.358346in}}%
\pgfpathlineto{\pgfqpoint{4.899584in}{0.358023in}}%
\pgfpathlineto{\pgfqpoint{4.902806in}{0.358546in}}%
\pgfpathlineto{\pgfqpoint{4.903880in}{0.359961in}}%
\pgfpathlineto{\pgfqpoint{4.906027in}{0.360436in}}%
\pgfpathlineto{\pgfqpoint{4.907101in}{0.358664in}}%
\pgfpathlineto{\pgfqpoint{4.911396in}{0.359801in}}%
\pgfpathlineto{\pgfqpoint{4.912470in}{0.361282in}}%
\pgfpathlineto{\pgfqpoint{4.914618in}{0.361235in}}%
\pgfpathlineto{\pgfqpoint{4.917839in}{0.362538in}}%
\pgfpathlineto{\pgfqpoint{4.918913in}{0.362315in}}%
\pgfpathlineto{\pgfqpoint{4.921061in}{0.358978in}}%
\pgfpathlineto{\pgfqpoint{4.922135in}{0.359763in}}%
\pgfpathlineto{\pgfqpoint{4.925356in}{0.360983in}}%
\pgfpathlineto{\pgfqpoint{4.928578in}{0.359118in}}%
\pgfpathlineto{\pgfqpoint{4.929652in}{0.359480in}}%
\pgfpathlineto{\pgfqpoint{4.932873in}{0.362025in}}%
\pgfpathlineto{\pgfqpoint{4.933947in}{0.361806in}}%
\pgfpathlineto{\pgfqpoint{4.937169in}{0.361761in}}%
\pgfpathlineto{\pgfqpoint{4.940390in}{0.361381in}}%
\pgfpathlineto{\pgfqpoint{4.941464in}{0.362083in}}%
\pgfpathlineto{\pgfqpoint{4.942538in}{0.363602in}}%
\pgfpathlineto{\pgfqpoint{4.943612in}{0.363151in}}%
\pgfpathlineto{\pgfqpoint{4.944685in}{0.363843in}}%
\pgfpathlineto{\pgfqpoint{4.950055in}{0.368782in}}%
\pgfpathlineto{\pgfqpoint{4.956498in}{0.368670in}}%
\pgfpathlineto{\pgfqpoint{4.957571in}{0.367039in}}%
\pgfpathlineto{\pgfqpoint{4.958645in}{0.367784in}}%
\pgfpathlineto{\pgfqpoint{4.959719in}{0.364778in}}%
\pgfpathlineto{\pgfqpoint{4.962941in}{0.364086in}}%
\pgfpathlineto{\pgfqpoint{4.964014in}{0.364999in}}%
\pgfpathlineto{\pgfqpoint{4.967236in}{0.371059in}}%
\pgfpathlineto{\pgfqpoint{4.970458in}{0.367018in}}%
\pgfpathlineto{\pgfqpoint{4.971531in}{0.368654in}}%
\pgfpathlineto{\pgfqpoint{4.972605in}{0.367552in}}%
\pgfpathlineto{\pgfqpoint{4.973679in}{0.361791in}}%
\pgfpathlineto{\pgfqpoint{4.974753in}{0.361643in}}%
\pgfpathlineto{\pgfqpoint{4.980122in}{0.358650in}}%
\pgfpathlineto{\pgfqpoint{4.981196in}{0.358663in}}%
\pgfpathlineto{\pgfqpoint{4.982270in}{0.360366in}}%
\pgfpathlineto{\pgfqpoint{4.986565in}{0.359910in}}%
\pgfpathlineto{\pgfqpoint{4.988713in}{0.361234in}}%
\pgfpathlineto{\pgfqpoint{4.989787in}{0.360019in}}%
\pgfpathlineto{\pgfqpoint{4.993008in}{0.359269in}}%
\pgfpathlineto{\pgfqpoint{4.994082in}{0.359914in}}%
\pgfpathlineto{\pgfqpoint{4.995156in}{0.359750in}}%
\pgfpathlineto{\pgfqpoint{4.996230in}{0.360254in}}%
\pgfpathlineto{\pgfqpoint{4.997303in}{0.359832in}}%
\pgfpathlineto{\pgfqpoint{5.000525in}{0.361520in}}%
\pgfpathlineto{\pgfqpoint{5.001599in}{0.364424in}}%
\pgfpathlineto{\pgfqpoint{5.003747in}{0.362336in}}%
\pgfpathlineto{\pgfqpoint{5.004820in}{0.363745in}}%
\pgfpathlineto{\pgfqpoint{5.008042in}{0.363441in}}%
\pgfpathlineto{\pgfqpoint{5.009116in}{0.362269in}}%
\pgfpathlineto{\pgfqpoint{5.010190in}{0.362683in}}%
\pgfpathlineto{\pgfqpoint{5.011263in}{0.364530in}}%
\pgfpathlineto{\pgfqpoint{5.012337in}{0.364716in}}%
\pgfpathlineto{\pgfqpoint{5.016633in}{0.362199in}}%
\pgfpathlineto{\pgfqpoint{5.017706in}{0.364051in}}%
\pgfpathlineto{\pgfqpoint{5.019854in}{0.365040in}}%
\pgfpathlineto{\pgfqpoint{5.026297in}{0.362569in}}%
\pgfpathlineto{\pgfqpoint{5.027371in}{0.363240in}}%
\pgfpathlineto{\pgfqpoint{5.030592in}{0.361812in}}%
\pgfpathlineto{\pgfqpoint{5.031666in}{0.359881in}}%
\pgfpathlineto{\pgfqpoint{5.033814in}{0.359787in}}%
\pgfpathlineto{\pgfqpoint{5.034888in}{0.358225in}}%
\pgfpathlineto{\pgfqpoint{5.038109in}{0.356623in}}%
\pgfpathlineto{\pgfqpoint{5.039183in}{0.357170in}}%
\pgfpathlineto{\pgfqpoint{5.040257in}{0.356062in}}%
\pgfpathlineto{\pgfqpoint{5.041331in}{0.359639in}}%
\pgfpathlineto{\pgfqpoint{5.042405in}{0.358992in}}%
\pgfpathlineto{\pgfqpoint{5.046700in}{0.359723in}}%
\pgfpathlineto{\pgfqpoint{5.047774in}{0.361393in}}%
\pgfpathlineto{\pgfqpoint{5.048848in}{0.360032in}}%
\pgfpathlineto{\pgfqpoint{5.056365in}{0.361614in}}%
\pgfpathlineto{\pgfqpoint{5.057438in}{0.359856in}}%
\pgfpathlineto{\pgfqpoint{5.060660in}{0.359270in}}%
\pgfpathlineto{\pgfqpoint{5.061734in}{0.360386in}}%
\pgfpathlineto{\pgfqpoint{5.064955in}{0.358264in}}%
\pgfpathlineto{\pgfqpoint{5.068177in}{0.357127in}}%
\pgfpathlineto{\pgfqpoint{5.069251in}{0.357363in}}%
\pgfpathlineto{\pgfqpoint{5.070325in}{0.356456in}}%
\pgfpathlineto{\pgfqpoint{5.071398in}{0.358059in}}%
\pgfpathlineto{\pgfqpoint{5.072472in}{0.357639in}}%
\pgfpathlineto{\pgfqpoint{5.075694in}{0.357066in}}%
\pgfpathlineto{\pgfqpoint{5.076768in}{0.357478in}}%
\pgfpathlineto{\pgfqpoint{5.078915in}{0.355166in}}%
\pgfpathlineto{\pgfqpoint{5.079989in}{0.355452in}}%
\pgfpathlineto{\pgfqpoint{5.084284in}{0.354609in}}%
\pgfpathlineto{\pgfqpoint{5.085358in}{0.355898in}}%
\pgfpathlineto{\pgfqpoint{5.086432in}{0.355706in}}%
\pgfpathlineto{\pgfqpoint{5.087506in}{0.354562in}}%
\pgfpathlineto{\pgfqpoint{5.092875in}{0.354238in}}%
\pgfpathlineto{\pgfqpoint{5.095023in}{0.354756in}}%
\pgfpathlineto{\pgfqpoint{5.099318in}{0.355496in}}%
\pgfpathlineto{\pgfqpoint{5.100392in}{0.355907in}}%
\pgfpathlineto{\pgfqpoint{5.101466in}{0.355284in}}%
\pgfpathlineto{\pgfqpoint{5.105761in}{0.356481in}}%
\pgfpathlineto{\pgfqpoint{5.106835in}{0.357320in}}%
\pgfpathlineto{\pgfqpoint{5.107909in}{0.356818in}}%
\pgfpathlineto{\pgfqpoint{5.108983in}{0.355060in}}%
\pgfpathlineto{\pgfqpoint{5.110057in}{0.355060in}}%
\pgfpathlineto{\pgfqpoint{5.116500in}{0.356065in}}%
\pgfpathlineto{\pgfqpoint{5.117573in}{0.357169in}}%
\pgfpathlineto{\pgfqpoint{5.121869in}{0.357002in}}%
\pgfpathlineto{\pgfqpoint{5.122943in}{0.358451in}}%
\pgfpathlineto{\pgfqpoint{5.125090in}{0.356810in}}%
\pgfpathlineto{\pgfqpoint{5.128312in}{0.356301in}}%
\pgfpathlineto{\pgfqpoint{5.129386in}{0.355466in}}%
\pgfpathlineto{\pgfqpoint{5.136903in}{0.355401in}}%
\pgfpathlineto{\pgfqpoint{5.137976in}{0.356932in}}%
\pgfpathlineto{\pgfqpoint{5.140124in}{0.357369in}}%
\pgfpathlineto{\pgfqpoint{5.143346in}{0.357222in}}%
\pgfpathlineto{\pgfqpoint{5.146567in}{0.355250in}}%
\pgfpathlineto{\pgfqpoint{5.147641in}{0.356144in}}%
\pgfpathlineto{\pgfqpoint{5.151936in}{0.356770in}}%
\pgfpathlineto{\pgfqpoint{5.153010in}{0.356966in}}%
\pgfpathlineto{\pgfqpoint{5.155158in}{0.358482in}}%
\pgfpathlineto{\pgfqpoint{5.158379in}{0.359556in}}%
\pgfpathlineto{\pgfqpoint{5.161601in}{0.363411in}}%
\pgfpathlineto{\pgfqpoint{5.162675in}{0.362806in}}%
\pgfpathlineto{\pgfqpoint{5.166970in}{0.365661in}}%
\pgfpathlineto{\pgfqpoint{5.168044in}{0.365461in}}%
\pgfpathlineto{\pgfqpoint{5.169118in}{0.364131in}}%
\pgfpathlineto{\pgfqpoint{5.170191in}{0.365780in}}%
\pgfpathlineto{\pgfqpoint{5.173413in}{0.364982in}}%
\pgfpathlineto{\pgfqpoint{5.175561in}{0.365811in}}%
\pgfpathlineto{\pgfqpoint{5.177708in}{0.363772in}}%
\pgfpathlineto{\pgfqpoint{5.183078in}{0.362535in}}%
\pgfpathlineto{\pgfqpoint{5.184151in}{0.363158in}}%
\pgfpathlineto{\pgfqpoint{5.190594in}{0.362897in}}%
\pgfpathlineto{\pgfqpoint{5.191668in}{0.363058in}}%
\pgfpathlineto{\pgfqpoint{5.192742in}{0.362646in}}%
\pgfpathlineto{\pgfqpoint{5.199185in}{0.364491in}}%
\pgfpathlineto{\pgfqpoint{5.200259in}{0.365428in}}%
\pgfpathlineto{\pgfqpoint{5.203480in}{0.365871in}}%
\pgfpathlineto{\pgfqpoint{5.204554in}{0.364735in}}%
\pgfpathlineto{\pgfqpoint{5.205628in}{0.364645in}}%
\pgfpathlineto{\pgfqpoint{5.207776in}{0.365539in}}%
\pgfpathlineto{\pgfqpoint{5.212071in}{0.365809in}}%
\pgfpathlineto{\pgfqpoint{5.215293in}{0.365924in}}%
\pgfpathlineto{\pgfqpoint{5.218514in}{0.367343in}}%
\pgfpathlineto{\pgfqpoint{5.219588in}{0.366149in}}%
\pgfpathlineto{\pgfqpoint{5.222810in}{0.370060in}}%
\pgfpathlineto{\pgfqpoint{5.227105in}{0.370209in}}%
\pgfpathlineto{\pgfqpoint{5.228179in}{0.372141in}}%
\pgfpathlineto{\pgfqpoint{5.229253in}{0.371232in}}%
\pgfpathlineto{\pgfqpoint{5.230326in}{0.373387in}}%
\pgfpathlineto{\pgfqpoint{5.233548in}{0.373247in}}%
\pgfpathlineto{\pgfqpoint{5.234622in}{0.371376in}}%
\pgfpathlineto{\pgfqpoint{5.235696in}{0.372660in}}%
\pgfpathlineto{\pgfqpoint{5.236769in}{0.375759in}}%
\pgfpathlineto{\pgfqpoint{5.237843in}{0.376525in}}%
\pgfpathlineto{\pgfqpoint{5.242139in}{0.380065in}}%
\pgfpathlineto{\pgfqpoint{5.243213in}{0.383038in}}%
\pgfpathlineto{\pgfqpoint{5.244286in}{0.382532in}}%
\pgfpathlineto{\pgfqpoint{5.245360in}{0.385687in}}%
\pgfpathlineto{\pgfqpoint{5.248582in}{0.383120in}}%
\pgfpathlineto{\pgfqpoint{5.249656in}{0.380184in}}%
\pgfpathlineto{\pgfqpoint{5.251803in}{0.382039in}}%
\pgfpathlineto{\pgfqpoint{5.252877in}{0.381161in}}%
\pgfpathlineto{\pgfqpoint{5.256099in}{0.381385in}}%
\pgfpathlineto{\pgfqpoint{5.257172in}{0.383436in}}%
\pgfpathlineto{\pgfqpoint{5.259320in}{0.384956in}}%
\pgfpathlineto{\pgfqpoint{5.260394in}{0.383340in}}%
\pgfpathlineto{\pgfqpoint{5.263615in}{0.382884in}}%
\pgfpathlineto{\pgfqpoint{5.265763in}{0.385628in}}%
\pgfpathlineto{\pgfqpoint{5.266837in}{0.384228in}}%
\pgfpathlineto{\pgfqpoint{5.267911in}{0.386967in}}%
\pgfpathlineto{\pgfqpoint{5.272206in}{0.386952in}}%
\pgfpathlineto{\pgfqpoint{5.273280in}{0.387075in}}%
\pgfpathlineto{\pgfqpoint{5.274354in}{0.387922in}}%
\pgfpathlineto{\pgfqpoint{5.275428in}{0.387475in}}%
\pgfpathlineto{\pgfqpoint{5.279723in}{0.387856in}}%
\pgfpathlineto{\pgfqpoint{5.280797in}{0.387208in}}%
\pgfpathlineto{\pgfqpoint{5.281871in}{0.385895in}}%
\pgfpathlineto{\pgfqpoint{5.282945in}{0.385973in}}%
\pgfpathlineto{\pgfqpoint{5.286166in}{0.388289in}}%
\pgfpathlineto{\pgfqpoint{5.287240in}{0.386760in}}%
\pgfpathlineto{\pgfqpoint{5.290461in}{0.385559in}}%
\pgfpathlineto{\pgfqpoint{5.293683in}{0.385307in}}%
\pgfpathlineto{\pgfqpoint{5.294757in}{0.384246in}}%
\pgfpathlineto{\pgfqpoint{5.296904in}{0.384914in}}%
\pgfpathlineto{\pgfqpoint{5.297978in}{0.384198in}}%
\pgfpathlineto{\pgfqpoint{5.301200in}{0.384155in}}%
\pgfpathlineto{\pgfqpoint{5.303347in}{0.381253in}}%
\pgfpathlineto{\pgfqpoint{5.304421in}{0.382971in}}%
\pgfpathlineto{\pgfqpoint{5.305495in}{0.383317in}}%
\pgfpathlineto{\pgfqpoint{5.308717in}{0.382519in}}%
\pgfpathlineto{\pgfqpoint{5.309791in}{0.381518in}}%
\pgfpathlineto{\pgfqpoint{5.310864in}{0.383172in}}%
\pgfpathlineto{\pgfqpoint{5.311938in}{0.381503in}}%
\pgfpathlineto{\pgfqpoint{5.313012in}{0.381136in}}%
\pgfpathlineto{\pgfqpoint{5.317307in}{0.378780in}}%
\pgfpathlineto{\pgfqpoint{5.320529in}{0.379892in}}%
\pgfpathlineto{\pgfqpoint{5.323750in}{0.379294in}}%
\pgfpathlineto{\pgfqpoint{5.324824in}{0.380304in}}%
\pgfpathlineto{\pgfqpoint{5.325898in}{0.379647in}}%
\pgfpathlineto{\pgfqpoint{5.328046in}{0.379086in}}%
\pgfpathlineto{\pgfqpoint{5.332341in}{0.376705in}}%
\pgfpathlineto{\pgfqpoint{5.333415in}{0.377909in}}%
\pgfpathlineto{\pgfqpoint{5.335563in}{0.376101in}}%
\pgfpathlineto{\pgfqpoint{5.338784in}{0.375836in}}%
\pgfpathlineto{\pgfqpoint{5.339858in}{0.374937in}}%
\pgfpathlineto{\pgfqpoint{5.340932in}{0.376108in}}%
\pgfpathlineto{\pgfqpoint{5.342006in}{0.376254in}}%
\pgfpathlineto{\pgfqpoint{5.343079in}{0.377290in}}%
\pgfpathlineto{\pgfqpoint{5.347375in}{0.377834in}}%
\pgfpathlineto{\pgfqpoint{5.348449in}{0.377111in}}%
\pgfpathlineto{\pgfqpoint{5.350596in}{0.378879in}}%
\pgfpathlineto{\pgfqpoint{5.353818in}{0.379190in}}%
\pgfpathlineto{\pgfqpoint{5.354892in}{0.380448in}}%
\pgfpathlineto{\pgfqpoint{5.355966in}{0.378763in}}%
\pgfpathlineto{\pgfqpoint{5.357039in}{0.379468in}}%
\pgfpathlineto{\pgfqpoint{5.358113in}{0.382394in}}%
\pgfpathlineto{\pgfqpoint{5.361335in}{0.383713in}}%
\pgfpathlineto{\pgfqpoint{5.362409in}{0.380757in}}%
\pgfpathlineto{\pgfqpoint{5.363482in}{0.371455in}}%
\pgfpathlineto{\pgfqpoint{5.364556in}{0.369948in}}%
\pgfpathlineto{\pgfqpoint{5.365630in}{0.370618in}}%
\pgfpathlineto{\pgfqpoint{5.370999in}{0.370571in}}%
\pgfpathlineto{\pgfqpoint{5.372073in}{0.372455in}}%
\pgfpathlineto{\pgfqpoint{5.373147in}{0.371812in}}%
\pgfpathlineto{\pgfqpoint{5.384959in}{0.372214in}}%
\pgfpathlineto{\pgfqpoint{5.387107in}{0.374271in}}%
\pgfpathlineto{\pgfqpoint{5.388181in}{0.373741in}}%
\pgfpathlineto{\pgfqpoint{5.391402in}{0.375731in}}%
\pgfpathlineto{\pgfqpoint{5.392476in}{0.375225in}}%
\pgfpathlineto{\pgfqpoint{5.393550in}{0.376077in}}%
\pgfpathlineto{\pgfqpoint{5.394624in}{0.375485in}}%
\pgfpathlineto{\pgfqpoint{5.395698in}{0.374134in}}%
\pgfpathlineto{\pgfqpoint{5.398919in}{0.375734in}}%
\pgfpathlineto{\pgfqpoint{5.399993in}{0.378139in}}%
\pgfpathlineto{\pgfqpoint{5.401067in}{0.377904in}}%
\pgfpathlineto{\pgfqpoint{5.402141in}{0.382855in}}%
\pgfpathlineto{\pgfqpoint{5.403214in}{0.383694in}}%
\pgfpathlineto{\pgfqpoint{5.407510in}{0.385258in}}%
\pgfpathlineto{\pgfqpoint{5.410731in}{0.388412in}}%
\pgfpathlineto{\pgfqpoint{5.413953in}{0.388654in}}%
\pgfpathlineto{\pgfqpoint{5.415027in}{0.387323in}}%
\pgfpathlineto{\pgfqpoint{5.416101in}{0.384789in}}%
\pgfpathlineto{\pgfqpoint{5.417174in}{0.385071in}}%
\pgfpathlineto{\pgfqpoint{5.421470in}{0.384012in}}%
\pgfpathlineto{\pgfqpoint{5.423617in}{0.386142in}}%
\pgfpathlineto{\pgfqpoint{5.424691in}{0.385459in}}%
\pgfpathlineto{\pgfqpoint{5.430060in}{0.385661in}}%
\pgfpathlineto{\pgfqpoint{5.431134in}{0.386778in}}%
\pgfpathlineto{\pgfqpoint{5.433282in}{0.386061in}}%
\pgfpathlineto{\pgfqpoint{5.438651in}{0.386389in}}%
\pgfpathlineto{\pgfqpoint{5.439725in}{0.385054in}}%
\pgfpathlineto{\pgfqpoint{5.440799in}{0.386698in}}%
\pgfpathlineto{\pgfqpoint{5.446168in}{0.389501in}}%
\pgfpathlineto{\pgfqpoint{5.448316in}{0.387576in}}%
\pgfpathlineto{\pgfqpoint{5.452611in}{0.387122in}}%
\pgfpathlineto{\pgfqpoint{5.453685in}{0.386997in}}%
\pgfpathlineto{\pgfqpoint{5.454759in}{0.385727in}}%
\pgfpathlineto{\pgfqpoint{5.455833in}{0.387819in}}%
\pgfpathlineto{\pgfqpoint{5.460128in}{0.387468in}}%
\pgfpathlineto{\pgfqpoint{5.462276in}{0.385660in}}%
\pgfpathlineto{\pgfqpoint{5.463349in}{0.386651in}}%
\pgfpathlineto{\pgfqpoint{5.469792in}{0.390103in}}%
\pgfpathlineto{\pgfqpoint{5.470866in}{0.389657in}}%
\pgfpathlineto{\pgfqpoint{5.475162in}{0.389884in}}%
\pgfpathlineto{\pgfqpoint{5.478383in}{0.390696in}}%
\pgfpathlineto{\pgfqpoint{5.482679in}{0.389714in}}%
\pgfpathlineto{\pgfqpoint{5.483752in}{0.390824in}}%
\pgfpathlineto{\pgfqpoint{5.485900in}{0.390823in}}%
\pgfpathlineto{\pgfqpoint{5.491269in}{0.392292in}}%
\pgfpathlineto{\pgfqpoint{5.492343in}{0.393330in}}%
\pgfpathlineto{\pgfqpoint{5.493417in}{0.393330in}}%
\pgfpathlineto{\pgfqpoint{5.497712in}{0.391157in}}%
\pgfpathlineto{\pgfqpoint{5.498786in}{0.392965in}}%
\pgfpathlineto{\pgfqpoint{5.499860in}{0.392211in}}%
\pgfpathlineto{\pgfqpoint{5.500934in}{0.393021in}}%
\pgfpathlineto{\pgfqpoint{5.504155in}{0.393295in}}%
\pgfpathlineto{\pgfqpoint{5.507377in}{0.389642in}}%
\pgfpathlineto{\pgfqpoint{5.508451in}{0.390332in}}%
\pgfpathlineto{\pgfqpoint{5.511672in}{0.391404in}}%
\pgfpathlineto{\pgfqpoint{5.512746in}{0.390382in}}%
\pgfpathlineto{\pgfqpoint{5.514894in}{0.391912in}}%
\pgfpathlineto{\pgfqpoint{5.515967in}{0.391292in}}%
\pgfpathlineto{\pgfqpoint{5.520263in}{0.394812in}}%
\pgfpathlineto{\pgfqpoint{5.521337in}{0.392842in}}%
\pgfpathlineto{\pgfqpoint{5.526706in}{0.393882in}}%
\pgfpathlineto{\pgfqpoint{5.527780in}{0.392248in}}%
\pgfpathlineto{\pgfqpoint{5.528854in}{0.394423in}}%
\pgfpathlineto{\pgfqpoint{5.529927in}{0.394910in}}%
\pgfpathlineto{\pgfqpoint{5.531001in}{0.393923in}}%
\pgfpathlineto{\pgfqpoint{5.534223in}{0.392053in}}%
\pgfpathlineto{\pgfqpoint{5.536370in}{0.392785in}}%
\pgfpathlineto{\pgfqpoint{5.537444in}{0.391908in}}%
\pgfpathlineto{\pgfqpoint{5.543887in}{0.393264in}}%
\pgfpathlineto{\pgfqpoint{5.544961in}{0.392666in}}%
\pgfpathlineto{\pgfqpoint{5.546035in}{0.392798in}}%
\pgfpathlineto{\pgfqpoint{5.550330in}{0.392284in}}%
\pgfpathlineto{\pgfqpoint{5.551404in}{0.390718in}}%
\pgfpathlineto{\pgfqpoint{5.553552in}{0.390543in}}%
\pgfpathlineto{\pgfqpoint{5.556773in}{0.391766in}}%
\pgfpathlineto{\pgfqpoint{5.557847in}{0.391137in}}%
\pgfpathlineto{\pgfqpoint{5.558921in}{0.403268in}}%
\pgfpathlineto{\pgfqpoint{5.559995in}{0.400676in}}%
\pgfpathlineto{\pgfqpoint{5.561069in}{0.401715in}}%
\pgfpathlineto{\pgfqpoint{5.564290in}{0.399554in}}%
\pgfpathlineto{\pgfqpoint{5.565364in}{0.403359in}}%
\pgfpathlineto{\pgfqpoint{5.566438in}{0.402010in}}%
\pgfpathlineto{\pgfqpoint{5.571807in}{0.405746in}}%
\pgfpathlineto{\pgfqpoint{5.572881in}{0.408589in}}%
\pgfpathlineto{\pgfqpoint{5.573955in}{0.407906in}}%
\pgfpathlineto{\pgfqpoint{5.575029in}{0.398891in}}%
\pgfpathlineto{\pgfqpoint{5.579324in}{0.392371in}}%
\pgfpathlineto{\pgfqpoint{5.580398in}{0.393113in}}%
\pgfpathlineto{\pgfqpoint{5.582545in}{0.384598in}}%
\pgfpathlineto{\pgfqpoint{5.583619in}{0.384219in}}%
\pgfpathlineto{\pgfqpoint{5.586841in}{0.384946in}}%
\pgfpathlineto{\pgfqpoint{5.587915in}{0.382005in}}%
\pgfpathlineto{\pgfqpoint{5.588989in}{0.384959in}}%
\pgfpathlineto{\pgfqpoint{5.590062in}{0.384841in}}%
\pgfpathlineto{\pgfqpoint{5.591136in}{0.383583in}}%
\pgfpathlineto{\pgfqpoint{5.595432in}{0.387344in}}%
\pgfpathlineto{\pgfqpoint{5.596505in}{0.389939in}}%
\pgfpathlineto{\pgfqpoint{5.597579in}{0.390825in}}%
\pgfpathlineto{\pgfqpoint{5.598653in}{0.388410in}}%
\pgfpathlineto{\pgfqpoint{5.601875in}{0.389226in}}%
\pgfpathlineto{\pgfqpoint{5.602948in}{0.388733in}}%
\pgfpathlineto{\pgfqpoint{5.604022in}{0.389399in}}%
\pgfpathlineto{\pgfqpoint{5.605096in}{0.389093in}}%
\pgfpathlineto{\pgfqpoint{5.606170in}{0.390803in}}%
\pgfpathlineto{\pgfqpoint{5.609391in}{0.391538in}}%
\pgfpathlineto{\pgfqpoint{5.610465in}{0.392722in}}%
\pgfpathlineto{\pgfqpoint{5.613687in}{0.389868in}}%
\pgfpathlineto{\pgfqpoint{5.616908in}{0.387499in}}%
\pgfpathlineto{\pgfqpoint{5.617982in}{0.388718in}}%
\pgfpathlineto{\pgfqpoint{5.619056in}{0.385088in}}%
\pgfpathlineto{\pgfqpoint{5.621204in}{0.384109in}}%
\pgfpathlineto{\pgfqpoint{5.626573in}{0.382708in}}%
\pgfpathlineto{\pgfqpoint{5.627647in}{0.384169in}}%
\pgfpathlineto{\pgfqpoint{5.628721in}{0.383025in}}%
\pgfpathlineto{\pgfqpoint{5.633016in}{0.381819in}}%
\pgfpathlineto{\pgfqpoint{5.634090in}{0.382811in}}%
\pgfpathlineto{\pgfqpoint{5.635164in}{0.385348in}}%
\pgfpathlineto{\pgfqpoint{5.641607in}{0.382805in}}%
\pgfpathlineto{\pgfqpoint{5.642680in}{0.379246in}}%
\pgfpathlineto{\pgfqpoint{5.648050in}{0.379619in}}%
\pgfpathlineto{\pgfqpoint{5.650197in}{0.378275in}}%
\pgfpathlineto{\pgfqpoint{5.651271in}{0.379629in}}%
\pgfpathlineto{\pgfqpoint{5.654493in}{0.381015in}}%
\pgfpathlineto{\pgfqpoint{5.655567in}{0.380486in}}%
\pgfpathlineto{\pgfqpoint{5.656640in}{0.382898in}}%
\pgfpathlineto{\pgfqpoint{5.657714in}{0.382619in}}%
\pgfpathlineto{\pgfqpoint{5.658788in}{0.385547in}}%
\pgfpathlineto{\pgfqpoint{5.662010in}{0.384722in}}%
\pgfpathlineto{\pgfqpoint{5.663083in}{0.383644in}}%
\pgfpathlineto{\pgfqpoint{5.664157in}{0.384602in}}%
\pgfpathlineto{\pgfqpoint{5.665231in}{0.384270in}}%
\pgfpathlineto{\pgfqpoint{5.666305in}{0.382793in}}%
\pgfpathlineto{\pgfqpoint{5.669526in}{0.383961in}}%
\pgfpathlineto{\pgfqpoint{5.670600in}{0.383732in}}%
\pgfpathlineto{\pgfqpoint{5.671674in}{0.381576in}}%
\pgfpathlineto{\pgfqpoint{5.672748in}{0.380981in}}%
\pgfpathlineto{\pgfqpoint{5.673822in}{0.379584in}}%
\pgfpathlineto{\pgfqpoint{5.677043in}{0.380235in}}%
\pgfpathlineto{\pgfqpoint{5.678117in}{0.378736in}}%
\pgfpathlineto{\pgfqpoint{5.679191in}{0.379469in}}%
\pgfpathlineto{\pgfqpoint{5.681339in}{0.383071in}}%
\pgfpathlineto{\pgfqpoint{5.684560in}{0.381285in}}%
\pgfpathlineto{\pgfqpoint{5.686708in}{0.384856in}}%
\pgfpathlineto{\pgfqpoint{5.687782in}{0.382523in}}%
\pgfpathlineto{\pgfqpoint{5.688856in}{0.385114in}}%
\pgfpathlineto{\pgfqpoint{5.692077in}{0.385559in}}%
\pgfpathlineto{\pgfqpoint{5.694225in}{0.382926in}}%
\pgfpathlineto{\pgfqpoint{5.695299in}{0.382926in}}%
\pgfpathlineto{\pgfqpoint{5.696372in}{0.379880in}}%
\pgfpathlineto{\pgfqpoint{5.699594in}{0.381344in}}%
\pgfpathlineto{\pgfqpoint{5.701742in}{0.376453in}}%
\pgfpathlineto{\pgfqpoint{5.702815in}{0.378294in}}%
\pgfpathlineto{\pgfqpoint{5.703889in}{0.373720in}}%
\pgfpathlineto{\pgfqpoint{5.708185in}{0.372675in}}%
\pgfpathlineto{\pgfqpoint{5.709258in}{0.373933in}}%
\pgfpathlineto{\pgfqpoint{5.710332in}{0.374255in}}%
\pgfpathlineto{\pgfqpoint{5.716775in}{0.372131in}}%
\pgfpathlineto{\pgfqpoint{5.717849in}{0.370784in}}%
\pgfpathlineto{\pgfqpoint{5.722144in}{0.368550in}}%
\pgfpathlineto{\pgfqpoint{5.723218in}{0.366327in}}%
\pgfpathlineto{\pgfqpoint{5.726440in}{0.364575in}}%
\pgfpathlineto{\pgfqpoint{5.729661in}{0.365282in}}%
\pgfpathlineto{\pgfqpoint{5.730735in}{0.363632in}}%
\pgfpathlineto{\pgfqpoint{5.731809in}{0.366712in}}%
\pgfpathlineto{\pgfqpoint{5.732883in}{0.367393in}}%
\pgfpathlineto{\pgfqpoint{5.733957in}{0.373096in}}%
\pgfpathlineto{\pgfqpoint{5.738252in}{0.373186in}}%
\pgfpathlineto{\pgfqpoint{5.739326in}{0.371442in}}%
\pgfpathlineto{\pgfqpoint{5.740400in}{0.373237in}}%
\pgfpathlineto{\pgfqpoint{5.741474in}{0.369724in}}%
\pgfpathlineto{\pgfqpoint{5.745769in}{0.372728in}}%
\pgfpathlineto{\pgfqpoint{5.746843in}{0.375035in}}%
\pgfpathlineto{\pgfqpoint{5.747917in}{0.374061in}}%
\pgfpathlineto{\pgfqpoint{5.748990in}{0.376879in}}%
\pgfpathlineto{\pgfqpoint{5.752212in}{0.377697in}}%
\pgfpathlineto{\pgfqpoint{5.753286in}{0.375156in}}%
\pgfpathlineto{\pgfqpoint{5.754360in}{0.377683in}}%
\pgfpathlineto{\pgfqpoint{5.755433in}{0.377328in}}%
\pgfpathlineto{\pgfqpoint{5.756507in}{0.379225in}}%
\pgfpathlineto{\pgfqpoint{5.759729in}{0.376942in}}%
\pgfpathlineto{\pgfqpoint{5.760803in}{0.377199in}}%
\pgfpathlineto{\pgfqpoint{5.761877in}{0.381644in}}%
\pgfpathlineto{\pgfqpoint{5.762950in}{0.383654in}}%
\pgfpathlineto{\pgfqpoint{5.764024in}{0.382489in}}%
\pgfpathlineto{\pgfqpoint{5.768320in}{0.380101in}}%
\pgfpathlineto{\pgfqpoint{5.769393in}{0.376718in}}%
\pgfpathlineto{\pgfqpoint{5.770467in}{0.377121in}}%
\pgfpathlineto{\pgfqpoint{5.771541in}{0.376911in}}%
\pgfpathlineto{\pgfqpoint{5.774763in}{0.378601in}}%
\pgfpathlineto{\pgfqpoint{5.775836in}{0.379832in}}%
\pgfpathlineto{\pgfqpoint{5.777984in}{0.380180in}}%
\pgfpathlineto{\pgfqpoint{5.779058in}{0.380608in}}%
\pgfpathlineto{\pgfqpoint{5.782279in}{0.380877in}}%
\pgfpathlineto{\pgfqpoint{5.783353in}{0.378157in}}%
\pgfpathlineto{\pgfqpoint{5.784427in}{0.378955in}}%
\pgfpathlineto{\pgfqpoint{5.785501in}{0.381213in}}%
\pgfpathlineto{\pgfqpoint{5.789796in}{0.382761in}}%
\pgfpathlineto{\pgfqpoint{5.790870in}{0.384719in}}%
\pgfpathlineto{\pgfqpoint{5.791944in}{0.384515in}}%
\pgfpathlineto{\pgfqpoint{5.793018in}{0.383715in}}%
\pgfpathlineto{\pgfqpoint{5.794092in}{0.384868in}}%
\pgfpathlineto{\pgfqpoint{5.797313in}{0.383742in}}%
\pgfpathlineto{\pgfqpoint{5.799461in}{0.384718in}}%
\pgfpathlineto{\pgfqpoint{5.800535in}{0.383235in}}%
\pgfpathlineto{\pgfqpoint{5.801609in}{0.383736in}}%
\pgfpathlineto{\pgfqpoint{5.804830in}{0.382807in}}%
\pgfpathlineto{\pgfqpoint{5.806978in}{0.380750in}}%
\pgfpathlineto{\pgfqpoint{5.808052in}{0.381241in}}%
\pgfpathlineto{\pgfqpoint{5.813421in}{0.380061in}}%
\pgfpathlineto{\pgfqpoint{5.814495in}{0.380999in}}%
\pgfpathlineto{\pgfqpoint{5.815568in}{0.380517in}}%
\pgfpathlineto{\pgfqpoint{5.816642in}{0.380801in}}%
\pgfpathlineto{\pgfqpoint{5.819864in}{0.380321in}}%
\pgfpathlineto{\pgfqpoint{5.820938in}{0.378245in}}%
\pgfpathlineto{\pgfqpoint{5.822011in}{0.378842in}}%
\pgfpathlineto{\pgfqpoint{5.823085in}{0.380464in}}%
\pgfpathlineto{\pgfqpoint{5.824159in}{0.380784in}}%
\pgfpathlineto{\pgfqpoint{5.827381in}{0.380970in}}%
\pgfpathlineto{\pgfqpoint{5.828455in}{0.382334in}}%
\pgfpathlineto{\pgfqpoint{5.829528in}{0.379620in}}%
\pgfpathlineto{\pgfqpoint{5.830602in}{0.380656in}}%
\pgfpathlineto{\pgfqpoint{5.831676in}{0.380604in}}%
\pgfpathlineto{\pgfqpoint{5.834898in}{0.384171in}}%
\pgfpathlineto{\pgfqpoint{5.837045in}{0.381976in}}%
\pgfpathlineto{\pgfqpoint{5.838119in}{0.382403in}}%
\pgfpathlineto{\pgfqpoint{5.839193in}{0.383440in}}%
\pgfpathlineto{\pgfqpoint{5.844562in}{0.381647in}}%
\pgfpathlineto{\pgfqpoint{5.845636in}{0.383110in}}%
\pgfpathlineto{\pgfqpoint{5.846710in}{0.382191in}}%
\pgfpathlineto{\pgfqpoint{5.852079in}{0.384065in}}%
\pgfpathlineto{\pgfqpoint{5.853153in}{0.385593in}}%
\pgfpathlineto{\pgfqpoint{5.854227in}{0.384854in}}%
\pgfpathlineto{\pgfqpoint{5.857448in}{0.385093in}}%
\pgfpathlineto{\pgfqpoint{5.858522in}{0.386596in}}%
\pgfpathlineto{\pgfqpoint{5.859596in}{0.391763in}}%
\pgfpathlineto{\pgfqpoint{5.860670in}{0.391009in}}%
\pgfpathlineto{\pgfqpoint{5.861744in}{0.389462in}}%
\pgfpathlineto{\pgfqpoint{5.866039in}{0.388708in}}%
\pgfpathlineto{\pgfqpoint{5.868187in}{0.386723in}}%
\pgfpathlineto{\pgfqpoint{5.869260in}{0.388503in}}%
\pgfpathlineto{\pgfqpoint{5.874630in}{0.389242in}}%
\pgfpathlineto{\pgfqpoint{5.876777in}{0.389931in}}%
\pgfpathlineto{\pgfqpoint{5.881073in}{0.391322in}}%
\pgfpathlineto{\pgfqpoint{5.882146in}{0.390379in}}%
\pgfpathlineto{\pgfqpoint{5.884294in}{0.390605in}}%
\pgfpathlineto{\pgfqpoint{5.887516in}{0.390647in}}%
\pgfpathlineto{\pgfqpoint{5.891811in}{0.388720in}}%
\pgfpathlineto{\pgfqpoint{5.895032in}{0.389030in}}%
\pgfpathlineto{\pgfqpoint{5.896106in}{0.387919in}}%
\pgfpathlineto{\pgfqpoint{5.898254in}{0.388223in}}%
\pgfpathlineto{\pgfqpoint{5.899328in}{0.387405in}}%
\pgfpathlineto{\pgfqpoint{5.902549in}{0.386666in}}%
\pgfpathlineto{\pgfqpoint{5.903623in}{0.387627in}}%
\pgfpathlineto{\pgfqpoint{5.905771in}{0.387888in}}%
\pgfpathlineto{\pgfqpoint{5.906845in}{0.392187in}}%
\pgfpathlineto{\pgfqpoint{5.910066in}{0.390322in}}%
\pgfpathlineto{\pgfqpoint{5.911140in}{0.392642in}}%
\pgfpathlineto{\pgfqpoint{5.913288in}{0.390198in}}%
\pgfpathlineto{\pgfqpoint{5.914362in}{0.389915in}}%
\pgfpathlineto{\pgfqpoint{5.918657in}{0.390408in}}%
\pgfpathlineto{\pgfqpoint{5.919731in}{0.391473in}}%
\pgfpathlineto{\pgfqpoint{5.921878in}{0.388940in}}%
\pgfpathlineto{\pgfqpoint{5.927248in}{0.388609in}}%
\pgfpathlineto{\pgfqpoint{5.933691in}{0.389846in}}%
\pgfpathlineto{\pgfqpoint{5.934765in}{0.388206in}}%
\pgfpathlineto{\pgfqpoint{5.942281in}{0.385712in}}%
\pgfpathlineto{\pgfqpoint{5.944429in}{0.385216in}}%
\pgfpathlineto{\pgfqpoint{5.948724in}{0.383976in}}%
\pgfpathlineto{\pgfqpoint{5.951946in}{0.387521in}}%
\pgfpathlineto{\pgfqpoint{5.955167in}{0.387619in}}%
\pgfpathlineto{\pgfqpoint{5.956241in}{0.388857in}}%
\pgfpathlineto{\pgfqpoint{5.957315in}{0.387267in}}%
\pgfpathlineto{\pgfqpoint{5.958389in}{0.387377in}}%
\pgfpathlineto{\pgfqpoint{5.959463in}{0.386153in}}%
\pgfpathlineto{\pgfqpoint{5.965906in}{0.386482in}}%
\pgfpathlineto{\pgfqpoint{5.970201in}{0.385445in}}%
\pgfpathlineto{\pgfqpoint{5.973423in}{0.385430in}}%
\pgfpathlineto{\pgfqpoint{5.974497in}{0.384987in}}%
\pgfpathlineto{\pgfqpoint{5.982013in}{0.383933in}}%
\pgfpathlineto{\pgfqpoint{5.988456in}{0.385144in}}%
\pgfpathlineto{\pgfqpoint{5.989530in}{0.387392in}}%
\pgfpathlineto{\pgfqpoint{5.992752in}{0.389094in}}%
\pgfpathlineto{\pgfqpoint{5.993826in}{0.390404in}}%
\pgfpathlineto{\pgfqpoint{5.994899in}{0.389781in}}%
\pgfpathlineto{\pgfqpoint{5.997047in}{0.390033in}}%
\pgfpathlineto{\pgfqpoint{6.001343in}{0.390582in}}%
\pgfpathlineto{\pgfqpoint{6.004564in}{0.393049in}}%
\pgfpathlineto{\pgfqpoint{6.008859in}{0.390806in}}%
\pgfpathlineto{\pgfqpoint{6.011007in}{0.392070in}}%
\pgfpathlineto{\pgfqpoint{6.012081in}{0.393604in}}%
\pgfpathlineto{\pgfqpoint{6.018524in}{0.395055in}}%
\pgfpathlineto{\pgfqpoint{6.019598in}{0.395547in}}%
\pgfpathlineto{\pgfqpoint{6.022819in}{0.395547in}}%
\pgfpathlineto{\pgfqpoint{6.024967in}{0.393954in}}%
\pgfpathlineto{\pgfqpoint{6.026041in}{0.393521in}}%
\pgfpathlineto{\pgfqpoint{6.031410in}{0.394995in}}%
\pgfpathlineto{\pgfqpoint{6.032484in}{0.393870in}}%
\pgfpathlineto{\pgfqpoint{6.033558in}{0.393717in}}%
\pgfpathlineto{\pgfqpoint{6.034632in}{0.392244in}}%
\pgfpathlineto{\pgfqpoint{6.037853in}{0.391764in}}%
\pgfpathlineto{\pgfqpoint{6.040001in}{0.393715in}}%
\pgfpathlineto{\pgfqpoint{6.041075in}{0.392960in}}%
\pgfpathlineto{\pgfqpoint{6.042148in}{0.393199in}}%
\pgfpathlineto{\pgfqpoint{6.047518in}{0.390428in}}%
\pgfpathlineto{\pgfqpoint{6.049665in}{0.393828in}}%
\pgfpathlineto{\pgfqpoint{6.052887in}{0.396715in}}%
\pgfpathlineto{\pgfqpoint{6.056108in}{0.396670in}}%
\pgfpathlineto{\pgfqpoint{6.057182in}{0.392721in}}%
\pgfpathlineto{\pgfqpoint{6.061477in}{0.392691in}}%
\pgfpathlineto{\pgfqpoint{6.062551in}{0.394640in}}%
\pgfpathlineto{\pgfqpoint{6.063625in}{0.394296in}}%
\pgfpathlineto{\pgfqpoint{6.064699in}{0.395837in}}%
\pgfpathlineto{\pgfqpoint{6.068994in}{0.395093in}}%
\pgfpathlineto{\pgfqpoint{6.070068in}{0.395866in}}%
\pgfpathlineto{\pgfqpoint{6.072216in}{0.395093in}}%
\pgfpathlineto{\pgfqpoint{6.075437in}{0.394891in}}%
\pgfpathlineto{\pgfqpoint{6.076511in}{0.393927in}}%
\pgfpathlineto{\pgfqpoint{6.077585in}{0.394664in}}%
\pgfpathlineto{\pgfqpoint{6.079733in}{0.393818in}}%
\pgfpathlineto{\pgfqpoint{6.082954in}{0.395811in}}%
\pgfpathlineto{\pgfqpoint{6.084028in}{0.394851in}}%
\pgfpathlineto{\pgfqpoint{6.086176in}{0.390178in}}%
\pgfpathlineto{\pgfqpoint{6.087250in}{0.388291in}}%
\pgfpathlineto{\pgfqpoint{6.090471in}{0.389278in}}%
\pgfpathlineto{\pgfqpoint{6.093693in}{0.388849in}}%
\pgfpathlineto{\pgfqpoint{6.097988in}{0.391192in}}%
\pgfpathlineto{\pgfqpoint{6.102283in}{0.390360in}}%
\pgfpathlineto{\pgfqpoint{6.106579in}{0.390386in}}%
\pgfpathlineto{\pgfqpoint{6.107653in}{0.389295in}}%
\pgfpathlineto{\pgfqpoint{6.109800in}{0.390048in}}%
\pgfpathlineto{\pgfqpoint{6.114096in}{0.392388in}}%
\pgfpathlineto{\pgfqpoint{6.115169in}{0.390676in}}%
\pgfpathlineto{\pgfqpoint{6.116243in}{0.390739in}}%
\pgfpathlineto{\pgfqpoint{6.117317in}{0.392703in}}%
\pgfpathlineto{\pgfqpoint{6.121612in}{0.393494in}}%
\pgfpathlineto{\pgfqpoint{6.122686in}{0.394810in}}%
\pgfpathlineto{\pgfqpoint{6.123760in}{0.397184in}}%
\pgfpathlineto{\pgfqpoint{6.124834in}{0.397865in}}%
\pgfpathlineto{\pgfqpoint{6.130203in}{0.398240in}}%
\pgfpathlineto{\pgfqpoint{6.132351in}{0.399880in}}%
\pgfpathlineto{\pgfqpoint{6.135572in}{0.400601in}}%
\pgfpathlineto{\pgfqpoint{6.136646in}{0.401645in}}%
\pgfpathlineto{\pgfqpoint{6.138794in}{0.401439in}}%
\pgfpathlineto{\pgfqpoint{6.139868in}{0.403070in}}%
\pgfpathlineto{\pgfqpoint{6.143089in}{0.400905in}}%
\pgfpathlineto{\pgfqpoint{6.145237in}{0.402093in}}%
\pgfpathlineto{\pgfqpoint{6.146311in}{0.402975in}}%
\pgfpathlineto{\pgfqpoint{6.152754in}{0.400878in}}%
\pgfpathlineto{\pgfqpoint{6.154901in}{0.401811in}}%
\pgfpathlineto{\pgfqpoint{6.158123in}{0.402314in}}%
\pgfpathlineto{\pgfqpoint{6.159197in}{0.401417in}}%
\pgfpathlineto{\pgfqpoint{6.161344in}{0.402304in}}%
\pgfpathlineto{\pgfqpoint{6.162418in}{0.403139in}}%
\pgfpathlineto{\pgfqpoint{6.167787in}{0.403207in}}%
\pgfpathlineto{\pgfqpoint{6.169935in}{0.404492in}}%
\pgfpathlineto{\pgfqpoint{6.174231in}{0.404437in}}%
\pgfpathlineto{\pgfqpoint{6.176378in}{0.406277in}}%
\pgfpathlineto{\pgfqpoint{6.177452in}{0.407135in}}%
\pgfpathlineto{\pgfqpoint{6.188190in}{0.409380in}}%
\pgfpathlineto{\pgfqpoint{6.189264in}{0.408315in}}%
\pgfpathlineto{\pgfqpoint{6.190338in}{0.408896in}}%
\pgfpathlineto{\pgfqpoint{6.192486in}{0.408742in}}%
\pgfpathlineto{\pgfqpoint{6.195707in}{0.410024in}}%
\pgfpathlineto{\pgfqpoint{6.197855in}{0.411744in}}%
\pgfpathlineto{\pgfqpoint{6.200003in}{0.411659in}}%
\pgfpathlineto{\pgfqpoint{6.203224in}{0.411984in}}%
\pgfpathlineto{\pgfqpoint{6.204298in}{0.411064in}}%
\pgfpathlineto{\pgfqpoint{6.206446in}{0.412638in}}%
\pgfpathlineto{\pgfqpoint{6.207520in}{0.412342in}}%
\pgfpathlineto{\pgfqpoint{6.213963in}{0.412411in}}%
\pgfpathlineto{\pgfqpoint{6.215036in}{0.411763in}}%
\pgfpathlineto{\pgfqpoint{6.218258in}{0.411565in}}%
\pgfpathlineto{\pgfqpoint{6.219332in}{0.412440in}}%
\pgfpathlineto{\pgfqpoint{6.221479in}{0.412694in}}%
\pgfpathlineto{\pgfqpoint{6.227922in}{0.411969in}}%
\pgfpathlineto{\pgfqpoint{6.228996in}{0.413394in}}%
\pgfpathlineto{\pgfqpoint{6.230070in}{0.413856in}}%
\pgfpathlineto{\pgfqpoint{6.233292in}{0.412843in}}%
\pgfpathlineto{\pgfqpoint{6.234365in}{0.414854in}}%
\pgfpathlineto{\pgfqpoint{6.236513in}{0.413957in}}%
\pgfpathlineto{\pgfqpoint{6.237587in}{0.414290in}}%
\pgfpathlineto{\pgfqpoint{6.241882in}{0.412643in}}%
\pgfpathlineto{\pgfqpoint{6.242956in}{0.408943in}}%
\pgfpathlineto{\pgfqpoint{6.244030in}{0.408343in}}%
\pgfpathlineto{\pgfqpoint{6.245104in}{0.409446in}}%
\pgfpathlineto{\pgfqpoint{6.248325in}{0.410213in}}%
\pgfpathlineto{\pgfqpoint{6.249399in}{0.411088in}}%
\pgfpathlineto{\pgfqpoint{6.250473in}{0.414360in}}%
\pgfpathlineto{\pgfqpoint{6.252621in}{0.414405in}}%
\pgfpathlineto{\pgfqpoint{6.255842in}{0.415200in}}%
\pgfpathlineto{\pgfqpoint{6.257990in}{0.411153in}}%
\pgfpathlineto{\pgfqpoint{6.259064in}{0.411689in}}%
\pgfpathlineto{\pgfqpoint{6.260138in}{0.410484in}}%
\pgfpathlineto{\pgfqpoint{6.263359in}{0.410323in}}%
\pgfpathlineto{\pgfqpoint{6.267654in}{0.414420in}}%
\pgfpathlineto{\pgfqpoint{6.271950in}{0.414525in}}%
\pgfpathlineto{\pgfqpoint{6.273024in}{0.413943in}}%
\pgfpathlineto{\pgfqpoint{6.274097in}{0.414331in}}%
\pgfpathlineto{\pgfqpoint{6.275171in}{0.415808in}}%
\pgfpathlineto{\pgfqpoint{6.278393in}{0.414850in}}%
\pgfpathlineto{\pgfqpoint{6.279467in}{0.413344in}}%
\pgfpathlineto{\pgfqpoint{6.280541in}{0.413968in}}%
\pgfpathlineto{\pgfqpoint{6.282688in}{0.418268in}}%
\pgfpathlineto{\pgfqpoint{6.285910in}{0.416121in}}%
\pgfpathlineto{\pgfqpoint{6.286984in}{0.416823in}}%
\pgfpathlineto{\pgfqpoint{6.290205in}{0.415253in}}%
\pgfpathlineto{\pgfqpoint{6.293427in}{0.415053in}}%
\pgfpathlineto{\pgfqpoint{6.294500in}{0.412929in}}%
\pgfpathlineto{\pgfqpoint{6.296648in}{0.415068in}}%
\pgfpathlineto{\pgfqpoint{6.297722in}{0.415286in}}%
\pgfpathlineto{\pgfqpoint{6.300943in}{0.413481in}}%
\pgfpathlineto{\pgfqpoint{6.302017in}{0.413709in}}%
\pgfpathlineto{\pgfqpoint{6.304165in}{0.417507in}}%
\pgfpathlineto{\pgfqpoint{6.305239in}{0.418318in}}%
\pgfpathlineto{\pgfqpoint{6.308460in}{0.416510in}}%
\pgfpathlineto{\pgfqpoint{6.310608in}{0.419191in}}%
\pgfpathlineto{\pgfqpoint{6.311682in}{0.416890in}}%
\pgfpathlineto{\pgfqpoint{6.312756in}{0.416843in}}%
\pgfpathlineto{\pgfqpoint{6.317051in}{0.416890in}}%
\pgfpathlineto{\pgfqpoint{6.318125in}{0.416074in}}%
\pgfpathlineto{\pgfqpoint{6.319199in}{0.416198in}}%
\pgfpathlineto{\pgfqpoint{6.320273in}{0.417396in}}%
\pgfpathlineto{\pgfqpoint{6.323494in}{0.417847in}}%
\pgfpathlineto{\pgfqpoint{6.324568in}{0.419301in}}%
\pgfpathlineto{\pgfqpoint{6.326716in}{0.416863in}}%
\pgfpathlineto{\pgfqpoint{6.327789in}{0.417272in}}%
\pgfpathlineto{\pgfqpoint{6.331011in}{0.417150in}}%
\pgfpathlineto{\pgfqpoint{6.332085in}{0.416281in}}%
\pgfpathlineto{\pgfqpoint{6.333159in}{0.417059in}}%
\pgfpathlineto{\pgfqpoint{6.334232in}{0.412530in}}%
\pgfpathlineto{\pgfqpoint{6.338528in}{0.412545in}}%
\pgfpathlineto{\pgfqpoint{6.339602in}{0.413496in}}%
\pgfpathlineto{\pgfqpoint{6.340675in}{0.416205in}}%
\pgfpathlineto{\pgfqpoint{6.341749in}{0.416846in}}%
\pgfpathlineto{\pgfqpoint{6.342823in}{0.418906in}}%
\pgfpathlineto{\pgfqpoint{6.346045in}{0.416914in}}%
\pgfpathlineto{\pgfqpoint{6.347119in}{0.417849in}}%
\pgfpathlineto{\pgfqpoint{6.348192in}{0.424022in}}%
\pgfpathlineto{\pgfqpoint{6.349266in}{0.421638in}}%
\pgfpathlineto{\pgfqpoint{6.350340in}{0.422664in}}%
\pgfpathlineto{\pgfqpoint{6.354635in}{0.423792in}}%
\pgfpathlineto{\pgfqpoint{6.355709in}{0.422670in}}%
\pgfpathlineto{\pgfqpoint{6.356783in}{0.423992in}}%
\pgfpathlineto{\pgfqpoint{6.357857in}{0.422884in}}%
\pgfpathlineto{\pgfqpoint{6.361078in}{0.423656in}}%
\pgfpathlineto{\pgfqpoint{6.362152in}{0.421054in}}%
\pgfpathlineto{\pgfqpoint{6.363226in}{0.423055in}}%
\pgfpathlineto{\pgfqpoint{6.364300in}{0.423088in}}%
\pgfpathlineto{\pgfqpoint{6.365374in}{0.424522in}}%
\pgfpathlineto{\pgfqpoint{6.369669in}{0.424255in}}%
\pgfpathlineto{\pgfqpoint{6.370743in}{0.424737in}}%
\pgfpathlineto{\pgfqpoint{6.371817in}{0.427429in}}%
\pgfpathlineto{\pgfqpoint{6.372891in}{0.427927in}}%
\pgfpathlineto{\pgfqpoint{6.378260in}{0.427927in}}%
\pgfpathlineto{\pgfqpoint{6.379334in}{0.420528in}}%
\pgfpathlineto{\pgfqpoint{6.383629in}{0.420579in}}%
\pgfpathlineto{\pgfqpoint{6.384703in}{0.421901in}}%
\pgfpathlineto{\pgfqpoint{6.385777in}{0.420339in}}%
\pgfpathlineto{\pgfqpoint{6.387924in}{0.422901in}}%
\pgfpathlineto{\pgfqpoint{6.392220in}{0.424159in}}%
\pgfpathlineto{\pgfqpoint{6.393294in}{0.422843in}}%
\pgfpathlineto{\pgfqpoint{6.394367in}{0.423371in}}%
\pgfpathlineto{\pgfqpoint{6.395441in}{0.422874in}}%
\pgfpathlineto{\pgfqpoint{6.398663in}{0.424503in}}%
\pgfpathlineto{\pgfqpoint{6.402958in}{0.430016in}}%
\pgfpathlineto{\pgfqpoint{6.406180in}{0.427761in}}%
\pgfpathlineto{\pgfqpoint{6.407253in}{0.426192in}}%
\pgfpathlineto{\pgfqpoint{6.408327in}{0.426586in}}%
\pgfpathlineto{\pgfqpoint{6.410475in}{0.425778in}}%
\pgfpathlineto{\pgfqpoint{6.414770in}{0.424952in}}%
\pgfpathlineto{\pgfqpoint{6.416918in}{0.420534in}}%
\pgfpathlineto{\pgfqpoint{6.417992in}{0.421290in}}%
\pgfpathlineto{\pgfqpoint{6.422287in}{0.419663in}}%
\pgfpathlineto{\pgfqpoint{6.423361in}{0.419848in}}%
\pgfpathlineto{\pgfqpoint{6.424435in}{0.421098in}}%
\pgfpathlineto{\pgfqpoint{6.425509in}{0.420456in}}%
\pgfpathlineto{\pgfqpoint{6.428730in}{0.421596in}}%
\pgfpathlineto{\pgfqpoint{6.430878in}{0.420117in}}%
\pgfpathlineto{\pgfqpoint{6.433026in}{0.421766in}}%
\pgfpathlineto{\pgfqpoint{6.437321in}{0.420946in}}%
\pgfpathlineto{\pgfqpoint{6.439469in}{0.424089in}}%
\pgfpathlineto{\pgfqpoint{6.440542in}{0.424573in}}%
\pgfpathlineto{\pgfqpoint{6.444838in}{0.419651in}}%
\pgfpathlineto{\pgfqpoint{6.445912in}{0.420320in}}%
\pgfpathlineto{\pgfqpoint{6.446985in}{0.422705in}}%
\pgfpathlineto{\pgfqpoint{6.448059in}{0.419349in}}%
\pgfpathlineto{\pgfqpoint{6.451281in}{0.419413in}}%
\pgfpathlineto{\pgfqpoint{6.452355in}{0.417011in}}%
\pgfpathlineto{\pgfqpoint{6.454502in}{0.417927in}}%
\pgfpathlineto{\pgfqpoint{6.459872in}{0.416994in}}%
\pgfpathlineto{\pgfqpoint{6.460945in}{0.417389in}}%
\pgfpathlineto{\pgfqpoint{6.467388in}{0.416515in}}%
\pgfpathlineto{\pgfqpoint{6.468462in}{0.413718in}}%
\pgfpathlineto{\pgfqpoint{6.470610in}{0.414981in}}%
\pgfpathlineto{\pgfqpoint{6.473831in}{0.407612in}}%
\pgfpathlineto{\pgfqpoint{6.474905in}{0.411660in}}%
\pgfpathlineto{\pgfqpoint{6.475979in}{0.409149in}}%
\pgfpathlineto{\pgfqpoint{6.478127in}{0.408590in}}%
\pgfpathlineto{\pgfqpoint{6.481348in}{0.412336in}}%
\pgfpathlineto{\pgfqpoint{6.483496in}{0.411223in}}%
\pgfpathlineto{\pgfqpoint{6.484570in}{0.407039in}}%
\pgfpathlineto{\pgfqpoint{6.485644in}{0.406093in}}%
\pgfpathlineto{\pgfqpoint{6.488865in}{0.406418in}}%
\pgfpathlineto{\pgfqpoint{6.489939in}{0.407464in}}%
\pgfpathlineto{\pgfqpoint{6.491013in}{0.410291in}}%
\pgfpathlineto{\pgfqpoint{6.492087in}{0.410136in}}%
\pgfpathlineto{\pgfqpoint{6.493161in}{0.408893in}}%
\pgfpathlineto{\pgfqpoint{6.500677in}{0.407990in}}%
\pgfpathlineto{\pgfqpoint{6.500677in}{0.407990in}}%
\pgfusepath{stroke}%
\end{pgfscope}%
\begin{pgfscope}%
\pgfsetrectcap%
\pgfsetmiterjoin%
\pgfsetlinewidth{0.803000pt}%
\definecolor{currentstroke}{rgb}{1.000000,1.000000,1.000000}%
\pgfsetstrokecolor{currentstroke}%
\pgfsetdash{}{0pt}%
\pgfpathmoveto{\pgfqpoint{4.034768in}{0.331635in}}%
\pgfpathlineto{\pgfqpoint{4.034768in}{0.732520in}}%
\pgfusepath{stroke}%
\end{pgfscope}%
\begin{pgfscope}%
\pgfsetrectcap%
\pgfsetmiterjoin%
\pgfsetlinewidth{0.803000pt}%
\definecolor{currentstroke}{rgb}{1.000000,1.000000,1.000000}%
\pgfsetstrokecolor{currentstroke}%
\pgfsetdash{}{0pt}%
\pgfpathmoveto{\pgfqpoint{6.618102in}{0.331635in}}%
\pgfpathlineto{\pgfqpoint{6.618102in}{0.732520in}}%
\pgfusepath{stroke}%
\end{pgfscope}%
\begin{pgfscope}%
\pgfsetrectcap%
\pgfsetmiterjoin%
\pgfsetlinewidth{0.803000pt}%
\definecolor{currentstroke}{rgb}{1.000000,1.000000,1.000000}%
\pgfsetstrokecolor{currentstroke}%
\pgfsetdash{}{0pt}%
\pgfpathmoveto{\pgfqpoint{4.034768in}{0.331635in}}%
\pgfpathlineto{\pgfqpoint{6.618102in}{0.331635in}}%
\pgfusepath{stroke}%
\end{pgfscope}%
\begin{pgfscope}%
\pgfsetrectcap%
\pgfsetmiterjoin%
\pgfsetlinewidth{0.803000pt}%
\definecolor{currentstroke}{rgb}{1.000000,1.000000,1.000000}%
\pgfsetstrokecolor{currentstroke}%
\pgfsetdash{}{0pt}%
\pgfpathmoveto{\pgfqpoint{4.034768in}{0.732520in}}%
\pgfpathlineto{\pgfqpoint{6.618102in}{0.732520in}}%
\pgfusepath{stroke}%
\end{pgfscope}%
\begin{pgfscope}%
\definecolor{textcolor}{rgb}{0.150000,0.150000,0.150000}%
\pgfsetstrokecolor{textcolor}%
\pgfsetfillcolor{textcolor}%
\pgftext[x=5.326435in,y=0.815853in,,base]{\color{textcolor}\rmfamily\fontsize{12.000000}{14.400000}\selectfont DIS}%
\end{pgfscope}%
\begin{pgfscope}%
\definecolor{textcolor}{rgb}{0.150000,0.150000,0.150000}%
\pgfsetstrokecolor{textcolor}%
\pgfsetfillcolor{textcolor}%
\pgftext[x=3.418102in,y=5.781016in,,top]{\color{textcolor}\rmfamily\fontsize{12.000000}{14.400000}\selectfont Return Trading Strategy vs. Buy and Hold}%
\end{pgfscope}%
\end{pgfpicture}%
\makeatother%
\endgroup%

    \end{adjustbox}  
    \caption{Mean Reversion based on past returns for single stocks}
    \label{fig:mean_reversion_returns}
\end{figure}{}

\subsubsection{Comparison of 90d and 30d moving averages}
Idea: If 30d averages is above 90d average, then sell. And vice versa

\subsubsection{Mean Reversion Portfolio}
Idea: Look at entire portfolio. 




\subsection{Momentum Based Trading}


\subsection{Pairs Trading}



\section{Trading Strategies Based on Our Predictions}



\section{Hybrid Trading Strategies}




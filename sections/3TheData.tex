\chapter{The Data}

\section{Stock Selection}
The stock data comprises 10 selected companies from the NASDAQ stock index. The stocks were determined as those are the stocks we have Ravenpack data and analyst reports about. 


\section{Ravenpack Sentiment Data}


\section{Analyst Reports}


\section{Stock Data}
\subsection{Overview over the Stock Data}
The stock data were downloaded from Yahoo Financial Data Base. Table \ref{tab:stocks_overview} provides a small overview over the raw data. Figure \ref{fig:Daily Stock Prices for all Stocks in the Data Set} shows the Closing Prices of the selected assets. The closing prices have been provided adjusted for dividends by Yahoo. Time series analysis is easiest with data that are at least weakly stationary. Weak stationarity implies that the mean of the time series is constant over time and that the covariance between two observations $y_t$ and $y_{t+h}$ depends only on h, not on t (see [Shumway and Stoffer 2011]). 

\begin{table}[]
    \figuretitle{Stock Data}
    \centering
    \begin{adjustbox}{width = 0.9\linewidth}
    \setlength{\tabcolsep}{15pt}
    \begin{tabular}{lrrrrrrr}
\toprule
Date &    Open &    High &     Low &   Close &  Adj Close &    Volume & ticker \\
\midrule
2012-01-03 &   83.76 &   84.44 &   83.36 &   83.49 &      68.41 &   3380100 & MMM\\
2012-01-04 &   83.13 &   84.26 &   83.11 &   84.18 &      68.98 &   3007400 & MMM \\
2012-01-05 &   83.53 &   83.87 &   82.70 &   83.80 &      68.67 &   3116400 & MMM \\
...        &   ...   &   ...   &   ...   &   ...   &      ...   &   ...     & ... \\
2015-11-27 &   29.11 &   29.21 &   29.03 &   29.19 &      25.97 &  34469600 & GE \\
2015-11-30 &   29.16 &   29.28 &   28.79 &   28.79 &      25.61 &  82905200 & GE \\
2015-12-01 &   28.84 &   29.09 &   28.72 &   29.01 &      25.80 &  56414600 & GE \\
...        &   ...   &   ...   &   ...   &   ...   &      ...   &   ...     & ... \\
2017-12-27 &  108.42 &  108.55 &  107.46 &  107.64 &     105.31 &   5624000 & DIS \\
2017-12-28 &  108.00 &  108.05 &  107.06 &  107.77 &     105.43 &   3477700 & DIS \\
2017-12-29 &  108.05 &  108.34 &  107.51 &  107.51 &     105.18 &   4538400 & DIS \\
\bottomrule
\end{tabular}

    \end{adjustbox}
    \caption{}
    \label{tab:stocks_overview}
\end{table}{}

\begin{figure}[h]
    \figuretitle{Stock Prices}
    \centering
    \begin{adjustbox}{width=.9\textwidth,center}
    %% Creator: Matplotlib, PGF backend
%%
%% To include the figure in your LaTeX document, write
%%   \input{<filename>.pgf}
%%
%% Make sure the required packages are loaded in your preamble
%%   \usepackage{pgf}
%%
%% Figures using additional raster images can only be included by \input if
%% they are in the same directory as the main LaTeX file. For loading figures
%% from other directories you can use the `import` package
%%   \usepackage{import}
%% and then include the figures with
%%   \import{<path to file>}{<filename>.pgf}
%%
%% Matplotlib used the following preamble
%%   \usepackage{fontspec}
%%   \setmainfont{DejaVuSerif.ttf}[Path=/opt/tljh/user/lib/python3.6/site-packages/matplotlib/mpl-data/fonts/ttf/]
%%   \setsansfont{DejaVuSans.ttf}[Path=/opt/tljh/user/lib/python3.6/site-packages/matplotlib/mpl-data/fonts/ttf/]
%%   \setmonofont{DejaVuSansMono.ttf}[Path=/opt/tljh/user/lib/python3.6/site-packages/matplotlib/mpl-data/fonts/ttf/]
%%
\begingroup%
\makeatletter%
\begin{pgfpicture}%
\pgfpathrectangle{\pgfpointorigin}{\pgfqpoint{6.952287in}{5.361564in}}%
\pgfusepath{use as bounding box, clip}%
\begin{pgfscope}%
\pgfsetbuttcap%
\pgfsetmiterjoin%
\definecolor{currentfill}{rgb}{1.000000,1.000000,1.000000}%
\pgfsetfillcolor{currentfill}%
\pgfsetlinewidth{0.000000pt}%
\definecolor{currentstroke}{rgb}{1.000000,1.000000,1.000000}%
\pgfsetstrokecolor{currentstroke}%
\pgfsetdash{}{0pt}%
\pgfpathmoveto{\pgfqpoint{0.000000in}{0.000000in}}%
\pgfpathlineto{\pgfqpoint{6.952287in}{0.000000in}}%
\pgfpathlineto{\pgfqpoint{6.952287in}{5.361564in}}%
\pgfpathlineto{\pgfqpoint{0.000000in}{5.361564in}}%
\pgfpathclose%
\pgfusepath{fill}%
\end{pgfscope}%
\begin{pgfscope}%
\pgfsetbuttcap%
\pgfsetmiterjoin%
\definecolor{currentfill}{rgb}{0.917647,0.917647,0.949020}%
\pgfsetfillcolor{currentfill}%
\pgfsetlinewidth{0.000000pt}%
\definecolor{currentstroke}{rgb}{0.000000,0.000000,0.000000}%
\pgfsetstrokecolor{currentstroke}%
\pgfsetstrokeopacity{0.000000}%
\pgfsetdash{}{0pt}%
\pgfpathmoveto{\pgfqpoint{0.652287in}{0.521603in}}%
\pgfpathlineto{\pgfqpoint{6.852287in}{0.521603in}}%
\pgfpathlineto{\pgfqpoint{6.852287in}{5.051603in}}%
\pgfpathlineto{\pgfqpoint{0.652287in}{5.051603in}}%
\pgfpathclose%
\pgfusepath{fill}%
\end{pgfscope}%
\begin{pgfscope}%
\pgfpathrectangle{\pgfqpoint{0.652287in}{0.521603in}}{\pgfqpoint{6.200000in}{4.530000in}}%
\pgfusepath{clip}%
\pgfsetroundcap%
\pgfsetroundjoin%
\pgfsetlinewidth{0.803000pt}%
\definecolor{currentstroke}{rgb}{1.000000,1.000000,1.000000}%
\pgfsetstrokecolor{currentstroke}%
\pgfsetdash{}{0pt}%
\pgfpathmoveto{\pgfqpoint{0.928950in}{0.521603in}}%
\pgfpathlineto{\pgfqpoint{0.928950in}{5.051603in}}%
\pgfusepath{stroke}%
\end{pgfscope}%
\begin{pgfscope}%
\definecolor{textcolor}{rgb}{0.150000,0.150000,0.150000}%
\pgfsetstrokecolor{textcolor}%
\pgfsetfillcolor{textcolor}%
\pgftext[x=0.928950in,y=0.424381in,,top]{\color{textcolor}\rmfamily\fontsize{10.000000}{12.000000}\selectfont 2012}%
\end{pgfscope}%
\begin{pgfscope}%
\pgfpathrectangle{\pgfqpoint{0.652287in}{0.521603in}}{\pgfqpoint{6.200000in}{4.530000in}}%
\pgfusepath{clip}%
\pgfsetroundcap%
\pgfsetroundjoin%
\pgfsetlinewidth{0.803000pt}%
\definecolor{currentstroke}{rgb}{1.000000,1.000000,1.000000}%
\pgfsetstrokecolor{currentstroke}%
\pgfsetdash{}{0pt}%
\pgfpathmoveto{\pgfqpoint{1.872210in}{0.521603in}}%
\pgfpathlineto{\pgfqpoint{1.872210in}{5.051603in}}%
\pgfusepath{stroke}%
\end{pgfscope}%
\begin{pgfscope}%
\definecolor{textcolor}{rgb}{0.150000,0.150000,0.150000}%
\pgfsetstrokecolor{textcolor}%
\pgfsetfillcolor{textcolor}%
\pgftext[x=1.872210in,y=0.424381in,,top]{\color{textcolor}\rmfamily\fontsize{10.000000}{12.000000}\selectfont 2013}%
\end{pgfscope}%
\begin{pgfscope}%
\pgfpathrectangle{\pgfqpoint{0.652287in}{0.521603in}}{\pgfqpoint{6.200000in}{4.530000in}}%
\pgfusepath{clip}%
\pgfsetroundcap%
\pgfsetroundjoin%
\pgfsetlinewidth{0.803000pt}%
\definecolor{currentstroke}{rgb}{1.000000,1.000000,1.000000}%
\pgfsetstrokecolor{currentstroke}%
\pgfsetdash{}{0pt}%
\pgfpathmoveto{\pgfqpoint{2.812893in}{0.521603in}}%
\pgfpathlineto{\pgfqpoint{2.812893in}{5.051603in}}%
\pgfusepath{stroke}%
\end{pgfscope}%
\begin{pgfscope}%
\definecolor{textcolor}{rgb}{0.150000,0.150000,0.150000}%
\pgfsetstrokecolor{textcolor}%
\pgfsetfillcolor{textcolor}%
\pgftext[x=2.812893in,y=0.424381in,,top]{\color{textcolor}\rmfamily\fontsize{10.000000}{12.000000}\selectfont 2014}%
\end{pgfscope}%
\begin{pgfscope}%
\pgfpathrectangle{\pgfqpoint{0.652287in}{0.521603in}}{\pgfqpoint{6.200000in}{4.530000in}}%
\pgfusepath{clip}%
\pgfsetroundcap%
\pgfsetroundjoin%
\pgfsetlinewidth{0.803000pt}%
\definecolor{currentstroke}{rgb}{1.000000,1.000000,1.000000}%
\pgfsetstrokecolor{currentstroke}%
\pgfsetdash{}{0pt}%
\pgfpathmoveto{\pgfqpoint{3.753575in}{0.521603in}}%
\pgfpathlineto{\pgfqpoint{3.753575in}{5.051603in}}%
\pgfusepath{stroke}%
\end{pgfscope}%
\begin{pgfscope}%
\definecolor{textcolor}{rgb}{0.150000,0.150000,0.150000}%
\pgfsetstrokecolor{textcolor}%
\pgfsetfillcolor{textcolor}%
\pgftext[x=3.753575in,y=0.424381in,,top]{\color{textcolor}\rmfamily\fontsize{10.000000}{12.000000}\selectfont 2015}%
\end{pgfscope}%
\begin{pgfscope}%
\pgfpathrectangle{\pgfqpoint{0.652287in}{0.521603in}}{\pgfqpoint{6.200000in}{4.530000in}}%
\pgfusepath{clip}%
\pgfsetroundcap%
\pgfsetroundjoin%
\pgfsetlinewidth{0.803000pt}%
\definecolor{currentstroke}{rgb}{1.000000,1.000000,1.000000}%
\pgfsetstrokecolor{currentstroke}%
\pgfsetdash{}{0pt}%
\pgfpathmoveto{\pgfqpoint{4.694258in}{0.521603in}}%
\pgfpathlineto{\pgfqpoint{4.694258in}{5.051603in}}%
\pgfusepath{stroke}%
\end{pgfscope}%
\begin{pgfscope}%
\definecolor{textcolor}{rgb}{0.150000,0.150000,0.150000}%
\pgfsetstrokecolor{textcolor}%
\pgfsetfillcolor{textcolor}%
\pgftext[x=4.694258in,y=0.424381in,,top]{\color{textcolor}\rmfamily\fontsize{10.000000}{12.000000}\selectfont 2016}%
\end{pgfscope}%
\begin{pgfscope}%
\pgfpathrectangle{\pgfqpoint{0.652287in}{0.521603in}}{\pgfqpoint{6.200000in}{4.530000in}}%
\pgfusepath{clip}%
\pgfsetroundcap%
\pgfsetroundjoin%
\pgfsetlinewidth{0.803000pt}%
\definecolor{currentstroke}{rgb}{1.000000,1.000000,1.000000}%
\pgfsetstrokecolor{currentstroke}%
\pgfsetdash{}{0pt}%
\pgfpathmoveto{\pgfqpoint{5.637517in}{0.521603in}}%
\pgfpathlineto{\pgfqpoint{5.637517in}{5.051603in}}%
\pgfusepath{stroke}%
\end{pgfscope}%
\begin{pgfscope}%
\definecolor{textcolor}{rgb}{0.150000,0.150000,0.150000}%
\pgfsetstrokecolor{textcolor}%
\pgfsetfillcolor{textcolor}%
\pgftext[x=5.637517in,y=0.424381in,,top]{\color{textcolor}\rmfamily\fontsize{10.000000}{12.000000}\selectfont 2017}%
\end{pgfscope}%
\begin{pgfscope}%
\pgfpathrectangle{\pgfqpoint{0.652287in}{0.521603in}}{\pgfqpoint{6.200000in}{4.530000in}}%
\pgfusepath{clip}%
\pgfsetroundcap%
\pgfsetroundjoin%
\pgfsetlinewidth{0.803000pt}%
\definecolor{currentstroke}{rgb}{1.000000,1.000000,1.000000}%
\pgfsetstrokecolor{currentstroke}%
\pgfsetdash{}{0pt}%
\pgfpathmoveto{\pgfqpoint{6.578200in}{0.521603in}}%
\pgfpathlineto{\pgfqpoint{6.578200in}{5.051603in}}%
\pgfusepath{stroke}%
\end{pgfscope}%
\begin{pgfscope}%
\definecolor{textcolor}{rgb}{0.150000,0.150000,0.150000}%
\pgfsetstrokecolor{textcolor}%
\pgfsetfillcolor{textcolor}%
\pgftext[x=6.578200in,y=0.424381in,,top]{\color{textcolor}\rmfamily\fontsize{10.000000}{12.000000}\selectfont 2018}%
\end{pgfscope}%
\begin{pgfscope}%
\definecolor{textcolor}{rgb}{0.150000,0.150000,0.150000}%
\pgfsetstrokecolor{textcolor}%
\pgfsetfillcolor{textcolor}%
\pgftext[x=3.752287in,y=0.234413in,,top]{\color{textcolor}\rmfamily\fontsize{10.000000}{12.000000}\selectfont Year}%
\end{pgfscope}%
\begin{pgfscope}%
\pgfpathrectangle{\pgfqpoint{0.652287in}{0.521603in}}{\pgfqpoint{6.200000in}{4.530000in}}%
\pgfusepath{clip}%
\pgfsetroundcap%
\pgfsetroundjoin%
\pgfsetlinewidth{0.803000pt}%
\definecolor{currentstroke}{rgb}{1.000000,1.000000,1.000000}%
\pgfsetstrokecolor{currentstroke}%
\pgfsetdash{}{0pt}%
\pgfpathmoveto{\pgfqpoint{0.652287in}{1.407684in}}%
\pgfpathlineto{\pgfqpoint{6.852287in}{1.407684in}}%
\pgfusepath{stroke}%
\end{pgfscope}%
\begin{pgfscope}%
\definecolor{textcolor}{rgb}{0.150000,0.150000,0.150000}%
\pgfsetstrokecolor{textcolor}%
\pgfsetfillcolor{textcolor}%
\pgftext[x=0.378334in,y=1.354923in,left,base]{\color{textcolor}\rmfamily\fontsize{10.000000}{12.000000}\selectfont 50}%
\end{pgfscope}%
\begin{pgfscope}%
\pgfpathrectangle{\pgfqpoint{0.652287in}{0.521603in}}{\pgfqpoint{6.200000in}{4.530000in}}%
\pgfusepath{clip}%
\pgfsetroundcap%
\pgfsetroundjoin%
\pgfsetlinewidth{0.803000pt}%
\definecolor{currentstroke}{rgb}{1.000000,1.000000,1.000000}%
\pgfsetstrokecolor{currentstroke}%
\pgfsetdash{}{0pt}%
\pgfpathmoveto{\pgfqpoint{0.652287in}{2.345852in}}%
\pgfpathlineto{\pgfqpoint{6.852287in}{2.345852in}}%
\pgfusepath{stroke}%
\end{pgfscope}%
\begin{pgfscope}%
\definecolor{textcolor}{rgb}{0.150000,0.150000,0.150000}%
\pgfsetstrokecolor{textcolor}%
\pgfsetfillcolor{textcolor}%
\pgftext[x=0.289968in,y=2.293091in,left,base]{\color{textcolor}\rmfamily\fontsize{10.000000}{12.000000}\selectfont 100}%
\end{pgfscope}%
\begin{pgfscope}%
\pgfpathrectangle{\pgfqpoint{0.652287in}{0.521603in}}{\pgfqpoint{6.200000in}{4.530000in}}%
\pgfusepath{clip}%
\pgfsetroundcap%
\pgfsetroundjoin%
\pgfsetlinewidth{0.803000pt}%
\definecolor{currentstroke}{rgb}{1.000000,1.000000,1.000000}%
\pgfsetstrokecolor{currentstroke}%
\pgfsetdash{}{0pt}%
\pgfpathmoveto{\pgfqpoint{0.652287in}{3.284020in}}%
\pgfpathlineto{\pgfqpoint{6.852287in}{3.284020in}}%
\pgfusepath{stroke}%
\end{pgfscope}%
\begin{pgfscope}%
\definecolor{textcolor}{rgb}{0.150000,0.150000,0.150000}%
\pgfsetstrokecolor{textcolor}%
\pgfsetfillcolor{textcolor}%
\pgftext[x=0.289968in,y=3.231258in,left,base]{\color{textcolor}\rmfamily\fontsize{10.000000}{12.000000}\selectfont 150}%
\end{pgfscope}%
\begin{pgfscope}%
\pgfpathrectangle{\pgfqpoint{0.652287in}{0.521603in}}{\pgfqpoint{6.200000in}{4.530000in}}%
\pgfusepath{clip}%
\pgfsetroundcap%
\pgfsetroundjoin%
\pgfsetlinewidth{0.803000pt}%
\definecolor{currentstroke}{rgb}{1.000000,1.000000,1.000000}%
\pgfsetstrokecolor{currentstroke}%
\pgfsetdash{}{0pt}%
\pgfpathmoveto{\pgfqpoint{0.652287in}{4.222188in}}%
\pgfpathlineto{\pgfqpoint{6.852287in}{4.222188in}}%
\pgfusepath{stroke}%
\end{pgfscope}%
\begin{pgfscope}%
\definecolor{textcolor}{rgb}{0.150000,0.150000,0.150000}%
\pgfsetstrokecolor{textcolor}%
\pgfsetfillcolor{textcolor}%
\pgftext[x=0.289968in,y=4.169426in,left,base]{\color{textcolor}\rmfamily\fontsize{10.000000}{12.000000}\selectfont 200}%
\end{pgfscope}%
\begin{pgfscope}%
\definecolor{textcolor}{rgb}{0.150000,0.150000,0.150000}%
\pgfsetstrokecolor{textcolor}%
\pgfsetfillcolor{textcolor}%
\pgftext[x=0.234413in,y=2.786603in,,bottom,rotate=90.000000]{\color{textcolor}\rmfamily\fontsize{10.000000}{12.000000}\selectfont Nominal Dollars}%
\end{pgfscope}%
\begin{pgfscope}%
\pgfpathrectangle{\pgfqpoint{0.652287in}{0.521603in}}{\pgfqpoint{6.200000in}{4.530000in}}%
\pgfusepath{clip}%
\pgfsetroundcap%
\pgfsetroundjoin%
\pgfsetlinewidth{1.505625pt}%
\definecolor{currentstroke}{rgb}{0.121569,0.466667,0.705882}%
\pgfsetstrokecolor{currentstroke}%
\pgfsetdash{}{0pt}%
\pgfpathmoveto{\pgfqpoint{0.934105in}{1.764938in}}%
\pgfpathlineto{\pgfqpoint{0.936682in}{1.775634in}}%
\pgfpathlineto{\pgfqpoint{0.941836in}{1.763062in}}%
\pgfpathlineto{\pgfqpoint{0.949568in}{1.770755in}}%
\pgfpathlineto{\pgfqpoint{0.952145in}{1.777510in}}%
\pgfpathlineto{\pgfqpoint{0.954722in}{1.769254in}}%
\pgfpathlineto{\pgfqpoint{0.957300in}{1.777135in}}%
\pgfpathlineto{\pgfqpoint{0.959877in}{1.766627in}}%
\pgfpathlineto{\pgfqpoint{0.970186in}{1.776384in}}%
\pgfpathlineto{\pgfqpoint{0.975340in}{1.800777in}}%
\pgfpathlineto{\pgfqpoint{0.977917in}{1.798525in}}%
\pgfpathlineto{\pgfqpoint{0.985649in}{1.797774in}}%
\pgfpathlineto{\pgfqpoint{0.988226in}{1.802840in}}%
\pgfpathlineto{\pgfqpoint{0.990803in}{1.811284in}}%
\pgfpathlineto{\pgfqpoint{0.993381in}{1.828359in}}%
\pgfpathlineto{\pgfqpoint{0.995958in}{1.826482in}}%
\pgfpathlineto{\pgfqpoint{1.003689in}{1.824606in}}%
\pgfpathlineto{\pgfqpoint{1.006267in}{1.814849in}}%
\pgfpathlineto{\pgfqpoint{1.008844in}{1.824794in}}%
\pgfpathlineto{\pgfqpoint{1.011421in}{1.826107in}}%
\pgfpathlineto{\pgfqpoint{1.013998in}{1.830798in}}%
\pgfpathlineto{\pgfqpoint{1.021730in}{1.828171in}}%
\pgfpathlineto{\pgfqpoint{1.024307in}{1.833237in}}%
\pgfpathlineto{\pgfqpoint{1.029462in}{1.835301in}}%
\pgfpathlineto{\pgfqpoint{1.032039in}{1.821604in}}%
\pgfpathlineto{\pgfqpoint{1.039770in}{1.835301in}}%
\pgfpathlineto{\pgfqpoint{1.042348in}{1.834738in}}%
\pgfpathlineto{\pgfqpoint{1.044925in}{1.828734in}}%
\pgfpathlineto{\pgfqpoint{1.047502in}{1.838678in}}%
\pgfpathlineto{\pgfqpoint{1.050079in}{1.837177in}}%
\pgfpathlineto{\pgfqpoint{1.060388in}{1.837928in}}%
\pgfpathlineto{\pgfqpoint{1.065543in}{1.842619in}}%
\pgfpathlineto{\pgfqpoint{1.068120in}{1.847310in}}%
\pgfpathlineto{\pgfqpoint{1.075851in}{1.845246in}}%
\pgfpathlineto{\pgfqpoint{1.078429in}{1.840555in}}%
\pgfpathlineto{\pgfqpoint{1.081006in}{1.837928in}}%
\pgfpathlineto{\pgfqpoint{1.083583in}{1.836239in}}%
\pgfpathlineto{\pgfqpoint{1.086160in}{1.836615in}}%
\pgfpathlineto{\pgfqpoint{1.093892in}{1.829484in}}%
\pgfpathlineto{\pgfqpoint{1.096469in}{1.796461in}}%
\pgfpathlineto{\pgfqpoint{1.099046in}{1.804529in}}%
\pgfpathlineto{\pgfqpoint{1.101624in}{1.824043in}}%
\pgfpathlineto{\pgfqpoint{1.104201in}{1.825357in}}%
\pgfpathlineto{\pgfqpoint{1.111932in}{1.836990in}}%
\pgfpathlineto{\pgfqpoint{1.114510in}{1.855941in}}%
\pgfpathlineto{\pgfqpoint{1.117087in}{1.857817in}}%
\pgfpathlineto{\pgfqpoint{1.119664in}{1.875455in}}%
\pgfpathlineto{\pgfqpoint{1.122241in}{1.868512in}}%
\pgfpathlineto{\pgfqpoint{1.129973in}{1.871327in}}%
\pgfpathlineto{\pgfqpoint{1.137704in}{1.853126in}}%
\pgfpathlineto{\pgfqpoint{1.140282in}{1.851250in}}%
\pgfpathlineto{\pgfqpoint{1.148013in}{1.861570in}}%
\pgfpathlineto{\pgfqpoint{1.150591in}{1.861945in}}%
\pgfpathlineto{\pgfqpoint{1.153168in}{1.851250in}}%
\pgfpathlineto{\pgfqpoint{1.155745in}{1.856128in}}%
\pgfpathlineto{\pgfqpoint{1.158322in}{1.863071in}}%
\pgfpathlineto{\pgfqpoint{1.166054in}{1.863446in}}%
\pgfpathlineto{\pgfqpoint{1.168631in}{1.856504in}}%
\pgfpathlineto{\pgfqpoint{1.171208in}{1.841681in}}%
\pgfpathlineto{\pgfqpoint{1.173785in}{1.832674in}}%
\pgfpathlineto{\pgfqpoint{1.184094in}{1.818039in}}%
\pgfpathlineto{\pgfqpoint{1.186672in}{1.790644in}}%
\pgfpathlineto{\pgfqpoint{1.189249in}{1.802653in}}%
\pgfpathlineto{\pgfqpoint{1.191826in}{1.826295in}}%
\pgfpathlineto{\pgfqpoint{1.194403in}{1.808094in}}%
\pgfpathlineto{\pgfqpoint{1.202135in}{1.819165in}}%
\pgfpathlineto{\pgfqpoint{1.204712in}{1.835489in}}%
\pgfpathlineto{\pgfqpoint{1.209866in}{1.825357in}}%
\pgfpathlineto{\pgfqpoint{1.212444in}{1.836052in}}%
\pgfpathlineto{\pgfqpoint{1.220175in}{1.830610in}}%
\pgfpathlineto{\pgfqpoint{1.222753in}{1.851813in}}%
\pgfpathlineto{\pgfqpoint{1.225330in}{1.856691in}}%
\pgfpathlineto{\pgfqpoint{1.227907in}{1.864760in}}%
\pgfpathlineto{\pgfqpoint{1.230484in}{1.865322in}}%
\pgfpathlineto{\pgfqpoint{1.238216in}{1.865322in}}%
\pgfpathlineto{\pgfqpoint{1.240793in}{1.869075in}}%
\pgfpathlineto{\pgfqpoint{1.243370in}{1.867949in}}%
\pgfpathlineto{\pgfqpoint{1.245947in}{1.865885in}}%
\pgfpathlineto{\pgfqpoint{1.248525in}{1.854627in}}%
\pgfpathlineto{\pgfqpoint{1.256256in}{1.844307in}}%
\pgfpathlineto{\pgfqpoint{1.258833in}{1.837177in}}%
\pgfpathlineto{\pgfqpoint{1.261411in}{1.832674in}}%
\pgfpathlineto{\pgfqpoint{1.263988in}{1.830610in}}%
\pgfpathlineto{\pgfqpoint{1.266565in}{1.824231in}}%
\pgfpathlineto{\pgfqpoint{1.274297in}{1.810158in}}%
\pgfpathlineto{\pgfqpoint{1.276874in}{1.809408in}}%
\pgfpathlineto{\pgfqpoint{1.279451in}{1.812035in}}%
\pgfpathlineto{\pgfqpoint{1.284606in}{1.782951in}}%
\pgfpathlineto{\pgfqpoint{1.292337in}{1.797962in}}%
\pgfpathlineto{\pgfqpoint{1.294914in}{1.792708in}}%
\pgfpathlineto{\pgfqpoint{1.297492in}{1.803403in}}%
\pgfpathlineto{\pgfqpoint{1.300069in}{1.806218in}}%
\pgfpathlineto{\pgfqpoint{1.302646in}{1.803028in}}%
\pgfpathlineto{\pgfqpoint{1.312955in}{1.818039in}}%
\pgfpathlineto{\pgfqpoint{1.315532in}{1.797774in}}%
\pgfpathlineto{\pgfqpoint{1.318109in}{1.797211in}}%
\pgfpathlineto{\pgfqpoint{1.320687in}{1.772631in}}%
\pgfpathlineto{\pgfqpoint{1.330995in}{1.767378in}}%
\pgfpathlineto{\pgfqpoint{1.333573in}{1.800589in}}%
\pgfpathlineto{\pgfqpoint{1.336150in}{1.814286in}}%
\pgfpathlineto{\pgfqpoint{1.338727in}{1.822167in}}%
\pgfpathlineto{\pgfqpoint{1.346459in}{1.809220in}}%
\pgfpathlineto{\pgfqpoint{1.349036in}{1.833612in}}%
\pgfpathlineto{\pgfqpoint{1.351613in}{1.824231in}}%
\pgfpathlineto{\pgfqpoint{1.356768in}{1.844870in}}%
\pgfpathlineto{\pgfqpoint{1.364499in}{1.842806in}}%
\pgfpathlineto{\pgfqpoint{1.367076in}{1.850875in}}%
\pgfpathlineto{\pgfqpoint{1.369654in}{1.846371in}}%
\pgfpathlineto{\pgfqpoint{1.372231in}{1.833612in}}%
\pgfpathlineto{\pgfqpoint{1.374808in}{1.835301in}}%
\pgfpathlineto{\pgfqpoint{1.382540in}{1.819915in}}%
\pgfpathlineto{\pgfqpoint{1.385117in}{1.824981in}}%
\pgfpathlineto{\pgfqpoint{1.387694in}{1.840367in}}%
\pgfpathlineto{\pgfqpoint{1.390271in}{1.840367in}}%
\pgfpathlineto{\pgfqpoint{1.392849in}{1.878832in}}%
\pgfpathlineto{\pgfqpoint{1.400580in}{1.873766in}}%
\pgfpathlineto{\pgfqpoint{1.403157in}{1.880333in}}%
\pgfpathlineto{\pgfqpoint{1.408312in}{1.878269in}}%
\pgfpathlineto{\pgfqpoint{1.410889in}{1.869263in}}%
\pgfpathlineto{\pgfqpoint{1.418621in}{1.868700in}}%
\pgfpathlineto{\pgfqpoint{1.421198in}{1.857629in}}%
\pgfpathlineto{\pgfqpoint{1.423775in}{1.851250in}}%
\pgfpathlineto{\pgfqpoint{1.426352in}{1.828734in}}%
\pgfpathlineto{\pgfqpoint{1.428930in}{1.847122in}}%
\pgfpathlineto{\pgfqpoint{1.436661in}{1.855190in}}%
\pgfpathlineto{\pgfqpoint{1.439238in}{1.869075in}}%
\pgfpathlineto{\pgfqpoint{1.441816in}{1.898721in}}%
\pgfpathlineto{\pgfqpoint{1.444393in}{1.898158in}}%
\pgfpathlineto{\pgfqpoint{1.446970in}{1.885024in}}%
\pgfpathlineto{\pgfqpoint{1.454702in}{1.875079in}}%
\pgfpathlineto{\pgfqpoint{1.457279in}{1.857442in}}%
\pgfpathlineto{\pgfqpoint{1.459856in}{1.865510in}}%
\pgfpathlineto{\pgfqpoint{1.462433in}{1.894406in}}%
\pgfpathlineto{\pgfqpoint{1.465010in}{1.912043in}}%
\pgfpathlineto{\pgfqpoint{1.472742in}{1.905288in}}%
\pgfpathlineto{\pgfqpoint{1.477897in}{1.903975in}}%
\pgfpathlineto{\pgfqpoint{1.480474in}{1.886337in}}%
\pgfpathlineto{\pgfqpoint{1.483051in}{1.911856in}}%
\pgfpathlineto{\pgfqpoint{1.490783in}{1.907352in}}%
\pgfpathlineto{\pgfqpoint{1.493360in}{1.911856in}}%
\pgfpathlineto{\pgfqpoint{1.498514in}{1.910167in}}%
\pgfpathlineto{\pgfqpoint{1.501091in}{1.921050in}}%
\pgfpathlineto{\pgfqpoint{1.508823in}{1.922926in}}%
\pgfpathlineto{\pgfqpoint{1.511400in}{1.921237in}}%
\pgfpathlineto{\pgfqpoint{1.513978in}{1.924990in}}%
\pgfpathlineto{\pgfqpoint{1.516555in}{1.943941in}}%
\pgfpathlineto{\pgfqpoint{1.519132in}{1.951822in}}%
\pgfpathlineto{\pgfqpoint{1.526864in}{1.946380in}}%
\pgfpathlineto{\pgfqpoint{1.529441in}{1.935685in}}%
\pgfpathlineto{\pgfqpoint{1.532018in}{1.936623in}}%
\pgfpathlineto{\pgfqpoint{1.534595in}{1.925553in}}%
\pgfpathlineto{\pgfqpoint{1.537172in}{1.938875in}}%
\pgfpathlineto{\pgfqpoint{1.544904in}{1.935122in}}%
\pgfpathlineto{\pgfqpoint{1.547481in}{1.930619in}}%
\pgfpathlineto{\pgfqpoint{1.550058in}{1.932683in}}%
\pgfpathlineto{\pgfqpoint{1.552636in}{1.921988in}}%
\pgfpathlineto{\pgfqpoint{1.555213in}{1.935310in}}%
\pgfpathlineto{\pgfqpoint{1.565522in}{1.920674in}}%
\pgfpathlineto{\pgfqpoint{1.568099in}{1.921800in}}%
\pgfpathlineto{\pgfqpoint{1.570676in}{1.946005in}}%
\pgfpathlineto{\pgfqpoint{1.573253in}{1.938687in}}%
\pgfpathlineto{\pgfqpoint{1.580985in}{1.904726in}}%
\pgfpathlineto{\pgfqpoint{1.583562in}{1.912606in}}%
\pgfpathlineto{\pgfqpoint{1.586139in}{1.906977in}}%
\pgfpathlineto{\pgfqpoint{1.588717in}{1.926679in}}%
\pgfpathlineto{\pgfqpoint{1.591294in}{1.957075in}}%
\pgfpathlineto{\pgfqpoint{1.599026in}{1.953886in}}%
\pgfpathlineto{\pgfqpoint{1.601603in}{1.948444in}}%
\pgfpathlineto{\pgfqpoint{1.604180in}{1.951634in}}%
\pgfpathlineto{\pgfqpoint{1.606757in}{1.950883in}}%
\pgfpathlineto{\pgfqpoint{1.609334in}{1.944879in}}%
\pgfpathlineto{\pgfqpoint{1.617066in}{1.953135in}}%
\pgfpathlineto{\pgfqpoint{1.619643in}{1.938875in}}%
\pgfpathlineto{\pgfqpoint{1.622220in}{1.935122in}}%
\pgfpathlineto{\pgfqpoint{1.624798in}{1.938687in}}%
\pgfpathlineto{\pgfqpoint{1.627375in}{1.932495in}}%
\pgfpathlineto{\pgfqpoint{1.637684in}{1.950133in}}%
\pgfpathlineto{\pgfqpoint{1.640261in}{1.953886in}}%
\pgfpathlineto{\pgfqpoint{1.645415in}{1.972649in}}%
\pgfpathlineto{\pgfqpoint{1.653147in}{1.979216in}}%
\pgfpathlineto{\pgfqpoint{1.655724in}{1.956512in}}%
\pgfpathlineto{\pgfqpoint{1.658301in}{1.946005in}}%
\pgfpathlineto{\pgfqpoint{1.660879in}{1.938875in}}%
\pgfpathlineto{\pgfqpoint{1.663456in}{1.937749in}}%
\pgfpathlineto{\pgfqpoint{1.671187in}{1.938312in}}%
\pgfpathlineto{\pgfqpoint{1.673765in}{1.961016in}}%
\pgfpathlineto{\pgfqpoint{1.676342in}{1.970210in}}%
\pgfpathlineto{\pgfqpoint{1.678919in}{1.969084in}}%
\pgfpathlineto{\pgfqpoint{1.681496in}{1.940751in}}%
\pgfpathlineto{\pgfqpoint{1.689228in}{1.934184in}}%
\pgfpathlineto{\pgfqpoint{1.691805in}{1.873954in}}%
\pgfpathlineto{\pgfqpoint{1.694382in}{1.868700in}}%
\pgfpathlineto{\pgfqpoint{1.696960in}{1.859506in}}%
\pgfpathlineto{\pgfqpoint{1.699537in}{1.862883in}}%
\pgfpathlineto{\pgfqpoint{1.712423in}{1.856128in}}%
\pgfpathlineto{\pgfqpoint{1.715000in}{1.882209in}}%
\pgfpathlineto{\pgfqpoint{1.717577in}{1.877894in}}%
\pgfpathlineto{\pgfqpoint{1.725309in}{1.887838in}}%
\pgfpathlineto{\pgfqpoint{1.727886in}{1.907728in}}%
\pgfpathlineto{\pgfqpoint{1.730463in}{1.884273in}}%
\pgfpathlineto{\pgfqpoint{1.733041in}{1.871139in}}%
\pgfpathlineto{\pgfqpoint{1.735618in}{1.875267in}}%
\pgfpathlineto{\pgfqpoint{1.743349in}{1.881084in}}%
\pgfpathlineto{\pgfqpoint{1.745927in}{1.879583in}}%
\pgfpathlineto{\pgfqpoint{1.748504in}{1.851625in}}%
\pgfpathlineto{\pgfqpoint{1.751081in}{1.862883in}}%
\pgfpathlineto{\pgfqpoint{1.753658in}{1.870389in}}%
\pgfpathlineto{\pgfqpoint{1.761390in}{1.887276in}}%
\pgfpathlineto{\pgfqpoint{1.763967in}{1.888026in}}%
\pgfpathlineto{\pgfqpoint{1.766544in}{1.886713in}}%
\pgfpathlineto{\pgfqpoint{1.771699in}{1.908103in}}%
\pgfpathlineto{\pgfqpoint{1.779430in}{1.905101in}}%
\pgfpathlineto{\pgfqpoint{1.782008in}{1.908478in}}%
\pgfpathlineto{\pgfqpoint{1.784585in}{1.918423in}}%
\pgfpathlineto{\pgfqpoint{1.787162in}{1.913920in}}%
\pgfpathlineto{\pgfqpoint{1.789739in}{1.918798in}}%
\pgfpathlineto{\pgfqpoint{1.800048in}{1.905476in}}%
\pgfpathlineto{\pgfqpoint{1.802625in}{1.915608in}}%
\pgfpathlineto{\pgfqpoint{1.805203in}{1.918798in}}%
\pgfpathlineto{\pgfqpoint{1.807780in}{1.927617in}}%
\pgfpathlineto{\pgfqpoint{1.815511in}{1.933433in}}%
\pgfpathlineto{\pgfqpoint{1.818089in}{1.962141in}}%
\pgfpathlineto{\pgfqpoint{1.823243in}{1.943378in}}%
\pgfpathlineto{\pgfqpoint{1.825820in}{1.940001in}}%
\pgfpathlineto{\pgfqpoint{1.833552in}{1.951822in}}%
\pgfpathlineto{\pgfqpoint{1.836129in}{1.964956in}}%
\pgfpathlineto{\pgfqpoint{1.838706in}{1.951634in}}%
\pgfpathlineto{\pgfqpoint{1.841283in}{1.969459in}}%
\pgfpathlineto{\pgfqpoint{1.843861in}{1.952947in}}%
\pgfpathlineto{\pgfqpoint{1.851592in}{1.954448in}}%
\pgfpathlineto{\pgfqpoint{1.856747in}{1.952572in}}%
\pgfpathlineto{\pgfqpoint{1.859324in}{1.945630in}}%
\pgfpathlineto{\pgfqpoint{1.861901in}{1.931932in}}%
\pgfpathlineto{\pgfqpoint{1.869633in}{1.949007in}}%
\pgfpathlineto{\pgfqpoint{1.874787in}{1.979779in}}%
\pgfpathlineto{\pgfqpoint{1.877364in}{1.977903in}}%
\pgfpathlineto{\pgfqpoint{1.879942in}{1.989161in}}%
\pgfpathlineto{\pgfqpoint{1.890251in}{1.991225in}}%
\pgfpathlineto{\pgfqpoint{1.892828in}{2.005672in}}%
\pgfpathlineto{\pgfqpoint{1.895405in}{2.013365in}}%
\pgfpathlineto{\pgfqpoint{1.897982in}{2.003608in}}%
\pgfpathlineto{\pgfqpoint{1.910868in}{2.024623in}}%
\pgfpathlineto{\pgfqpoint{1.913445in}{2.032316in}}%
\pgfpathlineto{\pgfqpoint{1.916023in}{2.042824in}}%
\pgfpathlineto{\pgfqpoint{1.931486in}{2.057647in}}%
\pgfpathlineto{\pgfqpoint{1.934063in}{2.072282in}}%
\pgfpathlineto{\pgfqpoint{1.941795in}{2.073220in}}%
\pgfpathlineto{\pgfqpoint{1.944372in}{2.091796in}}%
\pgfpathlineto{\pgfqpoint{1.946949in}{2.075660in}}%
\pgfpathlineto{\pgfqpoint{1.949526in}{2.071719in}}%
\pgfpathlineto{\pgfqpoint{1.952104in}{2.087856in}}%
\pgfpathlineto{\pgfqpoint{1.959835in}{2.075284in}}%
\pgfpathlineto{\pgfqpoint{1.962412in}{2.086730in}}%
\pgfpathlineto{\pgfqpoint{1.964990in}{2.105869in}}%
\pgfpathlineto{\pgfqpoint{1.967567in}{2.098363in}}%
\pgfpathlineto{\pgfqpoint{1.970144in}{2.105306in}}%
\pgfpathlineto{\pgfqpoint{1.977876in}{2.104743in}}%
\pgfpathlineto{\pgfqpoint{1.980453in}{2.118065in}}%
\pgfpathlineto{\pgfqpoint{1.983030in}{2.118628in}}%
\pgfpathlineto{\pgfqpoint{1.985607in}{2.117314in}}%
\pgfpathlineto{\pgfqpoint{1.988185in}{2.124632in}}%
\pgfpathlineto{\pgfqpoint{1.998493in}{2.139830in}}%
\pgfpathlineto{\pgfqpoint{2.001071in}{2.123319in}}%
\pgfpathlineto{\pgfqpoint{2.003648in}{2.116376in}}%
\pgfpathlineto{\pgfqpoint{2.006225in}{2.129511in}}%
\pgfpathlineto{\pgfqpoint{2.013957in}{2.100803in}}%
\pgfpathlineto{\pgfqpoint{2.016534in}{2.109809in}}%
\pgfpathlineto{\pgfqpoint{2.019111in}{2.130073in}}%
\pgfpathlineto{\pgfqpoint{2.021688in}{2.136828in}}%
\pgfpathlineto{\pgfqpoint{2.024266in}{2.133263in}}%
\pgfpathlineto{\pgfqpoint{2.031997in}{2.125383in}}%
\pgfpathlineto{\pgfqpoint{2.034574in}{2.144146in}}%
\pgfpathlineto{\pgfqpoint{2.037152in}{2.147523in}}%
\pgfpathlineto{\pgfqpoint{2.039729in}{2.145459in}}%
\pgfpathlineto{\pgfqpoint{2.042306in}{2.164223in}}%
\pgfpathlineto{\pgfqpoint{2.050038in}{2.165911in}}%
\pgfpathlineto{\pgfqpoint{2.052615in}{2.155029in}}%
\pgfpathlineto{\pgfqpoint{2.055192in}{2.154278in}}%
\pgfpathlineto{\pgfqpoint{2.057769in}{2.169289in}}%
\pgfpathlineto{\pgfqpoint{2.060347in}{2.175293in}}%
\pgfpathlineto{\pgfqpoint{2.068078in}{2.159532in}}%
\pgfpathlineto{\pgfqpoint{2.070655in}{2.155779in}}%
\pgfpathlineto{\pgfqpoint{2.073233in}{2.163472in}}%
\pgfpathlineto{\pgfqpoint{2.075810in}{2.152027in}}%
\pgfpathlineto{\pgfqpoint{2.078387in}{2.175668in}}%
\pgfpathlineto{\pgfqpoint{2.086119in}{2.155592in}}%
\pgfpathlineto{\pgfqpoint{2.088696in}{2.170039in}}%
\pgfpathlineto{\pgfqpoint{2.091273in}{2.157656in}}%
\pgfpathlineto{\pgfqpoint{2.093850in}{2.173980in}}%
\pgfpathlineto{\pgfqpoint{2.104159in}{2.163285in}}%
\pgfpathlineto{\pgfqpoint{2.106736in}{2.177357in}}%
\pgfpathlineto{\pgfqpoint{2.109314in}{2.163847in}}%
\pgfpathlineto{\pgfqpoint{2.111891in}{2.166099in}}%
\pgfpathlineto{\pgfqpoint{2.114468in}{2.165349in}}%
\pgfpathlineto{\pgfqpoint{2.122200in}{2.164973in}}%
\pgfpathlineto{\pgfqpoint{2.124777in}{2.168163in}}%
\pgfpathlineto{\pgfqpoint{2.127354in}{2.196120in}}%
\pgfpathlineto{\pgfqpoint{2.129931in}{2.204939in}}%
\pgfpathlineto{\pgfqpoint{2.132509in}{2.195933in}}%
\pgfpathlineto{\pgfqpoint{2.140240in}{2.165161in}}%
\pgfpathlineto{\pgfqpoint{2.142817in}{2.171353in}}%
\pgfpathlineto{\pgfqpoint{2.145395in}{2.160658in}}%
\pgfpathlineto{\pgfqpoint{2.147972in}{2.152589in}}%
\pgfpathlineto{\pgfqpoint{2.150549in}{2.164223in}}%
\pgfpathlineto{\pgfqpoint{2.158281in}{2.165724in}}%
\pgfpathlineto{\pgfqpoint{2.160858in}{2.191242in}}%
\pgfpathlineto{\pgfqpoint{2.163435in}{2.198935in}}%
\pgfpathlineto{\pgfqpoint{2.166012in}{2.151088in}}%
\pgfpathlineto{\pgfqpoint{2.168589in}{2.133451in}}%
\pgfpathlineto{\pgfqpoint{2.176321in}{2.134201in}}%
\pgfpathlineto{\pgfqpoint{2.178898in}{2.148274in}}%
\pgfpathlineto{\pgfqpoint{2.181476in}{2.145647in}}%
\pgfpathlineto{\pgfqpoint{2.186630in}{2.198372in}}%
\pgfpathlineto{\pgfqpoint{2.194362in}{2.198372in}}%
\pgfpathlineto{\pgfqpoint{2.199516in}{2.204189in}}%
\pgfpathlineto{\pgfqpoint{2.202093in}{2.231771in}}%
\pgfpathlineto{\pgfqpoint{2.204670in}{2.240777in}}%
\pgfpathlineto{\pgfqpoint{2.212402in}{2.241153in}}%
\pgfpathlineto{\pgfqpoint{2.214979in}{2.242466in}}%
\pgfpathlineto{\pgfqpoint{2.217557in}{2.257101in}}%
\pgfpathlineto{\pgfqpoint{2.220134in}{2.250159in}}%
\pgfpathlineto{\pgfqpoint{2.222711in}{2.255413in}}%
\pgfpathlineto{\pgfqpoint{2.230443in}{2.260666in}}%
\pgfpathlineto{\pgfqpoint{2.233020in}{2.265357in}}%
\pgfpathlineto{\pgfqpoint{2.235597in}{2.255788in}}%
\pgfpathlineto{\pgfqpoint{2.238174in}{2.249784in}}%
\pgfpathlineto{\pgfqpoint{2.240751in}{2.247532in}}%
\pgfpathlineto{\pgfqpoint{2.251060in}{2.268735in}}%
\pgfpathlineto{\pgfqpoint{2.253637in}{2.261417in}}%
\pgfpathlineto{\pgfqpoint{2.256215in}{2.266108in}}%
\pgfpathlineto{\pgfqpoint{2.258792in}{2.247532in}}%
\pgfpathlineto{\pgfqpoint{2.266524in}{2.252973in}}%
\pgfpathlineto{\pgfqpoint{2.269101in}{2.243404in}}%
\pgfpathlineto{\pgfqpoint{2.271678in}{2.220325in}}%
\pgfpathlineto{\pgfqpoint{2.274255in}{2.221639in}}%
\pgfpathlineto{\pgfqpoint{2.276832in}{2.261042in}}%
\pgfpathlineto{\pgfqpoint{2.284564in}{2.256163in}}%
\pgfpathlineto{\pgfqpoint{2.287141in}{2.246782in}}%
\pgfpathlineto{\pgfqpoint{2.289718in}{2.227455in}}%
\pgfpathlineto{\pgfqpoint{2.292296in}{2.262543in}}%
\pgfpathlineto{\pgfqpoint{2.294873in}{2.259728in}}%
\pgfpathlineto{\pgfqpoint{2.302605in}{2.273988in}}%
\pgfpathlineto{\pgfqpoint{2.305182in}{2.291063in}}%
\pgfpathlineto{\pgfqpoint{2.307759in}{2.268547in}}%
\pgfpathlineto{\pgfqpoint{2.310336in}{2.223515in}}%
\pgfpathlineto{\pgfqpoint{2.312913in}{2.236462in}}%
\pgfpathlineto{\pgfqpoint{2.320645in}{2.202875in}}%
\pgfpathlineto{\pgfqpoint{2.323222in}{2.214696in}}%
\pgfpathlineto{\pgfqpoint{2.325799in}{2.237963in}}%
\pgfpathlineto{\pgfqpoint{2.328377in}{2.246969in}}%
\pgfpathlineto{\pgfqpoint{2.330954in}{2.232709in}}%
\pgfpathlineto{\pgfqpoint{2.338685in}{2.231958in}}%
\pgfpathlineto{\pgfqpoint{2.341263in}{2.222577in}}%
\pgfpathlineto{\pgfqpoint{2.343840in}{2.234210in}}%
\pgfpathlineto{\pgfqpoint{2.348994in}{2.267984in}}%
\pgfpathlineto{\pgfqpoint{2.356726in}{2.277553in}}%
\pgfpathlineto{\pgfqpoint{2.359303in}{2.297067in}}%
\pgfpathlineto{\pgfqpoint{2.361880in}{2.298381in}}%
\pgfpathlineto{\pgfqpoint{2.364458in}{2.316206in}}%
\pgfpathlineto{\pgfqpoint{2.367035in}{2.325963in}}%
\pgfpathlineto{\pgfqpoint{2.374766in}{2.321272in}}%
\pgfpathlineto{\pgfqpoint{2.377344in}{2.313204in}}%
\pgfpathlineto{\pgfqpoint{2.379921in}{2.317895in}}%
\pgfpathlineto{\pgfqpoint{2.385075in}{2.343038in}}%
\pgfpathlineto{\pgfqpoint{2.392807in}{2.344351in}}%
\pgfpathlineto{\pgfqpoint{2.395384in}{2.352044in}}%
\pgfpathlineto{\pgfqpoint{2.397961in}{2.345289in}}%
\pgfpathlineto{\pgfqpoint{2.400539in}{2.348667in}}%
\pgfpathlineto{\pgfqpoint{2.403116in}{2.354483in}}%
\pgfpathlineto{\pgfqpoint{2.410847in}{2.349792in}}%
\pgfpathlineto{\pgfqpoint{2.413425in}{2.353357in}}%
\pgfpathlineto{\pgfqpoint{2.416002in}{2.362927in}}%
\pgfpathlineto{\pgfqpoint{2.418579in}{2.378688in}}%
\pgfpathlineto{\pgfqpoint{2.421156in}{2.376249in}}%
\pgfpathlineto{\pgfqpoint{2.428888in}{2.370807in}}%
\pgfpathlineto{\pgfqpoint{2.431465in}{2.364053in}}%
\pgfpathlineto{\pgfqpoint{2.434042in}{2.369306in}}%
\pgfpathlineto{\pgfqpoint{2.436620in}{2.383004in}}%
\pgfpathlineto{\pgfqpoint{2.439197in}{2.377187in}}%
\pgfpathlineto{\pgfqpoint{2.446928in}{2.378313in}}%
\pgfpathlineto{\pgfqpoint{2.449506in}{2.383004in}}%
\pgfpathlineto{\pgfqpoint{2.452083in}{2.365929in}}%
\pgfpathlineto{\pgfqpoint{2.454660in}{2.337409in}}%
\pgfpathlineto{\pgfqpoint{2.457237in}{2.338347in}}%
\pgfpathlineto{\pgfqpoint{2.464969in}{2.333656in}}%
\pgfpathlineto{\pgfqpoint{2.467546in}{2.330466in}}%
\pgfpathlineto{\pgfqpoint{2.470123in}{2.308325in}}%
\pgfpathlineto{\pgfqpoint{2.472701in}{2.328590in}}%
\pgfpathlineto{\pgfqpoint{2.475278in}{2.324274in}}%
\pgfpathlineto{\pgfqpoint{2.483009in}{2.323148in}}%
\pgfpathlineto{\pgfqpoint{2.485587in}{2.297255in}}%
\pgfpathlineto{\pgfqpoint{2.493318in}{2.310952in}}%
\pgfpathlineto{\pgfqpoint{2.503627in}{2.305136in}}%
\pgfpathlineto{\pgfqpoint{2.506204in}{2.326713in}}%
\pgfpathlineto{\pgfqpoint{2.508782in}{2.331967in}}%
\pgfpathlineto{\pgfqpoint{2.511359in}{2.334594in}}%
\pgfpathlineto{\pgfqpoint{2.519090in}{2.362176in}}%
\pgfpathlineto{\pgfqpoint{2.521668in}{2.376061in}}%
\pgfpathlineto{\pgfqpoint{2.524245in}{2.394262in}}%
\pgfpathlineto{\pgfqpoint{2.526822in}{2.385818in}}%
\pgfpathlineto{\pgfqpoint{2.529399in}{2.392385in}}%
\pgfpathlineto{\pgfqpoint{2.537131in}{2.402517in}}%
\pgfpathlineto{\pgfqpoint{2.539708in}{2.413963in}}%
\pgfpathlineto{\pgfqpoint{2.542285in}{2.435916in}}%
\pgfpathlineto{\pgfqpoint{2.544862in}{2.440607in}}%
\pgfpathlineto{\pgfqpoint{2.547440in}{2.415276in}}%
\pgfpathlineto{\pgfqpoint{2.555171in}{2.433102in}}%
\pgfpathlineto{\pgfqpoint{2.557749in}{2.428036in}}%
\pgfpathlineto{\pgfqpoint{2.560326in}{2.418279in}}%
\pgfpathlineto{\pgfqpoint{2.562903in}{2.425784in}}%
\pgfpathlineto{\pgfqpoint{2.565480in}{2.418654in}}%
\pgfpathlineto{\pgfqpoint{2.573212in}{2.405520in}}%
\pgfpathlineto{\pgfqpoint{2.575789in}{2.408897in}}%
\pgfpathlineto{\pgfqpoint{2.578366in}{2.402142in}}%
\pgfpathlineto{\pgfqpoint{2.580943in}{2.391635in}}%
\pgfpathlineto{\pgfqpoint{2.583521in}{2.405332in}}%
\pgfpathlineto{\pgfqpoint{2.591252in}{2.393886in}}%
\pgfpathlineto{\pgfqpoint{2.593830in}{2.369119in}}%
\pgfpathlineto{\pgfqpoint{2.596407in}{2.375686in}}%
\pgfpathlineto{\pgfqpoint{2.598984in}{2.404394in}}%
\pgfpathlineto{\pgfqpoint{2.601561in}{2.426722in}}%
\pgfpathlineto{\pgfqpoint{2.609293in}{2.437793in}}%
\pgfpathlineto{\pgfqpoint{2.611870in}{2.412087in}}%
\pgfpathlineto{\pgfqpoint{2.614447in}{2.430287in}}%
\pgfpathlineto{\pgfqpoint{2.617024in}{2.457494in}}%
\pgfpathlineto{\pgfqpoint{2.619602in}{2.461059in}}%
\pgfpathlineto{\pgfqpoint{2.627333in}{2.467814in}}%
\pgfpathlineto{\pgfqpoint{2.629911in}{2.476633in}}%
\pgfpathlineto{\pgfqpoint{2.632488in}{2.466876in}}%
\pgfpathlineto{\pgfqpoint{2.635065in}{2.471754in}}%
\pgfpathlineto{\pgfqpoint{2.637642in}{2.486765in}}%
\pgfpathlineto{\pgfqpoint{2.645374in}{2.494833in}}%
\pgfpathlineto{\pgfqpoint{2.647951in}{2.500650in}}%
\pgfpathlineto{\pgfqpoint{2.650528in}{2.492769in}}%
\pgfpathlineto{\pgfqpoint{2.653105in}{2.509844in}}%
\pgfpathlineto{\pgfqpoint{2.655683in}{2.510782in}}%
\pgfpathlineto{\pgfqpoint{2.663414in}{2.517537in}}%
\pgfpathlineto{\pgfqpoint{2.665991in}{2.514347in}}%
\pgfpathlineto{\pgfqpoint{2.668569in}{2.530671in}}%
\pgfpathlineto{\pgfqpoint{2.671146in}{2.519038in}}%
\pgfpathlineto{\pgfqpoint{2.673723in}{2.544556in}}%
\pgfpathlineto{\pgfqpoint{2.681455in}{2.543993in}}%
\pgfpathlineto{\pgfqpoint{2.684032in}{2.550560in}}%
\pgfpathlineto{\pgfqpoint{2.686609in}{2.554313in}}%
\pgfpathlineto{\pgfqpoint{2.689186in}{2.573827in}}%
\pgfpathlineto{\pgfqpoint{2.691764in}{2.574765in}}%
\pgfpathlineto{\pgfqpoint{2.699495in}{2.579268in}}%
\pgfpathlineto{\pgfqpoint{2.702072in}{2.577955in}}%
\pgfpathlineto{\pgfqpoint{2.704650in}{2.574202in}}%
\pgfpathlineto{\pgfqpoint{2.707227in}{2.593341in}}%
\pgfpathlineto{\pgfqpoint{2.709804in}{2.604223in}}%
\pgfpathlineto{\pgfqpoint{2.717536in}{2.607601in}}%
\pgfpathlineto{\pgfqpoint{2.720113in}{2.622612in}}%
\pgfpathlineto{\pgfqpoint{2.722690in}{2.644565in}}%
\pgfpathlineto{\pgfqpoint{2.727845in}{2.644752in}}%
\pgfpathlineto{\pgfqpoint{2.735576in}{2.549810in}}%
\pgfpathlineto{\pgfqpoint{2.738153in}{2.532172in}}%
\pgfpathlineto{\pgfqpoint{2.740731in}{2.529921in}}%
\pgfpathlineto{\pgfqpoint{2.743308in}{2.535925in}}%
\pgfpathlineto{\pgfqpoint{2.745885in}{2.564820in}}%
\pgfpathlineto{\pgfqpoint{2.753617in}{2.564258in}}%
\pgfpathlineto{\pgfqpoint{2.756194in}{2.547558in}}%
\pgfpathlineto{\pgfqpoint{2.758771in}{2.534987in}}%
\pgfpathlineto{\pgfqpoint{2.763926in}{2.529358in}}%
\pgfpathlineto{\pgfqpoint{2.771657in}{2.549434in}}%
\pgfpathlineto{\pgfqpoint{2.776812in}{2.682091in}}%
\pgfpathlineto{\pgfqpoint{2.779389in}{2.692036in}}%
\pgfpathlineto{\pgfqpoint{2.781966in}{2.697102in}}%
\pgfpathlineto{\pgfqpoint{2.789698in}{2.698416in}}%
\pgfpathlineto{\pgfqpoint{2.792275in}{2.701418in}}%
\pgfpathlineto{\pgfqpoint{2.797429in}{2.722620in}}%
\pgfpathlineto{\pgfqpoint{2.800007in}{2.739883in}}%
\pgfpathlineto{\pgfqpoint{2.807738in}{2.741008in}}%
\pgfpathlineto{\pgfqpoint{2.810315in}{2.754518in}}%
\pgfpathlineto{\pgfqpoint{2.815470in}{2.719993in}}%
\pgfpathlineto{\pgfqpoint{2.818047in}{2.725247in}}%
\pgfpathlineto{\pgfqpoint{2.825779in}{2.711925in}}%
\pgfpathlineto{\pgfqpoint{2.828356in}{2.712113in}}%
\pgfpathlineto{\pgfqpoint{2.830933in}{2.695601in}}%
\pgfpathlineto{\pgfqpoint{2.833510in}{2.692599in}}%
\pgfpathlineto{\pgfqpoint{2.836088in}{2.688283in}}%
\pgfpathlineto{\pgfqpoint{2.843819in}{2.663891in}}%
\pgfpathlineto{\pgfqpoint{2.846396in}{2.708360in}}%
\pgfpathlineto{\pgfqpoint{2.848974in}{2.725060in}}%
\pgfpathlineto{\pgfqpoint{2.851551in}{2.720556in}}%
\pgfpathlineto{\pgfqpoint{2.854128in}{2.706671in}}%
\pgfpathlineto{\pgfqpoint{2.864437in}{2.701418in}}%
\pgfpathlineto{\pgfqpoint{2.867014in}{2.693162in}}%
\pgfpathlineto{\pgfqpoint{2.869591in}{2.664266in}}%
\pgfpathlineto{\pgfqpoint{2.872168in}{2.591089in}}%
\pgfpathlineto{\pgfqpoint{2.879900in}{2.570637in}}%
\pgfpathlineto{\pgfqpoint{2.882477in}{2.584522in}}%
\pgfpathlineto{\pgfqpoint{2.885055in}{2.591652in}}%
\pgfpathlineto{\pgfqpoint{2.887632in}{2.555814in}}%
\pgfpathlineto{\pgfqpoint{2.890209in}{2.558066in}}%
\pgfpathlineto{\pgfqpoint{2.897941in}{2.488078in}}%
\pgfpathlineto{\pgfqpoint{2.900518in}{2.534049in}}%
\pgfpathlineto{\pgfqpoint{2.903095in}{2.544556in}}%
\pgfpathlineto{\pgfqpoint{2.908249in}{2.592965in}}%
\pgfpathlineto{\pgfqpoint{2.915981in}{2.582646in}}%
\pgfpathlineto{\pgfqpoint{2.918558in}{2.603473in}}%
\pgfpathlineto{\pgfqpoint{2.921136in}{2.608727in}}%
\pgfpathlineto{\pgfqpoint{2.923713in}{2.603848in}}%
\pgfpathlineto{\pgfqpoint{2.926290in}{2.636309in}}%
\pgfpathlineto{\pgfqpoint{2.936599in}{2.631055in}}%
\pgfpathlineto{\pgfqpoint{2.939176in}{2.610603in}}%
\pgfpathlineto{\pgfqpoint{2.941753in}{2.627115in}}%
\pgfpathlineto{\pgfqpoint{2.944330in}{2.627302in}}%
\pgfpathlineto{\pgfqpoint{2.952062in}{2.637622in}}%
\pgfpathlineto{\pgfqpoint{2.954639in}{2.649443in}}%
\pgfpathlineto{\pgfqpoint{2.957216in}{2.648317in}}%
\pgfpathlineto{\pgfqpoint{2.959794in}{2.672710in}}%
\pgfpathlineto{\pgfqpoint{2.962371in}{2.679089in}}%
\pgfpathlineto{\pgfqpoint{2.970103in}{2.637810in}}%
\pgfpathlineto{\pgfqpoint{2.972680in}{2.645503in}}%
\pgfpathlineto{\pgfqpoint{2.975257in}{2.664642in}}%
\pgfpathlineto{\pgfqpoint{2.977834in}{2.668394in}}%
\pgfpathlineto{\pgfqpoint{2.980411in}{2.668957in}}%
\pgfpathlineto{\pgfqpoint{2.988143in}{2.659763in}}%
\pgfpathlineto{\pgfqpoint{2.990720in}{2.642688in}}%
\pgfpathlineto{\pgfqpoint{2.993297in}{2.643627in}}%
\pgfpathlineto{\pgfqpoint{2.995875in}{2.614731in}}%
\pgfpathlineto{\pgfqpoint{2.998452in}{2.598594in}}%
\pgfpathlineto{\pgfqpoint{3.006184in}{2.638748in}}%
\pgfpathlineto{\pgfqpoint{3.008761in}{2.646441in}}%
\pgfpathlineto{\pgfqpoint{3.011338in}{2.621673in}}%
\pgfpathlineto{\pgfqpoint{3.013915in}{2.641375in}}%
\pgfpathlineto{\pgfqpoint{3.016492in}{2.652633in}}%
\pgfpathlineto{\pgfqpoint{3.024224in}{2.641187in}}%
\pgfpathlineto{\pgfqpoint{3.026801in}{2.668019in}}%
\pgfpathlineto{\pgfqpoint{3.029378in}{2.652070in}}%
\pgfpathlineto{\pgfqpoint{3.031956in}{2.648130in}}%
\pgfpathlineto{\pgfqpoint{3.034533in}{2.670271in}}%
\pgfpathlineto{\pgfqpoint{3.042264in}{2.694288in}}%
\pgfpathlineto{\pgfqpoint{3.044842in}{2.708548in}}%
\pgfpathlineto{\pgfqpoint{3.047419in}{2.699917in}}%
\pgfpathlineto{\pgfqpoint{3.049996in}{2.702168in}}%
\pgfpathlineto{\pgfqpoint{3.052573in}{2.697665in}}%
\pgfpathlineto{\pgfqpoint{3.060305in}{2.673648in}}%
\pgfpathlineto{\pgfqpoint{3.062882in}{2.680778in}}%
\pgfpathlineto{\pgfqpoint{3.065459in}{2.697290in}}%
\pgfpathlineto{\pgfqpoint{3.070614in}{2.640624in}}%
\pgfpathlineto{\pgfqpoint{3.078345in}{2.653196in}}%
\pgfpathlineto{\pgfqpoint{3.080923in}{2.668582in}}%
\pgfpathlineto{\pgfqpoint{3.083500in}{2.712488in}}%
\pgfpathlineto{\pgfqpoint{3.086077in}{2.728249in}}%
\pgfpathlineto{\pgfqpoint{3.098963in}{2.747200in}}%
\pgfpathlineto{\pgfqpoint{3.101540in}{2.732565in}}%
\pgfpathlineto{\pgfqpoint{3.104118in}{2.710612in}}%
\pgfpathlineto{\pgfqpoint{3.106695in}{2.709111in}}%
\pgfpathlineto{\pgfqpoint{3.114426in}{2.719993in}}%
\pgfpathlineto{\pgfqpoint{3.117004in}{2.725247in}}%
\pgfpathlineto{\pgfqpoint{3.122158in}{2.778723in}}%
\pgfpathlineto{\pgfqpoint{3.124735in}{2.767465in}}%
\pgfpathlineto{\pgfqpoint{3.132467in}{2.775533in}}%
\pgfpathlineto{\pgfqpoint{3.135044in}{2.753767in}}%
\pgfpathlineto{\pgfqpoint{3.137621in}{2.784164in}}%
\pgfpathlineto{\pgfqpoint{3.140199in}{2.779098in}}%
\pgfpathlineto{\pgfqpoint{3.150507in}{2.811746in}}%
\pgfpathlineto{\pgfqpoint{3.153085in}{2.805367in}}%
\pgfpathlineto{\pgfqpoint{3.155662in}{2.791669in}}%
\pgfpathlineto{\pgfqpoint{3.158239in}{2.781537in}}%
\pgfpathlineto{\pgfqpoint{3.160816in}{2.783976in}}%
\pgfpathlineto{\pgfqpoint{3.168548in}{2.792795in}}%
\pgfpathlineto{\pgfqpoint{3.171125in}{2.769153in}}%
\pgfpathlineto{\pgfqpoint{3.173702in}{2.790356in}}%
\pgfpathlineto{\pgfqpoint{3.176280in}{2.784727in}}%
\pgfpathlineto{\pgfqpoint{3.178857in}{2.798424in}}%
\pgfpathlineto{\pgfqpoint{3.189166in}{2.803303in}}%
\pgfpathlineto{\pgfqpoint{3.191743in}{2.803115in}}%
\pgfpathlineto{\pgfqpoint{3.194320in}{2.819064in}}%
\pgfpathlineto{\pgfqpoint{3.196897in}{2.821691in}}%
\pgfpathlineto{\pgfqpoint{3.204629in}{2.817751in}}%
\pgfpathlineto{\pgfqpoint{3.207206in}{2.827320in}}%
\pgfpathlineto{\pgfqpoint{3.209783in}{2.816812in}}%
\pgfpathlineto{\pgfqpoint{3.212361in}{2.840829in}}%
\pgfpathlineto{\pgfqpoint{3.214938in}{2.856028in}}%
\pgfpathlineto{\pgfqpoint{3.222669in}{2.867286in}}%
\pgfpathlineto{\pgfqpoint{3.225247in}{2.861469in}}%
\pgfpathlineto{\pgfqpoint{3.227824in}{2.852275in}}%
\pgfpathlineto{\pgfqpoint{3.230401in}{2.830134in}}%
\pgfpathlineto{\pgfqpoint{3.232978in}{2.835013in}}%
\pgfpathlineto{\pgfqpoint{3.240710in}{2.834262in}}%
\pgfpathlineto{\pgfqpoint{3.243287in}{2.840454in}}%
\pgfpathlineto{\pgfqpoint{3.245864in}{2.850962in}}%
\pgfpathlineto{\pgfqpoint{3.248441in}{2.853589in}}%
\pgfpathlineto{\pgfqpoint{3.251019in}{2.864659in}}%
\pgfpathlineto{\pgfqpoint{3.258750in}{2.847021in}}%
\pgfpathlineto{\pgfqpoint{3.261328in}{2.830510in}}%
\pgfpathlineto{\pgfqpoint{3.263905in}{2.839704in}}%
\pgfpathlineto{\pgfqpoint{3.266482in}{2.841017in}}%
\pgfpathlineto{\pgfqpoint{3.269059in}{2.840454in}}%
\pgfpathlineto{\pgfqpoint{3.276791in}{2.832949in}}%
\pgfpathlineto{\pgfqpoint{3.279368in}{2.854714in}}%
\pgfpathlineto{\pgfqpoint{3.281945in}{2.865785in}}%
\pgfpathlineto{\pgfqpoint{3.284522in}{2.868787in}}%
\pgfpathlineto{\pgfqpoint{3.294831in}{2.860531in}}%
\pgfpathlineto{\pgfqpoint{3.297409in}{2.854714in}}%
\pgfpathlineto{\pgfqpoint{3.299986in}{2.856778in}}%
\pgfpathlineto{\pgfqpoint{3.302563in}{2.843832in}}%
\pgfpathlineto{\pgfqpoint{3.305140in}{2.850586in}}%
\pgfpathlineto{\pgfqpoint{3.312872in}{2.862032in}}%
\pgfpathlineto{\pgfqpoint{3.315449in}{2.862970in}}%
\pgfpathlineto{\pgfqpoint{3.318026in}{2.881358in}}%
\pgfpathlineto{\pgfqpoint{3.320603in}{2.838015in}}%
\pgfpathlineto{\pgfqpoint{3.323181in}{2.859218in}}%
\pgfpathlineto{\pgfqpoint{3.330912in}{2.850586in}}%
\pgfpathlineto{\pgfqpoint{3.333490in}{2.864096in}}%
\pgfpathlineto{\pgfqpoint{3.336067in}{2.856778in}}%
\pgfpathlineto{\pgfqpoint{3.338644in}{2.864284in}}%
\pgfpathlineto{\pgfqpoint{3.341221in}{2.864096in}}%
\pgfpathlineto{\pgfqpoint{3.348953in}{2.870851in}}%
\pgfpathlineto{\pgfqpoint{3.351530in}{2.845896in}}%
\pgfpathlineto{\pgfqpoint{3.354107in}{2.841205in}}%
\pgfpathlineto{\pgfqpoint{3.356684in}{2.794296in}}%
\pgfpathlineto{\pgfqpoint{3.359262in}{2.781350in}}%
\pgfpathlineto{\pgfqpoint{3.366993in}{2.791857in}}%
\pgfpathlineto{\pgfqpoint{3.369570in}{2.775908in}}%
\pgfpathlineto{\pgfqpoint{3.372148in}{2.772156in}}%
\pgfpathlineto{\pgfqpoint{3.374725in}{2.765213in}}%
\pgfpathlineto{\pgfqpoint{3.377302in}{2.793546in}}%
\pgfpathlineto{\pgfqpoint{3.385034in}{2.789043in}}%
\pgfpathlineto{\pgfqpoint{3.387611in}{2.794109in}}%
\pgfpathlineto{\pgfqpoint{3.390188in}{2.809307in}}%
\pgfpathlineto{\pgfqpoint{3.392765in}{2.818126in}}%
\pgfpathlineto{\pgfqpoint{3.395343in}{2.811183in}}%
\pgfpathlineto{\pgfqpoint{3.403074in}{2.847584in}}%
\pgfpathlineto{\pgfqpoint{3.405651in}{2.849648in}}%
\pgfpathlineto{\pgfqpoint{3.408229in}{2.869725in}}%
\pgfpathlineto{\pgfqpoint{3.410806in}{2.868036in}}%
\pgfpathlineto{\pgfqpoint{3.413383in}{2.861844in}}%
\pgfpathlineto{\pgfqpoint{3.421115in}{2.871414in}}%
\pgfpathlineto{\pgfqpoint{3.423692in}{2.869725in}}%
\pgfpathlineto{\pgfqpoint{3.426269in}{2.858655in}}%
\pgfpathlineto{\pgfqpoint{3.441732in}{2.863345in}}%
\pgfpathlineto{\pgfqpoint{3.444310in}{2.856778in}}%
\pgfpathlineto{\pgfqpoint{3.446887in}{2.853213in}}%
\pgfpathlineto{\pgfqpoint{3.449464in}{2.862970in}}%
\pgfpathlineto{\pgfqpoint{3.457196in}{2.874416in}}%
\pgfpathlineto{\pgfqpoint{3.459773in}{2.867286in}}%
\pgfpathlineto{\pgfqpoint{3.462350in}{2.870288in}}%
\pgfpathlineto{\pgfqpoint{3.464927in}{2.865597in}}%
\pgfpathlineto{\pgfqpoint{3.467505in}{2.858655in}}%
\pgfpathlineto{\pgfqpoint{3.475236in}{2.867661in}}%
\pgfpathlineto{\pgfqpoint{3.477813in}{2.873853in}}%
\pgfpathlineto{\pgfqpoint{3.480391in}{2.882484in}}%
\pgfpathlineto{\pgfqpoint{3.482968in}{2.906876in}}%
\pgfpathlineto{\pgfqpoint{3.485545in}{2.904437in}}%
\pgfpathlineto{\pgfqpoint{3.493277in}{2.888113in}}%
\pgfpathlineto{\pgfqpoint{3.495854in}{2.865785in}}%
\pgfpathlineto{\pgfqpoint{3.498431in}{2.874416in}}%
\pgfpathlineto{\pgfqpoint{3.501008in}{2.834450in}}%
\pgfpathlineto{\pgfqpoint{3.503586in}{2.833512in}}%
\pgfpathlineto{\pgfqpoint{3.511317in}{2.829759in}}%
\pgfpathlineto{\pgfqpoint{3.513894in}{2.821128in}}%
\pgfpathlineto{\pgfqpoint{3.516472in}{2.779661in}}%
\pgfpathlineto{\pgfqpoint{3.519049in}{2.771217in}}%
\pgfpathlineto{\pgfqpoint{3.521626in}{2.796173in}}%
\pgfpathlineto{\pgfqpoint{3.529358in}{2.798987in}}%
\pgfpathlineto{\pgfqpoint{3.531935in}{2.754893in}}%
\pgfpathlineto{\pgfqpoint{3.534512in}{2.816812in}}%
\pgfpathlineto{\pgfqpoint{3.537089in}{2.770842in}}%
\pgfpathlineto{\pgfqpoint{3.539667in}{2.690910in}}%
\pgfpathlineto{\pgfqpoint{3.547398in}{2.675524in}}%
\pgfpathlineto{\pgfqpoint{3.549975in}{2.696727in}}%
\pgfpathlineto{\pgfqpoint{3.552553in}{2.697477in}}%
\pgfpathlineto{\pgfqpoint{3.555130in}{2.711175in}}%
\pgfpathlineto{\pgfqpoint{3.557707in}{2.750202in}}%
\pgfpathlineto{\pgfqpoint{3.565439in}{2.753580in}}%
\pgfpathlineto{\pgfqpoint{3.568016in}{2.808744in}}%
\pgfpathlineto{\pgfqpoint{3.570593in}{2.775908in}}%
\pgfpathlineto{\pgfqpoint{3.573170in}{2.877230in}}%
\pgfpathlineto{\pgfqpoint{3.575747in}{2.935960in}}%
\pgfpathlineto{\pgfqpoint{3.583479in}{2.952096in}}%
\pgfpathlineto{\pgfqpoint{3.586056in}{2.976864in}}%
\pgfpathlineto{\pgfqpoint{3.588634in}{2.976113in}}%
\pgfpathlineto{\pgfqpoint{3.591211in}{2.994126in}}%
\pgfpathlineto{\pgfqpoint{3.593788in}{3.021896in}}%
\pgfpathlineto{\pgfqpoint{3.601520in}{3.014015in}}%
\pgfpathlineto{\pgfqpoint{3.604097in}{3.043098in}}%
\pgfpathlineto{\pgfqpoint{3.606674in}{3.053794in}}%
\pgfpathlineto{\pgfqpoint{3.609251in}{3.059798in}}%
\pgfpathlineto{\pgfqpoint{3.611828in}{3.068054in}}%
\pgfpathlineto{\pgfqpoint{3.619560in}{3.086442in}}%
\pgfpathlineto{\pgfqpoint{3.622137in}{3.077248in}}%
\pgfpathlineto{\pgfqpoint{3.629869in}{3.106143in}}%
\pgfpathlineto{\pgfqpoint{3.637601in}{3.103516in}}%
\pgfpathlineto{\pgfqpoint{3.640178in}{3.127721in}}%
\pgfpathlineto{\pgfqpoint{3.642755in}{3.120028in}}%
\pgfpathlineto{\pgfqpoint{3.645332in}{3.128847in}}%
\pgfpathlineto{\pgfqpoint{3.647909in}{3.142169in}}%
\pgfpathlineto{\pgfqpoint{3.655641in}{3.137478in}}%
\pgfpathlineto{\pgfqpoint{3.658218in}{3.107269in}}%
\pgfpathlineto{\pgfqpoint{3.660795in}{3.111397in}}%
\pgfpathlineto{\pgfqpoint{3.665950in}{3.141043in}}%
\pgfpathlineto{\pgfqpoint{3.673682in}{3.108770in}}%
\pgfpathlineto{\pgfqpoint{3.676259in}{3.149487in}}%
\pgfpathlineto{\pgfqpoint{3.678836in}{3.177256in}}%
\pgfpathlineto{\pgfqpoint{3.683990in}{3.177444in}}%
\pgfpathlineto{\pgfqpoint{3.691722in}{3.155116in}}%
\pgfpathlineto{\pgfqpoint{3.694299in}{3.153427in}}%
\pgfpathlineto{\pgfqpoint{3.696876in}{3.110271in}}%
\pgfpathlineto{\pgfqpoint{3.699454in}{3.125282in}}%
\pgfpathlineto{\pgfqpoint{3.702031in}{3.091508in}}%
\pgfpathlineto{\pgfqpoint{3.709763in}{3.087005in}}%
\pgfpathlineto{\pgfqpoint{3.712340in}{3.123781in}}%
\pgfpathlineto{\pgfqpoint{3.714917in}{3.149487in}}%
\pgfpathlineto{\pgfqpoint{3.717494in}{3.227918in}}%
\pgfpathlineto{\pgfqpoint{3.720071in}{3.230920in}}%
\pgfpathlineto{\pgfqpoint{3.727803in}{3.260941in}}%
\pgfpathlineto{\pgfqpoint{3.730380in}{3.254186in}}%
\pgfpathlineto{\pgfqpoint{3.732957in}{3.255687in}}%
\pgfpathlineto{\pgfqpoint{3.738112in}{3.244054in}}%
\pgfpathlineto{\pgfqpoint{3.745843in}{3.251559in}}%
\pgfpathlineto{\pgfqpoint{3.748421in}{3.236924in}}%
\pgfpathlineto{\pgfqpoint{3.750998in}{3.211593in}}%
\pgfpathlineto{\pgfqpoint{3.756152in}{3.207278in}}%
\pgfpathlineto{\pgfqpoint{3.763884in}{3.145546in}}%
\pgfpathlineto{\pgfqpoint{3.766461in}{3.117026in}}%
\pgfpathlineto{\pgfqpoint{3.769038in}{3.136165in}}%
\pgfpathlineto{\pgfqpoint{3.771616in}{3.200148in}}%
\pgfpathlineto{\pgfqpoint{3.774193in}{3.166561in}}%
\pgfpathlineto{\pgfqpoint{3.781924in}{3.151926in}}%
\pgfpathlineto{\pgfqpoint{3.784502in}{3.149862in}}%
\pgfpathlineto{\pgfqpoint{3.787079in}{3.136915in}}%
\pgfpathlineto{\pgfqpoint{3.789656in}{3.133913in}}%
\pgfpathlineto{\pgfqpoint{3.792233in}{3.172941in}}%
\pgfpathlineto{\pgfqpoint{3.802542in}{3.172003in}}%
\pgfpathlineto{\pgfqpoint{3.805119in}{3.181947in}}%
\pgfpathlineto{\pgfqpoint{3.807697in}{3.237862in}}%
\pgfpathlineto{\pgfqpoint{3.810274in}{3.206715in}}%
\pgfpathlineto{\pgfqpoint{3.818005in}{3.210280in}}%
\pgfpathlineto{\pgfqpoint{3.820583in}{3.200148in}}%
\pgfpathlineto{\pgfqpoint{3.823160in}{3.205214in}}%
\pgfpathlineto{\pgfqpoint{3.825737in}{3.241802in}}%
\pgfpathlineto{\pgfqpoint{3.828314in}{3.178007in}}%
\pgfpathlineto{\pgfqpoint{3.836046in}{3.213282in}}%
\pgfpathlineto{\pgfqpoint{3.838623in}{3.238613in}}%
\pgfpathlineto{\pgfqpoint{3.841200in}{3.219849in}}%
\pgfpathlineto{\pgfqpoint{3.843778in}{3.248932in}}%
\pgfpathlineto{\pgfqpoint{3.846355in}{3.240864in}}%
\pgfpathlineto{\pgfqpoint{3.854086in}{3.220412in}}%
\pgfpathlineto{\pgfqpoint{3.856664in}{3.236736in}}%
\pgfpathlineto{\pgfqpoint{3.859241in}{3.231483in}}%
\pgfpathlineto{\pgfqpoint{3.861818in}{3.255312in}}%
\pgfpathlineto{\pgfqpoint{3.864395in}{3.255875in}}%
\pgfpathlineto{\pgfqpoint{3.874704in}{3.272387in}}%
\pgfpathlineto{\pgfqpoint{3.877281in}{3.280455in}}%
\pgfpathlineto{\pgfqpoint{3.879859in}{3.275201in}}%
\pgfpathlineto{\pgfqpoint{3.882436in}{3.292463in}}%
\pgfpathlineto{\pgfqpoint{3.890167in}{3.304097in}}%
\pgfpathlineto{\pgfqpoint{3.892745in}{3.309350in}}%
\pgfpathlineto{\pgfqpoint{3.895322in}{3.305410in}}%
\pgfpathlineto{\pgfqpoint{3.897899in}{3.317606in}}%
\pgfpathlineto{\pgfqpoint{3.900476in}{3.301470in}}%
\pgfpathlineto{\pgfqpoint{3.908208in}{3.332429in}}%
\pgfpathlineto{\pgfqpoint{3.910785in}{3.295090in}}%
\pgfpathlineto{\pgfqpoint{3.913362in}{3.276327in}}%
\pgfpathlineto{\pgfqpoint{3.915940in}{3.283269in}}%
\pgfpathlineto{\pgfqpoint{3.918517in}{3.229419in}}%
\pgfpathlineto{\pgfqpoint{3.926248in}{3.263193in}}%
\pgfpathlineto{\pgfqpoint{3.928826in}{3.194143in}}%
\pgfpathlineto{\pgfqpoint{3.931403in}{3.185137in}}%
\pgfpathlineto{\pgfqpoint{3.933980in}{3.231295in}}%
\pgfpathlineto{\pgfqpoint{3.936557in}{3.202212in}}%
\pgfpathlineto{\pgfqpoint{3.944289in}{3.260378in}}%
\pgfpathlineto{\pgfqpoint{3.946866in}{3.227167in}}%
\pgfpathlineto{\pgfqpoint{3.949443in}{3.264881in}}%
\pgfpathlineto{\pgfqpoint{3.952020in}{3.251184in}}%
\pgfpathlineto{\pgfqpoint{3.954598in}{3.265069in}}%
\pgfpathlineto{\pgfqpoint{3.962329in}{3.259065in}}%
\pgfpathlineto{\pgfqpoint{3.964907in}{3.262254in}}%
\pgfpathlineto{\pgfqpoint{3.967484in}{3.201461in}}%
\pgfpathlineto{\pgfqpoint{3.970061in}{3.199585in}}%
\pgfpathlineto{\pgfqpoint{3.972638in}{3.214971in}}%
\pgfpathlineto{\pgfqpoint{3.980370in}{3.257751in}}%
\pgfpathlineto{\pgfqpoint{3.982947in}{3.239363in}}%
\pgfpathlineto{\pgfqpoint{3.985524in}{3.198459in}}%
\pgfpathlineto{\pgfqpoint{3.988101in}{3.203150in}}%
\pgfpathlineto{\pgfqpoint{4.000988in}{3.260378in}}%
\pgfpathlineto{\pgfqpoint{4.003565in}{3.261504in}}%
\pgfpathlineto{\pgfqpoint{4.006142in}{3.270135in}}%
\pgfpathlineto{\pgfqpoint{4.008719in}{3.274826in}}%
\pgfpathlineto{\pgfqpoint{4.016451in}{3.254186in}}%
\pgfpathlineto{\pgfqpoint{4.019028in}{3.256813in}}%
\pgfpathlineto{\pgfqpoint{4.021605in}{3.264318in}}%
\pgfpathlineto{\pgfqpoint{4.024182in}{3.255124in}}%
\pgfpathlineto{\pgfqpoint{4.026760in}{3.184949in}}%
\pgfpathlineto{\pgfqpoint{4.034491in}{3.231858in}}%
\pgfpathlineto{\pgfqpoint{4.037069in}{3.223414in}}%
\pgfpathlineto{\pgfqpoint{4.039646in}{3.234672in}}%
\pgfpathlineto{\pgfqpoint{4.042223in}{3.150425in}}%
\pgfpathlineto{\pgfqpoint{4.044800in}{3.139167in}}%
\pgfpathlineto{\pgfqpoint{4.052532in}{3.122280in}}%
\pgfpathlineto{\pgfqpoint{4.055109in}{3.128096in}}%
\pgfpathlineto{\pgfqpoint{4.057686in}{3.105205in}}%
\pgfpathlineto{\pgfqpoint{4.060263in}{3.095636in}}%
\pgfpathlineto{\pgfqpoint{4.062841in}{3.117214in}}%
\pgfpathlineto{\pgfqpoint{4.070572in}{3.139730in}}%
\pgfpathlineto{\pgfqpoint{4.073149in}{3.121154in}}%
\pgfpathlineto{\pgfqpoint{4.075727in}{3.116463in}}%
\pgfpathlineto{\pgfqpoint{4.078304in}{3.132787in}}%
\pgfpathlineto{\pgfqpoint{4.080881in}{3.166186in}}%
\pgfpathlineto{\pgfqpoint{4.088613in}{3.154553in}}%
\pgfpathlineto{\pgfqpoint{4.091190in}{3.157367in}}%
\pgfpathlineto{\pgfqpoint{4.093767in}{3.178195in}}%
\pgfpathlineto{\pgfqpoint{4.096344in}{3.208967in}}%
\pgfpathlineto{\pgfqpoint{4.098922in}{3.211593in}}%
\pgfpathlineto{\pgfqpoint{4.106653in}{3.204276in}}%
\pgfpathlineto{\pgfqpoint{4.109230in}{3.208591in}}%
\pgfpathlineto{\pgfqpoint{4.111808in}{3.205214in}}%
\pgfpathlineto{\pgfqpoint{4.114385in}{3.208028in}}%
\pgfpathlineto{\pgfqpoint{4.116962in}{3.189640in}}%
\pgfpathlineto{\pgfqpoint{4.127271in}{3.166186in}}%
\pgfpathlineto{\pgfqpoint{4.129848in}{3.190391in}}%
\pgfpathlineto{\pgfqpoint{4.132425in}{3.187764in}}%
\pgfpathlineto{\pgfqpoint{4.135003in}{3.157555in}}%
\pgfpathlineto{\pgfqpoint{4.142734in}{3.155866in}}%
\pgfpathlineto{\pgfqpoint{4.145311in}{3.157555in}}%
\pgfpathlineto{\pgfqpoint{4.147889in}{3.175943in}}%
\pgfpathlineto{\pgfqpoint{4.150466in}{3.145359in}}%
\pgfpathlineto{\pgfqpoint{4.153043in}{3.123781in}}%
\pgfpathlineto{\pgfqpoint{4.160775in}{3.115525in}}%
\pgfpathlineto{\pgfqpoint{4.163352in}{3.122092in}}%
\pgfpathlineto{\pgfqpoint{4.165929in}{3.156992in}}%
\pgfpathlineto{\pgfqpoint{4.168506in}{3.170877in}}%
\pgfpathlineto{\pgfqpoint{4.171084in}{3.140105in}}%
\pgfpathlineto{\pgfqpoint{4.178815in}{3.103329in}}%
\pgfpathlineto{\pgfqpoint{4.181392in}{3.113836in}}%
\pgfpathlineto{\pgfqpoint{4.183970in}{3.121529in}}%
\pgfpathlineto{\pgfqpoint{4.186547in}{3.166749in}}%
\pgfpathlineto{\pgfqpoint{4.189124in}{3.155491in}}%
\pgfpathlineto{\pgfqpoint{4.196856in}{3.167687in}}%
\pgfpathlineto{\pgfqpoint{4.199433in}{3.170502in}}%
\pgfpathlineto{\pgfqpoint{4.202010in}{3.131661in}}%
\pgfpathlineto{\pgfqpoint{4.204587in}{3.104267in}}%
\pgfpathlineto{\pgfqpoint{4.207165in}{3.123969in}}%
\pgfpathlineto{\pgfqpoint{4.214896in}{3.070868in}}%
\pgfpathlineto{\pgfqpoint{4.217473in}{3.076872in}}%
\pgfpathlineto{\pgfqpoint{4.220051in}{3.099576in}}%
\pgfpathlineto{\pgfqpoint{4.222628in}{3.095073in}}%
\pgfpathlineto{\pgfqpoint{4.232937in}{3.090007in}}%
\pgfpathlineto{\pgfqpoint{4.235514in}{3.096011in}}%
\pgfpathlineto{\pgfqpoint{4.238091in}{3.050041in}}%
\pgfpathlineto{\pgfqpoint{4.243246in}{3.087755in}}%
\pgfpathlineto{\pgfqpoint{4.250977in}{3.113461in}}%
\pgfpathlineto{\pgfqpoint{4.253554in}{3.118715in}}%
\pgfpathlineto{\pgfqpoint{4.256132in}{3.105580in}}%
\pgfpathlineto{\pgfqpoint{4.258709in}{3.125282in}}%
\pgfpathlineto{\pgfqpoint{4.261286in}{3.117214in}}%
\pgfpathlineto{\pgfqpoint{4.269018in}{3.124531in}}%
\pgfpathlineto{\pgfqpoint{4.271595in}{3.101452in}}%
\pgfpathlineto{\pgfqpoint{4.274172in}{3.095636in}}%
\pgfpathlineto{\pgfqpoint{4.276749in}{2.995815in}}%
\pgfpathlineto{\pgfqpoint{4.279326in}{2.992813in}}%
\pgfpathlineto{\pgfqpoint{4.287058in}{2.982305in}}%
\pgfpathlineto{\pgfqpoint{4.289635in}{3.023022in}}%
\pgfpathlineto{\pgfqpoint{4.292213in}{3.028651in}}%
\pgfpathlineto{\pgfqpoint{4.294790in}{3.030715in}}%
\pgfpathlineto{\pgfqpoint{4.297367in}{3.026774in}}%
\pgfpathlineto{\pgfqpoint{4.305099in}{3.006135in}}%
\pgfpathlineto{\pgfqpoint{4.307676in}{3.011013in}}%
\pgfpathlineto{\pgfqpoint{4.310253in}{3.021333in}}%
\pgfpathlineto{\pgfqpoint{4.312830in}{2.991687in}}%
\pgfpathlineto{\pgfqpoint{4.315407in}{2.985495in}}%
\pgfpathlineto{\pgfqpoint{4.323139in}{3.025086in}}%
\pgfpathlineto{\pgfqpoint{4.325716in}{2.978552in}}%
\pgfpathlineto{\pgfqpoint{4.328294in}{2.979115in}}%
\pgfpathlineto{\pgfqpoint{4.330871in}{2.960164in}}%
\pgfpathlineto{\pgfqpoint{4.333448in}{2.975175in}}%
\pgfpathlineto{\pgfqpoint{4.341180in}{2.991312in}}%
\pgfpathlineto{\pgfqpoint{4.343757in}{2.975550in}}%
\pgfpathlineto{\pgfqpoint{4.346334in}{2.951533in}}%
\pgfpathlineto{\pgfqpoint{4.348911in}{2.908753in}}%
\pgfpathlineto{\pgfqpoint{4.351488in}{2.887175in}}%
\pgfpathlineto{\pgfqpoint{4.361797in}{2.811559in}}%
\pgfpathlineto{\pgfqpoint{4.364374in}{2.897870in}}%
\pgfpathlineto{\pgfqpoint{4.366952in}{2.917947in}}%
\pgfpathlineto{\pgfqpoint{4.369529in}{2.923388in}}%
\pgfpathlineto{\pgfqpoint{4.377261in}{2.888113in}}%
\pgfpathlineto{\pgfqpoint{4.379838in}{2.826194in}}%
\pgfpathlineto{\pgfqpoint{4.382415in}{2.872727in}}%
\pgfpathlineto{\pgfqpoint{4.384992in}{2.881358in}}%
\pgfpathlineto{\pgfqpoint{4.387569in}{2.848898in}}%
\pgfpathlineto{\pgfqpoint{4.397878in}{2.910066in}}%
\pgfpathlineto{\pgfqpoint{4.400455in}{2.866348in}}%
\pgfpathlineto{\pgfqpoint{4.403033in}{2.864847in}}%
\pgfpathlineto{\pgfqpoint{4.405610in}{2.873102in}}%
\pgfpathlineto{\pgfqpoint{4.413342in}{2.865409in}}%
\pgfpathlineto{\pgfqpoint{4.415919in}{2.912881in}}%
\pgfpathlineto{\pgfqpoint{4.418496in}{2.923201in}}%
\pgfpathlineto{\pgfqpoint{4.421073in}{2.901810in}}%
\pgfpathlineto{\pgfqpoint{4.423650in}{2.845145in}}%
\pgfpathlineto{\pgfqpoint{4.431382in}{2.852275in}}%
\pgfpathlineto{\pgfqpoint{4.433959in}{2.817938in}}%
\pgfpathlineto{\pgfqpoint{4.436536in}{2.811934in}}%
\pgfpathlineto{\pgfqpoint{4.439114in}{2.810433in}}%
\pgfpathlineto{\pgfqpoint{4.441691in}{2.844207in}}%
\pgfpathlineto{\pgfqpoint{4.449422in}{2.824130in}}%
\pgfpathlineto{\pgfqpoint{4.452000in}{2.877981in}}%
\pgfpathlineto{\pgfqpoint{4.454577in}{2.881734in}}%
\pgfpathlineto{\pgfqpoint{4.457154in}{2.865222in}}%
\pgfpathlineto{\pgfqpoint{4.459731in}{2.906126in}}%
\pgfpathlineto{\pgfqpoint{4.467463in}{2.959226in}}%
\pgfpathlineto{\pgfqpoint{4.470040in}{2.950220in}}%
\pgfpathlineto{\pgfqpoint{4.475195in}{3.012889in}}%
\pgfpathlineto{\pgfqpoint{4.477772in}{3.020020in}}%
\pgfpathlineto{\pgfqpoint{4.485503in}{3.022834in}}%
\pgfpathlineto{\pgfqpoint{4.490658in}{2.993188in}}%
\pgfpathlineto{\pgfqpoint{4.493235in}{3.008011in}}%
\pgfpathlineto{\pgfqpoint{4.495812in}{2.999755in}}%
\pgfpathlineto{\pgfqpoint{4.503544in}{2.987747in}}%
\pgfpathlineto{\pgfqpoint{4.506121in}{3.006698in}}%
\pgfpathlineto{\pgfqpoint{4.508698in}{3.018706in}}%
\pgfpathlineto{\pgfqpoint{4.511276in}{3.123969in}}%
\pgfpathlineto{\pgfqpoint{4.513853in}{3.120403in}}%
\pgfpathlineto{\pgfqpoint{4.524162in}{3.136352in}}%
\pgfpathlineto{\pgfqpoint{4.526739in}{3.159244in}}%
\pgfpathlineto{\pgfqpoint{4.531893in}{3.144421in}}%
\pgfpathlineto{\pgfqpoint{4.539625in}{3.190578in}}%
\pgfpathlineto{\pgfqpoint{4.542202in}{3.170689in}}%
\pgfpathlineto{\pgfqpoint{4.544779in}{3.173504in}}%
\pgfpathlineto{\pgfqpoint{4.547357in}{3.174817in}}%
\pgfpathlineto{\pgfqpoint{4.549934in}{3.179320in}}%
\pgfpathlineto{\pgfqpoint{4.557665in}{3.148924in}}%
\pgfpathlineto{\pgfqpoint{4.560243in}{3.150988in}}%
\pgfpathlineto{\pgfqpoint{4.562820in}{3.175755in}}%
\pgfpathlineto{\pgfqpoint{4.565397in}{3.129035in}}%
\pgfpathlineto{\pgfqpoint{4.567974in}{3.117964in}}%
\pgfpathlineto{\pgfqpoint{4.575706in}{3.161495in}}%
\pgfpathlineto{\pgfqpoint{4.578283in}{3.141418in}}%
\pgfpathlineto{\pgfqpoint{4.580860in}{3.165060in}}%
\pgfpathlineto{\pgfqpoint{4.583438in}{3.182135in}}%
\pgfpathlineto{\pgfqpoint{4.586015in}{3.191892in}}%
\pgfpathlineto{\pgfqpoint{4.593746in}{3.186826in}}%
\pgfpathlineto{\pgfqpoint{4.596324in}{3.171252in}}%
\pgfpathlineto{\pgfqpoint{4.598901in}{3.169188in}}%
\pgfpathlineto{\pgfqpoint{4.604055in}{3.175380in}}%
\pgfpathlineto{\pgfqpoint{4.611787in}{3.151363in}}%
\pgfpathlineto{\pgfqpoint{4.614364in}{3.156804in}}%
\pgfpathlineto{\pgfqpoint{4.619519in}{3.113649in}}%
\pgfpathlineto{\pgfqpoint{4.622096in}{3.179320in}}%
\pgfpathlineto{\pgfqpoint{4.629827in}{3.171252in}}%
\pgfpathlineto{\pgfqpoint{4.632405in}{3.154928in}}%
\pgfpathlineto{\pgfqpoint{4.634982in}{3.142169in}}%
\pgfpathlineto{\pgfqpoint{4.637559in}{3.167312in}}%
\pgfpathlineto{\pgfqpoint{4.640136in}{3.121342in}}%
\pgfpathlineto{\pgfqpoint{4.647868in}{3.169188in}}%
\pgfpathlineto{\pgfqpoint{4.650445in}{3.006510in}}%
\pgfpathlineto{\pgfqpoint{4.653022in}{3.037657in}}%
\pgfpathlineto{\pgfqpoint{4.655599in}{3.018894in}}%
\pgfpathlineto{\pgfqpoint{4.658177in}{2.985870in}}%
\pgfpathlineto{\pgfqpoint{4.665908in}{2.995440in}}%
\pgfpathlineto{\pgfqpoint{4.668486in}{3.020958in}}%
\pgfpathlineto{\pgfqpoint{4.671063in}{3.058484in}}%
\pgfpathlineto{\pgfqpoint{4.673640in}{3.059047in}}%
\pgfpathlineto{\pgfqpoint{4.683949in}{3.059423in}}%
\pgfpathlineto{\pgfqpoint{4.686526in}{3.088130in}}%
\pgfpathlineto{\pgfqpoint{4.689103in}{3.071243in}}%
\pgfpathlineto{\pgfqpoint{4.691680in}{3.049478in}}%
\pgfpathlineto{\pgfqpoint{4.701989in}{2.984181in}}%
\pgfpathlineto{\pgfqpoint{4.704567in}{2.995064in}}%
\pgfpathlineto{\pgfqpoint{4.709721in}{2.883985in}}%
\pgfpathlineto{\pgfqpoint{4.712298in}{2.875729in}}%
\pgfpathlineto{\pgfqpoint{4.720030in}{2.875166in}}%
\pgfpathlineto{\pgfqpoint{4.722607in}{2.882109in}}%
\pgfpathlineto{\pgfqpoint{4.725184in}{2.845333in}}%
\pgfpathlineto{\pgfqpoint{4.727761in}{2.887550in}}%
\pgfpathlineto{\pgfqpoint{4.730339in}{2.844957in}}%
\pgfpathlineto{\pgfqpoint{4.740648in}{2.839704in}}%
\pgfpathlineto{\pgfqpoint{4.743225in}{2.815311in}}%
\pgfpathlineto{\pgfqpoint{4.745802in}{2.829009in}}%
\pgfpathlineto{\pgfqpoint{4.748379in}{2.859030in}}%
\pgfpathlineto{\pgfqpoint{4.756111in}{2.825631in}}%
\pgfpathlineto{\pgfqpoint{4.758688in}{2.949094in}}%
\pgfpathlineto{\pgfqpoint{4.761265in}{2.962416in}}%
\pgfpathlineto{\pgfqpoint{4.763842in}{2.992625in}}%
\pgfpathlineto{\pgfqpoint{4.766420in}{3.055670in}}%
\pgfpathlineto{\pgfqpoint{4.776728in}{3.002194in}}%
\pgfpathlineto{\pgfqpoint{4.779306in}{3.081751in}}%
\pgfpathlineto{\pgfqpoint{4.781883in}{3.097512in}}%
\pgfpathlineto{\pgfqpoint{4.784460in}{3.098075in}}%
\pgfpathlineto{\pgfqpoint{4.792192in}{3.105393in}}%
\pgfpathlineto{\pgfqpoint{4.794769in}{3.119841in}}%
\pgfpathlineto{\pgfqpoint{4.797346in}{3.099389in}}%
\pgfpathlineto{\pgfqpoint{4.799923in}{3.068617in}}%
\pgfpathlineto{\pgfqpoint{4.802501in}{3.125470in}}%
\pgfpathlineto{\pgfqpoint{4.812809in}{3.152489in}}%
\pgfpathlineto{\pgfqpoint{4.815387in}{3.171440in}}%
\pgfpathlineto{\pgfqpoint{4.817964in}{3.174067in}}%
\pgfpathlineto{\pgfqpoint{4.820541in}{3.168438in}}%
\pgfpathlineto{\pgfqpoint{4.828273in}{3.190016in}}%
\pgfpathlineto{\pgfqpoint{4.830850in}{3.163559in}}%
\pgfpathlineto{\pgfqpoint{4.833427in}{3.181760in}}%
\pgfpathlineto{\pgfqpoint{4.836004in}{3.212156in}}%
\pgfpathlineto{\pgfqpoint{4.838582in}{3.199772in}}%
\pgfpathlineto{\pgfqpoint{4.846313in}{3.175568in}}%
\pgfpathlineto{\pgfqpoint{4.848890in}{3.222664in}}%
\pgfpathlineto{\pgfqpoint{4.851468in}{3.219474in}}%
\pgfpathlineto{\pgfqpoint{4.854045in}{3.218911in}}%
\pgfpathlineto{\pgfqpoint{4.856622in}{3.230920in}}%
\pgfpathlineto{\pgfqpoint{4.864354in}{3.239551in}}%
\pgfpathlineto{\pgfqpoint{4.866931in}{3.233171in}}%
\pgfpathlineto{\pgfqpoint{4.869508in}{3.231295in}}%
\pgfpathlineto{\pgfqpoint{4.872085in}{3.227730in}}%
\pgfpathlineto{\pgfqpoint{4.874663in}{3.262067in}}%
\pgfpathlineto{\pgfqpoint{4.882394in}{3.260941in}}%
\pgfpathlineto{\pgfqpoint{4.887549in}{3.279517in}}%
\pgfpathlineto{\pgfqpoint{4.890126in}{3.294340in}}%
\pgfpathlineto{\pgfqpoint{4.892703in}{3.321547in}}%
\pgfpathlineto{\pgfqpoint{4.900435in}{3.313103in}}%
\pgfpathlineto{\pgfqpoint{4.903012in}{3.314417in}}%
\pgfpathlineto{\pgfqpoint{4.905589in}{3.303346in}}%
\pgfpathlineto{\pgfqpoint{4.908166in}{3.306536in}}%
\pgfpathlineto{\pgfqpoint{4.918475in}{3.338058in}}%
\pgfpathlineto{\pgfqpoint{4.921052in}{3.305410in}}%
\pgfpathlineto{\pgfqpoint{4.923630in}{3.346127in}}%
\pgfpathlineto{\pgfqpoint{4.926207in}{3.344063in}}%
\pgfpathlineto{\pgfqpoint{4.928784in}{3.359636in}}%
\pgfpathlineto{\pgfqpoint{4.936516in}{3.340122in}}%
\pgfpathlineto{\pgfqpoint{4.939093in}{3.329803in}}%
\pgfpathlineto{\pgfqpoint{4.941670in}{3.347065in}}%
\pgfpathlineto{\pgfqpoint{4.944247in}{3.353257in}}%
\pgfpathlineto{\pgfqpoint{4.946825in}{3.344063in}}%
\pgfpathlineto{\pgfqpoint{4.954556in}{3.343312in}}%
\pgfpathlineto{\pgfqpoint{4.957133in}{3.368267in}}%
\pgfpathlineto{\pgfqpoint{4.959711in}{3.377837in}}%
\pgfpathlineto{\pgfqpoint{4.962288in}{3.370331in}}%
\pgfpathlineto{\pgfqpoint{4.964865in}{3.381027in}}%
\pgfpathlineto{\pgfqpoint{4.972597in}{3.394348in}}%
\pgfpathlineto{\pgfqpoint{4.975174in}{3.396037in}}%
\pgfpathlineto{\pgfqpoint{4.977751in}{3.383841in}}%
\pgfpathlineto{\pgfqpoint{4.980328in}{3.378963in}}%
\pgfpathlineto{\pgfqpoint{4.982905in}{3.378963in}}%
\pgfpathlineto{\pgfqpoint{4.990637in}{3.374272in}}%
\pgfpathlineto{\pgfqpoint{4.993214in}{3.336182in}}%
\pgfpathlineto{\pgfqpoint{4.995792in}{3.363577in}}%
\pgfpathlineto{\pgfqpoint{4.998369in}{3.350442in}}%
\pgfpathlineto{\pgfqpoint{5.000946in}{3.357009in}}%
\pgfpathlineto{\pgfqpoint{5.008678in}{3.373521in}}%
\pgfpathlineto{\pgfqpoint{5.011255in}{3.367142in}}%
\pgfpathlineto{\pgfqpoint{5.013832in}{3.353257in}}%
\pgfpathlineto{\pgfqpoint{5.016409in}{3.363764in}}%
\pgfpathlineto{\pgfqpoint{5.018986in}{3.383090in}}%
\pgfpathlineto{\pgfqpoint{5.026718in}{3.376711in}}%
\pgfpathlineto{\pgfqpoint{5.029295in}{3.406732in}}%
\pgfpathlineto{\pgfqpoint{5.031873in}{3.398101in}}%
\pgfpathlineto{\pgfqpoint{5.034450in}{3.404668in}}%
\pgfpathlineto{\pgfqpoint{5.037027in}{3.373146in}}%
\pgfpathlineto{\pgfqpoint{5.044759in}{3.394348in}}%
\pgfpathlineto{\pgfqpoint{5.047336in}{3.364327in}}%
\pgfpathlineto{\pgfqpoint{5.049913in}{3.366391in}}%
\pgfpathlineto{\pgfqpoint{5.052490in}{3.336745in}}%
\pgfpathlineto{\pgfqpoint{5.055067in}{3.335056in}}%
\pgfpathlineto{\pgfqpoint{5.062799in}{3.354195in}}%
\pgfpathlineto{\pgfqpoint{5.065376in}{3.397914in}}%
\pgfpathlineto{\pgfqpoint{5.067953in}{3.421555in}}%
\pgfpathlineto{\pgfqpoint{5.070531in}{3.402792in}}%
\pgfpathlineto{\pgfqpoint{5.073108in}{3.402417in}}%
\pgfpathlineto{\pgfqpoint{5.083417in}{3.392472in}}%
\pgfpathlineto{\pgfqpoint{5.085994in}{3.398852in}}%
\pgfpathlineto{\pgfqpoint{5.088571in}{3.387594in}}%
\pgfpathlineto{\pgfqpoint{5.091148in}{3.393598in}}%
\pgfpathlineto{\pgfqpoint{5.098880in}{3.423056in}}%
\pgfpathlineto{\pgfqpoint{5.101457in}{3.437504in}}%
\pgfpathlineto{\pgfqpoint{5.104034in}{3.446323in}}%
\pgfpathlineto{\pgfqpoint{5.106612in}{3.423995in}}%
\pgfpathlineto{\pgfqpoint{5.109189in}{3.396600in}}%
\pgfpathlineto{\pgfqpoint{5.116921in}{3.371832in}}%
\pgfpathlineto{\pgfqpoint{5.119498in}{3.380276in}}%
\pgfpathlineto{\pgfqpoint{5.122075in}{3.383466in}}%
\pgfpathlineto{\pgfqpoint{5.124652in}{3.420430in}}%
\pgfpathlineto{\pgfqpoint{5.127229in}{3.403355in}}%
\pgfpathlineto{\pgfqpoint{5.134961in}{3.443133in}}%
\pgfpathlineto{\pgfqpoint{5.137538in}{3.445572in}}%
\pgfpathlineto{\pgfqpoint{5.140115in}{3.443133in}}%
\pgfpathlineto{\pgfqpoint{5.142693in}{3.493044in}}%
\pgfpathlineto{\pgfqpoint{5.145270in}{3.406357in}}%
\pgfpathlineto{\pgfqpoint{5.153001in}{3.372958in}}%
\pgfpathlineto{\pgfqpoint{5.158156in}{3.448012in}}%
\pgfpathlineto{\pgfqpoint{5.160733in}{3.510681in}}%
\pgfpathlineto{\pgfqpoint{5.163310in}{3.517811in}}%
\pgfpathlineto{\pgfqpoint{5.173619in}{3.519500in}}%
\pgfpathlineto{\pgfqpoint{5.176196in}{3.511244in}}%
\pgfpathlineto{\pgfqpoint{5.178774in}{3.506178in}}%
\pgfpathlineto{\pgfqpoint{5.181351in}{3.545394in}}%
\pgfpathlineto{\pgfqpoint{5.189082in}{3.560779in}}%
\pgfpathlineto{\pgfqpoint{5.191660in}{3.580669in}}%
\pgfpathlineto{\pgfqpoint{5.194237in}{3.581607in}}%
\pgfpathlineto{\pgfqpoint{5.196814in}{3.611065in}}%
\pgfpathlineto{\pgfqpoint{5.199391in}{3.619696in}}%
\pgfpathlineto{\pgfqpoint{5.207123in}{3.614630in}}%
\pgfpathlineto{\pgfqpoint{5.209700in}{3.616882in}}%
\pgfpathlineto{\pgfqpoint{5.212277in}{3.620072in}}%
\pgfpathlineto{\pgfqpoint{5.214855in}{3.599807in}}%
\pgfpathlineto{\pgfqpoint{5.217432in}{3.602997in}}%
\pgfpathlineto{\pgfqpoint{5.225163in}{3.588925in}}%
\pgfpathlineto{\pgfqpoint{5.227741in}{3.554775in}}%
\pgfpathlineto{\pgfqpoint{5.230318in}{3.565283in}}%
\pgfpathlineto{\pgfqpoint{5.232895in}{3.560217in}}%
\pgfpathlineto{\pgfqpoint{5.235472in}{3.566784in}}%
\pgfpathlineto{\pgfqpoint{5.243204in}{3.565658in}}%
\pgfpathlineto{\pgfqpoint{5.245781in}{3.566784in}}%
\pgfpathlineto{\pgfqpoint{5.248358in}{3.567159in}}%
\pgfpathlineto{\pgfqpoint{5.250936in}{3.557402in}}%
\pgfpathlineto{\pgfqpoint{5.253513in}{3.570536in}}%
\pgfpathlineto{\pgfqpoint{5.261244in}{3.570724in}}%
\pgfpathlineto{\pgfqpoint{5.263822in}{3.567347in}}%
\pgfpathlineto{\pgfqpoint{5.266399in}{3.574852in}}%
\pgfpathlineto{\pgfqpoint{5.268976in}{3.612754in}}%
\pgfpathlineto{\pgfqpoint{5.271553in}{3.599995in}}%
\pgfpathlineto{\pgfqpoint{5.279285in}{3.605061in}}%
\pgfpathlineto{\pgfqpoint{5.281862in}{3.582357in}}%
\pgfpathlineto{\pgfqpoint{5.284439in}{3.612566in}}%
\pgfpathlineto{\pgfqpoint{5.287017in}{3.600558in}}%
\pgfpathlineto{\pgfqpoint{5.289594in}{3.608063in}}%
\pgfpathlineto{\pgfqpoint{5.297325in}{3.598494in}}%
\pgfpathlineto{\pgfqpoint{5.299903in}{3.610878in}}%
\pgfpathlineto{\pgfqpoint{5.302480in}{3.604498in}}%
\pgfpathlineto{\pgfqpoint{5.305057in}{3.607500in}}%
\pgfpathlineto{\pgfqpoint{5.307634in}{3.604498in}}%
\pgfpathlineto{\pgfqpoint{5.315366in}{3.623637in}}%
\pgfpathlineto{\pgfqpoint{5.317943in}{3.617445in}}%
\pgfpathlineto{\pgfqpoint{5.320520in}{3.601496in}}%
\pgfpathlineto{\pgfqpoint{5.325675in}{3.628890in}}%
\pgfpathlineto{\pgfqpoint{5.335984in}{3.622886in}}%
\pgfpathlineto{\pgfqpoint{5.338561in}{3.610502in}}%
\pgfpathlineto{\pgfqpoint{5.341138in}{3.616507in}}%
\pgfpathlineto{\pgfqpoint{5.343715in}{3.538639in}}%
\pgfpathlineto{\pgfqpoint{5.351447in}{3.581607in}}%
\pgfpathlineto{\pgfqpoint{5.354024in}{3.546144in}}%
\pgfpathlineto{\pgfqpoint{5.356601in}{3.538263in}}%
\pgfpathlineto{\pgfqpoint{5.359178in}{3.555150in}}%
\pgfpathlineto{\pgfqpoint{5.361756in}{3.528507in}}%
\pgfpathlineto{\pgfqpoint{5.369487in}{3.557590in}}%
\pgfpathlineto{\pgfqpoint{5.372065in}{3.572976in}}%
\pgfpathlineto{\pgfqpoint{5.374642in}{3.607688in}}%
\pgfpathlineto{\pgfqpoint{5.377219in}{3.612191in}}%
\pgfpathlineto{\pgfqpoint{5.379796in}{3.569223in}}%
\pgfpathlineto{\pgfqpoint{5.387528in}{3.544080in}}%
\pgfpathlineto{\pgfqpoint{5.390105in}{3.550084in}}%
\pgfpathlineto{\pgfqpoint{5.392682in}{3.572788in}}%
\pgfpathlineto{\pgfqpoint{5.395259in}{3.533948in}}%
\pgfpathlineto{\pgfqpoint{5.397837in}{3.548959in}}%
\pgfpathlineto{\pgfqpoint{5.405568in}{3.528131in}}%
\pgfpathlineto{\pgfqpoint{5.408146in}{3.469402in}}%
\pgfpathlineto{\pgfqpoint{5.410723in}{3.481786in}}%
\pgfpathlineto{\pgfqpoint{5.413300in}{3.468651in}}%
\pgfpathlineto{\pgfqpoint{5.415877in}{3.463398in}}%
\pgfpathlineto{\pgfqpoint{5.423609in}{3.460020in}}%
\pgfpathlineto{\pgfqpoint{5.426186in}{3.434502in}}%
\pgfpathlineto{\pgfqpoint{5.428763in}{3.435253in}}%
\pgfpathlineto{\pgfqpoint{5.433918in}{3.445760in}}%
\pgfpathlineto{\pgfqpoint{5.441649in}{3.443696in}}%
\pgfpathlineto{\pgfqpoint{5.444227in}{3.439193in}}%
\pgfpathlineto{\pgfqpoint{5.449381in}{3.437692in}}%
\pgfpathlineto{\pgfqpoint{5.451958in}{3.431312in}}%
\pgfpathlineto{\pgfqpoint{5.459690in}{3.462272in}}%
\pgfpathlineto{\pgfqpoint{5.462267in}{3.374272in}}%
\pgfpathlineto{\pgfqpoint{5.464844in}{3.379150in}}%
\pgfpathlineto{\pgfqpoint{5.467421in}{3.366016in}}%
\pgfpathlineto{\pgfqpoint{5.469999in}{3.366203in}}%
\pgfpathlineto{\pgfqpoint{5.477730in}{3.357948in}}%
\pgfpathlineto{\pgfqpoint{5.480307in}{3.339559in}}%
\pgfpathlineto{\pgfqpoint{5.482885in}{3.365453in}}%
\pgfpathlineto{\pgfqpoint{5.485462in}{3.384779in}}%
\pgfpathlineto{\pgfqpoint{5.488039in}{3.378587in}}%
\pgfpathlineto{\pgfqpoint{5.498348in}{3.458144in}}%
\pgfpathlineto{\pgfqpoint{5.500925in}{3.446886in}}%
\pgfpathlineto{\pgfqpoint{5.503502in}{3.514809in}}%
\pgfpathlineto{\pgfqpoint{5.506080in}{3.528882in}}%
\pgfpathlineto{\pgfqpoint{5.513811in}{3.490980in}}%
\pgfpathlineto{\pgfqpoint{5.516388in}{3.514246in}}%
\pgfpathlineto{\pgfqpoint{5.518966in}{3.494545in}}%
\pgfpathlineto{\pgfqpoint{5.521543in}{3.507867in}}%
\pgfpathlineto{\pgfqpoint{5.524120in}{3.511244in}}%
\pgfpathlineto{\pgfqpoint{5.531852in}{3.485914in}}%
\pgfpathlineto{\pgfqpoint{5.537006in}{3.498673in}}%
\pgfpathlineto{\pgfqpoint{5.542161in}{3.520438in}}%
\pgfpathlineto{\pgfqpoint{5.549892in}{3.503176in}}%
\pgfpathlineto{\pgfqpoint{5.552469in}{3.507867in}}%
\pgfpathlineto{\pgfqpoint{5.555047in}{3.489666in}}%
\pgfpathlineto{\pgfqpoint{5.557624in}{3.505428in}}%
\pgfpathlineto{\pgfqpoint{5.560201in}{3.501863in}}%
\pgfpathlineto{\pgfqpoint{5.567933in}{3.487602in}}%
\pgfpathlineto{\pgfqpoint{5.570510in}{3.490792in}}%
\pgfpathlineto{\pgfqpoint{5.573087in}{3.565470in}}%
\pgfpathlineto{\pgfqpoint{5.575664in}{3.562468in}}%
\pgfpathlineto{\pgfqpoint{5.578242in}{3.608438in}}%
\pgfpathlineto{\pgfqpoint{5.585973in}{3.628703in}}%
\pgfpathlineto{\pgfqpoint{5.588550in}{3.614443in}}%
\pgfpathlineto{\pgfqpoint{5.591128in}{3.575227in}}%
\pgfpathlineto{\pgfqpoint{5.593705in}{3.564907in}}%
\pgfpathlineto{\pgfqpoint{5.596282in}{3.590238in}}%
\pgfpathlineto{\pgfqpoint{5.604014in}{3.602622in}}%
\pgfpathlineto{\pgfqpoint{5.606591in}{3.611253in}}%
\pgfpathlineto{\pgfqpoint{5.609168in}{3.607313in}}%
\pgfpathlineto{\pgfqpoint{5.611745in}{3.621010in}}%
\pgfpathlineto{\pgfqpoint{5.614323in}{3.612942in}}%
\pgfpathlineto{\pgfqpoint{5.624631in}{3.615944in}}%
\pgfpathlineto{\pgfqpoint{5.627209in}{3.601308in}}%
\pgfpathlineto{\pgfqpoint{5.629786in}{3.606937in}}%
\pgfpathlineto{\pgfqpoint{5.632363in}{3.609752in}}%
\pgfpathlineto{\pgfqpoint{5.642672in}{3.600745in}}%
\pgfpathlineto{\pgfqpoint{5.645249in}{3.605436in}}%
\pgfpathlineto{\pgfqpoint{5.647826in}{3.594741in}}%
\pgfpathlineto{\pgfqpoint{5.650403in}{3.603935in}}%
\pgfpathlineto{\pgfqpoint{5.658135in}{3.587048in}}%
\pgfpathlineto{\pgfqpoint{5.660712in}{3.574852in}}%
\pgfpathlineto{\pgfqpoint{5.663290in}{3.597931in}}%
\pgfpathlineto{\pgfqpoint{5.665867in}{3.590050in}}%
\pgfpathlineto{\pgfqpoint{5.668444in}{3.589112in}}%
\pgfpathlineto{\pgfqpoint{5.678753in}{3.586861in}}%
\pgfpathlineto{\pgfqpoint{5.681330in}{3.608438in}}%
\pgfpathlineto{\pgfqpoint{5.683907in}{3.611816in}}%
\pgfpathlineto{\pgfqpoint{5.686484in}{3.608438in}}%
\pgfpathlineto{\pgfqpoint{5.694216in}{3.608814in}}%
\pgfpathlineto{\pgfqpoint{5.696793in}{3.564157in}}%
\pgfpathlineto{\pgfqpoint{5.699371in}{3.577479in}}%
\pgfpathlineto{\pgfqpoint{5.701948in}{3.578980in}}%
\pgfpathlineto{\pgfqpoint{5.704525in}{3.590613in}}%
\pgfpathlineto{\pgfqpoint{5.714834in}{3.543892in}}%
\pgfpathlineto{\pgfqpoint{5.717411in}{3.550084in}}%
\pgfpathlineto{\pgfqpoint{5.719988in}{3.532634in}}%
\pgfpathlineto{\pgfqpoint{5.722565in}{3.547833in}}%
\pgfpathlineto{\pgfqpoint{5.730297in}{3.548771in}}%
\pgfpathlineto{\pgfqpoint{5.732874in}{3.560404in}}%
\pgfpathlineto{\pgfqpoint{5.738029in}{3.602997in}}%
\pgfpathlineto{\pgfqpoint{5.740606in}{3.617445in}}%
\pgfpathlineto{\pgfqpoint{5.750915in}{3.663040in}}%
\pgfpathlineto{\pgfqpoint{5.753492in}{3.685743in}}%
\pgfpathlineto{\pgfqpoint{5.756069in}{3.715952in}}%
\pgfpathlineto{\pgfqpoint{5.758646in}{3.707884in}}%
\pgfpathlineto{\pgfqpoint{5.768955in}{3.715202in}}%
\pgfpathlineto{\pgfqpoint{5.771532in}{3.762486in}}%
\pgfpathlineto{\pgfqpoint{5.774110in}{3.782938in}}%
\pgfpathlineto{\pgfqpoint{5.776687in}{3.786690in}}%
\pgfpathlineto{\pgfqpoint{5.784419in}{3.777872in}}%
\pgfpathlineto{\pgfqpoint{5.786996in}{3.767927in}}%
\pgfpathlineto{\pgfqpoint{5.789573in}{3.830221in}}%
\pgfpathlineto{\pgfqpoint{5.792150in}{3.830597in}}%
\pgfpathlineto{\pgfqpoint{5.794727in}{3.820464in}}%
\pgfpathlineto{\pgfqpoint{5.802459in}{3.813147in}}%
\pgfpathlineto{\pgfqpoint{5.805036in}{3.816524in}}%
\pgfpathlineto{\pgfqpoint{5.810191in}{3.830784in}}%
\pgfpathlineto{\pgfqpoint{5.812768in}{3.854051in}}%
\pgfpathlineto{\pgfqpoint{5.820500in}{3.859492in}}%
\pgfpathlineto{\pgfqpoint{5.823077in}{3.838477in}}%
\pgfpathlineto{\pgfqpoint{5.825654in}{3.853863in}}%
\pgfpathlineto{\pgfqpoint{5.828231in}{3.838102in}}%
\pgfpathlineto{\pgfqpoint{5.830808in}{3.874315in}}%
\pgfpathlineto{\pgfqpoint{5.838540in}{3.885948in}}%
\pgfpathlineto{\pgfqpoint{5.841117in}{3.870375in}}%
\pgfpathlineto{\pgfqpoint{5.843694in}{3.871876in}}%
\pgfpathlineto{\pgfqpoint{5.846272in}{3.870563in}}%
\pgfpathlineto{\pgfqpoint{5.848849in}{3.859305in}}%
\pgfpathlineto{\pgfqpoint{5.856580in}{3.840166in}}%
\pgfpathlineto{\pgfqpoint{5.859158in}{3.850110in}}%
\pgfpathlineto{\pgfqpoint{5.861735in}{3.844669in}}%
\pgfpathlineto{\pgfqpoint{5.864312in}{3.855177in}}%
\pgfpathlineto{\pgfqpoint{5.866889in}{3.856115in}}%
\pgfpathlineto{\pgfqpoint{5.874621in}{3.845420in}}%
\pgfpathlineto{\pgfqpoint{5.877198in}{3.836038in}}%
\pgfpathlineto{\pgfqpoint{5.879775in}{3.836788in}}%
\pgfpathlineto{\pgfqpoint{5.882353in}{3.830597in}}%
\pgfpathlineto{\pgfqpoint{5.884930in}{3.832473in}}%
\pgfpathlineto{\pgfqpoint{5.892661in}{3.827407in}}%
\pgfpathlineto{\pgfqpoint{5.895239in}{3.833786in}}%
\pgfpathlineto{\pgfqpoint{5.897816in}{3.827219in}}%
\pgfpathlineto{\pgfqpoint{5.900393in}{3.808643in}}%
\pgfpathlineto{\pgfqpoint{5.910702in}{3.839040in}}%
\pgfpathlineto{\pgfqpoint{5.913279in}{3.836413in}}%
\pgfpathlineto{\pgfqpoint{5.915856in}{3.829658in}}%
\pgfpathlineto{\pgfqpoint{5.918434in}{3.853113in}}%
\pgfpathlineto{\pgfqpoint{5.921011in}{3.859117in}}%
\pgfpathlineto{\pgfqpoint{5.931320in}{3.923475in}}%
\pgfpathlineto{\pgfqpoint{5.933897in}{3.921036in}}%
\pgfpathlineto{\pgfqpoint{5.936474in}{3.940737in}}%
\pgfpathlineto{\pgfqpoint{5.939051in}{3.935859in}}%
\pgfpathlineto{\pgfqpoint{5.946783in}{3.917659in}}%
\pgfpathlineto{\pgfqpoint{5.949360in}{3.948243in}}%
\pgfpathlineto{\pgfqpoint{5.951937in}{3.967757in}}%
\pgfpathlineto{\pgfqpoint{5.954515in}{4.000968in}}%
\pgfpathlineto{\pgfqpoint{5.957092in}{3.998904in}}%
\pgfpathlineto{\pgfqpoint{5.964823in}{3.985770in}}%
\pgfpathlineto{\pgfqpoint{5.967401in}{3.973573in}}%
\pgfpathlineto{\pgfqpoint{5.969978in}{3.950119in}}%
\pgfpathlineto{\pgfqpoint{5.972555in}{3.951995in}}%
\pgfpathlineto{\pgfqpoint{5.975132in}{3.948993in}}%
\pgfpathlineto{\pgfqpoint{5.985441in}{3.972635in}}%
\pgfpathlineto{\pgfqpoint{5.988018in}{3.938861in}}%
\pgfpathlineto{\pgfqpoint{5.990596in}{3.945991in}}%
\pgfpathlineto{\pgfqpoint{5.993173in}{3.955936in}}%
\pgfpathlineto{\pgfqpoint{6.000904in}{4.004158in}}%
\pgfpathlineto{\pgfqpoint{6.003482in}{3.990273in}}%
\pgfpathlineto{\pgfqpoint{6.006059in}{3.986145in}}%
\pgfpathlineto{\pgfqpoint{6.008636in}{4.022546in}}%
\pgfpathlineto{\pgfqpoint{6.011213in}{4.042623in}}%
\pgfpathlineto{\pgfqpoint{6.021522in}{4.074145in}}%
\pgfpathlineto{\pgfqpoint{6.024099in}{4.110358in}}%
\pgfpathlineto{\pgfqpoint{6.026677in}{4.108294in}}%
\pgfpathlineto{\pgfqpoint{6.029254in}{4.150137in}}%
\pgfpathlineto{\pgfqpoint{6.036985in}{4.141505in}}%
\pgfpathlineto{\pgfqpoint{6.039563in}{4.127058in}}%
\pgfpathlineto{\pgfqpoint{6.042140in}{4.119928in}}%
\pgfpathlineto{\pgfqpoint{6.047294in}{4.154077in}}%
\pgfpathlineto{\pgfqpoint{6.055026in}{4.162333in}}%
\pgfpathlineto{\pgfqpoint{6.057603in}{4.192917in}}%
\pgfpathlineto{\pgfqpoint{6.060180in}{4.209054in}}%
\pgfpathlineto{\pgfqpoint{6.062757in}{4.231757in}}%
\pgfpathlineto{\pgfqpoint{6.065335in}{4.266469in}}%
\pgfpathlineto{\pgfqpoint{6.073066in}{4.268721in}}%
\pgfpathlineto{\pgfqpoint{6.075644in}{4.268721in}}%
\pgfpathlineto{\pgfqpoint{6.078221in}{4.259715in}}%
\pgfpathlineto{\pgfqpoint{6.080798in}{4.248269in}}%
\pgfpathlineto{\pgfqpoint{6.083375in}{4.260465in}}%
\pgfpathlineto{\pgfqpoint{6.091107in}{4.255211in}}%
\pgfpathlineto{\pgfqpoint{6.093684in}{4.204926in}}%
\pgfpathlineto{\pgfqpoint{6.096261in}{4.219936in}}%
\pgfpathlineto{\pgfqpoint{6.098838in}{4.170589in}}%
\pgfpathlineto{\pgfqpoint{6.101416in}{4.176593in}}%
\pgfpathlineto{\pgfqpoint{6.109147in}{4.205864in}}%
\pgfpathlineto{\pgfqpoint{6.114302in}{4.204550in}}%
\pgfpathlineto{\pgfqpoint{6.116879in}{4.173591in}}%
\pgfpathlineto{\pgfqpoint{6.119456in}{4.201548in}}%
\pgfpathlineto{\pgfqpoint{6.127188in}{4.217497in}}%
\pgfpathlineto{\pgfqpoint{6.129765in}{4.202674in}}%
\pgfpathlineto{\pgfqpoint{6.132342in}{4.231945in}}%
\pgfpathlineto{\pgfqpoint{6.134919in}{4.228192in}}%
\pgfpathlineto{\pgfqpoint{6.137497in}{4.240388in}}%
\pgfpathlineto{\pgfqpoint{6.145228in}{4.238700in}}%
\pgfpathlineto{\pgfqpoint{6.147806in}{4.232132in}}%
\pgfpathlineto{\pgfqpoint{6.150383in}{4.246205in}}%
\pgfpathlineto{\pgfqpoint{6.152960in}{4.252397in}}%
\pgfpathlineto{\pgfqpoint{6.155537in}{4.229506in}}%
\pgfpathlineto{\pgfqpoint{6.163269in}{4.208866in}}%
\pgfpathlineto{\pgfqpoint{6.165846in}{4.019919in}}%
\pgfpathlineto{\pgfqpoint{6.168423in}{4.013539in}}%
\pgfpathlineto{\pgfqpoint{6.171000in}{4.031740in}}%
\pgfpathlineto{\pgfqpoint{6.173578in}{4.025735in}}%
\pgfpathlineto{\pgfqpoint{6.181309in}{4.051629in}}%
\pgfpathlineto{\pgfqpoint{6.189041in}{4.166461in}}%
\pgfpathlineto{\pgfqpoint{6.191618in}{4.167024in}}%
\pgfpathlineto{\pgfqpoint{6.199350in}{4.163271in}}%
\pgfpathlineto{\pgfqpoint{6.201927in}{4.145258in}}%
\pgfpathlineto{\pgfqpoint{6.204504in}{4.146196in}}%
\pgfpathlineto{\pgfqpoint{6.209659in}{4.137190in}}%
\pgfpathlineto{\pgfqpoint{6.217390in}{4.161957in}}%
\pgfpathlineto{\pgfqpoint{6.219967in}{4.158580in}}%
\pgfpathlineto{\pgfqpoint{6.222545in}{4.171714in}}%
\pgfpathlineto{\pgfqpoint{6.225122in}{4.125932in}}%
\pgfpathlineto{\pgfqpoint{6.227699in}{4.093659in}}%
\pgfpathlineto{\pgfqpoint{6.235431in}{4.111859in}}%
\pgfpathlineto{\pgfqpoint{6.238008in}{4.127996in}}%
\pgfpathlineto{\pgfqpoint{6.240585in}{4.099100in}}%
\pgfpathlineto{\pgfqpoint{6.243162in}{4.089156in}}%
\pgfpathlineto{\pgfqpoint{6.245740in}{4.089343in}}%
\pgfpathlineto{\pgfqpoint{6.253471in}{4.095348in}}%
\pgfpathlineto{\pgfqpoint{6.256048in}{4.103603in}}%
\pgfpathlineto{\pgfqpoint{6.261203in}{4.128559in}}%
\pgfpathlineto{\pgfqpoint{6.263780in}{4.115049in}}%
\pgfpathlineto{\pgfqpoint{6.274089in}{4.068704in}}%
\pgfpathlineto{\pgfqpoint{6.276666in}{4.088030in}}%
\pgfpathlineto{\pgfqpoint{6.279243in}{4.131186in}}%
\pgfpathlineto{\pgfqpoint{6.289552in}{4.222375in}}%
\pgfpathlineto{\pgfqpoint{6.292129in}{4.223877in}}%
\pgfpathlineto{\pgfqpoint{6.294707in}{4.221250in}}%
\pgfpathlineto{\pgfqpoint{6.299861in}{4.290299in}}%
\pgfpathlineto{\pgfqpoint{6.307593in}{4.297617in}}%
\pgfpathlineto{\pgfqpoint{6.310170in}{4.294052in}}%
\pgfpathlineto{\pgfqpoint{6.312747in}{4.239450in}}%
\pgfpathlineto{\pgfqpoint{6.315324in}{4.238137in}}%
\pgfpathlineto{\pgfqpoint{6.317902in}{4.243953in}}%
\pgfpathlineto{\pgfqpoint{6.325633in}{4.243390in}}%
\pgfpathlineto{\pgfqpoint{6.328210in}{4.248832in}}%
\pgfpathlineto{\pgfqpoint{6.330788in}{4.221062in}}%
\pgfpathlineto{\pgfqpoint{6.333365in}{4.222188in}}%
\pgfpathlineto{\pgfqpoint{6.335942in}{4.228567in}}%
\pgfpathlineto{\pgfqpoint{6.343674in}{4.279791in}}%
\pgfpathlineto{\pgfqpoint{6.348828in}{4.347152in}}%
\pgfpathlineto{\pgfqpoint{6.351405in}{4.344337in}}%
\pgfpathlineto{\pgfqpoint{6.353982in}{4.347152in}}%
\pgfpathlineto{\pgfqpoint{6.364291in}{4.351280in}}%
\pgfpathlineto{\pgfqpoint{6.366869in}{4.346964in}}%
\pgfpathlineto{\pgfqpoint{6.369446in}{4.366290in}}%
\pgfpathlineto{\pgfqpoint{6.372023in}{4.368542in}}%
\pgfpathlineto{\pgfqpoint{6.379755in}{4.386555in}}%
\pgfpathlineto{\pgfqpoint{6.382332in}{4.369105in}}%
\pgfpathlineto{\pgfqpoint{6.384909in}{4.378487in}}%
\pgfpathlineto{\pgfqpoint{6.387486in}{4.395749in}}%
\pgfpathlineto{\pgfqpoint{6.390063in}{4.433088in}}%
\pgfpathlineto{\pgfqpoint{6.397795in}{4.437216in}}%
\pgfpathlineto{\pgfqpoint{6.400372in}{4.671758in}}%
\pgfpathlineto{\pgfqpoint{6.402950in}{4.725984in}}%
\pgfpathlineto{\pgfqpoint{6.405527in}{4.641174in}}%
\pgfpathlineto{\pgfqpoint{6.408104in}{4.673447in}}%
\pgfpathlineto{\pgfqpoint{6.415836in}{4.606837in}}%
\pgfpathlineto{\pgfqpoint{6.418413in}{4.591826in}}%
\pgfpathlineto{\pgfqpoint{6.420990in}{4.591638in}}%
\pgfpathlineto{\pgfqpoint{6.423567in}{4.628415in}}%
\pgfpathlineto{\pgfqpoint{6.426144in}{4.628227in}}%
\pgfpathlineto{\pgfqpoint{6.433876in}{4.594078in}}%
\pgfpathlineto{\pgfqpoint{6.439031in}{4.585446in}}%
\pgfpathlineto{\pgfqpoint{6.441608in}{4.559741in}}%
\pgfpathlineto{\pgfqpoint{6.444185in}{4.542854in}}%
\pgfpathlineto{\pgfqpoint{6.451917in}{4.556551in}}%
\pgfpathlineto{\pgfqpoint{6.454494in}{4.576440in}}%
\pgfpathlineto{\pgfqpoint{6.457071in}{4.541915in}}%
\pgfpathlineto{\pgfqpoint{6.459648in}{4.578129in}}%
\pgfpathlineto{\pgfqpoint{6.462225in}{4.577003in}}%
\pgfpathlineto{\pgfqpoint{6.469957in}{4.615093in}}%
\pgfpathlineto{\pgfqpoint{6.472534in}{4.661813in}}%
\pgfpathlineto{\pgfqpoint{6.475111in}{4.637796in}}%
\pgfpathlineto{\pgfqpoint{6.480266in}{4.634044in}}%
\pgfpathlineto{\pgfqpoint{6.487998in}{4.681327in}}%
\pgfpathlineto{\pgfqpoint{6.490575in}{4.710598in}}%
\pgfpathlineto{\pgfqpoint{6.493152in}{4.753379in}}%
\pgfpathlineto{\pgfqpoint{6.495729in}{4.845694in}}%
\pgfpathlineto{\pgfqpoint{6.498306in}{4.810044in}}%
\pgfpathlineto{\pgfqpoint{6.506038in}{4.775895in}}%
\pgfpathlineto{\pgfqpoint{6.508615in}{4.757882in}}%
\pgfpathlineto{\pgfqpoint{6.511192in}{4.765387in}}%
\pgfpathlineto{\pgfqpoint{6.513770in}{4.794658in}}%
\pgfpathlineto{\pgfqpoint{6.516347in}{4.755630in}}%
\pgfpathlineto{\pgfqpoint{6.524079in}{4.776645in}}%
\pgfpathlineto{\pgfqpoint{6.526656in}{4.727673in}}%
\pgfpathlineto{\pgfqpoint{6.529233in}{4.773268in}}%
\pgfpathlineto{\pgfqpoint{6.531810in}{4.754692in}}%
\pgfpathlineto{\pgfqpoint{6.534387in}{4.753191in}}%
\pgfpathlineto{\pgfqpoint{6.542119in}{4.760884in}}%
\pgfpathlineto{\pgfqpoint{6.544696in}{4.760321in}}%
\pgfpathlineto{\pgfqpoint{6.547273in}{4.733865in}}%
\pgfpathlineto{\pgfqpoint{6.549851in}{4.695400in}}%
\pgfpathlineto{\pgfqpoint{6.552428in}{4.694462in}}%
\pgfpathlineto{\pgfqpoint{6.562737in}{4.707408in}}%
\pgfpathlineto{\pgfqpoint{6.565314in}{4.720918in}}%
\pgfpathlineto{\pgfqpoint{6.567891in}{4.712287in}}%
\pgfpathlineto{\pgfqpoint{6.570468in}{4.705907in}}%
\pgfpathlineto{\pgfqpoint{6.570468in}{4.705907in}}%
\pgfusepath{stroke}%
\end{pgfscope}%
\begin{pgfscope}%
\pgfpathrectangle{\pgfqpoint{0.652287in}{0.521603in}}{\pgfqpoint{6.200000in}{4.530000in}}%
\pgfusepath{clip}%
\pgfsetroundcap%
\pgfsetroundjoin%
\pgfsetlinewidth{1.505625pt}%
\definecolor{currentstroke}{rgb}{1.000000,0.498039,0.054902}%
\pgfsetstrokecolor{currentstroke}%
\pgfsetdash{}{0pt}%
\pgfpathmoveto{\pgfqpoint{0.934105in}{1.281594in}}%
\pgfpathlineto{\pgfqpoint{0.936682in}{1.282157in}}%
\pgfpathlineto{\pgfqpoint{0.939259in}{1.291539in}}%
\pgfpathlineto{\pgfqpoint{0.941836in}{1.282533in}}%
\pgfpathlineto{\pgfqpoint{0.949568in}{1.284597in}}%
\pgfpathlineto{\pgfqpoint{0.954722in}{1.293978in}}%
\pgfpathlineto{\pgfqpoint{0.957300in}{1.305799in}}%
\pgfpathlineto{\pgfqpoint{0.970186in}{1.315368in}}%
\pgfpathlineto{\pgfqpoint{0.975340in}{1.327752in}}%
\pgfpathlineto{\pgfqpoint{0.977917in}{1.312366in}}%
\pgfpathlineto{\pgfqpoint{0.988226in}{1.298669in}}%
\pgfpathlineto{\pgfqpoint{0.990803in}{1.314618in}}%
\pgfpathlineto{\pgfqpoint{0.993381in}{1.311428in}}%
\pgfpathlineto{\pgfqpoint{0.995958in}{1.309177in}}%
\pgfpathlineto{\pgfqpoint{1.003689in}{1.296980in}}%
\pgfpathlineto{\pgfqpoint{1.006267in}{1.314055in}}%
\pgfpathlineto{\pgfqpoint{1.011421in}{1.331317in}}%
\pgfpathlineto{\pgfqpoint{1.013998in}{1.349705in}}%
\pgfpathlineto{\pgfqpoint{1.021730in}{1.342200in}}%
\pgfpathlineto{\pgfqpoint{1.024307in}{1.347641in}}%
\pgfpathlineto{\pgfqpoint{1.026884in}{1.339386in}}%
\pgfpathlineto{\pgfqpoint{1.029462in}{1.350644in}}%
\pgfpathlineto{\pgfqpoint{1.032039in}{1.342200in}}%
\pgfpathlineto{\pgfqpoint{1.039770in}{1.346516in}}%
\pgfpathlineto{\pgfqpoint{1.042348in}{1.344827in}}%
\pgfpathlineto{\pgfqpoint{1.044925in}{1.337322in}}%
\pgfpathlineto{\pgfqpoint{1.047502in}{1.360025in}}%
\pgfpathlineto{\pgfqpoint{1.050079in}{1.359838in}}%
\pgfpathlineto{\pgfqpoint{1.060388in}{1.359650in}}%
\pgfpathlineto{\pgfqpoint{1.062965in}{1.358712in}}%
\pgfpathlineto{\pgfqpoint{1.065543in}{1.356648in}}%
\pgfpathlineto{\pgfqpoint{1.068120in}{1.367906in}}%
\pgfpathlineto{\pgfqpoint{1.075851in}{1.382166in}}%
\pgfpathlineto{\pgfqpoint{1.078429in}{1.375036in}}%
\pgfpathlineto{\pgfqpoint{1.081006in}{1.360400in}}%
\pgfpathlineto{\pgfqpoint{1.083583in}{1.371846in}}%
\pgfpathlineto{\pgfqpoint{1.086160in}{1.362089in}}%
\pgfpathlineto{\pgfqpoint{1.093892in}{1.361714in}}%
\pgfpathlineto{\pgfqpoint{1.096469in}{1.341074in}}%
\pgfpathlineto{\pgfqpoint{1.099046in}{1.349893in}}%
\pgfpathlineto{\pgfqpoint{1.101624in}{1.361339in}}%
\pgfpathlineto{\pgfqpoint{1.104201in}{1.365654in}}%
\pgfpathlineto{\pgfqpoint{1.111932in}{1.358337in}}%
\pgfpathlineto{\pgfqpoint{1.114510in}{1.383292in}}%
\pgfpathlineto{\pgfqpoint{1.117087in}{1.415377in}}%
\pgfpathlineto{\pgfqpoint{1.119664in}{1.424946in}}%
\pgfpathlineto{\pgfqpoint{1.122241in}{1.422132in}}%
\pgfpathlineto{\pgfqpoint{1.129973in}{1.434140in}}%
\pgfpathlineto{\pgfqpoint{1.132550in}{1.428136in}}%
\pgfpathlineto{\pgfqpoint{1.135127in}{1.430388in}}%
\pgfpathlineto{\pgfqpoint{1.137704in}{1.434516in}}%
\pgfpathlineto{\pgfqpoint{1.140282in}{1.433765in}}%
\pgfpathlineto{\pgfqpoint{1.148013in}{1.457595in}}%
\pgfpathlineto{\pgfqpoint{1.150591in}{1.450277in}}%
\pgfpathlineto{\pgfqpoint{1.153168in}{1.464349in}}%
\pgfpathlineto{\pgfqpoint{1.155745in}{1.444648in}}%
\pgfpathlineto{\pgfqpoint{1.158322in}{1.444085in}}%
\pgfpathlineto{\pgfqpoint{1.166054in}{1.446712in}}%
\pgfpathlineto{\pgfqpoint{1.168631in}{1.456469in}}%
\pgfpathlineto{\pgfqpoint{1.171208in}{1.439207in}}%
\pgfpathlineto{\pgfqpoint{1.173785in}{1.451590in}}%
\pgfpathlineto{\pgfqpoint{1.184094in}{1.435642in}}%
\pgfpathlineto{\pgfqpoint{1.186672in}{1.419130in}}%
\pgfpathlineto{\pgfqpoint{1.189249in}{1.432077in}}%
\pgfpathlineto{\pgfqpoint{1.191826in}{1.450465in}}%
\pgfpathlineto{\pgfqpoint{1.194403in}{1.437706in}}%
\pgfpathlineto{\pgfqpoint{1.202135in}{1.446900in}}%
\pgfpathlineto{\pgfqpoint{1.204712in}{1.452904in}}%
\pgfpathlineto{\pgfqpoint{1.207289in}{1.450465in}}%
\pgfpathlineto{\pgfqpoint{1.209866in}{1.442584in}}%
\pgfpathlineto{\pgfqpoint{1.212444in}{1.440520in}}%
\pgfpathlineto{\pgfqpoint{1.220175in}{1.438456in}}%
\pgfpathlineto{\pgfqpoint{1.222753in}{1.443522in}}%
\pgfpathlineto{\pgfqpoint{1.225330in}{1.465288in}}%
\pgfpathlineto{\pgfqpoint{1.230484in}{1.486490in}}%
\pgfpathlineto{\pgfqpoint{1.238216in}{1.487241in}}%
\pgfpathlineto{\pgfqpoint{1.240793in}{1.500000in}}%
\pgfpathlineto{\pgfqpoint{1.243370in}{1.501501in}}%
\pgfpathlineto{\pgfqpoint{1.245947in}{1.498124in}}%
\pgfpathlineto{\pgfqpoint{1.248525in}{1.485364in}}%
\pgfpathlineto{\pgfqpoint{1.256256in}{1.485364in}}%
\pgfpathlineto{\pgfqpoint{1.258833in}{1.481799in}}%
\pgfpathlineto{\pgfqpoint{1.261411in}{1.474482in}}%
\pgfpathlineto{\pgfqpoint{1.263988in}{1.473919in}}%
\pgfpathlineto{\pgfqpoint{1.266565in}{1.477671in}}%
\pgfpathlineto{\pgfqpoint{1.276874in}{1.448776in}}%
\pgfpathlineto{\pgfqpoint{1.279451in}{1.439582in}}%
\pgfpathlineto{\pgfqpoint{1.282028in}{1.410123in}}%
\pgfpathlineto{\pgfqpoint{1.284606in}{1.405995in}}%
\pgfpathlineto{\pgfqpoint{1.292337in}{1.421569in}}%
\pgfpathlineto{\pgfqpoint{1.294914in}{1.421944in}}%
\pgfpathlineto{\pgfqpoint{1.297492in}{1.415565in}}%
\pgfpathlineto{\pgfqpoint{1.300069in}{1.421757in}}%
\pgfpathlineto{\pgfqpoint{1.302646in}{1.412750in}}%
\pgfpathlineto{\pgfqpoint{1.312955in}{1.425509in}}%
\pgfpathlineto{\pgfqpoint{1.315532in}{1.406934in}}%
\pgfpathlineto{\pgfqpoint{1.318109in}{1.413126in}}%
\pgfpathlineto{\pgfqpoint{1.320687in}{1.372597in}}%
\pgfpathlineto{\pgfqpoint{1.328418in}{1.373910in}}%
\pgfpathlineto{\pgfqpoint{1.330995in}{1.381978in}}%
\pgfpathlineto{\pgfqpoint{1.333573in}{1.405620in}}%
\pgfpathlineto{\pgfqpoint{1.336150in}{1.403181in}}%
\pgfpathlineto{\pgfqpoint{1.338727in}{1.413688in}}%
\pgfpathlineto{\pgfqpoint{1.346459in}{1.400929in}}%
\pgfpathlineto{\pgfqpoint{1.349036in}{1.424196in}}%
\pgfpathlineto{\pgfqpoint{1.351613in}{1.400929in}}%
\pgfpathlineto{\pgfqpoint{1.354190in}{1.399991in}}%
\pgfpathlineto{\pgfqpoint{1.356768in}{1.420819in}}%
\pgfpathlineto{\pgfqpoint{1.364499in}{1.413501in}}%
\pgfpathlineto{\pgfqpoint{1.367076in}{1.431889in}}%
\pgfpathlineto{\pgfqpoint{1.369654in}{1.440332in}}%
\pgfpathlineto{\pgfqpoint{1.372231in}{1.421381in}}%
\pgfpathlineto{\pgfqpoint{1.374808in}{1.429450in}}%
\pgfpathlineto{\pgfqpoint{1.382540in}{1.416878in}}%
\pgfpathlineto{\pgfqpoint{1.385117in}{1.418004in}}%
\pgfpathlineto{\pgfqpoint{1.387694in}{1.430951in}}%
\pgfpathlineto{\pgfqpoint{1.390271in}{1.427761in}}%
\pgfpathlineto{\pgfqpoint{1.392849in}{1.453467in}}%
\pgfpathlineto{\pgfqpoint{1.400580in}{1.467539in}}%
\pgfpathlineto{\pgfqpoint{1.403157in}{1.477109in}}%
\pgfpathlineto{\pgfqpoint{1.408312in}{1.474482in}}%
\pgfpathlineto{\pgfqpoint{1.410889in}{1.463787in}}%
\pgfpathlineto{\pgfqpoint{1.418621in}{1.460222in}}%
\pgfpathlineto{\pgfqpoint{1.421198in}{1.459846in}}%
\pgfpathlineto{\pgfqpoint{1.423775in}{1.454593in}}%
\pgfpathlineto{\pgfqpoint{1.426352in}{1.435079in}}%
\pgfpathlineto{\pgfqpoint{1.428930in}{1.451966in}}%
\pgfpathlineto{\pgfqpoint{1.436661in}{1.463974in}}%
\pgfpathlineto{\pgfqpoint{1.439238in}{1.464725in}}%
\pgfpathlineto{\pgfqpoint{1.441816in}{1.458158in}}%
\pgfpathlineto{\pgfqpoint{1.444393in}{1.423258in}}%
\pgfpathlineto{\pgfqpoint{1.446970in}{1.416128in}}%
\pgfpathlineto{\pgfqpoint{1.454702in}{1.414627in}}%
\pgfpathlineto{\pgfqpoint{1.457279in}{1.412938in}}%
\pgfpathlineto{\pgfqpoint{1.459856in}{1.420068in}}%
\pgfpathlineto{\pgfqpoint{1.462433in}{1.449151in}}%
\pgfpathlineto{\pgfqpoint{1.465010in}{1.462098in}}%
\pgfpathlineto{\pgfqpoint{1.472742in}{1.458158in}}%
\pgfpathlineto{\pgfqpoint{1.475319in}{1.448213in}}%
\pgfpathlineto{\pgfqpoint{1.477897in}{1.432827in}}%
\pgfpathlineto{\pgfqpoint{1.480474in}{1.427573in}}%
\pgfpathlineto{\pgfqpoint{1.483051in}{1.446524in}}%
\pgfpathlineto{\pgfqpoint{1.490783in}{1.437893in}}%
\pgfpathlineto{\pgfqpoint{1.493360in}{1.447087in}}%
\pgfpathlineto{\pgfqpoint{1.495937in}{1.451590in}}%
\pgfpathlineto{\pgfqpoint{1.498514in}{1.427198in}}%
\pgfpathlineto{\pgfqpoint{1.501091in}{1.416691in}}%
\pgfpathlineto{\pgfqpoint{1.508823in}{1.421381in}}%
\pgfpathlineto{\pgfqpoint{1.511400in}{1.420631in}}%
\pgfpathlineto{\pgfqpoint{1.513978in}{1.430388in}}%
\pgfpathlineto{\pgfqpoint{1.516555in}{1.442584in}}%
\pgfpathlineto{\pgfqpoint{1.519132in}{1.446149in}}%
\pgfpathlineto{\pgfqpoint{1.526864in}{1.434516in}}%
\pgfpathlineto{\pgfqpoint{1.529441in}{1.429450in}}%
\pgfpathlineto{\pgfqpoint{1.532018in}{1.433202in}}%
\pgfpathlineto{\pgfqpoint{1.534595in}{1.426448in}}%
\pgfpathlineto{\pgfqpoint{1.537172in}{1.444460in}}%
\pgfpathlineto{\pgfqpoint{1.544904in}{1.443335in}}%
\pgfpathlineto{\pgfqpoint{1.547481in}{1.445774in}}%
\pgfpathlineto{\pgfqpoint{1.550058in}{1.443335in}}%
\pgfpathlineto{\pgfqpoint{1.552636in}{1.439207in}}%
\pgfpathlineto{\pgfqpoint{1.555213in}{1.458345in}}%
\pgfpathlineto{\pgfqpoint{1.565522in}{1.463599in}}%
\pgfpathlineto{\pgfqpoint{1.568099in}{1.439394in}}%
\pgfpathlineto{\pgfqpoint{1.570676in}{1.443147in}}%
\pgfpathlineto{\pgfqpoint{1.573253in}{1.448588in}}%
\pgfpathlineto{\pgfqpoint{1.580985in}{1.445023in}}%
\pgfpathlineto{\pgfqpoint{1.583562in}{1.440145in}}%
\pgfpathlineto{\pgfqpoint{1.586139in}{1.440895in}}%
\pgfpathlineto{\pgfqpoint{1.588717in}{1.470917in}}%
\pgfpathlineto{\pgfqpoint{1.591294in}{1.474669in}}%
\pgfpathlineto{\pgfqpoint{1.599026in}{1.473168in}}%
\pgfpathlineto{\pgfqpoint{1.601603in}{1.464537in}}%
\pgfpathlineto{\pgfqpoint{1.604180in}{1.464537in}}%
\pgfpathlineto{\pgfqpoint{1.606757in}{1.458908in}}%
\pgfpathlineto{\pgfqpoint{1.609334in}{1.450840in}}%
\pgfpathlineto{\pgfqpoint{1.617066in}{1.447650in}}%
\pgfpathlineto{\pgfqpoint{1.619643in}{1.438456in}}%
\pgfpathlineto{\pgfqpoint{1.622220in}{1.421569in}}%
\pgfpathlineto{\pgfqpoint{1.624798in}{1.428887in}}%
\pgfpathlineto{\pgfqpoint{1.635106in}{1.448588in}}%
\pgfpathlineto{\pgfqpoint{1.637684in}{1.439207in}}%
\pgfpathlineto{\pgfqpoint{1.640261in}{1.446524in}}%
\pgfpathlineto{\pgfqpoint{1.642838in}{1.462848in}}%
\pgfpathlineto{\pgfqpoint{1.645415in}{1.466226in}}%
\pgfpathlineto{\pgfqpoint{1.653147in}{1.470541in}}%
\pgfpathlineto{\pgfqpoint{1.655724in}{1.460784in}}%
\pgfpathlineto{\pgfqpoint{1.658301in}{1.456094in}}%
\pgfpathlineto{\pgfqpoint{1.660879in}{1.464725in}}%
\pgfpathlineto{\pgfqpoint{1.663456in}{1.454780in}}%
\pgfpathlineto{\pgfqpoint{1.671187in}{1.449714in}}%
\pgfpathlineto{\pgfqpoint{1.673765in}{1.467352in}}%
\pgfpathlineto{\pgfqpoint{1.676342in}{1.479923in}}%
\pgfpathlineto{\pgfqpoint{1.678919in}{1.450089in}}%
\pgfpathlineto{\pgfqpoint{1.681496in}{1.437330in}}%
\pgfpathlineto{\pgfqpoint{1.689228in}{1.435266in}}%
\pgfpathlineto{\pgfqpoint{1.691805in}{1.412000in}}%
\pgfpathlineto{\pgfqpoint{1.694382in}{1.409560in}}%
\pgfpathlineto{\pgfqpoint{1.696960in}{1.413126in}}%
\pgfpathlineto{\pgfqpoint{1.699537in}{1.418379in}}%
\pgfpathlineto{\pgfqpoint{1.712423in}{1.422132in}}%
\pgfpathlineto{\pgfqpoint{1.715000in}{1.437143in}}%
\pgfpathlineto{\pgfqpoint{1.717577in}{1.434703in}}%
\pgfpathlineto{\pgfqpoint{1.725309in}{1.428699in}}%
\pgfpathlineto{\pgfqpoint{1.727886in}{1.442772in}}%
\pgfpathlineto{\pgfqpoint{1.730463in}{1.415377in}}%
\pgfpathlineto{\pgfqpoint{1.733041in}{1.415189in}}%
\pgfpathlineto{\pgfqpoint{1.735618in}{1.419693in}}%
\pgfpathlineto{\pgfqpoint{1.743349in}{1.414439in}}%
\pgfpathlineto{\pgfqpoint{1.745927in}{1.401117in}}%
\pgfpathlineto{\pgfqpoint{1.748504in}{1.382541in}}%
\pgfpathlineto{\pgfqpoint{1.751081in}{1.382354in}}%
\pgfpathlineto{\pgfqpoint{1.753658in}{1.393612in}}%
\pgfpathlineto{\pgfqpoint{1.761390in}{1.409560in}}%
\pgfpathlineto{\pgfqpoint{1.763967in}{1.420256in}}%
\pgfpathlineto{\pgfqpoint{1.766544in}{1.422320in}}%
\pgfpathlineto{\pgfqpoint{1.771699in}{1.431326in}}%
\pgfpathlineto{\pgfqpoint{1.779430in}{1.417253in}}%
\pgfpathlineto{\pgfqpoint{1.782008in}{1.396051in}}%
\pgfpathlineto{\pgfqpoint{1.784585in}{1.414251in}}%
\pgfpathlineto{\pgfqpoint{1.787162in}{1.421194in}}%
\pgfpathlineto{\pgfqpoint{1.789739in}{1.420819in}}%
\pgfpathlineto{\pgfqpoint{1.797471in}{1.422507in}}%
\pgfpathlineto{\pgfqpoint{1.800048in}{1.419880in}}%
\pgfpathlineto{\pgfqpoint{1.802625in}{1.429450in}}%
\pgfpathlineto{\pgfqpoint{1.805203in}{1.424571in}}%
\pgfpathlineto{\pgfqpoint{1.807780in}{1.433015in}}%
\pgfpathlineto{\pgfqpoint{1.815511in}{1.435454in}}%
\pgfpathlineto{\pgfqpoint{1.818089in}{1.440708in}}%
\pgfpathlineto{\pgfqpoint{1.820666in}{1.451028in}}%
\pgfpathlineto{\pgfqpoint{1.823243in}{1.451966in}}%
\pgfpathlineto{\pgfqpoint{1.825820in}{1.433578in}}%
\pgfpathlineto{\pgfqpoint{1.833552in}{1.444273in}}%
\pgfpathlineto{\pgfqpoint{1.836129in}{1.453654in}}%
\pgfpathlineto{\pgfqpoint{1.838706in}{1.436017in}}%
\pgfpathlineto{\pgfqpoint{1.841283in}{1.446524in}}%
\pgfpathlineto{\pgfqpoint{1.843861in}{1.450652in}}%
\pgfpathlineto{\pgfqpoint{1.851592in}{1.448588in}}%
\pgfpathlineto{\pgfqpoint{1.856747in}{1.441833in}}%
\pgfpathlineto{\pgfqpoint{1.859324in}{1.434328in}}%
\pgfpathlineto{\pgfqpoint{1.861901in}{1.433953in}}%
\pgfpathlineto{\pgfqpoint{1.869633in}{1.447838in}}%
\pgfpathlineto{\pgfqpoint{1.874787in}{1.472981in}}%
\pgfpathlineto{\pgfqpoint{1.877364in}{1.477109in}}%
\pgfpathlineto{\pgfqpoint{1.879942in}{1.487616in}}%
\pgfpathlineto{\pgfqpoint{1.887673in}{1.491932in}}%
\pgfpathlineto{\pgfqpoint{1.890251in}{1.497748in}}%
\pgfpathlineto{\pgfqpoint{1.892828in}{1.498686in}}%
\pgfpathlineto{\pgfqpoint{1.897982in}{1.515386in}}%
\pgfpathlineto{\pgfqpoint{1.905714in}{1.514823in}}%
\pgfpathlineto{\pgfqpoint{1.908291in}{1.507318in}}%
\pgfpathlineto{\pgfqpoint{1.910868in}{1.504878in}}%
\pgfpathlineto{\pgfqpoint{1.913445in}{1.506942in}}%
\pgfpathlineto{\pgfqpoint{1.916023in}{1.490431in}}%
\pgfpathlineto{\pgfqpoint{1.926332in}{1.484051in}}%
\pgfpathlineto{\pgfqpoint{1.928909in}{1.475983in}}%
\pgfpathlineto{\pgfqpoint{1.931486in}{1.482362in}}%
\pgfpathlineto{\pgfqpoint{1.934063in}{1.485740in}}%
\pgfpathlineto{\pgfqpoint{1.941795in}{1.478422in}}%
\pgfpathlineto{\pgfqpoint{1.944372in}{1.484802in}}%
\pgfpathlineto{\pgfqpoint{1.946949in}{1.482738in}}%
\pgfpathlineto{\pgfqpoint{1.949526in}{1.473919in}}%
\pgfpathlineto{\pgfqpoint{1.952104in}{1.492682in}}%
\pgfpathlineto{\pgfqpoint{1.959835in}{1.484239in}}%
\pgfpathlineto{\pgfqpoint{1.962412in}{1.505441in}}%
\pgfpathlineto{\pgfqpoint{1.964990in}{1.503753in}}%
\pgfpathlineto{\pgfqpoint{1.967567in}{1.530772in}}%
\pgfpathlineto{\pgfqpoint{1.970144in}{1.524955in}}%
\pgfpathlineto{\pgfqpoint{1.977876in}{1.528145in}}%
\pgfpathlineto{\pgfqpoint{1.980453in}{1.531898in}}%
\pgfpathlineto{\pgfqpoint{1.983030in}{1.530021in}}%
\pgfpathlineto{\pgfqpoint{1.985607in}{1.534149in}}%
\pgfpathlineto{\pgfqpoint{1.988185in}{1.523079in}}%
\pgfpathlineto{\pgfqpoint{1.998493in}{1.533211in}}%
\pgfpathlineto{\pgfqpoint{2.003648in}{1.520640in}}%
\pgfpathlineto{\pgfqpoint{2.006225in}{1.538089in}}%
\pgfpathlineto{\pgfqpoint{2.013957in}{1.528708in}}%
\pgfpathlineto{\pgfqpoint{2.016534in}{1.527019in}}%
\pgfpathlineto{\pgfqpoint{2.019111in}{1.536964in}}%
\pgfpathlineto{\pgfqpoint{2.021688in}{1.530959in}}%
\pgfpathlineto{\pgfqpoint{2.024266in}{1.534712in}}%
\pgfpathlineto{\pgfqpoint{2.031997in}{1.543531in}}%
\pgfpathlineto{\pgfqpoint{2.034574in}{1.564546in}}%
\pgfpathlineto{\pgfqpoint{2.037152in}{1.573740in}}%
\pgfpathlineto{\pgfqpoint{2.039729in}{1.572051in}}%
\pgfpathlineto{\pgfqpoint{2.042306in}{1.574490in}}%
\pgfpathlineto{\pgfqpoint{2.050038in}{1.588188in}}%
\pgfpathlineto{\pgfqpoint{2.052615in}{1.584623in}}%
\pgfpathlineto{\pgfqpoint{2.055192in}{1.584810in}}%
\pgfpathlineto{\pgfqpoint{2.057769in}{1.586124in}}%
\pgfpathlineto{\pgfqpoint{2.060347in}{1.598320in}}%
\pgfpathlineto{\pgfqpoint{2.068078in}{1.592878in}}%
\pgfpathlineto{\pgfqpoint{2.070655in}{1.581808in}}%
\pgfpathlineto{\pgfqpoint{2.073233in}{1.596631in}}%
\pgfpathlineto{\pgfqpoint{2.075810in}{1.586687in}}%
\pgfpathlineto{\pgfqpoint{2.078387in}{1.600384in}}%
\pgfpathlineto{\pgfqpoint{2.086119in}{1.598508in}}%
\pgfpathlineto{\pgfqpoint{2.088696in}{1.616708in}}%
\pgfpathlineto{\pgfqpoint{2.091273in}{1.616333in}}%
\pgfpathlineto{\pgfqpoint{2.093850in}{1.621586in}}%
\pgfpathlineto{\pgfqpoint{2.104159in}{1.618209in}}%
\pgfpathlineto{\pgfqpoint{2.106736in}{1.624776in}}%
\pgfpathlineto{\pgfqpoint{2.109314in}{1.604324in}}%
\pgfpathlineto{\pgfqpoint{2.111891in}{1.612580in}}%
\pgfpathlineto{\pgfqpoint{2.114468in}{1.588000in}}%
\pgfpathlineto{\pgfqpoint{2.122200in}{1.593254in}}%
\pgfpathlineto{\pgfqpoint{2.124777in}{1.586687in}}%
\pgfpathlineto{\pgfqpoint{2.127354in}{1.589876in}}%
\pgfpathlineto{\pgfqpoint{2.129931in}{1.595693in}}%
\pgfpathlineto{\pgfqpoint{2.132509in}{1.594567in}}%
\pgfpathlineto{\pgfqpoint{2.140240in}{1.567548in}}%
\pgfpathlineto{\pgfqpoint{2.142817in}{1.575991in}}%
\pgfpathlineto{\pgfqpoint{2.145395in}{1.568111in}}%
\pgfpathlineto{\pgfqpoint{2.147972in}{1.583684in}}%
\pgfpathlineto{\pgfqpoint{2.150549in}{1.621211in}}%
\pgfpathlineto{\pgfqpoint{2.158281in}{1.611454in}}%
\pgfpathlineto{\pgfqpoint{2.160858in}{1.624964in}}%
\pgfpathlineto{\pgfqpoint{2.163435in}{1.624401in}}%
\pgfpathlineto{\pgfqpoint{2.166012in}{1.636597in}}%
\pgfpathlineto{\pgfqpoint{2.168589in}{1.630030in}}%
\pgfpathlineto{\pgfqpoint{2.176321in}{1.627966in}}%
\pgfpathlineto{\pgfqpoint{2.178898in}{1.641288in}}%
\pgfpathlineto{\pgfqpoint{2.181476in}{1.639036in}}%
\pgfpathlineto{\pgfqpoint{2.184053in}{1.657987in}}%
\pgfpathlineto{\pgfqpoint{2.186630in}{1.672435in}}%
\pgfpathlineto{\pgfqpoint{2.194362in}{1.669621in}}%
\pgfpathlineto{\pgfqpoint{2.196939in}{1.672623in}}%
\pgfpathlineto{\pgfqpoint{2.199516in}{1.673936in}}%
\pgfpathlineto{\pgfqpoint{2.212402in}{1.664930in}}%
\pgfpathlineto{\pgfqpoint{2.214979in}{1.694388in}}%
\pgfpathlineto{\pgfqpoint{2.217557in}{1.716154in}}%
\pgfpathlineto{\pgfqpoint{2.220134in}{1.706772in}}%
\pgfpathlineto{\pgfqpoint{2.222711in}{1.725535in}}%
\pgfpathlineto{\pgfqpoint{2.230443in}{1.743924in}}%
\pgfpathlineto{\pgfqpoint{2.233020in}{1.756120in}}%
\pgfpathlineto{\pgfqpoint{2.235597in}{1.744674in}}%
\pgfpathlineto{\pgfqpoint{2.238174in}{1.748990in}}%
\pgfpathlineto{\pgfqpoint{2.240751in}{1.758934in}}%
\pgfpathlineto{\pgfqpoint{2.251060in}{1.774133in}}%
\pgfpathlineto{\pgfqpoint{2.253637in}{1.768504in}}%
\pgfpathlineto{\pgfqpoint{2.256215in}{1.773757in}}%
\pgfpathlineto{\pgfqpoint{2.258792in}{1.766440in}}%
\pgfpathlineto{\pgfqpoint{2.266524in}{1.779386in}}%
\pgfpathlineto{\pgfqpoint{2.269101in}{1.772444in}}%
\pgfpathlineto{\pgfqpoint{2.271678in}{1.750115in}}%
\pgfpathlineto{\pgfqpoint{2.276832in}{1.806218in}}%
\pgfpathlineto{\pgfqpoint{2.284564in}{1.810346in}}%
\pgfpathlineto{\pgfqpoint{2.289718in}{1.749365in}}%
\pgfpathlineto{\pgfqpoint{2.292296in}{1.757808in}}%
\pgfpathlineto{\pgfqpoint{2.294873in}{1.719531in}}%
\pgfpathlineto{\pgfqpoint{2.302605in}{1.734354in}}%
\pgfpathlineto{\pgfqpoint{2.305182in}{1.753868in}}%
\pgfpathlineto{\pgfqpoint{2.307759in}{1.741109in}}%
\pgfpathlineto{\pgfqpoint{2.310336in}{1.718593in}}%
\pgfpathlineto{\pgfqpoint{2.312913in}{1.725348in}}%
\pgfpathlineto{\pgfqpoint{2.320645in}{1.703019in}}%
\pgfpathlineto{\pgfqpoint{2.323222in}{1.723847in}}%
\pgfpathlineto{\pgfqpoint{2.325799in}{1.735292in}}%
\pgfpathlineto{\pgfqpoint{2.328377in}{1.756307in}}%
\pgfpathlineto{\pgfqpoint{2.330954in}{1.750115in}}%
\pgfpathlineto{\pgfqpoint{2.338685in}{1.765126in}}%
\pgfpathlineto{\pgfqpoint{2.341263in}{1.751617in}}%
\pgfpathlineto{\pgfqpoint{2.343840in}{1.750866in}}%
\pgfpathlineto{\pgfqpoint{2.348994in}{1.780700in}}%
\pgfpathlineto{\pgfqpoint{2.356726in}{1.793271in}}%
\pgfpathlineto{\pgfqpoint{2.359303in}{1.803028in}}%
\pgfpathlineto{\pgfqpoint{2.361880in}{1.781075in}}%
\pgfpathlineto{\pgfqpoint{2.364458in}{1.791395in}}%
\pgfpathlineto{\pgfqpoint{2.367035in}{1.815412in}}%
\pgfpathlineto{\pgfqpoint{2.374766in}{1.807156in}}%
\pgfpathlineto{\pgfqpoint{2.377344in}{1.814286in}}%
\pgfpathlineto{\pgfqpoint{2.379921in}{1.789143in}}%
\pgfpathlineto{\pgfqpoint{2.382498in}{1.741109in}}%
\pgfpathlineto{\pgfqpoint{2.385075in}{1.742047in}}%
\pgfpathlineto{\pgfqpoint{2.392807in}{1.753305in}}%
\pgfpathlineto{\pgfqpoint{2.395384in}{1.748051in}}%
\pgfpathlineto{\pgfqpoint{2.397961in}{1.764376in}}%
\pgfpathlineto{\pgfqpoint{2.400539in}{1.771318in}}%
\pgfpathlineto{\pgfqpoint{2.403116in}{1.764000in}}%
\pgfpathlineto{\pgfqpoint{2.410847in}{1.759122in}}%
\pgfpathlineto{\pgfqpoint{2.413425in}{1.761373in}}%
\pgfpathlineto{\pgfqpoint{2.416002in}{1.736981in}}%
\pgfpathlineto{\pgfqpoint{2.418579in}{1.768879in}}%
\pgfpathlineto{\pgfqpoint{2.421156in}{1.770005in}}%
\pgfpathlineto{\pgfqpoint{2.428888in}{1.774133in}}%
\pgfpathlineto{\pgfqpoint{2.431465in}{1.772069in}}%
\pgfpathlineto{\pgfqpoint{2.434042in}{1.760623in}}%
\pgfpathlineto{\pgfqpoint{2.436620in}{1.778636in}}%
\pgfpathlineto{\pgfqpoint{2.439197in}{1.766815in}}%
\pgfpathlineto{\pgfqpoint{2.446928in}{1.766627in}}%
\pgfpathlineto{\pgfqpoint{2.449506in}{1.779199in}}%
\pgfpathlineto{\pgfqpoint{2.452083in}{1.773194in}}%
\pgfpathlineto{\pgfqpoint{2.454660in}{1.756120in}}%
\pgfpathlineto{\pgfqpoint{2.457237in}{1.760998in}}%
\pgfpathlineto{\pgfqpoint{2.464969in}{1.746738in}}%
\pgfpathlineto{\pgfqpoint{2.467546in}{1.745612in}}%
\pgfpathlineto{\pgfqpoint{2.470123in}{1.731352in}}%
\pgfpathlineto{\pgfqpoint{2.472701in}{1.738482in}}%
\pgfpathlineto{\pgfqpoint{2.475278in}{1.734917in}}%
\pgfpathlineto{\pgfqpoint{2.483009in}{1.734354in}}%
\pgfpathlineto{\pgfqpoint{2.485587in}{1.705083in}}%
\pgfpathlineto{\pgfqpoint{2.488164in}{1.706772in}}%
\pgfpathlineto{\pgfqpoint{2.490741in}{1.710337in}}%
\pgfpathlineto{\pgfqpoint{2.493318in}{1.705083in}}%
\pgfpathlineto{\pgfqpoint{2.503627in}{1.713902in}}%
\pgfpathlineto{\pgfqpoint{2.508782in}{1.737544in}}%
\pgfpathlineto{\pgfqpoint{2.511359in}{1.729851in}}%
\pgfpathlineto{\pgfqpoint{2.519090in}{1.735105in}}%
\pgfpathlineto{\pgfqpoint{2.521668in}{1.751241in}}%
\pgfpathlineto{\pgfqpoint{2.524245in}{1.761186in}}%
\pgfpathlineto{\pgfqpoint{2.526822in}{1.763062in}}%
\pgfpathlineto{\pgfqpoint{2.529399in}{1.763250in}}%
\pgfpathlineto{\pgfqpoint{2.537131in}{1.768504in}}%
\pgfpathlineto{\pgfqpoint{2.539708in}{1.791207in}}%
\pgfpathlineto{\pgfqpoint{2.542285in}{1.803591in}}%
\pgfpathlineto{\pgfqpoint{2.544862in}{1.802278in}}%
\pgfpathlineto{\pgfqpoint{2.547440in}{1.797962in}}%
\pgfpathlineto{\pgfqpoint{2.555171in}{1.782951in}}%
\pgfpathlineto{\pgfqpoint{2.557749in}{1.776572in}}%
\pgfpathlineto{\pgfqpoint{2.560326in}{1.775071in}}%
\pgfpathlineto{\pgfqpoint{2.562903in}{1.780887in}}%
\pgfpathlineto{\pgfqpoint{2.565480in}{1.773382in}}%
\pgfpathlineto{\pgfqpoint{2.573212in}{1.767002in}}%
\pgfpathlineto{\pgfqpoint{2.575789in}{1.774133in}}%
\pgfpathlineto{\pgfqpoint{2.578366in}{1.754806in}}%
\pgfpathlineto{\pgfqpoint{2.580943in}{1.745237in}}%
\pgfpathlineto{\pgfqpoint{2.583521in}{1.750115in}}%
\pgfpathlineto{\pgfqpoint{2.591252in}{1.726474in}}%
\pgfpathlineto{\pgfqpoint{2.593830in}{1.714090in}}%
\pgfpathlineto{\pgfqpoint{2.596407in}{1.714090in}}%
\pgfpathlineto{\pgfqpoint{2.598984in}{1.756120in}}%
\pgfpathlineto{\pgfqpoint{2.601561in}{1.768879in}}%
\pgfpathlineto{\pgfqpoint{2.609293in}{1.780512in}}%
\pgfpathlineto{\pgfqpoint{2.611870in}{1.766440in}}%
\pgfpathlineto{\pgfqpoint{2.614447in}{1.784828in}}%
\pgfpathlineto{\pgfqpoint{2.617024in}{1.852188in}}%
\pgfpathlineto{\pgfqpoint{2.619602in}{1.857254in}}%
\pgfpathlineto{\pgfqpoint{2.627333in}{1.855190in}}%
\pgfpathlineto{\pgfqpoint{2.629911in}{1.863258in}}%
\pgfpathlineto{\pgfqpoint{2.632488in}{1.859131in}}%
\pgfpathlineto{\pgfqpoint{2.635065in}{1.863821in}}%
\pgfpathlineto{\pgfqpoint{2.637642in}{1.893280in}}%
\pgfpathlineto{\pgfqpoint{2.645374in}{1.895719in}}%
\pgfpathlineto{\pgfqpoint{2.647951in}{1.911293in}}%
\pgfpathlineto{\pgfqpoint{2.650528in}{1.901723in}}%
\pgfpathlineto{\pgfqpoint{2.653105in}{1.879207in}}%
\pgfpathlineto{\pgfqpoint{2.655683in}{1.885399in}}%
\pgfpathlineto{\pgfqpoint{2.663414in}{1.881459in}}%
\pgfpathlineto{\pgfqpoint{2.665991in}{1.880896in}}%
\pgfpathlineto{\pgfqpoint{2.668569in}{1.884836in}}%
\pgfpathlineto{\pgfqpoint{2.671146in}{1.864009in}}%
\pgfpathlineto{\pgfqpoint{2.673723in}{1.878832in}}%
\pgfpathlineto{\pgfqpoint{2.681455in}{1.872453in}}%
\pgfpathlineto{\pgfqpoint{2.684032in}{1.866073in}}%
\pgfpathlineto{\pgfqpoint{2.689186in}{1.880896in}}%
\pgfpathlineto{\pgfqpoint{2.691764in}{1.896470in}}%
\pgfpathlineto{\pgfqpoint{2.699495in}{1.888589in}}%
\pgfpathlineto{\pgfqpoint{2.702072in}{1.890653in}}%
\pgfpathlineto{\pgfqpoint{2.704650in}{1.887088in}}%
\pgfpathlineto{\pgfqpoint{2.707227in}{1.916359in}}%
\pgfpathlineto{\pgfqpoint{2.709804in}{1.914670in}}%
\pgfpathlineto{\pgfqpoint{2.717536in}{1.927429in}}%
\pgfpathlineto{\pgfqpoint{2.722690in}{1.944504in}}%
\pgfpathlineto{\pgfqpoint{2.727845in}{1.948256in}}%
\pgfpathlineto{\pgfqpoint{2.735576in}{1.939438in}}%
\pgfpathlineto{\pgfqpoint{2.738153in}{1.927617in}}%
\pgfpathlineto{\pgfqpoint{2.740731in}{1.925740in}}%
\pgfpathlineto{\pgfqpoint{2.743308in}{1.926491in}}%
\pgfpathlineto{\pgfqpoint{2.745885in}{1.950508in}}%
\pgfpathlineto{\pgfqpoint{2.753617in}{1.947318in}}%
\pgfpathlineto{\pgfqpoint{2.756194in}{1.939438in}}%
\pgfpathlineto{\pgfqpoint{2.758771in}{1.915421in}}%
\pgfpathlineto{\pgfqpoint{2.761348in}{1.905101in}}%
\pgfpathlineto{\pgfqpoint{2.763926in}{1.911668in}}%
\pgfpathlineto{\pgfqpoint{2.771657in}{1.926116in}}%
\pgfpathlineto{\pgfqpoint{2.774234in}{1.918986in}}%
\pgfpathlineto{\pgfqpoint{2.776812in}{1.951446in}}%
\pgfpathlineto{\pgfqpoint{2.779389in}{1.958764in}}%
\pgfpathlineto{\pgfqpoint{2.781966in}{1.978653in}}%
\pgfpathlineto{\pgfqpoint{2.789698in}{1.991412in}}%
\pgfpathlineto{\pgfqpoint{2.792275in}{1.997979in}}%
\pgfpathlineto{\pgfqpoint{2.807738in}{2.013553in}}%
\pgfpathlineto{\pgfqpoint{2.810315in}{2.033067in}}%
\pgfpathlineto{\pgfqpoint{2.815470in}{2.011114in}}%
\pgfpathlineto{\pgfqpoint{2.818047in}{2.016180in}}%
\pgfpathlineto{\pgfqpoint{2.825779in}{2.015429in}}%
\pgfpathlineto{\pgfqpoint{2.828356in}{2.009613in}}%
\pgfpathlineto{\pgfqpoint{2.830933in}{2.014304in}}%
\pgfpathlineto{\pgfqpoint{2.833510in}{2.005297in}}%
\pgfpathlineto{\pgfqpoint{2.836088in}{1.999480in}}%
\pgfpathlineto{\pgfqpoint{2.843819in}{1.972649in}}%
\pgfpathlineto{\pgfqpoint{2.846396in}{1.974900in}}%
\pgfpathlineto{\pgfqpoint{2.848974in}{1.994414in}}%
\pgfpathlineto{\pgfqpoint{2.851551in}{1.986158in}}%
\pgfpathlineto{\pgfqpoint{2.854128in}{2.041323in}}%
\pgfpathlineto{\pgfqpoint{2.864437in}{2.034756in}}%
\pgfpathlineto{\pgfqpoint{2.867014in}{2.044137in}}%
\pgfpathlineto{\pgfqpoint{2.872168in}{1.971898in}}%
\pgfpathlineto{\pgfqpoint{2.879900in}{1.950696in}}%
\pgfpathlineto{\pgfqpoint{2.882477in}{1.966457in}}%
\pgfpathlineto{\pgfqpoint{2.885055in}{1.947694in}}%
\pgfpathlineto{\pgfqpoint{2.887632in}{1.966082in}}%
\pgfpathlineto{\pgfqpoint{2.890209in}{1.938500in}}%
\pgfpathlineto{\pgfqpoint{2.897941in}{1.901160in}}%
\pgfpathlineto{\pgfqpoint{2.900518in}{1.920862in}}%
\pgfpathlineto{\pgfqpoint{2.903095in}{1.915984in}}%
\pgfpathlineto{\pgfqpoint{2.905672in}{1.950133in}}%
\pgfpathlineto{\pgfqpoint{2.908249in}{1.972649in}}%
\pgfpathlineto{\pgfqpoint{2.915981in}{1.995915in}}%
\pgfpathlineto{\pgfqpoint{2.918558in}{2.000043in}}%
\pgfpathlineto{\pgfqpoint{2.921136in}{2.007549in}}%
\pgfpathlineto{\pgfqpoint{2.923713in}{2.005485in}}%
\pgfpathlineto{\pgfqpoint{2.926290in}{2.007361in}}%
\pgfpathlineto{\pgfqpoint{2.936599in}{2.007549in}}%
\pgfpathlineto{\pgfqpoint{2.939176in}{2.004547in}}%
\pgfpathlineto{\pgfqpoint{2.941753in}{2.007549in}}%
\pgfpathlineto{\pgfqpoint{2.944330in}{2.003046in}}%
\pgfpathlineto{\pgfqpoint{2.952062in}{2.023122in}}%
\pgfpathlineto{\pgfqpoint{2.954639in}{2.022935in}}%
\pgfpathlineto{\pgfqpoint{2.957216in}{2.019933in}}%
\pgfpathlineto{\pgfqpoint{2.959794in}{2.029502in}}%
\pgfpathlineto{\pgfqpoint{2.962371in}{2.046764in}}%
\pgfpathlineto{\pgfqpoint{2.970103in}{2.024060in}}%
\pgfpathlineto{\pgfqpoint{2.972680in}{2.069655in}}%
\pgfpathlineto{\pgfqpoint{2.975257in}{2.061587in}}%
\pgfpathlineto{\pgfqpoint{2.977834in}{2.085417in}}%
\pgfpathlineto{\pgfqpoint{2.980411in}{2.091233in}}%
\pgfpathlineto{\pgfqpoint{2.988143in}{2.088419in}}%
\pgfpathlineto{\pgfqpoint{2.990720in}{2.079788in}}%
\pgfpathlineto{\pgfqpoint{2.993297in}{2.074722in}}%
\pgfpathlineto{\pgfqpoint{2.995875in}{2.036444in}}%
\pgfpathlineto{\pgfqpoint{2.998452in}{2.027438in}}%
\pgfpathlineto{\pgfqpoint{3.008761in}{2.052018in}}%
\pgfpathlineto{\pgfqpoint{3.011338in}{2.037195in}}%
\pgfpathlineto{\pgfqpoint{3.013915in}{2.053707in}}%
\pgfpathlineto{\pgfqpoint{3.026801in}{2.039071in}}%
\pgfpathlineto{\pgfqpoint{3.029378in}{2.018619in}}%
\pgfpathlineto{\pgfqpoint{3.031956in}{2.024436in}}%
\pgfpathlineto{\pgfqpoint{3.034533in}{2.032504in}}%
\pgfpathlineto{\pgfqpoint{3.042264in}{2.024999in}}%
\pgfpathlineto{\pgfqpoint{3.044842in}{2.044700in}}%
\pgfpathlineto{\pgfqpoint{3.047419in}{2.035506in}}%
\pgfpathlineto{\pgfqpoint{3.049996in}{2.045451in}}%
\pgfpathlineto{\pgfqpoint{3.052573in}{2.014116in}}%
\pgfpathlineto{\pgfqpoint{3.060305in}{1.969647in}}%
\pgfpathlineto{\pgfqpoint{3.062882in}{1.967770in}}%
\pgfpathlineto{\pgfqpoint{3.065459in}{2.006423in}}%
\pgfpathlineto{\pgfqpoint{3.068037in}{1.948069in}}%
\pgfpathlineto{\pgfqpoint{3.070614in}{1.933996in}}%
\pgfpathlineto{\pgfqpoint{3.078345in}{1.950508in}}%
\pgfpathlineto{\pgfqpoint{3.080923in}{1.959890in}}%
\pgfpathlineto{\pgfqpoint{3.083500in}{1.983532in}}%
\pgfpathlineto{\pgfqpoint{3.086077in}{1.963080in}}%
\pgfpathlineto{\pgfqpoint{3.096386in}{1.970773in}}%
\pgfpathlineto{\pgfqpoint{3.098963in}{1.977715in}}%
\pgfpathlineto{\pgfqpoint{3.101540in}{1.978841in}}%
\pgfpathlineto{\pgfqpoint{3.104118in}{1.983719in}}%
\pgfpathlineto{\pgfqpoint{3.106695in}{1.977152in}}%
\pgfpathlineto{\pgfqpoint{3.114426in}{1.977527in}}%
\pgfpathlineto{\pgfqpoint{3.117004in}{1.990286in}}%
\pgfpathlineto{\pgfqpoint{3.119581in}{1.983907in}}%
\pgfpathlineto{\pgfqpoint{3.122158in}{1.973399in}}%
\pgfpathlineto{\pgfqpoint{3.124735in}{1.975276in}}%
\pgfpathlineto{\pgfqpoint{3.132467in}{1.982781in}}%
\pgfpathlineto{\pgfqpoint{3.135044in}{1.962704in}}%
\pgfpathlineto{\pgfqpoint{3.137621in}{1.993476in}}%
\pgfpathlineto{\pgfqpoint{3.140199in}{2.004547in}}%
\pgfpathlineto{\pgfqpoint{3.142776in}{2.008487in}}%
\pgfpathlineto{\pgfqpoint{3.150507in}{2.022559in}}%
\pgfpathlineto{\pgfqpoint{3.155662in}{2.001920in}}%
\pgfpathlineto{\pgfqpoint{3.158239in}{1.986909in}}%
\pgfpathlineto{\pgfqpoint{3.160816in}{1.985220in}}%
\pgfpathlineto{\pgfqpoint{3.168548in}{1.995915in}}%
\pgfpathlineto{\pgfqpoint{3.171125in}{1.978090in}}%
\pgfpathlineto{\pgfqpoint{3.173702in}{1.991412in}}%
\pgfpathlineto{\pgfqpoint{3.176280in}{1.996478in}}%
\pgfpathlineto{\pgfqpoint{3.189166in}{2.052581in}}%
\pgfpathlineto{\pgfqpoint{3.191743in}{2.046952in}}%
\pgfpathlineto{\pgfqpoint{3.196897in}{2.054457in}}%
\pgfpathlineto{\pgfqpoint{3.204629in}{2.061212in}}%
\pgfpathlineto{\pgfqpoint{3.207206in}{2.058397in}}%
\pgfpathlineto{\pgfqpoint{3.209783in}{2.059898in}}%
\pgfpathlineto{\pgfqpoint{3.212361in}{2.076973in}}%
\pgfpathlineto{\pgfqpoint{3.214938in}{2.113562in}}%
\pgfpathlineto{\pgfqpoint{3.222669in}{2.125007in}}%
\pgfpathlineto{\pgfqpoint{3.225247in}{2.119941in}}%
\pgfpathlineto{\pgfqpoint{3.227824in}{2.116564in}}%
\pgfpathlineto{\pgfqpoint{3.230401in}{2.110935in}}%
\pgfpathlineto{\pgfqpoint{3.232978in}{2.112436in}}%
\pgfpathlineto{\pgfqpoint{3.240710in}{2.104180in}}%
\pgfpathlineto{\pgfqpoint{3.243287in}{2.109246in}}%
\pgfpathlineto{\pgfqpoint{3.245864in}{2.124820in}}%
\pgfpathlineto{\pgfqpoint{3.248441in}{2.116376in}}%
\pgfpathlineto{\pgfqpoint{3.251019in}{2.124444in}}%
\pgfpathlineto{\pgfqpoint{3.258750in}{2.123882in}}%
\pgfpathlineto{\pgfqpoint{3.261328in}{2.105306in}}%
\pgfpathlineto{\pgfqpoint{3.263905in}{2.108871in}}%
\pgfpathlineto{\pgfqpoint{3.266482in}{2.103054in}}%
\pgfpathlineto{\pgfqpoint{3.269059in}{2.113937in}}%
\pgfpathlineto{\pgfqpoint{3.276791in}{2.112811in}}%
\pgfpathlineto{\pgfqpoint{3.279368in}{2.121255in}}%
\pgfpathlineto{\pgfqpoint{3.281945in}{2.118815in}}%
\pgfpathlineto{\pgfqpoint{3.284522in}{2.129698in}}%
\pgfpathlineto{\pgfqpoint{3.294831in}{2.122005in}}%
\pgfpathlineto{\pgfqpoint{3.297409in}{2.107745in}}%
\pgfpathlineto{\pgfqpoint{3.299986in}{2.115063in}}%
\pgfpathlineto{\pgfqpoint{3.302563in}{2.110184in}}%
\pgfpathlineto{\pgfqpoint{3.305140in}{2.110372in}}%
\pgfpathlineto{\pgfqpoint{3.312872in}{2.111685in}}%
\pgfpathlineto{\pgfqpoint{3.315449in}{2.110184in}}%
\pgfpathlineto{\pgfqpoint{3.318026in}{2.110372in}}%
\pgfpathlineto{\pgfqpoint{3.320603in}{2.084666in}}%
\pgfpathlineto{\pgfqpoint{3.323181in}{2.094048in}}%
\pgfpathlineto{\pgfqpoint{3.330912in}{2.082790in}}%
\pgfpathlineto{\pgfqpoint{3.333490in}{2.091421in}}%
\pgfpathlineto{\pgfqpoint{3.338644in}{2.087481in}}%
\pgfpathlineto{\pgfqpoint{3.341221in}{2.066278in}}%
\pgfpathlineto{\pgfqpoint{3.348953in}{2.065152in}}%
\pgfpathlineto{\pgfqpoint{3.351530in}{2.062525in}}%
\pgfpathlineto{\pgfqpoint{3.354107in}{2.048640in}}%
\pgfpathlineto{\pgfqpoint{3.356684in}{1.997979in}}%
\pgfpathlineto{\pgfqpoint{3.359262in}{1.971523in}}%
\pgfpathlineto{\pgfqpoint{3.366993in}{1.979404in}}%
\pgfpathlineto{\pgfqpoint{3.369570in}{1.969647in}}%
\pgfpathlineto{\pgfqpoint{3.372148in}{1.970210in}}%
\pgfpathlineto{\pgfqpoint{3.374725in}{1.963642in}}%
\pgfpathlineto{\pgfqpoint{3.377302in}{1.988785in}}%
\pgfpathlineto{\pgfqpoint{3.385034in}{1.979779in}}%
\pgfpathlineto{\pgfqpoint{3.387611in}{1.982031in}}%
\pgfpathlineto{\pgfqpoint{3.390188in}{1.987660in}}%
\pgfpathlineto{\pgfqpoint{3.392765in}{1.985408in}}%
\pgfpathlineto{\pgfqpoint{3.395343in}{1.973775in}}%
\pgfpathlineto{\pgfqpoint{3.403074in}{1.983532in}}%
\pgfpathlineto{\pgfqpoint{3.405651in}{2.000231in}}%
\pgfpathlineto{\pgfqpoint{3.408229in}{2.006611in}}%
\pgfpathlineto{\pgfqpoint{3.410806in}{2.017869in}}%
\pgfpathlineto{\pgfqpoint{3.413383in}{2.013365in}}%
\pgfpathlineto{\pgfqpoint{3.421115in}{2.025749in}}%
\pgfpathlineto{\pgfqpoint{3.423692in}{2.018056in}}%
\pgfpathlineto{\pgfqpoint{3.426269in}{2.019557in}}%
\pgfpathlineto{\pgfqpoint{3.428846in}{2.015805in}}%
\pgfpathlineto{\pgfqpoint{3.431424in}{2.024999in}}%
\pgfpathlineto{\pgfqpoint{3.441732in}{2.028001in}}%
\pgfpathlineto{\pgfqpoint{3.444310in}{2.035318in}}%
\pgfpathlineto{\pgfqpoint{3.446887in}{2.026687in}}%
\pgfpathlineto{\pgfqpoint{3.449464in}{2.025937in}}%
\pgfpathlineto{\pgfqpoint{3.457196in}{2.014304in}}%
\pgfpathlineto{\pgfqpoint{3.459773in}{1.996291in}}%
\pgfpathlineto{\pgfqpoint{3.462350in}{2.005109in}}%
\pgfpathlineto{\pgfqpoint{3.464927in}{2.005297in}}%
\pgfpathlineto{\pgfqpoint{3.467505in}{1.991787in}}%
\pgfpathlineto{\pgfqpoint{3.475236in}{1.987284in}}%
\pgfpathlineto{\pgfqpoint{3.477813in}{2.004359in}}%
\pgfpathlineto{\pgfqpoint{3.480391in}{2.016743in}}%
\pgfpathlineto{\pgfqpoint{3.482968in}{2.034568in}}%
\pgfpathlineto{\pgfqpoint{3.485545in}{2.027626in}}%
\pgfpathlineto{\pgfqpoint{3.493277in}{2.015242in}}%
\pgfpathlineto{\pgfqpoint{3.495854in}{2.003421in}}%
\pgfpathlineto{\pgfqpoint{3.498431in}{2.007173in}}%
\pgfpathlineto{\pgfqpoint{3.501008in}{1.976589in}}%
\pgfpathlineto{\pgfqpoint{3.503586in}{2.004547in}}%
\pgfpathlineto{\pgfqpoint{3.511317in}{1.997229in}}%
\pgfpathlineto{\pgfqpoint{3.513894in}{1.990099in}}%
\pgfpathlineto{\pgfqpoint{3.516472in}{1.964393in}}%
\pgfpathlineto{\pgfqpoint{3.519049in}{1.965519in}}%
\pgfpathlineto{\pgfqpoint{3.521626in}{1.988035in}}%
\pgfpathlineto{\pgfqpoint{3.529358in}{1.985783in}}%
\pgfpathlineto{\pgfqpoint{3.531935in}{1.956512in}}%
\pgfpathlineto{\pgfqpoint{3.534512in}{1.992350in}}%
\pgfpathlineto{\pgfqpoint{3.537089in}{1.965894in}}%
\pgfpathlineto{\pgfqpoint{3.539667in}{1.950133in}}%
\pgfpathlineto{\pgfqpoint{3.547398in}{1.911668in}}%
\pgfpathlineto{\pgfqpoint{3.549975in}{1.910917in}}%
\pgfpathlineto{\pgfqpoint{3.552553in}{1.879395in}}%
\pgfpathlineto{\pgfqpoint{3.555130in}{1.867386in}}%
\pgfpathlineto{\pgfqpoint{3.557707in}{1.908103in}}%
\pgfpathlineto{\pgfqpoint{3.565439in}{1.933058in}}%
\pgfpathlineto{\pgfqpoint{3.568016in}{1.961578in}}%
\pgfpathlineto{\pgfqpoint{3.570593in}{1.932308in}}%
\pgfpathlineto{\pgfqpoint{3.573170in}{1.961016in}}%
\pgfpathlineto{\pgfqpoint{3.575747in}{1.974713in}}%
\pgfpathlineto{\pgfqpoint{3.583479in}{1.978841in}}%
\pgfpathlineto{\pgfqpoint{3.586056in}{2.002858in}}%
\pgfpathlineto{\pgfqpoint{3.591211in}{2.015429in}}%
\pgfpathlineto{\pgfqpoint{3.593788in}{2.036632in}}%
\pgfpathlineto{\pgfqpoint{3.601520in}{2.052206in}}%
\pgfpathlineto{\pgfqpoint{3.604097in}{2.061587in}}%
\pgfpathlineto{\pgfqpoint{3.606674in}{2.079600in}}%
\pgfpathlineto{\pgfqpoint{3.609251in}{2.064965in}}%
\pgfpathlineto{\pgfqpoint{3.611828in}{2.076785in}}%
\pgfpathlineto{\pgfqpoint{3.619560in}{2.079225in}}%
\pgfpathlineto{\pgfqpoint{3.622137in}{2.067779in}}%
\pgfpathlineto{\pgfqpoint{3.624715in}{2.064402in}}%
\pgfpathlineto{\pgfqpoint{3.629869in}{2.049203in}}%
\pgfpathlineto{\pgfqpoint{3.637601in}{2.039634in}}%
\pgfpathlineto{\pgfqpoint{3.640178in}{2.047515in}}%
\pgfpathlineto{\pgfqpoint{3.642755in}{2.046201in}}%
\pgfpathlineto{\pgfqpoint{3.645332in}{2.048265in}}%
\pgfpathlineto{\pgfqpoint{3.647909in}{2.044325in}}%
\pgfpathlineto{\pgfqpoint{3.655641in}{2.055583in}}%
\pgfpathlineto{\pgfqpoint{3.658218in}{2.061775in}}%
\pgfpathlineto{\pgfqpoint{3.660795in}{2.062713in}}%
\pgfpathlineto{\pgfqpoint{3.665950in}{2.079600in}}%
\pgfpathlineto{\pgfqpoint{3.673682in}{2.074722in}}%
\pgfpathlineto{\pgfqpoint{3.676259in}{2.089732in}}%
\pgfpathlineto{\pgfqpoint{3.678836in}{2.059148in}}%
\pgfpathlineto{\pgfqpoint{3.681413in}{2.069280in}}%
\pgfpathlineto{\pgfqpoint{3.683990in}{2.083540in}}%
\pgfpathlineto{\pgfqpoint{3.691722in}{2.099489in}}%
\pgfpathlineto{\pgfqpoint{3.694299in}{2.095924in}}%
\pgfpathlineto{\pgfqpoint{3.696876in}{2.085417in}}%
\pgfpathlineto{\pgfqpoint{3.699454in}{2.091984in}}%
\pgfpathlineto{\pgfqpoint{3.702031in}{2.054832in}}%
\pgfpathlineto{\pgfqpoint{3.709763in}{2.038133in}}%
\pgfpathlineto{\pgfqpoint{3.712340in}{2.006798in}}%
\pgfpathlineto{\pgfqpoint{3.714917in}{2.045075in}}%
\pgfpathlineto{\pgfqpoint{3.717494in}{2.092734in}}%
\pgfpathlineto{\pgfqpoint{3.720071in}{2.088044in}}%
\pgfpathlineto{\pgfqpoint{3.732957in}{2.108496in}}%
\pgfpathlineto{\pgfqpoint{3.738112in}{2.112248in}}%
\pgfpathlineto{\pgfqpoint{3.748421in}{2.111873in}}%
\pgfpathlineto{\pgfqpoint{3.750998in}{2.090483in}}%
\pgfpathlineto{\pgfqpoint{3.756152in}{2.090107in}}%
\pgfpathlineto{\pgfqpoint{3.763884in}{2.047139in}}%
\pgfpathlineto{\pgfqpoint{3.766461in}{2.013553in}}%
\pgfpathlineto{\pgfqpoint{3.769038in}{2.047327in}}%
\pgfpathlineto{\pgfqpoint{3.771616in}{2.069655in}}%
\pgfpathlineto{\pgfqpoint{3.774193in}{2.049391in}}%
\pgfpathlineto{\pgfqpoint{3.784502in}{2.028564in}}%
\pgfpathlineto{\pgfqpoint{3.787079in}{1.990849in}}%
\pgfpathlineto{\pgfqpoint{3.789656in}{1.970022in}}%
\pgfpathlineto{\pgfqpoint{3.792233in}{1.972836in}}%
\pgfpathlineto{\pgfqpoint{3.802542in}{1.994227in}}%
\pgfpathlineto{\pgfqpoint{3.805119in}{2.001357in}}%
\pgfpathlineto{\pgfqpoint{3.807697in}{1.943753in}}%
\pgfpathlineto{\pgfqpoint{3.810274in}{1.939438in}}%
\pgfpathlineto{\pgfqpoint{3.818005in}{1.924990in}}%
\pgfpathlineto{\pgfqpoint{3.820583in}{1.909229in}}%
\pgfpathlineto{\pgfqpoint{3.823160in}{1.897595in}}%
\pgfpathlineto{\pgfqpoint{3.825737in}{1.902286in}}%
\pgfpathlineto{\pgfqpoint{3.828314in}{1.879395in}}%
\pgfpathlineto{\pgfqpoint{3.836046in}{1.904538in}}%
\pgfpathlineto{\pgfqpoint{3.838623in}{1.932495in}}%
\pgfpathlineto{\pgfqpoint{3.841200in}{1.930431in}}%
\pgfpathlineto{\pgfqpoint{3.843778in}{1.949945in}}%
\pgfpathlineto{\pgfqpoint{3.846355in}{1.954824in}}%
\pgfpathlineto{\pgfqpoint{3.854086in}{1.954448in}}%
\pgfpathlineto{\pgfqpoint{3.856664in}{1.969459in}}%
\pgfpathlineto{\pgfqpoint{3.859241in}{1.972274in}}%
\pgfpathlineto{\pgfqpoint{3.861818in}{1.875642in}}%
\pgfpathlineto{\pgfqpoint{3.864395in}{1.833800in}}%
\pgfpathlineto{\pgfqpoint{3.874704in}{1.851250in}}%
\pgfpathlineto{\pgfqpoint{3.877281in}{1.863446in}}%
\pgfpathlineto{\pgfqpoint{3.879859in}{1.839429in}}%
\pgfpathlineto{\pgfqpoint{3.882436in}{1.864384in}}%
\pgfpathlineto{\pgfqpoint{3.890167in}{1.872640in}}%
\pgfpathlineto{\pgfqpoint{3.892745in}{1.882397in}}%
\pgfpathlineto{\pgfqpoint{3.897899in}{1.924052in}}%
\pgfpathlineto{\pgfqpoint{3.900476in}{1.895156in}}%
\pgfpathlineto{\pgfqpoint{3.908208in}{1.902849in}}%
\pgfpathlineto{\pgfqpoint{3.910785in}{1.900785in}}%
\pgfpathlineto{\pgfqpoint{3.913362in}{1.878269in}}%
\pgfpathlineto{\pgfqpoint{3.915940in}{1.887463in}}%
\pgfpathlineto{\pgfqpoint{3.918517in}{1.872828in}}%
\pgfpathlineto{\pgfqpoint{3.926248in}{1.876205in}}%
\pgfpathlineto{\pgfqpoint{3.928826in}{1.851813in}}%
\pgfpathlineto{\pgfqpoint{3.931403in}{1.857817in}}%
\pgfpathlineto{\pgfqpoint{3.933980in}{1.894593in}}%
\pgfpathlineto{\pgfqpoint{3.936557in}{1.877894in}}%
\pgfpathlineto{\pgfqpoint{3.944289in}{1.893467in}}%
\pgfpathlineto{\pgfqpoint{3.946866in}{1.885775in}}%
\pgfpathlineto{\pgfqpoint{3.949443in}{1.899847in}}%
\pgfpathlineto{\pgfqpoint{3.952020in}{1.894218in}}%
\pgfpathlineto{\pgfqpoint{3.954598in}{1.914482in}}%
\pgfpathlineto{\pgfqpoint{3.962329in}{1.905851in}}%
\pgfpathlineto{\pgfqpoint{3.964907in}{1.891216in}}%
\pgfpathlineto{\pgfqpoint{3.967484in}{1.869263in}}%
\pgfpathlineto{\pgfqpoint{3.970061in}{1.840742in}}%
\pgfpathlineto{\pgfqpoint{3.972638in}{1.831924in}}%
\pgfpathlineto{\pgfqpoint{3.980370in}{1.833237in}}%
\pgfpathlineto{\pgfqpoint{3.982947in}{1.839054in}}%
\pgfpathlineto{\pgfqpoint{3.988101in}{1.866824in}}%
\pgfpathlineto{\pgfqpoint{3.998410in}{1.865510in}}%
\pgfpathlineto{\pgfqpoint{4.000988in}{1.842806in}}%
\pgfpathlineto{\pgfqpoint{4.003565in}{1.848998in}}%
\pgfpathlineto{\pgfqpoint{4.008719in}{1.864760in}}%
\pgfpathlineto{\pgfqpoint{4.016451in}{1.859131in}}%
\pgfpathlineto{\pgfqpoint{4.021605in}{1.867574in}}%
\pgfpathlineto{\pgfqpoint{4.024182in}{1.888026in}}%
\pgfpathlineto{\pgfqpoint{4.026760in}{1.824981in}}%
\pgfpathlineto{\pgfqpoint{4.034491in}{1.823480in}}%
\pgfpathlineto{\pgfqpoint{4.037069in}{1.824418in}}%
\pgfpathlineto{\pgfqpoint{4.039646in}{1.844683in}}%
\pgfpathlineto{\pgfqpoint{4.042223in}{1.839617in}}%
\pgfpathlineto{\pgfqpoint{4.052532in}{1.827983in}}%
\pgfpathlineto{\pgfqpoint{4.055109in}{1.827983in}}%
\pgfpathlineto{\pgfqpoint{4.057686in}{1.822167in}}%
\pgfpathlineto{\pgfqpoint{4.062841in}{1.831361in}}%
\pgfpathlineto{\pgfqpoint{4.070572in}{1.841868in}}%
\pgfpathlineto{\pgfqpoint{4.073149in}{1.834175in}}%
\pgfpathlineto{\pgfqpoint{4.075727in}{1.834363in}}%
\pgfpathlineto{\pgfqpoint{4.080881in}{1.854064in}}%
\pgfpathlineto{\pgfqpoint{4.088613in}{1.866824in}}%
\pgfpathlineto{\pgfqpoint{4.091190in}{1.855941in}}%
\pgfpathlineto{\pgfqpoint{4.096344in}{1.885212in}}%
\pgfpathlineto{\pgfqpoint{4.098922in}{1.875830in}}%
\pgfpathlineto{\pgfqpoint{4.106653in}{1.875079in}}%
\pgfpathlineto{\pgfqpoint{4.109230in}{1.895907in}}%
\pgfpathlineto{\pgfqpoint{4.111808in}{1.888777in}}%
\pgfpathlineto{\pgfqpoint{4.114385in}{1.885399in}}%
\pgfpathlineto{\pgfqpoint{4.116962in}{1.893843in}}%
\pgfpathlineto{\pgfqpoint{4.127271in}{1.875267in}}%
\pgfpathlineto{\pgfqpoint{4.129848in}{1.873954in}}%
\pgfpathlineto{\pgfqpoint{4.132425in}{1.873578in}}%
\pgfpathlineto{\pgfqpoint{4.135003in}{1.867199in}}%
\pgfpathlineto{\pgfqpoint{4.142734in}{1.862696in}}%
\pgfpathlineto{\pgfqpoint{4.147889in}{1.882585in}}%
\pgfpathlineto{\pgfqpoint{4.150466in}{1.860819in}}%
\pgfpathlineto{\pgfqpoint{4.153043in}{1.861382in}}%
\pgfpathlineto{\pgfqpoint{4.160775in}{1.851062in}}%
\pgfpathlineto{\pgfqpoint{4.163352in}{1.857442in}}%
\pgfpathlineto{\pgfqpoint{4.165929in}{1.875267in}}%
\pgfpathlineto{\pgfqpoint{4.168506in}{1.877331in}}%
\pgfpathlineto{\pgfqpoint{4.171084in}{1.863821in}}%
\pgfpathlineto{\pgfqpoint{4.178815in}{1.858943in}}%
\pgfpathlineto{\pgfqpoint{4.181392in}{1.860819in}}%
\pgfpathlineto{\pgfqpoint{4.183970in}{1.877706in}}%
\pgfpathlineto{\pgfqpoint{4.186547in}{1.885587in}}%
\pgfpathlineto{\pgfqpoint{4.189124in}{1.875267in}}%
\pgfpathlineto{\pgfqpoint{4.196856in}{1.893843in}}%
\pgfpathlineto{\pgfqpoint{4.199433in}{1.895907in}}%
\pgfpathlineto{\pgfqpoint{4.202010in}{1.885024in}}%
\pgfpathlineto{\pgfqpoint{4.204587in}{1.869826in}}%
\pgfpathlineto{\pgfqpoint{4.207165in}{1.869826in}}%
\pgfpathlineto{\pgfqpoint{4.214896in}{1.833425in}}%
\pgfpathlineto{\pgfqpoint{4.217473in}{1.837177in}}%
\pgfpathlineto{\pgfqpoint{4.220051in}{1.849186in}}%
\pgfpathlineto{\pgfqpoint{4.222628in}{1.845809in}}%
\pgfpathlineto{\pgfqpoint{4.232937in}{1.834926in}}%
\pgfpathlineto{\pgfqpoint{4.235514in}{1.834175in}}%
\pgfpathlineto{\pgfqpoint{4.238091in}{1.805655in}}%
\pgfpathlineto{\pgfqpoint{4.240668in}{1.812785in}}%
\pgfpathlineto{\pgfqpoint{4.243246in}{1.829860in}}%
\pgfpathlineto{\pgfqpoint{4.253554in}{1.859131in}}%
\pgfpathlineto{\pgfqpoint{4.256132in}{1.851625in}}%
\pgfpathlineto{\pgfqpoint{4.261286in}{1.863446in}}%
\pgfpathlineto{\pgfqpoint{4.269018in}{1.864947in}}%
\pgfpathlineto{\pgfqpoint{4.271595in}{1.858755in}}%
\pgfpathlineto{\pgfqpoint{4.274172in}{1.859506in}}%
\pgfpathlineto{\pgfqpoint{4.276749in}{1.824606in}}%
\pgfpathlineto{\pgfqpoint{4.279326in}{1.805092in}}%
\pgfpathlineto{\pgfqpoint{4.287058in}{1.787830in}}%
\pgfpathlineto{\pgfqpoint{4.289635in}{1.791207in}}%
\pgfpathlineto{\pgfqpoint{4.292213in}{1.801527in}}%
\pgfpathlineto{\pgfqpoint{4.294790in}{1.809032in}}%
\pgfpathlineto{\pgfqpoint{4.297367in}{1.807907in}}%
\pgfpathlineto{\pgfqpoint{4.305099in}{1.806593in}}%
\pgfpathlineto{\pgfqpoint{4.307676in}{1.801902in}}%
\pgfpathlineto{\pgfqpoint{4.310253in}{1.799651in}}%
\pgfpathlineto{\pgfqpoint{4.312830in}{1.789331in}}%
\pgfpathlineto{\pgfqpoint{4.315407in}{1.872265in}}%
\pgfpathlineto{\pgfqpoint{4.323139in}{1.899472in}}%
\pgfpathlineto{\pgfqpoint{4.325716in}{1.900785in}}%
\pgfpathlineto{\pgfqpoint{4.328294in}{1.894781in}}%
\pgfpathlineto{\pgfqpoint{4.330871in}{1.890653in}}%
\pgfpathlineto{\pgfqpoint{4.333448in}{1.893280in}}%
\pgfpathlineto{\pgfqpoint{4.341180in}{1.894781in}}%
\pgfpathlineto{\pgfqpoint{4.343757in}{1.899096in}}%
\pgfpathlineto{\pgfqpoint{4.346334in}{1.894030in}}%
\pgfpathlineto{\pgfqpoint{4.351488in}{1.824981in}}%
\pgfpathlineto{\pgfqpoint{4.359220in}{1.783139in}}%
\pgfpathlineto{\pgfqpoint{4.361797in}{1.763250in}}%
\pgfpathlineto{\pgfqpoint{4.364374in}{1.800401in}}%
\pgfpathlineto{\pgfqpoint{4.366952in}{1.822354in}}%
\pgfpathlineto{\pgfqpoint{4.369529in}{1.818226in}}%
\pgfpathlineto{\pgfqpoint{4.377261in}{1.819540in}}%
\pgfpathlineto{\pgfqpoint{4.379838in}{1.771693in}}%
\pgfpathlineto{\pgfqpoint{4.382415in}{1.788205in}}%
\pgfpathlineto{\pgfqpoint{4.384992in}{1.793834in}}%
\pgfpathlineto{\pgfqpoint{4.387569in}{1.773007in}}%
\pgfpathlineto{\pgfqpoint{4.397878in}{1.797962in}}%
\pgfpathlineto{\pgfqpoint{4.400455in}{1.791395in}}%
\pgfpathlineto{\pgfqpoint{4.405610in}{1.798150in}}%
\pgfpathlineto{\pgfqpoint{4.413342in}{1.791770in}}%
\pgfpathlineto{\pgfqpoint{4.415919in}{1.815600in}}%
\pgfpathlineto{\pgfqpoint{4.418496in}{1.830047in}}%
\pgfpathlineto{\pgfqpoint{4.421073in}{1.825544in}}%
\pgfpathlineto{\pgfqpoint{4.423650in}{1.806030in}}%
\pgfpathlineto{\pgfqpoint{4.431382in}{1.819915in}}%
\pgfpathlineto{\pgfqpoint{4.433959in}{1.801715in}}%
\pgfpathlineto{\pgfqpoint{4.436536in}{1.800401in}}%
\pgfpathlineto{\pgfqpoint{4.439114in}{1.783889in}}%
\pgfpathlineto{\pgfqpoint{4.441691in}{1.791020in}}%
\pgfpathlineto{\pgfqpoint{4.449422in}{1.760060in}}%
\pgfpathlineto{\pgfqpoint{4.452000in}{1.755932in}}%
\pgfpathlineto{\pgfqpoint{4.454577in}{1.773945in}}%
\pgfpathlineto{\pgfqpoint{4.457154in}{1.769817in}}%
\pgfpathlineto{\pgfqpoint{4.459731in}{1.778823in}}%
\pgfpathlineto{\pgfqpoint{4.467463in}{1.829672in}}%
\pgfpathlineto{\pgfqpoint{4.470040in}{1.822917in}}%
\pgfpathlineto{\pgfqpoint{4.472617in}{1.832862in}}%
\pgfpathlineto{\pgfqpoint{4.475195in}{1.832862in}}%
\pgfpathlineto{\pgfqpoint{4.477772in}{1.835489in}}%
\pgfpathlineto{\pgfqpoint{4.485503in}{1.835113in}}%
\pgfpathlineto{\pgfqpoint{4.488081in}{1.822542in}}%
\pgfpathlineto{\pgfqpoint{4.490658in}{1.814849in}}%
\pgfpathlineto{\pgfqpoint{4.495812in}{1.833237in}}%
\pgfpathlineto{\pgfqpoint{4.506121in}{1.828734in}}%
\pgfpathlineto{\pgfqpoint{4.508698in}{1.821041in}}%
\pgfpathlineto{\pgfqpoint{4.511276in}{1.750115in}}%
\pgfpathlineto{\pgfqpoint{4.513853in}{1.787079in}}%
\pgfpathlineto{\pgfqpoint{4.521584in}{1.779949in}}%
\pgfpathlineto{\pgfqpoint{4.524162in}{1.776759in}}%
\pgfpathlineto{\pgfqpoint{4.526739in}{1.784640in}}%
\pgfpathlineto{\pgfqpoint{4.529316in}{1.780512in}}%
\pgfpathlineto{\pgfqpoint{4.531893in}{1.763625in}}%
\pgfpathlineto{\pgfqpoint{4.539625in}{1.775634in}}%
\pgfpathlineto{\pgfqpoint{4.542202in}{1.777135in}}%
\pgfpathlineto{\pgfqpoint{4.544779in}{1.777885in}}%
\pgfpathlineto{\pgfqpoint{4.547357in}{1.775446in}}%
\pgfpathlineto{\pgfqpoint{4.549934in}{1.781826in}}%
\pgfpathlineto{\pgfqpoint{4.557665in}{1.766440in}}%
\pgfpathlineto{\pgfqpoint{4.560243in}{1.765501in}}%
\pgfpathlineto{\pgfqpoint{4.562820in}{1.757433in}}%
\pgfpathlineto{\pgfqpoint{4.567974in}{1.727224in}}%
\pgfpathlineto{\pgfqpoint{4.575706in}{1.735292in}}%
\pgfpathlineto{\pgfqpoint{4.578283in}{1.725535in}}%
\pgfpathlineto{\pgfqpoint{4.580860in}{1.743924in}}%
\pgfpathlineto{\pgfqpoint{4.583438in}{1.754431in}}%
\pgfpathlineto{\pgfqpoint{4.586015in}{1.748614in}}%
\pgfpathlineto{\pgfqpoint{4.593746in}{1.745612in}}%
\pgfpathlineto{\pgfqpoint{4.596324in}{1.734729in}}%
\pgfpathlineto{\pgfqpoint{4.604055in}{1.738670in}}%
\pgfpathlineto{\pgfqpoint{4.611787in}{1.734917in}}%
\pgfpathlineto{\pgfqpoint{4.614364in}{1.744862in}}%
\pgfpathlineto{\pgfqpoint{4.619519in}{1.713339in}}%
\pgfpathlineto{\pgfqpoint{4.622096in}{1.725348in}}%
\pgfpathlineto{\pgfqpoint{4.629827in}{1.716717in}}%
\pgfpathlineto{\pgfqpoint{4.632405in}{1.704333in}}%
\pgfpathlineto{\pgfqpoint{4.634982in}{1.703395in}}%
\pgfpathlineto{\pgfqpoint{4.637559in}{1.707898in}}%
\pgfpathlineto{\pgfqpoint{4.640136in}{1.685757in}}%
\pgfpathlineto{\pgfqpoint{4.647868in}{1.685382in}}%
\pgfpathlineto{\pgfqpoint{4.650445in}{1.708648in}}%
\pgfpathlineto{\pgfqpoint{4.653022in}{1.718593in}}%
\pgfpathlineto{\pgfqpoint{4.655599in}{1.698329in}}%
\pgfpathlineto{\pgfqpoint{4.658177in}{1.668307in}}%
\pgfpathlineto{\pgfqpoint{4.665908in}{1.677689in}}%
\pgfpathlineto{\pgfqpoint{4.668486in}{1.685569in}}%
\pgfpathlineto{\pgfqpoint{4.671063in}{1.705459in}}%
\pgfpathlineto{\pgfqpoint{4.673640in}{1.708836in}}%
\pgfpathlineto{\pgfqpoint{4.683949in}{1.701706in}}%
\pgfpathlineto{\pgfqpoint{4.686526in}{1.715591in}}%
\pgfpathlineto{\pgfqpoint{4.689103in}{1.708836in}}%
\pgfpathlineto{\pgfqpoint{4.691680in}{1.697953in}}%
\pgfpathlineto{\pgfqpoint{4.701989in}{1.663429in}}%
\pgfpathlineto{\pgfqpoint{4.704567in}{1.645041in}}%
\pgfpathlineto{\pgfqpoint{4.707144in}{1.612392in}}%
\pgfpathlineto{\pgfqpoint{4.709721in}{1.602073in}}%
\pgfpathlineto{\pgfqpoint{4.712298in}{1.598320in}}%
\pgfpathlineto{\pgfqpoint{4.720030in}{1.605825in}}%
\pgfpathlineto{\pgfqpoint{4.722607in}{1.612017in}}%
\pgfpathlineto{\pgfqpoint{4.725184in}{1.584435in}}%
\pgfpathlineto{\pgfqpoint{4.727761in}{1.592316in}}%
\pgfpathlineto{\pgfqpoint{4.730339in}{1.585561in}}%
\pgfpathlineto{\pgfqpoint{4.740648in}{1.580870in}}%
\pgfpathlineto{\pgfqpoint{4.743225in}{1.587625in}}%
\pgfpathlineto{\pgfqpoint{4.745802in}{1.580870in}}%
\pgfpathlineto{\pgfqpoint{4.748379in}{1.446337in}}%
\pgfpathlineto{\pgfqpoint{4.756111in}{1.445586in}}%
\pgfpathlineto{\pgfqpoint{4.758688in}{1.446900in}}%
\pgfpathlineto{\pgfqpoint{4.761265in}{1.436767in}}%
\pgfpathlineto{\pgfqpoint{4.763842in}{1.407684in}}%
\pgfpathlineto{\pgfqpoint{4.766420in}{1.418567in}}%
\pgfpathlineto{\pgfqpoint{4.774151in}{1.439957in}}%
\pgfpathlineto{\pgfqpoint{4.776728in}{1.421569in}}%
\pgfpathlineto{\pgfqpoint{4.779306in}{1.429450in}}%
\pgfpathlineto{\pgfqpoint{4.781883in}{1.434328in}}%
\pgfpathlineto{\pgfqpoint{4.784460in}{1.427198in}}%
\pgfpathlineto{\pgfqpoint{4.792192in}{1.399053in}}%
\pgfpathlineto{\pgfqpoint{4.794769in}{1.403181in}}%
\pgfpathlineto{\pgfqpoint{4.797346in}{1.397177in}}%
\pgfpathlineto{\pgfqpoint{4.799923in}{1.376162in}}%
\pgfpathlineto{\pgfqpoint{4.802501in}{1.403744in}}%
\pgfpathlineto{\pgfqpoint{4.812809in}{1.412938in}}%
\pgfpathlineto{\pgfqpoint{4.815387in}{1.420631in}}%
\pgfpathlineto{\pgfqpoint{4.820541in}{1.440145in}}%
\pgfpathlineto{\pgfqpoint{4.828273in}{1.456469in}}%
\pgfpathlineto{\pgfqpoint{4.833427in}{1.438831in}}%
\pgfpathlineto{\pgfqpoint{4.836004in}{1.452153in}}%
\pgfpathlineto{\pgfqpoint{4.838582in}{1.451966in}}%
\pgfpathlineto{\pgfqpoint{4.846313in}{1.455531in}}%
\pgfpathlineto{\pgfqpoint{4.848890in}{1.477109in}}%
\pgfpathlineto{\pgfqpoint{4.851468in}{1.482925in}}%
\pgfpathlineto{\pgfqpoint{4.854045in}{1.500000in}}%
\pgfpathlineto{\pgfqpoint{4.856622in}{1.503565in}}%
\pgfpathlineto{\pgfqpoint{4.864354in}{1.516136in}}%
\pgfpathlineto{\pgfqpoint{4.866931in}{1.523829in}}%
\pgfpathlineto{\pgfqpoint{4.872085in}{1.511821in}}%
\pgfpathlineto{\pgfqpoint{4.874663in}{1.524392in}}%
\pgfpathlineto{\pgfqpoint{4.882394in}{1.526269in}}%
\pgfpathlineto{\pgfqpoint{4.884971in}{1.520264in}}%
\pgfpathlineto{\pgfqpoint{4.890126in}{1.535463in}}%
\pgfpathlineto{\pgfqpoint{4.892703in}{1.555539in}}%
\pgfpathlineto{\pgfqpoint{4.900435in}{1.555352in}}%
\pgfpathlineto{\pgfqpoint{4.903012in}{1.544844in}}%
\pgfpathlineto{\pgfqpoint{4.905589in}{1.545220in}}%
\pgfpathlineto{\pgfqpoint{4.908166in}{1.542217in}}%
\pgfpathlineto{\pgfqpoint{4.918475in}{1.538840in}}%
\pgfpathlineto{\pgfqpoint{4.921052in}{1.544657in}}%
\pgfpathlineto{\pgfqpoint{4.923630in}{1.539028in}}%
\pgfpathlineto{\pgfqpoint{4.926207in}{1.558729in}}%
\pgfpathlineto{\pgfqpoint{4.928784in}{1.553475in}}%
\pgfpathlineto{\pgfqpoint{4.936516in}{1.545032in}}%
\pgfpathlineto{\pgfqpoint{4.939093in}{1.537339in}}%
\pgfpathlineto{\pgfqpoint{4.941670in}{1.536964in}}%
\pgfpathlineto{\pgfqpoint{4.944247in}{1.518763in}}%
\pgfpathlineto{\pgfqpoint{4.946825in}{1.530209in}}%
\pgfpathlineto{\pgfqpoint{4.954556in}{1.535463in}}%
\pgfpathlineto{\pgfqpoint{4.957133in}{1.551224in}}%
\pgfpathlineto{\pgfqpoint{4.959711in}{1.577680in}}%
\pgfpathlineto{\pgfqpoint{4.962288in}{1.583872in}}%
\pgfpathlineto{\pgfqpoint{4.964865in}{1.577305in}}%
\pgfpathlineto{\pgfqpoint{4.972597in}{1.585373in}}%
\pgfpathlineto{\pgfqpoint{4.975174in}{1.602823in}}%
\pgfpathlineto{\pgfqpoint{4.977751in}{1.628529in}}%
\pgfpathlineto{\pgfqpoint{4.980328in}{1.639036in}}%
\pgfpathlineto{\pgfqpoint{4.982905in}{1.644853in}}%
\pgfpathlineto{\pgfqpoint{4.990637in}{1.640725in}}%
\pgfpathlineto{\pgfqpoint{4.993214in}{1.649731in}}%
\pgfpathlineto{\pgfqpoint{4.995792in}{1.650107in}}%
\pgfpathlineto{\pgfqpoint{5.000946in}{1.635847in}}%
\pgfpathlineto{\pgfqpoint{5.008678in}{1.640350in}}%
\pgfpathlineto{\pgfqpoint{5.011255in}{1.625527in}}%
\pgfpathlineto{\pgfqpoint{5.013832in}{1.615019in}}%
\pgfpathlineto{\pgfqpoint{5.016409in}{1.609015in}}%
\pgfpathlineto{\pgfqpoint{5.018986in}{1.619522in}}%
\pgfpathlineto{\pgfqpoint{5.026718in}{1.610141in}}%
\pgfpathlineto{\pgfqpoint{5.029295in}{1.625339in}}%
\pgfpathlineto{\pgfqpoint{5.037027in}{1.612580in}}%
\pgfpathlineto{\pgfqpoint{5.044759in}{1.611642in}}%
\pgfpathlineto{\pgfqpoint{5.047336in}{1.588375in}}%
\pgfpathlineto{\pgfqpoint{5.049913in}{1.601885in}}%
\pgfpathlineto{\pgfqpoint{5.052490in}{1.588000in}}%
\pgfpathlineto{\pgfqpoint{5.055067in}{1.609015in}}%
\pgfpathlineto{\pgfqpoint{5.062799in}{1.603198in}}%
\pgfpathlineto{\pgfqpoint{5.065376in}{1.625902in}}%
\pgfpathlineto{\pgfqpoint{5.067953in}{1.633783in}}%
\pgfpathlineto{\pgfqpoint{5.070531in}{1.632282in}}%
\pgfpathlineto{\pgfqpoint{5.073108in}{1.637535in}}%
\pgfpathlineto{\pgfqpoint{5.083417in}{1.641851in}}%
\pgfpathlineto{\pgfqpoint{5.085994in}{1.644478in}}%
\pgfpathlineto{\pgfqpoint{5.088571in}{1.653297in}}%
\pgfpathlineto{\pgfqpoint{5.091148in}{1.636972in}}%
\pgfpathlineto{\pgfqpoint{5.098880in}{1.645041in}}%
\pgfpathlineto{\pgfqpoint{5.101457in}{1.644102in}}%
\pgfpathlineto{\pgfqpoint{5.104034in}{1.650670in}}%
\pgfpathlineto{\pgfqpoint{5.106612in}{1.640913in}}%
\pgfpathlineto{\pgfqpoint{5.109189in}{1.627778in}}%
\pgfpathlineto{\pgfqpoint{5.116921in}{1.604512in}}%
\pgfpathlineto{\pgfqpoint{5.119498in}{1.558166in}}%
\pgfpathlineto{\pgfqpoint{5.122075in}{1.564358in}}%
\pgfpathlineto{\pgfqpoint{5.124652in}{1.573552in}}%
\pgfpathlineto{\pgfqpoint{5.127229in}{1.572239in}}%
\pgfpathlineto{\pgfqpoint{5.134961in}{1.580119in}}%
\pgfpathlineto{\pgfqpoint{5.137538in}{1.579932in}}%
\pgfpathlineto{\pgfqpoint{5.140115in}{1.573928in}}%
\pgfpathlineto{\pgfqpoint{5.142693in}{1.597006in}}%
\pgfpathlineto{\pgfqpoint{5.145270in}{1.540153in}}%
\pgfpathlineto{\pgfqpoint{5.153001in}{1.497561in}}%
\pgfpathlineto{\pgfqpoint{5.155579in}{1.501689in}}%
\pgfpathlineto{\pgfqpoint{5.158156in}{1.537902in}}%
\pgfpathlineto{\pgfqpoint{5.160733in}{1.558166in}}%
\pgfpathlineto{\pgfqpoint{5.163310in}{1.556853in}}%
\pgfpathlineto{\pgfqpoint{5.173619in}{1.529271in}}%
\pgfpathlineto{\pgfqpoint{5.178774in}{1.541092in}}%
\pgfpathlineto{\pgfqpoint{5.181351in}{1.571113in}}%
\pgfpathlineto{\pgfqpoint{5.189082in}{1.583122in}}%
\pgfpathlineto{\pgfqpoint{5.191660in}{1.598320in}}%
\pgfpathlineto{\pgfqpoint{5.194237in}{1.600009in}}%
\pgfpathlineto{\pgfqpoint{5.196814in}{1.609203in}}%
\pgfpathlineto{\pgfqpoint{5.199391in}{1.612205in}}%
\pgfpathlineto{\pgfqpoint{5.207123in}{1.615957in}}%
\pgfpathlineto{\pgfqpoint{5.209700in}{1.618960in}}%
\pgfpathlineto{\pgfqpoint{5.212277in}{1.624776in}}%
\pgfpathlineto{\pgfqpoint{5.214855in}{1.606013in}}%
\pgfpathlineto{\pgfqpoint{5.217432in}{1.621211in}}%
\pgfpathlineto{\pgfqpoint{5.227741in}{1.622712in}}%
\pgfpathlineto{\pgfqpoint{5.232895in}{1.630405in}}%
\pgfpathlineto{\pgfqpoint{5.235472in}{1.624401in}}%
\pgfpathlineto{\pgfqpoint{5.243204in}{1.618584in}}%
\pgfpathlineto{\pgfqpoint{5.245781in}{1.606763in}}%
\pgfpathlineto{\pgfqpoint{5.248358in}{1.613143in}}%
\pgfpathlineto{\pgfqpoint{5.250936in}{1.615019in}}%
\pgfpathlineto{\pgfqpoint{5.253513in}{1.643352in}}%
\pgfpathlineto{\pgfqpoint{5.261244in}{1.648981in}}%
\pgfpathlineto{\pgfqpoint{5.263822in}{1.641100in}}%
\pgfpathlineto{\pgfqpoint{5.266399in}{1.629467in}}%
\pgfpathlineto{\pgfqpoint{5.268976in}{1.642601in}}%
\pgfpathlineto{\pgfqpoint{5.271553in}{1.640913in}}%
\pgfpathlineto{\pgfqpoint{5.279285in}{1.645416in}}%
\pgfpathlineto{\pgfqpoint{5.281862in}{1.639412in}}%
\pgfpathlineto{\pgfqpoint{5.284439in}{1.646354in}}%
\pgfpathlineto{\pgfqpoint{5.287017in}{1.645416in}}%
\pgfpathlineto{\pgfqpoint{5.289594in}{1.643540in}}%
\pgfpathlineto{\pgfqpoint{5.297325in}{1.640537in}}%
\pgfpathlineto{\pgfqpoint{5.299903in}{1.646354in}}%
\pgfpathlineto{\pgfqpoint{5.302480in}{1.636972in}}%
\pgfpathlineto{\pgfqpoint{5.305057in}{1.634158in}}%
\pgfpathlineto{\pgfqpoint{5.307634in}{1.630405in}}%
\pgfpathlineto{\pgfqpoint{5.315366in}{1.643352in}}%
\pgfpathlineto{\pgfqpoint{5.317943in}{1.642226in}}%
\pgfpathlineto{\pgfqpoint{5.320520in}{1.644478in}}%
\pgfpathlineto{\pgfqpoint{5.323098in}{1.631531in}}%
\pgfpathlineto{\pgfqpoint{5.325675in}{1.637535in}}%
\pgfpathlineto{\pgfqpoint{5.335984in}{1.647668in}}%
\pgfpathlineto{\pgfqpoint{5.338561in}{1.655360in}}%
\pgfpathlineto{\pgfqpoint{5.341138in}{1.656299in}}%
\pgfpathlineto{\pgfqpoint{5.343715in}{1.636222in}}%
\pgfpathlineto{\pgfqpoint{5.351447in}{1.650294in}}%
\pgfpathlineto{\pgfqpoint{5.354024in}{1.622525in}}%
\pgfpathlineto{\pgfqpoint{5.356601in}{1.606763in}}%
\pgfpathlineto{\pgfqpoint{5.359178in}{1.613143in}}%
\pgfpathlineto{\pgfqpoint{5.361756in}{1.610141in}}%
\pgfpathlineto{\pgfqpoint{5.369487in}{1.617271in}}%
\pgfpathlineto{\pgfqpoint{5.372065in}{1.611079in}}%
\pgfpathlineto{\pgfqpoint{5.374642in}{1.621211in}}%
\pgfpathlineto{\pgfqpoint{5.377219in}{1.627403in}}%
\pgfpathlineto{\pgfqpoint{5.379796in}{1.613518in}}%
\pgfpathlineto{\pgfqpoint{5.387528in}{1.605825in}}%
\pgfpathlineto{\pgfqpoint{5.390105in}{1.621211in}}%
\pgfpathlineto{\pgfqpoint{5.392682in}{1.620085in}}%
\pgfpathlineto{\pgfqpoint{5.395259in}{1.604887in}}%
\pgfpathlineto{\pgfqpoint{5.397837in}{1.616896in}}%
\pgfpathlineto{\pgfqpoint{5.405568in}{1.612768in}}%
\pgfpathlineto{\pgfqpoint{5.408146in}{1.614644in}}%
\pgfpathlineto{\pgfqpoint{5.410723in}{1.628341in}}%
\pgfpathlineto{\pgfqpoint{5.413300in}{1.584810in}}%
\pgfpathlineto{\pgfqpoint{5.415877in}{1.581620in}}%
\pgfpathlineto{\pgfqpoint{5.423609in}{1.584060in}}%
\pgfpathlineto{\pgfqpoint{5.426186in}{1.565484in}}%
\pgfpathlineto{\pgfqpoint{5.428763in}{1.562294in}}%
\pgfpathlineto{\pgfqpoint{5.433918in}{1.552537in}}%
\pgfpathlineto{\pgfqpoint{5.441649in}{1.548034in}}%
\pgfpathlineto{\pgfqpoint{5.444227in}{1.551411in}}%
\pgfpathlineto{\pgfqpoint{5.446804in}{1.572426in}}%
\pgfpathlineto{\pgfqpoint{5.449381in}{1.672060in}}%
\pgfpathlineto{\pgfqpoint{5.451958in}{1.682380in}}%
\pgfpathlineto{\pgfqpoint{5.459690in}{1.677501in}}%
\pgfpathlineto{\pgfqpoint{5.462267in}{1.671309in}}%
\pgfpathlineto{\pgfqpoint{5.464844in}{1.672435in}}%
\pgfpathlineto{\pgfqpoint{5.467421in}{1.674687in}}%
\pgfpathlineto{\pgfqpoint{5.469999in}{1.666056in}}%
\pgfpathlineto{\pgfqpoint{5.477730in}{1.665493in}}%
\pgfpathlineto{\pgfqpoint{5.480307in}{1.662491in}}%
\pgfpathlineto{\pgfqpoint{5.482885in}{1.648043in}}%
\pgfpathlineto{\pgfqpoint{5.485462in}{1.645979in}}%
\pgfpathlineto{\pgfqpoint{5.488039in}{1.649169in}}%
\pgfpathlineto{\pgfqpoint{5.495771in}{1.676000in}}%
\pgfpathlineto{\pgfqpoint{5.498348in}{1.677126in}}%
\pgfpathlineto{\pgfqpoint{5.503502in}{1.731915in}}%
\pgfpathlineto{\pgfqpoint{5.506080in}{1.739045in}}%
\pgfpathlineto{\pgfqpoint{5.513811in}{1.773570in}}%
\pgfpathlineto{\pgfqpoint{5.516388in}{1.774508in}}%
\pgfpathlineto{\pgfqpoint{5.518966in}{1.760248in}}%
\pgfpathlineto{\pgfqpoint{5.521543in}{1.761936in}}%
\pgfpathlineto{\pgfqpoint{5.524120in}{1.748051in}}%
\pgfpathlineto{\pgfqpoint{5.534429in}{1.760998in}}%
\pgfpathlineto{\pgfqpoint{5.537006in}{1.781826in}}%
\pgfpathlineto{\pgfqpoint{5.542161in}{1.781450in}}%
\pgfpathlineto{\pgfqpoint{5.549892in}{1.768316in}}%
\pgfpathlineto{\pgfqpoint{5.552469in}{1.756870in}}%
\pgfpathlineto{\pgfqpoint{5.557624in}{1.775446in}}%
\pgfpathlineto{\pgfqpoint{5.560201in}{1.763437in}}%
\pgfpathlineto{\pgfqpoint{5.567933in}{1.766440in}}%
\pgfpathlineto{\pgfqpoint{5.570510in}{1.771506in}}%
\pgfpathlineto{\pgfqpoint{5.573087in}{1.807156in}}%
\pgfpathlineto{\pgfqpoint{5.575664in}{1.818414in}}%
\pgfpathlineto{\pgfqpoint{5.578242in}{1.815787in}}%
\pgfpathlineto{\pgfqpoint{5.585973in}{1.794397in}}%
\pgfpathlineto{\pgfqpoint{5.591128in}{1.803216in}}%
\pgfpathlineto{\pgfqpoint{5.593705in}{1.818977in}}%
\pgfpathlineto{\pgfqpoint{5.596282in}{1.819915in}}%
\pgfpathlineto{\pgfqpoint{5.604014in}{1.811847in}}%
\pgfpathlineto{\pgfqpoint{5.606591in}{1.821041in}}%
\pgfpathlineto{\pgfqpoint{5.609168in}{1.825732in}}%
\pgfpathlineto{\pgfqpoint{5.611745in}{1.812410in}}%
\pgfpathlineto{\pgfqpoint{5.614323in}{1.819540in}}%
\pgfpathlineto{\pgfqpoint{5.624631in}{1.819727in}}%
\pgfpathlineto{\pgfqpoint{5.627209in}{1.808657in}}%
\pgfpathlineto{\pgfqpoint{5.629786in}{1.800589in}}%
\pgfpathlineto{\pgfqpoint{5.632363in}{1.803403in}}%
\pgfpathlineto{\pgfqpoint{5.642672in}{1.826295in}}%
\pgfpathlineto{\pgfqpoint{5.645249in}{1.848623in}}%
\pgfpathlineto{\pgfqpoint{5.647826in}{1.831548in}}%
\pgfpathlineto{\pgfqpoint{5.658135in}{1.841305in}}%
\pgfpathlineto{\pgfqpoint{5.660712in}{1.855566in}}%
\pgfpathlineto{\pgfqpoint{5.663290in}{1.860256in}}%
\pgfpathlineto{\pgfqpoint{5.665867in}{1.859693in}}%
\pgfpathlineto{\pgfqpoint{5.668444in}{1.855003in}}%
\pgfpathlineto{\pgfqpoint{5.678753in}{1.854627in}}%
\pgfpathlineto{\pgfqpoint{5.681330in}{1.870764in}}%
\pgfpathlineto{\pgfqpoint{5.683907in}{1.856316in}}%
\pgfpathlineto{\pgfqpoint{5.686484in}{1.847497in}}%
\pgfpathlineto{\pgfqpoint{5.694216in}{1.843369in}}%
\pgfpathlineto{\pgfqpoint{5.696793in}{1.869638in}}%
\pgfpathlineto{\pgfqpoint{5.699371in}{1.859881in}}%
\pgfpathlineto{\pgfqpoint{5.701948in}{1.860632in}}%
\pgfpathlineto{\pgfqpoint{5.704525in}{1.859131in}}%
\pgfpathlineto{\pgfqpoint{5.712257in}{1.867011in}}%
\pgfpathlineto{\pgfqpoint{5.714834in}{1.850687in}}%
\pgfpathlineto{\pgfqpoint{5.717411in}{1.857629in}}%
\pgfpathlineto{\pgfqpoint{5.719988in}{1.853126in}}%
\pgfpathlineto{\pgfqpoint{5.722565in}{1.880708in}}%
\pgfpathlineto{\pgfqpoint{5.732874in}{1.874892in}}%
\pgfpathlineto{\pgfqpoint{5.735452in}{1.876393in}}%
\pgfpathlineto{\pgfqpoint{5.740606in}{1.888777in}}%
\pgfpathlineto{\pgfqpoint{5.748338in}{1.896470in}}%
\pgfpathlineto{\pgfqpoint{5.750915in}{1.905476in}}%
\pgfpathlineto{\pgfqpoint{5.753492in}{1.909041in}}%
\pgfpathlineto{\pgfqpoint{5.756069in}{1.907352in}}%
\pgfpathlineto{\pgfqpoint{5.758646in}{1.910917in}}%
\pgfpathlineto{\pgfqpoint{5.768955in}{1.915608in}}%
\pgfpathlineto{\pgfqpoint{5.771532in}{1.913732in}}%
\pgfpathlineto{\pgfqpoint{5.774110in}{1.917109in}}%
\pgfpathlineto{\pgfqpoint{5.776687in}{1.911856in}}%
\pgfpathlineto{\pgfqpoint{5.784419in}{1.919173in}}%
\pgfpathlineto{\pgfqpoint{5.786996in}{1.917297in}}%
\pgfpathlineto{\pgfqpoint{5.789573in}{1.950883in}}%
\pgfpathlineto{\pgfqpoint{5.792150in}{1.918047in}}%
\pgfpathlineto{\pgfqpoint{5.794727in}{1.914107in}}%
\pgfpathlineto{\pgfqpoint{5.802459in}{1.907165in}}%
\pgfpathlineto{\pgfqpoint{5.805036in}{1.908666in}}%
\pgfpathlineto{\pgfqpoint{5.807613in}{1.898909in}}%
\pgfpathlineto{\pgfqpoint{5.810191in}{1.903600in}}%
\pgfpathlineto{\pgfqpoint{5.812768in}{1.904913in}}%
\pgfpathlineto{\pgfqpoint{5.820500in}{1.901911in}}%
\pgfpathlineto{\pgfqpoint{5.823077in}{1.910542in}}%
\pgfpathlineto{\pgfqpoint{5.825654in}{1.902286in}}%
\pgfpathlineto{\pgfqpoint{5.828231in}{1.912043in}}%
\pgfpathlineto{\pgfqpoint{5.830808in}{1.902662in}}%
\pgfpathlineto{\pgfqpoint{5.838540in}{1.895344in}}%
\pgfpathlineto{\pgfqpoint{5.841117in}{1.871514in}}%
\pgfpathlineto{\pgfqpoint{5.846272in}{1.877143in}}%
\pgfpathlineto{\pgfqpoint{5.848849in}{1.883711in}}%
\pgfpathlineto{\pgfqpoint{5.856580in}{1.872828in}}%
\pgfpathlineto{\pgfqpoint{5.859158in}{1.891591in}}%
\pgfpathlineto{\pgfqpoint{5.861735in}{1.884461in}}%
\pgfpathlineto{\pgfqpoint{5.864312in}{1.901911in}}%
\pgfpathlineto{\pgfqpoint{5.866889in}{1.900035in}}%
\pgfpathlineto{\pgfqpoint{5.874621in}{1.890653in}}%
\pgfpathlineto{\pgfqpoint{5.877198in}{1.884649in}}%
\pgfpathlineto{\pgfqpoint{5.879775in}{1.881459in}}%
\pgfpathlineto{\pgfqpoint{5.882353in}{1.884273in}}%
\pgfpathlineto{\pgfqpoint{5.884930in}{1.881647in}}%
\pgfpathlineto{\pgfqpoint{5.892661in}{1.876580in}}%
\pgfpathlineto{\pgfqpoint{5.895239in}{1.872453in}}%
\pgfpathlineto{\pgfqpoint{5.897816in}{1.861757in}}%
\pgfpathlineto{\pgfqpoint{5.900393in}{1.845809in}}%
\pgfpathlineto{\pgfqpoint{5.910702in}{1.861570in}}%
\pgfpathlineto{\pgfqpoint{5.913279in}{1.845621in}}%
\pgfpathlineto{\pgfqpoint{5.915856in}{1.841305in}}%
\pgfpathlineto{\pgfqpoint{5.918434in}{1.922551in}}%
\pgfpathlineto{\pgfqpoint{5.921011in}{1.914670in}}%
\pgfpathlineto{\pgfqpoint{5.928742in}{1.930244in}}%
\pgfpathlineto{\pgfqpoint{5.931320in}{1.933621in}}%
\pgfpathlineto{\pgfqpoint{5.933897in}{1.931557in}}%
\pgfpathlineto{\pgfqpoint{5.936474in}{1.928180in}}%
\pgfpathlineto{\pgfqpoint{5.939051in}{1.908478in}}%
\pgfpathlineto{\pgfqpoint{5.946783in}{1.908103in}}%
\pgfpathlineto{\pgfqpoint{5.949360in}{1.913732in}}%
\pgfpathlineto{\pgfqpoint{5.951937in}{1.900973in}}%
\pgfpathlineto{\pgfqpoint{5.954515in}{1.891779in}}%
\pgfpathlineto{\pgfqpoint{5.957092in}{1.891591in}}%
\pgfpathlineto{\pgfqpoint{5.964823in}{1.888777in}}%
\pgfpathlineto{\pgfqpoint{5.967401in}{1.893843in}}%
\pgfpathlineto{\pgfqpoint{5.969978in}{1.897595in}}%
\pgfpathlineto{\pgfqpoint{5.972555in}{1.884273in}}%
\pgfpathlineto{\pgfqpoint{5.975132in}{1.876580in}}%
\pgfpathlineto{\pgfqpoint{5.982864in}{1.891779in}}%
\pgfpathlineto{\pgfqpoint{5.985441in}{1.888214in}}%
\pgfpathlineto{\pgfqpoint{5.988018in}{1.856128in}}%
\pgfpathlineto{\pgfqpoint{5.990596in}{1.856316in}}%
\pgfpathlineto{\pgfqpoint{5.993173in}{1.864009in}}%
\pgfpathlineto{\pgfqpoint{6.000904in}{1.867199in}}%
\pgfpathlineto{\pgfqpoint{6.003482in}{1.871327in}}%
\pgfpathlineto{\pgfqpoint{6.006059in}{1.869826in}}%
\pgfpathlineto{\pgfqpoint{6.008636in}{1.875642in}}%
\pgfpathlineto{\pgfqpoint{6.011213in}{1.876018in}}%
\pgfpathlineto{\pgfqpoint{6.021522in}{1.869263in}}%
\pgfpathlineto{\pgfqpoint{6.024099in}{1.866636in}}%
\pgfpathlineto{\pgfqpoint{6.026677in}{1.890841in}}%
\pgfpathlineto{\pgfqpoint{6.029254in}{1.894781in}}%
\pgfpathlineto{\pgfqpoint{6.036985in}{1.903412in}}%
\pgfpathlineto{\pgfqpoint{6.039563in}{1.901160in}}%
\pgfpathlineto{\pgfqpoint{6.042140in}{1.918610in}}%
\pgfpathlineto{\pgfqpoint{6.044717in}{1.921237in}}%
\pgfpathlineto{\pgfqpoint{6.047294in}{1.927804in}}%
\pgfpathlineto{\pgfqpoint{6.055026in}{1.925178in}}%
\pgfpathlineto{\pgfqpoint{6.057603in}{1.932871in}}%
\pgfpathlineto{\pgfqpoint{6.060180in}{1.937374in}}%
\pgfpathlineto{\pgfqpoint{6.062757in}{1.934747in}}%
\pgfpathlineto{\pgfqpoint{6.065335in}{1.948444in}}%
\pgfpathlineto{\pgfqpoint{6.073066in}{1.956325in}}%
\pgfpathlineto{\pgfqpoint{6.075644in}{1.967770in}}%
\pgfpathlineto{\pgfqpoint{6.078221in}{1.962329in}}%
\pgfpathlineto{\pgfqpoint{6.080798in}{1.962892in}}%
\pgfpathlineto{\pgfqpoint{6.083375in}{1.962517in}}%
\pgfpathlineto{\pgfqpoint{6.091107in}{1.975839in}}%
\pgfpathlineto{\pgfqpoint{6.093684in}{1.978090in}}%
\pgfpathlineto{\pgfqpoint{6.096261in}{1.994227in}}%
\pgfpathlineto{\pgfqpoint{6.098838in}{1.988035in}}%
\pgfpathlineto{\pgfqpoint{6.101416in}{1.999105in}}%
\pgfpathlineto{\pgfqpoint{6.109147in}{2.014679in}}%
\pgfpathlineto{\pgfqpoint{6.114302in}{2.016743in}}%
\pgfpathlineto{\pgfqpoint{6.116879in}{1.995728in}}%
\pgfpathlineto{\pgfqpoint{6.119456in}{2.006611in}}%
\pgfpathlineto{\pgfqpoint{6.127188in}{2.006048in}}%
\pgfpathlineto{\pgfqpoint{6.129765in}{2.003421in}}%
\pgfpathlineto{\pgfqpoint{6.134919in}{2.025562in}}%
\pgfpathlineto{\pgfqpoint{6.137497in}{2.023873in}}%
\pgfpathlineto{\pgfqpoint{6.145228in}{2.022747in}}%
\pgfpathlineto{\pgfqpoint{6.147806in}{2.028188in}}%
\pgfpathlineto{\pgfqpoint{6.150383in}{2.035694in}}%
\pgfpathlineto{\pgfqpoint{6.152960in}{2.025186in}}%
\pgfpathlineto{\pgfqpoint{6.155537in}{2.029502in}}%
\pgfpathlineto{\pgfqpoint{6.163269in}{2.018807in}}%
\pgfpathlineto{\pgfqpoint{6.165846in}{2.026124in}}%
\pgfpathlineto{\pgfqpoint{6.168423in}{2.024248in}}%
\pgfpathlineto{\pgfqpoint{6.171000in}{1.997792in}}%
\pgfpathlineto{\pgfqpoint{6.173578in}{2.015054in}}%
\pgfpathlineto{\pgfqpoint{6.181309in}{2.022935in}}%
\pgfpathlineto{\pgfqpoint{6.183886in}{2.023122in}}%
\pgfpathlineto{\pgfqpoint{6.186464in}{2.024248in}}%
\pgfpathlineto{\pgfqpoint{6.189041in}{2.028564in}}%
\pgfpathlineto{\pgfqpoint{6.191618in}{2.036444in}}%
\pgfpathlineto{\pgfqpoint{6.199350in}{2.034193in}}%
\pgfpathlineto{\pgfqpoint{6.201927in}{2.035881in}}%
\pgfpathlineto{\pgfqpoint{6.204504in}{2.031378in}}%
\pgfpathlineto{\pgfqpoint{6.207081in}{2.010738in}}%
\pgfpathlineto{\pgfqpoint{6.209659in}{2.005860in}}%
\pgfpathlineto{\pgfqpoint{6.217390in}{2.027250in}}%
\pgfpathlineto{\pgfqpoint{6.219967in}{2.051455in}}%
\pgfpathlineto{\pgfqpoint{6.222545in}{2.062338in}}%
\pgfpathlineto{\pgfqpoint{6.225122in}{2.038696in}}%
\pgfpathlineto{\pgfqpoint{6.227699in}{2.025749in}}%
\pgfpathlineto{\pgfqpoint{6.238008in}{2.024248in}}%
\pgfpathlineto{\pgfqpoint{6.240585in}{2.023873in}}%
\pgfpathlineto{\pgfqpoint{6.245740in}{2.027250in}}%
\pgfpathlineto{\pgfqpoint{6.253471in}{2.026124in}}%
\pgfpathlineto{\pgfqpoint{6.256048in}{2.026312in}}%
\pgfpathlineto{\pgfqpoint{6.258626in}{2.031378in}}%
\pgfpathlineto{\pgfqpoint{6.261203in}{2.038696in}}%
\pgfpathlineto{\pgfqpoint{6.263780in}{2.039446in}}%
\pgfpathlineto{\pgfqpoint{6.274089in}{2.024436in}}%
\pgfpathlineto{\pgfqpoint{6.276666in}{2.022935in}}%
\pgfpathlineto{\pgfqpoint{6.279243in}{2.007924in}}%
\pgfpathlineto{\pgfqpoint{6.281821in}{2.005109in}}%
\pgfpathlineto{\pgfqpoint{6.289552in}{2.031378in}}%
\pgfpathlineto{\pgfqpoint{6.292129in}{2.046952in}}%
\pgfpathlineto{\pgfqpoint{6.294707in}{2.048640in}}%
\pgfpathlineto{\pgfqpoint{6.297284in}{2.040385in}}%
\pgfpathlineto{\pgfqpoint{6.299861in}{2.055020in}}%
\pgfpathlineto{\pgfqpoint{6.307593in}{2.070969in}}%
\pgfpathlineto{\pgfqpoint{6.310170in}{2.091609in}}%
\pgfpathlineto{\pgfqpoint{6.312747in}{2.081289in}}%
\pgfpathlineto{\pgfqpoint{6.317902in}{2.080726in}}%
\pgfpathlineto{\pgfqpoint{6.325633in}{2.077161in}}%
\pgfpathlineto{\pgfqpoint{6.328210in}{2.085980in}}%
\pgfpathlineto{\pgfqpoint{6.333365in}{2.112436in}}%
\pgfpathlineto{\pgfqpoint{6.335942in}{2.118253in}}%
\pgfpathlineto{\pgfqpoint{6.343674in}{2.119754in}}%
\pgfpathlineto{\pgfqpoint{6.346251in}{2.135890in}}%
\pgfpathlineto{\pgfqpoint{6.348828in}{2.128197in}}%
\pgfpathlineto{\pgfqpoint{6.351405in}{2.137391in}}%
\pgfpathlineto{\pgfqpoint{6.353982in}{2.144521in}}%
\pgfpathlineto{\pgfqpoint{6.361714in}{2.147148in}}%
\pgfpathlineto{\pgfqpoint{6.364291in}{2.150713in}}%
\pgfpathlineto{\pgfqpoint{6.366869in}{2.152027in}}%
\pgfpathlineto{\pgfqpoint{6.369446in}{2.145647in}}%
\pgfpathlineto{\pgfqpoint{6.372023in}{2.168538in}}%
\pgfpathlineto{\pgfqpoint{6.382332in}{2.147148in}}%
\pgfpathlineto{\pgfqpoint{6.384909in}{2.154278in}}%
\pgfpathlineto{\pgfqpoint{6.387486in}{2.150901in}}%
\pgfpathlineto{\pgfqpoint{6.390063in}{2.154466in}}%
\pgfpathlineto{\pgfqpoint{6.397795in}{2.159720in}}%
\pgfpathlineto{\pgfqpoint{6.400372in}{2.186739in}}%
\pgfpathlineto{\pgfqpoint{6.402950in}{2.180735in}}%
\pgfpathlineto{\pgfqpoint{6.405527in}{2.220325in}}%
\pgfpathlineto{\pgfqpoint{6.408104in}{2.222202in}}%
\pgfpathlineto{\pgfqpoint{6.415836in}{2.208880in}}%
\pgfpathlineto{\pgfqpoint{6.418413in}{2.217135in}}%
\pgfpathlineto{\pgfqpoint{6.420990in}{2.222202in}}%
\pgfpathlineto{\pgfqpoint{6.423567in}{2.225579in}}%
\pgfpathlineto{\pgfqpoint{6.426144in}{2.233835in}}%
\pgfpathlineto{\pgfqpoint{6.433876in}{2.231208in}}%
\pgfpathlineto{\pgfqpoint{6.436453in}{2.214509in}}%
\pgfpathlineto{\pgfqpoint{6.439031in}{2.210005in}}%
\pgfpathlineto{\pgfqpoint{6.441608in}{2.184862in}}%
\pgfpathlineto{\pgfqpoint{6.444185in}{2.180547in}}%
\pgfpathlineto{\pgfqpoint{6.451917in}{2.187489in}}%
\pgfpathlineto{\pgfqpoint{6.454494in}{2.185050in}}%
\pgfpathlineto{\pgfqpoint{6.457071in}{2.175856in}}%
\pgfpathlineto{\pgfqpoint{6.459648in}{2.181297in}}%
\pgfpathlineto{\pgfqpoint{6.462225in}{2.183737in}}%
\pgfpathlineto{\pgfqpoint{6.469957in}{2.188427in}}%
\pgfpathlineto{\pgfqpoint{6.472534in}{2.196871in}}%
\pgfpathlineto{\pgfqpoint{6.475111in}{2.186176in}}%
\pgfpathlineto{\pgfqpoint{6.480266in}{2.179796in}}%
\pgfpathlineto{\pgfqpoint{6.487998in}{2.179609in}}%
\pgfpathlineto{\pgfqpoint{6.490575in}{2.212820in}}%
\pgfpathlineto{\pgfqpoint{6.495729in}{2.257289in}}%
\pgfpathlineto{\pgfqpoint{6.498306in}{2.260104in}}%
\pgfpathlineto{\pgfqpoint{6.506038in}{2.273425in}}%
\pgfpathlineto{\pgfqpoint{6.508615in}{2.275489in}}%
\pgfpathlineto{\pgfqpoint{6.511192in}{2.266483in}}%
\pgfpathlineto{\pgfqpoint{6.513770in}{2.273238in}}%
\pgfpathlineto{\pgfqpoint{6.516347in}{2.272675in}}%
\pgfpathlineto{\pgfqpoint{6.524079in}{2.280931in}}%
\pgfpathlineto{\pgfqpoint{6.526656in}{2.287686in}}%
\pgfpathlineto{\pgfqpoint{6.529233in}{2.258602in}}%
\pgfpathlineto{\pgfqpoint{6.531810in}{2.246969in}}%
\pgfpathlineto{\pgfqpoint{6.534387in}{2.272112in}}%
\pgfpathlineto{\pgfqpoint{6.542119in}{2.293315in}}%
\pgfpathlineto{\pgfqpoint{6.547273in}{2.271924in}}%
\pgfpathlineto{\pgfqpoint{6.549851in}{2.271737in}}%
\pgfpathlineto{\pgfqpoint{6.552428in}{2.276052in}}%
\pgfpathlineto{\pgfqpoint{6.562737in}{2.273050in}}%
\pgfpathlineto{\pgfqpoint{6.567891in}{2.293690in}}%
\pgfpathlineto{\pgfqpoint{6.570468in}{2.286560in}}%
\pgfpathlineto{\pgfqpoint{6.570468in}{2.286560in}}%
\pgfusepath{stroke}%
\end{pgfscope}%
\begin{pgfscope}%
\pgfpathrectangle{\pgfqpoint{0.652287in}{0.521603in}}{\pgfqpoint{6.200000in}{4.530000in}}%
\pgfusepath{clip}%
\pgfsetroundcap%
\pgfsetroundjoin%
\pgfsetlinewidth{1.505625pt}%
\definecolor{currentstroke}{rgb}{0.172549,0.627451,0.172549}%
\pgfsetstrokecolor{currentstroke}%
\pgfsetdash{}{0pt}%
\pgfpathmoveto{\pgfqpoint{0.934105in}{0.728263in}}%
\pgfpathlineto{\pgfqpoint{0.936682in}{0.731077in}}%
\pgfpathlineto{\pgfqpoint{0.939259in}{0.730890in}}%
\pgfpathlineto{\pgfqpoint{0.941836in}{0.732391in}}%
\pgfpathlineto{\pgfqpoint{0.949568in}{0.735393in}}%
\pgfpathlineto{\pgfqpoint{0.952145in}{0.733329in}}%
\pgfpathlineto{\pgfqpoint{0.954722in}{0.735581in}}%
\pgfpathlineto{\pgfqpoint{0.957300in}{0.736331in}}%
\pgfpathlineto{\pgfqpoint{0.959877in}{0.735018in}}%
\pgfpathlineto{\pgfqpoint{0.970186in}{0.733704in}}%
\pgfpathlineto{\pgfqpoint{0.972763in}{0.737645in}}%
\pgfpathlineto{\pgfqpoint{0.975340in}{0.739333in}}%
\pgfpathlineto{\pgfqpoint{0.977917in}{0.739333in}}%
\pgfpathlineto{\pgfqpoint{0.988226in}{0.735018in}}%
\pgfpathlineto{\pgfqpoint{0.990803in}{0.739146in}}%
\pgfpathlineto{\pgfqpoint{0.995958in}{0.737645in}}%
\pgfpathlineto{\pgfqpoint{1.003689in}{0.735956in}}%
\pgfpathlineto{\pgfqpoint{1.006267in}{0.733141in}}%
\pgfpathlineto{\pgfqpoint{1.008844in}{0.734080in}}%
\pgfpathlineto{\pgfqpoint{1.011421in}{0.733704in}}%
\pgfpathlineto{\pgfqpoint{1.013998in}{0.737645in}}%
\pgfpathlineto{\pgfqpoint{1.021730in}{0.738020in}}%
\pgfpathlineto{\pgfqpoint{1.024307in}{0.739896in}}%
\pgfpathlineto{\pgfqpoint{1.026884in}{0.740647in}}%
\pgfpathlineto{\pgfqpoint{1.029462in}{0.739146in}}%
\pgfpathlineto{\pgfqpoint{1.032039in}{0.735581in}}%
\pgfpathlineto{\pgfqpoint{1.039770in}{0.738208in}}%
\pgfpathlineto{\pgfqpoint{1.042348in}{0.736519in}}%
\pgfpathlineto{\pgfqpoint{1.044925in}{0.733892in}}%
\pgfpathlineto{\pgfqpoint{1.050079in}{0.741210in}}%
\pgfpathlineto{\pgfqpoint{1.060388in}{0.743086in}}%
\pgfpathlineto{\pgfqpoint{1.062965in}{0.742711in}}%
\pgfpathlineto{\pgfqpoint{1.065543in}{0.744024in}}%
\pgfpathlineto{\pgfqpoint{1.075851in}{0.740647in}}%
\pgfpathlineto{\pgfqpoint{1.078429in}{0.741960in}}%
\pgfpathlineto{\pgfqpoint{1.081006in}{0.740459in}}%
\pgfpathlineto{\pgfqpoint{1.083583in}{0.741397in}}%
\pgfpathlineto{\pgfqpoint{1.086160in}{0.739333in}}%
\pgfpathlineto{\pgfqpoint{1.093892in}{0.737457in}}%
\pgfpathlineto{\pgfqpoint{1.096469in}{0.731453in}}%
\pgfpathlineto{\pgfqpoint{1.101624in}{0.740084in}}%
\pgfpathlineto{\pgfqpoint{1.104201in}{0.740272in}}%
\pgfpathlineto{\pgfqpoint{1.111932in}{0.741585in}}%
\pgfpathlineto{\pgfqpoint{1.114510in}{0.748152in}}%
\pgfpathlineto{\pgfqpoint{1.117087in}{0.750967in}}%
\pgfpathlineto{\pgfqpoint{1.119664in}{0.756220in}}%
\pgfpathlineto{\pgfqpoint{1.122241in}{0.756783in}}%
\pgfpathlineto{\pgfqpoint{1.129973in}{0.756971in}}%
\pgfpathlineto{\pgfqpoint{1.132550in}{0.754907in}}%
\pgfpathlineto{\pgfqpoint{1.135127in}{0.754907in}}%
\pgfpathlineto{\pgfqpoint{1.137704in}{0.751717in}}%
\pgfpathlineto{\pgfqpoint{1.140282in}{0.750779in}}%
\pgfpathlineto{\pgfqpoint{1.148013in}{0.754532in}}%
\pgfpathlineto{\pgfqpoint{1.150591in}{0.754532in}}%
\pgfpathlineto{\pgfqpoint{1.155745in}{0.753218in}}%
\pgfpathlineto{\pgfqpoint{1.158322in}{0.754907in}}%
\pgfpathlineto{\pgfqpoint{1.168631in}{0.753406in}}%
\pgfpathlineto{\pgfqpoint{1.173785in}{0.746651in}}%
\pgfpathlineto{\pgfqpoint{1.184094in}{0.742523in}}%
\pgfpathlineto{\pgfqpoint{1.186672in}{0.735956in}}%
\pgfpathlineto{\pgfqpoint{1.191826in}{0.744024in}}%
\pgfpathlineto{\pgfqpoint{1.194403in}{0.738020in}}%
\pgfpathlineto{\pgfqpoint{1.202135in}{0.738208in}}%
\pgfpathlineto{\pgfqpoint{1.204712in}{0.744587in}}%
\pgfpathlineto{\pgfqpoint{1.207289in}{0.741022in}}%
\pgfpathlineto{\pgfqpoint{1.209866in}{0.741585in}}%
\pgfpathlineto{\pgfqpoint{1.212444in}{0.744775in}}%
\pgfpathlineto{\pgfqpoint{1.220175in}{0.740647in}}%
\pgfpathlineto{\pgfqpoint{1.222753in}{0.747402in}}%
\pgfpathlineto{\pgfqpoint{1.225330in}{0.746088in}}%
\pgfpathlineto{\pgfqpoint{1.230484in}{0.750779in}}%
\pgfpathlineto{\pgfqpoint{1.238216in}{0.747964in}}%
\pgfpathlineto{\pgfqpoint{1.240793in}{0.750967in}}%
\pgfpathlineto{\pgfqpoint{1.243370in}{0.750591in}}%
\pgfpathlineto{\pgfqpoint{1.245947in}{0.748340in}}%
\pgfpathlineto{\pgfqpoint{1.248525in}{0.744587in}}%
\pgfpathlineto{\pgfqpoint{1.256256in}{0.744212in}}%
\pgfpathlineto{\pgfqpoint{1.258833in}{0.743274in}}%
\pgfpathlineto{\pgfqpoint{1.261411in}{0.738395in}}%
\pgfpathlineto{\pgfqpoint{1.263988in}{0.741022in}}%
\pgfpathlineto{\pgfqpoint{1.266565in}{0.739896in}}%
\pgfpathlineto{\pgfqpoint{1.274297in}{0.733892in}}%
\pgfpathlineto{\pgfqpoint{1.276874in}{0.731077in}}%
\pgfpathlineto{\pgfqpoint{1.279451in}{0.739709in}}%
\pgfpathlineto{\pgfqpoint{1.282028in}{0.738020in}}%
\pgfpathlineto{\pgfqpoint{1.294914in}{0.742148in}}%
\pgfpathlineto{\pgfqpoint{1.297492in}{0.742148in}}%
\pgfpathlineto{\pgfqpoint{1.300069in}{0.743274in}}%
\pgfpathlineto{\pgfqpoint{1.302646in}{0.742523in}}%
\pgfpathlineto{\pgfqpoint{1.312955in}{0.744587in}}%
\pgfpathlineto{\pgfqpoint{1.315532in}{0.740272in}}%
\pgfpathlineto{\pgfqpoint{1.318109in}{0.741022in}}%
\pgfpathlineto{\pgfqpoint{1.320687in}{0.733141in}}%
\pgfpathlineto{\pgfqpoint{1.328418in}{0.727512in}}%
\pgfpathlineto{\pgfqpoint{1.330995in}{0.728826in}}%
\pgfpathlineto{\pgfqpoint{1.333573in}{0.738020in}}%
\pgfpathlineto{\pgfqpoint{1.336150in}{0.739709in}}%
\pgfpathlineto{\pgfqpoint{1.338727in}{0.742523in}}%
\pgfpathlineto{\pgfqpoint{1.346459in}{0.741210in}}%
\pgfpathlineto{\pgfqpoint{1.349036in}{0.746463in}}%
\pgfpathlineto{\pgfqpoint{1.351613in}{0.744962in}}%
\pgfpathlineto{\pgfqpoint{1.356768in}{0.754719in}}%
\pgfpathlineto{\pgfqpoint{1.364499in}{0.751154in}}%
\pgfpathlineto{\pgfqpoint{1.367076in}{0.754719in}}%
\pgfpathlineto{\pgfqpoint{1.369654in}{0.756033in}}%
\pgfpathlineto{\pgfqpoint{1.372231in}{0.750404in}}%
\pgfpathlineto{\pgfqpoint{1.374808in}{0.754344in}}%
\pgfpathlineto{\pgfqpoint{1.382540in}{0.750216in}}%
\pgfpathlineto{\pgfqpoint{1.387694in}{0.759035in}}%
\pgfpathlineto{\pgfqpoint{1.390271in}{0.759973in}}%
\pgfpathlineto{\pgfqpoint{1.392849in}{0.769167in}}%
\pgfpathlineto{\pgfqpoint{1.400580in}{0.764101in}}%
\pgfpathlineto{\pgfqpoint{1.408312in}{0.761849in}}%
\pgfpathlineto{\pgfqpoint{1.410889in}{0.757159in}}%
\pgfpathlineto{\pgfqpoint{1.418621in}{0.757721in}}%
\pgfpathlineto{\pgfqpoint{1.421198in}{0.751717in}}%
\pgfpathlineto{\pgfqpoint{1.423775in}{0.752468in}}%
\pgfpathlineto{\pgfqpoint{1.426352in}{0.749090in}}%
\pgfpathlineto{\pgfqpoint{1.428930in}{0.753781in}}%
\pgfpathlineto{\pgfqpoint{1.436661in}{0.751154in}}%
\pgfpathlineto{\pgfqpoint{1.441816in}{0.754907in}}%
\pgfpathlineto{\pgfqpoint{1.444393in}{0.754156in}}%
\pgfpathlineto{\pgfqpoint{1.454702in}{0.758472in}}%
\pgfpathlineto{\pgfqpoint{1.457279in}{0.756596in}}%
\pgfpathlineto{\pgfqpoint{1.459856in}{0.757159in}}%
\pgfpathlineto{\pgfqpoint{1.462433in}{0.765227in}}%
\pgfpathlineto{\pgfqpoint{1.465010in}{0.770293in}}%
\pgfpathlineto{\pgfqpoint{1.477897in}{0.767666in}}%
\pgfpathlineto{\pgfqpoint{1.480474in}{0.764664in}}%
\pgfpathlineto{\pgfqpoint{1.483051in}{0.770856in}}%
\pgfpathlineto{\pgfqpoint{1.490783in}{0.771043in}}%
\pgfpathlineto{\pgfqpoint{1.493360in}{0.773295in}}%
\pgfpathlineto{\pgfqpoint{1.495937in}{0.771606in}}%
\pgfpathlineto{\pgfqpoint{1.501091in}{0.772920in}}%
\pgfpathlineto{\pgfqpoint{1.513978in}{0.770856in}}%
\pgfpathlineto{\pgfqpoint{1.516555in}{0.772169in}}%
\pgfpathlineto{\pgfqpoint{1.519132in}{0.771419in}}%
\pgfpathlineto{\pgfqpoint{1.526864in}{0.770481in}}%
\pgfpathlineto{\pgfqpoint{1.532018in}{0.768417in}}%
\pgfpathlineto{\pgfqpoint{1.534595in}{0.766353in}}%
\pgfpathlineto{\pgfqpoint{1.537172in}{0.768604in}}%
\pgfpathlineto{\pgfqpoint{1.544904in}{0.769355in}}%
\pgfpathlineto{\pgfqpoint{1.547481in}{0.768792in}}%
\pgfpathlineto{\pgfqpoint{1.550058in}{0.768979in}}%
\pgfpathlineto{\pgfqpoint{1.552636in}{0.766353in}}%
\pgfpathlineto{\pgfqpoint{1.555213in}{0.767291in}}%
\pgfpathlineto{\pgfqpoint{1.565522in}{0.764476in}}%
\pgfpathlineto{\pgfqpoint{1.568099in}{0.766540in}}%
\pgfpathlineto{\pgfqpoint{1.570676in}{0.775922in}}%
\pgfpathlineto{\pgfqpoint{1.573253in}{0.780050in}}%
\pgfpathlineto{\pgfqpoint{1.580985in}{0.778361in}}%
\pgfpathlineto{\pgfqpoint{1.583562in}{0.780050in}}%
\pgfpathlineto{\pgfqpoint{1.586139in}{0.784365in}}%
\pgfpathlineto{\pgfqpoint{1.588717in}{0.786242in}}%
\pgfpathlineto{\pgfqpoint{1.591294in}{0.787368in}}%
\pgfpathlineto{\pgfqpoint{1.599026in}{0.786617in}}%
\pgfpathlineto{\pgfqpoint{1.606757in}{0.794498in}}%
\pgfpathlineto{\pgfqpoint{1.609334in}{0.795999in}}%
\pgfpathlineto{\pgfqpoint{1.619643in}{0.792809in}}%
\pgfpathlineto{\pgfqpoint{1.622220in}{0.789807in}}%
\pgfpathlineto{\pgfqpoint{1.624798in}{0.798813in}}%
\pgfpathlineto{\pgfqpoint{1.627375in}{0.798626in}}%
\pgfpathlineto{\pgfqpoint{1.635106in}{0.799939in}}%
\pgfpathlineto{\pgfqpoint{1.637684in}{0.799751in}}%
\pgfpathlineto{\pgfqpoint{1.640261in}{0.801440in}}%
\pgfpathlineto{\pgfqpoint{1.642838in}{0.802003in}}%
\pgfpathlineto{\pgfqpoint{1.645415in}{0.804442in}}%
\pgfpathlineto{\pgfqpoint{1.653147in}{0.801628in}}%
\pgfpathlineto{\pgfqpoint{1.655724in}{0.797312in}}%
\pgfpathlineto{\pgfqpoint{1.658301in}{0.794498in}}%
\pgfpathlineto{\pgfqpoint{1.660879in}{0.795623in}}%
\pgfpathlineto{\pgfqpoint{1.663456in}{0.795248in}}%
\pgfpathlineto{\pgfqpoint{1.671187in}{0.797500in}}%
\pgfpathlineto{\pgfqpoint{1.673765in}{0.797500in}}%
\pgfpathlineto{\pgfqpoint{1.676342in}{0.801440in}}%
\pgfpathlineto{\pgfqpoint{1.678919in}{0.799939in}}%
\pgfpathlineto{\pgfqpoint{1.681496in}{0.788681in}}%
\pgfpathlineto{\pgfqpoint{1.689228in}{0.783990in}}%
\pgfpathlineto{\pgfqpoint{1.691805in}{0.777798in}}%
\pgfpathlineto{\pgfqpoint{1.696960in}{0.777611in}}%
\pgfpathlineto{\pgfqpoint{1.699537in}{0.775359in}}%
\pgfpathlineto{\pgfqpoint{1.712423in}{0.774608in}}%
\pgfpathlineto{\pgfqpoint{1.715000in}{0.778736in}}%
\pgfpathlineto{\pgfqpoint{1.717577in}{0.778361in}}%
\pgfpathlineto{\pgfqpoint{1.725309in}{0.779675in}}%
\pgfpathlineto{\pgfqpoint{1.727886in}{0.782301in}}%
\pgfpathlineto{\pgfqpoint{1.730463in}{0.775734in}}%
\pgfpathlineto{\pgfqpoint{1.733041in}{0.772169in}}%
\pgfpathlineto{\pgfqpoint{1.735618in}{0.773858in}}%
\pgfpathlineto{\pgfqpoint{1.743349in}{0.772169in}}%
\pgfpathlineto{\pgfqpoint{1.745927in}{0.769167in}}%
\pgfpathlineto{\pgfqpoint{1.748504in}{0.759410in}}%
\pgfpathlineto{\pgfqpoint{1.753658in}{0.761474in}}%
\pgfpathlineto{\pgfqpoint{1.761390in}{0.768792in}}%
\pgfpathlineto{\pgfqpoint{1.763967in}{0.768229in}}%
\pgfpathlineto{\pgfqpoint{1.766544in}{0.769167in}}%
\pgfpathlineto{\pgfqpoint{1.771699in}{0.774421in}}%
\pgfpathlineto{\pgfqpoint{1.779430in}{0.774608in}}%
\pgfpathlineto{\pgfqpoint{1.782008in}{0.771982in}}%
\pgfpathlineto{\pgfqpoint{1.784585in}{0.775922in}}%
\pgfpathlineto{\pgfqpoint{1.789739in}{0.775734in}}%
\pgfpathlineto{\pgfqpoint{1.797471in}{0.771231in}}%
\pgfpathlineto{\pgfqpoint{1.800048in}{0.771794in}}%
\pgfpathlineto{\pgfqpoint{1.802625in}{0.777048in}}%
\pgfpathlineto{\pgfqpoint{1.805203in}{0.779112in}}%
\pgfpathlineto{\pgfqpoint{1.807780in}{0.780425in}}%
\pgfpathlineto{\pgfqpoint{1.815511in}{0.779487in}}%
\pgfpathlineto{\pgfqpoint{1.818089in}{0.781176in}}%
\pgfpathlineto{\pgfqpoint{1.820666in}{0.785116in}}%
\pgfpathlineto{\pgfqpoint{1.823243in}{0.782864in}}%
\pgfpathlineto{\pgfqpoint{1.825820in}{0.782864in}}%
\pgfpathlineto{\pgfqpoint{1.833552in}{0.787368in}}%
\pgfpathlineto{\pgfqpoint{1.836129in}{0.783802in}}%
\pgfpathlineto{\pgfqpoint{1.838706in}{0.773858in}}%
\pgfpathlineto{\pgfqpoint{1.841283in}{0.777235in}}%
\pgfpathlineto{\pgfqpoint{1.843861in}{0.774796in}}%
\pgfpathlineto{\pgfqpoint{1.856747in}{0.773295in}}%
\pgfpathlineto{\pgfqpoint{1.859324in}{0.771982in}}%
\pgfpathlineto{\pgfqpoint{1.861901in}{0.768417in}}%
\pgfpathlineto{\pgfqpoint{1.874787in}{0.781551in}}%
\pgfpathlineto{\pgfqpoint{1.877364in}{0.777986in}}%
\pgfpathlineto{\pgfqpoint{1.879942in}{0.779487in}}%
\pgfpathlineto{\pgfqpoint{1.887673in}{0.778549in}}%
\pgfpathlineto{\pgfqpoint{1.890251in}{0.775171in}}%
\pgfpathlineto{\pgfqpoint{1.892828in}{0.775922in}}%
\pgfpathlineto{\pgfqpoint{1.895405in}{0.779112in}}%
\pgfpathlineto{\pgfqpoint{1.897982in}{0.778549in}}%
\pgfpathlineto{\pgfqpoint{1.905714in}{0.778361in}}%
\pgfpathlineto{\pgfqpoint{1.908291in}{0.779487in}}%
\pgfpathlineto{\pgfqpoint{1.910868in}{0.778361in}}%
\pgfpathlineto{\pgfqpoint{1.913445in}{0.780988in}}%
\pgfpathlineto{\pgfqpoint{1.916023in}{0.791871in}}%
\pgfpathlineto{\pgfqpoint{1.926332in}{0.791308in}}%
\pgfpathlineto{\pgfqpoint{1.928909in}{0.790370in}}%
\pgfpathlineto{\pgfqpoint{1.931486in}{0.791871in}}%
\pgfpathlineto{\pgfqpoint{1.934063in}{0.795436in}}%
\pgfpathlineto{\pgfqpoint{1.941795in}{0.798438in}}%
\pgfpathlineto{\pgfqpoint{1.944372in}{0.798438in}}%
\pgfpathlineto{\pgfqpoint{1.946949in}{0.794498in}}%
\pgfpathlineto{\pgfqpoint{1.949526in}{0.795248in}}%
\pgfpathlineto{\pgfqpoint{1.952104in}{0.800314in}}%
\pgfpathlineto{\pgfqpoint{1.959835in}{0.795811in}}%
\pgfpathlineto{\pgfqpoint{1.962412in}{0.799188in}}%
\pgfpathlineto{\pgfqpoint{1.964990in}{0.797687in}}%
\pgfpathlineto{\pgfqpoint{1.970144in}{0.798438in}}%
\pgfpathlineto{\pgfqpoint{1.977876in}{0.797875in}}%
\pgfpathlineto{\pgfqpoint{1.980453in}{0.799751in}}%
\pgfpathlineto{\pgfqpoint{1.983030in}{0.811572in}}%
\pgfpathlineto{\pgfqpoint{1.985607in}{0.811760in}}%
\pgfpathlineto{\pgfqpoint{1.988185in}{0.810071in}}%
\pgfpathlineto{\pgfqpoint{1.998493in}{0.816826in}}%
\pgfpathlineto{\pgfqpoint{2.001071in}{0.811760in}}%
\pgfpathlineto{\pgfqpoint{2.003648in}{0.812323in}}%
\pgfpathlineto{\pgfqpoint{2.006225in}{0.814387in}}%
\pgfpathlineto{\pgfqpoint{2.013957in}{0.805756in}}%
\pgfpathlineto{\pgfqpoint{2.016534in}{0.809321in}}%
\pgfpathlineto{\pgfqpoint{2.019111in}{0.814012in}}%
\pgfpathlineto{\pgfqpoint{2.021688in}{0.811760in}}%
\pgfpathlineto{\pgfqpoint{2.024266in}{0.811385in}}%
\pgfpathlineto{\pgfqpoint{2.031997in}{0.812510in}}%
\pgfpathlineto{\pgfqpoint{2.034574in}{0.817201in}}%
\pgfpathlineto{\pgfqpoint{2.037152in}{0.818515in}}%
\pgfpathlineto{\pgfqpoint{2.039729in}{0.818515in}}%
\pgfpathlineto{\pgfqpoint{2.042306in}{0.820016in}}%
\pgfpathlineto{\pgfqpoint{2.050038in}{0.817764in}}%
\pgfpathlineto{\pgfqpoint{2.052615in}{0.814574in}}%
\pgfpathlineto{\pgfqpoint{2.055192in}{0.815888in}}%
\pgfpathlineto{\pgfqpoint{2.057769in}{0.818702in}}%
\pgfpathlineto{\pgfqpoint{2.060347in}{0.815137in}}%
\pgfpathlineto{\pgfqpoint{2.068078in}{0.812323in}}%
\pgfpathlineto{\pgfqpoint{2.070655in}{0.813261in}}%
\pgfpathlineto{\pgfqpoint{2.073233in}{0.815325in}}%
\pgfpathlineto{\pgfqpoint{2.075810in}{0.812886in}}%
\pgfpathlineto{\pgfqpoint{2.078387in}{0.814012in}}%
\pgfpathlineto{\pgfqpoint{2.086119in}{0.812135in}}%
\pgfpathlineto{\pgfqpoint{2.088696in}{0.810259in}}%
\pgfpathlineto{\pgfqpoint{2.093850in}{0.810259in}}%
\pgfpathlineto{\pgfqpoint{2.104159in}{0.809696in}}%
\pgfpathlineto{\pgfqpoint{2.106736in}{0.813636in}}%
\pgfpathlineto{\pgfqpoint{2.109314in}{0.808570in}}%
\pgfpathlineto{\pgfqpoint{2.111891in}{0.809696in}}%
\pgfpathlineto{\pgfqpoint{2.114468in}{0.807632in}}%
\pgfpathlineto{\pgfqpoint{2.122200in}{0.810259in}}%
\pgfpathlineto{\pgfqpoint{2.124777in}{0.809508in}}%
\pgfpathlineto{\pgfqpoint{2.127354in}{0.817201in}}%
\pgfpathlineto{\pgfqpoint{2.129931in}{0.817201in}}%
\pgfpathlineto{\pgfqpoint{2.132509in}{0.815325in}}%
\pgfpathlineto{\pgfqpoint{2.140240in}{0.805756in}}%
\pgfpathlineto{\pgfqpoint{2.142817in}{0.810071in}}%
\pgfpathlineto{\pgfqpoint{2.145395in}{0.805005in}}%
\pgfpathlineto{\pgfqpoint{2.147972in}{0.803692in}}%
\pgfpathlineto{\pgfqpoint{2.150549in}{0.790182in}}%
\pgfpathlineto{\pgfqpoint{2.158281in}{0.784178in}}%
\pgfpathlineto{\pgfqpoint{2.160858in}{0.786429in}}%
\pgfpathlineto{\pgfqpoint{2.163435in}{0.793184in}}%
\pgfpathlineto{\pgfqpoint{2.166012in}{0.793184in}}%
\pgfpathlineto{\pgfqpoint{2.168589in}{0.796937in}}%
\pgfpathlineto{\pgfqpoint{2.178898in}{0.798063in}}%
\pgfpathlineto{\pgfqpoint{2.181476in}{0.795999in}}%
\pgfpathlineto{\pgfqpoint{2.184053in}{0.798626in}}%
\pgfpathlineto{\pgfqpoint{2.186630in}{0.802191in}}%
\pgfpathlineto{\pgfqpoint{2.194362in}{0.802378in}}%
\pgfpathlineto{\pgfqpoint{2.196939in}{0.803879in}}%
\pgfpathlineto{\pgfqpoint{2.199516in}{0.808758in}}%
\pgfpathlineto{\pgfqpoint{2.202093in}{0.805380in}}%
\pgfpathlineto{\pgfqpoint{2.204670in}{0.807069in}}%
\pgfpathlineto{\pgfqpoint{2.212402in}{0.806319in}}%
\pgfpathlineto{\pgfqpoint{2.214979in}{0.808758in}}%
\pgfpathlineto{\pgfqpoint{2.217557in}{0.812135in}}%
\pgfpathlineto{\pgfqpoint{2.220134in}{0.812510in}}%
\pgfpathlineto{\pgfqpoint{2.222711in}{0.815325in}}%
\pgfpathlineto{\pgfqpoint{2.230443in}{0.817014in}}%
\pgfpathlineto{\pgfqpoint{2.233020in}{0.818327in}}%
\pgfpathlineto{\pgfqpoint{2.235597in}{0.821329in}}%
\pgfpathlineto{\pgfqpoint{2.238174in}{0.818327in}}%
\pgfpathlineto{\pgfqpoint{2.240751in}{0.816451in}}%
\pgfpathlineto{\pgfqpoint{2.256215in}{0.817389in}}%
\pgfpathlineto{\pgfqpoint{2.258792in}{0.813261in}}%
\pgfpathlineto{\pgfqpoint{2.266524in}{0.817952in}}%
\pgfpathlineto{\pgfqpoint{2.269101in}{0.818327in}}%
\pgfpathlineto{\pgfqpoint{2.271678in}{0.813261in}}%
\pgfpathlineto{\pgfqpoint{2.274255in}{0.814199in}}%
\pgfpathlineto{\pgfqpoint{2.276832in}{0.821329in}}%
\pgfpathlineto{\pgfqpoint{2.284564in}{0.820016in}}%
\pgfpathlineto{\pgfqpoint{2.287141in}{0.817201in}}%
\pgfpathlineto{\pgfqpoint{2.289718in}{0.815888in}}%
\pgfpathlineto{\pgfqpoint{2.292296in}{0.818515in}}%
\pgfpathlineto{\pgfqpoint{2.294873in}{0.816263in}}%
\pgfpathlineto{\pgfqpoint{2.302605in}{0.820016in}}%
\pgfpathlineto{\pgfqpoint{2.305182in}{0.828272in}}%
\pgfpathlineto{\pgfqpoint{2.307759in}{0.823018in}}%
\pgfpathlineto{\pgfqpoint{2.310336in}{0.814950in}}%
\pgfpathlineto{\pgfqpoint{2.312913in}{0.816638in}}%
\pgfpathlineto{\pgfqpoint{2.320645in}{0.810259in}}%
\pgfpathlineto{\pgfqpoint{2.325799in}{0.814950in}}%
\pgfpathlineto{\pgfqpoint{2.328377in}{0.816075in}}%
\pgfpathlineto{\pgfqpoint{2.330954in}{0.814199in}}%
\pgfpathlineto{\pgfqpoint{2.338685in}{0.816263in}}%
\pgfpathlineto{\pgfqpoint{2.341263in}{0.809884in}}%
\pgfpathlineto{\pgfqpoint{2.343840in}{0.809884in}}%
\pgfpathlineto{\pgfqpoint{2.348994in}{0.814762in}}%
\pgfpathlineto{\pgfqpoint{2.356726in}{0.816075in}}%
\pgfpathlineto{\pgfqpoint{2.359303in}{0.820579in}}%
\pgfpathlineto{\pgfqpoint{2.361880in}{0.819265in}}%
\pgfpathlineto{\pgfqpoint{2.364458in}{0.825270in}}%
\pgfpathlineto{\pgfqpoint{2.367035in}{0.822643in}}%
\pgfpathlineto{\pgfqpoint{2.374766in}{0.820579in}}%
\pgfpathlineto{\pgfqpoint{2.377344in}{0.817764in}}%
\pgfpathlineto{\pgfqpoint{2.382498in}{0.820579in}}%
\pgfpathlineto{\pgfqpoint{2.385075in}{0.836903in}}%
\pgfpathlineto{\pgfqpoint{2.392807in}{0.838967in}}%
\pgfpathlineto{\pgfqpoint{2.395384in}{0.836715in}}%
\pgfpathlineto{\pgfqpoint{2.397961in}{0.835402in}}%
\pgfpathlineto{\pgfqpoint{2.400539in}{0.836340in}}%
\pgfpathlineto{\pgfqpoint{2.403116in}{0.835777in}}%
\pgfpathlineto{\pgfqpoint{2.410847in}{0.833338in}}%
\pgfpathlineto{\pgfqpoint{2.413425in}{0.833338in}}%
\pgfpathlineto{\pgfqpoint{2.416002in}{0.831649in}}%
\pgfpathlineto{\pgfqpoint{2.418579in}{0.835402in}}%
\pgfpathlineto{\pgfqpoint{2.421156in}{0.836528in}}%
\pgfpathlineto{\pgfqpoint{2.428888in}{0.833901in}}%
\pgfpathlineto{\pgfqpoint{2.431465in}{0.830711in}}%
\pgfpathlineto{\pgfqpoint{2.434042in}{0.831274in}}%
\pgfpathlineto{\pgfqpoint{2.436620in}{0.831086in}}%
\pgfpathlineto{\pgfqpoint{2.439197in}{0.829773in}}%
\pgfpathlineto{\pgfqpoint{2.446928in}{0.830148in}}%
\pgfpathlineto{\pgfqpoint{2.449506in}{0.829210in}}%
\pgfpathlineto{\pgfqpoint{2.452083in}{0.827146in}}%
\pgfpathlineto{\pgfqpoint{2.457237in}{0.825457in}}%
\pgfpathlineto{\pgfqpoint{2.464969in}{0.823956in}}%
\pgfpathlineto{\pgfqpoint{2.470123in}{0.820391in}}%
\pgfpathlineto{\pgfqpoint{2.472701in}{0.822830in}}%
\pgfpathlineto{\pgfqpoint{2.475278in}{0.822830in}}%
\pgfpathlineto{\pgfqpoint{2.483009in}{0.820391in}}%
\pgfpathlineto{\pgfqpoint{2.485587in}{0.814012in}}%
\pgfpathlineto{\pgfqpoint{2.488164in}{0.814199in}}%
\pgfpathlineto{\pgfqpoint{2.490741in}{0.812886in}}%
\pgfpathlineto{\pgfqpoint{2.493318in}{0.813449in}}%
\pgfpathlineto{\pgfqpoint{2.503627in}{0.812135in}}%
\pgfpathlineto{\pgfqpoint{2.506204in}{0.813824in}}%
\pgfpathlineto{\pgfqpoint{2.511359in}{0.813636in}}%
\pgfpathlineto{\pgfqpoint{2.519090in}{0.817014in}}%
\pgfpathlineto{\pgfqpoint{2.521668in}{0.824144in}}%
\pgfpathlineto{\pgfqpoint{2.524245in}{0.827521in}}%
\pgfpathlineto{\pgfqpoint{2.526822in}{0.823956in}}%
\pgfpathlineto{\pgfqpoint{2.529399in}{0.822830in}}%
\pgfpathlineto{\pgfqpoint{2.537131in}{0.828272in}}%
\pgfpathlineto{\pgfqpoint{2.539708in}{0.832775in}}%
\pgfpathlineto{\pgfqpoint{2.542285in}{0.838967in}}%
\pgfpathlineto{\pgfqpoint{2.544862in}{0.835777in}}%
\pgfpathlineto{\pgfqpoint{2.547440in}{0.829022in}}%
\pgfpathlineto{\pgfqpoint{2.555171in}{0.833150in}}%
\pgfpathlineto{\pgfqpoint{2.557749in}{0.833713in}}%
\pgfpathlineto{\pgfqpoint{2.560326in}{0.832400in}}%
\pgfpathlineto{\pgfqpoint{2.562903in}{0.832587in}}%
\pgfpathlineto{\pgfqpoint{2.565480in}{0.829585in}}%
\pgfpathlineto{\pgfqpoint{2.573212in}{0.827333in}}%
\pgfpathlineto{\pgfqpoint{2.575789in}{0.831461in}}%
\pgfpathlineto{\pgfqpoint{2.578366in}{0.833901in}}%
\pgfpathlineto{\pgfqpoint{2.580943in}{0.830336in}}%
\pgfpathlineto{\pgfqpoint{2.583521in}{0.829585in}}%
\pgfpathlineto{\pgfqpoint{2.591252in}{0.828084in}}%
\pgfpathlineto{\pgfqpoint{2.593830in}{0.823956in}}%
\pgfpathlineto{\pgfqpoint{2.596407in}{0.822455in}}%
\pgfpathlineto{\pgfqpoint{2.598984in}{0.832587in}}%
\pgfpathlineto{\pgfqpoint{2.601561in}{0.834839in}}%
\pgfpathlineto{\pgfqpoint{2.609293in}{0.834651in}}%
\pgfpathlineto{\pgfqpoint{2.611870in}{0.831649in}}%
\pgfpathlineto{\pgfqpoint{2.614447in}{0.834276in}}%
\pgfpathlineto{\pgfqpoint{2.617024in}{0.839154in}}%
\pgfpathlineto{\pgfqpoint{2.619602in}{0.852101in}}%
\pgfpathlineto{\pgfqpoint{2.627333in}{0.860920in}}%
\pgfpathlineto{\pgfqpoint{2.629911in}{0.859231in}}%
\pgfpathlineto{\pgfqpoint{2.632488in}{0.854353in}}%
\pgfpathlineto{\pgfqpoint{2.635065in}{0.857918in}}%
\pgfpathlineto{\pgfqpoint{2.637642in}{0.856980in}}%
\pgfpathlineto{\pgfqpoint{2.645374in}{0.860169in}}%
\pgfpathlineto{\pgfqpoint{2.650528in}{0.864297in}}%
\pgfpathlineto{\pgfqpoint{2.653105in}{0.860920in}}%
\pgfpathlineto{\pgfqpoint{2.655683in}{0.866924in}}%
\pgfpathlineto{\pgfqpoint{2.665991in}{0.865048in}}%
\pgfpathlineto{\pgfqpoint{2.668569in}{0.872366in}}%
\pgfpathlineto{\pgfqpoint{2.671146in}{0.867862in}}%
\pgfpathlineto{\pgfqpoint{2.673723in}{0.874617in}}%
\pgfpathlineto{\pgfqpoint{2.681455in}{0.874054in}}%
\pgfpathlineto{\pgfqpoint{2.684032in}{0.874617in}}%
\pgfpathlineto{\pgfqpoint{2.686609in}{0.876118in}}%
\pgfpathlineto{\pgfqpoint{2.689186in}{0.873679in}}%
\pgfpathlineto{\pgfqpoint{2.691764in}{0.876869in}}%
\pgfpathlineto{\pgfqpoint{2.699495in}{0.877056in}}%
\pgfpathlineto{\pgfqpoint{2.702072in}{0.874242in}}%
\pgfpathlineto{\pgfqpoint{2.707227in}{0.872553in}}%
\pgfpathlineto{\pgfqpoint{2.709804in}{0.874992in}}%
\pgfpathlineto{\pgfqpoint{2.717536in}{0.869739in}}%
\pgfpathlineto{\pgfqpoint{2.722690in}{0.871240in}}%
\pgfpathlineto{\pgfqpoint{2.727845in}{0.868801in}}%
\pgfpathlineto{\pgfqpoint{2.735576in}{0.868801in}}%
\pgfpathlineto{\pgfqpoint{2.738153in}{0.867299in}}%
\pgfpathlineto{\pgfqpoint{2.740731in}{0.868425in}}%
\pgfpathlineto{\pgfqpoint{2.743308in}{0.865611in}}%
\pgfpathlineto{\pgfqpoint{2.745885in}{0.872928in}}%
\pgfpathlineto{\pgfqpoint{2.753617in}{0.876681in}}%
\pgfpathlineto{\pgfqpoint{2.756194in}{0.875931in}}%
\pgfpathlineto{\pgfqpoint{2.758771in}{0.867487in}}%
\pgfpathlineto{\pgfqpoint{2.761348in}{0.866924in}}%
\pgfpathlineto{\pgfqpoint{2.763926in}{0.871427in}}%
\pgfpathlineto{\pgfqpoint{2.774234in}{0.874242in}}%
\pgfpathlineto{\pgfqpoint{2.776812in}{0.879871in}}%
\pgfpathlineto{\pgfqpoint{2.779389in}{0.881935in}}%
\pgfpathlineto{\pgfqpoint{2.781966in}{0.882498in}}%
\pgfpathlineto{\pgfqpoint{2.789698in}{0.883061in}}%
\pgfpathlineto{\pgfqpoint{2.792275in}{0.886250in}}%
\pgfpathlineto{\pgfqpoint{2.797429in}{0.889628in}}%
\pgfpathlineto{\pgfqpoint{2.800007in}{0.889628in}}%
\pgfpathlineto{\pgfqpoint{2.807738in}{0.890566in}}%
\pgfpathlineto{\pgfqpoint{2.810315in}{0.892630in}}%
\pgfpathlineto{\pgfqpoint{2.815470in}{0.884562in}}%
\pgfpathlineto{\pgfqpoint{2.818047in}{0.884374in}}%
\pgfpathlineto{\pgfqpoint{2.825779in}{0.880997in}}%
\pgfpathlineto{\pgfqpoint{2.828356in}{0.881372in}}%
\pgfpathlineto{\pgfqpoint{2.830933in}{0.880246in}}%
\pgfpathlineto{\pgfqpoint{2.833510in}{0.880434in}}%
\pgfpathlineto{\pgfqpoint{2.836088in}{0.876493in}}%
\pgfpathlineto{\pgfqpoint{2.843819in}{0.872928in}}%
\pgfpathlineto{\pgfqpoint{2.846396in}{0.876681in}}%
\pgfpathlineto{\pgfqpoint{2.848974in}{0.882310in}}%
\pgfpathlineto{\pgfqpoint{2.851551in}{0.880059in}}%
\pgfpathlineto{\pgfqpoint{2.854128in}{0.870677in}}%
\pgfpathlineto{\pgfqpoint{2.864437in}{0.866361in}}%
\pgfpathlineto{\pgfqpoint{2.867014in}{0.861858in}}%
\pgfpathlineto{\pgfqpoint{2.869591in}{0.859231in}}%
\pgfpathlineto{\pgfqpoint{2.872168in}{0.846097in}}%
\pgfpathlineto{\pgfqpoint{2.879900in}{0.847973in}}%
\pgfpathlineto{\pgfqpoint{2.882477in}{0.853790in}}%
\pgfpathlineto{\pgfqpoint{2.885055in}{0.851351in}}%
\pgfpathlineto{\pgfqpoint{2.887632in}{0.854353in}}%
\pgfpathlineto{\pgfqpoint{2.890209in}{0.848911in}}%
\pgfpathlineto{\pgfqpoint{2.897941in}{0.837090in}}%
\pgfpathlineto{\pgfqpoint{2.900518in}{0.840468in}}%
\pgfpathlineto{\pgfqpoint{2.903095in}{0.839717in}}%
\pgfpathlineto{\pgfqpoint{2.905672in}{0.846097in}}%
\pgfpathlineto{\pgfqpoint{2.908249in}{0.849850in}}%
\pgfpathlineto{\pgfqpoint{2.915981in}{0.847598in}}%
\pgfpathlineto{\pgfqpoint{2.918558in}{0.853415in}}%
\pgfpathlineto{\pgfqpoint{2.921136in}{0.852852in}}%
\pgfpathlineto{\pgfqpoint{2.923713in}{0.853602in}}%
\pgfpathlineto{\pgfqpoint{2.926290in}{0.858105in}}%
\pgfpathlineto{\pgfqpoint{2.936599in}{0.856792in}}%
\pgfpathlineto{\pgfqpoint{2.939176in}{0.852852in}}%
\pgfpathlineto{\pgfqpoint{2.941753in}{0.852101in}}%
\pgfpathlineto{\pgfqpoint{2.944330in}{0.849287in}}%
\pgfpathlineto{\pgfqpoint{2.952062in}{0.854540in}}%
\pgfpathlineto{\pgfqpoint{2.954639in}{0.854353in}}%
\pgfpathlineto{\pgfqpoint{2.957216in}{0.854728in}}%
\pgfpathlineto{\pgfqpoint{2.959794in}{0.857730in}}%
\pgfpathlineto{\pgfqpoint{2.962371in}{0.857355in}}%
\pgfpathlineto{\pgfqpoint{2.970103in}{0.852101in}}%
\pgfpathlineto{\pgfqpoint{2.972680in}{0.860169in}}%
\pgfpathlineto{\pgfqpoint{2.977834in}{0.868801in}}%
\pgfpathlineto{\pgfqpoint{2.980411in}{0.867487in}}%
\pgfpathlineto{\pgfqpoint{2.988143in}{0.865986in}}%
\pgfpathlineto{\pgfqpoint{2.993297in}{0.861858in}}%
\pgfpathlineto{\pgfqpoint{2.995875in}{0.855291in}}%
\pgfpathlineto{\pgfqpoint{2.998452in}{0.851913in}}%
\pgfpathlineto{\pgfqpoint{3.006184in}{0.856792in}}%
\pgfpathlineto{\pgfqpoint{3.008761in}{0.860169in}}%
\pgfpathlineto{\pgfqpoint{3.011338in}{0.854540in}}%
\pgfpathlineto{\pgfqpoint{3.013915in}{0.854353in}}%
\pgfpathlineto{\pgfqpoint{3.016492in}{0.856229in}}%
\pgfpathlineto{\pgfqpoint{3.024224in}{0.856417in}}%
\pgfpathlineto{\pgfqpoint{3.026801in}{0.860920in}}%
\pgfpathlineto{\pgfqpoint{3.029378in}{0.859606in}}%
\pgfpathlineto{\pgfqpoint{3.031956in}{0.862609in}}%
\pgfpathlineto{\pgfqpoint{3.034533in}{0.863547in}}%
\pgfpathlineto{\pgfqpoint{3.044842in}{0.863359in}}%
\pgfpathlineto{\pgfqpoint{3.049996in}{0.868988in}}%
\pgfpathlineto{\pgfqpoint{3.052573in}{0.865798in}}%
\pgfpathlineto{\pgfqpoint{3.060305in}{0.863172in}}%
\pgfpathlineto{\pgfqpoint{3.062882in}{0.861670in}}%
\pgfpathlineto{\pgfqpoint{3.065459in}{0.864673in}}%
\pgfpathlineto{\pgfqpoint{3.068037in}{0.859044in}}%
\pgfpathlineto{\pgfqpoint{3.070614in}{0.856792in}}%
\pgfpathlineto{\pgfqpoint{3.080923in}{0.862609in}}%
\pgfpathlineto{\pgfqpoint{3.083500in}{0.867299in}}%
\pgfpathlineto{\pgfqpoint{3.086077in}{0.873867in}}%
\pgfpathlineto{\pgfqpoint{3.098963in}{0.874242in}}%
\pgfpathlineto{\pgfqpoint{3.101540in}{0.871803in}}%
\pgfpathlineto{\pgfqpoint{3.104118in}{0.872366in}}%
\pgfpathlineto{\pgfqpoint{3.106695in}{0.874617in}}%
\pgfpathlineto{\pgfqpoint{3.114426in}{0.877244in}}%
\pgfpathlineto{\pgfqpoint{3.117004in}{0.877056in}}%
\pgfpathlineto{\pgfqpoint{3.119581in}{0.878933in}}%
\pgfpathlineto{\pgfqpoint{3.124735in}{0.875743in}}%
\pgfpathlineto{\pgfqpoint{3.132467in}{0.874242in}}%
\pgfpathlineto{\pgfqpoint{3.135044in}{0.868238in}}%
\pgfpathlineto{\pgfqpoint{3.137621in}{0.873491in}}%
\pgfpathlineto{\pgfqpoint{3.140199in}{0.872178in}}%
\pgfpathlineto{\pgfqpoint{3.142776in}{0.871803in}}%
\pgfpathlineto{\pgfqpoint{3.150507in}{0.878370in}}%
\pgfpathlineto{\pgfqpoint{3.153085in}{0.879496in}}%
\pgfpathlineto{\pgfqpoint{3.158239in}{0.874617in}}%
\pgfpathlineto{\pgfqpoint{3.160816in}{0.875555in}}%
\pgfpathlineto{\pgfqpoint{3.168548in}{0.874617in}}%
\pgfpathlineto{\pgfqpoint{3.171125in}{0.869926in}}%
\pgfpathlineto{\pgfqpoint{3.173702in}{0.872928in}}%
\pgfpathlineto{\pgfqpoint{3.189166in}{0.874054in}}%
\pgfpathlineto{\pgfqpoint{3.194320in}{0.876681in}}%
\pgfpathlineto{\pgfqpoint{3.196897in}{0.877432in}}%
\pgfpathlineto{\pgfqpoint{3.204629in}{0.877995in}}%
\pgfpathlineto{\pgfqpoint{3.207206in}{0.877432in}}%
\pgfpathlineto{\pgfqpoint{3.209783in}{0.873867in}}%
\pgfpathlineto{\pgfqpoint{3.212361in}{0.877056in}}%
\pgfpathlineto{\pgfqpoint{3.214938in}{0.883436in}}%
\pgfpathlineto{\pgfqpoint{3.222669in}{0.887376in}}%
\pgfpathlineto{\pgfqpoint{3.225247in}{0.886813in}}%
\pgfpathlineto{\pgfqpoint{3.227824in}{0.882873in}}%
\pgfpathlineto{\pgfqpoint{3.230401in}{0.880059in}}%
\pgfpathlineto{\pgfqpoint{3.232978in}{0.881184in}}%
\pgfpathlineto{\pgfqpoint{3.240710in}{0.877995in}}%
\pgfpathlineto{\pgfqpoint{3.245864in}{0.878933in}}%
\pgfpathlineto{\pgfqpoint{3.248441in}{0.882873in}}%
\pgfpathlineto{\pgfqpoint{3.251019in}{0.883624in}}%
\pgfpathlineto{\pgfqpoint{3.261328in}{0.877619in}}%
\pgfpathlineto{\pgfqpoint{3.266482in}{0.873116in}}%
\pgfpathlineto{\pgfqpoint{3.269059in}{0.875368in}}%
\pgfpathlineto{\pgfqpoint{3.276791in}{0.872928in}}%
\pgfpathlineto{\pgfqpoint{3.279368in}{0.874805in}}%
\pgfpathlineto{\pgfqpoint{3.284522in}{0.881935in}}%
\pgfpathlineto{\pgfqpoint{3.294831in}{0.880246in}}%
\pgfpathlineto{\pgfqpoint{3.297409in}{0.874430in}}%
\pgfpathlineto{\pgfqpoint{3.299986in}{0.873679in}}%
\pgfpathlineto{\pgfqpoint{3.302563in}{0.871803in}}%
\pgfpathlineto{\pgfqpoint{3.305140in}{0.877056in}}%
\pgfpathlineto{\pgfqpoint{3.312872in}{0.878745in}}%
\pgfpathlineto{\pgfqpoint{3.315449in}{0.877995in}}%
\pgfpathlineto{\pgfqpoint{3.318026in}{0.884374in}}%
\pgfpathlineto{\pgfqpoint{3.320603in}{0.877995in}}%
\pgfpathlineto{\pgfqpoint{3.330912in}{0.868425in}}%
\pgfpathlineto{\pgfqpoint{3.333490in}{0.868988in}}%
\pgfpathlineto{\pgfqpoint{3.336067in}{0.867299in}}%
\pgfpathlineto{\pgfqpoint{3.338644in}{0.867675in}}%
\pgfpathlineto{\pgfqpoint{3.341221in}{0.865423in}}%
\pgfpathlineto{\pgfqpoint{3.348953in}{0.862421in}}%
\pgfpathlineto{\pgfqpoint{3.351530in}{0.860169in}}%
\pgfpathlineto{\pgfqpoint{3.354107in}{0.863172in}}%
\pgfpathlineto{\pgfqpoint{3.356684in}{0.855666in}}%
\pgfpathlineto{\pgfqpoint{3.359262in}{0.858668in}}%
\pgfpathlineto{\pgfqpoint{3.366993in}{0.857542in}}%
\pgfpathlineto{\pgfqpoint{3.369570in}{0.853602in}}%
\pgfpathlineto{\pgfqpoint{3.372148in}{0.860169in}}%
\pgfpathlineto{\pgfqpoint{3.374725in}{0.860920in}}%
\pgfpathlineto{\pgfqpoint{3.377302in}{0.863547in}}%
\pgfpathlineto{\pgfqpoint{3.385034in}{0.865423in}}%
\pgfpathlineto{\pgfqpoint{3.387611in}{0.862609in}}%
\pgfpathlineto{\pgfqpoint{3.390188in}{0.865986in}}%
\pgfpathlineto{\pgfqpoint{3.392765in}{0.866924in}}%
\pgfpathlineto{\pgfqpoint{3.395343in}{0.863172in}}%
\pgfpathlineto{\pgfqpoint{3.403074in}{0.869739in}}%
\pgfpathlineto{\pgfqpoint{3.405651in}{0.869363in}}%
\pgfpathlineto{\pgfqpoint{3.408229in}{0.874242in}}%
\pgfpathlineto{\pgfqpoint{3.410806in}{0.875368in}}%
\pgfpathlineto{\pgfqpoint{3.413383in}{0.871052in}}%
\pgfpathlineto{\pgfqpoint{3.421115in}{0.871803in}}%
\pgfpathlineto{\pgfqpoint{3.423692in}{0.868801in}}%
\pgfpathlineto{\pgfqpoint{3.426269in}{0.870677in}}%
\pgfpathlineto{\pgfqpoint{3.428846in}{0.868801in}}%
\pgfpathlineto{\pgfqpoint{3.431424in}{0.868425in}}%
\pgfpathlineto{\pgfqpoint{3.441732in}{0.866361in}}%
\pgfpathlineto{\pgfqpoint{3.444310in}{0.867862in}}%
\pgfpathlineto{\pgfqpoint{3.446887in}{0.868050in}}%
\pgfpathlineto{\pgfqpoint{3.449464in}{0.870302in}}%
\pgfpathlineto{\pgfqpoint{3.457196in}{0.869926in}}%
\pgfpathlineto{\pgfqpoint{3.459773in}{0.867112in}}%
\pgfpathlineto{\pgfqpoint{3.464927in}{0.868988in}}%
\pgfpathlineto{\pgfqpoint{3.467505in}{0.866737in}}%
\pgfpathlineto{\pgfqpoint{3.475236in}{0.867487in}}%
\pgfpathlineto{\pgfqpoint{3.477813in}{0.871990in}}%
\pgfpathlineto{\pgfqpoint{3.480391in}{0.872741in}}%
\pgfpathlineto{\pgfqpoint{3.482968in}{0.875368in}}%
\pgfpathlineto{\pgfqpoint{3.485545in}{0.876493in}}%
\pgfpathlineto{\pgfqpoint{3.498431in}{0.871052in}}%
\pgfpathlineto{\pgfqpoint{3.501008in}{0.865048in}}%
\pgfpathlineto{\pgfqpoint{3.503586in}{0.866361in}}%
\pgfpathlineto{\pgfqpoint{3.511317in}{0.862984in}}%
\pgfpathlineto{\pgfqpoint{3.513894in}{0.866174in}}%
\pgfpathlineto{\pgfqpoint{3.516472in}{0.859044in}}%
\pgfpathlineto{\pgfqpoint{3.519049in}{0.858481in}}%
\pgfpathlineto{\pgfqpoint{3.521626in}{0.862796in}}%
\pgfpathlineto{\pgfqpoint{3.529358in}{0.859982in}}%
\pgfpathlineto{\pgfqpoint{3.531935in}{0.853602in}}%
\pgfpathlineto{\pgfqpoint{3.534512in}{0.860357in}}%
\pgfpathlineto{\pgfqpoint{3.539667in}{0.845346in}}%
\pgfpathlineto{\pgfqpoint{3.547398in}{0.840280in}}%
\pgfpathlineto{\pgfqpoint{3.552553in}{0.845346in}}%
\pgfpathlineto{\pgfqpoint{3.555130in}{0.844971in}}%
\pgfpathlineto{\pgfqpoint{3.557707in}{0.853790in}}%
\pgfpathlineto{\pgfqpoint{3.565439in}{0.856980in}}%
\pgfpathlineto{\pgfqpoint{3.568016in}{0.863547in}}%
\pgfpathlineto{\pgfqpoint{3.570593in}{0.859606in}}%
\pgfpathlineto{\pgfqpoint{3.575747in}{0.866549in}}%
\pgfpathlineto{\pgfqpoint{3.583479in}{0.864673in}}%
\pgfpathlineto{\pgfqpoint{3.586056in}{0.870114in}}%
\pgfpathlineto{\pgfqpoint{3.588634in}{0.866737in}}%
\pgfpathlineto{\pgfqpoint{3.591211in}{0.866924in}}%
\pgfpathlineto{\pgfqpoint{3.593788in}{0.869176in}}%
\pgfpathlineto{\pgfqpoint{3.604097in}{0.867487in}}%
\pgfpathlineto{\pgfqpoint{3.606674in}{0.869363in}}%
\pgfpathlineto{\pgfqpoint{3.609251in}{0.877619in}}%
\pgfpathlineto{\pgfqpoint{3.611828in}{0.878370in}}%
\pgfpathlineto{\pgfqpoint{3.619560in}{0.879308in}}%
\pgfpathlineto{\pgfqpoint{3.622137in}{0.877995in}}%
\pgfpathlineto{\pgfqpoint{3.624715in}{0.880059in}}%
\pgfpathlineto{\pgfqpoint{3.627292in}{0.878557in}}%
\pgfpathlineto{\pgfqpoint{3.629869in}{0.879120in}}%
\pgfpathlineto{\pgfqpoint{3.637601in}{0.881560in}}%
\pgfpathlineto{\pgfqpoint{3.640178in}{0.887752in}}%
\pgfpathlineto{\pgfqpoint{3.645332in}{0.885125in}}%
\pgfpathlineto{\pgfqpoint{3.647909in}{0.887376in}}%
\pgfpathlineto{\pgfqpoint{3.655641in}{0.887564in}}%
\pgfpathlineto{\pgfqpoint{3.658218in}{0.885312in}}%
\pgfpathlineto{\pgfqpoint{3.660795in}{0.885500in}}%
\pgfpathlineto{\pgfqpoint{3.665950in}{0.879683in}}%
\pgfpathlineto{\pgfqpoint{3.673682in}{0.872366in}}%
\pgfpathlineto{\pgfqpoint{3.676259in}{0.872741in}}%
\pgfpathlineto{\pgfqpoint{3.678836in}{0.877995in}}%
\pgfpathlineto{\pgfqpoint{3.681413in}{0.873491in}}%
\pgfpathlineto{\pgfqpoint{3.683990in}{0.872178in}}%
\pgfpathlineto{\pgfqpoint{3.694299in}{0.865611in}}%
\pgfpathlineto{\pgfqpoint{3.696876in}{0.860732in}}%
\pgfpathlineto{\pgfqpoint{3.699454in}{0.862984in}}%
\pgfpathlineto{\pgfqpoint{3.702031in}{0.854916in}}%
\pgfpathlineto{\pgfqpoint{3.712340in}{0.848724in}}%
\pgfpathlineto{\pgfqpoint{3.714917in}{0.851351in}}%
\pgfpathlineto{\pgfqpoint{3.717494in}{0.862421in}}%
\pgfpathlineto{\pgfqpoint{3.720071in}{0.869926in}}%
\pgfpathlineto{\pgfqpoint{3.727803in}{0.871240in}}%
\pgfpathlineto{\pgfqpoint{3.730380in}{0.874054in}}%
\pgfpathlineto{\pgfqpoint{3.732957in}{0.873116in}}%
\pgfpathlineto{\pgfqpoint{3.745843in}{0.871240in}}%
\pgfpathlineto{\pgfqpoint{3.748421in}{0.869176in}}%
\pgfpathlineto{\pgfqpoint{3.750998in}{0.864485in}}%
\pgfpathlineto{\pgfqpoint{3.756152in}{0.861108in}}%
\pgfpathlineto{\pgfqpoint{3.763884in}{0.853977in}}%
\pgfpathlineto{\pgfqpoint{3.766461in}{0.845722in}}%
\pgfpathlineto{\pgfqpoint{3.769038in}{0.845909in}}%
\pgfpathlineto{\pgfqpoint{3.771616in}{0.850412in}}%
\pgfpathlineto{\pgfqpoint{3.774193in}{0.844971in}}%
\pgfpathlineto{\pgfqpoint{3.781924in}{0.844221in}}%
\pgfpathlineto{\pgfqpoint{3.784502in}{0.842344in}}%
\pgfpathlineto{\pgfqpoint{3.787079in}{0.841218in}}%
\pgfpathlineto{\pgfqpoint{3.789656in}{0.838029in}}%
\pgfpathlineto{\pgfqpoint{3.792233in}{0.838216in}}%
\pgfpathlineto{\pgfqpoint{3.802542in}{0.842157in}}%
\pgfpathlineto{\pgfqpoint{3.807697in}{0.848911in}}%
\pgfpathlineto{\pgfqpoint{3.810274in}{0.852101in}}%
\pgfpathlineto{\pgfqpoint{3.818005in}{0.853790in}}%
\pgfpathlineto{\pgfqpoint{3.820583in}{0.850600in}}%
\pgfpathlineto{\pgfqpoint{3.823160in}{0.842157in}}%
\pgfpathlineto{\pgfqpoint{3.825737in}{0.845909in}}%
\pgfpathlineto{\pgfqpoint{3.828314in}{0.842907in}}%
\pgfpathlineto{\pgfqpoint{3.836046in}{0.847786in}}%
\pgfpathlineto{\pgfqpoint{3.838623in}{0.851913in}}%
\pgfpathlineto{\pgfqpoint{3.841200in}{0.847035in}}%
\pgfpathlineto{\pgfqpoint{3.843778in}{0.852476in}}%
\pgfpathlineto{\pgfqpoint{3.846355in}{0.852664in}}%
\pgfpathlineto{\pgfqpoint{3.854086in}{0.854540in}}%
\pgfpathlineto{\pgfqpoint{3.861818in}{0.858481in}}%
\pgfpathlineto{\pgfqpoint{3.864395in}{0.862609in}}%
\pgfpathlineto{\pgfqpoint{3.874704in}{0.862796in}}%
\pgfpathlineto{\pgfqpoint{3.877281in}{0.864110in}}%
\pgfpathlineto{\pgfqpoint{3.879859in}{0.863922in}}%
\pgfpathlineto{\pgfqpoint{3.882436in}{0.867112in}}%
\pgfpathlineto{\pgfqpoint{3.890167in}{0.866549in}}%
\pgfpathlineto{\pgfqpoint{3.892745in}{0.869926in}}%
\pgfpathlineto{\pgfqpoint{3.895322in}{0.878182in}}%
\pgfpathlineto{\pgfqpoint{3.897899in}{0.877807in}}%
\pgfpathlineto{\pgfqpoint{3.900476in}{0.879496in}}%
\pgfpathlineto{\pgfqpoint{3.908208in}{0.881372in}}%
\pgfpathlineto{\pgfqpoint{3.913362in}{0.874242in}}%
\pgfpathlineto{\pgfqpoint{3.915940in}{0.876681in}}%
\pgfpathlineto{\pgfqpoint{3.918517in}{0.870489in}}%
\pgfpathlineto{\pgfqpoint{3.926248in}{0.873867in}}%
\pgfpathlineto{\pgfqpoint{3.928826in}{0.866549in}}%
\pgfpathlineto{\pgfqpoint{3.931403in}{0.866737in}}%
\pgfpathlineto{\pgfqpoint{3.933980in}{0.870114in}}%
\pgfpathlineto{\pgfqpoint{3.936557in}{0.864485in}}%
\pgfpathlineto{\pgfqpoint{3.944289in}{0.870864in}}%
\pgfpathlineto{\pgfqpoint{3.946866in}{0.868613in}}%
\pgfpathlineto{\pgfqpoint{3.949443in}{0.873867in}}%
\pgfpathlineto{\pgfqpoint{3.952020in}{0.868988in}}%
\pgfpathlineto{\pgfqpoint{3.954598in}{0.870114in}}%
\pgfpathlineto{\pgfqpoint{3.962329in}{0.871240in}}%
\pgfpathlineto{\pgfqpoint{3.964907in}{0.868050in}}%
\pgfpathlineto{\pgfqpoint{3.967484in}{0.862421in}}%
\pgfpathlineto{\pgfqpoint{3.970061in}{0.860732in}}%
\pgfpathlineto{\pgfqpoint{3.972638in}{0.861670in}}%
\pgfpathlineto{\pgfqpoint{3.980370in}{0.865611in}}%
\pgfpathlineto{\pgfqpoint{3.982947in}{0.860732in}}%
\pgfpathlineto{\pgfqpoint{3.985524in}{0.861295in}}%
\pgfpathlineto{\pgfqpoint{3.988101in}{0.862796in}}%
\pgfpathlineto{\pgfqpoint{3.998410in}{0.866737in}}%
\pgfpathlineto{\pgfqpoint{4.000988in}{0.864110in}}%
\pgfpathlineto{\pgfqpoint{4.003565in}{0.863922in}}%
\pgfpathlineto{\pgfqpoint{4.006142in}{0.875368in}}%
\pgfpathlineto{\pgfqpoint{4.008719in}{0.919086in}}%
\pgfpathlineto{\pgfqpoint{4.016451in}{0.905201in}}%
\pgfpathlineto{\pgfqpoint{4.019028in}{0.906890in}}%
\pgfpathlineto{\pgfqpoint{4.021605in}{0.902575in}}%
\pgfpathlineto{\pgfqpoint{4.024182in}{0.899760in}}%
\pgfpathlineto{\pgfqpoint{4.026760in}{0.899385in}}%
\pgfpathlineto{\pgfqpoint{4.034491in}{0.895632in}}%
\pgfpathlineto{\pgfqpoint{4.037069in}{0.889440in}}%
\pgfpathlineto{\pgfqpoint{4.039646in}{0.893943in}}%
\pgfpathlineto{\pgfqpoint{4.044800in}{0.892255in}}%
\pgfpathlineto{\pgfqpoint{4.052532in}{0.893381in}}%
\pgfpathlineto{\pgfqpoint{4.055109in}{0.897321in}}%
\pgfpathlineto{\pgfqpoint{4.060263in}{0.896570in}}%
\pgfpathlineto{\pgfqpoint{4.062841in}{0.900323in}}%
\pgfpathlineto{\pgfqpoint{4.070572in}{0.899572in}}%
\pgfpathlineto{\pgfqpoint{4.073149in}{0.894131in}}%
\pgfpathlineto{\pgfqpoint{4.075727in}{0.892442in}}%
\pgfpathlineto{\pgfqpoint{4.078304in}{0.896007in}}%
\pgfpathlineto{\pgfqpoint{4.080881in}{0.901073in}}%
\pgfpathlineto{\pgfqpoint{4.088613in}{0.894131in}}%
\pgfpathlineto{\pgfqpoint{4.091190in}{0.895820in}}%
\pgfpathlineto{\pgfqpoint{4.096344in}{0.901824in}}%
\pgfpathlineto{\pgfqpoint{4.098922in}{0.899572in}}%
\pgfpathlineto{\pgfqpoint{4.109230in}{0.900886in}}%
\pgfpathlineto{\pgfqpoint{4.111808in}{0.905389in}}%
\pgfpathlineto{\pgfqpoint{4.114385in}{0.906702in}}%
\pgfpathlineto{\pgfqpoint{4.129848in}{0.903513in}}%
\pgfpathlineto{\pgfqpoint{4.132425in}{0.905201in}}%
\pgfpathlineto{\pgfqpoint{4.135003in}{0.899572in}}%
\pgfpathlineto{\pgfqpoint{4.142734in}{0.899760in}}%
\pgfpathlineto{\pgfqpoint{4.145311in}{0.900511in}}%
\pgfpathlineto{\pgfqpoint{4.147889in}{0.903700in}}%
\pgfpathlineto{\pgfqpoint{4.150466in}{0.899385in}}%
\pgfpathlineto{\pgfqpoint{4.153043in}{0.899948in}}%
\pgfpathlineto{\pgfqpoint{4.160775in}{0.899197in}}%
\pgfpathlineto{\pgfqpoint{4.163352in}{0.900511in}}%
\pgfpathlineto{\pgfqpoint{4.165929in}{0.905201in}}%
\pgfpathlineto{\pgfqpoint{4.171084in}{0.901449in}}%
\pgfpathlineto{\pgfqpoint{4.178815in}{0.898634in}}%
\pgfpathlineto{\pgfqpoint{4.181392in}{0.898822in}}%
\pgfpathlineto{\pgfqpoint{4.183970in}{0.899572in}}%
\pgfpathlineto{\pgfqpoint{4.186547in}{0.904826in}}%
\pgfpathlineto{\pgfqpoint{4.189124in}{0.902762in}}%
\pgfpathlineto{\pgfqpoint{4.196856in}{0.905577in}}%
\pgfpathlineto{\pgfqpoint{4.199433in}{0.907641in}}%
\pgfpathlineto{\pgfqpoint{4.204587in}{0.899572in}}%
\pgfpathlineto{\pgfqpoint{4.207165in}{0.900323in}}%
\pgfpathlineto{\pgfqpoint{4.214896in}{0.893193in}}%
\pgfpathlineto{\pgfqpoint{4.217473in}{0.892067in}}%
\pgfpathlineto{\pgfqpoint{4.222628in}{0.895444in}}%
\pgfpathlineto{\pgfqpoint{4.232937in}{0.887939in}}%
\pgfpathlineto{\pgfqpoint{4.235514in}{0.890566in}}%
\pgfpathlineto{\pgfqpoint{4.238091in}{0.881372in}}%
\pgfpathlineto{\pgfqpoint{4.240668in}{0.883436in}}%
\pgfpathlineto{\pgfqpoint{4.243246in}{0.887376in}}%
\pgfpathlineto{\pgfqpoint{4.250977in}{0.890566in}}%
\pgfpathlineto{\pgfqpoint{4.253554in}{0.893568in}}%
\pgfpathlineto{\pgfqpoint{4.256132in}{0.895257in}}%
\pgfpathlineto{\pgfqpoint{4.258709in}{0.899572in}}%
\pgfpathlineto{\pgfqpoint{4.261286in}{0.902762in}}%
\pgfpathlineto{\pgfqpoint{4.269018in}{0.901261in}}%
\pgfpathlineto{\pgfqpoint{4.276749in}{0.887189in}}%
\pgfpathlineto{\pgfqpoint{4.279326in}{0.879120in}}%
\pgfpathlineto{\pgfqpoint{4.287058in}{0.882310in}}%
\pgfpathlineto{\pgfqpoint{4.292213in}{0.887189in}}%
\pgfpathlineto{\pgfqpoint{4.294790in}{0.884937in}}%
\pgfpathlineto{\pgfqpoint{4.297367in}{0.884749in}}%
\pgfpathlineto{\pgfqpoint{4.305099in}{0.880997in}}%
\pgfpathlineto{\pgfqpoint{4.307676in}{0.881560in}}%
\pgfpathlineto{\pgfqpoint{4.310253in}{0.884749in}}%
\pgfpathlineto{\pgfqpoint{4.312830in}{0.883624in}}%
\pgfpathlineto{\pgfqpoint{4.315407in}{0.879683in}}%
\pgfpathlineto{\pgfqpoint{4.323139in}{0.886813in}}%
\pgfpathlineto{\pgfqpoint{4.325716in}{0.878370in}}%
\pgfpathlineto{\pgfqpoint{4.328294in}{0.880809in}}%
\pgfpathlineto{\pgfqpoint{4.330871in}{0.879683in}}%
\pgfpathlineto{\pgfqpoint{4.333448in}{0.884374in}}%
\pgfpathlineto{\pgfqpoint{4.341180in}{0.886438in}}%
\pgfpathlineto{\pgfqpoint{4.343757in}{0.884186in}}%
\pgfpathlineto{\pgfqpoint{4.346334in}{0.878745in}}%
\pgfpathlineto{\pgfqpoint{4.351488in}{0.860732in}}%
\pgfpathlineto{\pgfqpoint{4.359220in}{0.849099in}}%
\pgfpathlineto{\pgfqpoint{4.361797in}{0.839717in}}%
\pgfpathlineto{\pgfqpoint{4.364374in}{0.851351in}}%
\pgfpathlineto{\pgfqpoint{4.366952in}{0.867299in}}%
\pgfpathlineto{\pgfqpoint{4.369529in}{0.869739in}}%
\pgfpathlineto{\pgfqpoint{4.377261in}{0.864297in}}%
\pgfpathlineto{\pgfqpoint{4.379838in}{0.849287in}}%
\pgfpathlineto{\pgfqpoint{4.382415in}{0.860357in}}%
\pgfpathlineto{\pgfqpoint{4.384992in}{0.859419in}}%
\pgfpathlineto{\pgfqpoint{4.387569in}{0.851351in}}%
\pgfpathlineto{\pgfqpoint{4.397878in}{0.866549in}}%
\pgfpathlineto{\pgfqpoint{4.400455in}{0.859982in}}%
\pgfpathlineto{\pgfqpoint{4.403033in}{0.862046in}}%
\pgfpathlineto{\pgfqpoint{4.405610in}{0.866361in}}%
\pgfpathlineto{\pgfqpoint{4.413342in}{0.863547in}}%
\pgfpathlineto{\pgfqpoint{4.418496in}{0.881935in}}%
\pgfpathlineto{\pgfqpoint{4.421073in}{0.876306in}}%
\pgfpathlineto{\pgfqpoint{4.423650in}{0.867487in}}%
\pgfpathlineto{\pgfqpoint{4.431382in}{0.872178in}}%
\pgfpathlineto{\pgfqpoint{4.436536in}{0.872928in}}%
\pgfpathlineto{\pgfqpoint{4.439114in}{0.869363in}}%
\pgfpathlineto{\pgfqpoint{4.441691in}{0.869363in}}%
\pgfpathlineto{\pgfqpoint{4.449422in}{0.859606in}}%
\pgfpathlineto{\pgfqpoint{4.452000in}{0.863734in}}%
\pgfpathlineto{\pgfqpoint{4.454577in}{0.874242in}}%
\pgfpathlineto{\pgfqpoint{4.457154in}{0.873679in}}%
\pgfpathlineto{\pgfqpoint{4.459731in}{0.878182in}}%
\pgfpathlineto{\pgfqpoint{4.472617in}{0.915146in}}%
\pgfpathlineto{\pgfqpoint{4.475195in}{0.919274in}}%
\pgfpathlineto{\pgfqpoint{4.477772in}{0.920024in}}%
\pgfpathlineto{\pgfqpoint{4.485503in}{0.920400in}}%
\pgfpathlineto{\pgfqpoint{4.490658in}{0.912519in}}%
\pgfpathlineto{\pgfqpoint{4.493235in}{0.919274in}}%
\pgfpathlineto{\pgfqpoint{4.495812in}{0.934660in}}%
\pgfpathlineto{\pgfqpoint{4.503544in}{0.934848in}}%
\pgfpathlineto{\pgfqpoint{4.506121in}{0.931470in}}%
\pgfpathlineto{\pgfqpoint{4.508698in}{0.932596in}}%
\pgfpathlineto{\pgfqpoint{4.511276in}{0.944229in}}%
\pgfpathlineto{\pgfqpoint{4.513853in}{0.943103in}}%
\pgfpathlineto{\pgfqpoint{4.521584in}{0.943666in}}%
\pgfpathlineto{\pgfqpoint{4.529316in}{0.940289in}}%
\pgfpathlineto{\pgfqpoint{4.531893in}{0.933722in}}%
\pgfpathlineto{\pgfqpoint{4.542202in}{0.944417in}}%
\pgfpathlineto{\pgfqpoint{4.544779in}{0.943666in}}%
\pgfpathlineto{\pgfqpoint{4.547357in}{0.945167in}}%
\pgfpathlineto{\pgfqpoint{4.549934in}{0.949671in}}%
\pgfpathlineto{\pgfqpoint{4.557665in}{0.947044in}}%
\pgfpathlineto{\pgfqpoint{4.560243in}{0.952860in}}%
\pgfpathlineto{\pgfqpoint{4.562820in}{0.961679in}}%
\pgfpathlineto{\pgfqpoint{4.565397in}{0.953611in}}%
\pgfpathlineto{\pgfqpoint{4.567974in}{0.955487in}}%
\pgfpathlineto{\pgfqpoint{4.575706in}{0.956801in}}%
\pgfpathlineto{\pgfqpoint{4.578283in}{0.956050in}}%
\pgfpathlineto{\pgfqpoint{4.580860in}{0.959240in}}%
\pgfpathlineto{\pgfqpoint{4.583438in}{0.955300in}}%
\pgfpathlineto{\pgfqpoint{4.586015in}{0.961491in}}%
\pgfpathlineto{\pgfqpoint{4.593746in}{0.960366in}}%
\pgfpathlineto{\pgfqpoint{4.596324in}{0.961491in}}%
\pgfpathlineto{\pgfqpoint{4.598901in}{0.956801in}}%
\pgfpathlineto{\pgfqpoint{4.604055in}{0.956801in}}%
\pgfpathlineto{\pgfqpoint{4.611787in}{0.950046in}}%
\pgfpathlineto{\pgfqpoint{4.614364in}{0.953611in}}%
\pgfpathlineto{\pgfqpoint{4.616941in}{0.950421in}}%
\pgfpathlineto{\pgfqpoint{4.619519in}{0.951359in}}%
\pgfpathlineto{\pgfqpoint{4.622096in}{0.958865in}}%
\pgfpathlineto{\pgfqpoint{4.629827in}{0.956988in}}%
\pgfpathlineto{\pgfqpoint{4.632405in}{0.953986in}}%
\pgfpathlineto{\pgfqpoint{4.634982in}{0.958489in}}%
\pgfpathlineto{\pgfqpoint{4.637559in}{0.961304in}}%
\pgfpathlineto{\pgfqpoint{4.640136in}{0.955112in}}%
\pgfpathlineto{\pgfqpoint{4.647868in}{0.955112in}}%
\pgfpathlineto{\pgfqpoint{4.650445in}{0.956050in}}%
\pgfpathlineto{\pgfqpoint{4.653022in}{0.966745in}}%
\pgfpathlineto{\pgfqpoint{4.655599in}{0.963368in}}%
\pgfpathlineto{\pgfqpoint{4.658177in}{0.959052in}}%
\pgfpathlineto{\pgfqpoint{4.665908in}{0.961116in}}%
\pgfpathlineto{\pgfqpoint{4.668486in}{0.962430in}}%
\pgfpathlineto{\pgfqpoint{4.671063in}{0.969935in}}%
\pgfpathlineto{\pgfqpoint{4.673640in}{0.968059in}}%
\pgfpathlineto{\pgfqpoint{4.683949in}{0.969184in}}%
\pgfpathlineto{\pgfqpoint{4.686526in}{0.975189in}}%
\pgfpathlineto{\pgfqpoint{4.689103in}{0.971624in}}%
\pgfpathlineto{\pgfqpoint{4.691680in}{0.973125in}}%
\pgfpathlineto{\pgfqpoint{4.701989in}{0.965995in}}%
\pgfpathlineto{\pgfqpoint{4.704567in}{0.966558in}}%
\pgfpathlineto{\pgfqpoint{4.707144in}{0.958677in}}%
\pgfpathlineto{\pgfqpoint{4.709721in}{0.937850in}}%
\pgfpathlineto{\pgfqpoint{4.712298in}{0.929594in}}%
\pgfpathlineto{\pgfqpoint{4.722607in}{0.932596in}}%
\pgfpathlineto{\pgfqpoint{4.725184in}{0.926029in}}%
\pgfpathlineto{\pgfqpoint{4.727761in}{0.939351in}}%
\pgfpathlineto{\pgfqpoint{4.730339in}{0.930157in}}%
\pgfpathlineto{\pgfqpoint{4.740648in}{0.930157in}}%
\pgfpathlineto{\pgfqpoint{4.743225in}{0.922276in}}%
\pgfpathlineto{\pgfqpoint{4.745802in}{0.931845in}}%
\pgfpathlineto{\pgfqpoint{4.748379in}{0.926029in}}%
\pgfpathlineto{\pgfqpoint{4.756111in}{0.922839in}}%
\pgfpathlineto{\pgfqpoint{4.758688in}{0.927155in}}%
\pgfpathlineto{\pgfqpoint{4.761265in}{0.922276in}}%
\pgfpathlineto{\pgfqpoint{4.763842in}{0.925653in}}%
\pgfpathlineto{\pgfqpoint{4.766420in}{0.940101in}}%
\pgfpathlineto{\pgfqpoint{4.774151in}{0.932596in}}%
\pgfpathlineto{\pgfqpoint{4.776728in}{0.926029in}}%
\pgfpathlineto{\pgfqpoint{4.781883in}{0.941227in}}%
\pgfpathlineto{\pgfqpoint{4.784460in}{0.930907in}}%
\pgfpathlineto{\pgfqpoint{4.792192in}{0.924903in}}%
\pgfpathlineto{\pgfqpoint{4.794769in}{0.926779in}}%
\pgfpathlineto{\pgfqpoint{4.797346in}{0.927155in}}%
\pgfpathlineto{\pgfqpoint{4.799923in}{0.913270in}}%
\pgfpathlineto{\pgfqpoint{4.802501in}{0.926404in}}%
\pgfpathlineto{\pgfqpoint{4.812809in}{0.936161in}}%
\pgfpathlineto{\pgfqpoint{4.815387in}{0.943854in}}%
\pgfpathlineto{\pgfqpoint{4.817964in}{0.939726in}}%
\pgfpathlineto{\pgfqpoint{4.820541in}{0.938788in}}%
\pgfpathlineto{\pgfqpoint{4.828273in}{0.944980in}}%
\pgfpathlineto{\pgfqpoint{4.830850in}{0.941978in}}%
\pgfpathlineto{\pgfqpoint{4.833427in}{0.937662in}}%
\pgfpathlineto{\pgfqpoint{4.836004in}{0.945918in}}%
\pgfpathlineto{\pgfqpoint{4.838582in}{0.948732in}}%
\pgfpathlineto{\pgfqpoint{4.846313in}{0.944417in}}%
\pgfpathlineto{\pgfqpoint{4.848890in}{0.956425in}}%
\pgfpathlineto{\pgfqpoint{4.851468in}{0.961304in}}%
\pgfpathlineto{\pgfqpoint{4.854045in}{0.962054in}}%
\pgfpathlineto{\pgfqpoint{4.856622in}{0.965995in}}%
\pgfpathlineto{\pgfqpoint{4.864354in}{0.963180in}}%
\pgfpathlineto{\pgfqpoint{4.866931in}{0.959428in}}%
\pgfpathlineto{\pgfqpoint{4.869508in}{0.959240in}}%
\pgfpathlineto{\pgfqpoint{4.872085in}{0.957551in}}%
\pgfpathlineto{\pgfqpoint{4.874663in}{0.963931in}}%
\pgfpathlineto{\pgfqpoint{4.882394in}{0.962805in}}%
\pgfpathlineto{\pgfqpoint{4.884971in}{0.962993in}}%
\pgfpathlineto{\pgfqpoint{4.887549in}{0.961304in}}%
\pgfpathlineto{\pgfqpoint{4.890126in}{0.974063in}}%
\pgfpathlineto{\pgfqpoint{4.892703in}{0.973500in}}%
\pgfpathlineto{\pgfqpoint{4.900435in}{0.976127in}}%
\pgfpathlineto{\pgfqpoint{4.903012in}{0.975752in}}%
\pgfpathlineto{\pgfqpoint{4.908166in}{0.976502in}}%
\pgfpathlineto{\pgfqpoint{4.918475in}{0.982694in}}%
\pgfpathlineto{\pgfqpoint{4.921052in}{0.982506in}}%
\pgfpathlineto{\pgfqpoint{4.923630in}{0.988323in}}%
\pgfpathlineto{\pgfqpoint{4.926207in}{0.987573in}}%
\pgfpathlineto{\pgfqpoint{4.928784in}{0.989824in}}%
\pgfpathlineto{\pgfqpoint{4.939093in}{0.974438in}}%
\pgfpathlineto{\pgfqpoint{4.941670in}{0.973125in}}%
\pgfpathlineto{\pgfqpoint{4.944247in}{0.968809in}}%
\pgfpathlineto{\pgfqpoint{4.946825in}{0.971248in}}%
\pgfpathlineto{\pgfqpoint{4.954556in}{0.969935in}}%
\pgfpathlineto{\pgfqpoint{4.957133in}{0.971624in}}%
\pgfpathlineto{\pgfqpoint{4.959711in}{0.974438in}}%
\pgfpathlineto{\pgfqpoint{4.964865in}{0.975189in}}%
\pgfpathlineto{\pgfqpoint{4.972597in}{0.975752in}}%
\pgfpathlineto{\pgfqpoint{4.975174in}{0.977253in}}%
\pgfpathlineto{\pgfqpoint{4.977751in}{0.977253in}}%
\pgfpathlineto{\pgfqpoint{4.982905in}{0.970873in}}%
\pgfpathlineto{\pgfqpoint{4.990637in}{0.969560in}}%
\pgfpathlineto{\pgfqpoint{4.993214in}{0.973125in}}%
\pgfpathlineto{\pgfqpoint{4.995792in}{0.973688in}}%
\pgfpathlineto{\pgfqpoint{4.998369in}{0.973125in}}%
\pgfpathlineto{\pgfqpoint{5.000946in}{0.970686in}}%
\pgfpathlineto{\pgfqpoint{5.008678in}{0.972937in}}%
\pgfpathlineto{\pgfqpoint{5.011255in}{0.968809in}}%
\pgfpathlineto{\pgfqpoint{5.013832in}{0.959615in}}%
\pgfpathlineto{\pgfqpoint{5.016409in}{0.956613in}}%
\pgfpathlineto{\pgfqpoint{5.018986in}{0.960366in}}%
\pgfpathlineto{\pgfqpoint{5.026718in}{0.956238in}}%
\pgfpathlineto{\pgfqpoint{5.029295in}{0.966182in}}%
\pgfpathlineto{\pgfqpoint{5.031873in}{0.963931in}}%
\pgfpathlineto{\pgfqpoint{5.034450in}{0.959990in}}%
\pgfpathlineto{\pgfqpoint{5.037027in}{0.952673in}}%
\pgfpathlineto{\pgfqpoint{5.044759in}{0.957739in}}%
\pgfpathlineto{\pgfqpoint{5.047336in}{0.953799in}}%
\pgfpathlineto{\pgfqpoint{5.049913in}{0.952110in}}%
\pgfpathlineto{\pgfqpoint{5.052490in}{0.947982in}}%
\pgfpathlineto{\pgfqpoint{5.055067in}{0.951359in}}%
\pgfpathlineto{\pgfqpoint{5.062799in}{0.950046in}}%
\pgfpathlineto{\pgfqpoint{5.065376in}{0.956050in}}%
\pgfpathlineto{\pgfqpoint{5.067953in}{0.959990in}}%
\pgfpathlineto{\pgfqpoint{5.070531in}{0.958865in}}%
\pgfpathlineto{\pgfqpoint{5.073108in}{0.960366in}}%
\pgfpathlineto{\pgfqpoint{5.083417in}{0.962242in}}%
\pgfpathlineto{\pgfqpoint{5.085994in}{0.960178in}}%
\pgfpathlineto{\pgfqpoint{5.088571in}{0.959240in}}%
\pgfpathlineto{\pgfqpoint{5.091148in}{0.957551in}}%
\pgfpathlineto{\pgfqpoint{5.101457in}{0.960741in}}%
\pgfpathlineto{\pgfqpoint{5.104034in}{0.963555in}}%
\pgfpathlineto{\pgfqpoint{5.106612in}{0.962430in}}%
\pgfpathlineto{\pgfqpoint{5.109189in}{0.959052in}}%
\pgfpathlineto{\pgfqpoint{5.116921in}{0.955675in}}%
\pgfpathlineto{\pgfqpoint{5.119498in}{0.965619in}}%
\pgfpathlineto{\pgfqpoint{5.122075in}{0.968059in}}%
\pgfpathlineto{\pgfqpoint{5.124652in}{0.972750in}}%
\pgfpathlineto{\pgfqpoint{5.127229in}{0.971999in}}%
\pgfpathlineto{\pgfqpoint{5.134961in}{0.975752in}}%
\pgfpathlineto{\pgfqpoint{5.137538in}{0.977628in}}%
\pgfpathlineto{\pgfqpoint{5.140115in}{0.975001in}}%
\pgfpathlineto{\pgfqpoint{5.142693in}{0.981756in}}%
\pgfpathlineto{\pgfqpoint{5.145270in}{0.959240in}}%
\pgfpathlineto{\pgfqpoint{5.153001in}{0.950984in}}%
\pgfpathlineto{\pgfqpoint{5.158156in}{0.971248in}}%
\pgfpathlineto{\pgfqpoint{5.160733in}{0.986447in}}%
\pgfpathlineto{\pgfqpoint{5.163310in}{0.986634in}}%
\pgfpathlineto{\pgfqpoint{5.173619in}{0.985884in}}%
\pgfpathlineto{\pgfqpoint{5.176196in}{0.990762in}}%
\pgfpathlineto{\pgfqpoint{5.178774in}{0.992076in}}%
\pgfpathlineto{\pgfqpoint{5.181351in}{0.998268in}}%
\pgfpathlineto{\pgfqpoint{5.189082in}{0.998455in}}%
\pgfpathlineto{\pgfqpoint{5.191660in}{0.999206in}}%
\pgfpathlineto{\pgfqpoint{5.194237in}{1.000895in}}%
\pgfpathlineto{\pgfqpoint{5.199391in}{1.009526in}}%
\pgfpathlineto{\pgfqpoint{5.209700in}{1.010276in}}%
\pgfpathlineto{\pgfqpoint{5.214855in}{1.004647in}}%
\pgfpathlineto{\pgfqpoint{5.217432in}{0.996016in}}%
\pgfpathlineto{\pgfqpoint{5.227741in}{0.986259in}}%
\pgfpathlineto{\pgfqpoint{5.230318in}{0.983257in}}%
\pgfpathlineto{\pgfqpoint{5.232895in}{0.982694in}}%
\pgfpathlineto{\pgfqpoint{5.235472in}{0.980818in}}%
\pgfpathlineto{\pgfqpoint{5.243204in}{0.981005in}}%
\pgfpathlineto{\pgfqpoint{5.245781in}{0.979317in}}%
\pgfpathlineto{\pgfqpoint{5.253513in}{0.983257in}}%
\pgfpathlineto{\pgfqpoint{5.279285in}{0.982506in}}%
\pgfpathlineto{\pgfqpoint{5.281862in}{0.981756in}}%
\pgfpathlineto{\pgfqpoint{5.284439in}{0.983257in}}%
\pgfpathlineto{\pgfqpoint{5.287017in}{0.985696in}}%
\pgfpathlineto{\pgfqpoint{5.289594in}{0.982694in}}%
\pgfpathlineto{\pgfqpoint{5.297325in}{0.983820in}}%
\pgfpathlineto{\pgfqpoint{5.299903in}{0.982319in}}%
\pgfpathlineto{\pgfqpoint{5.305057in}{0.981944in}}%
\pgfpathlineto{\pgfqpoint{5.307634in}{0.982319in}}%
\pgfpathlineto{\pgfqpoint{5.315366in}{0.984570in}}%
\pgfpathlineto{\pgfqpoint{5.317943in}{0.984570in}}%
\pgfpathlineto{\pgfqpoint{5.320520in}{0.982506in}}%
\pgfpathlineto{\pgfqpoint{5.323098in}{0.981944in}}%
\pgfpathlineto{\pgfqpoint{5.325675in}{0.983257in}}%
\pgfpathlineto{\pgfqpoint{5.335984in}{0.979317in}}%
\pgfpathlineto{\pgfqpoint{5.341138in}{0.979317in}}%
\pgfpathlineto{\pgfqpoint{5.343715in}{0.963931in}}%
\pgfpathlineto{\pgfqpoint{5.351447in}{0.970123in}}%
\pgfpathlineto{\pgfqpoint{5.354024in}{0.959615in}}%
\pgfpathlineto{\pgfqpoint{5.356601in}{0.957176in}}%
\pgfpathlineto{\pgfqpoint{5.359178in}{0.961867in}}%
\pgfpathlineto{\pgfqpoint{5.361756in}{0.960741in}}%
\pgfpathlineto{\pgfqpoint{5.369487in}{0.956613in}}%
\pgfpathlineto{\pgfqpoint{5.374642in}{0.963555in}}%
\pgfpathlineto{\pgfqpoint{5.377219in}{0.966745in}}%
\pgfpathlineto{\pgfqpoint{5.379796in}{0.964118in}}%
\pgfpathlineto{\pgfqpoint{5.387528in}{0.958302in}}%
\pgfpathlineto{\pgfqpoint{5.390105in}{0.963931in}}%
\pgfpathlineto{\pgfqpoint{5.392682in}{0.964306in}}%
\pgfpathlineto{\pgfqpoint{5.395259in}{0.958302in}}%
\pgfpathlineto{\pgfqpoint{5.397837in}{0.959615in}}%
\pgfpathlineto{\pgfqpoint{5.405568in}{0.959990in}}%
\pgfpathlineto{\pgfqpoint{5.408146in}{0.957739in}}%
\pgfpathlineto{\pgfqpoint{5.410723in}{0.957739in}}%
\pgfpathlineto{\pgfqpoint{5.415877in}{0.950796in}}%
\pgfpathlineto{\pgfqpoint{5.423609in}{0.947044in}}%
\pgfpathlineto{\pgfqpoint{5.426186in}{0.948170in}}%
\pgfpathlineto{\pgfqpoint{5.428763in}{0.947794in}}%
\pgfpathlineto{\pgfqpoint{5.431340in}{0.945730in}}%
\pgfpathlineto{\pgfqpoint{5.433918in}{0.947607in}}%
\pgfpathlineto{\pgfqpoint{5.441649in}{0.947044in}}%
\pgfpathlineto{\pgfqpoint{5.444227in}{0.949108in}}%
\pgfpathlineto{\pgfqpoint{5.446804in}{0.950421in}}%
\pgfpathlineto{\pgfqpoint{5.449381in}{0.950609in}}%
\pgfpathlineto{\pgfqpoint{5.451958in}{0.949108in}}%
\pgfpathlineto{\pgfqpoint{5.459690in}{0.948170in}}%
\pgfpathlineto{\pgfqpoint{5.462267in}{0.943666in}}%
\pgfpathlineto{\pgfqpoint{5.464844in}{0.947231in}}%
\pgfpathlineto{\pgfqpoint{5.467421in}{0.943291in}}%
\pgfpathlineto{\pgfqpoint{5.469999in}{0.953048in}}%
\pgfpathlineto{\pgfqpoint{5.477730in}{0.951172in}}%
\pgfpathlineto{\pgfqpoint{5.480307in}{0.947419in}}%
\pgfpathlineto{\pgfqpoint{5.482885in}{0.941039in}}%
\pgfpathlineto{\pgfqpoint{5.485462in}{0.937474in}}%
\pgfpathlineto{\pgfqpoint{5.488039in}{0.940101in}}%
\pgfpathlineto{\pgfqpoint{5.495771in}{0.954549in}}%
\pgfpathlineto{\pgfqpoint{5.498348in}{0.956425in}}%
\pgfpathlineto{\pgfqpoint{5.500925in}{0.959803in}}%
\pgfpathlineto{\pgfqpoint{5.503502in}{0.972750in}}%
\pgfpathlineto{\pgfqpoint{5.506080in}{0.977816in}}%
\pgfpathlineto{\pgfqpoint{5.513811in}{0.974438in}}%
\pgfpathlineto{\pgfqpoint{5.516388in}{0.978379in}}%
\pgfpathlineto{\pgfqpoint{5.518966in}{0.978191in}}%
\pgfpathlineto{\pgfqpoint{5.521543in}{0.979129in}}%
\pgfpathlineto{\pgfqpoint{5.524120in}{0.977065in}}%
\pgfpathlineto{\pgfqpoint{5.531852in}{0.980442in}}%
\pgfpathlineto{\pgfqpoint{5.534429in}{0.985509in}}%
\pgfpathlineto{\pgfqpoint{5.537006in}{0.988135in}}%
\pgfpathlineto{\pgfqpoint{5.542161in}{0.989824in}}%
\pgfpathlineto{\pgfqpoint{5.549892in}{0.986634in}}%
\pgfpathlineto{\pgfqpoint{5.552469in}{0.983445in}}%
\pgfpathlineto{\pgfqpoint{5.555047in}{0.978566in}}%
\pgfpathlineto{\pgfqpoint{5.557624in}{0.989074in}}%
\pgfpathlineto{\pgfqpoint{5.560201in}{0.988135in}}%
\pgfpathlineto{\pgfqpoint{5.567933in}{0.984383in}}%
\pgfpathlineto{\pgfqpoint{5.570510in}{0.985321in}}%
\pgfpathlineto{\pgfqpoint{5.573087in}{0.992451in}}%
\pgfpathlineto{\pgfqpoint{5.575664in}{0.991325in}}%
\pgfpathlineto{\pgfqpoint{5.578242in}{0.995453in}}%
\pgfpathlineto{\pgfqpoint{5.585973in}{0.996767in}}%
\pgfpathlineto{\pgfqpoint{5.588550in}{0.994703in}}%
\pgfpathlineto{\pgfqpoint{5.593705in}{0.986822in}}%
\pgfpathlineto{\pgfqpoint{5.596282in}{0.994890in}}%
\pgfpathlineto{\pgfqpoint{5.604014in}{0.997705in}}%
\pgfpathlineto{\pgfqpoint{5.606591in}{1.003146in}}%
\pgfpathlineto{\pgfqpoint{5.609168in}{1.001270in}}%
\pgfpathlineto{\pgfqpoint{5.611745in}{1.000144in}}%
\pgfpathlineto{\pgfqpoint{5.614323in}{1.001082in}}%
\pgfpathlineto{\pgfqpoint{5.624631in}{1.001457in}}%
\pgfpathlineto{\pgfqpoint{5.627209in}{0.998080in}}%
\pgfpathlineto{\pgfqpoint{5.629786in}{0.998268in}}%
\pgfpathlineto{\pgfqpoint{5.632363in}{0.996391in}}%
\pgfpathlineto{\pgfqpoint{5.645249in}{0.998080in}}%
\pgfpathlineto{\pgfqpoint{5.647826in}{0.995078in}}%
\pgfpathlineto{\pgfqpoint{5.650403in}{0.996579in}}%
\pgfpathlineto{\pgfqpoint{5.658135in}{0.994140in}}%
\pgfpathlineto{\pgfqpoint{5.660712in}{0.992639in}}%
\pgfpathlineto{\pgfqpoint{5.663290in}{0.994327in}}%
\pgfpathlineto{\pgfqpoint{5.665867in}{0.992826in}}%
\pgfpathlineto{\pgfqpoint{5.683907in}{0.989824in}}%
\pgfpathlineto{\pgfqpoint{5.686484in}{0.978566in}}%
\pgfpathlineto{\pgfqpoint{5.694216in}{0.965619in}}%
\pgfpathlineto{\pgfqpoint{5.696793in}{0.969747in}}%
\pgfpathlineto{\pgfqpoint{5.699371in}{0.975939in}}%
\pgfpathlineto{\pgfqpoint{5.701948in}{0.975001in}}%
\pgfpathlineto{\pgfqpoint{5.704525in}{0.969935in}}%
\pgfpathlineto{\pgfqpoint{5.712257in}{0.968997in}}%
\pgfpathlineto{\pgfqpoint{5.714834in}{0.964681in}}%
\pgfpathlineto{\pgfqpoint{5.719988in}{0.964306in}}%
\pgfpathlineto{\pgfqpoint{5.722565in}{0.964681in}}%
\pgfpathlineto{\pgfqpoint{5.730297in}{0.964118in}}%
\pgfpathlineto{\pgfqpoint{5.735452in}{0.960178in}}%
\pgfpathlineto{\pgfqpoint{5.740606in}{0.965057in}}%
\pgfpathlineto{\pgfqpoint{5.748338in}{0.970310in}}%
\pgfpathlineto{\pgfqpoint{5.750915in}{0.974438in}}%
\pgfpathlineto{\pgfqpoint{5.756069in}{0.977253in}}%
\pgfpathlineto{\pgfqpoint{5.758646in}{0.975939in}}%
\pgfpathlineto{\pgfqpoint{5.768955in}{0.978379in}}%
\pgfpathlineto{\pgfqpoint{5.771532in}{0.975189in}}%
\pgfpathlineto{\pgfqpoint{5.774110in}{0.974063in}}%
\pgfpathlineto{\pgfqpoint{5.776687in}{0.976877in}}%
\pgfpathlineto{\pgfqpoint{5.784419in}{0.972750in}}%
\pgfpathlineto{\pgfqpoint{5.786996in}{0.970498in}}%
\pgfpathlineto{\pgfqpoint{5.789573in}{0.976877in}}%
\pgfpathlineto{\pgfqpoint{5.792150in}{0.976877in}}%
\pgfpathlineto{\pgfqpoint{5.794727in}{0.975752in}}%
\pgfpathlineto{\pgfqpoint{5.802459in}{0.973688in}}%
\pgfpathlineto{\pgfqpoint{5.805036in}{0.971436in}}%
\pgfpathlineto{\pgfqpoint{5.807613in}{0.970310in}}%
\pgfpathlineto{\pgfqpoint{5.810191in}{0.968059in}}%
\pgfpathlineto{\pgfqpoint{5.812768in}{0.978379in}}%
\pgfpathlineto{\pgfqpoint{5.820500in}{0.971436in}}%
\pgfpathlineto{\pgfqpoint{5.823077in}{0.965995in}}%
\pgfpathlineto{\pgfqpoint{5.825654in}{0.969747in}}%
\pgfpathlineto{\pgfqpoint{5.828231in}{0.969560in}}%
\pgfpathlineto{\pgfqpoint{5.830808in}{0.971624in}}%
\pgfpathlineto{\pgfqpoint{5.838540in}{0.969372in}}%
\pgfpathlineto{\pgfqpoint{5.841117in}{0.963555in}}%
\pgfpathlineto{\pgfqpoint{5.843694in}{0.965807in}}%
\pgfpathlineto{\pgfqpoint{5.848849in}{0.968997in}}%
\pgfpathlineto{\pgfqpoint{5.856580in}{0.964306in}}%
\pgfpathlineto{\pgfqpoint{5.859158in}{0.967308in}}%
\pgfpathlineto{\pgfqpoint{5.861735in}{0.968434in}}%
\pgfpathlineto{\pgfqpoint{5.864312in}{0.971624in}}%
\pgfpathlineto{\pgfqpoint{5.866889in}{0.970310in}}%
\pgfpathlineto{\pgfqpoint{5.874621in}{0.971624in}}%
\pgfpathlineto{\pgfqpoint{5.877198in}{0.974063in}}%
\pgfpathlineto{\pgfqpoint{5.882353in}{0.972562in}}%
\pgfpathlineto{\pgfqpoint{5.884930in}{0.973500in}}%
\pgfpathlineto{\pgfqpoint{5.895239in}{0.974438in}}%
\pgfpathlineto{\pgfqpoint{5.900393in}{0.966370in}}%
\pgfpathlineto{\pgfqpoint{5.910702in}{0.967683in}}%
\pgfpathlineto{\pgfqpoint{5.918434in}{0.978191in}}%
\pgfpathlineto{\pgfqpoint{5.921011in}{0.966182in}}%
\pgfpathlineto{\pgfqpoint{5.928742in}{0.966182in}}%
\pgfpathlineto{\pgfqpoint{5.931320in}{0.964494in}}%
\pgfpathlineto{\pgfqpoint{5.936474in}{0.958302in}}%
\pgfpathlineto{\pgfqpoint{5.939051in}{0.956801in}}%
\pgfpathlineto{\pgfqpoint{5.946783in}{0.955862in}}%
\pgfpathlineto{\pgfqpoint{5.949360in}{0.956801in}}%
\pgfpathlineto{\pgfqpoint{5.951937in}{0.960741in}}%
\pgfpathlineto{\pgfqpoint{5.954515in}{0.960366in}}%
\pgfpathlineto{\pgfqpoint{5.957092in}{0.960553in}}%
\pgfpathlineto{\pgfqpoint{5.964823in}{0.958114in}}%
\pgfpathlineto{\pgfqpoint{5.967401in}{0.955675in}}%
\pgfpathlineto{\pgfqpoint{5.969978in}{0.951922in}}%
\pgfpathlineto{\pgfqpoint{5.972555in}{0.954737in}}%
\pgfpathlineto{\pgfqpoint{5.975132in}{0.944604in}}%
\pgfpathlineto{\pgfqpoint{5.982864in}{0.943103in}}%
\pgfpathlineto{\pgfqpoint{5.985441in}{0.940852in}}%
\pgfpathlineto{\pgfqpoint{5.988018in}{0.930157in}}%
\pgfpathlineto{\pgfqpoint{5.990596in}{0.931282in}}%
\pgfpathlineto{\pgfqpoint{5.993173in}{0.941039in}}%
\pgfpathlineto{\pgfqpoint{6.000904in}{0.943103in}}%
\pgfpathlineto{\pgfqpoint{6.003482in}{0.944792in}}%
\pgfpathlineto{\pgfqpoint{6.008636in}{0.931470in}}%
\pgfpathlineto{\pgfqpoint{6.011213in}{0.930907in}}%
\pgfpathlineto{\pgfqpoint{6.021522in}{0.929406in}}%
\pgfpathlineto{\pgfqpoint{6.024099in}{0.929781in}}%
\pgfpathlineto{\pgfqpoint{6.026677in}{0.935410in}}%
\pgfpathlineto{\pgfqpoint{6.029254in}{0.938037in}}%
\pgfpathlineto{\pgfqpoint{6.036985in}{0.939726in}}%
\pgfpathlineto{\pgfqpoint{6.039563in}{0.938975in}}%
\pgfpathlineto{\pgfqpoint{6.042140in}{0.934660in}}%
\pgfpathlineto{\pgfqpoint{6.044717in}{0.933159in}}%
\pgfpathlineto{\pgfqpoint{6.055026in}{0.955862in}}%
\pgfpathlineto{\pgfqpoint{6.057603in}{0.947607in}}%
\pgfpathlineto{\pgfqpoint{6.060180in}{0.951735in}}%
\pgfpathlineto{\pgfqpoint{6.062757in}{0.959990in}}%
\pgfpathlineto{\pgfqpoint{6.065335in}{0.965244in}}%
\pgfpathlineto{\pgfqpoint{6.073066in}{0.961867in}}%
\pgfpathlineto{\pgfqpoint{6.075644in}{0.950421in}}%
\pgfpathlineto{\pgfqpoint{6.078221in}{0.944417in}}%
\pgfpathlineto{\pgfqpoint{6.080798in}{0.940477in}}%
\pgfpathlineto{\pgfqpoint{6.091107in}{0.941602in}}%
\pgfpathlineto{\pgfqpoint{6.093684in}{0.934660in}}%
\pgfpathlineto{\pgfqpoint{6.096261in}{0.932408in}}%
\pgfpathlineto{\pgfqpoint{6.098838in}{0.931470in}}%
\pgfpathlineto{\pgfqpoint{6.101416in}{0.931282in}}%
\pgfpathlineto{\pgfqpoint{6.109147in}{0.938788in}}%
\pgfpathlineto{\pgfqpoint{6.114302in}{0.937099in}}%
\pgfpathlineto{\pgfqpoint{6.116879in}{0.919274in}}%
\pgfpathlineto{\pgfqpoint{6.119456in}{0.916647in}}%
\pgfpathlineto{\pgfqpoint{6.127188in}{0.914771in}}%
\pgfpathlineto{\pgfqpoint{6.129765in}{0.920587in}}%
\pgfpathlineto{\pgfqpoint{6.134919in}{0.927530in}}%
\pgfpathlineto{\pgfqpoint{6.137497in}{0.927342in}}%
\pgfpathlineto{\pgfqpoint{6.145228in}{0.928093in}}%
\pgfpathlineto{\pgfqpoint{6.150383in}{0.930157in}}%
\pgfpathlineto{\pgfqpoint{6.152960in}{0.925841in}}%
\pgfpathlineto{\pgfqpoint{6.155537in}{0.912519in}}%
\pgfpathlineto{\pgfqpoint{6.163269in}{0.904263in}}%
\pgfpathlineto{\pgfqpoint{6.165846in}{0.904451in}}%
\pgfpathlineto{\pgfqpoint{6.171000in}{0.910455in}}%
\pgfpathlineto{\pgfqpoint{6.173578in}{0.905952in}}%
\pgfpathlineto{\pgfqpoint{6.181309in}{0.907265in}}%
\pgfpathlineto{\pgfqpoint{6.183886in}{0.904451in}}%
\pgfpathlineto{\pgfqpoint{6.186464in}{0.905764in}}%
\pgfpathlineto{\pgfqpoint{6.189041in}{0.909892in}}%
\pgfpathlineto{\pgfqpoint{6.191618in}{0.910268in}}%
\pgfpathlineto{\pgfqpoint{6.201927in}{0.906515in}}%
\pgfpathlineto{\pgfqpoint{6.204504in}{0.909142in}}%
\pgfpathlineto{\pgfqpoint{6.207081in}{0.902012in}}%
\pgfpathlineto{\pgfqpoint{6.209659in}{0.900323in}}%
\pgfpathlineto{\pgfqpoint{6.217390in}{0.903137in}}%
\pgfpathlineto{\pgfqpoint{6.219967in}{0.899385in}}%
\pgfpathlineto{\pgfqpoint{6.222545in}{0.898634in}}%
\pgfpathlineto{\pgfqpoint{6.225122in}{0.892630in}}%
\pgfpathlineto{\pgfqpoint{6.227699in}{0.889253in}}%
\pgfpathlineto{\pgfqpoint{6.235431in}{0.888127in}}%
\pgfpathlineto{\pgfqpoint{6.238008in}{0.890003in}}%
\pgfpathlineto{\pgfqpoint{6.240585in}{0.886438in}}%
\pgfpathlineto{\pgfqpoint{6.243162in}{0.884937in}}%
\pgfpathlineto{\pgfqpoint{6.245740in}{0.888127in}}%
\pgfpathlineto{\pgfqpoint{6.256048in}{0.887376in}}%
\pgfpathlineto{\pgfqpoint{6.258626in}{0.884562in}}%
\pgfpathlineto{\pgfqpoint{6.261203in}{0.889253in}}%
\pgfpathlineto{\pgfqpoint{6.263780in}{0.899385in}}%
\pgfpathlineto{\pgfqpoint{6.274089in}{0.892630in}}%
\pgfpathlineto{\pgfqpoint{6.276666in}{0.895632in}}%
\pgfpathlineto{\pgfqpoint{6.279243in}{0.880246in}}%
\pgfpathlineto{\pgfqpoint{6.281821in}{0.876681in}}%
\pgfpathlineto{\pgfqpoint{6.289552in}{0.874992in}}%
\pgfpathlineto{\pgfqpoint{6.294707in}{0.881747in}}%
\pgfpathlineto{\pgfqpoint{6.297284in}{0.884186in}}%
\pgfpathlineto{\pgfqpoint{6.299861in}{0.882685in}}%
\pgfpathlineto{\pgfqpoint{6.307593in}{0.891879in}}%
\pgfpathlineto{\pgfqpoint{6.310170in}{0.887376in}}%
\pgfpathlineto{\pgfqpoint{6.312747in}{0.889440in}}%
\pgfpathlineto{\pgfqpoint{6.315324in}{0.896946in}}%
\pgfpathlineto{\pgfqpoint{6.317902in}{0.899010in}}%
\pgfpathlineto{\pgfqpoint{6.325633in}{0.903137in}}%
\pgfpathlineto{\pgfqpoint{6.328210in}{0.899948in}}%
\pgfpathlineto{\pgfqpoint{6.330788in}{0.890378in}}%
\pgfpathlineto{\pgfqpoint{6.333365in}{0.888127in}}%
\pgfpathlineto{\pgfqpoint{6.335942in}{0.887001in}}%
\pgfpathlineto{\pgfqpoint{6.343674in}{0.893756in}}%
\pgfpathlineto{\pgfqpoint{6.346251in}{0.897696in}}%
\pgfpathlineto{\pgfqpoint{6.348828in}{0.892255in}}%
\pgfpathlineto{\pgfqpoint{6.351405in}{0.893193in}}%
\pgfpathlineto{\pgfqpoint{6.353982in}{0.890566in}}%
\pgfpathlineto{\pgfqpoint{6.361714in}{0.874054in}}%
\pgfpathlineto{\pgfqpoint{6.364291in}{0.872928in}}%
\pgfpathlineto{\pgfqpoint{6.366869in}{0.867862in}}%
\pgfpathlineto{\pgfqpoint{6.369446in}{0.867487in}}%
\pgfpathlineto{\pgfqpoint{6.372023in}{0.866361in}}%
\pgfpathlineto{\pgfqpoint{6.379755in}{0.872928in}}%
\pgfpathlineto{\pgfqpoint{6.382332in}{0.869926in}}%
\pgfpathlineto{\pgfqpoint{6.384909in}{0.868801in}}%
\pgfpathlineto{\pgfqpoint{6.387486in}{0.876681in}}%
\pgfpathlineto{\pgfqpoint{6.390063in}{0.880997in}}%
\pgfpathlineto{\pgfqpoint{6.400372in}{0.847410in}}%
\pgfpathlineto{\pgfqpoint{6.402950in}{0.840843in}}%
\pgfpathlineto{\pgfqpoint{6.405527in}{0.837653in}}%
\pgfpathlineto{\pgfqpoint{6.408104in}{0.828459in}}%
\pgfpathlineto{\pgfqpoint{6.415836in}{0.821892in}}%
\pgfpathlineto{\pgfqpoint{6.418413in}{0.817577in}}%
\pgfpathlineto{\pgfqpoint{6.420990in}{0.815137in}}%
\pgfpathlineto{\pgfqpoint{6.423567in}{0.813824in}}%
\pgfpathlineto{\pgfqpoint{6.426144in}{0.817201in}}%
\pgfpathlineto{\pgfqpoint{6.433876in}{0.817014in}}%
\pgfpathlineto{\pgfqpoint{6.436453in}{0.818515in}}%
\pgfpathlineto{\pgfqpoint{6.439031in}{0.817014in}}%
\pgfpathlineto{\pgfqpoint{6.441608in}{0.814762in}}%
\pgfpathlineto{\pgfqpoint{6.444185in}{0.823393in}}%
\pgfpathlineto{\pgfqpoint{6.451917in}{0.797875in}}%
\pgfpathlineto{\pgfqpoint{6.454494in}{0.778549in}}%
\pgfpathlineto{\pgfqpoint{6.457071in}{0.784741in}}%
\pgfpathlineto{\pgfqpoint{6.462225in}{0.783990in}}%
\pgfpathlineto{\pgfqpoint{6.469957in}{0.780050in}}%
\pgfpathlineto{\pgfqpoint{6.472534in}{0.777423in}}%
\pgfpathlineto{\pgfqpoint{6.475111in}{0.782864in}}%
\pgfpathlineto{\pgfqpoint{6.480266in}{0.783615in}}%
\pgfpathlineto{\pgfqpoint{6.487998in}{0.782301in}}%
\pgfpathlineto{\pgfqpoint{6.490575in}{0.787368in}}%
\pgfpathlineto{\pgfqpoint{6.493152in}{0.788681in}}%
\pgfpathlineto{\pgfqpoint{6.495729in}{0.785304in}}%
\pgfpathlineto{\pgfqpoint{6.498306in}{0.778173in}}%
\pgfpathlineto{\pgfqpoint{6.506038in}{0.779487in}}%
\pgfpathlineto{\pgfqpoint{6.508615in}{0.776110in}}%
\pgfpathlineto{\pgfqpoint{6.516347in}{0.777423in}}%
\pgfpathlineto{\pgfqpoint{6.524079in}{0.776485in}}%
\pgfpathlineto{\pgfqpoint{6.526656in}{0.780988in}}%
\pgfpathlineto{\pgfqpoint{6.531810in}{0.776297in}}%
\pgfpathlineto{\pgfqpoint{6.534387in}{0.779299in}}%
\pgfpathlineto{\pgfqpoint{6.542119in}{0.778361in}}%
\pgfpathlineto{\pgfqpoint{6.547273in}{0.772920in}}%
\pgfpathlineto{\pgfqpoint{6.552428in}{0.773858in}}%
\pgfpathlineto{\pgfqpoint{6.562737in}{0.774608in}}%
\pgfpathlineto{\pgfqpoint{6.567891in}{0.773483in}}%
\pgfpathlineto{\pgfqpoint{6.570468in}{0.774984in}}%
\pgfpathlineto{\pgfqpoint{6.570468in}{0.774984in}}%
\pgfusepath{stroke}%
\end{pgfscope}%
\begin{pgfscope}%
\pgfpathrectangle{\pgfqpoint{0.652287in}{0.521603in}}{\pgfqpoint{6.200000in}{4.530000in}}%
\pgfusepath{clip}%
\pgfsetroundcap%
\pgfsetroundjoin%
\pgfsetlinewidth{1.505625pt}%
\definecolor{currentstroke}{rgb}{0.839216,0.152941,0.156863}%
\pgfsetstrokecolor{currentstroke}%
\pgfsetdash{}{0pt}%
\pgfpathmoveto{\pgfqpoint{0.934105in}{0.834651in}}%
\pgfpathlineto{\pgfqpoint{0.936682in}{0.843282in}}%
\pgfpathlineto{\pgfqpoint{0.939259in}{0.847598in}}%
\pgfpathlineto{\pgfqpoint{0.941836in}{0.845346in}}%
\pgfpathlineto{\pgfqpoint{0.949568in}{0.848536in}}%
\pgfpathlineto{\pgfqpoint{0.952145in}{0.850412in}}%
\pgfpathlineto{\pgfqpoint{0.954722in}{0.853415in}}%
\pgfpathlineto{\pgfqpoint{0.957300in}{0.852664in}}%
\pgfpathlineto{\pgfqpoint{0.959877in}{0.843658in}}%
\pgfpathlineto{\pgfqpoint{0.970186in}{0.842157in}}%
\pgfpathlineto{\pgfqpoint{0.972763in}{0.847410in}}%
\pgfpathlineto{\pgfqpoint{0.975340in}{0.850975in}}%
\pgfpathlineto{\pgfqpoint{0.977917in}{0.862046in}}%
\pgfpathlineto{\pgfqpoint{0.985649in}{0.866924in}}%
\pgfpathlineto{\pgfqpoint{0.988226in}{0.869739in}}%
\pgfpathlineto{\pgfqpoint{0.990803in}{0.869739in}}%
\pgfpathlineto{\pgfqpoint{0.993381in}{0.867675in}}%
\pgfpathlineto{\pgfqpoint{0.995958in}{0.867299in}}%
\pgfpathlineto{\pgfqpoint{1.003689in}{0.867487in}}%
\pgfpathlineto{\pgfqpoint{1.006267in}{0.862609in}}%
\pgfpathlineto{\pgfqpoint{1.008844in}{0.864673in}}%
\pgfpathlineto{\pgfqpoint{1.011421in}{0.863734in}}%
\pgfpathlineto{\pgfqpoint{1.013998in}{0.870677in}}%
\pgfpathlineto{\pgfqpoint{1.021730in}{0.870302in}}%
\pgfpathlineto{\pgfqpoint{1.024307in}{0.869176in}}%
\pgfpathlineto{\pgfqpoint{1.026884in}{0.872366in}}%
\pgfpathlineto{\pgfqpoint{1.029462in}{0.872366in}}%
\pgfpathlineto{\pgfqpoint{1.032039in}{0.870114in}}%
\pgfpathlineto{\pgfqpoint{1.039770in}{0.870114in}}%
\pgfpathlineto{\pgfqpoint{1.042348in}{0.871240in}}%
\pgfpathlineto{\pgfqpoint{1.044925in}{0.868238in}}%
\pgfpathlineto{\pgfqpoint{1.047502in}{0.871990in}}%
\pgfpathlineto{\pgfqpoint{1.050079in}{0.880059in}}%
\pgfpathlineto{\pgfqpoint{1.060388in}{0.876869in}}%
\pgfpathlineto{\pgfqpoint{1.062965in}{0.870489in}}%
\pgfpathlineto{\pgfqpoint{1.065543in}{0.869363in}}%
\pgfpathlineto{\pgfqpoint{1.075851in}{0.872928in}}%
\pgfpathlineto{\pgfqpoint{1.078429in}{0.878182in}}%
\pgfpathlineto{\pgfqpoint{1.081006in}{0.872741in}}%
\pgfpathlineto{\pgfqpoint{1.083583in}{0.872366in}}%
\pgfpathlineto{\pgfqpoint{1.086160in}{0.873304in}}%
\pgfpathlineto{\pgfqpoint{1.093892in}{0.867675in}}%
\pgfpathlineto{\pgfqpoint{1.096469in}{0.868613in}}%
\pgfpathlineto{\pgfqpoint{1.099046in}{0.873116in}}%
\pgfpathlineto{\pgfqpoint{1.101624in}{0.872178in}}%
\pgfpathlineto{\pgfqpoint{1.104201in}{0.875555in}}%
\pgfpathlineto{\pgfqpoint{1.111932in}{0.874430in}}%
\pgfpathlineto{\pgfqpoint{1.114510in}{0.881935in}}%
\pgfpathlineto{\pgfqpoint{1.117087in}{0.881372in}}%
\pgfpathlineto{\pgfqpoint{1.119664in}{0.885688in}}%
\pgfpathlineto{\pgfqpoint{1.122241in}{0.885500in}}%
\pgfpathlineto{\pgfqpoint{1.135127in}{0.886250in}}%
\pgfpathlineto{\pgfqpoint{1.137704in}{0.887939in}}%
\pgfpathlineto{\pgfqpoint{1.140282in}{0.887752in}}%
\pgfpathlineto{\pgfqpoint{1.148013in}{0.892442in}}%
\pgfpathlineto{\pgfqpoint{1.150591in}{0.892442in}}%
\pgfpathlineto{\pgfqpoint{1.153168in}{0.886438in}}%
\pgfpathlineto{\pgfqpoint{1.155745in}{0.891879in}}%
\pgfpathlineto{\pgfqpoint{1.158322in}{0.891317in}}%
\pgfpathlineto{\pgfqpoint{1.166054in}{0.895257in}}%
\pgfpathlineto{\pgfqpoint{1.168631in}{0.891129in}}%
\pgfpathlineto{\pgfqpoint{1.171208in}{0.888502in}}%
\pgfpathlineto{\pgfqpoint{1.173785in}{0.890566in}}%
\pgfpathlineto{\pgfqpoint{1.184094in}{0.885875in}}%
\pgfpathlineto{\pgfqpoint{1.186672in}{0.881372in}}%
\pgfpathlineto{\pgfqpoint{1.189249in}{0.887189in}}%
\pgfpathlineto{\pgfqpoint{1.191826in}{0.896758in}}%
\pgfpathlineto{\pgfqpoint{1.194403in}{0.890941in}}%
\pgfpathlineto{\pgfqpoint{1.204712in}{0.896570in}}%
\pgfpathlineto{\pgfqpoint{1.207289in}{0.888690in}}%
\pgfpathlineto{\pgfqpoint{1.209866in}{0.884937in}}%
\pgfpathlineto{\pgfqpoint{1.212444in}{0.883624in}}%
\pgfpathlineto{\pgfqpoint{1.220175in}{0.881372in}}%
\pgfpathlineto{\pgfqpoint{1.222753in}{0.879120in}}%
\pgfpathlineto{\pgfqpoint{1.225330in}{0.887376in}}%
\pgfpathlineto{\pgfqpoint{1.227907in}{0.892818in}}%
\pgfpathlineto{\pgfqpoint{1.230484in}{0.895257in}}%
\pgfpathlineto{\pgfqpoint{1.238216in}{0.895444in}}%
\pgfpathlineto{\pgfqpoint{1.240793in}{0.903700in}}%
\pgfpathlineto{\pgfqpoint{1.243370in}{0.907265in}}%
\pgfpathlineto{\pgfqpoint{1.245947in}{0.901073in}}%
\pgfpathlineto{\pgfqpoint{1.248525in}{0.891129in}}%
\pgfpathlineto{\pgfqpoint{1.256256in}{0.888877in}}%
\pgfpathlineto{\pgfqpoint{1.258833in}{0.883061in}}%
\pgfpathlineto{\pgfqpoint{1.261411in}{0.880246in}}%
\pgfpathlineto{\pgfqpoint{1.263988in}{0.881184in}}%
\pgfpathlineto{\pgfqpoint{1.266565in}{0.887001in}}%
\pgfpathlineto{\pgfqpoint{1.274297in}{0.877807in}}%
\pgfpathlineto{\pgfqpoint{1.276874in}{0.875743in}}%
\pgfpathlineto{\pgfqpoint{1.282028in}{0.865235in}}%
\pgfpathlineto{\pgfqpoint{1.284606in}{0.863359in}}%
\pgfpathlineto{\pgfqpoint{1.292337in}{0.864673in}}%
\pgfpathlineto{\pgfqpoint{1.294914in}{0.862796in}}%
\pgfpathlineto{\pgfqpoint{1.297492in}{0.853977in}}%
\pgfpathlineto{\pgfqpoint{1.300069in}{0.856980in}}%
\pgfpathlineto{\pgfqpoint{1.302646in}{0.858481in}}%
\pgfpathlineto{\pgfqpoint{1.312955in}{0.863734in}}%
\pgfpathlineto{\pgfqpoint{1.315532in}{0.864297in}}%
\pgfpathlineto{\pgfqpoint{1.318109in}{0.859982in}}%
\pgfpathlineto{\pgfqpoint{1.320687in}{0.849287in}}%
\pgfpathlineto{\pgfqpoint{1.328418in}{0.847786in}}%
\pgfpathlineto{\pgfqpoint{1.330995in}{0.853790in}}%
\pgfpathlineto{\pgfqpoint{1.333573in}{0.863359in}}%
\pgfpathlineto{\pgfqpoint{1.336150in}{0.861483in}}%
\pgfpathlineto{\pgfqpoint{1.338727in}{0.868613in}}%
\pgfpathlineto{\pgfqpoint{1.346459in}{0.862233in}}%
\pgfpathlineto{\pgfqpoint{1.349036in}{0.870302in}}%
\pgfpathlineto{\pgfqpoint{1.351613in}{0.870489in}}%
\pgfpathlineto{\pgfqpoint{1.356768in}{0.882685in}}%
\pgfpathlineto{\pgfqpoint{1.364499in}{0.883811in}}%
\pgfpathlineto{\pgfqpoint{1.367076in}{0.885125in}}%
\pgfpathlineto{\pgfqpoint{1.369654in}{0.887189in}}%
\pgfpathlineto{\pgfqpoint{1.372231in}{0.873116in}}%
\pgfpathlineto{\pgfqpoint{1.374808in}{0.876493in}}%
\pgfpathlineto{\pgfqpoint{1.382540in}{0.863172in}}%
\pgfpathlineto{\pgfqpoint{1.385117in}{0.862421in}}%
\pgfpathlineto{\pgfqpoint{1.387694in}{0.865611in}}%
\pgfpathlineto{\pgfqpoint{1.390271in}{0.859794in}}%
\pgfpathlineto{\pgfqpoint{1.392849in}{0.872178in}}%
\pgfpathlineto{\pgfqpoint{1.400580in}{0.872553in}}%
\pgfpathlineto{\pgfqpoint{1.403157in}{0.875368in}}%
\pgfpathlineto{\pgfqpoint{1.408312in}{0.870677in}}%
\pgfpathlineto{\pgfqpoint{1.410889in}{0.864860in}}%
\pgfpathlineto{\pgfqpoint{1.418621in}{0.864860in}}%
\pgfpathlineto{\pgfqpoint{1.421198in}{0.855666in}}%
\pgfpathlineto{\pgfqpoint{1.423775in}{0.853227in}}%
\pgfpathlineto{\pgfqpoint{1.426352in}{0.843282in}}%
\pgfpathlineto{\pgfqpoint{1.428930in}{0.850975in}}%
\pgfpathlineto{\pgfqpoint{1.436661in}{0.849287in}}%
\pgfpathlineto{\pgfqpoint{1.439238in}{0.853039in}}%
\pgfpathlineto{\pgfqpoint{1.441816in}{0.865611in}}%
\pgfpathlineto{\pgfqpoint{1.444393in}{0.863172in}}%
\pgfpathlineto{\pgfqpoint{1.446970in}{0.855103in}}%
\pgfpathlineto{\pgfqpoint{1.454702in}{0.851163in}}%
\pgfpathlineto{\pgfqpoint{1.457279in}{0.847410in}}%
\pgfpathlineto{\pgfqpoint{1.459856in}{0.849287in}}%
\pgfpathlineto{\pgfqpoint{1.462433in}{0.854728in}}%
\pgfpathlineto{\pgfqpoint{1.465010in}{0.862609in}}%
\pgfpathlineto{\pgfqpoint{1.475319in}{0.857730in}}%
\pgfpathlineto{\pgfqpoint{1.477897in}{0.861295in}}%
\pgfpathlineto{\pgfqpoint{1.480474in}{0.860920in}}%
\pgfpathlineto{\pgfqpoint{1.483051in}{0.869363in}}%
\pgfpathlineto{\pgfqpoint{1.490783in}{0.870489in}}%
\pgfpathlineto{\pgfqpoint{1.493360in}{0.873491in}}%
\pgfpathlineto{\pgfqpoint{1.498514in}{0.876493in}}%
\pgfpathlineto{\pgfqpoint{1.501091in}{0.879120in}}%
\pgfpathlineto{\pgfqpoint{1.508823in}{0.876306in}}%
\pgfpathlineto{\pgfqpoint{1.513978in}{0.869926in}}%
\pgfpathlineto{\pgfqpoint{1.516555in}{0.874805in}}%
\pgfpathlineto{\pgfqpoint{1.519132in}{0.870864in}}%
\pgfpathlineto{\pgfqpoint{1.526864in}{0.869363in}}%
\pgfpathlineto{\pgfqpoint{1.529441in}{0.867487in}}%
\pgfpathlineto{\pgfqpoint{1.532018in}{0.861670in}}%
\pgfpathlineto{\pgfqpoint{1.534595in}{0.851163in}}%
\pgfpathlineto{\pgfqpoint{1.537172in}{0.849099in}}%
\pgfpathlineto{\pgfqpoint{1.544904in}{0.848161in}}%
\pgfpathlineto{\pgfqpoint{1.547481in}{0.850600in}}%
\pgfpathlineto{\pgfqpoint{1.552636in}{0.839342in}}%
\pgfpathlineto{\pgfqpoint{1.555213in}{0.847973in}}%
\pgfpathlineto{\pgfqpoint{1.565522in}{0.841781in}}%
\pgfpathlineto{\pgfqpoint{1.568099in}{0.841218in}}%
\pgfpathlineto{\pgfqpoint{1.570676in}{0.852101in}}%
\pgfpathlineto{\pgfqpoint{1.573253in}{0.838216in}}%
\pgfpathlineto{\pgfqpoint{1.580985in}{0.823956in}}%
\pgfpathlineto{\pgfqpoint{1.583562in}{0.825270in}}%
\pgfpathlineto{\pgfqpoint{1.586139in}{0.823018in}}%
\pgfpathlineto{\pgfqpoint{1.588717in}{0.825645in}}%
\pgfpathlineto{\pgfqpoint{1.591294in}{0.825645in}}%
\pgfpathlineto{\pgfqpoint{1.599026in}{0.824707in}}%
\pgfpathlineto{\pgfqpoint{1.601603in}{0.825645in}}%
\pgfpathlineto{\pgfqpoint{1.604180in}{0.822267in}}%
\pgfpathlineto{\pgfqpoint{1.606757in}{0.822830in}}%
\pgfpathlineto{\pgfqpoint{1.609334in}{0.822080in}}%
\pgfpathlineto{\pgfqpoint{1.617066in}{0.817014in}}%
\pgfpathlineto{\pgfqpoint{1.619643in}{0.813073in}}%
\pgfpathlineto{\pgfqpoint{1.622220in}{0.814762in}}%
\pgfpathlineto{\pgfqpoint{1.624798in}{0.821517in}}%
\pgfpathlineto{\pgfqpoint{1.627375in}{0.814950in}}%
\pgfpathlineto{\pgfqpoint{1.635106in}{0.816451in}}%
\pgfpathlineto{\pgfqpoint{1.637684in}{0.817577in}}%
\pgfpathlineto{\pgfqpoint{1.640261in}{0.813261in}}%
\pgfpathlineto{\pgfqpoint{1.642838in}{0.811948in}}%
\pgfpathlineto{\pgfqpoint{1.645415in}{0.815137in}}%
\pgfpathlineto{\pgfqpoint{1.653147in}{0.812698in}}%
\pgfpathlineto{\pgfqpoint{1.655724in}{0.803316in}}%
\pgfpathlineto{\pgfqpoint{1.658301in}{0.801252in}}%
\pgfpathlineto{\pgfqpoint{1.660879in}{0.799939in}}%
\pgfpathlineto{\pgfqpoint{1.663456in}{0.796937in}}%
\pgfpathlineto{\pgfqpoint{1.671187in}{0.800690in}}%
\pgfpathlineto{\pgfqpoint{1.673765in}{0.810259in}}%
\pgfpathlineto{\pgfqpoint{1.676342in}{0.801628in}}%
\pgfpathlineto{\pgfqpoint{1.678919in}{0.799751in}}%
\pgfpathlineto{\pgfqpoint{1.681496in}{0.793747in}}%
\pgfpathlineto{\pgfqpoint{1.689228in}{0.796562in}}%
\pgfpathlineto{\pgfqpoint{1.691805in}{0.798626in}}%
\pgfpathlineto{\pgfqpoint{1.694382in}{0.796562in}}%
\pgfpathlineto{\pgfqpoint{1.699537in}{0.804067in}}%
\pgfpathlineto{\pgfqpoint{1.712423in}{0.799188in}}%
\pgfpathlineto{\pgfqpoint{1.715000in}{0.808758in}}%
\pgfpathlineto{\pgfqpoint{1.717577in}{0.805756in}}%
\pgfpathlineto{\pgfqpoint{1.725309in}{0.805756in}}%
\pgfpathlineto{\pgfqpoint{1.727886in}{0.804067in}}%
\pgfpathlineto{\pgfqpoint{1.730463in}{0.791495in}}%
\pgfpathlineto{\pgfqpoint{1.733041in}{0.790182in}}%
\pgfpathlineto{\pgfqpoint{1.735618in}{0.789807in}}%
\pgfpathlineto{\pgfqpoint{1.743349in}{0.789432in}}%
\pgfpathlineto{\pgfqpoint{1.745927in}{0.781739in}}%
\pgfpathlineto{\pgfqpoint{1.748504in}{0.776860in}}%
\pgfpathlineto{\pgfqpoint{1.751081in}{0.777986in}}%
\pgfpathlineto{\pgfqpoint{1.753658in}{0.780425in}}%
\pgfpathlineto{\pgfqpoint{1.761390in}{0.781363in}}%
\pgfpathlineto{\pgfqpoint{1.763967in}{0.769918in}}%
\pgfpathlineto{\pgfqpoint{1.766544in}{0.767666in}}%
\pgfpathlineto{\pgfqpoint{1.771699in}{0.773107in}}%
\pgfpathlineto{\pgfqpoint{1.782008in}{0.776485in}}%
\pgfpathlineto{\pgfqpoint{1.784585in}{0.778924in}}%
\pgfpathlineto{\pgfqpoint{1.787162in}{0.770293in}}%
\pgfpathlineto{\pgfqpoint{1.789739in}{0.770856in}}%
\pgfpathlineto{\pgfqpoint{1.797471in}{0.770481in}}%
\pgfpathlineto{\pgfqpoint{1.800048in}{0.777048in}}%
\pgfpathlineto{\pgfqpoint{1.802625in}{0.775171in}}%
\pgfpathlineto{\pgfqpoint{1.805203in}{0.779862in}}%
\pgfpathlineto{\pgfqpoint{1.807780in}{0.779862in}}%
\pgfpathlineto{\pgfqpoint{1.815511in}{0.778736in}}%
\pgfpathlineto{\pgfqpoint{1.818089in}{0.787555in}}%
\pgfpathlineto{\pgfqpoint{1.820666in}{0.787743in}}%
\pgfpathlineto{\pgfqpoint{1.823243in}{0.785116in}}%
\pgfpathlineto{\pgfqpoint{1.825820in}{0.785679in}}%
\pgfpathlineto{\pgfqpoint{1.833552in}{0.786242in}}%
\pgfpathlineto{\pgfqpoint{1.836129in}{0.792246in}}%
\pgfpathlineto{\pgfqpoint{1.838706in}{0.794498in}}%
\pgfpathlineto{\pgfqpoint{1.841283in}{0.793372in}}%
\pgfpathlineto{\pgfqpoint{1.843861in}{0.789432in}}%
\pgfpathlineto{\pgfqpoint{1.851592in}{0.787368in}}%
\pgfpathlineto{\pgfqpoint{1.856747in}{0.787555in}}%
\pgfpathlineto{\pgfqpoint{1.859324in}{0.785304in}}%
\pgfpathlineto{\pgfqpoint{1.861901in}{0.780988in}}%
\pgfpathlineto{\pgfqpoint{1.869633in}{0.786992in}}%
\pgfpathlineto{\pgfqpoint{1.874787in}{0.798813in}}%
\pgfpathlineto{\pgfqpoint{1.877364in}{0.797875in}}%
\pgfpathlineto{\pgfqpoint{1.879942in}{0.795436in}}%
\pgfpathlineto{\pgfqpoint{1.887673in}{0.796749in}}%
\pgfpathlineto{\pgfqpoint{1.890251in}{0.794310in}}%
\pgfpathlineto{\pgfqpoint{1.895405in}{0.805193in}}%
\pgfpathlineto{\pgfqpoint{1.897982in}{0.808195in}}%
\pgfpathlineto{\pgfqpoint{1.905714in}{0.808195in}}%
\pgfpathlineto{\pgfqpoint{1.908291in}{0.806506in}}%
\pgfpathlineto{\pgfqpoint{1.910868in}{0.810071in}}%
\pgfpathlineto{\pgfqpoint{1.913445in}{0.818702in}}%
\pgfpathlineto{\pgfqpoint{1.916023in}{0.796749in}}%
\pgfpathlineto{\pgfqpoint{1.926332in}{0.795436in}}%
\pgfpathlineto{\pgfqpoint{1.928909in}{0.794498in}}%
\pgfpathlineto{\pgfqpoint{1.931486in}{0.792058in}}%
\pgfpathlineto{\pgfqpoint{1.934063in}{0.792246in}}%
\pgfpathlineto{\pgfqpoint{1.941795in}{0.793747in}}%
\pgfpathlineto{\pgfqpoint{1.944372in}{0.797124in}}%
\pgfpathlineto{\pgfqpoint{1.946949in}{0.798626in}}%
\pgfpathlineto{\pgfqpoint{1.949526in}{0.793559in}}%
\pgfpathlineto{\pgfqpoint{1.952104in}{0.798438in}}%
\pgfpathlineto{\pgfqpoint{1.959835in}{0.795436in}}%
\pgfpathlineto{\pgfqpoint{1.962412in}{0.799188in}}%
\pgfpathlineto{\pgfqpoint{1.967567in}{0.793372in}}%
\pgfpathlineto{\pgfqpoint{1.970144in}{0.796374in}}%
\pgfpathlineto{\pgfqpoint{1.977876in}{0.796749in}}%
\pgfpathlineto{\pgfqpoint{1.980453in}{0.799376in}}%
\pgfpathlineto{\pgfqpoint{1.983030in}{0.800314in}}%
\pgfpathlineto{\pgfqpoint{1.985607in}{0.799939in}}%
\pgfpathlineto{\pgfqpoint{1.988185in}{0.798250in}}%
\pgfpathlineto{\pgfqpoint{1.998493in}{0.797687in}}%
\pgfpathlineto{\pgfqpoint{2.001071in}{0.792246in}}%
\pgfpathlineto{\pgfqpoint{2.003648in}{0.784741in}}%
\pgfpathlineto{\pgfqpoint{2.006225in}{0.787368in}}%
\pgfpathlineto{\pgfqpoint{2.013957in}{0.784365in}}%
\pgfpathlineto{\pgfqpoint{2.019111in}{0.795248in}}%
\pgfpathlineto{\pgfqpoint{2.021688in}{0.794498in}}%
\pgfpathlineto{\pgfqpoint{2.024266in}{0.796749in}}%
\pgfpathlineto{\pgfqpoint{2.031997in}{0.800502in}}%
\pgfpathlineto{\pgfqpoint{2.037152in}{0.808007in}}%
\pgfpathlineto{\pgfqpoint{2.039729in}{0.810259in}}%
\pgfpathlineto{\pgfqpoint{2.042306in}{0.805380in}}%
\pgfpathlineto{\pgfqpoint{2.050038in}{0.807069in}}%
\pgfpathlineto{\pgfqpoint{2.052615in}{0.806319in}}%
\pgfpathlineto{\pgfqpoint{2.055192in}{0.806694in}}%
\pgfpathlineto{\pgfqpoint{2.057769in}{0.806506in}}%
\pgfpathlineto{\pgfqpoint{2.060347in}{0.802191in}}%
\pgfpathlineto{\pgfqpoint{2.068078in}{0.800502in}}%
\pgfpathlineto{\pgfqpoint{2.070655in}{0.798626in}}%
\pgfpathlineto{\pgfqpoint{2.073233in}{0.799188in}}%
\pgfpathlineto{\pgfqpoint{2.075810in}{0.796937in}}%
\pgfpathlineto{\pgfqpoint{2.078387in}{0.801440in}}%
\pgfpathlineto{\pgfqpoint{2.086119in}{0.798626in}}%
\pgfpathlineto{\pgfqpoint{2.088696in}{0.808382in}}%
\pgfpathlineto{\pgfqpoint{2.091273in}{0.809321in}}%
\pgfpathlineto{\pgfqpoint{2.093850in}{0.809508in}}%
\pgfpathlineto{\pgfqpoint{2.104159in}{0.803129in}}%
\pgfpathlineto{\pgfqpoint{2.106736in}{0.803504in}}%
\pgfpathlineto{\pgfqpoint{2.109314in}{0.797124in}}%
\pgfpathlineto{\pgfqpoint{2.111891in}{0.798626in}}%
\pgfpathlineto{\pgfqpoint{2.114468in}{0.795436in}}%
\pgfpathlineto{\pgfqpoint{2.122200in}{0.797687in}}%
\pgfpathlineto{\pgfqpoint{2.124777in}{0.808007in}}%
\pgfpathlineto{\pgfqpoint{2.127354in}{0.815888in}}%
\pgfpathlineto{\pgfqpoint{2.129931in}{0.809321in}}%
\pgfpathlineto{\pgfqpoint{2.132509in}{0.806881in}}%
\pgfpathlineto{\pgfqpoint{2.140240in}{0.802191in}}%
\pgfpathlineto{\pgfqpoint{2.142817in}{0.810634in}}%
\pgfpathlineto{\pgfqpoint{2.145395in}{0.810822in}}%
\pgfpathlineto{\pgfqpoint{2.147972in}{0.815700in}}%
\pgfpathlineto{\pgfqpoint{2.150549in}{0.818702in}}%
\pgfpathlineto{\pgfqpoint{2.158281in}{0.825645in}}%
\pgfpathlineto{\pgfqpoint{2.160858in}{0.833338in}}%
\pgfpathlineto{\pgfqpoint{2.163435in}{0.837841in}}%
\pgfpathlineto{\pgfqpoint{2.166012in}{0.833338in}}%
\pgfpathlineto{\pgfqpoint{2.168589in}{0.833713in}}%
\pgfpathlineto{\pgfqpoint{2.176321in}{0.839342in}}%
\pgfpathlineto{\pgfqpoint{2.178898in}{0.842344in}}%
\pgfpathlineto{\pgfqpoint{2.181476in}{0.842907in}}%
\pgfpathlineto{\pgfqpoint{2.184053in}{0.844783in}}%
\pgfpathlineto{\pgfqpoint{2.186630in}{0.845909in}}%
\pgfpathlineto{\pgfqpoint{2.194362in}{0.845159in}}%
\pgfpathlineto{\pgfqpoint{2.196939in}{0.848911in}}%
\pgfpathlineto{\pgfqpoint{2.202093in}{0.852289in}}%
\pgfpathlineto{\pgfqpoint{2.204670in}{0.854353in}}%
\pgfpathlineto{\pgfqpoint{2.212402in}{0.847786in}}%
\pgfpathlineto{\pgfqpoint{2.214979in}{0.844033in}}%
\pgfpathlineto{\pgfqpoint{2.217557in}{0.849662in}}%
\pgfpathlineto{\pgfqpoint{2.220134in}{0.845722in}}%
\pgfpathlineto{\pgfqpoint{2.222711in}{0.847223in}}%
\pgfpathlineto{\pgfqpoint{2.230443in}{0.847786in}}%
\pgfpathlineto{\pgfqpoint{2.233020in}{0.848911in}}%
\pgfpathlineto{\pgfqpoint{2.235597in}{0.847598in}}%
\pgfpathlineto{\pgfqpoint{2.238174in}{0.847410in}}%
\pgfpathlineto{\pgfqpoint{2.240751in}{0.845346in}}%
\pgfpathlineto{\pgfqpoint{2.251060in}{0.847786in}}%
\pgfpathlineto{\pgfqpoint{2.253637in}{0.850788in}}%
\pgfpathlineto{\pgfqpoint{2.256215in}{0.849850in}}%
\pgfpathlineto{\pgfqpoint{2.258792in}{0.850975in}}%
\pgfpathlineto{\pgfqpoint{2.266524in}{0.865986in}}%
\pgfpathlineto{\pgfqpoint{2.269101in}{0.867862in}}%
\pgfpathlineto{\pgfqpoint{2.271678in}{0.857542in}}%
\pgfpathlineto{\pgfqpoint{2.276832in}{0.855854in}}%
\pgfpathlineto{\pgfqpoint{2.284564in}{0.862421in}}%
\pgfpathlineto{\pgfqpoint{2.289718in}{0.853790in}}%
\pgfpathlineto{\pgfqpoint{2.292296in}{0.862046in}}%
\pgfpathlineto{\pgfqpoint{2.294873in}{0.861108in}}%
\pgfpathlineto{\pgfqpoint{2.302605in}{0.863922in}}%
\pgfpathlineto{\pgfqpoint{2.305182in}{0.869739in}}%
\pgfpathlineto{\pgfqpoint{2.307759in}{0.862233in}}%
\pgfpathlineto{\pgfqpoint{2.310336in}{0.849474in}}%
\pgfpathlineto{\pgfqpoint{2.312913in}{0.849662in}}%
\pgfpathlineto{\pgfqpoint{2.320645in}{0.839905in}}%
\pgfpathlineto{\pgfqpoint{2.323222in}{0.844596in}}%
\pgfpathlineto{\pgfqpoint{2.325799in}{0.846660in}}%
\pgfpathlineto{\pgfqpoint{2.328377in}{0.847410in}}%
\pgfpathlineto{\pgfqpoint{2.330954in}{0.850225in}}%
\pgfpathlineto{\pgfqpoint{2.338685in}{0.844783in}}%
\pgfpathlineto{\pgfqpoint{2.341263in}{0.842157in}}%
\pgfpathlineto{\pgfqpoint{2.343840in}{0.842719in}}%
\pgfpathlineto{\pgfqpoint{2.348994in}{0.847598in}}%
\pgfpathlineto{\pgfqpoint{2.356726in}{0.833901in}}%
\pgfpathlineto{\pgfqpoint{2.359303in}{0.833150in}}%
\pgfpathlineto{\pgfqpoint{2.361880in}{0.834839in}}%
\pgfpathlineto{\pgfqpoint{2.364458in}{0.846472in}}%
\pgfpathlineto{\pgfqpoint{2.367035in}{0.844971in}}%
\pgfpathlineto{\pgfqpoint{2.374766in}{0.845722in}}%
\pgfpathlineto{\pgfqpoint{2.377344in}{0.850412in}}%
\pgfpathlineto{\pgfqpoint{2.379921in}{0.848911in}}%
\pgfpathlineto{\pgfqpoint{2.382498in}{0.834651in}}%
\pgfpathlineto{\pgfqpoint{2.385075in}{0.831461in}}%
\pgfpathlineto{\pgfqpoint{2.392807in}{0.827333in}}%
\pgfpathlineto{\pgfqpoint{2.395384in}{0.826958in}}%
\pgfpathlineto{\pgfqpoint{2.403116in}{0.835026in}}%
\pgfpathlineto{\pgfqpoint{2.410847in}{0.834651in}}%
\pgfpathlineto{\pgfqpoint{2.413425in}{0.836903in}}%
\pgfpathlineto{\pgfqpoint{2.416002in}{0.836152in}}%
\pgfpathlineto{\pgfqpoint{2.418579in}{0.834088in}}%
\pgfpathlineto{\pgfqpoint{2.421156in}{0.834276in}}%
\pgfpathlineto{\pgfqpoint{2.428888in}{0.833150in}}%
\pgfpathlineto{\pgfqpoint{2.434042in}{0.829585in}}%
\pgfpathlineto{\pgfqpoint{2.436620in}{0.825645in}}%
\pgfpathlineto{\pgfqpoint{2.439197in}{0.826583in}}%
\pgfpathlineto{\pgfqpoint{2.446928in}{0.828647in}}%
\pgfpathlineto{\pgfqpoint{2.449506in}{0.826771in}}%
\pgfpathlineto{\pgfqpoint{2.452083in}{0.827521in}}%
\pgfpathlineto{\pgfqpoint{2.454660in}{0.819078in}}%
\pgfpathlineto{\pgfqpoint{2.457237in}{0.817014in}}%
\pgfpathlineto{\pgfqpoint{2.464969in}{0.823018in}}%
\pgfpathlineto{\pgfqpoint{2.467546in}{0.826771in}}%
\pgfpathlineto{\pgfqpoint{2.470123in}{0.821142in}}%
\pgfpathlineto{\pgfqpoint{2.472701in}{0.822643in}}%
\pgfpathlineto{\pgfqpoint{2.475278in}{0.825457in}}%
\pgfpathlineto{\pgfqpoint{2.485587in}{0.821517in}}%
\pgfpathlineto{\pgfqpoint{2.488164in}{0.823206in}}%
\pgfpathlineto{\pgfqpoint{2.490741in}{0.819453in}}%
\pgfpathlineto{\pgfqpoint{2.493318in}{0.818139in}}%
\pgfpathlineto{\pgfqpoint{2.503627in}{0.819641in}}%
\pgfpathlineto{\pgfqpoint{2.506204in}{0.828647in}}%
\pgfpathlineto{\pgfqpoint{2.508782in}{0.828084in}}%
\pgfpathlineto{\pgfqpoint{2.521668in}{0.834088in}}%
\pgfpathlineto{\pgfqpoint{2.526822in}{0.828459in}}%
\pgfpathlineto{\pgfqpoint{2.529399in}{0.841406in}}%
\pgfpathlineto{\pgfqpoint{2.537131in}{0.840655in}}%
\pgfpathlineto{\pgfqpoint{2.539708in}{0.846097in}}%
\pgfpathlineto{\pgfqpoint{2.542285in}{0.848724in}}%
\pgfpathlineto{\pgfqpoint{2.544862in}{0.848911in}}%
\pgfpathlineto{\pgfqpoint{2.547440in}{0.846660in}}%
\pgfpathlineto{\pgfqpoint{2.555171in}{0.844221in}}%
\pgfpathlineto{\pgfqpoint{2.557749in}{0.845534in}}%
\pgfpathlineto{\pgfqpoint{2.560326in}{0.845534in}}%
\pgfpathlineto{\pgfqpoint{2.562903in}{0.840843in}}%
\pgfpathlineto{\pgfqpoint{2.565480in}{0.834088in}}%
\pgfpathlineto{\pgfqpoint{2.573212in}{0.833150in}}%
\pgfpathlineto{\pgfqpoint{2.575789in}{0.831649in}}%
\pgfpathlineto{\pgfqpoint{2.578366in}{0.832587in}}%
\pgfpathlineto{\pgfqpoint{2.580943in}{0.828084in}}%
\pgfpathlineto{\pgfqpoint{2.583521in}{0.831461in}}%
\pgfpathlineto{\pgfqpoint{2.591252in}{0.831649in}}%
\pgfpathlineto{\pgfqpoint{2.593830in}{0.826208in}}%
\pgfpathlineto{\pgfqpoint{2.596407in}{0.827896in}}%
\pgfpathlineto{\pgfqpoint{2.598984in}{0.835965in}}%
\pgfpathlineto{\pgfqpoint{2.601561in}{0.838592in}}%
\pgfpathlineto{\pgfqpoint{2.609293in}{0.841594in}}%
\pgfpathlineto{\pgfqpoint{2.611870in}{0.840655in}}%
\pgfpathlineto{\pgfqpoint{2.614447in}{0.845534in}}%
\pgfpathlineto{\pgfqpoint{2.617024in}{0.848911in}}%
\pgfpathlineto{\pgfqpoint{2.619602in}{0.848348in}}%
\pgfpathlineto{\pgfqpoint{2.627333in}{0.852476in}}%
\pgfpathlineto{\pgfqpoint{2.629911in}{0.851351in}}%
\pgfpathlineto{\pgfqpoint{2.632488in}{0.846097in}}%
\pgfpathlineto{\pgfqpoint{2.635065in}{0.846847in}}%
\pgfpathlineto{\pgfqpoint{2.637642in}{0.854165in}}%
\pgfpathlineto{\pgfqpoint{2.645374in}{0.856041in}}%
\pgfpathlineto{\pgfqpoint{2.647951in}{0.858481in}}%
\pgfpathlineto{\pgfqpoint{2.653105in}{0.857730in}}%
\pgfpathlineto{\pgfqpoint{2.655683in}{0.855479in}}%
\pgfpathlineto{\pgfqpoint{2.665991in}{0.854353in}}%
\pgfpathlineto{\pgfqpoint{2.668569in}{0.857918in}}%
\pgfpathlineto{\pgfqpoint{2.671146in}{0.854728in}}%
\pgfpathlineto{\pgfqpoint{2.681455in}{0.856604in}}%
\pgfpathlineto{\pgfqpoint{2.684032in}{0.860732in}}%
\pgfpathlineto{\pgfqpoint{2.686609in}{0.863359in}}%
\pgfpathlineto{\pgfqpoint{2.689186in}{0.859982in}}%
\pgfpathlineto{\pgfqpoint{2.691764in}{0.862233in}}%
\pgfpathlineto{\pgfqpoint{2.699495in}{0.863359in}}%
\pgfpathlineto{\pgfqpoint{2.702072in}{0.865048in}}%
\pgfpathlineto{\pgfqpoint{2.704650in}{0.862796in}}%
\pgfpathlineto{\pgfqpoint{2.707227in}{0.873491in}}%
\pgfpathlineto{\pgfqpoint{2.709804in}{0.851726in}}%
\pgfpathlineto{\pgfqpoint{2.717536in}{0.849850in}}%
\pgfpathlineto{\pgfqpoint{2.720113in}{0.848161in}}%
\pgfpathlineto{\pgfqpoint{2.722690in}{0.852289in}}%
\pgfpathlineto{\pgfqpoint{2.727845in}{0.851351in}}%
\pgfpathlineto{\pgfqpoint{2.735576in}{0.849099in}}%
\pgfpathlineto{\pgfqpoint{2.738153in}{0.846660in}}%
\pgfpathlineto{\pgfqpoint{2.740731in}{0.849662in}}%
\pgfpathlineto{\pgfqpoint{2.745885in}{0.866924in}}%
\pgfpathlineto{\pgfqpoint{2.753617in}{0.868801in}}%
\pgfpathlineto{\pgfqpoint{2.756194in}{0.866924in}}%
\pgfpathlineto{\pgfqpoint{2.758771in}{0.860545in}}%
\pgfpathlineto{\pgfqpoint{2.761348in}{0.861295in}}%
\pgfpathlineto{\pgfqpoint{2.763926in}{0.858481in}}%
\pgfpathlineto{\pgfqpoint{2.771657in}{0.861108in}}%
\pgfpathlineto{\pgfqpoint{2.774234in}{0.864485in}}%
\pgfpathlineto{\pgfqpoint{2.776812in}{0.872178in}}%
\pgfpathlineto{\pgfqpoint{2.779389in}{0.871990in}}%
\pgfpathlineto{\pgfqpoint{2.781966in}{0.870864in}}%
\pgfpathlineto{\pgfqpoint{2.792275in}{0.876681in}}%
\pgfpathlineto{\pgfqpoint{2.797429in}{0.880997in}}%
\pgfpathlineto{\pgfqpoint{2.800007in}{0.879496in}}%
\pgfpathlineto{\pgfqpoint{2.810315in}{0.885125in}}%
\pgfpathlineto{\pgfqpoint{2.815470in}{0.882498in}}%
\pgfpathlineto{\pgfqpoint{2.818047in}{0.882310in}}%
\pgfpathlineto{\pgfqpoint{2.825779in}{0.877244in}}%
\pgfpathlineto{\pgfqpoint{2.828356in}{0.879308in}}%
\pgfpathlineto{\pgfqpoint{2.833510in}{0.874805in}}%
\pgfpathlineto{\pgfqpoint{2.836088in}{0.878370in}}%
\pgfpathlineto{\pgfqpoint{2.843819in}{0.877807in}}%
\pgfpathlineto{\pgfqpoint{2.846396in}{0.893943in}}%
\pgfpathlineto{\pgfqpoint{2.848974in}{0.896570in}}%
\pgfpathlineto{\pgfqpoint{2.851551in}{0.894506in}}%
\pgfpathlineto{\pgfqpoint{2.854128in}{0.883436in}}%
\pgfpathlineto{\pgfqpoint{2.864437in}{0.879308in}}%
\pgfpathlineto{\pgfqpoint{2.867014in}{0.874805in}}%
\pgfpathlineto{\pgfqpoint{2.869591in}{0.871990in}}%
\pgfpathlineto{\pgfqpoint{2.872168in}{0.866737in}}%
\pgfpathlineto{\pgfqpoint{2.879900in}{0.865423in}}%
\pgfpathlineto{\pgfqpoint{2.882477in}{0.868238in}}%
\pgfpathlineto{\pgfqpoint{2.885055in}{0.864673in}}%
\pgfpathlineto{\pgfqpoint{2.887632in}{0.865611in}}%
\pgfpathlineto{\pgfqpoint{2.897941in}{0.853039in}}%
\pgfpathlineto{\pgfqpoint{2.900518in}{0.850975in}}%
\pgfpathlineto{\pgfqpoint{2.903095in}{0.849662in}}%
\pgfpathlineto{\pgfqpoint{2.905672in}{0.857355in}}%
\pgfpathlineto{\pgfqpoint{2.908249in}{0.860920in}}%
\pgfpathlineto{\pgfqpoint{2.915981in}{0.862233in}}%
\pgfpathlineto{\pgfqpoint{2.918558in}{0.865048in}}%
\pgfpathlineto{\pgfqpoint{2.921136in}{0.866361in}}%
\pgfpathlineto{\pgfqpoint{2.923713in}{0.868801in}}%
\pgfpathlineto{\pgfqpoint{2.926290in}{0.869739in}}%
\pgfpathlineto{\pgfqpoint{2.936599in}{0.869739in}}%
\pgfpathlineto{\pgfqpoint{2.939176in}{0.865611in}}%
\pgfpathlineto{\pgfqpoint{2.941753in}{0.869363in}}%
\pgfpathlineto{\pgfqpoint{2.944330in}{0.864297in}}%
\pgfpathlineto{\pgfqpoint{2.952062in}{0.867675in}}%
\pgfpathlineto{\pgfqpoint{2.954639in}{0.867487in}}%
\pgfpathlineto{\pgfqpoint{2.957216in}{0.870489in}}%
\pgfpathlineto{\pgfqpoint{2.959794in}{0.869739in}}%
\pgfpathlineto{\pgfqpoint{2.962371in}{0.869739in}}%
\pgfpathlineto{\pgfqpoint{2.970103in}{0.865611in}}%
\pgfpathlineto{\pgfqpoint{2.972680in}{0.867299in}}%
\pgfpathlineto{\pgfqpoint{2.975257in}{0.865611in}}%
\pgfpathlineto{\pgfqpoint{2.977834in}{0.867675in}}%
\pgfpathlineto{\pgfqpoint{2.980411in}{0.867862in}}%
\pgfpathlineto{\pgfqpoint{2.988143in}{0.871052in}}%
\pgfpathlineto{\pgfqpoint{2.990720in}{0.869363in}}%
\pgfpathlineto{\pgfqpoint{2.993297in}{0.869739in}}%
\pgfpathlineto{\pgfqpoint{2.995875in}{0.866737in}}%
\pgfpathlineto{\pgfqpoint{2.998452in}{0.865611in}}%
\pgfpathlineto{\pgfqpoint{3.006184in}{0.868801in}}%
\pgfpathlineto{\pgfqpoint{3.008761in}{0.870677in}}%
\pgfpathlineto{\pgfqpoint{3.011338in}{0.874054in}}%
\pgfpathlineto{\pgfqpoint{3.013915in}{0.880621in}}%
\pgfpathlineto{\pgfqpoint{3.016492in}{0.876493in}}%
\pgfpathlineto{\pgfqpoint{3.024224in}{0.875555in}}%
\pgfpathlineto{\pgfqpoint{3.026801in}{0.881184in}}%
\pgfpathlineto{\pgfqpoint{3.031956in}{0.878745in}}%
\pgfpathlineto{\pgfqpoint{3.034533in}{0.883624in}}%
\pgfpathlineto{\pgfqpoint{3.042264in}{0.886813in}}%
\pgfpathlineto{\pgfqpoint{3.044842in}{0.889628in}}%
\pgfpathlineto{\pgfqpoint{3.047419in}{0.888127in}}%
\pgfpathlineto{\pgfqpoint{3.049996in}{0.896383in}}%
\pgfpathlineto{\pgfqpoint{3.052573in}{0.892442in}}%
\pgfpathlineto{\pgfqpoint{3.060305in}{0.897696in}}%
\pgfpathlineto{\pgfqpoint{3.062882in}{0.904451in}}%
\pgfpathlineto{\pgfqpoint{3.065459in}{0.905577in}}%
\pgfpathlineto{\pgfqpoint{3.068037in}{0.896758in}}%
\pgfpathlineto{\pgfqpoint{3.070614in}{0.892818in}}%
\pgfpathlineto{\pgfqpoint{3.078345in}{0.898822in}}%
\pgfpathlineto{\pgfqpoint{3.083500in}{0.904826in}}%
\pgfpathlineto{\pgfqpoint{3.086077in}{0.906702in}}%
\pgfpathlineto{\pgfqpoint{3.096386in}{0.905201in}}%
\pgfpathlineto{\pgfqpoint{3.101540in}{0.902012in}}%
\pgfpathlineto{\pgfqpoint{3.104118in}{0.902012in}}%
\pgfpathlineto{\pgfqpoint{3.106695in}{0.893943in}}%
\pgfpathlineto{\pgfqpoint{3.114426in}{0.895069in}}%
\pgfpathlineto{\pgfqpoint{3.117004in}{0.897508in}}%
\pgfpathlineto{\pgfqpoint{3.119581in}{0.900886in}}%
\pgfpathlineto{\pgfqpoint{3.122158in}{0.897133in}}%
\pgfpathlineto{\pgfqpoint{3.124735in}{0.896383in}}%
\pgfpathlineto{\pgfqpoint{3.132467in}{0.896195in}}%
\pgfpathlineto{\pgfqpoint{3.135044in}{0.896758in}}%
\pgfpathlineto{\pgfqpoint{3.137621in}{0.899385in}}%
\pgfpathlineto{\pgfqpoint{3.142776in}{0.898259in}}%
\pgfpathlineto{\pgfqpoint{3.150507in}{0.899385in}}%
\pgfpathlineto{\pgfqpoint{3.153085in}{0.900698in}}%
\pgfpathlineto{\pgfqpoint{3.155662in}{0.898822in}}%
\pgfpathlineto{\pgfqpoint{3.158239in}{0.893568in}}%
\pgfpathlineto{\pgfqpoint{3.160816in}{0.890566in}}%
\pgfpathlineto{\pgfqpoint{3.168548in}{0.894131in}}%
\pgfpathlineto{\pgfqpoint{3.171125in}{0.894131in}}%
\pgfpathlineto{\pgfqpoint{3.173702in}{0.896758in}}%
\pgfpathlineto{\pgfqpoint{3.176280in}{0.895820in}}%
\pgfpathlineto{\pgfqpoint{3.178857in}{0.898071in}}%
\pgfpathlineto{\pgfqpoint{3.189166in}{0.905014in}}%
\pgfpathlineto{\pgfqpoint{3.191743in}{0.907828in}}%
\pgfpathlineto{\pgfqpoint{3.194320in}{0.909142in}}%
\pgfpathlineto{\pgfqpoint{3.196897in}{0.914958in}}%
\pgfpathlineto{\pgfqpoint{3.204629in}{0.914020in}}%
\pgfpathlineto{\pgfqpoint{3.207206in}{0.920587in}}%
\pgfpathlineto{\pgfqpoint{3.209783in}{0.919462in}}%
\pgfpathlineto{\pgfqpoint{3.212361in}{0.920587in}}%
\pgfpathlineto{\pgfqpoint{3.214938in}{0.928843in}}%
\pgfpathlineto{\pgfqpoint{3.222669in}{0.924528in}}%
\pgfpathlineto{\pgfqpoint{3.225247in}{0.929969in}}%
\pgfpathlineto{\pgfqpoint{3.227824in}{0.924903in}}%
\pgfpathlineto{\pgfqpoint{3.230401in}{0.925466in}}%
\pgfpathlineto{\pgfqpoint{3.232978in}{0.956613in}}%
\pgfpathlineto{\pgfqpoint{3.240710in}{0.958865in}}%
\pgfpathlineto{\pgfqpoint{3.245864in}{0.957551in}}%
\pgfpathlineto{\pgfqpoint{3.248441in}{0.960178in}}%
\pgfpathlineto{\pgfqpoint{3.251019in}{0.961867in}}%
\pgfpathlineto{\pgfqpoint{3.258750in}{0.962430in}}%
\pgfpathlineto{\pgfqpoint{3.261328in}{0.966745in}}%
\pgfpathlineto{\pgfqpoint{3.263905in}{0.972937in}}%
\pgfpathlineto{\pgfqpoint{3.266482in}{0.971436in}}%
\pgfpathlineto{\pgfqpoint{3.269059in}{0.973875in}}%
\pgfpathlineto{\pgfqpoint{3.276791in}{0.973312in}}%
\pgfpathlineto{\pgfqpoint{3.279368in}{0.974626in}}%
\pgfpathlineto{\pgfqpoint{3.281945in}{0.974626in}}%
\pgfpathlineto{\pgfqpoint{3.284522in}{0.977253in}}%
\pgfpathlineto{\pgfqpoint{3.294831in}{0.975376in}}%
\pgfpathlineto{\pgfqpoint{3.297409in}{0.971624in}}%
\pgfpathlineto{\pgfqpoint{3.299986in}{0.973125in}}%
\pgfpathlineto{\pgfqpoint{3.302563in}{0.979129in}}%
\pgfpathlineto{\pgfqpoint{3.305140in}{0.978941in}}%
\pgfpathlineto{\pgfqpoint{3.312872in}{0.982882in}}%
\pgfpathlineto{\pgfqpoint{3.315449in}{0.986447in}}%
\pgfpathlineto{\pgfqpoint{3.318026in}{1.034481in}}%
\pgfpathlineto{\pgfqpoint{3.320603in}{1.018907in}}%
\pgfpathlineto{\pgfqpoint{3.323181in}{1.018907in}}%
\pgfpathlineto{\pgfqpoint{3.330912in}{1.024912in}}%
\pgfpathlineto{\pgfqpoint{3.333490in}{1.036733in}}%
\pgfpathlineto{\pgfqpoint{3.338644in}{1.027914in}}%
\pgfpathlineto{\pgfqpoint{3.348953in}{1.027539in}}%
\pgfpathlineto{\pgfqpoint{3.351530in}{1.026976in}}%
\pgfpathlineto{\pgfqpoint{3.354107in}{1.029602in}}%
\pgfpathlineto{\pgfqpoint{3.356684in}{1.022097in}}%
\pgfpathlineto{\pgfqpoint{3.359262in}{1.019658in}}%
\pgfpathlineto{\pgfqpoint{3.366993in}{1.024724in}}%
\pgfpathlineto{\pgfqpoint{3.369570in}{1.008212in}}%
\pgfpathlineto{\pgfqpoint{3.372148in}{1.008775in}}%
\pgfpathlineto{\pgfqpoint{3.374725in}{1.005961in}}%
\pgfpathlineto{\pgfqpoint{3.377302in}{1.004647in}}%
\pgfpathlineto{\pgfqpoint{3.387611in}{1.013278in}}%
\pgfpathlineto{\pgfqpoint{3.390188in}{1.029227in}}%
\pgfpathlineto{\pgfqpoint{3.392765in}{1.026600in}}%
\pgfpathlineto{\pgfqpoint{3.395343in}{1.030353in}}%
\pgfpathlineto{\pgfqpoint{3.403074in}{1.034293in}}%
\pgfpathlineto{\pgfqpoint{3.405651in}{1.033168in}}%
\pgfpathlineto{\pgfqpoint{3.408229in}{1.035794in}}%
\pgfpathlineto{\pgfqpoint{3.410806in}{1.046490in}}%
\pgfpathlineto{\pgfqpoint{3.413383in}{1.042924in}}%
\pgfpathlineto{\pgfqpoint{3.423692in}{1.040673in}}%
\pgfpathlineto{\pgfqpoint{3.426269in}{1.040485in}}%
\pgfpathlineto{\pgfqpoint{3.428846in}{1.038234in}}%
\pgfpathlineto{\pgfqpoint{3.431424in}{1.042737in}}%
\pgfpathlineto{\pgfqpoint{3.441732in}{1.036920in}}%
\pgfpathlineto{\pgfqpoint{3.444310in}{1.036920in}}%
\pgfpathlineto{\pgfqpoint{3.446887in}{1.042549in}}%
\pgfpathlineto{\pgfqpoint{3.449464in}{1.044050in}}%
\pgfpathlineto{\pgfqpoint{3.457196in}{1.049304in}}%
\pgfpathlineto{\pgfqpoint{3.459773in}{1.042549in}}%
\pgfpathlineto{\pgfqpoint{3.462350in}{1.044238in}}%
\pgfpathlineto{\pgfqpoint{3.464927in}{1.044238in}}%
\pgfpathlineto{\pgfqpoint{3.467505in}{1.037671in}}%
\pgfpathlineto{\pgfqpoint{3.475236in}{1.036357in}}%
\pgfpathlineto{\pgfqpoint{3.477813in}{1.042737in}}%
\pgfpathlineto{\pgfqpoint{3.480391in}{1.043675in}}%
\pgfpathlineto{\pgfqpoint{3.482968in}{1.046677in}}%
\pgfpathlineto{\pgfqpoint{3.485545in}{1.041048in}}%
\pgfpathlineto{\pgfqpoint{3.493277in}{1.039172in}}%
\pgfpathlineto{\pgfqpoint{3.495854in}{1.034481in}}%
\pgfpathlineto{\pgfqpoint{3.498431in}{1.039922in}}%
\pgfpathlineto{\pgfqpoint{3.501008in}{1.029790in}}%
\pgfpathlineto{\pgfqpoint{3.503586in}{1.031854in}}%
\pgfpathlineto{\pgfqpoint{3.511317in}{1.042362in}}%
\pgfpathlineto{\pgfqpoint{3.513894in}{1.041048in}}%
\pgfpathlineto{\pgfqpoint{3.516472in}{1.027351in}}%
\pgfpathlineto{\pgfqpoint{3.519049in}{1.019658in}}%
\pgfpathlineto{\pgfqpoint{3.521626in}{1.028101in}}%
\pgfpathlineto{\pgfqpoint{3.529358in}{1.029415in}}%
\pgfpathlineto{\pgfqpoint{3.531935in}{1.018907in}}%
\pgfpathlineto{\pgfqpoint{3.534512in}{1.032042in}}%
\pgfpathlineto{\pgfqpoint{3.537089in}{1.021347in}}%
\pgfpathlineto{\pgfqpoint{3.539667in}{0.993202in}}%
\pgfpathlineto{\pgfqpoint{3.547398in}{0.986071in}}%
\pgfpathlineto{\pgfqpoint{3.549975in}{0.996954in}}%
\pgfpathlineto{\pgfqpoint{3.552553in}{0.982882in}}%
\pgfpathlineto{\pgfqpoint{3.555130in}{0.975939in}}%
\pgfpathlineto{\pgfqpoint{3.557707in}{0.984570in}}%
\pgfpathlineto{\pgfqpoint{3.565439in}{0.987760in}}%
\pgfpathlineto{\pgfqpoint{3.568016in}{1.004835in}}%
\pgfpathlineto{\pgfqpoint{3.570593in}{0.999206in}}%
\pgfpathlineto{\pgfqpoint{3.575747in}{1.014029in}}%
\pgfpathlineto{\pgfqpoint{3.583479in}{1.014404in}}%
\pgfpathlineto{\pgfqpoint{3.586056in}{1.023223in}}%
\pgfpathlineto{\pgfqpoint{3.588634in}{1.026225in}}%
\pgfpathlineto{\pgfqpoint{3.591211in}{1.004272in}}%
\pgfpathlineto{\pgfqpoint{3.593788in}{1.027726in}}%
\pgfpathlineto{\pgfqpoint{3.601520in}{1.032605in}}%
\pgfpathlineto{\pgfqpoint{3.604097in}{1.036357in}}%
\pgfpathlineto{\pgfqpoint{3.606674in}{1.027163in}}%
\pgfpathlineto{\pgfqpoint{3.609251in}{1.028289in}}%
\pgfpathlineto{\pgfqpoint{3.611828in}{1.024349in}}%
\pgfpathlineto{\pgfqpoint{3.619560in}{1.018907in}}%
\pgfpathlineto{\pgfqpoint{3.624715in}{1.020971in}}%
\pgfpathlineto{\pgfqpoint{3.629869in}{1.030353in}}%
\pgfpathlineto{\pgfqpoint{3.637601in}{1.035231in}}%
\pgfpathlineto{\pgfqpoint{3.640178in}{1.042924in}}%
\pgfpathlineto{\pgfqpoint{3.642755in}{1.036920in}}%
\pgfpathlineto{\pgfqpoint{3.645332in}{1.063377in}}%
\pgfpathlineto{\pgfqpoint{3.647909in}{1.057560in}}%
\pgfpathlineto{\pgfqpoint{3.655641in}{1.068443in}}%
\pgfpathlineto{\pgfqpoint{3.658218in}{1.069568in}}%
\pgfpathlineto{\pgfqpoint{3.660795in}{1.079138in}}%
\pgfpathlineto{\pgfqpoint{3.665950in}{1.084954in}}%
\pgfpathlineto{\pgfqpoint{3.673682in}{1.083641in}}%
\pgfpathlineto{\pgfqpoint{3.676259in}{1.090771in}}%
\pgfpathlineto{\pgfqpoint{3.678836in}{1.087957in}}%
\pgfpathlineto{\pgfqpoint{3.681413in}{1.088332in}}%
\pgfpathlineto{\pgfqpoint{3.683990in}{1.091897in}}%
\pgfpathlineto{\pgfqpoint{3.691722in}{1.084016in}}%
\pgfpathlineto{\pgfqpoint{3.694299in}{1.078950in}}%
\pgfpathlineto{\pgfqpoint{3.696876in}{1.071257in}}%
\pgfpathlineto{\pgfqpoint{3.699454in}{1.075760in}}%
\pgfpathlineto{\pgfqpoint{3.702031in}{1.068067in}}%
\pgfpathlineto{\pgfqpoint{3.709763in}{1.063001in}}%
\pgfpathlineto{\pgfqpoint{3.712340in}{1.056997in}}%
\pgfpathlineto{\pgfqpoint{3.717494in}{1.081202in}}%
\pgfpathlineto{\pgfqpoint{3.720071in}{1.070319in}}%
\pgfpathlineto{\pgfqpoint{3.730380in}{1.087957in}}%
\pgfpathlineto{\pgfqpoint{3.732957in}{1.088144in}}%
\pgfpathlineto{\pgfqpoint{3.738112in}{1.089833in}}%
\pgfpathlineto{\pgfqpoint{3.745843in}{1.083829in}}%
\pgfpathlineto{\pgfqpoint{3.750998in}{1.069006in}}%
\pgfpathlineto{\pgfqpoint{3.756152in}{1.070131in}}%
\pgfpathlineto{\pgfqpoint{3.763884in}{1.063377in}}%
\pgfpathlineto{\pgfqpoint{3.766461in}{1.052306in}}%
\pgfpathlineto{\pgfqpoint{3.771616in}{1.075573in}}%
\pgfpathlineto{\pgfqpoint{3.774193in}{1.076886in}}%
\pgfpathlineto{\pgfqpoint{3.781924in}{1.074259in}}%
\pgfpathlineto{\pgfqpoint{3.784502in}{1.072571in}}%
\pgfpathlineto{\pgfqpoint{3.789656in}{1.067317in}}%
\pgfpathlineto{\pgfqpoint{3.792233in}{1.071632in}}%
\pgfpathlineto{\pgfqpoint{3.802542in}{1.065816in}}%
\pgfpathlineto{\pgfqpoint{3.807697in}{1.079325in}}%
\pgfpathlineto{\pgfqpoint{3.810274in}{1.071632in}}%
\pgfpathlineto{\pgfqpoint{3.818005in}{1.061125in}}%
\pgfpathlineto{\pgfqpoint{3.820583in}{1.034293in}}%
\pgfpathlineto{\pgfqpoint{3.823160in}{1.027539in}}%
\pgfpathlineto{\pgfqpoint{3.825737in}{1.034669in}}%
\pgfpathlineto{\pgfqpoint{3.828314in}{1.015342in}}%
\pgfpathlineto{\pgfqpoint{3.836046in}{1.025475in}}%
\pgfpathlineto{\pgfqpoint{3.838623in}{1.026225in}}%
\pgfpathlineto{\pgfqpoint{3.841200in}{1.028664in}}%
\pgfpathlineto{\pgfqpoint{3.843778in}{1.034293in}}%
\pgfpathlineto{\pgfqpoint{3.846355in}{1.023598in}}%
\pgfpathlineto{\pgfqpoint{3.854086in}{1.017406in}}%
\pgfpathlineto{\pgfqpoint{3.856664in}{1.030165in}}%
\pgfpathlineto{\pgfqpoint{3.859241in}{1.027726in}}%
\pgfpathlineto{\pgfqpoint{3.861818in}{1.037483in}}%
\pgfpathlineto{\pgfqpoint{3.864395in}{1.041236in}}%
\pgfpathlineto{\pgfqpoint{3.874704in}{1.047615in}}%
\pgfpathlineto{\pgfqpoint{3.877281in}{1.039735in}}%
\pgfpathlineto{\pgfqpoint{3.879859in}{1.038797in}}%
\pgfpathlineto{\pgfqpoint{3.882436in}{1.041986in}}%
\pgfpathlineto{\pgfqpoint{3.890167in}{1.031291in}}%
\pgfpathlineto{\pgfqpoint{3.892745in}{1.041986in}}%
\pgfpathlineto{\pgfqpoint{3.895322in}{1.034481in}}%
\pgfpathlineto{\pgfqpoint{3.897899in}{1.029415in}}%
\pgfpathlineto{\pgfqpoint{3.900476in}{1.022848in}}%
\pgfpathlineto{\pgfqpoint{3.908208in}{1.036170in}}%
\pgfpathlineto{\pgfqpoint{3.913362in}{1.037295in}}%
\pgfpathlineto{\pgfqpoint{3.915940in}{1.030728in}}%
\pgfpathlineto{\pgfqpoint{3.918517in}{1.021722in}}%
\pgfpathlineto{\pgfqpoint{3.926248in}{1.014029in}}%
\pgfpathlineto{\pgfqpoint{3.928826in}{0.996954in}}%
\pgfpathlineto{\pgfqpoint{3.931403in}{1.007462in}}%
\pgfpathlineto{\pgfqpoint{3.933980in}{0.981944in}}%
\pgfpathlineto{\pgfqpoint{3.936557in}{0.984195in}}%
\pgfpathlineto{\pgfqpoint{3.944289in}{0.982506in}}%
\pgfpathlineto{\pgfqpoint{3.946866in}{0.978566in}}%
\pgfpathlineto{\pgfqpoint{3.949443in}{0.983445in}}%
\pgfpathlineto{\pgfqpoint{3.952020in}{0.981005in}}%
\pgfpathlineto{\pgfqpoint{3.954598in}{0.990575in}}%
\pgfpathlineto{\pgfqpoint{3.962329in}{0.988698in}}%
\pgfpathlineto{\pgfqpoint{3.964907in}{0.981756in}}%
\pgfpathlineto{\pgfqpoint{3.967484in}{0.966933in}}%
\pgfpathlineto{\pgfqpoint{3.970061in}{0.969935in}}%
\pgfpathlineto{\pgfqpoint{3.972638in}{1.002020in}}%
\pgfpathlineto{\pgfqpoint{3.982947in}{0.989824in}}%
\pgfpathlineto{\pgfqpoint{3.985524in}{0.982131in}}%
\pgfpathlineto{\pgfqpoint{3.988101in}{0.982131in}}%
\pgfpathlineto{\pgfqpoint{3.998410in}{0.986071in}}%
\pgfpathlineto{\pgfqpoint{4.000988in}{0.989637in}}%
\pgfpathlineto{\pgfqpoint{4.003565in}{0.990575in}}%
\pgfpathlineto{\pgfqpoint{4.006142in}{0.989261in}}%
\pgfpathlineto{\pgfqpoint{4.008719in}{1.000895in}}%
\pgfpathlineto{\pgfqpoint{4.016451in}{0.997517in}}%
\pgfpathlineto{\pgfqpoint{4.019028in}{0.993577in}}%
\pgfpathlineto{\pgfqpoint{4.021605in}{1.015718in}}%
\pgfpathlineto{\pgfqpoint{4.024182in}{1.016468in}}%
\pgfpathlineto{\pgfqpoint{4.026760in}{1.009713in}}%
\pgfpathlineto{\pgfqpoint{4.034491in}{1.014029in}}%
\pgfpathlineto{\pgfqpoint{4.037069in}{1.009150in}}%
\pgfpathlineto{\pgfqpoint{4.039646in}{1.013654in}}%
\pgfpathlineto{\pgfqpoint{4.044800in}{1.003334in}}%
\pgfpathlineto{\pgfqpoint{4.052532in}{1.010276in}}%
\pgfpathlineto{\pgfqpoint{4.055109in}{1.018907in}}%
\pgfpathlineto{\pgfqpoint{4.057686in}{1.016843in}}%
\pgfpathlineto{\pgfqpoint{4.060263in}{1.011214in}}%
\pgfpathlineto{\pgfqpoint{4.062841in}{1.025662in}}%
\pgfpathlineto{\pgfqpoint{4.070572in}{1.025850in}}%
\pgfpathlineto{\pgfqpoint{4.073149in}{1.016468in}}%
\pgfpathlineto{\pgfqpoint{4.075727in}{1.009526in}}%
\pgfpathlineto{\pgfqpoint{4.078304in}{1.009901in}}%
\pgfpathlineto{\pgfqpoint{4.080881in}{1.019283in}}%
\pgfpathlineto{\pgfqpoint{4.088613in}{1.017406in}}%
\pgfpathlineto{\pgfqpoint{4.091190in}{1.010089in}}%
\pgfpathlineto{\pgfqpoint{4.096344in}{1.022097in}}%
\pgfpathlineto{\pgfqpoint{4.098922in}{1.022472in}}%
\pgfpathlineto{\pgfqpoint{4.106653in}{1.029415in}}%
\pgfpathlineto{\pgfqpoint{4.109230in}{1.025099in}}%
\pgfpathlineto{\pgfqpoint{4.114385in}{1.031854in}}%
\pgfpathlineto{\pgfqpoint{4.116962in}{1.030165in}}%
\pgfpathlineto{\pgfqpoint{4.127271in}{1.024349in}}%
\pgfpathlineto{\pgfqpoint{4.129848in}{1.034481in}}%
\pgfpathlineto{\pgfqpoint{4.132425in}{1.039547in}}%
\pgfpathlineto{\pgfqpoint{4.135003in}{1.047052in}}%
\pgfpathlineto{\pgfqpoint{4.142734in}{1.037858in}}%
\pgfpathlineto{\pgfqpoint{4.147889in}{1.017969in}}%
\pgfpathlineto{\pgfqpoint{4.153043in}{1.003146in}}%
\pgfpathlineto{\pgfqpoint{4.160775in}{0.994140in}}%
\pgfpathlineto{\pgfqpoint{4.163352in}{0.993202in}}%
\pgfpathlineto{\pgfqpoint{4.165929in}{1.002771in}}%
\pgfpathlineto{\pgfqpoint{4.168506in}{1.003334in}}%
\pgfpathlineto{\pgfqpoint{4.171084in}{0.994515in}}%
\pgfpathlineto{\pgfqpoint{4.178815in}{0.995641in}}%
\pgfpathlineto{\pgfqpoint{4.181392in}{0.999769in}}%
\pgfpathlineto{\pgfqpoint{4.183970in}{1.005022in}}%
\pgfpathlineto{\pgfqpoint{4.186547in}{1.012153in}}%
\pgfpathlineto{\pgfqpoint{4.189124in}{1.007086in}}%
\pgfpathlineto{\pgfqpoint{4.196856in}{1.010089in}}%
\pgfpathlineto{\pgfqpoint{4.202010in}{1.004272in}}%
\pgfpathlineto{\pgfqpoint{4.204587in}{1.005585in}}%
\pgfpathlineto{\pgfqpoint{4.207165in}{0.989449in}}%
\pgfpathlineto{\pgfqpoint{4.214896in}{0.978754in}}%
\pgfpathlineto{\pgfqpoint{4.217473in}{0.979317in}}%
\pgfpathlineto{\pgfqpoint{4.220051in}{0.975376in}}%
\pgfpathlineto{\pgfqpoint{4.222628in}{0.981756in}}%
\pgfpathlineto{\pgfqpoint{4.235514in}{0.970686in}}%
\pgfpathlineto{\pgfqpoint{4.238091in}{0.963931in}}%
\pgfpathlineto{\pgfqpoint{4.240668in}{0.954361in}}%
\pgfpathlineto{\pgfqpoint{4.243246in}{0.958302in}}%
\pgfpathlineto{\pgfqpoint{4.250977in}{0.967683in}}%
\pgfpathlineto{\pgfqpoint{4.253554in}{0.966370in}}%
\pgfpathlineto{\pgfqpoint{4.256132in}{0.967121in}}%
\pgfpathlineto{\pgfqpoint{4.258709in}{0.970686in}}%
\pgfpathlineto{\pgfqpoint{4.261286in}{0.963368in}}%
\pgfpathlineto{\pgfqpoint{4.269018in}{0.957176in}}%
\pgfpathlineto{\pgfqpoint{4.271595in}{0.950796in}}%
\pgfpathlineto{\pgfqpoint{4.274172in}{0.948920in}}%
\pgfpathlineto{\pgfqpoint{4.276749in}{0.948920in}}%
\pgfpathlineto{\pgfqpoint{4.279326in}{0.939726in}}%
\pgfpathlineto{\pgfqpoint{4.287058in}{0.944604in}}%
\pgfpathlineto{\pgfqpoint{4.289635in}{0.954924in}}%
\pgfpathlineto{\pgfqpoint{4.292213in}{0.955675in}}%
\pgfpathlineto{\pgfqpoint{4.294790in}{0.953986in}}%
\pgfpathlineto{\pgfqpoint{4.307676in}{0.957739in}}%
\pgfpathlineto{\pgfqpoint{4.310253in}{0.961679in}}%
\pgfpathlineto{\pgfqpoint{4.315407in}{0.957551in}}%
\pgfpathlineto{\pgfqpoint{4.323139in}{0.970310in}}%
\pgfpathlineto{\pgfqpoint{4.325716in}{0.959052in}}%
\pgfpathlineto{\pgfqpoint{4.328294in}{0.967308in}}%
\pgfpathlineto{\pgfqpoint{4.330871in}{0.957364in}}%
\pgfpathlineto{\pgfqpoint{4.333448in}{0.959990in}}%
\pgfpathlineto{\pgfqpoint{4.341180in}{0.960929in}}%
\pgfpathlineto{\pgfqpoint{4.343757in}{0.958114in}}%
\pgfpathlineto{\pgfqpoint{4.346334in}{0.948732in}}%
\pgfpathlineto{\pgfqpoint{4.351488in}{0.918336in}}%
\pgfpathlineto{\pgfqpoint{4.359220in}{0.913082in}}%
\pgfpathlineto{\pgfqpoint{4.361797in}{0.906702in}}%
\pgfpathlineto{\pgfqpoint{4.364374in}{0.930907in}}%
\pgfpathlineto{\pgfqpoint{4.366952in}{0.938037in}}%
\pgfpathlineto{\pgfqpoint{4.369529in}{0.949858in}}%
\pgfpathlineto{\pgfqpoint{4.377261in}{0.951735in}}%
\pgfpathlineto{\pgfqpoint{4.379838in}{0.939726in}}%
\pgfpathlineto{\pgfqpoint{4.382415in}{0.952860in}}%
\pgfpathlineto{\pgfqpoint{4.384992in}{0.960929in}}%
\pgfpathlineto{\pgfqpoint{4.387569in}{0.951547in}}%
\pgfpathlineto{\pgfqpoint{4.397878in}{0.968059in}}%
\pgfpathlineto{\pgfqpoint{4.400455in}{0.963555in}}%
\pgfpathlineto{\pgfqpoint{4.403033in}{0.964118in}}%
\pgfpathlineto{\pgfqpoint{4.405610in}{0.967496in}}%
\pgfpathlineto{\pgfqpoint{4.413342in}{0.966182in}}%
\pgfpathlineto{\pgfqpoint{4.415919in}{0.971999in}}%
\pgfpathlineto{\pgfqpoint{4.418496in}{0.972562in}}%
\pgfpathlineto{\pgfqpoint{4.421073in}{0.971624in}}%
\pgfpathlineto{\pgfqpoint{4.423650in}{0.959990in}}%
\pgfpathlineto{\pgfqpoint{4.431382in}{0.962242in}}%
\pgfpathlineto{\pgfqpoint{4.433959in}{0.953986in}}%
\pgfpathlineto{\pgfqpoint{4.436536in}{0.955112in}}%
\pgfpathlineto{\pgfqpoint{4.439114in}{0.950796in}}%
\pgfpathlineto{\pgfqpoint{4.441691in}{0.956425in}}%
\pgfpathlineto{\pgfqpoint{4.449422in}{0.955487in}}%
\pgfpathlineto{\pgfqpoint{4.452000in}{0.963555in}}%
\pgfpathlineto{\pgfqpoint{4.454577in}{0.978754in}}%
\pgfpathlineto{\pgfqpoint{4.457154in}{0.976502in}}%
\pgfpathlineto{\pgfqpoint{4.459731in}{0.985133in}}%
\pgfpathlineto{\pgfqpoint{4.467463in}{0.996954in}}%
\pgfpathlineto{\pgfqpoint{4.472617in}{1.015718in}}%
\pgfpathlineto{\pgfqpoint{4.475195in}{1.019095in}}%
\pgfpathlineto{\pgfqpoint{4.477772in}{1.012715in}}%
\pgfpathlineto{\pgfqpoint{4.485503in}{1.013841in}}%
\pgfpathlineto{\pgfqpoint{4.488081in}{1.011027in}}%
\pgfpathlineto{\pgfqpoint{4.490658in}{1.023786in}}%
\pgfpathlineto{\pgfqpoint{4.493235in}{1.023035in}}%
\pgfpathlineto{\pgfqpoint{4.495812in}{1.027914in}}%
\pgfpathlineto{\pgfqpoint{4.503544in}{1.037108in}}%
\pgfpathlineto{\pgfqpoint{4.506121in}{1.034669in}}%
\pgfpathlineto{\pgfqpoint{4.508698in}{1.033355in}}%
\pgfpathlineto{\pgfqpoint{4.511276in}{1.050805in}}%
\pgfpathlineto{\pgfqpoint{4.513853in}{1.059249in}}%
\pgfpathlineto{\pgfqpoint{4.524162in}{1.051931in}}%
\pgfpathlineto{\pgfqpoint{4.526739in}{1.056059in}}%
\pgfpathlineto{\pgfqpoint{4.529316in}{1.044613in}}%
\pgfpathlineto{\pgfqpoint{4.531893in}{1.041799in}}%
\pgfpathlineto{\pgfqpoint{4.539625in}{1.045927in}}%
\pgfpathlineto{\pgfqpoint{4.542202in}{1.049304in}}%
\pgfpathlineto{\pgfqpoint{4.544779in}{1.050617in}}%
\pgfpathlineto{\pgfqpoint{4.549934in}{1.045364in}}%
\pgfpathlineto{\pgfqpoint{4.560243in}{1.034669in}}%
\pgfpathlineto{\pgfqpoint{4.562820in}{1.028664in}}%
\pgfpathlineto{\pgfqpoint{4.565397in}{1.020971in}}%
\pgfpathlineto{\pgfqpoint{4.567974in}{1.015905in}}%
\pgfpathlineto{\pgfqpoint{4.575706in}{1.015718in}}%
\pgfpathlineto{\pgfqpoint{4.580860in}{1.033918in}}%
\pgfpathlineto{\pgfqpoint{4.583438in}{1.053244in}}%
\pgfpathlineto{\pgfqpoint{4.586015in}{1.059436in}}%
\pgfpathlineto{\pgfqpoint{4.593746in}{1.056246in}}%
\pgfpathlineto{\pgfqpoint{4.596324in}{1.054182in}}%
\pgfpathlineto{\pgfqpoint{4.598901in}{1.055871in}}%
\pgfpathlineto{\pgfqpoint{4.604055in}{1.055871in}}%
\pgfpathlineto{\pgfqpoint{4.611787in}{1.061313in}}%
\pgfpathlineto{\pgfqpoint{4.614364in}{1.066754in}}%
\pgfpathlineto{\pgfqpoint{4.616941in}{1.062251in}}%
\pgfpathlineto{\pgfqpoint{4.619519in}{1.048741in}}%
\pgfpathlineto{\pgfqpoint{4.622096in}{1.064127in}}%
\pgfpathlineto{\pgfqpoint{4.629827in}{1.064878in}}%
\pgfpathlineto{\pgfqpoint{4.632405in}{1.060937in}}%
\pgfpathlineto{\pgfqpoint{4.634982in}{1.061875in}}%
\pgfpathlineto{\pgfqpoint{4.637559in}{1.061313in}}%
\pgfpathlineto{\pgfqpoint{4.640136in}{1.052681in}}%
\pgfpathlineto{\pgfqpoint{4.647868in}{1.056059in}}%
\pgfpathlineto{\pgfqpoint{4.650445in}{1.068255in}}%
\pgfpathlineto{\pgfqpoint{4.653022in}{1.070319in}}%
\pgfpathlineto{\pgfqpoint{4.655599in}{1.063564in}}%
\pgfpathlineto{\pgfqpoint{4.658177in}{1.045927in}}%
\pgfpathlineto{\pgfqpoint{4.665908in}{1.052119in}}%
\pgfpathlineto{\pgfqpoint{4.668486in}{1.060562in}}%
\pgfpathlineto{\pgfqpoint{4.671063in}{1.065065in}}%
\pgfpathlineto{\pgfqpoint{4.683949in}{1.063939in}}%
\pgfpathlineto{\pgfqpoint{4.686526in}{1.072571in}}%
\pgfpathlineto{\pgfqpoint{4.691680in}{1.055871in}}%
\pgfpathlineto{\pgfqpoint{4.701989in}{1.047991in}}%
\pgfpathlineto{\pgfqpoint{4.704567in}{1.045176in}}%
\pgfpathlineto{\pgfqpoint{4.707144in}{1.032417in}}%
\pgfpathlineto{\pgfqpoint{4.709721in}{1.011402in}}%
\pgfpathlineto{\pgfqpoint{4.712298in}{1.005773in}}%
\pgfpathlineto{\pgfqpoint{4.720030in}{1.015155in}}%
\pgfpathlineto{\pgfqpoint{4.722607in}{1.025662in}}%
\pgfpathlineto{\pgfqpoint{4.725184in}{1.012528in}}%
\pgfpathlineto{\pgfqpoint{4.727761in}{1.026600in}}%
\pgfpathlineto{\pgfqpoint{4.730339in}{0.975939in}}%
\pgfpathlineto{\pgfqpoint{4.740648in}{0.976690in}}%
\pgfpathlineto{\pgfqpoint{4.743225in}{0.973125in}}%
\pgfpathlineto{\pgfqpoint{4.745802in}{0.974251in}}%
\pgfpathlineto{\pgfqpoint{4.748379in}{0.978941in}}%
\pgfpathlineto{\pgfqpoint{4.756111in}{0.973312in}}%
\pgfpathlineto{\pgfqpoint{4.758688in}{0.978941in}}%
\pgfpathlineto{\pgfqpoint{4.761265in}{0.976877in}}%
\pgfpathlineto{\pgfqpoint{4.763842in}{0.979504in}}%
\pgfpathlineto{\pgfqpoint{4.766420in}{0.997330in}}%
\pgfpathlineto{\pgfqpoint{4.774151in}{0.993952in}}%
\pgfpathlineto{\pgfqpoint{4.776728in}{0.976690in}}%
\pgfpathlineto{\pgfqpoint{4.779306in}{0.973125in}}%
\pgfpathlineto{\pgfqpoint{4.781883in}{0.980630in}}%
\pgfpathlineto{\pgfqpoint{4.784460in}{0.968059in}}%
\pgfpathlineto{\pgfqpoint{4.792192in}{0.964306in}}%
\pgfpathlineto{\pgfqpoint{4.794769in}{0.964118in}}%
\pgfpathlineto{\pgfqpoint{4.797346in}{0.954174in}}%
\pgfpathlineto{\pgfqpoint{4.799923in}{0.953986in}}%
\pgfpathlineto{\pgfqpoint{4.802501in}{0.961116in}}%
\pgfpathlineto{\pgfqpoint{4.812809in}{0.963555in}}%
\pgfpathlineto{\pgfqpoint{4.815387in}{0.975376in}}%
\pgfpathlineto{\pgfqpoint{4.817964in}{0.974626in}}%
\pgfpathlineto{\pgfqpoint{4.820541in}{0.962430in}}%
\pgfpathlineto{\pgfqpoint{4.828273in}{0.973312in}}%
\pgfpathlineto{\pgfqpoint{4.830850in}{0.963931in}}%
\pgfpathlineto{\pgfqpoint{4.836004in}{0.978003in}}%
\pgfpathlineto{\pgfqpoint{4.838582in}{0.981193in}}%
\pgfpathlineto{\pgfqpoint{4.846313in}{0.977440in}}%
\pgfpathlineto{\pgfqpoint{4.848890in}{0.990950in}}%
\pgfpathlineto{\pgfqpoint{4.851468in}{0.993764in}}%
\pgfpathlineto{\pgfqpoint{4.856622in}{0.995266in}}%
\pgfpathlineto{\pgfqpoint{4.864354in}{1.000707in}}%
\pgfpathlineto{\pgfqpoint{4.866931in}{0.994140in}}%
\pgfpathlineto{\pgfqpoint{4.869508in}{0.999393in}}%
\pgfpathlineto{\pgfqpoint{4.872085in}{1.005961in}}%
\pgfpathlineto{\pgfqpoint{4.874663in}{1.014779in}}%
\pgfpathlineto{\pgfqpoint{4.882394in}{1.009150in}}%
\pgfpathlineto{\pgfqpoint{4.884971in}{1.012903in}}%
\pgfpathlineto{\pgfqpoint{4.887549in}{1.013466in}}%
\pgfpathlineto{\pgfqpoint{4.890126in}{1.018532in}}%
\pgfpathlineto{\pgfqpoint{4.892703in}{1.030541in}}%
\pgfpathlineto{\pgfqpoint{4.900435in}{1.024724in}}%
\pgfpathlineto{\pgfqpoint{4.903012in}{1.024349in}}%
\pgfpathlineto{\pgfqpoint{4.905589in}{1.018907in}}%
\pgfpathlineto{\pgfqpoint{4.908166in}{1.016843in}}%
\pgfpathlineto{\pgfqpoint{4.918475in}{1.017219in}}%
\pgfpathlineto{\pgfqpoint{4.921052in}{1.025475in}}%
\pgfpathlineto{\pgfqpoint{4.923630in}{1.031104in}}%
\pgfpathlineto{\pgfqpoint{4.926207in}{1.024912in}}%
\pgfpathlineto{\pgfqpoint{4.928784in}{1.026600in}}%
\pgfpathlineto{\pgfqpoint{4.936516in}{1.018907in}}%
\pgfpathlineto{\pgfqpoint{4.939093in}{1.017219in}}%
\pgfpathlineto{\pgfqpoint{4.941670in}{1.020221in}}%
\pgfpathlineto{\pgfqpoint{4.944247in}{1.011214in}}%
\pgfpathlineto{\pgfqpoint{4.946825in}{1.012528in}}%
\pgfpathlineto{\pgfqpoint{4.954556in}{1.013278in}}%
\pgfpathlineto{\pgfqpoint{4.957133in}{1.016468in}}%
\pgfpathlineto{\pgfqpoint{4.959711in}{1.021159in}}%
\pgfpathlineto{\pgfqpoint{4.964865in}{1.009526in}}%
\pgfpathlineto{\pgfqpoint{4.972597in}{1.012903in}}%
\pgfpathlineto{\pgfqpoint{4.975174in}{1.011965in}}%
\pgfpathlineto{\pgfqpoint{4.977751in}{1.018907in}}%
\pgfpathlineto{\pgfqpoint{4.980328in}{1.018344in}}%
\pgfpathlineto{\pgfqpoint{4.982905in}{1.012715in}}%
\pgfpathlineto{\pgfqpoint{4.990637in}{1.008400in}}%
\pgfpathlineto{\pgfqpoint{4.993214in}{1.008588in}}%
\pgfpathlineto{\pgfqpoint{4.995792in}{1.014592in}}%
\pgfpathlineto{\pgfqpoint{4.998369in}{1.003521in}}%
\pgfpathlineto{\pgfqpoint{5.000946in}{0.989261in}}%
\pgfpathlineto{\pgfqpoint{5.008678in}{0.995078in}}%
\pgfpathlineto{\pgfqpoint{5.013832in}{0.986447in}}%
\pgfpathlineto{\pgfqpoint{5.016409in}{0.987197in}}%
\pgfpathlineto{\pgfqpoint{5.018986in}{0.989637in}}%
\pgfpathlineto{\pgfqpoint{5.026718in}{0.985509in}}%
\pgfpathlineto{\pgfqpoint{5.029295in}{0.991325in}}%
\pgfpathlineto{\pgfqpoint{5.031873in}{0.990012in}}%
\pgfpathlineto{\pgfqpoint{5.034450in}{0.984758in}}%
\pgfpathlineto{\pgfqpoint{5.044759in}{0.995828in}}%
\pgfpathlineto{\pgfqpoint{5.047336in}{0.988698in}}%
\pgfpathlineto{\pgfqpoint{5.049913in}{0.988886in}}%
\pgfpathlineto{\pgfqpoint{5.052490in}{0.982506in}}%
\pgfpathlineto{\pgfqpoint{5.055067in}{0.991513in}}%
\pgfpathlineto{\pgfqpoint{5.062799in}{0.993014in}}%
\pgfpathlineto{\pgfqpoint{5.065376in}{1.007274in}}%
\pgfpathlineto{\pgfqpoint{5.067953in}{1.013091in}}%
\pgfpathlineto{\pgfqpoint{5.073108in}{1.016093in}}%
\pgfpathlineto{\pgfqpoint{5.083417in}{1.016468in}}%
\pgfpathlineto{\pgfqpoint{5.088571in}{1.019470in}}%
\pgfpathlineto{\pgfqpoint{5.091148in}{1.017031in}}%
\pgfpathlineto{\pgfqpoint{5.098880in}{1.018157in}}%
\pgfpathlineto{\pgfqpoint{5.101457in}{1.021534in}}%
\pgfpathlineto{\pgfqpoint{5.104034in}{1.021722in}}%
\pgfpathlineto{\pgfqpoint{5.106612in}{1.022660in}}%
\pgfpathlineto{\pgfqpoint{5.109189in}{1.024349in}}%
\pgfpathlineto{\pgfqpoint{5.116921in}{1.026600in}}%
\pgfpathlineto{\pgfqpoint{5.119498in}{1.026037in}}%
\pgfpathlineto{\pgfqpoint{5.122075in}{1.016843in}}%
\pgfpathlineto{\pgfqpoint{5.127229in}{1.019470in}}%
\pgfpathlineto{\pgfqpoint{5.137538in}{1.029227in}}%
\pgfpathlineto{\pgfqpoint{5.140115in}{1.028664in}}%
\pgfpathlineto{\pgfqpoint{5.142693in}{1.040673in}}%
\pgfpathlineto{\pgfqpoint{5.145270in}{1.015905in}}%
\pgfpathlineto{\pgfqpoint{5.153001in}{1.001457in}}%
\pgfpathlineto{\pgfqpoint{5.155579in}{1.009526in}}%
\pgfpathlineto{\pgfqpoint{5.160733in}{1.037483in}}%
\pgfpathlineto{\pgfqpoint{5.163310in}{1.036545in}}%
\pgfpathlineto{\pgfqpoint{5.173619in}{1.035419in}}%
\pgfpathlineto{\pgfqpoint{5.178774in}{1.044426in}}%
\pgfpathlineto{\pgfqpoint{5.181351in}{1.058310in}}%
\pgfpathlineto{\pgfqpoint{5.189082in}{1.064878in}}%
\pgfpathlineto{\pgfqpoint{5.191660in}{1.074447in}}%
\pgfpathlineto{\pgfqpoint{5.194237in}{1.075760in}}%
\pgfpathlineto{\pgfqpoint{5.196814in}{1.078950in}}%
\pgfpathlineto{\pgfqpoint{5.199391in}{1.076699in}}%
\pgfpathlineto{\pgfqpoint{5.207123in}{1.076511in}}%
\pgfpathlineto{\pgfqpoint{5.209700in}{1.078200in}}%
\pgfpathlineto{\pgfqpoint{5.212277in}{1.087581in}}%
\pgfpathlineto{\pgfqpoint{5.214855in}{1.063001in}}%
\pgfpathlineto{\pgfqpoint{5.217432in}{1.069756in}}%
\pgfpathlineto{\pgfqpoint{5.225163in}{1.070131in}}%
\pgfpathlineto{\pgfqpoint{5.227741in}{1.077074in}}%
\pgfpathlineto{\pgfqpoint{5.230318in}{1.072571in}}%
\pgfpathlineto{\pgfqpoint{5.232895in}{1.071632in}}%
\pgfpathlineto{\pgfqpoint{5.235472in}{1.073133in}}%
\pgfpathlineto{\pgfqpoint{5.243204in}{1.072946in}}%
\pgfpathlineto{\pgfqpoint{5.245781in}{1.067880in}}%
\pgfpathlineto{\pgfqpoint{5.248358in}{1.067129in}}%
\pgfpathlineto{\pgfqpoint{5.250936in}{1.072758in}}%
\pgfpathlineto{\pgfqpoint{5.253513in}{1.079888in}}%
\pgfpathlineto{\pgfqpoint{5.261244in}{1.080826in}}%
\pgfpathlineto{\pgfqpoint{5.263822in}{1.078762in}}%
\pgfpathlineto{\pgfqpoint{5.266399in}{1.072008in}}%
\pgfpathlineto{\pgfqpoint{5.268976in}{1.074635in}}%
\pgfpathlineto{\pgfqpoint{5.271553in}{1.072571in}}%
\pgfpathlineto{\pgfqpoint{5.279285in}{1.078575in}}%
\pgfpathlineto{\pgfqpoint{5.281862in}{1.083829in}}%
\pgfpathlineto{\pgfqpoint{5.284439in}{1.080451in}}%
\pgfpathlineto{\pgfqpoint{5.287017in}{1.079701in}}%
\pgfpathlineto{\pgfqpoint{5.289594in}{1.084391in}}%
\pgfpathlineto{\pgfqpoint{5.299903in}{1.087206in}}%
\pgfpathlineto{\pgfqpoint{5.302480in}{1.082703in}}%
\pgfpathlineto{\pgfqpoint{5.305057in}{1.081765in}}%
\pgfpathlineto{\pgfqpoint{5.307634in}{1.084767in}}%
\pgfpathlineto{\pgfqpoint{5.315366in}{1.089833in}}%
\pgfpathlineto{\pgfqpoint{5.323098in}{1.097901in}}%
\pgfpathlineto{\pgfqpoint{5.325675in}{1.099027in}}%
\pgfpathlineto{\pgfqpoint{5.335984in}{1.107470in}}%
\pgfpathlineto{\pgfqpoint{5.338561in}{1.105594in}}%
\pgfpathlineto{\pgfqpoint{5.341138in}{1.105219in}}%
\pgfpathlineto{\pgfqpoint{5.343715in}{1.087769in}}%
\pgfpathlineto{\pgfqpoint{5.351447in}{1.099027in}}%
\pgfpathlineto{\pgfqpoint{5.354024in}{1.090771in}}%
\pgfpathlineto{\pgfqpoint{5.356601in}{1.090959in}}%
\pgfpathlineto{\pgfqpoint{5.361756in}{1.126797in}}%
\pgfpathlineto{\pgfqpoint{5.369487in}{1.117790in}}%
\pgfpathlineto{\pgfqpoint{5.372065in}{1.117415in}}%
\pgfpathlineto{\pgfqpoint{5.374642in}{1.122856in}}%
\pgfpathlineto{\pgfqpoint{5.377219in}{1.124733in}}%
\pgfpathlineto{\pgfqpoint{5.379796in}{1.118353in}}%
\pgfpathlineto{\pgfqpoint{5.387528in}{1.108971in}}%
\pgfpathlineto{\pgfqpoint{5.390105in}{1.118166in}}%
\pgfpathlineto{\pgfqpoint{5.392682in}{1.122669in}}%
\pgfpathlineto{\pgfqpoint{5.395259in}{1.120605in}}%
\pgfpathlineto{\pgfqpoint{5.397837in}{1.128110in}}%
\pgfpathlineto{\pgfqpoint{5.405568in}{1.126609in}}%
\pgfpathlineto{\pgfqpoint{5.408146in}{1.124545in}}%
\pgfpathlineto{\pgfqpoint{5.410723in}{1.132238in}}%
\pgfpathlineto{\pgfqpoint{5.413300in}{1.133739in}}%
\pgfpathlineto{\pgfqpoint{5.415877in}{1.134302in}}%
\pgfpathlineto{\pgfqpoint{5.423609in}{1.132801in}}%
\pgfpathlineto{\pgfqpoint{5.426186in}{1.119854in}}%
\pgfpathlineto{\pgfqpoint{5.431340in}{1.114601in}}%
\pgfpathlineto{\pgfqpoint{5.433918in}{1.122856in}}%
\pgfpathlineto{\pgfqpoint{5.441649in}{1.120042in}}%
\pgfpathlineto{\pgfqpoint{5.444227in}{1.128110in}}%
\pgfpathlineto{\pgfqpoint{5.446804in}{1.089082in}}%
\pgfpathlineto{\pgfqpoint{5.449381in}{1.087581in}}%
\pgfpathlineto{\pgfqpoint{5.451958in}{1.082703in}}%
\pgfpathlineto{\pgfqpoint{5.459690in}{1.084767in}}%
\pgfpathlineto{\pgfqpoint{5.467421in}{1.076886in}}%
\pgfpathlineto{\pgfqpoint{5.469999in}{1.075573in}}%
\pgfpathlineto{\pgfqpoint{5.477730in}{1.077824in}}%
\pgfpathlineto{\pgfqpoint{5.480307in}{1.071820in}}%
\pgfpathlineto{\pgfqpoint{5.482885in}{1.073133in}}%
\pgfpathlineto{\pgfqpoint{5.488039in}{1.060374in}}%
\pgfpathlineto{\pgfqpoint{5.495771in}{1.079325in}}%
\pgfpathlineto{\pgfqpoint{5.498348in}{1.080264in}}%
\pgfpathlineto{\pgfqpoint{5.500925in}{1.080451in}}%
\pgfpathlineto{\pgfqpoint{5.503502in}{1.075948in}}%
\pgfpathlineto{\pgfqpoint{5.506080in}{1.078012in}}%
\pgfpathlineto{\pgfqpoint{5.513811in}{1.075573in}}%
\pgfpathlineto{\pgfqpoint{5.516388in}{1.083266in}}%
\pgfpathlineto{\pgfqpoint{5.518966in}{1.081952in}}%
\pgfpathlineto{\pgfqpoint{5.521543in}{1.085142in}}%
\pgfpathlineto{\pgfqpoint{5.524120in}{1.083829in}}%
\pgfpathlineto{\pgfqpoint{5.531852in}{1.084391in}}%
\pgfpathlineto{\pgfqpoint{5.534429in}{1.093210in}}%
\pgfpathlineto{\pgfqpoint{5.537006in}{1.088332in}}%
\pgfpathlineto{\pgfqpoint{5.542161in}{1.092460in}}%
\pgfpathlineto{\pgfqpoint{5.549892in}{1.093773in}}%
\pgfpathlineto{\pgfqpoint{5.552469in}{1.090208in}}%
\pgfpathlineto{\pgfqpoint{5.555047in}{1.079513in}}%
\pgfpathlineto{\pgfqpoint{5.557624in}{1.063001in}}%
\pgfpathlineto{\pgfqpoint{5.560201in}{1.069944in}}%
\pgfpathlineto{\pgfqpoint{5.567933in}{1.074072in}}%
\pgfpathlineto{\pgfqpoint{5.570510in}{1.079888in}}%
\pgfpathlineto{\pgfqpoint{5.573087in}{1.093586in}}%
\pgfpathlineto{\pgfqpoint{5.575664in}{1.097151in}}%
\pgfpathlineto{\pgfqpoint{5.578242in}{1.098089in}}%
\pgfpathlineto{\pgfqpoint{5.585973in}{1.101841in}}%
\pgfpathlineto{\pgfqpoint{5.588550in}{1.116477in}}%
\pgfpathlineto{\pgfqpoint{5.591128in}{1.111974in}}%
\pgfpathlineto{\pgfqpoint{5.593705in}{1.116289in}}%
\pgfpathlineto{\pgfqpoint{5.596282in}{1.107846in}}%
\pgfpathlineto{\pgfqpoint{5.604014in}{1.117978in}}%
\pgfpathlineto{\pgfqpoint{5.606591in}{1.123607in}}%
\pgfpathlineto{\pgfqpoint{5.609168in}{1.119667in}}%
\pgfpathlineto{\pgfqpoint{5.611745in}{1.118728in}}%
\pgfpathlineto{\pgfqpoint{5.614323in}{1.119479in}}%
\pgfpathlineto{\pgfqpoint{5.624631in}{1.121168in}}%
\pgfpathlineto{\pgfqpoint{5.627209in}{1.113475in}}%
\pgfpathlineto{\pgfqpoint{5.629786in}{1.114038in}}%
\pgfpathlineto{\pgfqpoint{5.632363in}{1.107095in}}%
\pgfpathlineto{\pgfqpoint{5.642672in}{1.112912in}}%
\pgfpathlineto{\pgfqpoint{5.645249in}{1.109534in}}%
\pgfpathlineto{\pgfqpoint{5.647826in}{1.108596in}}%
\pgfpathlineto{\pgfqpoint{5.650403in}{1.110848in}}%
\pgfpathlineto{\pgfqpoint{5.658135in}{1.113099in}}%
\pgfpathlineto{\pgfqpoint{5.660712in}{1.111786in}}%
\pgfpathlineto{\pgfqpoint{5.663290in}{1.119104in}}%
\pgfpathlineto{\pgfqpoint{5.665867in}{1.114788in}}%
\pgfpathlineto{\pgfqpoint{5.668444in}{1.116289in}}%
\pgfpathlineto{\pgfqpoint{5.678753in}{1.116477in}}%
\pgfpathlineto{\pgfqpoint{5.681330in}{1.115726in}}%
\pgfpathlineto{\pgfqpoint{5.683907in}{1.112349in}}%
\pgfpathlineto{\pgfqpoint{5.686484in}{1.118916in}}%
\pgfpathlineto{\pgfqpoint{5.694216in}{1.115914in}}%
\pgfpathlineto{\pgfqpoint{5.696793in}{1.130925in}}%
\pgfpathlineto{\pgfqpoint{5.699371in}{1.133927in}}%
\pgfpathlineto{\pgfqpoint{5.701948in}{1.129799in}}%
\pgfpathlineto{\pgfqpoint{5.704525in}{1.137117in}}%
\pgfpathlineto{\pgfqpoint{5.712257in}{1.127360in}}%
\pgfpathlineto{\pgfqpoint{5.714834in}{1.116852in}}%
\pgfpathlineto{\pgfqpoint{5.717411in}{1.111598in}}%
\pgfpathlineto{\pgfqpoint{5.719988in}{1.114413in}}%
\pgfpathlineto{\pgfqpoint{5.722565in}{1.116102in}}%
\pgfpathlineto{\pgfqpoint{5.730297in}{1.111598in}}%
\pgfpathlineto{\pgfqpoint{5.732874in}{1.113099in}}%
\pgfpathlineto{\pgfqpoint{5.735452in}{1.113662in}}%
\pgfpathlineto{\pgfqpoint{5.738029in}{1.097338in}}%
\pgfpathlineto{\pgfqpoint{5.740606in}{1.095274in}}%
\pgfpathlineto{\pgfqpoint{5.750915in}{1.105594in}}%
\pgfpathlineto{\pgfqpoint{5.753492in}{1.107846in}}%
\pgfpathlineto{\pgfqpoint{5.756069in}{1.114225in}}%
\pgfpathlineto{\pgfqpoint{5.758646in}{1.115351in}}%
\pgfpathlineto{\pgfqpoint{5.768955in}{1.116102in}}%
\pgfpathlineto{\pgfqpoint{5.771532in}{1.108033in}}%
\pgfpathlineto{\pgfqpoint{5.774110in}{1.110097in}}%
\pgfpathlineto{\pgfqpoint{5.776687in}{1.116289in}}%
\pgfpathlineto{\pgfqpoint{5.784419in}{1.115914in}}%
\pgfpathlineto{\pgfqpoint{5.789573in}{1.105594in}}%
\pgfpathlineto{\pgfqpoint{5.794727in}{1.105031in}}%
\pgfpathlineto{\pgfqpoint{5.802459in}{1.099215in}}%
\pgfpathlineto{\pgfqpoint{5.805036in}{1.103342in}}%
\pgfpathlineto{\pgfqpoint{5.807613in}{1.100153in}}%
\pgfpathlineto{\pgfqpoint{5.810191in}{1.103718in}}%
\pgfpathlineto{\pgfqpoint{5.812768in}{1.105219in}}%
\pgfpathlineto{\pgfqpoint{5.820500in}{1.092084in}}%
\pgfpathlineto{\pgfqpoint{5.823077in}{1.092460in}}%
\pgfpathlineto{\pgfqpoint{5.825654in}{1.090959in}}%
\pgfpathlineto{\pgfqpoint{5.828231in}{1.091709in}}%
\pgfpathlineto{\pgfqpoint{5.830808in}{1.093961in}}%
\pgfpathlineto{\pgfqpoint{5.838540in}{1.096775in}}%
\pgfpathlineto{\pgfqpoint{5.841117in}{1.089833in}}%
\pgfpathlineto{\pgfqpoint{5.843694in}{1.095650in}}%
\pgfpathlineto{\pgfqpoint{5.848849in}{1.092084in}}%
\pgfpathlineto{\pgfqpoint{5.856580in}{1.096025in}}%
\pgfpathlineto{\pgfqpoint{5.859158in}{1.099777in}}%
\pgfpathlineto{\pgfqpoint{5.861735in}{1.099215in}}%
\pgfpathlineto{\pgfqpoint{5.864312in}{1.102404in}}%
\pgfpathlineto{\pgfqpoint{5.866889in}{1.108033in}}%
\pgfpathlineto{\pgfqpoint{5.874621in}{1.109722in}}%
\pgfpathlineto{\pgfqpoint{5.877198in}{1.111786in}}%
\pgfpathlineto{\pgfqpoint{5.879775in}{1.110848in}}%
\pgfpathlineto{\pgfqpoint{5.882353in}{1.107470in}}%
\pgfpathlineto{\pgfqpoint{5.884930in}{1.107470in}}%
\pgfpathlineto{\pgfqpoint{5.897816in}{1.100340in}}%
\pgfpathlineto{\pgfqpoint{5.900393in}{1.093586in}}%
\pgfpathlineto{\pgfqpoint{5.910702in}{1.097713in}}%
\pgfpathlineto{\pgfqpoint{5.913279in}{1.102780in}}%
\pgfpathlineto{\pgfqpoint{5.915856in}{1.105219in}}%
\pgfpathlineto{\pgfqpoint{5.918434in}{1.110097in}}%
\pgfpathlineto{\pgfqpoint{5.931320in}{1.122293in}}%
\pgfpathlineto{\pgfqpoint{5.933897in}{1.123419in}}%
\pgfpathlineto{\pgfqpoint{5.936474in}{1.132238in}}%
\pgfpathlineto{\pgfqpoint{5.939051in}{1.109534in}}%
\pgfpathlineto{\pgfqpoint{5.946783in}{1.112349in}}%
\pgfpathlineto{\pgfqpoint{5.949360in}{1.123982in}}%
\pgfpathlineto{\pgfqpoint{5.951937in}{1.129048in}}%
\pgfpathlineto{\pgfqpoint{5.954515in}{1.126797in}}%
\pgfpathlineto{\pgfqpoint{5.957092in}{1.126234in}}%
\pgfpathlineto{\pgfqpoint{5.964823in}{1.121168in}}%
\pgfpathlineto{\pgfqpoint{5.967401in}{1.118166in}}%
\pgfpathlineto{\pgfqpoint{5.972555in}{1.106157in}}%
\pgfpathlineto{\pgfqpoint{5.975132in}{1.103155in}}%
\pgfpathlineto{\pgfqpoint{5.982864in}{1.105031in}}%
\pgfpathlineto{\pgfqpoint{5.985441in}{1.108409in}}%
\pgfpathlineto{\pgfqpoint{5.988018in}{1.094524in}}%
\pgfpathlineto{\pgfqpoint{5.993173in}{1.100903in}}%
\pgfpathlineto{\pgfqpoint{6.003482in}{1.109159in}}%
\pgfpathlineto{\pgfqpoint{6.006059in}{1.113850in}}%
\pgfpathlineto{\pgfqpoint{6.008636in}{1.116289in}}%
\pgfpathlineto{\pgfqpoint{6.011213in}{1.116289in}}%
\pgfpathlineto{\pgfqpoint{6.021522in}{1.114788in}}%
\pgfpathlineto{\pgfqpoint{6.024099in}{1.113662in}}%
\pgfpathlineto{\pgfqpoint{6.026677in}{1.113850in}}%
\pgfpathlineto{\pgfqpoint{6.029254in}{1.117415in}}%
\pgfpathlineto{\pgfqpoint{6.036985in}{1.117603in}}%
\pgfpathlineto{\pgfqpoint{6.039563in}{1.114038in}}%
\pgfpathlineto{\pgfqpoint{6.042140in}{1.116289in}}%
\pgfpathlineto{\pgfqpoint{6.044717in}{1.120230in}}%
\pgfpathlineto{\pgfqpoint{6.047294in}{1.106532in}}%
\pgfpathlineto{\pgfqpoint{6.055026in}{1.106720in}}%
\pgfpathlineto{\pgfqpoint{6.057603in}{1.109534in}}%
\pgfpathlineto{\pgfqpoint{6.060180in}{1.103155in}}%
\pgfpathlineto{\pgfqpoint{6.062757in}{1.099402in}}%
\pgfpathlineto{\pgfqpoint{6.065335in}{1.097526in}}%
\pgfpathlineto{\pgfqpoint{6.073066in}{1.102967in}}%
\pgfpathlineto{\pgfqpoint{6.075644in}{1.091334in}}%
\pgfpathlineto{\pgfqpoint{6.080798in}{1.082328in}}%
\pgfpathlineto{\pgfqpoint{6.083375in}{1.079325in}}%
\pgfpathlineto{\pgfqpoint{6.091107in}{1.077261in}}%
\pgfpathlineto{\pgfqpoint{6.093684in}{1.069756in}}%
\pgfpathlineto{\pgfqpoint{6.096261in}{1.079513in}}%
\pgfpathlineto{\pgfqpoint{6.098838in}{1.067692in}}%
\pgfpathlineto{\pgfqpoint{6.101416in}{1.071257in}}%
\pgfpathlineto{\pgfqpoint{6.109147in}{1.066379in}}%
\pgfpathlineto{\pgfqpoint{6.114302in}{1.081952in}}%
\pgfpathlineto{\pgfqpoint{6.116879in}{1.069381in}}%
\pgfpathlineto{\pgfqpoint{6.119456in}{1.073884in}}%
\pgfpathlineto{\pgfqpoint{6.127188in}{1.069756in}}%
\pgfpathlineto{\pgfqpoint{6.129765in}{1.074447in}}%
\pgfpathlineto{\pgfqpoint{6.132342in}{1.080451in}}%
\pgfpathlineto{\pgfqpoint{6.134919in}{1.080264in}}%
\pgfpathlineto{\pgfqpoint{6.137497in}{1.088144in}}%
\pgfpathlineto{\pgfqpoint{6.145228in}{1.084391in}}%
\pgfpathlineto{\pgfqpoint{6.150383in}{1.085893in}}%
\pgfpathlineto{\pgfqpoint{6.152960in}{1.089270in}}%
\pgfpathlineto{\pgfqpoint{6.155537in}{1.088895in}}%
\pgfpathlineto{\pgfqpoint{6.163269in}{1.084954in}}%
\pgfpathlineto{\pgfqpoint{6.165846in}{1.087957in}}%
\pgfpathlineto{\pgfqpoint{6.168423in}{1.089270in}}%
\pgfpathlineto{\pgfqpoint{6.171000in}{1.093210in}}%
\pgfpathlineto{\pgfqpoint{6.173578in}{1.099402in}}%
\pgfpathlineto{\pgfqpoint{6.181309in}{1.102217in}}%
\pgfpathlineto{\pgfqpoint{6.183886in}{1.117790in}}%
\pgfpathlineto{\pgfqpoint{6.186464in}{1.123044in}}%
\pgfpathlineto{\pgfqpoint{6.189041in}{1.125296in}}%
\pgfpathlineto{\pgfqpoint{6.191618in}{1.121918in}}%
\pgfpathlineto{\pgfqpoint{6.199350in}{1.124170in}}%
\pgfpathlineto{\pgfqpoint{6.201927in}{1.123795in}}%
\pgfpathlineto{\pgfqpoint{6.204504in}{1.126984in}}%
\pgfpathlineto{\pgfqpoint{6.207081in}{1.118916in}}%
\pgfpathlineto{\pgfqpoint{6.209659in}{1.114038in}}%
\pgfpathlineto{\pgfqpoint{6.217390in}{1.122481in}}%
\pgfpathlineto{\pgfqpoint{6.219967in}{1.116477in}}%
\pgfpathlineto{\pgfqpoint{6.222545in}{1.113099in}}%
\pgfpathlineto{\pgfqpoint{6.225122in}{1.101466in}}%
\pgfpathlineto{\pgfqpoint{6.227699in}{1.098652in}}%
\pgfpathlineto{\pgfqpoint{6.235431in}{1.096963in}}%
\pgfpathlineto{\pgfqpoint{6.238008in}{1.092272in}}%
\pgfpathlineto{\pgfqpoint{6.240585in}{1.092460in}}%
\pgfpathlineto{\pgfqpoint{6.243162in}{1.093210in}}%
\pgfpathlineto{\pgfqpoint{6.245740in}{1.092460in}}%
\pgfpathlineto{\pgfqpoint{6.253471in}{1.092272in}}%
\pgfpathlineto{\pgfqpoint{6.256048in}{1.093586in}}%
\pgfpathlineto{\pgfqpoint{6.261203in}{1.099777in}}%
\pgfpathlineto{\pgfqpoint{6.263780in}{1.100153in}}%
\pgfpathlineto{\pgfqpoint{6.274089in}{1.098839in}}%
\pgfpathlineto{\pgfqpoint{6.276666in}{1.112161in}}%
\pgfpathlineto{\pgfqpoint{6.279243in}{1.108221in}}%
\pgfpathlineto{\pgfqpoint{6.281821in}{1.101841in}}%
\pgfpathlineto{\pgfqpoint{6.289552in}{1.112349in}}%
\pgfpathlineto{\pgfqpoint{6.294707in}{1.122293in}}%
\pgfpathlineto{\pgfqpoint{6.297284in}{1.125108in}}%
\pgfpathlineto{\pgfqpoint{6.299861in}{1.134490in}}%
\pgfpathlineto{\pgfqpoint{6.307593in}{1.134490in}}%
\pgfpathlineto{\pgfqpoint{6.310170in}{1.138618in}}%
\pgfpathlineto{\pgfqpoint{6.312747in}{1.135615in}}%
\pgfpathlineto{\pgfqpoint{6.315324in}{1.138055in}}%
\pgfpathlineto{\pgfqpoint{6.317902in}{1.137679in}}%
\pgfpathlineto{\pgfqpoint{6.325633in}{1.137304in}}%
\pgfpathlineto{\pgfqpoint{6.328210in}{1.142933in}}%
\pgfpathlineto{\pgfqpoint{6.330788in}{1.144059in}}%
\pgfpathlineto{\pgfqpoint{6.343674in}{1.171078in}}%
\pgfpathlineto{\pgfqpoint{6.346251in}{1.177270in}}%
\pgfpathlineto{\pgfqpoint{6.348828in}{1.176520in}}%
\pgfpathlineto{\pgfqpoint{6.351405in}{1.179897in}}%
\pgfpathlineto{\pgfqpoint{6.353982in}{1.181586in}}%
\pgfpathlineto{\pgfqpoint{6.361714in}{1.185901in}}%
\pgfpathlineto{\pgfqpoint{6.364291in}{1.181961in}}%
\pgfpathlineto{\pgfqpoint{6.366869in}{1.175769in}}%
\pgfpathlineto{\pgfqpoint{6.369446in}{1.173705in}}%
\pgfpathlineto{\pgfqpoint{6.372023in}{1.182336in}}%
\pgfpathlineto{\pgfqpoint{6.382332in}{1.184588in}}%
\pgfpathlineto{\pgfqpoint{6.384909in}{1.192844in}}%
\pgfpathlineto{\pgfqpoint{6.387486in}{1.190029in}}%
\pgfpathlineto{\pgfqpoint{6.390063in}{1.196033in}}%
\pgfpathlineto{\pgfqpoint{6.400372in}{1.205415in}}%
\pgfpathlineto{\pgfqpoint{6.402950in}{1.202413in}}%
\pgfpathlineto{\pgfqpoint{6.405527in}{1.212545in}}%
\pgfpathlineto{\pgfqpoint{6.408104in}{1.267334in}}%
\pgfpathlineto{\pgfqpoint{6.415836in}{1.266771in}}%
\pgfpathlineto{\pgfqpoint{6.420990in}{1.308989in}}%
\pgfpathlineto{\pgfqpoint{6.423567in}{1.315931in}}%
\pgfpathlineto{\pgfqpoint{6.426144in}{1.302234in}}%
\pgfpathlineto{\pgfqpoint{6.433876in}{1.313680in}}%
\pgfpathlineto{\pgfqpoint{6.436453in}{1.315181in}}%
\pgfpathlineto{\pgfqpoint{6.439031in}{1.313680in}}%
\pgfpathlineto{\pgfqpoint{6.441608in}{1.306550in}}%
\pgfpathlineto{\pgfqpoint{6.444185in}{1.293415in}}%
\pgfpathlineto{\pgfqpoint{6.451917in}{1.296605in}}%
\pgfpathlineto{\pgfqpoint{6.454494in}{1.298481in}}%
\pgfpathlineto{\pgfqpoint{6.457071in}{1.291351in}}%
\pgfpathlineto{\pgfqpoint{6.459648in}{1.294729in}}%
\pgfpathlineto{\pgfqpoint{6.462225in}{1.276341in}}%
\pgfpathlineto{\pgfqpoint{6.469957in}{1.276153in}}%
\pgfpathlineto{\pgfqpoint{6.472534in}{1.281970in}}%
\pgfpathlineto{\pgfqpoint{6.475111in}{1.276716in}}%
\pgfpathlineto{\pgfqpoint{6.480266in}{1.278405in}}%
\pgfpathlineto{\pgfqpoint{6.487998in}{1.273714in}}%
\pgfpathlineto{\pgfqpoint{6.490575in}{1.278029in}}%
\pgfpathlineto{\pgfqpoint{6.493152in}{1.263957in}}%
\pgfpathlineto{\pgfqpoint{6.495729in}{1.280093in}}%
\pgfpathlineto{\pgfqpoint{6.498306in}{1.277279in}}%
\pgfpathlineto{\pgfqpoint{6.506038in}{1.273714in}}%
\pgfpathlineto{\pgfqpoint{6.508615in}{1.254763in}}%
\pgfpathlineto{\pgfqpoint{6.511192in}{1.254950in}}%
\pgfpathlineto{\pgfqpoint{6.513770in}{1.248196in}}%
\pgfpathlineto{\pgfqpoint{6.516347in}{1.253074in}}%
\pgfpathlineto{\pgfqpoint{6.524079in}{1.258703in}}%
\pgfpathlineto{\pgfqpoint{6.526656in}{1.252699in}}%
\pgfpathlineto{\pgfqpoint{6.529233in}{1.252886in}}%
\pgfpathlineto{\pgfqpoint{6.531810in}{1.251573in}}%
\pgfpathlineto{\pgfqpoint{6.534387in}{1.275027in}}%
\pgfpathlineto{\pgfqpoint{6.542119in}{1.305799in}}%
\pgfpathlineto{\pgfqpoint{6.544696in}{1.319872in}}%
\pgfpathlineto{\pgfqpoint{6.547273in}{1.329253in}}%
\pgfpathlineto{\pgfqpoint{6.549851in}{1.314806in}}%
\pgfpathlineto{\pgfqpoint{6.552428in}{1.313680in}}%
\pgfpathlineto{\pgfqpoint{6.562737in}{1.302422in}}%
\pgfpathlineto{\pgfqpoint{6.565314in}{1.302985in}}%
\pgfpathlineto{\pgfqpoint{6.567891in}{1.305049in}}%
\pgfpathlineto{\pgfqpoint{6.570468in}{1.303923in}}%
\pgfpathlineto{\pgfqpoint{6.570468in}{1.303923in}}%
\pgfusepath{stroke}%
\end{pgfscope}%
\begin{pgfscope}%
\pgfpathrectangle{\pgfqpoint{0.652287in}{0.521603in}}{\pgfqpoint{6.200000in}{4.530000in}}%
\pgfusepath{clip}%
\pgfsetroundcap%
\pgfsetroundjoin%
\pgfsetlinewidth{1.505625pt}%
\definecolor{currentstroke}{rgb}{0.580392,0.403922,0.741176}%
\pgfsetstrokecolor{currentstroke}%
\pgfsetdash{}{0pt}%
\pgfpathmoveto{\pgfqpoint{0.934105in}{1.464912in}}%
\pgfpathlineto{\pgfqpoint{0.936682in}{1.458908in}}%
\pgfpathlineto{\pgfqpoint{0.939259in}{1.457782in}}%
\pgfpathlineto{\pgfqpoint{0.941836in}{1.449151in}}%
\pgfpathlineto{\pgfqpoint{0.949568in}{1.450652in}}%
\pgfpathlineto{\pgfqpoint{0.952145in}{1.454780in}}%
\pgfpathlineto{\pgfqpoint{0.954722in}{1.453654in}}%
\pgfpathlineto{\pgfqpoint{0.957300in}{1.455155in}}%
\pgfpathlineto{\pgfqpoint{0.959877in}{1.455531in}}%
\pgfpathlineto{\pgfqpoint{0.970186in}{1.453467in}}%
\pgfpathlineto{\pgfqpoint{0.972763in}{1.455906in}}%
\pgfpathlineto{\pgfqpoint{0.975340in}{1.454593in}}%
\pgfpathlineto{\pgfqpoint{0.977917in}{1.455718in}}%
\pgfpathlineto{\pgfqpoint{0.985649in}{1.451778in}}%
\pgfpathlineto{\pgfqpoint{0.988226in}{1.451778in}}%
\pgfpathlineto{\pgfqpoint{0.990803in}{1.454968in}}%
\pgfpathlineto{\pgfqpoint{0.993381in}{1.462286in}}%
\pgfpathlineto{\pgfqpoint{0.995958in}{1.460222in}}%
\pgfpathlineto{\pgfqpoint{1.003689in}{1.462473in}}%
\pgfpathlineto{\pgfqpoint{1.006267in}{1.465475in}}%
\pgfpathlineto{\pgfqpoint{1.008844in}{1.462098in}}%
\pgfpathlineto{\pgfqpoint{1.011421in}{1.460597in}}%
\pgfpathlineto{\pgfqpoint{1.013998in}{1.461347in}}%
\pgfpathlineto{\pgfqpoint{1.021730in}{1.454593in}}%
\pgfpathlineto{\pgfqpoint{1.024307in}{1.455531in}}%
\pgfpathlineto{\pgfqpoint{1.026884in}{1.455343in}}%
\pgfpathlineto{\pgfqpoint{1.032039in}{1.445586in}}%
\pgfpathlineto{\pgfqpoint{1.039770in}{1.446900in}}%
\pgfpathlineto{\pgfqpoint{1.042348in}{1.445774in}}%
\pgfpathlineto{\pgfqpoint{1.044925in}{1.446337in}}%
\pgfpathlineto{\pgfqpoint{1.047502in}{1.450465in}}%
\pgfpathlineto{\pgfqpoint{1.050079in}{1.451590in}}%
\pgfpathlineto{\pgfqpoint{1.060388in}{1.452341in}}%
\pgfpathlineto{\pgfqpoint{1.062965in}{1.451778in}}%
\pgfpathlineto{\pgfqpoint{1.065543in}{1.453279in}}%
\pgfpathlineto{\pgfqpoint{1.068120in}{1.452153in}}%
\pgfpathlineto{\pgfqpoint{1.075851in}{1.451966in}}%
\pgfpathlineto{\pgfqpoint{1.078429in}{1.463036in}}%
\pgfpathlineto{\pgfqpoint{1.081006in}{1.461535in}}%
\pgfpathlineto{\pgfqpoint{1.083583in}{1.457782in}}%
\pgfpathlineto{\pgfqpoint{1.086160in}{1.456844in}}%
\pgfpathlineto{\pgfqpoint{1.093892in}{1.458908in}}%
\pgfpathlineto{\pgfqpoint{1.096469in}{1.450465in}}%
\pgfpathlineto{\pgfqpoint{1.099046in}{1.449714in}}%
\pgfpathlineto{\pgfqpoint{1.101624in}{1.458158in}}%
\pgfpathlineto{\pgfqpoint{1.104201in}{1.456469in}}%
\pgfpathlineto{\pgfqpoint{1.111932in}{1.461910in}}%
\pgfpathlineto{\pgfqpoint{1.114510in}{1.465475in}}%
\pgfpathlineto{\pgfqpoint{1.117087in}{1.461535in}}%
\pgfpathlineto{\pgfqpoint{1.119664in}{1.461347in}}%
\pgfpathlineto{\pgfqpoint{1.122241in}{1.462286in}}%
\pgfpathlineto{\pgfqpoint{1.129973in}{1.463599in}}%
\pgfpathlineto{\pgfqpoint{1.137704in}{1.452153in}}%
\pgfpathlineto{\pgfqpoint{1.140282in}{1.453467in}}%
\pgfpathlineto{\pgfqpoint{1.153168in}{1.469791in}}%
\pgfpathlineto{\pgfqpoint{1.155745in}{1.468665in}}%
\pgfpathlineto{\pgfqpoint{1.158322in}{1.475045in}}%
\pgfpathlineto{\pgfqpoint{1.166054in}{1.478797in}}%
\pgfpathlineto{\pgfqpoint{1.168631in}{1.473731in}}%
\pgfpathlineto{\pgfqpoint{1.171208in}{1.466226in}}%
\pgfpathlineto{\pgfqpoint{1.173785in}{1.465475in}}%
\pgfpathlineto{\pgfqpoint{1.184094in}{1.459096in}}%
\pgfpathlineto{\pgfqpoint{1.186672in}{1.448213in}}%
\pgfpathlineto{\pgfqpoint{1.189249in}{1.447087in}}%
\pgfpathlineto{\pgfqpoint{1.191826in}{1.447462in}}%
\pgfpathlineto{\pgfqpoint{1.194403in}{1.438081in}}%
\pgfpathlineto{\pgfqpoint{1.202135in}{1.444836in}}%
\pgfpathlineto{\pgfqpoint{1.204712in}{1.448401in}}%
\pgfpathlineto{\pgfqpoint{1.207289in}{1.433765in}}%
\pgfpathlineto{\pgfqpoint{1.209866in}{1.430388in}}%
\pgfpathlineto{\pgfqpoint{1.212444in}{1.440708in}}%
\pgfpathlineto{\pgfqpoint{1.220175in}{1.435454in}}%
\pgfpathlineto{\pgfqpoint{1.222753in}{1.441646in}}%
\pgfpathlineto{\pgfqpoint{1.225330in}{1.451590in}}%
\pgfpathlineto{\pgfqpoint{1.227907in}{1.456469in}}%
\pgfpathlineto{\pgfqpoint{1.243370in}{1.465475in}}%
\pgfpathlineto{\pgfqpoint{1.245947in}{1.465475in}}%
\pgfpathlineto{\pgfqpoint{1.248525in}{1.456469in}}%
\pgfpathlineto{\pgfqpoint{1.256256in}{1.457407in}}%
\pgfpathlineto{\pgfqpoint{1.258833in}{1.460034in}}%
\pgfpathlineto{\pgfqpoint{1.261411in}{1.449339in}}%
\pgfpathlineto{\pgfqpoint{1.263988in}{1.453842in}}%
\pgfpathlineto{\pgfqpoint{1.266565in}{1.450277in}}%
\pgfpathlineto{\pgfqpoint{1.274297in}{1.444085in}}%
\pgfpathlineto{\pgfqpoint{1.276874in}{1.439207in}}%
\pgfpathlineto{\pgfqpoint{1.279451in}{1.440708in}}%
\pgfpathlineto{\pgfqpoint{1.284606in}{1.435266in}}%
\pgfpathlineto{\pgfqpoint{1.294914in}{1.437893in}}%
\pgfpathlineto{\pgfqpoint{1.297492in}{1.433953in}}%
\pgfpathlineto{\pgfqpoint{1.300069in}{1.440708in}}%
\pgfpathlineto{\pgfqpoint{1.302646in}{1.431701in}}%
\pgfpathlineto{\pgfqpoint{1.312955in}{1.433015in}}%
\pgfpathlineto{\pgfqpoint{1.315532in}{1.427010in}}%
\pgfpathlineto{\pgfqpoint{1.318109in}{1.430388in}}%
\pgfpathlineto{\pgfqpoint{1.320687in}{1.420443in}}%
\pgfpathlineto{\pgfqpoint{1.328418in}{1.429074in}}%
\pgfpathlineto{\pgfqpoint{1.330995in}{1.427010in}}%
\pgfpathlineto{\pgfqpoint{1.333573in}{1.436204in}}%
\pgfpathlineto{\pgfqpoint{1.336150in}{1.436204in}}%
\pgfpathlineto{\pgfqpoint{1.338727in}{1.438831in}}%
\pgfpathlineto{\pgfqpoint{1.346459in}{1.425697in}}%
\pgfpathlineto{\pgfqpoint{1.349036in}{1.440520in}}%
\pgfpathlineto{\pgfqpoint{1.351613in}{1.461535in}}%
\pgfpathlineto{\pgfqpoint{1.354190in}{1.476921in}}%
\pgfpathlineto{\pgfqpoint{1.356768in}{1.485552in}}%
\pgfpathlineto{\pgfqpoint{1.364499in}{1.490055in}}%
\pgfpathlineto{\pgfqpoint{1.367076in}{1.496435in}}%
\pgfpathlineto{\pgfqpoint{1.369654in}{1.500750in}}%
\pgfpathlineto{\pgfqpoint{1.372231in}{1.491369in}}%
\pgfpathlineto{\pgfqpoint{1.374808in}{1.495121in}}%
\pgfpathlineto{\pgfqpoint{1.385117in}{1.491932in}}%
\pgfpathlineto{\pgfqpoint{1.387694in}{1.498311in}}%
\pgfpathlineto{\pgfqpoint{1.390271in}{1.499812in}}%
\pgfpathlineto{\pgfqpoint{1.392849in}{1.509382in}}%
\pgfpathlineto{\pgfqpoint{1.400580in}{1.516136in}}%
\pgfpathlineto{\pgfqpoint{1.403157in}{1.516887in}}%
\pgfpathlineto{\pgfqpoint{1.410889in}{1.510695in}}%
\pgfpathlineto{\pgfqpoint{1.418621in}{1.512759in}}%
\pgfpathlineto{\pgfqpoint{1.421198in}{1.514260in}}%
\pgfpathlineto{\pgfqpoint{1.423775in}{1.514635in}}%
\pgfpathlineto{\pgfqpoint{1.426352in}{1.511821in}}%
\pgfpathlineto{\pgfqpoint{1.428930in}{1.525518in}}%
\pgfpathlineto{\pgfqpoint{1.436661in}{1.523079in}}%
\pgfpathlineto{\pgfqpoint{1.439238in}{1.531522in}}%
\pgfpathlineto{\pgfqpoint{1.441816in}{1.537339in}}%
\pgfpathlineto{\pgfqpoint{1.444393in}{1.539778in}}%
\pgfpathlineto{\pgfqpoint{1.446970in}{1.525893in}}%
\pgfpathlineto{\pgfqpoint{1.454702in}{1.517825in}}%
\pgfpathlineto{\pgfqpoint{1.457279in}{1.506192in}}%
\pgfpathlineto{\pgfqpoint{1.459856in}{1.509006in}}%
\pgfpathlineto{\pgfqpoint{1.462433in}{1.527582in}}%
\pgfpathlineto{\pgfqpoint{1.465010in}{1.539591in}}%
\pgfpathlineto{\pgfqpoint{1.472742in}{1.538465in}}%
\pgfpathlineto{\pgfqpoint{1.475319in}{1.534900in}}%
\pgfpathlineto{\pgfqpoint{1.477897in}{1.537527in}}%
\pgfpathlineto{\pgfqpoint{1.480474in}{1.523079in}}%
\pgfpathlineto{\pgfqpoint{1.483051in}{1.533399in}}%
\pgfpathlineto{\pgfqpoint{1.490783in}{1.529083in}}%
\pgfpathlineto{\pgfqpoint{1.493360in}{1.520640in}}%
\pgfpathlineto{\pgfqpoint{1.495937in}{1.521578in}}%
\pgfpathlineto{\pgfqpoint{1.498514in}{1.521202in}}%
\pgfpathlineto{\pgfqpoint{1.501091in}{1.526081in}}%
\pgfpathlineto{\pgfqpoint{1.508823in}{1.523266in}}%
\pgfpathlineto{\pgfqpoint{1.511400in}{1.526081in}}%
\pgfpathlineto{\pgfqpoint{1.513978in}{1.521578in}}%
\pgfpathlineto{\pgfqpoint{1.516555in}{1.519326in}}%
\pgfpathlineto{\pgfqpoint{1.519132in}{1.513134in}}%
\pgfpathlineto{\pgfqpoint{1.526864in}{1.511633in}}%
\pgfpathlineto{\pgfqpoint{1.529441in}{1.512759in}}%
\pgfpathlineto{\pgfqpoint{1.534595in}{1.512196in}}%
\pgfpathlineto{\pgfqpoint{1.537172in}{1.519514in}}%
\pgfpathlineto{\pgfqpoint{1.544904in}{1.517825in}}%
\pgfpathlineto{\pgfqpoint{1.547481in}{1.518013in}}%
\pgfpathlineto{\pgfqpoint{1.552636in}{1.513509in}}%
\pgfpathlineto{\pgfqpoint{1.555213in}{1.516887in}}%
\pgfpathlineto{\pgfqpoint{1.565522in}{1.514260in}}%
\pgfpathlineto{\pgfqpoint{1.568099in}{1.514448in}}%
\pgfpathlineto{\pgfqpoint{1.570676in}{1.523266in}}%
\pgfpathlineto{\pgfqpoint{1.573253in}{1.523829in}}%
\pgfpathlineto{\pgfqpoint{1.580985in}{1.528520in}}%
\pgfpathlineto{\pgfqpoint{1.583562in}{1.528895in}}%
\pgfpathlineto{\pgfqpoint{1.586139in}{1.527957in}}%
\pgfpathlineto{\pgfqpoint{1.588717in}{1.541092in}}%
\pgfpathlineto{\pgfqpoint{1.591294in}{1.533023in}}%
\pgfpathlineto{\pgfqpoint{1.599026in}{1.529646in}}%
\pgfpathlineto{\pgfqpoint{1.601603in}{1.534337in}}%
\pgfpathlineto{\pgfqpoint{1.604180in}{1.535087in}}%
\pgfpathlineto{\pgfqpoint{1.606757in}{1.539591in}}%
\pgfpathlineto{\pgfqpoint{1.609334in}{1.542217in}}%
\pgfpathlineto{\pgfqpoint{1.617066in}{1.541279in}}%
\pgfpathlineto{\pgfqpoint{1.619643in}{1.546158in}}%
\pgfpathlineto{\pgfqpoint{1.622220in}{1.541279in}}%
\pgfpathlineto{\pgfqpoint{1.624798in}{1.541467in}}%
\pgfpathlineto{\pgfqpoint{1.627375in}{1.539778in}}%
\pgfpathlineto{\pgfqpoint{1.635106in}{1.543343in}}%
\pgfpathlineto{\pgfqpoint{1.637684in}{1.540529in}}%
\pgfpathlineto{\pgfqpoint{1.640261in}{1.541279in}}%
\pgfpathlineto{\pgfqpoint{1.642838in}{1.545407in}}%
\pgfpathlineto{\pgfqpoint{1.645415in}{1.551411in}}%
\pgfpathlineto{\pgfqpoint{1.653147in}{1.548034in}}%
\pgfpathlineto{\pgfqpoint{1.655724in}{1.532085in}}%
\pgfpathlineto{\pgfqpoint{1.660879in}{1.525330in}}%
\pgfpathlineto{\pgfqpoint{1.663456in}{1.525330in}}%
\pgfpathlineto{\pgfqpoint{1.671187in}{1.535087in}}%
\pgfpathlineto{\pgfqpoint{1.673765in}{1.549723in}}%
\pgfpathlineto{\pgfqpoint{1.678919in}{1.595881in}}%
\pgfpathlineto{\pgfqpoint{1.681496in}{1.585748in}}%
\pgfpathlineto{\pgfqpoint{1.689228in}{1.584435in}}%
\pgfpathlineto{\pgfqpoint{1.691805in}{1.570550in}}%
\pgfpathlineto{\pgfqpoint{1.694382in}{1.568298in}}%
\pgfpathlineto{\pgfqpoint{1.696960in}{1.574490in}}%
\pgfpathlineto{\pgfqpoint{1.699537in}{1.570738in}}%
\pgfpathlineto{\pgfqpoint{1.712423in}{1.569424in}}%
\pgfpathlineto{\pgfqpoint{1.715000in}{1.580119in}}%
\pgfpathlineto{\pgfqpoint{1.717577in}{1.570738in}}%
\pgfpathlineto{\pgfqpoint{1.725309in}{1.569049in}}%
\pgfpathlineto{\pgfqpoint{1.727886in}{1.572426in}}%
\pgfpathlineto{\pgfqpoint{1.733041in}{1.551411in}}%
\pgfpathlineto{\pgfqpoint{1.735618in}{1.554789in}}%
\pgfpathlineto{\pgfqpoint{1.743349in}{1.551787in}}%
\pgfpathlineto{\pgfqpoint{1.745927in}{1.549160in}}%
\pgfpathlineto{\pgfqpoint{1.751081in}{1.542405in}}%
\pgfpathlineto{\pgfqpoint{1.753658in}{1.544281in}}%
\pgfpathlineto{\pgfqpoint{1.761390in}{1.545032in}}%
\pgfpathlineto{\pgfqpoint{1.763967in}{1.551599in}}%
\pgfpathlineto{\pgfqpoint{1.766544in}{1.550473in}}%
\pgfpathlineto{\pgfqpoint{1.771699in}{1.559480in}}%
\pgfpathlineto{\pgfqpoint{1.779430in}{1.552162in}}%
\pgfpathlineto{\pgfqpoint{1.782008in}{1.547659in}}%
\pgfpathlineto{\pgfqpoint{1.784585in}{1.555352in}}%
\pgfpathlineto{\pgfqpoint{1.787162in}{1.554226in}}%
\pgfpathlineto{\pgfqpoint{1.789739in}{1.562107in}}%
\pgfpathlineto{\pgfqpoint{1.797471in}{1.560981in}}%
\pgfpathlineto{\pgfqpoint{1.800048in}{1.564171in}}%
\pgfpathlineto{\pgfqpoint{1.805203in}{1.567173in}}%
\pgfpathlineto{\pgfqpoint{1.807780in}{1.573365in}}%
\pgfpathlineto{\pgfqpoint{1.815511in}{1.575804in}}%
\pgfpathlineto{\pgfqpoint{1.818089in}{1.583684in}}%
\pgfpathlineto{\pgfqpoint{1.820666in}{1.581433in}}%
\pgfpathlineto{\pgfqpoint{1.823243in}{1.578055in}}%
\pgfpathlineto{\pgfqpoint{1.825820in}{1.577117in}}%
\pgfpathlineto{\pgfqpoint{1.833552in}{1.581058in}}%
\pgfpathlineto{\pgfqpoint{1.836129in}{1.581245in}}%
\pgfpathlineto{\pgfqpoint{1.838706in}{1.576179in}}%
\pgfpathlineto{\pgfqpoint{1.841283in}{1.578055in}}%
\pgfpathlineto{\pgfqpoint{1.843861in}{1.570550in}}%
\pgfpathlineto{\pgfqpoint{1.851592in}{1.566610in}}%
\pgfpathlineto{\pgfqpoint{1.856747in}{1.569049in}}%
\pgfpathlineto{\pgfqpoint{1.859324in}{1.567736in}}%
\pgfpathlineto{\pgfqpoint{1.861901in}{1.558166in}}%
\pgfpathlineto{\pgfqpoint{1.869633in}{1.567923in}}%
\pgfpathlineto{\pgfqpoint{1.874787in}{1.579557in}}%
\pgfpathlineto{\pgfqpoint{1.877364in}{1.578055in}}%
\pgfpathlineto{\pgfqpoint{1.879942in}{1.590627in}}%
\pgfpathlineto{\pgfqpoint{1.887673in}{1.588375in}}%
\pgfpathlineto{\pgfqpoint{1.890251in}{1.588563in}}%
\pgfpathlineto{\pgfqpoint{1.892828in}{1.593441in}}%
\pgfpathlineto{\pgfqpoint{1.895405in}{1.600759in}}%
\pgfpathlineto{\pgfqpoint{1.897982in}{1.603198in}}%
\pgfpathlineto{\pgfqpoint{1.905714in}{1.606576in}}%
\pgfpathlineto{\pgfqpoint{1.908291in}{1.603574in}}%
\pgfpathlineto{\pgfqpoint{1.910868in}{1.606763in}}%
\pgfpathlineto{\pgfqpoint{1.916023in}{1.617083in}}%
\pgfpathlineto{\pgfqpoint{1.926332in}{1.608452in}}%
\pgfpathlineto{\pgfqpoint{1.928909in}{1.611079in}}%
\pgfpathlineto{\pgfqpoint{1.931486in}{1.615019in}}%
\pgfpathlineto{\pgfqpoint{1.934063in}{1.627778in}}%
\pgfpathlineto{\pgfqpoint{1.941795in}{1.623088in}}%
\pgfpathlineto{\pgfqpoint{1.944372in}{1.635471in}}%
\pgfpathlineto{\pgfqpoint{1.946949in}{1.630405in}}%
\pgfpathlineto{\pgfqpoint{1.949526in}{1.627778in}}%
\pgfpathlineto{\pgfqpoint{1.952104in}{1.631906in}}%
\pgfpathlineto{\pgfqpoint{1.959835in}{1.630780in}}%
\pgfpathlineto{\pgfqpoint{1.962412in}{1.639412in}}%
\pgfpathlineto{\pgfqpoint{1.964990in}{1.650857in}}%
\pgfpathlineto{\pgfqpoint{1.967567in}{1.645604in}}%
\pgfpathlineto{\pgfqpoint{1.970144in}{1.652171in}}%
\pgfpathlineto{\pgfqpoint{1.977876in}{1.651233in}}%
\pgfpathlineto{\pgfqpoint{1.980453in}{1.657237in}}%
\pgfpathlineto{\pgfqpoint{1.983030in}{1.654985in}}%
\pgfpathlineto{\pgfqpoint{1.985607in}{1.657424in}}%
\pgfpathlineto{\pgfqpoint{1.988185in}{1.662866in}}%
\pgfpathlineto{\pgfqpoint{1.998493in}{1.675437in}}%
\pgfpathlineto{\pgfqpoint{2.001071in}{1.670559in}}%
\pgfpathlineto{\pgfqpoint{2.003648in}{1.674124in}}%
\pgfpathlineto{\pgfqpoint{2.006225in}{1.673936in}}%
\pgfpathlineto{\pgfqpoint{2.013957in}{1.663053in}}%
\pgfpathlineto{\pgfqpoint{2.016534in}{1.666056in}}%
\pgfpathlineto{\pgfqpoint{2.019111in}{1.675062in}}%
\pgfpathlineto{\pgfqpoint{2.021688in}{1.671685in}}%
\pgfpathlineto{\pgfqpoint{2.024266in}{1.681066in}}%
\pgfpathlineto{\pgfqpoint{2.031997in}{1.688947in}}%
\pgfpathlineto{\pgfqpoint{2.034574in}{1.696077in}}%
\pgfpathlineto{\pgfqpoint{2.037152in}{1.691949in}}%
\pgfpathlineto{\pgfqpoint{2.042306in}{1.704520in}}%
\pgfpathlineto{\pgfqpoint{2.052615in}{1.710337in}}%
\pgfpathlineto{\pgfqpoint{2.055192in}{1.710149in}}%
\pgfpathlineto{\pgfqpoint{2.057769in}{1.718968in}}%
\pgfpathlineto{\pgfqpoint{2.060347in}{1.720282in}}%
\pgfpathlineto{\pgfqpoint{2.068078in}{1.714277in}}%
\pgfpathlineto{\pgfqpoint{2.070655in}{1.715028in}}%
\pgfpathlineto{\pgfqpoint{2.073233in}{1.724410in}}%
\pgfpathlineto{\pgfqpoint{2.075810in}{1.717467in}}%
\pgfpathlineto{\pgfqpoint{2.078387in}{1.728913in}}%
\pgfpathlineto{\pgfqpoint{2.086119in}{1.727975in}}%
\pgfpathlineto{\pgfqpoint{2.088696in}{1.746550in}}%
\pgfpathlineto{\pgfqpoint{2.091273in}{1.753118in}}%
\pgfpathlineto{\pgfqpoint{2.093850in}{1.757246in}}%
\pgfpathlineto{\pgfqpoint{2.104159in}{1.763625in}}%
\pgfpathlineto{\pgfqpoint{2.106736in}{1.775634in}}%
\pgfpathlineto{\pgfqpoint{2.109314in}{1.765877in}}%
\pgfpathlineto{\pgfqpoint{2.111891in}{1.771130in}}%
\pgfpathlineto{\pgfqpoint{2.114468in}{1.765314in}}%
\pgfpathlineto{\pgfqpoint{2.122200in}{1.750678in}}%
\pgfpathlineto{\pgfqpoint{2.124777in}{1.757058in}}%
\pgfpathlineto{\pgfqpoint{2.127354in}{1.764938in}}%
\pgfpathlineto{\pgfqpoint{2.129931in}{1.769817in}}%
\pgfpathlineto{\pgfqpoint{2.132509in}{1.776384in}}%
\pgfpathlineto{\pgfqpoint{2.140240in}{1.760060in}}%
\pgfpathlineto{\pgfqpoint{2.142817in}{1.787455in}}%
\pgfpathlineto{\pgfqpoint{2.145395in}{1.794772in}}%
\pgfpathlineto{\pgfqpoint{2.147972in}{1.783327in}}%
\pgfpathlineto{\pgfqpoint{2.150549in}{1.803966in}}%
\pgfpathlineto{\pgfqpoint{2.158281in}{1.809408in}}%
\pgfpathlineto{\pgfqpoint{2.160858in}{1.819165in}}%
\pgfpathlineto{\pgfqpoint{2.163435in}{1.802465in}}%
\pgfpathlineto{\pgfqpoint{2.166012in}{1.815600in}}%
\pgfpathlineto{\pgfqpoint{2.168589in}{1.813911in}}%
\pgfpathlineto{\pgfqpoint{2.176321in}{1.821229in}}%
\pgfpathlineto{\pgfqpoint{2.178898in}{1.815787in}}%
\pgfpathlineto{\pgfqpoint{2.181476in}{1.799838in}}%
\pgfpathlineto{\pgfqpoint{2.184053in}{1.814661in}}%
\pgfpathlineto{\pgfqpoint{2.186630in}{1.823855in}}%
\pgfpathlineto{\pgfqpoint{2.194362in}{1.806968in}}%
\pgfpathlineto{\pgfqpoint{2.196939in}{1.820478in}}%
\pgfpathlineto{\pgfqpoint{2.199516in}{1.819352in}}%
\pgfpathlineto{\pgfqpoint{2.202093in}{1.814474in}}%
\pgfpathlineto{\pgfqpoint{2.204670in}{1.824043in}}%
\pgfpathlineto{\pgfqpoint{2.212402in}{1.825544in}}%
\pgfpathlineto{\pgfqpoint{2.214979in}{1.843182in}}%
\pgfpathlineto{\pgfqpoint{2.217557in}{1.853689in}}%
\pgfpathlineto{\pgfqpoint{2.220134in}{1.850687in}}%
\pgfpathlineto{\pgfqpoint{2.222711in}{1.860819in}}%
\pgfpathlineto{\pgfqpoint{2.230443in}{1.859693in}}%
\pgfpathlineto{\pgfqpoint{2.233020in}{1.868700in}}%
\pgfpathlineto{\pgfqpoint{2.235597in}{1.866824in}}%
\pgfpathlineto{\pgfqpoint{2.238174in}{1.857254in}}%
\pgfpathlineto{\pgfqpoint{2.240751in}{1.851062in}}%
\pgfpathlineto{\pgfqpoint{2.251060in}{1.863634in}}%
\pgfpathlineto{\pgfqpoint{2.253637in}{1.832487in}}%
\pgfpathlineto{\pgfqpoint{2.256215in}{1.837928in}}%
\pgfpathlineto{\pgfqpoint{2.258792in}{1.809032in}}%
\pgfpathlineto{\pgfqpoint{2.266524in}{1.817476in}}%
\pgfpathlineto{\pgfqpoint{2.269101in}{1.807907in}}%
\pgfpathlineto{\pgfqpoint{2.271678in}{1.801152in}}%
\pgfpathlineto{\pgfqpoint{2.274255in}{1.813536in}}%
\pgfpathlineto{\pgfqpoint{2.276832in}{1.820666in}}%
\pgfpathlineto{\pgfqpoint{2.284564in}{1.824418in}}%
\pgfpathlineto{\pgfqpoint{2.287141in}{1.817288in}}%
\pgfpathlineto{\pgfqpoint{2.289718in}{1.802090in}}%
\pgfpathlineto{\pgfqpoint{2.292296in}{1.820666in}}%
\pgfpathlineto{\pgfqpoint{2.294873in}{1.820666in}}%
\pgfpathlineto{\pgfqpoint{2.302605in}{1.832111in}}%
\pgfpathlineto{\pgfqpoint{2.305182in}{1.843745in}}%
\pgfpathlineto{\pgfqpoint{2.307759in}{1.819915in}}%
\pgfpathlineto{\pgfqpoint{2.310336in}{1.784265in}}%
\pgfpathlineto{\pgfqpoint{2.312913in}{1.793459in}}%
\pgfpathlineto{\pgfqpoint{2.320645in}{1.815975in}}%
\pgfpathlineto{\pgfqpoint{2.323222in}{1.827796in}}%
\pgfpathlineto{\pgfqpoint{2.325799in}{1.853877in}}%
\pgfpathlineto{\pgfqpoint{2.328377in}{1.849374in}}%
\pgfpathlineto{\pgfqpoint{2.330954in}{1.835864in}}%
\pgfpathlineto{\pgfqpoint{2.338685in}{1.848060in}}%
\pgfpathlineto{\pgfqpoint{2.341263in}{1.847122in}}%
\pgfpathlineto{\pgfqpoint{2.343840in}{1.850499in}}%
\pgfpathlineto{\pgfqpoint{2.348994in}{1.867762in}}%
\pgfpathlineto{\pgfqpoint{2.359303in}{1.883898in}}%
\pgfpathlineto{\pgfqpoint{2.367035in}{1.901536in}}%
\pgfpathlineto{\pgfqpoint{2.374766in}{1.908103in}}%
\pgfpathlineto{\pgfqpoint{2.377344in}{1.908103in}}%
\pgfpathlineto{\pgfqpoint{2.379921in}{1.903787in}}%
\pgfpathlineto{\pgfqpoint{2.382498in}{1.904538in}}%
\pgfpathlineto{\pgfqpoint{2.385075in}{1.937186in}}%
\pgfpathlineto{\pgfqpoint{2.392807in}{1.937937in}}%
\pgfpathlineto{\pgfqpoint{2.395384in}{1.940188in}}%
\pgfpathlineto{\pgfqpoint{2.397961in}{1.939250in}}%
\pgfpathlineto{\pgfqpoint{2.403116in}{1.946755in}}%
\pgfpathlineto{\pgfqpoint{2.410847in}{1.952760in}}%
\pgfpathlineto{\pgfqpoint{2.413425in}{1.952197in}}%
\pgfpathlineto{\pgfqpoint{2.418579in}{1.961766in}}%
\pgfpathlineto{\pgfqpoint{2.421156in}{1.971523in}}%
\pgfpathlineto{\pgfqpoint{2.428888in}{1.962141in}}%
\pgfpathlineto{\pgfqpoint{2.431465in}{1.962329in}}%
\pgfpathlineto{\pgfqpoint{2.434042in}{1.960077in}}%
\pgfpathlineto{\pgfqpoint{2.436620in}{1.954824in}}%
\pgfpathlineto{\pgfqpoint{2.439197in}{1.939250in}}%
\pgfpathlineto{\pgfqpoint{2.446928in}{1.933621in}}%
\pgfpathlineto{\pgfqpoint{2.449506in}{1.949570in}}%
\pgfpathlineto{\pgfqpoint{2.452083in}{1.912794in}}%
\pgfpathlineto{\pgfqpoint{2.454660in}{1.894593in}}%
\pgfpathlineto{\pgfqpoint{2.457237in}{1.891779in}}%
\pgfpathlineto{\pgfqpoint{2.464969in}{1.908853in}}%
\pgfpathlineto{\pgfqpoint{2.467546in}{1.897408in}}%
\pgfpathlineto{\pgfqpoint{2.470123in}{1.882585in}}%
\pgfpathlineto{\pgfqpoint{2.472701in}{1.874141in}}%
\pgfpathlineto{\pgfqpoint{2.475278in}{1.887088in}}%
\pgfpathlineto{\pgfqpoint{2.483009in}{1.873015in}}%
\pgfpathlineto{\pgfqpoint{2.485587in}{1.851062in}}%
\pgfpathlineto{\pgfqpoint{2.488164in}{1.856879in}}%
\pgfpathlineto{\pgfqpoint{2.490741in}{1.857629in}}%
\pgfpathlineto{\pgfqpoint{2.493318in}{1.855003in}}%
\pgfpathlineto{\pgfqpoint{2.503627in}{1.855190in}}%
\pgfpathlineto{\pgfqpoint{2.506204in}{1.862883in}}%
\pgfpathlineto{\pgfqpoint{2.511359in}{1.867011in}}%
\pgfpathlineto{\pgfqpoint{2.519090in}{1.873391in}}%
\pgfpathlineto{\pgfqpoint{2.521668in}{1.888964in}}%
\pgfpathlineto{\pgfqpoint{2.524245in}{1.900222in}}%
\pgfpathlineto{\pgfqpoint{2.526822in}{1.896657in}}%
\pgfpathlineto{\pgfqpoint{2.529399in}{1.889527in}}%
\pgfpathlineto{\pgfqpoint{2.537131in}{1.897033in}}%
\pgfpathlineto{\pgfqpoint{2.539708in}{1.897408in}}%
\pgfpathlineto{\pgfqpoint{2.542285in}{1.911105in}}%
\pgfpathlineto{\pgfqpoint{2.544862in}{1.913732in}}%
\pgfpathlineto{\pgfqpoint{2.547440in}{1.907352in}}%
\pgfpathlineto{\pgfqpoint{2.555171in}{1.897971in}}%
\pgfpathlineto{\pgfqpoint{2.557749in}{1.884086in}}%
\pgfpathlineto{\pgfqpoint{2.560326in}{1.865698in}}%
\pgfpathlineto{\pgfqpoint{2.562903in}{1.865510in}}%
\pgfpathlineto{\pgfqpoint{2.565480in}{1.860069in}}%
\pgfpathlineto{\pgfqpoint{2.573212in}{1.859506in}}%
\pgfpathlineto{\pgfqpoint{2.575789in}{1.871890in}}%
\pgfpathlineto{\pgfqpoint{2.578366in}{1.869075in}}%
\pgfpathlineto{\pgfqpoint{2.580943in}{1.857629in}}%
\pgfpathlineto{\pgfqpoint{2.583521in}{1.869450in}}%
\pgfpathlineto{\pgfqpoint{2.591252in}{1.857817in}}%
\pgfpathlineto{\pgfqpoint{2.593830in}{1.842056in}}%
\pgfpathlineto{\pgfqpoint{2.596407in}{1.847685in}}%
\pgfpathlineto{\pgfqpoint{2.601561in}{1.903787in}}%
\pgfpathlineto{\pgfqpoint{2.611870in}{1.911480in}}%
\pgfpathlineto{\pgfqpoint{2.614447in}{1.930244in}}%
\pgfpathlineto{\pgfqpoint{2.617024in}{1.944129in}}%
\pgfpathlineto{\pgfqpoint{2.619602in}{1.938687in}}%
\pgfpathlineto{\pgfqpoint{2.627333in}{1.931745in}}%
\pgfpathlineto{\pgfqpoint{2.629911in}{1.950320in}}%
\pgfpathlineto{\pgfqpoint{2.632488in}{1.946193in}}%
\pgfpathlineto{\pgfqpoint{2.635065in}{1.950133in}}%
\pgfpathlineto{\pgfqpoint{2.637642in}{1.946005in}}%
\pgfpathlineto{\pgfqpoint{2.645374in}{1.950883in}}%
\pgfpathlineto{\pgfqpoint{2.647951in}{1.962892in}}%
\pgfpathlineto{\pgfqpoint{2.650528in}{1.957263in}}%
\pgfpathlineto{\pgfqpoint{2.653105in}{1.954448in}}%
\pgfpathlineto{\pgfqpoint{2.655683in}{1.966645in}}%
\pgfpathlineto{\pgfqpoint{2.663414in}{1.961203in}}%
\pgfpathlineto{\pgfqpoint{2.665991in}{1.957638in}}%
\pgfpathlineto{\pgfqpoint{2.668569in}{1.961203in}}%
\pgfpathlineto{\pgfqpoint{2.671146in}{1.955574in}}%
\pgfpathlineto{\pgfqpoint{2.673723in}{1.977527in}}%
\pgfpathlineto{\pgfqpoint{2.681455in}{1.981280in}}%
\pgfpathlineto{\pgfqpoint{2.684032in}{1.969647in}}%
\pgfpathlineto{\pgfqpoint{2.686609in}{1.966082in}}%
\pgfpathlineto{\pgfqpoint{2.691764in}{1.982969in}}%
\pgfpathlineto{\pgfqpoint{2.699495in}{1.981468in}}%
\pgfpathlineto{\pgfqpoint{2.702072in}{1.990474in}}%
\pgfpathlineto{\pgfqpoint{2.704650in}{1.995165in}}%
\pgfpathlineto{\pgfqpoint{2.707227in}{1.995915in}}%
\pgfpathlineto{\pgfqpoint{2.709804in}{2.007361in}}%
\pgfpathlineto{\pgfqpoint{2.717536in}{2.013553in}}%
\pgfpathlineto{\pgfqpoint{2.720113in}{2.004359in}}%
\pgfpathlineto{\pgfqpoint{2.722690in}{2.003046in}}%
\pgfpathlineto{\pgfqpoint{2.727845in}{1.997792in}}%
\pgfpathlineto{\pgfqpoint{2.735576in}{1.991787in}}%
\pgfpathlineto{\pgfqpoint{2.740731in}{1.981280in}}%
\pgfpathlineto{\pgfqpoint{2.743308in}{1.970585in}}%
\pgfpathlineto{\pgfqpoint{2.745885in}{1.994227in}}%
\pgfpathlineto{\pgfqpoint{2.753617in}{1.994227in}}%
\pgfpathlineto{\pgfqpoint{2.756194in}{1.989536in}}%
\pgfpathlineto{\pgfqpoint{2.758771in}{1.972836in}}%
\pgfpathlineto{\pgfqpoint{2.761348in}{1.941314in}}%
\pgfpathlineto{\pgfqpoint{2.763926in}{1.944316in}}%
\pgfpathlineto{\pgfqpoint{2.771657in}{1.944691in}}%
\pgfpathlineto{\pgfqpoint{2.774234in}{1.933246in}}%
\pgfpathlineto{\pgfqpoint{2.776812in}{1.965144in}}%
\pgfpathlineto{\pgfqpoint{2.779389in}{1.954636in}}%
\pgfpathlineto{\pgfqpoint{2.781966in}{1.956325in}}%
\pgfpathlineto{\pgfqpoint{2.789698in}{1.955387in}}%
\pgfpathlineto{\pgfqpoint{2.792275in}{1.955949in}}%
\pgfpathlineto{\pgfqpoint{2.797429in}{1.963642in}}%
\pgfpathlineto{\pgfqpoint{2.800007in}{1.960640in}}%
\pgfpathlineto{\pgfqpoint{2.807738in}{1.959702in}}%
\pgfpathlineto{\pgfqpoint{2.810315in}{1.948256in}}%
\pgfpathlineto{\pgfqpoint{2.815470in}{1.939250in}}%
\pgfpathlineto{\pgfqpoint{2.818047in}{1.952384in}}%
\pgfpathlineto{\pgfqpoint{2.825779in}{1.960265in}}%
\pgfpathlineto{\pgfqpoint{2.828356in}{1.991787in}}%
\pgfpathlineto{\pgfqpoint{2.830933in}{1.989724in}}%
\pgfpathlineto{\pgfqpoint{2.833510in}{1.998918in}}%
\pgfpathlineto{\pgfqpoint{2.836088in}{1.999105in}}%
\pgfpathlineto{\pgfqpoint{2.843819in}{1.995353in}}%
\pgfpathlineto{\pgfqpoint{2.846396in}{1.998730in}}%
\pgfpathlineto{\pgfqpoint{2.848974in}{2.000043in}}%
\pgfpathlineto{\pgfqpoint{2.851551in}{1.997604in}}%
\pgfpathlineto{\pgfqpoint{2.854128in}{2.004359in}}%
\pgfpathlineto{\pgfqpoint{2.864437in}{1.987660in}}%
\pgfpathlineto{\pgfqpoint{2.867014in}{1.992350in}}%
\pgfpathlineto{\pgfqpoint{2.869591in}{1.966832in}}%
\pgfpathlineto{\pgfqpoint{2.872168in}{1.932495in}}%
\pgfpathlineto{\pgfqpoint{2.879900in}{1.921613in}}%
\pgfpathlineto{\pgfqpoint{2.882477in}{1.924239in}}%
\pgfpathlineto{\pgfqpoint{2.885055in}{1.904913in}}%
\pgfpathlineto{\pgfqpoint{2.887632in}{1.914482in}}%
\pgfpathlineto{\pgfqpoint{2.890209in}{1.897971in}}%
\pgfpathlineto{\pgfqpoint{2.897941in}{1.870576in}}%
\pgfpathlineto{\pgfqpoint{2.900518in}{1.867949in}}%
\pgfpathlineto{\pgfqpoint{2.903095in}{1.878644in}}%
\pgfpathlineto{\pgfqpoint{2.908249in}{1.923301in}}%
\pgfpathlineto{\pgfqpoint{2.915981in}{1.939813in}}%
\pgfpathlineto{\pgfqpoint{2.918558in}{1.970585in}}%
\pgfpathlineto{\pgfqpoint{2.921136in}{1.961766in}}%
\pgfpathlineto{\pgfqpoint{2.923713in}{1.964018in}}%
\pgfpathlineto{\pgfqpoint{2.926290in}{1.967207in}}%
\pgfpathlineto{\pgfqpoint{2.936599in}{1.957638in}}%
\pgfpathlineto{\pgfqpoint{2.939176in}{1.949007in}}%
\pgfpathlineto{\pgfqpoint{2.941753in}{1.961016in}}%
\pgfpathlineto{\pgfqpoint{2.944330in}{1.957826in}}%
\pgfpathlineto{\pgfqpoint{2.952062in}{1.951071in}}%
\pgfpathlineto{\pgfqpoint{2.957216in}{1.951071in}}%
\pgfpathlineto{\pgfqpoint{2.959794in}{1.955199in}}%
\pgfpathlineto{\pgfqpoint{2.962371in}{1.967583in}}%
\pgfpathlineto{\pgfqpoint{2.970103in}{1.958389in}}%
\pgfpathlineto{\pgfqpoint{2.972680in}{1.987284in}}%
\pgfpathlineto{\pgfqpoint{2.975257in}{1.975088in}}%
\pgfpathlineto{\pgfqpoint{2.977834in}{1.979967in}}%
\pgfpathlineto{\pgfqpoint{2.980411in}{1.987097in}}%
\pgfpathlineto{\pgfqpoint{2.990720in}{1.989724in}}%
\pgfpathlineto{\pgfqpoint{2.993297in}{1.991600in}}%
\pgfpathlineto{\pgfqpoint{2.995875in}{1.981843in}}%
\pgfpathlineto{\pgfqpoint{2.998452in}{1.978841in}}%
\pgfpathlineto{\pgfqpoint{3.006184in}{1.997041in}}%
\pgfpathlineto{\pgfqpoint{3.008761in}{1.997041in}}%
\pgfpathlineto{\pgfqpoint{3.011338in}{1.991412in}}%
\pgfpathlineto{\pgfqpoint{3.013915in}{2.000043in}}%
\pgfpathlineto{\pgfqpoint{3.016492in}{2.029502in}}%
\pgfpathlineto{\pgfqpoint{3.024224in}{2.017681in}}%
\pgfpathlineto{\pgfqpoint{3.026801in}{2.053144in}}%
\pgfpathlineto{\pgfqpoint{3.029378in}{2.047702in}}%
\pgfpathlineto{\pgfqpoint{3.031956in}{2.051455in}}%
\pgfpathlineto{\pgfqpoint{3.034533in}{2.054082in}}%
\pgfpathlineto{\pgfqpoint{3.042264in}{2.066841in}}%
\pgfpathlineto{\pgfqpoint{3.044842in}{2.062150in}}%
\pgfpathlineto{\pgfqpoint{3.047419in}{2.066841in}}%
\pgfpathlineto{\pgfqpoint{3.049996in}{2.067404in}}%
\pgfpathlineto{\pgfqpoint{3.052573in}{2.070031in}}%
\pgfpathlineto{\pgfqpoint{3.060305in}{2.061400in}}%
\pgfpathlineto{\pgfqpoint{3.062882in}{2.064214in}}%
\pgfpathlineto{\pgfqpoint{3.065459in}{2.078662in}}%
\pgfpathlineto{\pgfqpoint{3.068037in}{2.039446in}}%
\pgfpathlineto{\pgfqpoint{3.070614in}{2.044700in}}%
\pgfpathlineto{\pgfqpoint{3.078345in}{2.049203in}}%
\pgfpathlineto{\pgfqpoint{3.080923in}{2.082602in}}%
\pgfpathlineto{\pgfqpoint{3.083500in}{2.075284in}}%
\pgfpathlineto{\pgfqpoint{3.086077in}{2.078849in}}%
\pgfpathlineto{\pgfqpoint{3.096386in}{2.095736in}}%
\pgfpathlineto{\pgfqpoint{3.098963in}{2.098551in}}%
\pgfpathlineto{\pgfqpoint{3.101540in}{2.099302in}}%
\pgfpathlineto{\pgfqpoint{3.104118in}{2.094986in}}%
\pgfpathlineto{\pgfqpoint{3.106695in}{2.092171in}}%
\pgfpathlineto{\pgfqpoint{3.114426in}{2.117502in}}%
\pgfpathlineto{\pgfqpoint{3.117004in}{2.112436in}}%
\pgfpathlineto{\pgfqpoint{3.119581in}{2.116564in}}%
\pgfpathlineto{\pgfqpoint{3.122158in}{2.104368in}}%
\pgfpathlineto{\pgfqpoint{3.124735in}{2.084478in}}%
\pgfpathlineto{\pgfqpoint{3.132467in}{2.095736in}}%
\pgfpathlineto{\pgfqpoint{3.135044in}{2.087668in}}%
\pgfpathlineto{\pgfqpoint{3.137621in}{2.110560in}}%
\pgfpathlineto{\pgfqpoint{3.140199in}{2.103805in}}%
\pgfpathlineto{\pgfqpoint{3.142776in}{2.110560in}}%
\pgfpathlineto{\pgfqpoint{3.150507in}{2.104180in}}%
\pgfpathlineto{\pgfqpoint{3.153085in}{2.112248in}}%
\pgfpathlineto{\pgfqpoint{3.160816in}{2.105118in}}%
\pgfpathlineto{\pgfqpoint{3.168548in}{2.106056in}}%
\pgfpathlineto{\pgfqpoint{3.171125in}{2.099677in}}%
\pgfpathlineto{\pgfqpoint{3.173702in}{2.114125in}}%
\pgfpathlineto{\pgfqpoint{3.176280in}{2.122756in}}%
\pgfpathlineto{\pgfqpoint{3.178857in}{2.123131in}}%
\pgfpathlineto{\pgfqpoint{3.189166in}{2.120316in}}%
\pgfpathlineto{\pgfqpoint{3.191743in}{2.111873in}}%
\pgfpathlineto{\pgfqpoint{3.194320in}{2.119378in}}%
\pgfpathlineto{\pgfqpoint{3.196897in}{2.131012in}}%
\pgfpathlineto{\pgfqpoint{3.204629in}{2.142457in}}%
\pgfpathlineto{\pgfqpoint{3.207206in}{2.147336in}}%
\pgfpathlineto{\pgfqpoint{3.209783in}{2.150901in}}%
\pgfpathlineto{\pgfqpoint{3.212361in}{2.159720in}}%
\pgfpathlineto{\pgfqpoint{3.214938in}{2.159157in}}%
\pgfpathlineto{\pgfqpoint{3.222669in}{2.159720in}}%
\pgfpathlineto{\pgfqpoint{3.225247in}{2.174167in}}%
\pgfpathlineto{\pgfqpoint{3.230401in}{2.148462in}}%
\pgfpathlineto{\pgfqpoint{3.232978in}{2.148462in}}%
\pgfpathlineto{\pgfqpoint{3.240710in}{2.147148in}}%
\pgfpathlineto{\pgfqpoint{3.243287in}{2.138705in}}%
\pgfpathlineto{\pgfqpoint{3.248441in}{2.169476in}}%
\pgfpathlineto{\pgfqpoint{3.251019in}{2.193306in}}%
\pgfpathlineto{\pgfqpoint{3.261328in}{2.182236in}}%
\pgfpathlineto{\pgfqpoint{3.263905in}{2.201374in}}%
\pgfpathlineto{\pgfqpoint{3.266482in}{2.199498in}}%
\pgfpathlineto{\pgfqpoint{3.269059in}{2.188803in}}%
\pgfpathlineto{\pgfqpoint{3.276791in}{2.182611in}}%
\pgfpathlineto{\pgfqpoint{3.279368in}{2.203063in}}%
\pgfpathlineto{\pgfqpoint{3.281945in}{2.202875in}}%
\pgfpathlineto{\pgfqpoint{3.284522in}{2.195745in}}%
\pgfpathlineto{\pgfqpoint{3.294831in}{2.213007in}}%
\pgfpathlineto{\pgfqpoint{3.297409in}{2.200624in}}%
\pgfpathlineto{\pgfqpoint{3.299986in}{2.205877in}}%
\pgfpathlineto{\pgfqpoint{3.302563in}{2.201937in}}%
\pgfpathlineto{\pgfqpoint{3.305140in}{2.190491in}}%
\pgfpathlineto{\pgfqpoint{3.312872in}{2.195182in}}%
\pgfpathlineto{\pgfqpoint{3.315449in}{2.160658in}}%
\pgfpathlineto{\pgfqpoint{3.318026in}{2.143395in}}%
\pgfpathlineto{\pgfqpoint{3.320603in}{2.112999in}}%
\pgfpathlineto{\pgfqpoint{3.323181in}{2.136453in}}%
\pgfpathlineto{\pgfqpoint{3.330912in}{2.127822in}}%
\pgfpathlineto{\pgfqpoint{3.333490in}{2.147523in}}%
\pgfpathlineto{\pgfqpoint{3.336067in}{2.142833in}}%
\pgfpathlineto{\pgfqpoint{3.338644in}{2.143020in}}%
\pgfpathlineto{\pgfqpoint{3.341221in}{2.141519in}}%
\pgfpathlineto{\pgfqpoint{3.348953in}{2.141519in}}%
\pgfpathlineto{\pgfqpoint{3.351530in}{2.139080in}}%
\pgfpathlineto{\pgfqpoint{3.354107in}{2.144709in}}%
\pgfpathlineto{\pgfqpoint{3.356684in}{2.108496in}}%
\pgfpathlineto{\pgfqpoint{3.359262in}{2.105306in}}%
\pgfpathlineto{\pgfqpoint{3.366993in}{2.109621in}}%
\pgfpathlineto{\pgfqpoint{3.369570in}{2.103992in}}%
\pgfpathlineto{\pgfqpoint{3.372148in}{2.118628in}}%
\pgfpathlineto{\pgfqpoint{3.374725in}{2.105869in}}%
\pgfpathlineto{\pgfqpoint{3.377302in}{2.124632in}}%
\pgfpathlineto{\pgfqpoint{3.385034in}{2.125945in}}%
\pgfpathlineto{\pgfqpoint{3.387611in}{2.117314in}}%
\pgfpathlineto{\pgfqpoint{3.390188in}{2.135515in}}%
\pgfpathlineto{\pgfqpoint{3.392765in}{2.140018in}}%
\pgfpathlineto{\pgfqpoint{3.395343in}{2.126133in}}%
\pgfpathlineto{\pgfqpoint{3.403074in}{2.151276in}}%
\pgfpathlineto{\pgfqpoint{3.408229in}{2.159532in}}%
\pgfpathlineto{\pgfqpoint{3.410806in}{2.175668in}}%
\pgfpathlineto{\pgfqpoint{3.413383in}{2.169101in}}%
\pgfpathlineto{\pgfqpoint{3.421115in}{2.171353in}}%
\pgfpathlineto{\pgfqpoint{3.423692in}{2.174730in}}%
\pgfpathlineto{\pgfqpoint{3.428846in}{2.166662in}}%
\pgfpathlineto{\pgfqpoint{3.431424in}{2.179609in}}%
\pgfpathlineto{\pgfqpoint{3.441732in}{2.173417in}}%
\pgfpathlineto{\pgfqpoint{3.444310in}{2.179984in}}%
\pgfpathlineto{\pgfqpoint{3.446887in}{2.181485in}}%
\pgfpathlineto{\pgfqpoint{3.449464in}{2.190867in}}%
\pgfpathlineto{\pgfqpoint{3.457196in}{2.184862in}}%
\pgfpathlineto{\pgfqpoint{3.459773in}{2.180735in}}%
\pgfpathlineto{\pgfqpoint{3.462350in}{2.200436in}}%
\pgfpathlineto{\pgfqpoint{3.464927in}{2.193118in}}%
\pgfpathlineto{\pgfqpoint{3.475236in}{2.195933in}}%
\pgfpathlineto{\pgfqpoint{3.477813in}{2.215071in}}%
\pgfpathlineto{\pgfqpoint{3.480391in}{2.220138in}}%
\pgfpathlineto{\pgfqpoint{3.482968in}{2.239276in}}%
\pgfpathlineto{\pgfqpoint{3.485545in}{2.249784in}}%
\pgfpathlineto{\pgfqpoint{3.493277in}{2.247907in}}%
\pgfpathlineto{\pgfqpoint{3.495854in}{2.241153in}}%
\pgfpathlineto{\pgfqpoint{3.498431in}{2.260479in}}%
\pgfpathlineto{\pgfqpoint{3.501008in}{2.235148in}}%
\pgfpathlineto{\pgfqpoint{3.503586in}{2.235148in}}%
\pgfpathlineto{\pgfqpoint{3.511317in}{2.225954in}}%
\pgfpathlineto{\pgfqpoint{3.513894in}{2.226705in}}%
\pgfpathlineto{\pgfqpoint{3.516472in}{2.188990in}}%
\pgfpathlineto{\pgfqpoint{3.519049in}{2.181485in}}%
\pgfpathlineto{\pgfqpoint{3.521626in}{2.202688in}}%
\pgfpathlineto{\pgfqpoint{3.529358in}{2.198184in}}%
\pgfpathlineto{\pgfqpoint{3.531935in}{2.157468in}}%
\pgfpathlineto{\pgfqpoint{3.534512in}{2.199123in}}%
\pgfpathlineto{\pgfqpoint{3.537089in}{2.152402in}}%
\pgfpathlineto{\pgfqpoint{3.539667in}{2.138329in}}%
\pgfpathlineto{\pgfqpoint{3.547398in}{2.103617in}}%
\pgfpathlineto{\pgfqpoint{3.549975in}{2.068717in}}%
\pgfpathlineto{\pgfqpoint{3.552553in}{2.088606in}}%
\pgfpathlineto{\pgfqpoint{3.555130in}{2.064965in}}%
\pgfpathlineto{\pgfqpoint{3.557707in}{2.096675in}}%
\pgfpathlineto{\pgfqpoint{3.565439in}{2.104931in}}%
\pgfpathlineto{\pgfqpoint{3.568016in}{2.124069in}}%
\pgfpathlineto{\pgfqpoint{3.570593in}{2.138142in}}%
\pgfpathlineto{\pgfqpoint{3.573170in}{2.161408in}}%
\pgfpathlineto{\pgfqpoint{3.575747in}{2.169664in}}%
\pgfpathlineto{\pgfqpoint{3.583479in}{2.185238in}}%
\pgfpathlineto{\pgfqpoint{3.588634in}{2.209818in}}%
\pgfpathlineto{\pgfqpoint{3.591211in}{2.234210in}}%
\pgfpathlineto{\pgfqpoint{3.593788in}{2.246406in}}%
\pgfpathlineto{\pgfqpoint{3.601520in}{2.241153in}}%
\pgfpathlineto{\pgfqpoint{3.604097in}{2.260104in}}%
\pgfpathlineto{\pgfqpoint{3.609251in}{2.266671in}}%
\pgfpathlineto{\pgfqpoint{3.611828in}{2.253349in}}%
\pgfpathlineto{\pgfqpoint{3.619560in}{2.263481in}}%
\pgfpathlineto{\pgfqpoint{3.622137in}{2.264982in}}%
\pgfpathlineto{\pgfqpoint{3.624715in}{2.262355in}}%
\pgfpathlineto{\pgfqpoint{3.627292in}{2.267609in}}%
\pgfpathlineto{\pgfqpoint{3.629869in}{2.252598in}}%
\pgfpathlineto{\pgfqpoint{3.637601in}{2.254850in}}%
\pgfpathlineto{\pgfqpoint{3.640178in}{2.263669in}}%
\pgfpathlineto{\pgfqpoint{3.642755in}{2.262543in}}%
\pgfpathlineto{\pgfqpoint{3.645332in}{2.252973in}}%
\pgfpathlineto{\pgfqpoint{3.647909in}{2.259165in}}%
\pgfpathlineto{\pgfqpoint{3.655641in}{2.243029in}}%
\pgfpathlineto{\pgfqpoint{3.658218in}{2.240027in}}%
\pgfpathlineto{\pgfqpoint{3.665950in}{2.265733in}}%
\pgfpathlineto{\pgfqpoint{3.673682in}{2.261980in}}%
\pgfpathlineto{\pgfqpoint{3.676259in}{2.270048in}}%
\pgfpathlineto{\pgfqpoint{3.678836in}{2.256914in}}%
\pgfpathlineto{\pgfqpoint{3.681413in}{2.254287in}}%
\pgfpathlineto{\pgfqpoint{3.683990in}{2.270048in}}%
\pgfpathlineto{\pgfqpoint{3.691722in}{2.270236in}}%
\pgfpathlineto{\pgfqpoint{3.694299in}{2.262355in}}%
\pgfpathlineto{\pgfqpoint{3.696876in}{2.232334in}}%
\pgfpathlineto{\pgfqpoint{3.699454in}{2.240402in}}%
\pgfpathlineto{\pgfqpoint{3.702031in}{2.202312in}}%
\pgfpathlineto{\pgfqpoint{3.709763in}{2.194619in}}%
\pgfpathlineto{\pgfqpoint{3.712340in}{2.174543in}}%
\pgfpathlineto{\pgfqpoint{3.714917in}{2.196308in}}%
\pgfpathlineto{\pgfqpoint{3.717494in}{2.241903in}}%
\pgfpathlineto{\pgfqpoint{3.720071in}{2.220888in}}%
\pgfpathlineto{\pgfqpoint{3.727803in}{2.240590in}}%
\pgfpathlineto{\pgfqpoint{3.730380in}{2.199873in}}%
\pgfpathlineto{\pgfqpoint{3.732957in}{2.204939in}}%
\pgfpathlineto{\pgfqpoint{3.738112in}{2.212820in}}%
\pgfpathlineto{\pgfqpoint{3.745843in}{2.217323in}}%
\pgfpathlineto{\pgfqpoint{3.748421in}{2.217698in}}%
\pgfpathlineto{\pgfqpoint{3.750998in}{2.204564in}}%
\pgfpathlineto{\pgfqpoint{3.756152in}{2.203813in}}%
\pgfpathlineto{\pgfqpoint{3.763884in}{2.191805in}}%
\pgfpathlineto{\pgfqpoint{3.766461in}{2.183174in}}%
\pgfpathlineto{\pgfqpoint{3.769038in}{2.221076in}}%
\pgfpathlineto{\pgfqpoint{3.771616in}{2.234773in}}%
\pgfpathlineto{\pgfqpoint{3.774193in}{2.210756in}}%
\pgfpathlineto{\pgfqpoint{3.781924in}{2.204752in}}%
\pgfpathlineto{\pgfqpoint{3.784502in}{2.207754in}}%
\pgfpathlineto{\pgfqpoint{3.787079in}{2.195182in}}%
\pgfpathlineto{\pgfqpoint{3.789656in}{2.170227in}}%
\pgfpathlineto{\pgfqpoint{3.792233in}{2.195933in}}%
\pgfpathlineto{\pgfqpoint{3.802542in}{2.150150in}}%
\pgfpathlineto{\pgfqpoint{3.805119in}{2.160282in}}%
\pgfpathlineto{\pgfqpoint{3.807697in}{2.191242in}}%
\pgfpathlineto{\pgfqpoint{3.810274in}{2.165349in}}%
\pgfpathlineto{\pgfqpoint{3.818005in}{2.166287in}}%
\pgfpathlineto{\pgfqpoint{3.820583in}{2.163472in}}%
\pgfpathlineto{\pgfqpoint{3.823160in}{2.153340in}}%
\pgfpathlineto{\pgfqpoint{3.825737in}{2.168351in}}%
\pgfpathlineto{\pgfqpoint{3.828314in}{2.131199in}}%
\pgfpathlineto{\pgfqpoint{3.836046in}{2.142645in}}%
\pgfpathlineto{\pgfqpoint{3.838623in}{2.169664in}}%
\pgfpathlineto{\pgfqpoint{3.841200in}{2.151464in}}%
\pgfpathlineto{\pgfqpoint{3.843778in}{2.169664in}}%
\pgfpathlineto{\pgfqpoint{3.846355in}{2.147148in}}%
\pgfpathlineto{\pgfqpoint{3.854086in}{2.125195in}}%
\pgfpathlineto{\pgfqpoint{3.856664in}{2.134577in}}%
\pgfpathlineto{\pgfqpoint{3.859241in}{2.135140in}}%
\pgfpathlineto{\pgfqpoint{3.861818in}{2.102867in}}%
\pgfpathlineto{\pgfqpoint{3.864395in}{2.122568in}}%
\pgfpathlineto{\pgfqpoint{3.874704in}{2.136078in}}%
\pgfpathlineto{\pgfqpoint{3.877281in}{2.128197in}}%
\pgfpathlineto{\pgfqpoint{3.879859in}{2.140393in}}%
\pgfpathlineto{\pgfqpoint{3.882436in}{2.144709in}}%
\pgfpathlineto{\pgfqpoint{3.890167in}{2.143395in}}%
\pgfpathlineto{\pgfqpoint{3.895322in}{2.160658in}}%
\pgfpathlineto{\pgfqpoint{3.897899in}{2.187302in}}%
\pgfpathlineto{\pgfqpoint{3.900476in}{2.182423in}}%
\pgfpathlineto{\pgfqpoint{3.908208in}{2.194244in}}%
\pgfpathlineto{\pgfqpoint{3.913362in}{2.167975in}}%
\pgfpathlineto{\pgfqpoint{3.915940in}{2.182611in}}%
\pgfpathlineto{\pgfqpoint{3.918517in}{2.142270in}}%
\pgfpathlineto{\pgfqpoint{3.926248in}{2.151464in}}%
\pgfpathlineto{\pgfqpoint{3.931403in}{2.112436in}}%
\pgfpathlineto{\pgfqpoint{3.933980in}{2.137579in}}%
\pgfpathlineto{\pgfqpoint{3.936557in}{2.127259in}}%
\pgfpathlineto{\pgfqpoint{3.944289in}{2.158218in}}%
\pgfpathlineto{\pgfqpoint{3.946866in}{2.138517in}}%
\pgfpathlineto{\pgfqpoint{3.949443in}{2.164786in}}%
\pgfpathlineto{\pgfqpoint{3.952020in}{2.168726in}}%
\pgfpathlineto{\pgfqpoint{3.954598in}{2.180547in}}%
\pgfpathlineto{\pgfqpoint{3.962329in}{2.190304in}}%
\pgfpathlineto{\pgfqpoint{3.964907in}{2.173229in}}%
\pgfpathlineto{\pgfqpoint{3.967484in}{2.146210in}}%
\pgfpathlineto{\pgfqpoint{3.970061in}{2.142833in}}%
\pgfpathlineto{\pgfqpoint{3.972638in}{2.146210in}}%
\pgfpathlineto{\pgfqpoint{3.980370in}{2.166287in}}%
\pgfpathlineto{\pgfqpoint{3.982947in}{2.150525in}}%
\pgfpathlineto{\pgfqpoint{3.985524in}{2.126321in}}%
\pgfpathlineto{\pgfqpoint{3.988101in}{2.134389in}}%
\pgfpathlineto{\pgfqpoint{3.998410in}{2.126321in}}%
\pgfpathlineto{\pgfqpoint{4.000988in}{2.142082in}}%
\pgfpathlineto{\pgfqpoint{4.003565in}{2.143020in}}%
\pgfpathlineto{\pgfqpoint{4.006142in}{2.162722in}}%
\pgfpathlineto{\pgfqpoint{4.008719in}{2.174918in}}%
\pgfpathlineto{\pgfqpoint{4.016451in}{2.149587in}}%
\pgfpathlineto{\pgfqpoint{4.019028in}{2.149212in}}%
\pgfpathlineto{\pgfqpoint{4.021605in}{2.150525in}}%
\pgfpathlineto{\pgfqpoint{4.024182in}{2.137016in}}%
\pgfpathlineto{\pgfqpoint{4.026760in}{2.133451in}}%
\pgfpathlineto{\pgfqpoint{4.034491in}{2.143958in}}%
\pgfpathlineto{\pgfqpoint{4.037069in}{2.145459in}}%
\pgfpathlineto{\pgfqpoint{4.039646in}{2.147711in}}%
\pgfpathlineto{\pgfqpoint{4.042223in}{2.148086in}}%
\pgfpathlineto{\pgfqpoint{4.044800in}{2.158406in}}%
\pgfpathlineto{\pgfqpoint{4.052532in}{2.150150in}}%
\pgfpathlineto{\pgfqpoint{4.055109in}{2.152777in}}%
\pgfpathlineto{\pgfqpoint{4.057686in}{2.146960in}}%
\pgfpathlineto{\pgfqpoint{4.060263in}{2.127071in}}%
\pgfpathlineto{\pgfqpoint{4.062841in}{2.142645in}}%
\pgfpathlineto{\pgfqpoint{4.070572in}{2.146210in}}%
\pgfpathlineto{\pgfqpoint{4.073149in}{2.132137in}}%
\pgfpathlineto{\pgfqpoint{4.075727in}{2.126508in}}%
\pgfpathlineto{\pgfqpoint{4.078304in}{2.134764in}}%
\pgfpathlineto{\pgfqpoint{4.080881in}{2.164973in}}%
\pgfpathlineto{\pgfqpoint{4.088613in}{2.157656in}}%
\pgfpathlineto{\pgfqpoint{4.091190in}{2.148274in}}%
\pgfpathlineto{\pgfqpoint{4.093767in}{2.149587in}}%
\pgfpathlineto{\pgfqpoint{4.096344in}{2.170978in}}%
\pgfpathlineto{\pgfqpoint{4.098922in}{2.178858in}}%
\pgfpathlineto{\pgfqpoint{4.106653in}{2.197622in}}%
\pgfpathlineto{\pgfqpoint{4.109230in}{2.206628in}}%
\pgfpathlineto{\pgfqpoint{4.114385in}{2.193306in}}%
\pgfpathlineto{\pgfqpoint{4.116962in}{2.175293in}}%
\pgfpathlineto{\pgfqpoint{4.127271in}{2.166474in}}%
\pgfpathlineto{\pgfqpoint{4.129848in}{2.171728in}}%
\pgfpathlineto{\pgfqpoint{4.132425in}{2.171916in}}%
\pgfpathlineto{\pgfqpoint{4.135003in}{2.155029in}}%
\pgfpathlineto{\pgfqpoint{4.145311in}{2.152214in}}%
\pgfpathlineto{\pgfqpoint{4.147889in}{2.154091in}}%
\pgfpathlineto{\pgfqpoint{4.150466in}{2.139643in}}%
\pgfpathlineto{\pgfqpoint{4.153043in}{2.128948in}}%
\pgfpathlineto{\pgfqpoint{4.160775in}{2.118253in}}%
\pgfpathlineto{\pgfqpoint{4.163352in}{2.122568in}}%
\pgfpathlineto{\pgfqpoint{4.165929in}{2.133826in}}%
\pgfpathlineto{\pgfqpoint{4.168506in}{2.139830in}}%
\pgfpathlineto{\pgfqpoint{4.171084in}{2.124820in}}%
\pgfpathlineto{\pgfqpoint{4.178815in}{2.110372in}}%
\pgfpathlineto{\pgfqpoint{4.181392in}{2.125007in}}%
\pgfpathlineto{\pgfqpoint{4.183970in}{2.131199in}}%
\pgfpathlineto{\pgfqpoint{4.186547in}{2.158594in}}%
\pgfpathlineto{\pgfqpoint{4.189124in}{2.150338in}}%
\pgfpathlineto{\pgfqpoint{4.196856in}{2.154091in}}%
\pgfpathlineto{\pgfqpoint{4.199433in}{2.148837in}}%
\pgfpathlineto{\pgfqpoint{4.202010in}{2.141331in}}%
\pgfpathlineto{\pgfqpoint{4.204587in}{2.137766in}}%
\pgfpathlineto{\pgfqpoint{4.207165in}{2.146585in}}%
\pgfpathlineto{\pgfqpoint{4.214896in}{2.113562in}}%
\pgfpathlineto{\pgfqpoint{4.217473in}{2.109809in}}%
\pgfpathlineto{\pgfqpoint{4.220051in}{2.126884in}}%
\pgfpathlineto{\pgfqpoint{4.222628in}{2.126321in}}%
\pgfpathlineto{\pgfqpoint{4.232937in}{2.122380in}}%
\pgfpathlineto{\pgfqpoint{4.235514in}{2.134389in}}%
\pgfpathlineto{\pgfqpoint{4.238091in}{2.115063in}}%
\pgfpathlineto{\pgfqpoint{4.240668in}{2.125758in}}%
\pgfpathlineto{\pgfqpoint{4.243246in}{2.144709in}}%
\pgfpathlineto{\pgfqpoint{4.250977in}{2.157093in}}%
\pgfpathlineto{\pgfqpoint{4.253554in}{2.148837in}}%
\pgfpathlineto{\pgfqpoint{4.258709in}{2.171353in}}%
\pgfpathlineto{\pgfqpoint{4.261286in}{2.153903in}}%
\pgfpathlineto{\pgfqpoint{4.269018in}{2.158781in}}%
\pgfpathlineto{\pgfqpoint{4.271595in}{2.158406in}}%
\pgfpathlineto{\pgfqpoint{4.274172in}{2.155592in}}%
\pgfpathlineto{\pgfqpoint{4.276749in}{2.155967in}}%
\pgfpathlineto{\pgfqpoint{4.279326in}{2.138329in}}%
\pgfpathlineto{\pgfqpoint{4.287058in}{2.123694in}}%
\pgfpathlineto{\pgfqpoint{4.292213in}{2.148086in}}%
\pgfpathlineto{\pgfqpoint{4.294790in}{2.149963in}}%
\pgfpathlineto{\pgfqpoint{4.297367in}{2.156155in}}%
\pgfpathlineto{\pgfqpoint{4.305099in}{2.152965in}}%
\pgfpathlineto{\pgfqpoint{4.307676in}{2.149212in}}%
\pgfpathlineto{\pgfqpoint{4.310253in}{2.161408in}}%
\pgfpathlineto{\pgfqpoint{4.312830in}{2.137016in}}%
\pgfpathlineto{\pgfqpoint{4.315407in}{2.133263in}}%
\pgfpathlineto{\pgfqpoint{4.323139in}{2.149024in}}%
\pgfpathlineto{\pgfqpoint{4.325716in}{2.135890in}}%
\pgfpathlineto{\pgfqpoint{4.328294in}{2.130824in}}%
\pgfpathlineto{\pgfqpoint{4.330871in}{2.127071in}}%
\pgfpathlineto{\pgfqpoint{4.333448in}{2.132513in}}%
\pgfpathlineto{\pgfqpoint{4.341180in}{2.150525in}}%
\pgfpathlineto{\pgfqpoint{4.343757in}{2.142082in}}%
\pgfpathlineto{\pgfqpoint{4.346334in}{2.140956in}}%
\pgfpathlineto{\pgfqpoint{4.348911in}{2.132325in}}%
\pgfpathlineto{\pgfqpoint{4.351488in}{2.090107in}}%
\pgfpathlineto{\pgfqpoint{4.359220in}{2.043762in}}%
\pgfpathlineto{\pgfqpoint{4.361797in}{2.008299in}}%
\pgfpathlineto{\pgfqpoint{4.364374in}{2.082602in}}%
\pgfpathlineto{\pgfqpoint{4.366952in}{2.101365in}}%
\pgfpathlineto{\pgfqpoint{4.369529in}{2.083540in}}%
\pgfpathlineto{\pgfqpoint{4.377261in}{2.063464in}}%
\pgfpathlineto{\pgfqpoint{4.379838in}{2.031378in}}%
\pgfpathlineto{\pgfqpoint{4.382415in}{2.052768in}}%
\pgfpathlineto{\pgfqpoint{4.384992in}{2.040760in}}%
\pgfpathlineto{\pgfqpoint{4.387569in}{2.018056in}}%
\pgfpathlineto{\pgfqpoint{4.397878in}{2.062713in}}%
\pgfpathlineto{\pgfqpoint{4.400455in}{2.033630in}}%
\pgfpathlineto{\pgfqpoint{4.403033in}{2.042073in}}%
\pgfpathlineto{\pgfqpoint{4.405610in}{2.045638in}}%
\pgfpathlineto{\pgfqpoint{4.413342in}{2.052206in}}%
\pgfpathlineto{\pgfqpoint{4.415919in}{2.070594in}}%
\pgfpathlineto{\pgfqpoint{4.421073in}{2.077911in}}%
\pgfpathlineto{\pgfqpoint{4.423650in}{2.053331in}}%
\pgfpathlineto{\pgfqpoint{4.431382in}{2.049016in}}%
\pgfpathlineto{\pgfqpoint{4.433959in}{2.050892in}}%
\pgfpathlineto{\pgfqpoint{4.436536in}{2.046576in}}%
\pgfpathlineto{\pgfqpoint{4.439114in}{2.037945in}}%
\pgfpathlineto{\pgfqpoint{4.441691in}{2.012802in}}%
\pgfpathlineto{\pgfqpoint{4.449422in}{2.019182in}}%
\pgfpathlineto{\pgfqpoint{4.452000in}{2.047515in}}%
\pgfpathlineto{\pgfqpoint{4.454577in}{2.052768in}}%
\pgfpathlineto{\pgfqpoint{4.457154in}{2.049579in}}%
\pgfpathlineto{\pgfqpoint{4.459731in}{2.062525in}}%
\pgfpathlineto{\pgfqpoint{4.467463in}{2.076598in}}%
\pgfpathlineto{\pgfqpoint{4.470040in}{2.053894in}}%
\pgfpathlineto{\pgfqpoint{4.472617in}{2.080351in}}%
\pgfpathlineto{\pgfqpoint{4.475195in}{2.082039in}}%
\pgfpathlineto{\pgfqpoint{4.477772in}{2.086918in}}%
\pgfpathlineto{\pgfqpoint{4.485503in}{2.097425in}}%
\pgfpathlineto{\pgfqpoint{4.488081in}{2.088419in}}%
\pgfpathlineto{\pgfqpoint{4.490658in}{2.072658in}}%
\pgfpathlineto{\pgfqpoint{4.493235in}{2.117127in}}%
\pgfpathlineto{\pgfqpoint{4.495812in}{2.135702in}}%
\pgfpathlineto{\pgfqpoint{4.503544in}{2.130636in}}%
\pgfpathlineto{\pgfqpoint{4.506121in}{2.124632in}}%
\pgfpathlineto{\pgfqpoint{4.508698in}{2.125383in}}%
\pgfpathlineto{\pgfqpoint{4.511276in}{2.157468in}}%
\pgfpathlineto{\pgfqpoint{4.513853in}{2.170790in}}%
\pgfpathlineto{\pgfqpoint{4.521584in}{2.164410in}}%
\pgfpathlineto{\pgfqpoint{4.526739in}{2.173604in}}%
\pgfpathlineto{\pgfqpoint{4.529316in}{2.188803in}}%
\pgfpathlineto{\pgfqpoint{4.531893in}{2.182986in}}%
\pgfpathlineto{\pgfqpoint{4.539625in}{2.202312in}}%
\pgfpathlineto{\pgfqpoint{4.542202in}{2.198747in}}%
\pgfpathlineto{\pgfqpoint{4.544779in}{2.198372in}}%
\pgfpathlineto{\pgfqpoint{4.547357in}{2.204939in}}%
\pgfpathlineto{\pgfqpoint{4.549934in}{2.197997in}}%
\pgfpathlineto{\pgfqpoint{4.557665in}{2.179796in}}%
\pgfpathlineto{\pgfqpoint{4.560243in}{2.190116in}}%
\pgfpathlineto{\pgfqpoint{4.562820in}{2.197059in}}%
\pgfpathlineto{\pgfqpoint{4.565397in}{2.170039in}}%
\pgfpathlineto{\pgfqpoint{4.567974in}{2.163472in}}%
\pgfpathlineto{\pgfqpoint{4.575706in}{2.186176in}}%
\pgfpathlineto{\pgfqpoint{4.578283in}{2.190867in}}%
\pgfpathlineto{\pgfqpoint{4.580860in}{2.210756in}}%
\pgfpathlineto{\pgfqpoint{4.583438in}{2.207754in}}%
\pgfpathlineto{\pgfqpoint{4.586015in}{2.220325in}}%
\pgfpathlineto{\pgfqpoint{4.593746in}{2.225391in}}%
\pgfpathlineto{\pgfqpoint{4.596324in}{2.212820in}}%
\pgfpathlineto{\pgfqpoint{4.598901in}{2.211506in}}%
\pgfpathlineto{\pgfqpoint{4.604055in}{2.218449in}}%
\pgfpathlineto{\pgfqpoint{4.611787in}{2.199123in}}%
\pgfpathlineto{\pgfqpoint{4.614364in}{2.218261in}}%
\pgfpathlineto{\pgfqpoint{4.616941in}{2.213007in}}%
\pgfpathlineto{\pgfqpoint{4.619519in}{2.191430in}}%
\pgfpathlineto{\pgfqpoint{4.622096in}{2.228393in}}%
\pgfpathlineto{\pgfqpoint{4.629827in}{2.234773in}}%
\pgfpathlineto{\pgfqpoint{4.632405in}{2.219387in}}%
\pgfpathlineto{\pgfqpoint{4.634982in}{2.214696in}}%
\pgfpathlineto{\pgfqpoint{4.637559in}{2.223140in}}%
\pgfpathlineto{\pgfqpoint{4.640136in}{2.206628in}}%
\pgfpathlineto{\pgfqpoint{4.647868in}{2.214696in}}%
\pgfpathlineto{\pgfqpoint{4.650445in}{2.248470in}}%
\pgfpathlineto{\pgfqpoint{4.653022in}{2.267609in}}%
\pgfpathlineto{\pgfqpoint{4.658177in}{2.211319in}}%
\pgfpathlineto{\pgfqpoint{4.665908in}{2.204939in}}%
\pgfpathlineto{\pgfqpoint{4.668486in}{2.224265in}}%
\pgfpathlineto{\pgfqpoint{4.671063in}{2.237587in}}%
\pgfpathlineto{\pgfqpoint{4.673640in}{2.241528in}}%
\pgfpathlineto{\pgfqpoint{4.683949in}{2.233084in}}%
\pgfpathlineto{\pgfqpoint{4.686526in}{2.246782in}}%
\pgfpathlineto{\pgfqpoint{4.689103in}{2.242654in}}%
\pgfpathlineto{\pgfqpoint{4.691680in}{2.224453in}}%
\pgfpathlineto{\pgfqpoint{4.701989in}{2.186176in}}%
\pgfpathlineto{\pgfqpoint{4.704567in}{2.193306in}}%
\pgfpathlineto{\pgfqpoint{4.707144in}{2.184675in}}%
\pgfpathlineto{\pgfqpoint{4.712298in}{2.146585in}}%
\pgfpathlineto{\pgfqpoint{4.720030in}{2.136453in}}%
\pgfpathlineto{\pgfqpoint{4.722607in}{2.147899in}}%
\pgfpathlineto{\pgfqpoint{4.725184in}{2.127071in}}%
\pgfpathlineto{\pgfqpoint{4.727761in}{2.158969in}}%
\pgfpathlineto{\pgfqpoint{4.730339in}{2.126696in}}%
\pgfpathlineto{\pgfqpoint{4.740648in}{2.135327in}}%
\pgfpathlineto{\pgfqpoint{4.743225in}{2.105306in}}%
\pgfpathlineto{\pgfqpoint{4.745802in}{2.108683in}}%
\pgfpathlineto{\pgfqpoint{4.748379in}{2.122568in}}%
\pgfpathlineto{\pgfqpoint{4.756111in}{2.116564in}}%
\pgfpathlineto{\pgfqpoint{4.758688in}{2.198184in}}%
\pgfpathlineto{\pgfqpoint{4.761265in}{2.214884in}}%
\pgfpathlineto{\pgfqpoint{4.763842in}{2.216760in}}%
\pgfpathlineto{\pgfqpoint{4.766420in}{2.253912in}}%
\pgfpathlineto{\pgfqpoint{4.774151in}{2.252598in}}%
\pgfpathlineto{\pgfqpoint{4.776728in}{2.236274in}}%
\pgfpathlineto{\pgfqpoint{4.779306in}{2.248658in}}%
\pgfpathlineto{\pgfqpoint{4.781883in}{2.244718in}}%
\pgfpathlineto{\pgfqpoint{4.784460in}{2.187302in}}%
\pgfpathlineto{\pgfqpoint{4.792192in}{2.212257in}}%
\pgfpathlineto{\pgfqpoint{4.794769in}{2.211694in}}%
\pgfpathlineto{\pgfqpoint{4.797346in}{2.207566in}}%
\pgfpathlineto{\pgfqpoint{4.799923in}{2.207003in}}%
\pgfpathlineto{\pgfqpoint{4.812809in}{2.217698in}}%
\pgfpathlineto{\pgfqpoint{4.815387in}{2.220700in}}%
\pgfpathlineto{\pgfqpoint{4.817964in}{2.250534in}}%
\pgfpathlineto{\pgfqpoint{4.820541in}{2.261980in}}%
\pgfpathlineto{\pgfqpoint{4.828273in}{2.272112in}}%
\pgfpathlineto{\pgfqpoint{4.830850in}{2.260666in}}%
\pgfpathlineto{\pgfqpoint{4.833427in}{2.275677in}}%
\pgfpathlineto{\pgfqpoint{4.836004in}{2.300257in}}%
\pgfpathlineto{\pgfqpoint{4.838582in}{2.289937in}}%
\pgfpathlineto{\pgfqpoint{4.846313in}{2.279993in}}%
\pgfpathlineto{\pgfqpoint{4.848890in}{2.314705in}}%
\pgfpathlineto{\pgfqpoint{4.851468in}{2.311515in}}%
\pgfpathlineto{\pgfqpoint{4.854045in}{2.304760in}}%
\pgfpathlineto{\pgfqpoint{4.856622in}{2.302321in}}%
\pgfpathlineto{\pgfqpoint{4.864354in}{2.306449in}}%
\pgfpathlineto{\pgfqpoint{4.866931in}{2.297443in}}%
\pgfpathlineto{\pgfqpoint{4.869508in}{2.308138in}}%
\pgfpathlineto{\pgfqpoint{4.872085in}{2.313204in}}%
\pgfpathlineto{\pgfqpoint{4.874663in}{2.323148in}}%
\pgfpathlineto{\pgfqpoint{4.882394in}{2.322210in}}%
\pgfpathlineto{\pgfqpoint{4.884971in}{2.323899in}}%
\pgfpathlineto{\pgfqpoint{4.887549in}{2.317895in}}%
\pgfpathlineto{\pgfqpoint{4.890126in}{2.306449in}}%
\pgfpathlineto{\pgfqpoint{4.892703in}{2.319396in}}%
\pgfpathlineto{\pgfqpoint{4.900435in}{2.315643in}}%
\pgfpathlineto{\pgfqpoint{4.903012in}{2.317332in}}%
\pgfpathlineto{\pgfqpoint{4.905589in}{2.336658in}}%
\pgfpathlineto{\pgfqpoint{4.908166in}{2.333468in}}%
\pgfpathlineto{\pgfqpoint{4.918475in}{2.331967in}}%
\pgfpathlineto{\pgfqpoint{4.921052in}{2.347728in}}%
\pgfpathlineto{\pgfqpoint{4.923630in}{2.344914in}}%
\pgfpathlineto{\pgfqpoint{4.926207in}{2.331404in}}%
\pgfpathlineto{\pgfqpoint{4.928784in}{2.348479in}}%
\pgfpathlineto{\pgfqpoint{4.936516in}{2.338159in}}%
\pgfpathlineto{\pgfqpoint{4.939093in}{2.344351in}}%
\pgfpathlineto{\pgfqpoint{4.941670in}{2.352419in}}%
\pgfpathlineto{\pgfqpoint{4.946825in}{2.346978in}}%
\pgfpathlineto{\pgfqpoint{4.954556in}{2.344726in}}%
\pgfpathlineto{\pgfqpoint{4.957133in}{2.355609in}}%
\pgfpathlineto{\pgfqpoint{4.959711in}{2.360300in}}%
\pgfpathlineto{\pgfqpoint{4.962288in}{2.359737in}}%
\pgfpathlineto{\pgfqpoint{4.964865in}{2.365554in}}%
\pgfpathlineto{\pgfqpoint{4.972597in}{2.378500in}}%
\pgfpathlineto{\pgfqpoint{4.975174in}{2.408522in}}%
\pgfpathlineto{\pgfqpoint{4.977751in}{2.424095in}}%
\pgfpathlineto{\pgfqpoint{4.980328in}{2.423908in}}%
\pgfpathlineto{\pgfqpoint{4.982905in}{2.419592in}}%
\pgfpathlineto{\pgfqpoint{4.990637in}{2.422782in}}%
\pgfpathlineto{\pgfqpoint{4.993214in}{2.412462in}}%
\pgfpathlineto{\pgfqpoint{4.995792in}{2.410210in}}%
\pgfpathlineto{\pgfqpoint{4.998369in}{2.405144in}}%
\pgfpathlineto{\pgfqpoint{5.000946in}{2.398202in}}%
\pgfpathlineto{\pgfqpoint{5.008678in}{2.409835in}}%
\pgfpathlineto{\pgfqpoint{5.011255in}{2.408709in}}%
\pgfpathlineto{\pgfqpoint{5.013832in}{2.400641in}}%
\pgfpathlineto{\pgfqpoint{5.016409in}{2.412087in}}%
\pgfpathlineto{\pgfqpoint{5.018986in}{2.409647in}}%
\pgfpathlineto{\pgfqpoint{5.026718in}{2.426534in}}%
\pgfpathlineto{\pgfqpoint{5.029295in}{2.442859in}}%
\pgfpathlineto{\pgfqpoint{5.031873in}{2.437605in}}%
\pgfpathlineto{\pgfqpoint{5.034450in}{2.435353in}}%
\pgfpathlineto{\pgfqpoint{5.037027in}{2.423720in}}%
\pgfpathlineto{\pgfqpoint{5.044759in}{2.438918in}}%
\pgfpathlineto{\pgfqpoint{5.047336in}{2.428411in}}%
\pgfpathlineto{\pgfqpoint{5.049913in}{2.424283in}}%
\pgfpathlineto{\pgfqpoint{5.052490in}{2.411524in}}%
\pgfpathlineto{\pgfqpoint{5.055067in}{2.421656in}}%
\pgfpathlineto{\pgfqpoint{5.062799in}{2.413400in}}%
\pgfpathlineto{\pgfqpoint{5.065376in}{2.422594in}}%
\pgfpathlineto{\pgfqpoint{5.067953in}{2.434040in}}%
\pgfpathlineto{\pgfqpoint{5.070531in}{2.426159in}}%
\pgfpathlineto{\pgfqpoint{5.073108in}{2.428974in}}%
\pgfpathlineto{\pgfqpoint{5.083417in}{2.422594in}}%
\pgfpathlineto{\pgfqpoint{5.085994in}{2.424095in}}%
\pgfpathlineto{\pgfqpoint{5.088571in}{2.453741in}}%
\pgfpathlineto{\pgfqpoint{5.091148in}{2.458432in}}%
\pgfpathlineto{\pgfqpoint{5.098880in}{2.476070in}}%
\pgfpathlineto{\pgfqpoint{5.101457in}{2.475319in}}%
\pgfpathlineto{\pgfqpoint{5.104034in}{2.476445in}}%
\pgfpathlineto{\pgfqpoint{5.106612in}{2.497272in}}%
\pgfpathlineto{\pgfqpoint{5.109189in}{2.497460in}}%
\pgfpathlineto{\pgfqpoint{5.116921in}{2.493707in}}%
\pgfpathlineto{\pgfqpoint{5.119498in}{2.499336in}}%
\pgfpathlineto{\pgfqpoint{5.122075in}{2.487140in}}%
\pgfpathlineto{\pgfqpoint{5.124652in}{2.491080in}}%
\pgfpathlineto{\pgfqpoint{5.127229in}{2.471004in}}%
\pgfpathlineto{\pgfqpoint{5.134961in}{2.489392in}}%
\pgfpathlineto{\pgfqpoint{5.137538in}{2.483012in}}%
\pgfpathlineto{\pgfqpoint{5.140115in}{2.487891in}}%
\pgfpathlineto{\pgfqpoint{5.142693in}{2.503840in}}%
\pgfpathlineto{\pgfqpoint{5.145270in}{2.473631in}}%
\pgfpathlineto{\pgfqpoint{5.153001in}{2.489579in}}%
\pgfpathlineto{\pgfqpoint{5.155579in}{2.518663in}}%
\pgfpathlineto{\pgfqpoint{5.158156in}{2.537614in}}%
\pgfpathlineto{\pgfqpoint{5.160733in}{2.571763in}}%
\pgfpathlineto{\pgfqpoint{5.163310in}{2.571575in}}%
\pgfpathlineto{\pgfqpoint{5.173619in}{2.588650in}}%
\pgfpathlineto{\pgfqpoint{5.176196in}{2.595029in}}%
\pgfpathlineto{\pgfqpoint{5.178774in}{2.592778in}}%
\pgfpathlineto{\pgfqpoint{5.181351in}{2.598782in}}%
\pgfpathlineto{\pgfqpoint{5.189082in}{2.600096in}}%
\pgfpathlineto{\pgfqpoint{5.191660in}{2.599908in}}%
\pgfpathlineto{\pgfqpoint{5.194237in}{2.601221in}}%
\pgfpathlineto{\pgfqpoint{5.196814in}{2.604411in}}%
\pgfpathlineto{\pgfqpoint{5.199391in}{2.601221in}}%
\pgfpathlineto{\pgfqpoint{5.207123in}{2.603661in}}%
\pgfpathlineto{\pgfqpoint{5.209700in}{2.640249in}}%
\pgfpathlineto{\pgfqpoint{5.212277in}{2.638373in}}%
\pgfpathlineto{\pgfqpoint{5.214855in}{2.638560in}}%
\pgfpathlineto{\pgfqpoint{5.217432in}{2.636496in}}%
\pgfpathlineto{\pgfqpoint{5.225163in}{2.634057in}}%
\pgfpathlineto{\pgfqpoint{5.227741in}{2.638560in}}%
\pgfpathlineto{\pgfqpoint{5.232895in}{2.626552in}}%
\pgfpathlineto{\pgfqpoint{5.235472in}{2.639874in}}%
\pgfpathlineto{\pgfqpoint{5.243204in}{2.642876in}}%
\pgfpathlineto{\pgfqpoint{5.245781in}{2.633307in}}%
\pgfpathlineto{\pgfqpoint{5.248358in}{2.616983in}}%
\pgfpathlineto{\pgfqpoint{5.250936in}{2.616232in}}%
\pgfpathlineto{\pgfqpoint{5.253513in}{2.622799in}}%
\pgfpathlineto{\pgfqpoint{5.263822in}{2.608727in}}%
\pgfpathlineto{\pgfqpoint{5.266399in}{2.607601in}}%
\pgfpathlineto{\pgfqpoint{5.268976in}{2.614731in}}%
\pgfpathlineto{\pgfqpoint{5.271553in}{2.605162in}}%
\pgfpathlineto{\pgfqpoint{5.279285in}{2.589400in}}%
\pgfpathlineto{\pgfqpoint{5.281862in}{2.555063in}}%
\pgfpathlineto{\pgfqpoint{5.284439in}{2.571951in}}%
\pgfpathlineto{\pgfqpoint{5.287017in}{2.561818in}}%
\pgfpathlineto{\pgfqpoint{5.289594in}{2.561818in}}%
\pgfpathlineto{\pgfqpoint{5.297325in}{2.547933in}}%
\pgfpathlineto{\pgfqpoint{5.299903in}{2.553375in}}%
\pgfpathlineto{\pgfqpoint{5.302480in}{2.540616in}}%
\pgfpathlineto{\pgfqpoint{5.305057in}{2.538176in}}%
\pgfpathlineto{\pgfqpoint{5.307634in}{2.546432in}}%
\pgfpathlineto{\pgfqpoint{5.315366in}{2.561818in}}%
\pgfpathlineto{\pgfqpoint{5.317943in}{2.553938in}}%
\pgfpathlineto{\pgfqpoint{5.320520in}{2.551686in}}%
\pgfpathlineto{\pgfqpoint{5.323098in}{2.547183in}}%
\pgfpathlineto{\pgfqpoint{5.325675in}{2.551311in}}%
\pgfpathlineto{\pgfqpoint{5.335984in}{2.558816in}}%
\pgfpathlineto{\pgfqpoint{5.341138in}{2.553938in}}%
\pgfpathlineto{\pgfqpoint{5.343715in}{2.532360in}}%
\pgfpathlineto{\pgfqpoint{5.351447in}{2.548496in}}%
\pgfpathlineto{\pgfqpoint{5.354024in}{2.521477in}}%
\pgfpathlineto{\pgfqpoint{5.356601in}{2.525793in}}%
\pgfpathlineto{\pgfqpoint{5.359178in}{2.539302in}}%
\pgfpathlineto{\pgfqpoint{5.361756in}{2.532735in}}%
\pgfpathlineto{\pgfqpoint{5.369487in}{2.522415in}}%
\pgfpathlineto{\pgfqpoint{5.372065in}{2.527481in}}%
\pgfpathlineto{\pgfqpoint{5.374642in}{2.544181in}}%
\pgfpathlineto{\pgfqpoint{5.377219in}{2.553750in}}%
\pgfpathlineto{\pgfqpoint{5.379796in}{2.542492in}}%
\pgfpathlineto{\pgfqpoint{5.387528in}{2.524479in}}%
\pgfpathlineto{\pgfqpoint{5.390105in}{2.549622in}}%
\pgfpathlineto{\pgfqpoint{5.392682in}{2.552624in}}%
\pgfpathlineto{\pgfqpoint{5.395259in}{2.515473in}}%
\pgfpathlineto{\pgfqpoint{5.397837in}{2.530483in}}%
\pgfpathlineto{\pgfqpoint{5.405568in}{2.542492in}}%
\pgfpathlineto{\pgfqpoint{5.408146in}{2.542680in}}%
\pgfpathlineto{\pgfqpoint{5.410723in}{2.548872in}}%
\pgfpathlineto{\pgfqpoint{5.413300in}{2.541366in}}%
\pgfpathlineto{\pgfqpoint{5.415877in}{2.549997in}}%
\pgfpathlineto{\pgfqpoint{5.423609in}{2.559754in}}%
\pgfpathlineto{\pgfqpoint{5.426186in}{2.522040in}}%
\pgfpathlineto{\pgfqpoint{5.431340in}{2.532735in}}%
\pgfpathlineto{\pgfqpoint{5.433918in}{2.520539in}}%
\pgfpathlineto{\pgfqpoint{5.441649in}{2.536863in}}%
\pgfpathlineto{\pgfqpoint{5.444227in}{2.483012in}}%
\pgfpathlineto{\pgfqpoint{5.446804in}{2.468752in}}%
\pgfpathlineto{\pgfqpoint{5.449381in}{2.473631in}}%
\pgfpathlineto{\pgfqpoint{5.451958in}{2.448675in}}%
\pgfpathlineto{\pgfqpoint{5.459690in}{2.451677in}}%
\pgfpathlineto{\pgfqpoint{5.462267in}{2.457869in}}%
\pgfpathlineto{\pgfqpoint{5.464844in}{2.468189in}}%
\pgfpathlineto{\pgfqpoint{5.467421in}{2.488078in}}%
\pgfpathlineto{\pgfqpoint{5.469999in}{2.481699in}}%
\pgfpathlineto{\pgfqpoint{5.477730in}{2.493144in}}%
\pgfpathlineto{\pgfqpoint{5.480307in}{2.481886in}}%
\pgfpathlineto{\pgfqpoint{5.482885in}{2.473443in}}%
\pgfpathlineto{\pgfqpoint{5.485462in}{2.476445in}}%
\pgfpathlineto{\pgfqpoint{5.488039in}{2.477946in}}%
\pgfpathlineto{\pgfqpoint{5.495771in}{2.504965in}}%
\pgfpathlineto{\pgfqpoint{5.498348in}{2.511720in}}%
\pgfpathlineto{\pgfqpoint{5.500925in}{2.568573in}}%
\pgfpathlineto{\pgfqpoint{5.503502in}{2.555063in}}%
\pgfpathlineto{\pgfqpoint{5.506080in}{2.536488in}}%
\pgfpathlineto{\pgfqpoint{5.513811in}{2.503840in}}%
\pgfpathlineto{\pgfqpoint{5.516388in}{2.498961in}}%
\pgfpathlineto{\pgfqpoint{5.518966in}{2.499712in}}%
\pgfpathlineto{\pgfqpoint{5.521543in}{2.503277in}}%
\pgfpathlineto{\pgfqpoint{5.524120in}{2.496147in}}%
\pgfpathlineto{\pgfqpoint{5.531852in}{2.489767in}}%
\pgfpathlineto{\pgfqpoint{5.534429in}{2.450176in}}%
\pgfpathlineto{\pgfqpoint{5.537006in}{2.455805in}}%
\pgfpathlineto{\pgfqpoint{5.542161in}{2.474569in}}%
\pgfpathlineto{\pgfqpoint{5.549892in}{2.456931in}}%
\pgfpathlineto{\pgfqpoint{5.552469in}{2.445485in}}%
\pgfpathlineto{\pgfqpoint{5.555047in}{2.424846in}}%
\pgfpathlineto{\pgfqpoint{5.557624in}{2.426159in}}%
\pgfpathlineto{\pgfqpoint{5.560201in}{2.436479in}}%
\pgfpathlineto{\pgfqpoint{5.567933in}{2.436104in}}%
\pgfpathlineto{\pgfqpoint{5.570510in}{2.438168in}}%
\pgfpathlineto{\pgfqpoint{5.573087in}{2.421281in}}%
\pgfpathlineto{\pgfqpoint{5.575664in}{2.419404in}}%
\pgfpathlineto{\pgfqpoint{5.578242in}{2.441733in}}%
\pgfpathlineto{\pgfqpoint{5.585973in}{2.496147in}}%
\pgfpathlineto{\pgfqpoint{5.588550in}{2.505528in}}%
\pgfpathlineto{\pgfqpoint{5.591128in}{2.489579in}}%
\pgfpathlineto{\pgfqpoint{5.593705in}{2.505528in}}%
\pgfpathlineto{\pgfqpoint{5.596282in}{2.505341in}}%
\pgfpathlineto{\pgfqpoint{5.604014in}{2.507780in}}%
\pgfpathlineto{\pgfqpoint{5.609168in}{2.495208in}}%
\pgfpathlineto{\pgfqpoint{5.611745in}{2.497460in}}%
\pgfpathlineto{\pgfqpoint{5.614323in}{2.506654in}}%
\pgfpathlineto{\pgfqpoint{5.624631in}{2.505716in}}%
\pgfpathlineto{\pgfqpoint{5.627209in}{2.491643in}}%
\pgfpathlineto{\pgfqpoint{5.629786in}{2.498398in}}%
\pgfpathlineto{\pgfqpoint{5.632363in}{2.493520in}}%
\pgfpathlineto{\pgfqpoint{5.642672in}{2.504590in}}%
\pgfpathlineto{\pgfqpoint{5.645249in}{2.501213in}}%
\pgfpathlineto{\pgfqpoint{5.647826in}{2.522415in}}%
\pgfpathlineto{\pgfqpoint{5.650403in}{2.512658in}}%
\pgfpathlineto{\pgfqpoint{5.658135in}{2.512283in}}%
\pgfpathlineto{\pgfqpoint{5.660712in}{2.510219in}}%
\pgfpathlineto{\pgfqpoint{5.663290in}{2.485076in}}%
\pgfpathlineto{\pgfqpoint{5.665867in}{2.483200in}}%
\pgfpathlineto{\pgfqpoint{5.668444in}{2.482825in}}%
\pgfpathlineto{\pgfqpoint{5.678753in}{2.487515in}}%
\pgfpathlineto{\pgfqpoint{5.681330in}{2.484513in}}%
\pgfpathlineto{\pgfqpoint{5.683907in}{2.475694in}}%
\pgfpathlineto{\pgfqpoint{5.686484in}{2.474944in}}%
\pgfpathlineto{\pgfqpoint{5.694216in}{2.470628in}}%
\pgfpathlineto{\pgfqpoint{5.696793in}{2.432914in}}%
\pgfpathlineto{\pgfqpoint{5.699371in}{2.451114in}}%
\pgfpathlineto{\pgfqpoint{5.701948in}{2.434227in}}%
\pgfpathlineto{\pgfqpoint{5.704525in}{2.461434in}}%
\pgfpathlineto{\pgfqpoint{5.712257in}{2.456931in}}%
\pgfpathlineto{\pgfqpoint{5.714834in}{2.458995in}}%
\pgfpathlineto{\pgfqpoint{5.717411in}{2.458620in}}%
\pgfpathlineto{\pgfqpoint{5.719988in}{2.464624in}}%
\pgfpathlineto{\pgfqpoint{5.722565in}{2.465938in}}%
\pgfpathlineto{\pgfqpoint{5.730297in}{2.461622in}}%
\pgfpathlineto{\pgfqpoint{5.732874in}{2.463123in}}%
\pgfpathlineto{\pgfqpoint{5.735452in}{2.461622in}}%
\pgfpathlineto{\pgfqpoint{5.738029in}{2.473631in}}%
\pgfpathlineto{\pgfqpoint{5.740606in}{2.494083in}}%
\pgfpathlineto{\pgfqpoint{5.748338in}{2.505341in}}%
\pgfpathlineto{\pgfqpoint{5.750915in}{2.513784in}}%
\pgfpathlineto{\pgfqpoint{5.758646in}{2.557690in}}%
\pgfpathlineto{\pgfqpoint{5.768955in}{2.571763in}}%
\pgfpathlineto{\pgfqpoint{5.771532in}{2.569136in}}%
\pgfpathlineto{\pgfqpoint{5.776687in}{2.639874in}}%
\pgfpathlineto{\pgfqpoint{5.784419in}{2.634057in}}%
\pgfpathlineto{\pgfqpoint{5.786996in}{2.630680in}}%
\pgfpathlineto{\pgfqpoint{5.789573in}{2.659763in}}%
\pgfpathlineto{\pgfqpoint{5.792150in}{2.655823in}}%
\pgfpathlineto{\pgfqpoint{5.794727in}{2.658637in}}%
\pgfpathlineto{\pgfqpoint{5.802459in}{2.657136in}}%
\pgfpathlineto{\pgfqpoint{5.805036in}{2.659388in}}%
\pgfpathlineto{\pgfqpoint{5.807613in}{2.664079in}}%
\pgfpathlineto{\pgfqpoint{5.810191in}{2.696727in}}%
\pgfpathlineto{\pgfqpoint{5.812768in}{2.701418in}}%
\pgfpathlineto{\pgfqpoint{5.820500in}{2.709674in}}%
\pgfpathlineto{\pgfqpoint{5.823077in}{2.716241in}}%
\pgfpathlineto{\pgfqpoint{5.825654in}{2.750015in}}%
\pgfpathlineto{\pgfqpoint{5.830808in}{2.734066in}}%
\pgfpathlineto{\pgfqpoint{5.838540in}{2.734254in}}%
\pgfpathlineto{\pgfqpoint{5.843694in}{2.702356in}}%
\pgfpathlineto{\pgfqpoint{5.848849in}{2.688471in}}%
\pgfpathlineto{\pgfqpoint{5.856580in}{2.694100in}}%
\pgfpathlineto{\pgfqpoint{5.859158in}{2.691661in}}%
\pgfpathlineto{\pgfqpoint{5.861735in}{2.678526in}}%
\pgfpathlineto{\pgfqpoint{5.864312in}{2.674023in}}%
\pgfpathlineto{\pgfqpoint{5.866889in}{2.672147in}}%
\pgfpathlineto{\pgfqpoint{5.874621in}{2.674586in}}%
\pgfpathlineto{\pgfqpoint{5.877198in}{2.674398in}}%
\pgfpathlineto{\pgfqpoint{5.879775in}{2.676462in}}%
\pgfpathlineto{\pgfqpoint{5.882353in}{2.680966in}}%
\pgfpathlineto{\pgfqpoint{5.884930in}{2.678526in}}%
\pgfpathlineto{\pgfqpoint{5.892661in}{2.668394in}}%
\pgfpathlineto{\pgfqpoint{5.895239in}{2.666143in}}%
\pgfpathlineto{\pgfqpoint{5.897816in}{2.687158in}}%
\pgfpathlineto{\pgfqpoint{5.900393in}{2.679840in}}%
\pgfpathlineto{\pgfqpoint{5.910702in}{2.692787in}}%
\pgfpathlineto{\pgfqpoint{5.913279in}{2.623737in}}%
\pgfpathlineto{\pgfqpoint{5.915856in}{2.615857in}}%
\pgfpathlineto{\pgfqpoint{5.918434in}{2.624676in}}%
\pgfpathlineto{\pgfqpoint{5.921011in}{2.622799in}}%
\pgfpathlineto{\pgfqpoint{5.928742in}{2.642688in}}%
\pgfpathlineto{\pgfqpoint{5.931320in}{2.652258in}}%
\pgfpathlineto{\pgfqpoint{5.933897in}{2.653571in}}%
\pgfpathlineto{\pgfqpoint{5.936474in}{2.657699in}}%
\pgfpathlineto{\pgfqpoint{5.939051in}{2.653008in}}%
\pgfpathlineto{\pgfqpoint{5.946783in}{2.650569in}}%
\pgfpathlineto{\pgfqpoint{5.949360in}{2.656949in}}%
\pgfpathlineto{\pgfqpoint{5.951937in}{2.650569in}}%
\pgfpathlineto{\pgfqpoint{5.954515in}{2.661452in}}%
\pgfpathlineto{\pgfqpoint{5.957092in}{2.653571in}}%
\pgfpathlineto{\pgfqpoint{5.964823in}{2.649068in}}%
\pgfpathlineto{\pgfqpoint{5.967401in}{2.648317in}}%
\pgfpathlineto{\pgfqpoint{5.969978in}{2.641375in}}%
\pgfpathlineto{\pgfqpoint{5.975132in}{2.656010in}}%
\pgfpathlineto{\pgfqpoint{5.982864in}{2.715115in}}%
\pgfpathlineto{\pgfqpoint{5.985441in}{2.729000in}}%
\pgfpathlineto{\pgfqpoint{5.988018in}{2.709486in}}%
\pgfpathlineto{\pgfqpoint{5.990596in}{2.714927in}}%
\pgfpathlineto{\pgfqpoint{5.993173in}{2.715303in}}%
\pgfpathlineto{\pgfqpoint{6.000904in}{2.719993in}}%
\pgfpathlineto{\pgfqpoint{6.003482in}{2.724497in}}%
\pgfpathlineto{\pgfqpoint{6.006059in}{2.724309in}}%
\pgfpathlineto{\pgfqpoint{6.008636in}{2.739695in}}%
\pgfpathlineto{\pgfqpoint{6.011213in}{2.728812in}}%
\pgfpathlineto{\pgfqpoint{6.021522in}{2.732190in}}%
\pgfpathlineto{\pgfqpoint{6.024099in}{2.752454in}}%
\pgfpathlineto{\pgfqpoint{6.026677in}{2.762023in}}%
\pgfpathlineto{\pgfqpoint{6.029254in}{2.785102in}}%
\pgfpathlineto{\pgfqpoint{6.036985in}{2.790168in}}%
\pgfpathlineto{\pgfqpoint{6.039563in}{2.798424in}}%
\pgfpathlineto{\pgfqpoint{6.042140in}{2.797111in}}%
\pgfpathlineto{\pgfqpoint{6.044717in}{2.793546in}}%
\pgfpathlineto{\pgfqpoint{6.047294in}{2.810996in}}%
\pgfpathlineto{\pgfqpoint{6.055026in}{2.816062in}}%
\pgfpathlineto{\pgfqpoint{6.057603in}{2.819627in}}%
\pgfpathlineto{\pgfqpoint{6.060180in}{2.832949in}}%
\pgfpathlineto{\pgfqpoint{6.062757in}{2.837827in}}%
\pgfpathlineto{\pgfqpoint{6.065335in}{2.861094in}}%
\pgfpathlineto{\pgfqpoint{6.073066in}{2.856215in}}%
\pgfpathlineto{\pgfqpoint{6.075644in}{2.858842in}}%
\pgfpathlineto{\pgfqpoint{6.078221in}{2.871038in}}%
\pgfpathlineto{\pgfqpoint{6.080798in}{2.891303in}}%
\pgfpathlineto{\pgfqpoint{6.083375in}{2.898058in}}%
\pgfpathlineto{\pgfqpoint{6.091107in}{2.896557in}}%
\pgfpathlineto{\pgfqpoint{6.098838in}{2.830697in}}%
\pgfpathlineto{\pgfqpoint{6.101416in}{2.824505in}}%
\pgfpathlineto{\pgfqpoint{6.109147in}{2.835388in}}%
\pgfpathlineto{\pgfqpoint{6.114302in}{2.848522in}}%
\pgfpathlineto{\pgfqpoint{6.116879in}{2.828446in}}%
\pgfpathlineto{\pgfqpoint{6.119456in}{2.828821in}}%
\pgfpathlineto{\pgfqpoint{6.129765in}{2.805554in}}%
\pgfpathlineto{\pgfqpoint{6.132342in}{2.823380in}}%
\pgfpathlineto{\pgfqpoint{6.134919in}{2.816812in}}%
\pgfpathlineto{\pgfqpoint{6.137497in}{2.829947in}}%
\pgfpathlineto{\pgfqpoint{6.145228in}{2.821878in}}%
\pgfpathlineto{\pgfqpoint{6.147806in}{2.863158in}}%
\pgfpathlineto{\pgfqpoint{6.150383in}{2.876480in}}%
\pgfpathlineto{\pgfqpoint{6.152960in}{2.900685in}}%
\pgfpathlineto{\pgfqpoint{6.155537in}{2.878169in}}%
\pgfpathlineto{\pgfqpoint{6.163269in}{2.837264in}}%
\pgfpathlineto{\pgfqpoint{6.168423in}{2.800676in}}%
\pgfpathlineto{\pgfqpoint{6.171000in}{2.798424in}}%
\pgfpathlineto{\pgfqpoint{6.173578in}{2.816625in}}%
\pgfpathlineto{\pgfqpoint{6.181309in}{2.832011in}}%
\pgfpathlineto{\pgfqpoint{6.183886in}{2.828446in}}%
\pgfpathlineto{\pgfqpoint{6.186464in}{2.822066in}}%
\pgfpathlineto{\pgfqpoint{6.189041in}{2.843269in}}%
\pgfpathlineto{\pgfqpoint{6.191618in}{2.840267in}}%
\pgfpathlineto{\pgfqpoint{6.199350in}{2.834638in}}%
\pgfpathlineto{\pgfqpoint{6.201927in}{2.824130in}}%
\pgfpathlineto{\pgfqpoint{6.204504in}{2.841205in}}%
\pgfpathlineto{\pgfqpoint{6.207081in}{2.838765in}}%
\pgfpathlineto{\pgfqpoint{6.209659in}{2.838953in}}%
\pgfpathlineto{\pgfqpoint{6.217390in}{2.845708in}}%
\pgfpathlineto{\pgfqpoint{6.219967in}{2.843832in}}%
\pgfpathlineto{\pgfqpoint{6.222545in}{2.857904in}}%
\pgfpathlineto{\pgfqpoint{6.225122in}{2.837827in}}%
\pgfpathlineto{\pgfqpoint{6.227699in}{2.830510in}}%
\pgfpathlineto{\pgfqpoint{6.235431in}{2.845145in}}%
\pgfpathlineto{\pgfqpoint{6.238008in}{2.867286in}}%
\pgfpathlineto{\pgfqpoint{6.240585in}{2.833512in}}%
\pgfpathlineto{\pgfqpoint{6.243162in}{2.835200in}}%
\pgfpathlineto{\pgfqpoint{6.245740in}{2.828446in}}%
\pgfpathlineto{\pgfqpoint{6.253471in}{2.829571in}}%
\pgfpathlineto{\pgfqpoint{6.256048in}{2.838015in}}%
\pgfpathlineto{\pgfqpoint{6.258626in}{2.817563in}}%
\pgfpathlineto{\pgfqpoint{6.261203in}{2.840829in}}%
\pgfpathlineto{\pgfqpoint{6.263780in}{2.816812in}}%
\pgfpathlineto{\pgfqpoint{6.274089in}{2.796548in}}%
\pgfpathlineto{\pgfqpoint{6.276666in}{2.810433in}}%
\pgfpathlineto{\pgfqpoint{6.279243in}{2.837640in}}%
\pgfpathlineto{\pgfqpoint{6.281821in}{2.815874in}}%
\pgfpathlineto{\pgfqpoint{6.289552in}{2.855840in}}%
\pgfpathlineto{\pgfqpoint{6.292129in}{2.845520in}}%
\pgfpathlineto{\pgfqpoint{6.294707in}{2.842331in}}%
\pgfpathlineto{\pgfqpoint{6.297284in}{2.873290in}}%
\pgfpathlineto{\pgfqpoint{6.299861in}{2.878169in}}%
\pgfpathlineto{\pgfqpoint{6.307593in}{2.894868in}}%
\pgfpathlineto{\pgfqpoint{6.310170in}{2.891866in}}%
\pgfpathlineto{\pgfqpoint{6.312747in}{2.856028in}}%
\pgfpathlineto{\pgfqpoint{6.315324in}{2.829759in}}%
\pgfpathlineto{\pgfqpoint{6.317902in}{2.823380in}}%
\pgfpathlineto{\pgfqpoint{6.325633in}{2.819439in}}%
\pgfpathlineto{\pgfqpoint{6.328210in}{2.815311in}}%
\pgfpathlineto{\pgfqpoint{6.330788in}{2.793921in}}%
\pgfpathlineto{\pgfqpoint{6.333365in}{2.788855in}}%
\pgfpathlineto{\pgfqpoint{6.335942in}{2.798612in}}%
\pgfpathlineto{\pgfqpoint{6.343674in}{2.820190in}}%
\pgfpathlineto{\pgfqpoint{6.348828in}{2.850211in}}%
\pgfpathlineto{\pgfqpoint{6.351405in}{2.855465in}}%
\pgfpathlineto{\pgfqpoint{6.353982in}{2.856028in}}%
\pgfpathlineto{\pgfqpoint{6.361714in}{2.860156in}}%
\pgfpathlineto{\pgfqpoint{6.364291in}{2.868224in}}%
\pgfpathlineto{\pgfqpoint{6.366869in}{2.917572in}}%
\pgfpathlineto{\pgfqpoint{6.369446in}{2.920761in}}%
\pgfpathlineto{\pgfqpoint{6.372023in}{2.913631in}}%
\pgfpathlineto{\pgfqpoint{6.379755in}{2.908002in}}%
\pgfpathlineto{\pgfqpoint{6.382332in}{2.991687in}}%
\pgfpathlineto{\pgfqpoint{6.384909in}{2.989811in}}%
\pgfpathlineto{\pgfqpoint{6.387486in}{3.014015in}}%
\pgfpathlineto{\pgfqpoint{6.390063in}{3.020582in}}%
\pgfpathlineto{\pgfqpoint{6.397795in}{3.042348in}}%
\pgfpathlineto{\pgfqpoint{6.400372in}{3.006885in}}%
\pgfpathlineto{\pgfqpoint{6.402950in}{3.019832in}}%
\pgfpathlineto{\pgfqpoint{6.405527in}{3.009887in}}%
\pgfpathlineto{\pgfqpoint{6.408104in}{3.009512in}}%
\pgfpathlineto{\pgfqpoint{6.418413in}{2.966919in}}%
\pgfpathlineto{\pgfqpoint{6.420990in}{2.977239in}}%
\pgfpathlineto{\pgfqpoint{6.423567in}{2.976301in}}%
\pgfpathlineto{\pgfqpoint{6.426144in}{2.978928in}}%
\pgfpathlineto{\pgfqpoint{6.433876in}{2.973299in}}%
\pgfpathlineto{\pgfqpoint{6.436453in}{2.973486in}}%
\pgfpathlineto{\pgfqpoint{6.439031in}{3.001256in}}%
\pgfpathlineto{\pgfqpoint{6.444185in}{2.969734in}}%
\pgfpathlineto{\pgfqpoint{6.451917in}{2.973299in}}%
\pgfpathlineto{\pgfqpoint{6.454494in}{2.968420in}}%
\pgfpathlineto{\pgfqpoint{6.457071in}{2.961478in}}%
\pgfpathlineto{\pgfqpoint{6.459648in}{2.957350in}}%
\pgfpathlineto{\pgfqpoint{6.462225in}{2.941776in}}%
\pgfpathlineto{\pgfqpoint{6.469957in}{2.940463in}}%
\pgfpathlineto{\pgfqpoint{6.472534in}{2.947781in}}%
\pgfpathlineto{\pgfqpoint{6.475111in}{2.929017in}}%
\pgfpathlineto{\pgfqpoint{6.480266in}{2.941964in}}%
\pgfpathlineto{\pgfqpoint{6.487998in}{2.958663in}}%
\pgfpathlineto{\pgfqpoint{6.490575in}{2.993188in}}%
\pgfpathlineto{\pgfqpoint{6.493152in}{2.989435in}}%
\pgfpathlineto{\pgfqpoint{6.495729in}{2.980804in}}%
\pgfpathlineto{\pgfqpoint{6.498306in}{2.992625in}}%
\pgfpathlineto{\pgfqpoint{6.506038in}{2.974987in}}%
\pgfpathlineto{\pgfqpoint{6.508615in}{2.986996in}}%
\pgfpathlineto{\pgfqpoint{6.511192in}{3.011951in}}%
\pgfpathlineto{\pgfqpoint{6.513770in}{2.993000in}}%
\pgfpathlineto{\pgfqpoint{6.516347in}{3.003508in}}%
\pgfpathlineto{\pgfqpoint{6.524079in}{3.013452in}}%
\pgfpathlineto{\pgfqpoint{6.526656in}{3.039721in}}%
\pgfpathlineto{\pgfqpoint{6.529233in}{3.044975in}}%
\pgfpathlineto{\pgfqpoint{6.531810in}{3.022646in}}%
\pgfpathlineto{\pgfqpoint{6.534387in}{3.037282in}}%
\pgfpathlineto{\pgfqpoint{6.542119in}{3.025273in}}%
\pgfpathlineto{\pgfqpoint{6.544696in}{3.024898in}}%
\pgfpathlineto{\pgfqpoint{6.547273in}{3.013828in}}%
\pgfpathlineto{\pgfqpoint{6.549851in}{3.011951in}}%
\pgfpathlineto{\pgfqpoint{6.552428in}{2.995064in}}%
\pgfpathlineto{\pgfqpoint{6.562737in}{2.994501in}}%
\pgfpathlineto{\pgfqpoint{6.565314in}{3.003132in}}%
\pgfpathlineto{\pgfqpoint{6.567891in}{3.002945in}}%
\pgfpathlineto{\pgfqpoint{6.570468in}{2.987934in}}%
\pgfpathlineto{\pgfqpoint{6.570468in}{2.987934in}}%
\pgfusepath{stroke}%
\end{pgfscope}%
\begin{pgfscope}%
\pgfpathrectangle{\pgfqpoint{0.652287in}{0.521603in}}{\pgfqpoint{6.200000in}{4.530000in}}%
\pgfusepath{clip}%
\pgfsetroundcap%
\pgfsetroundjoin%
\pgfsetlinewidth{1.505625pt}%
\definecolor{currentstroke}{rgb}{0.549020,0.337255,0.294118}%
\pgfsetstrokecolor{currentstroke}%
\pgfsetdash{}{0pt}%
\pgfpathmoveto{\pgfqpoint{0.934105in}{1.456094in}}%
\pgfpathlineto{\pgfqpoint{0.936682in}{1.455718in}}%
\pgfpathlineto{\pgfqpoint{0.939259in}{1.451590in}}%
\pgfpathlineto{\pgfqpoint{0.941836in}{1.449151in}}%
\pgfpathlineto{\pgfqpoint{0.949568in}{1.453279in}}%
\pgfpathlineto{\pgfqpoint{0.952145in}{1.448776in}}%
\pgfpathlineto{\pgfqpoint{0.954722in}{1.439207in}}%
\pgfpathlineto{\pgfqpoint{0.957300in}{1.441083in}}%
\pgfpathlineto{\pgfqpoint{0.959877in}{1.441083in}}%
\pgfpathlineto{\pgfqpoint{0.970186in}{1.447838in}}%
\pgfpathlineto{\pgfqpoint{0.972763in}{1.452153in}}%
\pgfpathlineto{\pgfqpoint{0.975340in}{1.452904in}}%
\pgfpathlineto{\pgfqpoint{0.977917in}{1.455155in}}%
\pgfpathlineto{\pgfqpoint{0.988226in}{1.429450in}}%
\pgfpathlineto{\pgfqpoint{0.990803in}{1.436580in}}%
\pgfpathlineto{\pgfqpoint{0.993381in}{1.433765in}}%
\pgfpathlineto{\pgfqpoint{0.995958in}{1.426448in}}%
\pgfpathlineto{\pgfqpoint{1.003689in}{1.410123in}}%
\pgfpathlineto{\pgfqpoint{1.006267in}{1.407684in}}%
\pgfpathlineto{\pgfqpoint{1.011421in}{1.411812in}}%
\pgfpathlineto{\pgfqpoint{1.013998in}{1.403556in}}%
\pgfpathlineto{\pgfqpoint{1.024307in}{1.417441in}}%
\pgfpathlineto{\pgfqpoint{1.026884in}{1.416503in}}%
\pgfpathlineto{\pgfqpoint{1.029462in}{1.422507in}}%
\pgfpathlineto{\pgfqpoint{1.032039in}{1.420068in}}%
\pgfpathlineto{\pgfqpoint{1.039770in}{1.425322in}}%
\pgfpathlineto{\pgfqpoint{1.042348in}{1.429074in}}%
\pgfpathlineto{\pgfqpoint{1.044925in}{1.430200in}}%
\pgfpathlineto{\pgfqpoint{1.047502in}{1.439769in}}%
\pgfpathlineto{\pgfqpoint{1.050079in}{1.435454in}}%
\pgfpathlineto{\pgfqpoint{1.060388in}{1.428136in}}%
\pgfpathlineto{\pgfqpoint{1.062965in}{1.428511in}}%
\pgfpathlineto{\pgfqpoint{1.065543in}{1.457970in}}%
\pgfpathlineto{\pgfqpoint{1.068120in}{1.462286in}}%
\pgfpathlineto{\pgfqpoint{1.075851in}{1.462098in}}%
\pgfpathlineto{\pgfqpoint{1.078429in}{1.472418in}}%
\pgfpathlineto{\pgfqpoint{1.081006in}{1.475795in}}%
\pgfpathlineto{\pgfqpoint{1.083583in}{1.461535in}}%
\pgfpathlineto{\pgfqpoint{1.086160in}{1.461723in}}%
\pgfpathlineto{\pgfqpoint{1.093892in}{1.465851in}}%
\pgfpathlineto{\pgfqpoint{1.096469in}{1.464162in}}%
\pgfpathlineto{\pgfqpoint{1.099046in}{1.460409in}}%
\pgfpathlineto{\pgfqpoint{1.101624in}{1.465100in}}%
\pgfpathlineto{\pgfqpoint{1.104201in}{1.465475in}}%
\pgfpathlineto{\pgfqpoint{1.111932in}{1.477109in}}%
\pgfpathlineto{\pgfqpoint{1.114510in}{1.479923in}}%
\pgfpathlineto{\pgfqpoint{1.117087in}{1.479173in}}%
\pgfpathlineto{\pgfqpoint{1.119664in}{1.476733in}}%
\pgfpathlineto{\pgfqpoint{1.122241in}{1.470354in}}%
\pgfpathlineto{\pgfqpoint{1.135127in}{1.469603in}}%
\pgfpathlineto{\pgfqpoint{1.137704in}{1.474294in}}%
\pgfpathlineto{\pgfqpoint{1.140282in}{1.472981in}}%
\pgfpathlineto{\pgfqpoint{1.148013in}{1.473356in}}%
\pgfpathlineto{\pgfqpoint{1.150591in}{1.468853in}}%
\pgfpathlineto{\pgfqpoint{1.153168in}{1.469416in}}%
\pgfpathlineto{\pgfqpoint{1.155745in}{1.466789in}}%
\pgfpathlineto{\pgfqpoint{1.158322in}{1.469791in}}%
\pgfpathlineto{\pgfqpoint{1.166054in}{1.474857in}}%
\pgfpathlineto{\pgfqpoint{1.168631in}{1.467915in}}%
\pgfpathlineto{\pgfqpoint{1.171208in}{1.470541in}}%
\pgfpathlineto{\pgfqpoint{1.173785in}{1.471104in}}%
\pgfpathlineto{\pgfqpoint{1.184094in}{1.463787in}}%
\pgfpathlineto{\pgfqpoint{1.186672in}{1.456844in}}%
\pgfpathlineto{\pgfqpoint{1.189249in}{1.458158in}}%
\pgfpathlineto{\pgfqpoint{1.194403in}{1.448964in}}%
\pgfpathlineto{\pgfqpoint{1.202135in}{1.463224in}}%
\pgfpathlineto{\pgfqpoint{1.204712in}{1.466789in}}%
\pgfpathlineto{\pgfqpoint{1.207289in}{1.462848in}}%
\pgfpathlineto{\pgfqpoint{1.209866in}{1.460222in}}%
\pgfpathlineto{\pgfqpoint{1.212444in}{1.474106in}}%
\pgfpathlineto{\pgfqpoint{1.220175in}{1.461347in}}%
\pgfpathlineto{\pgfqpoint{1.222753in}{1.466601in}}%
\pgfpathlineto{\pgfqpoint{1.225330in}{1.473356in}}%
\pgfpathlineto{\pgfqpoint{1.227907in}{1.472981in}}%
\pgfpathlineto{\pgfqpoint{1.230484in}{1.436580in}}%
\pgfpathlineto{\pgfqpoint{1.238216in}{1.424571in}}%
\pgfpathlineto{\pgfqpoint{1.240793in}{1.423633in}}%
\pgfpathlineto{\pgfqpoint{1.243370in}{1.429637in}}%
\pgfpathlineto{\pgfqpoint{1.245947in}{1.437706in}}%
\pgfpathlineto{\pgfqpoint{1.248525in}{1.434140in}}%
\pgfpathlineto{\pgfqpoint{1.256256in}{1.433765in}}%
\pgfpathlineto{\pgfqpoint{1.258833in}{1.432452in}}%
\pgfpathlineto{\pgfqpoint{1.261411in}{1.425134in}}%
\pgfpathlineto{\pgfqpoint{1.263988in}{1.432077in}}%
\pgfpathlineto{\pgfqpoint{1.266565in}{1.425134in}}%
\pgfpathlineto{\pgfqpoint{1.274297in}{1.423633in}}%
\pgfpathlineto{\pgfqpoint{1.276874in}{1.425885in}}%
\pgfpathlineto{\pgfqpoint{1.279451in}{1.434328in}}%
\pgfpathlineto{\pgfqpoint{1.282028in}{1.429450in}}%
\pgfpathlineto{\pgfqpoint{1.284606in}{1.422882in}}%
\pgfpathlineto{\pgfqpoint{1.292337in}{1.420819in}}%
\pgfpathlineto{\pgfqpoint{1.294914in}{1.417253in}}%
\pgfpathlineto{\pgfqpoint{1.297492in}{1.405808in}}%
\pgfpathlineto{\pgfqpoint{1.300069in}{1.408622in}}%
\pgfpathlineto{\pgfqpoint{1.302646in}{1.407309in}}%
\pgfpathlineto{\pgfqpoint{1.312955in}{1.414251in}}%
\pgfpathlineto{\pgfqpoint{1.315532in}{1.404870in}}%
\pgfpathlineto{\pgfqpoint{1.318109in}{1.404307in}}%
\pgfpathlineto{\pgfqpoint{1.320687in}{1.393236in}}%
\pgfpathlineto{\pgfqpoint{1.328418in}{1.390985in}}%
\pgfpathlineto{\pgfqpoint{1.330995in}{1.387607in}}%
\pgfpathlineto{\pgfqpoint{1.333573in}{1.396989in}}%
\pgfpathlineto{\pgfqpoint{1.336150in}{1.411437in}}%
\pgfpathlineto{\pgfqpoint{1.338727in}{1.411249in}}%
\pgfpathlineto{\pgfqpoint{1.346459in}{1.408059in}}%
\pgfpathlineto{\pgfqpoint{1.349036in}{1.411437in}}%
\pgfpathlineto{\pgfqpoint{1.351613in}{1.408622in}}%
\pgfpathlineto{\pgfqpoint{1.354190in}{1.417629in}}%
\pgfpathlineto{\pgfqpoint{1.356768in}{1.413126in}}%
\pgfpathlineto{\pgfqpoint{1.364499in}{1.404307in}}%
\pgfpathlineto{\pgfqpoint{1.367076in}{1.403181in}}%
\pgfpathlineto{\pgfqpoint{1.369654in}{1.375786in}}%
\pgfpathlineto{\pgfqpoint{1.372231in}{1.366217in}}%
\pgfpathlineto{\pgfqpoint{1.374808in}{1.367343in}}%
\pgfpathlineto{\pgfqpoint{1.382540in}{1.359650in}}%
\pgfpathlineto{\pgfqpoint{1.385117in}{1.359087in}}%
\pgfpathlineto{\pgfqpoint{1.387694in}{1.369595in}}%
\pgfpathlineto{\pgfqpoint{1.390271in}{1.374285in}}%
\pgfpathlineto{\pgfqpoint{1.392849in}{1.388733in}}%
\pgfpathlineto{\pgfqpoint{1.400580in}{1.387795in}}%
\pgfpathlineto{\pgfqpoint{1.403157in}{1.390422in}}%
\pgfpathlineto{\pgfqpoint{1.408312in}{1.390234in}}%
\pgfpathlineto{\pgfqpoint{1.410889in}{1.389108in}}%
\pgfpathlineto{\pgfqpoint{1.418621in}{1.393236in}}%
\pgfpathlineto{\pgfqpoint{1.421198in}{1.395863in}}%
\pgfpathlineto{\pgfqpoint{1.423775in}{1.390985in}}%
\pgfpathlineto{\pgfqpoint{1.426352in}{1.425509in}}%
\pgfpathlineto{\pgfqpoint{1.428930in}{1.446337in}}%
\pgfpathlineto{\pgfqpoint{1.436661in}{1.442209in}}%
\pgfpathlineto{\pgfqpoint{1.439238in}{1.450277in}}%
\pgfpathlineto{\pgfqpoint{1.441816in}{1.450652in}}%
\pgfpathlineto{\pgfqpoint{1.444393in}{1.452341in}}%
\pgfpathlineto{\pgfqpoint{1.446970in}{1.449339in}}%
\pgfpathlineto{\pgfqpoint{1.454702in}{1.444273in}}%
\pgfpathlineto{\pgfqpoint{1.457279in}{1.438644in}}%
\pgfpathlineto{\pgfqpoint{1.459856in}{1.438644in}}%
\pgfpathlineto{\pgfqpoint{1.465010in}{1.454780in}}%
\pgfpathlineto{\pgfqpoint{1.472742in}{1.454968in}}%
\pgfpathlineto{\pgfqpoint{1.480474in}{1.430951in}}%
\pgfpathlineto{\pgfqpoint{1.483051in}{1.460972in}}%
\pgfpathlineto{\pgfqpoint{1.490783in}{1.465663in}}%
\pgfpathlineto{\pgfqpoint{1.493360in}{1.471855in}}%
\pgfpathlineto{\pgfqpoint{1.495937in}{1.479548in}}%
\pgfpathlineto{\pgfqpoint{1.501091in}{1.480298in}}%
\pgfpathlineto{\pgfqpoint{1.508823in}{1.475795in}}%
\pgfpathlineto{\pgfqpoint{1.511400in}{1.479735in}}%
\pgfpathlineto{\pgfqpoint{1.513978in}{1.478234in}}%
\pgfpathlineto{\pgfqpoint{1.516555in}{1.483676in}}%
\pgfpathlineto{\pgfqpoint{1.519132in}{1.483676in}}%
\pgfpathlineto{\pgfqpoint{1.526864in}{1.479923in}}%
\pgfpathlineto{\pgfqpoint{1.529441in}{1.480298in}}%
\pgfpathlineto{\pgfqpoint{1.532018in}{1.481424in}}%
\pgfpathlineto{\pgfqpoint{1.534595in}{1.478985in}}%
\pgfpathlineto{\pgfqpoint{1.537172in}{1.484051in}}%
\pgfpathlineto{\pgfqpoint{1.544904in}{1.485364in}}%
\pgfpathlineto{\pgfqpoint{1.550058in}{1.481799in}}%
\pgfpathlineto{\pgfqpoint{1.552636in}{1.481987in}}%
\pgfpathlineto{\pgfqpoint{1.555213in}{1.486678in}}%
\pgfpathlineto{\pgfqpoint{1.565522in}{1.489868in}}%
\pgfpathlineto{\pgfqpoint{1.568099in}{1.488179in}}%
\pgfpathlineto{\pgfqpoint{1.570676in}{1.502439in}}%
\pgfpathlineto{\pgfqpoint{1.573253in}{1.506755in}}%
\pgfpathlineto{\pgfqpoint{1.580985in}{1.506567in}}%
\pgfpathlineto{\pgfqpoint{1.583562in}{1.502814in}}%
\pgfpathlineto{\pgfqpoint{1.586139in}{1.500375in}}%
\pgfpathlineto{\pgfqpoint{1.588717in}{1.512571in}}%
\pgfpathlineto{\pgfqpoint{1.591294in}{1.516512in}}%
\pgfpathlineto{\pgfqpoint{1.599026in}{1.517825in}}%
\pgfpathlineto{\pgfqpoint{1.601603in}{1.517262in}}%
\pgfpathlineto{\pgfqpoint{1.604180in}{1.518013in}}%
\pgfpathlineto{\pgfqpoint{1.606757in}{1.522516in}}%
\pgfpathlineto{\pgfqpoint{1.609334in}{1.520452in}}%
\pgfpathlineto{\pgfqpoint{1.617066in}{1.525518in}}%
\pgfpathlineto{\pgfqpoint{1.619643in}{1.522891in}}%
\pgfpathlineto{\pgfqpoint{1.622220in}{1.518576in}}%
\pgfpathlineto{\pgfqpoint{1.624798in}{1.518576in}}%
\pgfpathlineto{\pgfqpoint{1.627375in}{1.519514in}}%
\pgfpathlineto{\pgfqpoint{1.635106in}{1.520827in}}%
\pgfpathlineto{\pgfqpoint{1.637684in}{1.510883in}}%
\pgfpathlineto{\pgfqpoint{1.640261in}{1.516512in}}%
\pgfpathlineto{\pgfqpoint{1.645415in}{1.523642in}}%
\pgfpathlineto{\pgfqpoint{1.653147in}{1.515573in}}%
\pgfpathlineto{\pgfqpoint{1.655724in}{1.509382in}}%
\pgfpathlineto{\pgfqpoint{1.658301in}{1.500938in}}%
\pgfpathlineto{\pgfqpoint{1.660879in}{1.498874in}}%
\pgfpathlineto{\pgfqpoint{1.663456in}{1.497936in}}%
\pgfpathlineto{\pgfqpoint{1.673765in}{1.514072in}}%
\pgfpathlineto{\pgfqpoint{1.676342in}{1.529834in}}%
\pgfpathlineto{\pgfqpoint{1.678919in}{1.529834in}}%
\pgfpathlineto{\pgfqpoint{1.681496in}{1.515949in}}%
\pgfpathlineto{\pgfqpoint{1.689228in}{1.515011in}}%
\pgfpathlineto{\pgfqpoint{1.691805in}{1.498874in}}%
\pgfpathlineto{\pgfqpoint{1.694382in}{1.508631in}}%
\pgfpathlineto{\pgfqpoint{1.696960in}{1.538840in}}%
\pgfpathlineto{\pgfqpoint{1.699537in}{1.529271in}}%
\pgfpathlineto{\pgfqpoint{1.712423in}{1.526269in}}%
\pgfpathlineto{\pgfqpoint{1.715000in}{1.526456in}}%
\pgfpathlineto{\pgfqpoint{1.717577in}{1.525518in}}%
\pgfpathlineto{\pgfqpoint{1.725309in}{1.517637in}}%
\pgfpathlineto{\pgfqpoint{1.727886in}{1.521578in}}%
\pgfpathlineto{\pgfqpoint{1.730463in}{1.508256in}}%
\pgfpathlineto{\pgfqpoint{1.733041in}{1.490618in}}%
\pgfpathlineto{\pgfqpoint{1.735618in}{1.492307in}}%
\pgfpathlineto{\pgfqpoint{1.743349in}{1.493245in}}%
\pgfpathlineto{\pgfqpoint{1.745927in}{1.489868in}}%
\pgfpathlineto{\pgfqpoint{1.748504in}{1.484802in}}%
\pgfpathlineto{\pgfqpoint{1.751081in}{1.481612in}}%
\pgfpathlineto{\pgfqpoint{1.753658in}{1.489305in}}%
\pgfpathlineto{\pgfqpoint{1.763967in}{1.511821in}}%
\pgfpathlineto{\pgfqpoint{1.766544in}{1.514448in}}%
\pgfpathlineto{\pgfqpoint{1.771699in}{1.531522in}}%
\pgfpathlineto{\pgfqpoint{1.779430in}{1.529834in}}%
\pgfpathlineto{\pgfqpoint{1.782008in}{1.522516in}}%
\pgfpathlineto{\pgfqpoint{1.784585in}{1.529271in}}%
\pgfpathlineto{\pgfqpoint{1.787162in}{1.530209in}}%
\pgfpathlineto{\pgfqpoint{1.789739in}{1.535275in}}%
\pgfpathlineto{\pgfqpoint{1.797471in}{1.531522in}}%
\pgfpathlineto{\pgfqpoint{1.800048in}{1.527394in}}%
\pgfpathlineto{\pgfqpoint{1.802625in}{1.528895in}}%
\pgfpathlineto{\pgfqpoint{1.805203in}{1.537151in}}%
\pgfpathlineto{\pgfqpoint{1.807780in}{1.542217in}}%
\pgfpathlineto{\pgfqpoint{1.815511in}{1.541279in}}%
\pgfpathlineto{\pgfqpoint{1.818089in}{1.547846in}}%
\pgfpathlineto{\pgfqpoint{1.820666in}{1.549535in}}%
\pgfpathlineto{\pgfqpoint{1.823243in}{1.539778in}}%
\pgfpathlineto{\pgfqpoint{1.825820in}{1.536776in}}%
\pgfpathlineto{\pgfqpoint{1.833552in}{1.536776in}}%
\pgfpathlineto{\pgfqpoint{1.836129in}{1.537339in}}%
\pgfpathlineto{\pgfqpoint{1.838706in}{1.527770in}}%
\pgfpathlineto{\pgfqpoint{1.841283in}{1.535087in}}%
\pgfpathlineto{\pgfqpoint{1.843861in}{1.518388in}}%
\pgfpathlineto{\pgfqpoint{1.851592in}{1.515198in}}%
\pgfpathlineto{\pgfqpoint{1.856747in}{1.507318in}}%
\pgfpathlineto{\pgfqpoint{1.859324in}{1.506942in}}%
\pgfpathlineto{\pgfqpoint{1.861901in}{1.494371in}}%
\pgfpathlineto{\pgfqpoint{1.869633in}{1.505629in}}%
\pgfpathlineto{\pgfqpoint{1.874787in}{1.528520in}}%
\pgfpathlineto{\pgfqpoint{1.877364in}{1.521765in}}%
\pgfpathlineto{\pgfqpoint{1.879942in}{1.524017in}}%
\pgfpathlineto{\pgfqpoint{1.887673in}{1.516699in}}%
\pgfpathlineto{\pgfqpoint{1.890251in}{1.515198in}}%
\pgfpathlineto{\pgfqpoint{1.895405in}{1.526644in}}%
\pgfpathlineto{\pgfqpoint{1.897982in}{1.525893in}}%
\pgfpathlineto{\pgfqpoint{1.905714in}{1.532273in}}%
\pgfpathlineto{\pgfqpoint{1.908291in}{1.536026in}}%
\pgfpathlineto{\pgfqpoint{1.910868in}{1.536401in}}%
\pgfpathlineto{\pgfqpoint{1.916023in}{1.545595in}}%
\pgfpathlineto{\pgfqpoint{1.926332in}{1.545782in}}%
\pgfpathlineto{\pgfqpoint{1.928909in}{1.557040in}}%
\pgfpathlineto{\pgfqpoint{1.931486in}{1.552913in}}%
\pgfpathlineto{\pgfqpoint{1.934063in}{1.596444in}}%
\pgfpathlineto{\pgfqpoint{1.941795in}{1.604512in}}%
\pgfpathlineto{\pgfqpoint{1.944372in}{1.623463in}}%
\pgfpathlineto{\pgfqpoint{1.949526in}{1.625902in}}%
\pgfpathlineto{\pgfqpoint{1.952104in}{1.637535in}}%
\pgfpathlineto{\pgfqpoint{1.959835in}{1.627215in}}%
\pgfpathlineto{\pgfqpoint{1.964990in}{1.641100in}}%
\pgfpathlineto{\pgfqpoint{1.967567in}{1.641100in}}%
\pgfpathlineto{\pgfqpoint{1.970144in}{1.634908in}}%
\pgfpathlineto{\pgfqpoint{1.977876in}{1.635847in}}%
\pgfpathlineto{\pgfqpoint{1.980453in}{1.638473in}}%
\pgfpathlineto{\pgfqpoint{1.983030in}{1.647480in}}%
\pgfpathlineto{\pgfqpoint{1.985607in}{1.650857in}}%
\pgfpathlineto{\pgfqpoint{1.988185in}{1.647105in}}%
\pgfpathlineto{\pgfqpoint{1.998493in}{1.660051in}}%
\pgfpathlineto{\pgfqpoint{2.001071in}{1.655360in}}%
\pgfpathlineto{\pgfqpoint{2.006225in}{1.654047in}}%
\pgfpathlineto{\pgfqpoint{2.013957in}{1.637535in}}%
\pgfpathlineto{\pgfqpoint{2.016534in}{1.639975in}}%
\pgfpathlineto{\pgfqpoint{2.019111in}{1.650294in}}%
\pgfpathlineto{\pgfqpoint{2.021688in}{1.641663in}}%
\pgfpathlineto{\pgfqpoint{2.024266in}{1.646354in}}%
\pgfpathlineto{\pgfqpoint{2.031997in}{1.649356in}}%
\pgfpathlineto{\pgfqpoint{2.034574in}{1.654985in}}%
\pgfpathlineto{\pgfqpoint{2.037152in}{1.657237in}}%
\pgfpathlineto{\pgfqpoint{2.039729in}{1.652734in}}%
\pgfpathlineto{\pgfqpoint{2.042306in}{1.657049in}}%
\pgfpathlineto{\pgfqpoint{2.050038in}{1.659676in}}%
\pgfpathlineto{\pgfqpoint{2.052615in}{1.656862in}}%
\pgfpathlineto{\pgfqpoint{2.055192in}{1.651045in}}%
\pgfpathlineto{\pgfqpoint{2.057769in}{1.660239in}}%
\pgfpathlineto{\pgfqpoint{2.060347in}{1.644102in}}%
\pgfpathlineto{\pgfqpoint{2.068078in}{1.641288in}}%
\pgfpathlineto{\pgfqpoint{2.070655in}{1.655923in}}%
\pgfpathlineto{\pgfqpoint{2.073233in}{1.663053in}}%
\pgfpathlineto{\pgfqpoint{2.075810in}{1.657424in}}%
\pgfpathlineto{\pgfqpoint{2.078387in}{1.658363in}}%
\pgfpathlineto{\pgfqpoint{2.086119in}{1.649356in}}%
\pgfpathlineto{\pgfqpoint{2.088696in}{1.660427in}}%
\pgfpathlineto{\pgfqpoint{2.091273in}{1.655173in}}%
\pgfpathlineto{\pgfqpoint{2.093850in}{1.655173in}}%
\pgfpathlineto{\pgfqpoint{2.104159in}{1.664930in}}%
\pgfpathlineto{\pgfqpoint{2.106736in}{1.684444in}}%
\pgfpathlineto{\pgfqpoint{2.109314in}{1.671497in}}%
\pgfpathlineto{\pgfqpoint{2.111891in}{1.677877in}}%
\pgfpathlineto{\pgfqpoint{2.114468in}{1.673186in}}%
\pgfpathlineto{\pgfqpoint{2.122200in}{1.681817in}}%
\pgfpathlineto{\pgfqpoint{2.124777in}{1.673561in}}%
\pgfpathlineto{\pgfqpoint{2.127354in}{1.688759in}}%
\pgfpathlineto{\pgfqpoint{2.132509in}{1.701518in}}%
\pgfpathlineto{\pgfqpoint{2.140240in}{1.694951in}}%
\pgfpathlineto{\pgfqpoint{2.142817in}{1.701894in}}%
\pgfpathlineto{\pgfqpoint{2.145395in}{1.685945in}}%
\pgfpathlineto{\pgfqpoint{2.147972in}{1.698329in}}%
\pgfpathlineto{\pgfqpoint{2.150549in}{1.722346in}}%
\pgfpathlineto{\pgfqpoint{2.158281in}{1.721970in}}%
\pgfpathlineto{\pgfqpoint{2.160858in}{1.739420in}}%
\pgfpathlineto{\pgfqpoint{2.163435in}{1.664742in}}%
\pgfpathlineto{\pgfqpoint{2.166012in}{1.656486in}}%
\pgfpathlineto{\pgfqpoint{2.168589in}{1.664555in}}%
\pgfpathlineto{\pgfqpoint{2.176321in}{1.673373in}}%
\pgfpathlineto{\pgfqpoint{2.178898in}{1.659301in}}%
\pgfpathlineto{\pgfqpoint{2.181476in}{1.662678in}}%
\pgfpathlineto{\pgfqpoint{2.184053in}{1.674687in}}%
\pgfpathlineto{\pgfqpoint{2.186630in}{1.681442in}}%
\pgfpathlineto{\pgfqpoint{2.194362in}{1.674499in}}%
\pgfpathlineto{\pgfqpoint{2.196939in}{1.677501in}}%
\pgfpathlineto{\pgfqpoint{2.199516in}{1.685194in}}%
\pgfpathlineto{\pgfqpoint{2.202093in}{1.681629in}}%
\pgfpathlineto{\pgfqpoint{2.204670in}{1.690260in}}%
\pgfpathlineto{\pgfqpoint{2.212402in}{1.687633in}}%
\pgfpathlineto{\pgfqpoint{2.214979in}{1.701143in}}%
\pgfpathlineto{\pgfqpoint{2.217557in}{1.719906in}}%
\pgfpathlineto{\pgfqpoint{2.220134in}{1.712589in}}%
\pgfpathlineto{\pgfqpoint{2.222711in}{1.709774in}}%
\pgfpathlineto{\pgfqpoint{2.233020in}{1.690823in}}%
\pgfpathlineto{\pgfqpoint{2.235597in}{1.691198in}}%
\pgfpathlineto{\pgfqpoint{2.238174in}{1.689322in}}%
\pgfpathlineto{\pgfqpoint{2.240751in}{1.738482in}}%
\pgfpathlineto{\pgfqpoint{2.251060in}{1.722721in}}%
\pgfpathlineto{\pgfqpoint{2.253637in}{1.692324in}}%
\pgfpathlineto{\pgfqpoint{2.256215in}{1.695326in}}%
\pgfpathlineto{\pgfqpoint{2.258792in}{1.659113in}}%
\pgfpathlineto{\pgfqpoint{2.266524in}{1.673186in}}%
\pgfpathlineto{\pgfqpoint{2.269101in}{1.668682in}}%
\pgfpathlineto{\pgfqpoint{2.271678in}{1.657612in}}%
\pgfpathlineto{\pgfqpoint{2.274255in}{1.660051in}}%
\pgfpathlineto{\pgfqpoint{2.276832in}{1.674499in}}%
\pgfpathlineto{\pgfqpoint{2.284564in}{1.679565in}}%
\pgfpathlineto{\pgfqpoint{2.287141in}{1.680316in}}%
\pgfpathlineto{\pgfqpoint{2.289718in}{1.672248in}}%
\pgfpathlineto{\pgfqpoint{2.292296in}{1.685007in}}%
\pgfpathlineto{\pgfqpoint{2.294873in}{1.678815in}}%
\pgfpathlineto{\pgfqpoint{2.302605in}{1.693262in}}%
\pgfpathlineto{\pgfqpoint{2.305182in}{1.694576in}}%
\pgfpathlineto{\pgfqpoint{2.307759in}{1.672435in}}%
\pgfpathlineto{\pgfqpoint{2.310336in}{1.635847in}}%
\pgfpathlineto{\pgfqpoint{2.312913in}{1.669621in}}%
\pgfpathlineto{\pgfqpoint{2.320645in}{1.656486in}}%
\pgfpathlineto{\pgfqpoint{2.323222in}{1.658175in}}%
\pgfpathlineto{\pgfqpoint{2.325799in}{1.669058in}}%
\pgfpathlineto{\pgfqpoint{2.328377in}{1.673373in}}%
\pgfpathlineto{\pgfqpoint{2.330954in}{1.662678in}}%
\pgfpathlineto{\pgfqpoint{2.341263in}{1.685194in}}%
\pgfpathlineto{\pgfqpoint{2.343840in}{1.687258in}}%
\pgfpathlineto{\pgfqpoint{2.348994in}{1.683693in}}%
\pgfpathlineto{\pgfqpoint{2.356726in}{1.690260in}}%
\pgfpathlineto{\pgfqpoint{2.359303in}{1.702832in}}%
\pgfpathlineto{\pgfqpoint{2.361880in}{1.706397in}}%
\pgfpathlineto{\pgfqpoint{2.367035in}{1.733416in}}%
\pgfpathlineto{\pgfqpoint{2.374766in}{1.732853in}}%
\pgfpathlineto{\pgfqpoint{2.377344in}{1.724410in}}%
\pgfpathlineto{\pgfqpoint{2.379921in}{1.720094in}}%
\pgfpathlineto{\pgfqpoint{2.382498in}{1.723284in}}%
\pgfpathlineto{\pgfqpoint{2.385075in}{1.740171in}}%
\pgfpathlineto{\pgfqpoint{2.392807in}{1.737732in}}%
\pgfpathlineto{\pgfqpoint{2.395384in}{1.733604in}}%
\pgfpathlineto{\pgfqpoint{2.397961in}{1.722909in}}%
\pgfpathlineto{\pgfqpoint{2.400539in}{1.725160in}}%
\pgfpathlineto{\pgfqpoint{2.403116in}{1.724785in}}%
\pgfpathlineto{\pgfqpoint{2.410847in}{1.720094in}}%
\pgfpathlineto{\pgfqpoint{2.413425in}{1.725535in}}%
\pgfpathlineto{\pgfqpoint{2.416002in}{1.723284in}}%
\pgfpathlineto{\pgfqpoint{2.418579in}{1.744299in}}%
\pgfpathlineto{\pgfqpoint{2.421156in}{1.738857in}}%
\pgfpathlineto{\pgfqpoint{2.428888in}{1.740546in}}%
\pgfpathlineto{\pgfqpoint{2.431465in}{1.745800in}}%
\pgfpathlineto{\pgfqpoint{2.436620in}{1.752555in}}%
\pgfpathlineto{\pgfqpoint{2.439197in}{1.744299in}}%
\pgfpathlineto{\pgfqpoint{2.446928in}{1.743924in}}%
\pgfpathlineto{\pgfqpoint{2.449506in}{1.744674in}}%
\pgfpathlineto{\pgfqpoint{2.452083in}{1.738295in}}%
\pgfpathlineto{\pgfqpoint{2.454660in}{1.726098in}}%
\pgfpathlineto{\pgfqpoint{2.457237in}{1.717092in}}%
\pgfpathlineto{\pgfqpoint{2.464969in}{1.712213in}}%
\pgfpathlineto{\pgfqpoint{2.467546in}{1.711275in}}%
\pgfpathlineto{\pgfqpoint{2.470123in}{1.709024in}}%
\pgfpathlineto{\pgfqpoint{2.472701in}{1.715028in}}%
\pgfpathlineto{\pgfqpoint{2.475278in}{1.718781in}}%
\pgfpathlineto{\pgfqpoint{2.485587in}{1.687071in}}%
\pgfpathlineto{\pgfqpoint{2.488164in}{1.669433in}}%
\pgfpathlineto{\pgfqpoint{2.490741in}{1.676751in}}%
\pgfpathlineto{\pgfqpoint{2.493318in}{1.685757in}}%
\pgfpathlineto{\pgfqpoint{2.503627in}{1.683506in}}%
\pgfpathlineto{\pgfqpoint{2.506204in}{1.679565in}}%
\pgfpathlineto{\pgfqpoint{2.508782in}{1.673936in}}%
\pgfpathlineto{\pgfqpoint{2.511359in}{1.674124in}}%
\pgfpathlineto{\pgfqpoint{2.519090in}{1.689885in}}%
\pgfpathlineto{\pgfqpoint{2.521668in}{1.686695in}}%
\pgfpathlineto{\pgfqpoint{2.524245in}{1.691761in}}%
\pgfpathlineto{\pgfqpoint{2.526822in}{1.691574in}}%
\pgfpathlineto{\pgfqpoint{2.529399in}{1.703770in}}%
\pgfpathlineto{\pgfqpoint{2.537131in}{1.721220in}}%
\pgfpathlineto{\pgfqpoint{2.539708in}{1.715966in}}%
\pgfpathlineto{\pgfqpoint{2.542285in}{1.723284in}}%
\pgfpathlineto{\pgfqpoint{2.544862in}{1.720469in}}%
\pgfpathlineto{\pgfqpoint{2.547440in}{1.709211in}}%
\pgfpathlineto{\pgfqpoint{2.555171in}{1.707523in}}%
\pgfpathlineto{\pgfqpoint{2.557749in}{1.697203in}}%
\pgfpathlineto{\pgfqpoint{2.560326in}{1.683130in}}%
\pgfpathlineto{\pgfqpoint{2.562903in}{1.688196in}}%
\pgfpathlineto{\pgfqpoint{2.565480in}{1.675062in}}%
\pgfpathlineto{\pgfqpoint{2.573212in}{1.649919in}}%
\pgfpathlineto{\pgfqpoint{2.575789in}{1.658738in}}%
\pgfpathlineto{\pgfqpoint{2.578366in}{1.655173in}}%
\pgfpathlineto{\pgfqpoint{2.580943in}{1.653672in}}%
\pgfpathlineto{\pgfqpoint{2.583521in}{1.656486in}}%
\pgfpathlineto{\pgfqpoint{2.591252in}{1.650670in}}%
\pgfpathlineto{\pgfqpoint{2.596407in}{1.671122in}}%
\pgfpathlineto{\pgfqpoint{2.598984in}{1.685757in}}%
\pgfpathlineto{\pgfqpoint{2.601561in}{1.694951in}}%
\pgfpathlineto{\pgfqpoint{2.609293in}{1.699079in}}%
\pgfpathlineto{\pgfqpoint{2.611870in}{1.681254in}}%
\pgfpathlineto{\pgfqpoint{2.617024in}{1.719344in}}%
\pgfpathlineto{\pgfqpoint{2.619602in}{1.719156in}}%
\pgfpathlineto{\pgfqpoint{2.627333in}{1.712213in}}%
\pgfpathlineto{\pgfqpoint{2.629911in}{1.734354in}}%
\pgfpathlineto{\pgfqpoint{2.632488in}{1.742798in}}%
\pgfpathlineto{\pgfqpoint{2.635065in}{1.738107in}}%
\pgfpathlineto{\pgfqpoint{2.637642in}{1.728538in}}%
\pgfpathlineto{\pgfqpoint{2.645374in}{1.748990in}}%
\pgfpathlineto{\pgfqpoint{2.647951in}{1.767190in}}%
\pgfpathlineto{\pgfqpoint{2.653105in}{1.740171in}}%
\pgfpathlineto{\pgfqpoint{2.655683in}{1.746550in}}%
\pgfpathlineto{\pgfqpoint{2.665991in}{1.750866in}}%
\pgfpathlineto{\pgfqpoint{2.668569in}{1.772631in}}%
\pgfpathlineto{\pgfqpoint{2.671146in}{1.764938in}}%
\pgfpathlineto{\pgfqpoint{2.673723in}{1.767941in}}%
\pgfpathlineto{\pgfqpoint{2.681455in}{1.764000in}}%
\pgfpathlineto{\pgfqpoint{2.684032in}{1.772444in}}%
\pgfpathlineto{\pgfqpoint{2.691764in}{1.804529in}}%
\pgfpathlineto{\pgfqpoint{2.699495in}{1.800401in}}%
\pgfpathlineto{\pgfqpoint{2.702072in}{1.796649in}}%
\pgfpathlineto{\pgfqpoint{2.704650in}{1.802278in}}%
\pgfpathlineto{\pgfqpoint{2.707227in}{1.801902in}}%
\pgfpathlineto{\pgfqpoint{2.709804in}{1.806406in}}%
\pgfpathlineto{\pgfqpoint{2.717536in}{1.813536in}}%
\pgfpathlineto{\pgfqpoint{2.720113in}{1.801527in}}%
\pgfpathlineto{\pgfqpoint{2.722690in}{1.795710in}}%
\pgfpathlineto{\pgfqpoint{2.727845in}{1.794772in}}%
\pgfpathlineto{\pgfqpoint{2.735576in}{1.781075in}}%
\pgfpathlineto{\pgfqpoint{2.738153in}{1.788768in}}%
\pgfpathlineto{\pgfqpoint{2.740731in}{1.781075in}}%
\pgfpathlineto{\pgfqpoint{2.743308in}{1.770755in}}%
\pgfpathlineto{\pgfqpoint{2.745885in}{1.799651in}}%
\pgfpathlineto{\pgfqpoint{2.753617in}{1.803591in}}%
\pgfpathlineto{\pgfqpoint{2.756194in}{1.785953in}}%
\pgfpathlineto{\pgfqpoint{2.758771in}{1.791770in}}%
\pgfpathlineto{\pgfqpoint{2.761348in}{1.764563in}}%
\pgfpathlineto{\pgfqpoint{2.763926in}{1.765689in}}%
\pgfpathlineto{\pgfqpoint{2.771657in}{1.754994in}}%
\pgfpathlineto{\pgfqpoint{2.774234in}{1.742798in}}%
\pgfpathlineto{\pgfqpoint{2.776812in}{1.765877in}}%
\pgfpathlineto{\pgfqpoint{2.779389in}{1.758371in}}%
\pgfpathlineto{\pgfqpoint{2.781966in}{1.757433in}}%
\pgfpathlineto{\pgfqpoint{2.789698in}{1.748990in}}%
\pgfpathlineto{\pgfqpoint{2.792275in}{1.748990in}}%
\pgfpathlineto{\pgfqpoint{2.800007in}{1.760060in}}%
\pgfpathlineto{\pgfqpoint{2.807738in}{1.759872in}}%
\pgfpathlineto{\pgfqpoint{2.810315in}{1.750678in}}%
\pgfpathlineto{\pgfqpoint{2.815470in}{1.736981in}}%
\pgfpathlineto{\pgfqpoint{2.818047in}{1.735480in}}%
\pgfpathlineto{\pgfqpoint{2.825779in}{1.738482in}}%
\pgfpathlineto{\pgfqpoint{2.828356in}{1.750866in}}%
\pgfpathlineto{\pgfqpoint{2.830933in}{1.732290in}}%
\pgfpathlineto{\pgfqpoint{2.833510in}{1.735105in}}%
\pgfpathlineto{\pgfqpoint{2.836088in}{1.733228in}}%
\pgfpathlineto{\pgfqpoint{2.843819in}{1.728538in}}%
\pgfpathlineto{\pgfqpoint{2.846396in}{1.742047in}}%
\pgfpathlineto{\pgfqpoint{2.848974in}{1.740921in}}%
\pgfpathlineto{\pgfqpoint{2.851551in}{1.737169in}}%
\pgfpathlineto{\pgfqpoint{2.854128in}{1.726474in}}%
\pgfpathlineto{\pgfqpoint{2.864437in}{1.731352in}}%
\pgfpathlineto{\pgfqpoint{2.867014in}{1.725723in}}%
\pgfpathlineto{\pgfqpoint{2.869591in}{1.709962in}}%
\pgfpathlineto{\pgfqpoint{2.872168in}{1.724973in}}%
\pgfpathlineto{\pgfqpoint{2.879900in}{1.713715in}}%
\pgfpathlineto{\pgfqpoint{2.882477in}{1.723847in}}%
\pgfpathlineto{\pgfqpoint{2.885055in}{1.700580in}}%
\pgfpathlineto{\pgfqpoint{2.887632in}{1.688384in}}%
\pgfpathlineto{\pgfqpoint{2.890209in}{1.684444in}}%
\pgfpathlineto{\pgfqpoint{2.897941in}{1.669808in}}%
\pgfpathlineto{\pgfqpoint{2.908249in}{1.695326in}}%
\pgfpathlineto{\pgfqpoint{2.915981in}{1.706772in}}%
\pgfpathlineto{\pgfqpoint{2.918558in}{1.719531in}}%
\pgfpathlineto{\pgfqpoint{2.921136in}{1.698141in}}%
\pgfpathlineto{\pgfqpoint{2.923713in}{1.703019in}}%
\pgfpathlineto{\pgfqpoint{2.926290in}{1.728538in}}%
\pgfpathlineto{\pgfqpoint{2.936599in}{1.705834in}}%
\pgfpathlineto{\pgfqpoint{2.939176in}{1.708461in}}%
\pgfpathlineto{\pgfqpoint{2.941753in}{1.704896in}}%
\pgfpathlineto{\pgfqpoint{2.944330in}{1.705834in}}%
\pgfpathlineto{\pgfqpoint{2.952062in}{1.703958in}}%
\pgfpathlineto{\pgfqpoint{2.954639in}{1.708086in}}%
\pgfpathlineto{\pgfqpoint{2.957216in}{1.703958in}}%
\pgfpathlineto{\pgfqpoint{2.959794in}{1.709211in}}%
\pgfpathlineto{\pgfqpoint{2.962371in}{1.716717in}}%
\pgfpathlineto{\pgfqpoint{2.970103in}{1.697953in}}%
\pgfpathlineto{\pgfqpoint{2.972680in}{1.713339in}}%
\pgfpathlineto{\pgfqpoint{2.975257in}{1.703395in}}%
\pgfpathlineto{\pgfqpoint{2.977834in}{1.707147in}}%
\pgfpathlineto{\pgfqpoint{2.980411in}{1.712213in}}%
\pgfpathlineto{\pgfqpoint{2.988143in}{1.714653in}}%
\pgfpathlineto{\pgfqpoint{2.990720in}{1.721220in}}%
\pgfpathlineto{\pgfqpoint{2.993297in}{1.725911in}}%
\pgfpathlineto{\pgfqpoint{2.995875in}{1.725160in}}%
\pgfpathlineto{\pgfqpoint{2.998452in}{1.721783in}}%
\pgfpathlineto{\pgfqpoint{3.006184in}{1.735480in}}%
\pgfpathlineto{\pgfqpoint{3.008761in}{1.734354in}}%
\pgfpathlineto{\pgfqpoint{3.011338in}{1.718593in}}%
\pgfpathlineto{\pgfqpoint{3.016492in}{1.704333in}}%
\pgfpathlineto{\pgfqpoint{3.026801in}{1.734917in}}%
\pgfpathlineto{\pgfqpoint{3.029378in}{1.730039in}}%
\pgfpathlineto{\pgfqpoint{3.034533in}{1.734167in}}%
\pgfpathlineto{\pgfqpoint{3.042264in}{1.747489in}}%
\pgfpathlineto{\pgfqpoint{3.047419in}{1.739983in}}%
\pgfpathlineto{\pgfqpoint{3.049996in}{1.739608in}}%
\pgfpathlineto{\pgfqpoint{3.052573in}{1.734354in}}%
\pgfpathlineto{\pgfqpoint{3.060305in}{1.745800in}}%
\pgfpathlineto{\pgfqpoint{3.062882in}{1.759309in}}%
\pgfpathlineto{\pgfqpoint{3.065459in}{1.761561in}}%
\pgfpathlineto{\pgfqpoint{3.070614in}{1.749928in}}%
\pgfpathlineto{\pgfqpoint{3.080923in}{1.751241in}}%
\pgfpathlineto{\pgfqpoint{3.083500in}{1.764188in}}%
\pgfpathlineto{\pgfqpoint{3.086077in}{1.765877in}}%
\pgfpathlineto{\pgfqpoint{3.096386in}{1.762687in}}%
\pgfpathlineto{\pgfqpoint{3.101540in}{1.753868in}}%
\pgfpathlineto{\pgfqpoint{3.104118in}{1.766440in}}%
\pgfpathlineto{\pgfqpoint{3.106695in}{1.770567in}}%
\pgfpathlineto{\pgfqpoint{3.114426in}{1.795147in}}%
\pgfpathlineto{\pgfqpoint{3.117004in}{1.787079in}}%
\pgfpathlineto{\pgfqpoint{3.119581in}{1.788768in}}%
\pgfpathlineto{\pgfqpoint{3.122158in}{1.785578in}}%
\pgfpathlineto{\pgfqpoint{3.124735in}{1.778823in}}%
\pgfpathlineto{\pgfqpoint{3.132467in}{1.775258in}}%
\pgfpathlineto{\pgfqpoint{3.135044in}{1.766064in}}%
\pgfpathlineto{\pgfqpoint{3.137621in}{1.781450in}}%
\pgfpathlineto{\pgfqpoint{3.140199in}{1.782576in}}%
\pgfpathlineto{\pgfqpoint{3.142776in}{1.786329in}}%
\pgfpathlineto{\pgfqpoint{3.150507in}{1.775821in}}%
\pgfpathlineto{\pgfqpoint{3.153085in}{1.773757in}}%
\pgfpathlineto{\pgfqpoint{3.155662in}{1.766815in}}%
\pgfpathlineto{\pgfqpoint{3.158239in}{1.756495in}}%
\pgfpathlineto{\pgfqpoint{3.160816in}{1.753305in}}%
\pgfpathlineto{\pgfqpoint{3.168548in}{1.746926in}}%
\pgfpathlineto{\pgfqpoint{3.173702in}{1.755932in}}%
\pgfpathlineto{\pgfqpoint{3.176280in}{1.758559in}}%
\pgfpathlineto{\pgfqpoint{3.178857in}{1.756307in}}%
\pgfpathlineto{\pgfqpoint{3.189166in}{1.749365in}}%
\pgfpathlineto{\pgfqpoint{3.191743in}{1.749740in}}%
\pgfpathlineto{\pgfqpoint{3.194320in}{1.754431in}}%
\pgfpathlineto{\pgfqpoint{3.196897in}{1.760623in}}%
\pgfpathlineto{\pgfqpoint{3.204629in}{1.753868in}}%
\pgfpathlineto{\pgfqpoint{3.207206in}{1.746926in}}%
\pgfpathlineto{\pgfqpoint{3.209783in}{1.745800in}}%
\pgfpathlineto{\pgfqpoint{3.212361in}{1.749928in}}%
\pgfpathlineto{\pgfqpoint{3.214938in}{1.748614in}}%
\pgfpathlineto{\pgfqpoint{3.222669in}{1.749553in}}%
\pgfpathlineto{\pgfqpoint{3.225247in}{1.750491in}}%
\pgfpathlineto{\pgfqpoint{3.227824in}{1.748990in}}%
\pgfpathlineto{\pgfqpoint{3.230401in}{1.744299in}}%
\pgfpathlineto{\pgfqpoint{3.232978in}{1.742422in}}%
\pgfpathlineto{\pgfqpoint{3.240710in}{1.743173in}}%
\pgfpathlineto{\pgfqpoint{3.243287in}{1.741297in}}%
\pgfpathlineto{\pgfqpoint{3.245864in}{1.744674in}}%
\pgfpathlineto{\pgfqpoint{3.248441in}{1.751992in}}%
\pgfpathlineto{\pgfqpoint{3.251019in}{1.746926in}}%
\pgfpathlineto{\pgfqpoint{3.258750in}{1.740358in}}%
\pgfpathlineto{\pgfqpoint{3.261328in}{1.732290in}}%
\pgfpathlineto{\pgfqpoint{3.263905in}{1.737169in}}%
\pgfpathlineto{\pgfqpoint{3.266482in}{1.726098in}}%
\pgfpathlineto{\pgfqpoint{3.269059in}{1.732478in}}%
\pgfpathlineto{\pgfqpoint{3.276791in}{1.725535in}}%
\pgfpathlineto{\pgfqpoint{3.279368in}{1.736606in}}%
\pgfpathlineto{\pgfqpoint{3.281945in}{1.741109in}}%
\pgfpathlineto{\pgfqpoint{3.284522in}{1.747676in}}%
\pgfpathlineto{\pgfqpoint{3.294831in}{1.751054in}}%
\pgfpathlineto{\pgfqpoint{3.297409in}{1.757058in}}%
\pgfpathlineto{\pgfqpoint{3.299986in}{1.774695in}}%
\pgfpathlineto{\pgfqpoint{3.302563in}{1.773757in}}%
\pgfpathlineto{\pgfqpoint{3.305140in}{1.766627in}}%
\pgfpathlineto{\pgfqpoint{3.312872in}{1.769254in}}%
\pgfpathlineto{\pgfqpoint{3.315449in}{1.768316in}}%
\pgfpathlineto{\pgfqpoint{3.318026in}{1.773382in}}%
\pgfpathlineto{\pgfqpoint{3.320603in}{1.764751in}}%
\pgfpathlineto{\pgfqpoint{3.323181in}{1.767190in}}%
\pgfpathlineto{\pgfqpoint{3.330912in}{1.762875in}}%
\pgfpathlineto{\pgfqpoint{3.333490in}{1.759872in}}%
\pgfpathlineto{\pgfqpoint{3.336067in}{1.758184in}}%
\pgfpathlineto{\pgfqpoint{3.338644in}{1.762499in}}%
\pgfpathlineto{\pgfqpoint{3.341221in}{1.751241in}}%
\pgfpathlineto{\pgfqpoint{3.348953in}{1.746363in}}%
\pgfpathlineto{\pgfqpoint{3.356684in}{1.715216in}}%
\pgfpathlineto{\pgfqpoint{3.359262in}{1.752742in}}%
\pgfpathlineto{\pgfqpoint{3.366993in}{1.745800in}}%
\pgfpathlineto{\pgfqpoint{3.369570in}{1.748802in}}%
\pgfpathlineto{\pgfqpoint{3.372148in}{1.775821in}}%
\pgfpathlineto{\pgfqpoint{3.374725in}{1.760623in}}%
\pgfpathlineto{\pgfqpoint{3.377302in}{1.773570in}}%
\pgfpathlineto{\pgfqpoint{3.385034in}{1.782201in}}%
\pgfpathlineto{\pgfqpoint{3.387611in}{1.781263in}}%
\pgfpathlineto{\pgfqpoint{3.390188in}{1.782201in}}%
\pgfpathlineto{\pgfqpoint{3.392765in}{1.789706in}}%
\pgfpathlineto{\pgfqpoint{3.395343in}{1.786892in}}%
\pgfpathlineto{\pgfqpoint{3.405651in}{1.801715in}}%
\pgfpathlineto{\pgfqpoint{3.408229in}{1.803591in}}%
\pgfpathlineto{\pgfqpoint{3.410806in}{1.811096in}}%
\pgfpathlineto{\pgfqpoint{3.413383in}{1.812973in}}%
\pgfpathlineto{\pgfqpoint{3.421115in}{1.815412in}}%
\pgfpathlineto{\pgfqpoint{3.423692in}{1.812785in}}%
\pgfpathlineto{\pgfqpoint{3.426269in}{1.811659in}}%
\pgfpathlineto{\pgfqpoint{3.428846in}{1.807156in}}%
\pgfpathlineto{\pgfqpoint{3.431424in}{1.808469in}}%
\pgfpathlineto{\pgfqpoint{3.444310in}{1.805092in}}%
\pgfpathlineto{\pgfqpoint{3.446887in}{1.817851in}}%
\pgfpathlineto{\pgfqpoint{3.449464in}{1.818977in}}%
\pgfpathlineto{\pgfqpoint{3.457196in}{1.811847in}}%
\pgfpathlineto{\pgfqpoint{3.459773in}{1.806406in}}%
\pgfpathlineto{\pgfqpoint{3.462350in}{1.816913in}}%
\pgfpathlineto{\pgfqpoint{3.464927in}{1.814474in}}%
\pgfpathlineto{\pgfqpoint{3.467505in}{1.810721in}}%
\pgfpathlineto{\pgfqpoint{3.477813in}{1.824043in}}%
\pgfpathlineto{\pgfqpoint{3.480391in}{1.825357in}}%
\pgfpathlineto{\pgfqpoint{3.482968in}{1.825732in}}%
\pgfpathlineto{\pgfqpoint{3.485545in}{1.830235in}}%
\pgfpathlineto{\pgfqpoint{3.493277in}{1.835864in}}%
\pgfpathlineto{\pgfqpoint{3.495854in}{1.829860in}}%
\pgfpathlineto{\pgfqpoint{3.498431in}{1.842619in}}%
\pgfpathlineto{\pgfqpoint{3.501008in}{1.827983in}}%
\pgfpathlineto{\pgfqpoint{3.503586in}{1.832111in}}%
\pgfpathlineto{\pgfqpoint{3.511317in}{1.829860in}}%
\pgfpathlineto{\pgfqpoint{3.516472in}{1.808845in}}%
\pgfpathlineto{\pgfqpoint{3.519049in}{1.807344in}}%
\pgfpathlineto{\pgfqpoint{3.521626in}{1.819352in}}%
\pgfpathlineto{\pgfqpoint{3.529358in}{1.815787in}}%
\pgfpathlineto{\pgfqpoint{3.531935in}{1.809220in}}%
\pgfpathlineto{\pgfqpoint{3.534512in}{1.825544in}}%
\pgfpathlineto{\pgfqpoint{3.537089in}{1.817288in}}%
\pgfpathlineto{\pgfqpoint{3.539667in}{1.833800in}}%
\pgfpathlineto{\pgfqpoint{3.547398in}{1.812597in}}%
\pgfpathlineto{\pgfqpoint{3.549975in}{1.815412in}}%
\pgfpathlineto{\pgfqpoint{3.555130in}{1.794397in}}%
\pgfpathlineto{\pgfqpoint{3.557707in}{1.810909in}}%
\pgfpathlineto{\pgfqpoint{3.565439in}{1.825544in}}%
\pgfpathlineto{\pgfqpoint{3.568016in}{1.832487in}}%
\pgfpathlineto{\pgfqpoint{3.570593in}{1.836802in}}%
\pgfpathlineto{\pgfqpoint{3.573170in}{1.820666in}}%
\pgfpathlineto{\pgfqpoint{3.575747in}{1.852000in}}%
\pgfpathlineto{\pgfqpoint{3.583479in}{1.864760in}}%
\pgfpathlineto{\pgfqpoint{3.586056in}{1.873203in}}%
\pgfpathlineto{\pgfqpoint{3.588634in}{1.874329in}}%
\pgfpathlineto{\pgfqpoint{3.593788in}{1.886150in}}%
\pgfpathlineto{\pgfqpoint{3.601520in}{1.888026in}}%
\pgfpathlineto{\pgfqpoint{3.604097in}{1.908478in}}%
\pgfpathlineto{\pgfqpoint{3.606674in}{1.914295in}}%
\pgfpathlineto{\pgfqpoint{3.609251in}{1.912606in}}%
\pgfpathlineto{\pgfqpoint{3.611828in}{1.916359in}}%
\pgfpathlineto{\pgfqpoint{3.619560in}{1.921613in}}%
\pgfpathlineto{\pgfqpoint{3.622137in}{1.925178in}}%
\pgfpathlineto{\pgfqpoint{3.624715in}{1.921988in}}%
\pgfpathlineto{\pgfqpoint{3.627292in}{1.907728in}}%
\pgfpathlineto{\pgfqpoint{3.629869in}{1.899847in}}%
\pgfpathlineto{\pgfqpoint{3.637601in}{1.895531in}}%
\pgfpathlineto{\pgfqpoint{3.640178in}{1.897220in}}%
\pgfpathlineto{\pgfqpoint{3.642755in}{1.909979in}}%
\pgfpathlineto{\pgfqpoint{3.645332in}{1.905664in}}%
\pgfpathlineto{\pgfqpoint{3.647909in}{1.907728in}}%
\pgfpathlineto{\pgfqpoint{3.655641in}{1.899847in}}%
\pgfpathlineto{\pgfqpoint{3.658218in}{1.911105in}}%
\pgfpathlineto{\pgfqpoint{3.660795in}{1.912418in}}%
\pgfpathlineto{\pgfqpoint{3.665950in}{1.937561in}}%
\pgfpathlineto{\pgfqpoint{3.673682in}{1.931745in}}%
\pgfpathlineto{\pgfqpoint{3.676259in}{1.947881in}}%
\pgfpathlineto{\pgfqpoint{3.678836in}{1.930431in}}%
\pgfpathlineto{\pgfqpoint{3.681413in}{1.940001in}}%
\pgfpathlineto{\pgfqpoint{3.683990in}{1.936623in}}%
\pgfpathlineto{\pgfqpoint{3.691722in}{1.942815in}}%
\pgfpathlineto{\pgfqpoint{3.694299in}{1.942065in}}%
\pgfpathlineto{\pgfqpoint{3.696876in}{1.930431in}}%
\pgfpathlineto{\pgfqpoint{3.699454in}{1.937186in}}%
\pgfpathlineto{\pgfqpoint{3.702031in}{1.923114in}}%
\pgfpathlineto{\pgfqpoint{3.709763in}{1.917485in}}%
\pgfpathlineto{\pgfqpoint{3.712340in}{1.920111in}}%
\pgfpathlineto{\pgfqpoint{3.717494in}{1.962892in}}%
\pgfpathlineto{\pgfqpoint{3.720071in}{1.963830in}}%
\pgfpathlineto{\pgfqpoint{3.727803in}{1.972649in}}%
\pgfpathlineto{\pgfqpoint{3.730380in}{1.983907in}}%
\pgfpathlineto{\pgfqpoint{3.732957in}{1.981468in}}%
\pgfpathlineto{\pgfqpoint{3.738112in}{1.986721in}}%
\pgfpathlineto{\pgfqpoint{3.748421in}{1.969459in}}%
\pgfpathlineto{\pgfqpoint{3.750998in}{1.948256in}}%
\pgfpathlineto{\pgfqpoint{3.756152in}{1.937561in}}%
\pgfpathlineto{\pgfqpoint{3.763884in}{1.930619in}}%
\pgfpathlineto{\pgfqpoint{3.766461in}{1.924052in}}%
\pgfpathlineto{\pgfqpoint{3.769038in}{1.931557in}}%
\pgfpathlineto{\pgfqpoint{3.771616in}{1.948444in}}%
\pgfpathlineto{\pgfqpoint{3.774193in}{1.934559in}}%
\pgfpathlineto{\pgfqpoint{3.781924in}{1.929118in}}%
\pgfpathlineto{\pgfqpoint{3.784502in}{1.935310in}}%
\pgfpathlineto{\pgfqpoint{3.787079in}{1.930431in}}%
\pgfpathlineto{\pgfqpoint{3.789656in}{1.928180in}}%
\pgfpathlineto{\pgfqpoint{3.792233in}{1.950883in}}%
\pgfpathlineto{\pgfqpoint{3.802542in}{1.949758in}}%
\pgfpathlineto{\pgfqpoint{3.805119in}{1.952760in}}%
\pgfpathlineto{\pgfqpoint{3.807697in}{1.967395in}}%
\pgfpathlineto{\pgfqpoint{3.810274in}{1.942252in}}%
\pgfpathlineto{\pgfqpoint{3.818005in}{1.933996in}}%
\pgfpathlineto{\pgfqpoint{3.820583in}{1.883523in}}%
\pgfpathlineto{\pgfqpoint{3.823160in}{1.861570in}}%
\pgfpathlineto{\pgfqpoint{3.825737in}{1.870013in}}%
\pgfpathlineto{\pgfqpoint{3.828314in}{1.847497in}}%
\pgfpathlineto{\pgfqpoint{3.836046in}{1.861195in}}%
\pgfpathlineto{\pgfqpoint{3.838623in}{1.874704in}}%
\pgfpathlineto{\pgfqpoint{3.841200in}{1.872077in}}%
\pgfpathlineto{\pgfqpoint{3.843778in}{1.886900in}}%
\pgfpathlineto{\pgfqpoint{3.846355in}{1.869075in}}%
\pgfpathlineto{\pgfqpoint{3.854086in}{1.859881in}}%
\pgfpathlineto{\pgfqpoint{3.856664in}{1.865510in}}%
\pgfpathlineto{\pgfqpoint{3.859241in}{1.869638in}}%
\pgfpathlineto{\pgfqpoint{3.861818in}{1.876018in}}%
\pgfpathlineto{\pgfqpoint{3.864395in}{1.873954in}}%
\pgfpathlineto{\pgfqpoint{3.874704in}{1.867199in}}%
\pgfpathlineto{\pgfqpoint{3.877281in}{1.879770in}}%
\pgfpathlineto{\pgfqpoint{3.879859in}{1.862508in}}%
\pgfpathlineto{\pgfqpoint{3.882436in}{1.857067in}}%
\pgfpathlineto{\pgfqpoint{3.890167in}{1.865510in}}%
\pgfpathlineto{\pgfqpoint{3.892745in}{1.867386in}}%
\pgfpathlineto{\pgfqpoint{3.895322in}{1.866636in}}%
\pgfpathlineto{\pgfqpoint{3.897899in}{1.861945in}}%
\pgfpathlineto{\pgfqpoint{3.900476in}{1.861195in}}%
\pgfpathlineto{\pgfqpoint{3.908208in}{1.865885in}}%
\pgfpathlineto{\pgfqpoint{3.910785in}{1.861757in}}%
\pgfpathlineto{\pgfqpoint{3.913362in}{1.848435in}}%
\pgfpathlineto{\pgfqpoint{3.915940in}{1.853126in}}%
\pgfpathlineto{\pgfqpoint{3.918517in}{1.820853in}}%
\pgfpathlineto{\pgfqpoint{3.926248in}{1.827983in}}%
\pgfpathlineto{\pgfqpoint{3.928826in}{1.802653in}}%
\pgfpathlineto{\pgfqpoint{3.931403in}{1.800214in}}%
\pgfpathlineto{\pgfqpoint{3.933980in}{1.811659in}}%
\pgfpathlineto{\pgfqpoint{3.936557in}{1.807344in}}%
\pgfpathlineto{\pgfqpoint{3.944289in}{1.835676in}}%
\pgfpathlineto{\pgfqpoint{3.946866in}{1.824043in}}%
\pgfpathlineto{\pgfqpoint{3.949443in}{1.838678in}}%
\pgfpathlineto{\pgfqpoint{3.952020in}{1.832674in}}%
\pgfpathlineto{\pgfqpoint{3.954598in}{1.854815in}}%
\pgfpathlineto{\pgfqpoint{3.962329in}{1.856879in}}%
\pgfpathlineto{\pgfqpoint{3.970061in}{1.812597in}}%
\pgfpathlineto{\pgfqpoint{3.972638in}{1.815224in}}%
\pgfpathlineto{\pgfqpoint{3.980370in}{1.821791in}}%
\pgfpathlineto{\pgfqpoint{3.982947in}{1.809032in}}%
\pgfpathlineto{\pgfqpoint{3.985524in}{1.815412in}}%
\pgfpathlineto{\pgfqpoint{3.988101in}{1.817101in}}%
\pgfpathlineto{\pgfqpoint{3.998410in}{1.827045in}}%
\pgfpathlineto{\pgfqpoint{4.000988in}{1.816538in}}%
\pgfpathlineto{\pgfqpoint{4.003565in}{1.822730in}}%
\pgfpathlineto{\pgfqpoint{4.006142in}{1.824606in}}%
\pgfpathlineto{\pgfqpoint{4.008719in}{1.832111in}}%
\pgfpathlineto{\pgfqpoint{4.016451in}{1.833425in}}%
\pgfpathlineto{\pgfqpoint{4.019028in}{1.836239in}}%
\pgfpathlineto{\pgfqpoint{4.021605in}{1.834738in}}%
\pgfpathlineto{\pgfqpoint{4.024182in}{1.834551in}}%
\pgfpathlineto{\pgfqpoint{4.026760in}{1.818789in}}%
\pgfpathlineto{\pgfqpoint{4.034491in}{1.824418in}}%
\pgfpathlineto{\pgfqpoint{4.037069in}{1.827796in}}%
\pgfpathlineto{\pgfqpoint{4.039646in}{1.827983in}}%
\pgfpathlineto{\pgfqpoint{4.042223in}{1.803591in}}%
\pgfpathlineto{\pgfqpoint{4.044800in}{1.804342in}}%
\pgfpathlineto{\pgfqpoint{4.052532in}{1.797774in}}%
\pgfpathlineto{\pgfqpoint{4.055109in}{1.794772in}}%
\pgfpathlineto{\pgfqpoint{4.057686in}{1.785391in}}%
\pgfpathlineto{\pgfqpoint{4.060263in}{1.779762in}}%
\pgfpathlineto{\pgfqpoint{4.062841in}{1.792708in}}%
\pgfpathlineto{\pgfqpoint{4.070572in}{1.793646in}}%
\pgfpathlineto{\pgfqpoint{4.073149in}{1.789143in}}%
\pgfpathlineto{\pgfqpoint{4.075727in}{1.794585in}}%
\pgfpathlineto{\pgfqpoint{4.078304in}{1.791207in}}%
\pgfpathlineto{\pgfqpoint{4.080881in}{1.803779in}}%
\pgfpathlineto{\pgfqpoint{4.088613in}{1.792333in}}%
\pgfpathlineto{\pgfqpoint{4.093767in}{1.782951in}}%
\pgfpathlineto{\pgfqpoint{4.096344in}{1.797399in}}%
\pgfpathlineto{\pgfqpoint{4.098922in}{1.805280in}}%
\pgfpathlineto{\pgfqpoint{4.106653in}{1.800026in}}%
\pgfpathlineto{\pgfqpoint{4.109230in}{1.801527in}}%
\pgfpathlineto{\pgfqpoint{4.111808in}{1.795898in}}%
\pgfpathlineto{\pgfqpoint{4.114385in}{1.794772in}}%
\pgfpathlineto{\pgfqpoint{4.116962in}{1.787079in}}%
\pgfpathlineto{\pgfqpoint{4.127271in}{1.773757in}}%
\pgfpathlineto{\pgfqpoint{4.129848in}{1.777885in}}%
\pgfpathlineto{\pgfqpoint{4.132425in}{1.776947in}}%
\pgfpathlineto{\pgfqpoint{4.135003in}{1.761373in}}%
\pgfpathlineto{\pgfqpoint{4.142734in}{1.768879in}}%
\pgfpathlineto{\pgfqpoint{4.145311in}{1.763813in}}%
\pgfpathlineto{\pgfqpoint{4.147889in}{1.764188in}}%
\pgfpathlineto{\pgfqpoint{4.150466in}{1.757433in}}%
\pgfpathlineto{\pgfqpoint{4.153043in}{1.745612in}}%
\pgfpathlineto{\pgfqpoint{4.160775in}{1.750115in}}%
\pgfpathlineto{\pgfqpoint{4.163352in}{1.769817in}}%
\pgfpathlineto{\pgfqpoint{4.165929in}{1.780324in}}%
\pgfpathlineto{\pgfqpoint{4.168506in}{1.778260in}}%
\pgfpathlineto{\pgfqpoint{4.171084in}{1.769254in}}%
\pgfpathlineto{\pgfqpoint{4.178815in}{1.756870in}}%
\pgfpathlineto{\pgfqpoint{4.183970in}{1.789331in}}%
\pgfpathlineto{\pgfqpoint{4.186547in}{1.801527in}}%
\pgfpathlineto{\pgfqpoint{4.189124in}{1.796836in}}%
\pgfpathlineto{\pgfqpoint{4.196856in}{1.795335in}}%
\pgfpathlineto{\pgfqpoint{4.199433in}{1.784452in}}%
\pgfpathlineto{\pgfqpoint{4.202010in}{1.779762in}}%
\pgfpathlineto{\pgfqpoint{4.204587in}{1.777885in}}%
\pgfpathlineto{\pgfqpoint{4.207165in}{1.777135in}}%
\pgfpathlineto{\pgfqpoint{4.214896in}{1.760248in}}%
\pgfpathlineto{\pgfqpoint{4.217473in}{1.758934in}}%
\pgfpathlineto{\pgfqpoint{4.220051in}{1.783327in}}%
\pgfpathlineto{\pgfqpoint{4.222628in}{1.786704in}}%
\pgfpathlineto{\pgfqpoint{4.232937in}{1.788768in}}%
\pgfpathlineto{\pgfqpoint{4.235514in}{1.816350in}}%
\pgfpathlineto{\pgfqpoint{4.238091in}{1.804154in}}%
\pgfpathlineto{\pgfqpoint{4.240668in}{1.798713in}}%
\pgfpathlineto{\pgfqpoint{4.250977in}{1.819352in}}%
\pgfpathlineto{\pgfqpoint{4.258709in}{1.825919in}}%
\pgfpathlineto{\pgfqpoint{4.261286in}{1.824794in}}%
\pgfpathlineto{\pgfqpoint{4.269018in}{1.824043in}}%
\pgfpathlineto{\pgfqpoint{4.271595in}{1.814849in}}%
\pgfpathlineto{\pgfqpoint{4.276749in}{1.810346in}}%
\pgfpathlineto{\pgfqpoint{4.279326in}{1.803591in}}%
\pgfpathlineto{\pgfqpoint{4.287058in}{1.798150in}}%
\pgfpathlineto{\pgfqpoint{4.289635in}{1.802465in}}%
\pgfpathlineto{\pgfqpoint{4.292213in}{1.809032in}}%
\pgfpathlineto{\pgfqpoint{4.294790in}{1.755369in}}%
\pgfpathlineto{\pgfqpoint{4.297367in}{1.743924in}}%
\pgfpathlineto{\pgfqpoint{4.305099in}{1.738857in}}%
\pgfpathlineto{\pgfqpoint{4.307676in}{1.730789in}}%
\pgfpathlineto{\pgfqpoint{4.310253in}{1.728350in}}%
\pgfpathlineto{\pgfqpoint{4.312830in}{1.727975in}}%
\pgfpathlineto{\pgfqpoint{4.315407in}{1.723659in}}%
\pgfpathlineto{\pgfqpoint{4.323139in}{1.738482in}}%
\pgfpathlineto{\pgfqpoint{4.325716in}{1.735855in}}%
\pgfpathlineto{\pgfqpoint{4.328294in}{1.738670in}}%
\pgfpathlineto{\pgfqpoint{4.330871in}{1.728538in}}%
\pgfpathlineto{\pgfqpoint{4.333448in}{1.725911in}}%
\pgfpathlineto{\pgfqpoint{4.341180in}{1.724410in}}%
\pgfpathlineto{\pgfqpoint{4.343757in}{1.717842in}}%
\pgfpathlineto{\pgfqpoint{4.346334in}{1.700955in}}%
\pgfpathlineto{\pgfqpoint{4.348911in}{1.697578in}}%
\pgfpathlineto{\pgfqpoint{4.351488in}{1.663053in}}%
\pgfpathlineto{\pgfqpoint{4.361797in}{1.606388in}}%
\pgfpathlineto{\pgfqpoint{4.364374in}{1.647480in}}%
\pgfpathlineto{\pgfqpoint{4.366952in}{1.657237in}}%
\pgfpathlineto{\pgfqpoint{4.369529in}{1.652734in}}%
\pgfpathlineto{\pgfqpoint{4.377261in}{1.643727in}}%
\pgfpathlineto{\pgfqpoint{4.379838in}{1.614269in}}%
\pgfpathlineto{\pgfqpoint{4.382415in}{1.629467in}}%
\pgfpathlineto{\pgfqpoint{4.384992in}{1.631343in}}%
\pgfpathlineto{\pgfqpoint{4.387569in}{1.612017in}}%
\pgfpathlineto{\pgfqpoint{4.397878in}{1.632282in}}%
\pgfpathlineto{\pgfqpoint{4.400455in}{1.607326in}}%
\pgfpathlineto{\pgfqpoint{4.403033in}{1.604699in}}%
\pgfpathlineto{\pgfqpoint{4.405610in}{1.606388in}}%
\pgfpathlineto{\pgfqpoint{4.413342in}{1.600384in}}%
\pgfpathlineto{\pgfqpoint{4.415919in}{1.623463in}}%
\pgfpathlineto{\pgfqpoint{4.418496in}{1.634158in}}%
\pgfpathlineto{\pgfqpoint{4.421073in}{1.636597in}}%
\pgfpathlineto{\pgfqpoint{4.423650in}{1.631531in}}%
\pgfpathlineto{\pgfqpoint{4.431382in}{1.643352in}}%
\pgfpathlineto{\pgfqpoint{4.433959in}{1.635659in}}%
\pgfpathlineto{\pgfqpoint{4.436536in}{1.636972in}}%
\pgfpathlineto{\pgfqpoint{4.441691in}{1.676938in}}%
\pgfpathlineto{\pgfqpoint{4.449422in}{1.661928in}}%
\pgfpathlineto{\pgfqpoint{4.452000in}{1.670371in}}%
\pgfpathlineto{\pgfqpoint{4.454577in}{1.664742in}}%
\pgfpathlineto{\pgfqpoint{4.457154in}{1.664930in}}%
\pgfpathlineto{\pgfqpoint{4.459731in}{1.672810in}}%
\pgfpathlineto{\pgfqpoint{4.467463in}{1.686132in}}%
\pgfpathlineto{\pgfqpoint{4.470040in}{1.689135in}}%
\pgfpathlineto{\pgfqpoint{4.472617in}{1.694388in}}%
\pgfpathlineto{\pgfqpoint{4.475195in}{1.705646in}}%
\pgfpathlineto{\pgfqpoint{4.477772in}{1.706960in}}%
\pgfpathlineto{\pgfqpoint{4.485503in}{1.704520in}}%
\pgfpathlineto{\pgfqpoint{4.488081in}{1.700768in}}%
\pgfpathlineto{\pgfqpoint{4.493235in}{1.703582in}}%
\pgfpathlineto{\pgfqpoint{4.495812in}{1.713902in}}%
\pgfpathlineto{\pgfqpoint{4.503544in}{1.718218in}}%
\pgfpathlineto{\pgfqpoint{4.506121in}{1.706209in}}%
\pgfpathlineto{\pgfqpoint{4.508698in}{1.703207in}}%
\pgfpathlineto{\pgfqpoint{4.511276in}{1.724410in}}%
\pgfpathlineto{\pgfqpoint{4.513853in}{1.760811in}}%
\pgfpathlineto{\pgfqpoint{4.521584in}{1.768504in}}%
\pgfpathlineto{\pgfqpoint{4.524162in}{1.765314in}}%
\pgfpathlineto{\pgfqpoint{4.526739in}{1.752179in}}%
\pgfpathlineto{\pgfqpoint{4.529316in}{1.760623in}}%
\pgfpathlineto{\pgfqpoint{4.531893in}{1.749928in}}%
\pgfpathlineto{\pgfqpoint{4.539625in}{1.753680in}}%
\pgfpathlineto{\pgfqpoint{4.542202in}{1.761186in}}%
\pgfpathlineto{\pgfqpoint{4.544779in}{1.761373in}}%
\pgfpathlineto{\pgfqpoint{4.549934in}{1.736418in}}%
\pgfpathlineto{\pgfqpoint{4.557665in}{1.733604in}}%
\pgfpathlineto{\pgfqpoint{4.560243in}{1.739796in}}%
\pgfpathlineto{\pgfqpoint{4.562820in}{1.742985in}}%
\pgfpathlineto{\pgfqpoint{4.565397in}{1.721220in}}%
\pgfpathlineto{\pgfqpoint{4.567974in}{1.709399in}}%
\pgfpathlineto{\pgfqpoint{4.575706in}{1.731915in}}%
\pgfpathlineto{\pgfqpoint{4.578283in}{1.727975in}}%
\pgfpathlineto{\pgfqpoint{4.580860in}{1.741860in}}%
\pgfpathlineto{\pgfqpoint{4.583438in}{1.747301in}}%
\pgfpathlineto{\pgfqpoint{4.586015in}{1.740546in}}%
\pgfpathlineto{\pgfqpoint{4.593746in}{1.743173in}}%
\pgfpathlineto{\pgfqpoint{4.596324in}{1.751241in}}%
\pgfpathlineto{\pgfqpoint{4.598901in}{1.741860in}}%
\pgfpathlineto{\pgfqpoint{4.604055in}{1.738670in}}%
\pgfpathlineto{\pgfqpoint{4.611787in}{1.724222in}}%
\pgfpathlineto{\pgfqpoint{4.614364in}{1.742610in}}%
\pgfpathlineto{\pgfqpoint{4.616941in}{1.740171in}}%
\pgfpathlineto{\pgfqpoint{4.619519in}{1.739233in}}%
\pgfpathlineto{\pgfqpoint{4.622096in}{1.774320in}}%
\pgfpathlineto{\pgfqpoint{4.629827in}{1.783327in}}%
\pgfpathlineto{\pgfqpoint{4.632405in}{1.773007in}}%
\pgfpathlineto{\pgfqpoint{4.634982in}{1.772069in}}%
\pgfpathlineto{\pgfqpoint{4.637559in}{1.773570in}}%
\pgfpathlineto{\pgfqpoint{4.640136in}{1.773382in}}%
\pgfpathlineto{\pgfqpoint{4.647868in}{1.782013in}}%
\pgfpathlineto{\pgfqpoint{4.653022in}{1.827233in}}%
\pgfpathlineto{\pgfqpoint{4.655599in}{1.815600in}}%
\pgfpathlineto{\pgfqpoint{4.658177in}{1.779386in}}%
\pgfpathlineto{\pgfqpoint{4.665908in}{1.793084in}}%
\pgfpathlineto{\pgfqpoint{4.668486in}{1.803966in}}%
\pgfpathlineto{\pgfqpoint{4.671063in}{1.809408in}}%
\pgfpathlineto{\pgfqpoint{4.673640in}{1.807156in}}%
\pgfpathlineto{\pgfqpoint{4.683949in}{1.809408in}}%
\pgfpathlineto{\pgfqpoint{4.686526in}{1.816725in}}%
\pgfpathlineto{\pgfqpoint{4.689103in}{1.811847in}}%
\pgfpathlineto{\pgfqpoint{4.691680in}{1.800777in}}%
\pgfpathlineto{\pgfqpoint{4.701989in}{1.783327in}}%
\pgfpathlineto{\pgfqpoint{4.704567in}{1.787455in}}%
\pgfpathlineto{\pgfqpoint{4.709721in}{1.763437in}}%
\pgfpathlineto{\pgfqpoint{4.712298in}{1.743173in}}%
\pgfpathlineto{\pgfqpoint{4.720030in}{1.754806in}}%
\pgfpathlineto{\pgfqpoint{4.722607in}{1.752179in}}%
\pgfpathlineto{\pgfqpoint{4.725184in}{1.741109in}}%
\pgfpathlineto{\pgfqpoint{4.727761in}{1.746175in}}%
\pgfpathlineto{\pgfqpoint{4.730339in}{1.726474in}}%
\pgfpathlineto{\pgfqpoint{4.740648in}{1.755932in}}%
\pgfpathlineto{\pgfqpoint{4.743225in}{1.751804in}}%
\pgfpathlineto{\pgfqpoint{4.745802in}{1.766815in}}%
\pgfpathlineto{\pgfqpoint{4.748379in}{1.777698in}}%
\pgfpathlineto{\pgfqpoint{4.756111in}{1.769066in}}%
\pgfpathlineto{\pgfqpoint{4.758688in}{1.802278in}}%
\pgfpathlineto{\pgfqpoint{4.761265in}{1.802090in}}%
\pgfpathlineto{\pgfqpoint{4.763842in}{1.819352in}}%
\pgfpathlineto{\pgfqpoint{4.766420in}{1.850875in}}%
\pgfpathlineto{\pgfqpoint{4.774151in}{1.841305in}}%
\pgfpathlineto{\pgfqpoint{4.776728in}{1.826107in}}%
\pgfpathlineto{\pgfqpoint{4.779306in}{1.840930in}}%
\pgfpathlineto{\pgfqpoint{4.781883in}{1.834175in}}%
\pgfpathlineto{\pgfqpoint{4.792192in}{1.866636in}}%
\pgfpathlineto{\pgfqpoint{4.794769in}{1.867011in}}%
\pgfpathlineto{\pgfqpoint{4.797346in}{1.849749in}}%
\pgfpathlineto{\pgfqpoint{4.799923in}{1.820666in}}%
\pgfpathlineto{\pgfqpoint{4.802501in}{1.839054in}}%
\pgfpathlineto{\pgfqpoint{4.812809in}{1.847122in}}%
\pgfpathlineto{\pgfqpoint{4.815387in}{1.863821in}}%
\pgfpathlineto{\pgfqpoint{4.817964in}{1.855753in}}%
\pgfpathlineto{\pgfqpoint{4.820541in}{1.852563in}}%
\pgfpathlineto{\pgfqpoint{4.828273in}{1.858380in}}%
\pgfpathlineto{\pgfqpoint{4.830850in}{1.852939in}}%
\pgfpathlineto{\pgfqpoint{4.833427in}{1.848811in}}%
\pgfpathlineto{\pgfqpoint{4.836004in}{1.861945in}}%
\pgfpathlineto{\pgfqpoint{4.838582in}{1.840930in}}%
\pgfpathlineto{\pgfqpoint{4.846313in}{1.827233in}}%
\pgfpathlineto{\pgfqpoint{4.848890in}{1.843182in}}%
\pgfpathlineto{\pgfqpoint{4.851468in}{1.865510in}}%
\pgfpathlineto{\pgfqpoint{4.854045in}{1.870389in}}%
\pgfpathlineto{\pgfqpoint{4.856622in}{1.881459in}}%
\pgfpathlineto{\pgfqpoint{4.864354in}{1.874704in}}%
\pgfpathlineto{\pgfqpoint{4.866931in}{1.874141in}}%
\pgfpathlineto{\pgfqpoint{4.869508in}{1.872453in}}%
\pgfpathlineto{\pgfqpoint{4.872085in}{1.860819in}}%
\pgfpathlineto{\pgfqpoint{4.874663in}{1.852000in}}%
\pgfpathlineto{\pgfqpoint{4.882394in}{1.842056in}}%
\pgfpathlineto{\pgfqpoint{4.884971in}{1.844495in}}%
\pgfpathlineto{\pgfqpoint{4.887549in}{1.845058in}}%
\pgfpathlineto{\pgfqpoint{4.890126in}{1.868887in}}%
\pgfpathlineto{\pgfqpoint{4.892703in}{1.875642in}}%
\pgfpathlineto{\pgfqpoint{4.900435in}{1.878457in}}%
\pgfpathlineto{\pgfqpoint{4.903012in}{1.868700in}}%
\pgfpathlineto{\pgfqpoint{4.908166in}{1.871139in}}%
\pgfpathlineto{\pgfqpoint{4.918475in}{1.866636in}}%
\pgfpathlineto{\pgfqpoint{4.921052in}{1.869638in}}%
\pgfpathlineto{\pgfqpoint{4.923630in}{1.867762in}}%
\pgfpathlineto{\pgfqpoint{4.926207in}{1.861382in}}%
\pgfpathlineto{\pgfqpoint{4.928784in}{1.882022in}}%
\pgfpathlineto{\pgfqpoint{4.936516in}{1.876580in}}%
\pgfpathlineto{\pgfqpoint{4.939093in}{1.875830in}}%
\pgfpathlineto{\pgfqpoint{4.941670in}{1.886713in}}%
\pgfpathlineto{\pgfqpoint{4.944247in}{1.877143in}}%
\pgfpathlineto{\pgfqpoint{4.946825in}{1.876393in}}%
\pgfpathlineto{\pgfqpoint{4.954556in}{1.868512in}}%
\pgfpathlineto{\pgfqpoint{4.957133in}{1.870201in}}%
\pgfpathlineto{\pgfqpoint{4.959711in}{1.864009in}}%
\pgfpathlineto{\pgfqpoint{4.962288in}{1.867762in}}%
\pgfpathlineto{\pgfqpoint{4.964865in}{1.872640in}}%
\pgfpathlineto{\pgfqpoint{4.972597in}{1.881647in}}%
\pgfpathlineto{\pgfqpoint{4.975174in}{1.889340in}}%
\pgfpathlineto{\pgfqpoint{4.977751in}{1.859881in}}%
\pgfpathlineto{\pgfqpoint{4.980328in}{1.847122in}}%
\pgfpathlineto{\pgfqpoint{4.990637in}{1.857442in}}%
\pgfpathlineto{\pgfqpoint{4.993214in}{1.825732in}}%
\pgfpathlineto{\pgfqpoint{4.995792in}{1.831548in}}%
\pgfpathlineto{\pgfqpoint{4.998369in}{1.829297in}}%
\pgfpathlineto{\pgfqpoint{5.000946in}{1.835489in}}%
\pgfpathlineto{\pgfqpoint{5.008678in}{1.849937in}}%
\pgfpathlineto{\pgfqpoint{5.011255in}{1.852188in}}%
\pgfpathlineto{\pgfqpoint{5.013832in}{1.860632in}}%
\pgfpathlineto{\pgfqpoint{5.016409in}{1.855566in}}%
\pgfpathlineto{\pgfqpoint{5.018986in}{1.869826in}}%
\pgfpathlineto{\pgfqpoint{5.026718in}{1.869638in}}%
\pgfpathlineto{\pgfqpoint{5.029295in}{1.875642in}}%
\pgfpathlineto{\pgfqpoint{5.031873in}{1.870013in}}%
\pgfpathlineto{\pgfqpoint{5.034450in}{1.874516in}}%
\pgfpathlineto{\pgfqpoint{5.037027in}{1.854440in}}%
\pgfpathlineto{\pgfqpoint{5.044759in}{1.861195in}}%
\pgfpathlineto{\pgfqpoint{5.047336in}{1.843932in}}%
\pgfpathlineto{\pgfqpoint{5.049913in}{1.830798in}}%
\pgfpathlineto{\pgfqpoint{5.052490in}{1.836615in}}%
\pgfpathlineto{\pgfqpoint{5.055067in}{1.833800in}}%
\pgfpathlineto{\pgfqpoint{5.062799in}{1.836802in}}%
\pgfpathlineto{\pgfqpoint{5.065376in}{1.849937in}}%
\pgfpathlineto{\pgfqpoint{5.067953in}{1.858568in}}%
\pgfpathlineto{\pgfqpoint{5.070531in}{1.854252in}}%
\pgfpathlineto{\pgfqpoint{5.073108in}{1.857817in}}%
\pgfpathlineto{\pgfqpoint{5.083417in}{1.851062in}}%
\pgfpathlineto{\pgfqpoint{5.085994in}{1.864009in}}%
\pgfpathlineto{\pgfqpoint{5.088571in}{1.866636in}}%
\pgfpathlineto{\pgfqpoint{5.091148in}{1.875455in}}%
\pgfpathlineto{\pgfqpoint{5.098880in}{1.880708in}}%
\pgfpathlineto{\pgfqpoint{5.101457in}{1.873015in}}%
\pgfpathlineto{\pgfqpoint{5.104034in}{1.878644in}}%
\pgfpathlineto{\pgfqpoint{5.106612in}{1.887463in}}%
\pgfpathlineto{\pgfqpoint{5.109189in}{1.888026in}}%
\pgfpathlineto{\pgfqpoint{5.116921in}{1.877331in}}%
\pgfpathlineto{\pgfqpoint{5.119498in}{1.890465in}}%
\pgfpathlineto{\pgfqpoint{5.122075in}{1.883711in}}%
\pgfpathlineto{\pgfqpoint{5.124652in}{1.891591in}}%
\pgfpathlineto{\pgfqpoint{5.127229in}{1.886713in}}%
\pgfpathlineto{\pgfqpoint{5.134961in}{1.885212in}}%
\pgfpathlineto{\pgfqpoint{5.137538in}{1.891591in}}%
\pgfpathlineto{\pgfqpoint{5.140115in}{1.894218in}}%
\pgfpathlineto{\pgfqpoint{5.142693in}{1.905288in}}%
\pgfpathlineto{\pgfqpoint{5.145270in}{1.871890in}}%
\pgfpathlineto{\pgfqpoint{5.153001in}{1.854440in}}%
\pgfpathlineto{\pgfqpoint{5.160733in}{1.912981in}}%
\pgfpathlineto{\pgfqpoint{5.163310in}{1.914858in}}%
\pgfpathlineto{\pgfqpoint{5.173619in}{1.926116in}}%
\pgfpathlineto{\pgfqpoint{5.176196in}{1.919173in}}%
\pgfpathlineto{\pgfqpoint{5.178774in}{1.915796in}}%
\pgfpathlineto{\pgfqpoint{5.181351in}{1.931745in}}%
\pgfpathlineto{\pgfqpoint{5.191660in}{1.931369in}}%
\pgfpathlineto{\pgfqpoint{5.194237in}{1.933809in}}%
\pgfpathlineto{\pgfqpoint{5.196814in}{1.933433in}}%
\pgfpathlineto{\pgfqpoint{5.199391in}{1.935873in}}%
\pgfpathlineto{\pgfqpoint{5.207123in}{1.934559in}}%
\pgfpathlineto{\pgfqpoint{5.209700in}{1.938687in}}%
\pgfpathlineto{\pgfqpoint{5.212277in}{1.935497in}}%
\pgfpathlineto{\pgfqpoint{5.214855in}{1.934559in}}%
\pgfpathlineto{\pgfqpoint{5.217432in}{1.942440in}}%
\pgfpathlineto{\pgfqpoint{5.225163in}{1.943753in}}%
\pgfpathlineto{\pgfqpoint{5.227741in}{1.934747in}}%
\pgfpathlineto{\pgfqpoint{5.230318in}{1.920674in}}%
\pgfpathlineto{\pgfqpoint{5.232895in}{1.926679in}}%
\pgfpathlineto{\pgfqpoint{5.235472in}{1.940188in}}%
\pgfpathlineto{\pgfqpoint{5.243204in}{1.954261in}}%
\pgfpathlineto{\pgfqpoint{5.245781in}{1.960265in}}%
\pgfpathlineto{\pgfqpoint{5.248358in}{1.946755in}}%
\pgfpathlineto{\pgfqpoint{5.250936in}{1.948069in}}%
\pgfpathlineto{\pgfqpoint{5.253513in}{1.943378in}}%
\pgfpathlineto{\pgfqpoint{5.261244in}{1.943003in}}%
\pgfpathlineto{\pgfqpoint{5.263822in}{1.946943in}}%
\pgfpathlineto{\pgfqpoint{5.266399in}{1.952572in}}%
\pgfpathlineto{\pgfqpoint{5.268976in}{1.959702in}}%
\pgfpathlineto{\pgfqpoint{5.271553in}{1.965144in}}%
\pgfpathlineto{\pgfqpoint{5.279285in}{1.964768in}}%
\pgfpathlineto{\pgfqpoint{5.281862in}{1.957075in}}%
\pgfpathlineto{\pgfqpoint{5.284439in}{1.963642in}}%
\pgfpathlineto{\pgfqpoint{5.287017in}{1.971898in}}%
\pgfpathlineto{\pgfqpoint{5.289594in}{1.969647in}}%
\pgfpathlineto{\pgfqpoint{5.297325in}{1.961766in}}%
\pgfpathlineto{\pgfqpoint{5.299903in}{1.971335in}}%
\pgfpathlineto{\pgfqpoint{5.302480in}{1.969647in}}%
\pgfpathlineto{\pgfqpoint{5.305057in}{1.979779in}}%
\pgfpathlineto{\pgfqpoint{5.307634in}{1.974338in}}%
\pgfpathlineto{\pgfqpoint{5.315366in}{1.986721in}}%
\pgfpathlineto{\pgfqpoint{5.317943in}{1.973587in}}%
\pgfpathlineto{\pgfqpoint{5.320520in}{1.969647in}}%
\pgfpathlineto{\pgfqpoint{5.323098in}{1.986909in}}%
\pgfpathlineto{\pgfqpoint{5.325675in}{1.985033in}}%
\pgfpathlineto{\pgfqpoint{5.335984in}{1.992538in}}%
\pgfpathlineto{\pgfqpoint{5.338561in}{1.980717in}}%
\pgfpathlineto{\pgfqpoint{5.341138in}{1.977715in}}%
\pgfpathlineto{\pgfqpoint{5.343715in}{1.951259in}}%
\pgfpathlineto{\pgfqpoint{5.351447in}{1.985783in}}%
\pgfpathlineto{\pgfqpoint{5.354024in}{1.965331in}}%
\pgfpathlineto{\pgfqpoint{5.356601in}{1.964581in}}%
\pgfpathlineto{\pgfqpoint{5.359178in}{1.982593in}}%
\pgfpathlineto{\pgfqpoint{5.361756in}{1.982406in}}%
\pgfpathlineto{\pgfqpoint{5.369487in}{1.987847in}}%
\pgfpathlineto{\pgfqpoint{5.372065in}{1.991600in}}%
\pgfpathlineto{\pgfqpoint{5.374642in}{1.978090in}}%
\pgfpathlineto{\pgfqpoint{5.377219in}{1.998542in}}%
\pgfpathlineto{\pgfqpoint{5.379796in}{1.977527in}}%
\pgfpathlineto{\pgfqpoint{5.387528in}{1.979028in}}%
\pgfpathlineto{\pgfqpoint{5.390105in}{1.987660in}}%
\pgfpathlineto{\pgfqpoint{5.392682in}{2.006611in}}%
\pgfpathlineto{\pgfqpoint{5.395259in}{1.985596in}}%
\pgfpathlineto{\pgfqpoint{5.397837in}{2.011677in}}%
\pgfpathlineto{\pgfqpoint{5.408146in}{1.987660in}}%
\pgfpathlineto{\pgfqpoint{5.410723in}{1.996103in}}%
\pgfpathlineto{\pgfqpoint{5.413300in}{2.002483in}}%
\pgfpathlineto{\pgfqpoint{5.415877in}{2.015992in}}%
\pgfpathlineto{\pgfqpoint{5.423609in}{1.999856in}}%
\pgfpathlineto{\pgfqpoint{5.426186in}{1.990849in}}%
\pgfpathlineto{\pgfqpoint{5.428763in}{1.991412in}}%
\pgfpathlineto{\pgfqpoint{5.431340in}{1.985596in}}%
\pgfpathlineto{\pgfqpoint{5.433918in}{1.988973in}}%
\pgfpathlineto{\pgfqpoint{5.441649in}{1.978653in}}%
\pgfpathlineto{\pgfqpoint{5.444227in}{1.972086in}}%
\pgfpathlineto{\pgfqpoint{5.446804in}{1.950696in}}%
\pgfpathlineto{\pgfqpoint{5.451958in}{1.929681in}}%
\pgfpathlineto{\pgfqpoint{5.459690in}{1.925740in}}%
\pgfpathlineto{\pgfqpoint{5.462267in}{1.975463in}}%
\pgfpathlineto{\pgfqpoint{5.464844in}{1.982781in}}%
\pgfpathlineto{\pgfqpoint{5.467421in}{1.968709in}}%
\pgfpathlineto{\pgfqpoint{5.469999in}{1.973212in}}%
\pgfpathlineto{\pgfqpoint{5.477730in}{1.972461in}}%
\pgfpathlineto{\pgfqpoint{5.480307in}{1.973399in}}%
\pgfpathlineto{\pgfqpoint{5.485462in}{1.968896in}}%
\pgfpathlineto{\pgfqpoint{5.488039in}{1.942627in}}%
\pgfpathlineto{\pgfqpoint{5.495771in}{1.968333in}}%
\pgfpathlineto{\pgfqpoint{5.498348in}{1.983907in}}%
\pgfpathlineto{\pgfqpoint{5.500925in}{1.957451in}}%
\pgfpathlineto{\pgfqpoint{5.503502in}{1.906039in}}%
\pgfpathlineto{\pgfqpoint{5.506080in}{1.916734in}}%
\pgfpathlineto{\pgfqpoint{5.513811in}{1.906602in}}%
\pgfpathlineto{\pgfqpoint{5.516388in}{1.917297in}}%
\pgfpathlineto{\pgfqpoint{5.518966in}{1.909979in}}%
\pgfpathlineto{\pgfqpoint{5.521543in}{1.907915in}}%
\pgfpathlineto{\pgfqpoint{5.524120in}{1.889340in}}%
\pgfpathlineto{\pgfqpoint{5.531852in}{1.900410in}}%
\pgfpathlineto{\pgfqpoint{5.534429in}{1.902474in}}%
\pgfpathlineto{\pgfqpoint{5.537006in}{1.901160in}}%
\pgfpathlineto{\pgfqpoint{5.542161in}{1.914670in}}%
\pgfpathlineto{\pgfqpoint{5.549892in}{1.907915in}}%
\pgfpathlineto{\pgfqpoint{5.552469in}{1.904726in}}%
\pgfpathlineto{\pgfqpoint{5.555047in}{1.897220in}}%
\pgfpathlineto{\pgfqpoint{5.557624in}{1.886900in}}%
\pgfpathlineto{\pgfqpoint{5.560201in}{1.896282in}}%
\pgfpathlineto{\pgfqpoint{5.567933in}{1.906414in}}%
\pgfpathlineto{\pgfqpoint{5.570510in}{1.905101in}}%
\pgfpathlineto{\pgfqpoint{5.573087in}{1.927054in}}%
\pgfpathlineto{\pgfqpoint{5.575664in}{1.915233in}}%
\pgfpathlineto{\pgfqpoint{5.578242in}{1.930431in}}%
\pgfpathlineto{\pgfqpoint{5.585973in}{1.943566in}}%
\pgfpathlineto{\pgfqpoint{5.588550in}{1.944316in}}%
\pgfpathlineto{\pgfqpoint{5.591128in}{1.930431in}}%
\pgfpathlineto{\pgfqpoint{5.593705in}{1.935685in}}%
\pgfpathlineto{\pgfqpoint{5.596282in}{1.935685in}}%
\pgfpathlineto{\pgfqpoint{5.604014in}{1.936248in}}%
\pgfpathlineto{\pgfqpoint{5.606591in}{1.933809in}}%
\pgfpathlineto{\pgfqpoint{5.609168in}{1.928743in}}%
\pgfpathlineto{\pgfqpoint{5.611745in}{1.932120in}}%
\pgfpathlineto{\pgfqpoint{5.614323in}{1.940564in}}%
\pgfpathlineto{\pgfqpoint{5.624631in}{1.934372in}}%
\pgfpathlineto{\pgfqpoint{5.627209in}{1.925178in}}%
\pgfpathlineto{\pgfqpoint{5.629786in}{1.930056in}}%
\pgfpathlineto{\pgfqpoint{5.632363in}{1.925365in}}%
\pgfpathlineto{\pgfqpoint{5.642672in}{1.927429in}}%
\pgfpathlineto{\pgfqpoint{5.645249in}{1.932683in}}%
\pgfpathlineto{\pgfqpoint{5.647826in}{1.942252in}}%
\pgfpathlineto{\pgfqpoint{5.650403in}{1.941877in}}%
\pgfpathlineto{\pgfqpoint{5.658135in}{1.930807in}}%
\pgfpathlineto{\pgfqpoint{5.660712in}{1.915045in}}%
\pgfpathlineto{\pgfqpoint{5.663290in}{1.919549in}}%
\pgfpathlineto{\pgfqpoint{5.665867in}{1.921237in}}%
\pgfpathlineto{\pgfqpoint{5.668444in}{1.924052in}}%
\pgfpathlineto{\pgfqpoint{5.678753in}{1.944879in}}%
\pgfpathlineto{\pgfqpoint{5.681330in}{1.951634in}}%
\pgfpathlineto{\pgfqpoint{5.683907in}{1.947694in}}%
\pgfpathlineto{\pgfqpoint{5.686484in}{1.995728in}}%
\pgfpathlineto{\pgfqpoint{5.694216in}{1.987097in}}%
\pgfpathlineto{\pgfqpoint{5.696793in}{2.002858in}}%
\pgfpathlineto{\pgfqpoint{5.701948in}{1.980905in}}%
\pgfpathlineto{\pgfqpoint{5.704525in}{1.982969in}}%
\pgfpathlineto{\pgfqpoint{5.712257in}{1.983532in}}%
\pgfpathlineto{\pgfqpoint{5.714834in}{1.998355in}}%
\pgfpathlineto{\pgfqpoint{5.717411in}{1.993664in}}%
\pgfpathlineto{\pgfqpoint{5.719988in}{2.001169in}}%
\pgfpathlineto{\pgfqpoint{5.722565in}{1.994977in}}%
\pgfpathlineto{\pgfqpoint{5.730297in}{1.994790in}}%
\pgfpathlineto{\pgfqpoint{5.732874in}{2.005485in}}%
\pgfpathlineto{\pgfqpoint{5.738029in}{2.016930in}}%
\pgfpathlineto{\pgfqpoint{5.740606in}{2.004734in}}%
\pgfpathlineto{\pgfqpoint{5.748338in}{2.010738in}}%
\pgfpathlineto{\pgfqpoint{5.750915in}{2.002858in}}%
\pgfpathlineto{\pgfqpoint{5.753492in}{2.059711in}}%
\pgfpathlineto{\pgfqpoint{5.756069in}{2.053894in}}%
\pgfpathlineto{\pgfqpoint{5.758646in}{2.059148in}}%
\pgfpathlineto{\pgfqpoint{5.768955in}{2.069280in}}%
\pgfpathlineto{\pgfqpoint{5.771532in}{2.065340in}}%
\pgfpathlineto{\pgfqpoint{5.774110in}{2.059898in}}%
\pgfpathlineto{\pgfqpoint{5.776687in}{2.058585in}}%
\pgfpathlineto{\pgfqpoint{5.784419in}{2.055771in}}%
\pgfpathlineto{\pgfqpoint{5.786996in}{2.058960in}}%
\pgfpathlineto{\pgfqpoint{5.789573in}{2.069093in}}%
\pgfpathlineto{\pgfqpoint{5.792150in}{2.056146in}}%
\pgfpathlineto{\pgfqpoint{5.794727in}{2.049016in}}%
\pgfpathlineto{\pgfqpoint{5.805036in}{2.045263in}}%
\pgfpathlineto{\pgfqpoint{5.807613in}{2.042636in}}%
\pgfpathlineto{\pgfqpoint{5.810191in}{2.046201in}}%
\pgfpathlineto{\pgfqpoint{5.812768in}{2.058960in}}%
\pgfpathlineto{\pgfqpoint{5.820500in}{2.063088in}}%
\pgfpathlineto{\pgfqpoint{5.823077in}{2.057647in}}%
\pgfpathlineto{\pgfqpoint{5.825654in}{2.064589in}}%
\pgfpathlineto{\pgfqpoint{5.828231in}{2.065340in}}%
\pgfpathlineto{\pgfqpoint{5.830808in}{2.057647in}}%
\pgfpathlineto{\pgfqpoint{5.838540in}{2.061400in}}%
\pgfpathlineto{\pgfqpoint{5.841117in}{2.061024in}}%
\pgfpathlineto{\pgfqpoint{5.848849in}{2.050142in}}%
\pgfpathlineto{\pgfqpoint{5.856580in}{2.048828in}}%
\pgfpathlineto{\pgfqpoint{5.859158in}{2.053519in}}%
\pgfpathlineto{\pgfqpoint{5.861735in}{2.050704in}}%
\pgfpathlineto{\pgfqpoint{5.866889in}{2.037570in}}%
\pgfpathlineto{\pgfqpoint{5.874621in}{2.034568in}}%
\pgfpathlineto{\pgfqpoint{5.877198in}{2.038696in}}%
\pgfpathlineto{\pgfqpoint{5.879775in}{2.039634in}}%
\pgfpathlineto{\pgfqpoint{5.882353in}{2.029689in}}%
\pgfpathlineto{\pgfqpoint{5.884930in}{2.026687in}}%
\pgfpathlineto{\pgfqpoint{5.892661in}{2.031378in}}%
\pgfpathlineto{\pgfqpoint{5.895239in}{2.036632in}}%
\pgfpathlineto{\pgfqpoint{5.897816in}{2.045638in}}%
\pgfpathlineto{\pgfqpoint{5.900393in}{2.040760in}}%
\pgfpathlineto{\pgfqpoint{5.910702in}{2.046952in}}%
\pgfpathlineto{\pgfqpoint{5.913279in}{2.054082in}}%
\pgfpathlineto{\pgfqpoint{5.915856in}{2.045263in}}%
\pgfpathlineto{\pgfqpoint{5.918434in}{2.040385in}}%
\pgfpathlineto{\pgfqpoint{5.921011in}{2.028001in}}%
\pgfpathlineto{\pgfqpoint{5.928742in}{2.044325in}}%
\pgfpathlineto{\pgfqpoint{5.931320in}{2.052206in}}%
\pgfpathlineto{\pgfqpoint{5.933897in}{2.012427in}}%
\pgfpathlineto{\pgfqpoint{5.936474in}{2.011677in}}%
\pgfpathlineto{\pgfqpoint{5.939051in}{2.005297in}}%
\pgfpathlineto{\pgfqpoint{5.946783in}{2.000982in}}%
\pgfpathlineto{\pgfqpoint{5.949360in}{1.985783in}}%
\pgfpathlineto{\pgfqpoint{5.951937in}{1.989348in}}%
\pgfpathlineto{\pgfqpoint{5.954515in}{1.989724in}}%
\pgfpathlineto{\pgfqpoint{5.957092in}{1.990662in}}%
\pgfpathlineto{\pgfqpoint{5.964823in}{1.991600in}}%
\pgfpathlineto{\pgfqpoint{5.967401in}{1.988973in}}%
\pgfpathlineto{\pgfqpoint{5.969978in}{1.990474in}}%
\pgfpathlineto{\pgfqpoint{5.972555in}{1.984845in}}%
\pgfpathlineto{\pgfqpoint{5.975132in}{1.985220in}}%
\pgfpathlineto{\pgfqpoint{5.982864in}{1.987660in}}%
\pgfpathlineto{\pgfqpoint{5.985441in}{1.986158in}}%
\pgfpathlineto{\pgfqpoint{5.988018in}{1.986534in}}%
\pgfpathlineto{\pgfqpoint{5.990596in}{1.979591in}}%
\pgfpathlineto{\pgfqpoint{5.993173in}{1.986158in}}%
\pgfpathlineto{\pgfqpoint{6.000904in}{1.985596in}}%
\pgfpathlineto{\pgfqpoint{6.003482in}{1.983344in}}%
\pgfpathlineto{\pgfqpoint{6.011213in}{2.003796in}}%
\pgfpathlineto{\pgfqpoint{6.021522in}{2.006423in}}%
\pgfpathlineto{\pgfqpoint{6.024099in}{2.018619in}}%
\pgfpathlineto{\pgfqpoint{6.026677in}{2.019370in}}%
\pgfpathlineto{\pgfqpoint{6.029254in}{2.027438in}}%
\pgfpathlineto{\pgfqpoint{6.039563in}{2.031191in}}%
\pgfpathlineto{\pgfqpoint{6.042140in}{2.030628in}}%
\pgfpathlineto{\pgfqpoint{6.044717in}{2.014491in}}%
\pgfpathlineto{\pgfqpoint{6.047294in}{2.019933in}}%
\pgfpathlineto{\pgfqpoint{6.055026in}{2.021246in}}%
\pgfpathlineto{\pgfqpoint{6.057603in}{2.018056in}}%
\pgfpathlineto{\pgfqpoint{6.060180in}{2.024811in}}%
\pgfpathlineto{\pgfqpoint{6.062757in}{2.041323in}}%
\pgfpathlineto{\pgfqpoint{6.065335in}{2.046201in}}%
\pgfpathlineto{\pgfqpoint{6.073066in}{2.049766in}}%
\pgfpathlineto{\pgfqpoint{6.078221in}{2.041698in}}%
\pgfpathlineto{\pgfqpoint{6.080798in}{2.035131in}}%
\pgfpathlineto{\pgfqpoint{6.083375in}{2.042073in}}%
\pgfpathlineto{\pgfqpoint{6.091107in}{2.040947in}}%
\pgfpathlineto{\pgfqpoint{6.093684in}{2.027813in}}%
\pgfpathlineto{\pgfqpoint{6.096261in}{2.023498in}}%
\pgfpathlineto{\pgfqpoint{6.098838in}{1.999293in}}%
\pgfpathlineto{\pgfqpoint{6.101416in}{2.002107in}}%
\pgfpathlineto{\pgfqpoint{6.109147in}{2.012427in}}%
\pgfpathlineto{\pgfqpoint{6.114302in}{2.011301in}}%
\pgfpathlineto{\pgfqpoint{6.116879in}{2.006048in}}%
\pgfpathlineto{\pgfqpoint{6.119456in}{2.010926in}}%
\pgfpathlineto{\pgfqpoint{6.127188in}{2.000982in}}%
\pgfpathlineto{\pgfqpoint{6.129765in}{1.995353in}}%
\pgfpathlineto{\pgfqpoint{6.132342in}{1.998730in}}%
\pgfpathlineto{\pgfqpoint{6.134919in}{1.994227in}}%
\pgfpathlineto{\pgfqpoint{6.137497in}{2.001169in}}%
\pgfpathlineto{\pgfqpoint{6.145228in}{2.009050in}}%
\pgfpathlineto{\pgfqpoint{6.147806in}{2.028376in}}%
\pgfpathlineto{\pgfqpoint{6.150383in}{2.035318in}}%
\pgfpathlineto{\pgfqpoint{6.152960in}{2.039822in}}%
\pgfpathlineto{\pgfqpoint{6.155537in}{2.040009in}}%
\pgfpathlineto{\pgfqpoint{6.163269in}{2.032316in}}%
\pgfpathlineto{\pgfqpoint{6.165846in}{2.049391in}}%
\pgfpathlineto{\pgfqpoint{6.168423in}{2.052206in}}%
\pgfpathlineto{\pgfqpoint{6.171000in}{2.076598in}}%
\pgfpathlineto{\pgfqpoint{6.173578in}{2.068342in}}%
\pgfpathlineto{\pgfqpoint{6.181309in}{2.079225in}}%
\pgfpathlineto{\pgfqpoint{6.183886in}{2.084103in}}%
\pgfpathlineto{\pgfqpoint{6.186464in}{2.082977in}}%
\pgfpathlineto{\pgfqpoint{6.191618in}{2.076410in}}%
\pgfpathlineto{\pgfqpoint{6.199350in}{2.090107in}}%
\pgfpathlineto{\pgfqpoint{6.201927in}{2.092547in}}%
\pgfpathlineto{\pgfqpoint{6.204504in}{2.098739in}}%
\pgfpathlineto{\pgfqpoint{6.207081in}{2.095549in}}%
\pgfpathlineto{\pgfqpoint{6.209659in}{2.088419in}}%
\pgfpathlineto{\pgfqpoint{6.217390in}{2.093860in}}%
\pgfpathlineto{\pgfqpoint{6.219967in}{2.103617in}}%
\pgfpathlineto{\pgfqpoint{6.222545in}{2.107933in}}%
\pgfpathlineto{\pgfqpoint{6.225122in}{2.101365in}}%
\pgfpathlineto{\pgfqpoint{6.227699in}{2.108496in}}%
\pgfpathlineto{\pgfqpoint{6.235431in}{2.115250in}}%
\pgfpathlineto{\pgfqpoint{6.238008in}{2.114125in}}%
\pgfpathlineto{\pgfqpoint{6.240585in}{2.107933in}}%
\pgfpathlineto{\pgfqpoint{6.243162in}{2.104931in}}%
\pgfpathlineto{\pgfqpoint{6.245740in}{2.109058in}}%
\pgfpathlineto{\pgfqpoint{6.253471in}{2.108496in}}%
\pgfpathlineto{\pgfqpoint{6.256048in}{2.105681in}}%
\pgfpathlineto{\pgfqpoint{6.258626in}{2.097800in}}%
\pgfpathlineto{\pgfqpoint{6.261203in}{2.104931in}}%
\pgfpathlineto{\pgfqpoint{6.263780in}{2.109434in}}%
\pgfpathlineto{\pgfqpoint{6.274089in}{2.112811in}}%
\pgfpathlineto{\pgfqpoint{6.276666in}{2.112811in}}%
\pgfpathlineto{\pgfqpoint{6.279243in}{2.117314in}}%
\pgfpathlineto{\pgfqpoint{6.281821in}{2.114875in}}%
\pgfpathlineto{\pgfqpoint{6.289552in}{2.135327in}}%
\pgfpathlineto{\pgfqpoint{6.292129in}{2.126884in}}%
\pgfpathlineto{\pgfqpoint{6.294707in}{2.127634in}}%
\pgfpathlineto{\pgfqpoint{6.297284in}{2.127634in}}%
\pgfpathlineto{\pgfqpoint{6.299861in}{2.122568in}}%
\pgfpathlineto{\pgfqpoint{6.307593in}{2.120504in}}%
\pgfpathlineto{\pgfqpoint{6.310170in}{2.138517in}}%
\pgfpathlineto{\pgfqpoint{6.312747in}{2.142645in}}%
\pgfpathlineto{\pgfqpoint{6.315324in}{2.111498in}}%
\pgfpathlineto{\pgfqpoint{6.317902in}{2.104368in}}%
\pgfpathlineto{\pgfqpoint{6.325633in}{2.112811in}}%
\pgfpathlineto{\pgfqpoint{6.328210in}{2.111685in}}%
\pgfpathlineto{\pgfqpoint{6.330788in}{2.079975in}}%
\pgfpathlineto{\pgfqpoint{6.333365in}{2.080351in}}%
\pgfpathlineto{\pgfqpoint{6.335942in}{2.082039in}}%
\pgfpathlineto{\pgfqpoint{6.343674in}{2.095924in}}%
\pgfpathlineto{\pgfqpoint{6.348828in}{2.107557in}}%
\pgfpathlineto{\pgfqpoint{6.351405in}{2.100615in}}%
\pgfpathlineto{\pgfqpoint{6.353982in}{2.105869in}}%
\pgfpathlineto{\pgfqpoint{6.361714in}{2.102116in}}%
\pgfpathlineto{\pgfqpoint{6.364291in}{2.093297in}}%
\pgfpathlineto{\pgfqpoint{6.366869in}{2.090483in}}%
\pgfpathlineto{\pgfqpoint{6.369446in}{2.102679in}}%
\pgfpathlineto{\pgfqpoint{6.372023in}{2.118440in}}%
\pgfpathlineto{\pgfqpoint{6.379755in}{2.120316in}}%
\pgfpathlineto{\pgfqpoint{6.382332in}{2.114312in}}%
\pgfpathlineto{\pgfqpoint{6.384909in}{2.113749in}}%
\pgfpathlineto{\pgfqpoint{6.387486in}{2.104931in}}%
\pgfpathlineto{\pgfqpoint{6.390063in}{2.045263in}}%
\pgfpathlineto{\pgfqpoint{6.400372in}{2.022559in}}%
\pgfpathlineto{\pgfqpoint{6.402950in}{2.020495in}}%
\pgfpathlineto{\pgfqpoint{6.405527in}{2.031941in}}%
\pgfpathlineto{\pgfqpoint{6.408104in}{2.023685in}}%
\pgfpathlineto{\pgfqpoint{6.415836in}{2.009988in}}%
\pgfpathlineto{\pgfqpoint{6.418413in}{2.011301in}}%
\pgfpathlineto{\pgfqpoint{6.420990in}{2.021058in}}%
\pgfpathlineto{\pgfqpoint{6.423567in}{2.014304in}}%
\pgfpathlineto{\pgfqpoint{6.426144in}{2.015429in}}%
\pgfpathlineto{\pgfqpoint{6.433876in}{2.006048in}}%
\pgfpathlineto{\pgfqpoint{6.436453in}{2.022559in}}%
\pgfpathlineto{\pgfqpoint{6.439031in}{2.033442in}}%
\pgfpathlineto{\pgfqpoint{6.441608in}{2.037007in}}%
\pgfpathlineto{\pgfqpoint{6.444185in}{2.043762in}}%
\pgfpathlineto{\pgfqpoint{6.451917in}{2.058773in}}%
\pgfpathlineto{\pgfqpoint{6.454494in}{2.056333in}}%
\pgfpathlineto{\pgfqpoint{6.457071in}{2.044888in}}%
\pgfpathlineto{\pgfqpoint{6.459648in}{2.063276in}}%
\pgfpathlineto{\pgfqpoint{6.462225in}{2.048453in}}%
\pgfpathlineto{\pgfqpoint{6.469957in}{2.045638in}}%
\pgfpathlineto{\pgfqpoint{6.472534in}{2.053707in}}%
\pgfpathlineto{\pgfqpoint{6.475111in}{2.046764in}}%
\pgfpathlineto{\pgfqpoint{6.480266in}{2.048828in}}%
\pgfpathlineto{\pgfqpoint{6.487998in}{2.058022in}}%
\pgfpathlineto{\pgfqpoint{6.490575in}{2.065903in}}%
\pgfpathlineto{\pgfqpoint{6.493152in}{2.065527in}}%
\pgfpathlineto{\pgfqpoint{6.495729in}{2.076410in}}%
\pgfpathlineto{\pgfqpoint{6.506038in}{2.101741in}}%
\pgfpathlineto{\pgfqpoint{6.508615in}{2.101553in}}%
\pgfpathlineto{\pgfqpoint{6.511192in}{2.098926in}}%
\pgfpathlineto{\pgfqpoint{6.513770in}{2.078287in}}%
\pgfpathlineto{\pgfqpoint{6.516347in}{2.083165in}}%
\pgfpathlineto{\pgfqpoint{6.524079in}{2.080726in}}%
\pgfpathlineto{\pgfqpoint{6.526656in}{2.073971in}}%
\pgfpathlineto{\pgfqpoint{6.529233in}{2.092359in}}%
\pgfpathlineto{\pgfqpoint{6.531810in}{2.094423in}}%
\pgfpathlineto{\pgfqpoint{6.534387in}{2.110372in}}%
\pgfpathlineto{\pgfqpoint{6.542119in}{2.110184in}}%
\pgfpathlineto{\pgfqpoint{6.547273in}{2.103805in}}%
\pgfpathlineto{\pgfqpoint{6.549851in}{2.106432in}}%
\pgfpathlineto{\pgfqpoint{6.552428in}{2.114687in}}%
\pgfpathlineto{\pgfqpoint{6.562737in}{2.120879in}}%
\pgfpathlineto{\pgfqpoint{6.565314in}{2.114125in}}%
\pgfpathlineto{\pgfqpoint{6.567891in}{2.113562in}}%
\pgfpathlineto{\pgfqpoint{6.570468in}{2.110184in}}%
\pgfpathlineto{\pgfqpoint{6.570468in}{2.110184in}}%
\pgfusepath{stroke}%
\end{pgfscope}%
\begin{pgfscope}%
\pgfpathrectangle{\pgfqpoint{0.652287in}{0.521603in}}{\pgfqpoint{6.200000in}{4.530000in}}%
\pgfusepath{clip}%
\pgfsetroundcap%
\pgfsetroundjoin%
\pgfsetlinewidth{1.505625pt}%
\definecolor{currentstroke}{rgb}{0.890196,0.466667,0.760784}%
\pgfsetstrokecolor{currentstroke}%
\pgfsetdash{}{0pt}%
\pgfpathmoveto{\pgfqpoint{0.934105in}{1.643915in}}%
\pgfpathlineto{\pgfqpoint{0.936682in}{1.649919in}}%
\pgfpathlineto{\pgfqpoint{0.939259in}{1.638661in}}%
\pgfpathlineto{\pgfqpoint{0.941836in}{1.631906in}}%
\pgfpathlineto{\pgfqpoint{0.949568in}{1.635096in}}%
\pgfpathlineto{\pgfqpoint{0.952145in}{1.665305in}}%
\pgfpathlineto{\pgfqpoint{0.957300in}{1.684444in}}%
\pgfpathlineto{\pgfqpoint{0.959877in}{1.666243in}}%
\pgfpathlineto{\pgfqpoint{0.970186in}{1.681254in}}%
\pgfpathlineto{\pgfqpoint{0.972763in}{1.690073in}}%
\pgfpathlineto{\pgfqpoint{0.975340in}{1.683693in}}%
\pgfpathlineto{\pgfqpoint{0.977917in}{1.675813in}}%
\pgfpathlineto{\pgfqpoint{0.985649in}{1.678439in}}%
\pgfpathlineto{\pgfqpoint{0.988226in}{1.692887in}}%
\pgfpathlineto{\pgfqpoint{0.990803in}{1.690823in}}%
\pgfpathlineto{\pgfqpoint{0.993381in}{1.687071in}}%
\pgfpathlineto{\pgfqpoint{0.995958in}{1.690448in}}%
\pgfpathlineto{\pgfqpoint{1.003689in}{1.690260in}}%
\pgfpathlineto{\pgfqpoint{1.006267in}{1.701894in}}%
\pgfpathlineto{\pgfqpoint{1.008844in}{1.731352in}}%
\pgfpathlineto{\pgfqpoint{1.011421in}{1.728162in}}%
\pgfpathlineto{\pgfqpoint{1.013998in}{1.744299in}}%
\pgfpathlineto{\pgfqpoint{1.021730in}{1.736793in}}%
\pgfpathlineto{\pgfqpoint{1.024307in}{1.732290in}}%
\pgfpathlineto{\pgfqpoint{1.026884in}{1.755182in}}%
\pgfpathlineto{\pgfqpoint{1.029462in}{1.787267in}}%
\pgfpathlineto{\pgfqpoint{1.032039in}{1.782951in}}%
\pgfpathlineto{\pgfqpoint{1.039770in}{1.804529in}}%
\pgfpathlineto{\pgfqpoint{1.042348in}{1.800777in}}%
\pgfpathlineto{\pgfqpoint{1.044925in}{1.776197in}}%
\pgfpathlineto{\pgfqpoint{1.050079in}{1.794772in}}%
\pgfpathlineto{\pgfqpoint{1.060388in}{1.797587in}}%
\pgfpathlineto{\pgfqpoint{1.062965in}{1.797587in}}%
\pgfpathlineto{\pgfqpoint{1.065543in}{1.790269in}}%
\pgfpathlineto{\pgfqpoint{1.068120in}{1.797774in}}%
\pgfpathlineto{\pgfqpoint{1.075851in}{1.792521in}}%
\pgfpathlineto{\pgfqpoint{1.078429in}{1.789894in}}%
\pgfpathlineto{\pgfqpoint{1.081006in}{1.796273in}}%
\pgfpathlineto{\pgfqpoint{1.083583in}{1.800777in}}%
\pgfpathlineto{\pgfqpoint{1.086160in}{1.806781in}}%
\pgfpathlineto{\pgfqpoint{1.093892in}{1.787079in}}%
\pgfpathlineto{\pgfqpoint{1.096469in}{1.757058in}}%
\pgfpathlineto{\pgfqpoint{1.101624in}{1.790832in}}%
\pgfpathlineto{\pgfqpoint{1.104201in}{1.791207in}}%
\pgfpathlineto{\pgfqpoint{1.111932in}{1.798150in}}%
\pgfpathlineto{\pgfqpoint{1.114510in}{1.838866in}}%
\pgfpathlineto{\pgfqpoint{1.117087in}{1.842806in}}%
\pgfpathlineto{\pgfqpoint{1.119664in}{1.843932in}}%
\pgfpathlineto{\pgfqpoint{1.122241in}{1.821791in}}%
\pgfpathlineto{\pgfqpoint{1.129973in}{1.808094in}}%
\pgfpathlineto{\pgfqpoint{1.132550in}{1.786516in}}%
\pgfpathlineto{\pgfqpoint{1.137704in}{1.767753in}}%
\pgfpathlineto{\pgfqpoint{1.140282in}{1.763437in}}%
\pgfpathlineto{\pgfqpoint{1.148013in}{1.790457in}}%
\pgfpathlineto{\pgfqpoint{1.150591in}{1.784265in}}%
\pgfpathlineto{\pgfqpoint{1.153168in}{1.759122in}}%
\pgfpathlineto{\pgfqpoint{1.155745in}{1.771693in}}%
\pgfpathlineto{\pgfqpoint{1.158322in}{1.781450in}}%
\pgfpathlineto{\pgfqpoint{1.166054in}{1.778260in}}%
\pgfpathlineto{\pgfqpoint{1.168631in}{1.769817in}}%
\pgfpathlineto{\pgfqpoint{1.173785in}{1.760248in}}%
\pgfpathlineto{\pgfqpoint{1.184094in}{1.738670in}}%
\pgfpathlineto{\pgfqpoint{1.186672in}{1.710900in}}%
\pgfpathlineto{\pgfqpoint{1.189249in}{1.729100in}}%
\pgfpathlineto{\pgfqpoint{1.191826in}{1.753493in}}%
\pgfpathlineto{\pgfqpoint{1.194403in}{1.731915in}}%
\pgfpathlineto{\pgfqpoint{1.202135in}{1.733041in}}%
\pgfpathlineto{\pgfqpoint{1.204712in}{1.754619in}}%
\pgfpathlineto{\pgfqpoint{1.207289in}{1.754056in}}%
\pgfpathlineto{\pgfqpoint{1.209866in}{1.741297in}}%
\pgfpathlineto{\pgfqpoint{1.212444in}{1.750866in}}%
\pgfpathlineto{\pgfqpoint{1.220175in}{1.730977in}}%
\pgfpathlineto{\pgfqpoint{1.222753in}{1.732666in}}%
\pgfpathlineto{\pgfqpoint{1.225330in}{1.732103in}}%
\pgfpathlineto{\pgfqpoint{1.227907in}{1.754806in}}%
\pgfpathlineto{\pgfqpoint{1.230484in}{1.766627in}}%
\pgfpathlineto{\pgfqpoint{1.238216in}{1.760998in}}%
\pgfpathlineto{\pgfqpoint{1.240793in}{1.757621in}}%
\pgfpathlineto{\pgfqpoint{1.243370in}{1.759497in}}%
\pgfpathlineto{\pgfqpoint{1.245947in}{1.749365in}}%
\pgfpathlineto{\pgfqpoint{1.248525in}{1.726849in}}%
\pgfpathlineto{\pgfqpoint{1.256256in}{1.717842in}}%
\pgfpathlineto{\pgfqpoint{1.258833in}{1.716341in}}%
\pgfpathlineto{\pgfqpoint{1.261411in}{1.687633in}}%
\pgfpathlineto{\pgfqpoint{1.263988in}{1.692137in}}%
\pgfpathlineto{\pgfqpoint{1.266565in}{1.690448in}}%
\pgfpathlineto{\pgfqpoint{1.274297in}{1.670934in}}%
\pgfpathlineto{\pgfqpoint{1.276874in}{1.674499in}}%
\pgfpathlineto{\pgfqpoint{1.279451in}{1.658738in}}%
\pgfpathlineto{\pgfqpoint{1.282028in}{1.636785in}}%
\pgfpathlineto{\pgfqpoint{1.284606in}{1.621774in}}%
\pgfpathlineto{\pgfqpoint{1.292337in}{1.642789in}}%
\pgfpathlineto{\pgfqpoint{1.294914in}{1.646542in}}%
\pgfpathlineto{\pgfqpoint{1.297492in}{1.648418in}}%
\pgfpathlineto{\pgfqpoint{1.302646in}{1.631906in}}%
\pgfpathlineto{\pgfqpoint{1.312955in}{1.664930in}}%
\pgfpathlineto{\pgfqpoint{1.315532in}{1.643352in}}%
\pgfpathlineto{\pgfqpoint{1.318109in}{1.649356in}}%
\pgfpathlineto{\pgfqpoint{1.320687in}{1.615957in}}%
\pgfpathlineto{\pgfqpoint{1.328418in}{1.606951in}}%
\pgfpathlineto{\pgfqpoint{1.330995in}{1.597945in}}%
\pgfpathlineto{\pgfqpoint{1.333573in}{1.641851in}}%
\pgfpathlineto{\pgfqpoint{1.336150in}{1.669808in}}%
\pgfpathlineto{\pgfqpoint{1.338727in}{1.671497in}}%
\pgfpathlineto{\pgfqpoint{1.349036in}{1.653109in}}%
\pgfpathlineto{\pgfqpoint{1.351613in}{1.640162in}}%
\pgfpathlineto{\pgfqpoint{1.354190in}{1.649731in}}%
\pgfpathlineto{\pgfqpoint{1.356768in}{1.655360in}}%
\pgfpathlineto{\pgfqpoint{1.364499in}{1.663429in}}%
\pgfpathlineto{\pgfqpoint{1.367076in}{1.685757in}}%
\pgfpathlineto{\pgfqpoint{1.369654in}{1.677877in}}%
\pgfpathlineto{\pgfqpoint{1.372231in}{1.660614in}}%
\pgfpathlineto{\pgfqpoint{1.374808in}{1.666618in}}%
\pgfpathlineto{\pgfqpoint{1.382540in}{1.643540in}}%
\pgfpathlineto{\pgfqpoint{1.385117in}{1.641663in}}%
\pgfpathlineto{\pgfqpoint{1.387694in}{1.650107in}}%
\pgfpathlineto{\pgfqpoint{1.390271in}{1.625151in}}%
\pgfpathlineto{\pgfqpoint{1.392849in}{1.671872in}}%
\pgfpathlineto{\pgfqpoint{1.400580in}{1.664179in}}%
\pgfpathlineto{\pgfqpoint{1.403157in}{1.675437in}}%
\pgfpathlineto{\pgfqpoint{1.408312in}{1.669621in}}%
\pgfpathlineto{\pgfqpoint{1.410889in}{1.648981in}}%
\pgfpathlineto{\pgfqpoint{1.418621in}{1.652734in}}%
\pgfpathlineto{\pgfqpoint{1.421198in}{1.649919in}}%
\pgfpathlineto{\pgfqpoint{1.423775in}{1.623838in}}%
\pgfpathlineto{\pgfqpoint{1.426352in}{1.613143in}}%
\pgfpathlineto{\pgfqpoint{1.428930in}{1.640913in}}%
\pgfpathlineto{\pgfqpoint{1.436661in}{1.634908in}}%
\pgfpathlineto{\pgfqpoint{1.439238in}{1.641100in}}%
\pgfpathlineto{\pgfqpoint{1.444393in}{1.676563in}}%
\pgfpathlineto{\pgfqpoint{1.446970in}{1.651233in}}%
\pgfpathlineto{\pgfqpoint{1.454702in}{1.636034in}}%
\pgfpathlineto{\pgfqpoint{1.457279in}{1.614644in}}%
\pgfpathlineto{\pgfqpoint{1.459856in}{1.625339in}}%
\pgfpathlineto{\pgfqpoint{1.462433in}{1.630405in}}%
\pgfpathlineto{\pgfqpoint{1.465010in}{1.651983in}}%
\pgfpathlineto{\pgfqpoint{1.472742in}{1.663053in}}%
\pgfpathlineto{\pgfqpoint{1.475319in}{1.654610in}}%
\pgfpathlineto{\pgfqpoint{1.477897in}{1.660051in}}%
\pgfpathlineto{\pgfqpoint{1.480474in}{1.654610in}}%
\pgfpathlineto{\pgfqpoint{1.483051in}{1.692324in}}%
\pgfpathlineto{\pgfqpoint{1.490783in}{1.689697in}}%
\pgfpathlineto{\pgfqpoint{1.493360in}{1.713152in}}%
\pgfpathlineto{\pgfqpoint{1.495937in}{1.703019in}}%
\pgfpathlineto{\pgfqpoint{1.498514in}{1.695702in}}%
\pgfpathlineto{\pgfqpoint{1.501091in}{1.709399in}}%
\pgfpathlineto{\pgfqpoint{1.508823in}{1.702832in}}%
\pgfpathlineto{\pgfqpoint{1.511400in}{1.707710in}}%
\pgfpathlineto{\pgfqpoint{1.516555in}{1.732103in}}%
\pgfpathlineto{\pgfqpoint{1.519132in}{1.757808in}}%
\pgfpathlineto{\pgfqpoint{1.526864in}{1.753118in}}%
\pgfpathlineto{\pgfqpoint{1.529441in}{1.740171in}}%
\pgfpathlineto{\pgfqpoint{1.532018in}{1.746363in}}%
\pgfpathlineto{\pgfqpoint{1.534595in}{1.739045in}}%
\pgfpathlineto{\pgfqpoint{1.537172in}{1.753118in}}%
\pgfpathlineto{\pgfqpoint{1.544904in}{1.760811in}}%
\pgfpathlineto{\pgfqpoint{1.547481in}{1.762124in}}%
\pgfpathlineto{\pgfqpoint{1.550058in}{1.754619in}}%
\pgfpathlineto{\pgfqpoint{1.552636in}{1.736793in}}%
\pgfpathlineto{\pgfqpoint{1.555213in}{1.749553in}}%
\pgfpathlineto{\pgfqpoint{1.568099in}{1.720282in}}%
\pgfpathlineto{\pgfqpoint{1.570676in}{1.742235in}}%
\pgfpathlineto{\pgfqpoint{1.573253in}{1.742422in}}%
\pgfpathlineto{\pgfqpoint{1.580985in}{1.729663in}}%
\pgfpathlineto{\pgfqpoint{1.583562in}{1.731915in}}%
\pgfpathlineto{\pgfqpoint{1.586139in}{1.733041in}}%
\pgfpathlineto{\pgfqpoint{1.591294in}{1.791207in}}%
\pgfpathlineto{\pgfqpoint{1.599026in}{1.786329in}}%
\pgfpathlineto{\pgfqpoint{1.601603in}{1.775634in}}%
\pgfpathlineto{\pgfqpoint{1.604180in}{1.779386in}}%
\pgfpathlineto{\pgfqpoint{1.606757in}{1.766627in}}%
\pgfpathlineto{\pgfqpoint{1.609334in}{1.763813in}}%
\pgfpathlineto{\pgfqpoint{1.617066in}{1.752367in}}%
\pgfpathlineto{\pgfqpoint{1.619643in}{1.732103in}}%
\pgfpathlineto{\pgfqpoint{1.624798in}{1.725348in}}%
\pgfpathlineto{\pgfqpoint{1.627375in}{1.724410in}}%
\pgfpathlineto{\pgfqpoint{1.637684in}{1.726098in}}%
\pgfpathlineto{\pgfqpoint{1.640261in}{1.728725in}}%
\pgfpathlineto{\pgfqpoint{1.645415in}{1.727975in}}%
\pgfpathlineto{\pgfqpoint{1.653147in}{1.725723in}}%
\pgfpathlineto{\pgfqpoint{1.655724in}{1.709024in}}%
\pgfpathlineto{\pgfqpoint{1.658301in}{1.688009in}}%
\pgfpathlineto{\pgfqpoint{1.660879in}{1.688572in}}%
\pgfpathlineto{\pgfqpoint{1.663456in}{1.687071in}}%
\pgfpathlineto{\pgfqpoint{1.671187in}{1.690073in}}%
\pgfpathlineto{\pgfqpoint{1.673765in}{1.713339in}}%
\pgfpathlineto{\pgfqpoint{1.676342in}{1.728538in}}%
\pgfpathlineto{\pgfqpoint{1.678919in}{1.739608in}}%
\pgfpathlineto{\pgfqpoint{1.681496in}{1.719719in}}%
\pgfpathlineto{\pgfqpoint{1.689228in}{1.717092in}}%
\pgfpathlineto{\pgfqpoint{1.691805in}{1.704896in}}%
\pgfpathlineto{\pgfqpoint{1.694382in}{1.718218in}}%
\pgfpathlineto{\pgfqpoint{1.696960in}{1.708461in}}%
\pgfpathlineto{\pgfqpoint{1.699537in}{1.723096in}}%
\pgfpathlineto{\pgfqpoint{1.712423in}{1.722346in}}%
\pgfpathlineto{\pgfqpoint{1.715000in}{1.736793in}}%
\pgfpathlineto{\pgfqpoint{1.717577in}{1.720845in}}%
\pgfpathlineto{\pgfqpoint{1.725309in}{1.718218in}}%
\pgfpathlineto{\pgfqpoint{1.727886in}{1.751429in}}%
\pgfpathlineto{\pgfqpoint{1.730463in}{1.714653in}}%
\pgfpathlineto{\pgfqpoint{1.733041in}{1.690260in}}%
\pgfpathlineto{\pgfqpoint{1.735618in}{1.685194in}}%
\pgfpathlineto{\pgfqpoint{1.743349in}{1.703019in}}%
\pgfpathlineto{\pgfqpoint{1.745927in}{1.703770in}}%
\pgfpathlineto{\pgfqpoint{1.748504in}{1.674499in}}%
\pgfpathlineto{\pgfqpoint{1.751081in}{1.677501in}}%
\pgfpathlineto{\pgfqpoint{1.761390in}{1.705646in}}%
\pgfpathlineto{\pgfqpoint{1.763967in}{1.707335in}}%
\pgfpathlineto{\pgfqpoint{1.771699in}{1.738482in}}%
\pgfpathlineto{\pgfqpoint{1.779430in}{1.739608in}}%
\pgfpathlineto{\pgfqpoint{1.782008in}{1.741860in}}%
\pgfpathlineto{\pgfqpoint{1.784585in}{1.757621in}}%
\pgfpathlineto{\pgfqpoint{1.787162in}{1.757058in}}%
\pgfpathlineto{\pgfqpoint{1.789739in}{1.762687in}}%
\pgfpathlineto{\pgfqpoint{1.797471in}{1.757621in}}%
\pgfpathlineto{\pgfqpoint{1.800048in}{1.763062in}}%
\pgfpathlineto{\pgfqpoint{1.802625in}{1.764563in}}%
\pgfpathlineto{\pgfqpoint{1.805203in}{1.774883in}}%
\pgfpathlineto{\pgfqpoint{1.807780in}{1.776759in}}%
\pgfpathlineto{\pgfqpoint{1.815511in}{1.777135in}}%
\pgfpathlineto{\pgfqpoint{1.818089in}{1.780137in}}%
\pgfpathlineto{\pgfqpoint{1.820666in}{1.776947in}}%
\pgfpathlineto{\pgfqpoint{1.823243in}{1.766815in}}%
\pgfpathlineto{\pgfqpoint{1.825820in}{1.760435in}}%
\pgfpathlineto{\pgfqpoint{1.833552in}{1.760623in}}%
\pgfpathlineto{\pgfqpoint{1.836129in}{1.796836in}}%
\pgfpathlineto{\pgfqpoint{1.838706in}{1.809783in}}%
\pgfpathlineto{\pgfqpoint{1.841283in}{1.812035in}}%
\pgfpathlineto{\pgfqpoint{1.843861in}{1.801902in}}%
\pgfpathlineto{\pgfqpoint{1.851592in}{1.797024in}}%
\pgfpathlineto{\pgfqpoint{1.859324in}{1.794209in}}%
\pgfpathlineto{\pgfqpoint{1.861901in}{1.773945in}}%
\pgfpathlineto{\pgfqpoint{1.869633in}{1.793271in}}%
\pgfpathlineto{\pgfqpoint{1.874787in}{1.825357in}}%
\pgfpathlineto{\pgfqpoint{1.877364in}{1.830423in}}%
\pgfpathlineto{\pgfqpoint{1.879942in}{1.841305in}}%
\pgfpathlineto{\pgfqpoint{1.887673in}{1.834551in}}%
\pgfpathlineto{\pgfqpoint{1.890251in}{1.818226in}}%
\pgfpathlineto{\pgfqpoint{1.892828in}{1.834363in}}%
\pgfpathlineto{\pgfqpoint{1.895405in}{1.841118in}}%
\pgfpathlineto{\pgfqpoint{1.905714in}{1.854440in}}%
\pgfpathlineto{\pgfqpoint{1.908291in}{1.857067in}}%
\pgfpathlineto{\pgfqpoint{1.910868in}{1.850687in}}%
\pgfpathlineto{\pgfqpoint{1.913445in}{1.864197in}}%
\pgfpathlineto{\pgfqpoint{1.916023in}{1.873015in}}%
\pgfpathlineto{\pgfqpoint{1.926332in}{1.881459in}}%
\pgfpathlineto{\pgfqpoint{1.928909in}{1.891028in}}%
\pgfpathlineto{\pgfqpoint{1.934063in}{1.918798in}}%
\pgfpathlineto{\pgfqpoint{1.941795in}{1.918986in}}%
\pgfpathlineto{\pgfqpoint{1.944372in}{1.918423in}}%
\pgfpathlineto{\pgfqpoint{1.946949in}{1.907728in}}%
\pgfpathlineto{\pgfqpoint{1.949526in}{1.882960in}}%
\pgfpathlineto{\pgfqpoint{1.952104in}{1.919736in}}%
\pgfpathlineto{\pgfqpoint{1.959835in}{1.914858in}}%
\pgfpathlineto{\pgfqpoint{1.962412in}{1.910355in}}%
\pgfpathlineto{\pgfqpoint{1.964990in}{1.912043in}}%
\pgfpathlineto{\pgfqpoint{1.967567in}{1.921425in}}%
\pgfpathlineto{\pgfqpoint{1.970144in}{1.923676in}}%
\pgfpathlineto{\pgfqpoint{1.977876in}{1.915984in}}%
\pgfpathlineto{\pgfqpoint{1.980453in}{1.921988in}}%
\pgfpathlineto{\pgfqpoint{1.983030in}{1.922738in}}%
\pgfpathlineto{\pgfqpoint{1.985607in}{1.925928in}}%
\pgfpathlineto{\pgfqpoint{1.988185in}{1.943566in}}%
\pgfpathlineto{\pgfqpoint{1.998493in}{1.947506in}}%
\pgfpathlineto{\pgfqpoint{2.003648in}{1.921425in}}%
\pgfpathlineto{\pgfqpoint{2.006225in}{1.938875in}}%
\pgfpathlineto{\pgfqpoint{2.013957in}{1.904538in}}%
\pgfpathlineto{\pgfqpoint{2.016534in}{1.917860in}}%
\pgfpathlineto{\pgfqpoint{2.019111in}{1.940188in}}%
\pgfpathlineto{\pgfqpoint{2.021688in}{1.939813in}}%
\pgfpathlineto{\pgfqpoint{2.024266in}{1.933058in}}%
\pgfpathlineto{\pgfqpoint{2.031997in}{1.916734in}}%
\pgfpathlineto{\pgfqpoint{2.034574in}{1.947506in}}%
\pgfpathlineto{\pgfqpoint{2.037152in}{1.948256in}}%
\pgfpathlineto{\pgfqpoint{2.039729in}{1.958764in}}%
\pgfpathlineto{\pgfqpoint{2.042306in}{1.964581in}}%
\pgfpathlineto{\pgfqpoint{2.052615in}{1.981843in}}%
\pgfpathlineto{\pgfqpoint{2.055192in}{1.980905in}}%
\pgfpathlineto{\pgfqpoint{2.057769in}{1.986909in}}%
\pgfpathlineto{\pgfqpoint{2.068078in}{1.975839in}}%
\pgfpathlineto{\pgfqpoint{2.073233in}{1.986909in}}%
\pgfpathlineto{\pgfqpoint{2.075810in}{1.970397in}}%
\pgfpathlineto{\pgfqpoint{2.078387in}{1.989161in}}%
\pgfpathlineto{\pgfqpoint{2.086119in}{1.976402in}}%
\pgfpathlineto{\pgfqpoint{2.088696in}{1.973962in}}%
\pgfpathlineto{\pgfqpoint{2.091273in}{1.973399in}}%
\pgfpathlineto{\pgfqpoint{2.093850in}{1.986721in}}%
\pgfpathlineto{\pgfqpoint{2.104159in}{1.978278in}}%
\pgfpathlineto{\pgfqpoint{2.106736in}{1.979404in}}%
\pgfpathlineto{\pgfqpoint{2.109314in}{1.982969in}}%
\pgfpathlineto{\pgfqpoint{2.111891in}{1.982218in}}%
\pgfpathlineto{\pgfqpoint{2.114468in}{1.973775in}}%
\pgfpathlineto{\pgfqpoint{2.122200in}{1.991225in}}%
\pgfpathlineto{\pgfqpoint{2.129931in}{2.025374in}}%
\pgfpathlineto{\pgfqpoint{2.132509in}{2.023310in}}%
\pgfpathlineto{\pgfqpoint{2.140240in}{1.988035in}}%
\pgfpathlineto{\pgfqpoint{2.142817in}{2.004171in}}%
\pgfpathlineto{\pgfqpoint{2.145395in}{1.977152in}}%
\pgfpathlineto{\pgfqpoint{2.147972in}{1.956137in}}%
\pgfpathlineto{\pgfqpoint{2.150549in}{1.983156in}}%
\pgfpathlineto{\pgfqpoint{2.158281in}{1.989911in}}%
\pgfpathlineto{\pgfqpoint{2.163435in}{1.963455in}}%
\pgfpathlineto{\pgfqpoint{2.166012in}{1.965144in}}%
\pgfpathlineto{\pgfqpoint{2.168589in}{1.949570in}}%
\pgfpathlineto{\pgfqpoint{2.176321in}{1.957263in}}%
\pgfpathlineto{\pgfqpoint{2.178898in}{1.951822in}}%
\pgfpathlineto{\pgfqpoint{2.181476in}{1.948069in}}%
\pgfpathlineto{\pgfqpoint{2.184053in}{1.959327in}}%
\pgfpathlineto{\pgfqpoint{2.186630in}{1.981468in}}%
\pgfpathlineto{\pgfqpoint{2.194362in}{1.987660in}}%
\pgfpathlineto{\pgfqpoint{2.196939in}{1.993851in}}%
\pgfpathlineto{\pgfqpoint{2.199516in}{2.002107in}}%
\pgfpathlineto{\pgfqpoint{2.202093in}{2.008299in}}%
\pgfpathlineto{\pgfqpoint{2.204670in}{2.016180in}}%
\pgfpathlineto{\pgfqpoint{2.212402in}{2.011301in}}%
\pgfpathlineto{\pgfqpoint{2.214979in}{2.027250in}}%
\pgfpathlineto{\pgfqpoint{2.217557in}{2.034005in}}%
\pgfpathlineto{\pgfqpoint{2.220134in}{2.023498in}}%
\pgfpathlineto{\pgfqpoint{2.222711in}{2.059148in}}%
\pgfpathlineto{\pgfqpoint{2.230443in}{2.057272in}}%
\pgfpathlineto{\pgfqpoint{2.233020in}{2.062338in}}%
\pgfpathlineto{\pgfqpoint{2.235597in}{2.041135in}}%
\pgfpathlineto{\pgfqpoint{2.238174in}{2.028751in}}%
\pgfpathlineto{\pgfqpoint{2.240751in}{2.021434in}}%
\pgfpathlineto{\pgfqpoint{2.251060in}{2.036632in}}%
\pgfpathlineto{\pgfqpoint{2.253637in}{2.022747in}}%
\pgfpathlineto{\pgfqpoint{2.256215in}{2.036444in}}%
\pgfpathlineto{\pgfqpoint{2.258792in}{2.019182in}}%
\pgfpathlineto{\pgfqpoint{2.266524in}{2.024623in}}%
\pgfpathlineto{\pgfqpoint{2.269101in}{2.009050in}}%
\pgfpathlineto{\pgfqpoint{2.271678in}{1.988222in}}%
\pgfpathlineto{\pgfqpoint{2.274255in}{1.984095in}}%
\pgfpathlineto{\pgfqpoint{2.276832in}{2.012615in}}%
\pgfpathlineto{\pgfqpoint{2.284564in}{2.006798in}}%
\pgfpathlineto{\pgfqpoint{2.287141in}{1.999293in}}%
\pgfpathlineto{\pgfqpoint{2.289718in}{1.982969in}}%
\pgfpathlineto{\pgfqpoint{2.292296in}{2.009237in}}%
\pgfpathlineto{\pgfqpoint{2.294873in}{2.004734in}}%
\pgfpathlineto{\pgfqpoint{2.302605in}{2.020495in}}%
\pgfpathlineto{\pgfqpoint{2.305182in}{2.039822in}}%
\pgfpathlineto{\pgfqpoint{2.310336in}{1.977527in}}%
\pgfpathlineto{\pgfqpoint{2.312913in}{1.974713in}}%
\pgfpathlineto{\pgfqpoint{2.320645in}{1.963455in}}%
\pgfpathlineto{\pgfqpoint{2.323222in}{1.968146in}}%
\pgfpathlineto{\pgfqpoint{2.325799in}{1.987847in}}%
\pgfpathlineto{\pgfqpoint{2.328377in}{1.996478in}}%
\pgfpathlineto{\pgfqpoint{2.330954in}{1.987097in}}%
\pgfpathlineto{\pgfqpoint{2.338685in}{2.016555in}}%
\pgfpathlineto{\pgfqpoint{2.341263in}{2.001169in}}%
\pgfpathlineto{\pgfqpoint{2.348994in}{2.045826in}}%
\pgfpathlineto{\pgfqpoint{2.356726in}{2.053144in}}%
\pgfpathlineto{\pgfqpoint{2.359303in}{2.070031in}}%
\pgfpathlineto{\pgfqpoint{2.361880in}{2.066278in}}%
\pgfpathlineto{\pgfqpoint{2.364458in}{2.097425in}}%
\pgfpathlineto{\pgfqpoint{2.374766in}{2.106432in}}%
\pgfpathlineto{\pgfqpoint{2.377344in}{2.102304in}}%
\pgfpathlineto{\pgfqpoint{2.379921in}{2.116564in}}%
\pgfpathlineto{\pgfqpoint{2.382498in}{2.124257in}}%
\pgfpathlineto{\pgfqpoint{2.385075in}{2.142833in}}%
\pgfpathlineto{\pgfqpoint{2.392807in}{2.136828in}}%
\pgfpathlineto{\pgfqpoint{2.395384in}{2.185988in}}%
\pgfpathlineto{\pgfqpoint{2.397961in}{2.183924in}}%
\pgfpathlineto{\pgfqpoint{2.400539in}{2.180735in}}%
\pgfpathlineto{\pgfqpoint{2.403116in}{2.183549in}}%
\pgfpathlineto{\pgfqpoint{2.410847in}{2.185238in}}%
\pgfpathlineto{\pgfqpoint{2.413425in}{2.193306in}}%
\pgfpathlineto{\pgfqpoint{2.416002in}{2.193306in}}%
\pgfpathlineto{\pgfqpoint{2.418579in}{2.219762in}}%
\pgfpathlineto{\pgfqpoint{2.421156in}{2.229332in}}%
\pgfpathlineto{\pgfqpoint{2.428888in}{2.210944in}}%
\pgfpathlineto{\pgfqpoint{2.431465in}{2.187302in}}%
\pgfpathlineto{\pgfqpoint{2.434042in}{2.200811in}}%
\pgfpathlineto{\pgfqpoint{2.436620in}{2.204564in}}%
\pgfpathlineto{\pgfqpoint{2.439197in}{2.194244in}}%
\pgfpathlineto{\pgfqpoint{2.446928in}{2.192931in}}%
\pgfpathlineto{\pgfqpoint{2.449506in}{2.213383in}}%
\pgfpathlineto{\pgfqpoint{2.452083in}{2.193681in}}%
\pgfpathlineto{\pgfqpoint{2.454660in}{2.159720in}}%
\pgfpathlineto{\pgfqpoint{2.457237in}{2.161221in}}%
\pgfpathlineto{\pgfqpoint{2.464969in}{2.153903in}}%
\pgfpathlineto{\pgfqpoint{2.467546in}{2.149963in}}%
\pgfpathlineto{\pgfqpoint{2.470123in}{2.138517in}}%
\pgfpathlineto{\pgfqpoint{2.472701in}{2.159157in}}%
\pgfpathlineto{\pgfqpoint{2.483009in}{2.148462in}}%
\pgfpathlineto{\pgfqpoint{2.485587in}{2.109621in}}%
\pgfpathlineto{\pgfqpoint{2.488164in}{2.110372in}}%
\pgfpathlineto{\pgfqpoint{2.490741in}{2.118253in}}%
\pgfpathlineto{\pgfqpoint{2.493318in}{2.112248in}}%
\pgfpathlineto{\pgfqpoint{2.506204in}{2.165724in}}%
\pgfpathlineto{\pgfqpoint{2.508782in}{2.170978in}}%
\pgfpathlineto{\pgfqpoint{2.511359in}{2.163660in}}%
\pgfpathlineto{\pgfqpoint{2.519090in}{2.184112in}}%
\pgfpathlineto{\pgfqpoint{2.524245in}{2.239089in}}%
\pgfpathlineto{\pgfqpoint{2.526822in}{2.239089in}}%
\pgfpathlineto{\pgfqpoint{2.529399in}{2.248283in}}%
\pgfpathlineto{\pgfqpoint{2.537131in}{2.269860in}}%
\pgfpathlineto{\pgfqpoint{2.542285in}{2.291251in}}%
\pgfpathlineto{\pgfqpoint{2.544862in}{2.307575in}}%
\pgfpathlineto{\pgfqpoint{2.547440in}{2.267796in}}%
\pgfpathlineto{\pgfqpoint{2.555171in}{2.265170in}}%
\pgfpathlineto{\pgfqpoint{2.557749in}{2.274739in}}%
\pgfpathlineto{\pgfqpoint{2.560326in}{2.262543in}}%
\pgfpathlineto{\pgfqpoint{2.562903in}{2.269110in}}%
\pgfpathlineto{\pgfqpoint{2.565480in}{2.264231in}}%
\pgfpathlineto{\pgfqpoint{2.575789in}{2.231771in}}%
\pgfpathlineto{\pgfqpoint{2.578366in}{2.192368in}}%
\pgfpathlineto{\pgfqpoint{2.580943in}{2.171165in}}%
\pgfpathlineto{\pgfqpoint{2.583521in}{2.180735in}}%
\pgfpathlineto{\pgfqpoint{2.591252in}{2.176794in}}%
\pgfpathlineto{\pgfqpoint{2.593830in}{2.155967in}}%
\pgfpathlineto{\pgfqpoint{2.596407in}{2.157280in}}%
\pgfpathlineto{\pgfqpoint{2.598984in}{2.207754in}}%
\pgfpathlineto{\pgfqpoint{2.601561in}{2.225767in}}%
\pgfpathlineto{\pgfqpoint{2.609293in}{2.223890in}}%
\pgfpathlineto{\pgfqpoint{2.611870in}{2.205877in}}%
\pgfpathlineto{\pgfqpoint{2.614447in}{2.216197in}}%
\pgfpathlineto{\pgfqpoint{2.617024in}{2.242278in}}%
\pgfpathlineto{\pgfqpoint{2.619602in}{2.237587in}}%
\pgfpathlineto{\pgfqpoint{2.627333in}{2.235711in}}%
\pgfpathlineto{\pgfqpoint{2.629911in}{2.211319in}}%
\pgfpathlineto{\pgfqpoint{2.632488in}{2.215447in}}%
\pgfpathlineto{\pgfqpoint{2.635065in}{2.226517in}}%
\pgfpathlineto{\pgfqpoint{2.637642in}{2.234022in}}%
\pgfpathlineto{\pgfqpoint{2.645374in}{2.208880in}}%
\pgfpathlineto{\pgfqpoint{2.647951in}{2.216010in}}%
\pgfpathlineto{\pgfqpoint{2.650528in}{2.208317in}}%
\pgfpathlineto{\pgfqpoint{2.653105in}{2.213195in}}%
\pgfpathlineto{\pgfqpoint{2.655683in}{2.233835in}}%
\pgfpathlineto{\pgfqpoint{2.663414in}{2.241528in}}%
\pgfpathlineto{\pgfqpoint{2.665991in}{2.236649in}}%
\pgfpathlineto{\pgfqpoint{2.668569in}{2.252973in}}%
\pgfpathlineto{\pgfqpoint{2.671146in}{2.229895in}}%
\pgfpathlineto{\pgfqpoint{2.673723in}{2.250909in}}%
\pgfpathlineto{\pgfqpoint{2.681455in}{2.243216in}}%
\pgfpathlineto{\pgfqpoint{2.684032in}{2.232897in}}%
\pgfpathlineto{\pgfqpoint{2.686609in}{2.242841in}}%
\pgfpathlineto{\pgfqpoint{2.689186in}{2.263106in}}%
\pgfpathlineto{\pgfqpoint{2.691764in}{2.261417in}}%
\pgfpathlineto{\pgfqpoint{2.699495in}{2.270423in}}%
\pgfpathlineto{\pgfqpoint{2.702072in}{2.269860in}}%
\pgfpathlineto{\pgfqpoint{2.704650in}{2.265357in}}%
\pgfpathlineto{\pgfqpoint{2.707227in}{2.280931in}}%
\pgfpathlineto{\pgfqpoint{2.709804in}{2.288061in}}%
\pgfpathlineto{\pgfqpoint{2.717536in}{2.289937in}}%
\pgfpathlineto{\pgfqpoint{2.722690in}{2.309639in}}%
\pgfpathlineto{\pgfqpoint{2.727845in}{2.298944in}}%
\pgfpathlineto{\pgfqpoint{2.735576in}{2.289937in}}%
\pgfpathlineto{\pgfqpoint{2.740731in}{2.267234in}}%
\pgfpathlineto{\pgfqpoint{2.743308in}{2.269110in}}%
\pgfpathlineto{\pgfqpoint{2.745885in}{2.303259in}}%
\pgfpathlineto{\pgfqpoint{2.753617in}{2.304197in}}%
\pgfpathlineto{\pgfqpoint{2.756194in}{2.300820in}}%
\pgfpathlineto{\pgfqpoint{2.758771in}{2.262543in}}%
\pgfpathlineto{\pgfqpoint{2.761348in}{2.253161in}}%
\pgfpathlineto{\pgfqpoint{2.763926in}{2.240965in}}%
\pgfpathlineto{\pgfqpoint{2.771657in}{2.260479in}}%
\pgfpathlineto{\pgfqpoint{2.774234in}{2.245468in}}%
\pgfpathlineto{\pgfqpoint{2.776812in}{2.281681in}}%
\pgfpathlineto{\pgfqpoint{2.779389in}{2.276428in}}%
\pgfpathlineto{\pgfqpoint{2.781966in}{2.295566in}}%
\pgfpathlineto{\pgfqpoint{2.789698in}{2.298193in}}%
\pgfpathlineto{\pgfqpoint{2.797429in}{2.329153in}}%
\pgfpathlineto{\pgfqpoint{2.800007in}{2.331029in}}%
\pgfpathlineto{\pgfqpoint{2.807738in}{2.329716in}}%
\pgfpathlineto{\pgfqpoint{2.810315in}{2.347353in}}%
\pgfpathlineto{\pgfqpoint{2.815470in}{2.325775in}}%
\pgfpathlineto{\pgfqpoint{2.818047in}{2.332530in}}%
\pgfpathlineto{\pgfqpoint{2.825779in}{2.330841in}}%
\pgfpathlineto{\pgfqpoint{2.828356in}{2.342662in}}%
\pgfpathlineto{\pgfqpoint{2.830933in}{2.344914in}}%
\pgfpathlineto{\pgfqpoint{2.833510in}{2.346040in}}%
\pgfpathlineto{\pgfqpoint{2.836088in}{2.347916in}}%
\pgfpathlineto{\pgfqpoint{2.843819in}{2.331029in}}%
\pgfpathlineto{\pgfqpoint{2.846396in}{2.328214in}}%
\pgfpathlineto{\pgfqpoint{2.848974in}{2.351856in}}%
\pgfpathlineto{\pgfqpoint{2.851551in}{2.354296in}}%
\pgfpathlineto{\pgfqpoint{2.854128in}{2.354296in}}%
\pgfpathlineto{\pgfqpoint{2.864437in}{2.367055in}}%
\pgfpathlineto{\pgfqpoint{2.867014in}{2.385630in}}%
\pgfpathlineto{\pgfqpoint{2.869591in}{2.364615in}}%
\pgfpathlineto{\pgfqpoint{2.872168in}{2.314517in}}%
\pgfpathlineto{\pgfqpoint{2.879900in}{2.347353in}}%
\pgfpathlineto{\pgfqpoint{2.882477in}{2.348479in}}%
\pgfpathlineto{\pgfqpoint{2.885055in}{2.338722in}}%
\pgfpathlineto{\pgfqpoint{2.887632in}{2.361989in}}%
\pgfpathlineto{\pgfqpoint{2.890209in}{2.351106in}}%
\pgfpathlineto{\pgfqpoint{2.900518in}{2.269860in}}%
\pgfpathlineto{\pgfqpoint{2.903095in}{2.250159in}}%
\pgfpathlineto{\pgfqpoint{2.908249in}{2.296692in}}%
\pgfpathlineto{\pgfqpoint{2.915981in}{2.311140in}}%
\pgfpathlineto{\pgfqpoint{2.918558in}{2.333656in}}%
\pgfpathlineto{\pgfqpoint{2.921136in}{2.340411in}}%
\pgfpathlineto{\pgfqpoint{2.926290in}{2.358424in}}%
\pgfpathlineto{\pgfqpoint{2.936599in}{2.356360in}}%
\pgfpathlineto{\pgfqpoint{2.939176in}{2.361238in}}%
\pgfpathlineto{\pgfqpoint{2.941753in}{2.378125in}}%
\pgfpathlineto{\pgfqpoint{2.944330in}{2.383566in}}%
\pgfpathlineto{\pgfqpoint{2.952062in}{2.401954in}}%
\pgfpathlineto{\pgfqpoint{2.954639in}{2.392010in}}%
\pgfpathlineto{\pgfqpoint{2.957216in}{2.395575in}}%
\pgfpathlineto{\pgfqpoint{2.962371in}{2.410773in}}%
\pgfpathlineto{\pgfqpoint{2.970103in}{2.406458in}}%
\pgfpathlineto{\pgfqpoint{2.972680in}{2.420155in}}%
\pgfpathlineto{\pgfqpoint{2.975257in}{2.417153in}}%
\pgfpathlineto{\pgfqpoint{2.977834in}{2.422969in}}%
\pgfpathlineto{\pgfqpoint{2.980411in}{2.432164in}}%
\pgfpathlineto{\pgfqpoint{2.988143in}{2.423157in}}%
\pgfpathlineto{\pgfqpoint{2.990720in}{2.387882in}}%
\pgfpathlineto{\pgfqpoint{2.993297in}{2.390696in}}%
\pgfpathlineto{\pgfqpoint{2.995875in}{2.342287in}}%
\pgfpathlineto{\pgfqpoint{2.998452in}{2.337409in}}%
\pgfpathlineto{\pgfqpoint{3.006184in}{2.367430in}}%
\pgfpathlineto{\pgfqpoint{3.008761in}{2.371933in}}%
\pgfpathlineto{\pgfqpoint{3.011338in}{2.360863in}}%
\pgfpathlineto{\pgfqpoint{3.013915in}{2.356735in}}%
\pgfpathlineto{\pgfqpoint{3.016492in}{2.370057in}}%
\pgfpathlineto{\pgfqpoint{3.024224in}{2.356547in}}%
\pgfpathlineto{\pgfqpoint{3.026801in}{2.380564in}}%
\pgfpathlineto{\pgfqpoint{3.029378in}{2.370244in}}%
\pgfpathlineto{\pgfqpoint{3.031956in}{2.357298in}}%
\pgfpathlineto{\pgfqpoint{3.034533in}{2.373997in}}%
\pgfpathlineto{\pgfqpoint{3.042264in}{2.407771in}}%
\pgfpathlineto{\pgfqpoint{3.044842in}{2.425596in}}%
\pgfpathlineto{\pgfqpoint{3.047419in}{2.458057in}}%
\pgfpathlineto{\pgfqpoint{3.049996in}{2.456368in}}%
\pgfpathlineto{\pgfqpoint{3.052573in}{2.429912in}}%
\pgfpathlineto{\pgfqpoint{3.060305in}{2.396701in}}%
\pgfpathlineto{\pgfqpoint{3.062882in}{2.389946in}}%
\pgfpathlineto{\pgfqpoint{3.065459in}{2.410961in}}%
\pgfpathlineto{\pgfqpoint{3.068037in}{2.369869in}}%
\pgfpathlineto{\pgfqpoint{3.070614in}{2.359549in}}%
\pgfpathlineto{\pgfqpoint{3.078345in}{2.376249in}}%
\pgfpathlineto{\pgfqpoint{3.080923in}{2.391072in}}%
\pgfpathlineto{\pgfqpoint{3.083500in}{2.428223in}}%
\pgfpathlineto{\pgfqpoint{3.086077in}{2.436479in}}%
\pgfpathlineto{\pgfqpoint{3.096386in}{2.431976in}}%
\pgfpathlineto{\pgfqpoint{3.098963in}{2.446799in}}%
\pgfpathlineto{\pgfqpoint{3.101540in}{2.454304in}}%
\pgfpathlineto{\pgfqpoint{3.104118in}{2.443234in}}%
\pgfpathlineto{\pgfqpoint{3.106695in}{2.413775in}}%
\pgfpathlineto{\pgfqpoint{3.114426in}{2.422031in}}%
\pgfpathlineto{\pgfqpoint{3.117004in}{2.419592in}}%
\pgfpathlineto{\pgfqpoint{3.119581in}{2.432539in}}%
\pgfpathlineto{\pgfqpoint{3.122158in}{2.407021in}}%
\pgfpathlineto{\pgfqpoint{3.124735in}{2.402330in}}%
\pgfpathlineto{\pgfqpoint{3.132467in}{2.407396in}}%
\pgfpathlineto{\pgfqpoint{3.135044in}{2.393886in}}%
\pgfpathlineto{\pgfqpoint{3.137621in}{2.408897in}}%
\pgfpathlineto{\pgfqpoint{3.140199in}{2.410773in}}%
\pgfpathlineto{\pgfqpoint{3.142776in}{2.410210in}}%
\pgfpathlineto{\pgfqpoint{3.150507in}{2.437417in}}%
\pgfpathlineto{\pgfqpoint{3.153085in}{2.439294in}}%
\pgfpathlineto{\pgfqpoint{3.155662in}{2.425409in}}%
\pgfpathlineto{\pgfqpoint{3.158239in}{2.398389in}}%
\pgfpathlineto{\pgfqpoint{3.160816in}{2.379814in}}%
\pgfpathlineto{\pgfqpoint{3.168548in}{2.387131in}}%
\pgfpathlineto{\pgfqpoint{3.171125in}{2.355046in}}%
\pgfpathlineto{\pgfqpoint{3.173702in}{2.384129in}}%
\pgfpathlineto{\pgfqpoint{3.176280in}{2.387319in}}%
\pgfpathlineto{\pgfqpoint{3.178857in}{2.395763in}}%
\pgfpathlineto{\pgfqpoint{3.191743in}{2.402517in}}%
\pgfpathlineto{\pgfqpoint{3.194320in}{2.409272in}}%
\pgfpathlineto{\pgfqpoint{3.196897in}{2.407021in}}%
\pgfpathlineto{\pgfqpoint{3.207206in}{2.433665in}}%
\pgfpathlineto{\pgfqpoint{3.209783in}{2.422407in}}%
\pgfpathlineto{\pgfqpoint{3.212361in}{2.439669in}}%
\pgfpathlineto{\pgfqpoint{3.214938in}{2.451677in}}%
\pgfpathlineto{\pgfqpoint{3.222669in}{2.471567in}}%
\pgfpathlineto{\pgfqpoint{3.227824in}{2.441733in}}%
\pgfpathlineto{\pgfqpoint{3.230401in}{2.417528in}}%
\pgfpathlineto{\pgfqpoint{3.232978in}{2.416590in}}%
\pgfpathlineto{\pgfqpoint{3.240710in}{2.417903in}}%
\pgfpathlineto{\pgfqpoint{3.243287in}{2.420155in}}%
\pgfpathlineto{\pgfqpoint{3.245864in}{2.423908in}}%
\pgfpathlineto{\pgfqpoint{3.251019in}{2.436667in}}%
\pgfpathlineto{\pgfqpoint{3.258750in}{2.422219in}}%
\pgfpathlineto{\pgfqpoint{3.261328in}{2.398014in}}%
\pgfpathlineto{\pgfqpoint{3.263905in}{2.405332in}}%
\pgfpathlineto{\pgfqpoint{3.266482in}{2.399328in}}%
\pgfpathlineto{\pgfqpoint{3.269059in}{2.413025in}}%
\pgfpathlineto{\pgfqpoint{3.276791in}{2.394262in}}%
\pgfpathlineto{\pgfqpoint{3.279368in}{2.402330in}}%
\pgfpathlineto{\pgfqpoint{3.281945in}{2.389008in}}%
\pgfpathlineto{\pgfqpoint{3.284522in}{2.395575in}}%
\pgfpathlineto{\pgfqpoint{3.294831in}{2.388445in}}%
\pgfpathlineto{\pgfqpoint{3.297409in}{2.371183in}}%
\pgfpathlineto{\pgfqpoint{3.299986in}{2.367430in}}%
\pgfpathlineto{\pgfqpoint{3.302563in}{2.361801in}}%
\pgfpathlineto{\pgfqpoint{3.305140in}{2.372308in}}%
\pgfpathlineto{\pgfqpoint{3.312872in}{2.384692in}}%
\pgfpathlineto{\pgfqpoint{3.315449in}{2.384129in}}%
\pgfpathlineto{\pgfqpoint{3.318026in}{2.376061in}}%
\pgfpathlineto{\pgfqpoint{3.320603in}{2.349605in}}%
\pgfpathlineto{\pgfqpoint{3.323181in}{2.362927in}}%
\pgfpathlineto{\pgfqpoint{3.330912in}{2.352982in}}%
\pgfpathlineto{\pgfqpoint{3.333490in}{2.317707in}}%
\pgfpathlineto{\pgfqpoint{3.336067in}{2.298944in}}%
\pgfpathlineto{\pgfqpoint{3.338644in}{2.285809in}}%
\pgfpathlineto{\pgfqpoint{3.341221in}{2.284684in}}%
\pgfpathlineto{\pgfqpoint{3.348953in}{2.286560in}}%
\pgfpathlineto{\pgfqpoint{3.351530in}{2.258790in}}%
\pgfpathlineto{\pgfqpoint{3.354107in}{2.243029in}}%
\pgfpathlineto{\pgfqpoint{3.356684in}{2.222577in}}%
\pgfpathlineto{\pgfqpoint{3.359262in}{2.215822in}}%
\pgfpathlineto{\pgfqpoint{3.366993in}{2.220888in}}%
\pgfpathlineto{\pgfqpoint{3.369570in}{2.220325in}}%
\pgfpathlineto{\pgfqpoint{3.372148in}{2.199873in}}%
\pgfpathlineto{\pgfqpoint{3.374725in}{2.207003in}}%
\pgfpathlineto{\pgfqpoint{3.377302in}{2.236086in}}%
\pgfpathlineto{\pgfqpoint{3.385034in}{2.231958in}}%
\pgfpathlineto{\pgfqpoint{3.387611in}{2.218449in}}%
\pgfpathlineto{\pgfqpoint{3.390188in}{2.239276in}}%
\pgfpathlineto{\pgfqpoint{3.392765in}{2.242841in}}%
\pgfpathlineto{\pgfqpoint{3.395343in}{2.240402in}}%
\pgfpathlineto{\pgfqpoint{3.403074in}{2.281681in}}%
\pgfpathlineto{\pgfqpoint{3.405651in}{2.289562in}}%
\pgfpathlineto{\pgfqpoint{3.408229in}{2.308888in}}%
\pgfpathlineto{\pgfqpoint{3.410806in}{2.314705in}}%
\pgfpathlineto{\pgfqpoint{3.413383in}{2.303447in}}%
\pgfpathlineto{\pgfqpoint{3.421115in}{2.310765in}}%
\pgfpathlineto{\pgfqpoint{3.423692in}{2.308325in}}%
\pgfpathlineto{\pgfqpoint{3.426269in}{2.298568in}}%
\pgfpathlineto{\pgfqpoint{3.428846in}{2.298756in}}%
\pgfpathlineto{\pgfqpoint{3.431424in}{2.279805in}}%
\pgfpathlineto{\pgfqpoint{3.441732in}{2.297630in}}%
\pgfpathlineto{\pgfqpoint{3.444310in}{2.300257in}}%
\pgfpathlineto{\pgfqpoint{3.446887in}{2.300632in}}%
\pgfpathlineto{\pgfqpoint{3.449464in}{2.292001in}}%
\pgfpathlineto{\pgfqpoint{3.457196in}{2.290313in}}%
\pgfpathlineto{\pgfqpoint{3.459773in}{2.291438in}}%
\pgfpathlineto{\pgfqpoint{3.462350in}{2.288624in}}%
\pgfpathlineto{\pgfqpoint{3.464927in}{2.288811in}}%
\pgfpathlineto{\pgfqpoint{3.467505in}{2.285997in}}%
\pgfpathlineto{\pgfqpoint{3.475236in}{2.285622in}}%
\pgfpathlineto{\pgfqpoint{3.477813in}{2.289187in}}%
\pgfpathlineto{\pgfqpoint{3.480391in}{2.280743in}}%
\pgfpathlineto{\pgfqpoint{3.482968in}{2.288999in}}%
\pgfpathlineto{\pgfqpoint{3.485545in}{2.287686in}}%
\pgfpathlineto{\pgfqpoint{3.493277in}{2.254474in}}%
\pgfpathlineto{\pgfqpoint{3.495854in}{2.237025in}}%
\pgfpathlineto{\pgfqpoint{3.498431in}{2.247907in}}%
\pgfpathlineto{\pgfqpoint{3.501008in}{2.221451in}}%
\pgfpathlineto{\pgfqpoint{3.503586in}{2.234022in}}%
\pgfpathlineto{\pgfqpoint{3.511317in}{2.231208in}}%
\pgfpathlineto{\pgfqpoint{3.513894in}{2.240027in}}%
\pgfpathlineto{\pgfqpoint{3.516472in}{2.210381in}}%
\pgfpathlineto{\pgfqpoint{3.519049in}{2.200436in}}%
\pgfpathlineto{\pgfqpoint{3.521626in}{2.220138in}}%
\pgfpathlineto{\pgfqpoint{3.529358in}{2.216760in}}%
\pgfpathlineto{\pgfqpoint{3.531935in}{2.169852in}}%
\pgfpathlineto{\pgfqpoint{3.534512in}{2.190116in}}%
\pgfpathlineto{\pgfqpoint{3.537089in}{2.145084in}}%
\pgfpathlineto{\pgfqpoint{3.539667in}{2.145084in}}%
\pgfpathlineto{\pgfqpoint{3.547398in}{2.134577in}}%
\pgfpathlineto{\pgfqpoint{3.549975in}{2.148086in}}%
\pgfpathlineto{\pgfqpoint{3.552553in}{2.132137in}}%
\pgfpathlineto{\pgfqpoint{3.555130in}{2.133263in}}%
\pgfpathlineto{\pgfqpoint{3.557707in}{2.171728in}}%
\pgfpathlineto{\pgfqpoint{3.565439in}{2.170978in}}%
\pgfpathlineto{\pgfqpoint{3.568016in}{2.179233in}}%
\pgfpathlineto{\pgfqpoint{3.570593in}{2.165911in}}%
\pgfpathlineto{\pgfqpoint{3.573170in}{2.199498in}}%
\pgfpathlineto{\pgfqpoint{3.575747in}{2.210193in}}%
\pgfpathlineto{\pgfqpoint{3.583479in}{2.216385in}}%
\pgfpathlineto{\pgfqpoint{3.586056in}{2.251285in}}%
\pgfpathlineto{\pgfqpoint{3.588634in}{2.244155in}}%
\pgfpathlineto{\pgfqpoint{3.591211in}{2.252411in}}%
\pgfpathlineto{\pgfqpoint{3.593788in}{2.263481in}}%
\pgfpathlineto{\pgfqpoint{3.601520in}{2.251660in}}%
\pgfpathlineto{\pgfqpoint{3.604097in}{2.261417in}}%
\pgfpathlineto{\pgfqpoint{3.606674in}{2.279054in}}%
\pgfpathlineto{\pgfqpoint{3.609251in}{2.289937in}}%
\pgfpathlineto{\pgfqpoint{3.611828in}{2.298381in}}%
\pgfpathlineto{\pgfqpoint{3.619560in}{2.297067in}}%
\pgfpathlineto{\pgfqpoint{3.622137in}{2.284496in}}%
\pgfpathlineto{\pgfqpoint{3.624715in}{2.292564in}}%
\pgfpathlineto{\pgfqpoint{3.627292in}{2.292564in}}%
\pgfpathlineto{\pgfqpoint{3.629869in}{2.280931in}}%
\pgfpathlineto{\pgfqpoint{3.637601in}{2.279430in}}%
\pgfpathlineto{\pgfqpoint{3.640178in}{2.303634in}}%
\pgfpathlineto{\pgfqpoint{3.642755in}{2.301195in}}%
\pgfpathlineto{\pgfqpoint{3.645332in}{2.304010in}}%
\pgfpathlineto{\pgfqpoint{3.647909in}{2.328965in}}%
\pgfpathlineto{\pgfqpoint{3.655641in}{2.303447in}}%
\pgfpathlineto{\pgfqpoint{3.658218in}{2.352607in}}%
\pgfpathlineto{\pgfqpoint{3.660795in}{2.326526in}}%
\pgfpathlineto{\pgfqpoint{3.665950in}{2.325212in}}%
\pgfpathlineto{\pgfqpoint{3.673682in}{2.318645in}}%
\pgfpathlineto{\pgfqpoint{3.676259in}{2.318458in}}%
\pgfpathlineto{\pgfqpoint{3.678836in}{2.340598in}}%
\pgfpathlineto{\pgfqpoint{3.681413in}{2.344163in}}%
\pgfpathlineto{\pgfqpoint{3.683990in}{2.345664in}}%
\pgfpathlineto{\pgfqpoint{3.691722in}{2.375311in}}%
\pgfpathlineto{\pgfqpoint{3.694299in}{2.407959in}}%
\pgfpathlineto{\pgfqpoint{3.696876in}{2.382816in}}%
\pgfpathlineto{\pgfqpoint{3.699454in}{2.392010in}}%
\pgfpathlineto{\pgfqpoint{3.702031in}{2.360112in}}%
\pgfpathlineto{\pgfqpoint{3.709763in}{2.355984in}}%
\pgfpathlineto{\pgfqpoint{3.712340in}{2.378313in}}%
\pgfpathlineto{\pgfqpoint{3.714917in}{2.389571in}}%
\pgfpathlineto{\pgfqpoint{3.717494in}{2.441170in}}%
\pgfpathlineto{\pgfqpoint{3.720071in}{2.418091in}}%
\pgfpathlineto{\pgfqpoint{3.727803in}{2.444172in}}%
\pgfpathlineto{\pgfqpoint{3.730380in}{2.445485in}}%
\pgfpathlineto{\pgfqpoint{3.732957in}{2.440419in}}%
\pgfpathlineto{\pgfqpoint{3.738112in}{2.445861in}}%
\pgfpathlineto{\pgfqpoint{3.745843in}{2.439481in}}%
\pgfpathlineto{\pgfqpoint{3.748421in}{2.428411in}}%
\pgfpathlineto{\pgfqpoint{3.750998in}{2.408146in}}%
\pgfpathlineto{\pgfqpoint{3.756152in}{2.408897in}}%
\pgfpathlineto{\pgfqpoint{3.763884in}{2.376436in}}%
\pgfpathlineto{\pgfqpoint{3.766461in}{2.349417in}}%
\pgfpathlineto{\pgfqpoint{3.769038in}{2.369869in}}%
\pgfpathlineto{\pgfqpoint{3.771616in}{2.402330in}}%
\pgfpathlineto{\pgfqpoint{3.774193in}{2.391635in}}%
\pgfpathlineto{\pgfqpoint{3.781924in}{2.398952in}}%
\pgfpathlineto{\pgfqpoint{3.784502in}{2.397076in}}%
\pgfpathlineto{\pgfqpoint{3.787079in}{2.383191in}}%
\pgfpathlineto{\pgfqpoint{3.789656in}{2.383191in}}%
\pgfpathlineto{\pgfqpoint{3.792233in}{2.427473in}}%
\pgfpathlineto{\pgfqpoint{3.802542in}{2.450739in}}%
\pgfpathlineto{\pgfqpoint{3.807697in}{2.500087in}}%
\pgfpathlineto{\pgfqpoint{3.810274in}{2.492019in}}%
\pgfpathlineto{\pgfqpoint{3.818005in}{2.471379in}}%
\pgfpathlineto{\pgfqpoint{3.820583in}{2.478321in}}%
\pgfpathlineto{\pgfqpoint{3.823160in}{2.440419in}}%
\pgfpathlineto{\pgfqpoint{3.825737in}{2.432539in}}%
\pgfpathlineto{\pgfqpoint{3.828314in}{2.404394in}}%
\pgfpathlineto{\pgfqpoint{3.836046in}{2.434790in}}%
\pgfpathlineto{\pgfqpoint{3.838623in}{2.473631in}}%
\pgfpathlineto{\pgfqpoint{3.841200in}{2.455242in}}%
\pgfpathlineto{\pgfqpoint{3.843778in}{2.494270in}}%
\pgfpathlineto{\pgfqpoint{3.846355in}{2.489204in}}%
\pgfpathlineto{\pgfqpoint{3.854086in}{2.480948in}}%
\pgfpathlineto{\pgfqpoint{3.856664in}{2.482262in}}%
\pgfpathlineto{\pgfqpoint{3.859241in}{2.481136in}}%
\pgfpathlineto{\pgfqpoint{3.861818in}{2.496334in}}%
\pgfpathlineto{\pgfqpoint{3.864395in}{2.524479in}}%
\pgfpathlineto{\pgfqpoint{3.874704in}{2.525793in}}%
\pgfpathlineto{\pgfqpoint{3.877281in}{2.537051in}}%
\pgfpathlineto{\pgfqpoint{3.879859in}{2.552624in}}%
\pgfpathlineto{\pgfqpoint{3.882436in}{2.573076in}}%
\pgfpathlineto{\pgfqpoint{3.890167in}{2.566322in}}%
\pgfpathlineto{\pgfqpoint{3.892745in}{2.567635in}}%
\pgfpathlineto{\pgfqpoint{3.895322in}{2.560505in}}%
\pgfpathlineto{\pgfqpoint{3.900476in}{2.535737in}}%
\pgfpathlineto{\pgfqpoint{3.908208in}{2.557315in}}%
\pgfpathlineto{\pgfqpoint{3.910785in}{2.532172in}}%
\pgfpathlineto{\pgfqpoint{3.913362in}{2.520914in}}%
\pgfpathlineto{\pgfqpoint{3.915940in}{2.516786in}}%
\pgfpathlineto{\pgfqpoint{3.918517in}{2.493707in}}%
\pgfpathlineto{\pgfqpoint{3.926248in}{2.532923in}}%
\pgfpathlineto{\pgfqpoint{3.928826in}{2.459183in}}%
\pgfpathlineto{\pgfqpoint{3.931403in}{2.475132in}}%
\pgfpathlineto{\pgfqpoint{3.933980in}{2.524292in}}%
\pgfpathlineto{\pgfqpoint{3.936557in}{2.481886in}}%
\pgfpathlineto{\pgfqpoint{3.944289in}{2.504590in}}%
\pgfpathlineto{\pgfqpoint{3.946866in}{2.501025in}}%
\pgfpathlineto{\pgfqpoint{3.949443in}{2.508530in}}%
\pgfpathlineto{\pgfqpoint{3.952020in}{2.492769in}}%
\pgfpathlineto{\pgfqpoint{3.954598in}{2.494270in}}%
\pgfpathlineto{\pgfqpoint{3.962329in}{2.480948in}}%
\pgfpathlineto{\pgfqpoint{3.964907in}{2.485076in}}%
\pgfpathlineto{\pgfqpoint{3.967484in}{2.443422in}}%
\pgfpathlineto{\pgfqpoint{3.970061in}{2.436291in}}%
\pgfpathlineto{\pgfqpoint{3.972638in}{2.450739in}}%
\pgfpathlineto{\pgfqpoint{3.980370in}{2.483575in}}%
\pgfpathlineto{\pgfqpoint{3.982947in}{2.455805in}}%
\pgfpathlineto{\pgfqpoint{3.985524in}{2.434227in}}%
\pgfpathlineto{\pgfqpoint{3.988101in}{2.454680in}}%
\pgfpathlineto{\pgfqpoint{3.998410in}{2.467063in}}%
\pgfpathlineto{\pgfqpoint{4.000988in}{2.460871in}}%
\pgfpathlineto{\pgfqpoint{4.003565in}{2.466688in}}%
\pgfpathlineto{\pgfqpoint{4.006142in}{2.467439in}}%
\pgfpathlineto{\pgfqpoint{4.008719in}{2.477196in}}%
\pgfpathlineto{\pgfqpoint{4.016451in}{2.459370in}}%
\pgfpathlineto{\pgfqpoint{4.019028in}{2.464249in}}%
\pgfpathlineto{\pgfqpoint{4.021605in}{2.467251in}}%
\pgfpathlineto{\pgfqpoint{4.024182in}{2.460684in}}%
\pgfpathlineto{\pgfqpoint{4.026760in}{2.420530in}}%
\pgfpathlineto{\pgfqpoint{4.037069in}{2.451677in}}%
\pgfpathlineto{\pgfqpoint{4.039646in}{2.451865in}}%
\pgfpathlineto{\pgfqpoint{4.042223in}{2.456931in}}%
\pgfpathlineto{\pgfqpoint{4.044800in}{2.438168in}}%
\pgfpathlineto{\pgfqpoint{4.052532in}{2.431038in}}%
\pgfpathlineto{\pgfqpoint{4.055109in}{2.436854in}}%
\pgfpathlineto{\pgfqpoint{4.057686in}{2.425033in}}%
\pgfpathlineto{\pgfqpoint{4.060263in}{2.397451in}}%
\pgfpathlineto{\pgfqpoint{4.062841in}{2.425972in}}%
\pgfpathlineto{\pgfqpoint{4.070572in}{2.443234in}}%
\pgfpathlineto{\pgfqpoint{4.073149in}{2.419780in}}%
\pgfpathlineto{\pgfqpoint{4.075727in}{2.419967in}}%
\pgfpathlineto{\pgfqpoint{4.078304in}{2.436291in}}%
\pgfpathlineto{\pgfqpoint{4.080881in}{2.476633in}}%
\pgfpathlineto{\pgfqpoint{4.088613in}{2.464436in}}%
\pgfpathlineto{\pgfqpoint{4.091190in}{2.458620in}}%
\pgfpathlineto{\pgfqpoint{4.093767in}{2.469690in}}%
\pgfpathlineto{\pgfqpoint{4.096344in}{2.499899in}}%
\pgfpathlineto{\pgfqpoint{4.098922in}{2.488829in}}%
\pgfpathlineto{\pgfqpoint{4.106653in}{2.489016in}}%
\pgfpathlineto{\pgfqpoint{4.109230in}{2.498586in}}%
\pgfpathlineto{\pgfqpoint{4.111808in}{2.495208in}}%
\pgfpathlineto{\pgfqpoint{4.114385in}{2.499524in}}%
\pgfpathlineto{\pgfqpoint{4.116962in}{2.490893in}}%
\pgfpathlineto{\pgfqpoint{4.127271in}{2.461059in}}%
\pgfpathlineto{\pgfqpoint{4.129848in}{2.474193in}}%
\pgfpathlineto{\pgfqpoint{4.132425in}{2.475132in}}%
\pgfpathlineto{\pgfqpoint{4.135003in}{2.466313in}}%
\pgfpathlineto{\pgfqpoint{4.142734in}{2.463311in}}%
\pgfpathlineto{\pgfqpoint{4.145311in}{2.469127in}}%
\pgfpathlineto{\pgfqpoint{4.147889in}{2.489016in}}%
\pgfpathlineto{\pgfqpoint{4.150466in}{2.465750in}}%
\pgfpathlineto{\pgfqpoint{4.153043in}{2.463311in}}%
\pgfpathlineto{\pgfqpoint{4.160775in}{2.449426in}}%
\pgfpathlineto{\pgfqpoint{4.163352in}{2.451490in}}%
\pgfpathlineto{\pgfqpoint{4.165929in}{2.471942in}}%
\pgfpathlineto{\pgfqpoint{4.168506in}{2.485827in}}%
\pgfpathlineto{\pgfqpoint{4.171084in}{2.473631in}}%
\pgfpathlineto{\pgfqpoint{4.178815in}{2.422594in}}%
\pgfpathlineto{\pgfqpoint{4.181392in}{2.428223in}}%
\pgfpathlineto{\pgfqpoint{4.183970in}{2.430475in}}%
\pgfpathlineto{\pgfqpoint{4.186547in}{2.444547in}}%
\pgfpathlineto{\pgfqpoint{4.189124in}{2.427473in}}%
\pgfpathlineto{\pgfqpoint{4.196856in}{2.434040in}}%
\pgfpathlineto{\pgfqpoint{4.199433in}{2.433102in}}%
\pgfpathlineto{\pgfqpoint{4.202010in}{2.411711in}}%
\pgfpathlineto{\pgfqpoint{4.204587in}{2.400266in}}%
\pgfpathlineto{\pgfqpoint{4.207165in}{2.404018in}}%
\pgfpathlineto{\pgfqpoint{4.217473in}{2.359925in}}%
\pgfpathlineto{\pgfqpoint{4.220051in}{2.358048in}}%
\pgfpathlineto{\pgfqpoint{4.222628in}{2.336658in}}%
\pgfpathlineto{\pgfqpoint{4.232937in}{2.333093in}}%
\pgfpathlineto{\pgfqpoint{4.235514in}{2.345664in}}%
\pgfpathlineto{\pgfqpoint{4.238091in}{2.321460in}}%
\pgfpathlineto{\pgfqpoint{4.240668in}{2.325588in}}%
\pgfpathlineto{\pgfqpoint{4.243246in}{2.347165in}}%
\pgfpathlineto{\pgfqpoint{4.250977in}{2.370995in}}%
\pgfpathlineto{\pgfqpoint{4.253554in}{2.369869in}}%
\pgfpathlineto{\pgfqpoint{4.256132in}{2.365366in}}%
\pgfpathlineto{\pgfqpoint{4.258709in}{2.365554in}}%
\pgfpathlineto{\pgfqpoint{4.261286in}{2.356735in}}%
\pgfpathlineto{\pgfqpoint{4.269018in}{2.352232in}}%
\pgfpathlineto{\pgfqpoint{4.271595in}{2.219762in}}%
\pgfpathlineto{\pgfqpoint{4.274172in}{2.200061in}}%
\pgfpathlineto{\pgfqpoint{4.276749in}{2.193118in}}%
\pgfpathlineto{\pgfqpoint{4.279326in}{2.161971in}}%
\pgfpathlineto{\pgfqpoint{4.287058in}{2.154466in}}%
\pgfpathlineto{\pgfqpoint{4.289635in}{2.156155in}}%
\pgfpathlineto{\pgfqpoint{4.292213in}{2.162722in}}%
\pgfpathlineto{\pgfqpoint{4.294790in}{2.186176in}}%
\pgfpathlineto{\pgfqpoint{4.297367in}{2.178858in}}%
\pgfpathlineto{\pgfqpoint{4.305099in}{2.163847in}}%
\pgfpathlineto{\pgfqpoint{4.307676in}{2.157093in}}%
\pgfpathlineto{\pgfqpoint{4.310253in}{2.154466in}}%
\pgfpathlineto{\pgfqpoint{4.312830in}{2.156342in}}%
\pgfpathlineto{\pgfqpoint{4.315407in}{2.144521in}}%
\pgfpathlineto{\pgfqpoint{4.323139in}{2.165911in}}%
\pgfpathlineto{\pgfqpoint{4.325716in}{2.149963in}}%
\pgfpathlineto{\pgfqpoint{4.328294in}{2.162346in}}%
\pgfpathlineto{\pgfqpoint{4.330871in}{2.157093in}}%
\pgfpathlineto{\pgfqpoint{4.333448in}{2.162534in}}%
\pgfpathlineto{\pgfqpoint{4.341180in}{2.173792in}}%
\pgfpathlineto{\pgfqpoint{4.343757in}{2.174730in}}%
\pgfpathlineto{\pgfqpoint{4.346334in}{2.154091in}}%
\pgfpathlineto{\pgfqpoint{4.351488in}{2.064965in}}%
\pgfpathlineto{\pgfqpoint{4.359220in}{2.028188in}}%
\pgfpathlineto{\pgfqpoint{4.361797in}{1.992726in}}%
\pgfpathlineto{\pgfqpoint{4.364374in}{2.040009in}}%
\pgfpathlineto{\pgfqpoint{4.366952in}{2.069280in}}%
\pgfpathlineto{\pgfqpoint{4.369529in}{2.068905in}}%
\pgfpathlineto{\pgfqpoint{4.377261in}{2.040947in}}%
\pgfpathlineto{\pgfqpoint{4.379838in}{2.008487in}}%
\pgfpathlineto{\pgfqpoint{4.382415in}{2.034568in}}%
\pgfpathlineto{\pgfqpoint{4.384992in}{2.046014in}}%
\pgfpathlineto{\pgfqpoint{4.387569in}{2.024999in}}%
\pgfpathlineto{\pgfqpoint{4.397878in}{2.061587in}}%
\pgfpathlineto{\pgfqpoint{4.400455in}{2.045075in}}%
\pgfpathlineto{\pgfqpoint{4.403033in}{2.037195in}}%
\pgfpathlineto{\pgfqpoint{4.405610in}{2.053144in}}%
\pgfpathlineto{\pgfqpoint{4.413342in}{2.044700in}}%
\pgfpathlineto{\pgfqpoint{4.415919in}{2.059148in}}%
\pgfpathlineto{\pgfqpoint{4.418496in}{2.080913in}}%
\pgfpathlineto{\pgfqpoint{4.421073in}{2.070031in}}%
\pgfpathlineto{\pgfqpoint{4.423650in}{2.031566in}}%
\pgfpathlineto{\pgfqpoint{4.431382in}{2.039822in}}%
\pgfpathlineto{\pgfqpoint{4.433959in}{1.982406in}}%
\pgfpathlineto{\pgfqpoint{4.436536in}{1.961203in}}%
\pgfpathlineto{\pgfqpoint{4.439114in}{1.958764in}}%
\pgfpathlineto{\pgfqpoint{4.441691in}{1.967020in}}%
\pgfpathlineto{\pgfqpoint{4.449422in}{1.958952in}}%
\pgfpathlineto{\pgfqpoint{4.454577in}{1.995915in}}%
\pgfpathlineto{\pgfqpoint{4.457154in}{1.985220in}}%
\pgfpathlineto{\pgfqpoint{4.459731in}{2.009237in}}%
\pgfpathlineto{\pgfqpoint{4.467463in}{2.052018in}}%
\pgfpathlineto{\pgfqpoint{4.470040in}{2.057084in}}%
\pgfpathlineto{\pgfqpoint{4.477772in}{2.105306in}}%
\pgfpathlineto{\pgfqpoint{4.485503in}{2.106432in}}%
\pgfpathlineto{\pgfqpoint{4.488081in}{2.086730in}}%
\pgfpathlineto{\pgfqpoint{4.490658in}{2.050517in}}%
\pgfpathlineto{\pgfqpoint{4.493235in}{2.067779in}}%
\pgfpathlineto{\pgfqpoint{4.495812in}{2.064777in}}%
\pgfpathlineto{\pgfqpoint{4.503544in}{2.048453in}}%
\pgfpathlineto{\pgfqpoint{4.506121in}{2.109621in}}%
\pgfpathlineto{\pgfqpoint{4.508698in}{2.149963in}}%
\pgfpathlineto{\pgfqpoint{4.511276in}{2.181110in}}%
\pgfpathlineto{\pgfqpoint{4.513853in}{2.195370in}}%
\pgfpathlineto{\pgfqpoint{4.521584in}{2.188803in}}%
\pgfpathlineto{\pgfqpoint{4.524162in}{2.165536in}}%
\pgfpathlineto{\pgfqpoint{4.526739in}{2.173042in}}%
\pgfpathlineto{\pgfqpoint{4.529316in}{2.168538in}}%
\pgfpathlineto{\pgfqpoint{4.531893in}{2.157468in}}%
\pgfpathlineto{\pgfqpoint{4.539625in}{2.174355in}}%
\pgfpathlineto{\pgfqpoint{4.542202in}{2.184862in}}%
\pgfpathlineto{\pgfqpoint{4.544779in}{2.190491in}}%
\pgfpathlineto{\pgfqpoint{4.547357in}{2.198560in}}%
\pgfpathlineto{\pgfqpoint{4.549934in}{2.198560in}}%
\pgfpathlineto{\pgfqpoint{4.562820in}{2.165911in}}%
\pgfpathlineto{\pgfqpoint{4.565397in}{2.182423in}}%
\pgfpathlineto{\pgfqpoint{4.567974in}{2.135140in}}%
\pgfpathlineto{\pgfqpoint{4.575706in}{2.158406in}}%
\pgfpathlineto{\pgfqpoint{4.578283in}{2.153528in}}%
\pgfpathlineto{\pgfqpoint{4.580860in}{2.156342in}}%
\pgfpathlineto{\pgfqpoint{4.583438in}{2.166474in}}%
\pgfpathlineto{\pgfqpoint{4.586015in}{2.165161in}}%
\pgfpathlineto{\pgfqpoint{4.593746in}{2.162909in}}%
\pgfpathlineto{\pgfqpoint{4.596324in}{2.150338in}}%
\pgfpathlineto{\pgfqpoint{4.598901in}{2.148837in}}%
\pgfpathlineto{\pgfqpoint{4.611787in}{2.127634in}}%
\pgfpathlineto{\pgfqpoint{4.614364in}{2.138892in}}%
\pgfpathlineto{\pgfqpoint{4.616941in}{2.113937in}}%
\pgfpathlineto{\pgfqpoint{4.619519in}{2.104368in}}%
\pgfpathlineto{\pgfqpoint{4.622096in}{2.122193in}}%
\pgfpathlineto{\pgfqpoint{4.629827in}{2.125195in}}%
\pgfpathlineto{\pgfqpoint{4.632405in}{2.096299in}}%
\pgfpathlineto{\pgfqpoint{4.634982in}{2.094986in}}%
\pgfpathlineto{\pgfqpoint{4.637559in}{2.090107in}}%
\pgfpathlineto{\pgfqpoint{4.640136in}{2.080163in}}%
\pgfpathlineto{\pgfqpoint{4.647868in}{2.075472in}}%
\pgfpathlineto{\pgfqpoint{4.650445in}{2.079600in}}%
\pgfpathlineto{\pgfqpoint{4.653022in}{2.111873in}}%
\pgfpathlineto{\pgfqpoint{4.658177in}{2.063651in}}%
\pgfpathlineto{\pgfqpoint{4.665908in}{2.085604in}}%
\pgfpathlineto{\pgfqpoint{4.671063in}{2.131950in}}%
\pgfpathlineto{\pgfqpoint{4.673640in}{2.131950in}}%
\pgfpathlineto{\pgfqpoint{4.683949in}{2.128385in}}%
\pgfpathlineto{\pgfqpoint{4.686526in}{2.148837in}}%
\pgfpathlineto{\pgfqpoint{4.689103in}{2.142270in}}%
\pgfpathlineto{\pgfqpoint{4.691680in}{2.128009in}}%
\pgfpathlineto{\pgfqpoint{4.701989in}{2.119378in}}%
\pgfpathlineto{\pgfqpoint{4.704567in}{2.122005in}}%
\pgfpathlineto{\pgfqpoint{4.707144in}{2.077161in}}%
\pgfpathlineto{\pgfqpoint{4.709721in}{2.055958in}}%
\pgfpathlineto{\pgfqpoint{4.712298in}{2.030065in}}%
\pgfpathlineto{\pgfqpoint{4.720030in}{2.031941in}}%
\pgfpathlineto{\pgfqpoint{4.722607in}{2.031941in}}%
\pgfpathlineto{\pgfqpoint{4.725184in}{2.003046in}}%
\pgfpathlineto{\pgfqpoint{4.727761in}{2.006235in}}%
\pgfpathlineto{\pgfqpoint{4.730339in}{1.947881in}}%
\pgfpathlineto{\pgfqpoint{4.740648in}{1.941126in}}%
\pgfpathlineto{\pgfqpoint{4.743225in}{1.933996in}}%
\pgfpathlineto{\pgfqpoint{4.748379in}{1.960265in}}%
\pgfpathlineto{\pgfqpoint{4.756111in}{1.935122in}}%
\pgfpathlineto{\pgfqpoint{4.758688in}{1.948069in}}%
\pgfpathlineto{\pgfqpoint{4.761265in}{1.950883in}}%
\pgfpathlineto{\pgfqpoint{4.763842in}{1.961954in}}%
\pgfpathlineto{\pgfqpoint{4.766420in}{1.983344in}}%
\pgfpathlineto{\pgfqpoint{4.774151in}{1.981092in}}%
\pgfpathlineto{\pgfqpoint{4.776728in}{1.944129in}}%
\pgfpathlineto{\pgfqpoint{4.779306in}{1.953510in}}%
\pgfpathlineto{\pgfqpoint{4.781883in}{1.990849in}}%
\pgfpathlineto{\pgfqpoint{4.784460in}{1.985971in}}%
\pgfpathlineto{\pgfqpoint{4.792192in}{1.967583in}}%
\pgfpathlineto{\pgfqpoint{4.794769in}{1.975651in}}%
\pgfpathlineto{\pgfqpoint{4.797346in}{1.970773in}}%
\pgfpathlineto{\pgfqpoint{4.799923in}{1.930994in}}%
\pgfpathlineto{\pgfqpoint{4.802501in}{1.953323in}}%
\pgfpathlineto{\pgfqpoint{4.812809in}{1.962517in}}%
\pgfpathlineto{\pgfqpoint{4.815387in}{2.002107in}}%
\pgfpathlineto{\pgfqpoint{4.817964in}{2.006235in}}%
\pgfpathlineto{\pgfqpoint{4.820541in}{2.003984in}}%
\pgfpathlineto{\pgfqpoint{4.828273in}{2.076035in}}%
\pgfpathlineto{\pgfqpoint{4.830850in}{2.062713in}}%
\pgfpathlineto{\pgfqpoint{4.833427in}{2.097613in}}%
\pgfpathlineto{\pgfqpoint{4.836004in}{2.175105in}}%
\pgfpathlineto{\pgfqpoint{4.846313in}{2.149963in}}%
\pgfpathlineto{\pgfqpoint{4.848890in}{2.122568in}}%
\pgfpathlineto{\pgfqpoint{4.854045in}{2.141144in}}%
\pgfpathlineto{\pgfqpoint{4.856622in}{2.156530in}}%
\pgfpathlineto{\pgfqpoint{4.866931in}{2.154841in}}%
\pgfpathlineto{\pgfqpoint{4.869508in}{2.149587in}}%
\pgfpathlineto{\pgfqpoint{4.872085in}{2.142270in}}%
\pgfpathlineto{\pgfqpoint{4.874663in}{2.152214in}}%
\pgfpathlineto{\pgfqpoint{4.882394in}{2.153528in}}%
\pgfpathlineto{\pgfqpoint{4.884971in}{2.144146in}}%
\pgfpathlineto{\pgfqpoint{4.890126in}{2.188615in}}%
\pgfpathlineto{\pgfqpoint{4.892703in}{2.192180in}}%
\pgfpathlineto{\pgfqpoint{4.900435in}{2.194244in}}%
\pgfpathlineto{\pgfqpoint{4.903012in}{2.185425in}}%
\pgfpathlineto{\pgfqpoint{4.905589in}{2.193306in}}%
\pgfpathlineto{\pgfqpoint{4.908166in}{2.192368in}}%
\pgfpathlineto{\pgfqpoint{4.918475in}{2.189741in}}%
\pgfpathlineto{\pgfqpoint{4.921052in}{2.210756in}}%
\pgfpathlineto{\pgfqpoint{4.923630in}{2.213007in}}%
\pgfpathlineto{\pgfqpoint{4.928784in}{2.208129in}}%
\pgfpathlineto{\pgfqpoint{4.936516in}{2.213383in}}%
\pgfpathlineto{\pgfqpoint{4.939093in}{2.206253in}}%
\pgfpathlineto{\pgfqpoint{4.946825in}{2.231771in}}%
\pgfpathlineto{\pgfqpoint{4.954556in}{2.246031in}}%
\pgfpathlineto{\pgfqpoint{4.957133in}{2.262355in}}%
\pgfpathlineto{\pgfqpoint{4.959711in}{2.288999in}}%
\pgfpathlineto{\pgfqpoint{4.962288in}{2.290125in}}%
\pgfpathlineto{\pgfqpoint{4.964865in}{2.288249in}}%
\pgfpathlineto{\pgfqpoint{4.972597in}{2.298005in}}%
\pgfpathlineto{\pgfqpoint{4.975174in}{2.295754in}}%
\pgfpathlineto{\pgfqpoint{4.977751in}{2.304948in}}%
\pgfpathlineto{\pgfqpoint{4.980328in}{2.303259in}}%
\pgfpathlineto{\pgfqpoint{4.982905in}{2.307950in}}%
\pgfpathlineto{\pgfqpoint{4.990637in}{2.299131in}}%
\pgfpathlineto{\pgfqpoint{4.993214in}{2.292189in}}%
\pgfpathlineto{\pgfqpoint{4.995792in}{2.311140in}}%
\pgfpathlineto{\pgfqpoint{4.998369in}{2.282244in}}%
\pgfpathlineto{\pgfqpoint{5.000946in}{2.284684in}}%
\pgfpathlineto{\pgfqpoint{5.008678in}{2.284684in}}%
\pgfpathlineto{\pgfqpoint{5.011255in}{2.245843in}}%
\pgfpathlineto{\pgfqpoint{5.013832in}{2.220325in}}%
\pgfpathlineto{\pgfqpoint{5.016409in}{2.213758in}}%
\pgfpathlineto{\pgfqpoint{5.018986in}{2.228206in}}%
\pgfpathlineto{\pgfqpoint{5.026718in}{2.210381in}}%
\pgfpathlineto{\pgfqpoint{5.029295in}{2.246219in}}%
\pgfpathlineto{\pgfqpoint{5.031873in}{2.236649in}}%
\pgfpathlineto{\pgfqpoint{5.034450in}{2.234210in}}%
\pgfpathlineto{\pgfqpoint{5.037027in}{2.213383in}}%
\pgfpathlineto{\pgfqpoint{5.044759in}{2.240402in}}%
\pgfpathlineto{\pgfqpoint{5.047336in}{2.209818in}}%
\pgfpathlineto{\pgfqpoint{5.049913in}{2.207941in}}%
\pgfpathlineto{\pgfqpoint{5.052490in}{2.194244in}}%
\pgfpathlineto{\pgfqpoint{5.055067in}{2.204564in}}%
\pgfpathlineto{\pgfqpoint{5.062799in}{2.200999in}}%
\pgfpathlineto{\pgfqpoint{5.065376in}{2.218636in}}%
\pgfpathlineto{\pgfqpoint{5.067953in}{2.225954in}}%
\pgfpathlineto{\pgfqpoint{5.070531in}{2.227455in}}%
\pgfpathlineto{\pgfqpoint{5.073108in}{2.233647in}}%
\pgfpathlineto{\pgfqpoint{5.083417in}{2.230457in}}%
\pgfpathlineto{\pgfqpoint{5.085994in}{2.226329in}}%
\pgfpathlineto{\pgfqpoint{5.088571in}{2.234210in}}%
\pgfpathlineto{\pgfqpoint{5.091148in}{2.227831in}}%
\pgfpathlineto{\pgfqpoint{5.098880in}{2.244155in}}%
\pgfpathlineto{\pgfqpoint{5.101457in}{2.245656in}}%
\pgfpathlineto{\pgfqpoint{5.104034in}{2.256538in}}%
\pgfpathlineto{\pgfqpoint{5.106612in}{2.260854in}}%
\pgfpathlineto{\pgfqpoint{5.109189in}{2.255225in}}%
\pgfpathlineto{\pgfqpoint{5.116921in}{2.241528in}}%
\pgfpathlineto{\pgfqpoint{5.119498in}{2.241153in}}%
\pgfpathlineto{\pgfqpoint{5.122075in}{2.229707in}}%
\pgfpathlineto{\pgfqpoint{5.124652in}{2.239839in}}%
\pgfpathlineto{\pgfqpoint{5.127229in}{2.241340in}}%
\pgfpathlineto{\pgfqpoint{5.134961in}{2.249971in}}%
\pgfpathlineto{\pgfqpoint{5.137538in}{2.247720in}}%
\pgfpathlineto{\pgfqpoint{5.140115in}{2.243967in}}%
\pgfpathlineto{\pgfqpoint{5.142693in}{2.261042in}}%
\pgfpathlineto{\pgfqpoint{5.145270in}{2.200811in}}%
\pgfpathlineto{\pgfqpoint{5.153001in}{2.171353in}}%
\pgfpathlineto{\pgfqpoint{5.160733in}{2.264982in}}%
\pgfpathlineto{\pgfqpoint{5.163310in}{2.268172in}}%
\pgfpathlineto{\pgfqpoint{5.173619in}{2.231208in}}%
\pgfpathlineto{\pgfqpoint{5.176196in}{2.244530in}}%
\pgfpathlineto{\pgfqpoint{5.178774in}{2.254850in}}%
\pgfpathlineto{\pgfqpoint{5.181351in}{2.284308in}}%
\pgfpathlineto{\pgfqpoint{5.189082in}{2.289562in}}%
\pgfpathlineto{\pgfqpoint{5.194237in}{2.309451in}}%
\pgfpathlineto{\pgfqpoint{5.196814in}{2.310202in}}%
\pgfpathlineto{\pgfqpoint{5.199391in}{2.316581in}}%
\pgfpathlineto{\pgfqpoint{5.207123in}{2.316394in}}%
\pgfpathlineto{\pgfqpoint{5.209700in}{2.318645in}}%
\pgfpathlineto{\pgfqpoint{5.212277in}{2.325963in}}%
\pgfpathlineto{\pgfqpoint{5.214855in}{2.322773in}}%
\pgfpathlineto{\pgfqpoint{5.217432in}{2.310014in}}%
\pgfpathlineto{\pgfqpoint{5.225163in}{2.301758in}}%
\pgfpathlineto{\pgfqpoint{5.227741in}{2.358424in}}%
\pgfpathlineto{\pgfqpoint{5.230318in}{2.354858in}}%
\pgfpathlineto{\pgfqpoint{5.232895in}{2.352982in}}%
\pgfpathlineto{\pgfqpoint{5.235472in}{2.354296in}}%
\pgfpathlineto{\pgfqpoint{5.243204in}{2.341724in}}%
\pgfpathlineto{\pgfqpoint{5.245781in}{2.330091in}}%
\pgfpathlineto{\pgfqpoint{5.248358in}{2.330466in}}%
\pgfpathlineto{\pgfqpoint{5.250936in}{2.331592in}}%
\pgfpathlineto{\pgfqpoint{5.253513in}{2.355797in}}%
\pgfpathlineto{\pgfqpoint{5.261244in}{2.356735in}}%
\pgfpathlineto{\pgfqpoint{5.263822in}{2.366304in}}%
\pgfpathlineto{\pgfqpoint{5.266399in}{2.362364in}}%
\pgfpathlineto{\pgfqpoint{5.268976in}{2.380940in}}%
\pgfpathlineto{\pgfqpoint{5.271553in}{2.375498in}}%
\pgfpathlineto{\pgfqpoint{5.279285in}{2.389946in}}%
\pgfpathlineto{\pgfqpoint{5.281862in}{2.380752in}}%
\pgfpathlineto{\pgfqpoint{5.287017in}{2.395200in}}%
\pgfpathlineto{\pgfqpoint{5.289594in}{2.392010in}}%
\pgfpathlineto{\pgfqpoint{5.297325in}{2.380752in}}%
\pgfpathlineto{\pgfqpoint{5.299903in}{2.371933in}}%
\pgfpathlineto{\pgfqpoint{5.302480in}{2.371183in}}%
\pgfpathlineto{\pgfqpoint{5.305057in}{2.366867in}}%
\pgfpathlineto{\pgfqpoint{5.307634in}{2.359737in}}%
\pgfpathlineto{\pgfqpoint{5.315366in}{2.371370in}}%
\pgfpathlineto{\pgfqpoint{5.317943in}{2.360487in}}%
\pgfpathlineto{\pgfqpoint{5.320520in}{2.344163in}}%
\pgfpathlineto{\pgfqpoint{5.323098in}{2.349229in}}%
\pgfpathlineto{\pgfqpoint{5.325675in}{2.352982in}}%
\pgfpathlineto{\pgfqpoint{5.335984in}{2.336846in}}%
\pgfpathlineto{\pgfqpoint{5.338561in}{2.334031in}}%
\pgfpathlineto{\pgfqpoint{5.341138in}{2.332155in}}%
\pgfpathlineto{\pgfqpoint{5.343715in}{2.278116in}}%
\pgfpathlineto{\pgfqpoint{5.351447in}{2.301946in}}%
\pgfpathlineto{\pgfqpoint{5.354024in}{2.271737in}}%
\pgfpathlineto{\pgfqpoint{5.356601in}{2.260104in}}%
\pgfpathlineto{\pgfqpoint{5.359178in}{2.278679in}}%
\pgfpathlineto{\pgfqpoint{5.361756in}{2.232709in}}%
\pgfpathlineto{\pgfqpoint{5.369487in}{2.238713in}}%
\pgfpathlineto{\pgfqpoint{5.372065in}{2.235148in}}%
\pgfpathlineto{\pgfqpoint{5.374642in}{2.265545in}}%
\pgfpathlineto{\pgfqpoint{5.377219in}{2.283745in}}%
\pgfpathlineto{\pgfqpoint{5.379796in}{2.276052in}}%
\pgfpathlineto{\pgfqpoint{5.387528in}{2.270236in}}%
\pgfpathlineto{\pgfqpoint{5.390105in}{2.272300in}}%
\pgfpathlineto{\pgfqpoint{5.392682in}{2.272300in}}%
\pgfpathlineto{\pgfqpoint{5.395259in}{2.249596in}}%
\pgfpathlineto{\pgfqpoint{5.397837in}{2.259165in}}%
\pgfpathlineto{\pgfqpoint{5.405568in}{2.273801in}}%
\pgfpathlineto{\pgfqpoint{5.408146in}{2.256163in}}%
\pgfpathlineto{\pgfqpoint{5.410723in}{2.270611in}}%
\pgfpathlineto{\pgfqpoint{5.413300in}{2.267609in}}%
\pgfpathlineto{\pgfqpoint{5.415877in}{2.241153in}}%
\pgfpathlineto{\pgfqpoint{5.423609in}{2.230645in}}%
\pgfpathlineto{\pgfqpoint{5.426186in}{2.209255in}}%
\pgfpathlineto{\pgfqpoint{5.428763in}{2.212069in}}%
\pgfpathlineto{\pgfqpoint{5.431340in}{2.228393in}}%
\pgfpathlineto{\pgfqpoint{5.433918in}{2.233835in}}%
\pgfpathlineto{\pgfqpoint{5.441649in}{2.225767in}}%
\pgfpathlineto{\pgfqpoint{5.444227in}{2.230833in}}%
\pgfpathlineto{\pgfqpoint{5.446804in}{2.226705in}}%
\pgfpathlineto{\pgfqpoint{5.449381in}{2.218449in}}%
\pgfpathlineto{\pgfqpoint{5.451958in}{2.207566in}}%
\pgfpathlineto{\pgfqpoint{5.459690in}{2.222577in}}%
\pgfpathlineto{\pgfqpoint{5.462267in}{2.254850in}}%
\pgfpathlineto{\pgfqpoint{5.464844in}{2.248658in}}%
\pgfpathlineto{\pgfqpoint{5.467421in}{2.232146in}}%
\pgfpathlineto{\pgfqpoint{5.469999in}{2.263293in}}%
\pgfpathlineto{\pgfqpoint{5.477730in}{2.269673in}}%
\pgfpathlineto{\pgfqpoint{5.480307in}{2.266295in}}%
\pgfpathlineto{\pgfqpoint{5.485462in}{2.249408in}}%
\pgfpathlineto{\pgfqpoint{5.488039in}{2.254287in}}%
\pgfpathlineto{\pgfqpoint{5.495771in}{2.284684in}}%
\pgfpathlineto{\pgfqpoint{5.498348in}{2.290313in}}%
\pgfpathlineto{\pgfqpoint{5.500925in}{2.315643in}}%
\pgfpathlineto{\pgfqpoint{5.503502in}{2.379063in}}%
\pgfpathlineto{\pgfqpoint{5.506080in}{2.386944in}}%
\pgfpathlineto{\pgfqpoint{5.513811in}{2.369306in}}%
\pgfpathlineto{\pgfqpoint{5.516388in}{2.366304in}}%
\pgfpathlineto{\pgfqpoint{5.518966in}{2.365178in}}%
\pgfpathlineto{\pgfqpoint{5.524120in}{2.356922in}}%
\pgfpathlineto{\pgfqpoint{5.534429in}{2.364803in}}%
\pgfpathlineto{\pgfqpoint{5.537006in}{2.385630in}}%
\pgfpathlineto{\pgfqpoint{5.542161in}{2.397451in}}%
\pgfpathlineto{\pgfqpoint{5.549892in}{2.391635in}}%
\pgfpathlineto{\pgfqpoint{5.552469in}{2.398389in}}%
\pgfpathlineto{\pgfqpoint{5.555047in}{2.378688in}}%
\pgfpathlineto{\pgfqpoint{5.557624in}{2.374372in}}%
\pgfpathlineto{\pgfqpoint{5.560201in}{2.387507in}}%
\pgfpathlineto{\pgfqpoint{5.567933in}{2.373247in}}%
\pgfpathlineto{\pgfqpoint{5.570510in}{2.373434in}}%
\pgfpathlineto{\pgfqpoint{5.573087in}{2.413400in}}%
\pgfpathlineto{\pgfqpoint{5.575664in}{2.391259in}}%
\pgfpathlineto{\pgfqpoint{5.578242in}{2.415276in}}%
\pgfpathlineto{\pgfqpoint{5.585973in}{2.425972in}}%
\pgfpathlineto{\pgfqpoint{5.588550in}{2.424471in}}%
\pgfpathlineto{\pgfqpoint{5.593705in}{2.385818in}}%
\pgfpathlineto{\pgfqpoint{5.596282in}{2.392760in}}%
\pgfpathlineto{\pgfqpoint{5.604014in}{2.433477in}}%
\pgfpathlineto{\pgfqpoint{5.609168in}{2.427660in}}%
\pgfpathlineto{\pgfqpoint{5.611745in}{2.427097in}}%
\pgfpathlineto{\pgfqpoint{5.614323in}{2.430475in}}%
\pgfpathlineto{\pgfqpoint{5.624631in}{2.436479in}}%
\pgfpathlineto{\pgfqpoint{5.627209in}{2.422407in}}%
\pgfpathlineto{\pgfqpoint{5.629786in}{2.428786in}}%
\pgfpathlineto{\pgfqpoint{5.632363in}{2.412274in}}%
\pgfpathlineto{\pgfqpoint{5.642672in}{2.433665in}}%
\pgfpathlineto{\pgfqpoint{5.645249in}{2.434978in}}%
\pgfpathlineto{\pgfqpoint{5.647826in}{2.443046in}}%
\pgfpathlineto{\pgfqpoint{5.650403in}{2.464249in}}%
\pgfpathlineto{\pgfqpoint{5.658135in}{2.445673in}}%
\pgfpathlineto{\pgfqpoint{5.665867in}{2.433477in}}%
\pgfpathlineto{\pgfqpoint{5.668444in}{2.422969in}}%
\pgfpathlineto{\pgfqpoint{5.678753in}{2.418466in}}%
\pgfpathlineto{\pgfqpoint{5.681330in}{2.426534in}}%
\pgfpathlineto{\pgfqpoint{5.683907in}{2.431976in}}%
\pgfpathlineto{\pgfqpoint{5.686484in}{2.433102in}}%
\pgfpathlineto{\pgfqpoint{5.694216in}{2.425033in}}%
\pgfpathlineto{\pgfqpoint{5.696793in}{2.447549in}}%
\pgfpathlineto{\pgfqpoint{5.704525in}{2.413775in}}%
\pgfpathlineto{\pgfqpoint{5.712257in}{2.404394in}}%
\pgfpathlineto{\pgfqpoint{5.714834in}{2.413213in}}%
\pgfpathlineto{\pgfqpoint{5.717411in}{2.386756in}}%
\pgfpathlineto{\pgfqpoint{5.719988in}{2.389946in}}%
\pgfpathlineto{\pgfqpoint{5.722565in}{2.412837in}}%
\pgfpathlineto{\pgfqpoint{5.730297in}{2.431038in}}%
\pgfpathlineto{\pgfqpoint{5.732874in}{2.441170in}}%
\pgfpathlineto{\pgfqpoint{5.735452in}{2.426159in}}%
\pgfpathlineto{\pgfqpoint{5.738029in}{2.420718in}}%
\pgfpathlineto{\pgfqpoint{5.740606in}{2.437605in}}%
\pgfpathlineto{\pgfqpoint{5.748338in}{2.454867in}}%
\pgfpathlineto{\pgfqpoint{5.750915in}{2.446799in}}%
\pgfpathlineto{\pgfqpoint{5.753492in}{2.464624in}}%
\pgfpathlineto{\pgfqpoint{5.756069in}{2.467626in}}%
\pgfpathlineto{\pgfqpoint{5.758646in}{2.468940in}}%
\pgfpathlineto{\pgfqpoint{5.768955in}{2.475132in}}%
\pgfpathlineto{\pgfqpoint{5.771532in}{2.478134in}}%
\pgfpathlineto{\pgfqpoint{5.774110in}{2.467814in}}%
\pgfpathlineto{\pgfqpoint{5.776687in}{2.474569in}}%
\pgfpathlineto{\pgfqpoint{5.784419in}{2.480761in}}%
\pgfpathlineto{\pgfqpoint{5.786996in}{2.476070in}}%
\pgfpathlineto{\pgfqpoint{5.789573in}{2.496334in}}%
\pgfpathlineto{\pgfqpoint{5.792150in}{2.478509in}}%
\pgfpathlineto{\pgfqpoint{5.794727in}{2.472317in}}%
\pgfpathlineto{\pgfqpoint{5.802459in}{2.460309in}}%
\pgfpathlineto{\pgfqpoint{5.805036in}{2.471191in}}%
\pgfpathlineto{\pgfqpoint{5.807613in}{2.461810in}}%
\pgfpathlineto{\pgfqpoint{5.812768in}{2.468752in}}%
\pgfpathlineto{\pgfqpoint{5.820500in}{2.471754in}}%
\pgfpathlineto{\pgfqpoint{5.823077in}{2.463311in}}%
\pgfpathlineto{\pgfqpoint{5.825654in}{2.485639in}}%
\pgfpathlineto{\pgfqpoint{5.828231in}{2.472317in}}%
\pgfpathlineto{\pgfqpoint{5.830808in}{2.492206in}}%
\pgfpathlineto{\pgfqpoint{5.838540in}{2.493144in}}%
\pgfpathlineto{\pgfqpoint{5.841117in}{2.469127in}}%
\pgfpathlineto{\pgfqpoint{5.843694in}{2.464999in}}%
\pgfpathlineto{\pgfqpoint{5.846272in}{2.463498in}}%
\pgfpathlineto{\pgfqpoint{5.848849in}{2.462748in}}%
\pgfpathlineto{\pgfqpoint{5.856580in}{2.463311in}}%
\pgfpathlineto{\pgfqpoint{5.859158in}{2.479447in}}%
\pgfpathlineto{\pgfqpoint{5.861735in}{2.467439in}}%
\pgfpathlineto{\pgfqpoint{5.864312in}{2.474381in}}%
\pgfpathlineto{\pgfqpoint{5.866889in}{2.470065in}}%
\pgfpathlineto{\pgfqpoint{5.874621in}{2.464999in}}%
\pgfpathlineto{\pgfqpoint{5.877198in}{2.483950in}}%
\pgfpathlineto{\pgfqpoint{5.879775in}{2.475132in}}%
\pgfpathlineto{\pgfqpoint{5.882353in}{2.473631in}}%
\pgfpathlineto{\pgfqpoint{5.884930in}{2.483950in}}%
\pgfpathlineto{\pgfqpoint{5.892661in}{2.482074in}}%
\pgfpathlineto{\pgfqpoint{5.895239in}{2.486577in}}%
\pgfpathlineto{\pgfqpoint{5.897816in}{2.470628in}}%
\pgfpathlineto{\pgfqpoint{5.900393in}{2.467814in}}%
\pgfpathlineto{\pgfqpoint{5.913279in}{2.494083in}}%
\pgfpathlineto{\pgfqpoint{5.915856in}{2.482637in}}%
\pgfpathlineto{\pgfqpoint{5.921011in}{2.519601in}}%
\pgfpathlineto{\pgfqpoint{5.928742in}{2.543243in}}%
\pgfpathlineto{\pgfqpoint{5.931320in}{2.553187in}}%
\pgfpathlineto{\pgfqpoint{5.933897in}{2.576829in}}%
\pgfpathlineto{\pgfqpoint{5.936474in}{2.587149in}}%
\pgfpathlineto{\pgfqpoint{5.939051in}{2.590902in}}%
\pgfpathlineto{\pgfqpoint{5.946783in}{2.589588in}}%
\pgfpathlineto{\pgfqpoint{5.949360in}{2.596718in}}%
\pgfpathlineto{\pgfqpoint{5.951937in}{2.614356in}}%
\pgfpathlineto{\pgfqpoint{5.954515in}{2.626740in}}%
\pgfpathlineto{\pgfqpoint{5.957092in}{2.632931in}}%
\pgfpathlineto{\pgfqpoint{5.964823in}{2.628241in}}%
\pgfpathlineto{\pgfqpoint{5.967401in}{2.634995in}}%
\pgfpathlineto{\pgfqpoint{5.969978in}{2.627115in}}%
\pgfpathlineto{\pgfqpoint{5.972555in}{2.631806in}}%
\pgfpathlineto{\pgfqpoint{5.975132in}{2.623737in}}%
\pgfpathlineto{\pgfqpoint{5.982864in}{2.625238in}}%
\pgfpathlineto{\pgfqpoint{5.985441in}{2.633870in}}%
\pgfpathlineto{\pgfqpoint{5.988018in}{2.613042in}}%
\pgfpathlineto{\pgfqpoint{5.990596in}{2.609290in}}%
\pgfpathlineto{\pgfqpoint{5.993173in}{2.641375in}}%
\pgfpathlineto{\pgfqpoint{6.000904in}{2.650381in}}%
\pgfpathlineto{\pgfqpoint{6.003482in}{2.657699in}}%
\pgfpathlineto{\pgfqpoint{6.006059in}{2.657887in}}%
\pgfpathlineto{\pgfqpoint{6.008636in}{2.660889in}}%
\pgfpathlineto{\pgfqpoint{6.011213in}{2.653759in}}%
\pgfpathlineto{\pgfqpoint{6.021522in}{2.643439in}}%
\pgfpathlineto{\pgfqpoint{6.024099in}{2.643627in}}%
\pgfpathlineto{\pgfqpoint{6.026677in}{2.652633in}}%
\pgfpathlineto{\pgfqpoint{6.029254in}{2.658637in}}%
\pgfpathlineto{\pgfqpoint{6.036985in}{2.638185in}}%
\pgfpathlineto{\pgfqpoint{6.039563in}{2.622799in}}%
\pgfpathlineto{\pgfqpoint{6.042140in}{2.616607in}}%
\pgfpathlineto{\pgfqpoint{6.044717in}{2.619609in}}%
\pgfpathlineto{\pgfqpoint{6.047294in}{2.631993in}}%
\pgfpathlineto{\pgfqpoint{6.055026in}{2.618484in}}%
\pgfpathlineto{\pgfqpoint{6.057603in}{2.621673in}}%
\pgfpathlineto{\pgfqpoint{6.060180in}{2.620735in}}%
\pgfpathlineto{\pgfqpoint{6.062757in}{2.634057in}}%
\pgfpathlineto{\pgfqpoint{6.065335in}{2.628241in}}%
\pgfpathlineto{\pgfqpoint{6.073066in}{2.653759in}}%
\pgfpathlineto{\pgfqpoint{6.075644in}{2.647754in}}%
\pgfpathlineto{\pgfqpoint{6.078221in}{2.652070in}}%
\pgfpathlineto{\pgfqpoint{6.080798in}{2.659575in}}%
\pgfpathlineto{\pgfqpoint{6.083375in}{2.660326in}}%
\pgfpathlineto{\pgfqpoint{6.091107in}{2.654134in}}%
\pgfpathlineto{\pgfqpoint{6.093684in}{2.649068in}}%
\pgfpathlineto{\pgfqpoint{6.096261in}{2.665392in}}%
\pgfpathlineto{\pgfqpoint{6.098838in}{2.650194in}}%
\pgfpathlineto{\pgfqpoint{6.101416in}{2.658450in}}%
\pgfpathlineto{\pgfqpoint{6.109147in}{2.657887in}}%
\pgfpathlineto{\pgfqpoint{6.114302in}{2.670646in}}%
\pgfpathlineto{\pgfqpoint{6.116879in}{2.654509in}}%
\pgfpathlineto{\pgfqpoint{6.119456in}{2.667456in}}%
\pgfpathlineto{\pgfqpoint{6.127188in}{2.675524in}}%
\pgfpathlineto{\pgfqpoint{6.129765in}{2.684155in}}%
\pgfpathlineto{\pgfqpoint{6.132342in}{2.687158in}}%
\pgfpathlineto{\pgfqpoint{6.134919in}{2.675712in}}%
\pgfpathlineto{\pgfqpoint{6.137497in}{2.681716in}}%
\pgfpathlineto{\pgfqpoint{6.145228in}{2.676275in}}%
\pgfpathlineto{\pgfqpoint{6.147806in}{2.667831in}}%
\pgfpathlineto{\pgfqpoint{6.150383in}{2.675900in}}%
\pgfpathlineto{\pgfqpoint{6.152960in}{2.664454in}}%
\pgfpathlineto{\pgfqpoint{6.155537in}{2.683217in}}%
\pgfpathlineto{\pgfqpoint{6.163269in}{2.676650in}}%
\pgfpathlineto{\pgfqpoint{6.165846in}{2.628053in}}%
\pgfpathlineto{\pgfqpoint{6.171000in}{2.597093in}}%
\pgfpathlineto{\pgfqpoint{6.173578in}{2.600283in}}%
\pgfpathlineto{\pgfqpoint{6.181309in}{2.595029in}}%
\pgfpathlineto{\pgfqpoint{6.183886in}{2.600471in}}%
\pgfpathlineto{\pgfqpoint{6.186464in}{2.626927in}}%
\pgfpathlineto{\pgfqpoint{6.189041in}{2.640249in}}%
\pgfpathlineto{\pgfqpoint{6.191618in}{2.647379in}}%
\pgfpathlineto{\pgfqpoint{6.199350in}{2.594091in}}%
\pgfpathlineto{\pgfqpoint{6.201927in}{2.588087in}}%
\pgfpathlineto{\pgfqpoint{6.204504in}{2.570449in}}%
\pgfpathlineto{\pgfqpoint{6.207081in}{2.562569in}}%
\pgfpathlineto{\pgfqpoint{6.209659in}{2.564820in}}%
\pgfpathlineto{\pgfqpoint{6.217390in}{2.569511in}}%
\pgfpathlineto{\pgfqpoint{6.219967in}{2.548121in}}%
\pgfpathlineto{\pgfqpoint{6.222545in}{2.597656in}}%
\pgfpathlineto{\pgfqpoint{6.225122in}{2.563319in}}%
\pgfpathlineto{\pgfqpoint{6.227699in}{2.552249in}}%
\pgfpathlineto{\pgfqpoint{6.235431in}{2.548496in}}%
\pgfpathlineto{\pgfqpoint{6.238008in}{2.556002in}}%
\pgfpathlineto{\pgfqpoint{6.240585in}{2.580206in}}%
\pgfpathlineto{\pgfqpoint{6.243162in}{2.548872in}}%
\pgfpathlineto{\pgfqpoint{6.245740in}{2.544744in}}%
\pgfpathlineto{\pgfqpoint{6.253471in}{2.549434in}}%
\pgfpathlineto{\pgfqpoint{6.256048in}{2.610228in}}%
\pgfpathlineto{\pgfqpoint{6.258626in}{2.626552in}}%
\pgfpathlineto{\pgfqpoint{6.261203in}{2.628616in}}%
\pgfpathlineto{\pgfqpoint{6.263780in}{2.596155in}}%
\pgfpathlineto{\pgfqpoint{6.276666in}{2.446236in}}%
\pgfpathlineto{\pgfqpoint{6.279243in}{2.452991in}}%
\pgfpathlineto{\pgfqpoint{6.281821in}{2.445298in}}%
\pgfpathlineto{\pgfqpoint{6.289552in}{2.446799in}}%
\pgfpathlineto{\pgfqpoint{6.292129in}{2.450739in}}%
\pgfpathlineto{\pgfqpoint{6.294707in}{2.458432in}}%
\pgfpathlineto{\pgfqpoint{6.297284in}{2.510031in}}%
\pgfpathlineto{\pgfqpoint{6.299861in}{2.508906in}}%
\pgfpathlineto{\pgfqpoint{6.307593in}{2.503277in}}%
\pgfpathlineto{\pgfqpoint{6.310170in}{2.519225in}}%
\pgfpathlineto{\pgfqpoint{6.317902in}{2.543618in}}%
\pgfpathlineto{\pgfqpoint{6.325633in}{2.530671in}}%
\pgfpathlineto{\pgfqpoint{6.328210in}{2.538364in}}%
\pgfpathlineto{\pgfqpoint{6.330788in}{2.582833in}}%
\pgfpathlineto{\pgfqpoint{6.333365in}{2.558629in}}%
\pgfpathlineto{\pgfqpoint{6.335942in}{2.562944in}}%
\pgfpathlineto{\pgfqpoint{6.343674in}{2.589963in}}%
\pgfpathlineto{\pgfqpoint{6.346251in}{2.592403in}}%
\pgfpathlineto{\pgfqpoint{6.348828in}{2.590902in}}%
\pgfpathlineto{\pgfqpoint{6.351405in}{2.600658in}}%
\pgfpathlineto{\pgfqpoint{6.353982in}{2.601784in}}%
\pgfpathlineto{\pgfqpoint{6.361714in}{2.608914in}}%
\pgfpathlineto{\pgfqpoint{6.364291in}{2.599720in}}%
\pgfpathlineto{\pgfqpoint{6.366869in}{2.593153in}}%
\pgfpathlineto{\pgfqpoint{6.369446in}{2.612479in}}%
\pgfpathlineto{\pgfqpoint{6.372023in}{2.610603in}}%
\pgfpathlineto{\pgfqpoint{6.379755in}{2.615857in}}%
\pgfpathlineto{\pgfqpoint{6.382332in}{2.622049in}}%
\pgfpathlineto{\pgfqpoint{6.384909in}{2.618859in}}%
\pgfpathlineto{\pgfqpoint{6.387486in}{2.624488in}}%
\pgfpathlineto{\pgfqpoint{6.390063in}{2.650381in}}%
\pgfpathlineto{\pgfqpoint{6.397795in}{2.649818in}}%
\pgfpathlineto{\pgfqpoint{6.400372in}{2.628991in}}%
\pgfpathlineto{\pgfqpoint{6.402950in}{2.615482in}}%
\pgfpathlineto{\pgfqpoint{6.405527in}{2.632369in}}%
\pgfpathlineto{\pgfqpoint{6.408104in}{2.616983in}}%
\pgfpathlineto{\pgfqpoint{6.415836in}{2.630680in}}%
\pgfpathlineto{\pgfqpoint{6.418413in}{2.629366in}}%
\pgfpathlineto{\pgfqpoint{6.420990in}{2.635934in}}%
\pgfpathlineto{\pgfqpoint{6.423567in}{2.659951in}}%
\pgfpathlineto{\pgfqpoint{6.426144in}{2.653008in}}%
\pgfpathlineto{\pgfqpoint{6.433876in}{2.640624in}}%
\pgfpathlineto{\pgfqpoint{6.436453in}{2.647192in}}%
\pgfpathlineto{\pgfqpoint{6.439031in}{2.638748in}}%
\pgfpathlineto{\pgfqpoint{6.441608in}{2.604223in}}%
\pgfpathlineto{\pgfqpoint{6.444185in}{2.599533in}}%
\pgfpathlineto{\pgfqpoint{6.451917in}{2.580769in}}%
\pgfpathlineto{\pgfqpoint{6.454494in}{2.612104in}}%
\pgfpathlineto{\pgfqpoint{6.457071in}{2.589776in}}%
\pgfpathlineto{\pgfqpoint{6.459648in}{2.608351in}}%
\pgfpathlineto{\pgfqpoint{6.462225in}{2.583771in}}%
\pgfpathlineto{\pgfqpoint{6.469957in}{2.580957in}}%
\pgfpathlineto{\pgfqpoint{6.472534in}{2.592965in}}%
\pgfpathlineto{\pgfqpoint{6.475111in}{2.587336in}}%
\pgfpathlineto{\pgfqpoint{6.480266in}{2.590526in}}%
\pgfpathlineto{\pgfqpoint{6.487998in}{2.594467in}}%
\pgfpathlineto{\pgfqpoint{6.490575in}{2.605162in}}%
\pgfpathlineto{\pgfqpoint{6.493152in}{2.613230in}}%
\pgfpathlineto{\pgfqpoint{6.495729in}{2.672897in}}%
\pgfpathlineto{\pgfqpoint{6.498306in}{2.648880in}}%
\pgfpathlineto{\pgfqpoint{6.506038in}{2.647379in}}%
\pgfpathlineto{\pgfqpoint{6.508615in}{2.651882in}}%
\pgfpathlineto{\pgfqpoint{6.511192in}{2.668394in}}%
\pgfpathlineto{\pgfqpoint{6.513770in}{2.690160in}}%
\pgfpathlineto{\pgfqpoint{6.516347in}{2.697665in}}%
\pgfpathlineto{\pgfqpoint{6.526656in}{2.709861in}}%
\pgfpathlineto{\pgfqpoint{6.529233in}{2.724684in}}%
\pgfpathlineto{\pgfqpoint{6.531810in}{2.714927in}}%
\pgfpathlineto{\pgfqpoint{6.534387in}{2.758646in}}%
\pgfpathlineto{\pgfqpoint{6.542119in}{2.768403in}}%
\pgfpathlineto{\pgfqpoint{6.544696in}{2.769716in}}%
\pgfpathlineto{\pgfqpoint{6.547273in}{2.773657in}}%
\pgfpathlineto{\pgfqpoint{6.549851in}{2.779286in}}%
\pgfpathlineto{\pgfqpoint{6.552428in}{2.777785in}}%
\pgfpathlineto{\pgfqpoint{6.562737in}{2.776096in}}%
\pgfpathlineto{\pgfqpoint{6.567891in}{2.793921in}}%
\pgfpathlineto{\pgfqpoint{6.570468in}{2.783976in}}%
\pgfpathlineto{\pgfqpoint{6.570468in}{2.783976in}}%
\pgfusepath{stroke}%
\end{pgfscope}%
\begin{pgfscope}%
\pgfpathrectangle{\pgfqpoint{0.652287in}{0.521603in}}{\pgfqpoint{6.200000in}{4.530000in}}%
\pgfusepath{clip}%
\pgfsetroundcap%
\pgfsetroundjoin%
\pgfsetlinewidth{1.505625pt}%
\definecolor{currentstroke}{rgb}{0.498039,0.498039,0.498039}%
\pgfsetstrokecolor{currentstroke}%
\pgfsetdash{}{0pt}%
\pgfpathmoveto{\pgfqpoint{0.934105in}{0.993764in}}%
\pgfpathlineto{\pgfqpoint{0.936682in}{0.987010in}}%
\pgfpathlineto{\pgfqpoint{0.939259in}{0.983445in}}%
\pgfpathlineto{\pgfqpoint{0.941836in}{0.981944in}}%
\pgfpathlineto{\pgfqpoint{0.949568in}{0.982506in}}%
\pgfpathlineto{\pgfqpoint{0.952145in}{0.985133in}}%
\pgfpathlineto{\pgfqpoint{0.954722in}{0.989637in}}%
\pgfpathlineto{\pgfqpoint{0.975340in}{0.990950in}}%
\pgfpathlineto{\pgfqpoint{0.977917in}{0.990575in}}%
\pgfpathlineto{\pgfqpoint{0.985649in}{0.982882in}}%
\pgfpathlineto{\pgfqpoint{0.988226in}{0.974813in}}%
\pgfpathlineto{\pgfqpoint{0.990803in}{0.973312in}}%
\pgfpathlineto{\pgfqpoint{0.993381in}{0.968622in}}%
\pgfpathlineto{\pgfqpoint{0.995958in}{0.966933in}}%
\pgfpathlineto{\pgfqpoint{1.003689in}{0.972374in}}%
\pgfpathlineto{\pgfqpoint{1.006267in}{0.972937in}}%
\pgfpathlineto{\pgfqpoint{1.008844in}{0.974813in}}%
\pgfpathlineto{\pgfqpoint{1.011421in}{0.971624in}}%
\pgfpathlineto{\pgfqpoint{1.013998in}{0.975376in}}%
\pgfpathlineto{\pgfqpoint{1.021730in}{0.979317in}}%
\pgfpathlineto{\pgfqpoint{1.024307in}{0.976502in}}%
\pgfpathlineto{\pgfqpoint{1.029462in}{0.976502in}}%
\pgfpathlineto{\pgfqpoint{1.032039in}{0.973312in}}%
\pgfpathlineto{\pgfqpoint{1.039770in}{0.979317in}}%
\pgfpathlineto{\pgfqpoint{1.042348in}{0.978003in}}%
\pgfpathlineto{\pgfqpoint{1.044925in}{0.975189in}}%
\pgfpathlineto{\pgfqpoint{1.047502in}{0.978191in}}%
\pgfpathlineto{\pgfqpoint{1.050079in}{0.983632in}}%
\pgfpathlineto{\pgfqpoint{1.060388in}{0.984008in}}%
\pgfpathlineto{\pgfqpoint{1.062965in}{0.980255in}}%
\pgfpathlineto{\pgfqpoint{1.065543in}{0.979317in}}%
\pgfpathlineto{\pgfqpoint{1.081006in}{0.978941in}}%
\pgfpathlineto{\pgfqpoint{1.083583in}{0.983257in}}%
\pgfpathlineto{\pgfqpoint{1.086160in}{0.986447in}}%
\pgfpathlineto{\pgfqpoint{1.093892in}{0.990950in}}%
\pgfpathlineto{\pgfqpoint{1.096469in}{0.986822in}}%
\pgfpathlineto{\pgfqpoint{1.099046in}{0.989261in}}%
\pgfpathlineto{\pgfqpoint{1.101624in}{0.993764in}}%
\pgfpathlineto{\pgfqpoint{1.104201in}{0.992263in}}%
\pgfpathlineto{\pgfqpoint{1.111932in}{0.995266in}}%
\pgfpathlineto{\pgfqpoint{1.114510in}{0.997330in}}%
\pgfpathlineto{\pgfqpoint{1.117087in}{0.997142in}}%
\pgfpathlineto{\pgfqpoint{1.122241in}{0.998455in}}%
\pgfpathlineto{\pgfqpoint{1.129973in}{0.999581in}}%
\pgfpathlineto{\pgfqpoint{1.132550in}{0.999393in}}%
\pgfpathlineto{\pgfqpoint{1.135127in}{1.001270in}}%
\pgfpathlineto{\pgfqpoint{1.137704in}{0.999769in}}%
\pgfpathlineto{\pgfqpoint{1.140282in}{0.996579in}}%
\pgfpathlineto{\pgfqpoint{1.148013in}{0.995266in}}%
\pgfpathlineto{\pgfqpoint{1.150591in}{0.986447in}}%
\pgfpathlineto{\pgfqpoint{1.153168in}{0.981005in}}%
\pgfpathlineto{\pgfqpoint{1.155745in}{0.978566in}}%
\pgfpathlineto{\pgfqpoint{1.158322in}{0.980630in}}%
\pgfpathlineto{\pgfqpoint{1.166054in}{0.984570in}}%
\pgfpathlineto{\pgfqpoint{1.171208in}{0.982694in}}%
\pgfpathlineto{\pgfqpoint{1.173785in}{0.979692in}}%
\pgfpathlineto{\pgfqpoint{1.184094in}{0.976877in}}%
\pgfpathlineto{\pgfqpoint{1.186672in}{0.968059in}}%
\pgfpathlineto{\pgfqpoint{1.189249in}{0.975752in}}%
\pgfpathlineto{\pgfqpoint{1.191826in}{0.978191in}}%
\pgfpathlineto{\pgfqpoint{1.194403in}{0.974251in}}%
\pgfpathlineto{\pgfqpoint{1.202135in}{0.976502in}}%
\pgfpathlineto{\pgfqpoint{1.204712in}{0.980630in}}%
\pgfpathlineto{\pgfqpoint{1.207289in}{0.979692in}}%
\pgfpathlineto{\pgfqpoint{1.212444in}{0.994140in}}%
\pgfpathlineto{\pgfqpoint{1.220175in}{0.991888in}}%
\pgfpathlineto{\pgfqpoint{1.222753in}{1.004647in}}%
\pgfpathlineto{\pgfqpoint{1.225330in}{1.004272in}}%
\pgfpathlineto{\pgfqpoint{1.227907in}{1.013278in}}%
\pgfpathlineto{\pgfqpoint{1.230484in}{1.014404in}}%
\pgfpathlineto{\pgfqpoint{1.238216in}{1.016468in}}%
\pgfpathlineto{\pgfqpoint{1.240793in}{1.018907in}}%
\pgfpathlineto{\pgfqpoint{1.245947in}{1.020033in}}%
\pgfpathlineto{\pgfqpoint{1.248525in}{1.014779in}}%
\pgfpathlineto{\pgfqpoint{1.256256in}{1.018532in}}%
\pgfpathlineto{\pgfqpoint{1.258833in}{1.018720in}}%
\pgfpathlineto{\pgfqpoint{1.261411in}{1.014779in}}%
\pgfpathlineto{\pgfqpoint{1.263988in}{1.018720in}}%
\pgfpathlineto{\pgfqpoint{1.266565in}{1.026976in}}%
\pgfpathlineto{\pgfqpoint{1.274297in}{1.023411in}}%
\pgfpathlineto{\pgfqpoint{1.276874in}{1.025475in}}%
\pgfpathlineto{\pgfqpoint{1.279451in}{1.023223in}}%
\pgfpathlineto{\pgfqpoint{1.282028in}{1.029978in}}%
\pgfpathlineto{\pgfqpoint{1.284606in}{1.032042in}}%
\pgfpathlineto{\pgfqpoint{1.292337in}{1.029415in}}%
\pgfpathlineto{\pgfqpoint{1.294914in}{1.030165in}}%
\pgfpathlineto{\pgfqpoint{1.297492in}{1.028664in}}%
\pgfpathlineto{\pgfqpoint{1.300069in}{1.030165in}}%
\pgfpathlineto{\pgfqpoint{1.302646in}{1.030916in}}%
\pgfpathlineto{\pgfqpoint{1.312955in}{1.035044in}}%
\pgfpathlineto{\pgfqpoint{1.315532in}{1.030541in}}%
\pgfpathlineto{\pgfqpoint{1.318109in}{1.033543in}}%
\pgfpathlineto{\pgfqpoint{1.320687in}{1.025287in}}%
\pgfpathlineto{\pgfqpoint{1.328418in}{1.029415in}}%
\pgfpathlineto{\pgfqpoint{1.330995in}{1.027539in}}%
\pgfpathlineto{\pgfqpoint{1.333573in}{1.035231in}}%
\pgfpathlineto{\pgfqpoint{1.336150in}{1.033543in}}%
\pgfpathlineto{\pgfqpoint{1.338727in}{1.044426in}}%
\pgfpathlineto{\pgfqpoint{1.346459in}{1.045927in}}%
\pgfpathlineto{\pgfqpoint{1.349036in}{1.051180in}}%
\pgfpathlineto{\pgfqpoint{1.351613in}{1.051743in}}%
\pgfpathlineto{\pgfqpoint{1.354190in}{1.062063in}}%
\pgfpathlineto{\pgfqpoint{1.356768in}{1.059436in}}%
\pgfpathlineto{\pgfqpoint{1.364499in}{1.063001in}}%
\pgfpathlineto{\pgfqpoint{1.367076in}{1.061875in}}%
\pgfpathlineto{\pgfqpoint{1.369654in}{1.056059in}}%
\pgfpathlineto{\pgfqpoint{1.372231in}{1.056434in}}%
\pgfpathlineto{\pgfqpoint{1.374808in}{1.064878in}}%
\pgfpathlineto{\pgfqpoint{1.382540in}{1.060750in}}%
\pgfpathlineto{\pgfqpoint{1.385117in}{1.063377in}}%
\pgfpathlineto{\pgfqpoint{1.387694in}{1.062626in}}%
\pgfpathlineto{\pgfqpoint{1.390271in}{1.065065in}}%
\pgfpathlineto{\pgfqpoint{1.392849in}{1.071445in}}%
\pgfpathlineto{\pgfqpoint{1.400580in}{1.078200in}}%
\pgfpathlineto{\pgfqpoint{1.403157in}{1.078387in}}%
\pgfpathlineto{\pgfqpoint{1.408312in}{1.077637in}}%
\pgfpathlineto{\pgfqpoint{1.410889in}{1.078012in}}%
\pgfpathlineto{\pgfqpoint{1.418621in}{1.082328in}}%
\pgfpathlineto{\pgfqpoint{1.421198in}{1.081765in}}%
\pgfpathlineto{\pgfqpoint{1.423775in}{1.084579in}}%
\pgfpathlineto{\pgfqpoint{1.426352in}{1.081389in}}%
\pgfpathlineto{\pgfqpoint{1.428930in}{1.088895in}}%
\pgfpathlineto{\pgfqpoint{1.436661in}{1.089645in}}%
\pgfpathlineto{\pgfqpoint{1.439238in}{1.094899in}}%
\pgfpathlineto{\pgfqpoint{1.441816in}{1.098089in}}%
\pgfpathlineto{\pgfqpoint{1.444393in}{1.079701in}}%
\pgfpathlineto{\pgfqpoint{1.454702in}{1.077074in}}%
\pgfpathlineto{\pgfqpoint{1.457279in}{1.069006in}}%
\pgfpathlineto{\pgfqpoint{1.459856in}{1.068443in}}%
\pgfpathlineto{\pgfqpoint{1.462433in}{1.078575in}}%
\pgfpathlineto{\pgfqpoint{1.465010in}{1.084579in}}%
\pgfpathlineto{\pgfqpoint{1.472742in}{1.085330in}}%
\pgfpathlineto{\pgfqpoint{1.475319in}{1.087769in}}%
\pgfpathlineto{\pgfqpoint{1.477897in}{1.088895in}}%
\pgfpathlineto{\pgfqpoint{1.480474in}{1.080639in}}%
\pgfpathlineto{\pgfqpoint{1.483051in}{1.078575in}}%
\pgfpathlineto{\pgfqpoint{1.490783in}{1.081765in}}%
\pgfpathlineto{\pgfqpoint{1.495937in}{1.074635in}}%
\pgfpathlineto{\pgfqpoint{1.498514in}{1.076886in}}%
\pgfpathlineto{\pgfqpoint{1.501091in}{1.080451in}}%
\pgfpathlineto{\pgfqpoint{1.508823in}{1.075948in}}%
\pgfpathlineto{\pgfqpoint{1.511400in}{1.076699in}}%
\pgfpathlineto{\pgfqpoint{1.513978in}{1.074822in}}%
\pgfpathlineto{\pgfqpoint{1.519132in}{1.073133in}}%
\pgfpathlineto{\pgfqpoint{1.526864in}{1.068067in}}%
\pgfpathlineto{\pgfqpoint{1.529441in}{1.056997in}}%
\pgfpathlineto{\pgfqpoint{1.532018in}{1.053244in}}%
\pgfpathlineto{\pgfqpoint{1.534595in}{1.048178in}}%
\pgfpathlineto{\pgfqpoint{1.537172in}{1.060937in}}%
\pgfpathlineto{\pgfqpoint{1.547481in}{1.053995in}}%
\pgfpathlineto{\pgfqpoint{1.550058in}{1.059811in}}%
\pgfpathlineto{\pgfqpoint{1.552636in}{1.055308in}}%
\pgfpathlineto{\pgfqpoint{1.568099in}{1.069568in}}%
\pgfpathlineto{\pgfqpoint{1.570676in}{1.074259in}}%
\pgfpathlineto{\pgfqpoint{1.573253in}{1.068443in}}%
\pgfpathlineto{\pgfqpoint{1.580985in}{1.073133in}}%
\pgfpathlineto{\pgfqpoint{1.583562in}{1.075573in}}%
\pgfpathlineto{\pgfqpoint{1.588717in}{1.093961in}}%
\pgfpathlineto{\pgfqpoint{1.591294in}{1.079513in}}%
\pgfpathlineto{\pgfqpoint{1.599026in}{1.080264in}}%
\pgfpathlineto{\pgfqpoint{1.604180in}{1.089645in}}%
\pgfpathlineto{\pgfqpoint{1.606757in}{1.092647in}}%
\pgfpathlineto{\pgfqpoint{1.609334in}{1.094711in}}%
\pgfpathlineto{\pgfqpoint{1.617066in}{1.095274in}}%
\pgfpathlineto{\pgfqpoint{1.622220in}{1.093961in}}%
\pgfpathlineto{\pgfqpoint{1.624798in}{1.096400in}}%
\pgfpathlineto{\pgfqpoint{1.627375in}{1.093773in}}%
\pgfpathlineto{\pgfqpoint{1.637684in}{1.097713in}}%
\pgfpathlineto{\pgfqpoint{1.640261in}{1.103530in}}%
\pgfpathlineto{\pgfqpoint{1.642838in}{1.116852in}}%
\pgfpathlineto{\pgfqpoint{1.645415in}{1.121168in}}%
\pgfpathlineto{\pgfqpoint{1.653147in}{1.114413in}}%
\pgfpathlineto{\pgfqpoint{1.655724in}{1.108033in}}%
\pgfpathlineto{\pgfqpoint{1.658301in}{1.103530in}}%
\pgfpathlineto{\pgfqpoint{1.663456in}{1.087394in}}%
\pgfpathlineto{\pgfqpoint{1.671187in}{1.085705in}}%
\pgfpathlineto{\pgfqpoint{1.673765in}{1.079888in}}%
\pgfpathlineto{\pgfqpoint{1.676342in}{1.088895in}}%
\pgfpathlineto{\pgfqpoint{1.678919in}{1.103530in}}%
\pgfpathlineto{\pgfqpoint{1.681496in}{1.094899in}}%
\pgfpathlineto{\pgfqpoint{1.689228in}{1.089458in}}%
\pgfpathlineto{\pgfqpoint{1.691805in}{1.079888in}}%
\pgfpathlineto{\pgfqpoint{1.694382in}{1.082140in}}%
\pgfpathlineto{\pgfqpoint{1.696960in}{1.082328in}}%
\pgfpathlineto{\pgfqpoint{1.699537in}{1.088895in}}%
\pgfpathlineto{\pgfqpoint{1.712423in}{1.087769in}}%
\pgfpathlineto{\pgfqpoint{1.715000in}{1.094711in}}%
\pgfpathlineto{\pgfqpoint{1.717577in}{1.086080in}}%
\pgfpathlineto{\pgfqpoint{1.725309in}{1.081577in}}%
\pgfpathlineto{\pgfqpoint{1.727886in}{1.083078in}}%
\pgfpathlineto{\pgfqpoint{1.730463in}{1.067692in}}%
\pgfpathlineto{\pgfqpoint{1.733041in}{1.059624in}}%
\pgfpathlineto{\pgfqpoint{1.735618in}{1.059999in}}%
\pgfpathlineto{\pgfqpoint{1.745927in}{1.058686in}}%
\pgfpathlineto{\pgfqpoint{1.748504in}{1.054558in}}%
\pgfpathlineto{\pgfqpoint{1.751081in}{1.047052in}}%
\pgfpathlineto{\pgfqpoint{1.753658in}{1.042924in}}%
\pgfpathlineto{\pgfqpoint{1.761390in}{1.062438in}}%
\pgfpathlineto{\pgfqpoint{1.763967in}{1.062438in}}%
\pgfpathlineto{\pgfqpoint{1.766544in}{1.067317in}}%
\pgfpathlineto{\pgfqpoint{1.771699in}{1.075573in}}%
\pgfpathlineto{\pgfqpoint{1.779430in}{1.069193in}}%
\pgfpathlineto{\pgfqpoint{1.782008in}{1.064690in}}%
\pgfpathlineto{\pgfqpoint{1.784585in}{1.072758in}}%
\pgfpathlineto{\pgfqpoint{1.787162in}{1.078950in}}%
\pgfpathlineto{\pgfqpoint{1.789739in}{1.080451in}}%
\pgfpathlineto{\pgfqpoint{1.797471in}{1.080264in}}%
\pgfpathlineto{\pgfqpoint{1.800048in}{1.074259in}}%
\pgfpathlineto{\pgfqpoint{1.805203in}{1.085142in}}%
\pgfpathlineto{\pgfqpoint{1.807780in}{1.084579in}}%
\pgfpathlineto{\pgfqpoint{1.815511in}{1.079325in}}%
\pgfpathlineto{\pgfqpoint{1.820666in}{1.089833in}}%
\pgfpathlineto{\pgfqpoint{1.825820in}{1.081765in}}%
\pgfpathlineto{\pgfqpoint{1.833552in}{1.080451in}}%
\pgfpathlineto{\pgfqpoint{1.836129in}{1.077074in}}%
\pgfpathlineto{\pgfqpoint{1.838706in}{1.070319in}}%
\pgfpathlineto{\pgfqpoint{1.841283in}{1.076323in}}%
\pgfpathlineto{\pgfqpoint{1.843861in}{1.072946in}}%
\pgfpathlineto{\pgfqpoint{1.851592in}{1.072571in}}%
\pgfpathlineto{\pgfqpoint{1.856747in}{1.071257in}}%
\pgfpathlineto{\pgfqpoint{1.859324in}{1.071632in}}%
\pgfpathlineto{\pgfqpoint{1.861901in}{1.063564in}}%
\pgfpathlineto{\pgfqpoint{1.869633in}{1.068818in}}%
\pgfpathlineto{\pgfqpoint{1.874787in}{1.082515in}}%
\pgfpathlineto{\pgfqpoint{1.877364in}{1.079701in}}%
\pgfpathlineto{\pgfqpoint{1.879942in}{1.083078in}}%
\pgfpathlineto{\pgfqpoint{1.887673in}{1.088332in}}%
\pgfpathlineto{\pgfqpoint{1.890251in}{1.073321in}}%
\pgfpathlineto{\pgfqpoint{1.892828in}{1.072008in}}%
\pgfpathlineto{\pgfqpoint{1.895405in}{1.080264in}}%
\pgfpathlineto{\pgfqpoint{1.897982in}{1.076136in}}%
\pgfpathlineto{\pgfqpoint{1.905714in}{1.066191in}}%
\pgfpathlineto{\pgfqpoint{1.908291in}{1.057560in}}%
\pgfpathlineto{\pgfqpoint{1.910868in}{1.050993in}}%
\pgfpathlineto{\pgfqpoint{1.913445in}{1.059811in}}%
\pgfpathlineto{\pgfqpoint{1.916023in}{1.065441in}}%
\pgfpathlineto{\pgfqpoint{1.926332in}{1.071070in}}%
\pgfpathlineto{\pgfqpoint{1.928909in}{1.069006in}}%
\pgfpathlineto{\pgfqpoint{1.931486in}{1.066191in}}%
\pgfpathlineto{\pgfqpoint{1.934063in}{1.067317in}}%
\pgfpathlineto{\pgfqpoint{1.941795in}{1.068818in}}%
\pgfpathlineto{\pgfqpoint{1.944372in}{1.078950in}}%
\pgfpathlineto{\pgfqpoint{1.946949in}{1.080451in}}%
\pgfpathlineto{\pgfqpoint{1.949526in}{1.080451in}}%
\pgfpathlineto{\pgfqpoint{1.952104in}{1.093773in}}%
\pgfpathlineto{\pgfqpoint{1.959835in}{1.093210in}}%
\pgfpathlineto{\pgfqpoint{1.962412in}{1.093773in}}%
\pgfpathlineto{\pgfqpoint{1.964990in}{1.096400in}}%
\pgfpathlineto{\pgfqpoint{1.967567in}{1.092460in}}%
\pgfpathlineto{\pgfqpoint{1.970144in}{1.090959in}}%
\pgfpathlineto{\pgfqpoint{1.977876in}{1.090396in}}%
\pgfpathlineto{\pgfqpoint{1.983030in}{1.093210in}}%
\pgfpathlineto{\pgfqpoint{1.985607in}{1.090583in}}%
\pgfpathlineto{\pgfqpoint{1.988185in}{1.091522in}}%
\pgfpathlineto{\pgfqpoint{1.998493in}{1.093023in}}%
\pgfpathlineto{\pgfqpoint{2.001071in}{1.098839in}}%
\pgfpathlineto{\pgfqpoint{2.003648in}{1.101654in}}%
\pgfpathlineto{\pgfqpoint{2.006225in}{1.105594in}}%
\pgfpathlineto{\pgfqpoint{2.013957in}{1.110097in}}%
\pgfpathlineto{\pgfqpoint{2.016534in}{1.115726in}}%
\pgfpathlineto{\pgfqpoint{2.021688in}{1.121355in}}%
\pgfpathlineto{\pgfqpoint{2.024266in}{1.124170in}}%
\pgfpathlineto{\pgfqpoint{2.031997in}{1.129611in}}%
\pgfpathlineto{\pgfqpoint{2.034574in}{1.137679in}}%
\pgfpathlineto{\pgfqpoint{2.037152in}{1.131863in}}%
\pgfpathlineto{\pgfqpoint{2.039729in}{1.134677in}}%
\pgfpathlineto{\pgfqpoint{2.042306in}{1.141432in}}%
\pgfpathlineto{\pgfqpoint{2.050038in}{1.139368in}}%
\pgfpathlineto{\pgfqpoint{2.052615in}{1.147061in}}%
\pgfpathlineto{\pgfqpoint{2.055192in}{1.141244in}}%
\pgfpathlineto{\pgfqpoint{2.057769in}{1.148750in}}%
\pgfpathlineto{\pgfqpoint{2.060347in}{1.142370in}}%
\pgfpathlineto{\pgfqpoint{2.070655in}{1.155317in}}%
\pgfpathlineto{\pgfqpoint{2.073233in}{1.150439in}}%
\pgfpathlineto{\pgfqpoint{2.078387in}{1.156255in}}%
\pgfpathlineto{\pgfqpoint{2.086119in}{1.158319in}}%
\pgfpathlineto{\pgfqpoint{2.088696in}{1.162822in}}%
\pgfpathlineto{\pgfqpoint{2.091273in}{1.155129in}}%
\pgfpathlineto{\pgfqpoint{2.093850in}{1.158131in}}%
\pgfpathlineto{\pgfqpoint{2.104159in}{1.159070in}}%
\pgfpathlineto{\pgfqpoint{2.106736in}{1.163010in}}%
\pgfpathlineto{\pgfqpoint{2.109314in}{1.155880in}}%
\pgfpathlineto{\pgfqpoint{2.114468in}{1.163948in}}%
\pgfpathlineto{\pgfqpoint{2.122200in}{1.169390in}}%
\pgfpathlineto{\pgfqpoint{2.124777in}{1.168264in}}%
\pgfpathlineto{\pgfqpoint{2.127354in}{1.175394in}}%
\pgfpathlineto{\pgfqpoint{2.129931in}{1.184588in}}%
\pgfpathlineto{\pgfqpoint{2.132509in}{1.189654in}}%
\pgfpathlineto{\pgfqpoint{2.140240in}{1.186464in}}%
\pgfpathlineto{\pgfqpoint{2.142817in}{1.183837in}}%
\pgfpathlineto{\pgfqpoint{2.145395in}{1.170891in}}%
\pgfpathlineto{\pgfqpoint{2.150549in}{1.209168in}}%
\pgfpathlineto{\pgfqpoint{2.158281in}{1.208417in}}%
\pgfpathlineto{\pgfqpoint{2.160858in}{1.210294in}}%
\pgfpathlineto{\pgfqpoint{2.163435in}{1.202788in}}%
\pgfpathlineto{\pgfqpoint{2.166012in}{1.223053in}}%
\pgfpathlineto{\pgfqpoint{2.168589in}{1.228869in}}%
\pgfpathlineto{\pgfqpoint{2.176321in}{1.226430in}}%
\pgfpathlineto{\pgfqpoint{2.178898in}{1.232810in}}%
\pgfpathlineto{\pgfqpoint{2.181476in}{1.211419in}}%
\pgfpathlineto{\pgfqpoint{2.186630in}{1.215360in}}%
\pgfpathlineto{\pgfqpoint{2.194362in}{1.206353in}}%
\pgfpathlineto{\pgfqpoint{2.196939in}{1.218737in}}%
\pgfpathlineto{\pgfqpoint{2.199516in}{1.221364in}}%
\pgfpathlineto{\pgfqpoint{2.202093in}{1.215735in}}%
\pgfpathlineto{\pgfqpoint{2.204670in}{1.218362in}}%
\pgfpathlineto{\pgfqpoint{2.212402in}{1.213483in}}%
\pgfpathlineto{\pgfqpoint{2.214979in}{1.222302in}}%
\pgfpathlineto{\pgfqpoint{2.217557in}{1.228306in}}%
\pgfpathlineto{\pgfqpoint{2.220134in}{1.222677in}}%
\pgfpathlineto{\pgfqpoint{2.222711in}{1.224741in}}%
\pgfpathlineto{\pgfqpoint{2.230443in}{1.216110in}}%
\pgfpathlineto{\pgfqpoint{2.235597in}{1.198285in}}%
\pgfpathlineto{\pgfqpoint{2.238174in}{1.204102in}}%
\pgfpathlineto{\pgfqpoint{2.240751in}{1.197159in}}%
\pgfpathlineto{\pgfqpoint{2.251060in}{1.189091in}}%
\pgfpathlineto{\pgfqpoint{2.253637in}{1.171266in}}%
\pgfpathlineto{\pgfqpoint{2.258792in}{1.155880in}}%
\pgfpathlineto{\pgfqpoint{2.266524in}{1.158507in}}%
\pgfpathlineto{\pgfqpoint{2.269101in}{1.160946in}}%
\pgfpathlineto{\pgfqpoint{2.271678in}{1.153253in}}%
\pgfpathlineto{\pgfqpoint{2.274255in}{1.176895in}}%
\pgfpathlineto{\pgfqpoint{2.276832in}{1.180835in}}%
\pgfpathlineto{\pgfqpoint{2.284564in}{1.184963in}}%
\pgfpathlineto{\pgfqpoint{2.289718in}{1.175957in}}%
\pgfpathlineto{\pgfqpoint{2.292296in}{1.186464in}}%
\pgfpathlineto{\pgfqpoint{2.294873in}{1.192468in}}%
\pgfpathlineto{\pgfqpoint{2.302605in}{1.187402in}}%
\pgfpathlineto{\pgfqpoint{2.305182in}{1.199411in}}%
\pgfpathlineto{\pgfqpoint{2.307759in}{1.178021in}}%
\pgfpathlineto{\pgfqpoint{2.310336in}{1.162635in}}%
\pgfpathlineto{\pgfqpoint{2.312913in}{1.170515in}}%
\pgfpathlineto{\pgfqpoint{2.320645in}{1.164886in}}%
\pgfpathlineto{\pgfqpoint{2.323222in}{1.183650in}}%
\pgfpathlineto{\pgfqpoint{2.325799in}{1.186652in}}%
\pgfpathlineto{\pgfqpoint{2.328377in}{1.191530in}}%
\pgfpathlineto{\pgfqpoint{2.330954in}{1.182149in}}%
\pgfpathlineto{\pgfqpoint{2.338685in}{1.182524in}}%
\pgfpathlineto{\pgfqpoint{2.341263in}{1.186277in}}%
\pgfpathlineto{\pgfqpoint{2.343840in}{1.191718in}}%
\pgfpathlineto{\pgfqpoint{2.348994in}{1.195846in}}%
\pgfpathlineto{\pgfqpoint{2.356726in}{1.201287in}}%
\pgfpathlineto{\pgfqpoint{2.359303in}{1.198285in}}%
\pgfpathlineto{\pgfqpoint{2.361880in}{1.192281in}}%
\pgfpathlineto{\pgfqpoint{2.364458in}{1.201850in}}%
\pgfpathlineto{\pgfqpoint{2.367035in}{1.190404in}}%
\pgfpathlineto{\pgfqpoint{2.374766in}{1.184025in}}%
\pgfpathlineto{\pgfqpoint{2.377344in}{1.188528in}}%
\pgfpathlineto{\pgfqpoint{2.379921in}{1.195095in}}%
\pgfpathlineto{\pgfqpoint{2.382498in}{1.184213in}}%
\pgfpathlineto{\pgfqpoint{2.385075in}{1.183837in}}%
\pgfpathlineto{\pgfqpoint{2.395384in}{1.189654in}}%
\pgfpathlineto{\pgfqpoint{2.397961in}{1.190029in}}%
\pgfpathlineto{\pgfqpoint{2.403116in}{1.199223in}}%
\pgfpathlineto{\pgfqpoint{2.410847in}{1.205790in}}%
\pgfpathlineto{\pgfqpoint{2.416002in}{1.177082in}}%
\pgfpathlineto{\pgfqpoint{2.418579in}{1.184775in}}%
\pgfpathlineto{\pgfqpoint{2.421156in}{1.188153in}}%
\pgfpathlineto{\pgfqpoint{2.428888in}{1.187590in}}%
\pgfpathlineto{\pgfqpoint{2.434042in}{1.183650in}}%
\pgfpathlineto{\pgfqpoint{2.439197in}{1.174831in}}%
\pgfpathlineto{\pgfqpoint{2.446928in}{1.179522in}}%
\pgfpathlineto{\pgfqpoint{2.449506in}{1.172392in}}%
\pgfpathlineto{\pgfqpoint{2.452083in}{1.168639in}}%
\pgfpathlineto{\pgfqpoint{2.454660in}{1.163760in}}%
\pgfpathlineto{\pgfqpoint{2.457237in}{1.151752in}}%
\pgfpathlineto{\pgfqpoint{2.464969in}{1.148750in}}%
\pgfpathlineto{\pgfqpoint{2.467546in}{1.154754in}}%
\pgfpathlineto{\pgfqpoint{2.470123in}{1.145560in}}%
\pgfpathlineto{\pgfqpoint{2.472701in}{1.141995in}}%
\pgfpathlineto{\pgfqpoint{2.475278in}{1.150439in}}%
\pgfpathlineto{\pgfqpoint{2.483009in}{1.140869in}}%
\pgfpathlineto{\pgfqpoint{2.485587in}{1.140869in}}%
\pgfpathlineto{\pgfqpoint{2.488164in}{1.135428in}}%
\pgfpathlineto{\pgfqpoint{2.490741in}{1.153441in}}%
\pgfpathlineto{\pgfqpoint{2.493318in}{1.147061in}}%
\pgfpathlineto{\pgfqpoint{2.503627in}{1.127547in}}%
\pgfpathlineto{\pgfqpoint{2.506204in}{1.138430in}}%
\pgfpathlineto{\pgfqpoint{2.508782in}{1.136554in}}%
\pgfpathlineto{\pgfqpoint{2.511359in}{1.132238in}}%
\pgfpathlineto{\pgfqpoint{2.519090in}{1.126046in}}%
\pgfpathlineto{\pgfqpoint{2.521668in}{1.134114in}}%
\pgfpathlineto{\pgfqpoint{2.524245in}{1.134865in}}%
\pgfpathlineto{\pgfqpoint{2.526822in}{1.146686in}}%
\pgfpathlineto{\pgfqpoint{2.529399in}{1.152502in}}%
\pgfpathlineto{\pgfqpoint{2.537131in}{1.160195in}}%
\pgfpathlineto{\pgfqpoint{2.539708in}{1.164136in}}%
\pgfpathlineto{\pgfqpoint{2.542285in}{1.166200in}}%
\pgfpathlineto{\pgfqpoint{2.544862in}{1.163198in}}%
\pgfpathlineto{\pgfqpoint{2.547440in}{1.152878in}}%
\pgfpathlineto{\pgfqpoint{2.555171in}{1.155692in}}%
\pgfpathlineto{\pgfqpoint{2.557749in}{1.145560in}}%
\pgfpathlineto{\pgfqpoint{2.560326in}{1.140869in}}%
\pgfpathlineto{\pgfqpoint{2.562903in}{1.151189in}}%
\pgfpathlineto{\pgfqpoint{2.565480in}{1.141620in}}%
\pgfpathlineto{\pgfqpoint{2.573212in}{1.136929in}}%
\pgfpathlineto{\pgfqpoint{2.575789in}{1.141432in}}%
\pgfpathlineto{\pgfqpoint{2.578366in}{1.138618in}}%
\pgfpathlineto{\pgfqpoint{2.580943in}{1.141807in}}%
\pgfpathlineto{\pgfqpoint{2.583521in}{1.143121in}}%
\pgfpathlineto{\pgfqpoint{2.591252in}{1.145748in}}%
\pgfpathlineto{\pgfqpoint{2.593830in}{1.135615in}}%
\pgfpathlineto{\pgfqpoint{2.596407in}{1.137679in}}%
\pgfpathlineto{\pgfqpoint{2.598984in}{1.147249in}}%
\pgfpathlineto{\pgfqpoint{2.601561in}{1.150626in}}%
\pgfpathlineto{\pgfqpoint{2.609293in}{1.146498in}}%
\pgfpathlineto{\pgfqpoint{2.611870in}{1.139368in}}%
\pgfpathlineto{\pgfqpoint{2.614447in}{1.152878in}}%
\pgfpathlineto{\pgfqpoint{2.617024in}{1.176707in}}%
\pgfpathlineto{\pgfqpoint{2.619602in}{1.192844in}}%
\pgfpathlineto{\pgfqpoint{2.627333in}{1.201100in}}%
\pgfpathlineto{\pgfqpoint{2.629911in}{1.209168in}}%
\pgfpathlineto{\pgfqpoint{2.635065in}{1.198660in}}%
\pgfpathlineto{\pgfqpoint{2.637642in}{1.202976in}}%
\pgfpathlineto{\pgfqpoint{2.645374in}{1.200912in}}%
\pgfpathlineto{\pgfqpoint{2.647951in}{1.208417in}}%
\pgfpathlineto{\pgfqpoint{2.650528in}{1.200349in}}%
\pgfpathlineto{\pgfqpoint{2.655683in}{1.199786in}}%
\pgfpathlineto{\pgfqpoint{2.663414in}{1.208230in}}%
\pgfpathlineto{\pgfqpoint{2.665991in}{1.194157in}}%
\pgfpathlineto{\pgfqpoint{2.668569in}{1.201475in}}%
\pgfpathlineto{\pgfqpoint{2.671146in}{1.195095in}}%
\pgfpathlineto{\pgfqpoint{2.673723in}{1.195658in}}%
\pgfpathlineto{\pgfqpoint{2.681455in}{1.192093in}}%
\pgfpathlineto{\pgfqpoint{2.684032in}{1.194908in}}%
\pgfpathlineto{\pgfqpoint{2.686609in}{1.192468in}}%
\pgfpathlineto{\pgfqpoint{2.689186in}{1.196596in}}%
\pgfpathlineto{\pgfqpoint{2.691764in}{1.197159in}}%
\pgfpathlineto{\pgfqpoint{2.699495in}{1.203726in}}%
\pgfpathlineto{\pgfqpoint{2.702072in}{1.203914in}}%
\pgfpathlineto{\pgfqpoint{2.704650in}{1.198473in}}%
\pgfpathlineto{\pgfqpoint{2.707227in}{1.198097in}}%
\pgfpathlineto{\pgfqpoint{2.709804in}{1.195846in}}%
\pgfpathlineto{\pgfqpoint{2.717536in}{1.192844in}}%
\pgfpathlineto{\pgfqpoint{2.720113in}{1.193407in}}%
\pgfpathlineto{\pgfqpoint{2.722690in}{1.191718in}}%
\pgfpathlineto{\pgfqpoint{2.727845in}{1.187215in}}%
\pgfpathlineto{\pgfqpoint{2.735576in}{1.181961in}}%
\pgfpathlineto{\pgfqpoint{2.738153in}{1.186839in}}%
\pgfpathlineto{\pgfqpoint{2.740731in}{1.183650in}}%
\pgfpathlineto{\pgfqpoint{2.743308in}{1.176895in}}%
\pgfpathlineto{\pgfqpoint{2.745885in}{1.185151in}}%
\pgfpathlineto{\pgfqpoint{2.753617in}{1.186464in}}%
\pgfpathlineto{\pgfqpoint{2.761348in}{1.165637in}}%
\pgfpathlineto{\pgfqpoint{2.763926in}{1.161509in}}%
\pgfpathlineto{\pgfqpoint{2.771657in}{1.167513in}}%
\pgfpathlineto{\pgfqpoint{2.774234in}{1.157381in}}%
\pgfpathlineto{\pgfqpoint{2.776812in}{1.170515in}}%
\pgfpathlineto{\pgfqpoint{2.779389in}{1.169952in}}%
\pgfpathlineto{\pgfqpoint{2.781966in}{1.164886in}}%
\pgfpathlineto{\pgfqpoint{2.789698in}{1.172204in}}%
\pgfpathlineto{\pgfqpoint{2.792275in}{1.177833in}}%
\pgfpathlineto{\pgfqpoint{2.797429in}{1.180835in}}%
\pgfpathlineto{\pgfqpoint{2.810315in}{1.180272in}}%
\pgfpathlineto{\pgfqpoint{2.815470in}{1.178208in}}%
\pgfpathlineto{\pgfqpoint{2.818047in}{1.169765in}}%
\pgfpathlineto{\pgfqpoint{2.825779in}{1.173705in}}%
\pgfpathlineto{\pgfqpoint{2.828356in}{1.182524in}}%
\pgfpathlineto{\pgfqpoint{2.830933in}{1.178584in}}%
\pgfpathlineto{\pgfqpoint{2.833510in}{1.163948in}}%
\pgfpathlineto{\pgfqpoint{2.836088in}{1.167701in}}%
\pgfpathlineto{\pgfqpoint{2.843819in}{1.157193in}}%
\pgfpathlineto{\pgfqpoint{2.846396in}{1.157756in}}%
\pgfpathlineto{\pgfqpoint{2.848974in}{1.175206in}}%
\pgfpathlineto{\pgfqpoint{2.851551in}{1.178959in}}%
\pgfpathlineto{\pgfqpoint{2.854128in}{1.176332in}}%
\pgfpathlineto{\pgfqpoint{2.864437in}{1.166950in}}%
\pgfpathlineto{\pgfqpoint{2.867014in}{1.161509in}}%
\pgfpathlineto{\pgfqpoint{2.869591in}{1.169202in}}%
\pgfpathlineto{\pgfqpoint{2.872168in}{1.165824in}}%
\pgfpathlineto{\pgfqpoint{2.879900in}{1.166763in}}%
\pgfpathlineto{\pgfqpoint{2.882477in}{1.161884in}}%
\pgfpathlineto{\pgfqpoint{2.885055in}{1.166763in}}%
\pgfpathlineto{\pgfqpoint{2.887632in}{1.165824in}}%
\pgfpathlineto{\pgfqpoint{2.890209in}{1.171641in}}%
\pgfpathlineto{\pgfqpoint{2.897941in}{1.147999in}}%
\pgfpathlineto{\pgfqpoint{2.900518in}{1.154004in}}%
\pgfpathlineto{\pgfqpoint{2.903095in}{1.152127in}}%
\pgfpathlineto{\pgfqpoint{2.905672in}{1.151940in}}%
\pgfpathlineto{\pgfqpoint{2.908249in}{1.153816in}}%
\pgfpathlineto{\pgfqpoint{2.915981in}{1.155317in}}%
\pgfpathlineto{\pgfqpoint{2.918558in}{1.159820in}}%
\pgfpathlineto{\pgfqpoint{2.921136in}{1.162072in}}%
\pgfpathlineto{\pgfqpoint{2.923713in}{1.161134in}}%
\pgfpathlineto{\pgfqpoint{2.926290in}{1.149500in}}%
\pgfpathlineto{\pgfqpoint{2.936599in}{1.141807in}}%
\pgfpathlineto{\pgfqpoint{2.939176in}{1.149876in}}%
\pgfpathlineto{\pgfqpoint{2.941753in}{1.172955in}}%
\pgfpathlineto{\pgfqpoint{2.944330in}{1.160571in}}%
\pgfpathlineto{\pgfqpoint{2.952062in}{1.145372in}}%
\pgfpathlineto{\pgfqpoint{2.957216in}{1.147249in}}%
\pgfpathlineto{\pgfqpoint{2.959794in}{1.163948in}}%
\pgfpathlineto{\pgfqpoint{2.962371in}{1.165074in}}%
\pgfpathlineto{\pgfqpoint{2.970103in}{1.161134in}}%
\pgfpathlineto{\pgfqpoint{2.972680in}{1.169765in}}%
\pgfpathlineto{\pgfqpoint{2.975257in}{1.162259in}}%
\pgfpathlineto{\pgfqpoint{2.977834in}{1.163385in}}%
\pgfpathlineto{\pgfqpoint{2.980411in}{1.158882in}}%
\pgfpathlineto{\pgfqpoint{2.988143in}{1.157193in}}%
\pgfpathlineto{\pgfqpoint{2.995875in}{1.142558in}}%
\pgfpathlineto{\pgfqpoint{2.998452in}{1.143308in}}%
\pgfpathlineto{\pgfqpoint{3.006184in}{1.146498in}}%
\pgfpathlineto{\pgfqpoint{3.008761in}{1.152315in}}%
\pgfpathlineto{\pgfqpoint{3.011338in}{1.147249in}}%
\pgfpathlineto{\pgfqpoint{3.013915in}{1.159820in}}%
\pgfpathlineto{\pgfqpoint{3.016492in}{1.155317in}}%
\pgfpathlineto{\pgfqpoint{3.024224in}{1.156818in}}%
\pgfpathlineto{\pgfqpoint{3.026801in}{1.159820in}}%
\pgfpathlineto{\pgfqpoint{3.029378in}{1.156818in}}%
\pgfpathlineto{\pgfqpoint{3.031956in}{1.166763in}}%
\pgfpathlineto{\pgfqpoint{3.034533in}{1.162822in}}%
\pgfpathlineto{\pgfqpoint{3.042264in}{1.165074in}}%
\pgfpathlineto{\pgfqpoint{3.044842in}{1.167701in}}%
\pgfpathlineto{\pgfqpoint{3.047419in}{1.168639in}}%
\pgfpathlineto{\pgfqpoint{3.049996in}{1.172955in}}%
\pgfpathlineto{\pgfqpoint{3.052573in}{1.171829in}}%
\pgfpathlineto{\pgfqpoint{3.060305in}{1.172955in}}%
\pgfpathlineto{\pgfqpoint{3.062882in}{1.182149in}}%
\pgfpathlineto{\pgfqpoint{3.065459in}{1.178771in}}%
\pgfpathlineto{\pgfqpoint{3.070614in}{1.165449in}}%
\pgfpathlineto{\pgfqpoint{3.078345in}{1.168264in}}%
\pgfpathlineto{\pgfqpoint{3.080923in}{1.163198in}}%
\pgfpathlineto{\pgfqpoint{3.083500in}{1.165824in}}%
\pgfpathlineto{\pgfqpoint{3.086077in}{1.173142in}}%
\pgfpathlineto{\pgfqpoint{3.096386in}{1.178771in}}%
\pgfpathlineto{\pgfqpoint{3.098963in}{1.177833in}}%
\pgfpathlineto{\pgfqpoint{3.101540in}{1.170703in}}%
\pgfpathlineto{\pgfqpoint{3.104118in}{1.153628in}}%
\pgfpathlineto{\pgfqpoint{3.106695in}{1.148562in}}%
\pgfpathlineto{\pgfqpoint{3.117004in}{1.160946in}}%
\pgfpathlineto{\pgfqpoint{3.119581in}{1.160383in}}%
\pgfpathlineto{\pgfqpoint{3.122158in}{1.167513in}}%
\pgfpathlineto{\pgfqpoint{3.124735in}{1.166012in}}%
\pgfpathlineto{\pgfqpoint{3.135044in}{1.171266in}}%
\pgfpathlineto{\pgfqpoint{3.137621in}{1.180648in}}%
\pgfpathlineto{\pgfqpoint{3.140199in}{1.186277in}}%
\pgfpathlineto{\pgfqpoint{3.142776in}{1.186277in}}%
\pgfpathlineto{\pgfqpoint{3.150507in}{1.181398in}}%
\pgfpathlineto{\pgfqpoint{3.153085in}{1.176895in}}%
\pgfpathlineto{\pgfqpoint{3.155662in}{1.179334in}}%
\pgfpathlineto{\pgfqpoint{3.158239in}{1.178584in}}%
\pgfpathlineto{\pgfqpoint{3.160816in}{1.194908in}}%
\pgfpathlineto{\pgfqpoint{3.168548in}{1.196033in}}%
\pgfpathlineto{\pgfqpoint{3.171125in}{1.188716in}}%
\pgfpathlineto{\pgfqpoint{3.173702in}{1.196409in}}%
\pgfpathlineto{\pgfqpoint{3.178857in}{1.204852in}}%
\pgfpathlineto{\pgfqpoint{3.189166in}{1.202976in}}%
\pgfpathlineto{\pgfqpoint{3.191743in}{1.204852in}}%
\pgfpathlineto{\pgfqpoint{3.194320in}{1.204477in}}%
\pgfpathlineto{\pgfqpoint{3.196897in}{1.208042in}}%
\pgfpathlineto{\pgfqpoint{3.204629in}{1.209355in}}%
\pgfpathlineto{\pgfqpoint{3.207206in}{1.198097in}}%
\pgfpathlineto{\pgfqpoint{3.209783in}{1.196033in}}%
\pgfpathlineto{\pgfqpoint{3.214938in}{1.200161in}}%
\pgfpathlineto{\pgfqpoint{3.222669in}{1.202225in}}%
\pgfpathlineto{\pgfqpoint{3.225247in}{1.201662in}}%
\pgfpathlineto{\pgfqpoint{3.227824in}{1.199411in}}%
\pgfpathlineto{\pgfqpoint{3.230401in}{1.194532in}}%
\pgfpathlineto{\pgfqpoint{3.232978in}{1.196596in}}%
\pgfpathlineto{\pgfqpoint{3.240710in}{1.198473in}}%
\pgfpathlineto{\pgfqpoint{3.243287in}{1.196972in}}%
\pgfpathlineto{\pgfqpoint{3.245864in}{1.200161in}}%
\pgfpathlineto{\pgfqpoint{3.248441in}{1.200912in}}%
\pgfpathlineto{\pgfqpoint{3.251019in}{1.199599in}}%
\pgfpathlineto{\pgfqpoint{3.258750in}{1.204665in}}%
\pgfpathlineto{\pgfqpoint{3.261328in}{1.198097in}}%
\pgfpathlineto{\pgfqpoint{3.263905in}{1.199974in}}%
\pgfpathlineto{\pgfqpoint{3.266482in}{1.196972in}}%
\pgfpathlineto{\pgfqpoint{3.269059in}{1.198660in}}%
\pgfpathlineto{\pgfqpoint{3.276791in}{1.192844in}}%
\pgfpathlineto{\pgfqpoint{3.279368in}{1.199411in}}%
\pgfpathlineto{\pgfqpoint{3.281945in}{1.203726in}}%
\pgfpathlineto{\pgfqpoint{3.284522in}{1.204477in}}%
\pgfpathlineto{\pgfqpoint{3.294831in}{1.205040in}}%
\pgfpathlineto{\pgfqpoint{3.297409in}{1.198097in}}%
\pgfpathlineto{\pgfqpoint{3.299986in}{1.200161in}}%
\pgfpathlineto{\pgfqpoint{3.305140in}{1.221364in}}%
\pgfpathlineto{\pgfqpoint{3.312872in}{1.224554in}}%
\pgfpathlineto{\pgfqpoint{3.315449in}{1.227181in}}%
\pgfpathlineto{\pgfqpoint{3.318026in}{1.230933in}}%
\pgfpathlineto{\pgfqpoint{3.320603in}{1.221364in}}%
\pgfpathlineto{\pgfqpoint{3.323181in}{1.227931in}}%
\pgfpathlineto{\pgfqpoint{3.330912in}{1.227181in}}%
\pgfpathlineto{\pgfqpoint{3.333490in}{1.231309in}}%
\pgfpathlineto{\pgfqpoint{3.336067in}{1.230183in}}%
\pgfpathlineto{\pgfqpoint{3.338644in}{1.232247in}}%
\pgfpathlineto{\pgfqpoint{3.341221in}{1.235812in}}%
\pgfpathlineto{\pgfqpoint{3.348953in}{1.240315in}}%
\pgfpathlineto{\pgfqpoint{3.351530in}{1.246132in}}%
\pgfpathlineto{\pgfqpoint{3.354107in}{1.242942in}}%
\pgfpathlineto{\pgfqpoint{3.356684in}{1.222865in}}%
\pgfpathlineto{\pgfqpoint{3.359262in}{1.214046in}}%
\pgfpathlineto{\pgfqpoint{3.366993in}{1.219863in}}%
\pgfpathlineto{\pgfqpoint{3.374725in}{1.196409in}}%
\pgfpathlineto{\pgfqpoint{3.377302in}{1.197159in}}%
\pgfpathlineto{\pgfqpoint{3.385034in}{1.196784in}}%
\pgfpathlineto{\pgfqpoint{3.387611in}{1.199223in}}%
\pgfpathlineto{\pgfqpoint{3.392765in}{1.201662in}}%
\pgfpathlineto{\pgfqpoint{3.395343in}{1.198660in}}%
\pgfpathlineto{\pgfqpoint{3.403074in}{1.198473in}}%
\pgfpathlineto{\pgfqpoint{3.405651in}{1.197159in}}%
\pgfpathlineto{\pgfqpoint{3.408229in}{1.199036in}}%
\pgfpathlineto{\pgfqpoint{3.410806in}{1.199786in}}%
\pgfpathlineto{\pgfqpoint{3.413383in}{1.196221in}}%
\pgfpathlineto{\pgfqpoint{3.421115in}{1.203914in}}%
\pgfpathlineto{\pgfqpoint{3.423692in}{1.205415in}}%
\pgfpathlineto{\pgfqpoint{3.426269in}{1.208042in}}%
\pgfpathlineto{\pgfqpoint{3.428846in}{1.207854in}}%
\pgfpathlineto{\pgfqpoint{3.431424in}{1.213859in}}%
\pgfpathlineto{\pgfqpoint{3.441732in}{1.213108in}}%
\pgfpathlineto{\pgfqpoint{3.444310in}{1.214797in}}%
\pgfpathlineto{\pgfqpoint{3.446887in}{1.212358in}}%
\pgfpathlineto{\pgfqpoint{3.449464in}{1.215735in}}%
\pgfpathlineto{\pgfqpoint{3.457196in}{1.209543in}}%
\pgfpathlineto{\pgfqpoint{3.459773in}{1.200161in}}%
\pgfpathlineto{\pgfqpoint{3.462350in}{1.197910in}}%
\pgfpathlineto{\pgfqpoint{3.464927in}{1.201850in}}%
\pgfpathlineto{\pgfqpoint{3.467505in}{1.192656in}}%
\pgfpathlineto{\pgfqpoint{3.475236in}{1.195095in}}%
\pgfpathlineto{\pgfqpoint{3.477813in}{1.201100in}}%
\pgfpathlineto{\pgfqpoint{3.480391in}{1.204852in}}%
\pgfpathlineto{\pgfqpoint{3.482968in}{1.211982in}}%
\pgfpathlineto{\pgfqpoint{3.485545in}{1.221927in}}%
\pgfpathlineto{\pgfqpoint{3.493277in}{1.219300in}}%
\pgfpathlineto{\pgfqpoint{3.495854in}{1.215360in}}%
\pgfpathlineto{\pgfqpoint{3.498431in}{1.217611in}}%
\pgfpathlineto{\pgfqpoint{3.501008in}{1.210857in}}%
\pgfpathlineto{\pgfqpoint{3.503586in}{1.213108in}}%
\pgfpathlineto{\pgfqpoint{3.511317in}{1.212920in}}%
\pgfpathlineto{\pgfqpoint{3.513894in}{1.216486in}}%
\pgfpathlineto{\pgfqpoint{3.516472in}{1.208042in}}%
\pgfpathlineto{\pgfqpoint{3.519049in}{1.205978in}}%
\pgfpathlineto{\pgfqpoint{3.521626in}{1.212358in}}%
\pgfpathlineto{\pgfqpoint{3.529358in}{1.217799in}}%
\pgfpathlineto{\pgfqpoint{3.531935in}{1.211795in}}%
\pgfpathlineto{\pgfqpoint{3.534512in}{1.223053in}}%
\pgfpathlineto{\pgfqpoint{3.537089in}{1.208793in}}%
\pgfpathlineto{\pgfqpoint{3.539667in}{1.208980in}}%
\pgfpathlineto{\pgfqpoint{3.549975in}{1.198097in}}%
\pgfpathlineto{\pgfqpoint{3.555130in}{1.189842in}}%
\pgfpathlineto{\pgfqpoint{3.557707in}{1.195846in}}%
\pgfpathlineto{\pgfqpoint{3.565439in}{1.202038in}}%
\pgfpathlineto{\pgfqpoint{3.568016in}{1.205228in}}%
\pgfpathlineto{\pgfqpoint{3.570593in}{1.199411in}}%
\pgfpathlineto{\pgfqpoint{3.573170in}{1.198097in}}%
\pgfpathlineto{\pgfqpoint{3.575747in}{1.206353in}}%
\pgfpathlineto{\pgfqpoint{3.583479in}{1.216298in}}%
\pgfpathlineto{\pgfqpoint{3.586056in}{1.224366in}}%
\pgfpathlineto{\pgfqpoint{3.588634in}{1.222490in}}%
\pgfpathlineto{\pgfqpoint{3.591211in}{1.223428in}}%
\pgfpathlineto{\pgfqpoint{3.593788in}{1.228682in}}%
\pgfpathlineto{\pgfqpoint{3.601520in}{1.230933in}}%
\pgfpathlineto{\pgfqpoint{3.604097in}{1.229995in}}%
\pgfpathlineto{\pgfqpoint{3.606674in}{1.230183in}}%
\pgfpathlineto{\pgfqpoint{3.609251in}{1.229245in}}%
\pgfpathlineto{\pgfqpoint{3.611828in}{1.238063in}}%
\pgfpathlineto{\pgfqpoint{3.619560in}{1.235812in}}%
\pgfpathlineto{\pgfqpoint{3.622137in}{1.234123in}}%
\pgfpathlineto{\pgfqpoint{3.624715in}{1.237313in}}%
\pgfpathlineto{\pgfqpoint{3.629869in}{1.247633in}}%
\pgfpathlineto{\pgfqpoint{3.637601in}{1.246132in}}%
\pgfpathlineto{\pgfqpoint{3.640178in}{1.243505in}}%
\pgfpathlineto{\pgfqpoint{3.642755in}{1.232434in}}%
\pgfpathlineto{\pgfqpoint{3.645332in}{1.227931in}}%
\pgfpathlineto{\pgfqpoint{3.647909in}{1.228119in}}%
\pgfpathlineto{\pgfqpoint{3.655641in}{1.217424in}}%
\pgfpathlineto{\pgfqpoint{3.658218in}{1.214984in}}%
\pgfpathlineto{\pgfqpoint{3.660795in}{1.225680in}}%
\pgfpathlineto{\pgfqpoint{3.665950in}{1.233935in}}%
\pgfpathlineto{\pgfqpoint{3.673682in}{1.225492in}}%
\pgfpathlineto{\pgfqpoint{3.676259in}{1.211607in}}%
\pgfpathlineto{\pgfqpoint{3.678836in}{1.206729in}}%
\pgfpathlineto{\pgfqpoint{3.681413in}{1.206541in}}%
\pgfpathlineto{\pgfqpoint{3.683990in}{1.203914in}}%
\pgfpathlineto{\pgfqpoint{3.691722in}{1.208417in}}%
\pgfpathlineto{\pgfqpoint{3.694299in}{1.178396in}}%
\pgfpathlineto{\pgfqpoint{3.696876in}{1.167326in}}%
\pgfpathlineto{\pgfqpoint{3.699454in}{1.169952in}}%
\pgfpathlineto{\pgfqpoint{3.702031in}{1.158131in}}%
\pgfpathlineto{\pgfqpoint{3.709763in}{1.155880in}}%
\pgfpathlineto{\pgfqpoint{3.712340in}{1.157381in}}%
\pgfpathlineto{\pgfqpoint{3.714917in}{1.171266in}}%
\pgfpathlineto{\pgfqpoint{3.717494in}{1.180460in}}%
\pgfpathlineto{\pgfqpoint{3.720071in}{1.179897in}}%
\pgfpathlineto{\pgfqpoint{3.730380in}{1.189842in}}%
\pgfpathlineto{\pgfqpoint{3.732957in}{1.189842in}}%
\pgfpathlineto{\pgfqpoint{3.738112in}{1.192656in}}%
\pgfpathlineto{\pgfqpoint{3.745843in}{1.187965in}}%
\pgfpathlineto{\pgfqpoint{3.748421in}{1.184588in}}%
\pgfpathlineto{\pgfqpoint{3.750998in}{1.176332in}}%
\pgfpathlineto{\pgfqpoint{3.756152in}{1.179146in}}%
\pgfpathlineto{\pgfqpoint{3.763884in}{1.173142in}}%
\pgfpathlineto{\pgfqpoint{3.766461in}{1.180272in}}%
\pgfpathlineto{\pgfqpoint{3.769038in}{1.175769in}}%
\pgfpathlineto{\pgfqpoint{3.771616in}{1.190780in}}%
\pgfpathlineto{\pgfqpoint{3.774193in}{1.184400in}}%
\pgfpathlineto{\pgfqpoint{3.784502in}{1.190780in}}%
\pgfpathlineto{\pgfqpoint{3.787079in}{1.187590in}}%
\pgfpathlineto{\pgfqpoint{3.789656in}{1.189654in}}%
\pgfpathlineto{\pgfqpoint{3.792233in}{1.203164in}}%
\pgfpathlineto{\pgfqpoint{3.805119in}{1.207104in}}%
\pgfpathlineto{\pgfqpoint{3.807697in}{1.200349in}}%
\pgfpathlineto{\pgfqpoint{3.810274in}{1.190404in}}%
\pgfpathlineto{\pgfqpoint{3.818005in}{1.187402in}}%
\pgfpathlineto{\pgfqpoint{3.820583in}{1.178208in}}%
\pgfpathlineto{\pgfqpoint{3.823160in}{1.173517in}}%
\pgfpathlineto{\pgfqpoint{3.825737in}{1.174456in}}%
\pgfpathlineto{\pgfqpoint{3.828314in}{1.168264in}}%
\pgfpathlineto{\pgfqpoint{3.836046in}{1.187778in}}%
\pgfpathlineto{\pgfqpoint{3.838623in}{1.200724in}}%
\pgfpathlineto{\pgfqpoint{3.841200in}{1.200349in}}%
\pgfpathlineto{\pgfqpoint{3.843778in}{1.201287in}}%
\pgfpathlineto{\pgfqpoint{3.846355in}{1.223616in}}%
\pgfpathlineto{\pgfqpoint{3.854086in}{1.220051in}}%
\pgfpathlineto{\pgfqpoint{3.856664in}{1.226430in}}%
\pgfpathlineto{\pgfqpoint{3.859241in}{1.231121in}}%
\pgfpathlineto{\pgfqpoint{3.861818in}{1.226618in}}%
\pgfpathlineto{\pgfqpoint{3.864395in}{1.223428in}}%
\pgfpathlineto{\pgfqpoint{3.874704in}{1.221364in}}%
\pgfpathlineto{\pgfqpoint{3.877281in}{1.217799in}}%
\pgfpathlineto{\pgfqpoint{3.879859in}{1.217236in}}%
\pgfpathlineto{\pgfqpoint{3.882436in}{1.218174in}}%
\pgfpathlineto{\pgfqpoint{3.890167in}{1.215172in}}%
\pgfpathlineto{\pgfqpoint{3.892745in}{1.221927in}}%
\pgfpathlineto{\pgfqpoint{3.895322in}{1.221739in}}%
\pgfpathlineto{\pgfqpoint{3.897899in}{1.224366in}}%
\pgfpathlineto{\pgfqpoint{3.900476in}{1.225492in}}%
\pgfpathlineto{\pgfqpoint{3.908208in}{1.225867in}}%
\pgfpathlineto{\pgfqpoint{3.910785in}{1.226993in}}%
\pgfpathlineto{\pgfqpoint{3.913362in}{1.219675in}}%
\pgfpathlineto{\pgfqpoint{3.915940in}{1.217424in}}%
\pgfpathlineto{\pgfqpoint{3.918517in}{1.207854in}}%
\pgfpathlineto{\pgfqpoint{3.926248in}{1.206916in}}%
\pgfpathlineto{\pgfqpoint{3.928826in}{1.195846in}}%
\pgfpathlineto{\pgfqpoint{3.931403in}{1.198473in}}%
\pgfpathlineto{\pgfqpoint{3.933980in}{1.214609in}}%
\pgfpathlineto{\pgfqpoint{3.936557in}{1.216110in}}%
\pgfpathlineto{\pgfqpoint{3.944289in}{1.222865in}}%
\pgfpathlineto{\pgfqpoint{3.946866in}{1.217799in}}%
\pgfpathlineto{\pgfqpoint{3.949443in}{1.226993in}}%
\pgfpathlineto{\pgfqpoint{3.952020in}{1.223240in}}%
\pgfpathlineto{\pgfqpoint{3.954598in}{1.227181in}}%
\pgfpathlineto{\pgfqpoint{3.962329in}{1.228494in}}%
\pgfpathlineto{\pgfqpoint{3.964907in}{1.224929in}}%
\pgfpathlineto{\pgfqpoint{3.967484in}{1.214609in}}%
\pgfpathlineto{\pgfqpoint{3.970061in}{1.209731in}}%
\pgfpathlineto{\pgfqpoint{3.972638in}{1.211982in}}%
\pgfpathlineto{\pgfqpoint{3.980370in}{1.220426in}}%
\pgfpathlineto{\pgfqpoint{3.982947in}{1.212920in}}%
\pgfpathlineto{\pgfqpoint{3.985524in}{1.217424in}}%
\pgfpathlineto{\pgfqpoint{3.988101in}{1.225867in}}%
\pgfpathlineto{\pgfqpoint{3.998410in}{1.228306in}}%
\pgfpathlineto{\pgfqpoint{4.000988in}{1.222865in}}%
\pgfpathlineto{\pgfqpoint{4.003565in}{1.229057in}}%
\pgfpathlineto{\pgfqpoint{4.006142in}{1.227181in}}%
\pgfpathlineto{\pgfqpoint{4.008719in}{1.230558in}}%
\pgfpathlineto{\pgfqpoint{4.016451in}{1.227744in}}%
\pgfpathlineto{\pgfqpoint{4.019028in}{1.229808in}}%
\pgfpathlineto{\pgfqpoint{4.021605in}{1.233185in}}%
\pgfpathlineto{\pgfqpoint{4.024182in}{1.231309in}}%
\pgfpathlineto{\pgfqpoint{4.026760in}{1.225492in}}%
\pgfpathlineto{\pgfqpoint{4.034491in}{1.232997in}}%
\pgfpathlineto{\pgfqpoint{4.037069in}{1.229808in}}%
\pgfpathlineto{\pgfqpoint{4.042223in}{1.243317in}}%
\pgfpathlineto{\pgfqpoint{4.044800in}{1.242942in}}%
\pgfpathlineto{\pgfqpoint{4.052532in}{1.243880in}}%
\pgfpathlineto{\pgfqpoint{4.055109in}{1.251010in}}%
\pgfpathlineto{\pgfqpoint{4.060263in}{1.249321in}}%
\pgfpathlineto{\pgfqpoint{4.062841in}{1.248946in}}%
\pgfpathlineto{\pgfqpoint{4.070572in}{1.250635in}}%
\pgfpathlineto{\pgfqpoint{4.075727in}{1.236938in}}%
\pgfpathlineto{\pgfqpoint{4.078304in}{1.238439in}}%
\pgfpathlineto{\pgfqpoint{4.080881in}{1.244818in}}%
\pgfpathlineto{\pgfqpoint{4.088613in}{1.239564in}}%
\pgfpathlineto{\pgfqpoint{4.091190in}{1.236750in}}%
\pgfpathlineto{\pgfqpoint{4.093767in}{1.238439in}}%
\pgfpathlineto{\pgfqpoint{4.096344in}{1.242191in}}%
\pgfpathlineto{\pgfqpoint{4.098922in}{1.239377in}}%
\pgfpathlineto{\pgfqpoint{4.109230in}{1.235624in}}%
\pgfpathlineto{\pgfqpoint{4.111808in}{1.237688in}}%
\pgfpathlineto{\pgfqpoint{4.114385in}{1.241066in}}%
\pgfpathlineto{\pgfqpoint{4.116962in}{1.236562in}}%
\pgfpathlineto{\pgfqpoint{4.127271in}{1.233560in}}%
\pgfpathlineto{\pgfqpoint{4.129848in}{1.235999in}}%
\pgfpathlineto{\pgfqpoint{4.132425in}{1.235437in}}%
\pgfpathlineto{\pgfqpoint{4.135003in}{1.233935in}}%
\pgfpathlineto{\pgfqpoint{4.147889in}{1.228306in}}%
\pgfpathlineto{\pgfqpoint{4.153043in}{1.199786in}}%
\pgfpathlineto{\pgfqpoint{4.160775in}{1.202976in}}%
\pgfpathlineto{\pgfqpoint{4.163352in}{1.201475in}}%
\pgfpathlineto{\pgfqpoint{4.165929in}{1.203539in}}%
\pgfpathlineto{\pgfqpoint{4.168506in}{1.207854in}}%
\pgfpathlineto{\pgfqpoint{4.171084in}{1.199974in}}%
\pgfpathlineto{\pgfqpoint{4.178815in}{1.196221in}}%
\pgfpathlineto{\pgfqpoint{4.181392in}{1.202601in}}%
\pgfpathlineto{\pgfqpoint{4.183970in}{1.200349in}}%
\pgfpathlineto{\pgfqpoint{4.186547in}{1.208042in}}%
\pgfpathlineto{\pgfqpoint{4.189124in}{1.203351in}}%
\pgfpathlineto{\pgfqpoint{4.196856in}{1.204289in}}%
\pgfpathlineto{\pgfqpoint{4.199433in}{1.208042in}}%
\pgfpathlineto{\pgfqpoint{4.202010in}{1.200724in}}%
\pgfpathlineto{\pgfqpoint{4.207165in}{1.205790in}}%
\pgfpathlineto{\pgfqpoint{4.214896in}{1.192656in}}%
\pgfpathlineto{\pgfqpoint{4.217473in}{1.190217in}}%
\pgfpathlineto{\pgfqpoint{4.220051in}{1.196221in}}%
\pgfpathlineto{\pgfqpoint{4.222628in}{1.199223in}}%
\pgfpathlineto{\pgfqpoint{4.232937in}{1.196784in}}%
\pgfpathlineto{\pgfqpoint{4.235514in}{1.199974in}}%
\pgfpathlineto{\pgfqpoint{4.238091in}{1.198097in}}%
\pgfpathlineto{\pgfqpoint{4.240668in}{1.193219in}}%
\pgfpathlineto{\pgfqpoint{4.243246in}{1.204665in}}%
\pgfpathlineto{\pgfqpoint{4.250977in}{1.207854in}}%
\pgfpathlineto{\pgfqpoint{4.253554in}{1.211232in}}%
\pgfpathlineto{\pgfqpoint{4.256132in}{1.209918in}}%
\pgfpathlineto{\pgfqpoint{4.258709in}{1.217799in}}%
\pgfpathlineto{\pgfqpoint{4.261286in}{1.214046in}}%
\pgfpathlineto{\pgfqpoint{4.269018in}{1.221927in}}%
\pgfpathlineto{\pgfqpoint{4.271595in}{1.204289in}}%
\pgfpathlineto{\pgfqpoint{4.274172in}{1.196221in}}%
\pgfpathlineto{\pgfqpoint{4.276749in}{1.194532in}}%
\pgfpathlineto{\pgfqpoint{4.279326in}{1.189654in}}%
\pgfpathlineto{\pgfqpoint{4.287058in}{1.186464in}}%
\pgfpathlineto{\pgfqpoint{4.289635in}{1.187402in}}%
\pgfpathlineto{\pgfqpoint{4.292213in}{1.197910in}}%
\pgfpathlineto{\pgfqpoint{4.297367in}{1.201475in}}%
\pgfpathlineto{\pgfqpoint{4.305099in}{1.204289in}}%
\pgfpathlineto{\pgfqpoint{4.307676in}{1.199599in}}%
\pgfpathlineto{\pgfqpoint{4.312830in}{1.199036in}}%
\pgfpathlineto{\pgfqpoint{4.315407in}{1.194720in}}%
\pgfpathlineto{\pgfqpoint{4.328294in}{1.217987in}}%
\pgfpathlineto{\pgfqpoint{4.333448in}{1.212358in}}%
\pgfpathlineto{\pgfqpoint{4.341180in}{1.212920in}}%
\pgfpathlineto{\pgfqpoint{4.346334in}{1.211607in}}%
\pgfpathlineto{\pgfqpoint{4.348911in}{1.202788in}}%
\pgfpathlineto{\pgfqpoint{4.351488in}{1.190592in}}%
\pgfpathlineto{\pgfqpoint{4.359220in}{1.169390in}}%
\pgfpathlineto{\pgfqpoint{4.361797in}{1.150063in}}%
\pgfpathlineto{\pgfqpoint{4.364374in}{1.175957in}}%
\pgfpathlineto{\pgfqpoint{4.366952in}{1.192093in}}%
\pgfpathlineto{\pgfqpoint{4.369529in}{1.190217in}}%
\pgfpathlineto{\pgfqpoint{4.377261in}{1.189279in}}%
\pgfpathlineto{\pgfqpoint{4.379838in}{1.171829in}}%
\pgfpathlineto{\pgfqpoint{4.384992in}{1.184775in}}%
\pgfpathlineto{\pgfqpoint{4.387569in}{1.170703in}}%
\pgfpathlineto{\pgfqpoint{4.397878in}{1.187215in}}%
\pgfpathlineto{\pgfqpoint{4.400455in}{1.179522in}}%
\pgfpathlineto{\pgfqpoint{4.403033in}{1.180648in}}%
\pgfpathlineto{\pgfqpoint{4.405610in}{1.184963in}}%
\pgfpathlineto{\pgfqpoint{4.413342in}{1.183650in}}%
\pgfpathlineto{\pgfqpoint{4.415919in}{1.194908in}}%
\pgfpathlineto{\pgfqpoint{4.418496in}{1.192093in}}%
\pgfpathlineto{\pgfqpoint{4.421073in}{1.177082in}}%
\pgfpathlineto{\pgfqpoint{4.423650in}{1.166763in}}%
\pgfpathlineto{\pgfqpoint{4.431382in}{1.170328in}}%
\pgfpathlineto{\pgfqpoint{4.436536in}{1.157569in}}%
\pgfpathlineto{\pgfqpoint{4.441691in}{1.161321in}}%
\pgfpathlineto{\pgfqpoint{4.449422in}{1.153816in}}%
\pgfpathlineto{\pgfqpoint{4.452000in}{1.150439in}}%
\pgfpathlineto{\pgfqpoint{4.454577in}{1.150063in}}%
\pgfpathlineto{\pgfqpoint{4.457154in}{1.141620in}}%
\pgfpathlineto{\pgfqpoint{4.459731in}{1.139743in}}%
\pgfpathlineto{\pgfqpoint{4.467463in}{1.157756in}}%
\pgfpathlineto{\pgfqpoint{4.470040in}{1.158694in}}%
\pgfpathlineto{\pgfqpoint{4.475195in}{1.170328in}}%
\pgfpathlineto{\pgfqpoint{4.477772in}{1.169202in}}%
\pgfpathlineto{\pgfqpoint{4.488081in}{1.172392in}}%
\pgfpathlineto{\pgfqpoint{4.490658in}{1.166575in}}%
\pgfpathlineto{\pgfqpoint{4.493235in}{1.177458in}}%
\pgfpathlineto{\pgfqpoint{4.495812in}{1.177833in}}%
\pgfpathlineto{\pgfqpoint{4.503544in}{1.177833in}}%
\pgfpathlineto{\pgfqpoint{4.506121in}{1.186464in}}%
\pgfpathlineto{\pgfqpoint{4.508698in}{1.180460in}}%
\pgfpathlineto{\pgfqpoint{4.511276in}{1.196784in}}%
\pgfpathlineto{\pgfqpoint{4.513853in}{1.200912in}}%
\pgfpathlineto{\pgfqpoint{4.521584in}{1.204102in}}%
\pgfpathlineto{\pgfqpoint{4.524162in}{1.201100in}}%
\pgfpathlineto{\pgfqpoint{4.526739in}{1.205978in}}%
\pgfpathlineto{\pgfqpoint{4.529316in}{1.204852in}}%
\pgfpathlineto{\pgfqpoint{4.531893in}{1.212358in}}%
\pgfpathlineto{\pgfqpoint{4.539625in}{1.210857in}}%
\pgfpathlineto{\pgfqpoint{4.544779in}{1.200724in}}%
\pgfpathlineto{\pgfqpoint{4.547357in}{1.201662in}}%
\pgfpathlineto{\pgfqpoint{4.549934in}{1.194908in}}%
\pgfpathlineto{\pgfqpoint{4.560243in}{1.184213in}}%
\pgfpathlineto{\pgfqpoint{4.562820in}{1.187590in}}%
\pgfpathlineto{\pgfqpoint{4.565397in}{1.179897in}}%
\pgfpathlineto{\pgfqpoint{4.567974in}{1.170328in}}%
\pgfpathlineto{\pgfqpoint{4.575706in}{1.183274in}}%
\pgfpathlineto{\pgfqpoint{4.578283in}{1.183837in}}%
\pgfpathlineto{\pgfqpoint{4.580860in}{1.188528in}}%
\pgfpathlineto{\pgfqpoint{4.583438in}{1.194720in}}%
\pgfpathlineto{\pgfqpoint{4.586015in}{1.188716in}}%
\pgfpathlineto{\pgfqpoint{4.593746in}{1.182524in}}%
\pgfpathlineto{\pgfqpoint{4.596324in}{1.185526in}}%
\pgfpathlineto{\pgfqpoint{4.598901in}{1.181398in}}%
\pgfpathlineto{\pgfqpoint{4.604055in}{1.186277in}}%
\pgfpathlineto{\pgfqpoint{4.611787in}{1.189654in}}%
\pgfpathlineto{\pgfqpoint{4.614364in}{1.191718in}}%
\pgfpathlineto{\pgfqpoint{4.616941in}{1.181398in}}%
\pgfpathlineto{\pgfqpoint{4.619519in}{1.175581in}}%
\pgfpathlineto{\pgfqpoint{4.622096in}{1.193782in}}%
\pgfpathlineto{\pgfqpoint{4.629827in}{1.199411in}}%
\pgfpathlineto{\pgfqpoint{4.634982in}{1.188340in}}%
\pgfpathlineto{\pgfqpoint{4.637559in}{1.187590in}}%
\pgfpathlineto{\pgfqpoint{4.640136in}{1.179709in}}%
\pgfpathlineto{\pgfqpoint{4.647868in}{1.189654in}}%
\pgfpathlineto{\pgfqpoint{4.650445in}{1.191343in}}%
\pgfpathlineto{\pgfqpoint{4.653022in}{1.206541in}}%
\pgfpathlineto{\pgfqpoint{4.655599in}{1.199974in}}%
\pgfpathlineto{\pgfqpoint{4.658177in}{1.191530in}}%
\pgfpathlineto{\pgfqpoint{4.665908in}{1.196784in}}%
\pgfpathlineto{\pgfqpoint{4.668486in}{1.203914in}}%
\pgfpathlineto{\pgfqpoint{4.671063in}{1.213483in}}%
\pgfpathlineto{\pgfqpoint{4.673640in}{1.209731in}}%
\pgfpathlineto{\pgfqpoint{4.683949in}{1.210294in}}%
\pgfpathlineto{\pgfqpoint{4.686526in}{1.217611in}}%
\pgfpathlineto{\pgfqpoint{4.689103in}{1.210669in}}%
\pgfpathlineto{\pgfqpoint{4.691680in}{1.201850in}}%
\pgfpathlineto{\pgfqpoint{4.701989in}{1.196409in}}%
\pgfpathlineto{\pgfqpoint{4.704567in}{1.206353in}}%
\pgfpathlineto{\pgfqpoint{4.707144in}{1.199786in}}%
\pgfpathlineto{\pgfqpoint{4.709721in}{1.195658in}}%
\pgfpathlineto{\pgfqpoint{4.712298in}{1.188716in}}%
\pgfpathlineto{\pgfqpoint{4.720030in}{1.192844in}}%
\pgfpathlineto{\pgfqpoint{4.722607in}{1.190217in}}%
\pgfpathlineto{\pgfqpoint{4.725184in}{1.177645in}}%
\pgfpathlineto{\pgfqpoint{4.727761in}{1.189279in}}%
\pgfpathlineto{\pgfqpoint{4.730339in}{1.182149in}}%
\pgfpathlineto{\pgfqpoint{4.740648in}{1.189279in}}%
\pgfpathlineto{\pgfqpoint{4.743225in}{1.182149in}}%
\pgfpathlineto{\pgfqpoint{4.748379in}{1.224179in}}%
\pgfpathlineto{\pgfqpoint{4.756111in}{1.223991in}}%
\pgfpathlineto{\pgfqpoint{4.758688in}{1.243505in}}%
\pgfpathlineto{\pgfqpoint{4.761265in}{1.256076in}}%
\pgfpathlineto{\pgfqpoint{4.763842in}{1.255701in}}%
\pgfpathlineto{\pgfqpoint{4.766420in}{1.271087in}}%
\pgfpathlineto{\pgfqpoint{4.774151in}{1.283846in}}%
\pgfpathlineto{\pgfqpoint{4.776728in}{1.270149in}}%
\pgfpathlineto{\pgfqpoint{4.779306in}{1.281594in}}%
\pgfpathlineto{\pgfqpoint{4.781883in}{1.278405in}}%
\pgfpathlineto{\pgfqpoint{4.784460in}{1.287223in}}%
\pgfpathlineto{\pgfqpoint{4.792192in}{1.283471in}}%
\pgfpathlineto{\pgfqpoint{4.794769in}{1.273901in}}%
\pgfpathlineto{\pgfqpoint{4.797346in}{1.271275in}}%
\pgfpathlineto{\pgfqpoint{4.799923in}{1.261705in}}%
\pgfpathlineto{\pgfqpoint{4.802501in}{1.273339in}}%
\pgfpathlineto{\pgfqpoint{4.812809in}{1.275402in}}%
\pgfpathlineto{\pgfqpoint{4.815387in}{1.276716in}}%
\pgfpathlineto{\pgfqpoint{4.817964in}{1.286660in}}%
\pgfpathlineto{\pgfqpoint{4.820541in}{1.285347in}}%
\pgfpathlineto{\pgfqpoint{4.828273in}{1.288724in}}%
\pgfpathlineto{\pgfqpoint{4.830850in}{1.281594in}}%
\pgfpathlineto{\pgfqpoint{4.833427in}{1.284784in}}%
\pgfpathlineto{\pgfqpoint{4.836004in}{1.289287in}}%
\pgfpathlineto{\pgfqpoint{4.838582in}{1.287974in}}%
\pgfpathlineto{\pgfqpoint{4.846313in}{1.283283in}}%
\pgfpathlineto{\pgfqpoint{4.851468in}{1.305611in}}%
\pgfpathlineto{\pgfqpoint{4.854045in}{1.301859in}}%
\pgfpathlineto{\pgfqpoint{4.856622in}{1.300545in}}%
\pgfpathlineto{\pgfqpoint{4.864354in}{1.307113in}}%
\pgfpathlineto{\pgfqpoint{4.866931in}{1.311053in}}%
\pgfpathlineto{\pgfqpoint{4.869508in}{1.309177in}}%
\pgfpathlineto{\pgfqpoint{4.872085in}{1.308801in}}%
\pgfpathlineto{\pgfqpoint{4.874663in}{1.312179in}}%
\pgfpathlineto{\pgfqpoint{4.882394in}{1.312366in}}%
\pgfpathlineto{\pgfqpoint{4.884971in}{1.314430in}}%
\pgfpathlineto{\pgfqpoint{4.890126in}{1.329816in}}%
\pgfpathlineto{\pgfqpoint{4.892703in}{1.323624in}}%
\pgfpathlineto{\pgfqpoint{4.900435in}{1.326814in}}%
\pgfpathlineto{\pgfqpoint{4.903012in}{1.323061in}}%
\pgfpathlineto{\pgfqpoint{4.905589in}{1.318183in}}%
\pgfpathlineto{\pgfqpoint{4.908166in}{1.328690in}}%
\pgfpathlineto{\pgfqpoint{4.918475in}{1.326064in}}%
\pgfpathlineto{\pgfqpoint{4.921052in}{1.336571in}}%
\pgfpathlineto{\pgfqpoint{4.923630in}{1.336383in}}%
\pgfpathlineto{\pgfqpoint{4.926207in}{1.336946in}}%
\pgfpathlineto{\pgfqpoint{4.928784in}{1.335820in}}%
\pgfpathlineto{\pgfqpoint{4.936516in}{1.342388in}}%
\pgfpathlineto{\pgfqpoint{4.939093in}{1.337134in}}%
\pgfpathlineto{\pgfqpoint{4.941670in}{1.337134in}}%
\pgfpathlineto{\pgfqpoint{4.944247in}{1.312554in}}%
\pgfpathlineto{\pgfqpoint{4.946825in}{1.315368in}}%
\pgfpathlineto{\pgfqpoint{4.954556in}{1.306174in}}%
\pgfpathlineto{\pgfqpoint{4.957133in}{1.311616in}}%
\pgfpathlineto{\pgfqpoint{4.959711in}{1.300921in}}%
\pgfpathlineto{\pgfqpoint{4.962288in}{1.302046in}}%
\pgfpathlineto{\pgfqpoint{4.964865in}{1.301859in}}%
\pgfpathlineto{\pgfqpoint{4.972597in}{1.308051in}}%
\pgfpathlineto{\pgfqpoint{4.975174in}{1.313680in}}%
\pgfpathlineto{\pgfqpoint{4.977751in}{1.308426in}}%
\pgfpathlineto{\pgfqpoint{4.980328in}{1.280469in}}%
\pgfpathlineto{\pgfqpoint{4.982905in}{1.288912in}}%
\pgfpathlineto{\pgfqpoint{4.990637in}{1.292290in}}%
\pgfpathlineto{\pgfqpoint{4.993214in}{1.287223in}}%
\pgfpathlineto{\pgfqpoint{4.995792in}{1.307488in}}%
\pgfpathlineto{\pgfqpoint{4.998369in}{1.296605in}}%
\pgfpathlineto{\pgfqpoint{5.000946in}{1.295292in}}%
\pgfpathlineto{\pgfqpoint{5.008678in}{1.301484in}}%
\pgfpathlineto{\pgfqpoint{5.011255in}{1.290976in}}%
\pgfpathlineto{\pgfqpoint{5.013832in}{1.293603in}}%
\pgfpathlineto{\pgfqpoint{5.016409in}{1.293603in}}%
\pgfpathlineto{\pgfqpoint{5.018986in}{1.298106in}}%
\pgfpathlineto{\pgfqpoint{5.026718in}{1.297543in}}%
\pgfpathlineto{\pgfqpoint{5.029295in}{1.305049in}}%
\pgfpathlineto{\pgfqpoint{5.031873in}{1.298669in}}%
\pgfpathlineto{\pgfqpoint{5.034450in}{1.303923in}}%
\pgfpathlineto{\pgfqpoint{5.037027in}{1.295292in}}%
\pgfpathlineto{\pgfqpoint{5.044759in}{1.299982in}}%
\pgfpathlineto{\pgfqpoint{5.047336in}{1.294541in}}%
\pgfpathlineto{\pgfqpoint{5.049913in}{1.286285in}}%
\pgfpathlineto{\pgfqpoint{5.052490in}{1.274089in}}%
\pgfpathlineto{\pgfqpoint{5.055067in}{1.274464in}}%
\pgfpathlineto{\pgfqpoint{5.062799in}{1.266021in}}%
\pgfpathlineto{\pgfqpoint{5.065376in}{1.273151in}}%
\pgfpathlineto{\pgfqpoint{5.070531in}{1.282720in}}%
\pgfpathlineto{\pgfqpoint{5.073108in}{1.290038in}}%
\pgfpathlineto{\pgfqpoint{5.083417in}{1.294541in}}%
\pgfpathlineto{\pgfqpoint{5.085994in}{1.287036in}}%
\pgfpathlineto{\pgfqpoint{5.088571in}{1.292477in}}%
\pgfpathlineto{\pgfqpoint{5.091148in}{1.294916in}}%
\pgfpathlineto{\pgfqpoint{5.098880in}{1.291539in}}%
\pgfpathlineto{\pgfqpoint{5.101457in}{1.308426in}}%
\pgfpathlineto{\pgfqpoint{5.104034in}{1.304673in}}%
\pgfpathlineto{\pgfqpoint{5.106612in}{1.311616in}}%
\pgfpathlineto{\pgfqpoint{5.109189in}{1.323249in}}%
\pgfpathlineto{\pgfqpoint{5.116921in}{1.321748in}}%
\pgfpathlineto{\pgfqpoint{5.119498in}{1.328503in}}%
\pgfpathlineto{\pgfqpoint{5.122075in}{1.326064in}}%
\pgfpathlineto{\pgfqpoint{5.124652in}{1.336196in}}%
\pgfpathlineto{\pgfqpoint{5.127229in}{1.341262in}}%
\pgfpathlineto{\pgfqpoint{5.134961in}{1.341074in}}%
\pgfpathlineto{\pgfqpoint{5.137538in}{1.346516in}}%
\pgfpathlineto{\pgfqpoint{5.140115in}{1.345390in}}%
\pgfpathlineto{\pgfqpoint{5.142693in}{1.355710in}}%
\pgfpathlineto{\pgfqpoint{5.145270in}{1.351769in}}%
\pgfpathlineto{\pgfqpoint{5.155579in}{1.358149in}}%
\pgfpathlineto{\pgfqpoint{5.158156in}{1.362089in}}%
\pgfpathlineto{\pgfqpoint{5.160733in}{1.374661in}}%
\pgfpathlineto{\pgfqpoint{5.163310in}{1.381040in}}%
\pgfpathlineto{\pgfqpoint{5.173619in}{1.385919in}}%
\pgfpathlineto{\pgfqpoint{5.176196in}{1.390797in}}%
\pgfpathlineto{\pgfqpoint{5.178774in}{1.376349in}}%
\pgfpathlineto{\pgfqpoint{5.181351in}{1.384793in}}%
\pgfpathlineto{\pgfqpoint{5.189082in}{1.385356in}}%
\pgfpathlineto{\pgfqpoint{5.191660in}{1.377850in}}%
\pgfpathlineto{\pgfqpoint{5.194237in}{1.386482in}}%
\pgfpathlineto{\pgfqpoint{5.196814in}{1.383855in}}%
\pgfpathlineto{\pgfqpoint{5.199391in}{1.383855in}}%
\pgfpathlineto{\pgfqpoint{5.207123in}{1.385356in}}%
\pgfpathlineto{\pgfqpoint{5.209700in}{1.381603in}}%
\pgfpathlineto{\pgfqpoint{5.212277in}{1.380290in}}%
\pgfpathlineto{\pgfqpoint{5.214855in}{1.376162in}}%
\pgfpathlineto{\pgfqpoint{5.217432in}{1.388170in}}%
\pgfpathlineto{\pgfqpoint{5.225163in}{1.384418in}}%
\pgfpathlineto{\pgfqpoint{5.227741in}{1.366968in}}%
\pgfpathlineto{\pgfqpoint{5.230318in}{1.375411in}}%
\pgfpathlineto{\pgfqpoint{5.232895in}{1.367718in}}%
\pgfpathlineto{\pgfqpoint{5.235472in}{1.376725in}}%
\pgfpathlineto{\pgfqpoint{5.243204in}{1.361902in}}%
\pgfpathlineto{\pgfqpoint{5.245781in}{1.353646in}}%
\pgfpathlineto{\pgfqpoint{5.248358in}{1.352145in}}%
\pgfpathlineto{\pgfqpoint{5.250936in}{1.352520in}}%
\pgfpathlineto{\pgfqpoint{5.253513in}{1.347829in}}%
\pgfpathlineto{\pgfqpoint{5.261244in}{1.347079in}}%
\pgfpathlineto{\pgfqpoint{5.263822in}{1.348204in}}%
\pgfpathlineto{\pgfqpoint{5.266399in}{1.350644in}}%
\pgfpathlineto{\pgfqpoint{5.268976in}{1.351394in}}%
\pgfpathlineto{\pgfqpoint{5.271553in}{1.348017in}}%
\pgfpathlineto{\pgfqpoint{5.279285in}{1.347266in}}%
\pgfpathlineto{\pgfqpoint{5.281862in}{1.333381in}}%
\pgfpathlineto{\pgfqpoint{5.284439in}{1.340136in}}%
\pgfpathlineto{\pgfqpoint{5.287017in}{1.335258in}}%
\pgfpathlineto{\pgfqpoint{5.289594in}{1.328315in}}%
\pgfpathlineto{\pgfqpoint{5.297325in}{1.330004in}}%
\pgfpathlineto{\pgfqpoint{5.299903in}{1.331317in}}%
\pgfpathlineto{\pgfqpoint{5.302480in}{1.329629in}}%
\pgfpathlineto{\pgfqpoint{5.305057in}{1.333194in}}%
\pgfpathlineto{\pgfqpoint{5.307634in}{1.322123in}}%
\pgfpathlineto{\pgfqpoint{5.315366in}{1.329066in}}%
\pgfpathlineto{\pgfqpoint{5.317943in}{1.325313in}}%
\pgfpathlineto{\pgfqpoint{5.320520in}{1.326439in}}%
\pgfpathlineto{\pgfqpoint{5.323098in}{1.330191in}}%
\pgfpathlineto{\pgfqpoint{5.325675in}{1.335445in}}%
\pgfpathlineto{\pgfqpoint{5.338561in}{1.348955in}}%
\pgfpathlineto{\pgfqpoint{5.341138in}{1.347079in}}%
\pgfpathlineto{\pgfqpoint{5.343715in}{1.317995in}}%
\pgfpathlineto{\pgfqpoint{5.351447in}{1.330379in}}%
\pgfpathlineto{\pgfqpoint{5.354024in}{1.311991in}}%
\pgfpathlineto{\pgfqpoint{5.356601in}{1.312554in}}%
\pgfpathlineto{\pgfqpoint{5.359178in}{1.320622in}}%
\pgfpathlineto{\pgfqpoint{5.361756in}{1.318933in}}%
\pgfpathlineto{\pgfqpoint{5.369487in}{1.307863in}}%
\pgfpathlineto{\pgfqpoint{5.372065in}{1.308989in}}%
\pgfpathlineto{\pgfqpoint{5.374642in}{1.318933in}}%
\pgfpathlineto{\pgfqpoint{5.377219in}{1.326626in}}%
\pgfpathlineto{\pgfqpoint{5.379796in}{1.330191in}}%
\pgfpathlineto{\pgfqpoint{5.387528in}{1.323437in}}%
\pgfpathlineto{\pgfqpoint{5.390105in}{1.329066in}}%
\pgfpathlineto{\pgfqpoint{5.392682in}{1.321936in}}%
\pgfpathlineto{\pgfqpoint{5.395259in}{1.322874in}}%
\pgfpathlineto{\pgfqpoint{5.397837in}{1.320622in}}%
\pgfpathlineto{\pgfqpoint{5.405568in}{1.318933in}}%
\pgfpathlineto{\pgfqpoint{5.408146in}{1.308801in}}%
\pgfpathlineto{\pgfqpoint{5.410723in}{1.302046in}}%
\pgfpathlineto{\pgfqpoint{5.413300in}{1.301859in}}%
\pgfpathlineto{\pgfqpoint{5.415877in}{1.296230in}}%
\pgfpathlineto{\pgfqpoint{5.423609in}{1.300733in}}%
\pgfpathlineto{\pgfqpoint{5.426186in}{1.295855in}}%
\pgfpathlineto{\pgfqpoint{5.428763in}{1.302609in}}%
\pgfpathlineto{\pgfqpoint{5.433918in}{1.302234in}}%
\pgfpathlineto{\pgfqpoint{5.441649in}{1.304673in}}%
\pgfpathlineto{\pgfqpoint{5.444227in}{1.302046in}}%
\pgfpathlineto{\pgfqpoint{5.446804in}{1.303923in}}%
\pgfpathlineto{\pgfqpoint{5.449381in}{1.283283in}}%
\pgfpathlineto{\pgfqpoint{5.451958in}{1.267710in}}%
\pgfpathlineto{\pgfqpoint{5.459690in}{1.267897in}}%
\pgfpathlineto{\pgfqpoint{5.462267in}{1.261705in}}%
\pgfpathlineto{\pgfqpoint{5.464844in}{1.258328in}}%
\pgfpathlineto{\pgfqpoint{5.467421in}{1.273339in}}%
\pgfpathlineto{\pgfqpoint{5.469999in}{1.267897in}}%
\pgfpathlineto{\pgfqpoint{5.477730in}{1.266021in}}%
\pgfpathlineto{\pgfqpoint{5.480307in}{1.258891in}}%
\pgfpathlineto{\pgfqpoint{5.482885in}{1.246882in}}%
\pgfpathlineto{\pgfqpoint{5.485462in}{1.245756in}}%
\pgfpathlineto{\pgfqpoint{5.488039in}{1.249134in}}%
\pgfpathlineto{\pgfqpoint{5.495771in}{1.255513in}}%
\pgfpathlineto{\pgfqpoint{5.500925in}{1.262080in}}%
\pgfpathlineto{\pgfqpoint{5.503502in}{1.242754in}}%
\pgfpathlineto{\pgfqpoint{5.506080in}{1.242754in}}%
\pgfpathlineto{\pgfqpoint{5.513811in}{1.234311in}}%
\pgfpathlineto{\pgfqpoint{5.516388in}{1.254012in}}%
\pgfpathlineto{\pgfqpoint{5.518966in}{1.263206in}}%
\pgfpathlineto{\pgfqpoint{5.521543in}{1.261705in}}%
\pgfpathlineto{\pgfqpoint{5.524120in}{1.265646in}}%
\pgfpathlineto{\pgfqpoint{5.531852in}{1.269586in}}%
\pgfpathlineto{\pgfqpoint{5.534429in}{1.289100in}}%
\pgfpathlineto{\pgfqpoint{5.537006in}{1.301296in}}%
\pgfpathlineto{\pgfqpoint{5.542161in}{1.308614in}}%
\pgfpathlineto{\pgfqpoint{5.549892in}{1.316119in}}%
\pgfpathlineto{\pgfqpoint{5.552469in}{1.313492in}}%
\pgfpathlineto{\pgfqpoint{5.555047in}{1.295855in}}%
\pgfpathlineto{\pgfqpoint{5.557624in}{1.295479in}}%
\pgfpathlineto{\pgfqpoint{5.560201in}{1.294353in}}%
\pgfpathlineto{\pgfqpoint{5.567933in}{1.293415in}}%
\pgfpathlineto{\pgfqpoint{5.570510in}{1.303548in}}%
\pgfpathlineto{\pgfqpoint{5.573087in}{1.320435in}}%
\pgfpathlineto{\pgfqpoint{5.575664in}{1.316307in}}%
\pgfpathlineto{\pgfqpoint{5.578242in}{1.322311in}}%
\pgfpathlineto{\pgfqpoint{5.585973in}{1.326626in}}%
\pgfpathlineto{\pgfqpoint{5.588550in}{1.336571in}}%
\pgfpathlineto{\pgfqpoint{5.591128in}{1.324562in}}%
\pgfpathlineto{\pgfqpoint{5.593705in}{1.327565in}}%
\pgfpathlineto{\pgfqpoint{5.596282in}{1.335070in}}%
\pgfpathlineto{\pgfqpoint{5.606591in}{1.349142in}}%
\pgfpathlineto{\pgfqpoint{5.609168in}{1.346703in}}%
\pgfpathlineto{\pgfqpoint{5.611745in}{1.357961in}}%
\pgfpathlineto{\pgfqpoint{5.614323in}{1.358524in}}%
\pgfpathlineto{\pgfqpoint{5.624631in}{1.357774in}}%
\pgfpathlineto{\pgfqpoint{5.627209in}{1.354584in}}%
\pgfpathlineto{\pgfqpoint{5.629786in}{1.359462in}}%
\pgfpathlineto{\pgfqpoint{5.632363in}{1.353458in}}%
\pgfpathlineto{\pgfqpoint{5.642672in}{1.373347in}}%
\pgfpathlineto{\pgfqpoint{5.645249in}{1.372409in}}%
\pgfpathlineto{\pgfqpoint{5.647826in}{1.374473in}}%
\pgfpathlineto{\pgfqpoint{5.650403in}{1.360963in}}%
\pgfpathlineto{\pgfqpoint{5.658135in}{1.351206in}}%
\pgfpathlineto{\pgfqpoint{5.660712in}{1.352520in}}%
\pgfpathlineto{\pgfqpoint{5.663290in}{1.347641in}}%
\pgfpathlineto{\pgfqpoint{5.665867in}{1.351206in}}%
\pgfpathlineto{\pgfqpoint{5.668444in}{1.349142in}}%
\pgfpathlineto{\pgfqpoint{5.678753in}{1.352332in}}%
\pgfpathlineto{\pgfqpoint{5.681330in}{1.344076in}}%
\pgfpathlineto{\pgfqpoint{5.683907in}{1.345953in}}%
\pgfpathlineto{\pgfqpoint{5.686484in}{1.351957in}}%
\pgfpathlineto{\pgfqpoint{5.694216in}{1.346703in}}%
\pgfpathlineto{\pgfqpoint{5.696793in}{1.308426in}}%
\pgfpathlineto{\pgfqpoint{5.699371in}{1.302609in}}%
\pgfpathlineto{\pgfqpoint{5.701948in}{1.291727in}}%
\pgfpathlineto{\pgfqpoint{5.704525in}{1.299607in}}%
\pgfpathlineto{\pgfqpoint{5.712257in}{1.295855in}}%
\pgfpathlineto{\pgfqpoint{5.714834in}{1.289850in}}%
\pgfpathlineto{\pgfqpoint{5.717411in}{1.279530in}}%
\pgfpathlineto{\pgfqpoint{5.719988in}{1.277654in}}%
\pgfpathlineto{\pgfqpoint{5.722565in}{1.282720in}}%
\pgfpathlineto{\pgfqpoint{5.730297in}{1.273339in}}%
\pgfpathlineto{\pgfqpoint{5.732874in}{1.273526in}}%
\pgfpathlineto{\pgfqpoint{5.735452in}{1.279155in}}%
\pgfpathlineto{\pgfqpoint{5.738029in}{1.286473in}}%
\pgfpathlineto{\pgfqpoint{5.740606in}{1.289287in}}%
\pgfpathlineto{\pgfqpoint{5.748338in}{1.282157in}}%
\pgfpathlineto{\pgfqpoint{5.750915in}{1.277466in}}%
\pgfpathlineto{\pgfqpoint{5.753492in}{1.274277in}}%
\pgfpathlineto{\pgfqpoint{5.756069in}{1.280656in}}%
\pgfpathlineto{\pgfqpoint{5.758646in}{1.292852in}}%
\pgfpathlineto{\pgfqpoint{5.768955in}{1.296793in}}%
\pgfpathlineto{\pgfqpoint{5.771532in}{1.300921in}}%
\pgfpathlineto{\pgfqpoint{5.774110in}{1.311616in}}%
\pgfpathlineto{\pgfqpoint{5.776687in}{1.316494in}}%
\pgfpathlineto{\pgfqpoint{5.784419in}{1.305424in}}%
\pgfpathlineto{\pgfqpoint{5.786996in}{1.300170in}}%
\pgfpathlineto{\pgfqpoint{5.792150in}{1.305987in}}%
\pgfpathlineto{\pgfqpoint{5.794727in}{1.307863in}}%
\pgfpathlineto{\pgfqpoint{5.802459in}{1.306925in}}%
\pgfpathlineto{\pgfqpoint{5.805036in}{1.296980in}}%
\pgfpathlineto{\pgfqpoint{5.807613in}{1.292290in}}%
\pgfpathlineto{\pgfqpoint{5.810191in}{1.294353in}}%
\pgfpathlineto{\pgfqpoint{5.812768in}{1.295479in}}%
\pgfpathlineto{\pgfqpoint{5.820500in}{1.297543in}}%
\pgfpathlineto{\pgfqpoint{5.823077in}{1.295667in}}%
\pgfpathlineto{\pgfqpoint{5.825654in}{1.308801in}}%
\pgfpathlineto{\pgfqpoint{5.828231in}{1.307113in}}%
\pgfpathlineto{\pgfqpoint{5.830808in}{1.312929in}}%
\pgfpathlineto{\pgfqpoint{5.838540in}{1.310490in}}%
\pgfpathlineto{\pgfqpoint{5.841117in}{1.308989in}}%
\pgfpathlineto{\pgfqpoint{5.843694in}{1.301484in}}%
\pgfpathlineto{\pgfqpoint{5.846272in}{1.300358in}}%
\pgfpathlineto{\pgfqpoint{5.848849in}{1.301108in}}%
\pgfpathlineto{\pgfqpoint{5.856580in}{1.291914in}}%
\pgfpathlineto{\pgfqpoint{5.859158in}{1.294729in}}%
\pgfpathlineto{\pgfqpoint{5.861735in}{1.291914in}}%
\pgfpathlineto{\pgfqpoint{5.864312in}{1.290601in}}%
\pgfpathlineto{\pgfqpoint{5.866889in}{1.285535in}}%
\pgfpathlineto{\pgfqpoint{5.877198in}{1.294916in}}%
\pgfpathlineto{\pgfqpoint{5.879775in}{1.290038in}}%
\pgfpathlineto{\pgfqpoint{5.882353in}{1.289850in}}%
\pgfpathlineto{\pgfqpoint{5.884930in}{1.293603in}}%
\pgfpathlineto{\pgfqpoint{5.892661in}{1.291727in}}%
\pgfpathlineto{\pgfqpoint{5.895239in}{1.294353in}}%
\pgfpathlineto{\pgfqpoint{5.897816in}{1.298106in}}%
\pgfpathlineto{\pgfqpoint{5.900393in}{1.293040in}}%
\pgfpathlineto{\pgfqpoint{5.910702in}{1.296230in}}%
\pgfpathlineto{\pgfqpoint{5.913279in}{1.303172in}}%
\pgfpathlineto{\pgfqpoint{5.915856in}{1.298481in}}%
\pgfpathlineto{\pgfqpoint{5.918434in}{1.289475in}}%
\pgfpathlineto{\pgfqpoint{5.921011in}{1.269773in}}%
\pgfpathlineto{\pgfqpoint{5.928742in}{1.266396in}}%
\pgfpathlineto{\pgfqpoint{5.931320in}{1.260392in}}%
\pgfpathlineto{\pgfqpoint{5.933897in}{1.271650in}}%
\pgfpathlineto{\pgfqpoint{5.939051in}{1.247070in}}%
\pgfpathlineto{\pgfqpoint{5.946783in}{1.246507in}}%
\pgfpathlineto{\pgfqpoint{5.949360in}{1.247070in}}%
\pgfpathlineto{\pgfqpoint{5.951937in}{1.251198in}}%
\pgfpathlineto{\pgfqpoint{5.954515in}{1.246507in}}%
\pgfpathlineto{\pgfqpoint{5.957092in}{1.260392in}}%
\pgfpathlineto{\pgfqpoint{5.964823in}{1.259266in}}%
\pgfpathlineto{\pgfqpoint{5.967401in}{1.255701in}}%
\pgfpathlineto{\pgfqpoint{5.969978in}{1.255138in}}%
\pgfpathlineto{\pgfqpoint{5.972555in}{1.248946in}}%
\pgfpathlineto{\pgfqpoint{5.975132in}{1.245944in}}%
\pgfpathlineto{\pgfqpoint{5.982864in}{1.238063in}}%
\pgfpathlineto{\pgfqpoint{5.985441in}{1.236938in}}%
\pgfpathlineto{\pgfqpoint{5.988018in}{1.222865in}}%
\pgfpathlineto{\pgfqpoint{5.990596in}{1.232434in}}%
\pgfpathlineto{\pgfqpoint{5.993173in}{1.238814in}}%
\pgfpathlineto{\pgfqpoint{6.003482in}{1.239752in}}%
\pgfpathlineto{\pgfqpoint{6.006059in}{1.232434in}}%
\pgfpathlineto{\pgfqpoint{6.008636in}{1.236938in}}%
\pgfpathlineto{\pgfqpoint{6.011213in}{1.237125in}}%
\pgfpathlineto{\pgfqpoint{6.021522in}{1.251948in}}%
\pgfpathlineto{\pgfqpoint{6.024099in}{1.259454in}}%
\pgfpathlineto{\pgfqpoint{6.026677in}{1.257202in}}%
\pgfpathlineto{\pgfqpoint{6.029254in}{1.256076in}}%
\pgfpathlineto{\pgfqpoint{6.036985in}{1.254950in}}%
\pgfpathlineto{\pgfqpoint{6.042140in}{1.257014in}}%
\pgfpathlineto{\pgfqpoint{6.044717in}{1.251761in}}%
\pgfpathlineto{\pgfqpoint{6.047294in}{1.260767in}}%
\pgfpathlineto{\pgfqpoint{6.055026in}{1.268835in}}%
\pgfpathlineto{\pgfqpoint{6.057603in}{1.256451in}}%
\pgfpathlineto{\pgfqpoint{6.060180in}{1.260392in}}%
\pgfpathlineto{\pgfqpoint{6.062757in}{1.259454in}}%
\pgfpathlineto{\pgfqpoint{6.073066in}{1.258328in}}%
\pgfpathlineto{\pgfqpoint{6.078221in}{1.238626in}}%
\pgfpathlineto{\pgfqpoint{6.080798in}{1.238814in}}%
\pgfpathlineto{\pgfqpoint{6.083375in}{1.238251in}}%
\pgfpathlineto{\pgfqpoint{6.091107in}{1.244443in}}%
\pgfpathlineto{\pgfqpoint{6.093684in}{1.229057in}}%
\pgfpathlineto{\pgfqpoint{6.096261in}{1.229057in}}%
\pgfpathlineto{\pgfqpoint{6.098838in}{1.221739in}}%
\pgfpathlineto{\pgfqpoint{6.101416in}{1.225867in}}%
\pgfpathlineto{\pgfqpoint{6.109147in}{1.231684in}}%
\pgfpathlineto{\pgfqpoint{6.114302in}{1.225680in}}%
\pgfpathlineto{\pgfqpoint{6.116879in}{1.216298in}}%
\pgfpathlineto{\pgfqpoint{6.119456in}{1.215547in}}%
\pgfpathlineto{\pgfqpoint{6.127188in}{1.210857in}}%
\pgfpathlineto{\pgfqpoint{6.129765in}{1.205415in}}%
\pgfpathlineto{\pgfqpoint{6.134919in}{1.215735in}}%
\pgfpathlineto{\pgfqpoint{6.137497in}{1.217048in}}%
\pgfpathlineto{\pgfqpoint{6.145228in}{1.218737in}}%
\pgfpathlineto{\pgfqpoint{6.147806in}{1.213483in}}%
\pgfpathlineto{\pgfqpoint{6.150383in}{1.214984in}}%
\pgfpathlineto{\pgfqpoint{6.152960in}{1.228306in}}%
\pgfpathlineto{\pgfqpoint{6.155537in}{1.228494in}}%
\pgfpathlineto{\pgfqpoint{6.163269in}{1.219488in}}%
\pgfpathlineto{\pgfqpoint{6.165846in}{1.224179in}}%
\pgfpathlineto{\pgfqpoint{6.168423in}{1.231309in}}%
\pgfpathlineto{\pgfqpoint{6.171000in}{1.289850in}}%
\pgfpathlineto{\pgfqpoint{6.173578in}{1.292102in}}%
\pgfpathlineto{\pgfqpoint{6.181309in}{1.299982in}}%
\pgfpathlineto{\pgfqpoint{6.183886in}{1.308426in}}%
\pgfpathlineto{\pgfqpoint{6.186464in}{1.296793in}}%
\pgfpathlineto{\pgfqpoint{6.191618in}{1.308801in}}%
\pgfpathlineto{\pgfqpoint{6.199350in}{1.307863in}}%
\pgfpathlineto{\pgfqpoint{6.201927in}{1.303360in}}%
\pgfpathlineto{\pgfqpoint{6.204504in}{1.296980in}}%
\pgfpathlineto{\pgfqpoint{6.207081in}{1.293415in}}%
\pgfpathlineto{\pgfqpoint{6.209659in}{1.294166in}}%
\pgfpathlineto{\pgfqpoint{6.217390in}{1.306550in}}%
\pgfpathlineto{\pgfqpoint{6.219967in}{1.301296in}}%
\pgfpathlineto{\pgfqpoint{6.222545in}{1.300170in}}%
\pgfpathlineto{\pgfqpoint{6.225122in}{1.291351in}}%
\pgfpathlineto{\pgfqpoint{6.227699in}{1.287786in}}%
\pgfpathlineto{\pgfqpoint{6.235431in}{1.295479in}}%
\pgfpathlineto{\pgfqpoint{6.238008in}{1.299232in}}%
\pgfpathlineto{\pgfqpoint{6.240585in}{1.297919in}}%
\pgfpathlineto{\pgfqpoint{6.243162in}{1.298481in}}%
\pgfpathlineto{\pgfqpoint{6.245740in}{1.304861in}}%
\pgfpathlineto{\pgfqpoint{6.253471in}{1.303548in}}%
\pgfpathlineto{\pgfqpoint{6.256048in}{1.301859in}}%
\pgfpathlineto{\pgfqpoint{6.258626in}{1.295292in}}%
\pgfpathlineto{\pgfqpoint{6.261203in}{1.292665in}}%
\pgfpathlineto{\pgfqpoint{6.263780in}{1.291727in}}%
\pgfpathlineto{\pgfqpoint{6.274089in}{1.282157in}}%
\pgfpathlineto{\pgfqpoint{6.276666in}{1.274464in}}%
\pgfpathlineto{\pgfqpoint{6.279243in}{1.262268in}}%
\pgfpathlineto{\pgfqpoint{6.281821in}{1.260767in}}%
\pgfpathlineto{\pgfqpoint{6.289552in}{1.263957in}}%
\pgfpathlineto{\pgfqpoint{6.294707in}{1.280281in}}%
\pgfpathlineto{\pgfqpoint{6.297284in}{1.279155in}}%
\pgfpathlineto{\pgfqpoint{6.299861in}{1.290788in}}%
\pgfpathlineto{\pgfqpoint{6.307593in}{1.294729in}}%
\pgfpathlineto{\pgfqpoint{6.310170in}{1.316119in}}%
\pgfpathlineto{\pgfqpoint{6.312747in}{1.318558in}}%
\pgfpathlineto{\pgfqpoint{6.315324in}{1.309364in}}%
\pgfpathlineto{\pgfqpoint{6.317902in}{1.325688in}}%
\pgfpathlineto{\pgfqpoint{6.325633in}{1.325688in}}%
\pgfpathlineto{\pgfqpoint{6.328210in}{1.319121in}}%
\pgfpathlineto{\pgfqpoint{6.330788in}{1.319121in}}%
\pgfpathlineto{\pgfqpoint{6.333365in}{1.317432in}}%
\pgfpathlineto{\pgfqpoint{6.335942in}{1.318746in}}%
\pgfpathlineto{\pgfqpoint{6.343674in}{1.316494in}}%
\pgfpathlineto{\pgfqpoint{6.346251in}{1.324938in}}%
\pgfpathlineto{\pgfqpoint{6.348828in}{1.325688in}}%
\pgfpathlineto{\pgfqpoint{6.351405in}{1.323437in}}%
\pgfpathlineto{\pgfqpoint{6.353982in}{1.317057in}}%
\pgfpathlineto{\pgfqpoint{6.364291in}{1.322874in}}%
\pgfpathlineto{\pgfqpoint{6.366869in}{1.317995in}}%
\pgfpathlineto{\pgfqpoint{6.372023in}{1.300545in}}%
\pgfpathlineto{\pgfqpoint{6.379755in}{1.304673in}}%
\pgfpathlineto{\pgfqpoint{6.384909in}{1.314243in}}%
\pgfpathlineto{\pgfqpoint{6.387486in}{1.324000in}}%
\pgfpathlineto{\pgfqpoint{6.390063in}{1.329629in}}%
\pgfpathlineto{\pgfqpoint{6.397795in}{1.320247in}}%
\pgfpathlineto{\pgfqpoint{6.400372in}{1.319309in}}%
\pgfpathlineto{\pgfqpoint{6.402950in}{1.314055in}}%
\pgfpathlineto{\pgfqpoint{6.405527in}{1.318558in}}%
\pgfpathlineto{\pgfqpoint{6.408104in}{1.318183in}}%
\pgfpathlineto{\pgfqpoint{6.415836in}{1.300170in}}%
\pgfpathlineto{\pgfqpoint{6.418413in}{1.300733in}}%
\pgfpathlineto{\pgfqpoint{6.420990in}{1.300170in}}%
\pgfpathlineto{\pgfqpoint{6.423567in}{1.293603in}}%
\pgfpathlineto{\pgfqpoint{6.426144in}{1.293040in}}%
\pgfpathlineto{\pgfqpoint{6.433876in}{1.260204in}}%
\pgfpathlineto{\pgfqpoint{6.436453in}{1.260767in}}%
\pgfpathlineto{\pgfqpoint{6.439031in}{1.258891in}}%
\pgfpathlineto{\pgfqpoint{6.441608in}{1.252136in}}%
\pgfpathlineto{\pgfqpoint{6.444185in}{1.248946in}}%
\pgfpathlineto{\pgfqpoint{6.451917in}{1.246507in}}%
\pgfpathlineto{\pgfqpoint{6.454494in}{1.237313in}}%
\pgfpathlineto{\pgfqpoint{6.457071in}{1.235437in}}%
\pgfpathlineto{\pgfqpoint{6.462225in}{1.258140in}}%
\pgfpathlineto{\pgfqpoint{6.469957in}{1.271837in}}%
\pgfpathlineto{\pgfqpoint{6.472534in}{1.271462in}}%
\pgfpathlineto{\pgfqpoint{6.475111in}{1.287411in}}%
\pgfpathlineto{\pgfqpoint{6.480266in}{1.285910in}}%
\pgfpathlineto{\pgfqpoint{6.487998in}{1.297168in}}%
\pgfpathlineto{\pgfqpoint{6.495729in}{1.353270in}}%
\pgfpathlineto{\pgfqpoint{6.498306in}{1.359462in}}%
\pgfpathlineto{\pgfqpoint{6.506038in}{1.367718in}}%
\pgfpathlineto{\pgfqpoint{6.508615in}{1.353646in}}%
\pgfpathlineto{\pgfqpoint{6.513770in}{1.345015in}}%
\pgfpathlineto{\pgfqpoint{6.516347in}{1.356648in}}%
\pgfpathlineto{\pgfqpoint{6.524079in}{1.369782in}}%
\pgfpathlineto{\pgfqpoint{6.526656in}{1.393236in}}%
\pgfpathlineto{\pgfqpoint{6.529233in}{1.387983in}}%
\pgfpathlineto{\pgfqpoint{6.531810in}{1.378413in}}%
\pgfpathlineto{\pgfqpoint{6.534387in}{1.384042in}}%
\pgfpathlineto{\pgfqpoint{6.542119in}{1.394175in}}%
\pgfpathlineto{\pgfqpoint{6.544696in}{1.386857in}}%
\pgfpathlineto{\pgfqpoint{6.547273in}{1.385919in}}%
\pgfpathlineto{\pgfqpoint{6.552428in}{1.393236in}}%
\pgfpathlineto{\pgfqpoint{6.562737in}{1.393612in}}%
\pgfpathlineto{\pgfqpoint{6.565314in}{1.394737in}}%
\pgfpathlineto{\pgfqpoint{6.567891in}{1.397364in}}%
\pgfpathlineto{\pgfqpoint{6.570468in}{1.388733in}}%
\pgfpathlineto{\pgfqpoint{6.570468in}{1.388733in}}%
\pgfusepath{stroke}%
\end{pgfscope}%
\begin{pgfscope}%
\pgfpathrectangle{\pgfqpoint{0.652287in}{0.521603in}}{\pgfqpoint{6.200000in}{4.530000in}}%
\pgfusepath{clip}%
\pgfsetroundcap%
\pgfsetroundjoin%
\pgfsetlinewidth{1.505625pt}%
\definecolor{currentstroke}{rgb}{0.737255,0.741176,0.133333}%
\pgfsetstrokecolor{currentstroke}%
\pgfsetdash{}{0pt}%
\pgfpathmoveto{\pgfqpoint{0.934105in}{0.893005in}}%
\pgfpathlineto{\pgfqpoint{0.936682in}{0.885312in}}%
\pgfpathlineto{\pgfqpoint{0.939259in}{0.888502in}}%
\pgfpathlineto{\pgfqpoint{0.941836in}{0.883624in}}%
\pgfpathlineto{\pgfqpoint{0.949568in}{0.880246in}}%
\pgfpathlineto{\pgfqpoint{0.952145in}{0.879871in}}%
\pgfpathlineto{\pgfqpoint{0.954722in}{0.877056in}}%
\pgfpathlineto{\pgfqpoint{0.957300in}{0.886250in}}%
\pgfpathlineto{\pgfqpoint{0.959877in}{0.883624in}}%
\pgfpathlineto{\pgfqpoint{0.970186in}{0.891129in}}%
\pgfpathlineto{\pgfqpoint{0.972763in}{0.896195in}}%
\pgfpathlineto{\pgfqpoint{0.975340in}{0.890378in}}%
\pgfpathlineto{\pgfqpoint{0.977917in}{0.883061in}}%
\pgfpathlineto{\pgfqpoint{0.985649in}{0.878933in}}%
\pgfpathlineto{\pgfqpoint{0.988226in}{0.884749in}}%
\pgfpathlineto{\pgfqpoint{0.990803in}{0.882873in}}%
\pgfpathlineto{\pgfqpoint{0.995958in}{0.884937in}}%
\pgfpathlineto{\pgfqpoint{1.003689in}{0.880246in}}%
\pgfpathlineto{\pgfqpoint{1.006267in}{0.883248in}}%
\pgfpathlineto{\pgfqpoint{1.008844in}{0.890941in}}%
\pgfpathlineto{\pgfqpoint{1.011421in}{0.905577in}}%
\pgfpathlineto{\pgfqpoint{1.013998in}{0.909517in}}%
\pgfpathlineto{\pgfqpoint{1.021730in}{0.910830in}}%
\pgfpathlineto{\pgfqpoint{1.024307in}{0.909329in}}%
\pgfpathlineto{\pgfqpoint{1.026884in}{0.914958in}}%
\pgfpathlineto{\pgfqpoint{1.029462in}{0.931658in}}%
\pgfpathlineto{\pgfqpoint{1.032039in}{0.937850in}}%
\pgfpathlineto{\pgfqpoint{1.039770in}{0.932971in}}%
\pgfpathlineto{\pgfqpoint{1.042348in}{0.943291in}}%
\pgfpathlineto{\pgfqpoint{1.044925in}{0.946293in}}%
\pgfpathlineto{\pgfqpoint{1.047502in}{0.941039in}}%
\pgfpathlineto{\pgfqpoint{1.050079in}{0.945918in}}%
\pgfpathlineto{\pgfqpoint{1.060388in}{0.942728in}}%
\pgfpathlineto{\pgfqpoint{1.062965in}{0.951547in}}%
\pgfpathlineto{\pgfqpoint{1.065543in}{0.952110in}}%
\pgfpathlineto{\pgfqpoint{1.068120in}{0.956425in}}%
\pgfpathlineto{\pgfqpoint{1.075851in}{0.953611in}}%
\pgfpathlineto{\pgfqpoint{1.078429in}{0.962242in}}%
\pgfpathlineto{\pgfqpoint{1.081006in}{0.951547in}}%
\pgfpathlineto{\pgfqpoint{1.083583in}{0.954924in}}%
\pgfpathlineto{\pgfqpoint{1.086160in}{0.950796in}}%
\pgfpathlineto{\pgfqpoint{1.093892in}{0.951172in}}%
\pgfpathlineto{\pgfqpoint{1.096469in}{0.945355in}}%
\pgfpathlineto{\pgfqpoint{1.099046in}{0.948545in}}%
\pgfpathlineto{\pgfqpoint{1.101624in}{0.958302in}}%
\pgfpathlineto{\pgfqpoint{1.104201in}{0.954924in}}%
\pgfpathlineto{\pgfqpoint{1.111932in}{0.952297in}}%
\pgfpathlineto{\pgfqpoint{1.114510in}{0.955300in}}%
\pgfpathlineto{\pgfqpoint{1.117087in}{0.953236in}}%
\pgfpathlineto{\pgfqpoint{1.119664in}{0.954174in}}%
\pgfpathlineto{\pgfqpoint{1.122241in}{0.952860in}}%
\pgfpathlineto{\pgfqpoint{1.129973in}{0.961867in}}%
\pgfpathlineto{\pgfqpoint{1.132550in}{0.952297in}}%
\pgfpathlineto{\pgfqpoint{1.135127in}{0.953236in}}%
\pgfpathlineto{\pgfqpoint{1.137704in}{0.955487in}}%
\pgfpathlineto{\pgfqpoint{1.140282in}{0.961679in}}%
\pgfpathlineto{\pgfqpoint{1.148013in}{0.966933in}}%
\pgfpathlineto{\pgfqpoint{1.150591in}{0.965807in}}%
\pgfpathlineto{\pgfqpoint{1.155745in}{0.962617in}}%
\pgfpathlineto{\pgfqpoint{1.158322in}{0.958302in}}%
\pgfpathlineto{\pgfqpoint{1.166054in}{0.962430in}}%
\pgfpathlineto{\pgfqpoint{1.168631in}{0.968246in}}%
\pgfpathlineto{\pgfqpoint{1.171208in}{0.962430in}}%
\pgfpathlineto{\pgfqpoint{1.173785in}{0.970873in}}%
\pgfpathlineto{\pgfqpoint{1.184094in}{0.964681in}}%
\pgfpathlineto{\pgfqpoint{1.186672in}{0.953236in}}%
\pgfpathlineto{\pgfqpoint{1.189249in}{0.955862in}}%
\pgfpathlineto{\pgfqpoint{1.191826in}{0.970873in}}%
\pgfpathlineto{\pgfqpoint{1.194403in}{0.979692in}}%
\pgfpathlineto{\pgfqpoint{1.202135in}{0.970123in}}%
\pgfpathlineto{\pgfqpoint{1.204712in}{0.975376in}}%
\pgfpathlineto{\pgfqpoint{1.207289in}{0.974063in}}%
\pgfpathlineto{\pgfqpoint{1.209866in}{0.971624in}}%
\pgfpathlineto{\pgfqpoint{1.212444in}{0.970873in}}%
\pgfpathlineto{\pgfqpoint{1.220175in}{0.958865in}}%
\pgfpathlineto{\pgfqpoint{1.222753in}{0.962242in}}%
\pgfpathlineto{\pgfqpoint{1.225330in}{0.974063in}}%
\pgfpathlineto{\pgfqpoint{1.227907in}{0.979880in}}%
\pgfpathlineto{\pgfqpoint{1.230484in}{0.981193in}}%
\pgfpathlineto{\pgfqpoint{1.238216in}{0.978941in}}%
\pgfpathlineto{\pgfqpoint{1.240793in}{0.979504in}}%
\pgfpathlineto{\pgfqpoint{1.243370in}{0.975752in}}%
\pgfpathlineto{\pgfqpoint{1.245947in}{0.951735in}}%
\pgfpathlineto{\pgfqpoint{1.248525in}{0.957551in}}%
\pgfpathlineto{\pgfqpoint{1.256256in}{0.960553in}}%
\pgfpathlineto{\pgfqpoint{1.258833in}{0.959052in}}%
\pgfpathlineto{\pgfqpoint{1.261411in}{0.956050in}}%
\pgfpathlineto{\pgfqpoint{1.263988in}{0.959052in}}%
\pgfpathlineto{\pgfqpoint{1.266565in}{0.958114in}}%
\pgfpathlineto{\pgfqpoint{1.274297in}{0.953048in}}%
\pgfpathlineto{\pgfqpoint{1.276874in}{0.952673in}}%
\pgfpathlineto{\pgfqpoint{1.279451in}{0.960178in}}%
\pgfpathlineto{\pgfqpoint{1.284606in}{0.939726in}}%
\pgfpathlineto{\pgfqpoint{1.292337in}{0.954737in}}%
\pgfpathlineto{\pgfqpoint{1.294914in}{0.963743in}}%
\pgfpathlineto{\pgfqpoint{1.300069in}{0.969560in}}%
\pgfpathlineto{\pgfqpoint{1.302646in}{0.967871in}}%
\pgfpathlineto{\pgfqpoint{1.312955in}{0.971624in}}%
\pgfpathlineto{\pgfqpoint{1.320687in}{0.938037in}}%
\pgfpathlineto{\pgfqpoint{1.328418in}{0.946106in}}%
\pgfpathlineto{\pgfqpoint{1.330995in}{0.944604in}}%
\pgfpathlineto{\pgfqpoint{1.333573in}{0.955112in}}%
\pgfpathlineto{\pgfqpoint{1.336150in}{0.957176in}}%
\pgfpathlineto{\pgfqpoint{1.338727in}{0.956238in}}%
\pgfpathlineto{\pgfqpoint{1.346459in}{0.955300in}}%
\pgfpathlineto{\pgfqpoint{1.349036in}{0.958677in}}%
\pgfpathlineto{\pgfqpoint{1.351613in}{0.949108in}}%
\pgfpathlineto{\pgfqpoint{1.356768in}{0.963743in}}%
\pgfpathlineto{\pgfqpoint{1.364499in}{0.970873in}}%
\pgfpathlineto{\pgfqpoint{1.369654in}{0.981193in}}%
\pgfpathlineto{\pgfqpoint{1.372231in}{0.968246in}}%
\pgfpathlineto{\pgfqpoint{1.374808in}{0.991138in}}%
\pgfpathlineto{\pgfqpoint{1.382540in}{0.975752in}}%
\pgfpathlineto{\pgfqpoint{1.385117in}{0.983632in}}%
\pgfpathlineto{\pgfqpoint{1.387694in}{0.984758in}}%
\pgfpathlineto{\pgfqpoint{1.390271in}{0.976877in}}%
\pgfpathlineto{\pgfqpoint{1.392849in}{0.985696in}}%
\pgfpathlineto{\pgfqpoint{1.400580in}{0.998080in}}%
\pgfpathlineto{\pgfqpoint{1.403157in}{0.996954in}}%
\pgfpathlineto{\pgfqpoint{1.408312in}{0.999769in}}%
\pgfpathlineto{\pgfqpoint{1.410889in}{0.992451in}}%
\pgfpathlineto{\pgfqpoint{1.418621in}{0.985696in}}%
\pgfpathlineto{\pgfqpoint{1.423775in}{0.973500in}}%
\pgfpathlineto{\pgfqpoint{1.428930in}{0.987573in}}%
\pgfpathlineto{\pgfqpoint{1.439238in}{1.004835in}}%
\pgfpathlineto{\pgfqpoint{1.441816in}{1.003146in}}%
\pgfpathlineto{\pgfqpoint{1.444393in}{0.993014in}}%
\pgfpathlineto{\pgfqpoint{1.446970in}{0.994515in}}%
\pgfpathlineto{\pgfqpoint{1.454702in}{0.989074in}}%
\pgfpathlineto{\pgfqpoint{1.457279in}{0.981568in}}%
\pgfpathlineto{\pgfqpoint{1.459856in}{0.979692in}}%
\pgfpathlineto{\pgfqpoint{1.462433in}{0.998643in}}%
\pgfpathlineto{\pgfqpoint{1.465010in}{1.008588in}}%
\pgfpathlineto{\pgfqpoint{1.472742in}{1.015718in}}%
\pgfpathlineto{\pgfqpoint{1.477897in}{1.001645in}}%
\pgfpathlineto{\pgfqpoint{1.483051in}{1.016281in}}%
\pgfpathlineto{\pgfqpoint{1.490783in}{1.016468in}}%
\pgfpathlineto{\pgfqpoint{1.493360in}{1.013841in}}%
\pgfpathlineto{\pgfqpoint{1.495937in}{1.017031in}}%
\pgfpathlineto{\pgfqpoint{1.498514in}{1.005961in}}%
\pgfpathlineto{\pgfqpoint{1.501091in}{1.008400in}}%
\pgfpathlineto{\pgfqpoint{1.508823in}{1.005210in}}%
\pgfpathlineto{\pgfqpoint{1.511400in}{1.011777in}}%
\pgfpathlineto{\pgfqpoint{1.513978in}{1.011965in}}%
\pgfpathlineto{\pgfqpoint{1.516555in}{1.014592in}}%
\pgfpathlineto{\pgfqpoint{1.519132in}{1.010464in}}%
\pgfpathlineto{\pgfqpoint{1.526864in}{1.008212in}}%
\pgfpathlineto{\pgfqpoint{1.529441in}{1.005398in}}%
\pgfpathlineto{\pgfqpoint{1.532018in}{1.007837in}}%
\pgfpathlineto{\pgfqpoint{1.534595in}{1.003897in}}%
\pgfpathlineto{\pgfqpoint{1.537172in}{1.001833in}}%
\pgfpathlineto{\pgfqpoint{1.544904in}{1.006148in}}%
\pgfpathlineto{\pgfqpoint{1.547481in}{1.005022in}}%
\pgfpathlineto{\pgfqpoint{1.550058in}{1.007649in}}%
\pgfpathlineto{\pgfqpoint{1.552636in}{1.002020in}}%
\pgfpathlineto{\pgfqpoint{1.555213in}{1.008588in}}%
\pgfpathlineto{\pgfqpoint{1.565522in}{1.009713in}}%
\pgfpathlineto{\pgfqpoint{1.568099in}{1.005773in}}%
\pgfpathlineto{\pgfqpoint{1.570676in}{1.013841in}}%
\pgfpathlineto{\pgfqpoint{1.573253in}{1.014592in}}%
\pgfpathlineto{\pgfqpoint{1.580985in}{1.010276in}}%
\pgfpathlineto{\pgfqpoint{1.583562in}{1.022097in}}%
\pgfpathlineto{\pgfqpoint{1.588717in}{1.036920in}}%
\pgfpathlineto{\pgfqpoint{1.591294in}{1.033730in}}%
\pgfpathlineto{\pgfqpoint{1.601603in}{1.031291in}}%
\pgfpathlineto{\pgfqpoint{1.604180in}{1.035982in}}%
\pgfpathlineto{\pgfqpoint{1.606757in}{1.035231in}}%
\pgfpathlineto{\pgfqpoint{1.609334in}{1.036920in}}%
\pgfpathlineto{\pgfqpoint{1.617066in}{1.031666in}}%
\pgfpathlineto{\pgfqpoint{1.619643in}{1.035044in}}%
\pgfpathlineto{\pgfqpoint{1.622220in}{1.026413in}}%
\pgfpathlineto{\pgfqpoint{1.624798in}{1.032417in}}%
\pgfpathlineto{\pgfqpoint{1.627375in}{1.033918in}}%
\pgfpathlineto{\pgfqpoint{1.635106in}{1.044238in}}%
\pgfpathlineto{\pgfqpoint{1.637684in}{1.041048in}}%
\pgfpathlineto{\pgfqpoint{1.640261in}{1.051743in}}%
\pgfpathlineto{\pgfqpoint{1.642838in}{1.056622in}}%
\pgfpathlineto{\pgfqpoint{1.645415in}{1.059061in}}%
\pgfpathlineto{\pgfqpoint{1.653147in}{1.052494in}}%
\pgfpathlineto{\pgfqpoint{1.655724in}{1.044238in}}%
\pgfpathlineto{\pgfqpoint{1.658301in}{1.047052in}}%
\pgfpathlineto{\pgfqpoint{1.660879in}{1.053995in}}%
\pgfpathlineto{\pgfqpoint{1.663456in}{1.054182in}}%
\pgfpathlineto{\pgfqpoint{1.671187in}{1.057560in}}%
\pgfpathlineto{\pgfqpoint{1.676342in}{1.068630in}}%
\pgfpathlineto{\pgfqpoint{1.678919in}{1.065816in}}%
\pgfpathlineto{\pgfqpoint{1.681496in}{1.057748in}}%
\pgfpathlineto{\pgfqpoint{1.689228in}{1.053807in}}%
\pgfpathlineto{\pgfqpoint{1.691805in}{1.043863in}}%
\pgfpathlineto{\pgfqpoint{1.694382in}{1.043112in}}%
\pgfpathlineto{\pgfqpoint{1.696960in}{1.047615in}}%
\pgfpathlineto{\pgfqpoint{1.699537in}{1.050805in}}%
\pgfpathlineto{\pgfqpoint{1.712423in}{1.052681in}}%
\pgfpathlineto{\pgfqpoint{1.715000in}{1.074259in}}%
\pgfpathlineto{\pgfqpoint{1.717577in}{1.072195in}}%
\pgfpathlineto{\pgfqpoint{1.725309in}{1.064315in}}%
\pgfpathlineto{\pgfqpoint{1.727886in}{1.074072in}}%
\pgfpathlineto{\pgfqpoint{1.730463in}{1.068630in}}%
\pgfpathlineto{\pgfqpoint{1.733041in}{1.066566in}}%
\pgfpathlineto{\pgfqpoint{1.735618in}{1.070319in}}%
\pgfpathlineto{\pgfqpoint{1.743349in}{1.071632in}}%
\pgfpathlineto{\pgfqpoint{1.745927in}{1.068255in}}%
\pgfpathlineto{\pgfqpoint{1.748504in}{1.063189in}}%
\pgfpathlineto{\pgfqpoint{1.751081in}{1.063377in}}%
\pgfpathlineto{\pgfqpoint{1.753658in}{1.075010in}}%
\pgfpathlineto{\pgfqpoint{1.761390in}{1.087394in}}%
\pgfpathlineto{\pgfqpoint{1.763967in}{1.090396in}}%
\pgfpathlineto{\pgfqpoint{1.766544in}{1.091709in}}%
\pgfpathlineto{\pgfqpoint{1.771699in}{1.097901in}}%
\pgfpathlineto{\pgfqpoint{1.779430in}{1.094148in}}%
\pgfpathlineto{\pgfqpoint{1.782008in}{1.092272in}}%
\pgfpathlineto{\pgfqpoint{1.784585in}{1.094336in}}%
\pgfpathlineto{\pgfqpoint{1.787162in}{1.098089in}}%
\pgfpathlineto{\pgfqpoint{1.789739in}{1.104656in}}%
\pgfpathlineto{\pgfqpoint{1.797471in}{1.100153in}}%
\pgfpathlineto{\pgfqpoint{1.800048in}{1.095650in}}%
\pgfpathlineto{\pgfqpoint{1.805203in}{1.099402in}}%
\pgfpathlineto{\pgfqpoint{1.807780in}{1.099777in}}%
\pgfpathlineto{\pgfqpoint{1.815511in}{1.100153in}}%
\pgfpathlineto{\pgfqpoint{1.818089in}{1.101279in}}%
\pgfpathlineto{\pgfqpoint{1.820666in}{1.096775in}}%
\pgfpathlineto{\pgfqpoint{1.823243in}{1.093961in}}%
\pgfpathlineto{\pgfqpoint{1.825820in}{1.092272in}}%
\pgfpathlineto{\pgfqpoint{1.833552in}{1.102217in}}%
\pgfpathlineto{\pgfqpoint{1.836129in}{1.108971in}}%
\pgfpathlineto{\pgfqpoint{1.838706in}{1.102217in}}%
\pgfpathlineto{\pgfqpoint{1.841283in}{1.116289in}}%
\pgfpathlineto{\pgfqpoint{1.843861in}{1.109159in}}%
\pgfpathlineto{\pgfqpoint{1.851592in}{1.110848in}}%
\pgfpathlineto{\pgfqpoint{1.856747in}{1.104844in}}%
\pgfpathlineto{\pgfqpoint{1.859324in}{1.103342in}}%
\pgfpathlineto{\pgfqpoint{1.861901in}{1.100153in}}%
\pgfpathlineto{\pgfqpoint{1.869633in}{1.112537in}}%
\pgfpathlineto{\pgfqpoint{1.874787in}{1.128673in}}%
\pgfpathlineto{\pgfqpoint{1.877364in}{1.129236in}}%
\pgfpathlineto{\pgfqpoint{1.879942in}{1.134490in}}%
\pgfpathlineto{\pgfqpoint{1.887673in}{1.139368in}}%
\pgfpathlineto{\pgfqpoint{1.890251in}{1.145560in}}%
\pgfpathlineto{\pgfqpoint{1.892828in}{1.155880in}}%
\pgfpathlineto{\pgfqpoint{1.895405in}{1.150439in}}%
\pgfpathlineto{\pgfqpoint{1.897982in}{1.153253in}}%
\pgfpathlineto{\pgfqpoint{1.910868in}{1.149125in}}%
\pgfpathlineto{\pgfqpoint{1.913445in}{1.147812in}}%
\pgfpathlineto{\pgfqpoint{1.916023in}{1.140869in}}%
\pgfpathlineto{\pgfqpoint{1.926332in}{1.144247in}}%
\pgfpathlineto{\pgfqpoint{1.928909in}{1.144247in}}%
\pgfpathlineto{\pgfqpoint{1.931486in}{1.146498in}}%
\pgfpathlineto{\pgfqpoint{1.934063in}{1.147624in}}%
\pgfpathlineto{\pgfqpoint{1.941795in}{1.132801in}}%
\pgfpathlineto{\pgfqpoint{1.944372in}{1.134114in}}%
\pgfpathlineto{\pgfqpoint{1.946949in}{1.127172in}}%
\pgfpathlineto{\pgfqpoint{1.949526in}{1.139368in}}%
\pgfpathlineto{\pgfqpoint{1.952104in}{1.142183in}}%
\pgfpathlineto{\pgfqpoint{1.959835in}{1.134114in}}%
\pgfpathlineto{\pgfqpoint{1.962412in}{1.146311in}}%
\pgfpathlineto{\pgfqpoint{1.964990in}{1.151752in}}%
\pgfpathlineto{\pgfqpoint{1.967567in}{1.135803in}}%
\pgfpathlineto{\pgfqpoint{1.970144in}{1.138993in}}%
\pgfpathlineto{\pgfqpoint{1.977876in}{1.132801in}}%
\pgfpathlineto{\pgfqpoint{1.980453in}{1.134677in}}%
\pgfpathlineto{\pgfqpoint{1.983030in}{1.131675in}}%
\pgfpathlineto{\pgfqpoint{1.985607in}{1.136929in}}%
\pgfpathlineto{\pgfqpoint{1.988185in}{1.145372in}}%
\pgfpathlineto{\pgfqpoint{1.998493in}{1.143871in}}%
\pgfpathlineto{\pgfqpoint{2.001071in}{1.134490in}}%
\pgfpathlineto{\pgfqpoint{2.006225in}{1.151564in}}%
\pgfpathlineto{\pgfqpoint{2.013957in}{1.136366in}}%
\pgfpathlineto{\pgfqpoint{2.019111in}{1.153441in}}%
\pgfpathlineto{\pgfqpoint{2.021688in}{1.148187in}}%
\pgfpathlineto{\pgfqpoint{2.024266in}{1.145748in}}%
\pgfpathlineto{\pgfqpoint{2.031997in}{1.151940in}}%
\pgfpathlineto{\pgfqpoint{2.037152in}{1.158507in}}%
\pgfpathlineto{\pgfqpoint{2.039729in}{1.156255in}}%
\pgfpathlineto{\pgfqpoint{2.050038in}{1.159070in}}%
\pgfpathlineto{\pgfqpoint{2.052615in}{1.153253in}}%
\pgfpathlineto{\pgfqpoint{2.055192in}{1.151377in}}%
\pgfpathlineto{\pgfqpoint{2.057769in}{1.156818in}}%
\pgfpathlineto{\pgfqpoint{2.060347in}{1.147812in}}%
\pgfpathlineto{\pgfqpoint{2.068078in}{1.145748in}}%
\pgfpathlineto{\pgfqpoint{2.070655in}{1.136929in}}%
\pgfpathlineto{\pgfqpoint{2.073233in}{1.151189in}}%
\pgfpathlineto{\pgfqpoint{2.075810in}{1.144622in}}%
\pgfpathlineto{\pgfqpoint{2.078387in}{1.154379in}}%
\pgfpathlineto{\pgfqpoint{2.086119in}{1.171078in}}%
\pgfpathlineto{\pgfqpoint{2.088696in}{1.185901in}}%
\pgfpathlineto{\pgfqpoint{2.093850in}{1.196221in}}%
\pgfpathlineto{\pgfqpoint{2.104159in}{1.185526in}}%
\pgfpathlineto{\pgfqpoint{2.106736in}{1.187965in}}%
\pgfpathlineto{\pgfqpoint{2.109314in}{1.174456in}}%
\pgfpathlineto{\pgfqpoint{2.111891in}{1.181773in}}%
\pgfpathlineto{\pgfqpoint{2.114468in}{1.176144in}}%
\pgfpathlineto{\pgfqpoint{2.122200in}{1.181023in}}%
\pgfpathlineto{\pgfqpoint{2.124777in}{1.174456in}}%
\pgfpathlineto{\pgfqpoint{2.127354in}{1.184025in}}%
\pgfpathlineto{\pgfqpoint{2.129931in}{1.186652in}}%
\pgfpathlineto{\pgfqpoint{2.132509in}{1.178584in}}%
\pgfpathlineto{\pgfqpoint{2.140240in}{1.159633in}}%
\pgfpathlineto{\pgfqpoint{2.142817in}{1.174268in}}%
\pgfpathlineto{\pgfqpoint{2.145395in}{1.163573in}}%
\pgfpathlineto{\pgfqpoint{2.147972in}{1.160008in}}%
\pgfpathlineto{\pgfqpoint{2.150549in}{1.170891in}}%
\pgfpathlineto{\pgfqpoint{2.158281in}{1.169202in}}%
\pgfpathlineto{\pgfqpoint{2.160858in}{1.178208in}}%
\pgfpathlineto{\pgfqpoint{2.166012in}{1.191530in}}%
\pgfpathlineto{\pgfqpoint{2.168589in}{1.184963in}}%
\pgfpathlineto{\pgfqpoint{2.176321in}{1.187778in}}%
\pgfpathlineto{\pgfqpoint{2.178898in}{1.190217in}}%
\pgfpathlineto{\pgfqpoint{2.181476in}{1.179709in}}%
\pgfpathlineto{\pgfqpoint{2.184053in}{1.219863in}}%
\pgfpathlineto{\pgfqpoint{2.186630in}{1.237688in}}%
\pgfpathlineto{\pgfqpoint{2.194362in}{1.234874in}}%
\pgfpathlineto{\pgfqpoint{2.196939in}{1.238626in}}%
\pgfpathlineto{\pgfqpoint{2.202093in}{1.233560in}}%
\pgfpathlineto{\pgfqpoint{2.204670in}{1.234311in}}%
\pgfpathlineto{\pgfqpoint{2.212402in}{1.235061in}}%
\pgfpathlineto{\pgfqpoint{2.214979in}{1.241066in}}%
\pgfpathlineto{\pgfqpoint{2.217557in}{1.252886in}}%
\pgfpathlineto{\pgfqpoint{2.220134in}{1.244443in}}%
\pgfpathlineto{\pgfqpoint{2.222711in}{1.264895in}}%
\pgfpathlineto{\pgfqpoint{2.230443in}{1.252136in}}%
\pgfpathlineto{\pgfqpoint{2.233020in}{1.251948in}}%
\pgfpathlineto{\pgfqpoint{2.235597in}{1.245944in}}%
\pgfpathlineto{\pgfqpoint{2.238174in}{1.237125in}}%
\pgfpathlineto{\pgfqpoint{2.240751in}{1.247257in}}%
\pgfpathlineto{\pgfqpoint{2.251060in}{1.245381in}}%
\pgfpathlineto{\pgfqpoint{2.253637in}{1.237313in}}%
\pgfpathlineto{\pgfqpoint{2.256215in}{1.249697in}}%
\pgfpathlineto{\pgfqpoint{2.258792in}{1.237313in}}%
\pgfpathlineto{\pgfqpoint{2.266524in}{1.246132in}}%
\pgfpathlineto{\pgfqpoint{2.269101in}{1.246132in}}%
\pgfpathlineto{\pgfqpoint{2.271678in}{1.233373in}}%
\pgfpathlineto{\pgfqpoint{2.274255in}{1.242567in}}%
\pgfpathlineto{\pgfqpoint{2.276832in}{1.245006in}}%
\pgfpathlineto{\pgfqpoint{2.284564in}{1.255326in}}%
\pgfpathlineto{\pgfqpoint{2.287141in}{1.243880in}}%
\pgfpathlineto{\pgfqpoint{2.289718in}{1.241628in}}%
\pgfpathlineto{\pgfqpoint{2.292296in}{1.254950in}}%
\pgfpathlineto{\pgfqpoint{2.294873in}{1.249321in}}%
\pgfpathlineto{\pgfqpoint{2.302605in}{1.255326in}}%
\pgfpathlineto{\pgfqpoint{2.305182in}{1.262643in}}%
\pgfpathlineto{\pgfqpoint{2.307759in}{1.256639in}}%
\pgfpathlineto{\pgfqpoint{2.310336in}{1.238626in}}%
\pgfpathlineto{\pgfqpoint{2.312913in}{1.243130in}}%
\pgfpathlineto{\pgfqpoint{2.320645in}{1.237876in}}%
\pgfpathlineto{\pgfqpoint{2.325799in}{1.256639in}}%
\pgfpathlineto{\pgfqpoint{2.328377in}{1.263582in}}%
\pgfpathlineto{\pgfqpoint{2.330954in}{1.257202in}}%
\pgfpathlineto{\pgfqpoint{2.338685in}{1.267710in}}%
\pgfpathlineto{\pgfqpoint{2.341263in}{1.267897in}}%
\pgfpathlineto{\pgfqpoint{2.348994in}{1.291727in}}%
\pgfpathlineto{\pgfqpoint{2.359303in}{1.276341in}}%
\pgfpathlineto{\pgfqpoint{2.361880in}{1.274089in}}%
\pgfpathlineto{\pgfqpoint{2.364458in}{1.288162in}}%
\pgfpathlineto{\pgfqpoint{2.367035in}{1.291539in}}%
\pgfpathlineto{\pgfqpoint{2.374766in}{1.289475in}}%
\pgfpathlineto{\pgfqpoint{2.377344in}{1.285722in}}%
\pgfpathlineto{\pgfqpoint{2.379921in}{1.286848in}}%
\pgfpathlineto{\pgfqpoint{2.382498in}{1.292665in}}%
\pgfpathlineto{\pgfqpoint{2.385075in}{1.287974in}}%
\pgfpathlineto{\pgfqpoint{2.392807in}{1.293603in}}%
\pgfpathlineto{\pgfqpoint{2.395384in}{1.282345in}}%
\pgfpathlineto{\pgfqpoint{2.397961in}{1.274464in}}%
\pgfpathlineto{\pgfqpoint{2.400539in}{1.308238in}}%
\pgfpathlineto{\pgfqpoint{2.403116in}{1.302234in}}%
\pgfpathlineto{\pgfqpoint{2.413425in}{1.294541in}}%
\pgfpathlineto{\pgfqpoint{2.416002in}{1.232434in}}%
\pgfpathlineto{\pgfqpoint{2.418579in}{1.241628in}}%
\pgfpathlineto{\pgfqpoint{2.421156in}{1.262456in}}%
\pgfpathlineto{\pgfqpoint{2.428888in}{1.264895in}}%
\pgfpathlineto{\pgfqpoint{2.434042in}{1.251198in}}%
\pgfpathlineto{\pgfqpoint{2.436620in}{1.248571in}}%
\pgfpathlineto{\pgfqpoint{2.439197in}{1.244443in}}%
\pgfpathlineto{\pgfqpoint{2.446928in}{1.244255in}}%
\pgfpathlineto{\pgfqpoint{2.449506in}{1.242004in}}%
\pgfpathlineto{\pgfqpoint{2.452083in}{1.243880in}}%
\pgfpathlineto{\pgfqpoint{2.454660in}{1.224741in}}%
\pgfpathlineto{\pgfqpoint{2.457237in}{1.221176in}}%
\pgfpathlineto{\pgfqpoint{2.464969in}{1.229245in}}%
\pgfpathlineto{\pgfqpoint{2.467546in}{1.221552in}}%
\pgfpathlineto{\pgfqpoint{2.470123in}{1.244068in}}%
\pgfpathlineto{\pgfqpoint{2.472701in}{1.246319in}}%
\pgfpathlineto{\pgfqpoint{2.475278in}{1.247070in}}%
\pgfpathlineto{\pgfqpoint{2.483009in}{1.229245in}}%
\pgfpathlineto{\pgfqpoint{2.485587in}{1.225680in}}%
\pgfpathlineto{\pgfqpoint{2.488164in}{1.230183in}}%
\pgfpathlineto{\pgfqpoint{2.490741in}{1.230933in}}%
\pgfpathlineto{\pgfqpoint{2.493318in}{1.226805in}}%
\pgfpathlineto{\pgfqpoint{2.503627in}{1.238063in}}%
\pgfpathlineto{\pgfqpoint{2.506204in}{1.234498in}}%
\pgfpathlineto{\pgfqpoint{2.508782in}{1.234498in}}%
\pgfpathlineto{\pgfqpoint{2.511359in}{1.236562in}}%
\pgfpathlineto{\pgfqpoint{2.519090in}{1.244818in}}%
\pgfpathlineto{\pgfqpoint{2.521668in}{1.270899in}}%
\pgfpathlineto{\pgfqpoint{2.524245in}{1.278405in}}%
\pgfpathlineto{\pgfqpoint{2.526822in}{1.272963in}}%
\pgfpathlineto{\pgfqpoint{2.529399in}{1.290038in}}%
\pgfpathlineto{\pgfqpoint{2.537131in}{1.291727in}}%
\pgfpathlineto{\pgfqpoint{2.542285in}{1.310302in}}%
\pgfpathlineto{\pgfqpoint{2.544862in}{1.314993in}}%
\pgfpathlineto{\pgfqpoint{2.547440in}{1.332818in}}%
\pgfpathlineto{\pgfqpoint{2.555171in}{1.321560in}}%
\pgfpathlineto{\pgfqpoint{2.557749in}{1.308989in}}%
\pgfpathlineto{\pgfqpoint{2.560326in}{1.301296in}}%
\pgfpathlineto{\pgfqpoint{2.562903in}{1.309927in}}%
\pgfpathlineto{\pgfqpoint{2.565480in}{1.307675in}}%
\pgfpathlineto{\pgfqpoint{2.573212in}{1.299232in}}%
\pgfpathlineto{\pgfqpoint{2.575789in}{1.308426in}}%
\pgfpathlineto{\pgfqpoint{2.578366in}{1.302422in}}%
\pgfpathlineto{\pgfqpoint{2.580943in}{1.288537in}}%
\pgfpathlineto{\pgfqpoint{2.583521in}{1.296605in}}%
\pgfpathlineto{\pgfqpoint{2.591252in}{1.278592in}}%
\pgfpathlineto{\pgfqpoint{2.593830in}{1.262080in}}%
\pgfpathlineto{\pgfqpoint{2.596407in}{1.267710in}}%
\pgfpathlineto{\pgfqpoint{2.598984in}{1.290226in}}%
\pgfpathlineto{\pgfqpoint{2.601561in}{1.303923in}}%
\pgfpathlineto{\pgfqpoint{2.609293in}{1.309364in}}%
\pgfpathlineto{\pgfqpoint{2.611870in}{1.300358in}}%
\pgfpathlineto{\pgfqpoint{2.614447in}{1.318933in}}%
\pgfpathlineto{\pgfqpoint{2.619602in}{1.339761in}}%
\pgfpathlineto{\pgfqpoint{2.629911in}{1.337697in}}%
\pgfpathlineto{\pgfqpoint{2.632488in}{1.333006in}}%
\pgfpathlineto{\pgfqpoint{2.635065in}{1.350456in}}%
\pgfpathlineto{\pgfqpoint{2.637642in}{1.351206in}}%
\pgfpathlineto{\pgfqpoint{2.645374in}{1.351206in}}%
\pgfpathlineto{\pgfqpoint{2.647951in}{1.356273in}}%
\pgfpathlineto{\pgfqpoint{2.650528in}{1.354396in}}%
\pgfpathlineto{\pgfqpoint{2.653105in}{1.323437in}}%
\pgfpathlineto{\pgfqpoint{2.655683in}{1.334319in}}%
\pgfpathlineto{\pgfqpoint{2.663414in}{1.322311in}}%
\pgfpathlineto{\pgfqpoint{2.665991in}{1.326439in}}%
\pgfpathlineto{\pgfqpoint{2.668569in}{1.333569in}}%
\pgfpathlineto{\pgfqpoint{2.671146in}{1.320997in}}%
\pgfpathlineto{\pgfqpoint{2.673723in}{1.330379in}}%
\pgfpathlineto{\pgfqpoint{2.681455in}{1.334695in}}%
\pgfpathlineto{\pgfqpoint{2.684032in}{1.330379in}}%
\pgfpathlineto{\pgfqpoint{2.686609in}{1.343889in}}%
\pgfpathlineto{\pgfqpoint{2.689186in}{1.345577in}}%
\pgfpathlineto{\pgfqpoint{2.691764in}{1.353646in}}%
\pgfpathlineto{\pgfqpoint{2.699495in}{1.344452in}}%
\pgfpathlineto{\pgfqpoint{2.702072in}{1.333569in}}%
\pgfpathlineto{\pgfqpoint{2.704650in}{1.336759in}}%
\pgfpathlineto{\pgfqpoint{2.707227in}{1.351957in}}%
\pgfpathlineto{\pgfqpoint{2.709804in}{1.354209in}}%
\pgfpathlineto{\pgfqpoint{2.717536in}{1.353458in}}%
\pgfpathlineto{\pgfqpoint{2.720113in}{1.360400in}}%
\pgfpathlineto{\pgfqpoint{2.722690in}{1.362464in}}%
\pgfpathlineto{\pgfqpoint{2.727845in}{1.360025in}}%
\pgfpathlineto{\pgfqpoint{2.735576in}{1.367343in}}%
\pgfpathlineto{\pgfqpoint{2.738153in}{1.352520in}}%
\pgfpathlineto{\pgfqpoint{2.740731in}{1.357023in}}%
\pgfpathlineto{\pgfqpoint{2.743308in}{1.352332in}}%
\pgfpathlineto{\pgfqpoint{2.745885in}{1.353083in}}%
\pgfpathlineto{\pgfqpoint{2.753617in}{1.351957in}}%
\pgfpathlineto{\pgfqpoint{2.756194in}{1.342388in}}%
\pgfpathlineto{\pgfqpoint{2.758771in}{1.369782in}}%
\pgfpathlineto{\pgfqpoint{2.761348in}{1.360025in}}%
\pgfpathlineto{\pgfqpoint{2.763926in}{1.377100in}}%
\pgfpathlineto{\pgfqpoint{2.771657in}{1.378789in}}%
\pgfpathlineto{\pgfqpoint{2.774234in}{1.402993in}}%
\pgfpathlineto{\pgfqpoint{2.776812in}{1.412187in}}%
\pgfpathlineto{\pgfqpoint{2.779389in}{1.415377in}}%
\pgfpathlineto{\pgfqpoint{2.781966in}{1.414814in}}%
\pgfpathlineto{\pgfqpoint{2.797429in}{1.433390in}}%
\pgfpathlineto{\pgfqpoint{2.800007in}{1.430951in}}%
\pgfpathlineto{\pgfqpoint{2.807738in}{1.436392in}}%
\pgfpathlineto{\pgfqpoint{2.810315in}{1.444273in}}%
\pgfpathlineto{\pgfqpoint{2.815470in}{1.436955in}}%
\pgfpathlineto{\pgfqpoint{2.818047in}{1.437518in}}%
\pgfpathlineto{\pgfqpoint{2.825779in}{1.431701in}}%
\pgfpathlineto{\pgfqpoint{2.828356in}{1.439019in}}%
\pgfpathlineto{\pgfqpoint{2.830933in}{1.442209in}}%
\pgfpathlineto{\pgfqpoint{2.833510in}{1.440895in}}%
\pgfpathlineto{\pgfqpoint{2.843819in}{1.427761in}}%
\pgfpathlineto{\pgfqpoint{2.846396in}{1.444085in}}%
\pgfpathlineto{\pgfqpoint{2.848974in}{1.448964in}}%
\pgfpathlineto{\pgfqpoint{2.851551in}{1.440145in}}%
\pgfpathlineto{\pgfqpoint{2.854128in}{1.485740in}}%
\pgfpathlineto{\pgfqpoint{2.864437in}{1.484614in}}%
\pgfpathlineto{\pgfqpoint{2.867014in}{1.489305in}}%
\pgfpathlineto{\pgfqpoint{2.869591in}{1.468665in}}%
\pgfpathlineto{\pgfqpoint{2.872168in}{1.437893in}}%
\pgfpathlineto{\pgfqpoint{2.879900in}{1.415940in}}%
\pgfpathlineto{\pgfqpoint{2.882477in}{1.436767in}}%
\pgfpathlineto{\pgfqpoint{2.885055in}{1.419880in}}%
\pgfpathlineto{\pgfqpoint{2.887632in}{1.436392in}}%
\pgfpathlineto{\pgfqpoint{2.890209in}{1.412563in}}%
\pgfpathlineto{\pgfqpoint{2.897941in}{1.403931in}}%
\pgfpathlineto{\pgfqpoint{2.903095in}{1.413313in}}%
\pgfpathlineto{\pgfqpoint{2.908249in}{1.440332in}}%
\pgfpathlineto{\pgfqpoint{2.915981in}{1.434891in}}%
\pgfpathlineto{\pgfqpoint{2.918558in}{1.442772in}}%
\pgfpathlineto{\pgfqpoint{2.921136in}{1.457782in}}%
\pgfpathlineto{\pgfqpoint{2.923713in}{1.457407in}}%
\pgfpathlineto{\pgfqpoint{2.926290in}{1.465851in}}%
\pgfpathlineto{\pgfqpoint{2.936599in}{1.466226in}}%
\pgfpathlineto{\pgfqpoint{2.939176in}{1.456657in}}%
\pgfpathlineto{\pgfqpoint{2.941753in}{1.454780in}}%
\pgfpathlineto{\pgfqpoint{2.944330in}{1.454217in}}%
\pgfpathlineto{\pgfqpoint{2.954639in}{1.470729in}}%
\pgfpathlineto{\pgfqpoint{2.957216in}{1.466413in}}%
\pgfpathlineto{\pgfqpoint{2.959794in}{1.467164in}}%
\pgfpathlineto{\pgfqpoint{2.962371in}{1.465663in}}%
\pgfpathlineto{\pgfqpoint{2.970103in}{1.445774in}}%
\pgfpathlineto{\pgfqpoint{2.972680in}{1.463787in}}%
\pgfpathlineto{\pgfqpoint{2.975257in}{1.451778in}}%
\pgfpathlineto{\pgfqpoint{2.977834in}{1.456469in}}%
\pgfpathlineto{\pgfqpoint{2.980411in}{1.463974in}}%
\pgfpathlineto{\pgfqpoint{2.988143in}{1.463787in}}%
\pgfpathlineto{\pgfqpoint{2.990720in}{1.471104in}}%
\pgfpathlineto{\pgfqpoint{2.993297in}{1.466413in}}%
\pgfpathlineto{\pgfqpoint{2.995875in}{1.442959in}}%
\pgfpathlineto{\pgfqpoint{2.998452in}{1.442772in}}%
\pgfpathlineto{\pgfqpoint{3.006184in}{1.456094in}}%
\pgfpathlineto{\pgfqpoint{3.008761in}{1.467727in}}%
\pgfpathlineto{\pgfqpoint{3.011338in}{1.456281in}}%
\pgfpathlineto{\pgfqpoint{3.013915in}{1.447462in}}%
\pgfpathlineto{\pgfqpoint{3.016492in}{1.454405in}}%
\pgfpathlineto{\pgfqpoint{3.024224in}{1.442772in}}%
\pgfpathlineto{\pgfqpoint{3.029378in}{1.420443in}}%
\pgfpathlineto{\pgfqpoint{3.031956in}{1.420819in}}%
\pgfpathlineto{\pgfqpoint{3.034533in}{1.404682in}}%
\pgfpathlineto{\pgfqpoint{3.042264in}{1.421194in}}%
\pgfpathlineto{\pgfqpoint{3.044842in}{1.416128in}}%
\pgfpathlineto{\pgfqpoint{3.047419in}{1.415940in}}%
\pgfpathlineto{\pgfqpoint{3.049996in}{1.417441in}}%
\pgfpathlineto{\pgfqpoint{3.052573in}{1.385168in}}%
\pgfpathlineto{\pgfqpoint{3.060305in}{1.366405in}}%
\pgfpathlineto{\pgfqpoint{3.062882in}{1.362652in}}%
\pgfpathlineto{\pgfqpoint{3.065459in}{1.384605in}}%
\pgfpathlineto{\pgfqpoint{3.070614in}{1.336383in}}%
\pgfpathlineto{\pgfqpoint{3.078345in}{1.355710in}}%
\pgfpathlineto{\pgfqpoint{3.080923in}{1.369219in}}%
\pgfpathlineto{\pgfqpoint{3.083500in}{1.392486in}}%
\pgfpathlineto{\pgfqpoint{3.086077in}{1.386294in}}%
\pgfpathlineto{\pgfqpoint{3.096386in}{1.391548in}}%
\pgfpathlineto{\pgfqpoint{3.098963in}{1.395300in}}%
\pgfpathlineto{\pgfqpoint{3.101540in}{1.390234in}}%
\pgfpathlineto{\pgfqpoint{3.104118in}{1.392673in}}%
\pgfpathlineto{\pgfqpoint{3.106695in}{1.346516in}}%
\pgfpathlineto{\pgfqpoint{3.114426in}{1.357586in}}%
\pgfpathlineto{\pgfqpoint{3.117004in}{1.363027in}}%
\pgfpathlineto{\pgfqpoint{3.119581in}{1.362840in}}%
\pgfpathlineto{\pgfqpoint{3.122158in}{1.378226in}}%
\pgfpathlineto{\pgfqpoint{3.124735in}{1.370720in}}%
\pgfpathlineto{\pgfqpoint{3.132467in}{1.382729in}}%
\pgfpathlineto{\pgfqpoint{3.135044in}{1.374848in}}%
\pgfpathlineto{\pgfqpoint{3.137621in}{1.389671in}}%
\pgfpathlineto{\pgfqpoint{3.140199in}{1.399428in}}%
\pgfpathlineto{\pgfqpoint{3.142776in}{1.398865in}}%
\pgfpathlineto{\pgfqpoint{3.153085in}{1.404494in}}%
\pgfpathlineto{\pgfqpoint{3.155662in}{1.401868in}}%
\pgfpathlineto{\pgfqpoint{3.158239in}{1.391172in}}%
\pgfpathlineto{\pgfqpoint{3.160816in}{1.401492in}}%
\pgfpathlineto{\pgfqpoint{3.168548in}{1.403931in}}%
\pgfpathlineto{\pgfqpoint{3.171125in}{1.393236in}}%
\pgfpathlineto{\pgfqpoint{3.173702in}{1.402806in}}%
\pgfpathlineto{\pgfqpoint{3.176280in}{1.399616in}}%
\pgfpathlineto{\pgfqpoint{3.178857in}{1.411624in}}%
\pgfpathlineto{\pgfqpoint{3.189166in}{1.422320in}}%
\pgfpathlineto{\pgfqpoint{3.191743in}{1.419317in}}%
\pgfpathlineto{\pgfqpoint{3.194320in}{1.423070in}}%
\pgfpathlineto{\pgfqpoint{3.196897in}{1.423821in}}%
\pgfpathlineto{\pgfqpoint{3.204629in}{1.418004in}}%
\pgfpathlineto{\pgfqpoint{3.207206in}{1.408247in}}%
\pgfpathlineto{\pgfqpoint{3.209783in}{1.408810in}}%
\pgfpathlineto{\pgfqpoint{3.214938in}{1.415752in}}%
\pgfpathlineto{\pgfqpoint{3.222669in}{1.413876in}}%
\pgfpathlineto{\pgfqpoint{3.225247in}{1.421381in}}%
\pgfpathlineto{\pgfqpoint{3.227824in}{1.414251in}}%
\pgfpathlineto{\pgfqpoint{3.230401in}{1.410311in}}%
\pgfpathlineto{\pgfqpoint{3.232978in}{1.408059in}}%
\pgfpathlineto{\pgfqpoint{3.240710in}{1.403556in}}%
\pgfpathlineto{\pgfqpoint{3.245864in}{1.408059in}}%
\pgfpathlineto{\pgfqpoint{3.251019in}{1.400179in}}%
\pgfpathlineto{\pgfqpoint{3.258750in}{1.399991in}}%
\pgfpathlineto{\pgfqpoint{3.261328in}{1.392673in}}%
\pgfpathlineto{\pgfqpoint{3.263905in}{1.398302in}}%
\pgfpathlineto{\pgfqpoint{3.266482in}{1.397740in}}%
\pgfpathlineto{\pgfqpoint{3.269059in}{1.399241in}}%
\pgfpathlineto{\pgfqpoint{3.276791in}{1.405620in}}%
\pgfpathlineto{\pgfqpoint{3.279368in}{1.421381in}}%
\pgfpathlineto{\pgfqpoint{3.281945in}{1.424008in}}%
\pgfpathlineto{\pgfqpoint{3.284522in}{1.431326in}}%
\pgfpathlineto{\pgfqpoint{3.294831in}{1.432264in}}%
\pgfpathlineto{\pgfqpoint{3.297409in}{1.425697in}}%
\pgfpathlineto{\pgfqpoint{3.299986in}{1.429825in}}%
\pgfpathlineto{\pgfqpoint{3.302563in}{1.427010in}}%
\pgfpathlineto{\pgfqpoint{3.305140in}{1.433578in}}%
\pgfpathlineto{\pgfqpoint{3.312872in}{1.451403in}}%
\pgfpathlineto{\pgfqpoint{3.318026in}{1.458720in}}%
\pgfpathlineto{\pgfqpoint{3.320603in}{1.437330in}}%
\pgfpathlineto{\pgfqpoint{3.323181in}{1.447650in}}%
\pgfpathlineto{\pgfqpoint{3.330912in}{1.443335in}}%
\pgfpathlineto{\pgfqpoint{3.333490in}{1.452529in}}%
\pgfpathlineto{\pgfqpoint{3.336067in}{1.452153in}}%
\pgfpathlineto{\pgfqpoint{3.338644in}{1.459096in}}%
\pgfpathlineto{\pgfqpoint{3.341221in}{1.423633in}}%
\pgfpathlineto{\pgfqpoint{3.348953in}{1.421194in}}%
\pgfpathlineto{\pgfqpoint{3.351530in}{1.418004in}}%
\pgfpathlineto{\pgfqpoint{3.354107in}{1.420443in}}%
\pgfpathlineto{\pgfqpoint{3.356684in}{1.406934in}}%
\pgfpathlineto{\pgfqpoint{3.359262in}{1.410499in}}%
\pgfpathlineto{\pgfqpoint{3.366993in}{1.412000in}}%
\pgfpathlineto{\pgfqpoint{3.369570in}{1.404870in}}%
\pgfpathlineto{\pgfqpoint{3.372148in}{1.405433in}}%
\pgfpathlineto{\pgfqpoint{3.374725in}{1.398115in}}%
\pgfpathlineto{\pgfqpoint{3.377302in}{1.404494in}}%
\pgfpathlineto{\pgfqpoint{3.385034in}{1.405057in}}%
\pgfpathlineto{\pgfqpoint{3.387611in}{1.403744in}}%
\pgfpathlineto{\pgfqpoint{3.390188in}{1.415940in}}%
\pgfpathlineto{\pgfqpoint{3.392765in}{1.420819in}}%
\pgfpathlineto{\pgfqpoint{3.395343in}{1.410499in}}%
\pgfpathlineto{\pgfqpoint{3.403074in}{1.428887in}}%
\pgfpathlineto{\pgfqpoint{3.405651in}{1.431889in}}%
\pgfpathlineto{\pgfqpoint{3.408229in}{1.437330in}}%
\pgfpathlineto{\pgfqpoint{3.410806in}{1.435266in}}%
\pgfpathlineto{\pgfqpoint{3.413383in}{1.436767in}}%
\pgfpathlineto{\pgfqpoint{3.421115in}{1.436580in}}%
\pgfpathlineto{\pgfqpoint{3.423692in}{1.439769in}}%
\pgfpathlineto{\pgfqpoint{3.426269in}{1.441458in}}%
\pgfpathlineto{\pgfqpoint{3.431424in}{1.420819in}}%
\pgfpathlineto{\pgfqpoint{3.441732in}{1.430763in}}%
\pgfpathlineto{\pgfqpoint{3.444310in}{1.431514in}}%
\pgfpathlineto{\pgfqpoint{3.446887in}{1.428699in}}%
\pgfpathlineto{\pgfqpoint{3.449464in}{1.428324in}}%
\pgfpathlineto{\pgfqpoint{3.457196in}{1.435454in}}%
\pgfpathlineto{\pgfqpoint{3.459773in}{1.428699in}}%
\pgfpathlineto{\pgfqpoint{3.462350in}{1.440332in}}%
\pgfpathlineto{\pgfqpoint{3.464927in}{1.431701in}}%
\pgfpathlineto{\pgfqpoint{3.467505in}{1.427573in}}%
\pgfpathlineto{\pgfqpoint{3.475236in}{1.430388in}}%
\pgfpathlineto{\pgfqpoint{3.477813in}{1.442209in}}%
\pgfpathlineto{\pgfqpoint{3.480391in}{1.434328in}}%
\pgfpathlineto{\pgfqpoint{3.482968in}{1.438456in}}%
\pgfpathlineto{\pgfqpoint{3.485545in}{1.437518in}}%
\pgfpathlineto{\pgfqpoint{3.493277in}{1.427010in}}%
\pgfpathlineto{\pgfqpoint{3.495854in}{1.420631in}}%
\pgfpathlineto{\pgfqpoint{3.498431in}{1.429262in}}%
\pgfpathlineto{\pgfqpoint{3.501008in}{1.412750in}}%
\pgfpathlineto{\pgfqpoint{3.503586in}{1.418192in}}%
\pgfpathlineto{\pgfqpoint{3.511317in}{1.413688in}}%
\pgfpathlineto{\pgfqpoint{3.513894in}{1.424571in}}%
\pgfpathlineto{\pgfqpoint{3.516472in}{1.409560in}}%
\pgfpathlineto{\pgfqpoint{3.519049in}{1.408247in}}%
\pgfpathlineto{\pgfqpoint{3.521626in}{1.418567in}}%
\pgfpathlineto{\pgfqpoint{3.529358in}{1.417629in}}%
\pgfpathlineto{\pgfqpoint{3.531935in}{1.400929in}}%
\pgfpathlineto{\pgfqpoint{3.534512in}{1.420068in}}%
\pgfpathlineto{\pgfqpoint{3.537089in}{1.399241in}}%
\pgfpathlineto{\pgfqpoint{3.539667in}{1.387044in}}%
\pgfpathlineto{\pgfqpoint{3.547398in}{1.384042in}}%
\pgfpathlineto{\pgfqpoint{3.549975in}{1.376912in}}%
\pgfpathlineto{\pgfqpoint{3.552553in}{1.365842in}}%
\pgfpathlineto{\pgfqpoint{3.557707in}{1.391735in}}%
\pgfpathlineto{\pgfqpoint{3.565439in}{1.399991in}}%
\pgfpathlineto{\pgfqpoint{3.568016in}{1.424384in}}%
\pgfpathlineto{\pgfqpoint{3.570593in}{1.413688in}}%
\pgfpathlineto{\pgfqpoint{3.573170in}{1.428699in}}%
\pgfpathlineto{\pgfqpoint{3.575747in}{1.425134in}}%
\pgfpathlineto{\pgfqpoint{3.583479in}{1.424759in}}%
\pgfpathlineto{\pgfqpoint{3.586056in}{1.439582in}}%
\pgfpathlineto{\pgfqpoint{3.588634in}{1.430388in}}%
\pgfpathlineto{\pgfqpoint{3.591211in}{1.528895in}}%
\pgfpathlineto{\pgfqpoint{3.593788in}{1.550286in}}%
\pgfpathlineto{\pgfqpoint{3.601520in}{1.550661in}}%
\pgfpathlineto{\pgfqpoint{3.604097in}{1.557228in}}%
\pgfpathlineto{\pgfqpoint{3.606674in}{1.586499in}}%
\pgfpathlineto{\pgfqpoint{3.609251in}{1.588938in}}%
\pgfpathlineto{\pgfqpoint{3.611828in}{1.599446in}}%
\pgfpathlineto{\pgfqpoint{3.619560in}{1.589501in}}%
\pgfpathlineto{\pgfqpoint{3.622137in}{1.587437in}}%
\pgfpathlineto{\pgfqpoint{3.624715in}{1.606013in}}%
\pgfpathlineto{\pgfqpoint{3.627292in}{1.601510in}}%
\pgfpathlineto{\pgfqpoint{3.629869in}{1.592128in}}%
\pgfpathlineto{\pgfqpoint{3.637601in}{1.596444in}}%
\pgfpathlineto{\pgfqpoint{3.640178in}{1.596068in}}%
\pgfpathlineto{\pgfqpoint{3.642755in}{1.596631in}}%
\pgfpathlineto{\pgfqpoint{3.645332in}{1.604887in}}%
\pgfpathlineto{\pgfqpoint{3.647909in}{1.616333in}}%
\pgfpathlineto{\pgfqpoint{3.655641in}{1.618209in}}%
\pgfpathlineto{\pgfqpoint{3.658218in}{1.630030in}}%
\pgfpathlineto{\pgfqpoint{3.660795in}{1.630030in}}%
\pgfpathlineto{\pgfqpoint{3.665950in}{1.634158in}}%
\pgfpathlineto{\pgfqpoint{3.673682in}{1.633970in}}%
\pgfpathlineto{\pgfqpoint{3.678836in}{1.649544in}}%
\pgfpathlineto{\pgfqpoint{3.681413in}{1.647480in}}%
\pgfpathlineto{\pgfqpoint{3.683990in}{1.657424in}}%
\pgfpathlineto{\pgfqpoint{3.691722in}{1.656299in}}%
\pgfpathlineto{\pgfqpoint{3.694299in}{1.660989in}}%
\pgfpathlineto{\pgfqpoint{3.696876in}{1.650107in}}%
\pgfpathlineto{\pgfqpoint{3.699454in}{1.656486in}}%
\pgfpathlineto{\pgfqpoint{3.702031in}{1.627966in}}%
\pgfpathlineto{\pgfqpoint{3.709763in}{1.627591in}}%
\pgfpathlineto{\pgfqpoint{3.712340in}{1.612768in}}%
\pgfpathlineto{\pgfqpoint{3.717494in}{1.661177in}}%
\pgfpathlineto{\pgfqpoint{3.720071in}{1.649919in}}%
\pgfpathlineto{\pgfqpoint{3.730380in}{1.666056in}}%
\pgfpathlineto{\pgfqpoint{3.732957in}{1.676751in}}%
\pgfpathlineto{\pgfqpoint{3.738112in}{1.672248in}}%
\pgfpathlineto{\pgfqpoint{3.745843in}{1.666994in}}%
\pgfpathlineto{\pgfqpoint{3.748421in}{1.662866in}}%
\pgfpathlineto{\pgfqpoint{3.750998in}{1.652358in}}%
\pgfpathlineto{\pgfqpoint{3.756152in}{1.665117in}}%
\pgfpathlineto{\pgfqpoint{3.766461in}{1.631156in}}%
\pgfpathlineto{\pgfqpoint{3.771616in}{1.662491in}}%
\pgfpathlineto{\pgfqpoint{3.774193in}{1.644853in}}%
\pgfpathlineto{\pgfqpoint{3.781924in}{1.642414in}}%
\pgfpathlineto{\pgfqpoint{3.784502in}{1.645979in}}%
\pgfpathlineto{\pgfqpoint{3.787079in}{1.622337in}}%
\pgfpathlineto{\pgfqpoint{3.789656in}{1.611454in}}%
\pgfpathlineto{\pgfqpoint{3.792233in}{1.619710in}}%
\pgfpathlineto{\pgfqpoint{3.805119in}{1.630218in}}%
\pgfpathlineto{\pgfqpoint{3.807697in}{1.631531in}}%
\pgfpathlineto{\pgfqpoint{3.810274in}{1.634721in}}%
\pgfpathlineto{\pgfqpoint{3.818005in}{1.626840in}}%
\pgfpathlineto{\pgfqpoint{3.823160in}{1.580870in}}%
\pgfpathlineto{\pgfqpoint{3.825737in}{1.588188in}}%
\pgfpathlineto{\pgfqpoint{3.828314in}{1.619522in}}%
\pgfpathlineto{\pgfqpoint{3.836046in}{1.621399in}}%
\pgfpathlineto{\pgfqpoint{3.841200in}{1.664555in}}%
\pgfpathlineto{\pgfqpoint{3.843778in}{1.695702in}}%
\pgfpathlineto{\pgfqpoint{3.846355in}{1.675813in}}%
\pgfpathlineto{\pgfqpoint{3.854086in}{1.666994in}}%
\pgfpathlineto{\pgfqpoint{3.856664in}{1.662866in}}%
\pgfpathlineto{\pgfqpoint{3.859241in}{1.678252in}}%
\pgfpathlineto{\pgfqpoint{3.861818in}{1.700580in}}%
\pgfpathlineto{\pgfqpoint{3.864395in}{1.694764in}}%
\pgfpathlineto{\pgfqpoint{3.874704in}{1.700393in}}%
\pgfpathlineto{\pgfqpoint{3.877281in}{1.692512in}}%
\pgfpathlineto{\pgfqpoint{3.879859in}{1.692324in}}%
\pgfpathlineto{\pgfqpoint{3.882436in}{1.709962in}}%
\pgfpathlineto{\pgfqpoint{3.890167in}{1.710149in}}%
\pgfpathlineto{\pgfqpoint{3.892745in}{1.707710in}}%
\pgfpathlineto{\pgfqpoint{3.895322in}{1.710149in}}%
\pgfpathlineto{\pgfqpoint{3.897899in}{1.713527in}}%
\pgfpathlineto{\pgfqpoint{3.900476in}{1.702457in}}%
\pgfpathlineto{\pgfqpoint{3.908208in}{1.734167in}}%
\pgfpathlineto{\pgfqpoint{3.910785in}{1.720469in}}%
\pgfpathlineto{\pgfqpoint{3.913362in}{1.713527in}}%
\pgfpathlineto{\pgfqpoint{3.915940in}{1.715216in}}%
\pgfpathlineto{\pgfqpoint{3.918517in}{1.693450in}}%
\pgfpathlineto{\pgfqpoint{3.926248in}{1.702832in}}%
\pgfpathlineto{\pgfqpoint{3.928826in}{1.674687in}}%
\pgfpathlineto{\pgfqpoint{3.931403in}{1.672623in}}%
\pgfpathlineto{\pgfqpoint{3.933980in}{1.694576in}}%
\pgfpathlineto{\pgfqpoint{3.936557in}{1.673749in}}%
\pgfpathlineto{\pgfqpoint{3.944289in}{1.691949in}}%
\pgfpathlineto{\pgfqpoint{3.946866in}{1.671497in}}%
\pgfpathlineto{\pgfqpoint{3.949443in}{1.685757in}}%
\pgfpathlineto{\pgfqpoint{3.952020in}{1.683881in}}%
\pgfpathlineto{\pgfqpoint{3.954598in}{1.694764in}}%
\pgfpathlineto{\pgfqpoint{3.962329in}{1.688759in}}%
\pgfpathlineto{\pgfqpoint{3.964907in}{1.689135in}}%
\pgfpathlineto{\pgfqpoint{3.967484in}{1.664179in}}%
\pgfpathlineto{\pgfqpoint{3.970061in}{1.661740in}}%
\pgfpathlineto{\pgfqpoint{3.972638in}{1.660802in}}%
\pgfpathlineto{\pgfqpoint{3.980370in}{1.663053in}}%
\pgfpathlineto{\pgfqpoint{3.985524in}{1.654235in}}%
\pgfpathlineto{\pgfqpoint{3.988101in}{1.656299in}}%
\pgfpathlineto{\pgfqpoint{3.998410in}{1.654610in}}%
\pgfpathlineto{\pgfqpoint{4.003565in}{1.678252in}}%
\pgfpathlineto{\pgfqpoint{4.006142in}{1.677126in}}%
\pgfpathlineto{\pgfqpoint{4.008719in}{1.675250in}}%
\pgfpathlineto{\pgfqpoint{4.016451in}{1.662115in}}%
\pgfpathlineto{\pgfqpoint{4.019028in}{1.660614in}}%
\pgfpathlineto{\pgfqpoint{4.021605in}{1.663429in}}%
\pgfpathlineto{\pgfqpoint{4.024182in}{1.662866in}}%
\pgfpathlineto{\pgfqpoint{4.026760in}{1.642226in}}%
\pgfpathlineto{\pgfqpoint{4.034491in}{1.645979in}}%
\pgfpathlineto{\pgfqpoint{4.037069in}{1.657424in}}%
\pgfpathlineto{\pgfqpoint{4.039646in}{1.705646in}}%
\pgfpathlineto{\pgfqpoint{4.044800in}{1.696077in}}%
\pgfpathlineto{\pgfqpoint{4.052532in}{1.689135in}}%
\pgfpathlineto{\pgfqpoint{4.055109in}{1.683130in}}%
\pgfpathlineto{\pgfqpoint{4.057686in}{1.693450in}}%
\pgfpathlineto{\pgfqpoint{4.060263in}{1.669996in}}%
\pgfpathlineto{\pgfqpoint{4.062841in}{1.664930in}}%
\pgfpathlineto{\pgfqpoint{4.070572in}{1.661552in}}%
\pgfpathlineto{\pgfqpoint{4.073149in}{1.668495in}}%
\pgfpathlineto{\pgfqpoint{4.075727in}{1.663241in}}%
\pgfpathlineto{\pgfqpoint{4.078304in}{1.679753in}}%
\pgfpathlineto{\pgfqpoint{4.080881in}{1.732290in}}%
\pgfpathlineto{\pgfqpoint{4.088613in}{1.724034in}}%
\pgfpathlineto{\pgfqpoint{4.091190in}{1.718030in}}%
\pgfpathlineto{\pgfqpoint{4.093767in}{1.719531in}}%
\pgfpathlineto{\pgfqpoint{4.096344in}{1.744111in}}%
\pgfpathlineto{\pgfqpoint{4.098922in}{1.736231in}}%
\pgfpathlineto{\pgfqpoint{4.109230in}{1.746926in}}%
\pgfpathlineto{\pgfqpoint{4.114385in}{1.732666in}}%
\pgfpathlineto{\pgfqpoint{4.116962in}{1.737169in}}%
\pgfpathlineto{\pgfqpoint{4.127271in}{1.717467in}}%
\pgfpathlineto{\pgfqpoint{4.129848in}{1.734729in}}%
\pgfpathlineto{\pgfqpoint{4.132425in}{1.736043in}}%
\pgfpathlineto{\pgfqpoint{4.135003in}{1.720094in}}%
\pgfpathlineto{\pgfqpoint{4.142734in}{1.727975in}}%
\pgfpathlineto{\pgfqpoint{4.147889in}{1.724973in}}%
\pgfpathlineto{\pgfqpoint{4.150466in}{1.711463in}}%
\pgfpathlineto{\pgfqpoint{4.153043in}{1.714465in}}%
\pgfpathlineto{\pgfqpoint{4.160775in}{1.701894in}}%
\pgfpathlineto{\pgfqpoint{4.163352in}{1.706772in}}%
\pgfpathlineto{\pgfqpoint{4.165929in}{1.737356in}}%
\pgfpathlineto{\pgfqpoint{4.168506in}{1.737544in}}%
\pgfpathlineto{\pgfqpoint{4.171084in}{1.731915in}}%
\pgfpathlineto{\pgfqpoint{4.178815in}{1.718030in}}%
\pgfpathlineto{\pgfqpoint{4.181392in}{1.726098in}}%
\pgfpathlineto{\pgfqpoint{4.183970in}{1.721407in}}%
\pgfpathlineto{\pgfqpoint{4.186547in}{1.734917in}}%
\pgfpathlineto{\pgfqpoint{4.189124in}{1.720282in}}%
\pgfpathlineto{\pgfqpoint{4.196856in}{1.727599in}}%
\pgfpathlineto{\pgfqpoint{4.199433in}{1.733604in}}%
\pgfpathlineto{\pgfqpoint{4.202010in}{1.723284in}}%
\pgfpathlineto{\pgfqpoint{4.204587in}{1.718968in}}%
\pgfpathlineto{\pgfqpoint{4.207165in}{1.721407in}}%
\pgfpathlineto{\pgfqpoint{4.214896in}{1.684444in}}%
\pgfpathlineto{\pgfqpoint{4.217473in}{1.692137in}}%
\pgfpathlineto{\pgfqpoint{4.220051in}{1.704896in}}%
\pgfpathlineto{\pgfqpoint{4.222628in}{1.712026in}}%
\pgfpathlineto{\pgfqpoint{4.232937in}{1.709774in}}%
\pgfpathlineto{\pgfqpoint{4.235514in}{1.703395in}}%
\pgfpathlineto{\pgfqpoint{4.238091in}{1.684631in}}%
\pgfpathlineto{\pgfqpoint{4.240668in}{1.690448in}}%
\pgfpathlineto{\pgfqpoint{4.243246in}{1.715403in}}%
\pgfpathlineto{\pgfqpoint{4.250977in}{1.735292in}}%
\pgfpathlineto{\pgfqpoint{4.253554in}{1.745987in}}%
\pgfpathlineto{\pgfqpoint{4.256132in}{1.744486in}}%
\pgfpathlineto{\pgfqpoint{4.258709in}{1.754431in}}%
\pgfpathlineto{\pgfqpoint{4.261286in}{1.760060in}}%
\pgfpathlineto{\pgfqpoint{4.269018in}{1.793271in}}%
\pgfpathlineto{\pgfqpoint{4.271595in}{1.780887in}}%
\pgfpathlineto{\pgfqpoint{4.274172in}{1.779949in}}%
\pgfpathlineto{\pgfqpoint{4.276749in}{1.776009in}}%
\pgfpathlineto{\pgfqpoint{4.279326in}{1.831548in}}%
\pgfpathlineto{\pgfqpoint{4.287058in}{1.814286in}}%
\pgfpathlineto{\pgfqpoint{4.289635in}{1.830423in}}%
\pgfpathlineto{\pgfqpoint{4.292213in}{1.853502in}}%
\pgfpathlineto{\pgfqpoint{4.294790in}{1.860256in}}%
\pgfpathlineto{\pgfqpoint{4.297367in}{1.841305in}}%
\pgfpathlineto{\pgfqpoint{4.305099in}{1.849186in}}%
\pgfpathlineto{\pgfqpoint{4.307676in}{1.839992in}}%
\pgfpathlineto{\pgfqpoint{4.310253in}{1.834738in}}%
\pgfpathlineto{\pgfqpoint{4.312830in}{1.809220in}}%
\pgfpathlineto{\pgfqpoint{4.315407in}{1.820666in}}%
\pgfpathlineto{\pgfqpoint{4.323139in}{1.823855in}}%
\pgfpathlineto{\pgfqpoint{4.325716in}{1.805280in}}%
\pgfpathlineto{\pgfqpoint{4.328294in}{1.810721in}}%
\pgfpathlineto{\pgfqpoint{4.330871in}{1.817851in}}%
\pgfpathlineto{\pgfqpoint{4.333448in}{1.823105in}}%
\pgfpathlineto{\pgfqpoint{4.343757in}{1.827608in}}%
\pgfpathlineto{\pgfqpoint{4.346334in}{1.826482in}}%
\pgfpathlineto{\pgfqpoint{4.348911in}{1.818226in}}%
\pgfpathlineto{\pgfqpoint{4.351488in}{1.767941in}}%
\pgfpathlineto{\pgfqpoint{4.359220in}{1.716341in}}%
\pgfpathlineto{\pgfqpoint{4.361797in}{1.690823in}}%
\pgfpathlineto{\pgfqpoint{4.364374in}{1.758747in}}%
\pgfpathlineto{\pgfqpoint{4.366952in}{1.789894in}}%
\pgfpathlineto{\pgfqpoint{4.369529in}{1.791020in}}%
\pgfpathlineto{\pgfqpoint{4.377261in}{1.769817in}}%
\pgfpathlineto{\pgfqpoint{4.379838in}{1.727224in}}%
\pgfpathlineto{\pgfqpoint{4.384992in}{1.753493in}}%
\pgfpathlineto{\pgfqpoint{4.387569in}{1.730789in}}%
\pgfpathlineto{\pgfqpoint{4.397878in}{1.755932in}}%
\pgfpathlineto{\pgfqpoint{4.400455in}{1.738857in}}%
\pgfpathlineto{\pgfqpoint{4.405610in}{1.760060in}}%
\pgfpathlineto{\pgfqpoint{4.413342in}{1.746363in}}%
\pgfpathlineto{\pgfqpoint{4.418496in}{1.764188in}}%
\pgfpathlineto{\pgfqpoint{4.421073in}{1.763625in}}%
\pgfpathlineto{\pgfqpoint{4.423650in}{1.742422in}}%
\pgfpathlineto{\pgfqpoint{4.431382in}{1.760435in}}%
\pgfpathlineto{\pgfqpoint{4.433959in}{1.750678in}}%
\pgfpathlineto{\pgfqpoint{4.436536in}{1.763437in}}%
\pgfpathlineto{\pgfqpoint{4.439114in}{1.750491in}}%
\pgfpathlineto{\pgfqpoint{4.441691in}{1.758747in}}%
\pgfpathlineto{\pgfqpoint{4.449422in}{1.695139in}}%
\pgfpathlineto{\pgfqpoint{4.454577in}{1.739983in}}%
\pgfpathlineto{\pgfqpoint{4.457154in}{1.745987in}}%
\pgfpathlineto{\pgfqpoint{4.459731in}{1.758371in}}%
\pgfpathlineto{\pgfqpoint{4.467463in}{1.786892in}}%
\pgfpathlineto{\pgfqpoint{4.470040in}{1.784077in}}%
\pgfpathlineto{\pgfqpoint{4.472617in}{1.804904in}}%
\pgfpathlineto{\pgfqpoint{4.475195in}{1.817288in}}%
\pgfpathlineto{\pgfqpoint{4.477772in}{1.818789in}}%
\pgfpathlineto{\pgfqpoint{4.485503in}{1.837177in}}%
\pgfpathlineto{\pgfqpoint{4.488081in}{1.837365in}}%
\pgfpathlineto{\pgfqpoint{4.490658in}{1.822730in}}%
\pgfpathlineto{\pgfqpoint{4.493235in}{1.843557in}}%
\pgfpathlineto{\pgfqpoint{4.495812in}{1.855566in}}%
\pgfpathlineto{\pgfqpoint{4.503544in}{1.873578in}}%
\pgfpathlineto{\pgfqpoint{4.508698in}{1.845809in}}%
\pgfpathlineto{\pgfqpoint{4.511276in}{1.863258in}}%
\pgfpathlineto{\pgfqpoint{4.513853in}{1.875079in}}%
\pgfpathlineto{\pgfqpoint{4.521584in}{1.895344in}}%
\pgfpathlineto{\pgfqpoint{4.524162in}{1.883335in}}%
\pgfpathlineto{\pgfqpoint{4.526739in}{1.907915in}}%
\pgfpathlineto{\pgfqpoint{4.529316in}{1.901348in}}%
\pgfpathlineto{\pgfqpoint{4.531893in}{1.884461in}}%
\pgfpathlineto{\pgfqpoint{4.539625in}{1.841305in}}%
\pgfpathlineto{\pgfqpoint{4.542202in}{1.890278in}}%
\pgfpathlineto{\pgfqpoint{4.544779in}{1.898534in}}%
\pgfpathlineto{\pgfqpoint{4.547357in}{1.915045in}}%
\pgfpathlineto{\pgfqpoint{4.549934in}{1.905851in}}%
\pgfpathlineto{\pgfqpoint{4.557665in}{1.892717in}}%
\pgfpathlineto{\pgfqpoint{4.560243in}{1.921613in}}%
\pgfpathlineto{\pgfqpoint{4.562820in}{1.916171in}}%
\pgfpathlineto{\pgfqpoint{4.565397in}{1.900222in}}%
\pgfpathlineto{\pgfqpoint{4.567974in}{1.896657in}}%
\pgfpathlineto{\pgfqpoint{4.575706in}{1.911105in}}%
\pgfpathlineto{\pgfqpoint{4.578283in}{1.909792in}}%
\pgfpathlineto{\pgfqpoint{4.580860in}{1.939625in}}%
\pgfpathlineto{\pgfqpoint{4.583438in}{1.933996in}}%
\pgfpathlineto{\pgfqpoint{4.586015in}{1.934559in}}%
\pgfpathlineto{\pgfqpoint{4.593746in}{1.933246in}}%
\pgfpathlineto{\pgfqpoint{4.596324in}{1.927429in}}%
\pgfpathlineto{\pgfqpoint{4.598901in}{1.923114in}}%
\pgfpathlineto{\pgfqpoint{4.604055in}{1.928180in}}%
\pgfpathlineto{\pgfqpoint{4.611787in}{1.913169in}}%
\pgfpathlineto{\pgfqpoint{4.614364in}{1.929493in}}%
\pgfpathlineto{\pgfqpoint{4.619519in}{1.899847in}}%
\pgfpathlineto{\pgfqpoint{4.622096in}{1.938500in}}%
\pgfpathlineto{\pgfqpoint{4.629827in}{1.922926in}}%
\pgfpathlineto{\pgfqpoint{4.632405in}{1.914107in}}%
\pgfpathlineto{\pgfqpoint{4.634982in}{1.892529in}}%
\pgfpathlineto{\pgfqpoint{4.637559in}{1.896845in}}%
\pgfpathlineto{\pgfqpoint{4.640136in}{1.860069in}}%
\pgfpathlineto{\pgfqpoint{4.647868in}{1.873954in}}%
\pgfpathlineto{\pgfqpoint{4.650445in}{1.906039in}}%
\pgfpathlineto{\pgfqpoint{4.653022in}{1.928555in}}%
\pgfpathlineto{\pgfqpoint{4.655599in}{1.907165in}}%
\pgfpathlineto{\pgfqpoint{4.658177in}{1.864009in}}%
\pgfpathlineto{\pgfqpoint{4.668486in}{1.883335in}}%
\pgfpathlineto{\pgfqpoint{4.671063in}{1.904913in}}%
\pgfpathlineto{\pgfqpoint{4.673640in}{1.899472in}}%
\pgfpathlineto{\pgfqpoint{4.683949in}{1.904350in}}%
\pgfpathlineto{\pgfqpoint{4.686526in}{1.916734in}}%
\pgfpathlineto{\pgfqpoint{4.691680in}{1.886337in}}%
\pgfpathlineto{\pgfqpoint{4.701989in}{1.852563in}}%
\pgfpathlineto{\pgfqpoint{4.704567in}{1.863071in}}%
\pgfpathlineto{\pgfqpoint{4.707144in}{1.844683in}}%
\pgfpathlineto{\pgfqpoint{4.709721in}{1.817664in}}%
\pgfpathlineto{\pgfqpoint{4.712298in}{1.801152in}}%
\pgfpathlineto{\pgfqpoint{4.720030in}{1.820103in}}%
\pgfpathlineto{\pgfqpoint{4.722607in}{1.835489in}}%
\pgfpathlineto{\pgfqpoint{4.725184in}{1.805092in}}%
\pgfpathlineto{\pgfqpoint{4.727761in}{1.817851in}}%
\pgfpathlineto{\pgfqpoint{4.730339in}{1.781826in}}%
\pgfpathlineto{\pgfqpoint{4.740648in}{1.773945in}}%
\pgfpathlineto{\pgfqpoint{4.743225in}{1.760811in}}%
\pgfpathlineto{\pgfqpoint{4.745802in}{1.776947in}}%
\pgfpathlineto{\pgfqpoint{4.748379in}{1.797774in}}%
\pgfpathlineto{\pgfqpoint{4.756111in}{1.780324in}}%
\pgfpathlineto{\pgfqpoint{4.758688in}{1.782764in}}%
\pgfpathlineto{\pgfqpoint{4.761265in}{1.765126in}}%
\pgfpathlineto{\pgfqpoint{4.763842in}{1.736231in}}%
\pgfpathlineto{\pgfqpoint{4.766420in}{1.830423in}}%
\pgfpathlineto{\pgfqpoint{4.774151in}{1.828546in}}%
\pgfpathlineto{\pgfqpoint{4.776728in}{1.810533in}}%
\pgfpathlineto{\pgfqpoint{4.779306in}{1.828546in}}%
\pgfpathlineto{\pgfqpoint{4.781883in}{1.815787in}}%
\pgfpathlineto{\pgfqpoint{4.784460in}{1.776572in}}%
\pgfpathlineto{\pgfqpoint{4.792192in}{1.707710in}}%
\pgfpathlineto{\pgfqpoint{4.794769in}{1.718030in}}%
\pgfpathlineto{\pgfqpoint{4.797346in}{1.753868in}}%
\pgfpathlineto{\pgfqpoint{4.799923in}{1.723096in}}%
\pgfpathlineto{\pgfqpoint{4.802501in}{1.758747in}}%
\pgfpathlineto{\pgfqpoint{4.812809in}{1.771318in}}%
\pgfpathlineto{\pgfqpoint{4.815387in}{1.790832in}}%
\pgfpathlineto{\pgfqpoint{4.817964in}{1.776384in}}%
\pgfpathlineto{\pgfqpoint{4.820541in}{1.781638in}}%
\pgfpathlineto{\pgfqpoint{4.828273in}{1.809783in}}%
\pgfpathlineto{\pgfqpoint{4.830850in}{1.793084in}}%
\pgfpathlineto{\pgfqpoint{4.833427in}{1.787642in}}%
\pgfpathlineto{\pgfqpoint{4.836004in}{1.813536in}}%
\pgfpathlineto{\pgfqpoint{4.838582in}{1.803779in}}%
\pgfpathlineto{\pgfqpoint{4.846313in}{1.797399in}}%
\pgfpathlineto{\pgfqpoint{4.848890in}{1.838678in}}%
\pgfpathlineto{\pgfqpoint{4.854045in}{1.825357in}}%
\pgfpathlineto{\pgfqpoint{4.856622in}{1.825169in}}%
\pgfpathlineto{\pgfqpoint{4.864354in}{1.789143in}}%
\pgfpathlineto{\pgfqpoint{4.866931in}{1.764938in}}%
\pgfpathlineto{\pgfqpoint{4.869508in}{1.766064in}}%
\pgfpathlineto{\pgfqpoint{4.872085in}{1.757621in}}%
\pgfpathlineto{\pgfqpoint{4.874663in}{1.783514in}}%
\pgfpathlineto{\pgfqpoint{4.882394in}{1.780887in}}%
\pgfpathlineto{\pgfqpoint{4.887549in}{1.797024in}}%
\pgfpathlineto{\pgfqpoint{4.892703in}{1.822917in}}%
\pgfpathlineto{\pgfqpoint{4.900435in}{1.822730in}}%
\pgfpathlineto{\pgfqpoint{4.903012in}{1.807907in}}%
\pgfpathlineto{\pgfqpoint{4.905589in}{1.825357in}}%
\pgfpathlineto{\pgfqpoint{4.908166in}{1.829484in}}%
\pgfpathlineto{\pgfqpoint{4.918475in}{1.828546in}}%
\pgfpathlineto{\pgfqpoint{4.923630in}{1.878082in}}%
\pgfpathlineto{\pgfqpoint{4.926207in}{1.872453in}}%
\pgfpathlineto{\pgfqpoint{4.928784in}{1.892905in}}%
\pgfpathlineto{\pgfqpoint{4.936516in}{1.897220in}}%
\pgfpathlineto{\pgfqpoint{4.939093in}{1.881459in}}%
\pgfpathlineto{\pgfqpoint{4.941670in}{1.904163in}}%
\pgfpathlineto{\pgfqpoint{4.944247in}{1.892717in}}%
\pgfpathlineto{\pgfqpoint{4.946825in}{1.900973in}}%
\pgfpathlineto{\pgfqpoint{4.954556in}{1.897033in}}%
\pgfpathlineto{\pgfqpoint{4.957133in}{1.909979in}}%
\pgfpathlineto{\pgfqpoint{4.959711in}{1.931745in}}%
\pgfpathlineto{\pgfqpoint{4.962288in}{1.943190in}}%
\pgfpathlineto{\pgfqpoint{4.964865in}{1.938500in}}%
\pgfpathlineto{\pgfqpoint{4.972597in}{1.963642in}}%
\pgfpathlineto{\pgfqpoint{4.975174in}{1.950883in}}%
\pgfpathlineto{\pgfqpoint{4.977751in}{1.958201in}}%
\pgfpathlineto{\pgfqpoint{4.980328in}{1.951634in}}%
\pgfpathlineto{\pgfqpoint{4.982905in}{1.920674in}}%
\pgfpathlineto{\pgfqpoint{4.990637in}{1.903037in}}%
\pgfpathlineto{\pgfqpoint{4.993214in}{1.910167in}}%
\pgfpathlineto{\pgfqpoint{4.995792in}{1.914295in}}%
\pgfpathlineto{\pgfqpoint{4.998369in}{1.894593in}}%
\pgfpathlineto{\pgfqpoint{5.000946in}{1.886525in}}%
\pgfpathlineto{\pgfqpoint{5.008678in}{1.908853in}}%
\pgfpathlineto{\pgfqpoint{5.011255in}{1.885587in}}%
\pgfpathlineto{\pgfqpoint{5.013832in}{1.883335in}}%
\pgfpathlineto{\pgfqpoint{5.016409in}{1.888589in}}%
\pgfpathlineto{\pgfqpoint{5.018986in}{1.895156in}}%
\pgfpathlineto{\pgfqpoint{5.026718in}{1.904350in}}%
\pgfpathlineto{\pgfqpoint{5.029295in}{1.922363in}}%
\pgfpathlineto{\pgfqpoint{5.031873in}{1.890090in}}%
\pgfpathlineto{\pgfqpoint{5.034450in}{1.900973in}}%
\pgfpathlineto{\pgfqpoint{5.037027in}{1.881459in}}%
\pgfpathlineto{\pgfqpoint{5.044759in}{1.899284in}}%
\pgfpathlineto{\pgfqpoint{5.047336in}{1.880708in}}%
\pgfpathlineto{\pgfqpoint{5.049913in}{1.892529in}}%
\pgfpathlineto{\pgfqpoint{5.052490in}{1.882397in}}%
\pgfpathlineto{\pgfqpoint{5.055067in}{1.896845in}}%
\pgfpathlineto{\pgfqpoint{5.062799in}{1.888401in}}%
\pgfpathlineto{\pgfqpoint{5.065376in}{1.928367in}}%
\pgfpathlineto{\pgfqpoint{5.067953in}{1.922363in}}%
\pgfpathlineto{\pgfqpoint{5.070531in}{1.921237in}}%
\pgfpathlineto{\pgfqpoint{5.073108in}{1.933433in}}%
\pgfpathlineto{\pgfqpoint{5.083417in}{1.920111in}}%
\pgfpathlineto{\pgfqpoint{5.085994in}{1.925365in}}%
\pgfpathlineto{\pgfqpoint{5.088571in}{1.938687in}}%
\pgfpathlineto{\pgfqpoint{5.091148in}{1.938500in}}%
\pgfpathlineto{\pgfqpoint{5.098880in}{1.949570in}}%
\pgfpathlineto{\pgfqpoint{5.101457in}{1.950696in}}%
\pgfpathlineto{\pgfqpoint{5.104034in}{1.970022in}}%
\pgfpathlineto{\pgfqpoint{5.106612in}{1.962892in}}%
\pgfpathlineto{\pgfqpoint{5.109189in}{1.943003in}}%
\pgfpathlineto{\pgfqpoint{5.116921in}{1.910355in}}%
\pgfpathlineto{\pgfqpoint{5.119498in}{1.913357in}}%
\pgfpathlineto{\pgfqpoint{5.122075in}{1.906039in}}%
\pgfpathlineto{\pgfqpoint{5.124652in}{1.909416in}}%
\pgfpathlineto{\pgfqpoint{5.127229in}{1.884273in}}%
\pgfpathlineto{\pgfqpoint{5.134961in}{1.890841in}}%
\pgfpathlineto{\pgfqpoint{5.137538in}{1.890653in}}%
\pgfpathlineto{\pgfqpoint{5.140115in}{1.875830in}}%
\pgfpathlineto{\pgfqpoint{5.142693in}{1.907165in}}%
\pgfpathlineto{\pgfqpoint{5.145270in}{1.848623in}}%
\pgfpathlineto{\pgfqpoint{5.153001in}{1.817288in}}%
\pgfpathlineto{\pgfqpoint{5.158156in}{1.879770in}}%
\pgfpathlineto{\pgfqpoint{5.160733in}{1.832487in}}%
\pgfpathlineto{\pgfqpoint{5.163310in}{1.838303in}}%
\pgfpathlineto{\pgfqpoint{5.173619in}{1.841681in}}%
\pgfpathlineto{\pgfqpoint{5.176196in}{1.830423in}}%
\pgfpathlineto{\pgfqpoint{5.178774in}{1.838866in}}%
\pgfpathlineto{\pgfqpoint{5.181351in}{1.873954in}}%
\pgfpathlineto{\pgfqpoint{5.189082in}{1.875642in}}%
\pgfpathlineto{\pgfqpoint{5.191660in}{1.893280in}}%
\pgfpathlineto{\pgfqpoint{5.194237in}{1.892905in}}%
\pgfpathlineto{\pgfqpoint{5.196814in}{1.905476in}}%
\pgfpathlineto{\pgfqpoint{5.199391in}{1.908478in}}%
\pgfpathlineto{\pgfqpoint{5.207123in}{1.908666in}}%
\pgfpathlineto{\pgfqpoint{5.209700in}{1.916171in}}%
\pgfpathlineto{\pgfqpoint{5.212277in}{1.927992in}}%
\pgfpathlineto{\pgfqpoint{5.214855in}{1.917485in}}%
\pgfpathlineto{\pgfqpoint{5.217432in}{1.937937in}}%
\pgfpathlineto{\pgfqpoint{5.225163in}{1.920111in}}%
\pgfpathlineto{\pgfqpoint{5.227741in}{1.911856in}}%
\pgfpathlineto{\pgfqpoint{5.230318in}{1.912606in}}%
\pgfpathlineto{\pgfqpoint{5.232895in}{1.924802in}}%
\pgfpathlineto{\pgfqpoint{5.235472in}{1.903787in}}%
\pgfpathlineto{\pgfqpoint{5.243204in}{1.907728in}}%
\pgfpathlineto{\pgfqpoint{5.245781in}{1.908103in}}%
\pgfpathlineto{\pgfqpoint{5.248358in}{1.915984in}}%
\pgfpathlineto{\pgfqpoint{5.253513in}{1.942252in}}%
\pgfpathlineto{\pgfqpoint{5.261244in}{1.938687in}}%
\pgfpathlineto{\pgfqpoint{5.263822in}{1.940751in}}%
\pgfpathlineto{\pgfqpoint{5.266399in}{1.933809in}}%
\pgfpathlineto{\pgfqpoint{5.268976in}{1.941877in}}%
\pgfpathlineto{\pgfqpoint{5.271553in}{1.940001in}}%
\pgfpathlineto{\pgfqpoint{5.279285in}{1.956325in}}%
\pgfpathlineto{\pgfqpoint{5.281862in}{1.954636in}}%
\pgfpathlineto{\pgfqpoint{5.284439in}{1.957638in}}%
\pgfpathlineto{\pgfqpoint{5.287017in}{1.947131in}}%
\pgfpathlineto{\pgfqpoint{5.297325in}{1.961766in}}%
\pgfpathlineto{\pgfqpoint{5.299903in}{1.956888in}}%
\pgfpathlineto{\pgfqpoint{5.302480in}{1.947506in}}%
\pgfpathlineto{\pgfqpoint{5.305057in}{1.948069in}}%
\pgfpathlineto{\pgfqpoint{5.307634in}{1.952760in}}%
\pgfpathlineto{\pgfqpoint{5.315366in}{1.958201in}}%
\pgfpathlineto{\pgfqpoint{5.317943in}{1.963830in}}%
\pgfpathlineto{\pgfqpoint{5.320520in}{1.958764in}}%
\pgfpathlineto{\pgfqpoint{5.323098in}{1.966082in}}%
\pgfpathlineto{\pgfqpoint{5.325675in}{1.979028in}}%
\pgfpathlineto{\pgfqpoint{5.335984in}{1.988785in}}%
\pgfpathlineto{\pgfqpoint{5.338561in}{2.001920in}}%
\pgfpathlineto{\pgfqpoint{5.341138in}{1.996854in}}%
\pgfpathlineto{\pgfqpoint{5.343715in}{1.966645in}}%
\pgfpathlineto{\pgfqpoint{5.351447in}{1.996854in}}%
\pgfpathlineto{\pgfqpoint{5.354024in}{1.976964in}}%
\pgfpathlineto{\pgfqpoint{5.356601in}{1.969647in}}%
\pgfpathlineto{\pgfqpoint{5.359178in}{1.979216in}}%
\pgfpathlineto{\pgfqpoint{5.361756in}{1.980342in}}%
\pgfpathlineto{\pgfqpoint{5.369487in}{1.988410in}}%
\pgfpathlineto{\pgfqpoint{5.372065in}{1.987847in}}%
\pgfpathlineto{\pgfqpoint{5.374642in}{2.001544in}}%
\pgfpathlineto{\pgfqpoint{5.377219in}{2.003984in}}%
\pgfpathlineto{\pgfqpoint{5.379796in}{1.988973in}}%
\pgfpathlineto{\pgfqpoint{5.387528in}{1.975463in}}%
\pgfpathlineto{\pgfqpoint{5.390105in}{1.982031in}}%
\pgfpathlineto{\pgfqpoint{5.392682in}{1.996854in}}%
\pgfpathlineto{\pgfqpoint{5.395259in}{1.977527in}}%
\pgfpathlineto{\pgfqpoint{5.397837in}{1.991975in}}%
\pgfpathlineto{\pgfqpoint{5.405568in}{1.994790in}}%
\pgfpathlineto{\pgfqpoint{5.408146in}{1.992538in}}%
\pgfpathlineto{\pgfqpoint{5.410723in}{2.003796in}}%
\pgfpathlineto{\pgfqpoint{5.413300in}{2.003984in}}%
\pgfpathlineto{\pgfqpoint{5.415877in}{1.995165in}}%
\pgfpathlineto{\pgfqpoint{5.423609in}{1.999668in}}%
\pgfpathlineto{\pgfqpoint{5.426186in}{1.979779in}}%
\pgfpathlineto{\pgfqpoint{5.428763in}{1.983719in}}%
\pgfpathlineto{\pgfqpoint{5.431340in}{1.976589in}}%
\pgfpathlineto{\pgfqpoint{5.433918in}{1.987284in}}%
\pgfpathlineto{\pgfqpoint{5.441649in}{1.981843in}}%
\pgfpathlineto{\pgfqpoint{5.444227in}{1.971335in}}%
\pgfpathlineto{\pgfqpoint{5.446804in}{1.994039in}}%
\pgfpathlineto{\pgfqpoint{5.449381in}{1.988222in}}%
\pgfpathlineto{\pgfqpoint{5.451958in}{1.985408in}}%
\pgfpathlineto{\pgfqpoint{5.459690in}{2.000606in}}%
\pgfpathlineto{\pgfqpoint{5.462267in}{1.979591in}}%
\pgfpathlineto{\pgfqpoint{5.464844in}{1.974525in}}%
\pgfpathlineto{\pgfqpoint{5.467421in}{1.977527in}}%
\pgfpathlineto{\pgfqpoint{5.469999in}{1.983156in}}%
\pgfpathlineto{\pgfqpoint{5.477730in}{1.988410in}}%
\pgfpathlineto{\pgfqpoint{5.482885in}{1.952947in}}%
\pgfpathlineto{\pgfqpoint{5.485462in}{1.954261in}}%
\pgfpathlineto{\pgfqpoint{5.488039in}{1.948819in}}%
\pgfpathlineto{\pgfqpoint{5.495771in}{1.988410in}}%
\pgfpathlineto{\pgfqpoint{5.500925in}{2.001920in}}%
\pgfpathlineto{\pgfqpoint{5.503502in}{1.976589in}}%
\pgfpathlineto{\pgfqpoint{5.506080in}{1.976777in}}%
\pgfpathlineto{\pgfqpoint{5.513811in}{1.912418in}}%
\pgfpathlineto{\pgfqpoint{5.516388in}{1.915984in}}%
\pgfpathlineto{\pgfqpoint{5.518966in}{1.946755in}}%
\pgfpathlineto{\pgfqpoint{5.521543in}{1.965144in}}%
\pgfpathlineto{\pgfqpoint{5.524120in}{1.960453in}}%
\pgfpathlineto{\pgfqpoint{5.531852in}{1.976402in}}%
\pgfpathlineto{\pgfqpoint{5.534429in}{1.943941in}}%
\pgfpathlineto{\pgfqpoint{5.537006in}{1.937374in}}%
\pgfpathlineto{\pgfqpoint{5.542161in}{1.947694in}}%
\pgfpathlineto{\pgfqpoint{5.549892in}{1.928367in}}%
\pgfpathlineto{\pgfqpoint{5.552469in}{1.929681in}}%
\pgfpathlineto{\pgfqpoint{5.557624in}{1.861007in}}%
\pgfpathlineto{\pgfqpoint{5.560201in}{1.866448in}}%
\pgfpathlineto{\pgfqpoint{5.567933in}{1.895719in}}%
\pgfpathlineto{\pgfqpoint{5.570510in}{1.891966in}}%
\pgfpathlineto{\pgfqpoint{5.573087in}{1.932683in}}%
\pgfpathlineto{\pgfqpoint{5.575664in}{1.932120in}}%
\pgfpathlineto{\pgfqpoint{5.578242in}{1.929493in}}%
\pgfpathlineto{\pgfqpoint{5.585973in}{1.917672in}}%
\pgfpathlineto{\pgfqpoint{5.588550in}{1.930619in}}%
\pgfpathlineto{\pgfqpoint{5.591128in}{1.929306in}}%
\pgfpathlineto{\pgfqpoint{5.593705in}{1.936060in}}%
\pgfpathlineto{\pgfqpoint{5.596282in}{1.914858in}}%
\pgfpathlineto{\pgfqpoint{5.604014in}{1.910355in}}%
\pgfpathlineto{\pgfqpoint{5.606591in}{1.915045in}}%
\pgfpathlineto{\pgfqpoint{5.611745in}{1.906602in}}%
\pgfpathlineto{\pgfqpoint{5.614323in}{1.911105in}}%
\pgfpathlineto{\pgfqpoint{5.624631in}{1.914858in}}%
\pgfpathlineto{\pgfqpoint{5.627209in}{1.913920in}}%
\pgfpathlineto{\pgfqpoint{5.629786in}{1.914482in}}%
\pgfpathlineto{\pgfqpoint{5.632363in}{1.908853in}}%
\pgfpathlineto{\pgfqpoint{5.642672in}{1.936060in}}%
\pgfpathlineto{\pgfqpoint{5.645249in}{1.948069in}}%
\pgfpathlineto{\pgfqpoint{5.650403in}{1.986158in}}%
\pgfpathlineto{\pgfqpoint{5.658135in}{1.977527in}}%
\pgfpathlineto{\pgfqpoint{5.660712in}{1.969459in}}%
\pgfpathlineto{\pgfqpoint{5.663290in}{1.978466in}}%
\pgfpathlineto{\pgfqpoint{5.665867in}{1.970585in}}%
\pgfpathlineto{\pgfqpoint{5.668444in}{1.966832in}}%
\pgfpathlineto{\pgfqpoint{5.678753in}{1.968521in}}%
\pgfpathlineto{\pgfqpoint{5.681330in}{1.974713in}}%
\pgfpathlineto{\pgfqpoint{5.686484in}{1.979216in}}%
\pgfpathlineto{\pgfqpoint{5.694216in}{1.985033in}}%
\pgfpathlineto{\pgfqpoint{5.696793in}{2.004922in}}%
\pgfpathlineto{\pgfqpoint{5.699371in}{2.017306in}}%
\pgfpathlineto{\pgfqpoint{5.701948in}{2.005109in}}%
\pgfpathlineto{\pgfqpoint{5.704525in}{2.014866in}}%
\pgfpathlineto{\pgfqpoint{5.712257in}{2.013553in}}%
\pgfpathlineto{\pgfqpoint{5.714834in}{1.995353in}}%
\pgfpathlineto{\pgfqpoint{5.717411in}{1.990286in}}%
\pgfpathlineto{\pgfqpoint{5.719988in}{1.987660in}}%
\pgfpathlineto{\pgfqpoint{5.722565in}{2.057459in}}%
\pgfpathlineto{\pgfqpoint{5.732874in}{2.052018in}}%
\pgfpathlineto{\pgfqpoint{5.735452in}{2.039259in}}%
\pgfpathlineto{\pgfqpoint{5.738029in}{2.048265in}}%
\pgfpathlineto{\pgfqpoint{5.740606in}{2.054082in}}%
\pgfpathlineto{\pgfqpoint{5.748338in}{2.064214in}}%
\pgfpathlineto{\pgfqpoint{5.750915in}{2.071719in}}%
\pgfpathlineto{\pgfqpoint{5.753492in}{2.087481in}}%
\pgfpathlineto{\pgfqpoint{5.756069in}{2.085041in}}%
\pgfpathlineto{\pgfqpoint{5.758646in}{2.085980in}}%
\pgfpathlineto{\pgfqpoint{5.768955in}{2.094423in}}%
\pgfpathlineto{\pgfqpoint{5.771532in}{2.092359in}}%
\pgfpathlineto{\pgfqpoint{5.774110in}{2.099302in}}%
\pgfpathlineto{\pgfqpoint{5.776687in}{2.103992in}}%
\pgfpathlineto{\pgfqpoint{5.784419in}{2.095924in}}%
\pgfpathlineto{\pgfqpoint{5.786996in}{2.094798in}}%
\pgfpathlineto{\pgfqpoint{5.789573in}{2.114312in}}%
\pgfpathlineto{\pgfqpoint{5.792150in}{2.105869in}}%
\pgfpathlineto{\pgfqpoint{5.794727in}{2.110560in}}%
\pgfpathlineto{\pgfqpoint{5.802459in}{2.113374in}}%
\pgfpathlineto{\pgfqpoint{5.805036in}{2.115626in}}%
\pgfpathlineto{\pgfqpoint{5.807613in}{2.113749in}}%
\pgfpathlineto{\pgfqpoint{5.810191in}{2.116564in}}%
\pgfpathlineto{\pgfqpoint{5.812768in}{2.128009in}}%
\pgfpathlineto{\pgfqpoint{5.820500in}{2.134952in}}%
\pgfpathlineto{\pgfqpoint{5.823077in}{2.124632in}}%
\pgfpathlineto{\pgfqpoint{5.825654in}{2.131387in}}%
\pgfpathlineto{\pgfqpoint{5.828231in}{2.133638in}}%
\pgfpathlineto{\pgfqpoint{5.830808in}{2.137391in}}%
\pgfpathlineto{\pgfqpoint{5.838540in}{2.117877in}}%
\pgfpathlineto{\pgfqpoint{5.841117in}{2.097050in}}%
\pgfpathlineto{\pgfqpoint{5.843694in}{2.105681in}}%
\pgfpathlineto{\pgfqpoint{5.848849in}{2.118065in}}%
\pgfpathlineto{\pgfqpoint{5.856580in}{2.113562in}}%
\pgfpathlineto{\pgfqpoint{5.859158in}{2.116751in}}%
\pgfpathlineto{\pgfqpoint{5.861735in}{2.117877in}}%
\pgfpathlineto{\pgfqpoint{5.866889in}{2.112061in}}%
\pgfpathlineto{\pgfqpoint{5.874621in}{2.121067in}}%
\pgfpathlineto{\pgfqpoint{5.877198in}{2.110372in}}%
\pgfpathlineto{\pgfqpoint{5.879775in}{2.115063in}}%
\pgfpathlineto{\pgfqpoint{5.882353in}{2.116189in}}%
\pgfpathlineto{\pgfqpoint{5.884930in}{2.109621in}}%
\pgfpathlineto{\pgfqpoint{5.892661in}{2.111873in}}%
\pgfpathlineto{\pgfqpoint{5.895239in}{2.111685in}}%
\pgfpathlineto{\pgfqpoint{5.897816in}{2.108496in}}%
\pgfpathlineto{\pgfqpoint{5.900393in}{2.112061in}}%
\pgfpathlineto{\pgfqpoint{5.910702in}{2.129511in}}%
\pgfpathlineto{\pgfqpoint{5.913279in}{2.128009in}}%
\pgfpathlineto{\pgfqpoint{5.915856in}{2.128385in}}%
\pgfpathlineto{\pgfqpoint{5.918434in}{2.154278in}}%
\pgfpathlineto{\pgfqpoint{5.921011in}{2.154278in}}%
\pgfpathlineto{\pgfqpoint{5.931320in}{2.171916in}}%
\pgfpathlineto{\pgfqpoint{5.939051in}{2.155592in}}%
\pgfpathlineto{\pgfqpoint{5.946783in}{2.156155in}}%
\pgfpathlineto{\pgfqpoint{5.949360in}{2.179984in}}%
\pgfpathlineto{\pgfqpoint{5.951937in}{2.178108in}}%
\pgfpathlineto{\pgfqpoint{5.954515in}{2.181297in}}%
\pgfpathlineto{\pgfqpoint{5.957092in}{2.171540in}}%
\pgfpathlineto{\pgfqpoint{5.964823in}{2.168351in}}%
\pgfpathlineto{\pgfqpoint{5.967401in}{2.169852in}}%
\pgfpathlineto{\pgfqpoint{5.969978in}{2.174543in}}%
\pgfpathlineto{\pgfqpoint{5.972555in}{2.171728in}}%
\pgfpathlineto{\pgfqpoint{5.975132in}{2.183361in}}%
\pgfpathlineto{\pgfqpoint{5.982864in}{2.192743in}}%
\pgfpathlineto{\pgfqpoint{5.985441in}{2.191242in}}%
\pgfpathlineto{\pgfqpoint{5.988018in}{2.168538in}}%
\pgfpathlineto{\pgfqpoint{5.990596in}{2.167600in}}%
\pgfpathlineto{\pgfqpoint{5.993173in}{2.181860in}}%
\pgfpathlineto{\pgfqpoint{6.000904in}{2.197246in}}%
\pgfpathlineto{\pgfqpoint{6.003482in}{2.207378in}}%
\pgfpathlineto{\pgfqpoint{6.006059in}{2.225016in}}%
\pgfpathlineto{\pgfqpoint{6.008636in}{2.229144in}}%
\pgfpathlineto{\pgfqpoint{6.011213in}{2.222389in}}%
\pgfpathlineto{\pgfqpoint{6.021522in}{2.223327in}}%
\pgfpathlineto{\pgfqpoint{6.024099in}{2.232709in}}%
\pgfpathlineto{\pgfqpoint{6.026677in}{2.235899in}}%
\pgfpathlineto{\pgfqpoint{6.029254in}{2.249784in}}%
\pgfpathlineto{\pgfqpoint{6.036985in}{2.257101in}}%
\pgfpathlineto{\pgfqpoint{6.039563in}{2.243029in}}%
\pgfpathlineto{\pgfqpoint{6.042140in}{2.248658in}}%
\pgfpathlineto{\pgfqpoint{6.044717in}{2.248658in}}%
\pgfpathlineto{\pgfqpoint{6.047294in}{2.220325in}}%
\pgfpathlineto{\pgfqpoint{6.055026in}{2.200624in}}%
\pgfpathlineto{\pgfqpoint{6.057603in}{2.229895in}}%
\pgfpathlineto{\pgfqpoint{6.060180in}{2.234398in}}%
\pgfpathlineto{\pgfqpoint{6.062757in}{2.213195in}}%
\pgfpathlineto{\pgfqpoint{6.065335in}{2.213195in}}%
\pgfpathlineto{\pgfqpoint{6.073066in}{2.224641in}}%
\pgfpathlineto{\pgfqpoint{6.075644in}{2.216948in}}%
\pgfpathlineto{\pgfqpoint{6.078221in}{2.219950in}}%
\pgfpathlineto{\pgfqpoint{6.080798in}{2.209067in}}%
\pgfpathlineto{\pgfqpoint{6.083375in}{2.239276in}}%
\pgfpathlineto{\pgfqpoint{6.091107in}{2.232521in}}%
\pgfpathlineto{\pgfqpoint{6.093684in}{2.226705in}}%
\pgfpathlineto{\pgfqpoint{6.096261in}{2.251285in}}%
\pgfpathlineto{\pgfqpoint{6.098838in}{2.217698in}}%
\pgfpathlineto{\pgfqpoint{6.101416in}{2.205877in}}%
\pgfpathlineto{\pgfqpoint{6.109147in}{2.197997in}}%
\pgfpathlineto{\pgfqpoint{6.114302in}{2.210193in}}%
\pgfpathlineto{\pgfqpoint{6.116879in}{2.196120in}}%
\pgfpathlineto{\pgfqpoint{6.119456in}{2.208504in}}%
\pgfpathlineto{\pgfqpoint{6.129765in}{2.236086in}}%
\pgfpathlineto{\pgfqpoint{6.132342in}{2.250347in}}%
\pgfpathlineto{\pgfqpoint{6.134919in}{2.245843in}}%
\pgfpathlineto{\pgfqpoint{6.137497in}{2.264231in}}%
\pgfpathlineto{\pgfqpoint{6.145228in}{2.262355in}}%
\pgfpathlineto{\pgfqpoint{6.150383in}{2.288624in}}%
\pgfpathlineto{\pgfqpoint{6.152960in}{2.285997in}}%
\pgfpathlineto{\pgfqpoint{6.155537in}{2.313579in}}%
\pgfpathlineto{\pgfqpoint{6.163269in}{2.327839in}}%
\pgfpathlineto{\pgfqpoint{6.165846in}{2.320897in}}%
\pgfpathlineto{\pgfqpoint{6.168423in}{2.336846in}}%
\pgfpathlineto{\pgfqpoint{6.171000in}{2.313016in}}%
\pgfpathlineto{\pgfqpoint{6.173578in}{2.305323in}}%
\pgfpathlineto{\pgfqpoint{6.181309in}{2.312829in}}%
\pgfpathlineto{\pgfqpoint{6.183886in}{2.337221in}}%
\pgfpathlineto{\pgfqpoint{6.186464in}{2.344726in}}%
\pgfpathlineto{\pgfqpoint{6.189041in}{2.331967in}}%
\pgfpathlineto{\pgfqpoint{6.191618in}{2.337596in}}%
\pgfpathlineto{\pgfqpoint{6.199350in}{2.348667in}}%
\pgfpathlineto{\pgfqpoint{6.201927in}{2.344726in}}%
\pgfpathlineto{\pgfqpoint{6.204504in}{2.339097in}}%
\pgfpathlineto{\pgfqpoint{6.207081in}{2.311703in}}%
\pgfpathlineto{\pgfqpoint{6.217390in}{2.355609in}}%
\pgfpathlineto{\pgfqpoint{6.222545in}{2.385630in}}%
\pgfpathlineto{\pgfqpoint{6.225122in}{2.364053in}}%
\pgfpathlineto{\pgfqpoint{6.227699in}{2.370620in}}%
\pgfpathlineto{\pgfqpoint{6.235431in}{2.383754in}}%
\pgfpathlineto{\pgfqpoint{6.238008in}{2.399515in}}%
\pgfpathlineto{\pgfqpoint{6.240585in}{2.383004in}}%
\pgfpathlineto{\pgfqpoint{6.243162in}{2.383754in}}%
\pgfpathlineto{\pgfqpoint{6.253471in}{2.394074in}}%
\pgfpathlineto{\pgfqpoint{6.256048in}{2.393886in}}%
\pgfpathlineto{\pgfqpoint{6.258626in}{2.393136in}}%
\pgfpathlineto{\pgfqpoint{6.261203in}{2.389383in}}%
\pgfpathlineto{\pgfqpoint{6.263780in}{2.396325in}}%
\pgfpathlineto{\pgfqpoint{6.274089in}{2.379814in}}%
\pgfpathlineto{\pgfqpoint{6.276666in}{2.383004in}}%
\pgfpathlineto{\pgfqpoint{6.279243in}{2.408522in}}%
\pgfpathlineto{\pgfqpoint{6.281821in}{2.406270in}}%
\pgfpathlineto{\pgfqpoint{6.289552in}{2.438168in}}%
\pgfpathlineto{\pgfqpoint{6.292129in}{2.439294in}}%
\pgfpathlineto{\pgfqpoint{6.294707in}{2.432164in}}%
\pgfpathlineto{\pgfqpoint{6.297284in}{2.436854in}}%
\pgfpathlineto{\pgfqpoint{6.299861in}{2.422407in}}%
\pgfpathlineto{\pgfqpoint{6.307593in}{2.413588in}}%
\pgfpathlineto{\pgfqpoint{6.310170in}{2.424283in}}%
\pgfpathlineto{\pgfqpoint{6.312747in}{2.416215in}}%
\pgfpathlineto{\pgfqpoint{6.315324in}{2.422594in}}%
\pgfpathlineto{\pgfqpoint{6.317902in}{2.427097in}}%
\pgfpathlineto{\pgfqpoint{6.325633in}{2.380001in}}%
\pgfpathlineto{\pgfqpoint{6.328210in}{2.378500in}}%
\pgfpathlineto{\pgfqpoint{6.330788in}{2.396138in}}%
\pgfpathlineto{\pgfqpoint{6.335942in}{2.421281in}}%
\pgfpathlineto{\pgfqpoint{6.343674in}{2.424846in}}%
\pgfpathlineto{\pgfqpoint{6.346251in}{2.427660in}}%
\pgfpathlineto{\pgfqpoint{6.348828in}{2.422594in}}%
\pgfpathlineto{\pgfqpoint{6.351405in}{2.439856in}}%
\pgfpathlineto{\pgfqpoint{6.353982in}{2.448863in}}%
\pgfpathlineto{\pgfqpoint{6.361714in}{2.454117in}}%
\pgfpathlineto{\pgfqpoint{6.364291in}{2.459558in}}%
\pgfpathlineto{\pgfqpoint{6.366869in}{2.480573in}}%
\pgfpathlineto{\pgfqpoint{6.369446in}{2.474381in}}%
\pgfpathlineto{\pgfqpoint{6.372023in}{2.484701in}}%
\pgfpathlineto{\pgfqpoint{6.379755in}{2.477946in}}%
\pgfpathlineto{\pgfqpoint{6.382332in}{2.463874in}}%
\pgfpathlineto{\pgfqpoint{6.384909in}{2.468752in}}%
\pgfpathlineto{\pgfqpoint{6.387486in}{2.454304in}}%
\pgfpathlineto{\pgfqpoint{6.390063in}{2.464061in}}%
\pgfpathlineto{\pgfqpoint{6.397795in}{2.463686in}}%
\pgfpathlineto{\pgfqpoint{6.400372in}{2.480010in}}%
\pgfpathlineto{\pgfqpoint{6.402950in}{2.500087in}}%
\pgfpathlineto{\pgfqpoint{6.405527in}{2.505716in}}%
\pgfpathlineto{\pgfqpoint{6.408104in}{2.504027in}}%
\pgfpathlineto{\pgfqpoint{6.415836in}{2.510219in}}%
\pgfpathlineto{\pgfqpoint{6.418413in}{2.509093in}}%
\pgfpathlineto{\pgfqpoint{6.420990in}{2.529358in}}%
\pgfpathlineto{\pgfqpoint{6.423567in}{2.527669in}}%
\pgfpathlineto{\pgfqpoint{6.426144in}{2.534799in}}%
\pgfpathlineto{\pgfqpoint{6.436453in}{2.548309in}}%
\pgfpathlineto{\pgfqpoint{6.439031in}{2.555251in}}%
\pgfpathlineto{\pgfqpoint{6.444185in}{2.544368in}}%
\pgfpathlineto{\pgfqpoint{6.451917in}{2.535550in}}%
\pgfpathlineto{\pgfqpoint{6.454494in}{2.546245in}}%
\pgfpathlineto{\pgfqpoint{6.457071in}{2.514159in}}%
\pgfpathlineto{\pgfqpoint{6.459648in}{2.531985in}}%
\pgfpathlineto{\pgfqpoint{6.462225in}{2.509844in}}%
\pgfpathlineto{\pgfqpoint{6.469957in}{2.512095in}}%
\pgfpathlineto{\pgfqpoint{6.472534in}{2.540053in}}%
\pgfpathlineto{\pgfqpoint{6.475111in}{2.528420in}}%
\pgfpathlineto{\pgfqpoint{6.480266in}{2.549622in}}%
\pgfpathlineto{\pgfqpoint{6.487998in}{2.557315in}}%
\pgfpathlineto{\pgfqpoint{6.490575in}{2.575516in}}%
\pgfpathlineto{\pgfqpoint{6.493152in}{2.510594in}}%
\pgfpathlineto{\pgfqpoint{6.495729in}{2.561255in}}%
\pgfpathlineto{\pgfqpoint{6.498306in}{2.526731in}}%
\pgfpathlineto{\pgfqpoint{6.506038in}{2.465375in}}%
\pgfpathlineto{\pgfqpoint{6.511192in}{2.508343in}}%
\pgfpathlineto{\pgfqpoint{6.513770in}{2.539115in}}%
\pgfpathlineto{\pgfqpoint{6.516347in}{2.561443in}}%
\pgfpathlineto{\pgfqpoint{6.524079in}{2.557127in}}%
\pgfpathlineto{\pgfqpoint{6.526656in}{2.577392in}}%
\pgfpathlineto{\pgfqpoint{6.529233in}{2.574577in}}%
\pgfpathlineto{\pgfqpoint{6.531810in}{2.567447in}}%
\pgfpathlineto{\pgfqpoint{6.534387in}{2.584147in}}%
\pgfpathlineto{\pgfqpoint{6.542119in}{2.579081in}}%
\pgfpathlineto{\pgfqpoint{6.544696in}{2.552812in}}%
\pgfpathlineto{\pgfqpoint{6.547273in}{2.552437in}}%
\pgfpathlineto{\pgfqpoint{6.552428in}{2.563132in}}%
\pgfpathlineto{\pgfqpoint{6.562737in}{2.568573in}}%
\pgfpathlineto{\pgfqpoint{6.565314in}{2.587712in}}%
\pgfpathlineto{\pgfqpoint{6.567891in}{2.593904in}}%
\pgfpathlineto{\pgfqpoint{6.570468in}{2.587712in}}%
\pgfpathlineto{\pgfqpoint{6.570468in}{2.587712in}}%
\pgfusepath{stroke}%
\end{pgfscope}%
\begin{pgfscope}%
\pgfpathrectangle{\pgfqpoint{0.652287in}{0.521603in}}{\pgfqpoint{6.200000in}{4.530000in}}%
\pgfusepath{clip}%
\pgfsetroundcap%
\pgfsetroundjoin%
\pgfsetlinewidth{1.505625pt}%
\definecolor{currentstroke}{rgb}{0.090196,0.745098,0.811765}%
\pgfsetstrokecolor{currentstroke}%
\pgfsetdash{}{0pt}%
\pgfpathmoveto{\pgfqpoint{0.934105in}{1.117040in}}%
\pgfpathlineto{\pgfqpoint{0.936682in}{1.126046in}}%
\pgfpathlineto{\pgfqpoint{0.939259in}{1.137117in}}%
\pgfpathlineto{\pgfqpoint{0.941836in}{1.144059in}}%
\pgfpathlineto{\pgfqpoint{0.949568in}{1.141244in}}%
\pgfpathlineto{\pgfqpoint{0.952145in}{1.139368in}}%
\pgfpathlineto{\pgfqpoint{0.954722in}{1.123607in}}%
\pgfpathlineto{\pgfqpoint{0.957300in}{1.123982in}}%
\pgfpathlineto{\pgfqpoint{0.959877in}{1.118541in}}%
\pgfpathlineto{\pgfqpoint{0.970186in}{1.119854in}}%
\pgfpathlineto{\pgfqpoint{0.972763in}{1.129048in}}%
\pgfpathlineto{\pgfqpoint{0.975340in}{1.135991in}}%
\pgfpathlineto{\pgfqpoint{0.977917in}{1.133927in}}%
\pgfpathlineto{\pgfqpoint{0.988226in}{1.132801in}}%
\pgfpathlineto{\pgfqpoint{0.990803in}{1.138055in}}%
\pgfpathlineto{\pgfqpoint{0.993381in}{1.134490in}}%
\pgfpathlineto{\pgfqpoint{0.995958in}{1.132801in}}%
\pgfpathlineto{\pgfqpoint{1.006267in}{1.126984in}}%
\pgfpathlineto{\pgfqpoint{1.008844in}{1.134114in}}%
\pgfpathlineto{\pgfqpoint{1.011421in}{1.127172in}}%
\pgfpathlineto{\pgfqpoint{1.013998in}{1.145560in}}%
\pgfpathlineto{\pgfqpoint{1.021730in}{1.153253in}}%
\pgfpathlineto{\pgfqpoint{1.024307in}{1.162072in}}%
\pgfpathlineto{\pgfqpoint{1.029462in}{1.171453in}}%
\pgfpathlineto{\pgfqpoint{1.032039in}{1.169952in}}%
\pgfpathlineto{\pgfqpoint{1.039770in}{1.175769in}}%
\pgfpathlineto{\pgfqpoint{1.042348in}{1.172579in}}%
\pgfpathlineto{\pgfqpoint{1.044925in}{1.166575in}}%
\pgfpathlineto{\pgfqpoint{1.047502in}{1.171641in}}%
\pgfpathlineto{\pgfqpoint{1.050079in}{1.175019in}}%
\pgfpathlineto{\pgfqpoint{1.060388in}{1.172016in}}%
\pgfpathlineto{\pgfqpoint{1.062965in}{1.166950in}}%
\pgfpathlineto{\pgfqpoint{1.065543in}{1.170515in}}%
\pgfpathlineto{\pgfqpoint{1.068120in}{1.167701in}}%
\pgfpathlineto{\pgfqpoint{1.075851in}{1.173330in}}%
\pgfpathlineto{\pgfqpoint{1.078429in}{1.178208in}}%
\pgfpathlineto{\pgfqpoint{1.081006in}{1.179146in}}%
\pgfpathlineto{\pgfqpoint{1.083583in}{1.185901in}}%
\pgfpathlineto{\pgfqpoint{1.086160in}{1.185338in}}%
\pgfpathlineto{\pgfqpoint{1.093892in}{1.191155in}}%
\pgfpathlineto{\pgfqpoint{1.096469in}{1.179334in}}%
\pgfpathlineto{\pgfqpoint{1.099046in}{1.175019in}}%
\pgfpathlineto{\pgfqpoint{1.104201in}{1.183462in}}%
\pgfpathlineto{\pgfqpoint{1.111932in}{1.185151in}}%
\pgfpathlineto{\pgfqpoint{1.114510in}{1.213296in}}%
\pgfpathlineto{\pgfqpoint{1.117087in}{1.204289in}}%
\pgfpathlineto{\pgfqpoint{1.119664in}{1.204102in}}%
\pgfpathlineto{\pgfqpoint{1.122241in}{1.199411in}}%
\pgfpathlineto{\pgfqpoint{1.129973in}{1.203726in}}%
\pgfpathlineto{\pgfqpoint{1.132550in}{1.200349in}}%
\pgfpathlineto{\pgfqpoint{1.137704in}{1.201100in}}%
\pgfpathlineto{\pgfqpoint{1.140282in}{1.207291in}}%
\pgfpathlineto{\pgfqpoint{1.148013in}{1.219488in}}%
\pgfpathlineto{\pgfqpoint{1.150591in}{1.215735in}}%
\pgfpathlineto{\pgfqpoint{1.153168in}{1.204852in}}%
\pgfpathlineto{\pgfqpoint{1.155745in}{1.196409in}}%
\pgfpathlineto{\pgfqpoint{1.158322in}{1.209355in}}%
\pgfpathlineto{\pgfqpoint{1.166054in}{1.210481in}}%
\pgfpathlineto{\pgfqpoint{1.168631in}{1.201100in}}%
\pgfpathlineto{\pgfqpoint{1.171208in}{1.195095in}}%
\pgfpathlineto{\pgfqpoint{1.173785in}{1.197535in}}%
\pgfpathlineto{\pgfqpoint{1.184094in}{1.181210in}}%
\pgfpathlineto{\pgfqpoint{1.186672in}{1.162259in}}%
\pgfpathlineto{\pgfqpoint{1.189249in}{1.168639in}}%
\pgfpathlineto{\pgfqpoint{1.191826in}{1.181773in}}%
\pgfpathlineto{\pgfqpoint{1.194403in}{1.176707in}}%
\pgfpathlineto{\pgfqpoint{1.202135in}{1.173517in}}%
\pgfpathlineto{\pgfqpoint{1.204712in}{1.190780in}}%
\pgfpathlineto{\pgfqpoint{1.207289in}{1.187590in}}%
\pgfpathlineto{\pgfqpoint{1.209866in}{1.180648in}}%
\pgfpathlineto{\pgfqpoint{1.212444in}{1.185151in}}%
\pgfpathlineto{\pgfqpoint{1.220175in}{1.179522in}}%
\pgfpathlineto{\pgfqpoint{1.222753in}{1.182336in}}%
\pgfpathlineto{\pgfqpoint{1.225330in}{1.191155in}}%
\pgfpathlineto{\pgfqpoint{1.227907in}{1.202225in}}%
\pgfpathlineto{\pgfqpoint{1.230484in}{1.202225in}}%
\pgfpathlineto{\pgfqpoint{1.238216in}{1.198097in}}%
\pgfpathlineto{\pgfqpoint{1.240793in}{1.209543in}}%
\pgfpathlineto{\pgfqpoint{1.243370in}{1.205415in}}%
\pgfpathlineto{\pgfqpoint{1.245947in}{1.209918in}}%
\pgfpathlineto{\pgfqpoint{1.248525in}{1.195095in}}%
\pgfpathlineto{\pgfqpoint{1.256256in}{1.210106in}}%
\pgfpathlineto{\pgfqpoint{1.258833in}{1.218174in}}%
\pgfpathlineto{\pgfqpoint{1.261411in}{1.230370in}}%
\pgfpathlineto{\pgfqpoint{1.266565in}{1.239564in}}%
\pgfpathlineto{\pgfqpoint{1.276874in}{1.230183in}}%
\pgfpathlineto{\pgfqpoint{1.279451in}{1.231309in}}%
\pgfpathlineto{\pgfqpoint{1.282028in}{1.218737in}}%
\pgfpathlineto{\pgfqpoint{1.284606in}{1.209918in}}%
\pgfpathlineto{\pgfqpoint{1.292337in}{1.219675in}}%
\pgfpathlineto{\pgfqpoint{1.294914in}{1.219675in}}%
\pgfpathlineto{\pgfqpoint{1.297492in}{1.216673in}}%
\pgfpathlineto{\pgfqpoint{1.300069in}{1.220613in}}%
\pgfpathlineto{\pgfqpoint{1.302646in}{1.221552in}}%
\pgfpathlineto{\pgfqpoint{1.312955in}{1.238063in}}%
\pgfpathlineto{\pgfqpoint{1.315532in}{1.233373in}}%
\pgfpathlineto{\pgfqpoint{1.318109in}{1.242004in}}%
\pgfpathlineto{\pgfqpoint{1.320687in}{1.219863in}}%
\pgfpathlineto{\pgfqpoint{1.328418in}{1.220051in}}%
\pgfpathlineto{\pgfqpoint{1.330995in}{1.227368in}}%
\pgfpathlineto{\pgfqpoint{1.333573in}{1.239002in}}%
\pgfpathlineto{\pgfqpoint{1.336150in}{1.240690in}}%
\pgfpathlineto{\pgfqpoint{1.338727in}{1.251010in}}%
\pgfpathlineto{\pgfqpoint{1.346459in}{1.243505in}}%
\pgfpathlineto{\pgfqpoint{1.349036in}{1.253262in}}%
\pgfpathlineto{\pgfqpoint{1.351613in}{1.250822in}}%
\pgfpathlineto{\pgfqpoint{1.354190in}{1.266959in}}%
\pgfpathlineto{\pgfqpoint{1.356768in}{1.265270in}}%
\pgfpathlineto{\pgfqpoint{1.364499in}{1.265458in}}%
\pgfpathlineto{\pgfqpoint{1.367076in}{1.272400in}}%
\pgfpathlineto{\pgfqpoint{1.369654in}{1.276153in}}%
\pgfpathlineto{\pgfqpoint{1.372231in}{1.270524in}}%
\pgfpathlineto{\pgfqpoint{1.374808in}{1.271837in}}%
\pgfpathlineto{\pgfqpoint{1.382540in}{1.258703in}}%
\pgfpathlineto{\pgfqpoint{1.387694in}{1.278592in}}%
\pgfpathlineto{\pgfqpoint{1.390271in}{1.277654in}}%
\pgfpathlineto{\pgfqpoint{1.392849in}{1.289100in}}%
\pgfpathlineto{\pgfqpoint{1.400580in}{1.292852in}}%
\pgfpathlineto{\pgfqpoint{1.403157in}{1.290788in}}%
\pgfpathlineto{\pgfqpoint{1.408312in}{1.282533in}}%
\pgfpathlineto{\pgfqpoint{1.410889in}{1.281407in}}%
\pgfpathlineto{\pgfqpoint{1.418621in}{1.280656in}}%
\pgfpathlineto{\pgfqpoint{1.421198in}{1.269961in}}%
\pgfpathlineto{\pgfqpoint{1.423775in}{1.268460in}}%
\pgfpathlineto{\pgfqpoint{1.426352in}{1.270712in}}%
\pgfpathlineto{\pgfqpoint{1.428930in}{1.284034in}}%
\pgfpathlineto{\pgfqpoint{1.436661in}{1.278405in}}%
\pgfpathlineto{\pgfqpoint{1.439238in}{1.303548in}}%
\pgfpathlineto{\pgfqpoint{1.441816in}{1.303548in}}%
\pgfpathlineto{\pgfqpoint{1.446970in}{1.290788in}}%
\pgfpathlineto{\pgfqpoint{1.454702in}{1.280469in}}%
\pgfpathlineto{\pgfqpoint{1.459856in}{1.285722in}}%
\pgfpathlineto{\pgfqpoint{1.462433in}{1.309552in}}%
\pgfpathlineto{\pgfqpoint{1.465010in}{1.313492in}}%
\pgfpathlineto{\pgfqpoint{1.472742in}{1.311240in}}%
\pgfpathlineto{\pgfqpoint{1.475319in}{1.299982in}}%
\pgfpathlineto{\pgfqpoint{1.477897in}{1.294541in}}%
\pgfpathlineto{\pgfqpoint{1.480474in}{1.297356in}}%
\pgfpathlineto{\pgfqpoint{1.483051in}{1.310678in}}%
\pgfpathlineto{\pgfqpoint{1.490783in}{1.308614in}}%
\pgfpathlineto{\pgfqpoint{1.493360in}{1.311240in}}%
\pgfpathlineto{\pgfqpoint{1.495937in}{1.322874in}}%
\pgfpathlineto{\pgfqpoint{1.498514in}{1.313867in}}%
\pgfpathlineto{\pgfqpoint{1.501091in}{1.308614in}}%
\pgfpathlineto{\pgfqpoint{1.508823in}{1.312179in}}%
\pgfpathlineto{\pgfqpoint{1.511400in}{1.309364in}}%
\pgfpathlineto{\pgfqpoint{1.513978in}{1.312742in}}%
\pgfpathlineto{\pgfqpoint{1.516555in}{1.318746in}}%
\pgfpathlineto{\pgfqpoint{1.519132in}{1.322311in}}%
\pgfpathlineto{\pgfqpoint{1.526864in}{1.322123in}}%
\pgfpathlineto{\pgfqpoint{1.529441in}{1.308426in}}%
\pgfpathlineto{\pgfqpoint{1.532018in}{1.308801in}}%
\pgfpathlineto{\pgfqpoint{1.534595in}{1.300545in}}%
\pgfpathlineto{\pgfqpoint{1.537172in}{1.307113in}}%
\pgfpathlineto{\pgfqpoint{1.547481in}{1.308238in}}%
\pgfpathlineto{\pgfqpoint{1.550058in}{1.315181in}}%
\pgfpathlineto{\pgfqpoint{1.552636in}{1.304673in}}%
\pgfpathlineto{\pgfqpoint{1.565522in}{1.308801in}}%
\pgfpathlineto{\pgfqpoint{1.570676in}{1.345953in}}%
\pgfpathlineto{\pgfqpoint{1.573253in}{1.343889in}}%
\pgfpathlineto{\pgfqpoint{1.580985in}{1.340136in}}%
\pgfpathlineto{\pgfqpoint{1.583562in}{1.340887in}}%
\pgfpathlineto{\pgfqpoint{1.586139in}{1.343326in}}%
\pgfpathlineto{\pgfqpoint{1.588717in}{1.358524in}}%
\pgfpathlineto{\pgfqpoint{1.591294in}{1.354209in}}%
\pgfpathlineto{\pgfqpoint{1.599026in}{1.350831in}}%
\pgfpathlineto{\pgfqpoint{1.601603in}{1.346703in}}%
\pgfpathlineto{\pgfqpoint{1.604180in}{1.360213in}}%
\pgfpathlineto{\pgfqpoint{1.606757in}{1.359462in}}%
\pgfpathlineto{\pgfqpoint{1.609334in}{1.360776in}}%
\pgfpathlineto{\pgfqpoint{1.617066in}{1.363966in}}%
\pgfpathlineto{\pgfqpoint{1.619643in}{1.357398in}}%
\pgfpathlineto{\pgfqpoint{1.622220in}{1.347266in}}%
\pgfpathlineto{\pgfqpoint{1.624798in}{1.358149in}}%
\pgfpathlineto{\pgfqpoint{1.627375in}{1.353083in}}%
\pgfpathlineto{\pgfqpoint{1.635106in}{1.349518in}}%
\pgfpathlineto{\pgfqpoint{1.637684in}{1.342200in}}%
\pgfpathlineto{\pgfqpoint{1.640261in}{1.356273in}}%
\pgfpathlineto{\pgfqpoint{1.642838in}{1.358899in}}%
\pgfpathlineto{\pgfqpoint{1.645415in}{1.364716in}}%
\pgfpathlineto{\pgfqpoint{1.653147in}{1.353833in}}%
\pgfpathlineto{\pgfqpoint{1.655724in}{1.340136in}}%
\pgfpathlineto{\pgfqpoint{1.658301in}{1.335070in}}%
\pgfpathlineto{\pgfqpoint{1.660879in}{1.320247in}}%
\pgfpathlineto{\pgfqpoint{1.663456in}{1.324562in}}%
\pgfpathlineto{\pgfqpoint{1.671187in}{1.327940in}}%
\pgfpathlineto{\pgfqpoint{1.673765in}{1.335445in}}%
\pgfpathlineto{\pgfqpoint{1.676342in}{1.353270in}}%
\pgfpathlineto{\pgfqpoint{1.678919in}{1.355522in}}%
\pgfpathlineto{\pgfqpoint{1.681496in}{1.346703in}}%
\pgfpathlineto{\pgfqpoint{1.689228in}{1.344827in}}%
\pgfpathlineto{\pgfqpoint{1.691805in}{1.327377in}}%
\pgfpathlineto{\pgfqpoint{1.694382in}{1.325501in}}%
\pgfpathlineto{\pgfqpoint{1.696960in}{1.318933in}}%
\pgfpathlineto{\pgfqpoint{1.699537in}{1.315931in}}%
\pgfpathlineto{\pgfqpoint{1.712423in}{1.299607in}}%
\pgfpathlineto{\pgfqpoint{1.715000in}{1.310865in}}%
\pgfpathlineto{\pgfqpoint{1.717577in}{1.312179in}}%
\pgfpathlineto{\pgfqpoint{1.727886in}{1.322499in}}%
\pgfpathlineto{\pgfqpoint{1.730463in}{1.315931in}}%
\pgfpathlineto{\pgfqpoint{1.733041in}{1.315181in}}%
\pgfpathlineto{\pgfqpoint{1.735618in}{1.264895in}}%
\pgfpathlineto{\pgfqpoint{1.743349in}{1.271462in}}%
\pgfpathlineto{\pgfqpoint{1.745927in}{1.280093in}}%
\pgfpathlineto{\pgfqpoint{1.748504in}{1.266771in}}%
\pgfpathlineto{\pgfqpoint{1.751081in}{1.271837in}}%
\pgfpathlineto{\pgfqpoint{1.753658in}{1.270899in}}%
\pgfpathlineto{\pgfqpoint{1.761390in}{1.279155in}}%
\pgfpathlineto{\pgfqpoint{1.763967in}{1.288162in}}%
\pgfpathlineto{\pgfqpoint{1.766544in}{1.292290in}}%
\pgfpathlineto{\pgfqpoint{1.771699in}{1.302046in}}%
\pgfpathlineto{\pgfqpoint{1.779430in}{1.298106in}}%
\pgfpathlineto{\pgfqpoint{1.782008in}{1.290788in}}%
\pgfpathlineto{\pgfqpoint{1.787162in}{1.309739in}}%
\pgfpathlineto{\pgfqpoint{1.789739in}{1.308801in}}%
\pgfpathlineto{\pgfqpoint{1.797471in}{1.302609in}}%
\pgfpathlineto{\pgfqpoint{1.800048in}{1.302609in}}%
\pgfpathlineto{\pgfqpoint{1.802625in}{1.307675in}}%
\pgfpathlineto{\pgfqpoint{1.807780in}{1.314430in}}%
\pgfpathlineto{\pgfqpoint{1.815511in}{1.315556in}}%
\pgfpathlineto{\pgfqpoint{1.820666in}{1.321373in}}%
\pgfpathlineto{\pgfqpoint{1.825820in}{1.304673in}}%
\pgfpathlineto{\pgfqpoint{1.833552in}{1.315181in}}%
\pgfpathlineto{\pgfqpoint{1.836129in}{1.331317in}}%
\pgfpathlineto{\pgfqpoint{1.838706in}{1.326439in}}%
\pgfpathlineto{\pgfqpoint{1.841283in}{1.343513in}}%
\pgfpathlineto{\pgfqpoint{1.843861in}{1.327565in}}%
\pgfpathlineto{\pgfqpoint{1.851592in}{1.325501in}}%
\pgfpathlineto{\pgfqpoint{1.856747in}{1.324938in}}%
\pgfpathlineto{\pgfqpoint{1.861901in}{1.312929in}}%
\pgfpathlineto{\pgfqpoint{1.869633in}{1.323812in}}%
\pgfpathlineto{\pgfqpoint{1.874787in}{1.346328in}}%
\pgfpathlineto{\pgfqpoint{1.877364in}{1.348204in}}%
\pgfpathlineto{\pgfqpoint{1.879942in}{1.365091in}}%
\pgfpathlineto{\pgfqpoint{1.887673in}{1.344076in}}%
\pgfpathlineto{\pgfqpoint{1.890251in}{1.340511in}}%
\pgfpathlineto{\pgfqpoint{1.895405in}{1.341074in}}%
\pgfpathlineto{\pgfqpoint{1.897982in}{1.337509in}}%
\pgfpathlineto{\pgfqpoint{1.905714in}{1.337697in}}%
\pgfpathlineto{\pgfqpoint{1.910868in}{1.353833in}}%
\pgfpathlineto{\pgfqpoint{1.913445in}{1.368844in}}%
\pgfpathlineto{\pgfqpoint{1.916023in}{1.367718in}}%
\pgfpathlineto{\pgfqpoint{1.926332in}{1.374285in}}%
\pgfpathlineto{\pgfqpoint{1.928909in}{1.395300in}}%
\pgfpathlineto{\pgfqpoint{1.931486in}{1.395300in}}%
\pgfpathlineto{\pgfqpoint{1.934063in}{1.402618in}}%
\pgfpathlineto{\pgfqpoint{1.941795in}{1.402243in}}%
\pgfpathlineto{\pgfqpoint{1.944372in}{1.396051in}}%
\pgfpathlineto{\pgfqpoint{1.946949in}{1.392486in}}%
\pgfpathlineto{\pgfqpoint{1.949526in}{1.394175in}}%
\pgfpathlineto{\pgfqpoint{1.952104in}{1.406183in}}%
\pgfpathlineto{\pgfqpoint{1.959835in}{1.394362in}}%
\pgfpathlineto{\pgfqpoint{1.962412in}{1.401117in}}%
\pgfpathlineto{\pgfqpoint{1.964990in}{1.405057in}}%
\pgfpathlineto{\pgfqpoint{1.967567in}{1.402243in}}%
\pgfpathlineto{\pgfqpoint{1.970144in}{1.407497in}}%
\pgfpathlineto{\pgfqpoint{1.977876in}{1.408998in}}%
\pgfpathlineto{\pgfqpoint{1.980453in}{1.412375in}}%
\pgfpathlineto{\pgfqpoint{1.983030in}{1.412563in}}%
\pgfpathlineto{\pgfqpoint{1.985607in}{1.411249in}}%
\pgfpathlineto{\pgfqpoint{1.988185in}{1.423821in}}%
\pgfpathlineto{\pgfqpoint{1.998493in}{1.425885in}}%
\pgfpathlineto{\pgfqpoint{2.001071in}{1.406371in}}%
\pgfpathlineto{\pgfqpoint{2.003648in}{1.399053in}}%
\pgfpathlineto{\pgfqpoint{2.006225in}{1.400366in}}%
\pgfpathlineto{\pgfqpoint{2.013957in}{1.389108in}}%
\pgfpathlineto{\pgfqpoint{2.016534in}{1.394362in}}%
\pgfpathlineto{\pgfqpoint{2.019111in}{1.404307in}}%
\pgfpathlineto{\pgfqpoint{2.021688in}{1.406183in}}%
\pgfpathlineto{\pgfqpoint{2.024266in}{1.418942in}}%
\pgfpathlineto{\pgfqpoint{2.031997in}{1.427010in}}%
\pgfpathlineto{\pgfqpoint{2.034574in}{1.438644in}}%
\pgfpathlineto{\pgfqpoint{2.037152in}{1.436580in}}%
\pgfpathlineto{\pgfqpoint{2.039729in}{1.436017in}}%
\pgfpathlineto{\pgfqpoint{2.042306in}{1.454405in}}%
\pgfpathlineto{\pgfqpoint{2.050038in}{1.458908in}}%
\pgfpathlineto{\pgfqpoint{2.052615in}{1.449526in}}%
\pgfpathlineto{\pgfqpoint{2.055192in}{1.453467in}}%
\pgfpathlineto{\pgfqpoint{2.057769in}{1.460409in}}%
\pgfpathlineto{\pgfqpoint{2.060347in}{1.457595in}}%
\pgfpathlineto{\pgfqpoint{2.068078in}{1.444648in}}%
\pgfpathlineto{\pgfqpoint{2.070655in}{1.435829in}}%
\pgfpathlineto{\pgfqpoint{2.073233in}{1.446524in}}%
\pgfpathlineto{\pgfqpoint{2.075810in}{1.435829in}}%
\pgfpathlineto{\pgfqpoint{2.078387in}{1.443897in}}%
\pgfpathlineto{\pgfqpoint{2.086119in}{1.434140in}}%
\pgfpathlineto{\pgfqpoint{2.088696in}{1.441271in}}%
\pgfpathlineto{\pgfqpoint{2.091273in}{1.438456in}}%
\pgfpathlineto{\pgfqpoint{2.093850in}{1.444273in}}%
\pgfpathlineto{\pgfqpoint{2.104159in}{1.442396in}}%
\pgfpathlineto{\pgfqpoint{2.106736in}{1.455531in}}%
\pgfpathlineto{\pgfqpoint{2.109314in}{1.451966in}}%
\pgfpathlineto{\pgfqpoint{2.111891in}{1.457782in}}%
\pgfpathlineto{\pgfqpoint{2.114468in}{1.459659in}}%
\pgfpathlineto{\pgfqpoint{2.124777in}{1.484426in}}%
\pgfpathlineto{\pgfqpoint{2.127354in}{1.500938in}}%
\pgfpathlineto{\pgfqpoint{2.129931in}{1.508631in}}%
\pgfpathlineto{\pgfqpoint{2.132509in}{1.508631in}}%
\pgfpathlineto{\pgfqpoint{2.140240in}{1.479923in}}%
\pgfpathlineto{\pgfqpoint{2.142817in}{1.512008in}}%
\pgfpathlineto{\pgfqpoint{2.145395in}{1.510695in}}%
\pgfpathlineto{\pgfqpoint{2.147972in}{1.498874in}}%
\pgfpathlineto{\pgfqpoint{2.150549in}{1.525893in}}%
\pgfpathlineto{\pgfqpoint{2.158281in}{1.533586in}}%
\pgfpathlineto{\pgfqpoint{2.160858in}{1.543531in}}%
\pgfpathlineto{\pgfqpoint{2.163435in}{1.532460in}}%
\pgfpathlineto{\pgfqpoint{2.166012in}{1.533399in}}%
\pgfpathlineto{\pgfqpoint{2.168589in}{1.531147in}}%
\pgfpathlineto{\pgfqpoint{2.176321in}{1.550661in}}%
\pgfpathlineto{\pgfqpoint{2.178898in}{1.547846in}}%
\pgfpathlineto{\pgfqpoint{2.181476in}{1.554226in}}%
\pgfpathlineto{\pgfqpoint{2.184053in}{1.565672in}}%
\pgfpathlineto{\pgfqpoint{2.186630in}{1.581433in}}%
\pgfpathlineto{\pgfqpoint{2.194362in}{1.585936in}}%
\pgfpathlineto{\pgfqpoint{2.196939in}{1.603198in}}%
\pgfpathlineto{\pgfqpoint{2.199516in}{1.601885in}}%
\pgfpathlineto{\pgfqpoint{2.204670in}{1.622712in}}%
\pgfpathlineto{\pgfqpoint{2.212402in}{1.624776in}}%
\pgfpathlineto{\pgfqpoint{2.214979in}{1.627215in}}%
\pgfpathlineto{\pgfqpoint{2.217557in}{1.630780in}}%
\pgfpathlineto{\pgfqpoint{2.220134in}{1.610141in}}%
\pgfpathlineto{\pgfqpoint{2.222711in}{1.612017in}}%
\pgfpathlineto{\pgfqpoint{2.230443in}{1.604137in}}%
\pgfpathlineto{\pgfqpoint{2.238174in}{1.588938in}}%
\pgfpathlineto{\pgfqpoint{2.240751in}{1.593254in}}%
\pgfpathlineto{\pgfqpoint{2.251060in}{1.613893in}}%
\pgfpathlineto{\pgfqpoint{2.253637in}{1.606576in}}%
\pgfpathlineto{\pgfqpoint{2.258792in}{1.551974in}}%
\pgfpathlineto{\pgfqpoint{2.266524in}{1.564358in}}%
\pgfpathlineto{\pgfqpoint{2.269101in}{1.573740in}}%
\pgfpathlineto{\pgfqpoint{2.271678in}{1.552725in}}%
\pgfpathlineto{\pgfqpoint{2.274255in}{1.552913in}}%
\pgfpathlineto{\pgfqpoint{2.276832in}{1.582371in}}%
\pgfpathlineto{\pgfqpoint{2.284564in}{1.564921in}}%
\pgfpathlineto{\pgfqpoint{2.287141in}{1.564358in}}%
\pgfpathlineto{\pgfqpoint{2.289718in}{1.550473in}}%
\pgfpathlineto{\pgfqpoint{2.292296in}{1.572802in}}%
\pgfpathlineto{\pgfqpoint{2.294873in}{1.564358in}}%
\pgfpathlineto{\pgfqpoint{2.302605in}{1.576179in}}%
\pgfpathlineto{\pgfqpoint{2.305182in}{1.589876in}}%
\pgfpathlineto{\pgfqpoint{2.307759in}{1.573365in}}%
\pgfpathlineto{\pgfqpoint{2.310336in}{1.533023in}}%
\pgfpathlineto{\pgfqpoint{2.312913in}{1.545970in}}%
\pgfpathlineto{\pgfqpoint{2.320645in}{1.540904in}}%
\pgfpathlineto{\pgfqpoint{2.323222in}{1.543156in}}%
\pgfpathlineto{\pgfqpoint{2.328377in}{1.562857in}}%
\pgfpathlineto{\pgfqpoint{2.330954in}{1.553100in}}%
\pgfpathlineto{\pgfqpoint{2.338685in}{1.566610in}}%
\pgfpathlineto{\pgfqpoint{2.341263in}{1.554977in}}%
\pgfpathlineto{\pgfqpoint{2.343840in}{1.560981in}}%
\pgfpathlineto{\pgfqpoint{2.348994in}{1.564733in}}%
\pgfpathlineto{\pgfqpoint{2.359303in}{1.583872in}}%
\pgfpathlineto{\pgfqpoint{2.361880in}{1.583309in}}%
\pgfpathlineto{\pgfqpoint{2.364458in}{1.612017in}}%
\pgfpathlineto{\pgfqpoint{2.367035in}{1.618960in}}%
\pgfpathlineto{\pgfqpoint{2.374766in}{1.600947in}}%
\pgfpathlineto{\pgfqpoint{2.377344in}{1.585373in}}%
\pgfpathlineto{\pgfqpoint{2.379921in}{1.591002in}}%
\pgfpathlineto{\pgfqpoint{2.382498in}{1.598883in}}%
\pgfpathlineto{\pgfqpoint{2.385075in}{1.587625in}}%
\pgfpathlineto{\pgfqpoint{2.392807in}{1.574678in}}%
\pgfpathlineto{\pgfqpoint{2.395384in}{1.575429in}}%
\pgfpathlineto{\pgfqpoint{2.397961in}{1.578806in}}%
\pgfpathlineto{\pgfqpoint{2.400539in}{1.577117in}}%
\pgfpathlineto{\pgfqpoint{2.403116in}{1.584623in}}%
\pgfpathlineto{\pgfqpoint{2.410847in}{1.578243in}}%
\pgfpathlineto{\pgfqpoint{2.413425in}{1.570925in}}%
\pgfpathlineto{\pgfqpoint{2.416002in}{1.578806in}}%
\pgfpathlineto{\pgfqpoint{2.418579in}{1.591002in}}%
\pgfpathlineto{\pgfqpoint{2.421156in}{1.610891in}}%
\pgfpathlineto{\pgfqpoint{2.428888in}{1.602448in}}%
\pgfpathlineto{\pgfqpoint{2.431465in}{1.620085in}}%
\pgfpathlineto{\pgfqpoint{2.434042in}{1.600571in}}%
\pgfpathlineto{\pgfqpoint{2.436620in}{1.597757in}}%
\pgfpathlineto{\pgfqpoint{2.439197in}{1.580307in}}%
\pgfpathlineto{\pgfqpoint{2.446928in}{1.566610in}}%
\pgfpathlineto{\pgfqpoint{2.449506in}{1.567360in}}%
\pgfpathlineto{\pgfqpoint{2.452083in}{1.566985in}}%
\pgfpathlineto{\pgfqpoint{2.454660in}{1.539966in}}%
\pgfpathlineto{\pgfqpoint{2.457237in}{1.536401in}}%
\pgfpathlineto{\pgfqpoint{2.464969in}{1.530584in}}%
\pgfpathlineto{\pgfqpoint{2.467546in}{1.531335in}}%
\pgfpathlineto{\pgfqpoint{2.470123in}{1.518576in}}%
\pgfpathlineto{\pgfqpoint{2.472701in}{1.527207in}}%
\pgfpathlineto{\pgfqpoint{2.475278in}{1.528708in}}%
\pgfpathlineto{\pgfqpoint{2.483009in}{1.522328in}}%
\pgfpathlineto{\pgfqpoint{2.485587in}{1.510883in}}%
\pgfpathlineto{\pgfqpoint{2.488164in}{1.512947in}}%
\pgfpathlineto{\pgfqpoint{2.490741in}{1.517637in}}%
\pgfpathlineto{\pgfqpoint{2.493318in}{1.513322in}}%
\pgfpathlineto{\pgfqpoint{2.503627in}{1.515011in}}%
\pgfpathlineto{\pgfqpoint{2.508782in}{1.521390in}}%
\pgfpathlineto{\pgfqpoint{2.511359in}{1.522891in}}%
\pgfpathlineto{\pgfqpoint{2.519090in}{1.526456in}}%
\pgfpathlineto{\pgfqpoint{2.529399in}{1.613893in}}%
\pgfpathlineto{\pgfqpoint{2.537131in}{1.618209in}}%
\pgfpathlineto{\pgfqpoint{2.539708in}{1.618960in}}%
\pgfpathlineto{\pgfqpoint{2.542285in}{1.621024in}}%
\pgfpathlineto{\pgfqpoint{2.544862in}{1.597194in}}%
\pgfpathlineto{\pgfqpoint{2.547440in}{1.584998in}}%
\pgfpathlineto{\pgfqpoint{2.555171in}{1.580682in}}%
\pgfpathlineto{\pgfqpoint{2.557749in}{1.573177in}}%
\pgfpathlineto{\pgfqpoint{2.560326in}{1.575429in}}%
\pgfpathlineto{\pgfqpoint{2.562903in}{1.588938in}}%
\pgfpathlineto{\pgfqpoint{2.565480in}{1.588188in}}%
\pgfpathlineto{\pgfqpoint{2.573212in}{1.576179in}}%
\pgfpathlineto{\pgfqpoint{2.575789in}{1.581996in}}%
\pgfpathlineto{\pgfqpoint{2.578366in}{1.582934in}}%
\pgfpathlineto{\pgfqpoint{2.580943in}{1.568111in}}%
\pgfpathlineto{\pgfqpoint{2.583521in}{1.590064in}}%
\pgfpathlineto{\pgfqpoint{2.591252in}{1.577868in}}%
\pgfpathlineto{\pgfqpoint{2.593830in}{1.567736in}}%
\pgfpathlineto{\pgfqpoint{2.596407in}{1.560793in}}%
\pgfpathlineto{\pgfqpoint{2.598984in}{1.594942in}}%
\pgfpathlineto{\pgfqpoint{2.601561in}{1.605638in}}%
\pgfpathlineto{\pgfqpoint{2.609293in}{1.616333in}}%
\pgfpathlineto{\pgfqpoint{2.611870in}{1.609578in}}%
\pgfpathlineto{\pgfqpoint{2.614447in}{1.608077in}}%
\pgfpathlineto{\pgfqpoint{2.617024in}{1.609203in}}%
\pgfpathlineto{\pgfqpoint{2.619602in}{1.621774in}}%
\pgfpathlineto{\pgfqpoint{2.627333in}{1.629655in}}%
\pgfpathlineto{\pgfqpoint{2.629911in}{1.653484in}}%
\pgfpathlineto{\pgfqpoint{2.632488in}{1.638473in}}%
\pgfpathlineto{\pgfqpoint{2.635065in}{1.654422in}}%
\pgfpathlineto{\pgfqpoint{2.637642in}{1.657987in}}%
\pgfpathlineto{\pgfqpoint{2.647951in}{1.652171in}}%
\pgfpathlineto{\pgfqpoint{2.650528in}{1.644478in}}%
\pgfpathlineto{\pgfqpoint{2.653105in}{1.646542in}}%
\pgfpathlineto{\pgfqpoint{2.655683in}{1.653672in}}%
\pgfpathlineto{\pgfqpoint{2.663414in}{1.650294in}}%
\pgfpathlineto{\pgfqpoint{2.665991in}{1.651045in}}%
\pgfpathlineto{\pgfqpoint{2.668569in}{1.653484in}}%
\pgfpathlineto{\pgfqpoint{2.671146in}{1.621774in}}%
\pgfpathlineto{\pgfqpoint{2.673723in}{1.646354in}}%
\pgfpathlineto{\pgfqpoint{2.681455in}{1.642226in}}%
\pgfpathlineto{\pgfqpoint{2.684032in}{1.632469in}}%
\pgfpathlineto{\pgfqpoint{2.689186in}{1.671122in}}%
\pgfpathlineto{\pgfqpoint{2.691764in}{1.670746in}}%
\pgfpathlineto{\pgfqpoint{2.699495in}{1.662115in}}%
\pgfpathlineto{\pgfqpoint{2.702072in}{1.655548in}}%
\pgfpathlineto{\pgfqpoint{2.704650in}{1.657424in}}%
\pgfpathlineto{\pgfqpoint{2.707227in}{1.669621in}}%
\pgfpathlineto{\pgfqpoint{2.709804in}{1.674124in}}%
\pgfpathlineto{\pgfqpoint{2.717536in}{1.666243in}}%
\pgfpathlineto{\pgfqpoint{2.720113in}{1.691011in}}%
\pgfpathlineto{\pgfqpoint{2.722690in}{1.683881in}}%
\pgfpathlineto{\pgfqpoint{2.727845in}{1.679940in}}%
\pgfpathlineto{\pgfqpoint{2.735576in}{1.686320in}}%
\pgfpathlineto{\pgfqpoint{2.738153in}{1.669058in}}%
\pgfpathlineto{\pgfqpoint{2.740731in}{1.670184in}}%
\pgfpathlineto{\pgfqpoint{2.743308in}{1.674687in}}%
\pgfpathlineto{\pgfqpoint{2.745885in}{1.695702in}}%
\pgfpathlineto{\pgfqpoint{2.753617in}{1.689697in}}%
\pgfpathlineto{\pgfqpoint{2.756194in}{1.697578in}}%
\pgfpathlineto{\pgfqpoint{2.758771in}{1.679565in}}%
\pgfpathlineto{\pgfqpoint{2.763926in}{1.679002in}}%
\pgfpathlineto{\pgfqpoint{2.771657in}{1.694388in}}%
\pgfpathlineto{\pgfqpoint{2.774234in}{1.697015in}}%
\pgfpathlineto{\pgfqpoint{2.776812in}{1.723659in}}%
\pgfpathlineto{\pgfqpoint{2.779389in}{1.737169in}}%
\pgfpathlineto{\pgfqpoint{2.781966in}{1.727224in}}%
\pgfpathlineto{\pgfqpoint{2.789698in}{1.742422in}}%
\pgfpathlineto{\pgfqpoint{2.792275in}{1.752367in}}%
\pgfpathlineto{\pgfqpoint{2.797429in}{1.765877in}}%
\pgfpathlineto{\pgfqpoint{2.800007in}{1.761186in}}%
\pgfpathlineto{\pgfqpoint{2.807738in}{1.793834in}}%
\pgfpathlineto{\pgfqpoint{2.810315in}{1.796649in}}%
\pgfpathlineto{\pgfqpoint{2.815470in}{1.794397in}}%
\pgfpathlineto{\pgfqpoint{2.818047in}{1.791582in}}%
\pgfpathlineto{\pgfqpoint{2.825779in}{1.786704in}}%
\pgfpathlineto{\pgfqpoint{2.828356in}{1.795710in}}%
\pgfpathlineto{\pgfqpoint{2.830933in}{1.776197in}}%
\pgfpathlineto{\pgfqpoint{2.833510in}{1.770567in}}%
\pgfpathlineto{\pgfqpoint{2.836088in}{1.779199in}}%
\pgfpathlineto{\pgfqpoint{2.843819in}{1.742422in}}%
\pgfpathlineto{\pgfqpoint{2.846396in}{1.762875in}}%
\pgfpathlineto{\pgfqpoint{2.848974in}{1.759872in}}%
\pgfpathlineto{\pgfqpoint{2.851551in}{1.758747in}}%
\pgfpathlineto{\pgfqpoint{2.854128in}{1.754619in}}%
\pgfpathlineto{\pgfqpoint{2.864437in}{1.758559in}}%
\pgfpathlineto{\pgfqpoint{2.867014in}{1.777698in}}%
\pgfpathlineto{\pgfqpoint{2.869591in}{1.768691in}}%
\pgfpathlineto{\pgfqpoint{2.872168in}{1.732853in}}%
\pgfpathlineto{\pgfqpoint{2.879900in}{1.724597in}}%
\pgfpathlineto{\pgfqpoint{2.882477in}{1.735480in}}%
\pgfpathlineto{\pgfqpoint{2.885055in}{1.708648in}}%
\pgfpathlineto{\pgfqpoint{2.887632in}{1.741484in}}%
\pgfpathlineto{\pgfqpoint{2.890209in}{1.730789in}}%
\pgfpathlineto{\pgfqpoint{2.897941in}{1.685382in}}%
\pgfpathlineto{\pgfqpoint{2.900518in}{1.703770in}}%
\pgfpathlineto{\pgfqpoint{2.903095in}{1.716154in}}%
\pgfpathlineto{\pgfqpoint{2.905672in}{1.782201in}}%
\pgfpathlineto{\pgfqpoint{2.908249in}{1.784077in}}%
\pgfpathlineto{\pgfqpoint{2.915981in}{1.808094in}}%
\pgfpathlineto{\pgfqpoint{2.918558in}{1.820853in}}%
\pgfpathlineto{\pgfqpoint{2.921136in}{1.822917in}}%
\pgfpathlineto{\pgfqpoint{2.923713in}{1.822730in}}%
\pgfpathlineto{\pgfqpoint{2.926290in}{1.845809in}}%
\pgfpathlineto{\pgfqpoint{2.936599in}{1.852000in}}%
\pgfpathlineto{\pgfqpoint{2.939176in}{1.839617in}}%
\pgfpathlineto{\pgfqpoint{2.941753in}{1.845246in}}%
\pgfpathlineto{\pgfqpoint{2.944330in}{1.861570in}}%
\pgfpathlineto{\pgfqpoint{2.952062in}{1.871890in}}%
\pgfpathlineto{\pgfqpoint{2.954639in}{1.862883in}}%
\pgfpathlineto{\pgfqpoint{2.957216in}{1.860632in}}%
\pgfpathlineto{\pgfqpoint{2.962371in}{1.873391in}}%
\pgfpathlineto{\pgfqpoint{2.970103in}{1.849937in}}%
\pgfpathlineto{\pgfqpoint{2.972680in}{1.888964in}}%
\pgfpathlineto{\pgfqpoint{2.975257in}{1.905851in}}%
\pgfpathlineto{\pgfqpoint{2.977834in}{1.917297in}}%
\pgfpathlineto{\pgfqpoint{2.980411in}{1.897595in}}%
\pgfpathlineto{\pgfqpoint{2.988143in}{1.893467in}}%
\pgfpathlineto{\pgfqpoint{2.990720in}{1.877519in}}%
\pgfpathlineto{\pgfqpoint{2.993297in}{1.883148in}}%
\pgfpathlineto{\pgfqpoint{2.995875in}{1.858005in}}%
\pgfpathlineto{\pgfqpoint{2.998452in}{1.860444in}}%
\pgfpathlineto{\pgfqpoint{3.006184in}{1.883335in}}%
\pgfpathlineto{\pgfqpoint{3.008761in}{1.893843in}}%
\pgfpathlineto{\pgfqpoint{3.011338in}{1.868325in}}%
\pgfpathlineto{\pgfqpoint{3.013915in}{1.873391in}}%
\pgfpathlineto{\pgfqpoint{3.016492in}{1.865322in}}%
\pgfpathlineto{\pgfqpoint{3.024224in}{1.850312in}}%
\pgfpathlineto{\pgfqpoint{3.026801in}{1.851438in}}%
\pgfpathlineto{\pgfqpoint{3.029378in}{1.835301in}}%
\pgfpathlineto{\pgfqpoint{3.031956in}{1.832862in}}%
\pgfpathlineto{\pgfqpoint{3.034533in}{1.841681in}}%
\pgfpathlineto{\pgfqpoint{3.042264in}{1.860444in}}%
\pgfpathlineto{\pgfqpoint{3.044842in}{1.886525in}}%
\pgfpathlineto{\pgfqpoint{3.047419in}{1.888214in}}%
\pgfpathlineto{\pgfqpoint{3.049996in}{1.888589in}}%
\pgfpathlineto{\pgfqpoint{3.052573in}{1.866636in}}%
\pgfpathlineto{\pgfqpoint{3.060305in}{1.844120in}}%
\pgfpathlineto{\pgfqpoint{3.062882in}{1.851813in}}%
\pgfpathlineto{\pgfqpoint{3.065459in}{1.867386in}}%
\pgfpathlineto{\pgfqpoint{3.068037in}{1.815975in}}%
\pgfpathlineto{\pgfqpoint{3.070614in}{1.807344in}}%
\pgfpathlineto{\pgfqpoint{3.078345in}{1.817851in}}%
\pgfpathlineto{\pgfqpoint{3.080923in}{1.818602in}}%
\pgfpathlineto{\pgfqpoint{3.086077in}{1.859131in}}%
\pgfpathlineto{\pgfqpoint{3.096386in}{1.843745in}}%
\pgfpathlineto{\pgfqpoint{3.098963in}{1.849749in}}%
\pgfpathlineto{\pgfqpoint{3.104118in}{1.852188in}}%
\pgfpathlineto{\pgfqpoint{3.106695in}{1.828546in}}%
\pgfpathlineto{\pgfqpoint{3.114426in}{1.820666in}}%
\pgfpathlineto{\pgfqpoint{3.119581in}{1.847873in}}%
\pgfpathlineto{\pgfqpoint{3.122158in}{1.851625in}}%
\pgfpathlineto{\pgfqpoint{3.124735in}{1.864572in}}%
\pgfpathlineto{\pgfqpoint{3.132467in}{1.880521in}}%
\pgfpathlineto{\pgfqpoint{3.135044in}{1.877143in}}%
\pgfpathlineto{\pgfqpoint{3.137621in}{1.864197in}}%
\pgfpathlineto{\pgfqpoint{3.140199in}{1.887088in}}%
\pgfpathlineto{\pgfqpoint{3.142776in}{1.893092in}}%
\pgfpathlineto{\pgfqpoint{3.150507in}{1.901348in}}%
\pgfpathlineto{\pgfqpoint{3.153085in}{1.895344in}}%
\pgfpathlineto{\pgfqpoint{3.155662in}{1.875267in}}%
\pgfpathlineto{\pgfqpoint{3.158239in}{1.861945in}}%
\pgfpathlineto{\pgfqpoint{3.168548in}{1.877519in}}%
\pgfpathlineto{\pgfqpoint{3.171125in}{1.878269in}}%
\pgfpathlineto{\pgfqpoint{3.173702in}{1.897220in}}%
\pgfpathlineto{\pgfqpoint{3.176280in}{1.900035in}}%
\pgfpathlineto{\pgfqpoint{3.178857in}{1.916922in}}%
\pgfpathlineto{\pgfqpoint{3.189166in}{1.924239in}}%
\pgfpathlineto{\pgfqpoint{3.191743in}{1.921800in}}%
\pgfpathlineto{\pgfqpoint{3.194320in}{1.929306in}}%
\pgfpathlineto{\pgfqpoint{3.196897in}{1.928930in}}%
\pgfpathlineto{\pgfqpoint{3.204629in}{1.933433in}}%
\pgfpathlineto{\pgfqpoint{3.207206in}{1.926679in}}%
\pgfpathlineto{\pgfqpoint{3.209783in}{1.932871in}}%
\pgfpathlineto{\pgfqpoint{3.212361in}{1.942252in}}%
\pgfpathlineto{\pgfqpoint{3.214938in}{1.939250in}}%
\pgfpathlineto{\pgfqpoint{3.222669in}{1.954448in}}%
\pgfpathlineto{\pgfqpoint{3.225247in}{1.941689in}}%
\pgfpathlineto{\pgfqpoint{3.227824in}{1.934184in}}%
\pgfpathlineto{\pgfqpoint{3.230401in}{1.907915in}}%
\pgfpathlineto{\pgfqpoint{3.232978in}{1.907915in}}%
\pgfpathlineto{\pgfqpoint{3.240710in}{1.916546in}}%
\pgfpathlineto{\pgfqpoint{3.243287in}{1.914107in}}%
\pgfpathlineto{\pgfqpoint{3.245864in}{1.921425in}}%
\pgfpathlineto{\pgfqpoint{3.248441in}{1.924802in}}%
\pgfpathlineto{\pgfqpoint{3.251019in}{1.908291in}}%
\pgfpathlineto{\pgfqpoint{3.258750in}{1.907165in}}%
\pgfpathlineto{\pgfqpoint{3.261328in}{1.905851in}}%
\pgfpathlineto{\pgfqpoint{3.263905in}{1.927054in}}%
\pgfpathlineto{\pgfqpoint{3.266482in}{1.936623in}}%
\pgfpathlineto{\pgfqpoint{3.269059in}{1.951259in}}%
\pgfpathlineto{\pgfqpoint{3.276791in}{1.958952in}}%
\pgfpathlineto{\pgfqpoint{3.279368in}{1.971711in}}%
\pgfpathlineto{\pgfqpoint{3.281945in}{1.971148in}}%
\pgfpathlineto{\pgfqpoint{3.284522in}{1.978090in}}%
\pgfpathlineto{\pgfqpoint{3.294831in}{1.973775in}}%
\pgfpathlineto{\pgfqpoint{3.297409in}{1.961016in}}%
\pgfpathlineto{\pgfqpoint{3.299986in}{1.984657in}}%
\pgfpathlineto{\pgfqpoint{3.302563in}{1.978466in}}%
\pgfpathlineto{\pgfqpoint{3.305140in}{1.979028in}}%
\pgfpathlineto{\pgfqpoint{3.312872in}{1.976777in}}%
\pgfpathlineto{\pgfqpoint{3.315449in}{1.966082in}}%
\pgfpathlineto{\pgfqpoint{3.318026in}{1.952197in}}%
\pgfpathlineto{\pgfqpoint{3.320603in}{1.946568in}}%
\pgfpathlineto{\pgfqpoint{3.323181in}{1.960265in}}%
\pgfpathlineto{\pgfqpoint{3.330912in}{1.958952in}}%
\pgfpathlineto{\pgfqpoint{3.333490in}{1.967207in}}%
\pgfpathlineto{\pgfqpoint{3.336067in}{1.964205in}}%
\pgfpathlineto{\pgfqpoint{3.338644in}{1.977340in}}%
\pgfpathlineto{\pgfqpoint{3.341221in}{1.967395in}}%
\pgfpathlineto{\pgfqpoint{3.348953in}{1.983532in}}%
\pgfpathlineto{\pgfqpoint{3.351530in}{1.967020in}}%
\pgfpathlineto{\pgfqpoint{3.354107in}{1.984470in}}%
\pgfpathlineto{\pgfqpoint{3.356684in}{1.961391in}}%
\pgfpathlineto{\pgfqpoint{3.359262in}{1.952760in}}%
\pgfpathlineto{\pgfqpoint{3.366993in}{1.985033in}}%
\pgfpathlineto{\pgfqpoint{3.369570in}{1.976589in}}%
\pgfpathlineto{\pgfqpoint{3.372148in}{1.973775in}}%
\pgfpathlineto{\pgfqpoint{3.374725in}{1.955011in}}%
\pgfpathlineto{\pgfqpoint{3.377302in}{1.978278in}}%
\pgfpathlineto{\pgfqpoint{3.385034in}{1.989536in}}%
\pgfpathlineto{\pgfqpoint{3.387611in}{1.984470in}}%
\pgfpathlineto{\pgfqpoint{3.390188in}{1.991225in}}%
\pgfpathlineto{\pgfqpoint{3.392765in}{2.008862in}}%
\pgfpathlineto{\pgfqpoint{3.395343in}{2.020495in}}%
\pgfpathlineto{\pgfqpoint{3.403074in}{2.032504in}}%
\pgfpathlineto{\pgfqpoint{3.405651in}{2.034568in}}%
\pgfpathlineto{\pgfqpoint{3.408229in}{2.030252in}}%
\pgfpathlineto{\pgfqpoint{3.410806in}{2.039634in}}%
\pgfpathlineto{\pgfqpoint{3.413383in}{2.041510in}}%
\pgfpathlineto{\pgfqpoint{3.421115in}{2.039634in}}%
\pgfpathlineto{\pgfqpoint{3.423692in}{2.033255in}}%
\pgfpathlineto{\pgfqpoint{3.426269in}{2.039446in}}%
\pgfpathlineto{\pgfqpoint{3.428846in}{2.037007in}}%
\pgfpathlineto{\pgfqpoint{3.431424in}{2.030815in}}%
\pgfpathlineto{\pgfqpoint{3.441732in}{2.046952in}}%
\pgfpathlineto{\pgfqpoint{3.444310in}{2.049203in}}%
\pgfpathlineto{\pgfqpoint{3.446887in}{2.035318in}}%
\pgfpathlineto{\pgfqpoint{3.449464in}{2.049203in}}%
\pgfpathlineto{\pgfqpoint{3.457196in}{2.042636in}}%
\pgfpathlineto{\pgfqpoint{3.459773in}{2.025937in}}%
\pgfpathlineto{\pgfqpoint{3.462350in}{2.024623in}}%
\pgfpathlineto{\pgfqpoint{3.464927in}{2.032504in}}%
\pgfpathlineto{\pgfqpoint{3.467505in}{2.027250in}}%
\pgfpathlineto{\pgfqpoint{3.475236in}{2.034380in}}%
\pgfpathlineto{\pgfqpoint{3.477813in}{2.037570in}}%
\pgfpathlineto{\pgfqpoint{3.480391in}{2.038696in}}%
\pgfpathlineto{\pgfqpoint{3.482968in}{2.038884in}}%
\pgfpathlineto{\pgfqpoint{3.485545in}{2.041510in}}%
\pgfpathlineto{\pgfqpoint{3.493277in}{2.020683in}}%
\pgfpathlineto{\pgfqpoint{3.495854in}{2.003608in}}%
\pgfpathlineto{\pgfqpoint{3.498431in}{2.023498in}}%
\pgfpathlineto{\pgfqpoint{3.501008in}{1.999480in}}%
\pgfpathlineto{\pgfqpoint{3.503586in}{2.011114in}}%
\pgfpathlineto{\pgfqpoint{3.511317in}{2.012615in}}%
\pgfpathlineto{\pgfqpoint{3.513894in}{2.016180in}}%
\pgfpathlineto{\pgfqpoint{3.516472in}{1.989348in}}%
\pgfpathlineto{\pgfqpoint{3.519049in}{1.977152in}}%
\pgfpathlineto{\pgfqpoint{3.521626in}{2.006048in}}%
\pgfpathlineto{\pgfqpoint{3.529358in}{2.007924in}}%
\pgfpathlineto{\pgfqpoint{3.531935in}{1.982781in}}%
\pgfpathlineto{\pgfqpoint{3.534512in}{2.000043in}}%
\pgfpathlineto{\pgfqpoint{3.537089in}{1.958389in}}%
\pgfpathlineto{\pgfqpoint{3.539667in}{1.968146in}}%
\pgfpathlineto{\pgfqpoint{3.547398in}{1.927242in}}%
\pgfpathlineto{\pgfqpoint{3.549975in}{1.931182in}}%
\pgfpathlineto{\pgfqpoint{3.552553in}{1.895344in}}%
\pgfpathlineto{\pgfqpoint{3.555130in}{1.889527in}}%
\pgfpathlineto{\pgfqpoint{3.557707in}{1.925740in}}%
\pgfpathlineto{\pgfqpoint{3.565439in}{1.955199in}}%
\pgfpathlineto{\pgfqpoint{3.568016in}{1.990286in}}%
\pgfpathlineto{\pgfqpoint{3.570593in}{1.982593in}}%
\pgfpathlineto{\pgfqpoint{3.573170in}{1.997979in}}%
\pgfpathlineto{\pgfqpoint{3.575747in}{2.008862in}}%
\pgfpathlineto{\pgfqpoint{3.583479in}{2.006048in}}%
\pgfpathlineto{\pgfqpoint{3.586056in}{2.031753in}}%
\pgfpathlineto{\pgfqpoint{3.588634in}{2.024811in}}%
\pgfpathlineto{\pgfqpoint{3.591211in}{2.036820in}}%
\pgfpathlineto{\pgfqpoint{3.593788in}{2.056896in}}%
\pgfpathlineto{\pgfqpoint{3.601520in}{2.062713in}}%
\pgfpathlineto{\pgfqpoint{3.604097in}{2.039071in}}%
\pgfpathlineto{\pgfqpoint{3.606674in}{2.050329in}}%
\pgfpathlineto{\pgfqpoint{3.609251in}{2.067779in}}%
\pgfpathlineto{\pgfqpoint{3.611828in}{2.032879in}}%
\pgfpathlineto{\pgfqpoint{3.619560in}{2.029127in}}%
\pgfpathlineto{\pgfqpoint{3.622137in}{2.032692in}}%
\pgfpathlineto{\pgfqpoint{3.624715in}{2.031191in}}%
\pgfpathlineto{\pgfqpoint{3.627292in}{2.041510in}}%
\pgfpathlineto{\pgfqpoint{3.629869in}{2.046952in}}%
\pgfpathlineto{\pgfqpoint{3.640178in}{2.037758in}}%
\pgfpathlineto{\pgfqpoint{3.642755in}{2.029877in}}%
\pgfpathlineto{\pgfqpoint{3.645332in}{2.013928in}}%
\pgfpathlineto{\pgfqpoint{3.647909in}{2.014866in}}%
\pgfpathlineto{\pgfqpoint{3.655641in}{2.043011in}}%
\pgfpathlineto{\pgfqpoint{3.658218in}{2.061587in}}%
\pgfpathlineto{\pgfqpoint{3.665950in}{2.076598in}}%
\pgfpathlineto{\pgfqpoint{3.673682in}{2.079788in}}%
\pgfpathlineto{\pgfqpoint{3.676259in}{2.093297in}}%
\pgfpathlineto{\pgfqpoint{3.678836in}{2.086918in}}%
\pgfpathlineto{\pgfqpoint{3.681413in}{2.089169in}}%
\pgfpathlineto{\pgfqpoint{3.683990in}{2.098363in}}%
\pgfpathlineto{\pgfqpoint{3.691722in}{2.098926in}}%
\pgfpathlineto{\pgfqpoint{3.694299in}{2.084103in}}%
\pgfpathlineto{\pgfqpoint{3.696876in}{2.061212in}}%
\pgfpathlineto{\pgfqpoint{3.699454in}{2.083540in}}%
\pgfpathlineto{\pgfqpoint{3.702031in}{2.079037in}}%
\pgfpathlineto{\pgfqpoint{3.709763in}{2.068717in}}%
\pgfpathlineto{\pgfqpoint{3.712340in}{2.055771in}}%
\pgfpathlineto{\pgfqpoint{3.717494in}{2.098739in}}%
\pgfpathlineto{\pgfqpoint{3.720071in}{2.103617in}}%
\pgfpathlineto{\pgfqpoint{3.730380in}{2.135327in}}%
\pgfpathlineto{\pgfqpoint{3.732957in}{2.130636in}}%
\pgfpathlineto{\pgfqpoint{3.738112in}{2.141331in}}%
\pgfpathlineto{\pgfqpoint{3.745843in}{2.149587in}}%
\pgfpathlineto{\pgfqpoint{3.748421in}{2.135515in}}%
\pgfpathlineto{\pgfqpoint{3.750998in}{2.126508in}}%
\pgfpathlineto{\pgfqpoint{3.756152in}{2.118815in}}%
\pgfpathlineto{\pgfqpoint{3.766461in}{2.086167in}}%
\pgfpathlineto{\pgfqpoint{3.771616in}{2.119566in}}%
\pgfpathlineto{\pgfqpoint{3.774193in}{2.127634in}}%
\pgfpathlineto{\pgfqpoint{3.781924in}{2.131387in}}%
\pgfpathlineto{\pgfqpoint{3.784502in}{2.144146in}}%
\pgfpathlineto{\pgfqpoint{3.787079in}{2.127259in}}%
\pgfpathlineto{\pgfqpoint{3.789656in}{2.129323in}}%
\pgfpathlineto{\pgfqpoint{3.792233in}{2.143958in}}%
\pgfpathlineto{\pgfqpoint{3.802542in}{2.136265in}}%
\pgfpathlineto{\pgfqpoint{3.805119in}{2.126133in}}%
\pgfpathlineto{\pgfqpoint{3.807697in}{2.143395in}}%
\pgfpathlineto{\pgfqpoint{3.810274in}{2.135890in}}%
\pgfpathlineto{\pgfqpoint{3.818005in}{2.140206in}}%
\pgfpathlineto{\pgfqpoint{3.820583in}{2.122756in}}%
\pgfpathlineto{\pgfqpoint{3.823160in}{2.099864in}}%
\pgfpathlineto{\pgfqpoint{3.825737in}{2.109434in}}%
\pgfpathlineto{\pgfqpoint{3.828314in}{2.069655in}}%
\pgfpathlineto{\pgfqpoint{3.836046in}{2.086730in}}%
\pgfpathlineto{\pgfqpoint{3.838623in}{2.125007in}}%
\pgfpathlineto{\pgfqpoint{3.841200in}{2.251285in}}%
\pgfpathlineto{\pgfqpoint{3.843778in}{2.275114in}}%
\pgfpathlineto{\pgfqpoint{3.846355in}{2.264231in}}%
\pgfpathlineto{\pgfqpoint{3.854086in}{2.259165in}}%
\pgfpathlineto{\pgfqpoint{3.856664in}{2.262543in}}%
\pgfpathlineto{\pgfqpoint{3.859241in}{2.261605in}}%
\pgfpathlineto{\pgfqpoint{3.861818in}{2.291814in}}%
\pgfpathlineto{\pgfqpoint{3.864395in}{2.302133in}}%
\pgfpathlineto{\pgfqpoint{3.874704in}{2.301195in}}%
\pgfpathlineto{\pgfqpoint{3.877281in}{2.296692in}}%
\pgfpathlineto{\pgfqpoint{3.879859in}{2.297255in}}%
\pgfpathlineto{\pgfqpoint{3.882436in}{2.308888in}}%
\pgfpathlineto{\pgfqpoint{3.890167in}{2.316581in}}%
\pgfpathlineto{\pgfqpoint{3.892745in}{2.310952in}}%
\pgfpathlineto{\pgfqpoint{3.895322in}{2.326713in}}%
\pgfpathlineto{\pgfqpoint{3.897899in}{2.309076in}}%
\pgfpathlineto{\pgfqpoint{3.900476in}{2.300632in}}%
\pgfpathlineto{\pgfqpoint{3.908208in}{2.332342in}}%
\pgfpathlineto{\pgfqpoint{3.910785in}{2.340411in}}%
\pgfpathlineto{\pgfqpoint{3.913362in}{2.326713in}}%
\pgfpathlineto{\pgfqpoint{3.915940in}{2.317332in}}%
\pgfpathlineto{\pgfqpoint{3.918517in}{2.295942in}}%
\pgfpathlineto{\pgfqpoint{3.926248in}{2.321272in}}%
\pgfpathlineto{\pgfqpoint{3.928826in}{2.283182in}}%
\pgfpathlineto{\pgfqpoint{3.931403in}{2.279617in}}%
\pgfpathlineto{\pgfqpoint{3.933980in}{2.354858in}}%
\pgfpathlineto{\pgfqpoint{3.936557in}{2.342099in}}%
\pgfpathlineto{\pgfqpoint{3.944289in}{2.358424in}}%
\pgfpathlineto{\pgfqpoint{3.946866in}{2.351293in}}%
\pgfpathlineto{\pgfqpoint{3.949443in}{2.368931in}}%
\pgfpathlineto{\pgfqpoint{3.952020in}{2.358424in}}%
\pgfpathlineto{\pgfqpoint{3.954598in}{2.376999in}}%
\pgfpathlineto{\pgfqpoint{3.962329in}{2.373434in}}%
\pgfpathlineto{\pgfqpoint{3.964907in}{2.353920in}}%
\pgfpathlineto{\pgfqpoint{3.967484in}{2.316769in}}%
\pgfpathlineto{\pgfqpoint{3.980370in}{2.336470in}}%
\pgfpathlineto{\pgfqpoint{3.982947in}{2.314705in}}%
\pgfpathlineto{\pgfqpoint{3.988101in}{2.334406in}}%
\pgfpathlineto{\pgfqpoint{3.998410in}{2.327839in}}%
\pgfpathlineto{\pgfqpoint{4.000988in}{2.324274in}}%
\pgfpathlineto{\pgfqpoint{4.003565in}{2.340598in}}%
\pgfpathlineto{\pgfqpoint{4.006142in}{2.347916in}}%
\pgfpathlineto{\pgfqpoint{4.008719in}{2.351106in}}%
\pgfpathlineto{\pgfqpoint{4.016451in}{2.343225in}}%
\pgfpathlineto{\pgfqpoint{4.019028in}{2.345852in}}%
\pgfpathlineto{\pgfqpoint{4.021605in}{2.351481in}}%
\pgfpathlineto{\pgfqpoint{4.024182in}{2.371183in}}%
\pgfpathlineto{\pgfqpoint{4.026760in}{2.346415in}}%
\pgfpathlineto{\pgfqpoint{4.034491in}{2.373434in}}%
\pgfpathlineto{\pgfqpoint{4.037069in}{2.363865in}}%
\pgfpathlineto{\pgfqpoint{4.039646in}{2.368368in}}%
\pgfpathlineto{\pgfqpoint{4.042223in}{2.387131in}}%
\pgfpathlineto{\pgfqpoint{4.044800in}{2.396513in}}%
\pgfpathlineto{\pgfqpoint{4.052532in}{2.407584in}}%
\pgfpathlineto{\pgfqpoint{4.055109in}{2.403268in}}%
\pgfpathlineto{\pgfqpoint{4.057686in}{2.401392in}}%
\pgfpathlineto{\pgfqpoint{4.060263in}{2.382253in}}%
\pgfpathlineto{\pgfqpoint{4.062841in}{2.413775in}}%
\pgfpathlineto{\pgfqpoint{4.070572in}{2.422782in}}%
\pgfpathlineto{\pgfqpoint{4.073149in}{2.419029in}}%
\pgfpathlineto{\pgfqpoint{4.075727in}{2.399703in}}%
\pgfpathlineto{\pgfqpoint{4.078304in}{2.391635in}}%
\pgfpathlineto{\pgfqpoint{4.080881in}{2.406645in}}%
\pgfpathlineto{\pgfqpoint{4.088613in}{2.380001in}}%
\pgfpathlineto{\pgfqpoint{4.091190in}{2.391259in}}%
\pgfpathlineto{\pgfqpoint{4.093767in}{2.390509in}}%
\pgfpathlineto{\pgfqpoint{4.096344in}{2.403456in}}%
\pgfpathlineto{\pgfqpoint{4.098922in}{2.410023in}}%
\pgfpathlineto{\pgfqpoint{4.106653in}{2.410586in}}%
\pgfpathlineto{\pgfqpoint{4.109230in}{2.414526in}}%
\pgfpathlineto{\pgfqpoint{4.111808in}{2.408146in}}%
\pgfpathlineto{\pgfqpoint{4.114385in}{2.411524in}}%
\pgfpathlineto{\pgfqpoint{4.116962in}{2.409272in}}%
\pgfpathlineto{\pgfqpoint{4.127271in}{2.394824in}}%
\pgfpathlineto{\pgfqpoint{4.129848in}{2.411149in}}%
\pgfpathlineto{\pgfqpoint{4.132425in}{2.413963in}}%
\pgfpathlineto{\pgfqpoint{4.135003in}{2.411149in}}%
\pgfpathlineto{\pgfqpoint{4.142734in}{2.421656in}}%
\pgfpathlineto{\pgfqpoint{4.145311in}{2.417903in}}%
\pgfpathlineto{\pgfqpoint{4.147889in}{2.425221in}}%
\pgfpathlineto{\pgfqpoint{4.150466in}{2.410023in}}%
\pgfpathlineto{\pgfqpoint{4.153043in}{2.410023in}}%
\pgfpathlineto{\pgfqpoint{4.160775in}{2.392198in}}%
\pgfpathlineto{\pgfqpoint{4.163352in}{2.378688in}}%
\pgfpathlineto{\pgfqpoint{4.165929in}{2.404769in}}%
\pgfpathlineto{\pgfqpoint{4.168506in}{2.415652in}}%
\pgfpathlineto{\pgfqpoint{4.171084in}{2.403831in}}%
\pgfpathlineto{\pgfqpoint{4.178815in}{2.407771in}}%
\pgfpathlineto{\pgfqpoint{4.181392in}{2.423345in}}%
\pgfpathlineto{\pgfqpoint{4.183970in}{2.430850in}}%
\pgfpathlineto{\pgfqpoint{4.186547in}{2.461247in}}%
\pgfpathlineto{\pgfqpoint{4.189124in}{2.450739in}}%
\pgfpathlineto{\pgfqpoint{4.196856in}{2.466876in}}%
\pgfpathlineto{\pgfqpoint{4.199433in}{2.482262in}}%
\pgfpathlineto{\pgfqpoint{4.202010in}{2.471004in}}%
\pgfpathlineto{\pgfqpoint{4.207165in}{2.492394in}}%
\pgfpathlineto{\pgfqpoint{4.214896in}{2.458432in}}%
\pgfpathlineto{\pgfqpoint{4.217473in}{2.477571in}}%
\pgfpathlineto{\pgfqpoint{4.220051in}{2.506654in}}%
\pgfpathlineto{\pgfqpoint{4.222628in}{2.503840in}}%
\pgfpathlineto{\pgfqpoint{4.232937in}{2.516786in}}%
\pgfpathlineto{\pgfqpoint{4.235514in}{2.541554in}}%
\pgfpathlineto{\pgfqpoint{4.238091in}{2.507780in}}%
\pgfpathlineto{\pgfqpoint{4.240668in}{2.515098in}}%
\pgfpathlineto{\pgfqpoint{4.243246in}{2.529921in}}%
\pgfpathlineto{\pgfqpoint{4.250977in}{2.558441in}}%
\pgfpathlineto{\pgfqpoint{4.253554in}{2.554876in}}%
\pgfpathlineto{\pgfqpoint{4.256132in}{2.562756in}}%
\pgfpathlineto{\pgfqpoint{4.258709in}{2.576454in}}%
\pgfpathlineto{\pgfqpoint{4.261286in}{2.572701in}}%
\pgfpathlineto{\pgfqpoint{4.269018in}{2.585460in}}%
\pgfpathlineto{\pgfqpoint{4.271595in}{2.580769in}}%
\pgfpathlineto{\pgfqpoint{4.274172in}{2.580957in}}%
\pgfpathlineto{\pgfqpoint{4.276749in}{2.571575in}}%
\pgfpathlineto{\pgfqpoint{4.279326in}{2.573639in}}%
\pgfpathlineto{\pgfqpoint{4.287058in}{2.562006in}}%
\pgfpathlineto{\pgfqpoint{4.289635in}{2.565571in}}%
\pgfpathlineto{\pgfqpoint{4.292213in}{2.590151in}}%
\pgfpathlineto{\pgfqpoint{4.294790in}{2.593341in}}%
\pgfpathlineto{\pgfqpoint{4.297367in}{2.592965in}}%
\pgfpathlineto{\pgfqpoint{4.305099in}{2.612667in}}%
\pgfpathlineto{\pgfqpoint{4.307676in}{2.622799in}}%
\pgfpathlineto{\pgfqpoint{4.310253in}{2.425409in}}%
\pgfpathlineto{\pgfqpoint{4.312830in}{2.390321in}}%
\pgfpathlineto{\pgfqpoint{4.315407in}{2.404394in}}%
\pgfpathlineto{\pgfqpoint{4.323139in}{2.433665in}}%
\pgfpathlineto{\pgfqpoint{4.325716in}{2.380564in}}%
\pgfpathlineto{\pgfqpoint{4.328294in}{2.362739in}}%
\pgfpathlineto{\pgfqpoint{4.330871in}{2.372121in}}%
\pgfpathlineto{\pgfqpoint{4.333448in}{2.365741in}}%
\pgfpathlineto{\pgfqpoint{4.341180in}{2.399140in}}%
\pgfpathlineto{\pgfqpoint{4.343757in}{2.361801in}}%
\pgfpathlineto{\pgfqpoint{4.346334in}{2.353170in}}%
\pgfpathlineto{\pgfqpoint{4.348911in}{2.239276in}}%
\pgfpathlineto{\pgfqpoint{4.359220in}{2.156905in}}%
\pgfpathlineto{\pgfqpoint{4.361797in}{2.166287in}}%
\pgfpathlineto{\pgfqpoint{4.366952in}{2.277366in}}%
\pgfpathlineto{\pgfqpoint{4.369529in}{2.282807in}}%
\pgfpathlineto{\pgfqpoint{4.377261in}{2.272300in}}%
\pgfpathlineto{\pgfqpoint{4.379838in}{2.230270in}}%
\pgfpathlineto{\pgfqpoint{4.382415in}{2.272487in}}%
\pgfpathlineto{\pgfqpoint{4.384992in}{2.274176in}}%
\pgfpathlineto{\pgfqpoint{4.387569in}{2.256163in}}%
\pgfpathlineto{\pgfqpoint{4.397878in}{2.310014in}}%
\pgfpathlineto{\pgfqpoint{4.400455in}{2.272863in}}%
\pgfpathlineto{\pgfqpoint{4.403033in}{2.285059in}}%
\pgfpathlineto{\pgfqpoint{4.405610in}{2.318270in}}%
\pgfpathlineto{\pgfqpoint{4.413342in}{2.306637in}}%
\pgfpathlineto{\pgfqpoint{4.415919in}{2.299694in}}%
\pgfpathlineto{\pgfqpoint{4.418496in}{2.309076in}}%
\pgfpathlineto{\pgfqpoint{4.421073in}{2.313391in}}%
\pgfpathlineto{\pgfqpoint{4.423650in}{2.289187in}}%
\pgfpathlineto{\pgfqpoint{4.431382in}{2.299319in}}%
\pgfpathlineto{\pgfqpoint{4.439114in}{2.249971in}}%
\pgfpathlineto{\pgfqpoint{4.441691in}{2.244342in}}%
\pgfpathlineto{\pgfqpoint{4.449422in}{2.212257in}}%
\pgfpathlineto{\pgfqpoint{4.452000in}{2.228769in}}%
\pgfpathlineto{\pgfqpoint{4.454577in}{2.277929in}}%
\pgfpathlineto{\pgfqpoint{4.457154in}{2.286185in}}%
\pgfpathlineto{\pgfqpoint{4.459731in}{2.292001in}}%
\pgfpathlineto{\pgfqpoint{4.467463in}{2.307200in}}%
\pgfpathlineto{\pgfqpoint{4.470040in}{2.305698in}}%
\pgfpathlineto{\pgfqpoint{4.472617in}{2.298944in}}%
\pgfpathlineto{\pgfqpoint{4.475195in}{2.320522in}}%
\pgfpathlineto{\pgfqpoint{4.477772in}{2.337409in}}%
\pgfpathlineto{\pgfqpoint{4.488081in}{2.355609in}}%
\pgfpathlineto{\pgfqpoint{4.490658in}{2.340411in}}%
\pgfpathlineto{\pgfqpoint{4.493235in}{2.378688in}}%
\pgfpathlineto{\pgfqpoint{4.495812in}{2.384880in}}%
\pgfpathlineto{\pgfqpoint{4.506121in}{2.413025in}}%
\pgfpathlineto{\pgfqpoint{4.508698in}{2.417528in}}%
\pgfpathlineto{\pgfqpoint{4.511276in}{2.473443in}}%
\pgfpathlineto{\pgfqpoint{4.513853in}{2.470628in}}%
\pgfpathlineto{\pgfqpoint{4.521584in}{2.478321in}}%
\pgfpathlineto{\pgfqpoint{4.524162in}{2.482637in}}%
\pgfpathlineto{\pgfqpoint{4.529316in}{2.505153in}}%
\pgfpathlineto{\pgfqpoint{4.531893in}{2.482074in}}%
\pgfpathlineto{\pgfqpoint{4.542202in}{2.513972in}}%
\pgfpathlineto{\pgfqpoint{4.544779in}{2.473443in}}%
\pgfpathlineto{\pgfqpoint{4.547357in}{2.468940in}}%
\pgfpathlineto{\pgfqpoint{4.549934in}{2.516223in}}%
\pgfpathlineto{\pgfqpoint{4.557665in}{2.529545in}}%
\pgfpathlineto{\pgfqpoint{4.560243in}{2.547183in}}%
\pgfpathlineto{\pgfqpoint{4.562820in}{2.531234in}}%
\pgfpathlineto{\pgfqpoint{4.565397in}{2.525793in}}%
\pgfpathlineto{\pgfqpoint{4.567974in}{2.501588in}}%
\pgfpathlineto{\pgfqpoint{4.575706in}{2.520727in}}%
\pgfpathlineto{\pgfqpoint{4.578283in}{2.524479in}}%
\pgfpathlineto{\pgfqpoint{4.580860in}{2.559942in}}%
\pgfpathlineto{\pgfqpoint{4.583438in}{2.570074in}}%
\pgfpathlineto{\pgfqpoint{4.586015in}{2.594091in}}%
\pgfpathlineto{\pgfqpoint{4.593746in}{2.582646in}}%
\pgfpathlineto{\pgfqpoint{4.596324in}{2.556565in}}%
\pgfpathlineto{\pgfqpoint{4.598901in}{2.569324in}}%
\pgfpathlineto{\pgfqpoint{4.604055in}{2.506654in}}%
\pgfpathlineto{\pgfqpoint{4.611787in}{2.477383in}}%
\pgfpathlineto{\pgfqpoint{4.614364in}{2.511345in}}%
\pgfpathlineto{\pgfqpoint{4.616941in}{2.486765in}}%
\pgfpathlineto{\pgfqpoint{4.619519in}{2.449426in}}%
\pgfpathlineto{\pgfqpoint{4.622096in}{2.490893in}}%
\pgfpathlineto{\pgfqpoint{4.629827in}{2.483763in}}%
\pgfpathlineto{\pgfqpoint{4.632405in}{2.459746in}}%
\pgfpathlineto{\pgfqpoint{4.634982in}{2.441920in}}%
\pgfpathlineto{\pgfqpoint{4.637559in}{2.441920in}}%
\pgfpathlineto{\pgfqpoint{4.640136in}{2.393511in}}%
\pgfpathlineto{\pgfqpoint{4.647868in}{2.416778in}}%
\pgfpathlineto{\pgfqpoint{4.650445in}{2.466876in}}%
\pgfpathlineto{\pgfqpoint{4.653022in}{2.495959in}}%
\pgfpathlineto{\pgfqpoint{4.655599in}{2.464249in}}%
\pgfpathlineto{\pgfqpoint{4.658177in}{2.387882in}}%
\pgfpathlineto{\pgfqpoint{4.665908in}{2.367618in}}%
\pgfpathlineto{\pgfqpoint{4.668486in}{2.370432in}}%
\pgfpathlineto{\pgfqpoint{4.671063in}{2.349417in}}%
\pgfpathlineto{\pgfqpoint{4.673640in}{2.354671in}}%
\pgfpathlineto{\pgfqpoint{4.683949in}{2.379438in}}%
\pgfpathlineto{\pgfqpoint{4.686526in}{2.376436in}}%
\pgfpathlineto{\pgfqpoint{4.689103in}{2.363302in}}%
\pgfpathlineto{\pgfqpoint{4.691680in}{2.340786in}}%
\pgfpathlineto{\pgfqpoint{4.701989in}{2.303447in}}%
\pgfpathlineto{\pgfqpoint{4.704567in}{2.266295in}}%
\pgfpathlineto{\pgfqpoint{4.707144in}{2.256726in}}%
\pgfpathlineto{\pgfqpoint{4.709721in}{2.241528in}}%
\pgfpathlineto{\pgfqpoint{4.712298in}{2.237025in}}%
\pgfpathlineto{\pgfqpoint{4.720030in}{2.248845in}}%
\pgfpathlineto{\pgfqpoint{4.722607in}{2.276428in}}%
\pgfpathlineto{\pgfqpoint{4.725184in}{2.223327in}}%
\pgfpathlineto{\pgfqpoint{4.727761in}{2.234398in}}%
\pgfpathlineto{\pgfqpoint{4.730339in}{2.141707in}}%
\pgfpathlineto{\pgfqpoint{4.740648in}{2.143020in}}%
\pgfpathlineto{\pgfqpoint{4.743225in}{2.117502in}}%
\pgfpathlineto{\pgfqpoint{4.745802in}{2.143771in}}%
\pgfpathlineto{\pgfqpoint{4.748379in}{2.195182in}}%
\pgfpathlineto{\pgfqpoint{4.756111in}{2.166474in}}%
\pgfpathlineto{\pgfqpoint{4.758688in}{2.183924in}}%
\pgfpathlineto{\pgfqpoint{4.761265in}{2.149212in}}%
\pgfpathlineto{\pgfqpoint{4.763842in}{2.135140in}}%
\pgfpathlineto{\pgfqpoint{4.766420in}{2.175856in}}%
\pgfpathlineto{\pgfqpoint{4.774151in}{2.164035in}}%
\pgfpathlineto{\pgfqpoint{4.776728in}{2.127822in}}%
\pgfpathlineto{\pgfqpoint{4.779306in}{2.163847in}}%
\pgfpathlineto{\pgfqpoint{4.781883in}{2.168914in}}%
\pgfpathlineto{\pgfqpoint{4.784460in}{2.141707in}}%
\pgfpathlineto{\pgfqpoint{4.792192in}{2.109997in}}%
\pgfpathlineto{\pgfqpoint{4.794769in}{2.113562in}}%
\pgfpathlineto{\pgfqpoint{4.797346in}{2.051830in}}%
\pgfpathlineto{\pgfqpoint{4.799923in}{2.077724in}}%
\pgfpathlineto{\pgfqpoint{4.802501in}{2.092734in}}%
\pgfpathlineto{\pgfqpoint{4.812809in}{2.124069in}}%
\pgfpathlineto{\pgfqpoint{4.815387in}{2.170227in}}%
\pgfpathlineto{\pgfqpoint{4.817964in}{2.164410in}}%
\pgfpathlineto{\pgfqpoint{4.820541in}{2.161408in}}%
\pgfpathlineto{\pgfqpoint{4.828273in}{2.185613in}}%
\pgfpathlineto{\pgfqpoint{4.830850in}{2.167975in}}%
\pgfpathlineto{\pgfqpoint{4.833427in}{2.168914in}}%
\pgfpathlineto{\pgfqpoint{4.836004in}{2.172854in}}%
\pgfpathlineto{\pgfqpoint{4.838582in}{2.166850in}}%
\pgfpathlineto{\pgfqpoint{4.846313in}{2.170602in}}%
\pgfpathlineto{\pgfqpoint{4.848890in}{2.208504in}}%
\pgfpathlineto{\pgfqpoint{4.851468in}{2.196871in}}%
\pgfpathlineto{\pgfqpoint{4.854045in}{2.229332in}}%
\pgfpathlineto{\pgfqpoint{4.856622in}{2.223327in}}%
\pgfpathlineto{\pgfqpoint{4.864354in}{2.239464in}}%
\pgfpathlineto{\pgfqpoint{4.866931in}{2.211506in}}%
\pgfpathlineto{\pgfqpoint{4.869508in}{2.208692in}}%
\pgfpathlineto{\pgfqpoint{4.872085in}{2.197622in}}%
\pgfpathlineto{\pgfqpoint{4.874663in}{2.213570in}}%
\pgfpathlineto{\pgfqpoint{4.882394in}{2.229144in}}%
\pgfpathlineto{\pgfqpoint{4.884971in}{2.219012in}}%
\pgfpathlineto{\pgfqpoint{4.887549in}{2.222577in}}%
\pgfpathlineto{\pgfqpoint{4.890126in}{2.243216in}}%
\pgfpathlineto{\pgfqpoint{4.892703in}{2.236086in}}%
\pgfpathlineto{\pgfqpoint{4.900435in}{2.222952in}}%
\pgfpathlineto{\pgfqpoint{4.905589in}{2.193869in}}%
\pgfpathlineto{\pgfqpoint{4.908166in}{2.200811in}}%
\pgfpathlineto{\pgfqpoint{4.918475in}{2.216385in}}%
\pgfpathlineto{\pgfqpoint{4.921052in}{2.217511in}}%
\pgfpathlineto{\pgfqpoint{4.923630in}{2.231020in}}%
\pgfpathlineto{\pgfqpoint{4.926207in}{2.237963in}}%
\pgfpathlineto{\pgfqpoint{4.928784in}{2.233835in}}%
\pgfpathlineto{\pgfqpoint{4.936516in}{2.226892in}}%
\pgfpathlineto{\pgfqpoint{4.939093in}{2.196871in}}%
\pgfpathlineto{\pgfqpoint{4.941670in}{2.205502in}}%
\pgfpathlineto{\pgfqpoint{4.944247in}{2.182048in}}%
\pgfpathlineto{\pgfqpoint{4.946825in}{2.186551in}}%
\pgfpathlineto{\pgfqpoint{4.954556in}{2.183924in}}%
\pgfpathlineto{\pgfqpoint{4.957133in}{2.203063in}}%
\pgfpathlineto{\pgfqpoint{4.959711in}{2.241153in}}%
\pgfpathlineto{\pgfqpoint{4.962288in}{2.225954in}}%
\pgfpathlineto{\pgfqpoint{4.964865in}{2.225204in}}%
\pgfpathlineto{\pgfqpoint{4.977751in}{2.308513in}}%
\pgfpathlineto{\pgfqpoint{4.980328in}{2.302133in}}%
\pgfpathlineto{\pgfqpoint{4.982905in}{2.317519in}}%
\pgfpathlineto{\pgfqpoint{4.993214in}{2.337409in}}%
\pgfpathlineto{\pgfqpoint{4.995792in}{2.344351in}}%
\pgfpathlineto{\pgfqpoint{4.998369in}{2.322023in}}%
\pgfpathlineto{\pgfqpoint{5.000946in}{2.308325in}}%
\pgfpathlineto{\pgfqpoint{5.008678in}{2.328027in}}%
\pgfpathlineto{\pgfqpoint{5.011255in}{2.317707in}}%
\pgfpathlineto{\pgfqpoint{5.013832in}{2.315643in}}%
\pgfpathlineto{\pgfqpoint{5.016409in}{2.338159in}}%
\pgfpathlineto{\pgfqpoint{5.018986in}{2.349042in}}%
\pgfpathlineto{\pgfqpoint{5.026718in}{2.345477in}}%
\pgfpathlineto{\pgfqpoint{5.029295in}{2.367805in}}%
\pgfpathlineto{\pgfqpoint{5.031873in}{2.291063in}}%
\pgfpathlineto{\pgfqpoint{5.034450in}{2.280743in}}%
\pgfpathlineto{\pgfqpoint{5.037027in}{2.259541in}}%
\pgfpathlineto{\pgfqpoint{5.044759in}{2.256726in}}%
\pgfpathlineto{\pgfqpoint{5.047336in}{2.249221in}}%
\pgfpathlineto{\pgfqpoint{5.049913in}{2.232521in}}%
\pgfpathlineto{\pgfqpoint{5.052490in}{2.222014in}}%
\pgfpathlineto{\pgfqpoint{5.055067in}{2.246406in}}%
\pgfpathlineto{\pgfqpoint{5.062799in}{2.235711in}}%
\pgfpathlineto{\pgfqpoint{5.067953in}{2.247907in}}%
\pgfpathlineto{\pgfqpoint{5.070531in}{2.246969in}}%
\pgfpathlineto{\pgfqpoint{5.073108in}{2.255413in}}%
\pgfpathlineto{\pgfqpoint{5.083417in}{2.236462in}}%
\pgfpathlineto{\pgfqpoint{5.085994in}{2.223890in}}%
\pgfpathlineto{\pgfqpoint{5.088571in}{2.227455in}}%
\pgfpathlineto{\pgfqpoint{5.091148in}{2.228018in}}%
\pgfpathlineto{\pgfqpoint{5.098880in}{2.228581in}}%
\pgfpathlineto{\pgfqpoint{5.101457in}{2.220888in}}%
\pgfpathlineto{\pgfqpoint{5.104034in}{2.215447in}}%
\pgfpathlineto{\pgfqpoint{5.106612in}{2.211882in}}%
\pgfpathlineto{\pgfqpoint{5.109189in}{2.202875in}}%
\pgfpathlineto{\pgfqpoint{5.116921in}{2.207003in}}%
\pgfpathlineto{\pgfqpoint{5.119498in}{2.221826in}}%
\pgfpathlineto{\pgfqpoint{5.122075in}{2.219575in}}%
\pgfpathlineto{\pgfqpoint{5.124652in}{2.221451in}}%
\pgfpathlineto{\pgfqpoint{5.127229in}{2.232521in}}%
\pgfpathlineto{\pgfqpoint{5.134961in}{2.242654in}}%
\pgfpathlineto{\pgfqpoint{5.137538in}{2.229332in}}%
\pgfpathlineto{\pgfqpoint{5.140115in}{2.228769in}}%
\pgfpathlineto{\pgfqpoint{5.142693in}{2.232897in}}%
\pgfpathlineto{\pgfqpoint{5.145270in}{2.174167in}}%
\pgfpathlineto{\pgfqpoint{5.153001in}{2.150338in}}%
\pgfpathlineto{\pgfqpoint{5.155579in}{2.179984in}}%
\pgfpathlineto{\pgfqpoint{5.160733in}{2.211506in}}%
\pgfpathlineto{\pgfqpoint{5.163310in}{2.215259in}}%
\pgfpathlineto{\pgfqpoint{5.173619in}{2.208692in}}%
\pgfpathlineto{\pgfqpoint{5.178774in}{2.234773in}}%
\pgfpathlineto{\pgfqpoint{5.181351in}{2.256538in}}%
\pgfpathlineto{\pgfqpoint{5.189082in}{2.262918in}}%
\pgfpathlineto{\pgfqpoint{5.191660in}{2.266858in}}%
\pgfpathlineto{\pgfqpoint{5.194237in}{2.261042in}}%
\pgfpathlineto{\pgfqpoint{5.196814in}{2.262730in}}%
\pgfpathlineto{\pgfqpoint{5.199391in}{2.259728in}}%
\pgfpathlineto{\pgfqpoint{5.207123in}{2.265920in}}%
\pgfpathlineto{\pgfqpoint{5.209700in}{2.253724in}}%
\pgfpathlineto{\pgfqpoint{5.212277in}{2.231396in}}%
\pgfpathlineto{\pgfqpoint{5.214855in}{2.227643in}}%
\pgfpathlineto{\pgfqpoint{5.217432in}{2.222202in}}%
\pgfpathlineto{\pgfqpoint{5.225163in}{2.216385in}}%
\pgfpathlineto{\pgfqpoint{5.227741in}{2.203813in}}%
\pgfpathlineto{\pgfqpoint{5.232895in}{2.189929in}}%
\pgfpathlineto{\pgfqpoint{5.235472in}{2.190679in}}%
\pgfpathlineto{\pgfqpoint{5.243204in}{2.183361in}}%
\pgfpathlineto{\pgfqpoint{5.245781in}{2.173792in}}%
\pgfpathlineto{\pgfqpoint{5.248358in}{2.193118in}}%
\pgfpathlineto{\pgfqpoint{5.250936in}{2.176419in}}%
\pgfpathlineto{\pgfqpoint{5.253513in}{2.188427in}}%
\pgfpathlineto{\pgfqpoint{5.261244in}{2.187114in}}%
\pgfpathlineto{\pgfqpoint{5.263822in}{2.203626in}}%
\pgfpathlineto{\pgfqpoint{5.266399in}{2.224828in}}%
\pgfpathlineto{\pgfqpoint{5.268976in}{2.223327in}}%
\pgfpathlineto{\pgfqpoint{5.271553in}{2.206628in}}%
\pgfpathlineto{\pgfqpoint{5.279285in}{2.211319in}}%
\pgfpathlineto{\pgfqpoint{5.281862in}{2.207378in}}%
\pgfpathlineto{\pgfqpoint{5.284439in}{2.207191in}}%
\pgfpathlineto{\pgfqpoint{5.289594in}{2.198560in}}%
\pgfpathlineto{\pgfqpoint{5.297325in}{2.189178in}}%
\pgfpathlineto{\pgfqpoint{5.299903in}{2.191054in}}%
\pgfpathlineto{\pgfqpoint{5.302480in}{2.188240in}}%
\pgfpathlineto{\pgfqpoint{5.307634in}{2.177357in}}%
\pgfpathlineto{\pgfqpoint{5.315366in}{2.171165in}}%
\pgfpathlineto{\pgfqpoint{5.317943in}{2.171165in}}%
\pgfpathlineto{\pgfqpoint{5.320520in}{2.163847in}}%
\pgfpathlineto{\pgfqpoint{5.323098in}{2.160282in}}%
\pgfpathlineto{\pgfqpoint{5.325675in}{2.163097in}}%
\pgfpathlineto{\pgfqpoint{5.335984in}{2.152027in}}%
\pgfpathlineto{\pgfqpoint{5.338561in}{2.150525in}}%
\pgfpathlineto{\pgfqpoint{5.341138in}{2.157280in}}%
\pgfpathlineto{\pgfqpoint{5.343715in}{2.127259in}}%
\pgfpathlineto{\pgfqpoint{5.351447in}{2.149212in}}%
\pgfpathlineto{\pgfqpoint{5.354024in}{2.132325in}}%
\pgfpathlineto{\pgfqpoint{5.356601in}{2.124444in}}%
\pgfpathlineto{\pgfqpoint{5.359178in}{2.128760in}}%
\pgfpathlineto{\pgfqpoint{5.361756in}{2.129886in}}%
\pgfpathlineto{\pgfqpoint{5.369487in}{2.131012in}}%
\pgfpathlineto{\pgfqpoint{5.372065in}{2.136828in}}%
\pgfpathlineto{\pgfqpoint{5.374642in}{2.126696in}}%
\pgfpathlineto{\pgfqpoint{5.377219in}{2.145084in}}%
\pgfpathlineto{\pgfqpoint{5.379796in}{2.142645in}}%
\pgfpathlineto{\pgfqpoint{5.387528in}{2.119003in}}%
\pgfpathlineto{\pgfqpoint{5.390105in}{2.114687in}}%
\pgfpathlineto{\pgfqpoint{5.392682in}{2.123319in}}%
\pgfpathlineto{\pgfqpoint{5.395259in}{2.116189in}}%
\pgfpathlineto{\pgfqpoint{5.397837in}{2.135140in}}%
\pgfpathlineto{\pgfqpoint{5.405568in}{2.128572in}}%
\pgfpathlineto{\pgfqpoint{5.408146in}{2.130449in}}%
\pgfpathlineto{\pgfqpoint{5.410723in}{2.127822in}}%
\pgfpathlineto{\pgfqpoint{5.413300in}{2.134577in}}%
\pgfpathlineto{\pgfqpoint{5.415877in}{2.128572in}}%
\pgfpathlineto{\pgfqpoint{5.423609in}{2.128572in}}%
\pgfpathlineto{\pgfqpoint{5.426186in}{2.117314in}}%
\pgfpathlineto{\pgfqpoint{5.428763in}{2.109246in}}%
\pgfpathlineto{\pgfqpoint{5.431340in}{2.103992in}}%
\pgfpathlineto{\pgfqpoint{5.433918in}{2.107182in}}%
\pgfpathlineto{\pgfqpoint{5.441649in}{2.098739in}}%
\pgfpathlineto{\pgfqpoint{5.444227in}{2.104931in}}%
\pgfpathlineto{\pgfqpoint{5.446804in}{2.118440in}}%
\pgfpathlineto{\pgfqpoint{5.449381in}{2.120316in}}%
\pgfpathlineto{\pgfqpoint{5.451958in}{2.138329in}}%
\pgfpathlineto{\pgfqpoint{5.459690in}{2.144334in}}%
\pgfpathlineto{\pgfqpoint{5.462267in}{2.133263in}}%
\pgfpathlineto{\pgfqpoint{5.464844in}{2.146585in}}%
\pgfpathlineto{\pgfqpoint{5.467421in}{2.155967in}}%
\pgfpathlineto{\pgfqpoint{5.469999in}{2.152965in}}%
\pgfpathlineto{\pgfqpoint{5.482885in}{2.118253in}}%
\pgfpathlineto{\pgfqpoint{5.485462in}{2.144334in}}%
\pgfpathlineto{\pgfqpoint{5.488039in}{2.127822in}}%
\pgfpathlineto{\pgfqpoint{5.495771in}{2.163285in}}%
\pgfpathlineto{\pgfqpoint{5.498348in}{2.162534in}}%
\pgfpathlineto{\pgfqpoint{5.500925in}{2.167037in}}%
\pgfpathlineto{\pgfqpoint{5.503502in}{2.172854in}}%
\pgfpathlineto{\pgfqpoint{5.506080in}{2.221639in}}%
\pgfpathlineto{\pgfqpoint{5.513811in}{2.225954in}}%
\pgfpathlineto{\pgfqpoint{5.516388in}{2.222014in}}%
\pgfpathlineto{\pgfqpoint{5.518966in}{2.247532in}}%
\pgfpathlineto{\pgfqpoint{5.521543in}{2.252035in}}%
\pgfpathlineto{\pgfqpoint{5.524120in}{2.231771in}}%
\pgfpathlineto{\pgfqpoint{5.531852in}{2.220700in}}%
\pgfpathlineto{\pgfqpoint{5.534429in}{2.222202in}}%
\pgfpathlineto{\pgfqpoint{5.537006in}{2.232146in}}%
\pgfpathlineto{\pgfqpoint{5.542161in}{2.242091in}}%
\pgfpathlineto{\pgfqpoint{5.549892in}{2.244718in}}%
\pgfpathlineto{\pgfqpoint{5.552469in}{2.257289in}}%
\pgfpathlineto{\pgfqpoint{5.555047in}{2.247532in}}%
\pgfpathlineto{\pgfqpoint{5.557624in}{2.244342in}}%
\pgfpathlineto{\pgfqpoint{5.560201in}{2.236462in}}%
\pgfpathlineto{\pgfqpoint{5.567933in}{2.262543in}}%
\pgfpathlineto{\pgfqpoint{5.570510in}{2.275114in}}%
\pgfpathlineto{\pgfqpoint{5.573087in}{2.298944in}}%
\pgfpathlineto{\pgfqpoint{5.575664in}{2.338159in}}%
\pgfpathlineto{\pgfqpoint{5.578242in}{2.364991in}}%
\pgfpathlineto{\pgfqpoint{5.588550in}{2.346603in}}%
\pgfpathlineto{\pgfqpoint{5.591128in}{2.350355in}}%
\pgfpathlineto{\pgfqpoint{5.593705in}{2.356360in}}%
\pgfpathlineto{\pgfqpoint{5.596282in}{2.347728in}}%
\pgfpathlineto{\pgfqpoint{5.604014in}{2.372871in}}%
\pgfpathlineto{\pgfqpoint{5.606591in}{2.375873in}}%
\pgfpathlineto{\pgfqpoint{5.609168in}{2.377562in}}%
\pgfpathlineto{\pgfqpoint{5.611745in}{2.375123in}}%
\pgfpathlineto{\pgfqpoint{5.614323in}{2.370244in}}%
\pgfpathlineto{\pgfqpoint{5.624631in}{2.370620in}}%
\pgfpathlineto{\pgfqpoint{5.627209in}{2.354858in}}%
\pgfpathlineto{\pgfqpoint{5.629786in}{2.359549in}}%
\pgfpathlineto{\pgfqpoint{5.632363in}{2.353357in}}%
\pgfpathlineto{\pgfqpoint{5.642672in}{2.386944in}}%
\pgfpathlineto{\pgfqpoint{5.645249in}{2.411524in}}%
\pgfpathlineto{\pgfqpoint{5.647826in}{2.410586in}}%
\pgfpathlineto{\pgfqpoint{5.650403in}{2.439481in}}%
\pgfpathlineto{\pgfqpoint{5.658135in}{2.428223in}}%
\pgfpathlineto{\pgfqpoint{5.660712in}{2.428598in}}%
\pgfpathlineto{\pgfqpoint{5.663290in}{2.447737in}}%
\pgfpathlineto{\pgfqpoint{5.665867in}{2.413213in}}%
\pgfpathlineto{\pgfqpoint{5.668444in}{2.422782in}}%
\pgfpathlineto{\pgfqpoint{5.678753in}{2.421093in}}%
\pgfpathlineto{\pgfqpoint{5.681330in}{2.424658in}}%
\pgfpathlineto{\pgfqpoint{5.683907in}{2.408709in}}%
\pgfpathlineto{\pgfqpoint{5.686484in}{2.415652in}}%
\pgfpathlineto{\pgfqpoint{5.694216in}{2.405707in}}%
\pgfpathlineto{\pgfqpoint{5.696793in}{2.419967in}}%
\pgfpathlineto{\pgfqpoint{5.699371in}{2.422407in}}%
\pgfpathlineto{\pgfqpoint{5.701948in}{2.422782in}}%
\pgfpathlineto{\pgfqpoint{5.704525in}{2.445110in}}%
\pgfpathlineto{\pgfqpoint{5.712257in}{2.474756in}}%
\pgfpathlineto{\pgfqpoint{5.714834in}{2.469690in}}%
\pgfpathlineto{\pgfqpoint{5.717411in}{2.481323in}}%
\pgfpathlineto{\pgfqpoint{5.719988in}{2.468940in}}%
\pgfpathlineto{\pgfqpoint{5.722565in}{2.463311in}}%
\pgfpathlineto{\pgfqpoint{5.730297in}{2.450176in}}%
\pgfpathlineto{\pgfqpoint{5.732874in}{2.439856in}}%
\pgfpathlineto{\pgfqpoint{5.735452in}{2.439856in}}%
\pgfpathlineto{\pgfqpoint{5.738029in}{2.448675in}}%
\pgfpathlineto{\pgfqpoint{5.740606in}{2.444547in}}%
\pgfpathlineto{\pgfqpoint{5.748338in}{2.451490in}}%
\pgfpathlineto{\pgfqpoint{5.750915in}{2.463874in}}%
\pgfpathlineto{\pgfqpoint{5.753492in}{2.461059in}}%
\pgfpathlineto{\pgfqpoint{5.756069in}{2.470628in}}%
\pgfpathlineto{\pgfqpoint{5.758646in}{2.458995in}}%
\pgfpathlineto{\pgfqpoint{5.768955in}{2.458057in}}%
\pgfpathlineto{\pgfqpoint{5.771532in}{2.459933in}}%
\pgfpathlineto{\pgfqpoint{5.774110in}{2.452991in}}%
\pgfpathlineto{\pgfqpoint{5.776687in}{2.463686in}}%
\pgfpathlineto{\pgfqpoint{5.784419in}{2.461997in}}%
\pgfpathlineto{\pgfqpoint{5.786996in}{2.459558in}}%
\pgfpathlineto{\pgfqpoint{5.789573in}{2.476633in}}%
\pgfpathlineto{\pgfqpoint{5.792150in}{2.468564in}}%
\pgfpathlineto{\pgfqpoint{5.794727in}{2.480198in}}%
\pgfpathlineto{\pgfqpoint{5.802459in}{2.469878in}}%
\pgfpathlineto{\pgfqpoint{5.805036in}{2.473443in}}%
\pgfpathlineto{\pgfqpoint{5.807613in}{2.473068in}}%
\pgfpathlineto{\pgfqpoint{5.810191in}{2.476445in}}%
\pgfpathlineto{\pgfqpoint{5.812768in}{2.474569in}}%
\pgfpathlineto{\pgfqpoint{5.820500in}{2.485264in}}%
\pgfpathlineto{\pgfqpoint{5.823077in}{2.499524in}}%
\pgfpathlineto{\pgfqpoint{5.825654in}{2.491643in}}%
\pgfpathlineto{\pgfqpoint{5.828231in}{2.488829in}}%
\pgfpathlineto{\pgfqpoint{5.830808in}{2.489579in}}%
\pgfpathlineto{\pgfqpoint{5.838540in}{2.506842in}}%
\pgfpathlineto{\pgfqpoint{5.841117in}{2.489579in}}%
\pgfpathlineto{\pgfqpoint{5.843694in}{2.495396in}}%
\pgfpathlineto{\pgfqpoint{5.846272in}{2.498398in}}%
\pgfpathlineto{\pgfqpoint{5.848849in}{2.496522in}}%
\pgfpathlineto{\pgfqpoint{5.856580in}{2.500837in}}%
\pgfpathlineto{\pgfqpoint{5.859158in}{2.513034in}}%
\pgfpathlineto{\pgfqpoint{5.861735in}{2.503652in}}%
\pgfpathlineto{\pgfqpoint{5.864312in}{2.515285in}}%
\pgfpathlineto{\pgfqpoint{5.866889in}{2.519225in}}%
\pgfpathlineto{\pgfqpoint{5.874621in}{2.515660in}}%
\pgfpathlineto{\pgfqpoint{5.877198in}{2.512846in}}%
\pgfpathlineto{\pgfqpoint{5.879775in}{2.512095in}}%
\pgfpathlineto{\pgfqpoint{5.882353in}{2.513034in}}%
\pgfpathlineto{\pgfqpoint{5.884930in}{2.504402in}}%
\pgfpathlineto{\pgfqpoint{5.892661in}{2.501776in}}%
\pgfpathlineto{\pgfqpoint{5.895239in}{2.513409in}}%
\pgfpathlineto{\pgfqpoint{5.897816in}{2.512846in}}%
\pgfpathlineto{\pgfqpoint{5.910702in}{2.526168in}}%
\pgfpathlineto{\pgfqpoint{5.913279in}{2.533673in}}%
\pgfpathlineto{\pgfqpoint{5.915856in}{2.525230in}}%
\pgfpathlineto{\pgfqpoint{5.918434in}{2.544368in}}%
\pgfpathlineto{\pgfqpoint{5.921011in}{2.538176in}}%
\pgfpathlineto{\pgfqpoint{5.928742in}{2.524667in}}%
\pgfpathlineto{\pgfqpoint{5.931320in}{2.551498in}}%
\pgfpathlineto{\pgfqpoint{5.933897in}{2.558629in}}%
\pgfpathlineto{\pgfqpoint{5.936474in}{2.563507in}}%
\pgfpathlineto{\pgfqpoint{5.939051in}{2.559004in}}%
\pgfpathlineto{\pgfqpoint{5.946783in}{2.543430in}}%
\pgfpathlineto{\pgfqpoint{5.949360in}{2.536863in}}%
\pgfpathlineto{\pgfqpoint{5.951937in}{2.487140in}}%
\pgfpathlineto{\pgfqpoint{5.954515in}{2.479072in}}%
\pgfpathlineto{\pgfqpoint{5.957092in}{2.493895in}}%
\pgfpathlineto{\pgfqpoint{5.964823in}{2.483575in}}%
\pgfpathlineto{\pgfqpoint{5.967401in}{2.495208in}}%
\pgfpathlineto{\pgfqpoint{5.969978in}{2.451677in}}%
\pgfpathlineto{\pgfqpoint{5.972555in}{2.450176in}}%
\pgfpathlineto{\pgfqpoint{5.975132in}{2.452240in}}%
\pgfpathlineto{\pgfqpoint{5.982864in}{2.442108in}}%
\pgfpathlineto{\pgfqpoint{5.985441in}{2.421281in}}%
\pgfpathlineto{\pgfqpoint{5.988018in}{2.391072in}}%
\pgfpathlineto{\pgfqpoint{5.990596in}{2.397827in}}%
\pgfpathlineto{\pgfqpoint{5.993173in}{2.413025in}}%
\pgfpathlineto{\pgfqpoint{6.000904in}{2.415089in}}%
\pgfpathlineto{\pgfqpoint{6.003482in}{2.404018in}}%
\pgfpathlineto{\pgfqpoint{6.006059in}{2.416402in}}%
\pgfpathlineto{\pgfqpoint{6.008636in}{2.408522in}}%
\pgfpathlineto{\pgfqpoint{6.011213in}{2.429161in}}%
\pgfpathlineto{\pgfqpoint{6.021522in}{2.427848in}}%
\pgfpathlineto{\pgfqpoint{6.024099in}{2.420530in}}%
\pgfpathlineto{\pgfqpoint{6.026677in}{2.425409in}}%
\pgfpathlineto{\pgfqpoint{6.029254in}{2.406833in}}%
\pgfpathlineto{\pgfqpoint{6.036985in}{2.395012in}}%
\pgfpathlineto{\pgfqpoint{6.039563in}{2.376436in}}%
\pgfpathlineto{\pgfqpoint{6.042140in}{2.384129in}}%
\pgfpathlineto{\pgfqpoint{6.044717in}{2.355234in}}%
\pgfpathlineto{\pgfqpoint{6.047294in}{2.378688in}}%
\pgfpathlineto{\pgfqpoint{6.055026in}{2.404394in}}%
\pgfpathlineto{\pgfqpoint{6.060180in}{2.388070in}}%
\pgfpathlineto{\pgfqpoint{6.062757in}{2.385255in}}%
\pgfpathlineto{\pgfqpoint{6.065335in}{2.376624in}}%
\pgfpathlineto{\pgfqpoint{6.073066in}{2.374185in}}%
\pgfpathlineto{\pgfqpoint{6.075644in}{2.348291in}}%
\pgfpathlineto{\pgfqpoint{6.078221in}{2.363865in}}%
\pgfpathlineto{\pgfqpoint{6.080798in}{2.353357in}}%
\pgfpathlineto{\pgfqpoint{6.083375in}{2.355984in}}%
\pgfpathlineto{\pgfqpoint{6.091107in}{2.377750in}}%
\pgfpathlineto{\pgfqpoint{6.093684in}{2.374935in}}%
\pgfpathlineto{\pgfqpoint{6.096261in}{2.401016in}}%
\pgfpathlineto{\pgfqpoint{6.098838in}{2.380377in}}%
\pgfpathlineto{\pgfqpoint{6.101416in}{2.390134in}}%
\pgfpathlineto{\pgfqpoint{6.109147in}{2.411899in}}%
\pgfpathlineto{\pgfqpoint{6.114302in}{2.378876in}}%
\pgfpathlineto{\pgfqpoint{6.116879in}{2.351481in}}%
\pgfpathlineto{\pgfqpoint{6.119456in}{2.350918in}}%
\pgfpathlineto{\pgfqpoint{6.127188in}{2.355984in}}%
\pgfpathlineto{\pgfqpoint{6.129765in}{2.360487in}}%
\pgfpathlineto{\pgfqpoint{6.132342in}{2.370244in}}%
\pgfpathlineto{\pgfqpoint{6.134919in}{2.368743in}}%
\pgfpathlineto{\pgfqpoint{6.137497in}{2.383191in}}%
\pgfpathlineto{\pgfqpoint{6.145228in}{2.377750in}}%
\pgfpathlineto{\pgfqpoint{6.150383in}{2.414714in}}%
\pgfpathlineto{\pgfqpoint{6.152960in}{2.424658in}}%
\pgfpathlineto{\pgfqpoint{6.155537in}{2.419592in}}%
\pgfpathlineto{\pgfqpoint{6.163269in}{2.418091in}}%
\pgfpathlineto{\pgfqpoint{6.165846in}{2.407396in}}%
\pgfpathlineto{\pgfqpoint{6.168423in}{2.416965in}}%
\pgfpathlineto{\pgfqpoint{6.171000in}{2.472692in}}%
\pgfpathlineto{\pgfqpoint{6.173578in}{2.471942in}}%
\pgfpathlineto{\pgfqpoint{6.181309in}{2.471379in}}%
\pgfpathlineto{\pgfqpoint{6.183886in}{2.483763in}}%
\pgfpathlineto{\pgfqpoint{6.186464in}{2.448488in}}%
\pgfpathlineto{\pgfqpoint{6.189041in}{2.456556in}}%
\pgfpathlineto{\pgfqpoint{6.191618in}{2.430662in}}%
\pgfpathlineto{\pgfqpoint{6.199350in}{2.406270in}}%
\pgfpathlineto{\pgfqpoint{6.201927in}{2.417716in}}%
\pgfpathlineto{\pgfqpoint{6.204504in}{2.342099in}}%
\pgfpathlineto{\pgfqpoint{6.207081in}{2.315080in}}%
\pgfpathlineto{\pgfqpoint{6.209659in}{2.326713in}}%
\pgfpathlineto{\pgfqpoint{6.217390in}{2.316018in}}%
\pgfpathlineto{\pgfqpoint{6.219967in}{2.318082in}}%
\pgfpathlineto{\pgfqpoint{6.222545in}{2.330654in}}%
\pgfpathlineto{\pgfqpoint{6.227699in}{2.303259in}}%
\pgfpathlineto{\pgfqpoint{6.235431in}{2.311890in}}%
\pgfpathlineto{\pgfqpoint{6.238008in}{2.340974in}}%
\pgfpathlineto{\pgfqpoint{6.240585in}{2.317895in}}%
\pgfpathlineto{\pgfqpoint{6.243162in}{2.318270in}}%
\pgfpathlineto{\pgfqpoint{6.245740in}{2.334406in}}%
\pgfpathlineto{\pgfqpoint{6.253471in}{2.337221in}}%
\pgfpathlineto{\pgfqpoint{6.256048in}{2.337409in}}%
\pgfpathlineto{\pgfqpoint{6.258626in}{2.342850in}}%
\pgfpathlineto{\pgfqpoint{6.261203in}{2.312453in}}%
\pgfpathlineto{\pgfqpoint{6.263780in}{2.317895in}}%
\pgfpathlineto{\pgfqpoint{6.274089in}{2.319771in}}%
\pgfpathlineto{\pgfqpoint{6.276666in}{2.317895in}}%
\pgfpathlineto{\pgfqpoint{6.279243in}{2.237025in}}%
\pgfpathlineto{\pgfqpoint{6.289552in}{2.237587in}}%
\pgfpathlineto{\pgfqpoint{6.294707in}{2.269110in}}%
\pgfpathlineto{\pgfqpoint{6.297284in}{2.252223in}}%
\pgfpathlineto{\pgfqpoint{6.299861in}{2.263669in}}%
\pgfpathlineto{\pgfqpoint{6.307593in}{2.255976in}}%
\pgfpathlineto{\pgfqpoint{6.310170in}{2.261980in}}%
\pgfpathlineto{\pgfqpoint{6.312747in}{2.276240in}}%
\pgfpathlineto{\pgfqpoint{6.317902in}{2.264982in}}%
\pgfpathlineto{\pgfqpoint{6.325633in}{2.282807in}}%
\pgfpathlineto{\pgfqpoint{6.328210in}{2.265545in}}%
\pgfpathlineto{\pgfqpoint{6.330788in}{2.276615in}}%
\pgfpathlineto{\pgfqpoint{6.333365in}{2.255037in}}%
\pgfpathlineto{\pgfqpoint{6.335942in}{2.264419in}}%
\pgfpathlineto{\pgfqpoint{6.343674in}{2.288061in}}%
\pgfpathlineto{\pgfqpoint{6.346251in}{2.304948in}}%
\pgfpathlineto{\pgfqpoint{6.348828in}{2.300632in}}%
\pgfpathlineto{\pgfqpoint{6.351405in}{2.292564in}}%
\pgfpathlineto{\pgfqpoint{6.353982in}{2.291814in}}%
\pgfpathlineto{\pgfqpoint{6.361714in}{2.282807in}}%
\pgfpathlineto{\pgfqpoint{6.364291in}{2.282807in}}%
\pgfpathlineto{\pgfqpoint{6.366869in}{2.264231in}}%
\pgfpathlineto{\pgfqpoint{6.369446in}{2.234585in}}%
\pgfpathlineto{\pgfqpoint{6.372023in}{2.242841in}}%
\pgfpathlineto{\pgfqpoint{6.382332in}{2.260666in}}%
\pgfpathlineto{\pgfqpoint{6.384909in}{2.258602in}}%
\pgfpathlineto{\pgfqpoint{6.387486in}{2.272487in}}%
\pgfpathlineto{\pgfqpoint{6.390063in}{2.279617in}}%
\pgfpathlineto{\pgfqpoint{6.397795in}{2.266858in}}%
\pgfpathlineto{\pgfqpoint{6.402950in}{2.250534in}}%
\pgfpathlineto{\pgfqpoint{6.405527in}{2.264231in}}%
\pgfpathlineto{\pgfqpoint{6.408104in}{2.259728in}}%
\pgfpathlineto{\pgfqpoint{6.415836in}{2.254850in}}%
\pgfpathlineto{\pgfqpoint{6.418413in}{2.250722in}}%
\pgfpathlineto{\pgfqpoint{6.420990in}{2.272863in}}%
\pgfpathlineto{\pgfqpoint{6.423567in}{2.260479in}}%
\pgfpathlineto{\pgfqpoint{6.426144in}{2.265733in}}%
\pgfpathlineto{\pgfqpoint{6.433876in}{2.302133in}}%
\pgfpathlineto{\pgfqpoint{6.436453in}{2.319771in}}%
\pgfpathlineto{\pgfqpoint{6.439031in}{2.312078in}}%
\pgfpathlineto{\pgfqpoint{6.441608in}{2.339285in}}%
\pgfpathlineto{\pgfqpoint{6.444185in}{2.377562in}}%
\pgfpathlineto{\pgfqpoint{6.451917in}{2.376812in}}%
\pgfpathlineto{\pgfqpoint{6.454494in}{2.348291in}}%
\pgfpathlineto{\pgfqpoint{6.457071in}{2.357673in}}%
\pgfpathlineto{\pgfqpoint{6.459648in}{2.356172in}}%
\pgfpathlineto{\pgfqpoint{6.462225in}{2.353170in}}%
\pgfpathlineto{\pgfqpoint{6.469957in}{2.340598in}}%
\pgfpathlineto{\pgfqpoint{6.472534in}{2.345102in}}%
\pgfpathlineto{\pgfqpoint{6.475111in}{2.340411in}}%
\pgfpathlineto{\pgfqpoint{6.480266in}{2.338534in}}%
\pgfpathlineto{\pgfqpoint{6.487998in}{2.341349in}}%
\pgfpathlineto{\pgfqpoint{6.490575in}{2.352607in}}%
\pgfpathlineto{\pgfqpoint{6.493152in}{2.386006in}}%
\pgfpathlineto{\pgfqpoint{6.495729in}{2.378313in}}%
\pgfpathlineto{\pgfqpoint{6.498306in}{2.386193in}}%
\pgfpathlineto{\pgfqpoint{6.506038in}{2.476633in}}%
\pgfpathlineto{\pgfqpoint{6.508615in}{2.422031in}}%
\pgfpathlineto{\pgfqpoint{6.511192in}{2.389946in}}%
\pgfpathlineto{\pgfqpoint{6.516347in}{2.382816in}}%
\pgfpathlineto{\pgfqpoint{6.524079in}{2.430662in}}%
\pgfpathlineto{\pgfqpoint{6.526656in}{2.441545in}}%
\pgfpathlineto{\pgfqpoint{6.529233in}{2.444923in}}%
\pgfpathlineto{\pgfqpoint{6.531810in}{2.499149in}}%
\pgfpathlineto{\pgfqpoint{6.534387in}{2.512095in}}%
\pgfpathlineto{\pgfqpoint{6.542119in}{2.507592in}}%
\pgfpathlineto{\pgfqpoint{6.544696in}{2.522040in}}%
\pgfpathlineto{\pgfqpoint{6.547273in}{2.483012in}}%
\pgfpathlineto{\pgfqpoint{6.549851in}{2.480948in}}%
\pgfpathlineto{\pgfqpoint{6.552428in}{2.464436in}}%
\pgfpathlineto{\pgfqpoint{6.562737in}{2.454304in}}%
\pgfpathlineto{\pgfqpoint{6.565314in}{2.445485in}}%
\pgfpathlineto{\pgfqpoint{6.567891in}{2.447737in}}%
\pgfpathlineto{\pgfqpoint{6.570468in}{2.443046in}}%
\pgfpathlineto{\pgfqpoint{6.570468in}{2.443046in}}%
\pgfusepath{stroke}%
\end{pgfscope}%
\begin{pgfscope}%
\pgfsetrectcap%
\pgfsetmiterjoin%
\pgfsetlinewidth{0.803000pt}%
\definecolor{currentstroke}{rgb}{1.000000,1.000000,1.000000}%
\pgfsetstrokecolor{currentstroke}%
\pgfsetdash{}{0pt}%
\pgfpathmoveto{\pgfqpoint{0.652287in}{0.521603in}}%
\pgfpathlineto{\pgfqpoint{0.652287in}{5.051603in}}%
\pgfusepath{stroke}%
\end{pgfscope}%
\begin{pgfscope}%
\pgfsetrectcap%
\pgfsetmiterjoin%
\pgfsetlinewidth{0.803000pt}%
\definecolor{currentstroke}{rgb}{1.000000,1.000000,1.000000}%
\pgfsetstrokecolor{currentstroke}%
\pgfsetdash{}{0pt}%
\pgfpathmoveto{\pgfqpoint{6.852287in}{0.521603in}}%
\pgfpathlineto{\pgfqpoint{6.852287in}{5.051603in}}%
\pgfusepath{stroke}%
\end{pgfscope}%
\begin{pgfscope}%
\pgfsetrectcap%
\pgfsetmiterjoin%
\pgfsetlinewidth{0.803000pt}%
\definecolor{currentstroke}{rgb}{1.000000,1.000000,1.000000}%
\pgfsetstrokecolor{currentstroke}%
\pgfsetdash{}{0pt}%
\pgfpathmoveto{\pgfqpoint{0.652287in}{0.521603in}}%
\pgfpathlineto{\pgfqpoint{6.852287in}{0.521603in}}%
\pgfusepath{stroke}%
\end{pgfscope}%
\begin{pgfscope}%
\pgfsetrectcap%
\pgfsetmiterjoin%
\pgfsetlinewidth{0.803000pt}%
\definecolor{currentstroke}{rgb}{1.000000,1.000000,1.000000}%
\pgfsetstrokecolor{currentstroke}%
\pgfsetdash{}{0pt}%
\pgfpathmoveto{\pgfqpoint{0.652287in}{5.051603in}}%
\pgfpathlineto{\pgfqpoint{6.852287in}{5.051603in}}%
\pgfusepath{stroke}%
\end{pgfscope}%
\begin{pgfscope}%
\definecolor{textcolor}{rgb}{0.150000,0.150000,0.150000}%
\pgfsetstrokecolor{textcolor}%
\pgfsetfillcolor{textcolor}%
\pgftext[x=3.752287in,y=5.134937in,,base]{\color{textcolor}\rmfamily\fontsize{12.000000}{14.400000}\selectfont Adjusted Closing Value of Selected Stocks}%
\end{pgfscope}%
\begin{pgfscope}%
\pgfsetbuttcap%
\pgfsetmiterjoin%
\definecolor{currentfill}{rgb}{0.917647,0.917647,0.949020}%
\pgfsetfillcolor{currentfill}%
\pgfsetfillopacity{0.800000}%
\pgfsetlinewidth{1.003750pt}%
\definecolor{currentstroke}{rgb}{0.800000,0.800000,0.800000}%
\pgfsetstrokecolor{currentstroke}%
\pgfsetstrokeopacity{0.800000}%
\pgfsetdash{}{0pt}%
\pgfpathmoveto{\pgfqpoint{0.749509in}{2.901920in}}%
\pgfpathlineto{\pgfqpoint{1.620589in}{2.901920in}}%
\pgfpathquadraticcurveto{\pgfqpoint{1.648367in}{2.901920in}}{\pgfqpoint{1.648367in}{2.929698in}}%
\pgfpathlineto{\pgfqpoint{1.648367in}{4.954381in}}%
\pgfpathquadraticcurveto{\pgfqpoint{1.648367in}{4.982159in}}{\pgfqpoint{1.620589in}{4.982159in}}%
\pgfpathlineto{\pgfqpoint{0.749509in}{4.982159in}}%
\pgfpathquadraticcurveto{\pgfqpoint{0.721731in}{4.982159in}}{\pgfqpoint{0.721731in}{4.954381in}}%
\pgfpathlineto{\pgfqpoint{0.721731in}{2.929698in}}%
\pgfpathquadraticcurveto{\pgfqpoint{0.721731in}{2.901920in}}{\pgfqpoint{0.749509in}{2.901920in}}%
\pgfpathclose%
\pgfusepath{stroke,fill}%
\end{pgfscope}%
\begin{pgfscope}%
\pgfsetroundcap%
\pgfsetroundjoin%
\pgfsetlinewidth{1.505625pt}%
\definecolor{currentstroke}{rgb}{0.121569,0.466667,0.705882}%
\pgfsetstrokecolor{currentstroke}%
\pgfsetdash{}{0pt}%
\pgfpathmoveto{\pgfqpoint{0.777287in}{4.869691in}}%
\pgfpathlineto{\pgfqpoint{1.055064in}{4.869691in}}%
\pgfusepath{stroke}%
\end{pgfscope}%
\begin{pgfscope}%
\definecolor{textcolor}{rgb}{0.150000,0.150000,0.150000}%
\pgfsetstrokecolor{textcolor}%
\pgfsetfillcolor{textcolor}%
\pgftext[x=1.166175in,y=4.821080in,left,base]{\color{textcolor}\rmfamily\fontsize{10.000000}{12.000000}\selectfont MMM}%
\end{pgfscope}%
\begin{pgfscope}%
\pgfsetroundcap%
\pgfsetroundjoin%
\pgfsetlinewidth{1.505625pt}%
\definecolor{currentstroke}{rgb}{1.000000,0.498039,0.054902}%
\pgfsetstrokecolor{currentstroke}%
\pgfsetdash{}{0pt}%
\pgfpathmoveto{\pgfqpoint{0.777287in}{4.665834in}}%
\pgfpathlineto{\pgfqpoint{1.055064in}{4.665834in}}%
\pgfusepath{stroke}%
\end{pgfscope}%
\begin{pgfscope}%
\definecolor{textcolor}{rgb}{0.150000,0.150000,0.150000}%
\pgfsetstrokecolor{textcolor}%
\pgfsetfillcolor{textcolor}%
\pgftext[x=1.166175in,y=4.617223in,left,base]{\color{textcolor}\rmfamily\fontsize{10.000000}{12.000000}\selectfont AXP}%
\end{pgfscope}%
\begin{pgfscope}%
\pgfsetroundcap%
\pgfsetroundjoin%
\pgfsetlinewidth{1.505625pt}%
\definecolor{currentstroke}{rgb}{0.172549,0.627451,0.172549}%
\pgfsetstrokecolor{currentstroke}%
\pgfsetdash{}{0pt}%
\pgfpathmoveto{\pgfqpoint{0.777287in}{4.461977in}}%
\pgfpathlineto{\pgfqpoint{1.055064in}{4.461977in}}%
\pgfusepath{stroke}%
\end{pgfscope}%
\begin{pgfscope}%
\definecolor{textcolor}{rgb}{0.150000,0.150000,0.150000}%
\pgfsetstrokecolor{textcolor}%
\pgfsetfillcolor{textcolor}%
\pgftext[x=1.166175in,y=4.413366in,left,base]{\color{textcolor}\rmfamily\fontsize{10.000000}{12.000000}\selectfont GE}%
\end{pgfscope}%
\begin{pgfscope}%
\pgfsetroundcap%
\pgfsetroundjoin%
\pgfsetlinewidth{1.505625pt}%
\definecolor{currentstroke}{rgb}{0.839216,0.152941,0.156863}%
\pgfsetstrokecolor{currentstroke}%
\pgfsetdash{}{0pt}%
\pgfpathmoveto{\pgfqpoint{0.777287in}{4.258120in}}%
\pgfpathlineto{\pgfqpoint{1.055064in}{4.258120in}}%
\pgfusepath{stroke}%
\end{pgfscope}%
\begin{pgfscope}%
\definecolor{textcolor}{rgb}{0.150000,0.150000,0.150000}%
\pgfsetstrokecolor{textcolor}%
\pgfsetfillcolor{textcolor}%
\pgftext[x=1.166175in,y=4.209509in,left,base]{\color{textcolor}\rmfamily\fontsize{10.000000}{12.000000}\selectfont INTC}%
\end{pgfscope}%
\begin{pgfscope}%
\pgfsetroundcap%
\pgfsetroundjoin%
\pgfsetlinewidth{1.505625pt}%
\definecolor{currentstroke}{rgb}{0.580392,0.403922,0.741176}%
\pgfsetstrokecolor{currentstroke}%
\pgfsetdash{}{0pt}%
\pgfpathmoveto{\pgfqpoint{0.777287in}{4.054262in}}%
\pgfpathlineto{\pgfqpoint{1.055064in}{4.054262in}}%
\pgfusepath{stroke}%
\end{pgfscope}%
\begin{pgfscope}%
\definecolor{textcolor}{rgb}{0.150000,0.150000,0.150000}%
\pgfsetstrokecolor{textcolor}%
\pgfsetfillcolor{textcolor}%
\pgftext[x=1.166175in,y=4.005651in,left,base]{\color{textcolor}\rmfamily\fontsize{10.000000}{12.000000}\selectfont JNJ}%
\end{pgfscope}%
\begin{pgfscope}%
\pgfsetroundcap%
\pgfsetroundjoin%
\pgfsetlinewidth{1.505625pt}%
\definecolor{currentstroke}{rgb}{0.549020,0.337255,0.294118}%
\pgfsetstrokecolor{currentstroke}%
\pgfsetdash{}{0pt}%
\pgfpathmoveto{\pgfqpoint{0.777287in}{3.850405in}}%
\pgfpathlineto{\pgfqpoint{1.055064in}{3.850405in}}%
\pgfusepath{stroke}%
\end{pgfscope}%
\begin{pgfscope}%
\definecolor{textcolor}{rgb}{0.150000,0.150000,0.150000}%
\pgfsetstrokecolor{textcolor}%
\pgfsetfillcolor{textcolor}%
\pgftext[x=1.166175in,y=3.801794in,left,base]{\color{textcolor}\rmfamily\fontsize{10.000000}{12.000000}\selectfont PG}%
\end{pgfscope}%
\begin{pgfscope}%
\pgfsetroundcap%
\pgfsetroundjoin%
\pgfsetlinewidth{1.505625pt}%
\definecolor{currentstroke}{rgb}{0.890196,0.466667,0.760784}%
\pgfsetstrokecolor{currentstroke}%
\pgfsetdash{}{0pt}%
\pgfpathmoveto{\pgfqpoint{0.777287in}{3.646548in}}%
\pgfpathlineto{\pgfqpoint{1.055064in}{3.646548in}}%
\pgfusepath{stroke}%
\end{pgfscope}%
\begin{pgfscope}%
\definecolor{textcolor}{rgb}{0.150000,0.150000,0.150000}%
\pgfsetstrokecolor{textcolor}%
\pgfsetfillcolor{textcolor}%
\pgftext[x=1.166175in,y=3.597937in,left,base]{\color{textcolor}\rmfamily\fontsize{10.000000}{12.000000}\selectfont UTX}%
\end{pgfscope}%
\begin{pgfscope}%
\pgfsetroundcap%
\pgfsetroundjoin%
\pgfsetlinewidth{1.505625pt}%
\definecolor{currentstroke}{rgb}{0.498039,0.498039,0.498039}%
\pgfsetstrokecolor{currentstroke}%
\pgfsetdash{}{0pt}%
\pgfpathmoveto{\pgfqpoint{0.777287in}{3.442691in}}%
\pgfpathlineto{\pgfqpoint{1.055064in}{3.442691in}}%
\pgfusepath{stroke}%
\end{pgfscope}%
\begin{pgfscope}%
\definecolor{textcolor}{rgb}{0.150000,0.150000,0.150000}%
\pgfsetstrokecolor{textcolor}%
\pgfsetfillcolor{textcolor}%
\pgftext[x=1.166175in,y=3.394080in,left,base]{\color{textcolor}\rmfamily\fontsize{10.000000}{12.000000}\selectfont VZ}%
\end{pgfscope}%
\begin{pgfscope}%
\pgfsetroundcap%
\pgfsetroundjoin%
\pgfsetlinewidth{1.505625pt}%
\definecolor{currentstroke}{rgb}{0.737255,0.741176,0.133333}%
\pgfsetstrokecolor{currentstroke}%
\pgfsetdash{}{0pt}%
\pgfpathmoveto{\pgfqpoint{0.777287in}{3.238834in}}%
\pgfpathlineto{\pgfqpoint{1.055064in}{3.238834in}}%
\pgfusepath{stroke}%
\end{pgfscope}%
\begin{pgfscope}%
\definecolor{textcolor}{rgb}{0.150000,0.150000,0.150000}%
\pgfsetstrokecolor{textcolor}%
\pgfsetfillcolor{textcolor}%
\pgftext[x=1.166175in,y=3.190222in,left,base]{\color{textcolor}\rmfamily\fontsize{10.000000}{12.000000}\selectfont V}%
\end{pgfscope}%
\begin{pgfscope}%
\pgfsetroundcap%
\pgfsetroundjoin%
\pgfsetlinewidth{1.505625pt}%
\definecolor{currentstroke}{rgb}{0.090196,0.745098,0.811765}%
\pgfsetstrokecolor{currentstroke}%
\pgfsetdash{}{0pt}%
\pgfpathmoveto{\pgfqpoint{0.777287in}{3.034976in}}%
\pgfpathlineto{\pgfqpoint{1.055064in}{3.034976in}}%
\pgfusepath{stroke}%
\end{pgfscope}%
\begin{pgfscope}%
\definecolor{textcolor}{rgb}{0.150000,0.150000,0.150000}%
\pgfsetstrokecolor{textcolor}%
\pgfsetfillcolor{textcolor}%
\pgftext[x=1.166175in,y=2.986365in,left,base]{\color{textcolor}\rmfamily\fontsize{10.000000}{12.000000}\selectfont DIS}%
\end{pgfscope}%
\end{pgfpicture}%
\makeatother%
\endgroup%

    \end{adjustbox}  
    \caption{Time series of the adjusted closing prices of all 10 stocks looked at in this paper}
    \label{fig:Daily Stock Prices for all Stocks in the Data Set}
\end{figure}{}

From figure \ref{fig:Daily Stock Prices for all Stocks in the Data Set} it can be clearly seen that most of the stocks exhibit a strong trend. Also the variance of most stocks increases steadily with time over the observed period. This increase in variance is illustrated in figure \ref{fig:cum_sd_all} where the standard deviation of the time series is shown. The data are therefore clearly not stationary. We can also formally test for stationarity using the augmented Dickey-Fuller test [REFERENCE]. This test, applied to all time series in no case is able to reject the null hypothesis of a unit root (implying non-stationarity) at any reasonable confidence level. 

\begin{figure}[H]
    \figuretitle{'Cumulative' Standard Deviation of Stock Prices}
    \centering
    \begin{adjustbox}{width=.9\textwidth,center}
    %% Creator: Matplotlib, PGF backend
%%
%% To include the figure in your LaTeX document, write
%%   \input{<filename>.pgf}
%%
%% Make sure the required packages are loaded in your preamble
%%   \usepackage{pgf}
%%
%% Figures using additional raster images can only be included by \input if
%% they are in the same directory as the main LaTeX file. For loading figures
%% from other directories you can use the `import` package
%%   \usepackage{import}
%% and then include the figures with
%%   \import{<path to file>}{<filename>.pgf}
%%
%% Matplotlib used the following preamble
%%   \usepackage{fontspec}
%%   \setmainfont{DejaVuSerif.ttf}[Path=/opt/tljh/user/lib/python3.6/site-packages/matplotlib/mpl-data/fonts/ttf/]
%%   \setsansfont{DejaVuSans.ttf}[Path=/opt/tljh/user/lib/python3.6/site-packages/matplotlib/mpl-data/fonts/ttf/]
%%   \setmonofont{DejaVuSansMono.ttf}[Path=/opt/tljh/user/lib/python3.6/site-packages/matplotlib/mpl-data/fonts/ttf/]
%%
\begingroup%
\makeatletter%
\begin{pgfpicture}%
\pgfpathrectangle{\pgfpointorigin}{\pgfqpoint{6.863921in}{3.474064in}}%
\pgfusepath{use as bounding box, clip}%
\begin{pgfscope}%
\pgfsetbuttcap%
\pgfsetmiterjoin%
\definecolor{currentfill}{rgb}{1.000000,1.000000,1.000000}%
\pgfsetfillcolor{currentfill}%
\pgfsetlinewidth{0.000000pt}%
\definecolor{currentstroke}{rgb}{1.000000,1.000000,1.000000}%
\pgfsetstrokecolor{currentstroke}%
\pgfsetdash{}{0pt}%
\pgfpathmoveto{\pgfqpoint{0.000000in}{0.000000in}}%
\pgfpathlineto{\pgfqpoint{6.863921in}{0.000000in}}%
\pgfpathlineto{\pgfqpoint{6.863921in}{3.474064in}}%
\pgfpathlineto{\pgfqpoint{0.000000in}{3.474064in}}%
\pgfpathclose%
\pgfusepath{fill}%
\end{pgfscope}%
\begin{pgfscope}%
\pgfsetbuttcap%
\pgfsetmiterjoin%
\definecolor{currentfill}{rgb}{0.917647,0.917647,0.949020}%
\pgfsetfillcolor{currentfill}%
\pgfsetlinewidth{0.000000pt}%
\definecolor{currentstroke}{rgb}{0.000000,0.000000,0.000000}%
\pgfsetstrokecolor{currentstroke}%
\pgfsetstrokeopacity{0.000000}%
\pgfsetdash{}{0pt}%
\pgfpathmoveto{\pgfqpoint{0.563921in}{0.521603in}}%
\pgfpathlineto{\pgfqpoint{6.763921in}{0.521603in}}%
\pgfpathlineto{\pgfqpoint{6.763921in}{3.164103in}}%
\pgfpathlineto{\pgfqpoint{0.563921in}{3.164103in}}%
\pgfpathclose%
\pgfusepath{fill}%
\end{pgfscope}%
\begin{pgfscope}%
\pgfpathrectangle{\pgfqpoint{0.563921in}{0.521603in}}{\pgfqpoint{6.200000in}{2.642500in}}%
\pgfusepath{clip}%
\pgfsetroundcap%
\pgfsetroundjoin%
\pgfsetlinewidth{0.803000pt}%
\definecolor{currentstroke}{rgb}{1.000000,1.000000,1.000000}%
\pgfsetstrokecolor{currentstroke}%
\pgfsetdash{}{0pt}%
\pgfpathmoveto{\pgfqpoint{0.840585in}{0.521603in}}%
\pgfpathlineto{\pgfqpoint{0.840585in}{3.164103in}}%
\pgfusepath{stroke}%
\end{pgfscope}%
\begin{pgfscope}%
\definecolor{textcolor}{rgb}{0.150000,0.150000,0.150000}%
\pgfsetstrokecolor{textcolor}%
\pgfsetfillcolor{textcolor}%
\pgftext[x=0.840585in,y=0.424381in,,top]{\color{textcolor}\rmfamily\fontsize{10.000000}{12.000000}\selectfont 2012}%
\end{pgfscope}%
\begin{pgfscope}%
\pgfpathrectangle{\pgfqpoint{0.563921in}{0.521603in}}{\pgfqpoint{6.200000in}{2.642500in}}%
\pgfusepath{clip}%
\pgfsetroundcap%
\pgfsetroundjoin%
\pgfsetlinewidth{0.803000pt}%
\definecolor{currentstroke}{rgb}{1.000000,1.000000,1.000000}%
\pgfsetstrokecolor{currentstroke}%
\pgfsetdash{}{0pt}%
\pgfpathmoveto{\pgfqpoint{1.783845in}{0.521603in}}%
\pgfpathlineto{\pgfqpoint{1.783845in}{3.164103in}}%
\pgfusepath{stroke}%
\end{pgfscope}%
\begin{pgfscope}%
\definecolor{textcolor}{rgb}{0.150000,0.150000,0.150000}%
\pgfsetstrokecolor{textcolor}%
\pgfsetfillcolor{textcolor}%
\pgftext[x=1.783845in,y=0.424381in,,top]{\color{textcolor}\rmfamily\fontsize{10.000000}{12.000000}\selectfont 2013}%
\end{pgfscope}%
\begin{pgfscope}%
\pgfpathrectangle{\pgfqpoint{0.563921in}{0.521603in}}{\pgfqpoint{6.200000in}{2.642500in}}%
\pgfusepath{clip}%
\pgfsetroundcap%
\pgfsetroundjoin%
\pgfsetlinewidth{0.803000pt}%
\definecolor{currentstroke}{rgb}{1.000000,1.000000,1.000000}%
\pgfsetstrokecolor{currentstroke}%
\pgfsetdash{}{0pt}%
\pgfpathmoveto{\pgfqpoint{2.724527in}{0.521603in}}%
\pgfpathlineto{\pgfqpoint{2.724527in}{3.164103in}}%
\pgfusepath{stroke}%
\end{pgfscope}%
\begin{pgfscope}%
\definecolor{textcolor}{rgb}{0.150000,0.150000,0.150000}%
\pgfsetstrokecolor{textcolor}%
\pgfsetfillcolor{textcolor}%
\pgftext[x=2.724527in,y=0.424381in,,top]{\color{textcolor}\rmfamily\fontsize{10.000000}{12.000000}\selectfont 2014}%
\end{pgfscope}%
\begin{pgfscope}%
\pgfpathrectangle{\pgfqpoint{0.563921in}{0.521603in}}{\pgfqpoint{6.200000in}{2.642500in}}%
\pgfusepath{clip}%
\pgfsetroundcap%
\pgfsetroundjoin%
\pgfsetlinewidth{0.803000pt}%
\definecolor{currentstroke}{rgb}{1.000000,1.000000,1.000000}%
\pgfsetstrokecolor{currentstroke}%
\pgfsetdash{}{0pt}%
\pgfpathmoveto{\pgfqpoint{3.665210in}{0.521603in}}%
\pgfpathlineto{\pgfqpoint{3.665210in}{3.164103in}}%
\pgfusepath{stroke}%
\end{pgfscope}%
\begin{pgfscope}%
\definecolor{textcolor}{rgb}{0.150000,0.150000,0.150000}%
\pgfsetstrokecolor{textcolor}%
\pgfsetfillcolor{textcolor}%
\pgftext[x=3.665210in,y=0.424381in,,top]{\color{textcolor}\rmfamily\fontsize{10.000000}{12.000000}\selectfont 2015}%
\end{pgfscope}%
\begin{pgfscope}%
\pgfpathrectangle{\pgfqpoint{0.563921in}{0.521603in}}{\pgfqpoint{6.200000in}{2.642500in}}%
\pgfusepath{clip}%
\pgfsetroundcap%
\pgfsetroundjoin%
\pgfsetlinewidth{0.803000pt}%
\definecolor{currentstroke}{rgb}{1.000000,1.000000,1.000000}%
\pgfsetstrokecolor{currentstroke}%
\pgfsetdash{}{0pt}%
\pgfpathmoveto{\pgfqpoint{4.605892in}{0.521603in}}%
\pgfpathlineto{\pgfqpoint{4.605892in}{3.164103in}}%
\pgfusepath{stroke}%
\end{pgfscope}%
\begin{pgfscope}%
\definecolor{textcolor}{rgb}{0.150000,0.150000,0.150000}%
\pgfsetstrokecolor{textcolor}%
\pgfsetfillcolor{textcolor}%
\pgftext[x=4.605892in,y=0.424381in,,top]{\color{textcolor}\rmfamily\fontsize{10.000000}{12.000000}\selectfont 2016}%
\end{pgfscope}%
\begin{pgfscope}%
\pgfpathrectangle{\pgfqpoint{0.563921in}{0.521603in}}{\pgfqpoint{6.200000in}{2.642500in}}%
\pgfusepath{clip}%
\pgfsetroundcap%
\pgfsetroundjoin%
\pgfsetlinewidth{0.803000pt}%
\definecolor{currentstroke}{rgb}{1.000000,1.000000,1.000000}%
\pgfsetstrokecolor{currentstroke}%
\pgfsetdash{}{0pt}%
\pgfpathmoveto{\pgfqpoint{5.549152in}{0.521603in}}%
\pgfpathlineto{\pgfqpoint{5.549152in}{3.164103in}}%
\pgfusepath{stroke}%
\end{pgfscope}%
\begin{pgfscope}%
\definecolor{textcolor}{rgb}{0.150000,0.150000,0.150000}%
\pgfsetstrokecolor{textcolor}%
\pgfsetfillcolor{textcolor}%
\pgftext[x=5.549152in,y=0.424381in,,top]{\color{textcolor}\rmfamily\fontsize{10.000000}{12.000000}\selectfont 2017}%
\end{pgfscope}%
\begin{pgfscope}%
\pgfpathrectangle{\pgfqpoint{0.563921in}{0.521603in}}{\pgfqpoint{6.200000in}{2.642500in}}%
\pgfusepath{clip}%
\pgfsetroundcap%
\pgfsetroundjoin%
\pgfsetlinewidth{0.803000pt}%
\definecolor{currentstroke}{rgb}{1.000000,1.000000,1.000000}%
\pgfsetstrokecolor{currentstroke}%
\pgfsetdash{}{0pt}%
\pgfpathmoveto{\pgfqpoint{6.489835in}{0.521603in}}%
\pgfpathlineto{\pgfqpoint{6.489835in}{3.164103in}}%
\pgfusepath{stroke}%
\end{pgfscope}%
\begin{pgfscope}%
\definecolor{textcolor}{rgb}{0.150000,0.150000,0.150000}%
\pgfsetstrokecolor{textcolor}%
\pgfsetfillcolor{textcolor}%
\pgftext[x=6.489835in,y=0.424381in,,top]{\color{textcolor}\rmfamily\fontsize{10.000000}{12.000000}\selectfont 2018}%
\end{pgfscope}%
\begin{pgfscope}%
\definecolor{textcolor}{rgb}{0.150000,0.150000,0.150000}%
\pgfsetstrokecolor{textcolor}%
\pgfsetfillcolor{textcolor}%
\pgftext[x=3.663921in,y=0.234413in,,top]{\color{textcolor}\rmfamily\fontsize{10.000000}{12.000000}\selectfont Time t}%
\end{pgfscope}%
\begin{pgfscope}%
\pgfpathrectangle{\pgfqpoint{0.563921in}{0.521603in}}{\pgfqpoint{6.200000in}{2.642500in}}%
\pgfusepath{clip}%
\pgfsetroundcap%
\pgfsetroundjoin%
\pgfsetlinewidth{0.803000pt}%
\definecolor{currentstroke}{rgb}{1.000000,1.000000,1.000000}%
\pgfsetstrokecolor{currentstroke}%
\pgfsetdash{}{0pt}%
\pgfpathmoveto{\pgfqpoint{0.563921in}{0.641717in}}%
\pgfpathlineto{\pgfqpoint{6.763921in}{0.641717in}}%
\pgfusepath{stroke}%
\end{pgfscope}%
\begin{pgfscope}%
\definecolor{textcolor}{rgb}{0.150000,0.150000,0.150000}%
\pgfsetstrokecolor{textcolor}%
\pgfsetfillcolor{textcolor}%
\pgftext[x=0.378334in,y=0.588955in,left,base]{\color{textcolor}\rmfamily\fontsize{10.000000}{12.000000}\selectfont 0}%
\end{pgfscope}%
\begin{pgfscope}%
\pgfpathrectangle{\pgfqpoint{0.563921in}{0.521603in}}{\pgfqpoint{6.200000in}{2.642500in}}%
\pgfusepath{clip}%
\pgfsetroundcap%
\pgfsetroundjoin%
\pgfsetlinewidth{0.803000pt}%
\definecolor{currentstroke}{rgb}{1.000000,1.000000,1.000000}%
\pgfsetstrokecolor{currentstroke}%
\pgfsetdash{}{0pt}%
\pgfpathmoveto{\pgfqpoint{0.563921in}{0.939662in}}%
\pgfpathlineto{\pgfqpoint{6.763921in}{0.939662in}}%
\pgfusepath{stroke}%
\end{pgfscope}%
\begin{pgfscope}%
\definecolor{textcolor}{rgb}{0.150000,0.150000,0.150000}%
\pgfsetstrokecolor{textcolor}%
\pgfsetfillcolor{textcolor}%
\pgftext[x=0.378334in,y=0.886901in,left,base]{\color{textcolor}\rmfamily\fontsize{10.000000}{12.000000}\selectfont 5}%
\end{pgfscope}%
\begin{pgfscope}%
\pgfpathrectangle{\pgfqpoint{0.563921in}{0.521603in}}{\pgfqpoint{6.200000in}{2.642500in}}%
\pgfusepath{clip}%
\pgfsetroundcap%
\pgfsetroundjoin%
\pgfsetlinewidth{0.803000pt}%
\definecolor{currentstroke}{rgb}{1.000000,1.000000,1.000000}%
\pgfsetstrokecolor{currentstroke}%
\pgfsetdash{}{0pt}%
\pgfpathmoveto{\pgfqpoint{0.563921in}{1.237607in}}%
\pgfpathlineto{\pgfqpoint{6.763921in}{1.237607in}}%
\pgfusepath{stroke}%
\end{pgfscope}%
\begin{pgfscope}%
\definecolor{textcolor}{rgb}{0.150000,0.150000,0.150000}%
\pgfsetstrokecolor{textcolor}%
\pgfsetfillcolor{textcolor}%
\pgftext[x=0.289968in,y=1.184846in,left,base]{\color{textcolor}\rmfamily\fontsize{10.000000}{12.000000}\selectfont 10}%
\end{pgfscope}%
\begin{pgfscope}%
\pgfpathrectangle{\pgfqpoint{0.563921in}{0.521603in}}{\pgfqpoint{6.200000in}{2.642500in}}%
\pgfusepath{clip}%
\pgfsetroundcap%
\pgfsetroundjoin%
\pgfsetlinewidth{0.803000pt}%
\definecolor{currentstroke}{rgb}{1.000000,1.000000,1.000000}%
\pgfsetstrokecolor{currentstroke}%
\pgfsetdash{}{0pt}%
\pgfpathmoveto{\pgfqpoint{0.563921in}{1.535553in}}%
\pgfpathlineto{\pgfqpoint{6.763921in}{1.535553in}}%
\pgfusepath{stroke}%
\end{pgfscope}%
\begin{pgfscope}%
\definecolor{textcolor}{rgb}{0.150000,0.150000,0.150000}%
\pgfsetstrokecolor{textcolor}%
\pgfsetfillcolor{textcolor}%
\pgftext[x=0.289968in,y=1.482791in,left,base]{\color{textcolor}\rmfamily\fontsize{10.000000}{12.000000}\selectfont 15}%
\end{pgfscope}%
\begin{pgfscope}%
\pgfpathrectangle{\pgfqpoint{0.563921in}{0.521603in}}{\pgfqpoint{6.200000in}{2.642500in}}%
\pgfusepath{clip}%
\pgfsetroundcap%
\pgfsetroundjoin%
\pgfsetlinewidth{0.803000pt}%
\definecolor{currentstroke}{rgb}{1.000000,1.000000,1.000000}%
\pgfsetstrokecolor{currentstroke}%
\pgfsetdash{}{0pt}%
\pgfpathmoveto{\pgfqpoint{0.563921in}{1.833498in}}%
\pgfpathlineto{\pgfqpoint{6.763921in}{1.833498in}}%
\pgfusepath{stroke}%
\end{pgfscope}%
\begin{pgfscope}%
\definecolor{textcolor}{rgb}{0.150000,0.150000,0.150000}%
\pgfsetstrokecolor{textcolor}%
\pgfsetfillcolor{textcolor}%
\pgftext[x=0.289968in,y=1.780736in,left,base]{\color{textcolor}\rmfamily\fontsize{10.000000}{12.000000}\selectfont 20}%
\end{pgfscope}%
\begin{pgfscope}%
\pgfpathrectangle{\pgfqpoint{0.563921in}{0.521603in}}{\pgfqpoint{6.200000in}{2.642500in}}%
\pgfusepath{clip}%
\pgfsetroundcap%
\pgfsetroundjoin%
\pgfsetlinewidth{0.803000pt}%
\definecolor{currentstroke}{rgb}{1.000000,1.000000,1.000000}%
\pgfsetstrokecolor{currentstroke}%
\pgfsetdash{}{0pt}%
\pgfpathmoveto{\pgfqpoint{0.563921in}{2.131443in}}%
\pgfpathlineto{\pgfqpoint{6.763921in}{2.131443in}}%
\pgfusepath{stroke}%
\end{pgfscope}%
\begin{pgfscope}%
\definecolor{textcolor}{rgb}{0.150000,0.150000,0.150000}%
\pgfsetstrokecolor{textcolor}%
\pgfsetfillcolor{textcolor}%
\pgftext[x=0.289968in,y=2.078682in,left,base]{\color{textcolor}\rmfamily\fontsize{10.000000}{12.000000}\selectfont 25}%
\end{pgfscope}%
\begin{pgfscope}%
\pgfpathrectangle{\pgfqpoint{0.563921in}{0.521603in}}{\pgfqpoint{6.200000in}{2.642500in}}%
\pgfusepath{clip}%
\pgfsetroundcap%
\pgfsetroundjoin%
\pgfsetlinewidth{0.803000pt}%
\definecolor{currentstroke}{rgb}{1.000000,1.000000,1.000000}%
\pgfsetstrokecolor{currentstroke}%
\pgfsetdash{}{0pt}%
\pgfpathmoveto{\pgfqpoint{0.563921in}{2.429388in}}%
\pgfpathlineto{\pgfqpoint{6.763921in}{2.429388in}}%
\pgfusepath{stroke}%
\end{pgfscope}%
\begin{pgfscope}%
\definecolor{textcolor}{rgb}{0.150000,0.150000,0.150000}%
\pgfsetstrokecolor{textcolor}%
\pgfsetfillcolor{textcolor}%
\pgftext[x=0.289968in,y=2.376627in,left,base]{\color{textcolor}\rmfamily\fontsize{10.000000}{12.000000}\selectfont 30}%
\end{pgfscope}%
\begin{pgfscope}%
\pgfpathrectangle{\pgfqpoint{0.563921in}{0.521603in}}{\pgfqpoint{6.200000in}{2.642500in}}%
\pgfusepath{clip}%
\pgfsetroundcap%
\pgfsetroundjoin%
\pgfsetlinewidth{0.803000pt}%
\definecolor{currentstroke}{rgb}{1.000000,1.000000,1.000000}%
\pgfsetstrokecolor{currentstroke}%
\pgfsetdash{}{0pt}%
\pgfpathmoveto{\pgfqpoint{0.563921in}{2.727334in}}%
\pgfpathlineto{\pgfqpoint{6.763921in}{2.727334in}}%
\pgfusepath{stroke}%
\end{pgfscope}%
\begin{pgfscope}%
\definecolor{textcolor}{rgb}{0.150000,0.150000,0.150000}%
\pgfsetstrokecolor{textcolor}%
\pgfsetfillcolor{textcolor}%
\pgftext[x=0.289968in,y=2.674572in,left,base]{\color{textcolor}\rmfamily\fontsize{10.000000}{12.000000}\selectfont 35}%
\end{pgfscope}%
\begin{pgfscope}%
\pgfpathrectangle{\pgfqpoint{0.563921in}{0.521603in}}{\pgfqpoint{6.200000in}{2.642500in}}%
\pgfusepath{clip}%
\pgfsetroundcap%
\pgfsetroundjoin%
\pgfsetlinewidth{0.803000pt}%
\definecolor{currentstroke}{rgb}{1.000000,1.000000,1.000000}%
\pgfsetstrokecolor{currentstroke}%
\pgfsetdash{}{0pt}%
\pgfpathmoveto{\pgfqpoint{0.563921in}{3.025279in}}%
\pgfpathlineto{\pgfqpoint{6.763921in}{3.025279in}}%
\pgfusepath{stroke}%
\end{pgfscope}%
\begin{pgfscope}%
\definecolor{textcolor}{rgb}{0.150000,0.150000,0.150000}%
\pgfsetstrokecolor{textcolor}%
\pgfsetfillcolor{textcolor}%
\pgftext[x=0.289968in,y=2.972517in,left,base]{\color{textcolor}\rmfamily\fontsize{10.000000}{12.000000}\selectfont 40}%
\end{pgfscope}%
\begin{pgfscope}%
\definecolor{textcolor}{rgb}{0.150000,0.150000,0.150000}%
\pgfsetstrokecolor{textcolor}%
\pgfsetfillcolor{textcolor}%
\pgftext[x=0.234413in,y=1.842853in,,bottom,rotate=90.000000]{\color{textcolor}\rmfamily\fontsize{10.000000}{12.000000}\selectfont Standard Deviation}%
\end{pgfscope}%
\begin{pgfscope}%
\pgfpathrectangle{\pgfqpoint{0.563921in}{0.521603in}}{\pgfqpoint{6.200000in}{2.642500in}}%
\pgfusepath{clip}%
\pgfsetroundcap%
\pgfsetroundjoin%
\pgfsetlinewidth{1.505625pt}%
\definecolor{currentstroke}{rgb}{0.121569,0.466667,0.705882}%
\pgfsetstrokecolor{currentstroke}%
\pgfsetdash{}{0pt}%
\pgfpathmoveto{\pgfqpoint{0.845739in}{0.641717in}}%
\pgfpathlineto{\pgfqpoint{0.848317in}{0.658700in}}%
\pgfpathlineto{\pgfqpoint{0.850894in}{0.655601in}}%
\pgfpathlineto{\pgfqpoint{0.853471in}{0.657178in}}%
\pgfpathlineto{\pgfqpoint{0.861203in}{0.655876in}}%
\pgfpathlineto{\pgfqpoint{0.863780in}{0.658219in}}%
\pgfpathlineto{\pgfqpoint{0.866357in}{0.657038in}}%
\pgfpathlineto{\pgfqpoint{0.868934in}{0.657822in}}%
\pgfpathlineto{\pgfqpoint{0.871512in}{0.657520in}}%
\pgfpathlineto{\pgfqpoint{0.881820in}{0.657714in}}%
\pgfpathlineto{\pgfqpoint{0.884398in}{0.664412in}}%
\pgfpathlineto{\pgfqpoint{0.886975in}{0.674516in}}%
\pgfpathlineto{\pgfqpoint{0.889552in}{0.678938in}}%
\pgfpathlineto{\pgfqpoint{0.897284in}{0.681434in}}%
\pgfpathlineto{\pgfqpoint{0.899861in}{0.684693in}}%
\pgfpathlineto{\pgfqpoint{0.902438in}{0.689764in}}%
\pgfpathlineto{\pgfqpoint{0.905015in}{0.699797in}}%
\pgfpathlineto{\pgfqpoint{0.907593in}{0.705818in}}%
\pgfpathlineto{\pgfqpoint{0.920479in}{0.712517in}}%
\pgfpathlineto{\pgfqpoint{0.925633in}{0.716874in}}%
\pgfpathlineto{\pgfqpoint{0.933365in}{0.718223in}}%
\pgfpathlineto{\pgfqpoint{0.938519in}{0.721714in}}%
\pgfpathlineto{\pgfqpoint{0.941096in}{0.723123in}}%
\pgfpathlineto{\pgfqpoint{0.943674in}{0.722600in}}%
\pgfpathlineto{\pgfqpoint{0.959137in}{0.724998in}}%
\pgfpathlineto{\pgfqpoint{0.961714in}{0.725513in}}%
\pgfpathlineto{\pgfqpoint{0.974600in}{0.726569in}}%
\pgfpathlineto{\pgfqpoint{0.979754in}{0.728341in}}%
\pgfpathlineto{\pgfqpoint{0.992641in}{0.728982in}}%
\pgfpathlineto{\pgfqpoint{1.008104in}{0.727125in}}%
\pgfpathlineto{\pgfqpoint{1.015835in}{0.724771in}}%
\pgfpathlineto{\pgfqpoint{1.023567in}{0.724558in}}%
\pgfpathlineto{\pgfqpoint{1.028722in}{0.727060in}}%
\pgfpathlineto{\pgfqpoint{1.031299in}{0.730154in}}%
\pgfpathlineto{\pgfqpoint{1.033876in}{0.732095in}}%
\pgfpathlineto{\pgfqpoint{1.049339in}{0.736017in}}%
\pgfpathlineto{\pgfqpoint{1.051916in}{0.736007in}}%
\pgfpathlineto{\pgfqpoint{1.080266in}{0.738117in}}%
\pgfpathlineto{\pgfqpoint{1.085420in}{0.736871in}}%
\pgfpathlineto{\pgfqpoint{1.106038in}{0.735102in}}%
\pgfpathlineto{\pgfqpoint{1.116347in}{0.733921in}}%
\pgfpathlineto{\pgfqpoint{1.124078in}{0.732187in}}%
\pgfpathlineto{\pgfqpoint{1.136964in}{0.731630in}}%
\pgfpathlineto{\pgfqpoint{1.142119in}{0.732548in}}%
\pgfpathlineto{\pgfqpoint{1.152428in}{0.733541in}}%
\pgfpathlineto{\pgfqpoint{1.157582in}{0.734353in}}%
\pgfpathlineto{\pgfqpoint{1.170468in}{0.733340in}}%
\pgfpathlineto{\pgfqpoint{1.178200in}{0.731841in}}%
\pgfpathlineto{\pgfqpoint{1.193663in}{0.731199in}}%
\pgfpathlineto{\pgfqpoint{1.196240in}{0.731943in}}%
\pgfpathlineto{\pgfqpoint{1.214281in}{0.731766in}}%
\pgfpathlineto{\pgfqpoint{1.229744in}{0.731411in}}%
\pgfpathlineto{\pgfqpoint{1.232321in}{0.732498in}}%
\pgfpathlineto{\pgfqpoint{1.240053in}{0.733725in}}%
\pgfpathlineto{\pgfqpoint{1.242630in}{0.734979in}}%
\pgfpathlineto{\pgfqpoint{1.258093in}{0.733729in}}%
\pgfpathlineto{\pgfqpoint{1.268402in}{0.732423in}}%
\pgfpathlineto{\pgfqpoint{1.283866in}{0.731618in}}%
\pgfpathlineto{\pgfqpoint{1.286443in}{0.731290in}}%
\pgfpathlineto{\pgfqpoint{1.304483in}{0.730991in}}%
\pgfpathlineto{\pgfqpoint{1.312215in}{0.731656in}}%
\pgfpathlineto{\pgfqpoint{1.314792in}{0.732580in}}%
\pgfpathlineto{\pgfqpoint{1.322524in}{0.733766in}}%
\pgfpathlineto{\pgfqpoint{1.335410in}{0.734017in}}%
\pgfpathlineto{\pgfqpoint{1.340564in}{0.733473in}}%
\pgfpathlineto{\pgfqpoint{1.350873in}{0.733751in}}%
\pgfpathlineto{\pgfqpoint{1.356027in}{0.736834in}}%
\pgfpathlineto{\pgfqpoint{1.358605in}{0.737646in}}%
\pgfpathlineto{\pgfqpoint{1.371491in}{0.738087in}}%
\pgfpathlineto{\pgfqpoint{1.374068in}{0.739209in}}%
\pgfpathlineto{\pgfqpoint{1.376645in}{0.741173in}}%
\pgfpathlineto{\pgfqpoint{1.384377in}{0.742688in}}%
\pgfpathlineto{\pgfqpoint{1.389531in}{0.745424in}}%
\pgfpathlineto{\pgfqpoint{1.392108in}{0.745986in}}%
\pgfpathlineto{\pgfqpoint{1.394686in}{0.747591in}}%
\pgfpathlineto{\pgfqpoint{1.402417in}{0.748914in}}%
\pgfpathlineto{\pgfqpoint{1.412726in}{0.754756in}}%
\pgfpathlineto{\pgfqpoint{1.420458in}{0.756498in}}%
\pgfpathlineto{\pgfqpoint{1.425612in}{0.759785in}}%
\pgfpathlineto{\pgfqpoint{1.430767in}{0.765275in}}%
\pgfpathlineto{\pgfqpoint{1.438498in}{0.767742in}}%
\pgfpathlineto{\pgfqpoint{1.448807in}{0.774412in}}%
\pgfpathlineto{\pgfqpoint{1.456539in}{0.775953in}}%
\pgfpathlineto{\pgfqpoint{1.466848in}{0.780839in}}%
\pgfpathlineto{\pgfqpoint{1.477156in}{0.781629in}}%
\pgfpathlineto{\pgfqpoint{1.479734in}{0.782426in}}%
\pgfpathlineto{\pgfqpoint{1.484888in}{0.785367in}}%
\pgfpathlineto{\pgfqpoint{1.497774in}{0.786314in}}%
\pgfpathlineto{\pgfqpoint{1.500351in}{0.787103in}}%
\pgfpathlineto{\pgfqpoint{1.502929in}{0.788951in}}%
\pgfpathlineto{\pgfqpoint{1.510660in}{0.790605in}}%
\pgfpathlineto{\pgfqpoint{1.520969in}{0.795969in}}%
\pgfpathlineto{\pgfqpoint{1.531278in}{0.798212in}}%
\pgfpathlineto{\pgfqpoint{1.539010in}{0.800370in}}%
\pgfpathlineto{\pgfqpoint{1.546741in}{0.801352in}}%
\pgfpathlineto{\pgfqpoint{1.554473in}{0.805040in}}%
\pgfpathlineto{\pgfqpoint{1.557050in}{0.806751in}}%
\pgfpathlineto{\pgfqpoint{1.575091in}{0.811491in}}%
\pgfpathlineto{\pgfqpoint{1.582822in}{0.811984in}}%
\pgfpathlineto{\pgfqpoint{1.593131in}{0.816090in}}%
\pgfpathlineto{\pgfqpoint{1.603440in}{0.815990in}}%
\pgfpathlineto{\pgfqpoint{1.611172in}{0.814736in}}%
\pgfpathlineto{\pgfqpoint{1.626635in}{0.813970in}}%
\pgfpathlineto{\pgfqpoint{1.629212in}{0.813594in}}%
\pgfpathlineto{\pgfqpoint{1.644675in}{0.812425in}}%
\pgfpathlineto{\pgfqpoint{1.647253in}{0.812050in}}%
\pgfpathlineto{\pgfqpoint{1.660139in}{0.810978in}}%
\pgfpathlineto{\pgfqpoint{1.665293in}{0.810218in}}%
\pgfpathlineto{\pgfqpoint{1.716837in}{0.808676in}}%
\pgfpathlineto{\pgfqpoint{1.729723in}{0.809852in}}%
\pgfpathlineto{\pgfqpoint{1.737455in}{0.811119in}}%
\pgfpathlineto{\pgfqpoint{1.745187in}{0.811641in}}%
\pgfpathlineto{\pgfqpoint{1.755495in}{0.814283in}}%
\pgfpathlineto{\pgfqpoint{1.781268in}{0.816010in}}%
\pgfpathlineto{\pgfqpoint{1.788999in}{0.817960in}}%
\pgfpathlineto{\pgfqpoint{1.791576in}{0.819160in}}%
\pgfpathlineto{\pgfqpoint{1.799308in}{0.820379in}}%
\pgfpathlineto{\pgfqpoint{1.804462in}{0.823129in}}%
\pgfpathlineto{\pgfqpoint{1.809617in}{0.826288in}}%
\pgfpathlineto{\pgfqpoint{1.817349in}{0.828031in}}%
\pgfpathlineto{\pgfqpoint{1.825080in}{0.833800in}}%
\pgfpathlineto{\pgfqpoint{1.827657in}{0.836154in}}%
\pgfpathlineto{\pgfqpoint{1.837966in}{0.838751in}}%
\pgfpathlineto{\pgfqpoint{1.845698in}{0.847037in}}%
\pgfpathlineto{\pgfqpoint{1.853430in}{0.850028in}}%
\pgfpathlineto{\pgfqpoint{1.858584in}{0.856477in}}%
\pgfpathlineto{\pgfqpoint{1.863738in}{0.862285in}}%
\pgfpathlineto{\pgfqpoint{1.871470in}{0.864917in}}%
\pgfpathlineto{\pgfqpoint{1.879202in}{0.874458in}}%
\pgfpathlineto{\pgfqpoint{1.881779in}{0.877728in}}%
\pgfpathlineto{\pgfqpoint{1.889510in}{0.880885in}}%
\pgfpathlineto{\pgfqpoint{1.899819in}{0.894540in}}%
\pgfpathlineto{\pgfqpoint{1.910128in}{0.898382in}}%
\pgfpathlineto{\pgfqpoint{1.917860in}{0.907689in}}%
\pgfpathlineto{\pgfqpoint{1.925591in}{0.909998in}}%
\pgfpathlineto{\pgfqpoint{1.930746in}{0.915511in}}%
\pgfpathlineto{\pgfqpoint{1.935900in}{0.921600in}}%
\pgfpathlineto{\pgfqpoint{1.943632in}{0.924257in}}%
\pgfpathlineto{\pgfqpoint{1.953941in}{0.937004in}}%
\pgfpathlineto{\pgfqpoint{1.961672in}{0.940440in}}%
\pgfpathlineto{\pgfqpoint{1.971981in}{0.953094in}}%
\pgfpathlineto{\pgfqpoint{1.979713in}{0.955958in}}%
\pgfpathlineto{\pgfqpoint{1.990022in}{0.967013in}}%
\pgfpathlineto{\pgfqpoint{1.997753in}{0.969468in}}%
\pgfpathlineto{\pgfqpoint{2.005485in}{0.977360in}}%
\pgfpathlineto{\pgfqpoint{2.015794in}{0.979781in}}%
\pgfpathlineto{\pgfqpoint{2.023526in}{0.987149in}}%
\pgfpathlineto{\pgfqpoint{2.026103in}{0.989415in}}%
\pgfpathlineto{\pgfqpoint{2.033834in}{0.991626in}}%
\pgfpathlineto{\pgfqpoint{2.038989in}{0.996708in}}%
\pgfpathlineto{\pgfqpoint{2.044143in}{1.002423in}}%
\pgfpathlineto{\pgfqpoint{2.051875in}{1.004399in}}%
\pgfpathlineto{\pgfqpoint{2.059606in}{1.009888in}}%
\pgfpathlineto{\pgfqpoint{2.062184in}{1.011699in}}%
\pgfpathlineto{\pgfqpoint{2.069915in}{1.013506in}}%
\pgfpathlineto{\pgfqpoint{2.075070in}{1.018226in}}%
\pgfpathlineto{\pgfqpoint{2.080224in}{1.020745in}}%
\pgfpathlineto{\pgfqpoint{2.087956in}{1.021834in}}%
\pgfpathlineto{\pgfqpoint{2.095687in}{1.026012in}}%
\pgfpathlineto{\pgfqpoint{2.098265in}{1.028180in}}%
\pgfpathlineto{\pgfqpoint{2.105996in}{1.030307in}}%
\pgfpathlineto{\pgfqpoint{2.113728in}{1.037323in}}%
\pgfpathlineto{\pgfqpoint{2.116305in}{1.040177in}}%
\pgfpathlineto{\pgfqpoint{2.124037in}{1.042983in}}%
\pgfpathlineto{\pgfqpoint{2.134346in}{1.054545in}}%
\pgfpathlineto{\pgfqpoint{2.142077in}{1.057500in}}%
\pgfpathlineto{\pgfqpoint{2.149809in}{1.065793in}}%
\pgfpathlineto{\pgfqpoint{2.152386in}{1.068257in}}%
\pgfpathlineto{\pgfqpoint{2.162695in}{1.071116in}}%
\pgfpathlineto{\pgfqpoint{2.170427in}{1.078742in}}%
\pgfpathlineto{\pgfqpoint{2.178158in}{1.081084in}}%
\pgfpathlineto{\pgfqpoint{2.183313in}{1.084874in}}%
\pgfpathlineto{\pgfqpoint{2.185890in}{1.086540in}}%
\pgfpathlineto{\pgfqpoint{2.188467in}{1.088887in}}%
\pgfpathlineto{\pgfqpoint{2.196199in}{1.091101in}}%
\pgfpathlineto{\pgfqpoint{2.206508in}{1.099103in}}%
\pgfpathlineto{\pgfqpoint{2.214239in}{1.101466in}}%
\pgfpathlineto{\pgfqpoint{2.219394in}{1.106293in}}%
\pgfpathlineto{\pgfqpoint{2.224548in}{1.109252in}}%
\pgfpathlineto{\pgfqpoint{2.232280in}{1.110295in}}%
\pgfpathlineto{\pgfqpoint{2.237434in}{1.113014in}}%
\pgfpathlineto{\pgfqpoint{2.242589in}{1.116052in}}%
\pgfpathlineto{\pgfqpoint{2.250320in}{1.117414in}}%
\pgfpathlineto{\pgfqpoint{2.260629in}{1.121831in}}%
\pgfpathlineto{\pgfqpoint{2.268361in}{1.123822in}}%
\pgfpathlineto{\pgfqpoint{2.278670in}{1.133675in}}%
\pgfpathlineto{\pgfqpoint{2.286401in}{1.136247in}}%
\pgfpathlineto{\pgfqpoint{2.296710in}{1.146498in}}%
\pgfpathlineto{\pgfqpoint{2.304442in}{1.149278in}}%
\pgfpathlineto{\pgfqpoint{2.314751in}{1.160362in}}%
\pgfpathlineto{\pgfqpoint{2.322482in}{1.163013in}}%
\pgfpathlineto{\pgfqpoint{2.332791in}{1.174477in}}%
\pgfpathlineto{\pgfqpoint{2.340523in}{1.177277in}}%
\pgfpathlineto{\pgfqpoint{2.350832in}{1.188218in}}%
\pgfpathlineto{\pgfqpoint{2.358563in}{1.190929in}}%
\pgfpathlineto{\pgfqpoint{2.363718in}{1.196094in}}%
\pgfpathlineto{\pgfqpoint{2.368872in}{1.199933in}}%
\pgfpathlineto{\pgfqpoint{2.376604in}{1.201747in}}%
\pgfpathlineto{\pgfqpoint{2.386912in}{1.208141in}}%
\pgfpathlineto{\pgfqpoint{2.394644in}{1.209683in}}%
\pgfpathlineto{\pgfqpoint{2.404953in}{1.214649in}}%
\pgfpathlineto{\pgfqpoint{2.415262in}{1.215858in}}%
\pgfpathlineto{\pgfqpoint{2.422993in}{1.220383in}}%
\pgfpathlineto{\pgfqpoint{2.430725in}{1.222281in}}%
\pgfpathlineto{\pgfqpoint{2.441034in}{1.231063in}}%
\pgfpathlineto{\pgfqpoint{2.448766in}{1.233412in}}%
\pgfpathlineto{\pgfqpoint{2.459074in}{1.243965in}}%
\pgfpathlineto{\pgfqpoint{2.466806in}{1.246609in}}%
\pgfpathlineto{\pgfqpoint{2.477115in}{1.256172in}}%
\pgfpathlineto{\pgfqpoint{2.484847in}{1.258227in}}%
\pgfpathlineto{\pgfqpoint{2.495155in}{1.265972in}}%
\pgfpathlineto{\pgfqpoint{2.502887in}{1.267732in}}%
\pgfpathlineto{\pgfqpoint{2.510619in}{1.272465in}}%
\pgfpathlineto{\pgfqpoint{2.513196in}{1.274580in}}%
\pgfpathlineto{\pgfqpoint{2.520928in}{1.276824in}}%
\pgfpathlineto{\pgfqpoint{2.526082in}{1.280761in}}%
\pgfpathlineto{\pgfqpoint{2.531236in}{1.285663in}}%
\pgfpathlineto{\pgfqpoint{2.538968in}{1.288189in}}%
\pgfpathlineto{\pgfqpoint{2.549277in}{1.298399in}}%
\pgfpathlineto{\pgfqpoint{2.557009in}{1.301151in}}%
\pgfpathlineto{\pgfqpoint{2.567317in}{1.312301in}}%
\pgfpathlineto{\pgfqpoint{2.575049in}{1.315202in}}%
\pgfpathlineto{\pgfqpoint{2.585358in}{1.326996in}}%
\pgfpathlineto{\pgfqpoint{2.593089in}{1.330097in}}%
\pgfpathlineto{\pgfqpoint{2.603398in}{1.343264in}}%
\pgfpathlineto{\pgfqpoint{2.611130in}{1.346687in}}%
\pgfpathlineto{\pgfqpoint{2.621439in}{1.360419in}}%
\pgfpathlineto{\pgfqpoint{2.629170in}{1.364046in}}%
\pgfpathlineto{\pgfqpoint{2.634325in}{1.371986in}}%
\pgfpathlineto{\pgfqpoint{2.639479in}{1.376051in}}%
\pgfpathlineto{\pgfqpoint{2.647211in}{1.378627in}}%
\pgfpathlineto{\pgfqpoint{2.657520in}{1.388132in}}%
\pgfpathlineto{\pgfqpoint{2.665251in}{1.390745in}}%
\pgfpathlineto{\pgfqpoint{2.672983in}{1.397367in}}%
\pgfpathlineto{\pgfqpoint{2.675560in}{1.399410in}}%
\pgfpathlineto{\pgfqpoint{2.683292in}{1.401679in}}%
\pgfpathlineto{\pgfqpoint{2.685869in}{1.404740in}}%
\pgfpathlineto{\pgfqpoint{2.693601in}{1.417294in}}%
\pgfpathlineto{\pgfqpoint{2.701332in}{1.421488in}}%
\pgfpathlineto{\pgfqpoint{2.703910in}{1.425671in}}%
\pgfpathlineto{\pgfqpoint{2.709064in}{1.430137in}}%
\pgfpathlineto{\pgfqpoint{2.711641in}{1.434823in}}%
\pgfpathlineto{\pgfqpoint{2.719373in}{1.439459in}}%
\pgfpathlineto{\pgfqpoint{2.721950in}{1.444253in}}%
\pgfpathlineto{\pgfqpoint{2.727105in}{1.448421in}}%
\pgfpathlineto{\pgfqpoint{2.729682in}{1.452614in}}%
\pgfpathlineto{\pgfqpoint{2.737413in}{1.456544in}}%
\pgfpathlineto{\pgfqpoint{2.745145in}{1.467506in}}%
\pgfpathlineto{\pgfqpoint{2.747722in}{1.470895in}}%
\pgfpathlineto{\pgfqpoint{2.755454in}{1.473906in}}%
\pgfpathlineto{\pgfqpoint{2.765763in}{1.488379in}}%
\pgfpathlineto{\pgfqpoint{2.776072in}{1.491689in}}%
\pgfpathlineto{\pgfqpoint{2.781226in}{1.497587in}}%
\pgfpathlineto{\pgfqpoint{2.783803in}{1.499431in}}%
\pgfpathlineto{\pgfqpoint{2.791535in}{1.501034in}}%
\pgfpathlineto{\pgfqpoint{2.801844in}{1.507354in}}%
\pgfpathlineto{\pgfqpoint{2.809575in}{1.508102in}}%
\pgfpathlineto{\pgfqpoint{2.817307in}{1.511937in}}%
\pgfpathlineto{\pgfqpoint{2.819884in}{1.513618in}}%
\pgfpathlineto{\pgfqpoint{2.827616in}{1.515174in}}%
\pgfpathlineto{\pgfqpoint{2.837925in}{1.522484in}}%
\pgfpathlineto{\pgfqpoint{2.848234in}{1.524451in}}%
\pgfpathlineto{\pgfqpoint{2.855965in}{1.529924in}}%
\pgfpathlineto{\pgfqpoint{2.863697in}{1.531877in}}%
\pgfpathlineto{\pgfqpoint{2.874006in}{1.540551in}}%
\pgfpathlineto{\pgfqpoint{2.881737in}{1.542394in}}%
\pgfpathlineto{\pgfqpoint{2.892046in}{1.550578in}}%
\pgfpathlineto{\pgfqpoint{2.899778in}{1.552544in}}%
\pgfpathlineto{\pgfqpoint{2.907509in}{1.557525in}}%
\pgfpathlineto{\pgfqpoint{2.910087in}{1.558814in}}%
\pgfpathlineto{\pgfqpoint{2.917818in}{1.560471in}}%
\pgfpathlineto{\pgfqpoint{2.928127in}{1.566996in}}%
\pgfpathlineto{\pgfqpoint{2.935859in}{1.568589in}}%
\pgfpathlineto{\pgfqpoint{2.946168in}{1.575510in}}%
\pgfpathlineto{\pgfqpoint{2.953899in}{1.577545in}}%
\pgfpathlineto{\pgfqpoint{2.964208in}{1.585789in}}%
\pgfpathlineto{\pgfqpoint{2.971940in}{1.587511in}}%
\pgfpathlineto{\pgfqpoint{2.982249in}{1.594143in}}%
\pgfpathlineto{\pgfqpoint{2.989980in}{1.595586in}}%
\pgfpathlineto{\pgfqpoint{2.995135in}{1.599136in}}%
\pgfpathlineto{\pgfqpoint{2.997712in}{1.601257in}}%
\pgfpathlineto{\pgfqpoint{3.008021in}{1.603509in}}%
\pgfpathlineto{\pgfqpoint{3.015752in}{1.609736in}}%
\pgfpathlineto{\pgfqpoint{3.018330in}{1.611559in}}%
\pgfpathlineto{\pgfqpoint{3.026061in}{1.613469in}}%
\pgfpathlineto{\pgfqpoint{3.031216in}{1.617589in}}%
\pgfpathlineto{\pgfqpoint{3.036370in}{1.622345in}}%
\pgfpathlineto{\pgfqpoint{3.044102in}{1.624712in}}%
\pgfpathlineto{\pgfqpoint{3.051833in}{1.631571in}}%
\pgfpathlineto{\pgfqpoint{3.054411in}{1.633966in}}%
\pgfpathlineto{\pgfqpoint{3.062142in}{1.636595in}}%
\pgfpathlineto{\pgfqpoint{3.072451in}{1.645965in}}%
\pgfpathlineto{\pgfqpoint{3.080183in}{1.648270in}}%
\pgfpathlineto{\pgfqpoint{3.090491in}{1.656975in}}%
\pgfpathlineto{\pgfqpoint{3.100800in}{1.659273in}}%
\pgfpathlineto{\pgfqpoint{3.108532in}{1.666372in}}%
\pgfpathlineto{\pgfqpoint{3.116264in}{1.668722in}}%
\pgfpathlineto{\pgfqpoint{3.126572in}{1.678602in}}%
\pgfpathlineto{\pgfqpoint{3.134304in}{1.681342in}}%
\pgfpathlineto{\pgfqpoint{3.142036in}{1.688802in}}%
\pgfpathlineto{\pgfqpoint{3.144613in}{1.691108in}}%
\pgfpathlineto{\pgfqpoint{3.152345in}{1.693385in}}%
\pgfpathlineto{\pgfqpoint{3.162653in}{1.702981in}}%
\pgfpathlineto{\pgfqpoint{3.170385in}{1.705268in}}%
\pgfpathlineto{\pgfqpoint{3.180694in}{1.713851in}}%
\pgfpathlineto{\pgfqpoint{3.188426in}{1.715898in}}%
\pgfpathlineto{\pgfqpoint{3.196157in}{1.722774in}}%
\pgfpathlineto{\pgfqpoint{3.206466in}{1.724998in}}%
\pgfpathlineto{\pgfqpoint{3.216775in}{1.733346in}}%
\pgfpathlineto{\pgfqpoint{3.224507in}{1.735481in}}%
\pgfpathlineto{\pgfqpoint{3.234815in}{1.743768in}}%
\pgfpathlineto{\pgfqpoint{3.242547in}{1.745701in}}%
\pgfpathlineto{\pgfqpoint{3.252856in}{1.753671in}}%
\pgfpathlineto{\pgfqpoint{3.260588in}{1.755693in}}%
\pgfpathlineto{\pgfqpoint{3.268319in}{1.760537in}}%
\pgfpathlineto{\pgfqpoint{3.270896in}{1.761756in}}%
\pgfpathlineto{\pgfqpoint{3.278628in}{1.763044in}}%
\pgfpathlineto{\pgfqpoint{3.288937in}{1.767627in}}%
\pgfpathlineto{\pgfqpoint{3.296668in}{1.768837in}}%
\pgfpathlineto{\pgfqpoint{3.306977in}{1.774142in}}%
\pgfpathlineto{\pgfqpoint{3.314709in}{1.775748in}}%
\pgfpathlineto{\pgfqpoint{3.325018in}{1.782505in}}%
\pgfpathlineto{\pgfqpoint{3.332749in}{1.784231in}}%
\pgfpathlineto{\pgfqpoint{3.343058in}{1.790674in}}%
\pgfpathlineto{\pgfqpoint{3.353367in}{1.792260in}}%
\pgfpathlineto{\pgfqpoint{3.361099in}{1.796801in}}%
\pgfpathlineto{\pgfqpoint{3.368830in}{1.798420in}}%
\pgfpathlineto{\pgfqpoint{3.379139in}{1.804474in}}%
\pgfpathlineto{\pgfqpoint{3.386871in}{1.805971in}}%
\pgfpathlineto{\pgfqpoint{3.397180in}{1.812586in}}%
\pgfpathlineto{\pgfqpoint{3.404911in}{1.814173in}}%
\pgfpathlineto{\pgfqpoint{3.415220in}{1.819312in}}%
\pgfpathlineto{\pgfqpoint{3.422952in}{1.820408in}}%
\pgfpathlineto{\pgfqpoint{3.428106in}{1.822182in}}%
\pgfpathlineto{\pgfqpoint{3.433261in}{1.823707in}}%
\pgfpathlineto{\pgfqpoint{3.443570in}{1.825122in}}%
\pgfpathlineto{\pgfqpoint{3.448724in}{1.826720in}}%
\pgfpathlineto{\pgfqpoint{3.479651in}{1.829627in}}%
\pgfpathlineto{\pgfqpoint{3.482228in}{1.830257in}}%
\pgfpathlineto{\pgfqpoint{3.487382in}{1.833254in}}%
\pgfpathlineto{\pgfqpoint{3.495114in}{1.835079in}}%
\pgfpathlineto{\pgfqpoint{3.505423in}{1.843526in}}%
\pgfpathlineto{\pgfqpoint{3.513154in}{1.845774in}}%
\pgfpathlineto{\pgfqpoint{3.523463in}{1.856025in}}%
\pgfpathlineto{\pgfqpoint{3.531195in}{1.858804in}}%
\pgfpathlineto{\pgfqpoint{3.541504in}{1.869885in}}%
\pgfpathlineto{\pgfqpoint{3.549235in}{1.872699in}}%
\pgfpathlineto{\pgfqpoint{3.559544in}{1.884664in}}%
\pgfpathlineto{\pgfqpoint{3.567276in}{1.887662in}}%
\pgfpathlineto{\pgfqpoint{3.572430in}{1.893076in}}%
\pgfpathlineto{\pgfqpoint{3.577585in}{1.896032in}}%
\pgfpathlineto{\pgfqpoint{3.585316in}{1.898676in}}%
\pgfpathlineto{\pgfqpoint{3.595625in}{1.911225in}}%
\pgfpathlineto{\pgfqpoint{3.603357in}{1.914156in}}%
\pgfpathlineto{\pgfqpoint{3.611088in}{1.922138in}}%
\pgfpathlineto{\pgfqpoint{3.613666in}{1.924433in}}%
\pgfpathlineto{\pgfqpoint{3.621397in}{1.926673in}}%
\pgfpathlineto{\pgfqpoint{3.626552in}{1.931922in}}%
\pgfpathlineto{\pgfqpoint{3.631706in}{1.938739in}}%
\pgfpathlineto{\pgfqpoint{3.639438in}{1.942404in}}%
\pgfpathlineto{\pgfqpoint{3.644592in}{1.949523in}}%
\pgfpathlineto{\pgfqpoint{3.649747in}{1.952934in}}%
\pgfpathlineto{\pgfqpoint{3.657478in}{1.956386in}}%
\pgfpathlineto{\pgfqpoint{3.662633in}{1.962703in}}%
\pgfpathlineto{\pgfqpoint{3.667787in}{1.965671in}}%
\pgfpathlineto{\pgfqpoint{3.675519in}{1.968093in}}%
\pgfpathlineto{\pgfqpoint{3.680673in}{1.972577in}}%
\pgfpathlineto{\pgfqpoint{3.685828in}{1.977910in}}%
\pgfpathlineto{\pgfqpoint{3.693559in}{1.980286in}}%
\pgfpathlineto{\pgfqpoint{3.703868in}{1.989496in}}%
\pgfpathlineto{\pgfqpoint{3.714177in}{1.991939in}}%
\pgfpathlineto{\pgfqpoint{3.721909in}{2.000056in}}%
\pgfpathlineto{\pgfqpoint{3.729640in}{2.002728in}}%
\pgfpathlineto{\pgfqpoint{3.739949in}{2.013086in}}%
\pgfpathlineto{\pgfqpoint{3.747681in}{2.015678in}}%
\pgfpathlineto{\pgfqpoint{3.757990in}{2.026615in}}%
\pgfpathlineto{\pgfqpoint{3.765721in}{2.029161in}}%
\pgfpathlineto{\pgfqpoint{3.776030in}{2.039926in}}%
\pgfpathlineto{\pgfqpoint{3.786339in}{2.042789in}}%
\pgfpathlineto{\pgfqpoint{3.794070in}{2.051498in}}%
\pgfpathlineto{\pgfqpoint{3.801802in}{2.054533in}}%
\pgfpathlineto{\pgfqpoint{3.812111in}{2.066582in}}%
\pgfpathlineto{\pgfqpoint{3.819843in}{2.069735in}}%
\pgfpathlineto{\pgfqpoint{3.827574in}{2.077884in}}%
\pgfpathlineto{\pgfqpoint{3.830151in}{2.080133in}}%
\pgfpathlineto{\pgfqpoint{3.837883in}{2.082620in}}%
\pgfpathlineto{\pgfqpoint{3.848192in}{2.090630in}}%
\pgfpathlineto{\pgfqpoint{3.855924in}{2.093010in}}%
\pgfpathlineto{\pgfqpoint{3.866232in}{2.102112in}}%
\pgfpathlineto{\pgfqpoint{3.873964in}{2.104395in}}%
\pgfpathlineto{\pgfqpoint{3.879118in}{2.108528in}}%
\pgfpathlineto{\pgfqpoint{3.884273in}{2.112252in}}%
\pgfpathlineto{\pgfqpoint{3.892005in}{2.114447in}}%
\pgfpathlineto{\pgfqpoint{3.899736in}{2.120020in}}%
\pgfpathlineto{\pgfqpoint{3.910045in}{2.122087in}}%
\pgfpathlineto{\pgfqpoint{3.920354in}{2.130723in}}%
\pgfpathlineto{\pgfqpoint{3.928086in}{2.132754in}}%
\pgfpathlineto{\pgfqpoint{3.938394in}{2.140376in}}%
\pgfpathlineto{\pgfqpoint{3.946126in}{2.142186in}}%
\pgfpathlineto{\pgfqpoint{3.953858in}{2.146995in}}%
\pgfpathlineto{\pgfqpoint{3.956435in}{2.148184in}}%
\pgfpathlineto{\pgfqpoint{3.964167in}{2.149268in}}%
\pgfpathlineto{\pgfqpoint{3.974475in}{2.153286in}}%
\pgfpathlineto{\pgfqpoint{3.982207in}{2.154429in}}%
\pgfpathlineto{\pgfqpoint{3.992516in}{2.158796in}}%
\pgfpathlineto{\pgfqpoint{4.000247in}{2.159981in}}%
\pgfpathlineto{\pgfqpoint{4.007979in}{2.163958in}}%
\pgfpathlineto{\pgfqpoint{4.010556in}{2.165443in}}%
\pgfpathlineto{\pgfqpoint{4.018288in}{2.166873in}}%
\pgfpathlineto{\pgfqpoint{4.028597in}{2.172459in}}%
\pgfpathlineto{\pgfqpoint{4.038906in}{2.173619in}}%
\pgfpathlineto{\pgfqpoint{4.046637in}{2.177261in}}%
\pgfpathlineto{\pgfqpoint{4.054369in}{2.178330in}}%
\pgfpathlineto{\pgfqpoint{4.064678in}{2.182422in}}%
\pgfpathlineto{\pgfqpoint{4.072409in}{2.183241in}}%
\pgfpathlineto{\pgfqpoint{4.082718in}{2.187121in}}%
\pgfpathlineto{\pgfqpoint{4.093027in}{2.188618in}}%
\pgfpathlineto{\pgfqpoint{4.100759in}{2.191422in}}%
\pgfpathlineto{\pgfqpoint{4.108490in}{2.192446in}}%
\pgfpathlineto{\pgfqpoint{4.116222in}{2.194981in}}%
\pgfpathlineto{\pgfqpoint{4.118799in}{2.195753in}}%
\pgfpathlineto{\pgfqpoint{4.129108in}{2.196805in}}%
\pgfpathlineto{\pgfqpoint{4.134263in}{2.198054in}}%
\pgfpathlineto{\pgfqpoint{4.147149in}{2.199244in}}%
\pgfpathlineto{\pgfqpoint{4.170343in}{2.203350in}}%
\pgfpathlineto{\pgfqpoint{4.172921in}{2.204016in}}%
\pgfpathlineto{\pgfqpoint{4.183230in}{2.205294in}}%
\pgfpathlineto{\pgfqpoint{4.188384in}{2.205975in}}%
\pgfpathlineto{\pgfqpoint{4.245083in}{2.208071in}}%
\pgfpathlineto{\pgfqpoint{4.260546in}{2.207880in}}%
\pgfpathlineto{\pgfqpoint{4.281164in}{2.206115in}}%
\pgfpathlineto{\pgfqpoint{4.291472in}{2.205406in}}%
\pgfpathlineto{\pgfqpoint{4.299204in}{2.204415in}}%
\pgfpathlineto{\pgfqpoint{4.312090in}{2.203866in}}%
\pgfpathlineto{\pgfqpoint{4.317245in}{2.203206in}}%
\pgfpathlineto{\pgfqpoint{4.332708in}{2.202238in}}%
\pgfpathlineto{\pgfqpoint{4.335285in}{2.201844in}}%
\pgfpathlineto{\pgfqpoint{4.345594in}{2.201009in}}%
\pgfpathlineto{\pgfqpoint{4.353326in}{2.199665in}}%
\pgfpathlineto{\pgfqpoint{4.368789in}{2.198263in}}%
\pgfpathlineto{\pgfqpoint{4.371366in}{2.198028in}}%
\pgfpathlineto{\pgfqpoint{4.399715in}{2.198265in}}%
\pgfpathlineto{\pgfqpoint{4.420333in}{2.198571in}}%
\pgfpathlineto{\pgfqpoint{4.425488in}{2.199567in}}%
\pgfpathlineto{\pgfqpoint{4.435796in}{2.200653in}}%
\pgfpathlineto{\pgfqpoint{4.443528in}{2.202470in}}%
\pgfpathlineto{\pgfqpoint{4.453837in}{2.203916in}}%
\pgfpathlineto{\pgfqpoint{4.461569in}{2.205983in}}%
\pgfpathlineto{\pgfqpoint{4.471877in}{2.207112in}}%
\pgfpathlineto{\pgfqpoint{4.479609in}{2.208666in}}%
\pgfpathlineto{\pgfqpoint{4.489918in}{2.209767in}}%
\pgfpathlineto{\pgfqpoint{4.497649in}{2.211751in}}%
\pgfpathlineto{\pgfqpoint{4.507958in}{2.213045in}}%
\pgfpathlineto{\pgfqpoint{4.510536in}{2.213642in}}%
\pgfpathlineto{\pgfqpoint{4.549194in}{2.218805in}}%
\pgfpathlineto{\pgfqpoint{4.559503in}{2.219716in}}%
\pgfpathlineto{\pgfqpoint{4.585275in}{2.219751in}}%
\pgfpathlineto{\pgfqpoint{4.618778in}{2.219838in}}%
\pgfpathlineto{\pgfqpoint{4.623933in}{2.219046in}}%
\pgfpathlineto{\pgfqpoint{4.634242in}{2.218248in}}%
\pgfpathlineto{\pgfqpoint{4.641973in}{2.216932in}}%
\pgfpathlineto{\pgfqpoint{4.654859in}{2.215929in}}%
\pgfpathlineto{\pgfqpoint{4.660014in}{2.214991in}}%
\pgfpathlineto{\pgfqpoint{4.678054in}{2.214024in}}%
\pgfpathlineto{\pgfqpoint{4.696095in}{2.214471in}}%
\pgfpathlineto{\pgfqpoint{4.711558in}{2.215300in}}%
\pgfpathlineto{\pgfqpoint{4.714135in}{2.215597in}}%
\pgfpathlineto{\pgfqpoint{4.727021in}{2.216453in}}%
\pgfpathlineto{\pgfqpoint{4.732176in}{2.217367in}}%
\pgfpathlineto{\pgfqpoint{4.745062in}{2.218798in}}%
\pgfpathlineto{\pgfqpoint{4.750216in}{2.219952in}}%
\pgfpathlineto{\pgfqpoint{4.760525in}{2.221038in}}%
\pgfpathlineto{\pgfqpoint{4.768257in}{2.222925in}}%
\pgfpathlineto{\pgfqpoint{4.778566in}{2.224272in}}%
\pgfpathlineto{\pgfqpoint{4.786297in}{2.226311in}}%
\pgfpathlineto{\pgfqpoint{4.796606in}{2.227865in}}%
\pgfpathlineto{\pgfqpoint{4.804338in}{2.230584in}}%
\pgfpathlineto{\pgfqpoint{4.812069in}{2.231545in}}%
\pgfpathlineto{\pgfqpoint{4.819801in}{2.234326in}}%
\pgfpathlineto{\pgfqpoint{4.830110in}{2.235378in}}%
\pgfpathlineto{\pgfqpoint{4.840419in}{2.239544in}}%
\pgfpathlineto{\pgfqpoint{4.848150in}{2.240576in}}%
\pgfpathlineto{\pgfqpoint{4.858459in}{2.244706in}}%
\pgfpathlineto{\pgfqpoint{4.866191in}{2.245723in}}%
\pgfpathlineto{\pgfqpoint{4.876500in}{2.250301in}}%
\pgfpathlineto{\pgfqpoint{4.884231in}{2.251524in}}%
\pgfpathlineto{\pgfqpoint{4.894540in}{2.256162in}}%
\pgfpathlineto{\pgfqpoint{4.904849in}{2.258178in}}%
\pgfpathlineto{\pgfqpoint{4.912581in}{2.261179in}}%
\pgfpathlineto{\pgfqpoint{4.920312in}{2.262242in}}%
\pgfpathlineto{\pgfqpoint{4.930621in}{2.266317in}}%
\pgfpathlineto{\pgfqpoint{4.938353in}{2.267366in}}%
\pgfpathlineto{\pgfqpoint{4.948662in}{2.271846in}}%
\pgfpathlineto{\pgfqpoint{4.956393in}{2.272944in}}%
\pgfpathlineto{\pgfqpoint{4.966702in}{2.276512in}}%
\pgfpathlineto{\pgfqpoint{4.974434in}{2.277406in}}%
\pgfpathlineto{\pgfqpoint{4.984743in}{2.281842in}}%
\pgfpathlineto{\pgfqpoint{4.995051in}{2.282874in}}%
\pgfpathlineto{\pgfqpoint{5.002783in}{2.285949in}}%
\pgfpathlineto{\pgfqpoint{5.010515in}{2.287095in}}%
\pgfpathlineto{\pgfqpoint{5.020824in}{2.291678in}}%
\pgfpathlineto{\pgfqpoint{5.028555in}{2.292570in}}%
\pgfpathlineto{\pgfqpoint{5.038864in}{2.296518in}}%
\pgfpathlineto{\pgfqpoint{5.046596in}{2.297696in}}%
\pgfpathlineto{\pgfqpoint{5.056905in}{2.302424in}}%
\pgfpathlineto{\pgfqpoint{5.064636in}{2.303268in}}%
\pgfpathlineto{\pgfqpoint{5.072368in}{2.306882in}}%
\pgfpathlineto{\pgfqpoint{5.074945in}{2.308352in}}%
\pgfpathlineto{\pgfqpoint{5.085254in}{2.309822in}}%
\pgfpathlineto{\pgfqpoint{5.092986in}{2.314208in}}%
\pgfpathlineto{\pgfqpoint{5.100717in}{2.315850in}}%
\pgfpathlineto{\pgfqpoint{5.111026in}{2.323092in}}%
\pgfpathlineto{\pgfqpoint{5.118758in}{2.324963in}}%
\pgfpathlineto{\pgfqpoint{5.129067in}{2.332251in}}%
\pgfpathlineto{\pgfqpoint{5.136798in}{2.333942in}}%
\pgfpathlineto{\pgfqpoint{5.147107in}{2.340088in}}%
\pgfpathlineto{\pgfqpoint{5.154839in}{2.341624in}}%
\pgfpathlineto{\pgfqpoint{5.165148in}{2.347682in}}%
\pgfpathlineto{\pgfqpoint{5.172879in}{2.349202in}}%
\pgfpathlineto{\pgfqpoint{5.183188in}{2.355553in}}%
\pgfpathlineto{\pgfqpoint{5.190920in}{2.357199in}}%
\pgfpathlineto{\pgfqpoint{5.201228in}{2.363619in}}%
\pgfpathlineto{\pgfqpoint{5.208960in}{2.365191in}}%
\pgfpathlineto{\pgfqpoint{5.219269in}{2.371560in}}%
\pgfpathlineto{\pgfqpoint{5.227001in}{2.373214in}}%
\pgfpathlineto{\pgfqpoint{5.237309in}{2.379588in}}%
\pgfpathlineto{\pgfqpoint{5.247618in}{2.381196in}}%
\pgfpathlineto{\pgfqpoint{5.255350in}{2.385498in}}%
\pgfpathlineto{\pgfqpoint{5.263082in}{2.386883in}}%
\pgfpathlineto{\pgfqpoint{5.273390in}{2.391654in}}%
\pgfpathlineto{\pgfqpoint{5.281122in}{2.392899in}}%
\pgfpathlineto{\pgfqpoint{5.291431in}{2.398402in}}%
\pgfpathlineto{\pgfqpoint{5.299163in}{2.399557in}}%
\pgfpathlineto{\pgfqpoint{5.309471in}{2.404243in}}%
\pgfpathlineto{\pgfqpoint{5.319780in}{2.406120in}}%
\pgfpathlineto{\pgfqpoint{5.327512in}{2.408572in}}%
\pgfpathlineto{\pgfqpoint{5.337821in}{2.410005in}}%
\pgfpathlineto{\pgfqpoint{5.345552in}{2.412049in}}%
\pgfpathlineto{\pgfqpoint{5.355861in}{2.413401in}}%
\pgfpathlineto{\pgfqpoint{5.363593in}{2.415339in}}%
\pgfpathlineto{\pgfqpoint{5.379056in}{2.417327in}}%
\pgfpathlineto{\pgfqpoint{5.381633in}{2.417717in}}%
\pgfpathlineto{\pgfqpoint{5.394519in}{2.418757in}}%
\pgfpathlineto{\pgfqpoint{5.399674in}{2.419614in}}%
\pgfpathlineto{\pgfqpoint{5.409983in}{2.420906in}}%
\pgfpathlineto{\pgfqpoint{5.417714in}{2.423375in}}%
\pgfpathlineto{\pgfqpoint{5.425446in}{2.424164in}}%
\pgfpathlineto{\pgfqpoint{5.435755in}{2.427520in}}%
\pgfpathlineto{\pgfqpoint{5.446064in}{2.429041in}}%
\pgfpathlineto{\pgfqpoint{5.448641in}{2.429830in}}%
\pgfpathlineto{\pgfqpoint{5.469259in}{2.433838in}}%
\pgfpathlineto{\pgfqpoint{5.471836in}{2.434614in}}%
\pgfpathlineto{\pgfqpoint{5.482145in}{2.436063in}}%
\pgfpathlineto{\pgfqpoint{5.489876in}{2.439230in}}%
\pgfpathlineto{\pgfqpoint{5.497608in}{2.440479in}}%
\pgfpathlineto{\pgfqpoint{5.507917in}{2.444725in}}%
\pgfpathlineto{\pgfqpoint{5.515648in}{2.445839in}}%
\pgfpathlineto{\pgfqpoint{5.525957in}{2.450409in}}%
\pgfpathlineto{\pgfqpoint{5.536266in}{2.451550in}}%
\pgfpathlineto{\pgfqpoint{5.543998in}{2.454818in}}%
\pgfpathlineto{\pgfqpoint{5.554307in}{2.455876in}}%
\pgfpathlineto{\pgfqpoint{5.562038in}{2.459028in}}%
\pgfpathlineto{\pgfqpoint{5.569770in}{2.460012in}}%
\pgfpathlineto{\pgfqpoint{5.580079in}{2.463917in}}%
\pgfpathlineto{\pgfqpoint{5.590388in}{2.464877in}}%
\pgfpathlineto{\pgfqpoint{5.598119in}{2.467992in}}%
\pgfpathlineto{\pgfqpoint{5.608428in}{2.469869in}}%
\pgfpathlineto{\pgfqpoint{5.616160in}{2.472600in}}%
\pgfpathlineto{\pgfqpoint{5.626469in}{2.474157in}}%
\pgfpathlineto{\pgfqpoint{5.634200in}{2.476402in}}%
\pgfpathlineto{\pgfqpoint{5.644509in}{2.477959in}}%
\pgfpathlineto{\pgfqpoint{5.652241in}{2.480778in}}%
\pgfpathlineto{\pgfqpoint{5.659972in}{2.481907in}}%
\pgfpathlineto{\pgfqpoint{5.670281in}{2.487019in}}%
\pgfpathlineto{\pgfqpoint{5.680590in}{2.488372in}}%
\pgfpathlineto{\pgfqpoint{5.688322in}{2.493194in}}%
\pgfpathlineto{\pgfqpoint{5.696053in}{2.494788in}}%
\pgfpathlineto{\pgfqpoint{5.703785in}{2.499954in}}%
\pgfpathlineto{\pgfqpoint{5.706362in}{2.501708in}}%
\pgfpathlineto{\pgfqpoint{5.714094in}{2.503420in}}%
\pgfpathlineto{\pgfqpoint{5.724403in}{2.510518in}}%
\pgfpathlineto{\pgfqpoint{5.732134in}{2.512401in}}%
\pgfpathlineto{\pgfqpoint{5.742443in}{2.519694in}}%
\pgfpathlineto{\pgfqpoint{5.750175in}{2.521656in}}%
\pgfpathlineto{\pgfqpoint{5.760484in}{2.529084in}}%
\pgfpathlineto{\pgfqpoint{5.768215in}{2.530794in}}%
\pgfpathlineto{\pgfqpoint{5.778524in}{2.537759in}}%
\pgfpathlineto{\pgfqpoint{5.786256in}{2.539453in}}%
\pgfpathlineto{\pgfqpoint{5.796565in}{2.545956in}}%
\pgfpathlineto{\pgfqpoint{5.804296in}{2.547535in}}%
\pgfpathlineto{\pgfqpoint{5.812028in}{2.552177in}}%
\pgfpathlineto{\pgfqpoint{5.822337in}{2.553777in}}%
\pgfpathlineto{\pgfqpoint{5.832646in}{2.560202in}}%
\pgfpathlineto{\pgfqpoint{5.840377in}{2.562066in}}%
\pgfpathlineto{\pgfqpoint{5.850686in}{2.569848in}}%
\pgfpathlineto{\pgfqpoint{5.858418in}{2.571714in}}%
\pgfpathlineto{\pgfqpoint{5.868727in}{2.580218in}}%
\pgfpathlineto{\pgfqpoint{5.876458in}{2.582351in}}%
\pgfpathlineto{\pgfqpoint{5.886767in}{2.590242in}}%
\pgfpathlineto{\pgfqpoint{5.894499in}{2.592245in}}%
\pgfpathlineto{\pgfqpoint{5.904807in}{2.599909in}}%
\pgfpathlineto{\pgfqpoint{5.912539in}{2.602035in}}%
\pgfpathlineto{\pgfqpoint{5.922848in}{2.610561in}}%
\pgfpathlineto{\pgfqpoint{5.933157in}{2.612969in}}%
\pgfpathlineto{\pgfqpoint{5.940888in}{2.620847in}}%
\pgfpathlineto{\pgfqpoint{5.948620in}{2.623541in}}%
\pgfpathlineto{\pgfqpoint{5.958929in}{2.634051in}}%
\pgfpathlineto{\pgfqpoint{5.966661in}{2.636788in}}%
\pgfpathlineto{\pgfqpoint{5.976969in}{2.648895in}}%
\pgfpathlineto{\pgfqpoint{5.984701in}{2.652116in}}%
\pgfpathlineto{\pgfqpoint{5.995010in}{2.664644in}}%
\pgfpathlineto{\pgfqpoint{6.002742in}{2.667717in}}%
\pgfpathlineto{\pgfqpoint{6.013050in}{2.678608in}}%
\pgfpathlineto{\pgfqpoint{6.020782in}{2.681362in}}%
\pgfpathlineto{\pgfqpoint{6.025936in}{2.684097in}}%
\pgfpathlineto{\pgfqpoint{6.031091in}{2.689363in}}%
\pgfpathlineto{\pgfqpoint{6.038823in}{2.692126in}}%
\pgfpathlineto{\pgfqpoint{6.049131in}{2.703213in}}%
\pgfpathlineto{\pgfqpoint{6.056863in}{2.706017in}}%
\pgfpathlineto{\pgfqpoint{6.067172in}{2.717132in}}%
\pgfpathlineto{\pgfqpoint{6.074904in}{2.719727in}}%
\pgfpathlineto{\pgfqpoint{6.085212in}{2.726683in}}%
\pgfpathlineto{\pgfqpoint{6.092944in}{2.728524in}}%
\pgfpathlineto{\pgfqpoint{6.100676in}{2.734980in}}%
\pgfpathlineto{\pgfqpoint{6.103253in}{2.737292in}}%
\pgfpathlineto{\pgfqpoint{6.110984in}{2.739576in}}%
\pgfpathlineto{\pgfqpoint{6.121293in}{2.748245in}}%
\pgfpathlineto{\pgfqpoint{6.129025in}{2.750475in}}%
\pgfpathlineto{\pgfqpoint{6.139334in}{2.758874in}}%
\pgfpathlineto{\pgfqpoint{6.147065in}{2.760839in}}%
\pgfpathlineto{\pgfqpoint{6.157374in}{2.768442in}}%
\pgfpathlineto{\pgfqpoint{6.165106in}{2.770297in}}%
\pgfpathlineto{\pgfqpoint{6.175415in}{2.777985in}}%
\pgfpathlineto{\pgfqpoint{6.185724in}{2.779692in}}%
\pgfpathlineto{\pgfqpoint{6.193455in}{2.785453in}}%
\pgfpathlineto{\pgfqpoint{6.201187in}{2.787781in}}%
\pgfpathlineto{\pgfqpoint{6.211496in}{2.797457in}}%
\pgfpathlineto{\pgfqpoint{6.219227in}{2.800078in}}%
\pgfpathlineto{\pgfqpoint{6.226959in}{2.807319in}}%
\pgfpathlineto{\pgfqpoint{6.229536in}{2.809651in}}%
\pgfpathlineto{\pgfqpoint{6.237268in}{2.811971in}}%
\pgfpathlineto{\pgfqpoint{6.247577in}{2.820921in}}%
\pgfpathlineto{\pgfqpoint{6.255308in}{2.823354in}}%
\pgfpathlineto{\pgfqpoint{6.265617in}{2.834041in}}%
\pgfpathlineto{\pgfqpoint{6.273349in}{2.836740in}}%
\pgfpathlineto{\pgfqpoint{6.283658in}{2.847574in}}%
\pgfpathlineto{\pgfqpoint{6.291389in}{2.850386in}}%
\pgfpathlineto{\pgfqpoint{6.301698in}{2.861655in}}%
\pgfpathlineto{\pgfqpoint{6.309430in}{2.864645in}}%
\pgfpathlineto{\pgfqpoint{6.319739in}{2.881428in}}%
\pgfpathlineto{\pgfqpoint{6.327470in}{2.885189in}}%
\pgfpathlineto{\pgfqpoint{6.337779in}{2.900127in}}%
\pgfpathlineto{\pgfqpoint{6.345511in}{2.903735in}}%
\pgfpathlineto{\pgfqpoint{6.355820in}{2.917508in}}%
\pgfpathlineto{\pgfqpoint{6.363551in}{2.920847in}}%
\pgfpathlineto{\pgfqpoint{6.373860in}{2.934293in}}%
\pgfpathlineto{\pgfqpoint{6.381592in}{2.937850in}}%
\pgfpathlineto{\pgfqpoint{6.386746in}{2.945269in}}%
\pgfpathlineto{\pgfqpoint{6.391901in}{2.948872in}}%
\pgfpathlineto{\pgfqpoint{6.399632in}{2.952702in}}%
\pgfpathlineto{\pgfqpoint{6.407364in}{2.965501in}}%
\pgfpathlineto{\pgfqpoint{6.409941in}{2.969942in}}%
\pgfpathlineto{\pgfqpoint{6.417673in}{2.974175in}}%
\pgfpathlineto{\pgfqpoint{6.427982in}{2.990750in}}%
\pgfpathlineto{\pgfqpoint{6.435713in}{2.994889in}}%
\pgfpathlineto{\pgfqpoint{6.446022in}{3.010745in}}%
\pgfpathlineto{\pgfqpoint{6.453754in}{3.014708in}}%
\pgfpathlineto{\pgfqpoint{6.464063in}{3.029576in}}%
\pgfpathlineto{\pgfqpoint{6.474371in}{3.033182in}}%
\pgfpathlineto{\pgfqpoint{6.482103in}{3.043990in}}%
\pgfpathlineto{\pgfqpoint{6.482103in}{3.043990in}}%
\pgfusepath{stroke}%
\end{pgfscope}%
\begin{pgfscope}%
\pgfpathrectangle{\pgfqpoint{0.563921in}{0.521603in}}{\pgfqpoint{6.200000in}{2.642500in}}%
\pgfusepath{clip}%
\pgfsetroundcap%
\pgfsetroundjoin%
\pgfsetlinewidth{1.505625pt}%
\definecolor{currentstroke}{rgb}{1.000000,0.498039,0.054902}%
\pgfsetstrokecolor{currentstroke}%
\pgfsetdash{}{0pt}%
\pgfpathmoveto{\pgfqpoint{0.845739in}{0.641717in}}%
\pgfpathlineto{\pgfqpoint{0.848317in}{0.642611in}}%
\pgfpathlineto{\pgfqpoint{0.850894in}{0.656202in}}%
\pgfpathlineto{\pgfqpoint{0.853471in}{0.654748in}}%
\pgfpathlineto{\pgfqpoint{0.861203in}{0.653373in}}%
\pgfpathlineto{\pgfqpoint{0.863780in}{0.653781in}}%
\pgfpathlineto{\pgfqpoint{0.866357in}{0.656485in}}%
\pgfpathlineto{\pgfqpoint{0.868934in}{0.666224in}}%
\pgfpathlineto{\pgfqpoint{0.871512in}{0.671444in}}%
\pgfpathlineto{\pgfqpoint{0.881820in}{0.678235in}}%
\pgfpathlineto{\pgfqpoint{0.886975in}{0.691486in}}%
\pgfpathlineto{\pgfqpoint{0.897284in}{0.689184in}}%
\pgfpathlineto{\pgfqpoint{0.899861in}{0.687584in}}%
\pgfpathlineto{\pgfqpoint{0.902438in}{0.687569in}}%
\pgfpathlineto{\pgfqpoint{0.907593in}{0.686010in}}%
\pgfpathlineto{\pgfqpoint{0.917901in}{0.684751in}}%
\pgfpathlineto{\pgfqpoint{0.920479in}{0.685833in}}%
\pgfpathlineto{\pgfqpoint{0.923056in}{0.688687in}}%
\pgfpathlineto{\pgfqpoint{0.925633in}{0.696231in}}%
\pgfpathlineto{\pgfqpoint{0.933365in}{0.699730in}}%
\pgfpathlineto{\pgfqpoint{0.935942in}{0.703718in}}%
\pgfpathlineto{\pgfqpoint{0.938519in}{0.705206in}}%
\pgfpathlineto{\pgfqpoint{0.941096in}{0.708489in}}%
\pgfpathlineto{\pgfqpoint{0.943674in}{0.709648in}}%
\pgfpathlineto{\pgfqpoint{0.956560in}{0.712167in}}%
\pgfpathlineto{\pgfqpoint{0.961714in}{0.717672in}}%
\pgfpathlineto{\pgfqpoint{0.972023in}{0.719708in}}%
\pgfpathlineto{\pgfqpoint{0.987486in}{0.728898in}}%
\pgfpathlineto{\pgfqpoint{0.990063in}{0.731394in}}%
\pgfpathlineto{\pgfqpoint{0.992641in}{0.731887in}}%
\pgfpathlineto{\pgfqpoint{0.995218in}{0.733458in}}%
\pgfpathlineto{\pgfqpoint{0.997795in}{0.733848in}}%
\pgfpathlineto{\pgfqpoint{1.005527in}{0.734120in}}%
\pgfpathlineto{\pgfqpoint{1.010681in}{0.732687in}}%
\pgfpathlineto{\pgfqpoint{1.023567in}{0.733091in}}%
\pgfpathlineto{\pgfqpoint{1.026144in}{0.735010in}}%
\pgfpathlineto{\pgfqpoint{1.033876in}{0.753113in}}%
\pgfpathlineto{\pgfqpoint{1.041608in}{0.759563in}}%
\pgfpathlineto{\pgfqpoint{1.049339in}{0.773502in}}%
\pgfpathlineto{\pgfqpoint{1.051916in}{0.777479in}}%
\pgfpathlineto{\pgfqpoint{1.059648in}{0.783911in}}%
\pgfpathlineto{\pgfqpoint{1.067380in}{0.797977in}}%
\pgfpathlineto{\pgfqpoint{1.069957in}{0.800878in}}%
\pgfpathlineto{\pgfqpoint{1.077689in}{0.803742in}}%
\pgfpathlineto{\pgfqpoint{1.080266in}{0.807217in}}%
\pgfpathlineto{\pgfqpoint{1.082843in}{0.808969in}}%
\pgfpathlineto{\pgfqpoint{1.085420in}{0.811509in}}%
\pgfpathlineto{\pgfqpoint{1.100883in}{0.813612in}}%
\pgfpathlineto{\pgfqpoint{1.103461in}{0.815481in}}%
\pgfpathlineto{\pgfqpoint{1.106038in}{0.816387in}}%
\pgfpathlineto{\pgfqpoint{1.113770in}{0.817757in}}%
\pgfpathlineto{\pgfqpoint{1.118924in}{0.820740in}}%
\pgfpathlineto{\pgfqpoint{1.124078in}{0.822137in}}%
\pgfpathlineto{\pgfqpoint{1.134387in}{0.823216in}}%
\pgfpathlineto{\pgfqpoint{1.139542in}{0.827415in}}%
\pgfpathlineto{\pgfqpoint{1.142119in}{0.830322in}}%
\pgfpathlineto{\pgfqpoint{1.149851in}{0.833073in}}%
\pgfpathlineto{\pgfqpoint{1.157582in}{0.842727in}}%
\pgfpathlineto{\pgfqpoint{1.160159in}{0.844597in}}%
\pgfpathlineto{\pgfqpoint{1.167891in}{0.846340in}}%
\pgfpathlineto{\pgfqpoint{1.173045in}{0.848740in}}%
\pgfpathlineto{\pgfqpoint{1.178200in}{0.850586in}}%
\pgfpathlineto{\pgfqpoint{1.188509in}{0.850427in}}%
\pgfpathlineto{\pgfqpoint{1.193663in}{0.848928in}}%
\pgfpathlineto{\pgfqpoint{1.196240in}{0.847902in}}%
\pgfpathlineto{\pgfqpoint{1.206549in}{0.846248in}}%
\pgfpathlineto{\pgfqpoint{1.214281in}{0.843643in}}%
\pgfpathlineto{\pgfqpoint{1.224590in}{0.842915in}}%
\pgfpathlineto{\pgfqpoint{1.232321in}{0.840266in}}%
\pgfpathlineto{\pgfqpoint{1.240053in}{0.839414in}}%
\pgfpathlineto{\pgfqpoint{1.250362in}{0.835978in}}%
\pgfpathlineto{\pgfqpoint{1.260671in}{0.834472in}}%
\pgfpathlineto{\pgfqpoint{1.268402in}{0.832132in}}%
\pgfpathlineto{\pgfqpoint{1.281288in}{0.830581in}}%
\pgfpathlineto{\pgfqpoint{1.286443in}{0.829422in}}%
\pgfpathlineto{\pgfqpoint{1.296752in}{0.828092in}}%
\pgfpathlineto{\pgfqpoint{1.304483in}{0.827044in}}%
\pgfpathlineto{\pgfqpoint{1.314792in}{0.828009in}}%
\pgfpathlineto{\pgfqpoint{1.322524in}{0.828678in}}%
\pgfpathlineto{\pgfqpoint{1.335410in}{0.828664in}}%
\pgfpathlineto{\pgfqpoint{1.340564in}{0.828039in}}%
\pgfpathlineto{\pgfqpoint{1.353450in}{0.828188in}}%
\pgfpathlineto{\pgfqpoint{1.358605in}{0.826964in}}%
\pgfpathlineto{\pgfqpoint{1.368914in}{0.825694in}}%
\pgfpathlineto{\pgfqpoint{1.374068in}{0.824873in}}%
\pgfpathlineto{\pgfqpoint{1.386954in}{0.824569in}}%
\pgfpathlineto{\pgfqpoint{1.394686in}{0.823297in}}%
\pgfpathlineto{\pgfqpoint{1.407572in}{0.822411in}}%
\pgfpathlineto{\pgfqpoint{1.412726in}{0.821339in}}%
\pgfpathlineto{\pgfqpoint{1.423035in}{0.820266in}}%
\pgfpathlineto{\pgfqpoint{1.430767in}{0.819196in}}%
\pgfpathlineto{\pgfqpoint{1.441076in}{0.818317in}}%
\pgfpathlineto{\pgfqpoint{1.448807in}{0.817120in}}%
\pgfpathlineto{\pgfqpoint{1.461693in}{0.816224in}}%
\pgfpathlineto{\pgfqpoint{1.466848in}{0.815780in}}%
\pgfpathlineto{\pgfqpoint{1.479734in}{0.815426in}}%
\pgfpathlineto{\pgfqpoint{1.484888in}{0.814871in}}%
\pgfpathlineto{\pgfqpoint{1.513237in}{0.814350in}}%
\pgfpathlineto{\pgfqpoint{1.528701in}{0.813704in}}%
\pgfpathlineto{\pgfqpoint{1.539010in}{0.812094in}}%
\pgfpathlineto{\pgfqpoint{1.600863in}{0.809671in}}%
\pgfpathlineto{\pgfqpoint{1.611172in}{0.808059in}}%
\pgfpathlineto{\pgfqpoint{1.629212in}{0.806990in}}%
\pgfpathlineto{\pgfqpoint{1.642098in}{0.805949in}}%
\pgfpathlineto{\pgfqpoint{1.647253in}{0.805189in}}%
\pgfpathlineto{\pgfqpoint{1.673025in}{0.803397in}}%
\pgfpathlineto{\pgfqpoint{1.683333in}{0.802356in}}%
\pgfpathlineto{\pgfqpoint{1.696220in}{0.801352in}}%
\pgfpathlineto{\pgfqpoint{1.701374in}{0.800663in}}%
\pgfpathlineto{\pgfqpoint{1.714260in}{0.799669in}}%
\pgfpathlineto{\pgfqpoint{1.719414in}{0.799043in}}%
\pgfpathlineto{\pgfqpoint{1.747764in}{0.797524in}}%
\pgfpathlineto{\pgfqpoint{1.755495in}{0.796876in}}%
\pgfpathlineto{\pgfqpoint{1.791576in}{0.796254in}}%
\pgfpathlineto{\pgfqpoint{1.801885in}{0.797123in}}%
\pgfpathlineto{\pgfqpoint{1.827657in}{0.801912in}}%
\pgfpathlineto{\pgfqpoint{1.871470in}{0.803088in}}%
\pgfpathlineto{\pgfqpoint{1.879202in}{0.804928in}}%
\pgfpathlineto{\pgfqpoint{1.881779in}{0.805741in}}%
\pgfpathlineto{\pgfqpoint{1.889510in}{0.806607in}}%
\pgfpathlineto{\pgfqpoint{1.899819in}{0.810036in}}%
\pgfpathlineto{\pgfqpoint{1.912705in}{0.811650in}}%
\pgfpathlineto{\pgfqpoint{1.917860in}{0.813183in}}%
\pgfpathlineto{\pgfqpoint{1.928169in}{0.814577in}}%
\pgfpathlineto{\pgfqpoint{1.935900in}{0.816944in}}%
\pgfpathlineto{\pgfqpoint{1.943632in}{0.817897in}}%
\pgfpathlineto{\pgfqpoint{1.953941in}{0.824121in}}%
\pgfpathlineto{\pgfqpoint{1.961672in}{0.826045in}}%
\pgfpathlineto{\pgfqpoint{1.971981in}{0.833330in}}%
\pgfpathlineto{\pgfqpoint{1.979713in}{0.835160in}}%
\pgfpathlineto{\pgfqpoint{1.987445in}{0.840063in}}%
\pgfpathlineto{\pgfqpoint{1.990022in}{0.841932in}}%
\pgfpathlineto{\pgfqpoint{1.997753in}{0.843708in}}%
\pgfpathlineto{\pgfqpoint{2.005485in}{0.850358in}}%
\pgfpathlineto{\pgfqpoint{2.015794in}{0.852468in}}%
\pgfpathlineto{\pgfqpoint{2.020948in}{0.856366in}}%
\pgfpathlineto{\pgfqpoint{2.026103in}{0.859407in}}%
\pgfpathlineto{\pgfqpoint{2.033834in}{0.860712in}}%
\pgfpathlineto{\pgfqpoint{2.044143in}{0.865554in}}%
\pgfpathlineto{\pgfqpoint{2.051875in}{0.866252in}}%
\pgfpathlineto{\pgfqpoint{2.059606in}{0.868710in}}%
\pgfpathlineto{\pgfqpoint{2.062184in}{0.870428in}}%
\pgfpathlineto{\pgfqpoint{2.069915in}{0.871890in}}%
\pgfpathlineto{\pgfqpoint{2.080224in}{0.879006in}}%
\pgfpathlineto{\pgfqpoint{2.087956in}{0.880669in}}%
\pgfpathlineto{\pgfqpoint{2.095687in}{0.886719in}}%
\pgfpathlineto{\pgfqpoint{2.098265in}{0.889308in}}%
\pgfpathlineto{\pgfqpoint{2.105996in}{0.891765in}}%
\pgfpathlineto{\pgfqpoint{2.116305in}{0.901319in}}%
\pgfpathlineto{\pgfqpoint{2.124037in}{0.903405in}}%
\pgfpathlineto{\pgfqpoint{2.131768in}{0.912450in}}%
\pgfpathlineto{\pgfqpoint{2.134346in}{0.915863in}}%
\pgfpathlineto{\pgfqpoint{2.142077in}{0.919721in}}%
\pgfpathlineto{\pgfqpoint{2.149809in}{0.931238in}}%
\pgfpathlineto{\pgfqpoint{2.152386in}{0.935143in}}%
\pgfpathlineto{\pgfqpoint{2.162695in}{0.939395in}}%
\pgfpathlineto{\pgfqpoint{2.170427in}{0.951129in}}%
\pgfpathlineto{\pgfqpoint{2.178158in}{0.955126in}}%
\pgfpathlineto{\pgfqpoint{2.188467in}{0.969904in}}%
\pgfpathlineto{\pgfqpoint{2.196199in}{0.974301in}}%
\pgfpathlineto{\pgfqpoint{2.201353in}{0.980462in}}%
\pgfpathlineto{\pgfqpoint{2.203930in}{0.983273in}}%
\pgfpathlineto{\pgfqpoint{2.206508in}{0.985217in}}%
\pgfpathlineto{\pgfqpoint{2.214239in}{0.987424in}}%
\pgfpathlineto{\pgfqpoint{2.219394in}{0.992263in}}%
\pgfpathlineto{\pgfqpoint{2.224548in}{0.995917in}}%
\pgfpathlineto{\pgfqpoint{2.232280in}{0.997359in}}%
\pgfpathlineto{\pgfqpoint{2.237434in}{1.001117in}}%
\pgfpathlineto{\pgfqpoint{2.242589in}{1.005654in}}%
\pgfpathlineto{\pgfqpoint{2.250320in}{1.008103in}}%
\pgfpathlineto{\pgfqpoint{2.255475in}{1.012320in}}%
\pgfpathlineto{\pgfqpoint{2.260629in}{1.014965in}}%
\pgfpathlineto{\pgfqpoint{2.268361in}{1.017829in}}%
\pgfpathlineto{\pgfqpoint{2.273515in}{1.023354in}}%
\pgfpathlineto{\pgfqpoint{2.278670in}{1.029154in}}%
\pgfpathlineto{\pgfqpoint{2.286401in}{1.032049in}}%
\pgfpathlineto{\pgfqpoint{2.291556in}{1.037464in}}%
\pgfpathlineto{\pgfqpoint{2.296710in}{1.040502in}}%
\pgfpathlineto{\pgfqpoint{2.304442in}{1.042177in}}%
\pgfpathlineto{\pgfqpoint{2.312173in}{1.047451in}}%
\pgfpathlineto{\pgfqpoint{2.314751in}{1.049196in}}%
\pgfpathlineto{\pgfqpoint{2.322482in}{1.050833in}}%
\pgfpathlineto{\pgfqpoint{2.332791in}{1.057165in}}%
\pgfpathlineto{\pgfqpoint{2.340523in}{1.058921in}}%
\pgfpathlineto{\pgfqpoint{2.350832in}{1.065390in}}%
\pgfpathlineto{\pgfqpoint{2.358563in}{1.066900in}}%
\pgfpathlineto{\pgfqpoint{2.366295in}{1.071434in}}%
\pgfpathlineto{\pgfqpoint{2.368872in}{1.072772in}}%
\pgfpathlineto{\pgfqpoint{2.376604in}{1.073889in}}%
\pgfpathlineto{\pgfqpoint{2.386912in}{1.077723in}}%
\pgfpathlineto{\pgfqpoint{2.399799in}{1.079676in}}%
\pgfpathlineto{\pgfqpoint{2.404953in}{1.080751in}}%
\pgfpathlineto{\pgfqpoint{2.415262in}{1.081343in}}%
\pgfpathlineto{\pgfqpoint{2.422993in}{1.083654in}}%
\pgfpathlineto{\pgfqpoint{2.430725in}{1.084448in}}%
\pgfpathlineto{\pgfqpoint{2.441034in}{1.088715in}}%
\pgfpathlineto{\pgfqpoint{2.448766in}{1.089859in}}%
\pgfpathlineto{\pgfqpoint{2.459074in}{1.095924in}}%
\pgfpathlineto{\pgfqpoint{2.466806in}{1.097168in}}%
\pgfpathlineto{\pgfqpoint{2.477115in}{1.101643in}}%
\pgfpathlineto{\pgfqpoint{2.484847in}{1.102605in}}%
\pgfpathlineto{\pgfqpoint{2.490001in}{1.104436in}}%
\pgfpathlineto{\pgfqpoint{2.495155in}{1.105832in}}%
\pgfpathlineto{\pgfqpoint{2.508041in}{1.106971in}}%
\pgfpathlineto{\pgfqpoint{2.513196in}{1.108611in}}%
\pgfpathlineto{\pgfqpoint{2.523505in}{1.110460in}}%
\pgfpathlineto{\pgfqpoint{2.526082in}{1.111499in}}%
\pgfpathlineto{\pgfqpoint{2.531236in}{1.115445in}}%
\pgfpathlineto{\pgfqpoint{2.538968in}{1.117383in}}%
\pgfpathlineto{\pgfqpoint{2.549277in}{1.125765in}}%
\pgfpathlineto{\pgfqpoint{2.557009in}{1.128190in}}%
\pgfpathlineto{\pgfqpoint{2.564740in}{1.135348in}}%
\pgfpathlineto{\pgfqpoint{2.567317in}{1.137475in}}%
\pgfpathlineto{\pgfqpoint{2.575049in}{1.139514in}}%
\pgfpathlineto{\pgfqpoint{2.585358in}{1.147149in}}%
\pgfpathlineto{\pgfqpoint{2.593089in}{1.148926in}}%
\pgfpathlineto{\pgfqpoint{2.603398in}{1.156181in}}%
\pgfpathlineto{\pgfqpoint{2.611130in}{1.158065in}}%
\pgfpathlineto{\pgfqpoint{2.621439in}{1.166129in}}%
\pgfpathlineto{\pgfqpoint{2.629170in}{1.168444in}}%
\pgfpathlineto{\pgfqpoint{2.634325in}{1.173366in}}%
\pgfpathlineto{\pgfqpoint{2.639479in}{1.175899in}}%
\pgfpathlineto{\pgfqpoint{2.647211in}{1.178264in}}%
\pgfpathlineto{\pgfqpoint{2.657520in}{1.187019in}}%
\pgfpathlineto{\pgfqpoint{2.665251in}{1.189347in}}%
\pgfpathlineto{\pgfqpoint{2.670406in}{1.193354in}}%
\pgfpathlineto{\pgfqpoint{2.675560in}{1.196750in}}%
\pgfpathlineto{\pgfqpoint{2.683292in}{1.198648in}}%
\pgfpathlineto{\pgfqpoint{2.688446in}{1.202627in}}%
\pgfpathlineto{\pgfqpoint{2.693601in}{1.207434in}}%
\pgfpathlineto{\pgfqpoint{2.701332in}{1.210125in}}%
\pgfpathlineto{\pgfqpoint{2.703910in}{1.212879in}}%
\pgfpathlineto{\pgfqpoint{2.709064in}{1.215680in}}%
\pgfpathlineto{\pgfqpoint{2.711641in}{1.218490in}}%
\pgfpathlineto{\pgfqpoint{2.719373in}{1.221370in}}%
\pgfpathlineto{\pgfqpoint{2.721950in}{1.224515in}}%
\pgfpathlineto{\pgfqpoint{2.727105in}{1.227277in}}%
\pgfpathlineto{\pgfqpoint{2.729682in}{1.230078in}}%
\pgfpathlineto{\pgfqpoint{2.737413in}{1.232830in}}%
\pgfpathlineto{\pgfqpoint{2.747722in}{1.243008in}}%
\pgfpathlineto{\pgfqpoint{2.755454in}{1.245003in}}%
\pgfpathlineto{\pgfqpoint{2.763185in}{1.251330in}}%
\pgfpathlineto{\pgfqpoint{2.765763in}{1.254167in}}%
\pgfpathlineto{\pgfqpoint{2.776072in}{1.256872in}}%
\pgfpathlineto{\pgfqpoint{2.781226in}{1.261971in}}%
\pgfpathlineto{\pgfqpoint{2.783803in}{1.263748in}}%
\pgfpathlineto{\pgfqpoint{2.791535in}{1.265254in}}%
\pgfpathlineto{\pgfqpoint{2.801844in}{1.271296in}}%
\pgfpathlineto{\pgfqpoint{2.809575in}{1.272204in}}%
\pgfpathlineto{\pgfqpoint{2.817307in}{1.275700in}}%
\pgfpathlineto{\pgfqpoint{2.819884in}{1.277310in}}%
\pgfpathlineto{\pgfqpoint{2.827616in}{1.279175in}}%
\pgfpathlineto{\pgfqpoint{2.837925in}{1.286861in}}%
\pgfpathlineto{\pgfqpoint{2.848234in}{1.288756in}}%
\pgfpathlineto{\pgfqpoint{2.855965in}{1.294223in}}%
\pgfpathlineto{\pgfqpoint{2.863697in}{1.296220in}}%
\pgfpathlineto{\pgfqpoint{2.874006in}{1.304305in}}%
\pgfpathlineto{\pgfqpoint{2.881737in}{1.306199in}}%
\pgfpathlineto{\pgfqpoint{2.892046in}{1.316154in}}%
\pgfpathlineto{\pgfqpoint{2.899778in}{1.318717in}}%
\pgfpathlineto{\pgfqpoint{2.907509in}{1.325308in}}%
\pgfpathlineto{\pgfqpoint{2.910087in}{1.327029in}}%
\pgfpathlineto{\pgfqpoint{2.917818in}{1.328946in}}%
\pgfpathlineto{\pgfqpoint{2.928127in}{1.336499in}}%
\pgfpathlineto{\pgfqpoint{2.935859in}{1.338264in}}%
\pgfpathlineto{\pgfqpoint{2.946168in}{1.344562in}}%
\pgfpathlineto{\pgfqpoint{2.953899in}{1.346056in}}%
\pgfpathlineto{\pgfqpoint{2.964208in}{1.352290in}}%
\pgfpathlineto{\pgfqpoint{2.971940in}{1.353184in}}%
\pgfpathlineto{\pgfqpoint{2.982249in}{1.356515in}}%
\pgfpathlineto{\pgfqpoint{2.992557in}{1.357964in}}%
\pgfpathlineto{\pgfqpoint{2.997712in}{1.359682in}}%
\pgfpathlineto{\pgfqpoint{3.008021in}{1.360505in}}%
\pgfpathlineto{\pgfqpoint{3.018330in}{1.363996in}}%
\pgfpathlineto{\pgfqpoint{3.026061in}{1.364831in}}%
\pgfpathlineto{\pgfqpoint{3.036370in}{1.368189in}}%
\pgfpathlineto{\pgfqpoint{3.046679in}{1.369688in}}%
\pgfpathlineto{\pgfqpoint{3.054411in}{1.372603in}}%
\pgfpathlineto{\pgfqpoint{3.062142in}{1.373733in}}%
\pgfpathlineto{\pgfqpoint{3.069874in}{1.376489in}}%
\pgfpathlineto{\pgfqpoint{3.072451in}{1.377263in}}%
\pgfpathlineto{\pgfqpoint{3.082760in}{1.378817in}}%
\pgfpathlineto{\pgfqpoint{3.090491in}{1.381360in}}%
\pgfpathlineto{\pgfqpoint{3.100800in}{1.382660in}}%
\pgfpathlineto{\pgfqpoint{3.108532in}{1.386433in}}%
\pgfpathlineto{\pgfqpoint{3.116264in}{1.387762in}}%
\pgfpathlineto{\pgfqpoint{3.123995in}{1.391781in}}%
\pgfpathlineto{\pgfqpoint{3.126572in}{1.393565in}}%
\pgfpathlineto{\pgfqpoint{3.134304in}{1.395449in}}%
\pgfpathlineto{\pgfqpoint{3.144613in}{1.402395in}}%
\pgfpathlineto{\pgfqpoint{3.152345in}{1.403982in}}%
\pgfpathlineto{\pgfqpoint{3.162653in}{1.410742in}}%
\pgfpathlineto{\pgfqpoint{3.170385in}{1.412441in}}%
\pgfpathlineto{\pgfqpoint{3.180694in}{1.418462in}}%
\pgfpathlineto{\pgfqpoint{3.188426in}{1.419978in}}%
\pgfpathlineto{\pgfqpoint{3.196157in}{1.424733in}}%
\pgfpathlineto{\pgfqpoint{3.206466in}{1.426276in}}%
\pgfpathlineto{\pgfqpoint{3.216775in}{1.431896in}}%
\pgfpathlineto{\pgfqpoint{3.224507in}{1.433276in}}%
\pgfpathlineto{\pgfqpoint{3.234815in}{1.438258in}}%
\pgfpathlineto{\pgfqpoint{3.242547in}{1.439332in}}%
\pgfpathlineto{\pgfqpoint{3.252856in}{1.443560in}}%
\pgfpathlineto{\pgfqpoint{3.260588in}{1.444446in}}%
\pgfpathlineto{\pgfqpoint{3.268319in}{1.446448in}}%
\pgfpathlineto{\pgfqpoint{3.288937in}{1.447886in}}%
\pgfpathlineto{\pgfqpoint{3.301823in}{1.448717in}}%
\pgfpathlineto{\pgfqpoint{3.306977in}{1.449212in}}%
\pgfpathlineto{\pgfqpoint{3.317286in}{1.449837in}}%
\pgfpathlineto{\pgfqpoint{3.325018in}{1.451118in}}%
\pgfpathlineto{\pgfqpoint{3.335327in}{1.452066in}}%
\pgfpathlineto{\pgfqpoint{3.343058in}{1.453416in}}%
\pgfpathlineto{\pgfqpoint{3.355944in}{1.454439in}}%
\pgfpathlineto{\pgfqpoint{3.361099in}{1.455373in}}%
\pgfpathlineto{\pgfqpoint{3.376562in}{1.456681in}}%
\pgfpathlineto{\pgfqpoint{3.415220in}{1.460057in}}%
\pgfpathlineto{\pgfqpoint{3.459033in}{1.460988in}}%
\pgfpathlineto{\pgfqpoint{3.482228in}{1.459799in}}%
\pgfpathlineto{\pgfqpoint{3.500268in}{1.460450in}}%
\pgfpathlineto{\pgfqpoint{3.518309in}{1.462770in}}%
\pgfpathlineto{\pgfqpoint{3.523463in}{1.463924in}}%
\pgfpathlineto{\pgfqpoint{3.533772in}{1.465090in}}%
\pgfpathlineto{\pgfqpoint{3.541504in}{1.466494in}}%
\pgfpathlineto{\pgfqpoint{3.554390in}{1.467648in}}%
\pgfpathlineto{\pgfqpoint{3.559544in}{1.468419in}}%
\pgfpathlineto{\pgfqpoint{3.569853in}{1.469315in}}%
\pgfpathlineto{\pgfqpoint{3.577585in}{1.470343in}}%
\pgfpathlineto{\pgfqpoint{3.587893in}{1.471485in}}%
\pgfpathlineto{\pgfqpoint{3.613666in}{1.475762in}}%
\pgfpathlineto{\pgfqpoint{3.626552in}{1.476471in}}%
\pgfpathlineto{\pgfqpoint{3.631706in}{1.477591in}}%
\pgfpathlineto{\pgfqpoint{3.642015in}{1.478836in}}%
\pgfpathlineto{\pgfqpoint{3.644592in}{1.479489in}}%
\pgfpathlineto{\pgfqpoint{3.667787in}{1.482520in}}%
\pgfpathlineto{\pgfqpoint{3.683250in}{1.483566in}}%
\pgfpathlineto{\pgfqpoint{3.685828in}{1.483842in}}%
\pgfpathlineto{\pgfqpoint{3.716754in}{1.484083in}}%
\pgfpathlineto{\pgfqpoint{3.809534in}{1.476952in}}%
\pgfpathlineto{\pgfqpoint{3.812111in}{1.476625in}}%
\pgfpathlineto{\pgfqpoint{3.824997in}{1.475638in}}%
\pgfpathlineto{\pgfqpoint{3.830151in}{1.474916in}}%
\pgfpathlineto{\pgfqpoint{3.840460in}{1.474130in}}%
\pgfpathlineto{\pgfqpoint{3.848192in}{1.473035in}}%
\pgfpathlineto{\pgfqpoint{3.861078in}{1.472047in}}%
\pgfpathlineto{\pgfqpoint{3.866232in}{1.471447in}}%
\pgfpathlineto{\pgfqpoint{3.879118in}{1.470438in}}%
\pgfpathlineto{\pgfqpoint{3.884273in}{1.469566in}}%
\pgfpathlineto{\pgfqpoint{3.897159in}{1.468290in}}%
\pgfpathlineto{\pgfqpoint{3.899736in}{1.467909in}}%
\pgfpathlineto{\pgfqpoint{3.912622in}{1.467103in}}%
\pgfpathlineto{\pgfqpoint{3.920354in}{1.465912in}}%
\pgfpathlineto{\pgfqpoint{3.933240in}{1.464759in}}%
\pgfpathlineto{\pgfqpoint{3.956435in}{1.461841in}}%
\pgfpathlineto{\pgfqpoint{3.966744in}{1.460972in}}%
\pgfpathlineto{\pgfqpoint{3.974475in}{1.459669in}}%
\pgfpathlineto{\pgfqpoint{3.984784in}{1.458835in}}%
\pgfpathlineto{\pgfqpoint{3.992516in}{1.457616in}}%
\pgfpathlineto{\pgfqpoint{4.005402in}{1.456500in}}%
\pgfpathlineto{\pgfqpoint{4.010556in}{1.455821in}}%
\pgfpathlineto{\pgfqpoint{4.023442in}{1.454844in}}%
\pgfpathlineto{\pgfqpoint{4.028597in}{1.454206in}}%
\pgfpathlineto{\pgfqpoint{4.044060in}{1.453157in}}%
\pgfpathlineto{\pgfqpoint{4.046637in}{1.452795in}}%
\pgfpathlineto{\pgfqpoint{4.059523in}{1.451742in}}%
\pgfpathlineto{\pgfqpoint{4.064678in}{1.451000in}}%
\pgfpathlineto{\pgfqpoint{4.077564in}{1.449895in}}%
\pgfpathlineto{\pgfqpoint{4.082718in}{1.449192in}}%
\pgfpathlineto{\pgfqpoint{4.095604in}{1.448115in}}%
\pgfpathlineto{\pgfqpoint{4.100759in}{1.447453in}}%
\pgfpathlineto{\pgfqpoint{4.113645in}{1.446529in}}%
\pgfpathlineto{\pgfqpoint{4.118799in}{1.445830in}}%
\pgfpathlineto{\pgfqpoint{4.131685in}{1.444646in}}%
\pgfpathlineto{\pgfqpoint{4.134263in}{1.444261in}}%
\pgfpathlineto{\pgfqpoint{4.147149in}{1.443461in}}%
\pgfpathlineto{\pgfqpoint{4.154880in}{1.442206in}}%
\pgfpathlineto{\pgfqpoint{4.167766in}{1.441097in}}%
\pgfpathlineto{\pgfqpoint{4.172921in}{1.440381in}}%
\pgfpathlineto{\pgfqpoint{4.185807in}{1.439312in}}%
\pgfpathlineto{\pgfqpoint{4.190961in}{1.438485in}}%
\pgfpathlineto{\pgfqpoint{4.201270in}{1.437621in}}%
\pgfpathlineto{\pgfqpoint{4.209002in}{1.436365in}}%
\pgfpathlineto{\pgfqpoint{4.219311in}{1.435527in}}%
\pgfpathlineto{\pgfqpoint{4.227042in}{1.434348in}}%
\pgfpathlineto{\pgfqpoint{4.239928in}{1.433508in}}%
\pgfpathlineto{\pgfqpoint{4.245083in}{1.432923in}}%
\pgfpathlineto{\pgfqpoint{4.257969in}{1.432071in}}%
\pgfpathlineto{\pgfqpoint{4.263123in}{1.431323in}}%
\pgfpathlineto{\pgfqpoint{4.273432in}{1.430471in}}%
\pgfpathlineto{\pgfqpoint{4.281164in}{1.429269in}}%
\pgfpathlineto{\pgfqpoint{4.291472in}{1.428451in}}%
\pgfpathlineto{\pgfqpoint{4.299204in}{1.427201in}}%
\pgfpathlineto{\pgfqpoint{4.312090in}{1.426380in}}%
\pgfpathlineto{\pgfqpoint{4.317245in}{1.425564in}}%
\pgfpathlineto{\pgfqpoint{4.330131in}{1.424385in}}%
\pgfpathlineto{\pgfqpoint{4.335285in}{1.423606in}}%
\pgfpathlineto{\pgfqpoint{4.348171in}{1.422420in}}%
\pgfpathlineto{\pgfqpoint{4.353326in}{1.421606in}}%
\pgfpathlineto{\pgfqpoint{4.366212in}{1.420368in}}%
\pgfpathlineto{\pgfqpoint{4.371366in}{1.419550in}}%
\pgfpathlineto{\pgfqpoint{4.384252in}{1.418439in}}%
\pgfpathlineto{\pgfqpoint{4.389407in}{1.417713in}}%
\pgfpathlineto{\pgfqpoint{4.402293in}{1.416598in}}%
\pgfpathlineto{\pgfqpoint{4.407447in}{1.415866in}}%
\pgfpathlineto{\pgfqpoint{4.420333in}{1.414764in}}%
\pgfpathlineto{\pgfqpoint{4.425488in}{1.413966in}}%
\pgfpathlineto{\pgfqpoint{4.438374in}{1.412777in}}%
\pgfpathlineto{\pgfqpoint{4.443528in}{1.411983in}}%
\pgfpathlineto{\pgfqpoint{4.456414in}{1.410799in}}%
\pgfpathlineto{\pgfqpoint{4.461569in}{1.410014in}}%
\pgfpathlineto{\pgfqpoint{4.474455in}{1.408830in}}%
\pgfpathlineto{\pgfqpoint{4.479609in}{1.408047in}}%
\pgfpathlineto{\pgfqpoint{4.492495in}{1.406877in}}%
\pgfpathlineto{\pgfqpoint{4.497649in}{1.406096in}}%
\pgfpathlineto{\pgfqpoint{4.510536in}{1.404932in}}%
\pgfpathlineto{\pgfqpoint{4.544039in}{1.401889in}}%
\pgfpathlineto{\pgfqpoint{4.551771in}{1.400799in}}%
\pgfpathlineto{\pgfqpoint{4.564657in}{1.399705in}}%
\pgfpathlineto{\pgfqpoint{4.569811in}{1.399011in}}%
\pgfpathlineto{\pgfqpoint{4.582697in}{1.397956in}}%
\pgfpathlineto{\pgfqpoint{4.585275in}{1.397591in}}%
\pgfpathlineto{\pgfqpoint{4.600738in}{1.396499in}}%
\pgfpathlineto{\pgfqpoint{4.603315in}{1.396143in}}%
\pgfpathlineto{\pgfqpoint{4.621356in}{1.395050in}}%
\pgfpathlineto{\pgfqpoint{4.641973in}{1.393769in}}%
\pgfpathlineto{\pgfqpoint{4.667746in}{1.393686in}}%
\pgfpathlineto{\pgfqpoint{4.711558in}{1.398090in}}%
\pgfpathlineto{\pgfqpoint{4.714135in}{1.398474in}}%
\pgfpathlineto{\pgfqpoint{4.739907in}{1.399776in}}%
\pgfpathlineto{\pgfqpoint{4.760525in}{1.400782in}}%
\pgfpathlineto{\pgfqpoint{4.778566in}{1.400737in}}%
\pgfpathlineto{\pgfqpoint{4.837842in}{1.399130in}}%
\pgfpathlineto{\pgfqpoint{4.858459in}{1.398502in}}%
\pgfpathlineto{\pgfqpoint{4.873922in}{1.397854in}}%
\pgfpathlineto{\pgfqpoint{4.966702in}{1.391064in}}%
\pgfpathlineto{\pgfqpoint{4.979588in}{1.390278in}}%
\pgfpathlineto{\pgfqpoint{4.984743in}{1.389718in}}%
\pgfpathlineto{\pgfqpoint{5.000206in}{1.388841in}}%
\pgfpathlineto{\pgfqpoint{5.020824in}{1.387134in}}%
\pgfpathlineto{\pgfqpoint{5.051750in}{1.385649in}}%
\pgfpathlineto{\pgfqpoint{5.056905in}{1.385313in}}%
\pgfpathlineto{\pgfqpoint{5.085254in}{1.384793in}}%
\pgfpathlineto{\pgfqpoint{5.118758in}{1.383006in}}%
\pgfpathlineto{\pgfqpoint{5.147107in}{1.380700in}}%
\pgfpathlineto{\pgfqpoint{5.162570in}{1.379725in}}%
\pgfpathlineto{\pgfqpoint{5.175456in}{1.378888in}}%
\pgfpathlineto{\pgfqpoint{5.201228in}{1.376694in}}%
\pgfpathlineto{\pgfqpoint{5.216692in}{1.375612in}}%
\pgfpathlineto{\pgfqpoint{5.237309in}{1.374005in}}%
\pgfpathlineto{\pgfqpoint{5.252773in}{1.373164in}}%
\pgfpathlineto{\pgfqpoint{5.265659in}{1.372375in}}%
\pgfpathlineto{\pgfqpoint{5.291431in}{1.370474in}}%
\pgfpathlineto{\pgfqpoint{5.306894in}{1.369539in}}%
\pgfpathlineto{\pgfqpoint{5.324935in}{1.368389in}}%
\pgfpathlineto{\pgfqpoint{5.345552in}{1.367421in}}%
\pgfpathlineto{\pgfqpoint{5.361016in}{1.366706in}}%
\pgfpathlineto{\pgfqpoint{5.363593in}{1.366415in}}%
\pgfpathlineto{\pgfqpoint{5.376479in}{1.365559in}}%
\pgfpathlineto{\pgfqpoint{5.381633in}{1.364993in}}%
\pgfpathlineto{\pgfqpoint{5.397097in}{1.363907in}}%
\pgfpathlineto{\pgfqpoint{5.417714in}{1.362193in}}%
\pgfpathlineto{\pgfqpoint{5.430600in}{1.361353in}}%
\pgfpathlineto{\pgfqpoint{5.435755in}{1.360781in}}%
\pgfpathlineto{\pgfqpoint{5.461527in}{1.359393in}}%
\pgfpathlineto{\pgfqpoint{5.471836in}{1.358278in}}%
\pgfpathlineto{\pgfqpoint{5.487299in}{1.357243in}}%
\pgfpathlineto{\pgfqpoint{5.507917in}{1.355778in}}%
\pgfpathlineto{\pgfqpoint{5.523380in}{1.354842in}}%
\pgfpathlineto{\pgfqpoint{5.525957in}{1.354610in}}%
\pgfpathlineto{\pgfqpoint{5.541421in}{1.353886in}}%
\pgfpathlineto{\pgfqpoint{5.543998in}{1.353639in}}%
\pgfpathlineto{\pgfqpoint{5.569770in}{1.352589in}}%
\pgfpathlineto{\pgfqpoint{5.580079in}{1.351867in}}%
\pgfpathlineto{\pgfqpoint{5.605851in}{1.350946in}}%
\pgfpathlineto{\pgfqpoint{5.641932in}{1.349244in}}%
\pgfpathlineto{\pgfqpoint{5.665127in}{1.348376in}}%
\pgfpathlineto{\pgfqpoint{5.696053in}{1.347810in}}%
\pgfpathlineto{\pgfqpoint{5.809451in}{1.344053in}}%
\pgfpathlineto{\pgfqpoint{5.812028in}{1.343859in}}%
\pgfpathlineto{\pgfqpoint{5.866149in}{1.342422in}}%
\pgfpathlineto{\pgfqpoint{5.922848in}{1.340057in}}%
\pgfpathlineto{\pgfqpoint{6.010473in}{1.339147in}}%
\pgfpathlineto{\pgfqpoint{6.219227in}{1.347803in}}%
\pgfpathlineto{\pgfqpoint{6.229536in}{1.349030in}}%
\pgfpathlineto{\pgfqpoint{6.242422in}{1.349979in}}%
\pgfpathlineto{\pgfqpoint{6.247577in}{1.350793in}}%
\pgfpathlineto{\pgfqpoint{6.257886in}{1.351691in}}%
\pgfpathlineto{\pgfqpoint{6.265617in}{1.353123in}}%
\pgfpathlineto{\pgfqpoint{6.275926in}{1.354160in}}%
\pgfpathlineto{\pgfqpoint{6.283658in}{1.355776in}}%
\pgfpathlineto{\pgfqpoint{6.293967in}{1.356797in}}%
\pgfpathlineto{\pgfqpoint{6.301698in}{1.358354in}}%
\pgfpathlineto{\pgfqpoint{6.312007in}{1.359539in}}%
\pgfpathlineto{\pgfqpoint{6.319739in}{1.361731in}}%
\pgfpathlineto{\pgfqpoint{6.330048in}{1.363219in}}%
\pgfpathlineto{\pgfqpoint{6.337779in}{1.365613in}}%
\pgfpathlineto{\pgfqpoint{6.348088in}{1.367155in}}%
\pgfpathlineto{\pgfqpoint{6.355820in}{1.369054in}}%
\pgfpathlineto{\pgfqpoint{6.366129in}{1.370261in}}%
\pgfpathlineto{\pgfqpoint{6.373860in}{1.371982in}}%
\pgfpathlineto{\pgfqpoint{6.384169in}{1.373216in}}%
\pgfpathlineto{\pgfqpoint{6.386746in}{1.373803in}}%
\pgfpathlineto{\pgfqpoint{6.402209in}{1.375608in}}%
\pgfpathlineto{\pgfqpoint{6.409941in}{1.378155in}}%
\pgfpathlineto{\pgfqpoint{6.417673in}{1.379096in}}%
\pgfpathlineto{\pgfqpoint{6.427982in}{1.382795in}}%
\pgfpathlineto{\pgfqpoint{6.435713in}{1.383751in}}%
\pgfpathlineto{\pgfqpoint{6.446022in}{1.387274in}}%
\pgfpathlineto{\pgfqpoint{6.453754in}{1.388268in}}%
\pgfpathlineto{\pgfqpoint{6.464063in}{1.391882in}}%
\pgfpathlineto{\pgfqpoint{6.474371in}{1.392764in}}%
\pgfpathlineto{\pgfqpoint{6.482103in}{1.395586in}}%
\pgfpathlineto{\pgfqpoint{6.482103in}{1.395586in}}%
\pgfusepath{stroke}%
\end{pgfscope}%
\begin{pgfscope}%
\pgfpathrectangle{\pgfqpoint{0.563921in}{0.521603in}}{\pgfqpoint{6.200000in}{2.642500in}}%
\pgfusepath{clip}%
\pgfsetroundcap%
\pgfsetroundjoin%
\pgfsetlinewidth{1.505625pt}%
\definecolor{currentstroke}{rgb}{0.172549,0.627451,0.172549}%
\pgfsetstrokecolor{currentstroke}%
\pgfsetdash{}{0pt}%
\pgfpathmoveto{\pgfqpoint{0.845739in}{0.641717in}}%
\pgfpathlineto{\pgfqpoint{0.848317in}{0.646186in}}%
\pgfpathlineto{\pgfqpoint{0.850894in}{0.645797in}}%
\pgfpathlineto{\pgfqpoint{0.853471in}{0.646472in}}%
\pgfpathlineto{\pgfqpoint{0.861203in}{0.649087in}}%
\pgfpathlineto{\pgfqpoint{0.863780in}{0.648748in}}%
\pgfpathlineto{\pgfqpoint{0.868934in}{0.650005in}}%
\pgfpathlineto{\pgfqpoint{0.881820in}{0.649413in}}%
\pgfpathlineto{\pgfqpoint{0.889552in}{0.651848in}}%
\pgfpathlineto{\pgfqpoint{0.935942in}{0.650998in}}%
\pgfpathlineto{\pgfqpoint{0.941096in}{0.651392in}}%
\pgfpathlineto{\pgfqpoint{0.953982in}{0.651004in}}%
\pgfpathlineto{\pgfqpoint{0.961714in}{0.651138in}}%
\pgfpathlineto{\pgfqpoint{0.974600in}{0.652163in}}%
\pgfpathlineto{\pgfqpoint{0.979754in}{0.653092in}}%
\pgfpathlineto{\pgfqpoint{1.023567in}{0.653098in}}%
\pgfpathlineto{\pgfqpoint{1.028722in}{0.655233in}}%
\pgfpathlineto{\pgfqpoint{1.033876in}{0.659190in}}%
\pgfpathlineto{\pgfqpoint{1.044185in}{0.661788in}}%
\pgfpathlineto{\pgfqpoint{1.049339in}{0.663127in}}%
\pgfpathlineto{\pgfqpoint{1.082843in}{0.666623in}}%
\pgfpathlineto{\pgfqpoint{1.136964in}{0.664768in}}%
\pgfpathlineto{\pgfqpoint{1.209126in}{0.663750in}}%
\pgfpathlineto{\pgfqpoint{1.232321in}{0.663311in}}%
\pgfpathlineto{\pgfqpoint{1.250362in}{0.663638in}}%
\pgfpathlineto{\pgfqpoint{1.276134in}{0.663776in}}%
\pgfpathlineto{\pgfqpoint{1.286443in}{0.664486in}}%
\pgfpathlineto{\pgfqpoint{1.299329in}{0.665051in}}%
\pgfpathlineto{\pgfqpoint{1.301906in}{0.665475in}}%
\pgfpathlineto{\pgfqpoint{1.304483in}{0.666532in}}%
\pgfpathlineto{\pgfqpoint{1.335410in}{0.668448in}}%
\pgfpathlineto{\pgfqpoint{1.353450in}{0.668482in}}%
\pgfpathlineto{\pgfqpoint{1.371491in}{0.668981in}}%
\pgfpathlineto{\pgfqpoint{1.394686in}{0.672655in}}%
\pgfpathlineto{\pgfqpoint{1.404995in}{0.673929in}}%
\pgfpathlineto{\pgfqpoint{1.412726in}{0.675643in}}%
\pgfpathlineto{\pgfqpoint{1.425612in}{0.676966in}}%
\pgfpathlineto{\pgfqpoint{1.430767in}{0.677831in}}%
\pgfpathlineto{\pgfqpoint{1.459116in}{0.679641in}}%
\pgfpathlineto{\pgfqpoint{1.466848in}{0.680174in}}%
\pgfpathlineto{\pgfqpoint{1.482311in}{0.680832in}}%
\pgfpathlineto{\pgfqpoint{1.484888in}{0.681434in}}%
\pgfpathlineto{\pgfqpoint{1.495197in}{0.682499in}}%
\pgfpathlineto{\pgfqpoint{1.502929in}{0.684836in}}%
\pgfpathlineto{\pgfqpoint{1.510660in}{0.685580in}}%
\pgfpathlineto{\pgfqpoint{1.520969in}{0.689437in}}%
\pgfpathlineto{\pgfqpoint{1.531278in}{0.691146in}}%
\pgfpathlineto{\pgfqpoint{1.536432in}{0.692843in}}%
\pgfpathlineto{\pgfqpoint{1.539010in}{0.693830in}}%
\pgfpathlineto{\pgfqpoint{1.546741in}{0.694837in}}%
\pgfpathlineto{\pgfqpoint{1.557050in}{0.698835in}}%
\pgfpathlineto{\pgfqpoint{1.567359in}{0.700412in}}%
\pgfpathlineto{\pgfqpoint{1.575091in}{0.702106in}}%
\pgfpathlineto{\pgfqpoint{1.585399in}{0.703308in}}%
\pgfpathlineto{\pgfqpoint{1.593131in}{0.704932in}}%
\pgfpathlineto{\pgfqpoint{1.629212in}{0.705200in}}%
\pgfpathlineto{\pgfqpoint{1.660139in}{0.704924in}}%
\pgfpathlineto{\pgfqpoint{1.678179in}{0.704367in}}%
\pgfpathlineto{\pgfqpoint{1.737455in}{0.704397in}}%
\pgfpathlineto{\pgfqpoint{1.763227in}{0.704437in}}%
\pgfpathlineto{\pgfqpoint{1.825080in}{0.704058in}}%
\pgfpathlineto{\pgfqpoint{1.827657in}{0.704245in}}%
\pgfpathlineto{\pgfqpoint{1.845698in}{0.704984in}}%
\pgfpathlineto{\pgfqpoint{1.861161in}{0.705998in}}%
\pgfpathlineto{\pgfqpoint{1.863738in}{0.706314in}}%
\pgfpathlineto{\pgfqpoint{1.879202in}{0.707301in}}%
\pgfpathlineto{\pgfqpoint{1.915283in}{0.711182in}}%
\pgfpathlineto{\pgfqpoint{1.917860in}{0.711714in}}%
\pgfpathlineto{\pgfqpoint{1.928169in}{0.712424in}}%
\pgfpathlineto{\pgfqpoint{1.935900in}{0.713752in}}%
\pgfpathlineto{\pgfqpoint{1.946209in}{0.714707in}}%
\pgfpathlineto{\pgfqpoint{1.953941in}{0.716353in}}%
\pgfpathlineto{\pgfqpoint{1.966827in}{0.717693in}}%
\pgfpathlineto{\pgfqpoint{1.971981in}{0.718577in}}%
\pgfpathlineto{\pgfqpoint{1.984867in}{0.719632in}}%
\pgfpathlineto{\pgfqpoint{1.990022in}{0.720294in}}%
\pgfpathlineto{\pgfqpoint{2.018371in}{0.721869in}}%
\pgfpathlineto{\pgfqpoint{2.051875in}{0.723973in}}%
\pgfpathlineto{\pgfqpoint{2.062184in}{0.724327in}}%
\pgfpathlineto{\pgfqpoint{2.105996in}{0.724332in}}%
\pgfpathlineto{\pgfqpoint{2.144655in}{0.726071in}}%
\pgfpathlineto{\pgfqpoint{2.152386in}{0.726835in}}%
\pgfpathlineto{\pgfqpoint{2.178158in}{0.727882in}}%
\pgfpathlineto{\pgfqpoint{2.196199in}{0.728895in}}%
\pgfpathlineto{\pgfqpoint{2.224548in}{0.730713in}}%
\pgfpathlineto{\pgfqpoint{2.255475in}{0.731533in}}%
\pgfpathlineto{\pgfqpoint{2.286401in}{0.732725in}}%
\pgfpathlineto{\pgfqpoint{2.304442in}{0.733999in}}%
\pgfpathlineto{\pgfqpoint{2.314751in}{0.735461in}}%
\pgfpathlineto{\pgfqpoint{2.327637in}{0.736348in}}%
\pgfpathlineto{\pgfqpoint{2.332791in}{0.737013in}}%
\pgfpathlineto{\pgfqpoint{2.348254in}{0.738042in}}%
\pgfpathlineto{\pgfqpoint{2.366295in}{0.739034in}}%
\pgfpathlineto{\pgfqpoint{2.386912in}{0.739747in}}%
\pgfpathlineto{\pgfqpoint{2.451343in}{0.740844in}}%
\pgfpathlineto{\pgfqpoint{2.459074in}{0.741503in}}%
\pgfpathlineto{\pgfqpoint{2.487424in}{0.742676in}}%
\pgfpathlineto{\pgfqpoint{2.502887in}{0.743254in}}%
\pgfpathlineto{\pgfqpoint{2.528659in}{0.744438in}}%
\pgfpathlineto{\pgfqpoint{2.531236in}{0.744831in}}%
\pgfpathlineto{\pgfqpoint{2.544122in}{0.746253in}}%
\pgfpathlineto{\pgfqpoint{2.549277in}{0.747145in}}%
\pgfpathlineto{\pgfqpoint{2.559586in}{0.748123in}}%
\pgfpathlineto{\pgfqpoint{2.567317in}{0.749680in}}%
\pgfpathlineto{\pgfqpoint{2.577626in}{0.750714in}}%
\pgfpathlineto{\pgfqpoint{2.585358in}{0.752523in}}%
\pgfpathlineto{\pgfqpoint{2.595667in}{0.753775in}}%
\pgfpathlineto{\pgfqpoint{2.603398in}{0.755644in}}%
\pgfpathlineto{\pgfqpoint{2.613707in}{0.756848in}}%
\pgfpathlineto{\pgfqpoint{2.621439in}{0.758495in}}%
\pgfpathlineto{\pgfqpoint{2.631748in}{0.759451in}}%
\pgfpathlineto{\pgfqpoint{2.639479in}{0.760379in}}%
\pgfpathlineto{\pgfqpoint{2.652365in}{0.761649in}}%
\pgfpathlineto{\pgfqpoint{2.675560in}{0.764698in}}%
\pgfpathlineto{\pgfqpoint{2.685869in}{0.765595in}}%
\pgfpathlineto{\pgfqpoint{2.693601in}{0.767215in}}%
\pgfpathlineto{\pgfqpoint{2.703910in}{0.768354in}}%
\pgfpathlineto{\pgfqpoint{2.765763in}{0.775984in}}%
\pgfpathlineto{\pgfqpoint{2.899778in}{0.779611in}}%
\pgfpathlineto{\pgfqpoint{2.910087in}{0.780062in}}%
\pgfpathlineto{\pgfqpoint{2.943590in}{0.780970in}}%
\pgfpathlineto{\pgfqpoint{2.977094in}{0.782232in}}%
\pgfpathlineto{\pgfqpoint{3.031216in}{0.784609in}}%
\pgfpathlineto{\pgfqpoint{3.046679in}{0.785340in}}%
\pgfpathlineto{\pgfqpoint{3.072451in}{0.786811in}}%
\pgfpathlineto{\pgfqpoint{3.105955in}{0.788065in}}%
\pgfpathlineto{\pgfqpoint{3.126572in}{0.789137in}}%
\pgfpathlineto{\pgfqpoint{3.154922in}{0.790552in}}%
\pgfpathlineto{\pgfqpoint{3.180694in}{0.791810in}}%
\pgfpathlineto{\pgfqpoint{3.214198in}{0.792841in}}%
\pgfpathlineto{\pgfqpoint{3.242547in}{0.793739in}}%
\pgfpathlineto{\pgfqpoint{3.270896in}{0.794033in}}%
\pgfpathlineto{\pgfqpoint{3.322441in}{0.794535in}}%
\pgfpathlineto{\pgfqpoint{3.361099in}{0.795046in}}%
\pgfpathlineto{\pgfqpoint{3.482228in}{0.795299in}}%
\pgfpathlineto{\pgfqpoint{3.536349in}{0.795893in}}%
\pgfpathlineto{\pgfqpoint{3.595625in}{0.797221in}}%
\pgfpathlineto{\pgfqpoint{3.685828in}{0.796790in}}%
\pgfpathlineto{\pgfqpoint{3.755412in}{0.795583in}}%
\pgfpathlineto{\pgfqpoint{3.794070in}{0.795314in}}%
\pgfpathlineto{\pgfqpoint{3.920354in}{0.795760in}}%
\pgfpathlineto{\pgfqpoint{3.969321in}{0.797467in}}%
\pgfpathlineto{\pgfqpoint{4.010556in}{0.798906in}}%
\pgfpathlineto{\pgfqpoint{4.041483in}{0.799939in}}%
\pgfpathlineto{\pgfqpoint{4.064678in}{0.800817in}}%
\pgfpathlineto{\pgfqpoint{4.095604in}{0.801754in}}%
\pgfpathlineto{\pgfqpoint{4.118799in}{0.802628in}}%
\pgfpathlineto{\pgfqpoint{4.281164in}{0.803621in}}%
\pgfpathlineto{\pgfqpoint{4.389407in}{0.803431in}}%
\pgfpathlineto{\pgfqpoint{4.407447in}{0.804344in}}%
\pgfpathlineto{\pgfqpoint{4.422910in}{0.805427in}}%
\pgfpathlineto{\pgfqpoint{4.425488in}{0.805744in}}%
\pgfpathlineto{\pgfqpoint{4.438374in}{0.806666in}}%
\pgfpathlineto{\pgfqpoint{4.443528in}{0.807197in}}%
\pgfpathlineto{\pgfqpoint{4.456414in}{0.808106in}}%
\pgfpathlineto{\pgfqpoint{4.461569in}{0.808762in}}%
\pgfpathlineto{\pgfqpoint{4.471877in}{0.809446in}}%
\pgfpathlineto{\pgfqpoint{4.479609in}{0.810609in}}%
\pgfpathlineto{\pgfqpoint{4.492495in}{0.811758in}}%
\pgfpathlineto{\pgfqpoint{4.497649in}{0.812528in}}%
\pgfpathlineto{\pgfqpoint{4.510536in}{0.813687in}}%
\pgfpathlineto{\pgfqpoint{4.546617in}{0.816709in}}%
\pgfpathlineto{\pgfqpoint{4.551771in}{0.817409in}}%
\pgfpathlineto{\pgfqpoint{4.564657in}{0.818466in}}%
\pgfpathlineto{\pgfqpoint{4.569811in}{0.819185in}}%
\pgfpathlineto{\pgfqpoint{4.582697in}{0.820318in}}%
\pgfpathlineto{\pgfqpoint{4.585275in}{0.820716in}}%
\pgfpathlineto{\pgfqpoint{4.598161in}{0.821566in}}%
\pgfpathlineto{\pgfqpoint{4.603315in}{0.822407in}}%
\pgfpathlineto{\pgfqpoint{4.618778in}{0.823463in}}%
\pgfpathlineto{\pgfqpoint{4.639396in}{0.824410in}}%
\pgfpathlineto{\pgfqpoint{4.660014in}{0.825054in}}%
\pgfpathlineto{\pgfqpoint{4.688363in}{0.825929in}}%
\pgfpathlineto{\pgfqpoint{4.714135in}{0.826884in}}%
\pgfpathlineto{\pgfqpoint{4.739907in}{0.827765in}}%
\pgfpathlineto{\pgfqpoint{4.814647in}{0.833403in}}%
\pgfpathlineto{\pgfqpoint{4.819801in}{0.834103in}}%
\pgfpathlineto{\pgfqpoint{4.832687in}{0.834880in}}%
\pgfpathlineto{\pgfqpoint{4.840419in}{0.836151in}}%
\pgfpathlineto{\pgfqpoint{4.853305in}{0.837135in}}%
\pgfpathlineto{\pgfqpoint{4.876500in}{0.839235in}}%
\pgfpathlineto{\pgfqpoint{4.889386in}{0.840189in}}%
\pgfpathlineto{\pgfqpoint{4.894540in}{0.840767in}}%
\pgfpathlineto{\pgfqpoint{4.907426in}{0.841612in}}%
\pgfpathlineto{\pgfqpoint{4.912581in}{0.842165in}}%
\pgfpathlineto{\pgfqpoint{4.940930in}{0.843712in}}%
\pgfpathlineto{\pgfqpoint{4.956393in}{0.844478in}}%
\pgfpathlineto{\pgfqpoint{4.984743in}{0.845964in}}%
\pgfpathlineto{\pgfqpoint{5.010515in}{0.846887in}}%
\pgfpathlineto{\pgfqpoint{5.056905in}{0.849903in}}%
\pgfpathlineto{\pgfqpoint{5.069791in}{0.850423in}}%
\pgfpathlineto{\pgfqpoint{5.074945in}{0.851029in}}%
\pgfpathlineto{\pgfqpoint{5.090408in}{0.851979in}}%
\pgfpathlineto{\pgfqpoint{5.092986in}{0.852344in}}%
\pgfpathlineto{\pgfqpoint{5.105872in}{0.853452in}}%
\pgfpathlineto{\pgfqpoint{5.111026in}{0.854280in}}%
\pgfpathlineto{\pgfqpoint{5.123912in}{0.855540in}}%
\pgfpathlineto{\pgfqpoint{5.129067in}{0.856256in}}%
\pgfpathlineto{\pgfqpoint{5.144530in}{0.857322in}}%
\pgfpathlineto{\pgfqpoint{5.165148in}{0.858745in}}%
\pgfpathlineto{\pgfqpoint{5.180611in}{0.859713in}}%
\pgfpathlineto{\pgfqpoint{5.201228in}{0.861124in}}%
\pgfpathlineto{\pgfqpoint{5.216692in}{0.862029in}}%
\pgfpathlineto{\pgfqpoint{5.237309in}{0.863373in}}%
\pgfpathlineto{\pgfqpoint{5.268236in}{0.864470in}}%
\pgfpathlineto{\pgfqpoint{5.309471in}{0.865841in}}%
\pgfpathlineto{\pgfqpoint{5.415137in}{0.867698in}}%
\pgfpathlineto{\pgfqpoint{5.435755in}{0.868651in}}%
\pgfpathlineto{\pgfqpoint{5.464104in}{0.869789in}}%
\pgfpathlineto{\pgfqpoint{5.562038in}{0.875350in}}%
\pgfpathlineto{\pgfqpoint{5.592965in}{0.876672in}}%
\pgfpathlineto{\pgfqpoint{5.605851in}{0.877053in}}%
\pgfpathlineto{\pgfqpoint{5.649663in}{0.878107in}}%
\pgfpathlineto{\pgfqpoint{5.760484in}{0.880796in}}%
\pgfpathlineto{\pgfqpoint{5.804296in}{0.881646in}}%
\pgfpathlineto{\pgfqpoint{5.832646in}{0.882234in}}%
\pgfpathlineto{\pgfqpoint{5.933157in}{0.882468in}}%
\pgfpathlineto{\pgfqpoint{5.976969in}{0.882394in}}%
\pgfpathlineto{\pgfqpoint{6.031091in}{0.882105in}}%
\pgfpathlineto{\pgfqpoint{6.100676in}{0.881034in}}%
\pgfpathlineto{\pgfqpoint{6.211496in}{0.878718in}}%
\pgfpathlineto{\pgfqpoint{6.263040in}{0.877660in}}%
\pgfpathlineto{\pgfqpoint{6.319739in}{0.876455in}}%
\pgfpathlineto{\pgfqpoint{6.404787in}{0.876523in}}%
\pgfpathlineto{\pgfqpoint{6.482103in}{0.877577in}}%
\pgfpathlineto{\pgfqpoint{6.482103in}{0.877577in}}%
\pgfusepath{stroke}%
\end{pgfscope}%
\begin{pgfscope}%
\pgfpathrectangle{\pgfqpoint{0.563921in}{0.521603in}}{\pgfqpoint{6.200000in}{2.642500in}}%
\pgfusepath{clip}%
\pgfsetroundcap%
\pgfsetroundjoin%
\pgfsetlinewidth{1.505625pt}%
\definecolor{currentstroke}{rgb}{0.839216,0.152941,0.156863}%
\pgfsetstrokecolor{currentstroke}%
\pgfsetdash{}{0pt}%
\pgfpathmoveto{\pgfqpoint{0.845739in}{0.641717in}}%
\pgfpathlineto{\pgfqpoint{0.848317in}{0.655422in}}%
\pgfpathlineto{\pgfqpoint{0.850894in}{0.658811in}}%
\pgfpathlineto{\pgfqpoint{0.853471in}{0.657284in}}%
\pgfpathlineto{\pgfqpoint{0.863780in}{0.658051in}}%
\pgfpathlineto{\pgfqpoint{0.866357in}{0.659514in}}%
\pgfpathlineto{\pgfqpoint{0.868934in}{0.659704in}}%
\pgfpathlineto{\pgfqpoint{0.871512in}{0.658998in}}%
\pgfpathlineto{\pgfqpoint{0.881820in}{0.658654in}}%
\pgfpathlineto{\pgfqpoint{0.884398in}{0.657905in}}%
\pgfpathlineto{\pgfqpoint{0.886975in}{0.657753in}}%
\pgfpathlineto{\pgfqpoint{0.889552in}{0.661881in}}%
\pgfpathlineto{\pgfqpoint{0.897284in}{0.666631in}}%
\pgfpathlineto{\pgfqpoint{0.899861in}{0.670761in}}%
\pgfpathlineto{\pgfqpoint{0.902438in}{0.673459in}}%
\pgfpathlineto{\pgfqpoint{0.907593in}{0.675500in}}%
\pgfpathlineto{\pgfqpoint{0.915324in}{0.676056in}}%
\pgfpathlineto{\pgfqpoint{0.923056in}{0.675251in}}%
\pgfpathlineto{\pgfqpoint{0.925633in}{0.675946in}}%
\pgfpathlineto{\pgfqpoint{0.938519in}{0.677104in}}%
\pgfpathlineto{\pgfqpoint{0.943674in}{0.677541in}}%
\pgfpathlineto{\pgfqpoint{0.959137in}{0.677399in}}%
\pgfpathlineto{\pgfqpoint{0.961714in}{0.678468in}}%
\pgfpathlineto{\pgfqpoint{0.974600in}{0.678711in}}%
\pgfpathlineto{\pgfqpoint{0.979754in}{0.678169in}}%
\pgfpathlineto{\pgfqpoint{1.008104in}{0.677418in}}%
\pgfpathlineto{\pgfqpoint{1.023567in}{0.677060in}}%
\pgfpathlineto{\pgfqpoint{1.033876in}{0.679369in}}%
\pgfpathlineto{\pgfqpoint{1.044185in}{0.680509in}}%
\pgfpathlineto{\pgfqpoint{1.069957in}{0.685428in}}%
\pgfpathlineto{\pgfqpoint{1.085420in}{0.687203in}}%
\pgfpathlineto{\pgfqpoint{1.103461in}{0.687789in}}%
\pgfpathlineto{\pgfqpoint{1.121501in}{0.688944in}}%
\pgfpathlineto{\pgfqpoint{1.136964in}{0.688271in}}%
\pgfpathlineto{\pgfqpoint{1.152428in}{0.689859in}}%
\pgfpathlineto{\pgfqpoint{1.157582in}{0.691364in}}%
\pgfpathlineto{\pgfqpoint{1.173045in}{0.690795in}}%
\pgfpathlineto{\pgfqpoint{1.178200in}{0.690419in}}%
\pgfpathlineto{\pgfqpoint{1.214281in}{0.689728in}}%
\pgfpathlineto{\pgfqpoint{1.232321in}{0.689910in}}%
\pgfpathlineto{\pgfqpoint{1.247785in}{0.690351in}}%
\pgfpathlineto{\pgfqpoint{1.268402in}{0.689318in}}%
\pgfpathlineto{\pgfqpoint{1.301906in}{0.688366in}}%
\pgfpathlineto{\pgfqpoint{1.314792in}{0.687818in}}%
\pgfpathlineto{\pgfqpoint{1.335410in}{0.687670in}}%
\pgfpathlineto{\pgfqpoint{1.340564in}{0.688441in}}%
\pgfpathlineto{\pgfqpoint{1.376645in}{0.689271in}}%
\pgfpathlineto{\pgfqpoint{1.404995in}{0.688564in}}%
\pgfpathlineto{\pgfqpoint{1.430767in}{0.687470in}}%
\pgfpathlineto{\pgfqpoint{1.448807in}{0.687446in}}%
\pgfpathlineto{\pgfqpoint{1.461693in}{0.688065in}}%
\pgfpathlineto{\pgfqpoint{1.466848in}{0.688736in}}%
\pgfpathlineto{\pgfqpoint{1.482311in}{0.689542in}}%
\pgfpathlineto{\pgfqpoint{1.495197in}{0.692087in}}%
\pgfpathlineto{\pgfqpoint{1.502929in}{0.694862in}}%
\pgfpathlineto{\pgfqpoint{1.513237in}{0.696493in}}%
\pgfpathlineto{\pgfqpoint{1.520969in}{0.699066in}}%
\pgfpathlineto{\pgfqpoint{1.528701in}{0.700098in}}%
\pgfpathlineto{\pgfqpoint{1.536432in}{0.703049in}}%
\pgfpathlineto{\pgfqpoint{1.539010in}{0.704018in}}%
\pgfpathlineto{\pgfqpoint{1.549318in}{0.705679in}}%
\pgfpathlineto{\pgfqpoint{1.557050in}{0.708376in}}%
\pgfpathlineto{\pgfqpoint{1.564782in}{0.709241in}}%
\pgfpathlineto{\pgfqpoint{1.575091in}{0.714406in}}%
\pgfpathlineto{\pgfqpoint{1.585399in}{0.716305in}}%
\pgfpathlineto{\pgfqpoint{1.593131in}{0.719722in}}%
\pgfpathlineto{\pgfqpoint{1.600863in}{0.720851in}}%
\pgfpathlineto{\pgfqpoint{1.611172in}{0.724529in}}%
\pgfpathlineto{\pgfqpoint{1.626635in}{0.725936in}}%
\pgfpathlineto{\pgfqpoint{1.629212in}{0.726548in}}%
\pgfpathlineto{\pgfqpoint{1.639521in}{0.727769in}}%
\pgfpathlineto{\pgfqpoint{1.647253in}{0.730846in}}%
\pgfpathlineto{\pgfqpoint{1.654984in}{0.731845in}}%
\pgfpathlineto{\pgfqpoint{1.665293in}{0.736961in}}%
\pgfpathlineto{\pgfqpoint{1.673025in}{0.738062in}}%
\pgfpathlineto{\pgfqpoint{1.678179in}{0.741079in}}%
\pgfpathlineto{\pgfqpoint{1.683333in}{0.742352in}}%
\pgfpathlineto{\pgfqpoint{1.691065in}{0.743493in}}%
\pgfpathlineto{\pgfqpoint{1.701374in}{0.747944in}}%
\pgfpathlineto{\pgfqpoint{1.711683in}{0.750003in}}%
\pgfpathlineto{\pgfqpoint{1.719414in}{0.752482in}}%
\pgfpathlineto{\pgfqpoint{1.732301in}{0.754283in}}%
\pgfpathlineto{\pgfqpoint{1.737455in}{0.755369in}}%
\pgfpathlineto{\pgfqpoint{1.752918in}{0.756856in}}%
\pgfpathlineto{\pgfqpoint{1.755495in}{0.757249in}}%
\pgfpathlineto{\pgfqpoint{1.770959in}{0.758537in}}%
\pgfpathlineto{\pgfqpoint{1.773536in}{0.759073in}}%
\pgfpathlineto{\pgfqpoint{1.809617in}{0.760598in}}%
\pgfpathlineto{\pgfqpoint{1.840543in}{0.760987in}}%
\pgfpathlineto{\pgfqpoint{1.853430in}{0.761586in}}%
\pgfpathlineto{\pgfqpoint{1.889510in}{0.762742in}}%
\pgfpathlineto{\pgfqpoint{1.964250in}{0.763954in}}%
\pgfpathlineto{\pgfqpoint{1.990022in}{0.763881in}}%
\pgfpathlineto{\pgfqpoint{2.038989in}{0.763439in}}%
\pgfpathlineto{\pgfqpoint{2.129191in}{0.759755in}}%
\pgfpathlineto{\pgfqpoint{2.152386in}{0.758658in}}%
\pgfpathlineto{\pgfqpoint{2.196199in}{0.757522in}}%
\pgfpathlineto{\pgfqpoint{2.242589in}{0.755885in}}%
\pgfpathlineto{\pgfqpoint{2.276092in}{0.754717in}}%
\pgfpathlineto{\pgfqpoint{2.296710in}{0.753888in}}%
\pgfpathlineto{\pgfqpoint{2.325059in}{0.752935in}}%
\pgfpathlineto{\pgfqpoint{2.350832in}{0.751880in}}%
\pgfpathlineto{\pgfqpoint{2.397221in}{0.750569in}}%
\pgfpathlineto{\pgfqpoint{2.441034in}{0.749209in}}%
\pgfpathlineto{\pgfqpoint{2.474538in}{0.748160in}}%
\pgfpathlineto{\pgfqpoint{2.513196in}{0.746893in}}%
\pgfpathlineto{\pgfqpoint{2.564740in}{0.745574in}}%
\pgfpathlineto{\pgfqpoint{2.603398in}{0.744809in}}%
\pgfpathlineto{\pgfqpoint{2.649788in}{0.744043in}}%
\pgfpathlineto{\pgfqpoint{2.675560in}{0.743639in}}%
\pgfpathlineto{\pgfqpoint{2.719373in}{0.743821in}}%
\pgfpathlineto{\pgfqpoint{2.742568in}{0.744180in}}%
\pgfpathlineto{\pgfqpoint{2.763185in}{0.744817in}}%
\pgfpathlineto{\pgfqpoint{2.783803in}{0.744890in}}%
\pgfpathlineto{\pgfqpoint{2.863697in}{0.743896in}}%
\pgfpathlineto{\pgfqpoint{2.935859in}{0.743433in}}%
\pgfpathlineto{\pgfqpoint{2.974517in}{0.744494in}}%
\pgfpathlineto{\pgfqpoint{2.982249in}{0.744992in}}%
\pgfpathlineto{\pgfqpoint{3.008021in}{0.746023in}}%
\pgfpathlineto{\pgfqpoint{3.018330in}{0.746691in}}%
\pgfpathlineto{\pgfqpoint{3.046679in}{0.747597in}}%
\pgfpathlineto{\pgfqpoint{3.072451in}{0.748582in}}%
\pgfpathlineto{\pgfqpoint{3.103378in}{0.749462in}}%
\pgfpathlineto{\pgfqpoint{3.142036in}{0.753001in}}%
\pgfpathlineto{\pgfqpoint{3.144613in}{0.753771in}}%
\pgfpathlineto{\pgfqpoint{3.154922in}{0.755336in}}%
\pgfpathlineto{\pgfqpoint{3.162653in}{0.757673in}}%
\pgfpathlineto{\pgfqpoint{3.170385in}{0.758471in}}%
\pgfpathlineto{\pgfqpoint{3.180694in}{0.762111in}}%
\pgfpathlineto{\pgfqpoint{3.188426in}{0.763022in}}%
\pgfpathlineto{\pgfqpoint{3.196157in}{0.765784in}}%
\pgfpathlineto{\pgfqpoint{3.206466in}{0.766677in}}%
\pgfpathlineto{\pgfqpoint{3.216775in}{0.770152in}}%
\pgfpathlineto{\pgfqpoint{3.224507in}{0.771104in}}%
\pgfpathlineto{\pgfqpoint{3.227084in}{0.772099in}}%
\pgfpathlineto{\pgfqpoint{3.232238in}{0.775524in}}%
\pgfpathlineto{\pgfqpoint{3.234815in}{0.777031in}}%
\pgfpathlineto{\pgfqpoint{3.242547in}{0.778627in}}%
\pgfpathlineto{\pgfqpoint{3.252856in}{0.785215in}}%
\pgfpathlineto{\pgfqpoint{3.260588in}{0.786721in}}%
\pgfpathlineto{\pgfqpoint{3.270896in}{0.792300in}}%
\pgfpathlineto{\pgfqpoint{3.278628in}{0.793643in}}%
\pgfpathlineto{\pgfqpoint{3.288937in}{0.797703in}}%
\pgfpathlineto{\pgfqpoint{3.296668in}{0.798752in}}%
\pgfpathlineto{\pgfqpoint{3.304400in}{0.802358in}}%
\pgfpathlineto{\pgfqpoint{3.306977in}{0.803641in}}%
\pgfpathlineto{\pgfqpoint{3.314709in}{0.804970in}}%
\pgfpathlineto{\pgfqpoint{3.325018in}{0.810444in}}%
\pgfpathlineto{\pgfqpoint{3.332749in}{0.811783in}}%
\pgfpathlineto{\pgfqpoint{3.343058in}{0.816937in}}%
\pgfpathlineto{\pgfqpoint{3.353367in}{0.818127in}}%
\pgfpathlineto{\pgfqpoint{3.361099in}{0.821797in}}%
\pgfpathlineto{\pgfqpoint{3.368830in}{0.823111in}}%
\pgfpathlineto{\pgfqpoint{3.379139in}{0.827781in}}%
\pgfpathlineto{\pgfqpoint{3.386871in}{0.828832in}}%
\pgfpathlineto{\pgfqpoint{3.397180in}{0.833300in}}%
\pgfpathlineto{\pgfqpoint{3.404911in}{0.834325in}}%
\pgfpathlineto{\pgfqpoint{3.415220in}{0.838049in}}%
\pgfpathlineto{\pgfqpoint{3.422952in}{0.839060in}}%
\pgfpathlineto{\pgfqpoint{3.430684in}{0.841556in}}%
\pgfpathlineto{\pgfqpoint{3.433261in}{0.842354in}}%
\pgfpathlineto{\pgfqpoint{3.443570in}{0.843837in}}%
\pgfpathlineto{\pgfqpoint{3.451301in}{0.845754in}}%
\pgfpathlineto{\pgfqpoint{3.464187in}{0.846842in}}%
\pgfpathlineto{\pgfqpoint{3.484805in}{0.849175in}}%
\pgfpathlineto{\pgfqpoint{3.487382in}{0.849739in}}%
\pgfpathlineto{\pgfqpoint{3.497691in}{0.850949in}}%
\pgfpathlineto{\pgfqpoint{3.505423in}{0.852750in}}%
\pgfpathlineto{\pgfqpoint{3.515732in}{0.854228in}}%
\pgfpathlineto{\pgfqpoint{3.523463in}{0.856146in}}%
\pgfpathlineto{\pgfqpoint{3.533772in}{0.857255in}}%
\pgfpathlineto{\pgfqpoint{3.541504in}{0.859065in}}%
\pgfpathlineto{\pgfqpoint{3.551813in}{0.860515in}}%
\pgfpathlineto{\pgfqpoint{3.577585in}{0.867446in}}%
\pgfpathlineto{\pgfqpoint{3.585316in}{0.868606in}}%
\pgfpathlineto{\pgfqpoint{3.595625in}{0.873425in}}%
\pgfpathlineto{\pgfqpoint{3.603357in}{0.874527in}}%
\pgfpathlineto{\pgfqpoint{3.613666in}{0.878339in}}%
\pgfpathlineto{\pgfqpoint{3.623974in}{0.879900in}}%
\pgfpathlineto{\pgfqpoint{3.631706in}{0.882608in}}%
\pgfpathlineto{\pgfqpoint{3.639438in}{0.883614in}}%
\pgfpathlineto{\pgfqpoint{3.644592in}{0.885684in}}%
\pgfpathlineto{\pgfqpoint{3.649747in}{0.886724in}}%
\pgfpathlineto{\pgfqpoint{3.660055in}{0.888564in}}%
\pgfpathlineto{\pgfqpoint{3.662633in}{0.889352in}}%
\pgfpathlineto{\pgfqpoint{3.683250in}{0.893011in}}%
\pgfpathlineto{\pgfqpoint{3.685828in}{0.893837in}}%
\pgfpathlineto{\pgfqpoint{3.696136in}{0.895400in}}%
\pgfpathlineto{\pgfqpoint{3.703868in}{0.897584in}}%
\pgfpathlineto{\pgfqpoint{3.714177in}{0.898263in}}%
\pgfpathlineto{\pgfqpoint{3.721909in}{0.900508in}}%
\pgfpathlineto{\pgfqpoint{3.752835in}{0.903448in}}%
\pgfpathlineto{\pgfqpoint{3.757990in}{0.904111in}}%
\pgfpathlineto{\pgfqpoint{3.770876in}{0.905006in}}%
\pgfpathlineto{\pgfqpoint{3.776030in}{0.905792in}}%
\pgfpathlineto{\pgfqpoint{3.791493in}{0.907007in}}%
\pgfpathlineto{\pgfqpoint{3.794070in}{0.907405in}}%
\pgfpathlineto{\pgfqpoint{3.806957in}{0.908453in}}%
\pgfpathlineto{\pgfqpoint{3.812111in}{0.909006in}}%
\pgfpathlineto{\pgfqpoint{3.824997in}{0.910033in}}%
\pgfpathlineto{\pgfqpoint{3.830151in}{0.910565in}}%
\pgfpathlineto{\pgfqpoint{3.879118in}{0.911233in}}%
\pgfpathlineto{\pgfqpoint{3.899736in}{0.911471in}}%
\pgfpathlineto{\pgfqpoint{3.933240in}{0.912065in}}%
\pgfpathlineto{\pgfqpoint{3.953858in}{0.912950in}}%
\pgfpathlineto{\pgfqpoint{4.026020in}{0.916312in}}%
\pgfpathlineto{\pgfqpoint{4.028597in}{0.916529in}}%
\pgfpathlineto{\pgfqpoint{4.044060in}{0.917219in}}%
\pgfpathlineto{\pgfqpoint{4.046637in}{0.917531in}}%
\pgfpathlineto{\pgfqpoint{4.100759in}{0.918848in}}%
\pgfpathlineto{\pgfqpoint{4.147149in}{0.918919in}}%
\pgfpathlineto{\pgfqpoint{4.317245in}{0.914412in}}%
\pgfpathlineto{\pgfqpoint{4.366212in}{0.913422in}}%
\pgfpathlineto{\pgfqpoint{4.384252in}{0.913535in}}%
\pgfpathlineto{\pgfqpoint{4.451260in}{0.916828in}}%
\pgfpathlineto{\pgfqpoint{4.461569in}{0.917749in}}%
\pgfpathlineto{\pgfqpoint{4.495072in}{0.918942in}}%
\pgfpathlineto{\pgfqpoint{4.497649in}{0.919218in}}%
\pgfpathlineto{\pgfqpoint{4.523422in}{0.920498in}}%
\pgfpathlineto{\pgfqpoint{4.533730in}{0.921580in}}%
\pgfpathlineto{\pgfqpoint{4.549194in}{0.922670in}}%
\pgfpathlineto{\pgfqpoint{4.569811in}{0.924176in}}%
\pgfpathlineto{\pgfqpoint{4.582697in}{0.924903in}}%
\pgfpathlineto{\pgfqpoint{4.585275in}{0.925172in}}%
\pgfpathlineto{\pgfqpoint{4.600738in}{0.926006in}}%
\pgfpathlineto{\pgfqpoint{4.603315in}{0.926222in}}%
\pgfpathlineto{\pgfqpoint{4.657437in}{0.926606in}}%
\pgfpathlineto{\pgfqpoint{4.750216in}{0.924619in}}%
\pgfpathlineto{\pgfqpoint{4.804338in}{0.924550in}}%
\pgfpathlineto{\pgfqpoint{4.930621in}{0.925049in}}%
\pgfpathlineto{\pgfqpoint{5.028555in}{0.924786in}}%
\pgfpathlineto{\pgfqpoint{5.108449in}{0.926598in}}%
\pgfpathlineto{\pgfqpoint{5.129067in}{0.927950in}}%
\pgfpathlineto{\pgfqpoint{5.144530in}{0.928780in}}%
\pgfpathlineto{\pgfqpoint{5.165148in}{0.929980in}}%
\pgfpathlineto{\pgfqpoint{5.190920in}{0.931246in}}%
\pgfpathlineto{\pgfqpoint{5.219269in}{0.933364in}}%
\pgfpathlineto{\pgfqpoint{5.232155in}{0.934177in}}%
\pgfpathlineto{\pgfqpoint{5.237309in}{0.934769in}}%
\pgfpathlineto{\pgfqpoint{5.252773in}{0.935760in}}%
\pgfpathlineto{\pgfqpoint{5.322357in}{0.942590in}}%
\pgfpathlineto{\pgfqpoint{5.327512in}{0.943469in}}%
\pgfpathlineto{\pgfqpoint{5.340398in}{0.944599in}}%
\pgfpathlineto{\pgfqpoint{5.345552in}{0.945294in}}%
\pgfpathlineto{\pgfqpoint{5.515648in}{0.953451in}}%
\pgfpathlineto{\pgfqpoint{5.525957in}{0.954638in}}%
\pgfpathlineto{\pgfqpoint{5.554307in}{0.955942in}}%
\pgfpathlineto{\pgfqpoint{5.580079in}{0.957939in}}%
\pgfpathlineto{\pgfqpoint{5.595542in}{0.958692in}}%
\pgfpathlineto{\pgfqpoint{5.616160in}{0.960513in}}%
\pgfpathlineto{\pgfqpoint{5.641932in}{0.961981in}}%
\pgfpathlineto{\pgfqpoint{5.652241in}{0.962752in}}%
\pgfpathlineto{\pgfqpoint{5.667704in}{0.963552in}}%
\pgfpathlineto{\pgfqpoint{5.670281in}{0.963781in}}%
\pgfpathlineto{\pgfqpoint{5.696053in}{0.964865in}}%
\pgfpathlineto{\pgfqpoint{5.716671in}{0.965935in}}%
\pgfpathlineto{\pgfqpoint{5.742443in}{0.967070in}}%
\pgfpathlineto{\pgfqpoint{5.773370in}{0.968136in}}%
\pgfpathlineto{\pgfqpoint{5.812028in}{0.969914in}}%
\pgfpathlineto{\pgfqpoint{5.832646in}{0.970701in}}%
\pgfpathlineto{\pgfqpoint{5.848109in}{0.971601in}}%
\pgfpathlineto{\pgfqpoint{5.886767in}{0.973691in}}%
\pgfpathlineto{\pgfqpoint{5.917694in}{0.974792in}}%
\pgfpathlineto{\pgfqpoint{5.935734in}{0.975473in}}%
\pgfpathlineto{\pgfqpoint{5.958929in}{0.976634in}}%
\pgfpathlineto{\pgfqpoint{6.113562in}{0.979861in}}%
\pgfpathlineto{\pgfqpoint{6.134179in}{0.980768in}}%
\pgfpathlineto{\pgfqpoint{6.165106in}{0.981359in}}%
\pgfpathlineto{\pgfqpoint{6.224382in}{0.983424in}}%
\pgfpathlineto{\pgfqpoint{6.247577in}{0.984953in}}%
\pgfpathlineto{\pgfqpoint{6.260463in}{0.985972in}}%
\pgfpathlineto{\pgfqpoint{6.265617in}{0.986694in}}%
\pgfpathlineto{\pgfqpoint{6.278503in}{0.987772in}}%
\pgfpathlineto{\pgfqpoint{6.283658in}{0.988456in}}%
\pgfpathlineto{\pgfqpoint{6.296544in}{0.989589in}}%
\pgfpathlineto{\pgfqpoint{6.301698in}{0.990390in}}%
\pgfpathlineto{\pgfqpoint{6.312007in}{0.991287in}}%
\pgfpathlineto{\pgfqpoint{6.330048in}{0.994639in}}%
\pgfpathlineto{\pgfqpoint{6.337779in}{0.997696in}}%
\pgfpathlineto{\pgfqpoint{6.345511in}{0.998734in}}%
\pgfpathlineto{\pgfqpoint{6.355820in}{1.002660in}}%
\pgfpathlineto{\pgfqpoint{6.363551in}{1.003560in}}%
\pgfpathlineto{\pgfqpoint{6.373860in}{1.006962in}}%
\pgfpathlineto{\pgfqpoint{6.384169in}{1.008505in}}%
\pgfpathlineto{\pgfqpoint{6.386746in}{1.009257in}}%
\pgfpathlineto{\pgfqpoint{6.404787in}{1.012159in}}%
\pgfpathlineto{\pgfqpoint{6.409941in}{1.013646in}}%
\pgfpathlineto{\pgfqpoint{6.420250in}{1.014962in}}%
\pgfpathlineto{\pgfqpoint{6.427982in}{1.016723in}}%
\pgfpathlineto{\pgfqpoint{6.438290in}{1.017921in}}%
\pgfpathlineto{\pgfqpoint{6.446022in}{1.019763in}}%
\pgfpathlineto{\pgfqpoint{6.453754in}{1.020623in}}%
\pgfpathlineto{\pgfqpoint{6.464063in}{1.024346in}}%
\pgfpathlineto{\pgfqpoint{6.474371in}{1.025164in}}%
\pgfpathlineto{\pgfqpoint{6.482103in}{1.027621in}}%
\pgfpathlineto{\pgfqpoint{6.482103in}{1.027621in}}%
\pgfusepath{stroke}%
\end{pgfscope}%
\begin{pgfscope}%
\pgfpathrectangle{\pgfqpoint{0.563921in}{0.521603in}}{\pgfqpoint{6.200000in}{2.642500in}}%
\pgfusepath{clip}%
\pgfsetroundcap%
\pgfsetroundjoin%
\pgfsetlinewidth{1.505625pt}%
\definecolor{currentstroke}{rgb}{0.580392,0.403922,0.741176}%
\pgfsetstrokecolor{currentstroke}%
\pgfsetdash{}{0pt}%
\pgfpathmoveto{\pgfqpoint{0.845739in}{0.641717in}}%
\pgfpathlineto{\pgfqpoint{0.848317in}{0.651251in}}%
\pgfpathlineto{\pgfqpoint{0.850894in}{0.651656in}}%
\pgfpathlineto{\pgfqpoint{0.853471in}{0.659581in}}%
\pgfpathlineto{\pgfqpoint{0.861203in}{0.660025in}}%
\pgfpathlineto{\pgfqpoint{0.866357in}{0.657500in}}%
\pgfpathlineto{\pgfqpoint{0.871512in}{0.655647in}}%
\pgfpathlineto{\pgfqpoint{0.884398in}{0.654476in}}%
\pgfpathlineto{\pgfqpoint{0.889552in}{0.653479in}}%
\pgfpathlineto{\pgfqpoint{0.902438in}{0.652975in}}%
\pgfpathlineto{\pgfqpoint{0.905015in}{0.653947in}}%
\pgfpathlineto{\pgfqpoint{0.915324in}{0.654728in}}%
\pgfpathlineto{\pgfqpoint{0.917901in}{0.655999in}}%
\pgfpathlineto{\pgfqpoint{0.925633in}{0.656042in}}%
\pgfpathlineto{\pgfqpoint{0.941096in}{0.655691in}}%
\pgfpathlineto{\pgfqpoint{0.943674in}{0.656902in}}%
\pgfpathlineto{\pgfqpoint{0.953982in}{0.658355in}}%
\pgfpathlineto{\pgfqpoint{0.956560in}{0.658885in}}%
\pgfpathlineto{\pgfqpoint{1.023567in}{0.657710in}}%
\pgfpathlineto{\pgfqpoint{1.028722in}{0.658256in}}%
\pgfpathlineto{\pgfqpoint{1.062225in}{0.658674in}}%
\pgfpathlineto{\pgfqpoint{1.080266in}{0.663740in}}%
\pgfpathlineto{\pgfqpoint{1.085420in}{0.663870in}}%
\pgfpathlineto{\pgfqpoint{1.100883in}{0.664058in}}%
\pgfpathlineto{\pgfqpoint{1.103461in}{0.664220in}}%
\pgfpathlineto{\pgfqpoint{1.106038in}{0.665201in}}%
\pgfpathlineto{\pgfqpoint{1.116347in}{0.665526in}}%
\pgfpathlineto{\pgfqpoint{1.124078in}{0.668825in}}%
\pgfpathlineto{\pgfqpoint{1.136964in}{0.669767in}}%
\pgfpathlineto{\pgfqpoint{1.149851in}{0.669355in}}%
\pgfpathlineto{\pgfqpoint{1.191086in}{0.669482in}}%
\pgfpathlineto{\pgfqpoint{1.196240in}{0.670465in}}%
\pgfpathlineto{\pgfqpoint{1.206549in}{0.671243in}}%
\pgfpathlineto{\pgfqpoint{1.214281in}{0.672654in}}%
\pgfpathlineto{\pgfqpoint{1.224590in}{0.673197in}}%
\pgfpathlineto{\pgfqpoint{1.232321in}{0.676148in}}%
\pgfpathlineto{\pgfqpoint{1.265825in}{0.679121in}}%
\pgfpathlineto{\pgfqpoint{1.268402in}{0.680240in}}%
\pgfpathlineto{\pgfqpoint{1.276134in}{0.681634in}}%
\pgfpathlineto{\pgfqpoint{1.283866in}{0.686715in}}%
\pgfpathlineto{\pgfqpoint{1.286443in}{0.688085in}}%
\pgfpathlineto{\pgfqpoint{1.294174in}{0.689227in}}%
\pgfpathlineto{\pgfqpoint{1.299329in}{0.691601in}}%
\pgfpathlineto{\pgfqpoint{1.304483in}{0.695012in}}%
\pgfpathlineto{\pgfqpoint{1.312215in}{0.697443in}}%
\pgfpathlineto{\pgfqpoint{1.314792in}{0.699760in}}%
\pgfpathlineto{\pgfqpoint{1.319947in}{0.701605in}}%
\pgfpathlineto{\pgfqpoint{1.322524in}{0.703195in}}%
\pgfpathlineto{\pgfqpoint{1.330255in}{0.704831in}}%
\pgfpathlineto{\pgfqpoint{1.340564in}{0.711491in}}%
\pgfpathlineto{\pgfqpoint{1.348296in}{0.713328in}}%
\pgfpathlineto{\pgfqpoint{1.358605in}{0.722348in}}%
\pgfpathlineto{\pgfqpoint{1.371491in}{0.724542in}}%
\pgfpathlineto{\pgfqpoint{1.376645in}{0.728018in}}%
\pgfpathlineto{\pgfqpoint{1.384377in}{0.729890in}}%
\pgfpathlineto{\pgfqpoint{1.392108in}{0.734019in}}%
\pgfpathlineto{\pgfqpoint{1.394686in}{0.735337in}}%
\pgfpathlineto{\pgfqpoint{1.404995in}{0.737086in}}%
\pgfpathlineto{\pgfqpoint{1.412726in}{0.739235in}}%
\pgfpathlineto{\pgfqpoint{1.423035in}{0.740654in}}%
\pgfpathlineto{\pgfqpoint{1.430767in}{0.741957in}}%
\pgfpathlineto{\pgfqpoint{1.446230in}{0.742842in}}%
\pgfpathlineto{\pgfqpoint{1.448807in}{0.743235in}}%
\pgfpathlineto{\pgfqpoint{1.464270in}{0.744332in}}%
\pgfpathlineto{\pgfqpoint{1.466848in}{0.744593in}}%
\pgfpathlineto{\pgfqpoint{1.482311in}{0.745363in}}%
\pgfpathlineto{\pgfqpoint{1.484888in}{0.745765in}}%
\pgfpathlineto{\pgfqpoint{1.495197in}{0.746804in}}%
\pgfpathlineto{\pgfqpoint{1.567359in}{0.759547in}}%
\pgfpathlineto{\pgfqpoint{1.585399in}{0.760932in}}%
\pgfpathlineto{\pgfqpoint{1.587977in}{0.762311in}}%
\pgfpathlineto{\pgfqpoint{1.593131in}{0.766396in}}%
\pgfpathlineto{\pgfqpoint{1.603440in}{0.769183in}}%
\pgfpathlineto{\pgfqpoint{1.611172in}{0.772402in}}%
\pgfpathlineto{\pgfqpoint{1.624058in}{0.773356in}}%
\pgfpathlineto{\pgfqpoint{1.629212in}{0.775556in}}%
\pgfpathlineto{\pgfqpoint{1.636944in}{0.776414in}}%
\pgfpathlineto{\pgfqpoint{1.642098in}{0.777969in}}%
\pgfpathlineto{\pgfqpoint{1.647253in}{0.778749in}}%
\pgfpathlineto{\pgfqpoint{1.678179in}{0.780611in}}%
\pgfpathlineto{\pgfqpoint{1.716837in}{0.784437in}}%
\pgfpathlineto{\pgfqpoint{1.719414in}{0.785070in}}%
\pgfpathlineto{\pgfqpoint{1.727146in}{0.785745in}}%
\pgfpathlineto{\pgfqpoint{1.737455in}{0.788684in}}%
\pgfpathlineto{\pgfqpoint{1.747764in}{0.790095in}}%
\pgfpathlineto{\pgfqpoint{1.755495in}{0.791679in}}%
\pgfpathlineto{\pgfqpoint{1.781268in}{0.793225in}}%
\pgfpathlineto{\pgfqpoint{1.788999in}{0.794276in}}%
\pgfpathlineto{\pgfqpoint{1.791576in}{0.795043in}}%
\pgfpathlineto{\pgfqpoint{1.801885in}{0.796425in}}%
\pgfpathlineto{\pgfqpoint{1.809617in}{0.799117in}}%
\pgfpathlineto{\pgfqpoint{1.817349in}{0.800154in}}%
\pgfpathlineto{\pgfqpoint{1.827657in}{0.804391in}}%
\pgfpathlineto{\pgfqpoint{1.837966in}{0.805353in}}%
\pgfpathlineto{\pgfqpoint{1.845698in}{0.808817in}}%
\pgfpathlineto{\pgfqpoint{1.853430in}{0.810043in}}%
\pgfpathlineto{\pgfqpoint{1.861161in}{0.814184in}}%
\pgfpathlineto{\pgfqpoint{1.863738in}{0.815515in}}%
\pgfpathlineto{\pgfqpoint{1.871470in}{0.816785in}}%
\pgfpathlineto{\pgfqpoint{1.881779in}{0.823270in}}%
\pgfpathlineto{\pgfqpoint{1.889510in}{0.824906in}}%
\pgfpathlineto{\pgfqpoint{1.899819in}{0.831805in}}%
\pgfpathlineto{\pgfqpoint{1.910128in}{0.833917in}}%
\pgfpathlineto{\pgfqpoint{1.917860in}{0.839720in}}%
\pgfpathlineto{\pgfqpoint{1.925591in}{0.841297in}}%
\pgfpathlineto{\pgfqpoint{1.935900in}{0.848315in}}%
\pgfpathlineto{\pgfqpoint{1.943632in}{0.850383in}}%
\pgfpathlineto{\pgfqpoint{1.953941in}{0.859105in}}%
\pgfpathlineto{\pgfqpoint{1.961672in}{0.861460in}}%
\pgfpathlineto{\pgfqpoint{1.971981in}{0.871033in}}%
\pgfpathlineto{\pgfqpoint{1.979713in}{0.873262in}}%
\pgfpathlineto{\pgfqpoint{1.990022in}{0.882430in}}%
\pgfpathlineto{\pgfqpoint{1.997753in}{0.884761in}}%
\pgfpathlineto{\pgfqpoint{2.005485in}{0.893428in}}%
\pgfpathlineto{\pgfqpoint{2.015794in}{0.896500in}}%
\pgfpathlineto{\pgfqpoint{2.023526in}{0.905886in}}%
\pgfpathlineto{\pgfqpoint{2.026103in}{0.908703in}}%
\pgfpathlineto{\pgfqpoint{2.033834in}{0.911065in}}%
\pgfpathlineto{\pgfqpoint{2.044143in}{0.921655in}}%
\pgfpathlineto{\pgfqpoint{2.051875in}{0.923976in}}%
\pgfpathlineto{\pgfqpoint{2.062184in}{0.936011in}}%
\pgfpathlineto{\pgfqpoint{2.069915in}{0.939302in}}%
\pgfpathlineto{\pgfqpoint{2.077647in}{0.948967in}}%
\pgfpathlineto{\pgfqpoint{2.080224in}{0.952078in}}%
\pgfpathlineto{\pgfqpoint{2.087956in}{0.955311in}}%
\pgfpathlineto{\pgfqpoint{2.098265in}{0.966780in}}%
\pgfpathlineto{\pgfqpoint{2.105996in}{0.969335in}}%
\pgfpathlineto{\pgfqpoint{2.116305in}{0.980200in}}%
\pgfpathlineto{\pgfqpoint{2.124037in}{0.982916in}}%
\pgfpathlineto{\pgfqpoint{2.134346in}{0.995721in}}%
\pgfpathlineto{\pgfqpoint{2.142077in}{0.998936in}}%
\pgfpathlineto{\pgfqpoint{2.149809in}{1.008511in}}%
\pgfpathlineto{\pgfqpoint{2.152386in}{1.011266in}}%
\pgfpathlineto{\pgfqpoint{2.162695in}{1.014252in}}%
\pgfpathlineto{\pgfqpoint{2.170427in}{1.020536in}}%
\pgfpathlineto{\pgfqpoint{2.178158in}{1.022378in}}%
\pgfpathlineto{\pgfqpoint{2.188467in}{1.028962in}}%
\pgfpathlineto{\pgfqpoint{2.196199in}{1.030779in}}%
\pgfpathlineto{\pgfqpoint{2.206508in}{1.037111in}}%
\pgfpathlineto{\pgfqpoint{2.214239in}{1.038921in}}%
\pgfpathlineto{\pgfqpoint{2.219394in}{1.042451in}}%
\pgfpathlineto{\pgfqpoint{2.224548in}{1.044488in}}%
\pgfpathlineto{\pgfqpoint{2.232280in}{1.045895in}}%
\pgfpathlineto{\pgfqpoint{2.237434in}{1.049484in}}%
\pgfpathlineto{\pgfqpoint{2.242589in}{1.053007in}}%
\pgfpathlineto{\pgfqpoint{2.250320in}{1.054818in}}%
\pgfpathlineto{\pgfqpoint{2.255475in}{1.058378in}}%
\pgfpathlineto{\pgfqpoint{2.260629in}{1.060448in}}%
\pgfpathlineto{\pgfqpoint{2.268361in}{1.062697in}}%
\pgfpathlineto{\pgfqpoint{2.278670in}{1.072320in}}%
\pgfpathlineto{\pgfqpoint{2.286401in}{1.074912in}}%
\pgfpathlineto{\pgfqpoint{2.296710in}{1.085262in}}%
\pgfpathlineto{\pgfqpoint{2.304442in}{1.088212in}}%
\pgfpathlineto{\pgfqpoint{2.314751in}{1.099834in}}%
\pgfpathlineto{\pgfqpoint{2.322482in}{1.102821in}}%
\pgfpathlineto{\pgfqpoint{2.332791in}{1.114878in}}%
\pgfpathlineto{\pgfqpoint{2.340523in}{1.117789in}}%
\pgfpathlineto{\pgfqpoint{2.348254in}{1.126061in}}%
\pgfpathlineto{\pgfqpoint{2.350832in}{1.128375in}}%
\pgfpathlineto{\pgfqpoint{2.358563in}{1.130557in}}%
\pgfpathlineto{\pgfqpoint{2.363718in}{1.134756in}}%
\pgfpathlineto{\pgfqpoint{2.368872in}{1.137657in}}%
\pgfpathlineto{\pgfqpoint{2.376604in}{1.139303in}}%
\pgfpathlineto{\pgfqpoint{2.384335in}{1.143092in}}%
\pgfpathlineto{\pgfqpoint{2.386912in}{1.144348in}}%
\pgfpathlineto{\pgfqpoint{2.397221in}{1.146179in}}%
\pgfpathlineto{\pgfqpoint{2.404953in}{1.148624in}}%
\pgfpathlineto{\pgfqpoint{2.415262in}{1.149400in}}%
\pgfpathlineto{\pgfqpoint{2.422993in}{1.151997in}}%
\pgfpathlineto{\pgfqpoint{2.430725in}{1.152937in}}%
\pgfpathlineto{\pgfqpoint{2.441034in}{1.157557in}}%
\pgfpathlineto{\pgfqpoint{2.448766in}{1.158712in}}%
\pgfpathlineto{\pgfqpoint{2.459074in}{1.163694in}}%
\pgfpathlineto{\pgfqpoint{2.469383in}{1.165684in}}%
\pgfpathlineto{\pgfqpoint{2.477115in}{1.167660in}}%
\pgfpathlineto{\pgfqpoint{2.487424in}{1.168972in}}%
\pgfpathlineto{\pgfqpoint{2.495155in}{1.170860in}}%
\pgfpathlineto{\pgfqpoint{2.508041in}{1.172195in}}%
\pgfpathlineto{\pgfqpoint{2.513196in}{1.173891in}}%
\pgfpathlineto{\pgfqpoint{2.520928in}{1.174934in}}%
\pgfpathlineto{\pgfqpoint{2.528659in}{1.178678in}}%
\pgfpathlineto{\pgfqpoint{2.531236in}{1.180014in}}%
\pgfpathlineto{\pgfqpoint{2.538968in}{1.181247in}}%
\pgfpathlineto{\pgfqpoint{2.549277in}{1.186823in}}%
\pgfpathlineto{\pgfqpoint{2.557009in}{1.188206in}}%
\pgfpathlineto{\pgfqpoint{2.567317in}{1.194030in}}%
\pgfpathlineto{\pgfqpoint{2.575049in}{1.195450in}}%
\pgfpathlineto{\pgfqpoint{2.585358in}{1.201041in}}%
\pgfpathlineto{\pgfqpoint{2.593089in}{1.202623in}}%
\pgfpathlineto{\pgfqpoint{2.603398in}{1.208368in}}%
\pgfpathlineto{\pgfqpoint{2.611130in}{1.209856in}}%
\pgfpathlineto{\pgfqpoint{2.621439in}{1.216409in}}%
\pgfpathlineto{\pgfqpoint{2.629170in}{1.218206in}}%
\pgfpathlineto{\pgfqpoint{2.634325in}{1.221478in}}%
\pgfpathlineto{\pgfqpoint{2.639479in}{1.223009in}}%
\pgfpathlineto{\pgfqpoint{2.647211in}{1.224448in}}%
\pgfpathlineto{\pgfqpoint{2.657520in}{1.229633in}}%
\pgfpathlineto{\pgfqpoint{2.665251in}{1.231016in}}%
\pgfpathlineto{\pgfqpoint{2.670406in}{1.233436in}}%
\pgfpathlineto{\pgfqpoint{2.675560in}{1.234997in}}%
\pgfpathlineto{\pgfqpoint{2.685869in}{1.236446in}}%
\pgfpathlineto{\pgfqpoint{2.693601in}{1.239125in}}%
\pgfpathlineto{\pgfqpoint{2.703910in}{1.240796in}}%
\pgfpathlineto{\pgfqpoint{2.739991in}{1.247400in}}%
\pgfpathlineto{\pgfqpoint{2.747722in}{1.250796in}}%
\pgfpathlineto{\pgfqpoint{2.755454in}{1.251895in}}%
\pgfpathlineto{\pgfqpoint{2.765763in}{1.256363in}}%
\pgfpathlineto{\pgfqpoint{2.776072in}{1.257321in}}%
\pgfpathlineto{\pgfqpoint{2.781226in}{1.259052in}}%
\pgfpathlineto{\pgfqpoint{2.783803in}{1.259486in}}%
\pgfpathlineto{\pgfqpoint{2.827616in}{1.261711in}}%
\pgfpathlineto{\pgfqpoint{2.837925in}{1.264267in}}%
\pgfpathlineto{\pgfqpoint{2.850811in}{1.265291in}}%
\pgfpathlineto{\pgfqpoint{2.855965in}{1.266388in}}%
\pgfpathlineto{\pgfqpoint{2.866274in}{1.267324in}}%
\pgfpathlineto{\pgfqpoint{2.874006in}{1.268850in}}%
\pgfpathlineto{\pgfqpoint{2.881737in}{1.269349in}}%
\pgfpathlineto{\pgfqpoint{2.892046in}{1.272068in}}%
\pgfpathlineto{\pgfqpoint{2.902355in}{1.273501in}}%
\pgfpathlineto{\pgfqpoint{2.910087in}{1.275457in}}%
\pgfpathlineto{\pgfqpoint{2.920395in}{1.276943in}}%
\pgfpathlineto{\pgfqpoint{2.928127in}{1.279380in}}%
\pgfpathlineto{\pgfqpoint{2.935859in}{1.280266in}}%
\pgfpathlineto{\pgfqpoint{2.946168in}{1.284986in}}%
\pgfpathlineto{\pgfqpoint{2.953899in}{1.286287in}}%
\pgfpathlineto{\pgfqpoint{2.964208in}{1.291344in}}%
\pgfpathlineto{\pgfqpoint{2.971940in}{1.292524in}}%
\pgfpathlineto{\pgfqpoint{2.982249in}{1.296958in}}%
\pgfpathlineto{\pgfqpoint{2.989980in}{1.297965in}}%
\pgfpathlineto{\pgfqpoint{2.997712in}{1.301762in}}%
\pgfpathlineto{\pgfqpoint{3.008021in}{1.303170in}}%
\pgfpathlineto{\pgfqpoint{3.018330in}{1.308677in}}%
\pgfpathlineto{\pgfqpoint{3.026061in}{1.310235in}}%
\pgfpathlineto{\pgfqpoint{3.036370in}{1.315790in}}%
\pgfpathlineto{\pgfqpoint{3.044102in}{1.317056in}}%
\pgfpathlineto{\pgfqpoint{3.054411in}{1.322281in}}%
\pgfpathlineto{\pgfqpoint{3.062142in}{1.323563in}}%
\pgfpathlineto{\pgfqpoint{3.072451in}{1.328727in}}%
\pgfpathlineto{\pgfqpoint{3.080183in}{1.329962in}}%
\pgfpathlineto{\pgfqpoint{3.090491in}{1.335118in}}%
\pgfpathlineto{\pgfqpoint{3.100800in}{1.336425in}}%
\pgfpathlineto{\pgfqpoint{3.108532in}{1.340282in}}%
\pgfpathlineto{\pgfqpoint{3.116264in}{1.341750in}}%
\pgfpathlineto{\pgfqpoint{3.126572in}{1.347941in}}%
\pgfpathlineto{\pgfqpoint{3.134304in}{1.349504in}}%
\pgfpathlineto{\pgfqpoint{3.142036in}{1.354153in}}%
\pgfpathlineto{\pgfqpoint{3.144613in}{1.355546in}}%
\pgfpathlineto{\pgfqpoint{3.152345in}{1.356913in}}%
\pgfpathlineto{\pgfqpoint{3.160076in}{1.361126in}}%
\pgfpathlineto{\pgfqpoint{3.162653in}{1.362891in}}%
\pgfpathlineto{\pgfqpoint{3.170385in}{1.364552in}}%
\pgfpathlineto{\pgfqpoint{3.180694in}{1.371361in}}%
\pgfpathlineto{\pgfqpoint{3.188426in}{1.372919in}}%
\pgfpathlineto{\pgfqpoint{3.196157in}{1.378029in}}%
\pgfpathlineto{\pgfqpoint{3.206466in}{1.379824in}}%
\pgfpathlineto{\pgfqpoint{3.216775in}{1.386312in}}%
\pgfpathlineto{\pgfqpoint{3.224507in}{1.387852in}}%
\pgfpathlineto{\pgfqpoint{3.232238in}{1.390917in}}%
\pgfpathlineto{\pgfqpoint{3.234815in}{1.391895in}}%
\pgfpathlineto{\pgfqpoint{3.242547in}{1.392794in}}%
\pgfpathlineto{\pgfqpoint{3.252856in}{1.396813in}}%
\pgfpathlineto{\pgfqpoint{3.260588in}{1.397776in}}%
\pgfpathlineto{\pgfqpoint{3.270896in}{1.401028in}}%
\pgfpathlineto{\pgfqpoint{3.281205in}{1.402346in}}%
\pgfpathlineto{\pgfqpoint{3.288937in}{1.404481in}}%
\pgfpathlineto{\pgfqpoint{3.299246in}{1.405945in}}%
\pgfpathlineto{\pgfqpoint{3.306977in}{1.408359in}}%
\pgfpathlineto{\pgfqpoint{3.314709in}{1.409280in}}%
\pgfpathlineto{\pgfqpoint{3.325018in}{1.413304in}}%
\pgfpathlineto{\pgfqpoint{3.332749in}{1.414337in}}%
\pgfpathlineto{\pgfqpoint{3.343058in}{1.418432in}}%
\pgfpathlineto{\pgfqpoint{3.353367in}{1.419436in}}%
\pgfpathlineto{\pgfqpoint{3.361099in}{1.422645in}}%
\pgfpathlineto{\pgfqpoint{3.368830in}{1.423701in}}%
\pgfpathlineto{\pgfqpoint{3.379139in}{1.428060in}}%
\pgfpathlineto{\pgfqpoint{3.386871in}{1.429156in}}%
\pgfpathlineto{\pgfqpoint{3.397180in}{1.434595in}}%
\pgfpathlineto{\pgfqpoint{3.404911in}{1.436067in}}%
\pgfpathlineto{\pgfqpoint{3.415220in}{1.441664in}}%
\pgfpathlineto{\pgfqpoint{3.422952in}{1.442891in}}%
\pgfpathlineto{\pgfqpoint{3.428106in}{1.445034in}}%
\pgfpathlineto{\pgfqpoint{3.433261in}{1.446899in}}%
\pgfpathlineto{\pgfqpoint{3.443570in}{1.448532in}}%
\pgfpathlineto{\pgfqpoint{3.448724in}{1.450111in}}%
\pgfpathlineto{\pgfqpoint{3.451301in}{1.450637in}}%
\pgfpathlineto{\pgfqpoint{3.477073in}{1.451977in}}%
\pgfpathlineto{\pgfqpoint{3.484805in}{1.453516in}}%
\pgfpathlineto{\pgfqpoint{3.487382in}{1.454200in}}%
\pgfpathlineto{\pgfqpoint{3.495114in}{1.454982in}}%
\pgfpathlineto{\pgfqpoint{3.502845in}{1.457896in}}%
\pgfpathlineto{\pgfqpoint{3.505423in}{1.459095in}}%
\pgfpathlineto{\pgfqpoint{3.513154in}{1.460244in}}%
\pgfpathlineto{\pgfqpoint{3.523463in}{1.465356in}}%
\pgfpathlineto{\pgfqpoint{3.531195in}{1.466628in}}%
\pgfpathlineto{\pgfqpoint{3.541504in}{1.471561in}}%
\pgfpathlineto{\pgfqpoint{3.549235in}{1.472715in}}%
\pgfpathlineto{\pgfqpoint{3.559544in}{1.477378in}}%
\pgfpathlineto{\pgfqpoint{3.567276in}{1.478398in}}%
\pgfpathlineto{\pgfqpoint{3.572430in}{1.480430in}}%
\pgfpathlineto{\pgfqpoint{3.577585in}{1.481591in}}%
\pgfpathlineto{\pgfqpoint{3.585316in}{1.482715in}}%
\pgfpathlineto{\pgfqpoint{3.595625in}{1.487146in}}%
\pgfpathlineto{\pgfqpoint{3.603357in}{1.488286in}}%
\pgfpathlineto{\pgfqpoint{3.611088in}{1.491114in}}%
\pgfpathlineto{\pgfqpoint{3.613666in}{1.491759in}}%
\pgfpathlineto{\pgfqpoint{3.623974in}{1.492820in}}%
\pgfpathlineto{\pgfqpoint{3.644592in}{1.497095in}}%
\pgfpathlineto{\pgfqpoint{3.680673in}{1.501935in}}%
\pgfpathlineto{\pgfqpoint{3.685828in}{1.503290in}}%
\pgfpathlineto{\pgfqpoint{3.696136in}{1.504423in}}%
\pgfpathlineto{\pgfqpoint{3.703868in}{1.505763in}}%
\pgfpathlineto{\pgfqpoint{3.716754in}{1.506299in}}%
\pgfpathlineto{\pgfqpoint{3.721909in}{1.507064in}}%
\pgfpathlineto{\pgfqpoint{3.737372in}{1.508228in}}%
\pgfpathlineto{\pgfqpoint{3.776030in}{1.509988in}}%
\pgfpathlineto{\pgfqpoint{3.804379in}{1.510863in}}%
\pgfpathlineto{\pgfqpoint{3.837883in}{1.513345in}}%
\pgfpathlineto{\pgfqpoint{3.873964in}{1.514907in}}%
\pgfpathlineto{\pgfqpoint{3.884273in}{1.515479in}}%
\pgfpathlineto{\pgfqpoint{4.026020in}{1.519635in}}%
\pgfpathlineto{\pgfqpoint{4.041483in}{1.520071in}}%
\pgfpathlineto{\pgfqpoint{4.056946in}{1.520415in}}%
\pgfpathlineto{\pgfqpoint{4.074987in}{1.520333in}}%
\pgfpathlineto{\pgfqpoint{4.219311in}{1.519921in}}%
\pgfpathlineto{\pgfqpoint{4.239928in}{1.519722in}}%
\pgfpathlineto{\pgfqpoint{4.288895in}{1.517689in}}%
\pgfpathlineto{\pgfqpoint{4.299204in}{1.516410in}}%
\pgfpathlineto{\pgfqpoint{4.314667in}{1.515516in}}%
\pgfpathlineto{\pgfqpoint{4.317245in}{1.515216in}}%
\pgfpathlineto{\pgfqpoint{4.332708in}{1.514223in}}%
\pgfpathlineto{\pgfqpoint{4.353326in}{1.512391in}}%
\pgfpathlineto{\pgfqpoint{4.368789in}{1.511177in}}%
\pgfpathlineto{\pgfqpoint{4.381675in}{1.510406in}}%
\pgfpathlineto{\pgfqpoint{4.402293in}{1.509146in}}%
\pgfpathlineto{\pgfqpoint{4.422910in}{1.508651in}}%
\pgfpathlineto{\pgfqpoint{4.492495in}{1.510251in}}%
\pgfpathlineto{\pgfqpoint{4.515690in}{1.511305in}}%
\pgfpathlineto{\pgfqpoint{4.533730in}{1.512031in}}%
\pgfpathlineto{\pgfqpoint{4.562080in}{1.513324in}}%
\pgfpathlineto{\pgfqpoint{4.569811in}{1.514051in}}%
\pgfpathlineto{\pgfqpoint{4.585275in}{1.514793in}}%
\pgfpathlineto{\pgfqpoint{4.636819in}{1.515272in}}%
\pgfpathlineto{\pgfqpoint{4.678054in}{1.514767in}}%
\pgfpathlineto{\pgfqpoint{4.742485in}{1.517349in}}%
\pgfpathlineto{\pgfqpoint{4.819801in}{1.526835in}}%
\pgfpathlineto{\pgfqpoint{4.832687in}{1.527892in}}%
\pgfpathlineto{\pgfqpoint{4.840419in}{1.529478in}}%
\pgfpathlineto{\pgfqpoint{4.850728in}{1.530516in}}%
\pgfpathlineto{\pgfqpoint{4.858459in}{1.532165in}}%
\pgfpathlineto{\pgfqpoint{4.868768in}{1.533252in}}%
\pgfpathlineto{\pgfqpoint{4.876500in}{1.535018in}}%
\pgfpathlineto{\pgfqpoint{4.884231in}{1.535677in}}%
\pgfpathlineto{\pgfqpoint{4.894540in}{1.539071in}}%
\pgfpathlineto{\pgfqpoint{4.904849in}{1.540726in}}%
\pgfpathlineto{\pgfqpoint{4.912581in}{1.542979in}}%
\pgfpathlineto{\pgfqpoint{4.922890in}{1.544506in}}%
\pgfpathlineto{\pgfqpoint{4.930621in}{1.546740in}}%
\pgfpathlineto{\pgfqpoint{4.938353in}{1.547571in}}%
\pgfpathlineto{\pgfqpoint{4.948662in}{1.551016in}}%
\pgfpathlineto{\pgfqpoint{4.958971in}{1.552698in}}%
\pgfpathlineto{\pgfqpoint{4.966702in}{1.554974in}}%
\pgfpathlineto{\pgfqpoint{4.977011in}{1.556459in}}%
\pgfpathlineto{\pgfqpoint{4.984743in}{1.558827in}}%
\pgfpathlineto{\pgfqpoint{4.995051in}{1.559573in}}%
\pgfpathlineto{\pgfqpoint{5.002783in}{1.562123in}}%
\pgfpathlineto{\pgfqpoint{5.010515in}{1.563120in}}%
\pgfpathlineto{\pgfqpoint{5.020824in}{1.567277in}}%
\pgfpathlineto{\pgfqpoint{5.028555in}{1.568340in}}%
\pgfpathlineto{\pgfqpoint{5.038864in}{1.572396in}}%
\pgfpathlineto{\pgfqpoint{5.046596in}{1.573408in}}%
\pgfpathlineto{\pgfqpoint{5.056905in}{1.577356in}}%
\pgfpathlineto{\pgfqpoint{5.064636in}{1.578342in}}%
\pgfpathlineto{\pgfqpoint{5.072368in}{1.582131in}}%
\pgfpathlineto{\pgfqpoint{5.074945in}{1.583545in}}%
\pgfpathlineto{\pgfqpoint{5.085254in}{1.585051in}}%
\pgfpathlineto{\pgfqpoint{5.092986in}{1.589644in}}%
\pgfpathlineto{\pgfqpoint{5.100717in}{1.591186in}}%
\pgfpathlineto{\pgfqpoint{5.111026in}{1.597304in}}%
\pgfpathlineto{\pgfqpoint{5.118758in}{1.598824in}}%
\pgfpathlineto{\pgfqpoint{5.129067in}{1.605662in}}%
\pgfpathlineto{\pgfqpoint{5.136798in}{1.607321in}}%
\pgfpathlineto{\pgfqpoint{5.147107in}{1.613867in}}%
\pgfpathlineto{\pgfqpoint{5.154839in}{1.615533in}}%
\pgfpathlineto{\pgfqpoint{5.165148in}{1.621617in}}%
\pgfpathlineto{\pgfqpoint{5.172879in}{1.623060in}}%
\pgfpathlineto{\pgfqpoint{5.183188in}{1.628676in}}%
\pgfpathlineto{\pgfqpoint{5.190920in}{1.629952in}}%
\pgfpathlineto{\pgfqpoint{5.201228in}{1.634426in}}%
\pgfpathlineto{\pgfqpoint{5.208960in}{1.635454in}}%
\pgfpathlineto{\pgfqpoint{5.219269in}{1.639452in}}%
\pgfpathlineto{\pgfqpoint{5.227001in}{1.640525in}}%
\pgfpathlineto{\pgfqpoint{5.237309in}{1.644545in}}%
\pgfpathlineto{\pgfqpoint{5.247618in}{1.645575in}}%
\pgfpathlineto{\pgfqpoint{5.255350in}{1.648472in}}%
\pgfpathlineto{\pgfqpoint{5.265659in}{1.650259in}}%
\pgfpathlineto{\pgfqpoint{5.273390in}{1.652871in}}%
\pgfpathlineto{\pgfqpoint{5.281122in}{1.653686in}}%
\pgfpathlineto{\pgfqpoint{5.291431in}{1.657266in}}%
\pgfpathlineto{\pgfqpoint{5.299163in}{1.658069in}}%
\pgfpathlineto{\pgfqpoint{5.309471in}{1.661472in}}%
\pgfpathlineto{\pgfqpoint{5.317203in}{1.662336in}}%
\pgfpathlineto{\pgfqpoint{5.327512in}{1.665809in}}%
\pgfpathlineto{\pgfqpoint{5.337821in}{1.667479in}}%
\pgfpathlineto{\pgfqpoint{5.345552in}{1.669763in}}%
\pgfpathlineto{\pgfqpoint{5.358438in}{1.671632in}}%
\pgfpathlineto{\pgfqpoint{5.363593in}{1.672580in}}%
\pgfpathlineto{\pgfqpoint{5.373902in}{1.673469in}}%
\pgfpathlineto{\pgfqpoint{5.381633in}{1.675066in}}%
\pgfpathlineto{\pgfqpoint{5.391942in}{1.676181in}}%
\pgfpathlineto{\pgfqpoint{5.399674in}{1.677698in}}%
\pgfpathlineto{\pgfqpoint{5.409983in}{1.678953in}}%
\pgfpathlineto{\pgfqpoint{5.417714in}{1.681385in}}%
\pgfpathlineto{\pgfqpoint{5.428023in}{1.682551in}}%
\pgfpathlineto{\pgfqpoint{5.435755in}{1.684257in}}%
\pgfpathlineto{\pgfqpoint{5.461527in}{1.686403in}}%
\pgfpathlineto{\pgfqpoint{5.471836in}{1.687629in}}%
\pgfpathlineto{\pgfqpoint{5.487299in}{1.688767in}}%
\pgfpathlineto{\pgfqpoint{5.523380in}{1.693802in}}%
\pgfpathlineto{\pgfqpoint{5.525957in}{1.694336in}}%
\pgfpathlineto{\pgfqpoint{5.538843in}{1.695336in}}%
\pgfpathlineto{\pgfqpoint{5.543998in}{1.696306in}}%
\pgfpathlineto{\pgfqpoint{5.556884in}{1.697318in}}%
\pgfpathlineto{\pgfqpoint{5.562038in}{1.698430in}}%
\pgfpathlineto{\pgfqpoint{5.572347in}{1.699484in}}%
\pgfpathlineto{\pgfqpoint{5.580079in}{1.700744in}}%
\pgfpathlineto{\pgfqpoint{5.592965in}{1.701589in}}%
\pgfpathlineto{\pgfqpoint{5.598119in}{1.702352in}}%
\pgfpathlineto{\pgfqpoint{5.613582in}{1.703483in}}%
\pgfpathlineto{\pgfqpoint{5.616160in}{1.703807in}}%
\pgfpathlineto{\pgfqpoint{5.629046in}{1.704737in}}%
\pgfpathlineto{\pgfqpoint{5.634200in}{1.705395in}}%
\pgfpathlineto{\pgfqpoint{5.647086in}{1.706339in}}%
\pgfpathlineto{\pgfqpoint{5.685744in}{1.712017in}}%
\pgfpathlineto{\pgfqpoint{5.688322in}{1.712990in}}%
\pgfpathlineto{\pgfqpoint{5.696053in}{1.713932in}}%
\pgfpathlineto{\pgfqpoint{5.706362in}{1.717964in}}%
\pgfpathlineto{\pgfqpoint{5.714094in}{1.718987in}}%
\pgfpathlineto{\pgfqpoint{5.724403in}{1.723462in}}%
\pgfpathlineto{\pgfqpoint{5.732134in}{1.724705in}}%
\pgfpathlineto{\pgfqpoint{5.742443in}{1.730127in}}%
\pgfpathlineto{\pgfqpoint{5.750175in}{1.731461in}}%
\pgfpathlineto{\pgfqpoint{5.760484in}{1.736107in}}%
\pgfpathlineto{\pgfqpoint{5.768215in}{1.737220in}}%
\pgfpathlineto{\pgfqpoint{5.778524in}{1.741349in}}%
\pgfpathlineto{\pgfqpoint{5.786256in}{1.742349in}}%
\pgfpathlineto{\pgfqpoint{5.796565in}{1.746357in}}%
\pgfpathlineto{\pgfqpoint{5.804296in}{1.747308in}}%
\pgfpathlineto{\pgfqpoint{5.812028in}{1.750256in}}%
\pgfpathlineto{\pgfqpoint{5.824914in}{1.752040in}}%
\pgfpathlineto{\pgfqpoint{5.832646in}{1.754222in}}%
\pgfpathlineto{\pgfqpoint{5.840377in}{1.755029in}}%
\pgfpathlineto{\pgfqpoint{5.850686in}{1.758408in}}%
\pgfpathlineto{\pgfqpoint{5.858418in}{1.759228in}}%
\pgfpathlineto{\pgfqpoint{5.868727in}{1.762557in}}%
\pgfpathlineto{\pgfqpoint{5.879035in}{1.764143in}}%
\pgfpathlineto{\pgfqpoint{5.886767in}{1.766494in}}%
\pgfpathlineto{\pgfqpoint{5.894499in}{1.767550in}}%
\pgfpathlineto{\pgfqpoint{5.904807in}{1.771764in}}%
\pgfpathlineto{\pgfqpoint{5.912539in}{1.772818in}}%
\pgfpathlineto{\pgfqpoint{5.922848in}{1.777152in}}%
\pgfpathlineto{\pgfqpoint{5.933157in}{1.778236in}}%
\pgfpathlineto{\pgfqpoint{5.940888in}{1.781924in}}%
\pgfpathlineto{\pgfqpoint{5.948620in}{1.783252in}}%
\pgfpathlineto{\pgfqpoint{5.958929in}{1.788693in}}%
\pgfpathlineto{\pgfqpoint{5.966661in}{1.790115in}}%
\pgfpathlineto{\pgfqpoint{5.976969in}{1.796164in}}%
\pgfpathlineto{\pgfqpoint{5.984701in}{1.797750in}}%
\pgfpathlineto{\pgfqpoint{5.995010in}{1.804501in}}%
\pgfpathlineto{\pgfqpoint{6.002742in}{1.806254in}}%
\pgfpathlineto{\pgfqpoint{6.013050in}{1.812158in}}%
\pgfpathlineto{\pgfqpoint{6.025936in}{1.815038in}}%
\pgfpathlineto{\pgfqpoint{6.031091in}{1.817767in}}%
\pgfpathlineto{\pgfqpoint{6.038823in}{1.819041in}}%
\pgfpathlineto{\pgfqpoint{6.049131in}{1.824223in}}%
\pgfpathlineto{\pgfqpoint{6.056863in}{1.825518in}}%
\pgfpathlineto{\pgfqpoint{6.067172in}{1.831729in}}%
\pgfpathlineto{\pgfqpoint{6.074904in}{1.833063in}}%
\pgfpathlineto{\pgfqpoint{6.085212in}{1.837820in}}%
\pgfpathlineto{\pgfqpoint{6.092944in}{1.839103in}}%
\pgfpathlineto{\pgfqpoint{6.103253in}{1.844203in}}%
\pgfpathlineto{\pgfqpoint{6.110984in}{1.845469in}}%
\pgfpathlineto{\pgfqpoint{6.121293in}{1.850499in}}%
\pgfpathlineto{\pgfqpoint{6.129025in}{1.851787in}}%
\pgfpathlineto{\pgfqpoint{6.139334in}{1.856828in}}%
\pgfpathlineto{\pgfqpoint{6.147065in}{1.858086in}}%
\pgfpathlineto{\pgfqpoint{6.157374in}{1.862996in}}%
\pgfpathlineto{\pgfqpoint{6.165106in}{1.864160in}}%
\pgfpathlineto{\pgfqpoint{6.175415in}{1.868745in}}%
\pgfpathlineto{\pgfqpoint{6.185724in}{1.869747in}}%
\pgfpathlineto{\pgfqpoint{6.193455in}{1.873034in}}%
\pgfpathlineto{\pgfqpoint{6.201187in}{1.874267in}}%
\pgfpathlineto{\pgfqpoint{6.211496in}{1.879218in}}%
\pgfpathlineto{\pgfqpoint{6.219227in}{1.880597in}}%
\pgfpathlineto{\pgfqpoint{6.226959in}{1.884234in}}%
\pgfpathlineto{\pgfqpoint{6.229536in}{1.885285in}}%
\pgfpathlineto{\pgfqpoint{6.237268in}{1.886315in}}%
\pgfpathlineto{\pgfqpoint{6.247577in}{1.890073in}}%
\pgfpathlineto{\pgfqpoint{6.255308in}{1.891085in}}%
\pgfpathlineto{\pgfqpoint{6.265617in}{1.895568in}}%
\pgfpathlineto{\pgfqpoint{6.273349in}{1.896722in}}%
\pgfpathlineto{\pgfqpoint{6.283658in}{1.902060in}}%
\pgfpathlineto{\pgfqpoint{6.291389in}{1.903392in}}%
\pgfpathlineto{\pgfqpoint{6.301698in}{1.910439in}}%
\pgfpathlineto{\pgfqpoint{6.309430in}{1.912370in}}%
\pgfpathlineto{\pgfqpoint{6.319739in}{1.919402in}}%
\pgfpathlineto{\pgfqpoint{6.327470in}{1.920980in}}%
\pgfpathlineto{\pgfqpoint{6.337779in}{1.927173in}}%
\pgfpathlineto{\pgfqpoint{6.345511in}{1.928698in}}%
\pgfpathlineto{\pgfqpoint{6.355820in}{1.934892in}}%
\pgfpathlineto{\pgfqpoint{6.363551in}{1.936383in}}%
\pgfpathlineto{\pgfqpoint{6.373860in}{1.942005in}}%
\pgfpathlineto{\pgfqpoint{6.381592in}{1.943323in}}%
\pgfpathlineto{\pgfqpoint{6.386746in}{1.945925in}}%
\pgfpathlineto{\pgfqpoint{6.391901in}{1.947232in}}%
\pgfpathlineto{\pgfqpoint{6.399632in}{1.948607in}}%
\pgfpathlineto{\pgfqpoint{6.409941in}{1.954577in}}%
\pgfpathlineto{\pgfqpoint{6.417673in}{1.955993in}}%
\pgfpathlineto{\pgfqpoint{6.427982in}{1.962015in}}%
\pgfpathlineto{\pgfqpoint{6.435713in}{1.963569in}}%
\pgfpathlineto{\pgfqpoint{6.446022in}{1.970135in}}%
\pgfpathlineto{\pgfqpoint{6.453754in}{1.971709in}}%
\pgfpathlineto{\pgfqpoint{6.464063in}{1.977696in}}%
\pgfpathlineto{\pgfqpoint{6.474371in}{1.979103in}}%
\pgfpathlineto{\pgfqpoint{6.482103in}{1.983336in}}%
\pgfpathlineto{\pgfqpoint{6.482103in}{1.983336in}}%
\pgfusepath{stroke}%
\end{pgfscope}%
\begin{pgfscope}%
\pgfpathrectangle{\pgfqpoint{0.563921in}{0.521603in}}{\pgfqpoint{6.200000in}{2.642500in}}%
\pgfusepath{clip}%
\pgfsetroundcap%
\pgfsetroundjoin%
\pgfsetlinewidth{1.505625pt}%
\definecolor{currentstroke}{rgb}{0.549020,0.337255,0.294118}%
\pgfsetstrokecolor{currentstroke}%
\pgfsetdash{}{0pt}%
\pgfpathmoveto{\pgfqpoint{0.845739in}{0.641717in}}%
\pgfpathlineto{\pgfqpoint{0.848317in}{0.642313in}}%
\pgfpathlineto{\pgfqpoint{0.850894in}{0.648196in}}%
\pgfpathlineto{\pgfqpoint{0.853471in}{0.650933in}}%
\pgfpathlineto{\pgfqpoint{0.861203in}{0.649962in}}%
\pgfpathlineto{\pgfqpoint{0.863780in}{0.650863in}}%
\pgfpathlineto{\pgfqpoint{0.866357in}{0.658682in}}%
\pgfpathlineto{\pgfqpoint{0.868934in}{0.660441in}}%
\pgfpathlineto{\pgfqpoint{0.871512in}{0.661209in}}%
\pgfpathlineto{\pgfqpoint{0.889552in}{0.659387in}}%
\pgfpathlineto{\pgfqpoint{0.897284in}{0.661694in}}%
\pgfpathlineto{\pgfqpoint{0.899861in}{0.666279in}}%
\pgfpathlineto{\pgfqpoint{0.905015in}{0.667984in}}%
\pgfpathlineto{\pgfqpoint{0.915324in}{0.679311in}}%
\pgfpathlineto{\pgfqpoint{0.917901in}{0.685767in}}%
\pgfpathlineto{\pgfqpoint{0.920479in}{0.689581in}}%
\pgfpathlineto{\pgfqpoint{0.923056in}{0.692003in}}%
\pgfpathlineto{\pgfqpoint{0.925633in}{0.695852in}}%
\pgfpathlineto{\pgfqpoint{0.938519in}{0.696841in}}%
\pgfpathlineto{\pgfqpoint{0.943674in}{0.695946in}}%
\pgfpathlineto{\pgfqpoint{0.951405in}{0.695219in}}%
\pgfpathlineto{\pgfqpoint{0.961714in}{0.692096in}}%
\pgfpathlineto{\pgfqpoint{0.974600in}{0.690777in}}%
\pgfpathlineto{\pgfqpoint{0.979754in}{0.693258in}}%
\pgfpathlineto{\pgfqpoint{0.987486in}{0.694469in}}%
\pgfpathlineto{\pgfqpoint{0.992641in}{0.699748in}}%
\pgfpathlineto{\pgfqpoint{0.997795in}{0.700708in}}%
\pgfpathlineto{\pgfqpoint{1.015835in}{0.702981in}}%
\pgfpathlineto{\pgfqpoint{1.023567in}{0.704499in}}%
\pgfpathlineto{\pgfqpoint{1.028722in}{0.707548in}}%
\pgfpathlineto{\pgfqpoint{1.031299in}{0.708533in}}%
\pgfpathlineto{\pgfqpoint{1.033876in}{0.708876in}}%
\pgfpathlineto{\pgfqpoint{1.046762in}{0.709498in}}%
\pgfpathlineto{\pgfqpoint{1.051916in}{0.710242in}}%
\pgfpathlineto{\pgfqpoint{1.067380in}{0.710350in}}%
\pgfpathlineto{\pgfqpoint{1.098306in}{0.709607in}}%
\pgfpathlineto{\pgfqpoint{1.106038in}{0.708231in}}%
\pgfpathlineto{\pgfqpoint{1.149851in}{0.706863in}}%
\pgfpathlineto{\pgfqpoint{1.155005in}{0.707244in}}%
\pgfpathlineto{\pgfqpoint{1.160159in}{0.706895in}}%
\pgfpathlineto{\pgfqpoint{1.178200in}{0.706962in}}%
\pgfpathlineto{\pgfqpoint{1.206549in}{0.707929in}}%
\pgfpathlineto{\pgfqpoint{1.214281in}{0.710530in}}%
\pgfpathlineto{\pgfqpoint{1.224590in}{0.710941in}}%
\pgfpathlineto{\pgfqpoint{1.229744in}{0.712656in}}%
\pgfpathlineto{\pgfqpoint{1.232321in}{0.714141in}}%
\pgfpathlineto{\pgfqpoint{1.240053in}{0.715688in}}%
\pgfpathlineto{\pgfqpoint{1.242630in}{0.717372in}}%
\pgfpathlineto{\pgfqpoint{1.245207in}{0.718370in}}%
\pgfpathlineto{\pgfqpoint{1.263248in}{0.719781in}}%
\pgfpathlineto{\pgfqpoint{1.268402in}{0.719898in}}%
\pgfpathlineto{\pgfqpoint{1.278711in}{0.720784in}}%
\pgfpathlineto{\pgfqpoint{1.283866in}{0.725189in}}%
\pgfpathlineto{\pgfqpoint{1.286443in}{0.727435in}}%
\pgfpathlineto{\pgfqpoint{1.294174in}{0.730094in}}%
\pgfpathlineto{\pgfqpoint{1.296752in}{0.732628in}}%
\pgfpathlineto{\pgfqpoint{1.301906in}{0.735681in}}%
\pgfpathlineto{\pgfqpoint{1.304483in}{0.736290in}}%
\pgfpathlineto{\pgfqpoint{1.319947in}{0.737867in}}%
\pgfpathlineto{\pgfqpoint{1.322524in}{0.738356in}}%
\pgfpathlineto{\pgfqpoint{1.340564in}{0.738626in}}%
\pgfpathlineto{\pgfqpoint{1.366336in}{0.736989in}}%
\pgfpathlineto{\pgfqpoint{1.376645in}{0.735867in}}%
\pgfpathlineto{\pgfqpoint{1.389531in}{0.735099in}}%
\pgfpathlineto{\pgfqpoint{1.394686in}{0.734717in}}%
\pgfpathlineto{\pgfqpoint{1.407572in}{0.735268in}}%
\pgfpathlineto{\pgfqpoint{1.412726in}{0.735991in}}%
\pgfpathlineto{\pgfqpoint{1.425612in}{0.736796in}}%
\pgfpathlineto{\pgfqpoint{1.430767in}{0.737602in}}%
\pgfpathlineto{\pgfqpoint{1.446230in}{0.738654in}}%
\pgfpathlineto{\pgfqpoint{1.448807in}{0.738996in}}%
\pgfpathlineto{\pgfqpoint{1.464270in}{0.740152in}}%
\pgfpathlineto{\pgfqpoint{1.466848in}{0.740505in}}%
\pgfpathlineto{\pgfqpoint{1.479734in}{0.741305in}}%
\pgfpathlineto{\pgfqpoint{1.484888in}{0.743040in}}%
\pgfpathlineto{\pgfqpoint{1.495197in}{0.744663in}}%
\pgfpathlineto{\pgfqpoint{1.500351in}{0.746316in}}%
\pgfpathlineto{\pgfqpoint{1.502929in}{0.747456in}}%
\pgfpathlineto{\pgfqpoint{1.510660in}{0.748606in}}%
\pgfpathlineto{\pgfqpoint{1.520969in}{0.753074in}}%
\pgfpathlineto{\pgfqpoint{1.528701in}{0.754317in}}%
\pgfpathlineto{\pgfqpoint{1.536432in}{0.757221in}}%
\pgfpathlineto{\pgfqpoint{1.539010in}{0.758109in}}%
\pgfpathlineto{\pgfqpoint{1.549318in}{0.759584in}}%
\pgfpathlineto{\pgfqpoint{1.557050in}{0.761984in}}%
\pgfpathlineto{\pgfqpoint{1.585399in}{0.764509in}}%
\pgfpathlineto{\pgfqpoint{1.593131in}{0.766838in}}%
\pgfpathlineto{\pgfqpoint{1.606017in}{0.767683in}}%
\pgfpathlineto{\pgfqpoint{1.611172in}{0.769542in}}%
\pgfpathlineto{\pgfqpoint{1.626635in}{0.770857in}}%
\pgfpathlineto{\pgfqpoint{1.629212in}{0.771465in}}%
\pgfpathlineto{\pgfqpoint{1.654984in}{0.772303in}}%
\pgfpathlineto{\pgfqpoint{1.673025in}{0.771857in}}%
\pgfpathlineto{\pgfqpoint{1.693642in}{0.773940in}}%
\pgfpathlineto{\pgfqpoint{1.701374in}{0.775665in}}%
\pgfpathlineto{\pgfqpoint{1.714260in}{0.777103in}}%
\pgfpathlineto{\pgfqpoint{1.719414in}{0.778482in}}%
\pgfpathlineto{\pgfqpoint{1.727146in}{0.779183in}}%
\pgfpathlineto{\pgfqpoint{1.737455in}{0.782043in}}%
\pgfpathlineto{\pgfqpoint{1.750341in}{0.783379in}}%
\pgfpathlineto{\pgfqpoint{1.755495in}{0.783955in}}%
\pgfpathlineto{\pgfqpoint{1.809617in}{0.785051in}}%
\pgfpathlineto{\pgfqpoint{1.822503in}{0.786063in}}%
\pgfpathlineto{\pgfqpoint{1.827657in}{0.787021in}}%
\pgfpathlineto{\pgfqpoint{1.837966in}{0.787529in}}%
\pgfpathlineto{\pgfqpoint{1.843121in}{0.788895in}}%
\pgfpathlineto{\pgfqpoint{1.845698in}{0.790672in}}%
\pgfpathlineto{\pgfqpoint{1.853430in}{0.792663in}}%
\pgfpathlineto{\pgfqpoint{1.863738in}{0.803210in}}%
\pgfpathlineto{\pgfqpoint{1.871470in}{0.805618in}}%
\pgfpathlineto{\pgfqpoint{1.881779in}{0.815972in}}%
\pgfpathlineto{\pgfqpoint{1.889510in}{0.818306in}}%
\pgfpathlineto{\pgfqpoint{1.899819in}{0.828291in}}%
\pgfpathlineto{\pgfqpoint{1.910128in}{0.831073in}}%
\pgfpathlineto{\pgfqpoint{1.917860in}{0.838450in}}%
\pgfpathlineto{\pgfqpoint{1.925591in}{0.840277in}}%
\pgfpathlineto{\pgfqpoint{1.935900in}{0.847912in}}%
\pgfpathlineto{\pgfqpoint{1.943632in}{0.849836in}}%
\pgfpathlineto{\pgfqpoint{1.953941in}{0.857733in}}%
\pgfpathlineto{\pgfqpoint{1.961672in}{0.859703in}}%
\pgfpathlineto{\pgfqpoint{1.971981in}{0.866492in}}%
\pgfpathlineto{\pgfqpoint{1.979713in}{0.867821in}}%
\pgfpathlineto{\pgfqpoint{1.990022in}{0.874433in}}%
\pgfpathlineto{\pgfqpoint{1.997753in}{0.875787in}}%
\pgfpathlineto{\pgfqpoint{2.005485in}{0.880179in}}%
\pgfpathlineto{\pgfqpoint{2.015794in}{0.881763in}}%
\pgfpathlineto{\pgfqpoint{2.023526in}{0.887221in}}%
\pgfpathlineto{\pgfqpoint{2.026103in}{0.888852in}}%
\pgfpathlineto{\pgfqpoint{2.033834in}{0.890648in}}%
\pgfpathlineto{\pgfqpoint{2.038989in}{0.894106in}}%
\pgfpathlineto{\pgfqpoint{2.044143in}{0.898223in}}%
\pgfpathlineto{\pgfqpoint{2.051875in}{0.900133in}}%
\pgfpathlineto{\pgfqpoint{2.062184in}{0.908078in}}%
\pgfpathlineto{\pgfqpoint{2.069915in}{0.910428in}}%
\pgfpathlineto{\pgfqpoint{2.072493in}{0.913176in}}%
\pgfpathlineto{\pgfqpoint{2.080224in}{0.916068in}}%
\pgfpathlineto{\pgfqpoint{2.090533in}{0.918066in}}%
\pgfpathlineto{\pgfqpoint{2.098265in}{0.921276in}}%
\pgfpathlineto{\pgfqpoint{2.105996in}{0.922333in}}%
\pgfpathlineto{\pgfqpoint{2.116305in}{0.927018in}}%
\pgfpathlineto{\pgfqpoint{2.124037in}{0.928208in}}%
\pgfpathlineto{\pgfqpoint{2.134346in}{0.934495in}}%
\pgfpathlineto{\pgfqpoint{2.142077in}{0.935703in}}%
\pgfpathlineto{\pgfqpoint{2.149809in}{0.938950in}}%
\pgfpathlineto{\pgfqpoint{2.152386in}{0.940913in}}%
\pgfpathlineto{\pgfqpoint{2.162695in}{0.942517in}}%
\pgfpathlineto{\pgfqpoint{2.170427in}{0.945108in}}%
\pgfpathlineto{\pgfqpoint{2.180735in}{0.946409in}}%
\pgfpathlineto{\pgfqpoint{2.224548in}{0.954400in}}%
\pgfpathlineto{\pgfqpoint{2.234857in}{0.955075in}}%
\pgfpathlineto{\pgfqpoint{2.242589in}{0.956429in}}%
\pgfpathlineto{\pgfqpoint{2.252897in}{0.957632in}}%
\pgfpathlineto{\pgfqpoint{2.255475in}{0.958290in}}%
\pgfpathlineto{\pgfqpoint{2.270938in}{0.960409in}}%
\pgfpathlineto{\pgfqpoint{2.276092in}{0.962383in}}%
\pgfpathlineto{\pgfqpoint{2.278670in}{0.963652in}}%
\pgfpathlineto{\pgfqpoint{2.286401in}{0.964893in}}%
\pgfpathlineto{\pgfqpoint{2.296710in}{0.969304in}}%
\pgfpathlineto{\pgfqpoint{2.304442in}{0.970530in}}%
\pgfpathlineto{\pgfqpoint{2.314751in}{0.974577in}}%
\pgfpathlineto{\pgfqpoint{2.322482in}{0.975457in}}%
\pgfpathlineto{\pgfqpoint{2.332791in}{0.979585in}}%
\pgfpathlineto{\pgfqpoint{2.340523in}{0.980687in}}%
\pgfpathlineto{\pgfqpoint{2.350832in}{0.985376in}}%
\pgfpathlineto{\pgfqpoint{2.358563in}{0.986443in}}%
\pgfpathlineto{\pgfqpoint{2.366295in}{0.989239in}}%
\pgfpathlineto{\pgfqpoint{2.368872in}{0.989891in}}%
\pgfpathlineto{\pgfqpoint{2.379181in}{0.991037in}}%
\pgfpathlineto{\pgfqpoint{2.386912in}{0.992784in}}%
\pgfpathlineto{\pgfqpoint{2.438457in}{0.995570in}}%
\pgfpathlineto{\pgfqpoint{2.441034in}{0.995948in}}%
\pgfpathlineto{\pgfqpoint{2.451343in}{0.997003in}}%
\pgfpathlineto{\pgfqpoint{2.459074in}{0.998501in}}%
\pgfpathlineto{\pgfqpoint{2.487424in}{0.999381in}}%
\pgfpathlineto{\pgfqpoint{2.510619in}{0.999154in}}%
\pgfpathlineto{\pgfqpoint{2.544122in}{1.002373in}}%
\pgfpathlineto{\pgfqpoint{2.549277in}{1.003444in}}%
\pgfpathlineto{\pgfqpoint{2.557009in}{1.004140in}}%
\pgfpathlineto{\pgfqpoint{2.564740in}{1.006332in}}%
\pgfpathlineto{\pgfqpoint{2.567317in}{1.006965in}}%
\pgfpathlineto{\pgfqpoint{2.577626in}{1.008285in}}%
\pgfpathlineto{\pgfqpoint{2.585358in}{1.010834in}}%
\pgfpathlineto{\pgfqpoint{2.593089in}{1.011609in}}%
\pgfpathlineto{\pgfqpoint{2.600821in}{1.014608in}}%
\pgfpathlineto{\pgfqpoint{2.603398in}{1.015846in}}%
\pgfpathlineto{\pgfqpoint{2.611130in}{1.017012in}}%
\pgfpathlineto{\pgfqpoint{2.621439in}{1.021603in}}%
\pgfpathlineto{\pgfqpoint{2.629170in}{1.022869in}}%
\pgfpathlineto{\pgfqpoint{2.634325in}{1.024967in}}%
\pgfpathlineto{\pgfqpoint{2.639479in}{1.025948in}}%
\pgfpathlineto{\pgfqpoint{2.647211in}{1.026755in}}%
\pgfpathlineto{\pgfqpoint{2.657520in}{1.030078in}}%
\pgfpathlineto{\pgfqpoint{2.667829in}{1.031900in}}%
\pgfpathlineto{\pgfqpoint{2.672983in}{1.033313in}}%
\pgfpathlineto{\pgfqpoint{2.675560in}{1.033873in}}%
\pgfpathlineto{\pgfqpoint{2.685869in}{1.034656in}}%
\pgfpathlineto{\pgfqpoint{2.693601in}{1.036112in}}%
\pgfpathlineto{\pgfqpoint{2.709064in}{1.037262in}}%
\pgfpathlineto{\pgfqpoint{2.711641in}{1.037714in}}%
\pgfpathlineto{\pgfqpoint{2.783803in}{1.041486in}}%
\pgfpathlineto{\pgfqpoint{2.809575in}{1.041081in}}%
\pgfpathlineto{\pgfqpoint{2.819884in}{1.040537in}}%
\pgfpathlineto{\pgfqpoint{2.943590in}{1.040644in}}%
\pgfpathlineto{\pgfqpoint{2.992557in}{1.042783in}}%
\pgfpathlineto{\pgfqpoint{2.997712in}{1.043322in}}%
\pgfpathlineto{\pgfqpoint{3.015752in}{1.044224in}}%
\pgfpathlineto{\pgfqpoint{3.036370in}{1.046560in}}%
\pgfpathlineto{\pgfqpoint{3.049256in}{1.047444in}}%
\pgfpathlineto{\pgfqpoint{3.054411in}{1.048167in}}%
\pgfpathlineto{\pgfqpoint{3.144613in}{1.051024in}}%
\pgfpathlineto{\pgfqpoint{3.216775in}{1.051802in}}%
\pgfpathlineto{\pgfqpoint{3.304400in}{1.054044in}}%
\pgfpathlineto{\pgfqpoint{3.392025in}{1.061992in}}%
\pgfpathlineto{\pgfqpoint{3.397180in}{1.062837in}}%
\pgfpathlineto{\pgfqpoint{3.407489in}{1.063738in}}%
\pgfpathlineto{\pgfqpoint{3.415220in}{1.065090in}}%
\pgfpathlineto{\pgfqpoint{3.430684in}{1.066350in}}%
\pgfpathlineto{\pgfqpoint{3.433261in}{1.066677in}}%
\pgfpathlineto{\pgfqpoint{3.446147in}{1.067593in}}%
\pgfpathlineto{\pgfqpoint{3.461610in}{1.068852in}}%
\pgfpathlineto{\pgfqpoint{3.495114in}{1.071992in}}%
\pgfpathlineto{\pgfqpoint{3.505423in}{1.074766in}}%
\pgfpathlineto{\pgfqpoint{3.513154in}{1.075521in}}%
\pgfpathlineto{\pgfqpoint{3.523463in}{1.079330in}}%
\pgfpathlineto{\pgfqpoint{3.531195in}{1.080339in}}%
\pgfpathlineto{\pgfqpoint{3.541504in}{1.084018in}}%
\pgfpathlineto{\pgfqpoint{3.551813in}{1.085517in}}%
\pgfpathlineto{\pgfqpoint{3.559544in}{1.088000in}}%
\pgfpathlineto{\pgfqpoint{3.567276in}{1.088749in}}%
\pgfpathlineto{\pgfqpoint{3.572430in}{1.090423in}}%
\pgfpathlineto{\pgfqpoint{3.577585in}{1.091474in}}%
\pgfpathlineto{\pgfqpoint{3.585316in}{1.092465in}}%
\pgfpathlineto{\pgfqpoint{3.595625in}{1.096592in}}%
\pgfpathlineto{\pgfqpoint{3.603357in}{1.097638in}}%
\pgfpathlineto{\pgfqpoint{3.613666in}{1.101403in}}%
\pgfpathlineto{\pgfqpoint{3.623974in}{1.103000in}}%
\pgfpathlineto{\pgfqpoint{3.631706in}{1.106302in}}%
\pgfpathlineto{\pgfqpoint{3.639438in}{1.107534in}}%
\pgfpathlineto{\pgfqpoint{3.644592in}{1.110157in}}%
\pgfpathlineto{\pgfqpoint{3.649747in}{1.111490in}}%
\pgfpathlineto{\pgfqpoint{3.657478in}{1.112686in}}%
\pgfpathlineto{\pgfqpoint{3.662633in}{1.114790in}}%
\pgfpathlineto{\pgfqpoint{3.680673in}{1.117973in}}%
\pgfpathlineto{\pgfqpoint{3.685828in}{1.119694in}}%
\pgfpathlineto{\pgfqpoint{3.696136in}{1.121241in}}%
\pgfpathlineto{\pgfqpoint{3.703868in}{1.123618in}}%
\pgfpathlineto{\pgfqpoint{3.714177in}{1.124502in}}%
\pgfpathlineto{\pgfqpoint{3.721909in}{1.127220in}}%
\pgfpathlineto{\pgfqpoint{3.776030in}{1.131952in}}%
\pgfpathlineto{\pgfqpoint{3.794070in}{1.132949in}}%
\pgfpathlineto{\pgfqpoint{3.809534in}{1.133880in}}%
\pgfpathlineto{\pgfqpoint{3.827574in}{1.134814in}}%
\pgfpathlineto{\pgfqpoint{3.969321in}{1.134957in}}%
\pgfpathlineto{\pgfqpoint{3.992516in}{1.134174in}}%
\pgfpathlineto{\pgfqpoint{4.038906in}{1.132943in}}%
\pgfpathlineto{\pgfqpoint{4.093027in}{1.130250in}}%
\pgfpathlineto{\pgfqpoint{4.134263in}{1.128581in}}%
\pgfpathlineto{\pgfqpoint{4.206424in}{1.127276in}}%
\pgfpathlineto{\pgfqpoint{4.227042in}{1.125868in}}%
\pgfpathlineto{\pgfqpoint{4.242505in}{1.124950in}}%
\pgfpathlineto{\pgfqpoint{4.263123in}{1.123442in}}%
\pgfpathlineto{\pgfqpoint{4.278586in}{1.122511in}}%
\pgfpathlineto{\pgfqpoint{4.291472in}{1.121797in}}%
\pgfpathlineto{\pgfqpoint{4.299204in}{1.121121in}}%
\pgfpathlineto{\pgfqpoint{4.324976in}{1.120100in}}%
\pgfpathlineto{\pgfqpoint{4.353326in}{1.117959in}}%
\pgfpathlineto{\pgfqpoint{4.368789in}{1.116955in}}%
\pgfpathlineto{\pgfqpoint{4.389407in}{1.115474in}}%
\pgfpathlineto{\pgfqpoint{4.404870in}{1.114507in}}%
\pgfpathlineto{\pgfqpoint{4.425488in}{1.113174in}}%
\pgfpathlineto{\pgfqpoint{4.451260in}{1.112136in}}%
\pgfpathlineto{\pgfqpoint{4.510536in}{1.108772in}}%
\pgfpathlineto{\pgfqpoint{4.585275in}{1.106314in}}%
\pgfpathlineto{\pgfqpoint{4.621356in}{1.105567in}}%
\pgfpathlineto{\pgfqpoint{4.641973in}{1.104443in}}%
\pgfpathlineto{\pgfqpoint{4.706404in}{1.103931in}}%
\pgfpathlineto{\pgfqpoint{4.750216in}{1.104639in}}%
\pgfpathlineto{\pgfqpoint{4.768257in}{1.105062in}}%
\pgfpathlineto{\pgfqpoint{4.837842in}{1.106830in}}%
\pgfpathlineto{\pgfqpoint{4.858459in}{1.107672in}}%
\pgfpathlineto{\pgfqpoint{5.031132in}{1.110366in}}%
\pgfpathlineto{\pgfqpoint{5.108449in}{1.114368in}}%
\pgfpathlineto{\pgfqpoint{5.111026in}{1.114671in}}%
\pgfpathlineto{\pgfqpoint{5.123912in}{1.115572in}}%
\pgfpathlineto{\pgfqpoint{5.139375in}{1.116798in}}%
\pgfpathlineto{\pgfqpoint{5.196074in}{1.122200in}}%
\pgfpathlineto{\pgfqpoint{5.201228in}{1.123034in}}%
\pgfpathlineto{\pgfqpoint{5.211537in}{1.123820in}}%
\pgfpathlineto{\pgfqpoint{5.219269in}{1.125096in}}%
\pgfpathlineto{\pgfqpoint{5.232155in}{1.126384in}}%
\pgfpathlineto{\pgfqpoint{5.237309in}{1.127319in}}%
\pgfpathlineto{\pgfqpoint{5.250196in}{1.128252in}}%
\pgfpathlineto{\pgfqpoint{5.255350in}{1.128973in}}%
\pgfpathlineto{\pgfqpoint{5.268236in}{1.130137in}}%
\pgfpathlineto{\pgfqpoint{5.273390in}{1.131000in}}%
\pgfpathlineto{\pgfqpoint{5.283699in}{1.131923in}}%
\pgfpathlineto{\pgfqpoint{5.291431in}{1.133221in}}%
\pgfpathlineto{\pgfqpoint{5.301740in}{1.134061in}}%
\pgfpathlineto{\pgfqpoint{5.309471in}{1.135565in}}%
\pgfpathlineto{\pgfqpoint{5.319780in}{1.136447in}}%
\pgfpathlineto{\pgfqpoint{5.327512in}{1.137964in}}%
\pgfpathlineto{\pgfqpoint{5.340398in}{1.139303in}}%
\pgfpathlineto{\pgfqpoint{5.345552in}{1.140122in}}%
\pgfpathlineto{\pgfqpoint{5.373902in}{1.141950in}}%
\pgfpathlineto{\pgfqpoint{5.381633in}{1.142969in}}%
\pgfpathlineto{\pgfqpoint{5.394519in}{1.143940in}}%
\pgfpathlineto{\pgfqpoint{5.399674in}{1.144449in}}%
\pgfpathlineto{\pgfqpoint{5.453795in}{1.146195in}}%
\pgfpathlineto{\pgfqpoint{5.497608in}{1.147111in}}%
\pgfpathlineto{\pgfqpoint{5.515648in}{1.147899in}}%
\pgfpathlineto{\pgfqpoint{5.543998in}{1.148978in}}%
\pgfpathlineto{\pgfqpoint{5.574924in}{1.149856in}}%
\pgfpathlineto{\pgfqpoint{5.605851in}{1.151271in}}%
\pgfpathlineto{\pgfqpoint{5.616160in}{1.152581in}}%
\pgfpathlineto{\pgfqpoint{5.629046in}{1.153576in}}%
\pgfpathlineto{\pgfqpoint{5.634200in}{1.154282in}}%
\pgfpathlineto{\pgfqpoint{5.644509in}{1.154998in}}%
\pgfpathlineto{\pgfqpoint{5.652241in}{1.156196in}}%
\pgfpathlineto{\pgfqpoint{5.662550in}{1.156950in}}%
\pgfpathlineto{\pgfqpoint{5.670281in}{1.158759in}}%
\pgfpathlineto{\pgfqpoint{5.683167in}{1.160046in}}%
\pgfpathlineto{\pgfqpoint{5.688322in}{1.161242in}}%
\pgfpathlineto{\pgfqpoint{5.698630in}{1.162408in}}%
\pgfpathlineto{\pgfqpoint{5.706362in}{1.164149in}}%
\pgfpathlineto{\pgfqpoint{5.716671in}{1.165181in}}%
\pgfpathlineto{\pgfqpoint{5.724403in}{1.166758in}}%
\pgfpathlineto{\pgfqpoint{5.734711in}{1.167900in}}%
\pgfpathlineto{\pgfqpoint{5.742443in}{1.169624in}}%
\pgfpathlineto{\pgfqpoint{5.752752in}{1.170747in}}%
\pgfpathlineto{\pgfqpoint{5.760484in}{1.172310in}}%
\pgfpathlineto{\pgfqpoint{5.770792in}{1.173317in}}%
\pgfpathlineto{\pgfqpoint{5.778524in}{1.174715in}}%
\pgfpathlineto{\pgfqpoint{5.788833in}{1.175573in}}%
\pgfpathlineto{\pgfqpoint{5.796565in}{1.176787in}}%
\pgfpathlineto{\pgfqpoint{5.806873in}{1.177604in}}%
\pgfpathlineto{\pgfqpoint{5.812028in}{1.178491in}}%
\pgfpathlineto{\pgfqpoint{5.824914in}{1.179433in}}%
\pgfpathlineto{\pgfqpoint{5.832646in}{1.180671in}}%
\pgfpathlineto{\pgfqpoint{5.845532in}{1.181877in}}%
\pgfpathlineto{\pgfqpoint{5.850686in}{1.182453in}}%
\pgfpathlineto{\pgfqpoint{5.876458in}{1.183786in}}%
\pgfpathlineto{\pgfqpoint{5.904807in}{1.185560in}}%
\pgfpathlineto{\pgfqpoint{5.920271in}{1.186364in}}%
\pgfpathlineto{\pgfqpoint{5.922848in}{1.186612in}}%
\pgfpathlineto{\pgfqpoint{5.938311in}{1.187467in}}%
\pgfpathlineto{\pgfqpoint{5.940888in}{1.187796in}}%
\pgfpathlineto{\pgfqpoint{5.953775in}{1.188811in}}%
\pgfpathlineto{\pgfqpoint{5.976969in}{1.191027in}}%
\pgfpathlineto{\pgfqpoint{5.989855in}{1.192171in}}%
\pgfpathlineto{\pgfqpoint{5.995010in}{1.192871in}}%
\pgfpathlineto{\pgfqpoint{6.028514in}{1.194952in}}%
\pgfpathlineto{\pgfqpoint{6.038823in}{1.195394in}}%
\pgfpathlineto{\pgfqpoint{6.080058in}{1.198683in}}%
\pgfpathlineto{\pgfqpoint{6.085212in}{1.199582in}}%
\pgfpathlineto{\pgfqpoint{6.095521in}{1.200549in}}%
\pgfpathlineto{\pgfqpoint{6.103253in}{1.201959in}}%
\pgfpathlineto{\pgfqpoint{6.113562in}{1.202987in}}%
\pgfpathlineto{\pgfqpoint{6.121293in}{1.204548in}}%
\pgfpathlineto{\pgfqpoint{6.131602in}{1.205617in}}%
\pgfpathlineto{\pgfqpoint{6.139334in}{1.207296in}}%
\pgfpathlineto{\pgfqpoint{6.149643in}{1.208480in}}%
\pgfpathlineto{\pgfqpoint{6.157374in}{1.210140in}}%
\pgfpathlineto{\pgfqpoint{6.167683in}{1.211233in}}%
\pgfpathlineto{\pgfqpoint{6.175415in}{1.212817in}}%
\pgfpathlineto{\pgfqpoint{6.188301in}{1.213936in}}%
\pgfpathlineto{\pgfqpoint{6.193455in}{1.215074in}}%
\pgfpathlineto{\pgfqpoint{6.203764in}{1.216335in}}%
\pgfpathlineto{\pgfqpoint{6.211496in}{1.218137in}}%
\pgfpathlineto{\pgfqpoint{6.221805in}{1.219358in}}%
\pgfpathlineto{\pgfqpoint{6.229536in}{1.221043in}}%
\pgfpathlineto{\pgfqpoint{6.239845in}{1.222090in}}%
\pgfpathlineto{\pgfqpoint{6.247577in}{1.223271in}}%
\pgfpathlineto{\pgfqpoint{6.257886in}{1.224192in}}%
\pgfpathlineto{\pgfqpoint{6.265617in}{1.225625in}}%
\pgfpathlineto{\pgfqpoint{6.278503in}{1.226930in}}%
\pgfpathlineto{\pgfqpoint{6.283658in}{1.227913in}}%
\pgfpathlineto{\pgfqpoint{6.293967in}{1.228941in}}%
\pgfpathlineto{\pgfqpoint{6.301698in}{1.230140in}}%
\pgfpathlineto{\pgfqpoint{6.330048in}{1.231306in}}%
\pgfpathlineto{\pgfqpoint{6.422827in}{1.237576in}}%
\pgfpathlineto{\pgfqpoint{6.427982in}{1.238228in}}%
\pgfpathlineto{\pgfqpoint{6.440868in}{1.239214in}}%
\pgfpathlineto{\pgfqpoint{6.446022in}{1.240007in}}%
\pgfpathlineto{\pgfqpoint{6.458908in}{1.241232in}}%
\pgfpathlineto{\pgfqpoint{6.464063in}{1.242068in}}%
\pgfpathlineto{\pgfqpoint{6.476949in}{1.242949in}}%
\pgfpathlineto{\pgfqpoint{6.482103in}{1.243782in}}%
\pgfpathlineto{\pgfqpoint{6.482103in}{1.243782in}}%
\pgfusepath{stroke}%
\end{pgfscope}%
\begin{pgfscope}%
\pgfpathrectangle{\pgfqpoint{0.563921in}{0.521603in}}{\pgfqpoint{6.200000in}{2.642500in}}%
\pgfusepath{clip}%
\pgfsetroundcap%
\pgfsetroundjoin%
\pgfsetlinewidth{1.505625pt}%
\definecolor{currentstroke}{rgb}{0.890196,0.466667,0.760784}%
\pgfsetstrokecolor{currentstroke}%
\pgfsetdash{}{0pt}%
\pgfpathmoveto{\pgfqpoint{0.845739in}{0.641717in}}%
\pgfpathlineto{\pgfqpoint{0.848317in}{0.651251in}}%
\pgfpathlineto{\pgfqpoint{0.850894in}{0.656324in}}%
\pgfpathlineto{\pgfqpoint{0.853471in}{0.662793in}}%
\pgfpathlineto{\pgfqpoint{0.861203in}{0.662053in}}%
\pgfpathlineto{\pgfqpoint{0.866357in}{0.689930in}}%
\pgfpathlineto{\pgfqpoint{0.868934in}{0.700406in}}%
\pgfpathlineto{\pgfqpoint{0.871512in}{0.698570in}}%
\pgfpathlineto{\pgfqpoint{0.881820in}{0.701328in}}%
\pgfpathlineto{\pgfqpoint{0.884398in}{0.705959in}}%
\pgfpathlineto{\pgfqpoint{0.886975in}{0.706574in}}%
\pgfpathlineto{\pgfqpoint{0.889552in}{0.705084in}}%
\pgfpathlineto{\pgfqpoint{0.897284in}{0.704029in}}%
\pgfpathlineto{\pgfqpoint{0.899861in}{0.706023in}}%
\pgfpathlineto{\pgfqpoint{0.902438in}{0.706786in}}%
\pgfpathlineto{\pgfqpoint{0.907593in}{0.706564in}}%
\pgfpathlineto{\pgfqpoint{0.915324in}{0.706432in}}%
\pgfpathlineto{\pgfqpoint{0.917901in}{0.708281in}}%
\pgfpathlineto{\pgfqpoint{0.920479in}{0.717828in}}%
\pgfpathlineto{\pgfqpoint{0.923056in}{0.723834in}}%
\pgfpathlineto{\pgfqpoint{0.925633in}{0.732820in}}%
\pgfpathlineto{\pgfqpoint{0.933365in}{0.737724in}}%
\pgfpathlineto{\pgfqpoint{0.935942in}{0.740638in}}%
\pgfpathlineto{\pgfqpoint{0.938519in}{0.747744in}}%
\pgfpathlineto{\pgfqpoint{0.941096in}{0.761658in}}%
\pgfpathlineto{\pgfqpoint{0.943674in}{0.771253in}}%
\pgfpathlineto{\pgfqpoint{0.951405in}{0.784017in}}%
\pgfpathlineto{\pgfqpoint{0.953982in}{0.793284in}}%
\pgfpathlineto{\pgfqpoint{0.959137in}{0.800632in}}%
\pgfpathlineto{\pgfqpoint{0.961714in}{0.805359in}}%
\pgfpathlineto{\pgfqpoint{0.972023in}{0.809715in}}%
\pgfpathlineto{\pgfqpoint{0.974600in}{0.813361in}}%
\pgfpathlineto{\pgfqpoint{0.979754in}{0.818161in}}%
\pgfpathlineto{\pgfqpoint{0.990063in}{0.820896in}}%
\pgfpathlineto{\pgfqpoint{0.995218in}{0.824039in}}%
\pgfpathlineto{\pgfqpoint{0.997795in}{0.826024in}}%
\pgfpathlineto{\pgfqpoint{1.005527in}{0.826036in}}%
\pgfpathlineto{\pgfqpoint{1.008104in}{0.824393in}}%
\pgfpathlineto{\pgfqpoint{1.010681in}{0.823601in}}%
\pgfpathlineto{\pgfqpoint{1.023567in}{0.824220in}}%
\pgfpathlineto{\pgfqpoint{1.031299in}{0.835928in}}%
\pgfpathlineto{\pgfqpoint{1.033876in}{0.837224in}}%
\pgfpathlineto{\pgfqpoint{1.041608in}{0.837409in}}%
\pgfpathlineto{\pgfqpoint{1.046762in}{0.835169in}}%
\pgfpathlineto{\pgfqpoint{1.051916in}{0.832110in}}%
\pgfpathlineto{\pgfqpoint{1.059648in}{0.831345in}}%
\pgfpathlineto{\pgfqpoint{1.064802in}{0.828831in}}%
\pgfpathlineto{\pgfqpoint{1.069957in}{0.826515in}}%
\pgfpathlineto{\pgfqpoint{1.077689in}{0.825416in}}%
\pgfpathlineto{\pgfqpoint{1.085420in}{0.821520in}}%
\pgfpathlineto{\pgfqpoint{1.098306in}{0.819531in}}%
\pgfpathlineto{\pgfqpoint{1.106038in}{0.816021in}}%
\pgfpathlineto{\pgfqpoint{1.113770in}{0.814905in}}%
\pgfpathlineto{\pgfqpoint{1.124078in}{0.810335in}}%
\pgfpathlineto{\pgfqpoint{1.131810in}{0.809356in}}%
\pgfpathlineto{\pgfqpoint{1.142119in}{0.805502in}}%
\pgfpathlineto{\pgfqpoint{1.149851in}{0.804560in}}%
\pgfpathlineto{\pgfqpoint{1.160159in}{0.800976in}}%
\pgfpathlineto{\pgfqpoint{1.173045in}{0.800244in}}%
\pgfpathlineto{\pgfqpoint{1.178200in}{0.800628in}}%
\pgfpathlineto{\pgfqpoint{1.188509in}{0.802479in}}%
\pgfpathlineto{\pgfqpoint{1.191086in}{0.804031in}}%
\pgfpathlineto{\pgfqpoint{1.193663in}{0.806827in}}%
\pgfpathlineto{\pgfqpoint{1.196240in}{0.810494in}}%
\pgfpathlineto{\pgfqpoint{1.203972in}{0.812549in}}%
\pgfpathlineto{\pgfqpoint{1.214281in}{0.819915in}}%
\pgfpathlineto{\pgfqpoint{1.224590in}{0.820430in}}%
\pgfpathlineto{\pgfqpoint{1.229744in}{0.822916in}}%
\pgfpathlineto{\pgfqpoint{1.232321in}{0.825668in}}%
\pgfpathlineto{\pgfqpoint{1.240053in}{0.828814in}}%
\pgfpathlineto{\pgfqpoint{1.242630in}{0.832347in}}%
\pgfpathlineto{\pgfqpoint{1.245207in}{0.833351in}}%
\pgfpathlineto{\pgfqpoint{1.258093in}{0.833539in}}%
\pgfpathlineto{\pgfqpoint{1.268402in}{0.835591in}}%
\pgfpathlineto{\pgfqpoint{1.278711in}{0.835093in}}%
\pgfpathlineto{\pgfqpoint{1.283866in}{0.834793in}}%
\pgfpathlineto{\pgfqpoint{1.286443in}{0.834676in}}%
\pgfpathlineto{\pgfqpoint{1.299329in}{0.836063in}}%
\pgfpathlineto{\pgfqpoint{1.301906in}{0.837172in}}%
\pgfpathlineto{\pgfqpoint{1.304483in}{0.836863in}}%
\pgfpathlineto{\pgfqpoint{1.332833in}{0.836505in}}%
\pgfpathlineto{\pgfqpoint{1.340564in}{0.839063in}}%
\pgfpathlineto{\pgfqpoint{1.353450in}{0.839588in}}%
\pgfpathlineto{\pgfqpoint{1.358605in}{0.839104in}}%
\pgfpathlineto{\pgfqpoint{1.366336in}{0.839461in}}%
\pgfpathlineto{\pgfqpoint{1.371491in}{0.841072in}}%
\pgfpathlineto{\pgfqpoint{1.376645in}{0.841404in}}%
\pgfpathlineto{\pgfqpoint{1.392108in}{0.840488in}}%
\pgfpathlineto{\pgfqpoint{1.394686in}{0.839863in}}%
\pgfpathlineto{\pgfqpoint{1.404995in}{0.838609in}}%
\pgfpathlineto{\pgfqpoint{1.412726in}{0.836712in}}%
\pgfpathlineto{\pgfqpoint{1.423035in}{0.835463in}}%
\pgfpathlineto{\pgfqpoint{1.430767in}{0.834146in}}%
\pgfpathlineto{\pgfqpoint{1.441076in}{0.833429in}}%
\pgfpathlineto{\pgfqpoint{1.448807in}{0.832339in}}%
\pgfpathlineto{\pgfqpoint{1.461693in}{0.831773in}}%
\pgfpathlineto{\pgfqpoint{1.466848in}{0.830988in}}%
\pgfpathlineto{\pgfqpoint{1.479734in}{0.829923in}}%
\pgfpathlineto{\pgfqpoint{1.484888in}{0.829132in}}%
\pgfpathlineto{\pgfqpoint{1.495197in}{0.828172in}}%
\pgfpathlineto{\pgfqpoint{1.500351in}{0.827545in}}%
\pgfpathlineto{\pgfqpoint{1.502929in}{0.827971in}}%
\pgfpathlineto{\pgfqpoint{1.520969in}{0.828225in}}%
\pgfpathlineto{\pgfqpoint{1.531278in}{0.827471in}}%
\pgfpathlineto{\pgfqpoint{1.539010in}{0.826041in}}%
\pgfpathlineto{\pgfqpoint{1.549318in}{0.825103in}}%
\pgfpathlineto{\pgfqpoint{1.557050in}{0.823750in}}%
\pgfpathlineto{\pgfqpoint{1.569936in}{0.822476in}}%
\pgfpathlineto{\pgfqpoint{1.575091in}{0.821767in}}%
\pgfpathlineto{\pgfqpoint{1.585399in}{0.820950in}}%
\pgfpathlineto{\pgfqpoint{1.593131in}{0.819738in}}%
\pgfpathlineto{\pgfqpoint{1.603440in}{0.818882in}}%
\pgfpathlineto{\pgfqpoint{1.611172in}{0.817616in}}%
\pgfpathlineto{\pgfqpoint{1.626635in}{0.816851in}}%
\pgfpathlineto{\pgfqpoint{1.629212in}{0.816440in}}%
\pgfpathlineto{\pgfqpoint{1.644675in}{0.815071in}}%
\pgfpathlineto{\pgfqpoint{1.662716in}{0.813640in}}%
\pgfpathlineto{\pgfqpoint{1.683333in}{0.811933in}}%
\pgfpathlineto{\pgfqpoint{1.719414in}{0.810946in}}%
\pgfpathlineto{\pgfqpoint{1.747764in}{0.811425in}}%
\pgfpathlineto{\pgfqpoint{1.755495in}{0.813294in}}%
\pgfpathlineto{\pgfqpoint{1.781268in}{0.814574in}}%
\pgfpathlineto{\pgfqpoint{1.786422in}{0.815501in}}%
\pgfpathlineto{\pgfqpoint{1.791576in}{0.817816in}}%
\pgfpathlineto{\pgfqpoint{1.801885in}{0.819570in}}%
\pgfpathlineto{\pgfqpoint{1.809617in}{0.823000in}}%
\pgfpathlineto{\pgfqpoint{1.817349in}{0.824466in}}%
\pgfpathlineto{\pgfqpoint{1.827657in}{0.830730in}}%
\pgfpathlineto{\pgfqpoint{1.837966in}{0.832777in}}%
\pgfpathlineto{\pgfqpoint{1.843121in}{0.837752in}}%
\pgfpathlineto{\pgfqpoint{1.845698in}{0.840839in}}%
\pgfpathlineto{\pgfqpoint{1.853430in}{0.843839in}}%
\pgfpathlineto{\pgfqpoint{1.858584in}{0.849184in}}%
\pgfpathlineto{\pgfqpoint{1.861161in}{0.850853in}}%
\pgfpathlineto{\pgfqpoint{1.863738in}{0.853572in}}%
\pgfpathlineto{\pgfqpoint{1.871470in}{0.856060in}}%
\pgfpathlineto{\pgfqpoint{1.881779in}{0.865618in}}%
\pgfpathlineto{\pgfqpoint{1.889510in}{0.867826in}}%
\pgfpathlineto{\pgfqpoint{1.899819in}{0.877551in}}%
\pgfpathlineto{\pgfqpoint{1.910128in}{0.880386in}}%
\pgfpathlineto{\pgfqpoint{1.917860in}{0.887089in}}%
\pgfpathlineto{\pgfqpoint{1.925591in}{0.888545in}}%
\pgfpathlineto{\pgfqpoint{1.930746in}{0.892563in}}%
\pgfpathlineto{\pgfqpoint{1.935900in}{0.896763in}}%
\pgfpathlineto{\pgfqpoint{1.943632in}{0.898310in}}%
\pgfpathlineto{\pgfqpoint{1.953941in}{0.907870in}}%
\pgfpathlineto{\pgfqpoint{1.961672in}{0.910763in}}%
\pgfpathlineto{\pgfqpoint{1.971981in}{0.922217in}}%
\pgfpathlineto{\pgfqpoint{1.979713in}{0.924705in}}%
\pgfpathlineto{\pgfqpoint{1.990022in}{0.934733in}}%
\pgfpathlineto{\pgfqpoint{1.997753in}{0.936959in}}%
\pgfpathlineto{\pgfqpoint{2.005485in}{0.943475in}}%
\pgfpathlineto{\pgfqpoint{2.015794in}{0.945560in}}%
\pgfpathlineto{\pgfqpoint{2.026103in}{0.953591in}}%
\pgfpathlineto{\pgfqpoint{2.033834in}{0.955760in}}%
\pgfpathlineto{\pgfqpoint{2.041566in}{0.963649in}}%
\pgfpathlineto{\pgfqpoint{2.044143in}{0.966377in}}%
\pgfpathlineto{\pgfqpoint{2.051875in}{0.968237in}}%
\pgfpathlineto{\pgfqpoint{2.057029in}{0.971975in}}%
\pgfpathlineto{\pgfqpoint{2.062184in}{0.974745in}}%
\pgfpathlineto{\pgfqpoint{2.069915in}{0.976476in}}%
\pgfpathlineto{\pgfqpoint{2.077647in}{0.980297in}}%
\pgfpathlineto{\pgfqpoint{2.080224in}{0.981199in}}%
\pgfpathlineto{\pgfqpoint{2.087956in}{0.982209in}}%
\pgfpathlineto{\pgfqpoint{2.098265in}{0.986287in}}%
\pgfpathlineto{\pgfqpoint{2.105996in}{0.987726in}}%
\pgfpathlineto{\pgfqpoint{2.113728in}{0.992655in}}%
\pgfpathlineto{\pgfqpoint{2.116305in}{0.994522in}}%
\pgfpathlineto{\pgfqpoint{2.124037in}{0.996260in}}%
\pgfpathlineto{\pgfqpoint{2.134346in}{1.004856in}}%
\pgfpathlineto{\pgfqpoint{2.142077in}{1.007345in}}%
\pgfpathlineto{\pgfqpoint{2.147232in}{1.011955in}}%
\pgfpathlineto{\pgfqpoint{2.152386in}{1.015351in}}%
\pgfpathlineto{\pgfqpoint{2.162695in}{1.017220in}}%
\pgfpathlineto{\pgfqpoint{2.170427in}{1.022063in}}%
\pgfpathlineto{\pgfqpoint{2.178158in}{1.023593in}}%
\pgfpathlineto{\pgfqpoint{2.183313in}{1.025741in}}%
\pgfpathlineto{\pgfqpoint{2.188467in}{1.027816in}}%
\pgfpathlineto{\pgfqpoint{2.196199in}{1.028949in}}%
\pgfpathlineto{\pgfqpoint{2.206508in}{1.032855in}}%
\pgfpathlineto{\pgfqpoint{2.214239in}{1.034112in}}%
\pgfpathlineto{\pgfqpoint{2.219394in}{1.036680in}}%
\pgfpathlineto{\pgfqpoint{2.224548in}{1.037826in}}%
\pgfpathlineto{\pgfqpoint{2.234857in}{1.038689in}}%
\pgfpathlineto{\pgfqpoint{2.242589in}{1.040818in}}%
\pgfpathlineto{\pgfqpoint{2.252897in}{1.042671in}}%
\pgfpathlineto{\pgfqpoint{2.260629in}{1.045117in}}%
\pgfpathlineto{\pgfqpoint{2.268361in}{1.046651in}}%
\pgfpathlineto{\pgfqpoint{2.276092in}{1.052371in}}%
\pgfpathlineto{\pgfqpoint{2.278670in}{1.054602in}}%
\pgfpathlineto{\pgfqpoint{2.286401in}{1.056922in}}%
\pgfpathlineto{\pgfqpoint{2.294133in}{1.064099in}}%
\pgfpathlineto{\pgfqpoint{2.296710in}{1.066963in}}%
\pgfpathlineto{\pgfqpoint{2.304442in}{1.069656in}}%
\pgfpathlineto{\pgfqpoint{2.314751in}{1.083739in}}%
\pgfpathlineto{\pgfqpoint{2.322482in}{1.087132in}}%
\pgfpathlineto{\pgfqpoint{2.330214in}{1.098017in}}%
\pgfpathlineto{\pgfqpoint{2.332791in}{1.102101in}}%
\pgfpathlineto{\pgfqpoint{2.340523in}{1.105706in}}%
\pgfpathlineto{\pgfqpoint{2.350832in}{1.118353in}}%
\pgfpathlineto{\pgfqpoint{2.358563in}{1.121306in}}%
\pgfpathlineto{\pgfqpoint{2.363718in}{1.127469in}}%
\pgfpathlineto{\pgfqpoint{2.368872in}{1.131890in}}%
\pgfpathlineto{\pgfqpoint{2.376604in}{1.133941in}}%
\pgfpathlineto{\pgfqpoint{2.386912in}{1.141666in}}%
\pgfpathlineto{\pgfqpoint{2.394644in}{1.143485in}}%
\pgfpathlineto{\pgfqpoint{2.404953in}{1.148491in}}%
\pgfpathlineto{\pgfqpoint{2.415262in}{1.150283in}}%
\pgfpathlineto{\pgfqpoint{2.422993in}{1.156067in}}%
\pgfpathlineto{\pgfqpoint{2.430725in}{1.158207in}}%
\pgfpathlineto{\pgfqpoint{2.438457in}{1.166745in}}%
\pgfpathlineto{\pgfqpoint{2.441034in}{1.169813in}}%
\pgfpathlineto{\pgfqpoint{2.448766in}{1.173229in}}%
\pgfpathlineto{\pgfqpoint{2.459074in}{1.187621in}}%
\pgfpathlineto{\pgfqpoint{2.466806in}{1.190674in}}%
\pgfpathlineto{\pgfqpoint{2.477115in}{1.202591in}}%
\pgfpathlineto{\pgfqpoint{2.484847in}{1.204988in}}%
\pgfpathlineto{\pgfqpoint{2.490001in}{1.208893in}}%
\pgfpathlineto{\pgfqpoint{2.495155in}{1.211703in}}%
\pgfpathlineto{\pgfqpoint{2.502887in}{1.213091in}}%
\pgfpathlineto{\pgfqpoint{2.508041in}{1.215320in}}%
\pgfpathlineto{\pgfqpoint{2.513196in}{1.219026in}}%
\pgfpathlineto{\pgfqpoint{2.520928in}{1.220939in}}%
\pgfpathlineto{\pgfqpoint{2.528659in}{1.226436in}}%
\pgfpathlineto{\pgfqpoint{2.531236in}{1.228439in}}%
\pgfpathlineto{\pgfqpoint{2.538968in}{1.230388in}}%
\pgfpathlineto{\pgfqpoint{2.549277in}{1.237191in}}%
\pgfpathlineto{\pgfqpoint{2.557009in}{1.238670in}}%
\pgfpathlineto{\pgfqpoint{2.567317in}{1.244843in}}%
\pgfpathlineto{\pgfqpoint{2.575049in}{1.246638in}}%
\pgfpathlineto{\pgfqpoint{2.585358in}{1.253652in}}%
\pgfpathlineto{\pgfqpoint{2.593089in}{1.255356in}}%
\pgfpathlineto{\pgfqpoint{2.600821in}{1.260459in}}%
\pgfpathlineto{\pgfqpoint{2.603398in}{1.262309in}}%
\pgfpathlineto{\pgfqpoint{2.611130in}{1.264253in}}%
\pgfpathlineto{\pgfqpoint{2.621439in}{1.272083in}}%
\pgfpathlineto{\pgfqpoint{2.629170in}{1.274161in}}%
\pgfpathlineto{\pgfqpoint{2.634325in}{1.278644in}}%
\pgfpathlineto{\pgfqpoint{2.639479in}{1.280759in}}%
\pgfpathlineto{\pgfqpoint{2.647211in}{1.282729in}}%
\pgfpathlineto{\pgfqpoint{2.657520in}{1.289866in}}%
\pgfpathlineto{\pgfqpoint{2.665251in}{1.291901in}}%
\pgfpathlineto{\pgfqpoint{2.670406in}{1.295342in}}%
\pgfpathlineto{\pgfqpoint{2.675560in}{1.297891in}}%
\pgfpathlineto{\pgfqpoint{2.683292in}{1.299292in}}%
\pgfpathlineto{\pgfqpoint{2.688446in}{1.302121in}}%
\pgfpathlineto{\pgfqpoint{2.693601in}{1.305390in}}%
\pgfpathlineto{\pgfqpoint{2.701332in}{1.307139in}}%
\pgfpathlineto{\pgfqpoint{2.703910in}{1.309000in}}%
\pgfpathlineto{\pgfqpoint{2.709064in}{1.311091in}}%
\pgfpathlineto{\pgfqpoint{2.711641in}{1.313180in}}%
\pgfpathlineto{\pgfqpoint{2.719373in}{1.315226in}}%
\pgfpathlineto{\pgfqpoint{2.721950in}{1.317475in}}%
\pgfpathlineto{\pgfqpoint{2.727105in}{1.319421in}}%
\pgfpathlineto{\pgfqpoint{2.729682in}{1.321428in}}%
\pgfpathlineto{\pgfqpoint{2.737413in}{1.323389in}}%
\pgfpathlineto{\pgfqpoint{2.747722in}{1.331714in}}%
\pgfpathlineto{\pgfqpoint{2.755454in}{1.333558in}}%
\pgfpathlineto{\pgfqpoint{2.763185in}{1.339459in}}%
\pgfpathlineto{\pgfqpoint{2.765763in}{1.341493in}}%
\pgfpathlineto{\pgfqpoint{2.776072in}{1.343661in}}%
\pgfpathlineto{\pgfqpoint{2.781226in}{1.348124in}}%
\pgfpathlineto{\pgfqpoint{2.783803in}{1.349605in}}%
\pgfpathlineto{\pgfqpoint{2.791535in}{1.351438in}}%
\pgfpathlineto{\pgfqpoint{2.801844in}{1.358689in}}%
\pgfpathlineto{\pgfqpoint{2.812153in}{1.360701in}}%
\pgfpathlineto{\pgfqpoint{2.817307in}{1.362383in}}%
\pgfpathlineto{\pgfqpoint{2.819884in}{1.363523in}}%
\pgfpathlineto{\pgfqpoint{2.827616in}{1.364793in}}%
\pgfpathlineto{\pgfqpoint{2.837925in}{1.371156in}}%
\pgfpathlineto{\pgfqpoint{2.848234in}{1.372820in}}%
\pgfpathlineto{\pgfqpoint{2.855965in}{1.378298in}}%
\pgfpathlineto{\pgfqpoint{2.863697in}{1.380401in}}%
\pgfpathlineto{\pgfqpoint{2.874006in}{1.388512in}}%
\pgfpathlineto{\pgfqpoint{2.881737in}{1.390550in}}%
\pgfpathlineto{\pgfqpoint{2.892046in}{1.399246in}}%
\pgfpathlineto{\pgfqpoint{2.899778in}{1.401359in}}%
\pgfpathlineto{\pgfqpoint{2.907509in}{1.405965in}}%
\pgfpathlineto{\pgfqpoint{2.910087in}{1.407107in}}%
\pgfpathlineto{\pgfqpoint{2.917818in}{1.408527in}}%
\pgfpathlineto{\pgfqpoint{2.928127in}{1.413957in}}%
\pgfpathlineto{\pgfqpoint{2.935859in}{1.415197in}}%
\pgfpathlineto{\pgfqpoint{2.946168in}{1.420561in}}%
\pgfpathlineto{\pgfqpoint{2.953899in}{1.422237in}}%
\pgfpathlineto{\pgfqpoint{2.964208in}{1.430243in}}%
\pgfpathlineto{\pgfqpoint{2.971940in}{1.431714in}}%
\pgfpathlineto{\pgfqpoint{2.982249in}{1.436926in}}%
\pgfpathlineto{\pgfqpoint{2.989980in}{1.438132in}}%
\pgfpathlineto{\pgfqpoint{2.995135in}{1.441145in}}%
\pgfpathlineto{\pgfqpoint{2.997712in}{1.442893in}}%
\pgfpathlineto{\pgfqpoint{3.008021in}{1.444576in}}%
\pgfpathlineto{\pgfqpoint{3.018330in}{1.451451in}}%
\pgfpathlineto{\pgfqpoint{3.026061in}{1.452950in}}%
\pgfpathlineto{\pgfqpoint{3.036370in}{1.458551in}}%
\pgfpathlineto{\pgfqpoint{3.044102in}{1.459841in}}%
\pgfpathlineto{\pgfqpoint{3.054411in}{1.464821in}}%
\pgfpathlineto{\pgfqpoint{3.062142in}{1.466323in}}%
\pgfpathlineto{\pgfqpoint{3.069874in}{1.470297in}}%
\pgfpathlineto{\pgfqpoint{3.072451in}{1.471238in}}%
\pgfpathlineto{\pgfqpoint{3.082760in}{1.472957in}}%
\pgfpathlineto{\pgfqpoint{3.090491in}{1.475894in}}%
\pgfpathlineto{\pgfqpoint{3.100800in}{1.476950in}}%
\pgfpathlineto{\pgfqpoint{3.108532in}{1.480190in}}%
\pgfpathlineto{\pgfqpoint{3.116264in}{1.481423in}}%
\pgfpathlineto{\pgfqpoint{3.126572in}{1.486588in}}%
\pgfpathlineto{\pgfqpoint{3.134304in}{1.488161in}}%
\pgfpathlineto{\pgfqpoint{3.142036in}{1.491904in}}%
\pgfpathlineto{\pgfqpoint{3.144613in}{1.492943in}}%
\pgfpathlineto{\pgfqpoint{3.152345in}{1.493983in}}%
\pgfpathlineto{\pgfqpoint{3.162653in}{1.498358in}}%
\pgfpathlineto{\pgfqpoint{3.172962in}{1.500206in}}%
\pgfpathlineto{\pgfqpoint{3.180694in}{1.502809in}}%
\pgfpathlineto{\pgfqpoint{3.188426in}{1.503573in}}%
\pgfpathlineto{\pgfqpoint{3.196157in}{1.505849in}}%
\pgfpathlineto{\pgfqpoint{3.209043in}{1.507104in}}%
\pgfpathlineto{\pgfqpoint{3.216775in}{1.508681in}}%
\pgfpathlineto{\pgfqpoint{3.227084in}{1.509935in}}%
\pgfpathlineto{\pgfqpoint{3.234815in}{1.511343in}}%
\pgfpathlineto{\pgfqpoint{3.252856in}{1.512009in}}%
\pgfpathlineto{\pgfqpoint{3.265742in}{1.511700in}}%
\pgfpathlineto{\pgfqpoint{3.288937in}{1.509646in}}%
\pgfpathlineto{\pgfqpoint{3.304400in}{1.508696in}}%
\pgfpathlineto{\pgfqpoint{3.306977in}{1.508485in}}%
\pgfpathlineto{\pgfqpoint{3.335327in}{1.508914in}}%
\pgfpathlineto{\pgfqpoint{3.376562in}{1.508884in}}%
\pgfpathlineto{\pgfqpoint{3.410066in}{1.507926in}}%
\pgfpathlineto{\pgfqpoint{3.415220in}{1.507352in}}%
\pgfpathlineto{\pgfqpoint{3.428106in}{1.506485in}}%
\pgfpathlineto{\pgfqpoint{3.433261in}{1.505797in}}%
\pgfpathlineto{\pgfqpoint{3.443570in}{1.505018in}}%
\pgfpathlineto{\pgfqpoint{3.469342in}{1.501085in}}%
\pgfpathlineto{\pgfqpoint{3.479651in}{1.500214in}}%
\pgfpathlineto{\pgfqpoint{3.487382in}{1.499048in}}%
\pgfpathlineto{\pgfqpoint{3.551813in}{1.496923in}}%
\pgfpathlineto{\pgfqpoint{3.572430in}{1.497259in}}%
\pgfpathlineto{\pgfqpoint{3.603357in}{1.498169in}}%
\pgfpathlineto{\pgfqpoint{3.613666in}{1.499595in}}%
\pgfpathlineto{\pgfqpoint{3.626552in}{1.500448in}}%
\pgfpathlineto{\pgfqpoint{3.631706in}{1.501615in}}%
\pgfpathlineto{\pgfqpoint{3.642015in}{1.502946in}}%
\pgfpathlineto{\pgfqpoint{3.644592in}{1.503575in}}%
\pgfpathlineto{\pgfqpoint{3.667787in}{1.506232in}}%
\pgfpathlineto{\pgfqpoint{3.683250in}{1.507175in}}%
\pgfpathlineto{\pgfqpoint{3.685828in}{1.507487in}}%
\pgfpathlineto{\pgfqpoint{3.701291in}{1.508687in}}%
\pgfpathlineto{\pgfqpoint{3.703868in}{1.509177in}}%
\pgfpathlineto{\pgfqpoint{3.714177in}{1.509798in}}%
\pgfpathlineto{\pgfqpoint{3.721909in}{1.512355in}}%
\pgfpathlineto{\pgfqpoint{3.732217in}{1.513840in}}%
\pgfpathlineto{\pgfqpoint{3.739949in}{1.515170in}}%
\pgfpathlineto{\pgfqpoint{3.750258in}{1.516360in}}%
\pgfpathlineto{\pgfqpoint{3.757990in}{1.518562in}}%
\pgfpathlineto{\pgfqpoint{3.768298in}{1.520022in}}%
\pgfpathlineto{\pgfqpoint{3.776030in}{1.522531in}}%
\pgfpathlineto{\pgfqpoint{3.786339in}{1.523516in}}%
\pgfpathlineto{\pgfqpoint{3.794070in}{1.527014in}}%
\pgfpathlineto{\pgfqpoint{3.801802in}{1.528247in}}%
\pgfpathlineto{\pgfqpoint{3.812111in}{1.532720in}}%
\pgfpathlineto{\pgfqpoint{3.819843in}{1.533845in}}%
\pgfpathlineto{\pgfqpoint{3.830151in}{1.537185in}}%
\pgfpathlineto{\pgfqpoint{3.843038in}{1.539158in}}%
\pgfpathlineto{\pgfqpoint{3.848192in}{1.540605in}}%
\pgfpathlineto{\pgfqpoint{3.858501in}{1.542020in}}%
\pgfpathlineto{\pgfqpoint{3.866232in}{1.544029in}}%
\pgfpathlineto{\pgfqpoint{3.876541in}{1.545165in}}%
\pgfpathlineto{\pgfqpoint{3.899736in}{1.547851in}}%
\pgfpathlineto{\pgfqpoint{3.912622in}{1.548714in}}%
\pgfpathlineto{\pgfqpoint{3.920354in}{1.550082in}}%
\pgfpathlineto{\pgfqpoint{3.930663in}{1.550886in}}%
\pgfpathlineto{\pgfqpoint{3.938394in}{1.551893in}}%
\pgfpathlineto{\pgfqpoint{3.951280in}{1.552861in}}%
\pgfpathlineto{\pgfqpoint{3.956435in}{1.553476in}}%
\pgfpathlineto{\pgfqpoint{3.984784in}{1.554849in}}%
\pgfpathlineto{\pgfqpoint{4.002825in}{1.556344in}}%
\pgfpathlineto{\pgfqpoint{4.028597in}{1.560129in}}%
\pgfpathlineto{\pgfqpoint{4.041483in}{1.560805in}}%
\pgfpathlineto{\pgfqpoint{4.046637in}{1.561498in}}%
\pgfpathlineto{\pgfqpoint{4.059523in}{1.562561in}}%
\pgfpathlineto{\pgfqpoint{4.064678in}{1.563166in}}%
\pgfpathlineto{\pgfqpoint{4.077564in}{1.563962in}}%
\pgfpathlineto{\pgfqpoint{4.082718in}{1.564682in}}%
\pgfpathlineto{\pgfqpoint{4.108490in}{1.565501in}}%
\pgfpathlineto{\pgfqpoint{4.118799in}{1.565717in}}%
\pgfpathlineto{\pgfqpoint{4.144571in}{1.564931in}}%
\pgfpathlineto{\pgfqpoint{4.172921in}{1.563503in}}%
\pgfpathlineto{\pgfqpoint{4.183230in}{1.562895in}}%
\pgfpathlineto{\pgfqpoint{4.190961in}{1.561452in}}%
\pgfpathlineto{\pgfqpoint{4.201270in}{1.560449in}}%
\pgfpathlineto{\pgfqpoint{4.209002in}{1.558992in}}%
\pgfpathlineto{\pgfqpoint{4.219311in}{1.558002in}}%
\pgfpathlineto{\pgfqpoint{4.227042in}{1.556511in}}%
\pgfpathlineto{\pgfqpoint{4.237351in}{1.555527in}}%
\pgfpathlineto{\pgfqpoint{4.245083in}{1.554061in}}%
\pgfpathlineto{\pgfqpoint{4.255392in}{1.553105in}}%
\pgfpathlineto{\pgfqpoint{4.263123in}{1.551634in}}%
\pgfpathlineto{\pgfqpoint{4.273432in}{1.550773in}}%
\pgfpathlineto{\pgfqpoint{4.281164in}{1.549345in}}%
\pgfpathlineto{\pgfqpoint{4.291472in}{1.548456in}}%
\pgfpathlineto{\pgfqpoint{4.299204in}{1.547094in}}%
\pgfpathlineto{\pgfqpoint{4.312090in}{1.546160in}}%
\pgfpathlineto{\pgfqpoint{4.317245in}{1.545240in}}%
\pgfpathlineto{\pgfqpoint{4.327553in}{1.544313in}}%
\pgfpathlineto{\pgfqpoint{4.335285in}{1.542918in}}%
\pgfpathlineto{\pgfqpoint{4.348171in}{1.541740in}}%
\pgfpathlineto{\pgfqpoint{4.353326in}{1.541044in}}%
\pgfpathlineto{\pgfqpoint{4.366212in}{1.539932in}}%
\pgfpathlineto{\pgfqpoint{4.371366in}{1.539134in}}%
\pgfpathlineto{\pgfqpoint{4.381675in}{1.538222in}}%
\pgfpathlineto{\pgfqpoint{4.389407in}{1.536816in}}%
\pgfpathlineto{\pgfqpoint{4.399715in}{1.535877in}}%
\pgfpathlineto{\pgfqpoint{4.407447in}{1.534514in}}%
\pgfpathlineto{\pgfqpoint{4.417756in}{1.533602in}}%
\pgfpathlineto{\pgfqpoint{4.425488in}{1.532288in}}%
\pgfpathlineto{\pgfqpoint{4.435796in}{1.531415in}}%
\pgfpathlineto{\pgfqpoint{4.443528in}{1.530086in}}%
\pgfpathlineto{\pgfqpoint{4.453837in}{1.529222in}}%
\pgfpathlineto{\pgfqpoint{4.461569in}{1.527975in}}%
\pgfpathlineto{\pgfqpoint{4.471877in}{1.527111in}}%
\pgfpathlineto{\pgfqpoint{4.479609in}{1.525797in}}%
\pgfpathlineto{\pgfqpoint{4.489918in}{1.524914in}}%
\pgfpathlineto{\pgfqpoint{4.497649in}{1.523606in}}%
\pgfpathlineto{\pgfqpoint{4.507958in}{1.522731in}}%
\pgfpathlineto{\pgfqpoint{4.515690in}{1.521847in}}%
\pgfpathlineto{\pgfqpoint{4.525999in}{1.520959in}}%
\pgfpathlineto{\pgfqpoint{4.533730in}{1.519624in}}%
\pgfpathlineto{\pgfqpoint{4.544039in}{1.518740in}}%
\pgfpathlineto{\pgfqpoint{4.551771in}{1.517424in}}%
\pgfpathlineto{\pgfqpoint{4.562080in}{1.516557in}}%
\pgfpathlineto{\pgfqpoint{4.569811in}{1.515257in}}%
\pgfpathlineto{\pgfqpoint{4.580120in}{1.514387in}}%
\pgfpathlineto{\pgfqpoint{4.585275in}{1.513520in}}%
\pgfpathlineto{\pgfqpoint{4.598161in}{1.512661in}}%
\pgfpathlineto{\pgfqpoint{4.603315in}{1.511802in}}%
\pgfpathlineto{\pgfqpoint{4.616201in}{1.510939in}}%
\pgfpathlineto{\pgfqpoint{4.623933in}{1.509711in}}%
\pgfpathlineto{\pgfqpoint{4.636819in}{1.508566in}}%
\pgfpathlineto{\pgfqpoint{4.641973in}{1.507928in}}%
\pgfpathlineto{\pgfqpoint{4.657437in}{1.507149in}}%
\pgfpathlineto{\pgfqpoint{4.660014in}{1.506853in}}%
\pgfpathlineto{\pgfqpoint{4.672900in}{1.506052in}}%
\pgfpathlineto{\pgfqpoint{4.685786in}{1.505097in}}%
\pgfpathlineto{\pgfqpoint{4.714135in}{1.502426in}}%
\pgfpathlineto{\pgfqpoint{4.727021in}{1.501779in}}%
\pgfpathlineto{\pgfqpoint{4.732176in}{1.501070in}}%
\pgfpathlineto{\pgfqpoint{4.745062in}{1.499849in}}%
\pgfpathlineto{\pgfqpoint{4.750216in}{1.499073in}}%
\pgfpathlineto{\pgfqpoint{4.763102in}{1.497858in}}%
\pgfpathlineto{\pgfqpoint{4.768257in}{1.497061in}}%
\pgfpathlineto{\pgfqpoint{4.781143in}{1.495876in}}%
\pgfpathlineto{\pgfqpoint{4.786297in}{1.495082in}}%
\pgfpathlineto{\pgfqpoint{4.799183in}{1.493907in}}%
\pgfpathlineto{\pgfqpoint{4.804338in}{1.493178in}}%
\pgfpathlineto{\pgfqpoint{4.817224in}{1.492089in}}%
\pgfpathlineto{\pgfqpoint{4.819801in}{1.491728in}}%
\pgfpathlineto{\pgfqpoint{4.835264in}{1.490686in}}%
\pgfpathlineto{\pgfqpoint{4.840419in}{1.490002in}}%
\pgfpathlineto{\pgfqpoint{4.853305in}{1.488988in}}%
\pgfpathlineto{\pgfqpoint{4.858459in}{1.488354in}}%
\pgfpathlineto{\pgfqpoint{4.886809in}{1.486752in}}%
\pgfpathlineto{\pgfqpoint{4.912581in}{1.485179in}}%
\pgfpathlineto{\pgfqpoint{4.925467in}{1.484343in}}%
\pgfpathlineto{\pgfqpoint{4.930621in}{1.483701in}}%
\pgfpathlineto{\pgfqpoint{4.943507in}{1.482781in}}%
\pgfpathlineto{\pgfqpoint{4.948662in}{1.482150in}}%
\pgfpathlineto{\pgfqpoint{4.961548in}{1.481191in}}%
\pgfpathlineto{\pgfqpoint{4.966702in}{1.480510in}}%
\pgfpathlineto{\pgfqpoint{4.979588in}{1.479542in}}%
\pgfpathlineto{\pgfqpoint{4.984743in}{1.478933in}}%
\pgfpathlineto{\pgfqpoint{5.000206in}{1.478021in}}%
\pgfpathlineto{\pgfqpoint{5.002783in}{1.477714in}}%
\pgfpathlineto{\pgfqpoint{5.018246in}{1.476622in}}%
\pgfpathlineto{\pgfqpoint{5.038864in}{1.474903in}}%
\pgfpathlineto{\pgfqpoint{5.054327in}{1.473813in}}%
\pgfpathlineto{\pgfqpoint{5.056905in}{1.473482in}}%
\pgfpathlineto{\pgfqpoint{5.072368in}{1.472252in}}%
\pgfpathlineto{\pgfqpoint{5.074945in}{1.472009in}}%
\pgfpathlineto{\pgfqpoint{5.100717in}{1.470748in}}%
\pgfpathlineto{\pgfqpoint{5.121335in}{1.469772in}}%
\pgfpathlineto{\pgfqpoint{5.147107in}{1.468886in}}%
\pgfpathlineto{\pgfqpoint{5.263082in}{1.466918in}}%
\pgfpathlineto{\pgfqpoint{5.291431in}{1.464637in}}%
\pgfpathlineto{\pgfqpoint{5.306894in}{1.463656in}}%
\pgfpathlineto{\pgfqpoint{5.327512in}{1.462152in}}%
\pgfpathlineto{\pgfqpoint{5.340398in}{1.461251in}}%
\pgfpathlineto{\pgfqpoint{5.345552in}{1.460679in}}%
\pgfpathlineto{\pgfqpoint{5.358438in}{1.459811in}}%
\pgfpathlineto{\pgfqpoint{5.363593in}{1.459208in}}%
\pgfpathlineto{\pgfqpoint{5.379056in}{1.458112in}}%
\pgfpathlineto{\pgfqpoint{5.399674in}{1.456622in}}%
\pgfpathlineto{\pgfqpoint{5.453795in}{1.455629in}}%
\pgfpathlineto{\pgfqpoint{5.520803in}{1.456162in}}%
\pgfpathlineto{\pgfqpoint{5.543998in}{1.456692in}}%
\pgfpathlineto{\pgfqpoint{5.572347in}{1.457499in}}%
\pgfpathlineto{\pgfqpoint{5.598119in}{1.458098in}}%
\pgfpathlineto{\pgfqpoint{5.662550in}{1.459207in}}%
\pgfpathlineto{\pgfqpoint{5.688322in}{1.460532in}}%
\pgfpathlineto{\pgfqpoint{5.703785in}{1.461414in}}%
\pgfpathlineto{\pgfqpoint{5.724403in}{1.462399in}}%
\pgfpathlineto{\pgfqpoint{5.742443in}{1.463353in}}%
\pgfpathlineto{\pgfqpoint{5.788833in}{1.465328in}}%
\pgfpathlineto{\pgfqpoint{5.812028in}{1.466531in}}%
\pgfpathlineto{\pgfqpoint{5.830068in}{1.467371in}}%
\pgfpathlineto{\pgfqpoint{5.863572in}{1.471764in}}%
\pgfpathlineto{\pgfqpoint{5.868727in}{1.473151in}}%
\pgfpathlineto{\pgfqpoint{5.879035in}{1.474541in}}%
\pgfpathlineto{\pgfqpoint{5.886767in}{1.476551in}}%
\pgfpathlineto{\pgfqpoint{5.897076in}{1.477893in}}%
\pgfpathlineto{\pgfqpoint{5.904807in}{1.479784in}}%
\pgfpathlineto{\pgfqpoint{5.915116in}{1.481305in}}%
\pgfpathlineto{\pgfqpoint{5.922848in}{1.483605in}}%
\pgfpathlineto{\pgfqpoint{5.935734in}{1.485003in}}%
\pgfpathlineto{\pgfqpoint{5.940888in}{1.486491in}}%
\pgfpathlineto{\pgfqpoint{5.951197in}{1.487757in}}%
\pgfpathlineto{\pgfqpoint{5.958929in}{1.489542in}}%
\pgfpathlineto{\pgfqpoint{5.969238in}{1.490698in}}%
\pgfpathlineto{\pgfqpoint{5.976969in}{1.492503in}}%
\pgfpathlineto{\pgfqpoint{5.987278in}{1.493879in}}%
\pgfpathlineto{\pgfqpoint{5.995010in}{1.496003in}}%
\pgfpathlineto{\pgfqpoint{6.005319in}{1.497356in}}%
\pgfpathlineto{\pgfqpoint{6.013050in}{1.499443in}}%
\pgfpathlineto{\pgfqpoint{6.025936in}{1.500871in}}%
\pgfpathlineto{\pgfqpoint{6.031091in}{1.502260in}}%
\pgfpathlineto{\pgfqpoint{6.041400in}{1.503794in}}%
\pgfpathlineto{\pgfqpoint{6.049131in}{1.506092in}}%
\pgfpathlineto{\pgfqpoint{6.059440in}{1.507527in}}%
\pgfpathlineto{\pgfqpoint{6.067172in}{1.509689in}}%
\pgfpathlineto{\pgfqpoint{6.080058in}{1.511410in}}%
\pgfpathlineto{\pgfqpoint{6.085212in}{1.512246in}}%
\pgfpathlineto{\pgfqpoint{6.095521in}{1.513068in}}%
\pgfpathlineto{\pgfqpoint{6.103253in}{1.514724in}}%
\pgfpathlineto{\pgfqpoint{6.116139in}{1.515791in}}%
\pgfpathlineto{\pgfqpoint{6.129025in}{1.516664in}}%
\pgfpathlineto{\pgfqpoint{6.157374in}{1.519064in}}%
\pgfpathlineto{\pgfqpoint{6.167683in}{1.519712in}}%
\pgfpathlineto{\pgfqpoint{6.175415in}{1.521035in}}%
\pgfpathlineto{\pgfqpoint{6.226959in}{1.521635in}}%
\pgfpathlineto{\pgfqpoint{6.291389in}{1.526856in}}%
\pgfpathlineto{\pgfqpoint{6.301698in}{1.528591in}}%
\pgfpathlineto{\pgfqpoint{6.314584in}{1.529900in}}%
\pgfpathlineto{\pgfqpoint{6.319739in}{1.530717in}}%
\pgfpathlineto{\pgfqpoint{6.330048in}{1.531562in}}%
\pgfpathlineto{\pgfqpoint{6.337779in}{1.533018in}}%
\pgfpathlineto{\pgfqpoint{6.348088in}{1.533938in}}%
\pgfpathlineto{\pgfqpoint{6.355820in}{1.535015in}}%
\pgfpathlineto{\pgfqpoint{6.368706in}{1.535892in}}%
\pgfpathlineto{\pgfqpoint{6.373860in}{1.536482in}}%
\pgfpathlineto{\pgfqpoint{6.399632in}{1.537829in}}%
\pgfpathlineto{\pgfqpoint{6.425404in}{1.541454in}}%
\pgfpathlineto{\pgfqpoint{6.427982in}{1.542064in}}%
\pgfpathlineto{\pgfqpoint{6.438290in}{1.543356in}}%
\pgfpathlineto{\pgfqpoint{6.446022in}{1.545560in}}%
\pgfpathlineto{\pgfqpoint{6.453754in}{1.546431in}}%
\pgfpathlineto{\pgfqpoint{6.464063in}{1.549994in}}%
\pgfpathlineto{\pgfqpoint{6.474371in}{1.550880in}}%
\pgfpathlineto{\pgfqpoint{6.482103in}{1.553657in}}%
\pgfpathlineto{\pgfqpoint{6.482103in}{1.553657in}}%
\pgfusepath{stroke}%
\end{pgfscope}%
\begin{pgfscope}%
\pgfpathrectangle{\pgfqpoint{0.563921in}{0.521603in}}{\pgfqpoint{6.200000in}{2.642500in}}%
\pgfusepath{clip}%
\pgfsetroundcap%
\pgfsetroundjoin%
\pgfsetlinewidth{1.505625pt}%
\definecolor{currentstroke}{rgb}{0.498039,0.498039,0.498039}%
\pgfsetstrokecolor{currentstroke}%
\pgfsetdash{}{0pt}%
\pgfpathmoveto{\pgfqpoint{0.845739in}{0.641717in}}%
\pgfpathlineto{\pgfqpoint{0.848317in}{0.652443in}}%
\pgfpathlineto{\pgfqpoint{0.850894in}{0.655308in}}%
\pgfpathlineto{\pgfqpoint{0.853471in}{0.656194in}}%
\pgfpathlineto{\pgfqpoint{0.861203in}{0.655643in}}%
\pgfpathlineto{\pgfqpoint{0.863780in}{0.654449in}}%
\pgfpathlineto{\pgfqpoint{0.889552in}{0.653292in}}%
\pgfpathlineto{\pgfqpoint{0.897284in}{0.653691in}}%
\pgfpathlineto{\pgfqpoint{0.907593in}{0.666884in}}%
\pgfpathlineto{\pgfqpoint{0.925633in}{0.668074in}}%
\pgfpathlineto{\pgfqpoint{0.956560in}{0.665628in}}%
\pgfpathlineto{\pgfqpoint{0.961714in}{0.664992in}}%
\pgfpathlineto{\pgfqpoint{0.977177in}{0.664086in}}%
\pgfpathlineto{\pgfqpoint{0.979754in}{0.663791in}}%
\pgfpathlineto{\pgfqpoint{1.010681in}{0.663094in}}%
\pgfpathlineto{\pgfqpoint{1.015835in}{0.663967in}}%
\pgfpathlineto{\pgfqpoint{1.023567in}{0.664577in}}%
\pgfpathlineto{\pgfqpoint{1.033876in}{0.667417in}}%
\pgfpathlineto{\pgfqpoint{1.044185in}{0.668762in}}%
\pgfpathlineto{\pgfqpoint{1.049339in}{0.670046in}}%
\pgfpathlineto{\pgfqpoint{1.059648in}{0.670345in}}%
\pgfpathlineto{\pgfqpoint{1.069957in}{0.669614in}}%
\pgfpathlineto{\pgfqpoint{1.098306in}{0.669303in}}%
\pgfpathlineto{\pgfqpoint{1.134387in}{0.669593in}}%
\pgfpathlineto{\pgfqpoint{1.136964in}{0.670324in}}%
\pgfpathlineto{\pgfqpoint{1.142119in}{0.673480in}}%
\pgfpathlineto{\pgfqpoint{1.149851in}{0.675124in}}%
\pgfpathlineto{\pgfqpoint{1.157582in}{0.680097in}}%
\pgfpathlineto{\pgfqpoint{1.160159in}{0.681052in}}%
\pgfpathlineto{\pgfqpoint{1.167891in}{0.682254in}}%
\pgfpathlineto{\pgfqpoint{1.178200in}{0.686717in}}%
\pgfpathlineto{\pgfqpoint{1.185931in}{0.687900in}}%
\pgfpathlineto{\pgfqpoint{1.196240in}{0.693154in}}%
\pgfpathlineto{\pgfqpoint{1.203972in}{0.694367in}}%
\pgfpathlineto{\pgfqpoint{1.214281in}{0.698635in}}%
\pgfpathlineto{\pgfqpoint{1.224590in}{0.699883in}}%
\pgfpathlineto{\pgfqpoint{1.232321in}{0.702264in}}%
\pgfpathlineto{\pgfqpoint{1.242630in}{0.703483in}}%
\pgfpathlineto{\pgfqpoint{1.250362in}{0.706600in}}%
\pgfpathlineto{\pgfqpoint{1.258093in}{0.708022in}}%
\pgfpathlineto{\pgfqpoint{1.265825in}{0.713642in}}%
\pgfpathlineto{\pgfqpoint{1.268402in}{0.715585in}}%
\pgfpathlineto{\pgfqpoint{1.276134in}{0.717663in}}%
\pgfpathlineto{\pgfqpoint{1.281288in}{0.720927in}}%
\pgfpathlineto{\pgfqpoint{1.286443in}{0.724051in}}%
\pgfpathlineto{\pgfqpoint{1.294174in}{0.725484in}}%
\pgfpathlineto{\pgfqpoint{1.304483in}{0.731590in}}%
\pgfpathlineto{\pgfqpoint{1.312215in}{0.733665in}}%
\pgfpathlineto{\pgfqpoint{1.314792in}{0.735638in}}%
\pgfpathlineto{\pgfqpoint{1.319947in}{0.737460in}}%
\pgfpathlineto{\pgfqpoint{1.322524in}{0.739209in}}%
\pgfpathlineto{\pgfqpoint{1.330255in}{0.741119in}}%
\pgfpathlineto{\pgfqpoint{1.340564in}{0.748244in}}%
\pgfpathlineto{\pgfqpoint{1.348296in}{0.750105in}}%
\pgfpathlineto{\pgfqpoint{1.353450in}{0.754305in}}%
\pgfpathlineto{\pgfqpoint{1.358605in}{0.756463in}}%
\pgfpathlineto{\pgfqpoint{1.371491in}{0.758497in}}%
\pgfpathlineto{\pgfqpoint{1.376645in}{0.760458in}}%
\pgfpathlineto{\pgfqpoint{1.384377in}{0.761527in}}%
\pgfpathlineto{\pgfqpoint{1.392108in}{0.764541in}}%
\pgfpathlineto{\pgfqpoint{1.394686in}{0.765203in}}%
\pgfpathlineto{\pgfqpoint{1.404995in}{0.766557in}}%
\pgfpathlineto{\pgfqpoint{1.412726in}{0.768131in}}%
\pgfpathlineto{\pgfqpoint{1.425612in}{0.769391in}}%
\pgfpathlineto{\pgfqpoint{1.430767in}{0.770022in}}%
\pgfpathlineto{\pgfqpoint{1.443653in}{0.769969in}}%
\pgfpathlineto{\pgfqpoint{1.448807in}{0.769746in}}%
\pgfpathlineto{\pgfqpoint{1.482311in}{0.769734in}}%
\pgfpathlineto{\pgfqpoint{1.497774in}{0.770697in}}%
\pgfpathlineto{\pgfqpoint{1.502929in}{0.771684in}}%
\pgfpathlineto{\pgfqpoint{1.513237in}{0.772373in}}%
\pgfpathlineto{\pgfqpoint{1.520969in}{0.774062in}}%
\pgfpathlineto{\pgfqpoint{1.531278in}{0.775230in}}%
\pgfpathlineto{\pgfqpoint{1.539010in}{0.776837in}}%
\pgfpathlineto{\pgfqpoint{1.549318in}{0.777961in}}%
\pgfpathlineto{\pgfqpoint{1.554473in}{0.779756in}}%
\pgfpathlineto{\pgfqpoint{1.557050in}{0.780954in}}%
\pgfpathlineto{\pgfqpoint{1.567359in}{0.782625in}}%
\pgfpathlineto{\pgfqpoint{1.572513in}{0.783579in}}%
\pgfpathlineto{\pgfqpoint{1.582822in}{0.783915in}}%
\pgfpathlineto{\pgfqpoint{1.587977in}{0.784150in}}%
\pgfpathlineto{\pgfqpoint{1.593131in}{0.784959in}}%
\pgfpathlineto{\pgfqpoint{1.642098in}{0.785673in}}%
\pgfpathlineto{\pgfqpoint{1.665293in}{0.783779in}}%
\pgfpathlineto{\pgfqpoint{1.729723in}{0.782377in}}%
\pgfpathlineto{\pgfqpoint{1.745187in}{0.782402in}}%
\pgfpathlineto{\pgfqpoint{1.863738in}{0.778461in}}%
\pgfpathlineto{\pgfqpoint{1.917860in}{0.779533in}}%
\pgfpathlineto{\pgfqpoint{1.928169in}{0.780169in}}%
\pgfpathlineto{\pgfqpoint{1.951364in}{0.784097in}}%
\pgfpathlineto{\pgfqpoint{1.953941in}{0.784877in}}%
\pgfpathlineto{\pgfqpoint{1.961672in}{0.785595in}}%
\pgfpathlineto{\pgfqpoint{1.971981in}{0.788789in}}%
\pgfpathlineto{\pgfqpoint{1.979713in}{0.789715in}}%
\pgfpathlineto{\pgfqpoint{1.990022in}{0.793345in}}%
\pgfpathlineto{\pgfqpoint{1.997753in}{0.794306in}}%
\pgfpathlineto{\pgfqpoint{2.005485in}{0.797099in}}%
\pgfpathlineto{\pgfqpoint{2.015794in}{0.797999in}}%
\pgfpathlineto{\pgfqpoint{2.026103in}{0.801569in}}%
\pgfpathlineto{\pgfqpoint{2.033834in}{0.802604in}}%
\pgfpathlineto{\pgfqpoint{2.041566in}{0.806045in}}%
\pgfpathlineto{\pgfqpoint{2.044143in}{0.807465in}}%
\pgfpathlineto{\pgfqpoint{2.051875in}{0.808776in}}%
\pgfpathlineto{\pgfqpoint{2.062184in}{0.814009in}}%
\pgfpathlineto{\pgfqpoint{2.069915in}{0.815723in}}%
\pgfpathlineto{\pgfqpoint{2.080224in}{0.823040in}}%
\pgfpathlineto{\pgfqpoint{2.087956in}{0.825026in}}%
\pgfpathlineto{\pgfqpoint{2.093110in}{0.828656in}}%
\pgfpathlineto{\pgfqpoint{2.098265in}{0.831729in}}%
\pgfpathlineto{\pgfqpoint{2.105996in}{0.833035in}}%
\pgfpathlineto{\pgfqpoint{2.116305in}{0.839084in}}%
\pgfpathlineto{\pgfqpoint{2.124037in}{0.840407in}}%
\pgfpathlineto{\pgfqpoint{2.134346in}{0.846402in}}%
\pgfpathlineto{\pgfqpoint{2.142077in}{0.847647in}}%
\pgfpathlineto{\pgfqpoint{2.152386in}{0.851302in}}%
\pgfpathlineto{\pgfqpoint{2.185890in}{0.853830in}}%
\pgfpathlineto{\pgfqpoint{2.188467in}{0.854307in}}%
\pgfpathlineto{\pgfqpoint{2.201353in}{0.855668in}}%
\pgfpathlineto{\pgfqpoint{2.206508in}{0.856799in}}%
\pgfpathlineto{\pgfqpoint{2.216816in}{0.858014in}}%
\pgfpathlineto{\pgfqpoint{2.224548in}{0.858845in}}%
\pgfpathlineto{\pgfqpoint{2.234857in}{0.859474in}}%
\pgfpathlineto{\pgfqpoint{2.242589in}{0.860841in}}%
\pgfpathlineto{\pgfqpoint{2.252897in}{0.861652in}}%
\pgfpathlineto{\pgfqpoint{2.260629in}{0.862688in}}%
\pgfpathlineto{\pgfqpoint{2.270938in}{0.863861in}}%
\pgfpathlineto{\pgfqpoint{2.278670in}{0.865350in}}%
\pgfpathlineto{\pgfqpoint{2.288978in}{0.866083in}}%
\pgfpathlineto{\pgfqpoint{2.296710in}{0.867195in}}%
\pgfpathlineto{\pgfqpoint{2.309596in}{0.868309in}}%
\pgfpathlineto{\pgfqpoint{2.314751in}{0.869213in}}%
\pgfpathlineto{\pgfqpoint{2.366295in}{0.872525in}}%
\pgfpathlineto{\pgfqpoint{2.430725in}{0.870682in}}%
\pgfpathlineto{\pgfqpoint{2.441034in}{0.870210in}}%
\pgfpathlineto{\pgfqpoint{2.477115in}{0.869738in}}%
\pgfpathlineto{\pgfqpoint{2.520928in}{0.868374in}}%
\pgfpathlineto{\pgfqpoint{2.528659in}{0.868271in}}%
\pgfpathlineto{\pgfqpoint{2.567317in}{0.871773in}}%
\pgfpathlineto{\pgfqpoint{2.600821in}{0.873855in}}%
\pgfpathlineto{\pgfqpoint{2.621439in}{0.875250in}}%
\pgfpathlineto{\pgfqpoint{2.683292in}{0.876126in}}%
\pgfpathlineto{\pgfqpoint{2.701332in}{0.875902in}}%
\pgfpathlineto{\pgfqpoint{2.755454in}{0.875847in}}%
\pgfpathlineto{\pgfqpoint{2.765763in}{0.875741in}}%
\pgfpathlineto{\pgfqpoint{2.809575in}{0.875018in}}%
\pgfpathlineto{\pgfqpoint{2.837925in}{0.873980in}}%
\pgfpathlineto{\pgfqpoint{2.889469in}{0.872631in}}%
\pgfpathlineto{\pgfqpoint{2.946168in}{0.870717in}}%
\pgfpathlineto{\pgfqpoint{3.031216in}{0.869317in}}%
\pgfpathlineto{\pgfqpoint{3.054411in}{0.869083in}}%
\pgfpathlineto{\pgfqpoint{3.100800in}{0.869578in}}%
\pgfpathlineto{\pgfqpoint{3.126572in}{0.870397in}}%
\pgfpathlineto{\pgfqpoint{3.214198in}{0.872007in}}%
\pgfpathlineto{\pgfqpoint{3.234815in}{0.873328in}}%
\pgfpathlineto{\pgfqpoint{3.250279in}{0.874302in}}%
\pgfpathlineto{\pgfqpoint{3.268319in}{0.875760in}}%
\pgfpathlineto{\pgfqpoint{3.288937in}{0.876202in}}%
\pgfpathlineto{\pgfqpoint{3.353367in}{0.876795in}}%
\pgfpathlineto{\pgfqpoint{3.376562in}{0.877181in}}%
\pgfpathlineto{\pgfqpoint{3.397180in}{0.877400in}}%
\pgfpathlineto{\pgfqpoint{3.531195in}{0.879886in}}%
\pgfpathlineto{\pgfqpoint{3.559544in}{0.881514in}}%
\pgfpathlineto{\pgfqpoint{3.613666in}{0.881494in}}%
\pgfpathlineto{\pgfqpoint{3.770876in}{0.879107in}}%
\pgfpathlineto{\pgfqpoint{3.794070in}{0.879409in}}%
\pgfpathlineto{\pgfqpoint{3.899736in}{0.880273in}}%
\pgfpathlineto{\pgfqpoint{3.951280in}{0.881154in}}%
\pgfpathlineto{\pgfqpoint{3.992516in}{0.882782in}}%
\pgfpathlineto{\pgfqpoint{4.093027in}{0.883607in}}%
\pgfpathlineto{\pgfqpoint{4.154880in}{0.882571in}}%
\pgfpathlineto{\pgfqpoint{4.216733in}{0.881719in}}%
\pgfpathlineto{\pgfqpoint{4.263123in}{0.881085in}}%
\pgfpathlineto{\pgfqpoint{4.314667in}{0.879743in}}%
\pgfpathlineto{\pgfqpoint{4.407447in}{0.876813in}}%
\pgfpathlineto{\pgfqpoint{4.492495in}{0.875158in}}%
\pgfpathlineto{\pgfqpoint{4.569811in}{0.873472in}}%
\pgfpathlineto{\pgfqpoint{4.685786in}{0.872971in}}%
\pgfpathlineto{\pgfqpoint{4.714135in}{0.874893in}}%
\pgfpathlineto{\pgfqpoint{4.729599in}{0.875571in}}%
\pgfpathlineto{\pgfqpoint{4.750216in}{0.877105in}}%
\pgfpathlineto{\pgfqpoint{4.763102in}{0.877996in}}%
\pgfpathlineto{\pgfqpoint{4.768257in}{0.878655in}}%
\pgfpathlineto{\pgfqpoint{4.781143in}{0.879774in}}%
\pgfpathlineto{\pgfqpoint{4.786297in}{0.880526in}}%
\pgfpathlineto{\pgfqpoint{4.796606in}{0.881305in}}%
\pgfpathlineto{\pgfqpoint{4.804338in}{0.882693in}}%
\pgfpathlineto{\pgfqpoint{4.817224in}{0.884001in}}%
\pgfpathlineto{\pgfqpoint{4.819801in}{0.884471in}}%
\pgfpathlineto{\pgfqpoint{4.832687in}{0.885439in}}%
\pgfpathlineto{\pgfqpoint{4.840419in}{0.886971in}}%
\pgfpathlineto{\pgfqpoint{4.850728in}{0.888023in}}%
\pgfpathlineto{\pgfqpoint{4.858459in}{0.889228in}}%
\pgfpathlineto{\pgfqpoint{4.871345in}{0.890133in}}%
\pgfpathlineto{\pgfqpoint{4.891963in}{0.891785in}}%
\pgfpathlineto{\pgfqpoint{4.912581in}{0.893136in}}%
\pgfpathlineto{\pgfqpoint{4.938353in}{0.894477in}}%
\pgfpathlineto{\pgfqpoint{4.948662in}{0.895443in}}%
\pgfpathlineto{\pgfqpoint{5.000206in}{0.897463in}}%
\pgfpathlineto{\pgfqpoint{5.049173in}{0.901709in}}%
\pgfpathlineto{\pgfqpoint{5.056905in}{0.903157in}}%
\pgfpathlineto{\pgfqpoint{5.067213in}{0.904192in}}%
\pgfpathlineto{\pgfqpoint{5.074945in}{0.906030in}}%
\pgfpathlineto{\pgfqpoint{5.087831in}{0.907462in}}%
\pgfpathlineto{\pgfqpoint{5.092986in}{0.908762in}}%
\pgfpathlineto{\pgfqpoint{5.103294in}{0.910060in}}%
\pgfpathlineto{\pgfqpoint{5.111026in}{0.912043in}}%
\pgfpathlineto{\pgfqpoint{5.121335in}{0.913328in}}%
\pgfpathlineto{\pgfqpoint{5.129067in}{0.915185in}}%
\pgfpathlineto{\pgfqpoint{5.139375in}{0.916328in}}%
\pgfpathlineto{\pgfqpoint{5.147107in}{0.917967in}}%
\pgfpathlineto{\pgfqpoint{5.159993in}{0.919258in}}%
\pgfpathlineto{\pgfqpoint{5.165148in}{0.920040in}}%
\pgfpathlineto{\pgfqpoint{5.178034in}{0.921173in}}%
\pgfpathlineto{\pgfqpoint{5.183188in}{0.921930in}}%
\pgfpathlineto{\pgfqpoint{5.198651in}{0.923193in}}%
\pgfpathlineto{\pgfqpoint{5.219269in}{0.924745in}}%
\pgfpathlineto{\pgfqpoint{5.234732in}{0.925727in}}%
\pgfpathlineto{\pgfqpoint{5.237309in}{0.926007in}}%
\pgfpathlineto{\pgfqpoint{5.252773in}{0.927015in}}%
\pgfpathlineto{\pgfqpoint{5.273390in}{0.928181in}}%
\pgfpathlineto{\pgfqpoint{5.288854in}{0.928888in}}%
\pgfpathlineto{\pgfqpoint{5.304317in}{0.929758in}}%
\pgfpathlineto{\pgfqpoint{5.319780in}{0.930468in}}%
\pgfpathlineto{\pgfqpoint{5.358438in}{0.931653in}}%
\pgfpathlineto{\pgfqpoint{5.389365in}{0.931663in}}%
\pgfpathlineto{\pgfqpoint{5.453795in}{0.931432in}}%
\pgfpathlineto{\pgfqpoint{5.497608in}{0.932770in}}%
\pgfpathlineto{\pgfqpoint{5.525957in}{0.935011in}}%
\pgfpathlineto{\pgfqpoint{5.541421in}{0.935937in}}%
\pgfpathlineto{\pgfqpoint{5.543998in}{0.936224in}}%
\pgfpathlineto{\pgfqpoint{5.556884in}{0.936986in}}%
\pgfpathlineto{\pgfqpoint{5.562038in}{0.937690in}}%
\pgfpathlineto{\pgfqpoint{5.577501in}{0.938750in}}%
\pgfpathlineto{\pgfqpoint{5.580079in}{0.939005in}}%
\pgfpathlineto{\pgfqpoint{5.598119in}{0.940005in}}%
\pgfpathlineto{\pgfqpoint{5.634200in}{0.940603in}}%
\pgfpathlineto{\pgfqpoint{5.696053in}{0.941056in}}%
\pgfpathlineto{\pgfqpoint{5.778524in}{0.942324in}}%
\pgfpathlineto{\pgfqpoint{5.868727in}{0.942140in}}%
\pgfpathlineto{\pgfqpoint{5.912539in}{0.941333in}}%
\pgfpathlineto{\pgfqpoint{5.953775in}{0.940618in}}%
\pgfpathlineto{\pgfqpoint{6.085212in}{0.937878in}}%
\pgfpathlineto{\pgfqpoint{6.245000in}{0.939179in}}%
\pgfpathlineto{\pgfqpoint{6.291389in}{0.940050in}}%
\pgfpathlineto{\pgfqpoint{6.348088in}{0.940686in}}%
\pgfpathlineto{\pgfqpoint{6.399632in}{0.940051in}}%
\pgfpathlineto{\pgfqpoint{6.422827in}{0.941173in}}%
\pgfpathlineto{\pgfqpoint{6.456331in}{0.943534in}}%
\pgfpathlineto{\pgfqpoint{6.464063in}{0.944445in}}%
\pgfpathlineto{\pgfqpoint{6.479526in}{0.945416in}}%
\pgfpathlineto{\pgfqpoint{6.482103in}{0.945709in}}%
\pgfpathlineto{\pgfqpoint{6.482103in}{0.945709in}}%
\pgfusepath{stroke}%
\end{pgfscope}%
\begin{pgfscope}%
\pgfpathrectangle{\pgfqpoint{0.563921in}{0.521603in}}{\pgfqpoint{6.200000in}{2.642500in}}%
\pgfusepath{clip}%
\pgfsetroundcap%
\pgfsetroundjoin%
\pgfsetlinewidth{1.505625pt}%
\definecolor{currentstroke}{rgb}{0.737255,0.741176,0.133333}%
\pgfsetstrokecolor{currentstroke}%
\pgfsetdash{}{0pt}%
\pgfpathmoveto{\pgfqpoint{0.845739in}{0.641717in}}%
\pgfpathlineto{\pgfqpoint{0.848317in}{0.653933in}}%
\pgfpathlineto{\pgfqpoint{0.850894in}{0.651739in}}%
\pgfpathlineto{\pgfqpoint{0.853471in}{0.653065in}}%
\pgfpathlineto{\pgfqpoint{0.863780in}{0.656339in}}%
\pgfpathlineto{\pgfqpoint{0.866357in}{0.657935in}}%
\pgfpathlineto{\pgfqpoint{0.871512in}{0.656214in}}%
\pgfpathlineto{\pgfqpoint{0.881820in}{0.656987in}}%
\pgfpathlineto{\pgfqpoint{0.884398in}{0.659578in}}%
\pgfpathlineto{\pgfqpoint{0.889552in}{0.658794in}}%
\pgfpathlineto{\pgfqpoint{0.897284in}{0.659164in}}%
\pgfpathlineto{\pgfqpoint{0.907593in}{0.657250in}}%
\pgfpathlineto{\pgfqpoint{0.920479in}{0.657087in}}%
\pgfpathlineto{\pgfqpoint{0.925633in}{0.666625in}}%
\pgfpathlineto{\pgfqpoint{0.933365in}{0.670374in}}%
\pgfpathlineto{\pgfqpoint{0.935942in}{0.672749in}}%
\pgfpathlineto{\pgfqpoint{0.938519in}{0.676044in}}%
\pgfpathlineto{\pgfqpoint{0.943674in}{0.691156in}}%
\pgfpathlineto{\pgfqpoint{0.951405in}{0.695499in}}%
\pgfpathlineto{\pgfqpoint{0.956560in}{0.706877in}}%
\pgfpathlineto{\pgfqpoint{0.961714in}{0.713774in}}%
\pgfpathlineto{\pgfqpoint{0.972023in}{0.716230in}}%
\pgfpathlineto{\pgfqpoint{0.979754in}{0.726012in}}%
\pgfpathlineto{\pgfqpoint{0.987486in}{0.728351in}}%
\pgfpathlineto{\pgfqpoint{0.990063in}{0.731584in}}%
\pgfpathlineto{\pgfqpoint{0.997795in}{0.735445in}}%
\pgfpathlineto{\pgfqpoint{1.010681in}{0.736817in}}%
\pgfpathlineto{\pgfqpoint{1.015835in}{0.738670in}}%
\pgfpathlineto{\pgfqpoint{1.026144in}{0.739602in}}%
\pgfpathlineto{\pgfqpoint{1.041608in}{0.741138in}}%
\pgfpathlineto{\pgfqpoint{1.049339in}{0.741326in}}%
\pgfpathlineto{\pgfqpoint{1.051916in}{0.741812in}}%
\pgfpathlineto{\pgfqpoint{1.085420in}{0.745347in}}%
\pgfpathlineto{\pgfqpoint{1.098306in}{0.745152in}}%
\pgfpathlineto{\pgfqpoint{1.100883in}{0.744887in}}%
\pgfpathlineto{\pgfqpoint{1.118924in}{0.747420in}}%
\pgfpathlineto{\pgfqpoint{1.124078in}{0.747832in}}%
\pgfpathlineto{\pgfqpoint{1.136964in}{0.747585in}}%
\pgfpathlineto{\pgfqpoint{1.142119in}{0.748648in}}%
\pgfpathlineto{\pgfqpoint{1.157582in}{0.749064in}}%
\pgfpathlineto{\pgfqpoint{1.160159in}{0.748650in}}%
\pgfpathlineto{\pgfqpoint{1.170468in}{0.747928in}}%
\pgfpathlineto{\pgfqpoint{1.178200in}{0.746731in}}%
\pgfpathlineto{\pgfqpoint{1.191086in}{0.745452in}}%
\pgfpathlineto{\pgfqpoint{1.196240in}{0.744416in}}%
\pgfpathlineto{\pgfqpoint{1.240053in}{0.741590in}}%
\pgfpathlineto{\pgfqpoint{1.250362in}{0.740051in}}%
\pgfpathlineto{\pgfqpoint{1.263248in}{0.738963in}}%
\pgfpathlineto{\pgfqpoint{1.268402in}{0.738400in}}%
\pgfpathlineto{\pgfqpoint{1.283866in}{0.738495in}}%
\pgfpathlineto{\pgfqpoint{1.286443in}{0.739059in}}%
\pgfpathlineto{\pgfqpoint{1.301906in}{0.739627in}}%
\pgfpathlineto{\pgfqpoint{1.319947in}{0.741993in}}%
\pgfpathlineto{\pgfqpoint{1.322524in}{0.742409in}}%
\pgfpathlineto{\pgfqpoint{1.348296in}{0.743319in}}%
\pgfpathlineto{\pgfqpoint{1.356027in}{0.745027in}}%
\pgfpathlineto{\pgfqpoint{1.358605in}{0.745347in}}%
\pgfpathlineto{\pgfqpoint{1.374068in}{0.745779in}}%
\pgfpathlineto{\pgfqpoint{1.376645in}{0.746511in}}%
\pgfpathlineto{\pgfqpoint{1.389531in}{0.748565in}}%
\pgfpathlineto{\pgfqpoint{1.394686in}{0.750083in}}%
\pgfpathlineto{\pgfqpoint{1.404995in}{0.751653in}}%
\pgfpathlineto{\pgfqpoint{1.410149in}{0.752868in}}%
\pgfpathlineto{\pgfqpoint{1.412726in}{0.753334in}}%
\pgfpathlineto{\pgfqpoint{1.423035in}{0.754221in}}%
\pgfpathlineto{\pgfqpoint{1.430767in}{0.755761in}}%
\pgfpathlineto{\pgfqpoint{1.459116in}{0.757492in}}%
\pgfpathlineto{\pgfqpoint{1.466848in}{0.758128in}}%
\pgfpathlineto{\pgfqpoint{1.482311in}{0.758944in}}%
\pgfpathlineto{\pgfqpoint{1.484888in}{0.759313in}}%
\pgfpathlineto{\pgfqpoint{1.495197in}{0.760112in}}%
\pgfpathlineto{\pgfqpoint{1.502929in}{0.762717in}}%
\pgfpathlineto{\pgfqpoint{1.513237in}{0.764189in}}%
\pgfpathlineto{\pgfqpoint{1.520969in}{0.766611in}}%
\pgfpathlineto{\pgfqpoint{1.531278in}{0.767940in}}%
\pgfpathlineto{\pgfqpoint{1.549318in}{0.771256in}}%
\pgfpathlineto{\pgfqpoint{1.557050in}{0.774829in}}%
\pgfpathlineto{\pgfqpoint{1.567359in}{0.776559in}}%
\pgfpathlineto{\pgfqpoint{1.575091in}{0.779255in}}%
\pgfpathlineto{\pgfqpoint{1.582822in}{0.780275in}}%
\pgfpathlineto{\pgfqpoint{1.593131in}{0.784821in}}%
\pgfpathlineto{\pgfqpoint{1.606017in}{0.786517in}}%
\pgfpathlineto{\pgfqpoint{1.611172in}{0.787670in}}%
\pgfpathlineto{\pgfqpoint{1.624058in}{0.788308in}}%
\pgfpathlineto{\pgfqpoint{1.629212in}{0.790676in}}%
\pgfpathlineto{\pgfqpoint{1.636944in}{0.791549in}}%
\pgfpathlineto{\pgfqpoint{1.644675in}{0.794474in}}%
\pgfpathlineto{\pgfqpoint{1.647253in}{0.795409in}}%
\pgfpathlineto{\pgfqpoint{1.657561in}{0.797183in}}%
\pgfpathlineto{\pgfqpoint{1.675602in}{0.802070in}}%
\pgfpathlineto{\pgfqpoint{1.678179in}{0.803401in}}%
\pgfpathlineto{\pgfqpoint{1.683333in}{0.804886in}}%
\pgfpathlineto{\pgfqpoint{1.691065in}{0.806213in}}%
\pgfpathlineto{\pgfqpoint{1.701374in}{0.811544in}}%
\pgfpathlineto{\pgfqpoint{1.709106in}{0.812866in}}%
\pgfpathlineto{\pgfqpoint{1.719414in}{0.817589in}}%
\pgfpathlineto{\pgfqpoint{1.727146in}{0.818748in}}%
\pgfpathlineto{\pgfqpoint{1.734878in}{0.821836in}}%
\pgfpathlineto{\pgfqpoint{1.737455in}{0.822689in}}%
\pgfpathlineto{\pgfqpoint{1.745187in}{0.823766in}}%
\pgfpathlineto{\pgfqpoint{1.755495in}{0.828519in}}%
\pgfpathlineto{\pgfqpoint{1.768381in}{0.830652in}}%
\pgfpathlineto{\pgfqpoint{1.773536in}{0.832395in}}%
\pgfpathlineto{\pgfqpoint{1.781268in}{0.833491in}}%
\pgfpathlineto{\pgfqpoint{1.786422in}{0.834982in}}%
\pgfpathlineto{\pgfqpoint{1.791576in}{0.838028in}}%
\pgfpathlineto{\pgfqpoint{1.799308in}{0.839702in}}%
\pgfpathlineto{\pgfqpoint{1.809617in}{0.847322in}}%
\pgfpathlineto{\pgfqpoint{1.817349in}{0.849095in}}%
\pgfpathlineto{\pgfqpoint{1.827657in}{0.855340in}}%
\pgfpathlineto{\pgfqpoint{1.837966in}{0.856736in}}%
\pgfpathlineto{\pgfqpoint{1.845698in}{0.860869in}}%
\pgfpathlineto{\pgfqpoint{1.853430in}{0.861876in}}%
\pgfpathlineto{\pgfqpoint{1.863738in}{0.865943in}}%
\pgfpathlineto{\pgfqpoint{1.871470in}{0.866869in}}%
\pgfpathlineto{\pgfqpoint{1.881779in}{0.871137in}}%
\pgfpathlineto{\pgfqpoint{1.892088in}{0.872748in}}%
\pgfpathlineto{\pgfqpoint{1.899819in}{0.875289in}}%
\pgfpathlineto{\pgfqpoint{1.912705in}{0.876939in}}%
\pgfpathlineto{\pgfqpoint{1.917860in}{0.878837in}}%
\pgfpathlineto{\pgfqpoint{1.925591in}{0.879547in}}%
\pgfpathlineto{\pgfqpoint{1.935900in}{0.883099in}}%
\pgfpathlineto{\pgfqpoint{1.943632in}{0.884016in}}%
\pgfpathlineto{\pgfqpoint{1.953941in}{0.887863in}}%
\pgfpathlineto{\pgfqpoint{1.964250in}{0.889653in}}%
\pgfpathlineto{\pgfqpoint{1.971981in}{0.891982in}}%
\pgfpathlineto{\pgfqpoint{1.982290in}{0.893105in}}%
\pgfpathlineto{\pgfqpoint{1.990022in}{0.895126in}}%
\pgfpathlineto{\pgfqpoint{1.997753in}{0.896148in}}%
\pgfpathlineto{\pgfqpoint{2.005485in}{0.900288in}}%
\pgfpathlineto{\pgfqpoint{2.015794in}{0.901499in}}%
\pgfpathlineto{\pgfqpoint{2.023526in}{0.904759in}}%
\pgfpathlineto{\pgfqpoint{2.026103in}{0.905705in}}%
\pgfpathlineto{\pgfqpoint{2.036412in}{0.907603in}}%
\pgfpathlineto{\pgfqpoint{2.044143in}{0.910604in}}%
\pgfpathlineto{\pgfqpoint{2.051875in}{0.911178in}}%
\pgfpathlineto{\pgfqpoint{2.062184in}{0.913833in}}%
\pgfpathlineto{\pgfqpoint{2.069915in}{0.914496in}}%
\pgfpathlineto{\pgfqpoint{2.080224in}{0.918021in}}%
\pgfpathlineto{\pgfqpoint{2.087956in}{0.918913in}}%
\pgfpathlineto{\pgfqpoint{2.095687in}{0.921988in}}%
\pgfpathlineto{\pgfqpoint{2.098265in}{0.923757in}}%
\pgfpathlineto{\pgfqpoint{2.105996in}{0.925433in}}%
\pgfpathlineto{\pgfqpoint{2.116305in}{0.931892in}}%
\pgfpathlineto{\pgfqpoint{2.124037in}{0.933419in}}%
\pgfpathlineto{\pgfqpoint{2.134346in}{0.940512in}}%
\pgfpathlineto{\pgfqpoint{2.142077in}{0.942226in}}%
\pgfpathlineto{\pgfqpoint{2.149809in}{0.946771in}}%
\pgfpathlineto{\pgfqpoint{2.152386in}{0.948273in}}%
\pgfpathlineto{\pgfqpoint{2.162695in}{0.949713in}}%
\pgfpathlineto{\pgfqpoint{2.170427in}{0.953676in}}%
\pgfpathlineto{\pgfqpoint{2.178158in}{0.955034in}}%
\pgfpathlineto{\pgfqpoint{2.188467in}{0.959941in}}%
\pgfpathlineto{\pgfqpoint{2.196199in}{0.961351in}}%
\pgfpathlineto{\pgfqpoint{2.206508in}{0.966224in}}%
\pgfpathlineto{\pgfqpoint{2.214239in}{0.967524in}}%
\pgfpathlineto{\pgfqpoint{2.221971in}{0.971174in}}%
\pgfpathlineto{\pgfqpoint{2.224548in}{0.972195in}}%
\pgfpathlineto{\pgfqpoint{2.232280in}{0.973120in}}%
\pgfpathlineto{\pgfqpoint{2.242589in}{0.977792in}}%
\pgfpathlineto{\pgfqpoint{2.250320in}{0.979099in}}%
\pgfpathlineto{\pgfqpoint{2.255475in}{0.981785in}}%
\pgfpathlineto{\pgfqpoint{2.260629in}{0.983441in}}%
\pgfpathlineto{\pgfqpoint{2.268361in}{0.984869in}}%
\pgfpathlineto{\pgfqpoint{2.278670in}{0.990512in}}%
\pgfpathlineto{\pgfqpoint{2.286401in}{0.991981in}}%
\pgfpathlineto{\pgfqpoint{2.296710in}{0.997548in}}%
\pgfpathlineto{\pgfqpoint{2.304442in}{0.998971in}}%
\pgfpathlineto{\pgfqpoint{2.314751in}{1.004358in}}%
\pgfpathlineto{\pgfqpoint{2.322482in}{1.005725in}}%
\pgfpathlineto{\pgfqpoint{2.325059in}{1.007035in}}%
\pgfpathlineto{\pgfqpoint{2.366295in}{1.013948in}}%
\pgfpathlineto{\pgfqpoint{2.404953in}{1.017602in}}%
\pgfpathlineto{\pgfqpoint{2.420416in}{1.018583in}}%
\pgfpathlineto{\pgfqpoint{2.422993in}{1.018908in}}%
\pgfpathlineto{\pgfqpoint{2.433302in}{1.019995in}}%
\pgfpathlineto{\pgfqpoint{2.441034in}{1.022342in}}%
\pgfpathlineto{\pgfqpoint{2.448766in}{1.023241in}}%
\pgfpathlineto{\pgfqpoint{2.456497in}{1.026545in}}%
\pgfpathlineto{\pgfqpoint{2.459074in}{1.027954in}}%
\pgfpathlineto{\pgfqpoint{2.466806in}{1.029181in}}%
\pgfpathlineto{\pgfqpoint{2.477115in}{1.033142in}}%
\pgfpathlineto{\pgfqpoint{2.484847in}{1.034001in}}%
\pgfpathlineto{\pgfqpoint{2.495155in}{1.037317in}}%
\pgfpathlineto{\pgfqpoint{2.508041in}{1.038716in}}%
\pgfpathlineto{\pgfqpoint{2.513196in}{1.040212in}}%
\pgfpathlineto{\pgfqpoint{2.523505in}{1.041848in}}%
\pgfpathlineto{\pgfqpoint{2.531236in}{1.045142in}}%
\pgfpathlineto{\pgfqpoint{2.538968in}{1.046318in}}%
\pgfpathlineto{\pgfqpoint{2.549277in}{1.051140in}}%
\pgfpathlineto{\pgfqpoint{2.557009in}{1.052413in}}%
\pgfpathlineto{\pgfqpoint{2.564740in}{1.055893in}}%
\pgfpathlineto{\pgfqpoint{2.567317in}{1.056886in}}%
\pgfpathlineto{\pgfqpoint{2.575049in}{1.057724in}}%
\pgfpathlineto{\pgfqpoint{2.585358in}{1.061223in}}%
\pgfpathlineto{\pgfqpoint{2.595667in}{1.063014in}}%
\pgfpathlineto{\pgfqpoint{2.603398in}{1.066155in}}%
\pgfpathlineto{\pgfqpoint{2.613707in}{1.067978in}}%
\pgfpathlineto{\pgfqpoint{2.621439in}{1.070931in}}%
\pgfpathlineto{\pgfqpoint{2.629170in}{1.071958in}}%
\pgfpathlineto{\pgfqpoint{2.634325in}{1.074166in}}%
\pgfpathlineto{\pgfqpoint{2.652365in}{1.078317in}}%
\pgfpathlineto{\pgfqpoint{2.657520in}{1.080165in}}%
\pgfpathlineto{\pgfqpoint{2.667829in}{1.081846in}}%
\pgfpathlineto{\pgfqpoint{2.675560in}{1.085030in}}%
\pgfpathlineto{\pgfqpoint{2.683292in}{1.086181in}}%
\pgfpathlineto{\pgfqpoint{2.693601in}{1.092281in}}%
\pgfpathlineto{\pgfqpoint{2.701332in}{1.093925in}}%
\pgfpathlineto{\pgfqpoint{2.703910in}{1.095592in}}%
\pgfpathlineto{\pgfqpoint{2.709064in}{1.097325in}}%
\pgfpathlineto{\pgfqpoint{2.711641in}{1.099003in}}%
\pgfpathlineto{\pgfqpoint{2.719373in}{1.100734in}}%
\pgfpathlineto{\pgfqpoint{2.721950in}{1.102553in}}%
\pgfpathlineto{\pgfqpoint{2.727105in}{1.104247in}}%
\pgfpathlineto{\pgfqpoint{2.729682in}{1.105928in}}%
\pgfpathlineto{\pgfqpoint{2.737413in}{1.107510in}}%
\pgfpathlineto{\pgfqpoint{2.747722in}{1.114073in}}%
\pgfpathlineto{\pgfqpoint{2.755454in}{1.115507in}}%
\pgfpathlineto{\pgfqpoint{2.765763in}{1.122478in}}%
\pgfpathlineto{\pgfqpoint{2.776072in}{1.124572in}}%
\pgfpathlineto{\pgfqpoint{2.781226in}{1.128526in}}%
\pgfpathlineto{\pgfqpoint{2.783803in}{1.129929in}}%
\pgfpathlineto{\pgfqpoint{2.791535in}{1.131059in}}%
\pgfpathlineto{\pgfqpoint{2.801844in}{1.135927in}}%
\pgfpathlineto{\pgfqpoint{2.809575in}{1.136862in}}%
\pgfpathlineto{\pgfqpoint{2.819884in}{1.141300in}}%
\pgfpathlineto{\pgfqpoint{2.827616in}{1.142511in}}%
\pgfpathlineto{\pgfqpoint{2.837925in}{1.148185in}}%
\pgfpathlineto{\pgfqpoint{2.848234in}{1.149684in}}%
\pgfpathlineto{\pgfqpoint{2.855965in}{1.153694in}}%
\pgfpathlineto{\pgfqpoint{2.863697in}{1.155135in}}%
\pgfpathlineto{\pgfqpoint{2.874006in}{1.160768in}}%
\pgfpathlineto{\pgfqpoint{2.881737in}{1.161896in}}%
\pgfpathlineto{\pgfqpoint{2.892046in}{1.166852in}}%
\pgfpathlineto{\pgfqpoint{2.899778in}{1.168109in}}%
\pgfpathlineto{\pgfqpoint{2.907509in}{1.171688in}}%
\pgfpathlineto{\pgfqpoint{2.910087in}{1.172673in}}%
\pgfpathlineto{\pgfqpoint{2.917818in}{1.173783in}}%
\pgfpathlineto{\pgfqpoint{2.928127in}{1.178124in}}%
\pgfpathlineto{\pgfqpoint{2.938436in}{1.179856in}}%
\pgfpathlineto{\pgfqpoint{2.946168in}{1.181786in}}%
\pgfpathlineto{\pgfqpoint{2.956476in}{1.183094in}}%
\pgfpathlineto{\pgfqpoint{2.964208in}{1.184702in}}%
\pgfpathlineto{\pgfqpoint{2.992557in}{1.186024in}}%
\pgfpathlineto{\pgfqpoint{2.997712in}{1.186747in}}%
\pgfpathlineto{\pgfqpoint{3.010598in}{1.187511in}}%
\pgfpathlineto{\pgfqpoint{3.018330in}{1.188289in}}%
\pgfpathlineto{\pgfqpoint{3.033793in}{1.188966in}}%
\pgfpathlineto{\pgfqpoint{3.069874in}{1.192177in}}%
\pgfpathlineto{\pgfqpoint{3.072451in}{1.192543in}}%
\pgfpathlineto{\pgfqpoint{3.085337in}{1.193587in}}%
\pgfpathlineto{\pgfqpoint{3.090491in}{1.194342in}}%
\pgfpathlineto{\pgfqpoint{3.103378in}{1.195300in}}%
\pgfpathlineto{\pgfqpoint{3.108532in}{1.196277in}}%
\pgfpathlineto{\pgfqpoint{3.121418in}{1.197449in}}%
\pgfpathlineto{\pgfqpoint{3.126572in}{1.198241in}}%
\pgfpathlineto{\pgfqpoint{3.136881in}{1.199067in}}%
\pgfpathlineto{\pgfqpoint{3.144613in}{1.200132in}}%
\pgfpathlineto{\pgfqpoint{3.157499in}{1.201061in}}%
\pgfpathlineto{\pgfqpoint{3.162653in}{1.201611in}}%
\pgfpathlineto{\pgfqpoint{3.178117in}{1.202550in}}%
\pgfpathlineto{\pgfqpoint{3.227084in}{1.207492in}}%
\pgfpathlineto{\pgfqpoint{3.234815in}{1.209044in}}%
\pgfpathlineto{\pgfqpoint{3.245124in}{1.210053in}}%
\pgfpathlineto{\pgfqpoint{3.252856in}{1.211475in}}%
\pgfpathlineto{\pgfqpoint{3.268319in}{1.212551in}}%
\pgfpathlineto{\pgfqpoint{3.283782in}{1.213374in}}%
\pgfpathlineto{\pgfqpoint{3.322441in}{1.216036in}}%
\pgfpathlineto{\pgfqpoint{3.325018in}{1.216377in}}%
\pgfpathlineto{\pgfqpoint{3.337904in}{1.217427in}}%
\pgfpathlineto{\pgfqpoint{3.343058in}{1.217940in}}%
\pgfpathlineto{\pgfqpoint{3.358522in}{1.218774in}}%
\pgfpathlineto{\pgfqpoint{3.379139in}{1.220421in}}%
\pgfpathlineto{\pgfqpoint{3.392025in}{1.221269in}}%
\pgfpathlineto{\pgfqpoint{3.397180in}{1.221847in}}%
\pgfpathlineto{\pgfqpoint{3.430684in}{1.223309in}}%
\pgfpathlineto{\pgfqpoint{3.448724in}{1.223868in}}%
\pgfpathlineto{\pgfqpoint{3.484805in}{1.224118in}}%
\pgfpathlineto{\pgfqpoint{3.502845in}{1.225677in}}%
\pgfpathlineto{\pgfqpoint{3.505423in}{1.226664in}}%
\pgfpathlineto{\pgfqpoint{3.513154in}{1.227646in}}%
\pgfpathlineto{\pgfqpoint{3.520886in}{1.231226in}}%
\pgfpathlineto{\pgfqpoint{3.523463in}{1.232593in}}%
\pgfpathlineto{\pgfqpoint{3.531195in}{1.233857in}}%
\pgfpathlineto{\pgfqpoint{3.541504in}{1.239067in}}%
\pgfpathlineto{\pgfqpoint{3.549235in}{1.240338in}}%
\pgfpathlineto{\pgfqpoint{3.559544in}{1.245565in}}%
\pgfpathlineto{\pgfqpoint{3.567276in}{1.246975in}}%
\pgfpathlineto{\pgfqpoint{3.572430in}{1.249976in}}%
\pgfpathlineto{\pgfqpoint{3.577585in}{1.251496in}}%
\pgfpathlineto{\pgfqpoint{3.585316in}{1.253000in}}%
\pgfpathlineto{\pgfqpoint{3.595625in}{1.259456in}}%
\pgfpathlineto{\pgfqpoint{3.603357in}{1.261101in}}%
\pgfpathlineto{\pgfqpoint{3.613666in}{1.267274in}}%
\pgfpathlineto{\pgfqpoint{3.621397in}{1.268593in}}%
\pgfpathlineto{\pgfqpoint{3.626552in}{1.271167in}}%
\pgfpathlineto{\pgfqpoint{3.631706in}{1.274223in}}%
\pgfpathlineto{\pgfqpoint{3.639438in}{1.275784in}}%
\pgfpathlineto{\pgfqpoint{3.644592in}{1.279052in}}%
\pgfpathlineto{\pgfqpoint{3.649747in}{1.280672in}}%
\pgfpathlineto{\pgfqpoint{3.657478in}{1.282231in}}%
\pgfpathlineto{\pgfqpoint{3.662633in}{1.285142in}}%
\pgfpathlineto{\pgfqpoint{3.667787in}{1.286646in}}%
\pgfpathlineto{\pgfqpoint{3.675519in}{1.287909in}}%
\pgfpathlineto{\pgfqpoint{3.685828in}{1.293118in}}%
\pgfpathlineto{\pgfqpoint{3.693559in}{1.294361in}}%
\pgfpathlineto{\pgfqpoint{3.701291in}{1.297658in}}%
\pgfpathlineto{\pgfqpoint{3.703868in}{1.298684in}}%
\pgfpathlineto{\pgfqpoint{3.714177in}{1.299768in}}%
\pgfpathlineto{\pgfqpoint{3.721909in}{1.303060in}}%
\pgfpathlineto{\pgfqpoint{3.732217in}{1.304946in}}%
\pgfpathlineto{\pgfqpoint{3.739949in}{1.307341in}}%
\pgfpathlineto{\pgfqpoint{3.747681in}{1.308304in}}%
\pgfpathlineto{\pgfqpoint{3.755412in}{1.312222in}}%
\pgfpathlineto{\pgfqpoint{3.757990in}{1.313574in}}%
\pgfpathlineto{\pgfqpoint{3.765721in}{1.314842in}}%
\pgfpathlineto{\pgfqpoint{3.773453in}{1.318925in}}%
\pgfpathlineto{\pgfqpoint{3.776030in}{1.320380in}}%
\pgfpathlineto{\pgfqpoint{3.786339in}{1.321871in}}%
\pgfpathlineto{\pgfqpoint{3.794070in}{1.326222in}}%
\pgfpathlineto{\pgfqpoint{3.801802in}{1.327749in}}%
\pgfpathlineto{\pgfqpoint{3.812111in}{1.333680in}}%
\pgfpathlineto{\pgfqpoint{3.819843in}{1.335353in}}%
\pgfpathlineto{\pgfqpoint{3.830151in}{1.341127in}}%
\pgfpathlineto{\pgfqpoint{3.837883in}{1.342478in}}%
\pgfpathlineto{\pgfqpoint{3.848192in}{1.347047in}}%
\pgfpathlineto{\pgfqpoint{3.855924in}{1.348266in}}%
\pgfpathlineto{\pgfqpoint{3.866232in}{1.352813in}}%
\pgfpathlineto{\pgfqpoint{3.873964in}{1.353964in}}%
\pgfpathlineto{\pgfqpoint{3.881696in}{1.356998in}}%
\pgfpathlineto{\pgfqpoint{3.884273in}{1.357918in}}%
\pgfpathlineto{\pgfqpoint{3.892005in}{1.358846in}}%
\pgfpathlineto{\pgfqpoint{3.899736in}{1.361454in}}%
\pgfpathlineto{\pgfqpoint{3.910045in}{1.362299in}}%
\pgfpathlineto{\pgfqpoint{3.920354in}{1.366140in}}%
\pgfpathlineto{\pgfqpoint{3.930663in}{1.367849in}}%
\pgfpathlineto{\pgfqpoint{3.938394in}{1.370267in}}%
\pgfpathlineto{\pgfqpoint{3.948703in}{1.371793in}}%
\pgfpathlineto{\pgfqpoint{3.956435in}{1.375017in}}%
\pgfpathlineto{\pgfqpoint{3.964167in}{1.375996in}}%
\pgfpathlineto{\pgfqpoint{3.974475in}{1.379547in}}%
\pgfpathlineto{\pgfqpoint{3.984784in}{1.381118in}}%
\pgfpathlineto{\pgfqpoint{3.992516in}{1.383971in}}%
\pgfpathlineto{\pgfqpoint{4.000247in}{1.385124in}}%
\pgfpathlineto{\pgfqpoint{4.010556in}{1.389812in}}%
\pgfpathlineto{\pgfqpoint{4.018288in}{1.391068in}}%
\pgfpathlineto{\pgfqpoint{4.028597in}{1.395862in}}%
\pgfpathlineto{\pgfqpoint{4.038906in}{1.396888in}}%
\pgfpathlineto{\pgfqpoint{4.046637in}{1.400186in}}%
\pgfpathlineto{\pgfqpoint{4.054369in}{1.401253in}}%
\pgfpathlineto{\pgfqpoint{4.064678in}{1.405222in}}%
\pgfpathlineto{\pgfqpoint{4.072409in}{1.406083in}}%
\pgfpathlineto{\pgfqpoint{4.082718in}{1.410157in}}%
\pgfpathlineto{\pgfqpoint{4.090450in}{1.411089in}}%
\pgfpathlineto{\pgfqpoint{4.100759in}{1.414950in}}%
\pgfpathlineto{\pgfqpoint{4.108490in}{1.415911in}}%
\pgfpathlineto{\pgfqpoint{4.118799in}{1.419607in}}%
\pgfpathlineto{\pgfqpoint{4.129108in}{1.420982in}}%
\pgfpathlineto{\pgfqpoint{4.134263in}{1.422576in}}%
\pgfpathlineto{\pgfqpoint{4.147149in}{1.424126in}}%
\pgfpathlineto{\pgfqpoint{4.154880in}{1.426244in}}%
\pgfpathlineto{\pgfqpoint{4.162612in}{1.427171in}}%
\pgfpathlineto{\pgfqpoint{4.172921in}{1.431220in}}%
\pgfpathlineto{\pgfqpoint{4.180652in}{1.432497in}}%
\pgfpathlineto{\pgfqpoint{4.190961in}{1.437500in}}%
\pgfpathlineto{\pgfqpoint{4.198693in}{1.438882in}}%
\pgfpathlineto{\pgfqpoint{4.209002in}{1.445252in}}%
\pgfpathlineto{\pgfqpoint{4.216733in}{1.446837in}}%
\pgfpathlineto{\pgfqpoint{4.227042in}{1.452405in}}%
\pgfpathlineto{\pgfqpoint{4.234774in}{1.453756in}}%
\pgfpathlineto{\pgfqpoint{4.245083in}{1.458807in}}%
\pgfpathlineto{\pgfqpoint{4.252814in}{1.460134in}}%
\pgfpathlineto{\pgfqpoint{4.263123in}{1.464922in}}%
\pgfpathlineto{\pgfqpoint{4.273432in}{1.466017in}}%
\pgfpathlineto{\pgfqpoint{4.281164in}{1.468915in}}%
\pgfpathlineto{\pgfqpoint{4.291472in}{1.470452in}}%
\pgfpathlineto{\pgfqpoint{4.299204in}{1.472590in}}%
\pgfpathlineto{\pgfqpoint{4.312090in}{1.474061in}}%
\pgfpathlineto{\pgfqpoint{4.317245in}{1.475598in}}%
\pgfpathlineto{\pgfqpoint{4.327553in}{1.477069in}}%
\pgfpathlineto{\pgfqpoint{4.335285in}{1.479337in}}%
\pgfpathlineto{\pgfqpoint{4.345594in}{1.480811in}}%
\pgfpathlineto{\pgfqpoint{4.353326in}{1.483020in}}%
\pgfpathlineto{\pgfqpoint{4.363634in}{1.483953in}}%
\pgfpathlineto{\pgfqpoint{4.371366in}{1.485945in}}%
\pgfpathlineto{\pgfqpoint{4.381675in}{1.487662in}}%
\pgfpathlineto{\pgfqpoint{4.389407in}{1.490691in}}%
\pgfpathlineto{\pgfqpoint{4.397138in}{1.491832in}}%
\pgfpathlineto{\pgfqpoint{4.407447in}{1.496394in}}%
\pgfpathlineto{\pgfqpoint{4.415179in}{1.497732in}}%
\pgfpathlineto{\pgfqpoint{4.425488in}{1.502684in}}%
\pgfpathlineto{\pgfqpoint{4.433219in}{1.504126in}}%
\pgfpathlineto{\pgfqpoint{4.443528in}{1.509776in}}%
\pgfpathlineto{\pgfqpoint{4.451260in}{1.510830in}}%
\pgfpathlineto{\pgfqpoint{4.461569in}{1.516521in}}%
\pgfpathlineto{\pgfqpoint{4.469300in}{1.517857in}}%
\pgfpathlineto{\pgfqpoint{4.479609in}{1.523534in}}%
\pgfpathlineto{\pgfqpoint{4.487341in}{1.524946in}}%
\pgfpathlineto{\pgfqpoint{4.497649in}{1.530996in}}%
\pgfpathlineto{\pgfqpoint{4.505381in}{1.532509in}}%
\pgfpathlineto{\pgfqpoint{4.510536in}{1.535400in}}%
\pgfpathlineto{\pgfqpoint{4.515690in}{1.536851in}}%
\pgfpathlineto{\pgfqpoint{4.523422in}{1.538195in}}%
\pgfpathlineto{\pgfqpoint{4.533730in}{1.543678in}}%
\pgfpathlineto{\pgfqpoint{4.541462in}{1.545043in}}%
\pgfpathlineto{\pgfqpoint{4.551771in}{1.549638in}}%
\pgfpathlineto{\pgfqpoint{4.559503in}{1.550668in}}%
\pgfpathlineto{\pgfqpoint{4.569811in}{1.555381in}}%
\pgfpathlineto{\pgfqpoint{4.577543in}{1.556410in}}%
\pgfpathlineto{\pgfqpoint{4.585275in}{1.559745in}}%
\pgfpathlineto{\pgfqpoint{4.595584in}{1.560893in}}%
\pgfpathlineto{\pgfqpoint{4.603315in}{1.564245in}}%
\pgfpathlineto{\pgfqpoint{4.613624in}{1.565077in}}%
\pgfpathlineto{\pgfqpoint{4.621356in}{1.567382in}}%
\pgfpathlineto{\pgfqpoint{4.623933in}{1.567941in}}%
\pgfpathlineto{\pgfqpoint{4.634242in}{1.569304in}}%
\pgfpathlineto{\pgfqpoint{4.641973in}{1.570946in}}%
\pgfpathlineto{\pgfqpoint{4.654859in}{1.571721in}}%
\pgfpathlineto{\pgfqpoint{4.660014in}{1.572658in}}%
\pgfpathlineto{\pgfqpoint{4.672900in}{1.573894in}}%
\pgfpathlineto{\pgfqpoint{4.685786in}{1.575428in}}%
\pgfpathlineto{\pgfqpoint{4.696095in}{1.577566in}}%
\pgfpathlineto{\pgfqpoint{4.711558in}{1.578316in}}%
\pgfpathlineto{\pgfqpoint{4.714135in}{1.578624in}}%
\pgfpathlineto{\pgfqpoint{4.727021in}{1.579416in}}%
\pgfpathlineto{\pgfqpoint{4.732176in}{1.580179in}}%
\pgfpathlineto{\pgfqpoint{4.745062in}{1.581532in}}%
\pgfpathlineto{\pgfqpoint{4.750216in}{1.582521in}}%
\pgfpathlineto{\pgfqpoint{4.760525in}{1.583586in}}%
\pgfpathlineto{\pgfqpoint{4.768257in}{1.585283in}}%
\pgfpathlineto{\pgfqpoint{4.783720in}{1.586523in}}%
\pgfpathlineto{\pgfqpoint{4.786297in}{1.586883in}}%
\pgfpathlineto{\pgfqpoint{4.799183in}{1.588014in}}%
\pgfpathlineto{\pgfqpoint{4.804338in}{1.588992in}}%
\pgfpathlineto{\pgfqpoint{4.814647in}{1.589944in}}%
\pgfpathlineto{\pgfqpoint{4.819801in}{1.590989in}}%
\pgfpathlineto{\pgfqpoint{4.832687in}{1.592138in}}%
\pgfpathlineto{\pgfqpoint{4.840419in}{1.594406in}}%
\pgfpathlineto{\pgfqpoint{4.850728in}{1.595977in}}%
\pgfpathlineto{\pgfqpoint{4.858459in}{1.598447in}}%
\pgfpathlineto{\pgfqpoint{4.866191in}{1.599250in}}%
\pgfpathlineto{\pgfqpoint{4.876500in}{1.603104in}}%
\pgfpathlineto{\pgfqpoint{4.884231in}{1.604229in}}%
\pgfpathlineto{\pgfqpoint{4.894540in}{1.608278in}}%
\pgfpathlineto{\pgfqpoint{4.904849in}{1.609874in}}%
\pgfpathlineto{\pgfqpoint{4.912581in}{1.612121in}}%
\pgfpathlineto{\pgfqpoint{4.922890in}{1.613586in}}%
\pgfpathlineto{\pgfqpoint{4.930621in}{1.615638in}}%
\pgfpathlineto{\pgfqpoint{4.938353in}{1.616386in}}%
\pgfpathlineto{\pgfqpoint{4.948662in}{1.619236in}}%
\pgfpathlineto{\pgfqpoint{4.958971in}{1.620556in}}%
\pgfpathlineto{\pgfqpoint{4.966702in}{1.622517in}}%
\pgfpathlineto{\pgfqpoint{4.974434in}{1.623153in}}%
\pgfpathlineto{\pgfqpoint{4.984743in}{1.626363in}}%
\pgfpathlineto{\pgfqpoint{4.995051in}{1.627126in}}%
\pgfpathlineto{\pgfqpoint{5.002783in}{1.629588in}}%
\pgfpathlineto{\pgfqpoint{5.010515in}{1.630475in}}%
\pgfpathlineto{\pgfqpoint{5.020824in}{1.634115in}}%
\pgfpathlineto{\pgfqpoint{5.031132in}{1.635483in}}%
\pgfpathlineto{\pgfqpoint{5.038864in}{1.637349in}}%
\pgfpathlineto{\pgfqpoint{5.049173in}{1.638496in}}%
\pgfpathlineto{\pgfqpoint{5.056905in}{1.640031in}}%
\pgfpathlineto{\pgfqpoint{5.067213in}{1.640700in}}%
\pgfpathlineto{\pgfqpoint{5.074945in}{1.641872in}}%
\pgfpathlineto{\pgfqpoint{5.090408in}{1.642870in}}%
\pgfpathlineto{\pgfqpoint{5.092986in}{1.643341in}}%
\pgfpathlineto{\pgfqpoint{5.103294in}{1.644360in}}%
\pgfpathlineto{\pgfqpoint{5.111026in}{1.646089in}}%
\pgfpathlineto{\pgfqpoint{5.121335in}{1.647312in}}%
\pgfpathlineto{\pgfqpoint{5.129067in}{1.649322in}}%
\pgfpathlineto{\pgfqpoint{5.139375in}{1.650542in}}%
\pgfpathlineto{\pgfqpoint{5.147107in}{1.652320in}}%
\pgfpathlineto{\pgfqpoint{5.157416in}{1.653440in}}%
\pgfpathlineto{\pgfqpoint{5.165148in}{1.655371in}}%
\pgfpathlineto{\pgfqpoint{5.175456in}{1.656730in}}%
\pgfpathlineto{\pgfqpoint{5.183188in}{1.658726in}}%
\pgfpathlineto{\pgfqpoint{5.193497in}{1.660188in}}%
\pgfpathlineto{\pgfqpoint{5.201228in}{1.662306in}}%
\pgfpathlineto{\pgfqpoint{5.211537in}{1.663765in}}%
\pgfpathlineto{\pgfqpoint{5.219269in}{1.665795in}}%
\pgfpathlineto{\pgfqpoint{5.229578in}{1.667232in}}%
\pgfpathlineto{\pgfqpoint{5.237309in}{1.669453in}}%
\pgfpathlineto{\pgfqpoint{5.247618in}{1.670279in}}%
\pgfpathlineto{\pgfqpoint{5.255350in}{1.672730in}}%
\pgfpathlineto{\pgfqpoint{5.265659in}{1.674328in}}%
\pgfpathlineto{\pgfqpoint{5.273390in}{1.676555in}}%
\pgfpathlineto{\pgfqpoint{5.283699in}{1.678124in}}%
\pgfpathlineto{\pgfqpoint{5.291431in}{1.680583in}}%
\pgfpathlineto{\pgfqpoint{5.301740in}{1.682030in}}%
\pgfpathlineto{\pgfqpoint{5.309471in}{1.684305in}}%
\pgfpathlineto{\pgfqpoint{5.319780in}{1.685845in}}%
\pgfpathlineto{\pgfqpoint{5.327512in}{1.688223in}}%
\pgfpathlineto{\pgfqpoint{5.337821in}{1.689691in}}%
\pgfpathlineto{\pgfqpoint{5.345552in}{1.691772in}}%
\pgfpathlineto{\pgfqpoint{5.355861in}{1.693093in}}%
\pgfpathlineto{\pgfqpoint{5.363593in}{1.695207in}}%
\pgfpathlineto{\pgfqpoint{5.373902in}{1.696608in}}%
\pgfpathlineto{\pgfqpoint{5.381633in}{1.698535in}}%
\pgfpathlineto{\pgfqpoint{5.391942in}{1.699820in}}%
\pgfpathlineto{\pgfqpoint{5.399674in}{1.701387in}}%
\pgfpathlineto{\pgfqpoint{5.409983in}{1.702735in}}%
\pgfpathlineto{\pgfqpoint{5.417714in}{1.704653in}}%
\pgfpathlineto{\pgfqpoint{5.428023in}{1.705379in}}%
\pgfpathlineto{\pgfqpoint{5.435755in}{1.706927in}}%
\pgfpathlineto{\pgfqpoint{5.448641in}{1.708399in}}%
\pgfpathlineto{\pgfqpoint{5.471836in}{1.710283in}}%
\pgfpathlineto{\pgfqpoint{5.484722in}{1.711212in}}%
\pgfpathlineto{\pgfqpoint{5.489876in}{1.711985in}}%
\pgfpathlineto{\pgfqpoint{5.502762in}{1.713077in}}%
\pgfpathlineto{\pgfqpoint{5.507917in}{1.713793in}}%
\pgfpathlineto{\pgfqpoint{5.520803in}{1.714716in}}%
\pgfpathlineto{\pgfqpoint{5.525957in}{1.715301in}}%
\pgfpathlineto{\pgfqpoint{5.541421in}{1.716221in}}%
\pgfpathlineto{\pgfqpoint{5.543998in}{1.716506in}}%
\pgfpathlineto{\pgfqpoint{5.556884in}{1.717295in}}%
\pgfpathlineto{\pgfqpoint{5.562038in}{1.718320in}}%
\pgfpathlineto{\pgfqpoint{5.572347in}{1.719316in}}%
\pgfpathlineto{\pgfqpoint{5.580079in}{1.720776in}}%
\pgfpathlineto{\pgfqpoint{5.592965in}{1.721734in}}%
\pgfpathlineto{\pgfqpoint{5.598119in}{1.722731in}}%
\pgfpathlineto{\pgfqpoint{5.608428in}{1.723843in}}%
\pgfpathlineto{\pgfqpoint{5.616160in}{1.725691in}}%
\pgfpathlineto{\pgfqpoint{5.626469in}{1.726848in}}%
\pgfpathlineto{\pgfqpoint{5.667704in}{1.735675in}}%
\pgfpathlineto{\pgfqpoint{5.670281in}{1.736538in}}%
\pgfpathlineto{\pgfqpoint{5.680590in}{1.737433in}}%
\pgfpathlineto{\pgfqpoint{5.688322in}{1.740147in}}%
\pgfpathlineto{\pgfqpoint{5.696053in}{1.741032in}}%
\pgfpathlineto{\pgfqpoint{5.706362in}{1.744712in}}%
\pgfpathlineto{\pgfqpoint{5.714094in}{1.745652in}}%
\pgfpathlineto{\pgfqpoint{5.724403in}{1.749457in}}%
\pgfpathlineto{\pgfqpoint{5.732134in}{1.750470in}}%
\pgfpathlineto{\pgfqpoint{5.742443in}{1.754419in}}%
\pgfpathlineto{\pgfqpoint{5.752752in}{1.756158in}}%
\pgfpathlineto{\pgfqpoint{5.760484in}{1.758792in}}%
\pgfpathlineto{\pgfqpoint{5.768215in}{1.759669in}}%
\pgfpathlineto{\pgfqpoint{5.778524in}{1.763167in}}%
\pgfpathlineto{\pgfqpoint{5.788833in}{1.764895in}}%
\pgfpathlineto{\pgfqpoint{5.796565in}{1.767433in}}%
\pgfpathlineto{\pgfqpoint{5.806873in}{1.769090in}}%
\pgfpathlineto{\pgfqpoint{5.812028in}{1.770720in}}%
\pgfpathlineto{\pgfqpoint{5.822337in}{1.771609in}}%
\pgfpathlineto{\pgfqpoint{5.832646in}{1.775325in}}%
\pgfpathlineto{\pgfqpoint{5.840377in}{1.776355in}}%
\pgfpathlineto{\pgfqpoint{5.850686in}{1.780376in}}%
\pgfpathlineto{\pgfqpoint{5.858418in}{1.781335in}}%
\pgfpathlineto{\pgfqpoint{5.868727in}{1.785501in}}%
\pgfpathlineto{\pgfqpoint{5.876458in}{1.786491in}}%
\pgfpathlineto{\pgfqpoint{5.886767in}{1.790516in}}%
\pgfpathlineto{\pgfqpoint{5.894499in}{1.791588in}}%
\pgfpathlineto{\pgfqpoint{5.904807in}{1.795566in}}%
\pgfpathlineto{\pgfqpoint{5.912539in}{1.796636in}}%
\pgfpathlineto{\pgfqpoint{5.922848in}{1.801286in}}%
\pgfpathlineto{\pgfqpoint{5.933157in}{1.802446in}}%
\pgfpathlineto{\pgfqpoint{5.940888in}{1.806115in}}%
\pgfpathlineto{\pgfqpoint{5.948620in}{1.807407in}}%
\pgfpathlineto{\pgfqpoint{5.958929in}{1.812213in}}%
\pgfpathlineto{\pgfqpoint{5.966661in}{1.813228in}}%
\pgfpathlineto{\pgfqpoint{5.976969in}{1.817621in}}%
\pgfpathlineto{\pgfqpoint{5.984701in}{1.818715in}}%
\pgfpathlineto{\pgfqpoint{5.995010in}{1.822989in}}%
\pgfpathlineto{\pgfqpoint{6.002742in}{1.824094in}}%
\pgfpathlineto{\pgfqpoint{6.013050in}{1.828346in}}%
\pgfpathlineto{\pgfqpoint{6.025936in}{1.830264in}}%
\pgfpathlineto{\pgfqpoint{6.031091in}{1.832152in}}%
\pgfpathlineto{\pgfqpoint{6.038823in}{1.833205in}}%
\pgfpathlineto{\pgfqpoint{6.049131in}{1.837699in}}%
\pgfpathlineto{\pgfqpoint{6.056863in}{1.838867in}}%
\pgfpathlineto{\pgfqpoint{6.067172in}{1.844003in}}%
\pgfpathlineto{\pgfqpoint{6.074904in}{1.845441in}}%
\pgfpathlineto{\pgfqpoint{6.085212in}{1.850971in}}%
\pgfpathlineto{\pgfqpoint{6.092944in}{1.852309in}}%
\pgfpathlineto{\pgfqpoint{6.103253in}{1.858064in}}%
\pgfpathlineto{\pgfqpoint{6.110984in}{1.859538in}}%
\pgfpathlineto{\pgfqpoint{6.121293in}{1.865016in}}%
\pgfpathlineto{\pgfqpoint{6.129025in}{1.866491in}}%
\pgfpathlineto{\pgfqpoint{6.139334in}{1.872643in}}%
\pgfpathlineto{\pgfqpoint{6.147065in}{1.874218in}}%
\pgfpathlineto{\pgfqpoint{6.157374in}{1.880530in}}%
\pgfpathlineto{\pgfqpoint{6.165106in}{1.882119in}}%
\pgfpathlineto{\pgfqpoint{6.175415in}{1.888388in}}%
\pgfpathlineto{\pgfqpoint{6.185724in}{1.889875in}}%
\pgfpathlineto{\pgfqpoint{6.193455in}{1.894570in}}%
\pgfpathlineto{\pgfqpoint{6.201187in}{1.896309in}}%
\pgfpathlineto{\pgfqpoint{6.211496in}{1.903081in}}%
\pgfpathlineto{\pgfqpoint{6.219227in}{1.904663in}}%
\pgfpathlineto{\pgfqpoint{6.229536in}{1.911093in}}%
\pgfpathlineto{\pgfqpoint{6.237268in}{1.912486in}}%
\pgfpathlineto{\pgfqpoint{6.247577in}{1.918386in}}%
\pgfpathlineto{\pgfqpoint{6.255308in}{1.919953in}}%
\pgfpathlineto{\pgfqpoint{6.265617in}{1.926340in}}%
\pgfpathlineto{\pgfqpoint{6.273349in}{1.928010in}}%
\pgfpathlineto{\pgfqpoint{6.283658in}{1.935015in}}%
\pgfpathlineto{\pgfqpoint{6.291389in}{1.936762in}}%
\pgfpathlineto{\pgfqpoint{6.301698in}{1.943383in}}%
\pgfpathlineto{\pgfqpoint{6.309430in}{1.945025in}}%
\pgfpathlineto{\pgfqpoint{6.319739in}{1.952162in}}%
\pgfpathlineto{\pgfqpoint{6.327470in}{1.953988in}}%
\pgfpathlineto{\pgfqpoint{6.337779in}{1.961505in}}%
\pgfpathlineto{\pgfqpoint{6.345511in}{1.963461in}}%
\pgfpathlineto{\pgfqpoint{6.355820in}{1.971281in}}%
\pgfpathlineto{\pgfqpoint{6.363551in}{1.973147in}}%
\pgfpathlineto{\pgfqpoint{6.373860in}{1.980338in}}%
\pgfpathlineto{\pgfqpoint{6.381592in}{1.982053in}}%
\pgfpathlineto{\pgfqpoint{6.386746in}{1.985668in}}%
\pgfpathlineto{\pgfqpoint{6.391901in}{1.987537in}}%
\pgfpathlineto{\pgfqpoint{6.399632in}{1.989436in}}%
\pgfpathlineto{\pgfqpoint{6.409941in}{1.996690in}}%
\pgfpathlineto{\pgfqpoint{6.417673in}{1.998128in}}%
\pgfpathlineto{\pgfqpoint{6.427982in}{2.004868in}}%
\pgfpathlineto{\pgfqpoint{6.435713in}{2.006688in}}%
\pgfpathlineto{\pgfqpoint{6.446022in}{2.014244in}}%
\pgfpathlineto{\pgfqpoint{6.453754in}{2.016128in}}%
\pgfpathlineto{\pgfqpoint{6.464063in}{2.023161in}}%
\pgfpathlineto{\pgfqpoint{6.474371in}{2.024956in}}%
\pgfpathlineto{\pgfqpoint{6.482103in}{2.030593in}}%
\pgfpathlineto{\pgfqpoint{6.482103in}{2.030593in}}%
\pgfusepath{stroke}%
\end{pgfscope}%
\begin{pgfscope}%
\pgfpathrectangle{\pgfqpoint{0.563921in}{0.521603in}}{\pgfqpoint{6.200000in}{2.642500in}}%
\pgfusepath{clip}%
\pgfsetroundcap%
\pgfsetroundjoin%
\pgfsetlinewidth{1.505625pt}%
\definecolor{currentstroke}{rgb}{0.090196,0.745098,0.811765}%
\pgfsetstrokecolor{currentstroke}%
\pgfsetdash{}{0pt}%
\pgfpathmoveto{\pgfqpoint{0.845739in}{0.641717in}}%
\pgfpathlineto{\pgfqpoint{0.850894in}{0.667793in}}%
\pgfpathlineto{\pgfqpoint{0.853471in}{0.674543in}}%
\pgfpathlineto{\pgfqpoint{0.861203in}{0.673799in}}%
\pgfpathlineto{\pgfqpoint{0.863780in}{0.671929in}}%
\pgfpathlineto{\pgfqpoint{0.866357in}{0.672040in}}%
\pgfpathlineto{\pgfqpoint{0.868934in}{0.671504in}}%
\pgfpathlineto{\pgfqpoint{0.871512in}{0.672660in}}%
\pgfpathlineto{\pgfqpoint{0.881820in}{0.672656in}}%
\pgfpathlineto{\pgfqpoint{0.884398in}{0.671216in}}%
\pgfpathlineto{\pgfqpoint{0.889552in}{0.669705in}}%
\pgfpathlineto{\pgfqpoint{0.902438in}{0.667795in}}%
\pgfpathlineto{\pgfqpoint{0.907593in}{0.666479in}}%
\pgfpathlineto{\pgfqpoint{0.920479in}{0.664980in}}%
\pgfpathlineto{\pgfqpoint{0.923056in}{0.664584in}}%
\pgfpathlineto{\pgfqpoint{0.925633in}{0.666039in}}%
\pgfpathlineto{\pgfqpoint{0.933365in}{0.669259in}}%
\pgfpathlineto{\pgfqpoint{0.941096in}{0.685017in}}%
\pgfpathlineto{\pgfqpoint{0.943674in}{0.688665in}}%
\pgfpathlineto{\pgfqpoint{0.951405in}{0.692931in}}%
\pgfpathlineto{\pgfqpoint{0.953982in}{0.695645in}}%
\pgfpathlineto{\pgfqpoint{0.959137in}{0.698610in}}%
\pgfpathlineto{\pgfqpoint{0.961714in}{0.700611in}}%
\pgfpathlineto{\pgfqpoint{0.987486in}{0.704062in}}%
\pgfpathlineto{\pgfqpoint{0.995218in}{0.708346in}}%
\pgfpathlineto{\pgfqpoint{0.997795in}{0.709880in}}%
\pgfpathlineto{\pgfqpoint{1.008104in}{0.712454in}}%
\pgfpathlineto{\pgfqpoint{1.013258in}{0.712944in}}%
\pgfpathlineto{\pgfqpoint{1.015835in}{0.713601in}}%
\pgfpathlineto{\pgfqpoint{1.023567in}{0.714318in}}%
\pgfpathlineto{\pgfqpoint{1.026144in}{0.718382in}}%
\pgfpathlineto{\pgfqpoint{1.031299in}{0.722755in}}%
\pgfpathlineto{\pgfqpoint{1.033876in}{0.724084in}}%
\pgfpathlineto{\pgfqpoint{1.041608in}{0.725688in}}%
\pgfpathlineto{\pgfqpoint{1.051916in}{0.730161in}}%
\pgfpathlineto{\pgfqpoint{1.059648in}{0.732594in}}%
\pgfpathlineto{\pgfqpoint{1.062225in}{0.734409in}}%
\pgfpathlineto{\pgfqpoint{1.064802in}{0.735153in}}%
\pgfpathlineto{\pgfqpoint{1.067380in}{0.735286in}}%
\pgfpathlineto{\pgfqpoint{1.069957in}{0.736226in}}%
\pgfpathlineto{\pgfqpoint{1.082843in}{0.737304in}}%
\pgfpathlineto{\pgfqpoint{1.085420in}{0.737305in}}%
\pgfpathlineto{\pgfqpoint{1.098306in}{0.736055in}}%
\pgfpathlineto{\pgfqpoint{1.106038in}{0.734247in}}%
\pgfpathlineto{\pgfqpoint{1.121501in}{0.732512in}}%
\pgfpathlineto{\pgfqpoint{1.124078in}{0.732104in}}%
\pgfpathlineto{\pgfqpoint{1.142119in}{0.731241in}}%
\pgfpathlineto{\pgfqpoint{1.152428in}{0.731676in}}%
\pgfpathlineto{\pgfqpoint{1.170468in}{0.733274in}}%
\pgfpathlineto{\pgfqpoint{1.178200in}{0.738292in}}%
\pgfpathlineto{\pgfqpoint{1.185931in}{0.739608in}}%
\pgfpathlineto{\pgfqpoint{1.191086in}{0.741755in}}%
\pgfpathlineto{\pgfqpoint{1.196240in}{0.742150in}}%
\pgfpathlineto{\pgfqpoint{1.211704in}{0.743409in}}%
\pgfpathlineto{\pgfqpoint{1.227167in}{0.745674in}}%
\pgfpathlineto{\pgfqpoint{1.229744in}{0.746879in}}%
\pgfpathlineto{\pgfqpoint{1.242630in}{0.747587in}}%
\pgfpathlineto{\pgfqpoint{1.250362in}{0.750791in}}%
\pgfpathlineto{\pgfqpoint{1.258093in}{0.751733in}}%
\pgfpathlineto{\pgfqpoint{1.265825in}{0.756335in}}%
\pgfpathlineto{\pgfqpoint{1.268402in}{0.758151in}}%
\pgfpathlineto{\pgfqpoint{1.276134in}{0.759879in}}%
\pgfpathlineto{\pgfqpoint{1.286443in}{0.767480in}}%
\pgfpathlineto{\pgfqpoint{1.294174in}{0.768458in}}%
\pgfpathlineto{\pgfqpoint{1.299329in}{0.771688in}}%
\pgfpathlineto{\pgfqpoint{1.304483in}{0.775601in}}%
\pgfpathlineto{\pgfqpoint{1.312215in}{0.777923in}}%
\pgfpathlineto{\pgfqpoint{1.314792in}{0.780012in}}%
\pgfpathlineto{\pgfqpoint{1.319947in}{0.781581in}}%
\pgfpathlineto{\pgfqpoint{1.322524in}{0.783024in}}%
\pgfpathlineto{\pgfqpoint{1.332833in}{0.785210in}}%
\pgfpathlineto{\pgfqpoint{1.337987in}{0.786754in}}%
\pgfpathlineto{\pgfqpoint{1.340564in}{0.788045in}}%
\pgfpathlineto{\pgfqpoint{1.348296in}{0.789050in}}%
\pgfpathlineto{\pgfqpoint{1.356027in}{0.794737in}}%
\pgfpathlineto{\pgfqpoint{1.358605in}{0.795997in}}%
\pgfpathlineto{\pgfqpoint{1.366336in}{0.796814in}}%
\pgfpathlineto{\pgfqpoint{1.371491in}{0.798634in}}%
\pgfpathlineto{\pgfqpoint{1.376645in}{0.802468in}}%
\pgfpathlineto{\pgfqpoint{1.384377in}{0.804250in}}%
\pgfpathlineto{\pgfqpoint{1.394686in}{0.809064in}}%
\pgfpathlineto{\pgfqpoint{1.402417in}{0.810443in}}%
\pgfpathlineto{\pgfqpoint{1.412726in}{0.816276in}}%
\pgfpathlineto{\pgfqpoint{1.420458in}{0.817519in}}%
\pgfpathlineto{\pgfqpoint{1.428189in}{0.821128in}}%
\pgfpathlineto{\pgfqpoint{1.430767in}{0.822549in}}%
\pgfpathlineto{\pgfqpoint{1.441076in}{0.824765in}}%
\pgfpathlineto{\pgfqpoint{1.448807in}{0.826894in}}%
\pgfpathlineto{\pgfqpoint{1.459116in}{0.828334in}}%
\pgfpathlineto{\pgfqpoint{1.464270in}{0.829773in}}%
\pgfpathlineto{\pgfqpoint{1.466848in}{0.830332in}}%
\pgfpathlineto{\pgfqpoint{1.477156in}{0.830955in}}%
\pgfpathlineto{\pgfqpoint{1.482311in}{0.833869in}}%
\pgfpathlineto{\pgfqpoint{1.484888in}{0.835481in}}%
\pgfpathlineto{\pgfqpoint{1.492620in}{0.836907in}}%
\pgfpathlineto{\pgfqpoint{1.500351in}{0.841670in}}%
\pgfpathlineto{\pgfqpoint{1.502929in}{0.843372in}}%
\pgfpathlineto{\pgfqpoint{1.510660in}{0.844900in}}%
\pgfpathlineto{\pgfqpoint{1.515815in}{0.847991in}}%
\pgfpathlineto{\pgfqpoint{1.520969in}{0.851309in}}%
\pgfpathlineto{\pgfqpoint{1.528701in}{0.853016in}}%
\pgfpathlineto{\pgfqpoint{1.536432in}{0.856910in}}%
\pgfpathlineto{\pgfqpoint{1.539010in}{0.858089in}}%
\pgfpathlineto{\pgfqpoint{1.549318in}{0.859941in}}%
\pgfpathlineto{\pgfqpoint{1.557050in}{0.863657in}}%
\pgfpathlineto{\pgfqpoint{1.587977in}{0.867741in}}%
\pgfpathlineto{\pgfqpoint{1.593131in}{0.869228in}}%
\pgfpathlineto{\pgfqpoint{1.611172in}{0.870185in}}%
\pgfpathlineto{\pgfqpoint{1.647253in}{0.869299in}}%
\pgfpathlineto{\pgfqpoint{1.657561in}{0.868422in}}%
\pgfpathlineto{\pgfqpoint{1.665293in}{0.867030in}}%
\pgfpathlineto{\pgfqpoint{1.737455in}{0.863496in}}%
\pgfpathlineto{\pgfqpoint{1.750341in}{0.863580in}}%
\pgfpathlineto{\pgfqpoint{1.755495in}{0.863936in}}%
\pgfpathlineto{\pgfqpoint{1.788999in}{0.864388in}}%
\pgfpathlineto{\pgfqpoint{1.791576in}{0.865011in}}%
\pgfpathlineto{\pgfqpoint{1.819926in}{0.866262in}}%
\pgfpathlineto{\pgfqpoint{1.840543in}{0.869543in}}%
\pgfpathlineto{\pgfqpoint{1.845698in}{0.871825in}}%
\pgfpathlineto{\pgfqpoint{1.853430in}{0.873000in}}%
\pgfpathlineto{\pgfqpoint{1.863738in}{0.877025in}}%
\pgfpathlineto{\pgfqpoint{1.871470in}{0.877916in}}%
\pgfpathlineto{\pgfqpoint{1.881779in}{0.882066in}}%
\pgfpathlineto{\pgfqpoint{1.889510in}{0.883152in}}%
\pgfpathlineto{\pgfqpoint{1.899819in}{0.887777in}}%
\pgfpathlineto{\pgfqpoint{1.910128in}{0.889107in}}%
\pgfpathlineto{\pgfqpoint{1.917860in}{0.891487in}}%
\pgfpathlineto{\pgfqpoint{1.928169in}{0.892637in}}%
\pgfpathlineto{\pgfqpoint{1.935900in}{0.895225in}}%
\pgfpathlineto{\pgfqpoint{1.943632in}{0.896387in}}%
\pgfpathlineto{\pgfqpoint{1.953941in}{0.902009in}}%
\pgfpathlineto{\pgfqpoint{1.961672in}{0.903732in}}%
\pgfpathlineto{\pgfqpoint{1.971981in}{0.909943in}}%
\pgfpathlineto{\pgfqpoint{1.979713in}{0.911192in}}%
\pgfpathlineto{\pgfqpoint{1.990022in}{0.915629in}}%
\pgfpathlineto{\pgfqpoint{1.997753in}{0.916566in}}%
\pgfpathlineto{\pgfqpoint{2.005485in}{0.919657in}}%
\pgfpathlineto{\pgfqpoint{2.015794in}{0.920667in}}%
\pgfpathlineto{\pgfqpoint{2.026103in}{0.925537in}}%
\pgfpathlineto{\pgfqpoint{2.033834in}{0.927152in}}%
\pgfpathlineto{\pgfqpoint{2.038989in}{0.930894in}}%
\pgfpathlineto{\pgfqpoint{2.044143in}{0.935202in}}%
\pgfpathlineto{\pgfqpoint{2.051875in}{0.936666in}}%
\pgfpathlineto{\pgfqpoint{2.062184in}{0.944933in}}%
\pgfpathlineto{\pgfqpoint{2.069915in}{0.947394in}}%
\pgfpathlineto{\pgfqpoint{2.077647in}{0.954678in}}%
\pgfpathlineto{\pgfqpoint{2.080224in}{0.956883in}}%
\pgfpathlineto{\pgfqpoint{2.087956in}{0.959498in}}%
\pgfpathlineto{\pgfqpoint{2.095687in}{0.967397in}}%
\pgfpathlineto{\pgfqpoint{2.098265in}{0.970545in}}%
\pgfpathlineto{\pgfqpoint{2.105996in}{0.973737in}}%
\pgfpathlineto{\pgfqpoint{2.116305in}{0.988295in}}%
\pgfpathlineto{\pgfqpoint{2.124037in}{0.992106in}}%
\pgfpathlineto{\pgfqpoint{2.131768in}{1.002880in}}%
\pgfpathlineto{\pgfqpoint{2.134346in}{1.006046in}}%
\pgfpathlineto{\pgfqpoint{2.142077in}{1.008961in}}%
\pgfpathlineto{\pgfqpoint{2.152386in}{1.019167in}}%
\pgfpathlineto{\pgfqpoint{2.162695in}{1.022031in}}%
\pgfpathlineto{\pgfqpoint{2.167849in}{1.026705in}}%
\pgfpathlineto{\pgfqpoint{2.170427in}{1.028205in}}%
\pgfpathlineto{\pgfqpoint{2.178158in}{1.029898in}}%
\pgfpathlineto{\pgfqpoint{2.183313in}{1.033164in}}%
\pgfpathlineto{\pgfqpoint{2.188467in}{1.036481in}}%
\pgfpathlineto{\pgfqpoint{2.196199in}{1.038042in}}%
\pgfpathlineto{\pgfqpoint{2.206508in}{1.043912in}}%
\pgfpathlineto{\pgfqpoint{2.214239in}{1.045538in}}%
\pgfpathlineto{\pgfqpoint{2.219394in}{1.048899in}}%
\pgfpathlineto{\pgfqpoint{2.224548in}{1.050837in}}%
\pgfpathlineto{\pgfqpoint{2.232280in}{1.051801in}}%
\pgfpathlineto{\pgfqpoint{2.242589in}{1.056213in}}%
\pgfpathlineto{\pgfqpoint{2.250320in}{1.057468in}}%
\pgfpathlineto{\pgfqpoint{2.255475in}{1.059664in}}%
\pgfpathlineto{\pgfqpoint{2.270938in}{1.063637in}}%
\pgfpathlineto{\pgfqpoint{2.276092in}{1.066864in}}%
\pgfpathlineto{\pgfqpoint{2.278670in}{1.068790in}}%
\pgfpathlineto{\pgfqpoint{2.286401in}{1.070383in}}%
\pgfpathlineto{\pgfqpoint{2.296710in}{1.075886in}}%
\pgfpathlineto{\pgfqpoint{2.304442in}{1.076976in}}%
\pgfpathlineto{\pgfqpoint{2.314751in}{1.081409in}}%
\pgfpathlineto{\pgfqpoint{2.322482in}{1.082465in}}%
\pgfpathlineto{\pgfqpoint{2.330214in}{1.085625in}}%
\pgfpathlineto{\pgfqpoint{2.332791in}{1.087082in}}%
\pgfpathlineto{\pgfqpoint{2.340523in}{1.088392in}}%
\pgfpathlineto{\pgfqpoint{2.345677in}{1.091186in}}%
\pgfpathlineto{\pgfqpoint{2.350832in}{1.093305in}}%
\pgfpathlineto{\pgfqpoint{2.361140in}{1.094805in}}%
\pgfpathlineto{\pgfqpoint{2.366295in}{1.095960in}}%
\pgfpathlineto{\pgfqpoint{2.368872in}{1.096340in}}%
\pgfpathlineto{\pgfqpoint{2.384335in}{1.097430in}}%
\pgfpathlineto{\pgfqpoint{2.386912in}{1.097705in}}%
\pgfpathlineto{\pgfqpoint{2.430725in}{1.099255in}}%
\pgfpathlineto{\pgfqpoint{2.435880in}{1.100248in}}%
\pgfpathlineto{\pgfqpoint{2.441034in}{1.102265in}}%
\pgfpathlineto{\pgfqpoint{2.448766in}{1.103437in}}%
\pgfpathlineto{\pgfqpoint{2.456497in}{1.106642in}}%
\pgfpathlineto{\pgfqpoint{2.459074in}{1.107353in}}%
\pgfpathlineto{\pgfqpoint{2.469383in}{1.108571in}}%
\pgfpathlineto{\pgfqpoint{2.477115in}{1.110566in}}%
\pgfpathlineto{\pgfqpoint{2.487424in}{1.111741in}}%
\pgfpathlineto{\pgfqpoint{2.495155in}{1.113484in}}%
\pgfpathlineto{\pgfqpoint{2.508041in}{1.114808in}}%
\pgfpathlineto{\pgfqpoint{2.513196in}{1.116301in}}%
\pgfpathlineto{\pgfqpoint{2.523505in}{1.118035in}}%
\pgfpathlineto{\pgfqpoint{2.531236in}{1.120554in}}%
\pgfpathlineto{\pgfqpoint{2.538968in}{1.121562in}}%
\pgfpathlineto{\pgfqpoint{2.549277in}{1.126507in}}%
\pgfpathlineto{\pgfqpoint{2.557009in}{1.127731in}}%
\pgfpathlineto{\pgfqpoint{2.567317in}{1.132266in}}%
\pgfpathlineto{\pgfqpoint{2.575049in}{1.133376in}}%
\pgfpathlineto{\pgfqpoint{2.585358in}{1.137358in}}%
\pgfpathlineto{\pgfqpoint{2.595667in}{1.139138in}}%
\pgfpathlineto{\pgfqpoint{2.603398in}{1.142674in}}%
\pgfpathlineto{\pgfqpoint{2.611130in}{1.143787in}}%
\pgfpathlineto{\pgfqpoint{2.621439in}{1.148202in}}%
\pgfpathlineto{\pgfqpoint{2.629170in}{1.149294in}}%
\pgfpathlineto{\pgfqpoint{2.634325in}{1.151941in}}%
\pgfpathlineto{\pgfqpoint{2.652365in}{1.156522in}}%
\pgfpathlineto{\pgfqpoint{2.657520in}{1.158935in}}%
\pgfpathlineto{\pgfqpoint{2.665251in}{1.160171in}}%
\pgfpathlineto{\pgfqpoint{2.672983in}{1.163643in}}%
\pgfpathlineto{\pgfqpoint{2.675560in}{1.164698in}}%
\pgfpathlineto{\pgfqpoint{2.683292in}{1.165916in}}%
\pgfpathlineto{\pgfqpoint{2.688446in}{1.168691in}}%
\pgfpathlineto{\pgfqpoint{2.693601in}{1.171935in}}%
\pgfpathlineto{\pgfqpoint{2.701332in}{1.173658in}}%
\pgfpathlineto{\pgfqpoint{2.703910in}{1.175491in}}%
\pgfpathlineto{\pgfqpoint{2.709064in}{1.177484in}}%
\pgfpathlineto{\pgfqpoint{2.711641in}{1.179387in}}%
\pgfpathlineto{\pgfqpoint{2.719373in}{1.181724in}}%
\pgfpathlineto{\pgfqpoint{2.721950in}{1.184071in}}%
\pgfpathlineto{\pgfqpoint{2.727105in}{1.186355in}}%
\pgfpathlineto{\pgfqpoint{2.729682in}{1.188570in}}%
\pgfpathlineto{\pgfqpoint{2.737413in}{1.190687in}}%
\pgfpathlineto{\pgfqpoint{2.745145in}{1.196649in}}%
\pgfpathlineto{\pgfqpoint{2.747722in}{1.198564in}}%
\pgfpathlineto{\pgfqpoint{2.755454in}{1.199997in}}%
\pgfpathlineto{\pgfqpoint{2.765763in}{1.206344in}}%
\pgfpathlineto{\pgfqpoint{2.776072in}{1.207879in}}%
\pgfpathlineto{\pgfqpoint{2.781226in}{1.211247in}}%
\pgfpathlineto{\pgfqpoint{2.783803in}{1.212435in}}%
\pgfpathlineto{\pgfqpoint{2.791535in}{1.213521in}}%
\pgfpathlineto{\pgfqpoint{2.801844in}{1.217940in}}%
\pgfpathlineto{\pgfqpoint{2.809575in}{1.218589in}}%
\pgfpathlineto{\pgfqpoint{2.814730in}{1.220318in}}%
\pgfpathlineto{\pgfqpoint{2.819884in}{1.223578in}}%
\pgfpathlineto{\pgfqpoint{2.827616in}{1.225486in}}%
\pgfpathlineto{\pgfqpoint{2.837925in}{1.233906in}}%
\pgfpathlineto{\pgfqpoint{2.848234in}{1.236263in}}%
\pgfpathlineto{\pgfqpoint{2.855965in}{1.243039in}}%
\pgfpathlineto{\pgfqpoint{2.863697in}{1.245552in}}%
\pgfpathlineto{\pgfqpoint{2.874006in}{1.254995in}}%
\pgfpathlineto{\pgfqpoint{2.881737in}{1.257077in}}%
\pgfpathlineto{\pgfqpoint{2.892046in}{1.267903in}}%
\pgfpathlineto{\pgfqpoint{2.899778in}{1.270404in}}%
\pgfpathlineto{\pgfqpoint{2.910087in}{1.278919in}}%
\pgfpathlineto{\pgfqpoint{2.917818in}{1.281154in}}%
\pgfpathlineto{\pgfqpoint{2.925550in}{1.287533in}}%
\pgfpathlineto{\pgfqpoint{2.928127in}{1.289451in}}%
\pgfpathlineto{\pgfqpoint{2.935859in}{1.291175in}}%
\pgfpathlineto{\pgfqpoint{2.946168in}{1.297452in}}%
\pgfpathlineto{\pgfqpoint{2.953899in}{1.299199in}}%
\pgfpathlineto{\pgfqpoint{2.964208in}{1.306998in}}%
\pgfpathlineto{\pgfqpoint{2.971940in}{1.308473in}}%
\pgfpathlineto{\pgfqpoint{2.979671in}{1.312851in}}%
\pgfpathlineto{\pgfqpoint{2.982249in}{1.313906in}}%
\pgfpathlineto{\pgfqpoint{2.989980in}{1.315048in}}%
\pgfpathlineto{\pgfqpoint{2.997712in}{1.319041in}}%
\pgfpathlineto{\pgfqpoint{3.008021in}{1.320382in}}%
\pgfpathlineto{\pgfqpoint{3.018330in}{1.325676in}}%
\pgfpathlineto{\pgfqpoint{3.026061in}{1.326732in}}%
\pgfpathlineto{\pgfqpoint{3.036370in}{1.331942in}}%
\pgfpathlineto{\pgfqpoint{3.044102in}{1.333515in}}%
\pgfpathlineto{\pgfqpoint{3.054411in}{1.339646in}}%
\pgfpathlineto{\pgfqpoint{3.062142in}{1.341355in}}%
\pgfpathlineto{\pgfqpoint{3.069874in}{1.345661in}}%
\pgfpathlineto{\pgfqpoint{3.072451in}{1.346955in}}%
\pgfpathlineto{\pgfqpoint{3.080183in}{1.348345in}}%
\pgfpathlineto{\pgfqpoint{3.087914in}{1.352855in}}%
\pgfpathlineto{\pgfqpoint{3.090491in}{1.354582in}}%
\pgfpathlineto{\pgfqpoint{3.100800in}{1.356367in}}%
\pgfpathlineto{\pgfqpoint{3.108532in}{1.361691in}}%
\pgfpathlineto{\pgfqpoint{3.116264in}{1.363501in}}%
\pgfpathlineto{\pgfqpoint{3.126572in}{1.370634in}}%
\pgfpathlineto{\pgfqpoint{3.134304in}{1.372573in}}%
\pgfpathlineto{\pgfqpoint{3.142036in}{1.377468in}}%
\pgfpathlineto{\pgfqpoint{3.144613in}{1.378872in}}%
\pgfpathlineto{\pgfqpoint{3.152345in}{1.380345in}}%
\pgfpathlineto{\pgfqpoint{3.162653in}{1.386116in}}%
\pgfpathlineto{\pgfqpoint{3.170385in}{1.387432in}}%
\pgfpathlineto{\pgfqpoint{3.178117in}{1.391748in}}%
\pgfpathlineto{\pgfqpoint{3.180694in}{1.393426in}}%
\pgfpathlineto{\pgfqpoint{3.188426in}{1.395164in}}%
\pgfpathlineto{\pgfqpoint{3.196157in}{1.400718in}}%
\pgfpathlineto{\pgfqpoint{3.206466in}{1.402535in}}%
\pgfpathlineto{\pgfqpoint{3.216775in}{1.409708in}}%
\pgfpathlineto{\pgfqpoint{3.224507in}{1.411467in}}%
\pgfpathlineto{\pgfqpoint{3.234815in}{1.417564in}}%
\pgfpathlineto{\pgfqpoint{3.242547in}{1.419076in}}%
\pgfpathlineto{\pgfqpoint{3.252856in}{1.425357in}}%
\pgfpathlineto{\pgfqpoint{3.260588in}{1.427026in}}%
\pgfpathlineto{\pgfqpoint{3.270896in}{1.432925in}}%
\pgfpathlineto{\pgfqpoint{3.278628in}{1.434534in}}%
\pgfpathlineto{\pgfqpoint{3.288937in}{1.440320in}}%
\pgfpathlineto{\pgfqpoint{3.296668in}{1.441900in}}%
\pgfpathlineto{\pgfqpoint{3.306977in}{1.448510in}}%
\pgfpathlineto{\pgfqpoint{3.314709in}{1.450418in}}%
\pgfpathlineto{\pgfqpoint{3.325018in}{1.458027in}}%
\pgfpathlineto{\pgfqpoint{3.332749in}{1.459915in}}%
\pgfpathlineto{\pgfqpoint{3.343058in}{1.467127in}}%
\pgfpathlineto{\pgfqpoint{3.353367in}{1.469000in}}%
\pgfpathlineto{\pgfqpoint{3.361099in}{1.474451in}}%
\pgfpathlineto{\pgfqpoint{3.368830in}{1.476217in}}%
\pgfpathlineto{\pgfqpoint{3.379139in}{1.482585in}}%
\pgfpathlineto{\pgfqpoint{3.386871in}{1.484201in}}%
\pgfpathlineto{\pgfqpoint{3.397180in}{1.490693in}}%
\pgfpathlineto{\pgfqpoint{3.404911in}{1.492121in}}%
\pgfpathlineto{\pgfqpoint{3.415220in}{1.497346in}}%
\pgfpathlineto{\pgfqpoint{3.422952in}{1.498649in}}%
\pgfpathlineto{\pgfqpoint{3.430684in}{1.502066in}}%
\pgfpathlineto{\pgfqpoint{3.433261in}{1.503274in}}%
\pgfpathlineto{\pgfqpoint{3.440992in}{1.504487in}}%
\pgfpathlineto{\pgfqpoint{3.451301in}{1.508329in}}%
\pgfpathlineto{\pgfqpoint{3.461610in}{1.509545in}}%
\pgfpathlineto{\pgfqpoint{3.479651in}{1.512609in}}%
\pgfpathlineto{\pgfqpoint{3.487382in}{1.515670in}}%
\pgfpathlineto{\pgfqpoint{3.495114in}{1.516742in}}%
\pgfpathlineto{\pgfqpoint{3.505423in}{1.521896in}}%
\pgfpathlineto{\pgfqpoint{3.513154in}{1.523356in}}%
\pgfpathlineto{\pgfqpoint{3.523463in}{1.528601in}}%
\pgfpathlineto{\pgfqpoint{3.531195in}{1.529744in}}%
\pgfpathlineto{\pgfqpoint{3.541504in}{1.534489in}}%
\pgfpathlineto{\pgfqpoint{3.549235in}{1.535665in}}%
\pgfpathlineto{\pgfqpoint{3.559544in}{1.539815in}}%
\pgfpathlineto{\pgfqpoint{3.567276in}{1.540967in}}%
\pgfpathlineto{\pgfqpoint{3.572430in}{1.543555in}}%
\pgfpathlineto{\pgfqpoint{3.577585in}{1.544930in}}%
\pgfpathlineto{\pgfqpoint{3.585316in}{1.546320in}}%
\pgfpathlineto{\pgfqpoint{3.595625in}{1.552141in}}%
\pgfpathlineto{\pgfqpoint{3.603357in}{1.553622in}}%
\pgfpathlineto{\pgfqpoint{3.613666in}{1.558759in}}%
\pgfpathlineto{\pgfqpoint{3.621397in}{1.559955in}}%
\pgfpathlineto{\pgfqpoint{3.629129in}{1.563681in}}%
\pgfpathlineto{\pgfqpoint{3.631706in}{1.565100in}}%
\pgfpathlineto{\pgfqpoint{3.639438in}{1.566687in}}%
\pgfpathlineto{\pgfqpoint{3.644592in}{1.569925in}}%
\pgfpathlineto{\pgfqpoint{3.649747in}{1.571591in}}%
\pgfpathlineto{\pgfqpoint{3.657478in}{1.573312in}}%
\pgfpathlineto{\pgfqpoint{3.662633in}{1.576414in}}%
\pgfpathlineto{\pgfqpoint{3.667787in}{1.577852in}}%
\pgfpathlineto{\pgfqpoint{3.675519in}{1.579099in}}%
\pgfpathlineto{\pgfqpoint{3.685828in}{1.584412in}}%
\pgfpathlineto{\pgfqpoint{3.693559in}{1.585880in}}%
\pgfpathlineto{\pgfqpoint{3.703868in}{1.591784in}}%
\pgfpathlineto{\pgfqpoint{3.714177in}{1.593231in}}%
\pgfpathlineto{\pgfqpoint{3.721909in}{1.597482in}}%
\pgfpathlineto{\pgfqpoint{3.729640in}{1.598914in}}%
\pgfpathlineto{\pgfqpoint{3.737372in}{1.602508in}}%
\pgfpathlineto{\pgfqpoint{3.739949in}{1.603412in}}%
\pgfpathlineto{\pgfqpoint{3.747681in}{1.604420in}}%
\pgfpathlineto{\pgfqpoint{3.750258in}{1.605683in}}%
\pgfpathlineto{\pgfqpoint{3.757990in}{1.612719in}}%
\pgfpathlineto{\pgfqpoint{3.765721in}{1.614986in}}%
\pgfpathlineto{\pgfqpoint{3.776030in}{1.624593in}}%
\pgfpathlineto{\pgfqpoint{3.786339in}{1.627138in}}%
\pgfpathlineto{\pgfqpoint{3.794070in}{1.634640in}}%
\pgfpathlineto{\pgfqpoint{3.801802in}{1.637240in}}%
\pgfpathlineto{\pgfqpoint{3.812111in}{1.647263in}}%
\pgfpathlineto{\pgfqpoint{3.819843in}{1.649901in}}%
\pgfpathlineto{\pgfqpoint{3.827574in}{1.657574in}}%
\pgfpathlineto{\pgfqpoint{3.830151in}{1.659812in}}%
\pgfpathlineto{\pgfqpoint{3.837883in}{1.662249in}}%
\pgfpathlineto{\pgfqpoint{3.845615in}{1.669073in}}%
\pgfpathlineto{\pgfqpoint{3.848192in}{1.671615in}}%
\pgfpathlineto{\pgfqpoint{3.855924in}{1.674282in}}%
\pgfpathlineto{\pgfqpoint{3.866232in}{1.684929in}}%
\pgfpathlineto{\pgfqpoint{3.873964in}{1.687621in}}%
\pgfpathlineto{\pgfqpoint{3.881696in}{1.694475in}}%
\pgfpathlineto{\pgfqpoint{3.884273in}{1.696679in}}%
\pgfpathlineto{\pgfqpoint{3.892005in}{1.698959in}}%
\pgfpathlineto{\pgfqpoint{3.899736in}{1.705404in}}%
\pgfpathlineto{\pgfqpoint{3.910045in}{1.707545in}}%
\pgfpathlineto{\pgfqpoint{3.920354in}{1.716372in}}%
\pgfpathlineto{\pgfqpoint{3.928086in}{1.718554in}}%
\pgfpathlineto{\pgfqpoint{3.938394in}{1.727455in}}%
\pgfpathlineto{\pgfqpoint{3.946126in}{1.729796in}}%
\pgfpathlineto{\pgfqpoint{3.956435in}{1.739169in}}%
\pgfpathlineto{\pgfqpoint{3.964167in}{1.741702in}}%
\pgfpathlineto{\pgfqpoint{3.974475in}{1.751401in}}%
\pgfpathlineto{\pgfqpoint{3.982207in}{1.753966in}}%
\pgfpathlineto{\pgfqpoint{3.992516in}{1.763430in}}%
\pgfpathlineto{\pgfqpoint{4.000247in}{1.765562in}}%
\pgfpathlineto{\pgfqpoint{4.010556in}{1.774511in}}%
\pgfpathlineto{\pgfqpoint{4.018288in}{1.776799in}}%
\pgfpathlineto{\pgfqpoint{4.028597in}{1.785791in}}%
\pgfpathlineto{\pgfqpoint{4.038906in}{1.787876in}}%
\pgfpathlineto{\pgfqpoint{4.046637in}{1.794431in}}%
\pgfpathlineto{\pgfqpoint{4.054369in}{1.796657in}}%
\pgfpathlineto{\pgfqpoint{4.064678in}{1.805226in}}%
\pgfpathlineto{\pgfqpoint{4.072409in}{1.807156in}}%
\pgfpathlineto{\pgfqpoint{4.082718in}{1.814990in}}%
\pgfpathlineto{\pgfqpoint{4.090450in}{1.816963in}}%
\pgfpathlineto{\pgfqpoint{4.100759in}{1.825709in}}%
\pgfpathlineto{\pgfqpoint{4.108490in}{1.828046in}}%
\pgfpathlineto{\pgfqpoint{4.118799in}{1.837695in}}%
\pgfpathlineto{\pgfqpoint{4.126531in}{1.839885in}}%
\pgfpathlineto{\pgfqpoint{4.134263in}{1.847211in}}%
\pgfpathlineto{\pgfqpoint{4.144571in}{1.849778in}}%
\pgfpathlineto{\pgfqpoint{4.154880in}{1.860076in}}%
\pgfpathlineto{\pgfqpoint{4.162612in}{1.862879in}}%
\pgfpathlineto{\pgfqpoint{4.172921in}{1.874150in}}%
\pgfpathlineto{\pgfqpoint{4.180652in}{1.877066in}}%
\pgfpathlineto{\pgfqpoint{4.190961in}{1.888245in}}%
\pgfpathlineto{\pgfqpoint{4.198693in}{1.890875in}}%
\pgfpathlineto{\pgfqpoint{4.209002in}{1.901933in}}%
\pgfpathlineto{\pgfqpoint{4.216733in}{1.904862in}}%
\pgfpathlineto{\pgfqpoint{4.219311in}{1.907850in}}%
\pgfpathlineto{\pgfqpoint{4.227042in}{1.912136in}}%
\pgfpathlineto{\pgfqpoint{4.234774in}{1.913712in}}%
\pgfpathlineto{\pgfqpoint{4.245083in}{1.918404in}}%
\pgfpathlineto{\pgfqpoint{4.252814in}{1.919724in}}%
\pgfpathlineto{\pgfqpoint{4.257969in}{1.921864in}}%
\pgfpathlineto{\pgfqpoint{4.263123in}{1.922708in}}%
\pgfpathlineto{\pgfqpoint{4.276009in}{1.923383in}}%
\pgfpathlineto{\pgfqpoint{4.281164in}{1.924660in}}%
\pgfpathlineto{\pgfqpoint{4.291472in}{1.925656in}}%
\pgfpathlineto{\pgfqpoint{4.299204in}{1.927339in}}%
\pgfpathlineto{\pgfqpoint{4.312090in}{1.928665in}}%
\pgfpathlineto{\pgfqpoint{4.317245in}{1.930071in}}%
\pgfpathlineto{\pgfqpoint{4.327553in}{1.931469in}}%
\pgfpathlineto{\pgfqpoint{4.335285in}{1.933541in}}%
\pgfpathlineto{\pgfqpoint{4.345594in}{1.934778in}}%
\pgfpathlineto{\pgfqpoint{4.353326in}{1.936099in}}%
\pgfpathlineto{\pgfqpoint{4.363634in}{1.936686in}}%
\pgfpathlineto{\pgfqpoint{4.371366in}{1.938374in}}%
\pgfpathlineto{\pgfqpoint{4.381675in}{1.939673in}}%
\pgfpathlineto{\pgfqpoint{4.389407in}{1.941765in}}%
\pgfpathlineto{\pgfqpoint{4.397138in}{1.942610in}}%
\pgfpathlineto{\pgfqpoint{4.407447in}{1.946209in}}%
\pgfpathlineto{\pgfqpoint{4.415179in}{1.947310in}}%
\pgfpathlineto{\pgfqpoint{4.422910in}{1.951037in}}%
\pgfpathlineto{\pgfqpoint{4.425488in}{1.952465in}}%
\pgfpathlineto{\pgfqpoint{4.433219in}{1.953929in}}%
\pgfpathlineto{\pgfqpoint{4.443528in}{1.959982in}}%
\pgfpathlineto{\pgfqpoint{4.451260in}{1.961558in}}%
\pgfpathlineto{\pgfqpoint{4.461569in}{1.967488in}}%
\pgfpathlineto{\pgfqpoint{4.469300in}{1.969162in}}%
\pgfpathlineto{\pgfqpoint{4.479609in}{1.975684in}}%
\pgfpathlineto{\pgfqpoint{4.487341in}{1.977255in}}%
\pgfpathlineto{\pgfqpoint{4.495072in}{1.982464in}}%
\pgfpathlineto{\pgfqpoint{4.497649in}{1.984442in}}%
\pgfpathlineto{\pgfqpoint{4.505381in}{1.986336in}}%
\pgfpathlineto{\pgfqpoint{4.510536in}{1.989846in}}%
\pgfpathlineto{\pgfqpoint{4.515690in}{1.991255in}}%
\pgfpathlineto{\pgfqpoint{4.523422in}{1.992495in}}%
\pgfpathlineto{\pgfqpoint{4.533730in}{1.997540in}}%
\pgfpathlineto{\pgfqpoint{4.541462in}{1.998776in}}%
\pgfpathlineto{\pgfqpoint{4.551771in}{2.002639in}}%
\pgfpathlineto{\pgfqpoint{4.559503in}{2.003502in}}%
\pgfpathlineto{\pgfqpoint{4.569811in}{2.007649in}}%
\pgfpathlineto{\pgfqpoint{4.580120in}{2.008880in}}%
\pgfpathlineto{\pgfqpoint{4.585275in}{2.009949in}}%
\pgfpathlineto{\pgfqpoint{4.598161in}{2.011226in}}%
\pgfpathlineto{\pgfqpoint{4.603315in}{2.012265in}}%
\pgfpathlineto{\pgfqpoint{4.641973in}{2.013265in}}%
\pgfpathlineto{\pgfqpoint{4.670323in}{2.012251in}}%
\pgfpathlineto{\pgfqpoint{4.696095in}{2.010746in}}%
\pgfpathlineto{\pgfqpoint{4.708981in}{2.009740in}}%
\pgfpathlineto{\pgfqpoint{4.714135in}{2.009036in}}%
\pgfpathlineto{\pgfqpoint{4.747639in}{2.007713in}}%
\pgfpathlineto{\pgfqpoint{4.763102in}{2.007265in}}%
\pgfpathlineto{\pgfqpoint{4.814647in}{2.007001in}}%
\pgfpathlineto{\pgfqpoint{4.832687in}{2.006718in}}%
\pgfpathlineto{\pgfqpoint{4.858459in}{2.006238in}}%
\pgfpathlineto{\pgfqpoint{4.886809in}{2.006260in}}%
\pgfpathlineto{\pgfqpoint{4.930621in}{2.009726in}}%
\pgfpathlineto{\pgfqpoint{4.964125in}{2.010628in}}%
\pgfpathlineto{\pgfqpoint{5.018246in}{2.009907in}}%
\pgfpathlineto{\pgfqpoint{5.038864in}{2.009294in}}%
\pgfpathlineto{\pgfqpoint{5.056905in}{2.008823in}}%
\pgfpathlineto{\pgfqpoint{5.105872in}{2.007663in}}%
\pgfpathlineto{\pgfqpoint{5.136798in}{2.007226in}}%
\pgfpathlineto{\pgfqpoint{5.183188in}{2.004748in}}%
\pgfpathlineto{\pgfqpoint{5.208960in}{2.003753in}}%
\pgfpathlineto{\pgfqpoint{5.237309in}{2.001674in}}%
\pgfpathlineto{\pgfqpoint{5.252773in}{2.000848in}}%
\pgfpathlineto{\pgfqpoint{5.255350in}{2.000525in}}%
\pgfpathlineto{\pgfqpoint{5.268236in}{1.999598in}}%
\pgfpathlineto{\pgfqpoint{5.273390in}{1.998958in}}%
\pgfpathlineto{\pgfqpoint{5.286276in}{1.998010in}}%
\pgfpathlineto{\pgfqpoint{5.309471in}{1.995758in}}%
\pgfpathlineto{\pgfqpoint{5.322357in}{1.994797in}}%
\pgfpathlineto{\pgfqpoint{5.327512in}{1.994167in}}%
\pgfpathlineto{\pgfqpoint{5.340398in}{1.993153in}}%
\pgfpathlineto{\pgfqpoint{5.345552in}{1.992437in}}%
\pgfpathlineto{\pgfqpoint{5.358438in}{1.991372in}}%
\pgfpathlineto{\pgfqpoint{5.363593in}{1.990735in}}%
\pgfpathlineto{\pgfqpoint{5.376479in}{1.989844in}}%
\pgfpathlineto{\pgfqpoint{5.399674in}{1.987711in}}%
\pgfpathlineto{\pgfqpoint{5.487299in}{1.985234in}}%
\pgfpathlineto{\pgfqpoint{5.500185in}{1.985812in}}%
\pgfpathlineto{\pgfqpoint{5.543998in}{1.988333in}}%
\pgfpathlineto{\pgfqpoint{5.559461in}{1.989290in}}%
\pgfpathlineto{\pgfqpoint{5.562038in}{1.989724in}}%
\pgfpathlineto{\pgfqpoint{5.572347in}{1.990511in}}%
\pgfpathlineto{\pgfqpoint{5.580079in}{1.991673in}}%
\pgfpathlineto{\pgfqpoint{5.595542in}{1.992718in}}%
\pgfpathlineto{\pgfqpoint{5.598119in}{1.993055in}}%
\pgfpathlineto{\pgfqpoint{5.611005in}{1.994056in}}%
\pgfpathlineto{\pgfqpoint{5.652241in}{1.999365in}}%
\pgfpathlineto{\pgfqpoint{5.662550in}{2.000239in}}%
\pgfpathlineto{\pgfqpoint{5.670281in}{2.001593in}}%
\pgfpathlineto{\pgfqpoint{5.683167in}{2.002453in}}%
\pgfpathlineto{\pgfqpoint{5.688322in}{2.003300in}}%
\pgfpathlineto{\pgfqpoint{5.698630in}{2.004153in}}%
\pgfpathlineto{\pgfqpoint{5.706362in}{2.005560in}}%
\pgfpathlineto{\pgfqpoint{5.716671in}{2.006462in}}%
\pgfpathlineto{\pgfqpoint{5.724403in}{2.007829in}}%
\pgfpathlineto{\pgfqpoint{5.734711in}{2.008847in}}%
\pgfpathlineto{\pgfqpoint{5.742443in}{2.010331in}}%
\pgfpathlineto{\pgfqpoint{5.752752in}{2.011364in}}%
\pgfpathlineto{\pgfqpoint{5.760484in}{2.012879in}}%
\pgfpathlineto{\pgfqpoint{5.770792in}{2.013946in}}%
\pgfpathlineto{\pgfqpoint{5.778524in}{2.015586in}}%
\pgfpathlineto{\pgfqpoint{5.788833in}{2.016677in}}%
\pgfpathlineto{\pgfqpoint{5.796565in}{2.018247in}}%
\pgfpathlineto{\pgfqpoint{5.806873in}{2.019267in}}%
\pgfpathlineto{\pgfqpoint{5.812028in}{2.020321in}}%
\pgfpathlineto{\pgfqpoint{5.824914in}{2.021472in}}%
\pgfpathlineto{\pgfqpoint{5.832646in}{2.023241in}}%
\pgfpathlineto{\pgfqpoint{5.842954in}{2.024422in}}%
\pgfpathlineto{\pgfqpoint{5.850686in}{2.026406in}}%
\pgfpathlineto{\pgfqpoint{5.860995in}{2.027572in}}%
\pgfpathlineto{\pgfqpoint{5.868727in}{2.028773in}}%
\pgfpathlineto{\pgfqpoint{5.881613in}{2.029866in}}%
\pgfpathlineto{\pgfqpoint{5.886767in}{2.030430in}}%
\pgfpathlineto{\pgfqpoint{5.935734in}{2.032496in}}%
\pgfpathlineto{\pgfqpoint{5.948620in}{2.032932in}}%
\pgfpathlineto{\pgfqpoint{6.020782in}{2.034081in}}%
\pgfpathlineto{\pgfqpoint{6.080058in}{2.035079in}}%
\pgfpathlineto{\pgfqpoint{6.085212in}{2.035627in}}%
\pgfpathlineto{\pgfqpoint{6.129025in}{2.036618in}}%
\pgfpathlineto{\pgfqpoint{6.167683in}{2.035542in}}%
\pgfpathlineto{\pgfqpoint{6.201187in}{2.034251in}}%
\pgfpathlineto{\pgfqpoint{6.221805in}{2.032928in}}%
\pgfpathlineto{\pgfqpoint{6.247577in}{2.031294in}}%
\pgfpathlineto{\pgfqpoint{6.275926in}{2.030126in}}%
\pgfpathlineto{\pgfqpoint{6.301698in}{2.028339in}}%
\pgfpathlineto{\pgfqpoint{6.317161in}{2.027452in}}%
\pgfpathlineto{\pgfqpoint{6.337779in}{2.026123in}}%
\pgfpathlineto{\pgfqpoint{6.409941in}{2.024894in}}%
\pgfpathlineto{\pgfqpoint{6.456331in}{2.026731in}}%
\pgfpathlineto{\pgfqpoint{6.464063in}{2.027334in}}%
\pgfpathlineto{\pgfqpoint{6.482103in}{2.027842in}}%
\pgfpathlineto{\pgfqpoint{6.482103in}{2.027842in}}%
\pgfusepath{stroke}%
\end{pgfscope}%
\begin{pgfscope}%
\pgfsetrectcap%
\pgfsetmiterjoin%
\pgfsetlinewidth{0.803000pt}%
\definecolor{currentstroke}{rgb}{1.000000,1.000000,1.000000}%
\pgfsetstrokecolor{currentstroke}%
\pgfsetdash{}{0pt}%
\pgfpathmoveto{\pgfqpoint{0.563921in}{0.521603in}}%
\pgfpathlineto{\pgfqpoint{0.563921in}{3.164103in}}%
\pgfusepath{stroke}%
\end{pgfscope}%
\begin{pgfscope}%
\pgfsetrectcap%
\pgfsetmiterjoin%
\pgfsetlinewidth{0.803000pt}%
\definecolor{currentstroke}{rgb}{1.000000,1.000000,1.000000}%
\pgfsetstrokecolor{currentstroke}%
\pgfsetdash{}{0pt}%
\pgfpathmoveto{\pgfqpoint{6.763921in}{0.521603in}}%
\pgfpathlineto{\pgfqpoint{6.763921in}{3.164103in}}%
\pgfusepath{stroke}%
\end{pgfscope}%
\begin{pgfscope}%
\pgfsetrectcap%
\pgfsetmiterjoin%
\pgfsetlinewidth{0.803000pt}%
\definecolor{currentstroke}{rgb}{1.000000,1.000000,1.000000}%
\pgfsetstrokecolor{currentstroke}%
\pgfsetdash{}{0pt}%
\pgfpathmoveto{\pgfqpoint{0.563921in}{0.521603in}}%
\pgfpathlineto{\pgfqpoint{6.763921in}{0.521603in}}%
\pgfusepath{stroke}%
\end{pgfscope}%
\begin{pgfscope}%
\pgfsetrectcap%
\pgfsetmiterjoin%
\pgfsetlinewidth{0.803000pt}%
\definecolor{currentstroke}{rgb}{1.000000,1.000000,1.000000}%
\pgfsetstrokecolor{currentstroke}%
\pgfsetdash{}{0pt}%
\pgfpathmoveto{\pgfqpoint{0.563921in}{3.164103in}}%
\pgfpathlineto{\pgfqpoint{6.763921in}{3.164103in}}%
\pgfusepath{stroke}%
\end{pgfscope}%
\begin{pgfscope}%
\definecolor{textcolor}{rgb}{0.150000,0.150000,0.150000}%
\pgfsetstrokecolor{textcolor}%
\pgfsetfillcolor{textcolor}%
\pgftext[x=3.663921in,y=3.247437in,,base]{\color{textcolor}\rmfamily\fontsize{12.000000}{14.400000}\selectfont 'Cumulative' Standard Deviation of Stock Prices}%
\end{pgfscope}%
\begin{pgfscope}%
\pgfsetbuttcap%
\pgfsetmiterjoin%
\definecolor{currentfill}{rgb}{0.917647,0.917647,0.949020}%
\pgfsetfillcolor{currentfill}%
\pgfsetfillopacity{0.800000}%
\pgfsetlinewidth{1.003750pt}%
\definecolor{currentstroke}{rgb}{0.800000,0.800000,0.800000}%
\pgfsetstrokecolor{currentstroke}%
\pgfsetstrokeopacity{0.800000}%
\pgfsetdash{}{0pt}%
\pgfpathmoveto{\pgfqpoint{0.661143in}{1.014420in}}%
\pgfpathlineto{\pgfqpoint{1.532224in}{1.014420in}}%
\pgfpathquadraticcurveto{\pgfqpoint{1.560001in}{1.014420in}}{\pgfqpoint{1.560001in}{1.042198in}}%
\pgfpathlineto{\pgfqpoint{1.560001in}{3.066881in}}%
\pgfpathquadraticcurveto{\pgfqpoint{1.560001in}{3.094659in}}{\pgfqpoint{1.532224in}{3.094659in}}%
\pgfpathlineto{\pgfqpoint{0.661143in}{3.094659in}}%
\pgfpathquadraticcurveto{\pgfqpoint{0.633366in}{3.094659in}}{\pgfqpoint{0.633366in}{3.066881in}}%
\pgfpathlineto{\pgfqpoint{0.633366in}{1.042198in}}%
\pgfpathquadraticcurveto{\pgfqpoint{0.633366in}{1.014420in}}{\pgfqpoint{0.661143in}{1.014420in}}%
\pgfpathclose%
\pgfusepath{stroke,fill}%
\end{pgfscope}%
\begin{pgfscope}%
\pgfsetroundcap%
\pgfsetroundjoin%
\pgfsetlinewidth{1.505625pt}%
\definecolor{currentstroke}{rgb}{0.121569,0.466667,0.705882}%
\pgfsetstrokecolor{currentstroke}%
\pgfsetdash{}{0pt}%
\pgfpathmoveto{\pgfqpoint{0.688921in}{2.982191in}}%
\pgfpathlineto{\pgfqpoint{0.966699in}{2.982191in}}%
\pgfusepath{stroke}%
\end{pgfscope}%
\begin{pgfscope}%
\definecolor{textcolor}{rgb}{0.150000,0.150000,0.150000}%
\pgfsetstrokecolor{textcolor}%
\pgfsetfillcolor{textcolor}%
\pgftext[x=1.077810in,y=2.933580in,left,base]{\color{textcolor}\rmfamily\fontsize{10.000000}{12.000000}\selectfont MMM}%
\end{pgfscope}%
\begin{pgfscope}%
\pgfsetroundcap%
\pgfsetroundjoin%
\pgfsetlinewidth{1.505625pt}%
\definecolor{currentstroke}{rgb}{1.000000,0.498039,0.054902}%
\pgfsetstrokecolor{currentstroke}%
\pgfsetdash{}{0pt}%
\pgfpathmoveto{\pgfqpoint{0.688921in}{2.778334in}}%
\pgfpathlineto{\pgfqpoint{0.966699in}{2.778334in}}%
\pgfusepath{stroke}%
\end{pgfscope}%
\begin{pgfscope}%
\definecolor{textcolor}{rgb}{0.150000,0.150000,0.150000}%
\pgfsetstrokecolor{textcolor}%
\pgfsetfillcolor{textcolor}%
\pgftext[x=1.077810in,y=2.729723in,left,base]{\color{textcolor}\rmfamily\fontsize{10.000000}{12.000000}\selectfont AXP}%
\end{pgfscope}%
\begin{pgfscope}%
\pgfsetroundcap%
\pgfsetroundjoin%
\pgfsetlinewidth{1.505625pt}%
\definecolor{currentstroke}{rgb}{0.172549,0.627451,0.172549}%
\pgfsetstrokecolor{currentstroke}%
\pgfsetdash{}{0pt}%
\pgfpathmoveto{\pgfqpoint{0.688921in}{2.574477in}}%
\pgfpathlineto{\pgfqpoint{0.966699in}{2.574477in}}%
\pgfusepath{stroke}%
\end{pgfscope}%
\begin{pgfscope}%
\definecolor{textcolor}{rgb}{0.150000,0.150000,0.150000}%
\pgfsetstrokecolor{textcolor}%
\pgfsetfillcolor{textcolor}%
\pgftext[x=1.077810in,y=2.525866in,left,base]{\color{textcolor}\rmfamily\fontsize{10.000000}{12.000000}\selectfont GE}%
\end{pgfscope}%
\begin{pgfscope}%
\pgfsetroundcap%
\pgfsetroundjoin%
\pgfsetlinewidth{1.505625pt}%
\definecolor{currentstroke}{rgb}{0.839216,0.152941,0.156863}%
\pgfsetstrokecolor{currentstroke}%
\pgfsetdash{}{0pt}%
\pgfpathmoveto{\pgfqpoint{0.688921in}{2.370620in}}%
\pgfpathlineto{\pgfqpoint{0.966699in}{2.370620in}}%
\pgfusepath{stroke}%
\end{pgfscope}%
\begin{pgfscope}%
\definecolor{textcolor}{rgb}{0.150000,0.150000,0.150000}%
\pgfsetstrokecolor{textcolor}%
\pgfsetfillcolor{textcolor}%
\pgftext[x=1.077810in,y=2.322009in,left,base]{\color{textcolor}\rmfamily\fontsize{10.000000}{12.000000}\selectfont INTC}%
\end{pgfscope}%
\begin{pgfscope}%
\pgfsetroundcap%
\pgfsetroundjoin%
\pgfsetlinewidth{1.505625pt}%
\definecolor{currentstroke}{rgb}{0.580392,0.403922,0.741176}%
\pgfsetstrokecolor{currentstroke}%
\pgfsetdash{}{0pt}%
\pgfpathmoveto{\pgfqpoint{0.688921in}{2.166762in}}%
\pgfpathlineto{\pgfqpoint{0.966699in}{2.166762in}}%
\pgfusepath{stroke}%
\end{pgfscope}%
\begin{pgfscope}%
\definecolor{textcolor}{rgb}{0.150000,0.150000,0.150000}%
\pgfsetstrokecolor{textcolor}%
\pgfsetfillcolor{textcolor}%
\pgftext[x=1.077810in,y=2.118151in,left,base]{\color{textcolor}\rmfamily\fontsize{10.000000}{12.000000}\selectfont JNJ}%
\end{pgfscope}%
\begin{pgfscope}%
\pgfsetroundcap%
\pgfsetroundjoin%
\pgfsetlinewidth{1.505625pt}%
\definecolor{currentstroke}{rgb}{0.549020,0.337255,0.294118}%
\pgfsetstrokecolor{currentstroke}%
\pgfsetdash{}{0pt}%
\pgfpathmoveto{\pgfqpoint{0.688921in}{1.962905in}}%
\pgfpathlineto{\pgfqpoint{0.966699in}{1.962905in}}%
\pgfusepath{stroke}%
\end{pgfscope}%
\begin{pgfscope}%
\definecolor{textcolor}{rgb}{0.150000,0.150000,0.150000}%
\pgfsetstrokecolor{textcolor}%
\pgfsetfillcolor{textcolor}%
\pgftext[x=1.077810in,y=1.914294in,left,base]{\color{textcolor}\rmfamily\fontsize{10.000000}{12.000000}\selectfont PG}%
\end{pgfscope}%
\begin{pgfscope}%
\pgfsetroundcap%
\pgfsetroundjoin%
\pgfsetlinewidth{1.505625pt}%
\definecolor{currentstroke}{rgb}{0.890196,0.466667,0.760784}%
\pgfsetstrokecolor{currentstroke}%
\pgfsetdash{}{0pt}%
\pgfpathmoveto{\pgfqpoint{0.688921in}{1.759048in}}%
\pgfpathlineto{\pgfqpoint{0.966699in}{1.759048in}}%
\pgfusepath{stroke}%
\end{pgfscope}%
\begin{pgfscope}%
\definecolor{textcolor}{rgb}{0.150000,0.150000,0.150000}%
\pgfsetstrokecolor{textcolor}%
\pgfsetfillcolor{textcolor}%
\pgftext[x=1.077810in,y=1.710437in,left,base]{\color{textcolor}\rmfamily\fontsize{10.000000}{12.000000}\selectfont UTX}%
\end{pgfscope}%
\begin{pgfscope}%
\pgfsetroundcap%
\pgfsetroundjoin%
\pgfsetlinewidth{1.505625pt}%
\definecolor{currentstroke}{rgb}{0.498039,0.498039,0.498039}%
\pgfsetstrokecolor{currentstroke}%
\pgfsetdash{}{0pt}%
\pgfpathmoveto{\pgfqpoint{0.688921in}{1.555191in}}%
\pgfpathlineto{\pgfqpoint{0.966699in}{1.555191in}}%
\pgfusepath{stroke}%
\end{pgfscope}%
\begin{pgfscope}%
\definecolor{textcolor}{rgb}{0.150000,0.150000,0.150000}%
\pgfsetstrokecolor{textcolor}%
\pgfsetfillcolor{textcolor}%
\pgftext[x=1.077810in,y=1.506580in,left,base]{\color{textcolor}\rmfamily\fontsize{10.000000}{12.000000}\selectfont VZ}%
\end{pgfscope}%
\begin{pgfscope}%
\pgfsetroundcap%
\pgfsetroundjoin%
\pgfsetlinewidth{1.505625pt}%
\definecolor{currentstroke}{rgb}{0.737255,0.741176,0.133333}%
\pgfsetstrokecolor{currentstroke}%
\pgfsetdash{}{0pt}%
\pgfpathmoveto{\pgfqpoint{0.688921in}{1.351334in}}%
\pgfpathlineto{\pgfqpoint{0.966699in}{1.351334in}}%
\pgfusepath{stroke}%
\end{pgfscope}%
\begin{pgfscope}%
\definecolor{textcolor}{rgb}{0.150000,0.150000,0.150000}%
\pgfsetstrokecolor{textcolor}%
\pgfsetfillcolor{textcolor}%
\pgftext[x=1.077810in,y=1.302722in,left,base]{\color{textcolor}\rmfamily\fontsize{10.000000}{12.000000}\selectfont V}%
\end{pgfscope}%
\begin{pgfscope}%
\pgfsetroundcap%
\pgfsetroundjoin%
\pgfsetlinewidth{1.505625pt}%
\definecolor{currentstroke}{rgb}{0.090196,0.745098,0.811765}%
\pgfsetstrokecolor{currentstroke}%
\pgfsetdash{}{0pt}%
\pgfpathmoveto{\pgfqpoint{0.688921in}{1.147476in}}%
\pgfpathlineto{\pgfqpoint{0.966699in}{1.147476in}}%
\pgfusepath{stroke}%
\end{pgfscope}%
\begin{pgfscope}%
\definecolor{textcolor}{rgb}{0.150000,0.150000,0.150000}%
\pgfsetstrokecolor{textcolor}%
\pgfsetfillcolor{textcolor}%
\pgftext[x=1.077810in,y=1.098865in,left,base]{\color{textcolor}\rmfamily\fontsize{10.000000}{12.000000}\selectfont DIS}%
\end{pgfscope}%
\end{pgfpicture}%
\makeatother%
\endgroup%

    \end{adjustbox}  
    \caption{Standard deviation for the time series of stock prices. The value of the graph at point t is calculated as the standard deviation of all recorded values of the respective stocks up to that point t.}
    \label{fig:cum_sd_all}
\end{figure}{}

\subsection{Data Transformation}

In order to obtain weakly stationary time series the data needs to be transformed. There are different ways to proceed that are often equivalent or very similar to each other. Economists usually work with either returns or log-returns, albeit the nomenclature may be a bit confusing. (Daily) returns can be calculated as
\begin{equation*}
    r^{\scriptscriptstyle{(1)} }_t = \frac{r_t}{r_{t-1}} \qquad \text{or as} \qquad r^{\scriptscriptstyle{(2)}}_t = \frac{r_t - r_{t-1}}{r_{t-1}} = r^{\scriptscriptstyle{(1)}}_t - 1
\end{equation*}{}
Usually $r^{\scriptscriptstyle{(2)}}_t$ is used and is called either returns or log-returns, even though no logging takes place. For increased conceptual clarity, $r^{\scriptscriptstyle{(1)}}_t$ will be called returns and $log(r^{\scriptscriptstyle{(1)}}_t)$ will be called log-returns. $r^{\scriptscriptstyle{(2)}}_t$ will not be explicitly used. Log-returns are computationally convenient and numerically stable. For very small values they are also very close to the returns $r^{\scriptscriptstyle{(2)}}_t$ often used in economic literature as $ \log(r^{\scriptscriptstyle{(1)}}_t) \approx r^{\scriptscriptstyle{(1)}}_t - 1 = r^{\scriptscriptstyle{(2)}}_t$ for values of $r^{\scriptscriptstyle{(1)}}_t$ close to 1. Using returns or log-returns instead of stock prices can make the time series stationary. Figure \ref{fig:returns_all_stocks} illustrates that the trends in the time series have vanished after looking at log-returns. The data visually now looks like white noise. 

\begin{figure}[H]
    \figuretitle{Log-returns of stock PG}
    \centering
    \begin{adjustbox}{width=.9\textwidth,center}
    %% Creator: Matplotlib, PGF backend
%%
%% To include the figure in your LaTeX document, write
%%   \input{<filename>.pgf}
%%
%% Make sure the required packages are loaded in your preamble
%%   \usepackage{pgf}
%%
%% Figures using additional raster images can only be included by \input if
%% they are in the same directory as the main LaTeX file. For loading figures
%% from other directories you can use the `import` package
%%   \usepackage{import}
%% and then include the figures with
%%   \import{<path to file>}{<filename>.pgf}
%%
%% Matplotlib used the following preamble
%%   \usepackage{fontspec}
%%   \setmainfont{DejaVuSerif.ttf}[Path=/opt/tljh/user/lib/python3.6/site-packages/matplotlib/mpl-data/fonts/ttf/]
%%   \setsansfont{DejaVuSans.ttf}[Path=/opt/tljh/user/lib/python3.6/site-packages/matplotlib/mpl-data/fonts/ttf/]
%%   \setmonofont{DejaVuSansMono.ttf}[Path=/opt/tljh/user/lib/python3.6/site-packages/matplotlib/mpl-data/fonts/ttf/]
%%
\begingroup%
\makeatletter%
\begin{pgfpicture}%
\pgfpathrectangle{\pgfpointorigin}{\pgfqpoint{7.114776in}{3.096564in}}%
\pgfusepath{use as bounding box, clip}%
\begin{pgfscope}%
\pgfsetbuttcap%
\pgfsetmiterjoin%
\definecolor{currentfill}{rgb}{1.000000,1.000000,1.000000}%
\pgfsetfillcolor{currentfill}%
\pgfsetlinewidth{0.000000pt}%
\definecolor{currentstroke}{rgb}{1.000000,1.000000,1.000000}%
\pgfsetstrokecolor{currentstroke}%
\pgfsetdash{}{0pt}%
\pgfpathmoveto{\pgfqpoint{0.000000in}{0.000000in}}%
\pgfpathlineto{\pgfqpoint{7.114776in}{0.000000in}}%
\pgfpathlineto{\pgfqpoint{7.114776in}{3.096564in}}%
\pgfpathlineto{\pgfqpoint{0.000000in}{3.096564in}}%
\pgfpathclose%
\pgfusepath{fill}%
\end{pgfscope}%
\begin{pgfscope}%
\pgfsetbuttcap%
\pgfsetmiterjoin%
\definecolor{currentfill}{rgb}{0.917647,0.917647,0.949020}%
\pgfsetfillcolor{currentfill}%
\pgfsetlinewidth{0.000000pt}%
\definecolor{currentstroke}{rgb}{0.000000,0.000000,0.000000}%
\pgfsetstrokecolor{currentstroke}%
\pgfsetstrokeopacity{0.000000}%
\pgfsetdash{}{0pt}%
\pgfpathmoveto{\pgfqpoint{0.814776in}{0.521603in}}%
\pgfpathlineto{\pgfqpoint{7.014776in}{0.521603in}}%
\pgfpathlineto{\pgfqpoint{7.014776in}{2.786603in}}%
\pgfpathlineto{\pgfqpoint{0.814776in}{2.786603in}}%
\pgfpathclose%
\pgfusepath{fill}%
\end{pgfscope}%
\begin{pgfscope}%
\pgfpathrectangle{\pgfqpoint{0.814776in}{0.521603in}}{\pgfqpoint{6.200000in}{2.265000in}}%
\pgfusepath{clip}%
\pgfsetroundcap%
\pgfsetroundjoin%
\pgfsetlinewidth{0.803000pt}%
\definecolor{currentstroke}{rgb}{1.000000,1.000000,1.000000}%
\pgfsetstrokecolor{currentstroke}%
\pgfsetdash{}{0pt}%
\pgfpathmoveto{\pgfqpoint{1.088859in}{0.521603in}}%
\pgfpathlineto{\pgfqpoint{1.088859in}{2.786603in}}%
\pgfusepath{stroke}%
\end{pgfscope}%
\begin{pgfscope}%
\definecolor{textcolor}{rgb}{0.150000,0.150000,0.150000}%
\pgfsetstrokecolor{textcolor}%
\pgfsetfillcolor{textcolor}%
\pgftext[x=1.088859in,y=0.424381in,,top]{\color{textcolor}\rmfamily\fontsize{10.000000}{12.000000}\selectfont 2012}%
\end{pgfscope}%
\begin{pgfscope}%
\pgfpathrectangle{\pgfqpoint{0.814776in}{0.521603in}}{\pgfqpoint{6.200000in}{2.265000in}}%
\pgfusepath{clip}%
\pgfsetroundcap%
\pgfsetroundjoin%
\pgfsetlinewidth{0.803000pt}%
\definecolor{currentstroke}{rgb}{1.000000,1.000000,1.000000}%
\pgfsetstrokecolor{currentstroke}%
\pgfsetdash{}{0pt}%
\pgfpathmoveto{\pgfqpoint{2.032550in}{0.521603in}}%
\pgfpathlineto{\pgfqpoint{2.032550in}{2.786603in}}%
\pgfusepath{stroke}%
\end{pgfscope}%
\begin{pgfscope}%
\definecolor{textcolor}{rgb}{0.150000,0.150000,0.150000}%
\pgfsetstrokecolor{textcolor}%
\pgfsetfillcolor{textcolor}%
\pgftext[x=2.032550in,y=0.424381in,,top]{\color{textcolor}\rmfamily\fontsize{10.000000}{12.000000}\selectfont 2013}%
\end{pgfscope}%
\begin{pgfscope}%
\pgfpathrectangle{\pgfqpoint{0.814776in}{0.521603in}}{\pgfqpoint{6.200000in}{2.265000in}}%
\pgfusepath{clip}%
\pgfsetroundcap%
\pgfsetroundjoin%
\pgfsetlinewidth{0.803000pt}%
\definecolor{currentstroke}{rgb}{1.000000,1.000000,1.000000}%
\pgfsetstrokecolor{currentstroke}%
\pgfsetdash{}{0pt}%
\pgfpathmoveto{\pgfqpoint{2.973663in}{0.521603in}}%
\pgfpathlineto{\pgfqpoint{2.973663in}{2.786603in}}%
\pgfusepath{stroke}%
\end{pgfscope}%
\begin{pgfscope}%
\definecolor{textcolor}{rgb}{0.150000,0.150000,0.150000}%
\pgfsetstrokecolor{textcolor}%
\pgfsetfillcolor{textcolor}%
\pgftext[x=2.973663in,y=0.424381in,,top]{\color{textcolor}\rmfamily\fontsize{10.000000}{12.000000}\selectfont 2014}%
\end{pgfscope}%
\begin{pgfscope}%
\pgfpathrectangle{\pgfqpoint{0.814776in}{0.521603in}}{\pgfqpoint{6.200000in}{2.265000in}}%
\pgfusepath{clip}%
\pgfsetroundcap%
\pgfsetroundjoin%
\pgfsetlinewidth{0.803000pt}%
\definecolor{currentstroke}{rgb}{1.000000,1.000000,1.000000}%
\pgfsetstrokecolor{currentstroke}%
\pgfsetdash{}{0pt}%
\pgfpathmoveto{\pgfqpoint{3.914776in}{0.521603in}}%
\pgfpathlineto{\pgfqpoint{3.914776in}{2.786603in}}%
\pgfusepath{stroke}%
\end{pgfscope}%
\begin{pgfscope}%
\definecolor{textcolor}{rgb}{0.150000,0.150000,0.150000}%
\pgfsetstrokecolor{textcolor}%
\pgfsetfillcolor{textcolor}%
\pgftext[x=3.914776in,y=0.424381in,,top]{\color{textcolor}\rmfamily\fontsize{10.000000}{12.000000}\selectfont 2015}%
\end{pgfscope}%
\begin{pgfscope}%
\pgfpathrectangle{\pgfqpoint{0.814776in}{0.521603in}}{\pgfqpoint{6.200000in}{2.265000in}}%
\pgfusepath{clip}%
\pgfsetroundcap%
\pgfsetroundjoin%
\pgfsetlinewidth{0.803000pt}%
\definecolor{currentstroke}{rgb}{1.000000,1.000000,1.000000}%
\pgfsetstrokecolor{currentstroke}%
\pgfsetdash{}{0pt}%
\pgfpathmoveto{\pgfqpoint{4.855889in}{0.521603in}}%
\pgfpathlineto{\pgfqpoint{4.855889in}{2.786603in}}%
\pgfusepath{stroke}%
\end{pgfscope}%
\begin{pgfscope}%
\definecolor{textcolor}{rgb}{0.150000,0.150000,0.150000}%
\pgfsetstrokecolor{textcolor}%
\pgfsetfillcolor{textcolor}%
\pgftext[x=4.855889in,y=0.424381in,,top]{\color{textcolor}\rmfamily\fontsize{10.000000}{12.000000}\selectfont 2016}%
\end{pgfscope}%
\begin{pgfscope}%
\pgfpathrectangle{\pgfqpoint{0.814776in}{0.521603in}}{\pgfqpoint{6.200000in}{2.265000in}}%
\pgfusepath{clip}%
\pgfsetroundcap%
\pgfsetroundjoin%
\pgfsetlinewidth{0.803000pt}%
\definecolor{currentstroke}{rgb}{1.000000,1.000000,1.000000}%
\pgfsetstrokecolor{currentstroke}%
\pgfsetdash{}{0pt}%
\pgfpathmoveto{\pgfqpoint{5.799580in}{0.521603in}}%
\pgfpathlineto{\pgfqpoint{5.799580in}{2.786603in}}%
\pgfusepath{stroke}%
\end{pgfscope}%
\begin{pgfscope}%
\definecolor{textcolor}{rgb}{0.150000,0.150000,0.150000}%
\pgfsetstrokecolor{textcolor}%
\pgfsetfillcolor{textcolor}%
\pgftext[x=5.799580in,y=0.424381in,,top]{\color{textcolor}\rmfamily\fontsize{10.000000}{12.000000}\selectfont 2017}%
\end{pgfscope}%
\begin{pgfscope}%
\pgfpathrectangle{\pgfqpoint{0.814776in}{0.521603in}}{\pgfqpoint{6.200000in}{2.265000in}}%
\pgfusepath{clip}%
\pgfsetroundcap%
\pgfsetroundjoin%
\pgfsetlinewidth{0.803000pt}%
\definecolor{currentstroke}{rgb}{1.000000,1.000000,1.000000}%
\pgfsetstrokecolor{currentstroke}%
\pgfsetdash{}{0pt}%
\pgfpathmoveto{\pgfqpoint{6.740693in}{0.521603in}}%
\pgfpathlineto{\pgfqpoint{6.740693in}{2.786603in}}%
\pgfusepath{stroke}%
\end{pgfscope}%
\begin{pgfscope}%
\definecolor{textcolor}{rgb}{0.150000,0.150000,0.150000}%
\pgfsetstrokecolor{textcolor}%
\pgfsetfillcolor{textcolor}%
\pgftext[x=6.740693in,y=0.424381in,,top]{\color{textcolor}\rmfamily\fontsize{10.000000}{12.000000}\selectfont 2018}%
\end{pgfscope}%
\begin{pgfscope}%
\definecolor{textcolor}{rgb}{0.150000,0.150000,0.150000}%
\pgfsetstrokecolor{textcolor}%
\pgfsetfillcolor{textcolor}%
\pgftext[x=3.914776in,y=0.234413in,,top]{\color{textcolor}\rmfamily\fontsize{10.000000}{12.000000}\selectfont Year}%
\end{pgfscope}%
\begin{pgfscope}%
\pgfpathrectangle{\pgfqpoint{0.814776in}{0.521603in}}{\pgfqpoint{6.200000in}{2.265000in}}%
\pgfusepath{clip}%
\pgfsetroundcap%
\pgfsetroundjoin%
\pgfsetlinewidth{0.803000pt}%
\definecolor{currentstroke}{rgb}{1.000000,1.000000,1.000000}%
\pgfsetstrokecolor{currentstroke}%
\pgfsetdash{}{0pt}%
\pgfpathmoveto{\pgfqpoint{0.814776in}{0.637025in}}%
\pgfpathlineto{\pgfqpoint{7.014776in}{0.637025in}}%
\pgfusepath{stroke}%
\end{pgfscope}%
\begin{pgfscope}%
\definecolor{textcolor}{rgb}{0.150000,0.150000,0.150000}%
\pgfsetstrokecolor{textcolor}%
\pgfsetfillcolor{textcolor}%
\pgftext[x=0.291935in,y=0.584263in,left,base]{\color{textcolor}\rmfamily\fontsize{10.000000}{12.000000}\selectfont −0.06}%
\end{pgfscope}%
\begin{pgfscope}%
\pgfpathrectangle{\pgfqpoint{0.814776in}{0.521603in}}{\pgfqpoint{6.200000in}{2.265000in}}%
\pgfusepath{clip}%
\pgfsetroundcap%
\pgfsetroundjoin%
\pgfsetlinewidth{0.803000pt}%
\definecolor{currentstroke}{rgb}{1.000000,1.000000,1.000000}%
\pgfsetstrokecolor{currentstroke}%
\pgfsetdash{}{0pt}%
\pgfpathmoveto{\pgfqpoint{0.814776in}{1.048363in}}%
\pgfpathlineto{\pgfqpoint{7.014776in}{1.048363in}}%
\pgfusepath{stroke}%
\end{pgfscope}%
\begin{pgfscope}%
\definecolor{textcolor}{rgb}{0.150000,0.150000,0.150000}%
\pgfsetstrokecolor{textcolor}%
\pgfsetfillcolor{textcolor}%
\pgftext[x=0.291935in,y=0.995601in,left,base]{\color{textcolor}\rmfamily\fontsize{10.000000}{12.000000}\selectfont −0.04}%
\end{pgfscope}%
\begin{pgfscope}%
\pgfpathrectangle{\pgfqpoint{0.814776in}{0.521603in}}{\pgfqpoint{6.200000in}{2.265000in}}%
\pgfusepath{clip}%
\pgfsetroundcap%
\pgfsetroundjoin%
\pgfsetlinewidth{0.803000pt}%
\definecolor{currentstroke}{rgb}{1.000000,1.000000,1.000000}%
\pgfsetstrokecolor{currentstroke}%
\pgfsetdash{}{0pt}%
\pgfpathmoveto{\pgfqpoint{0.814776in}{1.459701in}}%
\pgfpathlineto{\pgfqpoint{7.014776in}{1.459701in}}%
\pgfusepath{stroke}%
\end{pgfscope}%
\begin{pgfscope}%
\definecolor{textcolor}{rgb}{0.150000,0.150000,0.150000}%
\pgfsetstrokecolor{textcolor}%
\pgfsetfillcolor{textcolor}%
\pgftext[x=0.291935in,y=1.406939in,left,base]{\color{textcolor}\rmfamily\fontsize{10.000000}{12.000000}\selectfont −0.02}%
\end{pgfscope}%
\begin{pgfscope}%
\pgfpathrectangle{\pgfqpoint{0.814776in}{0.521603in}}{\pgfqpoint{6.200000in}{2.265000in}}%
\pgfusepath{clip}%
\pgfsetroundcap%
\pgfsetroundjoin%
\pgfsetlinewidth{0.803000pt}%
\definecolor{currentstroke}{rgb}{1.000000,1.000000,1.000000}%
\pgfsetstrokecolor{currentstroke}%
\pgfsetdash{}{0pt}%
\pgfpathmoveto{\pgfqpoint{0.814776in}{1.871039in}}%
\pgfpathlineto{\pgfqpoint{7.014776in}{1.871039in}}%
\pgfusepath{stroke}%
\end{pgfscope}%
\begin{pgfscope}%
\definecolor{textcolor}{rgb}{0.150000,0.150000,0.150000}%
\pgfsetstrokecolor{textcolor}%
\pgfsetfillcolor{textcolor}%
\pgftext[x=0.408309in,y=1.818277in,left,base]{\color{textcolor}\rmfamily\fontsize{10.000000}{12.000000}\selectfont 0.00}%
\end{pgfscope}%
\begin{pgfscope}%
\pgfpathrectangle{\pgfqpoint{0.814776in}{0.521603in}}{\pgfqpoint{6.200000in}{2.265000in}}%
\pgfusepath{clip}%
\pgfsetroundcap%
\pgfsetroundjoin%
\pgfsetlinewidth{0.803000pt}%
\definecolor{currentstroke}{rgb}{1.000000,1.000000,1.000000}%
\pgfsetstrokecolor{currentstroke}%
\pgfsetdash{}{0pt}%
\pgfpathmoveto{\pgfqpoint{0.814776in}{2.282377in}}%
\pgfpathlineto{\pgfqpoint{7.014776in}{2.282377in}}%
\pgfusepath{stroke}%
\end{pgfscope}%
\begin{pgfscope}%
\definecolor{textcolor}{rgb}{0.150000,0.150000,0.150000}%
\pgfsetstrokecolor{textcolor}%
\pgfsetfillcolor{textcolor}%
\pgftext[x=0.408309in,y=2.229615in,left,base]{\color{textcolor}\rmfamily\fontsize{10.000000}{12.000000}\selectfont 0.02}%
\end{pgfscope}%
\begin{pgfscope}%
\pgfpathrectangle{\pgfqpoint{0.814776in}{0.521603in}}{\pgfqpoint{6.200000in}{2.265000in}}%
\pgfusepath{clip}%
\pgfsetroundcap%
\pgfsetroundjoin%
\pgfsetlinewidth{0.803000pt}%
\definecolor{currentstroke}{rgb}{1.000000,1.000000,1.000000}%
\pgfsetstrokecolor{currentstroke}%
\pgfsetdash{}{0pt}%
\pgfpathmoveto{\pgfqpoint{0.814776in}{2.693715in}}%
\pgfpathlineto{\pgfqpoint{7.014776in}{2.693715in}}%
\pgfusepath{stroke}%
\end{pgfscope}%
\begin{pgfscope}%
\definecolor{textcolor}{rgb}{0.150000,0.150000,0.150000}%
\pgfsetstrokecolor{textcolor}%
\pgfsetfillcolor{textcolor}%
\pgftext[x=0.408309in,y=2.640953in,left,base]{\color{textcolor}\rmfamily\fontsize{10.000000}{12.000000}\selectfont 0.04}%
\end{pgfscope}%
\begin{pgfscope}%
\definecolor{textcolor}{rgb}{0.150000,0.150000,0.150000}%
\pgfsetstrokecolor{textcolor}%
\pgfsetfillcolor{textcolor}%
\pgftext[x=0.236379in,y=1.654103in,,bottom,rotate=90.000000]{\color{textcolor}\rmfamily\fontsize{10.000000}{12.000000}\selectfont Interday Price Change}%
\end{pgfscope}%
\begin{pgfscope}%
\pgfpathrectangle{\pgfqpoint{0.814776in}{0.521603in}}{\pgfqpoint{6.200000in}{2.265000in}}%
\pgfusepath{clip}%
\pgfsetroundcap%
\pgfsetroundjoin%
\pgfsetlinewidth{1.505625pt}%
\definecolor{currentstroke}{rgb}{0.549020,0.337255,0.294118}%
\pgfsetstrokecolor{currentstroke}%
\pgfsetdash{}{0pt}%
\pgfpathmoveto{\pgfqpoint{1.096594in}{1.863214in}}%
\pgfpathlineto{\pgfqpoint{1.099172in}{1.784771in}}%
\pgfpathlineto{\pgfqpoint{1.101751in}{1.819892in}}%
\pgfpathlineto{\pgfqpoint{1.109486in}{1.957521in}}%
\pgfpathlineto{\pgfqpoint{1.112064in}{1.776677in}}%
\pgfpathlineto{\pgfqpoint{1.114643in}{1.669071in}}%
\pgfpathlineto{\pgfqpoint{1.117221in}{1.910797in}}%
\pgfpathlineto{\pgfqpoint{1.119799in}{1.871039in}}%
\pgfpathlineto{\pgfqpoint{1.130113in}{2.013535in}}%
\pgfpathlineto{\pgfqpoint{1.132691in}{1.961564in}}%
\pgfpathlineto{\pgfqpoint{1.135270in}{1.886742in}}%
\pgfpathlineto{\pgfqpoint{1.137848in}{1.918076in}}%
\pgfpathlineto{\pgfqpoint{1.145583in}{1.483718in}}%
\pgfpathlineto{\pgfqpoint{1.148162in}{1.714849in}}%
\pgfpathlineto{\pgfqpoint{1.150740in}{2.023239in}}%
\pgfpathlineto{\pgfqpoint{1.153318in}{1.811094in}}%
\pgfpathlineto{\pgfqpoint{1.155897in}{1.714361in}}%
\pgfpathlineto{\pgfqpoint{1.163632in}{1.517165in}}%
\pgfpathlineto{\pgfqpoint{1.166210in}{1.817634in}}%
\pgfpathlineto{\pgfqpoint{1.168789in}{1.924443in}}%
\pgfpathlineto{\pgfqpoint{1.171367in}{1.907930in}}%
\pgfpathlineto{\pgfqpoint{1.173946in}{1.690049in}}%
\pgfpathlineto{\pgfqpoint{1.181681in}{2.113368in}}%
\pgfpathlineto{\pgfqpoint{1.184259in}{1.932195in}}%
\pgfpathlineto{\pgfqpoint{1.186838in}{1.850673in}}%
\pgfpathlineto{\pgfqpoint{1.189416in}{2.001030in}}%
\pgfpathlineto{\pgfqpoint{1.191994in}{1.818329in}}%
\pgfpathlineto{\pgfqpoint{1.199730in}{1.984400in}}%
\pgfpathlineto{\pgfqpoint{1.202308in}{1.951630in}}%
\pgfpathlineto{\pgfqpoint{1.204886in}{1.895155in}}%
\pgfpathlineto{\pgfqpoint{1.207465in}{2.074891in}}%
\pgfpathlineto{\pgfqpoint{1.210043in}{1.779356in}}%
\pgfpathlineto{\pgfqpoint{1.220357in}{1.714636in}}%
\pgfpathlineto{\pgfqpoint{1.222935in}{1.879088in}}%
\pgfpathlineto{\pgfqpoint{1.225513in}{2.493305in}}%
\pgfpathlineto{\pgfqpoint{1.228092in}{1.960638in}}%
\pgfpathlineto{\pgfqpoint{1.235827in}{1.867151in}}%
\pgfpathlineto{\pgfqpoint{1.238405in}{2.083768in}}%
\pgfpathlineto{\pgfqpoint{1.240984in}{1.940184in}}%
\pgfpathlineto{\pgfqpoint{1.243562in}{1.577497in}}%
\pgfpathlineto{\pgfqpoint{1.246141in}{1.874928in}}%
\pgfpathlineto{\pgfqpoint{1.253876in}{1.956427in}}%
\pgfpathlineto{\pgfqpoint{1.256454in}{1.836150in}}%
\pgfpathlineto{\pgfqpoint{1.259033in}{1.793296in}}%
\pgfpathlineto{\pgfqpoint{1.261611in}{1.968172in}}%
\pgfpathlineto{\pgfqpoint{1.264189in}{1.878790in}}%
\pgfpathlineto{\pgfqpoint{1.271924in}{2.109878in}}%
\pgfpathlineto{\pgfqpoint{1.274503in}{1.928408in}}%
\pgfpathlineto{\pgfqpoint{1.277081in}{1.855756in}}%
\pgfpathlineto{\pgfqpoint{1.279660in}{1.821291in}}%
\pgfpathlineto{\pgfqpoint{1.282238in}{1.740357in}}%
\pgfpathlineto{\pgfqpoint{1.289973in}{1.859468in}}%
\pgfpathlineto{\pgfqpoint{1.292552in}{1.871039in}}%
\pgfpathlineto{\pgfqpoint{1.295130in}{1.867180in}}%
\pgfpathlineto{\pgfqpoint{1.297708in}{1.967281in}}%
\pgfpathlineto{\pgfqpoint{1.300287in}{1.844136in}}%
\pgfpathlineto{\pgfqpoint{1.308022in}{1.878729in}}%
\pgfpathlineto{\pgfqpoint{1.310600in}{1.778568in}}%
\pgfpathlineto{\pgfqpoint{1.313179in}{1.882620in}}%
\pgfpathlineto{\pgfqpoint{1.315757in}{1.816936in}}%
\pgfpathlineto{\pgfqpoint{1.318336in}{1.932859in}}%
\pgfpathlineto{\pgfqpoint{1.326071in}{1.974941in}}%
\pgfpathlineto{\pgfqpoint{1.328649in}{1.728520in}}%
\pgfpathlineto{\pgfqpoint{1.331227in}{1.925081in}}%
\pgfpathlineto{\pgfqpoint{1.344119in}{1.720223in}}%
\pgfpathlineto{\pgfqpoint{1.346698in}{1.726928in}}%
\pgfpathlineto{\pgfqpoint{1.349276in}{1.898381in}}%
\pgfpathlineto{\pgfqpoint{1.351855in}{1.765377in}}%
\pgfpathlineto{\pgfqpoint{1.354433in}{1.784540in}}%
\pgfpathlineto{\pgfqpoint{1.362168in}{2.168321in}}%
\pgfpathlineto{\pgfqpoint{1.364747in}{1.944693in}}%
\pgfpathlineto{\pgfqpoint{1.367325in}{1.789616in}}%
\pgfpathlineto{\pgfqpoint{1.369903in}{1.816577in}}%
\pgfpathlineto{\pgfqpoint{1.372482in}{2.157286in}}%
\pgfpathlineto{\pgfqpoint{1.380217in}{1.608150in}}%
\pgfpathlineto{\pgfqpoint{1.382795in}{1.979694in}}%
\pgfpathlineto{\pgfqpoint{1.385374in}{2.009901in}}%
\pgfpathlineto{\pgfqpoint{1.387952in}{1.863349in}}%
\pgfpathlineto{\pgfqpoint{1.390530in}{1.111101in}}%
\pgfpathlineto{\pgfqpoint{1.398266in}{1.614050in}}%
\pgfpathlineto{\pgfqpoint{1.400844in}{1.850826in}}%
\pgfpathlineto{\pgfqpoint{1.403422in}{2.000061in}}%
\pgfpathlineto{\pgfqpoint{1.406001in}{2.043148in}}%
\pgfpathlineto{\pgfqpoint{1.408579in}{1.795168in}}%
\pgfpathlineto{\pgfqpoint{1.416314in}{1.863036in}}%
\pgfpathlineto{\pgfqpoint{1.418893in}{1.843005in}}%
\pgfpathlineto{\pgfqpoint{1.421471in}{1.714146in}}%
\pgfpathlineto{\pgfqpoint{1.424050in}{2.019915in}}%
\pgfpathlineto{\pgfqpoint{1.426628in}{1.722163in}}%
\pgfpathlineto{\pgfqpoint{1.434363in}{1.838707in}}%
\pgfpathlineto{\pgfqpoint{1.436942in}{1.919517in}}%
\pgfpathlineto{\pgfqpoint{1.439520in}{2.051821in}}%
\pgfpathlineto{\pgfqpoint{1.442098in}{1.766781in}}%
\pgfpathlineto{\pgfqpoint{1.444677in}{1.729851in}}%
\pgfpathlineto{\pgfqpoint{1.452412in}{1.826465in}}%
\pgfpathlineto{\pgfqpoint{1.454990in}{1.793819in}}%
\pgfpathlineto{\pgfqpoint{1.457569in}{1.621144in}}%
\pgfpathlineto{\pgfqpoint{1.460147in}{1.932770in}}%
\pgfpathlineto{\pgfqpoint{1.462725in}{1.842254in}}%
\pgfpathlineto{\pgfqpoint{1.473039in}{2.022734in}}%
\pgfpathlineto{\pgfqpoint{1.475617in}{1.665779in}}%
\pgfpathlineto{\pgfqpoint{1.478196in}{1.858658in}}%
\pgfpathlineto{\pgfqpoint{1.480774in}{1.626019in}}%
\pgfpathlineto{\pgfqpoint{1.488509in}{1.820845in}}%
\pgfpathlineto{\pgfqpoint{1.491088in}{1.795518in}}%
\pgfpathlineto{\pgfqpoint{1.493666in}{2.080139in}}%
\pgfpathlineto{\pgfqpoint{1.496245in}{2.188952in}}%
\pgfpathlineto{\pgfqpoint{1.498823in}{1.866941in}}%
\pgfpathlineto{\pgfqpoint{1.506558in}{1.801258in}}%
\pgfpathlineto{\pgfqpoint{1.509137in}{1.944917in}}%
\pgfpathlineto{\pgfqpoint{1.511715in}{1.809492in}}%
\pgfpathlineto{\pgfqpoint{1.514293in}{2.067344in}}%
\pgfpathlineto{\pgfqpoint{1.516872in}{1.773120in}}%
\pgfpathlineto{\pgfqpoint{1.524607in}{1.677921in}}%
\pgfpathlineto{\pgfqpoint{1.527185in}{1.846254in}}%
\pgfpathlineto{\pgfqpoint{1.529764in}{1.258559in}}%
\pgfpathlineto{\pgfqpoint{1.532342in}{1.652718in}}%
\pgfpathlineto{\pgfqpoint{1.534920in}{1.896844in}}%
\pgfpathlineto{\pgfqpoint{1.542656in}{1.694053in}}%
\pgfpathlineto{\pgfqpoint{1.545234in}{1.858029in}}%
\pgfpathlineto{\pgfqpoint{1.547812in}{2.112548in}}%
\pgfpathlineto{\pgfqpoint{1.550391in}{1.977947in}}%
\pgfpathlineto{\pgfqpoint{1.552969in}{2.196866in}}%
\pgfpathlineto{\pgfqpoint{1.560704in}{1.850037in}}%
\pgfpathlineto{\pgfqpoint{1.563283in}{1.929789in}}%
\pgfpathlineto{\pgfqpoint{1.568440in}{1.866848in}}%
\pgfpathlineto{\pgfqpoint{1.571018in}{1.845875in}}%
\pgfpathlineto{\pgfqpoint{1.578753in}{1.963155in}}%
\pgfpathlineto{\pgfqpoint{1.581331in}{1.929444in}}%
\pgfpathlineto{\pgfqpoint{1.583910in}{1.762440in}}%
\pgfpathlineto{\pgfqpoint{1.586488in}{2.627532in}}%
\pgfpathlineto{\pgfqpoint{1.589067in}{2.314300in}}%
\pgfpathlineto{\pgfqpoint{1.596802in}{1.783941in}}%
\pgfpathlineto{\pgfqpoint{1.599380in}{2.040933in}}%
\pgfpathlineto{\pgfqpoint{1.601959in}{1.878907in}}%
\pgfpathlineto{\pgfqpoint{1.604537in}{1.906407in}}%
\pgfpathlineto{\pgfqpoint{1.607115in}{1.808119in}}%
\pgfpathlineto{\pgfqpoint{1.617429in}{1.751925in}}%
\pgfpathlineto{\pgfqpoint{1.620007in}{1.871039in}}%
\pgfpathlineto{\pgfqpoint{1.622586in}{2.041555in}}%
\pgfpathlineto{\pgfqpoint{1.625164in}{2.040153in}}%
\pgfpathlineto{\pgfqpoint{1.632899in}{1.874955in}}%
\pgfpathlineto{\pgfqpoint{1.635478in}{1.694059in}}%
\pgfpathlineto{\pgfqpoint{1.638056in}{1.700490in}}%
\pgfpathlineto{\pgfqpoint{1.640634in}{1.711109in}}%
\pgfpathlineto{\pgfqpoint{1.643213in}{2.503430in}}%
\pgfpathlineto{\pgfqpoint{1.650948in}{1.968117in}}%
\pgfpathlineto{\pgfqpoint{1.656105in}{2.028288in}}%
\pgfpathlineto{\pgfqpoint{1.658683in}{1.874859in}}%
\pgfpathlineto{\pgfqpoint{1.661262in}{1.882496in}}%
\pgfpathlineto{\pgfqpoint{1.668997in}{1.779205in}}%
\pgfpathlineto{\pgfqpoint{1.671575in}{1.951416in}}%
\pgfpathlineto{\pgfqpoint{1.674154in}{1.840456in}}%
\pgfpathlineto{\pgfqpoint{1.676732in}{1.981686in}}%
\pgfpathlineto{\pgfqpoint{1.679310in}{1.871039in}}%
\pgfpathlineto{\pgfqpoint{1.687046in}{1.794794in}}%
\pgfpathlineto{\pgfqpoint{1.689624in}{1.878676in}}%
\pgfpathlineto{\pgfqpoint{1.692202in}{1.893933in}}%
\pgfpathlineto{\pgfqpoint{1.694781in}{1.821402in}}%
\pgfpathlineto{\pgfqpoint{1.697359in}{1.973997in}}%
\pgfpathlineto{\pgfqpoint{1.705094in}{1.897648in}}%
\pgfpathlineto{\pgfqpoint{1.707673in}{1.836821in}}%
\pgfpathlineto{\pgfqpoint{1.710251in}{1.832952in}}%
\pgfpathlineto{\pgfqpoint{1.712829in}{1.874851in}}%
\pgfpathlineto{\pgfqpoint{1.715408in}{1.966106in}}%
\pgfpathlineto{\pgfqpoint{1.725721in}{1.935435in}}%
\pgfpathlineto{\pgfqpoint{1.728300in}{1.836972in}}%
\pgfpathlineto{\pgfqpoint{1.730878in}{2.156956in}}%
\pgfpathlineto{\pgfqpoint{1.733457in}{1.956789in}}%
\pgfpathlineto{\pgfqpoint{1.741192in}{1.867318in}}%
\pgfpathlineto{\pgfqpoint{1.743770in}{1.796480in}}%
\pgfpathlineto{\pgfqpoint{1.746349in}{1.822430in}}%
\pgfpathlineto{\pgfqpoint{1.748927in}{2.112939in}}%
\pgfpathlineto{\pgfqpoint{1.751505in}{1.948587in}}%
\pgfpathlineto{\pgfqpoint{1.759240in}{1.896823in}}%
\pgfpathlineto{\pgfqpoint{1.761819in}{1.859992in}}%
\pgfpathlineto{\pgfqpoint{1.764397in}{1.885766in}}%
\pgfpathlineto{\pgfqpoint{1.766976in}{1.959183in}}%
\pgfpathlineto{\pgfqpoint{1.769554in}{1.830686in}}%
\pgfpathlineto{\pgfqpoint{1.777289in}{1.969945in}}%
\pgfpathlineto{\pgfqpoint{1.779868in}{1.819814in}}%
\pgfpathlineto{\pgfqpoint{1.782446in}{1.786605in}}%
\pgfpathlineto{\pgfqpoint{1.785024in}{1.871039in}}%
\pgfpathlineto{\pgfqpoint{1.787603in}{1.889423in}}%
\pgfpathlineto{\pgfqpoint{1.795338in}{1.896750in}}%
\pgfpathlineto{\pgfqpoint{1.797916in}{1.675566in}}%
\pgfpathlineto{\pgfqpoint{1.800495in}{1.981912in}}%
\pgfpathlineto{\pgfqpoint{1.803073in}{1.933603in}}%
\pgfpathlineto{\pgfqpoint{1.805652in}{1.948062in}}%
\pgfpathlineto{\pgfqpoint{1.813387in}{1.713015in}}%
\pgfpathlineto{\pgfqpoint{1.815965in}{1.748936in}}%
\pgfpathlineto{\pgfqpoint{1.818543in}{1.703358in}}%
\pgfpathlineto{\pgfqpoint{1.821122in}{1.829841in}}%
\pgfpathlineto{\pgfqpoint{1.823700in}{1.852285in}}%
\pgfpathlineto{\pgfqpoint{1.831435in}{2.102382in}}%
\pgfpathlineto{\pgfqpoint{1.834014in}{1.959897in}}%
\pgfpathlineto{\pgfqpoint{1.836592in}{2.179053in}}%
\pgfpathlineto{\pgfqpoint{1.841749in}{1.599936in}}%
\pgfpathlineto{\pgfqpoint{1.849484in}{1.852591in}}%
\pgfpathlineto{\pgfqpoint{1.852063in}{1.551128in}}%
\pgfpathlineto{\pgfqpoint{1.854641in}{2.065068in}}%
\pgfpathlineto{\pgfqpoint{1.857219in}{2.460431in}}%
\pgfpathlineto{\pgfqpoint{1.859798in}{1.686158in}}%
\pgfpathlineto{\pgfqpoint{1.872690in}{1.812693in}}%
\pgfpathlineto{\pgfqpoint{1.875268in}{1.874690in}}%
\pgfpathlineto{\pgfqpoint{1.877846in}{1.852775in}}%
\pgfpathlineto{\pgfqpoint{1.885582in}{1.716979in}}%
\pgfpathlineto{\pgfqpoint{1.888160in}{1.948213in}}%
\pgfpathlineto{\pgfqpoint{1.890738in}{1.608942in}}%
\pgfpathlineto{\pgfqpoint{1.893317in}{1.518819in}}%
\pgfpathlineto{\pgfqpoint{1.895895in}{1.905024in}}%
\pgfpathlineto{\pgfqpoint{1.903630in}{1.889895in}}%
\pgfpathlineto{\pgfqpoint{1.906209in}{1.803074in}}%
\pgfpathlineto{\pgfqpoint{1.908787in}{1.768668in}}%
\pgfpathlineto{\pgfqpoint{1.911366in}{1.806321in}}%
\pgfpathlineto{\pgfqpoint{1.913944in}{2.026777in}}%
\pgfpathlineto{\pgfqpoint{1.921679in}{2.208824in}}%
\pgfpathlineto{\pgfqpoint{1.924258in}{1.982412in}}%
\pgfpathlineto{\pgfqpoint{1.926836in}{1.922807in}}%
\pgfpathlineto{\pgfqpoint{1.931993in}{2.204395in}}%
\pgfpathlineto{\pgfqpoint{1.942306in}{1.728606in}}%
\pgfpathlineto{\pgfqpoint{1.944885in}{2.002550in}}%
\pgfpathlineto{\pgfqpoint{1.947463in}{1.889238in}}%
\pgfpathlineto{\pgfqpoint{1.950041in}{1.969037in}}%
\pgfpathlineto{\pgfqpoint{1.957777in}{1.798492in}}%
\pgfpathlineto{\pgfqpoint{1.960355in}{1.790941in}}%
\pgfpathlineto{\pgfqpoint{1.965512in}{2.030698in}}%
\pgfpathlineto{\pgfqpoint{1.968090in}{1.968401in}}%
\pgfpathlineto{\pgfqpoint{1.975825in}{1.853043in}}%
\pgfpathlineto{\pgfqpoint{1.978404in}{1.996677in}}%
\pgfpathlineto{\pgfqpoint{1.980982in}{1.903222in}}%
\pgfpathlineto{\pgfqpoint{1.983561in}{1.684392in}}%
\pgfpathlineto{\pgfqpoint{1.986139in}{1.813267in}}%
\pgfpathlineto{\pgfqpoint{1.993874in}{1.871039in}}%
\pgfpathlineto{\pgfqpoint{1.996452in}{1.881883in}}%
\pgfpathlineto{\pgfqpoint{1.999031in}{1.685897in}}%
\pgfpathlineto{\pgfqpoint{2.001609in}{2.012767in}}%
\pgfpathlineto{\pgfqpoint{2.004188in}{1.546167in}}%
\pgfpathlineto{\pgfqpoint{2.011923in}{1.808397in}}%
\pgfpathlineto{\pgfqpoint{2.017080in}{1.715453in}}%
\pgfpathlineto{\pgfqpoint{2.019658in}{1.863600in}}%
\pgfpathlineto{\pgfqpoint{2.022236in}{1.620288in}}%
\pgfpathlineto{\pgfqpoint{2.029972in}{2.095734in}}%
\pgfpathlineto{\pgfqpoint{2.035128in}{2.320486in}}%
\pgfpathlineto{\pgfqpoint{2.037707in}{1.739434in}}%
\pgfpathlineto{\pgfqpoint{2.040285in}{1.915001in}}%
\pgfpathlineto{\pgfqpoint{2.048020in}{1.727817in}}%
\pgfpathlineto{\pgfqpoint{2.050599in}{1.841536in}}%
\pgfpathlineto{\pgfqpoint{2.053177in}{1.981455in}}%
\pgfpathlineto{\pgfqpoint{2.055755in}{1.984517in}}%
\pgfpathlineto{\pgfqpoint{2.058334in}{1.856432in}}%
\pgfpathlineto{\pgfqpoint{2.066069in}{1.994870in}}%
\pgfpathlineto{\pgfqpoint{2.068647in}{1.943534in}}%
\pgfpathlineto{\pgfqpoint{2.071226in}{1.878274in}}%
\pgfpathlineto{\pgfqpoint{2.073804in}{1.968470in}}%
\pgfpathlineto{\pgfqpoint{2.086696in}{1.874625in}}%
\pgfpathlineto{\pgfqpoint{2.089275in}{2.085056in}}%
\pgfpathlineto{\pgfqpoint{2.091853in}{1.792824in}}%
\pgfpathlineto{\pgfqpoint{2.094431in}{2.681249in}}%
\pgfpathlineto{\pgfqpoint{2.102167in}{2.017763in}}%
\pgfpathlineto{\pgfqpoint{2.104745in}{2.211608in}}%
\pgfpathlineto{\pgfqpoint{2.107323in}{1.891094in}}%
\pgfpathlineto{\pgfqpoint{2.109902in}{1.894412in}}%
\pgfpathlineto{\pgfqpoint{2.112480in}{2.076909in}}%
\pgfpathlineto{\pgfqpoint{2.120215in}{1.688516in}}%
\pgfpathlineto{\pgfqpoint{2.122794in}{1.994005in}}%
\pgfpathlineto{\pgfqpoint{2.125372in}{1.993274in}}%
\pgfpathlineto{\pgfqpoint{2.130529in}{1.762053in}}%
\pgfpathlineto{\pgfqpoint{2.138264in}{1.887589in}}%
\pgfpathlineto{\pgfqpoint{2.140842in}{1.917309in}}%
\pgfpathlineto{\pgfqpoint{2.143421in}{2.028892in}}%
\pgfpathlineto{\pgfqpoint{2.145999in}{1.929923in}}%
\pgfpathlineto{\pgfqpoint{2.148578in}{1.805602in}}%
\pgfpathlineto{\pgfqpoint{2.158891in}{2.095923in}}%
\pgfpathlineto{\pgfqpoint{2.161470in}{1.789843in}}%
\pgfpathlineto{\pgfqpoint{2.164048in}{1.861274in}}%
\pgfpathlineto{\pgfqpoint{2.166626in}{1.858011in}}%
\pgfpathlineto{\pgfqpoint{2.174361in}{1.582330in}}%
\pgfpathlineto{\pgfqpoint{2.176940in}{1.913945in}}%
\pgfpathlineto{\pgfqpoint{2.179518in}{2.051581in}}%
\pgfpathlineto{\pgfqpoint{2.182097in}{1.720149in}}%
\pgfpathlineto{\pgfqpoint{2.184675in}{1.953182in}}%
\pgfpathlineto{\pgfqpoint{2.192410in}{1.923439in}}%
\pgfpathlineto{\pgfqpoint{2.194989in}{1.968930in}}%
\pgfpathlineto{\pgfqpoint{2.197567in}{1.910065in}}%
\pgfpathlineto{\pgfqpoint{2.200145in}{1.792912in}}%
\pgfpathlineto{\pgfqpoint{2.202724in}{1.945916in}}%
\pgfpathlineto{\pgfqpoint{2.210459in}{1.916483in}}%
\pgfpathlineto{\pgfqpoint{2.213037in}{1.822344in}}%
\pgfpathlineto{\pgfqpoint{2.215616in}{1.770037in}}%
\pgfpathlineto{\pgfqpoint{2.218194in}{2.030460in}}%
\pgfpathlineto{\pgfqpoint{2.220773in}{1.590413in}}%
\pgfpathlineto{\pgfqpoint{2.228508in}{1.821698in}}%
\pgfpathlineto{\pgfqpoint{2.231086in}{2.126328in}}%
\pgfpathlineto{\pgfqpoint{2.233664in}{1.994272in}}%
\pgfpathlineto{\pgfqpoint{2.236243in}{1.773811in}}%
\pgfpathlineto{\pgfqpoint{2.238821in}{1.887275in}}%
\pgfpathlineto{\pgfqpoint{2.246556in}{1.714636in}}%
\pgfpathlineto{\pgfqpoint{2.249135in}{2.063117in}}%
\pgfpathlineto{\pgfqpoint{2.251713in}{1.780106in}}%
\pgfpathlineto{\pgfqpoint{2.254292in}{1.871039in}}%
\pgfpathlineto{\pgfqpoint{2.264605in}{2.039594in}}%
\pgfpathlineto{\pgfqpoint{2.267184in}{2.204061in}}%
\pgfpathlineto{\pgfqpoint{2.269762in}{1.650694in}}%
\pgfpathlineto{\pgfqpoint{2.272340in}{1.979909in}}%
\pgfpathlineto{\pgfqpoint{2.274919in}{1.791043in}}%
\pgfpathlineto{\pgfqpoint{2.282654in}{2.017991in}}%
\pgfpathlineto{\pgfqpoint{2.285232in}{1.730497in}}%
\pgfpathlineto{\pgfqpoint{2.287811in}{2.129024in}}%
\pgfpathlineto{\pgfqpoint{2.290389in}{1.981520in}}%
\pgfpathlineto{\pgfqpoint{2.292967in}{1.974666in}}%
\pgfpathlineto{\pgfqpoint{2.300703in}{1.761114in}}%
\pgfpathlineto{\pgfqpoint{2.303281in}{1.987227in}}%
\pgfpathlineto{\pgfqpoint{2.305859in}{1.603134in}}%
\pgfpathlineto{\pgfqpoint{2.308438in}{2.079361in}}%
\pgfpathlineto{\pgfqpoint{2.311016in}{2.269140in}}%
\pgfpathlineto{\pgfqpoint{2.318751in}{1.864877in}}%
\pgfpathlineto{\pgfqpoint{2.321330in}{2.155611in}}%
\pgfpathlineto{\pgfqpoint{2.323908in}{0.624558in}}%
\pgfpathlineto{\pgfqpoint{2.326487in}{1.728482in}}%
\pgfpathlineto{\pgfqpoint{2.329065in}{2.010366in}}%
\pgfpathlineto{\pgfqpoint{2.336800in}{2.022255in}}%
\pgfpathlineto{\pgfqpoint{2.339379in}{1.629205in}}%
\pgfpathlineto{\pgfqpoint{2.341957in}{1.929339in}}%
\pgfpathlineto{\pgfqpoint{2.344535in}{2.076999in}}%
\pgfpathlineto{\pgfqpoint{2.347114in}{1.985991in}}%
\pgfpathlineto{\pgfqpoint{2.354849in}{1.752884in}}%
\pgfpathlineto{\pgfqpoint{2.357427in}{1.922216in}}%
\pgfpathlineto{\pgfqpoint{2.360006in}{2.001602in}}%
\pgfpathlineto{\pgfqpoint{2.362584in}{1.810637in}}%
\pgfpathlineto{\pgfqpoint{2.365162in}{2.016971in}}%
\pgfpathlineto{\pgfqpoint{2.372898in}{1.826734in}}%
\pgfpathlineto{\pgfqpoint{2.375476in}{2.097882in}}%
\pgfpathlineto{\pgfqpoint{2.378054in}{2.182005in}}%
\pgfpathlineto{\pgfqpoint{2.380633in}{1.750321in}}%
\pgfpathlineto{\pgfqpoint{2.383211in}{1.824419in}}%
\pgfpathlineto{\pgfqpoint{2.390946in}{1.630048in}}%
\pgfpathlineto{\pgfqpoint{2.393525in}{1.795344in}}%
\pgfpathlineto{\pgfqpoint{2.396103in}{1.877357in}}%
\pgfpathlineto{\pgfqpoint{2.398682in}{1.839427in}}%
\pgfpathlineto{\pgfqpoint{2.401260in}{2.683649in}}%
\pgfpathlineto{\pgfqpoint{2.414152in}{1.366036in}}%
\pgfpathlineto{\pgfqpoint{2.416730in}{1.921471in}}%
\pgfpathlineto{\pgfqpoint{2.419309in}{1.254289in}}%
\pgfpathlineto{\pgfqpoint{2.427044in}{2.112910in}}%
\pgfpathlineto{\pgfqpoint{2.429622in}{1.793949in}}%
\pgfpathlineto{\pgfqpoint{2.432201in}{1.680289in}}%
\pgfpathlineto{\pgfqpoint{2.437357in}{2.119127in}}%
\pgfpathlineto{\pgfqpoint{2.445093in}{1.957327in}}%
\pgfpathlineto{\pgfqpoint{2.447671in}{1.883791in}}%
\pgfpathlineto{\pgfqpoint{2.450249in}{1.733531in}}%
\pgfpathlineto{\pgfqpoint{2.452828in}{2.088072in}}%
\pgfpathlineto{\pgfqpoint{2.455406in}{1.766000in}}%
\pgfpathlineto{\pgfqpoint{2.463141in}{2.115300in}}%
\pgfpathlineto{\pgfqpoint{2.465720in}{1.893101in}}%
\pgfpathlineto{\pgfqpoint{2.468298in}{1.495929in}}%
\pgfpathlineto{\pgfqpoint{2.470877in}{1.235755in}}%
\pgfpathlineto{\pgfqpoint{2.473455in}{2.458145in}}%
\pgfpathlineto{\pgfqpoint{2.481190in}{1.644707in}}%
\pgfpathlineto{\pgfqpoint{2.483768in}{1.900279in}}%
\pgfpathlineto{\pgfqpoint{2.486347in}{2.058482in}}%
\pgfpathlineto{\pgfqpoint{2.488925in}{1.944899in}}%
\pgfpathlineto{\pgfqpoint{2.491504in}{1.687505in}}%
\pgfpathlineto{\pgfqpoint{2.499239in}{2.144133in}}%
\pgfpathlineto{\pgfqpoint{2.501817in}{1.982444in}}%
\pgfpathlineto{\pgfqpoint{2.504396in}{1.905927in}}%
\pgfpathlineto{\pgfqpoint{2.509552in}{1.810739in}}%
\pgfpathlineto{\pgfqpoint{2.517288in}{1.981980in}}%
\pgfpathlineto{\pgfqpoint{2.519866in}{2.081757in}}%
\pgfpathlineto{\pgfqpoint{2.522444in}{1.930404in}}%
\pgfpathlineto{\pgfqpoint{2.525023in}{2.119135in}}%
\pgfpathlineto{\pgfqpoint{2.527601in}{2.067383in}}%
\pgfpathlineto{\pgfqpoint{2.535336in}{1.861877in}}%
\pgfpathlineto{\pgfqpoint{2.537915in}{1.733118in}}%
\pgfpathlineto{\pgfqpoint{2.540493in}{1.800187in}}%
\pgfpathlineto{\pgfqpoint{2.543071in}{1.923431in}}%
\pgfpathlineto{\pgfqpoint{2.545650in}{2.146206in}}%
\pgfpathlineto{\pgfqpoint{2.553385in}{1.831519in}}%
\pgfpathlineto{\pgfqpoint{2.555963in}{1.803986in}}%
\pgfpathlineto{\pgfqpoint{2.558542in}{1.696287in}}%
\pgfpathlineto{\pgfqpoint{2.561120in}{1.907952in}}%
\pgfpathlineto{\pgfqpoint{2.563699in}{1.864891in}}%
\pgfpathlineto{\pgfqpoint{2.571434in}{1.794038in}}%
\pgfpathlineto{\pgfqpoint{2.574012in}{1.960333in}}%
\pgfpathlineto{\pgfqpoint{2.576591in}{1.834136in}}%
\pgfpathlineto{\pgfqpoint{2.579169in}{2.212913in}}%
\pgfpathlineto{\pgfqpoint{2.581747in}{1.783062in}}%
\pgfpathlineto{\pgfqpoint{2.589483in}{1.898382in}}%
\pgfpathlineto{\pgfqpoint{2.592061in}{1.955876in}}%
\pgfpathlineto{\pgfqpoint{2.594639in}{1.928408in}}%
\pgfpathlineto{\pgfqpoint{2.597218in}{1.922234in}}%
\pgfpathlineto{\pgfqpoint{2.599796in}{1.738271in}}%
\pgfpathlineto{\pgfqpoint{2.607531in}{1.864983in}}%
\pgfpathlineto{\pgfqpoint{2.610110in}{1.883148in}}%
\pgfpathlineto{\pgfqpoint{2.615266in}{1.672382in}}%
\pgfpathlineto{\pgfqpoint{2.617845in}{1.723097in}}%
\pgfpathlineto{\pgfqpoint{2.625580in}{1.790457in}}%
\pgfpathlineto{\pgfqpoint{2.628158in}{1.855506in}}%
\pgfpathlineto{\pgfqpoint{2.630737in}{1.833712in}}%
\pgfpathlineto{\pgfqpoint{2.633315in}{1.970426in}}%
\pgfpathlineto{\pgfqpoint{2.635894in}{1.932913in}}%
\pgfpathlineto{\pgfqpoint{2.643629in}{1.490679in}}%
\pgfpathlineto{\pgfqpoint{2.646207in}{1.722609in}}%
\pgfpathlineto{\pgfqpoint{2.648786in}{1.570926in}}%
\pgfpathlineto{\pgfqpoint{2.651364in}{1.996085in}}%
\pgfpathlineto{\pgfqpoint{2.653942in}{2.023906in}}%
\pgfpathlineto{\pgfqpoint{2.664256in}{1.832928in}}%
\pgfpathlineto{\pgfqpoint{2.669413in}{1.775141in}}%
\pgfpathlineto{\pgfqpoint{2.671991in}{1.874243in}}%
\pgfpathlineto{\pgfqpoint{2.679726in}{2.138393in}}%
\pgfpathlineto{\pgfqpoint{2.682305in}{1.817211in}}%
\pgfpathlineto{\pgfqpoint{2.684883in}{1.956464in}}%
\pgfpathlineto{\pgfqpoint{2.687461in}{1.867881in}}%
\pgfpathlineto{\pgfqpoint{2.690040in}{2.075280in}}%
\pgfpathlineto{\pgfqpoint{2.697775in}{2.159778in}}%
\pgfpathlineto{\pgfqpoint{2.700353in}{1.784532in}}%
\pgfpathlineto{\pgfqpoint{2.702932in}{1.991431in}}%
\pgfpathlineto{\pgfqpoint{2.708089in}{1.685108in}}%
\pgfpathlineto{\pgfqpoint{2.715824in}{1.843004in}}%
\pgfpathlineto{\pgfqpoint{2.718402in}{1.698877in}}%
\pgfpathlineto{\pgfqpoint{2.720980in}{1.633926in}}%
\pgfpathlineto{\pgfqpoint{2.723559in}{1.956714in}}%
\pgfpathlineto{\pgfqpoint{2.726137in}{1.648175in}}%
\pgfpathlineto{\pgfqpoint{2.733872in}{1.437559in}}%
\pgfpathlineto{\pgfqpoint{2.736451in}{2.024123in}}%
\pgfpathlineto{\pgfqpoint{2.739029in}{1.809291in}}%
\pgfpathlineto{\pgfqpoint{2.741608in}{1.844984in}}%
\pgfpathlineto{\pgfqpoint{2.744186in}{1.919864in}}%
\pgfpathlineto{\pgfqpoint{2.751921in}{1.770005in}}%
\pgfpathlineto{\pgfqpoint{2.754500in}{2.056430in}}%
\pgfpathlineto{\pgfqpoint{2.757078in}{2.038722in}}%
\pgfpathlineto{\pgfqpoint{2.759656in}{2.120028in}}%
\pgfpathlineto{\pgfqpoint{2.762235in}{2.025927in}}%
\pgfpathlineto{\pgfqpoint{2.769970in}{1.940203in}}%
\pgfpathlineto{\pgfqpoint{2.772548in}{1.570694in}}%
\pgfpathlineto{\pgfqpoint{2.775127in}{2.224669in}}%
\pgfpathlineto{\pgfqpoint{2.777705in}{2.153953in}}%
\pgfpathlineto{\pgfqpoint{2.780283in}{1.867951in}}%
\pgfpathlineto{\pgfqpoint{2.788019in}{1.756460in}}%
\pgfpathlineto{\pgfqpoint{2.790597in}{2.234247in}}%
\pgfpathlineto{\pgfqpoint{2.795754in}{1.795129in}}%
\pgfpathlineto{\pgfqpoint{2.798332in}{1.715309in}}%
\pgfpathlineto{\pgfqpoint{2.806067in}{2.202451in}}%
\pgfpathlineto{\pgfqpoint{2.808646in}{2.161541in}}%
\pgfpathlineto{\pgfqpoint{2.811224in}{1.634755in}}%
\pgfpathlineto{\pgfqpoint{2.813803in}{1.674572in}}%
\pgfpathlineto{\pgfqpoint{2.816381in}{1.974040in}}%
\pgfpathlineto{\pgfqpoint{2.826695in}{1.895146in}}%
\pgfpathlineto{\pgfqpoint{2.829273in}{2.217462in}}%
\pgfpathlineto{\pgfqpoint{2.831851in}{1.749262in}}%
\pgfpathlineto{\pgfqpoint{2.834430in}{1.918647in}}%
\pgfpathlineto{\pgfqpoint{2.842165in}{1.808530in}}%
\pgfpathlineto{\pgfqpoint{2.844743in}{2.004755in}}%
\pgfpathlineto{\pgfqpoint{2.847322in}{2.045048in}}%
\pgfpathlineto{\pgfqpoint{2.849900in}{2.072691in}}%
\pgfpathlineto{\pgfqpoint{2.852478in}{1.995713in}}%
\pgfpathlineto{\pgfqpoint{2.860214in}{1.807346in}}%
\pgfpathlineto{\pgfqpoint{2.862792in}{1.812965in}}%
\pgfpathlineto{\pgfqpoint{2.865370in}{1.958088in}}%
\pgfpathlineto{\pgfqpoint{2.867949in}{1.865247in}}%
\pgfpathlineto{\pgfqpoint{2.870527in}{1.940434in}}%
\pgfpathlineto{\pgfqpoint{2.878262in}{1.980437in}}%
\pgfpathlineto{\pgfqpoint{2.880841in}{1.686452in}}%
\pgfpathlineto{\pgfqpoint{2.883419in}{1.781030in}}%
\pgfpathlineto{\pgfqpoint{2.888576in}{1.856484in}}%
\pgfpathlineto{\pgfqpoint{2.896311in}{1.657362in}}%
\pgfpathlineto{\pgfqpoint{2.898889in}{1.991322in}}%
\pgfpathlineto{\pgfqpoint{2.901468in}{1.750756in}}%
\pgfpathlineto{\pgfqpoint{2.904046in}{1.708570in}}%
\pgfpathlineto{\pgfqpoint{2.906625in}{2.322754in}}%
\pgfpathlineto{\pgfqpoint{2.914360in}{1.931875in}}%
\pgfpathlineto{\pgfqpoint{2.916938in}{1.597314in}}%
\pgfpathlineto{\pgfqpoint{2.919517in}{1.961713in}}%
\pgfpathlineto{\pgfqpoint{2.922095in}{1.443437in}}%
\pgfpathlineto{\pgfqpoint{2.924673in}{1.888910in}}%
\pgfpathlineto{\pgfqpoint{2.932409in}{1.700631in}}%
\pgfpathlineto{\pgfqpoint{2.934987in}{1.674975in}}%
\pgfpathlineto{\pgfqpoint{2.937565in}{2.240487in}}%
\pgfpathlineto{\pgfqpoint{2.940144in}{1.751620in}}%
\pgfpathlineto{\pgfqpoint{2.942722in}{1.856063in}}%
\pgfpathlineto{\pgfqpoint{2.950457in}{1.735759in}}%
\pgfpathlineto{\pgfqpoint{2.953036in}{1.871039in}}%
\pgfpathlineto{\pgfqpoint{2.958192in}{1.985333in}}%
\pgfpathlineto{\pgfqpoint{2.960771in}{1.933930in}}%
\pgfpathlineto{\pgfqpoint{2.968506in}{1.868048in}}%
\pgfpathlineto{\pgfqpoint{2.971084in}{1.723971in}}%
\pgfpathlineto{\pgfqpoint{2.976241in}{1.649969in}}%
\pgfpathlineto{\pgfqpoint{2.978820in}{1.846667in}}%
\pgfpathlineto{\pgfqpoint{2.986555in}{1.919754in}}%
\pgfpathlineto{\pgfqpoint{2.989133in}{2.070778in}}%
\pgfpathlineto{\pgfqpoint{2.991712in}{1.570698in}}%
\pgfpathlineto{\pgfqpoint{2.994290in}{1.916828in}}%
\pgfpathlineto{\pgfqpoint{2.996868in}{1.840524in}}%
\pgfpathlineto{\pgfqpoint{3.004604in}{1.794553in}}%
\pgfpathlineto{\pgfqpoint{3.007182in}{2.090551in}}%
\pgfpathlineto{\pgfqpoint{3.009760in}{1.852835in}}%
\pgfpathlineto{\pgfqpoint{3.012339in}{1.810244in}}%
\pgfpathlineto{\pgfqpoint{3.014917in}{1.696781in}}%
\pgfpathlineto{\pgfqpoint{3.025231in}{1.950708in}}%
\pgfpathlineto{\pgfqpoint{3.030387in}{1.611360in}}%
\pgfpathlineto{\pgfqpoint{3.032966in}{2.118426in}}%
\pgfpathlineto{\pgfqpoint{3.040701in}{1.685778in}}%
\pgfpathlineto{\pgfqpoint{3.043279in}{2.037848in}}%
\pgfpathlineto{\pgfqpoint{3.045858in}{1.485961in}}%
\pgfpathlineto{\pgfqpoint{3.048436in}{1.666266in}}%
\pgfpathlineto{\pgfqpoint{3.051015in}{1.804443in}}%
\pgfpathlineto{\pgfqpoint{3.058750in}{1.621779in}}%
\pgfpathlineto{\pgfqpoint{3.061328in}{1.976864in}}%
\pgfpathlineto{\pgfqpoint{3.063907in}{1.966773in}}%
\pgfpathlineto{\pgfqpoint{3.066485in}{1.991666in}}%
\pgfpathlineto{\pgfqpoint{3.069063in}{1.981520in}}%
\pgfpathlineto{\pgfqpoint{3.076798in}{2.062185in}}%
\pgfpathlineto{\pgfqpoint{3.079377in}{2.082047in}}%
\pgfpathlineto{\pgfqpoint{3.081955in}{1.516052in}}%
\pgfpathlineto{\pgfqpoint{3.084534in}{1.952542in}}%
\pgfpathlineto{\pgfqpoint{3.087112in}{2.292176in}}%
\pgfpathlineto{\pgfqpoint{3.097426in}{1.496775in}}%
\pgfpathlineto{\pgfqpoint{3.100004in}{1.914692in}}%
\pgfpathlineto{\pgfqpoint{3.102582in}{1.811773in}}%
\pgfpathlineto{\pgfqpoint{3.105161in}{1.886652in}}%
\pgfpathlineto{\pgfqpoint{3.112896in}{1.839801in}}%
\pgfpathlineto{\pgfqpoint{3.115474in}{1.939699in}}%
\pgfpathlineto{\pgfqpoint{3.118053in}{1.802378in}}%
\pgfpathlineto{\pgfqpoint{3.120631in}{1.958385in}}%
\pgfpathlineto{\pgfqpoint{3.123210in}{1.995179in}}%
\pgfpathlineto{\pgfqpoint{3.130945in}{1.559271in}}%
\pgfpathlineto{\pgfqpoint{3.133523in}{2.127036in}}%
\pgfpathlineto{\pgfqpoint{3.136101in}{1.705942in}}%
\pgfpathlineto{\pgfqpoint{3.138680in}{1.933495in}}%
\pgfpathlineto{\pgfqpoint{3.141258in}{1.955055in}}%
\pgfpathlineto{\pgfqpoint{3.148993in}{1.911369in}}%
\pgfpathlineto{\pgfqpoint{3.151572in}{1.979229in}}%
\pgfpathlineto{\pgfqpoint{3.154150in}{1.947970in}}%
\pgfpathlineto{\pgfqpoint{3.156729in}{1.858749in}}%
\pgfpathlineto{\pgfqpoint{3.159307in}{1.815644in}}%
\pgfpathlineto{\pgfqpoint{3.167042in}{2.094777in}}%
\pgfpathlineto{\pgfqpoint{3.172199in}{1.613143in}}%
\pgfpathlineto{\pgfqpoint{3.174777in}{1.750193in}}%
\pgfpathlineto{\pgfqpoint{3.177356in}{1.755730in}}%
\pgfpathlineto{\pgfqpoint{3.185091in}{2.242684in}}%
\pgfpathlineto{\pgfqpoint{3.190248in}{1.791594in}}%
\pgfpathlineto{\pgfqpoint{3.192826in}{1.910799in}}%
\pgfpathlineto{\pgfqpoint{3.195404in}{1.898520in}}%
\pgfpathlineto{\pgfqpoint{3.203140in}{2.086560in}}%
\pgfpathlineto{\pgfqpoint{3.205718in}{1.804499in}}%
\pgfpathlineto{\pgfqpoint{3.208296in}{1.816436in}}%
\pgfpathlineto{\pgfqpoint{3.210875in}{1.864963in}}%
\pgfpathlineto{\pgfqpoint{3.213453in}{1.785787in}}%
\pgfpathlineto{\pgfqpoint{3.221188in}{2.056314in}}%
\pgfpathlineto{\pgfqpoint{3.223767in}{2.087598in}}%
\pgfpathlineto{\pgfqpoint{3.226345in}{1.906911in}}%
\pgfpathlineto{\pgfqpoint{3.228924in}{1.769237in}}%
\pgfpathlineto{\pgfqpoint{3.231502in}{1.786822in}}%
\pgfpathlineto{\pgfqpoint{3.239237in}{1.886103in}}%
\pgfpathlineto{\pgfqpoint{3.241816in}{1.877061in}}%
\pgfpathlineto{\pgfqpoint{3.244394in}{2.077743in}}%
\pgfpathlineto{\pgfqpoint{3.246972in}{1.897848in}}%
\pgfpathlineto{\pgfqpoint{3.257286in}{1.820370in}}%
\pgfpathlineto{\pgfqpoint{3.259864in}{1.793304in}}%
\pgfpathlineto{\pgfqpoint{3.262443in}{1.808037in}}%
\pgfpathlineto{\pgfqpoint{3.265021in}{2.071372in}}%
\pgfpathlineto{\pgfqpoint{3.267599in}{1.936397in}}%
\pgfpathlineto{\pgfqpoint{3.275335in}{2.255972in}}%
\pgfpathlineto{\pgfqpoint{3.277913in}{1.745479in}}%
\pgfpathlineto{\pgfqpoint{3.280491in}{1.897382in}}%
\pgfpathlineto{\pgfqpoint{3.283070in}{1.821251in}}%
\pgfpathlineto{\pgfqpoint{3.285648in}{1.765205in}}%
\pgfpathlineto{\pgfqpoint{3.293383in}{1.814962in}}%
\pgfpathlineto{\pgfqpoint{3.295962in}{1.725710in}}%
\pgfpathlineto{\pgfqpoint{3.298540in}{2.113667in}}%
\pgfpathlineto{\pgfqpoint{3.301119in}{1.888680in}}%
\pgfpathlineto{\pgfqpoint{3.303697in}{1.929734in}}%
\pgfpathlineto{\pgfqpoint{3.311432in}{1.706267in}}%
\pgfpathlineto{\pgfqpoint{3.314010in}{1.838517in}}%
\pgfpathlineto{\pgfqpoint{3.316589in}{1.761269in}}%
\pgfpathlineto{\pgfqpoint{3.319167in}{1.706777in}}%
\pgfpathlineto{\pgfqpoint{3.321746in}{1.820001in}}%
\pgfpathlineto{\pgfqpoint{3.329481in}{1.768581in}}%
\pgfpathlineto{\pgfqpoint{3.332059in}{1.949435in}}%
\pgfpathlineto{\pgfqpoint{3.334638in}{1.937141in}}%
\pgfpathlineto{\pgfqpoint{3.337216in}{1.912994in}}%
\pgfpathlineto{\pgfqpoint{3.339794in}{1.835083in}}%
\pgfpathlineto{\pgfqpoint{3.350108in}{1.759777in}}%
\pgfpathlineto{\pgfqpoint{3.352686in}{1.877068in}}%
\pgfpathlineto{\pgfqpoint{3.355265in}{1.946260in}}%
\pgfpathlineto{\pgfqpoint{3.357843in}{1.969911in}}%
\pgfpathlineto{\pgfqpoint{3.365578in}{1.763154in}}%
\pgfpathlineto{\pgfqpoint{3.368157in}{1.759565in}}%
\pgfpathlineto{\pgfqpoint{3.373313in}{1.937452in}}%
\pgfpathlineto{\pgfqpoint{3.375892in}{1.849931in}}%
\pgfpathlineto{\pgfqpoint{3.383627in}{1.886118in}}%
\pgfpathlineto{\pgfqpoint{3.386205in}{1.886107in}}%
\pgfpathlineto{\pgfqpoint{3.388784in}{1.846924in}}%
\pgfpathlineto{\pgfqpoint{3.391362in}{1.795497in}}%
\pgfpathlineto{\pgfqpoint{3.393941in}{1.840744in}}%
\pgfpathlineto{\pgfqpoint{3.401676in}{1.883162in}}%
\pgfpathlineto{\pgfqpoint{3.404254in}{1.840718in}}%
\pgfpathlineto{\pgfqpoint{3.406833in}{1.925585in}}%
\pgfpathlineto{\pgfqpoint{3.409411in}{1.988728in}}%
\pgfpathlineto{\pgfqpoint{3.411989in}{1.789633in}}%
\pgfpathlineto{\pgfqpoint{3.419725in}{1.765031in}}%
\pgfpathlineto{\pgfqpoint{3.422303in}{1.740049in}}%
\pgfpathlineto{\pgfqpoint{3.424881in}{1.950342in}}%
\pgfpathlineto{\pgfqpoint{3.427460in}{1.690640in}}%
\pgfpathlineto{\pgfqpoint{3.430038in}{1.975191in}}%
\pgfpathlineto{\pgfqpoint{3.437773in}{1.757672in}}%
\pgfpathlineto{\pgfqpoint{3.440352in}{2.051518in}}%
\pgfpathlineto{\pgfqpoint{3.442930in}{1.944003in}}%
\pgfpathlineto{\pgfqpoint{3.445508in}{1.976984in}}%
\pgfpathlineto{\pgfqpoint{3.455822in}{1.925313in}}%
\pgfpathlineto{\pgfqpoint{3.458400in}{1.967174in}}%
\pgfpathlineto{\pgfqpoint{3.460979in}{2.150865in}}%
\pgfpathlineto{\pgfqpoint{3.463557in}{1.856250in}}%
\pgfpathlineto{\pgfqpoint{3.466136in}{1.758294in}}%
\pgfpathlineto{\pgfqpoint{3.473871in}{1.912648in}}%
\pgfpathlineto{\pgfqpoint{3.476449in}{1.856188in}}%
\pgfpathlineto{\pgfqpoint{3.479028in}{1.951106in}}%
\pgfpathlineto{\pgfqpoint{3.481606in}{1.734440in}}%
\pgfpathlineto{\pgfqpoint{3.484184in}{1.909735in}}%
\pgfpathlineto{\pgfqpoint{3.491920in}{1.802527in}}%
\pgfpathlineto{\pgfqpoint{3.497076in}{1.844105in}}%
\pgfpathlineto{\pgfqpoint{3.499655in}{1.939799in}}%
\pgfpathlineto{\pgfqpoint{3.502233in}{1.691178in}}%
\pgfpathlineto{\pgfqpoint{3.509968in}{1.792608in}}%
\pgfpathlineto{\pgfqpoint{3.512547in}{1.713274in}}%
\pgfpathlineto{\pgfqpoint{3.515125in}{1.742724in}}%
\pgfpathlineto{\pgfqpoint{3.517703in}{1.649191in}}%
\pgfpathlineto{\pgfqpoint{3.520282in}{2.481468in}}%
\pgfpathlineto{\pgfqpoint{3.528017in}{1.759467in}}%
\pgfpathlineto{\pgfqpoint{3.530595in}{1.919360in}}%
\pgfpathlineto{\pgfqpoint{3.533174in}{2.300900in}}%
\pgfpathlineto{\pgfqpoint{3.535752in}{1.630348in}}%
\pgfpathlineto{\pgfqpoint{3.538331in}{2.076249in}}%
\pgfpathlineto{\pgfqpoint{3.546066in}{2.006717in}}%
\pgfpathlineto{\pgfqpoint{3.548644in}{1.856334in}}%
\pgfpathlineto{\pgfqpoint{3.551223in}{1.885743in}}%
\pgfpathlineto{\pgfqpoint{3.553801in}{1.988296in}}%
\pgfpathlineto{\pgfqpoint{3.556379in}{1.827146in}}%
\pgfpathlineto{\pgfqpoint{3.564114in}{2.037337in}}%
\pgfpathlineto{\pgfqpoint{3.566693in}{1.934866in}}%
\pgfpathlineto{\pgfqpoint{3.569271in}{1.899986in}}%
\pgfpathlineto{\pgfqpoint{3.571850in}{1.986421in}}%
\pgfpathlineto{\pgfqpoint{3.574428in}{1.899784in}}%
\pgfpathlineto{\pgfqpoint{3.582163in}{1.908347in}}%
\pgfpathlineto{\pgfqpoint{3.584742in}{1.830858in}}%
\pgfpathlineto{\pgfqpoint{3.587320in}{1.853794in}}%
\pgfpathlineto{\pgfqpoint{3.589898in}{1.801916in}}%
\pgfpathlineto{\pgfqpoint{3.592477in}{1.891224in}}%
\pgfpathlineto{\pgfqpoint{3.602790in}{1.836424in}}%
\pgfpathlineto{\pgfqpoint{3.605369in}{1.853710in}}%
\pgfpathlineto{\pgfqpoint{3.607947in}{2.066587in}}%
\pgfpathlineto{\pgfqpoint{3.610526in}{1.888204in}}%
\pgfpathlineto{\pgfqpoint{3.618261in}{1.762082in}}%
\pgfpathlineto{\pgfqpoint{3.620839in}{1.787498in}}%
\pgfpathlineto{\pgfqpoint{3.623417in}{2.032055in}}%
\pgfpathlineto{\pgfqpoint{3.625996in}{1.833772in}}%
\pgfpathlineto{\pgfqpoint{3.628574in}{1.813573in}}%
\pgfpathlineto{\pgfqpoint{3.636309in}{2.022973in}}%
\pgfpathlineto{\pgfqpoint{3.638888in}{1.922385in}}%
\pgfpathlineto{\pgfqpoint{3.641466in}{1.890972in}}%
\pgfpathlineto{\pgfqpoint{3.644045in}{1.876730in}}%
\pgfpathlineto{\pgfqpoint{3.646623in}{1.939216in}}%
\pgfpathlineto{\pgfqpoint{3.654358in}{1.955944in}}%
\pgfpathlineto{\pgfqpoint{3.656937in}{1.780460in}}%
\pgfpathlineto{\pgfqpoint{3.659515in}{2.063043in}}%
\pgfpathlineto{\pgfqpoint{3.662093in}{1.650647in}}%
\pgfpathlineto{\pgfqpoint{3.664672in}{1.933440in}}%
\pgfpathlineto{\pgfqpoint{3.672407in}{1.837025in}}%
\pgfpathlineto{\pgfqpoint{3.674985in}{1.700122in}}%
\pgfpathlineto{\pgfqpoint{3.677564in}{1.721753in}}%
\pgfpathlineto{\pgfqpoint{3.680142in}{1.847975in}}%
\pgfpathlineto{\pgfqpoint{3.682720in}{2.054827in}}%
\pgfpathlineto{\pgfqpoint{3.690456in}{1.816648in}}%
\pgfpathlineto{\pgfqpoint{3.693034in}{1.770467in}}%
\pgfpathlineto{\pgfqpoint{3.695612in}{2.120129in}}%
\pgfpathlineto{\pgfqpoint{3.698191in}{1.745439in}}%
\pgfpathlineto{\pgfqpoint{3.700769in}{2.121476in}}%
\pgfpathlineto{\pgfqpoint{3.708504in}{1.548895in}}%
\pgfpathlineto{\pgfqpoint{3.711083in}{1.914093in}}%
\pgfpathlineto{\pgfqpoint{3.713661in}{1.724286in}}%
\pgfpathlineto{\pgfqpoint{3.716240in}{1.694124in}}%
\pgfpathlineto{\pgfqpoint{3.718818in}{2.125776in}}%
\pgfpathlineto{\pgfqpoint{3.726553in}{2.094221in}}%
\pgfpathlineto{\pgfqpoint{3.729132in}{1.976066in}}%
\pgfpathlineto{\pgfqpoint{3.731710in}{1.936057in}}%
\pgfpathlineto{\pgfqpoint{3.734288in}{1.626868in}}%
\pgfpathlineto{\pgfqpoint{3.736867in}{2.342564in}}%
\pgfpathlineto{\pgfqpoint{3.744602in}{2.059982in}}%
\pgfpathlineto{\pgfqpoint{3.747180in}{1.995127in}}%
\pgfpathlineto{\pgfqpoint{3.749759in}{1.887527in}}%
\pgfpathlineto{\pgfqpoint{3.752337in}{1.966960in}}%
\pgfpathlineto{\pgfqpoint{3.754915in}{1.947455in}}%
\pgfpathlineto{\pgfqpoint{3.762651in}{1.898262in}}%
\pgfpathlineto{\pgfqpoint{3.765229in}{2.165455in}}%
\pgfpathlineto{\pgfqpoint{3.767807in}{1.954008in}}%
\pgfpathlineto{\pgfqpoint{3.770386in}{1.846985in}}%
\pgfpathlineto{\pgfqpoint{3.772964in}{1.924452in}}%
\pgfpathlineto{\pgfqpoint{3.780699in}{1.945586in}}%
\pgfpathlineto{\pgfqpoint{3.783278in}{1.921471in}}%
\pgfpathlineto{\pgfqpoint{3.785856in}{1.825921in}}%
\pgfpathlineto{\pgfqpoint{3.788435in}{1.668118in}}%
\pgfpathlineto{\pgfqpoint{3.791013in}{1.758034in}}%
\pgfpathlineto{\pgfqpoint{3.798748in}{1.808891in}}%
\pgfpathlineto{\pgfqpoint{3.801326in}{1.895380in}}%
\pgfpathlineto{\pgfqpoint{3.803905in}{2.054024in}}%
\pgfpathlineto{\pgfqpoint{3.806483in}{1.809329in}}%
\pgfpathlineto{\pgfqpoint{3.809062in}{1.900575in}}%
\pgfpathlineto{\pgfqpoint{3.816797in}{1.758034in}}%
\pgfpathlineto{\pgfqpoint{3.819375in}{2.032285in}}%
\pgfpathlineto{\pgfqpoint{3.821954in}{1.889769in}}%
\pgfpathlineto{\pgfqpoint{3.827110in}{2.226335in}}%
\pgfpathlineto{\pgfqpoint{3.834846in}{1.789387in}}%
\pgfpathlineto{\pgfqpoint{3.837424in}{2.096763in}}%
\pgfpathlineto{\pgfqpoint{3.840002in}{1.626833in}}%
\pgfpathlineto{\pgfqpoint{3.842581in}{2.005317in}}%
\pgfpathlineto{\pgfqpoint{3.845159in}{1.823746in}}%
\pgfpathlineto{\pgfqpoint{3.852894in}{1.957658in}}%
\pgfpathlineto{\pgfqpoint{3.855473in}{1.860559in}}%
\pgfpathlineto{\pgfqpoint{3.858051in}{1.707913in}}%
\pgfpathlineto{\pgfqpoint{3.860629in}{1.965914in}}%
\pgfpathlineto{\pgfqpoint{3.863208in}{1.672885in}}%
\pgfpathlineto{\pgfqpoint{3.870943in}{1.791240in}}%
\pgfpathlineto{\pgfqpoint{3.873521in}{1.908317in}}%
\pgfpathlineto{\pgfqpoint{3.876100in}{2.177343in}}%
\pgfpathlineto{\pgfqpoint{3.878678in}{2.162514in}}%
\pgfpathlineto{\pgfqpoint{3.881257in}{1.883955in}}%
\pgfpathlineto{\pgfqpoint{3.891570in}{2.024505in}}%
\pgfpathlineto{\pgfqpoint{3.894149in}{1.837885in}}%
\pgfpathlineto{\pgfqpoint{3.899305in}{1.942381in}}%
\pgfpathlineto{\pgfqpoint{3.907041in}{1.689651in}}%
\pgfpathlineto{\pgfqpoint{3.909619in}{1.817081in}}%
\pgfpathlineto{\pgfqpoint{3.912197in}{1.578239in}}%
\pgfpathlineto{\pgfqpoint{3.917354in}{1.721746in}}%
\pgfpathlineto{\pgfqpoint{3.925089in}{1.773546in}}%
\pgfpathlineto{\pgfqpoint{3.927668in}{1.778389in}}%
\pgfpathlineto{\pgfqpoint{3.930246in}{1.976890in}}%
\pgfpathlineto{\pgfqpoint{3.932824in}{2.107231in}}%
\pgfpathlineto{\pgfqpoint{3.935403in}{1.677035in}}%
\pgfpathlineto{\pgfqpoint{3.943138in}{1.794508in}}%
\pgfpathlineto{\pgfqpoint{3.945716in}{1.958103in}}%
\pgfpathlineto{\pgfqpoint{3.948295in}{1.802474in}}%
\pgfpathlineto{\pgfqpoint{3.950873in}{1.839316in}}%
\pgfpathlineto{\pgfqpoint{3.953452in}{2.188691in}}%
\pgfpathlineto{\pgfqpoint{3.963765in}{1.855402in}}%
\pgfpathlineto{\pgfqpoint{3.966344in}{1.912709in}}%
\pgfpathlineto{\pgfqpoint{3.968922in}{2.072981in}}%
\pgfpathlineto{\pgfqpoint{3.971500in}{1.522879in}}%
\pgfpathlineto{\pgfqpoint{3.979235in}{1.755420in}}%
\pgfpathlineto{\pgfqpoint{3.981814in}{1.149695in}}%
\pgfpathlineto{\pgfqpoint{3.986971in}{1.995411in}}%
\pgfpathlineto{\pgfqpoint{3.989549in}{1.537694in}}%
\pgfpathlineto{\pgfqpoint{3.997284in}{2.074466in}}%
\pgfpathlineto{\pgfqpoint{3.999863in}{2.069728in}}%
\pgfpathlineto{\pgfqpoint{4.002441in}{1.832555in}}%
\pgfpathlineto{\pgfqpoint{4.005019in}{2.087261in}}%
\pgfpathlineto{\pgfqpoint{4.007598in}{1.610747in}}%
\pgfpathlineto{\pgfqpoint{4.015333in}{1.735484in}}%
\pgfpathlineto{\pgfqpoint{4.017911in}{1.954137in}}%
\pgfpathlineto{\pgfqpoint{4.020490in}{1.931765in}}%
\pgfpathlineto{\pgfqpoint{4.023068in}{1.964537in}}%
\pgfpathlineto{\pgfqpoint{4.025647in}{1.840836in}}%
\pgfpathlineto{\pgfqpoint{4.035960in}{1.771881in}}%
\pgfpathlineto{\pgfqpoint{4.038538in}{2.055201in}}%
\pgfpathlineto{\pgfqpoint{4.041117in}{1.617735in}}%
\pgfpathlineto{\pgfqpoint{4.043695in}{1.790542in}}%
\pgfpathlineto{\pgfqpoint{4.051430in}{1.995813in}}%
\pgfpathlineto{\pgfqpoint{4.054009in}{1.898664in}}%
\pgfpathlineto{\pgfqpoint{4.056587in}{1.859993in}}%
\pgfpathlineto{\pgfqpoint{4.059166in}{1.801869in}}%
\pgfpathlineto{\pgfqpoint{4.061744in}{1.859950in}}%
\pgfpathlineto{\pgfqpoint{4.069479in}{1.940246in}}%
\pgfpathlineto{\pgfqpoint{4.074636in}{1.673292in}}%
\pgfpathlineto{\pgfqpoint{4.077214in}{1.940885in}}%
\pgfpathlineto{\pgfqpoint{4.079793in}{1.385628in}}%
\pgfpathlineto{\pgfqpoint{4.087528in}{1.979271in}}%
\pgfpathlineto{\pgfqpoint{4.090106in}{1.483919in}}%
\pgfpathlineto{\pgfqpoint{4.092685in}{1.833373in}}%
\pgfpathlineto{\pgfqpoint{4.095263in}{2.047183in}}%
\pgfpathlineto{\pgfqpoint{4.097841in}{1.804801in}}%
\pgfpathlineto{\pgfqpoint{4.105577in}{2.302059in}}%
\pgfpathlineto{\pgfqpoint{4.108155in}{1.695155in}}%
\pgfpathlineto{\pgfqpoint{4.110733in}{2.092069in}}%
\pgfpathlineto{\pgfqpoint{4.113312in}{1.780647in}}%
\pgfpathlineto{\pgfqpoint{4.115890in}{2.202408in}}%
\pgfpathlineto{\pgfqpoint{4.123625in}{1.901659in}}%
\pgfpathlineto{\pgfqpoint{4.126204in}{1.641676in}}%
\pgfpathlineto{\pgfqpoint{4.128782in}{1.647622in}}%
\pgfpathlineto{\pgfqpoint{4.131361in}{1.656664in}}%
\pgfpathlineto{\pgfqpoint{4.133939in}{1.911225in}}%
\pgfpathlineto{\pgfqpoint{4.141674in}{1.971163in}}%
\pgfpathlineto{\pgfqpoint{4.144253in}{1.676063in}}%
\pgfpathlineto{\pgfqpoint{4.146831in}{1.968758in}}%
\pgfpathlineto{\pgfqpoint{4.149409in}{1.896828in}}%
\pgfpathlineto{\pgfqpoint{4.159723in}{2.022256in}}%
\pgfpathlineto{\pgfqpoint{4.162301in}{1.711229in}}%
\pgfpathlineto{\pgfqpoint{4.164880in}{1.965363in}}%
\pgfpathlineto{\pgfqpoint{4.167458in}{1.899537in}}%
\pgfpathlineto{\pgfqpoint{4.170036in}{1.984637in}}%
\pgfpathlineto{\pgfqpoint{4.177772in}{1.890854in}}%
\pgfpathlineto{\pgfqpoint{4.180350in}{1.913436in}}%
\pgfpathlineto{\pgfqpoint{4.182928in}{1.848438in}}%
\pgfpathlineto{\pgfqpoint{4.185507in}{1.868212in}}%
\pgfpathlineto{\pgfqpoint{4.188085in}{1.632184in}}%
\pgfpathlineto{\pgfqpoint{4.195820in}{1.956663in}}%
\pgfpathlineto{\pgfqpoint{4.198399in}{1.922243in}}%
\pgfpathlineto{\pgfqpoint{4.200977in}{1.873880in}}%
\pgfpathlineto{\pgfqpoint{4.203556in}{1.498388in}}%
\pgfpathlineto{\pgfqpoint{4.206134in}{1.882606in}}%
\pgfpathlineto{\pgfqpoint{4.213869in}{1.769602in}}%
\pgfpathlineto{\pgfqpoint{4.216447in}{1.824501in}}%
\pgfpathlineto{\pgfqpoint{4.219026in}{1.724925in}}%
\pgfpathlineto{\pgfqpoint{4.221604in}{1.782870in}}%
\pgfpathlineto{\pgfqpoint{4.224183in}{2.073266in}}%
\pgfpathlineto{\pgfqpoint{4.231918in}{1.885616in}}%
\pgfpathlineto{\pgfqpoint{4.234496in}{1.800974in}}%
\pgfpathlineto{\pgfqpoint{4.237075in}{1.955670in}}%
\pgfpathlineto{\pgfqpoint{4.239653in}{1.818550in}}%
\pgfpathlineto{\pgfqpoint{4.242231in}{2.065739in}}%
\pgfpathlineto{\pgfqpoint{4.249967in}{1.693849in}}%
\pgfpathlineto{\pgfqpoint{4.252545in}{1.789192in}}%
\pgfpathlineto{\pgfqpoint{4.255123in}{1.806501in}}%
\pgfpathlineto{\pgfqpoint{4.257702in}{2.096039in}}%
\pgfpathlineto{\pgfqpoint{4.260280in}{1.992737in}}%
\pgfpathlineto{\pgfqpoint{4.268015in}{1.789987in}}%
\pgfpathlineto{\pgfqpoint{4.270594in}{1.894229in}}%
\pgfpathlineto{\pgfqpoint{4.273172in}{1.783940in}}%
\pgfpathlineto{\pgfqpoint{4.275750in}{1.853575in}}%
\pgfpathlineto{\pgfqpoint{4.278329in}{1.751302in}}%
\pgfpathlineto{\pgfqpoint{4.288642in}{1.662027in}}%
\pgfpathlineto{\pgfqpoint{4.291221in}{1.936030in}}%
\pgfpathlineto{\pgfqpoint{4.293799in}{1.856286in}}%
\pgfpathlineto{\pgfqpoint{4.296378in}{1.624583in}}%
\pgfpathlineto{\pgfqpoint{4.304113in}{1.990181in}}%
\pgfpathlineto{\pgfqpoint{4.306691in}{1.790693in}}%
\pgfpathlineto{\pgfqpoint{4.309270in}{1.877001in}}%
\pgfpathlineto{\pgfqpoint{4.311848in}{1.763452in}}%
\pgfpathlineto{\pgfqpoint{4.314426in}{1.681397in}}%
\pgfpathlineto{\pgfqpoint{4.322162in}{1.943489in}}%
\pgfpathlineto{\pgfqpoint{4.324740in}{2.185043in}}%
\pgfpathlineto{\pgfqpoint{4.327318in}{2.036568in}}%
\pgfpathlineto{\pgfqpoint{4.329897in}{1.838629in}}%
\pgfpathlineto{\pgfqpoint{4.332475in}{1.729014in}}%
\pgfpathlineto{\pgfqpoint{4.340210in}{1.674139in}}%
\pgfpathlineto{\pgfqpoint{4.342789in}{2.127235in}}%
\pgfpathlineto{\pgfqpoint{4.345367in}{2.127007in}}%
\pgfpathlineto{\pgfqpoint{4.347945in}{2.060221in}}%
\pgfpathlineto{\pgfqpoint{4.350524in}{1.798482in}}%
\pgfpathlineto{\pgfqpoint{4.358259in}{1.847766in}}%
\pgfpathlineto{\pgfqpoint{4.360837in}{1.701522in}}%
\pgfpathlineto{\pgfqpoint{4.363416in}{1.797538in}}%
\pgfpathlineto{\pgfqpoint{4.365994in}{1.841565in}}%
\pgfpathlineto{\pgfqpoint{4.368573in}{1.859237in}}%
\pgfpathlineto{\pgfqpoint{4.376308in}{1.603701in}}%
\pgfpathlineto{\pgfqpoint{4.378886in}{1.850099in}}%
\pgfpathlineto{\pgfqpoint{4.381465in}{2.256476in}}%
\pgfpathlineto{\pgfqpoint{4.384043in}{1.923842in}}%
\pgfpathlineto{\pgfqpoint{4.394356in}{1.903241in}}%
\pgfpathlineto{\pgfqpoint{4.396935in}{2.296606in}}%
\pgfpathlineto{\pgfqpoint{4.399513in}{1.683948in}}%
\pgfpathlineto{\pgfqpoint{4.402092in}{1.787015in}}%
\pgfpathlineto{\pgfqpoint{4.404670in}{1.946386in}}%
\pgfpathlineto{\pgfqpoint{4.412405in}{2.112599in}}%
\pgfpathlineto{\pgfqpoint{4.414984in}{1.905317in}}%
\pgfpathlineto{\pgfqpoint{4.417562in}{1.896710in}}%
\pgfpathlineto{\pgfqpoint{4.420140in}{1.910908in}}%
\pgfpathlineto{\pgfqpoint{4.422719in}{1.853961in}}%
\pgfpathlineto{\pgfqpoint{4.430454in}{1.859646in}}%
\pgfpathlineto{\pgfqpoint{4.433032in}{1.730962in}}%
\pgfpathlineto{\pgfqpoint{4.435611in}{1.836588in}}%
\pgfpathlineto{\pgfqpoint{4.438189in}{1.836531in}}%
\pgfpathlineto{\pgfqpoint{4.440768in}{1.767165in}}%
\pgfpathlineto{\pgfqpoint{4.448503in}{1.786980in}}%
\pgfpathlineto{\pgfqpoint{4.451081in}{1.937734in}}%
\pgfpathlineto{\pgfqpoint{4.453660in}{1.972119in}}%
\pgfpathlineto{\pgfqpoint{4.456238in}{1.030136in}}%
\pgfpathlineto{\pgfqpoint{4.458816in}{1.687149in}}%
\pgfpathlineto{\pgfqpoint{4.466551in}{1.789117in}}%
\pgfpathlineto{\pgfqpoint{4.469130in}{1.739893in}}%
\pgfpathlineto{\pgfqpoint{4.471708in}{1.831225in}}%
\pgfpathlineto{\pgfqpoint{4.474287in}{1.864907in}}%
\pgfpathlineto{\pgfqpoint{4.476865in}{1.800388in}}%
\pgfpathlineto{\pgfqpoint{4.484600in}{2.112699in}}%
\pgfpathlineto{\pgfqpoint{4.487179in}{1.828419in}}%
\pgfpathlineto{\pgfqpoint{4.489757in}{1.916699in}}%
\pgfpathlineto{\pgfqpoint{4.492335in}{1.706185in}}%
\pgfpathlineto{\pgfqpoint{4.494914in}{1.828082in}}%
\pgfpathlineto{\pgfqpoint{4.502649in}{1.846452in}}%
\pgfpathlineto{\pgfqpoint{4.505227in}{1.763124in}}%
\pgfpathlineto{\pgfqpoint{4.507806in}{1.590916in}}%
\pgfpathlineto{\pgfqpoint{4.510384in}{1.814553in}}%
\pgfpathlineto{\pgfqpoint{4.512963in}{1.284557in}}%
\pgfpathlineto{\pgfqpoint{4.520698in}{1.083393in}}%
\pgfpathlineto{\pgfqpoint{4.523276in}{1.658293in}}%
\pgfpathlineto{\pgfqpoint{4.525854in}{2.601301in}}%
\pgfpathlineto{\pgfqpoint{4.528433in}{2.040691in}}%
\pgfpathlineto{\pgfqpoint{4.531011in}{1.792912in}}%
\pgfpathlineto{\pgfqpoint{4.538746in}{1.713889in}}%
\pgfpathlineto{\pgfqpoint{4.541325in}{1.348476in}}%
\pgfpathlineto{\pgfqpoint{4.543903in}{2.142299in}}%
\pgfpathlineto{\pgfqpoint{4.546482in}{1.904281in}}%
\pgfpathlineto{\pgfqpoint{4.549060in}{1.526044in}}%
\pgfpathlineto{\pgfqpoint{4.559374in}{2.232635in}}%
\pgfpathlineto{\pgfqpoint{4.561952in}{1.424826in}}%
\pgfpathlineto{\pgfqpoint{4.564530in}{1.823501in}}%
\pgfpathlineto{\pgfqpoint{4.567109in}{1.901611in}}%
\pgfpathlineto{\pgfqpoint{4.574844in}{1.762129in}}%
\pgfpathlineto{\pgfqpoint{4.577422in}{2.286546in}}%
\pgfpathlineto{\pgfqpoint{4.580001in}{2.060781in}}%
\pgfpathlineto{\pgfqpoint{4.585157in}{1.781567in}}%
\pgfpathlineto{\pgfqpoint{4.592893in}{2.079204in}}%
\pgfpathlineto{\pgfqpoint{4.595471in}{1.735806in}}%
\pgfpathlineto{\pgfqpoint{4.598049in}{1.894190in}}%
\pgfpathlineto{\pgfqpoint{4.600628in}{2.189200in}}%
\pgfpathlineto{\pgfqpoint{4.603206in}{2.245169in}}%
\pgfpathlineto{\pgfqpoint{4.610941in}{1.613748in}}%
\pgfpathlineto{\pgfqpoint{4.613520in}{2.016161in}}%
\pgfpathlineto{\pgfqpoint{4.616098in}{1.774405in}}%
\pgfpathlineto{\pgfqpoint{4.618677in}{1.874267in}}%
\pgfpathlineto{\pgfqpoint{4.621255in}{2.006178in}}%
\pgfpathlineto{\pgfqpoint{4.628990in}{2.097489in}}%
\pgfpathlineto{\pgfqpoint{4.631569in}{1.921727in}}%
\pgfpathlineto{\pgfqpoint{4.634147in}{1.959444in}}%
\pgfpathlineto{\pgfqpoint{4.636725in}{2.059209in}}%
\pgfpathlineto{\pgfqpoint{4.639304in}{1.892880in}}%
\pgfpathlineto{\pgfqpoint{4.649617in}{1.808449in}}%
\pgfpathlineto{\pgfqpoint{4.652196in}{1.899228in}}%
\pgfpathlineto{\pgfqpoint{4.654774in}{1.889810in}}%
\pgfpathlineto{\pgfqpoint{4.657352in}{2.042314in}}%
\pgfpathlineto{\pgfqpoint{4.665088in}{1.942242in}}%
\pgfpathlineto{\pgfqpoint{4.667666in}{1.672293in}}%
\pgfpathlineto{\pgfqpoint{4.670244in}{1.821051in}}%
\pgfpathlineto{\pgfqpoint{4.672823in}{2.221505in}}%
\pgfpathlineto{\pgfqpoint{4.675401in}{2.459138in}}%
\pgfpathlineto{\pgfqpoint{4.688293in}{1.661507in}}%
\pgfpathlineto{\pgfqpoint{4.690872in}{2.005983in}}%
\pgfpathlineto{\pgfqpoint{4.693450in}{1.699960in}}%
\pgfpathlineto{\pgfqpoint{4.701185in}{1.931229in}}%
\pgfpathlineto{\pgfqpoint{4.703763in}{1.990893in}}%
\pgfpathlineto{\pgfqpoint{4.706342in}{1.874026in}}%
\pgfpathlineto{\pgfqpoint{4.708920in}{1.691021in}}%
\pgfpathlineto{\pgfqpoint{4.711499in}{1.649871in}}%
\pgfpathlineto{\pgfqpoint{4.719234in}{1.825297in}}%
\pgfpathlineto{\pgfqpoint{4.721812in}{1.971536in}}%
\pgfpathlineto{\pgfqpoint{4.724391in}{1.922619in}}%
\pgfpathlineto{\pgfqpoint{4.726969in}{1.516481in}}%
\pgfpathlineto{\pgfqpoint{4.729547in}{1.675885in}}%
\pgfpathlineto{\pgfqpoint{4.737283in}{2.241179in}}%
\pgfpathlineto{\pgfqpoint{4.739861in}{1.806743in}}%
\pgfpathlineto{\pgfqpoint{4.742439in}{2.096716in}}%
\pgfpathlineto{\pgfqpoint{4.745018in}{1.958809in}}%
\pgfpathlineto{\pgfqpoint{4.747596in}{1.762027in}}%
\pgfpathlineto{\pgfqpoint{4.755331in}{1.913501in}}%
\pgfpathlineto{\pgfqpoint{4.757910in}{2.000913in}}%
\pgfpathlineto{\pgfqpoint{4.760488in}{1.719944in}}%
\pgfpathlineto{\pgfqpoint{4.765645in}{1.819413in}}%
\pgfpathlineto{\pgfqpoint{4.773380in}{1.635566in}}%
\pgfpathlineto{\pgfqpoint{4.775958in}{2.170266in}}%
\pgfpathlineto{\pgfqpoint{4.778537in}{1.831595in}}%
\pgfpathlineto{\pgfqpoint{4.781115in}{1.855848in}}%
\pgfpathlineto{\pgfqpoint{4.783694in}{2.431676in}}%
\pgfpathlineto{\pgfqpoint{4.791429in}{2.012514in}}%
\pgfpathlineto{\pgfqpoint{4.794007in}{1.708850in}}%
\pgfpathlineto{\pgfqpoint{4.796586in}{1.856231in}}%
\pgfpathlineto{\pgfqpoint{4.799164in}{1.894727in}}%
\pgfpathlineto{\pgfqpoint{4.801742in}{1.868079in}}%
\pgfpathlineto{\pgfqpoint{4.809478in}{2.006736in}}%
\pgfpathlineto{\pgfqpoint{4.812056in}{2.232433in}}%
\pgfpathlineto{\pgfqpoint{4.814634in}{2.206306in}}%
\pgfpathlineto{\pgfqpoint{4.817213in}{1.694057in}}%
\pgfpathlineto{\pgfqpoint{4.819791in}{1.310155in}}%
\pgfpathlineto{\pgfqpoint{4.827526in}{2.084989in}}%
\pgfpathlineto{\pgfqpoint{4.830105in}{2.039454in}}%
\pgfpathlineto{\pgfqpoint{4.832683in}{1.954732in}}%
\pgfpathlineto{\pgfqpoint{4.835261in}{1.836448in}}%
\pgfpathlineto{\pgfqpoint{4.845575in}{1.905629in}}%
\pgfpathlineto{\pgfqpoint{4.848153in}{1.983058in}}%
\pgfpathlineto{\pgfqpoint{4.850732in}{1.796427in}}%
\pgfpathlineto{\pgfqpoint{4.853310in}{1.700717in}}%
\pgfpathlineto{\pgfqpoint{4.863624in}{1.599669in}}%
\pgfpathlineto{\pgfqpoint{4.866202in}{1.935558in}}%
\pgfpathlineto{\pgfqpoint{4.868781in}{1.670959in}}%
\pgfpathlineto{\pgfqpoint{4.871359in}{1.692866in}}%
\pgfpathlineto{\pgfqpoint{4.873937in}{1.546386in}}%
\pgfpathlineto{\pgfqpoint{4.881672in}{2.058039in}}%
\pgfpathlineto{\pgfqpoint{4.884251in}{1.828961in}}%
\pgfpathlineto{\pgfqpoint{4.886829in}{1.692760in}}%
\pgfpathlineto{\pgfqpoint{4.889408in}{1.952816in}}%
\pgfpathlineto{\pgfqpoint{4.891986in}{1.551174in}}%
\pgfpathlineto{\pgfqpoint{4.902300in}{2.347490in}}%
\pgfpathlineto{\pgfqpoint{4.904878in}{1.804936in}}%
\pgfpathlineto{\pgfqpoint{4.907456in}{2.110400in}}%
\pgfpathlineto{\pgfqpoint{4.910035in}{2.042850in}}%
\pgfpathlineto{\pgfqpoint{4.917770in}{1.734892in}}%
\pgfpathlineto{\pgfqpoint{4.920348in}{2.390040in}}%
\pgfpathlineto{\pgfqpoint{4.922927in}{1.868143in}}%
\pgfpathlineto{\pgfqpoint{4.928084in}{2.345811in}}%
\pgfpathlineto{\pgfqpoint{4.935819in}{1.728066in}}%
\pgfpathlineto{\pgfqpoint{4.938397in}{1.641902in}}%
\pgfpathlineto{\pgfqpoint{4.940975in}{2.094548in}}%
\pgfpathlineto{\pgfqpoint{4.943554in}{1.769488in}}%
\pgfpathlineto{\pgfqpoint{4.946132in}{1.997900in}}%
\pgfpathlineto{\pgfqpoint{4.953867in}{2.227667in}}%
\pgfpathlineto{\pgfqpoint{4.956446in}{1.876562in}}%
\pgfpathlineto{\pgfqpoint{4.959024in}{1.615408in}}%
\pgfpathlineto{\pgfqpoint{4.961603in}{1.433038in}}%
\pgfpathlineto{\pgfqpoint{4.964181in}{2.149051in}}%
\pgfpathlineto{\pgfqpoint{4.974495in}{1.991847in}}%
\pgfpathlineto{\pgfqpoint{4.977073in}{2.118853in}}%
\pgfpathlineto{\pgfqpoint{4.979651in}{1.751681in}}%
\pgfpathlineto{\pgfqpoint{4.982230in}{1.823659in}}%
\pgfpathlineto{\pgfqpoint{4.989965in}{1.957355in}}%
\pgfpathlineto{\pgfqpoint{4.992543in}{1.790302in}}%
\pgfpathlineto{\pgfqpoint{4.995122in}{1.809578in}}%
\pgfpathlineto{\pgfqpoint{4.997700in}{2.065961in}}%
\pgfpathlineto{\pgfqpoint{5.000278in}{1.558271in}}%
\pgfpathlineto{\pgfqpoint{5.008014in}{1.664590in}}%
\pgfpathlineto{\pgfqpoint{5.010592in}{2.111227in}}%
\pgfpathlineto{\pgfqpoint{5.013170in}{2.202658in}}%
\pgfpathlineto{\pgfqpoint{5.015749in}{1.942787in}}%
\pgfpathlineto{\pgfqpoint{5.018327in}{2.032930in}}%
\pgfpathlineto{\pgfqpoint{5.026062in}{1.772409in}}%
\pgfpathlineto{\pgfqpoint{5.028641in}{1.862798in}}%
\pgfpathlineto{\pgfqpoint{5.031219in}{1.846297in}}%
\pgfpathlineto{\pgfqpoint{5.033798in}{1.699785in}}%
\pgfpathlineto{\pgfqpoint{5.036376in}{1.740260in}}%
\pgfpathlineto{\pgfqpoint{5.044111in}{1.722561in}}%
\pgfpathlineto{\pgfqpoint{5.046690in}{1.907557in}}%
\pgfpathlineto{\pgfqpoint{5.049268in}{1.879457in}}%
\pgfpathlineto{\pgfqpoint{5.051846in}{2.224283in}}%
\pgfpathlineto{\pgfqpoint{5.054425in}{1.970077in}}%
\pgfpathlineto{\pgfqpoint{5.062160in}{1.912164in}}%
\pgfpathlineto{\pgfqpoint{5.064738in}{1.728117in}}%
\pgfpathlineto{\pgfqpoint{5.067317in}{1.890336in}}%
\pgfpathlineto{\pgfqpoint{5.069895in}{1.887565in}}%
\pgfpathlineto{\pgfqpoint{5.080209in}{1.804854in}}%
\pgfpathlineto{\pgfqpoint{5.082787in}{1.915186in}}%
\pgfpathlineto{\pgfqpoint{5.087944in}{1.776987in}}%
\pgfpathlineto{\pgfqpoint{5.090522in}{2.173782in}}%
\pgfpathlineto{\pgfqpoint{5.098257in}{1.791656in}}%
\pgfpathlineto{\pgfqpoint{5.100836in}{1.860065in}}%
\pgfpathlineto{\pgfqpoint{5.103414in}{2.029583in}}%
\pgfpathlineto{\pgfqpoint{5.105993in}{1.731694in}}%
\pgfpathlineto{\pgfqpoint{5.108571in}{1.860070in}}%
\pgfpathlineto{\pgfqpoint{5.116306in}{1.755510in}}%
\pgfpathlineto{\pgfqpoint{5.118884in}{1.895850in}}%
\pgfpathlineto{\pgfqpoint{5.121463in}{1.779919in}}%
\pgfpathlineto{\pgfqpoint{5.124041in}{1.926311in}}%
\pgfpathlineto{\pgfqpoint{5.126620in}{1.942672in}}%
\pgfpathlineto{\pgfqpoint{5.134355in}{2.002632in}}%
\pgfpathlineto{\pgfqpoint{5.136933in}{1.982779in}}%
\pgfpathlineto{\pgfqpoint{5.139512in}{1.439828in}}%
\pgfpathlineto{\pgfqpoint{5.144668in}{1.907423in}}%
\pgfpathlineto{\pgfqpoint{5.152404in}{1.988150in}}%
\pgfpathlineto{\pgfqpoint{5.154982in}{1.395694in}}%
\pgfpathlineto{\pgfqpoint{5.157560in}{1.959059in}}%
\pgfpathlineto{\pgfqpoint{5.160139in}{1.837011in}}%
\pgfpathlineto{\pgfqpoint{5.162717in}{1.964480in}}%
\pgfpathlineto{\pgfqpoint{5.170452in}{2.087431in}}%
\pgfpathlineto{\pgfqpoint{5.173031in}{1.904558in}}%
\pgfpathlineto{\pgfqpoint{5.175609in}{1.996252in}}%
\pgfpathlineto{\pgfqpoint{5.178187in}{1.796002in}}%
\pgfpathlineto{\pgfqpoint{5.180766in}{2.081557in}}%
\pgfpathlineto{\pgfqpoint{5.188501in}{1.868283in}}%
\pgfpathlineto{\pgfqpoint{5.191079in}{1.959049in}}%
\pgfpathlineto{\pgfqpoint{5.193658in}{1.788540in}}%
\pgfpathlineto{\pgfqpoint{5.196236in}{1.937064in}}%
\pgfpathlineto{\pgfqpoint{5.198815in}{1.575027in}}%
\pgfpathlineto{\pgfqpoint{5.206550in}{1.971108in}}%
\pgfpathlineto{\pgfqpoint{5.209128in}{1.614333in}}%
\pgfpathlineto{\pgfqpoint{5.211707in}{1.673550in}}%
\pgfpathlineto{\pgfqpoint{5.214285in}{1.958732in}}%
\pgfpathlineto{\pgfqpoint{5.216863in}{1.828653in}}%
\pgfpathlineto{\pgfqpoint{5.224599in}{1.916247in}}%
\pgfpathlineto{\pgfqpoint{5.227177in}{2.067665in}}%
\pgfpathlineto{\pgfqpoint{5.229755in}{1.999234in}}%
\pgfpathlineto{\pgfqpoint{5.232334in}{1.807041in}}%
\pgfpathlineto{\pgfqpoint{5.234912in}{1.923921in}}%
\pgfpathlineto{\pgfqpoint{5.245226in}{1.770726in}}%
\pgfpathlineto{\pgfqpoint{5.247804in}{2.062877in}}%
\pgfpathlineto{\pgfqpoint{5.250382in}{1.909745in}}%
\pgfpathlineto{\pgfqpoint{5.252961in}{2.000451in}}%
\pgfpathlineto{\pgfqpoint{5.260696in}{1.947750in}}%
\pgfpathlineto{\pgfqpoint{5.263274in}{1.758614in}}%
\pgfpathlineto{\pgfqpoint{5.265853in}{1.953361in}}%
\pgfpathlineto{\pgfqpoint{5.268431in}{1.999352in}}%
\pgfpathlineto{\pgfqpoint{5.271010in}{1.879202in}}%
\pgfpathlineto{\pgfqpoint{5.278745in}{1.715383in}}%
\pgfpathlineto{\pgfqpoint{5.281323in}{2.062030in}}%
\pgfpathlineto{\pgfqpoint{5.283902in}{1.773036in}}%
\pgfpathlineto{\pgfqpoint{5.286480in}{1.985330in}}%
\pgfpathlineto{\pgfqpoint{5.289058in}{1.800362in}}%
\pgfpathlineto{\pgfqpoint{5.296793in}{1.849243in}}%
\pgfpathlineto{\pgfqpoint{5.299372in}{1.963511in}}%
\pgfpathlineto{\pgfqpoint{5.301950in}{1.908995in}}%
\pgfpathlineto{\pgfqpoint{5.304529in}{2.030232in}}%
\pgfpathlineto{\pgfqpoint{5.307107in}{1.386961in}}%
\pgfpathlineto{\pgfqpoint{5.314842in}{1.613516in}}%
\pgfpathlineto{\pgfqpoint{5.317421in}{2.180779in}}%
\pgfpathlineto{\pgfqpoint{5.319999in}{2.227503in}}%
\pgfpathlineto{\pgfqpoint{5.325156in}{1.897756in}}%
\pgfpathlineto{\pgfqpoint{5.335469in}{2.030617in}}%
\pgfpathlineto{\pgfqpoint{5.338048in}{1.772778in}}%
\pgfpathlineto{\pgfqpoint{5.340626in}{1.823066in}}%
\pgfpathlineto{\pgfqpoint{5.343205in}{2.096599in}}%
\pgfpathlineto{\pgfqpoint{5.350940in}{1.865760in}}%
\pgfpathlineto{\pgfqpoint{5.353518in}{1.871039in}}%
\pgfpathlineto{\pgfqpoint{5.356096in}{1.905328in}}%
\pgfpathlineto{\pgfqpoint{5.358675in}{1.865767in}}%
\pgfpathlineto{\pgfqpoint{5.361253in}{1.905280in}}%
\pgfpathlineto{\pgfqpoint{5.368988in}{1.852608in}}%
\pgfpathlineto{\pgfqpoint{5.371567in}{1.928907in}}%
\pgfpathlineto{\pgfqpoint{5.374145in}{1.826337in}}%
\pgfpathlineto{\pgfqpoint{5.376724in}{1.857873in}}%
\pgfpathlineto{\pgfqpoint{5.379302in}{1.981374in}}%
\pgfpathlineto{\pgfqpoint{5.387037in}{1.889371in}}%
\pgfpathlineto{\pgfqpoint{5.389616in}{1.745006in}}%
\pgfpathlineto{\pgfqpoint{5.392194in}{1.672553in}}%
\pgfpathlineto{\pgfqpoint{5.394772in}{1.955960in}}%
\pgfpathlineto{\pgfqpoint{5.397351in}{2.060840in}}%
\pgfpathlineto{\pgfqpoint{5.405086in}{2.066903in}}%
\pgfpathlineto{\pgfqpoint{5.407664in}{1.954043in}}%
\pgfpathlineto{\pgfqpoint{5.410243in}{1.683806in}}%
\pgfpathlineto{\pgfqpoint{5.412821in}{1.889317in}}%
\pgfpathlineto{\pgfqpoint{5.415399in}{1.805685in}}%
\pgfpathlineto{\pgfqpoint{5.423135in}{1.865801in}}%
\pgfpathlineto{\pgfqpoint{5.425713in}{1.925964in}}%
\pgfpathlineto{\pgfqpoint{5.430870in}{1.969681in}}%
\pgfpathlineto{\pgfqpoint{5.433448in}{1.946001in}}%
\pgfpathlineto{\pgfqpoint{5.441183in}{1.865878in}}%
\pgfpathlineto{\pgfqpoint{5.443762in}{1.764950in}}%
\pgfpathlineto{\pgfqpoint{5.446340in}{1.961636in}}%
\pgfpathlineto{\pgfqpoint{5.448919in}{1.984370in}}%
\pgfpathlineto{\pgfqpoint{5.451497in}{1.840192in}}%
\pgfpathlineto{\pgfqpoint{5.459232in}{1.762710in}}%
\pgfpathlineto{\pgfqpoint{5.461811in}{2.002507in}}%
\pgfpathlineto{\pgfqpoint{5.464389in}{1.847900in}}%
\pgfpathlineto{\pgfqpoint{5.466967in}{2.009485in}}%
\pgfpathlineto{\pgfqpoint{5.469546in}{1.796804in}}%
\pgfpathlineto{\pgfqpoint{5.477281in}{2.039600in}}%
\pgfpathlineto{\pgfqpoint{5.479859in}{1.692217in}}%
\pgfpathlineto{\pgfqpoint{5.482438in}{1.817088in}}%
\pgfpathlineto{\pgfqpoint{5.485016in}{2.106355in}}%
\pgfpathlineto{\pgfqpoint{5.487594in}{1.845591in}}%
\pgfpathlineto{\pgfqpoint{5.497908in}{1.972642in}}%
\pgfpathlineto{\pgfqpoint{5.500486in}{1.710786in}}%
\pgfpathlineto{\pgfqpoint{5.503065in}{1.830140in}}%
\pgfpathlineto{\pgfqpoint{5.505643in}{1.507059in}}%
\pgfpathlineto{\pgfqpoint{5.513378in}{2.344750in}}%
\pgfpathlineto{\pgfqpoint{5.515957in}{1.591736in}}%
\pgfpathlineto{\pgfqpoint{5.521114in}{2.117352in}}%
\pgfpathlineto{\pgfqpoint{5.523692in}{1.868488in}}%
\pgfpathlineto{\pgfqpoint{5.531427in}{1.944879in}}%
\pgfpathlineto{\pgfqpoint{5.534006in}{1.921809in}}%
\pgfpathlineto{\pgfqpoint{5.536584in}{1.687677in}}%
\pgfpathlineto{\pgfqpoint{5.539162in}{2.147996in}}%
\pgfpathlineto{\pgfqpoint{5.541741in}{1.586406in}}%
\pgfpathlineto{\pgfqpoint{5.552054in}{1.988302in}}%
\pgfpathlineto{\pgfqpoint{5.554633in}{2.126186in}}%
\pgfpathlineto{\pgfqpoint{5.557211in}{1.587911in}}%
\pgfpathlineto{\pgfqpoint{5.559789in}{2.221842in}}%
\pgfpathlineto{\pgfqpoint{5.567525in}{1.619268in}}%
\pgfpathlineto{\pgfqpoint{5.572681in}{1.985110in}}%
\pgfpathlineto{\pgfqpoint{5.575260in}{1.956808in}}%
\pgfpathlineto{\pgfqpoint{5.577838in}{2.051495in}}%
\pgfpathlineto{\pgfqpoint{5.585573in}{1.655309in}}%
\pgfpathlineto{\pgfqpoint{5.588152in}{1.749640in}}%
\pgfpathlineto{\pgfqpoint{5.590730in}{1.878647in}}%
\pgfpathlineto{\pgfqpoint{5.593309in}{1.792282in}}%
\pgfpathlineto{\pgfqpoint{5.595887in}{1.916805in}}%
\pgfpathlineto{\pgfqpoint{5.603622in}{1.730876in}}%
\pgfpathlineto{\pgfqpoint{5.606200in}{1.781344in}}%
\pgfpathlineto{\pgfqpoint{5.608779in}{1.576149in}}%
\pgfpathlineto{\pgfqpoint{5.611357in}{1.721993in}}%
\pgfpathlineto{\pgfqpoint{5.613936in}{1.726192in}}%
\pgfpathlineto{\pgfqpoint{5.621671in}{1.815463in}}%
\pgfpathlineto{\pgfqpoint{5.624249in}{2.561574in}}%
\pgfpathlineto{\pgfqpoint{5.626828in}{1.970736in}}%
\pgfpathlineto{\pgfqpoint{5.629406in}{1.678883in}}%
\pgfpathlineto{\pgfqpoint{5.631984in}{1.932724in}}%
\pgfpathlineto{\pgfqpoint{5.639720in}{1.860771in}}%
\pgfpathlineto{\pgfqpoint{5.642298in}{1.883873in}}%
\pgfpathlineto{\pgfqpoint{5.644876in}{1.842793in}}%
\pgfpathlineto{\pgfqpoint{5.647455in}{1.837607in}}%
\pgfpathlineto{\pgfqpoint{5.650033in}{1.507519in}}%
\pgfpathlineto{\pgfqpoint{5.657768in}{2.226836in}}%
\pgfpathlineto{\pgfqpoint{5.660347in}{2.083638in}}%
\pgfpathlineto{\pgfqpoint{5.662925in}{1.508560in}}%
\pgfpathlineto{\pgfqpoint{5.665503in}{1.147837in}}%
\pgfpathlineto{\pgfqpoint{5.668082in}{2.023595in}}%
\pgfpathlineto{\pgfqpoint{5.675817in}{1.726540in}}%
\pgfpathlineto{\pgfqpoint{5.678395in}{2.023536in}}%
\pgfpathlineto{\pgfqpoint{5.680974in}{1.766821in}}%
\pgfpathlineto{\pgfqpoint{5.683552in}{1.841548in}}%
\pgfpathlineto{\pgfqpoint{5.686131in}{1.603704in}}%
\pgfpathlineto{\pgfqpoint{5.693866in}{2.030777in}}%
\pgfpathlineto{\pgfqpoint{5.696444in}{1.900684in}}%
\pgfpathlineto{\pgfqpoint{5.699023in}{1.852179in}}%
\pgfpathlineto{\pgfqpoint{5.704179in}{2.064207in}}%
\pgfpathlineto{\pgfqpoint{5.711915in}{1.774681in}}%
\pgfpathlineto{\pgfqpoint{5.714493in}{1.825379in}}%
\pgfpathlineto{\pgfqpoint{5.717071in}{1.763203in}}%
\pgfpathlineto{\pgfqpoint{5.719650in}{1.721835in}}%
\pgfpathlineto{\pgfqpoint{5.722228in}{2.006723in}}%
\pgfpathlineto{\pgfqpoint{5.729963in}{2.016579in}}%
\pgfpathlineto{\pgfqpoint{5.732542in}{1.852230in}}%
\pgfpathlineto{\pgfqpoint{5.735120in}{2.183170in}}%
\pgfpathlineto{\pgfqpoint{5.737698in}{1.703557in}}%
\pgfpathlineto{\pgfqpoint{5.740277in}{2.086123in}}%
\pgfpathlineto{\pgfqpoint{5.748012in}{2.055119in}}%
\pgfpathlineto{\pgfqpoint{5.753169in}{1.676489in}}%
\pgfpathlineto{\pgfqpoint{5.755747in}{1.944869in}}%
\pgfpathlineto{\pgfqpoint{5.758326in}{1.871039in}}%
\pgfpathlineto{\pgfqpoint{5.766061in}{1.878933in}}%
\pgfpathlineto{\pgfqpoint{5.771218in}{1.799759in}}%
\pgfpathlineto{\pgfqpoint{5.773796in}{1.918586in}}%
\pgfpathlineto{\pgfqpoint{5.776374in}{1.989429in}}%
\pgfpathlineto{\pgfqpoint{5.789266in}{1.741545in}}%
\pgfpathlineto{\pgfqpoint{5.791845in}{1.939851in}}%
\pgfpathlineto{\pgfqpoint{5.794423in}{1.804877in}}%
\pgfpathlineto{\pgfqpoint{5.804737in}{1.900176in}}%
\pgfpathlineto{\pgfqpoint{5.807315in}{1.945021in}}%
\pgfpathlineto{\pgfqpoint{5.809893in}{2.005111in}}%
\pgfpathlineto{\pgfqpoint{5.812472in}{1.865797in}}%
\pgfpathlineto{\pgfqpoint{5.820207in}{1.715816in}}%
\pgfpathlineto{\pgfqpoint{5.822785in}{1.648003in}}%
\pgfpathlineto{\pgfqpoint{5.825364in}{1.935011in}}%
\pgfpathlineto{\pgfqpoint{5.827942in}{1.894977in}}%
\pgfpathlineto{\pgfqpoint{5.830521in}{1.910874in}}%
\pgfpathlineto{\pgfqpoint{5.840834in}{2.163445in}}%
\pgfpathlineto{\pgfqpoint{5.843412in}{1.964987in}}%
\pgfpathlineto{\pgfqpoint{5.845991in}{1.816288in}}%
\pgfpathlineto{\pgfqpoint{5.848569in}{2.528742in}}%
\pgfpathlineto{\pgfqpoint{5.856304in}{1.754397in}}%
\pgfpathlineto{\pgfqpoint{5.858883in}{2.083540in}}%
\pgfpathlineto{\pgfqpoint{5.861461in}{1.706796in}}%
\pgfpathlineto{\pgfqpoint{5.864040in}{1.738693in}}%
\pgfpathlineto{\pgfqpoint{5.866618in}{1.899106in}}%
\pgfpathlineto{\pgfqpoint{5.874353in}{1.878687in}}%
\pgfpathlineto{\pgfqpoint{5.876932in}{2.071421in}}%
\pgfpathlineto{\pgfqpoint{5.879510in}{1.807838in}}%
\pgfpathlineto{\pgfqpoint{5.882088in}{1.972068in}}%
\pgfpathlineto{\pgfqpoint{5.884667in}{1.787726in}}%
\pgfpathlineto{\pgfqpoint{5.892402in}{1.868509in}}%
\pgfpathlineto{\pgfqpoint{5.894980in}{2.014749in}}%
\pgfpathlineto{\pgfqpoint{5.897559in}{1.946274in}}%
\pgfpathlineto{\pgfqpoint{5.900137in}{1.948494in}}%
\pgfpathlineto{\pgfqpoint{5.902715in}{1.708295in}}%
\pgfpathlineto{\pgfqpoint{5.910451in}{1.951320in}}%
\pgfpathlineto{\pgfqpoint{5.913029in}{1.765606in}}%
\pgfpathlineto{\pgfqpoint{5.915607in}{2.619818in}}%
\pgfpathlineto{\pgfqpoint{5.918186in}{1.795671in}}%
\pgfpathlineto{\pgfqpoint{5.920764in}{1.939125in}}%
\pgfpathlineto{\pgfqpoint{5.931078in}{2.001715in}}%
\pgfpathlineto{\pgfqpoint{5.933656in}{1.820319in}}%
\pgfpathlineto{\pgfqpoint{5.936235in}{1.800791in}}%
\pgfpathlineto{\pgfqpoint{5.938813in}{1.854046in}}%
\pgfpathlineto{\pgfqpoint{5.946548in}{1.834579in}}%
\pgfpathlineto{\pgfqpoint{5.951705in}{2.001730in}}%
\pgfpathlineto{\pgfqpoint{5.954283in}{1.703896in}}%
\pgfpathlineto{\pgfqpoint{5.956862in}{1.778406in}}%
\pgfpathlineto{\pgfqpoint{5.964597in}{1.839253in}}%
\pgfpathlineto{\pgfqpoint{5.967175in}{1.853903in}}%
\pgfpathlineto{\pgfqpoint{5.969754in}{1.836724in}}%
\pgfpathlineto{\pgfqpoint{5.972332in}{1.917595in}}%
\pgfpathlineto{\pgfqpoint{5.974910in}{2.036803in}}%
\pgfpathlineto{\pgfqpoint{5.982646in}{1.924384in}}%
\pgfpathlineto{\pgfqpoint{5.985224in}{1.800691in}}%
\pgfpathlineto{\pgfqpoint{5.987802in}{1.960750in}}%
\pgfpathlineto{\pgfqpoint{5.990381in}{1.880714in}}%
\pgfpathlineto{\pgfqpoint{5.992959in}{1.771652in}}%
\pgfpathlineto{\pgfqpoint{6.000694in}{1.919580in}}%
\pgfpathlineto{\pgfqpoint{6.003273in}{1.866190in}}%
\pgfpathlineto{\pgfqpoint{6.005851in}{1.824916in}}%
\pgfpathlineto{\pgfqpoint{6.008430in}{1.822377in}}%
\pgfpathlineto{\pgfqpoint{6.011008in}{1.824703in}}%
\pgfpathlineto{\pgfqpoint{6.018743in}{1.853941in}}%
\pgfpathlineto{\pgfqpoint{6.021321in}{1.932036in}}%
\pgfpathlineto{\pgfqpoint{6.023900in}{1.834462in}}%
\pgfpathlineto{\pgfqpoint{6.026478in}{1.780538in}}%
\pgfpathlineto{\pgfqpoint{6.029057in}{1.789984in}}%
\pgfpathlineto{\pgfqpoint{6.036792in}{1.831624in}}%
\pgfpathlineto{\pgfqpoint{6.039370in}{1.925214in}}%
\pgfpathlineto{\pgfqpoint{6.041949in}{1.883331in}}%
\pgfpathlineto{\pgfqpoint{6.044527in}{1.740361in}}%
\pgfpathlineto{\pgfqpoint{6.047105in}{1.831425in}}%
\pgfpathlineto{\pgfqpoint{6.054841in}{1.932902in}}%
\pgfpathlineto{\pgfqpoint{6.057419in}{1.940105in}}%
\pgfpathlineto{\pgfqpoint{6.059997in}{1.988901in}}%
\pgfpathlineto{\pgfqpoint{6.062576in}{1.807281in}}%
\pgfpathlineto{\pgfqpoint{6.072889in}{1.951929in}}%
\pgfpathlineto{\pgfqpoint{6.075468in}{1.963793in}}%
\pgfpathlineto{\pgfqpoint{6.078046in}{1.756256in}}%
\pgfpathlineto{\pgfqpoint{6.080624in}{1.807265in}}%
\pgfpathlineto{\pgfqpoint{6.083203in}{1.708259in}}%
\pgfpathlineto{\pgfqpoint{6.090938in}{2.085343in}}%
\pgfpathlineto{\pgfqpoint{6.093516in}{1.973702in}}%
\pgfpathlineto{\pgfqpoint{6.096095in}{1.347516in}}%
\pgfpathlineto{\pgfqpoint{6.098673in}{1.861032in}}%
\pgfpathlineto{\pgfqpoint{6.101252in}{1.785782in}}%
\pgfpathlineto{\pgfqpoint{6.108987in}{1.813164in}}%
\pgfpathlineto{\pgfqpoint{6.111565in}{1.665912in}}%
\pgfpathlineto{\pgfqpoint{6.114144in}{1.919339in}}%
\pgfpathlineto{\pgfqpoint{6.116722in}{1.876116in}}%
\pgfpathlineto{\pgfqpoint{6.119300in}{1.883727in}}%
\pgfpathlineto{\pgfqpoint{6.127036in}{1.883720in}}%
\pgfpathlineto{\pgfqpoint{6.129614in}{1.835513in}}%
\pgfpathlineto{\pgfqpoint{6.132192in}{1.891347in}}%
\pgfpathlineto{\pgfqpoint{6.134771in}{1.794780in}}%
\pgfpathlineto{\pgfqpoint{6.137349in}{1.876131in}}%
\pgfpathlineto{\pgfqpoint{6.145084in}{1.904111in}}%
\pgfpathlineto{\pgfqpoint{6.147663in}{1.850693in}}%
\pgfpathlineto{\pgfqpoint{6.150241in}{1.876127in}}%
\pgfpathlineto{\pgfqpoint{6.152819in}{1.776701in}}%
\pgfpathlineto{\pgfqpoint{6.155398in}{1.960288in}}%
\pgfpathlineto{\pgfqpoint{6.163133in}{1.863404in}}%
\pgfpathlineto{\pgfqpoint{6.165711in}{1.840471in}}%
\pgfpathlineto{\pgfqpoint{6.168290in}{1.970218in}}%
\pgfpathlineto{\pgfqpoint{6.170868in}{1.957114in}}%
\pgfpathlineto{\pgfqpoint{6.173447in}{1.961786in}}%
\pgfpathlineto{\pgfqpoint{6.183760in}{1.906222in}}%
\pgfpathlineto{\pgfqpoint{6.186339in}{2.033604in}}%
\pgfpathlineto{\pgfqpoint{6.188917in}{1.881001in}}%
\pgfpathlineto{\pgfqpoint{6.191495in}{1.977828in}}%
\pgfpathlineto{\pgfqpoint{6.201809in}{1.885871in}}%
\pgfpathlineto{\pgfqpoint{6.204387in}{1.863624in}}%
\pgfpathlineto{\pgfqpoint{6.206966in}{1.657342in}}%
\pgfpathlineto{\pgfqpoint{6.209544in}{1.943348in}}%
\pgfpathlineto{\pgfqpoint{6.217279in}{1.888455in}}%
\pgfpathlineto{\pgfqpoint{6.219858in}{1.828717in}}%
\pgfpathlineto{\pgfqpoint{6.225014in}{2.088236in}}%
\pgfpathlineto{\pgfqpoint{6.227593in}{1.934774in}}%
\pgfpathlineto{\pgfqpoint{6.235328in}{1.917490in}}%
\pgfpathlineto{\pgfqpoint{6.237906in}{1.817243in}}%
\pgfpathlineto{\pgfqpoint{6.240485in}{1.819557in}}%
\pgfpathlineto{\pgfqpoint{6.243063in}{1.784949in}}%
\pgfpathlineto{\pgfqpoint{6.245642in}{1.962037in}}%
\pgfpathlineto{\pgfqpoint{6.253377in}{1.856310in}}%
\pgfpathlineto{\pgfqpoint{6.255955in}{1.698414in}}%
\pgfpathlineto{\pgfqpoint{6.258533in}{1.814001in}}%
\pgfpathlineto{\pgfqpoint{6.261112in}{1.548169in}}%
\pgfpathlineto{\pgfqpoint{6.263690in}{1.908843in}}%
\pgfpathlineto{\pgfqpoint{6.271425in}{2.009064in}}%
\pgfpathlineto{\pgfqpoint{6.276582in}{1.856026in}}%
\pgfpathlineto{\pgfqpoint{6.279161in}{1.800836in}}%
\pgfpathlineto{\pgfqpoint{6.281739in}{1.936235in}}%
\pgfpathlineto{\pgfqpoint{6.289474in}{1.737919in}}%
\pgfpathlineto{\pgfqpoint{6.292053in}{1.795304in}}%
\pgfpathlineto{\pgfqpoint{6.294631in}{1.916513in}}%
\pgfpathlineto{\pgfqpoint{6.297209in}{1.810384in}}%
\pgfpathlineto{\pgfqpoint{6.299788in}{1.964473in}}%
\pgfpathlineto{\pgfqpoint{6.307523in}{1.976588in}}%
\pgfpathlineto{\pgfqpoint{6.310101in}{2.127615in}}%
\pgfpathlineto{\pgfqpoint{6.312680in}{1.962431in}}%
\pgfpathlineto{\pgfqpoint{6.315258in}{1.930104in}}%
\pgfpathlineto{\pgfqpoint{6.317836in}{1.873496in}}%
\pgfpathlineto{\pgfqpoint{6.325572in}{1.770045in}}%
\pgfpathlineto{\pgfqpoint{6.328150in}{2.094528in}}%
\pgfpathlineto{\pgfqpoint{6.330728in}{1.907646in}}%
\pgfpathlineto{\pgfqpoint{6.333307in}{2.185598in}}%
\pgfpathlineto{\pgfqpoint{6.335885in}{1.765110in}}%
\pgfpathlineto{\pgfqpoint{6.343620in}{2.010557in}}%
\pgfpathlineto{\pgfqpoint{6.348777in}{1.856693in}}%
\pgfpathlineto{\pgfqpoint{6.351356in}{1.830338in}}%
\pgfpathlineto{\pgfqpoint{6.353934in}{1.827856in}}%
\pgfpathlineto{\pgfqpoint{6.361669in}{2.045609in}}%
\pgfpathlineto{\pgfqpoint{6.364248in}{1.901972in}}%
\pgfpathlineto{\pgfqpoint{6.366826in}{1.949353in}}%
\pgfpathlineto{\pgfqpoint{6.369404in}{1.830732in}}%
\pgfpathlineto{\pgfqpoint{6.371983in}{1.780656in}}%
\pgfpathlineto{\pgfqpoint{6.382296in}{1.994208in}}%
\pgfpathlineto{\pgfqpoint{6.384875in}{1.925283in}}%
\pgfpathlineto{\pgfqpoint{6.387453in}{1.788436in}}%
\pgfpathlineto{\pgfqpoint{6.390031in}{1.960706in}}%
\pgfpathlineto{\pgfqpoint{6.397767in}{1.955628in}}%
\pgfpathlineto{\pgfqpoint{6.400345in}{1.856965in}}%
\pgfpathlineto{\pgfqpoint{6.402923in}{1.793459in}}%
\pgfpathlineto{\pgfqpoint{6.405502in}{1.833319in}}%
\pgfpathlineto{\pgfqpoint{6.408080in}{1.922886in}}%
\pgfpathlineto{\pgfqpoint{6.415815in}{1.863976in}}%
\pgfpathlineto{\pgfqpoint{6.418394in}{1.835690in}}%
\pgfpathlineto{\pgfqpoint{6.420972in}{1.771739in}}%
\pgfpathlineto{\pgfqpoint{6.423551in}{1.960902in}}%
\pgfpathlineto{\pgfqpoint{6.426129in}{1.927593in}}%
\pgfpathlineto{\pgfqpoint{6.436442in}{1.913353in}}%
\pgfpathlineto{\pgfqpoint{6.439021in}{1.871039in}}%
\pgfpathlineto{\pgfqpoint{6.441599in}{1.927322in}}%
\pgfpathlineto{\pgfqpoint{6.444178in}{1.840571in}}%
\pgfpathlineto{\pgfqpoint{6.451913in}{2.125113in}}%
\pgfpathlineto{\pgfqpoint{6.454491in}{1.766526in}}%
\pgfpathlineto{\pgfqpoint{6.457070in}{1.880350in}}%
\pgfpathlineto{\pgfqpoint{6.459648in}{1.871039in}}%
\pgfpathlineto{\pgfqpoint{6.462226in}{1.808104in}}%
\pgfpathlineto{\pgfqpoint{6.469962in}{1.845343in}}%
\pgfpathlineto{\pgfqpoint{6.472540in}{2.094215in}}%
\pgfpathlineto{\pgfqpoint{6.475118in}{1.921844in}}%
\pgfpathlineto{\pgfqpoint{6.477697in}{1.484554in}}%
\pgfpathlineto{\pgfqpoint{6.480275in}{1.781535in}}%
\pgfpathlineto{\pgfqpoint{6.488010in}{1.976987in}}%
\pgfpathlineto{\pgfqpoint{6.490589in}{1.856944in}}%
\pgfpathlineto{\pgfqpoint{6.493167in}{1.470010in}}%
\pgfpathlineto{\pgfqpoint{6.495746in}{1.875831in}}%
\pgfpathlineto{\pgfqpoint{6.498324in}{1.892589in}}%
\pgfpathlineto{\pgfqpoint{6.506059in}{2.047375in}}%
\pgfpathlineto{\pgfqpoint{6.508637in}{1.949190in}}%
\pgfpathlineto{\pgfqpoint{6.511216in}{1.939473in}}%
\pgfpathlineto{\pgfqpoint{6.513794in}{1.783686in}}%
\pgfpathlineto{\pgfqpoint{6.516373in}{1.937178in}}%
\pgfpathlineto{\pgfqpoint{6.524108in}{1.823818in}}%
\pgfpathlineto{\pgfqpoint{6.526686in}{1.759642in}}%
\pgfpathlineto{\pgfqpoint{6.529265in}{1.835359in}}%
\pgfpathlineto{\pgfqpoint{6.531843in}{2.025205in}}%
\pgfpathlineto{\pgfqpoint{6.534421in}{2.068573in}}%
\pgfpathlineto{\pgfqpoint{6.542157in}{1.894429in}}%
\pgfpathlineto{\pgfqpoint{6.544735in}{1.796097in}}%
\pgfpathlineto{\pgfqpoint{6.547313in}{1.863999in}}%
\pgfpathlineto{\pgfqpoint{6.549892in}{1.760432in}}%
\pgfpathlineto{\pgfqpoint{6.552470in}{1.106632in}}%
\pgfpathlineto{\pgfqpoint{6.562784in}{1.794153in}}%
\pgfpathlineto{\pgfqpoint{6.565362in}{1.843688in}}%
\pgfpathlineto{\pgfqpoint{6.567940in}{2.022258in}}%
\pgfpathlineto{\pgfqpoint{6.570519in}{1.762075in}}%
\pgfpathlineto{\pgfqpoint{6.578254in}{1.688974in}}%
\pgfpathlineto{\pgfqpoint{6.580832in}{1.888567in}}%
\pgfpathlineto{\pgfqpoint{6.583411in}{2.000783in}}%
\pgfpathlineto{\pgfqpoint{6.585989in}{1.781303in}}%
\pgfpathlineto{\pgfqpoint{6.588568in}{1.886022in}}%
\pgfpathlineto{\pgfqpoint{6.596303in}{1.745844in}}%
\pgfpathlineto{\pgfqpoint{6.598881in}{2.090874in}}%
\pgfpathlineto{\pgfqpoint{6.601460in}{2.014656in}}%
\pgfpathlineto{\pgfqpoint{6.604038in}{1.917869in}}%
\pgfpathlineto{\pgfqpoint{6.606616in}{1.959478in}}%
\pgfpathlineto{\pgfqpoint{6.614352in}{2.066218in}}%
\pgfpathlineto{\pgfqpoint{6.616930in}{1.839448in}}%
\pgfpathlineto{\pgfqpoint{6.619508in}{1.722153in}}%
\pgfpathlineto{\pgfqpoint{6.622087in}{2.109710in}}%
\pgfpathlineto{\pgfqpoint{6.624665in}{1.678857in}}%
\pgfpathlineto{\pgfqpoint{6.632400in}{1.834345in}}%
\pgfpathlineto{\pgfqpoint{6.634979in}{1.976053in}}%
\pgfpathlineto{\pgfqpoint{6.637557in}{1.780710in}}%
\pgfpathlineto{\pgfqpoint{6.642714in}{1.897935in}}%
\pgfpathlineto{\pgfqpoint{6.650449in}{1.990423in}}%
\pgfpathlineto{\pgfqpoint{6.653027in}{1.972819in}}%
\pgfpathlineto{\pgfqpoint{6.655606in}{1.866203in}}%
\pgfpathlineto{\pgfqpoint{6.658184in}{2.010803in}}%
\pgfpathlineto{\pgfqpoint{6.660763in}{1.954922in}}%
\pgfpathlineto{\pgfqpoint{6.668498in}{2.108837in}}%
\pgfpathlineto{\pgfqpoint{6.671076in}{1.868674in}}%
\pgfpathlineto{\pgfqpoint{6.673655in}{1.837908in}}%
\pgfpathlineto{\pgfqpoint{6.676233in}{1.608854in}}%
\pgfpathlineto{\pgfqpoint{6.678811in}{1.933312in}}%
\pgfpathlineto{\pgfqpoint{6.686546in}{1.839926in}}%
\pgfpathlineto{\pgfqpoint{6.689125in}{1.784633in}}%
\pgfpathlineto{\pgfqpoint{6.691703in}{2.105408in}}%
\pgfpathlineto{\pgfqpoint{6.694282in}{1.897180in}}%
\pgfpathlineto{\pgfqpoint{6.696860in}{2.071924in}}%
\pgfpathlineto{\pgfqpoint{6.704595in}{1.868687in}}%
\pgfpathlineto{\pgfqpoint{6.707174in}{1.826300in}}%
\pgfpathlineto{\pgfqpoint{6.709752in}{1.835650in}}%
\pgfpathlineto{\pgfqpoint{6.714909in}{1.974508in}}%
\pgfpathlineto{\pgfqpoint{6.725222in}{1.948301in}}%
\pgfpathlineto{\pgfqpoint{6.727801in}{1.786738in}}%
\pgfpathlineto{\pgfqpoint{6.730379in}{1.863998in}}%
\pgfpathlineto{\pgfqpoint{6.732958in}{1.828744in}}%
\pgfpathlineto{\pgfqpoint{6.732958in}{1.828744in}}%
\pgfusepath{stroke}%
\end{pgfscope}%
\begin{pgfscope}%
\pgfsetrectcap%
\pgfsetmiterjoin%
\pgfsetlinewidth{0.803000pt}%
\definecolor{currentstroke}{rgb}{1.000000,1.000000,1.000000}%
\pgfsetstrokecolor{currentstroke}%
\pgfsetdash{}{0pt}%
\pgfpathmoveto{\pgfqpoint{0.814776in}{0.521603in}}%
\pgfpathlineto{\pgfqpoint{0.814776in}{2.786603in}}%
\pgfusepath{stroke}%
\end{pgfscope}%
\begin{pgfscope}%
\pgfsetrectcap%
\pgfsetmiterjoin%
\pgfsetlinewidth{0.803000pt}%
\definecolor{currentstroke}{rgb}{1.000000,1.000000,1.000000}%
\pgfsetstrokecolor{currentstroke}%
\pgfsetdash{}{0pt}%
\pgfpathmoveto{\pgfqpoint{7.014776in}{0.521603in}}%
\pgfpathlineto{\pgfqpoint{7.014776in}{2.786603in}}%
\pgfusepath{stroke}%
\end{pgfscope}%
\begin{pgfscope}%
\pgfsetrectcap%
\pgfsetmiterjoin%
\pgfsetlinewidth{0.803000pt}%
\definecolor{currentstroke}{rgb}{1.000000,1.000000,1.000000}%
\pgfsetstrokecolor{currentstroke}%
\pgfsetdash{}{0pt}%
\pgfpathmoveto{\pgfqpoint{0.814776in}{0.521603in}}%
\pgfpathlineto{\pgfqpoint{7.014776in}{0.521603in}}%
\pgfusepath{stroke}%
\end{pgfscope}%
\begin{pgfscope}%
\pgfsetrectcap%
\pgfsetmiterjoin%
\pgfsetlinewidth{0.803000pt}%
\definecolor{currentstroke}{rgb}{1.000000,1.000000,1.000000}%
\pgfsetstrokecolor{currentstroke}%
\pgfsetdash{}{0pt}%
\pgfpathmoveto{\pgfqpoint{0.814776in}{2.786603in}}%
\pgfpathlineto{\pgfqpoint{7.014776in}{2.786603in}}%
\pgfusepath{stroke}%
\end{pgfscope}%
\begin{pgfscope}%
\definecolor{textcolor}{rgb}{0.150000,0.150000,0.150000}%
\pgfsetstrokecolor{textcolor}%
\pgfsetfillcolor{textcolor}%
\pgftext[x=3.914776in,y=2.869937in,,base]{\color{textcolor}\rmfamily\fontsize{12.000000}{14.400000}\selectfont First Difference of Log Adjusted Closing Values PG}%
\end{pgfscope}%
\end{pgfpicture}%
\makeatother%
\endgroup%

    \end{adjustbox}  
    \caption{Log-returns (or equivalently, first differences of the log of adjusted closing prices) of the PG. Visually the data looks similar to white noise. The entire data can be seen in the appendix in figure \ref{fig:all_fd_log_adjclose}}
    \label{fig:PG_fd_log_adjclose}
\end{figure}{}

That this is indeed a suitable transformation can be more intuitively understood by looking at a different route of transforming the data: taking the log of the stock prices and then using the first difference of the logged time series. Logging the time series transforms an exponential trend in the time series into a linear one and also serves to stabilize the variance. However, the trend does not vanish and after the log-transformation, the log value of a stock at time t is still mostly determined by its log-value at time t-1. To remove this effect, the time series needs to be differenced. To put this into a clearer perspective we can look at the autocorrelation and partial autocorrelation function of the series. 

The autocorrelation at lag j is the correlation between an observation at time t with the observation at t-j. As the series is assumed to be stationary, the autocorrelation function (ACF) does not depend on t, but only on the number of periods that lie between one observation $y_t$ and another $y_{t+h}$
\begin{equation}
    \text{ACF}(h) = corr(y_t, y_{t+h})
\end{equation}{}
Partial autocorrelation between an observation $y_t$ and another observation $y_{t+1}$ is the correlation between $y_t$ and $y_{t+h}$ that is not already explained by a linear dependence on the observations in between $y_t$ and $y_{t+h}$. Formally this can be defined as
\begin{equation}
    \text{PACF}(h) = corr(y_t - \hat{y}_t, y_{t+h} - \hat{y}_{t+h})
\end{equation}
\begin{flalign*}
    &\text{where} && \hat{y}_{t + h} = \beta_1 y_{t+h-1} + \beta_2 y_{t+h-2} + ... + \beta_{h-1} y_{t+1} &&\\
    &\text{and} && \hat{y_t} = \beta_1 y_{t+1} + \beta_2 y_{t+2} + ... + \beta_{h-1} y_{t+h-1} &&
\end{flalign*}{}
are the linear combinations $\{ y_{t+1}, ..., y_{t+h-1} \}$ that minimize the mean squared error of a regression of $y_{t+h}$, and $y_t$ respectively, on $\{ y_{t+1}, ..., y_{t+h-1}\}$. Both $y_t - \hat{y_t}$ and  $y_{t+h} - \hat{y}_{t+h}$ are uncorrelated with $\{ y_{t+1}, ..., y_{t+h-1} \}$. For now, however, it suffices to know that an ACF which is very slowly decaying to zero is an indicator that differencing may be appropriate (see Shumway and Stoffer 2011, p. 145) to make the series stationary. A large partial autocorrelation at lag 1, as shown in figure \ref{fig:acf_pacf_log_adjclose} also supports the conjecture that the dependence of the current on the previous value can be eliminated through differencing. After differencing we arrive again at the log-returns as $ \log{r^{\scriptscriptstyle{(1)}}_t} = \log{\frac{y_t}{y_{t-1}}} = \log{y_t} - \log{y_{t-1}} $. We can see now that the the partial autocorrelation at lag 1 has vanished after differecing and that the autocorrelation has also dropped to insignificance as is illustrated in figure \ref{fig:PG_autocorr_fd_log_adjclose}. We can also see that we have not induced any negative autocorrelation. The data therefore is not overdifferenced. The means of our time series is very close to zero (as shown in table \ref{tab:log return means}). Overall this pattern strongly suggests the time series are stationary now. However we can also perform a formal test whether our data is stationary or not. The augmented Dickey-Fuller test (ADF). P-values of the ADF for all log-returns are smaller than $10^{-12}$ even after correcting for multiple testing so we can safely assume the time series are stationary. 

\begin{figure}[h]
    \centering
    \figuretitle{ACF and PACF for prices of stock PG}
    \begin{adjustbox}{width=.95\textwidth,center}
    %% Creator: Matplotlib, PGF backend
%%
%% To include the figure in your LaTeX document, write
%%   \input{<filename>.pgf}
%%
%% Make sure the required packages are loaded in your preamble
%%   \usepackage{pgf}
%%
%% Figures using additional raster images can only be included by \input if
%% they are in the same directory as the main LaTeX file. For loading figures
%% from other directories you can use the `import` package
%%   \usepackage{import}
%% and then include the figures with
%%   \import{<path to file>}{<filename>.pgf}
%%
%% Matplotlib used the following preamble
%%   \usepackage{fontspec}
%%   \setmainfont{DejaVuSerif.ttf}[Path=/opt/tljh/user/lib/python3.6/site-packages/matplotlib/mpl-data/fonts/ttf/]
%%   \setsansfont{DejaVuSans.ttf}[Path=/opt/tljh/user/lib/python3.6/site-packages/matplotlib/mpl-data/fonts/ttf/]
%%   \setmonofont{DejaVuSansMono.ttf}[Path=/opt/tljh/user/lib/python3.6/site-packages/matplotlib/mpl-data/fonts/ttf/]
%%
\begingroup%
\makeatletter%
\begin{pgfpicture}%
\pgfpathrectangle{\pgfpointorigin}{\pgfqpoint{8.000000in}{4.000000in}}%
\pgfusepath{use as bounding box, clip}%
\begin{pgfscope}%
\pgfsetbuttcap%
\pgfsetmiterjoin%
\definecolor{currentfill}{rgb}{1.000000,1.000000,1.000000}%
\pgfsetfillcolor{currentfill}%
\pgfsetlinewidth{0.000000pt}%
\definecolor{currentstroke}{rgb}{1.000000,1.000000,1.000000}%
\pgfsetstrokecolor{currentstroke}%
\pgfsetdash{}{0pt}%
\pgfpathmoveto{\pgfqpoint{0.000000in}{0.000000in}}%
\pgfpathlineto{\pgfqpoint{8.000000in}{0.000000in}}%
\pgfpathlineto{\pgfqpoint{8.000000in}{4.000000in}}%
\pgfpathlineto{\pgfqpoint{0.000000in}{4.000000in}}%
\pgfpathclose%
\pgfusepath{fill}%
\end{pgfscope}%
\begin{pgfscope}%
\pgfsetbuttcap%
\pgfsetmiterjoin%
\definecolor{currentfill}{rgb}{0.917647,0.917647,0.949020}%
\pgfsetfillcolor{currentfill}%
\pgfsetlinewidth{0.000000pt}%
\definecolor{currentstroke}{rgb}{0.000000,0.000000,0.000000}%
\pgfsetstrokecolor{currentstroke}%
\pgfsetstrokeopacity{0.000000}%
\pgfsetdash{}{0pt}%
\pgfpathmoveto{\pgfqpoint{1.000000in}{0.500000in}}%
\pgfpathlineto{\pgfqpoint{3.818182in}{0.500000in}}%
\pgfpathlineto{\pgfqpoint{3.818182in}{3.520000in}}%
\pgfpathlineto{\pgfqpoint{1.000000in}{3.520000in}}%
\pgfpathclose%
\pgfusepath{fill}%
\end{pgfscope}%
\begin{pgfscope}%
\pgfpathrectangle{\pgfqpoint{1.000000in}{0.500000in}}{\pgfqpoint{2.818182in}{3.020000in}}%
\pgfusepath{clip}%
\pgfsetroundcap%
\pgfsetroundjoin%
\pgfsetlinewidth{0.803000pt}%
\definecolor{currentstroke}{rgb}{1.000000,1.000000,1.000000}%
\pgfsetstrokecolor{currentstroke}%
\pgfsetdash{}{0pt}%
\pgfpathmoveto{\pgfqpoint{1.128099in}{0.500000in}}%
\pgfpathlineto{\pgfqpoint{1.128099in}{3.520000in}}%
\pgfusepath{stroke}%
\end{pgfscope}%
\begin{pgfscope}%
\definecolor{textcolor}{rgb}{0.150000,0.150000,0.150000}%
\pgfsetstrokecolor{textcolor}%
\pgfsetfillcolor{textcolor}%
\pgftext[x=1.128099in,y=0.402778in,,top]{\color{textcolor}\rmfamily\fontsize{14.000000}{16.800000}\selectfont 0}%
\end{pgfscope}%
\begin{pgfscope}%
\pgfpathrectangle{\pgfqpoint{1.000000in}{0.500000in}}{\pgfqpoint{2.818182in}{3.020000in}}%
\pgfusepath{clip}%
\pgfsetroundcap%
\pgfsetroundjoin%
\pgfsetlinewidth{0.803000pt}%
\definecolor{currentstroke}{rgb}{1.000000,1.000000,1.000000}%
\pgfsetstrokecolor{currentstroke}%
\pgfsetdash{}{0pt}%
\pgfpathmoveto{\pgfqpoint{2.377847in}{0.500000in}}%
\pgfpathlineto{\pgfqpoint{2.377847in}{3.520000in}}%
\pgfusepath{stroke}%
\end{pgfscope}%
\begin{pgfscope}%
\definecolor{textcolor}{rgb}{0.150000,0.150000,0.150000}%
\pgfsetstrokecolor{textcolor}%
\pgfsetfillcolor{textcolor}%
\pgftext[x=2.377847in,y=0.402778in,,top]{\color{textcolor}\rmfamily\fontsize{14.000000}{16.800000}\selectfont 10}%
\end{pgfscope}%
\begin{pgfscope}%
\pgfpathrectangle{\pgfqpoint{1.000000in}{0.500000in}}{\pgfqpoint{2.818182in}{3.020000in}}%
\pgfusepath{clip}%
\pgfsetroundcap%
\pgfsetroundjoin%
\pgfsetlinewidth{0.803000pt}%
\definecolor{currentstroke}{rgb}{1.000000,1.000000,1.000000}%
\pgfsetstrokecolor{currentstroke}%
\pgfsetdash{}{0pt}%
\pgfpathmoveto{\pgfqpoint{3.627595in}{0.500000in}}%
\pgfpathlineto{\pgfqpoint{3.627595in}{3.520000in}}%
\pgfusepath{stroke}%
\end{pgfscope}%
\begin{pgfscope}%
\definecolor{textcolor}{rgb}{0.150000,0.150000,0.150000}%
\pgfsetstrokecolor{textcolor}%
\pgfsetfillcolor{textcolor}%
\pgftext[x=3.627595in,y=0.402778in,,top]{\color{textcolor}\rmfamily\fontsize{14.000000}{16.800000}\selectfont 20}%
\end{pgfscope}%
\begin{pgfscope}%
\pgfpathrectangle{\pgfqpoint{1.000000in}{0.500000in}}{\pgfqpoint{2.818182in}{3.020000in}}%
\pgfusepath{clip}%
\pgfsetroundcap%
\pgfsetroundjoin%
\pgfsetlinewidth{0.803000pt}%
\definecolor{currentstroke}{rgb}{1.000000,1.000000,1.000000}%
\pgfsetstrokecolor{currentstroke}%
\pgfsetdash{}{0pt}%
\pgfpathmoveto{\pgfqpoint{1.000000in}{0.752224in}}%
\pgfpathlineto{\pgfqpoint{3.818182in}{0.752224in}}%
\pgfusepath{stroke}%
\end{pgfscope}%
\begin{pgfscope}%
\definecolor{textcolor}{rgb}{0.150000,0.150000,0.150000}%
\pgfsetstrokecolor{textcolor}%
\pgfsetfillcolor{textcolor}%
\pgftext[x=0.306912in,y=0.678358in,left,base]{\color{textcolor}\rmfamily\fontsize{14.000000}{16.800000}\selectfont −0.25}%
\end{pgfscope}%
\begin{pgfscope}%
\pgfpathrectangle{\pgfqpoint{1.000000in}{0.500000in}}{\pgfqpoint{2.818182in}{3.020000in}}%
\pgfusepath{clip}%
\pgfsetroundcap%
\pgfsetroundjoin%
\pgfsetlinewidth{0.803000pt}%
\definecolor{currentstroke}{rgb}{1.000000,1.000000,1.000000}%
\pgfsetstrokecolor{currentstroke}%
\pgfsetdash{}{0pt}%
\pgfpathmoveto{\pgfqpoint{1.000000in}{1.278324in}}%
\pgfpathlineto{\pgfqpoint{3.818182in}{1.278324in}}%
\pgfusepath{stroke}%
\end{pgfscope}%
\begin{pgfscope}%
\definecolor{textcolor}{rgb}{0.150000,0.150000,0.150000}%
\pgfsetstrokecolor{textcolor}%
\pgfsetfillcolor{textcolor}%
\pgftext[x=0.469835in,y=1.204458in,left,base]{\color{textcolor}\rmfamily\fontsize{14.000000}{16.800000}\selectfont 0.00}%
\end{pgfscope}%
\begin{pgfscope}%
\pgfpathrectangle{\pgfqpoint{1.000000in}{0.500000in}}{\pgfqpoint{2.818182in}{3.020000in}}%
\pgfusepath{clip}%
\pgfsetroundcap%
\pgfsetroundjoin%
\pgfsetlinewidth{0.803000pt}%
\definecolor{currentstroke}{rgb}{1.000000,1.000000,1.000000}%
\pgfsetstrokecolor{currentstroke}%
\pgfsetdash{}{0pt}%
\pgfpathmoveto{\pgfqpoint{1.000000in}{1.804425in}}%
\pgfpathlineto{\pgfqpoint{3.818182in}{1.804425in}}%
\pgfusepath{stroke}%
\end{pgfscope}%
\begin{pgfscope}%
\definecolor{textcolor}{rgb}{0.150000,0.150000,0.150000}%
\pgfsetstrokecolor{textcolor}%
\pgfsetfillcolor{textcolor}%
\pgftext[x=0.469835in,y=1.730559in,left,base]{\color{textcolor}\rmfamily\fontsize{14.000000}{16.800000}\selectfont 0.25}%
\end{pgfscope}%
\begin{pgfscope}%
\pgfpathrectangle{\pgfqpoint{1.000000in}{0.500000in}}{\pgfqpoint{2.818182in}{3.020000in}}%
\pgfusepath{clip}%
\pgfsetroundcap%
\pgfsetroundjoin%
\pgfsetlinewidth{0.803000pt}%
\definecolor{currentstroke}{rgb}{1.000000,1.000000,1.000000}%
\pgfsetstrokecolor{currentstroke}%
\pgfsetdash{}{0pt}%
\pgfpathmoveto{\pgfqpoint{1.000000in}{2.330526in}}%
\pgfpathlineto{\pgfqpoint{3.818182in}{2.330526in}}%
\pgfusepath{stroke}%
\end{pgfscope}%
\begin{pgfscope}%
\definecolor{textcolor}{rgb}{0.150000,0.150000,0.150000}%
\pgfsetstrokecolor{textcolor}%
\pgfsetfillcolor{textcolor}%
\pgftext[x=0.469835in,y=2.256660in,left,base]{\color{textcolor}\rmfamily\fontsize{14.000000}{16.800000}\selectfont 0.50}%
\end{pgfscope}%
\begin{pgfscope}%
\pgfpathrectangle{\pgfqpoint{1.000000in}{0.500000in}}{\pgfqpoint{2.818182in}{3.020000in}}%
\pgfusepath{clip}%
\pgfsetroundcap%
\pgfsetroundjoin%
\pgfsetlinewidth{0.803000pt}%
\definecolor{currentstroke}{rgb}{1.000000,1.000000,1.000000}%
\pgfsetstrokecolor{currentstroke}%
\pgfsetdash{}{0pt}%
\pgfpathmoveto{\pgfqpoint{1.000000in}{2.856627in}}%
\pgfpathlineto{\pgfqpoint{3.818182in}{2.856627in}}%
\pgfusepath{stroke}%
\end{pgfscope}%
\begin{pgfscope}%
\definecolor{textcolor}{rgb}{0.150000,0.150000,0.150000}%
\pgfsetstrokecolor{textcolor}%
\pgfsetfillcolor{textcolor}%
\pgftext[x=0.469835in,y=2.782760in,left,base]{\color{textcolor}\rmfamily\fontsize{14.000000}{16.800000}\selectfont 0.75}%
\end{pgfscope}%
\begin{pgfscope}%
\pgfpathrectangle{\pgfqpoint{1.000000in}{0.500000in}}{\pgfqpoint{2.818182in}{3.020000in}}%
\pgfusepath{clip}%
\pgfsetroundcap%
\pgfsetroundjoin%
\pgfsetlinewidth{0.803000pt}%
\definecolor{currentstroke}{rgb}{1.000000,1.000000,1.000000}%
\pgfsetstrokecolor{currentstroke}%
\pgfsetdash{}{0pt}%
\pgfpathmoveto{\pgfqpoint{1.000000in}{3.382727in}}%
\pgfpathlineto{\pgfqpoint{3.818182in}{3.382727in}}%
\pgfusepath{stroke}%
\end{pgfscope}%
\begin{pgfscope}%
\definecolor{textcolor}{rgb}{0.150000,0.150000,0.150000}%
\pgfsetstrokecolor{textcolor}%
\pgfsetfillcolor{textcolor}%
\pgftext[x=0.469835in,y=3.308861in,left,base]{\color{textcolor}\rmfamily\fontsize{14.000000}{16.800000}\selectfont 1.00}%
\end{pgfscope}%
\begin{pgfscope}%
\pgfpathrectangle{\pgfqpoint{1.000000in}{0.500000in}}{\pgfqpoint{2.818182in}{3.020000in}}%
\pgfusepath{clip}%
\pgfsetbuttcap%
\pgfsetroundjoin%
\definecolor{currentfill}{rgb}{0.121569,0.466667,0.705882}%
\pgfsetfillcolor{currentfill}%
\pgfsetfillopacity{0.250000}%
\pgfsetlinewidth{1.003750pt}%
\definecolor{currentstroke}{rgb}{1.000000,1.000000,1.000000}%
\pgfsetstrokecolor{currentstroke}%
\pgfsetstrokeopacity{0.250000}%
\pgfsetdash{}{0pt}%
\pgfpathmoveto{\pgfqpoint{1.190587in}{1.384537in}}%
\pgfpathlineto{\pgfqpoint{1.190587in}{1.172112in}}%
\pgfpathlineto{\pgfqpoint{1.378049in}{1.094798in}}%
\pgfpathlineto{\pgfqpoint{1.503024in}{1.041846in}}%
\pgfpathlineto{\pgfqpoint{1.627998in}{0.999030in}}%
\pgfpathlineto{\pgfqpoint{1.752973in}{0.962200in}}%
\pgfpathlineto{\pgfqpoint{1.877948in}{0.929444in}}%
\pgfpathlineto{\pgfqpoint{2.002923in}{0.899713in}}%
\pgfpathlineto{\pgfqpoint{2.127898in}{0.872343in}}%
\pgfpathlineto{\pgfqpoint{2.252872in}{0.846885in}}%
\pgfpathlineto{\pgfqpoint{2.377847in}{0.823019in}}%
\pgfpathlineto{\pgfqpoint{2.502822in}{0.800499in}}%
\pgfpathlineto{\pgfqpoint{2.627797in}{0.779136in}}%
\pgfpathlineto{\pgfqpoint{2.752772in}{0.758787in}}%
\pgfpathlineto{\pgfqpoint{2.877746in}{0.739337in}}%
\pgfpathlineto{\pgfqpoint{3.002721in}{0.720697in}}%
\pgfpathlineto{\pgfqpoint{3.127696in}{0.702785in}}%
\pgfpathlineto{\pgfqpoint{3.252671in}{0.685531in}}%
\pgfpathlineto{\pgfqpoint{3.377646in}{0.668889in}}%
\pgfpathlineto{\pgfqpoint{3.502620in}{0.652817in}}%
\pgfpathlineto{\pgfqpoint{3.690083in}{0.637273in}}%
\pgfpathlineto{\pgfqpoint{3.690083in}{1.919376in}}%
\pgfpathlineto{\pgfqpoint{3.690083in}{1.919376in}}%
\pgfpathlineto{\pgfqpoint{3.502620in}{1.903832in}}%
\pgfpathlineto{\pgfqpoint{3.377646in}{1.887760in}}%
\pgfpathlineto{\pgfqpoint{3.252671in}{1.871118in}}%
\pgfpathlineto{\pgfqpoint{3.127696in}{1.853864in}}%
\pgfpathlineto{\pgfqpoint{3.002721in}{1.835952in}}%
\pgfpathlineto{\pgfqpoint{2.877746in}{1.817311in}}%
\pgfpathlineto{\pgfqpoint{2.752772in}{1.797862in}}%
\pgfpathlineto{\pgfqpoint{2.627797in}{1.777512in}}%
\pgfpathlineto{\pgfqpoint{2.502822in}{1.756150in}}%
\pgfpathlineto{\pgfqpoint{2.377847in}{1.733630in}}%
\pgfpathlineto{\pgfqpoint{2.252872in}{1.709763in}}%
\pgfpathlineto{\pgfqpoint{2.127898in}{1.684306in}}%
\pgfpathlineto{\pgfqpoint{2.002923in}{1.656936in}}%
\pgfpathlineto{\pgfqpoint{1.877948in}{1.627205in}}%
\pgfpathlineto{\pgfqpoint{1.752973in}{1.594449in}}%
\pgfpathlineto{\pgfqpoint{1.627998in}{1.557619in}}%
\pgfpathlineto{\pgfqpoint{1.503024in}{1.514803in}}%
\pgfpathlineto{\pgfqpoint{1.378049in}{1.461850in}}%
\pgfpathlineto{\pgfqpoint{1.190587in}{1.384537in}}%
\pgfpathclose%
\pgfusepath{stroke,fill}%
\end{pgfscope}%
\begin{pgfscope}%
\pgfpathrectangle{\pgfqpoint{1.000000in}{0.500000in}}{\pgfqpoint{2.818182in}{3.020000in}}%
\pgfusepath{clip}%
\pgfsetbuttcap%
\pgfsetroundjoin%
\pgfsetlinewidth{1.505625pt}%
\definecolor{currentstroke}{rgb}{0.000000,0.000000,0.000000}%
\pgfsetstrokecolor{currentstroke}%
\pgfsetdash{}{0pt}%
\pgfpathmoveto{\pgfqpoint{1.128099in}{1.278324in}}%
\pgfpathlineto{\pgfqpoint{1.128099in}{3.382727in}}%
\pgfusepath{stroke}%
\end{pgfscope}%
\begin{pgfscope}%
\pgfpathrectangle{\pgfqpoint{1.000000in}{0.500000in}}{\pgfqpoint{2.818182in}{3.020000in}}%
\pgfusepath{clip}%
\pgfsetbuttcap%
\pgfsetroundjoin%
\pgfsetlinewidth{1.505625pt}%
\definecolor{currentstroke}{rgb}{0.000000,0.000000,0.000000}%
\pgfsetstrokecolor{currentstroke}%
\pgfsetdash{}{0pt}%
\pgfpathmoveto{\pgfqpoint{1.253074in}{1.278324in}}%
\pgfpathlineto{\pgfqpoint{1.253074in}{3.375177in}}%
\pgfusepath{stroke}%
\end{pgfscope}%
\begin{pgfscope}%
\pgfpathrectangle{\pgfqpoint{1.000000in}{0.500000in}}{\pgfqpoint{2.818182in}{3.020000in}}%
\pgfusepath{clip}%
\pgfsetbuttcap%
\pgfsetroundjoin%
\pgfsetlinewidth{1.505625pt}%
\definecolor{currentstroke}{rgb}{0.000000,0.000000,0.000000}%
\pgfsetstrokecolor{currentstroke}%
\pgfsetdash{}{0pt}%
\pgfpathmoveto{\pgfqpoint{1.378049in}{1.278324in}}%
\pgfpathlineto{\pgfqpoint{1.378049in}{3.367658in}}%
\pgfusepath{stroke}%
\end{pgfscope}%
\begin{pgfscope}%
\pgfpathrectangle{\pgfqpoint{1.000000in}{0.500000in}}{\pgfqpoint{2.818182in}{3.020000in}}%
\pgfusepath{clip}%
\pgfsetbuttcap%
\pgfsetroundjoin%
\pgfsetlinewidth{1.505625pt}%
\definecolor{currentstroke}{rgb}{0.000000,0.000000,0.000000}%
\pgfsetstrokecolor{currentstroke}%
\pgfsetdash{}{0pt}%
\pgfpathmoveto{\pgfqpoint{1.503024in}{1.278324in}}%
\pgfpathlineto{\pgfqpoint{1.503024in}{3.360274in}}%
\pgfusepath{stroke}%
\end{pgfscope}%
\begin{pgfscope}%
\pgfpathrectangle{\pgfqpoint{1.000000in}{0.500000in}}{\pgfqpoint{2.818182in}{3.020000in}}%
\pgfusepath{clip}%
\pgfsetbuttcap%
\pgfsetroundjoin%
\pgfsetlinewidth{1.505625pt}%
\definecolor{currentstroke}{rgb}{0.000000,0.000000,0.000000}%
\pgfsetstrokecolor{currentstroke}%
\pgfsetdash{}{0pt}%
\pgfpathmoveto{\pgfqpoint{1.627998in}{1.278324in}}%
\pgfpathlineto{\pgfqpoint{1.627998in}{3.352997in}}%
\pgfusepath{stroke}%
\end{pgfscope}%
\begin{pgfscope}%
\pgfpathrectangle{\pgfqpoint{1.000000in}{0.500000in}}{\pgfqpoint{2.818182in}{3.020000in}}%
\pgfusepath{clip}%
\pgfsetbuttcap%
\pgfsetroundjoin%
\pgfsetlinewidth{1.505625pt}%
\definecolor{currentstroke}{rgb}{0.000000,0.000000,0.000000}%
\pgfsetstrokecolor{currentstroke}%
\pgfsetdash{}{0pt}%
\pgfpathmoveto{\pgfqpoint{1.752973in}{1.278324in}}%
\pgfpathlineto{\pgfqpoint{1.752973in}{3.346061in}}%
\pgfusepath{stroke}%
\end{pgfscope}%
\begin{pgfscope}%
\pgfpathrectangle{\pgfqpoint{1.000000in}{0.500000in}}{\pgfqpoint{2.818182in}{3.020000in}}%
\pgfusepath{clip}%
\pgfsetbuttcap%
\pgfsetroundjoin%
\pgfsetlinewidth{1.505625pt}%
\definecolor{currentstroke}{rgb}{0.000000,0.000000,0.000000}%
\pgfsetstrokecolor{currentstroke}%
\pgfsetdash{}{0pt}%
\pgfpathmoveto{\pgfqpoint{1.877948in}{1.278324in}}%
\pgfpathlineto{\pgfqpoint{1.877948in}{3.338751in}}%
\pgfusepath{stroke}%
\end{pgfscope}%
\begin{pgfscope}%
\pgfpathrectangle{\pgfqpoint{1.000000in}{0.500000in}}{\pgfqpoint{2.818182in}{3.020000in}}%
\pgfusepath{clip}%
\pgfsetbuttcap%
\pgfsetroundjoin%
\pgfsetlinewidth{1.505625pt}%
\definecolor{currentstroke}{rgb}{0.000000,0.000000,0.000000}%
\pgfsetstrokecolor{currentstroke}%
\pgfsetdash{}{0pt}%
\pgfpathmoveto{\pgfqpoint{2.002923in}{1.278324in}}%
\pgfpathlineto{\pgfqpoint{2.002923in}{3.331376in}}%
\pgfusepath{stroke}%
\end{pgfscope}%
\begin{pgfscope}%
\pgfpathrectangle{\pgfqpoint{1.000000in}{0.500000in}}{\pgfqpoint{2.818182in}{3.020000in}}%
\pgfusepath{clip}%
\pgfsetbuttcap%
\pgfsetroundjoin%
\pgfsetlinewidth{1.505625pt}%
\definecolor{currentstroke}{rgb}{0.000000,0.000000,0.000000}%
\pgfsetstrokecolor{currentstroke}%
\pgfsetdash{}{0pt}%
\pgfpathmoveto{\pgfqpoint{2.127898in}{1.278324in}}%
\pgfpathlineto{\pgfqpoint{2.127898in}{3.323893in}}%
\pgfusepath{stroke}%
\end{pgfscope}%
\begin{pgfscope}%
\pgfpathrectangle{\pgfqpoint{1.000000in}{0.500000in}}{\pgfqpoint{2.818182in}{3.020000in}}%
\pgfusepath{clip}%
\pgfsetbuttcap%
\pgfsetroundjoin%
\pgfsetlinewidth{1.505625pt}%
\definecolor{currentstroke}{rgb}{0.000000,0.000000,0.000000}%
\pgfsetstrokecolor{currentstroke}%
\pgfsetdash{}{0pt}%
\pgfpathmoveto{\pgfqpoint{2.252872in}{1.278324in}}%
\pgfpathlineto{\pgfqpoint{2.252872in}{3.316446in}}%
\pgfusepath{stroke}%
\end{pgfscope}%
\begin{pgfscope}%
\pgfpathrectangle{\pgfqpoint{1.000000in}{0.500000in}}{\pgfqpoint{2.818182in}{3.020000in}}%
\pgfusepath{clip}%
\pgfsetbuttcap%
\pgfsetroundjoin%
\pgfsetlinewidth{1.505625pt}%
\definecolor{currentstroke}{rgb}{0.000000,0.000000,0.000000}%
\pgfsetstrokecolor{currentstroke}%
\pgfsetdash{}{0pt}%
\pgfpathmoveto{\pgfqpoint{2.377847in}{1.278324in}}%
\pgfpathlineto{\pgfqpoint{2.377847in}{3.309238in}}%
\pgfusepath{stroke}%
\end{pgfscope}%
\begin{pgfscope}%
\pgfpathrectangle{\pgfqpoint{1.000000in}{0.500000in}}{\pgfqpoint{2.818182in}{3.020000in}}%
\pgfusepath{clip}%
\pgfsetbuttcap%
\pgfsetroundjoin%
\pgfsetlinewidth{1.505625pt}%
\definecolor{currentstroke}{rgb}{0.000000,0.000000,0.000000}%
\pgfsetstrokecolor{currentstroke}%
\pgfsetdash{}{0pt}%
\pgfpathmoveto{\pgfqpoint{2.502822in}{1.278324in}}%
\pgfpathlineto{\pgfqpoint{2.502822in}{3.302354in}}%
\pgfusepath{stroke}%
\end{pgfscope}%
\begin{pgfscope}%
\pgfpathrectangle{\pgfqpoint{1.000000in}{0.500000in}}{\pgfqpoint{2.818182in}{3.020000in}}%
\pgfusepath{clip}%
\pgfsetbuttcap%
\pgfsetroundjoin%
\pgfsetlinewidth{1.505625pt}%
\definecolor{currentstroke}{rgb}{0.000000,0.000000,0.000000}%
\pgfsetstrokecolor{currentstroke}%
\pgfsetdash{}{0pt}%
\pgfpathmoveto{\pgfqpoint{2.627797in}{1.278324in}}%
\pgfpathlineto{\pgfqpoint{2.627797in}{3.295513in}}%
\pgfusepath{stroke}%
\end{pgfscope}%
\begin{pgfscope}%
\pgfpathrectangle{\pgfqpoint{1.000000in}{0.500000in}}{\pgfqpoint{2.818182in}{3.020000in}}%
\pgfusepath{clip}%
\pgfsetbuttcap%
\pgfsetroundjoin%
\pgfsetlinewidth{1.505625pt}%
\definecolor{currentstroke}{rgb}{0.000000,0.000000,0.000000}%
\pgfsetstrokecolor{currentstroke}%
\pgfsetdash{}{0pt}%
\pgfpathmoveto{\pgfqpoint{2.752772in}{1.278324in}}%
\pgfpathlineto{\pgfqpoint{2.752772in}{3.288523in}}%
\pgfusepath{stroke}%
\end{pgfscope}%
\begin{pgfscope}%
\pgfpathrectangle{\pgfqpoint{1.000000in}{0.500000in}}{\pgfqpoint{2.818182in}{3.020000in}}%
\pgfusepath{clip}%
\pgfsetbuttcap%
\pgfsetroundjoin%
\pgfsetlinewidth{1.505625pt}%
\definecolor{currentstroke}{rgb}{0.000000,0.000000,0.000000}%
\pgfsetstrokecolor{currentstroke}%
\pgfsetdash{}{0pt}%
\pgfpathmoveto{\pgfqpoint{2.877746in}{1.278324in}}%
\pgfpathlineto{\pgfqpoint{2.877746in}{3.281378in}}%
\pgfusepath{stroke}%
\end{pgfscope}%
\begin{pgfscope}%
\pgfpathrectangle{\pgfqpoint{1.000000in}{0.500000in}}{\pgfqpoint{2.818182in}{3.020000in}}%
\pgfusepath{clip}%
\pgfsetbuttcap%
\pgfsetroundjoin%
\pgfsetlinewidth{1.505625pt}%
\definecolor{currentstroke}{rgb}{0.000000,0.000000,0.000000}%
\pgfsetstrokecolor{currentstroke}%
\pgfsetdash{}{0pt}%
\pgfpathmoveto{\pgfqpoint{3.002721in}{1.278324in}}%
\pgfpathlineto{\pgfqpoint{3.002721in}{3.274331in}}%
\pgfusepath{stroke}%
\end{pgfscope}%
\begin{pgfscope}%
\pgfpathrectangle{\pgfqpoint{1.000000in}{0.500000in}}{\pgfqpoint{2.818182in}{3.020000in}}%
\pgfusepath{clip}%
\pgfsetbuttcap%
\pgfsetroundjoin%
\pgfsetlinewidth{1.505625pt}%
\definecolor{currentstroke}{rgb}{0.000000,0.000000,0.000000}%
\pgfsetstrokecolor{currentstroke}%
\pgfsetdash{}{0pt}%
\pgfpathmoveto{\pgfqpoint{3.127696in}{1.278324in}}%
\pgfpathlineto{\pgfqpoint{3.127696in}{3.267464in}}%
\pgfusepath{stroke}%
\end{pgfscope}%
\begin{pgfscope}%
\pgfpathrectangle{\pgfqpoint{1.000000in}{0.500000in}}{\pgfqpoint{2.818182in}{3.020000in}}%
\pgfusepath{clip}%
\pgfsetbuttcap%
\pgfsetroundjoin%
\pgfsetlinewidth{1.505625pt}%
\definecolor{currentstroke}{rgb}{0.000000,0.000000,0.000000}%
\pgfsetstrokecolor{currentstroke}%
\pgfsetdash{}{0pt}%
\pgfpathmoveto{\pgfqpoint{3.252671in}{1.278324in}}%
\pgfpathlineto{\pgfqpoint{3.252671in}{3.260007in}}%
\pgfusepath{stroke}%
\end{pgfscope}%
\begin{pgfscope}%
\pgfpathrectangle{\pgfqpoint{1.000000in}{0.500000in}}{\pgfqpoint{2.818182in}{3.020000in}}%
\pgfusepath{clip}%
\pgfsetbuttcap%
\pgfsetroundjoin%
\pgfsetlinewidth{1.505625pt}%
\definecolor{currentstroke}{rgb}{0.000000,0.000000,0.000000}%
\pgfsetstrokecolor{currentstroke}%
\pgfsetdash{}{0pt}%
\pgfpathmoveto{\pgfqpoint{3.377646in}{1.278324in}}%
\pgfpathlineto{\pgfqpoint{3.377646in}{3.252088in}}%
\pgfusepath{stroke}%
\end{pgfscope}%
\begin{pgfscope}%
\pgfpathrectangle{\pgfqpoint{1.000000in}{0.500000in}}{\pgfqpoint{2.818182in}{3.020000in}}%
\pgfusepath{clip}%
\pgfsetbuttcap%
\pgfsetroundjoin%
\pgfsetlinewidth{1.505625pt}%
\definecolor{currentstroke}{rgb}{0.000000,0.000000,0.000000}%
\pgfsetstrokecolor{currentstroke}%
\pgfsetdash{}{0pt}%
\pgfpathmoveto{\pgfqpoint{3.502620in}{1.278324in}}%
\pgfpathlineto{\pgfqpoint{3.502620in}{3.244086in}}%
\pgfusepath{stroke}%
\end{pgfscope}%
\begin{pgfscope}%
\pgfpathrectangle{\pgfqpoint{1.000000in}{0.500000in}}{\pgfqpoint{2.818182in}{3.020000in}}%
\pgfusepath{clip}%
\pgfsetbuttcap%
\pgfsetroundjoin%
\pgfsetlinewidth{1.505625pt}%
\definecolor{currentstroke}{rgb}{0.000000,0.000000,0.000000}%
\pgfsetstrokecolor{currentstroke}%
\pgfsetdash{}{0pt}%
\pgfpathmoveto{\pgfqpoint{3.627595in}{1.278324in}}%
\pgfpathlineto{\pgfqpoint{3.627595in}{3.236240in}}%
\pgfusepath{stroke}%
\end{pgfscope}%
\begin{pgfscope}%
\pgfpathrectangle{\pgfqpoint{1.000000in}{0.500000in}}{\pgfqpoint{2.818182in}{3.020000in}}%
\pgfusepath{clip}%
\pgfsetroundcap%
\pgfsetroundjoin%
\pgfsetlinewidth{1.505625pt}%
\definecolor{currentstroke}{rgb}{0.549020,0.337255,0.294118}%
\pgfsetstrokecolor{currentstroke}%
\pgfsetdash{}{0pt}%
\pgfpathmoveto{\pgfqpoint{1.000000in}{1.278324in}}%
\pgfpathlineto{\pgfqpoint{3.818182in}{1.278324in}}%
\pgfusepath{stroke}%
\end{pgfscope}%
\begin{pgfscope}%
\pgfpathrectangle{\pgfqpoint{1.000000in}{0.500000in}}{\pgfqpoint{2.818182in}{3.020000in}}%
\pgfusepath{clip}%
\pgfsetbuttcap%
\pgfsetroundjoin%
\definecolor{currentfill}{rgb}{0.549020,0.337255,0.294118}%
\pgfsetfillcolor{currentfill}%
\pgfsetlinewidth{1.003750pt}%
\definecolor{currentstroke}{rgb}{0.549020,0.337255,0.294118}%
\pgfsetstrokecolor{currentstroke}%
\pgfsetdash{}{0pt}%
\pgfsys@defobject{currentmarker}{\pgfqpoint{-0.034722in}{-0.034722in}}{\pgfqpoint{0.034722in}{0.034722in}}{%
\pgfpathmoveto{\pgfqpoint{0.000000in}{-0.034722in}}%
\pgfpathcurveto{\pgfqpoint{0.009208in}{-0.034722in}}{\pgfqpoint{0.018041in}{-0.031064in}}{\pgfqpoint{0.024552in}{-0.024552in}}%
\pgfpathcurveto{\pgfqpoint{0.031064in}{-0.018041in}}{\pgfqpoint{0.034722in}{-0.009208in}}{\pgfqpoint{0.034722in}{0.000000in}}%
\pgfpathcurveto{\pgfqpoint{0.034722in}{0.009208in}}{\pgfqpoint{0.031064in}{0.018041in}}{\pgfqpoint{0.024552in}{0.024552in}}%
\pgfpathcurveto{\pgfqpoint{0.018041in}{0.031064in}}{\pgfqpoint{0.009208in}{0.034722in}}{\pgfqpoint{0.000000in}{0.034722in}}%
\pgfpathcurveto{\pgfqpoint{-0.009208in}{0.034722in}}{\pgfqpoint{-0.018041in}{0.031064in}}{\pgfqpoint{-0.024552in}{0.024552in}}%
\pgfpathcurveto{\pgfqpoint{-0.031064in}{0.018041in}}{\pgfqpoint{-0.034722in}{0.009208in}}{\pgfqpoint{-0.034722in}{0.000000in}}%
\pgfpathcurveto{\pgfqpoint{-0.034722in}{-0.009208in}}{\pgfqpoint{-0.031064in}{-0.018041in}}{\pgfqpoint{-0.024552in}{-0.024552in}}%
\pgfpathcurveto{\pgfqpoint{-0.018041in}{-0.031064in}}{\pgfqpoint{-0.009208in}{-0.034722in}}{\pgfqpoint{0.000000in}{-0.034722in}}%
\pgfpathclose%
\pgfusepath{stroke,fill}%
}%
\begin{pgfscope}%
\pgfsys@transformshift{1.128099in}{3.382727in}%
\pgfsys@useobject{currentmarker}{}%
\end{pgfscope}%
\begin{pgfscope}%
\pgfsys@transformshift{1.253074in}{3.375177in}%
\pgfsys@useobject{currentmarker}{}%
\end{pgfscope}%
\begin{pgfscope}%
\pgfsys@transformshift{1.378049in}{3.367658in}%
\pgfsys@useobject{currentmarker}{}%
\end{pgfscope}%
\begin{pgfscope}%
\pgfsys@transformshift{1.503024in}{3.360274in}%
\pgfsys@useobject{currentmarker}{}%
\end{pgfscope}%
\begin{pgfscope}%
\pgfsys@transformshift{1.627998in}{3.352997in}%
\pgfsys@useobject{currentmarker}{}%
\end{pgfscope}%
\begin{pgfscope}%
\pgfsys@transformshift{1.752973in}{3.346061in}%
\pgfsys@useobject{currentmarker}{}%
\end{pgfscope}%
\begin{pgfscope}%
\pgfsys@transformshift{1.877948in}{3.338751in}%
\pgfsys@useobject{currentmarker}{}%
\end{pgfscope}%
\begin{pgfscope}%
\pgfsys@transformshift{2.002923in}{3.331376in}%
\pgfsys@useobject{currentmarker}{}%
\end{pgfscope}%
\begin{pgfscope}%
\pgfsys@transformshift{2.127898in}{3.323893in}%
\pgfsys@useobject{currentmarker}{}%
\end{pgfscope}%
\begin{pgfscope}%
\pgfsys@transformshift{2.252872in}{3.316446in}%
\pgfsys@useobject{currentmarker}{}%
\end{pgfscope}%
\begin{pgfscope}%
\pgfsys@transformshift{2.377847in}{3.309238in}%
\pgfsys@useobject{currentmarker}{}%
\end{pgfscope}%
\begin{pgfscope}%
\pgfsys@transformshift{2.502822in}{3.302354in}%
\pgfsys@useobject{currentmarker}{}%
\end{pgfscope}%
\begin{pgfscope}%
\pgfsys@transformshift{2.627797in}{3.295513in}%
\pgfsys@useobject{currentmarker}{}%
\end{pgfscope}%
\begin{pgfscope}%
\pgfsys@transformshift{2.752772in}{3.288523in}%
\pgfsys@useobject{currentmarker}{}%
\end{pgfscope}%
\begin{pgfscope}%
\pgfsys@transformshift{2.877746in}{3.281378in}%
\pgfsys@useobject{currentmarker}{}%
\end{pgfscope}%
\begin{pgfscope}%
\pgfsys@transformshift{3.002721in}{3.274331in}%
\pgfsys@useobject{currentmarker}{}%
\end{pgfscope}%
\begin{pgfscope}%
\pgfsys@transformshift{3.127696in}{3.267464in}%
\pgfsys@useobject{currentmarker}{}%
\end{pgfscope}%
\begin{pgfscope}%
\pgfsys@transformshift{3.252671in}{3.260007in}%
\pgfsys@useobject{currentmarker}{}%
\end{pgfscope}%
\begin{pgfscope}%
\pgfsys@transformshift{3.377646in}{3.252088in}%
\pgfsys@useobject{currentmarker}{}%
\end{pgfscope}%
\begin{pgfscope}%
\pgfsys@transformshift{3.502620in}{3.244086in}%
\pgfsys@useobject{currentmarker}{}%
\end{pgfscope}%
\begin{pgfscope}%
\pgfsys@transformshift{3.627595in}{3.236240in}%
\pgfsys@useobject{currentmarker}{}%
\end{pgfscope}%
\end{pgfscope}%
\begin{pgfscope}%
\pgfsetrectcap%
\pgfsetmiterjoin%
\pgfsetlinewidth{0.803000pt}%
\definecolor{currentstroke}{rgb}{1.000000,1.000000,1.000000}%
\pgfsetstrokecolor{currentstroke}%
\pgfsetdash{}{0pt}%
\pgfpathmoveto{\pgfqpoint{1.000000in}{0.500000in}}%
\pgfpathlineto{\pgfqpoint{1.000000in}{3.520000in}}%
\pgfusepath{stroke}%
\end{pgfscope}%
\begin{pgfscope}%
\pgfsetrectcap%
\pgfsetmiterjoin%
\pgfsetlinewidth{0.803000pt}%
\definecolor{currentstroke}{rgb}{1.000000,1.000000,1.000000}%
\pgfsetstrokecolor{currentstroke}%
\pgfsetdash{}{0pt}%
\pgfpathmoveto{\pgfqpoint{3.818182in}{0.500000in}}%
\pgfpathlineto{\pgfqpoint{3.818182in}{3.520000in}}%
\pgfusepath{stroke}%
\end{pgfscope}%
\begin{pgfscope}%
\pgfsetrectcap%
\pgfsetmiterjoin%
\pgfsetlinewidth{0.803000pt}%
\definecolor{currentstroke}{rgb}{1.000000,1.000000,1.000000}%
\pgfsetstrokecolor{currentstroke}%
\pgfsetdash{}{0pt}%
\pgfpathmoveto{\pgfqpoint{1.000000in}{0.500000in}}%
\pgfpathlineto{\pgfqpoint{3.818182in}{0.500000in}}%
\pgfusepath{stroke}%
\end{pgfscope}%
\begin{pgfscope}%
\pgfsetrectcap%
\pgfsetmiterjoin%
\pgfsetlinewidth{0.803000pt}%
\definecolor{currentstroke}{rgb}{1.000000,1.000000,1.000000}%
\pgfsetstrokecolor{currentstroke}%
\pgfsetdash{}{0pt}%
\pgfpathmoveto{\pgfqpoint{1.000000in}{3.520000in}}%
\pgfpathlineto{\pgfqpoint{3.818182in}{3.520000in}}%
\pgfusepath{stroke}%
\end{pgfscope}%
\begin{pgfscope}%
\definecolor{textcolor}{rgb}{0.150000,0.150000,0.150000}%
\pgfsetstrokecolor{textcolor}%
\pgfsetfillcolor{textcolor}%
\pgftext[x=2.409091in,y=3.603333in,,base]{\color{textcolor}\rmfamily\fontsize{16.800000}{20.160000}\selectfont Autocorrelation}%
\end{pgfscope}%
\begin{pgfscope}%
\pgfsetbuttcap%
\pgfsetmiterjoin%
\definecolor{currentfill}{rgb}{0.917647,0.917647,0.949020}%
\pgfsetfillcolor{currentfill}%
\pgfsetlinewidth{0.000000pt}%
\definecolor{currentstroke}{rgb}{0.000000,0.000000,0.000000}%
\pgfsetstrokecolor{currentstroke}%
\pgfsetstrokeopacity{0.000000}%
\pgfsetdash{}{0pt}%
\pgfpathmoveto{\pgfqpoint{4.381818in}{0.500000in}}%
\pgfpathlineto{\pgfqpoint{7.200000in}{0.500000in}}%
\pgfpathlineto{\pgfqpoint{7.200000in}{3.520000in}}%
\pgfpathlineto{\pgfqpoint{4.381818in}{3.520000in}}%
\pgfpathclose%
\pgfusepath{fill}%
\end{pgfscope}%
\begin{pgfscope}%
\pgfpathrectangle{\pgfqpoint{4.381818in}{0.500000in}}{\pgfqpoint{2.818182in}{3.020000in}}%
\pgfusepath{clip}%
\pgfsetroundcap%
\pgfsetroundjoin%
\pgfsetlinewidth{0.803000pt}%
\definecolor{currentstroke}{rgb}{1.000000,1.000000,1.000000}%
\pgfsetstrokecolor{currentstroke}%
\pgfsetdash{}{0pt}%
\pgfpathmoveto{\pgfqpoint{4.509917in}{0.500000in}}%
\pgfpathlineto{\pgfqpoint{4.509917in}{3.520000in}}%
\pgfusepath{stroke}%
\end{pgfscope}%
\begin{pgfscope}%
\definecolor{textcolor}{rgb}{0.150000,0.150000,0.150000}%
\pgfsetstrokecolor{textcolor}%
\pgfsetfillcolor{textcolor}%
\pgftext[x=4.509917in,y=0.402778in,,top]{\color{textcolor}\rmfamily\fontsize{14.000000}{16.800000}\selectfont 0}%
\end{pgfscope}%
\begin{pgfscope}%
\pgfpathrectangle{\pgfqpoint{4.381818in}{0.500000in}}{\pgfqpoint{2.818182in}{3.020000in}}%
\pgfusepath{clip}%
\pgfsetroundcap%
\pgfsetroundjoin%
\pgfsetlinewidth{0.803000pt}%
\definecolor{currentstroke}{rgb}{1.000000,1.000000,1.000000}%
\pgfsetstrokecolor{currentstroke}%
\pgfsetdash{}{0pt}%
\pgfpathmoveto{\pgfqpoint{5.759665in}{0.500000in}}%
\pgfpathlineto{\pgfqpoint{5.759665in}{3.520000in}}%
\pgfusepath{stroke}%
\end{pgfscope}%
\begin{pgfscope}%
\definecolor{textcolor}{rgb}{0.150000,0.150000,0.150000}%
\pgfsetstrokecolor{textcolor}%
\pgfsetfillcolor{textcolor}%
\pgftext[x=5.759665in,y=0.402778in,,top]{\color{textcolor}\rmfamily\fontsize{14.000000}{16.800000}\selectfont 10}%
\end{pgfscope}%
\begin{pgfscope}%
\pgfpathrectangle{\pgfqpoint{4.381818in}{0.500000in}}{\pgfqpoint{2.818182in}{3.020000in}}%
\pgfusepath{clip}%
\pgfsetroundcap%
\pgfsetroundjoin%
\pgfsetlinewidth{0.803000pt}%
\definecolor{currentstroke}{rgb}{1.000000,1.000000,1.000000}%
\pgfsetstrokecolor{currentstroke}%
\pgfsetdash{}{0pt}%
\pgfpathmoveto{\pgfqpoint{7.009413in}{0.500000in}}%
\pgfpathlineto{\pgfqpoint{7.009413in}{3.520000in}}%
\pgfusepath{stroke}%
\end{pgfscope}%
\begin{pgfscope}%
\definecolor{textcolor}{rgb}{0.150000,0.150000,0.150000}%
\pgfsetstrokecolor{textcolor}%
\pgfsetfillcolor{textcolor}%
\pgftext[x=7.009413in,y=0.402778in,,top]{\color{textcolor}\rmfamily\fontsize{14.000000}{16.800000}\selectfont 20}%
\end{pgfscope}%
\begin{pgfscope}%
\pgfpathrectangle{\pgfqpoint{4.381818in}{0.500000in}}{\pgfqpoint{2.818182in}{3.020000in}}%
\pgfusepath{clip}%
\pgfsetroundcap%
\pgfsetroundjoin%
\pgfsetlinewidth{0.803000pt}%
\definecolor{currentstroke}{rgb}{1.000000,1.000000,1.000000}%
\pgfsetstrokecolor{currentstroke}%
\pgfsetdash{}{0pt}%
\pgfpathmoveto{\pgfqpoint{4.381818in}{0.773110in}}%
\pgfpathlineto{\pgfqpoint{7.200000in}{0.773110in}}%
\pgfusepath{stroke}%
\end{pgfscope}%
\begin{pgfscope}%
\definecolor{textcolor}{rgb}{0.150000,0.150000,0.150000}%
\pgfsetstrokecolor{textcolor}%
\pgfsetfillcolor{textcolor}%
\pgftext[x=3.975365in,y=0.699244in,left,base]{\color{textcolor}\rmfamily\fontsize{14.000000}{16.800000}\selectfont 0.0}%
\end{pgfscope}%
\begin{pgfscope}%
\pgfpathrectangle{\pgfqpoint{4.381818in}{0.500000in}}{\pgfqpoint{2.818182in}{3.020000in}}%
\pgfusepath{clip}%
\pgfsetroundcap%
\pgfsetroundjoin%
\pgfsetlinewidth{0.803000pt}%
\definecolor{currentstroke}{rgb}{1.000000,1.000000,1.000000}%
\pgfsetstrokecolor{currentstroke}%
\pgfsetdash{}{0pt}%
\pgfpathmoveto{\pgfqpoint{4.381818in}{1.295034in}}%
\pgfpathlineto{\pgfqpoint{7.200000in}{1.295034in}}%
\pgfusepath{stroke}%
\end{pgfscope}%
\begin{pgfscope}%
\definecolor{textcolor}{rgb}{0.150000,0.150000,0.150000}%
\pgfsetstrokecolor{textcolor}%
\pgfsetfillcolor{textcolor}%
\pgftext[x=3.975365in,y=1.221168in,left,base]{\color{textcolor}\rmfamily\fontsize{14.000000}{16.800000}\selectfont 0.2}%
\end{pgfscope}%
\begin{pgfscope}%
\pgfpathrectangle{\pgfqpoint{4.381818in}{0.500000in}}{\pgfqpoint{2.818182in}{3.020000in}}%
\pgfusepath{clip}%
\pgfsetroundcap%
\pgfsetroundjoin%
\pgfsetlinewidth{0.803000pt}%
\definecolor{currentstroke}{rgb}{1.000000,1.000000,1.000000}%
\pgfsetstrokecolor{currentstroke}%
\pgfsetdash{}{0pt}%
\pgfpathmoveto{\pgfqpoint{4.381818in}{1.816957in}}%
\pgfpathlineto{\pgfqpoint{7.200000in}{1.816957in}}%
\pgfusepath{stroke}%
\end{pgfscope}%
\begin{pgfscope}%
\definecolor{textcolor}{rgb}{0.150000,0.150000,0.150000}%
\pgfsetstrokecolor{textcolor}%
\pgfsetfillcolor{textcolor}%
\pgftext[x=3.975365in,y=1.743091in,left,base]{\color{textcolor}\rmfamily\fontsize{14.000000}{16.800000}\selectfont 0.4}%
\end{pgfscope}%
\begin{pgfscope}%
\pgfpathrectangle{\pgfqpoint{4.381818in}{0.500000in}}{\pgfqpoint{2.818182in}{3.020000in}}%
\pgfusepath{clip}%
\pgfsetroundcap%
\pgfsetroundjoin%
\pgfsetlinewidth{0.803000pt}%
\definecolor{currentstroke}{rgb}{1.000000,1.000000,1.000000}%
\pgfsetstrokecolor{currentstroke}%
\pgfsetdash{}{0pt}%
\pgfpathmoveto{\pgfqpoint{4.381818in}{2.338880in}}%
\pgfpathlineto{\pgfqpoint{7.200000in}{2.338880in}}%
\pgfusepath{stroke}%
\end{pgfscope}%
\begin{pgfscope}%
\definecolor{textcolor}{rgb}{0.150000,0.150000,0.150000}%
\pgfsetstrokecolor{textcolor}%
\pgfsetfillcolor{textcolor}%
\pgftext[x=3.975365in,y=2.265014in,left,base]{\color{textcolor}\rmfamily\fontsize{14.000000}{16.800000}\selectfont 0.6}%
\end{pgfscope}%
\begin{pgfscope}%
\pgfpathrectangle{\pgfqpoint{4.381818in}{0.500000in}}{\pgfqpoint{2.818182in}{3.020000in}}%
\pgfusepath{clip}%
\pgfsetroundcap%
\pgfsetroundjoin%
\pgfsetlinewidth{0.803000pt}%
\definecolor{currentstroke}{rgb}{1.000000,1.000000,1.000000}%
\pgfsetstrokecolor{currentstroke}%
\pgfsetdash{}{0pt}%
\pgfpathmoveto{\pgfqpoint{4.381818in}{2.860804in}}%
\pgfpathlineto{\pgfqpoint{7.200000in}{2.860804in}}%
\pgfusepath{stroke}%
\end{pgfscope}%
\begin{pgfscope}%
\definecolor{textcolor}{rgb}{0.150000,0.150000,0.150000}%
\pgfsetstrokecolor{textcolor}%
\pgfsetfillcolor{textcolor}%
\pgftext[x=3.975365in,y=2.786938in,left,base]{\color{textcolor}\rmfamily\fontsize{14.000000}{16.800000}\selectfont 0.8}%
\end{pgfscope}%
\begin{pgfscope}%
\pgfpathrectangle{\pgfqpoint{4.381818in}{0.500000in}}{\pgfqpoint{2.818182in}{3.020000in}}%
\pgfusepath{clip}%
\pgfsetroundcap%
\pgfsetroundjoin%
\pgfsetlinewidth{0.803000pt}%
\definecolor{currentstroke}{rgb}{1.000000,1.000000,1.000000}%
\pgfsetstrokecolor{currentstroke}%
\pgfsetdash{}{0pt}%
\pgfpathmoveto{\pgfqpoint{4.381818in}{3.382727in}}%
\pgfpathlineto{\pgfqpoint{7.200000in}{3.382727in}}%
\pgfusepath{stroke}%
\end{pgfscope}%
\begin{pgfscope}%
\definecolor{textcolor}{rgb}{0.150000,0.150000,0.150000}%
\pgfsetstrokecolor{textcolor}%
\pgfsetfillcolor{textcolor}%
\pgftext[x=3.975365in,y=3.308861in,left,base]{\color{textcolor}\rmfamily\fontsize{14.000000}{16.800000}\selectfont 1.0}%
\end{pgfscope}%
\begin{pgfscope}%
\pgfpathrectangle{\pgfqpoint{4.381818in}{0.500000in}}{\pgfqpoint{2.818182in}{3.020000in}}%
\pgfusepath{clip}%
\pgfsetbuttcap%
\pgfsetroundjoin%
\definecolor{currentfill}{rgb}{0.121569,0.466667,0.705882}%
\pgfsetfillcolor{currentfill}%
\pgfsetfillopacity{0.250000}%
\pgfsetlinewidth{1.003750pt}%
\definecolor{currentstroke}{rgb}{1.000000,1.000000,1.000000}%
\pgfsetstrokecolor{currentstroke}%
\pgfsetstrokeopacity{0.250000}%
\pgfsetdash{}{0pt}%
\pgfpathmoveto{\pgfqpoint{4.572405in}{0.904822in}}%
\pgfpathlineto{\pgfqpoint{4.572405in}{0.641399in}}%
\pgfpathlineto{\pgfqpoint{4.759867in}{0.641399in}}%
\pgfpathlineto{\pgfqpoint{4.884842in}{0.641399in}}%
\pgfpathlineto{\pgfqpoint{5.009817in}{0.641399in}}%
\pgfpathlineto{\pgfqpoint{5.134791in}{0.641399in}}%
\pgfpathlineto{\pgfqpoint{5.259766in}{0.641399in}}%
\pgfpathlineto{\pgfqpoint{5.384741in}{0.641399in}}%
\pgfpathlineto{\pgfqpoint{5.509716in}{0.641399in}}%
\pgfpathlineto{\pgfqpoint{5.634691in}{0.641399in}}%
\pgfpathlineto{\pgfqpoint{5.759665in}{0.641399in}}%
\pgfpathlineto{\pgfqpoint{5.884640in}{0.641399in}}%
\pgfpathlineto{\pgfqpoint{6.009615in}{0.641399in}}%
\pgfpathlineto{\pgfqpoint{6.134590in}{0.641399in}}%
\pgfpathlineto{\pgfqpoint{6.259565in}{0.641399in}}%
\pgfpathlineto{\pgfqpoint{6.384539in}{0.641399in}}%
\pgfpathlineto{\pgfqpoint{6.509514in}{0.641399in}}%
\pgfpathlineto{\pgfqpoint{6.634489in}{0.641399in}}%
\pgfpathlineto{\pgfqpoint{6.759464in}{0.641399in}}%
\pgfpathlineto{\pgfqpoint{6.884439in}{0.641399in}}%
\pgfpathlineto{\pgfqpoint{7.071901in}{0.641399in}}%
\pgfpathlineto{\pgfqpoint{7.071901in}{0.904822in}}%
\pgfpathlineto{\pgfqpoint{7.071901in}{0.904822in}}%
\pgfpathlineto{\pgfqpoint{6.884439in}{0.904822in}}%
\pgfpathlineto{\pgfqpoint{6.759464in}{0.904822in}}%
\pgfpathlineto{\pgfqpoint{6.634489in}{0.904822in}}%
\pgfpathlineto{\pgfqpoint{6.509514in}{0.904822in}}%
\pgfpathlineto{\pgfqpoint{6.384539in}{0.904822in}}%
\pgfpathlineto{\pgfqpoint{6.259565in}{0.904822in}}%
\pgfpathlineto{\pgfqpoint{6.134590in}{0.904822in}}%
\pgfpathlineto{\pgfqpoint{6.009615in}{0.904822in}}%
\pgfpathlineto{\pgfqpoint{5.884640in}{0.904822in}}%
\pgfpathlineto{\pgfqpoint{5.759665in}{0.904822in}}%
\pgfpathlineto{\pgfqpoint{5.634691in}{0.904822in}}%
\pgfpathlineto{\pgfqpoint{5.509716in}{0.904822in}}%
\pgfpathlineto{\pgfqpoint{5.384741in}{0.904822in}}%
\pgfpathlineto{\pgfqpoint{5.259766in}{0.904822in}}%
\pgfpathlineto{\pgfqpoint{5.134791in}{0.904822in}}%
\pgfpathlineto{\pgfqpoint{5.009817in}{0.904822in}}%
\pgfpathlineto{\pgfqpoint{4.884842in}{0.904822in}}%
\pgfpathlineto{\pgfqpoint{4.759867in}{0.904822in}}%
\pgfpathlineto{\pgfqpoint{4.572405in}{0.904822in}}%
\pgfpathclose%
\pgfusepath{stroke,fill}%
\end{pgfscope}%
\begin{pgfscope}%
\pgfpathrectangle{\pgfqpoint{4.381818in}{0.500000in}}{\pgfqpoint{2.818182in}{3.020000in}}%
\pgfusepath{clip}%
\pgfsetbuttcap%
\pgfsetroundjoin%
\pgfsetlinewidth{1.505625pt}%
\definecolor{currentstroke}{rgb}{0.000000,0.000000,0.000000}%
\pgfsetstrokecolor{currentstroke}%
\pgfsetdash{}{0pt}%
\pgfpathmoveto{\pgfqpoint{4.509917in}{0.773110in}}%
\pgfpathlineto{\pgfqpoint{4.509917in}{3.382727in}}%
\pgfusepath{stroke}%
\end{pgfscope}%
\begin{pgfscope}%
\pgfpathrectangle{\pgfqpoint{4.381818in}{0.500000in}}{\pgfqpoint{2.818182in}{3.020000in}}%
\pgfusepath{clip}%
\pgfsetbuttcap%
\pgfsetroundjoin%
\pgfsetlinewidth{1.505625pt}%
\definecolor{currentstroke}{rgb}{0.000000,0.000000,0.000000}%
\pgfsetstrokecolor{currentstroke}%
\pgfsetdash{}{0pt}%
\pgfpathmoveto{\pgfqpoint{4.634892in}{0.773110in}}%
\pgfpathlineto{\pgfqpoint{4.634892in}{3.375089in}}%
\pgfusepath{stroke}%
\end{pgfscope}%
\begin{pgfscope}%
\pgfpathrectangle{\pgfqpoint{4.381818in}{0.500000in}}{\pgfqpoint{2.818182in}{3.020000in}}%
\pgfusepath{clip}%
\pgfsetbuttcap%
\pgfsetroundjoin%
\pgfsetlinewidth{1.505625pt}%
\definecolor{currentstroke}{rgb}{0.000000,0.000000,0.000000}%
\pgfsetstrokecolor{currentstroke}%
\pgfsetdash{}{0pt}%
\pgfpathmoveto{\pgfqpoint{4.759867in}{0.773110in}}%
\pgfpathlineto{\pgfqpoint{4.759867in}{0.774419in}}%
\pgfusepath{stroke}%
\end{pgfscope}%
\begin{pgfscope}%
\pgfpathrectangle{\pgfqpoint{4.381818in}{0.500000in}}{\pgfqpoint{2.818182in}{3.020000in}}%
\pgfusepath{clip}%
\pgfsetbuttcap%
\pgfsetroundjoin%
\pgfsetlinewidth{1.505625pt}%
\definecolor{currentstroke}{rgb}{0.000000,0.000000,0.000000}%
\pgfsetstrokecolor{currentstroke}%
\pgfsetdash{}{0pt}%
\pgfpathmoveto{\pgfqpoint{4.884842in}{0.773110in}}%
\pgfpathlineto{\pgfqpoint{4.884842in}{0.796085in}}%
\pgfusepath{stroke}%
\end{pgfscope}%
\begin{pgfscope}%
\pgfpathrectangle{\pgfqpoint{4.381818in}{0.500000in}}{\pgfqpoint{2.818182in}{3.020000in}}%
\pgfusepath{clip}%
\pgfsetbuttcap%
\pgfsetroundjoin%
\pgfsetlinewidth{1.505625pt}%
\definecolor{currentstroke}{rgb}{0.000000,0.000000,0.000000}%
\pgfsetstrokecolor{currentstroke}%
\pgfsetdash{}{0pt}%
\pgfpathmoveto{\pgfqpoint{5.009817in}{0.773110in}}%
\pgfpathlineto{\pgfqpoint{5.009817in}{0.790749in}}%
\pgfusepath{stroke}%
\end{pgfscope}%
\begin{pgfscope}%
\pgfpathrectangle{\pgfqpoint{4.381818in}{0.500000in}}{\pgfqpoint{2.818182in}{3.020000in}}%
\pgfusepath{clip}%
\pgfsetbuttcap%
\pgfsetroundjoin%
\pgfsetlinewidth{1.505625pt}%
\definecolor{currentstroke}{rgb}{0.000000,0.000000,0.000000}%
\pgfsetstrokecolor{currentstroke}%
\pgfsetdash{}{0pt}%
\pgfpathmoveto{\pgfqpoint{5.134791in}{0.773110in}}%
\pgfpathlineto{\pgfqpoint{5.134791in}{0.840391in}}%
\pgfusepath{stroke}%
\end{pgfscope}%
\begin{pgfscope}%
\pgfpathrectangle{\pgfqpoint{4.381818in}{0.500000in}}{\pgfqpoint{2.818182in}{3.020000in}}%
\pgfusepath{clip}%
\pgfsetbuttcap%
\pgfsetroundjoin%
\pgfsetlinewidth{1.505625pt}%
\definecolor{currentstroke}{rgb}{0.000000,0.000000,0.000000}%
\pgfsetstrokecolor{currentstroke}%
\pgfsetdash{}{0pt}%
\pgfpathmoveto{\pgfqpoint{5.259766in}{0.773110in}}%
\pgfpathlineto{\pgfqpoint{5.259766in}{0.689396in}}%
\pgfusepath{stroke}%
\end{pgfscope}%
\begin{pgfscope}%
\pgfpathrectangle{\pgfqpoint{4.381818in}{0.500000in}}{\pgfqpoint{2.818182in}{3.020000in}}%
\pgfusepath{clip}%
\pgfsetbuttcap%
\pgfsetroundjoin%
\pgfsetlinewidth{1.505625pt}%
\definecolor{currentstroke}{rgb}{0.000000,0.000000,0.000000}%
\pgfsetstrokecolor{currentstroke}%
\pgfsetdash{}{0pt}%
\pgfpathmoveto{\pgfqpoint{5.384741in}{0.773110in}}%
\pgfpathlineto{\pgfqpoint{5.384741in}{0.754993in}}%
\pgfusepath{stroke}%
\end{pgfscope}%
\begin{pgfscope}%
\pgfpathrectangle{\pgfqpoint{4.381818in}{0.500000in}}{\pgfqpoint{2.818182in}{3.020000in}}%
\pgfusepath{clip}%
\pgfsetbuttcap%
\pgfsetroundjoin%
\pgfsetlinewidth{1.505625pt}%
\definecolor{currentstroke}{rgb}{0.000000,0.000000,0.000000}%
\pgfsetstrokecolor{currentstroke}%
\pgfsetdash{}{0pt}%
\pgfpathmoveto{\pgfqpoint{5.509716in}{0.773110in}}%
\pgfpathlineto{\pgfqpoint{5.509716in}{0.743890in}}%
\pgfusepath{stroke}%
\end{pgfscope}%
\begin{pgfscope}%
\pgfpathrectangle{\pgfqpoint{4.381818in}{0.500000in}}{\pgfqpoint{2.818182in}{3.020000in}}%
\pgfusepath{clip}%
\pgfsetbuttcap%
\pgfsetroundjoin%
\pgfsetlinewidth{1.505625pt}%
\definecolor{currentstroke}{rgb}{0.000000,0.000000,0.000000}%
\pgfsetstrokecolor{currentstroke}%
\pgfsetdash{}{0pt}%
\pgfpathmoveto{\pgfqpoint{5.634691in}{0.773110in}}%
\pgfpathlineto{\pgfqpoint{5.634691in}{0.775509in}}%
\pgfusepath{stroke}%
\end{pgfscope}%
\begin{pgfscope}%
\pgfpathrectangle{\pgfqpoint{4.381818in}{0.500000in}}{\pgfqpoint{2.818182in}{3.020000in}}%
\pgfusepath{clip}%
\pgfsetbuttcap%
\pgfsetroundjoin%
\pgfsetlinewidth{1.505625pt}%
\definecolor{currentstroke}{rgb}{0.000000,0.000000,0.000000}%
\pgfsetstrokecolor{currentstroke}%
\pgfsetdash{}{0pt}%
\pgfpathmoveto{\pgfqpoint{5.759665in}{0.773110in}}%
\pgfpathlineto{\pgfqpoint{5.759665in}{0.813737in}}%
\pgfusepath{stroke}%
\end{pgfscope}%
\begin{pgfscope}%
\pgfpathrectangle{\pgfqpoint{4.381818in}{0.500000in}}{\pgfqpoint{2.818182in}{3.020000in}}%
\pgfusepath{clip}%
\pgfsetbuttcap%
\pgfsetroundjoin%
\pgfsetlinewidth{1.505625pt}%
\definecolor{currentstroke}{rgb}{0.000000,0.000000,0.000000}%
\pgfsetstrokecolor{currentstroke}%
\pgfsetdash{}{0pt}%
\pgfpathmoveto{\pgfqpoint{5.884640in}{0.773110in}}%
\pgfpathlineto{\pgfqpoint{5.884640in}{0.838480in}}%
\pgfusepath{stroke}%
\end{pgfscope}%
\begin{pgfscope}%
\pgfpathrectangle{\pgfqpoint{4.381818in}{0.500000in}}{\pgfqpoint{2.818182in}{3.020000in}}%
\pgfusepath{clip}%
\pgfsetbuttcap%
\pgfsetroundjoin%
\pgfsetlinewidth{1.505625pt}%
\definecolor{currentstroke}{rgb}{0.000000,0.000000,0.000000}%
\pgfsetstrokecolor{currentstroke}%
\pgfsetdash{}{0pt}%
\pgfpathmoveto{\pgfqpoint{6.009615in}{0.773110in}}%
\pgfpathlineto{\pgfqpoint{6.009615in}{0.778278in}}%
\pgfusepath{stroke}%
\end{pgfscope}%
\begin{pgfscope}%
\pgfpathrectangle{\pgfqpoint{4.381818in}{0.500000in}}{\pgfqpoint{2.818182in}{3.020000in}}%
\pgfusepath{clip}%
\pgfsetbuttcap%
\pgfsetroundjoin%
\pgfsetlinewidth{1.505625pt}%
\definecolor{currentstroke}{rgb}{0.000000,0.000000,0.000000}%
\pgfsetstrokecolor{currentstroke}%
\pgfsetdash{}{0pt}%
\pgfpathmoveto{\pgfqpoint{6.134590in}{0.773110in}}%
\pgfpathlineto{\pgfqpoint{6.134590in}{0.740221in}}%
\pgfusepath{stroke}%
\end{pgfscope}%
\begin{pgfscope}%
\pgfpathrectangle{\pgfqpoint{4.381818in}{0.500000in}}{\pgfqpoint{2.818182in}{3.020000in}}%
\pgfusepath{clip}%
\pgfsetbuttcap%
\pgfsetroundjoin%
\pgfsetlinewidth{1.505625pt}%
\definecolor{currentstroke}{rgb}{0.000000,0.000000,0.000000}%
\pgfsetstrokecolor{currentstroke}%
\pgfsetdash{}{0pt}%
\pgfpathmoveto{\pgfqpoint{6.259565in}{0.773110in}}%
\pgfpathlineto{\pgfqpoint{6.259565in}{0.737875in}}%
\pgfusepath{stroke}%
\end{pgfscope}%
\begin{pgfscope}%
\pgfpathrectangle{\pgfqpoint{4.381818in}{0.500000in}}{\pgfqpoint{2.818182in}{3.020000in}}%
\pgfusepath{clip}%
\pgfsetbuttcap%
\pgfsetroundjoin%
\pgfsetlinewidth{1.505625pt}%
\definecolor{currentstroke}{rgb}{0.000000,0.000000,0.000000}%
\pgfsetstrokecolor{currentstroke}%
\pgfsetdash{}{0pt}%
\pgfpathmoveto{\pgfqpoint{6.384539in}{0.773110in}}%
\pgfpathlineto{\pgfqpoint{6.384539in}{0.789043in}}%
\pgfusepath{stroke}%
\end{pgfscope}%
\begin{pgfscope}%
\pgfpathrectangle{\pgfqpoint{4.381818in}{0.500000in}}{\pgfqpoint{2.818182in}{3.020000in}}%
\pgfusepath{clip}%
\pgfsetbuttcap%
\pgfsetroundjoin%
\pgfsetlinewidth{1.505625pt}%
\definecolor{currentstroke}{rgb}{0.000000,0.000000,0.000000}%
\pgfsetstrokecolor{currentstroke}%
\pgfsetdash{}{0pt}%
\pgfpathmoveto{\pgfqpoint{6.509514in}{0.773110in}}%
\pgfpathlineto{\pgfqpoint{6.509514in}{0.801117in}}%
\pgfusepath{stroke}%
\end{pgfscope}%
\begin{pgfscope}%
\pgfpathrectangle{\pgfqpoint{4.381818in}{0.500000in}}{\pgfqpoint{2.818182in}{3.020000in}}%
\pgfusepath{clip}%
\pgfsetbuttcap%
\pgfsetroundjoin%
\pgfsetlinewidth{1.505625pt}%
\definecolor{currentstroke}{rgb}{0.000000,0.000000,0.000000}%
\pgfsetstrokecolor{currentstroke}%
\pgfsetdash{}{0pt}%
\pgfpathmoveto{\pgfqpoint{6.634489in}{0.773110in}}%
\pgfpathlineto{\pgfqpoint{6.634489in}{0.637273in}}%
\pgfusepath{stroke}%
\end{pgfscope}%
\begin{pgfscope}%
\pgfpathrectangle{\pgfqpoint{4.381818in}{0.500000in}}{\pgfqpoint{2.818182in}{3.020000in}}%
\pgfusepath{clip}%
\pgfsetbuttcap%
\pgfsetroundjoin%
\pgfsetlinewidth{1.505625pt}%
\definecolor{currentstroke}{rgb}{0.000000,0.000000,0.000000}%
\pgfsetstrokecolor{currentstroke}%
\pgfsetdash{}{0pt}%
\pgfpathmoveto{\pgfqpoint{6.759464in}{0.773110in}}%
\pgfpathlineto{\pgfqpoint{6.759464in}{0.667214in}}%
\pgfusepath{stroke}%
\end{pgfscope}%
\begin{pgfscope}%
\pgfpathrectangle{\pgfqpoint{4.381818in}{0.500000in}}{\pgfqpoint{2.818182in}{3.020000in}}%
\pgfusepath{clip}%
\pgfsetbuttcap%
\pgfsetroundjoin%
\pgfsetlinewidth{1.505625pt}%
\definecolor{currentstroke}{rgb}{0.000000,0.000000,0.000000}%
\pgfsetstrokecolor{currentstroke}%
\pgfsetdash{}{0pt}%
\pgfpathmoveto{\pgfqpoint{6.884439in}{0.773110in}}%
\pgfpathlineto{\pgfqpoint{6.884439in}{0.750543in}}%
\pgfusepath{stroke}%
\end{pgfscope}%
\begin{pgfscope}%
\pgfpathrectangle{\pgfqpoint{4.381818in}{0.500000in}}{\pgfqpoint{2.818182in}{3.020000in}}%
\pgfusepath{clip}%
\pgfsetbuttcap%
\pgfsetroundjoin%
\pgfsetlinewidth{1.505625pt}%
\definecolor{currentstroke}{rgb}{0.000000,0.000000,0.000000}%
\pgfsetstrokecolor{currentstroke}%
\pgfsetdash{}{0pt}%
\pgfpathmoveto{\pgfqpoint{7.009413in}{0.773110in}}%
\pgfpathlineto{\pgfqpoint{7.009413in}{0.801174in}}%
\pgfusepath{stroke}%
\end{pgfscope}%
\begin{pgfscope}%
\pgfpathrectangle{\pgfqpoint{4.381818in}{0.500000in}}{\pgfqpoint{2.818182in}{3.020000in}}%
\pgfusepath{clip}%
\pgfsetroundcap%
\pgfsetroundjoin%
\pgfsetlinewidth{1.505625pt}%
\definecolor{currentstroke}{rgb}{0.549020,0.337255,0.294118}%
\pgfsetstrokecolor{currentstroke}%
\pgfsetdash{}{0pt}%
\pgfpathmoveto{\pgfqpoint{4.381818in}{0.773110in}}%
\pgfpathlineto{\pgfqpoint{7.200000in}{0.773110in}}%
\pgfusepath{stroke}%
\end{pgfscope}%
\begin{pgfscope}%
\pgfpathrectangle{\pgfqpoint{4.381818in}{0.500000in}}{\pgfqpoint{2.818182in}{3.020000in}}%
\pgfusepath{clip}%
\pgfsetbuttcap%
\pgfsetroundjoin%
\definecolor{currentfill}{rgb}{0.549020,0.337255,0.294118}%
\pgfsetfillcolor{currentfill}%
\pgfsetlinewidth{1.003750pt}%
\definecolor{currentstroke}{rgb}{0.549020,0.337255,0.294118}%
\pgfsetstrokecolor{currentstroke}%
\pgfsetdash{}{0pt}%
\pgfsys@defobject{currentmarker}{\pgfqpoint{-0.034722in}{-0.034722in}}{\pgfqpoint{0.034722in}{0.034722in}}{%
\pgfpathmoveto{\pgfqpoint{0.000000in}{-0.034722in}}%
\pgfpathcurveto{\pgfqpoint{0.009208in}{-0.034722in}}{\pgfqpoint{0.018041in}{-0.031064in}}{\pgfqpoint{0.024552in}{-0.024552in}}%
\pgfpathcurveto{\pgfqpoint{0.031064in}{-0.018041in}}{\pgfqpoint{0.034722in}{-0.009208in}}{\pgfqpoint{0.034722in}{0.000000in}}%
\pgfpathcurveto{\pgfqpoint{0.034722in}{0.009208in}}{\pgfqpoint{0.031064in}{0.018041in}}{\pgfqpoint{0.024552in}{0.024552in}}%
\pgfpathcurveto{\pgfqpoint{0.018041in}{0.031064in}}{\pgfqpoint{0.009208in}{0.034722in}}{\pgfqpoint{0.000000in}{0.034722in}}%
\pgfpathcurveto{\pgfqpoint{-0.009208in}{0.034722in}}{\pgfqpoint{-0.018041in}{0.031064in}}{\pgfqpoint{-0.024552in}{0.024552in}}%
\pgfpathcurveto{\pgfqpoint{-0.031064in}{0.018041in}}{\pgfqpoint{-0.034722in}{0.009208in}}{\pgfqpoint{-0.034722in}{0.000000in}}%
\pgfpathcurveto{\pgfqpoint{-0.034722in}{-0.009208in}}{\pgfqpoint{-0.031064in}{-0.018041in}}{\pgfqpoint{-0.024552in}{-0.024552in}}%
\pgfpathcurveto{\pgfqpoint{-0.018041in}{-0.031064in}}{\pgfqpoint{-0.009208in}{-0.034722in}}{\pgfqpoint{0.000000in}{-0.034722in}}%
\pgfpathclose%
\pgfusepath{stroke,fill}%
}%
\begin{pgfscope}%
\pgfsys@transformshift{4.509917in}{3.382727in}%
\pgfsys@useobject{currentmarker}{}%
\end{pgfscope}%
\begin{pgfscope}%
\pgfsys@transformshift{4.634892in}{3.375089in}%
\pgfsys@useobject{currentmarker}{}%
\end{pgfscope}%
\begin{pgfscope}%
\pgfsys@transformshift{4.759867in}{0.774419in}%
\pgfsys@useobject{currentmarker}{}%
\end{pgfscope}%
\begin{pgfscope}%
\pgfsys@transformshift{4.884842in}{0.796085in}%
\pgfsys@useobject{currentmarker}{}%
\end{pgfscope}%
\begin{pgfscope}%
\pgfsys@transformshift{5.009817in}{0.790749in}%
\pgfsys@useobject{currentmarker}{}%
\end{pgfscope}%
\begin{pgfscope}%
\pgfsys@transformshift{5.134791in}{0.840391in}%
\pgfsys@useobject{currentmarker}{}%
\end{pgfscope}%
\begin{pgfscope}%
\pgfsys@transformshift{5.259766in}{0.689396in}%
\pgfsys@useobject{currentmarker}{}%
\end{pgfscope}%
\begin{pgfscope}%
\pgfsys@transformshift{5.384741in}{0.754993in}%
\pgfsys@useobject{currentmarker}{}%
\end{pgfscope}%
\begin{pgfscope}%
\pgfsys@transformshift{5.509716in}{0.743890in}%
\pgfsys@useobject{currentmarker}{}%
\end{pgfscope}%
\begin{pgfscope}%
\pgfsys@transformshift{5.634691in}{0.775509in}%
\pgfsys@useobject{currentmarker}{}%
\end{pgfscope}%
\begin{pgfscope}%
\pgfsys@transformshift{5.759665in}{0.813737in}%
\pgfsys@useobject{currentmarker}{}%
\end{pgfscope}%
\begin{pgfscope}%
\pgfsys@transformshift{5.884640in}{0.838480in}%
\pgfsys@useobject{currentmarker}{}%
\end{pgfscope}%
\begin{pgfscope}%
\pgfsys@transformshift{6.009615in}{0.778278in}%
\pgfsys@useobject{currentmarker}{}%
\end{pgfscope}%
\begin{pgfscope}%
\pgfsys@transformshift{6.134590in}{0.740221in}%
\pgfsys@useobject{currentmarker}{}%
\end{pgfscope}%
\begin{pgfscope}%
\pgfsys@transformshift{6.259565in}{0.737875in}%
\pgfsys@useobject{currentmarker}{}%
\end{pgfscope}%
\begin{pgfscope}%
\pgfsys@transformshift{6.384539in}{0.789043in}%
\pgfsys@useobject{currentmarker}{}%
\end{pgfscope}%
\begin{pgfscope}%
\pgfsys@transformshift{6.509514in}{0.801117in}%
\pgfsys@useobject{currentmarker}{}%
\end{pgfscope}%
\begin{pgfscope}%
\pgfsys@transformshift{6.634489in}{0.637273in}%
\pgfsys@useobject{currentmarker}{}%
\end{pgfscope}%
\begin{pgfscope}%
\pgfsys@transformshift{6.759464in}{0.667214in}%
\pgfsys@useobject{currentmarker}{}%
\end{pgfscope}%
\begin{pgfscope}%
\pgfsys@transformshift{6.884439in}{0.750543in}%
\pgfsys@useobject{currentmarker}{}%
\end{pgfscope}%
\begin{pgfscope}%
\pgfsys@transformshift{7.009413in}{0.801174in}%
\pgfsys@useobject{currentmarker}{}%
\end{pgfscope}%
\end{pgfscope}%
\begin{pgfscope}%
\pgfsetrectcap%
\pgfsetmiterjoin%
\pgfsetlinewidth{0.803000pt}%
\definecolor{currentstroke}{rgb}{1.000000,1.000000,1.000000}%
\pgfsetstrokecolor{currentstroke}%
\pgfsetdash{}{0pt}%
\pgfpathmoveto{\pgfqpoint{4.381818in}{0.500000in}}%
\pgfpathlineto{\pgfqpoint{4.381818in}{3.520000in}}%
\pgfusepath{stroke}%
\end{pgfscope}%
\begin{pgfscope}%
\pgfsetrectcap%
\pgfsetmiterjoin%
\pgfsetlinewidth{0.803000pt}%
\definecolor{currentstroke}{rgb}{1.000000,1.000000,1.000000}%
\pgfsetstrokecolor{currentstroke}%
\pgfsetdash{}{0pt}%
\pgfpathmoveto{\pgfqpoint{7.200000in}{0.500000in}}%
\pgfpathlineto{\pgfqpoint{7.200000in}{3.520000in}}%
\pgfusepath{stroke}%
\end{pgfscope}%
\begin{pgfscope}%
\pgfsetrectcap%
\pgfsetmiterjoin%
\pgfsetlinewidth{0.803000pt}%
\definecolor{currentstroke}{rgb}{1.000000,1.000000,1.000000}%
\pgfsetstrokecolor{currentstroke}%
\pgfsetdash{}{0pt}%
\pgfpathmoveto{\pgfqpoint{4.381818in}{0.500000in}}%
\pgfpathlineto{\pgfqpoint{7.200000in}{0.500000in}}%
\pgfusepath{stroke}%
\end{pgfscope}%
\begin{pgfscope}%
\pgfsetrectcap%
\pgfsetmiterjoin%
\pgfsetlinewidth{0.803000pt}%
\definecolor{currentstroke}{rgb}{1.000000,1.000000,1.000000}%
\pgfsetstrokecolor{currentstroke}%
\pgfsetdash{}{0pt}%
\pgfpathmoveto{\pgfqpoint{4.381818in}{3.520000in}}%
\pgfpathlineto{\pgfqpoint{7.200000in}{3.520000in}}%
\pgfusepath{stroke}%
\end{pgfscope}%
\begin{pgfscope}%
\definecolor{textcolor}{rgb}{0.150000,0.150000,0.150000}%
\pgfsetstrokecolor{textcolor}%
\pgfsetfillcolor{textcolor}%
\pgftext[x=5.790909in,y=3.603333in,,base]{\color{textcolor}\rmfamily\fontsize{16.800000}{20.160000}\selectfont Partial Autocorrelation}%
\end{pgfscope}%
\end{pgfpicture}%
\makeatother%
\endgroup%

    \end{adjustbox}  
    \caption{Autocorrelation function (ACF) and partial autocorrelation function (PACF) for log-returns of PG. (For convenience, only one stock is shown. ACF and PACF for other stocks can be seen in the appendix in figure \ref{fig:acf_pacf_log_adjclose})}
    \label{fig:acf_pacf_log_adjclose_PG}
\end{figure}{}

\begin{table}[h!]
    \centering
    \figuretitle{Means of the log-returns for all Stocks}
    \begin{adjustbox}{width = 0.95\linewidth}
    \begin{tabular}{lrrrrrrrrrr}
\toprule
{} & MMM &  AXP & GE & INTC & JNJ & PG & UTX & VZ & V & DIS \\
\midrule
mean & 0.000786 & 0.000534  & 0.000110  & 0.000548  & 0.000616  & 0.000337  & 0.000450 & 0.000372 & 0.001068 & 0.000739 \\
\bottomrule
\end{tabular}

    \end{adjustbox}
    \caption{}
    \label{tab:log return means}
\end{table}{}

\begin{figure}[H]
    \centering
    \figuretitle{ACF and PACF for log-returns of stock PG}
    \begin{adjustbox}{width=.95\textwidth,center}
    %% Creator: Matplotlib, PGF backend
%%
%% To include the figure in your LaTeX document, write
%%   \input{<filename>.pgf}
%%
%% Make sure the required packages are loaded in your preamble
%%   \usepackage{pgf}
%%
%% Figures using additional raster images can only be included by \input if
%% they are in the same directory as the main LaTeX file. For loading figures
%% from other directories you can use the `import` package
%%   \usepackage{import}
%% and then include the figures with
%%   \import{<path to file>}{<filename>.pgf}
%%
%% Matplotlib used the following preamble
%%   \usepackage{fontspec}
%%   \setmainfont{DejaVuSerif.ttf}[Path=/opt/tljh/user/lib/python3.6/site-packages/matplotlib/mpl-data/fonts/ttf/]
%%   \setsansfont{DejaVuSans.ttf}[Path=/opt/tljh/user/lib/python3.6/site-packages/matplotlib/mpl-data/fonts/ttf/]
%%   \setmonofont{DejaVuSansMono.ttf}[Path=/opt/tljh/user/lib/python3.6/site-packages/matplotlib/mpl-data/fonts/ttf/]
%%
\begingroup%
\makeatletter%
\begin{pgfpicture}%
\pgfpathrectangle{\pgfpointorigin}{\pgfqpoint{8.000000in}{4.000000in}}%
\pgfusepath{use as bounding box, clip}%
\begin{pgfscope}%
\pgfsetbuttcap%
\pgfsetmiterjoin%
\definecolor{currentfill}{rgb}{1.000000,1.000000,1.000000}%
\pgfsetfillcolor{currentfill}%
\pgfsetlinewidth{0.000000pt}%
\definecolor{currentstroke}{rgb}{1.000000,1.000000,1.000000}%
\pgfsetstrokecolor{currentstroke}%
\pgfsetdash{}{0pt}%
\pgfpathmoveto{\pgfqpoint{0.000000in}{0.000000in}}%
\pgfpathlineto{\pgfqpoint{8.000000in}{0.000000in}}%
\pgfpathlineto{\pgfqpoint{8.000000in}{4.000000in}}%
\pgfpathlineto{\pgfqpoint{0.000000in}{4.000000in}}%
\pgfpathclose%
\pgfusepath{fill}%
\end{pgfscope}%
\begin{pgfscope}%
\pgfsetbuttcap%
\pgfsetmiterjoin%
\definecolor{currentfill}{rgb}{0.917647,0.917647,0.949020}%
\pgfsetfillcolor{currentfill}%
\pgfsetlinewidth{0.000000pt}%
\definecolor{currentstroke}{rgb}{0.000000,0.000000,0.000000}%
\pgfsetstrokecolor{currentstroke}%
\pgfsetstrokeopacity{0.000000}%
\pgfsetdash{}{0pt}%
\pgfpathmoveto{\pgfqpoint{1.000000in}{0.500000in}}%
\pgfpathlineto{\pgfqpoint{3.818182in}{0.500000in}}%
\pgfpathlineto{\pgfqpoint{3.818182in}{3.520000in}}%
\pgfpathlineto{\pgfqpoint{1.000000in}{3.520000in}}%
\pgfpathclose%
\pgfusepath{fill}%
\end{pgfscope}%
\begin{pgfscope}%
\pgfpathrectangle{\pgfqpoint{1.000000in}{0.500000in}}{\pgfqpoint{2.818182in}{3.020000in}}%
\pgfusepath{clip}%
\pgfsetroundcap%
\pgfsetroundjoin%
\pgfsetlinewidth{0.803000pt}%
\definecolor{currentstroke}{rgb}{1.000000,1.000000,1.000000}%
\pgfsetstrokecolor{currentstroke}%
\pgfsetdash{}{0pt}%
\pgfpathmoveto{\pgfqpoint{1.128099in}{0.500000in}}%
\pgfpathlineto{\pgfqpoint{1.128099in}{3.520000in}}%
\pgfusepath{stroke}%
\end{pgfscope}%
\begin{pgfscope}%
\definecolor{textcolor}{rgb}{0.150000,0.150000,0.150000}%
\pgfsetstrokecolor{textcolor}%
\pgfsetfillcolor{textcolor}%
\pgftext[x=1.128099in,y=0.402778in,,top]{\color{textcolor}\rmfamily\fontsize{14.000000}{16.800000}\selectfont 0}%
\end{pgfscope}%
\begin{pgfscope}%
\pgfpathrectangle{\pgfqpoint{1.000000in}{0.500000in}}{\pgfqpoint{2.818182in}{3.020000in}}%
\pgfusepath{clip}%
\pgfsetroundcap%
\pgfsetroundjoin%
\pgfsetlinewidth{0.803000pt}%
\definecolor{currentstroke}{rgb}{1.000000,1.000000,1.000000}%
\pgfsetstrokecolor{currentstroke}%
\pgfsetdash{}{0pt}%
\pgfpathmoveto{\pgfqpoint{2.377847in}{0.500000in}}%
\pgfpathlineto{\pgfqpoint{2.377847in}{3.520000in}}%
\pgfusepath{stroke}%
\end{pgfscope}%
\begin{pgfscope}%
\definecolor{textcolor}{rgb}{0.150000,0.150000,0.150000}%
\pgfsetstrokecolor{textcolor}%
\pgfsetfillcolor{textcolor}%
\pgftext[x=2.377847in,y=0.402778in,,top]{\color{textcolor}\rmfamily\fontsize{14.000000}{16.800000}\selectfont 10}%
\end{pgfscope}%
\begin{pgfscope}%
\pgfpathrectangle{\pgfqpoint{1.000000in}{0.500000in}}{\pgfqpoint{2.818182in}{3.020000in}}%
\pgfusepath{clip}%
\pgfsetroundcap%
\pgfsetroundjoin%
\pgfsetlinewidth{0.803000pt}%
\definecolor{currentstroke}{rgb}{1.000000,1.000000,1.000000}%
\pgfsetstrokecolor{currentstroke}%
\pgfsetdash{}{0pt}%
\pgfpathmoveto{\pgfqpoint{3.627595in}{0.500000in}}%
\pgfpathlineto{\pgfqpoint{3.627595in}{3.520000in}}%
\pgfusepath{stroke}%
\end{pgfscope}%
\begin{pgfscope}%
\definecolor{textcolor}{rgb}{0.150000,0.150000,0.150000}%
\pgfsetstrokecolor{textcolor}%
\pgfsetfillcolor{textcolor}%
\pgftext[x=3.627595in,y=0.402778in,,top]{\color{textcolor}\rmfamily\fontsize{14.000000}{16.800000}\selectfont 20}%
\end{pgfscope}%
\begin{pgfscope}%
\pgfpathrectangle{\pgfqpoint{1.000000in}{0.500000in}}{\pgfqpoint{2.818182in}{3.020000in}}%
\pgfusepath{clip}%
\pgfsetroundcap%
\pgfsetroundjoin%
\pgfsetlinewidth{0.803000pt}%
\definecolor{currentstroke}{rgb}{1.000000,1.000000,1.000000}%
\pgfsetstrokecolor{currentstroke}%
\pgfsetdash{}{0pt}%
\pgfpathmoveto{\pgfqpoint{1.000000in}{0.775215in}}%
\pgfpathlineto{\pgfqpoint{3.818182in}{0.775215in}}%
\pgfusepath{stroke}%
\end{pgfscope}%
\begin{pgfscope}%
\definecolor{textcolor}{rgb}{0.150000,0.150000,0.150000}%
\pgfsetstrokecolor{textcolor}%
\pgfsetfillcolor{textcolor}%
\pgftext[x=0.593547in,y=0.701349in,left,base]{\color{textcolor}\rmfamily\fontsize{14.000000}{16.800000}\selectfont 0.0}%
\end{pgfscope}%
\begin{pgfscope}%
\pgfpathrectangle{\pgfqpoint{1.000000in}{0.500000in}}{\pgfqpoint{2.818182in}{3.020000in}}%
\pgfusepath{clip}%
\pgfsetroundcap%
\pgfsetroundjoin%
\pgfsetlinewidth{0.803000pt}%
\definecolor{currentstroke}{rgb}{1.000000,1.000000,1.000000}%
\pgfsetstrokecolor{currentstroke}%
\pgfsetdash{}{0pt}%
\pgfpathmoveto{\pgfqpoint{1.000000in}{1.296717in}}%
\pgfpathlineto{\pgfqpoint{3.818182in}{1.296717in}}%
\pgfusepath{stroke}%
\end{pgfscope}%
\begin{pgfscope}%
\definecolor{textcolor}{rgb}{0.150000,0.150000,0.150000}%
\pgfsetstrokecolor{textcolor}%
\pgfsetfillcolor{textcolor}%
\pgftext[x=0.593547in,y=1.222851in,left,base]{\color{textcolor}\rmfamily\fontsize{14.000000}{16.800000}\selectfont 0.2}%
\end{pgfscope}%
\begin{pgfscope}%
\pgfpathrectangle{\pgfqpoint{1.000000in}{0.500000in}}{\pgfqpoint{2.818182in}{3.020000in}}%
\pgfusepath{clip}%
\pgfsetroundcap%
\pgfsetroundjoin%
\pgfsetlinewidth{0.803000pt}%
\definecolor{currentstroke}{rgb}{1.000000,1.000000,1.000000}%
\pgfsetstrokecolor{currentstroke}%
\pgfsetdash{}{0pt}%
\pgfpathmoveto{\pgfqpoint{1.000000in}{1.818220in}}%
\pgfpathlineto{\pgfqpoint{3.818182in}{1.818220in}}%
\pgfusepath{stroke}%
\end{pgfscope}%
\begin{pgfscope}%
\definecolor{textcolor}{rgb}{0.150000,0.150000,0.150000}%
\pgfsetstrokecolor{textcolor}%
\pgfsetfillcolor{textcolor}%
\pgftext[x=0.593547in,y=1.744354in,left,base]{\color{textcolor}\rmfamily\fontsize{14.000000}{16.800000}\selectfont 0.4}%
\end{pgfscope}%
\begin{pgfscope}%
\pgfpathrectangle{\pgfqpoint{1.000000in}{0.500000in}}{\pgfqpoint{2.818182in}{3.020000in}}%
\pgfusepath{clip}%
\pgfsetroundcap%
\pgfsetroundjoin%
\pgfsetlinewidth{0.803000pt}%
\definecolor{currentstroke}{rgb}{1.000000,1.000000,1.000000}%
\pgfsetstrokecolor{currentstroke}%
\pgfsetdash{}{0pt}%
\pgfpathmoveto{\pgfqpoint{1.000000in}{2.339722in}}%
\pgfpathlineto{\pgfqpoint{3.818182in}{2.339722in}}%
\pgfusepath{stroke}%
\end{pgfscope}%
\begin{pgfscope}%
\definecolor{textcolor}{rgb}{0.150000,0.150000,0.150000}%
\pgfsetstrokecolor{textcolor}%
\pgfsetfillcolor{textcolor}%
\pgftext[x=0.593547in,y=2.265856in,left,base]{\color{textcolor}\rmfamily\fontsize{14.000000}{16.800000}\selectfont 0.6}%
\end{pgfscope}%
\begin{pgfscope}%
\pgfpathrectangle{\pgfqpoint{1.000000in}{0.500000in}}{\pgfqpoint{2.818182in}{3.020000in}}%
\pgfusepath{clip}%
\pgfsetroundcap%
\pgfsetroundjoin%
\pgfsetlinewidth{0.803000pt}%
\definecolor{currentstroke}{rgb}{1.000000,1.000000,1.000000}%
\pgfsetstrokecolor{currentstroke}%
\pgfsetdash{}{0pt}%
\pgfpathmoveto{\pgfqpoint{1.000000in}{2.861225in}}%
\pgfpathlineto{\pgfqpoint{3.818182in}{2.861225in}}%
\pgfusepath{stroke}%
\end{pgfscope}%
\begin{pgfscope}%
\definecolor{textcolor}{rgb}{0.150000,0.150000,0.150000}%
\pgfsetstrokecolor{textcolor}%
\pgfsetfillcolor{textcolor}%
\pgftext[x=0.593547in,y=2.787359in,left,base]{\color{textcolor}\rmfamily\fontsize{14.000000}{16.800000}\selectfont 0.8}%
\end{pgfscope}%
\begin{pgfscope}%
\pgfpathrectangle{\pgfqpoint{1.000000in}{0.500000in}}{\pgfqpoint{2.818182in}{3.020000in}}%
\pgfusepath{clip}%
\pgfsetroundcap%
\pgfsetroundjoin%
\pgfsetlinewidth{0.803000pt}%
\definecolor{currentstroke}{rgb}{1.000000,1.000000,1.000000}%
\pgfsetstrokecolor{currentstroke}%
\pgfsetdash{}{0pt}%
\pgfpathmoveto{\pgfqpoint{1.000000in}{3.382727in}}%
\pgfpathlineto{\pgfqpoint{3.818182in}{3.382727in}}%
\pgfusepath{stroke}%
\end{pgfscope}%
\begin{pgfscope}%
\definecolor{textcolor}{rgb}{0.150000,0.150000,0.150000}%
\pgfsetstrokecolor{textcolor}%
\pgfsetfillcolor{textcolor}%
\pgftext[x=0.593547in,y=3.308861in,left,base]{\color{textcolor}\rmfamily\fontsize{14.000000}{16.800000}\selectfont 1.0}%
\end{pgfscope}%
\begin{pgfscope}%
\pgfpathrectangle{\pgfqpoint{1.000000in}{0.500000in}}{\pgfqpoint{2.818182in}{3.020000in}}%
\pgfusepath{clip}%
\pgfsetbuttcap%
\pgfsetroundjoin%
\definecolor{currentfill}{rgb}{0.121569,0.466667,0.705882}%
\pgfsetfillcolor{currentfill}%
\pgfsetfillopacity{0.250000}%
\pgfsetlinewidth{1.003750pt}%
\definecolor{currentstroke}{rgb}{1.000000,1.000000,1.000000}%
\pgfsetstrokecolor{currentstroke}%
\pgfsetstrokeopacity{0.250000}%
\pgfsetdash{}{0pt}%
\pgfpathmoveto{\pgfqpoint{1.190587in}{0.906820in}}%
\pgfpathlineto{\pgfqpoint{1.190587in}{0.643610in}}%
\pgfpathlineto{\pgfqpoint{1.378049in}{0.643555in}}%
\pgfpathlineto{\pgfqpoint{1.503024in}{0.643457in}}%
\pgfpathlineto{\pgfqpoint{1.627998in}{0.643427in}}%
\pgfpathlineto{\pgfqpoint{1.752973in}{0.643059in}}%
\pgfpathlineto{\pgfqpoint{1.877948in}{0.642867in}}%
\pgfpathlineto{\pgfqpoint{2.002923in}{0.642835in}}%
\pgfpathlineto{\pgfqpoint{2.127898in}{0.642821in}}%
\pgfpathlineto{\pgfqpoint{2.252872in}{0.642818in}}%
\pgfpathlineto{\pgfqpoint{2.377847in}{0.642734in}}%
\pgfpathlineto{\pgfqpoint{2.502822in}{0.642650in}}%
\pgfpathlineto{\pgfqpoint{2.627797in}{0.642650in}}%
\pgfpathlineto{\pgfqpoint{2.752772in}{0.642650in}}%
\pgfpathlineto{\pgfqpoint{2.877746in}{0.642645in}}%
\pgfpathlineto{\pgfqpoint{3.002721in}{0.642645in}}%
\pgfpathlineto{\pgfqpoint{3.127696in}{0.642544in}}%
\pgfpathlineto{\pgfqpoint{3.252671in}{0.642360in}}%
\pgfpathlineto{\pgfqpoint{3.377646in}{0.642310in}}%
\pgfpathlineto{\pgfqpoint{3.502620in}{0.642309in}}%
\pgfpathlineto{\pgfqpoint{3.690083in}{0.642304in}}%
\pgfpathlineto{\pgfqpoint{3.690083in}{0.908126in}}%
\pgfpathlineto{\pgfqpoint{3.690083in}{0.908126in}}%
\pgfpathlineto{\pgfqpoint{3.502620in}{0.908121in}}%
\pgfpathlineto{\pgfqpoint{3.377646in}{0.908119in}}%
\pgfpathlineto{\pgfqpoint{3.252671in}{0.908070in}}%
\pgfpathlineto{\pgfqpoint{3.127696in}{0.907886in}}%
\pgfpathlineto{\pgfqpoint{3.002721in}{0.907785in}}%
\pgfpathlineto{\pgfqpoint{2.877746in}{0.907785in}}%
\pgfpathlineto{\pgfqpoint{2.752772in}{0.907780in}}%
\pgfpathlineto{\pgfqpoint{2.627797in}{0.907780in}}%
\pgfpathlineto{\pgfqpoint{2.502822in}{0.907780in}}%
\pgfpathlineto{\pgfqpoint{2.377847in}{0.907696in}}%
\pgfpathlineto{\pgfqpoint{2.252872in}{0.907612in}}%
\pgfpathlineto{\pgfqpoint{2.127898in}{0.907609in}}%
\pgfpathlineto{\pgfqpoint{2.002923in}{0.907595in}}%
\pgfpathlineto{\pgfqpoint{1.877948in}{0.907562in}}%
\pgfpathlineto{\pgfqpoint{1.752973in}{0.907371in}}%
\pgfpathlineto{\pgfqpoint{1.627998in}{0.907003in}}%
\pgfpathlineto{\pgfqpoint{1.503024in}{0.906972in}}%
\pgfpathlineto{\pgfqpoint{1.378049in}{0.906875in}}%
\pgfpathlineto{\pgfqpoint{1.190587in}{0.906820in}}%
\pgfpathclose%
\pgfusepath{stroke,fill}%
\end{pgfscope}%
\begin{pgfscope}%
\pgfpathrectangle{\pgfqpoint{1.000000in}{0.500000in}}{\pgfqpoint{2.818182in}{3.020000in}}%
\pgfusepath{clip}%
\pgfsetbuttcap%
\pgfsetroundjoin%
\pgfsetlinewidth{1.505625pt}%
\definecolor{currentstroke}{rgb}{0.000000,0.000000,0.000000}%
\pgfsetstrokecolor{currentstroke}%
\pgfsetdash{}{0pt}%
\pgfpathmoveto{\pgfqpoint{1.128099in}{0.775215in}}%
\pgfpathlineto{\pgfqpoint{1.128099in}{3.382727in}}%
\pgfusepath{stroke}%
\end{pgfscope}%
\begin{pgfscope}%
\pgfpathrectangle{\pgfqpoint{1.000000in}{0.500000in}}{\pgfqpoint{2.818182in}{3.020000in}}%
\pgfusepath{clip}%
\pgfsetbuttcap%
\pgfsetroundjoin%
\pgfsetlinewidth{1.505625pt}%
\definecolor{currentstroke}{rgb}{0.000000,0.000000,0.000000}%
\pgfsetstrokecolor{currentstroke}%
\pgfsetdash{}{0pt}%
\pgfpathmoveto{\pgfqpoint{1.253074in}{0.775215in}}%
\pgfpathlineto{\pgfqpoint{1.253074in}{0.722064in}}%
\pgfusepath{stroke}%
\end{pgfscope}%
\begin{pgfscope}%
\pgfpathrectangle{\pgfqpoint{1.000000in}{0.500000in}}{\pgfqpoint{2.818182in}{3.020000in}}%
\pgfusepath{clip}%
\pgfsetbuttcap%
\pgfsetroundjoin%
\pgfsetlinewidth{1.505625pt}%
\definecolor{currentstroke}{rgb}{0.000000,0.000000,0.000000}%
\pgfsetstrokecolor{currentstroke}%
\pgfsetdash{}{0pt}%
\pgfpathmoveto{\pgfqpoint{1.378049in}{0.775215in}}%
\pgfpathlineto{\pgfqpoint{1.378049in}{0.704253in}}%
\pgfusepath{stroke}%
\end{pgfscope}%
\begin{pgfscope}%
\pgfpathrectangle{\pgfqpoint{1.000000in}{0.500000in}}{\pgfqpoint{2.818182in}{3.020000in}}%
\pgfusepath{clip}%
\pgfsetbuttcap%
\pgfsetroundjoin%
\pgfsetlinewidth{1.505625pt}%
\definecolor{currentstroke}{rgb}{0.000000,0.000000,0.000000}%
\pgfsetstrokecolor{currentstroke}%
\pgfsetdash{}{0pt}%
\pgfpathmoveto{\pgfqpoint{1.503024in}{0.775215in}}%
\pgfpathlineto{\pgfqpoint{1.503024in}{0.735298in}}%
\pgfusepath{stroke}%
\end{pgfscope}%
\begin{pgfscope}%
\pgfpathrectangle{\pgfqpoint{1.000000in}{0.500000in}}{\pgfqpoint{2.818182in}{3.020000in}}%
\pgfusepath{clip}%
\pgfsetbuttcap%
\pgfsetroundjoin%
\pgfsetlinewidth{1.505625pt}%
\definecolor{currentstroke}{rgb}{0.000000,0.000000,0.000000}%
\pgfsetstrokecolor{currentstroke}%
\pgfsetdash{}{0pt}%
\pgfpathmoveto{\pgfqpoint{1.627998in}{0.775215in}}%
\pgfpathlineto{\pgfqpoint{1.627998in}{0.637273in}}%
\pgfusepath{stroke}%
\end{pgfscope}%
\begin{pgfscope}%
\pgfpathrectangle{\pgfqpoint{1.000000in}{0.500000in}}{\pgfqpoint{2.818182in}{3.020000in}}%
\pgfusepath{clip}%
\pgfsetbuttcap%
\pgfsetroundjoin%
\pgfsetlinewidth{1.505625pt}%
\definecolor{currentstroke}{rgb}{0.000000,0.000000,0.000000}%
\pgfsetstrokecolor{currentstroke}%
\pgfsetdash{}{0pt}%
\pgfpathmoveto{\pgfqpoint{1.752973in}{0.775215in}}%
\pgfpathlineto{\pgfqpoint{1.752973in}{0.875024in}}%
\pgfusepath{stroke}%
\end{pgfscope}%
\begin{pgfscope}%
\pgfpathrectangle{\pgfqpoint{1.000000in}{0.500000in}}{\pgfqpoint{2.818182in}{3.020000in}}%
\pgfusepath{clip}%
\pgfsetbuttcap%
\pgfsetroundjoin%
\pgfsetlinewidth{1.505625pt}%
\definecolor{currentstroke}{rgb}{0.000000,0.000000,0.000000}%
\pgfsetstrokecolor{currentstroke}%
\pgfsetdash{}{0pt}%
\pgfpathmoveto{\pgfqpoint{1.877948in}{0.775215in}}%
\pgfpathlineto{\pgfqpoint{1.877948in}{0.816248in}}%
\pgfusepath{stroke}%
\end{pgfscope}%
\begin{pgfscope}%
\pgfpathrectangle{\pgfqpoint{1.000000in}{0.500000in}}{\pgfqpoint{2.818182in}{3.020000in}}%
\pgfusepath{clip}%
\pgfsetbuttcap%
\pgfsetroundjoin%
\pgfsetlinewidth{1.505625pt}%
\definecolor{currentstroke}{rgb}{0.000000,0.000000,0.000000}%
\pgfsetstrokecolor{currentstroke}%
\pgfsetdash{}{0pt}%
\pgfpathmoveto{\pgfqpoint{2.002923in}{0.775215in}}%
\pgfpathlineto{\pgfqpoint{2.002923in}{0.802020in}}%
\pgfusepath{stroke}%
\end{pgfscope}%
\begin{pgfscope}%
\pgfpathrectangle{\pgfqpoint{1.000000in}{0.500000in}}{\pgfqpoint{2.818182in}{3.020000in}}%
\pgfusepath{clip}%
\pgfsetbuttcap%
\pgfsetroundjoin%
\pgfsetlinewidth{1.505625pt}%
\definecolor{currentstroke}{rgb}{0.000000,0.000000,0.000000}%
\pgfsetstrokecolor{currentstroke}%
\pgfsetdash{}{0pt}%
\pgfpathmoveto{\pgfqpoint{2.127898in}{0.775215in}}%
\pgfpathlineto{\pgfqpoint{2.127898in}{0.789155in}}%
\pgfusepath{stroke}%
\end{pgfscope}%
\begin{pgfscope}%
\pgfpathrectangle{\pgfqpoint{1.000000in}{0.500000in}}{\pgfqpoint{2.818182in}{3.020000in}}%
\pgfusepath{clip}%
\pgfsetbuttcap%
\pgfsetroundjoin%
\pgfsetlinewidth{1.505625pt}%
\definecolor{currentstroke}{rgb}{0.000000,0.000000,0.000000}%
\pgfsetstrokecolor{currentstroke}%
\pgfsetdash{}{0pt}%
\pgfpathmoveto{\pgfqpoint{2.252872in}{0.775215in}}%
\pgfpathlineto{\pgfqpoint{2.252872in}{0.709290in}}%
\pgfusepath{stroke}%
\end{pgfscope}%
\begin{pgfscope}%
\pgfpathrectangle{\pgfqpoint{1.000000in}{0.500000in}}{\pgfqpoint{2.818182in}{3.020000in}}%
\pgfusepath{clip}%
\pgfsetbuttcap%
\pgfsetroundjoin%
\pgfsetlinewidth{1.505625pt}%
\definecolor{currentstroke}{rgb}{0.000000,0.000000,0.000000}%
\pgfsetstrokecolor{currentstroke}%
\pgfsetdash{}{0pt}%
\pgfpathmoveto{\pgfqpoint{2.377847in}{0.775215in}}%
\pgfpathlineto{\pgfqpoint{2.377847in}{0.709244in}}%
\pgfusepath{stroke}%
\end{pgfscope}%
\begin{pgfscope}%
\pgfpathrectangle{\pgfqpoint{1.000000in}{0.500000in}}{\pgfqpoint{2.818182in}{3.020000in}}%
\pgfusepath{clip}%
\pgfsetbuttcap%
\pgfsetroundjoin%
\pgfsetlinewidth{1.505625pt}%
\definecolor{currentstroke}{rgb}{0.000000,0.000000,0.000000}%
\pgfsetstrokecolor{currentstroke}%
\pgfsetdash{}{0pt}%
\pgfpathmoveto{\pgfqpoint{2.502822in}{0.775215in}}%
\pgfpathlineto{\pgfqpoint{2.502822in}{0.776255in}}%
\pgfusepath{stroke}%
\end{pgfscope}%
\begin{pgfscope}%
\pgfpathrectangle{\pgfqpoint{1.000000in}{0.500000in}}{\pgfqpoint{2.818182in}{3.020000in}}%
\pgfusepath{clip}%
\pgfsetbuttcap%
\pgfsetroundjoin%
\pgfsetlinewidth{1.505625pt}%
\definecolor{currentstroke}{rgb}{0.000000,0.000000,0.000000}%
\pgfsetstrokecolor{currentstroke}%
\pgfsetdash{}{0pt}%
\pgfpathmoveto{\pgfqpoint{2.627797in}{0.775215in}}%
\pgfpathlineto{\pgfqpoint{2.627797in}{0.770626in}}%
\pgfusepath{stroke}%
\end{pgfscope}%
\begin{pgfscope}%
\pgfpathrectangle{\pgfqpoint{1.000000in}{0.500000in}}{\pgfqpoint{2.818182in}{3.020000in}}%
\pgfusepath{clip}%
\pgfsetbuttcap%
\pgfsetroundjoin%
\pgfsetlinewidth{1.505625pt}%
\definecolor{currentstroke}{rgb}{0.000000,0.000000,0.000000}%
\pgfsetstrokecolor{currentstroke}%
\pgfsetdash{}{0pt}%
\pgfpathmoveto{\pgfqpoint{2.752772in}{0.775215in}}%
\pgfpathlineto{\pgfqpoint{2.752772in}{0.758920in}}%
\pgfusepath{stroke}%
\end{pgfscope}%
\begin{pgfscope}%
\pgfpathrectangle{\pgfqpoint{1.000000in}{0.500000in}}{\pgfqpoint{2.818182in}{3.020000in}}%
\pgfusepath{clip}%
\pgfsetbuttcap%
\pgfsetroundjoin%
\pgfsetlinewidth{1.505625pt}%
\definecolor{currentstroke}{rgb}{0.000000,0.000000,0.000000}%
\pgfsetstrokecolor{currentstroke}%
\pgfsetdash{}{0pt}%
\pgfpathmoveto{\pgfqpoint{2.877746in}{0.775215in}}%
\pgfpathlineto{\pgfqpoint{2.877746in}{0.772837in}}%
\pgfusepath{stroke}%
\end{pgfscope}%
\begin{pgfscope}%
\pgfpathrectangle{\pgfqpoint{1.000000in}{0.500000in}}{\pgfqpoint{2.818182in}{3.020000in}}%
\pgfusepath{clip}%
\pgfsetbuttcap%
\pgfsetroundjoin%
\pgfsetlinewidth{1.505625pt}%
\definecolor{currentstroke}{rgb}{0.000000,0.000000,0.000000}%
\pgfsetstrokecolor{currentstroke}%
\pgfsetdash{}{0pt}%
\pgfpathmoveto{\pgfqpoint{3.002721in}{0.775215in}}%
\pgfpathlineto{\pgfqpoint{3.002721in}{0.702752in}}%
\pgfusepath{stroke}%
\end{pgfscope}%
\begin{pgfscope}%
\pgfpathrectangle{\pgfqpoint{1.000000in}{0.500000in}}{\pgfqpoint{2.818182in}{3.020000in}}%
\pgfusepath{clip}%
\pgfsetbuttcap%
\pgfsetroundjoin%
\pgfsetlinewidth{1.505625pt}%
\definecolor{currentstroke}{rgb}{0.000000,0.000000,0.000000}%
\pgfsetstrokecolor{currentstroke}%
\pgfsetdash{}{0pt}%
\pgfpathmoveto{\pgfqpoint{3.127696in}{0.775215in}}%
\pgfpathlineto{\pgfqpoint{3.127696in}{0.873163in}}%
\pgfusepath{stroke}%
\end{pgfscope}%
\begin{pgfscope}%
\pgfpathrectangle{\pgfqpoint{1.000000in}{0.500000in}}{\pgfqpoint{2.818182in}{3.020000in}}%
\pgfusepath{clip}%
\pgfsetbuttcap%
\pgfsetroundjoin%
\pgfsetlinewidth{1.505625pt}%
\definecolor{currentstroke}{rgb}{0.000000,0.000000,0.000000}%
\pgfsetstrokecolor{currentstroke}%
\pgfsetdash{}{0pt}%
\pgfpathmoveto{\pgfqpoint{3.252671in}{0.775215in}}%
\pgfpathlineto{\pgfqpoint{3.252671in}{0.825894in}}%
\pgfusepath{stroke}%
\end{pgfscope}%
\begin{pgfscope}%
\pgfpathrectangle{\pgfqpoint{1.000000in}{0.500000in}}{\pgfqpoint{2.818182in}{3.020000in}}%
\pgfusepath{clip}%
\pgfsetbuttcap%
\pgfsetroundjoin%
\pgfsetlinewidth{1.505625pt}%
\definecolor{currentstroke}{rgb}{0.000000,0.000000,0.000000}%
\pgfsetstrokecolor{currentstroke}%
\pgfsetdash{}{0pt}%
\pgfpathmoveto{\pgfqpoint{3.377646in}{0.775215in}}%
\pgfpathlineto{\pgfqpoint{3.377646in}{0.785295in}}%
\pgfusepath{stroke}%
\end{pgfscope}%
\begin{pgfscope}%
\pgfpathrectangle{\pgfqpoint{1.000000in}{0.500000in}}{\pgfqpoint{2.818182in}{3.020000in}}%
\pgfusepath{clip}%
\pgfsetbuttcap%
\pgfsetroundjoin%
\pgfsetlinewidth{1.505625pt}%
\definecolor{currentstroke}{rgb}{0.000000,0.000000,0.000000}%
\pgfsetstrokecolor{currentstroke}%
\pgfsetdash{}{0pt}%
\pgfpathmoveto{\pgfqpoint{3.502620in}{0.775215in}}%
\pgfpathlineto{\pgfqpoint{3.502620in}{0.791248in}}%
\pgfusepath{stroke}%
\end{pgfscope}%
\begin{pgfscope}%
\pgfpathrectangle{\pgfqpoint{1.000000in}{0.500000in}}{\pgfqpoint{2.818182in}{3.020000in}}%
\pgfusepath{clip}%
\pgfsetbuttcap%
\pgfsetroundjoin%
\pgfsetlinewidth{1.505625pt}%
\definecolor{currentstroke}{rgb}{0.000000,0.000000,0.000000}%
\pgfsetstrokecolor{currentstroke}%
\pgfsetdash{}{0pt}%
\pgfpathmoveto{\pgfqpoint{3.627595in}{0.775215in}}%
\pgfpathlineto{\pgfqpoint{3.627595in}{0.722991in}}%
\pgfusepath{stroke}%
\end{pgfscope}%
\begin{pgfscope}%
\pgfpathrectangle{\pgfqpoint{1.000000in}{0.500000in}}{\pgfqpoint{2.818182in}{3.020000in}}%
\pgfusepath{clip}%
\pgfsetroundcap%
\pgfsetroundjoin%
\pgfsetlinewidth{1.505625pt}%
\definecolor{currentstroke}{rgb}{0.549020,0.337255,0.294118}%
\pgfsetstrokecolor{currentstroke}%
\pgfsetdash{}{0pt}%
\pgfpathmoveto{\pgfqpoint{1.000000in}{0.775215in}}%
\pgfpathlineto{\pgfqpoint{3.818182in}{0.775215in}}%
\pgfusepath{stroke}%
\end{pgfscope}%
\begin{pgfscope}%
\pgfpathrectangle{\pgfqpoint{1.000000in}{0.500000in}}{\pgfqpoint{2.818182in}{3.020000in}}%
\pgfusepath{clip}%
\pgfsetbuttcap%
\pgfsetroundjoin%
\definecolor{currentfill}{rgb}{0.549020,0.337255,0.294118}%
\pgfsetfillcolor{currentfill}%
\pgfsetlinewidth{1.003750pt}%
\definecolor{currentstroke}{rgb}{0.549020,0.337255,0.294118}%
\pgfsetstrokecolor{currentstroke}%
\pgfsetdash{}{0pt}%
\pgfsys@defobject{currentmarker}{\pgfqpoint{-0.034722in}{-0.034722in}}{\pgfqpoint{0.034722in}{0.034722in}}{%
\pgfpathmoveto{\pgfqpoint{0.000000in}{-0.034722in}}%
\pgfpathcurveto{\pgfqpoint{0.009208in}{-0.034722in}}{\pgfqpoint{0.018041in}{-0.031064in}}{\pgfqpoint{0.024552in}{-0.024552in}}%
\pgfpathcurveto{\pgfqpoint{0.031064in}{-0.018041in}}{\pgfqpoint{0.034722in}{-0.009208in}}{\pgfqpoint{0.034722in}{0.000000in}}%
\pgfpathcurveto{\pgfqpoint{0.034722in}{0.009208in}}{\pgfqpoint{0.031064in}{0.018041in}}{\pgfqpoint{0.024552in}{0.024552in}}%
\pgfpathcurveto{\pgfqpoint{0.018041in}{0.031064in}}{\pgfqpoint{0.009208in}{0.034722in}}{\pgfqpoint{0.000000in}{0.034722in}}%
\pgfpathcurveto{\pgfqpoint{-0.009208in}{0.034722in}}{\pgfqpoint{-0.018041in}{0.031064in}}{\pgfqpoint{-0.024552in}{0.024552in}}%
\pgfpathcurveto{\pgfqpoint{-0.031064in}{0.018041in}}{\pgfqpoint{-0.034722in}{0.009208in}}{\pgfqpoint{-0.034722in}{0.000000in}}%
\pgfpathcurveto{\pgfqpoint{-0.034722in}{-0.009208in}}{\pgfqpoint{-0.031064in}{-0.018041in}}{\pgfqpoint{-0.024552in}{-0.024552in}}%
\pgfpathcurveto{\pgfqpoint{-0.018041in}{-0.031064in}}{\pgfqpoint{-0.009208in}{-0.034722in}}{\pgfqpoint{0.000000in}{-0.034722in}}%
\pgfpathclose%
\pgfusepath{stroke,fill}%
}%
\begin{pgfscope}%
\pgfsys@transformshift{1.128099in}{3.382727in}%
\pgfsys@useobject{currentmarker}{}%
\end{pgfscope}%
\begin{pgfscope}%
\pgfsys@transformshift{1.253074in}{0.722064in}%
\pgfsys@useobject{currentmarker}{}%
\end{pgfscope}%
\begin{pgfscope}%
\pgfsys@transformshift{1.378049in}{0.704253in}%
\pgfsys@useobject{currentmarker}{}%
\end{pgfscope}%
\begin{pgfscope}%
\pgfsys@transformshift{1.503024in}{0.735298in}%
\pgfsys@useobject{currentmarker}{}%
\end{pgfscope}%
\begin{pgfscope}%
\pgfsys@transformshift{1.627998in}{0.637273in}%
\pgfsys@useobject{currentmarker}{}%
\end{pgfscope}%
\begin{pgfscope}%
\pgfsys@transformshift{1.752973in}{0.875024in}%
\pgfsys@useobject{currentmarker}{}%
\end{pgfscope}%
\begin{pgfscope}%
\pgfsys@transformshift{1.877948in}{0.816248in}%
\pgfsys@useobject{currentmarker}{}%
\end{pgfscope}%
\begin{pgfscope}%
\pgfsys@transformshift{2.002923in}{0.802020in}%
\pgfsys@useobject{currentmarker}{}%
\end{pgfscope}%
\begin{pgfscope}%
\pgfsys@transformshift{2.127898in}{0.789155in}%
\pgfsys@useobject{currentmarker}{}%
\end{pgfscope}%
\begin{pgfscope}%
\pgfsys@transformshift{2.252872in}{0.709290in}%
\pgfsys@useobject{currentmarker}{}%
\end{pgfscope}%
\begin{pgfscope}%
\pgfsys@transformshift{2.377847in}{0.709244in}%
\pgfsys@useobject{currentmarker}{}%
\end{pgfscope}%
\begin{pgfscope}%
\pgfsys@transformshift{2.502822in}{0.776255in}%
\pgfsys@useobject{currentmarker}{}%
\end{pgfscope}%
\begin{pgfscope}%
\pgfsys@transformshift{2.627797in}{0.770626in}%
\pgfsys@useobject{currentmarker}{}%
\end{pgfscope}%
\begin{pgfscope}%
\pgfsys@transformshift{2.752772in}{0.758920in}%
\pgfsys@useobject{currentmarker}{}%
\end{pgfscope}%
\begin{pgfscope}%
\pgfsys@transformshift{2.877746in}{0.772837in}%
\pgfsys@useobject{currentmarker}{}%
\end{pgfscope}%
\begin{pgfscope}%
\pgfsys@transformshift{3.002721in}{0.702752in}%
\pgfsys@useobject{currentmarker}{}%
\end{pgfscope}%
\begin{pgfscope}%
\pgfsys@transformshift{3.127696in}{0.873163in}%
\pgfsys@useobject{currentmarker}{}%
\end{pgfscope}%
\begin{pgfscope}%
\pgfsys@transformshift{3.252671in}{0.825894in}%
\pgfsys@useobject{currentmarker}{}%
\end{pgfscope}%
\begin{pgfscope}%
\pgfsys@transformshift{3.377646in}{0.785295in}%
\pgfsys@useobject{currentmarker}{}%
\end{pgfscope}%
\begin{pgfscope}%
\pgfsys@transformshift{3.502620in}{0.791248in}%
\pgfsys@useobject{currentmarker}{}%
\end{pgfscope}%
\begin{pgfscope}%
\pgfsys@transformshift{3.627595in}{0.722991in}%
\pgfsys@useobject{currentmarker}{}%
\end{pgfscope}%
\end{pgfscope}%
\begin{pgfscope}%
\pgfsetrectcap%
\pgfsetmiterjoin%
\pgfsetlinewidth{0.803000pt}%
\definecolor{currentstroke}{rgb}{1.000000,1.000000,1.000000}%
\pgfsetstrokecolor{currentstroke}%
\pgfsetdash{}{0pt}%
\pgfpathmoveto{\pgfqpoint{1.000000in}{0.500000in}}%
\pgfpathlineto{\pgfqpoint{1.000000in}{3.520000in}}%
\pgfusepath{stroke}%
\end{pgfscope}%
\begin{pgfscope}%
\pgfsetrectcap%
\pgfsetmiterjoin%
\pgfsetlinewidth{0.803000pt}%
\definecolor{currentstroke}{rgb}{1.000000,1.000000,1.000000}%
\pgfsetstrokecolor{currentstroke}%
\pgfsetdash{}{0pt}%
\pgfpathmoveto{\pgfqpoint{3.818182in}{0.500000in}}%
\pgfpathlineto{\pgfqpoint{3.818182in}{3.520000in}}%
\pgfusepath{stroke}%
\end{pgfscope}%
\begin{pgfscope}%
\pgfsetrectcap%
\pgfsetmiterjoin%
\pgfsetlinewidth{0.803000pt}%
\definecolor{currentstroke}{rgb}{1.000000,1.000000,1.000000}%
\pgfsetstrokecolor{currentstroke}%
\pgfsetdash{}{0pt}%
\pgfpathmoveto{\pgfqpoint{1.000000in}{0.500000in}}%
\pgfpathlineto{\pgfqpoint{3.818182in}{0.500000in}}%
\pgfusepath{stroke}%
\end{pgfscope}%
\begin{pgfscope}%
\pgfsetrectcap%
\pgfsetmiterjoin%
\pgfsetlinewidth{0.803000pt}%
\definecolor{currentstroke}{rgb}{1.000000,1.000000,1.000000}%
\pgfsetstrokecolor{currentstroke}%
\pgfsetdash{}{0pt}%
\pgfpathmoveto{\pgfqpoint{1.000000in}{3.520000in}}%
\pgfpathlineto{\pgfqpoint{3.818182in}{3.520000in}}%
\pgfusepath{stroke}%
\end{pgfscope}%
\begin{pgfscope}%
\definecolor{textcolor}{rgb}{0.150000,0.150000,0.150000}%
\pgfsetstrokecolor{textcolor}%
\pgfsetfillcolor{textcolor}%
\pgftext[x=2.409091in,y=3.603333in,,base]{\color{textcolor}\rmfamily\fontsize{16.800000}{20.160000}\selectfont Autocorrelation}%
\end{pgfscope}%
\begin{pgfscope}%
\pgfsetbuttcap%
\pgfsetmiterjoin%
\definecolor{currentfill}{rgb}{0.917647,0.917647,0.949020}%
\pgfsetfillcolor{currentfill}%
\pgfsetlinewidth{0.000000pt}%
\definecolor{currentstroke}{rgb}{0.000000,0.000000,0.000000}%
\pgfsetstrokecolor{currentstroke}%
\pgfsetstrokeopacity{0.000000}%
\pgfsetdash{}{0pt}%
\pgfpathmoveto{\pgfqpoint{4.381818in}{0.500000in}}%
\pgfpathlineto{\pgfqpoint{7.200000in}{0.500000in}}%
\pgfpathlineto{\pgfqpoint{7.200000in}{3.520000in}}%
\pgfpathlineto{\pgfqpoint{4.381818in}{3.520000in}}%
\pgfpathclose%
\pgfusepath{fill}%
\end{pgfscope}%
\begin{pgfscope}%
\pgfpathrectangle{\pgfqpoint{4.381818in}{0.500000in}}{\pgfqpoint{2.818182in}{3.020000in}}%
\pgfusepath{clip}%
\pgfsetroundcap%
\pgfsetroundjoin%
\pgfsetlinewidth{0.803000pt}%
\definecolor{currentstroke}{rgb}{1.000000,1.000000,1.000000}%
\pgfsetstrokecolor{currentstroke}%
\pgfsetdash{}{0pt}%
\pgfpathmoveto{\pgfqpoint{4.509917in}{0.500000in}}%
\pgfpathlineto{\pgfqpoint{4.509917in}{3.520000in}}%
\pgfusepath{stroke}%
\end{pgfscope}%
\begin{pgfscope}%
\definecolor{textcolor}{rgb}{0.150000,0.150000,0.150000}%
\pgfsetstrokecolor{textcolor}%
\pgfsetfillcolor{textcolor}%
\pgftext[x=4.509917in,y=0.402778in,,top]{\color{textcolor}\rmfamily\fontsize{14.000000}{16.800000}\selectfont 0}%
\end{pgfscope}%
\begin{pgfscope}%
\pgfpathrectangle{\pgfqpoint{4.381818in}{0.500000in}}{\pgfqpoint{2.818182in}{3.020000in}}%
\pgfusepath{clip}%
\pgfsetroundcap%
\pgfsetroundjoin%
\pgfsetlinewidth{0.803000pt}%
\definecolor{currentstroke}{rgb}{1.000000,1.000000,1.000000}%
\pgfsetstrokecolor{currentstroke}%
\pgfsetdash{}{0pt}%
\pgfpathmoveto{\pgfqpoint{5.759665in}{0.500000in}}%
\pgfpathlineto{\pgfqpoint{5.759665in}{3.520000in}}%
\pgfusepath{stroke}%
\end{pgfscope}%
\begin{pgfscope}%
\definecolor{textcolor}{rgb}{0.150000,0.150000,0.150000}%
\pgfsetstrokecolor{textcolor}%
\pgfsetfillcolor{textcolor}%
\pgftext[x=5.759665in,y=0.402778in,,top]{\color{textcolor}\rmfamily\fontsize{14.000000}{16.800000}\selectfont 10}%
\end{pgfscope}%
\begin{pgfscope}%
\pgfpathrectangle{\pgfqpoint{4.381818in}{0.500000in}}{\pgfqpoint{2.818182in}{3.020000in}}%
\pgfusepath{clip}%
\pgfsetroundcap%
\pgfsetroundjoin%
\pgfsetlinewidth{0.803000pt}%
\definecolor{currentstroke}{rgb}{1.000000,1.000000,1.000000}%
\pgfsetstrokecolor{currentstroke}%
\pgfsetdash{}{0pt}%
\pgfpathmoveto{\pgfqpoint{7.009413in}{0.500000in}}%
\pgfpathlineto{\pgfqpoint{7.009413in}{3.520000in}}%
\pgfusepath{stroke}%
\end{pgfscope}%
\begin{pgfscope}%
\definecolor{textcolor}{rgb}{0.150000,0.150000,0.150000}%
\pgfsetstrokecolor{textcolor}%
\pgfsetfillcolor{textcolor}%
\pgftext[x=7.009413in,y=0.402778in,,top]{\color{textcolor}\rmfamily\fontsize{14.000000}{16.800000}\selectfont 20}%
\end{pgfscope}%
\begin{pgfscope}%
\pgfpathrectangle{\pgfqpoint{4.381818in}{0.500000in}}{\pgfqpoint{2.818182in}{3.020000in}}%
\pgfusepath{clip}%
\pgfsetroundcap%
\pgfsetroundjoin%
\pgfsetlinewidth{0.803000pt}%
\definecolor{currentstroke}{rgb}{1.000000,1.000000,1.000000}%
\pgfsetstrokecolor{currentstroke}%
\pgfsetdash{}{0pt}%
\pgfpathmoveto{\pgfqpoint{4.381818in}{0.779291in}}%
\pgfpathlineto{\pgfqpoint{7.200000in}{0.779291in}}%
\pgfusepath{stroke}%
\end{pgfscope}%
\begin{pgfscope}%
\definecolor{textcolor}{rgb}{0.150000,0.150000,0.150000}%
\pgfsetstrokecolor{textcolor}%
\pgfsetfillcolor{textcolor}%
\pgftext[x=3.975365in,y=0.705425in,left,base]{\color{textcolor}\rmfamily\fontsize{14.000000}{16.800000}\selectfont 0.0}%
\end{pgfscope}%
\begin{pgfscope}%
\pgfpathrectangle{\pgfqpoint{4.381818in}{0.500000in}}{\pgfqpoint{2.818182in}{3.020000in}}%
\pgfusepath{clip}%
\pgfsetroundcap%
\pgfsetroundjoin%
\pgfsetlinewidth{0.803000pt}%
\definecolor{currentstroke}{rgb}{1.000000,1.000000,1.000000}%
\pgfsetstrokecolor{currentstroke}%
\pgfsetdash{}{0pt}%
\pgfpathmoveto{\pgfqpoint{4.381818in}{1.299979in}}%
\pgfpathlineto{\pgfqpoint{7.200000in}{1.299979in}}%
\pgfusepath{stroke}%
\end{pgfscope}%
\begin{pgfscope}%
\definecolor{textcolor}{rgb}{0.150000,0.150000,0.150000}%
\pgfsetstrokecolor{textcolor}%
\pgfsetfillcolor{textcolor}%
\pgftext[x=3.975365in,y=1.226112in,left,base]{\color{textcolor}\rmfamily\fontsize{14.000000}{16.800000}\selectfont 0.2}%
\end{pgfscope}%
\begin{pgfscope}%
\pgfpathrectangle{\pgfqpoint{4.381818in}{0.500000in}}{\pgfqpoint{2.818182in}{3.020000in}}%
\pgfusepath{clip}%
\pgfsetroundcap%
\pgfsetroundjoin%
\pgfsetlinewidth{0.803000pt}%
\definecolor{currentstroke}{rgb}{1.000000,1.000000,1.000000}%
\pgfsetstrokecolor{currentstroke}%
\pgfsetdash{}{0pt}%
\pgfpathmoveto{\pgfqpoint{4.381818in}{1.820666in}}%
\pgfpathlineto{\pgfqpoint{7.200000in}{1.820666in}}%
\pgfusepath{stroke}%
\end{pgfscope}%
\begin{pgfscope}%
\definecolor{textcolor}{rgb}{0.150000,0.150000,0.150000}%
\pgfsetstrokecolor{textcolor}%
\pgfsetfillcolor{textcolor}%
\pgftext[x=3.975365in,y=1.746800in,left,base]{\color{textcolor}\rmfamily\fontsize{14.000000}{16.800000}\selectfont 0.4}%
\end{pgfscope}%
\begin{pgfscope}%
\pgfpathrectangle{\pgfqpoint{4.381818in}{0.500000in}}{\pgfqpoint{2.818182in}{3.020000in}}%
\pgfusepath{clip}%
\pgfsetroundcap%
\pgfsetroundjoin%
\pgfsetlinewidth{0.803000pt}%
\definecolor{currentstroke}{rgb}{1.000000,1.000000,1.000000}%
\pgfsetstrokecolor{currentstroke}%
\pgfsetdash{}{0pt}%
\pgfpathmoveto{\pgfqpoint{4.381818in}{2.341353in}}%
\pgfpathlineto{\pgfqpoint{7.200000in}{2.341353in}}%
\pgfusepath{stroke}%
\end{pgfscope}%
\begin{pgfscope}%
\definecolor{textcolor}{rgb}{0.150000,0.150000,0.150000}%
\pgfsetstrokecolor{textcolor}%
\pgfsetfillcolor{textcolor}%
\pgftext[x=3.975365in,y=2.267487in,left,base]{\color{textcolor}\rmfamily\fontsize{14.000000}{16.800000}\selectfont 0.6}%
\end{pgfscope}%
\begin{pgfscope}%
\pgfpathrectangle{\pgfqpoint{4.381818in}{0.500000in}}{\pgfqpoint{2.818182in}{3.020000in}}%
\pgfusepath{clip}%
\pgfsetroundcap%
\pgfsetroundjoin%
\pgfsetlinewidth{0.803000pt}%
\definecolor{currentstroke}{rgb}{1.000000,1.000000,1.000000}%
\pgfsetstrokecolor{currentstroke}%
\pgfsetdash{}{0pt}%
\pgfpathmoveto{\pgfqpoint{4.381818in}{2.862040in}}%
\pgfpathlineto{\pgfqpoint{7.200000in}{2.862040in}}%
\pgfusepath{stroke}%
\end{pgfscope}%
\begin{pgfscope}%
\definecolor{textcolor}{rgb}{0.150000,0.150000,0.150000}%
\pgfsetstrokecolor{textcolor}%
\pgfsetfillcolor{textcolor}%
\pgftext[x=3.975365in,y=2.788174in,left,base]{\color{textcolor}\rmfamily\fontsize{14.000000}{16.800000}\selectfont 0.8}%
\end{pgfscope}%
\begin{pgfscope}%
\pgfpathrectangle{\pgfqpoint{4.381818in}{0.500000in}}{\pgfqpoint{2.818182in}{3.020000in}}%
\pgfusepath{clip}%
\pgfsetroundcap%
\pgfsetroundjoin%
\pgfsetlinewidth{0.803000pt}%
\definecolor{currentstroke}{rgb}{1.000000,1.000000,1.000000}%
\pgfsetstrokecolor{currentstroke}%
\pgfsetdash{}{0pt}%
\pgfpathmoveto{\pgfqpoint{4.381818in}{3.382727in}}%
\pgfpathlineto{\pgfqpoint{7.200000in}{3.382727in}}%
\pgfusepath{stroke}%
\end{pgfscope}%
\begin{pgfscope}%
\definecolor{textcolor}{rgb}{0.150000,0.150000,0.150000}%
\pgfsetstrokecolor{textcolor}%
\pgfsetfillcolor{textcolor}%
\pgftext[x=3.975365in,y=3.308861in,left,base]{\color{textcolor}\rmfamily\fontsize{14.000000}{16.800000}\selectfont 1.0}%
\end{pgfscope}%
\begin{pgfscope}%
\pgfpathrectangle{\pgfqpoint{4.381818in}{0.500000in}}{\pgfqpoint{2.818182in}{3.020000in}}%
\pgfusepath{clip}%
\pgfsetbuttcap%
\pgfsetroundjoin%
\definecolor{currentfill}{rgb}{0.121569,0.466667,0.705882}%
\pgfsetfillcolor{currentfill}%
\pgfsetfillopacity{0.250000}%
\pgfsetlinewidth{1.003750pt}%
\definecolor{currentstroke}{rgb}{1.000000,1.000000,1.000000}%
\pgfsetstrokecolor{currentstroke}%
\pgfsetstrokeopacity{0.250000}%
\pgfsetdash{}{0pt}%
\pgfpathmoveto{\pgfqpoint{4.572405in}{0.910691in}}%
\pgfpathlineto{\pgfqpoint{4.572405in}{0.647892in}}%
\pgfpathlineto{\pgfqpoint{4.759867in}{0.647892in}}%
\pgfpathlineto{\pgfqpoint{4.884842in}{0.647892in}}%
\pgfpathlineto{\pgfqpoint{5.009817in}{0.647892in}}%
\pgfpathlineto{\pgfqpoint{5.134791in}{0.647892in}}%
\pgfpathlineto{\pgfqpoint{5.259766in}{0.647892in}}%
\pgfpathlineto{\pgfqpoint{5.384741in}{0.647892in}}%
\pgfpathlineto{\pgfqpoint{5.509716in}{0.647892in}}%
\pgfpathlineto{\pgfqpoint{5.634691in}{0.647892in}}%
\pgfpathlineto{\pgfqpoint{5.759665in}{0.647892in}}%
\pgfpathlineto{\pgfqpoint{5.884640in}{0.647892in}}%
\pgfpathlineto{\pgfqpoint{6.009615in}{0.647892in}}%
\pgfpathlineto{\pgfqpoint{6.134590in}{0.647892in}}%
\pgfpathlineto{\pgfqpoint{6.259565in}{0.647892in}}%
\pgfpathlineto{\pgfqpoint{6.384539in}{0.647892in}}%
\pgfpathlineto{\pgfqpoint{6.509514in}{0.647892in}}%
\pgfpathlineto{\pgfqpoint{6.634489in}{0.647892in}}%
\pgfpathlineto{\pgfqpoint{6.759464in}{0.647892in}}%
\pgfpathlineto{\pgfqpoint{6.884439in}{0.647892in}}%
\pgfpathlineto{\pgfqpoint{7.071901in}{0.647892in}}%
\pgfpathlineto{\pgfqpoint{7.071901in}{0.910691in}}%
\pgfpathlineto{\pgfqpoint{7.071901in}{0.910691in}}%
\pgfpathlineto{\pgfqpoint{6.884439in}{0.910691in}}%
\pgfpathlineto{\pgfqpoint{6.759464in}{0.910691in}}%
\pgfpathlineto{\pgfqpoint{6.634489in}{0.910691in}}%
\pgfpathlineto{\pgfqpoint{6.509514in}{0.910691in}}%
\pgfpathlineto{\pgfqpoint{6.384539in}{0.910691in}}%
\pgfpathlineto{\pgfqpoint{6.259565in}{0.910691in}}%
\pgfpathlineto{\pgfqpoint{6.134590in}{0.910691in}}%
\pgfpathlineto{\pgfqpoint{6.009615in}{0.910691in}}%
\pgfpathlineto{\pgfqpoint{5.884640in}{0.910691in}}%
\pgfpathlineto{\pgfqpoint{5.759665in}{0.910691in}}%
\pgfpathlineto{\pgfqpoint{5.634691in}{0.910691in}}%
\pgfpathlineto{\pgfqpoint{5.509716in}{0.910691in}}%
\pgfpathlineto{\pgfqpoint{5.384741in}{0.910691in}}%
\pgfpathlineto{\pgfqpoint{5.259766in}{0.910691in}}%
\pgfpathlineto{\pgfqpoint{5.134791in}{0.910691in}}%
\pgfpathlineto{\pgfqpoint{5.009817in}{0.910691in}}%
\pgfpathlineto{\pgfqpoint{4.884842in}{0.910691in}}%
\pgfpathlineto{\pgfqpoint{4.759867in}{0.910691in}}%
\pgfpathlineto{\pgfqpoint{4.572405in}{0.910691in}}%
\pgfpathclose%
\pgfusepath{stroke,fill}%
\end{pgfscope}%
\begin{pgfscope}%
\pgfpathrectangle{\pgfqpoint{4.381818in}{0.500000in}}{\pgfqpoint{2.818182in}{3.020000in}}%
\pgfusepath{clip}%
\pgfsetbuttcap%
\pgfsetroundjoin%
\pgfsetlinewidth{1.505625pt}%
\definecolor{currentstroke}{rgb}{0.000000,0.000000,0.000000}%
\pgfsetstrokecolor{currentstroke}%
\pgfsetdash{}{0pt}%
\pgfpathmoveto{\pgfqpoint{4.509917in}{0.779291in}}%
\pgfpathlineto{\pgfqpoint{4.509917in}{3.382727in}}%
\pgfusepath{stroke}%
\end{pgfscope}%
\begin{pgfscope}%
\pgfpathrectangle{\pgfqpoint{4.381818in}{0.500000in}}{\pgfqpoint{2.818182in}{3.020000in}}%
\pgfusepath{clip}%
\pgfsetbuttcap%
\pgfsetroundjoin%
\pgfsetlinewidth{1.505625pt}%
\definecolor{currentstroke}{rgb}{0.000000,0.000000,0.000000}%
\pgfsetstrokecolor{currentstroke}%
\pgfsetdash{}{0pt}%
\pgfpathmoveto{\pgfqpoint{4.634892in}{0.779291in}}%
\pgfpathlineto{\pgfqpoint{4.634892in}{0.726188in}}%
\pgfusepath{stroke}%
\end{pgfscope}%
\begin{pgfscope}%
\pgfpathrectangle{\pgfqpoint{4.381818in}{0.500000in}}{\pgfqpoint{2.818182in}{3.020000in}}%
\pgfusepath{clip}%
\pgfsetbuttcap%
\pgfsetroundjoin%
\pgfsetlinewidth{1.505625pt}%
\definecolor{currentstroke}{rgb}{0.000000,0.000000,0.000000}%
\pgfsetstrokecolor{currentstroke}%
\pgfsetdash{}{0pt}%
\pgfpathmoveto{\pgfqpoint{4.759867in}{0.779291in}}%
\pgfpathlineto{\pgfqpoint{4.759867in}{0.707233in}}%
\pgfusepath{stroke}%
\end{pgfscope}%
\begin{pgfscope}%
\pgfpathrectangle{\pgfqpoint{4.381818in}{0.500000in}}{\pgfqpoint{2.818182in}{3.020000in}}%
\pgfusepath{clip}%
\pgfsetbuttcap%
\pgfsetroundjoin%
\pgfsetlinewidth{1.505625pt}%
\definecolor{currentstroke}{rgb}{0.000000,0.000000,0.000000}%
\pgfsetstrokecolor{currentstroke}%
\pgfsetdash{}{0pt}%
\pgfpathmoveto{\pgfqpoint{4.884842in}{0.779291in}}%
\pgfpathlineto{\pgfqpoint{4.884842in}{0.736350in}}%
\pgfusepath{stroke}%
\end{pgfscope}%
\begin{pgfscope}%
\pgfpathrectangle{\pgfqpoint{4.381818in}{0.500000in}}{\pgfqpoint{2.818182in}{3.020000in}}%
\pgfusepath{clip}%
\pgfsetbuttcap%
\pgfsetroundjoin%
\pgfsetlinewidth{1.505625pt}%
\definecolor{currentstroke}{rgb}{0.000000,0.000000,0.000000}%
\pgfsetstrokecolor{currentstroke}%
\pgfsetdash{}{0pt}%
\pgfpathmoveto{\pgfqpoint{5.009817in}{0.779291in}}%
\pgfpathlineto{\pgfqpoint{5.009817in}{0.637273in}}%
\pgfusepath{stroke}%
\end{pgfscope}%
\begin{pgfscope}%
\pgfpathrectangle{\pgfqpoint{4.381818in}{0.500000in}}{\pgfqpoint{2.818182in}{3.020000in}}%
\pgfusepath{clip}%
\pgfsetbuttcap%
\pgfsetroundjoin%
\pgfsetlinewidth{1.505625pt}%
\definecolor{currentstroke}{rgb}{0.000000,0.000000,0.000000}%
\pgfsetstrokecolor{currentstroke}%
\pgfsetdash{}{0pt}%
\pgfpathmoveto{\pgfqpoint{5.134791in}{0.779291in}}%
\pgfpathlineto{\pgfqpoint{5.134791in}{0.871271in}}%
\pgfusepath{stroke}%
\end{pgfscope}%
\begin{pgfscope}%
\pgfpathrectangle{\pgfqpoint{4.381818in}{0.500000in}}{\pgfqpoint{2.818182in}{3.020000in}}%
\pgfusepath{clip}%
\pgfsetbuttcap%
\pgfsetroundjoin%
\pgfsetlinewidth{1.505625pt}%
\definecolor{currentstroke}{rgb}{0.000000,0.000000,0.000000}%
\pgfsetstrokecolor{currentstroke}%
\pgfsetdash{}{0pt}%
\pgfpathmoveto{\pgfqpoint{5.259766in}{0.779291in}}%
\pgfpathlineto{\pgfqpoint{5.259766in}{0.816075in}}%
\pgfusepath{stroke}%
\end{pgfscope}%
\begin{pgfscope}%
\pgfpathrectangle{\pgfqpoint{4.381818in}{0.500000in}}{\pgfqpoint{2.818182in}{3.020000in}}%
\pgfusepath{clip}%
\pgfsetbuttcap%
\pgfsetroundjoin%
\pgfsetlinewidth{1.505625pt}%
\definecolor{currentstroke}{rgb}{0.000000,0.000000,0.000000}%
\pgfsetstrokecolor{currentstroke}%
\pgfsetdash{}{0pt}%
\pgfpathmoveto{\pgfqpoint{5.384741in}{0.779291in}}%
\pgfpathlineto{\pgfqpoint{5.384741in}{0.808913in}}%
\pgfusepath{stroke}%
\end{pgfscope}%
\begin{pgfscope}%
\pgfpathrectangle{\pgfqpoint{4.381818in}{0.500000in}}{\pgfqpoint{2.818182in}{3.020000in}}%
\pgfusepath{clip}%
\pgfsetbuttcap%
\pgfsetroundjoin%
\pgfsetlinewidth{1.505625pt}%
\definecolor{currentstroke}{rgb}{0.000000,0.000000,0.000000}%
\pgfsetstrokecolor{currentstroke}%
\pgfsetdash{}{0pt}%
\pgfpathmoveto{\pgfqpoint{5.509716in}{0.779291in}}%
\pgfpathlineto{\pgfqpoint{5.509716in}{0.792398in}}%
\pgfusepath{stroke}%
\end{pgfscope}%
\begin{pgfscope}%
\pgfpathrectangle{\pgfqpoint{4.381818in}{0.500000in}}{\pgfqpoint{2.818182in}{3.020000in}}%
\pgfusepath{clip}%
\pgfsetbuttcap%
\pgfsetroundjoin%
\pgfsetlinewidth{1.505625pt}%
\definecolor{currentstroke}{rgb}{0.000000,0.000000,0.000000}%
\pgfsetstrokecolor{currentstroke}%
\pgfsetdash{}{0pt}%
\pgfpathmoveto{\pgfqpoint{5.634691in}{0.779291in}}%
\pgfpathlineto{\pgfqpoint{5.634691in}{0.726402in}}%
\pgfusepath{stroke}%
\end{pgfscope}%
\begin{pgfscope}%
\pgfpathrectangle{\pgfqpoint{4.381818in}{0.500000in}}{\pgfqpoint{2.818182in}{3.020000in}}%
\pgfusepath{clip}%
\pgfsetbuttcap%
\pgfsetroundjoin%
\pgfsetlinewidth{1.505625pt}%
\definecolor{currentstroke}{rgb}{0.000000,0.000000,0.000000}%
\pgfsetstrokecolor{currentstroke}%
\pgfsetdash{}{0pt}%
\pgfpathmoveto{\pgfqpoint{5.759665in}{0.779291in}}%
\pgfpathlineto{\pgfqpoint{5.759665in}{0.712275in}}%
\pgfusepath{stroke}%
\end{pgfscope}%
\begin{pgfscope}%
\pgfpathrectangle{\pgfqpoint{4.381818in}{0.500000in}}{\pgfqpoint{2.818182in}{3.020000in}}%
\pgfusepath{clip}%
\pgfsetbuttcap%
\pgfsetroundjoin%
\pgfsetlinewidth{1.505625pt}%
\definecolor{currentstroke}{rgb}{0.000000,0.000000,0.000000}%
\pgfsetstrokecolor{currentstroke}%
\pgfsetdash{}{0pt}%
\pgfpathmoveto{\pgfqpoint{5.884640in}{0.779291in}}%
\pgfpathlineto{\pgfqpoint{5.884640in}{0.774611in}}%
\pgfusepath{stroke}%
\end{pgfscope}%
\begin{pgfscope}%
\pgfpathrectangle{\pgfqpoint{4.381818in}{0.500000in}}{\pgfqpoint{2.818182in}{3.020000in}}%
\pgfusepath{clip}%
\pgfsetbuttcap%
\pgfsetroundjoin%
\pgfsetlinewidth{1.505625pt}%
\definecolor{currentstroke}{rgb}{0.000000,0.000000,0.000000}%
\pgfsetstrokecolor{currentstroke}%
\pgfsetdash{}{0pt}%
\pgfpathmoveto{\pgfqpoint{6.009615in}{0.779291in}}%
\pgfpathlineto{\pgfqpoint{6.009615in}{0.767677in}}%
\pgfusepath{stroke}%
\end{pgfscope}%
\begin{pgfscope}%
\pgfpathrectangle{\pgfqpoint{4.381818in}{0.500000in}}{\pgfqpoint{2.818182in}{3.020000in}}%
\pgfusepath{clip}%
\pgfsetbuttcap%
\pgfsetroundjoin%
\pgfsetlinewidth{1.505625pt}%
\definecolor{currentstroke}{rgb}{0.000000,0.000000,0.000000}%
\pgfsetstrokecolor{currentstroke}%
\pgfsetdash{}{0pt}%
\pgfpathmoveto{\pgfqpoint{6.134590in}{0.779291in}}%
\pgfpathlineto{\pgfqpoint{6.134590in}{0.752214in}}%
\pgfusepath{stroke}%
\end{pgfscope}%
\begin{pgfscope}%
\pgfpathrectangle{\pgfqpoint{4.381818in}{0.500000in}}{\pgfqpoint{2.818182in}{3.020000in}}%
\pgfusepath{clip}%
\pgfsetbuttcap%
\pgfsetroundjoin%
\pgfsetlinewidth{1.505625pt}%
\definecolor{currentstroke}{rgb}{0.000000,0.000000,0.000000}%
\pgfsetstrokecolor{currentstroke}%
\pgfsetdash{}{0pt}%
\pgfpathmoveto{\pgfqpoint{6.259565in}{0.779291in}}%
\pgfpathlineto{\pgfqpoint{6.259565in}{0.772252in}}%
\pgfusepath{stroke}%
\end{pgfscope}%
\begin{pgfscope}%
\pgfpathrectangle{\pgfqpoint{4.381818in}{0.500000in}}{\pgfqpoint{2.818182in}{3.020000in}}%
\pgfusepath{clip}%
\pgfsetbuttcap%
\pgfsetroundjoin%
\pgfsetlinewidth{1.505625pt}%
\definecolor{currentstroke}{rgb}{0.000000,0.000000,0.000000}%
\pgfsetstrokecolor{currentstroke}%
\pgfsetdash{}{0pt}%
\pgfpathmoveto{\pgfqpoint{6.384539in}{0.779291in}}%
\pgfpathlineto{\pgfqpoint{6.384539in}{0.710284in}}%
\pgfusepath{stroke}%
\end{pgfscope}%
\begin{pgfscope}%
\pgfpathrectangle{\pgfqpoint{4.381818in}{0.500000in}}{\pgfqpoint{2.818182in}{3.020000in}}%
\pgfusepath{clip}%
\pgfsetbuttcap%
\pgfsetroundjoin%
\pgfsetlinewidth{1.505625pt}%
\definecolor{currentstroke}{rgb}{0.000000,0.000000,0.000000}%
\pgfsetstrokecolor{currentstroke}%
\pgfsetdash{}{0pt}%
\pgfpathmoveto{\pgfqpoint{6.509514in}{0.779291in}}%
\pgfpathlineto{\pgfqpoint{6.509514in}{0.877369in}}%
\pgfusepath{stroke}%
\end{pgfscope}%
\begin{pgfscope}%
\pgfpathrectangle{\pgfqpoint{4.381818in}{0.500000in}}{\pgfqpoint{2.818182in}{3.020000in}}%
\pgfusepath{clip}%
\pgfsetbuttcap%
\pgfsetroundjoin%
\pgfsetlinewidth{1.505625pt}%
\definecolor{currentstroke}{rgb}{0.000000,0.000000,0.000000}%
\pgfsetstrokecolor{currentstroke}%
\pgfsetdash{}{0pt}%
\pgfpathmoveto{\pgfqpoint{6.634489in}{0.779291in}}%
\pgfpathlineto{\pgfqpoint{6.634489in}{0.831358in}}%
\pgfusepath{stroke}%
\end{pgfscope}%
\begin{pgfscope}%
\pgfpathrectangle{\pgfqpoint{4.381818in}{0.500000in}}{\pgfqpoint{2.818182in}{3.020000in}}%
\pgfusepath{clip}%
\pgfsetbuttcap%
\pgfsetroundjoin%
\pgfsetlinewidth{1.505625pt}%
\definecolor{currentstroke}{rgb}{0.000000,0.000000,0.000000}%
\pgfsetstrokecolor{currentstroke}%
\pgfsetdash{}{0pt}%
\pgfpathmoveto{\pgfqpoint{6.759464in}{0.779291in}}%
\pgfpathlineto{\pgfqpoint{6.759464in}{0.795897in}}%
\pgfusepath{stroke}%
\end{pgfscope}%
\begin{pgfscope}%
\pgfpathrectangle{\pgfqpoint{4.381818in}{0.500000in}}{\pgfqpoint{2.818182in}{3.020000in}}%
\pgfusepath{clip}%
\pgfsetbuttcap%
\pgfsetroundjoin%
\pgfsetlinewidth{1.505625pt}%
\definecolor{currentstroke}{rgb}{0.000000,0.000000,0.000000}%
\pgfsetstrokecolor{currentstroke}%
\pgfsetdash{}{0pt}%
\pgfpathmoveto{\pgfqpoint{6.884439in}{0.779291in}}%
\pgfpathlineto{\pgfqpoint{6.884439in}{0.792836in}}%
\pgfusepath{stroke}%
\end{pgfscope}%
\begin{pgfscope}%
\pgfpathrectangle{\pgfqpoint{4.381818in}{0.500000in}}{\pgfqpoint{2.818182in}{3.020000in}}%
\pgfusepath{clip}%
\pgfsetbuttcap%
\pgfsetroundjoin%
\pgfsetlinewidth{1.505625pt}%
\definecolor{currentstroke}{rgb}{0.000000,0.000000,0.000000}%
\pgfsetstrokecolor{currentstroke}%
\pgfsetdash{}{0pt}%
\pgfpathmoveto{\pgfqpoint{7.009413in}{0.779291in}}%
\pgfpathlineto{\pgfqpoint{7.009413in}{0.743135in}}%
\pgfusepath{stroke}%
\end{pgfscope}%
\begin{pgfscope}%
\pgfpathrectangle{\pgfqpoint{4.381818in}{0.500000in}}{\pgfqpoint{2.818182in}{3.020000in}}%
\pgfusepath{clip}%
\pgfsetroundcap%
\pgfsetroundjoin%
\pgfsetlinewidth{1.505625pt}%
\definecolor{currentstroke}{rgb}{0.549020,0.337255,0.294118}%
\pgfsetstrokecolor{currentstroke}%
\pgfsetdash{}{0pt}%
\pgfpathmoveto{\pgfqpoint{4.381818in}{0.779291in}}%
\pgfpathlineto{\pgfqpoint{7.200000in}{0.779291in}}%
\pgfusepath{stroke}%
\end{pgfscope}%
\begin{pgfscope}%
\pgfpathrectangle{\pgfqpoint{4.381818in}{0.500000in}}{\pgfqpoint{2.818182in}{3.020000in}}%
\pgfusepath{clip}%
\pgfsetbuttcap%
\pgfsetroundjoin%
\definecolor{currentfill}{rgb}{0.549020,0.337255,0.294118}%
\pgfsetfillcolor{currentfill}%
\pgfsetlinewidth{1.003750pt}%
\definecolor{currentstroke}{rgb}{0.549020,0.337255,0.294118}%
\pgfsetstrokecolor{currentstroke}%
\pgfsetdash{}{0pt}%
\pgfsys@defobject{currentmarker}{\pgfqpoint{-0.034722in}{-0.034722in}}{\pgfqpoint{0.034722in}{0.034722in}}{%
\pgfpathmoveto{\pgfqpoint{0.000000in}{-0.034722in}}%
\pgfpathcurveto{\pgfqpoint{0.009208in}{-0.034722in}}{\pgfqpoint{0.018041in}{-0.031064in}}{\pgfqpoint{0.024552in}{-0.024552in}}%
\pgfpathcurveto{\pgfqpoint{0.031064in}{-0.018041in}}{\pgfqpoint{0.034722in}{-0.009208in}}{\pgfqpoint{0.034722in}{0.000000in}}%
\pgfpathcurveto{\pgfqpoint{0.034722in}{0.009208in}}{\pgfqpoint{0.031064in}{0.018041in}}{\pgfqpoint{0.024552in}{0.024552in}}%
\pgfpathcurveto{\pgfqpoint{0.018041in}{0.031064in}}{\pgfqpoint{0.009208in}{0.034722in}}{\pgfqpoint{0.000000in}{0.034722in}}%
\pgfpathcurveto{\pgfqpoint{-0.009208in}{0.034722in}}{\pgfqpoint{-0.018041in}{0.031064in}}{\pgfqpoint{-0.024552in}{0.024552in}}%
\pgfpathcurveto{\pgfqpoint{-0.031064in}{0.018041in}}{\pgfqpoint{-0.034722in}{0.009208in}}{\pgfqpoint{-0.034722in}{0.000000in}}%
\pgfpathcurveto{\pgfqpoint{-0.034722in}{-0.009208in}}{\pgfqpoint{-0.031064in}{-0.018041in}}{\pgfqpoint{-0.024552in}{-0.024552in}}%
\pgfpathcurveto{\pgfqpoint{-0.018041in}{-0.031064in}}{\pgfqpoint{-0.009208in}{-0.034722in}}{\pgfqpoint{0.000000in}{-0.034722in}}%
\pgfpathclose%
\pgfusepath{stroke,fill}%
}%
\begin{pgfscope}%
\pgfsys@transformshift{4.509917in}{3.382727in}%
\pgfsys@useobject{currentmarker}{}%
\end{pgfscope}%
\begin{pgfscope}%
\pgfsys@transformshift{4.634892in}{0.726188in}%
\pgfsys@useobject{currentmarker}{}%
\end{pgfscope}%
\begin{pgfscope}%
\pgfsys@transformshift{4.759867in}{0.707233in}%
\pgfsys@useobject{currentmarker}{}%
\end{pgfscope}%
\begin{pgfscope}%
\pgfsys@transformshift{4.884842in}{0.736350in}%
\pgfsys@useobject{currentmarker}{}%
\end{pgfscope}%
\begin{pgfscope}%
\pgfsys@transformshift{5.009817in}{0.637273in}%
\pgfsys@useobject{currentmarker}{}%
\end{pgfscope}%
\begin{pgfscope}%
\pgfsys@transformshift{5.134791in}{0.871271in}%
\pgfsys@useobject{currentmarker}{}%
\end{pgfscope}%
\begin{pgfscope}%
\pgfsys@transformshift{5.259766in}{0.816075in}%
\pgfsys@useobject{currentmarker}{}%
\end{pgfscope}%
\begin{pgfscope}%
\pgfsys@transformshift{5.384741in}{0.808913in}%
\pgfsys@useobject{currentmarker}{}%
\end{pgfscope}%
\begin{pgfscope}%
\pgfsys@transformshift{5.509716in}{0.792398in}%
\pgfsys@useobject{currentmarker}{}%
\end{pgfscope}%
\begin{pgfscope}%
\pgfsys@transformshift{5.634691in}{0.726402in}%
\pgfsys@useobject{currentmarker}{}%
\end{pgfscope}%
\begin{pgfscope}%
\pgfsys@transformshift{5.759665in}{0.712275in}%
\pgfsys@useobject{currentmarker}{}%
\end{pgfscope}%
\begin{pgfscope}%
\pgfsys@transformshift{5.884640in}{0.774611in}%
\pgfsys@useobject{currentmarker}{}%
\end{pgfscope}%
\begin{pgfscope}%
\pgfsys@transformshift{6.009615in}{0.767677in}%
\pgfsys@useobject{currentmarker}{}%
\end{pgfscope}%
\begin{pgfscope}%
\pgfsys@transformshift{6.134590in}{0.752214in}%
\pgfsys@useobject{currentmarker}{}%
\end{pgfscope}%
\begin{pgfscope}%
\pgfsys@transformshift{6.259565in}{0.772252in}%
\pgfsys@useobject{currentmarker}{}%
\end{pgfscope}%
\begin{pgfscope}%
\pgfsys@transformshift{6.384539in}{0.710284in}%
\pgfsys@useobject{currentmarker}{}%
\end{pgfscope}%
\begin{pgfscope}%
\pgfsys@transformshift{6.509514in}{0.877369in}%
\pgfsys@useobject{currentmarker}{}%
\end{pgfscope}%
\begin{pgfscope}%
\pgfsys@transformshift{6.634489in}{0.831358in}%
\pgfsys@useobject{currentmarker}{}%
\end{pgfscope}%
\begin{pgfscope}%
\pgfsys@transformshift{6.759464in}{0.795897in}%
\pgfsys@useobject{currentmarker}{}%
\end{pgfscope}%
\begin{pgfscope}%
\pgfsys@transformshift{6.884439in}{0.792836in}%
\pgfsys@useobject{currentmarker}{}%
\end{pgfscope}%
\begin{pgfscope}%
\pgfsys@transformshift{7.009413in}{0.743135in}%
\pgfsys@useobject{currentmarker}{}%
\end{pgfscope}%
\end{pgfscope}%
\begin{pgfscope}%
\pgfsetrectcap%
\pgfsetmiterjoin%
\pgfsetlinewidth{0.803000pt}%
\definecolor{currentstroke}{rgb}{1.000000,1.000000,1.000000}%
\pgfsetstrokecolor{currentstroke}%
\pgfsetdash{}{0pt}%
\pgfpathmoveto{\pgfqpoint{4.381818in}{0.500000in}}%
\pgfpathlineto{\pgfqpoint{4.381818in}{3.520000in}}%
\pgfusepath{stroke}%
\end{pgfscope}%
\begin{pgfscope}%
\pgfsetrectcap%
\pgfsetmiterjoin%
\pgfsetlinewidth{0.803000pt}%
\definecolor{currentstroke}{rgb}{1.000000,1.000000,1.000000}%
\pgfsetstrokecolor{currentstroke}%
\pgfsetdash{}{0pt}%
\pgfpathmoveto{\pgfqpoint{7.200000in}{0.500000in}}%
\pgfpathlineto{\pgfqpoint{7.200000in}{3.520000in}}%
\pgfusepath{stroke}%
\end{pgfscope}%
\begin{pgfscope}%
\pgfsetrectcap%
\pgfsetmiterjoin%
\pgfsetlinewidth{0.803000pt}%
\definecolor{currentstroke}{rgb}{1.000000,1.000000,1.000000}%
\pgfsetstrokecolor{currentstroke}%
\pgfsetdash{}{0pt}%
\pgfpathmoveto{\pgfqpoint{4.381818in}{0.500000in}}%
\pgfpathlineto{\pgfqpoint{7.200000in}{0.500000in}}%
\pgfusepath{stroke}%
\end{pgfscope}%
\begin{pgfscope}%
\pgfsetrectcap%
\pgfsetmiterjoin%
\pgfsetlinewidth{0.803000pt}%
\definecolor{currentstroke}{rgb}{1.000000,1.000000,1.000000}%
\pgfsetstrokecolor{currentstroke}%
\pgfsetdash{}{0pt}%
\pgfpathmoveto{\pgfqpoint{4.381818in}{3.520000in}}%
\pgfpathlineto{\pgfqpoint{7.200000in}{3.520000in}}%
\pgfusepath{stroke}%
\end{pgfscope}%
\begin{pgfscope}%
\definecolor{textcolor}{rgb}{0.150000,0.150000,0.150000}%
\pgfsetstrokecolor{textcolor}%
\pgfsetfillcolor{textcolor}%
\pgftext[x=5.790909in,y=3.603333in,,base]{\color{textcolor}\rmfamily\fontsize{16.800000}{20.160000}\selectfont Partial Autocorrelation}%
\end{pgfscope}%
\end{pgfpicture}%
\makeatother%
\endgroup%

    \end{adjustbox}
    \caption{Autocorrelation and partial autocorrelation for the log-returns of stock PG (For convenience, only one stock is shown. ACF and PACF for other stocks can be seen in the appendix in figure \ref{fig:all_autocorr_fd_log_adjclose}).}
    \label{fig:PG_autocorr_fd_log_adjclose}
\end{figure}{}

%Plot needed?: calculate returns and plot variance of returns. 







The data is, however, not quite normally distributed. While the mean of the log returns is close to zero, we can see by looking at figure \ref{fig:PG_qq_fd_log_adjclose} that the distribution has fat tails: extrem events appear more often than would be expected if the data was normally distributed. 

\begin{figure}[h]
    \centering
    \figuretitle{QQ-plot of log-returns of stock PG}
    \begin{adjustbox}{width=.9\textwidth,center}
    \includegraphics[]{figures/PG_log_adjclose_fd_and_qq.pdf}
    %%% Creator: Matplotlib, PGF backend
%%
%% To include the figure in your LaTeX document, write
%%   \input{<filename>.pgf}
%%
%% Make sure the required packages are loaded in your preamble
%%   \usepackage{pgf}
%%
%% Figures using additional raster images can only be included by \input if
%% they are in the same directory as the main LaTeX file. For loading figures
%% from other directories you can use the `import` package
%%   \usepackage{import}
%% and then include the figures with
%%   \import{<path to file>}{<filename>.pgf}
%%
%% Matplotlib used the following preamble
%%   \usepackage{fontspec}
%%   \setmainfont{DejaVuSerif.ttf}[Path=/opt/tljh/user/lib/python3.6/site-packages/matplotlib/mpl-data/fonts/ttf/]
%%   \setsansfont{DejaVuSans.ttf}[Path=/opt/tljh/user/lib/python3.6/site-packages/matplotlib/mpl-data/fonts/ttf/]
%%   \setmonofont{DejaVuSansMono.ttf}[Path=/opt/tljh/user/lib/python3.6/site-packages/matplotlib/mpl-data/fonts/ttf/]
%%
\begingroup%
\makeatletter%
\begin{pgfpicture}%
\pgfpathrectangle{\pgfpointorigin}{\pgfqpoint{17.000000in}{8.000000in}}%
\pgfusepath{use as bounding box, clip}%
\begin{pgfscope}%
\pgfsetbuttcap%
\pgfsetmiterjoin%
\definecolor{currentfill}{rgb}{1.000000,1.000000,1.000000}%
\pgfsetfillcolor{currentfill}%
\pgfsetlinewidth{0.000000pt}%
\definecolor{currentstroke}{rgb}{1.000000,1.000000,1.000000}%
\pgfsetstrokecolor{currentstroke}%
\pgfsetdash{}{0pt}%
\pgfpathmoveto{\pgfqpoint{0.000000in}{0.000000in}}%
\pgfpathlineto{\pgfqpoint{17.000000in}{0.000000in}}%
\pgfpathlineto{\pgfqpoint{17.000000in}{8.000000in}}%
\pgfpathlineto{\pgfqpoint{0.000000in}{8.000000in}}%
\pgfpathclose%
\pgfusepath{fill}%
\end{pgfscope}%
\begin{pgfscope}%
\pgfsetbuttcap%
\pgfsetmiterjoin%
\definecolor{currentfill}{rgb}{0.917647,0.917647,0.949020}%
\pgfsetfillcolor{currentfill}%
\pgfsetlinewidth{0.000000pt}%
\definecolor{currentstroke}{rgb}{0.000000,0.000000,0.000000}%
\pgfsetstrokecolor{currentstroke}%
\pgfsetstrokeopacity{0.000000}%
\pgfsetdash{}{0pt}%
\pgfpathmoveto{\pgfqpoint{2.125000in}{4.882857in}}%
\pgfpathlineto{\pgfqpoint{15.300000in}{4.882857in}}%
\pgfpathlineto{\pgfqpoint{15.300000in}{7.040000in}}%
\pgfpathlineto{\pgfqpoint{2.125000in}{7.040000in}}%
\pgfpathclose%
\pgfusepath{fill}%
\end{pgfscope}%
\begin{pgfscope}%
\pgfpathrectangle{\pgfqpoint{2.125000in}{4.882857in}}{\pgfqpoint{13.175000in}{2.157143in}}%
\pgfusepath{clip}%
\pgfsetroundcap%
\pgfsetroundjoin%
\pgfsetlinewidth{0.803000pt}%
\definecolor{currentstroke}{rgb}{1.000000,1.000000,1.000000}%
\pgfsetstrokecolor{currentstroke}%
\pgfsetdash{}{0pt}%
\pgfpathmoveto{\pgfqpoint{2.793535in}{4.882857in}}%
\pgfpathlineto{\pgfqpoint{2.793535in}{7.040000in}}%
\pgfusepath{stroke}%
\end{pgfscope}%
\begin{pgfscope}%
\definecolor{textcolor}{rgb}{0.150000,0.150000,0.150000}%
\pgfsetstrokecolor{textcolor}%
\pgfsetfillcolor{textcolor}%
\pgftext[x=2.793535in,y=4.785635in,,top]{\color{textcolor}\rmfamily\fontsize{10.000000}{12.000000}\selectfont −3}%
\end{pgfscope}%
\begin{pgfscope}%
\pgfpathrectangle{\pgfqpoint{2.125000in}{4.882857in}}{\pgfqpoint{13.175000in}{2.157143in}}%
\pgfusepath{clip}%
\pgfsetroundcap%
\pgfsetroundjoin%
\pgfsetlinewidth{0.803000pt}%
\definecolor{currentstroke}{rgb}{1.000000,1.000000,1.000000}%
\pgfsetstrokecolor{currentstroke}%
\pgfsetdash{}{0pt}%
\pgfpathmoveto{\pgfqpoint{4.766523in}{4.882857in}}%
\pgfpathlineto{\pgfqpoint{4.766523in}{7.040000in}}%
\pgfusepath{stroke}%
\end{pgfscope}%
\begin{pgfscope}%
\definecolor{textcolor}{rgb}{0.150000,0.150000,0.150000}%
\pgfsetstrokecolor{textcolor}%
\pgfsetfillcolor{textcolor}%
\pgftext[x=4.766523in,y=4.785635in,,top]{\color{textcolor}\rmfamily\fontsize{10.000000}{12.000000}\selectfont −2}%
\end{pgfscope}%
\begin{pgfscope}%
\pgfpathrectangle{\pgfqpoint{2.125000in}{4.882857in}}{\pgfqpoint{13.175000in}{2.157143in}}%
\pgfusepath{clip}%
\pgfsetroundcap%
\pgfsetroundjoin%
\pgfsetlinewidth{0.803000pt}%
\definecolor{currentstroke}{rgb}{1.000000,1.000000,1.000000}%
\pgfsetstrokecolor{currentstroke}%
\pgfsetdash{}{0pt}%
\pgfpathmoveto{\pgfqpoint{6.739512in}{4.882857in}}%
\pgfpathlineto{\pgfqpoint{6.739512in}{7.040000in}}%
\pgfusepath{stroke}%
\end{pgfscope}%
\begin{pgfscope}%
\definecolor{textcolor}{rgb}{0.150000,0.150000,0.150000}%
\pgfsetstrokecolor{textcolor}%
\pgfsetfillcolor{textcolor}%
\pgftext[x=6.739512in,y=4.785635in,,top]{\color{textcolor}\rmfamily\fontsize{10.000000}{12.000000}\selectfont −1}%
\end{pgfscope}%
\begin{pgfscope}%
\pgfpathrectangle{\pgfqpoint{2.125000in}{4.882857in}}{\pgfqpoint{13.175000in}{2.157143in}}%
\pgfusepath{clip}%
\pgfsetroundcap%
\pgfsetroundjoin%
\pgfsetlinewidth{0.803000pt}%
\definecolor{currentstroke}{rgb}{1.000000,1.000000,1.000000}%
\pgfsetstrokecolor{currentstroke}%
\pgfsetdash{}{0pt}%
\pgfpathmoveto{\pgfqpoint{8.712500in}{4.882857in}}%
\pgfpathlineto{\pgfqpoint{8.712500in}{7.040000in}}%
\pgfusepath{stroke}%
\end{pgfscope}%
\begin{pgfscope}%
\definecolor{textcolor}{rgb}{0.150000,0.150000,0.150000}%
\pgfsetstrokecolor{textcolor}%
\pgfsetfillcolor{textcolor}%
\pgftext[x=8.712500in,y=4.785635in,,top]{\color{textcolor}\rmfamily\fontsize{10.000000}{12.000000}\selectfont 0}%
\end{pgfscope}%
\begin{pgfscope}%
\pgfpathrectangle{\pgfqpoint{2.125000in}{4.882857in}}{\pgfqpoint{13.175000in}{2.157143in}}%
\pgfusepath{clip}%
\pgfsetroundcap%
\pgfsetroundjoin%
\pgfsetlinewidth{0.803000pt}%
\definecolor{currentstroke}{rgb}{1.000000,1.000000,1.000000}%
\pgfsetstrokecolor{currentstroke}%
\pgfsetdash{}{0pt}%
\pgfpathmoveto{\pgfqpoint{10.685488in}{4.882857in}}%
\pgfpathlineto{\pgfqpoint{10.685488in}{7.040000in}}%
\pgfusepath{stroke}%
\end{pgfscope}%
\begin{pgfscope}%
\definecolor{textcolor}{rgb}{0.150000,0.150000,0.150000}%
\pgfsetstrokecolor{textcolor}%
\pgfsetfillcolor{textcolor}%
\pgftext[x=10.685488in,y=4.785635in,,top]{\color{textcolor}\rmfamily\fontsize{10.000000}{12.000000}\selectfont 1}%
\end{pgfscope}%
\begin{pgfscope}%
\pgfpathrectangle{\pgfqpoint{2.125000in}{4.882857in}}{\pgfqpoint{13.175000in}{2.157143in}}%
\pgfusepath{clip}%
\pgfsetroundcap%
\pgfsetroundjoin%
\pgfsetlinewidth{0.803000pt}%
\definecolor{currentstroke}{rgb}{1.000000,1.000000,1.000000}%
\pgfsetstrokecolor{currentstroke}%
\pgfsetdash{}{0pt}%
\pgfpathmoveto{\pgfqpoint{12.658477in}{4.882857in}}%
\pgfpathlineto{\pgfqpoint{12.658477in}{7.040000in}}%
\pgfusepath{stroke}%
\end{pgfscope}%
\begin{pgfscope}%
\definecolor{textcolor}{rgb}{0.150000,0.150000,0.150000}%
\pgfsetstrokecolor{textcolor}%
\pgfsetfillcolor{textcolor}%
\pgftext[x=12.658477in,y=4.785635in,,top]{\color{textcolor}\rmfamily\fontsize{10.000000}{12.000000}\selectfont 2}%
\end{pgfscope}%
\begin{pgfscope}%
\pgfpathrectangle{\pgfqpoint{2.125000in}{4.882857in}}{\pgfqpoint{13.175000in}{2.157143in}}%
\pgfusepath{clip}%
\pgfsetroundcap%
\pgfsetroundjoin%
\pgfsetlinewidth{0.803000pt}%
\definecolor{currentstroke}{rgb}{1.000000,1.000000,1.000000}%
\pgfsetstrokecolor{currentstroke}%
\pgfsetdash{}{0pt}%
\pgfpathmoveto{\pgfqpoint{14.631465in}{4.882857in}}%
\pgfpathlineto{\pgfqpoint{14.631465in}{7.040000in}}%
\pgfusepath{stroke}%
\end{pgfscope}%
\begin{pgfscope}%
\definecolor{textcolor}{rgb}{0.150000,0.150000,0.150000}%
\pgfsetstrokecolor{textcolor}%
\pgfsetfillcolor{textcolor}%
\pgftext[x=14.631465in,y=4.785635in,,top]{\color{textcolor}\rmfamily\fontsize{10.000000}{12.000000}\selectfont 3}%
\end{pgfscope}%
\begin{pgfscope}%
\definecolor{textcolor}{rgb}{0.150000,0.150000,0.150000}%
\pgfsetstrokecolor{textcolor}%
\pgfsetfillcolor{textcolor}%
\pgftext[x=8.712500in,y=4.595667in,,top]{\color{textcolor}\rmfamily\fontsize{10.000000}{12.000000}\selectfont Theoretical Quantiles}%
\end{pgfscope}%
\begin{pgfscope}%
\pgfpathrectangle{\pgfqpoint{2.125000in}{4.882857in}}{\pgfqpoint{13.175000in}{2.157143in}}%
\pgfusepath{clip}%
\pgfsetroundcap%
\pgfsetroundjoin%
\pgfsetlinewidth{0.803000pt}%
\definecolor{currentstroke}{rgb}{1.000000,1.000000,1.000000}%
\pgfsetstrokecolor{currentstroke}%
\pgfsetdash{}{0pt}%
\pgfpathmoveto{\pgfqpoint{2.125000in}{4.992782in}}%
\pgfpathlineto{\pgfqpoint{15.300000in}{4.992782in}}%
\pgfusepath{stroke}%
\end{pgfscope}%
\begin{pgfscope}%
\definecolor{textcolor}{rgb}{0.150000,0.150000,0.150000}%
\pgfsetstrokecolor{textcolor}%
\pgfsetfillcolor{textcolor}%
\pgftext[x=1.602159in,y=4.940021in,left,base]{\color{textcolor}\rmfamily\fontsize{10.000000}{12.000000}\selectfont −0.06}%
\end{pgfscope}%
\begin{pgfscope}%
\pgfpathrectangle{\pgfqpoint{2.125000in}{4.882857in}}{\pgfqpoint{13.175000in}{2.157143in}}%
\pgfusepath{clip}%
\pgfsetroundcap%
\pgfsetroundjoin%
\pgfsetlinewidth{0.803000pt}%
\definecolor{currentstroke}{rgb}{1.000000,1.000000,1.000000}%
\pgfsetstrokecolor{currentstroke}%
\pgfsetdash{}{0pt}%
\pgfpathmoveto{\pgfqpoint{2.125000in}{5.384533in}}%
\pgfpathlineto{\pgfqpoint{15.300000in}{5.384533in}}%
\pgfusepath{stroke}%
\end{pgfscope}%
\begin{pgfscope}%
\definecolor{textcolor}{rgb}{0.150000,0.150000,0.150000}%
\pgfsetstrokecolor{textcolor}%
\pgfsetfillcolor{textcolor}%
\pgftext[x=1.602159in,y=5.331771in,left,base]{\color{textcolor}\rmfamily\fontsize{10.000000}{12.000000}\selectfont −0.04}%
\end{pgfscope}%
\begin{pgfscope}%
\pgfpathrectangle{\pgfqpoint{2.125000in}{4.882857in}}{\pgfqpoint{13.175000in}{2.157143in}}%
\pgfusepath{clip}%
\pgfsetroundcap%
\pgfsetroundjoin%
\pgfsetlinewidth{0.803000pt}%
\definecolor{currentstroke}{rgb}{1.000000,1.000000,1.000000}%
\pgfsetstrokecolor{currentstroke}%
\pgfsetdash{}{0pt}%
\pgfpathmoveto{\pgfqpoint{2.125000in}{5.776283in}}%
\pgfpathlineto{\pgfqpoint{15.300000in}{5.776283in}}%
\pgfusepath{stroke}%
\end{pgfscope}%
\begin{pgfscope}%
\definecolor{textcolor}{rgb}{0.150000,0.150000,0.150000}%
\pgfsetstrokecolor{textcolor}%
\pgfsetfillcolor{textcolor}%
\pgftext[x=1.602159in,y=5.723522in,left,base]{\color{textcolor}\rmfamily\fontsize{10.000000}{12.000000}\selectfont −0.02}%
\end{pgfscope}%
\begin{pgfscope}%
\pgfpathrectangle{\pgfqpoint{2.125000in}{4.882857in}}{\pgfqpoint{13.175000in}{2.157143in}}%
\pgfusepath{clip}%
\pgfsetroundcap%
\pgfsetroundjoin%
\pgfsetlinewidth{0.803000pt}%
\definecolor{currentstroke}{rgb}{1.000000,1.000000,1.000000}%
\pgfsetstrokecolor{currentstroke}%
\pgfsetdash{}{0pt}%
\pgfpathmoveto{\pgfqpoint{2.125000in}{6.168034in}}%
\pgfpathlineto{\pgfqpoint{15.300000in}{6.168034in}}%
\pgfusepath{stroke}%
\end{pgfscope}%
\begin{pgfscope}%
\definecolor{textcolor}{rgb}{0.150000,0.150000,0.150000}%
\pgfsetstrokecolor{textcolor}%
\pgfsetfillcolor{textcolor}%
\pgftext[x=1.718533in,y=6.115272in,left,base]{\color{textcolor}\rmfamily\fontsize{10.000000}{12.000000}\selectfont 0.00}%
\end{pgfscope}%
\begin{pgfscope}%
\pgfpathrectangle{\pgfqpoint{2.125000in}{4.882857in}}{\pgfqpoint{13.175000in}{2.157143in}}%
\pgfusepath{clip}%
\pgfsetroundcap%
\pgfsetroundjoin%
\pgfsetlinewidth{0.803000pt}%
\definecolor{currentstroke}{rgb}{1.000000,1.000000,1.000000}%
\pgfsetstrokecolor{currentstroke}%
\pgfsetdash{}{0pt}%
\pgfpathmoveto{\pgfqpoint{2.125000in}{6.559784in}}%
\pgfpathlineto{\pgfqpoint{15.300000in}{6.559784in}}%
\pgfusepath{stroke}%
\end{pgfscope}%
\begin{pgfscope}%
\definecolor{textcolor}{rgb}{0.150000,0.150000,0.150000}%
\pgfsetstrokecolor{textcolor}%
\pgfsetfillcolor{textcolor}%
\pgftext[x=1.718533in,y=6.507023in,left,base]{\color{textcolor}\rmfamily\fontsize{10.000000}{12.000000}\selectfont 0.02}%
\end{pgfscope}%
\begin{pgfscope}%
\pgfpathrectangle{\pgfqpoint{2.125000in}{4.882857in}}{\pgfqpoint{13.175000in}{2.157143in}}%
\pgfusepath{clip}%
\pgfsetroundcap%
\pgfsetroundjoin%
\pgfsetlinewidth{0.803000pt}%
\definecolor{currentstroke}{rgb}{1.000000,1.000000,1.000000}%
\pgfsetstrokecolor{currentstroke}%
\pgfsetdash{}{0pt}%
\pgfpathmoveto{\pgfqpoint{2.125000in}{6.951535in}}%
\pgfpathlineto{\pgfqpoint{15.300000in}{6.951535in}}%
\pgfusepath{stroke}%
\end{pgfscope}%
\begin{pgfscope}%
\definecolor{textcolor}{rgb}{0.150000,0.150000,0.150000}%
\pgfsetstrokecolor{textcolor}%
\pgfsetfillcolor{textcolor}%
\pgftext[x=1.718533in,y=6.898773in,left,base]{\color{textcolor}\rmfamily\fontsize{10.000000}{12.000000}\selectfont 0.04}%
\end{pgfscope}%
\begin{pgfscope}%
\definecolor{textcolor}{rgb}{0.150000,0.150000,0.150000}%
\pgfsetstrokecolor{textcolor}%
\pgfsetfillcolor{textcolor}%
\pgftext[x=1.546604in,y=5.961429in,,bottom,rotate=90.000000]{\color{textcolor}\rmfamily\fontsize{10.000000}{12.000000}\selectfont Sample Quantiles}%
\end{pgfscope}%
\begin{pgfscope}%
\pgfpathrectangle{\pgfqpoint{2.125000in}{4.882857in}}{\pgfqpoint{13.175000in}{2.157143in}}%
\pgfusepath{clip}%
\pgfsetbuttcap%
\pgfsetroundjoin%
\definecolor{currentfill}{rgb}{0.000000,0.000000,1.000000}%
\pgfsetfillcolor{currentfill}%
\pgfsetlinewidth{1.003750pt}%
\definecolor{currentstroke}{rgb}{0.000000,0.000000,1.000000}%
\pgfsetstrokecolor{currentstroke}%
\pgfsetdash{}{0pt}%
\pgfsys@defobject{currentmarker}{\pgfqpoint{-0.041667in}{-0.041667in}}{\pgfqpoint{0.041667in}{0.041667in}}{%
\pgfpathmoveto{\pgfqpoint{0.000000in}{-0.041667in}}%
\pgfpathcurveto{\pgfqpoint{0.011050in}{-0.041667in}}{\pgfqpoint{0.021649in}{-0.037276in}}{\pgfqpoint{0.029463in}{-0.029463in}}%
\pgfpathcurveto{\pgfqpoint{0.037276in}{-0.021649in}}{\pgfqpoint{0.041667in}{-0.011050in}}{\pgfqpoint{0.041667in}{0.000000in}}%
\pgfpathcurveto{\pgfqpoint{0.041667in}{0.011050in}}{\pgfqpoint{0.037276in}{0.021649in}}{\pgfqpoint{0.029463in}{0.029463in}}%
\pgfpathcurveto{\pgfqpoint{0.021649in}{0.037276in}}{\pgfqpoint{0.011050in}{0.041667in}}{\pgfqpoint{0.000000in}{0.041667in}}%
\pgfpathcurveto{\pgfqpoint{-0.011050in}{0.041667in}}{\pgfqpoint{-0.021649in}{0.037276in}}{\pgfqpoint{-0.029463in}{0.029463in}}%
\pgfpathcurveto{\pgfqpoint{-0.037276in}{0.021649in}}{\pgfqpoint{-0.041667in}{0.011050in}}{\pgfqpoint{-0.041667in}{0.000000in}}%
\pgfpathcurveto{\pgfqpoint{-0.041667in}{-0.011050in}}{\pgfqpoint{-0.037276in}{-0.021649in}}{\pgfqpoint{-0.029463in}{-0.029463in}}%
\pgfpathcurveto{\pgfqpoint{-0.021649in}{-0.037276in}}{\pgfqpoint{-0.011050in}{-0.041667in}}{\pgfqpoint{0.000000in}{-0.041667in}}%
\pgfpathclose%
\pgfusepath{stroke,fill}%
}%
\begin{pgfscope}%
\pgfsys@transformshift{2.378365in}{4.980909in}%
\pgfsys@useobject{currentmarker}{}%
\end{pgfscope}%
\begin{pgfscope}%
\pgfsys@transformshift{2.782528in}{5.367174in}%
\pgfsys@useobject{currentmarker}{}%
\end{pgfscope}%
\begin{pgfscope}%
\pgfsys@transformshift{3.030198in}{5.417895in}%
\pgfsys@useobject{currentmarker}{}%
\end{pgfscope}%
\begin{pgfscope}%
\pgfsys@transformshift{3.211576in}{5.440027in}%
\pgfsys@useobject{currentmarker}{}%
\end{pgfscope}%
\begin{pgfscope}%
\pgfsys@transformshift{3.355770in}{5.444283in}%
\pgfsys@useobject{currentmarker}{}%
\end{pgfscope}%
\begin{pgfscope}%
\pgfsys@transformshift{3.476013in}{5.479270in}%
\pgfsys@useobject{currentmarker}{}%
\end{pgfscope}%
\begin{pgfscope}%
\pgfsys@transformshift{3.579476in}{5.481040in}%
\pgfsys@useobject{currentmarker}{}%
\end{pgfscope}%
\begin{pgfscope}%
\pgfsys@transformshift{3.670496in}{5.563001in}%
\pgfsys@useobject{currentmarker}{}%
\end{pgfscope}%
\begin{pgfscope}%
\pgfsys@transformshift{3.751905in}{5.580653in}%
\pgfsys@useobject{currentmarker}{}%
\end{pgfscope}%
\begin{pgfscope}%
\pgfsys@transformshift{3.825653in}{5.584719in}%
\pgfsys@useobject{currentmarker}{}%
\end{pgfscope}%
\begin{pgfscope}%
\pgfsys@transformshift{3.893147in}{5.609480in}%
\pgfsys@useobject{currentmarker}{}%
\end{pgfscope}%
\begin{pgfscope}%
\pgfsys@transformshift{3.955431in}{5.633858in}%
\pgfsys@useobject{currentmarker}{}%
\end{pgfscope}%
\begin{pgfscope}%
\pgfsys@transformshift{4.013308in}{5.669441in}%
\pgfsys@useobject{currentmarker}{}%
\end{pgfscope}%
\begin{pgfscope}%
\pgfsys@transformshift{4.067403in}{5.670355in}%
\pgfsys@useobject{currentmarker}{}%
\end{pgfscope}%
\begin{pgfscope}%
\pgfsys@transformshift{4.118216in}{5.687079in}%
\pgfsys@useobject{currentmarker}{}%
\end{pgfscope}%
\begin{pgfscope}%
\pgfsys@transformshift{4.166153in}{5.705738in}%
\pgfsys@useobject{currentmarker}{}%
\end{pgfscope}%
\begin{pgfscope}%
\pgfsys@transformshift{4.211547in}{5.707008in}%
\pgfsys@useobject{currentmarker}{}%
\end{pgfscope}%
\begin{pgfscope}%
\pgfsys@transformshift{4.254677in}{5.715324in}%
\pgfsys@useobject{currentmarker}{}%
\end{pgfscope}%
\begin{pgfscope}%
\pgfsys@transformshift{4.295777in}{5.743070in}%
\pgfsys@useobject{currentmarker}{}%
\end{pgfscope}%
\begin{pgfscope}%
\pgfsys@transformshift{4.335044in}{5.750890in}%
\pgfsys@useobject{currentmarker}{}%
\end{pgfscope}%
\begin{pgfscope}%
\pgfsys@transformshift{4.372651in}{5.755195in}%
\pgfsys@useobject{currentmarker}{}%
\end{pgfscope}%
\begin{pgfscope}%
\pgfsys@transformshift{4.408744in}{5.757357in}%
\pgfsys@useobject{currentmarker}{}%
\end{pgfscope}%
\begin{pgfscope}%
\pgfsys@transformshift{4.443452in}{5.760794in}%
\pgfsys@useobject{currentmarker}{}%
\end{pgfscope}%
\begin{pgfscope}%
\pgfsys@transformshift{4.476887in}{5.786102in}%
\pgfsys@useobject{currentmarker}{}%
\end{pgfscope}%
\begin{pgfscope}%
\pgfsys@transformshift{4.509148in}{5.799157in}%
\pgfsys@useobject{currentmarker}{}%
\end{pgfscope}%
\begin{pgfscope}%
\pgfsys@transformshift{4.540323in}{5.799349in}%
\pgfsys@useobject{currentmarker}{}%
\end{pgfscope}%
\begin{pgfscope}%
\pgfsys@transformshift{4.570489in}{5.799953in}%
\pgfsys@useobject{currentmarker}{}%
\end{pgfscope}%
\begin{pgfscope}%
\pgfsys@transformshift{4.599718in}{5.801293in}%
\pgfsys@useobject{currentmarker}{}%
\end{pgfscope}%
\begin{pgfscope}%
\pgfsys@transformshift{4.628071in}{5.805786in}%
\pgfsys@useobject{currentmarker}{}%
\end{pgfscope}%
\begin{pgfscope}%
\pgfsys@transformshift{4.655604in}{5.810786in}%
\pgfsys@useobject{currentmarker}{}%
\end{pgfscope}%
\begin{pgfscope}%
\pgfsys@transformshift{4.682369in}{5.811592in}%
\pgfsys@useobject{currentmarker}{}%
\end{pgfscope}%
\begin{pgfscope}%
\pgfsys@transformshift{4.708413in}{5.813128in}%
\pgfsys@useobject{currentmarker}{}%
\end{pgfscope}%
\begin{pgfscope}%
\pgfsys@transformshift{4.733777in}{5.821386in}%
\pgfsys@useobject{currentmarker}{}%
\end{pgfscope}%
\begin{pgfscope}%
\pgfsys@transformshift{4.758500in}{5.821824in}%
\pgfsys@useobject{currentmarker}{}%
\end{pgfscope}%
\begin{pgfscope}%
\pgfsys@transformshift{4.782618in}{5.822816in}%
\pgfsys@useobject{currentmarker}{}%
\end{pgfscope}%
\begin{pgfscope}%
\pgfsys@transformshift{4.806162in}{5.829951in}%
\pgfsys@useobject{currentmarker}{}%
\end{pgfscope}%
\begin{pgfscope}%
\pgfsys@transformshift{4.829162in}{5.830360in}%
\pgfsys@useobject{currentmarker}{}%
\end{pgfscope}%
\begin{pgfscope}%
\pgfsys@transformshift{4.851647in}{5.831011in}%
\pgfsys@useobject{currentmarker}{}%
\end{pgfscope}%
\begin{pgfscope}%
\pgfsys@transformshift{4.873641in}{5.832586in}%
\pgfsys@useobject{currentmarker}{}%
\end{pgfscope}%
\begin{pgfscope}%
\pgfsys@transformshift{4.895168in}{5.836453in}%
\pgfsys@useobject{currentmarker}{}%
\end{pgfscope}%
\begin{pgfscope}%
\pgfsys@transformshift{4.916250in}{5.839467in}%
\pgfsys@useobject{currentmarker}{}%
\end{pgfscope}%
\begin{pgfscope}%
\pgfsys@transformshift{4.936908in}{5.850563in}%
\pgfsys@useobject{currentmarker}{}%
\end{pgfscope}%
\begin{pgfscope}%
\pgfsys@transformshift{4.957159in}{5.858632in}%
\pgfsys@useobject{currentmarker}{}%
\end{pgfscope}%
\begin{pgfscope}%
\pgfsys@transformshift{4.977023in}{5.858841in}%
\pgfsys@useobject{currentmarker}{}%
\end{pgfscope}%
\begin{pgfscope}%
\pgfsys@transformshift{4.996515in}{5.860538in}%
\pgfsys@useobject{currentmarker}{}%
\end{pgfscope}%
\begin{pgfscope}%
\pgfsys@transformshift{5.015651in}{5.861230in}%
\pgfsys@useobject{currentmarker}{}%
\end{pgfscope}%
\begin{pgfscope}%
\pgfsys@transformshift{5.034445in}{5.861543in}%
\pgfsys@useobject{currentmarker}{}%
\end{pgfscope}%
\begin{pgfscope}%
\pgfsys@transformshift{5.052911in}{5.863356in}%
\pgfsys@useobject{currentmarker}{}%
\end{pgfscope}%
\begin{pgfscope}%
\pgfsys@transformshift{5.071062in}{5.863400in}%
\pgfsys@useobject{currentmarker}{}%
\end{pgfscope}%
\begin{pgfscope}%
\pgfsys@transformshift{5.088911in}{5.870159in}%
\pgfsys@useobject{currentmarker}{}%
\end{pgfscope}%
\begin{pgfscope}%
\pgfsys@transformshift{5.106467in}{5.871112in}%
\pgfsys@useobject{currentmarker}{}%
\end{pgfscope}%
\begin{pgfscope}%
\pgfsys@transformshift{5.123742in}{5.881991in}%
\pgfsys@useobject{currentmarker}{}%
\end{pgfscope}%
\begin{pgfscope}%
\pgfsys@transformshift{5.140747in}{5.881995in}%
\pgfsys@useobject{currentmarker}{}%
\end{pgfscope}%
\begin{pgfscope}%
\pgfsys@transformshift{5.157490in}{5.882213in}%
\pgfsys@useobject{currentmarker}{}%
\end{pgfscope}%
\begin{pgfscope}%
\pgfsys@transformshift{5.173982in}{5.886118in}%
\pgfsys@useobject{currentmarker}{}%
\end{pgfscope}%
\begin{pgfscope}%
\pgfsys@transformshift{5.190229in}{5.887187in}%
\pgfsys@useobject{currentmarker}{}%
\end{pgfscope}%
\begin{pgfscope}%
\pgfsys@transformshift{5.206241in}{5.888470in}%
\pgfsys@useobject{currentmarker}{}%
\end{pgfscope}%
\begin{pgfscope}%
\pgfsys@transformshift{5.222026in}{5.889177in}%
\pgfsys@useobject{currentmarker}{}%
\end{pgfscope}%
\begin{pgfscope}%
\pgfsys@transformshift{5.237590in}{5.893073in}%
\pgfsys@useobject{currentmarker}{}%
\end{pgfscope}%
\begin{pgfscope}%
\pgfsys@transformshift{5.252941in}{5.896955in}%
\pgfsys@useobject{currentmarker}{}%
\end{pgfscope}%
\begin{pgfscope}%
\pgfsys@transformshift{5.268086in}{5.898388in}%
\pgfsys@useobject{currentmarker}{}%
\end{pgfscope}%
\begin{pgfscope}%
\pgfsys@transformshift{5.283030in}{5.900772in}%
\pgfsys@useobject{currentmarker}{}%
\end{pgfscope}%
\begin{pgfscope}%
\pgfsys@transformshift{5.297779in}{5.901250in}%
\pgfsys@useobject{currentmarker}{}%
\end{pgfscope}%
\begin{pgfscope}%
\pgfsys@transformshift{5.312341in}{5.902031in}%
\pgfsys@useobject{currentmarker}{}%
\end{pgfscope}%
\begin{pgfscope}%
\pgfsys@transformshift{5.326720in}{5.907344in}%
\pgfsys@useobject{currentmarker}{}%
\end{pgfscope}%
\begin{pgfscope}%
\pgfsys@transformshift{5.340920in}{5.909586in}%
\pgfsys@useobject{currentmarker}{}%
\end{pgfscope}%
\begin{pgfscope}%
\pgfsys@transformshift{5.354949in}{5.909840in}%
\pgfsys@useobject{currentmarker}{}%
\end{pgfscope}%
\begin{pgfscope}%
\pgfsys@transformshift{5.368810in}{5.912887in}%
\pgfsys@useobject{currentmarker}{}%
\end{pgfscope}%
\begin{pgfscope}%
\pgfsys@transformshift{5.382507in}{5.913426in}%
\pgfsys@useobject{currentmarker}{}%
\end{pgfscope}%
\begin{pgfscope}%
\pgfsys@transformshift{5.396046in}{5.913430in}%
\pgfsys@useobject{currentmarker}{}%
\end{pgfscope}%
\begin{pgfscope}%
\pgfsys@transformshift{5.409431in}{5.917663in}%
\pgfsys@useobject{currentmarker}{}%
\end{pgfscope}%
\begin{pgfscope}%
\pgfsys@transformshift{5.422665in}{5.918334in}%
\pgfsys@useobject{currentmarker}{}%
\end{pgfscope}%
\begin{pgfscope}%
\pgfsys@transformshift{5.435753in}{5.918418in}%
\pgfsys@useobject{currentmarker}{}%
\end{pgfscope}%
\begin{pgfscope}%
\pgfsys@transformshift{5.448698in}{5.920137in}%
\pgfsys@useobject{currentmarker}{}%
\end{pgfscope}%
\begin{pgfscope}%
\pgfsys@transformshift{5.461504in}{5.920721in}%
\pgfsys@useobject{currentmarker}{}%
\end{pgfscope}%
\begin{pgfscope}%
\pgfsys@transformshift{5.474175in}{5.922419in}%
\pgfsys@useobject{currentmarker}{}%
\end{pgfscope}%
\begin{pgfscope}%
\pgfsys@transformshift{5.486714in}{5.922774in}%
\pgfsys@useobject{currentmarker}{}%
\end{pgfscope}%
\begin{pgfscope}%
\pgfsys@transformshift{5.499123in}{5.922995in}%
\pgfsys@useobject{currentmarker}{}%
\end{pgfscope}%
\begin{pgfscope}%
\pgfsys@transformshift{5.511407in}{5.923223in}%
\pgfsys@useobject{currentmarker}{}%
\end{pgfscope}%
\begin{pgfscope}%
\pgfsys@transformshift{5.523568in}{5.923282in}%
\pgfsys@useobject{currentmarker}{}%
\end{pgfscope}%
\begin{pgfscope}%
\pgfsys@transformshift{5.535609in}{5.923552in}%
\pgfsys@useobject{currentmarker}{}%
\end{pgfscope}%
\begin{pgfscope}%
\pgfsys@transformshift{5.547533in}{5.924576in}%
\pgfsys@useobject{currentmarker}{}%
\end{pgfscope}%
\begin{pgfscope}%
\pgfsys@transformshift{5.559342in}{5.926792in}%
\pgfsys@useobject{currentmarker}{}%
\end{pgfscope}%
\begin{pgfscope}%
\pgfsys@transformshift{5.571040in}{5.928252in}%
\pgfsys@useobject{currentmarker}{}%
\end{pgfscope}%
\begin{pgfscope}%
\pgfsys@transformshift{5.582628in}{5.929224in}%
\pgfsys@useobject{currentmarker}{}%
\end{pgfscope}%
\begin{pgfscope}%
\pgfsys@transformshift{5.594109in}{5.930039in}%
\pgfsys@useobject{currentmarker}{}%
\end{pgfscope}%
\begin{pgfscope}%
\pgfsys@transformshift{5.605485in}{5.930643in}%
\pgfsys@useobject{currentmarker}{}%
\end{pgfscope}%
\begin{pgfscope}%
\pgfsys@transformshift{5.616759in}{5.933314in}%
\pgfsys@useobject{currentmarker}{}%
\end{pgfscope}%
\begin{pgfscope}%
\pgfsys@transformshift{5.627933in}{5.934681in}%
\pgfsys@useobject{currentmarker}{}%
\end{pgfscope}%
\begin{pgfscope}%
\pgfsys@transformshift{5.639009in}{5.935456in}%
\pgfsys@useobject{currentmarker}{}%
\end{pgfscope}%
\begin{pgfscope}%
\pgfsys@transformshift{5.649989in}{5.935490in}%
\pgfsys@useobject{currentmarker}{}%
\end{pgfscope}%
\begin{pgfscope}%
\pgfsys@transformshift{5.660874in}{5.937716in}%
\pgfsys@useobject{currentmarker}{}%
\end{pgfscope}%
\begin{pgfscope}%
\pgfsys@transformshift{5.671668in}{5.938519in}%
\pgfsys@useobject{currentmarker}{}%
\end{pgfscope}%
\begin{pgfscope}%
\pgfsys@transformshift{5.682371in}{5.938805in}%
\pgfsys@useobject{currentmarker}{}%
\end{pgfscope}%
\begin{pgfscope}%
\pgfsys@transformshift{5.692986in}{5.940553in}%
\pgfsys@useobject{currentmarker}{}%
\end{pgfscope}%
\begin{pgfscope}%
\pgfsys@transformshift{5.703514in}{5.942212in}%
\pgfsys@useobject{currentmarker}{}%
\end{pgfscope}%
\begin{pgfscope}%
\pgfsys@transformshift{5.713958in}{5.943002in}%
\pgfsys@useobject{currentmarker}{}%
\end{pgfscope}%
\begin{pgfscope}%
\pgfsys@transformshift{5.724318in}{5.943774in}%
\pgfsys@useobject{currentmarker}{}%
\end{pgfscope}%
\begin{pgfscope}%
\pgfsys@transformshift{5.734596in}{5.949593in}%
\pgfsys@useobject{currentmarker}{}%
\end{pgfscope}%
\begin{pgfscope}%
\pgfsys@transformshift{5.744794in}{5.949809in}%
\pgfsys@useobject{currentmarker}{}%
\end{pgfscope}%
\begin{pgfscope}%
\pgfsys@transformshift{5.754913in}{5.952479in}%
\pgfsys@useobject{currentmarker}{}%
\end{pgfscope}%
\begin{pgfscope}%
\pgfsys@transformshift{5.764955in}{5.955255in}%
\pgfsys@useobject{currentmarker}{}%
\end{pgfscope}%
\begin{pgfscope}%
\pgfsys@transformshift{5.774922in}{5.955619in}%
\pgfsys@useobject{currentmarker}{}%
\end{pgfscope}%
\begin{pgfscope}%
\pgfsys@transformshift{5.784814in}{5.955783in}%
\pgfsys@useobject{currentmarker}{}%
\end{pgfscope}%
\begin{pgfscope}%
\pgfsys@transformshift{5.794632in}{5.956751in}%
\pgfsys@useobject{currentmarker}{}%
\end{pgfscope}%
\begin{pgfscope}%
\pgfsys@transformshift{5.804380in}{5.956984in}%
\pgfsys@useobject{currentmarker}{}%
\end{pgfscope}%
\begin{pgfscope}%
\pgfsys@transformshift{5.814056in}{5.957398in}%
\pgfsys@useobject{currentmarker}{}%
\end{pgfscope}%
\begin{pgfscope}%
\pgfsys@transformshift{5.823664in}{5.957491in}%
\pgfsys@useobject{currentmarker}{}%
\end{pgfscope}%
\begin{pgfscope}%
\pgfsys@transformshift{5.833203in}{5.958137in}%
\pgfsys@useobject{currentmarker}{}%
\end{pgfscope}%
\begin{pgfscope}%
\pgfsys@transformshift{5.842676in}{5.958182in}%
\pgfsys@useobject{currentmarker}{}%
\end{pgfscope}%
\begin{pgfscope}%
\pgfsys@transformshift{5.852083in}{5.960109in}%
\pgfsys@useobject{currentmarker}{}%
\end{pgfscope}%
\begin{pgfscope}%
\pgfsys@transformshift{5.861425in}{5.962577in}%
\pgfsys@useobject{currentmarker}{}%
\end{pgfscope}%
\begin{pgfscope}%
\pgfsys@transformshift{5.870704in}{5.963868in}%
\pgfsys@useobject{currentmarker}{}%
\end{pgfscope}%
\begin{pgfscope}%
\pgfsys@transformshift{5.879920in}{5.964513in}%
\pgfsys@useobject{currentmarker}{}%
\end{pgfscope}%
\begin{pgfscope}%
\pgfsys@transformshift{5.889075in}{5.964532in}%
\pgfsys@useobject{currentmarker}{}%
\end{pgfscope}%
\begin{pgfscope}%
\pgfsys@transformshift{5.898170in}{5.965419in}%
\pgfsys@useobject{currentmarker}{}%
\end{pgfscope}%
\begin{pgfscope}%
\pgfsys@transformshift{5.907205in}{5.968479in}%
\pgfsys@useobject{currentmarker}{}%
\end{pgfscope}%
\begin{pgfscope}%
\pgfsys@transformshift{5.916182in}{5.968975in}%
\pgfsys@useobject{currentmarker}{}%
\end{pgfscope}%
\begin{pgfscope}%
\pgfsys@transformshift{5.925101in}{5.971416in}%
\pgfsys@useobject{currentmarker}{}%
\end{pgfscope}%
\begin{pgfscope}%
\pgfsys@transformshift{5.933964in}{5.972548in}%
\pgfsys@useobject{currentmarker}{}%
\end{pgfscope}%
\begin{pgfscope}%
\pgfsys@transformshift{5.942770in}{5.972675in}%
\pgfsys@useobject{currentmarker}{}%
\end{pgfscope}%
\begin{pgfscope}%
\pgfsys@transformshift{5.951522in}{5.973012in}%
\pgfsys@useobject{currentmarker}{}%
\end{pgfscope}%
\begin{pgfscope}%
\pgfsys@transformshift{5.960221in}{5.974776in}%
\pgfsys@useobject{currentmarker}{}%
\end{pgfscope}%
\begin{pgfscope}%
\pgfsys@transformshift{5.968865in}{5.975683in}%
\pgfsys@useobject{currentmarker}{}%
\end{pgfscope}%
\begin{pgfscope}%
\pgfsys@transformshift{5.977458in}{5.977481in}%
\pgfsys@useobject{currentmarker}{}%
\end{pgfscope}%
\begin{pgfscope}%
\pgfsys@transformshift{5.985999in}{5.978752in}%
\pgfsys@useobject{currentmarker}{}%
\end{pgfscope}%
\begin{pgfscope}%
\pgfsys@transformshift{5.994489in}{5.978837in}%
\pgfsys@useobject{currentmarker}{}%
\end{pgfscope}%
\begin{pgfscope}%
\pgfsys@transformshift{6.002929in}{5.979000in}%
\pgfsys@useobject{currentmarker}{}%
\end{pgfscope}%
\begin{pgfscope}%
\pgfsys@transformshift{6.011320in}{5.979316in}%
\pgfsys@useobject{currentmarker}{}%
\end{pgfscope}%
\begin{pgfscope}%
\pgfsys@transformshift{6.019663in}{5.979703in}%
\pgfsys@useobject{currentmarker}{}%
\end{pgfscope}%
\begin{pgfscope}%
\pgfsys@transformshift{6.027957in}{5.979949in}%
\pgfsys@useobject{currentmarker}{}%
\end{pgfscope}%
\begin{pgfscope}%
\pgfsys@transformshift{6.036204in}{5.980510in}%
\pgfsys@useobject{currentmarker}{}%
\end{pgfscope}%
\begin{pgfscope}%
\pgfsys@transformshift{6.044405in}{5.980922in}%
\pgfsys@useobject{currentmarker}{}%
\end{pgfscope}%
\begin{pgfscope}%
\pgfsys@transformshift{6.052560in}{5.981307in}%
\pgfsys@useobject{currentmarker}{}%
\end{pgfscope}%
\begin{pgfscope}%
\pgfsys@transformshift{6.060670in}{5.981869in}%
\pgfsys@useobject{currentmarker}{}%
\end{pgfscope}%
\begin{pgfscope}%
\pgfsys@transformshift{6.068735in}{5.982173in}%
\pgfsys@useobject{currentmarker}{}%
\end{pgfscope}%
\begin{pgfscope}%
\pgfsys@transformshift{6.076756in}{5.982343in}%
\pgfsys@useobject{currentmarker}{}%
\end{pgfscope}%
\begin{pgfscope}%
\pgfsys@transformshift{6.084734in}{5.982748in}%
\pgfsys@useobject{currentmarker}{}%
\end{pgfscope}%
\begin{pgfscope}%
\pgfsys@transformshift{6.092669in}{5.983268in}%
\pgfsys@useobject{currentmarker}{}%
\end{pgfscope}%
\begin{pgfscope}%
\pgfsys@transformshift{6.100562in}{5.984112in}%
\pgfsys@useobject{currentmarker}{}%
\end{pgfscope}%
\begin{pgfscope}%
\pgfsys@transformshift{6.108414in}{5.985004in}%
\pgfsys@useobject{currentmarker}{}%
\end{pgfscope}%
\begin{pgfscope}%
\pgfsys@transformshift{6.116224in}{5.985028in}%
\pgfsys@useobject{currentmarker}{}%
\end{pgfscope}%
\begin{pgfscope}%
\pgfsys@transformshift{6.123994in}{5.986367in}%
\pgfsys@useobject{currentmarker}{}%
\end{pgfscope}%
\begin{pgfscope}%
\pgfsys@transformshift{6.131724in}{5.987423in}%
\pgfsys@useobject{currentmarker}{}%
\end{pgfscope}%
\begin{pgfscope}%
\pgfsys@transformshift{6.139414in}{5.987453in}%
\pgfsys@useobject{currentmarker}{}%
\end{pgfscope}%
\begin{pgfscope}%
\pgfsys@transformshift{6.147066in}{5.989716in}%
\pgfsys@useobject{currentmarker}{}%
\end{pgfscope}%
\begin{pgfscope}%
\pgfsys@transformshift{6.154679in}{5.989852in}%
\pgfsys@useobject{currentmarker}{}%
\end{pgfscope}%
\begin{pgfscope}%
\pgfsys@transformshift{6.162255in}{5.990275in}%
\pgfsys@useobject{currentmarker}{}%
\end{pgfscope}%
\begin{pgfscope}%
\pgfsys@transformshift{6.169793in}{5.990957in}%
\pgfsys@useobject{currentmarker}{}%
\end{pgfscope}%
\begin{pgfscope}%
\pgfsys@transformshift{6.177294in}{5.991595in}%
\pgfsys@useobject{currentmarker}{}%
\end{pgfscope}%
\begin{pgfscope}%
\pgfsys@transformshift{6.184758in}{5.991709in}%
\pgfsys@useobject{currentmarker}{}%
\end{pgfscope}%
\begin{pgfscope}%
\pgfsys@transformshift{6.192187in}{5.991957in}%
\pgfsys@useobject{currentmarker}{}%
\end{pgfscope}%
\begin{pgfscope}%
\pgfsys@transformshift{6.199580in}{5.992237in}%
\pgfsys@useobject{currentmarker}{}%
\end{pgfscope}%
\begin{pgfscope}%
\pgfsys@transformshift{6.206938in}{5.992900in}%
\pgfsys@useobject{currentmarker}{}%
\end{pgfscope}%
\begin{pgfscope}%
\pgfsys@transformshift{6.214261in}{5.993240in}%
\pgfsys@useobject{currentmarker}{}%
\end{pgfscope}%
\begin{pgfscope}%
\pgfsys@transformshift{6.221550in}{5.993404in}%
\pgfsys@useobject{currentmarker}{}%
\end{pgfscope}%
\begin{pgfscope}%
\pgfsys@transformshift{6.228805in}{5.994202in}%
\pgfsys@useobject{currentmarker}{}%
\end{pgfscope}%
\begin{pgfscope}%
\pgfsys@transformshift{6.236027in}{5.994639in}%
\pgfsys@useobject{currentmarker}{}%
\end{pgfscope}%
\begin{pgfscope}%
\pgfsys@transformshift{6.243216in}{5.995284in}%
\pgfsys@useobject{currentmarker}{}%
\end{pgfscope}%
\begin{pgfscope}%
\pgfsys@transformshift{6.250372in}{5.995662in}%
\pgfsys@useobject{currentmarker}{}%
\end{pgfscope}%
\begin{pgfscope}%
\pgfsys@transformshift{6.257495in}{5.996225in}%
\pgfsys@useobject{currentmarker}{}%
\end{pgfscope}%
\begin{pgfscope}%
\pgfsys@transformshift{6.264587in}{5.996588in}%
\pgfsys@useobject{currentmarker}{}%
\end{pgfscope}%
\begin{pgfscope}%
\pgfsys@transformshift{6.271648in}{5.996738in}%
\pgfsys@useobject{currentmarker}{}%
\end{pgfscope}%
\begin{pgfscope}%
\pgfsys@transformshift{6.278677in}{5.997728in}%
\pgfsys@useobject{currentmarker}{}%
\end{pgfscope}%
\begin{pgfscope}%
\pgfsys@transformshift{6.285676in}{5.998244in}%
\pgfsys@useobject{currentmarker}{}%
\end{pgfscope}%
\begin{pgfscope}%
\pgfsys@transformshift{6.292644in}{5.998346in}%
\pgfsys@useobject{currentmarker}{}%
\end{pgfscope}%
\begin{pgfscope}%
\pgfsys@transformshift{6.299582in}{5.999282in}%
\pgfsys@useobject{currentmarker}{}%
\end{pgfscope}%
\begin{pgfscope}%
\pgfsys@transformshift{6.306490in}{5.999476in}%
\pgfsys@useobject{currentmarker}{}%
\end{pgfscope}%
\begin{pgfscope}%
\pgfsys@transformshift{6.313369in}{5.999479in}%
\pgfsys@useobject{currentmarker}{}%
\end{pgfscope}%
\begin{pgfscope}%
\pgfsys@transformshift{6.320219in}{5.999481in}%
\pgfsys@useobject{currentmarker}{}%
\end{pgfscope}%
\begin{pgfscope}%
\pgfsys@transformshift{6.327040in}{5.999544in}%
\pgfsys@useobject{currentmarker}{}%
\end{pgfscope}%
\begin{pgfscope}%
\pgfsys@transformshift{6.333833in}{6.000526in}%
\pgfsys@useobject{currentmarker}{}%
\end{pgfscope}%
\begin{pgfscope}%
\pgfsys@transformshift{6.340598in}{6.001604in}%
\pgfsys@useobject{currentmarker}{}%
\end{pgfscope}%
\begin{pgfscope}%
\pgfsys@transformshift{6.347335in}{6.002074in}%
\pgfsys@useobject{currentmarker}{}%
\end{pgfscope}%
\begin{pgfscope}%
\pgfsys@transformshift{6.354045in}{6.003629in}%
\pgfsys@useobject{currentmarker}{}%
\end{pgfscope}%
\begin{pgfscope}%
\pgfsys@transformshift{6.360727in}{6.004071in}%
\pgfsys@useobject{currentmarker}{}%
\end{pgfscope}%
\begin{pgfscope}%
\pgfsys@transformshift{6.367383in}{6.004935in}%
\pgfsys@useobject{currentmarker}{}%
\end{pgfscope}%
\begin{pgfscope}%
\pgfsys@transformshift{6.374012in}{6.005101in}%
\pgfsys@useobject{currentmarker}{}%
\end{pgfscope}%
\begin{pgfscope}%
\pgfsys@transformshift{6.380614in}{6.005256in}%
\pgfsys@useobject{currentmarker}{}%
\end{pgfscope}%
\begin{pgfscope}%
\pgfsys@transformshift{6.387191in}{6.005606in}%
\pgfsys@useobject{currentmarker}{}%
\end{pgfscope}%
\begin{pgfscope}%
\pgfsys@transformshift{6.393742in}{6.005741in}%
\pgfsys@useobject{currentmarker}{}%
\end{pgfscope}%
\begin{pgfscope}%
\pgfsys@transformshift{6.400267in}{6.005823in}%
\pgfsys@useobject{currentmarker}{}%
\end{pgfscope}%
\begin{pgfscope}%
\pgfsys@transformshift{6.406767in}{6.006590in}%
\pgfsys@useobject{currentmarker}{}%
\end{pgfscope}%
\begin{pgfscope}%
\pgfsys@transformshift{6.413243in}{6.008337in}%
\pgfsys@useobject{currentmarker}{}%
\end{pgfscope}%
\begin{pgfscope}%
\pgfsys@transformshift{6.419693in}{6.008527in}%
\pgfsys@useobject{currentmarker}{}%
\end{pgfscope}%
\begin{pgfscope}%
\pgfsys@transformshift{6.426119in}{6.008850in}%
\pgfsys@useobject{currentmarker}{}%
\end{pgfscope}%
\begin{pgfscope}%
\pgfsys@transformshift{6.432522in}{6.010799in}%
\pgfsys@useobject{currentmarker}{}%
\end{pgfscope}%
\begin{pgfscope}%
\pgfsys@transformshift{6.438900in}{6.011030in}%
\pgfsys@useobject{currentmarker}{}%
\end{pgfscope}%
\begin{pgfscope}%
\pgfsys@transformshift{6.445254in}{6.011109in}%
\pgfsys@useobject{currentmarker}{}%
\end{pgfscope}%
\begin{pgfscope}%
\pgfsys@transformshift{6.451585in}{6.011594in}%
\pgfsys@useobject{currentmarker}{}%
\end{pgfscope}%
\begin{pgfscope}%
\pgfsys@transformshift{6.457893in}{6.011612in}%
\pgfsys@useobject{currentmarker}{}%
\end{pgfscope}%
\begin{pgfscope}%
\pgfsys@transformshift{6.464178in}{6.012676in}%
\pgfsys@useobject{currentmarker}{}%
\end{pgfscope}%
\begin{pgfscope}%
\pgfsys@transformshift{6.470440in}{6.013005in}%
\pgfsys@useobject{currentmarker}{}%
\end{pgfscope}%
\begin{pgfscope}%
\pgfsys@transformshift{6.476680in}{6.013040in}%
\pgfsys@useobject{currentmarker}{}%
\end{pgfscope}%
\begin{pgfscope}%
\pgfsys@transformshift{6.482897in}{6.013302in}%
\pgfsys@useobject{currentmarker}{}%
\end{pgfscope}%
\begin{pgfscope}%
\pgfsys@transformshift{6.489092in}{6.013568in}%
\pgfsys@useobject{currentmarker}{}%
\end{pgfscope}%
\begin{pgfscope}%
\pgfsys@transformshift{6.495266in}{6.015412in}%
\pgfsys@useobject{currentmarker}{}%
\end{pgfscope}%
\begin{pgfscope}%
\pgfsys@transformshift{6.501418in}{6.015719in}%
\pgfsys@useobject{currentmarker}{}%
\end{pgfscope}%
\begin{pgfscope}%
\pgfsys@transformshift{6.507548in}{6.015834in}%
\pgfsys@useobject{currentmarker}{}%
\end{pgfscope}%
\begin{pgfscope}%
\pgfsys@transformshift{6.513657in}{6.017535in}%
\pgfsys@useobject{currentmarker}{}%
\end{pgfscope}%
\begin{pgfscope}%
\pgfsys@transformshift{6.519745in}{6.017782in}%
\pgfsys@useobject{currentmarker}{}%
\end{pgfscope}%
\begin{pgfscope}%
\pgfsys@transformshift{6.525813in}{6.018367in}%
\pgfsys@useobject{currentmarker}{}%
\end{pgfscope}%
\begin{pgfscope}%
\pgfsys@transformshift{6.531859in}{6.018612in}%
\pgfsys@useobject{currentmarker}{}%
\end{pgfscope}%
\begin{pgfscope}%
\pgfsys@transformshift{6.537886in}{6.018817in}%
\pgfsys@useobject{currentmarker}{}%
\end{pgfscope}%
\begin{pgfscope}%
\pgfsys@transformshift{6.543892in}{6.019078in}%
\pgfsys@useobject{currentmarker}{}%
\end{pgfscope}%
\begin{pgfscope}%
\pgfsys@transformshift{6.549878in}{6.019078in}%
\pgfsys@useobject{currentmarker}{}%
\end{pgfscope}%
\begin{pgfscope}%
\pgfsys@transformshift{6.555844in}{6.019281in}%
\pgfsys@useobject{currentmarker}{}%
\end{pgfscope}%
\begin{pgfscope}%
\pgfsys@transformshift{6.561791in}{6.019720in}%
\pgfsys@useobject{currentmarker}{}%
\end{pgfscope}%
\begin{pgfscope}%
\pgfsys@transformshift{6.567718in}{6.019790in}%
\pgfsys@useobject{currentmarker}{}%
\end{pgfscope}%
\begin{pgfscope}%
\pgfsys@transformshift{6.573626in}{6.019857in}%
\pgfsys@useobject{currentmarker}{}%
\end{pgfscope}%
\begin{pgfscope}%
\pgfsys@transformshift{6.579514in}{6.020203in}%
\pgfsys@useobject{currentmarker}{}%
\end{pgfscope}%
\begin{pgfscope}%
\pgfsys@transformshift{6.585384in}{6.021310in}%
\pgfsys@useobject{currentmarker}{}%
\end{pgfscope}%
\begin{pgfscope}%
\pgfsys@transformshift{6.591235in}{6.024134in}%
\pgfsys@useobject{currentmarker}{}%
\end{pgfscope}%
\begin{pgfscope}%
\pgfsys@transformshift{6.597068in}{6.024329in}%
\pgfsys@useobject{currentmarker}{}%
\end{pgfscope}%
\begin{pgfscope}%
\pgfsys@transformshift{6.602882in}{6.024400in}%
\pgfsys@useobject{currentmarker}{}%
\end{pgfscope}%
\begin{pgfscope}%
\pgfsys@transformshift{6.608677in}{6.025850in}%
\pgfsys@useobject{currentmarker}{}%
\end{pgfscope}%
\begin{pgfscope}%
\pgfsys@transformshift{6.614455in}{6.025857in}%
\pgfsys@useobject{currentmarker}{}%
\end{pgfscope}%
\begin{pgfscope}%
\pgfsys@transformshift{6.620215in}{6.025935in}%
\pgfsys@useobject{currentmarker}{}%
\end{pgfscope}%
\begin{pgfscope}%
\pgfsys@transformshift{6.625957in}{6.026085in}%
\pgfsys@useobject{currentmarker}{}%
\end{pgfscope}%
\begin{pgfscope}%
\pgfsys@transformshift{6.631681in}{6.026238in}%
\pgfsys@useobject{currentmarker}{}%
\end{pgfscope}%
\begin{pgfscope}%
\pgfsys@transformshift{6.637388in}{6.026247in}%
\pgfsys@useobject{currentmarker}{}%
\end{pgfscope}%
\begin{pgfscope}%
\pgfsys@transformshift{6.643077in}{6.026626in}%
\pgfsys@useobject{currentmarker}{}%
\end{pgfscope}%
\begin{pgfscope}%
\pgfsys@transformshift{6.648750in}{6.026672in}%
\pgfsys@useobject{currentmarker}{}%
\end{pgfscope}%
\begin{pgfscope}%
\pgfsys@transformshift{6.654405in}{6.027137in}%
\pgfsys@useobject{currentmarker}{}%
\end{pgfscope}%
\begin{pgfscope}%
\pgfsys@transformshift{6.660043in}{6.027969in}%
\pgfsys@useobject{currentmarker}{}%
\end{pgfscope}%
\begin{pgfscope}%
\pgfsys@transformshift{6.665665in}{6.028269in}%
\pgfsys@useobject{currentmarker}{}%
\end{pgfscope}%
\begin{pgfscope}%
\pgfsys@transformshift{6.671270in}{6.028878in}%
\pgfsys@useobject{currentmarker}{}%
\end{pgfscope}%
\begin{pgfscope}%
\pgfsys@transformshift{6.676859in}{6.029625in}%
\pgfsys@useobject{currentmarker}{}%
\end{pgfscope}%
\begin{pgfscope}%
\pgfsys@transformshift{6.682432in}{6.030084in}%
\pgfsys@useobject{currentmarker}{}%
\end{pgfscope}%
\begin{pgfscope}%
\pgfsys@transformshift{6.687988in}{6.030416in}%
\pgfsys@useobject{currentmarker}{}%
\end{pgfscope}%
\begin{pgfscope}%
\pgfsys@transformshift{6.693528in}{6.030785in}%
\pgfsys@useobject{currentmarker}{}%
\end{pgfscope}%
\begin{pgfscope}%
\pgfsys@transformshift{6.699053in}{6.031632in}%
\pgfsys@useobject{currentmarker}{}%
\end{pgfscope}%
\begin{pgfscope}%
\pgfsys@transformshift{6.704562in}{6.031870in}%
\pgfsys@useobject{currentmarker}{}%
\end{pgfscope}%
\begin{pgfscope}%
\pgfsys@transformshift{6.710055in}{6.031918in}%
\pgfsys@useobject{currentmarker}{}%
\end{pgfscope}%
\begin{pgfscope}%
\pgfsys@transformshift{6.715532in}{6.032266in}%
\pgfsys@useobject{currentmarker}{}%
\end{pgfscope}%
\begin{pgfscope}%
\pgfsys@transformshift{6.720995in}{6.032302in}%
\pgfsys@useobject{currentmarker}{}%
\end{pgfscope}%
\begin{pgfscope}%
\pgfsys@transformshift{6.726442in}{6.032383in}%
\pgfsys@useobject{currentmarker}{}%
\end{pgfscope}%
\begin{pgfscope}%
\pgfsys@transformshift{6.731874in}{6.032772in}%
\pgfsys@useobject{currentmarker}{}%
\end{pgfscope}%
\begin{pgfscope}%
\pgfsys@transformshift{6.737291in}{6.033569in}%
\pgfsys@useobject{currentmarker}{}%
\end{pgfscope}%
\begin{pgfscope}%
\pgfsys@transformshift{6.742693in}{6.034185in}%
\pgfsys@useobject{currentmarker}{}%
\end{pgfscope}%
\begin{pgfscope}%
\pgfsys@transformshift{6.748080in}{6.034545in}%
\pgfsys@useobject{currentmarker}{}%
\end{pgfscope}%
\begin{pgfscope}%
\pgfsys@transformshift{6.753453in}{6.034627in}%
\pgfsys@useobject{currentmarker}{}%
\end{pgfscope}%
\begin{pgfscope}%
\pgfsys@transformshift{6.758811in}{6.035325in}%
\pgfsys@useobject{currentmarker}{}%
\end{pgfscope}%
\begin{pgfscope}%
\pgfsys@transformshift{6.764155in}{6.036681in}%
\pgfsys@useobject{currentmarker}{}%
\end{pgfscope}%
\begin{pgfscope}%
\pgfsys@transformshift{6.769485in}{6.037074in}%
\pgfsys@useobject{currentmarker}{}%
\end{pgfscope}%
\begin{pgfscope}%
\pgfsys@transformshift{6.774801in}{6.037940in}%
\pgfsys@useobject{currentmarker}{}%
\end{pgfscope}%
\begin{pgfscope}%
\pgfsys@transformshift{6.780102in}{6.038371in}%
\pgfsys@useobject{currentmarker}{}%
\end{pgfscope}%
\begin{pgfscope}%
\pgfsys@transformshift{6.785390in}{6.038934in}%
\pgfsys@useobject{currentmarker}{}%
\end{pgfscope}%
\begin{pgfscope}%
\pgfsys@transformshift{6.790664in}{6.039196in}%
\pgfsys@useobject{currentmarker}{}%
\end{pgfscope}%
\begin{pgfscope}%
\pgfsys@transformshift{6.795924in}{6.039240in}%
\pgfsys@useobject{currentmarker}{}%
\end{pgfscope}%
\begin{pgfscope}%
\pgfsys@transformshift{6.801170in}{6.041253in}%
\pgfsys@useobject{currentmarker}{}%
\end{pgfscope}%
\begin{pgfscope}%
\pgfsys@transformshift{6.806403in}{6.041588in}%
\pgfsys@useobject{currentmarker}{}%
\end{pgfscope}%
\begin{pgfscope}%
\pgfsys@transformshift{6.811623in}{6.041991in}%
\pgfsys@useobject{currentmarker}{}%
\end{pgfscope}%
\begin{pgfscope}%
\pgfsys@transformshift{6.816829in}{6.042695in}%
\pgfsys@useobject{currentmarker}{}%
\end{pgfscope}%
\begin{pgfscope}%
\pgfsys@transformshift{6.822023in}{6.043133in}%
\pgfsys@useobject{currentmarker}{}%
\end{pgfscope}%
\begin{pgfscope}%
\pgfsys@transformshift{6.827203in}{6.043281in}%
\pgfsys@useobject{currentmarker}{}%
\end{pgfscope}%
\begin{pgfscope}%
\pgfsys@transformshift{6.832370in}{6.043483in}%
\pgfsys@useobject{currentmarker}{}%
\end{pgfscope}%
\begin{pgfscope}%
\pgfsys@transformshift{6.837524in}{6.043575in}%
\pgfsys@useobject{currentmarker}{}%
\end{pgfscope}%
\begin{pgfscope}%
\pgfsys@transformshift{6.842666in}{6.043579in}%
\pgfsys@useobject{currentmarker}{}%
\end{pgfscope}%
\begin{pgfscope}%
\pgfsys@transformshift{6.847795in}{6.044707in}%
\pgfsys@useobject{currentmarker}{}%
\end{pgfscope}%
\begin{pgfscope}%
\pgfsys@transformshift{6.852911in}{6.045830in}%
\pgfsys@useobject{currentmarker}{}%
\end{pgfscope}%
\begin{pgfscope}%
\pgfsys@transformshift{6.858015in}{6.048003in}%
\pgfsys@useobject{currentmarker}{}%
\end{pgfscope}%
\begin{pgfscope}%
\pgfsys@transformshift{6.863106in}{6.048415in}%
\pgfsys@useobject{currentmarker}{}%
\end{pgfscope}%
\begin{pgfscope}%
\pgfsys@transformshift{6.868186in}{6.048453in}%
\pgfsys@useobject{currentmarker}{}%
\end{pgfscope}%
\begin{pgfscope}%
\pgfsys@transformshift{6.873253in}{6.048801in}%
\pgfsys@useobject{currentmarker}{}%
\end{pgfscope}%
\begin{pgfscope}%
\pgfsys@transformshift{6.878308in}{6.051745in}%
\pgfsys@useobject{currentmarker}{}%
\end{pgfscope}%
\begin{pgfscope}%
\pgfsys@transformshift{6.883350in}{6.052056in}%
\pgfsys@useobject{currentmarker}{}%
\end{pgfscope}%
\begin{pgfscope}%
\pgfsys@transformshift{6.888381in}{6.052416in}%
\pgfsys@useobject{currentmarker}{}%
\end{pgfscope}%
\begin{pgfscope}%
\pgfsys@transformshift{6.893401in}{6.052943in}%
\pgfsys@useobject{currentmarker}{}%
\end{pgfscope}%
\begin{pgfscope}%
\pgfsys@transformshift{6.898408in}{6.053064in}%
\pgfsys@useobject{currentmarker}{}%
\end{pgfscope}%
\begin{pgfscope}%
\pgfsys@transformshift{6.903404in}{6.053478in}%
\pgfsys@useobject{currentmarker}{}%
\end{pgfscope}%
\begin{pgfscope}%
\pgfsys@transformshift{6.908388in}{6.053999in}%
\pgfsys@useobject{currentmarker}{}%
\end{pgfscope}%
\begin{pgfscope}%
\pgfsys@transformshift{6.913361in}{6.054301in}%
\pgfsys@useobject{currentmarker}{}%
\end{pgfscope}%
\begin{pgfscope}%
\pgfsys@transformshift{6.918322in}{6.054360in}%
\pgfsys@useobject{currentmarker}{}%
\end{pgfscope}%
\begin{pgfscope}%
\pgfsys@transformshift{6.923272in}{6.054592in}%
\pgfsys@useobject{currentmarker}{}%
\end{pgfscope}%
\begin{pgfscope}%
\pgfsys@transformshift{6.928211in}{6.055505in}%
\pgfsys@useobject{currentmarker}{}%
\end{pgfscope}%
\begin{pgfscope}%
\pgfsys@transformshift{6.933138in}{6.056946in}%
\pgfsys@useobject{currentmarker}{}%
\end{pgfscope}%
\begin{pgfscope}%
\pgfsys@transformshift{6.938055in}{6.057921in}%
\pgfsys@useobject{currentmarker}{}%
\end{pgfscope}%
\begin{pgfscope}%
\pgfsys@transformshift{6.942960in}{6.058006in}%
\pgfsys@useobject{currentmarker}{}%
\end{pgfscope}%
\begin{pgfscope}%
\pgfsys@transformshift{6.947855in}{6.058216in}%
\pgfsys@useobject{currentmarker}{}%
\end{pgfscope}%
\begin{pgfscope}%
\pgfsys@transformshift{6.952738in}{6.058716in}%
\pgfsys@useobject{currentmarker}{}%
\end{pgfscope}%
\begin{pgfscope}%
\pgfsys@transformshift{6.957611in}{6.058911in}%
\pgfsys@useobject{currentmarker}{}%
\end{pgfscope}%
\begin{pgfscope}%
\pgfsys@transformshift{6.962474in}{6.060065in}%
\pgfsys@useobject{currentmarker}{}%
\end{pgfscope}%
\begin{pgfscope}%
\pgfsys@transformshift{6.967325in}{6.060410in}%
\pgfsys@useobject{currentmarker}{}%
\end{pgfscope}%
\begin{pgfscope}%
\pgfsys@transformshift{6.972167in}{6.060410in}%
\pgfsys@useobject{currentmarker}{}%
\end{pgfscope}%
\begin{pgfscope}%
\pgfsys@transformshift{6.976997in}{6.060658in}%
\pgfsys@useobject{currentmarker}{}%
\end{pgfscope}%
\begin{pgfscope}%
\pgfsys@transformshift{6.981818in}{6.060962in}%
\pgfsys@useobject{currentmarker}{}%
\end{pgfscope}%
\begin{pgfscope}%
\pgfsys@transformshift{6.986628in}{6.061775in}%
\pgfsys@useobject{currentmarker}{}%
\end{pgfscope}%
\begin{pgfscope}%
\pgfsys@transformshift{6.991427in}{6.061868in}%
\pgfsys@useobject{currentmarker}{}%
\end{pgfscope}%
\begin{pgfscope}%
\pgfsys@transformshift{6.996217in}{6.061941in}%
\pgfsys@useobject{currentmarker}{}%
\end{pgfscope}%
\begin{pgfscope}%
\pgfsys@transformshift{7.000997in}{6.062070in}%
\pgfsys@useobject{currentmarker}{}%
\end{pgfscope}%
\begin{pgfscope}%
\pgfsys@transformshift{7.005766in}{6.062694in}%
\pgfsys@useobject{currentmarker}{}%
\end{pgfscope}%
\begin{pgfscope}%
\pgfsys@transformshift{7.010526in}{6.063343in}%
\pgfsys@useobject{currentmarker}{}%
\end{pgfscope}%
\begin{pgfscope}%
\pgfsys@transformshift{7.015275in}{6.063491in}%
\pgfsys@useobject{currentmarker}{}%
\end{pgfscope}%
\begin{pgfscope}%
\pgfsys@transformshift{7.020015in}{6.064213in}%
\pgfsys@useobject{currentmarker}{}%
\end{pgfscope}%
\begin{pgfscope}%
\pgfsys@transformshift{7.024745in}{6.064238in}%
\pgfsys@useobject{currentmarker}{}%
\end{pgfscope}%
\begin{pgfscope}%
\pgfsys@transformshift{7.029466in}{6.064259in}%
\pgfsys@useobject{currentmarker}{}%
\end{pgfscope}%
\begin{pgfscope}%
\pgfsys@transformshift{7.034176in}{6.064266in}%
\pgfsys@useobject{currentmarker}{}%
\end{pgfscope}%
\begin{pgfscope}%
\pgfsys@transformshift{7.038878in}{6.064310in}%
\pgfsys@useobject{currentmarker}{}%
\end{pgfscope}%
\begin{pgfscope}%
\pgfsys@transformshift{7.043570in}{6.064606in}%
\pgfsys@useobject{currentmarker}{}%
\end{pgfscope}%
\begin{pgfscope}%
\pgfsys@transformshift{7.048252in}{6.064864in}%
\pgfsys@useobject{currentmarker}{}%
\end{pgfscope}%
\begin{pgfscope}%
\pgfsys@transformshift{7.052925in}{6.065258in}%
\pgfsys@useobject{currentmarker}{}%
\end{pgfscope}%
\begin{pgfscope}%
\pgfsys@transformshift{7.057589in}{6.065287in}%
\pgfsys@useobject{currentmarker}{}%
\end{pgfscope}%
\begin{pgfscope}%
\pgfsys@transformshift{7.062243in}{6.065333in}%
\pgfsys@useobject{currentmarker}{}%
\end{pgfscope}%
\begin{pgfscope}%
\pgfsys@transformshift{7.066888in}{6.065570in}%
\pgfsys@useobject{currentmarker}{}%
\end{pgfscope}%
\begin{pgfscope}%
\pgfsys@transformshift{7.071525in}{6.066495in}%
\pgfsys@useobject{currentmarker}{}%
\end{pgfscope}%
\begin{pgfscope}%
\pgfsys@transformshift{7.076152in}{6.066997in}%
\pgfsys@useobject{currentmarker}{}%
\end{pgfscope}%
\begin{pgfscope}%
\pgfsys@transformshift{7.080770in}{6.067074in}%
\pgfsys@useobject{currentmarker}{}%
\end{pgfscope}%
\begin{pgfscope}%
\pgfsys@transformshift{7.085379in}{6.067150in}%
\pgfsys@useobject{currentmarker}{}%
\end{pgfscope}%
\begin{pgfscope}%
\pgfsys@transformshift{7.089980in}{6.067240in}%
\pgfsys@useobject{currentmarker}{}%
\end{pgfscope}%
\begin{pgfscope}%
\pgfsys@transformshift{7.094571in}{6.067403in}%
\pgfsys@useobject{currentmarker}{}%
\end{pgfscope}%
\begin{pgfscope}%
\pgfsys@transformshift{7.099154in}{6.067621in}%
\pgfsys@useobject{currentmarker}{}%
\end{pgfscope}%
\begin{pgfscope}%
\pgfsys@transformshift{7.103728in}{6.067997in}%
\pgfsys@useobject{currentmarker}{}%
\end{pgfscope}%
\begin{pgfscope}%
\pgfsys@transformshift{7.108294in}{6.068498in}%
\pgfsys@useobject{currentmarker}{}%
\end{pgfscope}%
\begin{pgfscope}%
\pgfsys@transformshift{7.112851in}{6.068740in}%
\pgfsys@useobject{currentmarker}{}%
\end{pgfscope}%
\begin{pgfscope}%
\pgfsys@transformshift{7.117399in}{6.068779in}%
\pgfsys@useobject{currentmarker}{}%
\end{pgfscope}%
\begin{pgfscope}%
\pgfsys@transformshift{7.121939in}{6.069107in}%
\pgfsys@useobject{currentmarker}{}%
\end{pgfscope}%
\begin{pgfscope}%
\pgfsys@transformshift{7.126471in}{6.069793in}%
\pgfsys@useobject{currentmarker}{}%
\end{pgfscope}%
\begin{pgfscope}%
\pgfsys@transformshift{7.130994in}{6.070455in}%
\pgfsys@useobject{currentmarker}{}%
\end{pgfscope}%
\begin{pgfscope}%
\pgfsys@transformshift{7.135509in}{6.070538in}%
\pgfsys@useobject{currentmarker}{}%
\end{pgfscope}%
\begin{pgfscope}%
\pgfsys@transformshift{7.140016in}{6.071080in}%
\pgfsys@useobject{currentmarker}{}%
\end{pgfscope}%
\begin{pgfscope}%
\pgfsys@transformshift{7.144514in}{6.071318in}%
\pgfsys@useobject{currentmarker}{}%
\end{pgfscope}%
\begin{pgfscope}%
\pgfsys@transformshift{7.149004in}{6.071428in}%
\pgfsys@useobject{currentmarker}{}%
\end{pgfscope}%
\begin{pgfscope}%
\pgfsys@transformshift{7.153487in}{6.071811in}%
\pgfsys@useobject{currentmarker}{}%
\end{pgfscope}%
\begin{pgfscope}%
\pgfsys@transformshift{7.157961in}{6.071841in}%
\pgfsys@useobject{currentmarker}{}%
\end{pgfscope}%
\begin{pgfscope}%
\pgfsys@transformshift{7.162427in}{6.071850in}%
\pgfsys@useobject{currentmarker}{}%
\end{pgfscope}%
\begin{pgfscope}%
\pgfsys@transformshift{7.166886in}{6.072251in}%
\pgfsys@useobject{currentmarker}{}%
\end{pgfscope}%
\begin{pgfscope}%
\pgfsys@transformshift{7.171336in}{6.072498in}%
\pgfsys@useobject{currentmarker}{}%
\end{pgfscope}%
\begin{pgfscope}%
\pgfsys@transformshift{7.175779in}{6.073380in}%
\pgfsys@useobject{currentmarker}{}%
\end{pgfscope}%
\begin{pgfscope}%
\pgfsys@transformshift{7.180213in}{6.073462in}%
\pgfsys@useobject{currentmarker}{}%
\end{pgfscope}%
\begin{pgfscope}%
\pgfsys@transformshift{7.184641in}{6.073598in}%
\pgfsys@useobject{currentmarker}{}%
\end{pgfscope}%
\begin{pgfscope}%
\pgfsys@transformshift{7.189060in}{6.074101in}%
\pgfsys@useobject{currentmarker}{}%
\end{pgfscope}%
\begin{pgfscope}%
\pgfsys@transformshift{7.193472in}{6.074453in}%
\pgfsys@useobject{currentmarker}{}%
\end{pgfscope}%
\begin{pgfscope}%
\pgfsys@transformshift{7.197876in}{6.074698in}%
\pgfsys@useobject{currentmarker}{}%
\end{pgfscope}%
\begin{pgfscope}%
\pgfsys@transformshift{7.202273in}{6.074778in}%
\pgfsys@useobject{currentmarker}{}%
\end{pgfscope}%
\begin{pgfscope}%
\pgfsys@transformshift{7.206662in}{6.075184in}%
\pgfsys@useobject{currentmarker}{}%
\end{pgfscope}%
\begin{pgfscope}%
\pgfsys@transformshift{7.211044in}{6.075436in}%
\pgfsys@useobject{currentmarker}{}%
\end{pgfscope}%
\begin{pgfscope}%
\pgfsys@transformshift{7.215418in}{6.076001in}%
\pgfsys@useobject{currentmarker}{}%
\end{pgfscope}%
\begin{pgfscope}%
\pgfsys@transformshift{7.219785in}{6.076265in}%
\pgfsys@useobject{currentmarker}{}%
\end{pgfscope}%
\begin{pgfscope}%
\pgfsys@transformshift{7.224145in}{6.076702in}%
\pgfsys@useobject{currentmarker}{}%
\end{pgfscope}%
\begin{pgfscope}%
\pgfsys@transformshift{7.228497in}{6.078165in}%
\pgfsys@useobject{currentmarker}{}%
\end{pgfscope}%
\begin{pgfscope}%
\pgfsys@transformshift{7.232843in}{6.078188in}%
\pgfsys@useobject{currentmarker}{}%
\end{pgfscope}%
\begin{pgfscope}%
\pgfsys@transformshift{7.237181in}{6.078461in}%
\pgfsys@useobject{currentmarker}{}%
\end{pgfscope}%
\begin{pgfscope}%
\pgfsys@transformshift{7.241512in}{6.079796in}%
\pgfsys@useobject{currentmarker}{}%
\end{pgfscope}%
\begin{pgfscope}%
\pgfsys@transformshift{7.245836in}{6.079812in}%
\pgfsys@useobject{currentmarker}{}%
\end{pgfscope}%
\begin{pgfscope}%
\pgfsys@transformshift{7.250152in}{6.079967in}%
\pgfsys@useobject{currentmarker}{}%
\end{pgfscope}%
\begin{pgfscope}%
\pgfsys@transformshift{7.254462in}{6.080459in}%
\pgfsys@useobject{currentmarker}{}%
\end{pgfscope}%
\begin{pgfscope}%
\pgfsys@transformshift{7.258765in}{6.080573in}%
\pgfsys@useobject{currentmarker}{}%
\end{pgfscope}%
\begin{pgfscope}%
\pgfsys@transformshift{7.263061in}{6.080716in}%
\pgfsys@useobject{currentmarker}{}%
\end{pgfscope}%
\begin{pgfscope}%
\pgfsys@transformshift{7.267350in}{6.081253in}%
\pgfsys@useobject{currentmarker}{}%
\end{pgfscope}%
\begin{pgfscope}%
\pgfsys@transformshift{7.271633in}{6.081432in}%
\pgfsys@useobject{currentmarker}{}%
\end{pgfscope}%
\begin{pgfscope}%
\pgfsys@transformshift{7.275908in}{6.081769in}%
\pgfsys@useobject{currentmarker}{}%
\end{pgfscope}%
\begin{pgfscope}%
\pgfsys@transformshift{7.280177in}{6.081842in}%
\pgfsys@useobject{currentmarker}{}%
\end{pgfscope}%
\begin{pgfscope}%
\pgfsys@transformshift{7.284439in}{6.081946in}%
\pgfsys@useobject{currentmarker}{}%
\end{pgfscope}%
\begin{pgfscope}%
\pgfsys@transformshift{7.288695in}{6.081955in}%
\pgfsys@useobject{currentmarker}{}%
\end{pgfscope}%
\begin{pgfscope}%
\pgfsys@transformshift{7.292944in}{6.082006in}%
\pgfsys@useobject{currentmarker}{}%
\end{pgfscope}%
\begin{pgfscope}%
\pgfsys@transformshift{7.297186in}{6.082311in}%
\pgfsys@useobject{currentmarker}{}%
\end{pgfscope}%
\begin{pgfscope}%
\pgfsys@transformshift{7.301422in}{6.082571in}%
\pgfsys@useobject{currentmarker}{}%
\end{pgfscope}%
\begin{pgfscope}%
\pgfsys@transformshift{7.305651in}{6.082610in}%
\pgfsys@useobject{currentmarker}{}%
\end{pgfscope}%
\begin{pgfscope}%
\pgfsys@transformshift{7.309874in}{6.082792in}%
\pgfsys@useobject{currentmarker}{}%
\end{pgfscope}%
\begin{pgfscope}%
\pgfsys@transformshift{7.314090in}{6.082822in}%
\pgfsys@useobject{currentmarker}{}%
\end{pgfscope}%
\begin{pgfscope}%
\pgfsys@transformshift{7.318300in}{6.084063in}%
\pgfsys@useobject{currentmarker}{}%
\end{pgfscope}%
\begin{pgfscope}%
\pgfsys@transformshift{7.322504in}{6.084246in}%
\pgfsys@useobject{currentmarker}{}%
\end{pgfscope}%
\begin{pgfscope}%
\pgfsys@transformshift{7.326701in}{6.084840in}%
\pgfsys@useobject{currentmarker}{}%
\end{pgfscope}%
\begin{pgfscope}%
\pgfsys@transformshift{7.330892in}{6.085082in}%
\pgfsys@useobject{currentmarker}{}%
\end{pgfscope}%
\begin{pgfscope}%
\pgfsys@transformshift{7.335077in}{6.085084in}%
\pgfsys@useobject{currentmarker}{}%
\end{pgfscope}%
\begin{pgfscope}%
\pgfsys@transformshift{7.339256in}{6.085412in}%
\pgfsys@useobject{currentmarker}{}%
\end{pgfscope}%
\begin{pgfscope}%
\pgfsys@transformshift{7.343429in}{6.085647in}%
\pgfsys@useobject{currentmarker}{}%
\end{pgfscope}%
\begin{pgfscope}%
\pgfsys@transformshift{7.347595in}{6.085654in}%
\pgfsys@useobject{currentmarker}{}%
\end{pgfscope}%
\begin{pgfscope}%
\pgfsys@transformshift{7.351755in}{6.085743in}%
\pgfsys@useobject{currentmarker}{}%
\end{pgfscope}%
\begin{pgfscope}%
\pgfsys@transformshift{7.355910in}{6.085874in}%
\pgfsys@useobject{currentmarker}{}%
\end{pgfscope}%
\begin{pgfscope}%
\pgfsys@transformshift{7.360058in}{6.086043in}%
\pgfsys@useobject{currentmarker}{}%
\end{pgfscope}%
\begin{pgfscope}%
\pgfsys@transformshift{7.364200in}{6.086837in}%
\pgfsys@useobject{currentmarker}{}%
\end{pgfscope}%
\begin{pgfscope}%
\pgfsys@transformshift{7.368337in}{6.086842in}%
\pgfsys@useobject{currentmarker}{}%
\end{pgfscope}%
\begin{pgfscope}%
\pgfsys@transformshift{7.372467in}{6.087621in}%
\pgfsys@useobject{currentmarker}{}%
\end{pgfscope}%
\begin{pgfscope}%
\pgfsys@transformshift{7.376592in}{6.087748in}%
\pgfsys@useobject{currentmarker}{}%
\end{pgfscope}%
\begin{pgfscope}%
\pgfsys@transformshift{7.380711in}{6.087827in}%
\pgfsys@useobject{currentmarker}{}%
\end{pgfscope}%
\begin{pgfscope}%
\pgfsys@transformshift{7.384824in}{6.087978in}%
\pgfsys@useobject{currentmarker}{}%
\end{pgfscope}%
\begin{pgfscope}%
\pgfsys@transformshift{7.388931in}{6.088011in}%
\pgfsys@useobject{currentmarker}{}%
\end{pgfscope}%
\begin{pgfscope}%
\pgfsys@transformshift{7.393033in}{6.088471in}%
\pgfsys@useobject{currentmarker}{}%
\end{pgfscope}%
\begin{pgfscope}%
\pgfsys@transformshift{7.397128in}{6.088688in}%
\pgfsys@useobject{currentmarker}{}%
\end{pgfscope}%
\begin{pgfscope}%
\pgfsys@transformshift{7.401219in}{6.089364in}%
\pgfsys@useobject{currentmarker}{}%
\end{pgfscope}%
\begin{pgfscope}%
\pgfsys@transformshift{7.405303in}{6.089464in}%
\pgfsys@useobject{currentmarker}{}%
\end{pgfscope}%
\begin{pgfscope}%
\pgfsys@transformshift{7.409382in}{6.090013in}%
\pgfsys@useobject{currentmarker}{}%
\end{pgfscope}%
\begin{pgfscope}%
\pgfsys@transformshift{7.413455in}{6.090084in}%
\pgfsys@useobject{currentmarker}{}%
\end{pgfscope}%
\begin{pgfscope}%
\pgfsys@transformshift{7.417523in}{6.090271in}%
\pgfsys@useobject{currentmarker}{}%
\end{pgfscope}%
\begin{pgfscope}%
\pgfsys@transformshift{7.421586in}{6.090489in}%
\pgfsys@useobject{currentmarker}{}%
\end{pgfscope}%
\begin{pgfscope}%
\pgfsys@transformshift{7.425643in}{6.090505in}%
\pgfsys@useobject{currentmarker}{}%
\end{pgfscope}%
\begin{pgfscope}%
\pgfsys@transformshift{7.429694in}{6.090704in}%
\pgfsys@useobject{currentmarker}{}%
\end{pgfscope}%
\begin{pgfscope}%
\pgfsys@transformshift{7.433740in}{6.090839in}%
\pgfsys@useobject{currentmarker}{}%
\end{pgfscope}%
\begin{pgfscope}%
\pgfsys@transformshift{7.437781in}{6.090841in}%
\pgfsys@useobject{currentmarker}{}%
\end{pgfscope}%
\begin{pgfscope}%
\pgfsys@transformshift{7.441816in}{6.091142in}%
\pgfsys@useobject{currentmarker}{}%
\end{pgfscope}%
\begin{pgfscope}%
\pgfsys@transformshift{7.445846in}{6.091289in}%
\pgfsys@useobject{currentmarker}{}%
\end{pgfscope}%
\begin{pgfscope}%
\pgfsys@transformshift{7.449871in}{6.091370in}%
\pgfsys@useobject{currentmarker}{}%
\end{pgfscope}%
\begin{pgfscope}%
\pgfsys@transformshift{7.453890in}{6.091514in}%
\pgfsys@useobject{currentmarker}{}%
\end{pgfscope}%
\begin{pgfscope}%
\pgfsys@transformshift{7.457905in}{6.091750in}%
\pgfsys@useobject{currentmarker}{}%
\end{pgfscope}%
\begin{pgfscope}%
\pgfsys@transformshift{7.461914in}{6.091847in}%
\pgfsys@useobject{currentmarker}{}%
\end{pgfscope}%
\begin{pgfscope}%
\pgfsys@transformshift{7.465918in}{6.092034in}%
\pgfsys@useobject{currentmarker}{}%
\end{pgfscope}%
\begin{pgfscope}%
\pgfsys@transformshift{7.469917in}{6.092372in}%
\pgfsys@useobject{currentmarker}{}%
\end{pgfscope}%
\begin{pgfscope}%
\pgfsys@transformshift{7.473910in}{6.092431in}%
\pgfsys@useobject{currentmarker}{}%
\end{pgfscope}%
\begin{pgfscope}%
\pgfsys@transformshift{7.477899in}{6.093027in}%
\pgfsys@useobject{currentmarker}{}%
\end{pgfscope}%
\begin{pgfscope}%
\pgfsys@transformshift{7.481883in}{6.093338in}%
\pgfsys@useobject{currentmarker}{}%
\end{pgfscope}%
\begin{pgfscope}%
\pgfsys@transformshift{7.485861in}{6.093543in}%
\pgfsys@useobject{currentmarker}{}%
\end{pgfscope}%
\begin{pgfscope}%
\pgfsys@transformshift{7.489835in}{6.093627in}%
\pgfsys@useobject{currentmarker}{}%
\end{pgfscope}%
\begin{pgfscope}%
\pgfsys@transformshift{7.493804in}{6.093627in}%
\pgfsys@useobject{currentmarker}{}%
\end{pgfscope}%
\begin{pgfscope}%
\pgfsys@transformshift{7.497767in}{6.093993in}%
\pgfsys@useobject{currentmarker}{}%
\end{pgfscope}%
\begin{pgfscope}%
\pgfsys@transformshift{7.501726in}{6.094000in}%
\pgfsys@useobject{currentmarker}{}%
\end{pgfscope}%
\begin{pgfscope}%
\pgfsys@transformshift{7.505680in}{6.094148in}%
\pgfsys@useobject{currentmarker}{}%
\end{pgfscope}%
\begin{pgfscope}%
\pgfsys@transformshift{7.509629in}{6.094491in}%
\pgfsys@useobject{currentmarker}{}%
\end{pgfscope}%
\begin{pgfscope}%
\pgfsys@transformshift{7.513574in}{6.094615in}%
\pgfsys@useobject{currentmarker}{}%
\end{pgfscope}%
\begin{pgfscope}%
\pgfsys@transformshift{7.517513in}{6.094700in}%
\pgfsys@useobject{currentmarker}{}%
\end{pgfscope}%
\begin{pgfscope}%
\pgfsys@transformshift{7.521448in}{6.094809in}%
\pgfsys@useobject{currentmarker}{}%
\end{pgfscope}%
\begin{pgfscope}%
\pgfsys@transformshift{7.525378in}{6.095148in}%
\pgfsys@useobject{currentmarker}{}%
\end{pgfscope}%
\begin{pgfscope}%
\pgfsys@transformshift{7.529303in}{6.095190in}%
\pgfsys@useobject{currentmarker}{}%
\end{pgfscope}%
\begin{pgfscope}%
\pgfsys@transformshift{7.533224in}{6.095407in}%
\pgfsys@useobject{currentmarker}{}%
\end{pgfscope}%
\begin{pgfscope}%
\pgfsys@transformshift{7.537140in}{6.095420in}%
\pgfsys@useobject{currentmarker}{}%
\end{pgfscope}%
\begin{pgfscope}%
\pgfsys@transformshift{7.541052in}{6.095739in}%
\pgfsys@useobject{currentmarker}{}%
\end{pgfscope}%
\begin{pgfscope}%
\pgfsys@transformshift{7.544958in}{6.095776in}%
\pgfsys@useobject{currentmarker}{}%
\end{pgfscope}%
\begin{pgfscope}%
\pgfsys@transformshift{7.548861in}{6.095906in}%
\pgfsys@useobject{currentmarker}{}%
\end{pgfscope}%
\begin{pgfscope}%
\pgfsys@transformshift{7.552758in}{6.095943in}%
\pgfsys@useobject{currentmarker}{}%
\end{pgfscope}%
\begin{pgfscope}%
\pgfsys@transformshift{7.556651in}{6.096089in}%
\pgfsys@useobject{currentmarker}{}%
\end{pgfscope}%
\begin{pgfscope}%
\pgfsys@transformshift{7.560540in}{6.096109in}%
\pgfsys@useobject{currentmarker}{}%
\end{pgfscope}%
\begin{pgfscope}%
\pgfsys@transformshift{7.564424in}{6.096255in}%
\pgfsys@useobject{currentmarker}{}%
\end{pgfscope}%
\begin{pgfscope}%
\pgfsys@transformshift{7.568304in}{6.096570in}%
\pgfsys@useobject{currentmarker}{}%
\end{pgfscope}%
\begin{pgfscope}%
\pgfsys@transformshift{7.572179in}{6.096660in}%
\pgfsys@useobject{currentmarker}{}%
\end{pgfscope}%
\begin{pgfscope}%
\pgfsys@transformshift{7.576050in}{6.096975in}%
\pgfsys@useobject{currentmarker}{}%
\end{pgfscope}%
\begin{pgfscope}%
\pgfsys@transformshift{7.579917in}{6.097026in}%
\pgfsys@useobject{currentmarker}{}%
\end{pgfscope}%
\begin{pgfscope}%
\pgfsys@transformshift{7.583779in}{6.097334in}%
\pgfsys@useobject{currentmarker}{}%
\end{pgfscope}%
\begin{pgfscope}%
\pgfsys@transformshift{7.587637in}{6.098033in}%
\pgfsys@useobject{currentmarker}{}%
\end{pgfscope}%
\begin{pgfscope}%
\pgfsys@transformshift{7.591491in}{6.098932in}%
\pgfsys@useobject{currentmarker}{}%
\end{pgfscope}%
\begin{pgfscope}%
\pgfsys@transformshift{7.595340in}{6.098942in}%
\pgfsys@useobject{currentmarker}{}%
\end{pgfscope}%
\begin{pgfscope}%
\pgfsys@transformshift{7.599185in}{6.100148in}%
\pgfsys@useobject{currentmarker}{}%
\end{pgfscope}%
\begin{pgfscope}%
\pgfsys@transformshift{7.603026in}{6.100365in}%
\pgfsys@useobject{currentmarker}{}%
\end{pgfscope}%
\begin{pgfscope}%
\pgfsys@transformshift{7.606863in}{6.100556in}%
\pgfsys@useobject{currentmarker}{}%
\end{pgfscope}%
\begin{pgfscope}%
\pgfsys@transformshift{7.610695in}{6.100723in}%
\pgfsys@useobject{currentmarker}{}%
\end{pgfscope}%
\begin{pgfscope}%
\pgfsys@transformshift{7.614523in}{6.100748in}%
\pgfsys@useobject{currentmarker}{}%
\end{pgfscope}%
\begin{pgfscope}%
\pgfsys@transformshift{7.618348in}{6.101036in}%
\pgfsys@useobject{currentmarker}{}%
\end{pgfscope}%
\begin{pgfscope}%
\pgfsys@transformshift{7.622168in}{6.101131in}%
\pgfsys@useobject{currentmarker}{}%
\end{pgfscope}%
\begin{pgfscope}%
\pgfsys@transformshift{7.625984in}{6.101174in}%
\pgfsys@useobject{currentmarker}{}%
\end{pgfscope}%
\begin{pgfscope}%
\pgfsys@transformshift{7.629796in}{6.101305in}%
\pgfsys@useobject{currentmarker}{}%
\end{pgfscope}%
\begin{pgfscope}%
\pgfsys@transformshift{7.633604in}{6.101576in}%
\pgfsys@useobject{currentmarker}{}%
\end{pgfscope}%
\begin{pgfscope}%
\pgfsys@transformshift{7.637407in}{6.102158in}%
\pgfsys@useobject{currentmarker}{}%
\end{pgfscope}%
\begin{pgfscope}%
\pgfsys@transformshift{7.641207in}{6.102202in}%
\pgfsys@useobject{currentmarker}{}%
\end{pgfscope}%
\begin{pgfscope}%
\pgfsys@transformshift{7.645003in}{6.102643in}%
\pgfsys@useobject{currentmarker}{}%
\end{pgfscope}%
\begin{pgfscope}%
\pgfsys@transformshift{7.648795in}{6.102734in}%
\pgfsys@useobject{currentmarker}{}%
\end{pgfscope}%
\begin{pgfscope}%
\pgfsys@transformshift{7.652583in}{6.102784in}%
\pgfsys@useobject{currentmarker}{}%
\end{pgfscope}%
\begin{pgfscope}%
\pgfsys@transformshift{7.656368in}{6.103305in}%
\pgfsys@useobject{currentmarker}{}%
\end{pgfscope}%
\begin{pgfscope}%
\pgfsys@transformshift{7.660148in}{6.104174in}%
\pgfsys@useobject{currentmarker}{}%
\end{pgfscope}%
\begin{pgfscope}%
\pgfsys@transformshift{7.663924in}{6.104354in}%
\pgfsys@useobject{currentmarker}{}%
\end{pgfscope}%
\begin{pgfscope}%
\pgfsys@transformshift{7.667697in}{6.104609in}%
\pgfsys@useobject{currentmarker}{}%
\end{pgfscope}%
\begin{pgfscope}%
\pgfsys@transformshift{7.671466in}{6.104662in}%
\pgfsys@useobject{currentmarker}{}%
\end{pgfscope}%
\begin{pgfscope}%
\pgfsys@transformshift{7.675231in}{6.104950in}%
\pgfsys@useobject{currentmarker}{}%
\end{pgfscope}%
\begin{pgfscope}%
\pgfsys@transformshift{7.678992in}{6.105001in}%
\pgfsys@useobject{currentmarker}{}%
\end{pgfscope}%
\begin{pgfscope}%
\pgfsys@transformshift{7.682749in}{6.105023in}%
\pgfsys@useobject{currentmarker}{}%
\end{pgfscope}%
\begin{pgfscope}%
\pgfsys@transformshift{7.686503in}{6.105079in}%
\pgfsys@useobject{currentmarker}{}%
\end{pgfscope}%
\begin{pgfscope}%
\pgfsys@transformshift{7.690253in}{6.105713in}%
\pgfsys@useobject{currentmarker}{}%
\end{pgfscope}%
\begin{pgfscope}%
\pgfsys@transformshift{7.693999in}{6.105792in}%
\pgfsys@useobject{currentmarker}{}%
\end{pgfscope}%
\begin{pgfscope}%
\pgfsys@transformshift{7.697742in}{6.106398in}%
\pgfsys@useobject{currentmarker}{}%
\end{pgfscope}%
\begin{pgfscope}%
\pgfsys@transformshift{7.701481in}{6.106569in}%
\pgfsys@useobject{currentmarker}{}%
\end{pgfscope}%
\begin{pgfscope}%
\pgfsys@transformshift{7.705216in}{6.106800in}%
\pgfsys@useobject{currentmarker}{}%
\end{pgfscope}%
\begin{pgfscope}%
\pgfsys@transformshift{7.708948in}{6.107084in}%
\pgfsys@useobject{currentmarker}{}%
\end{pgfscope}%
\begin{pgfscope}%
\pgfsys@transformshift{7.712676in}{6.107297in}%
\pgfsys@useobject{currentmarker}{}%
\end{pgfscope}%
\begin{pgfscope}%
\pgfsys@transformshift{7.716401in}{6.107312in}%
\pgfsys@useobject{currentmarker}{}%
\end{pgfscope}%
\begin{pgfscope}%
\pgfsys@transformshift{7.720122in}{6.107374in}%
\pgfsys@useobject{currentmarker}{}%
\end{pgfscope}%
\begin{pgfscope}%
\pgfsys@transformshift{7.723840in}{6.107842in}%
\pgfsys@useobject{currentmarker}{}%
\end{pgfscope}%
\begin{pgfscope}%
\pgfsys@transformshift{7.727554in}{6.108033in}%
\pgfsys@useobject{currentmarker}{}%
\end{pgfscope}%
\begin{pgfscope}%
\pgfsys@transformshift{7.731264in}{6.108096in}%
\pgfsys@useobject{currentmarker}{}%
\end{pgfscope}%
\begin{pgfscope}%
\pgfsys@transformshift{7.734971in}{6.108110in}%
\pgfsys@useobject{currentmarker}{}%
\end{pgfscope}%
\begin{pgfscope}%
\pgfsys@transformshift{7.738675in}{6.108374in}%
\pgfsys@useobject{currentmarker}{}%
\end{pgfscope}%
\begin{pgfscope}%
\pgfsys@transformshift{7.742375in}{6.108425in}%
\pgfsys@useobject{currentmarker}{}%
\end{pgfscope}%
\begin{pgfscope}%
\pgfsys@transformshift{7.746072in}{6.108501in}%
\pgfsys@useobject{currentmarker}{}%
\end{pgfscope}%
\begin{pgfscope}%
\pgfsys@transformshift{7.749765in}{6.108845in}%
\pgfsys@useobject{currentmarker}{}%
\end{pgfscope}%
\begin{pgfscope}%
\pgfsys@transformshift{7.753455in}{6.109226in}%
\pgfsys@useobject{currentmarker}{}%
\end{pgfscope}%
\begin{pgfscope}%
\pgfsys@transformshift{7.757142in}{6.109262in}%
\pgfsys@useobject{currentmarker}{}%
\end{pgfscope}%
\begin{pgfscope}%
\pgfsys@transformshift{7.760825in}{6.109418in}%
\pgfsys@useobject{currentmarker}{}%
\end{pgfscope}%
\begin{pgfscope}%
\pgfsys@transformshift{7.764505in}{6.109500in}%
\pgfsys@useobject{currentmarker}{}%
\end{pgfscope}%
\begin{pgfscope}%
\pgfsys@transformshift{7.768182in}{6.110043in}%
\pgfsys@useobject{currentmarker}{}%
\end{pgfscope}%
\begin{pgfscope}%
\pgfsys@transformshift{7.771856in}{6.110134in}%
\pgfsys@useobject{currentmarker}{}%
\end{pgfscope}%
\begin{pgfscope}%
\pgfsys@transformshift{7.775526in}{6.110268in}%
\pgfsys@useobject{currentmarker}{}%
\end{pgfscope}%
\begin{pgfscope}%
\pgfsys@transformshift{7.779193in}{6.110508in}%
\pgfsys@useobject{currentmarker}{}%
\end{pgfscope}%
\begin{pgfscope}%
\pgfsys@transformshift{7.782856in}{6.110606in}%
\pgfsys@useobject{currentmarker}{}%
\end{pgfscope}%
\begin{pgfscope}%
\pgfsys@transformshift{7.786517in}{6.110944in}%
\pgfsys@useobject{currentmarker}{}%
\end{pgfscope}%
\begin{pgfscope}%
\pgfsys@transformshift{7.790174in}{6.111590in}%
\pgfsys@useobject{currentmarker}{}%
\end{pgfscope}%
\begin{pgfscope}%
\pgfsys@transformshift{7.793829in}{6.112466in}%
\pgfsys@useobject{currentmarker}{}%
\end{pgfscope}%
\begin{pgfscope}%
\pgfsys@transformshift{7.797480in}{6.112725in}%
\pgfsys@useobject{currentmarker}{}%
\end{pgfscope}%
\begin{pgfscope}%
\pgfsys@transformshift{7.801128in}{6.112915in}%
\pgfsys@useobject{currentmarker}{}%
\end{pgfscope}%
\begin{pgfscope}%
\pgfsys@transformshift{7.804772in}{6.113013in}%
\pgfsys@useobject{currentmarker}{}%
\end{pgfscope}%
\begin{pgfscope}%
\pgfsys@transformshift{7.808414in}{6.113305in}%
\pgfsys@useobject{currentmarker}{}%
\end{pgfscope}%
\begin{pgfscope}%
\pgfsys@transformshift{7.812053in}{6.113712in}%
\pgfsys@useobject{currentmarker}{}%
\end{pgfscope}%
\begin{pgfscope}%
\pgfsys@transformshift{7.815688in}{6.114238in}%
\pgfsys@useobject{currentmarker}{}%
\end{pgfscope}%
\begin{pgfscope}%
\pgfsys@transformshift{7.819321in}{6.114627in}%
\pgfsys@useobject{currentmarker}{}%
\end{pgfscope}%
\begin{pgfscope}%
\pgfsys@transformshift{7.822950in}{6.115105in}%
\pgfsys@useobject{currentmarker}{}%
\end{pgfscope}%
\begin{pgfscope}%
\pgfsys@transformshift{7.826577in}{6.115277in}%
\pgfsys@useobject{currentmarker}{}%
\end{pgfscope}%
\begin{pgfscope}%
\pgfsys@transformshift{7.830200in}{6.115890in}%
\pgfsys@useobject{currentmarker}{}%
\end{pgfscope}%
\begin{pgfscope}%
\pgfsys@transformshift{7.833821in}{6.116031in}%
\pgfsys@useobject{currentmarker}{}%
\end{pgfscope}%
\begin{pgfscope}%
\pgfsys@transformshift{7.837438in}{6.116166in}%
\pgfsys@useobject{currentmarker}{}%
\end{pgfscope}%
\begin{pgfscope}%
\pgfsys@transformshift{7.841053in}{6.116233in}%
\pgfsys@useobject{currentmarker}{}%
\end{pgfscope}%
\begin{pgfscope}%
\pgfsys@transformshift{7.844665in}{6.116507in}%
\pgfsys@useobject{currentmarker}{}%
\end{pgfscope}%
\begin{pgfscope}%
\pgfsys@transformshift{7.848274in}{6.116645in}%
\pgfsys@useobject{currentmarker}{}%
\end{pgfscope}%
\begin{pgfscope}%
\pgfsys@transformshift{7.851879in}{6.116652in}%
\pgfsys@useobject{currentmarker}{}%
\end{pgfscope}%
\begin{pgfscope}%
\pgfsys@transformshift{7.855482in}{6.116769in}%
\pgfsys@useobject{currentmarker}{}%
\end{pgfscope}%
\begin{pgfscope}%
\pgfsys@transformshift{7.859083in}{6.116800in}%
\pgfsys@useobject{currentmarker}{}%
\end{pgfscope}%
\begin{pgfscope}%
\pgfsys@transformshift{7.862680in}{6.117172in}%
\pgfsys@useobject{currentmarker}{}%
\end{pgfscope}%
\begin{pgfscope}%
\pgfsys@transformshift{7.866275in}{6.117834in}%
\pgfsys@useobject{currentmarker}{}%
\end{pgfscope}%
\begin{pgfscope}%
\pgfsys@transformshift{7.869866in}{6.118044in}%
\pgfsys@useobject{currentmarker}{}%
\end{pgfscope}%
\begin{pgfscope}%
\pgfsys@transformshift{7.873455in}{6.118866in}%
\pgfsys@useobject{currentmarker}{}%
\end{pgfscope}%
\begin{pgfscope}%
\pgfsys@transformshift{7.877041in}{6.119004in}%
\pgfsys@useobject{currentmarker}{}%
\end{pgfscope}%
\begin{pgfscope}%
\pgfsys@transformshift{7.880625in}{6.119248in}%
\pgfsys@useobject{currentmarker}{}%
\end{pgfscope}%
\begin{pgfscope}%
\pgfsys@transformshift{7.884205in}{6.119323in}%
\pgfsys@useobject{currentmarker}{}%
\end{pgfscope}%
\begin{pgfscope}%
\pgfsys@transformshift{7.887783in}{6.119426in}%
\pgfsys@useobject{currentmarker}{}%
\end{pgfscope}%
\begin{pgfscope}%
\pgfsys@transformshift{7.891359in}{6.119729in}%
\pgfsys@useobject{currentmarker}{}%
\end{pgfscope}%
\begin{pgfscope}%
\pgfsys@transformshift{7.894931in}{6.119778in}%
\pgfsys@useobject{currentmarker}{}%
\end{pgfscope}%
\begin{pgfscope}%
\pgfsys@transformshift{7.898501in}{6.119876in}%
\pgfsys@useobject{currentmarker}{}%
\end{pgfscope}%
\begin{pgfscope}%
\pgfsys@transformshift{7.902068in}{6.120230in}%
\pgfsys@useobject{currentmarker}{}%
\end{pgfscope}%
\begin{pgfscope}%
\pgfsys@transformshift{7.905633in}{6.120426in}%
\pgfsys@useobject{currentmarker}{}%
\end{pgfscope}%
\begin{pgfscope}%
\pgfsys@transformshift{7.909195in}{6.120616in}%
\pgfsys@useobject{currentmarker}{}%
\end{pgfscope}%
\begin{pgfscope}%
\pgfsys@transformshift{7.912754in}{6.120655in}%
\pgfsys@useobject{currentmarker}{}%
\end{pgfscope}%
\begin{pgfscope}%
\pgfsys@transformshift{7.916310in}{6.120760in}%
\pgfsys@useobject{currentmarker}{}%
\end{pgfscope}%
\begin{pgfscope}%
\pgfsys@transformshift{7.919865in}{6.121042in}%
\pgfsys@useobject{currentmarker}{}%
\end{pgfscope}%
\begin{pgfscope}%
\pgfsys@transformshift{7.923416in}{6.121658in}%
\pgfsys@useobject{currentmarker}{}%
\end{pgfscope}%
\begin{pgfscope}%
\pgfsys@transformshift{7.926965in}{6.121689in}%
\pgfsys@useobject{currentmarker}{}%
\end{pgfscope}%
\begin{pgfscope}%
\pgfsys@transformshift{7.930512in}{6.121740in}%
\pgfsys@useobject{currentmarker}{}%
\end{pgfscope}%
\begin{pgfscope}%
\pgfsys@transformshift{7.934055in}{6.122346in}%
\pgfsys@useobject{currentmarker}{}%
\end{pgfscope}%
\begin{pgfscope}%
\pgfsys@transformshift{7.937597in}{6.122514in}%
\pgfsys@useobject{currentmarker}{}%
\end{pgfscope}%
\begin{pgfscope}%
\pgfsys@transformshift{7.941136in}{6.122760in}%
\pgfsys@useobject{currentmarker}{}%
\end{pgfscope}%
\begin{pgfscope}%
\pgfsys@transformshift{7.944672in}{6.122910in}%
\pgfsys@useobject{currentmarker}{}%
\end{pgfscope}%
\begin{pgfscope}%
\pgfsys@transformshift{7.948206in}{6.122993in}%
\pgfsys@useobject{currentmarker}{}%
\end{pgfscope}%
\begin{pgfscope}%
\pgfsys@transformshift{7.951738in}{6.123062in}%
\pgfsys@useobject{currentmarker}{}%
\end{pgfscope}%
\begin{pgfscope}%
\pgfsys@transformshift{7.955267in}{6.123634in}%
\pgfsys@useobject{currentmarker}{}%
\end{pgfscope}%
\begin{pgfscope}%
\pgfsys@transformshift{7.958793in}{6.123712in}%
\pgfsys@useobject{currentmarker}{}%
\end{pgfscope}%
\begin{pgfscope}%
\pgfsys@transformshift{7.962318in}{6.123905in}%
\pgfsys@useobject{currentmarker}{}%
\end{pgfscope}%
\begin{pgfscope}%
\pgfsys@transformshift{7.965839in}{6.124014in}%
\pgfsys@useobject{currentmarker}{}%
\end{pgfscope}%
\begin{pgfscope}%
\pgfsys@transformshift{7.969359in}{6.124108in}%
\pgfsys@useobject{currentmarker}{}%
\end{pgfscope}%
\begin{pgfscope}%
\pgfsys@transformshift{7.972876in}{6.124470in}%
\pgfsys@useobject{currentmarker}{}%
\end{pgfscope}%
\begin{pgfscope}%
\pgfsys@transformshift{7.976391in}{6.124549in}%
\pgfsys@useobject{currentmarker}{}%
\end{pgfscope}%
\begin{pgfscope}%
\pgfsys@transformshift{7.979903in}{6.125065in}%
\pgfsys@useobject{currentmarker}{}%
\end{pgfscope}%
\begin{pgfscope}%
\pgfsys@transformshift{7.983413in}{6.125425in}%
\pgfsys@useobject{currentmarker}{}%
\end{pgfscope}%
\begin{pgfscope}%
\pgfsys@transformshift{7.986921in}{6.125460in}%
\pgfsys@useobject{currentmarker}{}%
\end{pgfscope}%
\begin{pgfscope}%
\pgfsys@transformshift{7.990427in}{6.125582in}%
\pgfsys@useobject{currentmarker}{}%
\end{pgfscope}%
\begin{pgfscope}%
\pgfsys@transformshift{7.993930in}{6.125839in}%
\pgfsys@useobject{currentmarker}{}%
\end{pgfscope}%
\begin{pgfscope}%
\pgfsys@transformshift{7.997431in}{6.126231in}%
\pgfsys@useobject{currentmarker}{}%
\end{pgfscope}%
\begin{pgfscope}%
\pgfsys@transformshift{8.000930in}{6.126907in}%
\pgfsys@useobject{currentmarker}{}%
\end{pgfscope}%
\begin{pgfscope}%
\pgfsys@transformshift{8.004426in}{6.127123in}%
\pgfsys@useobject{currentmarker}{}%
\end{pgfscope}%
\begin{pgfscope}%
\pgfsys@transformshift{8.007920in}{6.127444in}%
\pgfsys@useobject{currentmarker}{}%
\end{pgfscope}%
\begin{pgfscope}%
\pgfsys@transformshift{8.011412in}{6.127666in}%
\pgfsys@useobject{currentmarker}{}%
\end{pgfscope}%
\begin{pgfscope}%
\pgfsys@transformshift{8.014902in}{6.127728in}%
\pgfsys@useobject{currentmarker}{}%
\end{pgfscope}%
\begin{pgfscope}%
\pgfsys@transformshift{8.018390in}{6.127753in}%
\pgfsys@useobject{currentmarker}{}%
\end{pgfscope}%
\begin{pgfscope}%
\pgfsys@transformshift{8.021875in}{6.127960in}%
\pgfsys@useobject{currentmarker}{}%
\end{pgfscope}%
\begin{pgfscope}%
\pgfsys@transformshift{8.025359in}{6.128798in}%
\pgfsys@useobject{currentmarker}{}%
\end{pgfscope}%
\begin{pgfscope}%
\pgfsys@transformshift{8.028840in}{6.129083in}%
\pgfsys@useobject{currentmarker}{}%
\end{pgfscope}%
\begin{pgfscope}%
\pgfsys@transformshift{8.032319in}{6.129271in}%
\pgfsys@useobject{currentmarker}{}%
\end{pgfscope}%
\begin{pgfscope}%
\pgfsys@transformshift{8.035796in}{6.129385in}%
\pgfsys@useobject{currentmarker}{}%
\end{pgfscope}%
\begin{pgfscope}%
\pgfsys@transformshift{8.039271in}{6.129603in}%
\pgfsys@useobject{currentmarker}{}%
\end{pgfscope}%
\begin{pgfscope}%
\pgfsys@transformshift{8.042744in}{6.129647in}%
\pgfsys@useobject{currentmarker}{}%
\end{pgfscope}%
\begin{pgfscope}%
\pgfsys@transformshift{8.046214in}{6.129766in}%
\pgfsys@useobject{currentmarker}{}%
\end{pgfscope}%
\begin{pgfscope}%
\pgfsys@transformshift{8.049683in}{6.130116in}%
\pgfsys@useobject{currentmarker}{}%
\end{pgfscope}%
\begin{pgfscope}%
\pgfsys@transformshift{8.053150in}{6.130306in}%
\pgfsys@useobject{currentmarker}{}%
\end{pgfscope}%
\begin{pgfscope}%
\pgfsys@transformshift{8.056614in}{6.130396in}%
\pgfsys@useobject{currentmarker}{}%
\end{pgfscope}%
\begin{pgfscope}%
\pgfsys@transformshift{8.060077in}{6.130468in}%
\pgfsys@useobject{currentmarker}{}%
\end{pgfscope}%
\begin{pgfscope}%
\pgfsys@transformshift{8.063537in}{6.130496in}%
\pgfsys@useobject{currentmarker}{}%
\end{pgfscope}%
\begin{pgfscope}%
\pgfsys@transformshift{8.066996in}{6.131382in}%
\pgfsys@useobject{currentmarker}{}%
\end{pgfscope}%
\begin{pgfscope}%
\pgfsys@transformshift{8.070453in}{6.131738in}%
\pgfsys@useobject{currentmarker}{}%
\end{pgfscope}%
\begin{pgfscope}%
\pgfsys@transformshift{8.073907in}{6.131761in}%
\pgfsys@useobject{currentmarker}{}%
\end{pgfscope}%
\begin{pgfscope}%
\pgfsys@transformshift{8.077360in}{6.132110in}%
\pgfsys@useobject{currentmarker}{}%
\end{pgfscope}%
\begin{pgfscope}%
\pgfsys@transformshift{8.080810in}{6.132162in}%
\pgfsys@useobject{currentmarker}{}%
\end{pgfscope}%
\begin{pgfscope}%
\pgfsys@transformshift{8.084259in}{6.132485in}%
\pgfsys@useobject{currentmarker}{}%
\end{pgfscope}%
\begin{pgfscope}%
\pgfsys@transformshift{8.087706in}{6.132542in}%
\pgfsys@useobject{currentmarker}{}%
\end{pgfscope}%
\begin{pgfscope}%
\pgfsys@transformshift{8.091151in}{6.132889in}%
\pgfsys@useobject{currentmarker}{}%
\end{pgfscope}%
\begin{pgfscope}%
\pgfsys@transformshift{8.094594in}{6.133087in}%
\pgfsys@useobject{currentmarker}{}%
\end{pgfscope}%
\begin{pgfscope}%
\pgfsys@transformshift{8.098035in}{6.133199in}%
\pgfsys@useobject{currentmarker}{}%
\end{pgfscope}%
\begin{pgfscope}%
\pgfsys@transformshift{8.101474in}{6.133310in}%
\pgfsys@useobject{currentmarker}{}%
\end{pgfscope}%
\begin{pgfscope}%
\pgfsys@transformshift{8.104912in}{6.133790in}%
\pgfsys@useobject{currentmarker}{}%
\end{pgfscope}%
\begin{pgfscope}%
\pgfsys@transformshift{8.108348in}{6.134053in}%
\pgfsys@useobject{currentmarker}{}%
\end{pgfscope}%
\begin{pgfscope}%
\pgfsys@transformshift{8.111781in}{6.134200in}%
\pgfsys@useobject{currentmarker}{}%
\end{pgfscope}%
\begin{pgfscope}%
\pgfsys@transformshift{8.115213in}{6.134330in}%
\pgfsys@useobject{currentmarker}{}%
\end{pgfscope}%
\begin{pgfscope}%
\pgfsys@transformshift{8.118643in}{6.134369in}%
\pgfsys@useobject{currentmarker}{}%
\end{pgfscope}%
\begin{pgfscope}%
\pgfsys@transformshift{8.122072in}{6.134806in}%
\pgfsys@useobject{currentmarker}{}%
\end{pgfscope}%
\begin{pgfscope}%
\pgfsys@transformshift{8.125498in}{6.135067in}%
\pgfsys@useobject{currentmarker}{}%
\end{pgfscope}%
\begin{pgfscope}%
\pgfsys@transformshift{8.128923in}{6.135090in}%
\pgfsys@useobject{currentmarker}{}%
\end{pgfscope}%
\begin{pgfscope}%
\pgfsys@transformshift{8.132346in}{6.135169in}%
\pgfsys@useobject{currentmarker}{}%
\end{pgfscope}%
\begin{pgfscope}%
\pgfsys@transformshift{8.135767in}{6.135224in}%
\pgfsys@useobject{currentmarker}{}%
\end{pgfscope}%
\begin{pgfscope}%
\pgfsys@transformshift{8.139187in}{6.135353in}%
\pgfsys@useobject{currentmarker}{}%
\end{pgfscope}%
\begin{pgfscope}%
\pgfsys@transformshift{8.142605in}{6.135432in}%
\pgfsys@useobject{currentmarker}{}%
\end{pgfscope}%
\begin{pgfscope}%
\pgfsys@transformshift{8.146021in}{6.135445in}%
\pgfsys@useobject{currentmarker}{}%
\end{pgfscope}%
\begin{pgfscope}%
\pgfsys@transformshift{8.149435in}{6.135589in}%
\pgfsys@useobject{currentmarker}{}%
\end{pgfscope}%
\begin{pgfscope}%
\pgfsys@transformshift{8.152848in}{6.135626in}%
\pgfsys@useobject{currentmarker}{}%
\end{pgfscope}%
\begin{pgfscope}%
\pgfsys@transformshift{8.156259in}{6.135640in}%
\pgfsys@useobject{currentmarker}{}%
\end{pgfscope}%
\begin{pgfscope}%
\pgfsys@transformshift{8.159669in}{6.136194in}%
\pgfsys@useobject{currentmarker}{}%
\end{pgfscope}%
\begin{pgfscope}%
\pgfsys@transformshift{8.163076in}{6.136459in}%
\pgfsys@useobject{currentmarker}{}%
\end{pgfscope}%
\begin{pgfscope}%
\pgfsys@transformshift{8.166482in}{6.136481in}%
\pgfsys@useobject{currentmarker}{}%
\end{pgfscope}%
\begin{pgfscope}%
\pgfsys@transformshift{8.169887in}{6.136863in}%
\pgfsys@useobject{currentmarker}{}%
\end{pgfscope}%
\begin{pgfscope}%
\pgfsys@transformshift{8.173290in}{6.137061in}%
\pgfsys@useobject{currentmarker}{}%
\end{pgfscope}%
\begin{pgfscope}%
\pgfsys@transformshift{8.176691in}{6.137167in}%
\pgfsys@useobject{currentmarker}{}%
\end{pgfscope}%
\begin{pgfscope}%
\pgfsys@transformshift{8.180091in}{6.137242in}%
\pgfsys@useobject{currentmarker}{}%
\end{pgfscope}%
\begin{pgfscope}%
\pgfsys@transformshift{8.183489in}{6.137761in}%
\pgfsys@useobject{currentmarker}{}%
\end{pgfscope}%
\begin{pgfscope}%
\pgfsys@transformshift{8.186885in}{6.137822in}%
\pgfsys@useobject{currentmarker}{}%
\end{pgfscope}%
\begin{pgfscope}%
\pgfsys@transformshift{8.190280in}{6.137927in}%
\pgfsys@useobject{currentmarker}{}%
\end{pgfscope}%
\begin{pgfscope}%
\pgfsys@transformshift{8.193674in}{6.137947in}%
\pgfsys@useobject{currentmarker}{}%
\end{pgfscope}%
\begin{pgfscope}%
\pgfsys@transformshift{8.197066in}{6.138284in}%
\pgfsys@useobject{currentmarker}{}%
\end{pgfscope}%
\begin{pgfscope}%
\pgfsys@transformshift{8.200456in}{6.138402in}%
\pgfsys@useobject{currentmarker}{}%
\end{pgfscope}%
\begin{pgfscope}%
\pgfsys@transformshift{8.203845in}{6.138656in}%
\pgfsys@useobject{currentmarker}{}%
\end{pgfscope}%
\begin{pgfscope}%
\pgfsys@transformshift{8.207232in}{6.138907in}%
\pgfsys@useobject{currentmarker}{}%
\end{pgfscope}%
\begin{pgfscope}%
\pgfsys@transformshift{8.210618in}{6.138922in}%
\pgfsys@useobject{currentmarker}{}%
\end{pgfscope}%
\begin{pgfscope}%
\pgfsys@transformshift{8.214003in}{6.138972in}%
\pgfsys@useobject{currentmarker}{}%
\end{pgfscope}%
\begin{pgfscope}%
\pgfsys@transformshift{8.217386in}{6.139017in}%
\pgfsys@useobject{currentmarker}{}%
\end{pgfscope}%
\begin{pgfscope}%
\pgfsys@transformshift{8.220767in}{6.139156in}%
\pgfsys@useobject{currentmarker}{}%
\end{pgfscope}%
\begin{pgfscope}%
\pgfsys@transformshift{8.224147in}{6.139182in}%
\pgfsys@useobject{currentmarker}{}%
\end{pgfscope}%
\begin{pgfscope}%
\pgfsys@transformshift{8.227526in}{6.139269in}%
\pgfsys@useobject{currentmarker}{}%
\end{pgfscope}%
\begin{pgfscope}%
\pgfsys@transformshift{8.230903in}{6.139936in}%
\pgfsys@useobject{currentmarker}{}%
\end{pgfscope}%
\begin{pgfscope}%
\pgfsys@transformshift{8.234279in}{6.139948in}%
\pgfsys@useobject{currentmarker}{}%
\end{pgfscope}%
\begin{pgfscope}%
\pgfsys@transformshift{8.237653in}{6.139963in}%
\pgfsys@useobject{currentmarker}{}%
\end{pgfscope}%
\begin{pgfscope}%
\pgfsys@transformshift{8.241026in}{6.140619in}%
\pgfsys@useobject{currentmarker}{}%
\end{pgfscope}%
\begin{pgfscope}%
\pgfsys@transformshift{8.244398in}{6.141133in}%
\pgfsys@useobject{currentmarker}{}%
\end{pgfscope}%
\begin{pgfscope}%
\pgfsys@transformshift{8.247768in}{6.141334in}%
\pgfsys@useobject{currentmarker}{}%
\end{pgfscope}%
\begin{pgfscope}%
\pgfsys@transformshift{8.251137in}{6.141335in}%
\pgfsys@useobject{currentmarker}{}%
\end{pgfscope}%
\begin{pgfscope}%
\pgfsys@transformshift{8.254504in}{6.141766in}%
\pgfsys@useobject{currentmarker}{}%
\end{pgfscope}%
\begin{pgfscope}%
\pgfsys@transformshift{8.257871in}{6.141985in}%
\pgfsys@useobject{currentmarker}{}%
\end{pgfscope}%
\begin{pgfscope}%
\pgfsys@transformshift{8.261235in}{6.142383in}%
\pgfsys@useobject{currentmarker}{}%
\end{pgfscope}%
\begin{pgfscope}%
\pgfsys@transformshift{8.264599in}{6.142412in}%
\pgfsys@useobject{currentmarker}{}%
\end{pgfscope}%
\begin{pgfscope}%
\pgfsys@transformshift{8.267961in}{6.143220in}%
\pgfsys@useobject{currentmarker}{}%
\end{pgfscope}%
\begin{pgfscope}%
\pgfsys@transformshift{8.271322in}{6.143562in}%
\pgfsys@useobject{currentmarker}{}%
\end{pgfscope}%
\begin{pgfscope}%
\pgfsys@transformshift{8.274682in}{6.143798in}%
\pgfsys@useobject{currentmarker}{}%
\end{pgfscope}%
\begin{pgfscope}%
\pgfsys@transformshift{8.278040in}{6.144069in}%
\pgfsys@useobject{currentmarker}{}%
\end{pgfscope}%
\begin{pgfscope}%
\pgfsys@transformshift{8.281398in}{6.144430in}%
\pgfsys@useobject{currentmarker}{}%
\end{pgfscope}%
\begin{pgfscope}%
\pgfsys@transformshift{8.284753in}{6.144471in}%
\pgfsys@useobject{currentmarker}{}%
\end{pgfscope}%
\begin{pgfscope}%
\pgfsys@transformshift{8.288108in}{6.144618in}%
\pgfsys@useobject{currentmarker}{}%
\end{pgfscope}%
\begin{pgfscope}%
\pgfsys@transformshift{8.291462in}{6.144822in}%
\pgfsys@useobject{currentmarker}{}%
\end{pgfscope}%
\begin{pgfscope}%
\pgfsys@transformshift{8.294814in}{6.145067in}%
\pgfsys@useobject{currentmarker}{}%
\end{pgfscope}%
\begin{pgfscope}%
\pgfsys@transformshift{8.298165in}{6.145126in}%
\pgfsys@useobject{currentmarker}{}%
\end{pgfscope}%
\begin{pgfscope}%
\pgfsys@transformshift{8.301515in}{6.145870in}%
\pgfsys@useobject{currentmarker}{}%
\end{pgfscope}%
\begin{pgfscope}%
\pgfsys@transformshift{8.304863in}{6.145996in}%
\pgfsys@useobject{currentmarker}{}%
\end{pgfscope}%
\begin{pgfscope}%
\pgfsys@transformshift{8.308211in}{6.146068in}%
\pgfsys@useobject{currentmarker}{}%
\end{pgfscope}%
\begin{pgfscope}%
\pgfsys@transformshift{8.311557in}{6.146509in}%
\pgfsys@useobject{currentmarker}{}%
\end{pgfscope}%
\begin{pgfscope}%
\pgfsys@transformshift{8.314902in}{6.147276in}%
\pgfsys@useobject{currentmarker}{}%
\end{pgfscope}%
\begin{pgfscope}%
\pgfsys@transformshift{8.318246in}{6.147931in}%
\pgfsys@useobject{currentmarker}{}%
\end{pgfscope}%
\begin{pgfscope}%
\pgfsys@transformshift{8.321589in}{6.148032in}%
\pgfsys@useobject{currentmarker}{}%
\end{pgfscope}%
\begin{pgfscope}%
\pgfsys@transformshift{8.324931in}{6.148092in}%
\pgfsys@useobject{currentmarker}{}%
\end{pgfscope}%
\begin{pgfscope}%
\pgfsys@transformshift{8.328272in}{6.148638in}%
\pgfsys@useobject{currentmarker}{}%
\end{pgfscope}%
\begin{pgfscope}%
\pgfsys@transformshift{8.331611in}{6.148657in}%
\pgfsys@useobject{currentmarker}{}%
\end{pgfscope}%
\begin{pgfscope}%
\pgfsys@transformshift{8.334950in}{6.148783in}%
\pgfsys@useobject{currentmarker}{}%
\end{pgfscope}%
\begin{pgfscope}%
\pgfsys@transformshift{8.338287in}{6.150072in}%
\pgfsys@useobject{currentmarker}{}%
\end{pgfscope}%
\begin{pgfscope}%
\pgfsys@transformshift{8.341623in}{6.150121in}%
\pgfsys@useobject{currentmarker}{}%
\end{pgfscope}%
\begin{pgfscope}%
\pgfsys@transformshift{8.344959in}{6.150173in}%
\pgfsys@useobject{currentmarker}{}%
\end{pgfscope}%
\begin{pgfscope}%
\pgfsys@transformshift{8.348293in}{6.150465in}%
\pgfsys@useobject{currentmarker}{}%
\end{pgfscope}%
\begin{pgfscope}%
\pgfsys@transformshift{8.351626in}{6.150481in}%
\pgfsys@useobject{currentmarker}{}%
\end{pgfscope}%
\begin{pgfscope}%
\pgfsys@transformshift{8.354958in}{6.150607in}%
\pgfsys@useobject{currentmarker}{}%
\end{pgfscope}%
\begin{pgfscope}%
\pgfsys@transformshift{8.358289in}{6.150640in}%
\pgfsys@useobject{currentmarker}{}%
\end{pgfscope}%
\begin{pgfscope}%
\pgfsys@transformshift{8.361619in}{6.150697in}%
\pgfsys@useobject{currentmarker}{}%
\end{pgfscope}%
\begin{pgfscope}%
\pgfsys@transformshift{8.364949in}{6.150763in}%
\pgfsys@useobject{currentmarker}{}%
\end{pgfscope}%
\begin{pgfscope}%
\pgfsys@transformshift{8.368277in}{6.150895in}%
\pgfsys@useobject{currentmarker}{}%
\end{pgfscope}%
\begin{pgfscope}%
\pgfsys@transformshift{8.371604in}{6.151401in}%
\pgfsys@useobject{currentmarker}{}%
\end{pgfscope}%
\begin{pgfscope}%
\pgfsys@transformshift{8.374930in}{6.151530in}%
\pgfsys@useobject{currentmarker}{}%
\end{pgfscope}%
\begin{pgfscope}%
\pgfsys@transformshift{8.378255in}{6.151610in}%
\pgfsys@useobject{currentmarker}{}%
\end{pgfscope}%
\begin{pgfscope}%
\pgfsys@transformshift{8.381579in}{6.151714in}%
\pgfsys@useobject{currentmarker}{}%
\end{pgfscope}%
\begin{pgfscope}%
\pgfsys@transformshift{8.384903in}{6.151751in}%
\pgfsys@useobject{currentmarker}{}%
\end{pgfscope}%
\begin{pgfscope}%
\pgfsys@transformshift{8.388225in}{6.151770in}%
\pgfsys@useobject{currentmarker}{}%
\end{pgfscope}%
\begin{pgfscope}%
\pgfsys@transformshift{8.391547in}{6.151850in}%
\pgfsys@useobject{currentmarker}{}%
\end{pgfscope}%
\begin{pgfscope}%
\pgfsys@transformshift{8.394867in}{6.153142in}%
\pgfsys@useobject{currentmarker}{}%
\end{pgfscope}%
\begin{pgfscope}%
\pgfsys@transformshift{8.398187in}{6.153241in}%
\pgfsys@useobject{currentmarker}{}%
\end{pgfscope}%
\begin{pgfscope}%
\pgfsys@transformshift{8.401506in}{6.153479in}%
\pgfsys@useobject{currentmarker}{}%
\end{pgfscope}%
\begin{pgfscope}%
\pgfsys@transformshift{8.404824in}{6.153566in}%
\pgfsys@useobject{currentmarker}{}%
\end{pgfscope}%
\begin{pgfscope}%
\pgfsys@transformshift{8.408141in}{6.153736in}%
\pgfsys@useobject{currentmarker}{}%
\end{pgfscope}%
\begin{pgfscope}%
\pgfsys@transformshift{8.411457in}{6.153771in}%
\pgfsys@useobject{currentmarker}{}%
\end{pgfscope}%
\begin{pgfscope}%
\pgfsys@transformshift{8.414772in}{6.153890in}%
\pgfsys@useobject{currentmarker}{}%
\end{pgfscope}%
\begin{pgfscope}%
\pgfsys@transformshift{8.418087in}{6.153931in}%
\pgfsys@useobject{currentmarker}{}%
\end{pgfscope}%
\begin{pgfscope}%
\pgfsys@transformshift{8.421400in}{6.153949in}%
\pgfsys@useobject{currentmarker}{}%
\end{pgfscope}%
\begin{pgfscope}%
\pgfsys@transformshift{8.424713in}{6.153983in}%
\pgfsys@useobject{currentmarker}{}%
\end{pgfscope}%
\begin{pgfscope}%
\pgfsys@transformshift{8.428025in}{6.154006in}%
\pgfsys@useobject{currentmarker}{}%
\end{pgfscope}%
\begin{pgfscope}%
\pgfsys@transformshift{8.431336in}{6.154030in}%
\pgfsys@useobject{currentmarker}{}%
\end{pgfscope}%
\begin{pgfscope}%
\pgfsys@transformshift{8.434647in}{6.154122in}%
\pgfsys@useobject{currentmarker}{}%
\end{pgfscope}%
\begin{pgfscope}%
\pgfsys@transformshift{8.437956in}{6.154172in}%
\pgfsys@useobject{currentmarker}{}%
\end{pgfscope}%
\begin{pgfscope}%
\pgfsys@transformshift{8.441265in}{6.154371in}%
\pgfsys@useobject{currentmarker}{}%
\end{pgfscope}%
\begin{pgfscope}%
\pgfsys@transformshift{8.444573in}{6.154610in}%
\pgfsys@useobject{currentmarker}{}%
\end{pgfscope}%
\begin{pgfscope}%
\pgfsys@transformshift{8.447881in}{6.154630in}%
\pgfsys@useobject{currentmarker}{}%
\end{pgfscope}%
\begin{pgfscope}%
\pgfsys@transformshift{8.451187in}{6.155495in}%
\pgfsys@useobject{currentmarker}{}%
\end{pgfscope}%
\begin{pgfscope}%
\pgfsys@transformshift{8.454493in}{6.155627in}%
\pgfsys@useobject{currentmarker}{}%
\end{pgfscope}%
\begin{pgfscope}%
\pgfsys@transformshift{8.457798in}{6.155643in}%
\pgfsys@useobject{currentmarker}{}%
\end{pgfscope}%
\begin{pgfscope}%
\pgfsys@transformshift{8.461103in}{6.156242in}%
\pgfsys@useobject{currentmarker}{}%
\end{pgfscope}%
\begin{pgfscope}%
\pgfsys@transformshift{8.464406in}{6.156329in}%
\pgfsys@useobject{currentmarker}{}%
\end{pgfscope}%
\begin{pgfscope}%
\pgfsys@transformshift{8.467709in}{6.156794in}%
\pgfsys@useobject{currentmarker}{}%
\end{pgfscope}%
\begin{pgfscope}%
\pgfsys@transformshift{8.471012in}{6.157014in}%
\pgfsys@useobject{currentmarker}{}%
\end{pgfscope}%
\begin{pgfscope}%
\pgfsys@transformshift{8.474314in}{6.157183in}%
\pgfsys@useobject{currentmarker}{}%
\end{pgfscope}%
\begin{pgfscope}%
\pgfsys@transformshift{8.477615in}{6.157473in}%
\pgfsys@useobject{currentmarker}{}%
\end{pgfscope}%
\begin{pgfscope}%
\pgfsys@transformshift{8.480915in}{6.157513in}%
\pgfsys@useobject{currentmarker}{}%
\end{pgfscope}%
\begin{pgfscope}%
\pgfsys@transformshift{8.484215in}{6.157514in}%
\pgfsys@useobject{currentmarker}{}%
\end{pgfscope}%
\begin{pgfscope}%
\pgfsys@transformshift{8.487514in}{6.157583in}%
\pgfsys@useobject{currentmarker}{}%
\end{pgfscope}%
\begin{pgfscope}%
\pgfsys@transformshift{8.490812in}{6.157587in}%
\pgfsys@useobject{currentmarker}{}%
\end{pgfscope}%
\begin{pgfscope}%
\pgfsys@transformshift{8.494110in}{6.158053in}%
\pgfsys@useobject{currentmarker}{}%
\end{pgfscope}%
\begin{pgfscope}%
\pgfsys@transformshift{8.497407in}{6.158203in}%
\pgfsys@useobject{currentmarker}{}%
\end{pgfscope}%
\begin{pgfscope}%
\pgfsys@transformshift{8.500704in}{6.158255in}%
\pgfsys@useobject{currentmarker}{}%
\end{pgfscope}%
\begin{pgfscope}%
\pgfsys@transformshift{8.504000in}{6.158503in}%
\pgfsys@useobject{currentmarker}{}%
\end{pgfscope}%
\begin{pgfscope}%
\pgfsys@transformshift{8.507295in}{6.158734in}%
\pgfsys@useobject{currentmarker}{}%
\end{pgfscope}%
\begin{pgfscope}%
\pgfsys@transformshift{8.510590in}{6.159308in}%
\pgfsys@useobject{currentmarker}{}%
\end{pgfscope}%
\begin{pgfscope}%
\pgfsys@transformshift{8.513884in}{6.160186in}%
\pgfsys@useobject{currentmarker}{}%
\end{pgfscope}%
\begin{pgfscope}%
\pgfsys@transformshift{8.517178in}{6.160412in}%
\pgfsys@useobject{currentmarker}{}%
\end{pgfscope}%
\begin{pgfscope}%
\pgfsys@transformshift{8.520471in}{6.160582in}%
\pgfsys@useobject{currentmarker}{}%
\end{pgfscope}%
\begin{pgfscope}%
\pgfsys@transformshift{8.523764in}{6.160710in}%
\pgfsys@useobject{currentmarker}{}%
\end{pgfscope}%
\begin{pgfscope}%
\pgfsys@transformshift{8.527056in}{6.160762in}%
\pgfsys@useobject{currentmarker}{}%
\end{pgfscope}%
\begin{pgfscope}%
\pgfsys@transformshift{8.530348in}{6.160950in}%
\pgfsys@useobject{currentmarker}{}%
\end{pgfscope}%
\begin{pgfscope}%
\pgfsys@transformshift{8.533639in}{6.160972in}%
\pgfsys@useobject{currentmarker}{}%
\end{pgfscope}%
\begin{pgfscope}%
\pgfsys@transformshift{8.536929in}{6.161308in}%
\pgfsys@useobject{currentmarker}{}%
\end{pgfscope}%
\begin{pgfscope}%
\pgfsys@transformshift{8.540219in}{6.161328in}%
\pgfsys@useobject{currentmarker}{}%
\end{pgfscope}%
\begin{pgfscope}%
\pgfsys@transformshift{8.543509in}{6.161329in}%
\pgfsys@useobject{currentmarker}{}%
\end{pgfscope}%
\begin{pgfscope}%
\pgfsys@transformshift{8.546798in}{6.162166in}%
\pgfsys@useobject{currentmarker}{}%
\end{pgfscope}%
\begin{pgfscope}%
\pgfsys@transformshift{8.550087in}{6.162179in}%
\pgfsys@useobject{currentmarker}{}%
\end{pgfscope}%
\begin{pgfscope}%
\pgfsys@transformshift{8.553375in}{6.162194in}%
\pgfsys@useobject{currentmarker}{}%
\end{pgfscope}%
\begin{pgfscope}%
\pgfsys@transformshift{8.556663in}{6.162247in}%
\pgfsys@useobject{currentmarker}{}%
\end{pgfscope}%
\begin{pgfscope}%
\pgfsys@transformshift{8.559951in}{6.162267in}%
\pgfsys@useobject{currentmarker}{}%
\end{pgfscope}%
\begin{pgfscope}%
\pgfsys@transformshift{8.563237in}{6.162518in}%
\pgfsys@useobject{currentmarker}{}%
\end{pgfscope}%
\begin{pgfscope}%
\pgfsys@transformshift{8.566524in}{6.163006in}%
\pgfsys@useobject{currentmarker}{}%
\end{pgfscope}%
\begin{pgfscope}%
\pgfsys@transformshift{8.569810in}{6.163013in}%
\pgfsys@useobject{currentmarker}{}%
\end{pgfscope}%
\begin{pgfscope}%
\pgfsys@transformshift{8.573096in}{6.163042in}%
\pgfsys@useobject{currentmarker}{}%
\end{pgfscope}%
\begin{pgfscope}%
\pgfsys@transformshift{8.576381in}{6.163046in}%
\pgfsys@useobject{currentmarker}{}%
\end{pgfscope}%
\begin{pgfscope}%
\pgfsys@transformshift{8.579666in}{6.163118in}%
\pgfsys@useobject{currentmarker}{}%
\end{pgfscope}%
\begin{pgfscope}%
\pgfsys@transformshift{8.582951in}{6.163416in}%
\pgfsys@useobject{currentmarker}{}%
\end{pgfscope}%
\begin{pgfscope}%
\pgfsys@transformshift{8.586235in}{6.163429in}%
\pgfsys@useobject{currentmarker}{}%
\end{pgfscope}%
\begin{pgfscope}%
\pgfsys@transformshift{8.589519in}{6.164042in}%
\pgfsys@useobject{currentmarker}{}%
\end{pgfscope}%
\begin{pgfscope}%
\pgfsys@transformshift{8.592803in}{6.164131in}%
\pgfsys@useobject{currentmarker}{}%
\end{pgfscope}%
\begin{pgfscope}%
\pgfsys@transformshift{8.596086in}{6.164331in}%
\pgfsys@useobject{currentmarker}{}%
\end{pgfscope}%
\begin{pgfscope}%
\pgfsys@transformshift{8.599369in}{6.164359in}%
\pgfsys@useobject{currentmarker}{}%
\end{pgfscope}%
\begin{pgfscope}%
\pgfsys@transformshift{8.602651in}{6.164490in}%
\pgfsys@useobject{currentmarker}{}%
\end{pgfscope}%
\begin{pgfscope}%
\pgfsys@transformshift{8.605934in}{6.165027in}%
\pgfsys@useobject{currentmarker}{}%
\end{pgfscope}%
\begin{pgfscope}%
\pgfsys@transformshift{8.609216in}{6.165093in}%
\pgfsys@useobject{currentmarker}{}%
\end{pgfscope}%
\begin{pgfscope}%
\pgfsys@transformshift{8.612497in}{6.165186in}%
\pgfsys@useobject{currentmarker}{}%
\end{pgfscope}%
\begin{pgfscope}%
\pgfsys@transformshift{8.615779in}{6.165215in}%
\pgfsys@useobject{currentmarker}{}%
\end{pgfscope}%
\begin{pgfscope}%
\pgfsys@transformshift{8.619060in}{6.165276in}%
\pgfsys@useobject{currentmarker}{}%
\end{pgfscope}%
\begin{pgfscope}%
\pgfsys@transformshift{8.622341in}{6.165341in}%
\pgfsys@useobject{currentmarker}{}%
\end{pgfscope}%
\begin{pgfscope}%
\pgfsys@transformshift{8.625622in}{6.165409in}%
\pgfsys@useobject{currentmarker}{}%
\end{pgfscope}%
\begin{pgfscope}%
\pgfsys@transformshift{8.628902in}{6.165605in}%
\pgfsys@useobject{currentmarker}{}%
\end{pgfscope}%
\begin{pgfscope}%
\pgfsys@transformshift{8.632182in}{6.165624in}%
\pgfsys@useobject{currentmarker}{}%
\end{pgfscope}%
\begin{pgfscope}%
\pgfsys@transformshift{8.635462in}{6.165782in}%
\pgfsys@useobject{currentmarker}{}%
\end{pgfscope}%
\begin{pgfscope}%
\pgfsys@transformshift{8.638742in}{6.165794in}%
\pgfsys@useobject{currentmarker}{}%
\end{pgfscope}%
\begin{pgfscope}%
\pgfsys@transformshift{8.642022in}{6.168034in}%
\pgfsys@useobject{currentmarker}{}%
\end{pgfscope}%
\begin{pgfscope}%
\pgfsys@transformshift{8.645301in}{6.168034in}%
\pgfsys@useobject{currentmarker}{}%
\end{pgfscope}%
\begin{pgfscope}%
\pgfsys@transformshift{8.648580in}{6.168034in}%
\pgfsys@useobject{currentmarker}{}%
\end{pgfscope}%
\begin{pgfscope}%
\pgfsys@transformshift{8.651859in}{6.168034in}%
\pgfsys@useobject{currentmarker}{}%
\end{pgfscope}%
\begin{pgfscope}%
\pgfsys@transformshift{8.655138in}{6.168034in}%
\pgfsys@useobject{currentmarker}{}%
\end{pgfscope}%
\begin{pgfscope}%
\pgfsys@transformshift{8.658417in}{6.168034in}%
\pgfsys@useobject{currentmarker}{}%
\end{pgfscope}%
\begin{pgfscope}%
\pgfsys@transformshift{8.661695in}{6.168034in}%
\pgfsys@useobject{currentmarker}{}%
\end{pgfscope}%
\begin{pgfscope}%
\pgfsys@transformshift{8.664974in}{6.168034in}%
\pgfsys@useobject{currentmarker}{}%
\end{pgfscope}%
\begin{pgfscope}%
\pgfsys@transformshift{8.668252in}{6.168034in}%
\pgfsys@useobject{currentmarker}{}%
\end{pgfscope}%
\begin{pgfscope}%
\pgfsys@transformshift{8.671530in}{6.168034in}%
\pgfsys@useobject{currentmarker}{}%
\end{pgfscope}%
\begin{pgfscope}%
\pgfsys@transformshift{8.674808in}{6.168034in}%
\pgfsys@useobject{currentmarker}{}%
\end{pgfscope}%
\begin{pgfscope}%
\pgfsys@transformshift{8.678086in}{6.168034in}%
\pgfsys@useobject{currentmarker}{}%
\end{pgfscope}%
\begin{pgfscope}%
\pgfsys@transformshift{8.681364in}{6.168034in}%
\pgfsys@useobject{currentmarker}{}%
\end{pgfscope}%
\begin{pgfscope}%
\pgfsys@transformshift{8.684641in}{6.168034in}%
\pgfsys@useobject{currentmarker}{}%
\end{pgfscope}%
\begin{pgfscope}%
\pgfsys@transformshift{8.687919in}{6.170374in}%
\pgfsys@useobject{currentmarker}{}%
\end{pgfscope}%
\begin{pgfscope}%
\pgfsys@transformshift{8.691197in}{6.170739in}%
\pgfsys@useobject{currentmarker}{}%
\end{pgfscope}%
\begin{pgfscope}%
\pgfsys@transformshift{8.694474in}{6.170879in}%
\pgfsys@useobject{currentmarker}{}%
\end{pgfscope}%
\begin{pgfscope}%
\pgfsys@transformshift{8.697752in}{6.171085in}%
\pgfsys@useobject{currentmarker}{}%
\end{pgfscope}%
\begin{pgfscope}%
\pgfsys@transformshift{8.701029in}{6.171108in}%
\pgfsys@useobject{currentmarker}{}%
\end{pgfscope}%
\begin{pgfscope}%
\pgfsys@transformshift{8.704307in}{6.171449in}%
\pgfsys@useobject{currentmarker}{}%
\end{pgfscope}%
\begin{pgfscope}%
\pgfsys@transformshift{8.707584in}{6.171511in}%
\pgfsys@useobject{currentmarker}{}%
\end{pgfscope}%
\begin{pgfscope}%
\pgfsys@transformshift{8.710861in}{6.171664in}%
\pgfsys@useobject{currentmarker}{}%
\end{pgfscope}%
\begin{pgfscope}%
\pgfsys@transformshift{8.714139in}{6.171672in}%
\pgfsys@useobject{currentmarker}{}%
\end{pgfscope}%
\begin{pgfscope}%
\pgfsys@transformshift{8.717416in}{6.171738in}%
\pgfsys@useobject{currentmarker}{}%
\end{pgfscope}%
\begin{pgfscope}%
\pgfsys@transformshift{8.720693in}{6.171764in}%
\pgfsys@useobject{currentmarker}{}%
\end{pgfscope}%
\begin{pgfscope}%
\pgfsys@transformshift{8.723971in}{6.172597in}%
\pgfsys@useobject{currentmarker}{}%
\end{pgfscope}%
\begin{pgfscope}%
\pgfsys@transformshift{8.727248in}{6.172870in}%
\pgfsys@useobject{currentmarker}{}%
\end{pgfscope}%
\begin{pgfscope}%
\pgfsys@transformshift{8.730526in}{6.172880in}%
\pgfsys@useobject{currentmarker}{}%
\end{pgfscope}%
\begin{pgfscope}%
\pgfsys@transformshift{8.733803in}{6.172884in}%
\pgfsys@useobject{currentmarker}{}%
\end{pgfscope}%
\begin{pgfscope}%
\pgfsys@transformshift{8.737081in}{6.173294in}%
\pgfsys@useobject{currentmarker}{}%
\end{pgfscope}%
\begin{pgfscope}%
\pgfsys@transformshift{8.740359in}{6.173454in}%
\pgfsys@useobject{currentmarker}{}%
\end{pgfscope}%
\begin{pgfscope}%
\pgfsys@transformshift{8.743636in}{6.173712in}%
\pgfsys@useobject{currentmarker}{}%
\end{pgfscope}%
\begin{pgfscope}%
\pgfsys@transformshift{8.746914in}{6.173769in}%
\pgfsys@useobject{currentmarker}{}%
\end{pgfscope}%
\begin{pgfscope}%
\pgfsys@transformshift{8.750192in}{6.173776in}%
\pgfsys@useobject{currentmarker}{}%
\end{pgfscope}%
\begin{pgfscope}%
\pgfsys@transformshift{8.753470in}{6.174051in}%
\pgfsys@useobject{currentmarker}{}%
\end{pgfscope}%
\begin{pgfscope}%
\pgfsys@transformshift{8.756748in}{6.174925in}%
\pgfsys@useobject{currentmarker}{}%
\end{pgfscope}%
\begin{pgfscope}%
\pgfsys@transformshift{8.760026in}{6.175280in}%
\pgfsys@useobject{currentmarker}{}%
\end{pgfscope}%
\begin{pgfscope}%
\pgfsys@transformshift{8.763305in}{6.175307in}%
\pgfsys@useobject{currentmarker}{}%
\end{pgfscope}%
\begin{pgfscope}%
\pgfsys@transformshift{8.766583in}{6.175318in}%
\pgfsys@useobject{currentmarker}{}%
\end{pgfscope}%
\begin{pgfscope}%
\pgfsys@transformshift{8.769862in}{6.175358in}%
\pgfsys@useobject{currentmarker}{}%
\end{pgfscope}%
\begin{pgfscope}%
\pgfsys@transformshift{8.773141in}{6.175416in}%
\pgfsys@useobject{currentmarker}{}%
\end{pgfscope}%
\begin{pgfscope}%
\pgfsys@transformshift{8.776420in}{6.175527in}%
\pgfsys@useobject{currentmarker}{}%
\end{pgfscope}%
\begin{pgfscope}%
\pgfsys@transformshift{8.779699in}{6.175552in}%
\pgfsys@useobject{currentmarker}{}%
\end{pgfscope}%
\begin{pgfscope}%
\pgfsys@transformshift{8.782978in}{6.175700in}%
\pgfsys@useobject{currentmarker}{}%
\end{pgfscope}%
\begin{pgfscope}%
\pgfsys@transformshift{8.786258in}{6.175808in}%
\pgfsys@useobject{currentmarker}{}%
\end{pgfscope}%
\begin{pgfscope}%
\pgfsys@transformshift{8.789538in}{6.176051in}%
\pgfsys@useobject{currentmarker}{}%
\end{pgfscope}%
\begin{pgfscope}%
\pgfsys@transformshift{8.792818in}{6.176902in}%
\pgfsys@useobject{currentmarker}{}%
\end{pgfscope}%
\begin{pgfscope}%
\pgfsys@transformshift{8.796098in}{6.177248in}%
\pgfsys@useobject{currentmarker}{}%
\end{pgfscope}%
\begin{pgfscope}%
\pgfsys@transformshift{8.799378in}{6.177522in}%
\pgfsys@useobject{currentmarker}{}%
\end{pgfscope}%
\begin{pgfscope}%
\pgfsys@transformshift{8.802659in}{6.178004in}%
\pgfsys@useobject{currentmarker}{}%
\end{pgfscope}%
\begin{pgfscope}%
\pgfsys@transformshift{8.805940in}{6.178362in}%
\pgfsys@useobject{currentmarker}{}%
\end{pgfscope}%
\begin{pgfscope}%
\pgfsys@transformshift{8.809221in}{6.178945in}%
\pgfsys@useobject{currentmarker}{}%
\end{pgfscope}%
\begin{pgfscope}%
\pgfsys@transformshift{8.812503in}{6.179045in}%
\pgfsys@useobject{currentmarker}{}%
\end{pgfscope}%
\begin{pgfscope}%
\pgfsys@transformshift{8.815784in}{6.179050in}%
\pgfsys@useobject{currentmarker}{}%
\end{pgfscope}%
\begin{pgfscope}%
\pgfsys@transformshift{8.819066in}{6.179064in}%
\pgfsys@useobject{currentmarker}{}%
\end{pgfscope}%
\begin{pgfscope}%
\pgfsys@transformshift{8.822349in}{6.179566in}%
\pgfsys@useobject{currentmarker}{}%
\end{pgfscope}%
\begin{pgfscope}%
\pgfsys@transformshift{8.825631in}{6.179580in}%
\pgfsys@useobject{currentmarker}{}%
\end{pgfscope}%
\begin{pgfscope}%
\pgfsys@transformshift{8.828914in}{6.179741in}%
\pgfsys@useobject{currentmarker}{}%
\end{pgfscope}%
\begin{pgfscope}%
\pgfsys@transformshift{8.832197in}{6.180111in}%
\pgfsys@useobject{currentmarker}{}%
\end{pgfscope}%
\begin{pgfscope}%
\pgfsys@transformshift{8.835481in}{6.180118in}%
\pgfsys@useobject{currentmarker}{}%
\end{pgfscope}%
\begin{pgfscope}%
\pgfsys@transformshift{8.838765in}{6.180179in}%
\pgfsys@useobject{currentmarker}{}%
\end{pgfscope}%
\begin{pgfscope}%
\pgfsys@transformshift{8.842049in}{6.180257in}%
\pgfsys@useobject{currentmarker}{}%
\end{pgfscope}%
\begin{pgfscope}%
\pgfsys@transformshift{8.845334in}{6.180335in}%
\pgfsys@useobject{currentmarker}{}%
\end{pgfscope}%
\begin{pgfscope}%
\pgfsys@transformshift{8.848619in}{6.181917in}%
\pgfsys@useobject{currentmarker}{}%
\end{pgfscope}%
\begin{pgfscope}%
\pgfsys@transformshift{8.851904in}{6.182038in}%
\pgfsys@useobject{currentmarker}{}%
\end{pgfscope}%
\begin{pgfscope}%
\pgfsys@transformshift{8.855190in}{6.182060in}%
\pgfsys@useobject{currentmarker}{}%
\end{pgfscope}%
\begin{pgfscope}%
\pgfsys@transformshift{8.858476in}{6.182159in}%
\pgfsys@useobject{currentmarker}{}%
\end{pgfscope}%
\begin{pgfscope}%
\pgfsys@transformshift{8.861763in}{6.182303in}%
\pgfsys@useobject{currentmarker}{}%
\end{pgfscope}%
\begin{pgfscope}%
\pgfsys@transformshift{8.865049in}{6.182380in}%
\pgfsys@useobject{currentmarker}{}%
\end{pgfscope}%
\begin{pgfscope}%
\pgfsys@transformshift{8.868337in}{6.182385in}%
\pgfsys@useobject{currentmarker}{}%
\end{pgfscope}%
\begin{pgfscope}%
\pgfsys@transformshift{8.871625in}{6.182395in}%
\pgfsys@useobject{currentmarker}{}%
\end{pgfscope}%
\begin{pgfscope}%
\pgfsys@transformshift{8.874913in}{6.182903in}%
\pgfsys@useobject{currentmarker}{}%
\end{pgfscope}%
\begin{pgfscope}%
\pgfsys@transformshift{8.878202in}{6.182989in}%
\pgfsys@useobject{currentmarker}{}%
\end{pgfscope}%
\begin{pgfscope}%
\pgfsys@transformshift{8.881491in}{6.183497in}%
\pgfsys@useobject{currentmarker}{}%
\end{pgfscope}%
\begin{pgfscope}%
\pgfsys@transformshift{8.884781in}{6.183737in}%
\pgfsys@useobject{currentmarker}{}%
\end{pgfscope}%
\begin{pgfscope}%
\pgfsys@transformshift{8.888071in}{6.183773in}%
\pgfsys@useobject{currentmarker}{}%
\end{pgfscope}%
\begin{pgfscope}%
\pgfsys@transformshift{8.891361in}{6.183796in}%
\pgfsys@useobject{currentmarker}{}%
\end{pgfscope}%
\begin{pgfscope}%
\pgfsys@transformshift{8.894652in}{6.184382in}%
\pgfsys@useobject{currentmarker}{}%
\end{pgfscope}%
\begin{pgfscope}%
\pgfsys@transformshift{8.897944in}{6.184620in}%
\pgfsys@useobject{currentmarker}{}%
\end{pgfscope}%
\begin{pgfscope}%
\pgfsys@transformshift{8.901236in}{6.184727in}%
\pgfsys@useobject{currentmarker}{}%
\end{pgfscope}%
\begin{pgfscope}%
\pgfsys@transformshift{8.904529in}{6.184835in}%
\pgfsys@useobject{currentmarker}{}%
\end{pgfscope}%
\begin{pgfscope}%
\pgfsys@transformshift{8.907822in}{6.185054in}%
\pgfsys@useobject{currentmarker}{}%
\end{pgfscope}%
\begin{pgfscope}%
\pgfsys@transformshift{8.911116in}{6.185366in}%
\pgfsys@useobject{currentmarker}{}%
\end{pgfscope}%
\begin{pgfscope}%
\pgfsys@transformshift{8.914410in}{6.185442in}%
\pgfsys@useobject{currentmarker}{}%
\end{pgfscope}%
\begin{pgfscope}%
\pgfsys@transformshift{8.917705in}{6.185493in}%
\pgfsys@useobject{currentmarker}{}%
\end{pgfscope}%
\begin{pgfscope}%
\pgfsys@transformshift{8.921000in}{6.185543in}%
\pgfsys@useobject{currentmarker}{}%
\end{pgfscope}%
\begin{pgfscope}%
\pgfsys@transformshift{8.924296in}{6.185872in}%
\pgfsys@useobject{currentmarker}{}%
\end{pgfscope}%
\begin{pgfscope}%
\pgfsys@transformshift{8.927593in}{6.185911in}%
\pgfsys@useobject{currentmarker}{}%
\end{pgfscope}%
\begin{pgfscope}%
\pgfsys@transformshift{8.930890in}{6.185992in}%
\pgfsys@useobject{currentmarker}{}%
\end{pgfscope}%
\begin{pgfscope}%
\pgfsys@transformshift{8.934188in}{6.186412in}%
\pgfsys@useobject{currentmarker}{}%
\end{pgfscope}%
\begin{pgfscope}%
\pgfsys@transformshift{8.937486in}{6.186905in}%
\pgfsys@useobject{currentmarker}{}%
\end{pgfscope}%
\begin{pgfscope}%
\pgfsys@transformshift{8.940785in}{6.187018in}%
\pgfsys@useobject{currentmarker}{}%
\end{pgfscope}%
\begin{pgfscope}%
\pgfsys@transformshift{8.944085in}{6.187134in}%
\pgfsys@useobject{currentmarker}{}%
\end{pgfscope}%
\begin{pgfscope}%
\pgfsys@transformshift{8.947385in}{6.187257in}%
\pgfsys@useobject{currentmarker}{}%
\end{pgfscope}%
\begin{pgfscope}%
\pgfsys@transformshift{8.950686in}{6.187375in}%
\pgfsys@useobject{currentmarker}{}%
\end{pgfscope}%
\begin{pgfscope}%
\pgfsys@transformshift{8.953988in}{6.187521in}%
\pgfsys@useobject{currentmarker}{}%
\end{pgfscope}%
\begin{pgfscope}%
\pgfsys@transformshift{8.957291in}{6.188557in}%
\pgfsys@useobject{currentmarker}{}%
\end{pgfscope}%
\begin{pgfscope}%
\pgfsys@transformshift{8.960594in}{6.188835in}%
\pgfsys@useobject{currentmarker}{}%
\end{pgfscope}%
\begin{pgfscope}%
\pgfsys@transformshift{8.963897in}{6.189046in}%
\pgfsys@useobject{currentmarker}{}%
\end{pgfscope}%
\begin{pgfscope}%
\pgfsys@transformshift{8.967202in}{6.189838in}%
\pgfsys@useobject{currentmarker}{}%
\end{pgfscope}%
\begin{pgfscope}%
\pgfsys@transformshift{8.970507in}{6.190083in}%
\pgfsys@useobject{currentmarker}{}%
\end{pgfscope}%
\begin{pgfscope}%
\pgfsys@transformshift{8.973813in}{6.190120in}%
\pgfsys@useobject{currentmarker}{}%
\end{pgfscope}%
\begin{pgfscope}%
\pgfsys@transformshift{8.977119in}{6.190294in}%
\pgfsys@useobject{currentmarker}{}%
\end{pgfscope}%
\begin{pgfscope}%
\pgfsys@transformshift{8.980427in}{6.190310in}%
\pgfsys@useobject{currentmarker}{}%
\end{pgfscope}%
\begin{pgfscope}%
\pgfsys@transformshift{8.983735in}{6.190594in}%
\pgfsys@useobject{currentmarker}{}%
\end{pgfscope}%
\begin{pgfscope}%
\pgfsys@transformshift{8.987044in}{6.190832in}%
\pgfsys@useobject{currentmarker}{}%
\end{pgfscope}%
\begin{pgfscope}%
\pgfsys@transformshift{8.990353in}{6.190993in}%
\pgfsys@useobject{currentmarker}{}%
\end{pgfscope}%
\begin{pgfscope}%
\pgfsys@transformshift{8.993664in}{6.191001in}%
\pgfsys@useobject{currentmarker}{}%
\end{pgfscope}%
\begin{pgfscope}%
\pgfsys@transformshift{8.996975in}{6.191216in}%
\pgfsys@useobject{currentmarker}{}%
\end{pgfscope}%
\begin{pgfscope}%
\pgfsys@transformshift{9.000287in}{6.191663in}%
\pgfsys@useobject{currentmarker}{}%
\end{pgfscope}%
\begin{pgfscope}%
\pgfsys@transformshift{9.003600in}{6.192483in}%
\pgfsys@useobject{currentmarker}{}%
\end{pgfscope}%
\begin{pgfscope}%
\pgfsys@transformshift{9.006913in}{6.192520in}%
\pgfsys@useobject{currentmarker}{}%
\end{pgfscope}%
\begin{pgfscope}%
\pgfsys@transformshift{9.010228in}{6.192591in}%
\pgfsys@useobject{currentmarker}{}%
\end{pgfscope}%
\begin{pgfscope}%
\pgfsys@transformshift{9.013543in}{6.192595in}%
\pgfsys@useobject{currentmarker}{}%
\end{pgfscope}%
\begin{pgfscope}%
\pgfsys@transformshift{9.016859in}{6.192610in}%
\pgfsys@useobject{currentmarker}{}%
\end{pgfscope}%
\begin{pgfscope}%
\pgfsys@transformshift{9.020176in}{6.192930in}%
\pgfsys@useobject{currentmarker}{}%
\end{pgfscope}%
\begin{pgfscope}%
\pgfsys@transformshift{9.023494in}{6.193123in}%
\pgfsys@useobject{currentmarker}{}%
\end{pgfscope}%
\begin{pgfscope}%
\pgfsys@transformshift{9.026813in}{6.193376in}%
\pgfsys@useobject{currentmarker}{}%
\end{pgfscope}%
\begin{pgfscope}%
\pgfsys@transformshift{9.030133in}{6.193479in}%
\pgfsys@useobject{currentmarker}{}%
\end{pgfscope}%
\begin{pgfscope}%
\pgfsys@transformshift{9.033453in}{6.193566in}%
\pgfsys@useobject{currentmarker}{}%
\end{pgfscope}%
\begin{pgfscope}%
\pgfsys@transformshift{9.036775in}{6.193649in}%
\pgfsys@useobject{currentmarker}{}%
\end{pgfscope}%
\begin{pgfscope}%
\pgfsys@transformshift{9.040097in}{6.193960in}%
\pgfsys@useobject{currentmarker}{}%
\end{pgfscope}%
\begin{pgfscope}%
\pgfsys@transformshift{9.043421in}{6.194074in}%
\pgfsys@useobject{currentmarker}{}%
\end{pgfscope}%
\begin{pgfscope}%
\pgfsys@transformshift{9.046745in}{6.194075in}%
\pgfsys@useobject{currentmarker}{}%
\end{pgfscope}%
\begin{pgfscope}%
\pgfsys@transformshift{9.050070in}{6.194207in}%
\pgfsys@useobject{currentmarker}{}%
\end{pgfscope}%
\begin{pgfscope}%
\pgfsys@transformshift{9.053396in}{6.194343in}%
\pgfsys@useobject{currentmarker}{}%
\end{pgfscope}%
\begin{pgfscope}%
\pgfsys@transformshift{9.056723in}{6.194764in}%
\pgfsys@useobject{currentmarker}{}%
\end{pgfscope}%
\begin{pgfscope}%
\pgfsys@transformshift{9.060051in}{6.194880in}%
\pgfsys@useobject{currentmarker}{}%
\end{pgfscope}%
\begin{pgfscope}%
\pgfsys@transformshift{9.063381in}{6.195175in}%
\pgfsys@useobject{currentmarker}{}%
\end{pgfscope}%
\begin{pgfscope}%
\pgfsys@transformshift{9.066711in}{6.195410in}%
\pgfsys@useobject{currentmarker}{}%
\end{pgfscope}%
\begin{pgfscope}%
\pgfsys@transformshift{9.070042in}{6.195602in}%
\pgfsys@useobject{currentmarker}{}%
\end{pgfscope}%
\begin{pgfscope}%
\pgfsys@transformshift{9.073374in}{6.195783in}%
\pgfsys@useobject{currentmarker}{}%
\end{pgfscope}%
\begin{pgfscope}%
\pgfsys@transformshift{9.076707in}{6.195808in}%
\pgfsys@useobject{currentmarker}{}%
\end{pgfscope}%
\begin{pgfscope}%
\pgfsys@transformshift{9.080041in}{6.195881in}%
\pgfsys@useobject{currentmarker}{}%
\end{pgfscope}%
\begin{pgfscope}%
\pgfsys@transformshift{9.083377in}{6.196164in}%
\pgfsys@useobject{currentmarker}{}%
\end{pgfscope}%
\begin{pgfscope}%
\pgfsys@transformshift{9.086713in}{6.196267in}%
\pgfsys@useobject{currentmarker}{}%
\end{pgfscope}%
\begin{pgfscope}%
\pgfsys@transformshift{9.090050in}{6.197151in}%
\pgfsys@useobject{currentmarker}{}%
\end{pgfscope}%
\begin{pgfscope}%
\pgfsys@transformshift{9.093389in}{6.197196in}%
\pgfsys@useobject{currentmarker}{}%
\end{pgfscope}%
\begin{pgfscope}%
\pgfsys@transformshift{9.096728in}{6.197494in}%
\pgfsys@useobject{currentmarker}{}%
\end{pgfscope}%
\begin{pgfscope}%
\pgfsys@transformshift{9.100069in}{6.198684in}%
\pgfsys@useobject{currentmarker}{}%
\end{pgfscope}%
\begin{pgfscope}%
\pgfsys@transformshift{9.103411in}{6.198702in}%
\pgfsys@useobject{currentmarker}{}%
\end{pgfscope}%
\begin{pgfscope}%
\pgfsys@transformshift{9.106754in}{6.199492in}%
\pgfsys@useobject{currentmarker}{}%
\end{pgfscope}%
\begin{pgfscope}%
\pgfsys@transformshift{9.110098in}{6.199531in}%
\pgfsys@useobject{currentmarker}{}%
\end{pgfscope}%
\begin{pgfscope}%
\pgfsys@transformshift{9.113443in}{6.199693in}%
\pgfsys@useobject{currentmarker}{}%
\end{pgfscope}%
\begin{pgfscope}%
\pgfsys@transformshift{9.116789in}{6.199957in}%
\pgfsys@useobject{currentmarker}{}%
\end{pgfscope}%
\begin{pgfscope}%
\pgfsys@transformshift{9.120137in}{6.200401in}%
\pgfsys@useobject{currentmarker}{}%
\end{pgfscope}%
\begin{pgfscope}%
\pgfsys@transformshift{9.123485in}{6.200644in}%
\pgfsys@useobject{currentmarker}{}%
\end{pgfscope}%
\begin{pgfscope}%
\pgfsys@transformshift{9.126835in}{6.200680in}%
\pgfsys@useobject{currentmarker}{}%
\end{pgfscope}%
\begin{pgfscope}%
\pgfsys@transformshift{9.130186in}{6.200690in}%
\pgfsys@useobject{currentmarker}{}%
\end{pgfscope}%
\begin{pgfscope}%
\pgfsys@transformshift{9.133538in}{6.200977in}%
\pgfsys@useobject{currentmarker}{}%
\end{pgfscope}%
\begin{pgfscope}%
\pgfsys@transformshift{9.136892in}{6.201033in}%
\pgfsys@useobject{currentmarker}{}%
\end{pgfscope}%
\begin{pgfscope}%
\pgfsys@transformshift{9.140247in}{6.201261in}%
\pgfsys@useobject{currentmarker}{}%
\end{pgfscope}%
\begin{pgfscope}%
\pgfsys@transformshift{9.143602in}{6.201541in}%
\pgfsys@useobject{currentmarker}{}%
\end{pgfscope}%
\begin{pgfscope}%
\pgfsys@transformshift{9.146960in}{6.201718in}%
\pgfsys@useobject{currentmarker}{}%
\end{pgfscope}%
\begin{pgfscope}%
\pgfsys@transformshift{9.150318in}{6.202198in}%
\pgfsys@useobject{currentmarker}{}%
\end{pgfscope}%
\begin{pgfscope}%
\pgfsys@transformshift{9.153678in}{6.202685in}%
\pgfsys@useobject{currentmarker}{}%
\end{pgfscope}%
\begin{pgfscope}%
\pgfsys@transformshift{9.157039in}{6.202813in}%
\pgfsys@useobject{currentmarker}{}%
\end{pgfscope}%
\begin{pgfscope}%
\pgfsys@transformshift{9.160401in}{6.202897in}%
\pgfsys@useobject{currentmarker}{}%
\end{pgfscope}%
\begin{pgfscope}%
\pgfsys@transformshift{9.163765in}{6.203168in}%
\pgfsys@useobject{currentmarker}{}%
\end{pgfscope}%
\begin{pgfscope}%
\pgfsys@transformshift{9.167129in}{6.203189in}%
\pgfsys@useobject{currentmarker}{}%
\end{pgfscope}%
\begin{pgfscope}%
\pgfsys@transformshift{9.170496in}{6.203537in}%
\pgfsys@useobject{currentmarker}{}%
\end{pgfscope}%
\begin{pgfscope}%
\pgfsys@transformshift{9.173863in}{6.203565in}%
\pgfsys@useobject{currentmarker}{}%
\end{pgfscope}%
\begin{pgfscope}%
\pgfsys@transformshift{9.177232in}{6.204038in}%
\pgfsys@useobject{currentmarker}{}%
\end{pgfscope}%
\begin{pgfscope}%
\pgfsys@transformshift{9.180602in}{6.204183in}%
\pgfsys@useobject{currentmarker}{}%
\end{pgfscope}%
\begin{pgfscope}%
\pgfsys@transformshift{9.183974in}{6.204887in}%
\pgfsys@useobject{currentmarker}{}%
\end{pgfscope}%
\begin{pgfscope}%
\pgfsys@transformshift{9.187347in}{6.204897in}%
\pgfsys@useobject{currentmarker}{}%
\end{pgfscope}%
\begin{pgfscope}%
\pgfsys@transformshift{9.190721in}{6.205202in}%
\pgfsys@useobject{currentmarker}{}%
\end{pgfscope}%
\begin{pgfscope}%
\pgfsys@transformshift{9.194097in}{6.205899in}%
\pgfsys@useobject{currentmarker}{}%
\end{pgfscope}%
\begin{pgfscope}%
\pgfsys@transformshift{9.197474in}{6.205901in}%
\pgfsys@useobject{currentmarker}{}%
\end{pgfscope}%
\begin{pgfscope}%
\pgfsys@transformshift{9.200853in}{6.205972in}%
\pgfsys@useobject{currentmarker}{}%
\end{pgfscope}%
\begin{pgfscope}%
\pgfsys@transformshift{9.204233in}{6.206005in}%
\pgfsys@useobject{currentmarker}{}%
\end{pgfscope}%
\begin{pgfscope}%
\pgfsys@transformshift{9.207614in}{6.206307in}%
\pgfsys@useobject{currentmarker}{}%
\end{pgfscope}%
\begin{pgfscope}%
\pgfsys@transformshift{9.210997in}{6.206444in}%
\pgfsys@useobject{currentmarker}{}%
\end{pgfscope}%
\begin{pgfscope}%
\pgfsys@transformshift{9.214382in}{6.207201in}%
\pgfsys@useobject{currentmarker}{}%
\end{pgfscope}%
\begin{pgfscope}%
\pgfsys@transformshift{9.217768in}{6.207382in}%
\pgfsys@useobject{currentmarker}{}%
\end{pgfscope}%
\begin{pgfscope}%
\pgfsys@transformshift{9.221155in}{6.207662in}%
\pgfsys@useobject{currentmarker}{}%
\end{pgfscope}%
\begin{pgfscope}%
\pgfsys@transformshift{9.224544in}{6.207720in}%
\pgfsys@useobject{currentmarker}{}%
\end{pgfscope}%
\begin{pgfscope}%
\pgfsys@transformshift{9.227934in}{6.207991in}%
\pgfsys@useobject{currentmarker}{}%
\end{pgfscope}%
\begin{pgfscope}%
\pgfsys@transformshift{9.231326in}{6.208207in}%
\pgfsys@useobject{currentmarker}{}%
\end{pgfscope}%
\begin{pgfscope}%
\pgfsys@transformshift{9.234720in}{6.208333in}%
\pgfsys@useobject{currentmarker}{}%
\end{pgfscope}%
\begin{pgfscope}%
\pgfsys@transformshift{9.238115in}{6.208412in}%
\pgfsys@useobject{currentmarker}{}%
\end{pgfscope}%
\begin{pgfscope}%
\pgfsys@transformshift{9.241511in}{6.208474in}%
\pgfsys@useobject{currentmarker}{}%
\end{pgfscope}%
\begin{pgfscope}%
\pgfsys@transformshift{9.244909in}{6.208897in}%
\pgfsys@useobject{currentmarker}{}%
\end{pgfscope}%
\begin{pgfscope}%
\pgfsys@transformshift{9.248309in}{6.209015in}%
\pgfsys@useobject{currentmarker}{}%
\end{pgfscope}%
\begin{pgfscope}%
\pgfsys@transformshift{9.251710in}{6.209038in}%
\pgfsys@useobject{currentmarker}{}%
\end{pgfscope}%
\begin{pgfscope}%
\pgfsys@transformshift{9.255113in}{6.209608in}%
\pgfsys@useobject{currentmarker}{}%
\end{pgfscope}%
\begin{pgfscope}%
\pgfsys@transformshift{9.258518in}{6.209902in}%
\pgfsys@useobject{currentmarker}{}%
\end{pgfscope}%
\begin{pgfscope}%
\pgfsys@transformshift{9.261924in}{6.210078in}%
\pgfsys@useobject{currentmarker}{}%
\end{pgfscope}%
\begin{pgfscope}%
\pgfsys@transformshift{9.265331in}{6.211089in}%
\pgfsys@useobject{currentmarker}{}%
\end{pgfscope}%
\begin{pgfscope}%
\pgfsys@transformshift{9.268741in}{6.211156in}%
\pgfsys@useobject{currentmarker}{}%
\end{pgfscope}%
\begin{pgfscope}%
\pgfsys@transformshift{9.272152in}{6.211314in}%
\pgfsys@useobject{currentmarker}{}%
\end{pgfscope}%
\begin{pgfscope}%
\pgfsys@transformshift{9.275565in}{6.211342in}%
\pgfsys@useobject{currentmarker}{}%
\end{pgfscope}%
\begin{pgfscope}%
\pgfsys@transformshift{9.278979in}{6.211520in}%
\pgfsys@useobject{currentmarker}{}%
\end{pgfscope}%
\begin{pgfscope}%
\pgfsys@transformshift{9.282395in}{6.211621in}%
\pgfsys@useobject{currentmarker}{}%
\end{pgfscope}%
\begin{pgfscope}%
\pgfsys@transformshift{9.285813in}{6.211642in}%
\pgfsys@useobject{currentmarker}{}%
\end{pgfscope}%
\begin{pgfscope}%
\pgfsys@transformshift{9.289233in}{6.212100in}%
\pgfsys@useobject{currentmarker}{}%
\end{pgfscope}%
\begin{pgfscope}%
\pgfsys@transformshift{9.292654in}{6.212273in}%
\pgfsys@useobject{currentmarker}{}%
\end{pgfscope}%
\begin{pgfscope}%
\pgfsys@transformshift{9.296077in}{6.212373in}%
\pgfsys@useobject{currentmarker}{}%
\end{pgfscope}%
\begin{pgfscope}%
\pgfsys@transformshift{9.299502in}{6.212634in}%
\pgfsys@useobject{currentmarker}{}%
\end{pgfscope}%
\begin{pgfscope}%
\pgfsys@transformshift{9.302928in}{6.212831in}%
\pgfsys@useobject{currentmarker}{}%
\end{pgfscope}%
\begin{pgfscope}%
\pgfsys@transformshift{9.306357in}{6.213317in}%
\pgfsys@useobject{currentmarker}{}%
\end{pgfscope}%
\begin{pgfscope}%
\pgfsys@transformshift{9.309787in}{6.213375in}%
\pgfsys@useobject{currentmarker}{}%
\end{pgfscope}%
\begin{pgfscope}%
\pgfsys@transformshift{9.313219in}{6.214034in}%
\pgfsys@useobject{currentmarker}{}%
\end{pgfscope}%
\begin{pgfscope}%
\pgfsys@transformshift{9.316652in}{6.214054in}%
\pgfsys@useobject{currentmarker}{}%
\end{pgfscope}%
\begin{pgfscope}%
\pgfsys@transformshift{9.320088in}{6.214204in}%
\pgfsys@useobject{currentmarker}{}%
\end{pgfscope}%
\begin{pgfscope}%
\pgfsys@transformshift{9.323526in}{6.214263in}%
\pgfsys@useobject{currentmarker}{}%
\end{pgfscope}%
\begin{pgfscope}%
\pgfsys@transformshift{9.326965in}{6.214429in}%
\pgfsys@useobject{currentmarker}{}%
\end{pgfscope}%
\begin{pgfscope}%
\pgfsys@transformshift{9.330406in}{6.214534in}%
\pgfsys@useobject{currentmarker}{}%
\end{pgfscope}%
\begin{pgfscope}%
\pgfsys@transformshift{9.333849in}{6.216064in}%
\pgfsys@useobject{currentmarker}{}%
\end{pgfscope}%
\begin{pgfscope}%
\pgfsys@transformshift{9.337294in}{6.216064in}%
\pgfsys@useobject{currentmarker}{}%
\end{pgfscope}%
\begin{pgfscope}%
\pgfsys@transformshift{9.340741in}{6.216309in}%
\pgfsys@useobject{currentmarker}{}%
\end{pgfscope}%
\begin{pgfscope}%
\pgfsys@transformshift{9.344190in}{6.216386in}%
\pgfsys@useobject{currentmarker}{}%
\end{pgfscope}%
\begin{pgfscope}%
\pgfsys@transformshift{9.347640in}{6.216420in}%
\pgfsys@useobject{currentmarker}{}%
\end{pgfscope}%
\begin{pgfscope}%
\pgfsys@transformshift{9.351093in}{6.216774in}%
\pgfsys@useobject{currentmarker}{}%
\end{pgfscope}%
\begin{pgfscope}%
\pgfsys@transformshift{9.354547in}{6.216791in}%
\pgfsys@useobject{currentmarker}{}%
\end{pgfscope}%
\begin{pgfscope}%
\pgfsys@transformshift{9.358004in}{6.216799in}%
\pgfsys@useobject{currentmarker}{}%
\end{pgfscope}%
\begin{pgfscope}%
\pgfsys@transformshift{9.361463in}{6.216935in}%
\pgfsys@useobject{currentmarker}{}%
\end{pgfscope}%
\begin{pgfscope}%
\pgfsys@transformshift{9.364923in}{6.217158in}%
\pgfsys@useobject{currentmarker}{}%
\end{pgfscope}%
\begin{pgfscope}%
\pgfsys@transformshift{9.368386in}{6.217337in}%
\pgfsys@useobject{currentmarker}{}%
\end{pgfscope}%
\begin{pgfscope}%
\pgfsys@transformshift{9.371850in}{6.217412in}%
\pgfsys@useobject{currentmarker}{}%
\end{pgfscope}%
\begin{pgfscope}%
\pgfsys@transformshift{9.375317in}{6.217931in}%
\pgfsys@useobject{currentmarker}{}%
\end{pgfscope}%
\begin{pgfscope}%
\pgfsys@transformshift{9.378786in}{6.217938in}%
\pgfsys@useobject{currentmarker}{}%
\end{pgfscope}%
\begin{pgfscope}%
\pgfsys@transformshift{9.382256in}{6.218323in}%
\pgfsys@useobject{currentmarker}{}%
\end{pgfscope}%
\begin{pgfscope}%
\pgfsys@transformshift{9.385729in}{6.218398in}%
\pgfsys@useobject{currentmarker}{}%
\end{pgfscope}%
\begin{pgfscope}%
\pgfsys@transformshift{9.389204in}{6.218838in}%
\pgfsys@useobject{currentmarker}{}%
\end{pgfscope}%
\begin{pgfscope}%
\pgfsys@transformshift{9.392681in}{6.218895in}%
\pgfsys@useobject{currentmarker}{}%
\end{pgfscope}%
\begin{pgfscope}%
\pgfsys@transformshift{9.396160in}{6.218904in}%
\pgfsys@useobject{currentmarker}{}%
\end{pgfscope}%
\begin{pgfscope}%
\pgfsys@transformshift{9.399641in}{6.219503in}%
\pgfsys@useobject{currentmarker}{}%
\end{pgfscope}%
\begin{pgfscope}%
\pgfsys@transformshift{9.403125in}{6.219629in}%
\pgfsys@useobject{currentmarker}{}%
\end{pgfscope}%
\begin{pgfscope}%
\pgfsys@transformshift{9.406610in}{6.219695in}%
\pgfsys@useobject{currentmarker}{}%
\end{pgfscope}%
\begin{pgfscope}%
\pgfsys@transformshift{9.410098in}{6.219723in}%
\pgfsys@useobject{currentmarker}{}%
\end{pgfscope}%
\begin{pgfscope}%
\pgfsys@transformshift{9.413588in}{6.219982in}%
\pgfsys@useobject{currentmarker}{}%
\end{pgfscope}%
\begin{pgfscope}%
\pgfsys@transformshift{9.417080in}{6.220344in}%
\pgfsys@useobject{currentmarker}{}%
\end{pgfscope}%
\begin{pgfscope}%
\pgfsys@transformshift{9.420574in}{6.220674in}%
\pgfsys@useobject{currentmarker}{}%
\end{pgfscope}%
\begin{pgfscope}%
\pgfsys@transformshift{9.424070in}{6.221637in}%
\pgfsys@useobject{currentmarker}{}%
\end{pgfscope}%
\begin{pgfscope}%
\pgfsys@transformshift{9.427569in}{6.221895in}%
\pgfsys@useobject{currentmarker}{}%
\end{pgfscope}%
\begin{pgfscope}%
\pgfsys@transformshift{9.431070in}{6.222671in}%
\pgfsys@useobject{currentmarker}{}%
\end{pgfscope}%
\begin{pgfscope}%
\pgfsys@transformshift{9.434573in}{6.222671in}%
\pgfsys@useobject{currentmarker}{}%
\end{pgfscope}%
\begin{pgfscope}%
\pgfsys@transformshift{9.438079in}{6.223146in}%
\pgfsys@useobject{currentmarker}{}%
\end{pgfscope}%
\begin{pgfscope}%
\pgfsys@transformshift{9.441587in}{6.223557in}%
\pgfsys@useobject{currentmarker}{}%
\end{pgfscope}%
\begin{pgfscope}%
\pgfsys@transformshift{9.445097in}{6.223658in}%
\pgfsys@useobject{currentmarker}{}%
\end{pgfscope}%
\begin{pgfscope}%
\pgfsys@transformshift{9.448609in}{6.223934in}%
\pgfsys@useobject{currentmarker}{}%
\end{pgfscope}%
\begin{pgfscope}%
\pgfsys@transformshift{9.452124in}{6.223987in}%
\pgfsys@useobject{currentmarker}{}%
\end{pgfscope}%
\begin{pgfscope}%
\pgfsys@transformshift{9.455641in}{6.224114in}%
\pgfsys@useobject{currentmarker}{}%
\end{pgfscope}%
\begin{pgfscope}%
\pgfsys@transformshift{9.459161in}{6.224286in}%
\pgfsys@useobject{currentmarker}{}%
\end{pgfscope}%
\begin{pgfscope}%
\pgfsys@transformshift{9.462682in}{6.224572in}%
\pgfsys@useobject{currentmarker}{}%
\end{pgfscope}%
\begin{pgfscope}%
\pgfsys@transformshift{9.466207in}{6.225358in}%
\pgfsys@useobject{currentmarker}{}%
\end{pgfscope}%
\begin{pgfscope}%
\pgfsys@transformshift{9.469733in}{6.225868in}%
\pgfsys@useobject{currentmarker}{}%
\end{pgfscope}%
\begin{pgfscope}%
\pgfsys@transformshift{9.473262in}{6.225973in}%
\pgfsys@useobject{currentmarker}{}%
\end{pgfscope}%
\begin{pgfscope}%
\pgfsys@transformshift{9.476794in}{6.226126in}%
\pgfsys@useobject{currentmarker}{}%
\end{pgfscope}%
\begin{pgfscope}%
\pgfsys@transformshift{9.480328in}{6.226278in}%
\pgfsys@useobject{currentmarker}{}%
\end{pgfscope}%
\begin{pgfscope}%
\pgfsys@transformshift{9.483864in}{6.226782in}%
\pgfsys@useobject{currentmarker}{}%
\end{pgfscope}%
\begin{pgfscope}%
\pgfsys@transformshift{9.487403in}{6.226826in}%
\pgfsys@useobject{currentmarker}{}%
\end{pgfscope}%
\begin{pgfscope}%
\pgfsys@transformshift{9.490945in}{6.226910in}%
\pgfsys@useobject{currentmarker}{}%
\end{pgfscope}%
\begin{pgfscope}%
\pgfsys@transformshift{9.494488in}{6.226951in}%
\pgfsys@useobject{currentmarker}{}%
\end{pgfscope}%
\begin{pgfscope}%
\pgfsys@transformshift{9.498035in}{6.226961in}%
\pgfsys@useobject{currentmarker}{}%
\end{pgfscope}%
\begin{pgfscope}%
\pgfsys@transformshift{9.501584in}{6.227307in}%
\pgfsys@useobject{currentmarker}{}%
\end{pgfscope}%
\begin{pgfscope}%
\pgfsys@transformshift{9.505135in}{6.227342in}%
\pgfsys@useobject{currentmarker}{}%
\end{pgfscope}%
\begin{pgfscope}%
\pgfsys@transformshift{9.508690in}{6.227464in}%
\pgfsys@useobject{currentmarker}{}%
\end{pgfscope}%
\begin{pgfscope}%
\pgfsys@transformshift{9.512246in}{6.227516in}%
\pgfsys@useobject{currentmarker}{}%
\end{pgfscope}%
\begin{pgfscope}%
\pgfsys@transformshift{9.515805in}{6.227618in}%
\pgfsys@useobject{currentmarker}{}%
\end{pgfscope}%
\begin{pgfscope}%
\pgfsys@transformshift{9.519367in}{6.227930in}%
\pgfsys@useobject{currentmarker}{}%
\end{pgfscope}%
\begin{pgfscope}%
\pgfsys@transformshift{9.522932in}{6.228734in}%
\pgfsys@useobject{currentmarker}{}%
\end{pgfscope}%
\begin{pgfscope}%
\pgfsys@transformshift{9.526499in}{6.228822in}%
\pgfsys@useobject{currentmarker}{}%
\end{pgfscope}%
\begin{pgfscope}%
\pgfsys@transformshift{9.530069in}{6.228959in}%
\pgfsys@useobject{currentmarker}{}%
\end{pgfscope}%
\begin{pgfscope}%
\pgfsys@transformshift{9.533641in}{6.229363in}%
\pgfsys@useobject{currentmarker}{}%
\end{pgfscope}%
\begin{pgfscope}%
\pgfsys@transformshift{9.537217in}{6.229480in}%
\pgfsys@useobject{currentmarker}{}%
\end{pgfscope}%
\begin{pgfscope}%
\pgfsys@transformshift{9.540795in}{6.229931in}%
\pgfsys@useobject{currentmarker}{}%
\end{pgfscope}%
\begin{pgfscope}%
\pgfsys@transformshift{9.544375in}{6.229956in}%
\pgfsys@useobject{currentmarker}{}%
\end{pgfscope}%
\begin{pgfscope}%
\pgfsys@transformshift{9.547959in}{6.230125in}%
\pgfsys@useobject{currentmarker}{}%
\end{pgfscope}%
\begin{pgfscope}%
\pgfsys@transformshift{9.551545in}{6.230279in}%
\pgfsys@useobject{currentmarker}{}%
\end{pgfscope}%
\begin{pgfscope}%
\pgfsys@transformshift{9.555134in}{6.230915in}%
\pgfsys@useobject{currentmarker}{}%
\end{pgfscope}%
\begin{pgfscope}%
\pgfsys@transformshift{9.558725in}{6.230989in}%
\pgfsys@useobject{currentmarker}{}%
\end{pgfscope}%
\begin{pgfscope}%
\pgfsys@transformshift{9.562320in}{6.231023in}%
\pgfsys@useobject{currentmarker}{}%
\end{pgfscope}%
\begin{pgfscope}%
\pgfsys@transformshift{9.565917in}{6.231284in}%
\pgfsys@useobject{currentmarker}{}%
\end{pgfscope}%
\begin{pgfscope}%
\pgfsys@transformshift{9.569518in}{6.231553in}%
\pgfsys@useobject{currentmarker}{}%
\end{pgfscope}%
\begin{pgfscope}%
\pgfsys@transformshift{9.573121in}{6.232878in}%
\pgfsys@useobject{currentmarker}{}%
\end{pgfscope}%
\begin{pgfscope}%
\pgfsys@transformshift{9.576726in}{6.232965in}%
\pgfsys@useobject{currentmarker}{}%
\end{pgfscope}%
\begin{pgfscope}%
\pgfsys@transformshift{9.580335in}{6.233209in}%
\pgfsys@useobject{currentmarker}{}%
\end{pgfscope}%
\begin{pgfscope}%
\pgfsys@transformshift{9.583947in}{6.233425in}%
\pgfsys@useobject{currentmarker}{}%
\end{pgfscope}%
\begin{pgfscope}%
\pgfsys@transformshift{9.587562in}{6.233520in}%
\pgfsys@useobject{currentmarker}{}%
\end{pgfscope}%
\begin{pgfscope}%
\pgfsys@transformshift{9.591179in}{6.233569in}%
\pgfsys@useobject{currentmarker}{}%
\end{pgfscope}%
\begin{pgfscope}%
\pgfsys@transformshift{9.594800in}{6.233760in}%
\pgfsys@useobject{currentmarker}{}%
\end{pgfscope}%
\begin{pgfscope}%
\pgfsys@transformshift{9.598423in}{6.233811in}%
\pgfsys@useobject{currentmarker}{}%
\end{pgfscope}%
\begin{pgfscope}%
\pgfsys@transformshift{9.602050in}{6.233887in}%
\pgfsys@useobject{currentmarker}{}%
\end{pgfscope}%
\begin{pgfscope}%
\pgfsys@transformshift{9.605679in}{6.233904in}%
\pgfsys@useobject{currentmarker}{}%
\end{pgfscope}%
\begin{pgfscope}%
\pgfsys@transformshift{9.609312in}{6.233945in}%
\pgfsys@useobject{currentmarker}{}%
\end{pgfscope}%
\begin{pgfscope}%
\pgfsys@transformshift{9.612947in}{6.234124in}%
\pgfsys@useobject{currentmarker}{}%
\end{pgfscope}%
\begin{pgfscope}%
\pgfsys@transformshift{9.616586in}{6.234554in}%
\pgfsys@useobject{currentmarker}{}%
\end{pgfscope}%
\begin{pgfscope}%
\pgfsys@transformshift{9.620228in}{6.235846in}%
\pgfsys@useobject{currentmarker}{}%
\end{pgfscope}%
\begin{pgfscope}%
\pgfsys@transformshift{9.623872in}{6.235979in}%
\pgfsys@useobject{currentmarker}{}%
\end{pgfscope}%
\begin{pgfscope}%
\pgfsys@transformshift{9.627520in}{6.236256in}%
\pgfsys@useobject{currentmarker}{}%
\end{pgfscope}%
\begin{pgfscope}%
\pgfsys@transformshift{9.631171in}{6.236365in}%
\pgfsys@useobject{currentmarker}{}%
\end{pgfscope}%
\begin{pgfscope}%
\pgfsys@transformshift{9.634826in}{6.236899in}%
\pgfsys@useobject{currentmarker}{}%
\end{pgfscope}%
\begin{pgfscope}%
\pgfsys@transformshift{9.638483in}{6.237034in}%
\pgfsys@useobject{currentmarker}{}%
\end{pgfscope}%
\begin{pgfscope}%
\pgfsys@transformshift{9.642144in}{6.237077in}%
\pgfsys@useobject{currentmarker}{}%
\end{pgfscope}%
\begin{pgfscope}%
\pgfsys@transformshift{9.645807in}{6.237524in}%
\pgfsys@useobject{currentmarker}{}%
\end{pgfscope}%
\begin{pgfscope}%
\pgfsys@transformshift{9.649474in}{6.238180in}%
\pgfsys@useobject{currentmarker}{}%
\end{pgfscope}%
\begin{pgfscope}%
\pgfsys@transformshift{9.653144in}{6.238348in}%
\pgfsys@useobject{currentmarker}{}%
\end{pgfscope}%
\begin{pgfscope}%
\pgfsys@transformshift{9.656818in}{6.238357in}%
\pgfsys@useobject{currentmarker}{}%
\end{pgfscope}%
\begin{pgfscope}%
\pgfsys@transformshift{9.660495in}{6.238377in}%
\pgfsys@useobject{currentmarker}{}%
\end{pgfscope}%
\begin{pgfscope}%
\pgfsys@transformshift{9.664175in}{6.238394in}%
\pgfsys@useobject{currentmarker}{}%
\end{pgfscope}%
\begin{pgfscope}%
\pgfsys@transformshift{9.667858in}{6.238493in}%
\pgfsys@useobject{currentmarker}{}%
\end{pgfscope}%
\begin{pgfscope}%
\pgfsys@transformshift{9.671545in}{6.239031in}%
\pgfsys@useobject{currentmarker}{}%
\end{pgfscope}%
\begin{pgfscope}%
\pgfsys@transformshift{9.675235in}{6.239346in}%
\pgfsys@useobject{currentmarker}{}%
\end{pgfscope}%
\begin{pgfscope}%
\pgfsys@transformshift{9.678928in}{6.239427in}%
\pgfsys@useobject{currentmarker}{}%
\end{pgfscope}%
\begin{pgfscope}%
\pgfsys@transformshift{9.682625in}{6.239673in}%
\pgfsys@useobject{currentmarker}{}%
\end{pgfscope}%
\begin{pgfscope}%
\pgfsys@transformshift{9.686325in}{6.239687in}%
\pgfsys@useobject{currentmarker}{}%
\end{pgfscope}%
\begin{pgfscope}%
\pgfsys@transformshift{9.690029in}{6.239793in}%
\pgfsys@useobject{currentmarker}{}%
\end{pgfscope}%
\begin{pgfscope}%
\pgfsys@transformshift{9.693736in}{6.240811in}%
\pgfsys@useobject{currentmarker}{}%
\end{pgfscope}%
\begin{pgfscope}%
\pgfsys@transformshift{9.697446in}{6.241092in}%
\pgfsys@useobject{currentmarker}{}%
\end{pgfscope}%
\begin{pgfscope}%
\pgfsys@transformshift{9.701160in}{6.241302in}%
\pgfsys@useobject{currentmarker}{}%
\end{pgfscope}%
\begin{pgfscope}%
\pgfsys@transformshift{9.704878in}{6.241389in}%
\pgfsys@useobject{currentmarker}{}%
\end{pgfscope}%
\begin{pgfscope}%
\pgfsys@transformshift{9.708599in}{6.241533in}%
\pgfsys@useobject{currentmarker}{}%
\end{pgfscope}%
\begin{pgfscope}%
\pgfsys@transformshift{9.712324in}{6.241617in}%
\pgfsys@useobject{currentmarker}{}%
\end{pgfscope}%
\begin{pgfscope}%
\pgfsys@transformshift{9.716052in}{6.241801in}%
\pgfsys@useobject{currentmarker}{}%
\end{pgfscope}%
\begin{pgfscope}%
\pgfsys@transformshift{9.719784in}{6.241889in}%
\pgfsys@useobject{currentmarker}{}%
\end{pgfscope}%
\begin{pgfscope}%
\pgfsys@transformshift{9.723519in}{6.242464in}%
\pgfsys@useobject{currentmarker}{}%
\end{pgfscope}%
\begin{pgfscope}%
\pgfsys@transformshift{9.727258in}{6.242521in}%
\pgfsys@useobject{currentmarker}{}%
\end{pgfscope}%
\begin{pgfscope}%
\pgfsys@transformshift{9.731001in}{6.242619in}%
\pgfsys@useobject{currentmarker}{}%
\end{pgfscope}%
\begin{pgfscope}%
\pgfsys@transformshift{9.734747in}{6.242697in}%
\pgfsys@useobject{currentmarker}{}%
\end{pgfscope}%
\begin{pgfscope}%
\pgfsys@transformshift{9.738497in}{6.243318in}%
\pgfsys@useobject{currentmarker}{}%
\end{pgfscope}%
\begin{pgfscope}%
\pgfsys@transformshift{9.742251in}{6.243560in}%
\pgfsys@useobject{currentmarker}{}%
\end{pgfscope}%
\begin{pgfscope}%
\pgfsys@transformshift{9.746008in}{6.243909in}%
\pgfsys@useobject{currentmarker}{}%
\end{pgfscope}%
\begin{pgfscope}%
\pgfsys@transformshift{9.749769in}{6.244288in}%
\pgfsys@useobject{currentmarker}{}%
\end{pgfscope}%
\begin{pgfscope}%
\pgfsys@transformshift{9.753534in}{6.244492in}%
\pgfsys@useobject{currentmarker}{}%
\end{pgfscope}%
\begin{pgfscope}%
\pgfsys@transformshift{9.757303in}{6.244583in}%
\pgfsys@useobject{currentmarker}{}%
\end{pgfscope}%
\begin{pgfscope}%
\pgfsys@transformshift{9.761076in}{6.244787in}%
\pgfsys@useobject{currentmarker}{}%
\end{pgfscope}%
\begin{pgfscope}%
\pgfsys@transformshift{9.764852in}{6.245072in}%
\pgfsys@useobject{currentmarker}{}%
\end{pgfscope}%
\begin{pgfscope}%
\pgfsys@transformshift{9.768632in}{6.245656in}%
\pgfsys@useobject{currentmarker}{}%
\end{pgfscope}%
\begin{pgfscope}%
\pgfsys@transformshift{9.772417in}{6.245917in}%
\pgfsys@useobject{currentmarker}{}%
\end{pgfscope}%
\begin{pgfscope}%
\pgfsys@transformshift{9.776205in}{6.246265in}%
\pgfsys@useobject{currentmarker}{}%
\end{pgfscope}%
\begin{pgfscope}%
\pgfsys@transformshift{9.779997in}{6.246436in}%
\pgfsys@useobject{currentmarker}{}%
\end{pgfscope}%
\begin{pgfscope}%
\pgfsys@transformshift{9.783793in}{6.247052in}%
\pgfsys@useobject{currentmarker}{}%
\end{pgfscope}%
\begin{pgfscope}%
\pgfsys@transformshift{9.787593in}{6.247086in}%
\pgfsys@useobject{currentmarker}{}%
\end{pgfscope}%
\begin{pgfscope}%
\pgfsys@transformshift{9.791396in}{6.247175in}%
\pgfsys@useobject{currentmarker}{}%
\end{pgfscope}%
\begin{pgfscope}%
\pgfsys@transformshift{9.795204in}{6.247742in}%
\pgfsys@useobject{currentmarker}{}%
\end{pgfscope}%
\begin{pgfscope}%
\pgfsys@transformshift{9.799016in}{6.247922in}%
\pgfsys@useobject{currentmarker}{}%
\end{pgfscope}%
\begin{pgfscope}%
\pgfsys@transformshift{9.802832in}{6.248049in}%
\pgfsys@useobject{currentmarker}{}%
\end{pgfscope}%
\begin{pgfscope}%
\pgfsys@transformshift{9.806652in}{6.248595in}%
\pgfsys@useobject{currentmarker}{}%
\end{pgfscope}%
\begin{pgfscope}%
\pgfsys@transformshift{9.810477in}{6.248635in}%
\pgfsys@useobject{currentmarker}{}%
\end{pgfscope}%
\begin{pgfscope}%
\pgfsys@transformshift{9.814305in}{6.248831in}%
\pgfsys@useobject{currentmarker}{}%
\end{pgfscope}%
\begin{pgfscope}%
\pgfsys@transformshift{9.818137in}{6.248896in}%
\pgfsys@useobject{currentmarker}{}%
\end{pgfscope}%
\begin{pgfscope}%
\pgfsys@transformshift{9.821974in}{6.248911in}%
\pgfsys@useobject{currentmarker}{}%
\end{pgfscope}%
\begin{pgfscope}%
\pgfsys@transformshift{9.825815in}{6.249356in}%
\pgfsys@useobject{currentmarker}{}%
\end{pgfscope}%
\begin{pgfscope}%
\pgfsys@transformshift{9.829660in}{6.249391in}%
\pgfsys@useobject{currentmarker}{}%
\end{pgfscope}%
\begin{pgfscope}%
\pgfsys@transformshift{9.833509in}{6.249581in}%
\pgfsys@useobject{currentmarker}{}%
\end{pgfscope}%
\begin{pgfscope}%
\pgfsys@transformshift{9.837363in}{6.249630in}%
\pgfsys@useobject{currentmarker}{}%
\end{pgfscope}%
\begin{pgfscope}%
\pgfsys@transformshift{9.841221in}{6.249700in}%
\pgfsys@useobject{currentmarker}{}%
\end{pgfscope}%
\begin{pgfscope}%
\pgfsys@transformshift{9.845083in}{6.249719in}%
\pgfsys@useobject{currentmarker}{}%
\end{pgfscope}%
\begin{pgfscope}%
\pgfsys@transformshift{9.848950in}{6.250010in}%
\pgfsys@useobject{currentmarker}{}%
\end{pgfscope}%
\begin{pgfscope}%
\pgfsys@transformshift{9.852821in}{6.250213in}%
\pgfsys@useobject{currentmarker}{}%
\end{pgfscope}%
\begin{pgfscope}%
\pgfsys@transformshift{9.856696in}{6.250240in}%
\pgfsys@useobject{currentmarker}{}%
\end{pgfscope}%
\begin{pgfscope}%
\pgfsys@transformshift{9.860576in}{6.250397in}%
\pgfsys@useobject{currentmarker}{}%
\end{pgfscope}%
\begin{pgfscope}%
\pgfsys@transformshift{9.864460in}{6.250529in}%
\pgfsys@useobject{currentmarker}{}%
\end{pgfscope}%
\begin{pgfscope}%
\pgfsys@transformshift{9.868349in}{6.250938in}%
\pgfsys@useobject{currentmarker}{}%
\end{pgfscope}%
\begin{pgfscope}%
\pgfsys@transformshift{9.872242in}{6.250952in}%
\pgfsys@useobject{currentmarker}{}%
\end{pgfscope}%
\begin{pgfscope}%
\pgfsys@transformshift{9.876139in}{6.251221in}%
\pgfsys@useobject{currentmarker}{}%
\end{pgfscope}%
\begin{pgfscope}%
\pgfsys@transformshift{9.880042in}{6.251551in}%
\pgfsys@useobject{currentmarker}{}%
\end{pgfscope}%
\begin{pgfscope}%
\pgfsys@transformshift{9.883948in}{6.251624in}%
\pgfsys@useobject{currentmarker}{}%
\end{pgfscope}%
\begin{pgfscope}%
\pgfsys@transformshift{9.887860in}{6.251853in}%
\pgfsys@useobject{currentmarker}{}%
\end{pgfscope}%
\begin{pgfscope}%
\pgfsys@transformshift{9.891776in}{6.251863in}%
\pgfsys@useobject{currentmarker}{}%
\end{pgfscope}%
\begin{pgfscope}%
\pgfsys@transformshift{9.895697in}{6.251980in}%
\pgfsys@useobject{currentmarker}{}%
\end{pgfscope}%
\begin{pgfscope}%
\pgfsys@transformshift{9.899622in}{6.252230in}%
\pgfsys@useobject{currentmarker}{}%
\end{pgfscope}%
\begin{pgfscope}%
\pgfsys@transformshift{9.903552in}{6.252261in}%
\pgfsys@useobject{currentmarker}{}%
\end{pgfscope}%
\begin{pgfscope}%
\pgfsys@transformshift{9.907487in}{6.252660in}%
\pgfsys@useobject{currentmarker}{}%
\end{pgfscope}%
\begin{pgfscope}%
\pgfsys@transformshift{9.911426in}{6.253033in}%
\pgfsys@useobject{currentmarker}{}%
\end{pgfscope}%
\begin{pgfscope}%
\pgfsys@transformshift{9.915371in}{6.253076in}%
\pgfsys@useobject{currentmarker}{}%
\end{pgfscope}%
\begin{pgfscope}%
\pgfsys@transformshift{9.919320in}{6.253290in}%
\pgfsys@useobject{currentmarker}{}%
\end{pgfscope}%
\begin{pgfscope}%
\pgfsys@transformshift{9.923274in}{6.253366in}%
\pgfsys@useobject{currentmarker}{}%
\end{pgfscope}%
\begin{pgfscope}%
\pgfsys@transformshift{9.927233in}{6.253432in}%
\pgfsys@useobject{currentmarker}{}%
\end{pgfscope}%
\begin{pgfscope}%
\pgfsys@transformshift{9.931196in}{6.253473in}%
\pgfsys@useobject{currentmarker}{}%
\end{pgfscope}%
\begin{pgfscope}%
\pgfsys@transformshift{9.935165in}{6.253618in}%
\pgfsys@useobject{currentmarker}{}%
\end{pgfscope}%
\begin{pgfscope}%
\pgfsys@transformshift{9.939139in}{6.254248in}%
\pgfsys@useobject{currentmarker}{}%
\end{pgfscope}%
\begin{pgfscope}%
\pgfsys@transformshift{9.943117in}{6.254317in}%
\pgfsys@useobject{currentmarker}{}%
\end{pgfscope}%
\begin{pgfscope}%
\pgfsys@transformshift{9.947101in}{6.254390in}%
\pgfsys@useobject{currentmarker}{}%
\end{pgfscope}%
\begin{pgfscope}%
\pgfsys@transformshift{9.951090in}{6.254460in}%
\pgfsys@useobject{currentmarker}{}%
\end{pgfscope}%
\begin{pgfscope}%
\pgfsys@transformshift{9.955083in}{6.254699in}%
\pgfsys@useobject{currentmarker}{}%
\end{pgfscope}%
\begin{pgfscope}%
\pgfsys@transformshift{9.959082in}{6.255074in}%
\pgfsys@useobject{currentmarker}{}%
\end{pgfscope}%
\begin{pgfscope}%
\pgfsys@transformshift{9.963086in}{6.255763in}%
\pgfsys@useobject{currentmarker}{}%
\end{pgfscope}%
\begin{pgfscope}%
\pgfsys@transformshift{9.967095in}{6.256103in}%
\pgfsys@useobject{currentmarker}{}%
\end{pgfscope}%
\begin{pgfscope}%
\pgfsys@transformshift{9.971110in}{6.256371in}%
\pgfsys@useobject{currentmarker}{}%
\end{pgfscope}%
\begin{pgfscope}%
\pgfsys@transformshift{9.975129in}{6.257019in}%
\pgfsys@useobject{currentmarker}{}%
\end{pgfscope}%
\begin{pgfscope}%
\pgfsys@transformshift{9.979154in}{6.257025in}%
\pgfsys@useobject{currentmarker}{}%
\end{pgfscope}%
\begin{pgfscope}%
\pgfsys@transformshift{9.983184in}{6.257080in}%
\pgfsys@useobject{currentmarker}{}%
\end{pgfscope}%
\begin{pgfscope}%
\pgfsys@transformshift{9.987219in}{6.257509in}%
\pgfsys@useobject{currentmarker}{}%
\end{pgfscope}%
\begin{pgfscope}%
\pgfsys@transformshift{9.991260in}{6.257866in}%
\pgfsys@useobject{currentmarker}{}%
\end{pgfscope}%
\begin{pgfscope}%
\pgfsys@transformshift{9.995306in}{6.258392in}%
\pgfsys@useobject{currentmarker}{}%
\end{pgfscope}%
\begin{pgfscope}%
\pgfsys@transformshift{9.999357in}{6.258574in}%
\pgfsys@useobject{currentmarker}{}%
\end{pgfscope}%
\begin{pgfscope}%
\pgfsys@transformshift{10.003414in}{6.259209in}%
\pgfsys@useobject{currentmarker}{}%
\end{pgfscope}%
\begin{pgfscope}%
\pgfsys@transformshift{10.007477in}{6.259388in}%
\pgfsys@useobject{currentmarker}{}%
\end{pgfscope}%
\begin{pgfscope}%
\pgfsys@transformshift{10.011545in}{6.259591in}%
\pgfsys@useobject{currentmarker}{}%
\end{pgfscope}%
\begin{pgfscope}%
\pgfsys@transformshift{10.015618in}{6.259693in}%
\pgfsys@useobject{currentmarker}{}%
\end{pgfscope}%
\begin{pgfscope}%
\pgfsys@transformshift{10.019697in}{6.260489in}%
\pgfsys@useobject{currentmarker}{}%
\end{pgfscope}%
\begin{pgfscope}%
\pgfsys@transformshift{10.023781in}{6.260541in}%
\pgfsys@useobject{currentmarker}{}%
\end{pgfscope}%
\begin{pgfscope}%
\pgfsys@transformshift{10.027872in}{6.260760in}%
\pgfsys@useobject{currentmarker}{}%
\end{pgfscope}%
\begin{pgfscope}%
\pgfsys@transformshift{10.031967in}{6.260825in}%
\pgfsys@useobject{currentmarker}{}%
\end{pgfscope}%
\begin{pgfscope}%
\pgfsys@transformshift{10.036069in}{6.261099in}%
\pgfsys@useobject{currentmarker}{}%
\end{pgfscope}%
\begin{pgfscope}%
\pgfsys@transformshift{10.040176in}{6.261263in}%
\pgfsys@useobject{currentmarker}{}%
\end{pgfscope}%
\begin{pgfscope}%
\pgfsys@transformshift{10.044289in}{6.261365in}%
\pgfsys@useobject{currentmarker}{}%
\end{pgfscope}%
\begin{pgfscope}%
\pgfsys@transformshift{10.048408in}{6.261979in}%
\pgfsys@useobject{currentmarker}{}%
\end{pgfscope}%
\begin{pgfscope}%
\pgfsys@transformshift{10.052533in}{6.262198in}%
\pgfsys@useobject{currentmarker}{}%
\end{pgfscope}%
\begin{pgfscope}%
\pgfsys@transformshift{10.056663in}{6.262230in}%
\pgfsys@useobject{currentmarker}{}%
\end{pgfscope}%
\begin{pgfscope}%
\pgfsys@transformshift{10.060800in}{6.262356in}%
\pgfsys@useobject{currentmarker}{}%
\end{pgfscope}%
\begin{pgfscope}%
\pgfsys@transformshift{10.064942in}{6.262490in}%
\pgfsys@useobject{currentmarker}{}%
\end{pgfscope}%
\begin{pgfscope}%
\pgfsys@transformshift{10.069090in}{6.262688in}%
\pgfsys@useobject{currentmarker}{}%
\end{pgfscope}%
\begin{pgfscope}%
\pgfsys@transformshift{10.073245in}{6.262983in}%
\pgfsys@useobject{currentmarker}{}%
\end{pgfscope}%
\begin{pgfscope}%
\pgfsys@transformshift{10.077405in}{6.263337in}%
\pgfsys@useobject{currentmarker}{}%
\end{pgfscope}%
\begin{pgfscope}%
\pgfsys@transformshift{10.081571in}{6.263390in}%
\pgfsys@useobject{currentmarker}{}%
\end{pgfscope}%
\begin{pgfscope}%
\pgfsys@transformshift{10.085744in}{6.263746in}%
\pgfsys@useobject{currentmarker}{}%
\end{pgfscope}%
\begin{pgfscope}%
\pgfsys@transformshift{10.089923in}{6.264252in}%
\pgfsys@useobject{currentmarker}{}%
\end{pgfscope}%
\begin{pgfscope}%
\pgfsys@transformshift{10.094108in}{6.264301in}%
\pgfsys@useobject{currentmarker}{}%
\end{pgfscope}%
\begin{pgfscope}%
\pgfsys@transformshift{10.098299in}{6.264798in}%
\pgfsys@useobject{currentmarker}{}%
\end{pgfscope}%
\begin{pgfscope}%
\pgfsys@transformshift{10.102496in}{6.264968in}%
\pgfsys@useobject{currentmarker}{}%
\end{pgfscope}%
\begin{pgfscope}%
\pgfsys@transformshift{10.106700in}{6.265809in}%
\pgfsys@useobject{currentmarker}{}%
\end{pgfscope}%
\begin{pgfscope}%
\pgfsys@transformshift{10.110910in}{6.266090in}%
\pgfsys@useobject{currentmarker}{}%
\end{pgfscope}%
\begin{pgfscope}%
\pgfsys@transformshift{10.115126in}{6.266130in}%
\pgfsys@useobject{currentmarker}{}%
\end{pgfscope}%
\begin{pgfscope}%
\pgfsys@transformshift{10.119349in}{6.266576in}%
\pgfsys@useobject{currentmarker}{}%
\end{pgfscope}%
\begin{pgfscope}%
\pgfsys@transformshift{10.123578in}{6.266727in}%
\pgfsys@useobject{currentmarker}{}%
\end{pgfscope}%
\begin{pgfscope}%
\pgfsys@transformshift{10.127814in}{6.266989in}%
\pgfsys@useobject{currentmarker}{}%
\end{pgfscope}%
\begin{pgfscope}%
\pgfsys@transformshift{10.132056in}{6.267226in}%
\pgfsys@useobject{currentmarker}{}%
\end{pgfscope}%
\begin{pgfscope}%
\pgfsys@transformshift{10.136305in}{6.268047in}%
\pgfsys@useobject{currentmarker}{}%
\end{pgfscope}%
\begin{pgfscope}%
\pgfsys@transformshift{10.140561in}{6.268060in}%
\pgfsys@useobject{currentmarker}{}%
\end{pgfscope}%
\begin{pgfscope}%
\pgfsys@transformshift{10.144823in}{6.268556in}%
\pgfsys@useobject{currentmarker}{}%
\end{pgfscope}%
\begin{pgfscope}%
\pgfsys@transformshift{10.149092in}{6.268820in}%
\pgfsys@useobject{currentmarker}{}%
\end{pgfscope}%
\begin{pgfscope}%
\pgfsys@transformshift{10.153367in}{6.268845in}%
\pgfsys@useobject{currentmarker}{}%
\end{pgfscope}%
\begin{pgfscope}%
\pgfsys@transformshift{10.157650in}{6.268934in}%
\pgfsys@useobject{currentmarker}{}%
\end{pgfscope}%
\begin{pgfscope}%
\pgfsys@transformshift{10.161939in}{6.268937in}%
\pgfsys@useobject{currentmarker}{}%
\end{pgfscope}%
\begin{pgfscope}%
\pgfsys@transformshift{10.166235in}{6.269738in}%
\pgfsys@useobject{currentmarker}{}%
\end{pgfscope}%
\begin{pgfscope}%
\pgfsys@transformshift{10.170538in}{6.269851in}%
\pgfsys@useobject{currentmarker}{}%
\end{pgfscope}%
\begin{pgfscope}%
\pgfsys@transformshift{10.174848in}{6.271072in}%
\pgfsys@useobject{currentmarker}{}%
\end{pgfscope}%
\begin{pgfscope}%
\pgfsys@transformshift{10.179164in}{6.271112in}%
\pgfsys@useobject{currentmarker}{}%
\end{pgfscope}%
\begin{pgfscope}%
\pgfsys@transformshift{10.183488in}{6.271515in}%
\pgfsys@useobject{currentmarker}{}%
\end{pgfscope}%
\begin{pgfscope}%
\pgfsys@transformshift{10.187819in}{6.271720in}%
\pgfsys@useobject{currentmarker}{}%
\end{pgfscope}%
\begin{pgfscope}%
\pgfsys@transformshift{10.192157in}{6.272223in}%
\pgfsys@useobject{currentmarker}{}%
\end{pgfscope}%
\begin{pgfscope}%
\pgfsys@transformshift{10.196503in}{6.273115in}%
\pgfsys@useobject{currentmarker}{}%
\end{pgfscope}%
\begin{pgfscope}%
\pgfsys@transformshift{10.200855in}{6.273193in}%
\pgfsys@useobject{currentmarker}{}%
\end{pgfscope}%
\begin{pgfscope}%
\pgfsys@transformshift{10.205215in}{6.273254in}%
\pgfsys@useobject{currentmarker}{}%
\end{pgfscope}%
\begin{pgfscope}%
\pgfsys@transformshift{10.209582in}{6.273254in}%
\pgfsys@useobject{currentmarker}{}%
\end{pgfscope}%
\begin{pgfscope}%
\pgfsys@transformshift{10.213956in}{6.273412in}%
\pgfsys@useobject{currentmarker}{}%
\end{pgfscope}%
\begin{pgfscope}%
\pgfsys@transformshift{10.218338in}{6.273627in}%
\pgfsys@useobject{currentmarker}{}%
\end{pgfscope}%
\begin{pgfscope}%
\pgfsys@transformshift{10.222727in}{6.273692in}%
\pgfsys@useobject{currentmarker}{}%
\end{pgfscope}%
\begin{pgfscope}%
\pgfsys@transformshift{10.227124in}{6.274104in}%
\pgfsys@useobject{currentmarker}{}%
\end{pgfscope}%
\begin{pgfscope}%
\pgfsys@transformshift{10.231528in}{6.274134in}%
\pgfsys@useobject{currentmarker}{}%
\end{pgfscope}%
\begin{pgfscope}%
\pgfsys@transformshift{10.235940in}{6.274453in}%
\pgfsys@useobject{currentmarker}{}%
\end{pgfscope}%
\begin{pgfscope}%
\pgfsys@transformshift{10.240359in}{6.274718in}%
\pgfsys@useobject{currentmarker}{}%
\end{pgfscope}%
\begin{pgfscope}%
\pgfsys@transformshift{10.244787in}{6.275968in}%
\pgfsys@useobject{currentmarker}{}%
\end{pgfscope}%
\begin{pgfscope}%
\pgfsys@transformshift{10.249221in}{6.275997in}%
\pgfsys@useobject{currentmarker}{}%
\end{pgfscope}%
\begin{pgfscope}%
\pgfsys@transformshift{10.253664in}{6.276108in}%
\pgfsys@useobject{currentmarker}{}%
\end{pgfscope}%
\begin{pgfscope}%
\pgfsys@transformshift{10.258114in}{6.276223in}%
\pgfsys@useobject{currentmarker}{}%
\end{pgfscope}%
\begin{pgfscope}%
\pgfsys@transformshift{10.262573in}{6.276673in}%
\pgfsys@useobject{currentmarker}{}%
\end{pgfscope}%
\begin{pgfscope}%
\pgfsys@transformshift{10.267039in}{6.276882in}%
\pgfsys@useobject{currentmarker}{}%
\end{pgfscope}%
\begin{pgfscope}%
\pgfsys@transformshift{10.271513in}{6.276885in}%
\pgfsys@useobject{currentmarker}{}%
\end{pgfscope}%
\begin{pgfscope}%
\pgfsys@transformshift{10.275996in}{6.277513in}%
\pgfsys@useobject{currentmarker}{}%
\end{pgfscope}%
\begin{pgfscope}%
\pgfsys@transformshift{10.280486in}{6.277922in}%
\pgfsys@useobject{currentmarker}{}%
\end{pgfscope}%
\begin{pgfscope}%
\pgfsys@transformshift{10.284984in}{6.278690in}%
\pgfsys@useobject{currentmarker}{}%
\end{pgfscope}%
\begin{pgfscope}%
\pgfsys@transformshift{10.289491in}{6.279568in}%
\pgfsys@useobject{currentmarker}{}%
\end{pgfscope}%
\begin{pgfscope}%
\pgfsys@transformshift{10.294006in}{6.279708in}%
\pgfsys@useobject{currentmarker}{}%
\end{pgfscope}%
\begin{pgfscope}%
\pgfsys@transformshift{10.298529in}{6.279713in}%
\pgfsys@useobject{currentmarker}{}%
\end{pgfscope}%
\begin{pgfscope}%
\pgfsys@transformshift{10.303061in}{6.280119in}%
\pgfsys@useobject{currentmarker}{}%
\end{pgfscope}%
\begin{pgfscope}%
\pgfsys@transformshift{10.307601in}{6.280283in}%
\pgfsys@useobject{currentmarker}{}%
\end{pgfscope}%
\begin{pgfscope}%
\pgfsys@transformshift{10.312149in}{6.280786in}%
\pgfsys@useobject{currentmarker}{}%
\end{pgfscope}%
\begin{pgfscope}%
\pgfsys@transformshift{10.316706in}{6.281503in}%
\pgfsys@useobject{currentmarker}{}%
\end{pgfscope}%
\begin{pgfscope}%
\pgfsys@transformshift{10.321272in}{6.281733in}%
\pgfsys@useobject{currentmarker}{}%
\end{pgfscope}%
\begin{pgfscope}%
\pgfsys@transformshift{10.325846in}{6.282180in}%
\pgfsys@useobject{currentmarker}{}%
\end{pgfscope}%
\begin{pgfscope}%
\pgfsys@transformshift{10.330429in}{6.282589in}%
\pgfsys@useobject{currentmarker}{}%
\end{pgfscope}%
\begin{pgfscope}%
\pgfsys@transformshift{10.335020in}{6.282693in}%
\pgfsys@useobject{currentmarker}{}%
\end{pgfscope}%
\begin{pgfscope}%
\pgfsys@transformshift{10.339621in}{6.282917in}%
\pgfsys@useobject{currentmarker}{}%
\end{pgfscope}%
\begin{pgfscope}%
\pgfsys@transformshift{10.344230in}{6.283090in}%
\pgfsys@useobject{currentmarker}{}%
\end{pgfscope}%
\begin{pgfscope}%
\pgfsys@transformshift{10.348848in}{6.283291in}%
\pgfsys@useobject{currentmarker}{}%
\end{pgfscope}%
\begin{pgfscope}%
\pgfsys@transformshift{10.353475in}{6.283937in}%
\pgfsys@useobject{currentmarker}{}%
\end{pgfscope}%
\begin{pgfscope}%
\pgfsys@transformshift{10.358112in}{6.284382in}%
\pgfsys@useobject{currentmarker}{}%
\end{pgfscope}%
\begin{pgfscope}%
\pgfsys@transformshift{10.362757in}{6.284449in}%
\pgfsys@useobject{currentmarker}{}%
\end{pgfscope}%
\begin{pgfscope}%
\pgfsys@transformshift{10.367411in}{6.285145in}%
\pgfsys@useobject{currentmarker}{}%
\end{pgfscope}%
\begin{pgfscope}%
\pgfsys@transformshift{10.372075in}{6.285338in}%
\pgfsys@useobject{currentmarker}{}%
\end{pgfscope}%
\begin{pgfscope}%
\pgfsys@transformshift{10.376748in}{6.285399in}%
\pgfsys@useobject{currentmarker}{}%
\end{pgfscope}%
\begin{pgfscope}%
\pgfsys@transformshift{10.381430in}{6.285969in}%
\pgfsys@useobject{currentmarker}{}%
\end{pgfscope}%
\begin{pgfscope}%
\pgfsys@transformshift{10.386122in}{6.286213in}%
\pgfsys@useobject{currentmarker}{}%
\end{pgfscope}%
\begin{pgfscope}%
\pgfsys@transformshift{10.390824in}{6.286263in}%
\pgfsys@useobject{currentmarker}{}%
\end{pgfscope}%
\begin{pgfscope}%
\pgfsys@transformshift{10.395534in}{6.286483in}%
\pgfsys@useobject{currentmarker}{}%
\end{pgfscope}%
\begin{pgfscope}%
\pgfsys@transformshift{10.400255in}{6.286771in}%
\pgfsys@useobject{currentmarker}{}%
\end{pgfscope}%
\begin{pgfscope}%
\pgfsys@transformshift{10.404985in}{6.286866in}%
\pgfsys@useobject{currentmarker}{}%
\end{pgfscope}%
\begin{pgfscope}%
\pgfsys@transformshift{10.409725in}{6.287126in}%
\pgfsys@useobject{currentmarker}{}%
\end{pgfscope}%
\begin{pgfscope}%
\pgfsys@transformshift{10.414474in}{6.287284in}%
\pgfsys@useobject{currentmarker}{}%
\end{pgfscope}%
\begin{pgfscope}%
\pgfsys@transformshift{10.419234in}{6.287689in}%
\pgfsys@useobject{currentmarker}{}%
\end{pgfscope}%
\begin{pgfscope}%
\pgfsys@transformshift{10.424003in}{6.288854in}%
\pgfsys@useobject{currentmarker}{}%
\end{pgfscope}%
\begin{pgfscope}%
\pgfsys@transformshift{10.428783in}{6.289410in}%
\pgfsys@useobject{currentmarker}{}%
\end{pgfscope}%
\begin{pgfscope}%
\pgfsys@transformshift{10.433573in}{6.290124in}%
\pgfsys@useobject{currentmarker}{}%
\end{pgfscope}%
\begin{pgfscope}%
\pgfsys@transformshift{10.438372in}{6.290237in}%
\pgfsys@useobject{currentmarker}{}%
\end{pgfscope}%
\begin{pgfscope}%
\pgfsys@transformshift{10.443182in}{6.290912in}%
\pgfsys@useobject{currentmarker}{}%
\end{pgfscope}%
\begin{pgfscope}%
\pgfsys@transformshift{10.448003in}{6.291284in}%
\pgfsys@useobject{currentmarker}{}%
\end{pgfscope}%
\begin{pgfscope}%
\pgfsys@transformshift{10.452833in}{6.291600in}%
\pgfsys@useobject{currentmarker}{}%
\end{pgfscope}%
\begin{pgfscope}%
\pgfsys@transformshift{10.457675in}{6.291724in}%
\pgfsys@useobject{currentmarker}{}%
\end{pgfscope}%
\begin{pgfscope}%
\pgfsys@transformshift{10.462526in}{6.291834in}%
\pgfsys@useobject{currentmarker}{}%
\end{pgfscope}%
\begin{pgfscope}%
\pgfsys@transformshift{10.467389in}{6.292380in}%
\pgfsys@useobject{currentmarker}{}%
\end{pgfscope}%
\begin{pgfscope}%
\pgfsys@transformshift{10.472262in}{6.292487in}%
\pgfsys@useobject{currentmarker}{}%
\end{pgfscope}%
\begin{pgfscope}%
\pgfsys@transformshift{10.477145in}{6.292502in}%
\pgfsys@useobject{currentmarker}{}%
\end{pgfscope}%
\begin{pgfscope}%
\pgfsys@transformshift{10.482040in}{6.293241in}%
\pgfsys@useobject{currentmarker}{}%
\end{pgfscope}%
\begin{pgfscope}%
\pgfsys@transformshift{10.486945in}{6.293283in}%
\pgfsys@useobject{currentmarker}{}%
\end{pgfscope}%
\begin{pgfscope}%
\pgfsys@transformshift{10.491862in}{6.293324in}%
\pgfsys@useobject{currentmarker}{}%
\end{pgfscope}%
\begin{pgfscope}%
\pgfsys@transformshift{10.496789in}{6.293361in}%
\pgfsys@useobject{currentmarker}{}%
\end{pgfscope}%
\begin{pgfscope}%
\pgfsys@transformshift{10.501728in}{6.295382in}%
\pgfsys@useobject{currentmarker}{}%
\end{pgfscope}%
\begin{pgfscope}%
\pgfsys@transformshift{10.506678in}{6.295722in}%
\pgfsys@useobject{currentmarker}{}%
\end{pgfscope}%
\begin{pgfscope}%
\pgfsys@transformshift{10.511639in}{6.295918in}%
\pgfsys@useobject{currentmarker}{}%
\end{pgfscope}%
\begin{pgfscope}%
\pgfsys@transformshift{10.516612in}{6.296552in}%
\pgfsys@useobject{currentmarker}{}%
\end{pgfscope}%
\begin{pgfscope}%
\pgfsys@transformshift{10.521596in}{6.296738in}%
\pgfsys@useobject{currentmarker}{}%
\end{pgfscope}%
\begin{pgfscope}%
\pgfsys@transformshift{10.526592in}{6.297251in}%
\pgfsys@useobject{currentmarker}{}%
\end{pgfscope}%
\begin{pgfscope}%
\pgfsys@transformshift{10.531599in}{6.297257in}%
\pgfsys@useobject{currentmarker}{}%
\end{pgfscope}%
\begin{pgfscope}%
\pgfsys@transformshift{10.536619in}{6.297269in}%
\pgfsys@useobject{currentmarker}{}%
\end{pgfscope}%
\begin{pgfscope}%
\pgfsys@transformshift{10.541650in}{6.298357in}%
\pgfsys@useobject{currentmarker}{}%
\end{pgfscope}%
\begin{pgfscope}%
\pgfsys@transformshift{10.546692in}{6.299486in}%
\pgfsys@useobject{currentmarker}{}%
\end{pgfscope}%
\begin{pgfscope}%
\pgfsys@transformshift{10.551747in}{6.299887in}%
\pgfsys@useobject{currentmarker}{}%
\end{pgfscope}%
\begin{pgfscope}%
\pgfsys@transformshift{10.556814in}{6.300283in}%
\pgfsys@useobject{currentmarker}{}%
\end{pgfscope}%
\begin{pgfscope}%
\pgfsys@transformshift{10.561894in}{6.300726in}%
\pgfsys@useobject{currentmarker}{}%
\end{pgfscope}%
\begin{pgfscope}%
\pgfsys@transformshift{10.566985in}{6.300909in}%
\pgfsys@useobject{currentmarker}{}%
\end{pgfscope}%
\begin{pgfscope}%
\pgfsys@transformshift{10.572089in}{6.301142in}%
\pgfsys@useobject{currentmarker}{}%
\end{pgfscope}%
\begin{pgfscope}%
\pgfsys@transformshift{10.577205in}{6.302772in}%
\pgfsys@useobject{currentmarker}{}%
\end{pgfscope}%
\begin{pgfscope}%
\pgfsys@transformshift{10.582334in}{6.303013in}%
\pgfsys@useobject{currentmarker}{}%
\end{pgfscope}%
\begin{pgfscope}%
\pgfsys@transformshift{10.587476in}{6.303745in}%
\pgfsys@useobject{currentmarker}{}%
\end{pgfscope}%
\begin{pgfscope}%
\pgfsys@transformshift{10.592630in}{6.304812in}%
\pgfsys@useobject{currentmarker}{}%
\end{pgfscope}%
\begin{pgfscope}%
\pgfsys@transformshift{10.597797in}{6.304901in}%
\pgfsys@useobject{currentmarker}{}%
\end{pgfscope}%
\begin{pgfscope}%
\pgfsys@transformshift{10.602977in}{6.306245in}%
\pgfsys@useobject{currentmarker}{}%
\end{pgfscope}%
\begin{pgfscope}%
\pgfsys@transformshift{10.608171in}{6.306643in}%
\pgfsys@useobject{currentmarker}{}%
\end{pgfscope}%
\begin{pgfscope}%
\pgfsys@transformshift{10.613377in}{6.307017in}%
\pgfsys@useobject{currentmarker}{}%
\end{pgfscope}%
\begin{pgfscope}%
\pgfsys@transformshift{10.618597in}{6.307771in}%
\pgfsys@useobject{currentmarker}{}%
\end{pgfscope}%
\begin{pgfscope}%
\pgfsys@transformshift{10.623830in}{6.307989in}%
\pgfsys@useobject{currentmarker}{}%
\end{pgfscope}%
\begin{pgfscope}%
\pgfsys@transformshift{10.629076in}{6.309820in}%
\pgfsys@useobject{currentmarker}{}%
\end{pgfscope}%
\begin{pgfscope}%
\pgfsys@transformshift{10.634336in}{6.312049in}%
\pgfsys@useobject{currentmarker}{}%
\end{pgfscope}%
\begin{pgfscope}%
\pgfsys@transformshift{10.639610in}{6.312050in}%
\pgfsys@useobject{currentmarker}{}%
\end{pgfscope}%
\begin{pgfscope}%
\pgfsys@transformshift{10.644898in}{6.312052in}%
\pgfsys@useobject{currentmarker}{}%
\end{pgfscope}%
\begin{pgfscope}%
\pgfsys@transformshift{10.650199in}{6.312505in}%
\pgfsys@useobject{currentmarker}{}%
\end{pgfscope}%
\begin{pgfscope}%
\pgfsys@transformshift{10.655515in}{6.312733in}%
\pgfsys@useobject{currentmarker}{}%
\end{pgfscope}%
\begin{pgfscope}%
\pgfsys@transformshift{10.660845in}{6.312986in}%
\pgfsys@useobject{currentmarker}{}%
\end{pgfscope}%
\begin{pgfscope}%
\pgfsys@transformshift{10.666189in}{6.313269in}%
\pgfsys@useobject{currentmarker}{}%
\end{pgfscope}%
\begin{pgfscope}%
\pgfsys@transformshift{10.671547in}{6.313325in}%
\pgfsys@useobject{currentmarker}{}%
\end{pgfscope}%
\begin{pgfscope}%
\pgfsys@transformshift{10.676920in}{6.313621in}%
\pgfsys@useobject{currentmarker}{}%
\end{pgfscope}%
\begin{pgfscope}%
\pgfsys@transformshift{10.682307in}{6.313828in}%
\pgfsys@useobject{currentmarker}{}%
\end{pgfscope}%
\begin{pgfscope}%
\pgfsys@transformshift{10.687709in}{6.314192in}%
\pgfsys@useobject{currentmarker}{}%
\end{pgfscope}%
\begin{pgfscope}%
\pgfsys@transformshift{10.693126in}{6.314859in}%
\pgfsys@useobject{currentmarker}{}%
\end{pgfscope}%
\begin{pgfscope}%
\pgfsys@transformshift{10.698558in}{6.315547in}%
\pgfsys@useobject{currentmarker}{}%
\end{pgfscope}%
\begin{pgfscope}%
\pgfsys@transformshift{10.704005in}{6.316356in}%
\pgfsys@useobject{currentmarker}{}%
\end{pgfscope}%
\begin{pgfscope}%
\pgfsys@transformshift{10.709468in}{6.317795in}%
\pgfsys@useobject{currentmarker}{}%
\end{pgfscope}%
\begin{pgfscope}%
\pgfsys@transformshift{10.714945in}{6.318370in}%
\pgfsys@useobject{currentmarker}{}%
\end{pgfscope}%
\begin{pgfscope}%
\pgfsys@transformshift{10.720438in}{6.319028in}%
\pgfsys@useobject{currentmarker}{}%
\end{pgfscope}%
\begin{pgfscope}%
\pgfsys@transformshift{10.725947in}{6.319647in}%
\pgfsys@useobject{currentmarker}{}%
\end{pgfscope}%
\begin{pgfscope}%
\pgfsys@transformshift{10.731472in}{6.319864in}%
\pgfsys@useobject{currentmarker}{}%
\end{pgfscope}%
\begin{pgfscope}%
\pgfsys@transformshift{10.737012in}{6.320013in}%
\pgfsys@useobject{currentmarker}{}%
\end{pgfscope}%
\begin{pgfscope}%
\pgfsys@transformshift{10.742568in}{6.320090in}%
\pgfsys@useobject{currentmarker}{}%
\end{pgfscope}%
\begin{pgfscope}%
\pgfsys@transformshift{10.748141in}{6.320166in}%
\pgfsys@useobject{currentmarker}{}%
\end{pgfscope}%
\begin{pgfscope}%
\pgfsys@transformshift{10.753730in}{6.321383in}%
\pgfsys@useobject{currentmarker}{}%
\end{pgfscope}%
\begin{pgfscope}%
\pgfsys@transformshift{10.759335in}{6.321602in}%
\pgfsys@useobject{currentmarker}{}%
\end{pgfscope}%
\begin{pgfscope}%
\pgfsys@transformshift{10.764957in}{6.322216in}%
\pgfsys@useobject{currentmarker}{}%
\end{pgfscope}%
\begin{pgfscope}%
\pgfsys@transformshift{10.770595in}{6.322858in}%
\pgfsys@useobject{currentmarker}{}%
\end{pgfscope}%
\begin{pgfscope}%
\pgfsys@transformshift{10.776250in}{6.325681in}%
\pgfsys@useobject{currentmarker}{}%
\end{pgfscope}%
\begin{pgfscope}%
\pgfsys@transformshift{10.781923in}{6.325905in}%
\pgfsys@useobject{currentmarker}{}%
\end{pgfscope}%
\begin{pgfscope}%
\pgfsys@transformshift{10.787612in}{6.326413in}%
\pgfsys@useobject{currentmarker}{}%
\end{pgfscope}%
\begin{pgfscope}%
\pgfsys@transformshift{10.793319in}{6.326900in}%
\pgfsys@useobject{currentmarker}{}%
\end{pgfscope}%
\begin{pgfscope}%
\pgfsys@transformshift{10.799043in}{6.327732in}%
\pgfsys@useobject{currentmarker}{}%
\end{pgfscope}%
\begin{pgfscope}%
\pgfsys@transformshift{10.804785in}{6.328429in}%
\pgfsys@useobject{currentmarker}{}%
\end{pgfscope}%
\begin{pgfscope}%
\pgfsys@transformshift{10.810545in}{6.328563in}%
\pgfsys@useobject{currentmarker}{}%
\end{pgfscope}%
\begin{pgfscope}%
\pgfsys@transformshift{10.816323in}{6.328568in}%
\pgfsys@useobject{currentmarker}{}%
\end{pgfscope}%
\begin{pgfscope}%
\pgfsys@transformshift{10.822118in}{6.329095in}%
\pgfsys@useobject{currentmarker}{}%
\end{pgfscope}%
\begin{pgfscope}%
\pgfsys@transformshift{10.827932in}{6.329607in}%
\pgfsys@useobject{currentmarker}{}%
\end{pgfscope}%
\begin{pgfscope}%
\pgfsys@transformshift{10.833765in}{6.329837in}%
\pgfsys@useobject{currentmarker}{}%
\end{pgfscope}%
\begin{pgfscope}%
\pgfsys@transformshift{10.839616in}{6.330430in}%
\pgfsys@useobject{currentmarker}{}%
\end{pgfscope}%
\begin{pgfscope}%
\pgfsys@transformshift{10.845486in}{6.331153in}%
\pgfsys@useobject{currentmarker}{}%
\end{pgfscope}%
\begin{pgfscope}%
\pgfsys@transformshift{10.851374in}{6.331664in}%
\pgfsys@useobject{currentmarker}{}%
\end{pgfscope}%
\begin{pgfscope}%
\pgfsys@transformshift{10.857282in}{6.331947in}%
\pgfsys@useobject{currentmarker}{}%
\end{pgfscope}%
\begin{pgfscope}%
\pgfsys@transformshift{10.863209in}{6.333757in}%
\pgfsys@useobject{currentmarker}{}%
\end{pgfscope}%
\begin{pgfscope}%
\pgfsys@transformshift{10.869156in}{6.334291in}%
\pgfsys@useobject{currentmarker}{}%
\end{pgfscope}%
\begin{pgfscope}%
\pgfsys@transformshift{10.875122in}{6.335790in}%
\pgfsys@useobject{currentmarker}{}%
\end{pgfscope}%
\begin{pgfscope}%
\pgfsys@transformshift{10.881108in}{6.335973in}%
\pgfsys@useobject{currentmarker}{}%
\end{pgfscope}%
\begin{pgfscope}%
\pgfsys@transformshift{10.887114in}{6.339897in}%
\pgfsys@useobject{currentmarker}{}%
\end{pgfscope}%
\begin{pgfscope}%
\pgfsys@transformshift{10.893141in}{6.339919in}%
\pgfsys@useobject{currentmarker}{}%
\end{pgfscope}%
\begin{pgfscope}%
\pgfsys@transformshift{10.899187in}{6.339979in}%
\pgfsys@useobject{currentmarker}{}%
\end{pgfscope}%
\begin{pgfscope}%
\pgfsys@transformshift{10.905255in}{6.340208in}%
\pgfsys@useobject{currentmarker}{}%
\end{pgfscope}%
\begin{pgfscope}%
\pgfsys@transformshift{10.911343in}{6.342306in}%
\pgfsys@useobject{currentmarker}{}%
\end{pgfscope}%
\begin{pgfscope}%
\pgfsys@transformshift{10.917452in}{6.343070in}%
\pgfsys@useobject{currentmarker}{}%
\end{pgfscope}%
\begin{pgfscope}%
\pgfsys@transformshift{10.923582in}{6.343348in}%
\pgfsys@useobject{currentmarker}{}%
\end{pgfscope}%
\begin{pgfscope}%
\pgfsys@transformshift{10.929734in}{6.343426in}%
\pgfsys@useobject{currentmarker}{}%
\end{pgfscope}%
\begin{pgfscope}%
\pgfsys@transformshift{10.935908in}{6.344486in}%
\pgfsys@useobject{currentmarker}{}%
\end{pgfscope}%
\begin{pgfscope}%
\pgfsys@transformshift{10.942103in}{6.344511in}%
\pgfsys@useobject{currentmarker}{}%
\end{pgfscope}%
\begin{pgfscope}%
\pgfsys@transformshift{10.948320in}{6.344597in}%
\pgfsys@useobject{currentmarker}{}%
\end{pgfscope}%
\begin{pgfscope}%
\pgfsys@transformshift{10.954560in}{6.346130in}%
\pgfsys@useobject{currentmarker}{}%
\end{pgfscope}%
\begin{pgfscope}%
\pgfsys@transformshift{10.960822in}{6.346551in}%
\pgfsys@useobject{currentmarker}{}%
\end{pgfscope}%
\begin{pgfscope}%
\pgfsys@transformshift{10.967107in}{6.347244in}%
\pgfsys@useobject{currentmarker}{}%
\end{pgfscope}%
\begin{pgfscope}%
\pgfsys@transformshift{10.973415in}{6.347980in}%
\pgfsys@useobject{currentmarker}{}%
\end{pgfscope}%
\begin{pgfscope}%
\pgfsys@transformshift{10.979746in}{6.348208in}%
\pgfsys@useobject{currentmarker}{}%
\end{pgfscope}%
\begin{pgfscope}%
\pgfsys@transformshift{10.986100in}{6.348741in}%
\pgfsys@useobject{currentmarker}{}%
\end{pgfscope}%
\begin{pgfscope}%
\pgfsys@transformshift{10.992478in}{6.348797in}%
\pgfsys@useobject{currentmarker}{}%
\end{pgfscope}%
\begin{pgfscope}%
\pgfsys@transformshift{10.998881in}{6.349930in}%
\pgfsys@useobject{currentmarker}{}%
\end{pgfscope}%
\begin{pgfscope}%
\pgfsys@transformshift{11.005307in}{6.350078in}%
\pgfsys@useobject{currentmarker}{}%
\end{pgfscope}%
\begin{pgfscope}%
\pgfsys@transformshift{11.011757in}{6.350737in}%
\pgfsys@useobject{currentmarker}{}%
\end{pgfscope}%
\begin{pgfscope}%
\pgfsys@transformshift{11.018233in}{6.350895in}%
\pgfsys@useobject{currentmarker}{}%
\end{pgfscope}%
\begin{pgfscope}%
\pgfsys@transformshift{11.024733in}{6.350966in}%
\pgfsys@useobject{currentmarker}{}%
\end{pgfscope}%
\begin{pgfscope}%
\pgfsys@transformshift{11.031258in}{6.352004in}%
\pgfsys@useobject{currentmarker}{}%
\end{pgfscope}%
\begin{pgfscope}%
\pgfsys@transformshift{11.037809in}{6.352823in}%
\pgfsys@useobject{currentmarker}{}%
\end{pgfscope}%
\begin{pgfscope}%
\pgfsys@transformshift{11.044386in}{6.353463in}%
\pgfsys@useobject{currentmarker}{}%
\end{pgfscope}%
\begin{pgfscope}%
\pgfsys@transformshift{11.050988in}{6.353674in}%
\pgfsys@useobject{currentmarker}{}%
\end{pgfscope}%
\begin{pgfscope}%
\pgfsys@transformshift{11.057617in}{6.353919in}%
\pgfsys@useobject{currentmarker}{}%
\end{pgfscope}%
\begin{pgfscope}%
\pgfsys@transformshift{11.064273in}{6.354270in}%
\pgfsys@useobject{currentmarker}{}%
\end{pgfscope}%
\begin{pgfscope}%
\pgfsys@transformshift{11.070955in}{6.354571in}%
\pgfsys@useobject{currentmarker}{}%
\end{pgfscope}%
\begin{pgfscope}%
\pgfsys@transformshift{11.077665in}{6.354991in}%
\pgfsys@useobject{currentmarker}{}%
\end{pgfscope}%
\begin{pgfscope}%
\pgfsys@transformshift{11.084402in}{6.355028in}%
\pgfsys@useobject{currentmarker}{}%
\end{pgfscope}%
\begin{pgfscope}%
\pgfsys@transformshift{11.091167in}{6.355296in}%
\pgfsys@useobject{currentmarker}{}%
\end{pgfscope}%
\begin{pgfscope}%
\pgfsys@transformshift{11.097960in}{6.356161in}%
\pgfsys@useobject{currentmarker}{}%
\end{pgfscope}%
\begin{pgfscope}%
\pgfsys@transformshift{11.104781in}{6.357261in}%
\pgfsys@useobject{currentmarker}{}%
\end{pgfscope}%
\begin{pgfscope}%
\pgfsys@transformshift{11.111631in}{6.358261in}%
\pgfsys@useobject{currentmarker}{}%
\end{pgfscope}%
\begin{pgfscope}%
\pgfsys@transformshift{11.118510in}{6.358827in}%
\pgfsys@useobject{currentmarker}{}%
\end{pgfscope}%
\begin{pgfscope}%
\pgfsys@transformshift{11.125418in}{6.358874in}%
\pgfsys@useobject{currentmarker}{}%
\end{pgfscope}%
\begin{pgfscope}%
\pgfsys@transformshift{11.132356in}{6.359353in}%
\pgfsys@useobject{currentmarker}{}%
\end{pgfscope}%
\begin{pgfscope}%
\pgfsys@transformshift{11.139324in}{6.360084in}%
\pgfsys@useobject{currentmarker}{}%
\end{pgfscope}%
\begin{pgfscope}%
\pgfsys@transformshift{11.146323in}{6.360360in}%
\pgfsys@useobject{currentmarker}{}%
\end{pgfscope}%
\begin{pgfscope}%
\pgfsys@transformshift{11.153352in}{6.360631in}%
\pgfsys@useobject{currentmarker}{}%
\end{pgfscope}%
\begin{pgfscope}%
\pgfsys@transformshift{11.160413in}{6.361774in}%
\pgfsys@useobject{currentmarker}{}%
\end{pgfscope}%
\begin{pgfscope}%
\pgfsys@transformshift{11.167505in}{6.362179in}%
\pgfsys@useobject{currentmarker}{}%
\end{pgfscope}%
\begin{pgfscope}%
\pgfsys@transformshift{11.174628in}{6.362549in}%
\pgfsys@useobject{currentmarker}{}%
\end{pgfscope}%
\begin{pgfscope}%
\pgfsys@transformshift{11.181784in}{6.363472in}%
\pgfsys@useobject{currentmarker}{}%
\end{pgfscope}%
\begin{pgfscope}%
\pgfsys@transformshift{11.188973in}{6.364100in}%
\pgfsys@useobject{currentmarker}{}%
\end{pgfscope}%
\begin{pgfscope}%
\pgfsys@transformshift{11.196195in}{6.364186in}%
\pgfsys@useobject{currentmarker}{}%
\end{pgfscope}%
\begin{pgfscope}%
\pgfsys@transformshift{11.203450in}{6.364895in}%
\pgfsys@useobject{currentmarker}{}%
\end{pgfscope}%
\begin{pgfscope}%
\pgfsys@transformshift{11.210739in}{6.366287in}%
\pgfsys@useobject{currentmarker}{}%
\end{pgfscope}%
\begin{pgfscope}%
\pgfsys@transformshift{11.218062in}{6.366436in}%
\pgfsys@useobject{currentmarker}{}%
\end{pgfscope}%
\begin{pgfscope}%
\pgfsys@transformshift{11.225420in}{6.367177in}%
\pgfsys@useobject{currentmarker}{}%
\end{pgfscope}%
\begin{pgfscope}%
\pgfsys@transformshift{11.232813in}{6.368527in}%
\pgfsys@useobject{currentmarker}{}%
\end{pgfscope}%
\begin{pgfscope}%
\pgfsys@transformshift{11.240242in}{6.368718in}%
\pgfsys@useobject{currentmarker}{}%
\end{pgfscope}%
\begin{pgfscope}%
\pgfsys@transformshift{11.247706in}{6.368994in}%
\pgfsys@useobject{currentmarker}{}%
\end{pgfscope}%
\begin{pgfscope}%
\pgfsys@transformshift{11.255207in}{6.370416in}%
\pgfsys@useobject{currentmarker}{}%
\end{pgfscope}%
\begin{pgfscope}%
\pgfsys@transformshift{11.262745in}{6.370510in}%
\pgfsys@useobject{currentmarker}{}%
\end{pgfscope}%
\begin{pgfscope}%
\pgfsys@transformshift{11.270321in}{6.370633in}%
\pgfsys@useobject{currentmarker}{}%
\end{pgfscope}%
\begin{pgfscope}%
\pgfsys@transformshift{11.277934in}{6.371796in}%
\pgfsys@useobject{currentmarker}{}%
\end{pgfscope}%
\begin{pgfscope}%
\pgfsys@transformshift{11.285586in}{6.371860in}%
\pgfsys@useobject{currentmarker}{}%
\end{pgfscope}%
\begin{pgfscope}%
\pgfsys@transformshift{11.293276in}{6.372133in}%
\pgfsys@useobject{currentmarker}{}%
\end{pgfscope}%
\begin{pgfscope}%
\pgfsys@transformshift{11.301006in}{6.372876in}%
\pgfsys@useobject{currentmarker}{}%
\end{pgfscope}%
\begin{pgfscope}%
\pgfsys@transformshift{11.308776in}{6.373292in}%
\pgfsys@useobject{currentmarker}{}%
\end{pgfscope}%
\begin{pgfscope}%
\pgfsys@transformshift{11.316586in}{6.373959in}%
\pgfsys@useobject{currentmarker}{}%
\end{pgfscope}%
\begin{pgfscope}%
\pgfsys@transformshift{11.324438in}{6.374122in}%
\pgfsys@useobject{currentmarker}{}%
\end{pgfscope}%
\begin{pgfscope}%
\pgfsys@transformshift{11.332331in}{6.374280in}%
\pgfsys@useobject{currentmarker}{}%
\end{pgfscope}%
\begin{pgfscope}%
\pgfsys@transformshift{11.340266in}{6.374732in}%
\pgfsys@useobject{currentmarker}{}%
\end{pgfscope}%
\begin{pgfscope}%
\pgfsys@transformshift{11.348244in}{6.374888in}%
\pgfsys@useobject{currentmarker}{}%
\end{pgfscope}%
\begin{pgfscope}%
\pgfsys@transformshift{11.356265in}{6.377093in}%
\pgfsys@useobject{currentmarker}{}%
\end{pgfscope}%
\begin{pgfscope}%
\pgfsys@transformshift{11.364330in}{6.377401in}%
\pgfsys@useobject{currentmarker}{}%
\end{pgfscope}%
\begin{pgfscope}%
\pgfsys@transformshift{11.372440in}{6.378538in}%
\pgfsys@useobject{currentmarker}{}%
\end{pgfscope}%
\begin{pgfscope}%
\pgfsys@transformshift{11.380595in}{6.380583in}%
\pgfsys@useobject{currentmarker}{}%
\end{pgfscope}%
\begin{pgfscope}%
\pgfsys@transformshift{11.388796in}{6.380588in}%
\pgfsys@useobject{currentmarker}{}%
\end{pgfscope}%
\begin{pgfscope}%
\pgfsys@transformshift{11.397043in}{6.380880in}%
\pgfsys@useobject{currentmarker}{}%
\end{pgfscope}%
\begin{pgfscope}%
\pgfsys@transformshift{11.405337in}{6.380900in}%
\pgfsys@useobject{currentmarker}{}%
\end{pgfscope}%
\begin{pgfscope}%
\pgfsys@transformshift{11.413680in}{6.381118in}%
\pgfsys@useobject{currentmarker}{}%
\end{pgfscope}%
\begin{pgfscope}%
\pgfsys@transformshift{11.422071in}{6.382029in}%
\pgfsys@useobject{currentmarker}{}%
\end{pgfscope}%
\begin{pgfscope}%
\pgfsys@transformshift{11.430511in}{6.382209in}%
\pgfsys@useobject{currentmarker}{}%
\end{pgfscope}%
\begin{pgfscope}%
\pgfsys@transformshift{11.439001in}{6.382320in}%
\pgfsys@useobject{currentmarker}{}%
\end{pgfscope}%
\begin{pgfscope}%
\pgfsys@transformshift{11.447542in}{6.382853in}%
\pgfsys@useobject{currentmarker}{}%
\end{pgfscope}%
\begin{pgfscope}%
\pgfsys@transformshift{11.456135in}{6.382964in}%
\pgfsys@useobject{currentmarker}{}%
\end{pgfscope}%
\begin{pgfscope}%
\pgfsys@transformshift{11.464779in}{6.383009in}%
\pgfsys@useobject{currentmarker}{}%
\end{pgfscope}%
\begin{pgfscope}%
\pgfsys@transformshift{11.473478in}{6.383701in}%
\pgfsys@useobject{currentmarker}{}%
\end{pgfscope}%
\begin{pgfscope}%
\pgfsys@transformshift{11.482230in}{6.384075in}%
\pgfsys@useobject{currentmarker}{}%
\end{pgfscope}%
\begin{pgfscope}%
\pgfsys@transformshift{11.491036in}{6.388360in}%
\pgfsys@useobject{currentmarker}{}%
\end{pgfscope}%
\begin{pgfscope}%
\pgfsys@transformshift{11.499899in}{6.391243in}%
\pgfsys@useobject{currentmarker}{}%
\end{pgfscope}%
\begin{pgfscope}%
\pgfsys@transformshift{11.508818in}{6.392144in}%
\pgfsys@useobject{currentmarker}{}%
\end{pgfscope}%
\begin{pgfscope}%
\pgfsys@transformshift{11.517795in}{6.392979in}%
\pgfsys@useobject{currentmarker}{}%
\end{pgfscope}%
\begin{pgfscope}%
\pgfsys@transformshift{11.526830in}{6.394508in}%
\pgfsys@useobject{currentmarker}{}%
\end{pgfscope}%
\begin{pgfscope}%
\pgfsys@transformshift{11.535925in}{6.395340in}%
\pgfsys@useobject{currentmarker}{}%
\end{pgfscope}%
\begin{pgfscope}%
\pgfsys@transformshift{11.545080in}{6.395500in}%
\pgfsys@useobject{currentmarker}{}%
\end{pgfscope}%
\begin{pgfscope}%
\pgfsys@transformshift{11.554296in}{6.395997in}%
\pgfsys@useobject{currentmarker}{}%
\end{pgfscope}%
\begin{pgfscope}%
\pgfsys@transformshift{11.563575in}{6.396784in}%
\pgfsys@useobject{currentmarker}{}%
\end{pgfscope}%
\begin{pgfscope}%
\pgfsys@transformshift{11.572917in}{6.398043in}%
\pgfsys@useobject{currentmarker}{}%
\end{pgfscope}%
\begin{pgfscope}%
\pgfsys@transformshift{11.582324in}{6.398091in}%
\pgfsys@useobject{currentmarker}{}%
\end{pgfscope}%
\begin{pgfscope}%
\pgfsys@transformshift{11.591797in}{6.398187in}%
\pgfsys@useobject{currentmarker}{}%
\end{pgfscope}%
\begin{pgfscope}%
\pgfsys@transformshift{11.601336in}{6.398388in}%
\pgfsys@useobject{currentmarker}{}%
\end{pgfscope}%
\begin{pgfscope}%
\pgfsys@transformshift{11.610944in}{6.398415in}%
\pgfsys@useobject{currentmarker}{}%
\end{pgfscope}%
\begin{pgfscope}%
\pgfsys@transformshift{11.620620in}{6.398823in}%
\pgfsys@useobject{currentmarker}{}%
\end{pgfscope}%
\begin{pgfscope}%
\pgfsys@transformshift{11.630368in}{6.399108in}%
\pgfsys@useobject{currentmarker}{}%
\end{pgfscope}%
\begin{pgfscope}%
\pgfsys@transformshift{11.640186in}{6.400664in}%
\pgfsys@useobject{currentmarker}{}%
\end{pgfscope}%
\begin{pgfscope}%
\pgfsys@transformshift{11.650078in}{6.402618in}%
\pgfsys@useobject{currentmarker}{}%
\end{pgfscope}%
\begin{pgfscope}%
\pgfsys@transformshift{11.660045in}{6.403641in}%
\pgfsys@useobject{currentmarker}{}%
\end{pgfscope}%
\begin{pgfscope}%
\pgfsys@transformshift{11.670087in}{6.404047in}%
\pgfsys@useobject{currentmarker}{}%
\end{pgfscope}%
\begin{pgfscope}%
\pgfsys@transformshift{11.680206in}{6.404308in}%
\pgfsys@useobject{currentmarker}{}%
\end{pgfscope}%
\begin{pgfscope}%
\pgfsys@transformshift{11.690404in}{6.404316in}%
\pgfsys@useobject{currentmarker}{}%
\end{pgfscope}%
\begin{pgfscope}%
\pgfsys@transformshift{11.700682in}{6.405166in}%
\pgfsys@useobject{currentmarker}{}%
\end{pgfscope}%
\begin{pgfscope}%
\pgfsys@transformshift{11.711042in}{6.405263in}%
\pgfsys@useobject{currentmarker}{}%
\end{pgfscope}%
\begin{pgfscope}%
\pgfsys@transformshift{11.721486in}{6.406545in}%
\pgfsys@useobject{currentmarker}{}%
\end{pgfscope}%
\begin{pgfscope}%
\pgfsys@transformshift{11.732014in}{6.410009in}%
\pgfsys@useobject{currentmarker}{}%
\end{pgfscope}%
\begin{pgfscope}%
\pgfsys@transformshift{11.742629in}{6.410641in}%
\pgfsys@useobject{currentmarker}{}%
\end{pgfscope}%
\begin{pgfscope}%
\pgfsys@transformshift{11.753332in}{6.411031in}%
\pgfsys@useobject{currentmarker}{}%
\end{pgfscope}%
\begin{pgfscope}%
\pgfsys@transformshift{11.764126in}{6.411167in}%
\pgfsys@useobject{currentmarker}{}%
\end{pgfscope}%
\begin{pgfscope}%
\pgfsys@transformshift{11.775011in}{6.411813in}%
\pgfsys@useobject{currentmarker}{}%
\end{pgfscope}%
\begin{pgfscope}%
\pgfsys@transformshift{11.785991in}{6.411840in}%
\pgfsys@useobject{currentmarker}{}%
\end{pgfscope}%
\begin{pgfscope}%
\pgfsys@transformshift{11.797067in}{6.412030in}%
\pgfsys@useobject{currentmarker}{}%
\end{pgfscope}%
\begin{pgfscope}%
\pgfsys@transformshift{11.808241in}{6.412392in}%
\pgfsys@useobject{currentmarker}{}%
\end{pgfscope}%
\begin{pgfscope}%
\pgfsys@transformshift{11.819515in}{6.413734in}%
\pgfsys@useobject{currentmarker}{}%
\end{pgfscope}%
\begin{pgfscope}%
\pgfsys@transformshift{11.830891in}{6.420143in}%
\pgfsys@useobject{currentmarker}{}%
\end{pgfscope}%
\begin{pgfscope}%
\pgfsys@transformshift{11.842372in}{6.422657in}%
\pgfsys@useobject{currentmarker}{}%
\end{pgfscope}%
\begin{pgfscope}%
\pgfsys@transformshift{11.853960in}{6.426377in}%
\pgfsys@useobject{currentmarker}{}%
\end{pgfscope}%
\begin{pgfscope}%
\pgfsys@transformshift{11.865658in}{6.428123in}%
\pgfsys@useobject{currentmarker}{}%
\end{pgfscope}%
\begin{pgfscope}%
\pgfsys@transformshift{11.877467in}{6.430098in}%
\pgfsys@useobject{currentmarker}{}%
\end{pgfscope}%
\begin{pgfscope}%
\pgfsys@transformshift{11.889391in}{6.431802in}%
\pgfsys@useobject{currentmarker}{}%
\end{pgfscope}%
\begin{pgfscope}%
\pgfsys@transformshift{11.901432in}{6.432807in}%
\pgfsys@useobject{currentmarker}{}%
\end{pgfscope}%
\begin{pgfscope}%
\pgfsys@transformshift{11.913593in}{6.434535in}%
\pgfsys@useobject{currentmarker}{}%
\end{pgfscope}%
\begin{pgfscope}%
\pgfsys@transformshift{11.925877in}{6.437476in}%
\pgfsys@useobject{currentmarker}{}%
\end{pgfscope}%
\begin{pgfscope}%
\pgfsys@transformshift{11.938286in}{6.439055in}%
\pgfsys@useobject{currentmarker}{}%
\end{pgfscope}%
\begin{pgfscope}%
\pgfsys@transformshift{11.950825in}{6.440336in}%
\pgfsys@useobject{currentmarker}{}%
\end{pgfscope}%
\begin{pgfscope}%
\pgfsys@transformshift{11.963496in}{6.440650in}%
\pgfsys@useobject{currentmarker}{}%
\end{pgfscope}%
\begin{pgfscope}%
\pgfsys@transformshift{11.976302in}{6.443024in}%
\pgfsys@useobject{currentmarker}{}%
\end{pgfscope}%
\begin{pgfscope}%
\pgfsys@transformshift{11.989247in}{6.444702in}%
\pgfsys@useobject{currentmarker}{}%
\end{pgfscope}%
\begin{pgfscope}%
\pgfsys@transformshift{12.002335in}{6.445630in}%
\pgfsys@useobject{currentmarker}{}%
\end{pgfscope}%
\begin{pgfscope}%
\pgfsys@transformshift{12.015569in}{6.446516in}%
\pgfsys@useobject{currentmarker}{}%
\end{pgfscope}%
\begin{pgfscope}%
\pgfsys@transformshift{12.028954in}{6.448430in}%
\pgfsys@useobject{currentmarker}{}%
\end{pgfscope}%
\begin{pgfscope}%
\pgfsys@transformshift{12.042493in}{6.451160in}%
\pgfsys@useobject{currentmarker}{}%
\end{pgfscope}%
\begin{pgfscope}%
\pgfsys@transformshift{12.056190in}{6.453012in}%
\pgfsys@useobject{currentmarker}{}%
\end{pgfscope}%
\begin{pgfscope}%
\pgfsys@transformshift{12.070051in}{6.456361in}%
\pgfsys@useobject{currentmarker}{}%
\end{pgfscope}%
\begin{pgfscope}%
\pgfsys@transformshift{12.084080in}{6.459752in}%
\pgfsys@useobject{currentmarker}{}%
\end{pgfscope}%
\begin{pgfscope}%
\pgfsys@transformshift{12.098280in}{6.461381in}%
\pgfsys@useobject{currentmarker}{}%
\end{pgfscope}%
\begin{pgfscope}%
\pgfsys@transformshift{12.112659in}{6.463024in}%
\pgfsys@useobject{currentmarker}{}%
\end{pgfscope}%
\begin{pgfscope}%
\pgfsys@transformshift{12.127221in}{6.464192in}%
\pgfsys@useobject{currentmarker}{}%
\end{pgfscope}%
\begin{pgfscope}%
\pgfsys@transformshift{12.141970in}{6.465301in}%
\pgfsys@useobject{currentmarker}{}%
\end{pgfscope}%
\begin{pgfscope}%
\pgfsys@transformshift{12.156914in}{6.467086in}%
\pgfsys@useobject{currentmarker}{}%
\end{pgfscope}%
\begin{pgfscope}%
\pgfsys@transformshift{12.172059in}{6.467614in}%
\pgfsys@useobject{currentmarker}{}%
\end{pgfscope}%
\begin{pgfscope}%
\pgfsys@transformshift{12.187410in}{6.470559in}%
\pgfsys@useobject{currentmarker}{}%
\end{pgfscope}%
\begin{pgfscope}%
\pgfsys@transformshift{12.202974in}{6.470808in}%
\pgfsys@useobject{currentmarker}{}%
\end{pgfscope}%
\begin{pgfscope}%
\pgfsys@transformshift{12.218759in}{6.471044in}%
\pgfsys@useobject{currentmarker}{}%
\end{pgfscope}%
\begin{pgfscope}%
\pgfsys@transformshift{12.234771in}{6.478346in}%
\pgfsys@useobject{currentmarker}{}%
\end{pgfscope}%
\begin{pgfscope}%
\pgfsys@transformshift{12.251018in}{6.483623in}%
\pgfsys@useobject{currentmarker}{}%
\end{pgfscope}%
\begin{pgfscope}%
\pgfsys@transformshift{12.267510in}{6.483665in}%
\pgfsys@useobject{currentmarker}{}%
\end{pgfscope}%
\begin{pgfscope}%
\pgfsys@transformshift{12.284253in}{6.483861in}%
\pgfsys@useobject{currentmarker}{}%
\end{pgfscope}%
\begin{pgfscope}%
\pgfsys@transformshift{12.301258in}{6.485198in}%
\pgfsys@useobject{currentmarker}{}%
\end{pgfscope}%
\begin{pgfscope}%
\pgfsys@transformshift{12.318533in}{6.485516in}%
\pgfsys@useobject{currentmarker}{}%
\end{pgfscope}%
\begin{pgfscope}%
\pgfsys@transformshift{12.336089in}{6.487336in}%
\pgfsys@useobject{currentmarker}{}%
\end{pgfscope}%
\begin{pgfscope}%
\pgfsys@transformshift{12.353938in}{6.489734in}%
\pgfsys@useobject{currentmarker}{}%
\end{pgfscope}%
\begin{pgfscope}%
\pgfsys@transformshift{12.372089in}{6.492386in}%
\pgfsys@useobject{currentmarker}{}%
\end{pgfscope}%
\begin{pgfscope}%
\pgfsys@transformshift{12.390555in}{6.493628in}%
\pgfsys@useobject{currentmarker}{}%
\end{pgfscope}%
\begin{pgfscope}%
\pgfsys@transformshift{12.409349in}{6.497961in}%
\pgfsys@useobject{currentmarker}{}%
\end{pgfscope}%
\begin{pgfscope}%
\pgfsys@transformshift{12.428485in}{6.501811in}%
\pgfsys@useobject{currentmarker}{}%
\end{pgfscope}%
\begin{pgfscope}%
\pgfsys@transformshift{12.447977in}{6.502132in}%
\pgfsys@useobject{currentmarker}{}%
\end{pgfscope}%
\begin{pgfscope}%
\pgfsys@transformshift{12.467841in}{6.504456in}%
\pgfsys@useobject{currentmarker}{}%
\end{pgfscope}%
\begin{pgfscope}%
\pgfsys@transformshift{12.488092in}{6.504825in}%
\pgfsys@useobject{currentmarker}{}%
\end{pgfscope}%
\begin{pgfscope}%
\pgfsys@transformshift{12.508750in}{6.506411in}%
\pgfsys@useobject{currentmarker}{}%
\end{pgfscope}%
\begin{pgfscope}%
\pgfsys@transformshift{12.529832in}{6.506888in}%
\pgfsys@useobject{currentmarker}{}%
\end{pgfscope}%
\begin{pgfscope}%
\pgfsys@transformshift{12.551359in}{6.507523in}%
\pgfsys@useobject{currentmarker}{}%
\end{pgfscope}%
\begin{pgfscope}%
\pgfsys@transformshift{12.573353in}{6.507680in}%
\pgfsys@useobject{currentmarker}{}%
\end{pgfscope}%
\begin{pgfscope}%
\pgfsys@transformshift{12.595838in}{6.512219in}%
\pgfsys@useobject{currentmarker}{}%
\end{pgfscope}%
\begin{pgfscope}%
\pgfsys@transformshift{12.618838in}{6.512411in}%
\pgfsys@useobject{currentmarker}{}%
\end{pgfscope}%
\begin{pgfscope}%
\pgfsys@transformshift{12.642382in}{6.513946in}%
\pgfsys@useobject{currentmarker}{}%
\end{pgfscope}%
\begin{pgfscope}%
\pgfsys@transformshift{12.666500in}{6.519889in}%
\pgfsys@useobject{currentmarker}{}%
\end{pgfscope}%
\begin{pgfscope}%
\pgfsys@transformshift{12.691223in}{6.520548in}%
\pgfsys@useobject{currentmarker}{}%
\end{pgfscope}%
\begin{pgfscope}%
\pgfsys@transformshift{12.716587in}{6.521982in}%
\pgfsys@useobject{currentmarker}{}%
\end{pgfscope}%
\begin{pgfscope}%
\pgfsys@transformshift{12.742631in}{6.524348in}%
\pgfsys@useobject{currentmarker}{}%
\end{pgfscope}%
\begin{pgfscope}%
\pgfsys@transformshift{12.769396in}{6.534637in}%
\pgfsys@useobject{currentmarker}{}%
\end{pgfscope}%
\begin{pgfscope}%
\pgfsys@transformshift{12.796929in}{6.535116in}%
\pgfsys@useobject{currentmarker}{}%
\end{pgfscope}%
\begin{pgfscope}%
\pgfsys@transformshift{12.825282in}{6.547178in}%
\pgfsys@useobject{currentmarker}{}%
\end{pgfscope}%
\begin{pgfscope}%
\pgfsys@transformshift{12.854511in}{6.563755in}%
\pgfsys@useobject{currentmarker}{}%
\end{pgfscope}%
\begin{pgfscope}%
\pgfsys@transformshift{12.884677in}{6.569117in}%
\pgfsys@useobject{currentmarker}{}%
\end{pgfscope}%
\begin{pgfscope}%
\pgfsys@transformshift{12.915852in}{6.573336in}%
\pgfsys@useobject{currentmarker}{}%
\end{pgfscope}%
\begin{pgfscope}%
\pgfsys@transformshift{12.948113in}{6.577425in}%
\pgfsys@useobject{currentmarker}{}%
\end{pgfscope}%
\begin{pgfscope}%
\pgfsys@transformshift{12.981548in}{6.578529in}%
\pgfsys@useobject{currentmarker}{}%
\end{pgfscope}%
\begin{pgfscope}%
\pgfsys@transformshift{13.016256in}{6.590187in}%
\pgfsys@useobject{currentmarker}{}%
\end{pgfscope}%
\begin{pgfscope}%
\pgfsys@transformshift{13.052349in}{6.596078in}%
\pgfsys@useobject{currentmarker}{}%
\end{pgfscope}%
\begin{pgfscope}%
\pgfsys@transformshift{13.089956in}{6.598239in}%
\pgfsys@useobject{currentmarker}{}%
\end{pgfscope}%
\begin{pgfscope}%
\pgfsys@transformshift{13.129223in}{6.617105in}%
\pgfsys@useobject{currentmarker}{}%
\end{pgfscope}%
\begin{pgfscope}%
\pgfsys@transformshift{13.170323in}{6.619187in}%
\pgfsys@useobject{currentmarker}{}%
\end{pgfscope}%
\begin{pgfscope}%
\pgfsys@transformshift{13.213453in}{6.620198in}%
\pgfsys@useobject{currentmarker}{}%
\end{pgfscope}%
\begin{pgfscope}%
\pgfsys@transformshift{13.258847in}{6.621797in}%
\pgfsys@useobject{currentmarker}{}%
\end{pgfscope}%
\begin{pgfscope}%
\pgfsys@transformshift{13.306784in}{6.662321in}%
\pgfsys@useobject{currentmarker}{}%
\end{pgfscope}%
\begin{pgfscope}%
\pgfsys@transformshift{13.357597in}{6.701974in}%
\pgfsys@useobject{currentmarker}{}%
\end{pgfscope}%
\begin{pgfscope}%
\pgfsys@transformshift{13.411692in}{6.727183in}%
\pgfsys@useobject{currentmarker}{}%
\end{pgfscope}%
\begin{pgfscope}%
\pgfsys@transformshift{13.469569in}{6.728128in}%
\pgfsys@useobject{currentmarker}{}%
\end{pgfscope}%
\begin{pgfscope}%
\pgfsys@transformshift{13.531853in}{6.729359in}%
\pgfsys@useobject{currentmarker}{}%
\end{pgfscope}%
\begin{pgfscope}%
\pgfsys@transformshift{13.599347in}{6.749395in}%
\pgfsys@useobject{currentmarker}{}%
\end{pgfscope}%
\begin{pgfscope}%
\pgfsys@transformshift{13.673095in}{6.760668in}%
\pgfsys@useobject{currentmarker}{}%
\end{pgfscope}%
\begin{pgfscope}%
\pgfsys@transformshift{13.754504in}{6.770311in}%
\pgfsys@useobject{currentmarker}{}%
\end{pgfscope}%
\begin{pgfscope}%
\pgfsys@transformshift{13.845524in}{6.794418in}%
\pgfsys@useobject{currentmarker}{}%
\end{pgfscope}%
\begin{pgfscope}%
\pgfsys@transformshift{13.948987in}{6.825687in}%
\pgfsys@useobject{currentmarker}{}%
\end{pgfscope}%
\begin{pgfscope}%
\pgfsys@transformshift{14.069230in}{6.863522in}%
\pgfsys@useobject{currentmarker}{}%
\end{pgfscope}%
\begin{pgfscope}%
\pgfsys@transformshift{14.213424in}{6.881156in}%
\pgfsys@useobject{currentmarker}{}%
\end{pgfscope}%
\begin{pgfscope}%
\pgfsys@transformshift{14.394802in}{6.888504in}%
\pgfsys@useobject{currentmarker}{}%
\end{pgfscope}%
\begin{pgfscope}%
\pgfsys@transformshift{14.642472in}{6.939663in}%
\pgfsys@useobject{currentmarker}{}%
\end{pgfscope}%
\begin{pgfscope}%
\pgfsys@transformshift{15.046635in}{6.941948in}%
\pgfsys@useobject{currentmarker}{}%
\end{pgfscope}%
\end{pgfscope}%
\begin{pgfscope}%
\pgfpathrectangle{\pgfqpoint{2.125000in}{4.882857in}}{\pgfqpoint{13.175000in}{2.157143in}}%
\pgfusepath{clip}%
\pgfsetroundcap%
\pgfsetroundjoin%
\pgfsetlinewidth{1.505625pt}%
\definecolor{currentstroke}{rgb}{1.000000,0.000000,0.000000}%
\pgfsetstrokecolor{currentstroke}%
\pgfsetdash{}{0pt}%
\pgfpathmoveto{\pgfqpoint{2.378365in}{5.627795in}}%
\pgfpathlineto{\pgfqpoint{15.046635in}{6.721485in}}%
\pgfpathlineto{\pgfqpoint{15.046635in}{6.721485in}}%
\pgfusepath{stroke}%
\end{pgfscope}%
\begin{pgfscope}%
\pgfsetrectcap%
\pgfsetmiterjoin%
\pgfsetlinewidth{0.803000pt}%
\definecolor{currentstroke}{rgb}{1.000000,1.000000,1.000000}%
\pgfsetstrokecolor{currentstroke}%
\pgfsetdash{}{0pt}%
\pgfpathmoveto{\pgfqpoint{2.125000in}{4.882857in}}%
\pgfpathlineto{\pgfqpoint{2.125000in}{7.040000in}}%
\pgfusepath{stroke}%
\end{pgfscope}%
\begin{pgfscope}%
\pgfsetrectcap%
\pgfsetmiterjoin%
\pgfsetlinewidth{0.803000pt}%
\definecolor{currentstroke}{rgb}{1.000000,1.000000,1.000000}%
\pgfsetstrokecolor{currentstroke}%
\pgfsetdash{}{0pt}%
\pgfpathmoveto{\pgfqpoint{15.300000in}{4.882857in}}%
\pgfpathlineto{\pgfqpoint{15.300000in}{7.040000in}}%
\pgfusepath{stroke}%
\end{pgfscope}%
\begin{pgfscope}%
\pgfsetrectcap%
\pgfsetmiterjoin%
\pgfsetlinewidth{0.803000pt}%
\definecolor{currentstroke}{rgb}{1.000000,1.000000,1.000000}%
\pgfsetstrokecolor{currentstroke}%
\pgfsetdash{}{0pt}%
\pgfpathmoveto{\pgfqpoint{2.125000in}{4.882857in}}%
\pgfpathlineto{\pgfqpoint{15.300000in}{4.882857in}}%
\pgfusepath{stroke}%
\end{pgfscope}%
\begin{pgfscope}%
\pgfsetrectcap%
\pgfsetmiterjoin%
\pgfsetlinewidth{0.803000pt}%
\definecolor{currentstroke}{rgb}{1.000000,1.000000,1.000000}%
\pgfsetstrokecolor{currentstroke}%
\pgfsetdash{}{0pt}%
\pgfpathmoveto{\pgfqpoint{2.125000in}{7.040000in}}%
\pgfpathlineto{\pgfqpoint{15.300000in}{7.040000in}}%
\pgfusepath{stroke}%
\end{pgfscope}%
\begin{pgfscope}%
\definecolor{textcolor}{rgb}{0.150000,0.150000,0.150000}%
\pgfsetstrokecolor{textcolor}%
\pgfsetfillcolor{textcolor}%
\pgftext[x=8.712500in,y=7.123333in,,base]{\color{textcolor}\rmfamily\fontsize{12.000000}{14.400000}\selectfont QQ Plot}%
\end{pgfscope}%
\begin{pgfscope}%
\pgfsetbuttcap%
\pgfsetmiterjoin%
\definecolor{currentfill}{rgb}{0.917647,0.917647,0.949020}%
\pgfsetfillcolor{currentfill}%
\pgfsetlinewidth{0.000000pt}%
\definecolor{currentstroke}{rgb}{0.000000,0.000000,0.000000}%
\pgfsetstrokecolor{currentstroke}%
\pgfsetstrokeopacity{0.000000}%
\pgfsetdash{}{0pt}%
\pgfpathmoveto{\pgfqpoint{2.125000in}{1.000000in}}%
\pgfpathlineto{\pgfqpoint{15.300000in}{1.000000in}}%
\pgfpathlineto{\pgfqpoint{15.300000in}{3.157143in}}%
\pgfpathlineto{\pgfqpoint{2.125000in}{3.157143in}}%
\pgfpathclose%
\pgfusepath{fill}%
\end{pgfscope}%
\begin{pgfscope}%
\pgfpathrectangle{\pgfqpoint{2.125000in}{1.000000in}}{\pgfqpoint{13.175000in}{2.157143in}}%
\pgfusepath{clip}%
\pgfsetroundcap%
\pgfsetroundjoin%
\pgfsetlinewidth{0.803000pt}%
\definecolor{currentstroke}{rgb}{1.000000,1.000000,1.000000}%
\pgfsetstrokecolor{currentstroke}%
\pgfsetdash{}{0pt}%
\pgfpathmoveto{\pgfqpoint{2.723864in}{1.000000in}}%
\pgfpathlineto{\pgfqpoint{2.723864in}{3.157143in}}%
\pgfusepath{stroke}%
\end{pgfscope}%
\begin{pgfscope}%
\definecolor{textcolor}{rgb}{0.150000,0.150000,0.150000}%
\pgfsetstrokecolor{textcolor}%
\pgfsetfillcolor{textcolor}%
\pgftext[x=2.723864in,y=0.902778in,,top]{\color{textcolor}\rmfamily\fontsize{10.000000}{12.000000}\selectfont 0}%
\end{pgfscope}%
\begin{pgfscope}%
\pgfpathrectangle{\pgfqpoint{2.125000in}{1.000000in}}{\pgfqpoint{13.175000in}{2.157143in}}%
\pgfusepath{clip}%
\pgfsetroundcap%
\pgfsetroundjoin%
\pgfsetlinewidth{0.803000pt}%
\definecolor{currentstroke}{rgb}{1.000000,1.000000,1.000000}%
\pgfsetstrokecolor{currentstroke}%
\pgfsetdash{}{0pt}%
\pgfpathmoveto{\pgfqpoint{4.313415in}{1.000000in}}%
\pgfpathlineto{\pgfqpoint{4.313415in}{3.157143in}}%
\pgfusepath{stroke}%
\end{pgfscope}%
\begin{pgfscope}%
\definecolor{textcolor}{rgb}{0.150000,0.150000,0.150000}%
\pgfsetstrokecolor{textcolor}%
\pgfsetfillcolor{textcolor}%
\pgftext[x=4.313415in,y=0.902778in,,top]{\color{textcolor}\rmfamily\fontsize{10.000000}{12.000000}\selectfont 200}%
\end{pgfscope}%
\begin{pgfscope}%
\pgfpathrectangle{\pgfqpoint{2.125000in}{1.000000in}}{\pgfqpoint{13.175000in}{2.157143in}}%
\pgfusepath{clip}%
\pgfsetroundcap%
\pgfsetroundjoin%
\pgfsetlinewidth{0.803000pt}%
\definecolor{currentstroke}{rgb}{1.000000,1.000000,1.000000}%
\pgfsetstrokecolor{currentstroke}%
\pgfsetdash{}{0pt}%
\pgfpathmoveto{\pgfqpoint{5.902967in}{1.000000in}}%
\pgfpathlineto{\pgfqpoint{5.902967in}{3.157143in}}%
\pgfusepath{stroke}%
\end{pgfscope}%
\begin{pgfscope}%
\definecolor{textcolor}{rgb}{0.150000,0.150000,0.150000}%
\pgfsetstrokecolor{textcolor}%
\pgfsetfillcolor{textcolor}%
\pgftext[x=5.902967in,y=0.902778in,,top]{\color{textcolor}\rmfamily\fontsize{10.000000}{12.000000}\selectfont 400}%
\end{pgfscope}%
\begin{pgfscope}%
\pgfpathrectangle{\pgfqpoint{2.125000in}{1.000000in}}{\pgfqpoint{13.175000in}{2.157143in}}%
\pgfusepath{clip}%
\pgfsetroundcap%
\pgfsetroundjoin%
\pgfsetlinewidth{0.803000pt}%
\definecolor{currentstroke}{rgb}{1.000000,1.000000,1.000000}%
\pgfsetstrokecolor{currentstroke}%
\pgfsetdash{}{0pt}%
\pgfpathmoveto{\pgfqpoint{7.492519in}{1.000000in}}%
\pgfpathlineto{\pgfqpoint{7.492519in}{3.157143in}}%
\pgfusepath{stroke}%
\end{pgfscope}%
\begin{pgfscope}%
\definecolor{textcolor}{rgb}{0.150000,0.150000,0.150000}%
\pgfsetstrokecolor{textcolor}%
\pgfsetfillcolor{textcolor}%
\pgftext[x=7.492519in,y=0.902778in,,top]{\color{textcolor}\rmfamily\fontsize{10.000000}{12.000000}\selectfont 600}%
\end{pgfscope}%
\begin{pgfscope}%
\pgfpathrectangle{\pgfqpoint{2.125000in}{1.000000in}}{\pgfqpoint{13.175000in}{2.157143in}}%
\pgfusepath{clip}%
\pgfsetroundcap%
\pgfsetroundjoin%
\pgfsetlinewidth{0.803000pt}%
\definecolor{currentstroke}{rgb}{1.000000,1.000000,1.000000}%
\pgfsetstrokecolor{currentstroke}%
\pgfsetdash{}{0pt}%
\pgfpathmoveto{\pgfqpoint{9.082071in}{1.000000in}}%
\pgfpathlineto{\pgfqpoint{9.082071in}{3.157143in}}%
\pgfusepath{stroke}%
\end{pgfscope}%
\begin{pgfscope}%
\definecolor{textcolor}{rgb}{0.150000,0.150000,0.150000}%
\pgfsetstrokecolor{textcolor}%
\pgfsetfillcolor{textcolor}%
\pgftext[x=9.082071in,y=0.902778in,,top]{\color{textcolor}\rmfamily\fontsize{10.000000}{12.000000}\selectfont 800}%
\end{pgfscope}%
\begin{pgfscope}%
\pgfpathrectangle{\pgfqpoint{2.125000in}{1.000000in}}{\pgfqpoint{13.175000in}{2.157143in}}%
\pgfusepath{clip}%
\pgfsetroundcap%
\pgfsetroundjoin%
\pgfsetlinewidth{0.803000pt}%
\definecolor{currentstroke}{rgb}{1.000000,1.000000,1.000000}%
\pgfsetstrokecolor{currentstroke}%
\pgfsetdash{}{0pt}%
\pgfpathmoveto{\pgfqpoint{10.671623in}{1.000000in}}%
\pgfpathlineto{\pgfqpoint{10.671623in}{3.157143in}}%
\pgfusepath{stroke}%
\end{pgfscope}%
\begin{pgfscope}%
\definecolor{textcolor}{rgb}{0.150000,0.150000,0.150000}%
\pgfsetstrokecolor{textcolor}%
\pgfsetfillcolor{textcolor}%
\pgftext[x=10.671623in,y=0.902778in,,top]{\color{textcolor}\rmfamily\fontsize{10.000000}{12.000000}\selectfont 1000}%
\end{pgfscope}%
\begin{pgfscope}%
\pgfpathrectangle{\pgfqpoint{2.125000in}{1.000000in}}{\pgfqpoint{13.175000in}{2.157143in}}%
\pgfusepath{clip}%
\pgfsetroundcap%
\pgfsetroundjoin%
\pgfsetlinewidth{0.803000pt}%
\definecolor{currentstroke}{rgb}{1.000000,1.000000,1.000000}%
\pgfsetstrokecolor{currentstroke}%
\pgfsetdash{}{0pt}%
\pgfpathmoveto{\pgfqpoint{12.261174in}{1.000000in}}%
\pgfpathlineto{\pgfqpoint{12.261174in}{3.157143in}}%
\pgfusepath{stroke}%
\end{pgfscope}%
\begin{pgfscope}%
\definecolor{textcolor}{rgb}{0.150000,0.150000,0.150000}%
\pgfsetstrokecolor{textcolor}%
\pgfsetfillcolor{textcolor}%
\pgftext[x=12.261174in,y=0.902778in,,top]{\color{textcolor}\rmfamily\fontsize{10.000000}{12.000000}\selectfont 1200}%
\end{pgfscope}%
\begin{pgfscope}%
\pgfpathrectangle{\pgfqpoint{2.125000in}{1.000000in}}{\pgfqpoint{13.175000in}{2.157143in}}%
\pgfusepath{clip}%
\pgfsetroundcap%
\pgfsetroundjoin%
\pgfsetlinewidth{0.803000pt}%
\definecolor{currentstroke}{rgb}{1.000000,1.000000,1.000000}%
\pgfsetstrokecolor{currentstroke}%
\pgfsetdash{}{0pt}%
\pgfpathmoveto{\pgfqpoint{13.850726in}{1.000000in}}%
\pgfpathlineto{\pgfqpoint{13.850726in}{3.157143in}}%
\pgfusepath{stroke}%
\end{pgfscope}%
\begin{pgfscope}%
\definecolor{textcolor}{rgb}{0.150000,0.150000,0.150000}%
\pgfsetstrokecolor{textcolor}%
\pgfsetfillcolor{textcolor}%
\pgftext[x=13.850726in,y=0.902778in,,top]{\color{textcolor}\rmfamily\fontsize{10.000000}{12.000000}\selectfont 1400}%
\end{pgfscope}%
\begin{pgfscope}%
\pgfpathrectangle{\pgfqpoint{2.125000in}{1.000000in}}{\pgfqpoint{13.175000in}{2.157143in}}%
\pgfusepath{clip}%
\pgfsetroundcap%
\pgfsetroundjoin%
\pgfsetlinewidth{0.803000pt}%
\definecolor{currentstroke}{rgb}{1.000000,1.000000,1.000000}%
\pgfsetstrokecolor{currentstroke}%
\pgfsetdash{}{0pt}%
\pgfpathmoveto{\pgfqpoint{2.125000in}{1.109925in}}%
\pgfpathlineto{\pgfqpoint{15.300000in}{1.109925in}}%
\pgfusepath{stroke}%
\end{pgfscope}%
\begin{pgfscope}%
\definecolor{textcolor}{rgb}{0.150000,0.150000,0.150000}%
\pgfsetstrokecolor{textcolor}%
\pgfsetfillcolor{textcolor}%
\pgftext[x=1.602159in,y=1.057163in,left,base]{\color{textcolor}\rmfamily\fontsize{10.000000}{12.000000}\selectfont −0.06}%
\end{pgfscope}%
\begin{pgfscope}%
\pgfpathrectangle{\pgfqpoint{2.125000in}{1.000000in}}{\pgfqpoint{13.175000in}{2.157143in}}%
\pgfusepath{clip}%
\pgfsetroundcap%
\pgfsetroundjoin%
\pgfsetlinewidth{0.803000pt}%
\definecolor{currentstroke}{rgb}{1.000000,1.000000,1.000000}%
\pgfsetstrokecolor{currentstroke}%
\pgfsetdash{}{0pt}%
\pgfpathmoveto{\pgfqpoint{2.125000in}{1.501676in}}%
\pgfpathlineto{\pgfqpoint{15.300000in}{1.501676in}}%
\pgfusepath{stroke}%
\end{pgfscope}%
\begin{pgfscope}%
\definecolor{textcolor}{rgb}{0.150000,0.150000,0.150000}%
\pgfsetstrokecolor{textcolor}%
\pgfsetfillcolor{textcolor}%
\pgftext[x=1.602159in,y=1.448914in,left,base]{\color{textcolor}\rmfamily\fontsize{10.000000}{12.000000}\selectfont −0.04}%
\end{pgfscope}%
\begin{pgfscope}%
\pgfpathrectangle{\pgfqpoint{2.125000in}{1.000000in}}{\pgfqpoint{13.175000in}{2.157143in}}%
\pgfusepath{clip}%
\pgfsetroundcap%
\pgfsetroundjoin%
\pgfsetlinewidth{0.803000pt}%
\definecolor{currentstroke}{rgb}{1.000000,1.000000,1.000000}%
\pgfsetstrokecolor{currentstroke}%
\pgfsetdash{}{0pt}%
\pgfpathmoveto{\pgfqpoint{2.125000in}{1.893426in}}%
\pgfpathlineto{\pgfqpoint{15.300000in}{1.893426in}}%
\pgfusepath{stroke}%
\end{pgfscope}%
\begin{pgfscope}%
\definecolor{textcolor}{rgb}{0.150000,0.150000,0.150000}%
\pgfsetstrokecolor{textcolor}%
\pgfsetfillcolor{textcolor}%
\pgftext[x=1.602159in,y=1.840665in,left,base]{\color{textcolor}\rmfamily\fontsize{10.000000}{12.000000}\selectfont −0.02}%
\end{pgfscope}%
\begin{pgfscope}%
\pgfpathrectangle{\pgfqpoint{2.125000in}{1.000000in}}{\pgfqpoint{13.175000in}{2.157143in}}%
\pgfusepath{clip}%
\pgfsetroundcap%
\pgfsetroundjoin%
\pgfsetlinewidth{0.803000pt}%
\definecolor{currentstroke}{rgb}{1.000000,1.000000,1.000000}%
\pgfsetstrokecolor{currentstroke}%
\pgfsetdash{}{0pt}%
\pgfpathmoveto{\pgfqpoint{2.125000in}{2.285177in}}%
\pgfpathlineto{\pgfqpoint{15.300000in}{2.285177in}}%
\pgfusepath{stroke}%
\end{pgfscope}%
\begin{pgfscope}%
\definecolor{textcolor}{rgb}{0.150000,0.150000,0.150000}%
\pgfsetstrokecolor{textcolor}%
\pgfsetfillcolor{textcolor}%
\pgftext[x=1.718533in,y=2.232415in,left,base]{\color{textcolor}\rmfamily\fontsize{10.000000}{12.000000}\selectfont 0.00}%
\end{pgfscope}%
\begin{pgfscope}%
\pgfpathrectangle{\pgfqpoint{2.125000in}{1.000000in}}{\pgfqpoint{13.175000in}{2.157143in}}%
\pgfusepath{clip}%
\pgfsetroundcap%
\pgfsetroundjoin%
\pgfsetlinewidth{0.803000pt}%
\definecolor{currentstroke}{rgb}{1.000000,1.000000,1.000000}%
\pgfsetstrokecolor{currentstroke}%
\pgfsetdash{}{0pt}%
\pgfpathmoveto{\pgfqpoint{2.125000in}{2.676927in}}%
\pgfpathlineto{\pgfqpoint{15.300000in}{2.676927in}}%
\pgfusepath{stroke}%
\end{pgfscope}%
\begin{pgfscope}%
\definecolor{textcolor}{rgb}{0.150000,0.150000,0.150000}%
\pgfsetstrokecolor{textcolor}%
\pgfsetfillcolor{textcolor}%
\pgftext[x=1.718533in,y=2.624166in,left,base]{\color{textcolor}\rmfamily\fontsize{10.000000}{12.000000}\selectfont 0.02}%
\end{pgfscope}%
\begin{pgfscope}%
\pgfpathrectangle{\pgfqpoint{2.125000in}{1.000000in}}{\pgfqpoint{13.175000in}{2.157143in}}%
\pgfusepath{clip}%
\pgfsetroundcap%
\pgfsetroundjoin%
\pgfsetlinewidth{0.803000pt}%
\definecolor{currentstroke}{rgb}{1.000000,1.000000,1.000000}%
\pgfsetstrokecolor{currentstroke}%
\pgfsetdash{}{0pt}%
\pgfpathmoveto{\pgfqpoint{2.125000in}{3.068678in}}%
\pgfpathlineto{\pgfqpoint{15.300000in}{3.068678in}}%
\pgfusepath{stroke}%
\end{pgfscope}%
\begin{pgfscope}%
\definecolor{textcolor}{rgb}{0.150000,0.150000,0.150000}%
\pgfsetstrokecolor{textcolor}%
\pgfsetfillcolor{textcolor}%
\pgftext[x=1.718533in,y=3.015916in,left,base]{\color{textcolor}\rmfamily\fontsize{10.000000}{12.000000}\selectfont 0.04}%
\end{pgfscope}%
\begin{pgfscope}%
\pgfpathrectangle{\pgfqpoint{2.125000in}{1.000000in}}{\pgfqpoint{13.175000in}{2.157143in}}%
\pgfusepath{clip}%
\pgfsetroundcap%
\pgfsetroundjoin%
\pgfsetlinewidth{1.505625pt}%
\definecolor{currentstroke}{rgb}{0.121569,0.466667,0.705882}%
\pgfsetstrokecolor{currentstroke}%
\pgfsetdash{}{0pt}%
\pgfpathmoveto{\pgfqpoint{2.723864in}{2.277725in}}%
\pgfpathlineto{\pgfqpoint{2.731811in}{2.203017in}}%
\pgfpathlineto{\pgfqpoint{2.739759in}{2.236465in}}%
\pgfpathlineto{\pgfqpoint{2.747707in}{2.367540in}}%
\pgfpathlineto{\pgfqpoint{2.755655in}{2.195308in}}%
\pgfpathlineto{\pgfqpoint{2.763602in}{2.092826in}}%
\pgfpathlineto{\pgfqpoint{2.771550in}{2.323042in}}%
\pgfpathlineto{\pgfqpoint{2.779498in}{2.285177in}}%
\pgfpathlineto{\pgfqpoint{2.787446in}{2.420887in}}%
\pgfpathlineto{\pgfqpoint{2.795393in}{2.371391in}}%
\pgfpathlineto{\pgfqpoint{2.803341in}{2.300132in}}%
\pgfpathlineto{\pgfqpoint{2.811289in}{2.329974in}}%
\pgfpathlineto{\pgfqpoint{2.819237in}{1.916300in}}%
\pgfpathlineto{\pgfqpoint{2.827185in}{2.136424in}}%
\pgfpathlineto{\pgfqpoint{2.835132in}{2.430129in}}%
\pgfpathlineto{\pgfqpoint{2.843080in}{2.228087in}}%
\pgfpathlineto{\pgfqpoint{2.851028in}{2.135959in}}%
\pgfpathlineto{\pgfqpoint{2.858976in}{1.948154in}}%
\pgfpathlineto{\pgfqpoint{2.866923in}{2.234315in}}%
\pgfpathlineto{\pgfqpoint{2.874871in}{2.336038in}}%
\pgfpathlineto{\pgfqpoint{2.882819in}{2.320311in}}%
\pgfpathlineto{\pgfqpoint{2.890767in}{2.112805in}}%
\pgfpathlineto{\pgfqpoint{2.898714in}{2.515966in}}%
\pgfpathlineto{\pgfqpoint{2.906662in}{2.343421in}}%
\pgfpathlineto{\pgfqpoint{2.914610in}{2.265781in}}%
\pgfpathlineto{\pgfqpoint{2.922558in}{2.408977in}}%
\pgfpathlineto{\pgfqpoint{2.930505in}{2.234977in}}%
\pgfpathlineto{\pgfqpoint{2.938453in}{2.393140in}}%
\pgfpathlineto{\pgfqpoint{2.946401in}{2.361930in}}%
\pgfpathlineto{\pgfqpoint{2.954349in}{2.308144in}}%
\pgfpathlineto{\pgfqpoint{2.962296in}{2.479321in}}%
\pgfpathlineto{\pgfqpoint{2.970244in}{2.197859in}}%
\pgfpathlineto{\pgfqpoint{2.978192in}{2.136221in}}%
\pgfpathlineto{\pgfqpoint{2.986140in}{2.292843in}}%
\pgfpathlineto{\pgfqpoint{2.994087in}{2.877811in}}%
\pgfpathlineto{\pgfqpoint{3.002035in}{2.370509in}}%
\pgfpathlineto{\pgfqpoint{3.009983in}{2.281474in}}%
\pgfpathlineto{\pgfqpoint{3.017931in}{2.487776in}}%
\pgfpathlineto{\pgfqpoint{3.025878in}{2.351029in}}%
\pgfpathlineto{\pgfqpoint{3.033826in}{2.005613in}}%
\pgfpathlineto{\pgfqpoint{3.041774in}{2.288881in}}%
\pgfpathlineto{\pgfqpoint{3.049722in}{2.366499in}}%
\pgfpathlineto{\pgfqpoint{3.057670in}{2.251949in}}%
\pgfpathlineto{\pgfqpoint{3.065617in}{2.211136in}}%
\pgfpathlineto{\pgfqpoint{3.073565in}{2.377684in}}%
\pgfpathlineto{\pgfqpoint{3.081513in}{2.292558in}}%
\pgfpathlineto{\pgfqpoint{3.089461in}{2.512642in}}%
\pgfpathlineto{\pgfqpoint{3.097408in}{2.339814in}}%
\pgfpathlineto{\pgfqpoint{3.105356in}{2.270622in}}%
\pgfpathlineto{\pgfqpoint{3.113304in}{2.237798in}}%
\pgfpathlineto{\pgfqpoint{3.121252in}{2.160718in}}%
\pgfpathlineto{\pgfqpoint{3.129199in}{2.274157in}}%
\pgfpathlineto{\pgfqpoint{3.137147in}{2.285177in}}%
\pgfpathlineto{\pgfqpoint{3.145095in}{2.281502in}}%
\pgfpathlineto{\pgfqpoint{3.153043in}{2.376836in}}%
\pgfpathlineto{\pgfqpoint{3.160990in}{2.259555in}}%
\pgfpathlineto{\pgfqpoint{3.168938in}{2.292500in}}%
\pgfpathlineto{\pgfqpoint{3.176886in}{2.197110in}}%
\pgfpathlineto{\pgfqpoint{3.184834in}{2.296207in}}%
\pgfpathlineto{\pgfqpoint{3.192781in}{2.233650in}}%
\pgfpathlineto{\pgfqpoint{3.200729in}{2.344053in}}%
\pgfpathlineto{\pgfqpoint{3.208677in}{2.384131in}}%
\pgfpathlineto{\pgfqpoint{3.216625in}{2.149445in}}%
\pgfpathlineto{\pgfqpoint{3.224572in}{2.336645in}}%
\pgfpathlineto{\pgfqpoint{3.232520in}{2.296188in}}%
\pgfpathlineto{\pgfqpoint{3.240468in}{2.141543in}}%
\pgfpathlineto{\pgfqpoint{3.248416in}{2.147928in}}%
\pgfpathlineto{\pgfqpoint{3.256363in}{2.311216in}}%
\pgfpathlineto{\pgfqpoint{3.264311in}{2.184546in}}%
\pgfpathlineto{\pgfqpoint{3.272259in}{2.202797in}}%
\pgfpathlineto{\pgfqpoint{3.280207in}{2.568303in}}%
\pgfpathlineto{\pgfqpoint{3.288155in}{2.355323in}}%
\pgfpathlineto{\pgfqpoint{3.296102in}{2.207631in}}%
\pgfpathlineto{\pgfqpoint{3.304050in}{2.233309in}}%
\pgfpathlineto{\pgfqpoint{3.311998in}{2.557793in}}%
\pgfpathlineto{\pgfqpoint{3.319946in}{2.034806in}}%
\pgfpathlineto{\pgfqpoint{3.327893in}{2.388658in}}%
\pgfpathlineto{\pgfqpoint{3.335841in}{2.417426in}}%
\pgfpathlineto{\pgfqpoint{3.343789in}{2.277853in}}%
\pgfpathlineto{\pgfqpoint{3.351737in}{1.561426in}}%
\pgfpathlineto{\pgfqpoint{3.359684in}{2.040425in}}%
\pgfpathlineto{\pgfqpoint{3.367632in}{2.265926in}}%
\pgfpathlineto{\pgfqpoint{3.375580in}{2.408055in}}%
\pgfpathlineto{\pgfqpoint{3.383528in}{2.449090in}}%
\pgfpathlineto{\pgfqpoint{3.391475in}{2.212919in}}%
\pgfpathlineto{\pgfqpoint{3.399423in}{2.277555in}}%
\pgfpathlineto{\pgfqpoint{3.407371in}{2.258478in}}%
\pgfpathlineto{\pgfqpoint{3.415319in}{2.135755in}}%
\pgfpathlineto{\pgfqpoint{3.423266in}{2.426963in}}%
\pgfpathlineto{\pgfqpoint{3.431214in}{2.143390in}}%
\pgfpathlineto{\pgfqpoint{3.439162in}{2.254385in}}%
\pgfpathlineto{\pgfqpoint{3.447110in}{2.331346in}}%
\pgfpathlineto{\pgfqpoint{3.455057in}{2.457351in}}%
\pgfpathlineto{\pgfqpoint{3.463005in}{2.185883in}}%
\pgfpathlineto{\pgfqpoint{3.470953in}{2.150712in}}%
\pgfpathlineto{\pgfqpoint{3.478901in}{2.242725in}}%
\pgfpathlineto{\pgfqpoint{3.486848in}{2.211634in}}%
\pgfpathlineto{\pgfqpoint{3.494796in}{2.047182in}}%
\pgfpathlineto{\pgfqpoint{3.502744in}{2.343969in}}%
\pgfpathlineto{\pgfqpoint{3.510692in}{2.257762in}}%
\pgfpathlineto{\pgfqpoint{3.518640in}{2.429648in}}%
\pgfpathlineto{\pgfqpoint{3.526587in}{2.089691in}}%
\pgfpathlineto{\pgfqpoint{3.534535in}{2.273385in}}%
\pgfpathlineto{\pgfqpoint{3.542483in}{2.051824in}}%
\pgfpathlineto{\pgfqpoint{3.550431in}{2.237373in}}%
\pgfpathlineto{\pgfqpoint{3.558378in}{2.213252in}}%
\pgfpathlineto{\pgfqpoint{3.566326in}{2.484320in}}%
\pgfpathlineto{\pgfqpoint{3.574274in}{2.587951in}}%
\pgfpathlineto{\pgfqpoint{3.582222in}{2.281274in}}%
\pgfpathlineto{\pgfqpoint{3.590169in}{2.218719in}}%
\pgfpathlineto{\pgfqpoint{3.598117in}{2.355537in}}%
\pgfpathlineto{\pgfqpoint{3.606065in}{2.226561in}}%
\pgfpathlineto{\pgfqpoint{3.614013in}{2.472134in}}%
\pgfpathlineto{\pgfqpoint{3.621960in}{2.191921in}}%
\pgfpathlineto{\pgfqpoint{3.629908in}{2.101255in}}%
\pgfpathlineto{\pgfqpoint{3.637856in}{2.261572in}}%
\pgfpathlineto{\pgfqpoint{3.645804in}{1.701862in}}%
\pgfpathlineto{\pgfqpoint{3.653751in}{2.077252in}}%
\pgfpathlineto{\pgfqpoint{3.661699in}{2.309753in}}%
\pgfpathlineto{\pgfqpoint{3.669647in}{2.116619in}}%
\pgfpathlineto{\pgfqpoint{3.677595in}{2.272786in}}%
\pgfpathlineto{\pgfqpoint{3.685542in}{2.515186in}}%
\pgfpathlineto{\pgfqpoint{3.693490in}{2.386994in}}%
\pgfpathlineto{\pgfqpoint{3.701438in}{2.595489in}}%
\pgfpathlineto{\pgfqpoint{3.709386in}{2.265175in}}%
\pgfpathlineto{\pgfqpoint{3.717334in}{2.341130in}}%
\pgfpathlineto{\pgfqpoint{3.725281in}{2.281185in}}%
\pgfpathlineto{\pgfqpoint{3.733229in}{2.261211in}}%
\pgfpathlineto{\pgfqpoint{3.741177in}{2.372906in}}%
\pgfpathlineto{\pgfqpoint{3.749125in}{2.340800in}}%
\pgfpathlineto{\pgfqpoint{3.757072in}{2.181749in}}%
\pgfpathlineto{\pgfqpoint{3.765020in}{3.005647in}}%
\pgfpathlineto{\pgfqpoint{3.772968in}{2.707330in}}%
\pgfpathlineto{\pgfqpoint{3.780916in}{2.202226in}}%
\pgfpathlineto{\pgfqpoint{3.788863in}{2.446980in}}%
\pgfpathlineto{\pgfqpoint{3.796811in}{2.292670in}}%
\pgfpathlineto{\pgfqpoint{3.804759in}{2.318861in}}%
\pgfpathlineto{\pgfqpoint{3.812707in}{2.225253in}}%
\pgfpathlineto{\pgfqpoint{3.820654in}{2.183638in}}%
\pgfpathlineto{\pgfqpoint{3.828602in}{2.171735in}}%
\pgfpathlineto{\pgfqpoint{3.836550in}{2.285177in}}%
\pgfpathlineto{\pgfqpoint{3.844498in}{2.447573in}}%
\pgfpathlineto{\pgfqpoint{3.852445in}{2.446238in}}%
\pgfpathlineto{\pgfqpoint{3.860393in}{2.288906in}}%
\pgfpathlineto{\pgfqpoint{3.868341in}{2.116624in}}%
\pgfpathlineto{\pgfqpoint{3.876289in}{2.122749in}}%
\pgfpathlineto{\pgfqpoint{3.884236in}{2.132862in}}%
\pgfpathlineto{\pgfqpoint{3.892184in}{2.887454in}}%
\pgfpathlineto{\pgfqpoint{3.900132in}{2.377632in}}%
\pgfpathlineto{\pgfqpoint{3.916027in}{2.434937in}}%
\pgfpathlineto{\pgfqpoint{3.923975in}{2.288815in}}%
\pgfpathlineto{\pgfqpoint{3.931923in}{2.296088in}}%
\pgfpathlineto{\pgfqpoint{3.939871in}{2.197716in}}%
\pgfpathlineto{\pgfqpoint{3.947819in}{2.361726in}}%
\pgfpathlineto{\pgfqpoint{3.955766in}{2.256050in}}%
\pgfpathlineto{\pgfqpoint{3.963714in}{2.390555in}}%
\pgfpathlineto{\pgfqpoint{3.971662in}{2.285177in}}%
\pgfpathlineto{\pgfqpoint{3.979610in}{2.212563in}}%
\pgfpathlineto{\pgfqpoint{3.987557in}{2.292450in}}%
\pgfpathlineto{\pgfqpoint{3.995505in}{2.306981in}}%
\pgfpathlineto{\pgfqpoint{4.003453in}{2.237903in}}%
\pgfpathlineto{\pgfqpoint{4.011401in}{2.383232in}}%
\pgfpathlineto{\pgfqpoint{4.019348in}{2.310519in}}%
\pgfpathlineto{\pgfqpoint{4.027296in}{2.252588in}}%
\pgfpathlineto{\pgfqpoint{4.035244in}{2.248903in}}%
\pgfpathlineto{\pgfqpoint{4.043192in}{2.288807in}}%
\pgfpathlineto{\pgfqpoint{4.051139in}{2.375717in}}%
\pgfpathlineto{\pgfqpoint{4.059087in}{2.346506in}}%
\pgfpathlineto{\pgfqpoint{4.067035in}{2.252732in}}%
\pgfpathlineto{\pgfqpoint{4.074983in}{2.557479in}}%
\pgfpathlineto{\pgfqpoint{4.082930in}{2.366843in}}%
\pgfpathlineto{\pgfqpoint{4.090878in}{2.281633in}}%
\pgfpathlineto{\pgfqpoint{4.098826in}{2.214169in}}%
\pgfpathlineto{\pgfqpoint{4.106774in}{2.238883in}}%
\pgfpathlineto{\pgfqpoint{4.114721in}{2.515558in}}%
\pgfpathlineto{\pgfqpoint{4.122669in}{2.359032in}}%
\pgfpathlineto{\pgfqpoint{4.130617in}{2.309733in}}%
\pgfpathlineto{\pgfqpoint{4.138565in}{2.274656in}}%
\pgfpathlineto{\pgfqpoint{4.146512in}{2.299203in}}%
\pgfpathlineto{\pgfqpoint{4.154460in}{2.369123in}}%
\pgfpathlineto{\pgfqpoint{4.162408in}{2.246746in}}%
\pgfpathlineto{\pgfqpoint{4.170356in}{2.379373in}}%
\pgfpathlineto{\pgfqpoint{4.178304in}{2.236391in}}%
\pgfpathlineto{\pgfqpoint{4.186251in}{2.204764in}}%
\pgfpathlineto{\pgfqpoint{4.194199in}{2.285177in}}%
\pgfpathlineto{\pgfqpoint{4.202147in}{2.302686in}}%
\pgfpathlineto{\pgfqpoint{4.210095in}{2.309663in}}%
\pgfpathlineto{\pgfqpoint{4.218042in}{2.099012in}}%
\pgfpathlineto{\pgfqpoint{4.225990in}{2.390770in}}%
\pgfpathlineto{\pgfqpoint{4.233938in}{2.344761in}}%
\pgfpathlineto{\pgfqpoint{4.241886in}{2.358532in}}%
\pgfpathlineto{\pgfqpoint{4.249833in}{2.134678in}}%
\pgfpathlineto{\pgfqpoint{4.257781in}{2.168888in}}%
\pgfpathlineto{\pgfqpoint{4.265729in}{2.125480in}}%
\pgfpathlineto{\pgfqpoint{4.273677in}{2.245941in}}%
\pgfpathlineto{\pgfqpoint{4.281624in}{2.267316in}}%
\pgfpathlineto{\pgfqpoint{4.289572in}{2.505503in}}%
\pgfpathlineto{\pgfqpoint{4.297520in}{2.369803in}}%
\pgfpathlineto{\pgfqpoint{4.305468in}{2.578524in}}%
\pgfpathlineto{\pgfqpoint{4.313415in}{2.285177in}}%
\pgfpathlineto{\pgfqpoint{4.321363in}{2.026983in}}%
\pgfpathlineto{\pgfqpoint{4.329311in}{2.267608in}}%
\pgfpathlineto{\pgfqpoint{4.337259in}{1.980499in}}%
\pgfpathlineto{\pgfqpoint{4.345206in}{2.469966in}}%
\pgfpathlineto{\pgfqpoint{4.353154in}{2.846502in}}%
\pgfpathlineto{\pgfqpoint{4.361102in}{2.109100in}}%
\pgfpathlineto{\pgfqpoint{4.369050in}{2.229609in}}%
\pgfpathlineto{\pgfqpoint{4.376997in}{2.288654in}}%
\pgfpathlineto{\pgfqpoint{4.384945in}{2.267782in}}%
\pgfpathlineto{\pgfqpoint{4.392893in}{2.138453in}}%
\pgfpathlineto{\pgfqpoint{4.400841in}{2.358676in}}%
\pgfpathlineto{\pgfqpoint{4.408789in}{2.035561in}}%
\pgfpathlineto{\pgfqpoint{4.416736in}{1.949729in}}%
\pgfpathlineto{\pgfqpoint{4.424684in}{2.317544in}}%
\pgfpathlineto{\pgfqpoint{4.432632in}{2.303135in}}%
\pgfpathlineto{\pgfqpoint{4.440580in}{2.220448in}}%
\pgfpathlineto{\pgfqpoint{4.448527in}{2.187681in}}%
\pgfpathlineto{\pgfqpoint{4.456475in}{2.223541in}}%
\pgfpathlineto{\pgfqpoint{4.464423in}{2.433499in}}%
\pgfpathlineto{\pgfqpoint{4.472371in}{2.606877in}}%
\pgfpathlineto{\pgfqpoint{4.480318in}{2.391247in}}%
\pgfpathlineto{\pgfqpoint{4.488266in}{2.334480in}}%
\pgfpathlineto{\pgfqpoint{4.496214in}{2.602659in}}%
\pgfpathlineto{\pgfqpoint{4.504162in}{2.254006in}}%
\pgfpathlineto{\pgfqpoint{4.512109in}{2.149526in}}%
\pgfpathlineto{\pgfqpoint{4.520057in}{2.410426in}}%
\pgfpathlineto{\pgfqpoint{4.528005in}{2.302509in}}%
\pgfpathlineto{\pgfqpoint{4.535953in}{2.378508in}}%
\pgfpathlineto{\pgfqpoint{4.543900in}{2.216085in}}%
\pgfpathlineto{\pgfqpoint{4.551848in}{2.208893in}}%
\pgfpathlineto{\pgfqpoint{4.559796in}{2.312950in}}%
\pgfpathlineto{\pgfqpoint{4.567744in}{2.437233in}}%
\pgfpathlineto{\pgfqpoint{4.575691in}{2.377903in}}%
\pgfpathlineto{\pgfqpoint{4.583639in}{2.268038in}}%
\pgfpathlineto{\pgfqpoint{4.591587in}{2.404832in}}%
\pgfpathlineto{\pgfqpoint{4.599535in}{2.315827in}}%
\pgfpathlineto{\pgfqpoint{4.607483in}{2.107418in}}%
\pgfpathlineto{\pgfqpoint{4.615430in}{2.230155in}}%
\pgfpathlineto{\pgfqpoint{4.623378in}{2.285177in}}%
\pgfpathlineto{\pgfqpoint{4.631326in}{2.295505in}}%
\pgfpathlineto{\pgfqpoint{4.639274in}{2.108851in}}%
\pgfpathlineto{\pgfqpoint{4.647221in}{2.420156in}}%
\pgfpathlineto{\pgfqpoint{4.655169in}{1.975774in}}%
\pgfpathlineto{\pgfqpoint{4.663117in}{2.225517in}}%
\pgfpathlineto{\pgfqpoint{4.671065in}{2.136999in}}%
\pgfpathlineto{\pgfqpoint{4.679012in}{2.278093in}}%
\pgfpathlineto{\pgfqpoint{4.686960in}{2.046367in}}%
\pgfpathlineto{\pgfqpoint{4.694908in}{2.499172in}}%
\pgfpathlineto{\pgfqpoint{4.702856in}{2.713221in}}%
\pgfpathlineto{\pgfqpoint{4.710803in}{2.159838in}}%
\pgfpathlineto{\pgfqpoint{4.718751in}{2.327045in}}%
\pgfpathlineto{\pgfqpoint{4.726699in}{2.148775in}}%
\pgfpathlineto{\pgfqpoint{4.734647in}{2.257079in}}%
\pgfpathlineto{\pgfqpoint{4.742594in}{2.390335in}}%
\pgfpathlineto{\pgfqpoint{4.750542in}{2.393251in}}%
\pgfpathlineto{\pgfqpoint{4.758490in}{2.271265in}}%
\pgfpathlineto{\pgfqpoint{4.766438in}{2.403111in}}%
\pgfpathlineto{\pgfqpoint{4.774385in}{2.354220in}}%
\pgfpathlineto{\pgfqpoint{4.782333in}{2.292068in}}%
\pgfpathlineto{\pgfqpoint{4.790281in}{2.377968in}}%
\pgfpathlineto{\pgfqpoint{4.798229in}{2.360461in}}%
\pgfpathlineto{\pgfqpoint{4.806176in}{2.288592in}}%
\pgfpathlineto{\pgfqpoint{4.814124in}{2.489003in}}%
\pgfpathlineto{\pgfqpoint{4.822072in}{2.210686in}}%
\pgfpathlineto{\pgfqpoint{4.830020in}{3.056805in}}%
\pgfpathlineto{\pgfqpoint{4.837968in}{2.424914in}}%
\pgfpathlineto{\pgfqpoint{4.845915in}{2.609529in}}%
\pgfpathlineto{\pgfqpoint{4.853863in}{2.304277in}}%
\pgfpathlineto{\pgfqpoint{4.861811in}{2.307437in}}%
\pgfpathlineto{\pgfqpoint{4.869759in}{2.481243in}}%
\pgfpathlineto{\pgfqpoint{4.877706in}{2.111345in}}%
\pgfpathlineto{\pgfqpoint{4.885654in}{2.402287in}}%
\pgfpathlineto{\pgfqpoint{4.893602in}{2.401591in}}%
\pgfpathlineto{\pgfqpoint{4.901550in}{2.285177in}}%
\pgfpathlineto{\pgfqpoint{4.909497in}{2.181381in}}%
\pgfpathlineto{\pgfqpoint{4.917445in}{2.300939in}}%
\pgfpathlineto{\pgfqpoint{4.925393in}{2.329243in}}%
\pgfpathlineto{\pgfqpoint{4.933341in}{2.435513in}}%
\pgfpathlineto{\pgfqpoint{4.941288in}{2.341257in}}%
\pgfpathlineto{\pgfqpoint{4.949236in}{2.222855in}}%
\pgfpathlineto{\pgfqpoint{4.957184in}{2.499352in}}%
\pgfpathlineto{\pgfqpoint{4.965132in}{2.207847in}}%
\pgfpathlineto{\pgfqpoint{4.973079in}{2.275877in}}%
\pgfpathlineto{\pgfqpoint{4.981027in}{2.272770in}}%
\pgfpathlineto{\pgfqpoint{4.988975in}{2.010216in}}%
\pgfpathlineto{\pgfqpoint{4.996923in}{2.326040in}}%
\pgfpathlineto{\pgfqpoint{5.004870in}{2.457122in}}%
\pgfpathlineto{\pgfqpoint{5.012818in}{2.141472in}}%
\pgfpathlineto{\pgfqpoint{5.020766in}{2.363408in}}%
\pgfpathlineto{\pgfqpoint{5.028714in}{2.335081in}}%
\pgfpathlineto{\pgfqpoint{5.036661in}{2.378406in}}%
\pgfpathlineto{\pgfqpoint{5.044609in}{2.322345in}}%
\pgfpathlineto{\pgfqpoint{5.052557in}{2.210770in}}%
\pgfpathlineto{\pgfqpoint{5.060505in}{2.356489in}}%
\pgfpathlineto{\pgfqpoint{5.068453in}{2.328457in}}%
\pgfpathlineto{\pgfqpoint{5.076400in}{2.238801in}}%
\pgfpathlineto{\pgfqpoint{5.084348in}{2.188984in}}%
\pgfpathlineto{\pgfqpoint{5.092296in}{2.437007in}}%
\pgfpathlineto{\pgfqpoint{5.100244in}{2.017914in}}%
\pgfpathlineto{\pgfqpoint{5.108191in}{2.238185in}}%
\pgfpathlineto{\pgfqpoint{5.116139in}{2.528309in}}%
\pgfpathlineto{\pgfqpoint{5.124087in}{2.402542in}}%
\pgfpathlineto{\pgfqpoint{5.132035in}{2.192579in}}%
\pgfpathlineto{\pgfqpoint{5.139982in}{2.300640in}}%
\pgfpathlineto{\pgfqpoint{5.147930in}{2.136221in}}%
\pgfpathlineto{\pgfqpoint{5.155878in}{2.468109in}}%
\pgfpathlineto{\pgfqpoint{5.163826in}{2.198574in}}%
\pgfpathlineto{\pgfqpoint{5.171773in}{2.285177in}}%
\pgfpathlineto{\pgfqpoint{5.187669in}{2.602341in}}%
\pgfpathlineto{\pgfqpoint{5.195617in}{2.075325in}}%
\pgfpathlineto{\pgfqpoint{5.203564in}{2.388863in}}%
\pgfpathlineto{\pgfqpoint{5.211512in}{2.208990in}}%
\pgfpathlineto{\pgfqpoint{5.219460in}{2.425132in}}%
\pgfpathlineto{\pgfqpoint{5.227408in}{2.151327in}}%
\pgfpathlineto{\pgfqpoint{5.235355in}{2.530877in}}%
\pgfpathlineto{\pgfqpoint{5.243303in}{2.390397in}}%
\pgfpathlineto{\pgfqpoint{5.251251in}{2.383870in}}%
\pgfpathlineto{\pgfqpoint{5.259199in}{2.180486in}}%
\pgfpathlineto{\pgfqpoint{5.267146in}{2.395832in}}%
\pgfpathlineto{\pgfqpoint{5.275094in}{2.030029in}}%
\pgfpathlineto{\pgfqpoint{5.283042in}{2.483579in}}%
\pgfpathlineto{\pgfqpoint{5.290990in}{2.664321in}}%
\pgfpathlineto{\pgfqpoint{5.298938in}{2.279309in}}%
\pgfpathlineto{\pgfqpoint{5.306885in}{2.556198in}}%
\pgfpathlineto{\pgfqpoint{5.314833in}{1.098052in}}%
\pgfpathlineto{\pgfqpoint{5.322781in}{2.149409in}}%
\pgfpathlineto{\pgfqpoint{5.330729in}{2.417869in}}%
\pgfpathlineto{\pgfqpoint{5.338676in}{2.429192in}}%
\pgfpathlineto{\pgfqpoint{5.346624in}{2.054859in}}%
\pgfpathlineto{\pgfqpoint{5.354572in}{2.340700in}}%
\pgfpathlineto{\pgfqpoint{5.362520in}{2.481329in}}%
\pgfpathlineto{\pgfqpoint{5.370467in}{2.394655in}}%
\pgfpathlineto{\pgfqpoint{5.378415in}{2.172648in}}%
\pgfpathlineto{\pgfqpoint{5.386363in}{2.333917in}}%
\pgfpathlineto{\pgfqpoint{5.394311in}{2.409523in}}%
\pgfpathlineto{\pgfqpoint{5.402258in}{2.227651in}}%
\pgfpathlineto{\pgfqpoint{5.410206in}{2.424160in}}%
\pgfpathlineto{\pgfqpoint{5.418154in}{2.242982in}}%
\pgfpathlineto{\pgfqpoint{5.426102in}{2.501218in}}%
\pgfpathlineto{\pgfqpoint{5.434049in}{2.581335in}}%
\pgfpathlineto{\pgfqpoint{5.441997in}{2.170207in}}%
\pgfpathlineto{\pgfqpoint{5.449945in}{2.240777in}}%
\pgfpathlineto{\pgfqpoint{5.457893in}{2.055662in}}%
\pgfpathlineto{\pgfqpoint{5.465840in}{2.213086in}}%
\pgfpathlineto{\pgfqpoint{5.473788in}{2.291194in}}%
\pgfpathlineto{\pgfqpoint{5.481736in}{2.255070in}}%
\pgfpathlineto{\pgfqpoint{5.489684in}{3.059091in}}%
\pgfpathlineto{\pgfqpoint{5.497632in}{2.040366in}}%
\pgfpathlineto{\pgfqpoint{5.505579in}{1.804222in}}%
\pgfpathlineto{\pgfqpoint{5.513527in}{2.333207in}}%
\pgfpathlineto{\pgfqpoint{5.521475in}{1.697796in}}%
\pgfpathlineto{\pgfqpoint{5.529423in}{2.515530in}}%
\pgfpathlineto{\pgfqpoint{5.537370in}{2.211758in}}%
\pgfpathlineto{\pgfqpoint{5.545318in}{2.103510in}}%
\pgfpathlineto{\pgfqpoint{5.553266in}{2.325350in}}%
\pgfpathlineto{\pgfqpoint{5.561214in}{2.521451in}}%
\pgfpathlineto{\pgfqpoint{5.569161in}{2.367356in}}%
\pgfpathlineto{\pgfqpoint{5.577109in}{2.297322in}}%
\pgfpathlineto{\pgfqpoint{5.585057in}{2.154217in}}%
\pgfpathlineto{\pgfqpoint{5.593005in}{2.491875in}}%
\pgfpathlineto{\pgfqpoint{5.600952in}{2.185139in}}%
\pgfpathlineto{\pgfqpoint{5.608900in}{2.517807in}}%
\pgfpathlineto{\pgfqpoint{5.616848in}{2.306188in}}%
\pgfpathlineto{\pgfqpoint{5.624796in}{1.927929in}}%
\pgfpathlineto{\pgfqpoint{5.632743in}{1.680144in}}%
\pgfpathlineto{\pgfqpoint{5.640691in}{2.844326in}}%
\pgfpathlineto{\pgfqpoint{5.648639in}{2.069622in}}%
\pgfpathlineto{\pgfqpoint{5.656587in}{2.313024in}}%
\pgfpathlineto{\pgfqpoint{5.664534in}{2.463694in}}%
\pgfpathlineto{\pgfqpoint{5.672482in}{2.355520in}}%
\pgfpathlineto{\pgfqpoint{5.680430in}{2.110383in}}%
\pgfpathlineto{\pgfqpoint{5.688378in}{2.545266in}}%
\pgfpathlineto{\pgfqpoint{5.696325in}{2.391277in}}%
\pgfpathlineto{\pgfqpoint{5.704273in}{2.318404in}}%
\pgfpathlineto{\pgfqpoint{5.712221in}{2.227748in}}%
\pgfpathlineto{\pgfqpoint{5.720169in}{2.390835in}}%
\pgfpathlineto{\pgfqpoint{5.728117in}{2.485861in}}%
\pgfpathlineto{\pgfqpoint{5.736064in}{2.341715in}}%
\pgfpathlineto{\pgfqpoint{5.744012in}{2.521458in}}%
\pgfpathlineto{\pgfqpoint{5.751960in}{2.472171in}}%
\pgfpathlineto{\pgfqpoint{5.759908in}{2.276451in}}%
\pgfpathlineto{\pgfqpoint{5.767855in}{2.153824in}}%
\pgfpathlineto{\pgfqpoint{5.775803in}{2.217699in}}%
\pgfpathlineto{\pgfqpoint{5.783751in}{2.335074in}}%
\pgfpathlineto{\pgfqpoint{5.791699in}{2.547240in}}%
\pgfpathlineto{\pgfqpoint{5.799646in}{2.247539in}}%
\pgfpathlineto{\pgfqpoint{5.807594in}{2.221317in}}%
\pgfpathlineto{\pgfqpoint{5.815542in}{2.118747in}}%
\pgfpathlineto{\pgfqpoint{5.823490in}{2.320332in}}%
\pgfpathlineto{\pgfqpoint{5.831437in}{2.279322in}}%
\pgfpathlineto{\pgfqpoint{5.839385in}{2.211842in}}%
\pgfpathlineto{\pgfqpoint{5.847333in}{2.370219in}}%
\pgfpathlineto{\pgfqpoint{5.855281in}{2.250032in}}%
\pgfpathlineto{\pgfqpoint{5.863228in}{2.610771in}}%
\pgfpathlineto{\pgfqpoint{5.871176in}{2.201389in}}%
\pgfpathlineto{\pgfqpoint{5.879124in}{2.311218in}}%
\pgfpathlineto{\pgfqpoint{5.887072in}{2.365974in}}%
\pgfpathlineto{\pgfqpoint{5.895019in}{2.339814in}}%
\pgfpathlineto{\pgfqpoint{5.902967in}{2.333934in}}%
\pgfpathlineto{\pgfqpoint{5.910915in}{2.158731in}}%
\pgfpathlineto{\pgfqpoint{5.918863in}{2.279410in}}%
\pgfpathlineto{\pgfqpoint{5.926810in}{2.296709in}}%
\pgfpathlineto{\pgfqpoint{5.934758in}{2.186935in}}%
\pgfpathlineto{\pgfqpoint{5.942706in}{2.095980in}}%
\pgfpathlineto{\pgfqpoint{5.950654in}{2.144280in}}%
\pgfpathlineto{\pgfqpoint{5.966549in}{2.270384in}}%
\pgfpathlineto{\pgfqpoint{5.974497in}{2.249628in}}%
\pgfpathlineto{\pgfqpoint{5.982445in}{2.379831in}}%
\pgfpathlineto{\pgfqpoint{5.990393in}{2.344104in}}%
\pgfpathlineto{\pgfqpoint{5.998340in}{1.922929in}}%
\pgfpathlineto{\pgfqpoint{6.006288in}{2.143815in}}%
\pgfpathlineto{\pgfqpoint{6.014236in}{1.999355in}}%
\pgfpathlineto{\pgfqpoint{6.022184in}{2.404269in}}%
\pgfpathlineto{\pgfqpoint{6.030131in}{2.430764in}}%
\pgfpathlineto{\pgfqpoint{6.038079in}{2.248881in}}%
\pgfpathlineto{\pgfqpoint{6.053975in}{2.193845in}}%
\pgfpathlineto{\pgfqpoint{6.061922in}{2.288228in}}%
\pgfpathlineto{\pgfqpoint{6.069870in}{2.539800in}}%
\pgfpathlineto{\pgfqpoint{6.077818in}{2.233912in}}%
\pgfpathlineto{\pgfqpoint{6.085766in}{2.366534in}}%
\pgfpathlineto{\pgfqpoint{6.093713in}{2.282169in}}%
\pgfpathlineto{\pgfqpoint{6.101661in}{2.479692in}}%
\pgfpathlineto{\pgfqpoint{6.109609in}{2.560166in}}%
\pgfpathlineto{\pgfqpoint{6.117557in}{2.202789in}}%
\pgfpathlineto{\pgfqpoint{6.125504in}{2.399836in}}%
\pgfpathlineto{\pgfqpoint{6.133452in}{2.241156in}}%
\pgfpathlineto{\pgfqpoint{6.141400in}{2.108100in}}%
\pgfpathlineto{\pgfqpoint{6.149348in}{2.258476in}}%
\pgfpathlineto{\pgfqpoint{6.157295in}{2.121213in}}%
\pgfpathlineto{\pgfqpoint{6.165243in}{2.059355in}}%
\pgfpathlineto{\pgfqpoint{6.173191in}{2.366773in}}%
\pgfpathlineto{\pgfqpoint{6.181139in}{2.072926in}}%
\pgfpathlineto{\pgfqpoint{6.189087in}{1.872338in}}%
\pgfpathlineto{\pgfqpoint{6.197034in}{2.430971in}}%
\pgfpathlineto{\pgfqpoint{6.204982in}{2.226369in}}%
\pgfpathlineto{\pgfqpoint{6.212930in}{2.260363in}}%
\pgfpathlineto{\pgfqpoint{6.220878in}{2.331677in}}%
\pgfpathlineto{\pgfqpoint{6.228825in}{2.188954in}}%
\pgfpathlineto{\pgfqpoint{6.236773in}{2.461740in}}%
\pgfpathlineto{\pgfqpoint{6.244721in}{2.444875in}}%
\pgfpathlineto{\pgfqpoint{6.252669in}{2.522309in}}%
\pgfpathlineto{\pgfqpoint{6.260616in}{2.432690in}}%
\pgfpathlineto{\pgfqpoint{6.268564in}{2.351047in}}%
\pgfpathlineto{\pgfqpoint{6.276512in}{1.999134in}}%
\pgfpathlineto{\pgfqpoint{6.284460in}{2.621968in}}%
\pgfpathlineto{\pgfqpoint{6.292407in}{2.554619in}}%
\pgfpathlineto{\pgfqpoint{6.300355in}{2.282236in}}%
\pgfpathlineto{\pgfqpoint{6.308303in}{2.176054in}}%
\pgfpathlineto{\pgfqpoint{6.316251in}{2.631089in}}%
\pgfpathlineto{\pgfqpoint{6.332146in}{2.212882in}}%
\pgfpathlineto{\pgfqpoint{6.340094in}{2.136863in}}%
\pgfpathlineto{\pgfqpoint{6.348042in}{2.600808in}}%
\pgfpathlineto{\pgfqpoint{6.355989in}{2.561845in}}%
\pgfpathlineto{\pgfqpoint{6.363937in}{2.060145in}}%
\pgfpathlineto{\pgfqpoint{6.371885in}{2.098065in}}%
\pgfpathlineto{\pgfqpoint{6.379833in}{2.383273in}}%
\pgfpathlineto{\pgfqpoint{6.387781in}{2.328299in}}%
\pgfpathlineto{\pgfqpoint{6.395728in}{2.308136in}}%
\pgfpathlineto{\pgfqpoint{6.403676in}{2.615104in}}%
\pgfpathlineto{\pgfqpoint{6.411624in}{2.169198in}}%
\pgfpathlineto{\pgfqpoint{6.419572in}{2.330518in}}%
\pgfpathlineto{\pgfqpoint{6.427519in}{2.225644in}}%
\pgfpathlineto{\pgfqpoint{6.435467in}{2.412525in}}%
\pgfpathlineto{\pgfqpoint{6.443415in}{2.450900in}}%
\pgfpathlineto{\pgfqpoint{6.451363in}{2.477226in}}%
\pgfpathlineto{\pgfqpoint{6.459310in}{2.403914in}}%
\pgfpathlineto{\pgfqpoint{6.467258in}{2.224517in}}%
\pgfpathlineto{\pgfqpoint{6.475206in}{2.229868in}}%
\pgfpathlineto{\pgfqpoint{6.483154in}{2.368081in}}%
\pgfpathlineto{\pgfqpoint{6.491101in}{2.279661in}}%
\pgfpathlineto{\pgfqpoint{6.499049in}{2.351267in}}%
\pgfpathlineto{\pgfqpoint{6.506997in}{2.389366in}}%
\pgfpathlineto{\pgfqpoint{6.514945in}{2.109379in}}%
\pgfpathlineto{\pgfqpoint{6.522892in}{2.199454in}}%
\pgfpathlineto{\pgfqpoint{6.530840in}{2.271315in}}%
\pgfpathlineto{\pgfqpoint{6.538788in}{2.081675in}}%
\pgfpathlineto{\pgfqpoint{6.546736in}{2.399732in}}%
\pgfpathlineto{\pgfqpoint{6.554683in}{2.170621in}}%
\pgfpathlineto{\pgfqpoint{6.562631in}{2.130445in}}%
\pgfpathlineto{\pgfqpoint{6.570579in}{2.715382in}}%
\pgfpathlineto{\pgfqpoint{6.578527in}{2.343116in}}%
\pgfpathlineto{\pgfqpoint{6.586474in}{2.024486in}}%
\pgfpathlineto{\pgfqpoint{6.594422in}{2.371533in}}%
\pgfpathlineto{\pgfqpoint{6.602370in}{1.877937in}}%
\pgfpathlineto{\pgfqpoint{6.610318in}{2.302197in}}%
\pgfpathlineto{\pgfqpoint{6.618266in}{2.122883in}}%
\pgfpathlineto{\pgfqpoint{6.626213in}{2.098449in}}%
\pgfpathlineto{\pgfqpoint{6.634161in}{2.637032in}}%
\pgfpathlineto{\pgfqpoint{6.642109in}{2.171444in}}%
\pgfpathlineto{\pgfqpoint{6.650057in}{2.270913in}}%
\pgfpathlineto{\pgfqpoint{6.658004in}{2.156339in}}%
\pgfpathlineto{\pgfqpoint{6.665952in}{2.285177in}}%
\pgfpathlineto{\pgfqpoint{6.673900in}{2.394028in}}%
\pgfpathlineto{\pgfqpoint{6.681848in}{2.345073in}}%
\pgfpathlineto{\pgfqpoint{6.689795in}{2.282329in}}%
\pgfpathlineto{\pgfqpoint{6.697743in}{2.145112in}}%
\pgfpathlineto{\pgfqpoint{6.705691in}{2.074634in}}%
\pgfpathlineto{\pgfqpoint{6.713639in}{2.261965in}}%
\pgfpathlineto{\pgfqpoint{6.721586in}{2.331572in}}%
\pgfpathlineto{\pgfqpoint{6.729534in}{2.475404in}}%
\pgfpathlineto{\pgfqpoint{6.737482in}{1.999138in}}%
\pgfpathlineto{\pgfqpoint{6.745430in}{2.328785in}}%
\pgfpathlineto{\pgfqpoint{6.753377in}{2.256115in}}%
\pgfpathlineto{\pgfqpoint{6.761325in}{2.212333in}}%
\pgfpathlineto{\pgfqpoint{6.769273in}{2.494236in}}%
\pgfpathlineto{\pgfqpoint{6.777221in}{2.267840in}}%
\pgfpathlineto{\pgfqpoint{6.785168in}{2.227277in}}%
\pgfpathlineto{\pgfqpoint{6.793116in}{2.119217in}}%
\pgfpathlineto{\pgfqpoint{6.801064in}{2.361052in}}%
\pgfpathlineto{\pgfqpoint{6.816959in}{2.037863in}}%
\pgfpathlineto{\pgfqpoint{6.824907in}{2.520783in}}%
\pgfpathlineto{\pgfqpoint{6.832855in}{2.108738in}}%
\pgfpathlineto{\pgfqpoint{6.840803in}{2.444043in}}%
\pgfpathlineto{\pgfqpoint{6.848751in}{1.918436in}}%
\pgfpathlineto{\pgfqpoint{6.856698in}{2.090155in}}%
\pgfpathlineto{\pgfqpoint{6.864646in}{2.221752in}}%
\pgfpathlineto{\pgfqpoint{6.872594in}{2.047786in}}%
\pgfpathlineto{\pgfqpoint{6.880542in}{2.385962in}}%
\pgfpathlineto{\pgfqpoint{6.888489in}{2.376352in}}%
\pgfpathlineto{\pgfqpoint{6.896437in}{2.400060in}}%
\pgfpathlineto{\pgfqpoint{6.904385in}{2.390397in}}%
\pgfpathlineto{\pgfqpoint{6.912333in}{2.467221in}}%
\pgfpathlineto{\pgfqpoint{6.920280in}{2.486137in}}%
\pgfpathlineto{\pgfqpoint{6.928228in}{1.947094in}}%
\pgfpathlineto{\pgfqpoint{6.936176in}{2.362798in}}%
\pgfpathlineto{\pgfqpoint{6.944124in}{2.686260in}}%
\pgfpathlineto{\pgfqpoint{6.952071in}{1.928735in}}%
\pgfpathlineto{\pgfqpoint{6.960019in}{2.326751in}}%
\pgfpathlineto{\pgfqpoint{6.967967in}{2.228733in}}%
\pgfpathlineto{\pgfqpoint{6.975915in}{2.300046in}}%
\pgfpathlineto{\pgfqpoint{6.983862in}{2.255426in}}%
\pgfpathlineto{\pgfqpoint{6.991810in}{2.350567in}}%
\pgfpathlineto{\pgfqpoint{6.999758in}{2.219786in}}%
\pgfpathlineto{\pgfqpoint{7.007706in}{2.368364in}}%
\pgfpathlineto{\pgfqpoint{7.015653in}{2.403406in}}%
\pgfpathlineto{\pgfqpoint{7.023601in}{1.988255in}}%
\pgfpathlineto{\pgfqpoint{7.031549in}{2.528983in}}%
\pgfpathlineto{\pgfqpoint{7.039497in}{2.127942in}}%
\pgfpathlineto{\pgfqpoint{7.047445in}{2.344659in}}%
\pgfpathlineto{\pgfqpoint{7.055392in}{2.365192in}}%
\pgfpathlineto{\pgfqpoint{7.063340in}{2.323586in}}%
\pgfpathlineto{\pgfqpoint{7.071288in}{2.388215in}}%
\pgfpathlineto{\pgfqpoint{7.079236in}{2.358445in}}%
\pgfpathlineto{\pgfqpoint{7.087183in}{2.273472in}}%
\pgfpathlineto{\pgfqpoint{7.095131in}{2.232420in}}%
\pgfpathlineto{\pgfqpoint{7.103079in}{2.498261in}}%
\pgfpathlineto{\pgfqpoint{7.118974in}{2.039562in}}%
\pgfpathlineto{\pgfqpoint{7.126922in}{2.170086in}}%
\pgfpathlineto{\pgfqpoint{7.134870in}{2.175359in}}%
\pgfpathlineto{\pgfqpoint{7.142818in}{2.639124in}}%
\pgfpathlineto{\pgfqpoint{7.150765in}{2.410467in}}%
\pgfpathlineto{\pgfqpoint{7.158713in}{2.209515in}}%
\pgfpathlineto{\pgfqpoint{7.166661in}{2.323044in}}%
\pgfpathlineto{\pgfqpoint{7.174609in}{2.311350in}}%
\pgfpathlineto{\pgfqpoint{7.182556in}{2.490435in}}%
\pgfpathlineto{\pgfqpoint{7.190504in}{2.221805in}}%
\pgfpathlineto{\pgfqpoint{7.198452in}{2.233174in}}%
\pgfpathlineto{\pgfqpoint{7.206400in}{2.279390in}}%
\pgfpathlineto{\pgfqpoint{7.214347in}{2.203985in}}%
\pgfpathlineto{\pgfqpoint{7.222295in}{2.461629in}}%
\pgfpathlineto{\pgfqpoint{7.230243in}{2.491423in}}%
\pgfpathlineto{\pgfqpoint{7.238191in}{2.319341in}}%
\pgfpathlineto{\pgfqpoint{7.246138in}{2.188223in}}%
\pgfpathlineto{\pgfqpoint{7.254086in}{2.204970in}}%
\pgfpathlineto{\pgfqpoint{7.262034in}{2.299523in}}%
\pgfpathlineto{\pgfqpoint{7.269982in}{2.290912in}}%
\pgfpathlineto{\pgfqpoint{7.277930in}{2.482038in}}%
\pgfpathlineto{\pgfqpoint{7.285877in}{2.310709in}}%
\pgfpathlineto{\pgfqpoint{7.293825in}{2.236921in}}%
\pgfpathlineto{\pgfqpoint{7.301773in}{2.211143in}}%
\pgfpathlineto{\pgfqpoint{7.309721in}{2.225175in}}%
\pgfpathlineto{\pgfqpoint{7.317668in}{2.475970in}}%
\pgfpathlineto{\pgfqpoint{7.325616in}{2.347422in}}%
\pgfpathlineto{\pgfqpoint{7.333564in}{2.651780in}}%
\pgfpathlineto{\pgfqpoint{7.341512in}{2.165596in}}%
\pgfpathlineto{\pgfqpoint{7.349459in}{2.310266in}}%
\pgfpathlineto{\pgfqpoint{7.357407in}{2.237759in}}%
\pgfpathlineto{\pgfqpoint{7.365355in}{2.184383in}}%
\pgfpathlineto{\pgfqpoint{7.373303in}{2.231770in}}%
\pgfpathlineto{\pgfqpoint{7.381250in}{2.146768in}}%
\pgfpathlineto{\pgfqpoint{7.389198in}{2.516251in}}%
\pgfpathlineto{\pgfqpoint{7.397146in}{2.301978in}}%
\pgfpathlineto{\pgfqpoint{7.405094in}{2.341077in}}%
\pgfpathlineto{\pgfqpoint{7.413041in}{2.128251in}}%
\pgfpathlineto{\pgfqpoint{7.420989in}{2.254204in}}%
\pgfpathlineto{\pgfqpoint{7.428937in}{2.180634in}}%
\pgfpathlineto{\pgfqpoint{7.436885in}{2.128737in}}%
\pgfpathlineto{\pgfqpoint{7.444832in}{2.236569in}}%
\pgfpathlineto{\pgfqpoint{7.452780in}{2.187598in}}%
\pgfpathlineto{\pgfqpoint{7.460728in}{2.359840in}}%
\pgfpathlineto{\pgfqpoint{7.468676in}{2.348131in}}%
\pgfpathlineto{\pgfqpoint{7.476623in}{2.325134in}}%
\pgfpathlineto{\pgfqpoint{7.492519in}{2.179213in}}%
\pgfpathlineto{\pgfqpoint{7.500467in}{2.290919in}}%
\pgfpathlineto{\pgfqpoint{7.508415in}{2.356816in}}%
\pgfpathlineto{\pgfqpoint{7.516362in}{2.379341in}}%
\pgfpathlineto{\pgfqpoint{7.524310in}{2.182430in}}%
\pgfpathlineto{\pgfqpoint{7.532258in}{2.179011in}}%
\pgfpathlineto{\pgfqpoint{7.540206in}{2.267906in}}%
\pgfpathlineto{\pgfqpoint{7.548153in}{2.348427in}}%
\pgfpathlineto{\pgfqpoint{7.556101in}{2.265074in}}%
\pgfpathlineto{\pgfqpoint{7.564049in}{2.299538in}}%
\pgfpathlineto{\pgfqpoint{7.571997in}{2.299527in}}%
\pgfpathlineto{\pgfqpoint{7.579944in}{2.262210in}}%
\pgfpathlineto{\pgfqpoint{7.587892in}{2.213232in}}%
\pgfpathlineto{\pgfqpoint{7.603788in}{2.296722in}}%
\pgfpathlineto{\pgfqpoint{7.611735in}{2.256299in}}%
\pgfpathlineto{\pgfqpoint{7.619683in}{2.337125in}}%
\pgfpathlineto{\pgfqpoint{7.627631in}{2.397262in}}%
\pgfpathlineto{\pgfqpoint{7.635579in}{2.207648in}}%
\pgfpathlineto{\pgfqpoint{7.651474in}{2.160424in}}%
\pgfpathlineto{\pgfqpoint{7.659422in}{2.360703in}}%
\pgfpathlineto{\pgfqpoint{7.667370in}{2.113368in}}%
\pgfpathlineto{\pgfqpoint{7.675317in}{2.384369in}}%
\pgfpathlineto{\pgfqpoint{7.683265in}{2.177208in}}%
\pgfpathlineto{\pgfqpoint{7.691213in}{2.457062in}}%
\pgfpathlineto{\pgfqpoint{7.699161in}{2.354667in}}%
\pgfpathlineto{\pgfqpoint{7.707108in}{2.386077in}}%
\pgfpathlineto{\pgfqpoint{7.715056in}{2.336866in}}%
\pgfpathlineto{\pgfqpoint{7.723004in}{2.376734in}}%
\pgfpathlineto{\pgfqpoint{7.730952in}{2.551678in}}%
\pgfpathlineto{\pgfqpoint{7.738900in}{2.271092in}}%
\pgfpathlineto{\pgfqpoint{7.746847in}{2.177801in}}%
\pgfpathlineto{\pgfqpoint{7.754795in}{2.324805in}}%
\pgfpathlineto{\pgfqpoint{7.762743in}{2.271033in}}%
\pgfpathlineto{\pgfqpoint{7.770691in}{2.361431in}}%
\pgfpathlineto{\pgfqpoint{7.778638in}{2.155083in}}%
\pgfpathlineto{\pgfqpoint{7.786586in}{2.322030in}}%
\pgfpathlineto{\pgfqpoint{7.794534in}{2.219927in}}%
\pgfpathlineto{\pgfqpoint{7.810429in}{2.259525in}}%
\pgfpathlineto{\pgfqpoint{7.818377in}{2.350663in}}%
\pgfpathlineto{\pgfqpoint{7.826325in}{2.113881in}}%
\pgfpathlineto{\pgfqpoint{7.834273in}{2.210481in}}%
\pgfpathlineto{\pgfqpoint{7.842220in}{2.134925in}}%
\pgfpathlineto{\pgfqpoint{7.850168in}{2.162972in}}%
\pgfpathlineto{\pgfqpoint{7.858116in}{2.073893in}}%
\pgfpathlineto{\pgfqpoint{7.866064in}{2.866538in}}%
\pgfpathlineto{\pgfqpoint{7.874011in}{2.178917in}}%
\pgfpathlineto{\pgfqpoint{7.881959in}{2.331197in}}%
\pgfpathlineto{\pgfqpoint{7.889907in}{2.694568in}}%
\pgfpathlineto{\pgfqpoint{7.897855in}{2.055948in}}%
\pgfpathlineto{\pgfqpoint{7.905802in}{2.480614in}}%
\pgfpathlineto{\pgfqpoint{7.913750in}{2.414394in}}%
\pgfpathlineto{\pgfqpoint{7.921698in}{2.271173in}}%
\pgfpathlineto{\pgfqpoint{7.929646in}{2.299181in}}%
\pgfpathlineto{\pgfqpoint{7.937594in}{2.396850in}}%
\pgfpathlineto{\pgfqpoint{7.945541in}{2.243374in}}%
\pgfpathlineto{\pgfqpoint{7.953489in}{2.443556in}}%
\pgfpathlineto{\pgfqpoint{7.961437in}{2.345965in}}%
\pgfpathlineto{\pgfqpoint{7.969385in}{2.312745in}}%
\pgfpathlineto{\pgfqpoint{7.977332in}{2.395065in}}%
\pgfpathlineto{\pgfqpoint{7.985280in}{2.312553in}}%
\pgfpathlineto{\pgfqpoint{7.993228in}{2.320708in}}%
\pgfpathlineto{\pgfqpoint{8.001176in}{2.246909in}}%
\pgfpathlineto{\pgfqpoint{8.009123in}{2.268753in}}%
\pgfpathlineto{\pgfqpoint{8.017071in}{2.219345in}}%
\pgfpathlineto{\pgfqpoint{8.025019in}{2.304400in}}%
\pgfpathlineto{\pgfqpoint{8.032967in}{2.252210in}}%
\pgfpathlineto{\pgfqpoint{8.040914in}{2.268673in}}%
\pgfpathlineto{\pgfqpoint{8.048862in}{2.471413in}}%
\pgfpathlineto{\pgfqpoint{8.056810in}{2.301525in}}%
\pgfpathlineto{\pgfqpoint{8.064758in}{2.181409in}}%
\pgfpathlineto{\pgfqpoint{8.072705in}{2.205614in}}%
\pgfpathlineto{\pgfqpoint{8.080653in}{2.438526in}}%
\pgfpathlineto{\pgfqpoint{8.088601in}{2.249685in}}%
\pgfpathlineto{\pgfqpoint{8.096549in}{2.230448in}}%
\pgfpathlineto{\pgfqpoint{8.104496in}{2.429876in}}%
\pgfpathlineto{\pgfqpoint{8.112444in}{2.334078in}}%
\pgfpathlineto{\pgfqpoint{8.120392in}{2.304161in}}%
\pgfpathlineto{\pgfqpoint{8.128340in}{2.290597in}}%
\pgfpathlineto{\pgfqpoint{8.136287in}{2.350108in}}%
\pgfpathlineto{\pgfqpoint{8.144235in}{2.366039in}}%
\pgfpathlineto{\pgfqpoint{8.152183in}{2.198912in}}%
\pgfpathlineto{\pgfqpoint{8.160131in}{2.468038in}}%
\pgfpathlineto{\pgfqpoint{8.168079in}{2.075279in}}%
\pgfpathlineto{\pgfqpoint{8.176026in}{2.344606in}}%
\pgfpathlineto{\pgfqpoint{8.183974in}{2.252783in}}%
\pgfpathlineto{\pgfqpoint{8.191922in}{2.122398in}}%
\pgfpathlineto{\pgfqpoint{8.199870in}{2.143000in}}%
\pgfpathlineto{\pgfqpoint{8.207817in}{2.263211in}}%
\pgfpathlineto{\pgfqpoint{8.215765in}{2.460213in}}%
\pgfpathlineto{\pgfqpoint{8.223713in}{2.233376in}}%
\pgfpathlineto{\pgfqpoint{8.231661in}{2.189394in}}%
\pgfpathlineto{\pgfqpoint{8.239608in}{2.522406in}}%
\pgfpathlineto{\pgfqpoint{8.247556in}{2.165558in}}%
\pgfpathlineto{\pgfqpoint{8.255504in}{2.523688in}}%
\pgfpathlineto{\pgfqpoint{8.263452in}{1.978373in}}%
\pgfpathlineto{\pgfqpoint{8.271399in}{2.326180in}}%
\pgfpathlineto{\pgfqpoint{8.279347in}{2.145412in}}%
\pgfpathlineto{\pgfqpoint{8.287295in}{2.116687in}}%
\pgfpathlineto{\pgfqpoint{8.295243in}{2.527784in}}%
\pgfpathlineto{\pgfqpoint{8.303190in}{2.497731in}}%
\pgfpathlineto{\pgfqpoint{8.311138in}{2.385203in}}%
\pgfpathlineto{\pgfqpoint{8.319086in}{2.347099in}}%
\pgfpathlineto{\pgfqpoint{8.327034in}{2.052633in}}%
\pgfpathlineto{\pgfqpoint{8.334981in}{2.734248in}}%
\pgfpathlineto{\pgfqpoint{8.342929in}{2.465123in}}%
\pgfpathlineto{\pgfqpoint{8.350877in}{2.403356in}}%
\pgfpathlineto{\pgfqpoint{8.358825in}{2.300880in}}%
\pgfpathlineto{\pgfqpoint{8.366772in}{2.376530in}}%
\pgfpathlineto{\pgfqpoint{8.374720in}{2.357954in}}%
\pgfpathlineto{\pgfqpoint{8.382668in}{2.311103in}}%
\pgfpathlineto{\pgfqpoint{8.390616in}{2.565573in}}%
\pgfpathlineto{\pgfqpoint{8.398564in}{2.364195in}}%
\pgfpathlineto{\pgfqpoint{8.406511in}{2.262269in}}%
\pgfpathlineto{\pgfqpoint{8.414459in}{2.336047in}}%
\pgfpathlineto{\pgfqpoint{8.422407in}{2.356174in}}%
\pgfpathlineto{\pgfqpoint{8.430355in}{2.333207in}}%
\pgfpathlineto{\pgfqpoint{8.438302in}{2.242208in}}%
\pgfpathlineto{\pgfqpoint{8.446250in}{2.091919in}}%
\pgfpathlineto{\pgfqpoint{8.454198in}{2.177553in}}%
\pgfpathlineto{\pgfqpoint{8.462146in}{2.225988in}}%
\pgfpathlineto{\pgfqpoint{8.470093in}{2.308359in}}%
\pgfpathlineto{\pgfqpoint{8.478041in}{2.459449in}}%
\pgfpathlineto{\pgfqpoint{8.485989in}{2.226405in}}%
\pgfpathlineto{\pgfqpoint{8.493937in}{2.313307in}}%
\pgfpathlineto{\pgfqpoint{8.501884in}{2.177553in}}%
\pgfpathlineto{\pgfqpoint{8.509832in}{2.438745in}}%
\pgfpathlineto{\pgfqpoint{8.517780in}{2.303015in}}%
\pgfpathlineto{\pgfqpoint{8.525728in}{2.623554in}}%
\pgfpathlineto{\pgfqpoint{8.533675in}{2.207413in}}%
\pgfpathlineto{\pgfqpoint{8.541623in}{2.500152in}}%
\pgfpathlineto{\pgfqpoint{8.549571in}{2.052599in}}%
\pgfpathlineto{\pgfqpoint{8.557519in}{2.413061in}}%
\pgfpathlineto{\pgfqpoint{8.565466in}{2.240136in}}%
\pgfpathlineto{\pgfqpoint{8.573414in}{2.367672in}}%
\pgfpathlineto{\pgfqpoint{8.581362in}{2.275196in}}%
\pgfpathlineto{\pgfqpoint{8.589310in}{2.129819in}}%
\pgfpathlineto{\pgfqpoint{8.597257in}{2.375534in}}%
\pgfpathlineto{\pgfqpoint{8.605205in}{2.096459in}}%
\pgfpathlineto{\pgfqpoint{8.621101in}{2.320680in}}%
\pgfpathlineto{\pgfqpoint{8.629049in}{2.576895in}}%
\pgfpathlineto{\pgfqpoint{8.636996in}{2.562772in}}%
\pgfpathlineto{\pgfqpoint{8.644944in}{2.297478in}}%
\pgfpathlineto{\pgfqpoint{8.652892in}{2.400434in}}%
\pgfpathlineto{\pgfqpoint{8.660840in}{2.431335in}}%
\pgfpathlineto{\pgfqpoint{8.668787in}{2.253601in}}%
\pgfpathlineto{\pgfqpoint{8.676735in}{2.353122in}}%
\pgfpathlineto{\pgfqpoint{8.684683in}{2.112427in}}%
\pgfpathlineto{\pgfqpoint{8.692631in}{2.233788in}}%
\pgfpathlineto{\pgfqpoint{8.700578in}{2.006320in}}%
\pgfpathlineto{\pgfqpoint{8.708526in}{2.142993in}}%
\pgfpathlineto{\pgfqpoint{8.716474in}{2.192327in}}%
\pgfpathlineto{\pgfqpoint{8.724422in}{2.196939in}}%
\pgfpathlineto{\pgfqpoint{8.732369in}{2.385988in}}%
\pgfpathlineto{\pgfqpoint{8.740317in}{2.510122in}}%
\pgfpathlineto{\pgfqpoint{8.748265in}{2.100411in}}%
\pgfpathlineto{\pgfqpoint{8.756213in}{2.212290in}}%
\pgfpathlineto{\pgfqpoint{8.764160in}{2.368095in}}%
\pgfpathlineto{\pgfqpoint{8.772108in}{2.219876in}}%
\pgfpathlineto{\pgfqpoint{8.780056in}{2.254964in}}%
\pgfpathlineto{\pgfqpoint{8.788004in}{2.587702in}}%
\pgfpathlineto{\pgfqpoint{8.795951in}{2.270285in}}%
\pgfpathlineto{\pgfqpoint{8.803899in}{2.324863in}}%
\pgfpathlineto{\pgfqpoint{8.811847in}{2.477503in}}%
\pgfpathlineto{\pgfqpoint{8.819795in}{1.953596in}}%
\pgfpathlineto{\pgfqpoint{8.827743in}{2.175064in}}%
\pgfpathlineto{\pgfqpoint{8.835690in}{1.598183in}}%
\pgfpathlineto{\pgfqpoint{8.843638in}{1.978685in}}%
\pgfpathlineto{\pgfqpoint{8.851586in}{2.403626in}}%
\pgfpathlineto{\pgfqpoint{8.859534in}{1.967706in}}%
\pgfpathlineto{\pgfqpoint{8.867481in}{2.478917in}}%
\pgfpathlineto{\pgfqpoint{8.875429in}{2.474404in}}%
\pgfpathlineto{\pgfqpoint{8.883377in}{2.248525in}}%
\pgfpathlineto{\pgfqpoint{8.891325in}{2.491102in}}%
\pgfpathlineto{\pgfqpoint{8.899272in}{2.037279in}}%
\pgfpathlineto{\pgfqpoint{8.907220in}{2.156077in}}%
\pgfpathlineto{\pgfqpoint{8.915168in}{2.364318in}}%
\pgfpathlineto{\pgfqpoint{8.923116in}{2.343011in}}%
\pgfpathlineto{\pgfqpoint{8.931063in}{2.374223in}}%
\pgfpathlineto{\pgfqpoint{8.939011in}{2.256412in}}%
\pgfpathlineto{\pgfqpoint{8.946959in}{2.190741in}}%
\pgfpathlineto{\pgfqpoint{8.954907in}{2.460569in}}%
\pgfpathlineto{\pgfqpoint{8.962854in}{2.043935in}}%
\pgfpathlineto{\pgfqpoint{8.970802in}{2.208513in}}%
\pgfpathlineto{\pgfqpoint{8.978750in}{2.404009in}}%
\pgfpathlineto{\pgfqpoint{8.986698in}{2.311486in}}%
\pgfpathlineto{\pgfqpoint{8.994645in}{2.274657in}}%
\pgfpathlineto{\pgfqpoint{9.002593in}{2.219301in}}%
\pgfpathlineto{\pgfqpoint{9.010541in}{2.274616in}}%
\pgfpathlineto{\pgfqpoint{9.018489in}{2.351088in}}%
\pgfpathlineto{\pgfqpoint{9.034384in}{2.096846in}}%
\pgfpathlineto{\pgfqpoint{9.042332in}{2.351697in}}%
\pgfpathlineto{\pgfqpoint{9.050280in}{1.822881in}}%
\pgfpathlineto{\pgfqpoint{9.058228in}{2.388255in}}%
\pgfpathlineto{\pgfqpoint{9.066175in}{1.916491in}}%
\pgfpathlineto{\pgfqpoint{9.074123in}{2.249305in}}%
\pgfpathlineto{\pgfqpoint{9.082071in}{2.452933in}}%
\pgfpathlineto{\pgfqpoint{9.090019in}{2.222093in}}%
\pgfpathlineto{\pgfqpoint{9.097966in}{2.695672in}}%
\pgfpathlineto{\pgfqpoint{9.105914in}{2.117668in}}%
\pgfpathlineto{\pgfqpoint{9.113862in}{2.495681in}}%
\pgfpathlineto{\pgfqpoint{9.121810in}{2.199089in}}%
\pgfpathlineto{\pgfqpoint{9.129757in}{2.600766in}}%
\pgfpathlineto{\pgfqpoint{9.137705in}{2.314339in}}%
\pgfpathlineto{\pgfqpoint{9.145653in}{2.066736in}}%
\pgfpathlineto{\pgfqpoint{9.153601in}{2.072398in}}%
\pgfpathlineto{\pgfqpoint{9.161548in}{2.081010in}}%
\pgfpathlineto{\pgfqpoint{9.169496in}{2.323450in}}%
\pgfpathlineto{\pgfqpoint{9.177444in}{2.380533in}}%
\pgfpathlineto{\pgfqpoint{9.185392in}{2.099486in}}%
\pgfpathlineto{\pgfqpoint{9.193339in}{2.378242in}}%
\pgfpathlineto{\pgfqpoint{9.201287in}{2.309738in}}%
\pgfpathlineto{\pgfqpoint{9.209235in}{2.429193in}}%
\pgfpathlineto{\pgfqpoint{9.217183in}{2.132976in}}%
\pgfpathlineto{\pgfqpoint{9.225130in}{2.375009in}}%
\pgfpathlineto{\pgfqpoint{9.233078in}{2.312317in}}%
\pgfpathlineto{\pgfqpoint{9.241026in}{2.393365in}}%
\pgfpathlineto{\pgfqpoint{9.248974in}{2.304048in}}%
\pgfpathlineto{\pgfqpoint{9.256921in}{2.325555in}}%
\pgfpathlineto{\pgfqpoint{9.264869in}{2.263652in}}%
\pgfpathlineto{\pgfqpoint{9.272817in}{2.282484in}}%
\pgfpathlineto{\pgfqpoint{9.280765in}{2.057696in}}%
\pgfpathlineto{\pgfqpoint{9.288713in}{2.366723in}}%
\pgfpathlineto{\pgfqpoint{9.296660in}{2.333942in}}%
\pgfpathlineto{\pgfqpoint{9.304608in}{2.287882in}}%
\pgfpathlineto{\pgfqpoint{9.312556in}{1.930271in}}%
\pgfpathlineto{\pgfqpoint{9.320504in}{2.296193in}}%
\pgfpathlineto{\pgfqpoint{9.328451in}{2.188571in}}%
\pgfpathlineto{\pgfqpoint{9.336399in}{2.240855in}}%
\pgfpathlineto{\pgfqpoint{9.344347in}{2.146021in}}%
\pgfpathlineto{\pgfqpoint{9.352295in}{2.201206in}}%
\pgfpathlineto{\pgfqpoint{9.360242in}{2.477774in}}%
\pgfpathlineto{\pgfqpoint{9.368190in}{2.299060in}}%
\pgfpathlineto{\pgfqpoint{9.376138in}{2.218448in}}%
\pgfpathlineto{\pgfqpoint{9.384086in}{2.365778in}}%
\pgfpathlineto{\pgfqpoint{9.392033in}{2.235187in}}%
\pgfpathlineto{\pgfqpoint{9.399981in}{2.470606in}}%
\pgfpathlineto{\pgfqpoint{9.407929in}{2.116425in}}%
\pgfpathlineto{\pgfqpoint{9.415877in}{2.207227in}}%
\pgfpathlineto{\pgfqpoint{9.423824in}{2.223712in}}%
\pgfpathlineto{\pgfqpoint{9.431772in}{2.499463in}}%
\pgfpathlineto{\pgfqpoint{9.439720in}{2.401079in}}%
\pgfpathlineto{\pgfqpoint{9.447668in}{2.207984in}}%
\pgfpathlineto{\pgfqpoint{9.455615in}{2.307263in}}%
\pgfpathlineto{\pgfqpoint{9.463563in}{2.202225in}}%
\pgfpathlineto{\pgfqpoint{9.471511in}{2.268544in}}%
\pgfpathlineto{\pgfqpoint{9.479459in}{2.171142in}}%
\pgfpathlineto{\pgfqpoint{9.487406in}{2.086118in}}%
\pgfpathlineto{\pgfqpoint{9.495354in}{2.347073in}}%
\pgfpathlineto{\pgfqpoint{9.503302in}{2.271126in}}%
\pgfpathlineto{\pgfqpoint{9.511250in}{2.050457in}}%
\pgfpathlineto{\pgfqpoint{9.519198in}{2.398646in}}%
\pgfpathlineto{\pgfqpoint{9.527145in}{2.208657in}}%
\pgfpathlineto{\pgfqpoint{9.535093in}{2.290855in}}%
\pgfpathlineto{\pgfqpoint{9.543041in}{2.182713in}}%
\pgfpathlineto{\pgfqpoint{9.550989in}{2.104566in}}%
\pgfpathlineto{\pgfqpoint{9.558936in}{2.354177in}}%
\pgfpathlineto{\pgfqpoint{9.566884in}{2.584229in}}%
\pgfpathlineto{\pgfqpoint{9.574832in}{2.442824in}}%
\pgfpathlineto{\pgfqpoint{9.582780in}{2.254310in}}%
\pgfpathlineto{\pgfqpoint{9.590727in}{2.149915in}}%
\pgfpathlineto{\pgfqpoint{9.598675in}{2.097653in}}%
\pgfpathlineto{\pgfqpoint{9.606623in}{2.529173in}}%
\pgfpathlineto{\pgfqpoint{9.614571in}{2.528956in}}%
\pgfpathlineto{\pgfqpoint{9.622518in}{2.465350in}}%
\pgfpathlineto{\pgfqpoint{9.630466in}{2.216075in}}%
\pgfpathlineto{\pgfqpoint{9.638414in}{2.263012in}}%
\pgfpathlineto{\pgfqpoint{9.646362in}{2.123732in}}%
\pgfpathlineto{\pgfqpoint{9.654309in}{2.215176in}}%
\pgfpathlineto{\pgfqpoint{9.662257in}{2.257106in}}%
\pgfpathlineto{\pgfqpoint{9.670205in}{2.273937in}}%
\pgfpathlineto{\pgfqpoint{9.678153in}{2.030569in}}%
\pgfpathlineto{\pgfqpoint{9.686100in}{2.265234in}}%
\pgfpathlineto{\pgfqpoint{9.694048in}{2.652259in}}%
\pgfpathlineto{\pgfqpoint{9.701996in}{2.335466in}}%
\pgfpathlineto{\pgfqpoint{9.709944in}{2.315845in}}%
\pgfpathlineto{\pgfqpoint{9.717892in}{2.690479in}}%
\pgfpathlineto{\pgfqpoint{9.725839in}{2.106995in}}%
\pgfpathlineto{\pgfqpoint{9.733787in}{2.205154in}}%
\pgfpathlineto{\pgfqpoint{9.749683in}{2.515234in}}%
\pgfpathlineto{\pgfqpoint{9.757630in}{2.317822in}}%
\pgfpathlineto{\pgfqpoint{9.765578in}{2.309625in}}%
\pgfpathlineto{\pgfqpoint{9.773526in}{2.323147in}}%
\pgfpathlineto{\pgfqpoint{9.781474in}{2.268912in}}%
\pgfpathlineto{\pgfqpoint{9.789421in}{2.274326in}}%
\pgfpathlineto{\pgfqpoint{9.797369in}{2.151770in}}%
\pgfpathlineto{\pgfqpoint{9.805317in}{2.252367in}}%
\pgfpathlineto{\pgfqpoint{9.813265in}{2.252312in}}%
\pgfpathlineto{\pgfqpoint{9.821212in}{2.186249in}}%
\pgfpathlineto{\pgfqpoint{9.829160in}{2.205120in}}%
\pgfpathlineto{\pgfqpoint{9.837108in}{2.348696in}}%
\pgfpathlineto{\pgfqpoint{9.845056in}{2.381444in}}%
\pgfpathlineto{\pgfqpoint{9.853003in}{1.484317in}}%
\pgfpathlineto{\pgfqpoint{9.860951in}{2.110043in}}%
\pgfpathlineto{\pgfqpoint{9.868899in}{2.207156in}}%
\pgfpathlineto{\pgfqpoint{9.876847in}{2.160276in}}%
\pgfpathlineto{\pgfqpoint{9.884794in}{2.247259in}}%
\pgfpathlineto{\pgfqpoint{9.892742in}{2.279337in}}%
\pgfpathlineto{\pgfqpoint{9.900690in}{2.217891in}}%
\pgfpathlineto{\pgfqpoint{9.908638in}{2.515329in}}%
\pgfpathlineto{\pgfqpoint{9.916585in}{2.244587in}}%
\pgfpathlineto{\pgfqpoint{9.924533in}{2.328663in}}%
\pgfpathlineto{\pgfqpoint{9.932481in}{2.128173in}}%
\pgfpathlineto{\pgfqpoint{9.940429in}{2.244266in}}%
\pgfpathlineto{\pgfqpoint{9.948377in}{2.261761in}}%
\pgfpathlineto{\pgfqpoint{9.956324in}{2.182401in}}%
\pgfpathlineto{\pgfqpoint{9.964272in}{2.018393in}}%
\pgfpathlineto{\pgfqpoint{9.972220in}{2.231381in}}%
\pgfpathlineto{\pgfqpoint{9.980168in}{1.726623in}}%
\pgfpathlineto{\pgfqpoint{9.988115in}{1.535038in}}%
\pgfpathlineto{\pgfqpoint{9.996063in}{2.082562in}}%
\pgfpathlineto{\pgfqpoint{10.004011in}{2.980665in}}%
\pgfpathlineto{\pgfqpoint{10.011959in}{2.446750in}}%
\pgfpathlineto{\pgfqpoint{10.019906in}{2.210770in}}%
\pgfpathlineto{\pgfqpoint{10.027854in}{2.135510in}}%
\pgfpathlineto{\pgfqpoint{10.035802in}{1.787498in}}%
\pgfpathlineto{\pgfqpoint{10.043750in}{2.543519in}}%
\pgfpathlineto{\pgfqpoint{10.051697in}{2.316836in}}%
\pgfpathlineto{\pgfqpoint{10.059645in}{1.956610in}}%
\pgfpathlineto{\pgfqpoint{10.067593in}{2.629554in}}%
\pgfpathlineto{\pgfqpoint{10.075541in}{1.860212in}}%
\pgfpathlineto{\pgfqpoint{10.083488in}{2.239902in}}%
\pgfpathlineto{\pgfqpoint{10.091436in}{2.314293in}}%
\pgfpathlineto{\pgfqpoint{10.099384in}{2.181453in}}%
\pgfpathlineto{\pgfqpoint{10.107332in}{2.680898in}}%
\pgfpathlineto{\pgfqpoint{10.115279in}{2.465883in}}%
\pgfpathlineto{\pgfqpoint{10.123227in}{2.326158in}}%
\pgfpathlineto{\pgfqpoint{10.131175in}{2.199965in}}%
\pgfpathlineto{\pgfqpoint{10.139123in}{2.483429in}}%
\pgfpathlineto{\pgfqpoint{10.147070in}{2.156383in}}%
\pgfpathlineto{\pgfqpoint{10.155018in}{2.307226in}}%
\pgfpathlineto{\pgfqpoint{10.162966in}{2.588187in}}%
\pgfpathlineto{\pgfqpoint{10.170914in}{2.641491in}}%
\pgfpathlineto{\pgfqpoint{10.178862in}{2.040138in}}%
\pgfpathlineto{\pgfqpoint{10.186809in}{2.423388in}}%
\pgfpathlineto{\pgfqpoint{10.194757in}{2.193144in}}%
\pgfpathlineto{\pgfqpoint{10.202705in}{2.288251in}}%
\pgfpathlineto{\pgfqpoint{10.210653in}{2.413881in}}%
\pgfpathlineto{\pgfqpoint{10.218600in}{2.500844in}}%
\pgfpathlineto{\pgfqpoint{10.226548in}{2.333451in}}%
\pgfpathlineto{\pgfqpoint{10.234496in}{2.369372in}}%
\pgfpathlineto{\pgfqpoint{10.242444in}{2.464387in}}%
\pgfpathlineto{\pgfqpoint{10.250391in}{2.305978in}}%
\pgfpathlineto{\pgfqpoint{10.258339in}{2.246528in}}%
\pgfpathlineto{\pgfqpoint{10.266287in}{2.225568in}}%
\pgfpathlineto{\pgfqpoint{10.274235in}{2.312023in}}%
\pgfpathlineto{\pgfqpoint{10.282182in}{2.303054in}}%
\pgfpathlineto{\pgfqpoint{10.290130in}{2.448296in}}%
\pgfpathlineto{\pgfqpoint{10.298078in}{2.352989in}}%
\pgfpathlineto{\pgfqpoint{10.306026in}{2.095895in}}%
\pgfpathlineto{\pgfqpoint{10.313973in}{2.237569in}}%
\pgfpathlineto{\pgfqpoint{10.321921in}{2.618954in}}%
\pgfpathlineto{\pgfqpoint{10.329869in}{2.845271in}}%
\pgfpathlineto{\pgfqpoint{10.337817in}{2.401524in}}%
\pgfpathlineto{\pgfqpoint{10.345764in}{2.237019in}}%
\pgfpathlineto{\pgfqpoint{10.353712in}{2.085622in}}%
\pgfpathlineto{\pgfqpoint{10.361660in}{2.413695in}}%
\pgfpathlineto{\pgfqpoint{10.369608in}{2.122244in}}%
\pgfpathlineto{\pgfqpoint{10.377555in}{2.342500in}}%
\pgfpathlineto{\pgfqpoint{10.385503in}{2.399323in}}%
\pgfpathlineto{\pgfqpoint{10.393451in}{2.288022in}}%
\pgfpathlineto{\pgfqpoint{10.401399in}{2.113731in}}%
\pgfpathlineto{\pgfqpoint{10.409347in}{2.074540in}}%
\pgfpathlineto{\pgfqpoint{10.417294in}{2.241613in}}%
\pgfpathlineto{\pgfqpoint{10.425242in}{2.380888in}}%
\pgfpathlineto{\pgfqpoint{10.433190in}{2.334301in}}%
\pgfpathlineto{\pgfqpoint{10.441138in}{1.947502in}}%
\pgfpathlineto{\pgfqpoint{10.449085in}{2.099316in}}%
\pgfpathlineto{\pgfqpoint{10.457033in}{2.637691in}}%
\pgfpathlineto{\pgfqpoint{10.464981in}{2.223943in}}%
\pgfpathlineto{\pgfqpoint{10.472929in}{2.500107in}}%
\pgfpathlineto{\pgfqpoint{10.480876in}{2.368767in}}%
\pgfpathlineto{\pgfqpoint{10.488824in}{2.181356in}}%
\pgfpathlineto{\pgfqpoint{10.496772in}{2.325617in}}%
\pgfpathlineto{\pgfqpoint{10.504720in}{2.408867in}}%
\pgfpathlineto{\pgfqpoint{10.512667in}{2.141277in}}%
\pgfpathlineto{\pgfqpoint{10.520615in}{2.236009in}}%
\pgfpathlineto{\pgfqpoint{10.528563in}{2.060917in}}%
\pgfpathlineto{\pgfqpoint{10.536511in}{2.570155in}}%
\pgfpathlineto{\pgfqpoint{10.544458in}{2.247611in}}%
\pgfpathlineto{\pgfqpoint{10.552406in}{2.270709in}}%
\pgfpathlineto{\pgfqpoint{10.560354in}{2.819116in}}%
\pgfpathlineto{\pgfqpoint{10.568302in}{2.419915in}}%
\pgfpathlineto{\pgfqpoint{10.576249in}{2.130711in}}%
\pgfpathlineto{\pgfqpoint{10.584197in}{2.271074in}}%
\pgfpathlineto{\pgfqpoint{10.592145in}{2.307736in}}%
\pgfpathlineto{\pgfqpoint{10.600093in}{2.282358in}}%
\pgfpathlineto{\pgfqpoint{10.608041in}{2.414412in}}%
\pgfpathlineto{\pgfqpoint{10.615988in}{2.629362in}}%
\pgfpathlineto{\pgfqpoint{10.623936in}{2.604479in}}%
\pgfpathlineto{\pgfqpoint{10.631884in}{2.116622in}}%
\pgfpathlineto{\pgfqpoint{10.639832in}{1.751001in}}%
\pgfpathlineto{\pgfqpoint{10.647779in}{2.488939in}}%
\pgfpathlineto{\pgfqpoint{10.655727in}{2.445572in}}%
\pgfpathlineto{\pgfqpoint{10.663675in}{2.364885in}}%
\pgfpathlineto{\pgfqpoint{10.671623in}{2.252233in}}%
\pgfpathlineto{\pgfqpoint{10.679570in}{2.318120in}}%
\pgfpathlineto{\pgfqpoint{10.687518in}{2.391861in}}%
\pgfpathlineto{\pgfqpoint{10.695466in}{2.214118in}}%
\pgfpathlineto{\pgfqpoint{10.711361in}{2.026729in}}%
\pgfpathlineto{\pgfqpoint{10.719309in}{2.346623in}}%
\pgfpathlineto{\pgfqpoint{10.727257in}{2.094624in}}%
\pgfpathlineto{\pgfqpoint{10.735205in}{2.115489in}}%
\pgfpathlineto{\pgfqpoint{10.743152in}{1.975984in}}%
\pgfpathlineto{\pgfqpoint{10.751100in}{2.463272in}}%
\pgfpathlineto{\pgfqpoint{10.759048in}{2.245103in}}%
\pgfpathlineto{\pgfqpoint{10.766996in}{2.115387in}}%
\pgfpathlineto{\pgfqpoint{10.774943in}{2.363060in}}%
\pgfpathlineto{\pgfqpoint{10.782891in}{1.980543in}}%
\pgfpathlineto{\pgfqpoint{10.790839in}{2.738940in}}%
\pgfpathlineto{\pgfqpoint{10.798787in}{2.222222in}}%
\pgfpathlineto{\pgfqpoint{10.806734in}{2.513140in}}%
\pgfpathlineto{\pgfqpoint{10.814682in}{2.448807in}}%
\pgfpathlineto{\pgfqpoint{10.822630in}{2.155513in}}%
\pgfpathlineto{\pgfqpoint{10.830578in}{2.779464in}}%
\pgfpathlineto{\pgfqpoint{10.838526in}{2.282419in}}%
\pgfpathlineto{\pgfqpoint{10.846473in}{2.537286in}}%
\pgfpathlineto{\pgfqpoint{10.854421in}{2.737341in}}%
\pgfpathlineto{\pgfqpoint{10.862369in}{2.149013in}}%
\pgfpathlineto{\pgfqpoint{10.870317in}{2.066951in}}%
\pgfpathlineto{\pgfqpoint{10.878264in}{2.498043in}}%
\pgfpathlineto{\pgfqpoint{10.886212in}{2.188461in}}%
\pgfpathlineto{\pgfqpoint{10.902108in}{2.624823in}}%
\pgfpathlineto{\pgfqpoint{10.910055in}{2.290437in}}%
\pgfpathlineto{\pgfqpoint{10.918003in}{2.041719in}}%
\pgfpathlineto{\pgfqpoint{10.925951in}{1.868033in}}%
\pgfpathlineto{\pgfqpoint{10.933899in}{2.549950in}}%
\pgfpathlineto{\pgfqpoint{10.941846in}{2.400232in}}%
\pgfpathlineto{\pgfqpoint{10.949794in}{2.521190in}}%
\pgfpathlineto{\pgfqpoint{10.957742in}{2.171503in}}%
\pgfpathlineto{\pgfqpoint{10.965690in}{2.240053in}}%
\pgfpathlineto{\pgfqpoint{10.973637in}{2.367383in}}%
\pgfpathlineto{\pgfqpoint{10.981585in}{2.208285in}}%
\pgfpathlineto{\pgfqpoint{10.989533in}{2.226643in}}%
\pgfpathlineto{\pgfqpoint{10.997481in}{2.470817in}}%
\pgfpathlineto{\pgfqpoint{11.005428in}{1.987302in}}%
\pgfpathlineto{\pgfqpoint{11.013376in}{2.088559in}}%
\pgfpathlineto{\pgfqpoint{11.021324in}{2.513927in}}%
\pgfpathlineto{\pgfqpoint{11.029272in}{2.601004in}}%
\pgfpathlineto{\pgfqpoint{11.037219in}{2.353508in}}%
\pgfpathlineto{\pgfqpoint{11.045167in}{2.439358in}}%
\pgfpathlineto{\pgfqpoint{11.053115in}{2.191244in}}%
\pgfpathlineto{\pgfqpoint{11.061063in}{2.277329in}}%
\pgfpathlineto{\pgfqpoint{11.069011in}{2.261613in}}%
\pgfpathlineto{\pgfqpoint{11.076958in}{2.122078in}}%
\pgfpathlineto{\pgfqpoint{11.084906in}{2.160626in}}%
\pgfpathlineto{\pgfqpoint{11.092854in}{2.143769in}}%
\pgfpathlineto{\pgfqpoint{11.100802in}{2.319956in}}%
\pgfpathlineto{\pgfqpoint{11.108749in}{2.293194in}}%
\pgfpathlineto{\pgfqpoint{11.116697in}{2.621599in}}%
\pgfpathlineto{\pgfqpoint{11.124645in}{2.379499in}}%
\pgfpathlineto{\pgfqpoint{11.132593in}{2.324344in}}%
\pgfpathlineto{\pgfqpoint{11.140540in}{2.149061in}}%
\pgfpathlineto{\pgfqpoint{11.148488in}{2.303555in}}%
\pgfpathlineto{\pgfqpoint{11.156436in}{2.300916in}}%
\pgfpathlineto{\pgfqpoint{11.164384in}{2.222143in}}%
\pgfpathlineto{\pgfqpoint{11.172331in}{2.327221in}}%
\pgfpathlineto{\pgfqpoint{11.180279in}{2.258909in}}%
\pgfpathlineto{\pgfqpoint{11.188227in}{2.195603in}}%
\pgfpathlineto{\pgfqpoint{11.196175in}{2.573503in}}%
\pgfpathlineto{\pgfqpoint{11.204122in}{2.209574in}}%
\pgfpathlineto{\pgfqpoint{11.212070in}{2.274726in}}%
\pgfpathlineto{\pgfqpoint{11.220018in}{2.436171in}}%
\pgfpathlineto{\pgfqpoint{11.227966in}{2.152468in}}%
\pgfpathlineto{\pgfqpoint{11.235913in}{2.274730in}}%
\pgfpathlineto{\pgfqpoint{11.243861in}{2.175149in}}%
\pgfpathlineto{\pgfqpoint{11.251809in}{2.308806in}}%
\pgfpathlineto{\pgfqpoint{11.259757in}{2.198395in}}%
\pgfpathlineto{\pgfqpoint{11.267705in}{2.337817in}}%
\pgfpathlineto{\pgfqpoint{11.275652in}{2.353398in}}%
\pgfpathlineto{\pgfqpoint{11.283600in}{2.410504in}}%
\pgfpathlineto{\pgfqpoint{11.291548in}{2.391596in}}%
\pgfpathlineto{\pgfqpoint{11.299496in}{1.874500in}}%
\pgfpathlineto{\pgfqpoint{11.315391in}{2.319828in}}%
\pgfpathlineto{\pgfqpoint{11.323339in}{2.396711in}}%
\pgfpathlineto{\pgfqpoint{11.331287in}{1.832467in}}%
\pgfpathlineto{\pgfqpoint{11.339234in}{2.369005in}}%
\pgfpathlineto{\pgfqpoint{11.347182in}{2.252769in}}%
\pgfpathlineto{\pgfqpoint{11.363078in}{2.491264in}}%
\pgfpathlineto{\pgfqpoint{11.371025in}{2.317100in}}%
\pgfpathlineto{\pgfqpoint{11.378973in}{2.404427in}}%
\pgfpathlineto{\pgfqpoint{11.386921in}{2.213713in}}%
\pgfpathlineto{\pgfqpoint{11.394869in}{2.485670in}}%
\pgfpathlineto{\pgfqpoint{11.402816in}{2.282552in}}%
\pgfpathlineto{\pgfqpoint{11.410764in}{2.368996in}}%
\pgfpathlineto{\pgfqpoint{11.418712in}{2.206606in}}%
\pgfpathlineto{\pgfqpoint{11.426660in}{2.348058in}}%
\pgfpathlineto{\pgfqpoint{11.434607in}{2.003261in}}%
\pgfpathlineto{\pgfqpoint{11.442555in}{2.380480in}}%
\pgfpathlineto{\pgfqpoint{11.450503in}{2.040695in}}%
\pgfpathlineto{\pgfqpoint{11.458451in}{2.097092in}}%
\pgfpathlineto{\pgfqpoint{11.466398in}{2.368694in}}%
\pgfpathlineto{\pgfqpoint{11.474346in}{2.244809in}}%
\pgfpathlineto{\pgfqpoint{11.482294in}{2.328232in}}%
\pgfpathlineto{\pgfqpoint{11.490242in}{2.472439in}}%
\pgfpathlineto{\pgfqpoint{11.498190in}{2.407267in}}%
\pgfpathlineto{\pgfqpoint{11.506137in}{2.224226in}}%
\pgfpathlineto{\pgfqpoint{11.514085in}{2.335540in}}%
\pgfpathlineto{\pgfqpoint{11.522033in}{2.189640in}}%
\pgfpathlineto{\pgfqpoint{11.529981in}{2.467880in}}%
\pgfpathlineto{\pgfqpoint{11.537928in}{2.322040in}}%
\pgfpathlineto{\pgfqpoint{11.545876in}{2.408427in}}%
\pgfpathlineto{\pgfqpoint{11.553824in}{2.358235in}}%
\pgfpathlineto{\pgfqpoint{11.561772in}{2.178105in}}%
\pgfpathlineto{\pgfqpoint{11.569719in}{2.363579in}}%
\pgfpathlineto{\pgfqpoint{11.577667in}{2.407380in}}%
\pgfpathlineto{\pgfqpoint{11.585615in}{2.292951in}}%
\pgfpathlineto{\pgfqpoint{11.593563in}{2.136933in}}%
\pgfpathlineto{\pgfqpoint{11.601510in}{2.467073in}}%
\pgfpathlineto{\pgfqpoint{11.609458in}{2.191841in}}%
\pgfpathlineto{\pgfqpoint{11.617406in}{2.394025in}}%
\pgfpathlineto{\pgfqpoint{11.625354in}{2.217865in}}%
\pgfpathlineto{\pgfqpoint{11.633301in}{2.264419in}}%
\pgfpathlineto{\pgfqpoint{11.641249in}{2.373246in}}%
\pgfpathlineto{\pgfqpoint{11.649197in}{2.321325in}}%
\pgfpathlineto{\pgfqpoint{11.657145in}{2.436790in}}%
\pgfpathlineto{\pgfqpoint{11.665092in}{1.824150in}}%
\pgfpathlineto{\pgfqpoint{11.673040in}{2.039917in}}%
\pgfpathlineto{\pgfqpoint{11.680988in}{2.580167in}}%
\pgfpathlineto{\pgfqpoint{11.688936in}{2.624666in}}%
\pgfpathlineto{\pgfqpoint{11.696883in}{2.461654in}}%
\pgfpathlineto{\pgfqpoint{11.704831in}{2.310622in}}%
\pgfpathlineto{\pgfqpoint{11.712779in}{2.437156in}}%
\pgfpathlineto{\pgfqpoint{11.720727in}{2.191595in}}%
\pgfpathlineto{\pgfqpoint{11.728675in}{2.239488in}}%
\pgfpathlineto{\pgfqpoint{11.736622in}{2.499996in}}%
\pgfpathlineto{\pgfqpoint{11.744570in}{2.280149in}}%
\pgfpathlineto{\pgfqpoint{11.752518in}{2.285177in}}%
\pgfpathlineto{\pgfqpoint{11.760466in}{2.317833in}}%
\pgfpathlineto{\pgfqpoint{11.768413in}{2.280156in}}%
\pgfpathlineto{\pgfqpoint{11.776361in}{2.317787in}}%
\pgfpathlineto{\pgfqpoint{11.784309in}{2.267624in}}%
\pgfpathlineto{\pgfqpoint{11.792257in}{2.340289in}}%
\pgfpathlineto{\pgfqpoint{11.800204in}{2.242603in}}%
\pgfpathlineto{\pgfqpoint{11.808152in}{2.272637in}}%
\pgfpathlineto{\pgfqpoint{11.816100in}{2.390257in}}%
\pgfpathlineto{\pgfqpoint{11.824048in}{2.302635in}}%
\pgfpathlineto{\pgfqpoint{11.831995in}{2.165145in}}%
\pgfpathlineto{\pgfqpoint{11.839943in}{2.096143in}}%
\pgfpathlineto{\pgfqpoint{11.847891in}{2.366054in}}%
\pgfpathlineto{\pgfqpoint{11.855839in}{2.465939in}}%
\pgfpathlineto{\pgfqpoint{11.863786in}{2.471714in}}%
\pgfpathlineto{\pgfqpoint{11.871734in}{2.364228in}}%
\pgfpathlineto{\pgfqpoint{11.879682in}{2.106859in}}%
\pgfpathlineto{\pgfqpoint{11.887630in}{2.302584in}}%
\pgfpathlineto{\pgfqpoint{11.895577in}{2.222935in}}%
\pgfpathlineto{\pgfqpoint{11.911473in}{2.337486in}}%
\pgfpathlineto{\pgfqpoint{11.919421in}{2.359663in}}%
\pgfpathlineto{\pgfqpoint{11.927368in}{2.379122in}}%
\pgfpathlineto{\pgfqpoint{11.935316in}{2.356570in}}%
\pgfpathlineto{\pgfqpoint{11.943264in}{2.280261in}}%
\pgfpathlineto{\pgfqpoint{11.951212in}{2.184140in}}%
\pgfpathlineto{\pgfqpoint{11.959160in}{2.371460in}}%
\pgfpathlineto{\pgfqpoint{11.967107in}{2.393111in}}%
\pgfpathlineto{\pgfqpoint{11.975055in}{2.255799in}}%
\pgfpathlineto{\pgfqpoint{11.983003in}{2.182007in}}%
\pgfpathlineto{\pgfqpoint{11.990951in}{2.410384in}}%
\pgfpathlineto{\pgfqpoint{11.998898in}{2.263139in}}%
\pgfpathlineto{\pgfqpoint{12.006846in}{2.417030in}}%
\pgfpathlineto{\pgfqpoint{12.014794in}{2.214477in}}%
\pgfpathlineto{\pgfqpoint{12.022742in}{2.445711in}}%
\pgfpathlineto{\pgfqpoint{12.030689in}{2.114870in}}%
\pgfpathlineto{\pgfqpoint{12.038637in}{2.233795in}}%
\pgfpathlineto{\pgfqpoint{12.046585in}{2.509287in}}%
\pgfpathlineto{\pgfqpoint{12.054533in}{2.260941in}}%
\pgfpathlineto{\pgfqpoint{12.062480in}{2.381941in}}%
\pgfpathlineto{\pgfqpoint{12.070428in}{2.132555in}}%
\pgfpathlineto{\pgfqpoint{12.078376in}{2.246226in}}%
\pgfpathlineto{\pgfqpoint{12.086324in}{1.938529in}}%
\pgfpathlineto{\pgfqpoint{12.094271in}{2.736330in}}%
\pgfpathlineto{\pgfqpoint{12.102219in}{2.019174in}}%
\pgfpathlineto{\pgfqpoint{12.118115in}{2.519761in}}%
\pgfpathlineto{\pgfqpoint{12.126062in}{2.282747in}}%
\pgfpathlineto{\pgfqpoint{12.134010in}{2.355500in}}%
\pgfpathlineto{\pgfqpoint{12.141958in}{2.333529in}}%
\pgfpathlineto{\pgfqpoint{12.149906in}{2.110547in}}%
\pgfpathlineto{\pgfqpoint{12.157854in}{2.548945in}}%
\pgfpathlineto{\pgfqpoint{12.165801in}{2.014098in}}%
\pgfpathlineto{\pgfqpoint{12.173749in}{2.304664in}}%
\pgfpathlineto{\pgfqpoint{12.181697in}{2.396856in}}%
\pgfpathlineto{\pgfqpoint{12.189645in}{2.528174in}}%
\pgfpathlineto{\pgfqpoint{12.197592in}{2.015531in}}%
\pgfpathlineto{\pgfqpoint{12.205540in}{2.619275in}}%
\pgfpathlineto{\pgfqpoint{12.213488in}{2.045395in}}%
\pgfpathlineto{\pgfqpoint{12.229383in}{2.393815in}}%
\pgfpathlineto{\pgfqpoint{12.237331in}{2.366861in}}%
\pgfpathlineto{\pgfqpoint{12.245279in}{2.457040in}}%
\pgfpathlineto{\pgfqpoint{12.253227in}{2.079720in}}%
\pgfpathlineto{\pgfqpoint{12.261174in}{2.169559in}}%
\pgfpathlineto{\pgfqpoint{12.269122in}{2.292423in}}%
\pgfpathlineto{\pgfqpoint{12.277070in}{2.210170in}}%
\pgfpathlineto{\pgfqpoint{12.285018in}{2.328764in}}%
\pgfpathlineto{\pgfqpoint{12.292965in}{2.151688in}}%
\pgfpathlineto{\pgfqpoint{12.300913in}{2.199753in}}%
\pgfpathlineto{\pgfqpoint{12.308861in}{2.004329in}}%
\pgfpathlineto{\pgfqpoint{12.316809in}{2.143228in}}%
\pgfpathlineto{\pgfqpoint{12.324756in}{2.147227in}}%
\pgfpathlineto{\pgfqpoint{12.332704in}{2.232248in}}%
\pgfpathlineto{\pgfqpoint{12.340652in}{2.942829in}}%
\pgfpathlineto{\pgfqpoint{12.348600in}{2.380126in}}%
\pgfpathlineto{\pgfqpoint{12.356547in}{2.102171in}}%
\pgfpathlineto{\pgfqpoint{12.364495in}{2.343925in}}%
\pgfpathlineto{\pgfqpoint{12.372443in}{2.275398in}}%
\pgfpathlineto{\pgfqpoint{12.380391in}{2.297400in}}%
\pgfpathlineto{\pgfqpoint{12.388339in}{2.258276in}}%
\pgfpathlineto{\pgfqpoint{12.396286in}{2.253337in}}%
\pgfpathlineto{\pgfqpoint{12.404234in}{1.938967in}}%
\pgfpathlineto{\pgfqpoint{12.412182in}{2.624031in}}%
\pgfpathlineto{\pgfqpoint{12.420130in}{2.487652in}}%
\pgfpathlineto{\pgfqpoint{12.428077in}{1.939959in}}%
\pgfpathlineto{\pgfqpoint{12.436025in}{1.596413in}}%
\pgfpathlineto{\pgfqpoint{12.443973in}{2.430468in}}%
\pgfpathlineto{\pgfqpoint{12.451921in}{2.147559in}}%
\pgfpathlineto{\pgfqpoint{12.459868in}{2.430412in}}%
\pgfpathlineto{\pgfqpoint{12.467816in}{2.185922in}}%
\pgfpathlineto{\pgfqpoint{12.475764in}{2.257090in}}%
\pgfpathlineto{\pgfqpoint{12.483712in}{2.030572in}}%
\pgfpathlineto{\pgfqpoint{12.491659in}{2.437308in}}%
\pgfpathlineto{\pgfqpoint{12.499607in}{2.313410in}}%
\pgfpathlineto{\pgfqpoint{12.507555in}{2.267215in}}%
\pgfpathlineto{\pgfqpoint{12.515503in}{2.469146in}}%
\pgfpathlineto{\pgfqpoint{12.523450in}{2.193408in}}%
\pgfpathlineto{\pgfqpoint{12.531398in}{2.241691in}}%
\pgfpathlineto{\pgfqpoint{12.539346in}{2.182476in}}%
\pgfpathlineto{\pgfqpoint{12.547294in}{2.143078in}}%
\pgfpathlineto{\pgfqpoint{12.555241in}{2.414399in}}%
\pgfpathlineto{\pgfqpoint{12.563189in}{2.423786in}}%
\pgfpathlineto{\pgfqpoint{12.571137in}{2.267264in}}%
\pgfpathlineto{\pgfqpoint{12.579085in}{2.582444in}}%
\pgfpathlineto{\pgfqpoint{12.587032in}{2.125670in}}%
\pgfpathlineto{\pgfqpoint{12.594980in}{2.490019in}}%
\pgfpathlineto{\pgfqpoint{12.602928in}{2.460491in}}%
\pgfpathlineto{\pgfqpoint{12.610876in}{2.295147in}}%
\pgfpathlineto{\pgfqpoint{12.618824in}{2.099891in}}%
\pgfpathlineto{\pgfqpoint{12.626771in}{2.355491in}}%
\pgfpathlineto{\pgfqpoint{12.634719in}{2.285177in}}%
\pgfpathlineto{\pgfqpoint{12.642667in}{2.292695in}}%
\pgfpathlineto{\pgfqpoint{12.650615in}{2.252575in}}%
\pgfpathlineto{\pgfqpoint{12.658562in}{2.217291in}}%
\pgfpathlineto{\pgfqpoint{12.666510in}{2.330460in}}%
\pgfpathlineto{\pgfqpoint{12.674458in}{2.397929in}}%
\pgfpathlineto{\pgfqpoint{12.682406in}{2.202555in}}%
\pgfpathlineto{\pgfqpoint{12.690353in}{2.161850in}}%
\pgfpathlineto{\pgfqpoint{12.698301in}{2.350712in}}%
\pgfpathlineto{\pgfqpoint{12.706249in}{2.222166in}}%
\pgfpathlineto{\pgfqpoint{12.714197in}{2.312926in}}%
\pgfpathlineto{\pgfqpoint{12.722144in}{2.355635in}}%
\pgfpathlineto{\pgfqpoint{12.730092in}{2.412865in}}%
\pgfpathlineto{\pgfqpoint{12.738040in}{2.280185in}}%
\pgfpathlineto{\pgfqpoint{12.745988in}{2.137346in}}%
\pgfpathlineto{\pgfqpoint{12.753935in}{2.072761in}}%
\pgfpathlineto{\pgfqpoint{12.761883in}{2.346102in}}%
\pgfpathlineto{\pgfqpoint{12.769831in}{2.307975in}}%
\pgfpathlineto{\pgfqpoint{12.777779in}{2.323115in}}%
\pgfpathlineto{\pgfqpoint{12.785726in}{2.563659in}}%
\pgfpathlineto{\pgfqpoint{12.793674in}{2.374652in}}%
\pgfpathlineto{\pgfqpoint{12.801622in}{2.233033in}}%
\pgfpathlineto{\pgfqpoint{12.809570in}{2.911561in}}%
\pgfpathlineto{\pgfqpoint{12.817517in}{2.174089in}}%
\pgfpathlineto{\pgfqpoint{12.825465in}{2.487559in}}%
\pgfpathlineto{\pgfqpoint{12.833413in}{2.128755in}}%
\pgfpathlineto{\pgfqpoint{12.841361in}{2.159133in}}%
\pgfpathlineto{\pgfqpoint{12.849309in}{2.311907in}}%
\pgfpathlineto{\pgfqpoint{12.857256in}{2.292460in}}%
\pgfpathlineto{\pgfqpoint{12.865204in}{2.476017in}}%
\pgfpathlineto{\pgfqpoint{12.873152in}{2.224985in}}%
\pgfpathlineto{\pgfqpoint{12.881100in}{2.381395in}}%
\pgfpathlineto{\pgfqpoint{12.889047in}{2.205831in}}%
\pgfpathlineto{\pgfqpoint{12.896995in}{2.282767in}}%
\pgfpathlineto{\pgfqpoint{12.904943in}{2.422044in}}%
\pgfpathlineto{\pgfqpoint{12.912891in}{2.356830in}}%
\pgfpathlineto{\pgfqpoint{12.920838in}{2.358944in}}%
\pgfpathlineto{\pgfqpoint{12.928786in}{2.130183in}}%
\pgfpathlineto{\pgfqpoint{12.936734in}{2.361634in}}%
\pgfpathlineto{\pgfqpoint{12.944682in}{2.184764in}}%
\pgfpathlineto{\pgfqpoint{12.952629in}{2.998299in}}%
\pgfpathlineto{\pgfqpoint{12.960577in}{2.213398in}}%
\pgfpathlineto{\pgfqpoint{12.968525in}{2.350021in}}%
\pgfpathlineto{\pgfqpoint{12.976473in}{2.409630in}}%
\pgfpathlineto{\pgfqpoint{12.984420in}{2.236872in}}%
\pgfpathlineto{\pgfqpoint{12.992368in}{2.218274in}}%
\pgfpathlineto{\pgfqpoint{13.000316in}{2.268993in}}%
\pgfpathlineto{\pgfqpoint{13.008264in}{2.250453in}}%
\pgfpathlineto{\pgfqpoint{13.016211in}{2.324525in}}%
\pgfpathlineto{\pgfqpoint{13.024159in}{2.409645in}}%
\pgfpathlineto{\pgfqpoint{13.032107in}{2.125993in}}%
\pgfpathlineto{\pgfqpoint{13.040055in}{2.196955in}}%
\pgfpathlineto{\pgfqpoint{13.048003in}{2.254904in}}%
\pgfpathlineto{\pgfqpoint{13.055950in}{2.268857in}}%
\pgfpathlineto{\pgfqpoint{13.063898in}{2.252496in}}%
\pgfpathlineto{\pgfqpoint{13.071846in}{2.329516in}}%
\pgfpathlineto{\pgfqpoint{13.079794in}{2.443048in}}%
\pgfpathlineto{\pgfqpoint{13.087741in}{2.335981in}}%
\pgfpathlineto{\pgfqpoint{13.095689in}{2.218179in}}%
\pgfpathlineto{\pgfqpoint{13.103637in}{2.370616in}}%
\pgfpathlineto{\pgfqpoint{13.111585in}{2.294391in}}%
\pgfpathlineto{\pgfqpoint{13.119532in}{2.190523in}}%
\pgfpathlineto{\pgfqpoint{13.127480in}{2.331406in}}%
\pgfpathlineto{\pgfqpoint{13.135428in}{2.280559in}}%
\pgfpathlineto{\pgfqpoint{13.143376in}{2.241251in}}%
\pgfpathlineto{\pgfqpoint{13.151323in}{2.238832in}}%
\pgfpathlineto{\pgfqpoint{13.159271in}{2.241047in}}%
\pgfpathlineto{\pgfqpoint{13.167219in}{2.268893in}}%
\pgfpathlineto{\pgfqpoint{13.175167in}{2.343269in}}%
\pgfpathlineto{\pgfqpoint{13.183114in}{2.250342in}}%
\pgfpathlineto{\pgfqpoint{13.191062in}{2.198985in}}%
\pgfpathlineto{\pgfqpoint{13.199010in}{2.207982in}}%
\pgfpathlineto{\pgfqpoint{13.206958in}{2.247639in}}%
\pgfpathlineto{\pgfqpoint{13.214905in}{2.336772in}}%
\pgfpathlineto{\pgfqpoint{13.222853in}{2.296884in}}%
\pgfpathlineto{\pgfqpoint{13.230801in}{2.160721in}}%
\pgfpathlineto{\pgfqpoint{13.238749in}{2.247449in}}%
\pgfpathlineto{\pgfqpoint{13.246696in}{2.344094in}}%
\pgfpathlineto{\pgfqpoint{13.254644in}{2.350954in}}%
\pgfpathlineto{\pgfqpoint{13.262592in}{2.397426in}}%
\pgfpathlineto{\pgfqpoint{13.270540in}{2.224455in}}%
\pgfpathlineto{\pgfqpoint{13.278488in}{2.362215in}}%
\pgfpathlineto{\pgfqpoint{13.286435in}{2.373514in}}%
\pgfpathlineto{\pgfqpoint{13.294383in}{2.175859in}}%
\pgfpathlineto{\pgfqpoint{13.302331in}{2.224440in}}%
\pgfpathlineto{\pgfqpoint{13.310279in}{2.130148in}}%
\pgfpathlineto{\pgfqpoint{13.318226in}{2.489276in}}%
\pgfpathlineto{\pgfqpoint{13.326174in}{2.382951in}}%
\pgfpathlineto{\pgfqpoint{13.334122in}{1.786584in}}%
\pgfpathlineto{\pgfqpoint{13.342070in}{2.275646in}}%
\pgfpathlineto{\pgfqpoint{13.350017in}{2.203980in}}%
\pgfpathlineto{\pgfqpoint{13.357965in}{2.230058in}}%
\pgfpathlineto{\pgfqpoint{13.365913in}{2.089818in}}%
\pgfpathlineto{\pgfqpoint{13.373861in}{2.331177in}}%
\pgfpathlineto{\pgfqpoint{13.381808in}{2.290012in}}%
\pgfpathlineto{\pgfqpoint{13.389756in}{2.297261in}}%
\pgfpathlineto{\pgfqpoint{13.397704in}{2.297254in}}%
\pgfpathlineto{\pgfqpoint{13.405652in}{2.251342in}}%
\pgfpathlineto{\pgfqpoint{13.413599in}{2.304518in}}%
\pgfpathlineto{\pgfqpoint{13.421547in}{2.212549in}}%
\pgfpathlineto{\pgfqpoint{13.429495in}{2.290027in}}%
\pgfpathlineto{\pgfqpoint{13.437443in}{2.316674in}}%
\pgfpathlineto{\pgfqpoint{13.445390in}{2.265800in}}%
\pgfpathlineto{\pgfqpoint{13.453338in}{2.290023in}}%
\pgfpathlineto{\pgfqpoint{13.461286in}{2.195331in}}%
\pgfpathlineto{\pgfqpoint{13.469234in}{2.370176in}}%
\pgfpathlineto{\pgfqpoint{13.477181in}{2.277905in}}%
\pgfpathlineto{\pgfqpoint{13.485129in}{2.256065in}}%
\pgfpathlineto{\pgfqpoint{13.493077in}{2.379633in}}%
\pgfpathlineto{\pgfqpoint{13.501025in}{2.367153in}}%
\pgfpathlineto{\pgfqpoint{13.508973in}{2.371603in}}%
\pgfpathlineto{\pgfqpoint{13.516920in}{2.318684in}}%
\pgfpathlineto{\pgfqpoint{13.524868in}{2.440001in}}%
\pgfpathlineto{\pgfqpoint{13.532816in}{2.294664in}}%
\pgfpathlineto{\pgfqpoint{13.540764in}{2.386881in}}%
\pgfpathlineto{\pgfqpoint{13.548711in}{2.318176in}}%
\pgfpathlineto{\pgfqpoint{13.556659in}{2.299302in}}%
\pgfpathlineto{\pgfqpoint{13.564607in}{2.278115in}}%
\pgfpathlineto{\pgfqpoint{13.572555in}{2.081656in}}%
\pgfpathlineto{\pgfqpoint{13.580502in}{2.354042in}}%
\pgfpathlineto{\pgfqpoint{13.588450in}{2.301763in}}%
\pgfpathlineto{\pgfqpoint{13.596398in}{2.244871in}}%
\pgfpathlineto{\pgfqpoint{13.612293in}{2.492031in}}%
\pgfpathlineto{\pgfqpoint{13.620241in}{2.345877in}}%
\pgfpathlineto{\pgfqpoint{13.628189in}{2.329416in}}%
\pgfpathlineto{\pgfqpoint{13.636137in}{2.233943in}}%
\pgfpathlineto{\pgfqpoint{13.644084in}{2.236146in}}%
\pgfpathlineto{\pgfqpoint{13.652032in}{2.203186in}}%
\pgfpathlineto{\pgfqpoint{13.659980in}{2.371842in}}%
\pgfpathlineto{\pgfqpoint{13.667928in}{2.271149in}}%
\pgfpathlineto{\pgfqpoint{13.675875in}{2.120772in}}%
\pgfpathlineto{\pgfqpoint{13.683823in}{2.230855in}}%
\pgfpathlineto{\pgfqpoint{13.691771in}{1.977681in}}%
\pgfpathlineto{\pgfqpoint{13.699719in}{2.321181in}}%
\pgfpathlineto{\pgfqpoint{13.707666in}{2.416629in}}%
\pgfpathlineto{\pgfqpoint{13.715614in}{2.270879in}}%
\pgfpathlineto{\pgfqpoint{13.723562in}{2.218317in}}%
\pgfpathlineto{\pgfqpoint{13.731510in}{2.347268in}}%
\pgfpathlineto{\pgfqpoint{13.739458in}{2.158396in}}%
\pgfpathlineto{\pgfqpoint{13.747405in}{2.213049in}}%
\pgfpathlineto{\pgfqpoint{13.755353in}{2.328485in}}%
\pgfpathlineto{\pgfqpoint{13.763301in}{2.227410in}}%
\pgfpathlineto{\pgfqpoint{13.771249in}{2.374162in}}%
\pgfpathlineto{\pgfqpoint{13.779196in}{2.385699in}}%
\pgfpathlineto{\pgfqpoint{13.787144in}{2.529535in}}%
\pgfpathlineto{\pgfqpoint{13.795092in}{2.372217in}}%
\pgfpathlineto{\pgfqpoint{13.803040in}{2.341429in}}%
\pgfpathlineto{\pgfqpoint{13.810987in}{2.287517in}}%
\pgfpathlineto{\pgfqpoint{13.818935in}{2.188992in}}%
\pgfpathlineto{\pgfqpoint{13.826883in}{2.498023in}}%
\pgfpathlineto{\pgfqpoint{13.834831in}{2.320040in}}%
\pgfpathlineto{\pgfqpoint{13.842778in}{2.584756in}}%
\pgfpathlineto{\pgfqpoint{13.850726in}{2.184292in}}%
\pgfpathlineto{\pgfqpoint{13.858674in}{2.418052in}}%
\pgfpathlineto{\pgfqpoint{13.874569in}{2.271514in}}%
\pgfpathlineto{\pgfqpoint{13.882517in}{2.246414in}}%
\pgfpathlineto{\pgfqpoint{13.890465in}{2.244050in}}%
\pgfpathlineto{\pgfqpoint{13.898413in}{2.451434in}}%
\pgfpathlineto{\pgfqpoint{13.906360in}{2.314637in}}%
\pgfpathlineto{\pgfqpoint{13.914308in}{2.359761in}}%
\pgfpathlineto{\pgfqpoint{13.922256in}{2.246790in}}%
\pgfpathlineto{\pgfqpoint{13.930204in}{2.199097in}}%
\pgfpathlineto{\pgfqpoint{13.938152in}{2.350903in}}%
\pgfpathlineto{\pgfqpoint{13.946099in}{2.402481in}}%
\pgfpathlineto{\pgfqpoint{13.954047in}{2.336838in}}%
\pgfpathlineto{\pgfqpoint{13.961995in}{2.206507in}}%
\pgfpathlineto{\pgfqpoint{13.969943in}{2.370574in}}%
\pgfpathlineto{\pgfqpoint{13.977890in}{2.365738in}}%
\pgfpathlineto{\pgfqpoint{13.985838in}{2.271773in}}%
\pgfpathlineto{\pgfqpoint{13.993786in}{2.211291in}}%
\pgfpathlineto{\pgfqpoint{14.001734in}{2.249253in}}%
\pgfpathlineto{\pgfqpoint{14.009681in}{2.334555in}}%
\pgfpathlineto{\pgfqpoint{14.017629in}{2.278451in}}%
\pgfpathlineto{\pgfqpoint{14.025577in}{2.251511in}}%
\pgfpathlineto{\pgfqpoint{14.033525in}{2.190605in}}%
\pgfpathlineto{\pgfqpoint{14.041472in}{2.370761in}}%
\pgfpathlineto{\pgfqpoint{14.049420in}{2.339038in}}%
\pgfpathlineto{\pgfqpoint{14.057368in}{2.325476in}}%
\pgfpathlineto{\pgfqpoint{14.065316in}{2.285177in}}%
\pgfpathlineto{\pgfqpoint{14.073263in}{2.338780in}}%
\pgfpathlineto{\pgfqpoint{14.081211in}{2.256160in}}%
\pgfpathlineto{\pgfqpoint{14.089159in}{2.527152in}}%
\pgfpathlineto{\pgfqpoint{14.097107in}{2.185641in}}%
\pgfpathlineto{\pgfqpoint{14.105054in}{2.294045in}}%
\pgfpathlineto{\pgfqpoint{14.113002in}{2.285177in}}%
\pgfpathlineto{\pgfqpoint{14.120950in}{2.225239in}}%
\pgfpathlineto{\pgfqpoint{14.128898in}{2.260705in}}%
\pgfpathlineto{\pgfqpoint{14.136845in}{2.497726in}}%
\pgfpathlineto{\pgfqpoint{14.144793in}{2.333563in}}%
\pgfpathlineto{\pgfqpoint{14.152741in}{1.917096in}}%
\pgfpathlineto{\pgfqpoint{14.160689in}{2.199935in}}%
\pgfpathlineto{\pgfqpoint{14.168637in}{2.386080in}}%
\pgfpathlineto{\pgfqpoint{14.176584in}{2.271753in}}%
\pgfpathlineto{\pgfqpoint{14.184532in}{1.903244in}}%
\pgfpathlineto{\pgfqpoint{14.192480in}{2.289740in}}%
\pgfpathlineto{\pgfqpoint{14.200428in}{2.305700in}}%
\pgfpathlineto{\pgfqpoint{14.208375in}{2.453116in}}%
\pgfpathlineto{\pgfqpoint{14.216323in}{2.359607in}}%
\pgfpathlineto{\pgfqpoint{14.224271in}{2.350352in}}%
\pgfpathlineto{\pgfqpoint{14.232219in}{2.201983in}}%
\pgfpathlineto{\pgfqpoint{14.240166in}{2.348166in}}%
\pgfpathlineto{\pgfqpoint{14.248114in}{2.240205in}}%
\pgfpathlineto{\pgfqpoint{14.256062in}{2.179084in}}%
\pgfpathlineto{\pgfqpoint{14.264010in}{2.251196in}}%
\pgfpathlineto{\pgfqpoint{14.271957in}{2.432002in}}%
\pgfpathlineto{\pgfqpoint{14.279905in}{2.473304in}}%
\pgfpathlineto{\pgfqpoint{14.287853in}{2.307453in}}%
\pgfpathlineto{\pgfqpoint{14.295801in}{2.213803in}}%
\pgfpathlineto{\pgfqpoint{14.303748in}{2.278472in}}%
\pgfpathlineto{\pgfqpoint{14.311696in}{2.179837in}}%
\pgfpathlineto{\pgfqpoint{14.319644in}{1.557170in}}%
\pgfpathlineto{\pgfqpoint{14.327592in}{2.074127in}}%
\pgfpathlineto{\pgfqpoint{14.335539in}{2.211952in}}%
\pgfpathlineto{\pgfqpoint{14.343487in}{2.259128in}}%
\pgfpathlineto{\pgfqpoint{14.351435in}{2.429195in}}%
\pgfpathlineto{\pgfqpoint{14.359383in}{2.181401in}}%
\pgfpathlineto{\pgfqpoint{14.367330in}{2.111782in}}%
\pgfpathlineto{\pgfqpoint{14.375278in}{2.301870in}}%
\pgfpathlineto{\pgfqpoint{14.383226in}{2.408743in}}%
\pgfpathlineto{\pgfqpoint{14.391174in}{2.199714in}}%
\pgfpathlineto{\pgfqpoint{14.399122in}{2.299446in}}%
\pgfpathlineto{\pgfqpoint{14.407069in}{2.165944in}}%
\pgfpathlineto{\pgfqpoint{14.415017in}{2.494544in}}%
\pgfpathlineto{\pgfqpoint{14.422965in}{2.421955in}}%
\pgfpathlineto{\pgfqpoint{14.430913in}{2.329776in}}%
\pgfpathlineto{\pgfqpoint{14.438860in}{2.369404in}}%
\pgfpathlineto{\pgfqpoint{14.446808in}{2.471062in}}%
\pgfpathlineto{\pgfqpoint{14.454756in}{2.255090in}}%
\pgfpathlineto{\pgfqpoint{14.462704in}{2.143381in}}%
\pgfpathlineto{\pgfqpoint{14.470651in}{2.512483in}}%
\pgfpathlineto{\pgfqpoint{14.478599in}{2.102147in}}%
\pgfpathlineto{\pgfqpoint{14.486547in}{2.250230in}}%
\pgfpathlineto{\pgfqpoint{14.494495in}{2.385190in}}%
\pgfpathlineto{\pgfqpoint{14.502442in}{2.199149in}}%
\pgfpathlineto{\pgfqpoint{14.510390in}{2.310792in}}%
\pgfpathlineto{\pgfqpoint{14.518338in}{2.398876in}}%
\pgfpathlineto{\pgfqpoint{14.526286in}{2.382110in}}%
\pgfpathlineto{\pgfqpoint{14.534233in}{2.280572in}}%
\pgfpathlineto{\pgfqpoint{14.542181in}{2.418285in}}%
\pgfpathlineto{\pgfqpoint{14.550129in}{2.365065in}}%
\pgfpathlineto{\pgfqpoint{14.558077in}{2.511651in}}%
\pgfpathlineto{\pgfqpoint{14.566024in}{2.282925in}}%
\pgfpathlineto{\pgfqpoint{14.573972in}{2.253624in}}%
\pgfpathlineto{\pgfqpoint{14.581920in}{2.035477in}}%
\pgfpathlineto{\pgfqpoint{14.589868in}{2.344484in}}%
\pgfpathlineto{\pgfqpoint{14.597815in}{2.255545in}}%
\pgfpathlineto{\pgfqpoint{14.605763in}{2.202886in}}%
\pgfpathlineto{\pgfqpoint{14.613711in}{2.508386in}}%
\pgfpathlineto{\pgfqpoint{14.621659in}{2.310073in}}%
\pgfpathlineto{\pgfqpoint{14.629607in}{2.476496in}}%
\pgfpathlineto{\pgfqpoint{14.637554in}{2.282937in}}%
\pgfpathlineto{\pgfqpoint{14.645502in}{2.242568in}}%
\pgfpathlineto{\pgfqpoint{14.653450in}{2.251473in}}%
\pgfpathlineto{\pgfqpoint{14.669345in}{2.383719in}}%
\pgfpathlineto{\pgfqpoint{14.677293in}{2.358760in}}%
\pgfpathlineto{\pgfqpoint{14.685241in}{2.204891in}}%
\pgfpathlineto{\pgfqpoint{14.693189in}{2.278471in}}%
\pgfpathlineto{\pgfqpoint{14.701136in}{2.244896in}}%
\pgfpathlineto{\pgfqpoint{14.701136in}{2.244896in}}%
\pgfusepath{stroke}%
\end{pgfscope}%
\begin{pgfscope}%
\pgfsetrectcap%
\pgfsetmiterjoin%
\pgfsetlinewidth{0.803000pt}%
\definecolor{currentstroke}{rgb}{1.000000,1.000000,1.000000}%
\pgfsetstrokecolor{currentstroke}%
\pgfsetdash{}{0pt}%
\pgfpathmoveto{\pgfqpoint{2.125000in}{1.000000in}}%
\pgfpathlineto{\pgfqpoint{2.125000in}{3.157143in}}%
\pgfusepath{stroke}%
\end{pgfscope}%
\begin{pgfscope}%
\pgfsetrectcap%
\pgfsetmiterjoin%
\pgfsetlinewidth{0.803000pt}%
\definecolor{currentstroke}{rgb}{1.000000,1.000000,1.000000}%
\pgfsetstrokecolor{currentstroke}%
\pgfsetdash{}{0pt}%
\pgfpathmoveto{\pgfqpoint{15.300000in}{1.000000in}}%
\pgfpathlineto{\pgfqpoint{15.300000in}{3.157143in}}%
\pgfusepath{stroke}%
\end{pgfscope}%
\begin{pgfscope}%
\pgfsetrectcap%
\pgfsetmiterjoin%
\pgfsetlinewidth{0.803000pt}%
\definecolor{currentstroke}{rgb}{1.000000,1.000000,1.000000}%
\pgfsetstrokecolor{currentstroke}%
\pgfsetdash{}{0pt}%
\pgfpathmoveto{\pgfqpoint{2.125000in}{1.000000in}}%
\pgfpathlineto{\pgfqpoint{15.300000in}{1.000000in}}%
\pgfusepath{stroke}%
\end{pgfscope}%
\begin{pgfscope}%
\pgfsetrectcap%
\pgfsetmiterjoin%
\pgfsetlinewidth{0.803000pt}%
\definecolor{currentstroke}{rgb}{1.000000,1.000000,1.000000}%
\pgfsetstrokecolor{currentstroke}%
\pgfsetdash{}{0pt}%
\pgfpathmoveto{\pgfqpoint{2.125000in}{3.157143in}}%
\pgfpathlineto{\pgfqpoint{15.300000in}{3.157143in}}%
\pgfusepath{stroke}%
\end{pgfscope}%
\end{pgfpicture}%
\makeatother%
\endgroup%

    \end{adjustbox}  
    \caption{QQ-Plot for log-returns of stock PG. QQ-plots for the other stocks can be seen in the Appendix in figure \ref{fig:all_qq_fd_log_adjclose}.}
    \label{fig:PG_qq_fd_log_adjclose}
\end{figure}{}

The data are also not homoscedastic. While stationarity implies that the unconditional variance is constant over time, the variance of the time series fluctuates conditional on past observations [Beschreibung nachgucken, QUELLE!]. This conditional heteroskedasticity is quite common in financial data. (SOURCE). The pattern can be observed in figure \ref{fig:PG_squared_log_returns}. Figure \ref{fig:ACF_selected_squared_log_returns} shows the ACF and PACF of the squared residuals of some selected stocks. The patterns indicate that at least some of the volatility can be modeled using time series approaches. 

\begin{figure}[h]
    \centering
    \figuretitle{Squared log-returns of Stock PG}
    \begin{adjustbox}{width=.95\textwidth,center}
    %% Creator: Matplotlib, PGF backend
%%
%% To include the figure in your LaTeX document, write
%%   \input{<filename>.pgf}
%%
%% Make sure the required packages are loaded in your preamble
%%   \usepackage{pgf}
%%
%% Figures using additional raster images can only be included by \input if
%% they are in the same directory as the main LaTeX file. For loading figures
%% from other directories you can use the `import` package
%%   \usepackage{import}
%% and then include the figures with
%%   \import{<path to file>}{<filename>.pgf}
%%
%% Matplotlib used the following preamble
%%   \usepackage{fontspec}
%%   \setmainfont{DejaVuSerif.ttf}[Path=/opt/tljh/user/lib/python3.6/site-packages/matplotlib/mpl-data/fonts/ttf/]
%%   \setsansfont{DejaVuSans.ttf}[Path=/opt/tljh/user/lib/python3.6/site-packages/matplotlib/mpl-data/fonts/ttf/]
%%   \setmonofont{DejaVuSansMono.ttf}[Path=/opt/tljh/user/lib/python3.6/site-packages/matplotlib/mpl-data/fonts/ttf/]
%%
\begingroup%
\makeatletter%
\begin{pgfpicture}%
\pgfpathrectangle{\pgfpointorigin}{\pgfqpoint{7.175133in}{3.641603in}}%
\pgfusepath{use as bounding box, clip}%
\begin{pgfscope}%
\pgfsetbuttcap%
\pgfsetmiterjoin%
\definecolor{currentfill}{rgb}{1.000000,1.000000,1.000000}%
\pgfsetfillcolor{currentfill}%
\pgfsetlinewidth{0.000000pt}%
\definecolor{currentstroke}{rgb}{1.000000,1.000000,1.000000}%
\pgfsetstrokecolor{currentstroke}%
\pgfsetdash{}{0pt}%
\pgfpathmoveto{\pgfqpoint{0.000000in}{0.000000in}}%
\pgfpathlineto{\pgfqpoint{7.175133in}{0.000000in}}%
\pgfpathlineto{\pgfqpoint{7.175133in}{3.641603in}}%
\pgfpathlineto{\pgfqpoint{0.000000in}{3.641603in}}%
\pgfpathclose%
\pgfusepath{fill}%
\end{pgfscope}%
\begin{pgfscope}%
\pgfsetbuttcap%
\pgfsetmiterjoin%
\definecolor{currentfill}{rgb}{0.917647,0.917647,0.949020}%
\pgfsetfillcolor{currentfill}%
\pgfsetlinewidth{0.000000pt}%
\definecolor{currentstroke}{rgb}{0.000000,0.000000,0.000000}%
\pgfsetstrokecolor{currentstroke}%
\pgfsetstrokeopacity{0.000000}%
\pgfsetdash{}{0pt}%
\pgfpathmoveto{\pgfqpoint{0.875133in}{0.521603in}}%
\pgfpathlineto{\pgfqpoint{7.075133in}{0.521603in}}%
\pgfpathlineto{\pgfqpoint{7.075133in}{3.541603in}}%
\pgfpathlineto{\pgfqpoint{0.875133in}{3.541603in}}%
\pgfpathclose%
\pgfusepath{fill}%
\end{pgfscope}%
\begin{pgfscope}%
\pgfpathrectangle{\pgfqpoint{0.875133in}{0.521603in}}{\pgfqpoint{6.200000in}{3.020000in}}%
\pgfusepath{clip}%
\pgfsetroundcap%
\pgfsetroundjoin%
\pgfsetlinewidth{0.803000pt}%
\definecolor{currentstroke}{rgb}{1.000000,1.000000,1.000000}%
\pgfsetstrokecolor{currentstroke}%
\pgfsetdash{}{0pt}%
\pgfpathmoveto{\pgfqpoint{1.149216in}{0.521603in}}%
\pgfpathlineto{\pgfqpoint{1.149216in}{3.541603in}}%
\pgfusepath{stroke}%
\end{pgfscope}%
\begin{pgfscope}%
\definecolor{textcolor}{rgb}{0.150000,0.150000,0.150000}%
\pgfsetstrokecolor{textcolor}%
\pgfsetfillcolor{textcolor}%
\pgftext[x=1.149216in,y=0.424381in,,top]{\color{textcolor}\rmfamily\fontsize{10.000000}{12.000000}\selectfont 2012}%
\end{pgfscope}%
\begin{pgfscope}%
\pgfpathrectangle{\pgfqpoint{0.875133in}{0.521603in}}{\pgfqpoint{6.200000in}{3.020000in}}%
\pgfusepath{clip}%
\pgfsetroundcap%
\pgfsetroundjoin%
\pgfsetlinewidth{0.803000pt}%
\definecolor{currentstroke}{rgb}{1.000000,1.000000,1.000000}%
\pgfsetstrokecolor{currentstroke}%
\pgfsetdash{}{0pt}%
\pgfpathmoveto{\pgfqpoint{2.092907in}{0.521603in}}%
\pgfpathlineto{\pgfqpoint{2.092907in}{3.541603in}}%
\pgfusepath{stroke}%
\end{pgfscope}%
\begin{pgfscope}%
\definecolor{textcolor}{rgb}{0.150000,0.150000,0.150000}%
\pgfsetstrokecolor{textcolor}%
\pgfsetfillcolor{textcolor}%
\pgftext[x=2.092907in,y=0.424381in,,top]{\color{textcolor}\rmfamily\fontsize{10.000000}{12.000000}\selectfont 2013}%
\end{pgfscope}%
\begin{pgfscope}%
\pgfpathrectangle{\pgfqpoint{0.875133in}{0.521603in}}{\pgfqpoint{6.200000in}{3.020000in}}%
\pgfusepath{clip}%
\pgfsetroundcap%
\pgfsetroundjoin%
\pgfsetlinewidth{0.803000pt}%
\definecolor{currentstroke}{rgb}{1.000000,1.000000,1.000000}%
\pgfsetstrokecolor{currentstroke}%
\pgfsetdash{}{0pt}%
\pgfpathmoveto{\pgfqpoint{3.034020in}{0.521603in}}%
\pgfpathlineto{\pgfqpoint{3.034020in}{3.541603in}}%
\pgfusepath{stroke}%
\end{pgfscope}%
\begin{pgfscope}%
\definecolor{textcolor}{rgb}{0.150000,0.150000,0.150000}%
\pgfsetstrokecolor{textcolor}%
\pgfsetfillcolor{textcolor}%
\pgftext[x=3.034020in,y=0.424381in,,top]{\color{textcolor}\rmfamily\fontsize{10.000000}{12.000000}\selectfont 2014}%
\end{pgfscope}%
\begin{pgfscope}%
\pgfpathrectangle{\pgfqpoint{0.875133in}{0.521603in}}{\pgfqpoint{6.200000in}{3.020000in}}%
\pgfusepath{clip}%
\pgfsetroundcap%
\pgfsetroundjoin%
\pgfsetlinewidth{0.803000pt}%
\definecolor{currentstroke}{rgb}{1.000000,1.000000,1.000000}%
\pgfsetstrokecolor{currentstroke}%
\pgfsetdash{}{0pt}%
\pgfpathmoveto{\pgfqpoint{3.975133in}{0.521603in}}%
\pgfpathlineto{\pgfqpoint{3.975133in}{3.541603in}}%
\pgfusepath{stroke}%
\end{pgfscope}%
\begin{pgfscope}%
\definecolor{textcolor}{rgb}{0.150000,0.150000,0.150000}%
\pgfsetstrokecolor{textcolor}%
\pgfsetfillcolor{textcolor}%
\pgftext[x=3.975133in,y=0.424381in,,top]{\color{textcolor}\rmfamily\fontsize{10.000000}{12.000000}\selectfont 2015}%
\end{pgfscope}%
\begin{pgfscope}%
\pgfpathrectangle{\pgfqpoint{0.875133in}{0.521603in}}{\pgfqpoint{6.200000in}{3.020000in}}%
\pgfusepath{clip}%
\pgfsetroundcap%
\pgfsetroundjoin%
\pgfsetlinewidth{0.803000pt}%
\definecolor{currentstroke}{rgb}{1.000000,1.000000,1.000000}%
\pgfsetstrokecolor{currentstroke}%
\pgfsetdash{}{0pt}%
\pgfpathmoveto{\pgfqpoint{4.916246in}{0.521603in}}%
\pgfpathlineto{\pgfqpoint{4.916246in}{3.541603in}}%
\pgfusepath{stroke}%
\end{pgfscope}%
\begin{pgfscope}%
\definecolor{textcolor}{rgb}{0.150000,0.150000,0.150000}%
\pgfsetstrokecolor{textcolor}%
\pgfsetfillcolor{textcolor}%
\pgftext[x=4.916246in,y=0.424381in,,top]{\color{textcolor}\rmfamily\fontsize{10.000000}{12.000000}\selectfont 2016}%
\end{pgfscope}%
\begin{pgfscope}%
\pgfpathrectangle{\pgfqpoint{0.875133in}{0.521603in}}{\pgfqpoint{6.200000in}{3.020000in}}%
\pgfusepath{clip}%
\pgfsetroundcap%
\pgfsetroundjoin%
\pgfsetlinewidth{0.803000pt}%
\definecolor{currentstroke}{rgb}{1.000000,1.000000,1.000000}%
\pgfsetstrokecolor{currentstroke}%
\pgfsetdash{}{0pt}%
\pgfpathmoveto{\pgfqpoint{5.859937in}{0.521603in}}%
\pgfpathlineto{\pgfqpoint{5.859937in}{3.541603in}}%
\pgfusepath{stroke}%
\end{pgfscope}%
\begin{pgfscope}%
\definecolor{textcolor}{rgb}{0.150000,0.150000,0.150000}%
\pgfsetstrokecolor{textcolor}%
\pgfsetfillcolor{textcolor}%
\pgftext[x=5.859937in,y=0.424381in,,top]{\color{textcolor}\rmfamily\fontsize{10.000000}{12.000000}\selectfont 2017}%
\end{pgfscope}%
\begin{pgfscope}%
\pgfpathrectangle{\pgfqpoint{0.875133in}{0.521603in}}{\pgfqpoint{6.200000in}{3.020000in}}%
\pgfusepath{clip}%
\pgfsetroundcap%
\pgfsetroundjoin%
\pgfsetlinewidth{0.803000pt}%
\definecolor{currentstroke}{rgb}{1.000000,1.000000,1.000000}%
\pgfsetstrokecolor{currentstroke}%
\pgfsetdash{}{0pt}%
\pgfpathmoveto{\pgfqpoint{6.801050in}{0.521603in}}%
\pgfpathlineto{\pgfqpoint{6.801050in}{3.541603in}}%
\pgfusepath{stroke}%
\end{pgfscope}%
\begin{pgfscope}%
\definecolor{textcolor}{rgb}{0.150000,0.150000,0.150000}%
\pgfsetstrokecolor{textcolor}%
\pgfsetfillcolor{textcolor}%
\pgftext[x=6.801050in,y=0.424381in,,top]{\color{textcolor}\rmfamily\fontsize{10.000000}{12.000000}\selectfont 2018}%
\end{pgfscope}%
\begin{pgfscope}%
\definecolor{textcolor}{rgb}{0.150000,0.150000,0.150000}%
\pgfsetstrokecolor{textcolor}%
\pgfsetfillcolor{textcolor}%
\pgftext[x=3.975133in,y=0.234413in,,top]{\color{textcolor}\rmfamily\fontsize{10.000000}{12.000000}\selectfont Year}%
\end{pgfscope}%
\begin{pgfscope}%
\pgfpathrectangle{\pgfqpoint{0.875133in}{0.521603in}}{\pgfqpoint{6.200000in}{3.020000in}}%
\pgfusepath{clip}%
\pgfsetroundcap%
\pgfsetroundjoin%
\pgfsetlinewidth{0.803000pt}%
\definecolor{currentstroke}{rgb}{1.000000,1.000000,1.000000}%
\pgfsetstrokecolor{currentstroke}%
\pgfsetdash{}{0pt}%
\pgfpathmoveto{\pgfqpoint{0.875133in}{0.658876in}}%
\pgfpathlineto{\pgfqpoint{7.075133in}{0.658876in}}%
\pgfusepath{stroke}%
\end{pgfscope}%
\begin{pgfscope}%
\definecolor{textcolor}{rgb}{0.150000,0.150000,0.150000}%
\pgfsetstrokecolor{textcolor}%
\pgfsetfillcolor{textcolor}%
\pgftext[x=0.291935in,y=0.606115in,left,base]{\color{textcolor}\rmfamily\fontsize{10.000000}{12.000000}\selectfont 0.0000}%
\end{pgfscope}%
\begin{pgfscope}%
\pgfpathrectangle{\pgfqpoint{0.875133in}{0.521603in}}{\pgfqpoint{6.200000in}{3.020000in}}%
\pgfusepath{clip}%
\pgfsetroundcap%
\pgfsetroundjoin%
\pgfsetlinewidth{0.803000pt}%
\definecolor{currentstroke}{rgb}{1.000000,1.000000,1.000000}%
\pgfsetstrokecolor{currentstroke}%
\pgfsetdash{}{0pt}%
\pgfpathmoveto{\pgfqpoint{0.875133in}{1.031572in}}%
\pgfpathlineto{\pgfqpoint{7.075133in}{1.031572in}}%
\pgfusepath{stroke}%
\end{pgfscope}%
\begin{pgfscope}%
\definecolor{textcolor}{rgb}{0.150000,0.150000,0.150000}%
\pgfsetstrokecolor{textcolor}%
\pgfsetfillcolor{textcolor}%
\pgftext[x=0.291935in,y=0.978810in,left,base]{\color{textcolor}\rmfamily\fontsize{10.000000}{12.000000}\selectfont 0.0005}%
\end{pgfscope}%
\begin{pgfscope}%
\pgfpathrectangle{\pgfqpoint{0.875133in}{0.521603in}}{\pgfqpoint{6.200000in}{3.020000in}}%
\pgfusepath{clip}%
\pgfsetroundcap%
\pgfsetroundjoin%
\pgfsetlinewidth{0.803000pt}%
\definecolor{currentstroke}{rgb}{1.000000,1.000000,1.000000}%
\pgfsetstrokecolor{currentstroke}%
\pgfsetdash{}{0pt}%
\pgfpathmoveto{\pgfqpoint{0.875133in}{1.404268in}}%
\pgfpathlineto{\pgfqpoint{7.075133in}{1.404268in}}%
\pgfusepath{stroke}%
\end{pgfscope}%
\begin{pgfscope}%
\definecolor{textcolor}{rgb}{0.150000,0.150000,0.150000}%
\pgfsetstrokecolor{textcolor}%
\pgfsetfillcolor{textcolor}%
\pgftext[x=0.291935in,y=1.351506in,left,base]{\color{textcolor}\rmfamily\fontsize{10.000000}{12.000000}\selectfont 0.0010}%
\end{pgfscope}%
\begin{pgfscope}%
\pgfpathrectangle{\pgfqpoint{0.875133in}{0.521603in}}{\pgfqpoint{6.200000in}{3.020000in}}%
\pgfusepath{clip}%
\pgfsetroundcap%
\pgfsetroundjoin%
\pgfsetlinewidth{0.803000pt}%
\definecolor{currentstroke}{rgb}{1.000000,1.000000,1.000000}%
\pgfsetstrokecolor{currentstroke}%
\pgfsetdash{}{0pt}%
\pgfpathmoveto{\pgfqpoint{0.875133in}{1.776964in}}%
\pgfpathlineto{\pgfqpoint{7.075133in}{1.776964in}}%
\pgfusepath{stroke}%
\end{pgfscope}%
\begin{pgfscope}%
\definecolor{textcolor}{rgb}{0.150000,0.150000,0.150000}%
\pgfsetstrokecolor{textcolor}%
\pgfsetfillcolor{textcolor}%
\pgftext[x=0.291935in,y=1.724202in,left,base]{\color{textcolor}\rmfamily\fontsize{10.000000}{12.000000}\selectfont 0.0015}%
\end{pgfscope}%
\begin{pgfscope}%
\pgfpathrectangle{\pgfqpoint{0.875133in}{0.521603in}}{\pgfqpoint{6.200000in}{3.020000in}}%
\pgfusepath{clip}%
\pgfsetroundcap%
\pgfsetroundjoin%
\pgfsetlinewidth{0.803000pt}%
\definecolor{currentstroke}{rgb}{1.000000,1.000000,1.000000}%
\pgfsetstrokecolor{currentstroke}%
\pgfsetdash{}{0pt}%
\pgfpathmoveto{\pgfqpoint{0.875133in}{2.149660in}}%
\pgfpathlineto{\pgfqpoint{7.075133in}{2.149660in}}%
\pgfusepath{stroke}%
\end{pgfscope}%
\begin{pgfscope}%
\definecolor{textcolor}{rgb}{0.150000,0.150000,0.150000}%
\pgfsetstrokecolor{textcolor}%
\pgfsetfillcolor{textcolor}%
\pgftext[x=0.291935in,y=2.096898in,left,base]{\color{textcolor}\rmfamily\fontsize{10.000000}{12.000000}\selectfont 0.0020}%
\end{pgfscope}%
\begin{pgfscope}%
\pgfpathrectangle{\pgfqpoint{0.875133in}{0.521603in}}{\pgfqpoint{6.200000in}{3.020000in}}%
\pgfusepath{clip}%
\pgfsetroundcap%
\pgfsetroundjoin%
\pgfsetlinewidth{0.803000pt}%
\definecolor{currentstroke}{rgb}{1.000000,1.000000,1.000000}%
\pgfsetstrokecolor{currentstroke}%
\pgfsetdash{}{0pt}%
\pgfpathmoveto{\pgfqpoint{0.875133in}{2.522355in}}%
\pgfpathlineto{\pgfqpoint{7.075133in}{2.522355in}}%
\pgfusepath{stroke}%
\end{pgfscope}%
\begin{pgfscope}%
\definecolor{textcolor}{rgb}{0.150000,0.150000,0.150000}%
\pgfsetstrokecolor{textcolor}%
\pgfsetfillcolor{textcolor}%
\pgftext[x=0.291935in,y=2.469594in,left,base]{\color{textcolor}\rmfamily\fontsize{10.000000}{12.000000}\selectfont 0.0025}%
\end{pgfscope}%
\begin{pgfscope}%
\pgfpathrectangle{\pgfqpoint{0.875133in}{0.521603in}}{\pgfqpoint{6.200000in}{3.020000in}}%
\pgfusepath{clip}%
\pgfsetroundcap%
\pgfsetroundjoin%
\pgfsetlinewidth{0.803000pt}%
\definecolor{currentstroke}{rgb}{1.000000,1.000000,1.000000}%
\pgfsetstrokecolor{currentstroke}%
\pgfsetdash{}{0pt}%
\pgfpathmoveto{\pgfqpoint{0.875133in}{2.895051in}}%
\pgfpathlineto{\pgfqpoint{7.075133in}{2.895051in}}%
\pgfusepath{stroke}%
\end{pgfscope}%
\begin{pgfscope}%
\definecolor{textcolor}{rgb}{0.150000,0.150000,0.150000}%
\pgfsetstrokecolor{textcolor}%
\pgfsetfillcolor{textcolor}%
\pgftext[x=0.291935in,y=2.842290in,left,base]{\color{textcolor}\rmfamily\fontsize{10.000000}{12.000000}\selectfont 0.0030}%
\end{pgfscope}%
\begin{pgfscope}%
\pgfpathrectangle{\pgfqpoint{0.875133in}{0.521603in}}{\pgfqpoint{6.200000in}{3.020000in}}%
\pgfusepath{clip}%
\pgfsetroundcap%
\pgfsetroundjoin%
\pgfsetlinewidth{0.803000pt}%
\definecolor{currentstroke}{rgb}{1.000000,1.000000,1.000000}%
\pgfsetstrokecolor{currentstroke}%
\pgfsetdash{}{0pt}%
\pgfpathmoveto{\pgfqpoint{0.875133in}{3.267747in}}%
\pgfpathlineto{\pgfqpoint{7.075133in}{3.267747in}}%
\pgfusepath{stroke}%
\end{pgfscope}%
\begin{pgfscope}%
\definecolor{textcolor}{rgb}{0.150000,0.150000,0.150000}%
\pgfsetstrokecolor{textcolor}%
\pgfsetfillcolor{textcolor}%
\pgftext[x=0.291935in,y=3.214986in,left,base]{\color{textcolor}\rmfamily\fontsize{10.000000}{12.000000}\selectfont 0.0035}%
\end{pgfscope}%
\begin{pgfscope}%
\definecolor{textcolor}{rgb}{0.150000,0.150000,0.150000}%
\pgfsetstrokecolor{textcolor}%
\pgfsetfillcolor{textcolor}%
\pgftext[x=0.236379in,y=2.031603in,,bottom,rotate=90.000000]{\color{textcolor}\rmfamily\fontsize{10.000000}{12.000000}\selectfont Squared log-returns}%
\end{pgfscope}%
\begin{pgfscope}%
\pgfpathrectangle{\pgfqpoint{0.875133in}{0.521603in}}{\pgfqpoint{6.200000in}{3.020000in}}%
\pgfusepath{clip}%
\pgfsetroundcap%
\pgfsetroundjoin%
\pgfsetlinewidth{1.505625pt}%
\definecolor{currentstroke}{rgb}{0.549020,0.337255,0.294118}%
\pgfsetstrokecolor{currentstroke}%
\pgfsetdash{}{0pt}%
\pgfpathmoveto{\pgfqpoint{1.156951in}{0.659122in}}%
\pgfpathlineto{\pgfqpoint{1.159529in}{0.672162in}}%
\pgfpathlineto{\pgfqpoint{1.162108in}{0.662855in}}%
\pgfpathlineto{\pgfqpoint{1.172421in}{0.676122in}}%
\pgfpathlineto{\pgfqpoint{1.175000in}{0.728865in}}%
\pgfpathlineto{\pgfqpoint{1.177578in}{0.661697in}}%
\pgfpathlineto{\pgfqpoint{1.180156in}{0.658876in}}%
\pgfpathlineto{\pgfqpoint{1.190470in}{0.693137in}}%
\pgfpathlineto{\pgfqpoint{1.193048in}{0.673506in}}%
\pgfpathlineto{\pgfqpoint{1.195627in}{0.659316in}}%
\pgfpathlineto{\pgfqpoint{1.198205in}{0.662826in}}%
\pgfpathlineto{\pgfqpoint{1.205940in}{0.921231in}}%
\pgfpathlineto{\pgfqpoint{1.208519in}{0.702422in}}%
\pgfpathlineto{\pgfqpoint{1.211097in}{0.700226in}}%
\pgfpathlineto{\pgfqpoint{1.213675in}{0.665290in}}%
\pgfpathlineto{\pgfqpoint{1.216254in}{0.702697in}}%
\pgfpathlineto{\pgfqpoint{1.223989in}{0.877302in}}%
\pgfpathlineto{\pgfqpoint{1.226567in}{0.663967in}}%
\pgfpathlineto{\pgfqpoint{1.229146in}{0.663967in}}%
\pgfpathlineto{\pgfqpoint{1.231724in}{0.661305in}}%
\pgfpathlineto{\pgfqpoint{1.234303in}{0.717346in}}%
\pgfpathlineto{\pgfqpoint{1.242038in}{0.763692in}}%
\pgfpathlineto{\pgfqpoint{1.244616in}{0.665551in}}%
\pgfpathlineto{\pgfqpoint{1.247195in}{0.659616in}}%
\pgfpathlineto{\pgfqpoint{1.249773in}{0.687183in}}%
\pgfpathlineto{\pgfqpoint{1.252351in}{0.663102in}}%
\pgfpathlineto{\pgfqpoint{1.260086in}{0.680205in}}%
\pgfpathlineto{\pgfqpoint{1.265243in}{0.659914in}}%
\pgfpathlineto{\pgfqpoint{1.267822in}{0.733052in}}%
\pgfpathlineto{\pgfqpoint{1.270400in}{0.673880in}}%
\pgfpathlineto{\pgfqpoint{1.280714in}{0.702542in}}%
\pgfpathlineto{\pgfqpoint{1.283292in}{0.658992in}}%
\pgfpathlineto{\pgfqpoint{1.285870in}{1.341331in}}%
\pgfpathlineto{\pgfqpoint{1.288449in}{0.673207in}}%
\pgfpathlineto{\pgfqpoint{1.296184in}{0.658903in}}%
\pgfpathlineto{\pgfqpoint{1.298762in}{0.736755in}}%
\pgfpathlineto{\pgfqpoint{1.301341in}{0.668385in}}%
\pgfpathlineto{\pgfqpoint{1.303919in}{0.812674in}}%
\pgfpathlineto{\pgfqpoint{1.306498in}{0.658903in}}%
\pgfpathlineto{\pgfqpoint{1.314233in}{0.671891in}}%
\pgfpathlineto{\pgfqpoint{1.316811in}{0.661049in}}%
\pgfpathlineto{\pgfqpoint{1.319389in}{0.669665in}}%
\pgfpathlineto{\pgfqpoint{1.321968in}{0.675717in}}%
\pgfpathlineto{\pgfqpoint{1.324546in}{0.658983in}}%
\pgfpathlineto{\pgfqpoint{1.332281in}{0.760690in}}%
\pgfpathlineto{\pgfqpoint{1.334860in}{0.664750in}}%
\pgfpathlineto{\pgfqpoint{1.337438in}{0.659293in}}%
\pgfpathlineto{\pgfqpoint{1.340017in}{0.663999in}}%
\pgfpathlineto{\pgfqpoint{1.342595in}{0.689367in}}%
\pgfpathlineto{\pgfqpoint{1.350330in}{0.659115in}}%
\pgfpathlineto{\pgfqpoint{1.355487in}{0.658903in}}%
\pgfpathlineto{\pgfqpoint{1.358065in}{0.676760in}}%
\pgfpathlineto{\pgfqpoint{1.360644in}{0.660168in}}%
\pgfpathlineto{\pgfqpoint{1.368379in}{0.658982in}}%
\pgfpathlineto{\pgfqpoint{1.370957in}{0.674138in}}%
\pgfpathlineto{\pgfqpoint{1.373536in}{0.659115in}}%
\pgfpathlineto{\pgfqpoint{1.376114in}{0.664875in}}%
\pgfpathlineto{\pgfqpoint{1.378692in}{0.665700in}}%
\pgfpathlineto{\pgfqpoint{1.386428in}{0.679602in}}%
\pgfpathlineto{\pgfqpoint{1.389006in}{0.695129in}}%
\pgfpathlineto{\pgfqpoint{1.391584in}{0.663372in}}%
\pgfpathlineto{\pgfqpoint{1.394163in}{0.659300in}}%
\pgfpathlineto{\pgfqpoint{1.404476in}{0.699475in}}%
\pgfpathlineto{\pgfqpoint{1.407055in}{0.695949in}}%
\pgfpathlineto{\pgfqpoint{1.409633in}{0.660619in}}%
\pgfpathlineto{\pgfqpoint{1.412212in}{0.680306in}}%
\pgfpathlineto{\pgfqpoint{1.414790in}{0.672234in}}%
\pgfpathlineto{\pgfqpoint{1.422525in}{0.816645in}}%
\pgfpathlineto{\pgfqpoint{1.425104in}{0.668559in}}%
\pgfpathlineto{\pgfqpoint{1.427682in}{0.670710in}}%
\pgfpathlineto{\pgfqpoint{1.430260in}{0.664171in}}%
\pgfpathlineto{\pgfqpoint{1.432839in}{0.805128in}}%
\pgfpathlineto{\pgfqpoint{1.440574in}{0.782232in}}%
\pgfpathlineto{\pgfqpoint{1.443152in}{0.679950in}}%
\pgfpathlineto{\pgfqpoint{1.445731in}{0.693293in}}%
\pgfpathlineto{\pgfqpoint{1.448309in}{0.658982in}}%
\pgfpathlineto{\pgfqpoint{1.450887in}{1.679048in}}%
\pgfpathlineto{\pgfqpoint{1.458623in}{0.776770in}}%
\pgfpathlineto{\pgfqpoint{1.461201in}{0.659605in}}%
\pgfpathlineto{\pgfqpoint{1.463779in}{0.688594in}}%
\pgfpathlineto{\pgfqpoint{1.466358in}{0.709330in}}%
\pgfpathlineto{\pgfqpoint{1.468936in}{0.668100in}}%
\pgfpathlineto{\pgfqpoint{1.476671in}{0.658990in}}%
\pgfpathlineto{\pgfqpoint{1.479250in}{0.660279in}}%
\pgfpathlineto{\pgfqpoint{1.481828in}{0.702818in}}%
\pgfpathlineto{\pgfqpoint{1.484407in}{0.698442in}}%
\pgfpathlineto{\pgfqpoint{1.486985in}{0.696325in}}%
\pgfpathlineto{\pgfqpoint{1.494720in}{0.660741in}}%
\pgfpathlineto{\pgfqpoint{1.497298in}{0.662401in}}%
\pgfpathlineto{\pgfqpoint{1.499877in}{0.717218in}}%
\pgfpathlineto{\pgfqpoint{1.502455in}{0.678279in}}%
\pgfpathlineto{\pgfqpoint{1.505034in}{0.694462in}}%
\pgfpathlineto{\pgfqpoint{1.512769in}{0.662423in}}%
\pgfpathlineto{\pgfqpoint{1.515347in}{0.669522in}}%
\pgfpathlineto{\pgfqpoint{1.517926in}{0.766732in}}%
\pgfpathlineto{\pgfqpoint{1.520504in}{0.664803in}}%
\pgfpathlineto{\pgfqpoint{1.523082in}{0.659963in}}%
\pgfpathlineto{\pgfqpoint{1.533396in}{0.699948in}}%
\pgfpathlineto{\pgfqpoint{1.535974in}{0.737131in}}%
\pgfpathlineto{\pgfqpoint{1.538553in}{0.658998in}}%
\pgfpathlineto{\pgfqpoint{1.541131in}{0.766048in}}%
\pgfpathlineto{\pgfqpoint{1.548866in}{0.663374in}}%
\pgfpathlineto{\pgfqpoint{1.551445in}{0.669059in}}%
\pgfpathlineto{\pgfqpoint{1.554023in}{0.736935in}}%
\pgfpathlineto{\pgfqpoint{1.556601in}{0.834666in}}%
\pgfpathlineto{\pgfqpoint{1.559180in}{0.658876in}}%
\pgfpathlineto{\pgfqpoint{1.566915in}{0.667567in}}%
\pgfpathlineto{\pgfqpoint{1.569493in}{0.667567in}}%
\pgfpathlineto{\pgfqpoint{1.572072in}{0.665640in}}%
\pgfpathlineto{\pgfqpoint{1.574650in}{0.730565in}}%
\pgfpathlineto{\pgfqpoint{1.577229in}{0.675988in}}%
\pgfpathlineto{\pgfqpoint{1.584964in}{0.725444in}}%
\pgfpathlineto{\pgfqpoint{1.587542in}{0.659973in}}%
\pgfpathlineto{\pgfqpoint{1.590121in}{1.319355in}}%
\pgfpathlineto{\pgfqpoint{1.592699in}{0.743962in}}%
\pgfpathlineto{\pgfqpoint{1.595277in}{0.660494in}}%
\pgfpathlineto{\pgfqpoint{1.603013in}{0.717548in}}%
\pgfpathlineto{\pgfqpoint{1.605591in}{0.659178in}}%
\pgfpathlineto{\pgfqpoint{1.608169in}{0.766737in}}%
\pgfpathlineto{\pgfqpoint{1.610748in}{0.677675in}}%
\pgfpathlineto{\pgfqpoint{1.613326in}{0.843512in}}%
\pgfpathlineto{\pgfqpoint{1.621061in}{0.659380in}}%
\pgfpathlineto{\pgfqpoint{1.623640in}{0.664189in}}%
\pgfpathlineto{\pgfqpoint{1.628796in}{0.658907in}}%
\pgfpathlineto{\pgfqpoint{1.631375in}{0.659661in}}%
\pgfpathlineto{\pgfqpoint{1.639110in}{0.672677in}}%
\pgfpathlineto{\pgfqpoint{1.641688in}{0.665865in}}%
\pgfpathlineto{\pgfqpoint{1.644267in}{0.681577in}}%
\pgfpathlineto{\pgfqpoint{1.646845in}{1.669489in}}%
\pgfpathlineto{\pgfqpoint{1.649424in}{1.009587in}}%
\pgfpathlineto{\pgfqpoint{1.657159in}{0.673677in}}%
\pgfpathlineto{\pgfqpoint{1.659737in}{0.710410in}}%
\pgfpathlineto{\pgfqpoint{1.662316in}{0.659125in}}%
\pgfpathlineto{\pgfqpoint{1.664894in}{0.660640in}}%
\pgfpathlineto{\pgfqpoint{1.667472in}{0.665087in}}%
\pgfpathlineto{\pgfqpoint{1.675207in}{0.680697in}}%
\pgfpathlineto{\pgfqpoint{1.677786in}{0.682544in}}%
\pgfpathlineto{\pgfqpoint{1.680364in}{0.658876in}}%
\pgfpathlineto{\pgfqpoint{1.682943in}{0.710771in}}%
\pgfpathlineto{\pgfqpoint{1.685521in}{0.707578in}}%
\pgfpathlineto{\pgfqpoint{1.693256in}{0.658903in}}%
\pgfpathlineto{\pgfqpoint{1.695835in}{0.714795in}}%
\pgfpathlineto{\pgfqpoint{1.698413in}{0.708414in}}%
\pgfpathlineto{\pgfqpoint{1.700991in}{0.704532in}}%
\pgfpathlineto{\pgfqpoint{1.703570in}{1.363865in}}%
\pgfpathlineto{\pgfqpoint{1.711305in}{0.675698in}}%
\pgfpathlineto{\pgfqpoint{1.716462in}{0.700889in}}%
\pgfpathlineto{\pgfqpoint{1.719040in}{0.658902in}}%
\pgfpathlineto{\pgfqpoint{1.721619in}{0.659110in}}%
\pgfpathlineto{\pgfqpoint{1.729354in}{0.672701in}}%
\pgfpathlineto{\pgfqpoint{1.731932in}{0.669336in}}%
\pgfpathlineto{\pgfqpoint{1.734510in}{0.660154in}}%
\pgfpathlineto{\pgfqpoint{1.737089in}{0.680725in}}%
\pgfpathlineto{\pgfqpoint{1.739667in}{0.658876in}}%
\pgfpathlineto{\pgfqpoint{1.747402in}{0.669250in}}%
\pgfpathlineto{\pgfqpoint{1.749981in}{0.658902in}}%
\pgfpathlineto{\pgfqpoint{1.752559in}{0.660150in}}%
\pgfpathlineto{\pgfqpoint{1.755138in}{0.663977in}}%
\pgfpathlineto{\pgfqpoint{1.757716in}{0.677801in}}%
\pgfpathlineto{\pgfqpoint{1.765451in}{0.660140in}}%
\pgfpathlineto{\pgfqpoint{1.770608in}{0.661466in}}%
\pgfpathlineto{\pgfqpoint{1.773186in}{0.658902in}}%
\pgfpathlineto{\pgfqpoint{1.775765in}{0.675010in}}%
\pgfpathlineto{\pgfqpoint{1.786078in}{0.666279in}}%
\pgfpathlineto{\pgfqpoint{1.788657in}{0.660948in}}%
\pgfpathlineto{\pgfqpoint{1.791235in}{0.804793in}}%
\pgfpathlineto{\pgfqpoint{1.793814in}{0.670885in}}%
\pgfpathlineto{\pgfqpoint{1.801549in}{0.658876in}}%
\pgfpathlineto{\pgfqpoint{1.804127in}{0.668797in}}%
\pgfpathlineto{\pgfqpoint{1.806705in}{0.663093in}}%
\pgfpathlineto{\pgfqpoint{1.809284in}{0.763309in}}%
\pgfpathlineto{\pgfqpoint{1.811862in}{0.668612in}}%
\pgfpathlineto{\pgfqpoint{1.819597in}{0.660063in}}%
\pgfpathlineto{\pgfqpoint{1.822176in}{0.658973in}}%
\pgfpathlineto{\pgfqpoint{1.824754in}{0.659094in}}%
\pgfpathlineto{\pgfqpoint{1.827333in}{0.672745in}}%
\pgfpathlineto{\pgfqpoint{1.829911in}{0.661783in}}%
\pgfpathlineto{\pgfqpoint{1.837646in}{0.676337in}}%
\pgfpathlineto{\pgfqpoint{1.840225in}{0.663560in}}%
\pgfpathlineto{\pgfqpoint{1.842803in}{0.671602in}}%
\pgfpathlineto{\pgfqpoint{1.845381in}{0.658876in}}%
\pgfpathlineto{\pgfqpoint{1.847960in}{0.659479in}}%
\pgfpathlineto{\pgfqpoint{1.855695in}{0.660056in}}%
\pgfpathlineto{\pgfqpoint{1.858273in}{0.727086in}}%
\pgfpathlineto{\pgfqpoint{1.860852in}{0.680821in}}%
\pgfpathlineto{\pgfqpoint{1.863430in}{0.665863in}}%
\pgfpathlineto{\pgfqpoint{1.866008in}{0.668483in}}%
\pgfpathlineto{\pgfqpoint{1.873744in}{0.701410in}}%
\pgfpathlineto{\pgfqpoint{1.876322in}{0.683899in}}%
\pgfpathlineto{\pgfqpoint{1.878900in}{0.709057in}}%
\pgfpathlineto{\pgfqpoint{1.881479in}{0.661905in}}%
\pgfpathlineto{\pgfqpoint{1.884057in}{0.659504in}}%
\pgfpathlineto{\pgfqpoint{1.891792in}{0.754395in}}%
\pgfpathlineto{\pgfqpoint{1.894371in}{0.671819in}}%
\pgfpathlineto{\pgfqpoint{1.896949in}{0.824246in}}%
\pgfpathlineto{\pgfqpoint{1.899528in}{0.658876in}}%
\pgfpathlineto{\pgfqpoint{1.902106in}{0.783071in}}%
\pgfpathlineto{\pgfqpoint{1.909841in}{0.659751in}}%
\pgfpathlineto{\pgfqpoint{1.912420in}{0.841594in}}%
\pgfpathlineto{\pgfqpoint{1.914998in}{0.726092in}}%
\pgfpathlineto{\pgfqpoint{1.917576in}{1.278944in}}%
\pgfpathlineto{\pgfqpoint{1.920155in}{0.719880in}}%
\pgfpathlineto{\pgfqpoint{1.933047in}{0.664952in}}%
\pgfpathlineto{\pgfqpoint{1.935625in}{0.658876in}}%
\pgfpathlineto{\pgfqpoint{1.938203in}{0.659257in}}%
\pgfpathlineto{\pgfqpoint{1.945939in}{0.701242in}}%
\pgfpathlineto{\pgfqpoint{1.948517in}{0.669508in}}%
\pgfpathlineto{\pgfqpoint{1.953674in}{0.875657in}}%
\pgfpathlineto{\pgfqpoint{1.956252in}{0.660505in}}%
\pgfpathlineto{\pgfqpoint{1.963987in}{0.659790in}}%
\pgfpathlineto{\pgfqpoint{1.966566in}{0.667122in}}%
\pgfpathlineto{\pgfqpoint{1.969144in}{0.676220in}}%
\pgfpathlineto{\pgfqpoint{1.971723in}{0.666350in}}%
\pgfpathlineto{\pgfqpoint{1.974301in}{0.700080in}}%
\pgfpathlineto{\pgfqpoint{1.982036in}{0.858066in}}%
\pgfpathlineto{\pgfqpoint{1.984614in}{0.682518in}}%
\pgfpathlineto{\pgfqpoint{1.987193in}{0.663001in}}%
\pgfpathlineto{\pgfqpoint{1.992350in}{0.857235in}}%
\pgfpathlineto{\pgfqpoint{2.000085in}{0.660788in}}%
\pgfpathlineto{\pgfqpoint{2.002663in}{0.696975in}}%
\pgfpathlineto{\pgfqpoint{2.005242in}{0.691492in}}%
\pgfpathlineto{\pgfqpoint{2.007820in}{0.659467in}}%
\pgfpathlineto{\pgfqpoint{2.010398in}{0.674773in}}%
\pgfpathlineto{\pgfqpoint{2.018134in}{0.668273in}}%
\pgfpathlineto{\pgfqpoint{2.020712in}{0.670331in}}%
\pgfpathlineto{\pgfqpoint{2.023290in}{0.660395in}}%
\pgfpathlineto{\pgfqpoint{2.025869in}{0.704388in}}%
\pgfpathlineto{\pgfqpoint{2.028447in}{0.677073in}}%
\pgfpathlineto{\pgfqpoint{2.036182in}{0.659708in}}%
\pgfpathlineto{\pgfqpoint{2.038761in}{0.687055in}}%
\pgfpathlineto{\pgfqpoint{2.041339in}{0.660725in}}%
\pgfpathlineto{\pgfqpoint{2.043917in}{0.718686in}}%
\pgfpathlineto{\pgfqpoint{2.046496in}{0.665602in}}%
\pgfpathlineto{\pgfqpoint{2.054231in}{0.658876in}}%
\pgfpathlineto{\pgfqpoint{2.056809in}{0.659249in}}%
\pgfpathlineto{\pgfqpoint{2.059388in}{0.720053in}}%
\pgfpathlineto{\pgfqpoint{2.061966in}{0.694727in}}%
\pgfpathlineto{\pgfqpoint{2.064545in}{0.847258in}}%
\pgfpathlineto{\pgfqpoint{2.072280in}{0.665881in}}%
\pgfpathlineto{\pgfqpoint{2.077437in}{0.702091in}}%
\pgfpathlineto{\pgfqpoint{2.080015in}{0.658975in}}%
\pgfpathlineto{\pgfqpoint{2.082593in}{0.767794in}}%
\pgfpathlineto{\pgfqpoint{2.090329in}{0.748989in}}%
\pgfpathlineto{\pgfqpoint{2.095485in}{1.013522in}}%
\pgfpathlineto{\pgfqpoint{2.098064in}{0.689791in}}%
\pgfpathlineto{\pgfqpoint{2.100642in}{0.662326in}}%
\pgfpathlineto{\pgfqpoint{2.108377in}{0.693632in}}%
\pgfpathlineto{\pgfqpoint{2.110956in}{0.660842in}}%
\pgfpathlineto{\pgfqpoint{2.113534in}{0.680640in}}%
\pgfpathlineto{\pgfqpoint{2.116112in}{0.681861in}}%
\pgfpathlineto{\pgfqpoint{2.118691in}{0.659257in}}%
\pgfpathlineto{\pgfqpoint{2.126426in}{0.684664in}}%
\pgfpathlineto{\pgfqpoint{2.129004in}{0.668259in}}%
\pgfpathlineto{\pgfqpoint{2.131583in}{0.658970in}}%
\pgfpathlineto{\pgfqpoint{2.134161in}{0.675823in}}%
\pgfpathlineto{\pgfqpoint{2.136740in}{0.670031in}}%
\pgfpathlineto{\pgfqpoint{2.147053in}{0.658899in}}%
\pgfpathlineto{\pgfqpoint{2.149632in}{0.740635in}}%
\pgfpathlineto{\pgfqpoint{2.152210in}{0.669795in}}%
\pgfpathlineto{\pgfqpoint{2.154788in}{1.820428in}}%
\pgfpathlineto{\pgfqpoint{2.162523in}{0.695535in}}%
\pgfpathlineto{\pgfqpoint{2.165102in}{0.861859in}}%
\pgfpathlineto{\pgfqpoint{2.167680in}{0.659853in}}%
\pgfpathlineto{\pgfqpoint{2.170259in}{0.659593in}}%
\pgfpathlineto{\pgfqpoint{2.172837in}{0.734535in}}%
\pgfpathlineto{\pgfqpoint{2.180572in}{0.716195in}}%
\pgfpathlineto{\pgfqpoint{2.183151in}{0.684426in}}%
\pgfpathlineto{\pgfqpoint{2.185729in}{0.685547in}}%
\pgfpathlineto{\pgfqpoint{2.188307in}{0.658876in}}%
\pgfpathlineto{\pgfqpoint{2.190886in}{0.680078in}}%
\pgfpathlineto{\pgfqpoint{2.198621in}{0.659365in}}%
\pgfpathlineto{\pgfqpoint{2.201199in}{0.662698in}}%
\pgfpathlineto{\pgfqpoint{2.203778in}{0.701525in}}%
\pgfpathlineto{\pgfqpoint{2.206356in}{0.665067in}}%
\pgfpathlineto{\pgfqpoint{2.208935in}{0.665775in}}%
\pgfpathlineto{\pgfqpoint{2.219248in}{0.746551in}}%
\pgfpathlineto{\pgfqpoint{2.221826in}{0.669721in}}%
\pgfpathlineto{\pgfqpoint{2.224405in}{0.659046in}}%
\pgfpathlineto{\pgfqpoint{2.226983in}{0.659349in}}%
\pgfpathlineto{\pgfqpoint{2.234718in}{0.804312in}}%
\pgfpathlineto{\pgfqpoint{2.237297in}{0.662162in}}%
\pgfpathlineto{\pgfqpoint{2.239875in}{0.717055in}}%
\pgfpathlineto{\pgfqpoint{2.242454in}{0.701307in}}%
\pgfpathlineto{\pgfqpoint{2.245032in}{0.671904in}}%
\pgfpathlineto{\pgfqpoint{2.252767in}{0.663184in}}%
\pgfpathlineto{\pgfqpoint{2.255346in}{0.675984in}}%
\pgfpathlineto{\pgfqpoint{2.257924in}{0.662067in}}%
\pgfpathlineto{\pgfqpoint{2.260502in}{0.670698in}}%
\pgfpathlineto{\pgfqpoint{2.263081in}{0.668885in}}%
\pgfpathlineto{\pgfqpoint{2.270816in}{0.662563in}}%
\pgfpathlineto{\pgfqpoint{2.273394in}{0.663109in}}%
\pgfpathlineto{\pgfqpoint{2.275973in}{0.675930in}}%
\pgfpathlineto{\pgfqpoint{2.278551in}{0.702409in}}%
\pgfpathlineto{\pgfqpoint{2.281129in}{0.799471in}}%
\pgfpathlineto{\pgfqpoint{2.288865in}{0.662662in}}%
\pgfpathlineto{\pgfqpoint{2.291443in}{0.772252in}}%
\pgfpathlineto{\pgfqpoint{2.294021in}{0.687428in}}%
\pgfpathlineto{\pgfqpoint{2.296600in}{0.676893in}}%
\pgfpathlineto{\pgfqpoint{2.299178in}{0.659347in}}%
\pgfpathlineto{\pgfqpoint{2.306913in}{0.702548in}}%
\pgfpathlineto{\pgfqpoint{2.309492in}{0.724742in}}%
\pgfpathlineto{\pgfqpoint{2.312070in}{0.673637in}}%
\pgfpathlineto{\pgfqpoint{2.314649in}{0.658876in}}%
\pgfpathlineto{\pgfqpoint{2.324962in}{0.709594in}}%
\pgfpathlineto{\pgfqpoint{2.327541in}{0.853069in}}%
\pgfpathlineto{\pgfqpoint{2.330119in}{0.743052in}}%
\pgfpathlineto{\pgfqpoint{2.332697in}{0.680032in}}%
\pgfpathlineto{\pgfqpoint{2.335276in}{0.671232in}}%
\pgfpathlineto{\pgfqpoint{2.343011in}{0.697432in}}%
\pgfpathlineto{\pgfqpoint{2.345589in}{0.692551in}}%
\pgfpathlineto{\pgfqpoint{2.348168in}{0.774766in}}%
\pgfpathlineto{\pgfqpoint{2.350746in}{0.680665in}}%
\pgfpathlineto{\pgfqpoint{2.353324in}{0.678044in}}%
\pgfpathlineto{\pgfqpoint{2.361060in}{0.680445in}}%
\pgfpathlineto{\pgfqpoint{2.363638in}{0.682973in}}%
\pgfpathlineto{\pgfqpoint{2.366216in}{0.786996in}}%
\pgfpathlineto{\pgfqpoint{2.368795in}{0.736346in}}%
\pgfpathlineto{\pgfqpoint{2.371373in}{0.937374in}}%
\pgfpathlineto{\pgfqpoint{2.379108in}{0.658944in}}%
\pgfpathlineto{\pgfqpoint{2.381687in}{0.803422in}}%
\pgfpathlineto{\pgfqpoint{2.384265in}{3.404331in}}%
\pgfpathlineto{\pgfqpoint{2.386844in}{0.695154in}}%
\pgfpathlineto{\pgfqpoint{2.389422in}{0.693530in}}%
\pgfpathlineto{\pgfqpoint{2.397157in}{0.699692in}}%
\pgfpathlineto{\pgfqpoint{2.399735in}{0.763271in}}%
\pgfpathlineto{\pgfqpoint{2.402314in}{0.664944in}}%
\pgfpathlineto{\pgfqpoint{2.404892in}{0.734595in}}%
\pgfpathlineto{\pgfqpoint{2.407471in}{0.681172in}}%
\pgfpathlineto{\pgfqpoint{2.415206in}{0.682468in}}%
\pgfpathlineto{\pgfqpoint{2.417784in}{0.662985in}}%
\pgfpathlineto{\pgfqpoint{2.420363in}{0.689310in}}%
\pgfpathlineto{\pgfqpoint{2.422941in}{0.664721in}}%
\pgfpathlineto{\pgfqpoint{2.425519in}{0.695254in}}%
\pgfpathlineto{\pgfqpoint{2.433255in}{0.662380in}}%
\pgfpathlineto{\pgfqpoint{2.438411in}{0.828053in}}%
\pgfpathlineto{\pgfqpoint{2.440990in}{0.684891in}}%
\pgfpathlineto{\pgfqpoint{2.443568in}{0.662756in}}%
\pgfpathlineto{\pgfqpoint{2.451303in}{0.759878in}}%
\pgfpathlineto{\pgfqpoint{2.453882in}{0.669104in}}%
\pgfpathlineto{\pgfqpoint{2.456460in}{0.658947in}}%
\pgfpathlineto{\pgfqpoint{2.459038in}{0.660660in}}%
\pgfpathlineto{\pgfqpoint{2.461617in}{1.828575in}}%
\pgfpathlineto{\pgfqpoint{2.471930in}{0.776816in}}%
\pgfpathlineto{\pgfqpoint{2.474509in}{1.108520in}}%
\pgfpathlineto{\pgfqpoint{2.477087in}{0.663416in}}%
\pgfpathlineto{\pgfqpoint{2.479666in}{1.323792in}}%
\pgfpathlineto{\pgfqpoint{2.487401in}{0.757809in}}%
\pgfpathlineto{\pgfqpoint{2.489979in}{0.668620in}}%
\pgfpathlineto{\pgfqpoint{2.492558in}{0.723828in}}%
\pgfpathlineto{\pgfqpoint{2.495136in}{0.662559in}}%
\pgfpathlineto{\pgfqpoint{2.497714in}{0.765888in}}%
\pgfpathlineto{\pgfqpoint{2.505450in}{0.671201in}}%
\pgfpathlineto{\pgfqpoint{2.508028in}{0.659166in}}%
\pgfpathlineto{\pgfqpoint{2.510606in}{0.691078in}}%
\pgfpathlineto{\pgfqpoint{2.513185in}{0.742947in}}%
\pgfpathlineto{\pgfqpoint{2.515763in}{0.678567in}}%
\pgfpathlineto{\pgfqpoint{2.523498in}{0.762624in}}%
\pgfpathlineto{\pgfqpoint{2.526077in}{0.659745in}}%
\pgfpathlineto{\pgfqpoint{2.528655in}{0.910077in}}%
\pgfpathlineto{\pgfqpoint{2.531233in}{1.364684in}}%
\pgfpathlineto{\pgfqpoint{2.533812in}{1.267345in}}%
\pgfpathlineto{\pgfqpoint{2.541547in}{0.750320in}}%
\pgfpathlineto{\pgfqpoint{2.544125in}{0.660402in}}%
\pgfpathlineto{\pgfqpoint{2.546704in}{0.721595in}}%
\pgfpathlineto{\pgfqpoint{2.549282in}{0.667787in}}%
\pgfpathlineto{\pgfqpoint{2.551861in}{0.716921in}}%
\pgfpathlineto{\pgfqpoint{2.559596in}{0.791999in}}%
\pgfpathlineto{\pgfqpoint{2.562174in}{0.679782in}}%
\pgfpathlineto{\pgfqpoint{2.564753in}{0.661049in}}%
\pgfpathlineto{\pgfqpoint{2.569909in}{0.665367in}}%
\pgfpathlineto{\pgfqpoint{2.577644in}{0.680848in}}%
\pgfpathlineto{\pgfqpoint{2.580223in}{0.735799in}}%
\pgfpathlineto{\pgfqpoint{2.582801in}{0.665168in}}%
\pgfpathlineto{\pgfqpoint{2.585380in}{0.768760in}}%
\pgfpathlineto{\pgfqpoint{2.587958in}{0.727689in}}%
\pgfpathlineto{\pgfqpoint{2.595693in}{0.659142in}}%
\pgfpathlineto{\pgfqpoint{2.598272in}{0.691342in}}%
\pgfpathlineto{\pgfqpoint{2.600850in}{0.667837in}}%
\pgfpathlineto{\pgfqpoint{2.603428in}{0.663776in}}%
\pgfpathlineto{\pgfqpoint{2.606007in}{0.791057in}}%
\pgfpathlineto{\pgfqpoint{2.613742in}{0.661252in}}%
\pgfpathlineto{\pgfqpoint{2.616320in}{0.667648in}}%
\pgfpathlineto{\pgfqpoint{2.618899in}{0.711498in}}%
\pgfpathlineto{\pgfqpoint{2.621477in}{0.661308in}}%
\pgfpathlineto{\pgfqpoint{2.624056in}{0.658944in}}%
\pgfpathlineto{\pgfqpoint{2.634369in}{0.673109in}}%
\pgfpathlineto{\pgfqpoint{2.636947in}{0.661307in}}%
\pgfpathlineto{\pgfqpoint{2.639526in}{0.863810in}}%
\pgfpathlineto{\pgfqpoint{2.642104in}{0.672694in}}%
\pgfpathlineto{\pgfqpoint{2.649839in}{0.660211in}}%
\pgfpathlineto{\pgfqpoint{2.652418in}{0.672657in}}%
\pgfpathlineto{\pgfqpoint{2.654996in}{0.664148in}}%
\pgfpathlineto{\pgfqpoint{2.657575in}{0.663555in}}%
\pgfpathlineto{\pgfqpoint{2.660153in}{0.690343in}}%
\pgfpathlineto{\pgfqpoint{2.667888in}{0.658892in}}%
\pgfpathlineto{\pgfqpoint{2.670467in}{0.659023in}}%
\pgfpathlineto{\pgfqpoint{2.673045in}{0.677872in}}%
\pgfpathlineto{\pgfqpoint{2.675623in}{0.727176in}}%
\pgfpathlineto{\pgfqpoint{2.678202in}{0.697946in}}%
\pgfpathlineto{\pgfqpoint{2.688515in}{0.659307in}}%
\pgfpathlineto{\pgfqpoint{2.691094in}{0.661795in}}%
\pgfpathlineto{\pgfqpoint{2.693672in}{0.677627in}}%
\pgfpathlineto{\pgfqpoint{2.696250in}{0.665043in}}%
\pgfpathlineto{\pgfqpoint{2.703986in}{0.912915in}}%
\pgfpathlineto{\pgfqpoint{2.706564in}{0.698205in}}%
\pgfpathlineto{\pgfqpoint{2.709142in}{0.816252in}}%
\pgfpathlineto{\pgfqpoint{2.711721in}{0.685377in}}%
\pgfpathlineto{\pgfqpoint{2.714299in}{0.700596in}}%
\pgfpathlineto{\pgfqpoint{2.724613in}{0.661054in}}%
\pgfpathlineto{\pgfqpoint{2.729770in}{0.674217in}}%
\pgfpathlineto{\pgfqpoint{2.732348in}{0.658894in}}%
\pgfpathlineto{\pgfqpoint{2.740083in}{0.786447in}}%
\pgfpathlineto{\pgfqpoint{2.742662in}{0.664674in}}%
\pgfpathlineto{\pgfqpoint{2.745240in}{0.671903in}}%
\pgfpathlineto{\pgfqpoint{2.747818in}{0.658894in}}%
\pgfpathlineto{\pgfqpoint{2.750397in}{0.733336in}}%
\pgfpathlineto{\pgfqpoint{2.758132in}{0.804509in}}%
\pgfpathlineto{\pgfqpoint{2.760710in}{0.671296in}}%
\pgfpathlineto{\pgfqpoint{2.763289in}{0.683437in}}%
\pgfpathlineto{\pgfqpoint{2.765867in}{0.662198in}}%
\pgfpathlineto{\pgfqpoint{2.768445in}{0.722656in}}%
\pgfpathlineto{\pgfqpoint{2.776181in}{0.660279in}}%
\pgfpathlineto{\pgfqpoint{2.781337in}{0.759241in}}%
\pgfpathlineto{\pgfqpoint{2.783916in}{0.671980in}}%
\pgfpathlineto{\pgfqpoint{2.786494in}{0.745020in}}%
\pgfpathlineto{\pgfqpoint{2.794229in}{0.994308in}}%
\pgfpathlineto{\pgfqpoint{2.796808in}{0.700713in}}%
\pgfpathlineto{\pgfqpoint{2.799386in}{0.665683in}}%
\pgfpathlineto{\pgfqpoint{2.801965in}{0.659804in}}%
\pgfpathlineto{\pgfqpoint{2.804543in}{0.663131in}}%
\pgfpathlineto{\pgfqpoint{2.812278in}{0.677094in}}%
\pgfpathlineto{\pgfqpoint{2.814857in}{0.718091in}}%
\pgfpathlineto{\pgfqpoint{2.817435in}{0.707163in}}%
\pgfpathlineto{\pgfqpoint{2.820013in}{0.769561in}}%
\pgfpathlineto{\pgfqpoint{2.822592in}{0.701702in}}%
\pgfpathlineto{\pgfqpoint{2.830327in}{0.666658in}}%
\pgfpathlineto{\pgfqpoint{2.832905in}{0.816560in}}%
\pgfpathlineto{\pgfqpoint{2.835484in}{0.882113in}}%
\pgfpathlineto{\pgfqpoint{2.838062in}{0.798627in}}%
\pgfpathlineto{\pgfqpoint{2.840640in}{0.658893in}}%
\pgfpathlineto{\pgfqpoint{2.848376in}{0.681060in}}%
\pgfpathlineto{\pgfqpoint{2.850954in}{0.890353in}}%
\pgfpathlineto{\pgfqpoint{2.853532in}{0.690832in}}%
\pgfpathlineto{\pgfqpoint{2.856111in}{0.669163in}}%
\pgfpathlineto{\pgfqpoint{2.858689in}{0.702175in}}%
\pgfpathlineto{\pgfqpoint{2.866424in}{0.854965in}}%
\pgfpathlineto{\pgfqpoint{2.871581in}{0.756036in}}%
\pgfpathlineto{\pgfqpoint{2.874160in}{0.725669in}}%
\pgfpathlineto{\pgfqpoint{2.876738in}{0.676716in}}%
\pgfpathlineto{\pgfqpoint{2.884473in}{0.662536in}}%
\pgfpathlineto{\pgfqpoint{2.887051in}{0.659914in}}%
\pgfpathlineto{\pgfqpoint{2.889630in}{0.869434in}}%
\pgfpathlineto{\pgfqpoint{2.892208in}{0.684072in}}%
\pgfpathlineto{\pgfqpoint{2.894787in}{0.662432in}}%
\pgfpathlineto{\pgfqpoint{2.902522in}{0.665203in}}%
\pgfpathlineto{\pgfqpoint{2.907679in}{0.711104in}}%
\pgfpathlineto{\pgfqpoint{2.910257in}{0.729379in}}%
\pgfpathlineto{\pgfqpoint{2.912835in}{0.687927in}}%
\pgfpathlineto{\pgfqpoint{2.920571in}{0.666790in}}%
\pgfpathlineto{\pgfqpoint{2.923149in}{0.664309in}}%
\pgfpathlineto{\pgfqpoint{2.925727in}{0.671516in}}%
\pgfpathlineto{\pgfqpoint{2.928306in}{0.658936in}}%
\pgfpathlineto{\pgfqpoint{2.930884in}{0.667473in}}%
\pgfpathlineto{\pgfqpoint{2.938619in}{0.680240in}}%
\pgfpathlineto{\pgfqpoint{2.941198in}{0.719699in}}%
\pgfpathlineto{\pgfqpoint{2.943776in}{0.672420in}}%
\pgfpathlineto{\pgfqpoint{2.948933in}{0.659254in}}%
\pgfpathlineto{\pgfqpoint{2.956668in}{0.740373in}}%
\pgfpathlineto{\pgfqpoint{2.959246in}{0.683461in}}%
\pgfpathlineto{\pgfqpoint{2.961825in}{0.683461in}}%
\pgfpathlineto{\pgfqpoint{2.964403in}{0.705997in}}%
\pgfpathlineto{\pgfqpoint{2.966982in}{1.018437in}}%
\pgfpathlineto{\pgfqpoint{2.974717in}{0.666126in}}%
\pgfpathlineto{\pgfqpoint{2.977295in}{0.795485in}}%
\pgfpathlineto{\pgfqpoint{2.979874in}{0.673556in}}%
\pgfpathlineto{\pgfqpoint{2.982452in}{0.976332in}}%
\pgfpathlineto{\pgfqpoint{2.985030in}{0.659446in}}%
\pgfpathlineto{\pgfqpoint{2.995344in}{0.727498in}}%
\pgfpathlineto{\pgfqpoint{2.997922in}{0.902519in}}%
\pgfpathlineto{\pgfqpoint{3.000501in}{0.684330in}}%
\pgfpathlineto{\pgfqpoint{3.003079in}{0.659276in}}%
\pgfpathlineto{\pgfqpoint{3.010814in}{0.691543in}}%
\pgfpathlineto{\pgfqpoint{3.013393in}{0.658876in}}%
\pgfpathlineto{\pgfqpoint{3.018549in}{0.682195in}}%
\pgfpathlineto{\pgfqpoint{3.021128in}{0.665936in}}%
\pgfpathlineto{\pgfqpoint{3.028863in}{0.658892in}}%
\pgfpathlineto{\pgfqpoint{3.031441in}{0.697484in}}%
\pgfpathlineto{\pgfqpoint{3.036598in}{0.746124in}}%
\pgfpathlineto{\pgfqpoint{3.039177in}{0.659688in}}%
\pgfpathlineto{\pgfqpoint{3.046912in}{0.663112in}}%
\pgfpathlineto{\pgfqpoint{3.049490in}{0.727946in}}%
\pgfpathlineto{\pgfqpoint{3.052069in}{0.819915in}}%
\pgfpathlineto{\pgfqpoint{3.054647in}{0.662619in}}%
\pgfpathlineto{\pgfqpoint{3.057225in}{0.660539in}}%
\pgfpathlineto{\pgfqpoint{3.064960in}{0.668501in}}%
\pgfpathlineto{\pgfqpoint{3.067539in}{0.744880in}}%
\pgfpathlineto{\pgfqpoint{3.070117in}{0.659681in}}%
\pgfpathlineto{\pgfqpoint{3.072696in}{0.664830in}}%
\pgfpathlineto{\pgfqpoint{3.075274in}{0.713076in}}%
\pgfpathlineto{\pgfqpoint{3.085588in}{0.669352in}}%
\pgfpathlineto{\pgfqpoint{3.088166in}{0.672981in}}%
\pgfpathlineto{\pgfqpoint{3.090744in}{0.776388in}}%
\pgfpathlineto{\pgfqpoint{3.093323in}{0.765399in}}%
\pgfpathlineto{\pgfqpoint{3.101058in}{0.720145in}}%
\pgfpathlineto{\pgfqpoint{3.103636in}{0.708549in}}%
\pgfpathlineto{\pgfqpoint{3.106215in}{0.919313in}}%
\pgfpathlineto{\pgfqpoint{3.108793in}{0.733727in}}%
\pgfpathlineto{\pgfqpoint{3.111372in}{0.666793in}}%
\pgfpathlineto{\pgfqpoint{3.119107in}{0.766961in}}%
\pgfpathlineto{\pgfqpoint{3.121685in}{0.677675in}}%
\pgfpathlineto{\pgfqpoint{3.124263in}{0.676345in}}%
\pgfpathlineto{\pgfqpoint{3.126842in}{0.683500in}}%
\pgfpathlineto{\pgfqpoint{3.129420in}{0.680665in}}%
\pgfpathlineto{\pgfqpoint{3.139734in}{0.738364in}}%
\pgfpathlineto{\pgfqpoint{3.142312in}{0.879892in}}%
\pgfpathlineto{\pgfqpoint{3.144891in}{0.670733in}}%
\pgfpathlineto{\pgfqpoint{3.147469in}{0.966251in}}%
\pgfpathlineto{\pgfqpoint{3.157783in}{0.904853in}}%
\pgfpathlineto{\pgfqpoint{3.160361in}{0.662781in}}%
\pgfpathlineto{\pgfqpoint{3.162939in}{0.665145in}}%
\pgfpathlineto{\pgfqpoint{3.165518in}{0.659155in}}%
\pgfpathlineto{\pgfqpoint{3.173253in}{0.660287in}}%
\pgfpathlineto{\pgfqpoint{3.175831in}{0.667290in}}%
\pgfpathlineto{\pgfqpoint{3.178410in}{0.667290in}}%
\pgfpathlineto{\pgfqpoint{3.180988in}{0.672493in}}%
\pgfpathlineto{\pgfqpoint{3.183566in}{0.685026in}}%
\pgfpathlineto{\pgfqpoint{3.191302in}{0.828932in}}%
\pgfpathlineto{\pgfqpoint{3.193880in}{0.775845in}}%
\pgfpathlineto{\pgfqpoint{3.196458in}{0.707524in}}%
\pgfpathlineto{\pgfqpoint{3.199037in}{0.665838in}}%
\pgfpathlineto{\pgfqpoint{3.201615in}{0.671474in}}%
\pgfpathlineto{\pgfqpoint{3.209350in}{0.661350in}}%
\pgfpathlineto{\pgfqpoint{3.211929in}{0.680979in}}%
\pgfpathlineto{\pgfqpoint{3.214507in}{0.668611in}}%
\pgfpathlineto{\pgfqpoint{3.217086in}{0.659146in}}%
\pgfpathlineto{\pgfqpoint{3.219664in}{0.664354in}}%
\pgfpathlineto{\pgfqpoint{3.227399in}{0.745812in}}%
\pgfpathlineto{\pgfqpoint{3.229978in}{0.659474in}}%
\pgfpathlineto{\pgfqpoint{3.232556in}{0.774805in}}%
\pgfpathlineto{\pgfqpoint{3.235134in}{0.683623in}}%
\pgfpathlineto{\pgfqpoint{3.237713in}{0.682607in}}%
\pgfpathlineto{\pgfqpoint{3.245448in}{0.901320in}}%
\pgfpathlineto{\pgfqpoint{3.248026in}{0.689767in}}%
\pgfpathlineto{\pgfqpoint{3.250605in}{0.670141in}}%
\pgfpathlineto{\pgfqpoint{3.253183in}{0.661281in}}%
\pgfpathlineto{\pgfqpoint{3.255761in}{0.660541in}}%
\pgfpathlineto{\pgfqpoint{3.263497in}{0.739470in}}%
\pgfpathlineto{\pgfqpoint{3.266075in}{0.666780in}}%
\pgfpathlineto{\pgfqpoint{3.268653in}{0.663623in}}%
\pgfpathlineto{\pgfqpoint{3.271232in}{0.659024in}}%
\pgfpathlineto{\pgfqpoint{3.273810in}{0.671852in}}%
\pgfpathlineto{\pgfqpoint{3.281545in}{0.720159in}}%
\pgfpathlineto{\pgfqpoint{3.284124in}{0.742592in}}%
\pgfpathlineto{\pgfqpoint{3.286702in}{0.661173in}}%
\pgfpathlineto{\pgfqpoint{3.289281in}{0.677375in}}%
\pgfpathlineto{\pgfqpoint{3.291859in}{0.670647in}}%
\pgfpathlineto{\pgfqpoint{3.299594in}{0.659135in}}%
\pgfpathlineto{\pgfqpoint{3.302172in}{0.658941in}}%
\pgfpathlineto{\pgfqpoint{3.304751in}{0.732958in}}%
\pgfpathlineto{\pgfqpoint{3.307329in}{0.660159in}}%
\pgfpathlineto{\pgfqpoint{3.317643in}{0.662936in}}%
\pgfpathlineto{\pgfqpoint{3.320221in}{0.669662in}}%
\pgfpathlineto{\pgfqpoint{3.322800in}{0.665961in}}%
\pgfpathlineto{\pgfqpoint{3.325378in}{0.730511in}}%
\pgfpathlineto{\pgfqpoint{3.327956in}{0.666500in}}%
\pgfpathlineto{\pgfqpoint{3.335692in}{0.915358in}}%
\pgfpathlineto{\pgfqpoint{3.338270in}{0.685726in}}%
\pgfpathlineto{\pgfqpoint{3.340848in}{0.660115in}}%
\pgfpathlineto{\pgfqpoint{3.343427in}{0.663837in}}%
\pgfpathlineto{\pgfqpoint{3.346005in}{0.677777in}}%
\pgfpathlineto{\pgfqpoint{3.353740in}{0.664490in}}%
\pgfpathlineto{\pgfqpoint{3.356319in}{0.695053in}}%
\pgfpathlineto{\pgfqpoint{3.358897in}{0.761405in}}%
\pgfpathlineto{\pgfqpoint{3.361475in}{0.659432in}}%
\pgfpathlineto{\pgfqpoint{3.364054in}{0.665026in}}%
\pgfpathlineto{\pgfqpoint{3.371789in}{0.707340in}}%
\pgfpathlineto{\pgfqpoint{3.374367in}{0.660436in}}%
\pgfpathlineto{\pgfqpoint{3.379524in}{0.705307in}}%
\pgfpathlineto{\pgfqpoint{3.382103in}{0.663526in}}%
\pgfpathlineto{\pgfqpoint{3.389838in}{0.677616in}}%
\pgfpathlineto{\pgfqpoint{3.392416in}{0.669847in}}%
\pgfpathlineto{\pgfqpoint{3.394995in}{0.666676in}}%
\pgfpathlineto{\pgfqpoint{3.397573in}{0.661586in}}%
\pgfpathlineto{\pgfqpoint{3.400151in}{0.660815in}}%
\pgfpathlineto{\pgfqpoint{3.410465in}{0.680974in}}%
\pgfpathlineto{\pgfqpoint{3.413043in}{0.658941in}}%
\pgfpathlineto{\pgfqpoint{3.415622in}{0.668976in}}%
\pgfpathlineto{\pgfqpoint{3.418200in}{0.676326in}}%
\pgfpathlineto{\pgfqpoint{3.428514in}{0.681059in}}%
\pgfpathlineto{\pgfqpoint{3.431092in}{0.659284in}}%
\pgfpathlineto{\pgfqpoint{3.433670in}{0.666049in}}%
\pgfpathlineto{\pgfqpoint{3.436249in}{0.659671in}}%
\pgfpathlineto{\pgfqpoint{3.446562in}{0.659281in}}%
\pgfpathlineto{\pgfqpoint{3.449141in}{0.659914in}}%
\pgfpathlineto{\pgfqpoint{3.451719in}{0.669063in}}%
\pgfpathlineto{\pgfqpoint{3.454298in}{0.660514in}}%
\pgfpathlineto{\pgfqpoint{3.462033in}{0.659138in}}%
\pgfpathlineto{\pgfqpoint{3.464611in}{0.660205in}}%
\pgfpathlineto{\pgfqpoint{3.467190in}{0.664186in}}%
\pgfpathlineto{\pgfqpoint{3.469768in}{0.682346in}}%
\pgfpathlineto{\pgfqpoint{3.472346in}{0.670706in}}%
\pgfpathlineto{\pgfqpoint{3.480081in}{0.677805in}}%
\pgfpathlineto{\pgfqpoint{3.482660in}{0.689501in}}%
\pgfpathlineto{\pgfqpoint{3.485238in}{0.670101in}}%
\pgfpathlineto{\pgfqpoint{3.487817in}{0.716964in}}%
\pgfpathlineto{\pgfqpoint{3.490395in}{0.677119in}}%
\pgfpathlineto{\pgfqpoint{3.498130in}{0.680597in}}%
\pgfpathlineto{\pgfqpoint{3.500709in}{0.715070in}}%
\pgfpathlineto{\pgfqpoint{3.503287in}{0.668381in}}%
\pgfpathlineto{\pgfqpoint{3.505865in}{0.680072in}}%
\pgfpathlineto{\pgfqpoint{3.516179in}{0.663566in}}%
\pgfpathlineto{\pgfqpoint{3.518757in}{0.675373in}}%
\pgfpathlineto{\pgfqpoint{3.521336in}{0.798633in}}%
\pgfpathlineto{\pgfqpoint{3.523914in}{0.659266in}}%
\pgfpathlineto{\pgfqpoint{3.526493in}{0.681563in}}%
\pgfpathlineto{\pgfqpoint{3.534228in}{0.661541in}}%
\pgfpathlineto{\pgfqpoint{3.536806in}{0.659270in}}%
\pgfpathlineto{\pgfqpoint{3.539384in}{0.671183in}}%
\pgfpathlineto{\pgfqpoint{3.541963in}{0.692178in}}%
\pgfpathlineto{\pgfqpoint{3.544541in}{0.661154in}}%
\pgfpathlineto{\pgfqpoint{3.552276in}{0.667256in}}%
\pgfpathlineto{\pgfqpoint{3.554855in}{0.662460in}}%
\pgfpathlineto{\pgfqpoint{3.557433in}{0.660171in}}%
\pgfpathlineto{\pgfqpoint{3.560012in}{0.667315in}}%
\pgfpathlineto{\pgfqpoint{3.562590in}{0.716620in}}%
\pgfpathlineto{\pgfqpoint{3.570325in}{0.669857in}}%
\pgfpathlineto{\pgfqpoint{3.572904in}{0.703311in}}%
\pgfpathlineto{\pgfqpoint{3.575482in}{0.688272in}}%
\pgfpathlineto{\pgfqpoint{3.578060in}{0.744321in}}%
\pgfpathlineto{\pgfqpoint{3.580639in}{1.317426in}}%
\pgfpathlineto{\pgfqpoint{3.588374in}{0.681098in}}%
\pgfpathlineto{\pgfqpoint{3.590952in}{0.663044in}}%
\pgfpathlineto{\pgfqpoint{3.593531in}{0.984171in}}%
\pgfpathlineto{\pgfqpoint{3.596109in}{0.762302in}}%
\pgfpathlineto{\pgfqpoint{3.598687in}{0.734058in}}%
\pgfpathlineto{\pgfqpoint{3.606423in}{0.691738in}}%
\pgfpathlineto{\pgfqpoint{3.609001in}{0.659262in}}%
\pgfpathlineto{\pgfqpoint{3.611579in}{0.659262in}}%
\pgfpathlineto{\pgfqpoint{3.614158in}{0.683418in}}%
\pgfpathlineto{\pgfqpoint{3.616736in}{0.662315in}}%
\pgfpathlineto{\pgfqpoint{3.624471in}{0.708237in}}%
\pgfpathlineto{\pgfqpoint{3.627050in}{0.665502in}}%
\pgfpathlineto{\pgfqpoint{3.629628in}{0.660372in}}%
\pgfpathlineto{\pgfqpoint{3.632207in}{0.682642in}}%
\pgfpathlineto{\pgfqpoint{3.634785in}{0.660351in}}%
\pgfpathlineto{\pgfqpoint{3.645099in}{0.661361in}}%
\pgfpathlineto{\pgfqpoint{3.647677in}{0.659407in}}%
\pgfpathlineto{\pgfqpoint{3.650255in}{0.667405in}}%
\pgfpathlineto{\pgfqpoint{3.652834in}{0.659603in}}%
\pgfpathlineto{\pgfqpoint{3.663147in}{0.660673in}}%
\pgfpathlineto{\pgfqpoint{3.665726in}{0.659606in}}%
\pgfpathlineto{\pgfqpoint{3.668304in}{0.727135in}}%
\pgfpathlineto{\pgfqpoint{3.670882in}{0.659402in}}%
\pgfpathlineto{\pgfqpoint{3.678618in}{0.680067in}}%
\pgfpathlineto{\pgfqpoint{3.681196in}{0.671334in}}%
\pgfpathlineto{\pgfqpoint{3.683774in}{0.705156in}}%
\pgfpathlineto{\pgfqpoint{3.686353in}{0.661355in}}%
\pgfpathlineto{\pgfqpoint{3.688931in}{0.664771in}}%
\pgfpathlineto{\pgfqpoint{3.696666in}{0.700080in}}%
\pgfpathlineto{\pgfqpoint{3.699245in}{0.663074in}}%
\pgfpathlineto{\pgfqpoint{3.701823in}{0.659585in}}%
\pgfpathlineto{\pgfqpoint{3.704402in}{0.659006in}}%
\pgfpathlineto{\pgfqpoint{3.706980in}{0.667172in}}%
\pgfpathlineto{\pgfqpoint{3.714715in}{0.670900in}}%
\pgfpathlineto{\pgfqpoint{3.717293in}{0.673522in}}%
\pgfpathlineto{\pgfqpoint{3.719872in}{0.724686in}}%
\pgfpathlineto{\pgfqpoint{3.722450in}{0.743365in}}%
\pgfpathlineto{\pgfqpoint{3.725029in}{0.665209in}}%
\pgfpathlineto{\pgfqpoint{3.732764in}{0.660941in}}%
\pgfpathlineto{\pgfqpoint{3.735342in}{0.711030in}}%
\pgfpathlineto{\pgfqpoint{3.737921in}{0.697147in}}%
\pgfpathlineto{\pgfqpoint{3.740499in}{0.659603in}}%
\pgfpathlineto{\pgfqpoint{3.743077in}{0.717293in}}%
\pgfpathlineto{\pgfqpoint{3.750813in}{0.664157in}}%
\pgfpathlineto{\pgfqpoint{3.753391in}{0.676931in}}%
\pgfpathlineto{\pgfqpoint{3.755969in}{0.769626in}}%
\pgfpathlineto{\pgfqpoint{3.758548in}{0.687033in}}%
\pgfpathlineto{\pgfqpoint{3.761126in}{0.768303in}}%
\pgfpathlineto{\pgfqpoint{3.768861in}{0.840864in}}%
\pgfpathlineto{\pgfqpoint{3.771440in}{0.662185in}}%
\pgfpathlineto{\pgfqpoint{3.774018in}{0.697320in}}%
\pgfpathlineto{\pgfqpoint{3.776596in}{0.714752in}}%
\pgfpathlineto{\pgfqpoint{3.779175in}{0.774720in}}%
\pgfpathlineto{\pgfqpoint{3.786910in}{0.747784in}}%
\pgfpathlineto{\pgfqpoint{3.789488in}{0.678563in}}%
\pgfpathlineto{\pgfqpoint{3.792067in}{0.665780in}}%
\pgfpathlineto{\pgfqpoint{3.794645in}{0.765312in}}%
\pgfpathlineto{\pgfqpoint{3.797224in}{1.051089in}}%
\pgfpathlineto{\pgfqpoint{3.804959in}{0.722609in}}%
\pgfpathlineto{\pgfqpoint{3.807537in}{0.686362in}}%
\pgfpathlineto{\pgfqpoint{3.810116in}{0.659361in}}%
\pgfpathlineto{\pgfqpoint{3.812694in}{0.674376in}}%
\pgfpathlineto{\pgfqpoint{3.815272in}{0.670058in}}%
\pgfpathlineto{\pgfqpoint{3.823008in}{0.659948in}}%
\pgfpathlineto{\pgfqpoint{3.825586in}{0.810801in}}%
\pgfpathlineto{\pgfqpoint{3.828164in}{0.671166in}}%
\pgfpathlineto{\pgfqpoint{3.830743in}{0.659692in}}%
\pgfpathlineto{\pgfqpoint{3.833321in}{0.663968in}}%
\pgfpathlineto{\pgfqpoint{3.841056in}{0.668100in}}%
\pgfpathlineto{\pgfqpoint{3.843635in}{0.663416in}}%
\pgfpathlineto{\pgfqpoint{3.846213in}{0.662094in}}%
\pgfpathlineto{\pgfqpoint{3.848791in}{0.732368in}}%
\pgfpathlineto{\pgfqpoint{3.851370in}{0.681671in}}%
\pgfpathlineto{\pgfqpoint{3.861683in}{0.659934in}}%
\pgfpathlineto{\pgfqpoint{3.864262in}{0.716906in}}%
\pgfpathlineto{\pgfqpoint{3.866840in}{0.665096in}}%
\pgfpathlineto{\pgfqpoint{3.869419in}{0.660433in}}%
\pgfpathlineto{\pgfqpoint{3.877154in}{0.681671in}}%
\pgfpathlineto{\pgfqpoint{3.879732in}{0.703757in}}%
\pgfpathlineto{\pgfqpoint{3.882311in}{0.659502in}}%
\pgfpathlineto{\pgfqpoint{3.887467in}{0.880898in}}%
\pgfpathlineto{\pgfqpoint{3.895203in}{0.670022in}}%
\pgfpathlineto{\pgfqpoint{3.897781in}{0.747726in}}%
\pgfpathlineto{\pgfqpoint{3.900359in}{0.763057in}}%
\pgfpathlineto{\pgfqpoint{3.902938in}{0.689814in}}%
\pgfpathlineto{\pgfqpoint{3.905516in}{0.662437in}}%
\pgfpathlineto{\pgfqpoint{3.913251in}{0.672268in}}%
\pgfpathlineto{\pgfqpoint{3.915830in}{0.659182in}}%
\pgfpathlineto{\pgfqpoint{3.918408in}{0.704859in}}%
\pgfpathlineto{\pgfqpoint{3.920986in}{0.674943in}}%
\pgfpathlineto{\pgfqpoint{3.923565in}{0.727100in}}%
\pgfpathlineto{\pgfqpoint{3.931300in}{0.670241in}}%
\pgfpathlineto{\pgfqpoint{3.933878in}{0.661015in}}%
\pgfpathlineto{\pgfqpoint{3.936457in}{0.823496in}}%
\pgfpathlineto{\pgfqpoint{3.939035in}{0.810537in}}%
\pgfpathlineto{\pgfqpoint{3.941614in}{0.659174in}}%
\pgfpathlineto{\pgfqpoint{3.949349in}{0.685017in}}%
\pgfpathlineto{\pgfqpoint{3.951927in}{0.699525in}}%
\pgfpathlineto{\pgfqpoint{3.954506in}{0.660838in}}%
\pgfpathlineto{\pgfqpoint{3.959662in}{0.667962in}}%
\pgfpathlineto{\pgfqpoint{3.967397in}{0.717610in}}%
\pgfpathlineto{\pgfqpoint{3.969976in}{0.663590in}}%
\pgfpathlineto{\pgfqpoint{3.972554in}{0.811910in}}%
\pgfpathlineto{\pgfqpoint{3.977711in}{0.697278in}}%
\pgfpathlineto{\pgfqpoint{3.985446in}{0.675842in}}%
\pgfpathlineto{\pgfqpoint{3.988025in}{0.674199in}}%
\pgfpathlineto{\pgfqpoint{3.990603in}{0.678877in}}%
\pgfpathlineto{\pgfqpoint{3.993181in}{0.756260in}}%
\pgfpathlineto{\pgfqpoint{3.995760in}{0.724258in}}%
\pgfpathlineto{\pgfqpoint{4.003495in}{0.669331in}}%
\pgfpathlineto{\pgfqpoint{4.006073in}{0.672407in}}%
\pgfpathlineto{\pgfqpoint{4.008652in}{0.667927in}}%
\pgfpathlineto{\pgfqpoint{4.011230in}{0.660386in}}%
\pgfpathlineto{\pgfqpoint{4.013809in}{0.836033in}}%
\pgfpathlineto{\pgfqpoint{4.024122in}{0.659179in}}%
\pgfpathlineto{\pgfqpoint{4.026700in}{0.661600in}}%
\pgfpathlineto{\pgfqpoint{4.029279in}{0.731670in}}%
\pgfpathlineto{\pgfqpoint{4.031857in}{0.875256in}}%
\pgfpathlineto{\pgfqpoint{4.039592in}{0.681667in}}%
\pgfpathlineto{\pgfqpoint{4.042171in}{1.580852in}}%
\pgfpathlineto{\pgfqpoint{4.044749in}{0.840619in}}%
\pgfpathlineto{\pgfqpoint{4.047328in}{0.686489in}}%
\pgfpathlineto{\pgfqpoint{4.049906in}{0.853930in}}%
\pgfpathlineto{\pgfqpoint{4.057641in}{0.730737in}}%
\pgfpathlineto{\pgfqpoint{4.060220in}{0.729346in}}%
\pgfpathlineto{\pgfqpoint{4.062798in}{0.661520in}}%
\pgfpathlineto{\pgfqpoint{4.065376in}{0.742323in}}%
\pgfpathlineto{\pgfqpoint{4.067955in}{0.779808in}}%
\pgfpathlineto{\pgfqpoint{4.075690in}{0.690349in}}%
\pgfpathlineto{\pgfqpoint{4.078268in}{0.670394in}}%
\pgfpathlineto{\pgfqpoint{4.080847in}{0.665459in}}%
\pgfpathlineto{\pgfqpoint{4.083425in}{0.674481in}}%
\pgfpathlineto{\pgfqpoint{4.086003in}{0.660814in}}%
\pgfpathlineto{\pgfqpoint{4.096317in}{0.675468in}}%
\pgfpathlineto{\pgfqpoint{4.101474in}{0.770917in}}%
\pgfpathlineto{\pgfqpoint{4.104052in}{0.671254in}}%
\pgfpathlineto{\pgfqpoint{4.111787in}{0.686669in}}%
\pgfpathlineto{\pgfqpoint{4.114366in}{0.660238in}}%
\pgfpathlineto{\pgfqpoint{4.116944in}{0.659094in}}%
\pgfpathlineto{\pgfqpoint{4.119523in}{0.667417in}}%
\pgfpathlineto{\pgfqpoint{4.122101in}{0.659000in}}%
\pgfpathlineto{\pgfqpoint{4.129836in}{0.666755in}}%
\pgfpathlineto{\pgfqpoint{4.132415in}{0.665495in}}%
\pgfpathlineto{\pgfqpoint{4.134993in}{0.726726in}}%
\pgfpathlineto{\pgfqpoint{4.137571in}{0.666902in}}%
\pgfpathlineto{\pgfqpoint{4.140150in}{1.069766in}}%
\pgfpathlineto{\pgfqpoint{4.147885in}{0.678701in}}%
\pgfpathlineto{\pgfqpoint{4.150463in}{0.922466in}}%
\pgfpathlineto{\pgfqpoint{4.153042in}{0.661409in}}%
\pgfpathlineto{\pgfqpoint{4.155620in}{0.714264in}}%
\pgfpathlineto{\pgfqpoint{4.158198in}{0.666708in}}%
\pgfpathlineto{\pgfqpoint{4.165934in}{0.986137in}}%
\pgfpathlineto{\pgfqpoint{4.168512in}{0.714102in}}%
\pgfpathlineto{\pgfqpoint{4.171090in}{0.746091in}}%
\pgfpathlineto{\pgfqpoint{4.173669in}{0.673462in}}%
\pgfpathlineto{\pgfqpoint{4.176247in}{0.854879in}}%
\pgfpathlineto{\pgfqpoint{4.183982in}{0.660259in}}%
\pgfpathlineto{\pgfqpoint{4.186561in}{0.750512in}}%
\pgfpathlineto{\pgfqpoint{4.189139in}{0.747983in}}%
\pgfpathlineto{\pgfqpoint{4.191718in}{0.738743in}}%
\pgfpathlineto{\pgfqpoint{4.194296in}{0.661759in}}%
\pgfpathlineto{\pgfqpoint{4.202031in}{0.676770in}}%
\pgfpathlineto{\pgfqpoint{4.204609in}{0.724743in}}%
\pgfpathlineto{\pgfqpoint{4.207188in}{0.673975in}}%
\pgfpathlineto{\pgfqpoint{4.209766in}{0.660342in}}%
\pgfpathlineto{\pgfqpoint{4.220080in}{0.699690in}}%
\pgfpathlineto{\pgfqpoint{4.222658in}{0.706110in}}%
\pgfpathlineto{\pgfqpoint{4.225237in}{0.674761in}}%
\pgfpathlineto{\pgfqpoint{4.227815in}{0.660630in}}%
\pgfpathlineto{\pgfqpoint{4.230393in}{0.681908in}}%
\pgfpathlineto{\pgfqpoint{4.238129in}{0.659577in}}%
\pgfpathlineto{\pgfqpoint{4.240707in}{0.661671in}}%
\pgfpathlineto{\pgfqpoint{4.243285in}{0.659788in}}%
\pgfpathlineto{\pgfqpoint{4.245864in}{0.658876in}}%
\pgfpathlineto{\pgfqpoint{4.248442in}{0.760702in}}%
\pgfpathlineto{\pgfqpoint{4.256177in}{0.671106in}}%
\pgfpathlineto{\pgfqpoint{4.258756in}{0.663557in}}%
\pgfpathlineto{\pgfqpoint{4.261334in}{0.658890in}}%
\pgfpathlineto{\pgfqpoint{4.263912in}{0.902989in}}%
\pgfpathlineto{\pgfqpoint{4.266491in}{0.659249in}}%
\pgfpathlineto{\pgfqpoint{4.274226in}{0.677240in}}%
\pgfpathlineto{\pgfqpoint{4.276804in}{0.662742in}}%
\pgfpathlineto{\pgfqpoint{4.279383in}{0.696983in}}%
\pgfpathlineto{\pgfqpoint{4.281961in}{0.671841in}}%
\pgfpathlineto{\pgfqpoint{4.284540in}{0.729764in}}%
\pgfpathlineto{\pgfqpoint{4.292275in}{0.659255in}}%
\pgfpathlineto{\pgfqpoint{4.294853in}{0.668385in}}%
\pgfpathlineto{\pgfqpoint{4.297432in}{0.671664in}}%
\pgfpathlineto{\pgfqpoint{4.300010in}{0.663263in}}%
\pgfpathlineto{\pgfqpoint{4.302588in}{0.724539in}}%
\pgfpathlineto{\pgfqpoint{4.310324in}{0.713097in}}%
\pgfpathlineto{\pgfqpoint{4.312902in}{0.670833in}}%
\pgfpathlineto{\pgfqpoint{4.315480in}{0.666311in}}%
\pgfpathlineto{\pgfqpoint{4.318059in}{0.746919in}}%
\pgfpathlineto{\pgfqpoint{4.320637in}{0.685316in}}%
\pgfpathlineto{\pgfqpoint{4.328372in}{0.669780in}}%
\pgfpathlineto{\pgfqpoint{4.330951in}{0.659836in}}%
\pgfpathlineto{\pgfqpoint{4.333529in}{0.673335in}}%
\pgfpathlineto{\pgfqpoint{4.336107in}{0.659421in}}%
\pgfpathlineto{\pgfqpoint{4.338686in}{0.683236in}}%
\pgfpathlineto{\pgfqpoint{4.348999in}{0.736859in}}%
\pgfpathlineto{\pgfqpoint{4.351578in}{0.666416in}}%
\pgfpathlineto{\pgfqpoint{4.354156in}{0.659265in}}%
\pgfpathlineto{\pgfqpoint{4.356735in}{0.764702in}}%
\pgfpathlineto{\pgfqpoint{4.364470in}{0.684212in}}%
\pgfpathlineto{\pgfqpoint{4.369627in}{0.658940in}}%
\pgfpathlineto{\pgfqpoint{4.372205in}{0.679536in}}%
\pgfpathlineto{\pgfqpoint{4.374783in}{0.723074in}}%
\pgfpathlineto{\pgfqpoint{4.382518in}{0.668246in}}%
\pgfpathlineto{\pgfqpoint{4.385097in}{0.831557in}}%
\pgfpathlineto{\pgfqpoint{4.387675in}{0.707790in}}%
\pgfpathlineto{\pgfqpoint{4.390254in}{0.660751in}}%
\pgfpathlineto{\pgfqpoint{4.392832in}{0.693396in}}%
\pgfpathlineto{\pgfqpoint{4.400567in}{0.728075in}}%
\pgfpathlineto{\pgfqpoint{4.403146in}{0.776027in}}%
\pgfpathlineto{\pgfqpoint{4.405724in}{0.770518in}}%
\pgfpathlineto{\pgfqpoint{4.410881in}{0.668275in}}%
\pgfpathlineto{\pgfqpoint{4.418616in}{0.659616in}}%
\pgfpathlineto{\pgfqpoint{4.421194in}{0.710167in}}%
\pgfpathlineto{\pgfqpoint{4.423773in}{0.667762in}}%
\pgfpathlineto{\pgfqpoint{4.426351in}{0.660752in}}%
\pgfpathlineto{\pgfqpoint{4.428930in}{0.659125in}}%
\pgfpathlineto{\pgfqpoint{4.436665in}{0.783632in}}%
\pgfpathlineto{\pgfqpoint{4.439243in}{0.659659in}}%
\pgfpathlineto{\pgfqpoint{4.441821in}{0.919993in}}%
\pgfpathlineto{\pgfqpoint{4.444400in}{0.663853in}}%
\pgfpathlineto{\pgfqpoint{4.454713in}{0.660727in}}%
\pgfpathlineto{\pgfqpoint{4.457292in}{0.973453in}}%
\pgfpathlineto{\pgfqpoint{4.459870in}{0.717573in}}%
\pgfpathlineto{\pgfqpoint{4.462449in}{0.671476in}}%
\pgfpathlineto{\pgfqpoint{4.465027in}{0.668245in}}%
\pgfpathlineto{\pgfqpoint{4.472762in}{0.763037in}}%
\pgfpathlineto{\pgfqpoint{4.475341in}{0.660639in}}%
\pgfpathlineto{\pgfqpoint{4.477919in}{0.660328in}}%
\pgfpathlineto{\pgfqpoint{4.480497in}{0.661323in}}%
\pgfpathlineto{\pgfqpoint{4.483076in}{0.659238in}}%
\pgfpathlineto{\pgfqpoint{4.490811in}{0.659238in}}%
\pgfpathlineto{\pgfqpoint{4.493389in}{0.692489in}}%
\pgfpathlineto{\pgfqpoint{4.495968in}{0.661363in}}%
\pgfpathlineto{\pgfqpoint{4.498546in}{0.660662in}}%
\pgfpathlineto{\pgfqpoint{4.501124in}{0.678137in}}%
\pgfpathlineto{\pgfqpoint{4.508860in}{0.670634in}}%
\pgfpathlineto{\pgfqpoint{4.511438in}{0.666815in}}%
\pgfpathlineto{\pgfqpoint{4.514016in}{0.676086in}}%
\pgfpathlineto{\pgfqpoint{4.516595in}{1.903527in}}%
\pgfpathlineto{\pgfqpoint{4.519173in}{0.719241in}}%
\pgfpathlineto{\pgfqpoint{4.526908in}{0.669985in}}%
\pgfpathlineto{\pgfqpoint{4.529487in}{0.689574in}}%
\pgfpathlineto{\pgfqpoint{4.532065in}{0.661705in}}%
\pgfpathlineto{\pgfqpoint{4.534644in}{0.658943in}}%
\pgfpathlineto{\pgfqpoint{4.537222in}{0.667786in}}%
\pgfpathlineto{\pgfqpoint{4.544957in}{0.763115in}}%
\pgfpathlineto{\pgfqpoint{4.547536in}{0.662118in}}%
\pgfpathlineto{\pgfqpoint{4.550114in}{0.662597in}}%
\pgfpathlineto{\pgfqpoint{4.552692in}{0.707383in}}%
\pgfpathlineto{\pgfqpoint{4.555271in}{0.662170in}}%
\pgfpathlineto{\pgfqpoint{4.563006in}{0.659955in}}%
\pgfpathlineto{\pgfqpoint{4.565584in}{0.679665in}}%
\pgfpathlineto{\pgfqpoint{4.568163in}{0.795851in}}%
\pgfpathlineto{\pgfqpoint{4.570741in}{0.664571in}}%
\pgfpathlineto{\pgfqpoint{4.573319in}{1.259494in}}%
\pgfpathlineto{\pgfqpoint{4.581055in}{1.756824in}}%
\pgfpathlineto{\pgfqpoint{4.583633in}{0.739664in}}%
\pgfpathlineto{\pgfqpoint{4.586211in}{1.602089in}}%
\pgfpathlineto{\pgfqpoint{4.588790in}{0.708295in}}%
\pgfpathlineto{\pgfqpoint{4.591368in}{0.669773in}}%
\pgfpathlineto{\pgfqpoint{4.599103in}{0.701144in}}%
\pgfpathlineto{\pgfqpoint{4.601682in}{1.140096in}}%
\pgfpathlineto{\pgfqpoint{4.604260in}{0.787001in}}%
\pgfpathlineto{\pgfqpoint{4.606839in}{0.660849in}}%
\pgfpathlineto{\pgfqpoint{4.609417in}{0.867231in}}%
\pgfpathlineto{\pgfqpoint{4.619730in}{0.887972in}}%
\pgfpathlineto{\pgfqpoint{4.622309in}{1.008974in}}%
\pgfpathlineto{\pgfqpoint{4.624887in}{0.662910in}}%
\pgfpathlineto{\pgfqpoint{4.627466in}{0.660545in}}%
\pgfpathlineto{\pgfqpoint{4.635201in}{0.680050in}}%
\pgfpathlineto{\pgfqpoint{4.637779in}{0.962102in}}%
\pgfpathlineto{\pgfqpoint{4.640358in}{0.723147in}}%
\pgfpathlineto{\pgfqpoint{4.642936in}{0.662181in}}%
\pgfpathlineto{\pgfqpoint{4.645514in}{0.673166in}}%
\pgfpathlineto{\pgfqpoint{4.653250in}{0.736227in}}%
\pgfpathlineto{\pgfqpoint{4.655828in}{0.691520in}}%
\pgfpathlineto{\pgfqpoint{4.658406in}{0.659833in}}%
\pgfpathlineto{\pgfqpoint{4.660985in}{0.835874in}}%
\pgfpathlineto{\pgfqpoint{4.663563in}{0.908735in}}%
\pgfpathlineto{\pgfqpoint{4.671298in}{0.777039in}}%
\pgfpathlineto{\pgfqpoint{4.673877in}{0.696470in}}%
\pgfpathlineto{\pgfqpoint{4.679033in}{0.658895in}}%
\pgfpathlineto{\pgfqpoint{4.681612in}{0.691475in}}%
\pgfpathlineto{\pgfqpoint{4.689347in}{0.747853in}}%
\pgfpathlineto{\pgfqpoint{4.691925in}{0.663463in}}%
\pgfpathlineto{\pgfqpoint{4.694504in}{0.672828in}}%
\pgfpathlineto{\pgfqpoint{4.697082in}{0.722078in}}%
\pgfpathlineto{\pgfqpoint{4.699661in}{0.659728in}}%
\pgfpathlineto{\pgfqpoint{4.707396in}{0.661815in}}%
\pgfpathlineto{\pgfqpoint{4.709974in}{0.665868in}}%
\pgfpathlineto{\pgfqpoint{4.712553in}{0.660294in}}%
\pgfpathlineto{\pgfqpoint{4.715131in}{0.659313in}}%
\pgfpathlineto{\pgfqpoint{4.717709in}{0.713163in}}%
\pgfpathlineto{\pgfqpoint{4.725445in}{0.667924in}}%
\pgfpathlineto{\pgfqpoint{4.728023in}{0.731604in}}%
\pgfpathlineto{\pgfqpoint{4.730601in}{0.662797in}}%
\pgfpathlineto{\pgfqpoint{4.733180in}{0.874255in}}%
\pgfpathlineto{\pgfqpoint{4.735758in}{1.269882in}}%
\pgfpathlineto{\pgfqpoint{4.743493in}{0.685514in}}%
\pgfpathlineto{\pgfqpoint{4.746072in}{0.663993in}}%
\pgfpathlineto{\pgfqpoint{4.748650in}{0.735030in}}%
\pgfpathlineto{\pgfqpoint{4.751228in}{0.691381in}}%
\pgfpathlineto{\pgfqpoint{4.753807in}{0.711120in}}%
\pgfpathlineto{\pgfqpoint{4.761542in}{0.665343in}}%
\pgfpathlineto{\pgfqpoint{4.764120in}{0.684517in}}%
\pgfpathlineto{\pgfqpoint{4.766699in}{0.658892in}}%
\pgfpathlineto{\pgfqpoint{4.769277in}{0.716722in}}%
\pgfpathlineto{\pgfqpoint{4.771856in}{0.746201in}}%
\pgfpathlineto{\pgfqpoint{4.779591in}{0.662612in}}%
\pgfpathlineto{\pgfqpoint{4.782169in}{0.676908in}}%
\pgfpathlineto{\pgfqpoint{4.784748in}{0.663626in}}%
\pgfpathlineto{\pgfqpoint{4.787326in}{0.879440in}}%
\pgfpathlineto{\pgfqpoint{4.789904in}{0.724715in}}%
\pgfpathlineto{\pgfqpoint{4.797640in}{0.895359in}}%
\pgfpathlineto{\pgfqpoint{4.800218in}{0.665570in}}%
\pgfpathlineto{\pgfqpoint{4.802796in}{0.749779in}}%
\pgfpathlineto{\pgfqpoint{4.805375in}{0.671695in}}%
\pgfpathlineto{\pgfqpoint{4.807953in}{0.678926in}}%
\pgfpathlineto{\pgfqpoint{4.815688in}{0.661651in}}%
\pgfpathlineto{\pgfqpoint{4.818267in}{0.688987in}}%
\pgfpathlineto{\pgfqpoint{4.820845in}{0.698011in}}%
\pgfpathlineto{\pgfqpoint{4.826002in}{0.664210in}}%
\pgfpathlineto{\pgfqpoint{4.833737in}{0.755305in}}%
\pgfpathlineto{\pgfqpoint{4.836315in}{0.815467in}}%
\pgfpathlineto{\pgfqpoint{4.838894in}{0.661654in}}%
\pgfpathlineto{\pgfqpoint{4.841472in}{0.659140in}}%
\pgfpathlineto{\pgfqpoint{4.844051in}{1.207985in}}%
\pgfpathlineto{\pgfqpoint{4.851786in}{0.694606in}}%
\pgfpathlineto{\pgfqpoint{4.854364in}{0.705834in}}%
\pgfpathlineto{\pgfqpoint{4.856943in}{0.659127in}}%
\pgfpathlineto{\pgfqpoint{4.859521in}{0.659877in}}%
\pgfpathlineto{\pgfqpoint{4.862099in}{0.658892in}}%
\pgfpathlineto{\pgfqpoint{4.869834in}{0.690329in}}%
\pgfpathlineto{\pgfqpoint{4.872413in}{0.888277in}}%
\pgfpathlineto{\pgfqpoint{4.874991in}{0.859514in}}%
\pgfpathlineto{\pgfqpoint{4.877570in}{0.716609in}}%
\pgfpathlineto{\pgfqpoint{4.880148in}{1.208934in}}%
\pgfpathlineto{\pgfqpoint{4.887883in}{0.738372in}}%
\pgfpathlineto{\pgfqpoint{4.890462in}{0.709513in}}%
\pgfpathlineto{\pgfqpoint{4.893040in}{0.671380in}}%
\pgfpathlineto{\pgfqpoint{4.895618in}{0.661012in}}%
\pgfpathlineto{\pgfqpoint{4.905932in}{0.661012in}}%
\pgfpathlineto{\pgfqpoint{4.908510in}{0.681275in}}%
\pgfpathlineto{\pgfqpoint{4.911089in}{0.668813in}}%
\pgfpathlineto{\pgfqpoint{4.913667in}{0.710662in}}%
\pgfpathlineto{\pgfqpoint{4.923981in}{0.787524in}}%
\pgfpathlineto{\pgfqpoint{4.926559in}{0.666306in}}%
\pgfpathlineto{\pgfqpoint{4.929137in}{0.728240in}}%
\pgfpathlineto{\pgfqpoint{4.931716in}{0.717447in}}%
\pgfpathlineto{\pgfqpoint{4.934294in}{0.843578in}}%
\pgfpathlineto{\pgfqpoint{4.942029in}{0.721301in}}%
\pgfpathlineto{\pgfqpoint{4.944608in}{0.662037in}}%
\pgfpathlineto{\pgfqpoint{4.947186in}{0.715615in}}%
\pgfpathlineto{\pgfqpoint{4.949765in}{0.670815in}}%
\pgfpathlineto{\pgfqpoint{4.952343in}{0.838053in}}%
\pgfpathlineto{\pgfqpoint{4.962657in}{1.053883in}}%
\pgfpathlineto{\pgfqpoint{4.965235in}{0.665984in}}%
\pgfpathlineto{\pgfqpoint{4.967813in}{0.761142in}}%
\pgfpathlineto{\pgfqpoint{4.970392in}{0.709766in}}%
\pgfpathlineto{\pgfqpoint{4.978127in}{0.690539in}}%
\pgfpathlineto{\pgfqpoint{4.980705in}{1.128882in}}%
\pgfpathlineto{\pgfqpoint{4.983284in}{0.658891in}}%
\pgfpathlineto{\pgfqpoint{4.985862in}{0.781275in}}%
\pgfpathlineto{\pgfqpoint{4.988440in}{1.061264in}}%
\pgfpathlineto{\pgfqpoint{4.996176in}{0.695363in}}%
\pgfpathlineto{\pgfqpoint{4.998754in}{0.752604in}}%
\pgfpathlineto{\pgfqpoint{5.001332in}{0.748057in}}%
\pgfpathlineto{\pgfqpoint{5.003911in}{0.677285in}}%
\pgfpathlineto{\pgfqpoint{5.006489in}{0.687605in}}%
\pgfpathlineto{\pgfqpoint{5.014224in}{0.882374in}}%
\pgfpathlineto{\pgfqpoint{5.016803in}{0.658931in}}%
\pgfpathlineto{\pgfqpoint{5.019381in}{0.772999in}}%
\pgfpathlineto{\pgfqpoint{5.021960in}{0.996896in}}%
\pgfpathlineto{\pgfqpoint{5.024538in}{0.794038in}}%
\pgfpathlineto{\pgfqpoint{5.034852in}{0.684928in}}%
\pgfpathlineto{\pgfqpoint{5.037430in}{0.766049in}}%
\pgfpathlineto{\pgfqpoint{5.040008in}{0.683136in}}%
\pgfpathlineto{\pgfqpoint{5.042587in}{0.662883in}}%
\pgfpathlineto{\pgfqpoint{5.050322in}{0.671331in}}%
\pgfpathlineto{\pgfqpoint{5.052900in}{0.669723in}}%
\pgfpathlineto{\pgfqpoint{5.055479in}{0.666246in}}%
\pgfpathlineto{\pgfqpoint{5.058057in}{0.726708in}}%
\pgfpathlineto{\pgfqpoint{5.060635in}{0.830399in}}%
\pgfpathlineto{\pgfqpoint{5.068371in}{0.732881in}}%
\pgfpathlineto{\pgfqpoint{5.070949in}{0.759439in}}%
\pgfpathlineto{\pgfqpoint{5.073527in}{0.851897in}}%
\pgfpathlineto{\pgfqpoint{5.076106in}{0.668065in}}%
\pgfpathlineto{\pgfqpoint{5.078684in}{0.704091in}}%
\pgfpathlineto{\pgfqpoint{5.086419in}{0.675290in}}%
\pgfpathlineto{\pgfqpoint{5.088998in}{0.658997in}}%
\pgfpathlineto{\pgfqpoint{5.091576in}{0.659969in}}%
\pgfpathlineto{\pgfqpoint{5.094155in}{0.709547in}}%
\pgfpathlineto{\pgfqpoint{5.096733in}{0.690718in}}%
\pgfpathlineto{\pgfqpoint{5.104468in}{0.696761in}}%
\pgfpathlineto{\pgfqpoint{5.107046in}{0.661257in}}%
\pgfpathlineto{\pgfqpoint{5.109625in}{0.659003in}}%
\pgfpathlineto{\pgfqpoint{5.112203in}{0.878120in}}%
\pgfpathlineto{\pgfqpoint{5.114782in}{0.676385in}}%
\pgfpathlineto{\pgfqpoint{5.122517in}{0.661895in}}%
\pgfpathlineto{\pgfqpoint{5.125095in}{0.695337in}}%
\pgfpathlineto{\pgfqpoint{5.127674in}{0.659541in}}%
\pgfpathlineto{\pgfqpoint{5.130252in}{0.659364in}}%
\pgfpathlineto{\pgfqpoint{5.140566in}{0.666695in}}%
\pgfpathlineto{\pgfqpoint{5.143144in}{0.662355in}}%
\pgfpathlineto{\pgfqpoint{5.145722in}{0.660234in}}%
\pgfpathlineto{\pgfqpoint{5.148301in}{0.674667in}}%
\pgfpathlineto{\pgfqpoint{5.150879in}{0.822480in}}%
\pgfpathlineto{\pgfqpoint{5.158614in}{0.670124in}}%
\pgfpathlineto{\pgfqpoint{5.161193in}{0.659091in}}%
\pgfpathlineto{\pgfqpoint{5.163771in}{0.703741in}}%
\pgfpathlineto{\pgfqpoint{5.166349in}{0.693532in}}%
\pgfpathlineto{\pgfqpoint{5.168928in}{0.659091in}}%
\pgfpathlineto{\pgfqpoint{5.176663in}{0.682701in}}%
\pgfpathlineto{\pgfqpoint{5.179241in}{0.659975in}}%
\pgfpathlineto{\pgfqpoint{5.181820in}{0.673698in}}%
\pgfpathlineto{\pgfqpoint{5.184398in}{0.664330in}}%
\pgfpathlineto{\pgfqpoint{5.186977in}{0.668036in}}%
\pgfpathlineto{\pgfqpoint{5.194712in}{0.689786in}}%
\pgfpathlineto{\pgfqpoint{5.197290in}{0.680090in}}%
\pgfpathlineto{\pgfqpoint{5.199869in}{0.986609in}}%
\pgfpathlineto{\pgfqpoint{5.202447in}{0.723062in}}%
\pgfpathlineto{\pgfqpoint{5.205025in}{0.661617in}}%
\pgfpathlineto{\pgfqpoint{5.212761in}{0.683355in}}%
\pgfpathlineto{\pgfqpoint{5.215339in}{1.057397in}}%
\pgfpathlineto{\pgfqpoint{5.217917in}{0.671828in}}%
\pgfpathlineto{\pgfqpoint{5.220496in}{0.660613in}}%
\pgfpathlineto{\pgfqpoint{5.223074in}{0.673531in}}%
\pgfpathlineto{\pgfqpoint{5.230809in}{0.742461in}}%
\pgfpathlineto{\pgfqpoint{5.233388in}{0.660881in}}%
\pgfpathlineto{\pgfqpoint{5.235966in}{0.686859in}}%
\pgfpathlineto{\pgfqpoint{5.238544in}{0.668925in}}%
\pgfpathlineto{\pgfqpoint{5.241123in}{0.735912in}}%
\pgfpathlineto{\pgfqpoint{5.248858in}{0.658890in}}%
\pgfpathlineto{\pgfqpoint{5.251436in}{0.672702in}}%
\pgfpathlineto{\pgfqpoint{5.254015in}{0.671025in}}%
\pgfpathlineto{\pgfqpoint{5.256593in}{0.666658in}}%
\pgfpathlineto{\pgfqpoint{5.259172in}{0.815295in}}%
\pgfpathlineto{\pgfqpoint{5.266907in}{0.676753in}}%
\pgfpathlineto{\pgfqpoint{5.269485in}{0.773967in}}%
\pgfpathlineto{\pgfqpoint{5.274642in}{0.671731in}}%
\pgfpathlineto{\pgfqpoint{5.277220in}{0.662083in}}%
\pgfpathlineto{\pgfqpoint{5.284955in}{0.662525in}}%
\pgfpathlineto{\pgfqpoint{5.287534in}{0.727888in}}%
\pgfpathlineto{\pgfqpoint{5.290112in}{0.688208in}}%
\pgfpathlineto{\pgfqpoint{5.292691in}{0.666837in}}%
\pgfpathlineto{\pgfqpoint{5.295269in}{0.663869in}}%
\pgfpathlineto{\pgfqpoint{5.305583in}{0.675855in}}%
\pgfpathlineto{\pgfqpoint{5.308161in}{0.722678in}}%
\pgfpathlineto{\pgfqpoint{5.310739in}{0.661551in}}%
\pgfpathlineto{\pgfqpoint{5.313318in}{0.688771in}}%
\pgfpathlineto{\pgfqpoint{5.321053in}{0.668644in}}%
\pgfpathlineto{\pgfqpoint{5.323631in}{0.680352in}}%
\pgfpathlineto{\pgfqpoint{5.326210in}{0.670181in}}%
\pgfpathlineto{\pgfqpoint{5.328788in}{0.688270in}}%
\pgfpathlineto{\pgfqpoint{5.331367in}{0.658995in}}%
\pgfpathlineto{\pgfqpoint{5.339102in}{0.702132in}}%
\pgfpathlineto{\pgfqpoint{5.341680in}{0.725865in}}%
\pgfpathlineto{\pgfqpoint{5.344258in}{0.676018in}}%
\pgfpathlineto{\pgfqpoint{5.346837in}{0.681095in}}%
\pgfpathlineto{\pgfqpoint{5.349415in}{0.667120in}}%
\pgfpathlineto{\pgfqpoint{5.357150in}{0.659724in}}%
\pgfpathlineto{\pgfqpoint{5.359729in}{0.673255in}}%
\pgfpathlineto{\pgfqpoint{5.362307in}{0.661828in}}%
\pgfpathlineto{\pgfqpoint{5.364886in}{0.701099in}}%
\pgfpathlineto{\pgfqpoint{5.367464in}{1.067861in}}%
\pgfpathlineto{\pgfqpoint{5.375199in}{0.777264in}}%
\pgfpathlineto{\pgfqpoint{5.380356in}{0.885720in}}%
\pgfpathlineto{\pgfqpoint{5.382934in}{0.720164in}}%
\pgfpathlineto{\pgfqpoint{5.385513in}{0.660150in}}%
\pgfpathlineto{\pgfqpoint{5.395826in}{0.702828in}}%
\pgfpathlineto{\pgfqpoint{5.398405in}{0.676112in}}%
\pgfpathlineto{\pgfqpoint{5.400983in}{0.662985in}}%
\pgfpathlineto{\pgfqpoint{5.403561in}{0.749699in}}%
\pgfpathlineto{\pgfqpoint{5.411297in}{0.658926in}}%
\pgfpathlineto{\pgfqpoint{5.413875in}{0.658876in}}%
\pgfpathlineto{\pgfqpoint{5.416453in}{0.660975in}}%
\pgfpathlineto{\pgfqpoint{5.419032in}{0.658926in}}%
\pgfpathlineto{\pgfqpoint{5.421610in}{0.660969in}}%
\pgfpathlineto{\pgfqpoint{5.429345in}{0.659482in}}%
\pgfpathlineto{\pgfqpoint{5.431924in}{0.664323in}}%
\pgfpathlineto{\pgfqpoint{5.437081in}{0.659322in}}%
\pgfpathlineto{\pgfqpoint{5.439659in}{0.680611in}}%
\pgfpathlineto{\pgfqpoint{5.447394in}{0.659476in}}%
\pgfpathlineto{\pgfqpoint{5.449973in}{0.687235in}}%
\pgfpathlineto{\pgfqpoint{5.452551in}{0.725500in}}%
\pgfpathlineto{\pgfqpoint{5.455129in}{0.670956in}}%
\pgfpathlineto{\pgfqpoint{5.457708in}{0.721413in}}%
\pgfpathlineto{\pgfqpoint{5.465443in}{0.727362in}}%
\pgfpathlineto{\pgfqpoint{5.468021in}{0.671175in}}%
\pgfpathlineto{\pgfqpoint{5.470600in}{0.721455in}}%
\pgfpathlineto{\pgfqpoint{5.473178in}{0.659472in}}%
\pgfpathlineto{\pgfqpoint{5.475756in}{0.665902in}}%
\pgfpathlineto{\pgfqpoint{5.483492in}{0.658925in}}%
\pgfpathlineto{\pgfqpoint{5.488648in}{0.669079in}}%
\pgfpathlineto{\pgfqpoint{5.491227in}{0.676245in}}%
\pgfpathlineto{\pgfqpoint{5.493805in}{0.668228in}}%
\pgfpathlineto{\pgfqpoint{5.501540in}{0.658924in}}%
\pgfpathlineto{\pgfqpoint{5.504119in}{0.678000in}}%
\pgfpathlineto{\pgfqpoint{5.506697in}{0.673527in}}%
\pgfpathlineto{\pgfqpoint{5.509276in}{0.681801in}}%
\pgfpathlineto{\pgfqpoint{5.511854in}{0.660574in}}%
\pgfpathlineto{\pgfqpoint{5.519589in}{0.679822in}}%
\pgfpathlineto{\pgfqpoint{5.522167in}{0.688532in}}%
\pgfpathlineto{\pgfqpoint{5.524746in}{0.659631in}}%
\pgfpathlineto{\pgfqpoint{5.527324in}{0.693086in}}%
\pgfpathlineto{\pgfqpoint{5.529903in}{0.669403in}}%
\pgfpathlineto{\pgfqpoint{5.537638in}{0.709597in}}%
\pgfpathlineto{\pgfqpoint{5.540216in}{0.714334in}}%
\pgfpathlineto{\pgfqpoint{5.542795in}{0.664071in}}%
\pgfpathlineto{\pgfqpoint{5.545373in}{0.755579in}}%
\pgfpathlineto{\pgfqpoint{5.547951in}{0.660032in}}%
\pgfpathlineto{\pgfqpoint{5.558265in}{0.677303in}}%
\pgfpathlineto{\pgfqpoint{5.560843in}{0.704718in}}%
\pgfpathlineto{\pgfqpoint{5.563422in}{0.661500in}}%
\pgfpathlineto{\pgfqpoint{5.566000in}{0.891976in}}%
\pgfpathlineto{\pgfqpoint{5.573735in}{1.050747in}}%
\pgfpathlineto{\pgfqpoint{5.576314in}{0.798142in}}%
\pgfpathlineto{\pgfqpoint{5.578892in}{0.658983in}}%
\pgfpathlineto{\pgfqpoint{5.581470in}{0.764930in}}%
\pgfpathlineto{\pgfqpoint{5.584049in}{0.658888in}}%
\pgfpathlineto{\pgfqpoint{5.591784in}{0.668609in}}%
\pgfpathlineto{\pgfqpoint{5.594362in}{0.663029in}}%
\pgfpathlineto{\pgfqpoint{5.596941in}{0.717245in}}%
\pgfpathlineto{\pgfqpoint{5.599519in}{0.795797in}}%
\pgfpathlineto{\pgfqpoint{5.602098in}{0.803491in}}%
\pgfpathlineto{\pgfqpoint{5.609833in}{0.659624in}}%
\pgfpathlineto{\pgfqpoint{5.612411in}{0.683422in}}%
\pgfpathlineto{\pgfqpoint{5.614990in}{0.772798in}}%
\pgfpathlineto{\pgfqpoint{5.617568in}{0.799431in}}%
\pgfpathlineto{\pgfqpoint{5.620146in}{0.875405in}}%
\pgfpathlineto{\pgfqpoint{5.627882in}{0.769775in}}%
\pgfpathlineto{\pgfqpoint{5.630460in}{0.667887in}}%
\pgfpathlineto{\pgfqpoint{5.633038in}{0.682102in}}%
\pgfpathlineto{\pgfqpoint{5.635617in}{0.672006in}}%
\pgfpathlineto{\pgfqpoint{5.638195in}{0.715398in}}%
\pgfpathlineto{\pgfqpoint{5.645930in}{0.741957in}}%
\pgfpathlineto{\pgfqpoint{5.648509in}{0.684100in}}%
\pgfpathlineto{\pgfqpoint{5.651087in}{0.658979in}}%
\pgfpathlineto{\pgfqpoint{5.653665in}{0.669948in}}%
\pgfpathlineto{\pgfqpoint{5.656244in}{0.662615in}}%
\pgfpathlineto{\pgfqpoint{5.663979in}{0.693945in}}%
\pgfpathlineto{\pgfqpoint{5.666557in}{0.672428in}}%
\pgfpathlineto{\pgfqpoint{5.669136in}{0.814099in}}%
\pgfpathlineto{\pgfqpoint{5.671714in}{0.697150in}}%
\pgfpathlineto{\pgfqpoint{5.674293in}{0.696326in}}%
\pgfpathlineto{\pgfqpoint{5.682028in}{0.664389in}}%
\pgfpathlineto{\pgfqpoint{5.684606in}{1.497309in}}%
\pgfpathlineto{\pgfqpoint{5.687185in}{0.676620in}}%
\pgfpathlineto{\pgfqpoint{5.689763in}{0.724793in}}%
\pgfpathlineto{\pgfqpoint{5.692341in}{0.665669in}}%
\pgfpathlineto{\pgfqpoint{5.700076in}{0.659064in}}%
\pgfpathlineto{\pgfqpoint{5.702655in}{0.659170in}}%
\pgfpathlineto{\pgfqpoint{5.707812in}{0.660871in}}%
\pgfpathlineto{\pgfqpoint{5.710390in}{0.891381in}}%
\pgfpathlineto{\pgfqpoint{5.718125in}{0.878303in}}%
\pgfpathlineto{\pgfqpoint{5.720704in}{0.739565in}}%
\pgfpathlineto{\pgfqpoint{5.723282in}{0.890103in}}%
\pgfpathlineto{\pgfqpoint{5.725860in}{1.585786in}}%
\pgfpathlineto{\pgfqpoint{5.728439in}{0.700427in}}%
\pgfpathlineto{\pgfqpoint{5.736174in}{0.694781in}}%
\pgfpathlineto{\pgfqpoint{5.738752in}{0.700383in}}%
\pgfpathlineto{\pgfqpoint{5.743909in}{0.660429in}}%
\pgfpathlineto{\pgfqpoint{5.746488in}{0.783902in}}%
\pgfpathlineto{\pgfqpoint{5.754223in}{0.704426in}}%
\pgfpathlineto{\pgfqpoint{5.756801in}{0.660445in}}%
\pgfpathlineto{\pgfqpoint{5.759379in}{0.659511in}}%
\pgfpathlineto{\pgfqpoint{5.764536in}{0.723649in}}%
\pgfpathlineto{\pgfqpoint{5.774850in}{0.662173in}}%
\pgfpathlineto{\pgfqpoint{5.780007in}{0.698617in}}%
\pgfpathlineto{\pgfqpoint{5.782585in}{0.691741in}}%
\pgfpathlineto{\pgfqpoint{5.790320in}{0.696686in}}%
\pgfpathlineto{\pgfqpoint{5.792899in}{0.659507in}}%
\pgfpathlineto{\pgfqpoint{5.795477in}{0.829822in}}%
\pgfpathlineto{\pgfqpoint{5.798055in}{0.707367in}}%
\pgfpathlineto{\pgfqpoint{5.800634in}{0.739422in}}%
\pgfpathlineto{\pgfqpoint{5.808369in}{0.717651in}}%
\pgfpathlineto{\pgfqpoint{5.810947in}{0.659182in}}%
\pgfpathlineto{\pgfqpoint{5.813526in}{0.726434in}}%
\pgfpathlineto{\pgfqpoint{5.816104in}{0.668606in}}%
\pgfpathlineto{\pgfqpoint{5.818683in}{0.658876in}}%
\pgfpathlineto{\pgfqpoint{5.826418in}{0.658987in}}%
\pgfpathlineto{\pgfqpoint{5.828996in}{0.660968in}}%
\pgfpathlineto{\pgfqpoint{5.831574in}{0.667945in}}%
\pgfpathlineto{\pgfqpoint{5.834153in}{0.662912in}}%
\pgfpathlineto{\pgfqpoint{5.836731in}{0.683894in}}%
\pgfpathlineto{\pgfqpoint{5.847045in}{0.672309in}}%
\pgfpathlineto{\pgfqpoint{5.849623in}{0.688808in}}%
\pgfpathlineto{\pgfqpoint{5.852202in}{0.667329in}}%
\pgfpathlineto{\pgfqpoint{5.854780in}{0.666690in}}%
\pgfpathlineto{\pgfqpoint{5.865094in}{0.660392in}}%
\pgfpathlineto{\pgfqpoint{5.867672in}{0.668646in}}%
\pgfpathlineto{\pgfqpoint{5.870250in}{0.690961in}}%
\pgfpathlineto{\pgfqpoint{5.872829in}{0.658986in}}%
\pgfpathlineto{\pgfqpoint{5.880564in}{0.700442in}}%
\pgfpathlineto{\pgfqpoint{5.883142in}{0.745565in}}%
\pgfpathlineto{\pgfqpoint{5.885721in}{0.666180in}}%
\pgfpathlineto{\pgfqpoint{5.888299in}{0.659684in}}%
\pgfpathlineto{\pgfqpoint{5.890877in}{0.662099in}}%
\pgfpathlineto{\pgfqpoint{5.901191in}{0.808736in}}%
\pgfpathlineto{\pgfqpoint{5.903769in}{0.674629in}}%
\pgfpathlineto{\pgfqpoint{5.906348in}{0.664749in}}%
\pgfpathlineto{\pgfqpoint{5.908926in}{1.419123in}}%
\pgfpathlineto{\pgfqpoint{5.916661in}{0.682120in}}%
\pgfpathlineto{\pgfqpoint{5.919240in}{0.737580in}}%
\pgfpathlineto{\pgfqpoint{5.921818in}{0.705561in}}%
\pgfpathlineto{\pgfqpoint{5.924397in}{0.690142in}}%
\pgfpathlineto{\pgfqpoint{5.926975in}{0.660282in}}%
\pgfpathlineto{\pgfqpoint{5.934710in}{0.658980in}}%
\pgfpathlineto{\pgfqpoint{5.937289in}{0.728754in}}%
\pgfpathlineto{\pgfqpoint{5.939867in}{0.666007in}}%
\pgfpathlineto{\pgfqpoint{5.942445in}{0.677099in}}%
\pgfpathlineto{\pgfqpoint{5.945024in}{0.670527in}}%
\pgfpathlineto{\pgfqpoint{5.952759in}{0.658887in}}%
\pgfpathlineto{\pgfqpoint{5.955337in}{0.694460in}}%
\pgfpathlineto{\pgfqpoint{5.957916in}{0.668981in}}%
\pgfpathlineto{\pgfqpoint{5.960494in}{0.669585in}}%
\pgfpathlineto{\pgfqpoint{5.963072in}{0.704707in}}%
\pgfpathlineto{\pgfqpoint{5.970808in}{0.669672in}}%
\pgfpathlineto{\pgfqpoint{5.973386in}{0.678720in}}%
\pgfpathlineto{\pgfqpoint{5.975964in}{1.653099in}}%
\pgfpathlineto{\pgfqpoint{5.978543in}{0.669015in}}%
\pgfpathlineto{\pgfqpoint{5.981121in}{0.667150in}}%
\pgfpathlineto{\pgfqpoint{5.991435in}{0.688238in}}%
\pgfpathlineto{\pgfqpoint{5.994013in}{0.663468in}}%
\pgfpathlineto{\pgfqpoint{5.996592in}{0.667088in}}%
\pgfpathlineto{\pgfqpoint{5.999170in}{0.659549in}}%
\pgfpathlineto{\pgfqpoint{6.009483in}{0.661924in}}%
\pgfpathlineto{\pgfqpoint{6.012062in}{0.690504in}}%
\pgfpathlineto{\pgfqpoint{6.014640in}{0.710199in}}%
\pgfpathlineto{\pgfqpoint{6.017219in}{0.674196in}}%
\pgfpathlineto{\pgfqpoint{6.024954in}{0.660413in}}%
\pgfpathlineto{\pgfqpoint{6.027532in}{0.659400in}}%
\pgfpathlineto{\pgfqpoint{6.032689in}{0.662745in}}%
\pgfpathlineto{\pgfqpoint{6.035267in}{0.706496in}}%
\pgfpathlineto{\pgfqpoint{6.043003in}{0.663956in}}%
\pgfpathlineto{\pgfqpoint{6.045581in}{0.667111in}}%
\pgfpathlineto{\pgfqpoint{6.048159in}{0.673241in}}%
\pgfpathlineto{\pgfqpoint{6.050738in}{0.658970in}}%
\pgfpathlineto{\pgfqpoint{6.053316in}{0.675658in}}%
\pgfpathlineto{\pgfqpoint{6.063630in}{0.658970in}}%
\pgfpathlineto{\pgfqpoint{6.066208in}{0.662284in}}%
\pgfpathlineto{\pgfqpoint{6.068786in}{0.663536in}}%
\pgfpathlineto{\pgfqpoint{6.071365in}{0.662316in}}%
\pgfpathlineto{\pgfqpoint{6.079100in}{0.659558in}}%
\pgfpathlineto{\pgfqpoint{6.081678in}{0.665519in}}%
\pgfpathlineto{\pgfqpoint{6.084257in}{0.660956in}}%
\pgfpathlineto{\pgfqpoint{6.086835in}{0.673496in}}%
\pgfpathlineto{\pgfqpoint{6.097149in}{0.661313in}}%
\pgfpathlineto{\pgfqpoint{6.099727in}{0.663649in}}%
\pgfpathlineto{\pgfqpoint{6.102306in}{0.659146in}}%
\pgfpathlineto{\pgfqpoint{6.104884in}{0.688218in}}%
\pgfpathlineto{\pgfqpoint{6.107462in}{0.661677in}}%
\pgfpathlineto{\pgfqpoint{6.115198in}{0.665172in}}%
\pgfpathlineto{\pgfqpoint{6.117776in}{0.668010in}}%
\pgfpathlineto{\pgfqpoint{6.120354in}{0.682649in}}%
\pgfpathlineto{\pgfqpoint{6.122933in}{0.666133in}}%
\pgfpathlineto{\pgfqpoint{6.133246in}{0.670556in}}%
\pgfpathlineto{\pgfqpoint{6.135825in}{0.674233in}}%
\pgfpathlineto{\pgfqpoint{6.138403in}{0.682394in}}%
\pgfpathlineto{\pgfqpoint{6.140981in}{0.666136in}}%
\pgfpathlineto{\pgfqpoint{6.143560in}{0.706179in}}%
\pgfpathlineto{\pgfqpoint{6.151295in}{0.740863in}}%
\pgfpathlineto{\pgfqpoint{6.153873in}{0.677690in}}%
\pgfpathlineto{\pgfqpoint{6.156452in}{1.143504in}}%
\pgfpathlineto{\pgfqpoint{6.159030in}{0.659055in}}%
\pgfpathlineto{\pgfqpoint{6.161609in}{0.671852in}}%
\pgfpathlineto{\pgfqpoint{6.169344in}{0.664346in}}%
\pgfpathlineto{\pgfqpoint{6.171922in}{0.733981in}}%
\pgfpathlineto{\pgfqpoint{6.174501in}{0.663040in}}%
\pgfpathlineto{\pgfqpoint{6.177079in}{0.658922in}}%
\pgfpathlineto{\pgfqpoint{6.179657in}{0.659163in}}%
\pgfpathlineto{\pgfqpoint{6.187392in}{0.659163in}}%
\pgfpathlineto{\pgfqpoint{6.189971in}{0.661129in}}%
\pgfpathlineto{\pgfqpoint{6.192549in}{0.659612in}}%
\pgfpathlineto{\pgfqpoint{6.195128in}{0.669257in}}%
\pgfpathlineto{\pgfqpoint{6.197706in}{0.658922in}}%
\pgfpathlineto{\pgfqpoint{6.205441in}{0.660828in}}%
\pgfpathlineto{\pgfqpoint{6.210598in}{0.658922in}}%
\pgfpathlineto{\pgfqpoint{6.213176in}{0.674763in}}%
\pgfpathlineto{\pgfqpoint{6.215755in}{0.673095in}}%
\pgfpathlineto{\pgfqpoint{6.223490in}{0.658980in}}%
\pgfpathlineto{\pgfqpoint{6.226068in}{0.660544in}}%
\pgfpathlineto{\pgfqpoint{6.228647in}{0.676435in}}%
\pgfpathlineto{\pgfqpoint{6.231225in}{0.671336in}}%
\pgfpathlineto{\pgfqpoint{6.233804in}{0.674404in}}%
\pgfpathlineto{\pgfqpoint{6.244117in}{0.661085in}}%
\pgfpathlineto{\pgfqpoint{6.246695in}{0.704606in}}%
\pgfpathlineto{\pgfqpoint{6.249274in}{0.659053in}}%
\pgfpathlineto{\pgfqpoint{6.251852in}{0.678298in}}%
\pgfpathlineto{\pgfqpoint{6.259587in}{0.661019in}}%
\pgfpathlineto{\pgfqpoint{6.262166in}{0.659269in}}%
\pgfpathlineto{\pgfqpoint{6.264744in}{0.658974in}}%
\pgfpathlineto{\pgfqpoint{6.267323in}{0.738513in}}%
\pgfpathlineto{\pgfqpoint{6.269901in}{0.667578in}}%
\pgfpathlineto{\pgfqpoint{6.277636in}{0.659583in}}%
\pgfpathlineto{\pgfqpoint{6.280215in}{0.662073in}}%
\pgfpathlineto{\pgfqpoint{6.282793in}{0.673180in}}%
\pgfpathlineto{\pgfqpoint{6.285371in}{0.741180in}}%
\pgfpathlineto{\pgfqpoint{6.287950in}{0.666127in}}%
\pgfpathlineto{\pgfqpoint{6.295685in}{0.662728in}}%
\pgfpathlineto{\pgfqpoint{6.298263in}{0.664042in}}%
\pgfpathlineto{\pgfqpoint{6.300842in}{0.663607in}}%
\pgfpathlineto{\pgfqpoint{6.303420in}{0.672107in}}%
\pgfpathlineto{\pgfqpoint{6.305998in}{0.673658in}}%
\pgfpathlineto{\pgfqpoint{6.313734in}{0.659263in}}%
\pgfpathlineto{\pgfqpoint{6.316312in}{0.712074in}}%
\pgfpathlineto{\pgfqpoint{6.318890in}{0.664190in}}%
\pgfpathlineto{\pgfqpoint{6.321469in}{0.842070in}}%
\pgfpathlineto{\pgfqpoint{6.324047in}{0.661098in}}%
\pgfpathlineto{\pgfqpoint{6.331782in}{0.692885in}}%
\pgfpathlineto{\pgfqpoint{6.336939in}{0.659278in}}%
\pgfpathlineto{\pgfqpoint{6.339518in}{0.667674in}}%
\pgfpathlineto{\pgfqpoint{6.342096in}{0.666464in}}%
\pgfpathlineto{\pgfqpoint{6.349831in}{0.689325in}}%
\pgfpathlineto{\pgfqpoint{6.352410in}{0.669114in}}%
\pgfpathlineto{\pgfqpoint{6.354988in}{0.662567in}}%
\pgfpathlineto{\pgfqpoint{6.357566in}{0.665443in}}%
\pgfpathlineto{\pgfqpoint{6.360145in}{0.674458in}}%
\pgfpathlineto{\pgfqpoint{6.367880in}{0.678759in}}%
\pgfpathlineto{\pgfqpoint{6.370458in}{0.771866in}}%
\pgfpathlineto{\pgfqpoint{6.373037in}{0.674602in}}%
\pgfpathlineto{\pgfqpoint{6.375615in}{0.664595in}}%
\pgfpathlineto{\pgfqpoint{6.378193in}{0.658887in}}%
\pgfpathlineto{\pgfqpoint{6.385929in}{0.676205in}}%
\pgfpathlineto{\pgfqpoint{6.388507in}{0.746078in}}%
\pgfpathlineto{\pgfqpoint{6.391085in}{0.661268in}}%
\pgfpathlineto{\pgfqpoint{6.393664in}{0.832788in}}%
\pgfpathlineto{\pgfqpoint{6.396242in}{0.678906in}}%
\pgfpathlineto{\pgfqpoint{6.403977in}{0.692439in}}%
\pgfpathlineto{\pgfqpoint{6.406556in}{0.666333in}}%
\pgfpathlineto{\pgfqpoint{6.409134in}{0.659243in}}%
\pgfpathlineto{\pgfqpoint{6.411713in}{0.661833in}}%
\pgfpathlineto{\pgfqpoint{6.414291in}{0.662205in}}%
\pgfpathlineto{\pgfqpoint{6.422026in}{0.713271in}}%
\pgfpathlineto{\pgfqpoint{6.424604in}{0.660584in}}%
\pgfpathlineto{\pgfqpoint{6.427183in}{0.669170in}}%
\pgfpathlineto{\pgfqpoint{6.429761in}{0.661776in}}%
\pgfpathlineto{\pgfqpoint{6.432340in}{0.673460in}}%
\pgfpathlineto{\pgfqpoint{6.440075in}{0.667379in}}%
\pgfpathlineto{\pgfqpoint{6.442653in}{0.685958in}}%
\pgfpathlineto{\pgfqpoint{6.445232in}{0.664128in}}%
\pgfpathlineto{\pgfqpoint{6.447810in}{0.671056in}}%
\pgfpathlineto{\pgfqpoint{6.450388in}{0.672484in}}%
\pgfpathlineto{\pgfqpoint{6.458124in}{0.672369in}}%
\pgfpathlineto{\pgfqpoint{6.460702in}{0.659357in}}%
\pgfpathlineto{\pgfqpoint{6.463280in}{0.668979in}}%
\pgfpathlineto{\pgfqpoint{6.465859in}{0.661416in}}%
\pgfpathlineto{\pgfqpoint{6.468437in}{0.663674in}}%
\pgfpathlineto{\pgfqpoint{6.476172in}{0.659034in}}%
\pgfpathlineto{\pgfqpoint{6.478751in}{0.660819in}}%
\pgfpathlineto{\pgfqpoint{6.481329in}{0.676478in}}%
\pgfpathlineto{\pgfqpoint{6.483907in}{0.673291in}}%
\pgfpathlineto{\pgfqpoint{6.486486in}{0.664585in}}%
\pgfpathlineto{\pgfqpoint{6.496799in}{0.662072in}}%
\pgfpathlineto{\pgfqpoint{6.499378in}{0.658876in}}%
\pgfpathlineto{\pgfqpoint{6.501956in}{0.664070in}}%
\pgfpathlineto{\pgfqpoint{6.504535in}{0.660288in}}%
\pgfpathlineto{\pgfqpoint{6.512270in}{0.771996in}}%
\pgfpathlineto{\pgfqpoint{6.514848in}{0.678374in}}%
\pgfpathlineto{\pgfqpoint{6.517427in}{0.659031in}}%
\pgfpathlineto{\pgfqpoint{6.520005in}{0.658876in}}%
\pgfpathlineto{\pgfqpoint{6.522583in}{0.665947in}}%
\pgfpathlineto{\pgfqpoint{6.530319in}{0.660055in}}%
\pgfpathlineto{\pgfqpoint{6.532897in}{0.747788in}}%
\pgfpathlineto{\pgfqpoint{6.535475in}{0.663074in}}%
\pgfpathlineto{\pgfqpoint{6.538054in}{0.922349in}}%
\pgfpathlineto{\pgfqpoint{6.540632in}{0.672434in}}%
\pgfpathlineto{\pgfqpoint{6.548367in}{0.678913in}}%
\pgfpathlineto{\pgfqpoint{6.550946in}{0.659359in}}%
\pgfpathlineto{\pgfqpoint{6.553524in}{0.939217in}}%
\pgfpathlineto{\pgfqpoint{6.556102in}{0.658917in}}%
\pgfpathlineto{\pgfqpoint{6.558681in}{0.659531in}}%
\pgfpathlineto{\pgfqpoint{6.566416in}{0.715892in}}%
\pgfpathlineto{\pgfqpoint{6.568994in}{0.669127in}}%
\pgfpathlineto{\pgfqpoint{6.571573in}{0.667236in}}%
\pgfpathlineto{\pgfqpoint{6.574151in}{0.672497in}}%
\pgfpathlineto{\pgfqpoint{6.576730in}{0.666685in}}%
\pgfpathlineto{\pgfqpoint{6.584465in}{0.662856in}}%
\pgfpathlineto{\pgfqpoint{6.587043in}{0.680093in}}%
\pgfpathlineto{\pgfqpoint{6.589622in}{0.661148in}}%
\pgfpathlineto{\pgfqpoint{6.592200in}{0.700004in}}%
\pgfpathlineto{\pgfqpoint{6.594778in}{0.728524in}}%
\pgfpathlineto{\pgfqpoint{6.602513in}{0.659667in}}%
\pgfpathlineto{\pgfqpoint{6.605092in}{0.668903in}}%
\pgfpathlineto{\pgfqpoint{6.607670in}{0.658965in}}%
\pgfpathlineto{\pgfqpoint{6.610249in}{0.679796in}}%
\pgfpathlineto{\pgfqpoint{6.612827in}{1.688827in}}%
\pgfpathlineto{\pgfqpoint{6.620562in}{0.746541in}}%
\pgfpathlineto{\pgfqpoint{6.623141in}{0.668759in}}%
\pgfpathlineto{\pgfqpoint{6.625719in}{0.660211in}}%
\pgfpathlineto{\pgfqpoint{6.628297in}{0.698369in}}%
\pgfpathlineto{\pgfqpoint{6.630876in}{0.679118in}}%
\pgfpathlineto{\pgfqpoint{6.638611in}{0.718045in}}%
\pgfpathlineto{\pgfqpoint{6.641189in}{0.659425in}}%
\pgfpathlineto{\pgfqpoint{6.643768in}{0.688925in}}%
\pgfpathlineto{\pgfqpoint{6.648925in}{0.659277in}}%
\pgfpathlineto{\pgfqpoint{6.656660in}{0.686856in}}%
\pgfpathlineto{\pgfqpoint{6.659238in}{0.745144in}}%
\pgfpathlineto{\pgfqpoint{6.661816in}{0.694436in}}%
\pgfpathlineto{\pgfqpoint{6.664395in}{0.662791in}}%
\pgfpathlineto{\pgfqpoint{6.666973in}{0.672838in}}%
\pgfpathlineto{\pgfqpoint{6.674708in}{0.725193in}}%
\pgfpathlineto{\pgfqpoint{6.677287in}{0.660394in}}%
\pgfpathlineto{\pgfqpoint{6.679865in}{0.698443in}}%
\pgfpathlineto{\pgfqpoint{6.682444in}{0.758500in}}%
\pgfpathlineto{\pgfqpoint{6.685022in}{0.723148in}}%
\pgfpathlineto{\pgfqpoint{6.692757in}{0.661279in}}%
\pgfpathlineto{\pgfqpoint{6.695336in}{0.678561in}}%
\pgfpathlineto{\pgfqpoint{6.697914in}{0.673440in}}%
\pgfpathlineto{\pgfqpoint{6.703071in}{0.660167in}}%
\pgfpathlineto{\pgfqpoint{6.710806in}{0.683291in}}%
\pgfpathlineto{\pgfqpoint{6.713384in}{0.677370in}}%
\pgfpathlineto{\pgfqpoint{6.715963in}{0.658918in}}%
\pgfpathlineto{\pgfqpoint{6.718541in}{0.693746in}}%
\pgfpathlineto{\pgfqpoint{6.721119in}{0.671436in}}%
\pgfpathlineto{\pgfqpoint{6.728855in}{0.757805in}}%
\pgfpathlineto{\pgfqpoint{6.731433in}{0.658886in}}%
\pgfpathlineto{\pgfqpoint{6.734011in}{0.660835in}}%
\pgfpathlineto{\pgfqpoint{6.736590in}{0.779361in}}%
\pgfpathlineto{\pgfqpoint{6.739168in}{0.665798in}}%
\pgfpathlineto{\pgfqpoint{6.746903in}{0.660604in}}%
\pgfpathlineto{\pgfqpoint{6.749482in}{0.672203in}}%
\pgfpathlineto{\pgfqpoint{6.752060in}{0.754941in}}%
\pgfpathlineto{\pgfqpoint{6.754639in}{0.660096in}}%
\pgfpathlineto{\pgfqpoint{6.757217in}{0.729235in}}%
\pgfpathlineto{\pgfqpoint{6.764952in}{0.658876in}}%
\pgfpathlineto{\pgfqpoint{6.767531in}{0.662449in}}%
\pgfpathlineto{\pgfqpoint{6.770109in}{0.661112in}}%
\pgfpathlineto{\pgfqpoint{6.772687in}{0.660824in}}%
\pgfpathlineto{\pgfqpoint{6.775266in}{0.677129in}}%
\pgfpathlineto{\pgfqpoint{6.785579in}{0.669533in}}%
\pgfpathlineto{\pgfqpoint{6.788158in}{0.671563in}}%
\pgfpathlineto{\pgfqpoint{6.790736in}{0.658965in}}%
\pgfpathlineto{\pgfqpoint{6.793314in}{0.661724in}}%
\pgfpathlineto{\pgfqpoint{6.793314in}{0.661724in}}%
\pgfusepath{stroke}%
\end{pgfscope}%
\begin{pgfscope}%
\pgfsetrectcap%
\pgfsetmiterjoin%
\pgfsetlinewidth{0.803000pt}%
\definecolor{currentstroke}{rgb}{1.000000,1.000000,1.000000}%
\pgfsetstrokecolor{currentstroke}%
\pgfsetdash{}{0pt}%
\pgfpathmoveto{\pgfqpoint{0.875133in}{0.521603in}}%
\pgfpathlineto{\pgfqpoint{0.875133in}{3.541603in}}%
\pgfusepath{stroke}%
\end{pgfscope}%
\begin{pgfscope}%
\pgfsetrectcap%
\pgfsetmiterjoin%
\pgfsetlinewidth{0.803000pt}%
\definecolor{currentstroke}{rgb}{1.000000,1.000000,1.000000}%
\pgfsetstrokecolor{currentstroke}%
\pgfsetdash{}{0pt}%
\pgfpathmoveto{\pgfqpoint{7.075133in}{0.521603in}}%
\pgfpathlineto{\pgfqpoint{7.075133in}{3.541603in}}%
\pgfusepath{stroke}%
\end{pgfscope}%
\begin{pgfscope}%
\pgfsetrectcap%
\pgfsetmiterjoin%
\pgfsetlinewidth{0.803000pt}%
\definecolor{currentstroke}{rgb}{1.000000,1.000000,1.000000}%
\pgfsetstrokecolor{currentstroke}%
\pgfsetdash{}{0pt}%
\pgfpathmoveto{\pgfqpoint{0.875133in}{0.521603in}}%
\pgfpathlineto{\pgfqpoint{7.075133in}{0.521603in}}%
\pgfusepath{stroke}%
\end{pgfscope}%
\begin{pgfscope}%
\pgfsetrectcap%
\pgfsetmiterjoin%
\pgfsetlinewidth{0.803000pt}%
\definecolor{currentstroke}{rgb}{1.000000,1.000000,1.000000}%
\pgfsetstrokecolor{currentstroke}%
\pgfsetdash{}{0pt}%
\pgfpathmoveto{\pgfqpoint{0.875133in}{3.541603in}}%
\pgfpathlineto{\pgfqpoint{7.075133in}{3.541603in}}%
\pgfusepath{stroke}%
\end{pgfscope}%
\end{pgfpicture}%
\makeatother%
\endgroup%

    \end{adjustbox}  
    \caption{Plot of squared log returns of stock PG. This serves as an approximation of the variance of the log returns, as $Var(x) = E [(x - E(x))^2$ and the mean of the log returns is close to zero. The pattern looks similar for all stocks, therefore only one is shown.}
    \label{fig:PG_squared_log_returns}
\end{figure}{}



\begin{figure}[H]
    \centering
    \figuretitle{ACF / PACF of Squared log-returns of Stocks MMM, GE, JNJ}
    \begin{adjustbox}{width=.95\textwidth,center}
    %% Creator: Matplotlib, PGF backend
%%
%% To include the figure in your LaTeX document, write
%%   \input{<filename>.pgf}
%%
%% Make sure the required packages are loaded in your preamble
%%   \usepackage{pgf}
%%
%% Figures using additional raster images can only be included by \input if
%% they are in the same directory as the main LaTeX file. For loading figures
%% from other directories you can use the `import` package
%%   \usepackage{import}
%% and then include the figures with
%%   \import{<path to file>}{<filename>.pgf}
%%
%% Matplotlib used the following preamble
%%   \usepackage{fontspec}
%%   \setmainfont{DejaVuSerif.ttf}[Path=/opt/tljh/user/lib/python3.6/site-packages/matplotlib/mpl-data/fonts/ttf/]
%%   \setsansfont{DejaVuSans.ttf}[Path=/opt/tljh/user/lib/python3.6/site-packages/matplotlib/mpl-data/fonts/ttf/]
%%   \setmonofont{DejaVuSansMono.ttf}[Path=/opt/tljh/user/lib/python3.6/site-packages/matplotlib/mpl-data/fonts/ttf/]
%%
\begingroup%
\makeatletter%
\begin{pgfpicture}%
\pgfpathrectangle{\pgfpointorigin}{\pgfqpoint{6.718102in}{1.774096in}}%
\pgfusepath{use as bounding box, clip}%
\begin{pgfscope}%
\pgfsetbuttcap%
\pgfsetmiterjoin%
\definecolor{currentfill}{rgb}{1.000000,1.000000,1.000000}%
\pgfsetfillcolor{currentfill}%
\pgfsetlinewidth{0.000000pt}%
\definecolor{currentstroke}{rgb}{1.000000,1.000000,1.000000}%
\pgfsetstrokecolor{currentstroke}%
\pgfsetdash{}{0pt}%
\pgfpathmoveto{\pgfqpoint{0.000000in}{0.000000in}}%
\pgfpathlineto{\pgfqpoint{6.718102in}{0.000000in}}%
\pgfpathlineto{\pgfqpoint{6.718102in}{1.774096in}}%
\pgfpathlineto{\pgfqpoint{0.000000in}{1.774096in}}%
\pgfpathclose%
\pgfusepath{fill}%
\end{pgfscope}%
\begin{pgfscope}%
\pgfsetbuttcap%
\pgfsetmiterjoin%
\definecolor{currentfill}{rgb}{0.917647,0.917647,0.949020}%
\pgfsetfillcolor{currentfill}%
\pgfsetlinewidth{0.000000pt}%
\definecolor{currentstroke}{rgb}{0.000000,0.000000,0.000000}%
\pgfsetstrokecolor{currentstroke}%
\pgfsetstrokeopacity{0.000000}%
\pgfsetdash{}{0pt}%
\pgfpathmoveto{\pgfqpoint{0.418102in}{0.331635in}}%
\pgfpathlineto{\pgfqpoint{3.236283in}{0.331635in}}%
\pgfpathlineto{\pgfqpoint{3.236283in}{1.464135in}}%
\pgfpathlineto{\pgfqpoint{0.418102in}{1.464135in}}%
\pgfpathclose%
\pgfusepath{fill}%
\end{pgfscope}%
\begin{pgfscope}%
\pgfpathrectangle{\pgfqpoint{0.418102in}{0.331635in}}{\pgfqpoint{2.818182in}{1.132500in}}%
\pgfusepath{clip}%
\pgfsetroundcap%
\pgfsetroundjoin%
\pgfsetlinewidth{0.803000pt}%
\definecolor{currentstroke}{rgb}{1.000000,1.000000,1.000000}%
\pgfsetstrokecolor{currentstroke}%
\pgfsetdash{}{0pt}%
\pgfpathmoveto{\pgfqpoint{0.546201in}{0.331635in}}%
\pgfpathlineto{\pgfqpoint{0.546201in}{1.464135in}}%
\pgfusepath{stroke}%
\end{pgfscope}%
\begin{pgfscope}%
\definecolor{textcolor}{rgb}{0.150000,0.150000,0.150000}%
\pgfsetstrokecolor{textcolor}%
\pgfsetfillcolor{textcolor}%
\pgftext[x=0.546201in,y=0.234413in,,top]{\color{textcolor}\rmfamily\fontsize{10.000000}{12.000000}\selectfont 0}%
\end{pgfscope}%
\begin{pgfscope}%
\pgfpathrectangle{\pgfqpoint{0.418102in}{0.331635in}}{\pgfqpoint{2.818182in}{1.132500in}}%
\pgfusepath{clip}%
\pgfsetroundcap%
\pgfsetroundjoin%
\pgfsetlinewidth{0.803000pt}%
\definecolor{currentstroke}{rgb}{1.000000,1.000000,1.000000}%
\pgfsetstrokecolor{currentstroke}%
\pgfsetdash{}{0pt}%
\pgfpathmoveto{\pgfqpoint{1.171075in}{0.331635in}}%
\pgfpathlineto{\pgfqpoint{1.171075in}{1.464135in}}%
\pgfusepath{stroke}%
\end{pgfscope}%
\begin{pgfscope}%
\definecolor{textcolor}{rgb}{0.150000,0.150000,0.150000}%
\pgfsetstrokecolor{textcolor}%
\pgfsetfillcolor{textcolor}%
\pgftext[x=1.171075in,y=0.234413in,,top]{\color{textcolor}\rmfamily\fontsize{10.000000}{12.000000}\selectfont 5}%
\end{pgfscope}%
\begin{pgfscope}%
\pgfpathrectangle{\pgfqpoint{0.418102in}{0.331635in}}{\pgfqpoint{2.818182in}{1.132500in}}%
\pgfusepath{clip}%
\pgfsetroundcap%
\pgfsetroundjoin%
\pgfsetlinewidth{0.803000pt}%
\definecolor{currentstroke}{rgb}{1.000000,1.000000,1.000000}%
\pgfsetstrokecolor{currentstroke}%
\pgfsetdash{}{0pt}%
\pgfpathmoveto{\pgfqpoint{1.795949in}{0.331635in}}%
\pgfpathlineto{\pgfqpoint{1.795949in}{1.464135in}}%
\pgfusepath{stroke}%
\end{pgfscope}%
\begin{pgfscope}%
\definecolor{textcolor}{rgb}{0.150000,0.150000,0.150000}%
\pgfsetstrokecolor{textcolor}%
\pgfsetfillcolor{textcolor}%
\pgftext[x=1.795949in,y=0.234413in,,top]{\color{textcolor}\rmfamily\fontsize{10.000000}{12.000000}\selectfont 10}%
\end{pgfscope}%
\begin{pgfscope}%
\pgfpathrectangle{\pgfqpoint{0.418102in}{0.331635in}}{\pgfqpoint{2.818182in}{1.132500in}}%
\pgfusepath{clip}%
\pgfsetroundcap%
\pgfsetroundjoin%
\pgfsetlinewidth{0.803000pt}%
\definecolor{currentstroke}{rgb}{1.000000,1.000000,1.000000}%
\pgfsetstrokecolor{currentstroke}%
\pgfsetdash{}{0pt}%
\pgfpathmoveto{\pgfqpoint{2.420823in}{0.331635in}}%
\pgfpathlineto{\pgfqpoint{2.420823in}{1.464135in}}%
\pgfusepath{stroke}%
\end{pgfscope}%
\begin{pgfscope}%
\definecolor{textcolor}{rgb}{0.150000,0.150000,0.150000}%
\pgfsetstrokecolor{textcolor}%
\pgfsetfillcolor{textcolor}%
\pgftext[x=2.420823in,y=0.234413in,,top]{\color{textcolor}\rmfamily\fontsize{10.000000}{12.000000}\selectfont 15}%
\end{pgfscope}%
\begin{pgfscope}%
\pgfpathrectangle{\pgfqpoint{0.418102in}{0.331635in}}{\pgfqpoint{2.818182in}{1.132500in}}%
\pgfusepath{clip}%
\pgfsetroundcap%
\pgfsetroundjoin%
\pgfsetlinewidth{0.803000pt}%
\definecolor{currentstroke}{rgb}{1.000000,1.000000,1.000000}%
\pgfsetstrokecolor{currentstroke}%
\pgfsetdash{}{0pt}%
\pgfpathmoveto{\pgfqpoint{3.045697in}{0.331635in}}%
\pgfpathlineto{\pgfqpoint{3.045697in}{1.464135in}}%
\pgfusepath{stroke}%
\end{pgfscope}%
\begin{pgfscope}%
\definecolor{textcolor}{rgb}{0.150000,0.150000,0.150000}%
\pgfsetstrokecolor{textcolor}%
\pgfsetfillcolor{textcolor}%
\pgftext[x=3.045697in,y=0.234413in,,top]{\color{textcolor}\rmfamily\fontsize{10.000000}{12.000000}\selectfont 20}%
\end{pgfscope}%
\begin{pgfscope}%
\pgfpathrectangle{\pgfqpoint{0.418102in}{0.331635in}}{\pgfqpoint{2.818182in}{1.132500in}}%
\pgfusepath{clip}%
\pgfsetroundcap%
\pgfsetroundjoin%
\pgfsetlinewidth{0.803000pt}%
\definecolor{currentstroke}{rgb}{1.000000,1.000000,1.000000}%
\pgfsetstrokecolor{currentstroke}%
\pgfsetdash{}{0pt}%
\pgfpathmoveto{\pgfqpoint{0.418102in}{0.433693in}}%
\pgfpathlineto{\pgfqpoint{3.236283in}{0.433693in}}%
\pgfusepath{stroke}%
\end{pgfscope}%
\begin{pgfscope}%
\definecolor{textcolor}{rgb}{0.150000,0.150000,0.150000}%
\pgfsetstrokecolor{textcolor}%
\pgfsetfillcolor{textcolor}%
\pgftext[x=0.100000in,y=0.380931in,left,base]{\color{textcolor}\rmfamily\fontsize{10.000000}{12.000000}\selectfont 0.0}%
\end{pgfscope}%
\begin{pgfscope}%
\pgfpathrectangle{\pgfqpoint{0.418102in}{0.331635in}}{\pgfqpoint{2.818182in}{1.132500in}}%
\pgfusepath{clip}%
\pgfsetroundcap%
\pgfsetroundjoin%
\pgfsetlinewidth{0.803000pt}%
\definecolor{currentstroke}{rgb}{1.000000,1.000000,1.000000}%
\pgfsetstrokecolor{currentstroke}%
\pgfsetdash{}{0pt}%
\pgfpathmoveto{\pgfqpoint{0.418102in}{0.923175in}}%
\pgfpathlineto{\pgfqpoint{3.236283in}{0.923175in}}%
\pgfusepath{stroke}%
\end{pgfscope}%
\begin{pgfscope}%
\definecolor{textcolor}{rgb}{0.150000,0.150000,0.150000}%
\pgfsetstrokecolor{textcolor}%
\pgfsetfillcolor{textcolor}%
\pgftext[x=0.100000in,y=0.870414in,left,base]{\color{textcolor}\rmfamily\fontsize{10.000000}{12.000000}\selectfont 0.5}%
\end{pgfscope}%
\begin{pgfscope}%
\pgfpathrectangle{\pgfqpoint{0.418102in}{0.331635in}}{\pgfqpoint{2.818182in}{1.132500in}}%
\pgfusepath{clip}%
\pgfsetroundcap%
\pgfsetroundjoin%
\pgfsetlinewidth{0.803000pt}%
\definecolor{currentstroke}{rgb}{1.000000,1.000000,1.000000}%
\pgfsetstrokecolor{currentstroke}%
\pgfsetdash{}{0pt}%
\pgfpathmoveto{\pgfqpoint{0.418102in}{1.412658in}}%
\pgfpathlineto{\pgfqpoint{3.236283in}{1.412658in}}%
\pgfusepath{stroke}%
\end{pgfscope}%
\begin{pgfscope}%
\definecolor{textcolor}{rgb}{0.150000,0.150000,0.150000}%
\pgfsetstrokecolor{textcolor}%
\pgfsetfillcolor{textcolor}%
\pgftext[x=0.100000in,y=1.359896in,left,base]{\color{textcolor}\rmfamily\fontsize{10.000000}{12.000000}\selectfont 1.0}%
\end{pgfscope}%
\begin{pgfscope}%
\pgfpathrectangle{\pgfqpoint{0.418102in}{0.331635in}}{\pgfqpoint{2.818182in}{1.132500in}}%
\pgfusepath{clip}%
\pgfsetbuttcap%
\pgfsetroundjoin%
\definecolor{currentfill}{rgb}{0.121569,0.466667,0.705882}%
\pgfsetfillcolor{currentfill}%
\pgfsetfillopacity{0.250000}%
\pgfsetlinewidth{1.003750pt}%
\definecolor{currentstroke}{rgb}{1.000000,1.000000,1.000000}%
\pgfsetstrokecolor{currentstroke}%
\pgfsetstrokeopacity{0.250000}%
\pgfsetdash{}{0pt}%
\pgfpathmoveto{\pgfqpoint{0.608688in}{0.483103in}}%
\pgfpathlineto{\pgfqpoint{0.608688in}{0.384283in}}%
\pgfpathlineto{\pgfqpoint{0.796150in}{0.384149in}}%
\pgfpathlineto{\pgfqpoint{0.921125in}{0.383882in}}%
\pgfpathlineto{\pgfqpoint{1.046100in}{0.383837in}}%
\pgfpathlineto{\pgfqpoint{1.171075in}{0.383784in}}%
\pgfpathlineto{\pgfqpoint{1.296050in}{0.383779in}}%
\pgfpathlineto{\pgfqpoint{1.421024in}{0.383662in}}%
\pgfpathlineto{\pgfqpoint{1.545999in}{0.383563in}}%
\pgfpathlineto{\pgfqpoint{1.670974in}{0.383562in}}%
\pgfpathlineto{\pgfqpoint{1.795949in}{0.383509in}}%
\pgfpathlineto{\pgfqpoint{1.920924in}{0.383501in}}%
\pgfpathlineto{\pgfqpoint{2.045898in}{0.383463in}}%
\pgfpathlineto{\pgfqpoint{2.170873in}{0.383212in}}%
\pgfpathlineto{\pgfqpoint{2.295848in}{0.383198in}}%
\pgfpathlineto{\pgfqpoint{2.420823in}{0.383192in}}%
\pgfpathlineto{\pgfqpoint{2.545798in}{0.383128in}}%
\pgfpathlineto{\pgfqpoint{2.670772in}{0.383120in}}%
\pgfpathlineto{\pgfqpoint{2.795747in}{0.383113in}}%
\pgfpathlineto{\pgfqpoint{2.920722in}{0.383113in}}%
\pgfpathlineto{\pgfqpoint{3.108184in}{0.383112in}}%
\pgfpathlineto{\pgfqpoint{3.108184in}{0.484274in}}%
\pgfpathlineto{\pgfqpoint{3.108184in}{0.484274in}}%
\pgfpathlineto{\pgfqpoint{2.920722in}{0.484273in}}%
\pgfpathlineto{\pgfqpoint{2.795747in}{0.484273in}}%
\pgfpathlineto{\pgfqpoint{2.670772in}{0.484266in}}%
\pgfpathlineto{\pgfqpoint{2.545798in}{0.484258in}}%
\pgfpathlineto{\pgfqpoint{2.420823in}{0.484194in}}%
\pgfpathlineto{\pgfqpoint{2.295848in}{0.484188in}}%
\pgfpathlineto{\pgfqpoint{2.170873in}{0.484174in}}%
\pgfpathlineto{\pgfqpoint{2.045898in}{0.483923in}}%
\pgfpathlineto{\pgfqpoint{1.920924in}{0.483885in}}%
\pgfpathlineto{\pgfqpoint{1.795949in}{0.483877in}}%
\pgfpathlineto{\pgfqpoint{1.670974in}{0.483824in}}%
\pgfpathlineto{\pgfqpoint{1.545999in}{0.483823in}}%
\pgfpathlineto{\pgfqpoint{1.421024in}{0.483724in}}%
\pgfpathlineto{\pgfqpoint{1.296050in}{0.483607in}}%
\pgfpathlineto{\pgfqpoint{1.171075in}{0.483602in}}%
\pgfpathlineto{\pgfqpoint{1.046100in}{0.483549in}}%
\pgfpathlineto{\pgfqpoint{0.921125in}{0.483504in}}%
\pgfpathlineto{\pgfqpoint{0.796150in}{0.483237in}}%
\pgfpathlineto{\pgfqpoint{0.608688in}{0.483103in}}%
\pgfpathclose%
\pgfusepath{stroke,fill}%
\end{pgfscope}%
\begin{pgfscope}%
\pgfpathrectangle{\pgfqpoint{0.418102in}{0.331635in}}{\pgfqpoint{2.818182in}{1.132500in}}%
\pgfusepath{clip}%
\pgfsetbuttcap%
\pgfsetroundjoin%
\pgfsetlinewidth{1.505625pt}%
\definecolor{currentstroke}{rgb}{0.000000,0.000000,0.000000}%
\pgfsetstrokecolor{currentstroke}%
\pgfsetdash{}{0pt}%
\pgfpathmoveto{\pgfqpoint{0.546201in}{0.433693in}}%
\pgfpathlineto{\pgfqpoint{0.546201in}{1.412658in}}%
\pgfusepath{stroke}%
\end{pgfscope}%
\begin{pgfscope}%
\pgfpathrectangle{\pgfqpoint{0.418102in}{0.331635in}}{\pgfqpoint{2.818182in}{1.132500in}}%
\pgfusepath{clip}%
\pgfsetbuttcap%
\pgfsetroundjoin%
\pgfsetlinewidth{1.505625pt}%
\definecolor{currentstroke}{rgb}{0.000000,0.000000,0.000000}%
\pgfsetstrokecolor{currentstroke}%
\pgfsetdash{}{0pt}%
\pgfpathmoveto{\pgfqpoint{0.671176in}{0.433693in}}%
\pgfpathlineto{\pgfqpoint{0.671176in}{0.484740in}}%
\pgfusepath{stroke}%
\end{pgfscope}%
\begin{pgfscope}%
\pgfpathrectangle{\pgfqpoint{0.418102in}{0.331635in}}{\pgfqpoint{2.818182in}{1.132500in}}%
\pgfusepath{clip}%
\pgfsetbuttcap%
\pgfsetroundjoin%
\pgfsetlinewidth{1.505625pt}%
\definecolor{currentstroke}{rgb}{0.000000,0.000000,0.000000}%
\pgfsetstrokecolor{currentstroke}%
\pgfsetdash{}{0pt}%
\pgfpathmoveto{\pgfqpoint{0.796150in}{0.433693in}}%
\pgfpathlineto{\pgfqpoint{0.796150in}{0.505861in}}%
\pgfusepath{stroke}%
\end{pgfscope}%
\begin{pgfscope}%
\pgfpathrectangle{\pgfqpoint{0.418102in}{0.331635in}}{\pgfqpoint{2.818182in}{1.132500in}}%
\pgfusepath{clip}%
\pgfsetbuttcap%
\pgfsetroundjoin%
\pgfsetlinewidth{1.505625pt}%
\definecolor{currentstroke}{rgb}{0.000000,0.000000,0.000000}%
\pgfsetstrokecolor{currentstroke}%
\pgfsetdash{}{0pt}%
\pgfpathmoveto{\pgfqpoint{0.921125in}{0.433693in}}%
\pgfpathlineto{\pgfqpoint{0.921125in}{0.463224in}}%
\pgfusepath{stroke}%
\end{pgfscope}%
\begin{pgfscope}%
\pgfpathrectangle{\pgfqpoint{0.418102in}{0.331635in}}{\pgfqpoint{2.818182in}{1.132500in}}%
\pgfusepath{clip}%
\pgfsetbuttcap%
\pgfsetroundjoin%
\pgfsetlinewidth{1.505625pt}%
\definecolor{currentstroke}{rgb}{0.000000,0.000000,0.000000}%
\pgfsetstrokecolor{currentstroke}%
\pgfsetdash{}{0pt}%
\pgfpathmoveto{\pgfqpoint{1.046100in}{0.433693in}}%
\pgfpathlineto{\pgfqpoint{1.046100in}{0.465961in}}%
\pgfusepath{stroke}%
\end{pgfscope}%
\begin{pgfscope}%
\pgfpathrectangle{\pgfqpoint{0.418102in}{0.331635in}}{\pgfqpoint{2.818182in}{1.132500in}}%
\pgfusepath{clip}%
\pgfsetbuttcap%
\pgfsetroundjoin%
\pgfsetlinewidth{1.505625pt}%
\definecolor{currentstroke}{rgb}{0.000000,0.000000,0.000000}%
\pgfsetstrokecolor{currentstroke}%
\pgfsetdash{}{0pt}%
\pgfpathmoveto{\pgfqpoint{1.171075in}{0.433693in}}%
\pgfpathlineto{\pgfqpoint{1.171075in}{0.443550in}}%
\pgfusepath{stroke}%
\end{pgfscope}%
\begin{pgfscope}%
\pgfpathrectangle{\pgfqpoint{0.418102in}{0.331635in}}{\pgfqpoint{2.818182in}{1.132500in}}%
\pgfusepath{clip}%
\pgfsetbuttcap%
\pgfsetroundjoin%
\pgfsetlinewidth{1.505625pt}%
\definecolor{currentstroke}{rgb}{0.000000,0.000000,0.000000}%
\pgfsetstrokecolor{currentstroke}%
\pgfsetdash{}{0pt}%
\pgfpathmoveto{\pgfqpoint{1.296050in}{0.433693in}}%
\pgfpathlineto{\pgfqpoint{1.296050in}{0.481527in}}%
\pgfusepath{stroke}%
\end{pgfscope}%
\begin{pgfscope}%
\pgfpathrectangle{\pgfqpoint{0.418102in}{0.331635in}}{\pgfqpoint{2.818182in}{1.132500in}}%
\pgfusepath{clip}%
\pgfsetbuttcap%
\pgfsetroundjoin%
\pgfsetlinewidth{1.505625pt}%
\definecolor{currentstroke}{rgb}{0.000000,0.000000,0.000000}%
\pgfsetstrokecolor{currentstroke}%
\pgfsetdash{}{0pt}%
\pgfpathmoveto{\pgfqpoint{1.421024in}{0.433693in}}%
\pgfpathlineto{\pgfqpoint{1.421024in}{0.477879in}}%
\pgfusepath{stroke}%
\end{pgfscope}%
\begin{pgfscope}%
\pgfpathrectangle{\pgfqpoint{0.418102in}{0.331635in}}{\pgfqpoint{2.818182in}{1.132500in}}%
\pgfusepath{clip}%
\pgfsetbuttcap%
\pgfsetroundjoin%
\pgfsetlinewidth{1.505625pt}%
\definecolor{currentstroke}{rgb}{0.000000,0.000000,0.000000}%
\pgfsetstrokecolor{currentstroke}%
\pgfsetdash{}{0pt}%
\pgfpathmoveto{\pgfqpoint{1.545999in}{0.433693in}}%
\pgfpathlineto{\pgfqpoint{1.545999in}{0.437739in}}%
\pgfusepath{stroke}%
\end{pgfscope}%
\begin{pgfscope}%
\pgfpathrectangle{\pgfqpoint{0.418102in}{0.331635in}}{\pgfqpoint{2.818182in}{1.132500in}}%
\pgfusepath{clip}%
\pgfsetbuttcap%
\pgfsetroundjoin%
\pgfsetlinewidth{1.505625pt}%
\definecolor{currentstroke}{rgb}{0.000000,0.000000,0.000000}%
\pgfsetstrokecolor{currentstroke}%
\pgfsetdash{}{0pt}%
\pgfpathmoveto{\pgfqpoint{1.670974in}{0.433693in}}%
\pgfpathlineto{\pgfqpoint{1.670974in}{0.466042in}}%
\pgfusepath{stroke}%
\end{pgfscope}%
\begin{pgfscope}%
\pgfpathrectangle{\pgfqpoint{0.418102in}{0.331635in}}{\pgfqpoint{2.818182in}{1.132500in}}%
\pgfusepath{clip}%
\pgfsetbuttcap%
\pgfsetroundjoin%
\pgfsetlinewidth{1.505625pt}%
\definecolor{currentstroke}{rgb}{0.000000,0.000000,0.000000}%
\pgfsetstrokecolor{currentstroke}%
\pgfsetdash{}{0pt}%
\pgfpathmoveto{\pgfqpoint{1.795949in}{0.433693in}}%
\pgfpathlineto{\pgfqpoint{1.795949in}{0.446375in}}%
\pgfusepath{stroke}%
\end{pgfscope}%
\begin{pgfscope}%
\pgfpathrectangle{\pgfqpoint{0.418102in}{0.331635in}}{\pgfqpoint{2.818182in}{1.132500in}}%
\pgfusepath{clip}%
\pgfsetbuttcap%
\pgfsetroundjoin%
\pgfsetlinewidth{1.505625pt}%
\definecolor{currentstroke}{rgb}{0.000000,0.000000,0.000000}%
\pgfsetstrokecolor{currentstroke}%
\pgfsetdash{}{0pt}%
\pgfpathmoveto{\pgfqpoint{1.920924in}{0.433693in}}%
\pgfpathlineto{\pgfqpoint{1.920924in}{0.460964in}}%
\pgfusepath{stroke}%
\end{pgfscope}%
\begin{pgfscope}%
\pgfpathrectangle{\pgfqpoint{0.418102in}{0.331635in}}{\pgfqpoint{2.818182in}{1.132500in}}%
\pgfusepath{clip}%
\pgfsetbuttcap%
\pgfsetroundjoin%
\pgfsetlinewidth{1.505625pt}%
\definecolor{currentstroke}{rgb}{0.000000,0.000000,0.000000}%
\pgfsetstrokecolor{currentstroke}%
\pgfsetdash{}{0pt}%
\pgfpathmoveto{\pgfqpoint{2.045898in}{0.433693in}}%
\pgfpathlineto{\pgfqpoint{2.045898in}{0.504186in}}%
\pgfusepath{stroke}%
\end{pgfscope}%
\begin{pgfscope}%
\pgfpathrectangle{\pgfqpoint{0.418102in}{0.331635in}}{\pgfqpoint{2.818182in}{1.132500in}}%
\pgfusepath{clip}%
\pgfsetbuttcap%
\pgfsetroundjoin%
\pgfsetlinewidth{1.505625pt}%
\definecolor{currentstroke}{rgb}{0.000000,0.000000,0.000000}%
\pgfsetstrokecolor{currentstroke}%
\pgfsetdash{}{0pt}%
\pgfpathmoveto{\pgfqpoint{2.170873in}{0.433693in}}%
\pgfpathlineto{\pgfqpoint{2.170873in}{0.450332in}}%
\pgfusepath{stroke}%
\end{pgfscope}%
\begin{pgfscope}%
\pgfpathrectangle{\pgfqpoint{0.418102in}{0.331635in}}{\pgfqpoint{2.818182in}{1.132500in}}%
\pgfusepath{clip}%
\pgfsetbuttcap%
\pgfsetroundjoin%
\pgfsetlinewidth{1.505625pt}%
\definecolor{currentstroke}{rgb}{0.000000,0.000000,0.000000}%
\pgfsetstrokecolor{currentstroke}%
\pgfsetdash{}{0pt}%
\pgfpathmoveto{\pgfqpoint{2.295848in}{0.433693in}}%
\pgfpathlineto{\pgfqpoint{2.295848in}{0.444208in}}%
\pgfusepath{stroke}%
\end{pgfscope}%
\begin{pgfscope}%
\pgfpathrectangle{\pgfqpoint{0.418102in}{0.331635in}}{\pgfqpoint{2.818182in}{1.132500in}}%
\pgfusepath{clip}%
\pgfsetbuttcap%
\pgfsetroundjoin%
\pgfsetlinewidth{1.505625pt}%
\definecolor{currentstroke}{rgb}{0.000000,0.000000,0.000000}%
\pgfsetstrokecolor{currentstroke}%
\pgfsetdash{}{0pt}%
\pgfpathmoveto{\pgfqpoint{2.420823in}{0.433693in}}%
\pgfpathlineto{\pgfqpoint{2.420823in}{0.469459in}}%
\pgfusepath{stroke}%
\end{pgfscope}%
\begin{pgfscope}%
\pgfpathrectangle{\pgfqpoint{0.418102in}{0.331635in}}{\pgfqpoint{2.818182in}{1.132500in}}%
\pgfusepath{clip}%
\pgfsetbuttcap%
\pgfsetroundjoin%
\pgfsetlinewidth{1.505625pt}%
\definecolor{currentstroke}{rgb}{0.000000,0.000000,0.000000}%
\pgfsetstrokecolor{currentstroke}%
\pgfsetdash{}{0pt}%
\pgfpathmoveto{\pgfqpoint{2.545798in}{0.433693in}}%
\pgfpathlineto{\pgfqpoint{2.545798in}{0.421343in}}%
\pgfusepath{stroke}%
\end{pgfscope}%
\begin{pgfscope}%
\pgfpathrectangle{\pgfqpoint{0.418102in}{0.331635in}}{\pgfqpoint{2.818182in}{1.132500in}}%
\pgfusepath{clip}%
\pgfsetbuttcap%
\pgfsetroundjoin%
\pgfsetlinewidth{1.505625pt}%
\definecolor{currentstroke}{rgb}{0.000000,0.000000,0.000000}%
\pgfsetstrokecolor{currentstroke}%
\pgfsetdash{}{0pt}%
\pgfpathmoveto{\pgfqpoint{2.670772in}{0.433693in}}%
\pgfpathlineto{\pgfqpoint{2.670772in}{0.445469in}}%
\pgfusepath{stroke}%
\end{pgfscope}%
\begin{pgfscope}%
\pgfpathrectangle{\pgfqpoint{0.418102in}{0.331635in}}{\pgfqpoint{2.818182in}{1.132500in}}%
\pgfusepath{clip}%
\pgfsetbuttcap%
\pgfsetroundjoin%
\pgfsetlinewidth{1.505625pt}%
\definecolor{currentstroke}{rgb}{0.000000,0.000000,0.000000}%
\pgfsetstrokecolor{currentstroke}%
\pgfsetdash{}{0pt}%
\pgfpathmoveto{\pgfqpoint{2.795747in}{0.433693in}}%
\pgfpathlineto{\pgfqpoint{2.795747in}{0.430669in}}%
\pgfusepath{stroke}%
\end{pgfscope}%
\begin{pgfscope}%
\pgfpathrectangle{\pgfqpoint{0.418102in}{0.331635in}}{\pgfqpoint{2.818182in}{1.132500in}}%
\pgfusepath{clip}%
\pgfsetbuttcap%
\pgfsetroundjoin%
\pgfsetlinewidth{1.505625pt}%
\definecolor{currentstroke}{rgb}{0.000000,0.000000,0.000000}%
\pgfsetstrokecolor{currentstroke}%
\pgfsetdash{}{0pt}%
\pgfpathmoveto{\pgfqpoint{2.920722in}{0.433693in}}%
\pgfpathlineto{\pgfqpoint{2.920722in}{0.436685in}}%
\pgfusepath{stroke}%
\end{pgfscope}%
\begin{pgfscope}%
\pgfpathrectangle{\pgfqpoint{0.418102in}{0.331635in}}{\pgfqpoint{2.818182in}{1.132500in}}%
\pgfusepath{clip}%
\pgfsetbuttcap%
\pgfsetroundjoin%
\pgfsetlinewidth{1.505625pt}%
\definecolor{currentstroke}{rgb}{0.000000,0.000000,0.000000}%
\pgfsetstrokecolor{currentstroke}%
\pgfsetdash{}{0pt}%
\pgfpathmoveto{\pgfqpoint{3.045697in}{0.433693in}}%
\pgfpathlineto{\pgfqpoint{3.045697in}{0.438304in}}%
\pgfusepath{stroke}%
\end{pgfscope}%
\begin{pgfscope}%
\pgfpathrectangle{\pgfqpoint{0.418102in}{0.331635in}}{\pgfqpoint{2.818182in}{1.132500in}}%
\pgfusepath{clip}%
\pgfsetroundcap%
\pgfsetroundjoin%
\pgfsetlinewidth{1.505625pt}%
\definecolor{currentstroke}{rgb}{0.121569,0.466667,0.705882}%
\pgfsetstrokecolor{currentstroke}%
\pgfsetdash{}{0pt}%
\pgfpathmoveto{\pgfqpoint{0.418102in}{0.433693in}}%
\pgfpathlineto{\pgfqpoint{3.236283in}{0.433693in}}%
\pgfusepath{stroke}%
\end{pgfscope}%
\begin{pgfscope}%
\pgfpathrectangle{\pgfqpoint{0.418102in}{0.331635in}}{\pgfqpoint{2.818182in}{1.132500in}}%
\pgfusepath{clip}%
\pgfsetbuttcap%
\pgfsetroundjoin%
\definecolor{currentfill}{rgb}{0.121569,0.466667,0.705882}%
\pgfsetfillcolor{currentfill}%
\pgfsetlinewidth{1.003750pt}%
\definecolor{currentstroke}{rgb}{0.121569,0.466667,0.705882}%
\pgfsetstrokecolor{currentstroke}%
\pgfsetdash{}{0pt}%
\pgfsys@defobject{currentmarker}{\pgfqpoint{-0.034722in}{-0.034722in}}{\pgfqpoint{0.034722in}{0.034722in}}{%
\pgfpathmoveto{\pgfqpoint{0.000000in}{-0.034722in}}%
\pgfpathcurveto{\pgfqpoint{0.009208in}{-0.034722in}}{\pgfqpoint{0.018041in}{-0.031064in}}{\pgfqpoint{0.024552in}{-0.024552in}}%
\pgfpathcurveto{\pgfqpoint{0.031064in}{-0.018041in}}{\pgfqpoint{0.034722in}{-0.009208in}}{\pgfqpoint{0.034722in}{0.000000in}}%
\pgfpathcurveto{\pgfqpoint{0.034722in}{0.009208in}}{\pgfqpoint{0.031064in}{0.018041in}}{\pgfqpoint{0.024552in}{0.024552in}}%
\pgfpathcurveto{\pgfqpoint{0.018041in}{0.031064in}}{\pgfqpoint{0.009208in}{0.034722in}}{\pgfqpoint{0.000000in}{0.034722in}}%
\pgfpathcurveto{\pgfqpoint{-0.009208in}{0.034722in}}{\pgfqpoint{-0.018041in}{0.031064in}}{\pgfqpoint{-0.024552in}{0.024552in}}%
\pgfpathcurveto{\pgfqpoint{-0.031064in}{0.018041in}}{\pgfqpoint{-0.034722in}{0.009208in}}{\pgfqpoint{-0.034722in}{0.000000in}}%
\pgfpathcurveto{\pgfqpoint{-0.034722in}{-0.009208in}}{\pgfqpoint{-0.031064in}{-0.018041in}}{\pgfqpoint{-0.024552in}{-0.024552in}}%
\pgfpathcurveto{\pgfqpoint{-0.018041in}{-0.031064in}}{\pgfqpoint{-0.009208in}{-0.034722in}}{\pgfqpoint{0.000000in}{-0.034722in}}%
\pgfpathclose%
\pgfusepath{stroke,fill}%
}%
\begin{pgfscope}%
\pgfsys@transformshift{0.546201in}{1.412658in}%
\pgfsys@useobject{currentmarker}{}%
\end{pgfscope}%
\begin{pgfscope}%
\pgfsys@transformshift{0.671176in}{0.484740in}%
\pgfsys@useobject{currentmarker}{}%
\end{pgfscope}%
\begin{pgfscope}%
\pgfsys@transformshift{0.796150in}{0.505861in}%
\pgfsys@useobject{currentmarker}{}%
\end{pgfscope}%
\begin{pgfscope}%
\pgfsys@transformshift{0.921125in}{0.463224in}%
\pgfsys@useobject{currentmarker}{}%
\end{pgfscope}%
\begin{pgfscope}%
\pgfsys@transformshift{1.046100in}{0.465961in}%
\pgfsys@useobject{currentmarker}{}%
\end{pgfscope}%
\begin{pgfscope}%
\pgfsys@transformshift{1.171075in}{0.443550in}%
\pgfsys@useobject{currentmarker}{}%
\end{pgfscope}%
\begin{pgfscope}%
\pgfsys@transformshift{1.296050in}{0.481527in}%
\pgfsys@useobject{currentmarker}{}%
\end{pgfscope}%
\begin{pgfscope}%
\pgfsys@transformshift{1.421024in}{0.477879in}%
\pgfsys@useobject{currentmarker}{}%
\end{pgfscope}%
\begin{pgfscope}%
\pgfsys@transformshift{1.545999in}{0.437739in}%
\pgfsys@useobject{currentmarker}{}%
\end{pgfscope}%
\begin{pgfscope}%
\pgfsys@transformshift{1.670974in}{0.466042in}%
\pgfsys@useobject{currentmarker}{}%
\end{pgfscope}%
\begin{pgfscope}%
\pgfsys@transformshift{1.795949in}{0.446375in}%
\pgfsys@useobject{currentmarker}{}%
\end{pgfscope}%
\begin{pgfscope}%
\pgfsys@transformshift{1.920924in}{0.460964in}%
\pgfsys@useobject{currentmarker}{}%
\end{pgfscope}%
\begin{pgfscope}%
\pgfsys@transformshift{2.045898in}{0.504186in}%
\pgfsys@useobject{currentmarker}{}%
\end{pgfscope}%
\begin{pgfscope}%
\pgfsys@transformshift{2.170873in}{0.450332in}%
\pgfsys@useobject{currentmarker}{}%
\end{pgfscope}%
\begin{pgfscope}%
\pgfsys@transformshift{2.295848in}{0.444208in}%
\pgfsys@useobject{currentmarker}{}%
\end{pgfscope}%
\begin{pgfscope}%
\pgfsys@transformshift{2.420823in}{0.469459in}%
\pgfsys@useobject{currentmarker}{}%
\end{pgfscope}%
\begin{pgfscope}%
\pgfsys@transformshift{2.545798in}{0.421343in}%
\pgfsys@useobject{currentmarker}{}%
\end{pgfscope}%
\begin{pgfscope}%
\pgfsys@transformshift{2.670772in}{0.445469in}%
\pgfsys@useobject{currentmarker}{}%
\end{pgfscope}%
\begin{pgfscope}%
\pgfsys@transformshift{2.795747in}{0.430669in}%
\pgfsys@useobject{currentmarker}{}%
\end{pgfscope}%
\begin{pgfscope}%
\pgfsys@transformshift{2.920722in}{0.436685in}%
\pgfsys@useobject{currentmarker}{}%
\end{pgfscope}%
\begin{pgfscope}%
\pgfsys@transformshift{3.045697in}{0.438304in}%
\pgfsys@useobject{currentmarker}{}%
\end{pgfscope}%
\end{pgfscope}%
\begin{pgfscope}%
\pgfsetrectcap%
\pgfsetmiterjoin%
\pgfsetlinewidth{0.803000pt}%
\definecolor{currentstroke}{rgb}{1.000000,1.000000,1.000000}%
\pgfsetstrokecolor{currentstroke}%
\pgfsetdash{}{0pt}%
\pgfpathmoveto{\pgfqpoint{0.418102in}{0.331635in}}%
\pgfpathlineto{\pgfqpoint{0.418102in}{1.464135in}}%
\pgfusepath{stroke}%
\end{pgfscope}%
\begin{pgfscope}%
\pgfsetrectcap%
\pgfsetmiterjoin%
\pgfsetlinewidth{0.803000pt}%
\definecolor{currentstroke}{rgb}{1.000000,1.000000,1.000000}%
\pgfsetstrokecolor{currentstroke}%
\pgfsetdash{}{0pt}%
\pgfpathmoveto{\pgfqpoint{3.236283in}{0.331635in}}%
\pgfpathlineto{\pgfqpoint{3.236283in}{1.464135in}}%
\pgfusepath{stroke}%
\end{pgfscope}%
\begin{pgfscope}%
\pgfsetrectcap%
\pgfsetmiterjoin%
\pgfsetlinewidth{0.803000pt}%
\definecolor{currentstroke}{rgb}{1.000000,1.000000,1.000000}%
\pgfsetstrokecolor{currentstroke}%
\pgfsetdash{}{0pt}%
\pgfpathmoveto{\pgfqpoint{0.418102in}{0.331635in}}%
\pgfpathlineto{\pgfqpoint{3.236283in}{0.331635in}}%
\pgfusepath{stroke}%
\end{pgfscope}%
\begin{pgfscope}%
\pgfsetrectcap%
\pgfsetmiterjoin%
\pgfsetlinewidth{0.803000pt}%
\definecolor{currentstroke}{rgb}{1.000000,1.000000,1.000000}%
\pgfsetstrokecolor{currentstroke}%
\pgfsetdash{}{0pt}%
\pgfpathmoveto{\pgfqpoint{0.418102in}{1.464135in}}%
\pgfpathlineto{\pgfqpoint{3.236283in}{1.464135in}}%
\pgfusepath{stroke}%
\end{pgfscope}%
\begin{pgfscope}%
\definecolor{textcolor}{rgb}{0.150000,0.150000,0.150000}%
\pgfsetstrokecolor{textcolor}%
\pgfsetfillcolor{textcolor}%
\pgftext[x=1.827193in,y=1.547468in,,base]{\color{textcolor}\rmfamily\fontsize{12.000000}{14.400000}\selectfont Autocorrelation}%
\end{pgfscope}%
\begin{pgfscope}%
\pgfsetbuttcap%
\pgfsetmiterjoin%
\definecolor{currentfill}{rgb}{0.917647,0.917647,0.949020}%
\pgfsetfillcolor{currentfill}%
\pgfsetlinewidth{0.000000pt}%
\definecolor{currentstroke}{rgb}{0.000000,0.000000,0.000000}%
\pgfsetstrokecolor{currentstroke}%
\pgfsetstrokeopacity{0.000000}%
\pgfsetdash{}{0pt}%
\pgfpathmoveto{\pgfqpoint{3.799920in}{0.331635in}}%
\pgfpathlineto{\pgfqpoint{6.618102in}{0.331635in}}%
\pgfpathlineto{\pgfqpoint{6.618102in}{1.464135in}}%
\pgfpathlineto{\pgfqpoint{3.799920in}{1.464135in}}%
\pgfpathclose%
\pgfusepath{fill}%
\end{pgfscope}%
\begin{pgfscope}%
\pgfpathrectangle{\pgfqpoint{3.799920in}{0.331635in}}{\pgfqpoint{2.818182in}{1.132500in}}%
\pgfusepath{clip}%
\pgfsetroundcap%
\pgfsetroundjoin%
\pgfsetlinewidth{0.803000pt}%
\definecolor{currentstroke}{rgb}{1.000000,1.000000,1.000000}%
\pgfsetstrokecolor{currentstroke}%
\pgfsetdash{}{0pt}%
\pgfpathmoveto{\pgfqpoint{3.928019in}{0.331635in}}%
\pgfpathlineto{\pgfqpoint{3.928019in}{1.464135in}}%
\pgfusepath{stroke}%
\end{pgfscope}%
\begin{pgfscope}%
\definecolor{textcolor}{rgb}{0.150000,0.150000,0.150000}%
\pgfsetstrokecolor{textcolor}%
\pgfsetfillcolor{textcolor}%
\pgftext[x=3.928019in,y=0.234413in,,top]{\color{textcolor}\rmfamily\fontsize{10.000000}{12.000000}\selectfont 0}%
\end{pgfscope}%
\begin{pgfscope}%
\pgfpathrectangle{\pgfqpoint{3.799920in}{0.331635in}}{\pgfqpoint{2.818182in}{1.132500in}}%
\pgfusepath{clip}%
\pgfsetroundcap%
\pgfsetroundjoin%
\pgfsetlinewidth{0.803000pt}%
\definecolor{currentstroke}{rgb}{1.000000,1.000000,1.000000}%
\pgfsetstrokecolor{currentstroke}%
\pgfsetdash{}{0pt}%
\pgfpathmoveto{\pgfqpoint{4.552893in}{0.331635in}}%
\pgfpathlineto{\pgfqpoint{4.552893in}{1.464135in}}%
\pgfusepath{stroke}%
\end{pgfscope}%
\begin{pgfscope}%
\definecolor{textcolor}{rgb}{0.150000,0.150000,0.150000}%
\pgfsetstrokecolor{textcolor}%
\pgfsetfillcolor{textcolor}%
\pgftext[x=4.552893in,y=0.234413in,,top]{\color{textcolor}\rmfamily\fontsize{10.000000}{12.000000}\selectfont 5}%
\end{pgfscope}%
\begin{pgfscope}%
\pgfpathrectangle{\pgfqpoint{3.799920in}{0.331635in}}{\pgfqpoint{2.818182in}{1.132500in}}%
\pgfusepath{clip}%
\pgfsetroundcap%
\pgfsetroundjoin%
\pgfsetlinewidth{0.803000pt}%
\definecolor{currentstroke}{rgb}{1.000000,1.000000,1.000000}%
\pgfsetstrokecolor{currentstroke}%
\pgfsetdash{}{0pt}%
\pgfpathmoveto{\pgfqpoint{5.177767in}{0.331635in}}%
\pgfpathlineto{\pgfqpoint{5.177767in}{1.464135in}}%
\pgfusepath{stroke}%
\end{pgfscope}%
\begin{pgfscope}%
\definecolor{textcolor}{rgb}{0.150000,0.150000,0.150000}%
\pgfsetstrokecolor{textcolor}%
\pgfsetfillcolor{textcolor}%
\pgftext[x=5.177767in,y=0.234413in,,top]{\color{textcolor}\rmfamily\fontsize{10.000000}{12.000000}\selectfont 10}%
\end{pgfscope}%
\begin{pgfscope}%
\pgfpathrectangle{\pgfqpoint{3.799920in}{0.331635in}}{\pgfqpoint{2.818182in}{1.132500in}}%
\pgfusepath{clip}%
\pgfsetroundcap%
\pgfsetroundjoin%
\pgfsetlinewidth{0.803000pt}%
\definecolor{currentstroke}{rgb}{1.000000,1.000000,1.000000}%
\pgfsetstrokecolor{currentstroke}%
\pgfsetdash{}{0pt}%
\pgfpathmoveto{\pgfqpoint{5.802641in}{0.331635in}}%
\pgfpathlineto{\pgfqpoint{5.802641in}{1.464135in}}%
\pgfusepath{stroke}%
\end{pgfscope}%
\begin{pgfscope}%
\definecolor{textcolor}{rgb}{0.150000,0.150000,0.150000}%
\pgfsetstrokecolor{textcolor}%
\pgfsetfillcolor{textcolor}%
\pgftext[x=5.802641in,y=0.234413in,,top]{\color{textcolor}\rmfamily\fontsize{10.000000}{12.000000}\selectfont 15}%
\end{pgfscope}%
\begin{pgfscope}%
\pgfpathrectangle{\pgfqpoint{3.799920in}{0.331635in}}{\pgfqpoint{2.818182in}{1.132500in}}%
\pgfusepath{clip}%
\pgfsetroundcap%
\pgfsetroundjoin%
\pgfsetlinewidth{0.803000pt}%
\definecolor{currentstroke}{rgb}{1.000000,1.000000,1.000000}%
\pgfsetstrokecolor{currentstroke}%
\pgfsetdash{}{0pt}%
\pgfpathmoveto{\pgfqpoint{6.427515in}{0.331635in}}%
\pgfpathlineto{\pgfqpoint{6.427515in}{1.464135in}}%
\pgfusepath{stroke}%
\end{pgfscope}%
\begin{pgfscope}%
\definecolor{textcolor}{rgb}{0.150000,0.150000,0.150000}%
\pgfsetstrokecolor{textcolor}%
\pgfsetfillcolor{textcolor}%
\pgftext[x=6.427515in,y=0.234413in,,top]{\color{textcolor}\rmfamily\fontsize{10.000000}{12.000000}\selectfont 20}%
\end{pgfscope}%
\begin{pgfscope}%
\pgfpathrectangle{\pgfqpoint{3.799920in}{0.331635in}}{\pgfqpoint{2.818182in}{1.132500in}}%
\pgfusepath{clip}%
\pgfsetroundcap%
\pgfsetroundjoin%
\pgfsetlinewidth{0.803000pt}%
\definecolor{currentstroke}{rgb}{1.000000,1.000000,1.000000}%
\pgfsetstrokecolor{currentstroke}%
\pgfsetdash{}{0pt}%
\pgfpathmoveto{\pgfqpoint{3.799920in}{0.432578in}}%
\pgfpathlineto{\pgfqpoint{6.618102in}{0.432578in}}%
\pgfusepath{stroke}%
\end{pgfscope}%
\begin{pgfscope}%
\definecolor{textcolor}{rgb}{0.150000,0.150000,0.150000}%
\pgfsetstrokecolor{textcolor}%
\pgfsetfillcolor{textcolor}%
\pgftext[x=3.481818in,y=0.379817in,left,base]{\color{textcolor}\rmfamily\fontsize{10.000000}{12.000000}\selectfont 0.0}%
\end{pgfscope}%
\begin{pgfscope}%
\pgfpathrectangle{\pgfqpoint{3.799920in}{0.331635in}}{\pgfqpoint{2.818182in}{1.132500in}}%
\pgfusepath{clip}%
\pgfsetroundcap%
\pgfsetroundjoin%
\pgfsetlinewidth{0.803000pt}%
\definecolor{currentstroke}{rgb}{1.000000,1.000000,1.000000}%
\pgfsetstrokecolor{currentstroke}%
\pgfsetdash{}{0pt}%
\pgfpathmoveto{\pgfqpoint{3.799920in}{0.922618in}}%
\pgfpathlineto{\pgfqpoint{6.618102in}{0.922618in}}%
\pgfusepath{stroke}%
\end{pgfscope}%
\begin{pgfscope}%
\definecolor{textcolor}{rgb}{0.150000,0.150000,0.150000}%
\pgfsetstrokecolor{textcolor}%
\pgfsetfillcolor{textcolor}%
\pgftext[x=3.481818in,y=0.869857in,left,base]{\color{textcolor}\rmfamily\fontsize{10.000000}{12.000000}\selectfont 0.5}%
\end{pgfscope}%
\begin{pgfscope}%
\pgfpathrectangle{\pgfqpoint{3.799920in}{0.331635in}}{\pgfqpoint{2.818182in}{1.132500in}}%
\pgfusepath{clip}%
\pgfsetroundcap%
\pgfsetroundjoin%
\pgfsetlinewidth{0.803000pt}%
\definecolor{currentstroke}{rgb}{1.000000,1.000000,1.000000}%
\pgfsetstrokecolor{currentstroke}%
\pgfsetdash{}{0pt}%
\pgfpathmoveto{\pgfqpoint{3.799920in}{1.412658in}}%
\pgfpathlineto{\pgfqpoint{6.618102in}{1.412658in}}%
\pgfusepath{stroke}%
\end{pgfscope}%
\begin{pgfscope}%
\definecolor{textcolor}{rgb}{0.150000,0.150000,0.150000}%
\pgfsetstrokecolor{textcolor}%
\pgfsetfillcolor{textcolor}%
\pgftext[x=3.481818in,y=1.359896in,left,base]{\color{textcolor}\rmfamily\fontsize{10.000000}{12.000000}\selectfont 1.0}%
\end{pgfscope}%
\begin{pgfscope}%
\pgfpathrectangle{\pgfqpoint{3.799920in}{0.331635in}}{\pgfqpoint{2.818182in}{1.132500in}}%
\pgfusepath{clip}%
\pgfsetbuttcap%
\pgfsetroundjoin%
\definecolor{currentfill}{rgb}{0.121569,0.466667,0.705882}%
\pgfsetfillcolor{currentfill}%
\pgfsetfillopacity{0.250000}%
\pgfsetlinewidth{1.003750pt}%
\definecolor{currentstroke}{rgb}{1.000000,1.000000,1.000000}%
\pgfsetstrokecolor{currentstroke}%
\pgfsetstrokeopacity{0.250000}%
\pgfsetdash{}{0pt}%
\pgfpathmoveto{\pgfqpoint{3.990506in}{0.482045in}}%
\pgfpathlineto{\pgfqpoint{3.990506in}{0.383112in}}%
\pgfpathlineto{\pgfqpoint{4.177969in}{0.383112in}}%
\pgfpathlineto{\pgfqpoint{4.302943in}{0.383112in}}%
\pgfpathlineto{\pgfqpoint{4.427918in}{0.383112in}}%
\pgfpathlineto{\pgfqpoint{4.552893in}{0.383112in}}%
\pgfpathlineto{\pgfqpoint{4.677868in}{0.383112in}}%
\pgfpathlineto{\pgfqpoint{4.802843in}{0.383112in}}%
\pgfpathlineto{\pgfqpoint{4.927817in}{0.383112in}}%
\pgfpathlineto{\pgfqpoint{5.052792in}{0.383112in}}%
\pgfpathlineto{\pgfqpoint{5.177767in}{0.383112in}}%
\pgfpathlineto{\pgfqpoint{5.302742in}{0.383112in}}%
\pgfpathlineto{\pgfqpoint{5.427717in}{0.383112in}}%
\pgfpathlineto{\pgfqpoint{5.552691in}{0.383112in}}%
\pgfpathlineto{\pgfqpoint{5.677666in}{0.383112in}}%
\pgfpathlineto{\pgfqpoint{5.802641in}{0.383112in}}%
\pgfpathlineto{\pgfqpoint{5.927616in}{0.383112in}}%
\pgfpathlineto{\pgfqpoint{6.052591in}{0.383112in}}%
\pgfpathlineto{\pgfqpoint{6.177565in}{0.383112in}}%
\pgfpathlineto{\pgfqpoint{6.302540in}{0.383112in}}%
\pgfpathlineto{\pgfqpoint{6.490002in}{0.383112in}}%
\pgfpathlineto{\pgfqpoint{6.490002in}{0.482045in}}%
\pgfpathlineto{\pgfqpoint{6.490002in}{0.482045in}}%
\pgfpathlineto{\pgfqpoint{6.302540in}{0.482045in}}%
\pgfpathlineto{\pgfqpoint{6.177565in}{0.482045in}}%
\pgfpathlineto{\pgfqpoint{6.052591in}{0.482045in}}%
\pgfpathlineto{\pgfqpoint{5.927616in}{0.482045in}}%
\pgfpathlineto{\pgfqpoint{5.802641in}{0.482045in}}%
\pgfpathlineto{\pgfqpoint{5.677666in}{0.482045in}}%
\pgfpathlineto{\pgfqpoint{5.552691in}{0.482045in}}%
\pgfpathlineto{\pgfqpoint{5.427717in}{0.482045in}}%
\pgfpathlineto{\pgfqpoint{5.302742in}{0.482045in}}%
\pgfpathlineto{\pgfqpoint{5.177767in}{0.482045in}}%
\pgfpathlineto{\pgfqpoint{5.052792in}{0.482045in}}%
\pgfpathlineto{\pgfqpoint{4.927817in}{0.482045in}}%
\pgfpathlineto{\pgfqpoint{4.802843in}{0.482045in}}%
\pgfpathlineto{\pgfqpoint{4.677868in}{0.482045in}}%
\pgfpathlineto{\pgfqpoint{4.552893in}{0.482045in}}%
\pgfpathlineto{\pgfqpoint{4.427918in}{0.482045in}}%
\pgfpathlineto{\pgfqpoint{4.302943in}{0.482045in}}%
\pgfpathlineto{\pgfqpoint{4.177969in}{0.482045in}}%
\pgfpathlineto{\pgfqpoint{3.990506in}{0.482045in}}%
\pgfpathclose%
\pgfusepath{stroke,fill}%
\end{pgfscope}%
\begin{pgfscope}%
\pgfpathrectangle{\pgfqpoint{3.799920in}{0.331635in}}{\pgfqpoint{2.818182in}{1.132500in}}%
\pgfusepath{clip}%
\pgfsetbuttcap%
\pgfsetroundjoin%
\pgfsetlinewidth{1.505625pt}%
\definecolor{currentstroke}{rgb}{0.000000,0.000000,0.000000}%
\pgfsetstrokecolor{currentstroke}%
\pgfsetdash{}{0pt}%
\pgfpathmoveto{\pgfqpoint{3.928019in}{0.432578in}}%
\pgfpathlineto{\pgfqpoint{3.928019in}{1.412658in}}%
\pgfusepath{stroke}%
\end{pgfscope}%
\begin{pgfscope}%
\pgfpathrectangle{\pgfqpoint{3.799920in}{0.331635in}}{\pgfqpoint{2.818182in}{1.132500in}}%
\pgfusepath{clip}%
\pgfsetbuttcap%
\pgfsetroundjoin%
\pgfsetlinewidth{1.505625pt}%
\definecolor{currentstroke}{rgb}{0.000000,0.000000,0.000000}%
\pgfsetstrokecolor{currentstroke}%
\pgfsetdash{}{0pt}%
\pgfpathmoveto{\pgfqpoint{4.052994in}{0.432578in}}%
\pgfpathlineto{\pgfqpoint{4.052994in}{0.483717in}}%
\pgfusepath{stroke}%
\end{pgfscope}%
\begin{pgfscope}%
\pgfpathrectangle{\pgfqpoint{3.799920in}{0.331635in}}{\pgfqpoint{2.818182in}{1.132500in}}%
\pgfusepath{clip}%
\pgfsetbuttcap%
\pgfsetroundjoin%
\pgfsetlinewidth{1.505625pt}%
\definecolor{currentstroke}{rgb}{0.000000,0.000000,0.000000}%
\pgfsetstrokecolor{currentstroke}%
\pgfsetdash{}{0pt}%
\pgfpathmoveto{\pgfqpoint{4.177969in}{0.432578in}}%
\pgfpathlineto{\pgfqpoint{4.177969in}{0.502446in}}%
\pgfusepath{stroke}%
\end{pgfscope}%
\begin{pgfscope}%
\pgfpathrectangle{\pgfqpoint{3.799920in}{0.331635in}}{\pgfqpoint{2.818182in}{1.132500in}}%
\pgfusepath{clip}%
\pgfsetbuttcap%
\pgfsetroundjoin%
\pgfsetlinewidth{1.505625pt}%
\definecolor{currentstroke}{rgb}{0.000000,0.000000,0.000000}%
\pgfsetstrokecolor{currentstroke}%
\pgfsetdash{}{0pt}%
\pgfpathmoveto{\pgfqpoint{4.302943in}{0.432578in}}%
\pgfpathlineto{\pgfqpoint{4.302943in}{0.455227in}}%
\pgfusepath{stroke}%
\end{pgfscope}%
\begin{pgfscope}%
\pgfpathrectangle{\pgfqpoint{3.799920in}{0.331635in}}{\pgfqpoint{2.818182in}{1.132500in}}%
\pgfusepath{clip}%
\pgfsetbuttcap%
\pgfsetroundjoin%
\pgfsetlinewidth{1.505625pt}%
\definecolor{currentstroke}{rgb}{0.000000,0.000000,0.000000}%
\pgfsetstrokecolor{currentstroke}%
\pgfsetdash{}{0pt}%
\pgfpathmoveto{\pgfqpoint{4.427918in}{0.432578in}}%
\pgfpathlineto{\pgfqpoint{4.427918in}{0.457532in}}%
\pgfusepath{stroke}%
\end{pgfscope}%
\begin{pgfscope}%
\pgfpathrectangle{\pgfqpoint{3.799920in}{0.331635in}}{\pgfqpoint{2.818182in}{1.132500in}}%
\pgfusepath{clip}%
\pgfsetbuttcap%
\pgfsetroundjoin%
\pgfsetlinewidth{1.505625pt}%
\definecolor{currentstroke}{rgb}{0.000000,0.000000,0.000000}%
\pgfsetstrokecolor{currentstroke}%
\pgfsetdash{}{0pt}%
\pgfpathmoveto{\pgfqpoint{4.552893in}{0.432578in}}%
\pgfpathlineto{\pgfqpoint{4.552893in}{0.436101in}}%
\pgfusepath{stroke}%
\end{pgfscope}%
\begin{pgfscope}%
\pgfpathrectangle{\pgfqpoint{3.799920in}{0.331635in}}{\pgfqpoint{2.818182in}{1.132500in}}%
\pgfusepath{clip}%
\pgfsetbuttcap%
\pgfsetroundjoin%
\pgfsetlinewidth{1.505625pt}%
\definecolor{currentstroke}{rgb}{0.000000,0.000000,0.000000}%
\pgfsetstrokecolor{currentstroke}%
\pgfsetdash{}{0pt}%
\pgfpathmoveto{\pgfqpoint{4.677868in}{0.432578in}}%
\pgfpathlineto{\pgfqpoint{4.677868in}{0.475721in}}%
\pgfusepath{stroke}%
\end{pgfscope}%
\begin{pgfscope}%
\pgfpathrectangle{\pgfqpoint{3.799920in}{0.331635in}}{\pgfqpoint{2.818182in}{1.132500in}}%
\pgfusepath{clip}%
\pgfsetbuttcap%
\pgfsetroundjoin%
\pgfsetlinewidth{1.505625pt}%
\definecolor{currentstroke}{rgb}{0.000000,0.000000,0.000000}%
\pgfsetstrokecolor{currentstroke}%
\pgfsetdash{}{0pt}%
\pgfpathmoveto{\pgfqpoint{4.802843in}{0.432578in}}%
\pgfpathlineto{\pgfqpoint{4.802843in}{0.470867in}}%
\pgfusepath{stroke}%
\end{pgfscope}%
\begin{pgfscope}%
\pgfpathrectangle{\pgfqpoint{3.799920in}{0.331635in}}{\pgfqpoint{2.818182in}{1.132500in}}%
\pgfusepath{clip}%
\pgfsetbuttcap%
\pgfsetroundjoin%
\pgfsetlinewidth{1.505625pt}%
\definecolor{currentstroke}{rgb}{0.000000,0.000000,0.000000}%
\pgfsetstrokecolor{currentstroke}%
\pgfsetdash{}{0pt}%
\pgfpathmoveto{\pgfqpoint{4.927817in}{0.432578in}}%
\pgfpathlineto{\pgfqpoint{4.927817in}{0.425455in}}%
\pgfusepath{stroke}%
\end{pgfscope}%
\begin{pgfscope}%
\pgfpathrectangle{\pgfqpoint{3.799920in}{0.331635in}}{\pgfqpoint{2.818182in}{1.132500in}}%
\pgfusepath{clip}%
\pgfsetbuttcap%
\pgfsetroundjoin%
\pgfsetlinewidth{1.505625pt}%
\definecolor{currentstroke}{rgb}{0.000000,0.000000,0.000000}%
\pgfsetstrokecolor{currentstroke}%
\pgfsetdash{}{0pt}%
\pgfpathmoveto{\pgfqpoint{5.052792in}{0.432578in}}%
\pgfpathlineto{\pgfqpoint{5.052792in}{0.457408in}}%
\pgfusepath{stroke}%
\end{pgfscope}%
\begin{pgfscope}%
\pgfpathrectangle{\pgfqpoint{3.799920in}{0.331635in}}{\pgfqpoint{2.818182in}{1.132500in}}%
\pgfusepath{clip}%
\pgfsetbuttcap%
\pgfsetroundjoin%
\pgfsetlinewidth{1.505625pt}%
\definecolor{currentstroke}{rgb}{0.000000,0.000000,0.000000}%
\pgfsetstrokecolor{currentstroke}%
\pgfsetdash{}{0pt}%
\pgfpathmoveto{\pgfqpoint{5.177767in}{0.432578in}}%
\pgfpathlineto{\pgfqpoint{5.177767in}{0.438680in}}%
\pgfusepath{stroke}%
\end{pgfscope}%
\begin{pgfscope}%
\pgfpathrectangle{\pgfqpoint{3.799920in}{0.331635in}}{\pgfqpoint{2.818182in}{1.132500in}}%
\pgfusepath{clip}%
\pgfsetbuttcap%
\pgfsetroundjoin%
\pgfsetlinewidth{1.505625pt}%
\definecolor{currentstroke}{rgb}{0.000000,0.000000,0.000000}%
\pgfsetstrokecolor{currentstroke}%
\pgfsetdash{}{0pt}%
\pgfpathmoveto{\pgfqpoint{5.302742in}{0.432578in}}%
\pgfpathlineto{\pgfqpoint{5.302742in}{0.453156in}}%
\pgfusepath{stroke}%
\end{pgfscope}%
\begin{pgfscope}%
\pgfpathrectangle{\pgfqpoint{3.799920in}{0.331635in}}{\pgfqpoint{2.818182in}{1.132500in}}%
\pgfusepath{clip}%
\pgfsetbuttcap%
\pgfsetroundjoin%
\pgfsetlinewidth{1.505625pt}%
\definecolor{currentstroke}{rgb}{0.000000,0.000000,0.000000}%
\pgfsetstrokecolor{currentstroke}%
\pgfsetdash{}{0pt}%
\pgfpathmoveto{\pgfqpoint{5.427717in}{0.432578in}}%
\pgfpathlineto{\pgfqpoint{5.427717in}{0.497695in}}%
\pgfusepath{stroke}%
\end{pgfscope}%
\begin{pgfscope}%
\pgfpathrectangle{\pgfqpoint{3.799920in}{0.331635in}}{\pgfqpoint{2.818182in}{1.132500in}}%
\pgfusepath{clip}%
\pgfsetbuttcap%
\pgfsetroundjoin%
\pgfsetlinewidth{1.505625pt}%
\definecolor{currentstroke}{rgb}{0.000000,0.000000,0.000000}%
\pgfsetstrokecolor{currentstroke}%
\pgfsetdash{}{0pt}%
\pgfpathmoveto{\pgfqpoint{5.552691in}{0.432578in}}%
\pgfpathlineto{\pgfqpoint{5.552691in}{0.434774in}}%
\pgfusepath{stroke}%
\end{pgfscope}%
\begin{pgfscope}%
\pgfpathrectangle{\pgfqpoint{3.799920in}{0.331635in}}{\pgfqpoint{2.818182in}{1.132500in}}%
\pgfusepath{clip}%
\pgfsetbuttcap%
\pgfsetroundjoin%
\pgfsetlinewidth{1.505625pt}%
\definecolor{currentstroke}{rgb}{0.000000,0.000000,0.000000}%
\pgfsetstrokecolor{currentstroke}%
\pgfsetdash{}{0pt}%
\pgfpathmoveto{\pgfqpoint{5.677666in}{0.432578in}}%
\pgfpathlineto{\pgfqpoint{5.677666in}{0.430192in}}%
\pgfusepath{stroke}%
\end{pgfscope}%
\begin{pgfscope}%
\pgfpathrectangle{\pgfqpoint{3.799920in}{0.331635in}}{\pgfqpoint{2.818182in}{1.132500in}}%
\pgfusepath{clip}%
\pgfsetbuttcap%
\pgfsetroundjoin%
\pgfsetlinewidth{1.505625pt}%
\definecolor{currentstroke}{rgb}{0.000000,0.000000,0.000000}%
\pgfsetstrokecolor{currentstroke}%
\pgfsetdash{}{0pt}%
\pgfpathmoveto{\pgfqpoint{5.802641in}{0.432578in}}%
\pgfpathlineto{\pgfqpoint{5.802641in}{0.461312in}}%
\pgfusepath{stroke}%
\end{pgfscope}%
\begin{pgfscope}%
\pgfpathrectangle{\pgfqpoint{3.799920in}{0.331635in}}{\pgfqpoint{2.818182in}{1.132500in}}%
\pgfusepath{clip}%
\pgfsetbuttcap%
\pgfsetroundjoin%
\pgfsetlinewidth{1.505625pt}%
\definecolor{currentstroke}{rgb}{0.000000,0.000000,0.000000}%
\pgfsetstrokecolor{currentstroke}%
\pgfsetdash{}{0pt}%
\pgfpathmoveto{\pgfqpoint{5.927616in}{0.432578in}}%
\pgfpathlineto{\pgfqpoint{5.927616in}{0.409587in}}%
\pgfusepath{stroke}%
\end{pgfscope}%
\begin{pgfscope}%
\pgfpathrectangle{\pgfqpoint{3.799920in}{0.331635in}}{\pgfqpoint{2.818182in}{1.132500in}}%
\pgfusepath{clip}%
\pgfsetbuttcap%
\pgfsetroundjoin%
\pgfsetlinewidth{1.505625pt}%
\definecolor{currentstroke}{rgb}{0.000000,0.000000,0.000000}%
\pgfsetstrokecolor{currentstroke}%
\pgfsetdash{}{0pt}%
\pgfpathmoveto{\pgfqpoint{6.052591in}{0.432578in}}%
\pgfpathlineto{\pgfqpoint{6.052591in}{0.438790in}}%
\pgfusepath{stroke}%
\end{pgfscope}%
\begin{pgfscope}%
\pgfpathrectangle{\pgfqpoint{3.799920in}{0.331635in}}{\pgfqpoint{2.818182in}{1.132500in}}%
\pgfusepath{clip}%
\pgfsetbuttcap%
\pgfsetroundjoin%
\pgfsetlinewidth{1.505625pt}%
\definecolor{currentstroke}{rgb}{0.000000,0.000000,0.000000}%
\pgfsetstrokecolor{currentstroke}%
\pgfsetdash{}{0pt}%
\pgfpathmoveto{\pgfqpoint{6.177565in}{0.432578in}}%
\pgfpathlineto{\pgfqpoint{6.177565in}{0.421143in}}%
\pgfusepath{stroke}%
\end{pgfscope}%
\begin{pgfscope}%
\pgfpathrectangle{\pgfqpoint{3.799920in}{0.331635in}}{\pgfqpoint{2.818182in}{1.132500in}}%
\pgfusepath{clip}%
\pgfsetbuttcap%
\pgfsetroundjoin%
\pgfsetlinewidth{1.505625pt}%
\definecolor{currentstroke}{rgb}{0.000000,0.000000,0.000000}%
\pgfsetstrokecolor{currentstroke}%
\pgfsetdash{}{0pt}%
\pgfpathmoveto{\pgfqpoint{6.302540in}{0.432578in}}%
\pgfpathlineto{\pgfqpoint{6.302540in}{0.427680in}}%
\pgfusepath{stroke}%
\end{pgfscope}%
\begin{pgfscope}%
\pgfpathrectangle{\pgfqpoint{3.799920in}{0.331635in}}{\pgfqpoint{2.818182in}{1.132500in}}%
\pgfusepath{clip}%
\pgfsetbuttcap%
\pgfsetroundjoin%
\pgfsetlinewidth{1.505625pt}%
\definecolor{currentstroke}{rgb}{0.000000,0.000000,0.000000}%
\pgfsetstrokecolor{currentstroke}%
\pgfsetdash{}{0pt}%
\pgfpathmoveto{\pgfqpoint{6.427515in}{0.432578in}}%
\pgfpathlineto{\pgfqpoint{6.427515in}{0.437361in}}%
\pgfusepath{stroke}%
\end{pgfscope}%
\begin{pgfscope}%
\pgfpathrectangle{\pgfqpoint{3.799920in}{0.331635in}}{\pgfqpoint{2.818182in}{1.132500in}}%
\pgfusepath{clip}%
\pgfsetroundcap%
\pgfsetroundjoin%
\pgfsetlinewidth{1.505625pt}%
\definecolor{currentstroke}{rgb}{0.121569,0.466667,0.705882}%
\pgfsetstrokecolor{currentstroke}%
\pgfsetdash{}{0pt}%
\pgfpathmoveto{\pgfqpoint{3.799920in}{0.432578in}}%
\pgfpathlineto{\pgfqpoint{6.618102in}{0.432578in}}%
\pgfusepath{stroke}%
\end{pgfscope}%
\begin{pgfscope}%
\pgfpathrectangle{\pgfqpoint{3.799920in}{0.331635in}}{\pgfqpoint{2.818182in}{1.132500in}}%
\pgfusepath{clip}%
\pgfsetbuttcap%
\pgfsetroundjoin%
\definecolor{currentfill}{rgb}{0.121569,0.466667,0.705882}%
\pgfsetfillcolor{currentfill}%
\pgfsetlinewidth{1.003750pt}%
\definecolor{currentstroke}{rgb}{0.121569,0.466667,0.705882}%
\pgfsetstrokecolor{currentstroke}%
\pgfsetdash{}{0pt}%
\pgfsys@defobject{currentmarker}{\pgfqpoint{-0.034722in}{-0.034722in}}{\pgfqpoint{0.034722in}{0.034722in}}{%
\pgfpathmoveto{\pgfqpoint{0.000000in}{-0.034722in}}%
\pgfpathcurveto{\pgfqpoint{0.009208in}{-0.034722in}}{\pgfqpoint{0.018041in}{-0.031064in}}{\pgfqpoint{0.024552in}{-0.024552in}}%
\pgfpathcurveto{\pgfqpoint{0.031064in}{-0.018041in}}{\pgfqpoint{0.034722in}{-0.009208in}}{\pgfqpoint{0.034722in}{0.000000in}}%
\pgfpathcurveto{\pgfqpoint{0.034722in}{0.009208in}}{\pgfqpoint{0.031064in}{0.018041in}}{\pgfqpoint{0.024552in}{0.024552in}}%
\pgfpathcurveto{\pgfqpoint{0.018041in}{0.031064in}}{\pgfqpoint{0.009208in}{0.034722in}}{\pgfqpoint{0.000000in}{0.034722in}}%
\pgfpathcurveto{\pgfqpoint{-0.009208in}{0.034722in}}{\pgfqpoint{-0.018041in}{0.031064in}}{\pgfqpoint{-0.024552in}{0.024552in}}%
\pgfpathcurveto{\pgfqpoint{-0.031064in}{0.018041in}}{\pgfqpoint{-0.034722in}{0.009208in}}{\pgfqpoint{-0.034722in}{0.000000in}}%
\pgfpathcurveto{\pgfqpoint{-0.034722in}{-0.009208in}}{\pgfqpoint{-0.031064in}{-0.018041in}}{\pgfqpoint{-0.024552in}{-0.024552in}}%
\pgfpathcurveto{\pgfqpoint{-0.018041in}{-0.031064in}}{\pgfqpoint{-0.009208in}{-0.034722in}}{\pgfqpoint{0.000000in}{-0.034722in}}%
\pgfpathclose%
\pgfusepath{stroke,fill}%
}%
\begin{pgfscope}%
\pgfsys@transformshift{3.928019in}{1.412658in}%
\pgfsys@useobject{currentmarker}{}%
\end{pgfscope}%
\begin{pgfscope}%
\pgfsys@transformshift{4.052994in}{0.483717in}%
\pgfsys@useobject{currentmarker}{}%
\end{pgfscope}%
\begin{pgfscope}%
\pgfsys@transformshift{4.177969in}{0.502446in}%
\pgfsys@useobject{currentmarker}{}%
\end{pgfscope}%
\begin{pgfscope}%
\pgfsys@transformshift{4.302943in}{0.455227in}%
\pgfsys@useobject{currentmarker}{}%
\end{pgfscope}%
\begin{pgfscope}%
\pgfsys@transformshift{4.427918in}{0.457532in}%
\pgfsys@useobject{currentmarker}{}%
\end{pgfscope}%
\begin{pgfscope}%
\pgfsys@transformshift{4.552893in}{0.436101in}%
\pgfsys@useobject{currentmarker}{}%
\end{pgfscope}%
\begin{pgfscope}%
\pgfsys@transformshift{4.677868in}{0.475721in}%
\pgfsys@useobject{currentmarker}{}%
\end{pgfscope}%
\begin{pgfscope}%
\pgfsys@transformshift{4.802843in}{0.470867in}%
\pgfsys@useobject{currentmarker}{}%
\end{pgfscope}%
\begin{pgfscope}%
\pgfsys@transformshift{4.927817in}{0.425455in}%
\pgfsys@useobject{currentmarker}{}%
\end{pgfscope}%
\begin{pgfscope}%
\pgfsys@transformshift{5.052792in}{0.457408in}%
\pgfsys@useobject{currentmarker}{}%
\end{pgfscope}%
\begin{pgfscope}%
\pgfsys@transformshift{5.177767in}{0.438680in}%
\pgfsys@useobject{currentmarker}{}%
\end{pgfscope}%
\begin{pgfscope}%
\pgfsys@transformshift{5.302742in}{0.453156in}%
\pgfsys@useobject{currentmarker}{}%
\end{pgfscope}%
\begin{pgfscope}%
\pgfsys@transformshift{5.427717in}{0.497695in}%
\pgfsys@useobject{currentmarker}{}%
\end{pgfscope}%
\begin{pgfscope}%
\pgfsys@transformshift{5.552691in}{0.434774in}%
\pgfsys@useobject{currentmarker}{}%
\end{pgfscope}%
\begin{pgfscope}%
\pgfsys@transformshift{5.677666in}{0.430192in}%
\pgfsys@useobject{currentmarker}{}%
\end{pgfscope}%
\begin{pgfscope}%
\pgfsys@transformshift{5.802641in}{0.461312in}%
\pgfsys@useobject{currentmarker}{}%
\end{pgfscope}%
\begin{pgfscope}%
\pgfsys@transformshift{5.927616in}{0.409587in}%
\pgfsys@useobject{currentmarker}{}%
\end{pgfscope}%
\begin{pgfscope}%
\pgfsys@transformshift{6.052591in}{0.438790in}%
\pgfsys@useobject{currentmarker}{}%
\end{pgfscope}%
\begin{pgfscope}%
\pgfsys@transformshift{6.177565in}{0.421143in}%
\pgfsys@useobject{currentmarker}{}%
\end{pgfscope}%
\begin{pgfscope}%
\pgfsys@transformshift{6.302540in}{0.427680in}%
\pgfsys@useobject{currentmarker}{}%
\end{pgfscope}%
\begin{pgfscope}%
\pgfsys@transformshift{6.427515in}{0.437361in}%
\pgfsys@useobject{currentmarker}{}%
\end{pgfscope}%
\end{pgfscope}%
\begin{pgfscope}%
\pgfsetrectcap%
\pgfsetmiterjoin%
\pgfsetlinewidth{0.803000pt}%
\definecolor{currentstroke}{rgb}{1.000000,1.000000,1.000000}%
\pgfsetstrokecolor{currentstroke}%
\pgfsetdash{}{0pt}%
\pgfpathmoveto{\pgfqpoint{3.799920in}{0.331635in}}%
\pgfpathlineto{\pgfqpoint{3.799920in}{1.464135in}}%
\pgfusepath{stroke}%
\end{pgfscope}%
\begin{pgfscope}%
\pgfsetrectcap%
\pgfsetmiterjoin%
\pgfsetlinewidth{0.803000pt}%
\definecolor{currentstroke}{rgb}{1.000000,1.000000,1.000000}%
\pgfsetstrokecolor{currentstroke}%
\pgfsetdash{}{0pt}%
\pgfpathmoveto{\pgfqpoint{6.618102in}{0.331635in}}%
\pgfpathlineto{\pgfqpoint{6.618102in}{1.464135in}}%
\pgfusepath{stroke}%
\end{pgfscope}%
\begin{pgfscope}%
\pgfsetrectcap%
\pgfsetmiterjoin%
\pgfsetlinewidth{0.803000pt}%
\definecolor{currentstroke}{rgb}{1.000000,1.000000,1.000000}%
\pgfsetstrokecolor{currentstroke}%
\pgfsetdash{}{0pt}%
\pgfpathmoveto{\pgfqpoint{3.799920in}{0.331635in}}%
\pgfpathlineto{\pgfqpoint{6.618102in}{0.331635in}}%
\pgfusepath{stroke}%
\end{pgfscope}%
\begin{pgfscope}%
\pgfsetrectcap%
\pgfsetmiterjoin%
\pgfsetlinewidth{0.803000pt}%
\definecolor{currentstroke}{rgb}{1.000000,1.000000,1.000000}%
\pgfsetstrokecolor{currentstroke}%
\pgfsetdash{}{0pt}%
\pgfpathmoveto{\pgfqpoint{3.799920in}{1.464135in}}%
\pgfpathlineto{\pgfqpoint{6.618102in}{1.464135in}}%
\pgfusepath{stroke}%
\end{pgfscope}%
\begin{pgfscope}%
\definecolor{textcolor}{rgb}{0.150000,0.150000,0.150000}%
\pgfsetstrokecolor{textcolor}%
\pgfsetfillcolor{textcolor}%
\pgftext[x=5.209011in,y=1.547468in,,base]{\color{textcolor}\rmfamily\fontsize{12.000000}{14.400000}\selectfont Partial Autocorrelation}%
\end{pgfscope}%
\end{pgfpicture}%
\makeatother%
\endgroup%

    \end{adjustbox}
    \begin{adjustbox}{width=.95\textwidth,center}
    %% Creator: Matplotlib, PGF backend
%%
%% To include the figure in your LaTeX document, write
%%   \input{<filename>.pgf}
%%
%% Make sure the required packages are loaded in your preamble
%%   \usepackage{pgf}
%%
%% Figures using additional raster images can only be included by \input if
%% they are in the same directory as the main LaTeX file. For loading figures
%% from other directories you can use the `import` package
%%   \usepackage{import}
%% and then include the figures with
%%   \import{<path to file>}{<filename>.pgf}
%%
%% Matplotlib used the following preamble
%%   \usepackage{fontspec}
%%   \setmainfont{DejaVuSerif.ttf}[Path=/opt/tljh/user/lib/python3.6/site-packages/matplotlib/mpl-data/fonts/ttf/]
%%   \setsansfont{DejaVuSans.ttf}[Path=/opt/tljh/user/lib/python3.6/site-packages/matplotlib/mpl-data/fonts/ttf/]
%%   \setmonofont{DejaVuSansMono.ttf}[Path=/opt/tljh/user/lib/python3.6/site-packages/matplotlib/mpl-data/fonts/ttf/]
%%
\begingroup%
\makeatletter%
\begin{pgfpicture}%
\pgfpathrectangle{\pgfpointorigin}{\pgfqpoint{6.718102in}{1.774096in}}%
\pgfusepath{use as bounding box, clip}%
\begin{pgfscope}%
\pgfsetbuttcap%
\pgfsetmiterjoin%
\definecolor{currentfill}{rgb}{1.000000,1.000000,1.000000}%
\pgfsetfillcolor{currentfill}%
\pgfsetlinewidth{0.000000pt}%
\definecolor{currentstroke}{rgb}{1.000000,1.000000,1.000000}%
\pgfsetstrokecolor{currentstroke}%
\pgfsetdash{}{0pt}%
\pgfpathmoveto{\pgfqpoint{0.000000in}{0.000000in}}%
\pgfpathlineto{\pgfqpoint{6.718102in}{0.000000in}}%
\pgfpathlineto{\pgfqpoint{6.718102in}{1.774096in}}%
\pgfpathlineto{\pgfqpoint{0.000000in}{1.774096in}}%
\pgfpathclose%
\pgfusepath{fill}%
\end{pgfscope}%
\begin{pgfscope}%
\pgfsetbuttcap%
\pgfsetmiterjoin%
\definecolor{currentfill}{rgb}{0.917647,0.917647,0.949020}%
\pgfsetfillcolor{currentfill}%
\pgfsetlinewidth{0.000000pt}%
\definecolor{currentstroke}{rgb}{0.000000,0.000000,0.000000}%
\pgfsetstrokecolor{currentstroke}%
\pgfsetstrokeopacity{0.000000}%
\pgfsetdash{}{0pt}%
\pgfpathmoveto{\pgfqpoint{0.418102in}{0.331635in}}%
\pgfpathlineto{\pgfqpoint{3.236283in}{0.331635in}}%
\pgfpathlineto{\pgfqpoint{3.236283in}{1.464135in}}%
\pgfpathlineto{\pgfqpoint{0.418102in}{1.464135in}}%
\pgfpathclose%
\pgfusepath{fill}%
\end{pgfscope}%
\begin{pgfscope}%
\pgfpathrectangle{\pgfqpoint{0.418102in}{0.331635in}}{\pgfqpoint{2.818182in}{1.132500in}}%
\pgfusepath{clip}%
\pgfsetroundcap%
\pgfsetroundjoin%
\pgfsetlinewidth{0.803000pt}%
\definecolor{currentstroke}{rgb}{1.000000,1.000000,1.000000}%
\pgfsetstrokecolor{currentstroke}%
\pgfsetdash{}{0pt}%
\pgfpathmoveto{\pgfqpoint{0.546201in}{0.331635in}}%
\pgfpathlineto{\pgfqpoint{0.546201in}{1.464135in}}%
\pgfusepath{stroke}%
\end{pgfscope}%
\begin{pgfscope}%
\definecolor{textcolor}{rgb}{0.150000,0.150000,0.150000}%
\pgfsetstrokecolor{textcolor}%
\pgfsetfillcolor{textcolor}%
\pgftext[x=0.546201in,y=0.234413in,,top]{\color{textcolor}\rmfamily\fontsize{10.000000}{12.000000}\selectfont 0}%
\end{pgfscope}%
\begin{pgfscope}%
\pgfpathrectangle{\pgfqpoint{0.418102in}{0.331635in}}{\pgfqpoint{2.818182in}{1.132500in}}%
\pgfusepath{clip}%
\pgfsetroundcap%
\pgfsetroundjoin%
\pgfsetlinewidth{0.803000pt}%
\definecolor{currentstroke}{rgb}{1.000000,1.000000,1.000000}%
\pgfsetstrokecolor{currentstroke}%
\pgfsetdash{}{0pt}%
\pgfpathmoveto{\pgfqpoint{1.171075in}{0.331635in}}%
\pgfpathlineto{\pgfqpoint{1.171075in}{1.464135in}}%
\pgfusepath{stroke}%
\end{pgfscope}%
\begin{pgfscope}%
\definecolor{textcolor}{rgb}{0.150000,0.150000,0.150000}%
\pgfsetstrokecolor{textcolor}%
\pgfsetfillcolor{textcolor}%
\pgftext[x=1.171075in,y=0.234413in,,top]{\color{textcolor}\rmfamily\fontsize{10.000000}{12.000000}\selectfont 5}%
\end{pgfscope}%
\begin{pgfscope}%
\pgfpathrectangle{\pgfqpoint{0.418102in}{0.331635in}}{\pgfqpoint{2.818182in}{1.132500in}}%
\pgfusepath{clip}%
\pgfsetroundcap%
\pgfsetroundjoin%
\pgfsetlinewidth{0.803000pt}%
\definecolor{currentstroke}{rgb}{1.000000,1.000000,1.000000}%
\pgfsetstrokecolor{currentstroke}%
\pgfsetdash{}{0pt}%
\pgfpathmoveto{\pgfqpoint{1.795949in}{0.331635in}}%
\pgfpathlineto{\pgfqpoint{1.795949in}{1.464135in}}%
\pgfusepath{stroke}%
\end{pgfscope}%
\begin{pgfscope}%
\definecolor{textcolor}{rgb}{0.150000,0.150000,0.150000}%
\pgfsetstrokecolor{textcolor}%
\pgfsetfillcolor{textcolor}%
\pgftext[x=1.795949in,y=0.234413in,,top]{\color{textcolor}\rmfamily\fontsize{10.000000}{12.000000}\selectfont 10}%
\end{pgfscope}%
\begin{pgfscope}%
\pgfpathrectangle{\pgfqpoint{0.418102in}{0.331635in}}{\pgfqpoint{2.818182in}{1.132500in}}%
\pgfusepath{clip}%
\pgfsetroundcap%
\pgfsetroundjoin%
\pgfsetlinewidth{0.803000pt}%
\definecolor{currentstroke}{rgb}{1.000000,1.000000,1.000000}%
\pgfsetstrokecolor{currentstroke}%
\pgfsetdash{}{0pt}%
\pgfpathmoveto{\pgfqpoint{2.420823in}{0.331635in}}%
\pgfpathlineto{\pgfqpoint{2.420823in}{1.464135in}}%
\pgfusepath{stroke}%
\end{pgfscope}%
\begin{pgfscope}%
\definecolor{textcolor}{rgb}{0.150000,0.150000,0.150000}%
\pgfsetstrokecolor{textcolor}%
\pgfsetfillcolor{textcolor}%
\pgftext[x=2.420823in,y=0.234413in,,top]{\color{textcolor}\rmfamily\fontsize{10.000000}{12.000000}\selectfont 15}%
\end{pgfscope}%
\begin{pgfscope}%
\pgfpathrectangle{\pgfqpoint{0.418102in}{0.331635in}}{\pgfqpoint{2.818182in}{1.132500in}}%
\pgfusepath{clip}%
\pgfsetroundcap%
\pgfsetroundjoin%
\pgfsetlinewidth{0.803000pt}%
\definecolor{currentstroke}{rgb}{1.000000,1.000000,1.000000}%
\pgfsetstrokecolor{currentstroke}%
\pgfsetdash{}{0pt}%
\pgfpathmoveto{\pgfqpoint{3.045697in}{0.331635in}}%
\pgfpathlineto{\pgfqpoint{3.045697in}{1.464135in}}%
\pgfusepath{stroke}%
\end{pgfscope}%
\begin{pgfscope}%
\definecolor{textcolor}{rgb}{0.150000,0.150000,0.150000}%
\pgfsetstrokecolor{textcolor}%
\pgfsetfillcolor{textcolor}%
\pgftext[x=3.045697in,y=0.234413in,,top]{\color{textcolor}\rmfamily\fontsize{10.000000}{12.000000}\selectfont 20}%
\end{pgfscope}%
\begin{pgfscope}%
\pgfpathrectangle{\pgfqpoint{0.418102in}{0.331635in}}{\pgfqpoint{2.818182in}{1.132500in}}%
\pgfusepath{clip}%
\pgfsetroundcap%
\pgfsetroundjoin%
\pgfsetlinewidth{0.803000pt}%
\definecolor{currentstroke}{rgb}{1.000000,1.000000,1.000000}%
\pgfsetstrokecolor{currentstroke}%
\pgfsetdash{}{0pt}%
\pgfpathmoveto{\pgfqpoint{0.418102in}{0.435254in}}%
\pgfpathlineto{\pgfqpoint{3.236283in}{0.435254in}}%
\pgfusepath{stroke}%
\end{pgfscope}%
\begin{pgfscope}%
\definecolor{textcolor}{rgb}{0.150000,0.150000,0.150000}%
\pgfsetstrokecolor{textcolor}%
\pgfsetfillcolor{textcolor}%
\pgftext[x=0.100000in,y=0.382492in,left,base]{\color{textcolor}\rmfamily\fontsize{10.000000}{12.000000}\selectfont 0.0}%
\end{pgfscope}%
\begin{pgfscope}%
\pgfpathrectangle{\pgfqpoint{0.418102in}{0.331635in}}{\pgfqpoint{2.818182in}{1.132500in}}%
\pgfusepath{clip}%
\pgfsetroundcap%
\pgfsetroundjoin%
\pgfsetlinewidth{0.803000pt}%
\definecolor{currentstroke}{rgb}{1.000000,1.000000,1.000000}%
\pgfsetstrokecolor{currentstroke}%
\pgfsetdash{}{0pt}%
\pgfpathmoveto{\pgfqpoint{0.418102in}{0.923956in}}%
\pgfpathlineto{\pgfqpoint{3.236283in}{0.923956in}}%
\pgfusepath{stroke}%
\end{pgfscope}%
\begin{pgfscope}%
\definecolor{textcolor}{rgb}{0.150000,0.150000,0.150000}%
\pgfsetstrokecolor{textcolor}%
\pgfsetfillcolor{textcolor}%
\pgftext[x=0.100000in,y=0.871194in,left,base]{\color{textcolor}\rmfamily\fontsize{10.000000}{12.000000}\selectfont 0.5}%
\end{pgfscope}%
\begin{pgfscope}%
\pgfpathrectangle{\pgfqpoint{0.418102in}{0.331635in}}{\pgfqpoint{2.818182in}{1.132500in}}%
\pgfusepath{clip}%
\pgfsetroundcap%
\pgfsetroundjoin%
\pgfsetlinewidth{0.803000pt}%
\definecolor{currentstroke}{rgb}{1.000000,1.000000,1.000000}%
\pgfsetstrokecolor{currentstroke}%
\pgfsetdash{}{0pt}%
\pgfpathmoveto{\pgfqpoint{0.418102in}{1.412658in}}%
\pgfpathlineto{\pgfqpoint{3.236283in}{1.412658in}}%
\pgfusepath{stroke}%
\end{pgfscope}%
\begin{pgfscope}%
\definecolor{textcolor}{rgb}{0.150000,0.150000,0.150000}%
\pgfsetstrokecolor{textcolor}%
\pgfsetfillcolor{textcolor}%
\pgftext[x=0.100000in,y=1.359896in,left,base]{\color{textcolor}\rmfamily\fontsize{10.000000}{12.000000}\selectfont 1.0}%
\end{pgfscope}%
\begin{pgfscope}%
\pgfpathrectangle{\pgfqpoint{0.418102in}{0.331635in}}{\pgfqpoint{2.818182in}{1.132500in}}%
\pgfusepath{clip}%
\pgfsetbuttcap%
\pgfsetroundjoin%
\definecolor{currentfill}{rgb}{0.121569,0.466667,0.705882}%
\pgfsetfillcolor{currentfill}%
\pgfsetfillopacity{0.250000}%
\pgfsetlinewidth{1.003750pt}%
\definecolor{currentstroke}{rgb}{1.000000,1.000000,1.000000}%
\pgfsetstrokecolor{currentstroke}%
\pgfsetstrokeopacity{0.250000}%
\pgfsetdash{}{0pt}%
\pgfpathmoveto{\pgfqpoint{0.608688in}{0.484585in}}%
\pgfpathlineto{\pgfqpoint{0.608688in}{0.385922in}}%
\pgfpathlineto{\pgfqpoint{0.796150in}{0.384028in}}%
\pgfpathlineto{\pgfqpoint{0.921125in}{0.383990in}}%
\pgfpathlineto{\pgfqpoint{1.046100in}{0.383949in}}%
\pgfpathlineto{\pgfqpoint{1.171075in}{0.383913in}}%
\pgfpathlineto{\pgfqpoint{1.296050in}{0.383908in}}%
\pgfpathlineto{\pgfqpoint{1.421024in}{0.383902in}}%
\pgfpathlineto{\pgfqpoint{1.545999in}{0.383888in}}%
\pgfpathlineto{\pgfqpoint{1.670974in}{0.383886in}}%
\pgfpathlineto{\pgfqpoint{1.795949in}{0.383879in}}%
\pgfpathlineto{\pgfqpoint{1.920924in}{0.383840in}}%
\pgfpathlineto{\pgfqpoint{2.045898in}{0.383808in}}%
\pgfpathlineto{\pgfqpoint{2.170873in}{0.383788in}}%
\pgfpathlineto{\pgfqpoint{2.295848in}{0.383786in}}%
\pgfpathlineto{\pgfqpoint{2.420823in}{0.383763in}}%
\pgfpathlineto{\pgfqpoint{2.545798in}{0.383274in}}%
\pgfpathlineto{\pgfqpoint{2.670772in}{0.383122in}}%
\pgfpathlineto{\pgfqpoint{2.795747in}{0.383118in}}%
\pgfpathlineto{\pgfqpoint{2.920722in}{0.383118in}}%
\pgfpathlineto{\pgfqpoint{3.108184in}{0.383112in}}%
\pgfpathlineto{\pgfqpoint{3.108184in}{0.487395in}}%
\pgfpathlineto{\pgfqpoint{3.108184in}{0.487395in}}%
\pgfpathlineto{\pgfqpoint{2.920722in}{0.487389in}}%
\pgfpathlineto{\pgfqpoint{2.795747in}{0.487389in}}%
\pgfpathlineto{\pgfqpoint{2.670772in}{0.487385in}}%
\pgfpathlineto{\pgfqpoint{2.545798in}{0.487233in}}%
\pgfpathlineto{\pgfqpoint{2.420823in}{0.486744in}}%
\pgfpathlineto{\pgfqpoint{2.295848in}{0.486721in}}%
\pgfpathlineto{\pgfqpoint{2.170873in}{0.486719in}}%
\pgfpathlineto{\pgfqpoint{2.045898in}{0.486699in}}%
\pgfpathlineto{\pgfqpoint{1.920924in}{0.486667in}}%
\pgfpathlineto{\pgfqpoint{1.795949in}{0.486628in}}%
\pgfpathlineto{\pgfqpoint{1.670974in}{0.486621in}}%
\pgfpathlineto{\pgfqpoint{1.545999in}{0.486619in}}%
\pgfpathlineto{\pgfqpoint{1.421024in}{0.486605in}}%
\pgfpathlineto{\pgfqpoint{1.296050in}{0.486599in}}%
\pgfpathlineto{\pgfqpoint{1.171075in}{0.486594in}}%
\pgfpathlineto{\pgfqpoint{1.046100in}{0.486558in}}%
\pgfpathlineto{\pgfqpoint{0.921125in}{0.486518in}}%
\pgfpathlineto{\pgfqpoint{0.796150in}{0.486479in}}%
\pgfpathlineto{\pgfqpoint{0.608688in}{0.484585in}}%
\pgfpathclose%
\pgfusepath{stroke,fill}%
\end{pgfscope}%
\begin{pgfscope}%
\pgfpathrectangle{\pgfqpoint{0.418102in}{0.331635in}}{\pgfqpoint{2.818182in}{1.132500in}}%
\pgfusepath{clip}%
\pgfsetbuttcap%
\pgfsetroundjoin%
\pgfsetlinewidth{1.505625pt}%
\definecolor{currentstroke}{rgb}{0.000000,0.000000,0.000000}%
\pgfsetstrokecolor{currentstroke}%
\pgfsetdash{}{0pt}%
\pgfpathmoveto{\pgfqpoint{0.546201in}{0.435254in}}%
\pgfpathlineto{\pgfqpoint{0.546201in}{1.412658in}}%
\pgfusepath{stroke}%
\end{pgfscope}%
\begin{pgfscope}%
\pgfpathrectangle{\pgfqpoint{0.418102in}{0.331635in}}{\pgfqpoint{2.818182in}{1.132500in}}%
\pgfusepath{clip}%
\pgfsetbuttcap%
\pgfsetroundjoin%
\pgfsetlinewidth{1.505625pt}%
\definecolor{currentstroke}{rgb}{0.000000,0.000000,0.000000}%
\pgfsetstrokecolor{currentstroke}%
\pgfsetdash{}{0pt}%
\pgfpathmoveto{\pgfqpoint{0.671176in}{0.435254in}}%
\pgfpathlineto{\pgfqpoint{0.671176in}{0.628614in}}%
\pgfusepath{stroke}%
\end{pgfscope}%
\begin{pgfscope}%
\pgfpathrectangle{\pgfqpoint{0.418102in}{0.331635in}}{\pgfqpoint{2.818182in}{1.132500in}}%
\pgfusepath{clip}%
\pgfsetbuttcap%
\pgfsetroundjoin%
\pgfsetlinewidth{1.505625pt}%
\definecolor{currentstroke}{rgb}{0.000000,0.000000,0.000000}%
\pgfsetstrokecolor{currentstroke}%
\pgfsetdash{}{0pt}%
\pgfpathmoveto{\pgfqpoint{0.796150in}{0.435254in}}%
\pgfpathlineto{\pgfqpoint{0.796150in}{0.463103in}}%
\pgfusepath{stroke}%
\end{pgfscope}%
\begin{pgfscope}%
\pgfpathrectangle{\pgfqpoint{0.418102in}{0.331635in}}{\pgfqpoint{2.818182in}{1.132500in}}%
\pgfusepath{clip}%
\pgfsetbuttcap%
\pgfsetroundjoin%
\pgfsetlinewidth{1.505625pt}%
\definecolor{currentstroke}{rgb}{0.000000,0.000000,0.000000}%
\pgfsetstrokecolor{currentstroke}%
\pgfsetdash{}{0pt}%
\pgfpathmoveto{\pgfqpoint{0.921125in}{0.435254in}}%
\pgfpathlineto{\pgfqpoint{0.921125in}{0.463814in}}%
\pgfusepath{stroke}%
\end{pgfscope}%
\begin{pgfscope}%
\pgfpathrectangle{\pgfqpoint{0.418102in}{0.331635in}}{\pgfqpoint{2.818182in}{1.132500in}}%
\pgfusepath{clip}%
\pgfsetbuttcap%
\pgfsetroundjoin%
\pgfsetlinewidth{1.505625pt}%
\definecolor{currentstroke}{rgb}{0.000000,0.000000,0.000000}%
\pgfsetstrokecolor{currentstroke}%
\pgfsetdash{}{0pt}%
\pgfpathmoveto{\pgfqpoint{1.046100in}{0.435254in}}%
\pgfpathlineto{\pgfqpoint{1.046100in}{0.462023in}}%
\pgfusepath{stroke}%
\end{pgfscope}%
\begin{pgfscope}%
\pgfpathrectangle{\pgfqpoint{0.418102in}{0.331635in}}{\pgfqpoint{2.818182in}{1.132500in}}%
\pgfusepath{clip}%
\pgfsetbuttcap%
\pgfsetroundjoin%
\pgfsetlinewidth{1.505625pt}%
\definecolor{currentstroke}{rgb}{0.000000,0.000000,0.000000}%
\pgfsetstrokecolor{currentstroke}%
\pgfsetdash{}{0pt}%
\pgfpathmoveto{\pgfqpoint{1.171075in}{0.435254in}}%
\pgfpathlineto{\pgfqpoint{1.171075in}{0.445255in}}%
\pgfusepath{stroke}%
\end{pgfscope}%
\begin{pgfscope}%
\pgfpathrectangle{\pgfqpoint{0.418102in}{0.331635in}}{\pgfqpoint{2.818182in}{1.132500in}}%
\pgfusepath{clip}%
\pgfsetbuttcap%
\pgfsetroundjoin%
\pgfsetlinewidth{1.505625pt}%
\definecolor{currentstroke}{rgb}{0.000000,0.000000,0.000000}%
\pgfsetstrokecolor{currentstroke}%
\pgfsetdash{}{0pt}%
\pgfpathmoveto{\pgfqpoint{1.296050in}{0.435254in}}%
\pgfpathlineto{\pgfqpoint{1.296050in}{0.446342in}}%
\pgfusepath{stroke}%
\end{pgfscope}%
\begin{pgfscope}%
\pgfpathrectangle{\pgfqpoint{0.418102in}{0.331635in}}{\pgfqpoint{2.818182in}{1.132500in}}%
\pgfusepath{clip}%
\pgfsetbuttcap%
\pgfsetroundjoin%
\pgfsetlinewidth{1.505625pt}%
\definecolor{currentstroke}{rgb}{0.000000,0.000000,0.000000}%
\pgfsetstrokecolor{currentstroke}%
\pgfsetdash{}{0pt}%
\pgfpathmoveto{\pgfqpoint{1.421024in}{0.435254in}}%
\pgfpathlineto{\pgfqpoint{1.421024in}{0.452082in}}%
\pgfusepath{stroke}%
\end{pgfscope}%
\begin{pgfscope}%
\pgfpathrectangle{\pgfqpoint{0.418102in}{0.331635in}}{\pgfqpoint{2.818182in}{1.132500in}}%
\pgfusepath{clip}%
\pgfsetbuttcap%
\pgfsetroundjoin%
\pgfsetlinewidth{1.505625pt}%
\definecolor{currentstroke}{rgb}{0.000000,0.000000,0.000000}%
\pgfsetstrokecolor{currentstroke}%
\pgfsetdash{}{0pt}%
\pgfpathmoveto{\pgfqpoint{1.545999in}{0.435254in}}%
\pgfpathlineto{\pgfqpoint{1.545999in}{0.441562in}}%
\pgfusepath{stroke}%
\end{pgfscope}%
\begin{pgfscope}%
\pgfpathrectangle{\pgfqpoint{0.418102in}{0.331635in}}{\pgfqpoint{2.818182in}{1.132500in}}%
\pgfusepath{clip}%
\pgfsetbuttcap%
\pgfsetroundjoin%
\pgfsetlinewidth{1.505625pt}%
\definecolor{currentstroke}{rgb}{0.000000,0.000000,0.000000}%
\pgfsetstrokecolor{currentstroke}%
\pgfsetdash{}{0pt}%
\pgfpathmoveto{\pgfqpoint{1.670974in}{0.435254in}}%
\pgfpathlineto{\pgfqpoint{1.670974in}{0.447244in}}%
\pgfusepath{stroke}%
\end{pgfscope}%
\begin{pgfscope}%
\pgfpathrectangle{\pgfqpoint{0.418102in}{0.331635in}}{\pgfqpoint{2.818182in}{1.132500in}}%
\pgfusepath{clip}%
\pgfsetbuttcap%
\pgfsetroundjoin%
\pgfsetlinewidth{1.505625pt}%
\definecolor{currentstroke}{rgb}{0.000000,0.000000,0.000000}%
\pgfsetstrokecolor{currentstroke}%
\pgfsetdash{}{0pt}%
\pgfpathmoveto{\pgfqpoint{1.795949in}{0.435254in}}%
\pgfpathlineto{\pgfqpoint{1.795949in}{0.463439in}}%
\pgfusepath{stroke}%
\end{pgfscope}%
\begin{pgfscope}%
\pgfpathrectangle{\pgfqpoint{0.418102in}{0.331635in}}{\pgfqpoint{2.818182in}{1.132500in}}%
\pgfusepath{clip}%
\pgfsetbuttcap%
\pgfsetroundjoin%
\pgfsetlinewidth{1.505625pt}%
\definecolor{currentstroke}{rgb}{0.000000,0.000000,0.000000}%
\pgfsetstrokecolor{currentstroke}%
\pgfsetdash{}{0pt}%
\pgfpathmoveto{\pgfqpoint{1.920924in}{0.435254in}}%
\pgfpathlineto{\pgfqpoint{1.920924in}{0.460454in}}%
\pgfusepath{stroke}%
\end{pgfscope}%
\begin{pgfscope}%
\pgfpathrectangle{\pgfqpoint{0.418102in}{0.331635in}}{\pgfqpoint{2.818182in}{1.132500in}}%
\pgfusepath{clip}%
\pgfsetbuttcap%
\pgfsetroundjoin%
\pgfsetlinewidth{1.505625pt}%
\definecolor{currentstroke}{rgb}{0.000000,0.000000,0.000000}%
\pgfsetstrokecolor{currentstroke}%
\pgfsetdash{}{0pt}%
\pgfpathmoveto{\pgfqpoint{2.045898in}{0.435254in}}%
\pgfpathlineto{\pgfqpoint{2.045898in}{0.455353in}}%
\pgfusepath{stroke}%
\end{pgfscope}%
\begin{pgfscope}%
\pgfpathrectangle{\pgfqpoint{0.418102in}{0.331635in}}{\pgfqpoint{2.818182in}{1.132500in}}%
\pgfusepath{clip}%
\pgfsetbuttcap%
\pgfsetroundjoin%
\pgfsetlinewidth{1.505625pt}%
\definecolor{currentstroke}{rgb}{0.000000,0.000000,0.000000}%
\pgfsetstrokecolor{currentstroke}%
\pgfsetdash{}{0pt}%
\pgfpathmoveto{\pgfqpoint{2.170873in}{0.435254in}}%
\pgfpathlineto{\pgfqpoint{2.170873in}{0.442378in}}%
\pgfusepath{stroke}%
\end{pgfscope}%
\begin{pgfscope}%
\pgfpathrectangle{\pgfqpoint{0.418102in}{0.331635in}}{\pgfqpoint{2.818182in}{1.132500in}}%
\pgfusepath{clip}%
\pgfsetbuttcap%
\pgfsetroundjoin%
\pgfsetlinewidth{1.505625pt}%
\definecolor{currentstroke}{rgb}{0.000000,0.000000,0.000000}%
\pgfsetstrokecolor{currentstroke}%
\pgfsetdash{}{0pt}%
\pgfpathmoveto{\pgfqpoint{2.295848in}{0.435254in}}%
\pgfpathlineto{\pgfqpoint{2.295848in}{0.456597in}}%
\pgfusepath{stroke}%
\end{pgfscope}%
\begin{pgfscope}%
\pgfpathrectangle{\pgfqpoint{0.418102in}{0.331635in}}{\pgfqpoint{2.818182in}{1.132500in}}%
\pgfusepath{clip}%
\pgfsetbuttcap%
\pgfsetroundjoin%
\pgfsetlinewidth{1.505625pt}%
\definecolor{currentstroke}{rgb}{0.000000,0.000000,0.000000}%
\pgfsetstrokecolor{currentstroke}%
\pgfsetdash{}{0pt}%
\pgfpathmoveto{\pgfqpoint{2.420823in}{0.435254in}}%
\pgfpathlineto{\pgfqpoint{2.420823in}{0.534967in}}%
\pgfusepath{stroke}%
\end{pgfscope}%
\begin{pgfscope}%
\pgfpathrectangle{\pgfqpoint{0.418102in}{0.331635in}}{\pgfqpoint{2.818182in}{1.132500in}}%
\pgfusepath{clip}%
\pgfsetbuttcap%
\pgfsetroundjoin%
\pgfsetlinewidth{1.505625pt}%
\definecolor{currentstroke}{rgb}{0.000000,0.000000,0.000000}%
\pgfsetstrokecolor{currentstroke}%
\pgfsetdash{}{0pt}%
\pgfpathmoveto{\pgfqpoint{2.545798in}{0.435254in}}%
\pgfpathlineto{\pgfqpoint{2.545798in}{0.490984in}}%
\pgfusepath{stroke}%
\end{pgfscope}%
\begin{pgfscope}%
\pgfpathrectangle{\pgfqpoint{0.418102in}{0.331635in}}{\pgfqpoint{2.818182in}{1.132500in}}%
\pgfusepath{clip}%
\pgfsetbuttcap%
\pgfsetroundjoin%
\pgfsetlinewidth{1.505625pt}%
\definecolor{currentstroke}{rgb}{0.000000,0.000000,0.000000}%
\pgfsetstrokecolor{currentstroke}%
\pgfsetdash{}{0pt}%
\pgfpathmoveto{\pgfqpoint{2.670772in}{0.435254in}}%
\pgfpathlineto{\pgfqpoint{2.670772in}{0.443648in}}%
\pgfusepath{stroke}%
\end{pgfscope}%
\begin{pgfscope}%
\pgfpathrectangle{\pgfqpoint{0.418102in}{0.331635in}}{\pgfqpoint{2.818182in}{1.132500in}}%
\pgfusepath{clip}%
\pgfsetbuttcap%
\pgfsetroundjoin%
\pgfsetlinewidth{1.505625pt}%
\definecolor{currentstroke}{rgb}{0.000000,0.000000,0.000000}%
\pgfsetstrokecolor{currentstroke}%
\pgfsetdash{}{0pt}%
\pgfpathmoveto{\pgfqpoint{2.795747in}{0.435254in}}%
\pgfpathlineto{\pgfqpoint{2.795747in}{0.437568in}}%
\pgfusepath{stroke}%
\end{pgfscope}%
\begin{pgfscope}%
\pgfpathrectangle{\pgfqpoint{0.418102in}{0.331635in}}{\pgfqpoint{2.818182in}{1.132500in}}%
\pgfusepath{clip}%
\pgfsetbuttcap%
\pgfsetroundjoin%
\pgfsetlinewidth{1.505625pt}%
\definecolor{currentstroke}{rgb}{0.000000,0.000000,0.000000}%
\pgfsetstrokecolor{currentstroke}%
\pgfsetdash{}{0pt}%
\pgfpathmoveto{\pgfqpoint{2.920722in}{0.435254in}}%
\pgfpathlineto{\pgfqpoint{2.920722in}{0.446164in}}%
\pgfusepath{stroke}%
\end{pgfscope}%
\begin{pgfscope}%
\pgfpathrectangle{\pgfqpoint{0.418102in}{0.331635in}}{\pgfqpoint{2.818182in}{1.132500in}}%
\pgfusepath{clip}%
\pgfsetbuttcap%
\pgfsetroundjoin%
\pgfsetlinewidth{1.505625pt}%
\definecolor{currentstroke}{rgb}{0.000000,0.000000,0.000000}%
\pgfsetstrokecolor{currentstroke}%
\pgfsetdash{}{0pt}%
\pgfpathmoveto{\pgfqpoint{3.045697in}{0.435254in}}%
\pgfpathlineto{\pgfqpoint{3.045697in}{0.456685in}}%
\pgfusepath{stroke}%
\end{pgfscope}%
\begin{pgfscope}%
\pgfpathrectangle{\pgfqpoint{0.418102in}{0.331635in}}{\pgfqpoint{2.818182in}{1.132500in}}%
\pgfusepath{clip}%
\pgfsetroundcap%
\pgfsetroundjoin%
\pgfsetlinewidth{1.505625pt}%
\definecolor{currentstroke}{rgb}{0.172549,0.627451,0.172549}%
\pgfsetstrokecolor{currentstroke}%
\pgfsetdash{}{0pt}%
\pgfpathmoveto{\pgfqpoint{0.418102in}{0.435254in}}%
\pgfpathlineto{\pgfqpoint{3.236283in}{0.435254in}}%
\pgfusepath{stroke}%
\end{pgfscope}%
\begin{pgfscope}%
\pgfpathrectangle{\pgfqpoint{0.418102in}{0.331635in}}{\pgfqpoint{2.818182in}{1.132500in}}%
\pgfusepath{clip}%
\pgfsetbuttcap%
\pgfsetroundjoin%
\definecolor{currentfill}{rgb}{0.172549,0.627451,0.172549}%
\pgfsetfillcolor{currentfill}%
\pgfsetlinewidth{1.003750pt}%
\definecolor{currentstroke}{rgb}{0.172549,0.627451,0.172549}%
\pgfsetstrokecolor{currentstroke}%
\pgfsetdash{}{0pt}%
\pgfsys@defobject{currentmarker}{\pgfqpoint{-0.034722in}{-0.034722in}}{\pgfqpoint{0.034722in}{0.034722in}}{%
\pgfpathmoveto{\pgfqpoint{0.000000in}{-0.034722in}}%
\pgfpathcurveto{\pgfqpoint{0.009208in}{-0.034722in}}{\pgfqpoint{0.018041in}{-0.031064in}}{\pgfqpoint{0.024552in}{-0.024552in}}%
\pgfpathcurveto{\pgfqpoint{0.031064in}{-0.018041in}}{\pgfqpoint{0.034722in}{-0.009208in}}{\pgfqpoint{0.034722in}{0.000000in}}%
\pgfpathcurveto{\pgfqpoint{0.034722in}{0.009208in}}{\pgfqpoint{0.031064in}{0.018041in}}{\pgfqpoint{0.024552in}{0.024552in}}%
\pgfpathcurveto{\pgfqpoint{0.018041in}{0.031064in}}{\pgfqpoint{0.009208in}{0.034722in}}{\pgfqpoint{0.000000in}{0.034722in}}%
\pgfpathcurveto{\pgfqpoint{-0.009208in}{0.034722in}}{\pgfqpoint{-0.018041in}{0.031064in}}{\pgfqpoint{-0.024552in}{0.024552in}}%
\pgfpathcurveto{\pgfqpoint{-0.031064in}{0.018041in}}{\pgfqpoint{-0.034722in}{0.009208in}}{\pgfqpoint{-0.034722in}{0.000000in}}%
\pgfpathcurveto{\pgfqpoint{-0.034722in}{-0.009208in}}{\pgfqpoint{-0.031064in}{-0.018041in}}{\pgfqpoint{-0.024552in}{-0.024552in}}%
\pgfpathcurveto{\pgfqpoint{-0.018041in}{-0.031064in}}{\pgfqpoint{-0.009208in}{-0.034722in}}{\pgfqpoint{0.000000in}{-0.034722in}}%
\pgfpathclose%
\pgfusepath{stroke,fill}%
}%
\begin{pgfscope}%
\pgfsys@transformshift{0.546201in}{1.412658in}%
\pgfsys@useobject{currentmarker}{}%
\end{pgfscope}%
\begin{pgfscope}%
\pgfsys@transformshift{0.671176in}{0.628614in}%
\pgfsys@useobject{currentmarker}{}%
\end{pgfscope}%
\begin{pgfscope}%
\pgfsys@transformshift{0.796150in}{0.463103in}%
\pgfsys@useobject{currentmarker}{}%
\end{pgfscope}%
\begin{pgfscope}%
\pgfsys@transformshift{0.921125in}{0.463814in}%
\pgfsys@useobject{currentmarker}{}%
\end{pgfscope}%
\begin{pgfscope}%
\pgfsys@transformshift{1.046100in}{0.462023in}%
\pgfsys@useobject{currentmarker}{}%
\end{pgfscope}%
\begin{pgfscope}%
\pgfsys@transformshift{1.171075in}{0.445255in}%
\pgfsys@useobject{currentmarker}{}%
\end{pgfscope}%
\begin{pgfscope}%
\pgfsys@transformshift{1.296050in}{0.446342in}%
\pgfsys@useobject{currentmarker}{}%
\end{pgfscope}%
\begin{pgfscope}%
\pgfsys@transformshift{1.421024in}{0.452082in}%
\pgfsys@useobject{currentmarker}{}%
\end{pgfscope}%
\begin{pgfscope}%
\pgfsys@transformshift{1.545999in}{0.441562in}%
\pgfsys@useobject{currentmarker}{}%
\end{pgfscope}%
\begin{pgfscope}%
\pgfsys@transformshift{1.670974in}{0.447244in}%
\pgfsys@useobject{currentmarker}{}%
\end{pgfscope}%
\begin{pgfscope}%
\pgfsys@transformshift{1.795949in}{0.463439in}%
\pgfsys@useobject{currentmarker}{}%
\end{pgfscope}%
\begin{pgfscope}%
\pgfsys@transformshift{1.920924in}{0.460454in}%
\pgfsys@useobject{currentmarker}{}%
\end{pgfscope}%
\begin{pgfscope}%
\pgfsys@transformshift{2.045898in}{0.455353in}%
\pgfsys@useobject{currentmarker}{}%
\end{pgfscope}%
\begin{pgfscope}%
\pgfsys@transformshift{2.170873in}{0.442378in}%
\pgfsys@useobject{currentmarker}{}%
\end{pgfscope}%
\begin{pgfscope}%
\pgfsys@transformshift{2.295848in}{0.456597in}%
\pgfsys@useobject{currentmarker}{}%
\end{pgfscope}%
\begin{pgfscope}%
\pgfsys@transformshift{2.420823in}{0.534967in}%
\pgfsys@useobject{currentmarker}{}%
\end{pgfscope}%
\begin{pgfscope}%
\pgfsys@transformshift{2.545798in}{0.490984in}%
\pgfsys@useobject{currentmarker}{}%
\end{pgfscope}%
\begin{pgfscope}%
\pgfsys@transformshift{2.670772in}{0.443648in}%
\pgfsys@useobject{currentmarker}{}%
\end{pgfscope}%
\begin{pgfscope}%
\pgfsys@transformshift{2.795747in}{0.437568in}%
\pgfsys@useobject{currentmarker}{}%
\end{pgfscope}%
\begin{pgfscope}%
\pgfsys@transformshift{2.920722in}{0.446164in}%
\pgfsys@useobject{currentmarker}{}%
\end{pgfscope}%
\begin{pgfscope}%
\pgfsys@transformshift{3.045697in}{0.456685in}%
\pgfsys@useobject{currentmarker}{}%
\end{pgfscope}%
\end{pgfscope}%
\begin{pgfscope}%
\pgfsetrectcap%
\pgfsetmiterjoin%
\pgfsetlinewidth{0.803000pt}%
\definecolor{currentstroke}{rgb}{1.000000,1.000000,1.000000}%
\pgfsetstrokecolor{currentstroke}%
\pgfsetdash{}{0pt}%
\pgfpathmoveto{\pgfqpoint{0.418102in}{0.331635in}}%
\pgfpathlineto{\pgfqpoint{0.418102in}{1.464135in}}%
\pgfusepath{stroke}%
\end{pgfscope}%
\begin{pgfscope}%
\pgfsetrectcap%
\pgfsetmiterjoin%
\pgfsetlinewidth{0.803000pt}%
\definecolor{currentstroke}{rgb}{1.000000,1.000000,1.000000}%
\pgfsetstrokecolor{currentstroke}%
\pgfsetdash{}{0pt}%
\pgfpathmoveto{\pgfqpoint{3.236283in}{0.331635in}}%
\pgfpathlineto{\pgfqpoint{3.236283in}{1.464135in}}%
\pgfusepath{stroke}%
\end{pgfscope}%
\begin{pgfscope}%
\pgfsetrectcap%
\pgfsetmiterjoin%
\pgfsetlinewidth{0.803000pt}%
\definecolor{currentstroke}{rgb}{1.000000,1.000000,1.000000}%
\pgfsetstrokecolor{currentstroke}%
\pgfsetdash{}{0pt}%
\pgfpathmoveto{\pgfqpoint{0.418102in}{0.331635in}}%
\pgfpathlineto{\pgfqpoint{3.236283in}{0.331635in}}%
\pgfusepath{stroke}%
\end{pgfscope}%
\begin{pgfscope}%
\pgfsetrectcap%
\pgfsetmiterjoin%
\pgfsetlinewidth{0.803000pt}%
\definecolor{currentstroke}{rgb}{1.000000,1.000000,1.000000}%
\pgfsetstrokecolor{currentstroke}%
\pgfsetdash{}{0pt}%
\pgfpathmoveto{\pgfqpoint{0.418102in}{1.464135in}}%
\pgfpathlineto{\pgfqpoint{3.236283in}{1.464135in}}%
\pgfusepath{stroke}%
\end{pgfscope}%
\begin{pgfscope}%
\definecolor{textcolor}{rgb}{0.150000,0.150000,0.150000}%
\pgfsetstrokecolor{textcolor}%
\pgfsetfillcolor{textcolor}%
\pgftext[x=1.827193in,y=1.547468in,,base]{\color{textcolor}\rmfamily\fontsize{12.000000}{14.400000}\selectfont Autocorrelation}%
\end{pgfscope}%
\begin{pgfscope}%
\pgfsetbuttcap%
\pgfsetmiterjoin%
\definecolor{currentfill}{rgb}{0.917647,0.917647,0.949020}%
\pgfsetfillcolor{currentfill}%
\pgfsetlinewidth{0.000000pt}%
\definecolor{currentstroke}{rgb}{0.000000,0.000000,0.000000}%
\pgfsetstrokecolor{currentstroke}%
\pgfsetstrokeopacity{0.000000}%
\pgfsetdash{}{0pt}%
\pgfpathmoveto{\pgfqpoint{3.799920in}{0.331635in}}%
\pgfpathlineto{\pgfqpoint{6.618102in}{0.331635in}}%
\pgfpathlineto{\pgfqpoint{6.618102in}{1.464135in}}%
\pgfpathlineto{\pgfqpoint{3.799920in}{1.464135in}}%
\pgfpathclose%
\pgfusepath{fill}%
\end{pgfscope}%
\begin{pgfscope}%
\pgfpathrectangle{\pgfqpoint{3.799920in}{0.331635in}}{\pgfqpoint{2.818182in}{1.132500in}}%
\pgfusepath{clip}%
\pgfsetroundcap%
\pgfsetroundjoin%
\pgfsetlinewidth{0.803000pt}%
\definecolor{currentstroke}{rgb}{1.000000,1.000000,1.000000}%
\pgfsetstrokecolor{currentstroke}%
\pgfsetdash{}{0pt}%
\pgfpathmoveto{\pgfqpoint{3.928019in}{0.331635in}}%
\pgfpathlineto{\pgfqpoint{3.928019in}{1.464135in}}%
\pgfusepath{stroke}%
\end{pgfscope}%
\begin{pgfscope}%
\definecolor{textcolor}{rgb}{0.150000,0.150000,0.150000}%
\pgfsetstrokecolor{textcolor}%
\pgfsetfillcolor{textcolor}%
\pgftext[x=3.928019in,y=0.234413in,,top]{\color{textcolor}\rmfamily\fontsize{10.000000}{12.000000}\selectfont 0}%
\end{pgfscope}%
\begin{pgfscope}%
\pgfpathrectangle{\pgfqpoint{3.799920in}{0.331635in}}{\pgfqpoint{2.818182in}{1.132500in}}%
\pgfusepath{clip}%
\pgfsetroundcap%
\pgfsetroundjoin%
\pgfsetlinewidth{0.803000pt}%
\definecolor{currentstroke}{rgb}{1.000000,1.000000,1.000000}%
\pgfsetstrokecolor{currentstroke}%
\pgfsetdash{}{0pt}%
\pgfpathmoveto{\pgfqpoint{4.552893in}{0.331635in}}%
\pgfpathlineto{\pgfqpoint{4.552893in}{1.464135in}}%
\pgfusepath{stroke}%
\end{pgfscope}%
\begin{pgfscope}%
\definecolor{textcolor}{rgb}{0.150000,0.150000,0.150000}%
\pgfsetstrokecolor{textcolor}%
\pgfsetfillcolor{textcolor}%
\pgftext[x=4.552893in,y=0.234413in,,top]{\color{textcolor}\rmfamily\fontsize{10.000000}{12.000000}\selectfont 5}%
\end{pgfscope}%
\begin{pgfscope}%
\pgfpathrectangle{\pgfqpoint{3.799920in}{0.331635in}}{\pgfqpoint{2.818182in}{1.132500in}}%
\pgfusepath{clip}%
\pgfsetroundcap%
\pgfsetroundjoin%
\pgfsetlinewidth{0.803000pt}%
\definecolor{currentstroke}{rgb}{1.000000,1.000000,1.000000}%
\pgfsetstrokecolor{currentstroke}%
\pgfsetdash{}{0pt}%
\pgfpathmoveto{\pgfqpoint{5.177767in}{0.331635in}}%
\pgfpathlineto{\pgfqpoint{5.177767in}{1.464135in}}%
\pgfusepath{stroke}%
\end{pgfscope}%
\begin{pgfscope}%
\definecolor{textcolor}{rgb}{0.150000,0.150000,0.150000}%
\pgfsetstrokecolor{textcolor}%
\pgfsetfillcolor{textcolor}%
\pgftext[x=5.177767in,y=0.234413in,,top]{\color{textcolor}\rmfamily\fontsize{10.000000}{12.000000}\selectfont 10}%
\end{pgfscope}%
\begin{pgfscope}%
\pgfpathrectangle{\pgfqpoint{3.799920in}{0.331635in}}{\pgfqpoint{2.818182in}{1.132500in}}%
\pgfusepath{clip}%
\pgfsetroundcap%
\pgfsetroundjoin%
\pgfsetlinewidth{0.803000pt}%
\definecolor{currentstroke}{rgb}{1.000000,1.000000,1.000000}%
\pgfsetstrokecolor{currentstroke}%
\pgfsetdash{}{0pt}%
\pgfpathmoveto{\pgfqpoint{5.802641in}{0.331635in}}%
\pgfpathlineto{\pgfqpoint{5.802641in}{1.464135in}}%
\pgfusepath{stroke}%
\end{pgfscope}%
\begin{pgfscope}%
\definecolor{textcolor}{rgb}{0.150000,0.150000,0.150000}%
\pgfsetstrokecolor{textcolor}%
\pgfsetfillcolor{textcolor}%
\pgftext[x=5.802641in,y=0.234413in,,top]{\color{textcolor}\rmfamily\fontsize{10.000000}{12.000000}\selectfont 15}%
\end{pgfscope}%
\begin{pgfscope}%
\pgfpathrectangle{\pgfqpoint{3.799920in}{0.331635in}}{\pgfqpoint{2.818182in}{1.132500in}}%
\pgfusepath{clip}%
\pgfsetroundcap%
\pgfsetroundjoin%
\pgfsetlinewidth{0.803000pt}%
\definecolor{currentstroke}{rgb}{1.000000,1.000000,1.000000}%
\pgfsetstrokecolor{currentstroke}%
\pgfsetdash{}{0pt}%
\pgfpathmoveto{\pgfqpoint{6.427515in}{0.331635in}}%
\pgfpathlineto{\pgfqpoint{6.427515in}{1.464135in}}%
\pgfusepath{stroke}%
\end{pgfscope}%
\begin{pgfscope}%
\definecolor{textcolor}{rgb}{0.150000,0.150000,0.150000}%
\pgfsetstrokecolor{textcolor}%
\pgfsetfillcolor{textcolor}%
\pgftext[x=6.427515in,y=0.234413in,,top]{\color{textcolor}\rmfamily\fontsize{10.000000}{12.000000}\selectfont 20}%
\end{pgfscope}%
\begin{pgfscope}%
\pgfpathrectangle{\pgfqpoint{3.799920in}{0.331635in}}{\pgfqpoint{2.818182in}{1.132500in}}%
\pgfusepath{clip}%
\pgfsetroundcap%
\pgfsetroundjoin%
\pgfsetlinewidth{0.803000pt}%
\definecolor{currentstroke}{rgb}{1.000000,1.000000,1.000000}%
\pgfsetstrokecolor{currentstroke}%
\pgfsetdash{}{0pt}%
\pgfpathmoveto{\pgfqpoint{3.799920in}{0.432578in}}%
\pgfpathlineto{\pgfqpoint{6.618102in}{0.432578in}}%
\pgfusepath{stroke}%
\end{pgfscope}%
\begin{pgfscope}%
\definecolor{textcolor}{rgb}{0.150000,0.150000,0.150000}%
\pgfsetstrokecolor{textcolor}%
\pgfsetfillcolor{textcolor}%
\pgftext[x=3.481818in,y=0.379817in,left,base]{\color{textcolor}\rmfamily\fontsize{10.000000}{12.000000}\selectfont 0.0}%
\end{pgfscope}%
\begin{pgfscope}%
\pgfpathrectangle{\pgfqpoint{3.799920in}{0.331635in}}{\pgfqpoint{2.818182in}{1.132500in}}%
\pgfusepath{clip}%
\pgfsetroundcap%
\pgfsetroundjoin%
\pgfsetlinewidth{0.803000pt}%
\definecolor{currentstroke}{rgb}{1.000000,1.000000,1.000000}%
\pgfsetstrokecolor{currentstroke}%
\pgfsetdash{}{0pt}%
\pgfpathmoveto{\pgfqpoint{3.799920in}{0.922618in}}%
\pgfpathlineto{\pgfqpoint{6.618102in}{0.922618in}}%
\pgfusepath{stroke}%
\end{pgfscope}%
\begin{pgfscope}%
\definecolor{textcolor}{rgb}{0.150000,0.150000,0.150000}%
\pgfsetstrokecolor{textcolor}%
\pgfsetfillcolor{textcolor}%
\pgftext[x=3.481818in,y=0.869857in,left,base]{\color{textcolor}\rmfamily\fontsize{10.000000}{12.000000}\selectfont 0.5}%
\end{pgfscope}%
\begin{pgfscope}%
\pgfpathrectangle{\pgfqpoint{3.799920in}{0.331635in}}{\pgfqpoint{2.818182in}{1.132500in}}%
\pgfusepath{clip}%
\pgfsetroundcap%
\pgfsetroundjoin%
\pgfsetlinewidth{0.803000pt}%
\definecolor{currentstroke}{rgb}{1.000000,1.000000,1.000000}%
\pgfsetstrokecolor{currentstroke}%
\pgfsetdash{}{0pt}%
\pgfpathmoveto{\pgfqpoint{3.799920in}{1.412658in}}%
\pgfpathlineto{\pgfqpoint{6.618102in}{1.412658in}}%
\pgfusepath{stroke}%
\end{pgfscope}%
\begin{pgfscope}%
\definecolor{textcolor}{rgb}{0.150000,0.150000,0.150000}%
\pgfsetstrokecolor{textcolor}%
\pgfsetfillcolor{textcolor}%
\pgftext[x=3.481818in,y=1.359896in,left,base]{\color{textcolor}\rmfamily\fontsize{10.000000}{12.000000}\selectfont 1.0}%
\end{pgfscope}%
\begin{pgfscope}%
\pgfpathrectangle{\pgfqpoint{3.799920in}{0.331635in}}{\pgfqpoint{2.818182in}{1.132500in}}%
\pgfusepath{clip}%
\pgfsetbuttcap%
\pgfsetroundjoin%
\definecolor{currentfill}{rgb}{0.121569,0.466667,0.705882}%
\pgfsetfillcolor{currentfill}%
\pgfsetfillopacity{0.250000}%
\pgfsetlinewidth{1.003750pt}%
\definecolor{currentstroke}{rgb}{1.000000,1.000000,1.000000}%
\pgfsetstrokecolor{currentstroke}%
\pgfsetstrokeopacity{0.250000}%
\pgfsetdash{}{0pt}%
\pgfpathmoveto{\pgfqpoint{3.990506in}{0.482045in}}%
\pgfpathlineto{\pgfqpoint{3.990506in}{0.383112in}}%
\pgfpathlineto{\pgfqpoint{4.177969in}{0.383112in}}%
\pgfpathlineto{\pgfqpoint{4.302943in}{0.383112in}}%
\pgfpathlineto{\pgfqpoint{4.427918in}{0.383112in}}%
\pgfpathlineto{\pgfqpoint{4.552893in}{0.383112in}}%
\pgfpathlineto{\pgfqpoint{4.677868in}{0.383112in}}%
\pgfpathlineto{\pgfqpoint{4.802843in}{0.383112in}}%
\pgfpathlineto{\pgfqpoint{4.927817in}{0.383112in}}%
\pgfpathlineto{\pgfqpoint{5.052792in}{0.383112in}}%
\pgfpathlineto{\pgfqpoint{5.177767in}{0.383112in}}%
\pgfpathlineto{\pgfqpoint{5.302742in}{0.383112in}}%
\pgfpathlineto{\pgfqpoint{5.427717in}{0.383112in}}%
\pgfpathlineto{\pgfqpoint{5.552691in}{0.383112in}}%
\pgfpathlineto{\pgfqpoint{5.677666in}{0.383112in}}%
\pgfpathlineto{\pgfqpoint{5.802641in}{0.383112in}}%
\pgfpathlineto{\pgfqpoint{5.927616in}{0.383112in}}%
\pgfpathlineto{\pgfqpoint{6.052591in}{0.383112in}}%
\pgfpathlineto{\pgfqpoint{6.177565in}{0.383112in}}%
\pgfpathlineto{\pgfqpoint{6.302540in}{0.383112in}}%
\pgfpathlineto{\pgfqpoint{6.490002in}{0.383112in}}%
\pgfpathlineto{\pgfqpoint{6.490002in}{0.482045in}}%
\pgfpathlineto{\pgfqpoint{6.490002in}{0.482045in}}%
\pgfpathlineto{\pgfqpoint{6.302540in}{0.482045in}}%
\pgfpathlineto{\pgfqpoint{6.177565in}{0.482045in}}%
\pgfpathlineto{\pgfqpoint{6.052591in}{0.482045in}}%
\pgfpathlineto{\pgfqpoint{5.927616in}{0.482045in}}%
\pgfpathlineto{\pgfqpoint{5.802641in}{0.482045in}}%
\pgfpathlineto{\pgfqpoint{5.677666in}{0.482045in}}%
\pgfpathlineto{\pgfqpoint{5.552691in}{0.482045in}}%
\pgfpathlineto{\pgfqpoint{5.427717in}{0.482045in}}%
\pgfpathlineto{\pgfqpoint{5.302742in}{0.482045in}}%
\pgfpathlineto{\pgfqpoint{5.177767in}{0.482045in}}%
\pgfpathlineto{\pgfqpoint{5.052792in}{0.482045in}}%
\pgfpathlineto{\pgfqpoint{4.927817in}{0.482045in}}%
\pgfpathlineto{\pgfqpoint{4.802843in}{0.482045in}}%
\pgfpathlineto{\pgfqpoint{4.677868in}{0.482045in}}%
\pgfpathlineto{\pgfqpoint{4.552893in}{0.482045in}}%
\pgfpathlineto{\pgfqpoint{4.427918in}{0.482045in}}%
\pgfpathlineto{\pgfqpoint{4.302943in}{0.482045in}}%
\pgfpathlineto{\pgfqpoint{4.177969in}{0.482045in}}%
\pgfpathlineto{\pgfqpoint{3.990506in}{0.482045in}}%
\pgfpathclose%
\pgfusepath{stroke,fill}%
\end{pgfscope}%
\begin{pgfscope}%
\pgfpathrectangle{\pgfqpoint{3.799920in}{0.331635in}}{\pgfqpoint{2.818182in}{1.132500in}}%
\pgfusepath{clip}%
\pgfsetbuttcap%
\pgfsetroundjoin%
\pgfsetlinewidth{1.505625pt}%
\definecolor{currentstroke}{rgb}{0.000000,0.000000,0.000000}%
\pgfsetstrokecolor{currentstroke}%
\pgfsetdash{}{0pt}%
\pgfpathmoveto{\pgfqpoint{3.928019in}{0.432578in}}%
\pgfpathlineto{\pgfqpoint{3.928019in}{1.412658in}}%
\pgfusepath{stroke}%
\end{pgfscope}%
\begin{pgfscope}%
\pgfpathrectangle{\pgfqpoint{3.799920in}{0.331635in}}{\pgfqpoint{2.818182in}{1.132500in}}%
\pgfusepath{clip}%
\pgfsetbuttcap%
\pgfsetroundjoin%
\pgfsetlinewidth{1.505625pt}%
\definecolor{currentstroke}{rgb}{0.000000,0.000000,0.000000}%
\pgfsetstrokecolor{currentstroke}%
\pgfsetdash{}{0pt}%
\pgfpathmoveto{\pgfqpoint{4.052994in}{0.432578in}}%
\pgfpathlineto{\pgfqpoint{4.052994in}{0.626597in}}%
\pgfusepath{stroke}%
\end{pgfscope}%
\begin{pgfscope}%
\pgfpathrectangle{\pgfqpoint{3.799920in}{0.331635in}}{\pgfqpoint{2.818182in}{1.132500in}}%
\pgfusepath{clip}%
\pgfsetbuttcap%
\pgfsetroundjoin%
\pgfsetlinewidth{1.505625pt}%
\definecolor{currentstroke}{rgb}{0.000000,0.000000,0.000000}%
\pgfsetstrokecolor{currentstroke}%
\pgfsetdash{}{0pt}%
\pgfpathmoveto{\pgfqpoint{4.177969in}{0.432578in}}%
\pgfpathlineto{\pgfqpoint{4.177969in}{0.421707in}}%
\pgfusepath{stroke}%
\end{pgfscope}%
\begin{pgfscope}%
\pgfpathrectangle{\pgfqpoint{3.799920in}{0.331635in}}{\pgfqpoint{2.818182in}{1.132500in}}%
\pgfusepath{clip}%
\pgfsetbuttcap%
\pgfsetroundjoin%
\pgfsetlinewidth{1.505625pt}%
\definecolor{currentstroke}{rgb}{0.000000,0.000000,0.000000}%
\pgfsetstrokecolor{currentstroke}%
\pgfsetdash{}{0pt}%
\pgfpathmoveto{\pgfqpoint{4.302943in}{0.432578in}}%
\pgfpathlineto{\pgfqpoint{4.302943in}{0.458862in}}%
\pgfusepath{stroke}%
\end{pgfscope}%
\begin{pgfscope}%
\pgfpathrectangle{\pgfqpoint{3.799920in}{0.331635in}}{\pgfqpoint{2.818182in}{1.132500in}}%
\pgfusepath{clip}%
\pgfsetbuttcap%
\pgfsetroundjoin%
\pgfsetlinewidth{1.505625pt}%
\definecolor{currentstroke}{rgb}{0.000000,0.000000,0.000000}%
\pgfsetstrokecolor{currentstroke}%
\pgfsetdash{}{0pt}%
\pgfpathmoveto{\pgfqpoint{4.427918in}{0.432578in}}%
\pgfpathlineto{\pgfqpoint{4.427918in}{0.449682in}}%
\pgfusepath{stroke}%
\end{pgfscope}%
\begin{pgfscope}%
\pgfpathrectangle{\pgfqpoint{3.799920in}{0.331635in}}{\pgfqpoint{2.818182in}{1.132500in}}%
\pgfusepath{clip}%
\pgfsetbuttcap%
\pgfsetroundjoin%
\pgfsetlinewidth{1.505625pt}%
\definecolor{currentstroke}{rgb}{0.000000,0.000000,0.000000}%
\pgfsetstrokecolor{currentstroke}%
\pgfsetdash{}{0pt}%
\pgfpathmoveto{\pgfqpoint{4.552893in}{0.432578in}}%
\pgfpathlineto{\pgfqpoint{4.552893in}{0.433730in}}%
\pgfusepath{stroke}%
\end{pgfscope}%
\begin{pgfscope}%
\pgfpathrectangle{\pgfqpoint{3.799920in}{0.331635in}}{\pgfqpoint{2.818182in}{1.132500in}}%
\pgfusepath{clip}%
\pgfsetbuttcap%
\pgfsetroundjoin%
\pgfsetlinewidth{1.505625pt}%
\definecolor{currentstroke}{rgb}{0.000000,0.000000,0.000000}%
\pgfsetstrokecolor{currentstroke}%
\pgfsetdash{}{0pt}%
\pgfpathmoveto{\pgfqpoint{4.677868in}{0.432578in}}%
\pgfpathlineto{\pgfqpoint{4.677868in}{0.441132in}}%
\pgfusepath{stroke}%
\end{pgfscope}%
\begin{pgfscope}%
\pgfpathrectangle{\pgfqpoint{3.799920in}{0.331635in}}{\pgfqpoint{2.818182in}{1.132500in}}%
\pgfusepath{clip}%
\pgfsetbuttcap%
\pgfsetroundjoin%
\pgfsetlinewidth{1.505625pt}%
\definecolor{currentstroke}{rgb}{0.000000,0.000000,0.000000}%
\pgfsetstrokecolor{currentstroke}%
\pgfsetdash{}{0pt}%
\pgfpathmoveto{\pgfqpoint{4.802843in}{0.432578in}}%
\pgfpathlineto{\pgfqpoint{4.802843in}{0.445174in}}%
\pgfusepath{stroke}%
\end{pgfscope}%
\begin{pgfscope}%
\pgfpathrectangle{\pgfqpoint{3.799920in}{0.331635in}}{\pgfqpoint{2.818182in}{1.132500in}}%
\pgfusepath{clip}%
\pgfsetbuttcap%
\pgfsetroundjoin%
\pgfsetlinewidth{1.505625pt}%
\definecolor{currentstroke}{rgb}{0.000000,0.000000,0.000000}%
\pgfsetstrokecolor{currentstroke}%
\pgfsetdash{}{0pt}%
\pgfpathmoveto{\pgfqpoint{4.927817in}{0.432578in}}%
\pgfpathlineto{\pgfqpoint{4.927817in}{0.432380in}}%
\pgfusepath{stroke}%
\end{pgfscope}%
\begin{pgfscope}%
\pgfpathrectangle{\pgfqpoint{3.799920in}{0.331635in}}{\pgfqpoint{2.818182in}{1.132500in}}%
\pgfusepath{clip}%
\pgfsetbuttcap%
\pgfsetroundjoin%
\pgfsetlinewidth{1.505625pt}%
\definecolor{currentstroke}{rgb}{0.000000,0.000000,0.000000}%
\pgfsetstrokecolor{currentstroke}%
\pgfsetdash{}{0pt}%
\pgfpathmoveto{\pgfqpoint{5.052792in}{0.432578in}}%
\pgfpathlineto{\pgfqpoint{5.052792in}{0.443194in}}%
\pgfusepath{stroke}%
\end{pgfscope}%
\begin{pgfscope}%
\pgfpathrectangle{\pgfqpoint{3.799920in}{0.331635in}}{\pgfqpoint{2.818182in}{1.132500in}}%
\pgfusepath{clip}%
\pgfsetbuttcap%
\pgfsetroundjoin%
\pgfsetlinewidth{1.505625pt}%
\definecolor{currentstroke}{rgb}{0.000000,0.000000,0.000000}%
\pgfsetstrokecolor{currentstroke}%
\pgfsetdash{}{0pt}%
\pgfpathmoveto{\pgfqpoint{5.177767in}{0.432578in}}%
\pgfpathlineto{\pgfqpoint{5.177767in}{0.456550in}}%
\pgfusepath{stroke}%
\end{pgfscope}%
\begin{pgfscope}%
\pgfpathrectangle{\pgfqpoint{3.799920in}{0.331635in}}{\pgfqpoint{2.818182in}{1.132500in}}%
\pgfusepath{clip}%
\pgfsetbuttcap%
\pgfsetroundjoin%
\pgfsetlinewidth{1.505625pt}%
\definecolor{currentstroke}{rgb}{0.000000,0.000000,0.000000}%
\pgfsetstrokecolor{currentstroke}%
\pgfsetdash{}{0pt}%
\pgfpathmoveto{\pgfqpoint{5.302742in}{0.432578in}}%
\pgfpathlineto{\pgfqpoint{5.302742in}{0.447498in}}%
\pgfusepath{stroke}%
\end{pgfscope}%
\begin{pgfscope}%
\pgfpathrectangle{\pgfqpoint{3.799920in}{0.331635in}}{\pgfqpoint{2.818182in}{1.132500in}}%
\pgfusepath{clip}%
\pgfsetbuttcap%
\pgfsetroundjoin%
\pgfsetlinewidth{1.505625pt}%
\definecolor{currentstroke}{rgb}{0.000000,0.000000,0.000000}%
\pgfsetstrokecolor{currentstroke}%
\pgfsetdash{}{0pt}%
\pgfpathmoveto{\pgfqpoint{5.427717in}{0.432578in}}%
\pgfpathlineto{\pgfqpoint{5.427717in}{0.444568in}}%
\pgfusepath{stroke}%
\end{pgfscope}%
\begin{pgfscope}%
\pgfpathrectangle{\pgfqpoint{3.799920in}{0.331635in}}{\pgfqpoint{2.818182in}{1.132500in}}%
\pgfusepath{clip}%
\pgfsetbuttcap%
\pgfsetroundjoin%
\pgfsetlinewidth{1.505625pt}%
\definecolor{currentstroke}{rgb}{0.000000,0.000000,0.000000}%
\pgfsetstrokecolor{currentstroke}%
\pgfsetdash{}{0pt}%
\pgfpathmoveto{\pgfqpoint{5.552691in}{0.432578in}}%
\pgfpathlineto{\pgfqpoint{5.552691in}{0.431574in}}%
\pgfusepath{stroke}%
\end{pgfscope}%
\begin{pgfscope}%
\pgfpathrectangle{\pgfqpoint{3.799920in}{0.331635in}}{\pgfqpoint{2.818182in}{1.132500in}}%
\pgfusepath{clip}%
\pgfsetbuttcap%
\pgfsetroundjoin%
\pgfsetlinewidth{1.505625pt}%
\definecolor{currentstroke}{rgb}{0.000000,0.000000,0.000000}%
\pgfsetstrokecolor{currentstroke}%
\pgfsetdash{}{0pt}%
\pgfpathmoveto{\pgfqpoint{5.677666in}{0.432578in}}%
\pgfpathlineto{\pgfqpoint{5.677666in}{0.451459in}}%
\pgfusepath{stroke}%
\end{pgfscope}%
\begin{pgfscope}%
\pgfpathrectangle{\pgfqpoint{3.799920in}{0.331635in}}{\pgfqpoint{2.818182in}{1.132500in}}%
\pgfusepath{clip}%
\pgfsetbuttcap%
\pgfsetroundjoin%
\pgfsetlinewidth{1.505625pt}%
\definecolor{currentstroke}{rgb}{0.000000,0.000000,0.000000}%
\pgfsetstrokecolor{currentstroke}%
\pgfsetdash{}{0pt}%
\pgfpathmoveto{\pgfqpoint{5.802641in}{0.432578in}}%
\pgfpathlineto{\pgfqpoint{5.802641in}{0.527865in}}%
\pgfusepath{stroke}%
\end{pgfscope}%
\begin{pgfscope}%
\pgfpathrectangle{\pgfqpoint{3.799920in}{0.331635in}}{\pgfqpoint{2.818182in}{1.132500in}}%
\pgfusepath{clip}%
\pgfsetbuttcap%
\pgfsetroundjoin%
\pgfsetlinewidth{1.505625pt}%
\definecolor{currentstroke}{rgb}{0.000000,0.000000,0.000000}%
\pgfsetstrokecolor{currentstroke}%
\pgfsetdash{}{0pt}%
\pgfpathmoveto{\pgfqpoint{5.927616in}{0.432578in}}%
\pgfpathlineto{\pgfqpoint{5.927616in}{0.450264in}}%
\pgfusepath{stroke}%
\end{pgfscope}%
\begin{pgfscope}%
\pgfpathrectangle{\pgfqpoint{3.799920in}{0.331635in}}{\pgfqpoint{2.818182in}{1.132500in}}%
\pgfusepath{clip}%
\pgfsetbuttcap%
\pgfsetroundjoin%
\pgfsetlinewidth{1.505625pt}%
\definecolor{currentstroke}{rgb}{0.000000,0.000000,0.000000}%
\pgfsetstrokecolor{currentstroke}%
\pgfsetdash{}{0pt}%
\pgfpathmoveto{\pgfqpoint{6.052591in}{0.432578in}}%
\pgfpathlineto{\pgfqpoint{6.052591in}{0.423702in}}%
\pgfusepath{stroke}%
\end{pgfscope}%
\begin{pgfscope}%
\pgfpathrectangle{\pgfqpoint{3.799920in}{0.331635in}}{\pgfqpoint{2.818182in}{1.132500in}}%
\pgfusepath{clip}%
\pgfsetbuttcap%
\pgfsetroundjoin%
\pgfsetlinewidth{1.505625pt}%
\definecolor{currentstroke}{rgb}{0.000000,0.000000,0.000000}%
\pgfsetstrokecolor{currentstroke}%
\pgfsetdash{}{0pt}%
\pgfpathmoveto{\pgfqpoint{6.177565in}{0.432578in}}%
\pgfpathlineto{\pgfqpoint{6.177565in}{0.429291in}}%
\pgfusepath{stroke}%
\end{pgfscope}%
\begin{pgfscope}%
\pgfpathrectangle{\pgfqpoint{3.799920in}{0.331635in}}{\pgfqpoint{2.818182in}{1.132500in}}%
\pgfusepath{clip}%
\pgfsetbuttcap%
\pgfsetroundjoin%
\pgfsetlinewidth{1.505625pt}%
\definecolor{currentstroke}{rgb}{0.000000,0.000000,0.000000}%
\pgfsetstrokecolor{currentstroke}%
\pgfsetdash{}{0pt}%
\pgfpathmoveto{\pgfqpoint{6.302540in}{0.432578in}}%
\pgfpathlineto{\pgfqpoint{6.302540in}{0.437947in}}%
\pgfusepath{stroke}%
\end{pgfscope}%
\begin{pgfscope}%
\pgfpathrectangle{\pgfqpoint{3.799920in}{0.331635in}}{\pgfqpoint{2.818182in}{1.132500in}}%
\pgfusepath{clip}%
\pgfsetbuttcap%
\pgfsetroundjoin%
\pgfsetlinewidth{1.505625pt}%
\definecolor{currentstroke}{rgb}{0.000000,0.000000,0.000000}%
\pgfsetstrokecolor{currentstroke}%
\pgfsetdash{}{0pt}%
\pgfpathmoveto{\pgfqpoint{6.427515in}{0.432578in}}%
\pgfpathlineto{\pgfqpoint{6.427515in}{0.449231in}}%
\pgfusepath{stroke}%
\end{pgfscope}%
\begin{pgfscope}%
\pgfpathrectangle{\pgfqpoint{3.799920in}{0.331635in}}{\pgfqpoint{2.818182in}{1.132500in}}%
\pgfusepath{clip}%
\pgfsetroundcap%
\pgfsetroundjoin%
\pgfsetlinewidth{1.505625pt}%
\definecolor{currentstroke}{rgb}{0.172549,0.627451,0.172549}%
\pgfsetstrokecolor{currentstroke}%
\pgfsetdash{}{0pt}%
\pgfpathmoveto{\pgfqpoint{3.799920in}{0.432578in}}%
\pgfpathlineto{\pgfqpoint{6.618102in}{0.432578in}}%
\pgfusepath{stroke}%
\end{pgfscope}%
\begin{pgfscope}%
\pgfpathrectangle{\pgfqpoint{3.799920in}{0.331635in}}{\pgfqpoint{2.818182in}{1.132500in}}%
\pgfusepath{clip}%
\pgfsetbuttcap%
\pgfsetroundjoin%
\definecolor{currentfill}{rgb}{0.172549,0.627451,0.172549}%
\pgfsetfillcolor{currentfill}%
\pgfsetlinewidth{1.003750pt}%
\definecolor{currentstroke}{rgb}{0.172549,0.627451,0.172549}%
\pgfsetstrokecolor{currentstroke}%
\pgfsetdash{}{0pt}%
\pgfsys@defobject{currentmarker}{\pgfqpoint{-0.034722in}{-0.034722in}}{\pgfqpoint{0.034722in}{0.034722in}}{%
\pgfpathmoveto{\pgfqpoint{0.000000in}{-0.034722in}}%
\pgfpathcurveto{\pgfqpoint{0.009208in}{-0.034722in}}{\pgfqpoint{0.018041in}{-0.031064in}}{\pgfqpoint{0.024552in}{-0.024552in}}%
\pgfpathcurveto{\pgfqpoint{0.031064in}{-0.018041in}}{\pgfqpoint{0.034722in}{-0.009208in}}{\pgfqpoint{0.034722in}{0.000000in}}%
\pgfpathcurveto{\pgfqpoint{0.034722in}{0.009208in}}{\pgfqpoint{0.031064in}{0.018041in}}{\pgfqpoint{0.024552in}{0.024552in}}%
\pgfpathcurveto{\pgfqpoint{0.018041in}{0.031064in}}{\pgfqpoint{0.009208in}{0.034722in}}{\pgfqpoint{0.000000in}{0.034722in}}%
\pgfpathcurveto{\pgfqpoint{-0.009208in}{0.034722in}}{\pgfqpoint{-0.018041in}{0.031064in}}{\pgfqpoint{-0.024552in}{0.024552in}}%
\pgfpathcurveto{\pgfqpoint{-0.031064in}{0.018041in}}{\pgfqpoint{-0.034722in}{0.009208in}}{\pgfqpoint{-0.034722in}{0.000000in}}%
\pgfpathcurveto{\pgfqpoint{-0.034722in}{-0.009208in}}{\pgfqpoint{-0.031064in}{-0.018041in}}{\pgfqpoint{-0.024552in}{-0.024552in}}%
\pgfpathcurveto{\pgfqpoint{-0.018041in}{-0.031064in}}{\pgfqpoint{-0.009208in}{-0.034722in}}{\pgfqpoint{0.000000in}{-0.034722in}}%
\pgfpathclose%
\pgfusepath{stroke,fill}%
}%
\begin{pgfscope}%
\pgfsys@transformshift{3.928019in}{1.412658in}%
\pgfsys@useobject{currentmarker}{}%
\end{pgfscope}%
\begin{pgfscope}%
\pgfsys@transformshift{4.052994in}{0.626597in}%
\pgfsys@useobject{currentmarker}{}%
\end{pgfscope}%
\begin{pgfscope}%
\pgfsys@transformshift{4.177969in}{0.421707in}%
\pgfsys@useobject{currentmarker}{}%
\end{pgfscope}%
\begin{pgfscope}%
\pgfsys@transformshift{4.302943in}{0.458862in}%
\pgfsys@useobject{currentmarker}{}%
\end{pgfscope}%
\begin{pgfscope}%
\pgfsys@transformshift{4.427918in}{0.449682in}%
\pgfsys@useobject{currentmarker}{}%
\end{pgfscope}%
\begin{pgfscope}%
\pgfsys@transformshift{4.552893in}{0.433730in}%
\pgfsys@useobject{currentmarker}{}%
\end{pgfscope}%
\begin{pgfscope}%
\pgfsys@transformshift{4.677868in}{0.441132in}%
\pgfsys@useobject{currentmarker}{}%
\end{pgfscope}%
\begin{pgfscope}%
\pgfsys@transformshift{4.802843in}{0.445174in}%
\pgfsys@useobject{currentmarker}{}%
\end{pgfscope}%
\begin{pgfscope}%
\pgfsys@transformshift{4.927817in}{0.432380in}%
\pgfsys@useobject{currentmarker}{}%
\end{pgfscope}%
\begin{pgfscope}%
\pgfsys@transformshift{5.052792in}{0.443194in}%
\pgfsys@useobject{currentmarker}{}%
\end{pgfscope}%
\begin{pgfscope}%
\pgfsys@transformshift{5.177767in}{0.456550in}%
\pgfsys@useobject{currentmarker}{}%
\end{pgfscope}%
\begin{pgfscope}%
\pgfsys@transformshift{5.302742in}{0.447498in}%
\pgfsys@useobject{currentmarker}{}%
\end{pgfscope}%
\begin{pgfscope}%
\pgfsys@transformshift{5.427717in}{0.444568in}%
\pgfsys@useobject{currentmarker}{}%
\end{pgfscope}%
\begin{pgfscope}%
\pgfsys@transformshift{5.552691in}{0.431574in}%
\pgfsys@useobject{currentmarker}{}%
\end{pgfscope}%
\begin{pgfscope}%
\pgfsys@transformshift{5.677666in}{0.451459in}%
\pgfsys@useobject{currentmarker}{}%
\end{pgfscope}%
\begin{pgfscope}%
\pgfsys@transformshift{5.802641in}{0.527865in}%
\pgfsys@useobject{currentmarker}{}%
\end{pgfscope}%
\begin{pgfscope}%
\pgfsys@transformshift{5.927616in}{0.450264in}%
\pgfsys@useobject{currentmarker}{}%
\end{pgfscope}%
\begin{pgfscope}%
\pgfsys@transformshift{6.052591in}{0.423702in}%
\pgfsys@useobject{currentmarker}{}%
\end{pgfscope}%
\begin{pgfscope}%
\pgfsys@transformshift{6.177565in}{0.429291in}%
\pgfsys@useobject{currentmarker}{}%
\end{pgfscope}%
\begin{pgfscope}%
\pgfsys@transformshift{6.302540in}{0.437947in}%
\pgfsys@useobject{currentmarker}{}%
\end{pgfscope}%
\begin{pgfscope}%
\pgfsys@transformshift{6.427515in}{0.449231in}%
\pgfsys@useobject{currentmarker}{}%
\end{pgfscope}%
\end{pgfscope}%
\begin{pgfscope}%
\pgfsetrectcap%
\pgfsetmiterjoin%
\pgfsetlinewidth{0.803000pt}%
\definecolor{currentstroke}{rgb}{1.000000,1.000000,1.000000}%
\pgfsetstrokecolor{currentstroke}%
\pgfsetdash{}{0pt}%
\pgfpathmoveto{\pgfqpoint{3.799920in}{0.331635in}}%
\pgfpathlineto{\pgfqpoint{3.799920in}{1.464135in}}%
\pgfusepath{stroke}%
\end{pgfscope}%
\begin{pgfscope}%
\pgfsetrectcap%
\pgfsetmiterjoin%
\pgfsetlinewidth{0.803000pt}%
\definecolor{currentstroke}{rgb}{1.000000,1.000000,1.000000}%
\pgfsetstrokecolor{currentstroke}%
\pgfsetdash{}{0pt}%
\pgfpathmoveto{\pgfqpoint{6.618102in}{0.331635in}}%
\pgfpathlineto{\pgfqpoint{6.618102in}{1.464135in}}%
\pgfusepath{stroke}%
\end{pgfscope}%
\begin{pgfscope}%
\pgfsetrectcap%
\pgfsetmiterjoin%
\pgfsetlinewidth{0.803000pt}%
\definecolor{currentstroke}{rgb}{1.000000,1.000000,1.000000}%
\pgfsetstrokecolor{currentstroke}%
\pgfsetdash{}{0pt}%
\pgfpathmoveto{\pgfqpoint{3.799920in}{0.331635in}}%
\pgfpathlineto{\pgfqpoint{6.618102in}{0.331635in}}%
\pgfusepath{stroke}%
\end{pgfscope}%
\begin{pgfscope}%
\pgfsetrectcap%
\pgfsetmiterjoin%
\pgfsetlinewidth{0.803000pt}%
\definecolor{currentstroke}{rgb}{1.000000,1.000000,1.000000}%
\pgfsetstrokecolor{currentstroke}%
\pgfsetdash{}{0pt}%
\pgfpathmoveto{\pgfqpoint{3.799920in}{1.464135in}}%
\pgfpathlineto{\pgfqpoint{6.618102in}{1.464135in}}%
\pgfusepath{stroke}%
\end{pgfscope}%
\begin{pgfscope}%
\definecolor{textcolor}{rgb}{0.150000,0.150000,0.150000}%
\pgfsetstrokecolor{textcolor}%
\pgfsetfillcolor{textcolor}%
\pgftext[x=5.209011in,y=1.547468in,,base]{\color{textcolor}\rmfamily\fontsize{12.000000}{14.400000}\selectfont Partial Autocorrelation}%
\end{pgfscope}%
\end{pgfpicture}%
\makeatother%
\endgroup%

    \end{adjustbox}  
    \begin{adjustbox}{width=.95\textwidth,center}
    %% Creator: Matplotlib, PGF backend
%%
%% To include the figure in your LaTeX document, write
%%   \input{<filename>.pgf}
%%
%% Make sure the required packages are loaded in your preamble
%%   \usepackage{pgf}
%%
%% Figures using additional raster images can only be included by \input if
%% they are in the same directory as the main LaTeX file. For loading figures
%% from other directories you can use the `import` package
%%   \usepackage{import}
%% and then include the figures with
%%   \import{<path to file>}{<filename>.pgf}
%%
%% Matplotlib used the following preamble
%%   \usepackage{fontspec}
%%   \setmainfont{DejaVuSerif.ttf}[Path=/opt/tljh/user/lib/python3.6/site-packages/matplotlib/mpl-data/fonts/ttf/]
%%   \setsansfont{DejaVuSans.ttf}[Path=/opt/tljh/user/lib/python3.6/site-packages/matplotlib/mpl-data/fonts/ttf/]
%%   \setmonofont{DejaVuSansMono.ttf}[Path=/opt/tljh/user/lib/python3.6/site-packages/matplotlib/mpl-data/fonts/ttf/]
%%
\begingroup%
\makeatletter%
\begin{pgfpicture}%
\pgfpathrectangle{\pgfpointorigin}{\pgfqpoint{6.718102in}{1.774096in}}%
\pgfusepath{use as bounding box, clip}%
\begin{pgfscope}%
\pgfsetbuttcap%
\pgfsetmiterjoin%
\definecolor{currentfill}{rgb}{1.000000,1.000000,1.000000}%
\pgfsetfillcolor{currentfill}%
\pgfsetlinewidth{0.000000pt}%
\definecolor{currentstroke}{rgb}{1.000000,1.000000,1.000000}%
\pgfsetstrokecolor{currentstroke}%
\pgfsetdash{}{0pt}%
\pgfpathmoveto{\pgfqpoint{0.000000in}{0.000000in}}%
\pgfpathlineto{\pgfqpoint{6.718102in}{0.000000in}}%
\pgfpathlineto{\pgfqpoint{6.718102in}{1.774096in}}%
\pgfpathlineto{\pgfqpoint{0.000000in}{1.774096in}}%
\pgfpathclose%
\pgfusepath{fill}%
\end{pgfscope}%
\begin{pgfscope}%
\pgfsetbuttcap%
\pgfsetmiterjoin%
\definecolor{currentfill}{rgb}{0.917647,0.917647,0.949020}%
\pgfsetfillcolor{currentfill}%
\pgfsetlinewidth{0.000000pt}%
\definecolor{currentstroke}{rgb}{0.000000,0.000000,0.000000}%
\pgfsetstrokecolor{currentstroke}%
\pgfsetstrokeopacity{0.000000}%
\pgfsetdash{}{0pt}%
\pgfpathmoveto{\pgfqpoint{0.418102in}{0.331635in}}%
\pgfpathlineto{\pgfqpoint{3.236283in}{0.331635in}}%
\pgfpathlineto{\pgfqpoint{3.236283in}{1.464135in}}%
\pgfpathlineto{\pgfqpoint{0.418102in}{1.464135in}}%
\pgfpathclose%
\pgfusepath{fill}%
\end{pgfscope}%
\begin{pgfscope}%
\pgfpathrectangle{\pgfqpoint{0.418102in}{0.331635in}}{\pgfqpoint{2.818182in}{1.132500in}}%
\pgfusepath{clip}%
\pgfsetroundcap%
\pgfsetroundjoin%
\pgfsetlinewidth{0.803000pt}%
\definecolor{currentstroke}{rgb}{1.000000,1.000000,1.000000}%
\pgfsetstrokecolor{currentstroke}%
\pgfsetdash{}{0pt}%
\pgfpathmoveto{\pgfqpoint{0.546201in}{0.331635in}}%
\pgfpathlineto{\pgfqpoint{0.546201in}{1.464135in}}%
\pgfusepath{stroke}%
\end{pgfscope}%
\begin{pgfscope}%
\definecolor{textcolor}{rgb}{0.150000,0.150000,0.150000}%
\pgfsetstrokecolor{textcolor}%
\pgfsetfillcolor{textcolor}%
\pgftext[x=0.546201in,y=0.234413in,,top]{\color{textcolor}\rmfamily\fontsize{10.000000}{12.000000}\selectfont 0}%
\end{pgfscope}%
\begin{pgfscope}%
\pgfpathrectangle{\pgfqpoint{0.418102in}{0.331635in}}{\pgfqpoint{2.818182in}{1.132500in}}%
\pgfusepath{clip}%
\pgfsetroundcap%
\pgfsetroundjoin%
\pgfsetlinewidth{0.803000pt}%
\definecolor{currentstroke}{rgb}{1.000000,1.000000,1.000000}%
\pgfsetstrokecolor{currentstroke}%
\pgfsetdash{}{0pt}%
\pgfpathmoveto{\pgfqpoint{1.171075in}{0.331635in}}%
\pgfpathlineto{\pgfqpoint{1.171075in}{1.464135in}}%
\pgfusepath{stroke}%
\end{pgfscope}%
\begin{pgfscope}%
\definecolor{textcolor}{rgb}{0.150000,0.150000,0.150000}%
\pgfsetstrokecolor{textcolor}%
\pgfsetfillcolor{textcolor}%
\pgftext[x=1.171075in,y=0.234413in,,top]{\color{textcolor}\rmfamily\fontsize{10.000000}{12.000000}\selectfont 5}%
\end{pgfscope}%
\begin{pgfscope}%
\pgfpathrectangle{\pgfqpoint{0.418102in}{0.331635in}}{\pgfqpoint{2.818182in}{1.132500in}}%
\pgfusepath{clip}%
\pgfsetroundcap%
\pgfsetroundjoin%
\pgfsetlinewidth{0.803000pt}%
\definecolor{currentstroke}{rgb}{1.000000,1.000000,1.000000}%
\pgfsetstrokecolor{currentstroke}%
\pgfsetdash{}{0pt}%
\pgfpathmoveto{\pgfqpoint{1.795949in}{0.331635in}}%
\pgfpathlineto{\pgfqpoint{1.795949in}{1.464135in}}%
\pgfusepath{stroke}%
\end{pgfscope}%
\begin{pgfscope}%
\definecolor{textcolor}{rgb}{0.150000,0.150000,0.150000}%
\pgfsetstrokecolor{textcolor}%
\pgfsetfillcolor{textcolor}%
\pgftext[x=1.795949in,y=0.234413in,,top]{\color{textcolor}\rmfamily\fontsize{10.000000}{12.000000}\selectfont 10}%
\end{pgfscope}%
\begin{pgfscope}%
\pgfpathrectangle{\pgfqpoint{0.418102in}{0.331635in}}{\pgfqpoint{2.818182in}{1.132500in}}%
\pgfusepath{clip}%
\pgfsetroundcap%
\pgfsetroundjoin%
\pgfsetlinewidth{0.803000pt}%
\definecolor{currentstroke}{rgb}{1.000000,1.000000,1.000000}%
\pgfsetstrokecolor{currentstroke}%
\pgfsetdash{}{0pt}%
\pgfpathmoveto{\pgfqpoint{2.420823in}{0.331635in}}%
\pgfpathlineto{\pgfqpoint{2.420823in}{1.464135in}}%
\pgfusepath{stroke}%
\end{pgfscope}%
\begin{pgfscope}%
\definecolor{textcolor}{rgb}{0.150000,0.150000,0.150000}%
\pgfsetstrokecolor{textcolor}%
\pgfsetfillcolor{textcolor}%
\pgftext[x=2.420823in,y=0.234413in,,top]{\color{textcolor}\rmfamily\fontsize{10.000000}{12.000000}\selectfont 15}%
\end{pgfscope}%
\begin{pgfscope}%
\pgfpathrectangle{\pgfqpoint{0.418102in}{0.331635in}}{\pgfqpoint{2.818182in}{1.132500in}}%
\pgfusepath{clip}%
\pgfsetroundcap%
\pgfsetroundjoin%
\pgfsetlinewidth{0.803000pt}%
\definecolor{currentstroke}{rgb}{1.000000,1.000000,1.000000}%
\pgfsetstrokecolor{currentstroke}%
\pgfsetdash{}{0pt}%
\pgfpathmoveto{\pgfqpoint{3.045697in}{0.331635in}}%
\pgfpathlineto{\pgfqpoint{3.045697in}{1.464135in}}%
\pgfusepath{stroke}%
\end{pgfscope}%
\begin{pgfscope}%
\definecolor{textcolor}{rgb}{0.150000,0.150000,0.150000}%
\pgfsetstrokecolor{textcolor}%
\pgfsetfillcolor{textcolor}%
\pgftext[x=3.045697in,y=0.234413in,,top]{\color{textcolor}\rmfamily\fontsize{10.000000}{12.000000}\selectfont 20}%
\end{pgfscope}%
\begin{pgfscope}%
\pgfpathrectangle{\pgfqpoint{0.418102in}{0.331635in}}{\pgfqpoint{2.818182in}{1.132500in}}%
\pgfusepath{clip}%
\pgfsetroundcap%
\pgfsetroundjoin%
\pgfsetlinewidth{0.803000pt}%
\definecolor{currentstroke}{rgb}{1.000000,1.000000,1.000000}%
\pgfsetstrokecolor{currentstroke}%
\pgfsetdash{}{0pt}%
\pgfpathmoveto{\pgfqpoint{0.418102in}{0.437731in}}%
\pgfpathlineto{\pgfqpoint{3.236283in}{0.437731in}}%
\pgfusepath{stroke}%
\end{pgfscope}%
\begin{pgfscope}%
\definecolor{textcolor}{rgb}{0.150000,0.150000,0.150000}%
\pgfsetstrokecolor{textcolor}%
\pgfsetfillcolor{textcolor}%
\pgftext[x=0.100000in,y=0.384969in,left,base]{\color{textcolor}\rmfamily\fontsize{10.000000}{12.000000}\selectfont 0.0}%
\end{pgfscope}%
\begin{pgfscope}%
\pgfpathrectangle{\pgfqpoint{0.418102in}{0.331635in}}{\pgfqpoint{2.818182in}{1.132500in}}%
\pgfusepath{clip}%
\pgfsetroundcap%
\pgfsetroundjoin%
\pgfsetlinewidth{0.803000pt}%
\definecolor{currentstroke}{rgb}{1.000000,1.000000,1.000000}%
\pgfsetstrokecolor{currentstroke}%
\pgfsetdash{}{0pt}%
\pgfpathmoveto{\pgfqpoint{0.418102in}{0.925194in}}%
\pgfpathlineto{\pgfqpoint{3.236283in}{0.925194in}}%
\pgfusepath{stroke}%
\end{pgfscope}%
\begin{pgfscope}%
\definecolor{textcolor}{rgb}{0.150000,0.150000,0.150000}%
\pgfsetstrokecolor{textcolor}%
\pgfsetfillcolor{textcolor}%
\pgftext[x=0.100000in,y=0.872433in,left,base]{\color{textcolor}\rmfamily\fontsize{10.000000}{12.000000}\selectfont 0.5}%
\end{pgfscope}%
\begin{pgfscope}%
\pgfpathrectangle{\pgfqpoint{0.418102in}{0.331635in}}{\pgfqpoint{2.818182in}{1.132500in}}%
\pgfusepath{clip}%
\pgfsetroundcap%
\pgfsetroundjoin%
\pgfsetlinewidth{0.803000pt}%
\definecolor{currentstroke}{rgb}{1.000000,1.000000,1.000000}%
\pgfsetstrokecolor{currentstroke}%
\pgfsetdash{}{0pt}%
\pgfpathmoveto{\pgfqpoint{0.418102in}{1.412658in}}%
\pgfpathlineto{\pgfqpoint{3.236283in}{1.412658in}}%
\pgfusepath{stroke}%
\end{pgfscope}%
\begin{pgfscope}%
\definecolor{textcolor}{rgb}{0.150000,0.150000,0.150000}%
\pgfsetstrokecolor{textcolor}%
\pgfsetfillcolor{textcolor}%
\pgftext[x=0.100000in,y=1.359896in,left,base]{\color{textcolor}\rmfamily\fontsize{10.000000}{12.000000}\selectfont 1.0}%
\end{pgfscope}%
\begin{pgfscope}%
\pgfpathrectangle{\pgfqpoint{0.418102in}{0.331635in}}{\pgfqpoint{2.818182in}{1.132500in}}%
\pgfusepath{clip}%
\pgfsetbuttcap%
\pgfsetroundjoin%
\definecolor{currentfill}{rgb}{0.121569,0.466667,0.705882}%
\pgfsetfillcolor{currentfill}%
\pgfsetfillopacity{0.250000}%
\pgfsetlinewidth{1.003750pt}%
\definecolor{currentstroke}{rgb}{1.000000,1.000000,1.000000}%
\pgfsetstrokecolor{currentstroke}%
\pgfsetstrokeopacity{0.250000}%
\pgfsetdash{}{0pt}%
\pgfpathmoveto{\pgfqpoint{0.608688in}{0.486937in}}%
\pgfpathlineto{\pgfqpoint{0.608688in}{0.388524in}}%
\pgfpathlineto{\pgfqpoint{0.796150in}{0.387625in}}%
\pgfpathlineto{\pgfqpoint{0.921125in}{0.386798in}}%
\pgfpathlineto{\pgfqpoint{1.046100in}{0.386014in}}%
\pgfpathlineto{\pgfqpoint{1.171075in}{0.385480in}}%
\pgfpathlineto{\pgfqpoint{1.296050in}{0.385330in}}%
\pgfpathlineto{\pgfqpoint{1.421024in}{0.385145in}}%
\pgfpathlineto{\pgfqpoint{1.545999in}{0.385036in}}%
\pgfpathlineto{\pgfqpoint{1.670974in}{0.383883in}}%
\pgfpathlineto{\pgfqpoint{1.795949in}{0.383644in}}%
\pgfpathlineto{\pgfqpoint{1.920924in}{0.383630in}}%
\pgfpathlineto{\pgfqpoint{2.045898in}{0.383622in}}%
\pgfpathlineto{\pgfqpoint{2.170873in}{0.383448in}}%
\pgfpathlineto{\pgfqpoint{2.295848in}{0.383447in}}%
\pgfpathlineto{\pgfqpoint{2.420823in}{0.383442in}}%
\pgfpathlineto{\pgfqpoint{2.545798in}{0.383379in}}%
\pgfpathlineto{\pgfqpoint{2.670772in}{0.383204in}}%
\pgfpathlineto{\pgfqpoint{2.795747in}{0.383164in}}%
\pgfpathlineto{\pgfqpoint{2.920722in}{0.383141in}}%
\pgfpathlineto{\pgfqpoint{3.108184in}{0.383112in}}%
\pgfpathlineto{\pgfqpoint{3.108184in}{0.492349in}}%
\pgfpathlineto{\pgfqpoint{3.108184in}{0.492349in}}%
\pgfpathlineto{\pgfqpoint{2.920722in}{0.492320in}}%
\pgfpathlineto{\pgfqpoint{2.795747in}{0.492297in}}%
\pgfpathlineto{\pgfqpoint{2.670772in}{0.492257in}}%
\pgfpathlineto{\pgfqpoint{2.545798in}{0.492083in}}%
\pgfpathlineto{\pgfqpoint{2.420823in}{0.492019in}}%
\pgfpathlineto{\pgfqpoint{2.295848in}{0.492014in}}%
\pgfpathlineto{\pgfqpoint{2.170873in}{0.492014in}}%
\pgfpathlineto{\pgfqpoint{2.045898in}{0.491839in}}%
\pgfpathlineto{\pgfqpoint{1.920924in}{0.491831in}}%
\pgfpathlineto{\pgfqpoint{1.795949in}{0.491818in}}%
\pgfpathlineto{\pgfqpoint{1.670974in}{0.491579in}}%
\pgfpathlineto{\pgfqpoint{1.545999in}{0.490426in}}%
\pgfpathlineto{\pgfqpoint{1.421024in}{0.490317in}}%
\pgfpathlineto{\pgfqpoint{1.296050in}{0.490131in}}%
\pgfpathlineto{\pgfqpoint{1.171075in}{0.489981in}}%
\pgfpathlineto{\pgfqpoint{1.046100in}{0.489448in}}%
\pgfpathlineto{\pgfqpoint{0.921125in}{0.488663in}}%
\pgfpathlineto{\pgfqpoint{0.796150in}{0.487836in}}%
\pgfpathlineto{\pgfqpoint{0.608688in}{0.486937in}}%
\pgfpathclose%
\pgfusepath{stroke,fill}%
\end{pgfscope}%
\begin{pgfscope}%
\pgfpathrectangle{\pgfqpoint{0.418102in}{0.331635in}}{\pgfqpoint{2.818182in}{1.132500in}}%
\pgfusepath{clip}%
\pgfsetbuttcap%
\pgfsetroundjoin%
\pgfsetlinewidth{1.505625pt}%
\definecolor{currentstroke}{rgb}{0.000000,0.000000,0.000000}%
\pgfsetstrokecolor{currentstroke}%
\pgfsetdash{}{0pt}%
\pgfpathmoveto{\pgfqpoint{0.546201in}{0.437731in}}%
\pgfpathlineto{\pgfqpoint{0.546201in}{1.412658in}}%
\pgfusepath{stroke}%
\end{pgfscope}%
\begin{pgfscope}%
\pgfpathrectangle{\pgfqpoint{0.418102in}{0.331635in}}{\pgfqpoint{2.818182in}{1.132500in}}%
\pgfusepath{clip}%
\pgfsetbuttcap%
\pgfsetroundjoin%
\pgfsetlinewidth{1.505625pt}%
\definecolor{currentstroke}{rgb}{0.000000,0.000000,0.000000}%
\pgfsetstrokecolor{currentstroke}%
\pgfsetdash{}{0pt}%
\pgfpathmoveto{\pgfqpoint{0.671176in}{0.437731in}}%
\pgfpathlineto{\pgfqpoint{0.671176in}{0.570148in}}%
\pgfusepath{stroke}%
\end{pgfscope}%
\begin{pgfscope}%
\pgfpathrectangle{\pgfqpoint{0.418102in}{0.331635in}}{\pgfqpoint{2.818182in}{1.132500in}}%
\pgfusepath{clip}%
\pgfsetbuttcap%
\pgfsetroundjoin%
\pgfsetlinewidth{1.505625pt}%
\definecolor{currentstroke}{rgb}{0.000000,0.000000,0.000000}%
\pgfsetstrokecolor{currentstroke}%
\pgfsetdash{}{0pt}%
\pgfpathmoveto{\pgfqpoint{0.796150in}{0.437731in}}%
\pgfpathlineto{\pgfqpoint{0.796150in}{0.565758in}}%
\pgfusepath{stroke}%
\end{pgfscope}%
\begin{pgfscope}%
\pgfpathrectangle{\pgfqpoint{0.418102in}{0.331635in}}{\pgfqpoint{2.818182in}{1.132500in}}%
\pgfusepath{clip}%
\pgfsetbuttcap%
\pgfsetroundjoin%
\pgfsetlinewidth{1.505625pt}%
\definecolor{currentstroke}{rgb}{0.000000,0.000000,0.000000}%
\pgfsetstrokecolor{currentstroke}%
\pgfsetdash{}{0pt}%
\pgfpathmoveto{\pgfqpoint{0.921125in}{0.437731in}}%
\pgfpathlineto{\pgfqpoint{0.921125in}{0.563487in}}%
\pgfusepath{stroke}%
\end{pgfscope}%
\begin{pgfscope}%
\pgfpathrectangle{\pgfqpoint{0.418102in}{0.331635in}}{\pgfqpoint{2.818182in}{1.132500in}}%
\pgfusepath{clip}%
\pgfsetbuttcap%
\pgfsetroundjoin%
\pgfsetlinewidth{1.505625pt}%
\definecolor{currentstroke}{rgb}{0.000000,0.000000,0.000000}%
\pgfsetstrokecolor{currentstroke}%
\pgfsetdash{}{0pt}%
\pgfpathmoveto{\pgfqpoint{1.046100in}{0.437731in}}%
\pgfpathlineto{\pgfqpoint{1.046100in}{0.542077in}}%
\pgfusepath{stroke}%
\end{pgfscope}%
\begin{pgfscope}%
\pgfpathrectangle{\pgfqpoint{0.418102in}{0.331635in}}{\pgfqpoint{2.818182in}{1.132500in}}%
\pgfusepath{clip}%
\pgfsetbuttcap%
\pgfsetroundjoin%
\pgfsetlinewidth{1.505625pt}%
\definecolor{currentstroke}{rgb}{0.000000,0.000000,0.000000}%
\pgfsetstrokecolor{currentstroke}%
\pgfsetdash{}{0pt}%
\pgfpathmoveto{\pgfqpoint{1.171075in}{0.437731in}}%
\pgfpathlineto{\pgfqpoint{1.171075in}{0.493178in}}%
\pgfusepath{stroke}%
\end{pgfscope}%
\begin{pgfscope}%
\pgfpathrectangle{\pgfqpoint{0.418102in}{0.331635in}}{\pgfqpoint{2.818182in}{1.132500in}}%
\pgfusepath{clip}%
\pgfsetbuttcap%
\pgfsetroundjoin%
\pgfsetlinewidth{1.505625pt}%
\definecolor{currentstroke}{rgb}{0.000000,0.000000,0.000000}%
\pgfsetstrokecolor{currentstroke}%
\pgfsetdash{}{0pt}%
\pgfpathmoveto{\pgfqpoint{1.296050in}{0.437731in}}%
\pgfpathlineto{\pgfqpoint{1.296050in}{0.499561in}}%
\pgfusepath{stroke}%
\end{pgfscope}%
\begin{pgfscope}%
\pgfpathrectangle{\pgfqpoint{0.418102in}{0.331635in}}{\pgfqpoint{2.818182in}{1.132500in}}%
\pgfusepath{clip}%
\pgfsetbuttcap%
\pgfsetroundjoin%
\pgfsetlinewidth{1.505625pt}%
\definecolor{currentstroke}{rgb}{0.000000,0.000000,0.000000}%
\pgfsetstrokecolor{currentstroke}%
\pgfsetdash{}{0pt}%
\pgfpathmoveto{\pgfqpoint{1.421024in}{0.437731in}}%
\pgfpathlineto{\pgfqpoint{1.421024in}{0.485241in}}%
\pgfusepath{stroke}%
\end{pgfscope}%
\begin{pgfscope}%
\pgfpathrectangle{\pgfqpoint{0.418102in}{0.331635in}}{\pgfqpoint{2.818182in}{1.132500in}}%
\pgfusepath{clip}%
\pgfsetbuttcap%
\pgfsetroundjoin%
\pgfsetlinewidth{1.505625pt}%
\definecolor{currentstroke}{rgb}{0.000000,0.000000,0.000000}%
\pgfsetstrokecolor{currentstroke}%
\pgfsetdash{}{0pt}%
\pgfpathmoveto{\pgfqpoint{1.545999in}{0.437731in}}%
\pgfpathlineto{\pgfqpoint{1.545999in}{0.593010in}}%
\pgfusepath{stroke}%
\end{pgfscope}%
\begin{pgfscope}%
\pgfpathrectangle{\pgfqpoint{0.418102in}{0.331635in}}{\pgfqpoint{2.818182in}{1.132500in}}%
\pgfusepath{clip}%
\pgfsetbuttcap%
\pgfsetroundjoin%
\pgfsetlinewidth{1.505625pt}%
\definecolor{currentstroke}{rgb}{0.000000,0.000000,0.000000}%
\pgfsetstrokecolor{currentstroke}%
\pgfsetdash{}{0pt}%
\pgfpathmoveto{\pgfqpoint{1.670974in}{0.437731in}}%
\pgfpathlineto{\pgfqpoint{1.670974in}{0.508872in}}%
\pgfusepath{stroke}%
\end{pgfscope}%
\begin{pgfscope}%
\pgfpathrectangle{\pgfqpoint{0.418102in}{0.331635in}}{\pgfqpoint{2.818182in}{1.132500in}}%
\pgfusepath{clip}%
\pgfsetbuttcap%
\pgfsetroundjoin%
\pgfsetlinewidth{1.505625pt}%
\definecolor{currentstroke}{rgb}{0.000000,0.000000,0.000000}%
\pgfsetstrokecolor{currentstroke}%
\pgfsetdash{}{0pt}%
\pgfpathmoveto{\pgfqpoint{1.795949in}{0.437731in}}%
\pgfpathlineto{\pgfqpoint{1.795949in}{0.454729in}}%
\pgfusepath{stroke}%
\end{pgfscope}%
\begin{pgfscope}%
\pgfpathrectangle{\pgfqpoint{0.418102in}{0.331635in}}{\pgfqpoint{2.818182in}{1.132500in}}%
\pgfusepath{clip}%
\pgfsetbuttcap%
\pgfsetroundjoin%
\pgfsetlinewidth{1.505625pt}%
\definecolor{currentstroke}{rgb}{0.000000,0.000000,0.000000}%
\pgfsetstrokecolor{currentstroke}%
\pgfsetdash{}{0pt}%
\pgfpathmoveto{\pgfqpoint{1.920924in}{0.437731in}}%
\pgfpathlineto{\pgfqpoint{1.920924in}{0.450595in}}%
\pgfusepath{stroke}%
\end{pgfscope}%
\begin{pgfscope}%
\pgfpathrectangle{\pgfqpoint{0.418102in}{0.331635in}}{\pgfqpoint{2.818182in}{1.132500in}}%
\pgfusepath{clip}%
\pgfsetbuttcap%
\pgfsetroundjoin%
\pgfsetlinewidth{1.505625pt}%
\definecolor{currentstroke}{rgb}{0.000000,0.000000,0.000000}%
\pgfsetstrokecolor{currentstroke}%
\pgfsetdash{}{0pt}%
\pgfpathmoveto{\pgfqpoint{2.045898in}{0.437731in}}%
\pgfpathlineto{\pgfqpoint{2.045898in}{0.498681in}}%
\pgfusepath{stroke}%
\end{pgfscope}%
\begin{pgfscope}%
\pgfpathrectangle{\pgfqpoint{0.418102in}{0.331635in}}{\pgfqpoint{2.818182in}{1.132500in}}%
\pgfusepath{clip}%
\pgfsetbuttcap%
\pgfsetroundjoin%
\pgfsetlinewidth{1.505625pt}%
\definecolor{currentstroke}{rgb}{0.000000,0.000000,0.000000}%
\pgfsetstrokecolor{currentstroke}%
\pgfsetdash{}{0pt}%
\pgfpathmoveto{\pgfqpoint{2.170873in}{0.437731in}}%
\pgfpathlineto{\pgfqpoint{2.170873in}{0.439571in}}%
\pgfusepath{stroke}%
\end{pgfscope}%
\begin{pgfscope}%
\pgfpathrectangle{\pgfqpoint{0.418102in}{0.331635in}}{\pgfqpoint{2.818182in}{1.132500in}}%
\pgfusepath{clip}%
\pgfsetbuttcap%
\pgfsetroundjoin%
\pgfsetlinewidth{1.505625pt}%
\definecolor{currentstroke}{rgb}{0.000000,0.000000,0.000000}%
\pgfsetstrokecolor{currentstroke}%
\pgfsetdash{}{0pt}%
\pgfpathmoveto{\pgfqpoint{2.295848in}{0.437731in}}%
\pgfpathlineto{\pgfqpoint{2.295848in}{0.448060in}}%
\pgfusepath{stroke}%
\end{pgfscope}%
\begin{pgfscope}%
\pgfpathrectangle{\pgfqpoint{0.418102in}{0.331635in}}{\pgfqpoint{2.818182in}{1.132500in}}%
\pgfusepath{clip}%
\pgfsetbuttcap%
\pgfsetroundjoin%
\pgfsetlinewidth{1.505625pt}%
\definecolor{currentstroke}{rgb}{0.000000,0.000000,0.000000}%
\pgfsetstrokecolor{currentstroke}%
\pgfsetdash{}{0pt}%
\pgfpathmoveto{\pgfqpoint{2.420823in}{0.437731in}}%
\pgfpathlineto{\pgfqpoint{2.420823in}{0.474649in}}%
\pgfusepath{stroke}%
\end{pgfscope}%
\begin{pgfscope}%
\pgfpathrectangle{\pgfqpoint{0.418102in}{0.331635in}}{\pgfqpoint{2.818182in}{1.132500in}}%
\pgfusepath{clip}%
\pgfsetbuttcap%
\pgfsetroundjoin%
\pgfsetlinewidth{1.505625pt}%
\definecolor{currentstroke}{rgb}{0.000000,0.000000,0.000000}%
\pgfsetstrokecolor{currentstroke}%
\pgfsetdash{}{0pt}%
\pgfpathmoveto{\pgfqpoint{2.545798in}{0.437731in}}%
\pgfpathlineto{\pgfqpoint{2.545798in}{0.498785in}}%
\pgfusepath{stroke}%
\end{pgfscope}%
\begin{pgfscope}%
\pgfpathrectangle{\pgfqpoint{0.418102in}{0.331635in}}{\pgfqpoint{2.818182in}{1.132500in}}%
\pgfusepath{clip}%
\pgfsetbuttcap%
\pgfsetroundjoin%
\pgfsetlinewidth{1.505625pt}%
\definecolor{currentstroke}{rgb}{0.000000,0.000000,0.000000}%
\pgfsetstrokecolor{currentstroke}%
\pgfsetdash{}{0pt}%
\pgfpathmoveto{\pgfqpoint{2.670772in}{0.437731in}}%
\pgfpathlineto{\pgfqpoint{2.670772in}{0.408553in}}%
\pgfusepath{stroke}%
\end{pgfscope}%
\begin{pgfscope}%
\pgfpathrectangle{\pgfqpoint{0.418102in}{0.331635in}}{\pgfqpoint{2.818182in}{1.132500in}}%
\pgfusepath{clip}%
\pgfsetbuttcap%
\pgfsetroundjoin%
\pgfsetlinewidth{1.505625pt}%
\definecolor{currentstroke}{rgb}{0.000000,0.000000,0.000000}%
\pgfsetstrokecolor{currentstroke}%
\pgfsetdash{}{0pt}%
\pgfpathmoveto{\pgfqpoint{2.795747in}{0.437731in}}%
\pgfpathlineto{\pgfqpoint{2.795747in}{0.459895in}}%
\pgfusepath{stroke}%
\end{pgfscope}%
\begin{pgfscope}%
\pgfpathrectangle{\pgfqpoint{0.418102in}{0.331635in}}{\pgfqpoint{2.818182in}{1.132500in}}%
\pgfusepath{clip}%
\pgfsetbuttcap%
\pgfsetroundjoin%
\pgfsetlinewidth{1.505625pt}%
\definecolor{currentstroke}{rgb}{0.000000,0.000000,0.000000}%
\pgfsetstrokecolor{currentstroke}%
\pgfsetdash{}{0pt}%
\pgfpathmoveto{\pgfqpoint{2.920722in}{0.437731in}}%
\pgfpathlineto{\pgfqpoint{2.920722in}{0.462741in}}%
\pgfusepath{stroke}%
\end{pgfscope}%
\begin{pgfscope}%
\pgfpathrectangle{\pgfqpoint{0.418102in}{0.331635in}}{\pgfqpoint{2.818182in}{1.132500in}}%
\pgfusepath{clip}%
\pgfsetbuttcap%
\pgfsetroundjoin%
\pgfsetlinewidth{1.505625pt}%
\definecolor{currentstroke}{rgb}{0.000000,0.000000,0.000000}%
\pgfsetstrokecolor{currentstroke}%
\pgfsetdash{}{0pt}%
\pgfpathmoveto{\pgfqpoint{3.045697in}{0.437731in}}%
\pgfpathlineto{\pgfqpoint{3.045697in}{0.453201in}}%
\pgfusepath{stroke}%
\end{pgfscope}%
\begin{pgfscope}%
\pgfpathrectangle{\pgfqpoint{0.418102in}{0.331635in}}{\pgfqpoint{2.818182in}{1.132500in}}%
\pgfusepath{clip}%
\pgfsetroundcap%
\pgfsetroundjoin%
\pgfsetlinewidth{1.505625pt}%
\definecolor{currentstroke}{rgb}{0.580392,0.403922,0.741176}%
\pgfsetstrokecolor{currentstroke}%
\pgfsetdash{}{0pt}%
\pgfpathmoveto{\pgfqpoint{0.418102in}{0.437731in}}%
\pgfpathlineto{\pgfqpoint{3.236283in}{0.437731in}}%
\pgfusepath{stroke}%
\end{pgfscope}%
\begin{pgfscope}%
\pgfpathrectangle{\pgfqpoint{0.418102in}{0.331635in}}{\pgfqpoint{2.818182in}{1.132500in}}%
\pgfusepath{clip}%
\pgfsetbuttcap%
\pgfsetroundjoin%
\definecolor{currentfill}{rgb}{0.580392,0.403922,0.741176}%
\pgfsetfillcolor{currentfill}%
\pgfsetlinewidth{1.003750pt}%
\definecolor{currentstroke}{rgb}{0.580392,0.403922,0.741176}%
\pgfsetstrokecolor{currentstroke}%
\pgfsetdash{}{0pt}%
\pgfsys@defobject{currentmarker}{\pgfqpoint{-0.034722in}{-0.034722in}}{\pgfqpoint{0.034722in}{0.034722in}}{%
\pgfpathmoveto{\pgfqpoint{0.000000in}{-0.034722in}}%
\pgfpathcurveto{\pgfqpoint{0.009208in}{-0.034722in}}{\pgfqpoint{0.018041in}{-0.031064in}}{\pgfqpoint{0.024552in}{-0.024552in}}%
\pgfpathcurveto{\pgfqpoint{0.031064in}{-0.018041in}}{\pgfqpoint{0.034722in}{-0.009208in}}{\pgfqpoint{0.034722in}{0.000000in}}%
\pgfpathcurveto{\pgfqpoint{0.034722in}{0.009208in}}{\pgfqpoint{0.031064in}{0.018041in}}{\pgfqpoint{0.024552in}{0.024552in}}%
\pgfpathcurveto{\pgfqpoint{0.018041in}{0.031064in}}{\pgfqpoint{0.009208in}{0.034722in}}{\pgfqpoint{0.000000in}{0.034722in}}%
\pgfpathcurveto{\pgfqpoint{-0.009208in}{0.034722in}}{\pgfqpoint{-0.018041in}{0.031064in}}{\pgfqpoint{-0.024552in}{0.024552in}}%
\pgfpathcurveto{\pgfqpoint{-0.031064in}{0.018041in}}{\pgfqpoint{-0.034722in}{0.009208in}}{\pgfqpoint{-0.034722in}{0.000000in}}%
\pgfpathcurveto{\pgfqpoint{-0.034722in}{-0.009208in}}{\pgfqpoint{-0.031064in}{-0.018041in}}{\pgfqpoint{-0.024552in}{-0.024552in}}%
\pgfpathcurveto{\pgfqpoint{-0.018041in}{-0.031064in}}{\pgfqpoint{-0.009208in}{-0.034722in}}{\pgfqpoint{0.000000in}{-0.034722in}}%
\pgfpathclose%
\pgfusepath{stroke,fill}%
}%
\begin{pgfscope}%
\pgfsys@transformshift{0.546201in}{1.412658in}%
\pgfsys@useobject{currentmarker}{}%
\end{pgfscope}%
\begin{pgfscope}%
\pgfsys@transformshift{0.671176in}{0.570148in}%
\pgfsys@useobject{currentmarker}{}%
\end{pgfscope}%
\begin{pgfscope}%
\pgfsys@transformshift{0.796150in}{0.565758in}%
\pgfsys@useobject{currentmarker}{}%
\end{pgfscope}%
\begin{pgfscope}%
\pgfsys@transformshift{0.921125in}{0.563487in}%
\pgfsys@useobject{currentmarker}{}%
\end{pgfscope}%
\begin{pgfscope}%
\pgfsys@transformshift{1.046100in}{0.542077in}%
\pgfsys@useobject{currentmarker}{}%
\end{pgfscope}%
\begin{pgfscope}%
\pgfsys@transformshift{1.171075in}{0.493178in}%
\pgfsys@useobject{currentmarker}{}%
\end{pgfscope}%
\begin{pgfscope}%
\pgfsys@transformshift{1.296050in}{0.499561in}%
\pgfsys@useobject{currentmarker}{}%
\end{pgfscope}%
\begin{pgfscope}%
\pgfsys@transformshift{1.421024in}{0.485241in}%
\pgfsys@useobject{currentmarker}{}%
\end{pgfscope}%
\begin{pgfscope}%
\pgfsys@transformshift{1.545999in}{0.593010in}%
\pgfsys@useobject{currentmarker}{}%
\end{pgfscope}%
\begin{pgfscope}%
\pgfsys@transformshift{1.670974in}{0.508872in}%
\pgfsys@useobject{currentmarker}{}%
\end{pgfscope}%
\begin{pgfscope}%
\pgfsys@transformshift{1.795949in}{0.454729in}%
\pgfsys@useobject{currentmarker}{}%
\end{pgfscope}%
\begin{pgfscope}%
\pgfsys@transformshift{1.920924in}{0.450595in}%
\pgfsys@useobject{currentmarker}{}%
\end{pgfscope}%
\begin{pgfscope}%
\pgfsys@transformshift{2.045898in}{0.498681in}%
\pgfsys@useobject{currentmarker}{}%
\end{pgfscope}%
\begin{pgfscope}%
\pgfsys@transformshift{2.170873in}{0.439571in}%
\pgfsys@useobject{currentmarker}{}%
\end{pgfscope}%
\begin{pgfscope}%
\pgfsys@transformshift{2.295848in}{0.448060in}%
\pgfsys@useobject{currentmarker}{}%
\end{pgfscope}%
\begin{pgfscope}%
\pgfsys@transformshift{2.420823in}{0.474649in}%
\pgfsys@useobject{currentmarker}{}%
\end{pgfscope}%
\begin{pgfscope}%
\pgfsys@transformshift{2.545798in}{0.498785in}%
\pgfsys@useobject{currentmarker}{}%
\end{pgfscope}%
\begin{pgfscope}%
\pgfsys@transformshift{2.670772in}{0.408553in}%
\pgfsys@useobject{currentmarker}{}%
\end{pgfscope}%
\begin{pgfscope}%
\pgfsys@transformshift{2.795747in}{0.459895in}%
\pgfsys@useobject{currentmarker}{}%
\end{pgfscope}%
\begin{pgfscope}%
\pgfsys@transformshift{2.920722in}{0.462741in}%
\pgfsys@useobject{currentmarker}{}%
\end{pgfscope}%
\begin{pgfscope}%
\pgfsys@transformshift{3.045697in}{0.453201in}%
\pgfsys@useobject{currentmarker}{}%
\end{pgfscope}%
\end{pgfscope}%
\begin{pgfscope}%
\pgfsetrectcap%
\pgfsetmiterjoin%
\pgfsetlinewidth{0.803000pt}%
\definecolor{currentstroke}{rgb}{1.000000,1.000000,1.000000}%
\pgfsetstrokecolor{currentstroke}%
\pgfsetdash{}{0pt}%
\pgfpathmoveto{\pgfqpoint{0.418102in}{0.331635in}}%
\pgfpathlineto{\pgfqpoint{0.418102in}{1.464135in}}%
\pgfusepath{stroke}%
\end{pgfscope}%
\begin{pgfscope}%
\pgfsetrectcap%
\pgfsetmiterjoin%
\pgfsetlinewidth{0.803000pt}%
\definecolor{currentstroke}{rgb}{1.000000,1.000000,1.000000}%
\pgfsetstrokecolor{currentstroke}%
\pgfsetdash{}{0pt}%
\pgfpathmoveto{\pgfqpoint{3.236283in}{0.331635in}}%
\pgfpathlineto{\pgfqpoint{3.236283in}{1.464135in}}%
\pgfusepath{stroke}%
\end{pgfscope}%
\begin{pgfscope}%
\pgfsetrectcap%
\pgfsetmiterjoin%
\pgfsetlinewidth{0.803000pt}%
\definecolor{currentstroke}{rgb}{1.000000,1.000000,1.000000}%
\pgfsetstrokecolor{currentstroke}%
\pgfsetdash{}{0pt}%
\pgfpathmoveto{\pgfqpoint{0.418102in}{0.331635in}}%
\pgfpathlineto{\pgfqpoint{3.236283in}{0.331635in}}%
\pgfusepath{stroke}%
\end{pgfscope}%
\begin{pgfscope}%
\pgfsetrectcap%
\pgfsetmiterjoin%
\pgfsetlinewidth{0.803000pt}%
\definecolor{currentstroke}{rgb}{1.000000,1.000000,1.000000}%
\pgfsetstrokecolor{currentstroke}%
\pgfsetdash{}{0pt}%
\pgfpathmoveto{\pgfqpoint{0.418102in}{1.464135in}}%
\pgfpathlineto{\pgfqpoint{3.236283in}{1.464135in}}%
\pgfusepath{stroke}%
\end{pgfscope}%
\begin{pgfscope}%
\definecolor{textcolor}{rgb}{0.150000,0.150000,0.150000}%
\pgfsetstrokecolor{textcolor}%
\pgfsetfillcolor{textcolor}%
\pgftext[x=1.827193in,y=1.547468in,,base]{\color{textcolor}\rmfamily\fontsize{12.000000}{14.400000}\selectfont Autocorrelation}%
\end{pgfscope}%
\begin{pgfscope}%
\pgfsetbuttcap%
\pgfsetmiterjoin%
\definecolor{currentfill}{rgb}{0.917647,0.917647,0.949020}%
\pgfsetfillcolor{currentfill}%
\pgfsetlinewidth{0.000000pt}%
\definecolor{currentstroke}{rgb}{0.000000,0.000000,0.000000}%
\pgfsetstrokecolor{currentstroke}%
\pgfsetstrokeopacity{0.000000}%
\pgfsetdash{}{0pt}%
\pgfpathmoveto{\pgfqpoint{3.799920in}{0.331635in}}%
\pgfpathlineto{\pgfqpoint{6.618102in}{0.331635in}}%
\pgfpathlineto{\pgfqpoint{6.618102in}{1.464135in}}%
\pgfpathlineto{\pgfqpoint{3.799920in}{1.464135in}}%
\pgfpathclose%
\pgfusepath{fill}%
\end{pgfscope}%
\begin{pgfscope}%
\pgfpathrectangle{\pgfqpoint{3.799920in}{0.331635in}}{\pgfqpoint{2.818182in}{1.132500in}}%
\pgfusepath{clip}%
\pgfsetroundcap%
\pgfsetroundjoin%
\pgfsetlinewidth{0.803000pt}%
\definecolor{currentstroke}{rgb}{1.000000,1.000000,1.000000}%
\pgfsetstrokecolor{currentstroke}%
\pgfsetdash{}{0pt}%
\pgfpathmoveto{\pgfqpoint{3.928019in}{0.331635in}}%
\pgfpathlineto{\pgfqpoint{3.928019in}{1.464135in}}%
\pgfusepath{stroke}%
\end{pgfscope}%
\begin{pgfscope}%
\definecolor{textcolor}{rgb}{0.150000,0.150000,0.150000}%
\pgfsetstrokecolor{textcolor}%
\pgfsetfillcolor{textcolor}%
\pgftext[x=3.928019in,y=0.234413in,,top]{\color{textcolor}\rmfamily\fontsize{10.000000}{12.000000}\selectfont 0}%
\end{pgfscope}%
\begin{pgfscope}%
\pgfpathrectangle{\pgfqpoint{3.799920in}{0.331635in}}{\pgfqpoint{2.818182in}{1.132500in}}%
\pgfusepath{clip}%
\pgfsetroundcap%
\pgfsetroundjoin%
\pgfsetlinewidth{0.803000pt}%
\definecolor{currentstroke}{rgb}{1.000000,1.000000,1.000000}%
\pgfsetstrokecolor{currentstroke}%
\pgfsetdash{}{0pt}%
\pgfpathmoveto{\pgfqpoint{4.552893in}{0.331635in}}%
\pgfpathlineto{\pgfqpoint{4.552893in}{1.464135in}}%
\pgfusepath{stroke}%
\end{pgfscope}%
\begin{pgfscope}%
\definecolor{textcolor}{rgb}{0.150000,0.150000,0.150000}%
\pgfsetstrokecolor{textcolor}%
\pgfsetfillcolor{textcolor}%
\pgftext[x=4.552893in,y=0.234413in,,top]{\color{textcolor}\rmfamily\fontsize{10.000000}{12.000000}\selectfont 5}%
\end{pgfscope}%
\begin{pgfscope}%
\pgfpathrectangle{\pgfqpoint{3.799920in}{0.331635in}}{\pgfqpoint{2.818182in}{1.132500in}}%
\pgfusepath{clip}%
\pgfsetroundcap%
\pgfsetroundjoin%
\pgfsetlinewidth{0.803000pt}%
\definecolor{currentstroke}{rgb}{1.000000,1.000000,1.000000}%
\pgfsetstrokecolor{currentstroke}%
\pgfsetdash{}{0pt}%
\pgfpathmoveto{\pgfqpoint{5.177767in}{0.331635in}}%
\pgfpathlineto{\pgfqpoint{5.177767in}{1.464135in}}%
\pgfusepath{stroke}%
\end{pgfscope}%
\begin{pgfscope}%
\definecolor{textcolor}{rgb}{0.150000,0.150000,0.150000}%
\pgfsetstrokecolor{textcolor}%
\pgfsetfillcolor{textcolor}%
\pgftext[x=5.177767in,y=0.234413in,,top]{\color{textcolor}\rmfamily\fontsize{10.000000}{12.000000}\selectfont 10}%
\end{pgfscope}%
\begin{pgfscope}%
\pgfpathrectangle{\pgfqpoint{3.799920in}{0.331635in}}{\pgfqpoint{2.818182in}{1.132500in}}%
\pgfusepath{clip}%
\pgfsetroundcap%
\pgfsetroundjoin%
\pgfsetlinewidth{0.803000pt}%
\definecolor{currentstroke}{rgb}{1.000000,1.000000,1.000000}%
\pgfsetstrokecolor{currentstroke}%
\pgfsetdash{}{0pt}%
\pgfpathmoveto{\pgfqpoint{5.802641in}{0.331635in}}%
\pgfpathlineto{\pgfqpoint{5.802641in}{1.464135in}}%
\pgfusepath{stroke}%
\end{pgfscope}%
\begin{pgfscope}%
\definecolor{textcolor}{rgb}{0.150000,0.150000,0.150000}%
\pgfsetstrokecolor{textcolor}%
\pgfsetfillcolor{textcolor}%
\pgftext[x=5.802641in,y=0.234413in,,top]{\color{textcolor}\rmfamily\fontsize{10.000000}{12.000000}\selectfont 15}%
\end{pgfscope}%
\begin{pgfscope}%
\pgfpathrectangle{\pgfqpoint{3.799920in}{0.331635in}}{\pgfqpoint{2.818182in}{1.132500in}}%
\pgfusepath{clip}%
\pgfsetroundcap%
\pgfsetroundjoin%
\pgfsetlinewidth{0.803000pt}%
\definecolor{currentstroke}{rgb}{1.000000,1.000000,1.000000}%
\pgfsetstrokecolor{currentstroke}%
\pgfsetdash{}{0pt}%
\pgfpathmoveto{\pgfqpoint{6.427515in}{0.331635in}}%
\pgfpathlineto{\pgfqpoint{6.427515in}{1.464135in}}%
\pgfusepath{stroke}%
\end{pgfscope}%
\begin{pgfscope}%
\definecolor{textcolor}{rgb}{0.150000,0.150000,0.150000}%
\pgfsetstrokecolor{textcolor}%
\pgfsetfillcolor{textcolor}%
\pgftext[x=6.427515in,y=0.234413in,,top]{\color{textcolor}\rmfamily\fontsize{10.000000}{12.000000}\selectfont 20}%
\end{pgfscope}%
\begin{pgfscope}%
\pgfpathrectangle{\pgfqpoint{3.799920in}{0.331635in}}{\pgfqpoint{2.818182in}{1.132500in}}%
\pgfusepath{clip}%
\pgfsetroundcap%
\pgfsetroundjoin%
\pgfsetlinewidth{0.803000pt}%
\definecolor{currentstroke}{rgb}{1.000000,1.000000,1.000000}%
\pgfsetstrokecolor{currentstroke}%
\pgfsetdash{}{0pt}%
\pgfpathmoveto{\pgfqpoint{3.799920in}{0.443206in}}%
\pgfpathlineto{\pgfqpoint{6.618102in}{0.443206in}}%
\pgfusepath{stroke}%
\end{pgfscope}%
\begin{pgfscope}%
\definecolor{textcolor}{rgb}{0.150000,0.150000,0.150000}%
\pgfsetstrokecolor{textcolor}%
\pgfsetfillcolor{textcolor}%
\pgftext[x=3.481818in,y=0.390444in,left,base]{\color{textcolor}\rmfamily\fontsize{10.000000}{12.000000}\selectfont 0.0}%
\end{pgfscope}%
\begin{pgfscope}%
\pgfpathrectangle{\pgfqpoint{3.799920in}{0.331635in}}{\pgfqpoint{2.818182in}{1.132500in}}%
\pgfusepath{clip}%
\pgfsetroundcap%
\pgfsetroundjoin%
\pgfsetlinewidth{0.803000pt}%
\definecolor{currentstroke}{rgb}{1.000000,1.000000,1.000000}%
\pgfsetstrokecolor{currentstroke}%
\pgfsetdash{}{0pt}%
\pgfpathmoveto{\pgfqpoint{3.799920in}{0.927932in}}%
\pgfpathlineto{\pgfqpoint{6.618102in}{0.927932in}}%
\pgfusepath{stroke}%
\end{pgfscope}%
\begin{pgfscope}%
\definecolor{textcolor}{rgb}{0.150000,0.150000,0.150000}%
\pgfsetstrokecolor{textcolor}%
\pgfsetfillcolor{textcolor}%
\pgftext[x=3.481818in,y=0.875170in,left,base]{\color{textcolor}\rmfamily\fontsize{10.000000}{12.000000}\selectfont 0.5}%
\end{pgfscope}%
\begin{pgfscope}%
\pgfpathrectangle{\pgfqpoint{3.799920in}{0.331635in}}{\pgfqpoint{2.818182in}{1.132500in}}%
\pgfusepath{clip}%
\pgfsetroundcap%
\pgfsetroundjoin%
\pgfsetlinewidth{0.803000pt}%
\definecolor{currentstroke}{rgb}{1.000000,1.000000,1.000000}%
\pgfsetstrokecolor{currentstroke}%
\pgfsetdash{}{0pt}%
\pgfpathmoveto{\pgfqpoint{3.799920in}{1.412658in}}%
\pgfpathlineto{\pgfqpoint{6.618102in}{1.412658in}}%
\pgfusepath{stroke}%
\end{pgfscope}%
\begin{pgfscope}%
\definecolor{textcolor}{rgb}{0.150000,0.150000,0.150000}%
\pgfsetstrokecolor{textcolor}%
\pgfsetfillcolor{textcolor}%
\pgftext[x=3.481818in,y=1.359896in,left,base]{\color{textcolor}\rmfamily\fontsize{10.000000}{12.000000}\selectfont 1.0}%
\end{pgfscope}%
\begin{pgfscope}%
\pgfpathrectangle{\pgfqpoint{3.799920in}{0.331635in}}{\pgfqpoint{2.818182in}{1.132500in}}%
\pgfusepath{clip}%
\pgfsetbuttcap%
\pgfsetroundjoin%
\definecolor{currentfill}{rgb}{0.121569,0.466667,0.705882}%
\pgfsetfillcolor{currentfill}%
\pgfsetfillopacity{0.250000}%
\pgfsetlinewidth{1.003750pt}%
\definecolor{currentstroke}{rgb}{1.000000,1.000000,1.000000}%
\pgfsetstrokecolor{currentstroke}%
\pgfsetstrokeopacity{0.250000}%
\pgfsetdash{}{0pt}%
\pgfpathmoveto{\pgfqpoint{3.990506in}{0.492135in}}%
\pgfpathlineto{\pgfqpoint{3.990506in}{0.394276in}}%
\pgfpathlineto{\pgfqpoint{4.177969in}{0.394276in}}%
\pgfpathlineto{\pgfqpoint{4.302943in}{0.394276in}}%
\pgfpathlineto{\pgfqpoint{4.427918in}{0.394276in}}%
\pgfpathlineto{\pgfqpoint{4.552893in}{0.394276in}}%
\pgfpathlineto{\pgfqpoint{4.677868in}{0.394276in}}%
\pgfpathlineto{\pgfqpoint{4.802843in}{0.394276in}}%
\pgfpathlineto{\pgfqpoint{4.927817in}{0.394276in}}%
\pgfpathlineto{\pgfqpoint{5.052792in}{0.394276in}}%
\pgfpathlineto{\pgfqpoint{5.177767in}{0.394276in}}%
\pgfpathlineto{\pgfqpoint{5.302742in}{0.394276in}}%
\pgfpathlineto{\pgfqpoint{5.427717in}{0.394276in}}%
\pgfpathlineto{\pgfqpoint{5.552691in}{0.394276in}}%
\pgfpathlineto{\pgfqpoint{5.677666in}{0.394276in}}%
\pgfpathlineto{\pgfqpoint{5.802641in}{0.394276in}}%
\pgfpathlineto{\pgfqpoint{5.927616in}{0.394276in}}%
\pgfpathlineto{\pgfqpoint{6.052591in}{0.394276in}}%
\pgfpathlineto{\pgfqpoint{6.177565in}{0.394276in}}%
\pgfpathlineto{\pgfqpoint{6.302540in}{0.394276in}}%
\pgfpathlineto{\pgfqpoint{6.490002in}{0.394276in}}%
\pgfpathlineto{\pgfqpoint{6.490002in}{0.492135in}}%
\pgfpathlineto{\pgfqpoint{6.490002in}{0.492135in}}%
\pgfpathlineto{\pgfqpoint{6.302540in}{0.492135in}}%
\pgfpathlineto{\pgfqpoint{6.177565in}{0.492135in}}%
\pgfpathlineto{\pgfqpoint{6.052591in}{0.492135in}}%
\pgfpathlineto{\pgfqpoint{5.927616in}{0.492135in}}%
\pgfpathlineto{\pgfqpoint{5.802641in}{0.492135in}}%
\pgfpathlineto{\pgfqpoint{5.677666in}{0.492135in}}%
\pgfpathlineto{\pgfqpoint{5.552691in}{0.492135in}}%
\pgfpathlineto{\pgfqpoint{5.427717in}{0.492135in}}%
\pgfpathlineto{\pgfqpoint{5.302742in}{0.492135in}}%
\pgfpathlineto{\pgfqpoint{5.177767in}{0.492135in}}%
\pgfpathlineto{\pgfqpoint{5.052792in}{0.492135in}}%
\pgfpathlineto{\pgfqpoint{4.927817in}{0.492135in}}%
\pgfpathlineto{\pgfqpoint{4.802843in}{0.492135in}}%
\pgfpathlineto{\pgfqpoint{4.677868in}{0.492135in}}%
\pgfpathlineto{\pgfqpoint{4.552893in}{0.492135in}}%
\pgfpathlineto{\pgfqpoint{4.427918in}{0.492135in}}%
\pgfpathlineto{\pgfqpoint{4.302943in}{0.492135in}}%
\pgfpathlineto{\pgfqpoint{4.177969in}{0.492135in}}%
\pgfpathlineto{\pgfqpoint{3.990506in}{0.492135in}}%
\pgfpathclose%
\pgfusepath{stroke,fill}%
\end{pgfscope}%
\begin{pgfscope}%
\pgfpathrectangle{\pgfqpoint{3.799920in}{0.331635in}}{\pgfqpoint{2.818182in}{1.132500in}}%
\pgfusepath{clip}%
\pgfsetbuttcap%
\pgfsetroundjoin%
\pgfsetlinewidth{1.505625pt}%
\definecolor{currentstroke}{rgb}{0.000000,0.000000,0.000000}%
\pgfsetstrokecolor{currentstroke}%
\pgfsetdash{}{0pt}%
\pgfpathmoveto{\pgfqpoint{3.928019in}{0.443206in}}%
\pgfpathlineto{\pgfqpoint{3.928019in}{1.412658in}}%
\pgfusepath{stroke}%
\end{pgfscope}%
\begin{pgfscope}%
\pgfpathrectangle{\pgfqpoint{3.799920in}{0.331635in}}{\pgfqpoint{2.818182in}{1.132500in}}%
\pgfusepath{clip}%
\pgfsetbuttcap%
\pgfsetroundjoin%
\pgfsetlinewidth{1.505625pt}%
\definecolor{currentstroke}{rgb}{0.000000,0.000000,0.000000}%
\pgfsetstrokecolor{currentstroke}%
\pgfsetdash{}{0pt}%
\pgfpathmoveto{\pgfqpoint{4.052994in}{0.443206in}}%
\pgfpathlineto{\pgfqpoint{4.052994in}{0.574967in}}%
\pgfusepath{stroke}%
\end{pgfscope}%
\begin{pgfscope}%
\pgfpathrectangle{\pgfqpoint{3.799920in}{0.331635in}}{\pgfqpoint{2.818182in}{1.132500in}}%
\pgfusepath{clip}%
\pgfsetbuttcap%
\pgfsetroundjoin%
\pgfsetlinewidth{1.505625pt}%
\definecolor{currentstroke}{rgb}{0.000000,0.000000,0.000000}%
\pgfsetstrokecolor{currentstroke}%
\pgfsetdash{}{0pt}%
\pgfpathmoveto{\pgfqpoint{4.177969in}{0.443206in}}%
\pgfpathlineto{\pgfqpoint{4.177969in}{0.554837in}}%
\pgfusepath{stroke}%
\end{pgfscope}%
\begin{pgfscope}%
\pgfpathrectangle{\pgfqpoint{3.799920in}{0.331635in}}{\pgfqpoint{2.818182in}{1.132500in}}%
\pgfusepath{clip}%
\pgfsetbuttcap%
\pgfsetroundjoin%
\pgfsetlinewidth{1.505625pt}%
\definecolor{currentstroke}{rgb}{0.000000,0.000000,0.000000}%
\pgfsetstrokecolor{currentstroke}%
\pgfsetdash{}{0pt}%
\pgfpathmoveto{\pgfqpoint{4.302943in}{0.443206in}}%
\pgfpathlineto{\pgfqpoint{4.302943in}{0.541084in}}%
\pgfusepath{stroke}%
\end{pgfscope}%
\begin{pgfscope}%
\pgfpathrectangle{\pgfqpoint{3.799920in}{0.331635in}}{\pgfqpoint{2.818182in}{1.132500in}}%
\pgfusepath{clip}%
\pgfsetbuttcap%
\pgfsetroundjoin%
\pgfsetlinewidth{1.505625pt}%
\definecolor{currentstroke}{rgb}{0.000000,0.000000,0.000000}%
\pgfsetstrokecolor{currentstroke}%
\pgfsetdash{}{0pt}%
\pgfpathmoveto{\pgfqpoint{4.427918in}{0.443206in}}%
\pgfpathlineto{\pgfqpoint{4.427918in}{0.509957in}}%
\pgfusepath{stroke}%
\end{pgfscope}%
\begin{pgfscope}%
\pgfpathrectangle{\pgfqpoint{3.799920in}{0.331635in}}{\pgfqpoint{2.818182in}{1.132500in}}%
\pgfusepath{clip}%
\pgfsetbuttcap%
\pgfsetroundjoin%
\pgfsetlinewidth{1.505625pt}%
\definecolor{currentstroke}{rgb}{0.000000,0.000000,0.000000}%
\pgfsetstrokecolor{currentstroke}%
\pgfsetdash{}{0pt}%
\pgfpathmoveto{\pgfqpoint{4.552893in}{0.443206in}}%
\pgfpathlineto{\pgfqpoint{4.552893in}{0.455503in}}%
\pgfusepath{stroke}%
\end{pgfscope}%
\begin{pgfscope}%
\pgfpathrectangle{\pgfqpoint{3.799920in}{0.331635in}}{\pgfqpoint{2.818182in}{1.132500in}}%
\pgfusepath{clip}%
\pgfsetbuttcap%
\pgfsetroundjoin%
\pgfsetlinewidth{1.505625pt}%
\definecolor{currentstroke}{rgb}{0.000000,0.000000,0.000000}%
\pgfsetstrokecolor{currentstroke}%
\pgfsetdash{}{0pt}%
\pgfpathmoveto{\pgfqpoint{4.677868in}{0.443206in}}%
\pgfpathlineto{\pgfqpoint{4.677868in}{0.468838in}}%
\pgfusepath{stroke}%
\end{pgfscope}%
\begin{pgfscope}%
\pgfpathrectangle{\pgfqpoint{3.799920in}{0.331635in}}{\pgfqpoint{2.818182in}{1.132500in}}%
\pgfusepath{clip}%
\pgfsetbuttcap%
\pgfsetroundjoin%
\pgfsetlinewidth{1.505625pt}%
\definecolor{currentstroke}{rgb}{0.000000,0.000000,0.000000}%
\pgfsetstrokecolor{currentstroke}%
\pgfsetdash{}{0pt}%
\pgfpathmoveto{\pgfqpoint{4.802843in}{0.443206in}}%
\pgfpathlineto{\pgfqpoint{4.802843in}{0.457747in}}%
\pgfusepath{stroke}%
\end{pgfscope}%
\begin{pgfscope}%
\pgfpathrectangle{\pgfqpoint{3.799920in}{0.331635in}}{\pgfqpoint{2.818182in}{1.132500in}}%
\pgfusepath{clip}%
\pgfsetbuttcap%
\pgfsetroundjoin%
\pgfsetlinewidth{1.505625pt}%
\definecolor{currentstroke}{rgb}{0.000000,0.000000,0.000000}%
\pgfsetstrokecolor{currentstroke}%
\pgfsetdash{}{0pt}%
\pgfpathmoveto{\pgfqpoint{4.927817in}{0.443206in}}%
\pgfpathlineto{\pgfqpoint{4.927817in}{0.576433in}}%
\pgfusepath{stroke}%
\end{pgfscope}%
\begin{pgfscope}%
\pgfpathrectangle{\pgfqpoint{3.799920in}{0.331635in}}{\pgfqpoint{2.818182in}{1.132500in}}%
\pgfusepath{clip}%
\pgfsetbuttcap%
\pgfsetroundjoin%
\pgfsetlinewidth{1.505625pt}%
\definecolor{currentstroke}{rgb}{0.000000,0.000000,0.000000}%
\pgfsetstrokecolor{currentstroke}%
\pgfsetdash{}{0pt}%
\pgfpathmoveto{\pgfqpoint{5.052792in}{0.443206in}}%
\pgfpathlineto{\pgfqpoint{5.052792in}{0.468699in}}%
\pgfusepath{stroke}%
\end{pgfscope}%
\begin{pgfscope}%
\pgfpathrectangle{\pgfqpoint{3.799920in}{0.331635in}}{\pgfqpoint{2.818182in}{1.132500in}}%
\pgfusepath{clip}%
\pgfsetbuttcap%
\pgfsetroundjoin%
\pgfsetlinewidth{1.505625pt}%
\definecolor{currentstroke}{rgb}{0.000000,0.000000,0.000000}%
\pgfsetstrokecolor{currentstroke}%
\pgfsetdash{}{0pt}%
\pgfpathmoveto{\pgfqpoint{5.177767in}{0.443206in}}%
\pgfpathlineto{\pgfqpoint{5.177767in}{0.408553in}}%
\pgfusepath{stroke}%
\end{pgfscope}%
\begin{pgfscope}%
\pgfpathrectangle{\pgfqpoint{3.799920in}{0.331635in}}{\pgfqpoint{2.818182in}{1.132500in}}%
\pgfusepath{clip}%
\pgfsetbuttcap%
\pgfsetroundjoin%
\pgfsetlinewidth{1.505625pt}%
\definecolor{currentstroke}{rgb}{0.000000,0.000000,0.000000}%
\pgfsetstrokecolor{currentstroke}%
\pgfsetdash{}{0pt}%
\pgfpathmoveto{\pgfqpoint{5.302742in}{0.443206in}}%
\pgfpathlineto{\pgfqpoint{5.302742in}{0.412632in}}%
\pgfusepath{stroke}%
\end{pgfscope}%
\begin{pgfscope}%
\pgfpathrectangle{\pgfqpoint{3.799920in}{0.331635in}}{\pgfqpoint{2.818182in}{1.132500in}}%
\pgfusepath{clip}%
\pgfsetbuttcap%
\pgfsetroundjoin%
\pgfsetlinewidth{1.505625pt}%
\definecolor{currentstroke}{rgb}{0.000000,0.000000,0.000000}%
\pgfsetstrokecolor{currentstroke}%
\pgfsetdash{}{0pt}%
\pgfpathmoveto{\pgfqpoint{5.427717in}{0.443206in}}%
\pgfpathlineto{\pgfqpoint{5.427717in}{0.477551in}}%
\pgfusepath{stroke}%
\end{pgfscope}%
\begin{pgfscope}%
\pgfpathrectangle{\pgfqpoint{3.799920in}{0.331635in}}{\pgfqpoint{2.818182in}{1.132500in}}%
\pgfusepath{clip}%
\pgfsetbuttcap%
\pgfsetroundjoin%
\pgfsetlinewidth{1.505625pt}%
\definecolor{currentstroke}{rgb}{0.000000,0.000000,0.000000}%
\pgfsetstrokecolor{currentstroke}%
\pgfsetdash{}{0pt}%
\pgfpathmoveto{\pgfqpoint{5.552691in}{0.443206in}}%
\pgfpathlineto{\pgfqpoint{5.552691in}{0.425170in}}%
\pgfusepath{stroke}%
\end{pgfscope}%
\begin{pgfscope}%
\pgfpathrectangle{\pgfqpoint{3.799920in}{0.331635in}}{\pgfqpoint{2.818182in}{1.132500in}}%
\pgfusepath{clip}%
\pgfsetbuttcap%
\pgfsetroundjoin%
\pgfsetlinewidth{1.505625pt}%
\definecolor{currentstroke}{rgb}{0.000000,0.000000,0.000000}%
\pgfsetstrokecolor{currentstroke}%
\pgfsetdash{}{0pt}%
\pgfpathmoveto{\pgfqpoint{5.677666in}{0.443206in}}%
\pgfpathlineto{\pgfqpoint{5.677666in}{0.438619in}}%
\pgfusepath{stroke}%
\end{pgfscope}%
\begin{pgfscope}%
\pgfpathrectangle{\pgfqpoint{3.799920in}{0.331635in}}{\pgfqpoint{2.818182in}{1.132500in}}%
\pgfusepath{clip}%
\pgfsetbuttcap%
\pgfsetroundjoin%
\pgfsetlinewidth{1.505625pt}%
\definecolor{currentstroke}{rgb}{0.000000,0.000000,0.000000}%
\pgfsetstrokecolor{currentstroke}%
\pgfsetdash{}{0pt}%
\pgfpathmoveto{\pgfqpoint{5.802641in}{0.443206in}}%
\pgfpathlineto{\pgfqpoint{5.802641in}{0.468304in}}%
\pgfusepath{stroke}%
\end{pgfscope}%
\begin{pgfscope}%
\pgfpathrectangle{\pgfqpoint{3.799920in}{0.331635in}}{\pgfqpoint{2.818182in}{1.132500in}}%
\pgfusepath{clip}%
\pgfsetbuttcap%
\pgfsetroundjoin%
\pgfsetlinewidth{1.505625pt}%
\definecolor{currentstroke}{rgb}{0.000000,0.000000,0.000000}%
\pgfsetstrokecolor{currentstroke}%
\pgfsetdash{}{0pt}%
\pgfpathmoveto{\pgfqpoint{5.927616in}{0.443206in}}%
\pgfpathlineto{\pgfqpoint{5.927616in}{0.476231in}}%
\pgfusepath{stroke}%
\end{pgfscope}%
\begin{pgfscope}%
\pgfpathrectangle{\pgfqpoint{3.799920in}{0.331635in}}{\pgfqpoint{2.818182in}{1.132500in}}%
\pgfusepath{clip}%
\pgfsetbuttcap%
\pgfsetroundjoin%
\pgfsetlinewidth{1.505625pt}%
\definecolor{currentstroke}{rgb}{0.000000,0.000000,0.000000}%
\pgfsetstrokecolor{currentstroke}%
\pgfsetdash{}{0pt}%
\pgfpathmoveto{\pgfqpoint{6.052591in}{0.443206in}}%
\pgfpathlineto{\pgfqpoint{6.052591in}{0.383112in}}%
\pgfusepath{stroke}%
\end{pgfscope}%
\begin{pgfscope}%
\pgfpathrectangle{\pgfqpoint{3.799920in}{0.331635in}}{\pgfqpoint{2.818182in}{1.132500in}}%
\pgfusepath{clip}%
\pgfsetbuttcap%
\pgfsetroundjoin%
\pgfsetlinewidth{1.505625pt}%
\definecolor{currentstroke}{rgb}{0.000000,0.000000,0.000000}%
\pgfsetstrokecolor{currentstroke}%
\pgfsetdash{}{0pt}%
\pgfpathmoveto{\pgfqpoint{6.177565in}{0.443206in}}%
\pgfpathlineto{\pgfqpoint{6.177565in}{0.460194in}}%
\pgfusepath{stroke}%
\end{pgfscope}%
\begin{pgfscope}%
\pgfpathrectangle{\pgfqpoint{3.799920in}{0.331635in}}{\pgfqpoint{2.818182in}{1.132500in}}%
\pgfusepath{clip}%
\pgfsetbuttcap%
\pgfsetroundjoin%
\pgfsetlinewidth{1.505625pt}%
\definecolor{currentstroke}{rgb}{0.000000,0.000000,0.000000}%
\pgfsetstrokecolor{currentstroke}%
\pgfsetdash{}{0pt}%
\pgfpathmoveto{\pgfqpoint{6.302540in}{0.443206in}}%
\pgfpathlineto{\pgfqpoint{6.302540in}{0.467276in}}%
\pgfusepath{stroke}%
\end{pgfscope}%
\begin{pgfscope}%
\pgfpathrectangle{\pgfqpoint{3.799920in}{0.331635in}}{\pgfqpoint{2.818182in}{1.132500in}}%
\pgfusepath{clip}%
\pgfsetbuttcap%
\pgfsetroundjoin%
\pgfsetlinewidth{1.505625pt}%
\definecolor{currentstroke}{rgb}{0.000000,0.000000,0.000000}%
\pgfsetstrokecolor{currentstroke}%
\pgfsetdash{}{0pt}%
\pgfpathmoveto{\pgfqpoint{6.427515in}{0.443206in}}%
\pgfpathlineto{\pgfqpoint{6.427515in}{0.441854in}}%
\pgfusepath{stroke}%
\end{pgfscope}%
\begin{pgfscope}%
\pgfpathrectangle{\pgfqpoint{3.799920in}{0.331635in}}{\pgfqpoint{2.818182in}{1.132500in}}%
\pgfusepath{clip}%
\pgfsetroundcap%
\pgfsetroundjoin%
\pgfsetlinewidth{1.505625pt}%
\definecolor{currentstroke}{rgb}{0.580392,0.403922,0.741176}%
\pgfsetstrokecolor{currentstroke}%
\pgfsetdash{}{0pt}%
\pgfpathmoveto{\pgfqpoint{3.799920in}{0.443206in}}%
\pgfpathlineto{\pgfqpoint{6.618102in}{0.443206in}}%
\pgfusepath{stroke}%
\end{pgfscope}%
\begin{pgfscope}%
\pgfpathrectangle{\pgfqpoint{3.799920in}{0.331635in}}{\pgfqpoint{2.818182in}{1.132500in}}%
\pgfusepath{clip}%
\pgfsetbuttcap%
\pgfsetroundjoin%
\definecolor{currentfill}{rgb}{0.580392,0.403922,0.741176}%
\pgfsetfillcolor{currentfill}%
\pgfsetlinewidth{1.003750pt}%
\definecolor{currentstroke}{rgb}{0.580392,0.403922,0.741176}%
\pgfsetstrokecolor{currentstroke}%
\pgfsetdash{}{0pt}%
\pgfsys@defobject{currentmarker}{\pgfqpoint{-0.034722in}{-0.034722in}}{\pgfqpoint{0.034722in}{0.034722in}}{%
\pgfpathmoveto{\pgfqpoint{0.000000in}{-0.034722in}}%
\pgfpathcurveto{\pgfqpoint{0.009208in}{-0.034722in}}{\pgfqpoint{0.018041in}{-0.031064in}}{\pgfqpoint{0.024552in}{-0.024552in}}%
\pgfpathcurveto{\pgfqpoint{0.031064in}{-0.018041in}}{\pgfqpoint{0.034722in}{-0.009208in}}{\pgfqpoint{0.034722in}{0.000000in}}%
\pgfpathcurveto{\pgfqpoint{0.034722in}{0.009208in}}{\pgfqpoint{0.031064in}{0.018041in}}{\pgfqpoint{0.024552in}{0.024552in}}%
\pgfpathcurveto{\pgfqpoint{0.018041in}{0.031064in}}{\pgfqpoint{0.009208in}{0.034722in}}{\pgfqpoint{0.000000in}{0.034722in}}%
\pgfpathcurveto{\pgfqpoint{-0.009208in}{0.034722in}}{\pgfqpoint{-0.018041in}{0.031064in}}{\pgfqpoint{-0.024552in}{0.024552in}}%
\pgfpathcurveto{\pgfqpoint{-0.031064in}{0.018041in}}{\pgfqpoint{-0.034722in}{0.009208in}}{\pgfqpoint{-0.034722in}{0.000000in}}%
\pgfpathcurveto{\pgfqpoint{-0.034722in}{-0.009208in}}{\pgfqpoint{-0.031064in}{-0.018041in}}{\pgfqpoint{-0.024552in}{-0.024552in}}%
\pgfpathcurveto{\pgfqpoint{-0.018041in}{-0.031064in}}{\pgfqpoint{-0.009208in}{-0.034722in}}{\pgfqpoint{0.000000in}{-0.034722in}}%
\pgfpathclose%
\pgfusepath{stroke,fill}%
}%
\begin{pgfscope}%
\pgfsys@transformshift{3.928019in}{1.412658in}%
\pgfsys@useobject{currentmarker}{}%
\end{pgfscope}%
\begin{pgfscope}%
\pgfsys@transformshift{4.052994in}{0.574967in}%
\pgfsys@useobject{currentmarker}{}%
\end{pgfscope}%
\begin{pgfscope}%
\pgfsys@transformshift{4.177969in}{0.554837in}%
\pgfsys@useobject{currentmarker}{}%
\end{pgfscope}%
\begin{pgfscope}%
\pgfsys@transformshift{4.302943in}{0.541084in}%
\pgfsys@useobject{currentmarker}{}%
\end{pgfscope}%
\begin{pgfscope}%
\pgfsys@transformshift{4.427918in}{0.509957in}%
\pgfsys@useobject{currentmarker}{}%
\end{pgfscope}%
\begin{pgfscope}%
\pgfsys@transformshift{4.552893in}{0.455503in}%
\pgfsys@useobject{currentmarker}{}%
\end{pgfscope}%
\begin{pgfscope}%
\pgfsys@transformshift{4.677868in}{0.468838in}%
\pgfsys@useobject{currentmarker}{}%
\end{pgfscope}%
\begin{pgfscope}%
\pgfsys@transformshift{4.802843in}{0.457747in}%
\pgfsys@useobject{currentmarker}{}%
\end{pgfscope}%
\begin{pgfscope}%
\pgfsys@transformshift{4.927817in}{0.576433in}%
\pgfsys@useobject{currentmarker}{}%
\end{pgfscope}%
\begin{pgfscope}%
\pgfsys@transformshift{5.052792in}{0.468699in}%
\pgfsys@useobject{currentmarker}{}%
\end{pgfscope}%
\begin{pgfscope}%
\pgfsys@transformshift{5.177767in}{0.408553in}%
\pgfsys@useobject{currentmarker}{}%
\end{pgfscope}%
\begin{pgfscope}%
\pgfsys@transformshift{5.302742in}{0.412632in}%
\pgfsys@useobject{currentmarker}{}%
\end{pgfscope}%
\begin{pgfscope}%
\pgfsys@transformshift{5.427717in}{0.477551in}%
\pgfsys@useobject{currentmarker}{}%
\end{pgfscope}%
\begin{pgfscope}%
\pgfsys@transformshift{5.552691in}{0.425170in}%
\pgfsys@useobject{currentmarker}{}%
\end{pgfscope}%
\begin{pgfscope}%
\pgfsys@transformshift{5.677666in}{0.438619in}%
\pgfsys@useobject{currentmarker}{}%
\end{pgfscope}%
\begin{pgfscope}%
\pgfsys@transformshift{5.802641in}{0.468304in}%
\pgfsys@useobject{currentmarker}{}%
\end{pgfscope}%
\begin{pgfscope}%
\pgfsys@transformshift{5.927616in}{0.476231in}%
\pgfsys@useobject{currentmarker}{}%
\end{pgfscope}%
\begin{pgfscope}%
\pgfsys@transformshift{6.052591in}{0.383112in}%
\pgfsys@useobject{currentmarker}{}%
\end{pgfscope}%
\begin{pgfscope}%
\pgfsys@transformshift{6.177565in}{0.460194in}%
\pgfsys@useobject{currentmarker}{}%
\end{pgfscope}%
\begin{pgfscope}%
\pgfsys@transformshift{6.302540in}{0.467276in}%
\pgfsys@useobject{currentmarker}{}%
\end{pgfscope}%
\begin{pgfscope}%
\pgfsys@transformshift{6.427515in}{0.441854in}%
\pgfsys@useobject{currentmarker}{}%
\end{pgfscope}%
\end{pgfscope}%
\begin{pgfscope}%
\pgfsetrectcap%
\pgfsetmiterjoin%
\pgfsetlinewidth{0.803000pt}%
\definecolor{currentstroke}{rgb}{1.000000,1.000000,1.000000}%
\pgfsetstrokecolor{currentstroke}%
\pgfsetdash{}{0pt}%
\pgfpathmoveto{\pgfqpoint{3.799920in}{0.331635in}}%
\pgfpathlineto{\pgfqpoint{3.799920in}{1.464135in}}%
\pgfusepath{stroke}%
\end{pgfscope}%
\begin{pgfscope}%
\pgfsetrectcap%
\pgfsetmiterjoin%
\pgfsetlinewidth{0.803000pt}%
\definecolor{currentstroke}{rgb}{1.000000,1.000000,1.000000}%
\pgfsetstrokecolor{currentstroke}%
\pgfsetdash{}{0pt}%
\pgfpathmoveto{\pgfqpoint{6.618102in}{0.331635in}}%
\pgfpathlineto{\pgfqpoint{6.618102in}{1.464135in}}%
\pgfusepath{stroke}%
\end{pgfscope}%
\begin{pgfscope}%
\pgfsetrectcap%
\pgfsetmiterjoin%
\pgfsetlinewidth{0.803000pt}%
\definecolor{currentstroke}{rgb}{1.000000,1.000000,1.000000}%
\pgfsetstrokecolor{currentstroke}%
\pgfsetdash{}{0pt}%
\pgfpathmoveto{\pgfqpoint{3.799920in}{0.331635in}}%
\pgfpathlineto{\pgfqpoint{6.618102in}{0.331635in}}%
\pgfusepath{stroke}%
\end{pgfscope}%
\begin{pgfscope}%
\pgfsetrectcap%
\pgfsetmiterjoin%
\pgfsetlinewidth{0.803000pt}%
\definecolor{currentstroke}{rgb}{1.000000,1.000000,1.000000}%
\pgfsetstrokecolor{currentstroke}%
\pgfsetdash{}{0pt}%
\pgfpathmoveto{\pgfqpoint{3.799920in}{1.464135in}}%
\pgfpathlineto{\pgfqpoint{6.618102in}{1.464135in}}%
\pgfusepath{stroke}%
\end{pgfscope}%
\begin{pgfscope}%
\definecolor{textcolor}{rgb}{0.150000,0.150000,0.150000}%
\pgfsetstrokecolor{textcolor}%
\pgfsetfillcolor{textcolor}%
\pgftext[x=5.209011in,y=1.547468in,,base]{\color{textcolor}\rmfamily\fontsize{12.000000}{14.400000}\selectfont Partial Autocorrelation}%
\end{pgfscope}%
\end{pgfpicture}%
\makeatother%
\endgroup%

    \end{adjustbox}  
    \caption{ACF and PACF of squared residuals of stocks MMM, GE and JNJ. We see that some of the squared log-returns exhibit indeed autocorrelation, while others do less so. Strong autocorrelation implies that there is information about the future in the time series that can be modeled.}
    \label{fig:ACF_selected_squared_log_returns}
\end{figure}{}


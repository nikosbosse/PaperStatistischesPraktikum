\chapter{\textcolor{yellow}{Predictions Using Time Series} - Nikos} \label{ch:ts}

The following chapter will first give a theoretical overview over different time series models and concepts. Then those models will be used to analyze a subset of the data (consisting of two stocks, Visa and Intel) in order to get a thorough understanding for the ways in which stock market data can or cannot be approached. Lastly time series predictions are generated and evaluated for all stocks

\section{\textcolor{yellow}{Theoretical Overview} - Nikos}
Time series predictions in this paper will be made using Random Walks, autoregressive (AR), moving average (MA) and generalized autoregressive conditional heteroscedasticity (GARCH) models. Those models are explained in the following.

\subsection{\textcolor{yellow}{Random Walks  - Nikos du }}
Random walks serve as the baseline against which every prediction can be compared. Assuming a random walk as the underlying process implies that we know nothing about the future and can do no better than assuming tomorrow's stock price will on average be the same as today. Formally, a random walk follows \begin{align}
    y_t &= y_{t-1} + w_t \label{eg:rw1}\\
\intertext{where $y_t$ is the value of the time series at time t and $w_t$ is a random realisation of a stationary white noise process with mean 0 and variance $\sigma^2$. Predictions for period t + 1 are therefore exactly the value at time t. We can expand equation \ref{eg:rw1} by allowing for a constant trend, a drift. A random walk with drift can be represented as}
    y_t &= \delta + y_{t-1} + w_t
\end{align}{}
where $\delta$ is a drift parameter. Note that the random walk (with or without drift) is not a stationary process. 

\subsection{\textcolor{yellow}{Autoregressive Models - Nikos}}
An autoregressive process of order p (AR(p)) implies that the current value of a time series can be described as a combination of the previous p values plus a random shock. As those previous values themselves depend on previous values, the current value is indirectly influenced by its entire past. Formally, an AR(p) process follows 
\begin{align}
    y_t &= \psi_1 y_{t-1} + \psi_2 y_{t-2} + ... + \psi_p y_{t-p} + w_t \label{eq:AR(p)}
\intertext{where $y_t$ is stationary, $\psi_1, ..., \psi_p$ are constants and $w_t$ is white noise. The mean of $y_t$ is assumed to be zero. If the mean is $\mu$ instead of zero, equation \ref{eq:AR(p)} can be rewritten as}
    y_t - \mu &= \phi_1 (y_{t-1} - \mu) + \phi_2 (y_{t-2} - \mu) + ... + \phi_p (y_{t-p} - \mu) + w_t \label{eq:AR(p)2} \\
\intertext{This can also be expressed as}
    y_t &= \alpha + \phi_1 y_{t-1} + \phi_2 y_{t-2} + ... + \phi_p y_{t-p} + w_t
\end{align}
\noindent with $\alpha$ = $\mu (1 - \phi_1 - ... - \phi_p)$.

\subsection{\textcolor{yellow}{Moving Average Models - Nikos}}
A moving average process of order q implies that the current value of a time series consists of the average of the previous q observations plus a random shock. As the mean of the time series $\mu$ is constant this average can also be simply expressed as an average of the past random shocks $\{w_{t-1}, ... w_{t-q}\} $. In contrast to the AR(p) process, the shocks affect the future directly (and not only indirectly through past values) and only affect the next q values. Formally, the MA(q) process can be expressed as
\begin{align}
    y_t = \mu + w_t + \theta w_{t-1} + ... + \theta w_{t-q}
\end{align}{}
\noindent where $w_t$ represents white noise and $\theta_1, ..., \theta_q$ are parameters and q is the number of lags in the moving average. 


\subsection{Autoregressive Moving Average Models - Nikos}
Autoregressive Moving Average Models of order p and q (ARMA(p,q)) form a combination of the above described AR(p) and MA(q) models. Formally, an ARMA(p,q) process follows
\begin{align}
    y_t = \phi_1 y_{t-1} + ... + \phi_p y_{t-p} + w_t + \theta_1 w_{t-1} + ... + \theta_q w_{t-q} \\
    \intertext{if the mean of $y_t$ is $\mu$, then the above results in}
    y_t = \alpha + \phi_1 y_{t-1} + ... + \phi_p y_{t-p} + w_t + \theta_1 w_{t-1} + ... + \theta_q w_{t-q}
\end{align}
\noindent with $\alpha = \mu (1 - \phi_1 - ... - \phi_p)$.


\subsection{GARCH Models - Nikos}
The GARCH(p,q) model is specified as follows:
\begin{align}
    r_{t} &= \sigma_t  \upepsilon_t \label{eq:garch1}\\
\intertext{where $\upepsilon_t$ is Gaussian white noise with  $\upepsilon_t \sim \mathcal{N}(0,1)$ and}
    \sigma^2_t &= \alpha_0 + \underbrace{\alpha_1 r^2_{t-1} + ... + \alpha_p r^2_{t-p}}_\text{autoregressive part} + \underbrace{\beta_1 \sigma^2_{t-1} + ... + \beta_q \sigma^2_{t-q}}_\text{moving average part} \label{eq:garch2}
\end{align}{}
In equation \ref{eq:garch1} the returns $r_t$ are therefore modelled as white noise with mean zero and variance variance $\sigma^2_t$. When compared to white Gaussian noise with constant variance this can produce a leptokurtic (fat-tailed) distribution similar to what we observed in the QQ-Plots in figure \ref{fig:PG_squared_log_returns}. Equation \ref{eq:garch1} is called the mean model of the GARCH(p,q) process. This mean model can also be altered as needed. The GARCH model can then be specified in the following way: 
\begin{equation}
    r_t = x_t + y_t
\end{equation}{}
where $x_t$ can be any constant mean, regression or time series process and $y_t$ is a GARCH process that satisfies equations \ref{eq:garch1} and \ref{eq:garch2}. In a similar way, the distribution of $\upepsilon_t$ can be altered. In praxis, researchers often assume a t-distribution instead of a standard normal distribution. 

A further expansion is the GJR-GARCH model (SOURCE). The GJR-GARCH model includes a separate term for past negative shocks. GJR-GARCH(1,1,1) is specified as follows

\begin{equation}
    \sigma^2_t = \alpha_0 + \alpha_1 \upepsilon^2_{t-1} + \gamma_1 \epsilon^2_{t-1} I(\upepsilon^2_{t-1} < 0) + \beta_1 \sigma^2_{t-1} \label{eg:garch3}
\end{equation}{}

where $I(\upepsilon^2_{t-1} < 0)$ is an indicator function that is one if $\upepsilon^2_{t-1}$ is negative and zero else. This means that negative shocks may have a different impact on future volatility than positive shocks, e.g. a sudden drop in a stock will cause the stock to be disproportionally more volatile in the near future. 

\section{\textcolor{red}{Full Example Analysis with Training Data  - Nikos}}
In the following we will analyze the stocks of Visa (V) and Intel (INTC) in depth. 

\subsection{\textcolor{red}{Data Exploration and Autocorrelation- Nikos}}
The entire time series could be seen back in figure \ref{fig:Daily Stock Prices for all Stocks in the Data Set}. The plots of the log-returns look very similar to the plot in figure \ref{fig:PG_squared_log_returns} and are therefore not shown. Figure \ref{fig:INTC_V_ACF_log_returns} shows the ACF and PACF for the log-returns of INTC and V. From looking at the plots we can presume that for V, AR and MA models of order one or two might be a reasonable try. For INTC it looks like there is very little information included as none of the lower order lags bear any significance. 

\begin{figure}[h!]
    \centering
    \figuretitle{ACF and PACF of log-returns of Stocks INTC and V}
    \begin{adjustbox}{width=.95\textwidth,center}
    %% Creator: Matplotlib, PGF backend
%%
%% To include the figure in your LaTeX document, write
%%   \input{<filename>.pgf}
%%
%% Make sure the required packages are loaded in your preamble
%%   \usepackage{pgf}
%%
%% Figures using additional raster images can only be included by \input if
%% they are in the same directory as the main LaTeX file. For loading figures
%% from other directories you can use the `import` package
%%   \usepackage{import}
%% and then include the figures with
%%   \import{<path to file>}{<filename>.pgf}
%%
%% Matplotlib used the following preamble
%%   \usepackage{fontspec}
%%   \setmainfont{DejaVuSerif.ttf}[Path=/opt/tljh/user/lib/python3.6/site-packages/matplotlib/mpl-data/fonts/ttf/]
%%   \setsansfont{DejaVuSans.ttf}[Path=/opt/tljh/user/lib/python3.6/site-packages/matplotlib/mpl-data/fonts/ttf/]
%%   \setmonofont{DejaVuSansMono.ttf}[Path=/opt/tljh/user/lib/python3.6/site-packages/matplotlib/mpl-data/fonts/ttf/]
%%
\begingroup%
\makeatletter%
\begin{pgfpicture}%
\pgfpathrectangle{\pgfpointorigin}{\pgfqpoint{6.718102in}{1.774096in}}%
\pgfusepath{use as bounding box, clip}%
\begin{pgfscope}%
\pgfsetbuttcap%
\pgfsetmiterjoin%
\definecolor{currentfill}{rgb}{1.000000,1.000000,1.000000}%
\pgfsetfillcolor{currentfill}%
\pgfsetlinewidth{0.000000pt}%
\definecolor{currentstroke}{rgb}{1.000000,1.000000,1.000000}%
\pgfsetstrokecolor{currentstroke}%
\pgfsetdash{}{0pt}%
\pgfpathmoveto{\pgfqpoint{0.000000in}{0.000000in}}%
\pgfpathlineto{\pgfqpoint{6.718102in}{0.000000in}}%
\pgfpathlineto{\pgfqpoint{6.718102in}{1.774096in}}%
\pgfpathlineto{\pgfqpoint{0.000000in}{1.774096in}}%
\pgfpathclose%
\pgfusepath{fill}%
\end{pgfscope}%
\begin{pgfscope}%
\pgfsetbuttcap%
\pgfsetmiterjoin%
\definecolor{currentfill}{rgb}{0.917647,0.917647,0.949020}%
\pgfsetfillcolor{currentfill}%
\pgfsetlinewidth{0.000000pt}%
\definecolor{currentstroke}{rgb}{0.000000,0.000000,0.000000}%
\pgfsetstrokecolor{currentstroke}%
\pgfsetstrokeopacity{0.000000}%
\pgfsetdash{}{0pt}%
\pgfpathmoveto{\pgfqpoint{0.418102in}{0.331635in}}%
\pgfpathlineto{\pgfqpoint{3.236283in}{0.331635in}}%
\pgfpathlineto{\pgfqpoint{3.236283in}{1.464135in}}%
\pgfpathlineto{\pgfqpoint{0.418102in}{1.464135in}}%
\pgfpathclose%
\pgfusepath{fill}%
\end{pgfscope}%
\begin{pgfscope}%
\pgfpathrectangle{\pgfqpoint{0.418102in}{0.331635in}}{\pgfqpoint{2.818182in}{1.132500in}}%
\pgfusepath{clip}%
\pgfsetroundcap%
\pgfsetroundjoin%
\pgfsetlinewidth{0.803000pt}%
\definecolor{currentstroke}{rgb}{1.000000,1.000000,1.000000}%
\pgfsetstrokecolor{currentstroke}%
\pgfsetdash{}{0pt}%
\pgfpathmoveto{\pgfqpoint{0.546201in}{0.331635in}}%
\pgfpathlineto{\pgfqpoint{0.546201in}{1.464135in}}%
\pgfusepath{stroke}%
\end{pgfscope}%
\begin{pgfscope}%
\definecolor{textcolor}{rgb}{0.150000,0.150000,0.150000}%
\pgfsetstrokecolor{textcolor}%
\pgfsetfillcolor{textcolor}%
\pgftext[x=0.546201in,y=0.234413in,,top]{\color{textcolor}\rmfamily\fontsize{10.000000}{12.000000}\selectfont 0}%
\end{pgfscope}%
\begin{pgfscope}%
\pgfpathrectangle{\pgfqpoint{0.418102in}{0.331635in}}{\pgfqpoint{2.818182in}{1.132500in}}%
\pgfusepath{clip}%
\pgfsetroundcap%
\pgfsetroundjoin%
\pgfsetlinewidth{0.803000pt}%
\definecolor{currentstroke}{rgb}{1.000000,1.000000,1.000000}%
\pgfsetstrokecolor{currentstroke}%
\pgfsetdash{}{0pt}%
\pgfpathmoveto{\pgfqpoint{1.171075in}{0.331635in}}%
\pgfpathlineto{\pgfqpoint{1.171075in}{1.464135in}}%
\pgfusepath{stroke}%
\end{pgfscope}%
\begin{pgfscope}%
\definecolor{textcolor}{rgb}{0.150000,0.150000,0.150000}%
\pgfsetstrokecolor{textcolor}%
\pgfsetfillcolor{textcolor}%
\pgftext[x=1.171075in,y=0.234413in,,top]{\color{textcolor}\rmfamily\fontsize{10.000000}{12.000000}\selectfont 5}%
\end{pgfscope}%
\begin{pgfscope}%
\pgfpathrectangle{\pgfqpoint{0.418102in}{0.331635in}}{\pgfqpoint{2.818182in}{1.132500in}}%
\pgfusepath{clip}%
\pgfsetroundcap%
\pgfsetroundjoin%
\pgfsetlinewidth{0.803000pt}%
\definecolor{currentstroke}{rgb}{1.000000,1.000000,1.000000}%
\pgfsetstrokecolor{currentstroke}%
\pgfsetdash{}{0pt}%
\pgfpathmoveto{\pgfqpoint{1.795949in}{0.331635in}}%
\pgfpathlineto{\pgfqpoint{1.795949in}{1.464135in}}%
\pgfusepath{stroke}%
\end{pgfscope}%
\begin{pgfscope}%
\definecolor{textcolor}{rgb}{0.150000,0.150000,0.150000}%
\pgfsetstrokecolor{textcolor}%
\pgfsetfillcolor{textcolor}%
\pgftext[x=1.795949in,y=0.234413in,,top]{\color{textcolor}\rmfamily\fontsize{10.000000}{12.000000}\selectfont 10}%
\end{pgfscope}%
\begin{pgfscope}%
\pgfpathrectangle{\pgfqpoint{0.418102in}{0.331635in}}{\pgfqpoint{2.818182in}{1.132500in}}%
\pgfusepath{clip}%
\pgfsetroundcap%
\pgfsetroundjoin%
\pgfsetlinewidth{0.803000pt}%
\definecolor{currentstroke}{rgb}{1.000000,1.000000,1.000000}%
\pgfsetstrokecolor{currentstroke}%
\pgfsetdash{}{0pt}%
\pgfpathmoveto{\pgfqpoint{2.420823in}{0.331635in}}%
\pgfpathlineto{\pgfqpoint{2.420823in}{1.464135in}}%
\pgfusepath{stroke}%
\end{pgfscope}%
\begin{pgfscope}%
\definecolor{textcolor}{rgb}{0.150000,0.150000,0.150000}%
\pgfsetstrokecolor{textcolor}%
\pgfsetfillcolor{textcolor}%
\pgftext[x=2.420823in,y=0.234413in,,top]{\color{textcolor}\rmfamily\fontsize{10.000000}{12.000000}\selectfont 15}%
\end{pgfscope}%
\begin{pgfscope}%
\pgfpathrectangle{\pgfqpoint{0.418102in}{0.331635in}}{\pgfqpoint{2.818182in}{1.132500in}}%
\pgfusepath{clip}%
\pgfsetroundcap%
\pgfsetroundjoin%
\pgfsetlinewidth{0.803000pt}%
\definecolor{currentstroke}{rgb}{1.000000,1.000000,1.000000}%
\pgfsetstrokecolor{currentstroke}%
\pgfsetdash{}{0pt}%
\pgfpathmoveto{\pgfqpoint{3.045697in}{0.331635in}}%
\pgfpathlineto{\pgfqpoint{3.045697in}{1.464135in}}%
\pgfusepath{stroke}%
\end{pgfscope}%
\begin{pgfscope}%
\definecolor{textcolor}{rgb}{0.150000,0.150000,0.150000}%
\pgfsetstrokecolor{textcolor}%
\pgfsetfillcolor{textcolor}%
\pgftext[x=3.045697in,y=0.234413in,,top]{\color{textcolor}\rmfamily\fontsize{10.000000}{12.000000}\selectfont 20}%
\end{pgfscope}%
\begin{pgfscope}%
\pgfpathrectangle{\pgfqpoint{0.418102in}{0.331635in}}{\pgfqpoint{2.818182in}{1.132500in}}%
\pgfusepath{clip}%
\pgfsetroundcap%
\pgfsetroundjoin%
\pgfsetlinewidth{0.803000pt}%
\definecolor{currentstroke}{rgb}{1.000000,1.000000,1.000000}%
\pgfsetstrokecolor{currentstroke}%
\pgfsetdash{}{0pt}%
\pgfpathmoveto{\pgfqpoint{0.418102in}{0.456180in}}%
\pgfpathlineto{\pgfqpoint{3.236283in}{0.456180in}}%
\pgfusepath{stroke}%
\end{pgfscope}%
\begin{pgfscope}%
\definecolor{textcolor}{rgb}{0.150000,0.150000,0.150000}%
\pgfsetstrokecolor{textcolor}%
\pgfsetfillcolor{textcolor}%
\pgftext[x=0.100000in,y=0.403418in,left,base]{\color{textcolor}\rmfamily\fontsize{10.000000}{12.000000}\selectfont 0.0}%
\end{pgfscope}%
\begin{pgfscope}%
\pgfpathrectangle{\pgfqpoint{0.418102in}{0.331635in}}{\pgfqpoint{2.818182in}{1.132500in}}%
\pgfusepath{clip}%
\pgfsetroundcap%
\pgfsetroundjoin%
\pgfsetlinewidth{0.803000pt}%
\definecolor{currentstroke}{rgb}{1.000000,1.000000,1.000000}%
\pgfsetstrokecolor{currentstroke}%
\pgfsetdash{}{0pt}%
\pgfpathmoveto{\pgfqpoint{0.418102in}{0.934419in}}%
\pgfpathlineto{\pgfqpoint{3.236283in}{0.934419in}}%
\pgfusepath{stroke}%
\end{pgfscope}%
\begin{pgfscope}%
\definecolor{textcolor}{rgb}{0.150000,0.150000,0.150000}%
\pgfsetstrokecolor{textcolor}%
\pgfsetfillcolor{textcolor}%
\pgftext[x=0.100000in,y=0.881657in,left,base]{\color{textcolor}\rmfamily\fontsize{10.000000}{12.000000}\selectfont 0.5}%
\end{pgfscope}%
\begin{pgfscope}%
\pgfpathrectangle{\pgfqpoint{0.418102in}{0.331635in}}{\pgfqpoint{2.818182in}{1.132500in}}%
\pgfusepath{clip}%
\pgfsetroundcap%
\pgfsetroundjoin%
\pgfsetlinewidth{0.803000pt}%
\definecolor{currentstroke}{rgb}{1.000000,1.000000,1.000000}%
\pgfsetstrokecolor{currentstroke}%
\pgfsetdash{}{0pt}%
\pgfpathmoveto{\pgfqpoint{0.418102in}{1.412658in}}%
\pgfpathlineto{\pgfqpoint{3.236283in}{1.412658in}}%
\pgfusepath{stroke}%
\end{pgfscope}%
\begin{pgfscope}%
\definecolor{textcolor}{rgb}{0.150000,0.150000,0.150000}%
\pgfsetstrokecolor{textcolor}%
\pgfsetfillcolor{textcolor}%
\pgftext[x=0.100000in,y=1.359896in,left,base]{\color{textcolor}\rmfamily\fontsize{10.000000}{12.000000}\selectfont 1.0}%
\end{pgfscope}%
\begin{pgfscope}%
\pgfpathrectangle{\pgfqpoint{0.418102in}{0.331635in}}{\pgfqpoint{2.818182in}{1.132500in}}%
\pgfusepath{clip}%
\pgfsetbuttcap%
\pgfsetroundjoin%
\definecolor{currentfill}{rgb}{0.121569,0.466667,0.705882}%
\pgfsetfillcolor{currentfill}%
\pgfsetfillopacity{0.250000}%
\pgfsetlinewidth{1.003750pt}%
\definecolor{currentstroke}{rgb}{1.000000,1.000000,1.000000}%
\pgfsetstrokecolor{currentstroke}%
\pgfsetstrokeopacity{0.250000}%
\pgfsetdash{}{0pt}%
\pgfpathmoveto{\pgfqpoint{0.608688in}{0.504455in}}%
\pgfpathlineto{\pgfqpoint{0.608688in}{0.407905in}}%
\pgfpathlineto{\pgfqpoint{0.796150in}{0.407896in}}%
\pgfpathlineto{\pgfqpoint{0.921125in}{0.407896in}}%
\pgfpathlineto{\pgfqpoint{1.046100in}{0.407896in}}%
\pgfpathlineto{\pgfqpoint{1.171075in}{0.407890in}}%
\pgfpathlineto{\pgfqpoint{1.296050in}{0.407889in}}%
\pgfpathlineto{\pgfqpoint{1.421024in}{0.407886in}}%
\pgfpathlineto{\pgfqpoint{1.545999in}{0.407861in}}%
\pgfpathlineto{\pgfqpoint{1.670974in}{0.407855in}}%
\pgfpathlineto{\pgfqpoint{1.795949in}{0.407791in}}%
\pgfpathlineto{\pgfqpoint{1.920924in}{0.407716in}}%
\pgfpathlineto{\pgfqpoint{2.045898in}{0.407671in}}%
\pgfpathlineto{\pgfqpoint{2.170873in}{0.407665in}}%
\pgfpathlineto{\pgfqpoint{2.295848in}{0.407548in}}%
\pgfpathlineto{\pgfqpoint{2.420823in}{0.407546in}}%
\pgfpathlineto{\pgfqpoint{2.545798in}{0.407267in}}%
\pgfpathlineto{\pgfqpoint{2.670772in}{0.407168in}}%
\pgfpathlineto{\pgfqpoint{2.795747in}{0.407042in}}%
\pgfpathlineto{\pgfqpoint{2.920722in}{0.406795in}}%
\pgfpathlineto{\pgfqpoint{3.108184in}{0.406794in}}%
\pgfpathlineto{\pgfqpoint{3.108184in}{0.505566in}}%
\pgfpathlineto{\pgfqpoint{3.108184in}{0.505566in}}%
\pgfpathlineto{\pgfqpoint{2.920722in}{0.505565in}}%
\pgfpathlineto{\pgfqpoint{2.795747in}{0.505318in}}%
\pgfpathlineto{\pgfqpoint{2.670772in}{0.505192in}}%
\pgfpathlineto{\pgfqpoint{2.545798in}{0.505093in}}%
\pgfpathlineto{\pgfqpoint{2.420823in}{0.504814in}}%
\pgfpathlineto{\pgfqpoint{2.295848in}{0.504812in}}%
\pgfpathlineto{\pgfqpoint{2.170873in}{0.504695in}}%
\pgfpathlineto{\pgfqpoint{2.045898in}{0.504689in}}%
\pgfpathlineto{\pgfqpoint{1.920924in}{0.504644in}}%
\pgfpathlineto{\pgfqpoint{1.795949in}{0.504569in}}%
\pgfpathlineto{\pgfqpoint{1.670974in}{0.504505in}}%
\pgfpathlineto{\pgfqpoint{1.545999in}{0.504499in}}%
\pgfpathlineto{\pgfqpoint{1.421024in}{0.504474in}}%
\pgfpathlineto{\pgfqpoint{1.296050in}{0.504471in}}%
\pgfpathlineto{\pgfqpoint{1.171075in}{0.504470in}}%
\pgfpathlineto{\pgfqpoint{1.046100in}{0.504464in}}%
\pgfpathlineto{\pgfqpoint{0.921125in}{0.504464in}}%
\pgfpathlineto{\pgfqpoint{0.796150in}{0.504464in}}%
\pgfpathlineto{\pgfqpoint{0.608688in}{0.504455in}}%
\pgfpathclose%
\pgfusepath{stroke,fill}%
\end{pgfscope}%
\begin{pgfscope}%
\pgfpathrectangle{\pgfqpoint{0.418102in}{0.331635in}}{\pgfqpoint{2.818182in}{1.132500in}}%
\pgfusepath{clip}%
\pgfsetbuttcap%
\pgfsetroundjoin%
\pgfsetlinewidth{1.505625pt}%
\definecolor{currentstroke}{rgb}{0.000000,0.000000,0.000000}%
\pgfsetstrokecolor{currentstroke}%
\pgfsetdash{}{0pt}%
\pgfpathmoveto{\pgfqpoint{0.546201in}{0.456180in}}%
\pgfpathlineto{\pgfqpoint{0.546201in}{1.412658in}}%
\pgfusepath{stroke}%
\end{pgfscope}%
\begin{pgfscope}%
\pgfpathrectangle{\pgfqpoint{0.418102in}{0.331635in}}{\pgfqpoint{2.818182in}{1.132500in}}%
\pgfusepath{clip}%
\pgfsetbuttcap%
\pgfsetroundjoin%
\pgfsetlinewidth{1.505625pt}%
\definecolor{currentstroke}{rgb}{0.000000,0.000000,0.000000}%
\pgfsetstrokecolor{currentstroke}%
\pgfsetdash{}{0pt}%
\pgfpathmoveto{\pgfqpoint{0.671176in}{0.456180in}}%
\pgfpathlineto{\pgfqpoint{0.671176in}{0.443061in}}%
\pgfusepath{stroke}%
\end{pgfscope}%
\begin{pgfscope}%
\pgfpathrectangle{\pgfqpoint{0.418102in}{0.331635in}}{\pgfqpoint{2.818182in}{1.132500in}}%
\pgfusepath{clip}%
\pgfsetbuttcap%
\pgfsetroundjoin%
\pgfsetlinewidth{1.505625pt}%
\definecolor{currentstroke}{rgb}{0.000000,0.000000,0.000000}%
\pgfsetstrokecolor{currentstroke}%
\pgfsetdash{}{0pt}%
\pgfpathmoveto{\pgfqpoint{0.796150in}{0.456180in}}%
\pgfpathlineto{\pgfqpoint{0.796150in}{0.457912in}}%
\pgfusepath{stroke}%
\end{pgfscope}%
\begin{pgfscope}%
\pgfpathrectangle{\pgfqpoint{0.418102in}{0.331635in}}{\pgfqpoint{2.818182in}{1.132500in}}%
\pgfusepath{clip}%
\pgfsetbuttcap%
\pgfsetroundjoin%
\pgfsetlinewidth{1.505625pt}%
\definecolor{currentstroke}{rgb}{0.000000,0.000000,0.000000}%
\pgfsetstrokecolor{currentstroke}%
\pgfsetdash{}{0pt}%
\pgfpathmoveto{\pgfqpoint{0.921125in}{0.456180in}}%
\pgfpathlineto{\pgfqpoint{0.921125in}{0.457136in}}%
\pgfusepath{stroke}%
\end{pgfscope}%
\begin{pgfscope}%
\pgfpathrectangle{\pgfqpoint{0.418102in}{0.331635in}}{\pgfqpoint{2.818182in}{1.132500in}}%
\pgfusepath{clip}%
\pgfsetbuttcap%
\pgfsetroundjoin%
\pgfsetlinewidth{1.505625pt}%
\definecolor{currentstroke}{rgb}{0.000000,0.000000,0.000000}%
\pgfsetstrokecolor{currentstroke}%
\pgfsetdash{}{0pt}%
\pgfpathmoveto{\pgfqpoint{1.046100in}{0.456180in}}%
\pgfpathlineto{\pgfqpoint{1.046100in}{0.445386in}}%
\pgfusepath{stroke}%
\end{pgfscope}%
\begin{pgfscope}%
\pgfpathrectangle{\pgfqpoint{0.418102in}{0.331635in}}{\pgfqpoint{2.818182in}{1.132500in}}%
\pgfusepath{clip}%
\pgfsetbuttcap%
\pgfsetroundjoin%
\pgfsetlinewidth{1.505625pt}%
\definecolor{currentstroke}{rgb}{0.000000,0.000000,0.000000}%
\pgfsetstrokecolor{currentstroke}%
\pgfsetdash{}{0pt}%
\pgfpathmoveto{\pgfqpoint{1.171075in}{0.456180in}}%
\pgfpathlineto{\pgfqpoint{1.171075in}{0.458094in}}%
\pgfusepath{stroke}%
\end{pgfscope}%
\begin{pgfscope}%
\pgfpathrectangle{\pgfqpoint{0.418102in}{0.331635in}}{\pgfqpoint{2.818182in}{1.132500in}}%
\pgfusepath{clip}%
\pgfsetbuttcap%
\pgfsetroundjoin%
\pgfsetlinewidth{1.505625pt}%
\definecolor{currentstroke}{rgb}{0.000000,0.000000,0.000000}%
\pgfsetstrokecolor{currentstroke}%
\pgfsetdash{}{0pt}%
\pgfpathmoveto{\pgfqpoint{1.296050in}{0.456180in}}%
\pgfpathlineto{\pgfqpoint{1.296050in}{0.447725in}}%
\pgfusepath{stroke}%
\end{pgfscope}%
\begin{pgfscope}%
\pgfpathrectangle{\pgfqpoint{0.418102in}{0.331635in}}{\pgfqpoint{2.818182in}{1.132500in}}%
\pgfusepath{clip}%
\pgfsetbuttcap%
\pgfsetroundjoin%
\pgfsetlinewidth{1.505625pt}%
\definecolor{currentstroke}{rgb}{0.000000,0.000000,0.000000}%
\pgfsetstrokecolor{currentstroke}%
\pgfsetdash{}{0pt}%
\pgfpathmoveto{\pgfqpoint{1.421024in}{0.456180in}}%
\pgfpathlineto{\pgfqpoint{1.421024in}{0.477812in}}%
\pgfusepath{stroke}%
\end{pgfscope}%
\begin{pgfscope}%
\pgfpathrectangle{\pgfqpoint{0.418102in}{0.331635in}}{\pgfqpoint{2.818182in}{1.132500in}}%
\pgfusepath{clip}%
\pgfsetbuttcap%
\pgfsetroundjoin%
\pgfsetlinewidth{1.505625pt}%
\definecolor{currentstroke}{rgb}{0.000000,0.000000,0.000000}%
\pgfsetstrokecolor{currentstroke}%
\pgfsetdash{}{0pt}%
\pgfpathmoveto{\pgfqpoint{1.545999in}{0.456180in}}%
\pgfpathlineto{\pgfqpoint{1.545999in}{0.467110in}}%
\pgfusepath{stroke}%
\end{pgfscope}%
\begin{pgfscope}%
\pgfpathrectangle{\pgfqpoint{0.418102in}{0.331635in}}{\pgfqpoint{2.818182in}{1.132500in}}%
\pgfusepath{clip}%
\pgfsetbuttcap%
\pgfsetroundjoin%
\pgfsetlinewidth{1.505625pt}%
\definecolor{currentstroke}{rgb}{0.000000,0.000000,0.000000}%
\pgfsetstrokecolor{currentstroke}%
\pgfsetdash{}{0pt}%
\pgfpathmoveto{\pgfqpoint{1.670974in}{0.456180in}}%
\pgfpathlineto{\pgfqpoint{1.670974in}{0.421349in}}%
\pgfusepath{stroke}%
\end{pgfscope}%
\begin{pgfscope}%
\pgfpathrectangle{\pgfqpoint{0.418102in}{0.331635in}}{\pgfqpoint{2.818182in}{1.132500in}}%
\pgfusepath{clip}%
\pgfsetbuttcap%
\pgfsetroundjoin%
\pgfsetlinewidth{1.505625pt}%
\definecolor{currentstroke}{rgb}{0.000000,0.000000,0.000000}%
\pgfsetstrokecolor{currentstroke}%
\pgfsetdash{}{0pt}%
\pgfpathmoveto{\pgfqpoint{1.795949in}{0.456180in}}%
\pgfpathlineto{\pgfqpoint{1.795949in}{0.493934in}}%
\pgfusepath{stroke}%
\end{pgfscope}%
\begin{pgfscope}%
\pgfpathrectangle{\pgfqpoint{0.418102in}{0.331635in}}{\pgfqpoint{2.818182in}{1.132500in}}%
\pgfusepath{clip}%
\pgfsetbuttcap%
\pgfsetroundjoin%
\pgfsetlinewidth{1.505625pt}%
\definecolor{currentstroke}{rgb}{0.000000,0.000000,0.000000}%
\pgfsetstrokecolor{currentstroke}%
\pgfsetdash{}{0pt}%
\pgfpathmoveto{\pgfqpoint{1.920924in}{0.456180in}}%
\pgfpathlineto{\pgfqpoint{1.920924in}{0.485433in}}%
\pgfusepath{stroke}%
\end{pgfscope}%
\begin{pgfscope}%
\pgfpathrectangle{\pgfqpoint{0.418102in}{0.331635in}}{\pgfqpoint{2.818182in}{1.132500in}}%
\pgfusepath{clip}%
\pgfsetbuttcap%
\pgfsetroundjoin%
\pgfsetlinewidth{1.505625pt}%
\definecolor{currentstroke}{rgb}{0.000000,0.000000,0.000000}%
\pgfsetstrokecolor{currentstroke}%
\pgfsetdash{}{0pt}%
\pgfpathmoveto{\pgfqpoint{2.045898in}{0.456180in}}%
\pgfpathlineto{\pgfqpoint{2.045898in}{0.466527in}}%
\pgfusepath{stroke}%
\end{pgfscope}%
\begin{pgfscope}%
\pgfpathrectangle{\pgfqpoint{0.418102in}{0.331635in}}{\pgfqpoint{2.818182in}{1.132500in}}%
\pgfusepath{clip}%
\pgfsetbuttcap%
\pgfsetroundjoin%
\pgfsetlinewidth{1.505625pt}%
\definecolor{currentstroke}{rgb}{0.000000,0.000000,0.000000}%
\pgfsetstrokecolor{currentstroke}%
\pgfsetdash{}{0pt}%
\pgfpathmoveto{\pgfqpoint{2.170873in}{0.456180in}}%
\pgfpathlineto{\pgfqpoint{2.170873in}{0.408852in}}%
\pgfusepath{stroke}%
\end{pgfscope}%
\begin{pgfscope}%
\pgfpathrectangle{\pgfqpoint{0.418102in}{0.331635in}}{\pgfqpoint{2.818182in}{1.132500in}}%
\pgfusepath{clip}%
\pgfsetbuttcap%
\pgfsetroundjoin%
\pgfsetlinewidth{1.505625pt}%
\definecolor{currentstroke}{rgb}{0.000000,0.000000,0.000000}%
\pgfsetstrokecolor{currentstroke}%
\pgfsetdash{}{0pt}%
\pgfpathmoveto{\pgfqpoint{2.295848in}{0.456180in}}%
\pgfpathlineto{\pgfqpoint{2.295848in}{0.461296in}}%
\pgfusepath{stroke}%
\end{pgfscope}%
\begin{pgfscope}%
\pgfpathrectangle{\pgfqpoint{0.418102in}{0.331635in}}{\pgfqpoint{2.818182in}{1.132500in}}%
\pgfusepath{clip}%
\pgfsetbuttcap%
\pgfsetroundjoin%
\pgfsetlinewidth{1.505625pt}%
\definecolor{currentstroke}{rgb}{0.000000,0.000000,0.000000}%
\pgfsetstrokecolor{currentstroke}%
\pgfsetdash{}{0pt}%
\pgfpathmoveto{\pgfqpoint{2.420823in}{0.456180in}}%
\pgfpathlineto{\pgfqpoint{2.420823in}{0.383112in}}%
\pgfusepath{stroke}%
\end{pgfscope}%
\begin{pgfscope}%
\pgfpathrectangle{\pgfqpoint{0.418102in}{0.331635in}}{\pgfqpoint{2.818182in}{1.132500in}}%
\pgfusepath{clip}%
\pgfsetbuttcap%
\pgfsetroundjoin%
\pgfsetlinewidth{1.505625pt}%
\definecolor{currentstroke}{rgb}{0.000000,0.000000,0.000000}%
\pgfsetstrokecolor{currentstroke}%
\pgfsetdash{}{0pt}%
\pgfpathmoveto{\pgfqpoint{2.545798in}{0.456180in}}%
\pgfpathlineto{\pgfqpoint{2.545798in}{0.412363in}}%
\pgfusepath{stroke}%
\end{pgfscope}%
\begin{pgfscope}%
\pgfpathrectangle{\pgfqpoint{0.418102in}{0.331635in}}{\pgfqpoint{2.818182in}{1.132500in}}%
\pgfusepath{clip}%
\pgfsetbuttcap%
\pgfsetroundjoin%
\pgfsetlinewidth{1.505625pt}%
\definecolor{currentstroke}{rgb}{0.000000,0.000000,0.000000}%
\pgfsetstrokecolor{currentstroke}%
\pgfsetdash{}{0pt}%
\pgfpathmoveto{\pgfqpoint{2.670772in}{0.456180in}}%
\pgfpathlineto{\pgfqpoint{2.670772in}{0.505278in}}%
\pgfusepath{stroke}%
\end{pgfscope}%
\begin{pgfscope}%
\pgfpathrectangle{\pgfqpoint{0.418102in}{0.331635in}}{\pgfqpoint{2.818182in}{1.132500in}}%
\pgfusepath{clip}%
\pgfsetbuttcap%
\pgfsetroundjoin%
\pgfsetlinewidth{1.505625pt}%
\definecolor{currentstroke}{rgb}{0.000000,0.000000,0.000000}%
\pgfsetstrokecolor{currentstroke}%
\pgfsetdash{}{0pt}%
\pgfpathmoveto{\pgfqpoint{2.795747in}{0.456180in}}%
\pgfpathlineto{\pgfqpoint{2.795747in}{0.525411in}}%
\pgfusepath{stroke}%
\end{pgfscope}%
\begin{pgfscope}%
\pgfpathrectangle{\pgfqpoint{0.418102in}{0.331635in}}{\pgfqpoint{2.818182in}{1.132500in}}%
\pgfusepath{clip}%
\pgfsetbuttcap%
\pgfsetroundjoin%
\pgfsetlinewidth{1.505625pt}%
\definecolor{currentstroke}{rgb}{0.000000,0.000000,0.000000}%
\pgfsetstrokecolor{currentstroke}%
\pgfsetdash{}{0pt}%
\pgfpathmoveto{\pgfqpoint{2.920722in}{0.456180in}}%
\pgfpathlineto{\pgfqpoint{2.920722in}{0.459955in}}%
\pgfusepath{stroke}%
\end{pgfscope}%
\begin{pgfscope}%
\pgfpathrectangle{\pgfqpoint{0.418102in}{0.331635in}}{\pgfqpoint{2.818182in}{1.132500in}}%
\pgfusepath{clip}%
\pgfsetbuttcap%
\pgfsetroundjoin%
\pgfsetlinewidth{1.505625pt}%
\definecolor{currentstroke}{rgb}{0.000000,0.000000,0.000000}%
\pgfsetstrokecolor{currentstroke}%
\pgfsetdash{}{0pt}%
\pgfpathmoveto{\pgfqpoint{3.045697in}{0.456180in}}%
\pgfpathlineto{\pgfqpoint{3.045697in}{0.459635in}}%
\pgfusepath{stroke}%
\end{pgfscope}%
\begin{pgfscope}%
\pgfpathrectangle{\pgfqpoint{0.418102in}{0.331635in}}{\pgfqpoint{2.818182in}{1.132500in}}%
\pgfusepath{clip}%
\pgfsetroundcap%
\pgfsetroundjoin%
\pgfsetlinewidth{1.505625pt}%
\definecolor{currentstroke}{rgb}{0.839216,0.152941,0.156863}%
\pgfsetstrokecolor{currentstroke}%
\pgfsetdash{}{0pt}%
\pgfpathmoveto{\pgfqpoint{0.418102in}{0.456180in}}%
\pgfpathlineto{\pgfqpoint{3.236283in}{0.456180in}}%
\pgfusepath{stroke}%
\end{pgfscope}%
\begin{pgfscope}%
\pgfpathrectangle{\pgfqpoint{0.418102in}{0.331635in}}{\pgfqpoint{2.818182in}{1.132500in}}%
\pgfusepath{clip}%
\pgfsetbuttcap%
\pgfsetroundjoin%
\definecolor{currentfill}{rgb}{0.839216,0.152941,0.156863}%
\pgfsetfillcolor{currentfill}%
\pgfsetlinewidth{1.003750pt}%
\definecolor{currentstroke}{rgb}{0.839216,0.152941,0.156863}%
\pgfsetstrokecolor{currentstroke}%
\pgfsetdash{}{0pt}%
\pgfsys@defobject{currentmarker}{\pgfqpoint{-0.034722in}{-0.034722in}}{\pgfqpoint{0.034722in}{0.034722in}}{%
\pgfpathmoveto{\pgfqpoint{0.000000in}{-0.034722in}}%
\pgfpathcurveto{\pgfqpoint{0.009208in}{-0.034722in}}{\pgfqpoint{0.018041in}{-0.031064in}}{\pgfqpoint{0.024552in}{-0.024552in}}%
\pgfpathcurveto{\pgfqpoint{0.031064in}{-0.018041in}}{\pgfqpoint{0.034722in}{-0.009208in}}{\pgfqpoint{0.034722in}{0.000000in}}%
\pgfpathcurveto{\pgfqpoint{0.034722in}{0.009208in}}{\pgfqpoint{0.031064in}{0.018041in}}{\pgfqpoint{0.024552in}{0.024552in}}%
\pgfpathcurveto{\pgfqpoint{0.018041in}{0.031064in}}{\pgfqpoint{0.009208in}{0.034722in}}{\pgfqpoint{0.000000in}{0.034722in}}%
\pgfpathcurveto{\pgfqpoint{-0.009208in}{0.034722in}}{\pgfqpoint{-0.018041in}{0.031064in}}{\pgfqpoint{-0.024552in}{0.024552in}}%
\pgfpathcurveto{\pgfqpoint{-0.031064in}{0.018041in}}{\pgfqpoint{-0.034722in}{0.009208in}}{\pgfqpoint{-0.034722in}{0.000000in}}%
\pgfpathcurveto{\pgfqpoint{-0.034722in}{-0.009208in}}{\pgfqpoint{-0.031064in}{-0.018041in}}{\pgfqpoint{-0.024552in}{-0.024552in}}%
\pgfpathcurveto{\pgfqpoint{-0.018041in}{-0.031064in}}{\pgfqpoint{-0.009208in}{-0.034722in}}{\pgfqpoint{0.000000in}{-0.034722in}}%
\pgfpathclose%
\pgfusepath{stroke,fill}%
}%
\begin{pgfscope}%
\pgfsys@transformshift{0.546201in}{1.412658in}%
\pgfsys@useobject{currentmarker}{}%
\end{pgfscope}%
\begin{pgfscope}%
\pgfsys@transformshift{0.671176in}{0.443061in}%
\pgfsys@useobject{currentmarker}{}%
\end{pgfscope}%
\begin{pgfscope}%
\pgfsys@transformshift{0.796150in}{0.457912in}%
\pgfsys@useobject{currentmarker}{}%
\end{pgfscope}%
\begin{pgfscope}%
\pgfsys@transformshift{0.921125in}{0.457136in}%
\pgfsys@useobject{currentmarker}{}%
\end{pgfscope}%
\begin{pgfscope}%
\pgfsys@transformshift{1.046100in}{0.445386in}%
\pgfsys@useobject{currentmarker}{}%
\end{pgfscope}%
\begin{pgfscope}%
\pgfsys@transformshift{1.171075in}{0.458094in}%
\pgfsys@useobject{currentmarker}{}%
\end{pgfscope}%
\begin{pgfscope}%
\pgfsys@transformshift{1.296050in}{0.447725in}%
\pgfsys@useobject{currentmarker}{}%
\end{pgfscope}%
\begin{pgfscope}%
\pgfsys@transformshift{1.421024in}{0.477812in}%
\pgfsys@useobject{currentmarker}{}%
\end{pgfscope}%
\begin{pgfscope}%
\pgfsys@transformshift{1.545999in}{0.467110in}%
\pgfsys@useobject{currentmarker}{}%
\end{pgfscope}%
\begin{pgfscope}%
\pgfsys@transformshift{1.670974in}{0.421349in}%
\pgfsys@useobject{currentmarker}{}%
\end{pgfscope}%
\begin{pgfscope}%
\pgfsys@transformshift{1.795949in}{0.493934in}%
\pgfsys@useobject{currentmarker}{}%
\end{pgfscope}%
\begin{pgfscope}%
\pgfsys@transformshift{1.920924in}{0.485433in}%
\pgfsys@useobject{currentmarker}{}%
\end{pgfscope}%
\begin{pgfscope}%
\pgfsys@transformshift{2.045898in}{0.466527in}%
\pgfsys@useobject{currentmarker}{}%
\end{pgfscope}%
\begin{pgfscope}%
\pgfsys@transformshift{2.170873in}{0.408852in}%
\pgfsys@useobject{currentmarker}{}%
\end{pgfscope}%
\begin{pgfscope}%
\pgfsys@transformshift{2.295848in}{0.461296in}%
\pgfsys@useobject{currentmarker}{}%
\end{pgfscope}%
\begin{pgfscope}%
\pgfsys@transformshift{2.420823in}{0.383112in}%
\pgfsys@useobject{currentmarker}{}%
\end{pgfscope}%
\begin{pgfscope}%
\pgfsys@transformshift{2.545798in}{0.412363in}%
\pgfsys@useobject{currentmarker}{}%
\end{pgfscope}%
\begin{pgfscope}%
\pgfsys@transformshift{2.670772in}{0.505278in}%
\pgfsys@useobject{currentmarker}{}%
\end{pgfscope}%
\begin{pgfscope}%
\pgfsys@transformshift{2.795747in}{0.525411in}%
\pgfsys@useobject{currentmarker}{}%
\end{pgfscope}%
\begin{pgfscope}%
\pgfsys@transformshift{2.920722in}{0.459955in}%
\pgfsys@useobject{currentmarker}{}%
\end{pgfscope}%
\begin{pgfscope}%
\pgfsys@transformshift{3.045697in}{0.459635in}%
\pgfsys@useobject{currentmarker}{}%
\end{pgfscope}%
\end{pgfscope}%
\begin{pgfscope}%
\pgfsetrectcap%
\pgfsetmiterjoin%
\pgfsetlinewidth{0.803000pt}%
\definecolor{currentstroke}{rgb}{1.000000,1.000000,1.000000}%
\pgfsetstrokecolor{currentstroke}%
\pgfsetdash{}{0pt}%
\pgfpathmoveto{\pgfqpoint{0.418102in}{0.331635in}}%
\pgfpathlineto{\pgfqpoint{0.418102in}{1.464135in}}%
\pgfusepath{stroke}%
\end{pgfscope}%
\begin{pgfscope}%
\pgfsetrectcap%
\pgfsetmiterjoin%
\pgfsetlinewidth{0.803000pt}%
\definecolor{currentstroke}{rgb}{1.000000,1.000000,1.000000}%
\pgfsetstrokecolor{currentstroke}%
\pgfsetdash{}{0pt}%
\pgfpathmoveto{\pgfqpoint{3.236283in}{0.331635in}}%
\pgfpathlineto{\pgfqpoint{3.236283in}{1.464135in}}%
\pgfusepath{stroke}%
\end{pgfscope}%
\begin{pgfscope}%
\pgfsetrectcap%
\pgfsetmiterjoin%
\pgfsetlinewidth{0.803000pt}%
\definecolor{currentstroke}{rgb}{1.000000,1.000000,1.000000}%
\pgfsetstrokecolor{currentstroke}%
\pgfsetdash{}{0pt}%
\pgfpathmoveto{\pgfqpoint{0.418102in}{0.331635in}}%
\pgfpathlineto{\pgfqpoint{3.236283in}{0.331635in}}%
\pgfusepath{stroke}%
\end{pgfscope}%
\begin{pgfscope}%
\pgfsetrectcap%
\pgfsetmiterjoin%
\pgfsetlinewidth{0.803000pt}%
\definecolor{currentstroke}{rgb}{1.000000,1.000000,1.000000}%
\pgfsetstrokecolor{currentstroke}%
\pgfsetdash{}{0pt}%
\pgfpathmoveto{\pgfqpoint{0.418102in}{1.464135in}}%
\pgfpathlineto{\pgfqpoint{3.236283in}{1.464135in}}%
\pgfusepath{stroke}%
\end{pgfscope}%
\begin{pgfscope}%
\definecolor{textcolor}{rgb}{0.150000,0.150000,0.150000}%
\pgfsetstrokecolor{textcolor}%
\pgfsetfillcolor{textcolor}%
\pgftext[x=1.827193in,y=1.547468in,,base]{\color{textcolor}\rmfamily\fontsize{12.000000}{14.400000}\selectfont Autocorrelation INTC}%
\end{pgfscope}%
\begin{pgfscope}%
\pgfsetbuttcap%
\pgfsetmiterjoin%
\definecolor{currentfill}{rgb}{0.917647,0.917647,0.949020}%
\pgfsetfillcolor{currentfill}%
\pgfsetlinewidth{0.000000pt}%
\definecolor{currentstroke}{rgb}{0.000000,0.000000,0.000000}%
\pgfsetstrokecolor{currentstroke}%
\pgfsetstrokeopacity{0.000000}%
\pgfsetdash{}{0pt}%
\pgfpathmoveto{\pgfqpoint{3.799920in}{0.331635in}}%
\pgfpathlineto{\pgfqpoint{6.618102in}{0.331635in}}%
\pgfpathlineto{\pgfqpoint{6.618102in}{1.464135in}}%
\pgfpathlineto{\pgfqpoint{3.799920in}{1.464135in}}%
\pgfpathclose%
\pgfusepath{fill}%
\end{pgfscope}%
\begin{pgfscope}%
\pgfpathrectangle{\pgfqpoint{3.799920in}{0.331635in}}{\pgfqpoint{2.818182in}{1.132500in}}%
\pgfusepath{clip}%
\pgfsetroundcap%
\pgfsetroundjoin%
\pgfsetlinewidth{0.803000pt}%
\definecolor{currentstroke}{rgb}{1.000000,1.000000,1.000000}%
\pgfsetstrokecolor{currentstroke}%
\pgfsetdash{}{0pt}%
\pgfpathmoveto{\pgfqpoint{3.928019in}{0.331635in}}%
\pgfpathlineto{\pgfqpoint{3.928019in}{1.464135in}}%
\pgfusepath{stroke}%
\end{pgfscope}%
\begin{pgfscope}%
\definecolor{textcolor}{rgb}{0.150000,0.150000,0.150000}%
\pgfsetstrokecolor{textcolor}%
\pgfsetfillcolor{textcolor}%
\pgftext[x=3.928019in,y=0.234413in,,top]{\color{textcolor}\rmfamily\fontsize{10.000000}{12.000000}\selectfont 0}%
\end{pgfscope}%
\begin{pgfscope}%
\pgfpathrectangle{\pgfqpoint{3.799920in}{0.331635in}}{\pgfqpoint{2.818182in}{1.132500in}}%
\pgfusepath{clip}%
\pgfsetroundcap%
\pgfsetroundjoin%
\pgfsetlinewidth{0.803000pt}%
\definecolor{currentstroke}{rgb}{1.000000,1.000000,1.000000}%
\pgfsetstrokecolor{currentstroke}%
\pgfsetdash{}{0pt}%
\pgfpathmoveto{\pgfqpoint{4.552893in}{0.331635in}}%
\pgfpathlineto{\pgfqpoint{4.552893in}{1.464135in}}%
\pgfusepath{stroke}%
\end{pgfscope}%
\begin{pgfscope}%
\definecolor{textcolor}{rgb}{0.150000,0.150000,0.150000}%
\pgfsetstrokecolor{textcolor}%
\pgfsetfillcolor{textcolor}%
\pgftext[x=4.552893in,y=0.234413in,,top]{\color{textcolor}\rmfamily\fontsize{10.000000}{12.000000}\selectfont 5}%
\end{pgfscope}%
\begin{pgfscope}%
\pgfpathrectangle{\pgfqpoint{3.799920in}{0.331635in}}{\pgfqpoint{2.818182in}{1.132500in}}%
\pgfusepath{clip}%
\pgfsetroundcap%
\pgfsetroundjoin%
\pgfsetlinewidth{0.803000pt}%
\definecolor{currentstroke}{rgb}{1.000000,1.000000,1.000000}%
\pgfsetstrokecolor{currentstroke}%
\pgfsetdash{}{0pt}%
\pgfpathmoveto{\pgfqpoint{5.177767in}{0.331635in}}%
\pgfpathlineto{\pgfqpoint{5.177767in}{1.464135in}}%
\pgfusepath{stroke}%
\end{pgfscope}%
\begin{pgfscope}%
\definecolor{textcolor}{rgb}{0.150000,0.150000,0.150000}%
\pgfsetstrokecolor{textcolor}%
\pgfsetfillcolor{textcolor}%
\pgftext[x=5.177767in,y=0.234413in,,top]{\color{textcolor}\rmfamily\fontsize{10.000000}{12.000000}\selectfont 10}%
\end{pgfscope}%
\begin{pgfscope}%
\pgfpathrectangle{\pgfqpoint{3.799920in}{0.331635in}}{\pgfqpoint{2.818182in}{1.132500in}}%
\pgfusepath{clip}%
\pgfsetroundcap%
\pgfsetroundjoin%
\pgfsetlinewidth{0.803000pt}%
\definecolor{currentstroke}{rgb}{1.000000,1.000000,1.000000}%
\pgfsetstrokecolor{currentstroke}%
\pgfsetdash{}{0pt}%
\pgfpathmoveto{\pgfqpoint{5.802641in}{0.331635in}}%
\pgfpathlineto{\pgfqpoint{5.802641in}{1.464135in}}%
\pgfusepath{stroke}%
\end{pgfscope}%
\begin{pgfscope}%
\definecolor{textcolor}{rgb}{0.150000,0.150000,0.150000}%
\pgfsetstrokecolor{textcolor}%
\pgfsetfillcolor{textcolor}%
\pgftext[x=5.802641in,y=0.234413in,,top]{\color{textcolor}\rmfamily\fontsize{10.000000}{12.000000}\selectfont 15}%
\end{pgfscope}%
\begin{pgfscope}%
\pgfpathrectangle{\pgfqpoint{3.799920in}{0.331635in}}{\pgfqpoint{2.818182in}{1.132500in}}%
\pgfusepath{clip}%
\pgfsetroundcap%
\pgfsetroundjoin%
\pgfsetlinewidth{0.803000pt}%
\definecolor{currentstroke}{rgb}{1.000000,1.000000,1.000000}%
\pgfsetstrokecolor{currentstroke}%
\pgfsetdash{}{0pt}%
\pgfpathmoveto{\pgfqpoint{6.427515in}{0.331635in}}%
\pgfpathlineto{\pgfqpoint{6.427515in}{1.464135in}}%
\pgfusepath{stroke}%
\end{pgfscope}%
\begin{pgfscope}%
\definecolor{textcolor}{rgb}{0.150000,0.150000,0.150000}%
\pgfsetstrokecolor{textcolor}%
\pgfsetfillcolor{textcolor}%
\pgftext[x=6.427515in,y=0.234413in,,top]{\color{textcolor}\rmfamily\fontsize{10.000000}{12.000000}\selectfont 20}%
\end{pgfscope}%
\begin{pgfscope}%
\pgfpathrectangle{\pgfqpoint{3.799920in}{0.331635in}}{\pgfqpoint{2.818182in}{1.132500in}}%
\pgfusepath{clip}%
\pgfsetroundcap%
\pgfsetroundjoin%
\pgfsetlinewidth{0.803000pt}%
\definecolor{currentstroke}{rgb}{1.000000,1.000000,1.000000}%
\pgfsetstrokecolor{currentstroke}%
\pgfsetdash{}{0pt}%
\pgfpathmoveto{\pgfqpoint{3.799920in}{0.457429in}}%
\pgfpathlineto{\pgfqpoint{6.618102in}{0.457429in}}%
\pgfusepath{stroke}%
\end{pgfscope}%
\begin{pgfscope}%
\definecolor{textcolor}{rgb}{0.150000,0.150000,0.150000}%
\pgfsetstrokecolor{textcolor}%
\pgfsetfillcolor{textcolor}%
\pgftext[x=3.481818in,y=0.404668in,left,base]{\color{textcolor}\rmfamily\fontsize{10.000000}{12.000000}\selectfont 0.0}%
\end{pgfscope}%
\begin{pgfscope}%
\pgfpathrectangle{\pgfqpoint{3.799920in}{0.331635in}}{\pgfqpoint{2.818182in}{1.132500in}}%
\pgfusepath{clip}%
\pgfsetroundcap%
\pgfsetroundjoin%
\pgfsetlinewidth{0.803000pt}%
\definecolor{currentstroke}{rgb}{1.000000,1.000000,1.000000}%
\pgfsetstrokecolor{currentstroke}%
\pgfsetdash{}{0pt}%
\pgfpathmoveto{\pgfqpoint{3.799920in}{0.935043in}}%
\pgfpathlineto{\pgfqpoint{6.618102in}{0.935043in}}%
\pgfusepath{stroke}%
\end{pgfscope}%
\begin{pgfscope}%
\definecolor{textcolor}{rgb}{0.150000,0.150000,0.150000}%
\pgfsetstrokecolor{textcolor}%
\pgfsetfillcolor{textcolor}%
\pgftext[x=3.481818in,y=0.882282in,left,base]{\color{textcolor}\rmfamily\fontsize{10.000000}{12.000000}\selectfont 0.5}%
\end{pgfscope}%
\begin{pgfscope}%
\pgfpathrectangle{\pgfqpoint{3.799920in}{0.331635in}}{\pgfqpoint{2.818182in}{1.132500in}}%
\pgfusepath{clip}%
\pgfsetroundcap%
\pgfsetroundjoin%
\pgfsetlinewidth{0.803000pt}%
\definecolor{currentstroke}{rgb}{1.000000,1.000000,1.000000}%
\pgfsetstrokecolor{currentstroke}%
\pgfsetdash{}{0pt}%
\pgfpathmoveto{\pgfqpoint{3.799920in}{1.412658in}}%
\pgfpathlineto{\pgfqpoint{6.618102in}{1.412658in}}%
\pgfusepath{stroke}%
\end{pgfscope}%
\begin{pgfscope}%
\definecolor{textcolor}{rgb}{0.150000,0.150000,0.150000}%
\pgfsetstrokecolor{textcolor}%
\pgfsetfillcolor{textcolor}%
\pgftext[x=3.481818in,y=1.359896in,left,base]{\color{textcolor}\rmfamily\fontsize{10.000000}{12.000000}\selectfont 1.0}%
\end{pgfscope}%
\begin{pgfscope}%
\pgfpathrectangle{\pgfqpoint{3.799920in}{0.331635in}}{\pgfqpoint{2.818182in}{1.132500in}}%
\pgfusepath{clip}%
\pgfsetbuttcap%
\pgfsetroundjoin%
\definecolor{currentfill}{rgb}{0.121569,0.466667,0.705882}%
\pgfsetfillcolor{currentfill}%
\pgfsetfillopacity{0.250000}%
\pgfsetlinewidth{1.003750pt}%
\definecolor{currentstroke}{rgb}{1.000000,1.000000,1.000000}%
\pgfsetstrokecolor{currentstroke}%
\pgfsetstrokeopacity{0.250000}%
\pgfsetdash{}{0pt}%
\pgfpathmoveto{\pgfqpoint{3.990506in}{0.505641in}}%
\pgfpathlineto{\pgfqpoint{3.990506in}{0.409217in}}%
\pgfpathlineto{\pgfqpoint{4.177969in}{0.409217in}}%
\pgfpathlineto{\pgfqpoint{4.302943in}{0.409217in}}%
\pgfpathlineto{\pgfqpoint{4.427918in}{0.409217in}}%
\pgfpathlineto{\pgfqpoint{4.552893in}{0.409217in}}%
\pgfpathlineto{\pgfqpoint{4.677868in}{0.409217in}}%
\pgfpathlineto{\pgfqpoint{4.802843in}{0.409217in}}%
\pgfpathlineto{\pgfqpoint{4.927817in}{0.409217in}}%
\pgfpathlineto{\pgfqpoint{5.052792in}{0.409217in}}%
\pgfpathlineto{\pgfqpoint{5.177767in}{0.409217in}}%
\pgfpathlineto{\pgfqpoint{5.302742in}{0.409217in}}%
\pgfpathlineto{\pgfqpoint{5.427717in}{0.409217in}}%
\pgfpathlineto{\pgfqpoint{5.552691in}{0.409217in}}%
\pgfpathlineto{\pgfqpoint{5.677666in}{0.409217in}}%
\pgfpathlineto{\pgfqpoint{5.802641in}{0.409217in}}%
\pgfpathlineto{\pgfqpoint{5.927616in}{0.409217in}}%
\pgfpathlineto{\pgfqpoint{6.052591in}{0.409217in}}%
\pgfpathlineto{\pgfqpoint{6.177565in}{0.409217in}}%
\pgfpathlineto{\pgfqpoint{6.302540in}{0.409217in}}%
\pgfpathlineto{\pgfqpoint{6.490002in}{0.409217in}}%
\pgfpathlineto{\pgfqpoint{6.490002in}{0.505641in}}%
\pgfpathlineto{\pgfqpoint{6.490002in}{0.505641in}}%
\pgfpathlineto{\pgfqpoint{6.302540in}{0.505641in}}%
\pgfpathlineto{\pgfqpoint{6.177565in}{0.505641in}}%
\pgfpathlineto{\pgfqpoint{6.052591in}{0.505641in}}%
\pgfpathlineto{\pgfqpoint{5.927616in}{0.505641in}}%
\pgfpathlineto{\pgfqpoint{5.802641in}{0.505641in}}%
\pgfpathlineto{\pgfqpoint{5.677666in}{0.505641in}}%
\pgfpathlineto{\pgfqpoint{5.552691in}{0.505641in}}%
\pgfpathlineto{\pgfqpoint{5.427717in}{0.505641in}}%
\pgfpathlineto{\pgfqpoint{5.302742in}{0.505641in}}%
\pgfpathlineto{\pgfqpoint{5.177767in}{0.505641in}}%
\pgfpathlineto{\pgfqpoint{5.052792in}{0.505641in}}%
\pgfpathlineto{\pgfqpoint{4.927817in}{0.505641in}}%
\pgfpathlineto{\pgfqpoint{4.802843in}{0.505641in}}%
\pgfpathlineto{\pgfqpoint{4.677868in}{0.505641in}}%
\pgfpathlineto{\pgfqpoint{4.552893in}{0.505641in}}%
\pgfpathlineto{\pgfqpoint{4.427918in}{0.505641in}}%
\pgfpathlineto{\pgfqpoint{4.302943in}{0.505641in}}%
\pgfpathlineto{\pgfqpoint{4.177969in}{0.505641in}}%
\pgfpathlineto{\pgfqpoint{3.990506in}{0.505641in}}%
\pgfpathclose%
\pgfusepath{stroke,fill}%
\end{pgfscope}%
\begin{pgfscope}%
\pgfpathrectangle{\pgfqpoint{3.799920in}{0.331635in}}{\pgfqpoint{2.818182in}{1.132500in}}%
\pgfusepath{clip}%
\pgfsetbuttcap%
\pgfsetroundjoin%
\pgfsetlinewidth{1.505625pt}%
\definecolor{currentstroke}{rgb}{0.000000,0.000000,0.000000}%
\pgfsetstrokecolor{currentstroke}%
\pgfsetdash{}{0pt}%
\pgfpathmoveto{\pgfqpoint{3.928019in}{0.457429in}}%
\pgfpathlineto{\pgfqpoint{3.928019in}{1.412658in}}%
\pgfusepath{stroke}%
\end{pgfscope}%
\begin{pgfscope}%
\pgfpathrectangle{\pgfqpoint{3.799920in}{0.331635in}}{\pgfqpoint{2.818182in}{1.132500in}}%
\pgfusepath{clip}%
\pgfsetbuttcap%
\pgfsetroundjoin%
\pgfsetlinewidth{1.505625pt}%
\definecolor{currentstroke}{rgb}{0.000000,0.000000,0.000000}%
\pgfsetstrokecolor{currentstroke}%
\pgfsetdash{}{0pt}%
\pgfpathmoveto{\pgfqpoint{4.052994in}{0.457429in}}%
\pgfpathlineto{\pgfqpoint{4.052994in}{0.444318in}}%
\pgfusepath{stroke}%
\end{pgfscope}%
\begin{pgfscope}%
\pgfpathrectangle{\pgfqpoint{3.799920in}{0.331635in}}{\pgfqpoint{2.818182in}{1.132500in}}%
\pgfusepath{clip}%
\pgfsetbuttcap%
\pgfsetroundjoin%
\pgfsetlinewidth{1.505625pt}%
\definecolor{currentstroke}{rgb}{0.000000,0.000000,0.000000}%
\pgfsetstrokecolor{currentstroke}%
\pgfsetdash{}{0pt}%
\pgfpathmoveto{\pgfqpoint{4.177969in}{0.457429in}}%
\pgfpathlineto{\pgfqpoint{4.177969in}{0.458982in}}%
\pgfusepath{stroke}%
\end{pgfscope}%
\begin{pgfscope}%
\pgfpathrectangle{\pgfqpoint{3.799920in}{0.331635in}}{\pgfqpoint{2.818182in}{1.132500in}}%
\pgfusepath{clip}%
\pgfsetbuttcap%
\pgfsetroundjoin%
\pgfsetlinewidth{1.505625pt}%
\definecolor{currentstroke}{rgb}{0.000000,0.000000,0.000000}%
\pgfsetstrokecolor{currentstroke}%
\pgfsetdash{}{0pt}%
\pgfpathmoveto{\pgfqpoint{4.302943in}{0.457429in}}%
\pgfpathlineto{\pgfqpoint{4.302943in}{0.458431in}}%
\pgfusepath{stroke}%
\end{pgfscope}%
\begin{pgfscope}%
\pgfpathrectangle{\pgfqpoint{3.799920in}{0.331635in}}{\pgfqpoint{2.818182in}{1.132500in}}%
\pgfusepath{clip}%
\pgfsetbuttcap%
\pgfsetroundjoin%
\pgfsetlinewidth{1.505625pt}%
\definecolor{currentstroke}{rgb}{0.000000,0.000000,0.000000}%
\pgfsetstrokecolor{currentstroke}%
\pgfsetdash{}{0pt}%
\pgfpathmoveto{\pgfqpoint{4.427918in}{0.457429in}}%
\pgfpathlineto{\pgfqpoint{4.427918in}{0.446642in}}%
\pgfusepath{stroke}%
\end{pgfscope}%
\begin{pgfscope}%
\pgfpathrectangle{\pgfqpoint{3.799920in}{0.331635in}}{\pgfqpoint{2.818182in}{1.132500in}}%
\pgfusepath{clip}%
\pgfsetbuttcap%
\pgfsetroundjoin%
\pgfsetlinewidth{1.505625pt}%
\definecolor{currentstroke}{rgb}{0.000000,0.000000,0.000000}%
\pgfsetstrokecolor{currentstroke}%
\pgfsetdash{}{0pt}%
\pgfpathmoveto{\pgfqpoint{4.552893in}{0.457429in}}%
\pgfpathlineto{\pgfqpoint{4.552893in}{0.459049in}}%
\pgfusepath{stroke}%
\end{pgfscope}%
\begin{pgfscope}%
\pgfpathrectangle{\pgfqpoint{3.799920in}{0.331635in}}{\pgfqpoint{2.818182in}{1.132500in}}%
\pgfusepath{clip}%
\pgfsetbuttcap%
\pgfsetroundjoin%
\pgfsetlinewidth{1.505625pt}%
\definecolor{currentstroke}{rgb}{0.000000,0.000000,0.000000}%
\pgfsetstrokecolor{currentstroke}%
\pgfsetdash{}{0pt}%
\pgfpathmoveto{\pgfqpoint{4.677868in}{0.457429in}}%
\pgfpathlineto{\pgfqpoint{4.677868in}{0.449034in}}%
\pgfusepath{stroke}%
\end{pgfscope}%
\begin{pgfscope}%
\pgfpathrectangle{\pgfqpoint{3.799920in}{0.331635in}}{\pgfqpoint{2.818182in}{1.132500in}}%
\pgfusepath{clip}%
\pgfsetbuttcap%
\pgfsetroundjoin%
\pgfsetlinewidth{1.505625pt}%
\definecolor{currentstroke}{rgb}{0.000000,0.000000,0.000000}%
\pgfsetstrokecolor{currentstroke}%
\pgfsetdash{}{0pt}%
\pgfpathmoveto{\pgfqpoint{4.802843in}{0.457429in}}%
\pgfpathlineto{\pgfqpoint{4.802843in}{0.478926in}}%
\pgfusepath{stroke}%
\end{pgfscope}%
\begin{pgfscope}%
\pgfpathrectangle{\pgfqpoint{3.799920in}{0.331635in}}{\pgfqpoint{2.818182in}{1.132500in}}%
\pgfusepath{clip}%
\pgfsetbuttcap%
\pgfsetroundjoin%
\pgfsetlinewidth{1.505625pt}%
\definecolor{currentstroke}{rgb}{0.000000,0.000000,0.000000}%
\pgfsetstrokecolor{currentstroke}%
\pgfsetdash{}{0pt}%
\pgfpathmoveto{\pgfqpoint{4.927817in}{0.457429in}}%
\pgfpathlineto{\pgfqpoint{4.927817in}{0.468903in}}%
\pgfusepath{stroke}%
\end{pgfscope}%
\begin{pgfscope}%
\pgfpathrectangle{\pgfqpoint{3.799920in}{0.331635in}}{\pgfqpoint{2.818182in}{1.132500in}}%
\pgfusepath{clip}%
\pgfsetbuttcap%
\pgfsetroundjoin%
\pgfsetlinewidth{1.505625pt}%
\definecolor{currentstroke}{rgb}{0.000000,0.000000,0.000000}%
\pgfsetstrokecolor{currentstroke}%
\pgfsetdash{}{0pt}%
\pgfpathmoveto{\pgfqpoint{5.052792in}{0.457429in}}%
\pgfpathlineto{\pgfqpoint{5.052792in}{0.422688in}}%
\pgfusepath{stroke}%
\end{pgfscope}%
\begin{pgfscope}%
\pgfpathrectangle{\pgfqpoint{3.799920in}{0.331635in}}{\pgfqpoint{2.818182in}{1.132500in}}%
\pgfusepath{clip}%
\pgfsetbuttcap%
\pgfsetroundjoin%
\pgfsetlinewidth{1.505625pt}%
\definecolor{currentstroke}{rgb}{0.000000,0.000000,0.000000}%
\pgfsetstrokecolor{currentstroke}%
\pgfsetdash{}{0pt}%
\pgfpathmoveto{\pgfqpoint{5.177767in}{0.457429in}}%
\pgfpathlineto{\pgfqpoint{5.177767in}{0.494257in}}%
\pgfusepath{stroke}%
\end{pgfscope}%
\begin{pgfscope}%
\pgfpathrectangle{\pgfqpoint{3.799920in}{0.331635in}}{\pgfqpoint{2.818182in}{1.132500in}}%
\pgfusepath{clip}%
\pgfsetbuttcap%
\pgfsetroundjoin%
\pgfsetlinewidth{1.505625pt}%
\definecolor{currentstroke}{rgb}{0.000000,0.000000,0.000000}%
\pgfsetstrokecolor{currentstroke}%
\pgfsetdash{}{0pt}%
\pgfpathmoveto{\pgfqpoint{5.302742in}{0.457429in}}%
\pgfpathlineto{\pgfqpoint{5.302742in}{0.488561in}}%
\pgfusepath{stroke}%
\end{pgfscope}%
\begin{pgfscope}%
\pgfpathrectangle{\pgfqpoint{3.799920in}{0.331635in}}{\pgfqpoint{2.818182in}{1.132500in}}%
\pgfusepath{clip}%
\pgfsetbuttcap%
\pgfsetroundjoin%
\pgfsetlinewidth{1.505625pt}%
\definecolor{currentstroke}{rgb}{0.000000,0.000000,0.000000}%
\pgfsetstrokecolor{currentstroke}%
\pgfsetdash{}{0pt}%
\pgfpathmoveto{\pgfqpoint{5.427717in}{0.457429in}}%
\pgfpathlineto{\pgfqpoint{5.427717in}{0.468654in}}%
\pgfusepath{stroke}%
\end{pgfscope}%
\begin{pgfscope}%
\pgfpathrectangle{\pgfqpoint{3.799920in}{0.331635in}}{\pgfqpoint{2.818182in}{1.132500in}}%
\pgfusepath{clip}%
\pgfsetbuttcap%
\pgfsetroundjoin%
\pgfsetlinewidth{1.505625pt}%
\definecolor{currentstroke}{rgb}{0.000000,0.000000,0.000000}%
\pgfsetstrokecolor{currentstroke}%
\pgfsetdash{}{0pt}%
\pgfpathmoveto{\pgfqpoint{5.552691in}{0.457429in}}%
\pgfpathlineto{\pgfqpoint{5.552691in}{0.409169in}}%
\pgfusepath{stroke}%
\end{pgfscope}%
\begin{pgfscope}%
\pgfpathrectangle{\pgfqpoint{3.799920in}{0.331635in}}{\pgfqpoint{2.818182in}{1.132500in}}%
\pgfusepath{clip}%
\pgfsetbuttcap%
\pgfsetroundjoin%
\pgfsetlinewidth{1.505625pt}%
\definecolor{currentstroke}{rgb}{0.000000,0.000000,0.000000}%
\pgfsetstrokecolor{currentstroke}%
\pgfsetdash{}{0pt}%
\pgfpathmoveto{\pgfqpoint{5.677666in}{0.457429in}}%
\pgfpathlineto{\pgfqpoint{5.677666in}{0.461928in}}%
\pgfusepath{stroke}%
\end{pgfscope}%
\begin{pgfscope}%
\pgfpathrectangle{\pgfqpoint{3.799920in}{0.331635in}}{\pgfqpoint{2.818182in}{1.132500in}}%
\pgfusepath{clip}%
\pgfsetbuttcap%
\pgfsetroundjoin%
\pgfsetlinewidth{1.505625pt}%
\definecolor{currentstroke}{rgb}{0.000000,0.000000,0.000000}%
\pgfsetstrokecolor{currentstroke}%
\pgfsetdash{}{0pt}%
\pgfpathmoveto{\pgfqpoint{5.802641in}{0.457429in}}%
\pgfpathlineto{\pgfqpoint{5.802641in}{0.383112in}}%
\pgfusepath{stroke}%
\end{pgfscope}%
\begin{pgfscope}%
\pgfpathrectangle{\pgfqpoint{3.799920in}{0.331635in}}{\pgfqpoint{2.818182in}{1.132500in}}%
\pgfusepath{clip}%
\pgfsetbuttcap%
\pgfsetroundjoin%
\pgfsetlinewidth{1.505625pt}%
\definecolor{currentstroke}{rgb}{0.000000,0.000000,0.000000}%
\pgfsetstrokecolor{currentstroke}%
\pgfsetdash{}{0pt}%
\pgfpathmoveto{\pgfqpoint{5.927616in}{0.457429in}}%
\pgfpathlineto{\pgfqpoint{5.927616in}{0.413042in}}%
\pgfusepath{stroke}%
\end{pgfscope}%
\begin{pgfscope}%
\pgfpathrectangle{\pgfqpoint{3.799920in}{0.331635in}}{\pgfqpoint{2.818182in}{1.132500in}}%
\pgfusepath{clip}%
\pgfsetbuttcap%
\pgfsetroundjoin%
\pgfsetlinewidth{1.505625pt}%
\definecolor{currentstroke}{rgb}{0.000000,0.000000,0.000000}%
\pgfsetstrokecolor{currentstroke}%
\pgfsetdash{}{0pt}%
\pgfpathmoveto{\pgfqpoint{6.052591in}{0.457429in}}%
\pgfpathlineto{\pgfqpoint{6.052591in}{0.505228in}}%
\pgfusepath{stroke}%
\end{pgfscope}%
\begin{pgfscope}%
\pgfpathrectangle{\pgfqpoint{3.799920in}{0.331635in}}{\pgfqpoint{2.818182in}{1.132500in}}%
\pgfusepath{clip}%
\pgfsetbuttcap%
\pgfsetroundjoin%
\pgfsetlinewidth{1.505625pt}%
\definecolor{currentstroke}{rgb}{0.000000,0.000000,0.000000}%
\pgfsetstrokecolor{currentstroke}%
\pgfsetdash{}{0pt}%
\pgfpathmoveto{\pgfqpoint{6.177565in}{0.457429in}}%
\pgfpathlineto{\pgfqpoint{6.177565in}{0.527434in}}%
\pgfusepath{stroke}%
\end{pgfscope}%
\begin{pgfscope}%
\pgfpathrectangle{\pgfqpoint{3.799920in}{0.331635in}}{\pgfqpoint{2.818182in}{1.132500in}}%
\pgfusepath{clip}%
\pgfsetbuttcap%
\pgfsetroundjoin%
\pgfsetlinewidth{1.505625pt}%
\definecolor{currentstroke}{rgb}{0.000000,0.000000,0.000000}%
\pgfsetstrokecolor{currentstroke}%
\pgfsetdash{}{0pt}%
\pgfpathmoveto{\pgfqpoint{6.302540in}{0.457429in}}%
\pgfpathlineto{\pgfqpoint{6.302540in}{0.462351in}}%
\pgfusepath{stroke}%
\end{pgfscope}%
\begin{pgfscope}%
\pgfpathrectangle{\pgfqpoint{3.799920in}{0.331635in}}{\pgfqpoint{2.818182in}{1.132500in}}%
\pgfusepath{clip}%
\pgfsetbuttcap%
\pgfsetroundjoin%
\pgfsetlinewidth{1.505625pt}%
\definecolor{currentstroke}{rgb}{0.000000,0.000000,0.000000}%
\pgfsetstrokecolor{currentstroke}%
\pgfsetdash{}{0pt}%
\pgfpathmoveto{\pgfqpoint{6.427515in}{0.457429in}}%
\pgfpathlineto{\pgfqpoint{6.427515in}{0.462609in}}%
\pgfusepath{stroke}%
\end{pgfscope}%
\begin{pgfscope}%
\pgfpathrectangle{\pgfqpoint{3.799920in}{0.331635in}}{\pgfqpoint{2.818182in}{1.132500in}}%
\pgfusepath{clip}%
\pgfsetroundcap%
\pgfsetroundjoin%
\pgfsetlinewidth{1.505625pt}%
\definecolor{currentstroke}{rgb}{0.839216,0.152941,0.156863}%
\pgfsetstrokecolor{currentstroke}%
\pgfsetdash{}{0pt}%
\pgfpathmoveto{\pgfqpoint{3.799920in}{0.457429in}}%
\pgfpathlineto{\pgfqpoint{6.618102in}{0.457429in}}%
\pgfusepath{stroke}%
\end{pgfscope}%
\begin{pgfscope}%
\pgfpathrectangle{\pgfqpoint{3.799920in}{0.331635in}}{\pgfqpoint{2.818182in}{1.132500in}}%
\pgfusepath{clip}%
\pgfsetbuttcap%
\pgfsetroundjoin%
\definecolor{currentfill}{rgb}{0.839216,0.152941,0.156863}%
\pgfsetfillcolor{currentfill}%
\pgfsetlinewidth{1.003750pt}%
\definecolor{currentstroke}{rgb}{0.839216,0.152941,0.156863}%
\pgfsetstrokecolor{currentstroke}%
\pgfsetdash{}{0pt}%
\pgfsys@defobject{currentmarker}{\pgfqpoint{-0.034722in}{-0.034722in}}{\pgfqpoint{0.034722in}{0.034722in}}{%
\pgfpathmoveto{\pgfqpoint{0.000000in}{-0.034722in}}%
\pgfpathcurveto{\pgfqpoint{0.009208in}{-0.034722in}}{\pgfqpoint{0.018041in}{-0.031064in}}{\pgfqpoint{0.024552in}{-0.024552in}}%
\pgfpathcurveto{\pgfqpoint{0.031064in}{-0.018041in}}{\pgfqpoint{0.034722in}{-0.009208in}}{\pgfqpoint{0.034722in}{0.000000in}}%
\pgfpathcurveto{\pgfqpoint{0.034722in}{0.009208in}}{\pgfqpoint{0.031064in}{0.018041in}}{\pgfqpoint{0.024552in}{0.024552in}}%
\pgfpathcurveto{\pgfqpoint{0.018041in}{0.031064in}}{\pgfqpoint{0.009208in}{0.034722in}}{\pgfqpoint{0.000000in}{0.034722in}}%
\pgfpathcurveto{\pgfqpoint{-0.009208in}{0.034722in}}{\pgfqpoint{-0.018041in}{0.031064in}}{\pgfqpoint{-0.024552in}{0.024552in}}%
\pgfpathcurveto{\pgfqpoint{-0.031064in}{0.018041in}}{\pgfqpoint{-0.034722in}{0.009208in}}{\pgfqpoint{-0.034722in}{0.000000in}}%
\pgfpathcurveto{\pgfqpoint{-0.034722in}{-0.009208in}}{\pgfqpoint{-0.031064in}{-0.018041in}}{\pgfqpoint{-0.024552in}{-0.024552in}}%
\pgfpathcurveto{\pgfqpoint{-0.018041in}{-0.031064in}}{\pgfqpoint{-0.009208in}{-0.034722in}}{\pgfqpoint{0.000000in}{-0.034722in}}%
\pgfpathclose%
\pgfusepath{stroke,fill}%
}%
\begin{pgfscope}%
\pgfsys@transformshift{3.928019in}{1.412658in}%
\pgfsys@useobject{currentmarker}{}%
\end{pgfscope}%
\begin{pgfscope}%
\pgfsys@transformshift{4.052994in}{0.444318in}%
\pgfsys@useobject{currentmarker}{}%
\end{pgfscope}%
\begin{pgfscope}%
\pgfsys@transformshift{4.177969in}{0.458982in}%
\pgfsys@useobject{currentmarker}{}%
\end{pgfscope}%
\begin{pgfscope}%
\pgfsys@transformshift{4.302943in}{0.458431in}%
\pgfsys@useobject{currentmarker}{}%
\end{pgfscope}%
\begin{pgfscope}%
\pgfsys@transformshift{4.427918in}{0.446642in}%
\pgfsys@useobject{currentmarker}{}%
\end{pgfscope}%
\begin{pgfscope}%
\pgfsys@transformshift{4.552893in}{0.459049in}%
\pgfsys@useobject{currentmarker}{}%
\end{pgfscope}%
\begin{pgfscope}%
\pgfsys@transformshift{4.677868in}{0.449034in}%
\pgfsys@useobject{currentmarker}{}%
\end{pgfscope}%
\begin{pgfscope}%
\pgfsys@transformshift{4.802843in}{0.478926in}%
\pgfsys@useobject{currentmarker}{}%
\end{pgfscope}%
\begin{pgfscope}%
\pgfsys@transformshift{4.927817in}{0.468903in}%
\pgfsys@useobject{currentmarker}{}%
\end{pgfscope}%
\begin{pgfscope}%
\pgfsys@transformshift{5.052792in}{0.422688in}%
\pgfsys@useobject{currentmarker}{}%
\end{pgfscope}%
\begin{pgfscope}%
\pgfsys@transformshift{5.177767in}{0.494257in}%
\pgfsys@useobject{currentmarker}{}%
\end{pgfscope}%
\begin{pgfscope}%
\pgfsys@transformshift{5.302742in}{0.488561in}%
\pgfsys@useobject{currentmarker}{}%
\end{pgfscope}%
\begin{pgfscope}%
\pgfsys@transformshift{5.427717in}{0.468654in}%
\pgfsys@useobject{currentmarker}{}%
\end{pgfscope}%
\begin{pgfscope}%
\pgfsys@transformshift{5.552691in}{0.409169in}%
\pgfsys@useobject{currentmarker}{}%
\end{pgfscope}%
\begin{pgfscope}%
\pgfsys@transformshift{5.677666in}{0.461928in}%
\pgfsys@useobject{currentmarker}{}%
\end{pgfscope}%
\begin{pgfscope}%
\pgfsys@transformshift{5.802641in}{0.383112in}%
\pgfsys@useobject{currentmarker}{}%
\end{pgfscope}%
\begin{pgfscope}%
\pgfsys@transformshift{5.927616in}{0.413042in}%
\pgfsys@useobject{currentmarker}{}%
\end{pgfscope}%
\begin{pgfscope}%
\pgfsys@transformshift{6.052591in}{0.505228in}%
\pgfsys@useobject{currentmarker}{}%
\end{pgfscope}%
\begin{pgfscope}%
\pgfsys@transformshift{6.177565in}{0.527434in}%
\pgfsys@useobject{currentmarker}{}%
\end{pgfscope}%
\begin{pgfscope}%
\pgfsys@transformshift{6.302540in}{0.462351in}%
\pgfsys@useobject{currentmarker}{}%
\end{pgfscope}%
\begin{pgfscope}%
\pgfsys@transformshift{6.427515in}{0.462609in}%
\pgfsys@useobject{currentmarker}{}%
\end{pgfscope}%
\end{pgfscope}%
\begin{pgfscope}%
\pgfsetrectcap%
\pgfsetmiterjoin%
\pgfsetlinewidth{0.803000pt}%
\definecolor{currentstroke}{rgb}{1.000000,1.000000,1.000000}%
\pgfsetstrokecolor{currentstroke}%
\pgfsetdash{}{0pt}%
\pgfpathmoveto{\pgfqpoint{3.799920in}{0.331635in}}%
\pgfpathlineto{\pgfqpoint{3.799920in}{1.464135in}}%
\pgfusepath{stroke}%
\end{pgfscope}%
\begin{pgfscope}%
\pgfsetrectcap%
\pgfsetmiterjoin%
\pgfsetlinewidth{0.803000pt}%
\definecolor{currentstroke}{rgb}{1.000000,1.000000,1.000000}%
\pgfsetstrokecolor{currentstroke}%
\pgfsetdash{}{0pt}%
\pgfpathmoveto{\pgfqpoint{6.618102in}{0.331635in}}%
\pgfpathlineto{\pgfqpoint{6.618102in}{1.464135in}}%
\pgfusepath{stroke}%
\end{pgfscope}%
\begin{pgfscope}%
\pgfsetrectcap%
\pgfsetmiterjoin%
\pgfsetlinewidth{0.803000pt}%
\definecolor{currentstroke}{rgb}{1.000000,1.000000,1.000000}%
\pgfsetstrokecolor{currentstroke}%
\pgfsetdash{}{0pt}%
\pgfpathmoveto{\pgfqpoint{3.799920in}{0.331635in}}%
\pgfpathlineto{\pgfqpoint{6.618102in}{0.331635in}}%
\pgfusepath{stroke}%
\end{pgfscope}%
\begin{pgfscope}%
\pgfsetrectcap%
\pgfsetmiterjoin%
\pgfsetlinewidth{0.803000pt}%
\definecolor{currentstroke}{rgb}{1.000000,1.000000,1.000000}%
\pgfsetstrokecolor{currentstroke}%
\pgfsetdash{}{0pt}%
\pgfpathmoveto{\pgfqpoint{3.799920in}{1.464135in}}%
\pgfpathlineto{\pgfqpoint{6.618102in}{1.464135in}}%
\pgfusepath{stroke}%
\end{pgfscope}%
\begin{pgfscope}%
\definecolor{textcolor}{rgb}{0.150000,0.150000,0.150000}%
\pgfsetstrokecolor{textcolor}%
\pgfsetfillcolor{textcolor}%
\pgftext[x=5.209011in,y=1.547468in,,base]{\color{textcolor}\rmfamily\fontsize{12.000000}{14.400000}\selectfont Partial Autocorrelation INTC}%
\end{pgfscope}%
\end{pgfpicture}%
\makeatother%
\endgroup%

    \end{adjustbox}
    \begin{adjustbox}{width=.95\textwidth,center}
    %% Creator: Matplotlib, PGF backend
%%
%% To include the figure in your LaTeX document, write
%%   \input{<filename>.pgf}
%%
%% Make sure the required packages are loaded in your preamble
%%   \usepackage{pgf}
%%
%% Figures using additional raster images can only be included by \input if
%% they are in the same directory as the main LaTeX file. For loading figures
%% from other directories you can use the `import` package
%%   \usepackage{import}
%% and then include the figures with
%%   \import{<path to file>}{<filename>.pgf}
%%
%% Matplotlib used the following preamble
%%   \usepackage{fontspec}
%%   \setmainfont{DejaVuSerif.ttf}[Path=/opt/tljh/user/lib/python3.6/site-packages/matplotlib/mpl-data/fonts/ttf/]
%%   \setsansfont{DejaVuSans.ttf}[Path=/opt/tljh/user/lib/python3.6/site-packages/matplotlib/mpl-data/fonts/ttf/]
%%   \setmonofont{DejaVuSansMono.ttf}[Path=/opt/tljh/user/lib/python3.6/site-packages/matplotlib/mpl-data/fonts/ttf/]
%%
\begingroup%
\makeatletter%
\begin{pgfpicture}%
\pgfpathrectangle{\pgfpointorigin}{\pgfqpoint{6.806467in}{2.151596in}}%
\pgfusepath{use as bounding box, clip}%
\begin{pgfscope}%
\pgfsetbuttcap%
\pgfsetmiterjoin%
\definecolor{currentfill}{rgb}{1.000000,1.000000,1.000000}%
\pgfsetfillcolor{currentfill}%
\pgfsetlinewidth{0.000000pt}%
\definecolor{currentstroke}{rgb}{1.000000,1.000000,1.000000}%
\pgfsetstrokecolor{currentstroke}%
\pgfsetdash{}{0pt}%
\pgfpathmoveto{\pgfqpoint{0.000000in}{0.000000in}}%
\pgfpathlineto{\pgfqpoint{6.806467in}{0.000000in}}%
\pgfpathlineto{\pgfqpoint{6.806467in}{2.151596in}}%
\pgfpathlineto{\pgfqpoint{0.000000in}{2.151596in}}%
\pgfpathclose%
\pgfusepath{fill}%
\end{pgfscope}%
\begin{pgfscope}%
\pgfsetbuttcap%
\pgfsetmiterjoin%
\definecolor{currentfill}{rgb}{0.917647,0.917647,0.949020}%
\pgfsetfillcolor{currentfill}%
\pgfsetlinewidth{0.000000pt}%
\definecolor{currentstroke}{rgb}{0.000000,0.000000,0.000000}%
\pgfsetstrokecolor{currentstroke}%
\pgfsetstrokeopacity{0.000000}%
\pgfsetdash{}{0pt}%
\pgfpathmoveto{\pgfqpoint{0.506467in}{0.331635in}}%
\pgfpathlineto{\pgfqpoint{3.089800in}{0.331635in}}%
\pgfpathlineto{\pgfqpoint{3.089800in}{1.841635in}}%
\pgfpathlineto{\pgfqpoint{0.506467in}{1.841635in}}%
\pgfpathclose%
\pgfusepath{fill}%
\end{pgfscope}%
\begin{pgfscope}%
\pgfpathrectangle{\pgfqpoint{0.506467in}{0.331635in}}{\pgfqpoint{2.583333in}{1.510000in}}%
\pgfusepath{clip}%
\pgfsetroundcap%
\pgfsetroundjoin%
\pgfsetlinewidth{0.803000pt}%
\definecolor{currentstroke}{rgb}{1.000000,1.000000,1.000000}%
\pgfsetstrokecolor{currentstroke}%
\pgfsetdash{}{0pt}%
\pgfpathmoveto{\pgfqpoint{0.623891in}{0.331635in}}%
\pgfpathlineto{\pgfqpoint{0.623891in}{1.841635in}}%
\pgfusepath{stroke}%
\end{pgfscope}%
\begin{pgfscope}%
\definecolor{textcolor}{rgb}{0.150000,0.150000,0.150000}%
\pgfsetstrokecolor{textcolor}%
\pgfsetfillcolor{textcolor}%
\pgftext[x=0.623891in,y=0.234413in,,top]{\color{textcolor}\rmfamily\fontsize{10.000000}{12.000000}\selectfont 0}%
\end{pgfscope}%
\begin{pgfscope}%
\pgfpathrectangle{\pgfqpoint{0.506467in}{0.331635in}}{\pgfqpoint{2.583333in}{1.510000in}}%
\pgfusepath{clip}%
\pgfsetroundcap%
\pgfsetroundjoin%
\pgfsetlinewidth{0.803000pt}%
\definecolor{currentstroke}{rgb}{1.000000,1.000000,1.000000}%
\pgfsetstrokecolor{currentstroke}%
\pgfsetdash{}{0pt}%
\pgfpathmoveto{\pgfqpoint{1.196693in}{0.331635in}}%
\pgfpathlineto{\pgfqpoint{1.196693in}{1.841635in}}%
\pgfusepath{stroke}%
\end{pgfscope}%
\begin{pgfscope}%
\definecolor{textcolor}{rgb}{0.150000,0.150000,0.150000}%
\pgfsetstrokecolor{textcolor}%
\pgfsetfillcolor{textcolor}%
\pgftext[x=1.196693in,y=0.234413in,,top]{\color{textcolor}\rmfamily\fontsize{10.000000}{12.000000}\selectfont 5}%
\end{pgfscope}%
\begin{pgfscope}%
\pgfpathrectangle{\pgfqpoint{0.506467in}{0.331635in}}{\pgfqpoint{2.583333in}{1.510000in}}%
\pgfusepath{clip}%
\pgfsetroundcap%
\pgfsetroundjoin%
\pgfsetlinewidth{0.803000pt}%
\definecolor{currentstroke}{rgb}{1.000000,1.000000,1.000000}%
\pgfsetstrokecolor{currentstroke}%
\pgfsetdash{}{0pt}%
\pgfpathmoveto{\pgfqpoint{1.769494in}{0.331635in}}%
\pgfpathlineto{\pgfqpoint{1.769494in}{1.841635in}}%
\pgfusepath{stroke}%
\end{pgfscope}%
\begin{pgfscope}%
\definecolor{textcolor}{rgb}{0.150000,0.150000,0.150000}%
\pgfsetstrokecolor{textcolor}%
\pgfsetfillcolor{textcolor}%
\pgftext[x=1.769494in,y=0.234413in,,top]{\color{textcolor}\rmfamily\fontsize{10.000000}{12.000000}\selectfont 10}%
\end{pgfscope}%
\begin{pgfscope}%
\pgfpathrectangle{\pgfqpoint{0.506467in}{0.331635in}}{\pgfqpoint{2.583333in}{1.510000in}}%
\pgfusepath{clip}%
\pgfsetroundcap%
\pgfsetroundjoin%
\pgfsetlinewidth{0.803000pt}%
\definecolor{currentstroke}{rgb}{1.000000,1.000000,1.000000}%
\pgfsetstrokecolor{currentstroke}%
\pgfsetdash{}{0pt}%
\pgfpathmoveto{\pgfqpoint{2.342295in}{0.331635in}}%
\pgfpathlineto{\pgfqpoint{2.342295in}{1.841635in}}%
\pgfusepath{stroke}%
\end{pgfscope}%
\begin{pgfscope}%
\definecolor{textcolor}{rgb}{0.150000,0.150000,0.150000}%
\pgfsetstrokecolor{textcolor}%
\pgfsetfillcolor{textcolor}%
\pgftext[x=2.342295in,y=0.234413in,,top]{\color{textcolor}\rmfamily\fontsize{10.000000}{12.000000}\selectfont 15}%
\end{pgfscope}%
\begin{pgfscope}%
\pgfpathrectangle{\pgfqpoint{0.506467in}{0.331635in}}{\pgfqpoint{2.583333in}{1.510000in}}%
\pgfusepath{clip}%
\pgfsetroundcap%
\pgfsetroundjoin%
\pgfsetlinewidth{0.803000pt}%
\definecolor{currentstroke}{rgb}{1.000000,1.000000,1.000000}%
\pgfsetstrokecolor{currentstroke}%
\pgfsetdash{}{0pt}%
\pgfpathmoveto{\pgfqpoint{2.915096in}{0.331635in}}%
\pgfpathlineto{\pgfqpoint{2.915096in}{1.841635in}}%
\pgfusepath{stroke}%
\end{pgfscope}%
\begin{pgfscope}%
\definecolor{textcolor}{rgb}{0.150000,0.150000,0.150000}%
\pgfsetstrokecolor{textcolor}%
\pgfsetfillcolor{textcolor}%
\pgftext[x=2.915096in,y=0.234413in,,top]{\color{textcolor}\rmfamily\fontsize{10.000000}{12.000000}\selectfont 20}%
\end{pgfscope}%
\begin{pgfscope}%
\pgfpathrectangle{\pgfqpoint{0.506467in}{0.331635in}}{\pgfqpoint{2.583333in}{1.510000in}}%
\pgfusepath{clip}%
\pgfsetroundcap%
\pgfsetroundjoin%
\pgfsetlinewidth{0.803000pt}%
\definecolor{currentstroke}{rgb}{1.000000,1.000000,1.000000}%
\pgfsetstrokecolor{currentstroke}%
\pgfsetdash{}{0pt}%
\pgfpathmoveto{\pgfqpoint{0.506467in}{0.504733in}}%
\pgfpathlineto{\pgfqpoint{3.089800in}{0.504733in}}%
\pgfusepath{stroke}%
\end{pgfscope}%
\begin{pgfscope}%
\definecolor{textcolor}{rgb}{0.150000,0.150000,0.150000}%
\pgfsetstrokecolor{textcolor}%
\pgfsetfillcolor{textcolor}%
\pgftext[x=0.100000in,y=0.451972in,left,base]{\color{textcolor}\rmfamily\fontsize{10.000000}{12.000000}\selectfont 0.00}%
\end{pgfscope}%
\begin{pgfscope}%
\pgfpathrectangle{\pgfqpoint{0.506467in}{0.331635in}}{\pgfqpoint{2.583333in}{1.510000in}}%
\pgfusepath{clip}%
\pgfsetroundcap%
\pgfsetroundjoin%
\pgfsetlinewidth{0.803000pt}%
\definecolor{currentstroke}{rgb}{1.000000,1.000000,1.000000}%
\pgfsetstrokecolor{currentstroke}%
\pgfsetdash{}{0pt}%
\pgfpathmoveto{\pgfqpoint{0.506467in}{0.821800in}}%
\pgfpathlineto{\pgfqpoint{3.089800in}{0.821800in}}%
\pgfusepath{stroke}%
\end{pgfscope}%
\begin{pgfscope}%
\definecolor{textcolor}{rgb}{0.150000,0.150000,0.150000}%
\pgfsetstrokecolor{textcolor}%
\pgfsetfillcolor{textcolor}%
\pgftext[x=0.100000in,y=0.769038in,left,base]{\color{textcolor}\rmfamily\fontsize{10.000000}{12.000000}\selectfont 0.25}%
\end{pgfscope}%
\begin{pgfscope}%
\pgfpathrectangle{\pgfqpoint{0.506467in}{0.331635in}}{\pgfqpoint{2.583333in}{1.510000in}}%
\pgfusepath{clip}%
\pgfsetroundcap%
\pgfsetroundjoin%
\pgfsetlinewidth{0.803000pt}%
\definecolor{currentstroke}{rgb}{1.000000,1.000000,1.000000}%
\pgfsetstrokecolor{currentstroke}%
\pgfsetdash{}{0pt}%
\pgfpathmoveto{\pgfqpoint{0.506467in}{1.138866in}}%
\pgfpathlineto{\pgfqpoint{3.089800in}{1.138866in}}%
\pgfusepath{stroke}%
\end{pgfscope}%
\begin{pgfscope}%
\definecolor{textcolor}{rgb}{0.150000,0.150000,0.150000}%
\pgfsetstrokecolor{textcolor}%
\pgfsetfillcolor{textcolor}%
\pgftext[x=0.100000in,y=1.086105in,left,base]{\color{textcolor}\rmfamily\fontsize{10.000000}{12.000000}\selectfont 0.50}%
\end{pgfscope}%
\begin{pgfscope}%
\pgfpathrectangle{\pgfqpoint{0.506467in}{0.331635in}}{\pgfqpoint{2.583333in}{1.510000in}}%
\pgfusepath{clip}%
\pgfsetroundcap%
\pgfsetroundjoin%
\pgfsetlinewidth{0.803000pt}%
\definecolor{currentstroke}{rgb}{1.000000,1.000000,1.000000}%
\pgfsetstrokecolor{currentstroke}%
\pgfsetdash{}{0pt}%
\pgfpathmoveto{\pgfqpoint{0.506467in}{1.455932in}}%
\pgfpathlineto{\pgfqpoint{3.089800in}{1.455932in}}%
\pgfusepath{stroke}%
\end{pgfscope}%
\begin{pgfscope}%
\definecolor{textcolor}{rgb}{0.150000,0.150000,0.150000}%
\pgfsetstrokecolor{textcolor}%
\pgfsetfillcolor{textcolor}%
\pgftext[x=0.100000in,y=1.403171in,left,base]{\color{textcolor}\rmfamily\fontsize{10.000000}{12.000000}\selectfont 0.75}%
\end{pgfscope}%
\begin{pgfscope}%
\pgfpathrectangle{\pgfqpoint{0.506467in}{0.331635in}}{\pgfqpoint{2.583333in}{1.510000in}}%
\pgfusepath{clip}%
\pgfsetroundcap%
\pgfsetroundjoin%
\pgfsetlinewidth{0.803000pt}%
\definecolor{currentstroke}{rgb}{1.000000,1.000000,1.000000}%
\pgfsetstrokecolor{currentstroke}%
\pgfsetdash{}{0pt}%
\pgfpathmoveto{\pgfqpoint{0.506467in}{1.772999in}}%
\pgfpathlineto{\pgfqpoint{3.089800in}{1.772999in}}%
\pgfusepath{stroke}%
\end{pgfscope}%
\begin{pgfscope}%
\definecolor{textcolor}{rgb}{0.150000,0.150000,0.150000}%
\pgfsetstrokecolor{textcolor}%
\pgfsetfillcolor{textcolor}%
\pgftext[x=0.100000in,y=1.720237in,left,base]{\color{textcolor}\rmfamily\fontsize{10.000000}{12.000000}\selectfont 1.00}%
\end{pgfscope}%
\begin{pgfscope}%
\pgfpathrectangle{\pgfqpoint{0.506467in}{0.331635in}}{\pgfqpoint{2.583333in}{1.510000in}}%
\pgfusepath{clip}%
\pgfsetbuttcap%
\pgfsetroundjoin%
\definecolor{currentfill}{rgb}{0.121569,0.466667,0.705882}%
\pgfsetfillcolor{currentfill}%
\pgfsetfillopacity{0.250000}%
\pgfsetlinewidth{1.003750pt}%
\definecolor{currentstroke}{rgb}{1.000000,1.000000,1.000000}%
\pgfsetstrokecolor{currentstroke}%
\pgfsetstrokeopacity{0.250000}%
\pgfsetdash{}{0pt}%
\pgfpathmoveto{\pgfqpoint{0.681171in}{0.568745in}}%
\pgfpathlineto{\pgfqpoint{0.681171in}{0.440722in}}%
\pgfpathlineto{\pgfqpoint{0.853012in}{0.440375in}}%
\pgfpathlineto{\pgfqpoint{0.967572in}{0.440283in}}%
\pgfpathlineto{\pgfqpoint{1.082132in}{0.440228in}}%
\pgfpathlineto{\pgfqpoint{1.196693in}{0.440062in}}%
\pgfpathlineto{\pgfqpoint{1.311253in}{0.439989in}}%
\pgfpathlineto{\pgfqpoint{1.425813in}{0.439980in}}%
\pgfpathlineto{\pgfqpoint{1.540373in}{0.439979in}}%
\pgfpathlineto{\pgfqpoint{1.654933in}{0.439704in}}%
\pgfpathlineto{\pgfqpoint{1.769494in}{0.439686in}}%
\pgfpathlineto{\pgfqpoint{1.884054in}{0.439627in}}%
\pgfpathlineto{\pgfqpoint{1.998614in}{0.439622in}}%
\pgfpathlineto{\pgfqpoint{2.113174in}{0.439618in}}%
\pgfpathlineto{\pgfqpoint{2.227735in}{0.439615in}}%
\pgfpathlineto{\pgfqpoint{2.342295in}{0.439610in}}%
\pgfpathlineto{\pgfqpoint{2.456855in}{0.439185in}}%
\pgfpathlineto{\pgfqpoint{2.571415in}{0.439185in}}%
\pgfpathlineto{\pgfqpoint{2.685976in}{0.439175in}}%
\pgfpathlineto{\pgfqpoint{2.800536in}{0.439165in}}%
\pgfpathlineto{\pgfqpoint{2.972376in}{0.439156in}}%
\pgfpathlineto{\pgfqpoint{2.972376in}{0.570311in}}%
\pgfpathlineto{\pgfqpoint{2.972376in}{0.570311in}}%
\pgfpathlineto{\pgfqpoint{2.800536in}{0.570302in}}%
\pgfpathlineto{\pgfqpoint{2.685976in}{0.570292in}}%
\pgfpathlineto{\pgfqpoint{2.571415in}{0.570282in}}%
\pgfpathlineto{\pgfqpoint{2.456855in}{0.570282in}}%
\pgfpathlineto{\pgfqpoint{2.342295in}{0.569857in}}%
\pgfpathlineto{\pgfqpoint{2.227735in}{0.569852in}}%
\pgfpathlineto{\pgfqpoint{2.113174in}{0.569849in}}%
\pgfpathlineto{\pgfqpoint{1.998614in}{0.569845in}}%
\pgfpathlineto{\pgfqpoint{1.884054in}{0.569840in}}%
\pgfpathlineto{\pgfqpoint{1.769494in}{0.569781in}}%
\pgfpathlineto{\pgfqpoint{1.654933in}{0.569763in}}%
\pgfpathlineto{\pgfqpoint{1.540373in}{0.569488in}}%
\pgfpathlineto{\pgfqpoint{1.425813in}{0.569487in}}%
\pgfpathlineto{\pgfqpoint{1.311253in}{0.569478in}}%
\pgfpathlineto{\pgfqpoint{1.196693in}{0.569405in}}%
\pgfpathlineto{\pgfqpoint{1.082132in}{0.569239in}}%
\pgfpathlineto{\pgfqpoint{0.967572in}{0.569184in}}%
\pgfpathlineto{\pgfqpoint{0.853012in}{0.569092in}}%
\pgfpathlineto{\pgfqpoint{0.681171in}{0.568745in}}%
\pgfpathclose%
\pgfusepath{stroke,fill}%
\end{pgfscope}%
\begin{pgfscope}%
\pgfpathrectangle{\pgfqpoint{0.506467in}{0.331635in}}{\pgfqpoint{2.583333in}{1.510000in}}%
\pgfusepath{clip}%
\pgfsetbuttcap%
\pgfsetroundjoin%
\pgfsetlinewidth{1.505625pt}%
\definecolor{currentstroke}{rgb}{0.000000,0.000000,0.000000}%
\pgfsetstrokecolor{currentstroke}%
\pgfsetdash{}{0pt}%
\pgfpathmoveto{\pgfqpoint{0.623891in}{0.504733in}}%
\pgfpathlineto{\pgfqpoint{0.623891in}{1.772999in}}%
\pgfusepath{stroke}%
\end{pgfscope}%
\begin{pgfscope}%
\pgfpathrectangle{\pgfqpoint{0.506467in}{0.331635in}}{\pgfqpoint{2.583333in}{1.510000in}}%
\pgfusepath{clip}%
\pgfsetbuttcap%
\pgfsetroundjoin%
\pgfsetlinewidth{1.505625pt}%
\definecolor{currentstroke}{rgb}{0.000000,0.000000,0.000000}%
\pgfsetstrokecolor{currentstroke}%
\pgfsetdash{}{0pt}%
\pgfpathmoveto{\pgfqpoint{0.738452in}{0.504733in}}%
\pgfpathlineto{\pgfqpoint{0.738452in}{0.411212in}}%
\pgfusepath{stroke}%
\end{pgfscope}%
\begin{pgfscope}%
\pgfpathrectangle{\pgfqpoint{0.506467in}{0.331635in}}{\pgfqpoint{2.583333in}{1.510000in}}%
\pgfusepath{clip}%
\pgfsetbuttcap%
\pgfsetroundjoin%
\pgfsetlinewidth{1.505625pt}%
\definecolor{currentstroke}{rgb}{0.000000,0.000000,0.000000}%
\pgfsetstrokecolor{currentstroke}%
\pgfsetdash{}{0pt}%
\pgfpathmoveto{\pgfqpoint{0.853012in}{0.504733in}}%
\pgfpathlineto{\pgfqpoint{0.853012in}{0.456503in}}%
\pgfusepath{stroke}%
\end{pgfscope}%
\begin{pgfscope}%
\pgfpathrectangle{\pgfqpoint{0.506467in}{0.331635in}}{\pgfqpoint{2.583333in}{1.510000in}}%
\pgfusepath{clip}%
\pgfsetbuttcap%
\pgfsetroundjoin%
\pgfsetlinewidth{1.505625pt}%
\definecolor{currentstroke}{rgb}{0.000000,0.000000,0.000000}%
\pgfsetstrokecolor{currentstroke}%
\pgfsetdash{}{0pt}%
\pgfpathmoveto{\pgfqpoint{0.967572in}{0.504733in}}%
\pgfpathlineto{\pgfqpoint{0.967572in}{0.467381in}}%
\pgfusepath{stroke}%
\end{pgfscope}%
\begin{pgfscope}%
\pgfpathrectangle{\pgfqpoint{0.506467in}{0.331635in}}{\pgfqpoint{2.583333in}{1.510000in}}%
\pgfusepath{clip}%
\pgfsetbuttcap%
\pgfsetroundjoin%
\pgfsetlinewidth{1.505625pt}%
\definecolor{currentstroke}{rgb}{0.000000,0.000000,0.000000}%
\pgfsetstrokecolor{currentstroke}%
\pgfsetdash{}{0pt}%
\pgfpathmoveto{\pgfqpoint{1.082132in}{0.504733in}}%
\pgfpathlineto{\pgfqpoint{1.082132in}{0.439844in}}%
\pgfusepath{stroke}%
\end{pgfscope}%
\begin{pgfscope}%
\pgfpathrectangle{\pgfqpoint{0.506467in}{0.331635in}}{\pgfqpoint{2.583333in}{1.510000in}}%
\pgfusepath{clip}%
\pgfsetbuttcap%
\pgfsetroundjoin%
\pgfsetlinewidth{1.505625pt}%
\definecolor{currentstroke}{rgb}{0.000000,0.000000,0.000000}%
\pgfsetstrokecolor{currentstroke}%
\pgfsetdash{}{0pt}%
\pgfpathmoveto{\pgfqpoint{1.196693in}{0.504733in}}%
\pgfpathlineto{\pgfqpoint{1.196693in}{0.461770in}}%
\pgfusepath{stroke}%
\end{pgfscope}%
\begin{pgfscope}%
\pgfpathrectangle{\pgfqpoint{0.506467in}{0.331635in}}{\pgfqpoint{2.583333in}{1.510000in}}%
\pgfusepath{clip}%
\pgfsetbuttcap%
\pgfsetroundjoin%
\pgfsetlinewidth{1.505625pt}%
\definecolor{currentstroke}{rgb}{0.000000,0.000000,0.000000}%
\pgfsetstrokecolor{currentstroke}%
\pgfsetdash{}{0pt}%
\pgfpathmoveto{\pgfqpoint{1.311253in}{0.504733in}}%
\pgfpathlineto{\pgfqpoint{1.311253in}{0.519947in}}%
\pgfusepath{stroke}%
\end{pgfscope}%
\begin{pgfscope}%
\pgfpathrectangle{\pgfqpoint{0.506467in}{0.331635in}}{\pgfqpoint{2.583333in}{1.510000in}}%
\pgfusepath{clip}%
\pgfsetbuttcap%
\pgfsetroundjoin%
\pgfsetlinewidth{1.505625pt}%
\definecolor{currentstroke}{rgb}{0.000000,0.000000,0.000000}%
\pgfsetstrokecolor{currentstroke}%
\pgfsetdash{}{0pt}%
\pgfpathmoveto{\pgfqpoint{1.425813in}{0.504733in}}%
\pgfpathlineto{\pgfqpoint{1.425813in}{0.499056in}}%
\pgfusepath{stroke}%
\end{pgfscope}%
\begin{pgfscope}%
\pgfpathrectangle{\pgfqpoint{0.506467in}{0.331635in}}{\pgfqpoint{2.583333in}{1.510000in}}%
\pgfusepath{clip}%
\pgfsetbuttcap%
\pgfsetroundjoin%
\pgfsetlinewidth{1.505625pt}%
\definecolor{currentstroke}{rgb}{0.000000,0.000000,0.000000}%
\pgfsetstrokecolor{currentstroke}%
\pgfsetdash{}{0pt}%
\pgfpathmoveto{\pgfqpoint{1.540373in}{0.504733in}}%
\pgfpathlineto{\pgfqpoint{1.540373in}{0.588348in}}%
\pgfusepath{stroke}%
\end{pgfscope}%
\begin{pgfscope}%
\pgfpathrectangle{\pgfqpoint{0.506467in}{0.331635in}}{\pgfqpoint{2.583333in}{1.510000in}}%
\pgfusepath{clip}%
\pgfsetbuttcap%
\pgfsetroundjoin%
\pgfsetlinewidth{1.505625pt}%
\definecolor{currentstroke}{rgb}{0.000000,0.000000,0.000000}%
\pgfsetstrokecolor{currentstroke}%
\pgfsetdash{}{0pt}%
\pgfpathmoveto{\pgfqpoint{1.654933in}{0.504733in}}%
\pgfpathlineto{\pgfqpoint{1.654933in}{0.483209in}}%
\pgfusepath{stroke}%
\end{pgfscope}%
\begin{pgfscope}%
\pgfpathrectangle{\pgfqpoint{0.506467in}{0.331635in}}{\pgfqpoint{2.583333in}{1.510000in}}%
\pgfusepath{clip}%
\pgfsetbuttcap%
\pgfsetroundjoin%
\pgfsetlinewidth{1.505625pt}%
\definecolor{currentstroke}{rgb}{0.000000,0.000000,0.000000}%
\pgfsetstrokecolor{currentstroke}%
\pgfsetdash{}{0pt}%
\pgfpathmoveto{\pgfqpoint{1.769494in}{0.504733in}}%
\pgfpathlineto{\pgfqpoint{1.769494in}{0.543585in}}%
\pgfusepath{stroke}%
\end{pgfscope}%
\begin{pgfscope}%
\pgfpathrectangle{\pgfqpoint{0.506467in}{0.331635in}}{\pgfqpoint{2.583333in}{1.510000in}}%
\pgfusepath{clip}%
\pgfsetbuttcap%
\pgfsetroundjoin%
\pgfsetlinewidth{1.505625pt}%
\definecolor{currentstroke}{rgb}{0.000000,0.000000,0.000000}%
\pgfsetstrokecolor{currentstroke}%
\pgfsetdash{}{0pt}%
\pgfpathmoveto{\pgfqpoint{1.884054in}{0.504733in}}%
\pgfpathlineto{\pgfqpoint{1.884054in}{0.516218in}}%
\pgfusepath{stroke}%
\end{pgfscope}%
\begin{pgfscope}%
\pgfpathrectangle{\pgfqpoint{0.506467in}{0.331635in}}{\pgfqpoint{2.583333in}{1.510000in}}%
\pgfusepath{clip}%
\pgfsetbuttcap%
\pgfsetroundjoin%
\pgfsetlinewidth{1.505625pt}%
\definecolor{currentstroke}{rgb}{0.000000,0.000000,0.000000}%
\pgfsetstrokecolor{currentstroke}%
\pgfsetdash{}{0pt}%
\pgfpathmoveto{\pgfqpoint{1.998614in}{0.504733in}}%
\pgfpathlineto{\pgfqpoint{1.998614in}{0.495343in}}%
\pgfusepath{stroke}%
\end{pgfscope}%
\begin{pgfscope}%
\pgfpathrectangle{\pgfqpoint{0.506467in}{0.331635in}}{\pgfqpoint{2.583333in}{1.510000in}}%
\pgfusepath{clip}%
\pgfsetbuttcap%
\pgfsetroundjoin%
\pgfsetlinewidth{1.505625pt}%
\definecolor{currentstroke}{rgb}{0.000000,0.000000,0.000000}%
\pgfsetstrokecolor{currentstroke}%
\pgfsetdash{}{0pt}%
\pgfpathmoveto{\pgfqpoint{2.113174in}{0.504733in}}%
\pgfpathlineto{\pgfqpoint{2.113174in}{0.495276in}}%
\pgfusepath{stroke}%
\end{pgfscope}%
\begin{pgfscope}%
\pgfpathrectangle{\pgfqpoint{0.506467in}{0.331635in}}{\pgfqpoint{2.583333in}{1.510000in}}%
\pgfusepath{clip}%
\pgfsetbuttcap%
\pgfsetroundjoin%
\pgfsetlinewidth{1.505625pt}%
\definecolor{currentstroke}{rgb}{0.000000,0.000000,0.000000}%
\pgfsetstrokecolor{currentstroke}%
\pgfsetdash{}{0pt}%
\pgfpathmoveto{\pgfqpoint{2.227735in}{0.504733in}}%
\pgfpathlineto{\pgfqpoint{2.227735in}{0.493785in}}%
\pgfusepath{stroke}%
\end{pgfscope}%
\begin{pgfscope}%
\pgfpathrectangle{\pgfqpoint{0.506467in}{0.331635in}}{\pgfqpoint{2.583333in}{1.510000in}}%
\pgfusepath{clip}%
\pgfsetbuttcap%
\pgfsetroundjoin%
\pgfsetlinewidth{1.505625pt}%
\definecolor{currentstroke}{rgb}{0.000000,0.000000,0.000000}%
\pgfsetstrokecolor{currentstroke}%
\pgfsetdash{}{0pt}%
\pgfpathmoveto{\pgfqpoint{2.342295in}{0.504733in}}%
\pgfpathlineto{\pgfqpoint{2.342295in}{0.400271in}}%
\pgfusepath{stroke}%
\end{pgfscope}%
\begin{pgfscope}%
\pgfpathrectangle{\pgfqpoint{0.506467in}{0.331635in}}{\pgfqpoint{2.583333in}{1.510000in}}%
\pgfusepath{clip}%
\pgfsetbuttcap%
\pgfsetroundjoin%
\pgfsetlinewidth{1.505625pt}%
\definecolor{currentstroke}{rgb}{0.000000,0.000000,0.000000}%
\pgfsetstrokecolor{currentstroke}%
\pgfsetdash{}{0pt}%
\pgfpathmoveto{\pgfqpoint{2.456855in}{0.504733in}}%
\pgfpathlineto{\pgfqpoint{2.456855in}{0.504034in}}%
\pgfusepath{stroke}%
\end{pgfscope}%
\begin{pgfscope}%
\pgfpathrectangle{\pgfqpoint{0.506467in}{0.331635in}}{\pgfqpoint{2.583333in}{1.510000in}}%
\pgfusepath{clip}%
\pgfsetbuttcap%
\pgfsetroundjoin%
\pgfsetlinewidth{1.505625pt}%
\definecolor{currentstroke}{rgb}{0.000000,0.000000,0.000000}%
\pgfsetstrokecolor{currentstroke}%
\pgfsetdash{}{0pt}%
\pgfpathmoveto{\pgfqpoint{2.571415in}{0.504733in}}%
\pgfpathlineto{\pgfqpoint{2.571415in}{0.520419in}}%
\pgfusepath{stroke}%
\end{pgfscope}%
\begin{pgfscope}%
\pgfpathrectangle{\pgfqpoint{0.506467in}{0.331635in}}{\pgfqpoint{2.583333in}{1.510000in}}%
\pgfusepath{clip}%
\pgfsetbuttcap%
\pgfsetroundjoin%
\pgfsetlinewidth{1.505625pt}%
\definecolor{currentstroke}{rgb}{0.000000,0.000000,0.000000}%
\pgfsetstrokecolor{currentstroke}%
\pgfsetdash{}{0pt}%
\pgfpathmoveto{\pgfqpoint{2.685976in}{0.504733in}}%
\pgfpathlineto{\pgfqpoint{2.685976in}{0.488435in}}%
\pgfusepath{stroke}%
\end{pgfscope}%
\begin{pgfscope}%
\pgfpathrectangle{\pgfqpoint{0.506467in}{0.331635in}}{\pgfqpoint{2.583333in}{1.510000in}}%
\pgfusepath{clip}%
\pgfsetbuttcap%
\pgfsetroundjoin%
\pgfsetlinewidth{1.505625pt}%
\definecolor{currentstroke}{rgb}{0.000000,0.000000,0.000000}%
\pgfsetstrokecolor{currentstroke}%
\pgfsetdash{}{0pt}%
\pgfpathmoveto{\pgfqpoint{2.800536in}{0.504733in}}%
\pgfpathlineto{\pgfqpoint{2.800536in}{0.520056in}}%
\pgfusepath{stroke}%
\end{pgfscope}%
\begin{pgfscope}%
\pgfpathrectangle{\pgfqpoint{0.506467in}{0.331635in}}{\pgfqpoint{2.583333in}{1.510000in}}%
\pgfusepath{clip}%
\pgfsetbuttcap%
\pgfsetroundjoin%
\pgfsetlinewidth{1.505625pt}%
\definecolor{currentstroke}{rgb}{0.000000,0.000000,0.000000}%
\pgfsetstrokecolor{currentstroke}%
\pgfsetdash{}{0pt}%
\pgfpathmoveto{\pgfqpoint{2.915096in}{0.504733in}}%
\pgfpathlineto{\pgfqpoint{2.915096in}{0.539542in}}%
\pgfusepath{stroke}%
\end{pgfscope}%
\begin{pgfscope}%
\pgfpathrectangle{\pgfqpoint{0.506467in}{0.331635in}}{\pgfqpoint{2.583333in}{1.510000in}}%
\pgfusepath{clip}%
\pgfsetroundcap%
\pgfsetroundjoin%
\pgfsetlinewidth{1.505625pt}%
\definecolor{currentstroke}{rgb}{0.737255,0.741176,0.133333}%
\pgfsetstrokecolor{currentstroke}%
\pgfsetdash{}{0pt}%
\pgfpathmoveto{\pgfqpoint{0.506467in}{0.504733in}}%
\pgfpathlineto{\pgfqpoint{3.089800in}{0.504733in}}%
\pgfusepath{stroke}%
\end{pgfscope}%
\begin{pgfscope}%
\pgfpathrectangle{\pgfqpoint{0.506467in}{0.331635in}}{\pgfqpoint{2.583333in}{1.510000in}}%
\pgfusepath{clip}%
\pgfsetbuttcap%
\pgfsetroundjoin%
\definecolor{currentfill}{rgb}{0.737255,0.741176,0.133333}%
\pgfsetfillcolor{currentfill}%
\pgfsetlinewidth{1.003750pt}%
\definecolor{currentstroke}{rgb}{0.737255,0.741176,0.133333}%
\pgfsetstrokecolor{currentstroke}%
\pgfsetdash{}{0pt}%
\pgfsys@defobject{currentmarker}{\pgfqpoint{-0.034722in}{-0.034722in}}{\pgfqpoint{0.034722in}{0.034722in}}{%
\pgfpathmoveto{\pgfqpoint{0.000000in}{-0.034722in}}%
\pgfpathcurveto{\pgfqpoint{0.009208in}{-0.034722in}}{\pgfqpoint{0.018041in}{-0.031064in}}{\pgfqpoint{0.024552in}{-0.024552in}}%
\pgfpathcurveto{\pgfqpoint{0.031064in}{-0.018041in}}{\pgfqpoint{0.034722in}{-0.009208in}}{\pgfqpoint{0.034722in}{0.000000in}}%
\pgfpathcurveto{\pgfqpoint{0.034722in}{0.009208in}}{\pgfqpoint{0.031064in}{0.018041in}}{\pgfqpoint{0.024552in}{0.024552in}}%
\pgfpathcurveto{\pgfqpoint{0.018041in}{0.031064in}}{\pgfqpoint{0.009208in}{0.034722in}}{\pgfqpoint{0.000000in}{0.034722in}}%
\pgfpathcurveto{\pgfqpoint{-0.009208in}{0.034722in}}{\pgfqpoint{-0.018041in}{0.031064in}}{\pgfqpoint{-0.024552in}{0.024552in}}%
\pgfpathcurveto{\pgfqpoint{-0.031064in}{0.018041in}}{\pgfqpoint{-0.034722in}{0.009208in}}{\pgfqpoint{-0.034722in}{0.000000in}}%
\pgfpathcurveto{\pgfqpoint{-0.034722in}{-0.009208in}}{\pgfqpoint{-0.031064in}{-0.018041in}}{\pgfqpoint{-0.024552in}{-0.024552in}}%
\pgfpathcurveto{\pgfqpoint{-0.018041in}{-0.031064in}}{\pgfqpoint{-0.009208in}{-0.034722in}}{\pgfqpoint{0.000000in}{-0.034722in}}%
\pgfpathclose%
\pgfusepath{stroke,fill}%
}%
\begin{pgfscope}%
\pgfsys@transformshift{0.623891in}{1.772999in}%
\pgfsys@useobject{currentmarker}{}%
\end{pgfscope}%
\begin{pgfscope}%
\pgfsys@transformshift{0.738452in}{0.411212in}%
\pgfsys@useobject{currentmarker}{}%
\end{pgfscope}%
\begin{pgfscope}%
\pgfsys@transformshift{0.853012in}{0.456503in}%
\pgfsys@useobject{currentmarker}{}%
\end{pgfscope}%
\begin{pgfscope}%
\pgfsys@transformshift{0.967572in}{0.467381in}%
\pgfsys@useobject{currentmarker}{}%
\end{pgfscope}%
\begin{pgfscope}%
\pgfsys@transformshift{1.082132in}{0.439844in}%
\pgfsys@useobject{currentmarker}{}%
\end{pgfscope}%
\begin{pgfscope}%
\pgfsys@transformshift{1.196693in}{0.461770in}%
\pgfsys@useobject{currentmarker}{}%
\end{pgfscope}%
\begin{pgfscope}%
\pgfsys@transformshift{1.311253in}{0.519947in}%
\pgfsys@useobject{currentmarker}{}%
\end{pgfscope}%
\begin{pgfscope}%
\pgfsys@transformshift{1.425813in}{0.499056in}%
\pgfsys@useobject{currentmarker}{}%
\end{pgfscope}%
\begin{pgfscope}%
\pgfsys@transformshift{1.540373in}{0.588348in}%
\pgfsys@useobject{currentmarker}{}%
\end{pgfscope}%
\begin{pgfscope}%
\pgfsys@transformshift{1.654933in}{0.483209in}%
\pgfsys@useobject{currentmarker}{}%
\end{pgfscope}%
\begin{pgfscope}%
\pgfsys@transformshift{1.769494in}{0.543585in}%
\pgfsys@useobject{currentmarker}{}%
\end{pgfscope}%
\begin{pgfscope}%
\pgfsys@transformshift{1.884054in}{0.516218in}%
\pgfsys@useobject{currentmarker}{}%
\end{pgfscope}%
\begin{pgfscope}%
\pgfsys@transformshift{1.998614in}{0.495343in}%
\pgfsys@useobject{currentmarker}{}%
\end{pgfscope}%
\begin{pgfscope}%
\pgfsys@transformshift{2.113174in}{0.495276in}%
\pgfsys@useobject{currentmarker}{}%
\end{pgfscope}%
\begin{pgfscope}%
\pgfsys@transformshift{2.227735in}{0.493785in}%
\pgfsys@useobject{currentmarker}{}%
\end{pgfscope}%
\begin{pgfscope}%
\pgfsys@transformshift{2.342295in}{0.400271in}%
\pgfsys@useobject{currentmarker}{}%
\end{pgfscope}%
\begin{pgfscope}%
\pgfsys@transformshift{2.456855in}{0.504034in}%
\pgfsys@useobject{currentmarker}{}%
\end{pgfscope}%
\begin{pgfscope}%
\pgfsys@transformshift{2.571415in}{0.520419in}%
\pgfsys@useobject{currentmarker}{}%
\end{pgfscope}%
\begin{pgfscope}%
\pgfsys@transformshift{2.685976in}{0.488435in}%
\pgfsys@useobject{currentmarker}{}%
\end{pgfscope}%
\begin{pgfscope}%
\pgfsys@transformshift{2.800536in}{0.520056in}%
\pgfsys@useobject{currentmarker}{}%
\end{pgfscope}%
\begin{pgfscope}%
\pgfsys@transformshift{2.915096in}{0.539542in}%
\pgfsys@useobject{currentmarker}{}%
\end{pgfscope}%
\end{pgfscope}%
\begin{pgfscope}%
\pgfsetrectcap%
\pgfsetmiterjoin%
\pgfsetlinewidth{0.803000pt}%
\definecolor{currentstroke}{rgb}{1.000000,1.000000,1.000000}%
\pgfsetstrokecolor{currentstroke}%
\pgfsetdash{}{0pt}%
\pgfpathmoveto{\pgfqpoint{0.506467in}{0.331635in}}%
\pgfpathlineto{\pgfqpoint{0.506467in}{1.841635in}}%
\pgfusepath{stroke}%
\end{pgfscope}%
\begin{pgfscope}%
\pgfsetrectcap%
\pgfsetmiterjoin%
\pgfsetlinewidth{0.803000pt}%
\definecolor{currentstroke}{rgb}{1.000000,1.000000,1.000000}%
\pgfsetstrokecolor{currentstroke}%
\pgfsetdash{}{0pt}%
\pgfpathmoveto{\pgfqpoint{3.089800in}{0.331635in}}%
\pgfpathlineto{\pgfqpoint{3.089800in}{1.841635in}}%
\pgfusepath{stroke}%
\end{pgfscope}%
\begin{pgfscope}%
\pgfsetrectcap%
\pgfsetmiterjoin%
\pgfsetlinewidth{0.803000pt}%
\definecolor{currentstroke}{rgb}{1.000000,1.000000,1.000000}%
\pgfsetstrokecolor{currentstroke}%
\pgfsetdash{}{0pt}%
\pgfpathmoveto{\pgfqpoint{0.506467in}{0.331635in}}%
\pgfpathlineto{\pgfqpoint{3.089800in}{0.331635in}}%
\pgfusepath{stroke}%
\end{pgfscope}%
\begin{pgfscope}%
\pgfsetrectcap%
\pgfsetmiterjoin%
\pgfsetlinewidth{0.803000pt}%
\definecolor{currentstroke}{rgb}{1.000000,1.000000,1.000000}%
\pgfsetstrokecolor{currentstroke}%
\pgfsetdash{}{0pt}%
\pgfpathmoveto{\pgfqpoint{0.506467in}{1.841635in}}%
\pgfpathlineto{\pgfqpoint{3.089800in}{1.841635in}}%
\pgfusepath{stroke}%
\end{pgfscope}%
\begin{pgfscope}%
\definecolor{textcolor}{rgb}{0.150000,0.150000,0.150000}%
\pgfsetstrokecolor{textcolor}%
\pgfsetfillcolor{textcolor}%
\pgftext[x=1.798134in,y=1.924968in,,base]{\color{textcolor}\rmfamily\fontsize{12.000000}{14.400000}\selectfont Autocorrelation}%
\end{pgfscope}%
\begin{pgfscope}%
\pgfsetbuttcap%
\pgfsetmiterjoin%
\definecolor{currentfill}{rgb}{0.917647,0.917647,0.949020}%
\pgfsetfillcolor{currentfill}%
\pgfsetlinewidth{0.000000pt}%
\definecolor{currentstroke}{rgb}{0.000000,0.000000,0.000000}%
\pgfsetstrokecolor{currentstroke}%
\pgfsetstrokeopacity{0.000000}%
\pgfsetdash{}{0pt}%
\pgfpathmoveto{\pgfqpoint{4.123134in}{0.331635in}}%
\pgfpathlineto{\pgfqpoint{6.706467in}{0.331635in}}%
\pgfpathlineto{\pgfqpoint{6.706467in}{1.841635in}}%
\pgfpathlineto{\pgfqpoint{4.123134in}{1.841635in}}%
\pgfpathclose%
\pgfusepath{fill}%
\end{pgfscope}%
\begin{pgfscope}%
\pgfpathrectangle{\pgfqpoint{4.123134in}{0.331635in}}{\pgfqpoint{2.583333in}{1.510000in}}%
\pgfusepath{clip}%
\pgfsetroundcap%
\pgfsetroundjoin%
\pgfsetlinewidth{0.803000pt}%
\definecolor{currentstroke}{rgb}{1.000000,1.000000,1.000000}%
\pgfsetstrokecolor{currentstroke}%
\pgfsetdash{}{0pt}%
\pgfpathmoveto{\pgfqpoint{4.240558in}{0.331635in}}%
\pgfpathlineto{\pgfqpoint{4.240558in}{1.841635in}}%
\pgfusepath{stroke}%
\end{pgfscope}%
\begin{pgfscope}%
\definecolor{textcolor}{rgb}{0.150000,0.150000,0.150000}%
\pgfsetstrokecolor{textcolor}%
\pgfsetfillcolor{textcolor}%
\pgftext[x=4.240558in,y=0.234413in,,top]{\color{textcolor}\rmfamily\fontsize{10.000000}{12.000000}\selectfont 0}%
\end{pgfscope}%
\begin{pgfscope}%
\pgfpathrectangle{\pgfqpoint{4.123134in}{0.331635in}}{\pgfqpoint{2.583333in}{1.510000in}}%
\pgfusepath{clip}%
\pgfsetroundcap%
\pgfsetroundjoin%
\pgfsetlinewidth{0.803000pt}%
\definecolor{currentstroke}{rgb}{1.000000,1.000000,1.000000}%
\pgfsetstrokecolor{currentstroke}%
\pgfsetdash{}{0pt}%
\pgfpathmoveto{\pgfqpoint{4.813359in}{0.331635in}}%
\pgfpathlineto{\pgfqpoint{4.813359in}{1.841635in}}%
\pgfusepath{stroke}%
\end{pgfscope}%
\begin{pgfscope}%
\definecolor{textcolor}{rgb}{0.150000,0.150000,0.150000}%
\pgfsetstrokecolor{textcolor}%
\pgfsetfillcolor{textcolor}%
\pgftext[x=4.813359in,y=0.234413in,,top]{\color{textcolor}\rmfamily\fontsize{10.000000}{12.000000}\selectfont 5}%
\end{pgfscope}%
\begin{pgfscope}%
\pgfpathrectangle{\pgfqpoint{4.123134in}{0.331635in}}{\pgfqpoint{2.583333in}{1.510000in}}%
\pgfusepath{clip}%
\pgfsetroundcap%
\pgfsetroundjoin%
\pgfsetlinewidth{0.803000pt}%
\definecolor{currentstroke}{rgb}{1.000000,1.000000,1.000000}%
\pgfsetstrokecolor{currentstroke}%
\pgfsetdash{}{0pt}%
\pgfpathmoveto{\pgfqpoint{5.386160in}{0.331635in}}%
\pgfpathlineto{\pgfqpoint{5.386160in}{1.841635in}}%
\pgfusepath{stroke}%
\end{pgfscope}%
\begin{pgfscope}%
\definecolor{textcolor}{rgb}{0.150000,0.150000,0.150000}%
\pgfsetstrokecolor{textcolor}%
\pgfsetfillcolor{textcolor}%
\pgftext[x=5.386160in,y=0.234413in,,top]{\color{textcolor}\rmfamily\fontsize{10.000000}{12.000000}\selectfont 10}%
\end{pgfscope}%
\begin{pgfscope}%
\pgfpathrectangle{\pgfqpoint{4.123134in}{0.331635in}}{\pgfqpoint{2.583333in}{1.510000in}}%
\pgfusepath{clip}%
\pgfsetroundcap%
\pgfsetroundjoin%
\pgfsetlinewidth{0.803000pt}%
\definecolor{currentstroke}{rgb}{1.000000,1.000000,1.000000}%
\pgfsetstrokecolor{currentstroke}%
\pgfsetdash{}{0pt}%
\pgfpathmoveto{\pgfqpoint{5.958962in}{0.331635in}}%
\pgfpathlineto{\pgfqpoint{5.958962in}{1.841635in}}%
\pgfusepath{stroke}%
\end{pgfscope}%
\begin{pgfscope}%
\definecolor{textcolor}{rgb}{0.150000,0.150000,0.150000}%
\pgfsetstrokecolor{textcolor}%
\pgfsetfillcolor{textcolor}%
\pgftext[x=5.958962in,y=0.234413in,,top]{\color{textcolor}\rmfamily\fontsize{10.000000}{12.000000}\selectfont 15}%
\end{pgfscope}%
\begin{pgfscope}%
\pgfpathrectangle{\pgfqpoint{4.123134in}{0.331635in}}{\pgfqpoint{2.583333in}{1.510000in}}%
\pgfusepath{clip}%
\pgfsetroundcap%
\pgfsetroundjoin%
\pgfsetlinewidth{0.803000pt}%
\definecolor{currentstroke}{rgb}{1.000000,1.000000,1.000000}%
\pgfsetstrokecolor{currentstroke}%
\pgfsetdash{}{0pt}%
\pgfpathmoveto{\pgfqpoint{6.531763in}{0.331635in}}%
\pgfpathlineto{\pgfqpoint{6.531763in}{1.841635in}}%
\pgfusepath{stroke}%
\end{pgfscope}%
\begin{pgfscope}%
\definecolor{textcolor}{rgb}{0.150000,0.150000,0.150000}%
\pgfsetstrokecolor{textcolor}%
\pgfsetfillcolor{textcolor}%
\pgftext[x=6.531763in,y=0.234413in,,top]{\color{textcolor}\rmfamily\fontsize{10.000000}{12.000000}\selectfont 20}%
\end{pgfscope}%
\begin{pgfscope}%
\pgfpathrectangle{\pgfqpoint{4.123134in}{0.331635in}}{\pgfqpoint{2.583333in}{1.510000in}}%
\pgfusepath{clip}%
\pgfsetroundcap%
\pgfsetroundjoin%
\pgfsetlinewidth{0.803000pt}%
\definecolor{currentstroke}{rgb}{1.000000,1.000000,1.000000}%
\pgfsetstrokecolor{currentstroke}%
\pgfsetdash{}{0pt}%
\pgfpathmoveto{\pgfqpoint{4.123134in}{0.504323in}}%
\pgfpathlineto{\pgfqpoint{6.706467in}{0.504323in}}%
\pgfusepath{stroke}%
\end{pgfscope}%
\begin{pgfscope}%
\definecolor{textcolor}{rgb}{0.150000,0.150000,0.150000}%
\pgfsetstrokecolor{textcolor}%
\pgfsetfillcolor{textcolor}%
\pgftext[x=3.716667in,y=0.451562in,left,base]{\color{textcolor}\rmfamily\fontsize{10.000000}{12.000000}\selectfont 0.00}%
\end{pgfscope}%
\begin{pgfscope}%
\pgfpathrectangle{\pgfqpoint{4.123134in}{0.331635in}}{\pgfqpoint{2.583333in}{1.510000in}}%
\pgfusepath{clip}%
\pgfsetroundcap%
\pgfsetroundjoin%
\pgfsetlinewidth{0.803000pt}%
\definecolor{currentstroke}{rgb}{1.000000,1.000000,1.000000}%
\pgfsetstrokecolor{currentstroke}%
\pgfsetdash{}{0pt}%
\pgfpathmoveto{\pgfqpoint{4.123134in}{0.821492in}}%
\pgfpathlineto{\pgfqpoint{6.706467in}{0.821492in}}%
\pgfusepath{stroke}%
\end{pgfscope}%
\begin{pgfscope}%
\definecolor{textcolor}{rgb}{0.150000,0.150000,0.150000}%
\pgfsetstrokecolor{textcolor}%
\pgfsetfillcolor{textcolor}%
\pgftext[x=3.716667in,y=0.768731in,left,base]{\color{textcolor}\rmfamily\fontsize{10.000000}{12.000000}\selectfont 0.25}%
\end{pgfscope}%
\begin{pgfscope}%
\pgfpathrectangle{\pgfqpoint{4.123134in}{0.331635in}}{\pgfqpoint{2.583333in}{1.510000in}}%
\pgfusepath{clip}%
\pgfsetroundcap%
\pgfsetroundjoin%
\pgfsetlinewidth{0.803000pt}%
\definecolor{currentstroke}{rgb}{1.000000,1.000000,1.000000}%
\pgfsetstrokecolor{currentstroke}%
\pgfsetdash{}{0pt}%
\pgfpathmoveto{\pgfqpoint{4.123134in}{1.138661in}}%
\pgfpathlineto{\pgfqpoint{6.706467in}{1.138661in}}%
\pgfusepath{stroke}%
\end{pgfscope}%
\begin{pgfscope}%
\definecolor{textcolor}{rgb}{0.150000,0.150000,0.150000}%
\pgfsetstrokecolor{textcolor}%
\pgfsetfillcolor{textcolor}%
\pgftext[x=3.716667in,y=1.085900in,left,base]{\color{textcolor}\rmfamily\fontsize{10.000000}{12.000000}\selectfont 0.50}%
\end{pgfscope}%
\begin{pgfscope}%
\pgfpathrectangle{\pgfqpoint{4.123134in}{0.331635in}}{\pgfqpoint{2.583333in}{1.510000in}}%
\pgfusepath{clip}%
\pgfsetroundcap%
\pgfsetroundjoin%
\pgfsetlinewidth{0.803000pt}%
\definecolor{currentstroke}{rgb}{1.000000,1.000000,1.000000}%
\pgfsetstrokecolor{currentstroke}%
\pgfsetdash{}{0pt}%
\pgfpathmoveto{\pgfqpoint{4.123134in}{1.455830in}}%
\pgfpathlineto{\pgfqpoint{6.706467in}{1.455830in}}%
\pgfusepath{stroke}%
\end{pgfscope}%
\begin{pgfscope}%
\definecolor{textcolor}{rgb}{0.150000,0.150000,0.150000}%
\pgfsetstrokecolor{textcolor}%
\pgfsetfillcolor{textcolor}%
\pgftext[x=3.716667in,y=1.403068in,left,base]{\color{textcolor}\rmfamily\fontsize{10.000000}{12.000000}\selectfont 0.75}%
\end{pgfscope}%
\begin{pgfscope}%
\pgfpathrectangle{\pgfqpoint{4.123134in}{0.331635in}}{\pgfqpoint{2.583333in}{1.510000in}}%
\pgfusepath{clip}%
\pgfsetroundcap%
\pgfsetroundjoin%
\pgfsetlinewidth{0.803000pt}%
\definecolor{currentstroke}{rgb}{1.000000,1.000000,1.000000}%
\pgfsetstrokecolor{currentstroke}%
\pgfsetdash{}{0pt}%
\pgfpathmoveto{\pgfqpoint{4.123134in}{1.772999in}}%
\pgfpathlineto{\pgfqpoint{6.706467in}{1.772999in}}%
\pgfusepath{stroke}%
\end{pgfscope}%
\begin{pgfscope}%
\definecolor{textcolor}{rgb}{0.150000,0.150000,0.150000}%
\pgfsetstrokecolor{textcolor}%
\pgfsetfillcolor{textcolor}%
\pgftext[x=3.716667in,y=1.720237in,left,base]{\color{textcolor}\rmfamily\fontsize{10.000000}{12.000000}\selectfont 1.00}%
\end{pgfscope}%
\begin{pgfscope}%
\pgfpathrectangle{\pgfqpoint{4.123134in}{0.331635in}}{\pgfqpoint{2.583333in}{1.510000in}}%
\pgfusepath{clip}%
\pgfsetbuttcap%
\pgfsetroundjoin%
\definecolor{currentfill}{rgb}{0.121569,0.466667,0.705882}%
\pgfsetfillcolor{currentfill}%
\pgfsetfillopacity{0.250000}%
\pgfsetlinewidth{1.003750pt}%
\definecolor{currentstroke}{rgb}{1.000000,1.000000,1.000000}%
\pgfsetstrokecolor{currentstroke}%
\pgfsetstrokeopacity{0.250000}%
\pgfsetdash{}{0pt}%
\pgfpathmoveto{\pgfqpoint{4.297838in}{0.568356in}}%
\pgfpathlineto{\pgfqpoint{4.297838in}{0.440291in}}%
\pgfpathlineto{\pgfqpoint{4.469678in}{0.440291in}}%
\pgfpathlineto{\pgfqpoint{4.584239in}{0.440291in}}%
\pgfpathlineto{\pgfqpoint{4.698799in}{0.440291in}}%
\pgfpathlineto{\pgfqpoint{4.813359in}{0.440291in}}%
\pgfpathlineto{\pgfqpoint{4.927919in}{0.440291in}}%
\pgfpathlineto{\pgfqpoint{5.042480in}{0.440291in}}%
\pgfpathlineto{\pgfqpoint{5.157040in}{0.440291in}}%
\pgfpathlineto{\pgfqpoint{5.271600in}{0.440291in}}%
\pgfpathlineto{\pgfqpoint{5.386160in}{0.440291in}}%
\pgfpathlineto{\pgfqpoint{5.500721in}{0.440291in}}%
\pgfpathlineto{\pgfqpoint{5.615281in}{0.440291in}}%
\pgfpathlineto{\pgfqpoint{5.729841in}{0.440291in}}%
\pgfpathlineto{\pgfqpoint{5.844401in}{0.440291in}}%
\pgfpathlineto{\pgfqpoint{5.958962in}{0.440291in}}%
\pgfpathlineto{\pgfqpoint{6.073522in}{0.440291in}}%
\pgfpathlineto{\pgfqpoint{6.188082in}{0.440291in}}%
\pgfpathlineto{\pgfqpoint{6.302642in}{0.440291in}}%
\pgfpathlineto{\pgfqpoint{6.417202in}{0.440291in}}%
\pgfpathlineto{\pgfqpoint{6.589043in}{0.440291in}}%
\pgfpathlineto{\pgfqpoint{6.589043in}{0.568356in}}%
\pgfpathlineto{\pgfqpoint{6.589043in}{0.568356in}}%
\pgfpathlineto{\pgfqpoint{6.417202in}{0.568356in}}%
\pgfpathlineto{\pgfqpoint{6.302642in}{0.568356in}}%
\pgfpathlineto{\pgfqpoint{6.188082in}{0.568356in}}%
\pgfpathlineto{\pgfqpoint{6.073522in}{0.568356in}}%
\pgfpathlineto{\pgfqpoint{5.958962in}{0.568356in}}%
\pgfpathlineto{\pgfqpoint{5.844401in}{0.568356in}}%
\pgfpathlineto{\pgfqpoint{5.729841in}{0.568356in}}%
\pgfpathlineto{\pgfqpoint{5.615281in}{0.568356in}}%
\pgfpathlineto{\pgfqpoint{5.500721in}{0.568356in}}%
\pgfpathlineto{\pgfqpoint{5.386160in}{0.568356in}}%
\pgfpathlineto{\pgfqpoint{5.271600in}{0.568356in}}%
\pgfpathlineto{\pgfqpoint{5.157040in}{0.568356in}}%
\pgfpathlineto{\pgfqpoint{5.042480in}{0.568356in}}%
\pgfpathlineto{\pgfqpoint{4.927919in}{0.568356in}}%
\pgfpathlineto{\pgfqpoint{4.813359in}{0.568356in}}%
\pgfpathlineto{\pgfqpoint{4.698799in}{0.568356in}}%
\pgfpathlineto{\pgfqpoint{4.584239in}{0.568356in}}%
\pgfpathlineto{\pgfqpoint{4.469678in}{0.568356in}}%
\pgfpathlineto{\pgfqpoint{4.297838in}{0.568356in}}%
\pgfpathclose%
\pgfusepath{stroke,fill}%
\end{pgfscope}%
\begin{pgfscope}%
\pgfpathrectangle{\pgfqpoint{4.123134in}{0.331635in}}{\pgfqpoint{2.583333in}{1.510000in}}%
\pgfusepath{clip}%
\pgfsetbuttcap%
\pgfsetroundjoin%
\pgfsetlinewidth{1.505625pt}%
\definecolor{currentstroke}{rgb}{0.000000,0.000000,0.000000}%
\pgfsetstrokecolor{currentstroke}%
\pgfsetdash{}{0pt}%
\pgfpathmoveto{\pgfqpoint{4.240558in}{0.504323in}}%
\pgfpathlineto{\pgfqpoint{4.240558in}{1.772999in}}%
\pgfusepath{stroke}%
\end{pgfscope}%
\begin{pgfscope}%
\pgfpathrectangle{\pgfqpoint{4.123134in}{0.331635in}}{\pgfqpoint{2.583333in}{1.510000in}}%
\pgfusepath{clip}%
\pgfsetbuttcap%
\pgfsetroundjoin%
\pgfsetlinewidth{1.505625pt}%
\definecolor{currentstroke}{rgb}{0.000000,0.000000,0.000000}%
\pgfsetstrokecolor{currentstroke}%
\pgfsetdash{}{0pt}%
\pgfpathmoveto{\pgfqpoint{4.355118in}{0.504323in}}%
\pgfpathlineto{\pgfqpoint{4.355118in}{0.410710in}}%
\pgfusepath{stroke}%
\end{pgfscope}%
\begin{pgfscope}%
\pgfpathrectangle{\pgfqpoint{4.123134in}{0.331635in}}{\pgfqpoint{2.583333in}{1.510000in}}%
\pgfusepath{clip}%
\pgfsetbuttcap%
\pgfsetroundjoin%
\pgfsetlinewidth{1.505625pt}%
\definecolor{currentstroke}{rgb}{0.000000,0.000000,0.000000}%
\pgfsetstrokecolor{currentstroke}%
\pgfsetdash{}{0pt}%
\pgfpathmoveto{\pgfqpoint{4.469678in}{0.504323in}}%
\pgfpathlineto{\pgfqpoint{4.469678in}{0.448803in}}%
\pgfusepath{stroke}%
\end{pgfscope}%
\begin{pgfscope}%
\pgfpathrectangle{\pgfqpoint{4.123134in}{0.331635in}}{\pgfqpoint{2.583333in}{1.510000in}}%
\pgfusepath{clip}%
\pgfsetbuttcap%
\pgfsetroundjoin%
\pgfsetlinewidth{1.505625pt}%
\definecolor{currentstroke}{rgb}{0.000000,0.000000,0.000000}%
\pgfsetstrokecolor{currentstroke}%
\pgfsetdash{}{0pt}%
\pgfpathmoveto{\pgfqpoint{4.584239in}{0.504323in}}%
\pgfpathlineto{\pgfqpoint{4.584239in}{0.458732in}}%
\pgfusepath{stroke}%
\end{pgfscope}%
\begin{pgfscope}%
\pgfpathrectangle{\pgfqpoint{4.123134in}{0.331635in}}{\pgfqpoint{2.583333in}{1.510000in}}%
\pgfusepath{clip}%
\pgfsetbuttcap%
\pgfsetroundjoin%
\pgfsetlinewidth{1.505625pt}%
\definecolor{currentstroke}{rgb}{0.000000,0.000000,0.000000}%
\pgfsetstrokecolor{currentstroke}%
\pgfsetdash{}{0pt}%
\pgfpathmoveto{\pgfqpoint{4.698799in}{0.504323in}}%
\pgfpathlineto{\pgfqpoint{4.698799in}{0.430045in}}%
\pgfusepath{stroke}%
\end{pgfscope}%
\begin{pgfscope}%
\pgfpathrectangle{\pgfqpoint{4.123134in}{0.331635in}}{\pgfqpoint{2.583333in}{1.510000in}}%
\pgfusepath{clip}%
\pgfsetbuttcap%
\pgfsetroundjoin%
\pgfsetlinewidth{1.505625pt}%
\definecolor{currentstroke}{rgb}{0.000000,0.000000,0.000000}%
\pgfsetstrokecolor{currentstroke}%
\pgfsetdash{}{0pt}%
\pgfpathmoveto{\pgfqpoint{4.813359in}{0.504323in}}%
\pgfpathlineto{\pgfqpoint{4.813359in}{0.445966in}}%
\pgfusepath{stroke}%
\end{pgfscope}%
\begin{pgfscope}%
\pgfpathrectangle{\pgfqpoint{4.123134in}{0.331635in}}{\pgfqpoint{2.583333in}{1.510000in}}%
\pgfusepath{clip}%
\pgfsetbuttcap%
\pgfsetroundjoin%
\pgfsetlinewidth{1.505625pt}%
\definecolor{currentstroke}{rgb}{0.000000,0.000000,0.000000}%
\pgfsetstrokecolor{currentstroke}%
\pgfsetdash{}{0pt}%
\pgfpathmoveto{\pgfqpoint{4.927919in}{0.504323in}}%
\pgfpathlineto{\pgfqpoint{4.927919in}{0.503755in}}%
\pgfusepath{stroke}%
\end{pgfscope}%
\begin{pgfscope}%
\pgfpathrectangle{\pgfqpoint{4.123134in}{0.331635in}}{\pgfqpoint{2.583333in}{1.510000in}}%
\pgfusepath{clip}%
\pgfsetbuttcap%
\pgfsetroundjoin%
\pgfsetlinewidth{1.505625pt}%
\definecolor{currentstroke}{rgb}{0.000000,0.000000,0.000000}%
\pgfsetstrokecolor{currentstroke}%
\pgfsetdash{}{0pt}%
\pgfpathmoveto{\pgfqpoint{5.042480in}{0.504323in}}%
\pgfpathlineto{\pgfqpoint{5.042480in}{0.490101in}}%
\pgfusepath{stroke}%
\end{pgfscope}%
\begin{pgfscope}%
\pgfpathrectangle{\pgfqpoint{4.123134in}{0.331635in}}{\pgfqpoint{2.583333in}{1.510000in}}%
\pgfusepath{clip}%
\pgfsetbuttcap%
\pgfsetroundjoin%
\pgfsetlinewidth{1.505625pt}%
\definecolor{currentstroke}{rgb}{0.000000,0.000000,0.000000}%
\pgfsetstrokecolor{currentstroke}%
\pgfsetdash{}{0pt}%
\pgfpathmoveto{\pgfqpoint{5.157040in}{0.504323in}}%
\pgfpathlineto{\pgfqpoint{5.157040in}{0.580994in}}%
\pgfusepath{stroke}%
\end{pgfscope}%
\begin{pgfscope}%
\pgfpathrectangle{\pgfqpoint{4.123134in}{0.331635in}}{\pgfqpoint{2.583333in}{1.510000in}}%
\pgfusepath{clip}%
\pgfsetbuttcap%
\pgfsetroundjoin%
\pgfsetlinewidth{1.505625pt}%
\definecolor{currentstroke}{rgb}{0.000000,0.000000,0.000000}%
\pgfsetstrokecolor{currentstroke}%
\pgfsetdash{}{0pt}%
\pgfpathmoveto{\pgfqpoint{5.271600in}{0.504323in}}%
\pgfpathlineto{\pgfqpoint{5.271600in}{0.489705in}}%
\pgfusepath{stroke}%
\end{pgfscope}%
\begin{pgfscope}%
\pgfpathrectangle{\pgfqpoint{4.123134in}{0.331635in}}{\pgfqpoint{2.583333in}{1.510000in}}%
\pgfusepath{clip}%
\pgfsetbuttcap%
\pgfsetroundjoin%
\pgfsetlinewidth{1.505625pt}%
\definecolor{currentstroke}{rgb}{0.000000,0.000000,0.000000}%
\pgfsetstrokecolor{currentstroke}%
\pgfsetdash{}{0pt}%
\pgfpathmoveto{\pgfqpoint{5.386160in}{0.504323in}}%
\pgfpathlineto{\pgfqpoint{5.386160in}{0.547158in}}%
\pgfusepath{stroke}%
\end{pgfscope}%
\begin{pgfscope}%
\pgfpathrectangle{\pgfqpoint{4.123134in}{0.331635in}}{\pgfqpoint{2.583333in}{1.510000in}}%
\pgfusepath{clip}%
\pgfsetbuttcap%
\pgfsetroundjoin%
\pgfsetlinewidth{1.505625pt}%
\definecolor{currentstroke}{rgb}{0.000000,0.000000,0.000000}%
\pgfsetstrokecolor{currentstroke}%
\pgfsetdash{}{0pt}%
\pgfpathmoveto{\pgfqpoint{5.500721in}{0.504323in}}%
\pgfpathlineto{\pgfqpoint{5.500721in}{0.526902in}}%
\pgfusepath{stroke}%
\end{pgfscope}%
\begin{pgfscope}%
\pgfpathrectangle{\pgfqpoint{4.123134in}{0.331635in}}{\pgfqpoint{2.583333in}{1.510000in}}%
\pgfusepath{clip}%
\pgfsetbuttcap%
\pgfsetroundjoin%
\pgfsetlinewidth{1.505625pt}%
\definecolor{currentstroke}{rgb}{0.000000,0.000000,0.000000}%
\pgfsetstrokecolor{currentstroke}%
\pgfsetdash{}{0pt}%
\pgfpathmoveto{\pgfqpoint{5.615281in}{0.504323in}}%
\pgfpathlineto{\pgfqpoint{5.615281in}{0.508693in}}%
\pgfusepath{stroke}%
\end{pgfscope}%
\begin{pgfscope}%
\pgfpathrectangle{\pgfqpoint{4.123134in}{0.331635in}}{\pgfqpoint{2.583333in}{1.510000in}}%
\pgfusepath{clip}%
\pgfsetbuttcap%
\pgfsetroundjoin%
\pgfsetlinewidth{1.505625pt}%
\definecolor{currentstroke}{rgb}{0.000000,0.000000,0.000000}%
\pgfsetstrokecolor{currentstroke}%
\pgfsetdash{}{0pt}%
\pgfpathmoveto{\pgfqpoint{5.729841in}{0.504323in}}%
\pgfpathlineto{\pgfqpoint{5.729841in}{0.503358in}}%
\pgfusepath{stroke}%
\end{pgfscope}%
\begin{pgfscope}%
\pgfpathrectangle{\pgfqpoint{4.123134in}{0.331635in}}{\pgfqpoint{2.583333in}{1.510000in}}%
\pgfusepath{clip}%
\pgfsetbuttcap%
\pgfsetroundjoin%
\pgfsetlinewidth{1.505625pt}%
\definecolor{currentstroke}{rgb}{0.000000,0.000000,0.000000}%
\pgfsetstrokecolor{currentstroke}%
\pgfsetdash{}{0pt}%
\pgfpathmoveto{\pgfqpoint{5.844401in}{0.504323in}}%
\pgfpathlineto{\pgfqpoint{5.844401in}{0.495101in}}%
\pgfusepath{stroke}%
\end{pgfscope}%
\begin{pgfscope}%
\pgfpathrectangle{\pgfqpoint{4.123134in}{0.331635in}}{\pgfqpoint{2.583333in}{1.510000in}}%
\pgfusepath{clip}%
\pgfsetbuttcap%
\pgfsetroundjoin%
\pgfsetlinewidth{1.505625pt}%
\definecolor{currentstroke}{rgb}{0.000000,0.000000,0.000000}%
\pgfsetstrokecolor{currentstroke}%
\pgfsetdash{}{0pt}%
\pgfpathmoveto{\pgfqpoint{5.958962in}{0.504323in}}%
\pgfpathlineto{\pgfqpoint{5.958962in}{0.400271in}}%
\pgfusepath{stroke}%
\end{pgfscope}%
\begin{pgfscope}%
\pgfpathrectangle{\pgfqpoint{4.123134in}{0.331635in}}{\pgfqpoint{2.583333in}{1.510000in}}%
\pgfusepath{clip}%
\pgfsetbuttcap%
\pgfsetroundjoin%
\pgfsetlinewidth{1.505625pt}%
\definecolor{currentstroke}{rgb}{0.000000,0.000000,0.000000}%
\pgfsetstrokecolor{currentstroke}%
\pgfsetdash{}{0pt}%
\pgfpathmoveto{\pgfqpoint{6.073522in}{0.504323in}}%
\pgfpathlineto{\pgfqpoint{6.073522in}{0.479252in}}%
\pgfusepath{stroke}%
\end{pgfscope}%
\begin{pgfscope}%
\pgfpathrectangle{\pgfqpoint{4.123134in}{0.331635in}}{\pgfqpoint{2.583333in}{1.510000in}}%
\pgfusepath{clip}%
\pgfsetbuttcap%
\pgfsetroundjoin%
\pgfsetlinewidth{1.505625pt}%
\definecolor{currentstroke}{rgb}{0.000000,0.000000,0.000000}%
\pgfsetstrokecolor{currentstroke}%
\pgfsetdash{}{0pt}%
\pgfpathmoveto{\pgfqpoint{6.188082in}{0.504323in}}%
\pgfpathlineto{\pgfqpoint{6.188082in}{0.508120in}}%
\pgfusepath{stroke}%
\end{pgfscope}%
\begin{pgfscope}%
\pgfpathrectangle{\pgfqpoint{4.123134in}{0.331635in}}{\pgfqpoint{2.583333in}{1.510000in}}%
\pgfusepath{clip}%
\pgfsetbuttcap%
\pgfsetroundjoin%
\pgfsetlinewidth{1.505625pt}%
\definecolor{currentstroke}{rgb}{0.000000,0.000000,0.000000}%
\pgfsetstrokecolor{currentstroke}%
\pgfsetdash{}{0pt}%
\pgfpathmoveto{\pgfqpoint{6.302642in}{0.504323in}}%
\pgfpathlineto{\pgfqpoint{6.302642in}{0.472725in}}%
\pgfusepath{stroke}%
\end{pgfscope}%
\begin{pgfscope}%
\pgfpathrectangle{\pgfqpoint{4.123134in}{0.331635in}}{\pgfqpoint{2.583333in}{1.510000in}}%
\pgfusepath{clip}%
\pgfsetbuttcap%
\pgfsetroundjoin%
\pgfsetlinewidth{1.505625pt}%
\definecolor{currentstroke}{rgb}{0.000000,0.000000,0.000000}%
\pgfsetstrokecolor{currentstroke}%
\pgfsetdash{}{0pt}%
\pgfpathmoveto{\pgfqpoint{6.417202in}{0.504323in}}%
\pgfpathlineto{\pgfqpoint{6.417202in}{0.502799in}}%
\pgfusepath{stroke}%
\end{pgfscope}%
\begin{pgfscope}%
\pgfpathrectangle{\pgfqpoint{4.123134in}{0.331635in}}{\pgfqpoint{2.583333in}{1.510000in}}%
\pgfusepath{clip}%
\pgfsetbuttcap%
\pgfsetroundjoin%
\pgfsetlinewidth{1.505625pt}%
\definecolor{currentstroke}{rgb}{0.000000,0.000000,0.000000}%
\pgfsetstrokecolor{currentstroke}%
\pgfsetdash{}{0pt}%
\pgfpathmoveto{\pgfqpoint{6.531763in}{0.504323in}}%
\pgfpathlineto{\pgfqpoint{6.531763in}{0.530890in}}%
\pgfusepath{stroke}%
\end{pgfscope}%
\begin{pgfscope}%
\pgfpathrectangle{\pgfqpoint{4.123134in}{0.331635in}}{\pgfqpoint{2.583333in}{1.510000in}}%
\pgfusepath{clip}%
\pgfsetroundcap%
\pgfsetroundjoin%
\pgfsetlinewidth{1.505625pt}%
\definecolor{currentstroke}{rgb}{0.737255,0.741176,0.133333}%
\pgfsetstrokecolor{currentstroke}%
\pgfsetdash{}{0pt}%
\pgfpathmoveto{\pgfqpoint{4.123134in}{0.504323in}}%
\pgfpathlineto{\pgfqpoint{6.706467in}{0.504323in}}%
\pgfusepath{stroke}%
\end{pgfscope}%
\begin{pgfscope}%
\pgfpathrectangle{\pgfqpoint{4.123134in}{0.331635in}}{\pgfqpoint{2.583333in}{1.510000in}}%
\pgfusepath{clip}%
\pgfsetbuttcap%
\pgfsetroundjoin%
\definecolor{currentfill}{rgb}{0.737255,0.741176,0.133333}%
\pgfsetfillcolor{currentfill}%
\pgfsetlinewidth{1.003750pt}%
\definecolor{currentstroke}{rgb}{0.737255,0.741176,0.133333}%
\pgfsetstrokecolor{currentstroke}%
\pgfsetdash{}{0pt}%
\pgfsys@defobject{currentmarker}{\pgfqpoint{-0.034722in}{-0.034722in}}{\pgfqpoint{0.034722in}{0.034722in}}{%
\pgfpathmoveto{\pgfqpoint{0.000000in}{-0.034722in}}%
\pgfpathcurveto{\pgfqpoint{0.009208in}{-0.034722in}}{\pgfqpoint{0.018041in}{-0.031064in}}{\pgfqpoint{0.024552in}{-0.024552in}}%
\pgfpathcurveto{\pgfqpoint{0.031064in}{-0.018041in}}{\pgfqpoint{0.034722in}{-0.009208in}}{\pgfqpoint{0.034722in}{0.000000in}}%
\pgfpathcurveto{\pgfqpoint{0.034722in}{0.009208in}}{\pgfqpoint{0.031064in}{0.018041in}}{\pgfqpoint{0.024552in}{0.024552in}}%
\pgfpathcurveto{\pgfqpoint{0.018041in}{0.031064in}}{\pgfqpoint{0.009208in}{0.034722in}}{\pgfqpoint{0.000000in}{0.034722in}}%
\pgfpathcurveto{\pgfqpoint{-0.009208in}{0.034722in}}{\pgfqpoint{-0.018041in}{0.031064in}}{\pgfqpoint{-0.024552in}{0.024552in}}%
\pgfpathcurveto{\pgfqpoint{-0.031064in}{0.018041in}}{\pgfqpoint{-0.034722in}{0.009208in}}{\pgfqpoint{-0.034722in}{0.000000in}}%
\pgfpathcurveto{\pgfqpoint{-0.034722in}{-0.009208in}}{\pgfqpoint{-0.031064in}{-0.018041in}}{\pgfqpoint{-0.024552in}{-0.024552in}}%
\pgfpathcurveto{\pgfqpoint{-0.018041in}{-0.031064in}}{\pgfqpoint{-0.009208in}{-0.034722in}}{\pgfqpoint{0.000000in}{-0.034722in}}%
\pgfpathclose%
\pgfusepath{stroke,fill}%
}%
\begin{pgfscope}%
\pgfsys@transformshift{4.240558in}{1.772999in}%
\pgfsys@useobject{currentmarker}{}%
\end{pgfscope}%
\begin{pgfscope}%
\pgfsys@transformshift{4.355118in}{0.410710in}%
\pgfsys@useobject{currentmarker}{}%
\end{pgfscope}%
\begin{pgfscope}%
\pgfsys@transformshift{4.469678in}{0.448803in}%
\pgfsys@useobject{currentmarker}{}%
\end{pgfscope}%
\begin{pgfscope}%
\pgfsys@transformshift{4.584239in}{0.458732in}%
\pgfsys@useobject{currentmarker}{}%
\end{pgfscope}%
\begin{pgfscope}%
\pgfsys@transformshift{4.698799in}{0.430045in}%
\pgfsys@useobject{currentmarker}{}%
\end{pgfscope}%
\begin{pgfscope}%
\pgfsys@transformshift{4.813359in}{0.445966in}%
\pgfsys@useobject{currentmarker}{}%
\end{pgfscope}%
\begin{pgfscope}%
\pgfsys@transformshift{4.927919in}{0.503755in}%
\pgfsys@useobject{currentmarker}{}%
\end{pgfscope}%
\begin{pgfscope}%
\pgfsys@transformshift{5.042480in}{0.490101in}%
\pgfsys@useobject{currentmarker}{}%
\end{pgfscope}%
\begin{pgfscope}%
\pgfsys@transformshift{5.157040in}{0.580994in}%
\pgfsys@useobject{currentmarker}{}%
\end{pgfscope}%
\begin{pgfscope}%
\pgfsys@transformshift{5.271600in}{0.489705in}%
\pgfsys@useobject{currentmarker}{}%
\end{pgfscope}%
\begin{pgfscope}%
\pgfsys@transformshift{5.386160in}{0.547158in}%
\pgfsys@useobject{currentmarker}{}%
\end{pgfscope}%
\begin{pgfscope}%
\pgfsys@transformshift{5.500721in}{0.526902in}%
\pgfsys@useobject{currentmarker}{}%
\end{pgfscope}%
\begin{pgfscope}%
\pgfsys@transformshift{5.615281in}{0.508693in}%
\pgfsys@useobject{currentmarker}{}%
\end{pgfscope}%
\begin{pgfscope}%
\pgfsys@transformshift{5.729841in}{0.503358in}%
\pgfsys@useobject{currentmarker}{}%
\end{pgfscope}%
\begin{pgfscope}%
\pgfsys@transformshift{5.844401in}{0.495101in}%
\pgfsys@useobject{currentmarker}{}%
\end{pgfscope}%
\begin{pgfscope}%
\pgfsys@transformshift{5.958962in}{0.400271in}%
\pgfsys@useobject{currentmarker}{}%
\end{pgfscope}%
\begin{pgfscope}%
\pgfsys@transformshift{6.073522in}{0.479252in}%
\pgfsys@useobject{currentmarker}{}%
\end{pgfscope}%
\begin{pgfscope}%
\pgfsys@transformshift{6.188082in}{0.508120in}%
\pgfsys@useobject{currentmarker}{}%
\end{pgfscope}%
\begin{pgfscope}%
\pgfsys@transformshift{6.302642in}{0.472725in}%
\pgfsys@useobject{currentmarker}{}%
\end{pgfscope}%
\begin{pgfscope}%
\pgfsys@transformshift{6.417202in}{0.502799in}%
\pgfsys@useobject{currentmarker}{}%
\end{pgfscope}%
\begin{pgfscope}%
\pgfsys@transformshift{6.531763in}{0.530890in}%
\pgfsys@useobject{currentmarker}{}%
\end{pgfscope}%
\end{pgfscope}%
\begin{pgfscope}%
\pgfsetrectcap%
\pgfsetmiterjoin%
\pgfsetlinewidth{0.803000pt}%
\definecolor{currentstroke}{rgb}{1.000000,1.000000,1.000000}%
\pgfsetstrokecolor{currentstroke}%
\pgfsetdash{}{0pt}%
\pgfpathmoveto{\pgfqpoint{4.123134in}{0.331635in}}%
\pgfpathlineto{\pgfqpoint{4.123134in}{1.841635in}}%
\pgfusepath{stroke}%
\end{pgfscope}%
\begin{pgfscope}%
\pgfsetrectcap%
\pgfsetmiterjoin%
\pgfsetlinewidth{0.803000pt}%
\definecolor{currentstroke}{rgb}{1.000000,1.000000,1.000000}%
\pgfsetstrokecolor{currentstroke}%
\pgfsetdash{}{0pt}%
\pgfpathmoveto{\pgfqpoint{6.706467in}{0.331635in}}%
\pgfpathlineto{\pgfqpoint{6.706467in}{1.841635in}}%
\pgfusepath{stroke}%
\end{pgfscope}%
\begin{pgfscope}%
\pgfsetrectcap%
\pgfsetmiterjoin%
\pgfsetlinewidth{0.803000pt}%
\definecolor{currentstroke}{rgb}{1.000000,1.000000,1.000000}%
\pgfsetstrokecolor{currentstroke}%
\pgfsetdash{}{0pt}%
\pgfpathmoveto{\pgfqpoint{4.123134in}{0.331635in}}%
\pgfpathlineto{\pgfqpoint{6.706467in}{0.331635in}}%
\pgfusepath{stroke}%
\end{pgfscope}%
\begin{pgfscope}%
\pgfsetrectcap%
\pgfsetmiterjoin%
\pgfsetlinewidth{0.803000pt}%
\definecolor{currentstroke}{rgb}{1.000000,1.000000,1.000000}%
\pgfsetstrokecolor{currentstroke}%
\pgfsetdash{}{0pt}%
\pgfpathmoveto{\pgfqpoint{4.123134in}{1.841635in}}%
\pgfpathlineto{\pgfqpoint{6.706467in}{1.841635in}}%
\pgfusepath{stroke}%
\end{pgfscope}%
\begin{pgfscope}%
\definecolor{textcolor}{rgb}{0.150000,0.150000,0.150000}%
\pgfsetstrokecolor{textcolor}%
\pgfsetfillcolor{textcolor}%
\pgftext[x=5.414800in,y=1.924968in,,base]{\color{textcolor}\rmfamily\fontsize{12.000000}{14.400000}\selectfont Partial Autocorrelation}%
\end{pgfscope}%
\end{pgfpicture}%
\makeatother%
\endgroup%

    \end{adjustbox} 
    \caption{}
    \label{fig:INTC_V_ACF_log_returns}
\end{figure}{}

We start by applying a formal test for autocorrelation to the time series, the Box/Pierce and Ljung/Box tests. EXPLANATION / Source. They have slightly different properties regarding their handling of very large and very small numbers of observations, but for both the null hypothesis is that there is no autocorrelation in the series. Figure \ref{fig:ljungbox} shows the p-values for the first 40 lags. The test suggests that for V there may be some significant autocorrelations that could be used for modeling. Compared to the plot of ACF and PACF however, the test seems overly optimistic. For INTC, the test confirms non-significance for the first few lags. While higher order lags may be significant, modeling a time series process with that many coefficients is almost certain to overfit the data. 

\begin{figure}[h!]
    \centering
    \figuretitle{Ljung-Box and Box-Pierce Test for Autocorrelation}
    \begin{adjustbox}{width=.95\textwidth,center}
    %% Creator: Matplotlib, PGF backend
%%
%% To include the figure in your LaTeX document, write
%%   \input{<filename>.pgf}
%%
%% Make sure the required packages are loaded in your preamble
%%   \usepackage{pgf}
%%
%% Figures using additional raster images can only be included by \input if
%% they are in the same directory as the main LaTeX file. For loading figures
%% from other directories you can use the `import` package
%%   \usepackage{import}
%% and then include the figures with
%%   \import{<path to file>}{<filename>.pgf}
%%
%% Matplotlib used the following preamble
%%   \usepackage{fontspec}
%%   \setmainfont{DejaVuSerif.ttf}[Path=/opt/tljh/user/lib/python3.6/site-packages/matplotlib/mpl-data/fonts/ttf/]
%%   \setsansfont{DejaVuSans.ttf}[Path=/opt/tljh/user/lib/python3.6/site-packages/matplotlib/mpl-data/fonts/ttf/]
%%   \setmonofont{DejaVuSansMono.ttf}[Path=/opt/tljh/user/lib/python3.6/site-packages/matplotlib/mpl-data/fonts/ttf/]
%%
\begingroup%
\makeatletter%
\begin{pgfpicture}%
\pgfpathrectangle{\pgfpointorigin}{\pgfqpoint{5.446435in}{3.684685in}}%
\pgfusepath{use as bounding box, clip}%
\begin{pgfscope}%
\pgfsetbuttcap%
\pgfsetmiterjoin%
\definecolor{currentfill}{rgb}{1.000000,1.000000,1.000000}%
\pgfsetfillcolor{currentfill}%
\pgfsetlinewidth{0.000000pt}%
\definecolor{currentstroke}{rgb}{1.000000,1.000000,1.000000}%
\pgfsetstrokecolor{currentstroke}%
\pgfsetdash{}{0pt}%
\pgfpathmoveto{\pgfqpoint{0.000000in}{0.000000in}}%
\pgfpathlineto{\pgfqpoint{5.446435in}{0.000000in}}%
\pgfpathlineto{\pgfqpoint{5.446435in}{3.684685in}}%
\pgfpathlineto{\pgfqpoint{0.000000in}{3.684685in}}%
\pgfpathclose%
\pgfusepath{fill}%
\end{pgfscope}%
\begin{pgfscope}%
\pgfsetbuttcap%
\pgfsetmiterjoin%
\definecolor{currentfill}{rgb}{0.917647,0.917647,0.949020}%
\pgfsetfillcolor{currentfill}%
\pgfsetlinewidth{0.000000pt}%
\definecolor{currentstroke}{rgb}{0.000000,0.000000,0.000000}%
\pgfsetstrokecolor{currentstroke}%
\pgfsetstrokeopacity{0.000000}%
\pgfsetdash{}{0pt}%
\pgfpathmoveto{\pgfqpoint{0.696435in}{0.523570in}}%
\pgfpathlineto{\pgfqpoint{5.346435in}{0.523570in}}%
\pgfpathlineto{\pgfqpoint{5.346435in}{3.543570in}}%
\pgfpathlineto{\pgfqpoint{0.696435in}{3.543570in}}%
\pgfpathclose%
\pgfusepath{fill}%
\end{pgfscope}%
\begin{pgfscope}%
\pgfpathrectangle{\pgfqpoint{0.696435in}{0.523570in}}{\pgfqpoint{4.650000in}{3.020000in}}%
\pgfusepath{clip}%
\pgfsetroundcap%
\pgfsetroundjoin%
\pgfsetlinewidth{0.803000pt}%
\definecolor{currentstroke}{rgb}{1.000000,1.000000,1.000000}%
\pgfsetstrokecolor{currentstroke}%
\pgfsetdash{}{0pt}%
\pgfpathmoveto{\pgfqpoint{0.907799in}{0.523570in}}%
\pgfpathlineto{\pgfqpoint{0.907799in}{3.543570in}}%
\pgfusepath{stroke}%
\end{pgfscope}%
\begin{pgfscope}%
\definecolor{textcolor}{rgb}{0.150000,0.150000,0.150000}%
\pgfsetstrokecolor{textcolor}%
\pgfsetfillcolor{textcolor}%
\pgftext[x=0.907799in,y=0.426348in,,top]{\color{textcolor}\rmfamily\fontsize{10.000000}{12.000000}\selectfont 0}%
\end{pgfscope}%
\begin{pgfscope}%
\pgfpathrectangle{\pgfqpoint{0.696435in}{0.523570in}}{\pgfqpoint{4.650000in}{3.020000in}}%
\pgfusepath{clip}%
\pgfsetroundcap%
\pgfsetroundjoin%
\pgfsetlinewidth{0.803000pt}%
\definecolor{currentstroke}{rgb}{1.000000,1.000000,1.000000}%
\pgfsetstrokecolor{currentstroke}%
\pgfsetdash{}{0pt}%
\pgfpathmoveto{\pgfqpoint{1.449757in}{0.523570in}}%
\pgfpathlineto{\pgfqpoint{1.449757in}{3.543570in}}%
\pgfusepath{stroke}%
\end{pgfscope}%
\begin{pgfscope}%
\definecolor{textcolor}{rgb}{0.150000,0.150000,0.150000}%
\pgfsetstrokecolor{textcolor}%
\pgfsetfillcolor{textcolor}%
\pgftext[x=1.449757in,y=0.426348in,,top]{\color{textcolor}\rmfamily\fontsize{10.000000}{12.000000}\selectfont 5}%
\end{pgfscope}%
\begin{pgfscope}%
\pgfpathrectangle{\pgfqpoint{0.696435in}{0.523570in}}{\pgfqpoint{4.650000in}{3.020000in}}%
\pgfusepath{clip}%
\pgfsetroundcap%
\pgfsetroundjoin%
\pgfsetlinewidth{0.803000pt}%
\definecolor{currentstroke}{rgb}{1.000000,1.000000,1.000000}%
\pgfsetstrokecolor{currentstroke}%
\pgfsetdash{}{0pt}%
\pgfpathmoveto{\pgfqpoint{1.991715in}{0.523570in}}%
\pgfpathlineto{\pgfqpoint{1.991715in}{3.543570in}}%
\pgfusepath{stroke}%
\end{pgfscope}%
\begin{pgfscope}%
\definecolor{textcolor}{rgb}{0.150000,0.150000,0.150000}%
\pgfsetstrokecolor{textcolor}%
\pgfsetfillcolor{textcolor}%
\pgftext[x=1.991715in,y=0.426348in,,top]{\color{textcolor}\rmfamily\fontsize{10.000000}{12.000000}\selectfont 10}%
\end{pgfscope}%
\begin{pgfscope}%
\pgfpathrectangle{\pgfqpoint{0.696435in}{0.523570in}}{\pgfqpoint{4.650000in}{3.020000in}}%
\pgfusepath{clip}%
\pgfsetroundcap%
\pgfsetroundjoin%
\pgfsetlinewidth{0.803000pt}%
\definecolor{currentstroke}{rgb}{1.000000,1.000000,1.000000}%
\pgfsetstrokecolor{currentstroke}%
\pgfsetdash{}{0pt}%
\pgfpathmoveto{\pgfqpoint{2.533673in}{0.523570in}}%
\pgfpathlineto{\pgfqpoint{2.533673in}{3.543570in}}%
\pgfusepath{stroke}%
\end{pgfscope}%
\begin{pgfscope}%
\definecolor{textcolor}{rgb}{0.150000,0.150000,0.150000}%
\pgfsetstrokecolor{textcolor}%
\pgfsetfillcolor{textcolor}%
\pgftext[x=2.533673in,y=0.426348in,,top]{\color{textcolor}\rmfamily\fontsize{10.000000}{12.000000}\selectfont 15}%
\end{pgfscope}%
\begin{pgfscope}%
\pgfpathrectangle{\pgfqpoint{0.696435in}{0.523570in}}{\pgfqpoint{4.650000in}{3.020000in}}%
\pgfusepath{clip}%
\pgfsetroundcap%
\pgfsetroundjoin%
\pgfsetlinewidth{0.803000pt}%
\definecolor{currentstroke}{rgb}{1.000000,1.000000,1.000000}%
\pgfsetstrokecolor{currentstroke}%
\pgfsetdash{}{0pt}%
\pgfpathmoveto{\pgfqpoint{3.075631in}{0.523570in}}%
\pgfpathlineto{\pgfqpoint{3.075631in}{3.543570in}}%
\pgfusepath{stroke}%
\end{pgfscope}%
\begin{pgfscope}%
\definecolor{textcolor}{rgb}{0.150000,0.150000,0.150000}%
\pgfsetstrokecolor{textcolor}%
\pgfsetfillcolor{textcolor}%
\pgftext[x=3.075631in,y=0.426348in,,top]{\color{textcolor}\rmfamily\fontsize{10.000000}{12.000000}\selectfont 20}%
\end{pgfscope}%
\begin{pgfscope}%
\pgfpathrectangle{\pgfqpoint{0.696435in}{0.523570in}}{\pgfqpoint{4.650000in}{3.020000in}}%
\pgfusepath{clip}%
\pgfsetroundcap%
\pgfsetroundjoin%
\pgfsetlinewidth{0.803000pt}%
\definecolor{currentstroke}{rgb}{1.000000,1.000000,1.000000}%
\pgfsetstrokecolor{currentstroke}%
\pgfsetdash{}{0pt}%
\pgfpathmoveto{\pgfqpoint{3.617589in}{0.523570in}}%
\pgfpathlineto{\pgfqpoint{3.617589in}{3.543570in}}%
\pgfusepath{stroke}%
\end{pgfscope}%
\begin{pgfscope}%
\definecolor{textcolor}{rgb}{0.150000,0.150000,0.150000}%
\pgfsetstrokecolor{textcolor}%
\pgfsetfillcolor{textcolor}%
\pgftext[x=3.617589in,y=0.426348in,,top]{\color{textcolor}\rmfamily\fontsize{10.000000}{12.000000}\selectfont 25}%
\end{pgfscope}%
\begin{pgfscope}%
\pgfpathrectangle{\pgfqpoint{0.696435in}{0.523570in}}{\pgfqpoint{4.650000in}{3.020000in}}%
\pgfusepath{clip}%
\pgfsetroundcap%
\pgfsetroundjoin%
\pgfsetlinewidth{0.803000pt}%
\definecolor{currentstroke}{rgb}{1.000000,1.000000,1.000000}%
\pgfsetstrokecolor{currentstroke}%
\pgfsetdash{}{0pt}%
\pgfpathmoveto{\pgfqpoint{4.159547in}{0.523570in}}%
\pgfpathlineto{\pgfqpoint{4.159547in}{3.543570in}}%
\pgfusepath{stroke}%
\end{pgfscope}%
\begin{pgfscope}%
\definecolor{textcolor}{rgb}{0.150000,0.150000,0.150000}%
\pgfsetstrokecolor{textcolor}%
\pgfsetfillcolor{textcolor}%
\pgftext[x=4.159547in,y=0.426348in,,top]{\color{textcolor}\rmfamily\fontsize{10.000000}{12.000000}\selectfont 30}%
\end{pgfscope}%
\begin{pgfscope}%
\pgfpathrectangle{\pgfqpoint{0.696435in}{0.523570in}}{\pgfqpoint{4.650000in}{3.020000in}}%
\pgfusepath{clip}%
\pgfsetroundcap%
\pgfsetroundjoin%
\pgfsetlinewidth{0.803000pt}%
\definecolor{currentstroke}{rgb}{1.000000,1.000000,1.000000}%
\pgfsetstrokecolor{currentstroke}%
\pgfsetdash{}{0pt}%
\pgfpathmoveto{\pgfqpoint{4.701505in}{0.523570in}}%
\pgfpathlineto{\pgfqpoint{4.701505in}{3.543570in}}%
\pgfusepath{stroke}%
\end{pgfscope}%
\begin{pgfscope}%
\definecolor{textcolor}{rgb}{0.150000,0.150000,0.150000}%
\pgfsetstrokecolor{textcolor}%
\pgfsetfillcolor{textcolor}%
\pgftext[x=4.701505in,y=0.426348in,,top]{\color{textcolor}\rmfamily\fontsize{10.000000}{12.000000}\selectfont 35}%
\end{pgfscope}%
\begin{pgfscope}%
\pgfpathrectangle{\pgfqpoint{0.696435in}{0.523570in}}{\pgfqpoint{4.650000in}{3.020000in}}%
\pgfusepath{clip}%
\pgfsetroundcap%
\pgfsetroundjoin%
\pgfsetlinewidth{0.803000pt}%
\definecolor{currentstroke}{rgb}{1.000000,1.000000,1.000000}%
\pgfsetstrokecolor{currentstroke}%
\pgfsetdash{}{0pt}%
\pgfpathmoveto{\pgfqpoint{5.243463in}{0.523570in}}%
\pgfpathlineto{\pgfqpoint{5.243463in}{3.543570in}}%
\pgfusepath{stroke}%
\end{pgfscope}%
\begin{pgfscope}%
\definecolor{textcolor}{rgb}{0.150000,0.150000,0.150000}%
\pgfsetstrokecolor{textcolor}%
\pgfsetfillcolor{textcolor}%
\pgftext[x=5.243463in,y=0.426348in,,top]{\color{textcolor}\rmfamily\fontsize{10.000000}{12.000000}\selectfont 40}%
\end{pgfscope}%
\begin{pgfscope}%
\definecolor{textcolor}{rgb}{0.150000,0.150000,0.150000}%
\pgfsetstrokecolor{textcolor}%
\pgfsetfillcolor{textcolor}%
\pgftext[x=3.021435in,y=0.236379in,,top]{\color{textcolor}\rmfamily\fontsize{10.000000}{12.000000}\selectfont lag}%
\end{pgfscope}%
\begin{pgfscope}%
\pgfpathrectangle{\pgfqpoint{0.696435in}{0.523570in}}{\pgfqpoint{4.650000in}{3.020000in}}%
\pgfusepath{clip}%
\pgfsetroundcap%
\pgfsetroundjoin%
\pgfsetlinewidth{0.803000pt}%
\definecolor{currentstroke}{rgb}{1.000000,1.000000,1.000000}%
\pgfsetstrokecolor{currentstroke}%
\pgfsetdash{}{0pt}%
\pgfpathmoveto{\pgfqpoint{0.696435in}{0.605763in}}%
\pgfpathlineto{\pgfqpoint{5.346435in}{0.605763in}}%
\pgfusepath{stroke}%
\end{pgfscope}%
\begin{pgfscope}%
\definecolor{textcolor}{rgb}{0.150000,0.150000,0.150000}%
\pgfsetstrokecolor{textcolor}%
\pgfsetfillcolor{textcolor}%
\pgftext[x=0.289968in,y=0.553002in,left,base]{\color{textcolor}\rmfamily\fontsize{10.000000}{12.000000}\selectfont 0.00}%
\end{pgfscope}%
\begin{pgfscope}%
\pgfpathrectangle{\pgfqpoint{0.696435in}{0.523570in}}{\pgfqpoint{4.650000in}{3.020000in}}%
\pgfusepath{clip}%
\pgfsetroundcap%
\pgfsetroundjoin%
\pgfsetlinewidth{0.803000pt}%
\definecolor{currentstroke}{rgb}{1.000000,1.000000,1.000000}%
\pgfsetstrokecolor{currentstroke}%
\pgfsetdash{}{0pt}%
\pgfpathmoveto{\pgfqpoint{0.696435in}{1.023786in}}%
\pgfpathlineto{\pgfqpoint{5.346435in}{1.023786in}}%
\pgfusepath{stroke}%
\end{pgfscope}%
\begin{pgfscope}%
\definecolor{textcolor}{rgb}{0.150000,0.150000,0.150000}%
\pgfsetstrokecolor{textcolor}%
\pgfsetfillcolor{textcolor}%
\pgftext[x=0.289968in,y=0.971025in,left,base]{\color{textcolor}\rmfamily\fontsize{10.000000}{12.000000}\selectfont 0.01}%
\end{pgfscope}%
\begin{pgfscope}%
\pgfpathrectangle{\pgfqpoint{0.696435in}{0.523570in}}{\pgfqpoint{4.650000in}{3.020000in}}%
\pgfusepath{clip}%
\pgfsetroundcap%
\pgfsetroundjoin%
\pgfsetlinewidth{0.803000pt}%
\definecolor{currentstroke}{rgb}{1.000000,1.000000,1.000000}%
\pgfsetstrokecolor{currentstroke}%
\pgfsetdash{}{0pt}%
\pgfpathmoveto{\pgfqpoint{0.696435in}{1.441809in}}%
\pgfpathlineto{\pgfqpoint{5.346435in}{1.441809in}}%
\pgfusepath{stroke}%
\end{pgfscope}%
\begin{pgfscope}%
\definecolor{textcolor}{rgb}{0.150000,0.150000,0.150000}%
\pgfsetstrokecolor{textcolor}%
\pgfsetfillcolor{textcolor}%
\pgftext[x=0.289968in,y=1.389048in,left,base]{\color{textcolor}\rmfamily\fontsize{10.000000}{12.000000}\selectfont 0.02}%
\end{pgfscope}%
\begin{pgfscope}%
\pgfpathrectangle{\pgfqpoint{0.696435in}{0.523570in}}{\pgfqpoint{4.650000in}{3.020000in}}%
\pgfusepath{clip}%
\pgfsetroundcap%
\pgfsetroundjoin%
\pgfsetlinewidth{0.803000pt}%
\definecolor{currentstroke}{rgb}{1.000000,1.000000,1.000000}%
\pgfsetstrokecolor{currentstroke}%
\pgfsetdash{}{0pt}%
\pgfpathmoveto{\pgfqpoint{0.696435in}{1.859832in}}%
\pgfpathlineto{\pgfqpoint{5.346435in}{1.859832in}}%
\pgfusepath{stroke}%
\end{pgfscope}%
\begin{pgfscope}%
\definecolor{textcolor}{rgb}{0.150000,0.150000,0.150000}%
\pgfsetstrokecolor{textcolor}%
\pgfsetfillcolor{textcolor}%
\pgftext[x=0.289968in,y=1.807070in,left,base]{\color{textcolor}\rmfamily\fontsize{10.000000}{12.000000}\selectfont 0.03}%
\end{pgfscope}%
\begin{pgfscope}%
\pgfpathrectangle{\pgfqpoint{0.696435in}{0.523570in}}{\pgfqpoint{4.650000in}{3.020000in}}%
\pgfusepath{clip}%
\pgfsetroundcap%
\pgfsetroundjoin%
\pgfsetlinewidth{0.803000pt}%
\definecolor{currentstroke}{rgb}{1.000000,1.000000,1.000000}%
\pgfsetstrokecolor{currentstroke}%
\pgfsetdash{}{0pt}%
\pgfpathmoveto{\pgfqpoint{0.696435in}{2.277855in}}%
\pgfpathlineto{\pgfqpoint{5.346435in}{2.277855in}}%
\pgfusepath{stroke}%
\end{pgfscope}%
\begin{pgfscope}%
\definecolor{textcolor}{rgb}{0.150000,0.150000,0.150000}%
\pgfsetstrokecolor{textcolor}%
\pgfsetfillcolor{textcolor}%
\pgftext[x=0.289968in,y=2.225093in,left,base]{\color{textcolor}\rmfamily\fontsize{10.000000}{12.000000}\selectfont 0.04}%
\end{pgfscope}%
\begin{pgfscope}%
\pgfpathrectangle{\pgfqpoint{0.696435in}{0.523570in}}{\pgfqpoint{4.650000in}{3.020000in}}%
\pgfusepath{clip}%
\pgfsetroundcap%
\pgfsetroundjoin%
\pgfsetlinewidth{0.803000pt}%
\definecolor{currentstroke}{rgb}{1.000000,1.000000,1.000000}%
\pgfsetstrokecolor{currentstroke}%
\pgfsetdash{}{0pt}%
\pgfpathmoveto{\pgfqpoint{0.696435in}{2.695878in}}%
\pgfpathlineto{\pgfqpoint{5.346435in}{2.695878in}}%
\pgfusepath{stroke}%
\end{pgfscope}%
\begin{pgfscope}%
\definecolor{textcolor}{rgb}{0.150000,0.150000,0.150000}%
\pgfsetstrokecolor{textcolor}%
\pgfsetfillcolor{textcolor}%
\pgftext[x=0.289968in,y=2.643116in,left,base]{\color{textcolor}\rmfamily\fontsize{10.000000}{12.000000}\selectfont 0.05}%
\end{pgfscope}%
\begin{pgfscope}%
\pgfpathrectangle{\pgfqpoint{0.696435in}{0.523570in}}{\pgfqpoint{4.650000in}{3.020000in}}%
\pgfusepath{clip}%
\pgfsetroundcap%
\pgfsetroundjoin%
\pgfsetlinewidth{0.803000pt}%
\definecolor{currentstroke}{rgb}{1.000000,1.000000,1.000000}%
\pgfsetstrokecolor{currentstroke}%
\pgfsetdash{}{0pt}%
\pgfpathmoveto{\pgfqpoint{0.696435in}{3.113900in}}%
\pgfpathlineto{\pgfqpoint{5.346435in}{3.113900in}}%
\pgfusepath{stroke}%
\end{pgfscope}%
\begin{pgfscope}%
\definecolor{textcolor}{rgb}{0.150000,0.150000,0.150000}%
\pgfsetstrokecolor{textcolor}%
\pgfsetfillcolor{textcolor}%
\pgftext[x=0.289968in,y=3.061139in,left,base]{\color{textcolor}\rmfamily\fontsize{10.000000}{12.000000}\selectfont 0.06}%
\end{pgfscope}%
\begin{pgfscope}%
\pgfpathrectangle{\pgfqpoint{0.696435in}{0.523570in}}{\pgfqpoint{4.650000in}{3.020000in}}%
\pgfusepath{clip}%
\pgfsetroundcap%
\pgfsetroundjoin%
\pgfsetlinewidth{0.803000pt}%
\definecolor{currentstroke}{rgb}{1.000000,1.000000,1.000000}%
\pgfsetstrokecolor{currentstroke}%
\pgfsetdash{}{0pt}%
\pgfpathmoveto{\pgfqpoint{0.696435in}{3.531923in}}%
\pgfpathlineto{\pgfqpoint{5.346435in}{3.531923in}}%
\pgfusepath{stroke}%
\end{pgfscope}%
\begin{pgfscope}%
\definecolor{textcolor}{rgb}{0.150000,0.150000,0.150000}%
\pgfsetstrokecolor{textcolor}%
\pgfsetfillcolor{textcolor}%
\pgftext[x=0.289968in,y=3.479162in,left,base]{\color{textcolor}\rmfamily\fontsize{10.000000}{12.000000}\selectfont 0.07}%
\end{pgfscope}%
\begin{pgfscope}%
\definecolor{textcolor}{rgb}{0.150000,0.150000,0.150000}%
\pgfsetstrokecolor{textcolor}%
\pgfsetfillcolor{textcolor}%
\pgftext[x=0.234413in,y=2.033570in,,bottom,rotate=90.000000]{\color{textcolor}\rmfamily\fontsize{10.000000}{12.000000}\selectfont p-value}%
\end{pgfscope}%
\begin{pgfscope}%
\pgfpathrectangle{\pgfqpoint{0.696435in}{0.523570in}}{\pgfqpoint{4.650000in}{3.020000in}}%
\pgfusepath{clip}%
\pgfsetroundcap%
\pgfsetroundjoin%
\pgfsetlinewidth{1.505625pt}%
\definecolor{currentstroke}{rgb}{0.121569,0.466667,0.705882}%
\pgfsetstrokecolor{currentstroke}%
\pgfsetdash{}{0pt}%
\pgfpathmoveto{\pgfqpoint{0.907799in}{0.780583in}}%
\pgfpathlineto{\pgfqpoint{1.016191in}{0.837014in}}%
\pgfpathlineto{\pgfqpoint{1.124582in}{0.960010in}}%
\pgfpathlineto{\pgfqpoint{1.232974in}{0.752383in}}%
\pgfpathlineto{\pgfqpoint{1.341365in}{0.764556in}}%
\pgfpathlineto{\pgfqpoint{1.449757in}{0.908977in}}%
\pgfpathlineto{\pgfqpoint{1.558149in}{1.175991in}}%
\pgfpathlineto{\pgfqpoint{1.666540in}{0.692372in}}%
\pgfpathlineto{\pgfqpoint{1.774932in}{0.744485in}}%
\pgfpathlineto{\pgfqpoint{1.883324in}{0.755218in}}%
\pgfpathlineto{\pgfqpoint{1.991715in}{0.854582in}}%
\pgfpathlineto{\pgfqpoint{2.100107in}{1.008127in}}%
\pgfpathlineto{\pgfqpoint{2.208498in}{1.230096in}}%
\pgfpathlineto{\pgfqpoint{2.316890in}{1.530868in}}%
\pgfpathlineto{\pgfqpoint{2.425282in}{0.660843in}}%
\pgfpathlineto{\pgfqpoint{2.533673in}{0.696887in}}%
\pgfpathlineto{\pgfqpoint{2.642065in}{0.741513in}}%
\pgfpathlineto{\pgfqpoint{2.750456in}{0.802085in}}%
\pgfpathlineto{\pgfqpoint{2.858848in}{0.885659in}}%
\pgfpathlineto{\pgfqpoint{2.967240in}{0.906492in}}%
\pgfpathlineto{\pgfqpoint{3.075631in}{1.036846in}}%
\pgfpathlineto{\pgfqpoint{3.184023in}{1.010567in}}%
\pgfpathlineto{\pgfqpoint{3.292414in}{1.183684in}}%
\pgfpathlineto{\pgfqpoint{3.400806in}{1.413353in}}%
\pgfpathlineto{\pgfqpoint{3.509198in}{1.711652in}}%
\pgfpathlineto{\pgfqpoint{3.617589in}{2.049959in}}%
\pgfpathlineto{\pgfqpoint{3.725981in}{2.480307in}}%
\pgfpathlineto{\pgfqpoint{3.834372in}{2.662037in}}%
\pgfpathlineto{\pgfqpoint{3.942764in}{3.239471in}}%
\pgfpathlineto{\pgfqpoint{4.051156in}{2.332387in}}%
\pgfpathlineto{\pgfqpoint{4.159547in}{2.643171in}}%
\pgfpathlineto{\pgfqpoint{4.267939in}{2.003165in}}%
\pgfpathlineto{\pgfqpoint{4.376331in}{1.132292in}}%
\pgfpathlineto{\pgfqpoint{4.484722in}{1.289658in}}%
\pgfpathlineto{\pgfqpoint{4.593114in}{1.383251in}}%
\pgfpathlineto{\pgfqpoint{4.701505in}{1.581711in}}%
\pgfpathlineto{\pgfqpoint{4.809897in}{1.823074in}}%
\pgfpathlineto{\pgfqpoint{4.918289in}{2.148997in}}%
\pgfpathlineto{\pgfqpoint{5.026680in}{1.826133in}}%
\pgfpathlineto{\pgfqpoint{5.135072in}{1.944291in}}%
\pgfusepath{stroke}%
\end{pgfscope}%
\begin{pgfscope}%
\pgfpathrectangle{\pgfqpoint{0.696435in}{0.523570in}}{\pgfqpoint{4.650000in}{3.020000in}}%
\pgfusepath{clip}%
\pgfsetroundcap%
\pgfsetroundjoin%
\pgfsetlinewidth{1.505625pt}%
\definecolor{currentstroke}{rgb}{1.000000,0.498039,0.054902}%
\pgfsetstrokecolor{currentstroke}%
\pgfsetdash{}{0pt}%
\pgfpathmoveto{\pgfqpoint{0.907799in}{0.782160in}}%
\pgfpathlineto{\pgfqpoint{1.016191in}{0.839583in}}%
\pgfpathlineto{\pgfqpoint{1.124582in}{0.964371in}}%
\pgfpathlineto{\pgfqpoint{1.232974in}{0.755148in}}%
\pgfpathlineto{\pgfqpoint{1.341365in}{0.767964in}}%
\pgfpathlineto{\pgfqpoint{1.449757in}{0.915254in}}%
\pgfpathlineto{\pgfqpoint{1.558149in}{1.187111in}}%
\pgfpathlineto{\pgfqpoint{1.666540in}{0.695625in}}%
\pgfpathlineto{\pgfqpoint{1.774932in}{0.749644in}}%
\pgfpathlineto{\pgfqpoint{1.883324in}{0.761242in}}%
\pgfpathlineto{\pgfqpoint{1.991715in}{0.864247in}}%
\pgfpathlineto{\pgfqpoint{2.100107in}{1.023107in}}%
\pgfpathlineto{\pgfqpoint{2.208498in}{1.252351in}}%
\pgfpathlineto{\pgfqpoint{2.316890in}{1.562519in}}%
\pgfpathlineto{\pgfqpoint{2.425282in}{0.665261in}}%
\pgfpathlineto{\pgfqpoint{2.533673in}{0.703920in}}%
\pgfpathlineto{\pgfqpoint{2.642065in}{0.751750in}}%
\pgfpathlineto{\pgfqpoint{2.750456in}{0.816567in}}%
\pgfpathlineto{\pgfqpoint{2.858848in}{0.905808in}}%
\pgfpathlineto{\pgfqpoint{2.967240in}{0.929215in}}%
\pgfpathlineto{\pgfqpoint{3.075631in}{1.068285in}}%
\pgfpathlineto{\pgfqpoint{3.184023in}{1.042716in}}%
\pgfpathlineto{\pgfqpoint{3.292414in}{1.227670in}}%
\pgfpathlineto{\pgfqpoint{3.400806in}{1.472180in}}%
\pgfpathlineto{\pgfqpoint{3.509198in}{1.788638in}}%
\pgfpathlineto{\pgfqpoint{3.617589in}{2.147004in}}%
\pgfpathlineto{\pgfqpoint{3.725981in}{2.601256in}}%
\pgfpathlineto{\pgfqpoint{3.834372in}{2.798612in}}%
\pgfpathlineto{\pgfqpoint{3.942764in}{3.406297in}}%
\pgfpathlineto{\pgfqpoint{4.051156in}{2.473569in}}%
\pgfpathlineto{\pgfqpoint{4.159547in}{2.808192in}}%
\pgfpathlineto{\pgfqpoint{4.267939in}{2.142305in}}%
\pgfpathlineto{\pgfqpoint{4.376331in}{1.207309in}}%
\pgfpathlineto{\pgfqpoint{4.484722in}{1.384126in}}%
\pgfpathlineto{\pgfqpoint{4.593114in}{1.491229in}}%
\pgfpathlineto{\pgfqpoint{4.701505in}{1.713600in}}%
\pgfpathlineto{\pgfqpoint{4.809897in}{1.982842in}}%
\pgfpathlineto{\pgfqpoint{4.918289in}{2.343696in}}%
\pgfpathlineto{\pgfqpoint{5.026680in}{2.000818in}}%
\pgfpathlineto{\pgfqpoint{5.135072in}{2.137825in}}%
\pgfusepath{stroke}%
\end{pgfscope}%
\begin{pgfscope}%
\pgfsetrectcap%
\pgfsetmiterjoin%
\pgfsetlinewidth{0.803000pt}%
\definecolor{currentstroke}{rgb}{1.000000,1.000000,1.000000}%
\pgfsetstrokecolor{currentstroke}%
\pgfsetdash{}{0pt}%
\pgfpathmoveto{\pgfqpoint{0.696435in}{0.523570in}}%
\pgfpathlineto{\pgfqpoint{0.696435in}{3.543570in}}%
\pgfusepath{stroke}%
\end{pgfscope}%
\begin{pgfscope}%
\pgfsetrectcap%
\pgfsetmiterjoin%
\pgfsetlinewidth{0.803000pt}%
\definecolor{currentstroke}{rgb}{1.000000,1.000000,1.000000}%
\pgfsetstrokecolor{currentstroke}%
\pgfsetdash{}{0pt}%
\pgfpathmoveto{\pgfqpoint{5.346435in}{0.523570in}}%
\pgfpathlineto{\pgfqpoint{5.346435in}{3.543570in}}%
\pgfusepath{stroke}%
\end{pgfscope}%
\begin{pgfscope}%
\pgfsetrectcap%
\pgfsetmiterjoin%
\pgfsetlinewidth{0.803000pt}%
\definecolor{currentstroke}{rgb}{1.000000,1.000000,1.000000}%
\pgfsetstrokecolor{currentstroke}%
\pgfsetdash{}{0pt}%
\pgfpathmoveto{\pgfqpoint{0.696435in}{0.523570in}}%
\pgfpathlineto{\pgfqpoint{5.346435in}{0.523570in}}%
\pgfusepath{stroke}%
\end{pgfscope}%
\begin{pgfscope}%
\pgfsetrectcap%
\pgfsetmiterjoin%
\pgfsetlinewidth{0.803000pt}%
\definecolor{currentstroke}{rgb}{1.000000,1.000000,1.000000}%
\pgfsetstrokecolor{currentstroke}%
\pgfsetdash{}{0pt}%
\pgfpathmoveto{\pgfqpoint{0.696435in}{3.543570in}}%
\pgfpathlineto{\pgfqpoint{5.346435in}{3.543570in}}%
\pgfusepath{stroke}%
\end{pgfscope}%
\begin{pgfscope}%
\pgfsetbuttcap%
\pgfsetmiterjoin%
\definecolor{currentfill}{rgb}{0.917647,0.917647,0.949020}%
\pgfsetfillcolor{currentfill}%
\pgfsetfillopacity{0.800000}%
\pgfsetlinewidth{1.003750pt}%
\definecolor{currentstroke}{rgb}{0.800000,0.800000,0.800000}%
\pgfsetstrokecolor{currentstroke}%
\pgfsetstrokeopacity{0.800000}%
\pgfsetdash{}{0pt}%
\pgfpathmoveto{\pgfqpoint{4.049832in}{3.022778in}}%
\pgfpathlineto{\pgfqpoint{5.249213in}{3.022778in}}%
\pgfpathquadraticcurveto{\pgfqpoint{5.276991in}{3.022778in}}{\pgfqpoint{5.276991in}{3.050556in}}%
\pgfpathlineto{\pgfqpoint{5.276991in}{3.446348in}}%
\pgfpathquadraticcurveto{\pgfqpoint{5.276991in}{3.474126in}}{\pgfqpoint{5.249213in}{3.474126in}}%
\pgfpathlineto{\pgfqpoint{4.049832in}{3.474126in}}%
\pgfpathquadraticcurveto{\pgfqpoint{4.022054in}{3.474126in}}{\pgfqpoint{4.022054in}{3.446348in}}%
\pgfpathlineto{\pgfqpoint{4.022054in}{3.050556in}}%
\pgfpathquadraticcurveto{\pgfqpoint{4.022054in}{3.022778in}}{\pgfqpoint{4.049832in}{3.022778in}}%
\pgfpathclose%
\pgfusepath{stroke,fill}%
\end{pgfscope}%
\begin{pgfscope}%
\pgfsetroundcap%
\pgfsetroundjoin%
\pgfsetlinewidth{1.505625pt}%
\definecolor{currentstroke}{rgb}{0.121569,0.466667,0.705882}%
\pgfsetstrokecolor{currentstroke}%
\pgfsetdash{}{0pt}%
\pgfpathmoveto{\pgfqpoint{4.077609in}{3.361658in}}%
\pgfpathlineto{\pgfqpoint{4.355387in}{3.361658in}}%
\pgfusepath{stroke}%
\end{pgfscope}%
\begin{pgfscope}%
\definecolor{textcolor}{rgb}{0.150000,0.150000,0.150000}%
\pgfsetstrokecolor{textcolor}%
\pgfsetfillcolor{textcolor}%
\pgftext[x=4.466498in,y=3.313047in,left,base]{\color{textcolor}\rmfamily\fontsize{10.000000}{12.000000}\selectfont Ljung-Box}%
\end{pgfscope}%
\begin{pgfscope}%
\pgfsetroundcap%
\pgfsetroundjoin%
\pgfsetlinewidth{1.505625pt}%
\definecolor{currentstroke}{rgb}{1.000000,0.498039,0.054902}%
\pgfsetstrokecolor{currentstroke}%
\pgfsetdash{}{0pt}%
\pgfpathmoveto{\pgfqpoint{4.077609in}{3.155834in}}%
\pgfpathlineto{\pgfqpoint{4.355387in}{3.155834in}}%
\pgfusepath{stroke}%
\end{pgfscope}%
\begin{pgfscope}%
\definecolor{textcolor}{rgb}{0.150000,0.150000,0.150000}%
\pgfsetstrokecolor{textcolor}%
\pgfsetfillcolor{textcolor}%
\pgftext[x=4.466498in,y=3.107223in,left,base]{\color{textcolor}\rmfamily\fontsize{10.000000}{12.000000}\selectfont Box-Pierce}%
\end{pgfscope}%
\end{pgfpicture}%
\makeatother%
\endgroup%

    %% Creator: Matplotlib, PGF backend
%%
%% To include the figure in your LaTeX document, write
%%   \input{<filename>.pgf}
%%
%% Make sure the required packages are loaded in your preamble
%%   \usepackage{pgf}
%%
%% Figures using additional raster images can only be included by \input if
%% they are in the same directory as the main LaTeX file. For loading figures
%% from other directories you can use the `import` package
%%   \usepackage{import}
%% and then include the figures with
%%   \import{<path to file>}{<filename>.pgf}
%%
%% Matplotlib used the following preamble
%%   \usepackage{fontspec}
%%   \setmainfont{DejaVuSerif.ttf}[Path=/opt/tljh/user/lib/python3.6/site-packages/matplotlib/mpl-data/fonts/ttf/]
%%   \setsansfont{DejaVuSans.ttf}[Path=/opt/tljh/user/lib/python3.6/site-packages/matplotlib/mpl-data/fonts/ttf/]
%%   \setmonofont{DejaVuSansMono.ttf}[Path=/opt/tljh/user/lib/python3.6/site-packages/matplotlib/mpl-data/fonts/ttf/]
%%
\begingroup%
\makeatletter%
\begin{pgfpicture}%
\pgfpathrectangle{\pgfpointorigin}{\pgfqpoint{6.908070in}{2.888570in}}%
\pgfusepath{use as bounding box, clip}%
\begin{pgfscope}%
\pgfsetbuttcap%
\pgfsetmiterjoin%
\definecolor{currentfill}{rgb}{1.000000,1.000000,1.000000}%
\pgfsetfillcolor{currentfill}%
\pgfsetlinewidth{0.000000pt}%
\definecolor{currentstroke}{rgb}{1.000000,1.000000,1.000000}%
\pgfsetstrokecolor{currentstroke}%
\pgfsetdash{}{0pt}%
\pgfpathmoveto{\pgfqpoint{0.000000in}{0.000000in}}%
\pgfpathlineto{\pgfqpoint{6.908070in}{0.000000in}}%
\pgfpathlineto{\pgfqpoint{6.908070in}{2.888570in}}%
\pgfpathlineto{\pgfqpoint{0.000000in}{2.888570in}}%
\pgfpathclose%
\pgfusepath{fill}%
\end{pgfscope}%
\begin{pgfscope}%
\pgfsetbuttcap%
\pgfsetmiterjoin%
\definecolor{currentfill}{rgb}{0.917647,0.917647,0.949020}%
\pgfsetfillcolor{currentfill}%
\pgfsetlinewidth{0.000000pt}%
\definecolor{currentstroke}{rgb}{0.000000,0.000000,0.000000}%
\pgfsetstrokecolor{currentstroke}%
\pgfsetstrokeopacity{0.000000}%
\pgfsetdash{}{0pt}%
\pgfpathmoveto{\pgfqpoint{0.608070in}{0.523570in}}%
\pgfpathlineto{\pgfqpoint{6.808070in}{0.523570in}}%
\pgfpathlineto{\pgfqpoint{6.808070in}{2.788570in}}%
\pgfpathlineto{\pgfqpoint{0.608070in}{2.788570in}}%
\pgfpathclose%
\pgfusepath{fill}%
\end{pgfscope}%
\begin{pgfscope}%
\pgfpathrectangle{\pgfqpoint{0.608070in}{0.523570in}}{\pgfqpoint{6.200000in}{2.265000in}}%
\pgfusepath{clip}%
\pgfsetroundcap%
\pgfsetroundjoin%
\pgfsetlinewidth{0.803000pt}%
\definecolor{currentstroke}{rgb}{1.000000,1.000000,1.000000}%
\pgfsetstrokecolor{currentstroke}%
\pgfsetdash{}{0pt}%
\pgfpathmoveto{\pgfqpoint{0.889888in}{0.523570in}}%
\pgfpathlineto{\pgfqpoint{0.889888in}{2.788570in}}%
\pgfusepath{stroke}%
\end{pgfscope}%
\begin{pgfscope}%
\definecolor{textcolor}{rgb}{0.150000,0.150000,0.150000}%
\pgfsetstrokecolor{textcolor}%
\pgfsetfillcolor{textcolor}%
\pgftext[x=0.889888in,y=0.426348in,,top]{\color{textcolor}\rmfamily\fontsize{10.000000}{12.000000}\selectfont 0}%
\end{pgfscope}%
\begin{pgfscope}%
\pgfpathrectangle{\pgfqpoint{0.608070in}{0.523570in}}{\pgfqpoint{6.200000in}{2.265000in}}%
\pgfusepath{clip}%
\pgfsetroundcap%
\pgfsetroundjoin%
\pgfsetlinewidth{0.803000pt}%
\definecolor{currentstroke}{rgb}{1.000000,1.000000,1.000000}%
\pgfsetstrokecolor{currentstroke}%
\pgfsetdash{}{0pt}%
\pgfpathmoveto{\pgfqpoint{1.612499in}{0.523570in}}%
\pgfpathlineto{\pgfqpoint{1.612499in}{2.788570in}}%
\pgfusepath{stroke}%
\end{pgfscope}%
\begin{pgfscope}%
\definecolor{textcolor}{rgb}{0.150000,0.150000,0.150000}%
\pgfsetstrokecolor{textcolor}%
\pgfsetfillcolor{textcolor}%
\pgftext[x=1.612499in,y=0.426348in,,top]{\color{textcolor}\rmfamily\fontsize{10.000000}{12.000000}\selectfont 5}%
\end{pgfscope}%
\begin{pgfscope}%
\pgfpathrectangle{\pgfqpoint{0.608070in}{0.523570in}}{\pgfqpoint{6.200000in}{2.265000in}}%
\pgfusepath{clip}%
\pgfsetroundcap%
\pgfsetroundjoin%
\pgfsetlinewidth{0.803000pt}%
\definecolor{currentstroke}{rgb}{1.000000,1.000000,1.000000}%
\pgfsetstrokecolor{currentstroke}%
\pgfsetdash{}{0pt}%
\pgfpathmoveto{\pgfqpoint{2.335110in}{0.523570in}}%
\pgfpathlineto{\pgfqpoint{2.335110in}{2.788570in}}%
\pgfusepath{stroke}%
\end{pgfscope}%
\begin{pgfscope}%
\definecolor{textcolor}{rgb}{0.150000,0.150000,0.150000}%
\pgfsetstrokecolor{textcolor}%
\pgfsetfillcolor{textcolor}%
\pgftext[x=2.335110in,y=0.426348in,,top]{\color{textcolor}\rmfamily\fontsize{10.000000}{12.000000}\selectfont 10}%
\end{pgfscope}%
\begin{pgfscope}%
\pgfpathrectangle{\pgfqpoint{0.608070in}{0.523570in}}{\pgfqpoint{6.200000in}{2.265000in}}%
\pgfusepath{clip}%
\pgfsetroundcap%
\pgfsetroundjoin%
\pgfsetlinewidth{0.803000pt}%
\definecolor{currentstroke}{rgb}{1.000000,1.000000,1.000000}%
\pgfsetstrokecolor{currentstroke}%
\pgfsetdash{}{0pt}%
\pgfpathmoveto{\pgfqpoint{3.057720in}{0.523570in}}%
\pgfpathlineto{\pgfqpoint{3.057720in}{2.788570in}}%
\pgfusepath{stroke}%
\end{pgfscope}%
\begin{pgfscope}%
\definecolor{textcolor}{rgb}{0.150000,0.150000,0.150000}%
\pgfsetstrokecolor{textcolor}%
\pgfsetfillcolor{textcolor}%
\pgftext[x=3.057720in,y=0.426348in,,top]{\color{textcolor}\rmfamily\fontsize{10.000000}{12.000000}\selectfont 15}%
\end{pgfscope}%
\begin{pgfscope}%
\pgfpathrectangle{\pgfqpoint{0.608070in}{0.523570in}}{\pgfqpoint{6.200000in}{2.265000in}}%
\pgfusepath{clip}%
\pgfsetroundcap%
\pgfsetroundjoin%
\pgfsetlinewidth{0.803000pt}%
\definecolor{currentstroke}{rgb}{1.000000,1.000000,1.000000}%
\pgfsetstrokecolor{currentstroke}%
\pgfsetdash{}{0pt}%
\pgfpathmoveto{\pgfqpoint{3.780331in}{0.523570in}}%
\pgfpathlineto{\pgfqpoint{3.780331in}{2.788570in}}%
\pgfusepath{stroke}%
\end{pgfscope}%
\begin{pgfscope}%
\definecolor{textcolor}{rgb}{0.150000,0.150000,0.150000}%
\pgfsetstrokecolor{textcolor}%
\pgfsetfillcolor{textcolor}%
\pgftext[x=3.780331in,y=0.426348in,,top]{\color{textcolor}\rmfamily\fontsize{10.000000}{12.000000}\selectfont 20}%
\end{pgfscope}%
\begin{pgfscope}%
\pgfpathrectangle{\pgfqpoint{0.608070in}{0.523570in}}{\pgfqpoint{6.200000in}{2.265000in}}%
\pgfusepath{clip}%
\pgfsetroundcap%
\pgfsetroundjoin%
\pgfsetlinewidth{0.803000pt}%
\definecolor{currentstroke}{rgb}{1.000000,1.000000,1.000000}%
\pgfsetstrokecolor{currentstroke}%
\pgfsetdash{}{0pt}%
\pgfpathmoveto{\pgfqpoint{4.502942in}{0.523570in}}%
\pgfpathlineto{\pgfqpoint{4.502942in}{2.788570in}}%
\pgfusepath{stroke}%
\end{pgfscope}%
\begin{pgfscope}%
\definecolor{textcolor}{rgb}{0.150000,0.150000,0.150000}%
\pgfsetstrokecolor{textcolor}%
\pgfsetfillcolor{textcolor}%
\pgftext[x=4.502942in,y=0.426348in,,top]{\color{textcolor}\rmfamily\fontsize{10.000000}{12.000000}\selectfont 25}%
\end{pgfscope}%
\begin{pgfscope}%
\pgfpathrectangle{\pgfqpoint{0.608070in}{0.523570in}}{\pgfqpoint{6.200000in}{2.265000in}}%
\pgfusepath{clip}%
\pgfsetroundcap%
\pgfsetroundjoin%
\pgfsetlinewidth{0.803000pt}%
\definecolor{currentstroke}{rgb}{1.000000,1.000000,1.000000}%
\pgfsetstrokecolor{currentstroke}%
\pgfsetdash{}{0pt}%
\pgfpathmoveto{\pgfqpoint{5.225553in}{0.523570in}}%
\pgfpathlineto{\pgfqpoint{5.225553in}{2.788570in}}%
\pgfusepath{stroke}%
\end{pgfscope}%
\begin{pgfscope}%
\definecolor{textcolor}{rgb}{0.150000,0.150000,0.150000}%
\pgfsetstrokecolor{textcolor}%
\pgfsetfillcolor{textcolor}%
\pgftext[x=5.225553in,y=0.426348in,,top]{\color{textcolor}\rmfamily\fontsize{10.000000}{12.000000}\selectfont 30}%
\end{pgfscope}%
\begin{pgfscope}%
\pgfpathrectangle{\pgfqpoint{0.608070in}{0.523570in}}{\pgfqpoint{6.200000in}{2.265000in}}%
\pgfusepath{clip}%
\pgfsetroundcap%
\pgfsetroundjoin%
\pgfsetlinewidth{0.803000pt}%
\definecolor{currentstroke}{rgb}{1.000000,1.000000,1.000000}%
\pgfsetstrokecolor{currentstroke}%
\pgfsetdash{}{0pt}%
\pgfpathmoveto{\pgfqpoint{5.948163in}{0.523570in}}%
\pgfpathlineto{\pgfqpoint{5.948163in}{2.788570in}}%
\pgfusepath{stroke}%
\end{pgfscope}%
\begin{pgfscope}%
\definecolor{textcolor}{rgb}{0.150000,0.150000,0.150000}%
\pgfsetstrokecolor{textcolor}%
\pgfsetfillcolor{textcolor}%
\pgftext[x=5.948163in,y=0.426348in,,top]{\color{textcolor}\rmfamily\fontsize{10.000000}{12.000000}\selectfont 35}%
\end{pgfscope}%
\begin{pgfscope}%
\pgfpathrectangle{\pgfqpoint{0.608070in}{0.523570in}}{\pgfqpoint{6.200000in}{2.265000in}}%
\pgfusepath{clip}%
\pgfsetroundcap%
\pgfsetroundjoin%
\pgfsetlinewidth{0.803000pt}%
\definecolor{currentstroke}{rgb}{1.000000,1.000000,1.000000}%
\pgfsetstrokecolor{currentstroke}%
\pgfsetdash{}{0pt}%
\pgfpathmoveto{\pgfqpoint{6.670774in}{0.523570in}}%
\pgfpathlineto{\pgfqpoint{6.670774in}{2.788570in}}%
\pgfusepath{stroke}%
\end{pgfscope}%
\begin{pgfscope}%
\definecolor{textcolor}{rgb}{0.150000,0.150000,0.150000}%
\pgfsetstrokecolor{textcolor}%
\pgfsetfillcolor{textcolor}%
\pgftext[x=6.670774in,y=0.426348in,,top]{\color{textcolor}\rmfamily\fontsize{10.000000}{12.000000}\selectfont 40}%
\end{pgfscope}%
\begin{pgfscope}%
\definecolor{textcolor}{rgb}{0.150000,0.150000,0.150000}%
\pgfsetstrokecolor{textcolor}%
\pgfsetfillcolor{textcolor}%
\pgftext[x=3.708070in,y=0.236379in,,top]{\color{textcolor}\rmfamily\fontsize{10.000000}{12.000000}\selectfont lag}%
\end{pgfscope}%
\begin{pgfscope}%
\pgfpathrectangle{\pgfqpoint{0.608070in}{0.523570in}}{\pgfqpoint{6.200000in}{2.265000in}}%
\pgfusepath{clip}%
\pgfsetroundcap%
\pgfsetroundjoin%
\pgfsetlinewidth{0.803000pt}%
\definecolor{currentstroke}{rgb}{1.000000,1.000000,1.000000}%
\pgfsetstrokecolor{currentstroke}%
\pgfsetdash{}{0pt}%
\pgfpathmoveto{\pgfqpoint{0.608070in}{0.609470in}}%
\pgfpathlineto{\pgfqpoint{6.808070in}{0.609470in}}%
\pgfusepath{stroke}%
\end{pgfscope}%
\begin{pgfscope}%
\definecolor{textcolor}{rgb}{0.150000,0.150000,0.150000}%
\pgfsetstrokecolor{textcolor}%
\pgfsetfillcolor{textcolor}%
\pgftext[x=0.289968in,y=0.556709in,left,base]{\color{textcolor}\rmfamily\fontsize{10.000000}{12.000000}\selectfont 0.0}%
\end{pgfscope}%
\begin{pgfscope}%
\pgfpathrectangle{\pgfqpoint{0.608070in}{0.523570in}}{\pgfqpoint{6.200000in}{2.265000in}}%
\pgfusepath{clip}%
\pgfsetroundcap%
\pgfsetroundjoin%
\pgfsetlinewidth{0.803000pt}%
\definecolor{currentstroke}{rgb}{1.000000,1.000000,1.000000}%
\pgfsetstrokecolor{currentstroke}%
\pgfsetdash{}{0pt}%
\pgfpathmoveto{\pgfqpoint{0.608070in}{1.026264in}}%
\pgfpathlineto{\pgfqpoint{6.808070in}{1.026264in}}%
\pgfusepath{stroke}%
\end{pgfscope}%
\begin{pgfscope}%
\definecolor{textcolor}{rgb}{0.150000,0.150000,0.150000}%
\pgfsetstrokecolor{textcolor}%
\pgfsetfillcolor{textcolor}%
\pgftext[x=0.289968in,y=0.973503in,left,base]{\color{textcolor}\rmfamily\fontsize{10.000000}{12.000000}\selectfont 0.2}%
\end{pgfscope}%
\begin{pgfscope}%
\pgfpathrectangle{\pgfqpoint{0.608070in}{0.523570in}}{\pgfqpoint{6.200000in}{2.265000in}}%
\pgfusepath{clip}%
\pgfsetroundcap%
\pgfsetroundjoin%
\pgfsetlinewidth{0.803000pt}%
\definecolor{currentstroke}{rgb}{1.000000,1.000000,1.000000}%
\pgfsetstrokecolor{currentstroke}%
\pgfsetdash{}{0pt}%
\pgfpathmoveto{\pgfqpoint{0.608070in}{1.443058in}}%
\pgfpathlineto{\pgfqpoint{6.808070in}{1.443058in}}%
\pgfusepath{stroke}%
\end{pgfscope}%
\begin{pgfscope}%
\definecolor{textcolor}{rgb}{0.150000,0.150000,0.150000}%
\pgfsetstrokecolor{textcolor}%
\pgfsetfillcolor{textcolor}%
\pgftext[x=0.289968in,y=1.390297in,left,base]{\color{textcolor}\rmfamily\fontsize{10.000000}{12.000000}\selectfont 0.4}%
\end{pgfscope}%
\begin{pgfscope}%
\pgfpathrectangle{\pgfqpoint{0.608070in}{0.523570in}}{\pgfqpoint{6.200000in}{2.265000in}}%
\pgfusepath{clip}%
\pgfsetroundcap%
\pgfsetroundjoin%
\pgfsetlinewidth{0.803000pt}%
\definecolor{currentstroke}{rgb}{1.000000,1.000000,1.000000}%
\pgfsetstrokecolor{currentstroke}%
\pgfsetdash{}{0pt}%
\pgfpathmoveto{\pgfqpoint{0.608070in}{1.859853in}}%
\pgfpathlineto{\pgfqpoint{6.808070in}{1.859853in}}%
\pgfusepath{stroke}%
\end{pgfscope}%
\begin{pgfscope}%
\definecolor{textcolor}{rgb}{0.150000,0.150000,0.150000}%
\pgfsetstrokecolor{textcolor}%
\pgfsetfillcolor{textcolor}%
\pgftext[x=0.289968in,y=1.807091in,left,base]{\color{textcolor}\rmfamily\fontsize{10.000000}{12.000000}\selectfont 0.6}%
\end{pgfscope}%
\begin{pgfscope}%
\pgfpathrectangle{\pgfqpoint{0.608070in}{0.523570in}}{\pgfqpoint{6.200000in}{2.265000in}}%
\pgfusepath{clip}%
\pgfsetroundcap%
\pgfsetroundjoin%
\pgfsetlinewidth{0.803000pt}%
\definecolor{currentstroke}{rgb}{1.000000,1.000000,1.000000}%
\pgfsetstrokecolor{currentstroke}%
\pgfsetdash{}{0pt}%
\pgfpathmoveto{\pgfqpoint{0.608070in}{2.276647in}}%
\pgfpathlineto{\pgfqpoint{6.808070in}{2.276647in}}%
\pgfusepath{stroke}%
\end{pgfscope}%
\begin{pgfscope}%
\definecolor{textcolor}{rgb}{0.150000,0.150000,0.150000}%
\pgfsetstrokecolor{textcolor}%
\pgfsetfillcolor{textcolor}%
\pgftext[x=0.289968in,y=2.223885in,left,base]{\color{textcolor}\rmfamily\fontsize{10.000000}{12.000000}\selectfont 0.8}%
\end{pgfscope}%
\begin{pgfscope}%
\pgfpathrectangle{\pgfqpoint{0.608070in}{0.523570in}}{\pgfqpoint{6.200000in}{2.265000in}}%
\pgfusepath{clip}%
\pgfsetroundcap%
\pgfsetroundjoin%
\pgfsetlinewidth{0.803000pt}%
\definecolor{currentstroke}{rgb}{1.000000,1.000000,1.000000}%
\pgfsetstrokecolor{currentstroke}%
\pgfsetdash{}{0pt}%
\pgfpathmoveto{\pgfqpoint{0.608070in}{2.693441in}}%
\pgfpathlineto{\pgfqpoint{6.808070in}{2.693441in}}%
\pgfusepath{stroke}%
\end{pgfscope}%
\begin{pgfscope}%
\definecolor{textcolor}{rgb}{0.150000,0.150000,0.150000}%
\pgfsetstrokecolor{textcolor}%
\pgfsetfillcolor{textcolor}%
\pgftext[x=0.289968in,y=2.640679in,left,base]{\color{textcolor}\rmfamily\fontsize{10.000000}{12.000000}\selectfont 1.0}%
\end{pgfscope}%
\begin{pgfscope}%
\definecolor{textcolor}{rgb}{0.150000,0.150000,0.150000}%
\pgfsetstrokecolor{textcolor}%
\pgfsetfillcolor{textcolor}%
\pgftext[x=0.234413in,y=1.656070in,,bottom,rotate=90.000000]{\color{textcolor}\rmfamily\fontsize{10.000000}{12.000000}\selectfont p-value}%
\end{pgfscope}%
\begin{pgfscope}%
\pgfpathrectangle{\pgfqpoint{0.608070in}{0.523570in}}{\pgfqpoint{6.200000in}{2.265000in}}%
\pgfusepath{clip}%
\pgfsetroundcap%
\pgfsetroundjoin%
\pgfsetlinewidth{1.505625pt}%
\definecolor{currentstroke}{rgb}{0.121569,0.466667,0.705882}%
\pgfsetstrokecolor{currentstroke}%
\pgfsetdash{}{0pt}%
\pgfpathmoveto{\pgfqpoint{0.889888in}{1.844401in}}%
\pgfpathlineto{\pgfqpoint{1.034410in}{2.411099in}}%
\pgfpathlineto{\pgfqpoint{1.178932in}{2.612982in}}%
\pgfpathlineto{\pgfqpoint{1.323455in}{2.640979in}}%
\pgfpathlineto{\pgfqpoint{1.467977in}{2.677610in}}%
\pgfpathlineto{\pgfqpoint{1.612499in}{2.685544in}}%
\pgfpathlineto{\pgfqpoint{1.757021in}{2.664130in}}%
\pgfpathlineto{\pgfqpoint{1.901543in}{2.674991in}}%
\pgfpathlineto{\pgfqpoint{2.046065in}{2.559145in}}%
\pgfpathlineto{\pgfqpoint{2.190587in}{2.313754in}}%
\pgfpathlineto{\pgfqpoint{2.335110in}{2.206893in}}%
\pgfpathlineto{\pgfqpoint{2.479632in}{2.313957in}}%
\pgfpathlineto{\pgfqpoint{2.624154in}{1.829463in}}%
\pgfpathlineto{\pgfqpoint{2.768676in}{1.981978in}}%
\pgfpathlineto{\pgfqpoint{2.913198in}{0.947990in}}%
\pgfpathlineto{\pgfqpoint{3.057720in}{0.822179in}}%
\pgfpathlineto{\pgfqpoint{3.202242in}{0.716478in}}%
\pgfpathlineto{\pgfqpoint{3.346765in}{0.626525in}}%
\pgfpathlineto{\pgfqpoint{3.491287in}{0.634654in}}%
\pgfpathlineto{\pgfqpoint{3.635809in}{0.645759in}}%
\pgfpathlineto{\pgfqpoint{3.780331in}{0.647946in}}%
\pgfpathlineto{\pgfqpoint{3.924853in}{0.645321in}}%
\pgfpathlineto{\pgfqpoint{4.069375in}{0.656212in}}%
\pgfpathlineto{\pgfqpoint{4.213898in}{0.666054in}}%
\pgfpathlineto{\pgfqpoint{4.358420in}{0.685250in}}%
\pgfpathlineto{\pgfqpoint{4.502942in}{0.666359in}}%
\pgfpathlineto{\pgfqpoint{4.647464in}{0.684535in}}%
\pgfpathlineto{\pgfqpoint{4.791986in}{0.707153in}}%
\pgfpathlineto{\pgfqpoint{4.936508in}{0.735416in}}%
\pgfpathlineto{\pgfqpoint{5.081030in}{0.750931in}}%
\pgfpathlineto{\pgfqpoint{5.225553in}{0.744246in}}%
\pgfpathlineto{\pgfqpoint{5.370075in}{0.778652in}}%
\pgfpathlineto{\pgfqpoint{5.514597in}{0.818242in}}%
\pgfpathlineto{\pgfqpoint{5.659119in}{0.839037in}}%
\pgfpathlineto{\pgfqpoint{5.803641in}{0.820718in}}%
\pgfpathlineto{\pgfqpoint{5.948163in}{0.866080in}}%
\pgfpathlineto{\pgfqpoint{6.092685in}{0.863481in}}%
\pgfpathlineto{\pgfqpoint{6.237208in}{0.898435in}}%
\pgfpathlineto{\pgfqpoint{6.381730in}{0.906855in}}%
\pgfpathlineto{\pgfqpoint{6.526252in}{0.961073in}}%
\pgfusepath{stroke}%
\end{pgfscope}%
\begin{pgfscope}%
\pgfpathrectangle{\pgfqpoint{0.608070in}{0.523570in}}{\pgfqpoint{6.200000in}{2.265000in}}%
\pgfusepath{clip}%
\pgfsetroundcap%
\pgfsetroundjoin%
\pgfsetlinewidth{1.505625pt}%
\definecolor{currentstroke}{rgb}{1.000000,0.498039,0.054902}%
\pgfsetstrokecolor{currentstroke}%
\pgfsetdash{}{0pt}%
\pgfpathmoveto{\pgfqpoint{0.889888in}{1.845167in}}%
\pgfpathlineto{\pgfqpoint{1.034410in}{2.411623in}}%
\pgfpathlineto{\pgfqpoint{1.178932in}{2.613210in}}%
\pgfpathlineto{\pgfqpoint{1.323455in}{2.641247in}}%
\pgfpathlineto{\pgfqpoint{1.467977in}{2.677713in}}%
\pgfpathlineto{\pgfqpoint{1.612499in}{2.685615in}}%
\pgfpathlineto{\pgfqpoint{1.757021in}{2.664546in}}%
\pgfpathlineto{\pgfqpoint{1.901543in}{2.675302in}}%
\pgfpathlineto{\pgfqpoint{2.046065in}{2.561756in}}%
\pgfpathlineto{\pgfqpoint{2.190587in}{2.320945in}}%
\pgfpathlineto{\pgfqpoint{2.335110in}{2.216378in}}%
\pgfpathlineto{\pgfqpoint{2.479632in}{2.322387in}}%
\pgfpathlineto{\pgfqpoint{2.624154in}{1.845567in}}%
\pgfpathlineto{\pgfqpoint{2.768676in}{1.997352in}}%
\pgfpathlineto{\pgfqpoint{2.913198in}{0.963003in}}%
\pgfpathlineto{\pgfqpoint{3.057720in}{0.834220in}}%
\pgfpathlineto{\pgfqpoint{3.202242in}{0.724456in}}%
\pgfpathlineto{\pgfqpoint{3.346765in}{0.628565in}}%
\pgfpathlineto{\pgfqpoint{3.491287in}{0.637536in}}%
\pgfpathlineto{\pgfqpoint{3.635809in}{0.649724in}}%
\pgfpathlineto{\pgfqpoint{3.780331in}{0.652244in}}%
\pgfpathlineto{\pgfqpoint{3.924853in}{0.649543in}}%
\pgfpathlineto{\pgfqpoint{4.069375in}{0.661554in}}%
\pgfpathlineto{\pgfqpoint{4.213898in}{0.672430in}}%
\pgfpathlineto{\pgfqpoint{4.358420in}{0.693399in}}%
\pgfpathlineto{\pgfqpoint{4.502942in}{0.673261in}}%
\pgfpathlineto{\pgfqpoint{4.647464in}{0.693249in}}%
\pgfpathlineto{\pgfqpoint{4.791986in}{0.717977in}}%
\pgfpathlineto{\pgfqpoint{4.936508in}{0.748686in}}%
\pgfpathlineto{\pgfqpoint{5.081030in}{0.765768in}}%
\pgfpathlineto{\pgfqpoint{5.225553in}{0.759210in}}%
\pgfpathlineto{\pgfqpoint{5.370075in}{0.796496in}}%
\pgfpathlineto{\pgfqpoint{5.514597in}{0.839148in}}%
\pgfpathlineto{\pgfqpoint{5.659119in}{0.861921in}}%
\pgfpathlineto{\pgfqpoint{5.803641in}{0.843462in}}%
\pgfpathlineto{\pgfqpoint{5.948163in}{0.892247in}}%
\pgfpathlineto{\pgfqpoint{6.092685in}{0.890521in}}%
\pgfpathlineto{\pgfqpoint{6.237208in}{0.928341in}}%
\pgfpathlineto{\pgfqpoint{6.381730in}{0.938320in}}%
\pgfpathlineto{\pgfqpoint{6.526252in}{0.996238in}}%
\pgfusepath{stroke}%
\end{pgfscope}%
\begin{pgfscope}%
\pgfsetrectcap%
\pgfsetmiterjoin%
\pgfsetlinewidth{0.803000pt}%
\definecolor{currentstroke}{rgb}{1.000000,1.000000,1.000000}%
\pgfsetstrokecolor{currentstroke}%
\pgfsetdash{}{0pt}%
\pgfpathmoveto{\pgfqpoint{0.608070in}{0.523570in}}%
\pgfpathlineto{\pgfqpoint{0.608070in}{2.788570in}}%
\pgfusepath{stroke}%
\end{pgfscope}%
\begin{pgfscope}%
\pgfsetrectcap%
\pgfsetmiterjoin%
\pgfsetlinewidth{0.803000pt}%
\definecolor{currentstroke}{rgb}{1.000000,1.000000,1.000000}%
\pgfsetstrokecolor{currentstroke}%
\pgfsetdash{}{0pt}%
\pgfpathmoveto{\pgfqpoint{6.808070in}{0.523570in}}%
\pgfpathlineto{\pgfqpoint{6.808070in}{2.788570in}}%
\pgfusepath{stroke}%
\end{pgfscope}%
\begin{pgfscope}%
\pgfsetrectcap%
\pgfsetmiterjoin%
\pgfsetlinewidth{0.803000pt}%
\definecolor{currentstroke}{rgb}{1.000000,1.000000,1.000000}%
\pgfsetstrokecolor{currentstroke}%
\pgfsetdash{}{0pt}%
\pgfpathmoveto{\pgfqpoint{0.608070in}{0.523570in}}%
\pgfpathlineto{\pgfqpoint{6.808070in}{0.523570in}}%
\pgfusepath{stroke}%
\end{pgfscope}%
\begin{pgfscope}%
\pgfsetrectcap%
\pgfsetmiterjoin%
\pgfsetlinewidth{0.803000pt}%
\definecolor{currentstroke}{rgb}{1.000000,1.000000,1.000000}%
\pgfsetstrokecolor{currentstroke}%
\pgfsetdash{}{0pt}%
\pgfpathmoveto{\pgfqpoint{0.608070in}{2.788570in}}%
\pgfpathlineto{\pgfqpoint{6.808070in}{2.788570in}}%
\pgfusepath{stroke}%
\end{pgfscope}%
\begin{pgfscope}%
\pgfsetbuttcap%
\pgfsetmiterjoin%
\definecolor{currentfill}{rgb}{0.917647,0.917647,0.949020}%
\pgfsetfillcolor{currentfill}%
\pgfsetfillopacity{0.800000}%
\pgfsetlinewidth{1.003750pt}%
\definecolor{currentstroke}{rgb}{0.800000,0.800000,0.800000}%
\pgfsetstrokecolor{currentstroke}%
\pgfsetstrokeopacity{0.800000}%
\pgfsetdash{}{0pt}%
\pgfpathmoveto{\pgfqpoint{5.511466in}{2.267778in}}%
\pgfpathlineto{\pgfqpoint{6.710848in}{2.267778in}}%
\pgfpathquadraticcurveto{\pgfqpoint{6.738626in}{2.267778in}}{\pgfqpoint{6.738626in}{2.295556in}}%
\pgfpathlineto{\pgfqpoint{6.738626in}{2.691348in}}%
\pgfpathquadraticcurveto{\pgfqpoint{6.738626in}{2.719126in}}{\pgfqpoint{6.710848in}{2.719126in}}%
\pgfpathlineto{\pgfqpoint{5.511466in}{2.719126in}}%
\pgfpathquadraticcurveto{\pgfqpoint{5.483688in}{2.719126in}}{\pgfqpoint{5.483688in}{2.691348in}}%
\pgfpathlineto{\pgfqpoint{5.483688in}{2.295556in}}%
\pgfpathquadraticcurveto{\pgfqpoint{5.483688in}{2.267778in}}{\pgfqpoint{5.511466in}{2.267778in}}%
\pgfpathclose%
\pgfusepath{stroke,fill}%
\end{pgfscope}%
\begin{pgfscope}%
\pgfsetroundcap%
\pgfsetroundjoin%
\pgfsetlinewidth{1.505625pt}%
\definecolor{currentstroke}{rgb}{0.121569,0.466667,0.705882}%
\pgfsetstrokecolor{currentstroke}%
\pgfsetdash{}{0pt}%
\pgfpathmoveto{\pgfqpoint{5.539244in}{2.606658in}}%
\pgfpathlineto{\pgfqpoint{5.817022in}{2.606658in}}%
\pgfusepath{stroke}%
\end{pgfscope}%
\begin{pgfscope}%
\definecolor{textcolor}{rgb}{0.150000,0.150000,0.150000}%
\pgfsetstrokecolor{textcolor}%
\pgfsetfillcolor{textcolor}%
\pgftext[x=5.928133in,y=2.558047in,left,base]{\color{textcolor}\rmfamily\fontsize{10.000000}{12.000000}\selectfont Ljung-Box}%
\end{pgfscope}%
\begin{pgfscope}%
\pgfsetroundcap%
\pgfsetroundjoin%
\pgfsetlinewidth{1.505625pt}%
\definecolor{currentstroke}{rgb}{1.000000,0.498039,0.054902}%
\pgfsetstrokecolor{currentstroke}%
\pgfsetdash{}{0pt}%
\pgfpathmoveto{\pgfqpoint{5.539244in}{2.400834in}}%
\pgfpathlineto{\pgfqpoint{5.817022in}{2.400834in}}%
\pgfusepath{stroke}%
\end{pgfscope}%
\begin{pgfscope}%
\definecolor{textcolor}{rgb}{0.150000,0.150000,0.150000}%
\pgfsetstrokecolor{textcolor}%
\pgfsetfillcolor{textcolor}%
\pgftext[x=5.928133in,y=2.352223in,left,base]{\color{textcolor}\rmfamily\fontsize{10.000000}{12.000000}\selectfont Box-Pierce}%
\end{pgfscope}%
\end{pgfpicture}%
\makeatother%
\endgroup%

    \end{adjustbox}
    \caption{}
    \label{fig:ljungbox}
\end{figure}{}

\subsection{\textcolor{yellow}{Applying ARMA models}}
We start by fitting ARMA models to the time series. This turned out to be prone to numerical instability. While the BIC could always be calculated, for some of the models standard errors could not be computed as the algorithm was not able to invert the Hessian matrix. The underlying problem is exacerbated when dealing with GARCH models, as all residuals are squared. We eventually managed to alleviate the problem by multiplying log-returns by 100 (corresponding to an approximate percentage interpretation). Table \ref{tab:bic_arma} shows the BIC that were obtained from fitting different ARMA(p,q) models to the series of log-returns. To allow for comparison later on, models were fitted once with the original log-returns and once with the log-returns multiplied by 100. 

\begin{table}[h!]
    \centering
    \small
    \figuretitle{BIC for different combinations of ARMA(p,q) for V}
\begin{adjustbox}{width=.95\textwidth,center}
\begin{tabular}{lrrr}
\toprule
(p,q)   &          0     &       1       &     2 \\
\midrule
0 & -8870.52 & -8872.19 & -8868.22 \\
1 & -8871.40 & \textbf{-8872.93} & -8865.64 \\
2 & -8866.97 & -8865.65 & -8858.30 \\
3 & -8861.59 & -8859.10 & -8851.28 \\
4 & -8859.42 & -8854.27 & -8854.01 \\ 
\bottomrule
\end{tabular}
\quad
\begin{tabular}{lrrr}
\toprule
(p,q)   &          0     &       1       &     2 \\
\midrule
0 & 5027.88  & 5026.20  & 5030.17 \\
1 & 5026.99  & \textbf{5025.46}  & 5032.75 \\
2  & 5031.42  & 5032.75  & 5039.26 \\
3  & 5036.81  & 5039.29  & 5037.07 \\
4  & 5038.98  & 5044.13  & 5044.39 \\
\bottomrule
\end{tabular}
\end{adjustbox}

\hspace{6ex}
\newline
\hspace{6ex}
\newline
\figuretitle{BIC for different combinations of ARMA(p,q) for INTC}
\begin{adjustbox}{width=.95\textwidth,center}
\begin{tabular}{lrrr}
\toprule
(p,q)   &          0     &       1       &     2 \\
\midrule
0 & \textbf{-8692.98} & -8685.94 & -8678.63 \\
1 & -8685.94 & -8678.62 & -8671.31 \\
2 & -8678.63 & -8671.30 & -8672.74 \\
3 & -8671.31 & -8664.08 & -8664.26 \\
4 & -8664.18 & -8656.92 & -8652.19 \\
\bottomrule
\end{tabular}
\quad
\begin{tabular}{lrrr}
\toprule
(p,q)   &          0     &       1       &     2 \\
\midrule
0 & \textbf{5205.42} & 5212.45 & 5219.77 \\
1 & 5212.45 & 5219.77 & 5227.09 \\
2 & 5219.77 & 5227.09 & 5226.21 \\
3 & 5227.09 & 5234.32 & 5238.91 \\
4 & 5234.21 & 5241.47 & 5246.20 \\
\bottomrule
\end{tabular}
\end{adjustbox}
    \caption{BIC presented for different combinations of ARMA(p,q) fit to the log-returns of V (top) and INTC (bottom). On the right side, those returns were multiplied by 100 in order to allow for comparison with the GARCH models later on.}
    \label{tab:bic_arma}
\end{table}{}

For V the best model according to BIC is ARMA(1,1). However the difference to ARMA(0,0) seems as good as negligible. In order to avoid the peril of reading too much into random chance, it may therefore be more prudent to stick with a constant mean model (ARMA(0,0)). Recall that commons sense suggests that the past should not contain exploitable information about the future, which, however is an implication of assuming an ARMA(1,1) model. We will fit this model nevertheless, along with the constant mean model. For INTC the lowest BIC is reached by ARMA(0,0) which corresponds to our observation that there is no significant lower-level autocorrelation. 

\begin{table}[h!]
    \centering
    \figuretitle{Results for an ARMA(0,0) process fit to the log-returns of V}
    \vspace{-2ex}
    \small
    \begin{center}
\begin{tabular}{lclc}
\toprule
\textbf{Dep. Variable:} &   log\_returns   & \textbf{  No. Observations:  } &    1509     \\
\textbf{Model:}         &    ARMA(0, 0)    & \textbf{  Log Likelihood     } &  4442.581   \\
\textbf{Method:}        &       css        & \textbf{  S.D. of innovations} &   0.013     \\
\textbf{Date:}          & Wed, 04 Sep 2019 & \textbf{  AIC                } & -8881.162   \\
\textbf{Time:}          &     10:03:08     & \textbf{  BIC                } & -8870.524   \\
\textbf{Sample:}        &        0         & \textbf{  HQIC               } & -8877.200   \\
\bottomrule
\end{tabular}
\begin{tabular}{lcccccc}
               & \textbf{coef} & \textbf{std err} & \textbf{z} & \textbf{P$> |$z$|$} & \textbf{[0.025} & \textbf{0.975]}  \\
\midrule
\textbf{const} &       0.0011  &        0.000     &     3.253  &         0.001        &        0.000    &        0.002     \\
\bottomrule
\end{tabular}
%\caption{ARMA Model Results}
\end{center}

    \vspace{1ex}
    
    \figuretitle{Results for an ARMA(1,1) process fit to the log-returns of V}
    \vspace{-2ex}
    %\begin{adjustbox}{width = 0.75\linewidth}
    \small
    \begin{center}
\begin{tabular}{lclc}
\toprule
\textbf{Dep. Variable:}     &        log\_returns       & \textbf{  No. Observations:  } &            1509            \\
\textbf{Model:}             &         ARMA(1, 1)        & \textbf{  Log Likelihood     } &          4451.108          \\
\textbf{Method:}            &          css-mle          & \textbf{  S.D. of innovations} &           0.013            \\
\textbf{Date:}              &      Tue, 03 Sep 2019     & \textbf{  AIC                } &         -8894.215          \\
\textbf{Time:}              &          13:15:26         & \textbf{  BIC                } &         -8872.938          \\
\textbf{Sample:}            &             0             & \textbf{  HQIC               } &         -8886.291          \\
\bottomrule
\end{tabular}
\vspace{1ex}
\begin{tabular}{lcccccc}
                            & \textbf{coef} & \textbf{std err} & \textbf{z} & \textbf{P$> |$z$|$} & \textbf{[0.025} & \textbf{0.975]}  \\
\midrule
\textbf{const}              &       0.0011  &        0.000     &     4.194  &         0.000        &        0.001    &        0.002     \\
\textbf{ar.L1.log\_returns} &       0.6156  &        0.116     &     5.307  &         0.000        &        0.388    &        0.843     \\
\textbf{ma.L1.log\_returns} &      -0.6997  &        0.105     &    -6.681  &         0.000        &       -0.905    &       -0.494     \\
\bottomrule
\end{tabular}
\vspace{1ex}
\begin{tabular}{lcccc}
 \textbf{Roots: }             & \textbf{            Real} & \textbf{         Imaginary} & \textbf{         Modulus} & \textbf{        Frequency}  \\
\midrule
\textbf{AR.1} &                1.6243     &                +0.0000j     &                1.6243     &                0.0000       \\
\textbf{MA.1} &                1.4293     &                +0.0000j     &                1.4293     &                0.0000       \\
\bottomrule
\end{tabular}
%\caption{ARMA Model Results}
\end{center}

    %\end{adjustbox}
    \vspace{-2ex}
    \caption{Results for the ARMA(0,0) and ARMA(1,1) model fit to the log-returns of V.}
    \label{tab:V_ARMA_log_returns}
\end{table}

Table \ref{tab:V_ARMA_log_returns} shows the results of both models for V. The change in AIC is larger than the change in BIC as the latter more heavily penalizes the number of parameters included in the estimation. Notably, even though the difference in BIC is quite small, the AR(1) and MA(1) are estimated very significantly and are high in relative magnitude. However, note that their effects go in opposing directions in similar magnitude and may as well cancel each other out. This interpretation may be somewhat plausible as the individual effects of the AR(1) and MA(1) terms are about an order of magnitude smaller (albeit still significant) and go in the same direction when estimating ARMA(1,0) and ARMA(0,1) models separately (-0.0736 for the AR(1) and -0.0809 for the MA(1) term in separate models). 
Table \ref{tab:INTC_ARMA00_log_returns} shows the result of the ARMA(0,0) model to the log-returns of INTC. Not even the mean is significant, which is not too surprising, as Intel hardly gained in value in the observed period from 2012 to 2017 (See again figure \ref{fig:Daily Stock Prices for all Stocks in the Data Set}. When exploratively fitting an ARMA(1,0), ARMA(0,1) or ARMA(1,1) model, as expected none of the coefficients reach significance (p-values all > 0.5). 

\begin{table}[h!]
    \centering
    \figuretitle{Results for an ARMA(0,0) process fit to the log-returns of INTC}
    \vspace{-2ex}
    \small
    \begin{center}
\begin{tabular}{lclc}
\toprule
\textbf{Dep. Variable:} &   log\_returns   & \textbf{  No. Observations:  } &    1509     \\
\textbf{Model:}         &    ARMA(0, 0)    & \textbf{  Log Likelihood     } &  4353.809   \\
\textbf{Method:}        &       css        & \textbf{  S.D. of innovations} &   0.014     \\
\textbf{Date:}          & Tue, 03 Sep 2019 & \textbf{  AIC                } & -8703.618   \\
\textbf{Time:}          &     14:04:11     & \textbf{  BIC                } & -8692.979   \\
\textbf{Sample:}        &        0         & \textbf{  HQIC               } & -8699.656   \\
\bottomrule
\end{tabular}
\begin{tabular}{lcccccc}
               & \textbf{coef} & \textbf{std err} & \textbf{z} & \textbf{P$> |$z$|$} & \textbf{[0.025} & \textbf{0.975]}  \\
\midrule
\textbf{const} &       0.0005  &        0.000     &     1.574  &         0.116        &       -0.000    &        0.001     \\
\bottomrule
\end{tabular}
%\caption{ARMA Model Results}
\end{center}

    \caption{Results for an ARMA(0,0) process fit to the log-returns of INTC}
    \vspace{-2ex}
    \label{tab:INTC_ARMA00_log_returns}
\end{table}{}

\subsubsection{\textcolor{yellow}[ARMAX models]}
We proceed in our analysis by adding our own generated sentiments as well as the ravenpack sentiments as external regressors. To make matters more complicated the stock market observations don't perfectly match the external data data points. We have about a third as many analyst reports (and therefore sentiment scores) as stock market observations. In order to obtain an estimable model we decided to discard all observations on days where no analyst reports existed. This also implies that we need to confine our analysis to a constant mean model (ARMA(0,0) as autoregressive and moving-average components do not make sense when many observations in the time series are missing. The time series model hence reduces to a regression with an intercept and the sentiment data as independent variables. While analyst reports were rather scarce, the Ravenpack data usually had multiple entries per day with only few days missing. We therefore needed to aggregate sentiments to obtain one single observation per day. 


\textit{ARMAX Sentiments Analyst Reports}

The fact that we need to omit two-thirds of all data points changes the BIC considerably. To be able to compare the models according to BIC we have refit the baseline ARMA(0,0) model to the reduced data and obtained a BIC of -3001.07 for V and -4533.63 for INTC (the difference owing to the different number of observations). Table \ref{tab:result_ARMAX00_sentiment} shows the results of the ARMAX model for V and INTC. Even though sent\_mean is not too far from significance for V (and is even closer when only including sent\_mean, with a p-value of 0.067), all models fare worse in terms of BIC than the baseline. 

\begin{table}[h!]
    \centering
    \figuretitle{ARMAX(0,0) with analyst report sentiments fit to the log-returns of V}
    \vspace{-2ex}
    \small
    \begin{center}
\begin{tabular}{lclc}
\toprule
\textbf{Dep. Variable:} &   log\_returns   & \textbf{  No. Observations:  } &    535      \\
\textbf{Model:}         &    ARMA(0, 0)    & \textbf{  Log Likelihood     } &  1508.522   \\
\textbf{Method:}        &       css        & \textbf{  S.D. of innovations} &   0.014     \\
\textbf{Date:}          & Wed, 04 Sep 2019 & \textbf{  AIC                } & -3009.044   \\
\textbf{Time:}          &     17:06:21     & \textbf{  BIC                } & -2991.915   \\
\textbf{Sample:}        &        0         & \textbf{  HQIC               } & -3002.342   \\
\bottomrule
\end{tabular}
\begin{tabular}{lcccccc}
                       & \textbf{coef} & \textbf{std err} & \textbf{z} & \textbf{P$> |$z$|$} & \textbf{[0.025} & \textbf{0.975]}  \\
\midrule
\textbf{const}         &       0.0045  &        0.002     &     2.217  &         0.027        &        0.001    &        0.009     \\
\textbf{sent1\_mean}   &      -0.0047  &        0.003     &    -1.602  &         0.110        &       -0.010    &        0.001     \\
\textbf{sentBoW\_mean} &      -0.0007  &        0.003     &    -0.241  &         0.810        &       -0.006    &        0.005     \\
\bottomrule
\end{tabular}
%\caption{ARMA Model Results}
\end{center}

    \vspace{1ex}
    
    \figuretitle{ARMAX(0,0) with analyst report sentiments fit to the log-returns of INTC}
    \vspace{-2ex}
    \small
    \begin{center}
\begin{tabular}{lclc}
\toprule
\textbf{Dep. Variable:} &   log\_returns   & \textbf{  No. Observations:  } &    821      \\
\textbf{Model:}         &    ARMA(0, 0)    & \textbf{  Log Likelihood     } &  2274.402   \\
\textbf{Method:}        &       css        & \textbf{  S.D. of innovations} &   0.015     \\
\textbf{Date:}          & Sat, 14 Sep 2019 & \textbf{  AIC                } & -4540.805   \\
\textbf{Time:}          &     10:53:20     & \textbf{  BIC                } & -4521.963   \\
\textbf{Sample:}        &        0         & \textbf{  HQIC               } & -4533.575   \\
\bottomrule
\end{tabular}
\begin{tabular}{lcccccc}
                       & \textbf{coef} & \textbf{std err} & \textbf{z} & \textbf{P$> |$z$|$} & \textbf{[0.025} & \textbf{0.975]}  \\
\midrule
\textbf{const}         &       0.0031  &        0.002     &     1.766  &         0.078        &       -0.000    &        0.006     \\
\textbf{sent1\_mean}   &      -0.0035  &        0.003     &    -1.215  &         0.225        &       -0.009    &        0.002     \\
\textbf{sentBoW\_mean} &       0.0024  &        0.002     &     0.993  &         0.321        &       -0.002    &        0.007     \\
\bottomrule
\end{tabular}
%\caption{ARMA Model Results}
\end{center}

    \caption{Results for ARMAX(0,0), i.e. a regression with a constant and our own sentiment data as external regressors.}
    \label{tab:result_ARMAX00_sentiment}
\end{table}{}

\textit{ARMAX Sentiments Ravenpack}

In the Ravenpack data for V, 21 observations out of 1509 were missing. We decided it might still be interesting to continue having a look at ARMA(1,1) even though the estimation is now mildly distorted by the fact that some values are missing. We computed a new baseline BIC as we did in the previous analysis. The BIC for an ARMA(0,0) model with the 21 omitted values is now -8734.35 and for an ARMA(1,1) model on the same data -8735.23. For INTC there were no missing observations, so the baseline BIC stayed -8692.98. Table \ref{tab:result_ARMAX00_ravenpack} shows the results for the ARMAX(0,0) model fit with Ravenpack sentiments to V and INTC. We have tried different combinations of regressors to include in the model but only show the full model to avoid redundancy. The results for ARMAX(1,1) for V are not shown as they look similar to a combination of the tables for ARMA(1,1) and ARMAX(0,0) for V. In all cases the external information worsened BIC and did not improve the fit. For the following GARCH analysis we will therefore not include external information. 

\begin{table}[h!]
    \centering
    \figuretitle{ARMAX(0,0) with Ravenpack sentiments fit to the log-returns of V}
    \vspace{-2ex}
    \small
    \begin{center}
\begin{tabular}{lclc}
\toprule
\textbf{Dep. Variable:} &   log\_returns   & \textbf{  No. Observations:  } &    1488     \\
\textbf{Model:}         &    ARMA(0, 0)    & \textbf{  Log Likelihood     } &  4375.394   \\
\textbf{Method:}        &       css        & \textbf{  S.D. of innovations} &   0.013     \\
\textbf{Date:}          & Wed, 04 Sep 2019 & \textbf{  AIC                } & -8736.788   \\
\textbf{Time:}          &     17:23:40     & \textbf{  BIC                } & -8699.652   \\
\textbf{Sample:}        &        0         & \textbf{  HQIC               } & -8722.948   \\
\bottomrule
\end{tabular}
\begin{tabular}{lcccccc}
                   & \textbf{coef} & \textbf{std err} & \textbf{z} & \textbf{P$> |$z$|$} & \textbf{[0.025} & \textbf{0.975]}  \\
\midrule
\textbf{const}     &       0.0124  &        0.012     &     1.061  &         0.289        &       -0.010    &        0.035     \\
\textbf{relevance} &      -0.0001  &        0.000     &    -1.037  &         0.300        &       -0.000    &        0.000     \\
\textbf{aes\_min}  &   -1.536e-05  &     2.42e-05     &    -0.634  &         0.526        &    -6.29e-05    &     3.22e-05     \\
\textbf{aes\_max}  &    1.781e-05  &     3.35e-05     &     0.531  &         0.595        &    -4.79e-05    &     8.35e-05     \\
\textbf{count}     &    5.652e-06  &     1.41e-05     &     0.402  &         0.688        &    -2.19e-05    &     3.32e-05     \\
\textbf{aev\_min}  &    2.091e-07  &     2.53e-06     &     0.083  &         0.934        &    -4.76e-06    &     5.18e-06     \\
\bottomrule
\end{tabular}
%\caption{ARMA Model Results}
\end{center}


    \figuretitle{ARMAX(0,0) with Ravenpack sentiments fit to the log-returns of INTC}
    \vspace{-2ex}
    \small
    \begin{center}
\begin{tabular}{lclc}
\toprule
\textbf{Dep. Variable:} &   log\_returns   & \textbf{  No. Observations:  } &    1509     \\
\textbf{Model:}         &    ARMA(0, 0)    & \textbf{  Log Likelihood     } &  4358.128   \\
\textbf{Method:}        &       css        & \textbf{  S.D. of innovations} &   0.013     \\
\textbf{Date:}          & Tue, 03 Sep 2019 & \textbf{  AIC                } & -8700.255   \\
\textbf{Time:}          &     13:42:47     & \textbf{  BIC                } & -8657.702   \\
\textbf{Sample:}        &        0         & \textbf{  HQIC               } & -8684.407   \\
\bottomrule
\end{tabular}
\begin{tabular}{lcccccc}
                   & \textbf{coef} & \textbf{std err} & \textbf{z} & \textbf{P$> |$z$|$} & \textbf{[0.025} & \textbf{0.975]}  \\
\midrule
\textbf{const}     &      -0.0322  &        0.019     &    -1.734  &         0.083        &       -0.069    &        0.004     \\
\textbf{relevance} &       0.0004  &        0.000     &     1.865  &         0.062        &    -1.85e-05    &        0.001     \\
\textbf{aes\_min}  &     3.57e-05  &     2.71e-05     &     1.317  &         0.188        &    -1.74e-05    &     8.88e-05     \\
\textbf{aes\_max}  &   -2.438e-05  &      3.3e-05     &    -0.738  &         0.460        &    -8.91e-05    &     4.03e-05     \\
\textbf{count}     &   -7.535e-07  &     6.03e-06     &    -0.125  &         0.901        &    -1.26e-05    &     1.11e-05     \\
\textbf{aev\_min}  &   -1.312e-06  &     1.82e-06     &    -0.719  &         0.472        &    -4.89e-06    &     2.26e-06     \\
\textbf{aev\_max}  &   -2.601e-06  &     1.57e-06     &    -1.661  &         0.097        &    -5.67e-06    &     4.67e-07     \\
\bottomrule
\end{tabular}
%\caption{ARMA Model Results}
\end{center}

    \caption{Results for ARMAX(0,0), i.e. a regression with a constant and the Ravenpack sentiment data as external regressors.}
    \label{tab:INTC_result_ARMAX00_ravenpack}
\end{table}{}

\subsubsection{\textcolor{red}{GARCH models}}
Financial data often exhibit conditional heteroskedasticity. We therefore want to see whether GARCH models are able to improve the fit. Figure \ref{fig:V_INTC_squared} shows the squared log-returns of V and INTC as well as their ACF and PACF. For both the first lag visually seems to be significant. 

\begin{figure}[h!]
    \centering
    \figuretitle{Squared log-returns V and INTC}
    \begin{adjustbox}{width=.95\textwidth,center}
    %% Creator: Matplotlib, PGF backend
%%
%% To include the figure in your LaTeX document, write
%%   \input{<filename>.pgf}
%%
%% Make sure the required packages are loaded in your preamble
%%   \usepackage{pgf}
%%
%% Figures using additional raster images can only be included by \input if
%% they are in the same directory as the main LaTeX file. For loading figures
%% from other directories you can use the `import` package
%%   \usepackage{import}
%% and then include the figures with
%%   \import{<path to file>}{<filename>.pgf}
%%
%% Matplotlib used the following preamble
%%   \usepackage{fontspec}
%%   \setmainfont{DejaVuSerif.ttf}[Path=/opt/tljh/user/lib/python3.6/site-packages/matplotlib/mpl-data/fonts/ttf/]
%%   \setsansfont{DejaVuSans.ttf}[Path=/opt/tljh/user/lib/python3.6/site-packages/matplotlib/mpl-data/fonts/ttf/]
%%   \setmonofont{DejaVuSansMono.ttf}[Path=/opt/tljh/user/lib/python3.6/site-packages/matplotlib/mpl-data/fonts/ttf/]
%%
\begingroup%
\makeatletter%
\begin{pgfpicture}%
\pgfpathrectangle{\pgfpointorigin}{\pgfqpoint{3.922887in}{1.941635in}}%
\pgfusepath{use as bounding box, clip}%
\begin{pgfscope}%
\pgfsetbuttcap%
\pgfsetmiterjoin%
\definecolor{currentfill}{rgb}{1.000000,1.000000,1.000000}%
\pgfsetfillcolor{currentfill}%
\pgfsetlinewidth{0.000000pt}%
\definecolor{currentstroke}{rgb}{1.000000,1.000000,1.000000}%
\pgfsetstrokecolor{currentstroke}%
\pgfsetdash{}{0pt}%
\pgfpathmoveto{\pgfqpoint{0.000000in}{0.000000in}}%
\pgfpathlineto{\pgfqpoint{3.922887in}{0.000000in}}%
\pgfpathlineto{\pgfqpoint{3.922887in}{1.941635in}}%
\pgfpathlineto{\pgfqpoint{0.000000in}{1.941635in}}%
\pgfpathclose%
\pgfusepath{fill}%
\end{pgfscope}%
\begin{pgfscope}%
\pgfsetbuttcap%
\pgfsetmiterjoin%
\definecolor{currentfill}{rgb}{0.917647,0.917647,0.949020}%
\pgfsetfillcolor{currentfill}%
\pgfsetlinewidth{0.000000pt}%
\definecolor{currentstroke}{rgb}{0.000000,0.000000,0.000000}%
\pgfsetstrokecolor{currentstroke}%
\pgfsetstrokeopacity{0.000000}%
\pgfsetdash{}{0pt}%
\pgfpathmoveto{\pgfqpoint{0.683198in}{0.331635in}}%
\pgfpathlineto{\pgfqpoint{3.783198in}{0.331635in}}%
\pgfpathlineto{\pgfqpoint{3.783198in}{1.841635in}}%
\pgfpathlineto{\pgfqpoint{0.683198in}{1.841635in}}%
\pgfpathclose%
\pgfusepath{fill}%
\end{pgfscope}%
\begin{pgfscope}%
\pgfpathrectangle{\pgfqpoint{0.683198in}{0.331635in}}{\pgfqpoint{3.100000in}{1.510000in}}%
\pgfusepath{clip}%
\pgfsetroundcap%
\pgfsetroundjoin%
\pgfsetlinewidth{0.803000pt}%
\definecolor{currentstroke}{rgb}{1.000000,1.000000,1.000000}%
\pgfsetstrokecolor{currentstroke}%
\pgfsetdash{}{0pt}%
\pgfpathmoveto{\pgfqpoint{0.820239in}{0.331635in}}%
\pgfpathlineto{\pgfqpoint{0.820239in}{1.841635in}}%
\pgfusepath{stroke}%
\end{pgfscope}%
\begin{pgfscope}%
\definecolor{textcolor}{rgb}{0.150000,0.150000,0.150000}%
\pgfsetstrokecolor{textcolor}%
\pgfsetfillcolor{textcolor}%
\pgftext[x=0.820239in,y=0.234413in,,top]{\color{textcolor}\rmfamily\fontsize{10.000000}{12.000000}\selectfont 2012}%
\end{pgfscope}%
\begin{pgfscope}%
\pgfpathrectangle{\pgfqpoint{0.683198in}{0.331635in}}{\pgfqpoint{3.100000in}{1.510000in}}%
\pgfusepath{clip}%
\pgfsetroundcap%
\pgfsetroundjoin%
\pgfsetlinewidth{0.803000pt}%
\definecolor{currentstroke}{rgb}{1.000000,1.000000,1.000000}%
\pgfsetstrokecolor{currentstroke}%
\pgfsetdash{}{0pt}%
\pgfpathmoveto{\pgfqpoint{1.292085in}{0.331635in}}%
\pgfpathlineto{\pgfqpoint{1.292085in}{1.841635in}}%
\pgfusepath{stroke}%
\end{pgfscope}%
\begin{pgfscope}%
\definecolor{textcolor}{rgb}{0.150000,0.150000,0.150000}%
\pgfsetstrokecolor{textcolor}%
\pgfsetfillcolor{textcolor}%
\pgftext[x=1.292085in,y=0.234413in,,top]{\color{textcolor}\rmfamily\fontsize{10.000000}{12.000000}\selectfont 2013}%
\end{pgfscope}%
\begin{pgfscope}%
\pgfpathrectangle{\pgfqpoint{0.683198in}{0.331635in}}{\pgfqpoint{3.100000in}{1.510000in}}%
\pgfusepath{clip}%
\pgfsetroundcap%
\pgfsetroundjoin%
\pgfsetlinewidth{0.803000pt}%
\definecolor{currentstroke}{rgb}{1.000000,1.000000,1.000000}%
\pgfsetstrokecolor{currentstroke}%
\pgfsetdash{}{0pt}%
\pgfpathmoveto{\pgfqpoint{1.762641in}{0.331635in}}%
\pgfpathlineto{\pgfqpoint{1.762641in}{1.841635in}}%
\pgfusepath{stroke}%
\end{pgfscope}%
\begin{pgfscope}%
\definecolor{textcolor}{rgb}{0.150000,0.150000,0.150000}%
\pgfsetstrokecolor{textcolor}%
\pgfsetfillcolor{textcolor}%
\pgftext[x=1.762641in,y=0.234413in,,top]{\color{textcolor}\rmfamily\fontsize{10.000000}{12.000000}\selectfont 2014}%
\end{pgfscope}%
\begin{pgfscope}%
\pgfpathrectangle{\pgfqpoint{0.683198in}{0.331635in}}{\pgfqpoint{3.100000in}{1.510000in}}%
\pgfusepath{clip}%
\pgfsetroundcap%
\pgfsetroundjoin%
\pgfsetlinewidth{0.803000pt}%
\definecolor{currentstroke}{rgb}{1.000000,1.000000,1.000000}%
\pgfsetstrokecolor{currentstroke}%
\pgfsetdash{}{0pt}%
\pgfpathmoveto{\pgfqpoint{2.233198in}{0.331635in}}%
\pgfpathlineto{\pgfqpoint{2.233198in}{1.841635in}}%
\pgfusepath{stroke}%
\end{pgfscope}%
\begin{pgfscope}%
\definecolor{textcolor}{rgb}{0.150000,0.150000,0.150000}%
\pgfsetstrokecolor{textcolor}%
\pgfsetfillcolor{textcolor}%
\pgftext[x=2.233198in,y=0.234413in,,top]{\color{textcolor}\rmfamily\fontsize{10.000000}{12.000000}\selectfont 2015}%
\end{pgfscope}%
\begin{pgfscope}%
\pgfpathrectangle{\pgfqpoint{0.683198in}{0.331635in}}{\pgfqpoint{3.100000in}{1.510000in}}%
\pgfusepath{clip}%
\pgfsetroundcap%
\pgfsetroundjoin%
\pgfsetlinewidth{0.803000pt}%
\definecolor{currentstroke}{rgb}{1.000000,1.000000,1.000000}%
\pgfsetstrokecolor{currentstroke}%
\pgfsetdash{}{0pt}%
\pgfpathmoveto{\pgfqpoint{2.703754in}{0.331635in}}%
\pgfpathlineto{\pgfqpoint{2.703754in}{1.841635in}}%
\pgfusepath{stroke}%
\end{pgfscope}%
\begin{pgfscope}%
\definecolor{textcolor}{rgb}{0.150000,0.150000,0.150000}%
\pgfsetstrokecolor{textcolor}%
\pgfsetfillcolor{textcolor}%
\pgftext[x=2.703754in,y=0.234413in,,top]{\color{textcolor}\rmfamily\fontsize{10.000000}{12.000000}\selectfont 2016}%
\end{pgfscope}%
\begin{pgfscope}%
\pgfpathrectangle{\pgfqpoint{0.683198in}{0.331635in}}{\pgfqpoint{3.100000in}{1.510000in}}%
\pgfusepath{clip}%
\pgfsetroundcap%
\pgfsetroundjoin%
\pgfsetlinewidth{0.803000pt}%
\definecolor{currentstroke}{rgb}{1.000000,1.000000,1.000000}%
\pgfsetstrokecolor{currentstroke}%
\pgfsetdash{}{0pt}%
\pgfpathmoveto{\pgfqpoint{3.175600in}{0.331635in}}%
\pgfpathlineto{\pgfqpoint{3.175600in}{1.841635in}}%
\pgfusepath{stroke}%
\end{pgfscope}%
\begin{pgfscope}%
\definecolor{textcolor}{rgb}{0.150000,0.150000,0.150000}%
\pgfsetstrokecolor{textcolor}%
\pgfsetfillcolor{textcolor}%
\pgftext[x=3.175600in,y=0.234413in,,top]{\color{textcolor}\rmfamily\fontsize{10.000000}{12.000000}\selectfont 2017}%
\end{pgfscope}%
\begin{pgfscope}%
\pgfpathrectangle{\pgfqpoint{0.683198in}{0.331635in}}{\pgfqpoint{3.100000in}{1.510000in}}%
\pgfusepath{clip}%
\pgfsetroundcap%
\pgfsetroundjoin%
\pgfsetlinewidth{0.803000pt}%
\definecolor{currentstroke}{rgb}{1.000000,1.000000,1.000000}%
\pgfsetstrokecolor{currentstroke}%
\pgfsetdash{}{0pt}%
\pgfpathmoveto{\pgfqpoint{3.646156in}{0.331635in}}%
\pgfpathlineto{\pgfqpoint{3.646156in}{1.841635in}}%
\pgfusepath{stroke}%
\end{pgfscope}%
\begin{pgfscope}%
\definecolor{textcolor}{rgb}{0.150000,0.150000,0.150000}%
\pgfsetstrokecolor{textcolor}%
\pgfsetfillcolor{textcolor}%
\pgftext[x=3.646156in,y=0.234413in,,top]{\color{textcolor}\rmfamily\fontsize{10.000000}{12.000000}\selectfont 2018}%
\end{pgfscope}%
\begin{pgfscope}%
\pgfpathrectangle{\pgfqpoint{0.683198in}{0.331635in}}{\pgfqpoint{3.100000in}{1.510000in}}%
\pgfusepath{clip}%
\pgfsetroundcap%
\pgfsetroundjoin%
\pgfsetlinewidth{0.803000pt}%
\definecolor{currentstroke}{rgb}{1.000000,1.000000,1.000000}%
\pgfsetstrokecolor{currentstroke}%
\pgfsetdash{}{0pt}%
\pgfpathmoveto{\pgfqpoint{0.683198in}{0.400271in}}%
\pgfpathlineto{\pgfqpoint{3.783198in}{0.400271in}}%
\pgfusepath{stroke}%
\end{pgfscope}%
\begin{pgfscope}%
\definecolor{textcolor}{rgb}{0.150000,0.150000,0.150000}%
\pgfsetstrokecolor{textcolor}%
\pgfsetfillcolor{textcolor}%
\pgftext[x=0.100000in,y=0.347510in,left,base]{\color{textcolor}\rmfamily\fontsize{10.000000}{12.000000}\selectfont 0.0000}%
\end{pgfscope}%
\begin{pgfscope}%
\pgfpathrectangle{\pgfqpoint{0.683198in}{0.331635in}}{\pgfqpoint{3.100000in}{1.510000in}}%
\pgfusepath{clip}%
\pgfsetroundcap%
\pgfsetroundjoin%
\pgfsetlinewidth{0.803000pt}%
\definecolor{currentstroke}{rgb}{1.000000,1.000000,1.000000}%
\pgfsetstrokecolor{currentstroke}%
\pgfsetdash{}{0pt}%
\pgfpathmoveto{\pgfqpoint{0.683198in}{0.760926in}}%
\pgfpathlineto{\pgfqpoint{3.783198in}{0.760926in}}%
\pgfusepath{stroke}%
\end{pgfscope}%
\begin{pgfscope}%
\definecolor{textcolor}{rgb}{0.150000,0.150000,0.150000}%
\pgfsetstrokecolor{textcolor}%
\pgfsetfillcolor{textcolor}%
\pgftext[x=0.100000in,y=0.708165in,left,base]{\color{textcolor}\rmfamily\fontsize{10.000000}{12.000000}\selectfont 0.0025}%
\end{pgfscope}%
\begin{pgfscope}%
\pgfpathrectangle{\pgfqpoint{0.683198in}{0.331635in}}{\pgfqpoint{3.100000in}{1.510000in}}%
\pgfusepath{clip}%
\pgfsetroundcap%
\pgfsetroundjoin%
\pgfsetlinewidth{0.803000pt}%
\definecolor{currentstroke}{rgb}{1.000000,1.000000,1.000000}%
\pgfsetstrokecolor{currentstroke}%
\pgfsetdash{}{0pt}%
\pgfpathmoveto{\pgfqpoint{0.683198in}{1.121581in}}%
\pgfpathlineto{\pgfqpoint{3.783198in}{1.121581in}}%
\pgfusepath{stroke}%
\end{pgfscope}%
\begin{pgfscope}%
\definecolor{textcolor}{rgb}{0.150000,0.150000,0.150000}%
\pgfsetstrokecolor{textcolor}%
\pgfsetfillcolor{textcolor}%
\pgftext[x=0.100000in,y=1.068820in,left,base]{\color{textcolor}\rmfamily\fontsize{10.000000}{12.000000}\selectfont 0.0050}%
\end{pgfscope}%
\begin{pgfscope}%
\pgfpathrectangle{\pgfqpoint{0.683198in}{0.331635in}}{\pgfqpoint{3.100000in}{1.510000in}}%
\pgfusepath{clip}%
\pgfsetroundcap%
\pgfsetroundjoin%
\pgfsetlinewidth{0.803000pt}%
\definecolor{currentstroke}{rgb}{1.000000,1.000000,1.000000}%
\pgfsetstrokecolor{currentstroke}%
\pgfsetdash{}{0pt}%
\pgfpathmoveto{\pgfqpoint{0.683198in}{1.482236in}}%
\pgfpathlineto{\pgfqpoint{3.783198in}{1.482236in}}%
\pgfusepath{stroke}%
\end{pgfscope}%
\begin{pgfscope}%
\definecolor{textcolor}{rgb}{0.150000,0.150000,0.150000}%
\pgfsetstrokecolor{textcolor}%
\pgfsetfillcolor{textcolor}%
\pgftext[x=0.100000in,y=1.429475in,left,base]{\color{textcolor}\rmfamily\fontsize{10.000000}{12.000000}\selectfont 0.0075}%
\end{pgfscope}%
\begin{pgfscope}%
\pgfpathrectangle{\pgfqpoint{0.683198in}{0.331635in}}{\pgfqpoint{3.100000in}{1.510000in}}%
\pgfusepath{clip}%
\pgfsetroundcap%
\pgfsetroundjoin%
\pgfsetlinewidth{1.505625pt}%
\definecolor{currentstroke}{rgb}{0.737255,0.741176,0.133333}%
\pgfsetstrokecolor{currentstroke}%
\pgfsetdash{}{0pt}%
\pgfpathmoveto{\pgfqpoint{0.824107in}{0.446565in}}%
\pgfpathlineto{\pgfqpoint{0.825396in}{0.408720in}}%
\pgfpathlineto{\pgfqpoint{0.826685in}{0.420113in}}%
\pgfpathlineto{\pgfqpoint{0.830553in}{0.409973in}}%
\pgfpathlineto{\pgfqpoint{0.831842in}{0.400392in}}%
\pgfpathlineto{\pgfqpoint{0.833131in}{0.407123in}}%
\pgfpathlineto{\pgfqpoint{0.834420in}{0.469389in}}%
\pgfpathlineto{\pgfqpoint{0.835709in}{0.405261in}}%
\pgfpathlineto{\pgfqpoint{0.840866in}{0.444656in}}%
\pgfpathlineto{\pgfqpoint{0.842155in}{0.422475in}}%
\pgfpathlineto{\pgfqpoint{0.843445in}{0.427524in}}%
\pgfpathlineto{\pgfqpoint{0.844734in}{0.444776in}}%
\pgfpathlineto{\pgfqpoint{0.848601in}{0.414830in}}%
\pgfpathlineto{\pgfqpoint{0.849891in}{0.429060in}}%
\pgfpathlineto{\pgfqpoint{0.851180in}{0.403238in}}%
\pgfpathlineto{\pgfqpoint{0.853758in}{0.400744in}}%
\pgfpathlineto{\pgfqpoint{0.857626in}{0.420458in}}%
\pgfpathlineto{\pgfqpoint{0.858915in}{0.408937in}}%
\pgfpathlineto{\pgfqpoint{0.860204in}{0.449369in}}%
\pgfpathlineto{\pgfqpoint{0.861493in}{0.568847in}}%
\pgfpathlineto{\pgfqpoint{0.862783in}{0.411975in}}%
\pgfpathlineto{\pgfqpoint{0.866650in}{0.401556in}}%
\pgfpathlineto{\pgfqpoint{0.867939in}{0.401950in}}%
\pgfpathlineto{\pgfqpoint{0.869229in}{0.423663in}}%
\pgfpathlineto{\pgfqpoint{0.870518in}{0.596163in}}%
\pgfpathlineto{\pgfqpoint{0.871807in}{0.424368in}}%
\pgfpathlineto{\pgfqpoint{0.875675in}{0.416141in}}%
\pgfpathlineto{\pgfqpoint{0.876964in}{0.470464in}}%
\pgfpathlineto{\pgfqpoint{0.878253in}{0.406787in}}%
\pgfpathlineto{\pgfqpoint{0.879542in}{0.419324in}}%
\pgfpathlineto{\pgfqpoint{0.885988in}{0.407591in}}%
\pgfpathlineto{\pgfqpoint{0.887277in}{0.451712in}}%
\pgfpathlineto{\pgfqpoint{0.888567in}{0.400359in}}%
\pgfpathlineto{\pgfqpoint{0.889856in}{0.412755in}}%
\pgfpathlineto{\pgfqpoint{0.893723in}{0.405131in}}%
\pgfpathlineto{\pgfqpoint{0.896302in}{0.469907in}}%
\pgfpathlineto{\pgfqpoint{0.897591in}{0.407321in}}%
\pgfpathlineto{\pgfqpoint{0.898880in}{0.411803in}}%
\pgfpathlineto{\pgfqpoint{0.902748in}{0.400359in}}%
\pgfpathlineto{\pgfqpoint{0.904037in}{0.421634in}}%
\pgfpathlineto{\pgfqpoint{0.905326in}{0.407511in}}%
\pgfpathlineto{\pgfqpoint{0.906615in}{0.456843in}}%
\pgfpathlineto{\pgfqpoint{0.907904in}{0.407228in}}%
\pgfpathlineto{\pgfqpoint{0.911772in}{0.404532in}}%
\pgfpathlineto{\pgfqpoint{0.913061in}{0.406547in}}%
\pgfpathlineto{\pgfqpoint{0.915640in}{0.401053in}}%
\pgfpathlineto{\pgfqpoint{0.916929in}{0.401336in}}%
\pgfpathlineto{\pgfqpoint{0.920796in}{0.447542in}}%
\pgfpathlineto{\pgfqpoint{0.922086in}{0.456017in}}%
\pgfpathlineto{\pgfqpoint{0.923375in}{0.400817in}}%
\pgfpathlineto{\pgfqpoint{0.924664in}{0.403392in}}%
\pgfpathlineto{\pgfqpoint{0.925953in}{0.423467in}}%
\pgfpathlineto{\pgfqpoint{0.929821in}{0.416586in}}%
\pgfpathlineto{\pgfqpoint{0.931110in}{0.400787in}}%
\pgfpathlineto{\pgfqpoint{0.932399in}{0.402783in}}%
\pgfpathlineto{\pgfqpoint{0.933688in}{0.401295in}}%
\pgfpathlineto{\pgfqpoint{0.934978in}{0.410498in}}%
\pgfpathlineto{\pgfqpoint{0.938845in}{0.409593in}}%
\pgfpathlineto{\pgfqpoint{0.940134in}{0.420186in}}%
\pgfpathlineto{\pgfqpoint{0.941424in}{0.420186in}}%
\pgfpathlineto{\pgfqpoint{0.942713in}{0.442014in}}%
\pgfpathlineto{\pgfqpoint{0.947870in}{0.422618in}}%
\pgfpathlineto{\pgfqpoint{0.949159in}{0.479413in}}%
\pgfpathlineto{\pgfqpoint{0.950448in}{0.404516in}}%
\pgfpathlineto{\pgfqpoint{0.951737in}{0.533977in}}%
\pgfpathlineto{\pgfqpoint{0.953026in}{0.444264in}}%
\pgfpathlineto{\pgfqpoint{0.956894in}{0.452148in}}%
\pgfpathlineto{\pgfqpoint{0.958183in}{0.416042in}}%
\pgfpathlineto{\pgfqpoint{0.959472in}{0.401249in}}%
\pgfpathlineto{\pgfqpoint{0.960761in}{0.403670in}}%
\pgfpathlineto{\pgfqpoint{0.962051in}{0.400595in}}%
\pgfpathlineto{\pgfqpoint{0.965918in}{0.485319in}}%
\pgfpathlineto{\pgfqpoint{0.967207in}{0.407117in}}%
\pgfpathlineto{\pgfqpoint{0.968497in}{0.481599in}}%
\pgfpathlineto{\pgfqpoint{0.969786in}{0.419282in}}%
\pgfpathlineto{\pgfqpoint{0.971075in}{0.401227in}}%
\pgfpathlineto{\pgfqpoint{0.974943in}{0.403085in}}%
\pgfpathlineto{\pgfqpoint{0.976232in}{0.400448in}}%
\pgfpathlineto{\pgfqpoint{0.977521in}{0.408164in}}%
\pgfpathlineto{\pgfqpoint{0.978810in}{0.736668in}}%
\pgfpathlineto{\pgfqpoint{0.980099in}{0.419747in}}%
\pgfpathlineto{\pgfqpoint{0.985256in}{0.401628in}}%
\pgfpathlineto{\pgfqpoint{0.986545in}{0.405747in}}%
\pgfpathlineto{\pgfqpoint{0.987835in}{0.405747in}}%
\pgfpathlineto{\pgfqpoint{0.989124in}{0.400804in}}%
\pgfpathlineto{\pgfqpoint{0.992991in}{0.415992in}}%
\pgfpathlineto{\pgfqpoint{0.994281in}{0.400359in}}%
\pgfpathlineto{\pgfqpoint{0.996859in}{0.468084in}}%
\pgfpathlineto{\pgfqpoint{0.998148in}{0.463660in}}%
\pgfpathlineto{\pgfqpoint{1.002016in}{0.543178in}}%
\pgfpathlineto{\pgfqpoint{1.003305in}{0.449217in}}%
\pgfpathlineto{\pgfqpoint{1.004594in}{0.406260in}}%
\pgfpathlineto{\pgfqpoint{1.007173in}{0.401927in}}%
\pgfpathlineto{\pgfqpoint{1.012329in}{0.408415in}}%
\pgfpathlineto{\pgfqpoint{1.013618in}{0.479784in}}%
\pgfpathlineto{\pgfqpoint{1.014908in}{0.456449in}}%
\pgfpathlineto{\pgfqpoint{1.016197in}{0.495768in}}%
\pgfpathlineto{\pgfqpoint{1.020064in}{0.440505in}}%
\pgfpathlineto{\pgfqpoint{1.021354in}{0.401712in}}%
\pgfpathlineto{\pgfqpoint{1.022643in}{0.471992in}}%
\pgfpathlineto{\pgfqpoint{1.023932in}{0.402422in}}%
\pgfpathlineto{\pgfqpoint{1.025221in}{0.400808in}}%
\pgfpathlineto{\pgfqpoint{1.029089in}{0.400810in}}%
\pgfpathlineto{\pgfqpoint{1.030378in}{0.407218in}}%
\pgfpathlineto{\pgfqpoint{1.031667in}{0.454541in}}%
\pgfpathlineto{\pgfqpoint{1.032956in}{0.427018in}}%
\pgfpathlineto{\pgfqpoint{1.034246in}{0.437658in}}%
\pgfpathlineto{\pgfqpoint{1.038113in}{0.429959in}}%
\pgfpathlineto{\pgfqpoint{1.039402in}{0.415995in}}%
\pgfpathlineto{\pgfqpoint{1.040692in}{0.414597in}}%
\pgfpathlineto{\pgfqpoint{1.041981in}{0.495305in}}%
\pgfpathlineto{\pgfqpoint{1.043270in}{0.691664in}}%
\pgfpathlineto{\pgfqpoint{1.047138in}{0.529968in}}%
\pgfpathlineto{\pgfqpoint{1.048427in}{0.434796in}}%
\pgfpathlineto{\pgfqpoint{1.049716in}{0.400964in}}%
\pgfpathlineto{\pgfqpoint{1.051005in}{0.434644in}}%
\pgfpathlineto{\pgfqpoint{1.052294in}{0.443236in}}%
\pgfpathlineto{\pgfqpoint{1.056162in}{0.481590in}}%
\pgfpathlineto{\pgfqpoint{1.057451in}{0.400929in}}%
\pgfpathlineto{\pgfqpoint{1.060030in}{0.404369in}}%
\pgfpathlineto{\pgfqpoint{1.061319in}{0.428210in}}%
\pgfpathlineto{\pgfqpoint{1.065186in}{0.424727in}}%
\pgfpathlineto{\pgfqpoint{1.066476in}{0.417667in}}%
\pgfpathlineto{\pgfqpoint{1.067765in}{0.424523in}}%
\pgfpathlineto{\pgfqpoint{1.069054in}{0.436768in}}%
\pgfpathlineto{\pgfqpoint{1.070343in}{0.419934in}}%
\pgfpathlineto{\pgfqpoint{1.074211in}{0.485915in}}%
\pgfpathlineto{\pgfqpoint{1.075500in}{0.410596in}}%
\pgfpathlineto{\pgfqpoint{1.076789in}{0.401716in}}%
\pgfpathlineto{\pgfqpoint{1.078078in}{0.453436in}}%
\pgfpathlineto{\pgfqpoint{1.079367in}{0.401457in}}%
\pgfpathlineto{\pgfqpoint{1.083235in}{0.415976in}}%
\pgfpathlineto{\pgfqpoint{1.084524in}{0.430905in}}%
\pgfpathlineto{\pgfqpoint{1.085813in}{0.402221in}}%
\pgfpathlineto{\pgfqpoint{1.087103in}{0.592733in}}%
\pgfpathlineto{\pgfqpoint{1.088392in}{0.450428in}}%
\pgfpathlineto{\pgfqpoint{1.092259in}{0.425248in}}%
\pgfpathlineto{\pgfqpoint{1.093549in}{0.427966in}}%
\pgfpathlineto{\pgfqpoint{1.094838in}{0.422037in}}%
\pgfpathlineto{\pgfqpoint{1.096127in}{0.428650in}}%
\pgfpathlineto{\pgfqpoint{1.097416in}{0.425196in}}%
\pgfpathlineto{\pgfqpoint{1.101284in}{0.400288in}}%
\pgfpathlineto{\pgfqpoint{1.102573in}{0.403624in}}%
\pgfpathlineto{\pgfqpoint{1.103862in}{0.405210in}}%
\pgfpathlineto{\pgfqpoint{1.105151in}{0.460631in}}%
\pgfpathlineto{\pgfqpoint{1.106441in}{0.403249in}}%
\pgfpathlineto{\pgfqpoint{1.110308in}{0.405370in}}%
\pgfpathlineto{\pgfqpoint{1.111597in}{0.421749in}}%
\pgfpathlineto{\pgfqpoint{1.112887in}{0.400289in}}%
\pgfpathlineto{\pgfqpoint{1.114176in}{0.403647in}}%
\pgfpathlineto{\pgfqpoint{1.115465in}{0.409412in}}%
\pgfpathlineto{\pgfqpoint{1.119333in}{0.402387in}}%
\pgfpathlineto{\pgfqpoint{1.120622in}{0.404241in}}%
\pgfpathlineto{\pgfqpoint{1.121911in}{0.403255in}}%
\pgfpathlineto{\pgfqpoint{1.123200in}{0.408079in}}%
\pgfpathlineto{\pgfqpoint{1.124489in}{0.402438in}}%
\pgfpathlineto{\pgfqpoint{1.128357in}{0.408904in}}%
\pgfpathlineto{\pgfqpoint{1.129646in}{0.400715in}}%
\pgfpathlineto{\pgfqpoint{1.130935in}{0.403259in}}%
\pgfpathlineto{\pgfqpoint{1.133514in}{0.422006in}}%
\pgfpathlineto{\pgfqpoint{1.138670in}{0.400901in}}%
\pgfpathlineto{\pgfqpoint{1.139960in}{0.408025in}}%
\pgfpathlineto{\pgfqpoint{1.141249in}{0.432532in}}%
\pgfpathlineto{\pgfqpoint{1.142538in}{0.400546in}}%
\pgfpathlineto{\pgfqpoint{1.146406in}{0.409412in}}%
\pgfpathlineto{\pgfqpoint{1.147695in}{0.467915in}}%
\pgfpathlineto{\pgfqpoint{1.148984in}{0.432085in}}%
\pgfpathlineto{\pgfqpoint{1.151562in}{0.404345in}}%
\pgfpathlineto{\pgfqpoint{1.156719in}{0.400529in}}%
\pgfpathlineto{\pgfqpoint{1.158008in}{0.410273in}}%
\pgfpathlineto{\pgfqpoint{1.159298in}{0.400526in}}%
\pgfpathlineto{\pgfqpoint{1.160587in}{0.401556in}}%
\pgfpathlineto{\pgfqpoint{1.164454in}{0.413703in}}%
\pgfpathlineto{\pgfqpoint{1.165744in}{0.406056in}}%
\pgfpathlineto{\pgfqpoint{1.167033in}{0.434486in}}%
\pgfpathlineto{\pgfqpoint{1.169611in}{0.401570in}}%
\pgfpathlineto{\pgfqpoint{1.173479in}{0.446063in}}%
\pgfpathlineto{\pgfqpoint{1.174768in}{0.404241in}}%
\pgfpathlineto{\pgfqpoint{1.176057in}{0.448275in}}%
\pgfpathlineto{\pgfqpoint{1.177346in}{0.410348in}}%
\pgfpathlineto{\pgfqpoint{1.178636in}{0.402759in}}%
\pgfpathlineto{\pgfqpoint{1.182503in}{0.417403in}}%
\pgfpathlineto{\pgfqpoint{1.183792in}{0.429694in}}%
\pgfpathlineto{\pgfqpoint{1.185082in}{0.403723in}}%
\pgfpathlineto{\pgfqpoint{1.186371in}{0.419828in}}%
\pgfpathlineto{\pgfqpoint{1.187660in}{0.400331in}}%
\pgfpathlineto{\pgfqpoint{1.191528in}{0.404553in}}%
\pgfpathlineto{\pgfqpoint{1.192817in}{0.411721in}}%
\pgfpathlineto{\pgfqpoint{1.194106in}{0.414944in}}%
\pgfpathlineto{\pgfqpoint{1.195395in}{0.403478in}}%
\pgfpathlineto{\pgfqpoint{1.196684in}{0.427112in}}%
\pgfpathlineto{\pgfqpoint{1.200552in}{0.407444in}}%
\pgfpathlineto{\pgfqpoint{1.201841in}{0.442921in}}%
\pgfpathlineto{\pgfqpoint{1.203130in}{0.400411in}}%
\pgfpathlineto{\pgfqpoint{1.204419in}{0.409117in}}%
\pgfpathlineto{\pgfqpoint{1.205709in}{0.404651in}}%
\pgfpathlineto{\pgfqpoint{1.212155in}{0.401774in}}%
\pgfpathlineto{\pgfqpoint{1.213444in}{0.587946in}}%
\pgfpathlineto{\pgfqpoint{1.214733in}{0.401963in}}%
\pgfpathlineto{\pgfqpoint{1.218601in}{0.425342in}}%
\pgfpathlineto{\pgfqpoint{1.219890in}{0.440058in}}%
\pgfpathlineto{\pgfqpoint{1.221179in}{0.412094in}}%
\pgfpathlineto{\pgfqpoint{1.222468in}{0.401994in}}%
\pgfpathlineto{\pgfqpoint{1.223757in}{0.405397in}}%
\pgfpathlineto{\pgfqpoint{1.227625in}{0.400961in}}%
\pgfpathlineto{\pgfqpoint{1.230203in}{0.409958in}}%
\pgfpathlineto{\pgfqpoint{1.231493in}{0.400286in}}%
\pgfpathlineto{\pgfqpoint{1.232782in}{0.452976in}}%
\pgfpathlineto{\pgfqpoint{1.236649in}{0.459587in}}%
\pgfpathlineto{\pgfqpoint{1.237939in}{0.403671in}}%
\pgfpathlineto{\pgfqpoint{1.239228in}{0.400918in}}%
\pgfpathlineto{\pgfqpoint{1.241806in}{0.414460in}}%
\pgfpathlineto{\pgfqpoint{1.246963in}{0.401581in}}%
\pgfpathlineto{\pgfqpoint{1.248252in}{0.401856in}}%
\pgfpathlineto{\pgfqpoint{1.249541in}{0.405460in}}%
\pgfpathlineto{\pgfqpoint{1.250831in}{0.415902in}}%
\pgfpathlineto{\pgfqpoint{1.254698in}{0.407597in}}%
\pgfpathlineto{\pgfqpoint{1.255987in}{0.407702in}}%
\pgfpathlineto{\pgfqpoint{1.257276in}{0.401838in}}%
\pgfpathlineto{\pgfqpoint{1.259855in}{0.400323in}}%
\pgfpathlineto{\pgfqpoint{1.265012in}{0.400732in}}%
\pgfpathlineto{\pgfqpoint{1.266301in}{0.407070in}}%
\pgfpathlineto{\pgfqpoint{1.268879in}{0.401110in}}%
\pgfpathlineto{\pgfqpoint{1.272747in}{0.435219in}}%
\pgfpathlineto{\pgfqpoint{1.274036in}{0.416589in}}%
\pgfpathlineto{\pgfqpoint{1.275325in}{0.415690in}}%
\pgfpathlineto{\pgfqpoint{1.276614in}{0.468413in}}%
\pgfpathlineto{\pgfqpoint{1.277904in}{0.418051in}}%
\pgfpathlineto{\pgfqpoint{1.281771in}{0.401277in}}%
\pgfpathlineto{\pgfqpoint{1.284350in}{0.413073in}}%
\pgfpathlineto{\pgfqpoint{1.285639in}{0.401081in}}%
\pgfpathlineto{\pgfqpoint{1.286928in}{0.403954in}}%
\pgfpathlineto{\pgfqpoint{1.293374in}{0.489148in}}%
\pgfpathlineto{\pgfqpoint{1.294663in}{0.400377in}}%
\pgfpathlineto{\pgfqpoint{1.295952in}{0.409374in}}%
\pgfpathlineto{\pgfqpoint{1.299820in}{0.408001in}}%
\pgfpathlineto{\pgfqpoint{1.301109in}{0.412519in}}%
\pgfpathlineto{\pgfqpoint{1.302398in}{0.433473in}}%
\pgfpathlineto{\pgfqpoint{1.303688in}{0.409436in}}%
\pgfpathlineto{\pgfqpoint{1.304977in}{0.402416in}}%
\pgfpathlineto{\pgfqpoint{1.308844in}{0.401594in}}%
\pgfpathlineto{\pgfqpoint{1.310134in}{0.400546in}}%
\pgfpathlineto{\pgfqpoint{1.312712in}{0.400813in}}%
\pgfpathlineto{\pgfqpoint{1.314001in}{0.415583in}}%
\pgfpathlineto{\pgfqpoint{1.320447in}{0.400271in}}%
\pgfpathlineto{\pgfqpoint{1.321736in}{0.401621in}}%
\pgfpathlineto{\pgfqpoint{1.323025in}{0.400815in}}%
\pgfpathlineto{\pgfqpoint{1.326893in}{0.470940in}}%
\pgfpathlineto{\pgfqpoint{1.328182in}{0.400838in}}%
\pgfpathlineto{\pgfqpoint{1.329471in}{0.416225in}}%
\pgfpathlineto{\pgfqpoint{1.330761in}{0.449118in}}%
\pgfpathlineto{\pgfqpoint{1.332050in}{0.402815in}}%
\pgfpathlineto{\pgfqpoint{1.335917in}{0.421337in}}%
\pgfpathlineto{\pgfqpoint{1.337207in}{0.448111in}}%
\pgfpathlineto{\pgfqpoint{1.338496in}{0.409548in}}%
\pgfpathlineto{\pgfqpoint{1.339785in}{0.481219in}}%
\pgfpathlineto{\pgfqpoint{1.341074in}{0.403571in}}%
\pgfpathlineto{\pgfqpoint{1.344942in}{0.413534in}}%
\pgfpathlineto{\pgfqpoint{1.346231in}{0.401669in}}%
\pgfpathlineto{\pgfqpoint{1.347520in}{0.403232in}}%
\pgfpathlineto{\pgfqpoint{1.348809in}{0.408675in}}%
\pgfpathlineto{\pgfqpoint{1.350099in}{0.424170in}}%
\pgfpathlineto{\pgfqpoint{1.355255in}{0.400987in}}%
\pgfpathlineto{\pgfqpoint{1.356545in}{0.428666in}}%
\pgfpathlineto{\pgfqpoint{1.357834in}{0.424331in}}%
\pgfpathlineto{\pgfqpoint{1.359123in}{0.422719in}}%
\pgfpathlineto{\pgfqpoint{1.362991in}{0.473738in}}%
\pgfpathlineto{\pgfqpoint{1.364280in}{0.423174in}}%
\pgfpathlineto{\pgfqpoint{1.365569in}{0.423605in}}%
\pgfpathlineto{\pgfqpoint{1.366858in}{0.408874in}}%
\pgfpathlineto{\pgfqpoint{1.368147in}{0.402147in}}%
\pgfpathlineto{\pgfqpoint{1.372015in}{0.412290in}}%
\pgfpathlineto{\pgfqpoint{1.373304in}{0.403797in}}%
\pgfpathlineto{\pgfqpoint{1.374593in}{0.403386in}}%
\pgfpathlineto{\pgfqpoint{1.377172in}{0.400368in}}%
\pgfpathlineto{\pgfqpoint{1.381039in}{0.401819in}}%
\pgfpathlineto{\pgfqpoint{1.382328in}{0.411336in}}%
\pgfpathlineto{\pgfqpoint{1.383618in}{0.401156in}}%
\pgfpathlineto{\pgfqpoint{1.384907in}{0.409411in}}%
\pgfpathlineto{\pgfqpoint{1.386196in}{0.425441in}}%
\pgfpathlineto{\pgfqpoint{1.390064in}{0.401615in}}%
\pgfpathlineto{\pgfqpoint{1.391353in}{0.425199in}}%
\pgfpathlineto{\pgfqpoint{1.392642in}{0.464929in}}%
\pgfpathlineto{\pgfqpoint{1.393931in}{0.413828in}}%
\pgfpathlineto{\pgfqpoint{1.399088in}{0.484225in}}%
\pgfpathlineto{\pgfqpoint{1.400377in}{0.463505in}}%
\pgfpathlineto{\pgfqpoint{1.401666in}{0.407454in}}%
\pgfpathlineto{\pgfqpoint{1.402956in}{0.407354in}}%
\pgfpathlineto{\pgfqpoint{1.408112in}{0.432084in}}%
\pgfpathlineto{\pgfqpoint{1.409402in}{0.402212in}}%
\pgfpathlineto{\pgfqpoint{1.410691in}{0.453866in}}%
\pgfpathlineto{\pgfqpoint{1.411980in}{0.415706in}}%
\pgfpathlineto{\pgfqpoint{1.413269in}{0.409382in}}%
\pgfpathlineto{\pgfqpoint{1.417137in}{0.407122in}}%
\pgfpathlineto{\pgfqpoint{1.418426in}{0.412715in}}%
\pgfpathlineto{\pgfqpoint{1.419715in}{0.426582in}}%
\pgfpathlineto{\pgfqpoint{1.421004in}{0.402220in}}%
\pgfpathlineto{\pgfqpoint{1.422294in}{0.417941in}}%
\pgfpathlineto{\pgfqpoint{1.426161in}{0.508569in}}%
\pgfpathlineto{\pgfqpoint{1.428740in}{0.432930in}}%
\pgfpathlineto{\pgfqpoint{1.430029in}{0.404108in}}%
\pgfpathlineto{\pgfqpoint{1.431318in}{0.435645in}}%
\pgfpathlineto{\pgfqpoint{1.435185in}{0.401309in}}%
\pgfpathlineto{\pgfqpoint{1.436475in}{0.424936in}}%
\pgfpathlineto{\pgfqpoint{1.437764in}{0.410567in}}%
\pgfpathlineto{\pgfqpoint{1.439053in}{0.414525in}}%
\pgfpathlineto{\pgfqpoint{1.444210in}{0.402501in}}%
\pgfpathlineto{\pgfqpoint{1.445499in}{0.401934in}}%
\pgfpathlineto{\pgfqpoint{1.446788in}{0.431471in}}%
\pgfpathlineto{\pgfqpoint{1.448077in}{0.837763in}}%
\pgfpathlineto{\pgfqpoint{1.449367in}{0.478339in}}%
\pgfpathlineto{\pgfqpoint{1.453234in}{0.402221in}}%
\pgfpathlineto{\pgfqpoint{1.454523in}{0.404087in}}%
\pgfpathlineto{\pgfqpoint{1.455813in}{0.401731in}}%
\pgfpathlineto{\pgfqpoint{1.457102in}{0.402228in}}%
\pgfpathlineto{\pgfqpoint{1.458391in}{0.400411in}}%
\pgfpathlineto{\pgfqpoint{1.462259in}{0.400411in}}%
\pgfpathlineto{\pgfqpoint{1.463548in}{0.409102in}}%
\pgfpathlineto{\pgfqpoint{1.464837in}{0.433720in}}%
\pgfpathlineto{\pgfqpoint{1.466126in}{0.417262in}}%
\pgfpathlineto{\pgfqpoint{1.467415in}{0.498449in}}%
\pgfpathlineto{\pgfqpoint{1.471283in}{0.438107in}}%
\pgfpathlineto{\pgfqpoint{1.472572in}{0.400280in}}%
\pgfpathlineto{\pgfqpoint{1.476440in}{0.425151in}}%
\pgfpathlineto{\pgfqpoint{1.481597in}{0.401115in}}%
\pgfpathlineto{\pgfqpoint{1.484175in}{0.437312in}}%
\pgfpathlineto{\pgfqpoint{1.485464in}{0.437312in}}%
\pgfpathlineto{\pgfqpoint{1.489332in}{0.419142in}}%
\pgfpathlineto{\pgfqpoint{1.490621in}{0.400271in}}%
\pgfpathlineto{\pgfqpoint{1.491910in}{0.439975in}}%
\pgfpathlineto{\pgfqpoint{1.494488in}{0.401707in}}%
\pgfpathlineto{\pgfqpoint{1.499645in}{0.431418in}}%
\pgfpathlineto{\pgfqpoint{1.500934in}{0.401498in}}%
\pgfpathlineto{\pgfqpoint{1.502224in}{0.442611in}}%
\pgfpathlineto{\pgfqpoint{1.503513in}{0.407756in}}%
\pgfpathlineto{\pgfqpoint{1.507380in}{0.408783in}}%
\pgfpathlineto{\pgfqpoint{1.508670in}{0.412702in}}%
\pgfpathlineto{\pgfqpoint{1.509959in}{0.408626in}}%
\pgfpathlineto{\pgfqpoint{1.511248in}{0.477811in}}%
\pgfpathlineto{\pgfqpoint{1.512537in}{0.405203in}}%
\pgfpathlineto{\pgfqpoint{1.516405in}{0.406990in}}%
\pgfpathlineto{\pgfqpoint{1.517694in}{0.421581in}}%
\pgfpathlineto{\pgfqpoint{1.518983in}{0.421072in}}%
\pgfpathlineto{\pgfqpoint{1.520272in}{0.411427in}}%
\pgfpathlineto{\pgfqpoint{1.521562in}{0.409685in}}%
\pgfpathlineto{\pgfqpoint{1.525429in}{0.425676in}}%
\pgfpathlineto{\pgfqpoint{1.526718in}{0.400279in}}%
\pgfpathlineto{\pgfqpoint{1.528008in}{0.413575in}}%
\pgfpathlineto{\pgfqpoint{1.530586in}{0.457104in}}%
\pgfpathlineto{\pgfqpoint{1.534454in}{0.429646in}}%
\pgfpathlineto{\pgfqpoint{1.535743in}{0.403386in}}%
\pgfpathlineto{\pgfqpoint{1.537032in}{0.401401in}}%
\pgfpathlineto{\pgfqpoint{1.538321in}{0.443765in}}%
\pgfpathlineto{\pgfqpoint{1.539610in}{0.402459in}}%
\pgfpathlineto{\pgfqpoint{1.543478in}{0.401027in}}%
\pgfpathlineto{\pgfqpoint{1.544767in}{0.403315in}}%
\pgfpathlineto{\pgfqpoint{1.546056in}{0.400546in}}%
\pgfpathlineto{\pgfqpoint{1.547346in}{0.407546in}}%
\pgfpathlineto{\pgfqpoint{1.548635in}{0.404996in}}%
\pgfpathlineto{\pgfqpoint{1.552502in}{0.407067in}}%
\pgfpathlineto{\pgfqpoint{1.553792in}{0.428569in}}%
\pgfpathlineto{\pgfqpoint{1.555081in}{0.414009in}}%
\pgfpathlineto{\pgfqpoint{1.556370in}{0.647429in}}%
\pgfpathlineto{\pgfqpoint{1.557659in}{0.408214in}}%
\pgfpathlineto{\pgfqpoint{1.561527in}{0.406069in}}%
\pgfpathlineto{\pgfqpoint{1.562816in}{0.401346in}}%
\pgfpathlineto{\pgfqpoint{1.564105in}{1.286399in}}%
\pgfpathlineto{\pgfqpoint{1.565394in}{0.421034in}}%
\pgfpathlineto{\pgfqpoint{1.566683in}{0.502771in}}%
\pgfpathlineto{\pgfqpoint{1.570551in}{0.401636in}}%
\pgfpathlineto{\pgfqpoint{1.571840in}{0.412002in}}%
\pgfpathlineto{\pgfqpoint{1.573129in}{0.410398in}}%
\pgfpathlineto{\pgfqpoint{1.574419in}{0.401911in}}%
\pgfpathlineto{\pgfqpoint{1.575708in}{0.404355in}}%
\pgfpathlineto{\pgfqpoint{1.579575in}{0.400280in}}%
\pgfpathlineto{\pgfqpoint{1.580865in}{0.401497in}}%
\pgfpathlineto{\pgfqpoint{1.582154in}{0.401123in}}%
\pgfpathlineto{\pgfqpoint{1.583443in}{0.490888in}}%
\pgfpathlineto{\pgfqpoint{1.584732in}{0.403511in}}%
\pgfpathlineto{\pgfqpoint{1.588600in}{0.416765in}}%
\pgfpathlineto{\pgfqpoint{1.589889in}{0.415259in}}%
\pgfpathlineto{\pgfqpoint{1.591178in}{0.526192in}}%
\pgfpathlineto{\pgfqpoint{1.592467in}{0.401490in}}%
\pgfpathlineto{\pgfqpoint{1.593757in}{0.400406in}}%
\pgfpathlineto{\pgfqpoint{1.597624in}{0.476442in}}%
\pgfpathlineto{\pgfqpoint{1.598913in}{0.403817in}}%
\pgfpathlineto{\pgfqpoint{1.600203in}{0.405373in}}%
\pgfpathlineto{\pgfqpoint{1.601492in}{0.400412in}}%
\pgfpathlineto{\pgfqpoint{1.602781in}{0.404547in}}%
\pgfpathlineto{\pgfqpoint{1.607938in}{0.431779in}}%
\pgfpathlineto{\pgfqpoint{1.609227in}{0.403399in}}%
\pgfpathlineto{\pgfqpoint{1.610516in}{0.400280in}}%
\pgfpathlineto{\pgfqpoint{1.611805in}{0.401139in}}%
\pgfpathlineto{\pgfqpoint{1.615673in}{0.416854in}}%
\pgfpathlineto{\pgfqpoint{1.616962in}{0.558649in}}%
\pgfpathlineto{\pgfqpoint{1.618251in}{0.412843in}}%
\pgfpathlineto{\pgfqpoint{1.619540in}{0.406862in}}%
\pgfpathlineto{\pgfqpoint{1.620830in}{0.464245in}}%
\pgfpathlineto{\pgfqpoint{1.624697in}{0.400883in}}%
\pgfpathlineto{\pgfqpoint{1.625986in}{0.419640in}}%
\pgfpathlineto{\pgfqpoint{1.627276in}{0.417049in}}%
\pgfpathlineto{\pgfqpoint{1.628565in}{0.404399in}}%
\pgfpathlineto{\pgfqpoint{1.629854in}{0.463262in}}%
\pgfpathlineto{\pgfqpoint{1.633722in}{0.425204in}}%
\pgfpathlineto{\pgfqpoint{1.635011in}{0.432245in}}%
\pgfpathlineto{\pgfqpoint{1.636300in}{0.412537in}}%
\pgfpathlineto{\pgfqpoint{1.637589in}{0.415694in}}%
\pgfpathlineto{\pgfqpoint{1.638878in}{0.401146in}}%
\pgfpathlineto{\pgfqpoint{1.644035in}{0.417838in}}%
\pgfpathlineto{\pgfqpoint{1.645324in}{0.408210in}}%
\pgfpathlineto{\pgfqpoint{1.646614in}{0.440064in}}%
\pgfpathlineto{\pgfqpoint{1.647903in}{0.413536in}}%
\pgfpathlineto{\pgfqpoint{1.651770in}{0.470441in}}%
\pgfpathlineto{\pgfqpoint{1.653060in}{0.460402in}}%
\pgfpathlineto{\pgfqpoint{1.654349in}{0.407517in}}%
\pgfpathlineto{\pgfqpoint{1.655638in}{0.512220in}}%
\pgfpathlineto{\pgfqpoint{1.656927in}{0.439901in}}%
\pgfpathlineto{\pgfqpoint{1.660795in}{0.406383in}}%
\pgfpathlineto{\pgfqpoint{1.662084in}{0.417087in}}%
\pgfpathlineto{\pgfqpoint{1.663373in}{0.469587in}}%
\pgfpathlineto{\pgfqpoint{1.664662in}{0.427047in}}%
\pgfpathlineto{\pgfqpoint{1.665952in}{0.417264in}}%
\pgfpathlineto{\pgfqpoint{1.669819in}{0.400945in}}%
\pgfpathlineto{\pgfqpoint{1.671108in}{0.400278in}}%
\pgfpathlineto{\pgfqpoint{1.672398in}{0.404517in}}%
\pgfpathlineto{\pgfqpoint{1.673687in}{0.458167in}}%
\pgfpathlineto{\pgfqpoint{1.674976in}{0.400330in}}%
\pgfpathlineto{\pgfqpoint{1.678843in}{0.400271in}}%
\pgfpathlineto{\pgfqpoint{1.680133in}{0.405021in}}%
\pgfpathlineto{\pgfqpoint{1.681422in}{0.400797in}}%
\pgfpathlineto{\pgfqpoint{1.682711in}{0.585993in}}%
\pgfpathlineto{\pgfqpoint{1.684000in}{0.423478in}}%
\pgfpathlineto{\pgfqpoint{1.687868in}{0.427682in}}%
\pgfpathlineto{\pgfqpoint{1.689157in}{0.403644in}}%
\pgfpathlineto{\pgfqpoint{1.690446in}{0.410202in}}%
\pgfpathlineto{\pgfqpoint{1.691735in}{0.432283in}}%
\pgfpathlineto{\pgfqpoint{1.693025in}{0.418345in}}%
\pgfpathlineto{\pgfqpoint{1.696892in}{0.403581in}}%
\pgfpathlineto{\pgfqpoint{1.698181in}{0.403890in}}%
\pgfpathlineto{\pgfqpoint{1.699471in}{0.436329in}}%
\pgfpathlineto{\pgfqpoint{1.700760in}{0.400697in}}%
\pgfpathlineto{\pgfqpoint{1.702049in}{0.412432in}}%
\pgfpathlineto{\pgfqpoint{1.705917in}{0.415441in}}%
\pgfpathlineto{\pgfqpoint{1.707206in}{0.423723in}}%
\pgfpathlineto{\pgfqpoint{1.708495in}{0.402236in}}%
\pgfpathlineto{\pgfqpoint{1.709784in}{0.443944in}}%
\pgfpathlineto{\pgfqpoint{1.711073in}{0.401211in}}%
\pgfpathlineto{\pgfqpoint{1.714941in}{0.400376in}}%
\pgfpathlineto{\pgfqpoint{1.716230in}{0.409127in}}%
\pgfpathlineto{\pgfqpoint{1.717519in}{0.401046in}}%
\pgfpathlineto{\pgfqpoint{1.720098in}{0.401354in}}%
\pgfpathlineto{\pgfqpoint{1.723965in}{0.409962in}}%
\pgfpathlineto{\pgfqpoint{1.725255in}{0.439354in}}%
\pgfpathlineto{\pgfqpoint{1.726544in}{0.403709in}}%
\pgfpathlineto{\pgfqpoint{1.727833in}{0.404015in}}%
\pgfpathlineto{\pgfqpoint{1.729122in}{0.400330in}}%
\pgfpathlineto{\pgfqpoint{1.732990in}{0.400507in}}%
\pgfpathlineto{\pgfqpoint{1.734279in}{0.416802in}}%
\pgfpathlineto{\pgfqpoint{1.735568in}{0.536525in}}%
\pgfpathlineto{\pgfqpoint{1.736857in}{0.417452in}}%
\pgfpathlineto{\pgfqpoint{1.738146in}{0.452461in}}%
\pgfpathlineto{\pgfqpoint{1.742014in}{0.400771in}}%
\pgfpathlineto{\pgfqpoint{1.743303in}{0.500126in}}%
\pgfpathlineto{\pgfqpoint{1.744592in}{0.413609in}}%
\pgfpathlineto{\pgfqpoint{1.745882in}{0.401923in}}%
\pgfpathlineto{\pgfqpoint{1.747171in}{0.400294in}}%
\pgfpathlineto{\pgfqpoint{1.751038in}{0.413279in}}%
\pgfpathlineto{\pgfqpoint{1.752328in}{0.401699in}}%
\pgfpathlineto{\pgfqpoint{1.754906in}{0.406653in}}%
\pgfpathlineto{\pgfqpoint{1.756195in}{0.401063in}}%
\pgfpathlineto{\pgfqpoint{1.760063in}{0.404879in}}%
\pgfpathlineto{\pgfqpoint{1.761352in}{0.409356in}}%
\pgfpathlineto{\pgfqpoint{1.763930in}{0.408489in}}%
\pgfpathlineto{\pgfqpoint{1.765220in}{0.400358in}}%
\pgfpathlineto{\pgfqpoint{1.769087in}{0.405526in}}%
\pgfpathlineto{\pgfqpoint{1.770376in}{0.408575in}}%
\pgfpathlineto{\pgfqpoint{1.771666in}{0.401654in}}%
\pgfpathlineto{\pgfqpoint{1.772955in}{0.400536in}}%
\pgfpathlineto{\pgfqpoint{1.774244in}{0.402027in}}%
\pgfpathlineto{\pgfqpoint{1.778112in}{0.415123in}}%
\pgfpathlineto{\pgfqpoint{1.779401in}{0.442510in}}%
\pgfpathlineto{\pgfqpoint{1.780690in}{0.403606in}}%
\pgfpathlineto{\pgfqpoint{1.781979in}{0.411609in}}%
\pgfpathlineto{\pgfqpoint{1.783268in}{0.705072in}}%
\pgfpathlineto{\pgfqpoint{1.788425in}{0.400449in}}%
\pgfpathlineto{\pgfqpoint{1.789714in}{0.403105in}}%
\pgfpathlineto{\pgfqpoint{1.792293in}{0.540127in}}%
\pgfpathlineto{\pgfqpoint{1.796160in}{0.476337in}}%
\pgfpathlineto{\pgfqpoint{1.797449in}{0.467599in}}%
\pgfpathlineto{\pgfqpoint{1.798739in}{0.444164in}}%
\pgfpathlineto{\pgfqpoint{1.800028in}{0.442229in}}%
\pgfpathlineto{\pgfqpoint{1.801317in}{0.490391in}}%
\pgfpathlineto{\pgfqpoint{1.805185in}{0.411978in}}%
\pgfpathlineto{\pgfqpoint{1.806474in}{0.403899in}}%
\pgfpathlineto{\pgfqpoint{1.807763in}{0.403863in}}%
\pgfpathlineto{\pgfqpoint{1.809052in}{0.435403in}}%
\pgfpathlineto{\pgfqpoint{1.810341in}{0.422690in}}%
\pgfpathlineto{\pgfqpoint{1.814209in}{0.404532in}}%
\pgfpathlineto{\pgfqpoint{1.815498in}{0.409385in}}%
\pgfpathlineto{\pgfqpoint{1.816787in}{0.435015in}}%
\pgfpathlineto{\pgfqpoint{1.818077in}{0.400318in}}%
\pgfpathlineto{\pgfqpoint{1.819366in}{0.411221in}}%
\pgfpathlineto{\pgfqpoint{1.824523in}{0.400276in}}%
\pgfpathlineto{\pgfqpoint{1.825812in}{0.413739in}}%
\pgfpathlineto{\pgfqpoint{1.827101in}{0.400795in}}%
\pgfpathlineto{\pgfqpoint{1.828390in}{0.400292in}}%
\pgfpathlineto{\pgfqpoint{1.832258in}{0.422864in}}%
\pgfpathlineto{\pgfqpoint{1.833547in}{0.402740in}}%
\pgfpathlineto{\pgfqpoint{1.834836in}{0.403211in}}%
\pgfpathlineto{\pgfqpoint{1.836125in}{0.400353in}}%
\pgfpathlineto{\pgfqpoint{1.837415in}{0.400599in}}%
\pgfpathlineto{\pgfqpoint{1.841282in}{0.458013in}}%
\pgfpathlineto{\pgfqpoint{1.842571in}{0.447628in}}%
\pgfpathlineto{\pgfqpoint{1.845150in}{0.403298in}}%
\pgfpathlineto{\pgfqpoint{1.846439in}{0.408576in}}%
\pgfpathlineto{\pgfqpoint{1.850307in}{0.400277in}}%
\pgfpathlineto{\pgfqpoint{1.851596in}{0.408052in}}%
\pgfpathlineto{\pgfqpoint{1.852885in}{0.403460in}}%
\pgfpathlineto{\pgfqpoint{1.854174in}{0.482294in}}%
\pgfpathlineto{\pgfqpoint{1.855463in}{0.400271in}}%
\pgfpathlineto{\pgfqpoint{1.859331in}{0.427007in}}%
\pgfpathlineto{\pgfqpoint{1.860620in}{0.419516in}}%
\pgfpathlineto{\pgfqpoint{1.861909in}{0.418887in}}%
\pgfpathlineto{\pgfqpoint{1.864488in}{0.407125in}}%
\pgfpathlineto{\pgfqpoint{1.868355in}{0.420044in}}%
\pgfpathlineto{\pgfqpoint{1.869644in}{0.417316in}}%
\pgfpathlineto{\pgfqpoint{1.870934in}{0.422350in}}%
\pgfpathlineto{\pgfqpoint{1.872223in}{0.400294in}}%
\pgfpathlineto{\pgfqpoint{1.873512in}{0.443612in}}%
\pgfpathlineto{\pgfqpoint{1.877380in}{0.445609in}}%
\pgfpathlineto{\pgfqpoint{1.878669in}{0.404393in}}%
\pgfpathlineto{\pgfqpoint{1.879958in}{0.400277in}}%
\pgfpathlineto{\pgfqpoint{1.881247in}{0.400635in}}%
\pgfpathlineto{\pgfqpoint{1.882536in}{0.573855in}}%
\pgfpathlineto{\pgfqpoint{1.886404in}{0.462294in}}%
\pgfpathlineto{\pgfqpoint{1.887693in}{0.402815in}}%
\pgfpathlineto{\pgfqpoint{1.888982in}{0.485582in}}%
\pgfpathlineto{\pgfqpoint{1.890272in}{0.524798in}}%
\pgfpathlineto{\pgfqpoint{1.891561in}{0.487705in}}%
\pgfpathlineto{\pgfqpoint{1.895428in}{0.469245in}}%
\pgfpathlineto{\pgfqpoint{1.896718in}{0.433391in}}%
\pgfpathlineto{\pgfqpoint{1.898007in}{0.494582in}}%
\pgfpathlineto{\pgfqpoint{1.899296in}{0.406827in}}%
\pgfpathlineto{\pgfqpoint{1.904453in}{0.404995in}}%
\pgfpathlineto{\pgfqpoint{1.905742in}{0.402658in}}%
\pgfpathlineto{\pgfqpoint{1.907031in}{0.404627in}}%
\pgfpathlineto{\pgfqpoint{1.908320in}{0.401284in}}%
\pgfpathlineto{\pgfqpoint{1.909610in}{0.777767in}}%
\pgfpathlineto{\pgfqpoint{1.913477in}{0.422269in}}%
\pgfpathlineto{\pgfqpoint{1.914766in}{0.405670in}}%
\pgfpathlineto{\pgfqpoint{1.916055in}{0.400278in}}%
\pgfpathlineto{\pgfqpoint{1.917345in}{0.442463in}}%
\pgfpathlineto{\pgfqpoint{1.918634in}{0.410223in}}%
\pgfpathlineto{\pgfqpoint{1.922501in}{0.425622in}}%
\pgfpathlineto{\pgfqpoint{1.923791in}{0.411139in}}%
\pgfpathlineto{\pgfqpoint{1.925080in}{0.438431in}}%
\pgfpathlineto{\pgfqpoint{1.927658in}{0.400295in}}%
\pgfpathlineto{\pgfqpoint{1.931526in}{0.402861in}}%
\pgfpathlineto{\pgfqpoint{1.932815in}{0.400744in}}%
\pgfpathlineto{\pgfqpoint{1.934104in}{0.401586in}}%
\pgfpathlineto{\pgfqpoint{1.935393in}{0.419537in}}%
\pgfpathlineto{\pgfqpoint{1.936683in}{0.418870in}}%
\pgfpathlineto{\pgfqpoint{1.940550in}{0.401260in}}%
\pgfpathlineto{\pgfqpoint{1.941839in}{0.419443in}}%
\pgfpathlineto{\pgfqpoint{1.943129in}{0.415638in}}%
\pgfpathlineto{\pgfqpoint{1.944418in}{0.401967in}}%
\pgfpathlineto{\pgfqpoint{1.945707in}{0.424079in}}%
\pgfpathlineto{\pgfqpoint{1.950864in}{0.418705in}}%
\pgfpathlineto{\pgfqpoint{1.952153in}{0.401712in}}%
\pgfpathlineto{\pgfqpoint{1.953442in}{0.402521in}}%
\pgfpathlineto{\pgfqpoint{1.954731in}{0.400361in}}%
\pgfpathlineto{\pgfqpoint{1.958599in}{0.405679in}}%
\pgfpathlineto{\pgfqpoint{1.959888in}{0.415739in}}%
\pgfpathlineto{\pgfqpoint{1.961177in}{0.400323in}}%
\pgfpathlineto{\pgfqpoint{1.962467in}{0.402135in}}%
\pgfpathlineto{\pgfqpoint{1.963756in}{0.402333in}}%
\pgfpathlineto{\pgfqpoint{1.967623in}{0.400841in}}%
\pgfpathlineto{\pgfqpoint{1.968913in}{0.408891in}}%
\pgfpathlineto{\pgfqpoint{1.970202in}{0.408026in}}%
\pgfpathlineto{\pgfqpoint{1.971491in}{0.402798in}}%
\pgfpathlineto{\pgfqpoint{1.972780in}{0.401102in}}%
\pgfpathlineto{\pgfqpoint{1.976648in}{0.403903in}}%
\pgfpathlineto{\pgfqpoint{1.977937in}{0.401413in}}%
\pgfpathlineto{\pgfqpoint{1.979226in}{0.400851in}}%
\pgfpathlineto{\pgfqpoint{1.980515in}{0.403083in}}%
\pgfpathlineto{\pgfqpoint{1.985672in}{0.400277in}}%
\pgfpathlineto{\pgfqpoint{1.986961in}{0.409290in}}%
\pgfpathlineto{\pgfqpoint{1.989540in}{0.400325in}}%
\pgfpathlineto{\pgfqpoint{1.990829in}{0.400649in}}%
\pgfpathlineto{\pgfqpoint{1.994696in}{0.407036in}}%
\pgfpathlineto{\pgfqpoint{1.995986in}{0.439655in}}%
\pgfpathlineto{\pgfqpoint{1.997275in}{0.401370in}}%
\pgfpathlineto{\pgfqpoint{1.998564in}{0.409149in}}%
\pgfpathlineto{\pgfqpoint{2.003721in}{0.400409in}}%
\pgfpathlineto{\pgfqpoint{2.005010in}{0.407048in}}%
\pgfpathlineto{\pgfqpoint{2.006299in}{0.402718in}}%
\pgfpathlineto{\pgfqpoint{2.007588in}{0.401357in}}%
\pgfpathlineto{\pgfqpoint{2.008878in}{0.406650in}}%
\pgfpathlineto{\pgfqpoint{2.012745in}{0.449861in}}%
\pgfpathlineto{\pgfqpoint{2.014034in}{0.401793in}}%
\pgfpathlineto{\pgfqpoint{2.015324in}{0.402576in}}%
\pgfpathlineto{\pgfqpoint{2.016613in}{0.468205in}}%
\pgfpathlineto{\pgfqpoint{2.017902in}{0.416545in}}%
\pgfpathlineto{\pgfqpoint{2.021770in}{0.403100in}}%
\pgfpathlineto{\pgfqpoint{2.023059in}{0.412531in}}%
\pgfpathlineto{\pgfqpoint{2.024348in}{0.400277in}}%
\pgfpathlineto{\pgfqpoint{2.025637in}{0.407060in}}%
\pgfpathlineto{\pgfqpoint{2.026926in}{0.590925in}}%
\pgfpathlineto{\pgfqpoint{2.030794in}{0.401219in}}%
\pgfpathlineto{\pgfqpoint{2.032083in}{0.402100in}}%
\pgfpathlineto{\pgfqpoint{2.033372in}{0.401378in}}%
\pgfpathlineto{\pgfqpoint{2.034661in}{0.429890in}}%
\pgfpathlineto{\pgfqpoint{2.035951in}{0.402356in}}%
\pgfpathlineto{\pgfqpoint{2.039818in}{0.400639in}}%
\pgfpathlineto{\pgfqpoint{2.041107in}{0.408614in}}%
\pgfpathlineto{\pgfqpoint{2.042397in}{0.400324in}}%
\pgfpathlineto{\pgfqpoint{2.043686in}{0.409185in}}%
\pgfpathlineto{\pgfqpoint{2.048843in}{0.400295in}}%
\pgfpathlineto{\pgfqpoint{2.050132in}{0.400481in}}%
\pgfpathlineto{\pgfqpoint{2.051421in}{0.424608in}}%
\pgfpathlineto{\pgfqpoint{2.052710in}{0.404095in}}%
\pgfpathlineto{\pgfqpoint{2.057867in}{0.455575in}}%
\pgfpathlineto{\pgfqpoint{2.059156in}{0.401683in}}%
\pgfpathlineto{\pgfqpoint{2.060445in}{0.404870in}}%
\pgfpathlineto{\pgfqpoint{2.061735in}{0.400931in}}%
\pgfpathlineto{\pgfqpoint{2.063024in}{0.400620in}}%
\pgfpathlineto{\pgfqpoint{2.066891in}{0.400293in}}%
\pgfpathlineto{\pgfqpoint{2.068181in}{0.401841in}}%
\pgfpathlineto{\pgfqpoint{2.069470in}{0.400812in}}%
\pgfpathlineto{\pgfqpoint{2.070759in}{0.420570in}}%
\pgfpathlineto{\pgfqpoint{2.072048in}{0.413654in}}%
\pgfpathlineto{\pgfqpoint{2.077205in}{0.415334in}}%
\pgfpathlineto{\pgfqpoint{2.078494in}{0.400409in}}%
\pgfpathlineto{\pgfqpoint{2.079783in}{0.401685in}}%
\pgfpathlineto{\pgfqpoint{2.081073in}{0.400277in}}%
\pgfpathlineto{\pgfqpoint{2.084940in}{0.407801in}}%
\pgfpathlineto{\pgfqpoint{2.086229in}{0.407006in}}%
\pgfpathlineto{\pgfqpoint{2.087519in}{0.421294in}}%
\pgfpathlineto{\pgfqpoint{2.090097in}{0.402945in}}%
\pgfpathlineto{\pgfqpoint{2.093965in}{0.401356in}}%
\pgfpathlineto{\pgfqpoint{2.095254in}{0.422594in}}%
\pgfpathlineto{\pgfqpoint{2.096543in}{0.409844in}}%
\pgfpathlineto{\pgfqpoint{2.097832in}{0.402674in}}%
\pgfpathlineto{\pgfqpoint{2.099121in}{0.400358in}}%
\pgfpathlineto{\pgfqpoint{2.102989in}{0.418127in}}%
\pgfpathlineto{\pgfqpoint{2.104278in}{0.406739in}}%
\pgfpathlineto{\pgfqpoint{2.105567in}{0.412598in}}%
\pgfpathlineto{\pgfqpoint{2.106856in}{0.443838in}}%
\pgfpathlineto{\pgfqpoint{2.108146in}{0.405058in}}%
\pgfpathlineto{\pgfqpoint{2.112013in}{0.403547in}}%
\pgfpathlineto{\pgfqpoint{2.114592in}{0.436577in}}%
\pgfpathlineto{\pgfqpoint{2.115881in}{0.400554in}}%
\pgfpathlineto{\pgfqpoint{2.117170in}{0.416945in}}%
\pgfpathlineto{\pgfqpoint{2.121038in}{0.400413in}}%
\pgfpathlineto{\pgfqpoint{2.122327in}{0.445973in}}%
\pgfpathlineto{\pgfqpoint{2.124905in}{0.471276in}}%
\pgfpathlineto{\pgfqpoint{2.126194in}{0.425498in}}%
\pgfpathlineto{\pgfqpoint{2.130062in}{0.401825in}}%
\pgfpathlineto{\pgfqpoint{2.131351in}{0.409135in}}%
\pgfpathlineto{\pgfqpoint{2.132640in}{0.421335in}}%
\pgfpathlineto{\pgfqpoint{2.135219in}{0.431779in}}%
\pgfpathlineto{\pgfqpoint{2.139086in}{0.411763in}}%
\pgfpathlineto{\pgfqpoint{2.140376in}{0.497144in}}%
\pgfpathlineto{\pgfqpoint{2.141665in}{0.418625in}}%
\pgfpathlineto{\pgfqpoint{2.142954in}{0.436262in}}%
\pgfpathlineto{\pgfqpoint{2.144243in}{0.402277in}}%
\pgfpathlineto{\pgfqpoint{2.148111in}{0.400294in}}%
\pgfpathlineto{\pgfqpoint{2.149400in}{0.434571in}}%
\pgfpathlineto{\pgfqpoint{2.150689in}{0.413390in}}%
\pgfpathlineto{\pgfqpoint{2.151978in}{1.772999in}}%
\pgfpathlineto{\pgfqpoint{2.153268in}{0.458081in}}%
\pgfpathlineto{\pgfqpoint{2.157135in}{0.400289in}}%
\pgfpathlineto{\pgfqpoint{2.158424in}{0.405279in}}%
\pgfpathlineto{\pgfqpoint{2.159713in}{0.503606in}}%
\pgfpathlineto{\pgfqpoint{2.161003in}{0.400960in}}%
\pgfpathlineto{\pgfqpoint{2.162292in}{0.412897in}}%
\pgfpathlineto{\pgfqpoint{2.166159in}{0.411575in}}%
\pgfpathlineto{\pgfqpoint{2.167449in}{0.400857in}}%
\pgfpathlineto{\pgfqpoint{2.168738in}{0.440359in}}%
\pgfpathlineto{\pgfqpoint{2.170027in}{0.402746in}}%
\pgfpathlineto{\pgfqpoint{2.171316in}{0.410293in}}%
\pgfpathlineto{\pgfqpoint{2.176473in}{0.400287in}}%
\pgfpathlineto{\pgfqpoint{2.177762in}{0.400336in}}%
\pgfpathlineto{\pgfqpoint{2.180341in}{0.414829in}}%
\pgfpathlineto{\pgfqpoint{2.184208in}{0.400658in}}%
\pgfpathlineto{\pgfqpoint{2.185497in}{0.415437in}}%
\pgfpathlineto{\pgfqpoint{2.186787in}{0.400271in}}%
\pgfpathlineto{\pgfqpoint{2.189365in}{0.402095in}}%
\pgfpathlineto{\pgfqpoint{2.193233in}{0.400271in}}%
\pgfpathlineto{\pgfqpoint{2.194522in}{0.407490in}}%
\pgfpathlineto{\pgfqpoint{2.195811in}{0.405586in}}%
\pgfpathlineto{\pgfqpoint{2.197100in}{0.400715in}}%
\pgfpathlineto{\pgfqpoint{2.198389in}{0.410496in}}%
\pgfpathlineto{\pgfqpoint{2.202257in}{0.400401in}}%
\pgfpathlineto{\pgfqpoint{2.203546in}{0.402523in}}%
\pgfpathlineto{\pgfqpoint{2.204835in}{0.412452in}}%
\pgfpathlineto{\pgfqpoint{2.206125in}{0.404473in}}%
\pgfpathlineto{\pgfqpoint{2.207414in}{0.485852in}}%
\pgfpathlineto{\pgfqpoint{2.211281in}{0.400287in}}%
\pgfpathlineto{\pgfqpoint{2.212571in}{0.424283in}}%
\pgfpathlineto{\pgfqpoint{2.213860in}{0.470778in}}%
\pgfpathlineto{\pgfqpoint{2.215149in}{0.454702in}}%
\pgfpathlineto{\pgfqpoint{2.216438in}{0.413307in}}%
\pgfpathlineto{\pgfqpoint{2.220306in}{0.413743in}}%
\pgfpathlineto{\pgfqpoint{2.221595in}{0.402504in}}%
\pgfpathlineto{\pgfqpoint{2.222884in}{0.411727in}}%
\pgfpathlineto{\pgfqpoint{2.225462in}{0.402292in}}%
\pgfpathlineto{\pgfqpoint{2.229330in}{0.403246in}}%
\pgfpathlineto{\pgfqpoint{2.230619in}{0.401843in}}%
\pgfpathlineto{\pgfqpoint{2.231908in}{0.411587in}}%
\pgfpathlineto{\pgfqpoint{2.234487in}{0.416441in}}%
\pgfpathlineto{\pgfqpoint{2.238354in}{0.471699in}}%
\pgfpathlineto{\pgfqpoint{2.239644in}{0.406272in}}%
\pgfpathlineto{\pgfqpoint{2.240933in}{0.425931in}}%
\pgfpathlineto{\pgfqpoint{2.242222in}{0.425862in}}%
\pgfpathlineto{\pgfqpoint{2.243511in}{0.432369in}}%
\pgfpathlineto{\pgfqpoint{2.247379in}{0.400896in}}%
\pgfpathlineto{\pgfqpoint{2.248668in}{0.401604in}}%
\pgfpathlineto{\pgfqpoint{2.249957in}{0.459895in}}%
\pgfpathlineto{\pgfqpoint{2.251246in}{0.413288in}}%
\pgfpathlineto{\pgfqpoint{2.252536in}{0.407780in}}%
\pgfpathlineto{\pgfqpoint{2.257692in}{0.408012in}}%
\pgfpathlineto{\pgfqpoint{2.258982in}{0.400730in}}%
\pgfpathlineto{\pgfqpoint{2.260271in}{0.400456in}}%
\pgfpathlineto{\pgfqpoint{2.261560in}{0.401359in}}%
\pgfpathlineto{\pgfqpoint{2.265428in}{0.406934in}}%
\pgfpathlineto{\pgfqpoint{2.266717in}{0.469025in}}%
\pgfpathlineto{\pgfqpoint{2.268006in}{0.451048in}}%
\pgfpathlineto{\pgfqpoint{2.269295in}{0.406826in}}%
\pgfpathlineto{\pgfqpoint{2.270584in}{0.509360in}}%
\pgfpathlineto{\pgfqpoint{2.274452in}{0.400656in}}%
\pgfpathlineto{\pgfqpoint{2.275741in}{0.443470in}}%
\pgfpathlineto{\pgfqpoint{2.277030in}{0.454408in}}%
\pgfpathlineto{\pgfqpoint{2.278319in}{0.496073in}}%
\pgfpathlineto{\pgfqpoint{2.279609in}{0.438238in}}%
\pgfpathlineto{\pgfqpoint{2.284765in}{0.401996in}}%
\pgfpathlineto{\pgfqpoint{2.287344in}{0.448753in}}%
\pgfpathlineto{\pgfqpoint{2.288633in}{0.403517in}}%
\pgfpathlineto{\pgfqpoint{2.293790in}{0.403312in}}%
\pgfpathlineto{\pgfqpoint{2.295079in}{0.406241in}}%
\pgfpathlineto{\pgfqpoint{2.296368in}{0.400271in}}%
\pgfpathlineto{\pgfqpoint{2.297657in}{0.429939in}}%
\pgfpathlineto{\pgfqpoint{2.301525in}{0.400271in}}%
\pgfpathlineto{\pgfqpoint{2.304103in}{0.400832in}}%
\pgfpathlineto{\pgfqpoint{2.305393in}{0.401341in}}%
\pgfpathlineto{\pgfqpoint{2.306682in}{0.411833in}}%
\pgfpathlineto{\pgfqpoint{2.310549in}{0.493579in}}%
\pgfpathlineto{\pgfqpoint{2.311839in}{0.416961in}}%
\pgfpathlineto{\pgfqpoint{2.313128in}{0.404998in}}%
\pgfpathlineto{\pgfqpoint{2.314417in}{0.400538in}}%
\pgfpathlineto{\pgfqpoint{2.315706in}{0.445231in}}%
\pgfpathlineto{\pgfqpoint{2.319574in}{0.409048in}}%
\pgfpathlineto{\pgfqpoint{2.320863in}{0.478408in}}%
\pgfpathlineto{\pgfqpoint{2.322152in}{0.400697in}}%
\pgfpathlineto{\pgfqpoint{2.323441in}{0.447588in}}%
\pgfpathlineto{\pgfqpoint{2.324731in}{0.442050in}}%
\pgfpathlineto{\pgfqpoint{2.328598in}{0.432824in}}%
\pgfpathlineto{\pgfqpoint{2.329887in}{0.442219in}}%
\pgfpathlineto{\pgfqpoint{2.332466in}{0.400616in}}%
\pgfpathlineto{\pgfqpoint{2.333755in}{0.411786in}}%
\pgfpathlineto{\pgfqpoint{2.338912in}{0.400285in}}%
\pgfpathlineto{\pgfqpoint{2.340201in}{0.462103in}}%
\pgfpathlineto{\pgfqpoint{2.341490in}{0.400876in}}%
\pgfpathlineto{\pgfqpoint{2.342779in}{0.400361in}}%
\pgfpathlineto{\pgfqpoint{2.346647in}{0.400705in}}%
\pgfpathlineto{\pgfqpoint{2.347936in}{0.402339in}}%
\pgfpathlineto{\pgfqpoint{2.349225in}{0.402022in}}%
\pgfpathlineto{\pgfqpoint{2.350514in}{0.400710in}}%
\pgfpathlineto{\pgfqpoint{2.355671in}{0.400565in}}%
\pgfpathlineto{\pgfqpoint{2.356960in}{0.412360in}}%
\pgfpathlineto{\pgfqpoint{2.358250in}{0.416563in}}%
\pgfpathlineto{\pgfqpoint{2.359539in}{0.400397in}}%
\pgfpathlineto{\pgfqpoint{2.360828in}{0.400621in}}%
\pgfpathlineto{\pgfqpoint{2.364696in}{0.417627in}}%
\pgfpathlineto{\pgfqpoint{2.365985in}{0.400501in}}%
\pgfpathlineto{\pgfqpoint{2.367274in}{0.400973in}}%
\pgfpathlineto{\pgfqpoint{2.368563in}{0.400304in}}%
\pgfpathlineto{\pgfqpoint{2.369852in}{0.443519in}}%
\pgfpathlineto{\pgfqpoint{2.373720in}{0.401604in}}%
\pgfpathlineto{\pgfqpoint{2.375009in}{0.413837in}}%
\pgfpathlineto{\pgfqpoint{2.376298in}{0.631152in}}%
\pgfpathlineto{\pgfqpoint{2.377588in}{0.402896in}}%
\pgfpathlineto{\pgfqpoint{2.378877in}{0.402215in}}%
\pgfpathlineto{\pgfqpoint{2.382744in}{0.404684in}}%
\pgfpathlineto{\pgfqpoint{2.384034in}{0.403795in}}%
\pgfpathlineto{\pgfqpoint{2.385323in}{0.410643in}}%
\pgfpathlineto{\pgfqpoint{2.386612in}{0.454431in}}%
\pgfpathlineto{\pgfqpoint{2.387901in}{0.402859in}}%
\pgfpathlineto{\pgfqpoint{2.391769in}{0.401429in}}%
\pgfpathlineto{\pgfqpoint{2.393058in}{0.404891in}}%
\pgfpathlineto{\pgfqpoint{2.394347in}{0.403062in}}%
\pgfpathlineto{\pgfqpoint{2.395636in}{0.427581in}}%
\pgfpathlineto{\pgfqpoint{2.396925in}{0.661576in}}%
\pgfpathlineto{\pgfqpoint{2.400793in}{0.406497in}}%
\pgfpathlineto{\pgfqpoint{2.403371in}{0.400431in}}%
\pgfpathlineto{\pgfqpoint{2.404661in}{0.455138in}}%
\pgfpathlineto{\pgfqpoint{2.405950in}{0.405837in}}%
\pgfpathlineto{\pgfqpoint{2.409817in}{0.405837in}}%
\pgfpathlineto{\pgfqpoint{2.411107in}{0.401072in}}%
\pgfpathlineto{\pgfqpoint{2.412396in}{0.404567in}}%
\pgfpathlineto{\pgfqpoint{2.413685in}{0.405349in}}%
\pgfpathlineto{\pgfqpoint{2.414974in}{0.402104in}}%
\pgfpathlineto{\pgfqpoint{2.420131in}{0.435094in}}%
\pgfpathlineto{\pgfqpoint{2.421420in}{0.427570in}}%
\pgfpathlineto{\pgfqpoint{2.422709in}{0.400386in}}%
\pgfpathlineto{\pgfqpoint{2.423999in}{0.423508in}}%
\pgfpathlineto{\pgfqpoint{2.429155in}{0.400429in}}%
\pgfpathlineto{\pgfqpoint{2.430445in}{0.400533in}}%
\pgfpathlineto{\pgfqpoint{2.431734in}{0.416739in}}%
\pgfpathlineto{\pgfqpoint{2.433023in}{0.401013in}}%
\pgfpathlineto{\pgfqpoint{2.436891in}{0.415177in}}%
\pgfpathlineto{\pgfqpoint{2.438180in}{0.402530in}}%
\pgfpathlineto{\pgfqpoint{2.439469in}{0.486550in}}%
\pgfpathlineto{\pgfqpoint{2.440758in}{0.400275in}}%
\pgfpathlineto{\pgfqpoint{2.442047in}{0.403135in}}%
\pgfpathlineto{\pgfqpoint{2.445915in}{0.417492in}}%
\pgfpathlineto{\pgfqpoint{2.447204in}{0.405998in}}%
\pgfpathlineto{\pgfqpoint{2.448493in}{0.402295in}}%
\pgfpathlineto{\pgfqpoint{2.449783in}{0.417405in}}%
\pgfpathlineto{\pgfqpoint{2.451072in}{0.420355in}}%
\pgfpathlineto{\pgfqpoint{2.454939in}{0.405195in}}%
\pgfpathlineto{\pgfqpoint{2.456228in}{0.403551in}}%
\pgfpathlineto{\pgfqpoint{2.457518in}{0.409643in}}%
\pgfpathlineto{\pgfqpoint{2.458807in}{0.402144in}}%
\pgfpathlineto{\pgfqpoint{2.460096in}{0.400822in}}%
\pgfpathlineto{\pgfqpoint{2.463964in}{0.530252in}}%
\pgfpathlineto{\pgfqpoint{2.465253in}{0.406321in}}%
\pgfpathlineto{\pgfqpoint{2.466542in}{0.415863in}}%
\pgfpathlineto{\pgfqpoint{2.467831in}{0.405062in}}%
\pgfpathlineto{\pgfqpoint{2.472988in}{0.400747in}}%
\pgfpathlineto{\pgfqpoint{2.474277in}{0.404118in}}%
\pgfpathlineto{\pgfqpoint{2.475566in}{0.434928in}}%
\pgfpathlineto{\pgfqpoint{2.476856in}{0.403571in}}%
\pgfpathlineto{\pgfqpoint{2.478145in}{0.459511in}}%
\pgfpathlineto{\pgfqpoint{2.482012in}{0.437248in}}%
\pgfpathlineto{\pgfqpoint{2.483302in}{0.410157in}}%
\pgfpathlineto{\pgfqpoint{2.484591in}{0.400472in}}%
\pgfpathlineto{\pgfqpoint{2.485880in}{0.409006in}}%
\pgfpathlineto{\pgfqpoint{2.487169in}{0.403036in}}%
\pgfpathlineto{\pgfqpoint{2.491037in}{0.493675in}}%
\pgfpathlineto{\pgfqpoint{2.492326in}{0.413052in}}%
\pgfpathlineto{\pgfqpoint{2.493615in}{0.400345in}}%
\pgfpathlineto{\pgfqpoint{2.494904in}{0.401583in}}%
\pgfpathlineto{\pgfqpoint{2.496194in}{0.649354in}}%
\pgfpathlineto{\pgfqpoint{2.500061in}{0.423301in}}%
\pgfpathlineto{\pgfqpoint{2.501350in}{0.420381in}}%
\pgfpathlineto{\pgfqpoint{2.502640in}{0.441190in}}%
\pgfpathlineto{\pgfqpoint{2.503929in}{0.403701in}}%
\pgfpathlineto{\pgfqpoint{2.505218in}{0.426969in}}%
\pgfpathlineto{\pgfqpoint{2.509086in}{0.404795in}}%
\pgfpathlineto{\pgfqpoint{2.510375in}{0.406478in}}%
\pgfpathlineto{\pgfqpoint{2.511664in}{0.402405in}}%
\pgfpathlineto{\pgfqpoint{2.512953in}{0.451774in}}%
\pgfpathlineto{\pgfqpoint{2.514242in}{0.410741in}}%
\pgfpathlineto{\pgfqpoint{2.518110in}{0.401076in}}%
\pgfpathlineto{\pgfqpoint{2.519399in}{0.427864in}}%
\pgfpathlineto{\pgfqpoint{2.520688in}{0.402662in}}%
\pgfpathlineto{\pgfqpoint{2.521977in}{0.404338in}}%
\pgfpathlineto{\pgfqpoint{2.523267in}{0.402459in}}%
\pgfpathlineto{\pgfqpoint{2.529713in}{0.400407in}}%
\pgfpathlineto{\pgfqpoint{2.531002in}{0.405418in}}%
\pgfpathlineto{\pgfqpoint{2.532291in}{0.609147in}}%
\pgfpathlineto{\pgfqpoint{2.536159in}{0.638196in}}%
\pgfpathlineto{\pgfqpoint{2.537448in}{0.461232in}}%
\pgfpathlineto{\pgfqpoint{2.538737in}{0.821735in}}%
\pgfpathlineto{\pgfqpoint{2.540026in}{0.482721in}}%
\pgfpathlineto{\pgfqpoint{2.541315in}{0.400376in}}%
\pgfpathlineto{\pgfqpoint{2.545183in}{0.438120in}}%
\pgfpathlineto{\pgfqpoint{2.546472in}{0.560757in}}%
\pgfpathlineto{\pgfqpoint{2.547761in}{0.413337in}}%
\pgfpathlineto{\pgfqpoint{2.549051in}{0.418319in}}%
\pgfpathlineto{\pgfqpoint{2.550340in}{0.445559in}}%
\pgfpathlineto{\pgfqpoint{2.555497in}{0.455796in}}%
\pgfpathlineto{\pgfqpoint{2.556786in}{0.426104in}}%
\pgfpathlineto{\pgfqpoint{2.558075in}{0.412314in}}%
\pgfpathlineto{\pgfqpoint{2.559364in}{0.408286in}}%
\pgfpathlineto{\pgfqpoint{2.563232in}{0.416744in}}%
\pgfpathlineto{\pgfqpoint{2.564521in}{0.407719in}}%
\pgfpathlineto{\pgfqpoint{2.565810in}{0.406745in}}%
\pgfpathlineto{\pgfqpoint{2.567099in}{0.400299in}}%
\pgfpathlineto{\pgfqpoint{2.568389in}{0.439756in}}%
\pgfpathlineto{\pgfqpoint{2.572256in}{0.428840in}}%
\pgfpathlineto{\pgfqpoint{2.573545in}{0.408599in}}%
\pgfpathlineto{\pgfqpoint{2.574835in}{0.414480in}}%
\pgfpathlineto{\pgfqpoint{2.576124in}{0.414903in}}%
\pgfpathlineto{\pgfqpoint{2.577413in}{0.406243in}}%
\pgfpathlineto{\pgfqpoint{2.581280in}{0.770688in}}%
\pgfpathlineto{\pgfqpoint{2.582570in}{0.456466in}}%
\pgfpathlineto{\pgfqpoint{2.583859in}{0.438381in}}%
\pgfpathlineto{\pgfqpoint{2.585148in}{0.403488in}}%
\pgfpathlineto{\pgfqpoint{2.586437in}{0.413758in}}%
\pgfpathlineto{\pgfqpoint{2.590305in}{0.469579in}}%
\pgfpathlineto{\pgfqpoint{2.591594in}{0.400933in}}%
\pgfpathlineto{\pgfqpoint{2.592883in}{0.436019in}}%
\pgfpathlineto{\pgfqpoint{2.594172in}{0.412598in}}%
\pgfpathlineto{\pgfqpoint{2.595462in}{0.400451in}}%
\pgfpathlineto{\pgfqpoint{2.599329in}{0.426777in}}%
\pgfpathlineto{\pgfqpoint{2.600618in}{0.400274in}}%
\pgfpathlineto{\pgfqpoint{2.603197in}{0.434019in}}%
\pgfpathlineto{\pgfqpoint{2.604486in}{0.411224in}}%
\pgfpathlineto{\pgfqpoint{2.608354in}{0.424388in}}%
\pgfpathlineto{\pgfqpoint{2.609643in}{0.412685in}}%
\pgfpathlineto{\pgfqpoint{2.612221in}{0.423240in}}%
\pgfpathlineto{\pgfqpoint{2.613510in}{0.410592in}}%
\pgfpathlineto{\pgfqpoint{2.617378in}{0.429917in}}%
\pgfpathlineto{\pgfqpoint{2.618667in}{0.410621in}}%
\pgfpathlineto{\pgfqpoint{2.619956in}{0.443254in}}%
\pgfpathlineto{\pgfqpoint{2.621246in}{0.403301in}}%
\pgfpathlineto{\pgfqpoint{2.622535in}{0.420636in}}%
\pgfpathlineto{\pgfqpoint{2.627692in}{0.576913in}}%
\pgfpathlineto{\pgfqpoint{2.628981in}{0.405128in}}%
\pgfpathlineto{\pgfqpoint{2.630270in}{0.419366in}}%
\pgfpathlineto{\pgfqpoint{2.631559in}{0.406405in}}%
\pgfpathlineto{\pgfqpoint{2.635427in}{0.411793in}}%
\pgfpathlineto{\pgfqpoint{2.636716in}{0.457957in}}%
\pgfpathlineto{\pgfqpoint{2.638005in}{0.402310in}}%
\pgfpathlineto{\pgfqpoint{2.639294in}{0.418052in}}%
\pgfpathlineto{\pgfqpoint{2.640583in}{0.401172in}}%
\pgfpathlineto{\pgfqpoint{2.644451in}{0.414950in}}%
\pgfpathlineto{\pgfqpoint{2.645740in}{0.400392in}}%
\pgfpathlineto{\pgfqpoint{2.647029in}{0.460329in}}%
\pgfpathlineto{\pgfqpoint{2.648319in}{0.402401in}}%
\pgfpathlineto{\pgfqpoint{2.649608in}{0.400309in}}%
\pgfpathlineto{\pgfqpoint{2.653475in}{0.400423in}}%
\pgfpathlineto{\pgfqpoint{2.654765in}{0.402566in}}%
\pgfpathlineto{\pgfqpoint{2.656054in}{0.401435in}}%
\pgfpathlineto{\pgfqpoint{2.658632in}{0.402023in}}%
\pgfpathlineto{\pgfqpoint{2.663789in}{0.418988in}}%
\pgfpathlineto{\pgfqpoint{2.665078in}{0.417726in}}%
\pgfpathlineto{\pgfqpoint{2.666367in}{0.413413in}}%
\pgfpathlineto{\pgfqpoint{2.667657in}{0.503136in}}%
\pgfpathlineto{\pgfqpoint{2.671524in}{0.416706in}}%
\pgfpathlineto{\pgfqpoint{2.672813in}{0.405630in}}%
\pgfpathlineto{\pgfqpoint{2.674103in}{0.433040in}}%
\pgfpathlineto{\pgfqpoint{2.675392in}{0.401598in}}%
\pgfpathlineto{\pgfqpoint{2.676681in}{0.497842in}}%
\pgfpathlineto{\pgfqpoint{2.680549in}{0.414169in}}%
\pgfpathlineto{\pgfqpoint{2.681838in}{0.474095in}}%
\pgfpathlineto{\pgfqpoint{2.683127in}{0.435265in}}%
\pgfpathlineto{\pgfqpoint{2.684416in}{0.431828in}}%
\pgfpathlineto{\pgfqpoint{2.685705in}{0.534669in}}%
\pgfpathlineto{\pgfqpoint{2.689573in}{0.418118in}}%
\pgfpathlineto{\pgfqpoint{2.690862in}{0.401293in}}%
\pgfpathlineto{\pgfqpoint{2.692151in}{0.433465in}}%
\pgfpathlineto{\pgfqpoint{2.693441in}{0.402358in}}%
\pgfpathlineto{\pgfqpoint{2.698597in}{0.401950in}}%
\pgfpathlineto{\pgfqpoint{2.701176in}{0.417620in}}%
\pgfpathlineto{\pgfqpoint{2.702465in}{0.415160in}}%
\pgfpathlineto{\pgfqpoint{2.707622in}{0.484471in}}%
\pgfpathlineto{\pgfqpoint{2.708911in}{0.408265in}}%
\pgfpathlineto{\pgfqpoint{2.710200in}{0.425279in}}%
\pgfpathlineto{\pgfqpoint{2.711489in}{0.457234in}}%
\pgfpathlineto{\pgfqpoint{2.712778in}{0.422747in}}%
\pgfpathlineto{\pgfqpoint{2.716646in}{0.429738in}}%
\pgfpathlineto{\pgfqpoint{2.717935in}{0.418386in}}%
\pgfpathlineto{\pgfqpoint{2.719224in}{0.472642in}}%
\pgfpathlineto{\pgfqpoint{2.720514in}{0.413349in}}%
\pgfpathlineto{\pgfqpoint{2.721803in}{0.506378in}}%
\pgfpathlineto{\pgfqpoint{2.726960in}{0.405520in}}%
\pgfpathlineto{\pgfqpoint{2.730827in}{0.436408in}}%
\pgfpathlineto{\pgfqpoint{2.734695in}{0.425573in}}%
\pgfpathlineto{\pgfqpoint{2.735984in}{0.400771in}}%
\pgfpathlineto{\pgfqpoint{2.737273in}{0.426723in}}%
\pgfpathlineto{\pgfqpoint{2.738562in}{0.473878in}}%
\pgfpathlineto{\pgfqpoint{2.739852in}{1.144540in}}%
\pgfpathlineto{\pgfqpoint{2.743719in}{0.400605in}}%
\pgfpathlineto{\pgfqpoint{2.745008in}{0.425496in}}%
\pgfpathlineto{\pgfqpoint{2.746298in}{0.425496in}}%
\pgfpathlineto{\pgfqpoint{2.747587in}{0.413147in}}%
\pgfpathlineto{\pgfqpoint{2.752744in}{0.821738in}}%
\pgfpathlineto{\pgfqpoint{2.754033in}{0.409879in}}%
\pgfpathlineto{\pgfqpoint{2.755322in}{0.516155in}}%
\pgfpathlineto{\pgfqpoint{2.756611in}{0.485363in}}%
\pgfpathlineto{\pgfqpoint{2.757900in}{0.514048in}}%
\pgfpathlineto{\pgfqpoint{2.763057in}{0.413895in}}%
\pgfpathlineto{\pgfqpoint{2.764346in}{0.432299in}}%
\pgfpathlineto{\pgfqpoint{2.766925in}{0.402600in}}%
\pgfpathlineto{\pgfqpoint{2.770792in}{0.465435in}}%
\pgfpathlineto{\pgfqpoint{2.772081in}{0.423014in}}%
\pgfpathlineto{\pgfqpoint{2.773371in}{0.402899in}}%
\pgfpathlineto{\pgfqpoint{2.774660in}{0.455825in}}%
\pgfpathlineto{\pgfqpoint{2.775949in}{0.408250in}}%
\pgfpathlineto{\pgfqpoint{2.779817in}{0.403402in}}%
\pgfpathlineto{\pgfqpoint{2.781106in}{0.535846in}}%
\pgfpathlineto{\pgfqpoint{2.782395in}{0.403243in}}%
\pgfpathlineto{\pgfqpoint{2.783684in}{0.404251in}}%
\pgfpathlineto{\pgfqpoint{2.784973in}{0.400274in}}%
\pgfpathlineto{\pgfqpoint{2.788841in}{0.505218in}}%
\pgfpathlineto{\pgfqpoint{2.791419in}{0.400381in}}%
\pgfpathlineto{\pgfqpoint{2.792709in}{0.406448in}}%
\pgfpathlineto{\pgfqpoint{2.793998in}{0.457581in}}%
\pgfpathlineto{\pgfqpoint{2.797865in}{0.400851in}}%
\pgfpathlineto{\pgfqpoint{2.799155in}{0.405221in}}%
\pgfpathlineto{\pgfqpoint{2.800444in}{0.406162in}}%
\pgfpathlineto{\pgfqpoint{2.801733in}{0.416358in}}%
\pgfpathlineto{\pgfqpoint{2.803022in}{0.411405in}}%
\pgfpathlineto{\pgfqpoint{2.806890in}{0.400274in}}%
\pgfpathlineto{\pgfqpoint{2.808179in}{0.417822in}}%
\pgfpathlineto{\pgfqpoint{2.809468in}{0.424546in}}%
\pgfpathlineto{\pgfqpoint{2.810757in}{0.401608in}}%
\pgfpathlineto{\pgfqpoint{2.815914in}{0.400371in}}%
\pgfpathlineto{\pgfqpoint{2.817203in}{0.443990in}}%
\pgfpathlineto{\pgfqpoint{2.818492in}{0.449354in}}%
\pgfpathlineto{\pgfqpoint{2.819782in}{0.402591in}}%
\pgfpathlineto{\pgfqpoint{2.821071in}{0.430577in}}%
\pgfpathlineto{\pgfqpoint{2.824938in}{0.401597in}}%
\pgfpathlineto{\pgfqpoint{2.827517in}{0.437087in}}%
\pgfpathlineto{\pgfqpoint{2.828806in}{0.409554in}}%
\pgfpathlineto{\pgfqpoint{2.830095in}{0.405112in}}%
\pgfpathlineto{\pgfqpoint{2.833963in}{0.401371in}}%
\pgfpathlineto{\pgfqpoint{2.835252in}{0.412065in}}%
\pgfpathlineto{\pgfqpoint{2.836541in}{0.432810in}}%
\pgfpathlineto{\pgfqpoint{2.837830in}{0.408781in}}%
\pgfpathlineto{\pgfqpoint{2.839120in}{0.401627in}}%
\pgfpathlineto{\pgfqpoint{2.842987in}{0.441937in}}%
\pgfpathlineto{\pgfqpoint{2.844276in}{0.411228in}}%
\pgfpathlineto{\pgfqpoint{2.845566in}{0.403785in}}%
\pgfpathlineto{\pgfqpoint{2.846855in}{0.403100in}}%
\pgfpathlineto{\pgfqpoint{2.848144in}{0.463969in}}%
\pgfpathlineto{\pgfqpoint{2.853301in}{0.403833in}}%
\pgfpathlineto{\pgfqpoint{2.854590in}{0.401456in}}%
\pgfpathlineto{\pgfqpoint{2.855879in}{0.427546in}}%
\pgfpathlineto{\pgfqpoint{2.857168in}{0.404935in}}%
\pgfpathlineto{\pgfqpoint{2.861036in}{0.435637in}}%
\pgfpathlineto{\pgfqpoint{2.862325in}{0.438697in}}%
\pgfpathlineto{\pgfqpoint{2.863614in}{0.400638in}}%
\pgfpathlineto{\pgfqpoint{2.866193in}{0.403534in}}%
\pgfpathlineto{\pgfqpoint{2.870060in}{0.406008in}}%
\pgfpathlineto{\pgfqpoint{2.871350in}{0.422789in}}%
\pgfpathlineto{\pgfqpoint{2.872639in}{0.473280in}}%
\pgfpathlineto{\pgfqpoint{2.873928in}{0.408698in}}%
\pgfpathlineto{\pgfqpoint{2.875217in}{0.427530in}}%
\pgfpathlineto{\pgfqpoint{2.879085in}{0.423043in}}%
\pgfpathlineto{\pgfqpoint{2.880374in}{0.425014in}}%
\pgfpathlineto{\pgfqpoint{2.881663in}{0.410339in}}%
\pgfpathlineto{\pgfqpoint{2.882952in}{0.407659in}}%
\pgfpathlineto{\pgfqpoint{2.884241in}{0.415247in}}%
\pgfpathlineto{\pgfqpoint{2.888109in}{0.405365in}}%
\pgfpathlineto{\pgfqpoint{2.889398in}{0.510873in}}%
\pgfpathlineto{\pgfqpoint{2.890687in}{0.402581in}}%
\pgfpathlineto{\pgfqpoint{2.891977in}{0.400358in}}%
\pgfpathlineto{\pgfqpoint{2.893266in}{0.410397in}}%
\pgfpathlineto{\pgfqpoint{2.898423in}{0.412362in}}%
\pgfpathlineto{\pgfqpoint{2.899712in}{0.402162in}}%
\pgfpathlineto{\pgfqpoint{2.901001in}{0.412275in}}%
\pgfpathlineto{\pgfqpoint{2.902290in}{0.400274in}}%
\pgfpathlineto{\pgfqpoint{2.906158in}{0.408426in}}%
\pgfpathlineto{\pgfqpoint{2.907447in}{0.400355in}}%
\pgfpathlineto{\pgfqpoint{2.908736in}{0.424115in}}%
\pgfpathlineto{\pgfqpoint{2.910025in}{0.403554in}}%
\pgfpathlineto{\pgfqpoint{2.911315in}{0.425792in}}%
\pgfpathlineto{\pgfqpoint{2.915182in}{0.473749in}}%
\pgfpathlineto{\pgfqpoint{2.916471in}{0.400979in}}%
\pgfpathlineto{\pgfqpoint{2.917761in}{0.404007in}}%
\pgfpathlineto{\pgfqpoint{2.919050in}{0.401069in}}%
\pgfpathlineto{\pgfqpoint{2.920339in}{0.444495in}}%
\pgfpathlineto{\pgfqpoint{2.924207in}{0.403199in}}%
\pgfpathlineto{\pgfqpoint{2.925496in}{0.400274in}}%
\pgfpathlineto{\pgfqpoint{2.926785in}{0.415776in}}%
\pgfpathlineto{\pgfqpoint{2.928074in}{0.469684in}}%
\pgfpathlineto{\pgfqpoint{2.929363in}{0.648697in}}%
\pgfpathlineto{\pgfqpoint{2.933231in}{0.476685in}}%
\pgfpathlineto{\pgfqpoint{2.934520in}{0.486952in}}%
\pgfpathlineto{\pgfqpoint{2.935809in}{0.462284in}}%
\pgfpathlineto{\pgfqpoint{2.937098in}{0.567199in}}%
\pgfpathlineto{\pgfqpoint{2.938388in}{0.402729in}}%
\pgfpathlineto{\pgfqpoint{2.943544in}{0.401250in}}%
\pgfpathlineto{\pgfqpoint{2.944834in}{0.410090in}}%
\pgfpathlineto{\pgfqpoint{2.946123in}{0.405563in}}%
\pgfpathlineto{\pgfqpoint{2.947412in}{0.492892in}}%
\pgfpathlineto{\pgfqpoint{2.951280in}{0.400529in}}%
\pgfpathlineto{\pgfqpoint{2.952569in}{0.422274in}}%
\pgfpathlineto{\pgfqpoint{2.953858in}{0.400274in}}%
\pgfpathlineto{\pgfqpoint{2.955147in}{0.411125in}}%
\pgfpathlineto{\pgfqpoint{2.956436in}{0.400902in}}%
\pgfpathlineto{\pgfqpoint{2.960304in}{0.400274in}}%
\pgfpathlineto{\pgfqpoint{2.962882in}{0.409552in}}%
\pgfpathlineto{\pgfqpoint{2.964172in}{0.407568in}}%
\pgfpathlineto{\pgfqpoint{2.965461in}{0.428732in}}%
\pgfpathlineto{\pgfqpoint{2.969328in}{0.421851in}}%
\pgfpathlineto{\pgfqpoint{2.970618in}{0.404984in}}%
\pgfpathlineto{\pgfqpoint{2.971907in}{0.400311in}}%
\pgfpathlineto{\pgfqpoint{2.973196in}{0.410518in}}%
\pgfpathlineto{\pgfqpoint{2.974485in}{0.430881in}}%
\pgfpathlineto{\pgfqpoint{2.978353in}{0.401360in}}%
\pgfpathlineto{\pgfqpoint{2.979642in}{0.400281in}}%
\pgfpathlineto{\pgfqpoint{2.980931in}{0.404589in}}%
\pgfpathlineto{\pgfqpoint{2.982220in}{0.413833in}}%
\pgfpathlineto{\pgfqpoint{2.987377in}{0.401121in}}%
\pgfpathlineto{\pgfqpoint{2.988666in}{0.400556in}}%
\pgfpathlineto{\pgfqpoint{2.989956in}{0.403508in}}%
\pgfpathlineto{\pgfqpoint{2.991245in}{0.404639in}}%
\pgfpathlineto{\pgfqpoint{2.992534in}{0.400507in}}%
\pgfpathlineto{\pgfqpoint{2.996401in}{0.417903in}}%
\pgfpathlineto{\pgfqpoint{2.997691in}{0.400502in}}%
\pgfpathlineto{\pgfqpoint{2.998980in}{0.400937in}}%
\pgfpathlineto{\pgfqpoint{3.000269in}{0.407535in}}%
\pgfpathlineto{\pgfqpoint{3.001558in}{0.401202in}}%
\pgfpathlineto{\pgfqpoint{3.005426in}{0.407758in}}%
\pgfpathlineto{\pgfqpoint{3.006715in}{0.401706in}}%
\pgfpathlineto{\pgfqpoint{3.008004in}{0.406063in}}%
\pgfpathlineto{\pgfqpoint{3.009293in}{0.400292in}}%
\pgfpathlineto{\pgfqpoint{3.010583in}{0.401609in}}%
\pgfpathlineto{\pgfqpoint{3.015739in}{0.402197in}}%
\pgfpathlineto{\pgfqpoint{3.017029in}{0.401819in}}%
\pgfpathlineto{\pgfqpoint{3.018318in}{0.403747in}}%
\pgfpathlineto{\pgfqpoint{3.019607in}{0.410697in}}%
\pgfpathlineto{\pgfqpoint{3.024764in}{0.406278in}}%
\pgfpathlineto{\pgfqpoint{3.026053in}{0.410993in}}%
\pgfpathlineto{\pgfqpoint{3.027342in}{0.401858in}}%
\pgfpathlineto{\pgfqpoint{3.028631in}{0.457294in}}%
\pgfpathlineto{\pgfqpoint{3.032499in}{0.457294in}}%
\pgfpathlineto{\pgfqpoint{3.035077in}{0.403875in}}%
\pgfpathlineto{\pgfqpoint{3.036367in}{0.406122in}}%
\pgfpathlineto{\pgfqpoint{3.037656in}{0.400352in}}%
\pgfpathlineto{\pgfqpoint{3.041523in}{0.404376in}}%
\pgfpathlineto{\pgfqpoint{3.042813in}{0.400291in}}%
\pgfpathlineto{\pgfqpoint{3.044102in}{0.411941in}}%
\pgfpathlineto{\pgfqpoint{3.045391in}{0.400696in}}%
\pgfpathlineto{\pgfqpoint{3.046680in}{0.414604in}}%
\pgfpathlineto{\pgfqpoint{3.050548in}{0.411492in}}%
\pgfpathlineto{\pgfqpoint{3.051837in}{0.402856in}}%
\pgfpathlineto{\pgfqpoint{3.054415in}{0.423279in}}%
\pgfpathlineto{\pgfqpoint{3.055705in}{0.413085in}}%
\pgfpathlineto{\pgfqpoint{3.059572in}{0.400833in}}%
\pgfpathlineto{\pgfqpoint{3.060861in}{0.400642in}}%
\pgfpathlineto{\pgfqpoint{3.062150in}{0.408382in}}%
\pgfpathlineto{\pgfqpoint{3.063440in}{0.400274in}}%
\pgfpathlineto{\pgfqpoint{3.064729in}{0.405077in}}%
\pgfpathlineto{\pgfqpoint{3.068596in}{0.401426in}}%
\pgfpathlineto{\pgfqpoint{3.069886in}{0.425039in}}%
\pgfpathlineto{\pgfqpoint{3.071175in}{0.401254in}}%
\pgfpathlineto{\pgfqpoint{3.072464in}{0.403327in}}%
\pgfpathlineto{\pgfqpoint{3.073753in}{0.407505in}}%
\pgfpathlineto{\pgfqpoint{3.077621in}{0.402268in}}%
\pgfpathlineto{\pgfqpoint{3.078910in}{0.407305in}}%
\pgfpathlineto{\pgfqpoint{3.080199in}{0.432846in}}%
\pgfpathlineto{\pgfqpoint{3.081488in}{0.402386in}}%
\pgfpathlineto{\pgfqpoint{3.082778in}{0.400705in}}%
\pgfpathlineto{\pgfqpoint{3.086645in}{0.414316in}}%
\pgfpathlineto{\pgfqpoint{3.087934in}{0.427908in}}%
\pgfpathlineto{\pgfqpoint{3.089224in}{0.401905in}}%
\pgfpathlineto{\pgfqpoint{3.090513in}{0.400920in}}%
\pgfpathlineto{\pgfqpoint{3.091802in}{0.402148in}}%
\pgfpathlineto{\pgfqpoint{3.095670in}{0.402008in}}%
\pgfpathlineto{\pgfqpoint{3.096959in}{0.417166in}}%
\pgfpathlineto{\pgfqpoint{3.098248in}{0.424049in}}%
\pgfpathlineto{\pgfqpoint{3.099537in}{0.400385in}}%
\pgfpathlineto{\pgfqpoint{3.100826in}{0.402090in}}%
\pgfpathlineto{\pgfqpoint{3.104694in}{0.500237in}}%
\pgfpathlineto{\pgfqpoint{3.105983in}{0.403280in}}%
\pgfpathlineto{\pgfqpoint{3.107272in}{0.402940in}}%
\pgfpathlineto{\pgfqpoint{3.108562in}{0.439872in}}%
\pgfpathlineto{\pgfqpoint{3.109851in}{0.400274in}}%
\pgfpathlineto{\pgfqpoint{3.113718in}{0.675747in}}%
\pgfpathlineto{\pgfqpoint{3.115008in}{0.401062in}}%
\pgfpathlineto{\pgfqpoint{3.116297in}{0.465172in}}%
\pgfpathlineto{\pgfqpoint{3.117586in}{0.421963in}}%
\pgfpathlineto{\pgfqpoint{3.118875in}{0.401587in}}%
\pgfpathlineto{\pgfqpoint{3.122743in}{0.416649in}}%
\pgfpathlineto{\pgfqpoint{3.124032in}{0.468875in}}%
\pgfpathlineto{\pgfqpoint{3.125321in}{0.403154in}}%
\pgfpathlineto{\pgfqpoint{3.127899in}{0.407372in}}%
\pgfpathlineto{\pgfqpoint{3.131767in}{0.425328in}}%
\pgfpathlineto{\pgfqpoint{3.133056in}{0.400389in}}%
\pgfpathlineto{\pgfqpoint{3.134345in}{0.479503in}}%
\pgfpathlineto{\pgfqpoint{3.135635in}{0.489055in}}%
\pgfpathlineto{\pgfqpoint{3.136924in}{0.402475in}}%
\pgfpathlineto{\pgfqpoint{3.140791in}{0.462490in}}%
\pgfpathlineto{\pgfqpoint{3.142081in}{0.401276in}}%
\pgfpathlineto{\pgfqpoint{3.143370in}{0.514459in}}%
\pgfpathlineto{\pgfqpoint{3.144659in}{0.400293in}}%
\pgfpathlineto{\pgfqpoint{3.145948in}{0.400675in}}%
\pgfpathlineto{\pgfqpoint{3.149816in}{0.409833in}}%
\pgfpathlineto{\pgfqpoint{3.151105in}{0.411404in}}%
\pgfpathlineto{\pgfqpoint{3.152394in}{0.400357in}}%
\pgfpathlineto{\pgfqpoint{3.153683in}{0.403355in}}%
\pgfpathlineto{\pgfqpoint{3.154973in}{0.430954in}}%
\pgfpathlineto{\pgfqpoint{3.158840in}{0.401800in}}%
\pgfpathlineto{\pgfqpoint{3.160129in}{0.401925in}}%
\pgfpathlineto{\pgfqpoint{3.162708in}{0.401572in}}%
\pgfpathlineto{\pgfqpoint{3.163997in}{0.401688in}}%
\pgfpathlineto{\pgfqpoint{3.169154in}{0.401249in}}%
\pgfpathlineto{\pgfqpoint{3.170443in}{0.400332in}}%
\pgfpathlineto{\pgfqpoint{3.171732in}{0.400293in}}%
\pgfpathlineto{\pgfqpoint{3.173021in}{0.402476in}}%
\pgfpathlineto{\pgfqpoint{3.178178in}{0.451002in}}%
\pgfpathlineto{\pgfqpoint{3.179467in}{0.409892in}}%
\pgfpathlineto{\pgfqpoint{3.182046in}{0.427438in}}%
\pgfpathlineto{\pgfqpoint{3.185913in}{0.404781in}}%
\pgfpathlineto{\pgfqpoint{3.187202in}{0.404435in}}%
\pgfpathlineto{\pgfqpoint{3.188492in}{0.405456in}}%
\pgfpathlineto{\pgfqpoint{3.189781in}{0.404238in}}%
\pgfpathlineto{\pgfqpoint{3.191070in}{0.401178in}}%
\pgfpathlineto{\pgfqpoint{3.196227in}{0.400455in}}%
\pgfpathlineto{\pgfqpoint{3.197516in}{0.402583in}}%
\pgfpathlineto{\pgfqpoint{3.198805in}{0.400711in}}%
\pgfpathlineto{\pgfqpoint{3.200094in}{0.400542in}}%
\pgfpathlineto{\pgfqpoint{3.203962in}{0.402275in}}%
\pgfpathlineto{\pgfqpoint{3.205251in}{0.424868in}}%
\pgfpathlineto{\pgfqpoint{3.206540in}{0.409608in}}%
\pgfpathlineto{\pgfqpoint{3.207830in}{0.409326in}}%
\pgfpathlineto{\pgfqpoint{3.209119in}{0.406076in}}%
\pgfpathlineto{\pgfqpoint{3.212986in}{0.400376in}}%
\pgfpathlineto{\pgfqpoint{3.214276in}{0.420615in}}%
\pgfpathlineto{\pgfqpoint{3.215565in}{0.401755in}}%
\pgfpathlineto{\pgfqpoint{3.216854in}{0.400704in}}%
\pgfpathlineto{\pgfqpoint{3.218143in}{0.691013in}}%
\pgfpathlineto{\pgfqpoint{3.222011in}{0.401538in}}%
\pgfpathlineto{\pgfqpoint{3.223300in}{0.400322in}}%
\pgfpathlineto{\pgfqpoint{3.224589in}{0.409477in}}%
\pgfpathlineto{\pgfqpoint{3.227168in}{0.402228in}}%
\pgfpathlineto{\pgfqpoint{3.231035in}{0.405934in}}%
\pgfpathlineto{\pgfqpoint{3.232324in}{0.403623in}}%
\pgfpathlineto{\pgfqpoint{3.233614in}{0.414134in}}%
\pgfpathlineto{\pgfqpoint{3.234903in}{0.400601in}}%
\pgfpathlineto{\pgfqpoint{3.236192in}{0.400320in}}%
\pgfpathlineto{\pgfqpoint{3.241349in}{0.404198in}}%
\pgfpathlineto{\pgfqpoint{3.242638in}{0.400550in}}%
\pgfpathlineto{\pgfqpoint{3.243927in}{0.403052in}}%
\pgfpathlineto{\pgfqpoint{3.245216in}{0.401373in}}%
\pgfpathlineto{\pgfqpoint{3.249084in}{0.403651in}}%
\pgfpathlineto{\pgfqpoint{3.250373in}{0.400341in}}%
\pgfpathlineto{\pgfqpoint{3.251662in}{0.420485in}}%
\pgfpathlineto{\pgfqpoint{3.252951in}{0.404104in}}%
\pgfpathlineto{\pgfqpoint{3.254241in}{0.401554in}}%
\pgfpathlineto{\pgfqpoint{3.260687in}{0.400460in}}%
\pgfpathlineto{\pgfqpoint{3.261976in}{0.400695in}}%
\pgfpathlineto{\pgfqpoint{3.263265in}{0.407210in}}%
\pgfpathlineto{\pgfqpoint{3.267133in}{0.402796in}}%
\pgfpathlineto{\pgfqpoint{3.268422in}{0.405861in}}%
\pgfpathlineto{\pgfqpoint{3.271000in}{0.400537in}}%
\pgfpathlineto{\pgfqpoint{3.272289in}{0.400934in}}%
\pgfpathlineto{\pgfqpoint{3.276157in}{0.420311in}}%
\pgfpathlineto{\pgfqpoint{3.277446in}{0.423244in}}%
\pgfpathlineto{\pgfqpoint{3.278735in}{0.404145in}}%
\pgfpathlineto{\pgfqpoint{3.280025in}{0.402336in}}%
\pgfpathlineto{\pgfqpoint{3.281314in}{0.402320in}}%
\pgfpathlineto{\pgfqpoint{3.289049in}{0.400814in}}%
\pgfpathlineto{\pgfqpoint{3.290338in}{0.400641in}}%
\pgfpathlineto{\pgfqpoint{3.295495in}{0.406377in}}%
\pgfpathlineto{\pgfqpoint{3.296784in}{0.401358in}}%
\pgfpathlineto{\pgfqpoint{3.298073in}{0.400339in}}%
\pgfpathlineto{\pgfqpoint{3.299362in}{0.402452in}}%
\pgfpathlineto{\pgfqpoint{3.304519in}{0.400273in}}%
\pgfpathlineto{\pgfqpoint{3.307098in}{0.400884in}}%
\pgfpathlineto{\pgfqpoint{3.312254in}{0.416430in}}%
\pgfpathlineto{\pgfqpoint{3.313544in}{0.400390in}}%
\pgfpathlineto{\pgfqpoint{3.314833in}{0.400279in}}%
\pgfpathlineto{\pgfqpoint{3.316122in}{0.434980in}}%
\pgfpathlineto{\pgfqpoint{3.317411in}{0.400271in}}%
\pgfpathlineto{\pgfqpoint{3.321279in}{0.408750in}}%
\pgfpathlineto{\pgfqpoint{3.322568in}{0.401373in}}%
\pgfpathlineto{\pgfqpoint{3.323857in}{0.401654in}}%
\pgfpathlineto{\pgfqpoint{3.325146in}{0.401129in}}%
\pgfpathlineto{\pgfqpoint{3.326436in}{0.402714in}}%
\pgfpathlineto{\pgfqpoint{3.330303in}{0.400300in}}%
\pgfpathlineto{\pgfqpoint{3.331592in}{0.427858in}}%
\pgfpathlineto{\pgfqpoint{3.332882in}{0.400413in}}%
\pgfpathlineto{\pgfqpoint{3.334171in}{0.400775in}}%
\pgfpathlineto{\pgfqpoint{3.335460in}{0.404999in}}%
\pgfpathlineto{\pgfqpoint{3.339328in}{0.400780in}}%
\pgfpathlineto{\pgfqpoint{3.340617in}{0.400384in}}%
\pgfpathlineto{\pgfqpoint{3.341906in}{0.401369in}}%
\pgfpathlineto{\pgfqpoint{3.343195in}{0.400666in}}%
\pgfpathlineto{\pgfqpoint{3.344484in}{0.406983in}}%
\pgfpathlineto{\pgfqpoint{3.348352in}{0.404413in}}%
\pgfpathlineto{\pgfqpoint{3.349641in}{0.400381in}}%
\pgfpathlineto{\pgfqpoint{3.350930in}{0.425347in}}%
\pgfpathlineto{\pgfqpoint{3.352220in}{0.400316in}}%
\pgfpathlineto{\pgfqpoint{3.353509in}{0.410389in}}%
\pgfpathlineto{\pgfqpoint{3.357376in}{0.411568in}}%
\pgfpathlineto{\pgfqpoint{3.358665in}{0.405403in}}%
\pgfpathlineto{\pgfqpoint{3.359955in}{0.414712in}}%
\pgfpathlineto{\pgfqpoint{3.361244in}{0.401144in}}%
\pgfpathlineto{\pgfqpoint{3.362533in}{0.402532in}}%
\pgfpathlineto{\pgfqpoint{3.367690in}{0.400313in}}%
\pgfpathlineto{\pgfqpoint{3.368979in}{0.404556in}}%
\pgfpathlineto{\pgfqpoint{3.370268in}{0.400690in}}%
\pgfpathlineto{\pgfqpoint{3.371557in}{0.409142in}}%
\pgfpathlineto{\pgfqpoint{3.375425in}{0.402832in}}%
\pgfpathlineto{\pgfqpoint{3.376714in}{0.409308in}}%
\pgfpathlineto{\pgfqpoint{3.378003in}{0.401629in}}%
\pgfpathlineto{\pgfqpoint{3.379293in}{0.400271in}}%
\pgfpathlineto{\pgfqpoint{3.380582in}{0.437063in}}%
\pgfpathlineto{\pgfqpoint{3.384449in}{0.418798in}}%
\pgfpathlineto{\pgfqpoint{3.385739in}{0.440943in}}%
\pgfpathlineto{\pgfqpoint{3.387028in}{0.401216in}}%
\pgfpathlineto{\pgfqpoint{3.388317in}{0.421782in}}%
\pgfpathlineto{\pgfqpoint{3.389606in}{0.400271in}}%
\pgfpathlineto{\pgfqpoint{3.393474in}{0.406465in}}%
\pgfpathlineto{\pgfqpoint{3.396052in}{0.400698in}}%
\pgfpathlineto{\pgfqpoint{3.397341in}{0.405898in}}%
\pgfpathlineto{\pgfqpoint{3.398631in}{0.442626in}}%
\pgfpathlineto{\pgfqpoint{3.402498in}{0.402271in}}%
\pgfpathlineto{\pgfqpoint{3.403787in}{0.401955in}}%
\pgfpathlineto{\pgfqpoint{3.406366in}{0.452081in}}%
\pgfpathlineto{\pgfqpoint{3.407655in}{0.406931in}}%
\pgfpathlineto{\pgfqpoint{3.411523in}{0.403410in}}%
\pgfpathlineto{\pgfqpoint{3.415390in}{0.409552in}}%
\pgfpathlineto{\pgfqpoint{3.416679in}{0.407660in}}%
\pgfpathlineto{\pgfqpoint{3.420547in}{0.422272in}}%
\pgfpathlineto{\pgfqpoint{3.421836in}{0.401845in}}%
\pgfpathlineto{\pgfqpoint{3.423125in}{0.409624in}}%
\pgfpathlineto{\pgfqpoint{3.424414in}{0.401123in}}%
\pgfpathlineto{\pgfqpoint{3.425704in}{0.415304in}}%
\pgfpathlineto{\pgfqpoint{3.429571in}{0.400430in}}%
\pgfpathlineto{\pgfqpoint{3.430860in}{0.408883in}}%
\pgfpathlineto{\pgfqpoint{3.432150in}{0.407022in}}%
\pgfpathlineto{\pgfqpoint{3.433439in}{0.400573in}}%
\pgfpathlineto{\pgfqpoint{3.434728in}{0.433129in}}%
\pgfpathlineto{\pgfqpoint{3.439885in}{0.402298in}}%
\pgfpathlineto{\pgfqpoint{3.441174in}{0.410668in}}%
\pgfpathlineto{\pgfqpoint{3.442463in}{0.423763in}}%
\pgfpathlineto{\pgfqpoint{3.443752in}{0.402801in}}%
\pgfpathlineto{\pgfqpoint{3.447620in}{0.402680in}}%
\pgfpathlineto{\pgfqpoint{3.448909in}{0.424893in}}%
\pgfpathlineto{\pgfqpoint{3.450198in}{0.402716in}}%
\pgfpathlineto{\pgfqpoint{3.451488in}{0.407015in}}%
\pgfpathlineto{\pgfqpoint{3.452777in}{0.401503in}}%
\pgfpathlineto{\pgfqpoint{3.456644in}{0.405323in}}%
\pgfpathlineto{\pgfqpoint{3.457934in}{0.400850in}}%
\pgfpathlineto{\pgfqpoint{3.459223in}{0.401668in}}%
\pgfpathlineto{\pgfqpoint{3.460512in}{0.431800in}}%
\pgfpathlineto{\pgfqpoint{3.461801in}{0.405643in}}%
\pgfpathlineto{\pgfqpoint{3.465669in}{0.444895in}}%
\pgfpathlineto{\pgfqpoint{3.466958in}{0.408265in}}%
\pgfpathlineto{\pgfqpoint{3.468247in}{0.410142in}}%
\pgfpathlineto{\pgfqpoint{3.469536in}{0.418506in}}%
\pgfpathlineto{\pgfqpoint{3.470826in}{0.401906in}}%
\pgfpathlineto{\pgfqpoint{3.474693in}{0.407130in}}%
\pgfpathlineto{\pgfqpoint{3.475982in}{0.410000in}}%
\pgfpathlineto{\pgfqpoint{3.477271in}{0.410953in}}%
\pgfpathlineto{\pgfqpoint{3.478561in}{0.400294in}}%
\pgfpathlineto{\pgfqpoint{3.479850in}{0.400506in}}%
\pgfpathlineto{\pgfqpoint{3.483717in}{0.402824in}}%
\pgfpathlineto{\pgfqpoint{3.485007in}{0.400273in}}%
\pgfpathlineto{\pgfqpoint{3.486296in}{0.400293in}}%
\pgfpathlineto{\pgfqpoint{3.487585in}{0.400880in}}%
\pgfpathlineto{\pgfqpoint{3.488874in}{0.402156in}}%
\pgfpathlineto{\pgfqpoint{3.494031in}{0.410745in}}%
\pgfpathlineto{\pgfqpoint{3.495320in}{0.400628in}}%
\pgfpathlineto{\pgfqpoint{3.496609in}{0.426036in}}%
\pgfpathlineto{\pgfqpoint{3.497899in}{0.400501in}}%
\pgfpathlineto{\pgfqpoint{3.501766in}{0.438430in}}%
\pgfpathlineto{\pgfqpoint{3.503055in}{0.400319in}}%
\pgfpathlineto{\pgfqpoint{3.504345in}{0.402075in}}%
\pgfpathlineto{\pgfqpoint{3.505634in}{0.401096in}}%
\pgfpathlineto{\pgfqpoint{3.506923in}{0.408132in}}%
\pgfpathlineto{\pgfqpoint{3.510791in}{0.403235in}}%
\pgfpathlineto{\pgfqpoint{3.512080in}{0.404626in}}%
\pgfpathlineto{\pgfqpoint{3.514658in}{0.401820in}}%
\pgfpathlineto{\pgfqpoint{3.515947in}{0.401039in}}%
\pgfpathlineto{\pgfqpoint{3.519815in}{0.485383in}}%
\pgfpathlineto{\pgfqpoint{3.521104in}{0.400361in}}%
\pgfpathlineto{\pgfqpoint{3.522393in}{0.412249in}}%
\pgfpathlineto{\pgfqpoint{3.523683in}{0.406761in}}%
\pgfpathlineto{\pgfqpoint{3.524972in}{0.405956in}}%
\pgfpathlineto{\pgfqpoint{3.528839in}{0.400753in}}%
\pgfpathlineto{\pgfqpoint{3.530129in}{0.400571in}}%
\pgfpathlineto{\pgfqpoint{3.531418in}{0.401242in}}%
\pgfpathlineto{\pgfqpoint{3.532707in}{0.411233in}}%
\pgfpathlineto{\pgfqpoint{3.533996in}{0.403407in}}%
\pgfpathlineto{\pgfqpoint{3.537864in}{0.401217in}}%
\pgfpathlineto{\pgfqpoint{3.539153in}{0.401432in}}%
\pgfpathlineto{\pgfqpoint{3.540442in}{0.416235in}}%
\pgfpathlineto{\pgfqpoint{3.541731in}{0.401647in}}%
\pgfpathlineto{\pgfqpoint{3.543020in}{0.403948in}}%
\pgfpathlineto{\pgfqpoint{3.546888in}{0.401813in}}%
\pgfpathlineto{\pgfqpoint{3.548177in}{0.407424in}}%
\pgfpathlineto{\pgfqpoint{3.549466in}{0.401070in}}%
\pgfpathlineto{\pgfqpoint{3.550756in}{0.407883in}}%
\pgfpathlineto{\pgfqpoint{3.552045in}{0.403886in}}%
\pgfpathlineto{\pgfqpoint{3.555912in}{0.400276in}}%
\pgfpathlineto{\pgfqpoint{3.557202in}{0.409887in}}%
\pgfpathlineto{\pgfqpoint{3.558491in}{0.414290in}}%
\pgfpathlineto{\pgfqpoint{3.559780in}{0.401455in}}%
\pgfpathlineto{\pgfqpoint{3.561069in}{0.400371in}}%
\pgfpathlineto{\pgfqpoint{3.564937in}{0.401528in}}%
\pgfpathlineto{\pgfqpoint{3.566226in}{0.400315in}}%
\pgfpathlineto{\pgfqpoint{3.567515in}{0.414414in}}%
\pgfpathlineto{\pgfqpoint{3.568804in}{0.400369in}}%
\pgfpathlineto{\pgfqpoint{3.570094in}{0.402002in}}%
\pgfpathlineto{\pgfqpoint{3.573961in}{0.403866in}}%
\pgfpathlineto{\pgfqpoint{3.575250in}{0.400613in}}%
\pgfpathlineto{\pgfqpoint{3.576540in}{0.401880in}}%
\pgfpathlineto{\pgfqpoint{3.579118in}{0.401069in}}%
\pgfpathlineto{\pgfqpoint{3.582986in}{0.402896in}}%
\pgfpathlineto{\pgfqpoint{3.584275in}{0.404128in}}%
\pgfpathlineto{\pgfqpoint{3.585564in}{0.435349in}}%
\pgfpathlineto{\pgfqpoint{3.586853in}{0.411402in}}%
\pgfpathlineto{\pgfqpoint{3.588142in}{0.417411in}}%
\pgfpathlineto{\pgfqpoint{3.592010in}{0.400478in}}%
\pgfpathlineto{\pgfqpoint{3.593299in}{0.426650in}}%
\pgfpathlineto{\pgfqpoint{3.594588in}{0.404864in}}%
\pgfpathlineto{\pgfqpoint{3.597167in}{0.415458in}}%
\pgfpathlineto{\pgfqpoint{3.601034in}{0.402243in}}%
\pgfpathlineto{\pgfqpoint{3.602323in}{0.411171in}}%
\pgfpathlineto{\pgfqpoint{3.603613in}{0.542122in}}%
\pgfpathlineto{\pgfqpoint{3.606191in}{0.440346in}}%
\pgfpathlineto{\pgfqpoint{3.610059in}{0.532101in}}%
\pgfpathlineto{\pgfqpoint{3.611348in}{0.416426in}}%
\pgfpathlineto{\pgfqpoint{3.612637in}{0.416651in}}%
\pgfpathlineto{\pgfqpoint{3.613926in}{0.432739in}}%
\pgfpathlineto{\pgfqpoint{3.615215in}{0.416932in}}%
\pgfpathlineto{\pgfqpoint{3.619083in}{0.400888in}}%
\pgfpathlineto{\pgfqpoint{3.620372in}{0.413772in}}%
\pgfpathlineto{\pgfqpoint{3.621661in}{0.400530in}}%
\pgfpathlineto{\pgfqpoint{3.622951in}{0.402026in}}%
\pgfpathlineto{\pgfqpoint{3.624240in}{0.409367in}}%
\pgfpathlineto{\pgfqpoint{3.628107in}{0.401043in}}%
\pgfpathlineto{\pgfqpoint{3.629397in}{0.422987in}}%
\pgfpathlineto{\pgfqpoint{3.630686in}{0.400276in}}%
\pgfpathlineto{\pgfqpoint{3.631975in}{0.401256in}}%
\pgfpathlineto{\pgfqpoint{3.638421in}{0.401246in}}%
\pgfpathlineto{\pgfqpoint{3.639710in}{0.412190in}}%
\pgfpathlineto{\pgfqpoint{3.640999in}{0.401504in}}%
\pgfpathlineto{\pgfqpoint{3.642289in}{0.401504in}}%
\pgfpathlineto{\pgfqpoint{3.642289in}{0.401504in}}%
\pgfusepath{stroke}%
\end{pgfscope}%
\begin{pgfscope}%
\pgfsetrectcap%
\pgfsetmiterjoin%
\pgfsetlinewidth{0.803000pt}%
\definecolor{currentstroke}{rgb}{1.000000,1.000000,1.000000}%
\pgfsetstrokecolor{currentstroke}%
\pgfsetdash{}{0pt}%
\pgfpathmoveto{\pgfqpoint{0.683198in}{0.331635in}}%
\pgfpathlineto{\pgfqpoint{0.683198in}{1.841635in}}%
\pgfusepath{stroke}%
\end{pgfscope}%
\begin{pgfscope}%
\pgfsetrectcap%
\pgfsetmiterjoin%
\pgfsetlinewidth{0.803000pt}%
\definecolor{currentstroke}{rgb}{1.000000,1.000000,1.000000}%
\pgfsetstrokecolor{currentstroke}%
\pgfsetdash{}{0pt}%
\pgfpathmoveto{\pgfqpoint{3.783198in}{0.331635in}}%
\pgfpathlineto{\pgfqpoint{3.783198in}{1.841635in}}%
\pgfusepath{stroke}%
\end{pgfscope}%
\begin{pgfscope}%
\pgfsetrectcap%
\pgfsetmiterjoin%
\pgfsetlinewidth{0.803000pt}%
\definecolor{currentstroke}{rgb}{1.000000,1.000000,1.000000}%
\pgfsetstrokecolor{currentstroke}%
\pgfsetdash{}{0pt}%
\pgfpathmoveto{\pgfqpoint{0.683198in}{0.331635in}}%
\pgfpathlineto{\pgfqpoint{3.783198in}{0.331635in}}%
\pgfusepath{stroke}%
\end{pgfscope}%
\begin{pgfscope}%
\pgfsetrectcap%
\pgfsetmiterjoin%
\pgfsetlinewidth{0.803000pt}%
\definecolor{currentstroke}{rgb}{1.000000,1.000000,1.000000}%
\pgfsetstrokecolor{currentstroke}%
\pgfsetdash{}{0pt}%
\pgfpathmoveto{\pgfqpoint{0.683198in}{1.841635in}}%
\pgfpathlineto{\pgfqpoint{3.783198in}{1.841635in}}%
\pgfusepath{stroke}%
\end{pgfscope}%
\end{pgfpicture}%
\makeatother%
\endgroup%

    %% Creator: Matplotlib, PGF backend
%%
%% To include the figure in your LaTeX document, write
%%   \input{<filename>.pgf}
%%
%% Make sure the required packages are loaded in your preamble
%%   \usepackage{pgf}
%%
%% Figures using additional raster images can only be included by \input if
%% they are in the same directory as the main LaTeX file. For loading figures
%% from other directories you can use the `import` package
%%   \usepackage{import}
%% and then include the figures with
%%   \import{<path to file>}{<filename>.pgf}
%%
%% Matplotlib used the following preamble
%%   \usepackage{fontspec}
%%   \setmainfont{DejaVuSerif.ttf}[Path=/opt/tljh/user/lib/python3.6/site-packages/matplotlib/mpl-data/fonts/ttf/]
%%   \setsansfont{DejaVuSans.ttf}[Path=/opt/tljh/user/lib/python3.6/site-packages/matplotlib/mpl-data/fonts/ttf/]
%%   \setmonofont{DejaVuSansMono.ttf}[Path=/opt/tljh/user/lib/python3.6/site-packages/matplotlib/mpl-data/fonts/ttf/]
%%
\begingroup%
\makeatletter%
\begin{pgfpicture}%
\pgfpathrectangle{\pgfpointorigin}{\pgfqpoint{3.834522in}{1.941635in}}%
\pgfusepath{use as bounding box, clip}%
\begin{pgfscope}%
\pgfsetbuttcap%
\pgfsetmiterjoin%
\definecolor{currentfill}{rgb}{1.000000,1.000000,1.000000}%
\pgfsetfillcolor{currentfill}%
\pgfsetlinewidth{0.000000pt}%
\definecolor{currentstroke}{rgb}{1.000000,1.000000,1.000000}%
\pgfsetstrokecolor{currentstroke}%
\pgfsetdash{}{0pt}%
\pgfpathmoveto{\pgfqpoint{0.000000in}{0.000000in}}%
\pgfpathlineto{\pgfqpoint{3.834522in}{0.000000in}}%
\pgfpathlineto{\pgfqpoint{3.834522in}{1.941635in}}%
\pgfpathlineto{\pgfqpoint{0.000000in}{1.941635in}}%
\pgfpathclose%
\pgfusepath{fill}%
\end{pgfscope}%
\begin{pgfscope}%
\pgfsetbuttcap%
\pgfsetmiterjoin%
\definecolor{currentfill}{rgb}{0.917647,0.917647,0.949020}%
\pgfsetfillcolor{currentfill}%
\pgfsetlinewidth{0.000000pt}%
\definecolor{currentstroke}{rgb}{0.000000,0.000000,0.000000}%
\pgfsetstrokecolor{currentstroke}%
\pgfsetstrokeopacity{0.000000}%
\pgfsetdash{}{0pt}%
\pgfpathmoveto{\pgfqpoint{0.594832in}{0.331635in}}%
\pgfpathlineto{\pgfqpoint{3.694832in}{0.331635in}}%
\pgfpathlineto{\pgfqpoint{3.694832in}{1.841635in}}%
\pgfpathlineto{\pgfqpoint{0.594832in}{1.841635in}}%
\pgfpathclose%
\pgfusepath{fill}%
\end{pgfscope}%
\begin{pgfscope}%
\pgfpathrectangle{\pgfqpoint{0.594832in}{0.331635in}}{\pgfqpoint{3.100000in}{1.510000in}}%
\pgfusepath{clip}%
\pgfsetroundcap%
\pgfsetroundjoin%
\pgfsetlinewidth{0.803000pt}%
\definecolor{currentstroke}{rgb}{1.000000,1.000000,1.000000}%
\pgfsetstrokecolor{currentstroke}%
\pgfsetdash{}{0pt}%
\pgfpathmoveto{\pgfqpoint{0.731874in}{0.331635in}}%
\pgfpathlineto{\pgfqpoint{0.731874in}{1.841635in}}%
\pgfusepath{stroke}%
\end{pgfscope}%
\begin{pgfscope}%
\definecolor{textcolor}{rgb}{0.150000,0.150000,0.150000}%
\pgfsetstrokecolor{textcolor}%
\pgfsetfillcolor{textcolor}%
\pgftext[x=0.731874in,y=0.234413in,,top]{\color{textcolor}\rmfamily\fontsize{10.000000}{12.000000}\selectfont 2012}%
\end{pgfscope}%
\begin{pgfscope}%
\pgfpathrectangle{\pgfqpoint{0.594832in}{0.331635in}}{\pgfqpoint{3.100000in}{1.510000in}}%
\pgfusepath{clip}%
\pgfsetroundcap%
\pgfsetroundjoin%
\pgfsetlinewidth{0.803000pt}%
\definecolor{currentstroke}{rgb}{1.000000,1.000000,1.000000}%
\pgfsetstrokecolor{currentstroke}%
\pgfsetdash{}{0pt}%
\pgfpathmoveto{\pgfqpoint{1.203719in}{0.331635in}}%
\pgfpathlineto{\pgfqpoint{1.203719in}{1.841635in}}%
\pgfusepath{stroke}%
\end{pgfscope}%
\begin{pgfscope}%
\definecolor{textcolor}{rgb}{0.150000,0.150000,0.150000}%
\pgfsetstrokecolor{textcolor}%
\pgfsetfillcolor{textcolor}%
\pgftext[x=1.203719in,y=0.234413in,,top]{\color{textcolor}\rmfamily\fontsize{10.000000}{12.000000}\selectfont 2013}%
\end{pgfscope}%
\begin{pgfscope}%
\pgfpathrectangle{\pgfqpoint{0.594832in}{0.331635in}}{\pgfqpoint{3.100000in}{1.510000in}}%
\pgfusepath{clip}%
\pgfsetroundcap%
\pgfsetroundjoin%
\pgfsetlinewidth{0.803000pt}%
\definecolor{currentstroke}{rgb}{1.000000,1.000000,1.000000}%
\pgfsetstrokecolor{currentstroke}%
\pgfsetdash{}{0pt}%
\pgfpathmoveto{\pgfqpoint{1.674276in}{0.331635in}}%
\pgfpathlineto{\pgfqpoint{1.674276in}{1.841635in}}%
\pgfusepath{stroke}%
\end{pgfscope}%
\begin{pgfscope}%
\definecolor{textcolor}{rgb}{0.150000,0.150000,0.150000}%
\pgfsetstrokecolor{textcolor}%
\pgfsetfillcolor{textcolor}%
\pgftext[x=1.674276in,y=0.234413in,,top]{\color{textcolor}\rmfamily\fontsize{10.000000}{12.000000}\selectfont 2014}%
\end{pgfscope}%
\begin{pgfscope}%
\pgfpathrectangle{\pgfqpoint{0.594832in}{0.331635in}}{\pgfqpoint{3.100000in}{1.510000in}}%
\pgfusepath{clip}%
\pgfsetroundcap%
\pgfsetroundjoin%
\pgfsetlinewidth{0.803000pt}%
\definecolor{currentstroke}{rgb}{1.000000,1.000000,1.000000}%
\pgfsetstrokecolor{currentstroke}%
\pgfsetdash{}{0pt}%
\pgfpathmoveto{\pgfqpoint{2.144832in}{0.331635in}}%
\pgfpathlineto{\pgfqpoint{2.144832in}{1.841635in}}%
\pgfusepath{stroke}%
\end{pgfscope}%
\begin{pgfscope}%
\definecolor{textcolor}{rgb}{0.150000,0.150000,0.150000}%
\pgfsetstrokecolor{textcolor}%
\pgfsetfillcolor{textcolor}%
\pgftext[x=2.144832in,y=0.234413in,,top]{\color{textcolor}\rmfamily\fontsize{10.000000}{12.000000}\selectfont 2015}%
\end{pgfscope}%
\begin{pgfscope}%
\pgfpathrectangle{\pgfqpoint{0.594832in}{0.331635in}}{\pgfqpoint{3.100000in}{1.510000in}}%
\pgfusepath{clip}%
\pgfsetroundcap%
\pgfsetroundjoin%
\pgfsetlinewidth{0.803000pt}%
\definecolor{currentstroke}{rgb}{1.000000,1.000000,1.000000}%
\pgfsetstrokecolor{currentstroke}%
\pgfsetdash{}{0pt}%
\pgfpathmoveto{\pgfqpoint{2.615389in}{0.331635in}}%
\pgfpathlineto{\pgfqpoint{2.615389in}{1.841635in}}%
\pgfusepath{stroke}%
\end{pgfscope}%
\begin{pgfscope}%
\definecolor{textcolor}{rgb}{0.150000,0.150000,0.150000}%
\pgfsetstrokecolor{textcolor}%
\pgfsetfillcolor{textcolor}%
\pgftext[x=2.615389in,y=0.234413in,,top]{\color{textcolor}\rmfamily\fontsize{10.000000}{12.000000}\selectfont 2016}%
\end{pgfscope}%
\begin{pgfscope}%
\pgfpathrectangle{\pgfqpoint{0.594832in}{0.331635in}}{\pgfqpoint{3.100000in}{1.510000in}}%
\pgfusepath{clip}%
\pgfsetroundcap%
\pgfsetroundjoin%
\pgfsetlinewidth{0.803000pt}%
\definecolor{currentstroke}{rgb}{1.000000,1.000000,1.000000}%
\pgfsetstrokecolor{currentstroke}%
\pgfsetdash{}{0pt}%
\pgfpathmoveto{\pgfqpoint{3.087234in}{0.331635in}}%
\pgfpathlineto{\pgfqpoint{3.087234in}{1.841635in}}%
\pgfusepath{stroke}%
\end{pgfscope}%
\begin{pgfscope}%
\definecolor{textcolor}{rgb}{0.150000,0.150000,0.150000}%
\pgfsetstrokecolor{textcolor}%
\pgfsetfillcolor{textcolor}%
\pgftext[x=3.087234in,y=0.234413in,,top]{\color{textcolor}\rmfamily\fontsize{10.000000}{12.000000}\selectfont 2017}%
\end{pgfscope}%
\begin{pgfscope}%
\pgfpathrectangle{\pgfqpoint{0.594832in}{0.331635in}}{\pgfqpoint{3.100000in}{1.510000in}}%
\pgfusepath{clip}%
\pgfsetroundcap%
\pgfsetroundjoin%
\pgfsetlinewidth{0.803000pt}%
\definecolor{currentstroke}{rgb}{1.000000,1.000000,1.000000}%
\pgfsetstrokecolor{currentstroke}%
\pgfsetdash{}{0pt}%
\pgfpathmoveto{\pgfqpoint{3.557791in}{0.331635in}}%
\pgfpathlineto{\pgfqpoint{3.557791in}{1.841635in}}%
\pgfusepath{stroke}%
\end{pgfscope}%
\begin{pgfscope}%
\definecolor{textcolor}{rgb}{0.150000,0.150000,0.150000}%
\pgfsetstrokecolor{textcolor}%
\pgfsetfillcolor{textcolor}%
\pgftext[x=3.557791in,y=0.234413in,,top]{\color{textcolor}\rmfamily\fontsize{10.000000}{12.000000}\selectfont 2018}%
\end{pgfscope}%
\begin{pgfscope}%
\pgfpathrectangle{\pgfqpoint{0.594832in}{0.331635in}}{\pgfqpoint{3.100000in}{1.510000in}}%
\pgfusepath{clip}%
\pgfsetroundcap%
\pgfsetroundjoin%
\pgfsetlinewidth{0.803000pt}%
\definecolor{currentstroke}{rgb}{1.000000,1.000000,1.000000}%
\pgfsetstrokecolor{currentstroke}%
\pgfsetdash{}{0pt}%
\pgfpathmoveto{\pgfqpoint{0.594832in}{0.400271in}}%
\pgfpathlineto{\pgfqpoint{3.694832in}{0.400271in}}%
\pgfusepath{stroke}%
\end{pgfscope}%
\begin{pgfscope}%
\definecolor{textcolor}{rgb}{0.150000,0.150000,0.150000}%
\pgfsetstrokecolor{textcolor}%
\pgfsetfillcolor{textcolor}%
\pgftext[x=0.100000in,y=0.347510in,left,base]{\color{textcolor}\rmfamily\fontsize{10.000000}{12.000000}\selectfont 0.000}%
\end{pgfscope}%
\begin{pgfscope}%
\pgfpathrectangle{\pgfqpoint{0.594832in}{0.331635in}}{\pgfqpoint{3.100000in}{1.510000in}}%
\pgfusepath{clip}%
\pgfsetroundcap%
\pgfsetroundjoin%
\pgfsetlinewidth{0.803000pt}%
\definecolor{currentstroke}{rgb}{1.000000,1.000000,1.000000}%
\pgfsetstrokecolor{currentstroke}%
\pgfsetdash{}{0pt}%
\pgfpathmoveto{\pgfqpoint{0.594832in}{0.702715in}}%
\pgfpathlineto{\pgfqpoint{3.694832in}{0.702715in}}%
\pgfusepath{stroke}%
\end{pgfscope}%
\begin{pgfscope}%
\definecolor{textcolor}{rgb}{0.150000,0.150000,0.150000}%
\pgfsetstrokecolor{textcolor}%
\pgfsetfillcolor{textcolor}%
\pgftext[x=0.100000in,y=0.649954in,left,base]{\color{textcolor}\rmfamily\fontsize{10.000000}{12.000000}\selectfont 0.002}%
\end{pgfscope}%
\begin{pgfscope}%
\pgfpathrectangle{\pgfqpoint{0.594832in}{0.331635in}}{\pgfqpoint{3.100000in}{1.510000in}}%
\pgfusepath{clip}%
\pgfsetroundcap%
\pgfsetroundjoin%
\pgfsetlinewidth{0.803000pt}%
\definecolor{currentstroke}{rgb}{1.000000,1.000000,1.000000}%
\pgfsetstrokecolor{currentstroke}%
\pgfsetdash{}{0pt}%
\pgfpathmoveto{\pgfqpoint{0.594832in}{1.005159in}}%
\pgfpathlineto{\pgfqpoint{3.694832in}{1.005159in}}%
\pgfusepath{stroke}%
\end{pgfscope}%
\begin{pgfscope}%
\definecolor{textcolor}{rgb}{0.150000,0.150000,0.150000}%
\pgfsetstrokecolor{textcolor}%
\pgfsetfillcolor{textcolor}%
\pgftext[x=0.100000in,y=0.952398in,left,base]{\color{textcolor}\rmfamily\fontsize{10.000000}{12.000000}\selectfont 0.004}%
\end{pgfscope}%
\begin{pgfscope}%
\pgfpathrectangle{\pgfqpoint{0.594832in}{0.331635in}}{\pgfqpoint{3.100000in}{1.510000in}}%
\pgfusepath{clip}%
\pgfsetroundcap%
\pgfsetroundjoin%
\pgfsetlinewidth{0.803000pt}%
\definecolor{currentstroke}{rgb}{1.000000,1.000000,1.000000}%
\pgfsetstrokecolor{currentstroke}%
\pgfsetdash{}{0pt}%
\pgfpathmoveto{\pgfqpoint{0.594832in}{1.307603in}}%
\pgfpathlineto{\pgfqpoint{3.694832in}{1.307603in}}%
\pgfusepath{stroke}%
\end{pgfscope}%
\begin{pgfscope}%
\definecolor{textcolor}{rgb}{0.150000,0.150000,0.150000}%
\pgfsetstrokecolor{textcolor}%
\pgfsetfillcolor{textcolor}%
\pgftext[x=0.100000in,y=1.254841in,left,base]{\color{textcolor}\rmfamily\fontsize{10.000000}{12.000000}\selectfont 0.006}%
\end{pgfscope}%
\begin{pgfscope}%
\pgfpathrectangle{\pgfqpoint{0.594832in}{0.331635in}}{\pgfqpoint{3.100000in}{1.510000in}}%
\pgfusepath{clip}%
\pgfsetroundcap%
\pgfsetroundjoin%
\pgfsetlinewidth{0.803000pt}%
\definecolor{currentstroke}{rgb}{1.000000,1.000000,1.000000}%
\pgfsetstrokecolor{currentstroke}%
\pgfsetdash{}{0pt}%
\pgfpathmoveto{\pgfqpoint{0.594832in}{1.610047in}}%
\pgfpathlineto{\pgfqpoint{3.694832in}{1.610047in}}%
\pgfusepath{stroke}%
\end{pgfscope}%
\begin{pgfscope}%
\definecolor{textcolor}{rgb}{0.150000,0.150000,0.150000}%
\pgfsetstrokecolor{textcolor}%
\pgfsetfillcolor{textcolor}%
\pgftext[x=0.100000in,y=1.557285in,left,base]{\color{textcolor}\rmfamily\fontsize{10.000000}{12.000000}\selectfont 0.008}%
\end{pgfscope}%
\begin{pgfscope}%
\pgfpathrectangle{\pgfqpoint{0.594832in}{0.331635in}}{\pgfqpoint{3.100000in}{1.510000in}}%
\pgfusepath{clip}%
\pgfsetroundcap%
\pgfsetroundjoin%
\pgfsetlinewidth{1.505625pt}%
\definecolor{currentstroke}{rgb}{0.839216,0.152941,0.156863}%
\pgfsetstrokecolor{currentstroke}%
\pgfsetdash{}{0pt}%
\pgfpathmoveto{\pgfqpoint{0.735741in}{0.480358in}}%
\pgfpathlineto{\pgfqpoint{0.737031in}{0.420483in}}%
\pgfpathlineto{\pgfqpoint{0.738320in}{0.405743in}}%
\pgfpathlineto{\pgfqpoint{0.742187in}{0.412545in}}%
\pgfpathlineto{\pgfqpoint{0.743477in}{0.403299in}}%
\pgfpathlineto{\pgfqpoint{0.744766in}{0.410934in}}%
\pgfpathlineto{\pgfqpoint{0.746055in}{0.400858in}}%
\pgfpathlineto{\pgfqpoint{0.747344in}{0.486950in}}%
\pgfpathlineto{\pgfqpoint{0.752501in}{0.402748in}}%
\pgfpathlineto{\pgfqpoint{0.753790in}{0.428208in}}%
\pgfpathlineto{\pgfqpoint{0.755079in}{0.413791in}}%
\pgfpathlineto{\pgfqpoint{0.756369in}{0.525732in}}%
\pgfpathlineto{\pgfqpoint{0.760236in}{0.423652in}}%
\pgfpathlineto{\pgfqpoint{0.761525in}{0.407903in}}%
\pgfpathlineto{\pgfqpoint{0.762815in}{0.400271in}}%
\pgfpathlineto{\pgfqpoint{0.764104in}{0.405149in}}%
\pgfpathlineto{\pgfqpoint{0.765393in}{0.400305in}}%
\pgfpathlineto{\pgfqpoint{0.769261in}{0.400305in}}%
\pgfpathlineto{\pgfqpoint{0.770550in}{0.423585in}}%
\pgfpathlineto{\pgfqpoint{0.771839in}{0.404474in}}%
\pgfpathlineto{\pgfqpoint{0.773128in}{0.401137in}}%
\pgfpathlineto{\pgfqpoint{0.774417in}{0.444503in}}%
\pgfpathlineto{\pgfqpoint{0.778285in}{0.400305in}}%
\pgfpathlineto{\pgfqpoint{0.779574in}{0.401922in}}%
\pgfpathlineto{\pgfqpoint{0.780863in}{0.409963in}}%
\pgfpathlineto{\pgfqpoint{0.782152in}{0.400305in}}%
\pgfpathlineto{\pgfqpoint{0.783442in}{0.405923in}}%
\pgfpathlineto{\pgfqpoint{0.787309in}{0.400271in}}%
\pgfpathlineto{\pgfqpoint{0.788598in}{0.401915in}}%
\pgfpathlineto{\pgfqpoint{0.791177in}{0.413718in}}%
\pgfpathlineto{\pgfqpoint{0.792466in}{0.457876in}}%
\pgfpathlineto{\pgfqpoint{0.797623in}{0.408531in}}%
\pgfpathlineto{\pgfqpoint{0.798912in}{0.438445in}}%
\pgfpathlineto{\pgfqpoint{0.800201in}{0.401483in}}%
\pgfpathlineto{\pgfqpoint{0.801490in}{0.400575in}}%
\pgfpathlineto{\pgfqpoint{0.805358in}{0.407789in}}%
\pgfpathlineto{\pgfqpoint{0.806647in}{0.425944in}}%
\pgfpathlineto{\pgfqpoint{0.807936in}{0.425944in}}%
\pgfpathlineto{\pgfqpoint{0.809226in}{0.400404in}}%
\pgfpathlineto{\pgfqpoint{0.810515in}{0.401100in}}%
\pgfpathlineto{\pgfqpoint{0.814382in}{0.430469in}}%
\pgfpathlineto{\pgfqpoint{0.815672in}{0.401120in}}%
\pgfpathlineto{\pgfqpoint{0.816961in}{0.419561in}}%
\pgfpathlineto{\pgfqpoint{0.818250in}{0.401467in}}%
\pgfpathlineto{\pgfqpoint{0.819539in}{0.412186in}}%
\pgfpathlineto{\pgfqpoint{0.823407in}{0.401880in}}%
\pgfpathlineto{\pgfqpoint{0.824696in}{0.451990in}}%
\pgfpathlineto{\pgfqpoint{0.825985in}{0.400398in}}%
\pgfpathlineto{\pgfqpoint{0.827274in}{0.416913in}}%
\pgfpathlineto{\pgfqpoint{0.828564in}{0.400396in}}%
\pgfpathlineto{\pgfqpoint{0.835010in}{0.400396in}}%
\pgfpathlineto{\pgfqpoint{0.836299in}{0.403365in}}%
\pgfpathlineto{\pgfqpoint{0.837588in}{0.400395in}}%
\pgfpathlineto{\pgfqpoint{0.841456in}{0.419337in}}%
\pgfpathlineto{\pgfqpoint{0.842745in}{0.400271in}}%
\pgfpathlineto{\pgfqpoint{0.844034in}{0.429667in}}%
\pgfpathlineto{\pgfqpoint{0.845323in}{0.424285in}}%
\pgfpathlineto{\pgfqpoint{0.846612in}{0.400544in}}%
\pgfpathlineto{\pgfqpoint{0.850480in}{0.413521in}}%
\pgfpathlineto{\pgfqpoint{0.851769in}{0.414820in}}%
\pgfpathlineto{\pgfqpoint{0.853058in}{0.406258in}}%
\pgfpathlineto{\pgfqpoint{0.854347in}{0.403972in}}%
\pgfpathlineto{\pgfqpoint{0.859504in}{0.419510in}}%
\pgfpathlineto{\pgfqpoint{0.860793in}{0.418400in}}%
\pgfpathlineto{\pgfqpoint{0.862083in}{0.432383in}}%
\pgfpathlineto{\pgfqpoint{0.863372in}{0.475826in}}%
\pgfpathlineto{\pgfqpoint{0.864661in}{0.429068in}}%
\pgfpathlineto{\pgfqpoint{0.868529in}{0.419049in}}%
\pgfpathlineto{\pgfqpoint{0.869818in}{0.401013in}}%
\pgfpathlineto{\pgfqpoint{0.871107in}{0.453438in}}%
\pgfpathlineto{\pgfqpoint{0.872396in}{0.412668in}}%
\pgfpathlineto{\pgfqpoint{0.873685in}{0.401809in}}%
\pgfpathlineto{\pgfqpoint{0.877553in}{0.404828in}}%
\pgfpathlineto{\pgfqpoint{0.878842in}{0.404141in}}%
\pgfpathlineto{\pgfqpoint{0.880131in}{0.458548in}}%
\pgfpathlineto{\pgfqpoint{0.882710in}{0.405331in}}%
\pgfpathlineto{\pgfqpoint{0.886577in}{0.400301in}}%
\pgfpathlineto{\pgfqpoint{0.887867in}{0.456742in}}%
\pgfpathlineto{\pgfqpoint{0.889156in}{0.409469in}}%
\pgfpathlineto{\pgfqpoint{0.890445in}{0.431389in}}%
\pgfpathlineto{\pgfqpoint{0.891734in}{0.480478in}}%
\pgfpathlineto{\pgfqpoint{0.895602in}{0.404666in}}%
\pgfpathlineto{\pgfqpoint{0.896891in}{0.430176in}}%
\pgfpathlineto{\pgfqpoint{0.898180in}{0.406497in}}%
\pgfpathlineto{\pgfqpoint{0.899469in}{0.400782in}}%
\pgfpathlineto{\pgfqpoint{0.900759in}{0.430449in}}%
\pgfpathlineto{\pgfqpoint{0.904626in}{0.476294in}}%
\pgfpathlineto{\pgfqpoint{0.905915in}{0.404210in}}%
\pgfpathlineto{\pgfqpoint{0.907204in}{0.430130in}}%
\pgfpathlineto{\pgfqpoint{0.908494in}{0.421551in}}%
\pgfpathlineto{\pgfqpoint{0.909783in}{0.403733in}}%
\pgfpathlineto{\pgfqpoint{0.913650in}{0.401970in}}%
\pgfpathlineto{\pgfqpoint{0.914940in}{0.403743in}}%
\pgfpathlineto{\pgfqpoint{0.916229in}{0.479111in}}%
\pgfpathlineto{\pgfqpoint{0.917518in}{0.410737in}}%
\pgfpathlineto{\pgfqpoint{0.918807in}{0.402025in}}%
\pgfpathlineto{\pgfqpoint{0.923964in}{0.427861in}}%
\pgfpathlineto{\pgfqpoint{0.925253in}{0.400583in}}%
\pgfpathlineto{\pgfqpoint{0.926542in}{0.418789in}}%
\pgfpathlineto{\pgfqpoint{0.927832in}{0.514352in}}%
\pgfpathlineto{\pgfqpoint{0.931699in}{0.402674in}}%
\pgfpathlineto{\pgfqpoint{0.932988in}{0.435941in}}%
\pgfpathlineto{\pgfqpoint{0.934278in}{0.493013in}}%
\pgfpathlineto{\pgfqpoint{0.935567in}{0.403767in}}%
\pgfpathlineto{\pgfqpoint{0.936856in}{0.447510in}}%
\pgfpathlineto{\pgfqpoint{0.940724in}{0.437776in}}%
\pgfpathlineto{\pgfqpoint{0.942013in}{0.460764in}}%
\pgfpathlineto{\pgfqpoint{0.943302in}{0.400406in}}%
\pgfpathlineto{\pgfqpoint{0.944591in}{0.440705in}}%
\pgfpathlineto{\pgfqpoint{0.945880in}{0.427218in}}%
\pgfpathlineto{\pgfqpoint{0.949748in}{0.401406in}}%
\pgfpathlineto{\pgfqpoint{0.951037in}{0.401807in}}%
\pgfpathlineto{\pgfqpoint{0.952326in}{0.404033in}}%
\pgfpathlineto{\pgfqpoint{0.953616in}{0.580367in}}%
\pgfpathlineto{\pgfqpoint{0.954905in}{0.412131in}}%
\pgfpathlineto{\pgfqpoint{0.958772in}{0.569996in}}%
\pgfpathlineto{\pgfqpoint{0.960062in}{0.400830in}}%
\pgfpathlineto{\pgfqpoint{0.961351in}{0.410291in}}%
\pgfpathlineto{\pgfqpoint{0.962640in}{0.433814in}}%
\pgfpathlineto{\pgfqpoint{0.963929in}{0.549809in}}%
\pgfpathlineto{\pgfqpoint{0.967797in}{0.400305in}}%
\pgfpathlineto{\pgfqpoint{0.971664in}{0.419363in}}%
\pgfpathlineto{\pgfqpoint{0.972953in}{0.435120in}}%
\pgfpathlineto{\pgfqpoint{0.976821in}{0.400306in}}%
\pgfpathlineto{\pgfqpoint{0.978110in}{0.485138in}}%
\pgfpathlineto{\pgfqpoint{0.979399in}{0.406428in}}%
\pgfpathlineto{\pgfqpoint{0.980689in}{0.502000in}}%
\pgfpathlineto{\pgfqpoint{0.981978in}{0.463859in}}%
\pgfpathlineto{\pgfqpoint{0.985845in}{0.403996in}}%
\pgfpathlineto{\pgfqpoint{0.987135in}{0.415096in}}%
\pgfpathlineto{\pgfqpoint{0.988424in}{0.555074in}}%
\pgfpathlineto{\pgfqpoint{0.989713in}{0.405257in}}%
\pgfpathlineto{\pgfqpoint{0.991002in}{0.466006in}}%
\pgfpathlineto{\pgfqpoint{0.994870in}{0.416447in}}%
\pgfpathlineto{\pgfqpoint{0.996159in}{0.415244in}}%
\pgfpathlineto{\pgfqpoint{0.997448in}{0.404033in}}%
\pgfpathlineto{\pgfqpoint{1.000027in}{0.460149in}}%
\pgfpathlineto{\pgfqpoint{1.005183in}{0.400843in}}%
\pgfpathlineto{\pgfqpoint{1.006473in}{0.411761in}}%
\pgfpathlineto{\pgfqpoint{1.007762in}{0.400412in}}%
\pgfpathlineto{\pgfqpoint{1.009051in}{0.467044in}}%
\pgfpathlineto{\pgfqpoint{1.012919in}{0.401921in}}%
\pgfpathlineto{\pgfqpoint{1.014208in}{0.407768in}}%
\pgfpathlineto{\pgfqpoint{1.015497in}{0.402381in}}%
\pgfpathlineto{\pgfqpoint{1.016786in}{0.402365in}}%
\pgfpathlineto{\pgfqpoint{1.018075in}{0.407554in}}%
\pgfpathlineto{\pgfqpoint{1.021943in}{0.407554in}}%
\pgfpathlineto{\pgfqpoint{1.023232in}{0.409766in}}%
\pgfpathlineto{\pgfqpoint{1.024521in}{0.409918in}}%
\pgfpathlineto{\pgfqpoint{1.025810in}{0.421055in}}%
\pgfpathlineto{\pgfqpoint{1.027100in}{0.414909in}}%
\pgfpathlineto{\pgfqpoint{1.030967in}{0.402425in}}%
\pgfpathlineto{\pgfqpoint{1.032256in}{0.403019in}}%
\pgfpathlineto{\pgfqpoint{1.033546in}{0.433494in}}%
\pgfpathlineto{\pgfqpoint{1.034835in}{0.513351in}}%
\pgfpathlineto{\pgfqpoint{1.036124in}{0.403996in}}%
\pgfpathlineto{\pgfqpoint{1.039992in}{0.401623in}}%
\pgfpathlineto{\pgfqpoint{1.041281in}{0.406594in}}%
\pgfpathlineto{\pgfqpoint{1.043859in}{0.442514in}}%
\pgfpathlineto{\pgfqpoint{1.045148in}{0.478349in}}%
\pgfpathlineto{\pgfqpoint{1.050305in}{0.442005in}}%
\pgfpathlineto{\pgfqpoint{1.051594in}{0.400427in}}%
\pgfpathlineto{\pgfqpoint{1.054173in}{0.604120in}}%
\pgfpathlineto{\pgfqpoint{1.058040in}{0.632405in}}%
\pgfpathlineto{\pgfqpoint{1.059330in}{0.401812in}}%
\pgfpathlineto{\pgfqpoint{1.060619in}{0.406452in}}%
\pgfpathlineto{\pgfqpoint{1.061908in}{0.408675in}}%
\pgfpathlineto{\pgfqpoint{1.063197in}{0.400314in}}%
\pgfpathlineto{\pgfqpoint{1.068354in}{0.401337in}}%
\pgfpathlineto{\pgfqpoint{1.069643in}{0.414179in}}%
\pgfpathlineto{\pgfqpoint{1.070932in}{0.400445in}}%
\pgfpathlineto{\pgfqpoint{1.072222in}{0.400965in}}%
\pgfpathlineto{\pgfqpoint{1.076089in}{0.430045in}}%
\pgfpathlineto{\pgfqpoint{1.077378in}{0.420196in}}%
\pgfpathlineto{\pgfqpoint{1.078668in}{0.403955in}}%
\pgfpathlineto{\pgfqpoint{1.079957in}{0.454665in}}%
\pgfpathlineto{\pgfqpoint{1.081246in}{0.454665in}}%
\pgfpathlineto{\pgfqpoint{1.085113in}{0.403155in}}%
\pgfpathlineto{\pgfqpoint{1.086403in}{0.402461in}}%
\pgfpathlineto{\pgfqpoint{1.087692in}{0.426253in}}%
\pgfpathlineto{\pgfqpoint{1.088981in}{0.401922in}}%
\pgfpathlineto{\pgfqpoint{1.090270in}{0.413444in}}%
\pgfpathlineto{\pgfqpoint{1.094138in}{0.409190in}}%
\pgfpathlineto{\pgfqpoint{1.095427in}{0.513414in}}%
\pgfpathlineto{\pgfqpoint{1.096716in}{0.406168in}}%
\pgfpathlineto{\pgfqpoint{1.098005in}{0.402684in}}%
\pgfpathlineto{\pgfqpoint{1.099295in}{0.413043in}}%
\pgfpathlineto{\pgfqpoint{1.103162in}{0.420182in}}%
\pgfpathlineto{\pgfqpoint{1.104451in}{0.519876in}}%
\pgfpathlineto{\pgfqpoint{1.105741in}{0.496877in}}%
\pgfpathlineto{\pgfqpoint{1.107030in}{0.405192in}}%
\pgfpathlineto{\pgfqpoint{1.108319in}{0.451894in}}%
\pgfpathlineto{\pgfqpoint{1.113476in}{0.405283in}}%
\pgfpathlineto{\pgfqpoint{1.114765in}{0.405283in}}%
\pgfpathlineto{\pgfqpoint{1.116054in}{0.416436in}}%
\pgfpathlineto{\pgfqpoint{1.117343in}{0.421789in}}%
\pgfpathlineto{\pgfqpoint{1.123789in}{0.433349in}}%
\pgfpathlineto{\pgfqpoint{1.125079in}{0.525765in}}%
\pgfpathlineto{\pgfqpoint{1.126368in}{0.412381in}}%
\pgfpathlineto{\pgfqpoint{1.130235in}{0.400271in}}%
\pgfpathlineto{\pgfqpoint{1.131525in}{0.404157in}}%
\pgfpathlineto{\pgfqpoint{1.132814in}{0.618397in}}%
\pgfpathlineto{\pgfqpoint{1.134103in}{0.402831in}}%
\pgfpathlineto{\pgfqpoint{1.135392in}{0.400481in}}%
\pgfpathlineto{\pgfqpoint{1.139260in}{0.400745in}}%
\pgfpathlineto{\pgfqpoint{1.140549in}{0.486725in}}%
\pgfpathlineto{\pgfqpoint{1.143127in}{0.402320in}}%
\pgfpathlineto{\pgfqpoint{1.144416in}{0.409779in}}%
\pgfpathlineto{\pgfqpoint{1.148284in}{0.401663in}}%
\pgfpathlineto{\pgfqpoint{1.149573in}{0.614599in}}%
\pgfpathlineto{\pgfqpoint{1.150862in}{0.408950in}}%
\pgfpathlineto{\pgfqpoint{1.153441in}{0.453907in}}%
\pgfpathlineto{\pgfqpoint{1.158598in}{0.401190in}}%
\pgfpathlineto{\pgfqpoint{1.159887in}{0.409873in}}%
\pgfpathlineto{\pgfqpoint{1.161176in}{0.522971in}}%
\pgfpathlineto{\pgfqpoint{1.162465in}{0.400807in}}%
\pgfpathlineto{\pgfqpoint{1.166333in}{0.400509in}}%
\pgfpathlineto{\pgfqpoint{1.167622in}{0.471699in}}%
\pgfpathlineto{\pgfqpoint{1.168911in}{0.406012in}}%
\pgfpathlineto{\pgfqpoint{1.170200in}{0.435824in}}%
\pgfpathlineto{\pgfqpoint{1.171490in}{0.400271in}}%
\pgfpathlineto{\pgfqpoint{1.175357in}{0.402295in}}%
\pgfpathlineto{\pgfqpoint{1.176646in}{0.516408in}}%
\pgfpathlineto{\pgfqpoint{1.177936in}{0.400485in}}%
\pgfpathlineto{\pgfqpoint{1.179225in}{0.412363in}}%
\pgfpathlineto{\pgfqpoint{1.180514in}{0.401137in}}%
\pgfpathlineto{\pgfqpoint{1.184382in}{0.400756in}}%
\pgfpathlineto{\pgfqpoint{1.185671in}{0.454300in}}%
\pgfpathlineto{\pgfqpoint{1.186960in}{0.406496in}}%
\pgfpathlineto{\pgfqpoint{1.188249in}{0.402118in}}%
\pgfpathlineto{\pgfqpoint{1.189538in}{0.423253in}}%
\pgfpathlineto{\pgfqpoint{1.193406in}{0.405579in}}%
\pgfpathlineto{\pgfqpoint{1.195984in}{0.400271in}}%
\pgfpathlineto{\pgfqpoint{1.197274in}{0.406774in}}%
\pgfpathlineto{\pgfqpoint{1.198563in}{0.429285in}}%
\pgfpathlineto{\pgfqpoint{1.202430in}{0.456133in}}%
\pgfpathlineto{\pgfqpoint{1.205009in}{0.598646in}}%
\pgfpathlineto{\pgfqpoint{1.206298in}{0.401519in}}%
\pgfpathlineto{\pgfqpoint{1.207587in}{0.408796in}}%
\pgfpathlineto{\pgfqpoint{1.211455in}{0.402751in}}%
\pgfpathlineto{\pgfqpoint{1.212744in}{0.408855in}}%
\pgfpathlineto{\pgfqpoint{1.214033in}{0.445538in}}%
\pgfpathlineto{\pgfqpoint{1.215322in}{0.438417in}}%
\pgfpathlineto{\pgfqpoint{1.216611in}{0.412422in}}%
\pgfpathlineto{\pgfqpoint{1.220479in}{0.400271in}}%
\pgfpathlineto{\pgfqpoint{1.221768in}{0.404101in}}%
\pgfpathlineto{\pgfqpoint{1.223057in}{0.415512in}}%
\pgfpathlineto{\pgfqpoint{1.224347in}{0.500524in}}%
\pgfpathlineto{\pgfqpoint{1.225636in}{1.046442in}}%
\pgfpathlineto{\pgfqpoint{1.230793in}{0.402092in}}%
\pgfpathlineto{\pgfqpoint{1.232082in}{0.401544in}}%
\pgfpathlineto{\pgfqpoint{1.233371in}{0.408966in}}%
\pgfpathlineto{\pgfqpoint{1.234660in}{0.400323in}}%
\pgfpathlineto{\pgfqpoint{1.238528in}{0.402798in}}%
\pgfpathlineto{\pgfqpoint{1.239817in}{0.418607in}}%
\pgfpathlineto{\pgfqpoint{1.241106in}{0.402723in}}%
\pgfpathlineto{\pgfqpoint{1.242395in}{0.437171in}}%
\pgfpathlineto{\pgfqpoint{1.243685in}{0.434508in}}%
\pgfpathlineto{\pgfqpoint{1.247552in}{0.413162in}}%
\pgfpathlineto{\pgfqpoint{1.248841in}{0.420366in}}%
\pgfpathlineto{\pgfqpoint{1.250131in}{0.413102in}}%
\pgfpathlineto{\pgfqpoint{1.251420in}{0.410267in}}%
\pgfpathlineto{\pgfqpoint{1.252709in}{0.411739in}}%
\pgfpathlineto{\pgfqpoint{1.256577in}{0.400725in}}%
\pgfpathlineto{\pgfqpoint{1.257866in}{0.408717in}}%
\pgfpathlineto{\pgfqpoint{1.259155in}{0.401508in}}%
\pgfpathlineto{\pgfqpoint{1.260444in}{0.400469in}}%
\pgfpathlineto{\pgfqpoint{1.261733in}{0.404296in}}%
\pgfpathlineto{\pgfqpoint{1.266890in}{0.400471in}}%
\pgfpathlineto{\pgfqpoint{1.269468in}{0.485195in}}%
\pgfpathlineto{\pgfqpoint{1.270758in}{0.410836in}}%
\pgfpathlineto{\pgfqpoint{1.274625in}{0.412406in}}%
\pgfpathlineto{\pgfqpoint{1.275914in}{0.445253in}}%
\pgfpathlineto{\pgfqpoint{1.277204in}{0.440818in}}%
\pgfpathlineto{\pgfqpoint{1.278493in}{0.401087in}}%
\pgfpathlineto{\pgfqpoint{1.279782in}{0.408845in}}%
\pgfpathlineto{\pgfqpoint{1.283650in}{0.420182in}}%
\pgfpathlineto{\pgfqpoint{1.284939in}{0.417845in}}%
\pgfpathlineto{\pgfqpoint{1.286228in}{0.419320in}}%
\pgfpathlineto{\pgfqpoint{1.287517in}{0.407007in}}%
\pgfpathlineto{\pgfqpoint{1.288806in}{0.432142in}}%
\pgfpathlineto{\pgfqpoint{1.292674in}{0.404127in}}%
\pgfpathlineto{\pgfqpoint{1.293963in}{0.401031in}}%
\pgfpathlineto{\pgfqpoint{1.296542in}{0.400319in}}%
\pgfpathlineto{\pgfqpoint{1.297831in}{0.423555in}}%
\pgfpathlineto{\pgfqpoint{1.301698in}{0.405170in}}%
\pgfpathlineto{\pgfqpoint{1.302988in}{0.405226in}}%
\pgfpathlineto{\pgfqpoint{1.304277in}{0.400719in}}%
\pgfpathlineto{\pgfqpoint{1.305566in}{0.406319in}}%
\pgfpathlineto{\pgfqpoint{1.306855in}{0.426528in}}%
\pgfpathlineto{\pgfqpoint{1.310723in}{0.409950in}}%
\pgfpathlineto{\pgfqpoint{1.312012in}{0.526050in}}%
\pgfpathlineto{\pgfqpoint{1.313301in}{0.401443in}}%
\pgfpathlineto{\pgfqpoint{1.314590in}{0.400271in}}%
\pgfpathlineto{\pgfqpoint{1.319747in}{0.452105in}}%
\pgfpathlineto{\pgfqpoint{1.321036in}{0.400465in}}%
\pgfpathlineto{\pgfqpoint{1.322325in}{0.457291in}}%
\pgfpathlineto{\pgfqpoint{1.323615in}{0.403475in}}%
\pgfpathlineto{\pgfqpoint{1.324904in}{0.414815in}}%
\pgfpathlineto{\pgfqpoint{1.328771in}{0.408796in}}%
\pgfpathlineto{\pgfqpoint{1.330061in}{0.541844in}}%
\pgfpathlineto{\pgfqpoint{1.331350in}{0.481434in}}%
\pgfpathlineto{\pgfqpoint{1.332639in}{0.456416in}}%
\pgfpathlineto{\pgfqpoint{1.333928in}{0.408225in}}%
\pgfpathlineto{\pgfqpoint{1.337796in}{0.427950in}}%
\pgfpathlineto{\pgfqpoint{1.339085in}{0.492269in}}%
\pgfpathlineto{\pgfqpoint{1.340374in}{0.400318in}}%
\pgfpathlineto{\pgfqpoint{1.341663in}{0.428821in}}%
\pgfpathlineto{\pgfqpoint{1.342953in}{0.413173in}}%
\pgfpathlineto{\pgfqpoint{1.348109in}{0.470287in}}%
\pgfpathlineto{\pgfqpoint{1.349399in}{0.421582in}}%
\pgfpathlineto{\pgfqpoint{1.350688in}{0.421582in}}%
\pgfpathlineto{\pgfqpoint{1.351977in}{0.400434in}}%
\pgfpathlineto{\pgfqpoint{1.355845in}{0.436323in}}%
\pgfpathlineto{\pgfqpoint{1.357134in}{0.409079in}}%
\pgfpathlineto{\pgfqpoint{1.358423in}{0.400892in}}%
\pgfpathlineto{\pgfqpoint{1.359712in}{0.404121in}}%
\pgfpathlineto{\pgfqpoint{1.361001in}{0.401646in}}%
\pgfpathlineto{\pgfqpoint{1.364869in}{0.400882in}}%
\pgfpathlineto{\pgfqpoint{1.366158in}{0.415409in}}%
\pgfpathlineto{\pgfqpoint{1.367447in}{0.402660in}}%
\pgfpathlineto{\pgfqpoint{1.368737in}{0.403269in}}%
\pgfpathlineto{\pgfqpoint{1.370026in}{0.405545in}}%
\pgfpathlineto{\pgfqpoint{1.373893in}{0.445650in}}%
\pgfpathlineto{\pgfqpoint{1.375183in}{0.415500in}}%
\pgfpathlineto{\pgfqpoint{1.376472in}{0.434366in}}%
\pgfpathlineto{\pgfqpoint{1.379050in}{0.402703in}}%
\pgfpathlineto{\pgfqpoint{1.382918in}{0.400876in}}%
\pgfpathlineto{\pgfqpoint{1.385496in}{0.402114in}}%
\pgfpathlineto{\pgfqpoint{1.386785in}{0.400422in}}%
\pgfpathlineto{\pgfqpoint{1.388074in}{0.404871in}}%
\pgfpathlineto{\pgfqpoint{1.393231in}{0.407711in}}%
\pgfpathlineto{\pgfqpoint{1.394520in}{0.409844in}}%
\pgfpathlineto{\pgfqpoint{1.395810in}{0.401201in}}%
\pgfpathlineto{\pgfqpoint{1.397099in}{0.401201in}}%
\pgfpathlineto{\pgfqpoint{1.400966in}{0.628616in}}%
\pgfpathlineto{\pgfqpoint{1.402256in}{0.403687in}}%
\pgfpathlineto{\pgfqpoint{1.403545in}{0.505865in}}%
\pgfpathlineto{\pgfqpoint{1.404834in}{0.400846in}}%
\pgfpathlineto{\pgfqpoint{1.406123in}{0.401173in}}%
\pgfpathlineto{\pgfqpoint{1.409991in}{0.443821in}}%
\pgfpathlineto{\pgfqpoint{1.411280in}{0.422383in}}%
\pgfpathlineto{\pgfqpoint{1.412569in}{0.416226in}}%
\pgfpathlineto{\pgfqpoint{1.413858in}{0.469535in}}%
\pgfpathlineto{\pgfqpoint{1.415148in}{0.401149in}}%
\pgfpathlineto{\pgfqpoint{1.419015in}{0.408132in}}%
\pgfpathlineto{\pgfqpoint{1.421594in}{0.452505in}}%
\pgfpathlineto{\pgfqpoint{1.422883in}{0.562543in}}%
\pgfpathlineto{\pgfqpoint{1.424172in}{0.400309in}}%
\pgfpathlineto{\pgfqpoint{1.428040in}{0.503851in}}%
\pgfpathlineto{\pgfqpoint{1.429329in}{0.424539in}}%
\pgfpathlineto{\pgfqpoint{1.430618in}{0.404885in}}%
\pgfpathlineto{\pgfqpoint{1.431907in}{0.400612in}}%
\pgfpathlineto{\pgfqpoint{1.437064in}{0.429882in}}%
\pgfpathlineto{\pgfqpoint{1.438353in}{0.407832in}}%
\pgfpathlineto{\pgfqpoint{1.439642in}{0.400620in}}%
\pgfpathlineto{\pgfqpoint{1.442221in}{0.424175in}}%
\pgfpathlineto{\pgfqpoint{1.446088in}{0.603344in}}%
\pgfpathlineto{\pgfqpoint{1.447377in}{0.400923in}}%
\pgfpathlineto{\pgfqpoint{1.448667in}{0.403561in}}%
\pgfpathlineto{\pgfqpoint{1.449956in}{0.546099in}}%
\pgfpathlineto{\pgfqpoint{1.451245in}{0.402140in}}%
\pgfpathlineto{\pgfqpoint{1.455113in}{0.400615in}}%
\pgfpathlineto{\pgfqpoint{1.456402in}{0.425727in}}%
\pgfpathlineto{\pgfqpoint{1.457691in}{0.402660in}}%
\pgfpathlineto{\pgfqpoint{1.458980in}{0.625165in}}%
\pgfpathlineto{\pgfqpoint{1.460269in}{0.412071in}}%
\pgfpathlineto{\pgfqpoint{1.464137in}{0.420440in}}%
\pgfpathlineto{\pgfqpoint{1.465426in}{0.400440in}}%
\pgfpathlineto{\pgfqpoint{1.466715in}{0.409702in}}%
\pgfpathlineto{\pgfqpoint{1.468005in}{0.405274in}}%
\pgfpathlineto{\pgfqpoint{1.469294in}{0.412047in}}%
\pgfpathlineto{\pgfqpoint{1.473161in}{0.400433in}}%
\pgfpathlineto{\pgfqpoint{1.474451in}{0.406063in}}%
\pgfpathlineto{\pgfqpoint{1.475740in}{0.400912in}}%
\pgfpathlineto{\pgfqpoint{1.477029in}{0.405156in}}%
\pgfpathlineto{\pgfqpoint{1.478318in}{0.400312in}}%
\pgfpathlineto{\pgfqpoint{1.482186in}{0.401736in}}%
\pgfpathlineto{\pgfqpoint{1.483475in}{0.404374in}}%
\pgfpathlineto{\pgfqpoint{1.484764in}{0.403627in}}%
\pgfpathlineto{\pgfqpoint{1.486053in}{0.418833in}}%
\pgfpathlineto{\pgfqpoint{1.487343in}{0.401333in}}%
\pgfpathlineto{\pgfqpoint{1.491210in}{0.405364in}}%
\pgfpathlineto{\pgfqpoint{1.492499in}{0.404478in}}%
\pgfpathlineto{\pgfqpoint{1.493789in}{0.401326in}}%
\pgfpathlineto{\pgfqpoint{1.495078in}{0.491493in}}%
\pgfpathlineto{\pgfqpoint{1.496367in}{0.404714in}}%
\pgfpathlineto{\pgfqpoint{1.500235in}{0.442483in}}%
\pgfpathlineto{\pgfqpoint{1.501524in}{0.417368in}}%
\pgfpathlineto{\pgfqpoint{1.502813in}{0.436390in}}%
\pgfpathlineto{\pgfqpoint{1.504102in}{0.402401in}}%
\pgfpathlineto{\pgfqpoint{1.505391in}{0.411260in}}%
\pgfpathlineto{\pgfqpoint{1.509259in}{0.409924in}}%
\pgfpathlineto{\pgfqpoint{1.510548in}{0.401833in}}%
\pgfpathlineto{\pgfqpoint{1.511837in}{0.403045in}}%
\pgfpathlineto{\pgfqpoint{1.513126in}{0.416009in}}%
\pgfpathlineto{\pgfqpoint{1.514416in}{0.402438in}}%
\pgfpathlineto{\pgfqpoint{1.519572in}{0.403099in}}%
\pgfpathlineto{\pgfqpoint{1.520862in}{0.495047in}}%
\pgfpathlineto{\pgfqpoint{1.522151in}{0.400649in}}%
\pgfpathlineto{\pgfqpoint{1.523440in}{0.401778in}}%
\pgfpathlineto{\pgfqpoint{1.527308in}{0.416783in}}%
\pgfpathlineto{\pgfqpoint{1.528597in}{0.401737in}}%
\pgfpathlineto{\pgfqpoint{1.529886in}{0.408287in}}%
\pgfpathlineto{\pgfqpoint{1.531175in}{0.409613in}}%
\pgfpathlineto{\pgfqpoint{1.532464in}{0.587064in}}%
\pgfpathlineto{\pgfqpoint{1.536332in}{0.401249in}}%
\pgfpathlineto{\pgfqpoint{1.537621in}{0.435024in}}%
\pgfpathlineto{\pgfqpoint{1.538910in}{0.406657in}}%
\pgfpathlineto{\pgfqpoint{1.540200in}{0.400309in}}%
\pgfpathlineto{\pgfqpoint{1.541489in}{0.405705in}}%
\pgfpathlineto{\pgfqpoint{1.545356in}{0.405770in}}%
\pgfpathlineto{\pgfqpoint{1.546646in}{0.401650in}}%
\pgfpathlineto{\pgfqpoint{1.547935in}{0.400271in}}%
\pgfpathlineto{\pgfqpoint{1.550513in}{0.451921in}}%
\pgfpathlineto{\pgfqpoint{1.554381in}{0.401289in}}%
\pgfpathlineto{\pgfqpoint{1.555670in}{0.402894in}}%
\pgfpathlineto{\pgfqpoint{1.556959in}{0.401297in}}%
\pgfpathlineto{\pgfqpoint{1.558248in}{0.424148in}}%
\pgfpathlineto{\pgfqpoint{1.559538in}{0.413744in}}%
\pgfpathlineto{\pgfqpoint{1.563405in}{0.400313in}}%
\pgfpathlineto{\pgfqpoint{1.564694in}{0.435409in}}%
\pgfpathlineto{\pgfqpoint{1.565983in}{0.403691in}}%
\pgfpathlineto{\pgfqpoint{1.567273in}{0.476245in}}%
\pgfpathlineto{\pgfqpoint{1.568562in}{0.407017in}}%
\pgfpathlineto{\pgfqpoint{1.572429in}{0.410338in}}%
\pgfpathlineto{\pgfqpoint{1.573719in}{0.401249in}}%
\pgfpathlineto{\pgfqpoint{1.575008in}{0.426427in}}%
\pgfpathlineto{\pgfqpoint{1.577586in}{0.400609in}}%
\pgfpathlineto{\pgfqpoint{1.581454in}{0.416672in}}%
\pgfpathlineto{\pgfqpoint{1.582743in}{0.401600in}}%
\pgfpathlineto{\pgfqpoint{1.584032in}{0.427627in}}%
\pgfpathlineto{\pgfqpoint{1.585321in}{0.400613in}}%
\pgfpathlineto{\pgfqpoint{1.586611in}{0.456840in}}%
\pgfpathlineto{\pgfqpoint{1.590478in}{0.403902in}}%
\pgfpathlineto{\pgfqpoint{1.591767in}{0.406338in}}%
\pgfpathlineto{\pgfqpoint{1.593057in}{0.400307in}}%
\pgfpathlineto{\pgfqpoint{1.594346in}{0.400593in}}%
\pgfpathlineto{\pgfqpoint{1.595635in}{0.405458in}}%
\pgfpathlineto{\pgfqpoint{1.600792in}{0.400271in}}%
\pgfpathlineto{\pgfqpoint{1.602081in}{0.413306in}}%
\pgfpathlineto{\pgfqpoint{1.603370in}{0.409501in}}%
\pgfpathlineto{\pgfqpoint{1.604659in}{0.400417in}}%
\pgfpathlineto{\pgfqpoint{1.608527in}{0.402042in}}%
\pgfpathlineto{\pgfqpoint{1.609816in}{0.417519in}}%
\pgfpathlineto{\pgfqpoint{1.611105in}{0.407135in}}%
\pgfpathlineto{\pgfqpoint{1.612395in}{0.410407in}}%
\pgfpathlineto{\pgfqpoint{1.613684in}{0.404527in}}%
\pgfpathlineto{\pgfqpoint{1.617551in}{0.401527in}}%
\pgfpathlineto{\pgfqpoint{1.620130in}{0.405266in}}%
\pgfpathlineto{\pgfqpoint{1.621419in}{0.510589in}}%
\pgfpathlineto{\pgfqpoint{1.622708in}{0.870394in}}%
\pgfpathlineto{\pgfqpoint{1.626576in}{0.403985in}}%
\pgfpathlineto{\pgfqpoint{1.627865in}{0.402669in}}%
\pgfpathlineto{\pgfqpoint{1.629154in}{0.416688in}}%
\pgfpathlineto{\pgfqpoint{1.631732in}{0.401195in}}%
\pgfpathlineto{\pgfqpoint{1.636889in}{0.406645in}}%
\pgfpathlineto{\pgfqpoint{1.638178in}{0.409911in}}%
\pgfpathlineto{\pgfqpoint{1.639468in}{0.474259in}}%
\pgfpathlineto{\pgfqpoint{1.640757in}{0.477453in}}%
\pgfpathlineto{\pgfqpoint{1.644624in}{0.403027in}}%
\pgfpathlineto{\pgfqpoint{1.645914in}{0.403027in}}%
\pgfpathlineto{\pgfqpoint{1.647203in}{0.440409in}}%
\pgfpathlineto{\pgfqpoint{1.648492in}{0.400835in}}%
\pgfpathlineto{\pgfqpoint{1.649781in}{0.408239in}}%
\pgfpathlineto{\pgfqpoint{1.653649in}{0.407215in}}%
\pgfpathlineto{\pgfqpoint{1.654938in}{0.410358in}}%
\pgfpathlineto{\pgfqpoint{1.656227in}{0.460135in}}%
\pgfpathlineto{\pgfqpoint{1.657516in}{0.400305in}}%
\pgfpathlineto{\pgfqpoint{1.658806in}{0.401908in}}%
\pgfpathlineto{\pgfqpoint{1.662673in}{0.416329in}}%
\pgfpathlineto{\pgfqpoint{1.663962in}{0.403540in}}%
\pgfpathlineto{\pgfqpoint{1.666541in}{0.417299in}}%
\pgfpathlineto{\pgfqpoint{1.667830in}{0.402862in}}%
\pgfpathlineto{\pgfqpoint{1.671698in}{0.414298in}}%
\pgfpathlineto{\pgfqpoint{1.672987in}{0.403407in}}%
\pgfpathlineto{\pgfqpoint{1.675565in}{0.407343in}}%
\pgfpathlineto{\pgfqpoint{1.676854in}{0.400303in}}%
\pgfpathlineto{\pgfqpoint{1.680722in}{0.423651in}}%
\pgfpathlineto{\pgfqpoint{1.682011in}{0.404181in}}%
\pgfpathlineto{\pgfqpoint{1.683300in}{0.405736in}}%
\pgfpathlineto{\pgfqpoint{1.684589in}{0.403540in}}%
\pgfpathlineto{\pgfqpoint{1.685879in}{0.410822in}}%
\pgfpathlineto{\pgfqpoint{1.689746in}{0.400401in}}%
\pgfpathlineto{\pgfqpoint{1.691035in}{0.625146in}}%
\pgfpathlineto{\pgfqpoint{1.692325in}{0.406105in}}%
\pgfpathlineto{\pgfqpoint{1.693614in}{0.403868in}}%
\pgfpathlineto{\pgfqpoint{1.694903in}{0.507046in}}%
\pgfpathlineto{\pgfqpoint{1.700060in}{0.415673in}}%
\pgfpathlineto{\pgfqpoint{1.701349in}{0.417457in}}%
\pgfpathlineto{\pgfqpoint{1.702638in}{0.408741in}}%
\pgfpathlineto{\pgfqpoint{1.703927in}{0.424882in}}%
\pgfpathlineto{\pgfqpoint{1.707795in}{0.401952in}}%
\pgfpathlineto{\pgfqpoint{1.710373in}{0.412633in}}%
\pgfpathlineto{\pgfqpoint{1.711663in}{0.401133in}}%
\pgfpathlineto{\pgfqpoint{1.712952in}{0.410291in}}%
\pgfpathlineto{\pgfqpoint{1.716819in}{0.489802in}}%
\pgfpathlineto{\pgfqpoint{1.718109in}{0.404736in}}%
\pgfpathlineto{\pgfqpoint{1.719398in}{0.401611in}}%
\pgfpathlineto{\pgfqpoint{1.720687in}{0.458816in}}%
\pgfpathlineto{\pgfqpoint{1.721976in}{0.413105in}}%
\pgfpathlineto{\pgfqpoint{1.725844in}{0.401992in}}%
\pgfpathlineto{\pgfqpoint{1.727133in}{0.408087in}}%
\pgfpathlineto{\pgfqpoint{1.728422in}{0.401956in}}%
\pgfpathlineto{\pgfqpoint{1.729711in}{0.406025in}}%
\pgfpathlineto{\pgfqpoint{1.731001in}{0.401115in}}%
\pgfpathlineto{\pgfqpoint{1.736157in}{0.400271in}}%
\pgfpathlineto{\pgfqpoint{1.737447in}{0.416743in}}%
\pgfpathlineto{\pgfqpoint{1.738736in}{0.413897in}}%
\pgfpathlineto{\pgfqpoint{1.740025in}{0.425187in}}%
\pgfpathlineto{\pgfqpoint{1.743892in}{0.411392in}}%
\pgfpathlineto{\pgfqpoint{1.745182in}{0.400305in}}%
\pgfpathlineto{\pgfqpoint{1.746471in}{0.407881in}}%
\pgfpathlineto{\pgfqpoint{1.747760in}{0.400574in}}%
\pgfpathlineto{\pgfqpoint{1.749049in}{0.400271in}}%
\pgfpathlineto{\pgfqpoint{1.752917in}{0.416743in}}%
\pgfpathlineto{\pgfqpoint{1.754206in}{0.403045in}}%
\pgfpathlineto{\pgfqpoint{1.755495in}{0.403045in}}%
\pgfpathlineto{\pgfqpoint{1.756784in}{0.404411in}}%
\pgfpathlineto{\pgfqpoint{1.758074in}{0.400305in}}%
\pgfpathlineto{\pgfqpoint{1.761941in}{0.410018in}}%
\pgfpathlineto{\pgfqpoint{1.763230in}{0.403633in}}%
\pgfpathlineto{\pgfqpoint{1.764520in}{0.400575in}}%
\pgfpathlineto{\pgfqpoint{1.765809in}{0.408959in}}%
\pgfpathlineto{\pgfqpoint{1.767098in}{0.401506in}}%
\pgfpathlineto{\pgfqpoint{1.770966in}{0.410130in}}%
\pgfpathlineto{\pgfqpoint{1.772255in}{0.403639in}}%
\pgfpathlineto{\pgfqpoint{1.773544in}{0.409882in}}%
\pgfpathlineto{\pgfqpoint{1.774833in}{0.440033in}}%
\pgfpathlineto{\pgfqpoint{1.776122in}{0.415887in}}%
\pgfpathlineto{\pgfqpoint{1.779990in}{0.400794in}}%
\pgfpathlineto{\pgfqpoint{1.781279in}{0.427418in}}%
\pgfpathlineto{\pgfqpoint{1.782568in}{0.401837in}}%
\pgfpathlineto{\pgfqpoint{1.783858in}{0.401429in}}%
\pgfpathlineto{\pgfqpoint{1.785147in}{0.423479in}}%
\pgfpathlineto{\pgfqpoint{1.789014in}{0.408263in}}%
\pgfpathlineto{\pgfqpoint{1.790304in}{0.407197in}}%
\pgfpathlineto{\pgfqpoint{1.791593in}{0.402235in}}%
\pgfpathlineto{\pgfqpoint{1.792882in}{0.458727in}}%
\pgfpathlineto{\pgfqpoint{1.794171in}{0.413451in}}%
\pgfpathlineto{\pgfqpoint{1.798039in}{0.423629in}}%
\pgfpathlineto{\pgfqpoint{1.799328in}{0.437809in}}%
\pgfpathlineto{\pgfqpoint{1.800617in}{0.401295in}}%
\pgfpathlineto{\pgfqpoint{1.801906in}{0.464227in}}%
\pgfpathlineto{\pgfqpoint{1.803195in}{0.414716in}}%
\pgfpathlineto{\pgfqpoint{1.807063in}{0.432615in}}%
\pgfpathlineto{\pgfqpoint{1.808352in}{0.409680in}}%
\pgfpathlineto{\pgfqpoint{1.810931in}{0.402568in}}%
\pgfpathlineto{\pgfqpoint{1.816087in}{0.402085in}}%
\pgfpathlineto{\pgfqpoint{1.817377in}{0.402584in}}%
\pgfpathlineto{\pgfqpoint{1.818666in}{0.402112in}}%
\pgfpathlineto{\pgfqpoint{1.819955in}{0.400271in}}%
\pgfpathlineto{\pgfqpoint{1.821244in}{0.452134in}}%
\pgfpathlineto{\pgfqpoint{1.825112in}{0.401347in}}%
\pgfpathlineto{\pgfqpoint{1.828979in}{0.411969in}}%
\pgfpathlineto{\pgfqpoint{1.830269in}{0.400744in}}%
\pgfpathlineto{\pgfqpoint{1.835425in}{0.400390in}}%
\pgfpathlineto{\pgfqpoint{1.836715in}{0.406883in}}%
\pgfpathlineto{\pgfqpoint{1.838004in}{0.400534in}}%
\pgfpathlineto{\pgfqpoint{1.839293in}{0.400535in}}%
\pgfpathlineto{\pgfqpoint{1.844450in}{0.401697in}}%
\pgfpathlineto{\pgfqpoint{1.845739in}{0.403185in}}%
\pgfpathlineto{\pgfqpoint{1.847028in}{0.423505in}}%
\pgfpathlineto{\pgfqpoint{1.848317in}{0.408008in}}%
\pgfpathlineto{\pgfqpoint{1.852185in}{0.411166in}}%
\pgfpathlineto{\pgfqpoint{1.853474in}{0.400271in}}%
\pgfpathlineto{\pgfqpoint{1.854763in}{0.405299in}}%
\pgfpathlineto{\pgfqpoint{1.856053in}{0.400745in}}%
\pgfpathlineto{\pgfqpoint{1.857342in}{0.404524in}}%
\pgfpathlineto{\pgfqpoint{1.862499in}{0.437743in}}%
\pgfpathlineto{\pgfqpoint{1.863788in}{0.406634in}}%
\pgfpathlineto{\pgfqpoint{1.865077in}{0.401644in}}%
\pgfpathlineto{\pgfqpoint{1.866366in}{0.426756in}}%
\pgfpathlineto{\pgfqpoint{1.870234in}{0.400953in}}%
\pgfpathlineto{\pgfqpoint{1.871523in}{0.431395in}}%
\pgfpathlineto{\pgfqpoint{1.872812in}{0.400936in}}%
\pgfpathlineto{\pgfqpoint{1.874101in}{0.400936in}}%
\pgfpathlineto{\pgfqpoint{1.875390in}{0.450722in}}%
\pgfpathlineto{\pgfqpoint{1.879258in}{0.412769in}}%
\pgfpathlineto{\pgfqpoint{1.880547in}{0.420466in}}%
\pgfpathlineto{\pgfqpoint{1.881836in}{0.417670in}}%
\pgfpathlineto{\pgfqpoint{1.883126in}{0.400375in}}%
\pgfpathlineto{\pgfqpoint{1.884415in}{1.061854in}}%
\pgfpathlineto{\pgfqpoint{1.888282in}{0.403533in}}%
\pgfpathlineto{\pgfqpoint{1.889572in}{0.400836in}}%
\pgfpathlineto{\pgfqpoint{1.890861in}{0.400362in}}%
\pgfpathlineto{\pgfqpoint{1.892150in}{0.404690in}}%
\pgfpathlineto{\pgfqpoint{1.893439in}{0.402505in}}%
\pgfpathlineto{\pgfqpoint{1.897307in}{0.400360in}}%
\pgfpathlineto{\pgfqpoint{1.899885in}{0.422339in}}%
\pgfpathlineto{\pgfqpoint{1.901174in}{0.401638in}}%
\pgfpathlineto{\pgfqpoint{1.902464in}{0.403874in}}%
\pgfpathlineto{\pgfqpoint{1.906331in}{0.400462in}}%
\pgfpathlineto{\pgfqpoint{1.907620in}{0.401310in}}%
\pgfpathlineto{\pgfqpoint{1.908910in}{0.400271in}}%
\pgfpathlineto{\pgfqpoint{1.910199in}{0.404395in}}%
\pgfpathlineto{\pgfqpoint{1.915356in}{0.402372in}}%
\pgfpathlineto{\pgfqpoint{1.916645in}{0.408769in}}%
\pgfpathlineto{\pgfqpoint{1.917934in}{0.401637in}}%
\pgfpathlineto{\pgfqpoint{1.919223in}{0.421800in}}%
\pgfpathlineto{\pgfqpoint{1.920512in}{0.400292in}}%
\pgfpathlineto{\pgfqpoint{1.924380in}{0.409369in}}%
\pgfpathlineto{\pgfqpoint{1.925669in}{0.407610in}}%
\pgfpathlineto{\pgfqpoint{1.926958in}{1.591255in}}%
\pgfpathlineto{\pgfqpoint{1.928247in}{0.517128in}}%
\pgfpathlineto{\pgfqpoint{1.929537in}{0.400271in}}%
\pgfpathlineto{\pgfqpoint{1.933404in}{0.417263in}}%
\pgfpathlineto{\pgfqpoint{1.934693in}{0.468251in}}%
\pgfpathlineto{\pgfqpoint{1.935983in}{0.410840in}}%
\pgfpathlineto{\pgfqpoint{1.937272in}{0.408586in}}%
\pgfpathlineto{\pgfqpoint{1.938561in}{0.400271in}}%
\pgfpathlineto{\pgfqpoint{1.943718in}{0.400549in}}%
\pgfpathlineto{\pgfqpoint{1.945007in}{0.403659in}}%
\pgfpathlineto{\pgfqpoint{1.946296in}{0.428170in}}%
\pgfpathlineto{\pgfqpoint{1.947585in}{0.403272in}}%
\pgfpathlineto{\pgfqpoint{1.951453in}{0.413152in}}%
\pgfpathlineto{\pgfqpoint{1.952742in}{0.536815in}}%
\pgfpathlineto{\pgfqpoint{1.954031in}{0.400346in}}%
\pgfpathlineto{\pgfqpoint{1.955321in}{0.404471in}}%
\pgfpathlineto{\pgfqpoint{1.956610in}{0.400948in}}%
\pgfpathlineto{\pgfqpoint{1.960477in}{0.424384in}}%
\pgfpathlineto{\pgfqpoint{1.961767in}{0.402102in}}%
\pgfpathlineto{\pgfqpoint{1.963056in}{0.525346in}}%
\pgfpathlineto{\pgfqpoint{1.964345in}{0.403663in}}%
\pgfpathlineto{\pgfqpoint{1.965634in}{0.407180in}}%
\pgfpathlineto{\pgfqpoint{1.969502in}{0.407784in}}%
\pgfpathlineto{\pgfqpoint{1.970791in}{0.400882in}}%
\pgfpathlineto{\pgfqpoint{1.972080in}{0.403585in}}%
\pgfpathlineto{\pgfqpoint{1.973369in}{0.452065in}}%
\pgfpathlineto{\pgfqpoint{1.974659in}{0.405556in}}%
\pgfpathlineto{\pgfqpoint{1.978526in}{0.402264in}}%
\pgfpathlineto{\pgfqpoint{1.979815in}{0.400288in}}%
\pgfpathlineto{\pgfqpoint{1.981105in}{0.400288in}}%
\pgfpathlineto{\pgfqpoint{1.982394in}{0.402664in}}%
\pgfpathlineto{\pgfqpoint{1.983683in}{0.409028in}}%
\pgfpathlineto{\pgfqpoint{1.988840in}{0.415204in}}%
\pgfpathlineto{\pgfqpoint{1.990129in}{0.400271in}}%
\pgfpathlineto{\pgfqpoint{1.991418in}{0.414230in}}%
\pgfpathlineto{\pgfqpoint{1.992707in}{0.401321in}}%
\pgfpathlineto{\pgfqpoint{1.996575in}{0.413894in}}%
\pgfpathlineto{\pgfqpoint{1.997864in}{0.422505in}}%
\pgfpathlineto{\pgfqpoint{1.999153in}{0.401910in}}%
\pgfpathlineto{\pgfqpoint{2.000442in}{0.400271in}}%
\pgfpathlineto{\pgfqpoint{2.001732in}{0.420509in}}%
\pgfpathlineto{\pgfqpoint{2.005599in}{0.401092in}}%
\pgfpathlineto{\pgfqpoint{2.006888in}{0.419464in}}%
\pgfpathlineto{\pgfqpoint{2.008178in}{0.400534in}}%
\pgfpathlineto{\pgfqpoint{2.009467in}{0.404977in}}%
\pgfpathlineto{\pgfqpoint{2.010756in}{0.415991in}}%
\pgfpathlineto{\pgfqpoint{2.014624in}{0.401614in}}%
\pgfpathlineto{\pgfqpoint{2.015913in}{0.410749in}}%
\pgfpathlineto{\pgfqpoint{2.017202in}{0.413401in}}%
\pgfpathlineto{\pgfqpoint{2.018491in}{0.447709in}}%
\pgfpathlineto{\pgfqpoint{2.019780in}{0.402344in}}%
\pgfpathlineto{\pgfqpoint{2.023648in}{0.450949in}}%
\pgfpathlineto{\pgfqpoint{2.024937in}{0.401079in}}%
\pgfpathlineto{\pgfqpoint{2.026226in}{0.488026in}}%
\pgfpathlineto{\pgfqpoint{2.027516in}{0.429832in}}%
\pgfpathlineto{\pgfqpoint{2.028805in}{0.435834in}}%
\pgfpathlineto{\pgfqpoint{2.032672in}{0.401117in}}%
\pgfpathlineto{\pgfqpoint{2.033962in}{0.455289in}}%
\pgfpathlineto{\pgfqpoint{2.035251in}{0.485831in}}%
\pgfpathlineto{\pgfqpoint{2.036540in}{0.456751in}}%
\pgfpathlineto{\pgfqpoint{2.037829in}{0.809117in}}%
\pgfpathlineto{\pgfqpoint{2.041697in}{0.430609in}}%
\pgfpathlineto{\pgfqpoint{2.044275in}{0.512316in}}%
\pgfpathlineto{\pgfqpoint{2.045564in}{0.430249in}}%
\pgfpathlineto{\pgfqpoint{2.046853in}{0.445978in}}%
\pgfpathlineto{\pgfqpoint{2.050721in}{0.406112in}}%
\pgfpathlineto{\pgfqpoint{2.052010in}{0.554345in}}%
\pgfpathlineto{\pgfqpoint{2.053299in}{0.416281in}}%
\pgfpathlineto{\pgfqpoint{2.055878in}{0.434421in}}%
\pgfpathlineto{\pgfqpoint{2.059745in}{0.400344in}}%
\pgfpathlineto{\pgfqpoint{2.061035in}{0.439762in}}%
\pgfpathlineto{\pgfqpoint{2.062324in}{0.404209in}}%
\pgfpathlineto{\pgfqpoint{2.063613in}{0.644138in}}%
\pgfpathlineto{\pgfqpoint{2.064902in}{0.678175in}}%
\pgfpathlineto{\pgfqpoint{2.068770in}{0.411876in}}%
\pgfpathlineto{\pgfqpoint{2.070059in}{0.407033in}}%
\pgfpathlineto{\pgfqpoint{2.071348in}{0.439586in}}%
\pgfpathlineto{\pgfqpoint{2.072637in}{0.400704in}}%
\pgfpathlineto{\pgfqpoint{2.073927in}{0.407948in}}%
\pgfpathlineto{\pgfqpoint{2.077794in}{0.414148in}}%
\pgfpathlineto{\pgfqpoint{2.079083in}{0.400557in}}%
\pgfpathlineto{\pgfqpoint{2.080373in}{0.400912in}}%
\pgfpathlineto{\pgfqpoint{2.081662in}{0.413091in}}%
\pgfpathlineto{\pgfqpoint{2.082951in}{0.409417in}}%
\pgfpathlineto{\pgfqpoint{2.086819in}{0.411767in}}%
\pgfpathlineto{\pgfqpoint{2.088108in}{0.428228in}}%
\pgfpathlineto{\pgfqpoint{2.089397in}{0.417250in}}%
\pgfpathlineto{\pgfqpoint{2.090686in}{0.714043in}}%
\pgfpathlineto{\pgfqpoint{2.091975in}{0.415120in}}%
\pgfpathlineto{\pgfqpoint{2.095843in}{0.450058in}}%
\pgfpathlineto{\pgfqpoint{2.097132in}{0.400812in}}%
\pgfpathlineto{\pgfqpoint{2.098421in}{0.438648in}}%
\pgfpathlineto{\pgfqpoint{2.101000in}{0.414094in}}%
\pgfpathlineto{\pgfqpoint{2.104867in}{0.400971in}}%
\pgfpathlineto{\pgfqpoint{2.106156in}{0.419638in}}%
\pgfpathlineto{\pgfqpoint{2.107446in}{0.403025in}}%
\pgfpathlineto{\pgfqpoint{2.108735in}{0.400328in}}%
\pgfpathlineto{\pgfqpoint{2.110024in}{0.405329in}}%
\pgfpathlineto{\pgfqpoint{2.113892in}{0.425158in}}%
\pgfpathlineto{\pgfqpoint{2.115181in}{0.410776in}}%
\pgfpathlineto{\pgfqpoint{2.116470in}{0.425012in}}%
\pgfpathlineto{\pgfqpoint{2.117759in}{0.409516in}}%
\pgfpathlineto{\pgfqpoint{2.119048in}{0.425260in}}%
\pgfpathlineto{\pgfqpoint{2.122916in}{0.412182in}}%
\pgfpathlineto{\pgfqpoint{2.124205in}{0.415148in}}%
\pgfpathlineto{\pgfqpoint{2.125494in}{0.453681in}}%
\pgfpathlineto{\pgfqpoint{2.126784in}{0.470433in}}%
\pgfpathlineto{\pgfqpoint{2.128073in}{0.447971in}}%
\pgfpathlineto{\pgfqpoint{2.131940in}{0.478122in}}%
\pgfpathlineto{\pgfqpoint{2.133230in}{0.405949in}}%
\pgfpathlineto{\pgfqpoint{2.134519in}{0.400271in}}%
\pgfpathlineto{\pgfqpoint{2.137097in}{0.401678in}}%
\pgfpathlineto{\pgfqpoint{2.143543in}{0.425182in}}%
\pgfpathlineto{\pgfqpoint{2.146122in}{0.400811in}}%
\pgfpathlineto{\pgfqpoint{2.149989in}{0.419874in}}%
\pgfpathlineto{\pgfqpoint{2.151278in}{0.454542in}}%
\pgfpathlineto{\pgfqpoint{2.152568in}{0.466016in}}%
\pgfpathlineto{\pgfqpoint{2.153857in}{0.452346in}}%
\pgfpathlineto{\pgfqpoint{2.155146in}{0.400799in}}%
\pgfpathlineto{\pgfqpoint{2.159014in}{0.403151in}}%
\pgfpathlineto{\pgfqpoint{2.160303in}{0.401470in}}%
\pgfpathlineto{\pgfqpoint{2.161592in}{0.402790in}}%
\pgfpathlineto{\pgfqpoint{2.162881in}{0.403217in}}%
\pgfpathlineto{\pgfqpoint{2.164170in}{0.408200in}}%
\pgfpathlineto{\pgfqpoint{2.170616in}{0.417635in}}%
\pgfpathlineto{\pgfqpoint{2.173195in}{0.423783in}}%
\pgfpathlineto{\pgfqpoint{2.177062in}{0.447765in}}%
\pgfpathlineto{\pgfqpoint{2.178351in}{0.725824in}}%
\pgfpathlineto{\pgfqpoint{2.179641in}{0.422460in}}%
\pgfpathlineto{\pgfqpoint{2.180930in}{0.424977in}}%
\pgfpathlineto{\pgfqpoint{2.182219in}{0.582205in}}%
\pgfpathlineto{\pgfqpoint{2.187376in}{0.400550in}}%
\pgfpathlineto{\pgfqpoint{2.188665in}{0.403201in}}%
\pgfpathlineto{\pgfqpoint{2.189954in}{0.415649in}}%
\pgfpathlineto{\pgfqpoint{2.191243in}{0.456295in}}%
\pgfpathlineto{\pgfqpoint{2.195111in}{0.418470in}}%
\pgfpathlineto{\pgfqpoint{2.196400in}{0.479109in}}%
\pgfpathlineto{\pgfqpoint{2.197689in}{0.403185in}}%
\pgfpathlineto{\pgfqpoint{2.198979in}{0.444550in}}%
\pgfpathlineto{\pgfqpoint{2.200268in}{0.407600in}}%
\pgfpathlineto{\pgfqpoint{2.205425in}{0.418050in}}%
\pgfpathlineto{\pgfqpoint{2.206714in}{0.427788in}}%
\pgfpathlineto{\pgfqpoint{2.208003in}{0.400870in}}%
\pgfpathlineto{\pgfqpoint{2.209292in}{0.405636in}}%
\pgfpathlineto{\pgfqpoint{2.213160in}{0.454775in}}%
\pgfpathlineto{\pgfqpoint{2.214449in}{0.454775in}}%
\pgfpathlineto{\pgfqpoint{2.217027in}{0.411806in}}%
\pgfpathlineto{\pgfqpoint{2.218317in}{0.421609in}}%
\pgfpathlineto{\pgfqpoint{2.222184in}{0.487021in}}%
\pgfpathlineto{\pgfqpoint{2.223473in}{0.400540in}}%
\pgfpathlineto{\pgfqpoint{2.224762in}{0.400288in}}%
\pgfpathlineto{\pgfqpoint{2.227341in}{0.438724in}}%
\pgfpathlineto{\pgfqpoint{2.231208in}{0.431926in}}%
\pgfpathlineto{\pgfqpoint{2.232498in}{0.552485in}}%
\pgfpathlineto{\pgfqpoint{2.233787in}{0.459909in}}%
\pgfpathlineto{\pgfqpoint{2.235076in}{0.756893in}}%
\pgfpathlineto{\pgfqpoint{2.236365in}{0.402747in}}%
\pgfpathlineto{\pgfqpoint{2.240233in}{0.401580in}}%
\pgfpathlineto{\pgfqpoint{2.242811in}{0.414212in}}%
\pgfpathlineto{\pgfqpoint{2.244100in}{0.403740in}}%
\pgfpathlineto{\pgfqpoint{2.245390in}{0.450889in}}%
\pgfpathlineto{\pgfqpoint{2.249257in}{0.402267in}}%
\pgfpathlineto{\pgfqpoint{2.250546in}{0.426568in}}%
\pgfpathlineto{\pgfqpoint{2.251836in}{0.532430in}}%
\pgfpathlineto{\pgfqpoint{2.253125in}{0.406535in}}%
\pgfpathlineto{\pgfqpoint{2.254414in}{0.978246in}}%
\pgfpathlineto{\pgfqpoint{2.258282in}{0.444873in}}%
\pgfpathlineto{\pgfqpoint{2.259571in}{0.405998in}}%
\pgfpathlineto{\pgfqpoint{2.260860in}{0.432642in}}%
\pgfpathlineto{\pgfqpoint{2.262149in}{0.400271in}}%
\pgfpathlineto{\pgfqpoint{2.267306in}{0.408423in}}%
\pgfpathlineto{\pgfqpoint{2.268595in}{0.408305in}}%
\pgfpathlineto{\pgfqpoint{2.269884in}{0.400590in}}%
\pgfpathlineto{\pgfqpoint{2.271174in}{0.400989in}}%
\pgfpathlineto{\pgfqpoint{2.272463in}{0.472955in}}%
\pgfpathlineto{\pgfqpoint{2.276330in}{0.406503in}}%
\pgfpathlineto{\pgfqpoint{2.277620in}{0.408873in}}%
\pgfpathlineto{\pgfqpoint{2.278909in}{0.662700in}}%
\pgfpathlineto{\pgfqpoint{2.280198in}{0.400434in}}%
\pgfpathlineto{\pgfqpoint{2.281487in}{0.422645in}}%
\pgfpathlineto{\pgfqpoint{2.285355in}{0.409973in}}%
\pgfpathlineto{\pgfqpoint{2.286644in}{0.412682in}}%
\pgfpathlineto{\pgfqpoint{2.287933in}{0.409994in}}%
\pgfpathlineto{\pgfqpoint{2.289222in}{0.416853in}}%
\pgfpathlineto{\pgfqpoint{2.290511in}{0.411086in}}%
\pgfpathlineto{\pgfqpoint{2.294379in}{0.425858in}}%
\pgfpathlineto{\pgfqpoint{2.295668in}{0.438691in}}%
\pgfpathlineto{\pgfqpoint{2.296957in}{0.402855in}}%
\pgfpathlineto{\pgfqpoint{2.298247in}{0.416658in}}%
\pgfpathlineto{\pgfqpoint{2.299536in}{0.506491in}}%
\pgfpathlineto{\pgfqpoint{2.303403in}{0.400289in}}%
\pgfpathlineto{\pgfqpoint{2.304693in}{0.442842in}}%
\pgfpathlineto{\pgfqpoint{2.307271in}{0.400290in}}%
\pgfpathlineto{\pgfqpoint{2.308560in}{0.445695in}}%
\pgfpathlineto{\pgfqpoint{2.312428in}{0.402063in}}%
\pgfpathlineto{\pgfqpoint{2.313717in}{0.427993in}}%
\pgfpathlineto{\pgfqpoint{2.316295in}{0.415285in}}%
\pgfpathlineto{\pgfqpoint{2.317585in}{0.400342in}}%
\pgfpathlineto{\pgfqpoint{2.321452in}{0.424134in}}%
\pgfpathlineto{\pgfqpoint{2.322741in}{0.409448in}}%
\pgfpathlineto{\pgfqpoint{2.326609in}{0.401658in}}%
\pgfpathlineto{\pgfqpoint{2.331766in}{0.416943in}}%
\pgfpathlineto{\pgfqpoint{2.333055in}{0.450467in}}%
\pgfpathlineto{\pgfqpoint{2.334344in}{0.411600in}}%
\pgfpathlineto{\pgfqpoint{2.335633in}{0.426505in}}%
\pgfpathlineto{\pgfqpoint{2.339501in}{0.438148in}}%
\pgfpathlineto{\pgfqpoint{2.340790in}{0.455583in}}%
\pgfpathlineto{\pgfqpoint{2.343369in}{0.425136in}}%
\pgfpathlineto{\pgfqpoint{2.344658in}{0.433206in}}%
\pgfpathlineto{\pgfqpoint{2.348525in}{0.444683in}}%
\pgfpathlineto{\pgfqpoint{2.349814in}{0.400586in}}%
\pgfpathlineto{\pgfqpoint{2.351104in}{0.448564in}}%
\pgfpathlineto{\pgfqpoint{2.352393in}{0.400442in}}%
\pgfpathlineto{\pgfqpoint{2.353682in}{0.442806in}}%
\pgfpathlineto{\pgfqpoint{2.357550in}{0.400975in}}%
\pgfpathlineto{\pgfqpoint{2.361417in}{0.427081in}}%
\pgfpathlineto{\pgfqpoint{2.362706in}{0.414776in}}%
\pgfpathlineto{\pgfqpoint{2.366574in}{0.405639in}}%
\pgfpathlineto{\pgfqpoint{2.367863in}{0.404447in}}%
\pgfpathlineto{\pgfqpoint{2.369152in}{0.405075in}}%
\pgfpathlineto{\pgfqpoint{2.370442in}{0.401194in}}%
\pgfpathlineto{\pgfqpoint{2.371731in}{0.543463in}}%
\pgfpathlineto{\pgfqpoint{2.375598in}{0.464187in}}%
\pgfpathlineto{\pgfqpoint{2.376888in}{0.400458in}}%
\pgfpathlineto{\pgfqpoint{2.378177in}{0.409498in}}%
\pgfpathlineto{\pgfqpoint{2.379466in}{0.422954in}}%
\pgfpathlineto{\pgfqpoint{2.384623in}{0.444560in}}%
\pgfpathlineto{\pgfqpoint{2.385912in}{0.403351in}}%
\pgfpathlineto{\pgfqpoint{2.388490in}{0.456546in}}%
\pgfpathlineto{\pgfqpoint{2.389780in}{0.410310in}}%
\pgfpathlineto{\pgfqpoint{2.393647in}{0.455645in}}%
\pgfpathlineto{\pgfqpoint{2.394936in}{0.401339in}}%
\pgfpathlineto{\pgfqpoint{2.396226in}{0.400468in}}%
\pgfpathlineto{\pgfqpoint{2.397515in}{0.408083in}}%
\pgfpathlineto{\pgfqpoint{2.398804in}{0.431746in}}%
\pgfpathlineto{\pgfqpoint{2.402672in}{0.424661in}}%
\pgfpathlineto{\pgfqpoint{2.403961in}{0.426843in}}%
\pgfpathlineto{\pgfqpoint{2.405250in}{0.402610in}}%
\pgfpathlineto{\pgfqpoint{2.406539in}{0.400295in}}%
\pgfpathlineto{\pgfqpoint{2.407828in}{0.453150in}}%
\pgfpathlineto{\pgfqpoint{2.411696in}{0.415364in}}%
\pgfpathlineto{\pgfqpoint{2.412985in}{0.468518in}}%
\pgfpathlineto{\pgfqpoint{2.414274in}{0.400843in}}%
\pgfpathlineto{\pgfqpoint{2.415563in}{0.402127in}}%
\pgfpathlineto{\pgfqpoint{2.416853in}{0.400478in}}%
\pgfpathlineto{\pgfqpoint{2.420720in}{0.402122in}}%
\pgfpathlineto{\pgfqpoint{2.422009in}{0.401384in}}%
\pgfpathlineto{\pgfqpoint{2.423299in}{0.410179in}}%
\pgfpathlineto{\pgfqpoint{2.424588in}{0.402508in}}%
\pgfpathlineto{\pgfqpoint{2.425877in}{0.403000in}}%
\pgfpathlineto{\pgfqpoint{2.429745in}{0.502305in}}%
\pgfpathlineto{\pgfqpoint{2.431034in}{0.479465in}}%
\pgfpathlineto{\pgfqpoint{2.432323in}{0.441209in}}%
\pgfpathlineto{\pgfqpoint{2.433612in}{0.460348in}}%
\pgfpathlineto{\pgfqpoint{2.434901in}{0.404083in}}%
\pgfpathlineto{\pgfqpoint{2.438769in}{0.400831in}}%
\pgfpathlineto{\pgfqpoint{2.440058in}{0.405330in}}%
\pgfpathlineto{\pgfqpoint{2.441347in}{0.455612in}}%
\pgfpathlineto{\pgfqpoint{2.443926in}{0.595838in}}%
\pgfpathlineto{\pgfqpoint{2.447793in}{0.421522in}}%
\pgfpathlineto{\pgfqpoint{2.449083in}{0.432445in}}%
\pgfpathlineto{\pgfqpoint{2.450372in}{0.838621in}}%
\pgfpathlineto{\pgfqpoint{2.451661in}{0.436327in}}%
\pgfpathlineto{\pgfqpoint{2.452950in}{0.492475in}}%
\pgfpathlineto{\pgfqpoint{2.456818in}{0.403091in}}%
\pgfpathlineto{\pgfqpoint{2.458107in}{0.497742in}}%
\pgfpathlineto{\pgfqpoint{2.459396in}{0.516600in}}%
\pgfpathlineto{\pgfqpoint{2.460685in}{0.440348in}}%
\pgfpathlineto{\pgfqpoint{2.461975in}{0.457247in}}%
\pgfpathlineto{\pgfqpoint{2.467131in}{0.574207in}}%
\pgfpathlineto{\pgfqpoint{2.468420in}{0.411859in}}%
\pgfpathlineto{\pgfqpoint{2.469710in}{0.400360in}}%
\pgfpathlineto{\pgfqpoint{2.470999in}{0.407371in}}%
\pgfpathlineto{\pgfqpoint{2.474866in}{0.401341in}}%
\pgfpathlineto{\pgfqpoint{2.476156in}{0.421052in}}%
\pgfpathlineto{\pgfqpoint{2.477445in}{0.400463in}}%
\pgfpathlineto{\pgfqpoint{2.478734in}{0.400805in}}%
\pgfpathlineto{\pgfqpoint{2.480023in}{0.484511in}}%
\pgfpathlineto{\pgfqpoint{2.483891in}{0.404045in}}%
\pgfpathlineto{\pgfqpoint{2.485180in}{0.444018in}}%
\pgfpathlineto{\pgfqpoint{2.486469in}{0.401097in}}%
\pgfpathlineto{\pgfqpoint{2.489048in}{0.419640in}}%
\pgfpathlineto{\pgfqpoint{2.492915in}{0.400636in}}%
\pgfpathlineto{\pgfqpoint{2.494204in}{0.441811in}}%
\pgfpathlineto{\pgfqpoint{2.495494in}{0.537487in}}%
\pgfpathlineto{\pgfqpoint{2.496783in}{0.403281in}}%
\pgfpathlineto{\pgfqpoint{2.498072in}{0.442078in}}%
\pgfpathlineto{\pgfqpoint{2.501940in}{0.479041in}}%
\pgfpathlineto{\pgfqpoint{2.503229in}{0.442414in}}%
\pgfpathlineto{\pgfqpoint{2.504518in}{0.450090in}}%
\pgfpathlineto{\pgfqpoint{2.505807in}{0.406097in}}%
\pgfpathlineto{\pgfqpoint{2.507096in}{0.421173in}}%
\pgfpathlineto{\pgfqpoint{2.510964in}{0.400929in}}%
\pgfpathlineto{\pgfqpoint{2.512253in}{0.404392in}}%
\pgfpathlineto{\pgfqpoint{2.513542in}{0.483424in}}%
\pgfpathlineto{\pgfqpoint{2.514832in}{0.400553in}}%
\pgfpathlineto{\pgfqpoint{2.516121in}{0.412074in}}%
\pgfpathlineto{\pgfqpoint{2.519988in}{0.441145in}}%
\pgfpathlineto{\pgfqpoint{2.521278in}{0.403569in}}%
\pgfpathlineto{\pgfqpoint{2.522567in}{0.400881in}}%
\pgfpathlineto{\pgfqpoint{2.523856in}{0.539614in}}%
\pgfpathlineto{\pgfqpoint{2.525145in}{0.432148in}}%
\pgfpathlineto{\pgfqpoint{2.529013in}{0.412545in}}%
\pgfpathlineto{\pgfqpoint{2.530302in}{0.401856in}}%
\pgfpathlineto{\pgfqpoint{2.531591in}{0.407236in}}%
\pgfpathlineto{\pgfqpoint{2.532880in}{0.459804in}}%
\pgfpathlineto{\pgfqpoint{2.534169in}{0.403962in}}%
\pgfpathlineto{\pgfqpoint{2.538037in}{0.408191in}}%
\pgfpathlineto{\pgfqpoint{2.539326in}{0.405504in}}%
\pgfpathlineto{\pgfqpoint{2.540615in}{0.401296in}}%
\pgfpathlineto{\pgfqpoint{2.541905in}{0.403416in}}%
\pgfpathlineto{\pgfqpoint{2.543194in}{0.403445in}}%
\pgfpathlineto{\pgfqpoint{2.547061in}{0.433704in}}%
\pgfpathlineto{\pgfqpoint{2.548351in}{0.402694in}}%
\pgfpathlineto{\pgfqpoint{2.550929in}{0.429692in}}%
\pgfpathlineto{\pgfqpoint{2.552218in}{0.412377in}}%
\pgfpathlineto{\pgfqpoint{2.556086in}{0.400289in}}%
\pgfpathlineto{\pgfqpoint{2.557375in}{0.441249in}}%
\pgfpathlineto{\pgfqpoint{2.558664in}{0.438309in}}%
\pgfpathlineto{\pgfqpoint{2.559953in}{0.574073in}}%
\pgfpathlineto{\pgfqpoint{2.561243in}{0.416323in}}%
\pgfpathlineto{\pgfqpoint{2.565110in}{0.404264in}}%
\pgfpathlineto{\pgfqpoint{2.566399in}{0.402175in}}%
\pgfpathlineto{\pgfqpoint{2.570267in}{0.400271in}}%
\pgfpathlineto{\pgfqpoint{2.574135in}{0.412466in}}%
\pgfpathlineto{\pgfqpoint{2.575424in}{0.413117in}}%
\pgfpathlineto{\pgfqpoint{2.576713in}{0.408336in}}%
\pgfpathlineto{\pgfqpoint{2.578002in}{0.479475in}}%
\pgfpathlineto{\pgfqpoint{2.579291in}{0.503028in}}%
\pgfpathlineto{\pgfqpoint{2.583159in}{0.400515in}}%
\pgfpathlineto{\pgfqpoint{2.584448in}{0.407029in}}%
\pgfpathlineto{\pgfqpoint{2.585737in}{0.400656in}}%
\pgfpathlineto{\pgfqpoint{2.587026in}{0.400518in}}%
\pgfpathlineto{\pgfqpoint{2.588316in}{0.431943in}}%
\pgfpathlineto{\pgfqpoint{2.592183in}{0.405383in}}%
\pgfpathlineto{\pgfqpoint{2.593472in}{0.463218in}}%
\pgfpathlineto{\pgfqpoint{2.594762in}{0.402087in}}%
\pgfpathlineto{\pgfqpoint{2.596051in}{0.418794in}}%
\pgfpathlineto{\pgfqpoint{2.597340in}{0.539515in}}%
\pgfpathlineto{\pgfqpoint{2.601208in}{0.418844in}}%
\pgfpathlineto{\pgfqpoint{2.602497in}{0.430600in}}%
\pgfpathlineto{\pgfqpoint{2.603786in}{0.409100in}}%
\pgfpathlineto{\pgfqpoint{2.605075in}{0.400332in}}%
\pgfpathlineto{\pgfqpoint{2.610232in}{0.400515in}}%
\pgfpathlineto{\pgfqpoint{2.611521in}{0.432115in}}%
\pgfpathlineto{\pgfqpoint{2.612810in}{0.425529in}}%
\pgfpathlineto{\pgfqpoint{2.614100in}{0.435882in}}%
\pgfpathlineto{\pgfqpoint{2.619256in}{0.428337in}}%
\pgfpathlineto{\pgfqpoint{2.620546in}{0.403447in}}%
\pgfpathlineto{\pgfqpoint{2.621835in}{0.477252in}}%
\pgfpathlineto{\pgfqpoint{2.623124in}{0.622194in}}%
\pgfpathlineto{\pgfqpoint{2.624413in}{0.415891in}}%
\pgfpathlineto{\pgfqpoint{2.629570in}{0.456047in}}%
\pgfpathlineto{\pgfqpoint{2.630859in}{0.485330in}}%
\pgfpathlineto{\pgfqpoint{2.632148in}{0.497937in}}%
\pgfpathlineto{\pgfqpoint{2.633438in}{1.772999in}}%
\pgfpathlineto{\pgfqpoint{2.638594in}{0.400607in}}%
\pgfpathlineto{\pgfqpoint{2.639884in}{0.407898in}}%
\pgfpathlineto{\pgfqpoint{2.641173in}{0.401036in}}%
\pgfpathlineto{\pgfqpoint{2.642462in}{0.412363in}}%
\pgfpathlineto{\pgfqpoint{2.647619in}{0.419193in}}%
\pgfpathlineto{\pgfqpoint{2.648908in}{0.402797in}}%
\pgfpathlineto{\pgfqpoint{2.650197in}{0.404358in}}%
\pgfpathlineto{\pgfqpoint{2.651486in}{0.581102in}}%
\pgfpathlineto{\pgfqpoint{2.655354in}{0.406584in}}%
\pgfpathlineto{\pgfqpoint{2.656643in}{0.571916in}}%
\pgfpathlineto{\pgfqpoint{2.657932in}{0.407898in}}%
\pgfpathlineto{\pgfqpoint{2.659221in}{0.433809in}}%
\pgfpathlineto{\pgfqpoint{2.660511in}{0.495326in}}%
\pgfpathlineto{\pgfqpoint{2.664378in}{0.409023in}}%
\pgfpathlineto{\pgfqpoint{2.665667in}{0.400293in}}%
\pgfpathlineto{\pgfqpoint{2.666957in}{0.461138in}}%
\pgfpathlineto{\pgfqpoint{2.668246in}{0.400294in}}%
\pgfpathlineto{\pgfqpoint{2.669535in}{0.432976in}}%
\pgfpathlineto{\pgfqpoint{2.674692in}{0.404024in}}%
\pgfpathlineto{\pgfqpoint{2.675981in}{0.483243in}}%
\pgfpathlineto{\pgfqpoint{2.677270in}{0.400609in}}%
\pgfpathlineto{\pgfqpoint{2.678559in}{0.491848in}}%
\pgfpathlineto{\pgfqpoint{2.682427in}{0.473378in}}%
\pgfpathlineto{\pgfqpoint{2.685005in}{0.428499in}}%
\pgfpathlineto{\pgfqpoint{2.686295in}{0.432476in}}%
\pgfpathlineto{\pgfqpoint{2.687584in}{0.405582in}}%
\pgfpathlineto{\pgfqpoint{2.691451in}{0.407768in}}%
\pgfpathlineto{\pgfqpoint{2.692741in}{0.502980in}}%
\pgfpathlineto{\pgfqpoint{2.694030in}{0.405324in}}%
\pgfpathlineto{\pgfqpoint{2.695319in}{0.400448in}}%
\pgfpathlineto{\pgfqpoint{2.696608in}{0.400760in}}%
\pgfpathlineto{\pgfqpoint{2.700476in}{0.415415in}}%
\pgfpathlineto{\pgfqpoint{2.701765in}{0.423994in}}%
\pgfpathlineto{\pgfqpoint{2.703054in}{0.416592in}}%
\pgfpathlineto{\pgfqpoint{2.704343in}{0.422208in}}%
\pgfpathlineto{\pgfqpoint{2.705632in}{0.439311in}}%
\pgfpathlineto{\pgfqpoint{2.712078in}{0.400563in}}%
\pgfpathlineto{\pgfqpoint{2.713368in}{0.412486in}}%
\pgfpathlineto{\pgfqpoint{2.714657in}{0.472035in}}%
\pgfpathlineto{\pgfqpoint{2.719814in}{0.400341in}}%
\pgfpathlineto{\pgfqpoint{2.721103in}{0.415162in}}%
\pgfpathlineto{\pgfqpoint{2.722392in}{0.402443in}}%
\pgfpathlineto{\pgfqpoint{2.727549in}{0.400343in}}%
\pgfpathlineto{\pgfqpoint{2.728838in}{0.434586in}}%
\pgfpathlineto{\pgfqpoint{2.730127in}{0.415826in}}%
\pgfpathlineto{\pgfqpoint{2.731416in}{0.419112in}}%
\pgfpathlineto{\pgfqpoint{2.732706in}{0.401684in}}%
\pgfpathlineto{\pgfqpoint{2.736573in}{0.429913in}}%
\pgfpathlineto{\pgfqpoint{2.737862in}{0.401724in}}%
\pgfpathlineto{\pgfqpoint{2.739152in}{0.404852in}}%
\pgfpathlineto{\pgfqpoint{2.740441in}{0.441960in}}%
\pgfpathlineto{\pgfqpoint{2.741730in}{0.401446in}}%
\pgfpathlineto{\pgfqpoint{2.745598in}{0.400436in}}%
\pgfpathlineto{\pgfqpoint{2.749465in}{0.416399in}}%
\pgfpathlineto{\pgfqpoint{2.750754in}{0.417860in}}%
\pgfpathlineto{\pgfqpoint{2.755911in}{0.400564in}}%
\pgfpathlineto{\pgfqpoint{2.757200in}{0.423737in}}%
\pgfpathlineto{\pgfqpoint{2.758490in}{0.400432in}}%
\pgfpathlineto{\pgfqpoint{2.759779in}{0.416567in}}%
\pgfpathlineto{\pgfqpoint{2.763646in}{0.409195in}}%
\pgfpathlineto{\pgfqpoint{2.764935in}{0.400290in}}%
\pgfpathlineto{\pgfqpoint{2.766225in}{0.417921in}}%
\pgfpathlineto{\pgfqpoint{2.768803in}{0.509601in}}%
\pgfpathlineto{\pgfqpoint{2.772671in}{0.418048in}}%
\pgfpathlineto{\pgfqpoint{2.773960in}{0.410694in}}%
\pgfpathlineto{\pgfqpoint{2.775249in}{0.410869in}}%
\pgfpathlineto{\pgfqpoint{2.776538in}{0.400775in}}%
\pgfpathlineto{\pgfqpoint{2.777827in}{0.403657in}}%
\pgfpathlineto{\pgfqpoint{2.781695in}{0.409999in}}%
\pgfpathlineto{\pgfqpoint{2.782984in}{0.419523in}}%
\pgfpathlineto{\pgfqpoint{2.784273in}{0.401543in}}%
\pgfpathlineto{\pgfqpoint{2.785563in}{0.414939in}}%
\pgfpathlineto{\pgfqpoint{2.786852in}{0.404234in}}%
\pgfpathlineto{\pgfqpoint{2.790719in}{0.438595in}}%
\pgfpathlineto{\pgfqpoint{2.792009in}{0.428793in}}%
\pgfpathlineto{\pgfqpoint{2.793298in}{0.400291in}}%
\pgfpathlineto{\pgfqpoint{2.795876in}{0.446682in}}%
\pgfpathlineto{\pgfqpoint{2.799744in}{0.401239in}}%
\pgfpathlineto{\pgfqpoint{2.801033in}{0.510983in}}%
\pgfpathlineto{\pgfqpoint{2.802322in}{0.416887in}}%
\pgfpathlineto{\pgfqpoint{2.803611in}{0.402092in}}%
\pgfpathlineto{\pgfqpoint{2.804901in}{0.401158in}}%
\pgfpathlineto{\pgfqpoint{2.810057in}{0.400344in}}%
\pgfpathlineto{\pgfqpoint{2.812636in}{0.401721in}}%
\pgfpathlineto{\pgfqpoint{2.813925in}{0.403301in}}%
\pgfpathlineto{\pgfqpoint{2.817793in}{0.400918in}}%
\pgfpathlineto{\pgfqpoint{2.819082in}{0.406045in}}%
\pgfpathlineto{\pgfqpoint{2.820371in}{0.400289in}}%
\pgfpathlineto{\pgfqpoint{2.826817in}{0.402786in}}%
\pgfpathlineto{\pgfqpoint{2.828106in}{0.400428in}}%
\pgfpathlineto{\pgfqpoint{2.829395in}{0.441080in}}%
\pgfpathlineto{\pgfqpoint{2.830684in}{0.401152in}}%
\pgfpathlineto{\pgfqpoint{2.831974in}{0.400915in}}%
\pgfpathlineto{\pgfqpoint{2.835841in}{0.425708in}}%
\pgfpathlineto{\pgfqpoint{2.837130in}{0.403663in}}%
\pgfpathlineto{\pgfqpoint{2.838420in}{0.400427in}}%
\pgfpathlineto{\pgfqpoint{2.839709in}{0.469468in}}%
\pgfpathlineto{\pgfqpoint{2.840998in}{0.701510in}}%
\pgfpathlineto{\pgfqpoint{2.846155in}{0.435001in}}%
\pgfpathlineto{\pgfqpoint{2.847444in}{0.483827in}}%
\pgfpathlineto{\pgfqpoint{2.848733in}{0.510226in}}%
\pgfpathlineto{\pgfqpoint{2.850022in}{0.400690in}}%
\pgfpathlineto{\pgfqpoint{2.855179in}{0.400877in}}%
\pgfpathlineto{\pgfqpoint{2.856468in}{0.411562in}}%
\pgfpathlineto{\pgfqpoint{2.857758in}{0.407523in}}%
\pgfpathlineto{\pgfqpoint{2.859047in}{0.487572in}}%
\pgfpathlineto{\pgfqpoint{2.862914in}{0.418070in}}%
\pgfpathlineto{\pgfqpoint{2.864204in}{0.440785in}}%
\pgfpathlineto{\pgfqpoint{2.865493in}{0.400801in}}%
\pgfpathlineto{\pgfqpoint{2.866782in}{0.405003in}}%
\pgfpathlineto{\pgfqpoint{2.868071in}{0.402370in}}%
\pgfpathlineto{\pgfqpoint{2.871939in}{0.400330in}}%
\pgfpathlineto{\pgfqpoint{2.873228in}{0.401455in}}%
\pgfpathlineto{\pgfqpoint{2.874517in}{0.436131in}}%
\pgfpathlineto{\pgfqpoint{2.875806in}{0.652745in}}%
\pgfpathlineto{\pgfqpoint{2.877096in}{0.419911in}}%
\pgfpathlineto{\pgfqpoint{2.880963in}{0.400406in}}%
\pgfpathlineto{\pgfqpoint{2.882252in}{0.420508in}}%
\pgfpathlineto{\pgfqpoint{2.884831in}{0.400806in}}%
\pgfpathlineto{\pgfqpoint{2.886120in}{0.401222in}}%
\pgfpathlineto{\pgfqpoint{2.889987in}{0.400271in}}%
\pgfpathlineto{\pgfqpoint{2.891277in}{0.411163in}}%
\pgfpathlineto{\pgfqpoint{2.892566in}{0.400649in}}%
\pgfpathlineto{\pgfqpoint{2.893855in}{0.414656in}}%
\pgfpathlineto{\pgfqpoint{2.895144in}{0.420332in}}%
\pgfpathlineto{\pgfqpoint{2.899012in}{0.400633in}}%
\pgfpathlineto{\pgfqpoint{2.900301in}{0.402025in}}%
\pgfpathlineto{\pgfqpoint{2.901590in}{0.419327in}}%
\pgfpathlineto{\pgfqpoint{2.902879in}{0.403173in}}%
\pgfpathlineto{\pgfqpoint{2.904169in}{0.401750in}}%
\pgfpathlineto{\pgfqpoint{2.908036in}{0.414388in}}%
\pgfpathlineto{\pgfqpoint{2.909325in}{0.411579in}}%
\pgfpathlineto{\pgfqpoint{2.911904in}{0.400503in}}%
\pgfpathlineto{\pgfqpoint{2.913193in}{0.409261in}}%
\pgfpathlineto{\pgfqpoint{2.918350in}{0.400498in}}%
\pgfpathlineto{\pgfqpoint{2.919639in}{0.407806in}}%
\pgfpathlineto{\pgfqpoint{2.920928in}{0.400789in}}%
\pgfpathlineto{\pgfqpoint{2.922217in}{0.403939in}}%
\pgfpathlineto{\pgfqpoint{2.926085in}{0.409832in}}%
\pgfpathlineto{\pgfqpoint{2.927374in}{0.403010in}}%
\pgfpathlineto{\pgfqpoint{2.928663in}{0.404754in}}%
\pgfpathlineto{\pgfqpoint{2.931242in}{0.400612in}}%
\pgfpathlineto{\pgfqpoint{2.936399in}{0.428700in}}%
\pgfpathlineto{\pgfqpoint{2.937688in}{0.401880in}}%
\pgfpathlineto{\pgfqpoint{2.938977in}{0.400285in}}%
\pgfpathlineto{\pgfqpoint{2.940266in}{0.518983in}}%
\pgfpathlineto{\pgfqpoint{2.946712in}{0.400285in}}%
\pgfpathlineto{\pgfqpoint{2.948001in}{0.503287in}}%
\pgfpathlineto{\pgfqpoint{2.949290in}{0.534177in}}%
\pgfpathlineto{\pgfqpoint{2.954447in}{0.400323in}}%
\pgfpathlineto{\pgfqpoint{2.955736in}{0.410991in}}%
\pgfpathlineto{\pgfqpoint{2.957026in}{0.401292in}}%
\pgfpathlineto{\pgfqpoint{2.958315in}{0.414095in}}%
\pgfpathlineto{\pgfqpoint{2.962182in}{0.432785in}}%
\pgfpathlineto{\pgfqpoint{2.963472in}{0.431506in}}%
\pgfpathlineto{\pgfqpoint{2.964761in}{0.407607in}}%
\pgfpathlineto{\pgfqpoint{2.966050in}{0.401806in}}%
\pgfpathlineto{\pgfqpoint{2.967339in}{0.420401in}}%
\pgfpathlineto{\pgfqpoint{2.971207in}{0.401281in}}%
\pgfpathlineto{\pgfqpoint{2.972496in}{0.401789in}}%
\pgfpathlineto{\pgfqpoint{2.973785in}{0.422198in}}%
\pgfpathlineto{\pgfqpoint{2.975074in}{0.400872in}}%
\pgfpathlineto{\pgfqpoint{2.976364in}{0.400381in}}%
\pgfpathlineto{\pgfqpoint{2.980231in}{0.400871in}}%
\pgfpathlineto{\pgfqpoint{2.981520in}{0.461567in}}%
\pgfpathlineto{\pgfqpoint{2.982810in}{0.402437in}}%
\pgfpathlineto{\pgfqpoint{2.984099in}{0.402802in}}%
\pgfpathlineto{\pgfqpoint{2.985388in}{0.425055in}}%
\pgfpathlineto{\pgfqpoint{2.989256in}{0.403128in}}%
\pgfpathlineto{\pgfqpoint{2.990545in}{0.423554in}}%
\pgfpathlineto{\pgfqpoint{2.991834in}{0.966633in}}%
\pgfpathlineto{\pgfqpoint{2.993123in}{0.401173in}}%
\pgfpathlineto{\pgfqpoint{2.994412in}{0.409167in}}%
\pgfpathlineto{\pgfqpoint{2.998280in}{0.401701in}}%
\pgfpathlineto{\pgfqpoint{3.000858in}{0.404451in}}%
\pgfpathlineto{\pgfqpoint{3.002148in}{0.401730in}}%
\pgfpathlineto{\pgfqpoint{3.003437in}{0.400799in}}%
\pgfpathlineto{\pgfqpoint{3.007304in}{0.402378in}}%
\pgfpathlineto{\pgfqpoint{3.008593in}{0.416309in}}%
\pgfpathlineto{\pgfqpoint{3.009883in}{0.401221in}}%
\pgfpathlineto{\pgfqpoint{3.011172in}{0.421906in}}%
\pgfpathlineto{\pgfqpoint{3.012461in}{0.414049in}}%
\pgfpathlineto{\pgfqpoint{3.016329in}{0.550025in}}%
\pgfpathlineto{\pgfqpoint{3.017618in}{0.400634in}}%
\pgfpathlineto{\pgfqpoint{3.018907in}{0.400286in}}%
\pgfpathlineto{\pgfqpoint{3.020196in}{0.407975in}}%
\pgfpathlineto{\pgfqpoint{3.021485in}{0.401733in}}%
\pgfpathlineto{\pgfqpoint{3.025353in}{0.402378in}}%
\pgfpathlineto{\pgfqpoint{3.026642in}{0.423478in}}%
\pgfpathlineto{\pgfqpoint{3.027931in}{0.400975in}}%
\pgfpathlineto{\pgfqpoint{3.029221in}{0.404408in}}%
\pgfpathlineto{\pgfqpoint{3.030510in}{0.400971in}}%
\pgfpathlineto{\pgfqpoint{3.034377in}{0.400400in}}%
\pgfpathlineto{\pgfqpoint{3.035667in}{0.431351in}}%
\pgfpathlineto{\pgfqpoint{3.036956in}{0.409722in}}%
\pgfpathlineto{\pgfqpoint{3.039534in}{0.407046in}}%
\pgfpathlineto{\pgfqpoint{3.043402in}{0.400951in}}%
\pgfpathlineto{\pgfqpoint{3.044691in}{0.405298in}}%
\pgfpathlineto{\pgfqpoint{3.045980in}{0.446575in}}%
\pgfpathlineto{\pgfqpoint{3.047269in}{0.513083in}}%
\pgfpathlineto{\pgfqpoint{3.048559in}{0.420997in}}%
\pgfpathlineto{\pgfqpoint{3.052426in}{0.406827in}}%
\pgfpathlineto{\pgfqpoint{3.053715in}{0.414327in}}%
\pgfpathlineto{\pgfqpoint{3.055005in}{0.475757in}}%
\pgfpathlineto{\pgfqpoint{3.056294in}{0.404735in}}%
\pgfpathlineto{\pgfqpoint{3.057583in}{0.400764in}}%
\pgfpathlineto{\pgfqpoint{3.061451in}{0.405172in}}%
\pgfpathlineto{\pgfqpoint{3.062740in}{0.480526in}}%
\pgfpathlineto{\pgfqpoint{3.064029in}{0.407749in}}%
\pgfpathlineto{\pgfqpoint{3.065318in}{0.407141in}}%
\pgfpathlineto{\pgfqpoint{3.066607in}{0.426739in}}%
\pgfpathlineto{\pgfqpoint{3.070475in}{0.438284in}}%
\pgfpathlineto{\pgfqpoint{3.071764in}{0.411719in}}%
\pgfpathlineto{\pgfqpoint{3.074342in}{0.400476in}}%
\pgfpathlineto{\pgfqpoint{3.075632in}{0.400386in}}%
\pgfpathlineto{\pgfqpoint{3.080788in}{0.401546in}}%
\pgfpathlineto{\pgfqpoint{3.082078in}{0.421885in}}%
\pgfpathlineto{\pgfqpoint{3.083367in}{0.400388in}}%
\pgfpathlineto{\pgfqpoint{3.084656in}{0.418249in}}%
\pgfpathlineto{\pgfqpoint{3.089813in}{0.412913in}}%
\pgfpathlineto{\pgfqpoint{3.091102in}{0.404517in}}%
\pgfpathlineto{\pgfqpoint{3.092391in}{0.400601in}}%
\pgfpathlineto{\pgfqpoint{3.093680in}{0.402167in}}%
\pgfpathlineto{\pgfqpoint{3.097548in}{0.402154in}}%
\pgfpathlineto{\pgfqpoint{3.098837in}{0.400911in}}%
\pgfpathlineto{\pgfqpoint{3.100126in}{0.419942in}}%
\pgfpathlineto{\pgfqpoint{3.101416in}{0.407081in}}%
\pgfpathlineto{\pgfqpoint{3.102705in}{0.401099in}}%
\pgfpathlineto{\pgfqpoint{3.109151in}{0.400478in}}%
\pgfpathlineto{\pgfqpoint{3.110440in}{0.404480in}}%
\pgfpathlineto{\pgfqpoint{3.111729in}{0.416105in}}%
\pgfpathlineto{\pgfqpoint{3.115597in}{0.403562in}}%
\pgfpathlineto{\pgfqpoint{3.116886in}{0.479044in}}%
\pgfpathlineto{\pgfqpoint{3.118175in}{0.403820in}}%
\pgfpathlineto{\pgfqpoint{3.119464in}{0.406778in}}%
\pgfpathlineto{\pgfqpoint{3.120754in}{0.418893in}}%
\pgfpathlineto{\pgfqpoint{3.125910in}{0.440024in}}%
\pgfpathlineto{\pgfqpoint{3.127199in}{0.409738in}}%
\pgfpathlineto{\pgfqpoint{3.128489in}{0.402826in}}%
\pgfpathlineto{\pgfqpoint{3.129778in}{0.401566in}}%
\pgfpathlineto{\pgfqpoint{3.133645in}{0.407758in}}%
\pgfpathlineto{\pgfqpoint{3.134935in}{0.401107in}}%
\pgfpathlineto{\pgfqpoint{3.136224in}{0.400389in}}%
\pgfpathlineto{\pgfqpoint{3.137513in}{0.501277in}}%
\pgfpathlineto{\pgfqpoint{3.138802in}{0.401934in}}%
\pgfpathlineto{\pgfqpoint{3.142670in}{0.425434in}}%
\pgfpathlineto{\pgfqpoint{3.143959in}{0.402533in}}%
\pgfpathlineto{\pgfqpoint{3.145248in}{0.401879in}}%
\pgfpathlineto{\pgfqpoint{3.146537in}{0.415429in}}%
\pgfpathlineto{\pgfqpoint{3.147827in}{0.400738in}}%
\pgfpathlineto{\pgfqpoint{3.152983in}{0.400478in}}%
\pgfpathlineto{\pgfqpoint{3.154273in}{0.423321in}}%
\pgfpathlineto{\pgfqpoint{3.155562in}{0.401590in}}%
\pgfpathlineto{\pgfqpoint{3.156851in}{0.414455in}}%
\pgfpathlineto{\pgfqpoint{3.160719in}{0.400323in}}%
\pgfpathlineto{\pgfqpoint{3.162008in}{0.411225in}}%
\pgfpathlineto{\pgfqpoint{3.163297in}{0.408542in}}%
\pgfpathlineto{\pgfqpoint{3.164586in}{0.400325in}}%
\pgfpathlineto{\pgfqpoint{3.165875in}{0.400285in}}%
\pgfpathlineto{\pgfqpoint{3.169743in}{0.413225in}}%
\pgfpathlineto{\pgfqpoint{3.171032in}{0.406233in}}%
\pgfpathlineto{\pgfqpoint{3.172321in}{0.404174in}}%
\pgfpathlineto{\pgfqpoint{3.173611in}{0.405143in}}%
\pgfpathlineto{\pgfqpoint{3.174900in}{0.401355in}}%
\pgfpathlineto{\pgfqpoint{3.178767in}{0.469001in}}%
\pgfpathlineto{\pgfqpoint{3.180057in}{0.400327in}}%
\pgfpathlineto{\pgfqpoint{3.181346in}{0.400954in}}%
\pgfpathlineto{\pgfqpoint{3.182635in}{0.400397in}}%
\pgfpathlineto{\pgfqpoint{3.183924in}{0.402619in}}%
\pgfpathlineto{\pgfqpoint{3.187792in}{0.403370in}}%
\pgfpathlineto{\pgfqpoint{3.189081in}{0.419253in}}%
\pgfpathlineto{\pgfqpoint{3.190370in}{0.413620in}}%
\pgfpathlineto{\pgfqpoint{3.191659in}{0.401389in}}%
\pgfpathlineto{\pgfqpoint{3.192948in}{0.401951in}}%
\pgfpathlineto{\pgfqpoint{3.196816in}{0.406968in}}%
\pgfpathlineto{\pgfqpoint{3.198105in}{0.405736in}}%
\pgfpathlineto{\pgfqpoint{3.199394in}{0.400394in}}%
\pgfpathlineto{\pgfqpoint{3.200684in}{0.404183in}}%
\pgfpathlineto{\pgfqpoint{3.201973in}{0.412284in}}%
\pgfpathlineto{\pgfqpoint{3.205840in}{0.401116in}}%
\pgfpathlineto{\pgfqpoint{3.207130in}{0.401859in}}%
\pgfpathlineto{\pgfqpoint{3.208419in}{0.400599in}}%
\pgfpathlineto{\pgfqpoint{3.209708in}{0.404545in}}%
\pgfpathlineto{\pgfqpoint{3.210997in}{0.400271in}}%
\pgfpathlineto{\pgfqpoint{3.214865in}{0.406731in}}%
\pgfpathlineto{\pgfqpoint{3.216154in}{0.400608in}}%
\pgfpathlineto{\pgfqpoint{3.217443in}{0.401907in}}%
\pgfpathlineto{\pgfqpoint{3.218732in}{0.417063in}}%
\pgfpathlineto{\pgfqpoint{3.223889in}{0.406342in}}%
\pgfpathlineto{\pgfqpoint{3.225178in}{0.410906in}}%
\pgfpathlineto{\pgfqpoint{3.226468in}{0.402536in}}%
\pgfpathlineto{\pgfqpoint{3.227757in}{0.408552in}}%
\pgfpathlineto{\pgfqpoint{3.229046in}{0.402485in}}%
\pgfpathlineto{\pgfqpoint{3.232914in}{0.420910in}}%
\pgfpathlineto{\pgfqpoint{3.234203in}{0.402101in}}%
\pgfpathlineto{\pgfqpoint{3.235492in}{0.400587in}}%
\pgfpathlineto{\pgfqpoint{3.236781in}{0.427785in}}%
\pgfpathlineto{\pgfqpoint{3.238070in}{0.583473in}}%
\pgfpathlineto{\pgfqpoint{3.241938in}{0.403223in}}%
\pgfpathlineto{\pgfqpoint{3.243227in}{0.449568in}}%
\pgfpathlineto{\pgfqpoint{3.244516in}{0.409382in}}%
\pgfpathlineto{\pgfqpoint{3.245805in}{0.402375in}}%
\pgfpathlineto{\pgfqpoint{3.247095in}{0.400321in}}%
\pgfpathlineto{\pgfqpoint{3.250962in}{0.409461in}}%
\pgfpathlineto{\pgfqpoint{3.252251in}{0.403539in}}%
\pgfpathlineto{\pgfqpoint{3.253541in}{0.415244in}}%
\pgfpathlineto{\pgfqpoint{3.254830in}{0.412149in}}%
\pgfpathlineto{\pgfqpoint{3.256119in}{0.403281in}}%
\pgfpathlineto{\pgfqpoint{3.259987in}{0.401357in}}%
\pgfpathlineto{\pgfqpoint{3.261276in}{0.404578in}}%
\pgfpathlineto{\pgfqpoint{3.262565in}{0.472278in}}%
\pgfpathlineto{\pgfqpoint{3.263854in}{0.404242in}}%
\pgfpathlineto{\pgfqpoint{3.265143in}{0.404202in}}%
\pgfpathlineto{\pgfqpoint{3.269011in}{0.415756in}}%
\pgfpathlineto{\pgfqpoint{3.270300in}{0.401343in}}%
\pgfpathlineto{\pgfqpoint{3.271589in}{0.408455in}}%
\pgfpathlineto{\pgfqpoint{3.272879in}{0.402460in}}%
\pgfpathlineto{\pgfqpoint{3.274168in}{0.400271in}}%
\pgfpathlineto{\pgfqpoint{3.279325in}{0.401099in}}%
\pgfpathlineto{\pgfqpoint{3.281903in}{0.400284in}}%
\pgfpathlineto{\pgfqpoint{3.283192in}{0.404460in}}%
\pgfpathlineto{\pgfqpoint{3.287060in}{0.400323in}}%
\pgfpathlineto{\pgfqpoint{3.288349in}{0.405440in}}%
\pgfpathlineto{\pgfqpoint{3.289638in}{0.402460in}}%
\pgfpathlineto{\pgfqpoint{3.290927in}{0.405925in}}%
\pgfpathlineto{\pgfqpoint{3.292217in}{0.469641in}}%
\pgfpathlineto{\pgfqpoint{3.296084in}{0.400325in}}%
\pgfpathlineto{\pgfqpoint{3.297373in}{0.402864in}}%
\pgfpathlineto{\pgfqpoint{3.298663in}{0.414759in}}%
\pgfpathlineto{\pgfqpoint{3.301241in}{0.401376in}}%
\pgfpathlineto{\pgfqpoint{3.305108in}{0.410906in}}%
\pgfpathlineto{\pgfqpoint{3.306398in}{0.451248in}}%
\pgfpathlineto{\pgfqpoint{3.307687in}{0.410528in}}%
\pgfpathlineto{\pgfqpoint{3.310265in}{0.403968in}}%
\pgfpathlineto{\pgfqpoint{3.314133in}{0.402033in}}%
\pgfpathlineto{\pgfqpoint{3.316711in}{0.440130in}}%
\pgfpathlineto{\pgfqpoint{3.318000in}{0.457113in}}%
\pgfpathlineto{\pgfqpoint{3.319290in}{0.405682in}}%
\pgfpathlineto{\pgfqpoint{3.323157in}{0.411225in}}%
\pgfpathlineto{\pgfqpoint{3.325736in}{0.501986in}}%
\pgfpathlineto{\pgfqpoint{3.327025in}{0.466218in}}%
\pgfpathlineto{\pgfqpoint{3.328314in}{0.408847in}}%
\pgfpathlineto{\pgfqpoint{3.332182in}{0.407473in}}%
\pgfpathlineto{\pgfqpoint{3.334760in}{0.414301in}}%
\pgfpathlineto{\pgfqpoint{3.336049in}{0.400286in}}%
\pgfpathlineto{\pgfqpoint{3.337338in}{0.425465in}}%
\pgfpathlineto{\pgfqpoint{3.342495in}{0.400628in}}%
\pgfpathlineto{\pgfqpoint{3.343784in}{0.400399in}}%
\pgfpathlineto{\pgfqpoint{3.345074in}{0.404848in}}%
\pgfpathlineto{\pgfqpoint{3.346363in}{0.400328in}}%
\pgfpathlineto{\pgfqpoint{3.350230in}{0.406514in}}%
\pgfpathlineto{\pgfqpoint{3.352809in}{0.400961in}}%
\pgfpathlineto{\pgfqpoint{3.354098in}{0.406427in}}%
\pgfpathlineto{\pgfqpoint{3.355387in}{0.414337in}}%
\pgfpathlineto{\pgfqpoint{3.359255in}{0.403319in}}%
\pgfpathlineto{\pgfqpoint{3.360544in}{0.490904in}}%
\pgfpathlineto{\pgfqpoint{3.361833in}{0.410255in}}%
\pgfpathlineto{\pgfqpoint{3.363122in}{0.401795in}}%
\pgfpathlineto{\pgfqpoint{3.364412in}{0.404359in}}%
\pgfpathlineto{\pgfqpoint{3.368279in}{0.402407in}}%
\pgfpathlineto{\pgfqpoint{3.369568in}{0.400322in}}%
\pgfpathlineto{\pgfqpoint{3.370857in}{0.403896in}}%
\pgfpathlineto{\pgfqpoint{3.372147in}{0.423635in}}%
\pgfpathlineto{\pgfqpoint{3.373436in}{0.408986in}}%
\pgfpathlineto{\pgfqpoint{3.377303in}{0.426233in}}%
\pgfpathlineto{\pgfqpoint{3.379882in}{0.404470in}}%
\pgfpathlineto{\pgfqpoint{3.381171in}{0.449630in}}%
\pgfpathlineto{\pgfqpoint{3.382460in}{0.403324in}}%
\pgfpathlineto{\pgfqpoint{3.386328in}{0.401378in}}%
\pgfpathlineto{\pgfqpoint{3.387617in}{0.409608in}}%
\pgfpathlineto{\pgfqpoint{3.388906in}{0.400285in}}%
\pgfpathlineto{\pgfqpoint{3.391485in}{0.400494in}}%
\pgfpathlineto{\pgfqpoint{3.395352in}{0.400327in}}%
\pgfpathlineto{\pgfqpoint{3.396641in}{0.401160in}}%
\pgfpathlineto{\pgfqpoint{3.397931in}{0.403374in}}%
\pgfpathlineto{\pgfqpoint{3.399220in}{0.404218in}}%
\pgfpathlineto{\pgfqpoint{3.400509in}{0.400326in}}%
\pgfpathlineto{\pgfqpoint{3.405666in}{0.400938in}}%
\pgfpathlineto{\pgfqpoint{3.406955in}{0.467539in}}%
\pgfpathlineto{\pgfqpoint{3.408244in}{0.406069in}}%
\pgfpathlineto{\pgfqpoint{3.413401in}{0.440421in}}%
\pgfpathlineto{\pgfqpoint{3.414690in}{0.411921in}}%
\pgfpathlineto{\pgfqpoint{3.417269in}{0.402743in}}%
\pgfpathlineto{\pgfqpoint{3.418558in}{0.431222in}}%
\pgfpathlineto{\pgfqpoint{3.422425in}{0.400271in}}%
\pgfpathlineto{\pgfqpoint{3.423715in}{0.406142in}}%
\pgfpathlineto{\pgfqpoint{3.425004in}{0.402995in}}%
\pgfpathlineto{\pgfqpoint{3.426293in}{0.402016in}}%
\pgfpathlineto{\pgfqpoint{3.427582in}{0.400320in}}%
\pgfpathlineto{\pgfqpoint{3.431450in}{0.400320in}}%
\pgfpathlineto{\pgfqpoint{3.432739in}{0.411071in}}%
\pgfpathlineto{\pgfqpoint{3.434028in}{0.400699in}}%
\pgfpathlineto{\pgfqpoint{3.435317in}{0.409496in}}%
\pgfpathlineto{\pgfqpoint{3.436606in}{0.406952in}}%
\pgfpathlineto{\pgfqpoint{3.440474in}{0.493332in}}%
\pgfpathlineto{\pgfqpoint{3.441763in}{0.411402in}}%
\pgfpathlineto{\pgfqpoint{3.443052in}{0.400444in}}%
\pgfpathlineto{\pgfqpoint{3.444342in}{0.404150in}}%
\pgfpathlineto{\pgfqpoint{3.445631in}{0.401135in}}%
\pgfpathlineto{\pgfqpoint{3.449498in}{0.405391in}}%
\pgfpathlineto{\pgfqpoint{3.450788in}{0.404500in}}%
\pgfpathlineto{\pgfqpoint{3.452077in}{0.411948in}}%
\pgfpathlineto{\pgfqpoint{3.453366in}{0.401584in}}%
\pgfpathlineto{\pgfqpoint{3.454655in}{0.423014in}}%
\pgfpathlineto{\pgfqpoint{3.458523in}{0.400949in}}%
\pgfpathlineto{\pgfqpoint{3.459812in}{0.400366in}}%
\pgfpathlineto{\pgfqpoint{3.461101in}{0.420472in}}%
\pgfpathlineto{\pgfqpoint{3.462390in}{0.402601in}}%
\pgfpathlineto{\pgfqpoint{3.463680in}{0.410828in}}%
\pgfpathlineto{\pgfqpoint{3.467547in}{0.414889in}}%
\pgfpathlineto{\pgfqpoint{3.468836in}{0.401710in}}%
\pgfpathlineto{\pgfqpoint{3.470126in}{0.403163in}}%
\pgfpathlineto{\pgfqpoint{3.471415in}{0.430248in}}%
\pgfpathlineto{\pgfqpoint{3.472704in}{1.165285in}}%
\pgfpathlineto{\pgfqpoint{3.476572in}{0.400348in}}%
\pgfpathlineto{\pgfqpoint{3.477861in}{0.495038in}}%
\pgfpathlineto{\pgfqpoint{3.479150in}{0.506061in}}%
\pgfpathlineto{\pgfqpoint{3.480439in}{0.410667in}}%
\pgfpathlineto{\pgfqpoint{3.481728in}{0.441069in}}%
\pgfpathlineto{\pgfqpoint{3.485596in}{0.428835in}}%
\pgfpathlineto{\pgfqpoint{3.486885in}{0.400755in}}%
\pgfpathlineto{\pgfqpoint{3.488174in}{0.400755in}}%
\pgfpathlineto{\pgfqpoint{3.489463in}{0.411299in}}%
\pgfpathlineto{\pgfqpoint{3.490753in}{0.437521in}}%
\pgfpathlineto{\pgfqpoint{3.494620in}{0.402299in}}%
\pgfpathlineto{\pgfqpoint{3.495909in}{0.401224in}}%
\pgfpathlineto{\pgfqpoint{3.497199in}{0.412319in}}%
\pgfpathlineto{\pgfqpoint{3.498488in}{0.402850in}}%
\pgfpathlineto{\pgfqpoint{3.499777in}{0.476537in}}%
\pgfpathlineto{\pgfqpoint{3.503645in}{0.400280in}}%
\pgfpathlineto{\pgfqpoint{3.504934in}{0.408184in}}%
\pgfpathlineto{\pgfqpoint{3.506223in}{0.406722in}}%
\pgfpathlineto{\pgfqpoint{3.508801in}{0.400941in}}%
\pgfpathlineto{\pgfqpoint{3.512669in}{0.405456in}}%
\pgfpathlineto{\pgfqpoint{3.513958in}{0.404662in}}%
\pgfpathlineto{\pgfqpoint{3.515247in}{0.446271in}}%
\pgfpathlineto{\pgfqpoint{3.516537in}{0.460807in}}%
\pgfpathlineto{\pgfqpoint{3.517826in}{0.402126in}}%
\pgfpathlineto{\pgfqpoint{3.521693in}{0.403270in}}%
\pgfpathlineto{\pgfqpoint{3.522983in}{0.485721in}}%
\pgfpathlineto{\pgfqpoint{3.524272in}{0.400280in}}%
\pgfpathlineto{\pgfqpoint{3.525561in}{0.411077in}}%
\pgfpathlineto{\pgfqpoint{3.526850in}{0.405798in}}%
\pgfpathlineto{\pgfqpoint{3.532007in}{0.409211in}}%
\pgfpathlineto{\pgfqpoint{3.533296in}{0.400280in}}%
\pgfpathlineto{\pgfqpoint{3.534585in}{0.400703in}}%
\pgfpathlineto{\pgfqpoint{3.535875in}{0.531978in}}%
\pgfpathlineto{\pgfqpoint{3.539742in}{0.613091in}}%
\pgfpathlineto{\pgfqpoint{3.541031in}{0.441829in}}%
\pgfpathlineto{\pgfqpoint{3.542321in}{0.418727in}}%
\pgfpathlineto{\pgfqpoint{3.543610in}{0.443159in}}%
\pgfpathlineto{\pgfqpoint{3.544899in}{0.400544in}}%
\pgfpathlineto{\pgfqpoint{3.550056in}{0.426981in}}%
\pgfpathlineto{\pgfqpoint{3.551345in}{0.400302in}}%
\pgfpathlineto{\pgfqpoint{3.552634in}{0.401209in}}%
\pgfpathlineto{\pgfqpoint{3.553923in}{0.400550in}}%
\pgfpathlineto{\pgfqpoint{3.553923in}{0.400550in}}%
\pgfusepath{stroke}%
\end{pgfscope}%
\begin{pgfscope}%
\pgfsetrectcap%
\pgfsetmiterjoin%
\pgfsetlinewidth{0.803000pt}%
\definecolor{currentstroke}{rgb}{1.000000,1.000000,1.000000}%
\pgfsetstrokecolor{currentstroke}%
\pgfsetdash{}{0pt}%
\pgfpathmoveto{\pgfqpoint{0.594832in}{0.331635in}}%
\pgfpathlineto{\pgfqpoint{0.594832in}{1.841635in}}%
\pgfusepath{stroke}%
\end{pgfscope}%
\begin{pgfscope}%
\pgfsetrectcap%
\pgfsetmiterjoin%
\pgfsetlinewidth{0.803000pt}%
\definecolor{currentstroke}{rgb}{1.000000,1.000000,1.000000}%
\pgfsetstrokecolor{currentstroke}%
\pgfsetdash{}{0pt}%
\pgfpathmoveto{\pgfqpoint{3.694832in}{0.331635in}}%
\pgfpathlineto{\pgfqpoint{3.694832in}{1.841635in}}%
\pgfusepath{stroke}%
\end{pgfscope}%
\begin{pgfscope}%
\pgfsetrectcap%
\pgfsetmiterjoin%
\pgfsetlinewidth{0.803000pt}%
\definecolor{currentstroke}{rgb}{1.000000,1.000000,1.000000}%
\pgfsetstrokecolor{currentstroke}%
\pgfsetdash{}{0pt}%
\pgfpathmoveto{\pgfqpoint{0.594832in}{0.331635in}}%
\pgfpathlineto{\pgfqpoint{3.694832in}{0.331635in}}%
\pgfusepath{stroke}%
\end{pgfscope}%
\begin{pgfscope}%
\pgfsetrectcap%
\pgfsetmiterjoin%
\pgfsetlinewidth{0.803000pt}%
\definecolor{currentstroke}{rgb}{1.000000,1.000000,1.000000}%
\pgfsetstrokecolor{currentstroke}%
\pgfsetdash{}{0pt}%
\pgfpathmoveto{\pgfqpoint{0.594832in}{1.841635in}}%
\pgfpathlineto{\pgfqpoint{3.694832in}{1.841635in}}%
\pgfusepath{stroke}%
\end{pgfscope}%
\end{pgfpicture}%
\makeatother%
\endgroup%

    \end{adjustbox}
    \hspace{3ex}
    \figuretitle{ACF and PACF of Squared log-returns V and INTC}
    \begin{adjustbox}{width=.95\textwidth,center}
    %% Creator: Matplotlib, PGF backend
%%
%% To include the figure in your LaTeX document, write
%%   \input{<filename>.pgf}
%%
%% Make sure the required packages are loaded in your preamble
%%   \usepackage{pgf}
%%
%% Figures using additional raster images can only be included by \input if
%% they are in the same directory as the main LaTeX file. For loading figures
%% from other directories you can use the `import` package
%%   \usepackage{import}
%% and then include the figures with
%%   \import{<path to file>}{<filename>.pgf}
%%
%% Matplotlib used the following preamble
%%   \usepackage{fontspec}
%%   \setmainfont{DejaVuSerif.ttf}[Path=/opt/tljh/user/lib/python3.6/site-packages/matplotlib/mpl-data/fonts/ttf/]
%%   \setsansfont{DejaVuSans.ttf}[Path=/opt/tljh/user/lib/python3.6/site-packages/matplotlib/mpl-data/fonts/ttf/]
%%   \setmonofont{DejaVuSansMono.ttf}[Path=/opt/tljh/user/lib/python3.6/site-packages/matplotlib/mpl-data/fonts/ttf/]
%%
\begingroup%
\makeatletter%
\begin{pgfpicture}%
\pgfpathrectangle{\pgfpointorigin}{\pgfqpoint{6.806467in}{2.151596in}}%
\pgfusepath{use as bounding box, clip}%
\begin{pgfscope}%
\pgfsetbuttcap%
\pgfsetmiterjoin%
\definecolor{currentfill}{rgb}{1.000000,1.000000,1.000000}%
\pgfsetfillcolor{currentfill}%
\pgfsetlinewidth{0.000000pt}%
\definecolor{currentstroke}{rgb}{1.000000,1.000000,1.000000}%
\pgfsetstrokecolor{currentstroke}%
\pgfsetdash{}{0pt}%
\pgfpathmoveto{\pgfqpoint{0.000000in}{0.000000in}}%
\pgfpathlineto{\pgfqpoint{6.806467in}{0.000000in}}%
\pgfpathlineto{\pgfqpoint{6.806467in}{2.151596in}}%
\pgfpathlineto{\pgfqpoint{0.000000in}{2.151596in}}%
\pgfpathclose%
\pgfusepath{fill}%
\end{pgfscope}%
\begin{pgfscope}%
\pgfsetbuttcap%
\pgfsetmiterjoin%
\definecolor{currentfill}{rgb}{0.917647,0.917647,0.949020}%
\pgfsetfillcolor{currentfill}%
\pgfsetlinewidth{0.000000pt}%
\definecolor{currentstroke}{rgb}{0.000000,0.000000,0.000000}%
\pgfsetstrokecolor{currentstroke}%
\pgfsetstrokeopacity{0.000000}%
\pgfsetdash{}{0pt}%
\pgfpathmoveto{\pgfqpoint{0.506467in}{0.331635in}}%
\pgfpathlineto{\pgfqpoint{3.089800in}{0.331635in}}%
\pgfpathlineto{\pgfqpoint{3.089800in}{1.841635in}}%
\pgfpathlineto{\pgfqpoint{0.506467in}{1.841635in}}%
\pgfpathclose%
\pgfusepath{fill}%
\end{pgfscope}%
\begin{pgfscope}%
\pgfpathrectangle{\pgfqpoint{0.506467in}{0.331635in}}{\pgfqpoint{2.583333in}{1.510000in}}%
\pgfusepath{clip}%
\pgfsetroundcap%
\pgfsetroundjoin%
\pgfsetlinewidth{0.803000pt}%
\definecolor{currentstroke}{rgb}{1.000000,1.000000,1.000000}%
\pgfsetstrokecolor{currentstroke}%
\pgfsetdash{}{0pt}%
\pgfpathmoveto{\pgfqpoint{0.623891in}{0.331635in}}%
\pgfpathlineto{\pgfqpoint{0.623891in}{1.841635in}}%
\pgfusepath{stroke}%
\end{pgfscope}%
\begin{pgfscope}%
\definecolor{textcolor}{rgb}{0.150000,0.150000,0.150000}%
\pgfsetstrokecolor{textcolor}%
\pgfsetfillcolor{textcolor}%
\pgftext[x=0.623891in,y=0.234413in,,top]{\color{textcolor}\rmfamily\fontsize{10.000000}{12.000000}\selectfont 0}%
\end{pgfscope}%
\begin{pgfscope}%
\pgfpathrectangle{\pgfqpoint{0.506467in}{0.331635in}}{\pgfqpoint{2.583333in}{1.510000in}}%
\pgfusepath{clip}%
\pgfsetroundcap%
\pgfsetroundjoin%
\pgfsetlinewidth{0.803000pt}%
\definecolor{currentstroke}{rgb}{1.000000,1.000000,1.000000}%
\pgfsetstrokecolor{currentstroke}%
\pgfsetdash{}{0pt}%
\pgfpathmoveto{\pgfqpoint{1.196693in}{0.331635in}}%
\pgfpathlineto{\pgfqpoint{1.196693in}{1.841635in}}%
\pgfusepath{stroke}%
\end{pgfscope}%
\begin{pgfscope}%
\definecolor{textcolor}{rgb}{0.150000,0.150000,0.150000}%
\pgfsetstrokecolor{textcolor}%
\pgfsetfillcolor{textcolor}%
\pgftext[x=1.196693in,y=0.234413in,,top]{\color{textcolor}\rmfamily\fontsize{10.000000}{12.000000}\selectfont 5}%
\end{pgfscope}%
\begin{pgfscope}%
\pgfpathrectangle{\pgfqpoint{0.506467in}{0.331635in}}{\pgfqpoint{2.583333in}{1.510000in}}%
\pgfusepath{clip}%
\pgfsetroundcap%
\pgfsetroundjoin%
\pgfsetlinewidth{0.803000pt}%
\definecolor{currentstroke}{rgb}{1.000000,1.000000,1.000000}%
\pgfsetstrokecolor{currentstroke}%
\pgfsetdash{}{0pt}%
\pgfpathmoveto{\pgfqpoint{1.769494in}{0.331635in}}%
\pgfpathlineto{\pgfqpoint{1.769494in}{1.841635in}}%
\pgfusepath{stroke}%
\end{pgfscope}%
\begin{pgfscope}%
\definecolor{textcolor}{rgb}{0.150000,0.150000,0.150000}%
\pgfsetstrokecolor{textcolor}%
\pgfsetfillcolor{textcolor}%
\pgftext[x=1.769494in,y=0.234413in,,top]{\color{textcolor}\rmfamily\fontsize{10.000000}{12.000000}\selectfont 10}%
\end{pgfscope}%
\begin{pgfscope}%
\pgfpathrectangle{\pgfqpoint{0.506467in}{0.331635in}}{\pgfqpoint{2.583333in}{1.510000in}}%
\pgfusepath{clip}%
\pgfsetroundcap%
\pgfsetroundjoin%
\pgfsetlinewidth{0.803000pt}%
\definecolor{currentstroke}{rgb}{1.000000,1.000000,1.000000}%
\pgfsetstrokecolor{currentstroke}%
\pgfsetdash{}{0pt}%
\pgfpathmoveto{\pgfqpoint{2.342295in}{0.331635in}}%
\pgfpathlineto{\pgfqpoint{2.342295in}{1.841635in}}%
\pgfusepath{stroke}%
\end{pgfscope}%
\begin{pgfscope}%
\definecolor{textcolor}{rgb}{0.150000,0.150000,0.150000}%
\pgfsetstrokecolor{textcolor}%
\pgfsetfillcolor{textcolor}%
\pgftext[x=2.342295in,y=0.234413in,,top]{\color{textcolor}\rmfamily\fontsize{10.000000}{12.000000}\selectfont 15}%
\end{pgfscope}%
\begin{pgfscope}%
\pgfpathrectangle{\pgfqpoint{0.506467in}{0.331635in}}{\pgfqpoint{2.583333in}{1.510000in}}%
\pgfusepath{clip}%
\pgfsetroundcap%
\pgfsetroundjoin%
\pgfsetlinewidth{0.803000pt}%
\definecolor{currentstroke}{rgb}{1.000000,1.000000,1.000000}%
\pgfsetstrokecolor{currentstroke}%
\pgfsetdash{}{0pt}%
\pgfpathmoveto{\pgfqpoint{2.915096in}{0.331635in}}%
\pgfpathlineto{\pgfqpoint{2.915096in}{1.841635in}}%
\pgfusepath{stroke}%
\end{pgfscope}%
\begin{pgfscope}%
\definecolor{textcolor}{rgb}{0.150000,0.150000,0.150000}%
\pgfsetstrokecolor{textcolor}%
\pgfsetfillcolor{textcolor}%
\pgftext[x=2.915096in,y=0.234413in,,top]{\color{textcolor}\rmfamily\fontsize{10.000000}{12.000000}\selectfont 20}%
\end{pgfscope}%
\begin{pgfscope}%
\pgfpathrectangle{\pgfqpoint{0.506467in}{0.331635in}}{\pgfqpoint{2.583333in}{1.510000in}}%
\pgfusepath{clip}%
\pgfsetroundcap%
\pgfsetroundjoin%
\pgfsetlinewidth{0.803000pt}%
\definecolor{currentstroke}{rgb}{1.000000,1.000000,1.000000}%
\pgfsetstrokecolor{currentstroke}%
\pgfsetdash{}{0pt}%
\pgfpathmoveto{\pgfqpoint{0.506467in}{0.467674in}}%
\pgfpathlineto{\pgfqpoint{3.089800in}{0.467674in}}%
\pgfusepath{stroke}%
\end{pgfscope}%
\begin{pgfscope}%
\definecolor{textcolor}{rgb}{0.150000,0.150000,0.150000}%
\pgfsetstrokecolor{textcolor}%
\pgfsetfillcolor{textcolor}%
\pgftext[x=0.100000in,y=0.414913in,left,base]{\color{textcolor}\rmfamily\fontsize{10.000000}{12.000000}\selectfont 0.00}%
\end{pgfscope}%
\begin{pgfscope}%
\pgfpathrectangle{\pgfqpoint{0.506467in}{0.331635in}}{\pgfqpoint{2.583333in}{1.510000in}}%
\pgfusepath{clip}%
\pgfsetroundcap%
\pgfsetroundjoin%
\pgfsetlinewidth{0.803000pt}%
\definecolor{currentstroke}{rgb}{1.000000,1.000000,1.000000}%
\pgfsetstrokecolor{currentstroke}%
\pgfsetdash{}{0pt}%
\pgfpathmoveto{\pgfqpoint{0.506467in}{0.794005in}}%
\pgfpathlineto{\pgfqpoint{3.089800in}{0.794005in}}%
\pgfusepath{stroke}%
\end{pgfscope}%
\begin{pgfscope}%
\definecolor{textcolor}{rgb}{0.150000,0.150000,0.150000}%
\pgfsetstrokecolor{textcolor}%
\pgfsetfillcolor{textcolor}%
\pgftext[x=0.100000in,y=0.741244in,left,base]{\color{textcolor}\rmfamily\fontsize{10.000000}{12.000000}\selectfont 0.25}%
\end{pgfscope}%
\begin{pgfscope}%
\pgfpathrectangle{\pgfqpoint{0.506467in}{0.331635in}}{\pgfqpoint{2.583333in}{1.510000in}}%
\pgfusepath{clip}%
\pgfsetroundcap%
\pgfsetroundjoin%
\pgfsetlinewidth{0.803000pt}%
\definecolor{currentstroke}{rgb}{1.000000,1.000000,1.000000}%
\pgfsetstrokecolor{currentstroke}%
\pgfsetdash{}{0pt}%
\pgfpathmoveto{\pgfqpoint{0.506467in}{1.120336in}}%
\pgfpathlineto{\pgfqpoint{3.089800in}{1.120336in}}%
\pgfusepath{stroke}%
\end{pgfscope}%
\begin{pgfscope}%
\definecolor{textcolor}{rgb}{0.150000,0.150000,0.150000}%
\pgfsetstrokecolor{textcolor}%
\pgfsetfillcolor{textcolor}%
\pgftext[x=0.100000in,y=1.067575in,left,base]{\color{textcolor}\rmfamily\fontsize{10.000000}{12.000000}\selectfont 0.50}%
\end{pgfscope}%
\begin{pgfscope}%
\pgfpathrectangle{\pgfqpoint{0.506467in}{0.331635in}}{\pgfqpoint{2.583333in}{1.510000in}}%
\pgfusepath{clip}%
\pgfsetroundcap%
\pgfsetroundjoin%
\pgfsetlinewidth{0.803000pt}%
\definecolor{currentstroke}{rgb}{1.000000,1.000000,1.000000}%
\pgfsetstrokecolor{currentstroke}%
\pgfsetdash{}{0pt}%
\pgfpathmoveto{\pgfqpoint{0.506467in}{1.446668in}}%
\pgfpathlineto{\pgfqpoint{3.089800in}{1.446668in}}%
\pgfusepath{stroke}%
\end{pgfscope}%
\begin{pgfscope}%
\definecolor{textcolor}{rgb}{0.150000,0.150000,0.150000}%
\pgfsetstrokecolor{textcolor}%
\pgfsetfillcolor{textcolor}%
\pgftext[x=0.100000in,y=1.393906in,left,base]{\color{textcolor}\rmfamily\fontsize{10.000000}{12.000000}\selectfont 0.75}%
\end{pgfscope}%
\begin{pgfscope}%
\pgfpathrectangle{\pgfqpoint{0.506467in}{0.331635in}}{\pgfqpoint{2.583333in}{1.510000in}}%
\pgfusepath{clip}%
\pgfsetroundcap%
\pgfsetroundjoin%
\pgfsetlinewidth{0.803000pt}%
\definecolor{currentstroke}{rgb}{1.000000,1.000000,1.000000}%
\pgfsetstrokecolor{currentstroke}%
\pgfsetdash{}{0pt}%
\pgfpathmoveto{\pgfqpoint{0.506467in}{1.772999in}}%
\pgfpathlineto{\pgfqpoint{3.089800in}{1.772999in}}%
\pgfusepath{stroke}%
\end{pgfscope}%
\begin{pgfscope}%
\definecolor{textcolor}{rgb}{0.150000,0.150000,0.150000}%
\pgfsetstrokecolor{textcolor}%
\pgfsetfillcolor{textcolor}%
\pgftext[x=0.100000in,y=1.720237in,left,base]{\color{textcolor}\rmfamily\fontsize{10.000000}{12.000000}\selectfont 1.00}%
\end{pgfscope}%
\begin{pgfscope}%
\pgfpathrectangle{\pgfqpoint{0.506467in}{0.331635in}}{\pgfqpoint{2.583333in}{1.510000in}}%
\pgfusepath{clip}%
\pgfsetbuttcap%
\pgfsetroundjoin%
\definecolor{currentfill}{rgb}{0.121569,0.466667,0.705882}%
\pgfsetfillcolor{currentfill}%
\pgfsetfillopacity{0.250000}%
\pgfsetlinewidth{1.003750pt}%
\definecolor{currentstroke}{rgb}{1.000000,1.000000,1.000000}%
\pgfsetstrokecolor{currentstroke}%
\pgfsetstrokeopacity{0.250000}%
\pgfsetdash{}{0pt}%
\pgfpathmoveto{\pgfqpoint{0.681171in}{0.533556in}}%
\pgfpathlineto{\pgfqpoint{0.681171in}{0.401792in}}%
\pgfpathlineto{\pgfqpoint{0.853012in}{0.401526in}}%
\pgfpathlineto{\pgfqpoint{0.967572in}{0.401418in}}%
\pgfpathlineto{\pgfqpoint{1.082132in}{0.401389in}}%
\pgfpathlineto{\pgfqpoint{1.196693in}{0.400960in}}%
\pgfpathlineto{\pgfqpoint{1.311253in}{0.400887in}}%
\pgfpathlineto{\pgfqpoint{1.425813in}{0.400590in}}%
\pgfpathlineto{\pgfqpoint{1.540373in}{0.400586in}}%
\pgfpathlineto{\pgfqpoint{1.654933in}{0.400586in}}%
\pgfpathlineto{\pgfqpoint{1.769494in}{0.400501in}}%
\pgfpathlineto{\pgfqpoint{1.884054in}{0.400498in}}%
\pgfpathlineto{\pgfqpoint{1.998614in}{0.400469in}}%
\pgfpathlineto{\pgfqpoint{2.113174in}{0.400468in}}%
\pgfpathlineto{\pgfqpoint{2.227735in}{0.400466in}}%
\pgfpathlineto{\pgfqpoint{2.342295in}{0.400466in}}%
\pgfpathlineto{\pgfqpoint{2.456855in}{0.400351in}}%
\pgfpathlineto{\pgfqpoint{2.571415in}{0.400351in}}%
\pgfpathlineto{\pgfqpoint{2.685976in}{0.400351in}}%
\pgfpathlineto{\pgfqpoint{2.800536in}{0.400272in}}%
\pgfpathlineto{\pgfqpoint{2.972376in}{0.400271in}}%
\pgfpathlineto{\pgfqpoint{2.972376in}{0.535077in}}%
\pgfpathlineto{\pgfqpoint{2.972376in}{0.535077in}}%
\pgfpathlineto{\pgfqpoint{2.800536in}{0.535076in}}%
\pgfpathlineto{\pgfqpoint{2.685976in}{0.534998in}}%
\pgfpathlineto{\pgfqpoint{2.571415in}{0.534997in}}%
\pgfpathlineto{\pgfqpoint{2.456855in}{0.534997in}}%
\pgfpathlineto{\pgfqpoint{2.342295in}{0.534883in}}%
\pgfpathlineto{\pgfqpoint{2.227735in}{0.534882in}}%
\pgfpathlineto{\pgfqpoint{2.113174in}{0.534880in}}%
\pgfpathlineto{\pgfqpoint{1.998614in}{0.534879in}}%
\pgfpathlineto{\pgfqpoint{1.884054in}{0.534850in}}%
\pgfpathlineto{\pgfqpoint{1.769494in}{0.534848in}}%
\pgfpathlineto{\pgfqpoint{1.654933in}{0.534763in}}%
\pgfpathlineto{\pgfqpoint{1.540373in}{0.534762in}}%
\pgfpathlineto{\pgfqpoint{1.425813in}{0.534759in}}%
\pgfpathlineto{\pgfqpoint{1.311253in}{0.534461in}}%
\pgfpathlineto{\pgfqpoint{1.196693in}{0.534389in}}%
\pgfpathlineto{\pgfqpoint{1.082132in}{0.533959in}}%
\pgfpathlineto{\pgfqpoint{0.967572in}{0.533930in}}%
\pgfpathlineto{\pgfqpoint{0.853012in}{0.533823in}}%
\pgfpathlineto{\pgfqpoint{0.681171in}{0.533556in}}%
\pgfpathclose%
\pgfusepath{stroke,fill}%
\end{pgfscope}%
\begin{pgfscope}%
\pgfpathrectangle{\pgfqpoint{0.506467in}{0.331635in}}{\pgfqpoint{2.583333in}{1.510000in}}%
\pgfusepath{clip}%
\pgfsetbuttcap%
\pgfsetroundjoin%
\pgfsetlinewidth{1.505625pt}%
\definecolor{currentstroke}{rgb}{0.000000,0.000000,0.000000}%
\pgfsetstrokecolor{currentstroke}%
\pgfsetdash{}{0pt}%
\pgfpathmoveto{\pgfqpoint{0.623891in}{0.467674in}}%
\pgfpathlineto{\pgfqpoint{0.623891in}{1.772999in}}%
\pgfusepath{stroke}%
\end{pgfscope}%
\begin{pgfscope}%
\pgfpathrectangle{\pgfqpoint{0.506467in}{0.331635in}}{\pgfqpoint{2.583333in}{1.510000in}}%
\pgfusepath{clip}%
\pgfsetbuttcap%
\pgfsetroundjoin%
\pgfsetlinewidth{1.505625pt}%
\definecolor{currentstroke}{rgb}{0.000000,0.000000,0.000000}%
\pgfsetstrokecolor{currentstroke}%
\pgfsetdash{}{0pt}%
\pgfpathmoveto{\pgfqpoint{0.738452in}{0.467674in}}%
\pgfpathlineto{\pgfqpoint{0.738452in}{0.550775in}}%
\pgfusepath{stroke}%
\end{pgfscope}%
\begin{pgfscope}%
\pgfpathrectangle{\pgfqpoint{0.506467in}{0.331635in}}{\pgfqpoint{2.583333in}{1.510000in}}%
\pgfusepath{clip}%
\pgfsetbuttcap%
\pgfsetroundjoin%
\pgfsetlinewidth{1.505625pt}%
\definecolor{currentstroke}{rgb}{0.000000,0.000000,0.000000}%
\pgfsetstrokecolor{currentstroke}%
\pgfsetdash{}{0pt}%
\pgfpathmoveto{\pgfqpoint{0.853012in}{0.467674in}}%
\pgfpathlineto{\pgfqpoint{0.853012in}{0.520600in}}%
\pgfusepath{stroke}%
\end{pgfscope}%
\begin{pgfscope}%
\pgfpathrectangle{\pgfqpoint{0.506467in}{0.331635in}}{\pgfqpoint{2.583333in}{1.510000in}}%
\pgfusepath{clip}%
\pgfsetbuttcap%
\pgfsetroundjoin%
\pgfsetlinewidth{1.505625pt}%
\definecolor{currentstroke}{rgb}{0.000000,0.000000,0.000000}%
\pgfsetstrokecolor{currentstroke}%
\pgfsetdash{}{0pt}%
\pgfpathmoveto{\pgfqpoint{0.967572in}{0.467674in}}%
\pgfpathlineto{\pgfqpoint{0.967572in}{0.495129in}}%
\pgfusepath{stroke}%
\end{pgfscope}%
\begin{pgfscope}%
\pgfpathrectangle{\pgfqpoint{0.506467in}{0.331635in}}{\pgfqpoint{2.583333in}{1.510000in}}%
\pgfusepath{clip}%
\pgfsetbuttcap%
\pgfsetroundjoin%
\pgfsetlinewidth{1.505625pt}%
\definecolor{currentstroke}{rgb}{0.000000,0.000000,0.000000}%
\pgfsetstrokecolor{currentstroke}%
\pgfsetdash{}{0pt}%
\pgfpathmoveto{\pgfqpoint{1.082132in}{0.467674in}}%
\pgfpathlineto{\pgfqpoint{1.082132in}{0.573568in}}%
\pgfusepath{stroke}%
\end{pgfscope}%
\begin{pgfscope}%
\pgfpathrectangle{\pgfqpoint{0.506467in}{0.331635in}}{\pgfqpoint{2.583333in}{1.510000in}}%
\pgfusepath{clip}%
\pgfsetbuttcap%
\pgfsetroundjoin%
\pgfsetlinewidth{1.505625pt}%
\definecolor{currentstroke}{rgb}{0.000000,0.000000,0.000000}%
\pgfsetstrokecolor{currentstroke}%
\pgfsetdash{}{0pt}%
\pgfpathmoveto{\pgfqpoint{1.196693in}{0.467674in}}%
\pgfpathlineto{\pgfqpoint{1.196693in}{0.511267in}}%
\pgfusepath{stroke}%
\end{pgfscope}%
\begin{pgfscope}%
\pgfpathrectangle{\pgfqpoint{0.506467in}{0.331635in}}{\pgfqpoint{2.583333in}{1.510000in}}%
\pgfusepath{clip}%
\pgfsetbuttcap%
\pgfsetroundjoin%
\pgfsetlinewidth{1.505625pt}%
\definecolor{currentstroke}{rgb}{0.000000,0.000000,0.000000}%
\pgfsetstrokecolor{currentstroke}%
\pgfsetdash{}{0pt}%
\pgfpathmoveto{\pgfqpoint{1.311253in}{0.467674in}}%
\pgfpathlineto{\pgfqpoint{1.311253in}{0.556033in}}%
\pgfusepath{stroke}%
\end{pgfscope}%
\begin{pgfscope}%
\pgfpathrectangle{\pgfqpoint{0.506467in}{0.331635in}}{\pgfqpoint{2.583333in}{1.510000in}}%
\pgfusepath{clip}%
\pgfsetbuttcap%
\pgfsetroundjoin%
\pgfsetlinewidth{1.505625pt}%
\definecolor{currentstroke}{rgb}{0.000000,0.000000,0.000000}%
\pgfsetstrokecolor{currentstroke}%
\pgfsetdash{}{0pt}%
\pgfpathmoveto{\pgfqpoint{1.425813in}{0.467674in}}%
\pgfpathlineto{\pgfqpoint{1.425813in}{0.477615in}}%
\pgfusepath{stroke}%
\end{pgfscope}%
\begin{pgfscope}%
\pgfpathrectangle{\pgfqpoint{0.506467in}{0.331635in}}{\pgfqpoint{2.583333in}{1.510000in}}%
\pgfusepath{clip}%
\pgfsetbuttcap%
\pgfsetroundjoin%
\pgfsetlinewidth{1.505625pt}%
\definecolor{currentstroke}{rgb}{0.000000,0.000000,0.000000}%
\pgfsetstrokecolor{currentstroke}%
\pgfsetdash{}{0pt}%
\pgfpathmoveto{\pgfqpoint{1.540373in}{0.467674in}}%
\pgfpathlineto{\pgfqpoint{1.540373in}{0.471763in}}%
\pgfusepath{stroke}%
\end{pgfscope}%
\begin{pgfscope}%
\pgfpathrectangle{\pgfqpoint{0.506467in}{0.331635in}}{\pgfqpoint{2.583333in}{1.510000in}}%
\pgfusepath{clip}%
\pgfsetbuttcap%
\pgfsetroundjoin%
\pgfsetlinewidth{1.505625pt}%
\definecolor{currentstroke}{rgb}{0.000000,0.000000,0.000000}%
\pgfsetstrokecolor{currentstroke}%
\pgfsetdash{}{0pt}%
\pgfpathmoveto{\pgfqpoint{1.654933in}{0.467674in}}%
\pgfpathlineto{\pgfqpoint{1.654933in}{0.514965in}}%
\pgfusepath{stroke}%
\end{pgfscope}%
\begin{pgfscope}%
\pgfpathrectangle{\pgfqpoint{0.506467in}{0.331635in}}{\pgfqpoint{2.583333in}{1.510000in}}%
\pgfusepath{clip}%
\pgfsetbuttcap%
\pgfsetroundjoin%
\pgfsetlinewidth{1.505625pt}%
\definecolor{currentstroke}{rgb}{0.000000,0.000000,0.000000}%
\pgfsetstrokecolor{currentstroke}%
\pgfsetdash{}{0pt}%
\pgfpathmoveto{\pgfqpoint{1.769494in}{0.467674in}}%
\pgfpathlineto{\pgfqpoint{1.769494in}{0.475836in}}%
\pgfusepath{stroke}%
\end{pgfscope}%
\begin{pgfscope}%
\pgfpathrectangle{\pgfqpoint{0.506467in}{0.331635in}}{\pgfqpoint{2.583333in}{1.510000in}}%
\pgfusepath{clip}%
\pgfsetbuttcap%
\pgfsetroundjoin%
\pgfsetlinewidth{1.505625pt}%
\definecolor{currentstroke}{rgb}{0.000000,0.000000,0.000000}%
\pgfsetstrokecolor{currentstroke}%
\pgfsetdash{}{0pt}%
\pgfpathmoveto{\pgfqpoint{1.884054in}{0.467674in}}%
\pgfpathlineto{\pgfqpoint{1.884054in}{0.495331in}}%
\pgfusepath{stroke}%
\end{pgfscope}%
\begin{pgfscope}%
\pgfpathrectangle{\pgfqpoint{0.506467in}{0.331635in}}{\pgfqpoint{2.583333in}{1.510000in}}%
\pgfusepath{clip}%
\pgfsetbuttcap%
\pgfsetroundjoin%
\pgfsetlinewidth{1.505625pt}%
\definecolor{currentstroke}{rgb}{0.000000,0.000000,0.000000}%
\pgfsetstrokecolor{currentstroke}%
\pgfsetdash{}{0pt}%
\pgfpathmoveto{\pgfqpoint{1.998614in}{0.467674in}}%
\pgfpathlineto{\pgfqpoint{1.998614in}{0.472993in}}%
\pgfusepath{stroke}%
\end{pgfscope}%
\begin{pgfscope}%
\pgfpathrectangle{\pgfqpoint{0.506467in}{0.331635in}}{\pgfqpoint{2.583333in}{1.510000in}}%
\pgfusepath{clip}%
\pgfsetbuttcap%
\pgfsetroundjoin%
\pgfsetlinewidth{1.505625pt}%
\definecolor{currentstroke}{rgb}{0.000000,0.000000,0.000000}%
\pgfsetstrokecolor{currentstroke}%
\pgfsetdash{}{0pt}%
\pgfpathmoveto{\pgfqpoint{2.113174in}{0.467674in}}%
\pgfpathlineto{\pgfqpoint{2.113174in}{0.460501in}}%
\pgfusepath{stroke}%
\end{pgfscope}%
\begin{pgfscope}%
\pgfpathrectangle{\pgfqpoint{0.506467in}{0.331635in}}{\pgfqpoint{2.583333in}{1.510000in}}%
\pgfusepath{clip}%
\pgfsetbuttcap%
\pgfsetroundjoin%
\pgfsetlinewidth{1.505625pt}%
\definecolor{currentstroke}{rgb}{0.000000,0.000000,0.000000}%
\pgfsetstrokecolor{currentstroke}%
\pgfsetdash{}{0pt}%
\pgfpathmoveto{\pgfqpoint{2.227735in}{0.467674in}}%
\pgfpathlineto{\pgfqpoint{2.227735in}{0.471877in}}%
\pgfusepath{stroke}%
\end{pgfscope}%
\begin{pgfscope}%
\pgfpathrectangle{\pgfqpoint{0.506467in}{0.331635in}}{\pgfqpoint{2.583333in}{1.510000in}}%
\pgfusepath{clip}%
\pgfsetbuttcap%
\pgfsetroundjoin%
\pgfsetlinewidth{1.505625pt}%
\definecolor{currentstroke}{rgb}{0.000000,0.000000,0.000000}%
\pgfsetstrokecolor{currentstroke}%
\pgfsetdash{}{0pt}%
\pgfpathmoveto{\pgfqpoint{2.342295in}{0.467674in}}%
\pgfpathlineto{\pgfqpoint{2.342295in}{0.522638in}}%
\pgfusepath{stroke}%
\end{pgfscope}%
\begin{pgfscope}%
\pgfpathrectangle{\pgfqpoint{0.506467in}{0.331635in}}{\pgfqpoint{2.583333in}{1.510000in}}%
\pgfusepath{clip}%
\pgfsetbuttcap%
\pgfsetroundjoin%
\pgfsetlinewidth{1.505625pt}%
\definecolor{currentstroke}{rgb}{0.000000,0.000000,0.000000}%
\pgfsetstrokecolor{currentstroke}%
\pgfsetdash{}{0pt}%
\pgfpathmoveto{\pgfqpoint{2.456855in}{0.467674in}}%
\pgfpathlineto{\pgfqpoint{2.456855in}{0.467509in}}%
\pgfusepath{stroke}%
\end{pgfscope}%
\begin{pgfscope}%
\pgfpathrectangle{\pgfqpoint{0.506467in}{0.331635in}}{\pgfqpoint{2.583333in}{1.510000in}}%
\pgfusepath{clip}%
\pgfsetbuttcap%
\pgfsetroundjoin%
\pgfsetlinewidth{1.505625pt}%
\definecolor{currentstroke}{rgb}{0.000000,0.000000,0.000000}%
\pgfsetstrokecolor{currentstroke}%
\pgfsetdash{}{0pt}%
\pgfpathmoveto{\pgfqpoint{2.571415in}{0.467674in}}%
\pgfpathlineto{\pgfqpoint{2.571415in}{0.465208in}}%
\pgfusepath{stroke}%
\end{pgfscope}%
\begin{pgfscope}%
\pgfpathrectangle{\pgfqpoint{0.506467in}{0.331635in}}{\pgfqpoint{2.583333in}{1.510000in}}%
\pgfusepath{clip}%
\pgfsetbuttcap%
\pgfsetroundjoin%
\pgfsetlinewidth{1.505625pt}%
\definecolor{currentstroke}{rgb}{0.000000,0.000000,0.000000}%
\pgfsetstrokecolor{currentstroke}%
\pgfsetdash{}{0pt}%
\pgfpathmoveto{\pgfqpoint{2.685976in}{0.467674in}}%
\pgfpathlineto{\pgfqpoint{2.685976in}{0.513217in}}%
\pgfusepath{stroke}%
\end{pgfscope}%
\begin{pgfscope}%
\pgfpathrectangle{\pgfqpoint{0.506467in}{0.331635in}}{\pgfqpoint{2.583333in}{1.510000in}}%
\pgfusepath{clip}%
\pgfsetbuttcap%
\pgfsetroundjoin%
\pgfsetlinewidth{1.505625pt}%
\definecolor{currentstroke}{rgb}{0.000000,0.000000,0.000000}%
\pgfsetstrokecolor{currentstroke}%
\pgfsetdash{}{0pt}%
\pgfpathmoveto{\pgfqpoint{2.800536in}{0.467674in}}%
\pgfpathlineto{\pgfqpoint{2.800536in}{0.473129in}}%
\pgfusepath{stroke}%
\end{pgfscope}%
\begin{pgfscope}%
\pgfpathrectangle{\pgfqpoint{0.506467in}{0.331635in}}{\pgfqpoint{2.583333in}{1.510000in}}%
\pgfusepath{clip}%
\pgfsetbuttcap%
\pgfsetroundjoin%
\pgfsetlinewidth{1.505625pt}%
\definecolor{currentstroke}{rgb}{0.000000,0.000000,0.000000}%
\pgfsetstrokecolor{currentstroke}%
\pgfsetdash{}{0pt}%
\pgfpathmoveto{\pgfqpoint{2.915096in}{0.467674in}}%
\pgfpathlineto{\pgfqpoint{2.915096in}{0.457582in}}%
\pgfusepath{stroke}%
\end{pgfscope}%
\begin{pgfscope}%
\pgfpathrectangle{\pgfqpoint{0.506467in}{0.331635in}}{\pgfqpoint{2.583333in}{1.510000in}}%
\pgfusepath{clip}%
\pgfsetroundcap%
\pgfsetroundjoin%
\pgfsetlinewidth{1.505625pt}%
\definecolor{currentstroke}{rgb}{0.737255,0.741176,0.133333}%
\pgfsetstrokecolor{currentstroke}%
\pgfsetdash{}{0pt}%
\pgfpathmoveto{\pgfqpoint{0.506467in}{0.467674in}}%
\pgfpathlineto{\pgfqpoint{3.089800in}{0.467674in}}%
\pgfusepath{stroke}%
\end{pgfscope}%
\begin{pgfscope}%
\pgfpathrectangle{\pgfqpoint{0.506467in}{0.331635in}}{\pgfqpoint{2.583333in}{1.510000in}}%
\pgfusepath{clip}%
\pgfsetbuttcap%
\pgfsetroundjoin%
\definecolor{currentfill}{rgb}{0.737255,0.741176,0.133333}%
\pgfsetfillcolor{currentfill}%
\pgfsetlinewidth{1.003750pt}%
\definecolor{currentstroke}{rgb}{0.737255,0.741176,0.133333}%
\pgfsetstrokecolor{currentstroke}%
\pgfsetdash{}{0pt}%
\pgfsys@defobject{currentmarker}{\pgfqpoint{-0.034722in}{-0.034722in}}{\pgfqpoint{0.034722in}{0.034722in}}{%
\pgfpathmoveto{\pgfqpoint{0.000000in}{-0.034722in}}%
\pgfpathcurveto{\pgfqpoint{0.009208in}{-0.034722in}}{\pgfqpoint{0.018041in}{-0.031064in}}{\pgfqpoint{0.024552in}{-0.024552in}}%
\pgfpathcurveto{\pgfqpoint{0.031064in}{-0.018041in}}{\pgfqpoint{0.034722in}{-0.009208in}}{\pgfqpoint{0.034722in}{0.000000in}}%
\pgfpathcurveto{\pgfqpoint{0.034722in}{0.009208in}}{\pgfqpoint{0.031064in}{0.018041in}}{\pgfqpoint{0.024552in}{0.024552in}}%
\pgfpathcurveto{\pgfqpoint{0.018041in}{0.031064in}}{\pgfqpoint{0.009208in}{0.034722in}}{\pgfqpoint{0.000000in}{0.034722in}}%
\pgfpathcurveto{\pgfqpoint{-0.009208in}{0.034722in}}{\pgfqpoint{-0.018041in}{0.031064in}}{\pgfqpoint{-0.024552in}{0.024552in}}%
\pgfpathcurveto{\pgfqpoint{-0.031064in}{0.018041in}}{\pgfqpoint{-0.034722in}{0.009208in}}{\pgfqpoint{-0.034722in}{0.000000in}}%
\pgfpathcurveto{\pgfqpoint{-0.034722in}{-0.009208in}}{\pgfqpoint{-0.031064in}{-0.018041in}}{\pgfqpoint{-0.024552in}{-0.024552in}}%
\pgfpathcurveto{\pgfqpoint{-0.018041in}{-0.031064in}}{\pgfqpoint{-0.009208in}{-0.034722in}}{\pgfqpoint{0.000000in}{-0.034722in}}%
\pgfpathclose%
\pgfusepath{stroke,fill}%
}%
\begin{pgfscope}%
\pgfsys@transformshift{0.623891in}{1.772999in}%
\pgfsys@useobject{currentmarker}{}%
\end{pgfscope}%
\begin{pgfscope}%
\pgfsys@transformshift{0.738452in}{0.550775in}%
\pgfsys@useobject{currentmarker}{}%
\end{pgfscope}%
\begin{pgfscope}%
\pgfsys@transformshift{0.853012in}{0.520600in}%
\pgfsys@useobject{currentmarker}{}%
\end{pgfscope}%
\begin{pgfscope}%
\pgfsys@transformshift{0.967572in}{0.495129in}%
\pgfsys@useobject{currentmarker}{}%
\end{pgfscope}%
\begin{pgfscope}%
\pgfsys@transformshift{1.082132in}{0.573568in}%
\pgfsys@useobject{currentmarker}{}%
\end{pgfscope}%
\begin{pgfscope}%
\pgfsys@transformshift{1.196693in}{0.511267in}%
\pgfsys@useobject{currentmarker}{}%
\end{pgfscope}%
\begin{pgfscope}%
\pgfsys@transformshift{1.311253in}{0.556033in}%
\pgfsys@useobject{currentmarker}{}%
\end{pgfscope}%
\begin{pgfscope}%
\pgfsys@transformshift{1.425813in}{0.477615in}%
\pgfsys@useobject{currentmarker}{}%
\end{pgfscope}%
\begin{pgfscope}%
\pgfsys@transformshift{1.540373in}{0.471763in}%
\pgfsys@useobject{currentmarker}{}%
\end{pgfscope}%
\begin{pgfscope}%
\pgfsys@transformshift{1.654933in}{0.514965in}%
\pgfsys@useobject{currentmarker}{}%
\end{pgfscope}%
\begin{pgfscope}%
\pgfsys@transformshift{1.769494in}{0.475836in}%
\pgfsys@useobject{currentmarker}{}%
\end{pgfscope}%
\begin{pgfscope}%
\pgfsys@transformshift{1.884054in}{0.495331in}%
\pgfsys@useobject{currentmarker}{}%
\end{pgfscope}%
\begin{pgfscope}%
\pgfsys@transformshift{1.998614in}{0.472993in}%
\pgfsys@useobject{currentmarker}{}%
\end{pgfscope}%
\begin{pgfscope}%
\pgfsys@transformshift{2.113174in}{0.460501in}%
\pgfsys@useobject{currentmarker}{}%
\end{pgfscope}%
\begin{pgfscope}%
\pgfsys@transformshift{2.227735in}{0.471877in}%
\pgfsys@useobject{currentmarker}{}%
\end{pgfscope}%
\begin{pgfscope}%
\pgfsys@transformshift{2.342295in}{0.522638in}%
\pgfsys@useobject{currentmarker}{}%
\end{pgfscope}%
\begin{pgfscope}%
\pgfsys@transformshift{2.456855in}{0.467509in}%
\pgfsys@useobject{currentmarker}{}%
\end{pgfscope}%
\begin{pgfscope}%
\pgfsys@transformshift{2.571415in}{0.465208in}%
\pgfsys@useobject{currentmarker}{}%
\end{pgfscope}%
\begin{pgfscope}%
\pgfsys@transformshift{2.685976in}{0.513217in}%
\pgfsys@useobject{currentmarker}{}%
\end{pgfscope}%
\begin{pgfscope}%
\pgfsys@transformshift{2.800536in}{0.473129in}%
\pgfsys@useobject{currentmarker}{}%
\end{pgfscope}%
\begin{pgfscope}%
\pgfsys@transformshift{2.915096in}{0.457582in}%
\pgfsys@useobject{currentmarker}{}%
\end{pgfscope}%
\end{pgfscope}%
\begin{pgfscope}%
\pgfsetrectcap%
\pgfsetmiterjoin%
\pgfsetlinewidth{0.803000pt}%
\definecolor{currentstroke}{rgb}{1.000000,1.000000,1.000000}%
\pgfsetstrokecolor{currentstroke}%
\pgfsetdash{}{0pt}%
\pgfpathmoveto{\pgfqpoint{0.506467in}{0.331635in}}%
\pgfpathlineto{\pgfqpoint{0.506467in}{1.841635in}}%
\pgfusepath{stroke}%
\end{pgfscope}%
\begin{pgfscope}%
\pgfsetrectcap%
\pgfsetmiterjoin%
\pgfsetlinewidth{0.803000pt}%
\definecolor{currentstroke}{rgb}{1.000000,1.000000,1.000000}%
\pgfsetstrokecolor{currentstroke}%
\pgfsetdash{}{0pt}%
\pgfpathmoveto{\pgfqpoint{3.089800in}{0.331635in}}%
\pgfpathlineto{\pgfqpoint{3.089800in}{1.841635in}}%
\pgfusepath{stroke}%
\end{pgfscope}%
\begin{pgfscope}%
\pgfsetrectcap%
\pgfsetmiterjoin%
\pgfsetlinewidth{0.803000pt}%
\definecolor{currentstroke}{rgb}{1.000000,1.000000,1.000000}%
\pgfsetstrokecolor{currentstroke}%
\pgfsetdash{}{0pt}%
\pgfpathmoveto{\pgfqpoint{0.506467in}{0.331635in}}%
\pgfpathlineto{\pgfqpoint{3.089800in}{0.331635in}}%
\pgfusepath{stroke}%
\end{pgfscope}%
\begin{pgfscope}%
\pgfsetrectcap%
\pgfsetmiterjoin%
\pgfsetlinewidth{0.803000pt}%
\definecolor{currentstroke}{rgb}{1.000000,1.000000,1.000000}%
\pgfsetstrokecolor{currentstroke}%
\pgfsetdash{}{0pt}%
\pgfpathmoveto{\pgfqpoint{0.506467in}{1.841635in}}%
\pgfpathlineto{\pgfqpoint{3.089800in}{1.841635in}}%
\pgfusepath{stroke}%
\end{pgfscope}%
\begin{pgfscope}%
\definecolor{textcolor}{rgb}{0.150000,0.150000,0.150000}%
\pgfsetstrokecolor{textcolor}%
\pgfsetfillcolor{textcolor}%
\pgftext[x=1.798134in,y=1.924968in,,base]{\color{textcolor}\rmfamily\fontsize{12.000000}{14.400000}\selectfont Autocorrelation V}%
\end{pgfscope}%
\begin{pgfscope}%
\pgfsetbuttcap%
\pgfsetmiterjoin%
\definecolor{currentfill}{rgb}{0.917647,0.917647,0.949020}%
\pgfsetfillcolor{currentfill}%
\pgfsetlinewidth{0.000000pt}%
\definecolor{currentstroke}{rgb}{0.000000,0.000000,0.000000}%
\pgfsetstrokecolor{currentstroke}%
\pgfsetstrokeopacity{0.000000}%
\pgfsetdash{}{0pt}%
\pgfpathmoveto{\pgfqpoint{4.123134in}{0.331635in}}%
\pgfpathlineto{\pgfqpoint{6.706467in}{0.331635in}}%
\pgfpathlineto{\pgfqpoint{6.706467in}{1.841635in}}%
\pgfpathlineto{\pgfqpoint{4.123134in}{1.841635in}}%
\pgfpathclose%
\pgfusepath{fill}%
\end{pgfscope}%
\begin{pgfscope}%
\pgfpathrectangle{\pgfqpoint{4.123134in}{0.331635in}}{\pgfqpoint{2.583333in}{1.510000in}}%
\pgfusepath{clip}%
\pgfsetroundcap%
\pgfsetroundjoin%
\pgfsetlinewidth{0.803000pt}%
\definecolor{currentstroke}{rgb}{1.000000,1.000000,1.000000}%
\pgfsetstrokecolor{currentstroke}%
\pgfsetdash{}{0pt}%
\pgfpathmoveto{\pgfqpoint{4.240558in}{0.331635in}}%
\pgfpathlineto{\pgfqpoint{4.240558in}{1.841635in}}%
\pgfusepath{stroke}%
\end{pgfscope}%
\begin{pgfscope}%
\definecolor{textcolor}{rgb}{0.150000,0.150000,0.150000}%
\pgfsetstrokecolor{textcolor}%
\pgfsetfillcolor{textcolor}%
\pgftext[x=4.240558in,y=0.234413in,,top]{\color{textcolor}\rmfamily\fontsize{10.000000}{12.000000}\selectfont 0}%
\end{pgfscope}%
\begin{pgfscope}%
\pgfpathrectangle{\pgfqpoint{4.123134in}{0.331635in}}{\pgfqpoint{2.583333in}{1.510000in}}%
\pgfusepath{clip}%
\pgfsetroundcap%
\pgfsetroundjoin%
\pgfsetlinewidth{0.803000pt}%
\definecolor{currentstroke}{rgb}{1.000000,1.000000,1.000000}%
\pgfsetstrokecolor{currentstroke}%
\pgfsetdash{}{0pt}%
\pgfpathmoveto{\pgfqpoint{4.813359in}{0.331635in}}%
\pgfpathlineto{\pgfqpoint{4.813359in}{1.841635in}}%
\pgfusepath{stroke}%
\end{pgfscope}%
\begin{pgfscope}%
\definecolor{textcolor}{rgb}{0.150000,0.150000,0.150000}%
\pgfsetstrokecolor{textcolor}%
\pgfsetfillcolor{textcolor}%
\pgftext[x=4.813359in,y=0.234413in,,top]{\color{textcolor}\rmfamily\fontsize{10.000000}{12.000000}\selectfont 5}%
\end{pgfscope}%
\begin{pgfscope}%
\pgfpathrectangle{\pgfqpoint{4.123134in}{0.331635in}}{\pgfqpoint{2.583333in}{1.510000in}}%
\pgfusepath{clip}%
\pgfsetroundcap%
\pgfsetroundjoin%
\pgfsetlinewidth{0.803000pt}%
\definecolor{currentstroke}{rgb}{1.000000,1.000000,1.000000}%
\pgfsetstrokecolor{currentstroke}%
\pgfsetdash{}{0pt}%
\pgfpathmoveto{\pgfqpoint{5.386160in}{0.331635in}}%
\pgfpathlineto{\pgfqpoint{5.386160in}{1.841635in}}%
\pgfusepath{stroke}%
\end{pgfscope}%
\begin{pgfscope}%
\definecolor{textcolor}{rgb}{0.150000,0.150000,0.150000}%
\pgfsetstrokecolor{textcolor}%
\pgfsetfillcolor{textcolor}%
\pgftext[x=5.386160in,y=0.234413in,,top]{\color{textcolor}\rmfamily\fontsize{10.000000}{12.000000}\selectfont 10}%
\end{pgfscope}%
\begin{pgfscope}%
\pgfpathrectangle{\pgfqpoint{4.123134in}{0.331635in}}{\pgfqpoint{2.583333in}{1.510000in}}%
\pgfusepath{clip}%
\pgfsetroundcap%
\pgfsetroundjoin%
\pgfsetlinewidth{0.803000pt}%
\definecolor{currentstroke}{rgb}{1.000000,1.000000,1.000000}%
\pgfsetstrokecolor{currentstroke}%
\pgfsetdash{}{0pt}%
\pgfpathmoveto{\pgfqpoint{5.958962in}{0.331635in}}%
\pgfpathlineto{\pgfqpoint{5.958962in}{1.841635in}}%
\pgfusepath{stroke}%
\end{pgfscope}%
\begin{pgfscope}%
\definecolor{textcolor}{rgb}{0.150000,0.150000,0.150000}%
\pgfsetstrokecolor{textcolor}%
\pgfsetfillcolor{textcolor}%
\pgftext[x=5.958962in,y=0.234413in,,top]{\color{textcolor}\rmfamily\fontsize{10.000000}{12.000000}\selectfont 15}%
\end{pgfscope}%
\begin{pgfscope}%
\pgfpathrectangle{\pgfqpoint{4.123134in}{0.331635in}}{\pgfqpoint{2.583333in}{1.510000in}}%
\pgfusepath{clip}%
\pgfsetroundcap%
\pgfsetroundjoin%
\pgfsetlinewidth{0.803000pt}%
\definecolor{currentstroke}{rgb}{1.000000,1.000000,1.000000}%
\pgfsetstrokecolor{currentstroke}%
\pgfsetdash{}{0pt}%
\pgfpathmoveto{\pgfqpoint{6.531763in}{0.331635in}}%
\pgfpathlineto{\pgfqpoint{6.531763in}{1.841635in}}%
\pgfusepath{stroke}%
\end{pgfscope}%
\begin{pgfscope}%
\definecolor{textcolor}{rgb}{0.150000,0.150000,0.150000}%
\pgfsetstrokecolor{textcolor}%
\pgfsetfillcolor{textcolor}%
\pgftext[x=6.531763in,y=0.234413in,,top]{\color{textcolor}\rmfamily\fontsize{10.000000}{12.000000}\selectfont 20}%
\end{pgfscope}%
\begin{pgfscope}%
\pgfpathrectangle{\pgfqpoint{4.123134in}{0.331635in}}{\pgfqpoint{2.583333in}{1.510000in}}%
\pgfusepath{clip}%
\pgfsetroundcap%
\pgfsetroundjoin%
\pgfsetlinewidth{0.803000pt}%
\definecolor{currentstroke}{rgb}{1.000000,1.000000,1.000000}%
\pgfsetstrokecolor{currentstroke}%
\pgfsetdash{}{0pt}%
\pgfpathmoveto{\pgfqpoint{4.123134in}{0.466226in}}%
\pgfpathlineto{\pgfqpoint{6.706467in}{0.466226in}}%
\pgfusepath{stroke}%
\end{pgfscope}%
\begin{pgfscope}%
\definecolor{textcolor}{rgb}{0.150000,0.150000,0.150000}%
\pgfsetstrokecolor{textcolor}%
\pgfsetfillcolor{textcolor}%
\pgftext[x=3.716667in,y=0.413465in,left,base]{\color{textcolor}\rmfamily\fontsize{10.000000}{12.000000}\selectfont 0.00}%
\end{pgfscope}%
\begin{pgfscope}%
\pgfpathrectangle{\pgfqpoint{4.123134in}{0.331635in}}{\pgfqpoint{2.583333in}{1.510000in}}%
\pgfusepath{clip}%
\pgfsetroundcap%
\pgfsetroundjoin%
\pgfsetlinewidth{0.803000pt}%
\definecolor{currentstroke}{rgb}{1.000000,1.000000,1.000000}%
\pgfsetstrokecolor{currentstroke}%
\pgfsetdash{}{0pt}%
\pgfpathmoveto{\pgfqpoint{4.123134in}{0.792919in}}%
\pgfpathlineto{\pgfqpoint{6.706467in}{0.792919in}}%
\pgfusepath{stroke}%
\end{pgfscope}%
\begin{pgfscope}%
\definecolor{textcolor}{rgb}{0.150000,0.150000,0.150000}%
\pgfsetstrokecolor{textcolor}%
\pgfsetfillcolor{textcolor}%
\pgftext[x=3.716667in,y=0.740158in,left,base]{\color{textcolor}\rmfamily\fontsize{10.000000}{12.000000}\selectfont 0.25}%
\end{pgfscope}%
\begin{pgfscope}%
\pgfpathrectangle{\pgfqpoint{4.123134in}{0.331635in}}{\pgfqpoint{2.583333in}{1.510000in}}%
\pgfusepath{clip}%
\pgfsetroundcap%
\pgfsetroundjoin%
\pgfsetlinewidth{0.803000pt}%
\definecolor{currentstroke}{rgb}{1.000000,1.000000,1.000000}%
\pgfsetstrokecolor{currentstroke}%
\pgfsetdash{}{0pt}%
\pgfpathmoveto{\pgfqpoint{4.123134in}{1.119612in}}%
\pgfpathlineto{\pgfqpoint{6.706467in}{1.119612in}}%
\pgfusepath{stroke}%
\end{pgfscope}%
\begin{pgfscope}%
\definecolor{textcolor}{rgb}{0.150000,0.150000,0.150000}%
\pgfsetstrokecolor{textcolor}%
\pgfsetfillcolor{textcolor}%
\pgftext[x=3.716667in,y=1.066851in,left,base]{\color{textcolor}\rmfamily\fontsize{10.000000}{12.000000}\selectfont 0.50}%
\end{pgfscope}%
\begin{pgfscope}%
\pgfpathrectangle{\pgfqpoint{4.123134in}{0.331635in}}{\pgfqpoint{2.583333in}{1.510000in}}%
\pgfusepath{clip}%
\pgfsetroundcap%
\pgfsetroundjoin%
\pgfsetlinewidth{0.803000pt}%
\definecolor{currentstroke}{rgb}{1.000000,1.000000,1.000000}%
\pgfsetstrokecolor{currentstroke}%
\pgfsetdash{}{0pt}%
\pgfpathmoveto{\pgfqpoint{4.123134in}{1.446306in}}%
\pgfpathlineto{\pgfqpoint{6.706467in}{1.446306in}}%
\pgfusepath{stroke}%
\end{pgfscope}%
\begin{pgfscope}%
\definecolor{textcolor}{rgb}{0.150000,0.150000,0.150000}%
\pgfsetstrokecolor{textcolor}%
\pgfsetfillcolor{textcolor}%
\pgftext[x=3.716667in,y=1.393544in,left,base]{\color{textcolor}\rmfamily\fontsize{10.000000}{12.000000}\selectfont 0.75}%
\end{pgfscope}%
\begin{pgfscope}%
\pgfpathrectangle{\pgfqpoint{4.123134in}{0.331635in}}{\pgfqpoint{2.583333in}{1.510000in}}%
\pgfusepath{clip}%
\pgfsetroundcap%
\pgfsetroundjoin%
\pgfsetlinewidth{0.803000pt}%
\definecolor{currentstroke}{rgb}{1.000000,1.000000,1.000000}%
\pgfsetstrokecolor{currentstroke}%
\pgfsetdash{}{0pt}%
\pgfpathmoveto{\pgfqpoint{4.123134in}{1.772999in}}%
\pgfpathlineto{\pgfqpoint{6.706467in}{1.772999in}}%
\pgfusepath{stroke}%
\end{pgfscope}%
\begin{pgfscope}%
\definecolor{textcolor}{rgb}{0.150000,0.150000,0.150000}%
\pgfsetstrokecolor{textcolor}%
\pgfsetfillcolor{textcolor}%
\pgftext[x=3.716667in,y=1.720237in,left,base]{\color{textcolor}\rmfamily\fontsize{10.000000}{12.000000}\selectfont 1.00}%
\end{pgfscope}%
\begin{pgfscope}%
\pgfpathrectangle{\pgfqpoint{4.123134in}{0.331635in}}{\pgfqpoint{2.583333in}{1.510000in}}%
\pgfusepath{clip}%
\pgfsetbuttcap%
\pgfsetroundjoin%
\definecolor{currentfill}{rgb}{0.121569,0.466667,0.705882}%
\pgfsetfillcolor{currentfill}%
\pgfsetfillopacity{0.250000}%
\pgfsetlinewidth{1.003750pt}%
\definecolor{currentstroke}{rgb}{1.000000,1.000000,1.000000}%
\pgfsetstrokecolor{currentstroke}%
\pgfsetstrokeopacity{0.250000}%
\pgfsetdash{}{0pt}%
\pgfpathmoveto{\pgfqpoint{4.297838in}{0.532181in}}%
\pgfpathlineto{\pgfqpoint{4.297838in}{0.400271in}}%
\pgfpathlineto{\pgfqpoint{4.469678in}{0.400271in}}%
\pgfpathlineto{\pgfqpoint{4.584239in}{0.400271in}}%
\pgfpathlineto{\pgfqpoint{4.698799in}{0.400271in}}%
\pgfpathlineto{\pgfqpoint{4.813359in}{0.400271in}}%
\pgfpathlineto{\pgfqpoint{4.927919in}{0.400271in}}%
\pgfpathlineto{\pgfqpoint{5.042480in}{0.400271in}}%
\pgfpathlineto{\pgfqpoint{5.157040in}{0.400271in}}%
\pgfpathlineto{\pgfqpoint{5.271600in}{0.400271in}}%
\pgfpathlineto{\pgfqpoint{5.386160in}{0.400271in}}%
\pgfpathlineto{\pgfqpoint{5.500721in}{0.400271in}}%
\pgfpathlineto{\pgfqpoint{5.615281in}{0.400271in}}%
\pgfpathlineto{\pgfqpoint{5.729841in}{0.400271in}}%
\pgfpathlineto{\pgfqpoint{5.844401in}{0.400271in}}%
\pgfpathlineto{\pgfqpoint{5.958962in}{0.400271in}}%
\pgfpathlineto{\pgfqpoint{6.073522in}{0.400271in}}%
\pgfpathlineto{\pgfqpoint{6.188082in}{0.400271in}}%
\pgfpathlineto{\pgfqpoint{6.302642in}{0.400271in}}%
\pgfpathlineto{\pgfqpoint{6.417202in}{0.400271in}}%
\pgfpathlineto{\pgfqpoint{6.589043in}{0.400271in}}%
\pgfpathlineto{\pgfqpoint{6.589043in}{0.532181in}}%
\pgfpathlineto{\pgfqpoint{6.589043in}{0.532181in}}%
\pgfpathlineto{\pgfqpoint{6.417202in}{0.532181in}}%
\pgfpathlineto{\pgfqpoint{6.302642in}{0.532181in}}%
\pgfpathlineto{\pgfqpoint{6.188082in}{0.532181in}}%
\pgfpathlineto{\pgfqpoint{6.073522in}{0.532181in}}%
\pgfpathlineto{\pgfqpoint{5.958962in}{0.532181in}}%
\pgfpathlineto{\pgfqpoint{5.844401in}{0.532181in}}%
\pgfpathlineto{\pgfqpoint{5.729841in}{0.532181in}}%
\pgfpathlineto{\pgfqpoint{5.615281in}{0.532181in}}%
\pgfpathlineto{\pgfqpoint{5.500721in}{0.532181in}}%
\pgfpathlineto{\pgfqpoint{5.386160in}{0.532181in}}%
\pgfpathlineto{\pgfqpoint{5.271600in}{0.532181in}}%
\pgfpathlineto{\pgfqpoint{5.157040in}{0.532181in}}%
\pgfpathlineto{\pgfqpoint{5.042480in}{0.532181in}}%
\pgfpathlineto{\pgfqpoint{4.927919in}{0.532181in}}%
\pgfpathlineto{\pgfqpoint{4.813359in}{0.532181in}}%
\pgfpathlineto{\pgfqpoint{4.698799in}{0.532181in}}%
\pgfpathlineto{\pgfqpoint{4.584239in}{0.532181in}}%
\pgfpathlineto{\pgfqpoint{4.469678in}{0.532181in}}%
\pgfpathlineto{\pgfqpoint{4.297838in}{0.532181in}}%
\pgfpathclose%
\pgfusepath{stroke,fill}%
\end{pgfscope}%
\begin{pgfscope}%
\pgfpathrectangle{\pgfqpoint{4.123134in}{0.331635in}}{\pgfqpoint{2.583333in}{1.510000in}}%
\pgfusepath{clip}%
\pgfsetbuttcap%
\pgfsetroundjoin%
\pgfsetlinewidth{1.505625pt}%
\definecolor{currentstroke}{rgb}{0.000000,0.000000,0.000000}%
\pgfsetstrokecolor{currentstroke}%
\pgfsetdash{}{0pt}%
\pgfpathmoveto{\pgfqpoint{4.240558in}{0.466226in}}%
\pgfpathlineto{\pgfqpoint{4.240558in}{1.772999in}}%
\pgfusepath{stroke}%
\end{pgfscope}%
\begin{pgfscope}%
\pgfpathrectangle{\pgfqpoint{4.123134in}{0.331635in}}{\pgfqpoint{2.583333in}{1.510000in}}%
\pgfusepath{clip}%
\pgfsetbuttcap%
\pgfsetroundjoin%
\pgfsetlinewidth{1.505625pt}%
\definecolor{currentstroke}{rgb}{0.000000,0.000000,0.000000}%
\pgfsetstrokecolor{currentstroke}%
\pgfsetdash{}{0pt}%
\pgfpathmoveto{\pgfqpoint{4.355118in}{0.466226in}}%
\pgfpathlineto{\pgfqpoint{4.355118in}{0.549474in}}%
\pgfusepath{stroke}%
\end{pgfscope}%
\begin{pgfscope}%
\pgfpathrectangle{\pgfqpoint{4.123134in}{0.331635in}}{\pgfqpoint{2.583333in}{1.510000in}}%
\pgfusepath{clip}%
\pgfsetbuttcap%
\pgfsetroundjoin%
\pgfsetlinewidth{1.505625pt}%
\definecolor{currentstroke}{rgb}{0.000000,0.000000,0.000000}%
\pgfsetstrokecolor{currentstroke}%
\pgfsetdash{}{0pt}%
\pgfpathmoveto{\pgfqpoint{4.469678in}{0.466226in}}%
\pgfpathlineto{\pgfqpoint{4.469678in}{0.514172in}}%
\pgfusepath{stroke}%
\end{pgfscope}%
\begin{pgfscope}%
\pgfpathrectangle{\pgfqpoint{4.123134in}{0.331635in}}{\pgfqpoint{2.583333in}{1.510000in}}%
\pgfusepath{clip}%
\pgfsetbuttcap%
\pgfsetroundjoin%
\pgfsetlinewidth{1.505625pt}%
\definecolor{currentstroke}{rgb}{0.000000,0.000000,0.000000}%
\pgfsetstrokecolor{currentstroke}%
\pgfsetdash{}{0pt}%
\pgfpathmoveto{\pgfqpoint{4.584239in}{0.466226in}}%
\pgfpathlineto{\pgfqpoint{4.584239in}{0.487572in}}%
\pgfusepath{stroke}%
\end{pgfscope}%
\begin{pgfscope}%
\pgfpathrectangle{\pgfqpoint{4.123134in}{0.331635in}}{\pgfqpoint{2.583333in}{1.510000in}}%
\pgfusepath{clip}%
\pgfsetbuttcap%
\pgfsetroundjoin%
\pgfsetlinewidth{1.505625pt}%
\definecolor{currentstroke}{rgb}{0.000000,0.000000,0.000000}%
\pgfsetstrokecolor{currentstroke}%
\pgfsetdash{}{0pt}%
\pgfpathmoveto{\pgfqpoint{4.698799in}{0.466226in}}%
\pgfpathlineto{\pgfqpoint{4.698799in}{0.568170in}}%
\pgfusepath{stroke}%
\end{pgfscope}%
\begin{pgfscope}%
\pgfpathrectangle{\pgfqpoint{4.123134in}{0.331635in}}{\pgfqpoint{2.583333in}{1.510000in}}%
\pgfusepath{clip}%
\pgfsetbuttcap%
\pgfsetroundjoin%
\pgfsetlinewidth{1.505625pt}%
\definecolor{currentstroke}{rgb}{0.000000,0.000000,0.000000}%
\pgfsetstrokecolor{currentstroke}%
\pgfsetdash{}{0pt}%
\pgfpathmoveto{\pgfqpoint{4.813359in}{0.466226in}}%
\pgfpathlineto{\pgfqpoint{4.813359in}{0.496022in}}%
\pgfusepath{stroke}%
\end{pgfscope}%
\begin{pgfscope}%
\pgfpathrectangle{\pgfqpoint{4.123134in}{0.331635in}}{\pgfqpoint{2.583333in}{1.510000in}}%
\pgfusepath{clip}%
\pgfsetbuttcap%
\pgfsetroundjoin%
\pgfsetlinewidth{1.505625pt}%
\definecolor{currentstroke}{rgb}{0.000000,0.000000,0.000000}%
\pgfsetstrokecolor{currentstroke}%
\pgfsetdash{}{0pt}%
\pgfpathmoveto{\pgfqpoint{4.927919in}{0.466226in}}%
\pgfpathlineto{\pgfqpoint{4.927919in}{0.543724in}}%
\pgfusepath{stroke}%
\end{pgfscope}%
\begin{pgfscope}%
\pgfpathrectangle{\pgfqpoint{4.123134in}{0.331635in}}{\pgfqpoint{2.583333in}{1.510000in}}%
\pgfusepath{clip}%
\pgfsetbuttcap%
\pgfsetroundjoin%
\pgfsetlinewidth{1.505625pt}%
\definecolor{currentstroke}{rgb}{0.000000,0.000000,0.000000}%
\pgfsetstrokecolor{currentstroke}%
\pgfsetdash{}{0pt}%
\pgfpathmoveto{\pgfqpoint{5.042480in}{0.466226in}}%
\pgfpathlineto{\pgfqpoint{5.042480in}{0.460796in}}%
\pgfusepath{stroke}%
\end{pgfscope}%
\begin{pgfscope}%
\pgfpathrectangle{\pgfqpoint{4.123134in}{0.331635in}}{\pgfqpoint{2.583333in}{1.510000in}}%
\pgfusepath{clip}%
\pgfsetbuttcap%
\pgfsetroundjoin%
\pgfsetlinewidth{1.505625pt}%
\definecolor{currentstroke}{rgb}{0.000000,0.000000,0.000000}%
\pgfsetstrokecolor{currentstroke}%
\pgfsetdash{}{0pt}%
\pgfpathmoveto{\pgfqpoint{5.157040in}{0.466226in}}%
\pgfpathlineto{\pgfqpoint{5.157040in}{0.455345in}}%
\pgfusepath{stroke}%
\end{pgfscope}%
\begin{pgfscope}%
\pgfpathrectangle{\pgfqpoint{4.123134in}{0.331635in}}{\pgfqpoint{2.583333in}{1.510000in}}%
\pgfusepath{clip}%
\pgfsetbuttcap%
\pgfsetroundjoin%
\pgfsetlinewidth{1.505625pt}%
\definecolor{currentstroke}{rgb}{0.000000,0.000000,0.000000}%
\pgfsetstrokecolor{currentstroke}%
\pgfsetdash{}{0pt}%
\pgfpathmoveto{\pgfqpoint{5.271600in}{0.466226in}}%
\pgfpathlineto{\pgfqpoint{5.271600in}{0.506889in}}%
\pgfusepath{stroke}%
\end{pgfscope}%
\begin{pgfscope}%
\pgfpathrectangle{\pgfqpoint{4.123134in}{0.331635in}}{\pgfqpoint{2.583333in}{1.510000in}}%
\pgfusepath{clip}%
\pgfsetbuttcap%
\pgfsetroundjoin%
\pgfsetlinewidth{1.505625pt}%
\definecolor{currentstroke}{rgb}{0.000000,0.000000,0.000000}%
\pgfsetstrokecolor{currentstroke}%
\pgfsetdash{}{0pt}%
\pgfpathmoveto{\pgfqpoint{5.386160in}{0.466226in}}%
\pgfpathlineto{\pgfqpoint{5.386160in}{0.455719in}}%
\pgfusepath{stroke}%
\end{pgfscope}%
\begin{pgfscope}%
\pgfpathrectangle{\pgfqpoint{4.123134in}{0.331635in}}{\pgfqpoint{2.583333in}{1.510000in}}%
\pgfusepath{clip}%
\pgfsetbuttcap%
\pgfsetroundjoin%
\pgfsetlinewidth{1.505625pt}%
\definecolor{currentstroke}{rgb}{0.000000,0.000000,0.000000}%
\pgfsetstrokecolor{currentstroke}%
\pgfsetdash{}{0pt}%
\pgfpathmoveto{\pgfqpoint{5.500721in}{0.466226in}}%
\pgfpathlineto{\pgfqpoint{5.500721in}{0.487421in}}%
\pgfusepath{stroke}%
\end{pgfscope}%
\begin{pgfscope}%
\pgfpathrectangle{\pgfqpoint{4.123134in}{0.331635in}}{\pgfqpoint{2.583333in}{1.510000in}}%
\pgfusepath{clip}%
\pgfsetbuttcap%
\pgfsetroundjoin%
\pgfsetlinewidth{1.505625pt}%
\definecolor{currentstroke}{rgb}{0.000000,0.000000,0.000000}%
\pgfsetstrokecolor{currentstroke}%
\pgfsetdash{}{0pt}%
\pgfpathmoveto{\pgfqpoint{5.615281in}{0.466226in}}%
\pgfpathlineto{\pgfqpoint{5.615281in}{0.463088in}}%
\pgfusepath{stroke}%
\end{pgfscope}%
\begin{pgfscope}%
\pgfpathrectangle{\pgfqpoint{4.123134in}{0.331635in}}{\pgfqpoint{2.583333in}{1.510000in}}%
\pgfusepath{clip}%
\pgfsetbuttcap%
\pgfsetroundjoin%
\pgfsetlinewidth{1.505625pt}%
\definecolor{currentstroke}{rgb}{0.000000,0.000000,0.000000}%
\pgfsetstrokecolor{currentstroke}%
\pgfsetdash{}{0pt}%
\pgfpathmoveto{\pgfqpoint{5.729841in}{0.466226in}}%
\pgfpathlineto{\pgfqpoint{5.729841in}{0.450497in}}%
\pgfusepath{stroke}%
\end{pgfscope}%
\begin{pgfscope}%
\pgfpathrectangle{\pgfqpoint{4.123134in}{0.331635in}}{\pgfqpoint{2.583333in}{1.510000in}}%
\pgfusepath{clip}%
\pgfsetbuttcap%
\pgfsetroundjoin%
\pgfsetlinewidth{1.505625pt}%
\definecolor{currentstroke}{rgb}{0.000000,0.000000,0.000000}%
\pgfsetstrokecolor{currentstroke}%
\pgfsetdash{}{0pt}%
\pgfpathmoveto{\pgfqpoint{5.844401in}{0.466226in}}%
\pgfpathlineto{\pgfqpoint{5.844401in}{0.469976in}}%
\pgfusepath{stroke}%
\end{pgfscope}%
\begin{pgfscope}%
\pgfpathrectangle{\pgfqpoint{4.123134in}{0.331635in}}{\pgfqpoint{2.583333in}{1.510000in}}%
\pgfusepath{clip}%
\pgfsetbuttcap%
\pgfsetroundjoin%
\pgfsetlinewidth{1.505625pt}%
\definecolor{currentstroke}{rgb}{0.000000,0.000000,0.000000}%
\pgfsetstrokecolor{currentstroke}%
\pgfsetdash{}{0pt}%
\pgfpathmoveto{\pgfqpoint{5.958962in}{0.466226in}}%
\pgfpathlineto{\pgfqpoint{5.958962in}{0.513799in}}%
\pgfusepath{stroke}%
\end{pgfscope}%
\begin{pgfscope}%
\pgfpathrectangle{\pgfqpoint{4.123134in}{0.331635in}}{\pgfqpoint{2.583333in}{1.510000in}}%
\pgfusepath{clip}%
\pgfsetbuttcap%
\pgfsetroundjoin%
\pgfsetlinewidth{1.505625pt}%
\definecolor{currentstroke}{rgb}{0.000000,0.000000,0.000000}%
\pgfsetstrokecolor{currentstroke}%
\pgfsetdash{}{0pt}%
\pgfpathmoveto{\pgfqpoint{6.073522in}{0.466226in}}%
\pgfpathlineto{\pgfqpoint{6.073522in}{0.459066in}}%
\pgfusepath{stroke}%
\end{pgfscope}%
\begin{pgfscope}%
\pgfpathrectangle{\pgfqpoint{4.123134in}{0.331635in}}{\pgfqpoint{2.583333in}{1.510000in}}%
\pgfusepath{clip}%
\pgfsetbuttcap%
\pgfsetroundjoin%
\pgfsetlinewidth{1.505625pt}%
\definecolor{currentstroke}{rgb}{0.000000,0.000000,0.000000}%
\pgfsetstrokecolor{currentstroke}%
\pgfsetdash{}{0pt}%
\pgfpathmoveto{\pgfqpoint{6.188082in}{0.466226in}}%
\pgfpathlineto{\pgfqpoint{6.188082in}{0.459984in}}%
\pgfusepath{stroke}%
\end{pgfscope}%
\begin{pgfscope}%
\pgfpathrectangle{\pgfqpoint{4.123134in}{0.331635in}}{\pgfqpoint{2.583333in}{1.510000in}}%
\pgfusepath{clip}%
\pgfsetbuttcap%
\pgfsetroundjoin%
\pgfsetlinewidth{1.505625pt}%
\definecolor{currentstroke}{rgb}{0.000000,0.000000,0.000000}%
\pgfsetstrokecolor{currentstroke}%
\pgfsetdash{}{0pt}%
\pgfpathmoveto{\pgfqpoint{6.302642in}{0.466226in}}%
\pgfpathlineto{\pgfqpoint{6.302642in}{0.511292in}}%
\pgfusepath{stroke}%
\end{pgfscope}%
\begin{pgfscope}%
\pgfpathrectangle{\pgfqpoint{4.123134in}{0.331635in}}{\pgfqpoint{2.583333in}{1.510000in}}%
\pgfusepath{clip}%
\pgfsetbuttcap%
\pgfsetroundjoin%
\pgfsetlinewidth{1.505625pt}%
\definecolor{currentstroke}{rgb}{0.000000,0.000000,0.000000}%
\pgfsetstrokecolor{currentstroke}%
\pgfsetdash{}{0pt}%
\pgfpathmoveto{\pgfqpoint{6.417202in}{0.466226in}}%
\pgfpathlineto{\pgfqpoint{6.417202in}{0.460740in}}%
\pgfusepath{stroke}%
\end{pgfscope}%
\begin{pgfscope}%
\pgfpathrectangle{\pgfqpoint{4.123134in}{0.331635in}}{\pgfqpoint{2.583333in}{1.510000in}}%
\pgfusepath{clip}%
\pgfsetbuttcap%
\pgfsetroundjoin%
\pgfsetlinewidth{1.505625pt}%
\definecolor{currentstroke}{rgb}{0.000000,0.000000,0.000000}%
\pgfsetstrokecolor{currentstroke}%
\pgfsetdash{}{0pt}%
\pgfpathmoveto{\pgfqpoint{6.531763in}{0.466226in}}%
\pgfpathlineto{\pgfqpoint{6.531763in}{0.449443in}}%
\pgfusepath{stroke}%
\end{pgfscope}%
\begin{pgfscope}%
\pgfpathrectangle{\pgfqpoint{4.123134in}{0.331635in}}{\pgfqpoint{2.583333in}{1.510000in}}%
\pgfusepath{clip}%
\pgfsetroundcap%
\pgfsetroundjoin%
\pgfsetlinewidth{1.505625pt}%
\definecolor{currentstroke}{rgb}{0.737255,0.741176,0.133333}%
\pgfsetstrokecolor{currentstroke}%
\pgfsetdash{}{0pt}%
\pgfpathmoveto{\pgfqpoint{4.123134in}{0.466226in}}%
\pgfpathlineto{\pgfqpoint{6.706467in}{0.466226in}}%
\pgfusepath{stroke}%
\end{pgfscope}%
\begin{pgfscope}%
\pgfpathrectangle{\pgfqpoint{4.123134in}{0.331635in}}{\pgfqpoint{2.583333in}{1.510000in}}%
\pgfusepath{clip}%
\pgfsetbuttcap%
\pgfsetroundjoin%
\definecolor{currentfill}{rgb}{0.737255,0.741176,0.133333}%
\pgfsetfillcolor{currentfill}%
\pgfsetlinewidth{1.003750pt}%
\definecolor{currentstroke}{rgb}{0.737255,0.741176,0.133333}%
\pgfsetstrokecolor{currentstroke}%
\pgfsetdash{}{0pt}%
\pgfsys@defobject{currentmarker}{\pgfqpoint{-0.034722in}{-0.034722in}}{\pgfqpoint{0.034722in}{0.034722in}}{%
\pgfpathmoveto{\pgfqpoint{0.000000in}{-0.034722in}}%
\pgfpathcurveto{\pgfqpoint{0.009208in}{-0.034722in}}{\pgfqpoint{0.018041in}{-0.031064in}}{\pgfqpoint{0.024552in}{-0.024552in}}%
\pgfpathcurveto{\pgfqpoint{0.031064in}{-0.018041in}}{\pgfqpoint{0.034722in}{-0.009208in}}{\pgfqpoint{0.034722in}{0.000000in}}%
\pgfpathcurveto{\pgfqpoint{0.034722in}{0.009208in}}{\pgfqpoint{0.031064in}{0.018041in}}{\pgfqpoint{0.024552in}{0.024552in}}%
\pgfpathcurveto{\pgfqpoint{0.018041in}{0.031064in}}{\pgfqpoint{0.009208in}{0.034722in}}{\pgfqpoint{0.000000in}{0.034722in}}%
\pgfpathcurveto{\pgfqpoint{-0.009208in}{0.034722in}}{\pgfqpoint{-0.018041in}{0.031064in}}{\pgfqpoint{-0.024552in}{0.024552in}}%
\pgfpathcurveto{\pgfqpoint{-0.031064in}{0.018041in}}{\pgfqpoint{-0.034722in}{0.009208in}}{\pgfqpoint{-0.034722in}{0.000000in}}%
\pgfpathcurveto{\pgfqpoint{-0.034722in}{-0.009208in}}{\pgfqpoint{-0.031064in}{-0.018041in}}{\pgfqpoint{-0.024552in}{-0.024552in}}%
\pgfpathcurveto{\pgfqpoint{-0.018041in}{-0.031064in}}{\pgfqpoint{-0.009208in}{-0.034722in}}{\pgfqpoint{0.000000in}{-0.034722in}}%
\pgfpathclose%
\pgfusepath{stroke,fill}%
}%
\begin{pgfscope}%
\pgfsys@transformshift{4.240558in}{1.772999in}%
\pgfsys@useobject{currentmarker}{}%
\end{pgfscope}%
\begin{pgfscope}%
\pgfsys@transformshift{4.355118in}{0.549474in}%
\pgfsys@useobject{currentmarker}{}%
\end{pgfscope}%
\begin{pgfscope}%
\pgfsys@transformshift{4.469678in}{0.514172in}%
\pgfsys@useobject{currentmarker}{}%
\end{pgfscope}%
\begin{pgfscope}%
\pgfsys@transformshift{4.584239in}{0.487572in}%
\pgfsys@useobject{currentmarker}{}%
\end{pgfscope}%
\begin{pgfscope}%
\pgfsys@transformshift{4.698799in}{0.568170in}%
\pgfsys@useobject{currentmarker}{}%
\end{pgfscope}%
\begin{pgfscope}%
\pgfsys@transformshift{4.813359in}{0.496022in}%
\pgfsys@useobject{currentmarker}{}%
\end{pgfscope}%
\begin{pgfscope}%
\pgfsys@transformshift{4.927919in}{0.543724in}%
\pgfsys@useobject{currentmarker}{}%
\end{pgfscope}%
\begin{pgfscope}%
\pgfsys@transformshift{5.042480in}{0.460796in}%
\pgfsys@useobject{currentmarker}{}%
\end{pgfscope}%
\begin{pgfscope}%
\pgfsys@transformshift{5.157040in}{0.455345in}%
\pgfsys@useobject{currentmarker}{}%
\end{pgfscope}%
\begin{pgfscope}%
\pgfsys@transformshift{5.271600in}{0.506889in}%
\pgfsys@useobject{currentmarker}{}%
\end{pgfscope}%
\begin{pgfscope}%
\pgfsys@transformshift{5.386160in}{0.455719in}%
\pgfsys@useobject{currentmarker}{}%
\end{pgfscope}%
\begin{pgfscope}%
\pgfsys@transformshift{5.500721in}{0.487421in}%
\pgfsys@useobject{currentmarker}{}%
\end{pgfscope}%
\begin{pgfscope}%
\pgfsys@transformshift{5.615281in}{0.463088in}%
\pgfsys@useobject{currentmarker}{}%
\end{pgfscope}%
\begin{pgfscope}%
\pgfsys@transformshift{5.729841in}{0.450497in}%
\pgfsys@useobject{currentmarker}{}%
\end{pgfscope}%
\begin{pgfscope}%
\pgfsys@transformshift{5.844401in}{0.469976in}%
\pgfsys@useobject{currentmarker}{}%
\end{pgfscope}%
\begin{pgfscope}%
\pgfsys@transformshift{5.958962in}{0.513799in}%
\pgfsys@useobject{currentmarker}{}%
\end{pgfscope}%
\begin{pgfscope}%
\pgfsys@transformshift{6.073522in}{0.459066in}%
\pgfsys@useobject{currentmarker}{}%
\end{pgfscope}%
\begin{pgfscope}%
\pgfsys@transformshift{6.188082in}{0.459984in}%
\pgfsys@useobject{currentmarker}{}%
\end{pgfscope}%
\begin{pgfscope}%
\pgfsys@transformshift{6.302642in}{0.511292in}%
\pgfsys@useobject{currentmarker}{}%
\end{pgfscope}%
\begin{pgfscope}%
\pgfsys@transformshift{6.417202in}{0.460740in}%
\pgfsys@useobject{currentmarker}{}%
\end{pgfscope}%
\begin{pgfscope}%
\pgfsys@transformshift{6.531763in}{0.449443in}%
\pgfsys@useobject{currentmarker}{}%
\end{pgfscope}%
\end{pgfscope}%
\begin{pgfscope}%
\pgfsetrectcap%
\pgfsetmiterjoin%
\pgfsetlinewidth{0.803000pt}%
\definecolor{currentstroke}{rgb}{1.000000,1.000000,1.000000}%
\pgfsetstrokecolor{currentstroke}%
\pgfsetdash{}{0pt}%
\pgfpathmoveto{\pgfqpoint{4.123134in}{0.331635in}}%
\pgfpathlineto{\pgfqpoint{4.123134in}{1.841635in}}%
\pgfusepath{stroke}%
\end{pgfscope}%
\begin{pgfscope}%
\pgfsetrectcap%
\pgfsetmiterjoin%
\pgfsetlinewidth{0.803000pt}%
\definecolor{currentstroke}{rgb}{1.000000,1.000000,1.000000}%
\pgfsetstrokecolor{currentstroke}%
\pgfsetdash{}{0pt}%
\pgfpathmoveto{\pgfqpoint{6.706467in}{0.331635in}}%
\pgfpathlineto{\pgfqpoint{6.706467in}{1.841635in}}%
\pgfusepath{stroke}%
\end{pgfscope}%
\begin{pgfscope}%
\pgfsetrectcap%
\pgfsetmiterjoin%
\pgfsetlinewidth{0.803000pt}%
\definecolor{currentstroke}{rgb}{1.000000,1.000000,1.000000}%
\pgfsetstrokecolor{currentstroke}%
\pgfsetdash{}{0pt}%
\pgfpathmoveto{\pgfqpoint{4.123134in}{0.331635in}}%
\pgfpathlineto{\pgfqpoint{6.706467in}{0.331635in}}%
\pgfusepath{stroke}%
\end{pgfscope}%
\begin{pgfscope}%
\pgfsetrectcap%
\pgfsetmiterjoin%
\pgfsetlinewidth{0.803000pt}%
\definecolor{currentstroke}{rgb}{1.000000,1.000000,1.000000}%
\pgfsetstrokecolor{currentstroke}%
\pgfsetdash{}{0pt}%
\pgfpathmoveto{\pgfqpoint{4.123134in}{1.841635in}}%
\pgfpathlineto{\pgfqpoint{6.706467in}{1.841635in}}%
\pgfusepath{stroke}%
\end{pgfscope}%
\begin{pgfscope}%
\definecolor{textcolor}{rgb}{0.150000,0.150000,0.150000}%
\pgfsetstrokecolor{textcolor}%
\pgfsetfillcolor{textcolor}%
\pgftext[x=5.414800in,y=1.924968in,,base]{\color{textcolor}\rmfamily\fontsize{12.000000}{14.400000}\selectfont Partial Autocorrelation V}%
\end{pgfscope}%
\end{pgfpicture}%
\makeatother%
\endgroup%

    \end{adjustbox}
    \begin{adjustbox}{width=.95\textwidth,center}
    %% Creator: Matplotlib, PGF backend
%%
%% To include the figure in your LaTeX document, write
%%   \input{<filename>.pgf}
%%
%% Make sure the required packages are loaded in your preamble
%%   \usepackage{pgf}
%%
%% Figures using additional raster images can only be included by \input if
%% they are in the same directory as the main LaTeX file. For loading figures
%% from other directories you can use the `import` package
%%   \usepackage{import}
%% and then include the figures with
%%   \import{<path to file>}{<filename>.pgf}
%%
%% Matplotlib used the following preamble
%%   \usepackage{fontspec}
%%   \setmainfont{DejaVuSerif.ttf}[Path=/opt/tljh/user/lib/python3.6/site-packages/matplotlib/mpl-data/fonts/ttf/]
%%   \setsansfont{DejaVuSans.ttf}[Path=/opt/tljh/user/lib/python3.6/site-packages/matplotlib/mpl-data/fonts/ttf/]
%%   \setmonofont{DejaVuSansMono.ttf}[Path=/opt/tljh/user/lib/python3.6/site-packages/matplotlib/mpl-data/fonts/ttf/]
%%
\begingroup%
\makeatletter%
\begin{pgfpicture}%
\pgfpathrectangle{\pgfpointorigin}{\pgfqpoint{6.806467in}{2.151596in}}%
\pgfusepath{use as bounding box, clip}%
\begin{pgfscope}%
\pgfsetbuttcap%
\pgfsetmiterjoin%
\definecolor{currentfill}{rgb}{1.000000,1.000000,1.000000}%
\pgfsetfillcolor{currentfill}%
\pgfsetlinewidth{0.000000pt}%
\definecolor{currentstroke}{rgb}{1.000000,1.000000,1.000000}%
\pgfsetstrokecolor{currentstroke}%
\pgfsetdash{}{0pt}%
\pgfpathmoveto{\pgfqpoint{0.000000in}{0.000000in}}%
\pgfpathlineto{\pgfqpoint{6.806467in}{0.000000in}}%
\pgfpathlineto{\pgfqpoint{6.806467in}{2.151596in}}%
\pgfpathlineto{\pgfqpoint{0.000000in}{2.151596in}}%
\pgfpathclose%
\pgfusepath{fill}%
\end{pgfscope}%
\begin{pgfscope}%
\pgfsetbuttcap%
\pgfsetmiterjoin%
\definecolor{currentfill}{rgb}{0.917647,0.917647,0.949020}%
\pgfsetfillcolor{currentfill}%
\pgfsetlinewidth{0.000000pt}%
\definecolor{currentstroke}{rgb}{0.000000,0.000000,0.000000}%
\pgfsetstrokecolor{currentstroke}%
\pgfsetstrokeopacity{0.000000}%
\pgfsetdash{}{0pt}%
\pgfpathmoveto{\pgfqpoint{0.506467in}{0.331635in}}%
\pgfpathlineto{\pgfqpoint{3.089800in}{0.331635in}}%
\pgfpathlineto{\pgfqpoint{3.089800in}{1.841635in}}%
\pgfpathlineto{\pgfqpoint{0.506467in}{1.841635in}}%
\pgfpathclose%
\pgfusepath{fill}%
\end{pgfscope}%
\begin{pgfscope}%
\pgfpathrectangle{\pgfqpoint{0.506467in}{0.331635in}}{\pgfqpoint{2.583333in}{1.510000in}}%
\pgfusepath{clip}%
\pgfsetroundcap%
\pgfsetroundjoin%
\pgfsetlinewidth{0.803000pt}%
\definecolor{currentstroke}{rgb}{1.000000,1.000000,1.000000}%
\pgfsetstrokecolor{currentstroke}%
\pgfsetdash{}{0pt}%
\pgfpathmoveto{\pgfqpoint{0.623891in}{0.331635in}}%
\pgfpathlineto{\pgfqpoint{0.623891in}{1.841635in}}%
\pgfusepath{stroke}%
\end{pgfscope}%
\begin{pgfscope}%
\definecolor{textcolor}{rgb}{0.150000,0.150000,0.150000}%
\pgfsetstrokecolor{textcolor}%
\pgfsetfillcolor{textcolor}%
\pgftext[x=0.623891in,y=0.234413in,,top]{\color{textcolor}\rmfamily\fontsize{10.000000}{12.000000}\selectfont 0}%
\end{pgfscope}%
\begin{pgfscope}%
\pgfpathrectangle{\pgfqpoint{0.506467in}{0.331635in}}{\pgfqpoint{2.583333in}{1.510000in}}%
\pgfusepath{clip}%
\pgfsetroundcap%
\pgfsetroundjoin%
\pgfsetlinewidth{0.803000pt}%
\definecolor{currentstroke}{rgb}{1.000000,1.000000,1.000000}%
\pgfsetstrokecolor{currentstroke}%
\pgfsetdash{}{0pt}%
\pgfpathmoveto{\pgfqpoint{1.196693in}{0.331635in}}%
\pgfpathlineto{\pgfqpoint{1.196693in}{1.841635in}}%
\pgfusepath{stroke}%
\end{pgfscope}%
\begin{pgfscope}%
\definecolor{textcolor}{rgb}{0.150000,0.150000,0.150000}%
\pgfsetstrokecolor{textcolor}%
\pgfsetfillcolor{textcolor}%
\pgftext[x=1.196693in,y=0.234413in,,top]{\color{textcolor}\rmfamily\fontsize{10.000000}{12.000000}\selectfont 5}%
\end{pgfscope}%
\begin{pgfscope}%
\pgfpathrectangle{\pgfqpoint{0.506467in}{0.331635in}}{\pgfqpoint{2.583333in}{1.510000in}}%
\pgfusepath{clip}%
\pgfsetroundcap%
\pgfsetroundjoin%
\pgfsetlinewidth{0.803000pt}%
\definecolor{currentstroke}{rgb}{1.000000,1.000000,1.000000}%
\pgfsetstrokecolor{currentstroke}%
\pgfsetdash{}{0pt}%
\pgfpathmoveto{\pgfqpoint{1.769494in}{0.331635in}}%
\pgfpathlineto{\pgfqpoint{1.769494in}{1.841635in}}%
\pgfusepath{stroke}%
\end{pgfscope}%
\begin{pgfscope}%
\definecolor{textcolor}{rgb}{0.150000,0.150000,0.150000}%
\pgfsetstrokecolor{textcolor}%
\pgfsetfillcolor{textcolor}%
\pgftext[x=1.769494in,y=0.234413in,,top]{\color{textcolor}\rmfamily\fontsize{10.000000}{12.000000}\selectfont 10}%
\end{pgfscope}%
\begin{pgfscope}%
\pgfpathrectangle{\pgfqpoint{0.506467in}{0.331635in}}{\pgfqpoint{2.583333in}{1.510000in}}%
\pgfusepath{clip}%
\pgfsetroundcap%
\pgfsetroundjoin%
\pgfsetlinewidth{0.803000pt}%
\definecolor{currentstroke}{rgb}{1.000000,1.000000,1.000000}%
\pgfsetstrokecolor{currentstroke}%
\pgfsetdash{}{0pt}%
\pgfpathmoveto{\pgfqpoint{2.342295in}{0.331635in}}%
\pgfpathlineto{\pgfqpoint{2.342295in}{1.841635in}}%
\pgfusepath{stroke}%
\end{pgfscope}%
\begin{pgfscope}%
\definecolor{textcolor}{rgb}{0.150000,0.150000,0.150000}%
\pgfsetstrokecolor{textcolor}%
\pgfsetfillcolor{textcolor}%
\pgftext[x=2.342295in,y=0.234413in,,top]{\color{textcolor}\rmfamily\fontsize{10.000000}{12.000000}\selectfont 15}%
\end{pgfscope}%
\begin{pgfscope}%
\pgfpathrectangle{\pgfqpoint{0.506467in}{0.331635in}}{\pgfqpoint{2.583333in}{1.510000in}}%
\pgfusepath{clip}%
\pgfsetroundcap%
\pgfsetroundjoin%
\pgfsetlinewidth{0.803000pt}%
\definecolor{currentstroke}{rgb}{1.000000,1.000000,1.000000}%
\pgfsetstrokecolor{currentstroke}%
\pgfsetdash{}{0pt}%
\pgfpathmoveto{\pgfqpoint{2.915096in}{0.331635in}}%
\pgfpathlineto{\pgfqpoint{2.915096in}{1.841635in}}%
\pgfusepath{stroke}%
\end{pgfscope}%
\begin{pgfscope}%
\definecolor{textcolor}{rgb}{0.150000,0.150000,0.150000}%
\pgfsetstrokecolor{textcolor}%
\pgfsetfillcolor{textcolor}%
\pgftext[x=2.915096in,y=0.234413in,,top]{\color{textcolor}\rmfamily\fontsize{10.000000}{12.000000}\selectfont 20}%
\end{pgfscope}%
\begin{pgfscope}%
\pgfpathrectangle{\pgfqpoint{0.506467in}{0.331635in}}{\pgfqpoint{2.583333in}{1.510000in}}%
\pgfusepath{clip}%
\pgfsetroundcap%
\pgfsetroundjoin%
\pgfsetlinewidth{0.803000pt}%
\definecolor{currentstroke}{rgb}{1.000000,1.000000,1.000000}%
\pgfsetstrokecolor{currentstroke}%
\pgfsetdash{}{0pt}%
\pgfpathmoveto{\pgfqpoint{0.506467in}{0.466954in}}%
\pgfpathlineto{\pgfqpoint{3.089800in}{0.466954in}}%
\pgfusepath{stroke}%
\end{pgfscope}%
\begin{pgfscope}%
\definecolor{textcolor}{rgb}{0.150000,0.150000,0.150000}%
\pgfsetstrokecolor{textcolor}%
\pgfsetfillcolor{textcolor}%
\pgftext[x=0.100000in,y=0.414193in,left,base]{\color{textcolor}\rmfamily\fontsize{10.000000}{12.000000}\selectfont 0.00}%
\end{pgfscope}%
\begin{pgfscope}%
\pgfpathrectangle{\pgfqpoint{0.506467in}{0.331635in}}{\pgfqpoint{2.583333in}{1.510000in}}%
\pgfusepath{clip}%
\pgfsetroundcap%
\pgfsetroundjoin%
\pgfsetlinewidth{0.803000pt}%
\definecolor{currentstroke}{rgb}{1.000000,1.000000,1.000000}%
\pgfsetstrokecolor{currentstroke}%
\pgfsetdash{}{0pt}%
\pgfpathmoveto{\pgfqpoint{0.506467in}{0.793465in}}%
\pgfpathlineto{\pgfqpoint{3.089800in}{0.793465in}}%
\pgfusepath{stroke}%
\end{pgfscope}%
\begin{pgfscope}%
\definecolor{textcolor}{rgb}{0.150000,0.150000,0.150000}%
\pgfsetstrokecolor{textcolor}%
\pgfsetfillcolor{textcolor}%
\pgftext[x=0.100000in,y=0.740704in,left,base]{\color{textcolor}\rmfamily\fontsize{10.000000}{12.000000}\selectfont 0.25}%
\end{pgfscope}%
\begin{pgfscope}%
\pgfpathrectangle{\pgfqpoint{0.506467in}{0.331635in}}{\pgfqpoint{2.583333in}{1.510000in}}%
\pgfusepath{clip}%
\pgfsetroundcap%
\pgfsetroundjoin%
\pgfsetlinewidth{0.803000pt}%
\definecolor{currentstroke}{rgb}{1.000000,1.000000,1.000000}%
\pgfsetstrokecolor{currentstroke}%
\pgfsetdash{}{0pt}%
\pgfpathmoveto{\pgfqpoint{0.506467in}{1.119977in}}%
\pgfpathlineto{\pgfqpoint{3.089800in}{1.119977in}}%
\pgfusepath{stroke}%
\end{pgfscope}%
\begin{pgfscope}%
\definecolor{textcolor}{rgb}{0.150000,0.150000,0.150000}%
\pgfsetstrokecolor{textcolor}%
\pgfsetfillcolor{textcolor}%
\pgftext[x=0.100000in,y=1.067215in,left,base]{\color{textcolor}\rmfamily\fontsize{10.000000}{12.000000}\selectfont 0.50}%
\end{pgfscope}%
\begin{pgfscope}%
\pgfpathrectangle{\pgfqpoint{0.506467in}{0.331635in}}{\pgfqpoint{2.583333in}{1.510000in}}%
\pgfusepath{clip}%
\pgfsetroundcap%
\pgfsetroundjoin%
\pgfsetlinewidth{0.803000pt}%
\definecolor{currentstroke}{rgb}{1.000000,1.000000,1.000000}%
\pgfsetstrokecolor{currentstroke}%
\pgfsetdash{}{0pt}%
\pgfpathmoveto{\pgfqpoint{0.506467in}{1.446488in}}%
\pgfpathlineto{\pgfqpoint{3.089800in}{1.446488in}}%
\pgfusepath{stroke}%
\end{pgfscope}%
\begin{pgfscope}%
\definecolor{textcolor}{rgb}{0.150000,0.150000,0.150000}%
\pgfsetstrokecolor{textcolor}%
\pgfsetfillcolor{textcolor}%
\pgftext[x=0.100000in,y=1.393726in,left,base]{\color{textcolor}\rmfamily\fontsize{10.000000}{12.000000}\selectfont 0.75}%
\end{pgfscope}%
\begin{pgfscope}%
\pgfpathrectangle{\pgfqpoint{0.506467in}{0.331635in}}{\pgfqpoint{2.583333in}{1.510000in}}%
\pgfusepath{clip}%
\pgfsetroundcap%
\pgfsetroundjoin%
\pgfsetlinewidth{0.803000pt}%
\definecolor{currentstroke}{rgb}{1.000000,1.000000,1.000000}%
\pgfsetstrokecolor{currentstroke}%
\pgfsetdash{}{0pt}%
\pgfpathmoveto{\pgfqpoint{0.506467in}{1.772999in}}%
\pgfpathlineto{\pgfqpoint{3.089800in}{1.772999in}}%
\pgfusepath{stroke}%
\end{pgfscope}%
\begin{pgfscope}%
\definecolor{textcolor}{rgb}{0.150000,0.150000,0.150000}%
\pgfsetstrokecolor{textcolor}%
\pgfsetfillcolor{textcolor}%
\pgftext[x=0.100000in,y=1.720237in,left,base]{\color{textcolor}\rmfamily\fontsize{10.000000}{12.000000}\selectfont 1.00}%
\end{pgfscope}%
\begin{pgfscope}%
\pgfpathrectangle{\pgfqpoint{0.506467in}{0.331635in}}{\pgfqpoint{2.583333in}{1.510000in}}%
\pgfusepath{clip}%
\pgfsetbuttcap%
\pgfsetroundjoin%
\definecolor{currentfill}{rgb}{0.121569,0.466667,0.705882}%
\pgfsetfillcolor{currentfill}%
\pgfsetfillopacity{0.250000}%
\pgfsetlinewidth{1.003750pt}%
\definecolor{currentstroke}{rgb}{1.000000,1.000000,1.000000}%
\pgfsetstrokecolor{currentstroke}%
\pgfsetstrokeopacity{0.250000}%
\pgfsetdash{}{0pt}%
\pgfpathmoveto{\pgfqpoint{0.681171in}{0.532873in}}%
\pgfpathlineto{\pgfqpoint{0.681171in}{0.401036in}}%
\pgfpathlineto{\pgfqpoint{0.853012in}{0.400862in}}%
\pgfpathlineto{\pgfqpoint{0.967572in}{0.400817in}}%
\pgfpathlineto{\pgfqpoint{1.082132in}{0.400780in}}%
\pgfpathlineto{\pgfqpoint{1.196693in}{0.400774in}}%
\pgfpathlineto{\pgfqpoint{1.311253in}{0.400774in}}%
\pgfpathlineto{\pgfqpoint{1.425813in}{0.400750in}}%
\pgfpathlineto{\pgfqpoint{1.540373in}{0.400742in}}%
\pgfpathlineto{\pgfqpoint{1.654933in}{0.400738in}}%
\pgfpathlineto{\pgfqpoint{1.769494in}{0.400679in}}%
\pgfpathlineto{\pgfqpoint{1.884054in}{0.400669in}}%
\pgfpathlineto{\pgfqpoint{1.998614in}{0.400509in}}%
\pgfpathlineto{\pgfqpoint{2.113174in}{0.400431in}}%
\pgfpathlineto{\pgfqpoint{2.227735in}{0.400424in}}%
\pgfpathlineto{\pgfqpoint{2.342295in}{0.400315in}}%
\pgfpathlineto{\pgfqpoint{2.456855in}{0.400314in}}%
\pgfpathlineto{\pgfqpoint{2.571415in}{0.400286in}}%
\pgfpathlineto{\pgfqpoint{2.685976in}{0.400272in}}%
\pgfpathlineto{\pgfqpoint{2.800536in}{0.400272in}}%
\pgfpathlineto{\pgfqpoint{2.972376in}{0.400271in}}%
\pgfpathlineto{\pgfqpoint{2.972376in}{0.533637in}}%
\pgfpathlineto{\pgfqpoint{2.972376in}{0.533637in}}%
\pgfpathlineto{\pgfqpoint{2.800536in}{0.533637in}}%
\pgfpathlineto{\pgfqpoint{2.685976in}{0.533637in}}%
\pgfpathlineto{\pgfqpoint{2.571415in}{0.533623in}}%
\pgfpathlineto{\pgfqpoint{2.456855in}{0.533595in}}%
\pgfpathlineto{\pgfqpoint{2.342295in}{0.533594in}}%
\pgfpathlineto{\pgfqpoint{2.227735in}{0.533485in}}%
\pgfpathlineto{\pgfqpoint{2.113174in}{0.533478in}}%
\pgfpathlineto{\pgfqpoint{1.998614in}{0.533400in}}%
\pgfpathlineto{\pgfqpoint{1.884054in}{0.533240in}}%
\pgfpathlineto{\pgfqpoint{1.769494in}{0.533230in}}%
\pgfpathlineto{\pgfqpoint{1.654933in}{0.533171in}}%
\pgfpathlineto{\pgfqpoint{1.540373in}{0.533167in}}%
\pgfpathlineto{\pgfqpoint{1.425813in}{0.533159in}}%
\pgfpathlineto{\pgfqpoint{1.311253in}{0.533135in}}%
\pgfpathlineto{\pgfqpoint{1.196693in}{0.533135in}}%
\pgfpathlineto{\pgfqpoint{1.082132in}{0.533129in}}%
\pgfpathlineto{\pgfqpoint{0.967572in}{0.533091in}}%
\pgfpathlineto{\pgfqpoint{0.853012in}{0.533047in}}%
\pgfpathlineto{\pgfqpoint{0.681171in}{0.532873in}}%
\pgfpathclose%
\pgfusepath{stroke,fill}%
\end{pgfscope}%
\begin{pgfscope}%
\pgfpathrectangle{\pgfqpoint{0.506467in}{0.331635in}}{\pgfqpoint{2.583333in}{1.510000in}}%
\pgfusepath{clip}%
\pgfsetbuttcap%
\pgfsetroundjoin%
\pgfsetlinewidth{1.505625pt}%
\definecolor{currentstroke}{rgb}{0.000000,0.000000,0.000000}%
\pgfsetstrokecolor{currentstroke}%
\pgfsetdash{}{0pt}%
\pgfpathmoveto{\pgfqpoint{0.623891in}{0.466954in}}%
\pgfpathlineto{\pgfqpoint{0.623891in}{1.772999in}}%
\pgfusepath{stroke}%
\end{pgfscope}%
\begin{pgfscope}%
\pgfpathrectangle{\pgfqpoint{0.506467in}{0.331635in}}{\pgfqpoint{2.583333in}{1.510000in}}%
\pgfusepath{clip}%
\pgfsetbuttcap%
\pgfsetroundjoin%
\pgfsetlinewidth{1.505625pt}%
\definecolor{currentstroke}{rgb}{0.000000,0.000000,0.000000}%
\pgfsetstrokecolor{currentstroke}%
\pgfsetdash{}{0pt}%
\pgfpathmoveto{\pgfqpoint{0.738452in}{0.466954in}}%
\pgfpathlineto{\pgfqpoint{0.738452in}{0.534223in}}%
\pgfusepath{stroke}%
\end{pgfscope}%
\begin{pgfscope}%
\pgfpathrectangle{\pgfqpoint{0.506467in}{0.331635in}}{\pgfqpoint{2.583333in}{1.510000in}}%
\pgfusepath{clip}%
\pgfsetbuttcap%
\pgfsetroundjoin%
\pgfsetlinewidth{1.505625pt}%
\definecolor{currentstroke}{rgb}{0.000000,0.000000,0.000000}%
\pgfsetstrokecolor{currentstroke}%
\pgfsetdash{}{0pt}%
\pgfpathmoveto{\pgfqpoint{0.853012in}{0.466954in}}%
\pgfpathlineto{\pgfqpoint{0.853012in}{0.500834in}}%
\pgfusepath{stroke}%
\end{pgfscope}%
\begin{pgfscope}%
\pgfpathrectangle{\pgfqpoint{0.506467in}{0.331635in}}{\pgfqpoint{2.583333in}{1.510000in}}%
\pgfusepath{clip}%
\pgfsetbuttcap%
\pgfsetroundjoin%
\pgfsetlinewidth{1.505625pt}%
\definecolor{currentstroke}{rgb}{0.000000,0.000000,0.000000}%
\pgfsetstrokecolor{currentstroke}%
\pgfsetdash{}{0pt}%
\pgfpathmoveto{\pgfqpoint{0.967572in}{0.466954in}}%
\pgfpathlineto{\pgfqpoint{0.967572in}{0.498081in}}%
\pgfusepath{stroke}%
\end{pgfscope}%
\begin{pgfscope}%
\pgfpathrectangle{\pgfqpoint{0.506467in}{0.331635in}}{\pgfqpoint{2.583333in}{1.510000in}}%
\pgfusepath{clip}%
\pgfsetbuttcap%
\pgfsetroundjoin%
\pgfsetlinewidth{1.505625pt}%
\definecolor{currentstroke}{rgb}{0.000000,0.000000,0.000000}%
\pgfsetstrokecolor{currentstroke}%
\pgfsetdash{}{0pt}%
\pgfpathmoveto{\pgfqpoint{1.082132in}{0.466954in}}%
\pgfpathlineto{\pgfqpoint{1.082132in}{0.479279in}}%
\pgfusepath{stroke}%
\end{pgfscope}%
\begin{pgfscope}%
\pgfpathrectangle{\pgfqpoint{0.506467in}{0.331635in}}{\pgfqpoint{2.583333in}{1.510000in}}%
\pgfusepath{clip}%
\pgfsetbuttcap%
\pgfsetroundjoin%
\pgfsetlinewidth{1.505625pt}%
\definecolor{currentstroke}{rgb}{0.000000,0.000000,0.000000}%
\pgfsetstrokecolor{currentstroke}%
\pgfsetdash{}{0pt}%
\pgfpathmoveto{\pgfqpoint{1.196693in}{0.466954in}}%
\pgfpathlineto{\pgfqpoint{1.196693in}{0.468063in}}%
\pgfusepath{stroke}%
\end{pgfscope}%
\begin{pgfscope}%
\pgfpathrectangle{\pgfqpoint{0.506467in}{0.331635in}}{\pgfqpoint{2.583333in}{1.510000in}}%
\pgfusepath{clip}%
\pgfsetbuttcap%
\pgfsetroundjoin%
\pgfsetlinewidth{1.505625pt}%
\definecolor{currentstroke}{rgb}{0.000000,0.000000,0.000000}%
\pgfsetstrokecolor{currentstroke}%
\pgfsetdash{}{0pt}%
\pgfpathmoveto{\pgfqpoint{1.311253in}{0.466954in}}%
\pgfpathlineto{\pgfqpoint{1.311253in}{0.491989in}}%
\pgfusepath{stroke}%
\end{pgfscope}%
\begin{pgfscope}%
\pgfpathrectangle{\pgfqpoint{0.506467in}{0.331635in}}{\pgfqpoint{2.583333in}{1.510000in}}%
\pgfusepath{clip}%
\pgfsetbuttcap%
\pgfsetroundjoin%
\pgfsetlinewidth{1.505625pt}%
\definecolor{currentstroke}{rgb}{0.000000,0.000000,0.000000}%
\pgfsetstrokecolor{currentstroke}%
\pgfsetdash{}{0pt}%
\pgfpathmoveto{\pgfqpoint{1.425813in}{0.466954in}}%
\pgfpathlineto{\pgfqpoint{1.425813in}{0.452265in}}%
\pgfusepath{stroke}%
\end{pgfscope}%
\begin{pgfscope}%
\pgfpathrectangle{\pgfqpoint{0.506467in}{0.331635in}}{\pgfqpoint{2.583333in}{1.510000in}}%
\pgfusepath{clip}%
\pgfsetbuttcap%
\pgfsetroundjoin%
\pgfsetlinewidth{1.505625pt}%
\definecolor{currentstroke}{rgb}{0.000000,0.000000,0.000000}%
\pgfsetstrokecolor{currentstroke}%
\pgfsetdash{}{0pt}%
\pgfpathmoveto{\pgfqpoint{1.540373in}{0.466954in}}%
\pgfpathlineto{\pgfqpoint{1.540373in}{0.457182in}}%
\pgfusepath{stroke}%
\end{pgfscope}%
\begin{pgfscope}%
\pgfpathrectangle{\pgfqpoint{0.506467in}{0.331635in}}{\pgfqpoint{2.583333in}{1.510000in}}%
\pgfusepath{clip}%
\pgfsetbuttcap%
\pgfsetroundjoin%
\pgfsetlinewidth{1.505625pt}%
\definecolor{currentstroke}{rgb}{0.000000,0.000000,0.000000}%
\pgfsetstrokecolor{currentstroke}%
\pgfsetdash{}{0pt}%
\pgfpathmoveto{\pgfqpoint{1.654933in}{0.466954in}}%
\pgfpathlineto{\pgfqpoint{1.654933in}{0.506058in}}%
\pgfusepath{stroke}%
\end{pgfscope}%
\begin{pgfscope}%
\pgfpathrectangle{\pgfqpoint{0.506467in}{0.331635in}}{\pgfqpoint{2.583333in}{1.510000in}}%
\pgfusepath{clip}%
\pgfsetbuttcap%
\pgfsetroundjoin%
\pgfsetlinewidth{1.505625pt}%
\definecolor{currentstroke}{rgb}{0.000000,0.000000,0.000000}%
\pgfsetstrokecolor{currentstroke}%
\pgfsetdash{}{0pt}%
\pgfpathmoveto{\pgfqpoint{1.769494in}{0.466954in}}%
\pgfpathlineto{\pgfqpoint{1.769494in}{0.450638in}}%
\pgfusepath{stroke}%
\end{pgfscope}%
\begin{pgfscope}%
\pgfpathrectangle{\pgfqpoint{0.506467in}{0.331635in}}{\pgfqpoint{2.583333in}{1.510000in}}%
\pgfusepath{clip}%
\pgfsetbuttcap%
\pgfsetroundjoin%
\pgfsetlinewidth{1.505625pt}%
\definecolor{currentstroke}{rgb}{0.000000,0.000000,0.000000}%
\pgfsetstrokecolor{currentstroke}%
\pgfsetdash{}{0pt}%
\pgfpathmoveto{\pgfqpoint{1.884054in}{0.466954in}}%
\pgfpathlineto{\pgfqpoint{1.884054in}{0.531602in}}%
\pgfusepath{stroke}%
\end{pgfscope}%
\begin{pgfscope}%
\pgfpathrectangle{\pgfqpoint{0.506467in}{0.331635in}}{\pgfqpoint{2.583333in}{1.510000in}}%
\pgfusepath{clip}%
\pgfsetbuttcap%
\pgfsetroundjoin%
\pgfsetlinewidth{1.505625pt}%
\definecolor{currentstroke}{rgb}{0.000000,0.000000,0.000000}%
\pgfsetstrokecolor{currentstroke}%
\pgfsetdash{}{0pt}%
\pgfpathmoveto{\pgfqpoint{1.998614in}{0.466954in}}%
\pgfpathlineto{\pgfqpoint{1.998614in}{0.512094in}}%
\pgfusepath{stroke}%
\end{pgfscope}%
\begin{pgfscope}%
\pgfpathrectangle{\pgfqpoint{0.506467in}{0.331635in}}{\pgfqpoint{2.583333in}{1.510000in}}%
\pgfusepath{clip}%
\pgfsetbuttcap%
\pgfsetroundjoin%
\pgfsetlinewidth{1.505625pt}%
\definecolor{currentstroke}{rgb}{0.000000,0.000000,0.000000}%
\pgfsetstrokecolor{currentstroke}%
\pgfsetdash{}{0pt}%
\pgfpathmoveto{\pgfqpoint{2.113174in}{0.466954in}}%
\pgfpathlineto{\pgfqpoint{2.113174in}{0.480143in}}%
\pgfusepath{stroke}%
\end{pgfscope}%
\begin{pgfscope}%
\pgfpathrectangle{\pgfqpoint{0.506467in}{0.331635in}}{\pgfqpoint{2.583333in}{1.510000in}}%
\pgfusepath{clip}%
\pgfsetbuttcap%
\pgfsetroundjoin%
\pgfsetlinewidth{1.505625pt}%
\definecolor{currentstroke}{rgb}{0.000000,0.000000,0.000000}%
\pgfsetstrokecolor{currentstroke}%
\pgfsetdash{}{0pt}%
\pgfpathmoveto{\pgfqpoint{2.227735in}{0.466954in}}%
\pgfpathlineto{\pgfqpoint{2.227735in}{0.520374in}}%
\pgfusepath{stroke}%
\end{pgfscope}%
\begin{pgfscope}%
\pgfpathrectangle{\pgfqpoint{0.506467in}{0.331635in}}{\pgfqpoint{2.583333in}{1.510000in}}%
\pgfusepath{clip}%
\pgfsetbuttcap%
\pgfsetroundjoin%
\pgfsetlinewidth{1.505625pt}%
\definecolor{currentstroke}{rgb}{0.000000,0.000000,0.000000}%
\pgfsetstrokecolor{currentstroke}%
\pgfsetdash{}{0pt}%
\pgfpathmoveto{\pgfqpoint{2.342295in}{0.466954in}}%
\pgfpathlineto{\pgfqpoint{2.342295in}{0.463454in}}%
\pgfusepath{stroke}%
\end{pgfscope}%
\begin{pgfscope}%
\pgfpathrectangle{\pgfqpoint{0.506467in}{0.331635in}}{\pgfqpoint{2.583333in}{1.510000in}}%
\pgfusepath{clip}%
\pgfsetbuttcap%
\pgfsetroundjoin%
\pgfsetlinewidth{1.505625pt}%
\definecolor{currentstroke}{rgb}{0.000000,0.000000,0.000000}%
\pgfsetstrokecolor{currentstroke}%
\pgfsetdash{}{0pt}%
\pgfpathmoveto{\pgfqpoint{2.456855in}{0.466954in}}%
\pgfpathlineto{\pgfqpoint{2.456855in}{0.439680in}}%
\pgfusepath{stroke}%
\end{pgfscope}%
\begin{pgfscope}%
\pgfpathrectangle{\pgfqpoint{0.506467in}{0.331635in}}{\pgfqpoint{2.583333in}{1.510000in}}%
\pgfusepath{clip}%
\pgfsetbuttcap%
\pgfsetroundjoin%
\pgfsetlinewidth{1.505625pt}%
\definecolor{currentstroke}{rgb}{0.000000,0.000000,0.000000}%
\pgfsetstrokecolor{currentstroke}%
\pgfsetdash{}{0pt}%
\pgfpathmoveto{\pgfqpoint{2.571415in}{0.466954in}}%
\pgfpathlineto{\pgfqpoint{2.571415in}{0.447681in}}%
\pgfusepath{stroke}%
\end{pgfscope}%
\begin{pgfscope}%
\pgfpathrectangle{\pgfqpoint{0.506467in}{0.331635in}}{\pgfqpoint{2.583333in}{1.510000in}}%
\pgfusepath{clip}%
\pgfsetbuttcap%
\pgfsetroundjoin%
\pgfsetlinewidth{1.505625pt}%
\definecolor{currentstroke}{rgb}{0.000000,0.000000,0.000000}%
\pgfsetstrokecolor{currentstroke}%
\pgfsetdash{}{0pt}%
\pgfpathmoveto{\pgfqpoint{2.685976in}{0.466954in}}%
\pgfpathlineto{\pgfqpoint{2.685976in}{0.468569in}}%
\pgfusepath{stroke}%
\end{pgfscope}%
\begin{pgfscope}%
\pgfpathrectangle{\pgfqpoint{0.506467in}{0.331635in}}{\pgfqpoint{2.583333in}{1.510000in}}%
\pgfusepath{clip}%
\pgfsetbuttcap%
\pgfsetroundjoin%
\pgfsetlinewidth{1.505625pt}%
\definecolor{currentstroke}{rgb}{0.000000,0.000000,0.000000}%
\pgfsetstrokecolor{currentstroke}%
\pgfsetdash{}{0pt}%
\pgfpathmoveto{\pgfqpoint{2.800536in}{0.466954in}}%
\pgfpathlineto{\pgfqpoint{2.800536in}{0.468972in}}%
\pgfusepath{stroke}%
\end{pgfscope}%
\begin{pgfscope}%
\pgfpathrectangle{\pgfqpoint{0.506467in}{0.331635in}}{\pgfqpoint{2.583333in}{1.510000in}}%
\pgfusepath{clip}%
\pgfsetbuttcap%
\pgfsetroundjoin%
\pgfsetlinewidth{1.505625pt}%
\definecolor{currentstroke}{rgb}{0.000000,0.000000,0.000000}%
\pgfsetstrokecolor{currentstroke}%
\pgfsetdash{}{0pt}%
\pgfpathmoveto{\pgfqpoint{2.915096in}{0.466954in}}%
\pgfpathlineto{\pgfqpoint{2.915096in}{0.464769in}}%
\pgfusepath{stroke}%
\end{pgfscope}%
\begin{pgfscope}%
\pgfpathrectangle{\pgfqpoint{0.506467in}{0.331635in}}{\pgfqpoint{2.583333in}{1.510000in}}%
\pgfusepath{clip}%
\pgfsetroundcap%
\pgfsetroundjoin%
\pgfsetlinewidth{1.505625pt}%
\definecolor{currentstroke}{rgb}{0.839216,0.152941,0.156863}%
\pgfsetstrokecolor{currentstroke}%
\pgfsetdash{}{0pt}%
\pgfpathmoveto{\pgfqpoint{0.506467in}{0.466954in}}%
\pgfpathlineto{\pgfqpoint{3.089800in}{0.466954in}}%
\pgfusepath{stroke}%
\end{pgfscope}%
\begin{pgfscope}%
\pgfpathrectangle{\pgfqpoint{0.506467in}{0.331635in}}{\pgfqpoint{2.583333in}{1.510000in}}%
\pgfusepath{clip}%
\pgfsetbuttcap%
\pgfsetroundjoin%
\definecolor{currentfill}{rgb}{0.839216,0.152941,0.156863}%
\pgfsetfillcolor{currentfill}%
\pgfsetlinewidth{1.003750pt}%
\definecolor{currentstroke}{rgb}{0.839216,0.152941,0.156863}%
\pgfsetstrokecolor{currentstroke}%
\pgfsetdash{}{0pt}%
\pgfsys@defobject{currentmarker}{\pgfqpoint{-0.034722in}{-0.034722in}}{\pgfqpoint{0.034722in}{0.034722in}}{%
\pgfpathmoveto{\pgfqpoint{0.000000in}{-0.034722in}}%
\pgfpathcurveto{\pgfqpoint{0.009208in}{-0.034722in}}{\pgfqpoint{0.018041in}{-0.031064in}}{\pgfqpoint{0.024552in}{-0.024552in}}%
\pgfpathcurveto{\pgfqpoint{0.031064in}{-0.018041in}}{\pgfqpoint{0.034722in}{-0.009208in}}{\pgfqpoint{0.034722in}{0.000000in}}%
\pgfpathcurveto{\pgfqpoint{0.034722in}{0.009208in}}{\pgfqpoint{0.031064in}{0.018041in}}{\pgfqpoint{0.024552in}{0.024552in}}%
\pgfpathcurveto{\pgfqpoint{0.018041in}{0.031064in}}{\pgfqpoint{0.009208in}{0.034722in}}{\pgfqpoint{0.000000in}{0.034722in}}%
\pgfpathcurveto{\pgfqpoint{-0.009208in}{0.034722in}}{\pgfqpoint{-0.018041in}{0.031064in}}{\pgfqpoint{-0.024552in}{0.024552in}}%
\pgfpathcurveto{\pgfqpoint{-0.031064in}{0.018041in}}{\pgfqpoint{-0.034722in}{0.009208in}}{\pgfqpoint{-0.034722in}{0.000000in}}%
\pgfpathcurveto{\pgfqpoint{-0.034722in}{-0.009208in}}{\pgfqpoint{-0.031064in}{-0.018041in}}{\pgfqpoint{-0.024552in}{-0.024552in}}%
\pgfpathcurveto{\pgfqpoint{-0.018041in}{-0.031064in}}{\pgfqpoint{-0.009208in}{-0.034722in}}{\pgfqpoint{0.000000in}{-0.034722in}}%
\pgfpathclose%
\pgfusepath{stroke,fill}%
}%
\begin{pgfscope}%
\pgfsys@transformshift{0.623891in}{1.772999in}%
\pgfsys@useobject{currentmarker}{}%
\end{pgfscope}%
\begin{pgfscope}%
\pgfsys@transformshift{0.738452in}{0.534223in}%
\pgfsys@useobject{currentmarker}{}%
\end{pgfscope}%
\begin{pgfscope}%
\pgfsys@transformshift{0.853012in}{0.500834in}%
\pgfsys@useobject{currentmarker}{}%
\end{pgfscope}%
\begin{pgfscope}%
\pgfsys@transformshift{0.967572in}{0.498081in}%
\pgfsys@useobject{currentmarker}{}%
\end{pgfscope}%
\begin{pgfscope}%
\pgfsys@transformshift{1.082132in}{0.479279in}%
\pgfsys@useobject{currentmarker}{}%
\end{pgfscope}%
\begin{pgfscope}%
\pgfsys@transformshift{1.196693in}{0.468063in}%
\pgfsys@useobject{currentmarker}{}%
\end{pgfscope}%
\begin{pgfscope}%
\pgfsys@transformshift{1.311253in}{0.491989in}%
\pgfsys@useobject{currentmarker}{}%
\end{pgfscope}%
\begin{pgfscope}%
\pgfsys@transformshift{1.425813in}{0.452265in}%
\pgfsys@useobject{currentmarker}{}%
\end{pgfscope}%
\begin{pgfscope}%
\pgfsys@transformshift{1.540373in}{0.457182in}%
\pgfsys@useobject{currentmarker}{}%
\end{pgfscope}%
\begin{pgfscope}%
\pgfsys@transformshift{1.654933in}{0.506058in}%
\pgfsys@useobject{currentmarker}{}%
\end{pgfscope}%
\begin{pgfscope}%
\pgfsys@transformshift{1.769494in}{0.450638in}%
\pgfsys@useobject{currentmarker}{}%
\end{pgfscope}%
\begin{pgfscope}%
\pgfsys@transformshift{1.884054in}{0.531602in}%
\pgfsys@useobject{currentmarker}{}%
\end{pgfscope}%
\begin{pgfscope}%
\pgfsys@transformshift{1.998614in}{0.512094in}%
\pgfsys@useobject{currentmarker}{}%
\end{pgfscope}%
\begin{pgfscope}%
\pgfsys@transformshift{2.113174in}{0.480143in}%
\pgfsys@useobject{currentmarker}{}%
\end{pgfscope}%
\begin{pgfscope}%
\pgfsys@transformshift{2.227735in}{0.520374in}%
\pgfsys@useobject{currentmarker}{}%
\end{pgfscope}%
\begin{pgfscope}%
\pgfsys@transformshift{2.342295in}{0.463454in}%
\pgfsys@useobject{currentmarker}{}%
\end{pgfscope}%
\begin{pgfscope}%
\pgfsys@transformshift{2.456855in}{0.439680in}%
\pgfsys@useobject{currentmarker}{}%
\end{pgfscope}%
\begin{pgfscope}%
\pgfsys@transformshift{2.571415in}{0.447681in}%
\pgfsys@useobject{currentmarker}{}%
\end{pgfscope}%
\begin{pgfscope}%
\pgfsys@transformshift{2.685976in}{0.468569in}%
\pgfsys@useobject{currentmarker}{}%
\end{pgfscope}%
\begin{pgfscope}%
\pgfsys@transformshift{2.800536in}{0.468972in}%
\pgfsys@useobject{currentmarker}{}%
\end{pgfscope}%
\begin{pgfscope}%
\pgfsys@transformshift{2.915096in}{0.464769in}%
\pgfsys@useobject{currentmarker}{}%
\end{pgfscope}%
\end{pgfscope}%
\begin{pgfscope}%
\pgfsetrectcap%
\pgfsetmiterjoin%
\pgfsetlinewidth{0.803000pt}%
\definecolor{currentstroke}{rgb}{1.000000,1.000000,1.000000}%
\pgfsetstrokecolor{currentstroke}%
\pgfsetdash{}{0pt}%
\pgfpathmoveto{\pgfqpoint{0.506467in}{0.331635in}}%
\pgfpathlineto{\pgfqpoint{0.506467in}{1.841635in}}%
\pgfusepath{stroke}%
\end{pgfscope}%
\begin{pgfscope}%
\pgfsetrectcap%
\pgfsetmiterjoin%
\pgfsetlinewidth{0.803000pt}%
\definecolor{currentstroke}{rgb}{1.000000,1.000000,1.000000}%
\pgfsetstrokecolor{currentstroke}%
\pgfsetdash{}{0pt}%
\pgfpathmoveto{\pgfqpoint{3.089800in}{0.331635in}}%
\pgfpathlineto{\pgfqpoint{3.089800in}{1.841635in}}%
\pgfusepath{stroke}%
\end{pgfscope}%
\begin{pgfscope}%
\pgfsetrectcap%
\pgfsetmiterjoin%
\pgfsetlinewidth{0.803000pt}%
\definecolor{currentstroke}{rgb}{1.000000,1.000000,1.000000}%
\pgfsetstrokecolor{currentstroke}%
\pgfsetdash{}{0pt}%
\pgfpathmoveto{\pgfqpoint{0.506467in}{0.331635in}}%
\pgfpathlineto{\pgfqpoint{3.089800in}{0.331635in}}%
\pgfusepath{stroke}%
\end{pgfscope}%
\begin{pgfscope}%
\pgfsetrectcap%
\pgfsetmiterjoin%
\pgfsetlinewidth{0.803000pt}%
\definecolor{currentstroke}{rgb}{1.000000,1.000000,1.000000}%
\pgfsetstrokecolor{currentstroke}%
\pgfsetdash{}{0pt}%
\pgfpathmoveto{\pgfqpoint{0.506467in}{1.841635in}}%
\pgfpathlineto{\pgfqpoint{3.089800in}{1.841635in}}%
\pgfusepath{stroke}%
\end{pgfscope}%
\begin{pgfscope}%
\definecolor{textcolor}{rgb}{0.150000,0.150000,0.150000}%
\pgfsetstrokecolor{textcolor}%
\pgfsetfillcolor{textcolor}%
\pgftext[x=1.798134in,y=1.924968in,,base]{\color{textcolor}\rmfamily\fontsize{12.000000}{14.400000}\selectfont Autocorrelation INTC\^2}%
\end{pgfscope}%
\begin{pgfscope}%
\pgfsetbuttcap%
\pgfsetmiterjoin%
\definecolor{currentfill}{rgb}{0.917647,0.917647,0.949020}%
\pgfsetfillcolor{currentfill}%
\pgfsetlinewidth{0.000000pt}%
\definecolor{currentstroke}{rgb}{0.000000,0.000000,0.000000}%
\pgfsetstrokecolor{currentstroke}%
\pgfsetstrokeopacity{0.000000}%
\pgfsetdash{}{0pt}%
\pgfpathmoveto{\pgfqpoint{4.123134in}{0.331635in}}%
\pgfpathlineto{\pgfqpoint{6.706467in}{0.331635in}}%
\pgfpathlineto{\pgfqpoint{6.706467in}{1.841635in}}%
\pgfpathlineto{\pgfqpoint{4.123134in}{1.841635in}}%
\pgfpathclose%
\pgfusepath{fill}%
\end{pgfscope}%
\begin{pgfscope}%
\pgfpathrectangle{\pgfqpoint{4.123134in}{0.331635in}}{\pgfqpoint{2.583333in}{1.510000in}}%
\pgfusepath{clip}%
\pgfsetroundcap%
\pgfsetroundjoin%
\pgfsetlinewidth{0.803000pt}%
\definecolor{currentstroke}{rgb}{1.000000,1.000000,1.000000}%
\pgfsetstrokecolor{currentstroke}%
\pgfsetdash{}{0pt}%
\pgfpathmoveto{\pgfqpoint{4.240558in}{0.331635in}}%
\pgfpathlineto{\pgfqpoint{4.240558in}{1.841635in}}%
\pgfusepath{stroke}%
\end{pgfscope}%
\begin{pgfscope}%
\definecolor{textcolor}{rgb}{0.150000,0.150000,0.150000}%
\pgfsetstrokecolor{textcolor}%
\pgfsetfillcolor{textcolor}%
\pgftext[x=4.240558in,y=0.234413in,,top]{\color{textcolor}\rmfamily\fontsize{10.000000}{12.000000}\selectfont 0}%
\end{pgfscope}%
\begin{pgfscope}%
\pgfpathrectangle{\pgfqpoint{4.123134in}{0.331635in}}{\pgfqpoint{2.583333in}{1.510000in}}%
\pgfusepath{clip}%
\pgfsetroundcap%
\pgfsetroundjoin%
\pgfsetlinewidth{0.803000pt}%
\definecolor{currentstroke}{rgb}{1.000000,1.000000,1.000000}%
\pgfsetstrokecolor{currentstroke}%
\pgfsetdash{}{0pt}%
\pgfpathmoveto{\pgfqpoint{4.813359in}{0.331635in}}%
\pgfpathlineto{\pgfqpoint{4.813359in}{1.841635in}}%
\pgfusepath{stroke}%
\end{pgfscope}%
\begin{pgfscope}%
\definecolor{textcolor}{rgb}{0.150000,0.150000,0.150000}%
\pgfsetstrokecolor{textcolor}%
\pgfsetfillcolor{textcolor}%
\pgftext[x=4.813359in,y=0.234413in,,top]{\color{textcolor}\rmfamily\fontsize{10.000000}{12.000000}\selectfont 5}%
\end{pgfscope}%
\begin{pgfscope}%
\pgfpathrectangle{\pgfqpoint{4.123134in}{0.331635in}}{\pgfqpoint{2.583333in}{1.510000in}}%
\pgfusepath{clip}%
\pgfsetroundcap%
\pgfsetroundjoin%
\pgfsetlinewidth{0.803000pt}%
\definecolor{currentstroke}{rgb}{1.000000,1.000000,1.000000}%
\pgfsetstrokecolor{currentstroke}%
\pgfsetdash{}{0pt}%
\pgfpathmoveto{\pgfqpoint{5.386160in}{0.331635in}}%
\pgfpathlineto{\pgfqpoint{5.386160in}{1.841635in}}%
\pgfusepath{stroke}%
\end{pgfscope}%
\begin{pgfscope}%
\definecolor{textcolor}{rgb}{0.150000,0.150000,0.150000}%
\pgfsetstrokecolor{textcolor}%
\pgfsetfillcolor{textcolor}%
\pgftext[x=5.386160in,y=0.234413in,,top]{\color{textcolor}\rmfamily\fontsize{10.000000}{12.000000}\selectfont 10}%
\end{pgfscope}%
\begin{pgfscope}%
\pgfpathrectangle{\pgfqpoint{4.123134in}{0.331635in}}{\pgfqpoint{2.583333in}{1.510000in}}%
\pgfusepath{clip}%
\pgfsetroundcap%
\pgfsetroundjoin%
\pgfsetlinewidth{0.803000pt}%
\definecolor{currentstroke}{rgb}{1.000000,1.000000,1.000000}%
\pgfsetstrokecolor{currentstroke}%
\pgfsetdash{}{0pt}%
\pgfpathmoveto{\pgfqpoint{5.958962in}{0.331635in}}%
\pgfpathlineto{\pgfqpoint{5.958962in}{1.841635in}}%
\pgfusepath{stroke}%
\end{pgfscope}%
\begin{pgfscope}%
\definecolor{textcolor}{rgb}{0.150000,0.150000,0.150000}%
\pgfsetstrokecolor{textcolor}%
\pgfsetfillcolor{textcolor}%
\pgftext[x=5.958962in,y=0.234413in,,top]{\color{textcolor}\rmfamily\fontsize{10.000000}{12.000000}\selectfont 15}%
\end{pgfscope}%
\begin{pgfscope}%
\pgfpathrectangle{\pgfqpoint{4.123134in}{0.331635in}}{\pgfqpoint{2.583333in}{1.510000in}}%
\pgfusepath{clip}%
\pgfsetroundcap%
\pgfsetroundjoin%
\pgfsetlinewidth{0.803000pt}%
\definecolor{currentstroke}{rgb}{1.000000,1.000000,1.000000}%
\pgfsetstrokecolor{currentstroke}%
\pgfsetdash{}{0pt}%
\pgfpathmoveto{\pgfqpoint{6.531763in}{0.331635in}}%
\pgfpathlineto{\pgfqpoint{6.531763in}{1.841635in}}%
\pgfusepath{stroke}%
\end{pgfscope}%
\begin{pgfscope}%
\definecolor{textcolor}{rgb}{0.150000,0.150000,0.150000}%
\pgfsetstrokecolor{textcolor}%
\pgfsetfillcolor{textcolor}%
\pgftext[x=6.531763in,y=0.234413in,,top]{\color{textcolor}\rmfamily\fontsize{10.000000}{12.000000}\selectfont 20}%
\end{pgfscope}%
\begin{pgfscope}%
\pgfpathrectangle{\pgfqpoint{4.123134in}{0.331635in}}{\pgfqpoint{2.583333in}{1.510000in}}%
\pgfusepath{clip}%
\pgfsetroundcap%
\pgfsetroundjoin%
\pgfsetlinewidth{0.803000pt}%
\definecolor{currentstroke}{rgb}{1.000000,1.000000,1.000000}%
\pgfsetstrokecolor{currentstroke}%
\pgfsetdash{}{0pt}%
\pgfpathmoveto{\pgfqpoint{4.123134in}{0.466226in}}%
\pgfpathlineto{\pgfqpoint{6.706467in}{0.466226in}}%
\pgfusepath{stroke}%
\end{pgfscope}%
\begin{pgfscope}%
\definecolor{textcolor}{rgb}{0.150000,0.150000,0.150000}%
\pgfsetstrokecolor{textcolor}%
\pgfsetfillcolor{textcolor}%
\pgftext[x=3.716667in,y=0.413465in,left,base]{\color{textcolor}\rmfamily\fontsize{10.000000}{12.000000}\selectfont 0.00}%
\end{pgfscope}%
\begin{pgfscope}%
\pgfpathrectangle{\pgfqpoint{4.123134in}{0.331635in}}{\pgfqpoint{2.583333in}{1.510000in}}%
\pgfusepath{clip}%
\pgfsetroundcap%
\pgfsetroundjoin%
\pgfsetlinewidth{0.803000pt}%
\definecolor{currentstroke}{rgb}{1.000000,1.000000,1.000000}%
\pgfsetstrokecolor{currentstroke}%
\pgfsetdash{}{0pt}%
\pgfpathmoveto{\pgfqpoint{4.123134in}{0.792919in}}%
\pgfpathlineto{\pgfqpoint{6.706467in}{0.792919in}}%
\pgfusepath{stroke}%
\end{pgfscope}%
\begin{pgfscope}%
\definecolor{textcolor}{rgb}{0.150000,0.150000,0.150000}%
\pgfsetstrokecolor{textcolor}%
\pgfsetfillcolor{textcolor}%
\pgftext[x=3.716667in,y=0.740158in,left,base]{\color{textcolor}\rmfamily\fontsize{10.000000}{12.000000}\selectfont 0.25}%
\end{pgfscope}%
\begin{pgfscope}%
\pgfpathrectangle{\pgfqpoint{4.123134in}{0.331635in}}{\pgfqpoint{2.583333in}{1.510000in}}%
\pgfusepath{clip}%
\pgfsetroundcap%
\pgfsetroundjoin%
\pgfsetlinewidth{0.803000pt}%
\definecolor{currentstroke}{rgb}{1.000000,1.000000,1.000000}%
\pgfsetstrokecolor{currentstroke}%
\pgfsetdash{}{0pt}%
\pgfpathmoveto{\pgfqpoint{4.123134in}{1.119612in}}%
\pgfpathlineto{\pgfqpoint{6.706467in}{1.119612in}}%
\pgfusepath{stroke}%
\end{pgfscope}%
\begin{pgfscope}%
\definecolor{textcolor}{rgb}{0.150000,0.150000,0.150000}%
\pgfsetstrokecolor{textcolor}%
\pgfsetfillcolor{textcolor}%
\pgftext[x=3.716667in,y=1.066851in,left,base]{\color{textcolor}\rmfamily\fontsize{10.000000}{12.000000}\selectfont 0.50}%
\end{pgfscope}%
\begin{pgfscope}%
\pgfpathrectangle{\pgfqpoint{4.123134in}{0.331635in}}{\pgfqpoint{2.583333in}{1.510000in}}%
\pgfusepath{clip}%
\pgfsetroundcap%
\pgfsetroundjoin%
\pgfsetlinewidth{0.803000pt}%
\definecolor{currentstroke}{rgb}{1.000000,1.000000,1.000000}%
\pgfsetstrokecolor{currentstroke}%
\pgfsetdash{}{0pt}%
\pgfpathmoveto{\pgfqpoint{4.123134in}{1.446306in}}%
\pgfpathlineto{\pgfqpoint{6.706467in}{1.446306in}}%
\pgfusepath{stroke}%
\end{pgfscope}%
\begin{pgfscope}%
\definecolor{textcolor}{rgb}{0.150000,0.150000,0.150000}%
\pgfsetstrokecolor{textcolor}%
\pgfsetfillcolor{textcolor}%
\pgftext[x=3.716667in,y=1.393544in,left,base]{\color{textcolor}\rmfamily\fontsize{10.000000}{12.000000}\selectfont 0.75}%
\end{pgfscope}%
\begin{pgfscope}%
\pgfpathrectangle{\pgfqpoint{4.123134in}{0.331635in}}{\pgfqpoint{2.583333in}{1.510000in}}%
\pgfusepath{clip}%
\pgfsetroundcap%
\pgfsetroundjoin%
\pgfsetlinewidth{0.803000pt}%
\definecolor{currentstroke}{rgb}{1.000000,1.000000,1.000000}%
\pgfsetstrokecolor{currentstroke}%
\pgfsetdash{}{0pt}%
\pgfpathmoveto{\pgfqpoint{4.123134in}{1.772999in}}%
\pgfpathlineto{\pgfqpoint{6.706467in}{1.772999in}}%
\pgfusepath{stroke}%
\end{pgfscope}%
\begin{pgfscope}%
\definecolor{textcolor}{rgb}{0.150000,0.150000,0.150000}%
\pgfsetstrokecolor{textcolor}%
\pgfsetfillcolor{textcolor}%
\pgftext[x=3.716667in,y=1.720237in,left,base]{\color{textcolor}\rmfamily\fontsize{10.000000}{12.000000}\selectfont 1.00}%
\end{pgfscope}%
\begin{pgfscope}%
\pgfpathrectangle{\pgfqpoint{4.123134in}{0.331635in}}{\pgfqpoint{2.583333in}{1.510000in}}%
\pgfusepath{clip}%
\pgfsetbuttcap%
\pgfsetroundjoin%
\definecolor{currentfill}{rgb}{0.121569,0.466667,0.705882}%
\pgfsetfillcolor{currentfill}%
\pgfsetfillopacity{0.250000}%
\pgfsetlinewidth{1.003750pt}%
\definecolor{currentstroke}{rgb}{1.000000,1.000000,1.000000}%
\pgfsetstrokecolor{currentstroke}%
\pgfsetstrokeopacity{0.250000}%
\pgfsetdash{}{0pt}%
\pgfpathmoveto{\pgfqpoint{4.297838in}{0.532181in}}%
\pgfpathlineto{\pgfqpoint{4.297838in}{0.400271in}}%
\pgfpathlineto{\pgfqpoint{4.469678in}{0.400271in}}%
\pgfpathlineto{\pgfqpoint{4.584239in}{0.400271in}}%
\pgfpathlineto{\pgfqpoint{4.698799in}{0.400271in}}%
\pgfpathlineto{\pgfqpoint{4.813359in}{0.400271in}}%
\pgfpathlineto{\pgfqpoint{4.927919in}{0.400271in}}%
\pgfpathlineto{\pgfqpoint{5.042480in}{0.400271in}}%
\pgfpathlineto{\pgfqpoint{5.157040in}{0.400271in}}%
\pgfpathlineto{\pgfqpoint{5.271600in}{0.400271in}}%
\pgfpathlineto{\pgfqpoint{5.386160in}{0.400271in}}%
\pgfpathlineto{\pgfqpoint{5.500721in}{0.400271in}}%
\pgfpathlineto{\pgfqpoint{5.615281in}{0.400271in}}%
\pgfpathlineto{\pgfqpoint{5.729841in}{0.400271in}}%
\pgfpathlineto{\pgfqpoint{5.844401in}{0.400271in}}%
\pgfpathlineto{\pgfqpoint{5.958962in}{0.400271in}}%
\pgfpathlineto{\pgfqpoint{6.073522in}{0.400271in}}%
\pgfpathlineto{\pgfqpoint{6.188082in}{0.400271in}}%
\pgfpathlineto{\pgfqpoint{6.302642in}{0.400271in}}%
\pgfpathlineto{\pgfqpoint{6.417202in}{0.400271in}}%
\pgfpathlineto{\pgfqpoint{6.589043in}{0.400271in}}%
\pgfpathlineto{\pgfqpoint{6.589043in}{0.532181in}}%
\pgfpathlineto{\pgfqpoint{6.589043in}{0.532181in}}%
\pgfpathlineto{\pgfqpoint{6.417202in}{0.532181in}}%
\pgfpathlineto{\pgfqpoint{6.302642in}{0.532181in}}%
\pgfpathlineto{\pgfqpoint{6.188082in}{0.532181in}}%
\pgfpathlineto{\pgfqpoint{6.073522in}{0.532181in}}%
\pgfpathlineto{\pgfqpoint{5.958962in}{0.532181in}}%
\pgfpathlineto{\pgfqpoint{5.844401in}{0.532181in}}%
\pgfpathlineto{\pgfqpoint{5.729841in}{0.532181in}}%
\pgfpathlineto{\pgfqpoint{5.615281in}{0.532181in}}%
\pgfpathlineto{\pgfqpoint{5.500721in}{0.532181in}}%
\pgfpathlineto{\pgfqpoint{5.386160in}{0.532181in}}%
\pgfpathlineto{\pgfqpoint{5.271600in}{0.532181in}}%
\pgfpathlineto{\pgfqpoint{5.157040in}{0.532181in}}%
\pgfpathlineto{\pgfqpoint{5.042480in}{0.532181in}}%
\pgfpathlineto{\pgfqpoint{4.927919in}{0.532181in}}%
\pgfpathlineto{\pgfqpoint{4.813359in}{0.532181in}}%
\pgfpathlineto{\pgfqpoint{4.698799in}{0.532181in}}%
\pgfpathlineto{\pgfqpoint{4.584239in}{0.532181in}}%
\pgfpathlineto{\pgfqpoint{4.469678in}{0.532181in}}%
\pgfpathlineto{\pgfqpoint{4.297838in}{0.532181in}}%
\pgfpathclose%
\pgfusepath{stroke,fill}%
\end{pgfscope}%
\begin{pgfscope}%
\pgfpathrectangle{\pgfqpoint{4.123134in}{0.331635in}}{\pgfqpoint{2.583333in}{1.510000in}}%
\pgfusepath{clip}%
\pgfsetbuttcap%
\pgfsetroundjoin%
\pgfsetlinewidth{1.505625pt}%
\definecolor{currentstroke}{rgb}{0.000000,0.000000,0.000000}%
\pgfsetstrokecolor{currentstroke}%
\pgfsetdash{}{0pt}%
\pgfpathmoveto{\pgfqpoint{4.240558in}{0.466226in}}%
\pgfpathlineto{\pgfqpoint{4.240558in}{1.772999in}}%
\pgfusepath{stroke}%
\end{pgfscope}%
\begin{pgfscope}%
\pgfpathrectangle{\pgfqpoint{4.123134in}{0.331635in}}{\pgfqpoint{2.583333in}{1.510000in}}%
\pgfusepath{clip}%
\pgfsetbuttcap%
\pgfsetroundjoin%
\pgfsetlinewidth{1.505625pt}%
\definecolor{currentstroke}{rgb}{0.000000,0.000000,0.000000}%
\pgfsetstrokecolor{currentstroke}%
\pgfsetdash{}{0pt}%
\pgfpathmoveto{\pgfqpoint{4.355118in}{0.466226in}}%
\pgfpathlineto{\pgfqpoint{4.355118in}{0.533577in}}%
\pgfusepath{stroke}%
\end{pgfscope}%
\begin{pgfscope}%
\pgfpathrectangle{\pgfqpoint{4.123134in}{0.331635in}}{\pgfqpoint{2.583333in}{1.510000in}}%
\pgfusepath{clip}%
\pgfsetbuttcap%
\pgfsetroundjoin%
\pgfsetlinewidth{1.505625pt}%
\definecolor{currentstroke}{rgb}{0.000000,0.000000,0.000000}%
\pgfsetstrokecolor{currentstroke}%
\pgfsetdash{}{0pt}%
\pgfpathmoveto{\pgfqpoint{4.469678in}{0.466226in}}%
\pgfpathlineto{\pgfqpoint{4.469678in}{0.496780in}}%
\pgfusepath{stroke}%
\end{pgfscope}%
\begin{pgfscope}%
\pgfpathrectangle{\pgfqpoint{4.123134in}{0.331635in}}{\pgfqpoint{2.583333in}{1.510000in}}%
\pgfusepath{clip}%
\pgfsetbuttcap%
\pgfsetroundjoin%
\pgfsetlinewidth{1.505625pt}%
\definecolor{currentstroke}{rgb}{0.000000,0.000000,0.000000}%
\pgfsetstrokecolor{currentstroke}%
\pgfsetdash{}{0pt}%
\pgfpathmoveto{\pgfqpoint{4.584239in}{0.466226in}}%
\pgfpathlineto{\pgfqpoint{4.584239in}{0.494239in}}%
\pgfusepath{stroke}%
\end{pgfscope}%
\begin{pgfscope}%
\pgfpathrectangle{\pgfqpoint{4.123134in}{0.331635in}}{\pgfqpoint{2.583333in}{1.510000in}}%
\pgfusepath{clip}%
\pgfsetbuttcap%
\pgfsetroundjoin%
\pgfsetlinewidth{1.505625pt}%
\definecolor{currentstroke}{rgb}{0.000000,0.000000,0.000000}%
\pgfsetstrokecolor{currentstroke}%
\pgfsetdash{}{0pt}%
\pgfpathmoveto{\pgfqpoint{4.698799in}{0.466226in}}%
\pgfpathlineto{\pgfqpoint{4.698799in}{0.474866in}}%
\pgfusepath{stroke}%
\end{pgfscope}%
\begin{pgfscope}%
\pgfpathrectangle{\pgfqpoint{4.123134in}{0.331635in}}{\pgfqpoint{2.583333in}{1.510000in}}%
\pgfusepath{clip}%
\pgfsetbuttcap%
\pgfsetroundjoin%
\pgfsetlinewidth{1.505625pt}%
\definecolor{currentstroke}{rgb}{0.000000,0.000000,0.000000}%
\pgfsetstrokecolor{currentstroke}%
\pgfsetdash{}{0pt}%
\pgfpathmoveto{\pgfqpoint{4.813359in}{0.466226in}}%
\pgfpathlineto{\pgfqpoint{4.813359in}{0.464867in}}%
\pgfusepath{stroke}%
\end{pgfscope}%
\begin{pgfscope}%
\pgfpathrectangle{\pgfqpoint{4.123134in}{0.331635in}}{\pgfqpoint{2.583333in}{1.510000in}}%
\pgfusepath{clip}%
\pgfsetbuttcap%
\pgfsetroundjoin%
\pgfsetlinewidth{1.505625pt}%
\definecolor{currentstroke}{rgb}{0.000000,0.000000,0.000000}%
\pgfsetstrokecolor{currentstroke}%
\pgfsetdash{}{0pt}%
\pgfpathmoveto{\pgfqpoint{4.927919in}{0.466226in}}%
\pgfpathlineto{\pgfqpoint{4.927919in}{0.490319in}}%
\pgfusepath{stroke}%
\end{pgfscope}%
\begin{pgfscope}%
\pgfpathrectangle{\pgfqpoint{4.123134in}{0.331635in}}{\pgfqpoint{2.583333in}{1.510000in}}%
\pgfusepath{clip}%
\pgfsetbuttcap%
\pgfsetroundjoin%
\pgfsetlinewidth{1.505625pt}%
\definecolor{currentstroke}{rgb}{0.000000,0.000000,0.000000}%
\pgfsetstrokecolor{currentstroke}%
\pgfsetdash{}{0pt}%
\pgfpathmoveto{\pgfqpoint{5.042480in}{0.466226in}}%
\pgfpathlineto{\pgfqpoint{5.042480in}{0.448487in}}%
\pgfusepath{stroke}%
\end{pgfscope}%
\begin{pgfscope}%
\pgfpathrectangle{\pgfqpoint{4.123134in}{0.331635in}}{\pgfqpoint{2.583333in}{1.510000in}}%
\pgfusepath{clip}%
\pgfsetbuttcap%
\pgfsetroundjoin%
\pgfsetlinewidth{1.505625pt}%
\definecolor{currentstroke}{rgb}{0.000000,0.000000,0.000000}%
\pgfsetstrokecolor{currentstroke}%
\pgfsetdash{}{0pt}%
\pgfpathmoveto{\pgfqpoint{5.157040in}{0.466226in}}%
\pgfpathlineto{\pgfqpoint{5.157040in}{0.456753in}}%
\pgfusepath{stroke}%
\end{pgfscope}%
\begin{pgfscope}%
\pgfpathrectangle{\pgfqpoint{4.123134in}{0.331635in}}{\pgfqpoint{2.583333in}{1.510000in}}%
\pgfusepath{clip}%
\pgfsetbuttcap%
\pgfsetroundjoin%
\pgfsetlinewidth{1.505625pt}%
\definecolor{currentstroke}{rgb}{0.000000,0.000000,0.000000}%
\pgfsetstrokecolor{currentstroke}%
\pgfsetdash{}{0pt}%
\pgfpathmoveto{\pgfqpoint{5.271600in}{0.466226in}}%
\pgfpathlineto{\pgfqpoint{5.271600in}{0.506399in}}%
\pgfusepath{stroke}%
\end{pgfscope}%
\begin{pgfscope}%
\pgfpathrectangle{\pgfqpoint{4.123134in}{0.331635in}}{\pgfqpoint{2.583333in}{1.510000in}}%
\pgfusepath{clip}%
\pgfsetbuttcap%
\pgfsetroundjoin%
\pgfsetlinewidth{1.505625pt}%
\definecolor{currentstroke}{rgb}{0.000000,0.000000,0.000000}%
\pgfsetstrokecolor{currentstroke}%
\pgfsetdash{}{0pt}%
\pgfpathmoveto{\pgfqpoint{5.386160in}{0.466226in}}%
\pgfpathlineto{\pgfqpoint{5.386160in}{0.446503in}}%
\pgfusepath{stroke}%
\end{pgfscope}%
\begin{pgfscope}%
\pgfpathrectangle{\pgfqpoint{4.123134in}{0.331635in}}{\pgfqpoint{2.583333in}{1.510000in}}%
\pgfusepath{clip}%
\pgfsetbuttcap%
\pgfsetroundjoin%
\pgfsetlinewidth{1.505625pt}%
\definecolor{currentstroke}{rgb}{0.000000,0.000000,0.000000}%
\pgfsetstrokecolor{currentstroke}%
\pgfsetdash{}{0pt}%
\pgfpathmoveto{\pgfqpoint{5.500721in}{0.466226in}}%
\pgfpathlineto{\pgfqpoint{5.500721in}{0.532393in}}%
\pgfusepath{stroke}%
\end{pgfscope}%
\begin{pgfscope}%
\pgfpathrectangle{\pgfqpoint{4.123134in}{0.331635in}}{\pgfqpoint{2.583333in}{1.510000in}}%
\pgfusepath{clip}%
\pgfsetbuttcap%
\pgfsetroundjoin%
\pgfsetlinewidth{1.505625pt}%
\definecolor{currentstroke}{rgb}{0.000000,0.000000,0.000000}%
\pgfsetstrokecolor{currentstroke}%
\pgfsetdash{}{0pt}%
\pgfpathmoveto{\pgfqpoint{5.615281in}{0.466226in}}%
\pgfpathlineto{\pgfqpoint{5.615281in}{0.504187in}}%
\pgfusepath{stroke}%
\end{pgfscope}%
\begin{pgfscope}%
\pgfpathrectangle{\pgfqpoint{4.123134in}{0.331635in}}{\pgfqpoint{2.583333in}{1.510000in}}%
\pgfusepath{clip}%
\pgfsetbuttcap%
\pgfsetroundjoin%
\pgfsetlinewidth{1.505625pt}%
\definecolor{currentstroke}{rgb}{0.000000,0.000000,0.000000}%
\pgfsetstrokecolor{currentstroke}%
\pgfsetdash{}{0pt}%
\pgfpathmoveto{\pgfqpoint{5.729841in}{0.466226in}}%
\pgfpathlineto{\pgfqpoint{5.729841in}{0.473132in}}%
\pgfusepath{stroke}%
\end{pgfscope}%
\begin{pgfscope}%
\pgfpathrectangle{\pgfqpoint{4.123134in}{0.331635in}}{\pgfqpoint{2.583333in}{1.510000in}}%
\pgfusepath{clip}%
\pgfsetbuttcap%
\pgfsetroundjoin%
\pgfsetlinewidth{1.505625pt}%
\definecolor{currentstroke}{rgb}{0.000000,0.000000,0.000000}%
\pgfsetstrokecolor{currentstroke}%
\pgfsetdash{}{0pt}%
\pgfpathmoveto{\pgfqpoint{5.844401in}{0.466226in}}%
\pgfpathlineto{\pgfqpoint{5.844401in}{0.515473in}}%
\pgfusepath{stroke}%
\end{pgfscope}%
\begin{pgfscope}%
\pgfpathrectangle{\pgfqpoint{4.123134in}{0.331635in}}{\pgfqpoint{2.583333in}{1.510000in}}%
\pgfusepath{clip}%
\pgfsetbuttcap%
\pgfsetroundjoin%
\pgfsetlinewidth{1.505625pt}%
\definecolor{currentstroke}{rgb}{0.000000,0.000000,0.000000}%
\pgfsetstrokecolor{currentstroke}%
\pgfsetdash{}{0pt}%
\pgfpathmoveto{\pgfqpoint{5.958962in}{0.466226in}}%
\pgfpathlineto{\pgfqpoint{5.958962in}{0.452321in}}%
\pgfusepath{stroke}%
\end{pgfscope}%
\begin{pgfscope}%
\pgfpathrectangle{\pgfqpoint{4.123134in}{0.331635in}}{\pgfqpoint{2.583333in}{1.510000in}}%
\pgfusepath{clip}%
\pgfsetbuttcap%
\pgfsetroundjoin%
\pgfsetlinewidth{1.505625pt}%
\definecolor{currentstroke}{rgb}{0.000000,0.000000,0.000000}%
\pgfsetstrokecolor{currentstroke}%
\pgfsetdash{}{0pt}%
\pgfpathmoveto{\pgfqpoint{6.073522in}{0.466226in}}%
\pgfpathlineto{\pgfqpoint{6.073522in}{0.437586in}}%
\pgfusepath{stroke}%
\end{pgfscope}%
\begin{pgfscope}%
\pgfpathrectangle{\pgfqpoint{4.123134in}{0.331635in}}{\pgfqpoint{2.583333in}{1.510000in}}%
\pgfusepath{clip}%
\pgfsetbuttcap%
\pgfsetroundjoin%
\pgfsetlinewidth{1.505625pt}%
\definecolor{currentstroke}{rgb}{0.000000,0.000000,0.000000}%
\pgfsetstrokecolor{currentstroke}%
\pgfsetdash{}{0pt}%
\pgfpathmoveto{\pgfqpoint{6.188082in}{0.466226in}}%
\pgfpathlineto{\pgfqpoint{6.188082in}{0.445261in}}%
\pgfusepath{stroke}%
\end{pgfscope}%
\begin{pgfscope}%
\pgfpathrectangle{\pgfqpoint{4.123134in}{0.331635in}}{\pgfqpoint{2.583333in}{1.510000in}}%
\pgfusepath{clip}%
\pgfsetbuttcap%
\pgfsetroundjoin%
\pgfsetlinewidth{1.505625pt}%
\definecolor{currentstroke}{rgb}{0.000000,0.000000,0.000000}%
\pgfsetstrokecolor{currentstroke}%
\pgfsetdash{}{0pt}%
\pgfpathmoveto{\pgfqpoint{6.302642in}{0.466226in}}%
\pgfpathlineto{\pgfqpoint{6.302642in}{0.469069in}}%
\pgfusepath{stroke}%
\end{pgfscope}%
\begin{pgfscope}%
\pgfpathrectangle{\pgfqpoint{4.123134in}{0.331635in}}{\pgfqpoint{2.583333in}{1.510000in}}%
\pgfusepath{clip}%
\pgfsetbuttcap%
\pgfsetroundjoin%
\pgfsetlinewidth{1.505625pt}%
\definecolor{currentstroke}{rgb}{0.000000,0.000000,0.000000}%
\pgfsetstrokecolor{currentstroke}%
\pgfsetdash{}{0pt}%
\pgfpathmoveto{\pgfqpoint{6.417202in}{0.466226in}}%
\pgfpathlineto{\pgfqpoint{6.417202in}{0.473487in}}%
\pgfusepath{stroke}%
\end{pgfscope}%
\begin{pgfscope}%
\pgfpathrectangle{\pgfqpoint{4.123134in}{0.331635in}}{\pgfqpoint{2.583333in}{1.510000in}}%
\pgfusepath{clip}%
\pgfsetbuttcap%
\pgfsetroundjoin%
\pgfsetlinewidth{1.505625pt}%
\definecolor{currentstroke}{rgb}{0.000000,0.000000,0.000000}%
\pgfsetstrokecolor{currentstroke}%
\pgfsetdash{}{0pt}%
\pgfpathmoveto{\pgfqpoint{6.531763in}{0.466226in}}%
\pgfpathlineto{\pgfqpoint{6.531763in}{0.459594in}}%
\pgfusepath{stroke}%
\end{pgfscope}%
\begin{pgfscope}%
\pgfpathrectangle{\pgfqpoint{4.123134in}{0.331635in}}{\pgfqpoint{2.583333in}{1.510000in}}%
\pgfusepath{clip}%
\pgfsetroundcap%
\pgfsetroundjoin%
\pgfsetlinewidth{1.505625pt}%
\definecolor{currentstroke}{rgb}{0.839216,0.152941,0.156863}%
\pgfsetstrokecolor{currentstroke}%
\pgfsetdash{}{0pt}%
\pgfpathmoveto{\pgfqpoint{4.123134in}{0.466226in}}%
\pgfpathlineto{\pgfqpoint{6.706467in}{0.466226in}}%
\pgfusepath{stroke}%
\end{pgfscope}%
\begin{pgfscope}%
\pgfpathrectangle{\pgfqpoint{4.123134in}{0.331635in}}{\pgfqpoint{2.583333in}{1.510000in}}%
\pgfusepath{clip}%
\pgfsetbuttcap%
\pgfsetroundjoin%
\definecolor{currentfill}{rgb}{0.839216,0.152941,0.156863}%
\pgfsetfillcolor{currentfill}%
\pgfsetlinewidth{1.003750pt}%
\definecolor{currentstroke}{rgb}{0.839216,0.152941,0.156863}%
\pgfsetstrokecolor{currentstroke}%
\pgfsetdash{}{0pt}%
\pgfsys@defobject{currentmarker}{\pgfqpoint{-0.034722in}{-0.034722in}}{\pgfqpoint{0.034722in}{0.034722in}}{%
\pgfpathmoveto{\pgfqpoint{0.000000in}{-0.034722in}}%
\pgfpathcurveto{\pgfqpoint{0.009208in}{-0.034722in}}{\pgfqpoint{0.018041in}{-0.031064in}}{\pgfqpoint{0.024552in}{-0.024552in}}%
\pgfpathcurveto{\pgfqpoint{0.031064in}{-0.018041in}}{\pgfqpoint{0.034722in}{-0.009208in}}{\pgfqpoint{0.034722in}{0.000000in}}%
\pgfpathcurveto{\pgfqpoint{0.034722in}{0.009208in}}{\pgfqpoint{0.031064in}{0.018041in}}{\pgfqpoint{0.024552in}{0.024552in}}%
\pgfpathcurveto{\pgfqpoint{0.018041in}{0.031064in}}{\pgfqpoint{0.009208in}{0.034722in}}{\pgfqpoint{0.000000in}{0.034722in}}%
\pgfpathcurveto{\pgfqpoint{-0.009208in}{0.034722in}}{\pgfqpoint{-0.018041in}{0.031064in}}{\pgfqpoint{-0.024552in}{0.024552in}}%
\pgfpathcurveto{\pgfqpoint{-0.031064in}{0.018041in}}{\pgfqpoint{-0.034722in}{0.009208in}}{\pgfqpoint{-0.034722in}{0.000000in}}%
\pgfpathcurveto{\pgfqpoint{-0.034722in}{-0.009208in}}{\pgfqpoint{-0.031064in}{-0.018041in}}{\pgfqpoint{-0.024552in}{-0.024552in}}%
\pgfpathcurveto{\pgfqpoint{-0.018041in}{-0.031064in}}{\pgfqpoint{-0.009208in}{-0.034722in}}{\pgfqpoint{0.000000in}{-0.034722in}}%
\pgfpathclose%
\pgfusepath{stroke,fill}%
}%
\begin{pgfscope}%
\pgfsys@transformshift{4.240558in}{1.772999in}%
\pgfsys@useobject{currentmarker}{}%
\end{pgfscope}%
\begin{pgfscope}%
\pgfsys@transformshift{4.355118in}{0.533577in}%
\pgfsys@useobject{currentmarker}{}%
\end{pgfscope}%
\begin{pgfscope}%
\pgfsys@transformshift{4.469678in}{0.496780in}%
\pgfsys@useobject{currentmarker}{}%
\end{pgfscope}%
\begin{pgfscope}%
\pgfsys@transformshift{4.584239in}{0.494239in}%
\pgfsys@useobject{currentmarker}{}%
\end{pgfscope}%
\begin{pgfscope}%
\pgfsys@transformshift{4.698799in}{0.474866in}%
\pgfsys@useobject{currentmarker}{}%
\end{pgfscope}%
\begin{pgfscope}%
\pgfsys@transformshift{4.813359in}{0.464867in}%
\pgfsys@useobject{currentmarker}{}%
\end{pgfscope}%
\begin{pgfscope}%
\pgfsys@transformshift{4.927919in}{0.490319in}%
\pgfsys@useobject{currentmarker}{}%
\end{pgfscope}%
\begin{pgfscope}%
\pgfsys@transformshift{5.042480in}{0.448487in}%
\pgfsys@useobject{currentmarker}{}%
\end{pgfscope}%
\begin{pgfscope}%
\pgfsys@transformshift{5.157040in}{0.456753in}%
\pgfsys@useobject{currentmarker}{}%
\end{pgfscope}%
\begin{pgfscope}%
\pgfsys@transformshift{5.271600in}{0.506399in}%
\pgfsys@useobject{currentmarker}{}%
\end{pgfscope}%
\begin{pgfscope}%
\pgfsys@transformshift{5.386160in}{0.446503in}%
\pgfsys@useobject{currentmarker}{}%
\end{pgfscope}%
\begin{pgfscope}%
\pgfsys@transformshift{5.500721in}{0.532393in}%
\pgfsys@useobject{currentmarker}{}%
\end{pgfscope}%
\begin{pgfscope}%
\pgfsys@transformshift{5.615281in}{0.504187in}%
\pgfsys@useobject{currentmarker}{}%
\end{pgfscope}%
\begin{pgfscope}%
\pgfsys@transformshift{5.729841in}{0.473132in}%
\pgfsys@useobject{currentmarker}{}%
\end{pgfscope}%
\begin{pgfscope}%
\pgfsys@transformshift{5.844401in}{0.515473in}%
\pgfsys@useobject{currentmarker}{}%
\end{pgfscope}%
\begin{pgfscope}%
\pgfsys@transformshift{5.958962in}{0.452321in}%
\pgfsys@useobject{currentmarker}{}%
\end{pgfscope}%
\begin{pgfscope}%
\pgfsys@transformshift{6.073522in}{0.437586in}%
\pgfsys@useobject{currentmarker}{}%
\end{pgfscope}%
\begin{pgfscope}%
\pgfsys@transformshift{6.188082in}{0.445261in}%
\pgfsys@useobject{currentmarker}{}%
\end{pgfscope}%
\begin{pgfscope}%
\pgfsys@transformshift{6.302642in}{0.469069in}%
\pgfsys@useobject{currentmarker}{}%
\end{pgfscope}%
\begin{pgfscope}%
\pgfsys@transformshift{6.417202in}{0.473487in}%
\pgfsys@useobject{currentmarker}{}%
\end{pgfscope}%
\begin{pgfscope}%
\pgfsys@transformshift{6.531763in}{0.459594in}%
\pgfsys@useobject{currentmarker}{}%
\end{pgfscope}%
\end{pgfscope}%
\begin{pgfscope}%
\pgfsetrectcap%
\pgfsetmiterjoin%
\pgfsetlinewidth{0.803000pt}%
\definecolor{currentstroke}{rgb}{1.000000,1.000000,1.000000}%
\pgfsetstrokecolor{currentstroke}%
\pgfsetdash{}{0pt}%
\pgfpathmoveto{\pgfqpoint{4.123134in}{0.331635in}}%
\pgfpathlineto{\pgfqpoint{4.123134in}{1.841635in}}%
\pgfusepath{stroke}%
\end{pgfscope}%
\begin{pgfscope}%
\pgfsetrectcap%
\pgfsetmiterjoin%
\pgfsetlinewidth{0.803000pt}%
\definecolor{currentstroke}{rgb}{1.000000,1.000000,1.000000}%
\pgfsetstrokecolor{currentstroke}%
\pgfsetdash{}{0pt}%
\pgfpathmoveto{\pgfqpoint{6.706467in}{0.331635in}}%
\pgfpathlineto{\pgfqpoint{6.706467in}{1.841635in}}%
\pgfusepath{stroke}%
\end{pgfscope}%
\begin{pgfscope}%
\pgfsetrectcap%
\pgfsetmiterjoin%
\pgfsetlinewidth{0.803000pt}%
\definecolor{currentstroke}{rgb}{1.000000,1.000000,1.000000}%
\pgfsetstrokecolor{currentstroke}%
\pgfsetdash{}{0pt}%
\pgfpathmoveto{\pgfqpoint{4.123134in}{0.331635in}}%
\pgfpathlineto{\pgfqpoint{6.706467in}{0.331635in}}%
\pgfusepath{stroke}%
\end{pgfscope}%
\begin{pgfscope}%
\pgfsetrectcap%
\pgfsetmiterjoin%
\pgfsetlinewidth{0.803000pt}%
\definecolor{currentstroke}{rgb}{1.000000,1.000000,1.000000}%
\pgfsetstrokecolor{currentstroke}%
\pgfsetdash{}{0pt}%
\pgfpathmoveto{\pgfqpoint{4.123134in}{1.841635in}}%
\pgfpathlineto{\pgfqpoint{6.706467in}{1.841635in}}%
\pgfusepath{stroke}%
\end{pgfscope}%
\begin{pgfscope}%
\definecolor{textcolor}{rgb}{0.150000,0.150000,0.150000}%
\pgfsetstrokecolor{textcolor}%
\pgfsetfillcolor{textcolor}%
\pgftext[x=5.414800in,y=1.924968in,,base]{\color{textcolor}\rmfamily\fontsize{12.000000}{14.400000}\selectfont Partial Autocorrelation INTC\^2}%
\end{pgfscope}%
\end{pgfpicture}%
\makeatother%
\endgroup%

    \end{adjustbox}
    \caption{Squared log-returns and ACF and PACF for squared log-returns for V (top-left and middle) and INTC (top right and bottom). }
    \label{fig:V_INTC_squared}
\end{figure}{}

We formally test for the presence of a GARCH effect, i.e. autocorrelation in the squared residuals of the time series using the Lagrange Multiplier Test proposed by Engle (SOURCE). For both tests, the null hypothesis of 'No ARCH effect' cannot be rejected with p-values of 0.1160 (V) and 0.2693 (INTC). We nevertheless proceed to fit a GARCH model to the data. There is extensive literature on the optimal order for a GARCH model (e.g. does anything beat GARCH(1,1), does anything not beat GARCH(1,1), does anyone need GARCH(1,1)?). 

For simplicity, we will stick to a GARCH(1,1) model with a constant mean model (the mean is modeled by an ARMA(0,0) process). While we have tried other models such as for example GARCH(3,1) they did not perform significantly better than GARCH(1,1). To avoid issues with numerical instability all log-returns are multiplied by 100 and the new baseline BIC become 5025.46 for V (ARMA(1,1)) and 5205.42 for INTC (ARMA(0,0)) as shown on the right hand side of \ref{tab:bic_arma}. Table \ref{tab:V_result_GARCH11_100} shows the result for a GARCH(1,1) fit to V, table \ref{tab:INTC_result_GARCH11_100} shows the result for INTC. In both cases the fit improves and the BIC drops - not too much, but noticeably (from 5025.46 to 4948.68 (V) and from 5205.42 to 5169.75 (INTC)). 

\begin{table}[h!]
    \centering
    \figuretitle{Results for GARCH(1,1) with constant mean fit to the log-returns of V}
    \vspace{-2ex}
    \small
    \begin{center}
\begin{tabular}{lclc}
\toprule
\textbf{Dep. Variable:} &    log\_returns    & \textbf{  R-squared:         } &    -0.000   \\
\textbf{Mean Model:}    &   Constant Mean    & \textbf{  Adj. R-squared:    } &    -0.000   \\
\textbf{Vol Model:}     &       GARCH        & \textbf{  Log-Likelihood:    } &   -2459.70  \\
\textbf{Distribution:}  &       Normal       & \textbf{  AIC:               } &    4927.40  \\
\textbf{Method:}        & Maximum Likelihood & \textbf{  BIC:               } &    4948.68  \\
\textbf{}               &                    & \textbf{  No. Observations:  } &    1509     \\
\textbf{Date:}          &  Wed, Sep 04 2019  & \textbf{  Df Residuals:      } &    1505     \\
\bottomrule
\end{tabular}
\begin{tabular}{lccccc}
            & \textbf{coef} & \textbf{std err} & \textbf{t} & \textbf{P$> |$t$|$} & \textbf{95.0\% Conf. Int.}  \\
\midrule
\textbf{mu} &       0.1199  &    2.971e-02     &     4.034  &      5.473e-05       &    [6.164e-02,  0.178]      \\
                  & \textbf{coef} & \textbf{std err} & \textbf{t} & \textbf{P$> |$t$|$} & \textbf{95.0\% Conf. Int.}  \\
\midrule
\textbf{omega}    &       0.0811  &    5.148e-02     &     1.576  &          0.115       &    [-1.979e-02,  0.182]     \\
\textbf{alpha[1]} &       0.0985  &    3.488e-02     &     2.823  &      4.758e-03       &    [3.010e-02,  0.167]      \\
\textbf{beta[1]}  &       0.8585  &    5.388e-02     &    15.933  &      3.718e-57       &     [  0.753,  0.964]       \\
\bottomrule
\end{tabular}
%\caption{Constant Mean - GARCH Model Results}
\end{center}

Covariance estimator: robust

    \caption{}
    \label{tab:V_result_GARCH11_100}
\end{table}{}

\begin{table}[h!]
    \centering
    \figuretitle{Results for GARCH(1,1) with constant mean fit to the log-returns of INTC}
    \begin{center}
\begin{tabular}{lclc}
\toprule
\textbf{Dep. Variable:} &    log\_returns    & \textbf{  R-squared:         } &    -0.000   \\
\textbf{Mean Model:}    &   Constant Mean    & \textbf{  Adj. R-squared:    } &    -0.000   \\
\textbf{Vol Model:}     &       GARCH        & \textbf{  Log-Likelihood:    } &   -2570.24  \\
\textbf{Distribution:}  &       Normal       & \textbf{  AIC:               } &    5148.47  \\
\textbf{Method:}        & Maximum Likelihood & \textbf{  BIC:               } &    5169.75  \\
\textbf{}               &                    & \textbf{  No. Observations:  } &    1509     \\
\textbf{Date:}          &  Wed, Sep 04 2019  & \textbf{  Df Residuals:      } &    1505     \\
\bottomrule
\end{tabular}
\begin{tabular}{lccccc}
            & \textbf{coef} & \textbf{std err} & \textbf{t} & \textbf{P$> |$t$|$} & \textbf{95.0\% Conf. Int.}  \\
\midrule
\textbf{mu} &       0.0565  &    3.422e-02     &     1.650  &      9.897e-02       &    [-1.061e-02,  0.124]     \\
                  & \textbf{coef} & \textbf{std err} & \textbf{t} & \textbf{P$> |$t$|$} & \textbf{95.0\% Conf. Int.}  \\
\midrule
\textbf{omega}    &       0.9185  &        0.289     &     3.181  &      1.469e-03       &     [  0.353,  1.484]       \\
\textbf{alpha[1]} &       0.2295  &    9.374e-02     &     2.448  &      1.436e-02       &    [4.576e-02,  0.413]      \\
\textbf{beta[1]}  &       0.2977  &        0.168     &     1.772  &      7.642e-02       &    [-3.161e-02,  0.627]     \\
\bottomrule
\end{tabular}
%\caption{Constant Mean - GARCH Model Results}
\end{center}

Covariance estimator: robust

    \caption{}
    \label{tab:INTC_result_GARCH11_100}
\end{table}{}

We now relax the assumption of normally distributed error terms $\upepsilon_t$ (see \ref{eq:garch1}) and instead assume a Student t-distribution with fatter tails. This time the BIC drops considerably (from 4948.68 to 4794.91 (V) and from 5169.75 to 4986.41 (INTC)). The results are shown in table \ref{tab:result_GARCH11_students_100}. 

\begin{table}[h!]
    \centering
    \figuretitle{Results for GARCH(1,1) with t-Distribution (V)}
    \vspace{-2ex}
    \small
    \begin{center}
\begin{tabular}{lclc}
\toprule
\textbf{Dep. Variable:} &       log\_returns       & \textbf{  R-squared:         } &    -0.000   \\
\textbf{Mean Model:}    &      Constant Mean       & \textbf{  Adj. R-squared:    } &    -0.000   \\
\textbf{Vol Model:}     &          GARCH           & \textbf{  Log-Likelihood:    } &   -2379.16  \\
\textbf{Distribution:}  & Standardized Student's t & \textbf{  AIC:               } &    4768.31  \\
\textbf{Method:}        &    Maximum Likelihood    & \textbf{  BIC:               } &    4794.91  \\
\textbf{}               &                          & \textbf{  No. Observations:  } &    1509     \\
\textbf{Date:}          &     Wed, Sep 04 2019     & \textbf{  Df Residuals:      } &    1504     \\
\bottomrule
\end{tabular}
\begin{tabular}{lccccc}
            & \textbf{coef} & \textbf{std err} & \textbf{t} & \textbf{P$> |$t$|$} & \textbf{95.0\% Conf. Int.}  \\
\midrule
\textbf{mu} &       0.1150  &    2.551e-02     &     4.508  &      6.559e-06       &    [6.498e-02,  0.165]      \\
                  & \textbf{coef} & \textbf{std err} & \textbf{t} & \textbf{P$> |$t$|$} & \textbf{95.0\% Conf. Int.}  \\
\midrule
\textbf{omega}    &       0.1058  &    6.085e-02     &     1.739  &      8.196e-02       &    [-1.342e-02,  0.225]     \\
\textbf{alpha[1]} &       0.1297  &    4.143e-02     &     3.131  &      1.745e-03       &    [4.849e-02,  0.211]      \\
\textbf{beta[1]}  &       0.8158  &    6.683e-02     &    12.208  &      2.823e-34       &     [  0.685,  0.947]       \\
            & \textbf{coef} & \textbf{std err} & \textbf{t} & \textbf{P$> |$t$|$} & \textbf{95.0\% Conf. Int.}  \\
\midrule
\textbf{nu} &       4.9106  &        0.617     &     7.952  &      1.828e-15       &     [  3.700,  6.121]       \\
\bottomrule
\end{tabular}
%\caption{Constant Mean - GARCH Model Results}
\end{center}

Covariance estimator: robust

    \vspace{1ex}
    \figuretitle{Results for GARCH(1,1) with t-Distribution (INTC)}
    \vspace{-2ex}
    \small
    \begin{center}
\begin{tabular}{lclc}
\toprule
\textbf{Dep. Variable:} &       log\_returns       & \textbf{  R-squared:         } &    -0.000   \\
\textbf{Mean Model:}    &      Constant Mean       & \textbf{  Adj. R-squared:    } &    -0.000   \\
\textbf{Vol Model:}     &          GARCH           & \textbf{  Log-Likelihood:    } &   -2474.91  \\
\textbf{Distribution:}  & Standardized Student's t & \textbf{  AIC:               } &    4959.82  \\
\textbf{Method:}        &    Maximum Likelihood    & \textbf{  BIC:               } &    4986.41  \\
\textbf{}               &                          & \textbf{  No. Observations:  } &    1509     \\
\textbf{Date:}          &     Wed, Sep 04 2019     & \textbf{  Df Residuals:      } &    1504     \\
\bottomrule
\end{tabular}
\begin{tabular}{lccccc}
            & \textbf{coef} & \textbf{std err} & \textbf{t} & \textbf{P$> |$t$|$} & \textbf{95.0\% Conf. Int.}  \\
\midrule
\textbf{mu} &       0.0810  &    2.881e-02     &     2.810  &      4.950e-03       &    [2.449e-02,  0.137]      \\
                  & \textbf{coef} & \textbf{std err} & \textbf{t} & \textbf{P$> |$t$|$} & \textbf{95.0\% Conf. Int.}  \\
\midrule
\textbf{omega}    &   7.6664e-03  &    6.416e-03     &     1.195  &          0.232       &   [-4.909e-03,2.024e-02]    \\
\textbf{alpha[1]} &       0.0173  &    4.803e-03     &     3.604  &      3.137e-04       &   [7.896e-03,2.672e-02]     \\
\textbf{beta[1]}  &       0.9792  &    5.563e-03     &   176.017  &        0.000         &     [  0.968,  0.990]       \\
            & \textbf{coef} & \textbf{std err} & \textbf{t} & \textbf{P$> |$t$|$} & \textbf{95.0\% Conf. Int.}  \\
\midrule
\textbf{nu} &       4.1875  &        0.471     &     8.896  &      5.802e-19       &     [  3.265,  5.110]       \\
\bottomrule
\end{tabular}
%\caption{Constant Mean - GARCH Model Results}
\end{center}

Covariance estimator: robust

    \caption{}
    \label{tab:result_GARCH11_students_100}
\end{table}{}

Another expansion is the inclusion of asymmetric shocks (see \ref{eq:garch3}). This assumes that sudden drops in a stock's value lead to higher volatility than an increase of the same amount would. Results for the GJR-GARCH model are shown in table \ref{tab:result_GARCH11_students_GJR_100}. For V, the gamma-coefficient is positive and significant, indicating an asymmetric effect of shocks on volatility exists. The BIC drops accordingly by a small bit from 4794.91 to 4784.75. For INTC, the gamma coefficient is not significant and BIC even slightly increase from 4986.41 to 4993.23.

\begin{table}[h!]
    \centering
    \figuretitle{Results for GJR-GARCH(1,1) with t-Distribution (V)}
    \vspace{-2ex}
    \small
    \begin{center}
\begin{tabular}{lclc}
\toprule
\textbf{Dep. Variable:} &       log\_returns       & \textbf{  R-squared:         } &    -0.000   \\
\textbf{Mean Model:}    &      Constant Mean       & \textbf{  Adj. R-squared:    } &    -0.000   \\
\textbf{Vol Model:}     &        GJR-GARCH         & \textbf{  Log-Likelihood:    } &   -2370.42  \\
\textbf{Distribution:}  & Standardized Student's t & \textbf{  AIC:               } &    4752.84  \\
\textbf{Method:}        &    Maximum Likelihood    & \textbf{  BIC:               } &    4784.75  \\
\textbf{}               &                          & \textbf{  No. Observations:  } &    1509     \\
\textbf{Date:}          &     Wed, Sep 04 2019     & \textbf{  Df Residuals:      } &    1503     \\
\bottomrule
\end{tabular}
\begin{tabular}{lccccc}
            & \textbf{coef} & \textbf{std err} & \textbf{t} & \textbf{P$> |$t$|$} & \textbf{95.0\% Conf. Int.}  \\
\midrule
\textbf{mu} &       0.0962  &    2.631e-02     &     3.657  &      2.549e-04       &    [4.465e-02,  0.148]      \\
                  & \textbf{coef} & \textbf{std err} & \textbf{t} & \textbf{P$> |$t$|$} & \textbf{95.0\% Conf. Int.}  \\
\midrule
\textbf{omega}    &       0.0857  &    5.488e-02     &     1.561  &          0.118       &    [-2.187e-02,  0.193]     \\
\textbf{alpha[1]} &       0.0347  &    2.612e-02     &     1.327  &          0.185       &   [-1.653e-02,8.585e-02]    \\
\textbf{gamma[1]} &       0.1630  &    5.265e-02     &     3.096  &      1.961e-03       &    [5.981e-02,  0.266]      \\
\textbf{beta[1]}  &       0.8411  &    6.446e-02     &    13.050  &      6.383e-39       &     [  0.715,  0.967]       \\
            & \textbf{coef} & \textbf{std err} & \textbf{t} & \textbf{P$> |$t$|$} & \textbf{95.0\% Conf. Int.}  \\
\midrule
\textbf{nu} &       5.1057  &        0.675     &     7.568  &      3.795e-14       &     [  3.783,  6.428]       \\
\bottomrule
\end{tabular}
%\caption{Constant Mean - GJR-GARCH Model Results}
\end{center}

Covariance estimator: robust

    \vspace{1ex}
    \figuretitle{Results for GJR-GARCH(1,1) with t-Distribution (INTC)}
    \vspace{-2ex}
    \small
    \begin{center}
\begin{tabular}{lclc}
\toprule
\textbf{Dep. Variable:} &       log\_returns       & \textbf{  R-squared:         } &    -0.000   \\
\textbf{Mean Model:}    &      Constant Mean       & \textbf{  Adj. R-squared:    } &    -0.000   \\
\textbf{Vol Model:}     &        GJR-GARCH         & \textbf{  Log-Likelihood:    } &   -2474.66  \\
\textbf{Distribution:}  & Standardized Student's t & \textbf{  AIC:               } &    4961.31  \\
\textbf{Method:}        &    Maximum Likelihood    & \textbf{  BIC:               } &    4993.23  \\
\textbf{}               &                          & \textbf{  No. Observations:  } &    1509     \\
\textbf{Date:}          &     Wed, Sep 04 2019     & \textbf{  Df Residuals:      } &    1503     \\
\bottomrule
\end{tabular}
\begin{tabular}{lccccc}
            & \textbf{coef} & \textbf{std err} & \textbf{t} & \textbf{P$> |$t$|$} & \textbf{95.0\% Conf. Int.}  \\
\midrule
\textbf{mu} &       0.0808  &    2.862e-02     &     2.822  &      4.777e-03       &    [2.466e-02,  0.137]      \\
                  & \textbf{coef} & \textbf{std err} & \textbf{t} & \textbf{P$> |$t$|$} & \textbf{95.0\% Conf. Int.}  \\
\midrule
\textbf{omega}    &       0.0152  &    4.217e-02     &     0.360  &          0.719       &   [-6.748e-02,9.783e-02]    \\
\textbf{alpha[1]} &       0.0146  &    9.131e-03     &     1.601  &          0.109       &   [-3.276e-03,3.252e-02]    \\
\textbf{gamma[1]} &       0.0133  &    5.218e-02     &     0.255  &          0.799       &    [-8.898e-02,  0.116]     \\
\textbf{beta[1]}  &       0.9712  &    4.137e-02     &    23.476  &      7.225e-122      &     [  0.890,  1.052]       \\
            & \textbf{coef} & \textbf{std err} & \textbf{t} & \textbf{P$> |$t$|$} & \textbf{95.0\% Conf. Int.}  \\
\midrule
\textbf{nu} &       4.1624  &        0.473     &     8.808  &      1.272e-18       &     [  3.236,  5.089]       \\
\bottomrule
\end{tabular}
%\caption{Constant Mean - GJR-GARCH Model Results}
\end{center}

Covariance estimator: robust

    \caption{}
    \label{tab:result_GARCH11_students_GJR_100}
\end{table}{}

\subsubsection{Forecasting and Forecast Precision}
To assess the predictive performance we use the ARMA(0,0) model for V and INTC as well as the ARMA(1,1) model for V to generate predictions. To do so, we fit an ARMA model to the time series up to time $y_t$, predict the next value $\hat{y}_{t+1}$, then add the true value $y_{t+1}$ to the time series and fit a model that lets us predict $\hat{y}_{t + 2}$ and so forth. For the ARMA(0,0) model, this means predicting the next value on the basis of a cumulative mean of the past log-returns. Estimating an ARMA(1,1) consecutively for V turned out to be problematic because the result was not always a stationary process. Therefore, at some points the estimation of an ARMA(1,1) model was impossible. We therefore had to replace the prediction for those specific points with ARMA(0,0). 
To compare the results we compared the Error Sum of Squares (SSE) as well as a binary prediction accuracy that merely judges whether the direction of the prediction was correct. The predictions of the ARMA(0,0) model with constant mean (which corresponds to a random walk for the stock prices with drift) seem to be better then mere coin tosses. However they do not perform better than the naive strategy of always predicting an upwards movement of the stock (i.e. a positive return). Only the ARMA(1,1) model for V reaches a higher prediction accuracy. This might, however, also be due to chance. Table \ref{tab:V_INTC_ARMA_predictions} shows the results of those predictions. The forecasts are shown in figure \ref{fig:V_INTC_ARMA_predictions_plot}. 

\begin{table}[]
    \centering
    \figuretitle{SSE and Binary Accuracy of Predictions}
    %\vspace{-2ex}
    \small
    \begin{tabular}{lrrr}
    \toprule
    {}  & SSE & Binary Accuracy & Naive Binary Accuracy \\
    \midrule
    V - ARMA(0,0) & 0.2456 & 54.1528 \% & \textbf{54.2193} \% \\
    V - ARMA(1,1) & 0.2445 & \textbf{55.8139} & 54.2193 \\
    INTC - ARMA(0,0) & 0.2760 & 51.7608 & \textbf{52.3588} \\
    \bottomrule
    \end{tabular}
    \caption{Prediction accuracy for the models we tried for V and INTC. SSE is the Error Sum of Squares, Binary Accuracy is the Accuracy of the predicted sign of log-returns.}
    \label{tab:V_INTC_ARMA_predictions}
\end{table}{}

\begin{figure}[h]
    \centering
    \figuretitle{Forecasts for V and INTC}
    \begin{adjustbox}{width = 0.95\textwidth}
    %% Creator: Matplotlib, PGF backend
%%
%% To include the figure in your LaTeX document, write
%%   \input{<filename>.pgf}
%%
%% Make sure the required packages are loaded in your preamble
%%   \usepackage{pgf}
%%
%% Figures using additional raster images can only be included by \input if
%% they are in the same directory as the main LaTeX file. For loading figures
%% from other directories you can use the `import` package
%%   \usepackage{import}
%% and then include the figures with
%%   \import{<path to file>}{<filename>.pgf}
%%
%% Matplotlib used the following preamble
%%   \usepackage{fontspec}
%%   \setmainfont{DejaVuSerif.ttf}[Path=/opt/tljh/user/lib/python3.6/site-packages/matplotlib/mpl-data/fonts/ttf/]
%%   \setsansfont{DejaVuSans.ttf}[Path=/opt/tljh/user/lib/python3.6/site-packages/matplotlib/mpl-data/fonts/ttf/]
%%   \setmonofont{DejaVuSansMono.ttf}[Path=/opt/tljh/user/lib/python3.6/site-packages/matplotlib/mpl-data/fonts/ttf/]
%%
\begingroup%
\makeatletter%
\begin{pgfpicture}%
\pgfpathrectangle{\pgfpointorigin}{\pgfqpoint{5.461206in}{3.451635in}}%
\pgfusepath{use as bounding box, clip}%
\begin{pgfscope}%
\pgfsetbuttcap%
\pgfsetmiterjoin%
\definecolor{currentfill}{rgb}{1.000000,1.000000,1.000000}%
\pgfsetfillcolor{currentfill}%
\pgfsetlinewidth{0.000000pt}%
\definecolor{currentstroke}{rgb}{1.000000,1.000000,1.000000}%
\pgfsetstrokecolor{currentstroke}%
\pgfsetdash{}{0pt}%
\pgfpathmoveto{\pgfqpoint{0.000000in}{0.000000in}}%
\pgfpathlineto{\pgfqpoint{5.461206in}{0.000000in}}%
\pgfpathlineto{\pgfqpoint{5.461206in}{3.451635in}}%
\pgfpathlineto{\pgfqpoint{0.000000in}{3.451635in}}%
\pgfpathclose%
\pgfusepath{fill}%
\end{pgfscope}%
\begin{pgfscope}%
\pgfsetbuttcap%
\pgfsetmiterjoin%
\definecolor{currentfill}{rgb}{0.917647,0.917647,0.949020}%
\pgfsetfillcolor{currentfill}%
\pgfsetlinewidth{0.000000pt}%
\definecolor{currentstroke}{rgb}{0.000000,0.000000,0.000000}%
\pgfsetstrokecolor{currentstroke}%
\pgfsetstrokeopacity{0.000000}%
\pgfsetdash{}{0pt}%
\pgfpathmoveto{\pgfqpoint{0.711206in}{0.331635in}}%
\pgfpathlineto{\pgfqpoint{5.361206in}{0.331635in}}%
\pgfpathlineto{\pgfqpoint{5.361206in}{3.351635in}}%
\pgfpathlineto{\pgfqpoint{0.711206in}{3.351635in}}%
\pgfpathclose%
\pgfusepath{fill}%
\end{pgfscope}%
\begin{pgfscope}%
\pgfpathrectangle{\pgfqpoint{0.711206in}{0.331635in}}{\pgfqpoint{4.650000in}{3.020000in}}%
\pgfusepath{clip}%
\pgfsetroundcap%
\pgfsetroundjoin%
\pgfsetlinewidth{0.803000pt}%
\definecolor{currentstroke}{rgb}{1.000000,1.000000,1.000000}%
\pgfsetstrokecolor{currentstroke}%
\pgfsetdash{}{0pt}%
\pgfpathmoveto{\pgfqpoint{0.922570in}{0.331635in}}%
\pgfpathlineto{\pgfqpoint{0.922570in}{3.351635in}}%
\pgfusepath{stroke}%
\end{pgfscope}%
\begin{pgfscope}%
\definecolor{textcolor}{rgb}{0.150000,0.150000,0.150000}%
\pgfsetstrokecolor{textcolor}%
\pgfsetfillcolor{textcolor}%
\pgftext[x=0.922570in,y=0.234413in,,top]{\color{textcolor}\rmfamily\fontsize{10.000000}{12.000000}\selectfont 0}%
\end{pgfscope}%
\begin{pgfscope}%
\pgfpathrectangle{\pgfqpoint{0.711206in}{0.331635in}}{\pgfqpoint{4.650000in}{3.020000in}}%
\pgfusepath{clip}%
\pgfsetroundcap%
\pgfsetroundjoin%
\pgfsetlinewidth{0.803000pt}%
\definecolor{currentstroke}{rgb}{1.000000,1.000000,1.000000}%
\pgfsetstrokecolor{currentstroke}%
\pgfsetdash{}{0pt}%
\pgfpathmoveto{\pgfqpoint{1.484334in}{0.331635in}}%
\pgfpathlineto{\pgfqpoint{1.484334in}{3.351635in}}%
\pgfusepath{stroke}%
\end{pgfscope}%
\begin{pgfscope}%
\definecolor{textcolor}{rgb}{0.150000,0.150000,0.150000}%
\pgfsetstrokecolor{textcolor}%
\pgfsetfillcolor{textcolor}%
\pgftext[x=1.484334in,y=0.234413in,,top]{\color{textcolor}\rmfamily\fontsize{10.000000}{12.000000}\selectfont 200}%
\end{pgfscope}%
\begin{pgfscope}%
\pgfpathrectangle{\pgfqpoint{0.711206in}{0.331635in}}{\pgfqpoint{4.650000in}{3.020000in}}%
\pgfusepath{clip}%
\pgfsetroundcap%
\pgfsetroundjoin%
\pgfsetlinewidth{0.803000pt}%
\definecolor{currentstroke}{rgb}{1.000000,1.000000,1.000000}%
\pgfsetstrokecolor{currentstroke}%
\pgfsetdash{}{0pt}%
\pgfpathmoveto{\pgfqpoint{2.046097in}{0.331635in}}%
\pgfpathlineto{\pgfqpoint{2.046097in}{3.351635in}}%
\pgfusepath{stroke}%
\end{pgfscope}%
\begin{pgfscope}%
\definecolor{textcolor}{rgb}{0.150000,0.150000,0.150000}%
\pgfsetstrokecolor{textcolor}%
\pgfsetfillcolor{textcolor}%
\pgftext[x=2.046097in,y=0.234413in,,top]{\color{textcolor}\rmfamily\fontsize{10.000000}{12.000000}\selectfont 400}%
\end{pgfscope}%
\begin{pgfscope}%
\pgfpathrectangle{\pgfqpoint{0.711206in}{0.331635in}}{\pgfqpoint{4.650000in}{3.020000in}}%
\pgfusepath{clip}%
\pgfsetroundcap%
\pgfsetroundjoin%
\pgfsetlinewidth{0.803000pt}%
\definecolor{currentstroke}{rgb}{1.000000,1.000000,1.000000}%
\pgfsetstrokecolor{currentstroke}%
\pgfsetdash{}{0pt}%
\pgfpathmoveto{\pgfqpoint{2.607861in}{0.331635in}}%
\pgfpathlineto{\pgfqpoint{2.607861in}{3.351635in}}%
\pgfusepath{stroke}%
\end{pgfscope}%
\begin{pgfscope}%
\definecolor{textcolor}{rgb}{0.150000,0.150000,0.150000}%
\pgfsetstrokecolor{textcolor}%
\pgfsetfillcolor{textcolor}%
\pgftext[x=2.607861in,y=0.234413in,,top]{\color{textcolor}\rmfamily\fontsize{10.000000}{12.000000}\selectfont 600}%
\end{pgfscope}%
\begin{pgfscope}%
\pgfpathrectangle{\pgfqpoint{0.711206in}{0.331635in}}{\pgfqpoint{4.650000in}{3.020000in}}%
\pgfusepath{clip}%
\pgfsetroundcap%
\pgfsetroundjoin%
\pgfsetlinewidth{0.803000pt}%
\definecolor{currentstroke}{rgb}{1.000000,1.000000,1.000000}%
\pgfsetstrokecolor{currentstroke}%
\pgfsetdash{}{0pt}%
\pgfpathmoveto{\pgfqpoint{3.169625in}{0.331635in}}%
\pgfpathlineto{\pgfqpoint{3.169625in}{3.351635in}}%
\pgfusepath{stroke}%
\end{pgfscope}%
\begin{pgfscope}%
\definecolor{textcolor}{rgb}{0.150000,0.150000,0.150000}%
\pgfsetstrokecolor{textcolor}%
\pgfsetfillcolor{textcolor}%
\pgftext[x=3.169625in,y=0.234413in,,top]{\color{textcolor}\rmfamily\fontsize{10.000000}{12.000000}\selectfont 800}%
\end{pgfscope}%
\begin{pgfscope}%
\pgfpathrectangle{\pgfqpoint{0.711206in}{0.331635in}}{\pgfqpoint{4.650000in}{3.020000in}}%
\pgfusepath{clip}%
\pgfsetroundcap%
\pgfsetroundjoin%
\pgfsetlinewidth{0.803000pt}%
\definecolor{currentstroke}{rgb}{1.000000,1.000000,1.000000}%
\pgfsetstrokecolor{currentstroke}%
\pgfsetdash{}{0pt}%
\pgfpathmoveto{\pgfqpoint{3.731389in}{0.331635in}}%
\pgfpathlineto{\pgfqpoint{3.731389in}{3.351635in}}%
\pgfusepath{stroke}%
\end{pgfscope}%
\begin{pgfscope}%
\definecolor{textcolor}{rgb}{0.150000,0.150000,0.150000}%
\pgfsetstrokecolor{textcolor}%
\pgfsetfillcolor{textcolor}%
\pgftext[x=3.731389in,y=0.234413in,,top]{\color{textcolor}\rmfamily\fontsize{10.000000}{12.000000}\selectfont 1000}%
\end{pgfscope}%
\begin{pgfscope}%
\pgfpathrectangle{\pgfqpoint{0.711206in}{0.331635in}}{\pgfqpoint{4.650000in}{3.020000in}}%
\pgfusepath{clip}%
\pgfsetroundcap%
\pgfsetroundjoin%
\pgfsetlinewidth{0.803000pt}%
\definecolor{currentstroke}{rgb}{1.000000,1.000000,1.000000}%
\pgfsetstrokecolor{currentstroke}%
\pgfsetdash{}{0pt}%
\pgfpathmoveto{\pgfqpoint{4.293153in}{0.331635in}}%
\pgfpathlineto{\pgfqpoint{4.293153in}{3.351635in}}%
\pgfusepath{stroke}%
\end{pgfscope}%
\begin{pgfscope}%
\definecolor{textcolor}{rgb}{0.150000,0.150000,0.150000}%
\pgfsetstrokecolor{textcolor}%
\pgfsetfillcolor{textcolor}%
\pgftext[x=4.293153in,y=0.234413in,,top]{\color{textcolor}\rmfamily\fontsize{10.000000}{12.000000}\selectfont 1200}%
\end{pgfscope}%
\begin{pgfscope}%
\pgfpathrectangle{\pgfqpoint{0.711206in}{0.331635in}}{\pgfqpoint{4.650000in}{3.020000in}}%
\pgfusepath{clip}%
\pgfsetroundcap%
\pgfsetroundjoin%
\pgfsetlinewidth{0.803000pt}%
\definecolor{currentstroke}{rgb}{1.000000,1.000000,1.000000}%
\pgfsetstrokecolor{currentstroke}%
\pgfsetdash{}{0pt}%
\pgfpathmoveto{\pgfqpoint{4.854916in}{0.331635in}}%
\pgfpathlineto{\pgfqpoint{4.854916in}{3.351635in}}%
\pgfusepath{stroke}%
\end{pgfscope}%
\begin{pgfscope}%
\definecolor{textcolor}{rgb}{0.150000,0.150000,0.150000}%
\pgfsetstrokecolor{textcolor}%
\pgfsetfillcolor{textcolor}%
\pgftext[x=4.854916in,y=0.234413in,,top]{\color{textcolor}\rmfamily\fontsize{10.000000}{12.000000}\selectfont 1400}%
\end{pgfscope}%
\begin{pgfscope}%
\pgfpathrectangle{\pgfqpoint{0.711206in}{0.331635in}}{\pgfqpoint{4.650000in}{3.020000in}}%
\pgfusepath{clip}%
\pgfsetroundcap%
\pgfsetroundjoin%
\pgfsetlinewidth{0.803000pt}%
\definecolor{currentstroke}{rgb}{1.000000,1.000000,1.000000}%
\pgfsetstrokecolor{currentstroke}%
\pgfsetdash{}{0pt}%
\pgfpathmoveto{\pgfqpoint{0.711206in}{0.521563in}}%
\pgfpathlineto{\pgfqpoint{5.361206in}{0.521563in}}%
\pgfusepath{stroke}%
\end{pgfscope}%
\begin{pgfscope}%
\definecolor{textcolor}{rgb}{0.150000,0.150000,0.150000}%
\pgfsetstrokecolor{textcolor}%
\pgfsetfillcolor{textcolor}%
\pgftext[x=0.100000in,y=0.468802in,left,base]{\color{textcolor}\rmfamily\fontsize{10.000000}{12.000000}\selectfont −0.075}%
\end{pgfscope}%
\begin{pgfscope}%
\pgfpathrectangle{\pgfqpoint{0.711206in}{0.331635in}}{\pgfqpoint{4.650000in}{3.020000in}}%
\pgfusepath{clip}%
\pgfsetroundcap%
\pgfsetroundjoin%
\pgfsetlinewidth{0.803000pt}%
\definecolor{currentstroke}{rgb}{1.000000,1.000000,1.000000}%
\pgfsetstrokecolor{currentstroke}%
\pgfsetdash{}{0pt}%
\pgfpathmoveto{\pgfqpoint{0.711206in}{0.911716in}}%
\pgfpathlineto{\pgfqpoint{5.361206in}{0.911716in}}%
\pgfusepath{stroke}%
\end{pgfscope}%
\begin{pgfscope}%
\definecolor{textcolor}{rgb}{0.150000,0.150000,0.150000}%
\pgfsetstrokecolor{textcolor}%
\pgfsetfillcolor{textcolor}%
\pgftext[x=0.100000in,y=0.858955in,left,base]{\color{textcolor}\rmfamily\fontsize{10.000000}{12.000000}\selectfont −0.050}%
\end{pgfscope}%
\begin{pgfscope}%
\pgfpathrectangle{\pgfqpoint{0.711206in}{0.331635in}}{\pgfqpoint{4.650000in}{3.020000in}}%
\pgfusepath{clip}%
\pgfsetroundcap%
\pgfsetroundjoin%
\pgfsetlinewidth{0.803000pt}%
\definecolor{currentstroke}{rgb}{1.000000,1.000000,1.000000}%
\pgfsetstrokecolor{currentstroke}%
\pgfsetdash{}{0pt}%
\pgfpathmoveto{\pgfqpoint{0.711206in}{1.301870in}}%
\pgfpathlineto{\pgfqpoint{5.361206in}{1.301870in}}%
\pgfusepath{stroke}%
\end{pgfscope}%
\begin{pgfscope}%
\definecolor{textcolor}{rgb}{0.150000,0.150000,0.150000}%
\pgfsetstrokecolor{textcolor}%
\pgfsetfillcolor{textcolor}%
\pgftext[x=0.100000in,y=1.249108in,left,base]{\color{textcolor}\rmfamily\fontsize{10.000000}{12.000000}\selectfont −0.025}%
\end{pgfscope}%
\begin{pgfscope}%
\pgfpathrectangle{\pgfqpoint{0.711206in}{0.331635in}}{\pgfqpoint{4.650000in}{3.020000in}}%
\pgfusepath{clip}%
\pgfsetroundcap%
\pgfsetroundjoin%
\pgfsetlinewidth{0.803000pt}%
\definecolor{currentstroke}{rgb}{1.000000,1.000000,1.000000}%
\pgfsetstrokecolor{currentstroke}%
\pgfsetdash{}{0pt}%
\pgfpathmoveto{\pgfqpoint{0.711206in}{1.692023in}}%
\pgfpathlineto{\pgfqpoint{5.361206in}{1.692023in}}%
\pgfusepath{stroke}%
\end{pgfscope}%
\begin{pgfscope}%
\definecolor{textcolor}{rgb}{0.150000,0.150000,0.150000}%
\pgfsetstrokecolor{textcolor}%
\pgfsetfillcolor{textcolor}%
\pgftext[x=0.216374in,y=1.639261in,left,base]{\color{textcolor}\rmfamily\fontsize{10.000000}{12.000000}\selectfont 0.000}%
\end{pgfscope}%
\begin{pgfscope}%
\pgfpathrectangle{\pgfqpoint{0.711206in}{0.331635in}}{\pgfqpoint{4.650000in}{3.020000in}}%
\pgfusepath{clip}%
\pgfsetroundcap%
\pgfsetroundjoin%
\pgfsetlinewidth{0.803000pt}%
\definecolor{currentstroke}{rgb}{1.000000,1.000000,1.000000}%
\pgfsetstrokecolor{currentstroke}%
\pgfsetdash{}{0pt}%
\pgfpathmoveto{\pgfqpoint{0.711206in}{2.082176in}}%
\pgfpathlineto{\pgfqpoint{5.361206in}{2.082176in}}%
\pgfusepath{stroke}%
\end{pgfscope}%
\begin{pgfscope}%
\definecolor{textcolor}{rgb}{0.150000,0.150000,0.150000}%
\pgfsetstrokecolor{textcolor}%
\pgfsetfillcolor{textcolor}%
\pgftext[x=0.216374in,y=2.029415in,left,base]{\color{textcolor}\rmfamily\fontsize{10.000000}{12.000000}\selectfont 0.025}%
\end{pgfscope}%
\begin{pgfscope}%
\pgfpathrectangle{\pgfqpoint{0.711206in}{0.331635in}}{\pgfqpoint{4.650000in}{3.020000in}}%
\pgfusepath{clip}%
\pgfsetroundcap%
\pgfsetroundjoin%
\pgfsetlinewidth{0.803000pt}%
\definecolor{currentstroke}{rgb}{1.000000,1.000000,1.000000}%
\pgfsetstrokecolor{currentstroke}%
\pgfsetdash{}{0pt}%
\pgfpathmoveto{\pgfqpoint{0.711206in}{2.472330in}}%
\pgfpathlineto{\pgfqpoint{5.361206in}{2.472330in}}%
\pgfusepath{stroke}%
\end{pgfscope}%
\begin{pgfscope}%
\definecolor{textcolor}{rgb}{0.150000,0.150000,0.150000}%
\pgfsetstrokecolor{textcolor}%
\pgfsetfillcolor{textcolor}%
\pgftext[x=0.216374in,y=2.419568in,left,base]{\color{textcolor}\rmfamily\fontsize{10.000000}{12.000000}\selectfont 0.050}%
\end{pgfscope}%
\begin{pgfscope}%
\pgfpathrectangle{\pgfqpoint{0.711206in}{0.331635in}}{\pgfqpoint{4.650000in}{3.020000in}}%
\pgfusepath{clip}%
\pgfsetroundcap%
\pgfsetroundjoin%
\pgfsetlinewidth{0.803000pt}%
\definecolor{currentstroke}{rgb}{1.000000,1.000000,1.000000}%
\pgfsetstrokecolor{currentstroke}%
\pgfsetdash{}{0pt}%
\pgfpathmoveto{\pgfqpoint{0.711206in}{2.862483in}}%
\pgfpathlineto{\pgfqpoint{5.361206in}{2.862483in}}%
\pgfusepath{stroke}%
\end{pgfscope}%
\begin{pgfscope}%
\definecolor{textcolor}{rgb}{0.150000,0.150000,0.150000}%
\pgfsetstrokecolor{textcolor}%
\pgfsetfillcolor{textcolor}%
\pgftext[x=0.216374in,y=2.809721in,left,base]{\color{textcolor}\rmfamily\fontsize{10.000000}{12.000000}\selectfont 0.075}%
\end{pgfscope}%
\begin{pgfscope}%
\pgfpathrectangle{\pgfqpoint{0.711206in}{0.331635in}}{\pgfqpoint{4.650000in}{3.020000in}}%
\pgfusepath{clip}%
\pgfsetroundcap%
\pgfsetroundjoin%
\pgfsetlinewidth{0.803000pt}%
\definecolor{currentstroke}{rgb}{1.000000,1.000000,1.000000}%
\pgfsetstrokecolor{currentstroke}%
\pgfsetdash{}{0pt}%
\pgfpathmoveto{\pgfqpoint{0.711206in}{3.252636in}}%
\pgfpathlineto{\pgfqpoint{5.361206in}{3.252636in}}%
\pgfusepath{stroke}%
\end{pgfscope}%
\begin{pgfscope}%
\definecolor{textcolor}{rgb}{0.150000,0.150000,0.150000}%
\pgfsetstrokecolor{textcolor}%
\pgfsetfillcolor{textcolor}%
\pgftext[x=0.216374in,y=3.199875in,left,base]{\color{textcolor}\rmfamily\fontsize{10.000000}{12.000000}\selectfont 0.100}%
\end{pgfscope}%
\begin{pgfscope}%
\pgfpathrectangle{\pgfqpoint{0.711206in}{0.331635in}}{\pgfqpoint{4.650000in}{3.020000in}}%
\pgfusepath{clip}%
\pgfsetbuttcap%
\pgfsetroundjoin%
\definecolor{currentfill}{rgb}{1.000000,0.000000,0.000000}%
\pgfsetfillcolor{currentfill}%
\pgfsetfillopacity{0.400000}%
\pgfsetlinewidth{1.003750pt}%
\definecolor{currentstroke}{rgb}{1.000000,0.000000,0.000000}%
\pgfsetstrokecolor{currentstroke}%
\pgfsetstrokeopacity{0.400000}%
\pgfsetdash{}{0pt}%
\pgfpathmoveto{\pgfqpoint{0.922570in}{1.910758in}}%
\pgfpathlineto{\pgfqpoint{0.922570in}{1.244515in}}%
\pgfpathlineto{\pgfqpoint{0.925380in}{1.285516in}}%
\pgfpathlineto{\pgfqpoint{0.928191in}{1.324249in}}%
\pgfpathlineto{\pgfqpoint{0.931002in}{1.345971in}}%
\pgfpathlineto{\pgfqpoint{0.933812in}{1.277125in}}%
\pgfpathlineto{\pgfqpoint{0.936623in}{1.292774in}}%
\pgfpathlineto{\pgfqpoint{0.939434in}{1.295830in}}%
\pgfpathlineto{\pgfqpoint{0.942245in}{1.317353in}}%
\pgfpathlineto{\pgfqpoint{0.945055in}{1.293220in}}%
\pgfpathlineto{\pgfqpoint{0.947866in}{1.260910in}}%
\pgfpathlineto{\pgfqpoint{0.950677in}{1.261289in}}%
\pgfpathlineto{\pgfqpoint{0.953487in}{1.271797in}}%
\pgfpathlineto{\pgfqpoint{0.956298in}{1.281278in}}%
\pgfpathlineto{\pgfqpoint{0.959109in}{1.296520in}}%
\pgfpathlineto{\pgfqpoint{0.961919in}{1.309891in}}%
\pgfpathlineto{\pgfqpoint{0.964730in}{1.303004in}}%
\pgfpathlineto{\pgfqpoint{0.967541in}{1.314576in}}%
\pgfpathlineto{\pgfqpoint{0.970351in}{1.315246in}}%
\pgfpathlineto{\pgfqpoint{0.973162in}{1.286394in}}%
\pgfpathlineto{\pgfqpoint{0.975973in}{1.298824in}}%
\pgfpathlineto{\pgfqpoint{0.978783in}{1.308830in}}%
\pgfpathlineto{\pgfqpoint{0.981594in}{1.312832in}}%
\pgfpathlineto{\pgfqpoint{0.984405in}{1.322361in}}%
\pgfpathlineto{\pgfqpoint{0.987216in}{1.300998in}}%
\pgfpathlineto{\pgfqpoint{0.990026in}{1.311334in}}%
\pgfpathlineto{\pgfqpoint{0.992837in}{1.303822in}}%
\pgfpathlineto{\pgfqpoint{0.995648in}{1.308952in}}%
\pgfpathlineto{\pgfqpoint{0.998458in}{1.317607in}}%
\pgfpathlineto{\pgfqpoint{1.001269in}{1.308861in}}%
\pgfpathlineto{\pgfqpoint{1.004080in}{1.317559in}}%
\pgfpathlineto{\pgfqpoint{1.006890in}{1.315274in}}%
\pgfpathlineto{\pgfqpoint{1.009701in}{1.321225in}}%
\pgfpathlineto{\pgfqpoint{1.012512in}{1.325777in}}%
\pgfpathlineto{\pgfqpoint{1.015322in}{1.333194in}}%
\pgfpathlineto{\pgfqpoint{1.018133in}{1.332027in}}%
\pgfpathlineto{\pgfqpoint{1.020944in}{1.337597in}}%
\pgfpathlineto{\pgfqpoint{1.023754in}{1.313851in}}%
\pgfpathlineto{\pgfqpoint{1.026565in}{1.320331in}}%
\pgfpathlineto{\pgfqpoint{1.029376in}{1.316805in}}%
\pgfpathlineto{\pgfqpoint{1.032187in}{1.320869in}}%
\pgfpathlineto{\pgfqpoint{1.034997in}{1.314325in}}%
\pgfpathlineto{\pgfqpoint{1.037808in}{1.320198in}}%
\pgfpathlineto{\pgfqpoint{1.040619in}{1.323660in}}%
\pgfpathlineto{\pgfqpoint{1.043429in}{1.322328in}}%
\pgfpathlineto{\pgfqpoint{1.046240in}{1.322317in}}%
\pgfpathlineto{\pgfqpoint{1.049051in}{1.327544in}}%
\pgfpathlineto{\pgfqpoint{1.051861in}{1.328042in}}%
\pgfpathlineto{\pgfqpoint{1.054672in}{1.332001in}}%
\pgfpathlineto{\pgfqpoint{1.057483in}{1.333640in}}%
\pgfpathlineto{\pgfqpoint{1.060293in}{1.336820in}}%
\pgfpathlineto{\pgfqpoint{1.063104in}{1.323014in}}%
\pgfpathlineto{\pgfqpoint{1.065915in}{1.326647in}}%
\pgfpathlineto{\pgfqpoint{1.068725in}{1.330856in}}%
\pgfpathlineto{\pgfqpoint{1.071536in}{1.335146in}}%
\pgfpathlineto{\pgfqpoint{1.074347in}{1.339550in}}%
\pgfpathlineto{\pgfqpoint{1.077158in}{1.341265in}}%
\pgfpathlineto{\pgfqpoint{1.079968in}{1.341831in}}%
\pgfpathlineto{\pgfqpoint{1.082779in}{1.343175in}}%
\pgfpathlineto{\pgfqpoint{1.085590in}{1.341224in}}%
\pgfpathlineto{\pgfqpoint{1.088400in}{1.345170in}}%
\pgfpathlineto{\pgfqpoint{1.091211in}{1.348892in}}%
\pgfpathlineto{\pgfqpoint{1.094022in}{1.344531in}}%
\pgfpathlineto{\pgfqpoint{1.096832in}{1.347057in}}%
\pgfpathlineto{\pgfqpoint{1.099643in}{1.342323in}}%
\pgfpathlineto{\pgfqpoint{1.102454in}{1.327468in}}%
\pgfpathlineto{\pgfqpoint{1.105264in}{1.330973in}}%
\pgfpathlineto{\pgfqpoint{1.108075in}{1.326669in}}%
\pgfpathlineto{\pgfqpoint{1.110886in}{1.329221in}}%
\pgfpathlineto{\pgfqpoint{1.113696in}{1.319827in}}%
\pgfpathlineto{\pgfqpoint{1.116507in}{1.323400in}}%
\pgfpathlineto{\pgfqpoint{1.119318in}{1.324711in}}%
\pgfpathlineto{\pgfqpoint{1.122128in}{1.325109in}}%
\pgfpathlineto{\pgfqpoint{1.124939in}{1.326790in}}%
\pgfpathlineto{\pgfqpoint{1.127750in}{1.313374in}}%
\pgfpathlineto{\pgfqpoint{1.130561in}{1.316678in}}%
\pgfpathlineto{\pgfqpoint{1.133371in}{1.316748in}}%
\pgfpathlineto{\pgfqpoint{1.136182in}{1.319957in}}%
\pgfpathlineto{\pgfqpoint{1.138993in}{1.322561in}}%
\pgfpathlineto{\pgfqpoint{1.141803in}{1.323121in}}%
\pgfpathlineto{\pgfqpoint{1.144614in}{1.325352in}}%
\pgfpathlineto{\pgfqpoint{1.147425in}{1.324664in}}%
\pgfpathlineto{\pgfqpoint{1.150235in}{1.283982in}}%
\pgfpathlineto{\pgfqpoint{1.153046in}{1.287100in}}%
\pgfpathlineto{\pgfqpoint{1.155857in}{1.290174in}}%
\pgfpathlineto{\pgfqpoint{1.158667in}{1.291500in}}%
\pgfpathlineto{\pgfqpoint{1.161478in}{1.291783in}}%
\pgfpathlineto{\pgfqpoint{1.164289in}{1.294722in}}%
\pgfpathlineto{\pgfqpoint{1.167099in}{1.296307in}}%
\pgfpathlineto{\pgfqpoint{1.169910in}{1.294805in}}%
\pgfpathlineto{\pgfqpoint{1.172721in}{1.296655in}}%
\pgfpathlineto{\pgfqpoint{1.175532in}{1.298915in}}%
\pgfpathlineto{\pgfqpoint{1.178342in}{1.290916in}}%
\pgfpathlineto{\pgfqpoint{1.181153in}{1.283747in}}%
\pgfpathlineto{\pgfqpoint{1.183964in}{1.280525in}}%
\pgfpathlineto{\pgfqpoint{1.186774in}{1.282316in}}%
\pgfpathlineto{\pgfqpoint{1.189585in}{1.285054in}}%
\pgfpathlineto{\pgfqpoint{1.192396in}{1.287674in}}%
\pgfpathlineto{\pgfqpoint{1.195206in}{1.288764in}}%
\pgfpathlineto{\pgfqpoint{1.198017in}{1.291428in}}%
\pgfpathlineto{\pgfqpoint{1.200828in}{1.282959in}}%
\pgfpathlineto{\pgfqpoint{1.203638in}{1.277180in}}%
\pgfpathlineto{\pgfqpoint{1.206449in}{1.267793in}}%
\pgfpathlineto{\pgfqpoint{1.209260in}{1.269741in}}%
\pgfpathlineto{\pgfqpoint{1.212070in}{1.271006in}}%
\pgfpathlineto{\pgfqpoint{1.214881in}{1.271704in}}%
\pgfpathlineto{\pgfqpoint{1.217692in}{1.274055in}}%
\pgfpathlineto{\pgfqpoint{1.220503in}{1.275505in}}%
\pgfpathlineto{\pgfqpoint{1.223313in}{1.276937in}}%
\pgfpathlineto{\pgfqpoint{1.226124in}{1.279387in}}%
\pgfpathlineto{\pgfqpoint{1.228935in}{1.274346in}}%
\pgfpathlineto{\pgfqpoint{1.231745in}{1.276524in}}%
\pgfpathlineto{\pgfqpoint{1.234556in}{1.278397in}}%
\pgfpathlineto{\pgfqpoint{1.237367in}{1.280485in}}%
\pgfpathlineto{\pgfqpoint{1.240177in}{1.282826in}}%
\pgfpathlineto{\pgfqpoint{1.242988in}{1.285154in}}%
\pgfpathlineto{\pgfqpoint{1.245799in}{1.276632in}}%
\pgfpathlineto{\pgfqpoint{1.248609in}{1.267513in}}%
\pgfpathlineto{\pgfqpoint{1.251420in}{1.256700in}}%
\pgfpathlineto{\pgfqpoint{1.254231in}{1.258764in}}%
\pgfpathlineto{\pgfqpoint{1.257041in}{1.260719in}}%
\pgfpathlineto{\pgfqpoint{1.259852in}{1.258010in}}%
\pgfpathlineto{\pgfqpoint{1.262663in}{1.259799in}}%
\pgfpathlineto{\pgfqpoint{1.265474in}{1.260416in}}%
\pgfpathlineto{\pgfqpoint{1.268284in}{1.261634in}}%
\pgfpathlineto{\pgfqpoint{1.271095in}{1.263777in}}%
\pgfpathlineto{\pgfqpoint{1.273906in}{1.261687in}}%
\pgfpathlineto{\pgfqpoint{1.276716in}{1.259982in}}%
\pgfpathlineto{\pgfqpoint{1.279527in}{1.259007in}}%
\pgfpathlineto{\pgfqpoint{1.282338in}{1.257415in}}%
\pgfpathlineto{\pgfqpoint{1.285148in}{1.259222in}}%
\pgfpathlineto{\pgfqpoint{1.287959in}{1.261332in}}%
\pgfpathlineto{\pgfqpoint{1.290770in}{1.261714in}}%
\pgfpathlineto{\pgfqpoint{1.293580in}{1.263845in}}%
\pgfpathlineto{\pgfqpoint{1.296391in}{1.264778in}}%
\pgfpathlineto{\pgfqpoint{1.299202in}{1.260794in}}%
\pgfpathlineto{\pgfqpoint{1.302012in}{1.262583in}}%
\pgfpathlineto{\pgfqpoint{1.304823in}{1.261810in}}%
\pgfpathlineto{\pgfqpoint{1.307634in}{1.259756in}}%
\pgfpathlineto{\pgfqpoint{1.310445in}{1.260624in}}%
\pgfpathlineto{\pgfqpoint{1.313255in}{1.257177in}}%
\pgfpathlineto{\pgfqpoint{1.316066in}{1.258585in}}%
\pgfpathlineto{\pgfqpoint{1.318877in}{1.260502in}}%
\pgfpathlineto{\pgfqpoint{1.321687in}{1.258700in}}%
\pgfpathlineto{\pgfqpoint{1.324498in}{1.257435in}}%
\pgfpathlineto{\pgfqpoint{1.327309in}{1.259250in}}%
\pgfpathlineto{\pgfqpoint{1.330119in}{1.261107in}}%
\pgfpathlineto{\pgfqpoint{1.332930in}{1.262488in}}%
\pgfpathlineto{\pgfqpoint{1.335741in}{1.263040in}}%
\pgfpathlineto{\pgfqpoint{1.338551in}{1.264863in}}%
\pgfpathlineto{\pgfqpoint{1.341362in}{1.260776in}}%
\pgfpathlineto{\pgfqpoint{1.344173in}{1.262512in}}%
\pgfpathlineto{\pgfqpoint{1.346983in}{1.262849in}}%
\pgfpathlineto{\pgfqpoint{1.349794in}{1.264632in}}%
\pgfpathlineto{\pgfqpoint{1.352605in}{1.265948in}}%
\pgfpathlineto{\pgfqpoint{1.355415in}{1.267640in}}%
\pgfpathlineto{\pgfqpoint{1.358226in}{1.267519in}}%
\pgfpathlineto{\pgfqpoint{1.361037in}{1.268215in}}%
\pgfpathlineto{\pgfqpoint{1.363848in}{1.268658in}}%
\pgfpathlineto{\pgfqpoint{1.366658in}{1.270283in}}%
\pgfpathlineto{\pgfqpoint{1.369469in}{1.270300in}}%
\pgfpathlineto{\pgfqpoint{1.372280in}{1.270974in}}%
\pgfpathlineto{\pgfqpoint{1.375090in}{1.272672in}}%
\pgfpathlineto{\pgfqpoint{1.377901in}{1.273645in}}%
\pgfpathlineto{\pgfqpoint{1.380712in}{1.275206in}}%
\pgfpathlineto{\pgfqpoint{1.383522in}{1.274579in}}%
\pgfpathlineto{\pgfqpoint{1.386333in}{1.276149in}}%
\pgfpathlineto{\pgfqpoint{1.389144in}{1.277505in}}%
\pgfpathlineto{\pgfqpoint{1.391954in}{1.277504in}}%
\pgfpathlineto{\pgfqpoint{1.394765in}{1.278870in}}%
\pgfpathlineto{\pgfqpoint{1.397576in}{1.280132in}}%
\pgfpathlineto{\pgfqpoint{1.400386in}{1.279989in}}%
\pgfpathlineto{\pgfqpoint{1.403197in}{1.280531in}}%
\pgfpathlineto{\pgfqpoint{1.406008in}{1.281880in}}%
\pgfpathlineto{\pgfqpoint{1.408819in}{1.283405in}}%
\pgfpathlineto{\pgfqpoint{1.411629in}{1.283724in}}%
\pgfpathlineto{\pgfqpoint{1.414440in}{1.284409in}}%
\pgfpathlineto{\pgfqpoint{1.417251in}{1.285311in}}%
\pgfpathlineto{\pgfqpoint{1.420061in}{1.286823in}}%
\pgfpathlineto{\pgfqpoint{1.422872in}{1.287708in}}%
\pgfpathlineto{\pgfqpoint{1.425683in}{1.288996in}}%
\pgfpathlineto{\pgfqpoint{1.428493in}{1.288453in}}%
\pgfpathlineto{\pgfqpoint{1.431304in}{1.289887in}}%
\pgfpathlineto{\pgfqpoint{1.434115in}{1.287911in}}%
\pgfpathlineto{\pgfqpoint{1.436925in}{1.289342in}}%
\pgfpathlineto{\pgfqpoint{1.439736in}{1.290594in}}%
\pgfpathlineto{\pgfqpoint{1.442547in}{1.291536in}}%
\pgfpathlineto{\pgfqpoint{1.445357in}{1.291823in}}%
\pgfpathlineto{\pgfqpoint{1.448168in}{1.292711in}}%
\pgfpathlineto{\pgfqpoint{1.450979in}{1.294116in}}%
\pgfpathlineto{\pgfqpoint{1.453790in}{1.295388in}}%
\pgfpathlineto{\pgfqpoint{1.456600in}{1.294557in}}%
\pgfpathlineto{\pgfqpoint{1.459411in}{1.292927in}}%
\pgfpathlineto{\pgfqpoint{1.462222in}{1.294218in}}%
\pgfpathlineto{\pgfqpoint{1.465032in}{1.295529in}}%
\pgfpathlineto{\pgfqpoint{1.467843in}{1.296517in}}%
\pgfpathlineto{\pgfqpoint{1.470654in}{1.297801in}}%
\pgfpathlineto{\pgfqpoint{1.473464in}{1.299133in}}%
\pgfpathlineto{\pgfqpoint{1.476275in}{1.300446in}}%
\pgfpathlineto{\pgfqpoint{1.479086in}{1.300757in}}%
\pgfpathlineto{\pgfqpoint{1.481896in}{1.299307in}}%
\pgfpathlineto{\pgfqpoint{1.484707in}{1.299257in}}%
\pgfpathlineto{\pgfqpoint{1.487518in}{1.296917in}}%
\pgfpathlineto{\pgfqpoint{1.490328in}{1.297717in}}%
\pgfpathlineto{\pgfqpoint{1.493139in}{1.298994in}}%
\pgfpathlineto{\pgfqpoint{1.495950in}{1.300216in}}%
\pgfpathlineto{\pgfqpoint{1.498761in}{1.301328in}}%
\pgfpathlineto{\pgfqpoint{1.501571in}{1.298530in}}%
\pgfpathlineto{\pgfqpoint{1.504382in}{1.299011in}}%
\pgfpathlineto{\pgfqpoint{1.507193in}{1.297743in}}%
\pgfpathlineto{\pgfqpoint{1.510003in}{1.298658in}}%
\pgfpathlineto{\pgfqpoint{1.512814in}{1.298257in}}%
\pgfpathlineto{\pgfqpoint{1.515625in}{1.298732in}}%
\pgfpathlineto{\pgfqpoint{1.518435in}{1.299924in}}%
\pgfpathlineto{\pgfqpoint{1.521246in}{1.300930in}}%
\pgfpathlineto{\pgfqpoint{1.524057in}{1.301096in}}%
\pgfpathlineto{\pgfqpoint{1.526867in}{1.300857in}}%
\pgfpathlineto{\pgfqpoint{1.529678in}{1.301710in}}%
\pgfpathlineto{\pgfqpoint{1.532489in}{1.302323in}}%
\pgfpathlineto{\pgfqpoint{1.535299in}{1.302814in}}%
\pgfpathlineto{\pgfqpoint{1.538110in}{1.303923in}}%
\pgfpathlineto{\pgfqpoint{1.540921in}{1.304881in}}%
\pgfpathlineto{\pgfqpoint{1.543731in}{1.306039in}}%
\pgfpathlineto{\pgfqpoint{1.546542in}{1.306119in}}%
\pgfpathlineto{\pgfqpoint{1.549353in}{1.306593in}}%
\pgfpathlineto{\pgfqpoint{1.552164in}{1.307600in}}%
\pgfpathlineto{\pgfqpoint{1.554974in}{1.308704in}}%
\pgfpathlineto{\pgfqpoint{1.557785in}{1.309819in}}%
\pgfpathlineto{\pgfqpoint{1.560596in}{1.309715in}}%
\pgfpathlineto{\pgfqpoint{1.563406in}{1.309608in}}%
\pgfpathlineto{\pgfqpoint{1.566217in}{1.310586in}}%
\pgfpathlineto{\pgfqpoint{1.569028in}{1.311523in}}%
\pgfpathlineto{\pgfqpoint{1.571838in}{1.312314in}}%
\pgfpathlineto{\pgfqpoint{1.574649in}{1.313100in}}%
\pgfpathlineto{\pgfqpoint{1.577460in}{1.313966in}}%
\pgfpathlineto{\pgfqpoint{1.580270in}{1.313900in}}%
\pgfpathlineto{\pgfqpoint{1.583081in}{1.314123in}}%
\pgfpathlineto{\pgfqpoint{1.585892in}{1.314625in}}%
\pgfpathlineto{\pgfqpoint{1.588702in}{1.315433in}}%
\pgfpathlineto{\pgfqpoint{1.591513in}{1.316463in}}%
\pgfpathlineto{\pgfqpoint{1.594324in}{1.315822in}}%
\pgfpathlineto{\pgfqpoint{1.597135in}{1.316020in}}%
\pgfpathlineto{\pgfqpoint{1.599945in}{1.315243in}}%
\pgfpathlineto{\pgfqpoint{1.602756in}{1.316123in}}%
\pgfpathlineto{\pgfqpoint{1.605567in}{1.315654in}}%
\pgfpathlineto{\pgfqpoint{1.608377in}{1.316139in}}%
\pgfpathlineto{\pgfqpoint{1.611188in}{1.316319in}}%
\pgfpathlineto{\pgfqpoint{1.613999in}{1.316753in}}%
\pgfpathlineto{\pgfqpoint{1.616809in}{1.316518in}}%
\pgfpathlineto{\pgfqpoint{1.619620in}{1.317261in}}%
\pgfpathlineto{\pgfqpoint{1.622431in}{1.318267in}}%
\pgfpathlineto{\pgfqpoint{1.625241in}{1.319262in}}%
\pgfpathlineto{\pgfqpoint{1.628052in}{1.320257in}}%
\pgfpathlineto{\pgfqpoint{1.630863in}{1.321047in}}%
\pgfpathlineto{\pgfqpoint{1.633673in}{1.320786in}}%
\pgfpathlineto{\pgfqpoint{1.636484in}{1.321669in}}%
\pgfpathlineto{\pgfqpoint{1.639295in}{1.322039in}}%
\pgfpathlineto{\pgfqpoint{1.642106in}{1.322568in}}%
\pgfpathlineto{\pgfqpoint{1.644916in}{1.323095in}}%
\pgfpathlineto{\pgfqpoint{1.647727in}{1.323569in}}%
\pgfpathlineto{\pgfqpoint{1.650538in}{1.322947in}}%
\pgfpathlineto{\pgfqpoint{1.653348in}{1.323848in}}%
\pgfpathlineto{\pgfqpoint{1.656159in}{1.324479in}}%
\pgfpathlineto{\pgfqpoint{1.658970in}{1.325294in}}%
\pgfpathlineto{\pgfqpoint{1.661780in}{1.326046in}}%
\pgfpathlineto{\pgfqpoint{1.664591in}{1.322856in}}%
\pgfpathlineto{\pgfqpoint{1.667402in}{1.323611in}}%
\pgfpathlineto{\pgfqpoint{1.670212in}{1.322984in}}%
\pgfpathlineto{\pgfqpoint{1.673023in}{1.323455in}}%
\pgfpathlineto{\pgfqpoint{1.675834in}{1.324301in}}%
\pgfpathlineto{\pgfqpoint{1.678644in}{1.323408in}}%
\pgfpathlineto{\pgfqpoint{1.681455in}{1.323891in}}%
\pgfpathlineto{\pgfqpoint{1.684266in}{1.324800in}}%
\pgfpathlineto{\pgfqpoint{1.687077in}{1.321301in}}%
\pgfpathlineto{\pgfqpoint{1.689887in}{1.322155in}}%
\pgfpathlineto{\pgfqpoint{1.692698in}{1.321698in}}%
\pgfpathlineto{\pgfqpoint{1.695509in}{1.322484in}}%
\pgfpathlineto{\pgfqpoint{1.698319in}{1.322688in}}%
\pgfpathlineto{\pgfqpoint{1.701130in}{1.323575in}}%
\pgfpathlineto{\pgfqpoint{1.703941in}{1.324367in}}%
\pgfpathlineto{\pgfqpoint{1.706751in}{1.324787in}}%
\pgfpathlineto{\pgfqpoint{1.709562in}{1.323586in}}%
\pgfpathlineto{\pgfqpoint{1.712373in}{1.324366in}}%
\pgfpathlineto{\pgfqpoint{1.715183in}{1.325161in}}%
\pgfpathlineto{\pgfqpoint{1.717994in}{1.322092in}}%
\pgfpathlineto{\pgfqpoint{1.720805in}{1.322879in}}%
\pgfpathlineto{\pgfqpoint{1.723615in}{1.323661in}}%
\pgfpathlineto{\pgfqpoint{1.726426in}{1.323476in}}%
\pgfpathlineto{\pgfqpoint{1.729237in}{1.323762in}}%
\pgfpathlineto{\pgfqpoint{1.732048in}{1.324611in}}%
\pgfpathlineto{\pgfqpoint{1.734858in}{1.325417in}}%
\pgfpathlineto{\pgfqpoint{1.737669in}{1.326210in}}%
\pgfpathlineto{\pgfqpoint{1.740480in}{1.326516in}}%
\pgfpathlineto{\pgfqpoint{1.743290in}{1.327133in}}%
\pgfpathlineto{\pgfqpoint{1.746101in}{1.327868in}}%
\pgfpathlineto{\pgfqpoint{1.748912in}{1.327542in}}%
\pgfpathlineto{\pgfqpoint{1.751722in}{1.327915in}}%
\pgfpathlineto{\pgfqpoint{1.754533in}{1.328736in}}%
\pgfpathlineto{\pgfqpoint{1.757344in}{1.327732in}}%
\pgfpathlineto{\pgfqpoint{1.760154in}{1.328054in}}%
\pgfpathlineto{\pgfqpoint{1.762965in}{1.327077in}}%
\pgfpathlineto{\pgfqpoint{1.765776in}{1.327201in}}%
\pgfpathlineto{\pgfqpoint{1.768586in}{1.326763in}}%
\pgfpathlineto{\pgfqpoint{1.771397in}{1.327411in}}%
\pgfpathlineto{\pgfqpoint{1.774208in}{1.327209in}}%
\pgfpathlineto{\pgfqpoint{1.777018in}{1.327380in}}%
\pgfpathlineto{\pgfqpoint{1.779829in}{1.328176in}}%
\pgfpathlineto{\pgfqpoint{1.782640in}{1.328968in}}%
\pgfpathlineto{\pgfqpoint{1.785451in}{1.327700in}}%
\pgfpathlineto{\pgfqpoint{1.788261in}{1.328413in}}%
\pgfpathlineto{\pgfqpoint{1.791072in}{1.326305in}}%
\pgfpathlineto{\pgfqpoint{1.793883in}{1.327076in}}%
\pgfpathlineto{\pgfqpoint{1.796693in}{1.326870in}}%
\pgfpathlineto{\pgfqpoint{1.799504in}{1.327646in}}%
\pgfpathlineto{\pgfqpoint{1.802315in}{1.327271in}}%
\pgfpathlineto{\pgfqpoint{1.805125in}{1.327938in}}%
\pgfpathlineto{\pgfqpoint{1.807936in}{1.328635in}}%
\pgfpathlineto{\pgfqpoint{1.810747in}{1.328015in}}%
\pgfpathlineto{\pgfqpoint{1.813557in}{1.324064in}}%
\pgfpathlineto{\pgfqpoint{1.816368in}{1.324214in}}%
\pgfpathlineto{\pgfqpoint{1.819179in}{1.322997in}}%
\pgfpathlineto{\pgfqpoint{1.821989in}{1.323117in}}%
\pgfpathlineto{\pgfqpoint{1.824800in}{1.323671in}}%
\pgfpathlineto{\pgfqpoint{1.827611in}{1.324006in}}%
\pgfpathlineto{\pgfqpoint{1.830422in}{1.324675in}}%
\pgfpathlineto{\pgfqpoint{1.833232in}{1.325430in}}%
\pgfpathlineto{\pgfqpoint{1.836043in}{1.326172in}}%
\pgfpathlineto{\pgfqpoint{1.838854in}{1.325867in}}%
\pgfpathlineto{\pgfqpoint{1.841664in}{1.326550in}}%
\pgfpathlineto{\pgfqpoint{1.844475in}{1.327214in}}%
\pgfpathlineto{\pgfqpoint{1.847286in}{1.326075in}}%
\pgfpathlineto{\pgfqpoint{1.850096in}{1.319011in}}%
\pgfpathlineto{\pgfqpoint{1.852907in}{1.318996in}}%
\pgfpathlineto{\pgfqpoint{1.855718in}{1.319237in}}%
\pgfpathlineto{\pgfqpoint{1.858528in}{1.319950in}}%
\pgfpathlineto{\pgfqpoint{1.861339in}{1.320229in}}%
\pgfpathlineto{\pgfqpoint{1.864150in}{1.320466in}}%
\pgfpathlineto{\pgfqpoint{1.866960in}{1.321018in}}%
\pgfpathlineto{\pgfqpoint{1.869771in}{1.321567in}}%
\pgfpathlineto{\pgfqpoint{1.872582in}{1.322301in}}%
\pgfpathlineto{\pgfqpoint{1.875393in}{1.322874in}}%
\pgfpathlineto{\pgfqpoint{1.878203in}{1.322329in}}%
\pgfpathlineto{\pgfqpoint{1.881014in}{1.321993in}}%
\pgfpathlineto{\pgfqpoint{1.883825in}{1.320653in}}%
\pgfpathlineto{\pgfqpoint{1.886635in}{1.321125in}}%
\pgfpathlineto{\pgfqpoint{1.889446in}{1.320969in}}%
\pgfpathlineto{\pgfqpoint{1.892257in}{1.320365in}}%
\pgfpathlineto{\pgfqpoint{1.895067in}{1.321009in}}%
\pgfpathlineto{\pgfqpoint{1.897878in}{1.321334in}}%
\pgfpathlineto{\pgfqpoint{1.900689in}{1.320861in}}%
\pgfpathlineto{\pgfqpoint{1.903499in}{1.321377in}}%
\pgfpathlineto{\pgfqpoint{1.906310in}{1.320108in}}%
\pgfpathlineto{\pgfqpoint{1.909121in}{1.320786in}}%
\pgfpathlineto{\pgfqpoint{1.911931in}{1.321261in}}%
\pgfpathlineto{\pgfqpoint{1.914742in}{1.319914in}}%
\pgfpathlineto{\pgfqpoint{1.917553in}{1.320573in}}%
\pgfpathlineto{\pgfqpoint{1.920364in}{1.321185in}}%
\pgfpathlineto{\pgfqpoint{1.923174in}{1.321804in}}%
\pgfpathlineto{\pgfqpoint{1.925985in}{1.320765in}}%
\pgfpathlineto{\pgfqpoint{1.928796in}{1.321049in}}%
\pgfpathlineto{\pgfqpoint{1.931606in}{1.321487in}}%
\pgfpathlineto{\pgfqpoint{1.934417in}{1.321394in}}%
\pgfpathlineto{\pgfqpoint{1.937228in}{1.322081in}}%
\pgfpathlineto{\pgfqpoint{1.940038in}{1.322763in}}%
\pgfpathlineto{\pgfqpoint{1.942849in}{1.322626in}}%
\pgfpathlineto{\pgfqpoint{1.945660in}{1.320123in}}%
\pgfpathlineto{\pgfqpoint{1.948470in}{1.320785in}}%
\pgfpathlineto{\pgfqpoint{1.951281in}{1.320736in}}%
\pgfpathlineto{\pgfqpoint{1.954092in}{1.321364in}}%
\pgfpathlineto{\pgfqpoint{1.956902in}{1.321993in}}%
\pgfpathlineto{\pgfqpoint{1.959713in}{1.322664in}}%
\pgfpathlineto{\pgfqpoint{1.962524in}{1.322484in}}%
\pgfpathlineto{\pgfqpoint{1.965334in}{1.323072in}}%
\pgfpathlineto{\pgfqpoint{1.968145in}{1.323532in}}%
\pgfpathlineto{\pgfqpoint{1.970956in}{1.324189in}}%
\pgfpathlineto{\pgfqpoint{1.973767in}{1.324432in}}%
\pgfpathlineto{\pgfqpoint{1.976577in}{1.323490in}}%
\pgfpathlineto{\pgfqpoint{1.979388in}{1.323623in}}%
\pgfpathlineto{\pgfqpoint{1.982199in}{1.323896in}}%
\pgfpathlineto{\pgfqpoint{1.985009in}{1.324293in}}%
\pgfpathlineto{\pgfqpoint{1.987820in}{1.324884in}}%
\pgfpathlineto{\pgfqpoint{1.990631in}{1.325185in}}%
\pgfpathlineto{\pgfqpoint{1.993441in}{1.325317in}}%
\pgfpathlineto{\pgfqpoint{1.996252in}{1.325815in}}%
\pgfpathlineto{\pgfqpoint{1.999063in}{1.326453in}}%
\pgfpathlineto{\pgfqpoint{2.001873in}{1.326489in}}%
\pgfpathlineto{\pgfqpoint{2.004684in}{1.327121in}}%
\pgfpathlineto{\pgfqpoint{2.007495in}{1.326237in}}%
\pgfpathlineto{\pgfqpoint{2.010305in}{1.325882in}}%
\pgfpathlineto{\pgfqpoint{2.013116in}{1.323183in}}%
\pgfpathlineto{\pgfqpoint{2.015927in}{1.323070in}}%
\pgfpathlineto{\pgfqpoint{2.018738in}{1.323059in}}%
\pgfpathlineto{\pgfqpoint{2.021548in}{1.323327in}}%
\pgfpathlineto{\pgfqpoint{2.024359in}{1.301918in}}%
\pgfpathlineto{\pgfqpoint{2.027170in}{1.302540in}}%
\pgfpathlineto{\pgfqpoint{2.029980in}{1.302253in}}%
\pgfpathlineto{\pgfqpoint{2.032791in}{1.302826in}}%
\pgfpathlineto{\pgfqpoint{2.035602in}{1.302612in}}%
\pgfpathlineto{\pgfqpoint{2.038412in}{1.302463in}}%
\pgfpathlineto{\pgfqpoint{2.041223in}{1.302721in}}%
\pgfpathlineto{\pgfqpoint{2.044034in}{1.302838in}}%
\pgfpathlineto{\pgfqpoint{2.046844in}{1.303274in}}%
\pgfpathlineto{\pgfqpoint{2.049655in}{1.303559in}}%
\pgfpathlineto{\pgfqpoint{2.052466in}{1.304101in}}%
\pgfpathlineto{\pgfqpoint{2.055276in}{1.301619in}}%
\pgfpathlineto{\pgfqpoint{2.058087in}{1.301784in}}%
\pgfpathlineto{\pgfqpoint{2.060898in}{1.302405in}}%
\pgfpathlineto{\pgfqpoint{2.063709in}{1.302091in}}%
\pgfpathlineto{\pgfqpoint{2.066519in}{1.301479in}}%
\pgfpathlineto{\pgfqpoint{2.069330in}{1.302032in}}%
\pgfpathlineto{\pgfqpoint{2.072141in}{1.302513in}}%
\pgfpathlineto{\pgfqpoint{2.074951in}{1.300452in}}%
\pgfpathlineto{\pgfqpoint{2.077762in}{1.300601in}}%
\pgfpathlineto{\pgfqpoint{2.080573in}{1.301214in}}%
\pgfpathlineto{\pgfqpoint{2.083383in}{1.301694in}}%
\pgfpathlineto{\pgfqpoint{2.086194in}{1.301804in}}%
\pgfpathlineto{\pgfqpoint{2.089005in}{1.302311in}}%
\pgfpathlineto{\pgfqpoint{2.091815in}{1.302477in}}%
\pgfpathlineto{\pgfqpoint{2.094626in}{1.302920in}}%
\pgfpathlineto{\pgfqpoint{2.097437in}{1.303444in}}%
\pgfpathlineto{\pgfqpoint{2.100247in}{1.304042in}}%
\pgfpathlineto{\pgfqpoint{2.103058in}{1.303000in}}%
\pgfpathlineto{\pgfqpoint{2.105869in}{1.303612in}}%
\pgfpathlineto{\pgfqpoint{2.108680in}{1.303610in}}%
\pgfpathlineto{\pgfqpoint{2.111490in}{1.303793in}}%
\pgfpathlineto{\pgfqpoint{2.114301in}{1.304294in}}%
\pgfpathlineto{\pgfqpoint{2.117112in}{1.304875in}}%
\pgfpathlineto{\pgfqpoint{2.119922in}{1.305466in}}%
\pgfpathlineto{\pgfqpoint{2.122733in}{1.306044in}}%
\pgfpathlineto{\pgfqpoint{2.125544in}{1.306239in}}%
\pgfpathlineto{\pgfqpoint{2.128354in}{1.305606in}}%
\pgfpathlineto{\pgfqpoint{2.131165in}{1.304773in}}%
\pgfpathlineto{\pgfqpoint{2.133976in}{1.304555in}}%
\pgfpathlineto{\pgfqpoint{2.136786in}{1.305142in}}%
\pgfpathlineto{\pgfqpoint{2.139597in}{1.305428in}}%
\pgfpathlineto{\pgfqpoint{2.142408in}{1.305124in}}%
\pgfpathlineto{\pgfqpoint{2.145218in}{1.305698in}}%
\pgfpathlineto{\pgfqpoint{2.148029in}{1.305637in}}%
\pgfpathlineto{\pgfqpoint{2.150840in}{1.304609in}}%
\pgfpathlineto{\pgfqpoint{2.153651in}{1.305193in}}%
\pgfpathlineto{\pgfqpoint{2.156461in}{1.303396in}}%
\pgfpathlineto{\pgfqpoint{2.159272in}{1.301865in}}%
\pgfpathlineto{\pgfqpoint{2.162083in}{1.302447in}}%
\pgfpathlineto{\pgfqpoint{2.164893in}{1.302060in}}%
\pgfpathlineto{\pgfqpoint{2.167704in}{1.302474in}}%
\pgfpathlineto{\pgfqpoint{2.170515in}{1.303048in}}%
\pgfpathlineto{\pgfqpoint{2.173325in}{1.302696in}}%
\pgfpathlineto{\pgfqpoint{2.176136in}{1.302816in}}%
\pgfpathlineto{\pgfqpoint{2.178947in}{1.303328in}}%
\pgfpathlineto{\pgfqpoint{2.181757in}{1.303892in}}%
\pgfpathlineto{\pgfqpoint{2.184568in}{1.304184in}}%
\pgfpathlineto{\pgfqpoint{2.187379in}{1.304569in}}%
\pgfpathlineto{\pgfqpoint{2.190189in}{1.304660in}}%
\pgfpathlineto{\pgfqpoint{2.193000in}{1.304897in}}%
\pgfpathlineto{\pgfqpoint{2.195811in}{1.305314in}}%
\pgfpathlineto{\pgfqpoint{2.198621in}{1.305703in}}%
\pgfpathlineto{\pgfqpoint{2.201432in}{1.306251in}}%
\pgfpathlineto{\pgfqpoint{2.204243in}{1.306548in}}%
\pgfpathlineto{\pgfqpoint{2.207054in}{1.302261in}}%
\pgfpathlineto{\pgfqpoint{2.209864in}{1.302782in}}%
\pgfpathlineto{\pgfqpoint{2.212675in}{1.302137in}}%
\pgfpathlineto{\pgfqpoint{2.215486in}{1.302669in}}%
\pgfpathlineto{\pgfqpoint{2.218296in}{1.303230in}}%
\pgfpathlineto{\pgfqpoint{2.221107in}{1.302466in}}%
\pgfpathlineto{\pgfqpoint{2.223918in}{1.303006in}}%
\pgfpathlineto{\pgfqpoint{2.226728in}{1.303531in}}%
\pgfpathlineto{\pgfqpoint{2.229539in}{1.303650in}}%
\pgfpathlineto{\pgfqpoint{2.232350in}{1.304072in}}%
\pgfpathlineto{\pgfqpoint{2.235160in}{1.304516in}}%
\pgfpathlineto{\pgfqpoint{2.237971in}{1.305066in}}%
\pgfpathlineto{\pgfqpoint{2.240782in}{1.304775in}}%
\pgfpathlineto{\pgfqpoint{2.243592in}{1.304248in}}%
\pgfpathlineto{\pgfqpoint{2.246403in}{1.304743in}}%
\pgfpathlineto{\pgfqpoint{2.249214in}{1.305094in}}%
\pgfpathlineto{\pgfqpoint{2.252025in}{1.305556in}}%
\pgfpathlineto{\pgfqpoint{2.254835in}{1.305895in}}%
\pgfpathlineto{\pgfqpoint{2.257646in}{1.306435in}}%
\pgfpathlineto{\pgfqpoint{2.260457in}{1.306887in}}%
\pgfpathlineto{\pgfqpoint{2.263267in}{1.307130in}}%
\pgfpathlineto{\pgfqpoint{2.266078in}{1.307666in}}%
\pgfpathlineto{\pgfqpoint{2.268889in}{1.306737in}}%
\pgfpathlineto{\pgfqpoint{2.271699in}{1.307243in}}%
\pgfpathlineto{\pgfqpoint{2.274510in}{1.307348in}}%
\pgfpathlineto{\pgfqpoint{2.277321in}{1.307739in}}%
\pgfpathlineto{\pgfqpoint{2.280131in}{1.308050in}}%
\pgfpathlineto{\pgfqpoint{2.282942in}{1.307726in}}%
\pgfpathlineto{\pgfqpoint{2.285753in}{1.307074in}}%
\pgfpathlineto{\pgfqpoint{2.288563in}{1.306733in}}%
\pgfpathlineto{\pgfqpoint{2.291374in}{1.306994in}}%
\pgfpathlineto{\pgfqpoint{2.294185in}{1.307421in}}%
\pgfpathlineto{\pgfqpoint{2.296996in}{1.307201in}}%
\pgfpathlineto{\pgfqpoint{2.299806in}{1.307722in}}%
\pgfpathlineto{\pgfqpoint{2.302617in}{1.308185in}}%
\pgfpathlineto{\pgfqpoint{2.305428in}{1.308526in}}%
\pgfpathlineto{\pgfqpoint{2.308238in}{1.309043in}}%
\pgfpathlineto{\pgfqpoint{2.311049in}{1.309496in}}%
\pgfpathlineto{\pgfqpoint{2.313860in}{1.310005in}}%
\pgfpathlineto{\pgfqpoint{2.316670in}{1.310253in}}%
\pgfpathlineto{\pgfqpoint{2.319481in}{1.310749in}}%
\pgfpathlineto{\pgfqpoint{2.322292in}{1.311260in}}%
\pgfpathlineto{\pgfqpoint{2.325102in}{1.311185in}}%
\pgfpathlineto{\pgfqpoint{2.327913in}{1.311561in}}%
\pgfpathlineto{\pgfqpoint{2.330724in}{1.311591in}}%
\pgfpathlineto{\pgfqpoint{2.333534in}{1.312097in}}%
\pgfpathlineto{\pgfqpoint{2.336345in}{1.312538in}}%
\pgfpathlineto{\pgfqpoint{2.339156in}{1.312825in}}%
\pgfpathlineto{\pgfqpoint{2.341967in}{1.313006in}}%
\pgfpathlineto{\pgfqpoint{2.344777in}{1.312728in}}%
\pgfpathlineto{\pgfqpoint{2.347588in}{1.313054in}}%
\pgfpathlineto{\pgfqpoint{2.350399in}{1.313524in}}%
\pgfpathlineto{\pgfqpoint{2.353209in}{1.313350in}}%
\pgfpathlineto{\pgfqpoint{2.356020in}{1.310682in}}%
\pgfpathlineto{\pgfqpoint{2.358831in}{1.310976in}}%
\pgfpathlineto{\pgfqpoint{2.361641in}{1.311438in}}%
\pgfpathlineto{\pgfqpoint{2.364452in}{1.310069in}}%
\pgfpathlineto{\pgfqpoint{2.367263in}{1.307096in}}%
\pgfpathlineto{\pgfqpoint{2.370073in}{1.305426in}}%
\pgfpathlineto{\pgfqpoint{2.372884in}{1.305545in}}%
\pgfpathlineto{\pgfqpoint{2.375695in}{1.304579in}}%
\pgfpathlineto{\pgfqpoint{2.378505in}{1.304915in}}%
\pgfpathlineto{\pgfqpoint{2.381316in}{1.302982in}}%
\pgfpathlineto{\pgfqpoint{2.384127in}{1.302822in}}%
\pgfpathlineto{\pgfqpoint{2.386937in}{1.303294in}}%
\pgfpathlineto{\pgfqpoint{2.389748in}{1.303764in}}%
\pgfpathlineto{\pgfqpoint{2.392559in}{1.304147in}}%
\pgfpathlineto{\pgfqpoint{2.395370in}{1.304605in}}%
\pgfpathlineto{\pgfqpoint{2.398180in}{1.304683in}}%
\pgfpathlineto{\pgfqpoint{2.400991in}{1.305173in}}%
\pgfpathlineto{\pgfqpoint{2.403802in}{1.305555in}}%
\pgfpathlineto{\pgfqpoint{2.406612in}{1.305869in}}%
\pgfpathlineto{\pgfqpoint{2.409423in}{1.306356in}}%
\pgfpathlineto{\pgfqpoint{2.412234in}{1.306696in}}%
\pgfpathlineto{\pgfqpoint{2.415044in}{1.306480in}}%
\pgfpathlineto{\pgfqpoint{2.417855in}{1.306736in}}%
\pgfpathlineto{\pgfqpoint{2.420666in}{1.307053in}}%
\pgfpathlineto{\pgfqpoint{2.423476in}{1.307498in}}%
\pgfpathlineto{\pgfqpoint{2.426287in}{1.307939in}}%
\pgfpathlineto{\pgfqpoint{2.429098in}{1.308063in}}%
\pgfpathlineto{\pgfqpoint{2.431908in}{1.308416in}}%
\pgfpathlineto{\pgfqpoint{2.434719in}{1.308684in}}%
\pgfpathlineto{\pgfqpoint{2.437530in}{1.307455in}}%
\pgfpathlineto{\pgfqpoint{2.440341in}{1.307729in}}%
\pgfpathlineto{\pgfqpoint{2.443151in}{1.307331in}}%
\pgfpathlineto{\pgfqpoint{2.445962in}{1.307774in}}%
\pgfpathlineto{\pgfqpoint{2.448773in}{1.308244in}}%
\pgfpathlineto{\pgfqpoint{2.451583in}{1.308561in}}%
\pgfpathlineto{\pgfqpoint{2.454394in}{1.309028in}}%
\pgfpathlineto{\pgfqpoint{2.457205in}{1.309139in}}%
\pgfpathlineto{\pgfqpoint{2.460015in}{1.307444in}}%
\pgfpathlineto{\pgfqpoint{2.462826in}{1.307767in}}%
\pgfpathlineto{\pgfqpoint{2.465637in}{1.308175in}}%
\pgfpathlineto{\pgfqpoint{2.468447in}{1.308618in}}%
\pgfpathlineto{\pgfqpoint{2.471258in}{1.308278in}}%
\pgfpathlineto{\pgfqpoint{2.474069in}{1.308121in}}%
\pgfpathlineto{\pgfqpoint{2.476879in}{1.308579in}}%
\pgfpathlineto{\pgfqpoint{2.479690in}{1.308213in}}%
\pgfpathlineto{\pgfqpoint{2.482501in}{1.307918in}}%
\pgfpathlineto{\pgfqpoint{2.485312in}{1.307503in}}%
\pgfpathlineto{\pgfqpoint{2.488122in}{1.307835in}}%
\pgfpathlineto{\pgfqpoint{2.490933in}{1.306953in}}%
\pgfpathlineto{\pgfqpoint{2.493744in}{1.307225in}}%
\pgfpathlineto{\pgfqpoint{2.496554in}{1.307305in}}%
\pgfpathlineto{\pgfqpoint{2.499365in}{1.307616in}}%
\pgfpathlineto{\pgfqpoint{2.502176in}{1.307980in}}%
\pgfpathlineto{\pgfqpoint{2.504986in}{1.304701in}}%
\pgfpathlineto{\pgfqpoint{2.507797in}{1.303470in}}%
\pgfpathlineto{\pgfqpoint{2.510608in}{1.303617in}}%
\pgfpathlineto{\pgfqpoint{2.513418in}{1.303546in}}%
\pgfpathlineto{\pgfqpoint{2.516229in}{1.301177in}}%
\pgfpathlineto{\pgfqpoint{2.519040in}{1.299494in}}%
\pgfpathlineto{\pgfqpoint{2.521850in}{1.299575in}}%
\pgfpathlineto{\pgfqpoint{2.524661in}{1.299933in}}%
\pgfpathlineto{\pgfqpoint{2.527472in}{1.299793in}}%
\pgfpathlineto{\pgfqpoint{2.530283in}{1.299804in}}%
\pgfpathlineto{\pgfqpoint{2.533093in}{1.300244in}}%
\pgfpathlineto{\pgfqpoint{2.535904in}{1.300661in}}%
\pgfpathlineto{\pgfqpoint{2.538715in}{1.300741in}}%
\pgfpathlineto{\pgfqpoint{2.541525in}{1.301129in}}%
\pgfpathlineto{\pgfqpoint{2.544336in}{1.294714in}}%
\pgfpathlineto{\pgfqpoint{2.547147in}{1.295133in}}%
\pgfpathlineto{\pgfqpoint{2.549957in}{1.295576in}}%
\pgfpathlineto{\pgfqpoint{2.552768in}{1.295889in}}%
\pgfpathlineto{\pgfqpoint{2.555579in}{1.296187in}}%
\pgfpathlineto{\pgfqpoint{2.558389in}{1.296109in}}%
\pgfpathlineto{\pgfqpoint{2.561200in}{1.296509in}}%
\pgfpathlineto{\pgfqpoint{2.564011in}{1.296407in}}%
\pgfpathlineto{\pgfqpoint{2.566821in}{1.296730in}}%
\pgfpathlineto{\pgfqpoint{2.569632in}{1.297166in}}%
\pgfpathlineto{\pgfqpoint{2.572443in}{1.297467in}}%
\pgfpathlineto{\pgfqpoint{2.575253in}{1.297879in}}%
\pgfpathlineto{\pgfqpoint{2.578064in}{1.298242in}}%
\pgfpathlineto{\pgfqpoint{2.580875in}{1.298441in}}%
\pgfpathlineto{\pgfqpoint{2.583686in}{1.298136in}}%
\pgfpathlineto{\pgfqpoint{2.586496in}{1.298556in}}%
\pgfpathlineto{\pgfqpoint{2.589307in}{1.298935in}}%
\pgfpathlineto{\pgfqpoint{2.592118in}{1.298633in}}%
\pgfpathlineto{\pgfqpoint{2.594928in}{1.299061in}}%
\pgfpathlineto{\pgfqpoint{2.597739in}{1.299240in}}%
\pgfpathlineto{\pgfqpoint{2.600550in}{1.299634in}}%
\pgfpathlineto{\pgfqpoint{2.603360in}{1.300049in}}%
\pgfpathlineto{\pgfqpoint{2.606171in}{1.300237in}}%
\pgfpathlineto{\pgfqpoint{2.608982in}{1.300634in}}%
\pgfpathlineto{\pgfqpoint{2.611792in}{1.300961in}}%
\pgfpathlineto{\pgfqpoint{2.614603in}{1.301004in}}%
\pgfpathlineto{\pgfqpoint{2.617414in}{1.300788in}}%
\pgfpathlineto{\pgfqpoint{2.620224in}{1.301108in}}%
\pgfpathlineto{\pgfqpoint{2.623035in}{1.301495in}}%
\pgfpathlineto{\pgfqpoint{2.625846in}{1.301884in}}%
\pgfpathlineto{\pgfqpoint{2.628657in}{1.302116in}}%
\pgfpathlineto{\pgfqpoint{2.631467in}{1.302539in}}%
\pgfpathlineto{\pgfqpoint{2.634278in}{1.302514in}}%
\pgfpathlineto{\pgfqpoint{2.637089in}{1.302653in}}%
\pgfpathlineto{\pgfqpoint{2.639899in}{1.302867in}}%
\pgfpathlineto{\pgfqpoint{2.642710in}{1.302966in}}%
\pgfpathlineto{\pgfqpoint{2.645521in}{1.303332in}}%
\pgfpathlineto{\pgfqpoint{2.648331in}{1.303681in}}%
\pgfpathlineto{\pgfqpoint{2.651142in}{1.303807in}}%
\pgfpathlineto{\pgfqpoint{2.653953in}{1.303960in}}%
\pgfpathlineto{\pgfqpoint{2.656763in}{1.304248in}}%
\pgfpathlineto{\pgfqpoint{2.659574in}{1.304190in}}%
\pgfpathlineto{\pgfqpoint{2.662385in}{1.304596in}}%
\pgfpathlineto{\pgfqpoint{2.665195in}{1.304869in}}%
\pgfpathlineto{\pgfqpoint{2.668006in}{1.305204in}}%
\pgfpathlineto{\pgfqpoint{2.670817in}{1.305612in}}%
\pgfpathlineto{\pgfqpoint{2.673628in}{1.305889in}}%
\pgfpathlineto{\pgfqpoint{2.676438in}{1.306246in}}%
\pgfpathlineto{\pgfqpoint{2.679249in}{1.306653in}}%
\pgfpathlineto{\pgfqpoint{2.682060in}{1.306967in}}%
\pgfpathlineto{\pgfqpoint{2.684870in}{1.306965in}}%
\pgfpathlineto{\pgfqpoint{2.687681in}{1.307342in}}%
\pgfpathlineto{\pgfqpoint{2.690492in}{1.307533in}}%
\pgfpathlineto{\pgfqpoint{2.693302in}{1.307932in}}%
\pgfpathlineto{\pgfqpoint{2.696113in}{1.308134in}}%
\pgfpathlineto{\pgfqpoint{2.698924in}{1.308493in}}%
\pgfpathlineto{\pgfqpoint{2.701734in}{1.308864in}}%
\pgfpathlineto{\pgfqpoint{2.704545in}{1.307654in}}%
\pgfpathlineto{\pgfqpoint{2.707356in}{1.308042in}}%
\pgfpathlineto{\pgfqpoint{2.710166in}{1.308159in}}%
\pgfpathlineto{\pgfqpoint{2.712977in}{1.308555in}}%
\pgfpathlineto{\pgfqpoint{2.715788in}{1.308829in}}%
\pgfpathlineto{\pgfqpoint{2.718599in}{1.309223in}}%
\pgfpathlineto{\pgfqpoint{2.721409in}{1.306073in}}%
\pgfpathlineto{\pgfqpoint{2.724220in}{1.306269in}}%
\pgfpathlineto{\pgfqpoint{2.727031in}{1.306425in}}%
\pgfpathlineto{\pgfqpoint{2.729841in}{1.306771in}}%
\pgfpathlineto{\pgfqpoint{2.732652in}{1.306275in}}%
\pgfpathlineto{\pgfqpoint{2.735463in}{1.306638in}}%
\pgfpathlineto{\pgfqpoint{2.738273in}{1.306958in}}%
\pgfpathlineto{\pgfqpoint{2.741084in}{1.306918in}}%
\pgfpathlineto{\pgfqpoint{2.743895in}{1.307213in}}%
\pgfpathlineto{\pgfqpoint{2.746705in}{1.307160in}}%
\pgfpathlineto{\pgfqpoint{2.749516in}{1.307547in}}%
\pgfpathlineto{\pgfqpoint{2.752327in}{1.307835in}}%
\pgfpathlineto{\pgfqpoint{2.755137in}{1.308075in}}%
\pgfpathlineto{\pgfqpoint{2.757948in}{1.308417in}}%
\pgfpathlineto{\pgfqpoint{2.760759in}{1.308790in}}%
\pgfpathlineto{\pgfqpoint{2.763570in}{1.308533in}}%
\pgfpathlineto{\pgfqpoint{2.766380in}{1.308680in}}%
\pgfpathlineto{\pgfqpoint{2.769191in}{1.309023in}}%
\pgfpathlineto{\pgfqpoint{2.772002in}{1.309398in}}%
\pgfpathlineto{\pgfqpoint{2.774812in}{1.309601in}}%
\pgfpathlineto{\pgfqpoint{2.777623in}{1.309912in}}%
\pgfpathlineto{\pgfqpoint{2.780434in}{1.310171in}}%
\pgfpathlineto{\pgfqpoint{2.783244in}{1.310514in}}%
\pgfpathlineto{\pgfqpoint{2.786055in}{1.310830in}}%
\pgfpathlineto{\pgfqpoint{2.788866in}{1.310523in}}%
\pgfpathlineto{\pgfqpoint{2.791676in}{1.310364in}}%
\pgfpathlineto{\pgfqpoint{2.794487in}{1.310732in}}%
\pgfpathlineto{\pgfqpoint{2.797298in}{1.311025in}}%
\pgfpathlineto{\pgfqpoint{2.800108in}{1.311189in}}%
\pgfpathlineto{\pgfqpoint{2.802919in}{1.311450in}}%
\pgfpathlineto{\pgfqpoint{2.805730in}{1.311824in}}%
\pgfpathlineto{\pgfqpoint{2.808540in}{1.311821in}}%
\pgfpathlineto{\pgfqpoint{2.811351in}{1.312163in}}%
\pgfpathlineto{\pgfqpoint{2.814162in}{1.312044in}}%
\pgfpathlineto{\pgfqpoint{2.816973in}{1.312159in}}%
\pgfpathlineto{\pgfqpoint{2.819783in}{1.312484in}}%
\pgfpathlineto{\pgfqpoint{2.822594in}{1.312818in}}%
\pgfpathlineto{\pgfqpoint{2.825405in}{1.312744in}}%
\pgfpathlineto{\pgfqpoint{2.828215in}{1.313088in}}%
\pgfpathlineto{\pgfqpoint{2.831026in}{1.313328in}}%
\pgfpathlineto{\pgfqpoint{2.833837in}{1.313075in}}%
\pgfpathlineto{\pgfqpoint{2.836647in}{1.313079in}}%
\pgfpathlineto{\pgfqpoint{2.839458in}{1.313441in}}%
\pgfpathlineto{\pgfqpoint{2.842269in}{1.312717in}}%
\pgfpathlineto{\pgfqpoint{2.845079in}{1.313075in}}%
\pgfpathlineto{\pgfqpoint{2.847890in}{1.313168in}}%
\pgfpathlineto{\pgfqpoint{2.850701in}{1.313509in}}%
\pgfpathlineto{\pgfqpoint{2.853511in}{1.312915in}}%
\pgfpathlineto{\pgfqpoint{2.856322in}{1.313134in}}%
\pgfpathlineto{\pgfqpoint{2.859133in}{1.313480in}}%
\pgfpathlineto{\pgfqpoint{2.861944in}{1.313710in}}%
\pgfpathlineto{\pgfqpoint{2.864754in}{1.312959in}}%
\pgfpathlineto{\pgfqpoint{2.867565in}{1.313041in}}%
\pgfpathlineto{\pgfqpoint{2.870376in}{1.311882in}}%
\pgfpathlineto{\pgfqpoint{2.873186in}{1.311499in}}%
\pgfpathlineto{\pgfqpoint{2.875997in}{1.311654in}}%
\pgfpathlineto{\pgfqpoint{2.878808in}{1.311604in}}%
\pgfpathlineto{\pgfqpoint{2.881618in}{1.311303in}}%
\pgfpathlineto{\pgfqpoint{2.884429in}{1.311603in}}%
\pgfpathlineto{\pgfqpoint{2.887240in}{1.311877in}}%
\pgfpathlineto{\pgfqpoint{2.890050in}{1.312231in}}%
\pgfpathlineto{\pgfqpoint{2.892861in}{1.312037in}}%
\pgfpathlineto{\pgfqpoint{2.895672in}{1.311787in}}%
\pgfpathlineto{\pgfqpoint{2.898482in}{1.312035in}}%
\pgfpathlineto{\pgfqpoint{2.901293in}{1.312169in}}%
\pgfpathlineto{\pgfqpoint{2.904104in}{1.312412in}}%
\pgfpathlineto{\pgfqpoint{2.906915in}{1.312669in}}%
\pgfpathlineto{\pgfqpoint{2.909725in}{1.312524in}}%
\pgfpathlineto{\pgfqpoint{2.912536in}{1.299838in}}%
\pgfpathlineto{\pgfqpoint{2.915347in}{1.299980in}}%
\pgfpathlineto{\pgfqpoint{2.918157in}{1.300247in}}%
\pgfpathlineto{\pgfqpoint{2.920968in}{1.300602in}}%
\pgfpathlineto{\pgfqpoint{2.923779in}{1.300420in}}%
\pgfpathlineto{\pgfqpoint{2.926589in}{1.300726in}}%
\pgfpathlineto{\pgfqpoint{2.929400in}{1.301086in}}%
\pgfpathlineto{\pgfqpoint{2.932211in}{1.300986in}}%
\pgfpathlineto{\pgfqpoint{2.935021in}{1.301182in}}%
\pgfpathlineto{\pgfqpoint{2.937832in}{1.301430in}}%
\pgfpathlineto{\pgfqpoint{2.940643in}{1.301549in}}%
\pgfpathlineto{\pgfqpoint{2.943453in}{1.301477in}}%
\pgfpathlineto{\pgfqpoint{2.946264in}{1.301806in}}%
\pgfpathlineto{\pgfqpoint{2.949075in}{1.302049in}}%
\pgfpathlineto{\pgfqpoint{2.951886in}{1.302317in}}%
\pgfpathlineto{\pgfqpoint{2.954696in}{1.302672in}}%
\pgfpathlineto{\pgfqpoint{2.957507in}{1.303023in}}%
\pgfpathlineto{\pgfqpoint{2.960318in}{1.303310in}}%
\pgfpathlineto{\pgfqpoint{2.963128in}{1.303659in}}%
\pgfpathlineto{\pgfqpoint{2.965939in}{1.303907in}}%
\pgfpathlineto{\pgfqpoint{2.968750in}{1.304226in}}%
\pgfpathlineto{\pgfqpoint{2.971560in}{1.304473in}}%
\pgfpathlineto{\pgfqpoint{2.974371in}{1.304823in}}%
\pgfpathlineto{\pgfqpoint{2.977182in}{1.305167in}}%
\pgfpathlineto{\pgfqpoint{2.979992in}{1.305363in}}%
\pgfpathlineto{\pgfqpoint{2.982803in}{1.305713in}}%
\pgfpathlineto{\pgfqpoint{2.985614in}{1.305931in}}%
\pgfpathlineto{\pgfqpoint{2.988424in}{1.306252in}}%
\pgfpathlineto{\pgfqpoint{2.991235in}{1.306130in}}%
\pgfpathlineto{\pgfqpoint{2.994046in}{1.306465in}}%
\pgfpathlineto{\pgfqpoint{2.996856in}{1.305172in}}%
\pgfpathlineto{\pgfqpoint{2.999667in}{1.305407in}}%
\pgfpathlineto{\pgfqpoint{3.002478in}{1.305069in}}%
\pgfpathlineto{\pgfqpoint{3.005289in}{1.305115in}}%
\pgfpathlineto{\pgfqpoint{3.008099in}{1.305266in}}%
\pgfpathlineto{\pgfqpoint{3.010910in}{1.305130in}}%
\pgfpathlineto{\pgfqpoint{3.013721in}{1.305471in}}%
\pgfpathlineto{\pgfqpoint{3.016531in}{1.305787in}}%
\pgfpathlineto{\pgfqpoint{3.019342in}{1.306130in}}%
\pgfpathlineto{\pgfqpoint{3.022153in}{1.306254in}}%
\pgfpathlineto{\pgfqpoint{3.024963in}{1.306349in}}%
\pgfpathlineto{\pgfqpoint{3.027774in}{1.306489in}}%
\pgfpathlineto{\pgfqpoint{3.030585in}{1.306387in}}%
\pgfpathlineto{\pgfqpoint{3.033395in}{1.306718in}}%
\pgfpathlineto{\pgfqpoint{3.036206in}{1.305651in}}%
\pgfpathlineto{\pgfqpoint{3.039017in}{1.305667in}}%
\pgfpathlineto{\pgfqpoint{3.041827in}{1.305965in}}%
\pgfpathlineto{\pgfqpoint{3.044638in}{1.306263in}}%
\pgfpathlineto{\pgfqpoint{3.047449in}{1.305796in}}%
\pgfpathlineto{\pgfqpoint{3.050260in}{1.305976in}}%
\pgfpathlineto{\pgfqpoint{3.053070in}{1.306274in}}%
\pgfpathlineto{\pgfqpoint{3.055881in}{1.305389in}}%
\pgfpathlineto{\pgfqpoint{3.058692in}{1.305259in}}%
\pgfpathlineto{\pgfqpoint{3.061502in}{1.305593in}}%
\pgfpathlineto{\pgfqpoint{3.064313in}{1.305928in}}%
\pgfpathlineto{\pgfqpoint{3.067124in}{1.306202in}}%
\pgfpathlineto{\pgfqpoint{3.069934in}{1.306464in}}%
\pgfpathlineto{\pgfqpoint{3.072745in}{1.306754in}}%
\pgfpathlineto{\pgfqpoint{3.075556in}{1.306755in}}%
\pgfpathlineto{\pgfqpoint{3.078366in}{1.305749in}}%
\pgfpathlineto{\pgfqpoint{3.081177in}{1.305008in}}%
\pgfpathlineto{\pgfqpoint{3.083988in}{1.305337in}}%
\pgfpathlineto{\pgfqpoint{3.086798in}{1.305106in}}%
\pgfpathlineto{\pgfqpoint{3.089609in}{1.305376in}}%
\pgfpathlineto{\pgfqpoint{3.092420in}{1.305581in}}%
\pgfpathlineto{\pgfqpoint{3.095231in}{1.305724in}}%
\pgfpathlineto{\pgfqpoint{3.098041in}{1.305596in}}%
\pgfpathlineto{\pgfqpoint{3.100852in}{1.305049in}}%
\pgfpathlineto{\pgfqpoint{3.103663in}{1.305026in}}%
\pgfpathlineto{\pgfqpoint{3.106473in}{1.305158in}}%
\pgfpathlineto{\pgfqpoint{3.109284in}{1.305457in}}%
\pgfpathlineto{\pgfqpoint{3.112095in}{1.305633in}}%
\pgfpathlineto{\pgfqpoint{3.114905in}{1.305717in}}%
\pgfpathlineto{\pgfqpoint{3.117716in}{1.306026in}}%
\pgfpathlineto{\pgfqpoint{3.120527in}{1.306042in}}%
\pgfpathlineto{\pgfqpoint{3.123337in}{1.306271in}}%
\pgfpathlineto{\pgfqpoint{3.126148in}{1.306541in}}%
\pgfpathlineto{\pgfqpoint{3.128959in}{1.306770in}}%
\pgfpathlineto{\pgfqpoint{3.131769in}{1.306945in}}%
\pgfpathlineto{\pgfqpoint{3.134580in}{1.307214in}}%
\pgfpathlineto{\pgfqpoint{3.137391in}{1.307495in}}%
\pgfpathlineto{\pgfqpoint{3.140202in}{1.307393in}}%
\pgfpathlineto{\pgfqpoint{3.143012in}{1.307281in}}%
\pgfpathlineto{\pgfqpoint{3.145823in}{1.307085in}}%
\pgfpathlineto{\pgfqpoint{3.148634in}{1.307129in}}%
\pgfpathlineto{\pgfqpoint{3.151444in}{1.307384in}}%
\pgfpathlineto{\pgfqpoint{3.154255in}{1.306744in}}%
\pgfpathlineto{\pgfqpoint{3.157066in}{1.307065in}}%
\pgfpathlineto{\pgfqpoint{3.159876in}{1.305966in}}%
\pgfpathlineto{\pgfqpoint{3.162687in}{1.306148in}}%
\pgfpathlineto{\pgfqpoint{3.165498in}{1.306323in}}%
\pgfpathlineto{\pgfqpoint{3.168308in}{1.305737in}}%
\pgfpathlineto{\pgfqpoint{3.171119in}{1.305988in}}%
\pgfpathlineto{\pgfqpoint{3.173930in}{1.305402in}}%
\pgfpathlineto{\pgfqpoint{3.176740in}{1.305699in}}%
\pgfpathlineto{\pgfqpoint{3.179551in}{1.305885in}}%
\pgfpathlineto{\pgfqpoint{3.182362in}{1.306201in}}%
\pgfpathlineto{\pgfqpoint{3.185173in}{1.306277in}}%
\pgfpathlineto{\pgfqpoint{3.187983in}{1.306507in}}%
\pgfpathlineto{\pgfqpoint{3.190794in}{1.305646in}}%
\pgfpathlineto{\pgfqpoint{3.193605in}{1.305816in}}%
\pgfpathlineto{\pgfqpoint{3.196415in}{1.306021in}}%
\pgfpathlineto{\pgfqpoint{3.199226in}{1.306279in}}%
\pgfpathlineto{\pgfqpoint{3.202037in}{1.306395in}}%
\pgfpathlineto{\pgfqpoint{3.204847in}{1.306521in}}%
\pgfpathlineto{\pgfqpoint{3.207658in}{1.306778in}}%
\pgfpathlineto{\pgfqpoint{3.210469in}{1.306964in}}%
\pgfpathlineto{\pgfqpoint{3.213279in}{1.307275in}}%
\pgfpathlineto{\pgfqpoint{3.216090in}{1.307576in}}%
\pgfpathlineto{\pgfqpoint{3.218901in}{1.307774in}}%
\pgfpathlineto{\pgfqpoint{3.221711in}{1.307955in}}%
\pgfpathlineto{\pgfqpoint{3.224522in}{1.307757in}}%
\pgfpathlineto{\pgfqpoint{3.227333in}{1.307946in}}%
\pgfpathlineto{\pgfqpoint{3.230143in}{1.308208in}}%
\pgfpathlineto{\pgfqpoint{3.232954in}{1.308416in}}%
\pgfpathlineto{\pgfqpoint{3.235765in}{1.307825in}}%
\pgfpathlineto{\pgfqpoint{3.238576in}{1.308099in}}%
\pgfpathlineto{\pgfqpoint{3.241386in}{1.308401in}}%
\pgfpathlineto{\pgfqpoint{3.244197in}{1.307278in}}%
\pgfpathlineto{\pgfqpoint{3.247008in}{1.307374in}}%
\pgfpathlineto{\pgfqpoint{3.249818in}{1.307491in}}%
\pgfpathlineto{\pgfqpoint{3.252629in}{1.307542in}}%
\pgfpathlineto{\pgfqpoint{3.255440in}{1.307614in}}%
\pgfpathlineto{\pgfqpoint{3.258250in}{1.307919in}}%
\pgfpathlineto{\pgfqpoint{3.261061in}{1.307185in}}%
\pgfpathlineto{\pgfqpoint{3.263872in}{1.307283in}}%
\pgfpathlineto{\pgfqpoint{3.266682in}{1.307425in}}%
\pgfpathlineto{\pgfqpoint{3.269493in}{1.307722in}}%
\pgfpathlineto{\pgfqpoint{3.272304in}{1.307814in}}%
\pgfpathlineto{\pgfqpoint{3.275114in}{1.308070in}}%
\pgfpathlineto{\pgfqpoint{3.277925in}{1.306732in}}%
\pgfpathlineto{\pgfqpoint{3.280736in}{1.306743in}}%
\pgfpathlineto{\pgfqpoint{3.283547in}{1.306825in}}%
\pgfpathlineto{\pgfqpoint{3.286357in}{1.307061in}}%
\pgfpathlineto{\pgfqpoint{3.289168in}{1.307184in}}%
\pgfpathlineto{\pgfqpoint{3.291979in}{1.307208in}}%
\pgfpathlineto{\pgfqpoint{3.294789in}{1.307504in}}%
\pgfpathlineto{\pgfqpoint{3.297600in}{1.307761in}}%
\pgfpathlineto{\pgfqpoint{3.300411in}{1.307813in}}%
\pgfpathlineto{\pgfqpoint{3.303221in}{1.307847in}}%
\pgfpathlineto{\pgfqpoint{3.306032in}{1.308119in}}%
\pgfpathlineto{\pgfqpoint{3.308843in}{1.307664in}}%
\pgfpathlineto{\pgfqpoint{3.311653in}{1.307918in}}%
\pgfpathlineto{\pgfqpoint{3.314464in}{1.308147in}}%
\pgfpathlineto{\pgfqpoint{3.317275in}{1.307864in}}%
\pgfpathlineto{\pgfqpoint{3.320085in}{1.308158in}}%
\pgfpathlineto{\pgfqpoint{3.322896in}{1.308344in}}%
\pgfpathlineto{\pgfqpoint{3.325707in}{1.308522in}}%
\pgfpathlineto{\pgfqpoint{3.328518in}{1.308347in}}%
\pgfpathlineto{\pgfqpoint{3.331328in}{1.308599in}}%
\pgfpathlineto{\pgfqpoint{3.334139in}{1.308450in}}%
\pgfpathlineto{\pgfqpoint{3.336950in}{1.308723in}}%
\pgfpathlineto{\pgfqpoint{3.339760in}{1.308652in}}%
\pgfpathlineto{\pgfqpoint{3.342571in}{1.308864in}}%
\pgfpathlineto{\pgfqpoint{3.345382in}{1.308950in}}%
\pgfpathlineto{\pgfqpoint{3.348192in}{1.308763in}}%
\pgfpathlineto{\pgfqpoint{3.351003in}{1.309052in}}%
\pgfpathlineto{\pgfqpoint{3.353814in}{1.309161in}}%
\pgfpathlineto{\pgfqpoint{3.356624in}{1.309441in}}%
\pgfpathlineto{\pgfqpoint{3.359435in}{1.309211in}}%
\pgfpathlineto{\pgfqpoint{3.362246in}{1.309496in}}%
\pgfpathlineto{\pgfqpoint{3.365056in}{1.309773in}}%
\pgfpathlineto{\pgfqpoint{3.367867in}{1.309721in}}%
\pgfpathlineto{\pgfqpoint{3.370678in}{1.309833in}}%
\pgfpathlineto{\pgfqpoint{3.373489in}{1.310075in}}%
\pgfpathlineto{\pgfqpoint{3.376299in}{1.308450in}}%
\pgfpathlineto{\pgfqpoint{3.379110in}{1.308736in}}%
\pgfpathlineto{\pgfqpoint{3.381921in}{1.309017in}}%
\pgfpathlineto{\pgfqpoint{3.384731in}{1.309299in}}%
\pgfpathlineto{\pgfqpoint{3.387542in}{1.309461in}}%
\pgfpathlineto{\pgfqpoint{3.390353in}{1.309522in}}%
\pgfpathlineto{\pgfqpoint{3.393163in}{1.309085in}}%
\pgfpathlineto{\pgfqpoint{3.395974in}{1.309359in}}%
\pgfpathlineto{\pgfqpoint{3.398785in}{1.309444in}}%
\pgfpathlineto{\pgfqpoint{3.401595in}{1.309644in}}%
\pgfpathlineto{\pgfqpoint{3.404406in}{1.309930in}}%
\pgfpathlineto{\pgfqpoint{3.407217in}{1.310106in}}%
\pgfpathlineto{\pgfqpoint{3.410027in}{1.310391in}}%
\pgfpathlineto{\pgfqpoint{3.412838in}{1.310659in}}%
\pgfpathlineto{\pgfqpoint{3.415649in}{1.310542in}}%
\pgfpathlineto{\pgfqpoint{3.418459in}{1.310431in}}%
\pgfpathlineto{\pgfqpoint{3.421270in}{1.310616in}}%
\pgfpathlineto{\pgfqpoint{3.424081in}{1.310742in}}%
\pgfpathlineto{\pgfqpoint{3.426892in}{1.309558in}}%
\pgfpathlineto{\pgfqpoint{3.429702in}{1.309288in}}%
\pgfpathlineto{\pgfqpoint{3.432513in}{1.309553in}}%
\pgfpathlineto{\pgfqpoint{3.435324in}{1.309735in}}%
\pgfpathlineto{\pgfqpoint{3.438134in}{1.310005in}}%
\pgfpathlineto{\pgfqpoint{3.440945in}{1.309683in}}%
\pgfpathlineto{\pgfqpoint{3.443756in}{1.309957in}}%
\pgfpathlineto{\pgfqpoint{3.446566in}{1.309964in}}%
\pgfpathlineto{\pgfqpoint{3.449377in}{1.310064in}}%
\pgfpathlineto{\pgfqpoint{3.452188in}{1.309413in}}%
\pgfpathlineto{\pgfqpoint{3.454998in}{1.309694in}}%
\pgfpathlineto{\pgfqpoint{3.457809in}{1.309933in}}%
\pgfpathlineto{\pgfqpoint{3.460620in}{1.309602in}}%
\pgfpathlineto{\pgfqpoint{3.463430in}{1.309862in}}%
\pgfpathlineto{\pgfqpoint{3.466241in}{1.310132in}}%
\pgfpathlineto{\pgfqpoint{3.469052in}{1.310389in}}%
\pgfpathlineto{\pgfqpoint{3.471863in}{1.310629in}}%
\pgfpathlineto{\pgfqpoint{3.474673in}{1.310841in}}%
\pgfpathlineto{\pgfqpoint{3.477484in}{1.311016in}}%
\pgfpathlineto{\pgfqpoint{3.480295in}{1.311045in}}%
\pgfpathlineto{\pgfqpoint{3.483105in}{1.308624in}}%
\pgfpathlineto{\pgfqpoint{3.485916in}{1.305915in}}%
\pgfpathlineto{\pgfqpoint{3.488727in}{1.305169in}}%
\pgfpathlineto{\pgfqpoint{3.491537in}{1.302783in}}%
\pgfpathlineto{\pgfqpoint{3.494348in}{1.302749in}}%
\pgfpathlineto{\pgfqpoint{3.497159in}{1.302965in}}%
\pgfpathlineto{\pgfqpoint{3.499969in}{1.302513in}}%
\pgfpathlineto{\pgfqpoint{3.502780in}{1.300666in}}%
\pgfpathlineto{\pgfqpoint{3.505591in}{1.300943in}}%
\pgfpathlineto{\pgfqpoint{3.508401in}{1.301211in}}%
\pgfpathlineto{\pgfqpoint{3.511212in}{1.300671in}}%
\pgfpathlineto{\pgfqpoint{3.514023in}{1.300785in}}%
\pgfpathlineto{\pgfqpoint{3.516834in}{1.300499in}}%
\pgfpathlineto{\pgfqpoint{3.519644in}{1.300776in}}%
\pgfpathlineto{\pgfqpoint{3.522455in}{1.301054in}}%
\pgfpathlineto{\pgfqpoint{3.525266in}{1.300901in}}%
\pgfpathlineto{\pgfqpoint{3.528076in}{1.301178in}}%
\pgfpathlineto{\pgfqpoint{3.530887in}{1.301453in}}%
\pgfpathlineto{\pgfqpoint{3.533698in}{1.301642in}}%
\pgfpathlineto{\pgfqpoint{3.536508in}{1.301179in}}%
\pgfpathlineto{\pgfqpoint{3.539319in}{1.301410in}}%
\pgfpathlineto{\pgfqpoint{3.542130in}{1.301388in}}%
\pgfpathlineto{\pgfqpoint{3.544940in}{1.301659in}}%
\pgfpathlineto{\pgfqpoint{3.547751in}{1.301535in}}%
\pgfpathlineto{\pgfqpoint{3.550562in}{1.301807in}}%
\pgfpathlineto{\pgfqpoint{3.553372in}{1.297900in}}%
\pgfpathlineto{\pgfqpoint{3.556183in}{1.298010in}}%
\pgfpathlineto{\pgfqpoint{3.558994in}{1.298204in}}%
\pgfpathlineto{\pgfqpoint{3.561805in}{1.298467in}}%
\pgfpathlineto{\pgfqpoint{3.564615in}{1.298739in}}%
\pgfpathlineto{\pgfqpoint{3.567426in}{1.298783in}}%
\pgfpathlineto{\pgfqpoint{3.570237in}{1.298932in}}%
\pgfpathlineto{\pgfqpoint{3.573047in}{1.299136in}}%
\pgfpathlineto{\pgfqpoint{3.575858in}{1.299409in}}%
\pgfpathlineto{\pgfqpoint{3.578669in}{1.299624in}}%
\pgfpathlineto{\pgfqpoint{3.581479in}{1.299862in}}%
\pgfpathlineto{\pgfqpoint{3.584290in}{1.300059in}}%
\pgfpathlineto{\pgfqpoint{3.587101in}{1.299906in}}%
\pgfpathlineto{\pgfqpoint{3.589911in}{1.300116in}}%
\pgfpathlineto{\pgfqpoint{3.592722in}{1.300387in}}%
\pgfpathlineto{\pgfqpoint{3.595533in}{1.300631in}}%
\pgfpathlineto{\pgfqpoint{3.598343in}{1.300542in}}%
\pgfpathlineto{\pgfqpoint{3.601154in}{1.300385in}}%
\pgfpathlineto{\pgfqpoint{3.603965in}{1.300631in}}%
\pgfpathlineto{\pgfqpoint{3.606776in}{1.300902in}}%
\pgfpathlineto{\pgfqpoint{3.609586in}{1.301125in}}%
\pgfpathlineto{\pgfqpoint{3.612397in}{1.301069in}}%
\pgfpathlineto{\pgfqpoint{3.615208in}{1.301239in}}%
\pgfpathlineto{\pgfqpoint{3.618018in}{1.301317in}}%
\pgfpathlineto{\pgfqpoint{3.620829in}{1.301112in}}%
\pgfpathlineto{\pgfqpoint{3.623640in}{1.299587in}}%
\pgfpathlineto{\pgfqpoint{3.626450in}{1.299005in}}%
\pgfpathlineto{\pgfqpoint{3.629261in}{1.299268in}}%
\pgfpathlineto{\pgfqpoint{3.632072in}{1.299524in}}%
\pgfpathlineto{\pgfqpoint{3.634882in}{1.299540in}}%
\pgfpathlineto{\pgfqpoint{3.637693in}{1.299467in}}%
\pgfpathlineto{\pgfqpoint{3.640504in}{1.299570in}}%
\pgfpathlineto{\pgfqpoint{3.643314in}{1.299672in}}%
\pgfpathlineto{\pgfqpoint{3.646125in}{1.299507in}}%
\pgfpathlineto{\pgfqpoint{3.648936in}{1.299642in}}%
\pgfpathlineto{\pgfqpoint{3.651746in}{1.299904in}}%
\pgfpathlineto{\pgfqpoint{3.654557in}{1.300075in}}%
\pgfpathlineto{\pgfqpoint{3.657368in}{1.300166in}}%
\pgfpathlineto{\pgfqpoint{3.660179in}{1.300264in}}%
\pgfpathlineto{\pgfqpoint{3.662989in}{1.300462in}}%
\pgfpathlineto{\pgfqpoint{3.665800in}{1.300629in}}%
\pgfpathlineto{\pgfqpoint{3.668611in}{1.300723in}}%
\pgfpathlineto{\pgfqpoint{3.671421in}{1.300848in}}%
\pgfpathlineto{\pgfqpoint{3.674232in}{1.301087in}}%
\pgfpathlineto{\pgfqpoint{3.677043in}{1.300950in}}%
\pgfpathlineto{\pgfqpoint{3.679853in}{1.301200in}}%
\pgfpathlineto{\pgfqpoint{3.682664in}{1.301041in}}%
\pgfpathlineto{\pgfqpoint{3.685475in}{1.300943in}}%
\pgfpathlineto{\pgfqpoint{3.688285in}{1.300806in}}%
\pgfpathlineto{\pgfqpoint{3.691096in}{1.300661in}}%
\pgfpathlineto{\pgfqpoint{3.693907in}{1.300691in}}%
\pgfpathlineto{\pgfqpoint{3.696717in}{1.300336in}}%
\pgfpathlineto{\pgfqpoint{3.699528in}{1.300568in}}%
\pgfpathlineto{\pgfqpoint{3.702339in}{1.299496in}}%
\pgfpathlineto{\pgfqpoint{3.705150in}{1.299753in}}%
\pgfpathlineto{\pgfqpoint{3.707960in}{1.299772in}}%
\pgfpathlineto{\pgfqpoint{3.710771in}{1.299969in}}%
\pgfpathlineto{\pgfqpoint{3.713582in}{1.299631in}}%
\pgfpathlineto{\pgfqpoint{3.716392in}{1.298194in}}%
\pgfpathlineto{\pgfqpoint{3.719203in}{1.298444in}}%
\pgfpathlineto{\pgfqpoint{3.722014in}{1.298670in}}%
\pgfpathlineto{\pgfqpoint{3.724824in}{1.298873in}}%
\pgfpathlineto{\pgfqpoint{3.727635in}{1.298972in}}%
\pgfpathlineto{\pgfqpoint{3.730446in}{1.299207in}}%
\pgfpathlineto{\pgfqpoint{3.733256in}{1.299465in}}%
\pgfpathlineto{\pgfqpoint{3.736067in}{1.299313in}}%
\pgfpathlineto{\pgfqpoint{3.738878in}{1.299195in}}%
\pgfpathlineto{\pgfqpoint{3.741688in}{1.298280in}}%
\pgfpathlineto{\pgfqpoint{3.744499in}{1.298538in}}%
\pgfpathlineto{\pgfqpoint{3.747310in}{1.298288in}}%
\pgfpathlineto{\pgfqpoint{3.750121in}{1.297669in}}%
\pgfpathlineto{\pgfqpoint{3.752931in}{1.297454in}}%
\pgfpathlineto{\pgfqpoint{3.755742in}{1.297667in}}%
\pgfpathlineto{\pgfqpoint{3.758553in}{1.297913in}}%
\pgfpathlineto{\pgfqpoint{3.761363in}{1.297133in}}%
\pgfpathlineto{\pgfqpoint{3.764174in}{1.297387in}}%
\pgfpathlineto{\pgfqpoint{3.766985in}{1.296264in}}%
\pgfpathlineto{\pgfqpoint{3.769795in}{1.296301in}}%
\pgfpathlineto{\pgfqpoint{3.772606in}{1.296197in}}%
\pgfpathlineto{\pgfqpoint{3.775417in}{1.296432in}}%
\pgfpathlineto{\pgfqpoint{3.778227in}{1.296619in}}%
\pgfpathlineto{\pgfqpoint{3.781038in}{1.296373in}}%
\pgfpathlineto{\pgfqpoint{3.783849in}{1.296586in}}%
\pgfpathlineto{\pgfqpoint{3.786659in}{1.296326in}}%
\pgfpathlineto{\pgfqpoint{3.789470in}{1.295546in}}%
\pgfpathlineto{\pgfqpoint{3.792281in}{1.291331in}}%
\pgfpathlineto{\pgfqpoint{3.795092in}{1.291487in}}%
\pgfpathlineto{\pgfqpoint{3.797902in}{1.291247in}}%
\pgfpathlineto{\pgfqpoint{3.800713in}{1.291476in}}%
\pgfpathlineto{\pgfqpoint{3.803524in}{1.291397in}}%
\pgfpathlineto{\pgfqpoint{3.806334in}{1.290121in}}%
\pgfpathlineto{\pgfqpoint{3.809145in}{1.286206in}}%
\pgfpathlineto{\pgfqpoint{3.811956in}{1.286464in}}%
\pgfpathlineto{\pgfqpoint{3.814766in}{1.286283in}}%
\pgfpathlineto{\pgfqpoint{3.817577in}{1.285415in}}%
\pgfpathlineto{\pgfqpoint{3.820388in}{1.285248in}}%
\pgfpathlineto{\pgfqpoint{3.823198in}{1.285503in}}%
\pgfpathlineto{\pgfqpoint{3.826009in}{1.285712in}}%
\pgfpathlineto{\pgfqpoint{3.828820in}{1.285578in}}%
\pgfpathlineto{\pgfqpoint{3.831630in}{1.285817in}}%
\pgfpathlineto{\pgfqpoint{3.834441in}{1.285891in}}%
\pgfpathlineto{\pgfqpoint{3.837252in}{1.285692in}}%
\pgfpathlineto{\pgfqpoint{3.840062in}{1.285784in}}%
\pgfpathlineto{\pgfqpoint{3.842873in}{1.285901in}}%
\pgfpathlineto{\pgfqpoint{3.845684in}{1.285899in}}%
\pgfpathlineto{\pgfqpoint{3.848495in}{1.285980in}}%
\pgfpathlineto{\pgfqpoint{3.851305in}{1.285701in}}%
\pgfpathlineto{\pgfqpoint{3.854116in}{1.285786in}}%
\pgfpathlineto{\pgfqpoint{3.856927in}{1.285850in}}%
\pgfpathlineto{\pgfqpoint{3.859737in}{1.286032in}}%
\pgfpathlineto{\pgfqpoint{3.862548in}{1.284987in}}%
\pgfpathlineto{\pgfqpoint{3.865359in}{1.284493in}}%
\pgfpathlineto{\pgfqpoint{3.868169in}{1.284692in}}%
\pgfpathlineto{\pgfqpoint{3.870980in}{1.284720in}}%
\pgfpathlineto{\pgfqpoint{3.873791in}{1.284828in}}%
\pgfpathlineto{\pgfqpoint{3.876601in}{1.284973in}}%
\pgfpathlineto{\pgfqpoint{3.879412in}{1.285221in}}%
\pgfpathlineto{\pgfqpoint{3.882223in}{1.285471in}}%
\pgfpathlineto{\pgfqpoint{3.885033in}{1.285717in}}%
\pgfpathlineto{\pgfqpoint{3.887844in}{1.285969in}}%
\pgfpathlineto{\pgfqpoint{3.890655in}{1.286149in}}%
\pgfpathlineto{\pgfqpoint{3.893466in}{1.286017in}}%
\pgfpathlineto{\pgfqpoint{3.896276in}{1.286244in}}%
\pgfpathlineto{\pgfqpoint{3.899087in}{1.286467in}}%
\pgfpathlineto{\pgfqpoint{3.901898in}{1.286634in}}%
\pgfpathlineto{\pgfqpoint{3.904708in}{1.286795in}}%
\pgfpathlineto{\pgfqpoint{3.907519in}{1.286936in}}%
\pgfpathlineto{\pgfqpoint{3.910330in}{1.287031in}}%
\pgfpathlineto{\pgfqpoint{3.913140in}{1.287240in}}%
\pgfpathlineto{\pgfqpoint{3.915951in}{1.287462in}}%
\pgfpathlineto{\pgfqpoint{3.918762in}{1.287325in}}%
\pgfpathlineto{\pgfqpoint{3.921572in}{1.287511in}}%
\pgfpathlineto{\pgfqpoint{3.924383in}{1.287487in}}%
\pgfpathlineto{\pgfqpoint{3.927194in}{1.287730in}}%
\pgfpathlineto{\pgfqpoint{3.930004in}{1.287854in}}%
\pgfpathlineto{\pgfqpoint{3.932815in}{1.288101in}}%
\pgfpathlineto{\pgfqpoint{3.935626in}{1.288300in}}%
\pgfpathlineto{\pgfqpoint{3.938437in}{1.288547in}}%
\pgfpathlineto{\pgfqpoint{3.941247in}{1.288663in}}%
\pgfpathlineto{\pgfqpoint{3.944058in}{1.288830in}}%
\pgfpathlineto{\pgfqpoint{3.946869in}{1.288782in}}%
\pgfpathlineto{\pgfqpoint{3.949679in}{1.289017in}}%
\pgfpathlineto{\pgfqpoint{3.952490in}{1.289099in}}%
\pgfpathlineto{\pgfqpoint{3.955301in}{1.288469in}}%
\pgfpathlineto{\pgfqpoint{3.958111in}{1.288287in}}%
\pgfpathlineto{\pgfqpoint{3.960922in}{1.288522in}}%
\pgfpathlineto{\pgfqpoint{3.963733in}{1.288738in}}%
\pgfpathlineto{\pgfqpoint{3.966543in}{1.288491in}}%
\pgfpathlineto{\pgfqpoint{3.969354in}{1.288540in}}%
\pgfpathlineto{\pgfqpoint{3.972165in}{1.288726in}}%
\pgfpathlineto{\pgfqpoint{3.974975in}{1.288359in}}%
\pgfpathlineto{\pgfqpoint{3.977786in}{1.288506in}}%
\pgfpathlineto{\pgfqpoint{3.980597in}{1.288729in}}%
\pgfpathlineto{\pgfqpoint{3.983408in}{1.288960in}}%
\pgfpathlineto{\pgfqpoint{3.986218in}{1.289200in}}%
\pgfpathlineto{\pgfqpoint{3.989029in}{1.289422in}}%
\pgfpathlineto{\pgfqpoint{3.991840in}{1.288711in}}%
\pgfpathlineto{\pgfqpoint{3.994650in}{1.288952in}}%
\pgfpathlineto{\pgfqpoint{3.997461in}{1.288709in}}%
\pgfpathlineto{\pgfqpoint{4.000272in}{1.288930in}}%
\pgfpathlineto{\pgfqpoint{4.003082in}{1.288716in}}%
\pgfpathlineto{\pgfqpoint{4.005893in}{1.288957in}}%
\pgfpathlineto{\pgfqpoint{4.008704in}{1.288962in}}%
\pgfpathlineto{\pgfqpoint{4.011514in}{1.289198in}}%
\pgfpathlineto{\pgfqpoint{4.014325in}{1.289238in}}%
\pgfpathlineto{\pgfqpoint{4.017136in}{1.289092in}}%
\pgfpathlineto{\pgfqpoint{4.019946in}{1.289183in}}%
\pgfpathlineto{\pgfqpoint{4.022757in}{1.289343in}}%
\pgfpathlineto{\pgfqpoint{4.025568in}{1.289582in}}%
\pgfpathlineto{\pgfqpoint{4.028378in}{1.289521in}}%
\pgfpathlineto{\pgfqpoint{4.031189in}{1.289739in}}%
\pgfpathlineto{\pgfqpoint{4.034000in}{1.289976in}}%
\pgfpathlineto{\pgfqpoint{4.036811in}{1.290146in}}%
\pgfpathlineto{\pgfqpoint{4.039621in}{1.290383in}}%
\pgfpathlineto{\pgfqpoint{4.042432in}{1.290567in}}%
\pgfpathlineto{\pgfqpoint{4.045243in}{1.290781in}}%
\pgfpathlineto{\pgfqpoint{4.048053in}{1.290852in}}%
\pgfpathlineto{\pgfqpoint{4.050864in}{1.290629in}}%
\pgfpathlineto{\pgfqpoint{4.053675in}{1.289925in}}%
\pgfpathlineto{\pgfqpoint{4.056485in}{1.290127in}}%
\pgfpathlineto{\pgfqpoint{4.059296in}{1.290190in}}%
\pgfpathlineto{\pgfqpoint{4.062107in}{1.290393in}}%
\pgfpathlineto{\pgfqpoint{4.064917in}{1.289976in}}%
\pgfpathlineto{\pgfqpoint{4.067728in}{1.290199in}}%
\pgfpathlineto{\pgfqpoint{4.070539in}{1.290367in}}%
\pgfpathlineto{\pgfqpoint{4.073349in}{1.290264in}}%
\pgfpathlineto{\pgfqpoint{4.076160in}{1.290307in}}%
\pgfpathlineto{\pgfqpoint{4.078971in}{1.288108in}}%
\pgfpathlineto{\pgfqpoint{4.081782in}{1.287390in}}%
\pgfpathlineto{\pgfqpoint{4.084592in}{1.287359in}}%
\pgfpathlineto{\pgfqpoint{4.087403in}{1.287436in}}%
\pgfpathlineto{\pgfqpoint{4.090214in}{1.285931in}}%
\pgfpathlineto{\pgfqpoint{4.093024in}{1.286151in}}%
\pgfpathlineto{\pgfqpoint{4.095835in}{1.286357in}}%
\pgfpathlineto{\pgfqpoint{4.098646in}{1.286331in}}%
\pgfpathlineto{\pgfqpoint{4.101456in}{1.286561in}}%
\pgfpathlineto{\pgfqpoint{4.104267in}{1.286504in}}%
\pgfpathlineto{\pgfqpoint{4.107078in}{1.286694in}}%
\pgfpathlineto{\pgfqpoint{4.109888in}{1.286910in}}%
\pgfpathlineto{\pgfqpoint{4.112699in}{1.287077in}}%
\pgfpathlineto{\pgfqpoint{4.115510in}{1.287310in}}%
\pgfpathlineto{\pgfqpoint{4.118320in}{1.287508in}}%
\pgfpathlineto{\pgfqpoint{4.121131in}{1.287679in}}%
\pgfpathlineto{\pgfqpoint{4.123942in}{1.287903in}}%
\pgfpathlineto{\pgfqpoint{4.126753in}{1.288136in}}%
\pgfpathlineto{\pgfqpoint{4.129563in}{1.288143in}}%
\pgfpathlineto{\pgfqpoint{4.132374in}{1.288340in}}%
\pgfpathlineto{\pgfqpoint{4.135185in}{1.288170in}}%
\pgfpathlineto{\pgfqpoint{4.137995in}{1.288216in}}%
\pgfpathlineto{\pgfqpoint{4.140806in}{1.288392in}}%
\pgfpathlineto{\pgfqpoint{4.143617in}{1.288622in}}%
\pgfpathlineto{\pgfqpoint{4.146427in}{1.288356in}}%
\pgfpathlineto{\pgfqpoint{4.149238in}{1.288559in}}%
\pgfpathlineto{\pgfqpoint{4.152049in}{1.288730in}}%
\pgfpathlineto{\pgfqpoint{4.154859in}{1.288953in}}%
\pgfpathlineto{\pgfqpoint{4.157670in}{1.289179in}}%
\pgfpathlineto{\pgfqpoint{4.160481in}{1.289408in}}%
\pgfpathlineto{\pgfqpoint{4.163291in}{1.289530in}}%
\pgfpathlineto{\pgfqpoint{4.166102in}{1.289716in}}%
\pgfpathlineto{\pgfqpoint{4.168913in}{1.289786in}}%
\pgfpathlineto{\pgfqpoint{4.171724in}{1.290008in}}%
\pgfpathlineto{\pgfqpoint{4.174534in}{1.290151in}}%
\pgfpathlineto{\pgfqpoint{4.177345in}{1.290370in}}%
\pgfpathlineto{\pgfqpoint{4.180156in}{1.290513in}}%
\pgfpathlineto{\pgfqpoint{4.182966in}{1.290706in}}%
\pgfpathlineto{\pgfqpoint{4.185777in}{1.290712in}}%
\pgfpathlineto{\pgfqpoint{4.188588in}{1.290909in}}%
\pgfpathlineto{\pgfqpoint{4.191398in}{1.291134in}}%
\pgfpathlineto{\pgfqpoint{4.194209in}{1.291240in}}%
\pgfpathlineto{\pgfqpoint{4.197020in}{1.291267in}}%
\pgfpathlineto{\pgfqpoint{4.199830in}{1.291436in}}%
\pgfpathlineto{\pgfqpoint{4.202641in}{1.291637in}}%
\pgfpathlineto{\pgfqpoint{4.205452in}{1.291844in}}%
\pgfpathlineto{\pgfqpoint{4.208262in}{1.292050in}}%
\pgfpathlineto{\pgfqpoint{4.211073in}{1.292152in}}%
\pgfpathlineto{\pgfqpoint{4.213884in}{1.292366in}}%
\pgfpathlineto{\pgfqpoint{4.216695in}{1.292589in}}%
\pgfpathlineto{\pgfqpoint{4.219505in}{1.292810in}}%
\pgfpathlineto{\pgfqpoint{4.222316in}{1.293032in}}%
\pgfpathlineto{\pgfqpoint{4.225127in}{1.293132in}}%
\pgfpathlineto{\pgfqpoint{4.227937in}{1.292612in}}%
\pgfpathlineto{\pgfqpoint{4.230748in}{1.292700in}}%
\pgfpathlineto{\pgfqpoint{4.233559in}{1.292501in}}%
\pgfpathlineto{\pgfqpoint{4.236369in}{1.292562in}}%
\pgfpathlineto{\pgfqpoint{4.239180in}{1.292782in}}%
\pgfpathlineto{\pgfqpoint{4.241991in}{1.292953in}}%
\pgfpathlineto{\pgfqpoint{4.244801in}{1.293167in}}%
\pgfpathlineto{\pgfqpoint{4.247612in}{1.293321in}}%
\pgfpathlineto{\pgfqpoint{4.250423in}{1.293541in}}%
\pgfpathlineto{\pgfqpoint{4.253233in}{1.293724in}}%
\pgfpathlineto{\pgfqpoint{4.256044in}{1.293638in}}%
\pgfpathlineto{\pgfqpoint{4.258855in}{1.293590in}}%
\pgfpathlineto{\pgfqpoint{4.261665in}{1.293796in}}%
\pgfpathlineto{\pgfqpoint{4.264476in}{1.294013in}}%
\pgfpathlineto{\pgfqpoint{4.267287in}{1.293830in}}%
\pgfpathlineto{\pgfqpoint{4.270098in}{1.294047in}}%
\pgfpathlineto{\pgfqpoint{4.272908in}{1.294231in}}%
\pgfpathlineto{\pgfqpoint{4.275719in}{1.294363in}}%
\pgfpathlineto{\pgfqpoint{4.278530in}{1.294581in}}%
\pgfpathlineto{\pgfqpoint{4.281340in}{1.294741in}}%
\pgfpathlineto{\pgfqpoint{4.284151in}{1.294780in}}%
\pgfpathlineto{\pgfqpoint{4.286962in}{1.294973in}}%
\pgfpathlineto{\pgfqpoint{4.289772in}{1.294772in}}%
\pgfpathlineto{\pgfqpoint{4.292583in}{1.294962in}}%
\pgfpathlineto{\pgfqpoint{4.295394in}{1.295030in}}%
\pgfpathlineto{\pgfqpoint{4.298204in}{1.295246in}}%
\pgfpathlineto{\pgfqpoint{4.301015in}{1.295334in}}%
\pgfpathlineto{\pgfqpoint{4.303826in}{1.295341in}}%
\pgfpathlineto{\pgfqpoint{4.306636in}{1.295509in}}%
\pgfpathlineto{\pgfqpoint{4.309447in}{1.295595in}}%
\pgfpathlineto{\pgfqpoint{4.312258in}{1.295722in}}%
\pgfpathlineto{\pgfqpoint{4.315069in}{1.295934in}}%
\pgfpathlineto{\pgfqpoint{4.317879in}{1.295704in}}%
\pgfpathlineto{\pgfqpoint{4.320690in}{1.295800in}}%
\pgfpathlineto{\pgfqpoint{4.323501in}{1.295983in}}%
\pgfpathlineto{\pgfqpoint{4.326311in}{1.296179in}}%
\pgfpathlineto{\pgfqpoint{4.329122in}{1.296374in}}%
\pgfpathlineto{\pgfqpoint{4.331933in}{1.296260in}}%
\pgfpathlineto{\pgfqpoint{4.334743in}{1.296072in}}%
\pgfpathlineto{\pgfqpoint{4.337554in}{1.296239in}}%
\pgfpathlineto{\pgfqpoint{4.340365in}{1.296330in}}%
\pgfpathlineto{\pgfqpoint{4.343175in}{1.296227in}}%
\pgfpathlineto{\pgfqpoint{4.345986in}{1.296429in}}%
\pgfpathlineto{\pgfqpoint{4.348797in}{1.296629in}}%
\pgfpathlineto{\pgfqpoint{4.351607in}{1.296285in}}%
\pgfpathlineto{\pgfqpoint{4.354418in}{1.296441in}}%
\pgfpathlineto{\pgfqpoint{4.357229in}{1.294188in}}%
\pgfpathlineto{\pgfqpoint{4.360040in}{1.294372in}}%
\pgfpathlineto{\pgfqpoint{4.362850in}{1.294421in}}%
\pgfpathlineto{\pgfqpoint{4.365661in}{1.294616in}}%
\pgfpathlineto{\pgfqpoint{4.368472in}{1.294718in}}%
\pgfpathlineto{\pgfqpoint{4.371282in}{1.294923in}}%
\pgfpathlineto{\pgfqpoint{4.374093in}{1.294322in}}%
\pgfpathlineto{\pgfqpoint{4.376904in}{1.294392in}}%
\pgfpathlineto{\pgfqpoint{4.379714in}{1.294603in}}%
\pgfpathlineto{\pgfqpoint{4.382525in}{1.294406in}}%
\pgfpathlineto{\pgfqpoint{4.385336in}{1.294572in}}%
\pgfpathlineto{\pgfqpoint{4.388146in}{1.293884in}}%
\pgfpathlineto{\pgfqpoint{4.390957in}{1.293118in}}%
\pgfpathlineto{\pgfqpoint{4.393768in}{1.293315in}}%
\pgfpathlineto{\pgfqpoint{4.396578in}{1.293374in}}%
\pgfpathlineto{\pgfqpoint{4.399389in}{1.293483in}}%
\pgfpathlineto{\pgfqpoint{4.402200in}{1.293320in}}%
\pgfpathlineto{\pgfqpoint{4.405011in}{1.293468in}}%
\pgfpathlineto{\pgfqpoint{4.407821in}{1.293595in}}%
\pgfpathlineto{\pgfqpoint{4.410632in}{1.293571in}}%
\pgfpathlineto{\pgfqpoint{4.413443in}{1.293780in}}%
\pgfpathlineto{\pgfqpoint{4.416253in}{1.293921in}}%
\pgfpathlineto{\pgfqpoint{4.419064in}{1.294120in}}%
\pgfpathlineto{\pgfqpoint{4.421875in}{1.293871in}}%
\pgfpathlineto{\pgfqpoint{4.424685in}{1.293967in}}%
\pgfpathlineto{\pgfqpoint{4.427496in}{1.294157in}}%
\pgfpathlineto{\pgfqpoint{4.430307in}{1.294261in}}%
\pgfpathlineto{\pgfqpoint{4.433117in}{1.294362in}}%
\pgfpathlineto{\pgfqpoint{4.435928in}{1.294550in}}%
\pgfpathlineto{\pgfqpoint{4.438739in}{1.294732in}}%
\pgfpathlineto{\pgfqpoint{4.441549in}{1.294874in}}%
\pgfpathlineto{\pgfqpoint{4.444360in}{1.295031in}}%
\pgfpathlineto{\pgfqpoint{4.447171in}{1.295112in}}%
\pgfpathlineto{\pgfqpoint{4.449981in}{1.295212in}}%
\pgfpathlineto{\pgfqpoint{4.452792in}{1.295418in}}%
\pgfpathlineto{\pgfqpoint{4.455603in}{1.295611in}}%
\pgfpathlineto{\pgfqpoint{4.458414in}{1.295787in}}%
\pgfpathlineto{\pgfqpoint{4.461224in}{1.295829in}}%
\pgfpathlineto{\pgfqpoint{4.464035in}{1.295877in}}%
\pgfpathlineto{\pgfqpoint{4.466846in}{1.296079in}}%
\pgfpathlineto{\pgfqpoint{4.469656in}{1.296130in}}%
\pgfpathlineto{\pgfqpoint{4.472467in}{1.296238in}}%
\pgfpathlineto{\pgfqpoint{4.475278in}{1.296402in}}%
\pgfpathlineto{\pgfqpoint{4.478088in}{1.296593in}}%
\pgfpathlineto{\pgfqpoint{4.480899in}{1.296764in}}%
\pgfpathlineto{\pgfqpoint{4.483710in}{1.296930in}}%
\pgfpathlineto{\pgfqpoint{4.486520in}{1.297118in}}%
\pgfpathlineto{\pgfqpoint{4.489331in}{1.297297in}}%
\pgfpathlineto{\pgfqpoint{4.492142in}{1.297501in}}%
\pgfpathlineto{\pgfqpoint{4.494952in}{1.297481in}}%
\pgfpathlineto{\pgfqpoint{4.497763in}{1.297682in}}%
\pgfpathlineto{\pgfqpoint{4.500574in}{1.297816in}}%
\pgfpathlineto{\pgfqpoint{4.503385in}{1.297671in}}%
\pgfpathlineto{\pgfqpoint{4.506195in}{1.297764in}}%
\pgfpathlineto{\pgfqpoint{4.509006in}{1.297885in}}%
\pgfpathlineto{\pgfqpoint{4.511817in}{1.296865in}}%
\pgfpathlineto{\pgfqpoint{4.514627in}{1.296963in}}%
\pgfpathlineto{\pgfqpoint{4.517438in}{1.297101in}}%
\pgfpathlineto{\pgfqpoint{4.520249in}{1.297080in}}%
\pgfpathlineto{\pgfqpoint{4.523059in}{1.297278in}}%
\pgfpathlineto{\pgfqpoint{4.525870in}{1.297463in}}%
\pgfpathlineto{\pgfqpoint{4.528681in}{1.297662in}}%
\pgfpathlineto{\pgfqpoint{4.531491in}{1.297855in}}%
\pgfpathlineto{\pgfqpoint{4.534302in}{1.298053in}}%
\pgfpathlineto{\pgfqpoint{4.537113in}{1.298175in}}%
\pgfpathlineto{\pgfqpoint{4.539923in}{1.298328in}}%
\pgfpathlineto{\pgfqpoint{4.542734in}{1.298522in}}%
\pgfpathlineto{\pgfqpoint{4.545545in}{1.298646in}}%
\pgfpathlineto{\pgfqpoint{4.548356in}{1.298835in}}%
\pgfpathlineto{\pgfqpoint{4.551166in}{1.299012in}}%
\pgfpathlineto{\pgfqpoint{4.553977in}{1.299068in}}%
\pgfpathlineto{\pgfqpoint{4.556788in}{1.299203in}}%
\pgfpathlineto{\pgfqpoint{4.559598in}{1.299387in}}%
\pgfpathlineto{\pgfqpoint{4.562409in}{1.299437in}}%
\pgfpathlineto{\pgfqpoint{4.565220in}{1.299614in}}%
\pgfpathlineto{\pgfqpoint{4.568030in}{1.299778in}}%
\pgfpathlineto{\pgfqpoint{4.570841in}{1.299938in}}%
\pgfpathlineto{\pgfqpoint{4.573652in}{1.300065in}}%
\pgfpathlineto{\pgfqpoint{4.576462in}{1.300229in}}%
\pgfpathlineto{\pgfqpoint{4.579273in}{1.300426in}}%
\pgfpathlineto{\pgfqpoint{4.582084in}{1.300611in}}%
\pgfpathlineto{\pgfqpoint{4.584894in}{1.300634in}}%
\pgfpathlineto{\pgfqpoint{4.587705in}{1.300819in}}%
\pgfpathlineto{\pgfqpoint{4.590516in}{1.300978in}}%
\pgfpathlineto{\pgfqpoint{4.593327in}{1.301146in}}%
\pgfpathlineto{\pgfqpoint{4.596137in}{1.301004in}}%
\pgfpathlineto{\pgfqpoint{4.598948in}{1.300833in}}%
\pgfpathlineto{\pgfqpoint{4.601759in}{1.301023in}}%
\pgfpathlineto{\pgfqpoint{4.604569in}{1.301204in}}%
\pgfpathlineto{\pgfqpoint{4.607380in}{1.301385in}}%
\pgfpathlineto{\pgfqpoint{4.610191in}{1.301482in}}%
\pgfpathlineto{\pgfqpoint{4.613001in}{1.301647in}}%
\pgfpathlineto{\pgfqpoint{4.615812in}{1.301797in}}%
\pgfpathlineto{\pgfqpoint{4.618623in}{1.301908in}}%
\pgfpathlineto{\pgfqpoint{4.621433in}{1.302025in}}%
\pgfpathlineto{\pgfqpoint{4.624244in}{1.302215in}}%
\pgfpathlineto{\pgfqpoint{4.627055in}{1.302230in}}%
\pgfpathlineto{\pgfqpoint{4.629865in}{1.302401in}}%
\pgfpathlineto{\pgfqpoint{4.632676in}{1.302550in}}%
\pgfpathlineto{\pgfqpoint{4.635487in}{1.302624in}}%
\pgfpathlineto{\pgfqpoint{4.638298in}{1.302780in}}%
\pgfpathlineto{\pgfqpoint{4.641108in}{1.302918in}}%
\pgfpathlineto{\pgfqpoint{4.643919in}{1.303030in}}%
\pgfpathlineto{\pgfqpoint{4.646730in}{1.303194in}}%
\pgfpathlineto{\pgfqpoint{4.649540in}{1.303379in}}%
\pgfpathlineto{\pgfqpoint{4.652351in}{1.303506in}}%
\pgfpathlineto{\pgfqpoint{4.655162in}{1.303648in}}%
\pgfpathlineto{\pgfqpoint{4.657972in}{1.303787in}}%
\pgfpathlineto{\pgfqpoint{4.660783in}{1.303926in}}%
\pgfpathlineto{\pgfqpoint{4.663594in}{1.304117in}}%
\pgfpathlineto{\pgfqpoint{4.666404in}{1.304286in}}%
\pgfpathlineto{\pgfqpoint{4.669215in}{1.304374in}}%
\pgfpathlineto{\pgfqpoint{4.672026in}{1.304474in}}%
\pgfpathlineto{\pgfqpoint{4.674836in}{1.304542in}}%
\pgfpathlineto{\pgfqpoint{4.677647in}{1.304685in}}%
\pgfpathlineto{\pgfqpoint{4.680458in}{1.304843in}}%
\pgfpathlineto{\pgfqpoint{4.683268in}{1.304966in}}%
\pgfpathlineto{\pgfqpoint{4.686079in}{1.305126in}}%
\pgfpathlineto{\pgfqpoint{4.688890in}{1.305158in}}%
\pgfpathlineto{\pgfqpoint{4.691701in}{1.305266in}}%
\pgfpathlineto{\pgfqpoint{4.694511in}{1.305414in}}%
\pgfpathlineto{\pgfqpoint{4.697322in}{1.305582in}}%
\pgfpathlineto{\pgfqpoint{4.700133in}{1.305694in}}%
\pgfpathlineto{\pgfqpoint{4.702943in}{1.305881in}}%
\pgfpathlineto{\pgfqpoint{4.705754in}{1.306064in}}%
\pgfpathlineto{\pgfqpoint{4.708565in}{1.306188in}}%
\pgfpathlineto{\pgfqpoint{4.711375in}{1.305998in}}%
\pgfpathlineto{\pgfqpoint{4.714186in}{1.306127in}}%
\pgfpathlineto{\pgfqpoint{4.716997in}{1.306313in}}%
\pgfpathlineto{\pgfqpoint{4.719807in}{1.306499in}}%
\pgfpathlineto{\pgfqpoint{4.722618in}{1.306684in}}%
\pgfpathlineto{\pgfqpoint{4.725429in}{1.306866in}}%
\pgfpathlineto{\pgfqpoint{4.728239in}{1.307028in}}%
\pgfpathlineto{\pgfqpoint{4.731050in}{1.307097in}}%
\pgfpathlineto{\pgfqpoint{4.733861in}{1.307239in}}%
\pgfpathlineto{\pgfqpoint{4.736672in}{1.307420in}}%
\pgfpathlineto{\pgfqpoint{4.739482in}{1.307575in}}%
\pgfpathlineto{\pgfqpoint{4.742293in}{1.307760in}}%
\pgfpathlineto{\pgfqpoint{4.745104in}{1.307935in}}%
\pgfpathlineto{\pgfqpoint{4.747914in}{1.307910in}}%
\pgfpathlineto{\pgfqpoint{4.750725in}{1.308076in}}%
\pgfpathlineto{\pgfqpoint{4.753536in}{1.308210in}}%
\pgfpathlineto{\pgfqpoint{4.756346in}{1.307915in}}%
\pgfpathlineto{\pgfqpoint{4.759157in}{1.307789in}}%
\pgfpathlineto{\pgfqpoint{4.761968in}{1.307903in}}%
\pgfpathlineto{\pgfqpoint{4.764778in}{1.308065in}}%
\pgfpathlineto{\pgfqpoint{4.767589in}{1.307910in}}%
\pgfpathlineto{\pgfqpoint{4.770400in}{1.308043in}}%
\pgfpathlineto{\pgfqpoint{4.773210in}{1.308226in}}%
\pgfpathlineto{\pgfqpoint{4.776021in}{1.308287in}}%
\pgfpathlineto{\pgfqpoint{4.778832in}{1.308439in}}%
\pgfpathlineto{\pgfqpoint{4.781643in}{1.308457in}}%
\pgfpathlineto{\pgfqpoint{4.784453in}{1.308564in}}%
\pgfpathlineto{\pgfqpoint{4.787264in}{1.308636in}}%
\pgfpathlineto{\pgfqpoint{4.790075in}{1.308714in}}%
\pgfpathlineto{\pgfqpoint{4.792885in}{1.308864in}}%
\pgfpathlineto{\pgfqpoint{4.795696in}{1.308444in}}%
\pgfpathlineto{\pgfqpoint{4.798507in}{1.308448in}}%
\pgfpathlineto{\pgfqpoint{4.801317in}{1.308501in}}%
\pgfpathlineto{\pgfqpoint{4.804128in}{1.308683in}}%
\pgfpathlineto{\pgfqpoint{4.806939in}{1.308656in}}%
\pgfpathlineto{\pgfqpoint{4.809749in}{1.308837in}}%
\pgfpathlineto{\pgfqpoint{4.812560in}{1.309000in}}%
\pgfpathlineto{\pgfqpoint{4.815371in}{1.309165in}}%
\pgfpathlineto{\pgfqpoint{4.818181in}{1.309346in}}%
\pgfpathlineto{\pgfqpoint{4.820992in}{1.309440in}}%
\pgfpathlineto{\pgfqpoint{4.823803in}{1.309616in}}%
\pgfpathlineto{\pgfqpoint{4.826614in}{1.309732in}}%
\pgfpathlineto{\pgfqpoint{4.829424in}{1.309912in}}%
\pgfpathlineto{\pgfqpoint{4.832235in}{1.310092in}}%
\pgfpathlineto{\pgfqpoint{4.835046in}{1.310201in}}%
\pgfpathlineto{\pgfqpoint{4.837856in}{1.310335in}}%
\pgfpathlineto{\pgfqpoint{4.840667in}{1.310515in}}%
\pgfpathlineto{\pgfqpoint{4.843478in}{1.310584in}}%
\pgfpathlineto{\pgfqpoint{4.846288in}{1.310763in}}%
\pgfpathlineto{\pgfqpoint{4.849099in}{1.310590in}}%
\pgfpathlineto{\pgfqpoint{4.851910in}{1.310651in}}%
\pgfpathlineto{\pgfqpoint{4.854720in}{1.310819in}}%
\pgfpathlineto{\pgfqpoint{4.857531in}{1.310973in}}%
\pgfpathlineto{\pgfqpoint{4.860342in}{1.311141in}}%
\pgfpathlineto{\pgfqpoint{4.863152in}{1.311143in}}%
\pgfpathlineto{\pgfqpoint{4.865963in}{1.311302in}}%
\pgfpathlineto{\pgfqpoint{4.868774in}{1.311477in}}%
\pgfpathlineto{\pgfqpoint{4.871584in}{1.311576in}}%
\pgfpathlineto{\pgfqpoint{4.874395in}{1.311656in}}%
\pgfpathlineto{\pgfqpoint{4.877206in}{1.311412in}}%
\pgfpathlineto{\pgfqpoint{4.880017in}{1.311588in}}%
\pgfpathlineto{\pgfqpoint{4.882827in}{1.311683in}}%
\pgfpathlineto{\pgfqpoint{4.885638in}{1.311860in}}%
\pgfpathlineto{\pgfqpoint{4.888449in}{1.312036in}}%
\pgfpathlineto{\pgfqpoint{4.891259in}{1.311914in}}%
\pgfpathlineto{\pgfqpoint{4.894070in}{1.312075in}}%
\pgfpathlineto{\pgfqpoint{4.896881in}{1.312251in}}%
\pgfpathlineto{\pgfqpoint{4.899691in}{1.312427in}}%
\pgfpathlineto{\pgfqpoint{4.902502in}{1.312383in}}%
\pgfpathlineto{\pgfqpoint{4.905313in}{1.312514in}}%
\pgfpathlineto{\pgfqpoint{4.908123in}{1.312656in}}%
\pgfpathlineto{\pgfqpoint{4.910934in}{1.312822in}}%
\pgfpathlineto{\pgfqpoint{4.913745in}{1.312947in}}%
\pgfpathlineto{\pgfqpoint{4.916555in}{1.313068in}}%
\pgfpathlineto{\pgfqpoint{4.919366in}{1.313165in}}%
\pgfpathlineto{\pgfqpoint{4.922177in}{1.313326in}}%
\pgfpathlineto{\pgfqpoint{4.924988in}{1.313284in}}%
\pgfpathlineto{\pgfqpoint{4.927798in}{1.313427in}}%
\pgfpathlineto{\pgfqpoint{4.930609in}{1.313575in}}%
\pgfpathlineto{\pgfqpoint{4.933420in}{1.313683in}}%
\pgfpathlineto{\pgfqpoint{4.936230in}{1.313796in}}%
\pgfpathlineto{\pgfqpoint{4.939041in}{1.313928in}}%
\pgfpathlineto{\pgfqpoint{4.941852in}{1.313998in}}%
\pgfpathlineto{\pgfqpoint{4.944662in}{1.314148in}}%
\pgfpathlineto{\pgfqpoint{4.947473in}{1.314135in}}%
\pgfpathlineto{\pgfqpoint{4.950284in}{1.314186in}}%
\pgfpathlineto{\pgfqpoint{4.953094in}{1.314355in}}%
\pgfpathlineto{\pgfqpoint{4.955905in}{1.314413in}}%
\pgfpathlineto{\pgfqpoint{4.958716in}{1.314570in}}%
\pgfpathlineto{\pgfqpoint{4.961526in}{1.314719in}}%
\pgfpathlineto{\pgfqpoint{4.964337in}{1.314052in}}%
\pgfpathlineto{\pgfqpoint{4.967148in}{1.314166in}}%
\pgfpathlineto{\pgfqpoint{4.969959in}{1.314337in}}%
\pgfpathlineto{\pgfqpoint{4.972769in}{1.314508in}}%
\pgfpathlineto{\pgfqpoint{4.975580in}{1.314678in}}%
\pgfpathlineto{\pgfqpoint{4.978391in}{1.314822in}}%
\pgfpathlineto{\pgfqpoint{4.981201in}{1.314962in}}%
\pgfpathlineto{\pgfqpoint{4.984012in}{1.315048in}}%
\pgfpathlineto{\pgfqpoint{4.986823in}{1.315218in}}%
\pgfpathlineto{\pgfqpoint{4.989633in}{1.315382in}}%
\pgfpathlineto{\pgfqpoint{4.992444in}{1.315532in}}%
\pgfpathlineto{\pgfqpoint{4.995255in}{1.315684in}}%
\pgfpathlineto{\pgfqpoint{4.998065in}{1.315847in}}%
\pgfpathlineto{\pgfqpoint{5.000876in}{1.315923in}}%
\pgfpathlineto{\pgfqpoint{5.003687in}{1.316088in}}%
\pgfpathlineto{\pgfqpoint{5.006497in}{1.316161in}}%
\pgfpathlineto{\pgfqpoint{5.009308in}{1.316155in}}%
\pgfpathlineto{\pgfqpoint{5.012119in}{1.316302in}}%
\pgfpathlineto{\pgfqpoint{5.014930in}{1.316291in}}%
\pgfpathlineto{\pgfqpoint{5.017740in}{1.316455in}}%
\pgfpathlineto{\pgfqpoint{5.020551in}{1.316574in}}%
\pgfpathlineto{\pgfqpoint{5.023362in}{1.316743in}}%
\pgfpathlineto{\pgfqpoint{5.026172in}{1.316908in}}%
\pgfpathlineto{\pgfqpoint{5.028983in}{1.317058in}}%
\pgfpathlineto{\pgfqpoint{5.031794in}{1.317169in}}%
\pgfpathlineto{\pgfqpoint{5.034604in}{1.317319in}}%
\pgfpathlineto{\pgfqpoint{5.037415in}{1.317433in}}%
\pgfpathlineto{\pgfqpoint{5.040226in}{1.317597in}}%
\pgfpathlineto{\pgfqpoint{5.043036in}{1.317707in}}%
\pgfpathlineto{\pgfqpoint{5.045847in}{1.317861in}}%
\pgfpathlineto{\pgfqpoint{5.048658in}{1.318023in}}%
\pgfpathlineto{\pgfqpoint{5.051468in}{1.318160in}}%
\pgfpathlineto{\pgfqpoint{5.054279in}{1.318313in}}%
\pgfpathlineto{\pgfqpoint{5.057090in}{1.318391in}}%
\pgfpathlineto{\pgfqpoint{5.059901in}{1.318477in}}%
\pgfpathlineto{\pgfqpoint{5.062711in}{1.318530in}}%
\pgfpathlineto{\pgfqpoint{5.065522in}{1.318691in}}%
\pgfpathlineto{\pgfqpoint{5.068333in}{1.318422in}}%
\pgfpathlineto{\pgfqpoint{5.071143in}{1.318587in}}%
\pgfpathlineto{\pgfqpoint{5.073954in}{1.318477in}}%
\pgfpathlineto{\pgfqpoint{5.076765in}{1.318610in}}%
\pgfpathlineto{\pgfqpoint{5.079575in}{1.318747in}}%
\pgfpathlineto{\pgfqpoint{5.082386in}{1.318772in}}%
\pgfpathlineto{\pgfqpoint{5.085197in}{1.318931in}}%
\pgfpathlineto{\pgfqpoint{5.088007in}{1.319085in}}%
\pgfpathlineto{\pgfqpoint{5.090818in}{1.319249in}}%
\pgfpathlineto{\pgfqpoint{5.093629in}{1.318191in}}%
\pgfpathlineto{\pgfqpoint{5.096439in}{1.318125in}}%
\pgfpathlineto{\pgfqpoint{5.099250in}{1.317817in}}%
\pgfpathlineto{\pgfqpoint{5.102061in}{1.316835in}}%
\pgfpathlineto{\pgfqpoint{5.104871in}{1.316993in}}%
\pgfpathlineto{\pgfqpoint{5.107682in}{1.317151in}}%
\pgfpathlineto{\pgfqpoint{5.110493in}{1.317272in}}%
\pgfpathlineto{\pgfqpoint{5.113304in}{1.317429in}}%
\pgfpathlineto{\pgfqpoint{5.116114in}{1.317519in}}%
\pgfpathlineto{\pgfqpoint{5.118925in}{1.317681in}}%
\pgfpathlineto{\pgfqpoint{5.121736in}{1.317781in}}%
\pgfpathlineto{\pgfqpoint{5.124546in}{1.317848in}}%
\pgfpathlineto{\pgfqpoint{5.127357in}{1.318012in}}%
\pgfpathlineto{\pgfqpoint{5.130168in}{1.318098in}}%
\pgfpathlineto{\pgfqpoint{5.132978in}{1.317939in}}%
\pgfpathlineto{\pgfqpoint{5.135789in}{1.318055in}}%
\pgfpathlineto{\pgfqpoint{5.138600in}{1.318199in}}%
\pgfpathlineto{\pgfqpoint{5.141410in}{1.318342in}}%
\pgfpathlineto{\pgfqpoint{5.144221in}{1.318486in}}%
\pgfpathlineto{\pgfqpoint{5.147032in}{1.318647in}}%
\pgfpathlineto{\pgfqpoint{5.149842in}{1.318793in}}%
\pgfpathlineto{\pgfqpoint{5.149842in}{2.098662in}}%
\pgfpathlineto{\pgfqpoint{5.149842in}{2.098662in}}%
\pgfpathlineto{\pgfqpoint{5.147032in}{2.098769in}}%
\pgfpathlineto{\pgfqpoint{5.144221in}{2.098765in}}%
\pgfpathlineto{\pgfqpoint{5.141410in}{2.098876in}}%
\pgfpathlineto{\pgfqpoint{5.138600in}{2.098989in}}%
\pgfpathlineto{\pgfqpoint{5.135789in}{2.099101in}}%
\pgfpathlineto{\pgfqpoint{5.132978in}{2.099243in}}%
\pgfpathlineto{\pgfqpoint{5.130168in}{2.099367in}}%
\pgfpathlineto{\pgfqpoint{5.127357in}{2.099523in}}%
\pgfpathlineto{\pgfqpoint{5.124546in}{2.099544in}}%
\pgfpathlineto{\pgfqpoint{5.121736in}{2.099706in}}%
\pgfpathlineto{\pgfqpoint{5.118925in}{2.099857in}}%
\pgfpathlineto{\pgfqpoint{5.116114in}{2.099839in}}%
\pgfpathlineto{\pgfqpoint{5.113304in}{2.099995in}}%
\pgfpathlineto{\pgfqpoint{5.110493in}{2.099949in}}%
\pgfpathlineto{\pgfqpoint{5.107682in}{2.099779in}}%
\pgfpathlineto{\pgfqpoint{5.104871in}{2.099736in}}%
\pgfpathlineto{\pgfqpoint{5.102061in}{2.099694in}}%
\pgfpathlineto{\pgfqpoint{5.099250in}{2.099368in}}%
\pgfpathlineto{\pgfqpoint{5.096439in}{2.099432in}}%
\pgfpathlineto{\pgfqpoint{5.093629in}{2.098872in}}%
\pgfpathlineto{\pgfqpoint{5.090818in}{2.098495in}}%
\pgfpathlineto{\pgfqpoint{5.088007in}{2.098500in}}%
\pgfpathlineto{\pgfqpoint{5.085197in}{2.098598in}}%
\pgfpathlineto{\pgfqpoint{5.082386in}{2.098564in}}%
\pgfpathlineto{\pgfqpoint{5.079575in}{2.098730in}}%
\pgfpathlineto{\pgfqpoint{5.076765in}{2.098605in}}%
\pgfpathlineto{\pgfqpoint{5.073954in}{2.098735in}}%
\pgfpathlineto{\pgfqpoint{5.071143in}{2.098877in}}%
\pgfpathlineto{\pgfqpoint{5.068333in}{2.098879in}}%
\pgfpathlineto{\pgfqpoint{5.065522in}{2.098962in}}%
\pgfpathlineto{\pgfqpoint{5.062711in}{2.099037in}}%
\pgfpathlineto{\pgfqpoint{5.059901in}{2.099203in}}%
\pgfpathlineto{\pgfqpoint{5.057090in}{2.099362in}}%
\pgfpathlineto{\pgfqpoint{5.054279in}{2.099524in}}%
\pgfpathlineto{\pgfqpoint{5.051468in}{2.099628in}}%
\pgfpathlineto{\pgfqpoint{5.048658in}{2.099756in}}%
\pgfpathlineto{\pgfqpoint{5.045847in}{2.099835in}}%
\pgfpathlineto{\pgfqpoint{5.043036in}{2.099938in}}%
\pgfpathlineto{\pgfqpoint{5.040226in}{2.100089in}}%
\pgfpathlineto{\pgfqpoint{5.037415in}{2.100065in}}%
\pgfpathlineto{\pgfqpoint{5.034604in}{2.100213in}}%
\pgfpathlineto{\pgfqpoint{5.031794in}{2.100324in}}%
\pgfpathlineto{\pgfqpoint{5.028983in}{2.100475in}}%
\pgfpathlineto{\pgfqpoint{5.026172in}{2.100587in}}%
\pgfpathlineto{\pgfqpoint{5.023362in}{2.100564in}}%
\pgfpathlineto{\pgfqpoint{5.020551in}{2.100581in}}%
\pgfpathlineto{\pgfqpoint{5.017740in}{2.100727in}}%
\pgfpathlineto{\pgfqpoint{5.014930in}{2.100807in}}%
\pgfpathlineto{\pgfqpoint{5.012119in}{2.100974in}}%
\pgfpathlineto{\pgfqpoint{5.009308in}{2.101093in}}%
\pgfpathlineto{\pgfqpoint{5.006497in}{2.101261in}}%
\pgfpathlineto{\pgfqpoint{5.003687in}{2.101427in}}%
\pgfpathlineto{\pgfqpoint{5.000876in}{2.101506in}}%
\pgfpathlineto{\pgfqpoint{4.998065in}{2.101672in}}%
\pgfpathlineto{\pgfqpoint{4.995255in}{2.101633in}}%
\pgfpathlineto{\pgfqpoint{4.992444in}{2.101746in}}%
\pgfpathlineto{\pgfqpoint{4.989633in}{2.101864in}}%
\pgfpathlineto{\pgfqpoint{4.986823in}{2.101950in}}%
\pgfpathlineto{\pgfqpoint{4.984012in}{2.101955in}}%
\pgfpathlineto{\pgfqpoint{4.981201in}{2.102120in}}%
\pgfpathlineto{\pgfqpoint{4.978391in}{2.102252in}}%
\pgfpathlineto{\pgfqpoint{4.975580in}{2.102379in}}%
\pgfpathlineto{\pgfqpoint{4.972769in}{2.102437in}}%
\pgfpathlineto{\pgfqpoint{4.969959in}{2.102486in}}%
\pgfpathlineto{\pgfqpoint{4.967148in}{2.102481in}}%
\pgfpathlineto{\pgfqpoint{4.964337in}{2.102635in}}%
\pgfpathlineto{\pgfqpoint{4.961526in}{2.102517in}}%
\pgfpathlineto{\pgfqpoint{4.958716in}{2.102639in}}%
\pgfpathlineto{\pgfqpoint{4.955905in}{2.102748in}}%
\pgfpathlineto{\pgfqpoint{4.953094in}{2.102920in}}%
\pgfpathlineto{\pgfqpoint{4.950284in}{2.102992in}}%
\pgfpathlineto{\pgfqpoint{4.947473in}{2.103165in}}%
\pgfpathlineto{\pgfqpoint{4.944662in}{2.103335in}}%
\pgfpathlineto{\pgfqpoint{4.941852in}{2.103457in}}%
\pgfpathlineto{\pgfqpoint{4.939041in}{2.103628in}}%
\pgfpathlineto{\pgfqpoint{4.936230in}{2.103771in}}%
\pgfpathlineto{\pgfqpoint{4.933420in}{2.103552in}}%
\pgfpathlineto{\pgfqpoint{4.930609in}{2.103711in}}%
\pgfpathlineto{\pgfqpoint{4.927798in}{2.103589in}}%
\pgfpathlineto{\pgfqpoint{4.924988in}{2.103722in}}%
\pgfpathlineto{\pgfqpoint{4.922177in}{2.103889in}}%
\pgfpathlineto{\pgfqpoint{4.919366in}{2.103994in}}%
\pgfpathlineto{\pgfqpoint{4.916555in}{2.104159in}}%
\pgfpathlineto{\pgfqpoint{4.913745in}{2.104312in}}%
\pgfpathlineto{\pgfqpoint{4.910934in}{2.104463in}}%
\pgfpathlineto{\pgfqpoint{4.908123in}{2.104560in}}%
\pgfpathlineto{\pgfqpoint{4.905313in}{2.104697in}}%
\pgfpathlineto{\pgfqpoint{4.902502in}{2.104843in}}%
\pgfpathlineto{\pgfqpoint{4.899691in}{2.105011in}}%
\pgfpathlineto{\pgfqpoint{4.896881in}{2.105030in}}%
\pgfpathlineto{\pgfqpoint{4.894070in}{2.105077in}}%
\pgfpathlineto{\pgfqpoint{4.891259in}{2.105188in}}%
\pgfpathlineto{\pgfqpoint{4.888449in}{2.105337in}}%
\pgfpathlineto{\pgfqpoint{4.885638in}{2.105355in}}%
\pgfpathlineto{\pgfqpoint{4.882827in}{2.105391in}}%
\pgfpathlineto{\pgfqpoint{4.880017in}{2.105120in}}%
\pgfpathlineto{\pgfqpoint{4.877206in}{2.105183in}}%
\pgfpathlineto{\pgfqpoint{4.874395in}{2.105290in}}%
\pgfpathlineto{\pgfqpoint{4.871584in}{2.105462in}}%
\pgfpathlineto{\pgfqpoint{4.868774in}{2.105630in}}%
\pgfpathlineto{\pgfqpoint{4.865963in}{2.105697in}}%
\pgfpathlineto{\pgfqpoint{4.863152in}{2.105815in}}%
\pgfpathlineto{\pgfqpoint{4.860342in}{2.105992in}}%
\pgfpathlineto{\pgfqpoint{4.857531in}{2.106091in}}%
\pgfpathlineto{\pgfqpoint{4.854720in}{2.105978in}}%
\pgfpathlineto{\pgfqpoint{4.851910in}{2.106079in}}%
\pgfpathlineto{\pgfqpoint{4.849099in}{2.106256in}}%
\pgfpathlineto{\pgfqpoint{4.846288in}{2.106392in}}%
\pgfpathlineto{\pgfqpoint{4.843478in}{2.106405in}}%
\pgfpathlineto{\pgfqpoint{4.840667in}{2.106581in}}%
\pgfpathlineto{\pgfqpoint{4.837856in}{2.106612in}}%
\pgfpathlineto{\pgfqpoint{4.835046in}{2.106432in}}%
\pgfpathlineto{\pgfqpoint{4.832235in}{2.106597in}}%
\pgfpathlineto{\pgfqpoint{4.829424in}{2.106647in}}%
\pgfpathlineto{\pgfqpoint{4.826614in}{2.106677in}}%
\pgfpathlineto{\pgfqpoint{4.823803in}{2.106840in}}%
\pgfpathlineto{\pgfqpoint{4.820992in}{2.106810in}}%
\pgfpathlineto{\pgfqpoint{4.818181in}{2.106982in}}%
\pgfpathlineto{\pgfqpoint{4.815371in}{2.107005in}}%
\pgfpathlineto{\pgfqpoint{4.812560in}{2.107119in}}%
\pgfpathlineto{\pgfqpoint{4.809749in}{2.107027in}}%
\pgfpathlineto{\pgfqpoint{4.806939in}{2.107070in}}%
\pgfpathlineto{\pgfqpoint{4.804128in}{2.107247in}}%
\pgfpathlineto{\pgfqpoint{4.801317in}{2.107292in}}%
\pgfpathlineto{\pgfqpoint{4.798507in}{2.107474in}}%
\pgfpathlineto{\pgfqpoint{4.795696in}{2.107655in}}%
\pgfpathlineto{\pgfqpoint{4.792885in}{2.107687in}}%
\pgfpathlineto{\pgfqpoint{4.790075in}{2.107545in}}%
\pgfpathlineto{\pgfqpoint{4.787264in}{2.107724in}}%
\pgfpathlineto{\pgfqpoint{4.784453in}{2.107903in}}%
\pgfpathlineto{\pgfqpoint{4.781643in}{2.107645in}}%
\pgfpathlineto{\pgfqpoint{4.778832in}{2.107828in}}%
\pgfpathlineto{\pgfqpoint{4.776021in}{2.107965in}}%
\pgfpathlineto{\pgfqpoint{4.773210in}{2.108147in}}%
\pgfpathlineto{\pgfqpoint{4.770400in}{2.108204in}}%
\pgfpathlineto{\pgfqpoint{4.767589in}{2.108361in}}%
\pgfpathlineto{\pgfqpoint{4.764778in}{2.108508in}}%
\pgfpathlineto{\pgfqpoint{4.761968in}{2.108635in}}%
\pgfpathlineto{\pgfqpoint{4.759157in}{2.108390in}}%
\pgfpathlineto{\pgfqpoint{4.756346in}{2.108547in}}%
\pgfpathlineto{\pgfqpoint{4.753536in}{2.108640in}}%
\pgfpathlineto{\pgfqpoint{4.750725in}{2.108797in}}%
\pgfpathlineto{\pgfqpoint{4.747914in}{2.108918in}}%
\pgfpathlineto{\pgfqpoint{4.745104in}{2.109098in}}%
\pgfpathlineto{\pgfqpoint{4.742293in}{2.109201in}}%
\pgfpathlineto{\pgfqpoint{4.739482in}{2.109230in}}%
\pgfpathlineto{\pgfqpoint{4.736672in}{2.109370in}}%
\pgfpathlineto{\pgfqpoint{4.733861in}{2.109450in}}%
\pgfpathlineto{\pgfqpoint{4.731050in}{2.109603in}}%
\pgfpathlineto{\pgfqpoint{4.728239in}{2.109787in}}%
\pgfpathlineto{\pgfqpoint{4.725429in}{2.109917in}}%
\pgfpathlineto{\pgfqpoint{4.722618in}{2.109893in}}%
\pgfpathlineto{\pgfqpoint{4.719807in}{2.109964in}}%
\pgfpathlineto{\pgfqpoint{4.716997in}{2.109970in}}%
\pgfpathlineto{\pgfqpoint{4.714186in}{2.109987in}}%
\pgfpathlineto{\pgfqpoint{4.711375in}{2.110152in}}%
\pgfpathlineto{\pgfqpoint{4.708565in}{2.110291in}}%
\pgfpathlineto{\pgfqpoint{4.705754in}{2.110459in}}%
\pgfpathlineto{\pgfqpoint{4.702943in}{2.110542in}}%
\pgfpathlineto{\pgfqpoint{4.700133in}{2.110596in}}%
\pgfpathlineto{\pgfqpoint{4.697322in}{2.110770in}}%
\pgfpathlineto{\pgfqpoint{4.694511in}{2.110897in}}%
\pgfpathlineto{\pgfqpoint{4.691701in}{2.111049in}}%
\pgfpathlineto{\pgfqpoint{4.688890in}{2.111225in}}%
\pgfpathlineto{\pgfqpoint{4.686079in}{2.111414in}}%
\pgfpathlineto{\pgfqpoint{4.683268in}{2.111554in}}%
\pgfpathlineto{\pgfqpoint{4.680458in}{2.111725in}}%
\pgfpathlineto{\pgfqpoint{4.677647in}{2.111585in}}%
\pgfpathlineto{\pgfqpoint{4.674836in}{2.111742in}}%
\pgfpathlineto{\pgfqpoint{4.672026in}{2.111930in}}%
\pgfpathlineto{\pgfqpoint{4.669215in}{2.112111in}}%
\pgfpathlineto{\pgfqpoint{4.666404in}{2.112296in}}%
\pgfpathlineto{\pgfqpoint{4.663594in}{2.112425in}}%
\pgfpathlineto{\pgfqpoint{4.660783in}{2.112461in}}%
\pgfpathlineto{\pgfqpoint{4.657972in}{2.112623in}}%
\pgfpathlineto{\pgfqpoint{4.655162in}{2.112423in}}%
\pgfpathlineto{\pgfqpoint{4.652351in}{2.112584in}}%
\pgfpathlineto{\pgfqpoint{4.649540in}{2.112756in}}%
\pgfpathlineto{\pgfqpoint{4.646730in}{2.112716in}}%
\pgfpathlineto{\pgfqpoint{4.643919in}{2.112856in}}%
\pgfpathlineto{\pgfqpoint{4.641108in}{2.113035in}}%
\pgfpathlineto{\pgfqpoint{4.638298in}{2.113200in}}%
\pgfpathlineto{\pgfqpoint{4.635487in}{2.113350in}}%
\pgfpathlineto{\pgfqpoint{4.632676in}{2.113540in}}%
\pgfpathlineto{\pgfqpoint{4.629865in}{2.113697in}}%
\pgfpathlineto{\pgfqpoint{4.627055in}{2.113828in}}%
\pgfpathlineto{\pgfqpoint{4.624244in}{2.114021in}}%
\pgfpathlineto{\pgfqpoint{4.621433in}{2.114103in}}%
\pgfpathlineto{\pgfqpoint{4.618623in}{2.114282in}}%
\pgfpathlineto{\pgfqpoint{4.615812in}{2.114464in}}%
\pgfpathlineto{\pgfqpoint{4.613001in}{2.114622in}}%
\pgfpathlineto{\pgfqpoint{4.610191in}{2.114765in}}%
\pgfpathlineto{\pgfqpoint{4.607380in}{2.114952in}}%
\pgfpathlineto{\pgfqpoint{4.604569in}{2.115068in}}%
\pgfpathlineto{\pgfqpoint{4.601759in}{2.115184in}}%
\pgfpathlineto{\pgfqpoint{4.598948in}{2.115274in}}%
\pgfpathlineto{\pgfqpoint{4.596137in}{2.115429in}}%
\pgfpathlineto{\pgfqpoint{4.593327in}{2.115593in}}%
\pgfpathlineto{\pgfqpoint{4.590516in}{2.115734in}}%
\pgfpathlineto{\pgfqpoint{4.587705in}{2.115886in}}%
\pgfpathlineto{\pgfqpoint{4.584894in}{2.115996in}}%
\pgfpathlineto{\pgfqpoint{4.582084in}{2.116193in}}%
\pgfpathlineto{\pgfqpoint{4.579273in}{2.116303in}}%
\pgfpathlineto{\pgfqpoint{4.576462in}{2.116358in}}%
\pgfpathlineto{\pgfqpoint{4.573652in}{2.116507in}}%
\pgfpathlineto{\pgfqpoint{4.570841in}{2.116686in}}%
\pgfpathlineto{\pgfqpoint{4.568030in}{2.116839in}}%
\pgfpathlineto{\pgfqpoint{4.565220in}{2.116989in}}%
\pgfpathlineto{\pgfqpoint{4.562409in}{2.117120in}}%
\pgfpathlineto{\pgfqpoint{4.559598in}{2.117318in}}%
\pgfpathlineto{\pgfqpoint{4.556788in}{2.117243in}}%
\pgfpathlineto{\pgfqpoint{4.553977in}{2.117419in}}%
\pgfpathlineto{\pgfqpoint{4.551166in}{2.117617in}}%
\pgfpathlineto{\pgfqpoint{4.548356in}{2.117752in}}%
\pgfpathlineto{\pgfqpoint{4.545545in}{2.117860in}}%
\pgfpathlineto{\pgfqpoint{4.542734in}{2.118043in}}%
\pgfpathlineto{\pgfqpoint{4.539923in}{2.118136in}}%
\pgfpathlineto{\pgfqpoint{4.537113in}{2.118300in}}%
\pgfpathlineto{\pgfqpoint{4.534302in}{2.118485in}}%
\pgfpathlineto{\pgfqpoint{4.531491in}{2.118470in}}%
\pgfpathlineto{\pgfqpoint{4.528681in}{2.118570in}}%
\pgfpathlineto{\pgfqpoint{4.525870in}{2.118642in}}%
\pgfpathlineto{\pgfqpoint{4.523059in}{2.118764in}}%
\pgfpathlineto{\pgfqpoint{4.520249in}{2.118847in}}%
\pgfpathlineto{\pgfqpoint{4.517438in}{2.119044in}}%
\pgfpathlineto{\pgfqpoint{4.514627in}{2.119222in}}%
\pgfpathlineto{\pgfqpoint{4.511817in}{2.119417in}}%
\pgfpathlineto{\pgfqpoint{4.509006in}{2.117328in}}%
\pgfpathlineto{\pgfqpoint{4.506195in}{2.117515in}}%
\pgfpathlineto{\pgfqpoint{4.503385in}{2.117711in}}%
\pgfpathlineto{\pgfqpoint{4.500574in}{2.117881in}}%
\pgfpathlineto{\pgfqpoint{4.497763in}{2.118061in}}%
\pgfpathlineto{\pgfqpoint{4.494952in}{2.118131in}}%
\pgfpathlineto{\pgfqpoint{4.492142in}{2.118331in}}%
\pgfpathlineto{\pgfqpoint{4.489331in}{2.118361in}}%
\pgfpathlineto{\pgfqpoint{4.486520in}{2.118245in}}%
\pgfpathlineto{\pgfqpoint{4.483710in}{2.118366in}}%
\pgfpathlineto{\pgfqpoint{4.480899in}{2.118523in}}%
\pgfpathlineto{\pgfqpoint{4.478088in}{2.118676in}}%
\pgfpathlineto{\pgfqpoint{4.475278in}{2.118793in}}%
\pgfpathlineto{\pgfqpoint{4.472467in}{2.118954in}}%
\pgfpathlineto{\pgfqpoint{4.469656in}{2.119148in}}%
\pgfpathlineto{\pgfqpoint{4.466846in}{2.119353in}}%
\pgfpathlineto{\pgfqpoint{4.464035in}{2.119432in}}%
\pgfpathlineto{\pgfqpoint{4.461224in}{2.119637in}}%
\pgfpathlineto{\pgfqpoint{4.458414in}{2.119843in}}%
\pgfpathlineto{\pgfqpoint{4.455603in}{2.119703in}}%
\pgfpathlineto{\pgfqpoint{4.452792in}{2.119634in}}%
\pgfpathlineto{\pgfqpoint{4.449981in}{2.119662in}}%
\pgfpathlineto{\pgfqpoint{4.447171in}{2.119320in}}%
\pgfpathlineto{\pgfqpoint{4.444360in}{2.119523in}}%
\pgfpathlineto{\pgfqpoint{4.441549in}{2.119694in}}%
\pgfpathlineto{\pgfqpoint{4.438739in}{2.119877in}}%
\pgfpathlineto{\pgfqpoint{4.435928in}{2.120019in}}%
\pgfpathlineto{\pgfqpoint{4.433117in}{2.120152in}}%
\pgfpathlineto{\pgfqpoint{4.430307in}{2.120353in}}%
\pgfpathlineto{\pgfqpoint{4.427496in}{2.120553in}}%
\pgfpathlineto{\pgfqpoint{4.424685in}{2.120683in}}%
\pgfpathlineto{\pgfqpoint{4.421875in}{2.120885in}}%
\pgfpathlineto{\pgfqpoint{4.419064in}{2.121025in}}%
\pgfpathlineto{\pgfqpoint{4.416253in}{2.121134in}}%
\pgfpathlineto{\pgfqpoint{4.413443in}{2.121319in}}%
\pgfpathlineto{\pgfqpoint{4.410632in}{2.121333in}}%
\pgfpathlineto{\pgfqpoint{4.407821in}{2.121538in}}%
\pgfpathlineto{\pgfqpoint{4.405011in}{2.121732in}}%
\pgfpathlineto{\pgfqpoint{4.402200in}{2.121914in}}%
\pgfpathlineto{\pgfqpoint{4.399389in}{2.121068in}}%
\pgfpathlineto{\pgfqpoint{4.396578in}{2.121269in}}%
\pgfpathlineto{\pgfqpoint{4.393768in}{2.120829in}}%
\pgfpathlineto{\pgfqpoint{4.390957in}{2.120951in}}%
\pgfpathlineto{\pgfqpoint{4.388146in}{2.120836in}}%
\pgfpathlineto{\pgfqpoint{4.385336in}{2.120765in}}%
\pgfpathlineto{\pgfqpoint{4.382525in}{2.120934in}}%
\pgfpathlineto{\pgfqpoint{4.379714in}{2.121096in}}%
\pgfpathlineto{\pgfqpoint{4.376904in}{2.121155in}}%
\pgfpathlineto{\pgfqpoint{4.374093in}{2.121364in}}%
\pgfpathlineto{\pgfqpoint{4.371282in}{2.121342in}}%
\pgfpathlineto{\pgfqpoint{4.368472in}{2.121303in}}%
\pgfpathlineto{\pgfqpoint{4.365661in}{2.121508in}}%
\pgfpathlineto{\pgfqpoint{4.362850in}{2.121417in}}%
\pgfpathlineto{\pgfqpoint{4.360040in}{2.120952in}}%
\pgfpathlineto{\pgfqpoint{4.357229in}{2.121102in}}%
\pgfpathlineto{\pgfqpoint{4.354418in}{2.119989in}}%
\pgfpathlineto{\pgfqpoint{4.351607in}{2.120168in}}%
\pgfpathlineto{\pgfqpoint{4.348797in}{2.120273in}}%
\pgfpathlineto{\pgfqpoint{4.345986in}{2.120390in}}%
\pgfpathlineto{\pgfqpoint{4.343175in}{2.120502in}}%
\pgfpathlineto{\pgfqpoint{4.340365in}{2.119752in}}%
\pgfpathlineto{\pgfqpoint{4.337554in}{2.119960in}}%
\pgfpathlineto{\pgfqpoint{4.334743in}{2.120131in}}%
\pgfpathlineto{\pgfqpoint{4.331933in}{2.120299in}}%
\pgfpathlineto{\pgfqpoint{4.329122in}{2.120489in}}%
\pgfpathlineto{\pgfqpoint{4.326311in}{2.120622in}}%
\pgfpathlineto{\pgfqpoint{4.323501in}{2.120753in}}%
\pgfpathlineto{\pgfqpoint{4.320690in}{2.120908in}}%
\pgfpathlineto{\pgfqpoint{4.317879in}{2.121117in}}%
\pgfpathlineto{\pgfqpoint{4.315069in}{2.121272in}}%
\pgfpathlineto{\pgfqpoint{4.312258in}{2.121256in}}%
\pgfpathlineto{\pgfqpoint{4.309447in}{2.121456in}}%
\pgfpathlineto{\pgfqpoint{4.306636in}{2.121668in}}%
\pgfpathlineto{\pgfqpoint{4.303826in}{2.121475in}}%
\pgfpathlineto{\pgfqpoint{4.301015in}{2.121690in}}%
\pgfpathlineto{\pgfqpoint{4.298204in}{2.121902in}}%
\pgfpathlineto{\pgfqpoint{4.295394in}{2.121961in}}%
\pgfpathlineto{\pgfqpoint{4.292583in}{2.122176in}}%
\pgfpathlineto{\pgfqpoint{4.289772in}{2.122326in}}%
\pgfpathlineto{\pgfqpoint{4.286962in}{2.122494in}}%
\pgfpathlineto{\pgfqpoint{4.284151in}{2.122640in}}%
\pgfpathlineto{\pgfqpoint{4.281340in}{2.122858in}}%
\pgfpathlineto{\pgfqpoint{4.278530in}{2.123043in}}%
\pgfpathlineto{\pgfqpoint{4.275719in}{2.123093in}}%
\pgfpathlineto{\pgfqpoint{4.272908in}{2.123294in}}%
\pgfpathlineto{\pgfqpoint{4.270098in}{2.123455in}}%
\pgfpathlineto{\pgfqpoint{4.267287in}{2.123454in}}%
\pgfpathlineto{\pgfqpoint{4.264476in}{2.123630in}}%
\pgfpathlineto{\pgfqpoint{4.261665in}{2.123618in}}%
\pgfpathlineto{\pgfqpoint{4.258855in}{2.123742in}}%
\pgfpathlineto{\pgfqpoint{4.256044in}{2.123953in}}%
\pgfpathlineto{\pgfqpoint{4.253233in}{2.124157in}}%
\pgfpathlineto{\pgfqpoint{4.250423in}{2.124323in}}%
\pgfpathlineto{\pgfqpoint{4.247612in}{2.124335in}}%
\pgfpathlineto{\pgfqpoint{4.244801in}{2.124527in}}%
\pgfpathlineto{\pgfqpoint{4.241991in}{2.124630in}}%
\pgfpathlineto{\pgfqpoint{4.239180in}{2.124809in}}%
\pgfpathlineto{\pgfqpoint{4.236369in}{2.124889in}}%
\pgfpathlineto{\pgfqpoint{4.233559in}{2.125110in}}%
\pgfpathlineto{\pgfqpoint{4.230748in}{2.125284in}}%
\pgfpathlineto{\pgfqpoint{4.227937in}{2.124875in}}%
\pgfpathlineto{\pgfqpoint{4.225127in}{2.124909in}}%
\pgfpathlineto{\pgfqpoint{4.222316in}{2.125126in}}%
\pgfpathlineto{\pgfqpoint{4.219505in}{2.125149in}}%
\pgfpathlineto{\pgfqpoint{4.216695in}{2.125227in}}%
\pgfpathlineto{\pgfqpoint{4.213884in}{2.125253in}}%
\pgfpathlineto{\pgfqpoint{4.211073in}{2.125365in}}%
\pgfpathlineto{\pgfqpoint{4.208262in}{2.125583in}}%
\pgfpathlineto{\pgfqpoint{4.205452in}{2.125720in}}%
\pgfpathlineto{\pgfqpoint{4.202641in}{2.125855in}}%
\pgfpathlineto{\pgfqpoint{4.199830in}{2.126003in}}%
\pgfpathlineto{\pgfqpoint{4.197020in}{2.126190in}}%
\pgfpathlineto{\pgfqpoint{4.194209in}{2.126415in}}%
\pgfpathlineto{\pgfqpoint{4.191398in}{2.126634in}}%
\pgfpathlineto{\pgfqpoint{4.188588in}{2.126695in}}%
\pgfpathlineto{\pgfqpoint{4.185777in}{2.126853in}}%
\pgfpathlineto{\pgfqpoint{4.182966in}{2.127078in}}%
\pgfpathlineto{\pgfqpoint{4.180156in}{2.127242in}}%
\pgfpathlineto{\pgfqpoint{4.177345in}{2.127449in}}%
\pgfpathlineto{\pgfqpoint{4.174534in}{2.127399in}}%
\pgfpathlineto{\pgfqpoint{4.171724in}{2.127606in}}%
\pgfpathlineto{\pgfqpoint{4.168913in}{2.127708in}}%
\pgfpathlineto{\pgfqpoint{4.166102in}{2.127935in}}%
\pgfpathlineto{\pgfqpoint{4.163291in}{2.128112in}}%
\pgfpathlineto{\pgfqpoint{4.160481in}{2.128329in}}%
\pgfpathlineto{\pgfqpoint{4.157670in}{2.128361in}}%
\pgfpathlineto{\pgfqpoint{4.154859in}{2.128355in}}%
\pgfpathlineto{\pgfqpoint{4.152049in}{2.128458in}}%
\pgfpathlineto{\pgfqpoint{4.149238in}{2.128651in}}%
\pgfpathlineto{\pgfqpoint{4.146427in}{2.128808in}}%
\pgfpathlineto{\pgfqpoint{4.143617in}{2.128966in}}%
\pgfpathlineto{\pgfqpoint{4.140806in}{2.128997in}}%
\pgfpathlineto{\pgfqpoint{4.137995in}{2.129187in}}%
\pgfpathlineto{\pgfqpoint{4.135185in}{2.129418in}}%
\pgfpathlineto{\pgfqpoint{4.132374in}{2.129611in}}%
\pgfpathlineto{\pgfqpoint{4.129563in}{2.129455in}}%
\pgfpathlineto{\pgfqpoint{4.126753in}{2.129685in}}%
\pgfpathlineto{\pgfqpoint{4.123942in}{2.129728in}}%
\pgfpathlineto{\pgfqpoint{4.121131in}{2.129839in}}%
\pgfpathlineto{\pgfqpoint{4.118320in}{2.130036in}}%
\pgfpathlineto{\pgfqpoint{4.115510in}{2.130206in}}%
\pgfpathlineto{\pgfqpoint{4.112699in}{2.130230in}}%
\pgfpathlineto{\pgfqpoint{4.109888in}{2.130430in}}%
\pgfpathlineto{\pgfqpoint{4.107078in}{2.130337in}}%
\pgfpathlineto{\pgfqpoint{4.104267in}{2.130519in}}%
\pgfpathlineto{\pgfqpoint{4.101456in}{2.129793in}}%
\pgfpathlineto{\pgfqpoint{4.098646in}{2.129886in}}%
\pgfpathlineto{\pgfqpoint{4.095835in}{2.130115in}}%
\pgfpathlineto{\pgfqpoint{4.093024in}{2.130278in}}%
\pgfpathlineto{\pgfqpoint{4.090214in}{2.130412in}}%
\pgfpathlineto{\pgfqpoint{4.087403in}{2.129876in}}%
\pgfpathlineto{\pgfqpoint{4.084592in}{2.129409in}}%
\pgfpathlineto{\pgfqpoint{4.081782in}{2.128728in}}%
\pgfpathlineto{\pgfqpoint{4.078971in}{2.128676in}}%
\pgfpathlineto{\pgfqpoint{4.076160in}{2.127658in}}%
\pgfpathlineto{\pgfqpoint{4.073349in}{2.127122in}}%
\pgfpathlineto{\pgfqpoint{4.070539in}{2.127337in}}%
\pgfpathlineto{\pgfqpoint{4.067728in}{2.127539in}}%
\pgfpathlineto{\pgfqpoint{4.064917in}{2.127666in}}%
\pgfpathlineto{\pgfqpoint{4.062107in}{2.127767in}}%
\pgfpathlineto{\pgfqpoint{4.059296in}{2.127935in}}%
\pgfpathlineto{\pgfqpoint{4.056485in}{2.128170in}}%
\pgfpathlineto{\pgfqpoint{4.053675in}{2.128341in}}%
\pgfpathlineto{\pgfqpoint{4.050864in}{2.128299in}}%
\pgfpathlineto{\pgfqpoint{4.048053in}{2.128480in}}%
\pgfpathlineto{\pgfqpoint{4.045243in}{2.128716in}}%
\pgfpathlineto{\pgfqpoint{4.042432in}{2.128602in}}%
\pgfpathlineto{\pgfqpoint{4.039621in}{2.128795in}}%
\pgfpathlineto{\pgfqpoint{4.036811in}{2.128853in}}%
\pgfpathlineto{\pgfqpoint{4.034000in}{2.129058in}}%
\pgfpathlineto{\pgfqpoint{4.031189in}{2.129070in}}%
\pgfpathlineto{\pgfqpoint{4.028378in}{2.129218in}}%
\pgfpathlineto{\pgfqpoint{4.025568in}{2.129446in}}%
\pgfpathlineto{\pgfqpoint{4.022757in}{2.129480in}}%
\pgfpathlineto{\pgfqpoint{4.019946in}{2.129693in}}%
\pgfpathlineto{\pgfqpoint{4.017136in}{2.129930in}}%
\pgfpathlineto{\pgfqpoint{4.014325in}{2.129031in}}%
\pgfpathlineto{\pgfqpoint{4.011514in}{2.129271in}}%
\pgfpathlineto{\pgfqpoint{4.008704in}{2.129250in}}%
\pgfpathlineto{\pgfqpoint{4.005893in}{2.129489in}}%
\pgfpathlineto{\pgfqpoint{4.003082in}{2.129524in}}%
\pgfpathlineto{\pgfqpoint{4.000272in}{2.129712in}}%
\pgfpathlineto{\pgfqpoint{3.997461in}{2.129608in}}%
\pgfpathlineto{\pgfqpoint{3.994650in}{2.129787in}}%
\pgfpathlineto{\pgfqpoint{3.991840in}{2.129843in}}%
\pgfpathlineto{\pgfqpoint{3.989029in}{2.129804in}}%
\pgfpathlineto{\pgfqpoint{3.986218in}{2.129703in}}%
\pgfpathlineto{\pgfqpoint{3.983408in}{2.129793in}}%
\pgfpathlineto{\pgfqpoint{3.980597in}{2.129920in}}%
\pgfpathlineto{\pgfqpoint{3.977786in}{2.130069in}}%
\pgfpathlineto{\pgfqpoint{3.974975in}{2.130293in}}%
\pgfpathlineto{\pgfqpoint{3.972165in}{2.130426in}}%
\pgfpathlineto{\pgfqpoint{3.969354in}{2.130195in}}%
\pgfpathlineto{\pgfqpoint{3.966543in}{2.130439in}}%
\pgfpathlineto{\pgfqpoint{3.963733in}{2.130620in}}%
\pgfpathlineto{\pgfqpoint{3.960922in}{2.130787in}}%
\pgfpathlineto{\pgfqpoint{3.958111in}{2.130910in}}%
\pgfpathlineto{\pgfqpoint{3.955301in}{2.131114in}}%
\pgfpathlineto{\pgfqpoint{3.952490in}{2.131124in}}%
\pgfpathlineto{\pgfqpoint{3.949679in}{2.131368in}}%
\pgfpathlineto{\pgfqpoint{3.946869in}{2.131494in}}%
\pgfpathlineto{\pgfqpoint{3.944058in}{2.131732in}}%
\pgfpathlineto{\pgfqpoint{3.941247in}{2.131440in}}%
\pgfpathlineto{\pgfqpoint{3.938437in}{2.131678in}}%
\pgfpathlineto{\pgfqpoint{3.935626in}{2.131736in}}%
\pgfpathlineto{\pgfqpoint{3.932815in}{2.131532in}}%
\pgfpathlineto{\pgfqpoint{3.930004in}{2.131549in}}%
\pgfpathlineto{\pgfqpoint{3.927194in}{2.131787in}}%
\pgfpathlineto{\pgfqpoint{3.924383in}{2.131893in}}%
\pgfpathlineto{\pgfqpoint{3.921572in}{2.132137in}}%
\pgfpathlineto{\pgfqpoint{3.918762in}{2.131890in}}%
\pgfpathlineto{\pgfqpoint{3.915951in}{2.132111in}}%
\pgfpathlineto{\pgfqpoint{3.913140in}{2.132277in}}%
\pgfpathlineto{\pgfqpoint{3.910330in}{2.132094in}}%
\pgfpathlineto{\pgfqpoint{3.907519in}{2.132341in}}%
\pgfpathlineto{\pgfqpoint{3.904708in}{2.131973in}}%
\pgfpathlineto{\pgfqpoint{3.901898in}{2.131656in}}%
\pgfpathlineto{\pgfqpoint{3.899087in}{2.131879in}}%
\pgfpathlineto{\pgfqpoint{3.896276in}{2.132046in}}%
\pgfpathlineto{\pgfqpoint{3.893466in}{2.131923in}}%
\pgfpathlineto{\pgfqpoint{3.890655in}{2.132148in}}%
\pgfpathlineto{\pgfqpoint{3.887844in}{2.132364in}}%
\pgfpathlineto{\pgfqpoint{3.885033in}{2.132389in}}%
\pgfpathlineto{\pgfqpoint{3.882223in}{2.132356in}}%
\pgfpathlineto{\pgfqpoint{3.879412in}{2.132448in}}%
\pgfpathlineto{\pgfqpoint{3.876601in}{2.132555in}}%
\pgfpathlineto{\pgfqpoint{3.873791in}{2.132791in}}%
\pgfpathlineto{\pgfqpoint{3.870980in}{2.132340in}}%
\pgfpathlineto{\pgfqpoint{3.868169in}{2.132594in}}%
\pgfpathlineto{\pgfqpoint{3.865359in}{2.132798in}}%
\pgfpathlineto{\pgfqpoint{3.862548in}{2.132888in}}%
\pgfpathlineto{\pgfqpoint{3.859737in}{2.132678in}}%
\pgfpathlineto{\pgfqpoint{3.856927in}{2.132897in}}%
\pgfpathlineto{\pgfqpoint{3.854116in}{2.133151in}}%
\pgfpathlineto{\pgfqpoint{3.851305in}{2.133405in}}%
\pgfpathlineto{\pgfqpoint{3.848495in}{2.132243in}}%
\pgfpathlineto{\pgfqpoint{3.845684in}{2.132496in}}%
\pgfpathlineto{\pgfqpoint{3.842873in}{2.132750in}}%
\pgfpathlineto{\pgfqpoint{3.840062in}{2.132312in}}%
\pgfpathlineto{\pgfqpoint{3.837252in}{2.132565in}}%
\pgfpathlineto{\pgfqpoint{3.834441in}{2.132775in}}%
\pgfpathlineto{\pgfqpoint{3.831630in}{2.132244in}}%
\pgfpathlineto{\pgfqpoint{3.828820in}{2.132395in}}%
\pgfpathlineto{\pgfqpoint{3.826009in}{2.132625in}}%
\pgfpathlineto{\pgfqpoint{3.823198in}{2.132418in}}%
\pgfpathlineto{\pgfqpoint{3.820388in}{2.132413in}}%
\pgfpathlineto{\pgfqpoint{3.817577in}{2.131429in}}%
\pgfpathlineto{\pgfqpoint{3.814766in}{2.131328in}}%
\pgfpathlineto{\pgfqpoint{3.811956in}{2.130321in}}%
\pgfpathlineto{\pgfqpoint{3.809145in}{2.130364in}}%
\pgfpathlineto{\pgfqpoint{3.806334in}{2.128121in}}%
\pgfpathlineto{\pgfqpoint{3.803524in}{2.127772in}}%
\pgfpathlineto{\pgfqpoint{3.800713in}{2.128015in}}%
\pgfpathlineto{\pgfqpoint{3.797902in}{2.127875in}}%
\pgfpathlineto{\pgfqpoint{3.795092in}{2.128073in}}%
\pgfpathlineto{\pgfqpoint{3.792281in}{2.128310in}}%
\pgfpathlineto{\pgfqpoint{3.789470in}{2.121937in}}%
\pgfpathlineto{\pgfqpoint{3.786659in}{2.121880in}}%
\pgfpathlineto{\pgfqpoint{3.783849in}{2.122068in}}%
\pgfpathlineto{\pgfqpoint{3.781038in}{2.122258in}}%
\pgfpathlineto{\pgfqpoint{3.778227in}{2.122451in}}%
\pgfpathlineto{\pgfqpoint{3.775417in}{2.122188in}}%
\pgfpathlineto{\pgfqpoint{3.772606in}{2.122080in}}%
\pgfpathlineto{\pgfqpoint{3.769795in}{2.122316in}}%
\pgfpathlineto{\pgfqpoint{3.766985in}{2.122572in}}%
\pgfpathlineto{\pgfqpoint{3.764174in}{2.122319in}}%
\pgfpathlineto{\pgfqpoint{3.761363in}{2.122315in}}%
\pgfpathlineto{\pgfqpoint{3.758553in}{2.122260in}}%
\pgfpathlineto{\pgfqpoint{3.755742in}{2.122195in}}%
\pgfpathlineto{\pgfqpoint{3.752931in}{2.122002in}}%
\pgfpathlineto{\pgfqpoint{3.750121in}{2.122208in}}%
\pgfpathlineto{\pgfqpoint{3.747310in}{2.122240in}}%
\pgfpathlineto{\pgfqpoint{3.744499in}{2.122435in}}%
\pgfpathlineto{\pgfqpoint{3.741688in}{2.122498in}}%
\pgfpathlineto{\pgfqpoint{3.738878in}{2.122371in}}%
\pgfpathlineto{\pgfqpoint{3.736067in}{2.122606in}}%
\pgfpathlineto{\pgfqpoint{3.733256in}{2.122833in}}%
\pgfpathlineto{\pgfqpoint{3.730446in}{2.122861in}}%
\pgfpathlineto{\pgfqpoint{3.727635in}{2.123027in}}%
\pgfpathlineto{\pgfqpoint{3.724824in}{2.123283in}}%
\pgfpathlineto{\pgfqpoint{3.722014in}{2.123050in}}%
\pgfpathlineto{\pgfqpoint{3.719203in}{2.123231in}}%
\pgfpathlineto{\pgfqpoint{3.716392in}{2.123171in}}%
\pgfpathlineto{\pgfqpoint{3.713582in}{2.122727in}}%
\pgfpathlineto{\pgfqpoint{3.710771in}{2.122892in}}%
\pgfpathlineto{\pgfqpoint{3.707960in}{2.122639in}}%
\pgfpathlineto{\pgfqpoint{3.705150in}{2.121985in}}%
\pgfpathlineto{\pgfqpoint{3.702339in}{2.121972in}}%
\pgfpathlineto{\pgfqpoint{3.699528in}{2.121757in}}%
\pgfpathlineto{\pgfqpoint{3.696717in}{2.121932in}}%
\pgfpathlineto{\pgfqpoint{3.693907in}{2.122091in}}%
\pgfpathlineto{\pgfqpoint{3.691096in}{2.122352in}}%
\pgfpathlineto{\pgfqpoint{3.688285in}{2.122585in}}%
\pgfpathlineto{\pgfqpoint{3.685475in}{2.121642in}}%
\pgfpathlineto{\pgfqpoint{3.682664in}{2.121886in}}%
\pgfpathlineto{\pgfqpoint{3.679853in}{2.122115in}}%
\pgfpathlineto{\pgfqpoint{3.677043in}{2.122044in}}%
\pgfpathlineto{\pgfqpoint{3.674232in}{2.122279in}}%
\pgfpathlineto{\pgfqpoint{3.671421in}{2.122448in}}%
\pgfpathlineto{\pgfqpoint{3.668611in}{2.122703in}}%
\pgfpathlineto{\pgfqpoint{3.665800in}{2.122964in}}%
\pgfpathlineto{\pgfqpoint{3.662989in}{2.123205in}}%
\pgfpathlineto{\pgfqpoint{3.660179in}{2.123427in}}%
\pgfpathlineto{\pgfqpoint{3.657368in}{2.123689in}}%
\pgfpathlineto{\pgfqpoint{3.654557in}{2.123167in}}%
\pgfpathlineto{\pgfqpoint{3.651746in}{2.123408in}}%
\pgfpathlineto{\pgfqpoint{3.648936in}{2.123386in}}%
\pgfpathlineto{\pgfqpoint{3.646125in}{2.123642in}}%
\pgfpathlineto{\pgfqpoint{3.643314in}{2.123874in}}%
\pgfpathlineto{\pgfqpoint{3.640504in}{2.124138in}}%
\pgfpathlineto{\pgfqpoint{3.637693in}{2.123638in}}%
\pgfpathlineto{\pgfqpoint{3.634882in}{2.123893in}}%
\pgfpathlineto{\pgfqpoint{3.632072in}{2.124161in}}%
\pgfpathlineto{\pgfqpoint{3.629261in}{2.124085in}}%
\pgfpathlineto{\pgfqpoint{3.626450in}{2.124201in}}%
\pgfpathlineto{\pgfqpoint{3.623640in}{2.122528in}}%
\pgfpathlineto{\pgfqpoint{3.620829in}{2.122042in}}%
\pgfpathlineto{\pgfqpoint{3.618018in}{2.122264in}}%
\pgfpathlineto{\pgfqpoint{3.615208in}{2.122532in}}%
\pgfpathlineto{\pgfqpoint{3.612397in}{2.122182in}}%
\pgfpathlineto{\pgfqpoint{3.609586in}{2.122442in}}%
\pgfpathlineto{\pgfqpoint{3.606776in}{2.122240in}}%
\pgfpathlineto{\pgfqpoint{3.603965in}{2.122275in}}%
\pgfpathlineto{\pgfqpoint{3.601154in}{2.122150in}}%
\pgfpathlineto{\pgfqpoint{3.598343in}{2.122388in}}%
\pgfpathlineto{\pgfqpoint{3.595533in}{2.122644in}}%
\pgfpathlineto{\pgfqpoint{3.592722in}{2.122505in}}%
\pgfpathlineto{\pgfqpoint{3.589911in}{2.122532in}}%
\pgfpathlineto{\pgfqpoint{3.587101in}{2.122281in}}%
\pgfpathlineto{\pgfqpoint{3.584290in}{2.122522in}}%
\pgfpathlineto{\pgfqpoint{3.581479in}{2.122755in}}%
\pgfpathlineto{\pgfqpoint{3.578669in}{2.122587in}}%
\pgfpathlineto{\pgfqpoint{3.575858in}{2.122806in}}%
\pgfpathlineto{\pgfqpoint{3.573047in}{2.122814in}}%
\pgfpathlineto{\pgfqpoint{3.570237in}{2.122538in}}%
\pgfpathlineto{\pgfqpoint{3.567426in}{2.122798in}}%
\pgfpathlineto{\pgfqpoint{3.564615in}{2.122156in}}%
\pgfpathlineto{\pgfqpoint{3.561805in}{2.122147in}}%
\pgfpathlineto{\pgfqpoint{3.558994in}{2.122292in}}%
\pgfpathlineto{\pgfqpoint{3.556183in}{2.121985in}}%
\pgfpathlineto{\pgfqpoint{3.553372in}{2.121478in}}%
\pgfpathlineto{\pgfqpoint{3.550562in}{2.119294in}}%
\pgfpathlineto{\pgfqpoint{3.547751in}{2.119391in}}%
\pgfpathlineto{\pgfqpoint{3.544940in}{2.119642in}}%
\pgfpathlineto{\pgfqpoint{3.542130in}{2.119622in}}%
\pgfpathlineto{\pgfqpoint{3.539319in}{2.119893in}}%
\pgfpathlineto{\pgfqpoint{3.536508in}{2.119693in}}%
\pgfpathlineto{\pgfqpoint{3.533698in}{2.119824in}}%
\pgfpathlineto{\pgfqpoint{3.530887in}{2.120068in}}%
\pgfpathlineto{\pgfqpoint{3.528076in}{2.120158in}}%
\pgfpathlineto{\pgfqpoint{3.525266in}{2.120234in}}%
\pgfpathlineto{\pgfqpoint{3.522455in}{2.120481in}}%
\pgfpathlineto{\pgfqpoint{3.519644in}{2.120548in}}%
\pgfpathlineto{\pgfqpoint{3.516834in}{2.120558in}}%
\pgfpathlineto{\pgfqpoint{3.514023in}{2.120763in}}%
\pgfpathlineto{\pgfqpoint{3.511212in}{2.120255in}}%
\pgfpathlineto{\pgfqpoint{3.508401in}{2.120354in}}%
\pgfpathlineto{\pgfqpoint{3.505591in}{2.120283in}}%
\pgfpathlineto{\pgfqpoint{3.502780in}{2.120279in}}%
\pgfpathlineto{\pgfqpoint{3.499969in}{2.119604in}}%
\pgfpathlineto{\pgfqpoint{3.497159in}{2.119743in}}%
\pgfpathlineto{\pgfqpoint{3.494348in}{2.119972in}}%
\pgfpathlineto{\pgfqpoint{3.491537in}{2.119167in}}%
\pgfpathlineto{\pgfqpoint{3.488727in}{2.114980in}}%
\pgfpathlineto{\pgfqpoint{3.485916in}{2.114975in}}%
\pgfpathlineto{\pgfqpoint{3.483105in}{2.113693in}}%
\pgfpathlineto{\pgfqpoint{3.480295in}{2.112615in}}%
\pgfpathlineto{\pgfqpoint{3.477484in}{2.112892in}}%
\pgfpathlineto{\pgfqpoint{3.474673in}{2.113144in}}%
\pgfpathlineto{\pgfqpoint{3.471863in}{2.113371in}}%
\pgfpathlineto{\pgfqpoint{3.469052in}{2.113569in}}%
\pgfpathlineto{\pgfqpoint{3.466241in}{2.113737in}}%
\pgfpathlineto{\pgfqpoint{3.463430in}{2.113868in}}%
\pgfpathlineto{\pgfqpoint{3.460620in}{2.114031in}}%
\pgfpathlineto{\pgfqpoint{3.457809in}{2.114221in}}%
\pgfpathlineto{\pgfqpoint{3.454998in}{2.114423in}}%
\pgfpathlineto{\pgfqpoint{3.452188in}{2.114453in}}%
\pgfpathlineto{\pgfqpoint{3.449377in}{2.114500in}}%
\pgfpathlineto{\pgfqpoint{3.446566in}{2.114778in}}%
\pgfpathlineto{\pgfqpoint{3.443756in}{2.115058in}}%
\pgfpathlineto{\pgfqpoint{3.440945in}{2.115184in}}%
\pgfpathlineto{\pgfqpoint{3.438134in}{2.115379in}}%
\pgfpathlineto{\pgfqpoint{3.435324in}{2.115526in}}%
\pgfpathlineto{\pgfqpoint{3.432513in}{2.115168in}}%
\pgfpathlineto{\pgfqpoint{3.429702in}{2.115066in}}%
\pgfpathlineto{\pgfqpoint{3.426892in}{2.115283in}}%
\pgfpathlineto{\pgfqpoint{3.424081in}{2.112692in}}%
\pgfpathlineto{\pgfqpoint{3.421270in}{2.112967in}}%
\pgfpathlineto{\pgfqpoint{3.418459in}{2.113222in}}%
\pgfpathlineto{\pgfqpoint{3.415649in}{2.113485in}}%
\pgfpathlineto{\pgfqpoint{3.412838in}{2.112520in}}%
\pgfpathlineto{\pgfqpoint{3.410027in}{2.112678in}}%
\pgfpathlineto{\pgfqpoint{3.407217in}{2.112733in}}%
\pgfpathlineto{\pgfqpoint{3.404406in}{2.112995in}}%
\pgfpathlineto{\pgfqpoint{3.401595in}{2.113032in}}%
\pgfpathlineto{\pgfqpoint{3.398785in}{2.112711in}}%
\pgfpathlineto{\pgfqpoint{3.395974in}{2.112123in}}%
\pgfpathlineto{\pgfqpoint{3.393163in}{2.112270in}}%
\pgfpathlineto{\pgfqpoint{3.390353in}{2.112425in}}%
\pgfpathlineto{\pgfqpoint{3.387542in}{2.112712in}}%
\pgfpathlineto{\pgfqpoint{3.384731in}{2.112982in}}%
\pgfpathlineto{\pgfqpoint{3.381921in}{2.113103in}}%
\pgfpathlineto{\pgfqpoint{3.379110in}{2.113057in}}%
\pgfpathlineto{\pgfqpoint{3.376299in}{2.113156in}}%
\pgfpathlineto{\pgfqpoint{3.373489in}{2.112644in}}%
\pgfpathlineto{\pgfqpoint{3.370678in}{2.112860in}}%
\pgfpathlineto{\pgfqpoint{3.367867in}{2.113146in}}%
\pgfpathlineto{\pgfqpoint{3.365056in}{2.113427in}}%
\pgfpathlineto{\pgfqpoint{3.362246in}{2.113578in}}%
\pgfpathlineto{\pgfqpoint{3.359435in}{2.113698in}}%
\pgfpathlineto{\pgfqpoint{3.356624in}{2.113936in}}%
\pgfpathlineto{\pgfqpoint{3.353814in}{2.113870in}}%
\pgfpathlineto{\pgfqpoint{3.351003in}{2.114158in}}%
\pgfpathlineto{\pgfqpoint{3.348192in}{2.114265in}}%
\pgfpathlineto{\pgfqpoint{3.345382in}{2.114519in}}%
\pgfpathlineto{\pgfqpoint{3.342571in}{2.114811in}}%
\pgfpathlineto{\pgfqpoint{3.339760in}{2.115063in}}%
\pgfpathlineto{\pgfqpoint{3.336950in}{2.114152in}}%
\pgfpathlineto{\pgfqpoint{3.334139in}{2.114327in}}%
\pgfpathlineto{\pgfqpoint{3.331328in}{2.114593in}}%
\pgfpathlineto{\pgfqpoint{3.328518in}{2.114807in}}%
\pgfpathlineto{\pgfqpoint{3.325707in}{2.115067in}}%
\pgfpathlineto{\pgfqpoint{3.322896in}{2.115340in}}%
\pgfpathlineto{\pgfqpoint{3.320085in}{2.115611in}}%
\pgfpathlineto{\pgfqpoint{3.317275in}{2.115721in}}%
\pgfpathlineto{\pgfqpoint{3.314464in}{2.115948in}}%
\pgfpathlineto{\pgfqpoint{3.311653in}{2.116192in}}%
\pgfpathlineto{\pgfqpoint{3.308843in}{2.115989in}}%
\pgfpathlineto{\pgfqpoint{3.306032in}{2.116150in}}%
\pgfpathlineto{\pgfqpoint{3.303221in}{2.116339in}}%
\pgfpathlineto{\pgfqpoint{3.300411in}{2.116638in}}%
\pgfpathlineto{\pgfqpoint{3.297600in}{2.116939in}}%
\pgfpathlineto{\pgfqpoint{3.294789in}{2.117156in}}%
\pgfpathlineto{\pgfqpoint{3.291979in}{2.117272in}}%
\pgfpathlineto{\pgfqpoint{3.289168in}{2.117573in}}%
\pgfpathlineto{\pgfqpoint{3.286357in}{2.117023in}}%
\pgfpathlineto{\pgfqpoint{3.283547in}{2.117268in}}%
\pgfpathlineto{\pgfqpoint{3.280736in}{2.117570in}}%
\pgfpathlineto{\pgfqpoint{3.277925in}{2.117872in}}%
\pgfpathlineto{\pgfqpoint{3.275114in}{2.115001in}}%
\pgfpathlineto{\pgfqpoint{3.272304in}{2.114792in}}%
\pgfpathlineto{\pgfqpoint{3.269493in}{2.115094in}}%
\pgfpathlineto{\pgfqpoint{3.266682in}{2.115225in}}%
\pgfpathlineto{\pgfqpoint{3.263872in}{2.115520in}}%
\pgfpathlineto{\pgfqpoint{3.261061in}{2.115823in}}%
\pgfpathlineto{\pgfqpoint{3.258250in}{2.115860in}}%
\pgfpathlineto{\pgfqpoint{3.255440in}{2.115895in}}%
\pgfpathlineto{\pgfqpoint{3.252629in}{2.116201in}}%
\pgfpathlineto{\pgfqpoint{3.249818in}{2.116507in}}%
\pgfpathlineto{\pgfqpoint{3.247008in}{2.116810in}}%
\pgfpathlineto{\pgfqpoint{3.244197in}{2.117115in}}%
\pgfpathlineto{\pgfqpoint{3.241386in}{2.114532in}}%
\pgfpathlineto{\pgfqpoint{3.238576in}{2.114516in}}%
\pgfpathlineto{\pgfqpoint{3.235765in}{2.114721in}}%
\pgfpathlineto{\pgfqpoint{3.232954in}{2.114833in}}%
\pgfpathlineto{\pgfqpoint{3.230143in}{2.115106in}}%
\pgfpathlineto{\pgfqpoint{3.227333in}{2.115332in}}%
\pgfpathlineto{\pgfqpoint{3.224522in}{2.115616in}}%
\pgfpathlineto{\pgfqpoint{3.221711in}{2.115883in}}%
\pgfpathlineto{\pgfqpoint{3.218901in}{2.116172in}}%
\pgfpathlineto{\pgfqpoint{3.216090in}{2.116454in}}%
\pgfpathlineto{\pgfqpoint{3.213279in}{2.116399in}}%
\pgfpathlineto{\pgfqpoint{3.210469in}{2.116408in}}%
\pgfpathlineto{\pgfqpoint{3.207658in}{2.116697in}}%
\pgfpathlineto{\pgfqpoint{3.204847in}{2.116936in}}%
\pgfpathlineto{\pgfqpoint{3.202037in}{2.117244in}}%
\pgfpathlineto{\pgfqpoint{3.199226in}{2.117554in}}%
\pgfpathlineto{\pgfqpoint{3.196415in}{2.117794in}}%
\pgfpathlineto{\pgfqpoint{3.193605in}{2.118078in}}%
\pgfpathlineto{\pgfqpoint{3.190794in}{2.118377in}}%
\pgfpathlineto{\pgfqpoint{3.187983in}{2.118364in}}%
\pgfpathlineto{\pgfqpoint{3.185173in}{2.118633in}}%
\pgfpathlineto{\pgfqpoint{3.182362in}{2.118949in}}%
\pgfpathlineto{\pgfqpoint{3.179551in}{2.118971in}}%
\pgfpathlineto{\pgfqpoint{3.176740in}{2.119267in}}%
\pgfpathlineto{\pgfqpoint{3.173930in}{2.119150in}}%
\pgfpathlineto{\pgfqpoint{3.171119in}{2.119278in}}%
\pgfpathlineto{\pgfqpoint{3.168308in}{2.118995in}}%
\pgfpathlineto{\pgfqpoint{3.165498in}{2.119123in}}%
\pgfpathlineto{\pgfqpoint{3.162687in}{2.118644in}}%
\pgfpathlineto{\pgfqpoint{3.159876in}{2.118944in}}%
\pgfpathlineto{\pgfqpoint{3.157066in}{2.118806in}}%
\pgfpathlineto{\pgfqpoint{3.154255in}{2.118874in}}%
\pgfpathlineto{\pgfqpoint{3.151444in}{2.118979in}}%
\pgfpathlineto{\pgfqpoint{3.148634in}{2.119234in}}%
\pgfpathlineto{\pgfqpoint{3.145823in}{2.119556in}}%
\pgfpathlineto{\pgfqpoint{3.143012in}{2.119837in}}%
\pgfpathlineto{\pgfqpoint{3.140202in}{2.118777in}}%
\pgfpathlineto{\pgfqpoint{3.137391in}{2.119082in}}%
\pgfpathlineto{\pgfqpoint{3.134580in}{2.119309in}}%
\pgfpathlineto{\pgfqpoint{3.131769in}{2.119553in}}%
\pgfpathlineto{\pgfqpoint{3.128959in}{2.119861in}}%
\pgfpathlineto{\pgfqpoint{3.126148in}{2.120143in}}%
\pgfpathlineto{\pgfqpoint{3.123337in}{2.119898in}}%
\pgfpathlineto{\pgfqpoint{3.120527in}{2.120181in}}%
\pgfpathlineto{\pgfqpoint{3.117716in}{2.120507in}}%
\pgfpathlineto{\pgfqpoint{3.114905in}{2.120687in}}%
\pgfpathlineto{\pgfqpoint{3.112095in}{2.121015in}}%
\pgfpathlineto{\pgfqpoint{3.109284in}{2.120513in}}%
\pgfpathlineto{\pgfqpoint{3.106473in}{2.120358in}}%
\pgfpathlineto{\pgfqpoint{3.103663in}{2.120682in}}%
\pgfpathlineto{\pgfqpoint{3.100852in}{2.121007in}}%
\pgfpathlineto{\pgfqpoint{3.098041in}{2.121166in}}%
\pgfpathlineto{\pgfqpoint{3.095231in}{2.120056in}}%
\pgfpathlineto{\pgfqpoint{3.092420in}{2.119473in}}%
\pgfpathlineto{\pgfqpoint{3.089609in}{2.119033in}}%
\pgfpathlineto{\pgfqpoint{3.086798in}{2.119289in}}%
\pgfpathlineto{\pgfqpoint{3.083988in}{2.117999in}}%
\pgfpathlineto{\pgfqpoint{3.081177in}{2.118106in}}%
\pgfpathlineto{\pgfqpoint{3.078366in}{2.118177in}}%
\pgfpathlineto{\pgfqpoint{3.075556in}{2.118109in}}%
\pgfpathlineto{\pgfqpoint{3.072745in}{2.118439in}}%
\pgfpathlineto{\pgfqpoint{3.069934in}{2.118671in}}%
\pgfpathlineto{\pgfqpoint{3.067124in}{2.118940in}}%
\pgfpathlineto{\pgfqpoint{3.064313in}{2.119195in}}%
\pgfpathlineto{\pgfqpoint{3.061502in}{2.119284in}}%
\pgfpathlineto{\pgfqpoint{3.058692in}{2.119377in}}%
\pgfpathlineto{\pgfqpoint{3.055881in}{2.119690in}}%
\pgfpathlineto{\pgfqpoint{3.053070in}{2.119692in}}%
\pgfpathlineto{\pgfqpoint{3.050260in}{2.119921in}}%
\pgfpathlineto{\pgfqpoint{3.047449in}{2.120241in}}%
\pgfpathlineto{\pgfqpoint{3.044638in}{2.120444in}}%
\pgfpathlineto{\pgfqpoint{3.041827in}{2.120249in}}%
\pgfpathlineto{\pgfqpoint{3.039017in}{2.120052in}}%
\pgfpathlineto{\pgfqpoint{3.036206in}{2.120390in}}%
\pgfpathlineto{\pgfqpoint{3.033395in}{2.120299in}}%
\pgfpathlineto{\pgfqpoint{3.030585in}{2.120249in}}%
\pgfpathlineto{\pgfqpoint{3.027774in}{2.120572in}}%
\pgfpathlineto{\pgfqpoint{3.024963in}{2.120906in}}%
\pgfpathlineto{\pgfqpoint{3.022153in}{2.121247in}}%
\pgfpathlineto{\pgfqpoint{3.019342in}{2.121586in}}%
\pgfpathlineto{\pgfqpoint{3.016531in}{2.121615in}}%
\pgfpathlineto{\pgfqpoint{3.013721in}{2.121824in}}%
\pgfpathlineto{\pgfqpoint{3.010910in}{2.121819in}}%
\pgfpathlineto{\pgfqpoint{3.008099in}{2.122139in}}%
\pgfpathlineto{\pgfqpoint{3.005289in}{2.121534in}}%
\pgfpathlineto{\pgfqpoint{3.002478in}{2.120708in}}%
\pgfpathlineto{\pgfqpoint{2.999667in}{2.120969in}}%
\pgfpathlineto{\pgfqpoint{2.996856in}{2.121275in}}%
\pgfpathlineto{\pgfqpoint{2.994046in}{2.121066in}}%
\pgfpathlineto{\pgfqpoint{2.991235in}{2.121232in}}%
\pgfpathlineto{\pgfqpoint{2.988424in}{2.121558in}}%
\pgfpathlineto{\pgfqpoint{2.985614in}{2.121770in}}%
\pgfpathlineto{\pgfqpoint{2.982803in}{2.122088in}}%
\pgfpathlineto{\pgfqpoint{2.979992in}{2.122140in}}%
\pgfpathlineto{\pgfqpoint{2.977182in}{2.122470in}}%
\pgfpathlineto{\pgfqpoint{2.974371in}{2.122615in}}%
\pgfpathlineto{\pgfqpoint{2.971560in}{2.122722in}}%
\pgfpathlineto{\pgfqpoint{2.968750in}{2.123029in}}%
\pgfpathlineto{\pgfqpoint{2.965939in}{2.123255in}}%
\pgfpathlineto{\pgfqpoint{2.963128in}{2.123562in}}%
\pgfpathlineto{\pgfqpoint{2.960318in}{2.123530in}}%
\pgfpathlineto{\pgfqpoint{2.957507in}{2.123807in}}%
\pgfpathlineto{\pgfqpoint{2.954696in}{2.123784in}}%
\pgfpathlineto{\pgfqpoint{2.951886in}{2.123890in}}%
\pgfpathlineto{\pgfqpoint{2.949075in}{2.124188in}}%
\pgfpathlineto{\pgfqpoint{2.946264in}{2.124504in}}%
\pgfpathlineto{\pgfqpoint{2.943453in}{2.124725in}}%
\pgfpathlineto{\pgfqpoint{2.940643in}{2.125072in}}%
\pgfpathlineto{\pgfqpoint{2.937832in}{2.125430in}}%
\pgfpathlineto{\pgfqpoint{2.935021in}{2.125014in}}%
\pgfpathlineto{\pgfqpoint{2.932211in}{2.125356in}}%
\pgfpathlineto{\pgfqpoint{2.929400in}{2.125700in}}%
\pgfpathlineto{\pgfqpoint{2.926589in}{2.125713in}}%
\pgfpathlineto{\pgfqpoint{2.923779in}{2.125982in}}%
\pgfpathlineto{\pgfqpoint{2.920968in}{2.124689in}}%
\pgfpathlineto{\pgfqpoint{2.918157in}{2.124845in}}%
\pgfpathlineto{\pgfqpoint{2.915347in}{2.125154in}}%
\pgfpathlineto{\pgfqpoint{2.912536in}{2.124475in}}%
\pgfpathlineto{\pgfqpoint{2.909725in}{2.107557in}}%
\pgfpathlineto{\pgfqpoint{2.906915in}{2.107883in}}%
\pgfpathlineto{\pgfqpoint{2.904104in}{2.107512in}}%
\pgfpathlineto{\pgfqpoint{2.901293in}{2.107823in}}%
\pgfpathlineto{\pgfqpoint{2.898482in}{2.108173in}}%
\pgfpathlineto{\pgfqpoint{2.895672in}{2.107773in}}%
\pgfpathlineto{\pgfqpoint{2.892861in}{2.108073in}}%
\pgfpathlineto{\pgfqpoint{2.890050in}{2.106780in}}%
\pgfpathlineto{\pgfqpoint{2.887240in}{2.106787in}}%
\pgfpathlineto{\pgfqpoint{2.884429in}{2.106452in}}%
\pgfpathlineto{\pgfqpoint{2.881618in}{2.106198in}}%
\pgfpathlineto{\pgfqpoint{2.878808in}{2.106484in}}%
\pgfpathlineto{\pgfqpoint{2.875997in}{2.106834in}}%
\pgfpathlineto{\pgfqpoint{2.873186in}{2.107184in}}%
\pgfpathlineto{\pgfqpoint{2.870376in}{2.107444in}}%
\pgfpathlineto{\pgfqpoint{2.867565in}{2.107332in}}%
\pgfpathlineto{\pgfqpoint{2.864754in}{2.106542in}}%
\pgfpathlineto{\pgfqpoint{2.861944in}{2.106643in}}%
\pgfpathlineto{\pgfqpoint{2.859133in}{2.106970in}}%
\pgfpathlineto{\pgfqpoint{2.856322in}{2.106884in}}%
\pgfpathlineto{\pgfqpoint{2.853511in}{2.107218in}}%
\pgfpathlineto{\pgfqpoint{2.850701in}{2.107396in}}%
\pgfpathlineto{\pgfqpoint{2.847890in}{2.107269in}}%
\pgfpathlineto{\pgfqpoint{2.845079in}{2.107632in}}%
\pgfpathlineto{\pgfqpoint{2.842269in}{2.107781in}}%
\pgfpathlineto{\pgfqpoint{2.839458in}{2.107902in}}%
\pgfpathlineto{\pgfqpoint{2.836647in}{2.107897in}}%
\pgfpathlineto{\pgfqpoint{2.833837in}{2.108261in}}%
\pgfpathlineto{\pgfqpoint{2.831026in}{2.108572in}}%
\pgfpathlineto{\pgfqpoint{2.828215in}{2.108903in}}%
\pgfpathlineto{\pgfqpoint{2.825405in}{2.109115in}}%
\pgfpathlineto{\pgfqpoint{2.822594in}{2.109472in}}%
\pgfpathlineto{\pgfqpoint{2.819783in}{2.109289in}}%
\pgfpathlineto{\pgfqpoint{2.816973in}{2.109544in}}%
\pgfpathlineto{\pgfqpoint{2.814162in}{2.109913in}}%
\pgfpathlineto{\pgfqpoint{2.811351in}{2.110264in}}%
\pgfpathlineto{\pgfqpoint{2.808540in}{2.110104in}}%
\pgfpathlineto{\pgfqpoint{2.805730in}{2.110474in}}%
\pgfpathlineto{\pgfqpoint{2.802919in}{2.110569in}}%
\pgfpathlineto{\pgfqpoint{2.800108in}{2.110896in}}%
\pgfpathlineto{\pgfqpoint{2.797298in}{2.111263in}}%
\pgfpathlineto{\pgfqpoint{2.794487in}{2.111567in}}%
\pgfpathlineto{\pgfqpoint{2.791676in}{2.111515in}}%
\pgfpathlineto{\pgfqpoint{2.788866in}{2.111863in}}%
\pgfpathlineto{\pgfqpoint{2.786055in}{2.112170in}}%
\pgfpathlineto{\pgfqpoint{2.783244in}{2.112454in}}%
\pgfpathlineto{\pgfqpoint{2.780434in}{2.112700in}}%
\pgfpathlineto{\pgfqpoint{2.777623in}{2.113036in}}%
\pgfpathlineto{\pgfqpoint{2.774812in}{2.113332in}}%
\pgfpathlineto{\pgfqpoint{2.772002in}{2.113696in}}%
\pgfpathlineto{\pgfqpoint{2.769191in}{2.113862in}}%
\pgfpathlineto{\pgfqpoint{2.766380in}{2.114117in}}%
\pgfpathlineto{\pgfqpoint{2.763570in}{2.113393in}}%
\pgfpathlineto{\pgfqpoint{2.760759in}{2.113723in}}%
\pgfpathlineto{\pgfqpoint{2.757948in}{2.113909in}}%
\pgfpathlineto{\pgfqpoint{2.755137in}{2.113690in}}%
\pgfpathlineto{\pgfqpoint{2.752327in}{2.114045in}}%
\pgfpathlineto{\pgfqpoint{2.749516in}{2.114372in}}%
\pgfpathlineto{\pgfqpoint{2.746705in}{2.114490in}}%
\pgfpathlineto{\pgfqpoint{2.743895in}{2.114870in}}%
\pgfpathlineto{\pgfqpoint{2.741084in}{2.115195in}}%
\pgfpathlineto{\pgfqpoint{2.738273in}{2.115580in}}%
\pgfpathlineto{\pgfqpoint{2.735463in}{2.115882in}}%
\pgfpathlineto{\pgfqpoint{2.732652in}{2.116121in}}%
\pgfpathlineto{\pgfqpoint{2.729841in}{2.116377in}}%
\pgfpathlineto{\pgfqpoint{2.727031in}{2.116650in}}%
\pgfpathlineto{\pgfqpoint{2.724220in}{2.117039in}}%
\pgfpathlineto{\pgfqpoint{2.721409in}{2.117421in}}%
\pgfpathlineto{\pgfqpoint{2.718599in}{2.116099in}}%
\pgfpathlineto{\pgfqpoint{2.715788in}{2.116224in}}%
\pgfpathlineto{\pgfqpoint{2.712977in}{2.116572in}}%
\pgfpathlineto{\pgfqpoint{2.710166in}{2.116581in}}%
\pgfpathlineto{\pgfqpoint{2.707356in}{2.116979in}}%
\pgfpathlineto{\pgfqpoint{2.704545in}{2.116910in}}%
\pgfpathlineto{\pgfqpoint{2.701734in}{2.116829in}}%
\pgfpathlineto{\pgfqpoint{2.698924in}{2.117069in}}%
\pgfpathlineto{\pgfqpoint{2.696113in}{2.117334in}}%
\pgfpathlineto{\pgfqpoint{2.693302in}{2.116687in}}%
\pgfpathlineto{\pgfqpoint{2.690492in}{2.116822in}}%
\pgfpathlineto{\pgfqpoint{2.687681in}{2.117212in}}%
\pgfpathlineto{\pgfqpoint{2.684870in}{2.117449in}}%
\pgfpathlineto{\pgfqpoint{2.682060in}{2.117851in}}%
\pgfpathlineto{\pgfqpoint{2.679249in}{2.118182in}}%
\pgfpathlineto{\pgfqpoint{2.676438in}{2.118264in}}%
\pgfpathlineto{\pgfqpoint{2.673628in}{2.118548in}}%
\pgfpathlineto{\pgfqpoint{2.670817in}{2.118065in}}%
\pgfpathlineto{\pgfqpoint{2.668006in}{2.118193in}}%
\pgfpathlineto{\pgfqpoint{2.665195in}{2.118510in}}%
\pgfpathlineto{\pgfqpoint{2.662385in}{2.118877in}}%
\pgfpathlineto{\pgfqpoint{2.659574in}{2.119041in}}%
\pgfpathlineto{\pgfqpoint{2.656763in}{2.119444in}}%
\pgfpathlineto{\pgfqpoint{2.653953in}{2.119806in}}%
\pgfpathlineto{\pgfqpoint{2.651142in}{2.120217in}}%
\pgfpathlineto{\pgfqpoint{2.648331in}{2.120631in}}%
\pgfpathlineto{\pgfqpoint{2.645521in}{2.120944in}}%
\pgfpathlineto{\pgfqpoint{2.642710in}{2.121233in}}%
\pgfpathlineto{\pgfqpoint{2.639899in}{2.121652in}}%
\pgfpathlineto{\pgfqpoint{2.637089in}{2.122054in}}%
\pgfpathlineto{\pgfqpoint{2.634278in}{2.122473in}}%
\pgfpathlineto{\pgfqpoint{2.631467in}{2.122889in}}%
\pgfpathlineto{\pgfqpoint{2.628657in}{2.122985in}}%
\pgfpathlineto{\pgfqpoint{2.625846in}{2.123386in}}%
\pgfpathlineto{\pgfqpoint{2.623035in}{2.123649in}}%
\pgfpathlineto{\pgfqpoint{2.620224in}{2.123919in}}%
\pgfpathlineto{\pgfqpoint{2.617414in}{2.124276in}}%
\pgfpathlineto{\pgfqpoint{2.614603in}{2.124661in}}%
\pgfpathlineto{\pgfqpoint{2.611792in}{2.125090in}}%
\pgfpathlineto{\pgfqpoint{2.608982in}{2.125445in}}%
\pgfpathlineto{\pgfqpoint{2.606171in}{2.125707in}}%
\pgfpathlineto{\pgfqpoint{2.603360in}{2.126129in}}%
\pgfpathlineto{\pgfqpoint{2.600550in}{2.126027in}}%
\pgfpathlineto{\pgfqpoint{2.597739in}{2.125821in}}%
\pgfpathlineto{\pgfqpoint{2.594928in}{2.126248in}}%
\pgfpathlineto{\pgfqpoint{2.592118in}{2.126205in}}%
\pgfpathlineto{\pgfqpoint{2.589307in}{2.126575in}}%
\pgfpathlineto{\pgfqpoint{2.586496in}{2.126886in}}%
\pgfpathlineto{\pgfqpoint{2.583686in}{2.126779in}}%
\pgfpathlineto{\pgfqpoint{2.580875in}{2.127152in}}%
\pgfpathlineto{\pgfqpoint{2.578064in}{2.127580in}}%
\pgfpathlineto{\pgfqpoint{2.575253in}{2.127918in}}%
\pgfpathlineto{\pgfqpoint{2.572443in}{2.128177in}}%
\pgfpathlineto{\pgfqpoint{2.569632in}{2.128569in}}%
\pgfpathlineto{\pgfqpoint{2.566821in}{2.128527in}}%
\pgfpathlineto{\pgfqpoint{2.564011in}{2.128055in}}%
\pgfpathlineto{\pgfqpoint{2.561200in}{2.128485in}}%
\pgfpathlineto{\pgfqpoint{2.558389in}{2.128247in}}%
\pgfpathlineto{\pgfqpoint{2.555579in}{2.128682in}}%
\pgfpathlineto{\pgfqpoint{2.552768in}{2.128134in}}%
\pgfpathlineto{\pgfqpoint{2.549957in}{2.128527in}}%
\pgfpathlineto{\pgfqpoint{2.547147in}{2.128711in}}%
\pgfpathlineto{\pgfqpoint{2.544336in}{2.128534in}}%
\pgfpathlineto{\pgfqpoint{2.541525in}{2.124945in}}%
\pgfpathlineto{\pgfqpoint{2.538715in}{2.125262in}}%
\pgfpathlineto{\pgfqpoint{2.535904in}{2.125712in}}%
\pgfpathlineto{\pgfqpoint{2.533093in}{2.125982in}}%
\pgfpathlineto{\pgfqpoint{2.530283in}{2.126185in}}%
\pgfpathlineto{\pgfqpoint{2.527472in}{2.126635in}}%
\pgfpathlineto{\pgfqpoint{2.524661in}{2.125176in}}%
\pgfpathlineto{\pgfqpoint{2.521850in}{2.124779in}}%
\pgfpathlineto{\pgfqpoint{2.519040in}{2.123736in}}%
\pgfpathlineto{\pgfqpoint{2.516229in}{2.123469in}}%
\pgfpathlineto{\pgfqpoint{2.513418in}{2.122784in}}%
\pgfpathlineto{\pgfqpoint{2.510608in}{2.121451in}}%
\pgfpathlineto{\pgfqpoint{2.507797in}{2.121901in}}%
\pgfpathlineto{\pgfqpoint{2.504986in}{2.121886in}}%
\pgfpathlineto{\pgfqpoint{2.502176in}{2.120599in}}%
\pgfpathlineto{\pgfqpoint{2.499365in}{2.120954in}}%
\pgfpathlineto{\pgfqpoint{2.496554in}{2.121355in}}%
\pgfpathlineto{\pgfqpoint{2.493744in}{2.121811in}}%
\pgfpathlineto{\pgfqpoint{2.490933in}{2.121177in}}%
\pgfpathlineto{\pgfqpoint{2.488122in}{2.121341in}}%
\pgfpathlineto{\pgfqpoint{2.485312in}{2.121731in}}%
\pgfpathlineto{\pgfqpoint{2.482501in}{2.122089in}}%
\pgfpathlineto{\pgfqpoint{2.479690in}{2.122485in}}%
\pgfpathlineto{\pgfqpoint{2.476879in}{2.122861in}}%
\pgfpathlineto{\pgfqpoint{2.474069in}{2.123017in}}%
\pgfpathlineto{\pgfqpoint{2.471258in}{2.123451in}}%
\pgfpathlineto{\pgfqpoint{2.468447in}{2.123838in}}%
\pgfpathlineto{\pgfqpoint{2.465637in}{2.123716in}}%
\pgfpathlineto{\pgfqpoint{2.462826in}{2.123440in}}%
\pgfpathlineto{\pgfqpoint{2.460015in}{2.123849in}}%
\pgfpathlineto{\pgfqpoint{2.457205in}{2.123595in}}%
\pgfpathlineto{\pgfqpoint{2.454394in}{2.124063in}}%
\pgfpathlineto{\pgfqpoint{2.451583in}{2.124200in}}%
\pgfpathlineto{\pgfqpoint{2.448773in}{2.124617in}}%
\pgfpathlineto{\pgfqpoint{2.445962in}{2.124743in}}%
\pgfpathlineto{\pgfqpoint{2.443151in}{2.125012in}}%
\pgfpathlineto{\pgfqpoint{2.440341in}{2.125391in}}%
\pgfpathlineto{\pgfqpoint{2.437530in}{2.124712in}}%
\pgfpathlineto{\pgfqpoint{2.434719in}{2.124727in}}%
\pgfpathlineto{\pgfqpoint{2.431908in}{2.125173in}}%
\pgfpathlineto{\pgfqpoint{2.429098in}{2.125575in}}%
\pgfpathlineto{\pgfqpoint{2.426287in}{2.126052in}}%
\pgfpathlineto{\pgfqpoint{2.423476in}{2.126345in}}%
\pgfpathlineto{\pgfqpoint{2.420666in}{2.126155in}}%
\pgfpathlineto{\pgfqpoint{2.417855in}{2.126586in}}%
\pgfpathlineto{\pgfqpoint{2.415044in}{2.127046in}}%
\pgfpathlineto{\pgfqpoint{2.412234in}{2.127489in}}%
\pgfpathlineto{\pgfqpoint{2.409423in}{2.127912in}}%
\pgfpathlineto{\pgfqpoint{2.406612in}{2.127981in}}%
\pgfpathlineto{\pgfqpoint{2.403802in}{2.128423in}}%
\pgfpathlineto{\pgfqpoint{2.400991in}{2.127984in}}%
\pgfpathlineto{\pgfqpoint{2.398180in}{2.128097in}}%
\pgfpathlineto{\pgfqpoint{2.395370in}{2.128589in}}%
\pgfpathlineto{\pgfqpoint{2.392559in}{2.128401in}}%
\pgfpathlineto{\pgfqpoint{2.389748in}{2.127948in}}%
\pgfpathlineto{\pgfqpoint{2.386937in}{2.128212in}}%
\pgfpathlineto{\pgfqpoint{2.384127in}{2.128475in}}%
\pgfpathlineto{\pgfqpoint{2.381316in}{2.128944in}}%
\pgfpathlineto{\pgfqpoint{2.378505in}{2.128601in}}%
\pgfpathlineto{\pgfqpoint{2.375695in}{2.128009in}}%
\pgfpathlineto{\pgfqpoint{2.372884in}{2.128185in}}%
\pgfpathlineto{\pgfqpoint{2.370073in}{2.127098in}}%
\pgfpathlineto{\pgfqpoint{2.367263in}{2.126908in}}%
\pgfpathlineto{\pgfqpoint{2.364452in}{2.125916in}}%
\pgfpathlineto{\pgfqpoint{2.361641in}{2.125888in}}%
\pgfpathlineto{\pgfqpoint{2.358831in}{2.126185in}}%
\pgfpathlineto{\pgfqpoint{2.356020in}{2.126650in}}%
\pgfpathlineto{\pgfqpoint{2.353209in}{2.121284in}}%
\pgfpathlineto{\pgfqpoint{2.350399in}{2.121750in}}%
\pgfpathlineto{\pgfqpoint{2.347588in}{2.122026in}}%
\pgfpathlineto{\pgfqpoint{2.344777in}{2.121404in}}%
\pgfpathlineto{\pgfqpoint{2.341967in}{2.121848in}}%
\pgfpathlineto{\pgfqpoint{2.339156in}{2.122344in}}%
\pgfpathlineto{\pgfqpoint{2.336345in}{2.122815in}}%
\pgfpathlineto{\pgfqpoint{2.333534in}{2.123167in}}%
\pgfpathlineto{\pgfqpoint{2.330724in}{2.123304in}}%
\pgfpathlineto{\pgfqpoint{2.327913in}{2.123810in}}%
\pgfpathlineto{\pgfqpoint{2.325102in}{2.124240in}}%
\pgfpathlineto{\pgfqpoint{2.322292in}{2.124737in}}%
\pgfpathlineto{\pgfqpoint{2.319481in}{2.124857in}}%
\pgfpathlineto{\pgfqpoint{2.316670in}{2.125103in}}%
\pgfpathlineto{\pgfqpoint{2.313860in}{2.125600in}}%
\pgfpathlineto{\pgfqpoint{2.311049in}{2.125795in}}%
\pgfpathlineto{\pgfqpoint{2.308238in}{2.126154in}}%
\pgfpathlineto{\pgfqpoint{2.305428in}{2.126176in}}%
\pgfpathlineto{\pgfqpoint{2.302617in}{2.126644in}}%
\pgfpathlineto{\pgfqpoint{2.299806in}{2.126996in}}%
\pgfpathlineto{\pgfqpoint{2.296996in}{2.127010in}}%
\pgfpathlineto{\pgfqpoint{2.294185in}{2.125220in}}%
\pgfpathlineto{\pgfqpoint{2.291374in}{2.125627in}}%
\pgfpathlineto{\pgfqpoint{2.288563in}{2.124774in}}%
\pgfpathlineto{\pgfqpoint{2.285753in}{2.125228in}}%
\pgfpathlineto{\pgfqpoint{2.282942in}{2.122705in}}%
\pgfpathlineto{\pgfqpoint{2.280131in}{2.123164in}}%
\pgfpathlineto{\pgfqpoint{2.277321in}{2.123655in}}%
\pgfpathlineto{\pgfqpoint{2.274510in}{2.124103in}}%
\pgfpathlineto{\pgfqpoint{2.271699in}{2.124635in}}%
\pgfpathlineto{\pgfqpoint{2.268889in}{2.124924in}}%
\pgfpathlineto{\pgfqpoint{2.266078in}{2.125162in}}%
\pgfpathlineto{\pgfqpoint{2.263267in}{2.125267in}}%
\pgfpathlineto{\pgfqpoint{2.260457in}{2.125789in}}%
\pgfpathlineto{\pgfqpoint{2.257646in}{2.126191in}}%
\pgfpathlineto{\pgfqpoint{2.254835in}{2.126321in}}%
\pgfpathlineto{\pgfqpoint{2.252025in}{2.126816in}}%
\pgfpathlineto{\pgfqpoint{2.249214in}{2.127213in}}%
\pgfpathlineto{\pgfqpoint{2.246403in}{2.126520in}}%
\pgfpathlineto{\pgfqpoint{2.243592in}{2.126870in}}%
\pgfpathlineto{\pgfqpoint{2.240782in}{2.127287in}}%
\pgfpathlineto{\pgfqpoint{2.237971in}{2.127777in}}%
\pgfpathlineto{\pgfqpoint{2.235160in}{2.127821in}}%
\pgfpathlineto{\pgfqpoint{2.232350in}{2.128254in}}%
\pgfpathlineto{\pgfqpoint{2.229539in}{2.127724in}}%
\pgfpathlineto{\pgfqpoint{2.226728in}{2.128280in}}%
\pgfpathlineto{\pgfqpoint{2.223918in}{2.128587in}}%
\pgfpathlineto{\pgfqpoint{2.221107in}{2.128478in}}%
\pgfpathlineto{\pgfqpoint{2.218296in}{2.128816in}}%
\pgfpathlineto{\pgfqpoint{2.215486in}{2.128922in}}%
\pgfpathlineto{\pgfqpoint{2.212675in}{2.129230in}}%
\pgfpathlineto{\pgfqpoint{2.209864in}{2.129621in}}%
\pgfpathlineto{\pgfqpoint{2.207054in}{2.129387in}}%
\pgfpathlineto{\pgfqpoint{2.204243in}{2.127643in}}%
\pgfpathlineto{\pgfqpoint{2.201432in}{2.128180in}}%
\pgfpathlineto{\pgfqpoint{2.198621in}{2.128446in}}%
\pgfpathlineto{\pgfqpoint{2.195811in}{2.128945in}}%
\pgfpathlineto{\pgfqpoint{2.193000in}{2.129428in}}%
\pgfpathlineto{\pgfqpoint{2.190189in}{2.128399in}}%
\pgfpathlineto{\pgfqpoint{2.187379in}{2.128971in}}%
\pgfpathlineto{\pgfqpoint{2.184568in}{2.129481in}}%
\pgfpathlineto{\pgfqpoint{2.181757in}{2.130033in}}%
\pgfpathlineto{\pgfqpoint{2.178947in}{2.129955in}}%
\pgfpathlineto{\pgfqpoint{2.176136in}{2.129630in}}%
\pgfpathlineto{\pgfqpoint{2.173325in}{2.128331in}}%
\pgfpathlineto{\pgfqpoint{2.170515in}{2.128838in}}%
\pgfpathlineto{\pgfqpoint{2.167704in}{2.129063in}}%
\pgfpathlineto{\pgfqpoint{2.164893in}{2.128422in}}%
\pgfpathlineto{\pgfqpoint{2.162083in}{2.126181in}}%
\pgfpathlineto{\pgfqpoint{2.159272in}{2.126364in}}%
\pgfpathlineto{\pgfqpoint{2.156461in}{2.126374in}}%
\pgfpathlineto{\pgfqpoint{2.153651in}{2.126243in}}%
\pgfpathlineto{\pgfqpoint{2.150840in}{2.126254in}}%
\pgfpathlineto{\pgfqpoint{2.148029in}{2.126513in}}%
\pgfpathlineto{\pgfqpoint{2.145218in}{2.127091in}}%
\pgfpathlineto{\pgfqpoint{2.142408in}{2.126988in}}%
\pgfpathlineto{\pgfqpoint{2.139597in}{2.127521in}}%
\pgfpathlineto{\pgfqpoint{2.136786in}{2.128095in}}%
\pgfpathlineto{\pgfqpoint{2.133976in}{2.128052in}}%
\pgfpathlineto{\pgfqpoint{2.131165in}{2.128610in}}%
\pgfpathlineto{\pgfqpoint{2.128354in}{2.128968in}}%
\pgfpathlineto{\pgfqpoint{2.125544in}{2.129404in}}%
\pgfpathlineto{\pgfqpoint{2.122733in}{2.128202in}}%
\pgfpathlineto{\pgfqpoint{2.119922in}{2.128508in}}%
\pgfpathlineto{\pgfqpoint{2.117112in}{2.128429in}}%
\pgfpathlineto{\pgfqpoint{2.114301in}{2.128278in}}%
\pgfpathlineto{\pgfqpoint{2.111490in}{2.128743in}}%
\pgfpathlineto{\pgfqpoint{2.108680in}{2.127494in}}%
\pgfpathlineto{\pgfqpoint{2.105869in}{2.128100in}}%
\pgfpathlineto{\pgfqpoint{2.103058in}{2.128135in}}%
\pgfpathlineto{\pgfqpoint{2.100247in}{2.124755in}}%
\pgfpathlineto{\pgfqpoint{2.097437in}{2.124664in}}%
\pgfpathlineto{\pgfqpoint{2.094626in}{2.125111in}}%
\pgfpathlineto{\pgfqpoint{2.091815in}{2.125641in}}%
\pgfpathlineto{\pgfqpoint{2.089005in}{2.126259in}}%
\pgfpathlineto{\pgfqpoint{2.086194in}{2.125767in}}%
\pgfpathlineto{\pgfqpoint{2.083383in}{2.126390in}}%
\pgfpathlineto{\pgfqpoint{2.080573in}{2.126902in}}%
\pgfpathlineto{\pgfqpoint{2.077762in}{2.127173in}}%
\pgfpathlineto{\pgfqpoint{2.074951in}{2.127803in}}%
\pgfpathlineto{\pgfqpoint{2.072141in}{2.127590in}}%
\pgfpathlineto{\pgfqpoint{2.069330in}{2.128110in}}%
\pgfpathlineto{\pgfqpoint{2.066519in}{2.128553in}}%
\pgfpathlineto{\pgfqpoint{2.063709in}{2.125800in}}%
\pgfpathlineto{\pgfqpoint{2.060898in}{2.126373in}}%
\pgfpathlineto{\pgfqpoint{2.058087in}{2.126284in}}%
\pgfpathlineto{\pgfqpoint{2.055276in}{2.126922in}}%
\pgfpathlineto{\pgfqpoint{2.052466in}{2.126480in}}%
\pgfpathlineto{\pgfqpoint{2.049655in}{2.126950in}}%
\pgfpathlineto{\pgfqpoint{2.046844in}{2.127577in}}%
\pgfpathlineto{\pgfqpoint{2.044034in}{2.128148in}}%
\pgfpathlineto{\pgfqpoint{2.041223in}{2.128796in}}%
\pgfpathlineto{\pgfqpoint{2.038412in}{2.129435in}}%
\pgfpathlineto{\pgfqpoint{2.035602in}{2.130061in}}%
\pgfpathlineto{\pgfqpoint{2.032791in}{2.130677in}}%
\pgfpathlineto{\pgfqpoint{2.029980in}{2.131133in}}%
\pgfpathlineto{\pgfqpoint{2.027170in}{2.128870in}}%
\pgfpathlineto{\pgfqpoint{2.024359in}{2.128663in}}%
\pgfpathlineto{\pgfqpoint{2.021548in}{2.113582in}}%
\pgfpathlineto{\pgfqpoint{2.018738in}{2.114201in}}%
\pgfpathlineto{\pgfqpoint{2.015927in}{2.114830in}}%
\pgfpathlineto{\pgfqpoint{2.013116in}{2.115447in}}%
\pgfpathlineto{\pgfqpoint{2.010305in}{2.109575in}}%
\pgfpathlineto{\pgfqpoint{2.007495in}{2.110136in}}%
\pgfpathlineto{\pgfqpoint{2.004684in}{2.110513in}}%
\pgfpathlineto{\pgfqpoint{2.001873in}{2.110730in}}%
\pgfpathlineto{\pgfqpoint{1.999063in}{2.111367in}}%
\pgfpathlineto{\pgfqpoint{1.996252in}{2.111568in}}%
\pgfpathlineto{\pgfqpoint{1.993441in}{2.112093in}}%
\pgfpathlineto{\pgfqpoint{1.990631in}{2.112739in}}%
\pgfpathlineto{\pgfqpoint{1.987820in}{2.113368in}}%
\pgfpathlineto{\pgfqpoint{1.985009in}{2.113782in}}%
\pgfpathlineto{\pgfqpoint{1.982199in}{2.112895in}}%
\pgfpathlineto{\pgfqpoint{1.979388in}{2.113537in}}%
\pgfpathlineto{\pgfqpoint{1.976577in}{2.114195in}}%
\pgfpathlineto{\pgfqpoint{1.973767in}{2.114577in}}%
\pgfpathlineto{\pgfqpoint{1.970956in}{2.113318in}}%
\pgfpathlineto{\pgfqpoint{1.968145in}{2.113318in}}%
\pgfpathlineto{\pgfqpoint{1.965334in}{2.113900in}}%
\pgfpathlineto{\pgfqpoint{1.962524in}{2.113520in}}%
\pgfpathlineto{\pgfqpoint{1.959713in}{2.114157in}}%
\pgfpathlineto{\pgfqpoint{1.956902in}{2.114230in}}%
\pgfpathlineto{\pgfqpoint{1.954092in}{2.113988in}}%
\pgfpathlineto{\pgfqpoint{1.951281in}{2.113726in}}%
\pgfpathlineto{\pgfqpoint{1.948470in}{2.114393in}}%
\pgfpathlineto{\pgfqpoint{1.945660in}{2.114698in}}%
\pgfpathlineto{\pgfqpoint{1.942849in}{2.114310in}}%
\pgfpathlineto{\pgfqpoint{1.940038in}{2.114969in}}%
\pgfpathlineto{\pgfqpoint{1.937228in}{2.115003in}}%
\pgfpathlineto{\pgfqpoint{1.934417in}{2.115175in}}%
\pgfpathlineto{\pgfqpoint{1.931606in}{2.115848in}}%
\pgfpathlineto{\pgfqpoint{1.928796in}{2.114953in}}%
\pgfpathlineto{\pgfqpoint{1.925985in}{2.115634in}}%
\pgfpathlineto{\pgfqpoint{1.923174in}{2.116018in}}%
\pgfpathlineto{\pgfqpoint{1.920364in}{2.115633in}}%
\pgfpathlineto{\pgfqpoint{1.917553in}{2.116117in}}%
\pgfpathlineto{\pgfqpoint{1.914742in}{2.115874in}}%
\pgfpathlineto{\pgfqpoint{1.911931in}{2.116130in}}%
\pgfpathlineto{\pgfqpoint{1.909121in}{2.116757in}}%
\pgfpathlineto{\pgfqpoint{1.906310in}{2.116575in}}%
\pgfpathlineto{\pgfqpoint{1.903499in}{2.116876in}}%
\pgfpathlineto{\pgfqpoint{1.900689in}{2.116121in}}%
\pgfpathlineto{\pgfqpoint{1.897878in}{2.116733in}}%
\pgfpathlineto{\pgfqpoint{1.895067in}{2.117430in}}%
\pgfpathlineto{\pgfqpoint{1.892257in}{2.117052in}}%
\pgfpathlineto{\pgfqpoint{1.889446in}{2.117630in}}%
\pgfpathlineto{\pgfqpoint{1.886635in}{2.118328in}}%
\pgfpathlineto{\pgfqpoint{1.883825in}{2.118983in}}%
\pgfpathlineto{\pgfqpoint{1.881014in}{2.119274in}}%
\pgfpathlineto{\pgfqpoint{1.878203in}{2.116730in}}%
\pgfpathlineto{\pgfqpoint{1.875393in}{2.117337in}}%
\pgfpathlineto{\pgfqpoint{1.872582in}{2.116681in}}%
\pgfpathlineto{\pgfqpoint{1.869771in}{2.116858in}}%
\pgfpathlineto{\pgfqpoint{1.866960in}{2.117477in}}%
\pgfpathlineto{\pgfqpoint{1.864150in}{2.118099in}}%
\pgfpathlineto{\pgfqpoint{1.861339in}{2.118839in}}%
\pgfpathlineto{\pgfqpoint{1.858528in}{2.119577in}}%
\pgfpathlineto{\pgfqpoint{1.855718in}{2.119977in}}%
\pgfpathlineto{\pgfqpoint{1.852907in}{2.120726in}}%
\pgfpathlineto{\pgfqpoint{1.850096in}{2.118699in}}%
\pgfpathlineto{\pgfqpoint{1.847286in}{2.106619in}}%
\pgfpathlineto{\pgfqpoint{1.844475in}{2.107014in}}%
\pgfpathlineto{\pgfqpoint{1.841664in}{2.107508in}}%
\pgfpathlineto{\pgfqpoint{1.838854in}{2.107971in}}%
\pgfpathlineto{\pgfqpoint{1.836043in}{2.108664in}}%
\pgfpathlineto{\pgfqpoint{1.833232in}{2.108612in}}%
\pgfpathlineto{\pgfqpoint{1.830422in}{2.108712in}}%
\pgfpathlineto{\pgfqpoint{1.827611in}{2.108276in}}%
\pgfpathlineto{\pgfqpoint{1.824800in}{2.109019in}}%
\pgfpathlineto{\pgfqpoint{1.821989in}{2.108210in}}%
\pgfpathlineto{\pgfqpoint{1.819179in}{2.108977in}}%
\pgfpathlineto{\pgfqpoint{1.816368in}{2.109372in}}%
\pgfpathlineto{\pgfqpoint{1.813557in}{2.107627in}}%
\pgfpathlineto{\pgfqpoint{1.810747in}{2.106506in}}%
\pgfpathlineto{\pgfqpoint{1.807936in}{2.107131in}}%
\pgfpathlineto{\pgfqpoint{1.805125in}{2.107629in}}%
\pgfpathlineto{\pgfqpoint{1.802315in}{2.107126in}}%
\pgfpathlineto{\pgfqpoint{1.799504in}{2.107831in}}%
\pgfpathlineto{\pgfqpoint{1.796693in}{2.108085in}}%
\pgfpathlineto{\pgfqpoint{1.793883in}{2.108835in}}%
\pgfpathlineto{\pgfqpoint{1.791072in}{2.108737in}}%
\pgfpathlineto{\pgfqpoint{1.788261in}{2.108727in}}%
\pgfpathlineto{\pgfqpoint{1.785451in}{2.109242in}}%
\pgfpathlineto{\pgfqpoint{1.782640in}{2.109646in}}%
\pgfpathlineto{\pgfqpoint{1.779829in}{2.109905in}}%
\pgfpathlineto{\pgfqpoint{1.777018in}{2.110159in}}%
\pgfpathlineto{\pgfqpoint{1.774208in}{2.108370in}}%
\pgfpathlineto{\pgfqpoint{1.771397in}{2.105868in}}%
\pgfpathlineto{\pgfqpoint{1.768586in}{2.105203in}}%
\pgfpathlineto{\pgfqpoint{1.765776in}{2.105921in}}%
\pgfpathlineto{\pgfqpoint{1.762965in}{2.104019in}}%
\pgfpathlineto{\pgfqpoint{1.760154in}{2.104562in}}%
\pgfpathlineto{\pgfqpoint{1.757344in}{2.105363in}}%
\pgfpathlineto{\pgfqpoint{1.754533in}{2.105903in}}%
\pgfpathlineto{\pgfqpoint{1.751722in}{2.106060in}}%
\pgfpathlineto{\pgfqpoint{1.748912in}{2.106861in}}%
\pgfpathlineto{\pgfqpoint{1.746101in}{2.107628in}}%
\pgfpathlineto{\pgfqpoint{1.743290in}{2.108192in}}%
\pgfpathlineto{\pgfqpoint{1.740480in}{2.108896in}}%
\pgfpathlineto{\pgfqpoint{1.737669in}{2.109726in}}%
\pgfpathlineto{\pgfqpoint{1.734858in}{2.110200in}}%
\pgfpathlineto{\pgfqpoint{1.732048in}{2.110653in}}%
\pgfpathlineto{\pgfqpoint{1.729237in}{2.110697in}}%
\pgfpathlineto{\pgfqpoint{1.726426in}{2.111547in}}%
\pgfpathlineto{\pgfqpoint{1.723615in}{2.112375in}}%
\pgfpathlineto{\pgfqpoint{1.720805in}{2.111955in}}%
\pgfpathlineto{\pgfqpoint{1.717994in}{2.111545in}}%
\pgfpathlineto{\pgfqpoint{1.715183in}{2.111121in}}%
\pgfpathlineto{\pgfqpoint{1.712373in}{2.110729in}}%
\pgfpathlineto{\pgfqpoint{1.709562in}{2.110265in}}%
\pgfpathlineto{\pgfqpoint{1.706751in}{2.110794in}}%
\pgfpathlineto{\pgfqpoint{1.703941in}{2.111644in}}%
\pgfpathlineto{\pgfqpoint{1.701130in}{2.111187in}}%
\pgfpathlineto{\pgfqpoint{1.698319in}{2.111402in}}%
\pgfpathlineto{\pgfqpoint{1.695509in}{2.112294in}}%
\pgfpathlineto{\pgfqpoint{1.692698in}{2.112913in}}%
\pgfpathlineto{\pgfqpoint{1.689887in}{2.113724in}}%
\pgfpathlineto{\pgfqpoint{1.687077in}{2.114223in}}%
\pgfpathlineto{\pgfqpoint{1.684266in}{2.113610in}}%
\pgfpathlineto{\pgfqpoint{1.681455in}{2.113801in}}%
\pgfpathlineto{\pgfqpoint{1.678644in}{2.112392in}}%
\pgfpathlineto{\pgfqpoint{1.675834in}{2.113082in}}%
\pgfpathlineto{\pgfqpoint{1.673023in}{2.113641in}}%
\pgfpathlineto{\pgfqpoint{1.670212in}{2.112173in}}%
\pgfpathlineto{\pgfqpoint{1.667402in}{2.112961in}}%
\pgfpathlineto{\pgfqpoint{1.664591in}{2.113682in}}%
\pgfpathlineto{\pgfqpoint{1.661780in}{2.113287in}}%
\pgfpathlineto{\pgfqpoint{1.658970in}{2.114018in}}%
\pgfpathlineto{\pgfqpoint{1.656159in}{2.114681in}}%
\pgfpathlineto{\pgfqpoint{1.653348in}{2.115522in}}%
\pgfpathlineto{\pgfqpoint{1.650538in}{2.116035in}}%
\pgfpathlineto{\pgfqpoint{1.647727in}{2.116856in}}%
\pgfpathlineto{\pgfqpoint{1.644916in}{2.117779in}}%
\pgfpathlineto{\pgfqpoint{1.642106in}{2.118692in}}%
\pgfpathlineto{\pgfqpoint{1.639295in}{2.119610in}}%
\pgfpathlineto{\pgfqpoint{1.636484in}{2.120572in}}%
\pgfpathlineto{\pgfqpoint{1.633673in}{2.121212in}}%
\pgfpathlineto{\pgfqpoint{1.630863in}{2.122153in}}%
\pgfpathlineto{\pgfqpoint{1.628052in}{2.121313in}}%
\pgfpathlineto{\pgfqpoint{1.625241in}{2.121397in}}%
\pgfpathlineto{\pgfqpoint{1.622431in}{2.121711in}}%
\pgfpathlineto{\pgfqpoint{1.619620in}{2.121952in}}%
\pgfpathlineto{\pgfqpoint{1.616809in}{2.122809in}}%
\pgfpathlineto{\pgfqpoint{1.613999in}{2.119684in}}%
\pgfpathlineto{\pgfqpoint{1.611188in}{2.117879in}}%
\pgfpathlineto{\pgfqpoint{1.608377in}{2.118900in}}%
\pgfpathlineto{\pgfqpoint{1.605567in}{2.119893in}}%
\pgfpathlineto{\pgfqpoint{1.602756in}{2.120835in}}%
\pgfpathlineto{\pgfqpoint{1.599945in}{2.121594in}}%
\pgfpathlineto{\pgfqpoint{1.597135in}{2.122460in}}%
\pgfpathlineto{\pgfqpoint{1.594324in}{2.120079in}}%
\pgfpathlineto{\pgfqpoint{1.591513in}{2.120993in}}%
\pgfpathlineto{\pgfqpoint{1.588702in}{2.120863in}}%
\pgfpathlineto{\pgfqpoint{1.585892in}{2.119856in}}%
\pgfpathlineto{\pgfqpoint{1.583081in}{2.120886in}}%
\pgfpathlineto{\pgfqpoint{1.580270in}{2.121957in}}%
\pgfpathlineto{\pgfqpoint{1.577460in}{2.123018in}}%
\pgfpathlineto{\pgfqpoint{1.574649in}{2.123872in}}%
\pgfpathlineto{\pgfqpoint{1.571838in}{2.124804in}}%
\pgfpathlineto{\pgfqpoint{1.569028in}{2.125743in}}%
\pgfpathlineto{\pgfqpoint{1.566217in}{2.126549in}}%
\pgfpathlineto{\pgfqpoint{1.563406in}{2.127310in}}%
\pgfpathlineto{\pgfqpoint{1.560596in}{2.128408in}}%
\pgfpathlineto{\pgfqpoint{1.557785in}{2.129512in}}%
\pgfpathlineto{\pgfqpoint{1.554974in}{2.129444in}}%
\pgfpathlineto{\pgfqpoint{1.552164in}{2.129962in}}%
\pgfpathlineto{\pgfqpoint{1.549353in}{2.130747in}}%
\pgfpathlineto{\pgfqpoint{1.546542in}{2.131877in}}%
\pgfpathlineto{\pgfqpoint{1.543731in}{2.133033in}}%
\pgfpathlineto{\pgfqpoint{1.540921in}{2.133050in}}%
\pgfpathlineto{\pgfqpoint{1.538110in}{2.133953in}}%
\pgfpathlineto{\pgfqpoint{1.535299in}{2.134620in}}%
\pgfpathlineto{\pgfqpoint{1.532489in}{2.132477in}}%
\pgfpathlineto{\pgfqpoint{1.529678in}{2.130597in}}%
\pgfpathlineto{\pgfqpoint{1.526867in}{2.131627in}}%
\pgfpathlineto{\pgfqpoint{1.524057in}{2.132789in}}%
\pgfpathlineto{\pgfqpoint{1.521246in}{2.134000in}}%
\pgfpathlineto{\pgfqpoint{1.518435in}{2.134926in}}%
\pgfpathlineto{\pgfqpoint{1.515625in}{2.135485in}}%
\pgfpathlineto{\pgfqpoint{1.512814in}{2.136702in}}%
\pgfpathlineto{\pgfqpoint{1.510003in}{2.137874in}}%
\pgfpathlineto{\pgfqpoint{1.507193in}{2.136582in}}%
\pgfpathlineto{\pgfqpoint{1.504382in}{2.137512in}}%
\pgfpathlineto{\pgfqpoint{1.501571in}{2.138755in}}%
\pgfpathlineto{\pgfqpoint{1.498761in}{2.130801in}}%
\pgfpathlineto{\pgfqpoint{1.495950in}{2.131659in}}%
\pgfpathlineto{\pgfqpoint{1.493139in}{2.132279in}}%
\pgfpathlineto{\pgfqpoint{1.490328in}{2.132594in}}%
\pgfpathlineto{\pgfqpoint{1.487518in}{2.133770in}}%
\pgfpathlineto{\pgfqpoint{1.484707in}{2.134304in}}%
\pgfpathlineto{\pgfqpoint{1.481896in}{2.135588in}}%
\pgfpathlineto{\pgfqpoint{1.479086in}{2.136510in}}%
\pgfpathlineto{\pgfqpoint{1.476275in}{2.137823in}}%
\pgfpathlineto{\pgfqpoint{1.473464in}{2.137826in}}%
\pgfpathlineto{\pgfqpoint{1.470654in}{2.138021in}}%
\pgfpathlineto{\pgfqpoint{1.467843in}{2.138705in}}%
\pgfpathlineto{\pgfqpoint{1.465032in}{2.139851in}}%
\pgfpathlineto{\pgfqpoint{1.462222in}{2.139562in}}%
\pgfpathlineto{\pgfqpoint{1.459411in}{2.140323in}}%
\pgfpathlineto{\pgfqpoint{1.456600in}{2.141257in}}%
\pgfpathlineto{\pgfqpoint{1.453790in}{2.142468in}}%
\pgfpathlineto{\pgfqpoint{1.450979in}{2.143342in}}%
\pgfpathlineto{\pgfqpoint{1.448168in}{2.143655in}}%
\pgfpathlineto{\pgfqpoint{1.445357in}{2.141808in}}%
\pgfpathlineto{\pgfqpoint{1.442547in}{2.143230in}}%
\pgfpathlineto{\pgfqpoint{1.439736in}{2.141469in}}%
\pgfpathlineto{\pgfqpoint{1.436925in}{2.142476in}}%
\pgfpathlineto{\pgfqpoint{1.434115in}{2.142410in}}%
\pgfpathlineto{\pgfqpoint{1.431304in}{2.143297in}}%
\pgfpathlineto{\pgfqpoint{1.428493in}{2.143920in}}%
\pgfpathlineto{\pgfqpoint{1.425683in}{2.145297in}}%
\pgfpathlineto{\pgfqpoint{1.422872in}{2.146348in}}%
\pgfpathlineto{\pgfqpoint{1.420061in}{2.147741in}}%
\pgfpathlineto{\pgfqpoint{1.417251in}{2.148083in}}%
\pgfpathlineto{\pgfqpoint{1.414440in}{2.149497in}}%
\pgfpathlineto{\pgfqpoint{1.411629in}{2.150993in}}%
\pgfpathlineto{\pgfqpoint{1.408819in}{2.152542in}}%
\pgfpathlineto{\pgfqpoint{1.406008in}{2.152344in}}%
\pgfpathlineto{\pgfqpoint{1.403197in}{2.151318in}}%
\pgfpathlineto{\pgfqpoint{1.400386in}{2.148230in}}%
\pgfpathlineto{\pgfqpoint{1.397576in}{2.149788in}}%
\pgfpathlineto{\pgfqpoint{1.394765in}{2.151065in}}%
\pgfpathlineto{\pgfqpoint{1.391954in}{2.149956in}}%
\pgfpathlineto{\pgfqpoint{1.389144in}{2.151566in}}%
\pgfpathlineto{\pgfqpoint{1.386333in}{2.152802in}}%
\pgfpathlineto{\pgfqpoint{1.383522in}{2.152347in}}%
\pgfpathlineto{\pgfqpoint{1.380712in}{2.153892in}}%
\pgfpathlineto{\pgfqpoint{1.377901in}{2.154865in}}%
\pgfpathlineto{\pgfqpoint{1.375090in}{2.156443in}}%
\pgfpathlineto{\pgfqpoint{1.372280in}{2.156936in}}%
\pgfpathlineto{\pgfqpoint{1.369469in}{2.158628in}}%
\pgfpathlineto{\pgfqpoint{1.366658in}{2.160350in}}%
\pgfpathlineto{\pgfqpoint{1.363848in}{2.161379in}}%
\pgfpathlineto{\pgfqpoint{1.361037in}{2.163141in}}%
\pgfpathlineto{\pgfqpoint{1.358226in}{2.164893in}}%
\pgfpathlineto{\pgfqpoint{1.355415in}{2.166662in}}%
\pgfpathlineto{\pgfqpoint{1.352605in}{2.167708in}}%
\pgfpathlineto{\pgfqpoint{1.349794in}{2.169275in}}%
\pgfpathlineto{\pgfqpoint{1.346983in}{2.168908in}}%
\pgfpathlineto{\pgfqpoint{1.344173in}{2.170770in}}%
\pgfpathlineto{\pgfqpoint{1.341362in}{2.171897in}}%
\pgfpathlineto{\pgfqpoint{1.338551in}{2.172360in}}%
\pgfpathlineto{\pgfqpoint{1.335741in}{2.173320in}}%
\pgfpathlineto{\pgfqpoint{1.332930in}{2.175233in}}%
\pgfpathlineto{\pgfqpoint{1.330119in}{2.176904in}}%
\pgfpathlineto{\pgfqpoint{1.327309in}{2.176338in}}%
\pgfpathlineto{\pgfqpoint{1.324498in}{2.175508in}}%
\pgfpathlineto{\pgfqpoint{1.321687in}{2.177224in}}%
\pgfpathlineto{\pgfqpoint{1.318877in}{2.178785in}}%
\pgfpathlineto{\pgfqpoint{1.316066in}{2.178217in}}%
\pgfpathlineto{\pgfqpoint{1.313255in}{2.175898in}}%
\pgfpathlineto{\pgfqpoint{1.310445in}{2.164712in}}%
\pgfpathlineto{\pgfqpoint{1.307634in}{2.166696in}}%
\pgfpathlineto{\pgfqpoint{1.304823in}{2.168218in}}%
\pgfpathlineto{\pgfqpoint{1.302012in}{2.170138in}}%
\pgfpathlineto{\pgfqpoint{1.299202in}{2.171630in}}%
\pgfpathlineto{\pgfqpoint{1.296391in}{2.172407in}}%
\pgfpathlineto{\pgfqpoint{1.293580in}{2.174466in}}%
\pgfpathlineto{\pgfqpoint{1.290770in}{2.175031in}}%
\pgfpathlineto{\pgfqpoint{1.287959in}{2.170091in}}%
\pgfpathlineto{\pgfqpoint{1.285148in}{2.169799in}}%
\pgfpathlineto{\pgfqpoint{1.282338in}{2.168160in}}%
\pgfpathlineto{\pgfqpoint{1.279527in}{2.170000in}}%
\pgfpathlineto{\pgfqpoint{1.276716in}{2.172030in}}%
\pgfpathlineto{\pgfqpoint{1.273906in}{2.173875in}}%
\pgfpathlineto{\pgfqpoint{1.271095in}{2.175611in}}%
\pgfpathlineto{\pgfqpoint{1.268284in}{2.176873in}}%
\pgfpathlineto{\pgfqpoint{1.265474in}{2.179060in}}%
\pgfpathlineto{\pgfqpoint{1.262663in}{2.174148in}}%
\pgfpathlineto{\pgfqpoint{1.259852in}{2.171963in}}%
\pgfpathlineto{\pgfqpoint{1.257041in}{2.173580in}}%
\pgfpathlineto{\pgfqpoint{1.254231in}{2.175385in}}%
\pgfpathlineto{\pgfqpoint{1.251420in}{2.173842in}}%
\pgfpathlineto{\pgfqpoint{1.248609in}{2.171285in}}%
\pgfpathlineto{\pgfqpoint{1.245799in}{2.150742in}}%
\pgfpathlineto{\pgfqpoint{1.242988in}{2.149437in}}%
\pgfpathlineto{\pgfqpoint{1.240177in}{2.149520in}}%
\pgfpathlineto{\pgfqpoint{1.237367in}{2.149447in}}%
\pgfpathlineto{\pgfqpoint{1.234556in}{2.148011in}}%
\pgfpathlineto{\pgfqpoint{1.231745in}{2.145811in}}%
\pgfpathlineto{\pgfqpoint{1.228935in}{2.144537in}}%
\pgfpathlineto{\pgfqpoint{1.226124in}{2.145263in}}%
\pgfpathlineto{\pgfqpoint{1.223313in}{2.146113in}}%
\pgfpathlineto{\pgfqpoint{1.220503in}{2.148457in}}%
\pgfpathlineto{\pgfqpoint{1.217692in}{2.150834in}}%
\pgfpathlineto{\pgfqpoint{1.214881in}{2.152440in}}%
\pgfpathlineto{\pgfqpoint{1.212070in}{2.146951in}}%
\pgfpathlineto{\pgfqpoint{1.209260in}{2.149479in}}%
\pgfpathlineto{\pgfqpoint{1.206449in}{2.146752in}}%
\pgfpathlineto{\pgfqpoint{1.203638in}{2.145458in}}%
\pgfpathlineto{\pgfqpoint{1.200828in}{2.146096in}}%
\pgfpathlineto{\pgfqpoint{1.198017in}{2.145328in}}%
\pgfpathlineto{\pgfqpoint{1.195206in}{2.146174in}}%
\pgfpathlineto{\pgfqpoint{1.192396in}{2.148846in}}%
\pgfpathlineto{\pgfqpoint{1.189585in}{2.150321in}}%
\pgfpathlineto{\pgfqpoint{1.186774in}{2.151515in}}%
\pgfpathlineto{\pgfqpoint{1.183964in}{2.147836in}}%
\pgfpathlineto{\pgfqpoint{1.181153in}{2.134740in}}%
\pgfpathlineto{\pgfqpoint{1.178342in}{2.134897in}}%
\pgfpathlineto{\pgfqpoint{1.175532in}{2.134624in}}%
\pgfpathlineto{\pgfqpoint{1.172721in}{2.132185in}}%
\pgfpathlineto{\pgfqpoint{1.169910in}{2.134793in}}%
\pgfpathlineto{\pgfqpoint{1.167099in}{2.137418in}}%
\pgfpathlineto{\pgfqpoint{1.164289in}{2.140235in}}%
\pgfpathlineto{\pgfqpoint{1.161478in}{2.141567in}}%
\pgfpathlineto{\pgfqpoint{1.158667in}{2.144627in}}%
\pgfpathlineto{\pgfqpoint{1.155857in}{2.147672in}}%
\pgfpathlineto{\pgfqpoint{1.153046in}{2.149123in}}%
\pgfpathlineto{\pgfqpoint{1.150235in}{2.148545in}}%
\pgfpathlineto{\pgfqpoint{1.147425in}{2.126606in}}%
\pgfpathlineto{\pgfqpoint{1.144614in}{2.129554in}}%
\pgfpathlineto{\pgfqpoint{1.141803in}{2.132233in}}%
\pgfpathlineto{\pgfqpoint{1.138993in}{2.135407in}}%
\pgfpathlineto{\pgfqpoint{1.136182in}{2.137930in}}%
\pgfpathlineto{\pgfqpoint{1.133371in}{2.137493in}}%
\pgfpathlineto{\pgfqpoint{1.130561in}{2.128850in}}%
\pgfpathlineto{\pgfqpoint{1.127750in}{2.130134in}}%
\pgfpathlineto{\pgfqpoint{1.124939in}{2.127615in}}%
\pgfpathlineto{\pgfqpoint{1.122128in}{2.130879in}}%
\pgfpathlineto{\pgfqpoint{1.119318in}{2.134338in}}%
\pgfpathlineto{\pgfqpoint{1.116507in}{2.137819in}}%
\pgfpathlineto{\pgfqpoint{1.113696in}{2.137882in}}%
\pgfpathlineto{\pgfqpoint{1.110886in}{2.137997in}}%
\pgfpathlineto{\pgfqpoint{1.108075in}{2.133855in}}%
\pgfpathlineto{\pgfqpoint{1.105264in}{2.116702in}}%
\pgfpathlineto{\pgfqpoint{1.102454in}{2.118629in}}%
\pgfpathlineto{\pgfqpoint{1.099643in}{2.115791in}}%
\pgfpathlineto{\pgfqpoint{1.096832in}{2.118172in}}%
\pgfpathlineto{\pgfqpoint{1.094022in}{2.113672in}}%
\pgfpathlineto{\pgfqpoint{1.091211in}{2.116308in}}%
\pgfpathlineto{\pgfqpoint{1.088400in}{2.115425in}}%
\pgfpathlineto{\pgfqpoint{1.085590in}{2.116513in}}%
\pgfpathlineto{\pgfqpoint{1.082779in}{2.120170in}}%
\pgfpathlineto{\pgfqpoint{1.079968in}{2.124267in}}%
\pgfpathlineto{\pgfqpoint{1.077158in}{2.128492in}}%
\pgfpathlineto{\pgfqpoint{1.074347in}{2.132747in}}%
\pgfpathlineto{\pgfqpoint{1.071536in}{2.132800in}}%
\pgfpathlineto{\pgfqpoint{1.068725in}{2.131419in}}%
\pgfpathlineto{\pgfqpoint{1.065915in}{2.134388in}}%
\pgfpathlineto{\pgfqpoint{1.063104in}{2.138328in}}%
\pgfpathlineto{\pgfqpoint{1.060293in}{2.137809in}}%
\pgfpathlineto{\pgfqpoint{1.057483in}{2.131686in}}%
\pgfpathlineto{\pgfqpoint{1.054672in}{2.136647in}}%
\pgfpathlineto{\pgfqpoint{1.051861in}{2.140849in}}%
\pgfpathlineto{\pgfqpoint{1.049051in}{2.146139in}}%
\pgfpathlineto{\pgfqpoint{1.046240in}{2.148893in}}%
\pgfpathlineto{\pgfqpoint{1.043429in}{2.154465in}}%
\pgfpathlineto{\pgfqpoint{1.040619in}{2.160011in}}%
\pgfpathlineto{\pgfqpoint{1.037808in}{2.151690in}}%
\pgfpathlineto{\pgfqpoint{1.034997in}{2.154464in}}%
\pgfpathlineto{\pgfqpoint{1.032187in}{2.158980in}}%
\pgfpathlineto{\pgfqpoint{1.029376in}{2.164787in}}%
\pgfpathlineto{\pgfqpoint{1.026565in}{2.170676in}}%
\pgfpathlineto{\pgfqpoint{1.023754in}{2.174304in}}%
\pgfpathlineto{\pgfqpoint{1.020944in}{2.171344in}}%
\pgfpathlineto{\pgfqpoint{1.018133in}{2.165364in}}%
\pgfpathlineto{\pgfqpoint{1.015322in}{2.172377in}}%
\pgfpathlineto{\pgfqpoint{1.012512in}{2.174971in}}%
\pgfpathlineto{\pgfqpoint{1.009701in}{2.182241in}}%
\pgfpathlineto{\pgfqpoint{1.006890in}{2.173950in}}%
\pgfpathlineto{\pgfqpoint{1.004080in}{2.181900in}}%
\pgfpathlineto{\pgfqpoint{1.001269in}{2.183543in}}%
\pgfpathlineto{\pgfqpoint{0.998458in}{2.190366in}}%
\pgfpathlineto{\pgfqpoint{0.995648in}{2.196061in}}%
\pgfpathlineto{\pgfqpoint{0.992837in}{2.180922in}}%
\pgfpathlineto{\pgfqpoint{0.990026in}{2.189264in}}%
\pgfpathlineto{\pgfqpoint{0.987216in}{2.188568in}}%
\pgfpathlineto{\pgfqpoint{0.984405in}{2.125419in}}%
\pgfpathlineto{\pgfqpoint{0.981594in}{2.121044in}}%
\pgfpathlineto{\pgfqpoint{0.978783in}{2.131842in}}%
\pgfpathlineto{\pgfqpoint{0.975973in}{2.140188in}}%
\pgfpathlineto{\pgfqpoint{0.973162in}{2.141847in}}%
\pgfpathlineto{\pgfqpoint{0.970351in}{2.061857in}}%
\pgfpathlineto{\pgfqpoint{0.967541in}{2.031857in}}%
\pgfpathlineto{\pgfqpoint{0.964730in}{2.027900in}}%
\pgfpathlineto{\pgfqpoint{0.961919in}{2.039605in}}%
\pgfpathlineto{\pgfqpoint{0.959109in}{2.047284in}}%
\pgfpathlineto{\pgfqpoint{0.956298in}{2.053234in}}%
\pgfpathlineto{\pgfqpoint{0.953487in}{2.069287in}}%
\pgfpathlineto{\pgfqpoint{0.950677in}{2.042573in}}%
\pgfpathlineto{\pgfqpoint{0.947866in}{2.062399in}}%
\pgfpathlineto{\pgfqpoint{0.945055in}{2.074406in}}%
\pgfpathlineto{\pgfqpoint{0.942245in}{2.091530in}}%
\pgfpathlineto{\pgfqpoint{0.939434in}{2.072788in}}%
\pgfpathlineto{\pgfqpoint{0.936623in}{2.005482in}}%
\pgfpathlineto{\pgfqpoint{0.933812in}{2.035098in}}%
\pgfpathlineto{\pgfqpoint{0.931002in}{1.840416in}}%
\pgfpathlineto{\pgfqpoint{0.928191in}{1.865627in}}%
\pgfpathlineto{\pgfqpoint{0.925380in}{1.862961in}}%
\pgfpathlineto{\pgfqpoint{0.922570in}{1.910758in}}%
\pgfpathclose%
\pgfusepath{stroke,fill}%
\end{pgfscope}%
\begin{pgfscope}%
\pgfpathrectangle{\pgfqpoint{0.711206in}{0.331635in}}{\pgfqpoint{4.650000in}{3.020000in}}%
\pgfusepath{clip}%
\pgfsetroundcap%
\pgfsetroundjoin%
\pgfsetlinewidth{1.505625pt}%
\definecolor{currentstroke}{rgb}{0.121569,0.466667,0.705882}%
\pgfsetstrokecolor{currentstroke}%
\pgfsetdash{}{0pt}%
\pgfpathmoveto{\pgfqpoint{0.922570in}{1.564045in}}%
\pgfpathlineto{\pgfqpoint{0.925380in}{1.677738in}}%
\pgfpathlineto{\pgfqpoint{0.928191in}{1.584468in}}%
\pgfpathlineto{\pgfqpoint{0.931002in}{2.033621in}}%
\pgfpathlineto{\pgfqpoint{0.933812in}{1.600243in}}%
\pgfpathlineto{\pgfqpoint{0.936623in}{1.965760in}}%
\pgfpathlineto{\pgfqpoint{0.939434in}{1.885633in}}%
\pgfpathlineto{\pgfqpoint{0.942245in}{1.477526in}}%
\pgfpathlineto{\pgfqpoint{0.945055in}{1.417916in}}%
\pgfpathlineto{\pgfqpoint{0.947866in}{1.535248in}}%
\pgfpathlineto{\pgfqpoint{0.950677in}{1.912482in}}%
\pgfpathlineto{\pgfqpoint{0.953487in}{1.621247in}}%
\pgfpathlineto{\pgfqpoint{0.956298in}{1.741600in}}%
\pgfpathlineto{\pgfqpoint{0.959109in}{1.720282in}}%
\pgfpathlineto{\pgfqpoint{0.961919in}{1.507417in}}%
\pgfpathlineto{\pgfqpoint{0.964730in}{1.812974in}}%
\pgfpathlineto{\pgfqpoint{0.967541in}{1.979928in}}%
\pgfpathlineto{\pgfqpoint{0.970351in}{2.225501in}}%
\pgfpathlineto{\pgfqpoint{0.973162in}{1.832590in}}%
\pgfpathlineto{\pgfqpoint{0.978783in}{1.638782in}}%
\pgfpathlineto{\pgfqpoint{0.981594in}{1.890745in}}%
\pgfpathlineto{\pgfqpoint{0.984405in}{2.267102in}}%
\pgfpathlineto{\pgfqpoint{0.990026in}{1.528343in}}%
\pgfpathlineto{\pgfqpoint{0.992837in}{2.036267in}}%
\pgfpathlineto{\pgfqpoint{0.998458in}{1.512676in}}%
\pgfpathlineto{\pgfqpoint{1.001269in}{1.859067in}}%
\pgfpathlineto{\pgfqpoint{1.004080in}{1.580859in}}%
\pgfpathlineto{\pgfqpoint{1.006890in}{1.986719in}}%
\pgfpathlineto{\pgfqpoint{1.009701in}{1.704182in}}%
\pgfpathlineto{\pgfqpoint{1.012512in}{1.837197in}}%
\pgfpathlineto{\pgfqpoint{1.015322in}{1.601447in}}%
\pgfpathlineto{\pgfqpoint{1.018133in}{1.968139in}}%
\pgfpathlineto{\pgfqpoint{1.020944in}{1.349149in}}%
\pgfpathlineto{\pgfqpoint{1.023754in}{1.801115in}}%
\pgfpathlineto{\pgfqpoint{1.026565in}{1.552492in}}%
\pgfpathlineto{\pgfqpoint{1.029376in}{1.704206in}}%
\pgfpathlineto{\pgfqpoint{1.032187in}{1.502113in}}%
\pgfpathlineto{\pgfqpoint{1.034997in}{1.802575in}}%
\pgfpathlineto{\pgfqpoint{1.037808in}{2.001065in}}%
\pgfpathlineto{\pgfqpoint{1.040619in}{1.583647in}}%
\pgfpathlineto{\pgfqpoint{1.043429in}{1.607207in}}%
\pgfpathlineto{\pgfqpoint{1.046240in}{1.794954in}}%
\pgfpathlineto{\pgfqpoint{1.049051in}{1.619436in}}%
\pgfpathlineto{\pgfqpoint{1.051861in}{1.728359in}}%
\pgfpathlineto{\pgfqpoint{1.054672in}{1.649623in}}%
\pgfpathlineto{\pgfqpoint{1.057483in}{1.974522in}}%
\pgfpathlineto{\pgfqpoint{1.060293in}{1.385244in}}%
\pgfpathlineto{\pgfqpoint{1.063104in}{1.722367in}}%
\pgfpathlineto{\pgfqpoint{1.065915in}{1.764610in}}%
\pgfpathlineto{\pgfqpoint{1.068725in}{1.889913in}}%
\pgfpathlineto{\pgfqpoint{1.071536in}{1.857984in}}%
\pgfpathlineto{\pgfqpoint{1.074347in}{1.662516in}}%
\pgfpathlineto{\pgfqpoint{1.077158in}{1.626911in}}%
\pgfpathlineto{\pgfqpoint{1.079968in}{1.650446in}}%
\pgfpathlineto{\pgfqpoint{1.082779in}{1.560627in}}%
\pgfpathlineto{\pgfqpoint{1.085590in}{1.817470in}}%
\pgfpathlineto{\pgfqpoint{1.088400in}{1.875383in}}%
\pgfpathlineto{\pgfqpoint{1.091211in}{1.508663in}}%
\pgfpathlineto{\pgfqpoint{1.094022in}{1.957490in}}%
\pgfpathlineto{\pgfqpoint{1.096832in}{1.497790in}}%
\pgfpathlineto{\pgfqpoint{1.099643in}{1.326495in}}%
\pgfpathlineto{\pgfqpoint{1.105264in}{2.167132in}}%
\pgfpathlineto{\pgfqpoint{1.108075in}{1.964550in}}%
\pgfpathlineto{\pgfqpoint{1.110886in}{1.396081in}}%
\pgfpathlineto{\pgfqpoint{1.113696in}{1.855195in}}%
\pgfpathlineto{\pgfqpoint{1.116507in}{1.651390in}}%
\pgfpathlineto{\pgfqpoint{1.119318in}{1.616279in}}%
\pgfpathlineto{\pgfqpoint{1.122128in}{1.668643in}}%
\pgfpathlineto{\pgfqpoint{1.124939in}{1.313100in}}%
\pgfpathlineto{\pgfqpoint{1.127750in}{1.799528in}}%
\pgfpathlineto{\pgfqpoint{1.130561in}{2.062565in}}%
\pgfpathlineto{\pgfqpoint{1.133371in}{1.871173in}}%
\pgfpathlineto{\pgfqpoint{1.136182in}{1.732193in}}%
\pgfpathlineto{\pgfqpoint{1.138993in}{1.623096in}}%
\pgfpathlineto{\pgfqpoint{1.141803in}{1.709283in}}%
\pgfpathlineto{\pgfqpoint{1.144614in}{1.576592in}}%
\pgfpathlineto{\pgfqpoint{1.147425in}{0.938416in}}%
\pgfpathlineto{\pgfqpoint{1.150235in}{1.873351in}}%
\pgfpathlineto{\pgfqpoint{1.153046in}{1.787877in}}%
\pgfpathlineto{\pgfqpoint{1.155857in}{1.644170in}}%
\pgfpathlineto{\pgfqpoint{1.158667in}{1.595874in}}%
\pgfpathlineto{\pgfqpoint{1.161478in}{1.788172in}}%
\pgfpathlineto{\pgfqpoint{1.167099in}{1.529112in}}%
\pgfpathlineto{\pgfqpoint{1.169910in}{1.679888in}}%
\pgfpathlineto{\pgfqpoint{1.172721in}{1.932956in}}%
\pgfpathlineto{\pgfqpoint{1.175532in}{1.353665in}}%
\pgfpathlineto{\pgfqpoint{1.178342in}{1.364890in}}%
\pgfpathlineto{\pgfqpoint{1.181153in}{2.183208in}}%
\pgfpathlineto{\pgfqpoint{1.186774in}{1.792574in}}%
\pgfpathlineto{\pgfqpoint{1.189585in}{1.774346in}}%
\pgfpathlineto{\pgfqpoint{1.192396in}{1.639151in}}%
\pgfpathlineto{\pgfqpoint{1.195206in}{1.809275in}}%
\pgfpathlineto{\pgfqpoint{1.198017in}{1.325637in}}%
\pgfpathlineto{\pgfqpoint{1.200828in}{1.384057in}}%
\pgfpathlineto{\pgfqpoint{1.203638in}{1.290497in}}%
\pgfpathlineto{\pgfqpoint{1.206449in}{1.952648in}}%
\pgfpathlineto{\pgfqpoint{1.209260in}{1.642714in}}%
\pgfpathlineto{\pgfqpoint{1.212070in}{2.039993in}}%
\pgfpathlineto{\pgfqpoint{1.214881in}{1.752278in}}%
\pgfpathlineto{\pgfqpoint{1.217692in}{1.661924in}}%
\pgfpathlineto{\pgfqpoint{1.220503in}{1.661866in}}%
\pgfpathlineto{\pgfqpoint{1.223313in}{1.800316in}}%
\pgfpathlineto{\pgfqpoint{1.226124in}{1.389334in}}%
\pgfpathlineto{\pgfqpoint{1.228935in}{1.904520in}}%
\pgfpathlineto{\pgfqpoint{1.231745in}{1.943258in}}%
\pgfpathlineto{\pgfqpoint{1.234556in}{1.915897in}}%
\pgfpathlineto{\pgfqpoint{1.237367in}{1.854952in}}%
\pgfpathlineto{\pgfqpoint{1.240177in}{1.847538in}}%
\pgfpathlineto{\pgfqpoint{1.242988in}{1.291472in}}%
\pgfpathlineto{\pgfqpoint{1.245799in}{2.393411in}}%
\pgfpathlineto{\pgfqpoint{1.248609in}{1.224089in}}%
\pgfpathlineto{\pgfqpoint{1.251420in}{1.933449in}}%
\pgfpathlineto{\pgfqpoint{1.254231in}{1.726209in}}%
\pgfpathlineto{\pgfqpoint{1.257041in}{1.451130in}}%
\pgfpathlineto{\pgfqpoint{1.259852in}{1.961348in}}%
\pgfpathlineto{\pgfqpoint{1.262663in}{2.062546in}}%
\pgfpathlineto{\pgfqpoint{1.265474in}{1.658700in}}%
\pgfpathlineto{\pgfqpoint{1.268284in}{1.775197in}}%
\pgfpathlineto{\pgfqpoint{1.271095in}{1.474842in}}%
\pgfpathlineto{\pgfqpoint{1.273906in}{1.488830in}}%
\pgfpathlineto{\pgfqpoint{1.276716in}{1.520651in}}%
\pgfpathlineto{\pgfqpoint{1.279527in}{1.489681in}}%
\pgfpathlineto{\pgfqpoint{1.282338in}{1.940248in}}%
\pgfpathlineto{\pgfqpoint{1.285148in}{1.874220in}}%
\pgfpathlineto{\pgfqpoint{1.287959in}{2.072270in}}%
\pgfpathlineto{\pgfqpoint{1.290770in}{1.824046in}}%
\pgfpathlineto{\pgfqpoint{1.293580in}{1.642645in}}%
\pgfpathlineto{\pgfqpoint{1.296391in}{1.392429in}}%
\pgfpathlineto{\pgfqpoint{1.299202in}{1.736772in}}%
\pgfpathlineto{\pgfqpoint{1.302012in}{1.529194in}}%
\pgfpathlineto{\pgfqpoint{1.304823in}{1.464607in}}%
\pgfpathlineto{\pgfqpoint{1.307634in}{1.634647in}}%
\pgfpathlineto{\pgfqpoint{1.310445in}{2.262044in}}%
\pgfpathlineto{\pgfqpoint{1.313255in}{1.983016in}}%
\pgfpathlineto{\pgfqpoint{1.316066in}{1.897370in}}%
\pgfpathlineto{\pgfqpoint{1.318877in}{1.475793in}}%
\pgfpathlineto{\pgfqpoint{1.321687in}{1.500333in}}%
\pgfpathlineto{\pgfqpoint{1.324498in}{1.910906in}}%
\pgfpathlineto{\pgfqpoint{1.327309in}{1.897157in}}%
\pgfpathlineto{\pgfqpoint{1.330119in}{1.697385in}}%
\pgfpathlineto{\pgfqpoint{1.332930in}{1.616787in}}%
\pgfpathlineto{\pgfqpoint{1.335741in}{1.783334in}}%
\pgfpathlineto{\pgfqpoint{1.338551in}{1.372802in}}%
\pgfpathlineto{\pgfqpoint{1.341362in}{1.762923in}}%
\pgfpathlineto{\pgfqpoint{1.344173in}{1.599243in}}%
\pgfpathlineto{\pgfqpoint{1.346983in}{1.882443in}}%
\pgfpathlineto{\pgfqpoint{1.349794in}{1.697430in}}%
\pgfpathlineto{\pgfqpoint{1.352605in}{1.767519in}}%
\pgfpathlineto{\pgfqpoint{1.355415in}{1.567800in}}%
\pgfpathlineto{\pgfqpoint{1.358226in}{1.632260in}}%
\pgfpathlineto{\pgfqpoint{1.361037in}{1.610158in}}%
\pgfpathlineto{\pgfqpoint{1.363848in}{1.762997in}}%
\pgfpathlineto{\pgfqpoint{1.366658in}{1.577211in}}%
\pgfpathlineto{\pgfqpoint{1.369469in}{1.631545in}}%
\pgfpathlineto{\pgfqpoint{1.372280in}{1.812746in}}%
\pgfpathlineto{\pgfqpoint{1.375090in}{1.664668in}}%
\pgfpathlineto{\pgfqpoint{1.377901in}{1.763047in}}%
\pgfpathlineto{\pgfqpoint{1.380712in}{1.533139in}}%
\pgfpathlineto{\pgfqpoint{1.383522in}{1.883579in}}%
\pgfpathlineto{\pgfqpoint{1.389144in}{1.577612in}}%
\pgfpathlineto{\pgfqpoint{1.391954in}{1.925400in}}%
\pgfpathlineto{\pgfqpoint{1.394765in}{1.713556in}}%
\pgfpathlineto{\pgfqpoint{1.397576in}{1.567800in}}%
\pgfpathlineto{\pgfqpoint{1.400386in}{2.029959in}}%
\pgfpathlineto{\pgfqpoint{1.403197in}{1.923775in}}%
\pgfpathlineto{\pgfqpoint{1.406008in}{1.868776in}}%
\pgfpathlineto{\pgfqpoint{1.408819in}{1.609094in}}%
\pgfpathlineto{\pgfqpoint{1.411629in}{1.645181in}}%
\pgfpathlineto{\pgfqpoint{1.414440in}{1.671159in}}%
\pgfpathlineto{\pgfqpoint{1.417251in}{1.821966in}}%
\pgfpathlineto{\pgfqpoint{1.420061in}{1.671305in}}%
\pgfpathlineto{\pgfqpoint{1.422872in}{1.738601in}}%
\pgfpathlineto{\pgfqpoint{1.425683in}{1.541439in}}%
\pgfpathlineto{\pgfqpoint{1.428493in}{1.790846in}}%
\pgfpathlineto{\pgfqpoint{1.431304in}{1.451683in}}%
\pgfpathlineto{\pgfqpoint{1.434115in}{1.854397in}}%
\pgfpathlineto{\pgfqpoint{1.436925in}{1.738849in}}%
\pgfpathlineto{\pgfqpoint{1.439736in}{1.970068in}}%
\pgfpathlineto{\pgfqpoint{1.442547in}{1.610155in}}%
\pgfpathlineto{\pgfqpoint{1.445357in}{1.976703in}}%
\pgfpathlineto{\pgfqpoint{1.448168in}{1.822451in}}%
\pgfpathlineto{\pgfqpoint{1.450979in}{1.756830in}}%
\pgfpathlineto{\pgfqpoint{1.453790in}{1.521954in}}%
\pgfpathlineto{\pgfqpoint{1.456600in}{1.469147in}}%
\pgfpathlineto{\pgfqpoint{1.459411in}{1.768362in}}%
\pgfpathlineto{\pgfqpoint{1.462222in}{1.873726in}}%
\pgfpathlineto{\pgfqpoint{1.465032in}{1.702056in}}%
\pgfpathlineto{\pgfqpoint{1.467843in}{1.777043in}}%
\pgfpathlineto{\pgfqpoint{1.470654in}{1.831054in}}%
\pgfpathlineto{\pgfqpoint{1.473464in}{1.849414in}}%
\pgfpathlineto{\pgfqpoint{1.476275in}{1.618444in}}%
\pgfpathlineto{\pgfqpoint{1.479086in}{1.479153in}}%
\pgfpathlineto{\pgfqpoint{1.481896in}{1.581979in}}%
\pgfpathlineto{\pgfqpoint{1.484707in}{1.423687in}}%
\pgfpathlineto{\pgfqpoint{1.487518in}{1.676695in}}%
\pgfpathlineto{\pgfqpoint{1.490328in}{1.814225in}}%
\pgfpathlineto{\pgfqpoint{1.495950in}{1.742382in}}%
\pgfpathlineto{\pgfqpoint{1.498761in}{2.254912in}}%
\pgfpathlineto{\pgfqpoint{1.501571in}{1.638586in}}%
\pgfpathlineto{\pgfqpoint{1.504382in}{1.486289in}}%
\pgfpathlineto{\pgfqpoint{1.507193in}{1.951194in}}%
\pgfpathlineto{\pgfqpoint{1.510003in}{1.550746in}}%
\pgfpathlineto{\pgfqpoint{1.512814in}{1.638099in}}%
\pgfpathlineto{\pgfqpoint{1.515625in}{1.785048in}}%
\pgfpathlineto{\pgfqpoint{1.518435in}{1.726156in}}%
\pgfpathlineto{\pgfqpoint{1.521246in}{1.604101in}}%
\pgfpathlineto{\pgfqpoint{1.524057in}{1.564143in}}%
\pgfpathlineto{\pgfqpoint{1.526867in}{1.696961in}}%
\pgfpathlineto{\pgfqpoint{1.529678in}{1.990316in}}%
\pgfpathlineto{\pgfqpoint{1.532489in}{2.008471in}}%
\pgfpathlineto{\pgfqpoint{1.535299in}{1.767781in}}%
\pgfpathlineto{\pgfqpoint{1.538110in}{1.725052in}}%
\pgfpathlineto{\pgfqpoint{1.540921in}{1.846796in}}%
\pgfpathlineto{\pgfqpoint{1.543731in}{1.598405in}}%
\pgfpathlineto{\pgfqpoint{1.546542in}{1.645002in}}%
\pgfpathlineto{\pgfqpoint{1.549353in}{1.743738in}}%
\pgfpathlineto{\pgfqpoint{1.552164in}{1.785613in}}%
\pgfpathlineto{\pgfqpoint{1.554974in}{1.854467in}}%
\pgfpathlineto{\pgfqpoint{1.557785in}{1.580815in}}%
\pgfpathlineto{\pgfqpoint{1.560596in}{1.580017in}}%
\pgfpathlineto{\pgfqpoint{1.563406in}{1.743459in}}%
\pgfpathlineto{\pgfqpoint{1.566217in}{1.733981in}}%
\pgfpathlineto{\pgfqpoint{1.569028in}{1.701332in}}%
\pgfpathlineto{\pgfqpoint{1.571838in}{1.701326in}}%
\pgfpathlineto{\pgfqpoint{1.574649in}{1.719899in}}%
\pgfpathlineto{\pgfqpoint{1.577460in}{1.584892in}}%
\pgfpathlineto{\pgfqpoint{1.583081in}{1.654406in}}%
\pgfpathlineto{\pgfqpoint{1.585892in}{1.934925in}}%
\pgfpathlineto{\pgfqpoint{1.588702in}{1.857998in}}%
\pgfpathlineto{\pgfqpoint{1.591513in}{1.530682in}}%
\pgfpathlineto{\pgfqpoint{1.594324in}{2.031200in}}%
\pgfpathlineto{\pgfqpoint{1.597135in}{1.518772in}}%
\pgfpathlineto{\pgfqpoint{1.599945in}{1.733230in}}%
\pgfpathlineto{\pgfqpoint{1.602756in}{1.545011in}}%
\pgfpathlineto{\pgfqpoint{1.605567in}{1.655053in}}%
\pgfpathlineto{\pgfqpoint{1.608377in}{1.613169in}}%
\pgfpathlineto{\pgfqpoint{1.611188in}{1.995958in}}%
\pgfpathlineto{\pgfqpoint{1.613999in}{2.079381in}}%
\pgfpathlineto{\pgfqpoint{1.616809in}{1.705363in}}%
\pgfpathlineto{\pgfqpoint{1.619620in}{1.815987in}}%
\pgfpathlineto{\pgfqpoint{1.622431in}{1.806258in}}%
\pgfpathlineto{\pgfqpoint{1.625241in}{1.835819in}}%
\pgfpathlineto{\pgfqpoint{1.628052in}{1.928778in}}%
\pgfpathlineto{\pgfqpoint{1.630863in}{1.567636in}}%
\pgfpathlineto{\pgfqpoint{1.633673in}{1.752195in}}%
\pgfpathlineto{\pgfqpoint{1.636484in}{1.644764in}}%
\pgfpathlineto{\pgfqpoint{1.639295in}{1.670494in}}%
\pgfpathlineto{\pgfqpoint{1.642106in}{1.670465in}}%
\pgfpathlineto{\pgfqpoint{1.644916in}{1.661791in}}%
\pgfpathlineto{\pgfqpoint{1.647727in}{1.531246in}}%
\pgfpathlineto{\pgfqpoint{1.650538in}{1.770446in}}%
\pgfpathlineto{\pgfqpoint{1.653348in}{1.692023in}}%
\pgfpathlineto{\pgfqpoint{1.656159in}{1.739755in}}%
\pgfpathlineto{\pgfqpoint{1.658970in}{1.722322in}}%
\pgfpathlineto{\pgfqpoint{1.661780in}{1.346615in}}%
\pgfpathlineto{\pgfqpoint{1.664591in}{1.722939in}}%
\pgfpathlineto{\pgfqpoint{1.667402in}{1.527910in}}%
\pgfpathlineto{\pgfqpoint{1.670212in}{1.979193in}}%
\pgfpathlineto{\pgfqpoint{1.675834in}{1.503440in}}%
\pgfpathlineto{\pgfqpoint{1.678644in}{1.976216in}}%
\pgfpathlineto{\pgfqpoint{1.681455in}{1.817166in}}%
\pgfpathlineto{\pgfqpoint{1.684266in}{1.322347in}}%
\pgfpathlineto{\pgfqpoint{1.687077in}{1.766662in}}%
\pgfpathlineto{\pgfqpoint{1.689887in}{1.542386in}}%
\pgfpathlineto{\pgfqpoint{1.692698in}{1.740592in}}%
\pgfpathlineto{\pgfqpoint{1.695509in}{1.621327in}}%
\pgfpathlineto{\pgfqpoint{1.698319in}{1.811137in}}%
\pgfpathlineto{\pgfqpoint{1.701130in}{1.892888in}}%
\pgfpathlineto{\pgfqpoint{1.703941in}{1.657275in}}%
\pgfpathlineto{\pgfqpoint{1.706751in}{1.473078in}}%
\pgfpathlineto{\pgfqpoint{1.709562in}{1.893565in}}%
\pgfpathlineto{\pgfqpoint{1.712373in}{1.886697in}}%
\pgfpathlineto{\pgfqpoint{1.715183in}{1.339843in}}%
\pgfpathlineto{\pgfqpoint{1.717994in}{1.888659in}}%
\pgfpathlineto{\pgfqpoint{1.720805in}{1.890500in}}%
\pgfpathlineto{\pgfqpoint{1.723615in}{1.571512in}}%
\pgfpathlineto{\pgfqpoint{1.726426in}{1.635753in}}%
\pgfpathlineto{\pgfqpoint{1.729237in}{1.834467in}}%
\pgfpathlineto{\pgfqpoint{1.732048in}{1.769175in}}%
\pgfpathlineto{\pgfqpoint{1.734858in}{1.764540in}}%
\pgfpathlineto{\pgfqpoint{1.737669in}{1.640869in}}%
\pgfpathlineto{\pgfqpoint{1.740480in}{1.704827in}}%
\pgfpathlineto{\pgfqpoint{1.743290in}{1.743135in}}%
\pgfpathlineto{\pgfqpoint{1.746101in}{1.555351in}}%
\pgfpathlineto{\pgfqpoint{1.748912in}{1.653367in}}%
\pgfpathlineto{\pgfqpoint{1.751722in}{1.816239in}}%
\pgfpathlineto{\pgfqpoint{1.754533in}{1.485885in}}%
\pgfpathlineto{\pgfqpoint{1.757344in}{1.644397in}}%
\pgfpathlineto{\pgfqpoint{1.760154in}{1.486877in}}%
\pgfpathlineto{\pgfqpoint{1.762965in}{2.022416in}}%
\pgfpathlineto{\pgfqpoint{1.765776in}{1.540737in}}%
\pgfpathlineto{\pgfqpoint{1.768586in}{1.916265in}}%
\pgfpathlineto{\pgfqpoint{1.771397in}{2.068500in}}%
\pgfpathlineto{\pgfqpoint{1.774208in}{2.018757in}}%
\pgfpathlineto{\pgfqpoint{1.777018in}{1.802141in}}%
\pgfpathlineto{\pgfqpoint{1.779829in}{1.801370in}}%
\pgfpathlineto{\pgfqpoint{1.782640in}{1.460275in}}%
\pgfpathlineto{\pgfqpoint{1.785451in}{1.749263in}}%
\pgfpathlineto{\pgfqpoint{1.788261in}{1.391222in}}%
\pgfpathlineto{\pgfqpoint{1.791072in}{1.853446in}}%
\pgfpathlineto{\pgfqpoint{1.793883in}{1.568000in}}%
\pgfpathlineto{\pgfqpoint{1.796693in}{1.799566in}}%
\pgfpathlineto{\pgfqpoint{1.799504in}{1.547080in}}%
\pgfpathlineto{\pgfqpoint{1.802315in}{1.902780in}}%
\pgfpathlineto{\pgfqpoint{1.805125in}{1.749383in}}%
\pgfpathlineto{\pgfqpoint{1.810747in}{1.264431in}}%
\pgfpathlineto{\pgfqpoint{1.813557in}{2.020054in}}%
\pgfpathlineto{\pgfqpoint{1.816368in}{1.457214in}}%
\pgfpathlineto{\pgfqpoint{1.819179in}{1.611546in}}%
\pgfpathlineto{\pgfqpoint{1.821989in}{1.936400in}}%
\pgfpathlineto{\pgfqpoint{1.824800in}{1.650161in}}%
\pgfpathlineto{\pgfqpoint{1.827611in}{1.896082in}}%
\pgfpathlineto{\pgfqpoint{1.830422in}{1.823860in}}%
\pgfpathlineto{\pgfqpoint{1.833232in}{1.847147in}}%
\pgfpathlineto{\pgfqpoint{1.836043in}{1.553301in}}%
\pgfpathlineto{\pgfqpoint{1.838854in}{1.753376in}}%
\pgfpathlineto{\pgfqpoint{1.841664in}{1.745001in}}%
\pgfpathlineto{\pgfqpoint{1.844475in}{1.462517in}}%
\pgfpathlineto{\pgfqpoint{1.847286in}{2.551440in}}%
\pgfpathlineto{\pgfqpoint{1.852907in}{1.634654in}}%
\pgfpathlineto{\pgfqpoint{1.855718in}{1.772280in}}%
\pgfpathlineto{\pgfqpoint{1.858528in}{1.642389in}}%
\pgfpathlineto{\pgfqpoint{1.861339in}{1.634556in}}%
\pgfpathlineto{\pgfqpoint{1.864150in}{1.707368in}}%
\pgfpathlineto{\pgfqpoint{1.866960in}{1.707353in}}%
\pgfpathlineto{\pgfqpoint{1.872582in}{1.929656in}}%
\pgfpathlineto{\pgfqpoint{1.875393in}{1.522655in}}%
\pgfpathlineto{\pgfqpoint{1.878203in}{2.099146in}}%
\pgfpathlineto{\pgfqpoint{1.881014in}{1.439287in}}%
\pgfpathlineto{\pgfqpoint{1.883825in}{1.688276in}}%
\pgfpathlineto{\pgfqpoint{1.886635in}{1.571628in}}%
\pgfpathlineto{\pgfqpoint{1.889446in}{1.513492in}}%
\pgfpathlineto{\pgfqpoint{1.892257in}{1.896970in}}%
\pgfpathlineto{\pgfqpoint{1.895067in}{1.654272in}}%
\pgfpathlineto{\pgfqpoint{1.897878in}{1.528646in}}%
\pgfpathlineto{\pgfqpoint{1.900689in}{1.942090in}}%
\pgfpathlineto{\pgfqpoint{1.903499in}{1.441956in}}%
\pgfpathlineto{\pgfqpoint{1.906310in}{1.870511in}}%
\pgfpathlineto{\pgfqpoint{1.909121in}{1.692023in}}%
\pgfpathlineto{\pgfqpoint{1.911931in}{1.433120in}}%
\pgfpathlineto{\pgfqpoint{1.914742in}{1.879016in}}%
\pgfpathlineto{\pgfqpoint{1.917553in}{1.741260in}}%
\pgfpathlineto{\pgfqpoint{1.920364in}{1.898630in}}%
\pgfpathlineto{\pgfqpoint{1.923174in}{1.462710in}}%
\pgfpathlineto{\pgfqpoint{1.925985in}{1.646513in}}%
\pgfpathlineto{\pgfqpoint{1.928796in}{1.959380in}}%
\pgfpathlineto{\pgfqpoint{1.931606in}{1.579613in}}%
\pgfpathlineto{\pgfqpoint{1.934417in}{1.811898in}}%
\pgfpathlineto{\pgfqpoint{1.937228in}{1.836886in}}%
\pgfpathlineto{\pgfqpoint{1.942849in}{1.330213in}}%
\pgfpathlineto{\pgfqpoint{1.945660in}{1.783265in}}%
\pgfpathlineto{\pgfqpoint{1.948470in}{1.585522in}}%
\pgfpathlineto{\pgfqpoint{1.951281in}{1.881696in}}%
\pgfpathlineto{\pgfqpoint{1.954092in}{1.879419in}}%
\pgfpathlineto{\pgfqpoint{1.956902in}{1.829261in}}%
\pgfpathlineto{\pgfqpoint{1.959713in}{1.565957in}}%
\pgfpathlineto{\pgfqpoint{1.962524in}{1.899122in}}%
\pgfpathlineto{\pgfqpoint{1.965334in}{1.695696in}}%
\pgfpathlineto{\pgfqpoint{1.970956in}{2.001779in}}%
\pgfpathlineto{\pgfqpoint{1.973767in}{1.469331in}}%
\pgfpathlineto{\pgfqpoint{1.976577in}{1.619504in}}%
\pgfpathlineto{\pgfqpoint{1.979388in}{1.648349in}}%
\pgfpathlineto{\pgfqpoint{1.982199in}{1.963001in}}%
\pgfpathlineto{\pgfqpoint{1.985009in}{1.752796in}}%
\pgfpathlineto{\pgfqpoint{1.987820in}{1.656303in}}%
\pgfpathlineto{\pgfqpoint{1.990631in}{1.620336in}}%
\pgfpathlineto{\pgfqpoint{1.996252in}{1.802845in}}%
\pgfpathlineto{\pgfqpoint{1.999063in}{1.602712in}}%
\pgfpathlineto{\pgfqpoint{2.001873in}{1.799135in}}%
\pgfpathlineto{\pgfqpoint{2.004684in}{1.473451in}}%
\pgfpathlineto{\pgfqpoint{2.007495in}{1.539731in}}%
\pgfpathlineto{\pgfqpoint{2.010305in}{2.337985in}}%
\pgfpathlineto{\pgfqpoint{2.013116in}{1.576226in}}%
\pgfpathlineto{\pgfqpoint{2.015927in}{1.593093in}}%
\pgfpathlineto{\pgfqpoint{2.018738in}{1.649432in}}%
\pgfpathlineto{\pgfqpoint{2.021548in}{0.468908in}}%
\pgfpathlineto{\pgfqpoint{2.024359in}{1.879245in}}%
\pgfpathlineto{\pgfqpoint{2.027170in}{2.108011in}}%
\pgfpathlineto{\pgfqpoint{2.029980in}{1.740025in}}%
\pgfpathlineto{\pgfqpoint{2.032791in}{1.551293in}}%
\pgfpathlineto{\pgfqpoint{2.035602in}{1.561271in}}%
\pgfpathlineto{\pgfqpoint{2.038412in}{1.639414in}}%
\pgfpathlineto{\pgfqpoint{2.041223in}{1.608991in}}%
\pgfpathlineto{\pgfqpoint{2.044034in}{1.688238in}}%
\pgfpathlineto{\pgfqpoint{2.046844in}{1.646535in}}%
\pgfpathlineto{\pgfqpoint{2.049655in}{1.729939in}}%
\pgfpathlineto{\pgfqpoint{2.052466in}{1.300891in}}%
\pgfpathlineto{\pgfqpoint{2.055276in}{1.618069in}}%
\pgfpathlineto{\pgfqpoint{2.058087in}{1.858894in}}%
\pgfpathlineto{\pgfqpoint{2.060898in}{1.532953in}}%
\pgfpathlineto{\pgfqpoint{2.063709in}{2.153094in}}%
\pgfpathlineto{\pgfqpoint{2.066519in}{1.737390in}}%
\pgfpathlineto{\pgfqpoint{2.069330in}{1.707116in}}%
\pgfpathlineto{\pgfqpoint{2.072141in}{1.333421in}}%
\pgfpathlineto{\pgfqpoint{2.074951in}{1.614650in}}%
\pgfpathlineto{\pgfqpoint{2.077762in}{1.784825in}}%
\pgfpathlineto{\pgfqpoint{2.080573in}{1.707436in}}%
\pgfpathlineto{\pgfqpoint{2.083383in}{1.607060in}}%
\pgfpathlineto{\pgfqpoint{2.086194in}{1.922660in}}%
\pgfpathlineto{\pgfqpoint{2.089005in}{1.619356in}}%
\pgfpathlineto{\pgfqpoint{2.091815in}{1.695856in}}%
\pgfpathlineto{\pgfqpoint{2.094626in}{1.730301in}}%
\pgfpathlineto{\pgfqpoint{2.097437in}{1.859342in}}%
\pgfpathlineto{\pgfqpoint{2.100247in}{2.209113in}}%
\pgfpathlineto{\pgfqpoint{2.103058in}{1.837706in}}%
\pgfpathlineto{\pgfqpoint{2.105869in}{1.586539in}}%
\pgfpathlineto{\pgfqpoint{2.108680in}{2.020661in}}%
\pgfpathlineto{\pgfqpoint{2.111490in}{1.724153in}}%
\pgfpathlineto{\pgfqpoint{2.114301in}{1.872853in}}%
\pgfpathlineto{\pgfqpoint{2.117112in}{1.860323in}}%
\pgfpathlineto{\pgfqpoint{2.119922in}{1.775497in}}%
\pgfpathlineto{\pgfqpoint{2.122733in}{2.018128in}}%
\pgfpathlineto{\pgfqpoint{2.125544in}{1.486856in}}%
\pgfpathlineto{\pgfqpoint{2.128354in}{1.459687in}}%
\pgfpathlineto{\pgfqpoint{2.131165in}{1.548122in}}%
\pgfpathlineto{\pgfqpoint{2.133976in}{1.853382in}}%
\pgfpathlineto{\pgfqpoint{2.136786in}{1.653588in}}%
\pgfpathlineto{\pgfqpoint{2.139597in}{1.533798in}}%
\pgfpathlineto{\pgfqpoint{2.142408in}{1.864235in}}%
\pgfpathlineto{\pgfqpoint{2.145218in}{1.576253in}}%
\pgfpathlineto{\pgfqpoint{2.148029in}{1.432832in}}%
\pgfpathlineto{\pgfqpoint{2.150840in}{1.841672in}}%
\pgfpathlineto{\pgfqpoint{2.153651in}{1.347835in}}%
\pgfpathlineto{\pgfqpoint{2.156461in}{1.373407in}}%
\pgfpathlineto{\pgfqpoint{2.159272in}{1.802627in}}%
\pgfpathlineto{\pgfqpoint{2.162083in}{2.126762in}}%
\pgfpathlineto{\pgfqpoint{2.167704in}{1.793601in}}%
\pgfpathlineto{\pgfqpoint{2.170515in}{1.523534in}}%
\pgfpathlineto{\pgfqpoint{2.173325in}{2.034108in}}%
\pgfpathlineto{\pgfqpoint{2.176136in}{1.904634in}}%
\pgfpathlineto{\pgfqpoint{2.178947in}{1.861399in}}%
\pgfpathlineto{\pgfqpoint{2.181757in}{1.658295in}}%
\pgfpathlineto{\pgfqpoint{2.184568in}{1.688646in}}%
\pgfpathlineto{\pgfqpoint{2.187379in}{1.607363in}}%
\pgfpathlineto{\pgfqpoint{2.190189in}{2.004660in}}%
\pgfpathlineto{\pgfqpoint{2.193000in}{1.702004in}}%
\pgfpathlineto{\pgfqpoint{2.195811in}{1.692023in}}%
\pgfpathlineto{\pgfqpoint{2.198621in}{1.781571in}}%
\pgfpathlineto{\pgfqpoint{2.201432in}{1.662231in}}%
\pgfpathlineto{\pgfqpoint{2.204243in}{1.132071in}}%
\pgfpathlineto{\pgfqpoint{2.207054in}{1.889960in}}%
\pgfpathlineto{\pgfqpoint{2.209864in}{1.476904in}}%
\pgfpathlineto{\pgfqpoint{2.212675in}{1.767481in}}%
\pgfpathlineto{\pgfqpoint{2.215486in}{1.821507in}}%
\pgfpathlineto{\pgfqpoint{2.218296in}{1.459551in}}%
\pgfpathlineto{\pgfqpoint{2.221107in}{1.866701in}}%
\pgfpathlineto{\pgfqpoint{2.223918in}{1.766775in}}%
\pgfpathlineto{\pgfqpoint{2.226728in}{1.613865in}}%
\pgfpathlineto{\pgfqpoint{2.229539in}{1.938752in}}%
\pgfpathlineto{\pgfqpoint{2.232350in}{1.718826in}}%
\pgfpathlineto{\pgfqpoint{2.235160in}{1.835306in}}%
\pgfpathlineto{\pgfqpoint{2.237971in}{1.531993in}}%
\pgfpathlineto{\pgfqpoint{2.240782in}{1.493044in}}%
\pgfpathlineto{\pgfqpoint{2.246403in}{1.963557in}}%
\pgfpathlineto{\pgfqpoint{2.249214in}{1.731851in}}%
\pgfpathlineto{\pgfqpoint{2.252025in}{1.678758in}}%
\pgfpathlineto{\pgfqpoint{2.254835in}{1.814295in}}%
\pgfpathlineto{\pgfqpoint{2.260457in}{1.649271in}}%
\pgfpathlineto{\pgfqpoint{2.263267in}{1.819929in}}%
\pgfpathlineto{\pgfqpoint{2.266078in}{1.435153in}}%
\pgfpathlineto{\pgfqpoint{2.268889in}{1.768207in}}%
\pgfpathlineto{\pgfqpoint{2.271699in}{1.612518in}}%
\pgfpathlineto{\pgfqpoint{2.274510in}{1.701983in}}%
\pgfpathlineto{\pgfqpoint{2.277321in}{1.672096in}}%
\pgfpathlineto{\pgfqpoint{2.280131in}{1.524968in}}%
\pgfpathlineto{\pgfqpoint{2.282942in}{2.171639in}}%
\pgfpathlineto{\pgfqpoint{2.285753in}{1.521713in}}%
\pgfpathlineto{\pgfqpoint{2.288563in}{1.988857in}}%
\pgfpathlineto{\pgfqpoint{2.291374in}{1.721076in}}%
\pgfpathlineto{\pgfqpoint{2.294185in}{2.102609in}}%
\pgfpathlineto{\pgfqpoint{2.296996in}{1.842083in}}%
\pgfpathlineto{\pgfqpoint{2.299806in}{1.744825in}}%
\pgfpathlineto{\pgfqpoint{2.302617in}{1.685820in}}%
\pgfpathlineto{\pgfqpoint{2.305428in}{1.840213in}}%
\pgfpathlineto{\pgfqpoint{2.308238in}{1.741109in}}%
\pgfpathlineto{\pgfqpoint{2.311049in}{1.795820in}}%
\pgfpathlineto{\pgfqpoint{2.313860in}{1.655467in}}%
\pgfpathlineto{\pgfqpoint{2.316670in}{1.780219in}}%
\pgfpathlineto{\pgfqpoint{2.319481in}{1.815870in}}%
\pgfpathlineto{\pgfqpoint{2.322292in}{1.574240in}}%
\pgfpathlineto{\pgfqpoint{2.325102in}{1.704144in}}%
\pgfpathlineto{\pgfqpoint{2.327913in}{1.597836in}}%
\pgfpathlineto{\pgfqpoint{2.330724in}{1.810424in}}%
\pgfpathlineto{\pgfqpoint{2.336345in}{1.670903in}}%
\pgfpathlineto{\pgfqpoint{2.339156in}{1.637583in}}%
\pgfpathlineto{\pgfqpoint{2.341967in}{1.533676in}}%
\pgfpathlineto{\pgfqpoint{2.344777in}{1.959060in}}%
\pgfpathlineto{\pgfqpoint{2.350399in}{1.553675in}}%
\pgfpathlineto{\pgfqpoint{2.353209in}{2.409367in}}%
\pgfpathlineto{\pgfqpoint{2.356020in}{1.674702in}}%
\pgfpathlineto{\pgfqpoint{2.358831in}{1.761191in}}%
\pgfpathlineto{\pgfqpoint{2.361641in}{1.372452in}}%
\pgfpathlineto{\pgfqpoint{2.364452in}{1.206109in}}%
\pgfpathlineto{\pgfqpoint{2.367263in}{1.333668in}}%
\pgfpathlineto{\pgfqpoint{2.370073in}{2.029168in}}%
\pgfpathlineto{\pgfqpoint{2.372884in}{1.419807in}}%
\pgfpathlineto{\pgfqpoint{2.375695in}{1.958173in}}%
\pgfpathlineto{\pgfqpoint{2.378505in}{1.301965in}}%
\pgfpathlineto{\pgfqpoint{2.384127in}{1.770281in}}%
\pgfpathlineto{\pgfqpoint{2.386937in}{1.769890in}}%
\pgfpathlineto{\pgfqpoint{2.389748in}{1.935561in}}%
\pgfpathlineto{\pgfqpoint{2.392559in}{1.886570in}}%
\pgfpathlineto{\pgfqpoint{2.395370in}{1.607207in}}%
\pgfpathlineto{\pgfqpoint{2.398180in}{1.816062in}}%
\pgfpathlineto{\pgfqpoint{2.400991in}{1.934215in}}%
\pgfpathlineto{\pgfqpoint{2.403802in}{1.683120in}}%
\pgfpathlineto{\pgfqpoint{2.406612in}{1.827986in}}%
\pgfpathlineto{\pgfqpoint{2.412234in}{1.541238in}}%
\pgfpathlineto{\pgfqpoint{2.415044in}{1.662286in}}%
\pgfpathlineto{\pgfqpoint{2.417855in}{1.686069in}}%
\pgfpathlineto{\pgfqpoint{2.420666in}{1.887324in}}%
\pgfpathlineto{\pgfqpoint{2.426287in}{1.621579in}}%
\pgfpathlineto{\pgfqpoint{2.429098in}{1.703786in}}%
\pgfpathlineto{\pgfqpoint{2.431908in}{1.668489in}}%
\pgfpathlineto{\pgfqpoint{2.434719in}{1.379801in}}%
\pgfpathlineto{\pgfqpoint{2.437530in}{1.974777in}}%
\pgfpathlineto{\pgfqpoint{2.440341in}{1.505085in}}%
\pgfpathlineto{\pgfqpoint{2.443151in}{1.763502in}}%
\pgfpathlineto{\pgfqpoint{2.445962in}{1.810431in}}%
\pgfpathlineto{\pgfqpoint{2.448773in}{1.689074in}}%
\pgfpathlineto{\pgfqpoint{2.451583in}{1.806634in}}%
\pgfpathlineto{\pgfqpoint{2.454394in}{1.618651in}}%
\pgfpathlineto{\pgfqpoint{2.457205in}{1.319899in}}%
\pgfpathlineto{\pgfqpoint{2.460015in}{1.692023in}}%
\pgfpathlineto{\pgfqpoint{2.462826in}{1.904477in}}%
\pgfpathlineto{\pgfqpoint{2.465637in}{1.872272in}}%
\pgfpathlineto{\pgfqpoint{2.468447in}{1.514746in}}%
\pgfpathlineto{\pgfqpoint{2.471258in}{1.551735in}}%
\pgfpathlineto{\pgfqpoint{2.474069in}{1.799590in}}%
\pgfpathlineto{\pgfqpoint{2.476879in}{1.509318in}}%
\pgfpathlineto{\pgfqpoint{2.479690in}{1.522389in}}%
\pgfpathlineto{\pgfqpoint{2.482501in}{1.498955in}}%
\pgfpathlineto{\pgfqpoint{2.485312in}{1.698189in}}%
\pgfpathlineto{\pgfqpoint{2.488122in}{1.421524in}}%
\pgfpathlineto{\pgfqpoint{2.490933in}{1.968685in}}%
\pgfpathlineto{\pgfqpoint{2.493744in}{1.608609in}}%
\pgfpathlineto{\pgfqpoint{2.496554in}{1.688925in}}%
\pgfpathlineto{\pgfqpoint{2.499365in}{1.716790in}}%
\pgfpathlineto{\pgfqpoint{2.502176in}{1.150679in}}%
\pgfpathlineto{\pgfqpoint{2.507797in}{1.626492in}}%
\pgfpathlineto{\pgfqpoint{2.510608in}{2.071532in}}%
\pgfpathlineto{\pgfqpoint{2.513418in}{1.233512in}}%
\pgfpathlineto{\pgfqpoint{2.516229in}{1.307822in}}%
\pgfpathlineto{\pgfqpoint{2.519040in}{2.033264in}}%
\pgfpathlineto{\pgfqpoint{2.521850in}{1.928484in}}%
\pgfpathlineto{\pgfqpoint{2.524661in}{2.091047in}}%
\pgfpathlineto{\pgfqpoint{2.527472in}{1.586823in}}%
\pgfpathlineto{\pgfqpoint{2.530283in}{1.781329in}}%
\pgfpathlineto{\pgfqpoint{2.533093in}{1.755501in}}%
\pgfpathlineto{\pgfqpoint{2.535904in}{1.606266in}}%
\pgfpathlineto{\pgfqpoint{2.538715in}{1.733372in}}%
\pgfpathlineto{\pgfqpoint{2.541525in}{0.893706in}}%
\pgfpathlineto{\pgfqpoint{2.544336in}{1.884735in}}%
\pgfpathlineto{\pgfqpoint{2.549957in}{1.688741in}}%
\pgfpathlineto{\pgfqpoint{2.552768in}{1.958912in}}%
\pgfpathlineto{\pgfqpoint{2.555579in}{1.562403in}}%
\pgfpathlineto{\pgfqpoint{2.558389in}{1.898901in}}%
\pgfpathlineto{\pgfqpoint{2.561200in}{1.556569in}}%
\pgfpathlineto{\pgfqpoint{2.564011in}{1.945839in}}%
\pgfpathlineto{\pgfqpoint{2.566821in}{1.853713in}}%
\pgfpathlineto{\pgfqpoint{2.569632in}{1.685714in}}%
\pgfpathlineto{\pgfqpoint{2.572443in}{1.758144in}}%
\pgfpathlineto{\pgfqpoint{2.575253in}{1.720275in}}%
\pgfpathlineto{\pgfqpoint{2.578064in}{1.644908in}}%
\pgfpathlineto{\pgfqpoint{2.580875in}{1.511676in}}%
\pgfpathlineto{\pgfqpoint{2.583686in}{1.869223in}}%
\pgfpathlineto{\pgfqpoint{2.586496in}{1.732873in}}%
\pgfpathlineto{\pgfqpoint{2.589307in}{1.512114in}}%
\pgfpathlineto{\pgfqpoint{2.592118in}{1.853091in}}%
\pgfpathlineto{\pgfqpoint{2.594928in}{1.638518in}}%
\pgfpathlineto{\pgfqpoint{2.597739in}{1.892506in}}%
\pgfpathlineto{\pgfqpoint{2.600550in}{1.868435in}}%
\pgfpathlineto{\pgfqpoint{2.603360in}{1.642705in}}%
\pgfpathlineto{\pgfqpoint{2.606171in}{1.753646in}}%
\pgfpathlineto{\pgfqpoint{2.608982in}{1.704318in}}%
\pgfpathlineto{\pgfqpoint{2.611792in}{1.596478in}}%
\pgfpathlineto{\pgfqpoint{2.614603in}{1.530429in}}%
\pgfpathlineto{\pgfqpoint{2.617414in}{1.701391in}}%
\pgfpathlineto{\pgfqpoint{2.620224in}{1.748115in}}%
\pgfpathlineto{\pgfqpoint{2.623035in}{1.751014in}}%
\pgfpathlineto{\pgfqpoint{2.625846in}{1.661003in}}%
\pgfpathlineto{\pgfqpoint{2.628657in}{1.812654in}}%
\pgfpathlineto{\pgfqpoint{2.631467in}{1.577601in}}%
\pgfpathlineto{\pgfqpoint{2.634278in}{1.626706in}}%
\pgfpathlineto{\pgfqpoint{2.637089in}{1.654576in}}%
\pgfpathlineto{\pgfqpoint{2.639899in}{1.613718in}}%
\pgfpathlineto{\pgfqpoint{2.642710in}{1.735922in}}%
\pgfpathlineto{\pgfqpoint{2.645521in}{1.723304in}}%
\pgfpathlineto{\pgfqpoint{2.648331in}{1.623122in}}%
\pgfpathlineto{\pgfqpoint{2.651142in}{1.632271in}}%
\pgfpathlineto{\pgfqpoint{2.653953in}{1.688872in}}%
\pgfpathlineto{\pgfqpoint{2.656763in}{1.568629in}}%
\pgfpathlineto{\pgfqpoint{2.659574in}{1.787028in}}%
\pgfpathlineto{\pgfqpoint{2.662385in}{1.682548in}}%
\pgfpathlineto{\pgfqpoint{2.665195in}{1.717276in}}%
\pgfpathlineto{\pgfqpoint{2.668006in}{1.798893in}}%
\pgfpathlineto{\pgfqpoint{2.670817in}{1.949879in}}%
\pgfpathlineto{\pgfqpoint{2.673628in}{1.735100in}}%
\pgfpathlineto{\pgfqpoint{2.676438in}{1.814449in}}%
\pgfpathlineto{\pgfqpoint{2.682060in}{1.585058in}}%
\pgfpathlineto{\pgfqpoint{2.684870in}{1.756290in}}%
\pgfpathlineto{\pgfqpoint{2.687681in}{1.649208in}}%
\pgfpathlineto{\pgfqpoint{2.690492in}{1.795800in}}%
\pgfpathlineto{\pgfqpoint{2.693302in}{1.981369in}}%
\pgfpathlineto{\pgfqpoint{2.696113in}{1.742707in}}%
\pgfpathlineto{\pgfqpoint{2.698924in}{1.754406in}}%
\pgfpathlineto{\pgfqpoint{2.701734in}{1.353365in}}%
\pgfpathlineto{\pgfqpoint{2.704545in}{1.857775in}}%
\pgfpathlineto{\pgfqpoint{2.707356in}{1.622923in}}%
\pgfpathlineto{\pgfqpoint{2.710166in}{1.835887in}}%
\pgfpathlineto{\pgfqpoint{2.712977in}{1.689039in}}%
\pgfpathlineto{\pgfqpoint{2.715788in}{1.799078in}}%
\pgfpathlineto{\pgfqpoint{2.718599in}{1.124684in}}%
\pgfpathlineto{\pgfqpoint{2.721409in}{1.652019in}}%
\pgfpathlineto{\pgfqpoint{2.724220in}{1.636463in}}%
\pgfpathlineto{\pgfqpoint{2.727031in}{1.735253in}}%
\pgfpathlineto{\pgfqpoint{2.729841in}{1.468409in}}%
\pgfpathlineto{\pgfqpoint{2.732652in}{1.751344in}}%
\pgfpathlineto{\pgfqpoint{2.735463in}{1.716933in}}%
\pgfpathlineto{\pgfqpoint{2.738273in}{1.573345in}}%
\pgfpathlineto{\pgfqpoint{2.741084in}{1.701425in}}%
\pgfpathlineto{\pgfqpoint{2.743895in}{1.569350in}}%
\pgfpathlineto{\pgfqpoint{2.746705in}{1.799022in}}%
\pgfpathlineto{\pgfqpoint{2.749516in}{1.698294in}}%
\pgfpathlineto{\pgfqpoint{2.752327in}{1.673201in}}%
\pgfpathlineto{\pgfqpoint{2.755137in}{1.894723in}}%
\pgfpathlineto{\pgfqpoint{2.757948in}{1.772372in}}%
\pgfpathlineto{\pgfqpoint{2.760759in}{1.518447in}}%
\pgfpathlineto{\pgfqpoint{2.763570in}{1.997581in}}%
\pgfpathlineto{\pgfqpoint{2.766380in}{1.740849in}}%
\pgfpathlineto{\pgfqpoint{2.769191in}{1.780133in}}%
\pgfpathlineto{\pgfqpoint{2.772002in}{1.658660in}}%
\pgfpathlineto{\pgfqpoint{2.774812in}{1.716294in}}%
\pgfpathlineto{\pgfqpoint{2.777623in}{1.685959in}}%
\pgfpathlineto{\pgfqpoint{2.780434in}{1.743493in}}%
\pgfpathlineto{\pgfqpoint{2.783244in}{1.722221in}}%
\pgfpathlineto{\pgfqpoint{2.786055in}{1.506902in}}%
\pgfpathlineto{\pgfqpoint{2.788866in}{1.541712in}}%
\pgfpathlineto{\pgfqpoint{2.791676in}{1.851490in}}%
\pgfpathlineto{\pgfqpoint{2.794487in}{1.707271in}}%
\pgfpathlineto{\pgfqpoint{2.797298in}{1.643177in}}%
\pgfpathlineto{\pgfqpoint{2.800108in}{1.688965in}}%
\pgfpathlineto{\pgfqpoint{2.802919in}{1.804769in}}%
\pgfpathlineto{\pgfqpoint{2.805730in}{1.585392in}}%
\pgfpathlineto{\pgfqpoint{2.808540in}{1.880417in}}%
\pgfpathlineto{\pgfqpoint{2.811351in}{1.552465in}}%
\pgfpathlineto{\pgfqpoint{2.814162in}{1.624834in}}%
\pgfpathlineto{\pgfqpoint{2.816973in}{1.734813in}}%
\pgfpathlineto{\pgfqpoint{2.819783in}{1.886154in}}%
\pgfpathlineto{\pgfqpoint{2.822594in}{1.564897in}}%
\pgfpathlineto{\pgfqpoint{2.825405in}{1.755715in}}%
\pgfpathlineto{\pgfqpoint{2.828215in}{1.679911in}}%
\pgfpathlineto{\pgfqpoint{2.831026in}{1.518400in}}%
\pgfpathlineto{\pgfqpoint{2.833837in}{1.587531in}}%
\pgfpathlineto{\pgfqpoint{2.836647in}{1.836284in}}%
\pgfpathlineto{\pgfqpoint{2.839458in}{1.420820in}}%
\pgfpathlineto{\pgfqpoint{2.842269in}{1.781918in}}%
\pgfpathlineto{\pgfqpoint{2.845079in}{1.617664in}}%
\pgfpathlineto{\pgfqpoint{2.847890in}{1.871122in}}%
\pgfpathlineto{\pgfqpoint{2.850701in}{1.444449in}}%
\pgfpathlineto{\pgfqpoint{2.856322in}{1.859799in}}%
\pgfpathlineto{\pgfqpoint{2.859133in}{1.676564in}}%
\pgfpathlineto{\pgfqpoint{2.861944in}{1.414254in}}%
\pgfpathlineto{\pgfqpoint{2.864754in}{2.013039in}}%
\pgfpathlineto{\pgfqpoint{2.867565in}{1.345795in}}%
\pgfpathlineto{\pgfqpoint{2.873186in}{1.640803in}}%
\pgfpathlineto{\pgfqpoint{2.878808in}{1.503447in}}%
\pgfpathlineto{\pgfqpoint{2.881618in}{1.903205in}}%
\pgfpathlineto{\pgfqpoint{2.884429in}{1.922660in}}%
\pgfpathlineto{\pgfqpoint{2.887240in}{1.831308in}}%
\pgfpathlineto{\pgfqpoint{2.890050in}{2.096430in}}%
\pgfpathlineto{\pgfqpoint{2.892861in}{1.515995in}}%
\pgfpathlineto{\pgfqpoint{2.895672in}{1.938522in}}%
\pgfpathlineto{\pgfqpoint{2.898482in}{1.633831in}}%
\pgfpathlineto{\pgfqpoint{2.901293in}{1.685885in}}%
\pgfpathlineto{\pgfqpoint{2.904104in}{1.932661in}}%
\pgfpathlineto{\pgfqpoint{2.906915in}{1.543204in}}%
\pgfpathlineto{\pgfqpoint{2.909725in}{3.214362in}}%
\pgfpathlineto{\pgfqpoint{2.912536in}{2.004430in}}%
\pgfpathlineto{\pgfqpoint{2.915347in}{1.697448in}}%
\pgfpathlineto{\pgfqpoint{2.918157in}{1.783967in}}%
\pgfpathlineto{\pgfqpoint{2.920968in}{2.109701in}}%
\pgfpathlineto{\pgfqpoint{2.923779in}{1.726112in}}%
\pgfpathlineto{\pgfqpoint{2.926589in}{1.838022in}}%
\pgfpathlineto{\pgfqpoint{2.929400in}{1.553880in}}%
\pgfpathlineto{\pgfqpoint{2.932211in}{1.660575in}}%
\pgfpathlineto{\pgfqpoint{2.935021in}{1.952175in}}%
\pgfpathlineto{\pgfqpoint{2.937832in}{1.627390in}}%
\pgfpathlineto{\pgfqpoint{2.940643in}{1.561949in}}%
\pgfpathlineto{\pgfqpoint{2.943453in}{1.751991in}}%
\pgfpathlineto{\pgfqpoint{2.946264in}{1.686817in}}%
\pgfpathlineto{\pgfqpoint{2.949075in}{1.702432in}}%
\pgfpathlineto{\pgfqpoint{2.951886in}{1.803487in}}%
\pgfpathlineto{\pgfqpoint{2.954696in}{1.848792in}}%
\pgfpathlineto{\pgfqpoint{2.957507in}{1.717573in}}%
\pgfpathlineto{\pgfqpoint{2.960318in}{1.852035in}}%
\pgfpathlineto{\pgfqpoint{2.963128in}{1.692023in}}%
\pgfpathlineto{\pgfqpoint{2.965939in}{1.747516in}}%
\pgfpathlineto{\pgfqpoint{2.968750in}{1.692023in}}%
\pgfpathlineto{\pgfqpoint{2.971560in}{1.802420in}}%
\pgfpathlineto{\pgfqpoint{2.974371in}{1.786742in}}%
\pgfpathlineto{\pgfqpoint{2.977182in}{1.664663in}}%
\pgfpathlineto{\pgfqpoint{2.979992in}{1.823407in}}%
\pgfpathlineto{\pgfqpoint{2.982803in}{1.677205in}}%
\pgfpathlineto{\pgfqpoint{2.985614in}{1.753673in}}%
\pgfpathlineto{\pgfqpoint{2.988424in}{1.548620in}}%
\pgfpathlineto{\pgfqpoint{2.991235in}{1.776247in}}%
\pgfpathlineto{\pgfqpoint{2.994046in}{1.311916in}}%
\pgfpathlineto{\pgfqpoint{2.996856in}{1.686959in}}%
\pgfpathlineto{\pgfqpoint{2.999667in}{1.490684in}}%
\pgfpathlineto{\pgfqpoint{3.002478in}{2.037036in}}%
\pgfpathlineto{\pgfqpoint{3.005289in}{1.995161in}}%
\pgfpathlineto{\pgfqpoint{3.008099in}{1.543675in}}%
\pgfpathlineto{\pgfqpoint{3.010910in}{1.842832in}}%
\pgfpathlineto{\pgfqpoint{3.013721in}{1.753411in}}%
\pgfpathlineto{\pgfqpoint{3.016531in}{1.831092in}}%
\pgfpathlineto{\pgfqpoint{3.019342in}{1.633619in}}%
\pgfpathlineto{\pgfqpoint{3.022153in}{1.621158in}}%
\pgfpathlineto{\pgfqpoint{3.024963in}{1.640505in}}%
\pgfpathlineto{\pgfqpoint{3.027774in}{1.553805in}}%
\pgfpathlineto{\pgfqpoint{3.030585in}{1.857247in}}%
\pgfpathlineto{\pgfqpoint{3.033395in}{1.344765in}}%
\pgfpathlineto{\pgfqpoint{3.036206in}{1.591370in}}%
\pgfpathlineto{\pgfqpoint{3.039017in}{1.900158in}}%
\pgfpathlineto{\pgfqpoint{3.041827in}{1.899877in}}%
\pgfpathlineto{\pgfqpoint{3.044638in}{1.459239in}}%
\pgfpathlineto{\pgfqpoint{3.047449in}{1.659554in}}%
\pgfpathlineto{\pgfqpoint{3.050260in}{1.739454in}}%
\pgfpathlineto{\pgfqpoint{3.053070in}{1.374753in}}%
\pgfpathlineto{\pgfqpoint{3.055881in}{1.543781in}}%
\pgfpathlineto{\pgfqpoint{3.058692in}{1.804611in}}%
\pgfpathlineto{\pgfqpoint{3.061502in}{1.806335in}}%
\pgfpathlineto{\pgfqpoint{3.064313in}{1.719839in}}%
\pgfpathlineto{\pgfqpoint{3.067124in}{1.709698in}}%
\pgfpathlineto{\pgfqpoint{3.069934in}{1.734866in}}%
\pgfpathlineto{\pgfqpoint{3.072745in}{1.585961in}}%
\pgfpathlineto{\pgfqpoint{3.075556in}{1.351326in}}%
\pgfpathlineto{\pgfqpoint{3.078366in}{1.399236in}}%
\pgfpathlineto{\pgfqpoint{3.083988in}{2.121173in}}%
\pgfpathlineto{\pgfqpoint{3.086798in}{1.717502in}}%
\pgfpathlineto{\pgfqpoint{3.089609in}{1.962080in}}%
\pgfpathlineto{\pgfqpoint{3.092420in}{1.994342in}}%
\pgfpathlineto{\pgfqpoint{3.095231in}{2.094190in}}%
\pgfpathlineto{\pgfqpoint{3.098041in}{1.438847in}}%
\pgfpathlineto{\pgfqpoint{3.100852in}{1.577352in}}%
\pgfpathlineto{\pgfqpoint{3.103663in}{1.638056in}}%
\pgfpathlineto{\pgfqpoint{3.106473in}{1.889803in}}%
\pgfpathlineto{\pgfqpoint{3.109284in}{1.978117in}}%
\pgfpathlineto{\pgfqpoint{3.112095in}{1.617998in}}%
\pgfpathlineto{\pgfqpoint{3.114905in}{1.763666in}}%
\pgfpathlineto{\pgfqpoint{3.117716in}{1.591631in}}%
\pgfpathlineto{\pgfqpoint{3.120527in}{1.692023in}}%
\pgfpathlineto{\pgfqpoint{3.123337in}{1.915822in}}%
\pgfpathlineto{\pgfqpoint{3.126148in}{1.692023in}}%
\pgfpathlineto{\pgfqpoint{3.128959in}{1.661263in}}%
\pgfpathlineto{\pgfqpoint{3.131769in}{1.722783in}}%
\pgfpathlineto{\pgfqpoint{3.134580in}{1.734514in}}%
\pgfpathlineto{\pgfqpoint{3.137391in}{1.552312in}}%
\pgfpathlineto{\pgfqpoint{3.140202in}{2.088920in}}%
\pgfpathlineto{\pgfqpoint{3.143012in}{1.524163in}}%
\pgfpathlineto{\pgfqpoint{3.145823in}{1.602697in}}%
\pgfpathlineto{\pgfqpoint{3.148634in}{1.713225in}}%
\pgfpathlineto{\pgfqpoint{3.151444in}{1.416516in}}%
\pgfpathlineto{\pgfqpoint{3.154255in}{1.813751in}}%
\pgfpathlineto{\pgfqpoint{3.157066in}{1.328822in}}%
\pgfpathlineto{\pgfqpoint{3.162687in}{1.974659in}}%
\pgfpathlineto{\pgfqpoint{3.165498in}{1.426442in}}%
\pgfpathlineto{\pgfqpoint{3.168308in}{1.926452in}}%
\pgfpathlineto{\pgfqpoint{3.171119in}{1.425907in}}%
\pgfpathlineto{\pgfqpoint{3.173930in}{1.878780in}}%
\pgfpathlineto{\pgfqpoint{3.176740in}{1.667895in}}%
\pgfpathlineto{\pgfqpoint{3.179551in}{1.831450in}}%
\pgfpathlineto{\pgfqpoint{3.182362in}{1.615252in}}%
\pgfpathlineto{\pgfqpoint{3.185173in}{1.696832in}}%
\pgfpathlineto{\pgfqpoint{3.187983in}{1.368933in}}%
\pgfpathlineto{\pgfqpoint{3.190794in}{1.660081in}}%
\pgfpathlineto{\pgfqpoint{3.193605in}{1.679720in}}%
\pgfpathlineto{\pgfqpoint{3.196415in}{1.719076in}}%
\pgfpathlineto{\pgfqpoint{3.199226in}{1.632937in}}%
\pgfpathlineto{\pgfqpoint{3.202037in}{1.637663in}}%
\pgfpathlineto{\pgfqpoint{3.204847in}{1.719226in}}%
\pgfpathlineto{\pgfqpoint{3.207658in}{1.669769in}}%
\pgfpathlineto{\pgfqpoint{3.210469in}{1.834883in}}%
\pgfpathlineto{\pgfqpoint{3.213279in}{1.857866in}}%
\pgfpathlineto{\pgfqpoint{3.216090in}{1.677460in}}%
\pgfpathlineto{\pgfqpoint{3.218901in}{1.667722in}}%
\pgfpathlineto{\pgfqpoint{3.221711in}{1.520848in}}%
\pgfpathlineto{\pgfqpoint{3.224522in}{1.672340in}}%
\pgfpathlineto{\pgfqpoint{3.227333in}{1.726452in}}%
\pgfpathlineto{\pgfqpoint{3.230143in}{1.684652in}}%
\pgfpathlineto{\pgfqpoint{3.232954in}{1.421812in}}%
\pgfpathlineto{\pgfqpoint{3.235765in}{1.739462in}}%
\pgfpathlineto{\pgfqpoint{3.238576in}{1.843359in}}%
\pgfpathlineto{\pgfqpoint{3.241386in}{2.316351in}}%
\pgfpathlineto{\pgfqpoint{3.244197in}{1.625462in}}%
\pgfpathlineto{\pgfqpoint{3.247008in}{1.634744in}}%
\pgfpathlineto{\pgfqpoint{3.249818in}{1.605708in}}%
\pgfpathlineto{\pgfqpoint{3.252629in}{1.614896in}}%
\pgfpathlineto{\pgfqpoint{3.255440in}{1.824350in}}%
\pgfpathlineto{\pgfqpoint{3.258250in}{1.389639in}}%
\pgfpathlineto{\pgfqpoint{3.261061in}{1.625931in}}%
\pgfpathlineto{\pgfqpoint{3.263872in}{1.647806in}}%
\pgfpathlineto{\pgfqpoint{3.266682in}{1.780332in}}%
\pgfpathlineto{\pgfqpoint{3.269493in}{1.623381in}}%
\pgfpathlineto{\pgfqpoint{3.272304in}{1.906745in}}%
\pgfpathlineto{\pgfqpoint{3.275114in}{2.356214in}}%
\pgfpathlineto{\pgfqpoint{3.277925in}{1.589504in}}%
\pgfpathlineto{\pgfqpoint{3.280736in}{1.619387in}}%
\pgfpathlineto{\pgfqpoint{3.283547in}{1.708454in}}%
\pgfpathlineto{\pgfqpoint{3.286357in}{1.996372in}}%
\pgfpathlineto{\pgfqpoint{3.289168in}{1.595090in}}%
\pgfpathlineto{\pgfqpoint{3.291979in}{1.788956in}}%
\pgfpathlineto{\pgfqpoint{3.294789in}{1.728792in}}%
\pgfpathlineto{\pgfqpoint{3.297600in}{1.606863in}}%
\pgfpathlineto{\pgfqpoint{3.300411in}{1.599432in}}%
\pgfpathlineto{\pgfqpoint{3.303221in}{1.747643in}}%
\pgfpathlineto{\pgfqpoint{3.306032in}{1.449557in}}%
\pgfpathlineto{\pgfqpoint{3.308843in}{1.906704in}}%
\pgfpathlineto{\pgfqpoint{3.314464in}{1.493958in}}%
\pgfpathlineto{\pgfqpoint{3.317275in}{1.790204in}}%
\pgfpathlineto{\pgfqpoint{3.320085in}{1.675702in}}%
\pgfpathlineto{\pgfqpoint{3.322896in}{1.671014in}}%
\pgfpathlineto{\pgfqpoint{3.325707in}{1.525287in}}%
\pgfpathlineto{\pgfqpoint{3.328518in}{1.727398in}}%
\pgfpathlineto{\pgfqpoint{3.331328in}{1.533391in}}%
\pgfpathlineto{\pgfqpoint{3.334139in}{1.753773in}}%
\pgfpathlineto{\pgfqpoint{3.336950in}{2.073679in}}%
\pgfpathlineto{\pgfqpoint{3.339760in}{1.694336in}}%
\pgfpathlineto{\pgfqpoint{3.342571in}{1.622487in}}%
\pgfpathlineto{\pgfqpoint{3.345382in}{1.521513in}}%
\pgfpathlineto{\pgfqpoint{3.348192in}{1.790352in}}%
\pgfpathlineto{\pgfqpoint{3.351003in}{1.633569in}}%
\pgfpathlineto{\pgfqpoint{3.353814in}{1.862100in}}%
\pgfpathlineto{\pgfqpoint{3.356624in}{1.507885in}}%
\pgfpathlineto{\pgfqpoint{3.359435in}{1.783198in}}%
\pgfpathlineto{\pgfqpoint{3.362246in}{1.766438in}}%
\pgfpathlineto{\pgfqpoint{3.365056in}{1.566241in}}%
\pgfpathlineto{\pgfqpoint{3.367867in}{1.635793in}}%
\pgfpathlineto{\pgfqpoint{3.370678in}{1.722506in}}%
\pgfpathlineto{\pgfqpoint{3.373489in}{1.223578in}}%
\pgfpathlineto{\pgfqpoint{3.376299in}{1.793081in}}%
\pgfpathlineto{\pgfqpoint{3.379110in}{1.854265in}}%
\pgfpathlineto{\pgfqpoint{3.381921in}{1.781958in}}%
\pgfpathlineto{\pgfqpoint{3.384731in}{1.663678in}}%
\pgfpathlineto{\pgfqpoint{3.387542in}{1.611432in}}%
\pgfpathlineto{\pgfqpoint{3.390353in}{1.450137in}}%
\pgfpathlineto{\pgfqpoint{3.393163in}{1.766665in}}%
\pgfpathlineto{\pgfqpoint{3.395974in}{2.008270in}}%
\pgfpathlineto{\pgfqpoint{3.398785in}{1.941876in}}%
\pgfpathlineto{\pgfqpoint{3.401595in}{1.821210in}}%
\pgfpathlineto{\pgfqpoint{3.404406in}{1.673633in}}%
\pgfpathlineto{\pgfqpoint{3.407217in}{1.813454in}}%
\pgfpathlineto{\pgfqpoint{3.410027in}{1.760341in}}%
\pgfpathlineto{\pgfqpoint{3.412838in}{2.089124in}}%
\pgfpathlineto{\pgfqpoint{3.415649in}{1.545130in}}%
\pgfpathlineto{\pgfqpoint{3.418459in}{1.680838in}}%
\pgfpathlineto{\pgfqpoint{3.421270in}{1.644959in}}%
\pgfpathlineto{\pgfqpoint{3.424081in}{2.340494in}}%
\pgfpathlineto{\pgfqpoint{3.426892in}{1.494845in}}%
\pgfpathlineto{\pgfqpoint{3.429702in}{1.876277in}}%
\pgfpathlineto{\pgfqpoint{3.432513in}{1.954856in}}%
\pgfpathlineto{\pgfqpoint{3.435324in}{1.768120in}}%
\pgfpathlineto{\pgfqpoint{3.438134in}{1.479720in}}%
\pgfpathlineto{\pgfqpoint{3.440945in}{1.779417in}}%
\pgfpathlineto{\pgfqpoint{3.443756in}{1.589659in}}%
\pgfpathlineto{\pgfqpoint{3.446566in}{1.631999in}}%
\pgfpathlineto{\pgfqpoint{3.449377in}{1.397150in}}%
\pgfpathlineto{\pgfqpoint{3.452188in}{1.824972in}}%
\pgfpathlineto{\pgfqpoint{3.454998in}{1.728873in}}%
\pgfpathlineto{\pgfqpoint{3.457809in}{1.476191in}}%
\pgfpathlineto{\pgfqpoint{3.460620in}{1.755556in}}%
\pgfpathlineto{\pgfqpoint{3.463430in}{1.774884in}}%
\pgfpathlineto{\pgfqpoint{3.471863in}{1.704983in}}%
\pgfpathlineto{\pgfqpoint{3.474673in}{1.676902in}}%
\pgfpathlineto{\pgfqpoint{3.477484in}{1.598813in}}%
\pgfpathlineto{\pgfqpoint{3.480295in}{1.098192in}}%
\pgfpathlineto{\pgfqpoint{3.483105in}{1.058243in}}%
\pgfpathlineto{\pgfqpoint{3.485916in}{1.371216in}}%
\pgfpathlineto{\pgfqpoint{3.488727in}{2.535550in}}%
\pgfpathlineto{\pgfqpoint{3.491537in}{2.065114in}}%
\pgfpathlineto{\pgfqpoint{3.494348in}{1.705343in}}%
\pgfpathlineto{\pgfqpoint{3.499969in}{1.171503in}}%
\pgfpathlineto{\pgfqpoint{3.502780in}{1.840543in}}%
\pgfpathlineto{\pgfqpoint{3.505591in}{1.866575in}}%
\pgfpathlineto{\pgfqpoint{3.508401in}{1.415516in}}%
\pgfpathlineto{\pgfqpoint{3.511212in}{1.998193in}}%
\pgfpathlineto{\pgfqpoint{3.514023in}{1.483188in}}%
\pgfpathlineto{\pgfqpoint{3.516834in}{1.834609in}}%
\pgfpathlineto{\pgfqpoint{3.519644in}{1.808342in}}%
\pgfpathlineto{\pgfqpoint{3.522455in}{1.525257in}}%
\pgfpathlineto{\pgfqpoint{3.525266in}{1.804158in}}%
\pgfpathlineto{\pgfqpoint{3.528076in}{1.796564in}}%
\pgfpathlineto{\pgfqpoint{3.530887in}{1.685226in}}%
\pgfpathlineto{\pgfqpoint{3.533698in}{1.433838in}}%
\pgfpathlineto{\pgfqpoint{3.536508in}{1.911638in}}%
\pgfpathlineto{\pgfqpoint{3.539319in}{1.573449in}}%
\pgfpathlineto{\pgfqpoint{3.542130in}{1.846901in}}%
\pgfpathlineto{\pgfqpoint{3.544940in}{1.534856in}}%
\pgfpathlineto{\pgfqpoint{3.547751in}{1.792428in}}%
\pgfpathlineto{\pgfqpoint{3.550562in}{0.901227in}}%
\pgfpathlineto{\pgfqpoint{3.553372in}{2.000033in}}%
\pgfpathlineto{\pgfqpoint{3.556183in}{1.945673in}}%
\pgfpathlineto{\pgfqpoint{3.558994in}{1.765713in}}%
\pgfpathlineto{\pgfqpoint{3.561805in}{1.842918in}}%
\pgfpathlineto{\pgfqpoint{3.564615in}{2.034089in}}%
\pgfpathlineto{\pgfqpoint{3.567426in}{1.658598in}}%
\pgfpathlineto{\pgfqpoint{3.570237in}{1.937689in}}%
\pgfpathlineto{\pgfqpoint{3.573047in}{1.836282in}}%
\pgfpathlineto{\pgfqpoint{3.575858in}{1.709419in}}%
\pgfpathlineto{\pgfqpoint{3.578669in}{1.903562in}}%
\pgfpathlineto{\pgfqpoint{3.584290in}{1.523912in}}%
\pgfpathlineto{\pgfqpoint{3.587101in}{1.930716in}}%
\pgfpathlineto{\pgfqpoint{3.589911in}{1.828006in}}%
\pgfpathlineto{\pgfqpoint{3.592722in}{1.893802in}}%
\pgfpathlineto{\pgfqpoint{3.595533in}{1.547258in}}%
\pgfpathlineto{\pgfqpoint{3.598343in}{1.522482in}}%
\pgfpathlineto{\pgfqpoint{3.601154in}{1.888942in}}%
\pgfpathlineto{\pgfqpoint{3.603965in}{1.824022in}}%
\pgfpathlineto{\pgfqpoint{3.606776in}{1.915741in}}%
\pgfpathlineto{\pgfqpoint{3.609586in}{1.559837in}}%
\pgfpathlineto{\pgfqpoint{3.612397in}{1.961402in}}%
\pgfpathlineto{\pgfqpoint{3.615208in}{1.620505in}}%
\pgfpathlineto{\pgfqpoint{3.618018in}{1.506601in}}%
\pgfpathlineto{\pgfqpoint{3.620829in}{1.210048in}}%
\pgfpathlineto{\pgfqpoint{3.623640in}{2.238115in}}%
\pgfpathlineto{\pgfqpoint{3.626450in}{1.782577in}}%
\pgfpathlineto{\pgfqpoint{3.629261in}{1.871570in}}%
\pgfpathlineto{\pgfqpoint{3.632072in}{1.590261in}}%
\pgfpathlineto{\pgfqpoint{3.634882in}{1.552553in}}%
\pgfpathlineto{\pgfqpoint{3.637693in}{2.004095in}}%
\pgfpathlineto{\pgfqpoint{3.640504in}{1.633349in}}%
\pgfpathlineto{\pgfqpoint{3.643314in}{1.518767in}}%
\pgfpathlineto{\pgfqpoint{3.646125in}{1.653031in}}%
\pgfpathlineto{\pgfqpoint{3.648936in}{1.849445in}}%
\pgfpathlineto{\pgfqpoint{3.651746in}{1.677777in}}%
\pgfpathlineto{\pgfqpoint{3.654557in}{2.010444in}}%
\pgfpathlineto{\pgfqpoint{3.657368in}{1.632061in}}%
\pgfpathlineto{\pgfqpoint{3.660179in}{1.700031in}}%
\pgfpathlineto{\pgfqpoint{3.662989in}{1.676002in}}%
\pgfpathlineto{\pgfqpoint{3.665800in}{1.629787in}}%
\pgfpathlineto{\pgfqpoint{3.668611in}{1.647704in}}%
\pgfpathlineto{\pgfqpoint{3.671421in}{1.746396in}}%
\pgfpathlineto{\pgfqpoint{3.674232in}{1.528331in}}%
\pgfpathlineto{\pgfqpoint{3.677043in}{1.869780in}}%
\pgfpathlineto{\pgfqpoint{3.679853in}{1.520359in}}%
\pgfpathlineto{\pgfqpoint{3.682664in}{1.543071in}}%
\pgfpathlineto{\pgfqpoint{3.685475in}{2.108751in}}%
\pgfpathlineto{\pgfqpoint{3.688285in}{1.525454in}}%
\pgfpathlineto{\pgfqpoint{3.691096in}{1.596906in}}%
\pgfpathlineto{\pgfqpoint{3.693907in}{1.456817in}}%
\pgfpathlineto{\pgfqpoint{3.696717in}{1.739349in}}%
\pgfpathlineto{\pgfqpoint{3.699528in}{1.286159in}}%
\pgfpathlineto{\pgfqpoint{3.702339in}{1.845200in}}%
\pgfpathlineto{\pgfqpoint{3.705150in}{2.045057in}}%
\pgfpathlineto{\pgfqpoint{3.707960in}{1.935082in}}%
\pgfpathlineto{\pgfqpoint{3.710771in}{1.461207in}}%
\pgfpathlineto{\pgfqpoint{3.713582in}{1.215684in}}%
\pgfpathlineto{\pgfqpoint{3.716392in}{1.865600in}}%
\pgfpathlineto{\pgfqpoint{3.719203in}{1.733562in}}%
\pgfpathlineto{\pgfqpoint{3.722014in}{1.928749in}}%
\pgfpathlineto{\pgfqpoint{3.724824in}{1.632665in}}%
\pgfpathlineto{\pgfqpoint{3.727635in}{1.745251in}}%
\pgfpathlineto{\pgfqpoint{3.730446in}{1.826331in}}%
\pgfpathlineto{\pgfqpoint{3.733256in}{1.520883in}}%
\pgfpathlineto{\pgfqpoint{3.736067in}{1.533479in}}%
\pgfpathlineto{\pgfqpoint{3.738878in}{1.314994in}}%
\pgfpathlineto{\pgfqpoint{3.741688in}{1.808196in}}%
\pgfpathlineto{\pgfqpoint{3.744499in}{1.486551in}}%
\pgfpathlineto{\pgfqpoint{3.747310in}{1.381914in}}%
\pgfpathlineto{\pgfqpoint{3.750121in}{1.497231in}}%
\pgfpathlineto{\pgfqpoint{3.752931in}{1.915065in}}%
\pgfpathlineto{\pgfqpoint{3.755742in}{1.866902in}}%
\pgfpathlineto{\pgfqpoint{3.758553in}{1.342480in}}%
\pgfpathlineto{\pgfqpoint{3.761363in}{1.840612in}}%
\pgfpathlineto{\pgfqpoint{3.764174in}{1.268780in}}%
\pgfpathlineto{\pgfqpoint{3.766985in}{1.597888in}}%
\pgfpathlineto{\pgfqpoint{3.769795in}{1.536130in}}%
\pgfpathlineto{\pgfqpoint{3.772606in}{1.883844in}}%
\pgfpathlineto{\pgfqpoint{3.775417in}{1.939020in}}%
\pgfpathlineto{\pgfqpoint{3.778227in}{1.485346in}}%
\pgfpathlineto{\pgfqpoint{3.781038in}{1.721078in}}%
\pgfpathlineto{\pgfqpoint{3.783849in}{1.480701in}}%
\pgfpathlineto{\pgfqpoint{3.786659in}{1.339508in}}%
\pgfpathlineto{\pgfqpoint{3.789470in}{2.812968in}}%
\pgfpathlineto{\pgfqpoint{3.792281in}{1.668304in}}%
\pgfpathlineto{\pgfqpoint{3.795092in}{1.485661in}}%
\pgfpathlineto{\pgfqpoint{3.797902in}{1.898384in}}%
\pgfpathlineto{\pgfqpoint{3.806334in}{0.848492in}}%
\pgfpathlineto{\pgfqpoint{3.809145in}{1.819382in}}%
\pgfpathlineto{\pgfqpoint{3.811956in}{2.134336in}}%
\pgfpathlineto{\pgfqpoint{3.814766in}{1.313002in}}%
\pgfpathlineto{\pgfqpoint{3.817577in}{2.130297in}}%
\pgfpathlineto{\pgfqpoint{3.820388in}{1.843683in}}%
\pgfpathlineto{\pgfqpoint{3.823198in}{1.924554in}}%
\pgfpathlineto{\pgfqpoint{3.826009in}{1.520194in}}%
\pgfpathlineto{\pgfqpoint{3.831630in}{2.023704in}}%
\pgfpathlineto{\pgfqpoint{3.834441in}{1.496077in}}%
\pgfpathlineto{\pgfqpoint{3.837252in}{1.625415in}}%
\pgfpathlineto{\pgfqpoint{3.840062in}{1.998273in}}%
\pgfpathlineto{\pgfqpoint{3.842873in}{1.575959in}}%
\pgfpathlineto{\pgfqpoint{3.845684in}{1.619318in}}%
\pgfpathlineto{\pgfqpoint{3.848495in}{2.170443in}}%
\pgfpathlineto{\pgfqpoint{3.851305in}{1.621188in}}%
\pgfpathlineto{\pgfqpoint{3.854116in}{1.610056in}}%
\pgfpathlineto{\pgfqpoint{3.856927in}{1.689860in}}%
\pgfpathlineto{\pgfqpoint{3.859737in}{1.271099in}}%
\pgfpathlineto{\pgfqpoint{3.862548in}{1.402701in}}%
\pgfpathlineto{\pgfqpoint{3.865359in}{1.705599in}}%
\pgfpathlineto{\pgfqpoint{3.868169in}{1.589910in}}%
\pgfpathlineto{\pgfqpoint{3.870980in}{2.003074in}}%
\pgfpathlineto{\pgfqpoint{3.873791in}{1.660748in}}%
\pgfpathlineto{\pgfqpoint{3.876601in}{1.783437in}}%
\pgfpathlineto{\pgfqpoint{3.879412in}{1.791743in}}%
\pgfpathlineto{\pgfqpoint{3.882223in}{1.856820in}}%
\pgfpathlineto{\pgfqpoint{3.885033in}{1.829120in}}%
\pgfpathlineto{\pgfqpoint{3.887844in}{1.689856in}}%
\pgfpathlineto{\pgfqpoint{3.890655in}{1.519891in}}%
\pgfpathlineto{\pgfqpoint{3.893466in}{1.894463in}}%
\pgfpathlineto{\pgfqpoint{3.896276in}{1.739530in}}%
\pgfpathlineto{\pgfqpoint{3.899087in}{1.679081in}}%
\pgfpathlineto{\pgfqpoint{3.901898in}{1.963701in}}%
\pgfpathlineto{\pgfqpoint{3.904708in}{1.979886in}}%
\pgfpathlineto{\pgfqpoint{3.907519in}{1.629440in}}%
\pgfpathlineto{\pgfqpoint{3.910330in}{1.918218in}}%
\pgfpathlineto{\pgfqpoint{3.913140in}{1.739336in}}%
\pgfpathlineto{\pgfqpoint{3.915951in}{1.518528in}}%
\pgfpathlineto{\pgfqpoint{3.918762in}{1.941331in}}%
\pgfpathlineto{\pgfqpoint{3.921572in}{1.566837in}}%
\pgfpathlineto{\pgfqpoint{3.924383in}{1.782422in}}%
\pgfpathlineto{\pgfqpoint{3.927194in}{1.648943in}}%
\pgfpathlineto{\pgfqpoint{3.930004in}{1.833126in}}%
\pgfpathlineto{\pgfqpoint{3.932815in}{1.926402in}}%
\pgfpathlineto{\pgfqpoint{3.935626in}{1.811886in}}%
\pgfpathlineto{\pgfqpoint{3.938437in}{1.644188in}}%
\pgfpathlineto{\pgfqpoint{3.941247in}{1.957244in}}%
\pgfpathlineto{\pgfqpoint{3.944058in}{1.556017in}}%
\pgfpathlineto{\pgfqpoint{3.946869in}{1.769042in}}%
\pgfpathlineto{\pgfqpoint{3.949679in}{1.622921in}}%
\pgfpathlineto{\pgfqpoint{3.952490in}{1.364092in}}%
\pgfpathlineto{\pgfqpoint{3.955301in}{1.500908in}}%
\pgfpathlineto{\pgfqpoint{3.958111in}{1.769564in}}%
\pgfpathlineto{\pgfqpoint{3.960922in}{1.736740in}}%
\pgfpathlineto{\pgfqpoint{3.963733in}{1.477439in}}%
\pgfpathlineto{\pgfqpoint{3.966543in}{1.603287in}}%
\pgfpathlineto{\pgfqpoint{3.969354in}{1.936371in}}%
\pgfpathlineto{\pgfqpoint{3.972165in}{1.437324in}}%
\pgfpathlineto{\pgfqpoint{3.974975in}{1.667153in}}%
\pgfpathlineto{\pgfqpoint{3.977786in}{1.749992in}}%
\pgfpathlineto{\pgfqpoint{3.980597in}{1.766240in}}%
\pgfpathlineto{\pgfqpoint{3.983408in}{1.790433in}}%
\pgfpathlineto{\pgfqpoint{3.986218in}{1.887001in}}%
\pgfpathlineto{\pgfqpoint{3.989029in}{1.340942in}}%
\pgfpathlineto{\pgfqpoint{3.991840in}{1.811296in}}%
\pgfpathlineto{\pgfqpoint{3.994650in}{1.477502in}}%
\pgfpathlineto{\pgfqpoint{3.997461in}{1.888095in}}%
\pgfpathlineto{\pgfqpoint{4.000272in}{1.487640in}}%
\pgfpathlineto{\pgfqpoint{4.003082in}{1.822394in}}%
\pgfpathlineto{\pgfqpoint{4.005893in}{1.580343in}}%
\pgfpathlineto{\pgfqpoint{4.008704in}{1.851028in}}%
\pgfpathlineto{\pgfqpoint{4.011514in}{1.599295in}}%
\pgfpathlineto{\pgfqpoint{4.014325in}{2.124139in}}%
\pgfpathlineto{\pgfqpoint{4.017136in}{1.629578in}}%
\pgfpathlineto{\pgfqpoint{4.019946in}{1.679908in}}%
\pgfpathlineto{\pgfqpoint{4.022757in}{1.822770in}}%
\pgfpathlineto{\pgfqpoint{4.025568in}{1.549151in}}%
\pgfpathlineto{\pgfqpoint{4.028378in}{1.748523in}}%
\pgfpathlineto{\pgfqpoint{4.031189in}{1.834381in}}%
\pgfpathlineto{\pgfqpoint{4.034000in}{1.690027in}}%
\pgfpathlineto{\pgfqpoint{4.036811in}{1.809355in}}%
\pgfpathlineto{\pgfqpoint{4.039621in}{1.703906in}}%
\pgfpathlineto{\pgfqpoint{4.042432in}{1.892659in}}%
\pgfpathlineto{\pgfqpoint{4.045243in}{1.617577in}}%
\pgfpathlineto{\pgfqpoint{4.050864in}{1.339816in}}%
\pgfpathlineto{\pgfqpoint{4.053675in}{1.726588in}}%
\pgfpathlineto{\pgfqpoint{4.056485in}{1.612613in}}%
\pgfpathlineto{\pgfqpoint{4.059296in}{1.728724in}}%
\pgfpathlineto{\pgfqpoint{4.062107in}{1.418783in}}%
\pgfpathlineto{\pgfqpoint{4.064917in}{1.762330in}}%
\pgfpathlineto{\pgfqpoint{4.067728in}{1.689960in}}%
\pgfpathlineto{\pgfqpoint{4.070539in}{1.530235in}}%
\pgfpathlineto{\pgfqpoint{4.073349in}{2.034349in}}%
\pgfpathlineto{\pgfqpoint{4.076160in}{1.044407in}}%
\pgfpathlineto{\pgfqpoint{4.078971in}{1.332850in}}%
\pgfpathlineto{\pgfqpoint{4.081782in}{2.074566in}}%
\pgfpathlineto{\pgfqpoint{4.084592in}{2.015587in}}%
\pgfpathlineto{\pgfqpoint{4.087403in}{1.161158in}}%
\pgfpathlineto{\pgfqpoint{4.090214in}{1.756431in}}%
\pgfpathlineto{\pgfqpoint{4.093024in}{1.732678in}}%
\pgfpathlineto{\pgfqpoint{4.095835in}{1.563276in}}%
\pgfpathlineto{\pgfqpoint{4.098646in}{1.786541in}}%
\pgfpathlineto{\pgfqpoint{4.101456in}{2.087457in}}%
\pgfpathlineto{\pgfqpoint{4.104267in}{1.712890in}}%
\pgfpathlineto{\pgfqpoint{4.107078in}{1.884758in}}%
\pgfpathlineto{\pgfqpoint{4.109888in}{1.689963in}}%
\pgfpathlineto{\pgfqpoint{4.112699in}{1.827390in}}%
\pgfpathlineto{\pgfqpoint{4.115510in}{1.724663in}}%
\pgfpathlineto{\pgfqpoint{4.118320in}{1.694061in}}%
\pgfpathlineto{\pgfqpoint{4.121131in}{1.773316in}}%
\pgfpathlineto{\pgfqpoint{4.123942in}{1.817196in}}%
\pgfpathlineto{\pgfqpoint{4.126753in}{1.581033in}}%
\pgfpathlineto{\pgfqpoint{4.129563in}{1.911222in}}%
\pgfpathlineto{\pgfqpoint{4.132374in}{1.501151in}}%
\pgfpathlineto{\pgfqpoint{4.140806in}{1.823550in}}%
\pgfpathlineto{\pgfqpoint{4.143617in}{1.464696in}}%
\pgfpathlineto{\pgfqpoint{4.146427in}{1.734900in}}%
\pgfpathlineto{\pgfqpoint{4.149238in}{1.696100in}}%
\pgfpathlineto{\pgfqpoint{4.154859in}{1.843334in}}%
\pgfpathlineto{\pgfqpoint{4.157670in}{1.821983in}}%
\pgfpathlineto{\pgfqpoint{4.160481in}{1.654146in}}%
\pgfpathlineto{\pgfqpoint{4.163291in}{1.713963in}}%
\pgfpathlineto{\pgfqpoint{4.166102in}{1.618103in}}%
\pgfpathlineto{\pgfqpoint{4.168913in}{1.777897in}}%
\pgfpathlineto{\pgfqpoint{4.171724in}{1.672094in}}%
\pgfpathlineto{\pgfqpoint{4.174534in}{1.864556in}}%
\pgfpathlineto{\pgfqpoint{4.177345in}{1.672288in}}%
\pgfpathlineto{\pgfqpoint{4.180156in}{1.725557in}}%
\pgfpathlineto{\pgfqpoint{4.182966in}{1.581285in}}%
\pgfpathlineto{\pgfqpoint{4.185777in}{1.731663in}}%
\pgfpathlineto{\pgfqpoint{4.188588in}{1.804447in}}%
\pgfpathlineto{\pgfqpoint{4.191398in}{1.642814in}}%
\pgfpathlineto{\pgfqpoint{4.194209in}{1.593137in}}%
\pgfpathlineto{\pgfqpoint{4.197020in}{1.697974in}}%
\pgfpathlineto{\pgfqpoint{4.199830in}{1.739548in}}%
\pgfpathlineto{\pgfqpoint{4.202641in}{1.751227in}}%
\pgfpathlineto{\pgfqpoint{4.205452in}{1.749041in}}%
\pgfpathlineto{\pgfqpoint{4.208262in}{1.640913in}}%
\pgfpathlineto{\pgfqpoint{4.211073in}{1.768625in}}%
\pgfpathlineto{\pgfqpoint{4.213884in}{1.824692in}}%
\pgfpathlineto{\pgfqpoint{4.216695in}{1.792720in}}%
\pgfpathlineto{\pgfqpoint{4.219505in}{1.826559in}}%
\pgfpathlineto{\pgfqpoint{4.222316in}{1.640268in}}%
\pgfpathlineto{\pgfqpoint{4.225127in}{1.381752in}}%
\pgfpathlineto{\pgfqpoint{4.227937in}{2.002294in}}%
\pgfpathlineto{\pgfqpoint{4.230748in}{1.489105in}}%
\pgfpathlineto{\pgfqpoint{4.233559in}{1.614021in}}%
\pgfpathlineto{\pgfqpoint{4.236369in}{1.791407in}}%
\pgfpathlineto{\pgfqpoint{4.239180in}{1.703674in}}%
\pgfpathlineto{\pgfqpoint{4.241991in}{1.775266in}}%
\pgfpathlineto{\pgfqpoint{4.244801in}{1.686230in}}%
\pgfpathlineto{\pgfqpoint{4.247612in}{1.832386in}}%
\pgfpathlineto{\pgfqpoint{4.250423in}{1.718798in}}%
\pgfpathlineto{\pgfqpoint{4.253233in}{1.536469in}}%
\pgfpathlineto{\pgfqpoint{4.256044in}{1.554386in}}%
\pgfpathlineto{\pgfqpoint{4.258855in}{1.758085in}}%
\pgfpathlineto{\pgfqpoint{4.261665in}{1.844449in}}%
\pgfpathlineto{\pgfqpoint{4.264476in}{1.494939in}}%
\pgfpathlineto{\pgfqpoint{4.267287in}{1.839106in}}%
\pgfpathlineto{\pgfqpoint{4.270098in}{1.722812in}}%
\pgfpathlineto{\pgfqpoint{4.272908in}{1.667012in}}%
\pgfpathlineto{\pgfqpoint{4.275719in}{1.809038in}}%
\pgfpathlineto{\pgfqpoint{4.278530in}{1.693934in}}%
\pgfpathlineto{\pgfqpoint{4.281340in}{1.601952in}}%
\pgfpathlineto{\pgfqpoint{4.284151in}{1.736165in}}%
\pgfpathlineto{\pgfqpoint{4.286962in}{1.487539in}}%
\pgfpathlineto{\pgfqpoint{4.289772in}{1.732747in}}%
\pgfpathlineto{\pgfqpoint{4.292583in}{1.620199in}}%
\pgfpathlineto{\pgfqpoint{4.295394in}{1.802533in}}%
\pgfpathlineto{\pgfqpoint{4.298204in}{1.633957in}}%
\pgfpathlineto{\pgfqpoint{4.301015in}{1.583052in}}%
\pgfpathlineto{\pgfqpoint{4.303826in}{1.926531in}}%
\pgfpathlineto{\pgfqpoint{4.306636in}{1.632277in}}%
\pgfpathlineto{\pgfqpoint{4.309447in}{1.664966in}}%
\pgfpathlineto{\pgfqpoint{4.312258in}{1.846007in}}%
\pgfpathlineto{\pgfqpoint{4.315069in}{1.476018in}}%
\pgfpathlineto{\pgfqpoint{4.317879in}{1.639500in}}%
\pgfpathlineto{\pgfqpoint{4.320690in}{1.725114in}}%
\pgfpathlineto{\pgfqpoint{4.323501in}{1.748310in}}%
\pgfpathlineto{\pgfqpoint{4.326311in}{1.746177in}}%
\pgfpathlineto{\pgfqpoint{4.329122in}{1.523138in}}%
\pgfpathlineto{\pgfqpoint{4.331933in}{1.491665in}}%
\pgfpathlineto{\pgfqpoint{4.334743in}{1.705856in}}%
\pgfpathlineto{\pgfqpoint{4.337554in}{1.636619in}}%
\pgfpathlineto{\pgfqpoint{4.340365in}{2.102837in}}%
\pgfpathlineto{\pgfqpoint{4.343175in}{1.763297in}}%
\pgfpathlineto{\pgfqpoint{4.345986in}{1.759146in}}%
\pgfpathlineto{\pgfqpoint{4.348797in}{1.433458in}}%
\pgfpathlineto{\pgfqpoint{4.351607in}{1.693968in}}%
\pgfpathlineto{\pgfqpoint{4.354418in}{1.010059in}}%
\pgfpathlineto{\pgfqpoint{4.357229in}{1.728562in}}%
\pgfpathlineto{\pgfqpoint{4.360040in}{2.023035in}}%
\pgfpathlineto{\pgfqpoint{4.362850in}{1.883391in}}%
\pgfpathlineto{\pgfqpoint{4.365661in}{1.644892in}}%
\pgfpathlineto{\pgfqpoint{4.368472in}{1.858307in}}%
\pgfpathlineto{\pgfqpoint{4.371282in}{1.351698in}}%
\pgfpathlineto{\pgfqpoint{4.374093in}{1.622259in}}%
\pgfpathlineto{\pgfqpoint{4.376904in}{1.801512in}}%
\pgfpathlineto{\pgfqpoint{4.379714in}{1.486348in}}%
\pgfpathlineto{\pgfqpoint{4.382525in}{1.706087in}}%
\pgfpathlineto{\pgfqpoint{4.385336in}{1.326285in}}%
\pgfpathlineto{\pgfqpoint{4.388146in}{1.304868in}}%
\pgfpathlineto{\pgfqpoint{4.390957in}{1.753021in}}%
\pgfpathlineto{\pgfqpoint{4.393768in}{2.016124in}}%
\pgfpathlineto{\pgfqpoint{4.396578in}{1.650846in}}%
\pgfpathlineto{\pgfqpoint{4.399389in}{2.131089in}}%
\pgfpathlineto{\pgfqpoint{4.402200in}{1.686009in}}%
\pgfpathlineto{\pgfqpoint{4.405011in}{1.665934in}}%
\pgfpathlineto{\pgfqpoint{4.407821in}{1.564971in}}%
\pgfpathlineto{\pgfqpoint{4.410632in}{1.829114in}}%
\pgfpathlineto{\pgfqpoint{4.413443in}{1.679975in}}%
\pgfpathlineto{\pgfqpoint{4.416253in}{1.764172in}}%
\pgfpathlineto{\pgfqpoint{4.419064in}{1.464427in}}%
\pgfpathlineto{\pgfqpoint{4.421875in}{1.641218in}}%
\pgfpathlineto{\pgfqpoint{4.424685in}{1.744856in}}%
\pgfpathlineto{\pgfqpoint{4.427496in}{1.647329in}}%
\pgfpathlineto{\pgfqpoint{4.430307in}{1.645161in}}%
\pgfpathlineto{\pgfqpoint{4.433117in}{1.740919in}}%
\pgfpathlineto{\pgfqpoint{4.435928in}{1.732653in}}%
\pgfpathlineto{\pgfqpoint{4.438739in}{1.681875in}}%
\pgfpathlineto{\pgfqpoint{4.441549in}{1.698112in}}%
\pgfpathlineto{\pgfqpoint{4.444360in}{1.631022in}}%
\pgfpathlineto{\pgfqpoint{4.447171in}{1.984678in}}%
\pgfpathlineto{\pgfqpoint{4.449981in}{1.819469in}}%
\pgfpathlineto{\pgfqpoint{4.452792in}{1.873423in}}%
\pgfpathlineto{\pgfqpoint{4.455603in}{1.906182in}}%
\pgfpathlineto{\pgfqpoint{4.458414in}{1.604767in}}%
\pgfpathlineto{\pgfqpoint{4.461224in}{1.608187in}}%
\pgfpathlineto{\pgfqpoint{4.464035in}{1.785578in}}%
\pgfpathlineto{\pgfqpoint{4.466846in}{1.610193in}}%
\pgfpathlineto{\pgfqpoint{4.469656in}{1.652905in}}%
\pgfpathlineto{\pgfqpoint{4.472467in}{1.709638in}}%
\pgfpathlineto{\pgfqpoint{4.475278in}{1.754494in}}%
\pgfpathlineto{\pgfqpoint{4.478088in}{1.719276in}}%
\pgfpathlineto{\pgfqpoint{4.480899in}{1.713403in}}%
\pgfpathlineto{\pgfqpoint{4.483710in}{1.750183in}}%
\pgfpathlineto{\pgfqpoint{4.486520in}{1.895801in}}%
\pgfpathlineto{\pgfqpoint{4.489331in}{1.817573in}}%
\pgfpathlineto{\pgfqpoint{4.492142in}{1.568383in}}%
\pgfpathlineto{\pgfqpoint{4.494952in}{1.791013in}}%
\pgfpathlineto{\pgfqpoint{4.497763in}{1.678734in}}%
\pgfpathlineto{\pgfqpoint{4.500574in}{1.506700in}}%
\pgfpathlineto{\pgfqpoint{4.503385in}{1.641972in}}%
\pgfpathlineto{\pgfqpoint{4.506195in}{1.665006in}}%
\pgfpathlineto{\pgfqpoint{4.509006in}{2.392628in}}%
\pgfpathlineto{\pgfqpoint{4.511817in}{1.645788in}}%
\pgfpathlineto{\pgfqpoint{4.514627in}{1.682759in}}%
\pgfpathlineto{\pgfqpoint{4.517438in}{1.567359in}}%
\pgfpathlineto{\pgfqpoint{4.520249in}{1.781435in}}%
\pgfpathlineto{\pgfqpoint{4.523059in}{1.749497in}}%
\pgfpathlineto{\pgfqpoint{4.525870in}{1.789798in}}%
\pgfpathlineto{\pgfqpoint{4.528681in}{1.767242in}}%
\pgfpathlineto{\pgfqpoint{4.531491in}{1.845008in}}%
\pgfpathlineto{\pgfqpoint{4.534302in}{1.668445in}}%
\pgfpathlineto{\pgfqpoint{4.537113in}{1.701096in}}%
\pgfpathlineto{\pgfqpoint{4.539923in}{1.773442in}}%
\pgfpathlineto{\pgfqpoint{4.542734in}{1.670353in}}%
\pgfpathlineto{\pgfqpoint{4.545545in}{1.760542in}}%
\pgfpathlineto{\pgfqpoint{4.548356in}{1.735144in}}%
\pgfpathlineto{\pgfqpoint{4.551166in}{1.616483in}}%
\pgfpathlineto{\pgfqpoint{4.553977in}{1.681202in}}%
\pgfpathlineto{\pgfqpoint{4.556788in}{1.876755in}}%
\pgfpathlineto{\pgfqpoint{4.559598in}{1.611583in}}%
\pgfpathlineto{\pgfqpoint{4.562409in}{1.738550in}}%
\pgfpathlineto{\pgfqpoint{4.565220in}{1.717018in}}%
\pgfpathlineto{\pgfqpoint{4.568030in}{1.713416in}}%
\pgfpathlineto{\pgfqpoint{4.570841in}{1.674198in}}%
\pgfpathlineto{\pgfqpoint{4.573652in}{1.718753in}}%
\pgfpathlineto{\pgfqpoint{4.576462in}{1.800258in}}%
\pgfpathlineto{\pgfqpoint{4.579273in}{1.757310in}}%
\pgfpathlineto{\pgfqpoint{4.582084in}{1.594876in}}%
\pgfpathlineto{\pgfqpoint{4.584894in}{1.757443in}}%
\pgfpathlineto{\pgfqpoint{4.587705in}{1.713181in}}%
\pgfpathlineto{\pgfqpoint{4.590516in}{1.725465in}}%
\pgfpathlineto{\pgfqpoint{4.593327in}{1.508086in}}%
\pgfpathlineto{\pgfqpoint{4.596137in}{1.495086in}}%
\pgfpathlineto{\pgfqpoint{4.598948in}{1.772889in}}%
\pgfpathlineto{\pgfqpoint{4.601759in}{1.751059in}}%
\pgfpathlineto{\pgfqpoint{4.604569in}{1.750837in}}%
\pgfpathlineto{\pgfqpoint{4.607380in}{1.649271in}}%
\pgfpathlineto{\pgfqpoint{4.610191in}{1.722317in}}%
\pgfpathlineto{\pgfqpoint{4.613001in}{1.702701in}}%
\pgfpathlineto{\pgfqpoint{4.615812in}{1.661749in}}%
\pgfpathlineto{\pgfqpoint{4.618623in}{1.667047in}}%
\pgfpathlineto{\pgfqpoint{4.621433in}{1.779263in}}%
\pgfpathlineto{\pgfqpoint{4.624244in}{1.590493in}}%
\pgfpathlineto{\pgfqpoint{4.627055in}{1.734853in}}%
\pgfpathlineto{\pgfqpoint{4.629865in}{1.702712in}}%
\pgfpathlineto{\pgfqpoint{4.632676in}{1.631354in}}%
\pgfpathlineto{\pgfqpoint{4.635487in}{1.711677in}}%
\pgfpathlineto{\pgfqpoint{4.638298in}{1.690237in}}%
\pgfpathlineto{\pgfqpoint{4.641108in}{1.663424in}}%
\pgfpathlineto{\pgfqpoint{4.643919in}{1.724193in}}%
\pgfpathlineto{\pgfqpoint{4.646730in}{1.857188in}}%
\pgfpathlineto{\pgfqpoint{4.649540in}{1.677884in}}%
\pgfpathlineto{\pgfqpoint{4.652351in}{1.695559in}}%
\pgfpathlineto{\pgfqpoint{4.655162in}{1.934092in}}%
\pgfpathlineto{\pgfqpoint{4.657972in}{1.692023in}}%
\pgfpathlineto{\pgfqpoint{4.660783in}{1.811664in}}%
\pgfpathlineto{\pgfqpoint{4.666404in}{1.643717in}}%
\pgfpathlineto{\pgfqpoint{4.669215in}{1.653963in}}%
\pgfpathlineto{\pgfqpoint{4.672026in}{1.627804in}}%
\pgfpathlineto{\pgfqpoint{4.674836in}{1.698978in}}%
\pgfpathlineto{\pgfqpoint{4.677647in}{1.907831in}}%
\pgfpathlineto{\pgfqpoint{4.680458in}{1.676584in}}%
\pgfpathlineto{\pgfqpoint{4.683268in}{1.721172in}}%
\pgfpathlineto{\pgfqpoint{4.686079in}{1.602688in}}%
\pgfpathlineto{\pgfqpoint{4.688890in}{1.662706in}}%
\pgfpathlineto{\pgfqpoint{4.691701in}{1.705826in}}%
\pgfpathlineto{\pgfqpoint{4.694511in}{1.735079in}}%
\pgfpathlineto{\pgfqpoint{4.697322in}{1.666203in}}%
\pgfpathlineto{\pgfqpoint{4.700133in}{1.798468in}}%
\pgfpathlineto{\pgfqpoint{4.702943in}{1.775638in}}%
\pgfpathlineto{\pgfqpoint{4.705754in}{1.678402in}}%
\pgfpathlineto{\pgfqpoint{4.708565in}{1.486270in}}%
\pgfpathlineto{\pgfqpoint{4.711375in}{1.683391in}}%
\pgfpathlineto{\pgfqpoint{4.714186in}{1.822720in}}%
\pgfpathlineto{\pgfqpoint{4.716997in}{1.830124in}}%
\pgfpathlineto{\pgfqpoint{4.719807in}{1.785103in}}%
\pgfpathlineto{\pgfqpoint{4.722618in}{1.848161in}}%
\pgfpathlineto{\pgfqpoint{4.728239in}{1.630243in}}%
\pgfpathlineto{\pgfqpoint{4.733861in}{1.777070in}}%
\pgfpathlineto{\pgfqpoint{4.736672in}{1.718609in}}%
\pgfpathlineto{\pgfqpoint{4.739482in}{1.814399in}}%
\pgfpathlineto{\pgfqpoint{4.742293in}{1.757775in}}%
\pgfpathlineto{\pgfqpoint{4.745104in}{1.568510in}}%
\pgfpathlineto{\pgfqpoint{4.747914in}{1.739897in}}%
\pgfpathlineto{\pgfqpoint{4.750725in}{1.692023in}}%
\pgfpathlineto{\pgfqpoint{4.753536in}{1.442798in}}%
\pgfpathlineto{\pgfqpoint{4.756346in}{1.515167in}}%
\pgfpathlineto{\pgfqpoint{4.759157in}{1.954063in}}%
\pgfpathlineto{\pgfqpoint{4.764778in}{1.501457in}}%
\pgfpathlineto{\pgfqpoint{4.767589in}{1.692023in}}%
\pgfpathlineto{\pgfqpoint{4.770400in}{1.794282in}}%
\pgfpathlineto{\pgfqpoint{4.773210in}{1.625044in}}%
\pgfpathlineto{\pgfqpoint{4.776021in}{1.718849in}}%
\pgfpathlineto{\pgfqpoint{4.778832in}{1.594558in}}%
\pgfpathlineto{\pgfqpoint{4.781643in}{1.959429in}}%
\pgfpathlineto{\pgfqpoint{4.784453in}{1.633918in}}%
\pgfpathlineto{\pgfqpoint{4.787264in}{1.638708in}}%
\pgfpathlineto{\pgfqpoint{4.790075in}{1.909135in}}%
\pgfpathlineto{\pgfqpoint{4.792885in}{1.396273in}}%
\pgfpathlineto{\pgfqpoint{4.795696in}{1.585990in}}%
\pgfpathlineto{\pgfqpoint{4.798507in}{1.619235in}}%
\pgfpathlineto{\pgfqpoint{4.801317in}{1.801920in}}%
\pgfpathlineto{\pgfqpoint{4.804128in}{1.566848in}}%
\pgfpathlineto{\pgfqpoint{4.806939in}{1.803713in}}%
\pgfpathlineto{\pgfqpoint{4.809749in}{1.884745in}}%
\pgfpathlineto{\pgfqpoint{4.812560in}{1.743570in}}%
\pgfpathlineto{\pgfqpoint{4.815371in}{1.817680in}}%
\pgfpathlineto{\pgfqpoint{4.818181in}{1.654102in}}%
\pgfpathlineto{\pgfqpoint{4.820992in}{1.851329in}}%
\pgfpathlineto{\pgfqpoint{4.823803in}{1.675675in}}%
\pgfpathlineto{\pgfqpoint{4.826614in}{1.812597in}}%
\pgfpathlineto{\pgfqpoint{4.829424in}{1.798782in}}%
\pgfpathlineto{\pgfqpoint{4.832235in}{1.669438in}}%
\pgfpathlineto{\pgfqpoint{4.835046in}{1.927549in}}%
\pgfpathlineto{\pgfqpoint{4.837856in}{1.812412in}}%
\pgfpathlineto{\pgfqpoint{4.840667in}{1.633529in}}%
\pgfpathlineto{\pgfqpoint{4.843478in}{1.824506in}}%
\pgfpathlineto{\pgfqpoint{4.846288in}{1.492874in}}%
\pgfpathlineto{\pgfqpoint{4.854720in}{1.895904in}}%
\pgfpathlineto{\pgfqpoint{4.857531in}{1.756268in}}%
\pgfpathlineto{\pgfqpoint{4.860342in}{1.585325in}}%
\pgfpathlineto{\pgfqpoint{4.863152in}{1.737616in}}%
\pgfpathlineto{\pgfqpoint{4.865963in}{1.784372in}}%
\pgfpathlineto{\pgfqpoint{4.868774in}{1.660779in}}%
\pgfpathlineto{\pgfqpoint{4.871584in}{1.643472in}}%
\pgfpathlineto{\pgfqpoint{4.874395in}{1.461308in}}%
\pgfpathlineto{\pgfqpoint{4.877206in}{1.787250in}}%
\pgfpathlineto{\pgfqpoint{4.880017in}{1.966496in}}%
\pgfpathlineto{\pgfqpoint{4.882827in}{1.808192in}}%
\pgfpathlineto{\pgfqpoint{4.885638in}{1.821114in}}%
\pgfpathlineto{\pgfqpoint{4.888449in}{1.516568in}}%
\pgfpathlineto{\pgfqpoint{4.891259in}{1.744558in}}%
\pgfpathlineto{\pgfqpoint{4.894070in}{1.799631in}}%
\pgfpathlineto{\pgfqpoint{4.896881in}{1.820180in}}%
\pgfpathlineto{\pgfqpoint{4.899691in}{1.557736in}}%
\pgfpathlineto{\pgfqpoint{4.902502in}{1.698152in}}%
\pgfpathlineto{\pgfqpoint{4.905313in}{1.711926in}}%
\pgfpathlineto{\pgfqpoint{4.908123in}{1.757675in}}%
\pgfpathlineto{\pgfqpoint{4.910934in}{1.690499in}}%
\pgfpathlineto{\pgfqpoint{4.913745in}{1.685927in}}%
\pgfpathlineto{\pgfqpoint{4.916555in}{1.659979in}}%
\pgfpathlineto{\pgfqpoint{4.919366in}{1.748437in}}%
\pgfpathlineto{\pgfqpoint{4.922177in}{1.559048in}}%
\pgfpathlineto{\pgfqpoint{4.927798in}{1.900585in}}%
\pgfpathlineto{\pgfqpoint{4.930609in}{1.672352in}}%
\pgfpathlineto{\pgfqpoint{4.933420in}{1.945836in}}%
\pgfpathlineto{\pgfqpoint{4.936230in}{1.700959in}}%
\pgfpathlineto{\pgfqpoint{4.939041in}{1.636838in}}%
\pgfpathlineto{\pgfqpoint{4.941852in}{1.729331in}}%
\pgfpathlineto{\pgfqpoint{4.944662in}{1.576826in}}%
\pgfpathlineto{\pgfqpoint{4.947473in}{1.621288in}}%
\pgfpathlineto{\pgfqpoint{4.950284in}{1.777767in}}%
\pgfpathlineto{\pgfqpoint{4.953094in}{1.627383in}}%
\pgfpathlineto{\pgfqpoint{4.955905in}{1.743156in}}%
\pgfpathlineto{\pgfqpoint{4.958716in}{1.728016in}}%
\pgfpathlineto{\pgfqpoint{4.961526in}{1.312957in}}%
\pgfpathlineto{\pgfqpoint{4.964337in}{1.679739in}}%
\pgfpathlineto{\pgfqpoint{4.967148in}{1.834225in}}%
\pgfpathlineto{\pgfqpoint{4.969959in}{1.796696in}}%
\pgfpathlineto{\pgfqpoint{4.972769in}{1.789990in}}%
\pgfpathlineto{\pgfqpoint{4.975580in}{1.720544in}}%
\pgfpathlineto{\pgfqpoint{4.978391in}{1.714503in}}%
\pgfpathlineto{\pgfqpoint{4.981201in}{1.651536in}}%
\pgfpathlineto{\pgfqpoint{4.984012in}{1.828061in}}%
\pgfpathlineto{\pgfqpoint{4.986823in}{1.764786in}}%
\pgfpathlineto{\pgfqpoint{4.989633in}{1.731972in}}%
\pgfpathlineto{\pgfqpoint{4.992444in}{1.736291in}}%
\pgfpathlineto{\pgfqpoint{4.995255in}{1.856192in}}%
\pgfpathlineto{\pgfqpoint{4.998065in}{1.643831in}}%
\pgfpathlineto{\pgfqpoint{5.000876in}{1.770805in}}%
\pgfpathlineto{\pgfqpoint{5.003687in}{1.641006in}}%
\pgfpathlineto{\pgfqpoint{5.006497in}{1.582136in}}%
\pgfpathlineto{\pgfqpoint{5.009308in}{1.728738in}}%
\pgfpathlineto{\pgfqpoint{5.012119in}{1.578662in}}%
\pgfpathlineto{\pgfqpoint{5.014930in}{1.770139in}}%
\pgfpathlineto{\pgfqpoint{5.017740in}{1.689082in}}%
\pgfpathlineto{\pgfqpoint{5.020551in}{1.819433in}}%
\pgfpathlineto{\pgfqpoint{5.023362in}{1.845865in}}%
\pgfpathlineto{\pgfqpoint{5.026172in}{1.736729in}}%
\pgfpathlineto{\pgfqpoint{5.028983in}{1.679057in}}%
\pgfpathlineto{\pgfqpoint{5.031794in}{1.738076in}}%
\pgfpathlineto{\pgfqpoint{5.034604in}{1.683398in}}%
\pgfpathlineto{\pgfqpoint{5.037415in}{1.846540in}}%
\pgfpathlineto{\pgfqpoint{5.040226in}{1.679205in}}%
\pgfpathlineto{\pgfqpoint{5.043036in}{1.746073in}}%
\pgfpathlineto{\pgfqpoint{5.045847in}{1.769923in}}%
\pgfpathlineto{\pgfqpoint{5.048658in}{1.716023in}}%
\pgfpathlineto{\pgfqpoint{5.051468in}{1.744130in}}%
\pgfpathlineto{\pgfqpoint{5.054279in}{1.646967in}}%
\pgfpathlineto{\pgfqpoint{5.057090in}{1.655319in}}%
\pgfpathlineto{\pgfqpoint{5.059901in}{1.625454in}}%
\pgfpathlineto{\pgfqpoint{5.062711in}{1.772719in}}%
\pgfpathlineto{\pgfqpoint{5.065522in}{1.448673in}}%
\pgfpathlineto{\pgfqpoint{5.068333in}{1.829108in}}%
\pgfpathlineto{\pgfqpoint{5.071143in}{1.521915in}}%
\pgfpathlineto{\pgfqpoint{5.076765in}{1.903055in}}%
\pgfpathlineto{\pgfqpoint{5.079575in}{1.603965in}}%
\pgfpathlineto{\pgfqpoint{5.082386in}{1.852145in}}%
\pgfpathlineto{\pgfqpoint{5.085197in}{1.749717in}}%
\pgfpathlineto{\pgfqpoint{5.088007in}{1.827675in}}%
\pgfpathlineto{\pgfqpoint{5.090818in}{1.202656in}}%
\pgfpathlineto{\pgfqpoint{5.093629in}{2.075206in}}%
\pgfpathlineto{\pgfqpoint{5.096439in}{1.431915in}}%
\pgfpathlineto{\pgfqpoint{5.099250in}{1.220258in}}%
\pgfpathlineto{\pgfqpoint{5.102061in}{1.857169in}}%
\pgfpathlineto{\pgfqpoint{5.104871in}{1.858316in}}%
\pgfpathlineto{\pgfqpoint{5.107682in}{1.926148in}}%
\pgfpathlineto{\pgfqpoint{5.110493in}{1.859734in}}%
\pgfpathlineto{\pgfqpoint{5.113304in}{1.659748in}}%
\pgfpathlineto{\pgfqpoint{5.116114in}{1.842997in}}%
\pgfpathlineto{\pgfqpoint{5.118925in}{1.671141in}}%
\pgfpathlineto{\pgfqpoint{5.121736in}{1.637600in}}%
\pgfpathlineto{\pgfqpoint{5.124546in}{1.815943in}}%
\pgfpathlineto{\pgfqpoint{5.130168in}{1.496192in}}%
\pgfpathlineto{\pgfqpoint{5.132978in}{1.689207in}}%
\pgfpathlineto{\pgfqpoint{5.135789in}{1.732798in}}%
\pgfpathlineto{\pgfqpoint{5.138600in}{1.731291in}}%
\pgfpathlineto{\pgfqpoint{5.141410in}{1.732589in}}%
\pgfpathlineto{\pgfqpoint{5.144221in}{1.833873in}}%
\pgfpathlineto{\pgfqpoint{5.149842in}{1.646405in}}%
\pgfpathlineto{\pgfqpoint{5.149842in}{1.646405in}}%
\pgfusepath{stroke}%
\end{pgfscope}%
\begin{pgfscope}%
\pgfpathrectangle{\pgfqpoint{0.711206in}{0.331635in}}{\pgfqpoint{4.650000in}{3.020000in}}%
\pgfusepath{clip}%
\pgfsetroundcap%
\pgfsetroundjoin%
\pgfsetlinewidth{1.505625pt}%
\definecolor{currentstroke}{rgb}{1.000000,0.647059,0.000000}%
\pgfsetstrokecolor{currentstroke}%
\pgfsetdash{}{0pt}%
\pgfpathmoveto{\pgfqpoint{0.922570in}{1.577636in}}%
\pgfpathlineto{\pgfqpoint{0.925380in}{1.574238in}}%
\pgfpathlineto{\pgfqpoint{0.928191in}{1.594938in}}%
\pgfpathlineto{\pgfqpoint{0.931002in}{1.593193in}}%
\pgfpathlineto{\pgfqpoint{0.933812in}{1.656112in}}%
\pgfpathlineto{\pgfqpoint{0.936623in}{1.649128in}}%
\pgfpathlineto{\pgfqpoint{0.939434in}{1.684309in}}%
\pgfpathlineto{\pgfqpoint{0.942245in}{1.704442in}}%
\pgfpathlineto{\pgfqpoint{0.947866in}{1.661655in}}%
\pgfpathlineto{\pgfqpoint{0.950677in}{1.651931in}}%
\pgfpathlineto{\pgfqpoint{0.953487in}{1.670542in}}%
\pgfpathlineto{\pgfqpoint{0.956298in}{1.667256in}}%
\pgfpathlineto{\pgfqpoint{0.959109in}{1.671902in}}%
\pgfpathlineto{\pgfqpoint{0.961919in}{1.674748in}}%
\pgfpathlineto{\pgfqpoint{0.964730in}{1.665452in}}%
\pgfpathlineto{\pgfqpoint{0.967541in}{1.673216in}}%
\pgfpathlineto{\pgfqpoint{0.970351in}{1.688552in}}%
\pgfpathlineto{\pgfqpoint{0.973162in}{1.714121in}}%
\pgfpathlineto{\pgfqpoint{0.975973in}{1.719506in}}%
\pgfpathlineto{\pgfqpoint{0.978783in}{1.720336in}}%
\pgfpathlineto{\pgfqpoint{0.981594in}{1.716938in}}%
\pgfpathlineto{\pgfqpoint{0.984405in}{1.723890in}}%
\pgfpathlineto{\pgfqpoint{0.987216in}{1.744783in}}%
\pgfpathlineto{\pgfqpoint{0.990026in}{1.750299in}}%
\pgfpathlineto{\pgfqpoint{0.992837in}{1.742372in}}%
\pgfpathlineto{\pgfqpoint{0.995648in}{1.752506in}}%
\pgfpathlineto{\pgfqpoint{0.998458in}{1.753986in}}%
\pgfpathlineto{\pgfqpoint{1.001269in}{1.746202in}}%
\pgfpathlineto{\pgfqpoint{1.004080in}{1.749729in}}%
\pgfpathlineto{\pgfqpoint{1.006890in}{1.744612in}}%
\pgfpathlineto{\pgfqpoint{1.009701in}{1.751733in}}%
\pgfpathlineto{\pgfqpoint{1.012512in}{1.750374in}}%
\pgfpathlineto{\pgfqpoint{1.015322in}{1.752786in}}%
\pgfpathlineto{\pgfqpoint{1.018133in}{1.748696in}}%
\pgfpathlineto{\pgfqpoint{1.020944in}{1.754470in}}%
\pgfpathlineto{\pgfqpoint{1.023754in}{1.744078in}}%
\pgfpathlineto{\pgfqpoint{1.026565in}{1.745504in}}%
\pgfpathlineto{\pgfqpoint{1.029376in}{1.740796in}}%
\pgfpathlineto{\pgfqpoint{1.032187in}{1.739925in}}%
\pgfpathlineto{\pgfqpoint{1.034997in}{1.734394in}}%
\pgfpathlineto{\pgfqpoint{1.037808in}{1.735944in}}%
\pgfpathlineto{\pgfqpoint{1.040619in}{1.741835in}}%
\pgfpathlineto{\pgfqpoint{1.046240in}{1.735605in}}%
\pgfpathlineto{\pgfqpoint{1.049051in}{1.736842in}}%
\pgfpathlineto{\pgfqpoint{1.051861in}{1.734446in}}%
\pgfpathlineto{\pgfqpoint{1.054672in}{1.734324in}}%
\pgfpathlineto{\pgfqpoint{1.057483in}{1.732663in}}%
\pgfpathlineto{\pgfqpoint{1.060293in}{1.737314in}}%
\pgfpathlineto{\pgfqpoint{1.063104in}{1.730671in}}%
\pgfpathlineto{\pgfqpoint{1.065915in}{1.730518in}}%
\pgfpathlineto{\pgfqpoint{1.068725in}{1.731137in}}%
\pgfpathlineto{\pgfqpoint{1.074347in}{1.736148in}}%
\pgfpathlineto{\pgfqpoint{1.079968in}{1.733049in}}%
\pgfpathlineto{\pgfqpoint{1.082779in}{1.731672in}}%
\pgfpathlineto{\pgfqpoint{1.085590in}{1.728868in}}%
\pgfpathlineto{\pgfqpoint{1.088400in}{1.730297in}}%
\pgfpathlineto{\pgfqpoint{1.091211in}{1.732600in}}%
\pgfpathlineto{\pgfqpoint{1.094022in}{1.729101in}}%
\pgfpathlineto{\pgfqpoint{1.096832in}{1.732615in}}%
\pgfpathlineto{\pgfqpoint{1.099643in}{1.729057in}}%
\pgfpathlineto{\pgfqpoint{1.102454in}{1.723048in}}%
\pgfpathlineto{\pgfqpoint{1.105264in}{1.723837in}}%
\pgfpathlineto{\pgfqpoint{1.108075in}{1.730262in}}%
\pgfpathlineto{\pgfqpoint{1.110886in}{1.733609in}}%
\pgfpathlineto{\pgfqpoint{1.113696in}{1.728855in}}%
\pgfpathlineto{\pgfqpoint{1.116507in}{1.730609in}}%
\pgfpathlineto{\pgfqpoint{1.124939in}{1.727203in}}%
\pgfpathlineto{\pgfqpoint{1.127750in}{1.721754in}}%
\pgfpathlineto{\pgfqpoint{1.130561in}{1.722764in}}%
\pgfpathlineto{\pgfqpoint{1.133371in}{1.727120in}}%
\pgfpathlineto{\pgfqpoint{1.136182in}{1.728944in}}%
\pgfpathlineto{\pgfqpoint{1.138993in}{1.728984in}}%
\pgfpathlineto{\pgfqpoint{1.141803in}{1.727677in}}%
\pgfpathlineto{\pgfqpoint{1.144614in}{1.727453in}}%
\pgfpathlineto{\pgfqpoint{1.147425in}{1.725635in}}%
\pgfpathlineto{\pgfqpoint{1.150235in}{1.716264in}}%
\pgfpathlineto{\pgfqpoint{1.153046in}{1.718112in}}%
\pgfpathlineto{\pgfqpoint{1.155857in}{1.718923in}}%
\pgfpathlineto{\pgfqpoint{1.161478in}{1.716675in}}%
\pgfpathlineto{\pgfqpoint{1.164289in}{1.717478in}}%
\pgfpathlineto{\pgfqpoint{1.167099in}{1.716862in}}%
\pgfpathlineto{\pgfqpoint{1.169910in}{1.714799in}}%
\pgfpathlineto{\pgfqpoint{1.172721in}{1.714420in}}%
\pgfpathlineto{\pgfqpoint{1.175532in}{1.716770in}}%
\pgfpathlineto{\pgfqpoint{1.181153in}{1.709244in}}%
\pgfpathlineto{\pgfqpoint{1.183964in}{1.714181in}}%
\pgfpathlineto{\pgfqpoint{1.186774in}{1.716916in}}%
\pgfpathlineto{\pgfqpoint{1.192396in}{1.718260in}}%
\pgfpathlineto{\pgfqpoint{1.195206in}{1.717469in}}%
\pgfpathlineto{\pgfqpoint{1.198017in}{1.718378in}}%
\pgfpathlineto{\pgfqpoint{1.206449in}{1.707273in}}%
\pgfpathlineto{\pgfqpoint{1.209260in}{1.709610in}}%
\pgfpathlineto{\pgfqpoint{1.212070in}{1.708979in}}%
\pgfpathlineto{\pgfqpoint{1.214881in}{1.712072in}}%
\pgfpathlineto{\pgfqpoint{1.217692in}{1.712444in}}%
\pgfpathlineto{\pgfqpoint{1.223313in}{1.711525in}}%
\pgfpathlineto{\pgfqpoint{1.226124in}{1.712325in}}%
\pgfpathlineto{\pgfqpoint{1.228935in}{1.709441in}}%
\pgfpathlineto{\pgfqpoint{1.240177in}{1.716173in}}%
\pgfpathlineto{\pgfqpoint{1.242988in}{1.717296in}}%
\pgfpathlineto{\pgfqpoint{1.245799in}{1.713687in}}%
\pgfpathlineto{\pgfqpoint{1.248609in}{1.719399in}}%
\pgfpathlineto{\pgfqpoint{1.251420in}{1.715271in}}%
\pgfpathlineto{\pgfqpoint{1.254231in}{1.717075in}}%
\pgfpathlineto{\pgfqpoint{1.257041in}{1.717149in}}%
\pgfpathlineto{\pgfqpoint{1.259852in}{1.714987in}}%
\pgfpathlineto{\pgfqpoint{1.262663in}{1.716973in}}%
\pgfpathlineto{\pgfqpoint{1.265474in}{1.719738in}}%
\pgfpathlineto{\pgfqpoint{1.268284in}{1.719254in}}%
\pgfpathlineto{\pgfqpoint{1.271095in}{1.719694in}}%
\pgfpathlineto{\pgfqpoint{1.279527in}{1.714504in}}%
\pgfpathlineto{\pgfqpoint{1.282338in}{1.712787in}}%
\pgfpathlineto{\pgfqpoint{1.293580in}{1.719155in}}%
\pgfpathlineto{\pgfqpoint{1.296391in}{1.718593in}}%
\pgfpathlineto{\pgfqpoint{1.299202in}{1.716212in}}%
\pgfpathlineto{\pgfqpoint{1.302012in}{1.716361in}}%
\pgfpathlineto{\pgfqpoint{1.310445in}{1.712668in}}%
\pgfpathlineto{\pgfqpoint{1.313255in}{1.716537in}}%
\pgfpathlineto{\pgfqpoint{1.318877in}{1.719644in}}%
\pgfpathlineto{\pgfqpoint{1.324498in}{1.716471in}}%
\pgfpathlineto{\pgfqpoint{1.330119in}{1.719006in}}%
\pgfpathlineto{\pgfqpoint{1.338551in}{1.718612in}}%
\pgfpathlineto{\pgfqpoint{1.341362in}{1.716337in}}%
\pgfpathlineto{\pgfqpoint{1.344173in}{1.716641in}}%
\pgfpathlineto{\pgfqpoint{1.346983in}{1.715879in}}%
\pgfpathlineto{\pgfqpoint{1.349794in}{1.716953in}}%
\pgfpathlineto{\pgfqpoint{1.358226in}{1.716206in}}%
\pgfpathlineto{\pgfqpoint{1.363848in}{1.715018in}}%
\pgfpathlineto{\pgfqpoint{1.366658in}{1.715316in}}%
\pgfpathlineto{\pgfqpoint{1.372280in}{1.713955in}}%
\pgfpathlineto{\pgfqpoint{1.375090in}{1.714558in}}%
\pgfpathlineto{\pgfqpoint{1.377901in}{1.714255in}}%
\pgfpathlineto{\pgfqpoint{1.380712in}{1.714549in}}%
\pgfpathlineto{\pgfqpoint{1.383522in}{1.713463in}}%
\pgfpathlineto{\pgfqpoint{1.386333in}{1.714475in}}%
\pgfpathlineto{\pgfqpoint{1.389144in}{1.714535in}}%
\pgfpathlineto{\pgfqpoint{1.391954in}{1.713730in}}%
\pgfpathlineto{\pgfqpoint{1.394765in}{1.714968in}}%
\pgfpathlineto{\pgfqpoint{1.397576in}{1.714960in}}%
\pgfpathlineto{\pgfqpoint{1.400386in}{1.714109in}}%
\pgfpathlineto{\pgfqpoint{1.406008in}{1.717112in}}%
\pgfpathlineto{\pgfqpoint{1.408819in}{1.717974in}}%
\pgfpathlineto{\pgfqpoint{1.414440in}{1.716953in}}%
\pgfpathlineto{\pgfqpoint{1.417251in}{1.716697in}}%
\pgfpathlineto{\pgfqpoint{1.420061in}{1.717282in}}%
\pgfpathlineto{\pgfqpoint{1.431304in}{1.716592in}}%
\pgfpathlineto{\pgfqpoint{1.434115in}{1.715160in}}%
\pgfpathlineto{\pgfqpoint{1.436925in}{1.715909in}}%
\pgfpathlineto{\pgfqpoint{1.439736in}{1.716032in}}%
\pgfpathlineto{\pgfqpoint{1.442547in}{1.717383in}}%
\pgfpathlineto{\pgfqpoint{1.445357in}{1.716815in}}%
\pgfpathlineto{\pgfqpoint{1.448168in}{1.718183in}}%
\pgfpathlineto{\pgfqpoint{1.453790in}{1.718928in}}%
\pgfpathlineto{\pgfqpoint{1.459411in}{1.716625in}}%
\pgfpathlineto{\pgfqpoint{1.473464in}{1.718480in}}%
\pgfpathlineto{\pgfqpoint{1.476275in}{1.719134in}}%
\pgfpathlineto{\pgfqpoint{1.479086in}{1.718633in}}%
\pgfpathlineto{\pgfqpoint{1.487518in}{1.715344in}}%
\pgfpathlineto{\pgfqpoint{1.490328in}{1.715155in}}%
\pgfpathlineto{\pgfqpoint{1.498761in}{1.716065in}}%
\pgfpathlineto{\pgfqpoint{1.501571in}{1.718643in}}%
\pgfpathlineto{\pgfqpoint{1.504382in}{1.718262in}}%
\pgfpathlineto{\pgfqpoint{1.507193in}{1.717162in}}%
\pgfpathlineto{\pgfqpoint{1.510003in}{1.718266in}}%
\pgfpathlineto{\pgfqpoint{1.515625in}{1.717109in}}%
\pgfpathlineto{\pgfqpoint{1.521246in}{1.717465in}}%
\pgfpathlineto{\pgfqpoint{1.529678in}{1.716154in}}%
\pgfpathlineto{\pgfqpoint{1.535299in}{1.718717in}}%
\pgfpathlineto{\pgfqpoint{1.543731in}{1.719536in}}%
\pgfpathlineto{\pgfqpoint{1.549353in}{1.718670in}}%
\pgfpathlineto{\pgfqpoint{1.554974in}{1.719074in}}%
\pgfpathlineto{\pgfqpoint{1.557785in}{1.719665in}}%
\pgfpathlineto{\pgfqpoint{1.563406in}{1.718459in}}%
\pgfpathlineto{\pgfqpoint{1.577460in}{1.718492in}}%
\pgfpathlineto{\pgfqpoint{1.585892in}{1.717241in}}%
\pgfpathlineto{\pgfqpoint{1.591513in}{1.718728in}}%
\pgfpathlineto{\pgfqpoint{1.594324in}{1.717951in}}%
\pgfpathlineto{\pgfqpoint{1.597135in}{1.719240in}}%
\pgfpathlineto{\pgfqpoint{1.599945in}{1.718418in}}%
\pgfpathlineto{\pgfqpoint{1.602756in}{1.718479in}}%
\pgfpathlineto{\pgfqpoint{1.608377in}{1.717520in}}%
\pgfpathlineto{\pgfqpoint{1.611188in}{1.717099in}}%
\pgfpathlineto{\pgfqpoint{1.616809in}{1.719663in}}%
\pgfpathlineto{\pgfqpoint{1.622431in}{1.719989in}}%
\pgfpathlineto{\pgfqpoint{1.628052in}{1.720785in}}%
\pgfpathlineto{\pgfqpoint{1.630863in}{1.721600in}}%
\pgfpathlineto{\pgfqpoint{1.633673in}{1.720999in}}%
\pgfpathlineto{\pgfqpoint{1.639295in}{1.720824in}}%
\pgfpathlineto{\pgfqpoint{1.661780in}{1.719666in}}%
\pgfpathlineto{\pgfqpoint{1.664591in}{1.718269in}}%
\pgfpathlineto{\pgfqpoint{1.667402in}{1.718286in}}%
\pgfpathlineto{\pgfqpoint{1.670212in}{1.717579in}}%
\pgfpathlineto{\pgfqpoint{1.673023in}{1.718548in}}%
\pgfpathlineto{\pgfqpoint{1.675834in}{1.718692in}}%
\pgfpathlineto{\pgfqpoint{1.678644in}{1.717900in}}%
\pgfpathlineto{\pgfqpoint{1.681455in}{1.718846in}}%
\pgfpathlineto{\pgfqpoint{1.684266in}{1.719205in}}%
\pgfpathlineto{\pgfqpoint{1.687077in}{1.717762in}}%
\pgfpathlineto{\pgfqpoint{1.689887in}{1.717939in}}%
\pgfpathlineto{\pgfqpoint{1.692698in}{1.717306in}}%
\pgfpathlineto{\pgfqpoint{1.706751in}{1.717790in}}%
\pgfpathlineto{\pgfqpoint{1.709562in}{1.716926in}}%
\pgfpathlineto{\pgfqpoint{1.715183in}{1.718141in}}%
\pgfpathlineto{\pgfqpoint{1.717994in}{1.716818in}}%
\pgfpathlineto{\pgfqpoint{1.723615in}{1.718018in}}%
\pgfpathlineto{\pgfqpoint{1.729237in}{1.717229in}}%
\pgfpathlineto{\pgfqpoint{1.737669in}{1.717968in}}%
\pgfpathlineto{\pgfqpoint{1.743290in}{1.717662in}}%
\pgfpathlineto{\pgfqpoint{1.746101in}{1.717748in}}%
\pgfpathlineto{\pgfqpoint{1.751722in}{1.716987in}}%
\pgfpathlineto{\pgfqpoint{1.754533in}{1.717319in}}%
\pgfpathlineto{\pgfqpoint{1.762965in}{1.715548in}}%
\pgfpathlineto{\pgfqpoint{1.765776in}{1.716561in}}%
\pgfpathlineto{\pgfqpoint{1.768586in}{1.715983in}}%
\pgfpathlineto{\pgfqpoint{1.774208in}{1.717789in}}%
\pgfpathlineto{\pgfqpoint{1.777018in}{1.718770in}}%
\pgfpathlineto{\pgfqpoint{1.782640in}{1.719307in}}%
\pgfpathlineto{\pgfqpoint{1.785451in}{1.718471in}}%
\pgfpathlineto{\pgfqpoint{1.788261in}{1.718570in}}%
\pgfpathlineto{\pgfqpoint{1.791072in}{1.717521in}}%
\pgfpathlineto{\pgfqpoint{1.793883in}{1.717955in}}%
\pgfpathlineto{\pgfqpoint{1.796693in}{1.717478in}}%
\pgfpathlineto{\pgfqpoint{1.799504in}{1.717738in}}%
\pgfpathlineto{\pgfqpoint{1.802315in}{1.717198in}}%
\pgfpathlineto{\pgfqpoint{1.807936in}{1.717883in}}%
\pgfpathlineto{\pgfqpoint{1.810747in}{1.717260in}}%
\pgfpathlineto{\pgfqpoint{1.813557in}{1.715845in}}%
\pgfpathlineto{\pgfqpoint{1.816368in}{1.716793in}}%
\pgfpathlineto{\pgfqpoint{1.821989in}{1.715664in}}%
\pgfpathlineto{\pgfqpoint{1.824800in}{1.716345in}}%
\pgfpathlineto{\pgfqpoint{1.827611in}{1.716141in}}%
\pgfpathlineto{\pgfqpoint{1.836043in}{1.717418in}}%
\pgfpathlineto{\pgfqpoint{1.838854in}{1.716919in}}%
\pgfpathlineto{\pgfqpoint{1.844475in}{1.717114in}}%
\pgfpathlineto{\pgfqpoint{1.847286in}{1.716347in}}%
\pgfpathlineto{\pgfqpoint{1.850096in}{1.718855in}}%
\pgfpathlineto{\pgfqpoint{1.852907in}{1.719861in}}%
\pgfpathlineto{\pgfqpoint{1.864150in}{1.719283in}}%
\pgfpathlineto{\pgfqpoint{1.872582in}{1.719491in}}%
\pgfpathlineto{\pgfqpoint{1.875393in}{1.720105in}}%
\pgfpathlineto{\pgfqpoint{1.878203in}{1.719530in}}%
\pgfpathlineto{\pgfqpoint{1.881014in}{1.720633in}}%
\pgfpathlineto{\pgfqpoint{1.883825in}{1.719818in}}%
\pgfpathlineto{\pgfqpoint{1.889446in}{1.719300in}}%
\pgfpathlineto{\pgfqpoint{1.892257in}{1.718708in}}%
\pgfpathlineto{\pgfqpoint{1.895067in}{1.719219in}}%
\pgfpathlineto{\pgfqpoint{1.909121in}{1.718771in}}%
\pgfpathlineto{\pgfqpoint{1.911931in}{1.718696in}}%
\pgfpathlineto{\pgfqpoint{1.914742in}{1.717894in}}%
\pgfpathlineto{\pgfqpoint{1.923174in}{1.718911in}}%
\pgfpathlineto{\pgfqpoint{1.928796in}{1.718001in}}%
\pgfpathlineto{\pgfqpoint{1.931606in}{1.718668in}}%
\pgfpathlineto{\pgfqpoint{1.934417in}{1.718284in}}%
\pgfpathlineto{\pgfqpoint{1.940038in}{1.718866in}}%
\pgfpathlineto{\pgfqpoint{1.942849in}{1.718468in}}%
\pgfpathlineto{\pgfqpoint{1.945660in}{1.717410in}}%
\pgfpathlineto{\pgfqpoint{1.976577in}{1.718842in}}%
\pgfpathlineto{\pgfqpoint{1.982199in}{1.718396in}}%
\pgfpathlineto{\pgfqpoint{1.987820in}{1.719126in}}%
\pgfpathlineto{\pgfqpoint{2.001873in}{1.718609in}}%
\pgfpathlineto{\pgfqpoint{2.004684in}{1.718817in}}%
\pgfpathlineto{\pgfqpoint{2.010305in}{1.717729in}}%
\pgfpathlineto{\pgfqpoint{2.013116in}{1.719315in}}%
\pgfpathlineto{\pgfqpoint{2.021548in}{1.718454in}}%
\pgfpathlineto{\pgfqpoint{2.024359in}{1.715291in}}%
\pgfpathlineto{\pgfqpoint{2.027170in}{1.715705in}}%
\pgfpathlineto{\pgfqpoint{2.029980in}{1.716693in}}%
\pgfpathlineto{\pgfqpoint{2.035602in}{1.716337in}}%
\pgfpathlineto{\pgfqpoint{2.044034in}{1.715493in}}%
\pgfpathlineto{\pgfqpoint{2.052466in}{1.715291in}}%
\pgfpathlineto{\pgfqpoint{2.055276in}{1.714270in}}%
\pgfpathlineto{\pgfqpoint{2.058087in}{1.714034in}}%
\pgfpathlineto{\pgfqpoint{2.060898in}{1.714389in}}%
\pgfpathlineto{\pgfqpoint{2.063709in}{1.713945in}}%
\pgfpathlineto{\pgfqpoint{2.066519in}{1.715016in}}%
\pgfpathlineto{\pgfqpoint{2.072141in}{1.715051in}}%
\pgfpathlineto{\pgfqpoint{2.074951in}{1.714127in}}%
\pgfpathlineto{\pgfqpoint{2.080573in}{1.714058in}}%
\pgfpathlineto{\pgfqpoint{2.100247in}{1.714398in}}%
\pgfpathlineto{\pgfqpoint{2.103058in}{1.715568in}}%
\pgfpathlineto{\pgfqpoint{2.105869in}{1.715856in}}%
\pgfpathlineto{\pgfqpoint{2.108680in}{1.715552in}}%
\pgfpathlineto{\pgfqpoint{2.111490in}{1.716268in}}%
\pgfpathlineto{\pgfqpoint{2.117112in}{1.716652in}}%
\pgfpathlineto{\pgfqpoint{2.125544in}{1.717822in}}%
\pgfpathlineto{\pgfqpoint{2.133976in}{1.716304in}}%
\pgfpathlineto{\pgfqpoint{2.139597in}{1.716474in}}%
\pgfpathlineto{\pgfqpoint{2.142408in}{1.716056in}}%
\pgfpathlineto{\pgfqpoint{2.145218in}{1.716394in}}%
\pgfpathlineto{\pgfqpoint{2.156461in}{1.714885in}}%
\pgfpathlineto{\pgfqpoint{2.159272in}{1.714115in}}%
\pgfpathlineto{\pgfqpoint{2.162083in}{1.714314in}}%
\pgfpathlineto{\pgfqpoint{2.167704in}{1.715769in}}%
\pgfpathlineto{\pgfqpoint{2.170515in}{1.715943in}}%
\pgfpathlineto{\pgfqpoint{2.173325in}{1.715513in}}%
\pgfpathlineto{\pgfqpoint{2.178947in}{1.716642in}}%
\pgfpathlineto{\pgfqpoint{2.184568in}{1.716833in}}%
\pgfpathlineto{\pgfqpoint{2.198621in}{1.717075in}}%
\pgfpathlineto{\pgfqpoint{2.204243in}{1.717096in}}%
\pgfpathlineto{\pgfqpoint{2.207054in}{1.715824in}}%
\pgfpathlineto{\pgfqpoint{2.209864in}{1.716202in}}%
\pgfpathlineto{\pgfqpoint{2.212675in}{1.715684in}}%
\pgfpathlineto{\pgfqpoint{2.226728in}{1.715905in}}%
\pgfpathlineto{\pgfqpoint{2.229539in}{1.715687in}}%
\pgfpathlineto{\pgfqpoint{2.235160in}{1.716169in}}%
\pgfpathlineto{\pgfqpoint{2.237971in}{1.716422in}}%
\pgfpathlineto{\pgfqpoint{2.246403in}{1.715631in}}%
\pgfpathlineto{\pgfqpoint{2.252025in}{1.716186in}}%
\pgfpathlineto{\pgfqpoint{2.263267in}{1.716198in}}%
\pgfpathlineto{\pgfqpoint{2.266078in}{1.716414in}}%
\pgfpathlineto{\pgfqpoint{2.268889in}{1.715830in}}%
\pgfpathlineto{\pgfqpoint{2.277321in}{1.715697in}}%
\pgfpathlineto{\pgfqpoint{2.282942in}{1.715216in}}%
\pgfpathlineto{\pgfqpoint{2.285753in}{1.716151in}}%
\pgfpathlineto{\pgfqpoint{2.288563in}{1.715753in}}%
\pgfpathlineto{\pgfqpoint{2.296996in}{1.717105in}}%
\pgfpathlineto{\pgfqpoint{2.308238in}{1.717598in}}%
\pgfpathlineto{\pgfqpoint{2.339156in}{1.717584in}}%
\pgfpathlineto{\pgfqpoint{2.347588in}{1.717540in}}%
\pgfpathlineto{\pgfqpoint{2.353209in}{1.717317in}}%
\pgfpathlineto{\pgfqpoint{2.356020in}{1.718666in}}%
\pgfpathlineto{\pgfqpoint{2.364452in}{1.717992in}}%
\pgfpathlineto{\pgfqpoint{2.370073in}{1.716262in}}%
\pgfpathlineto{\pgfqpoint{2.372884in}{1.716865in}}%
\pgfpathlineto{\pgfqpoint{2.375695in}{1.716294in}}%
\pgfpathlineto{\pgfqpoint{2.378505in}{1.716758in}}%
\pgfpathlineto{\pgfqpoint{2.384127in}{1.715649in}}%
\pgfpathlineto{\pgfqpoint{2.400991in}{1.716578in}}%
\pgfpathlineto{\pgfqpoint{2.406612in}{1.716925in}}%
\pgfpathlineto{\pgfqpoint{2.412234in}{1.717092in}}%
\pgfpathlineto{\pgfqpoint{2.420666in}{1.716604in}}%
\pgfpathlineto{\pgfqpoint{2.426287in}{1.716996in}}%
\pgfpathlineto{\pgfqpoint{2.445962in}{1.716258in}}%
\pgfpathlineto{\pgfqpoint{2.454394in}{1.716545in}}%
\pgfpathlineto{\pgfqpoint{2.465637in}{1.715945in}}%
\pgfpathlineto{\pgfqpoint{2.468447in}{1.716228in}}%
\pgfpathlineto{\pgfqpoint{2.474069in}{1.715569in}}%
\pgfpathlineto{\pgfqpoint{2.479690in}{1.715349in}}%
\pgfpathlineto{\pgfqpoint{2.490933in}{1.714065in}}%
\pgfpathlineto{\pgfqpoint{2.493744in}{1.714518in}}%
\pgfpathlineto{\pgfqpoint{2.504986in}{1.713294in}}%
\pgfpathlineto{\pgfqpoint{2.510608in}{1.712534in}}%
\pgfpathlineto{\pgfqpoint{2.513418in}{1.713165in}}%
\pgfpathlineto{\pgfqpoint{2.519040in}{1.711615in}}%
\pgfpathlineto{\pgfqpoint{2.527472in}{1.713214in}}%
\pgfpathlineto{\pgfqpoint{2.533093in}{1.713113in}}%
\pgfpathlineto{\pgfqpoint{2.541525in}{1.713037in}}%
\pgfpathlineto{\pgfqpoint{2.544336in}{1.711624in}}%
\pgfpathlineto{\pgfqpoint{2.555579in}{1.712435in}}%
\pgfpathlineto{\pgfqpoint{2.558389in}{1.712178in}}%
\pgfpathlineto{\pgfqpoint{2.561200in}{1.712497in}}%
\pgfpathlineto{\pgfqpoint{2.564011in}{1.712231in}}%
\pgfpathlineto{\pgfqpoint{2.569632in}{1.712868in}}%
\pgfpathlineto{\pgfqpoint{2.600550in}{1.712830in}}%
\pgfpathlineto{\pgfqpoint{2.606171in}{1.712972in}}%
\pgfpathlineto{\pgfqpoint{2.614603in}{1.712833in}}%
\pgfpathlineto{\pgfqpoint{2.620224in}{1.712513in}}%
\pgfpathlineto{\pgfqpoint{2.637089in}{1.712354in}}%
\pgfpathlineto{\pgfqpoint{2.668006in}{1.711699in}}%
\pgfpathlineto{\pgfqpoint{2.687681in}{1.712277in}}%
\pgfpathlineto{\pgfqpoint{2.693302in}{1.712309in}}%
\pgfpathlineto{\pgfqpoint{2.698924in}{1.712781in}}%
\pgfpathlineto{\pgfqpoint{2.701734in}{1.712847in}}%
\pgfpathlineto{\pgfqpoint{2.704545in}{1.712282in}}%
\pgfpathlineto{\pgfqpoint{2.712977in}{1.712563in}}%
\pgfpathlineto{\pgfqpoint{2.718599in}{1.712661in}}%
\pgfpathlineto{\pgfqpoint{2.721409in}{1.711747in}}%
\pgfpathlineto{\pgfqpoint{2.755137in}{1.710883in}}%
\pgfpathlineto{\pgfqpoint{2.760759in}{1.711256in}}%
\pgfpathlineto{\pgfqpoint{2.763570in}{1.710963in}}%
\pgfpathlineto{\pgfqpoint{2.769191in}{1.711443in}}%
\pgfpathlineto{\pgfqpoint{2.783244in}{1.711484in}}%
\pgfpathlineto{\pgfqpoint{2.788866in}{1.711193in}}%
\pgfpathlineto{\pgfqpoint{2.794487in}{1.711150in}}%
\pgfpathlineto{\pgfqpoint{2.833837in}{1.710668in}}%
\pgfpathlineto{\pgfqpoint{2.847890in}{1.710219in}}%
\pgfpathlineto{\pgfqpoint{2.850701in}{1.710452in}}%
\pgfpathlineto{\pgfqpoint{2.856322in}{1.710009in}}%
\pgfpathlineto{\pgfqpoint{2.861944in}{1.710177in}}%
\pgfpathlineto{\pgfqpoint{2.864754in}{1.709750in}}%
\pgfpathlineto{\pgfqpoint{2.867565in}{1.710187in}}%
\pgfpathlineto{\pgfqpoint{2.873186in}{1.709342in}}%
\pgfpathlineto{\pgfqpoint{2.887240in}{1.709332in}}%
\pgfpathlineto{\pgfqpoint{2.906915in}{1.710276in}}%
\pgfpathlineto{\pgfqpoint{2.909725in}{1.710041in}}%
\pgfpathlineto{\pgfqpoint{2.912536in}{1.712157in}}%
\pgfpathlineto{\pgfqpoint{2.918157in}{1.712546in}}%
\pgfpathlineto{\pgfqpoint{2.949075in}{1.713118in}}%
\pgfpathlineto{\pgfqpoint{2.960318in}{1.713420in}}%
\pgfpathlineto{\pgfqpoint{2.968750in}{1.713627in}}%
\pgfpathlineto{\pgfqpoint{3.013721in}{1.713648in}}%
\pgfpathlineto{\pgfqpoint{3.027774in}{1.713530in}}%
\pgfpathlineto{\pgfqpoint{3.039017in}{1.712860in}}%
\pgfpathlineto{\pgfqpoint{3.044638in}{1.713353in}}%
\pgfpathlineto{\pgfqpoint{3.050260in}{1.712948in}}%
\pgfpathlineto{\pgfqpoint{3.055881in}{1.712539in}}%
\pgfpathlineto{\pgfqpoint{3.061502in}{1.712439in}}%
\pgfpathlineto{\pgfqpoint{3.072745in}{1.712596in}}%
\pgfpathlineto{\pgfqpoint{3.086798in}{1.712198in}}%
\pgfpathlineto{\pgfqpoint{3.092420in}{1.712527in}}%
\pgfpathlineto{\pgfqpoint{3.098041in}{1.713381in}}%
\pgfpathlineto{\pgfqpoint{3.106473in}{1.712758in}}%
\pgfpathlineto{\pgfqpoint{3.117716in}{1.713266in}}%
\pgfpathlineto{\pgfqpoint{3.126148in}{1.713342in}}%
\pgfpathlineto{\pgfqpoint{3.157066in}{1.712935in}}%
\pgfpathlineto{\pgfqpoint{3.162687in}{1.712396in}}%
\pgfpathlineto{\pgfqpoint{3.165498in}{1.712723in}}%
\pgfpathlineto{\pgfqpoint{3.168308in}{1.712366in}}%
\pgfpathlineto{\pgfqpoint{3.171119in}{1.712633in}}%
\pgfpathlineto{\pgfqpoint{3.173930in}{1.712276in}}%
\pgfpathlineto{\pgfqpoint{3.182362in}{1.712575in}}%
\pgfpathlineto{\pgfqpoint{3.210469in}{1.711686in}}%
\pgfpathlineto{\pgfqpoint{3.221711in}{1.711919in}}%
\pgfpathlineto{\pgfqpoint{3.230143in}{1.711657in}}%
\pgfpathlineto{\pgfqpoint{3.241386in}{1.711467in}}%
\pgfpathlineto{\pgfqpoint{3.244197in}{1.712196in}}%
\pgfpathlineto{\pgfqpoint{3.275114in}{1.711536in}}%
\pgfpathlineto{\pgfqpoint{3.277925in}{1.712302in}}%
\pgfpathlineto{\pgfqpoint{3.286357in}{1.712042in}}%
\pgfpathlineto{\pgfqpoint{3.291979in}{1.712240in}}%
\pgfpathlineto{\pgfqpoint{3.303221in}{1.712093in}}%
\pgfpathlineto{\pgfqpoint{3.322896in}{1.711842in}}%
\pgfpathlineto{\pgfqpoint{3.398785in}{1.711078in}}%
\pgfpathlineto{\pgfqpoint{3.407217in}{1.711420in}}%
\pgfpathlineto{\pgfqpoint{3.443756in}{1.712508in}}%
\pgfpathlineto{\pgfqpoint{3.466241in}{1.711934in}}%
\pgfpathlineto{\pgfqpoint{3.480295in}{1.711830in}}%
\pgfpathlineto{\pgfqpoint{3.488727in}{1.710075in}}%
\pgfpathlineto{\pgfqpoint{3.494348in}{1.711361in}}%
\pgfpathlineto{\pgfqpoint{3.499969in}{1.711058in}}%
\pgfpathlineto{\pgfqpoint{3.502780in}{1.710472in}}%
\pgfpathlineto{\pgfqpoint{3.514023in}{1.710774in}}%
\pgfpathlineto{\pgfqpoint{3.519644in}{1.710662in}}%
\pgfpathlineto{\pgfqpoint{3.533698in}{1.710733in}}%
\pgfpathlineto{\pgfqpoint{3.539319in}{1.710652in}}%
\pgfpathlineto{\pgfqpoint{3.556183in}{1.709997in}}%
\pgfpathlineto{\pgfqpoint{3.573047in}{1.710975in}}%
\pgfpathlineto{\pgfqpoint{3.595533in}{1.711637in}}%
\pgfpathlineto{\pgfqpoint{3.603965in}{1.711453in}}%
\pgfpathlineto{\pgfqpoint{3.618018in}{1.711790in}}%
\pgfpathlineto{\pgfqpoint{3.626450in}{1.711603in}}%
\pgfpathlineto{\pgfqpoint{3.646125in}{1.711574in}}%
\pgfpathlineto{\pgfqpoint{3.654557in}{1.711621in}}%
\pgfpathlineto{\pgfqpoint{3.660179in}{1.711845in}}%
\pgfpathlineto{\pgfqpoint{3.699528in}{1.711162in}}%
\pgfpathlineto{\pgfqpoint{3.702339in}{1.710734in}}%
\pgfpathlineto{\pgfqpoint{3.713582in}{1.711179in}}%
\pgfpathlineto{\pgfqpoint{3.716392in}{1.710682in}}%
\pgfpathlineto{\pgfqpoint{3.733256in}{1.711149in}}%
\pgfpathlineto{\pgfqpoint{3.755742in}{1.709931in}}%
\pgfpathlineto{\pgfqpoint{3.764174in}{1.709853in}}%
\pgfpathlineto{\pgfqpoint{3.769795in}{1.709308in}}%
\pgfpathlineto{\pgfqpoint{3.775417in}{1.709310in}}%
\pgfpathlineto{\pgfqpoint{3.781038in}{1.709315in}}%
\pgfpathlineto{\pgfqpoint{3.786659in}{1.709103in}}%
\pgfpathlineto{\pgfqpoint{3.789470in}{1.708742in}}%
\pgfpathlineto{\pgfqpoint{3.792281in}{1.709820in}}%
\pgfpathlineto{\pgfqpoint{3.806334in}{1.709121in}}%
\pgfpathlineto{\pgfqpoint{3.809145in}{1.708285in}}%
\pgfpathlineto{\pgfqpoint{3.845684in}{1.709198in}}%
\pgfpathlineto{\pgfqpoint{3.856927in}{1.709374in}}%
\pgfpathlineto{\pgfqpoint{3.862548in}{1.708937in}}%
\pgfpathlineto{\pgfqpoint{3.870980in}{1.708530in}}%
\pgfpathlineto{\pgfqpoint{3.879412in}{1.708835in}}%
\pgfpathlineto{\pgfqpoint{3.944058in}{1.710281in}}%
\pgfpathlineto{\pgfqpoint{3.969354in}{1.709368in}}%
\pgfpathlineto{\pgfqpoint{3.974975in}{1.709326in}}%
\pgfpathlineto{\pgfqpoint{4.014325in}{1.709135in}}%
\pgfpathlineto{\pgfqpoint{4.019946in}{1.709438in}}%
\pgfpathlineto{\pgfqpoint{4.078971in}{1.708392in}}%
\pgfpathlineto{\pgfqpoint{4.081782in}{1.708059in}}%
\pgfpathlineto{\pgfqpoint{4.087403in}{1.708656in}}%
\pgfpathlineto{\pgfqpoint{4.093024in}{1.708214in}}%
\pgfpathlineto{\pgfqpoint{4.233559in}{1.708806in}}%
\pgfpathlineto{\pgfqpoint{4.247612in}{1.708828in}}%
\pgfpathlineto{\pgfqpoint{4.256044in}{1.708796in}}%
\pgfpathlineto{\pgfqpoint{4.267287in}{1.708642in}}%
\pgfpathlineto{\pgfqpoint{4.281340in}{1.708799in}}%
\pgfpathlineto{\pgfqpoint{4.303826in}{1.708408in}}%
\pgfpathlineto{\pgfqpoint{4.315069in}{1.708603in}}%
\pgfpathlineto{\pgfqpoint{4.323501in}{1.708368in}}%
\pgfpathlineto{\pgfqpoint{4.354418in}{1.708215in}}%
\pgfpathlineto{\pgfqpoint{4.357229in}{1.707645in}}%
\pgfpathlineto{\pgfqpoint{4.388146in}{1.707360in}}%
\pgfpathlineto{\pgfqpoint{4.393768in}{1.707072in}}%
\pgfpathlineto{\pgfqpoint{4.405011in}{1.707600in}}%
\pgfpathlineto{\pgfqpoint{4.452792in}{1.707526in}}%
\pgfpathlineto{\pgfqpoint{4.466846in}{1.707716in}}%
\pgfpathlineto{\pgfqpoint{4.486520in}{1.707682in}}%
\pgfpathlineto{\pgfqpoint{4.497763in}{1.707871in}}%
\pgfpathlineto{\pgfqpoint{4.528681in}{1.708116in}}%
\pgfpathlineto{\pgfqpoint{4.570841in}{1.708312in}}%
\pgfpathlineto{\pgfqpoint{4.655162in}{1.708036in}}%
\pgfpathlineto{\pgfqpoint{4.666404in}{1.708291in}}%
\pgfpathlineto{\pgfqpoint{4.702943in}{1.708212in}}%
\pgfpathlineto{\pgfqpoint{4.745104in}{1.708516in}}%
\pgfpathlineto{\pgfqpoint{4.770400in}{1.708124in}}%
\pgfpathlineto{\pgfqpoint{4.792885in}{1.708275in}}%
\pgfpathlineto{\pgfqpoint{4.801317in}{1.707897in}}%
\pgfpathlineto{\pgfqpoint{4.846288in}{1.708577in}}%
\pgfpathlineto{\pgfqpoint{4.854720in}{1.708399in}}%
\pgfpathlineto{\pgfqpoint{4.868774in}{1.708553in}}%
\pgfpathlineto{\pgfqpoint{4.908123in}{1.708608in}}%
\pgfpathlineto{\pgfqpoint{4.953094in}{1.708637in}}%
\pgfpathlineto{\pgfqpoint{4.984012in}{1.708502in}}%
\pgfpathlineto{\pgfqpoint{5.006497in}{1.708711in}}%
\pgfpathlineto{\pgfqpoint{5.023362in}{1.708654in}}%
\pgfpathlineto{\pgfqpoint{5.043036in}{1.708823in}}%
\pgfpathlineto{\pgfqpoint{5.090818in}{1.708872in}}%
\pgfpathlineto{\pgfqpoint{5.093629in}{1.708532in}}%
\pgfpathlineto{\pgfqpoint{5.099250in}{1.708592in}}%
\pgfpathlineto{\pgfqpoint{5.104871in}{1.708364in}}%
\pgfpathlineto{\pgfqpoint{5.130168in}{1.708733in}}%
\pgfpathlineto{\pgfqpoint{5.141410in}{1.708609in}}%
\pgfpathlineto{\pgfqpoint{5.149842in}{1.708728in}}%
\pgfpathlineto{\pgfqpoint{5.149842in}{1.708728in}}%
\pgfusepath{stroke}%
\end{pgfscope}%
\begin{pgfscope}%
\pgfsetrectcap%
\pgfsetmiterjoin%
\pgfsetlinewidth{0.803000pt}%
\definecolor{currentstroke}{rgb}{1.000000,1.000000,1.000000}%
\pgfsetstrokecolor{currentstroke}%
\pgfsetdash{}{0pt}%
\pgfpathmoveto{\pgfqpoint{0.711206in}{0.331635in}}%
\pgfpathlineto{\pgfqpoint{0.711206in}{3.351635in}}%
\pgfusepath{stroke}%
\end{pgfscope}%
\begin{pgfscope}%
\pgfsetrectcap%
\pgfsetmiterjoin%
\pgfsetlinewidth{0.803000pt}%
\definecolor{currentstroke}{rgb}{1.000000,1.000000,1.000000}%
\pgfsetstrokecolor{currentstroke}%
\pgfsetdash{}{0pt}%
\pgfpathmoveto{\pgfqpoint{5.361206in}{0.331635in}}%
\pgfpathlineto{\pgfqpoint{5.361206in}{3.351635in}}%
\pgfusepath{stroke}%
\end{pgfscope}%
\begin{pgfscope}%
\pgfsetrectcap%
\pgfsetmiterjoin%
\pgfsetlinewidth{0.803000pt}%
\definecolor{currentstroke}{rgb}{1.000000,1.000000,1.000000}%
\pgfsetstrokecolor{currentstroke}%
\pgfsetdash{}{0pt}%
\pgfpathmoveto{\pgfqpoint{0.711206in}{0.331635in}}%
\pgfpathlineto{\pgfqpoint{5.361206in}{0.331635in}}%
\pgfusepath{stroke}%
\end{pgfscope}%
\begin{pgfscope}%
\pgfsetrectcap%
\pgfsetmiterjoin%
\pgfsetlinewidth{0.803000pt}%
\definecolor{currentstroke}{rgb}{1.000000,1.000000,1.000000}%
\pgfsetstrokecolor{currentstroke}%
\pgfsetdash{}{0pt}%
\pgfpathmoveto{\pgfqpoint{0.711206in}{3.351635in}}%
\pgfpathlineto{\pgfqpoint{5.361206in}{3.351635in}}%
\pgfusepath{stroke}%
\end{pgfscope}%
\end{pgfpicture}%
\makeatother%
\endgroup%

    %% Creator: Matplotlib, PGF backend
%%
%% To include the figure in your LaTeX document, write
%%   \input{<filename>.pgf}
%%
%% Make sure the required packages are loaded in your preamble
%%   \usepackage{pgf}
%%
%% Figures using additional raster images can only be included by \input if
%% they are in the same directory as the main LaTeX file. For loading figures
%% from other directories you can use the `import` package
%%   \usepackage{import}
%% and then include the figures with
%%   \import{<path to file>}{<filename>.pgf}
%%
%% Matplotlib used the following preamble
%%   \usepackage{fontspec}
%%   \setmainfont{DejaVuSerif.ttf}[Path=/opt/tljh/user/lib/python3.6/site-packages/matplotlib/mpl-data/fonts/ttf/]
%%   \setsansfont{DejaVuSans.ttf}[Path=/opt/tljh/user/lib/python3.6/site-packages/matplotlib/mpl-data/fonts/ttf/]
%%   \setmonofont{DejaVuSansMono.ttf}[Path=/opt/tljh/user/lib/python3.6/site-packages/matplotlib/mpl-data/fonts/ttf/]
%%
\begingroup%
\makeatletter%
\begin{pgfpicture}%
\pgfpathrectangle{\pgfpointorigin}{\pgfqpoint{5.461206in}{3.451635in}}%
\pgfusepath{use as bounding box, clip}%
\begin{pgfscope}%
\pgfsetbuttcap%
\pgfsetmiterjoin%
\definecolor{currentfill}{rgb}{1.000000,1.000000,1.000000}%
\pgfsetfillcolor{currentfill}%
\pgfsetlinewidth{0.000000pt}%
\definecolor{currentstroke}{rgb}{1.000000,1.000000,1.000000}%
\pgfsetstrokecolor{currentstroke}%
\pgfsetdash{}{0pt}%
\pgfpathmoveto{\pgfqpoint{0.000000in}{0.000000in}}%
\pgfpathlineto{\pgfqpoint{5.461206in}{0.000000in}}%
\pgfpathlineto{\pgfqpoint{5.461206in}{3.451635in}}%
\pgfpathlineto{\pgfqpoint{0.000000in}{3.451635in}}%
\pgfpathclose%
\pgfusepath{fill}%
\end{pgfscope}%
\begin{pgfscope}%
\pgfsetbuttcap%
\pgfsetmiterjoin%
\definecolor{currentfill}{rgb}{0.917647,0.917647,0.949020}%
\pgfsetfillcolor{currentfill}%
\pgfsetlinewidth{0.000000pt}%
\definecolor{currentstroke}{rgb}{0.000000,0.000000,0.000000}%
\pgfsetstrokecolor{currentstroke}%
\pgfsetstrokeopacity{0.000000}%
\pgfsetdash{}{0pt}%
\pgfpathmoveto{\pgfqpoint{0.711206in}{0.331635in}}%
\pgfpathlineto{\pgfqpoint{5.361206in}{0.331635in}}%
\pgfpathlineto{\pgfqpoint{5.361206in}{3.351635in}}%
\pgfpathlineto{\pgfqpoint{0.711206in}{3.351635in}}%
\pgfpathclose%
\pgfusepath{fill}%
\end{pgfscope}%
\begin{pgfscope}%
\pgfpathrectangle{\pgfqpoint{0.711206in}{0.331635in}}{\pgfqpoint{4.650000in}{3.020000in}}%
\pgfusepath{clip}%
\pgfsetroundcap%
\pgfsetroundjoin%
\pgfsetlinewidth{0.803000pt}%
\definecolor{currentstroke}{rgb}{1.000000,1.000000,1.000000}%
\pgfsetstrokecolor{currentstroke}%
\pgfsetdash{}{0pt}%
\pgfpathmoveto{\pgfqpoint{0.922570in}{0.331635in}}%
\pgfpathlineto{\pgfqpoint{0.922570in}{3.351635in}}%
\pgfusepath{stroke}%
\end{pgfscope}%
\begin{pgfscope}%
\definecolor{textcolor}{rgb}{0.150000,0.150000,0.150000}%
\pgfsetstrokecolor{textcolor}%
\pgfsetfillcolor{textcolor}%
\pgftext[x=0.922570in,y=0.234413in,,top]{\color{textcolor}\rmfamily\fontsize{10.000000}{12.000000}\selectfont 0}%
\end{pgfscope}%
\begin{pgfscope}%
\pgfpathrectangle{\pgfqpoint{0.711206in}{0.331635in}}{\pgfqpoint{4.650000in}{3.020000in}}%
\pgfusepath{clip}%
\pgfsetroundcap%
\pgfsetroundjoin%
\pgfsetlinewidth{0.803000pt}%
\definecolor{currentstroke}{rgb}{1.000000,1.000000,1.000000}%
\pgfsetstrokecolor{currentstroke}%
\pgfsetdash{}{0pt}%
\pgfpathmoveto{\pgfqpoint{1.484334in}{0.331635in}}%
\pgfpathlineto{\pgfqpoint{1.484334in}{3.351635in}}%
\pgfusepath{stroke}%
\end{pgfscope}%
\begin{pgfscope}%
\definecolor{textcolor}{rgb}{0.150000,0.150000,0.150000}%
\pgfsetstrokecolor{textcolor}%
\pgfsetfillcolor{textcolor}%
\pgftext[x=1.484334in,y=0.234413in,,top]{\color{textcolor}\rmfamily\fontsize{10.000000}{12.000000}\selectfont 200}%
\end{pgfscope}%
\begin{pgfscope}%
\pgfpathrectangle{\pgfqpoint{0.711206in}{0.331635in}}{\pgfqpoint{4.650000in}{3.020000in}}%
\pgfusepath{clip}%
\pgfsetroundcap%
\pgfsetroundjoin%
\pgfsetlinewidth{0.803000pt}%
\definecolor{currentstroke}{rgb}{1.000000,1.000000,1.000000}%
\pgfsetstrokecolor{currentstroke}%
\pgfsetdash{}{0pt}%
\pgfpathmoveto{\pgfqpoint{2.046097in}{0.331635in}}%
\pgfpathlineto{\pgfqpoint{2.046097in}{3.351635in}}%
\pgfusepath{stroke}%
\end{pgfscope}%
\begin{pgfscope}%
\definecolor{textcolor}{rgb}{0.150000,0.150000,0.150000}%
\pgfsetstrokecolor{textcolor}%
\pgfsetfillcolor{textcolor}%
\pgftext[x=2.046097in,y=0.234413in,,top]{\color{textcolor}\rmfamily\fontsize{10.000000}{12.000000}\selectfont 400}%
\end{pgfscope}%
\begin{pgfscope}%
\pgfpathrectangle{\pgfqpoint{0.711206in}{0.331635in}}{\pgfqpoint{4.650000in}{3.020000in}}%
\pgfusepath{clip}%
\pgfsetroundcap%
\pgfsetroundjoin%
\pgfsetlinewidth{0.803000pt}%
\definecolor{currentstroke}{rgb}{1.000000,1.000000,1.000000}%
\pgfsetstrokecolor{currentstroke}%
\pgfsetdash{}{0pt}%
\pgfpathmoveto{\pgfqpoint{2.607861in}{0.331635in}}%
\pgfpathlineto{\pgfqpoint{2.607861in}{3.351635in}}%
\pgfusepath{stroke}%
\end{pgfscope}%
\begin{pgfscope}%
\definecolor{textcolor}{rgb}{0.150000,0.150000,0.150000}%
\pgfsetstrokecolor{textcolor}%
\pgfsetfillcolor{textcolor}%
\pgftext[x=2.607861in,y=0.234413in,,top]{\color{textcolor}\rmfamily\fontsize{10.000000}{12.000000}\selectfont 600}%
\end{pgfscope}%
\begin{pgfscope}%
\pgfpathrectangle{\pgfqpoint{0.711206in}{0.331635in}}{\pgfqpoint{4.650000in}{3.020000in}}%
\pgfusepath{clip}%
\pgfsetroundcap%
\pgfsetroundjoin%
\pgfsetlinewidth{0.803000pt}%
\definecolor{currentstroke}{rgb}{1.000000,1.000000,1.000000}%
\pgfsetstrokecolor{currentstroke}%
\pgfsetdash{}{0pt}%
\pgfpathmoveto{\pgfqpoint{3.169625in}{0.331635in}}%
\pgfpathlineto{\pgfqpoint{3.169625in}{3.351635in}}%
\pgfusepath{stroke}%
\end{pgfscope}%
\begin{pgfscope}%
\definecolor{textcolor}{rgb}{0.150000,0.150000,0.150000}%
\pgfsetstrokecolor{textcolor}%
\pgfsetfillcolor{textcolor}%
\pgftext[x=3.169625in,y=0.234413in,,top]{\color{textcolor}\rmfamily\fontsize{10.000000}{12.000000}\selectfont 800}%
\end{pgfscope}%
\begin{pgfscope}%
\pgfpathrectangle{\pgfqpoint{0.711206in}{0.331635in}}{\pgfqpoint{4.650000in}{3.020000in}}%
\pgfusepath{clip}%
\pgfsetroundcap%
\pgfsetroundjoin%
\pgfsetlinewidth{0.803000pt}%
\definecolor{currentstroke}{rgb}{1.000000,1.000000,1.000000}%
\pgfsetstrokecolor{currentstroke}%
\pgfsetdash{}{0pt}%
\pgfpathmoveto{\pgfqpoint{3.731389in}{0.331635in}}%
\pgfpathlineto{\pgfqpoint{3.731389in}{3.351635in}}%
\pgfusepath{stroke}%
\end{pgfscope}%
\begin{pgfscope}%
\definecolor{textcolor}{rgb}{0.150000,0.150000,0.150000}%
\pgfsetstrokecolor{textcolor}%
\pgfsetfillcolor{textcolor}%
\pgftext[x=3.731389in,y=0.234413in,,top]{\color{textcolor}\rmfamily\fontsize{10.000000}{12.000000}\selectfont 1000}%
\end{pgfscope}%
\begin{pgfscope}%
\pgfpathrectangle{\pgfqpoint{0.711206in}{0.331635in}}{\pgfqpoint{4.650000in}{3.020000in}}%
\pgfusepath{clip}%
\pgfsetroundcap%
\pgfsetroundjoin%
\pgfsetlinewidth{0.803000pt}%
\definecolor{currentstroke}{rgb}{1.000000,1.000000,1.000000}%
\pgfsetstrokecolor{currentstroke}%
\pgfsetdash{}{0pt}%
\pgfpathmoveto{\pgfqpoint{4.293153in}{0.331635in}}%
\pgfpathlineto{\pgfqpoint{4.293153in}{3.351635in}}%
\pgfusepath{stroke}%
\end{pgfscope}%
\begin{pgfscope}%
\definecolor{textcolor}{rgb}{0.150000,0.150000,0.150000}%
\pgfsetstrokecolor{textcolor}%
\pgfsetfillcolor{textcolor}%
\pgftext[x=4.293153in,y=0.234413in,,top]{\color{textcolor}\rmfamily\fontsize{10.000000}{12.000000}\selectfont 1200}%
\end{pgfscope}%
\begin{pgfscope}%
\pgfpathrectangle{\pgfqpoint{0.711206in}{0.331635in}}{\pgfqpoint{4.650000in}{3.020000in}}%
\pgfusepath{clip}%
\pgfsetroundcap%
\pgfsetroundjoin%
\pgfsetlinewidth{0.803000pt}%
\definecolor{currentstroke}{rgb}{1.000000,1.000000,1.000000}%
\pgfsetstrokecolor{currentstroke}%
\pgfsetdash{}{0pt}%
\pgfpathmoveto{\pgfqpoint{4.854916in}{0.331635in}}%
\pgfpathlineto{\pgfqpoint{4.854916in}{3.351635in}}%
\pgfusepath{stroke}%
\end{pgfscope}%
\begin{pgfscope}%
\definecolor{textcolor}{rgb}{0.150000,0.150000,0.150000}%
\pgfsetstrokecolor{textcolor}%
\pgfsetfillcolor{textcolor}%
\pgftext[x=4.854916in,y=0.234413in,,top]{\color{textcolor}\rmfamily\fontsize{10.000000}{12.000000}\selectfont 1400}%
\end{pgfscope}%
\begin{pgfscope}%
\pgfpathrectangle{\pgfqpoint{0.711206in}{0.331635in}}{\pgfqpoint{4.650000in}{3.020000in}}%
\pgfusepath{clip}%
\pgfsetroundcap%
\pgfsetroundjoin%
\pgfsetlinewidth{0.803000pt}%
\definecolor{currentstroke}{rgb}{1.000000,1.000000,1.000000}%
\pgfsetstrokecolor{currentstroke}%
\pgfsetdash{}{0pt}%
\pgfpathmoveto{\pgfqpoint{0.711206in}{0.521563in}}%
\pgfpathlineto{\pgfqpoint{5.361206in}{0.521563in}}%
\pgfusepath{stroke}%
\end{pgfscope}%
\begin{pgfscope}%
\definecolor{textcolor}{rgb}{0.150000,0.150000,0.150000}%
\pgfsetstrokecolor{textcolor}%
\pgfsetfillcolor{textcolor}%
\pgftext[x=0.100000in,y=0.468802in,left,base]{\color{textcolor}\rmfamily\fontsize{10.000000}{12.000000}\selectfont −0.075}%
\end{pgfscope}%
\begin{pgfscope}%
\pgfpathrectangle{\pgfqpoint{0.711206in}{0.331635in}}{\pgfqpoint{4.650000in}{3.020000in}}%
\pgfusepath{clip}%
\pgfsetroundcap%
\pgfsetroundjoin%
\pgfsetlinewidth{0.803000pt}%
\definecolor{currentstroke}{rgb}{1.000000,1.000000,1.000000}%
\pgfsetstrokecolor{currentstroke}%
\pgfsetdash{}{0pt}%
\pgfpathmoveto{\pgfqpoint{0.711206in}{0.911716in}}%
\pgfpathlineto{\pgfqpoint{5.361206in}{0.911716in}}%
\pgfusepath{stroke}%
\end{pgfscope}%
\begin{pgfscope}%
\definecolor{textcolor}{rgb}{0.150000,0.150000,0.150000}%
\pgfsetstrokecolor{textcolor}%
\pgfsetfillcolor{textcolor}%
\pgftext[x=0.100000in,y=0.858955in,left,base]{\color{textcolor}\rmfamily\fontsize{10.000000}{12.000000}\selectfont −0.050}%
\end{pgfscope}%
\begin{pgfscope}%
\pgfpathrectangle{\pgfqpoint{0.711206in}{0.331635in}}{\pgfqpoint{4.650000in}{3.020000in}}%
\pgfusepath{clip}%
\pgfsetroundcap%
\pgfsetroundjoin%
\pgfsetlinewidth{0.803000pt}%
\definecolor{currentstroke}{rgb}{1.000000,1.000000,1.000000}%
\pgfsetstrokecolor{currentstroke}%
\pgfsetdash{}{0pt}%
\pgfpathmoveto{\pgfqpoint{0.711206in}{1.301870in}}%
\pgfpathlineto{\pgfqpoint{5.361206in}{1.301870in}}%
\pgfusepath{stroke}%
\end{pgfscope}%
\begin{pgfscope}%
\definecolor{textcolor}{rgb}{0.150000,0.150000,0.150000}%
\pgfsetstrokecolor{textcolor}%
\pgfsetfillcolor{textcolor}%
\pgftext[x=0.100000in,y=1.249108in,left,base]{\color{textcolor}\rmfamily\fontsize{10.000000}{12.000000}\selectfont −0.025}%
\end{pgfscope}%
\begin{pgfscope}%
\pgfpathrectangle{\pgfqpoint{0.711206in}{0.331635in}}{\pgfqpoint{4.650000in}{3.020000in}}%
\pgfusepath{clip}%
\pgfsetroundcap%
\pgfsetroundjoin%
\pgfsetlinewidth{0.803000pt}%
\definecolor{currentstroke}{rgb}{1.000000,1.000000,1.000000}%
\pgfsetstrokecolor{currentstroke}%
\pgfsetdash{}{0pt}%
\pgfpathmoveto{\pgfqpoint{0.711206in}{1.692023in}}%
\pgfpathlineto{\pgfqpoint{5.361206in}{1.692023in}}%
\pgfusepath{stroke}%
\end{pgfscope}%
\begin{pgfscope}%
\definecolor{textcolor}{rgb}{0.150000,0.150000,0.150000}%
\pgfsetstrokecolor{textcolor}%
\pgfsetfillcolor{textcolor}%
\pgftext[x=0.216374in,y=1.639261in,left,base]{\color{textcolor}\rmfamily\fontsize{10.000000}{12.000000}\selectfont 0.000}%
\end{pgfscope}%
\begin{pgfscope}%
\pgfpathrectangle{\pgfqpoint{0.711206in}{0.331635in}}{\pgfqpoint{4.650000in}{3.020000in}}%
\pgfusepath{clip}%
\pgfsetroundcap%
\pgfsetroundjoin%
\pgfsetlinewidth{0.803000pt}%
\definecolor{currentstroke}{rgb}{1.000000,1.000000,1.000000}%
\pgfsetstrokecolor{currentstroke}%
\pgfsetdash{}{0pt}%
\pgfpathmoveto{\pgfqpoint{0.711206in}{2.082176in}}%
\pgfpathlineto{\pgfqpoint{5.361206in}{2.082176in}}%
\pgfusepath{stroke}%
\end{pgfscope}%
\begin{pgfscope}%
\definecolor{textcolor}{rgb}{0.150000,0.150000,0.150000}%
\pgfsetstrokecolor{textcolor}%
\pgfsetfillcolor{textcolor}%
\pgftext[x=0.216374in,y=2.029415in,left,base]{\color{textcolor}\rmfamily\fontsize{10.000000}{12.000000}\selectfont 0.025}%
\end{pgfscope}%
\begin{pgfscope}%
\pgfpathrectangle{\pgfqpoint{0.711206in}{0.331635in}}{\pgfqpoint{4.650000in}{3.020000in}}%
\pgfusepath{clip}%
\pgfsetroundcap%
\pgfsetroundjoin%
\pgfsetlinewidth{0.803000pt}%
\definecolor{currentstroke}{rgb}{1.000000,1.000000,1.000000}%
\pgfsetstrokecolor{currentstroke}%
\pgfsetdash{}{0pt}%
\pgfpathmoveto{\pgfqpoint{0.711206in}{2.472330in}}%
\pgfpathlineto{\pgfqpoint{5.361206in}{2.472330in}}%
\pgfusepath{stroke}%
\end{pgfscope}%
\begin{pgfscope}%
\definecolor{textcolor}{rgb}{0.150000,0.150000,0.150000}%
\pgfsetstrokecolor{textcolor}%
\pgfsetfillcolor{textcolor}%
\pgftext[x=0.216374in,y=2.419568in,left,base]{\color{textcolor}\rmfamily\fontsize{10.000000}{12.000000}\selectfont 0.050}%
\end{pgfscope}%
\begin{pgfscope}%
\pgfpathrectangle{\pgfqpoint{0.711206in}{0.331635in}}{\pgfqpoint{4.650000in}{3.020000in}}%
\pgfusepath{clip}%
\pgfsetroundcap%
\pgfsetroundjoin%
\pgfsetlinewidth{0.803000pt}%
\definecolor{currentstroke}{rgb}{1.000000,1.000000,1.000000}%
\pgfsetstrokecolor{currentstroke}%
\pgfsetdash{}{0pt}%
\pgfpathmoveto{\pgfqpoint{0.711206in}{2.862483in}}%
\pgfpathlineto{\pgfqpoint{5.361206in}{2.862483in}}%
\pgfusepath{stroke}%
\end{pgfscope}%
\begin{pgfscope}%
\definecolor{textcolor}{rgb}{0.150000,0.150000,0.150000}%
\pgfsetstrokecolor{textcolor}%
\pgfsetfillcolor{textcolor}%
\pgftext[x=0.216374in,y=2.809721in,left,base]{\color{textcolor}\rmfamily\fontsize{10.000000}{12.000000}\selectfont 0.075}%
\end{pgfscope}%
\begin{pgfscope}%
\pgfpathrectangle{\pgfqpoint{0.711206in}{0.331635in}}{\pgfqpoint{4.650000in}{3.020000in}}%
\pgfusepath{clip}%
\pgfsetroundcap%
\pgfsetroundjoin%
\pgfsetlinewidth{0.803000pt}%
\definecolor{currentstroke}{rgb}{1.000000,1.000000,1.000000}%
\pgfsetstrokecolor{currentstroke}%
\pgfsetdash{}{0pt}%
\pgfpathmoveto{\pgfqpoint{0.711206in}{3.252636in}}%
\pgfpathlineto{\pgfqpoint{5.361206in}{3.252636in}}%
\pgfusepath{stroke}%
\end{pgfscope}%
\begin{pgfscope}%
\definecolor{textcolor}{rgb}{0.150000,0.150000,0.150000}%
\pgfsetstrokecolor{textcolor}%
\pgfsetfillcolor{textcolor}%
\pgftext[x=0.216374in,y=3.199875in,left,base]{\color{textcolor}\rmfamily\fontsize{10.000000}{12.000000}\selectfont 0.100}%
\end{pgfscope}%
\begin{pgfscope}%
\pgfpathrectangle{\pgfqpoint{0.711206in}{0.331635in}}{\pgfqpoint{4.650000in}{3.020000in}}%
\pgfusepath{clip}%
\pgfsetbuttcap%
\pgfsetroundjoin%
\definecolor{currentfill}{rgb}{1.000000,0.000000,0.000000}%
\pgfsetfillcolor{currentfill}%
\pgfsetfillopacity{0.400000}%
\pgfsetlinewidth{1.003750pt}%
\definecolor{currentstroke}{rgb}{1.000000,0.000000,0.000000}%
\pgfsetstrokecolor{currentstroke}%
\pgfsetstrokeopacity{0.400000}%
\pgfsetdash{}{0pt}%
\pgfpathmoveto{\pgfqpoint{0.922570in}{1.910758in}}%
\pgfpathlineto{\pgfqpoint{0.922570in}{1.244515in}}%
\pgfpathlineto{\pgfqpoint{0.925380in}{1.569253in}}%
\pgfpathlineto{\pgfqpoint{0.928191in}{1.324249in}}%
\pgfpathlineto{\pgfqpoint{0.931002in}{1.512089in}}%
\pgfpathlineto{\pgfqpoint{0.931002in}{1.696085in}}%
\pgfpathlineto{\pgfqpoint{0.931002in}{1.696085in}}%
\pgfpathlineto{\pgfqpoint{0.928191in}{1.865627in}}%
\pgfpathlineto{\pgfqpoint{0.925380in}{1.794838in}}%
\pgfpathlineto{\pgfqpoint{0.922570in}{1.910758in}}%
\pgfpathclose%
\pgfusepath{stroke,fill}%
\end{pgfscope}%
\begin{pgfscope}%
\pgfpathrectangle{\pgfqpoint{0.711206in}{0.331635in}}{\pgfqpoint{4.650000in}{3.020000in}}%
\pgfusepath{clip}%
\pgfsetbuttcap%
\pgfsetroundjoin%
\definecolor{currentfill}{rgb}{1.000000,0.000000,0.000000}%
\pgfsetfillcolor{currentfill}%
\pgfsetfillopacity{0.400000}%
\pgfsetlinewidth{1.003750pt}%
\definecolor{currentstroke}{rgb}{1.000000,0.000000,0.000000}%
\pgfsetstrokecolor{currentstroke}%
\pgfsetstrokeopacity{0.400000}%
\pgfsetdash{}{0pt}%
\pgfpathmoveto{\pgfqpoint{0.936623in}{2.034771in}}%
\pgfpathlineto{\pgfqpoint{0.936623in}{1.392247in}}%
\pgfpathlineto{\pgfqpoint{0.939434in}{1.295830in}}%
\pgfpathlineto{\pgfqpoint{0.942245in}{1.316917in}}%
\pgfpathlineto{\pgfqpoint{0.945055in}{1.397788in}}%
\pgfpathlineto{\pgfqpoint{0.947866in}{1.260910in}}%
\pgfpathlineto{\pgfqpoint{0.950677in}{1.266194in}}%
\pgfpathlineto{\pgfqpoint{0.953487in}{1.271797in}}%
\pgfpathlineto{\pgfqpoint{0.956298in}{1.281278in}}%
\pgfpathlineto{\pgfqpoint{0.959109in}{1.296520in}}%
\pgfpathlineto{\pgfqpoint{0.961919in}{1.310722in}}%
\pgfpathlineto{\pgfqpoint{0.964730in}{1.398196in}}%
\pgfpathlineto{\pgfqpoint{0.967541in}{1.355469in}}%
\pgfpathlineto{\pgfqpoint{0.970351in}{1.245360in}}%
\pgfpathlineto{\pgfqpoint{0.973162in}{1.437997in}}%
\pgfpathlineto{\pgfqpoint{0.975973in}{1.260319in}}%
\pgfpathlineto{\pgfqpoint{0.978783in}{1.317312in}}%
\pgfpathlineto{\pgfqpoint{0.981594in}{1.303178in}}%
\pgfpathlineto{\pgfqpoint{0.984405in}{1.348659in}}%
\pgfpathlineto{\pgfqpoint{0.987216in}{1.363450in}}%
\pgfpathlineto{\pgfqpoint{0.990026in}{1.384558in}}%
\pgfpathlineto{\pgfqpoint{0.992837in}{1.254146in}}%
\pgfpathlineto{\pgfqpoint{0.995648in}{1.440212in}}%
\pgfpathlineto{\pgfqpoint{0.998458in}{1.248620in}}%
\pgfpathlineto{\pgfqpoint{1.001269in}{1.365143in}}%
\pgfpathlineto{\pgfqpoint{1.004080in}{1.328161in}}%
\pgfpathlineto{\pgfqpoint{1.006890in}{1.338077in}}%
\pgfpathlineto{\pgfqpoint{1.009701in}{1.414676in}}%
\pgfpathlineto{\pgfqpoint{1.012512in}{1.287030in}}%
\pgfpathlineto{\pgfqpoint{1.015322in}{1.439467in}}%
\pgfpathlineto{\pgfqpoint{1.018133in}{1.255099in}}%
\pgfpathlineto{\pgfqpoint{1.020944in}{1.345777in}}%
\pgfpathlineto{\pgfqpoint{1.023754in}{1.313851in}}%
\pgfpathlineto{\pgfqpoint{1.026565in}{1.320331in}}%
\pgfpathlineto{\pgfqpoint{1.029376in}{1.316805in}}%
\pgfpathlineto{\pgfqpoint{1.032187in}{1.320869in}}%
\pgfpathlineto{\pgfqpoint{1.034997in}{1.314325in}}%
\pgfpathlineto{\pgfqpoint{1.037808in}{1.320198in}}%
\pgfpathlineto{\pgfqpoint{1.040619in}{1.323660in}}%
\pgfpathlineto{\pgfqpoint{1.043429in}{1.322328in}}%
\pgfpathlineto{\pgfqpoint{1.046240in}{1.322317in}}%
\pgfpathlineto{\pgfqpoint{1.049051in}{1.327544in}}%
\pgfpathlineto{\pgfqpoint{1.051861in}{1.328042in}}%
\pgfpathlineto{\pgfqpoint{1.054672in}{1.332001in}}%
\pgfpathlineto{\pgfqpoint{1.057483in}{1.333640in}}%
\pgfpathlineto{\pgfqpoint{1.060293in}{1.336820in}}%
\pgfpathlineto{\pgfqpoint{1.063104in}{1.375478in}}%
\pgfpathlineto{\pgfqpoint{1.065915in}{1.300501in}}%
\pgfpathlineto{\pgfqpoint{1.068725in}{1.347451in}}%
\pgfpathlineto{\pgfqpoint{1.071536in}{1.299942in}}%
\pgfpathlineto{\pgfqpoint{1.074347in}{1.347505in}}%
\pgfpathlineto{\pgfqpoint{1.077158in}{1.347206in}}%
\pgfpathlineto{\pgfqpoint{1.079968in}{1.351089in}}%
\pgfpathlineto{\pgfqpoint{1.082779in}{1.346761in}}%
\pgfpathlineto{\pgfqpoint{1.085590in}{1.341224in}}%
\pgfpathlineto{\pgfqpoint{1.088400in}{1.321788in}}%
\pgfpathlineto{\pgfqpoint{1.091211in}{1.363466in}}%
\pgfpathlineto{\pgfqpoint{1.094022in}{1.362305in}}%
\pgfpathlineto{\pgfqpoint{1.096832in}{1.318118in}}%
\pgfpathlineto{\pgfqpoint{1.099643in}{1.391390in}}%
\pgfpathlineto{\pgfqpoint{1.102454in}{1.342302in}}%
\pgfpathlineto{\pgfqpoint{1.105264in}{1.321846in}}%
\pgfpathlineto{\pgfqpoint{1.108075in}{1.301068in}}%
\pgfpathlineto{\pgfqpoint{1.110886in}{1.329221in}}%
\pgfpathlineto{\pgfqpoint{1.113696in}{1.319827in}}%
\pgfpathlineto{\pgfqpoint{1.116507in}{1.323400in}}%
\pgfpathlineto{\pgfqpoint{1.119318in}{1.324711in}}%
\pgfpathlineto{\pgfqpoint{1.122128in}{1.325109in}}%
\pgfpathlineto{\pgfqpoint{1.124939in}{1.326790in}}%
\pgfpathlineto{\pgfqpoint{1.127750in}{1.313374in}}%
\pgfpathlineto{\pgfqpoint{1.130561in}{1.331036in}}%
\pgfpathlineto{\pgfqpoint{1.133371in}{1.316748in}}%
\pgfpathlineto{\pgfqpoint{1.136182in}{1.319957in}}%
\pgfpathlineto{\pgfqpoint{1.138993in}{1.322561in}}%
\pgfpathlineto{\pgfqpoint{1.141803in}{1.323121in}}%
\pgfpathlineto{\pgfqpoint{1.144614in}{1.325352in}}%
\pgfpathlineto{\pgfqpoint{1.147425in}{1.324664in}}%
\pgfpathlineto{\pgfqpoint{1.150235in}{1.283982in}}%
\pgfpathlineto{\pgfqpoint{1.153046in}{1.287100in}}%
\pgfpathlineto{\pgfqpoint{1.155857in}{1.290174in}}%
\pgfpathlineto{\pgfqpoint{1.158667in}{1.291500in}}%
\pgfpathlineto{\pgfqpoint{1.161478in}{1.291783in}}%
\pgfpathlineto{\pgfqpoint{1.164289in}{1.294722in}}%
\pgfpathlineto{\pgfqpoint{1.167099in}{1.296307in}}%
\pgfpathlineto{\pgfqpoint{1.169910in}{1.294805in}}%
\pgfpathlineto{\pgfqpoint{1.172721in}{1.296655in}}%
\pgfpathlineto{\pgfqpoint{1.175532in}{1.298915in}}%
\pgfpathlineto{\pgfqpoint{1.178342in}{1.290916in}}%
\pgfpathlineto{\pgfqpoint{1.181153in}{1.283747in}}%
\pgfpathlineto{\pgfqpoint{1.183964in}{1.280525in}}%
\pgfpathlineto{\pgfqpoint{1.186774in}{1.282316in}}%
\pgfpathlineto{\pgfqpoint{1.189585in}{1.285054in}}%
\pgfpathlineto{\pgfqpoint{1.192396in}{1.287674in}}%
\pgfpathlineto{\pgfqpoint{1.195206in}{1.288764in}}%
\pgfpathlineto{\pgfqpoint{1.198017in}{1.291428in}}%
\pgfpathlineto{\pgfqpoint{1.200828in}{1.282959in}}%
\pgfpathlineto{\pgfqpoint{1.203638in}{1.277180in}}%
\pgfpathlineto{\pgfqpoint{1.206449in}{1.267793in}}%
\pgfpathlineto{\pgfqpoint{1.209260in}{1.269741in}}%
\pgfpathlineto{\pgfqpoint{1.212070in}{1.271006in}}%
\pgfpathlineto{\pgfqpoint{1.214881in}{1.271704in}}%
\pgfpathlineto{\pgfqpoint{1.217692in}{1.274055in}}%
\pgfpathlineto{\pgfqpoint{1.220503in}{1.275505in}}%
\pgfpathlineto{\pgfqpoint{1.223313in}{1.276937in}}%
\pgfpathlineto{\pgfqpoint{1.226124in}{1.279387in}}%
\pgfpathlineto{\pgfqpoint{1.228935in}{1.274346in}}%
\pgfpathlineto{\pgfqpoint{1.231745in}{1.276524in}}%
\pgfpathlineto{\pgfqpoint{1.234556in}{1.278397in}}%
\pgfpathlineto{\pgfqpoint{1.237367in}{1.280485in}}%
\pgfpathlineto{\pgfqpoint{1.240177in}{1.282826in}}%
\pgfpathlineto{\pgfqpoint{1.242988in}{1.285154in}}%
\pgfpathlineto{\pgfqpoint{1.245799in}{1.276632in}}%
\pgfpathlineto{\pgfqpoint{1.248609in}{1.267513in}}%
\pgfpathlineto{\pgfqpoint{1.251420in}{1.273515in}}%
\pgfpathlineto{\pgfqpoint{1.254231in}{1.249039in}}%
\pgfpathlineto{\pgfqpoint{1.257041in}{1.255282in}}%
\pgfpathlineto{\pgfqpoint{1.259852in}{1.286394in}}%
\pgfpathlineto{\pgfqpoint{1.262663in}{1.251411in}}%
\pgfpathlineto{\pgfqpoint{1.265474in}{1.216655in}}%
\pgfpathlineto{\pgfqpoint{1.268284in}{1.242604in}}%
\pgfpathlineto{\pgfqpoint{1.271095in}{1.247283in}}%
\pgfpathlineto{\pgfqpoint{1.273906in}{1.281393in}}%
\pgfpathlineto{\pgfqpoint{1.276716in}{1.299354in}}%
\pgfpathlineto{\pgfqpoint{1.279527in}{1.305530in}}%
\pgfpathlineto{\pgfqpoint{1.282338in}{1.308557in}}%
\pgfpathlineto{\pgfqpoint{1.285148in}{1.277048in}}%
\pgfpathlineto{\pgfqpoint{1.287959in}{1.260087in}}%
\pgfpathlineto{\pgfqpoint{1.290770in}{1.229045in}}%
\pgfpathlineto{\pgfqpoint{1.293580in}{1.232330in}}%
\pgfpathlineto{\pgfqpoint{1.296391in}{1.248706in}}%
\pgfpathlineto{\pgfqpoint{1.299202in}{1.281160in}}%
\pgfpathlineto{\pgfqpoint{1.302012in}{1.277298in}}%
\pgfpathlineto{\pgfqpoint{1.304823in}{1.291742in}}%
\pgfpathlineto{\pgfqpoint{1.307634in}{1.304342in}}%
\pgfpathlineto{\pgfqpoint{1.310445in}{1.301113in}}%
\pgfpathlineto{\pgfqpoint{1.313255in}{1.241476in}}%
\pgfpathlineto{\pgfqpoint{1.316066in}{1.224773in}}%
\pgfpathlineto{\pgfqpoint{1.318877in}{1.221584in}}%
\pgfpathlineto{\pgfqpoint{1.321687in}{1.248006in}}%
\pgfpathlineto{\pgfqpoint{1.324498in}{1.257435in}}%
\pgfpathlineto{\pgfqpoint{1.327309in}{1.259250in}}%
\pgfpathlineto{\pgfqpoint{1.330119in}{1.261107in}}%
\pgfpathlineto{\pgfqpoint{1.332930in}{1.262488in}}%
\pgfpathlineto{\pgfqpoint{1.335741in}{1.263040in}}%
\pgfpathlineto{\pgfqpoint{1.338551in}{1.264863in}}%
\pgfpathlineto{\pgfqpoint{1.341362in}{1.260776in}}%
\pgfpathlineto{\pgfqpoint{1.344173in}{1.262512in}}%
\pgfpathlineto{\pgfqpoint{1.346983in}{1.291821in}}%
\pgfpathlineto{\pgfqpoint{1.349794in}{1.273425in}}%
\pgfpathlineto{\pgfqpoint{1.352605in}{1.276077in}}%
\pgfpathlineto{\pgfqpoint{1.355415in}{1.271933in}}%
\pgfpathlineto{\pgfqpoint{1.358226in}{1.286881in}}%
\pgfpathlineto{\pgfqpoint{1.361037in}{1.293004in}}%
\pgfpathlineto{\pgfqpoint{1.363848in}{1.299539in}}%
\pgfpathlineto{\pgfqpoint{1.366658in}{1.291705in}}%
\pgfpathlineto{\pgfqpoint{1.369469in}{1.301352in}}%
\pgfpathlineto{\pgfqpoint{1.372280in}{1.304118in}}%
\pgfpathlineto{\pgfqpoint{1.375090in}{1.291357in}}%
\pgfpathlineto{\pgfqpoint{1.377901in}{1.294101in}}%
\pgfpathlineto{\pgfqpoint{1.380712in}{1.288102in}}%
\pgfpathlineto{\pgfqpoint{1.383522in}{1.302545in}}%
\pgfpathlineto{\pgfqpoint{1.386333in}{1.283626in}}%
\pgfpathlineto{\pgfqpoint{1.389144in}{1.283543in}}%
\pgfpathlineto{\pgfqpoint{1.391954in}{1.296458in}}%
\pgfpathlineto{\pgfqpoint{1.394765in}{1.274441in}}%
\pgfpathlineto{\pgfqpoint{1.397576in}{1.277838in}}%
\pgfpathlineto{\pgfqpoint{1.400386in}{1.293906in}}%
\pgfpathlineto{\pgfqpoint{1.403197in}{1.260681in}}%
\pgfpathlineto{\pgfqpoint{1.406008in}{1.247732in}}%
\pgfpathlineto{\pgfqpoint{1.408819in}{1.244383in}}%
\pgfpathlineto{\pgfqpoint{1.411629in}{1.262528in}}%
\pgfpathlineto{\pgfqpoint{1.414440in}{1.274797in}}%
\pgfpathlineto{\pgfqpoint{1.417251in}{1.282665in}}%
\pgfpathlineto{\pgfqpoint{1.420061in}{1.275975in}}%
\pgfpathlineto{\pgfqpoint{1.422872in}{1.284176in}}%
\pgfpathlineto{\pgfqpoint{1.425683in}{1.285107in}}%
\pgfpathlineto{\pgfqpoint{1.428493in}{1.302544in}}%
\pgfpathlineto{\pgfqpoint{1.431304in}{1.295444in}}%
\pgfpathlineto{\pgfqpoint{1.434115in}{1.317887in}}%
\pgfpathlineto{\pgfqpoint{1.436925in}{1.302177in}}%
\pgfpathlineto{\pgfqpoint{1.439736in}{1.299785in}}%
\pgfpathlineto{\pgfqpoint{1.442547in}{1.275282in}}%
\pgfpathlineto{\pgfqpoint{1.445357in}{1.290061in}}%
\pgfpathlineto{\pgfqpoint{1.448168in}{1.266745in}}%
\pgfpathlineto{\pgfqpoint{1.450979in}{1.264521in}}%
\pgfpathlineto{\pgfqpoint{1.453790in}{1.268950in}}%
\pgfpathlineto{\pgfqpoint{1.456600in}{1.292329in}}%
\pgfpathlineto{\pgfqpoint{1.459411in}{1.314987in}}%
\pgfpathlineto{\pgfqpoint{1.462222in}{1.308681in}}%
\pgfpathlineto{\pgfqpoint{1.465032in}{1.293419in}}%
\pgfpathlineto{\pgfqpoint{1.467843in}{1.297274in}}%
\pgfpathlineto{\pgfqpoint{1.470654in}{1.293557in}}%
\pgfpathlineto{\pgfqpoint{1.473464in}{1.285569in}}%
\pgfpathlineto{\pgfqpoint{1.476275in}{1.277800in}}%
\pgfpathlineto{\pgfqpoint{1.479086in}{1.292795in}}%
\pgfpathlineto{\pgfqpoint{1.481896in}{1.316909in}}%
\pgfpathlineto{\pgfqpoint{1.484707in}{1.326933in}}%
\pgfpathlineto{\pgfqpoint{1.487518in}{1.344336in}}%
\pgfpathlineto{\pgfqpoint{1.490328in}{1.340558in}}%
\pgfpathlineto{\pgfqpoint{1.493139in}{1.328656in}}%
\pgfpathlineto{\pgfqpoint{1.495950in}{1.321059in}}%
\pgfpathlineto{\pgfqpoint{1.498761in}{1.317612in}}%
\pgfpathlineto{\pgfqpoint{1.501571in}{1.267874in}}%
\pgfpathlineto{\pgfqpoint{1.504382in}{1.279416in}}%
\pgfpathlineto{\pgfqpoint{1.507193in}{1.302207in}}%
\pgfpathlineto{\pgfqpoint{1.510003in}{1.281446in}}%
\pgfpathlineto{\pgfqpoint{1.512814in}{1.300059in}}%
\pgfpathlineto{\pgfqpoint{1.515625in}{1.308686in}}%
\pgfpathlineto{\pgfqpoint{1.518435in}{1.302778in}}%
\pgfpathlineto{\pgfqpoint{1.521246in}{1.303359in}}%
\pgfpathlineto{\pgfqpoint{1.524057in}{1.314880in}}%
\pgfpathlineto{\pgfqpoint{1.526867in}{1.327746in}}%
\pgfpathlineto{\pgfqpoint{1.529678in}{1.327203in}}%
\pgfpathlineto{\pgfqpoint{1.532489in}{1.299022in}}%
\pgfpathlineto{\pgfqpoint{1.535299in}{1.302814in}}%
\pgfpathlineto{\pgfqpoint{1.538110in}{1.303923in}}%
\pgfpathlineto{\pgfqpoint{1.540921in}{1.304881in}}%
\pgfpathlineto{\pgfqpoint{1.543731in}{1.306039in}}%
\pgfpathlineto{\pgfqpoint{1.546542in}{1.306119in}}%
\pgfpathlineto{\pgfqpoint{1.549353in}{1.306593in}}%
\pgfpathlineto{\pgfqpoint{1.552164in}{1.307600in}}%
\pgfpathlineto{\pgfqpoint{1.554974in}{1.308704in}}%
\pgfpathlineto{\pgfqpoint{1.557785in}{1.309819in}}%
\pgfpathlineto{\pgfqpoint{1.560596in}{1.309715in}}%
\pgfpathlineto{\pgfqpoint{1.563406in}{1.309608in}}%
\pgfpathlineto{\pgfqpoint{1.566217in}{1.310586in}}%
\pgfpathlineto{\pgfqpoint{1.569028in}{1.311523in}}%
\pgfpathlineto{\pgfqpoint{1.571838in}{1.312314in}}%
\pgfpathlineto{\pgfqpoint{1.574649in}{1.313100in}}%
\pgfpathlineto{\pgfqpoint{1.577460in}{1.313966in}}%
\pgfpathlineto{\pgfqpoint{1.580270in}{1.313900in}}%
\pgfpathlineto{\pgfqpoint{1.583081in}{1.314123in}}%
\pgfpathlineto{\pgfqpoint{1.585892in}{1.314625in}}%
\pgfpathlineto{\pgfqpoint{1.588702in}{1.315433in}}%
\pgfpathlineto{\pgfqpoint{1.591513in}{1.316463in}}%
\pgfpathlineto{\pgfqpoint{1.594324in}{1.315822in}}%
\pgfpathlineto{\pgfqpoint{1.597135in}{1.316020in}}%
\pgfpathlineto{\pgfqpoint{1.599945in}{1.319665in}}%
\pgfpathlineto{\pgfqpoint{1.602756in}{1.319287in}}%
\pgfpathlineto{\pgfqpoint{1.605567in}{1.335911in}}%
\pgfpathlineto{\pgfqpoint{1.608377in}{1.340279in}}%
\pgfpathlineto{\pgfqpoint{1.611188in}{1.347511in}}%
\pgfpathlineto{\pgfqpoint{1.613999in}{1.318235in}}%
\pgfpathlineto{\pgfqpoint{1.616809in}{1.316518in}}%
\pgfpathlineto{\pgfqpoint{1.619620in}{1.317261in}}%
\pgfpathlineto{\pgfqpoint{1.622431in}{1.318267in}}%
\pgfpathlineto{\pgfqpoint{1.625241in}{1.319262in}}%
\pgfpathlineto{\pgfqpoint{1.628052in}{1.320257in}}%
\pgfpathlineto{\pgfqpoint{1.630863in}{1.321047in}}%
\pgfpathlineto{\pgfqpoint{1.633673in}{1.320786in}}%
\pgfpathlineto{\pgfqpoint{1.636484in}{1.292971in}}%
\pgfpathlineto{\pgfqpoint{1.639295in}{1.303344in}}%
\pgfpathlineto{\pgfqpoint{1.642106in}{1.310545in}}%
\pgfpathlineto{\pgfqpoint{1.644916in}{1.316800in}}%
\pgfpathlineto{\pgfqpoint{1.647727in}{1.322769in}}%
\pgfpathlineto{\pgfqpoint{1.650538in}{1.334332in}}%
\pgfpathlineto{\pgfqpoint{1.653348in}{1.331913in}}%
\pgfpathlineto{\pgfqpoint{1.656159in}{1.333855in}}%
\pgfpathlineto{\pgfqpoint{1.658970in}{1.332927in}}%
\pgfpathlineto{\pgfqpoint{1.661780in}{1.333218in}}%
\pgfpathlineto{\pgfqpoint{1.664591in}{1.323242in}}%
\pgfpathlineto{\pgfqpoint{1.667402in}{1.322584in}}%
\pgfpathlineto{\pgfqpoint{1.670212in}{1.333451in}}%
\pgfpathlineto{\pgfqpoint{1.673023in}{1.317493in}}%
\pgfpathlineto{\pgfqpoint{1.675834in}{1.316878in}}%
\pgfpathlineto{\pgfqpoint{1.678644in}{1.329254in}}%
\pgfpathlineto{\pgfqpoint{1.681455in}{1.313319in}}%
\pgfpathlineto{\pgfqpoint{1.684266in}{1.309257in}}%
\pgfpathlineto{\pgfqpoint{1.687077in}{1.330529in}}%
\pgfpathlineto{\pgfqpoint{1.689887in}{1.327077in}}%
\pgfpathlineto{\pgfqpoint{1.692698in}{1.337707in}}%
\pgfpathlineto{\pgfqpoint{1.695509in}{1.336403in}}%
\pgfpathlineto{\pgfqpoint{1.698319in}{1.342514in}}%
\pgfpathlineto{\pgfqpoint{1.701130in}{1.337136in}}%
\pgfpathlineto{\pgfqpoint{1.703941in}{1.327112in}}%
\pgfpathlineto{\pgfqpoint{1.706751in}{1.331864in}}%
\pgfpathlineto{\pgfqpoint{1.709562in}{1.357566in}}%
\pgfpathlineto{\pgfqpoint{1.712373in}{1.334708in}}%
\pgfpathlineto{\pgfqpoint{1.715183in}{1.325193in}}%
\pgfpathlineto{\pgfqpoint{1.717994in}{1.355144in}}%
\pgfpathlineto{\pgfqpoint{1.720805in}{1.337029in}}%
\pgfpathlineto{\pgfqpoint{1.723615in}{1.322422in}}%
\pgfpathlineto{\pgfqpoint{1.726426in}{1.335136in}}%
\pgfpathlineto{\pgfqpoint{1.729237in}{1.323762in}}%
\pgfpathlineto{\pgfqpoint{1.732048in}{1.324611in}}%
\pgfpathlineto{\pgfqpoint{1.734858in}{1.325417in}}%
\pgfpathlineto{\pgfqpoint{1.737669in}{1.326210in}}%
\pgfpathlineto{\pgfqpoint{1.740480in}{1.326516in}}%
\pgfpathlineto{\pgfqpoint{1.743290in}{1.327133in}}%
\pgfpathlineto{\pgfqpoint{1.746101in}{1.327868in}}%
\pgfpathlineto{\pgfqpoint{1.748912in}{1.327542in}}%
\pgfpathlineto{\pgfqpoint{1.751722in}{1.327915in}}%
\pgfpathlineto{\pgfqpoint{1.754533in}{1.328736in}}%
\pgfpathlineto{\pgfqpoint{1.757344in}{1.360814in}}%
\pgfpathlineto{\pgfqpoint{1.760154in}{1.328054in}}%
\pgfpathlineto{\pgfqpoint{1.762965in}{1.378629in}}%
\pgfpathlineto{\pgfqpoint{1.765776in}{1.345051in}}%
\pgfpathlineto{\pgfqpoint{1.768586in}{1.359054in}}%
\pgfpathlineto{\pgfqpoint{1.771397in}{1.336891in}}%
\pgfpathlineto{\pgfqpoint{1.774208in}{1.305435in}}%
\pgfpathlineto{\pgfqpoint{1.777018in}{1.310281in}}%
\pgfpathlineto{\pgfqpoint{1.779829in}{1.306652in}}%
\pgfpathlineto{\pgfqpoint{1.782640in}{1.302591in}}%
\pgfpathlineto{\pgfqpoint{1.785451in}{1.318791in}}%
\pgfpathlineto{\pgfqpoint{1.788261in}{1.316837in}}%
\pgfpathlineto{\pgfqpoint{1.791072in}{1.337242in}}%
\pgfpathlineto{\pgfqpoint{1.793883in}{1.327730in}}%
\pgfpathlineto{\pgfqpoint{1.796693in}{1.338200in}}%
\pgfpathlineto{\pgfqpoint{1.799504in}{1.332887in}}%
\pgfpathlineto{\pgfqpoint{1.802315in}{1.344435in}}%
\pgfpathlineto{\pgfqpoint{1.805125in}{1.331785in}}%
\pgfpathlineto{\pgfqpoint{1.807936in}{1.330759in}}%
\pgfpathlineto{\pgfqpoint{1.810747in}{1.344062in}}%
\pgfpathlineto{\pgfqpoint{1.813557in}{1.370485in}}%
\pgfpathlineto{\pgfqpoint{1.816368in}{1.350029in}}%
\pgfpathlineto{\pgfqpoint{1.819179in}{1.366045in}}%
\pgfpathlineto{\pgfqpoint{1.821989in}{1.371765in}}%
\pgfpathlineto{\pgfqpoint{1.824800in}{1.357154in}}%
\pgfpathlineto{\pgfqpoint{1.827611in}{1.362188in}}%
\pgfpathlineto{\pgfqpoint{1.830422in}{1.350120in}}%
\pgfpathlineto{\pgfqpoint{1.833232in}{1.343635in}}%
\pgfpathlineto{\pgfqpoint{1.836043in}{1.335278in}}%
\pgfpathlineto{\pgfqpoint{1.838854in}{1.346970in}}%
\pgfpathlineto{\pgfqpoint{1.841664in}{1.344472in}}%
\pgfpathlineto{\pgfqpoint{1.844475in}{1.343118in}}%
\pgfpathlineto{\pgfqpoint{1.847286in}{1.351831in}}%
\pgfpathlineto{\pgfqpoint{1.850096in}{1.291297in}}%
\pgfpathlineto{\pgfqpoint{1.852907in}{1.269862in}}%
\pgfpathlineto{\pgfqpoint{1.855718in}{1.277459in}}%
\pgfpathlineto{\pgfqpoint{1.858528in}{1.275032in}}%
\pgfpathlineto{\pgfqpoint{1.861339in}{1.281338in}}%
\pgfpathlineto{\pgfqpoint{1.864150in}{1.287768in}}%
\pgfpathlineto{\pgfqpoint{1.866960in}{1.289337in}}%
\pgfpathlineto{\pgfqpoint{1.869771in}{1.291104in}}%
\pgfpathlineto{\pgfqpoint{1.872582in}{1.285658in}}%
\pgfpathlineto{\pgfqpoint{1.875393in}{1.273104in}}%
\pgfpathlineto{\pgfqpoint{1.878203in}{1.286623in}}%
\pgfpathlineto{\pgfqpoint{1.881014in}{1.261935in}}%
\pgfpathlineto{\pgfqpoint{1.883825in}{1.279698in}}%
\pgfpathlineto{\pgfqpoint{1.886635in}{1.282166in}}%
\pgfpathlineto{\pgfqpoint{1.889446in}{1.292406in}}%
\pgfpathlineto{\pgfqpoint{1.892257in}{1.306166in}}%
\pgfpathlineto{\pgfqpoint{1.895067in}{1.293848in}}%
\pgfpathlineto{\pgfqpoint{1.897878in}{1.299443in}}%
\pgfpathlineto{\pgfqpoint{1.900689in}{1.312592in}}%
\pgfpathlineto{\pgfqpoint{1.903499in}{1.296760in}}%
\pgfpathlineto{\pgfqpoint{1.906310in}{1.316121in}}%
\pgfpathlineto{\pgfqpoint{1.909121in}{1.305131in}}%
\pgfpathlineto{\pgfqpoint{1.911931in}{1.308053in}}%
\pgfpathlineto{\pgfqpoint{1.914742in}{1.327690in}}%
\pgfpathlineto{\pgfqpoint{1.917553in}{1.315769in}}%
\pgfpathlineto{\pgfqpoint{1.920364in}{1.315150in}}%
\pgfpathlineto{\pgfqpoint{1.923174in}{1.302717in}}%
\pgfpathlineto{\pgfqpoint{1.925985in}{1.321064in}}%
\pgfpathlineto{\pgfqpoint{1.928796in}{1.326110in}}%
\pgfpathlineto{\pgfqpoint{1.931606in}{1.308457in}}%
\pgfpathlineto{\pgfqpoint{1.934417in}{1.319366in}}%
\pgfpathlineto{\pgfqpoint{1.937228in}{1.312821in}}%
\pgfpathlineto{\pgfqpoint{1.940038in}{1.305098in}}%
\pgfpathlineto{\pgfqpoint{1.942849in}{1.316112in}}%
\pgfpathlineto{\pgfqpoint{1.945660in}{1.342609in}}%
\pgfpathlineto{\pgfqpoint{1.948470in}{1.337293in}}%
\pgfpathlineto{\pgfqpoint{1.951281in}{1.347185in}}%
\pgfpathlineto{\pgfqpoint{1.954092in}{1.335054in}}%
\pgfpathlineto{\pgfqpoint{1.956902in}{1.323675in}}%
\pgfpathlineto{\pgfqpoint{1.959713in}{1.316370in}}%
\pgfpathlineto{\pgfqpoint{1.962524in}{1.328103in}}%
\pgfpathlineto{\pgfqpoint{1.965334in}{1.314558in}}%
\pgfpathlineto{\pgfqpoint{1.968145in}{1.317255in}}%
\pgfpathlineto{\pgfqpoint{1.970956in}{1.308590in}}%
\pgfpathlineto{\pgfqpoint{1.973767in}{1.288284in}}%
\pgfpathlineto{\pgfqpoint{1.976577in}{1.306725in}}%
\pgfpathlineto{\pgfqpoint{1.979388in}{1.314008in}}%
\pgfpathlineto{\pgfqpoint{1.982199in}{1.319526in}}%
\pgfpathlineto{\pgfqpoint{1.985009in}{1.301226in}}%
\pgfpathlineto{\pgfqpoint{1.987820in}{1.300110in}}%
\pgfpathlineto{\pgfqpoint{1.990631in}{1.305416in}}%
\pgfpathlineto{\pgfqpoint{1.993441in}{1.313051in}}%
\pgfpathlineto{\pgfqpoint{1.996252in}{1.313852in}}%
\pgfpathlineto{\pgfqpoint{1.999063in}{1.308261in}}%
\pgfpathlineto{\pgfqpoint{2.001873in}{1.317436in}}%
\pgfpathlineto{\pgfqpoint{2.004684in}{1.311809in}}%
\pgfpathlineto{\pgfqpoint{2.007495in}{1.329978in}}%
\pgfpathlineto{\pgfqpoint{2.010305in}{1.342843in}}%
\pgfpathlineto{\pgfqpoint{2.013116in}{1.291638in}}%
\pgfpathlineto{\pgfqpoint{2.015927in}{1.304140in}}%
\pgfpathlineto{\pgfqpoint{2.018738in}{1.313813in}}%
\pgfpathlineto{\pgfqpoint{2.021548in}{1.319371in}}%
\pgfpathlineto{\pgfqpoint{2.024359in}{1.398734in}}%
\pgfpathlineto{\pgfqpoint{2.027170in}{1.383763in}}%
\pgfpathlineto{\pgfqpoint{2.029980in}{1.353449in}}%
\pgfpathlineto{\pgfqpoint{2.032791in}{1.352450in}}%
\pgfpathlineto{\pgfqpoint{2.035602in}{1.365146in}}%
\pgfpathlineto{\pgfqpoint{2.038412in}{1.375849in}}%
\pgfpathlineto{\pgfqpoint{2.041223in}{1.380415in}}%
\pgfpathlineto{\pgfqpoint{2.044034in}{1.374939in}}%
\pgfpathlineto{\pgfqpoint{2.046844in}{1.365467in}}%
\pgfpathlineto{\pgfqpoint{2.049655in}{1.360133in}}%
\pgfpathlineto{\pgfqpoint{2.052466in}{1.349636in}}%
\pgfpathlineto{\pgfqpoint{2.055276in}{1.377936in}}%
\pgfpathlineto{\pgfqpoint{2.058087in}{1.371172in}}%
\pgfpathlineto{\pgfqpoint{2.060898in}{1.343391in}}%
\pgfpathlineto{\pgfqpoint{2.063709in}{1.352393in}}%
\pgfpathlineto{\pgfqpoint{2.066519in}{1.298198in}}%
\pgfpathlineto{\pgfqpoint{2.069330in}{1.298661in}}%
\pgfpathlineto{\pgfqpoint{2.072141in}{1.301784in}}%
\pgfpathlineto{\pgfqpoint{2.074951in}{1.343279in}}%
\pgfpathlineto{\pgfqpoint{2.077762in}{1.346192in}}%
\pgfpathlineto{\pgfqpoint{2.080573in}{1.329980in}}%
\pgfpathlineto{\pgfqpoint{2.083383in}{1.325987in}}%
\pgfpathlineto{\pgfqpoint{2.086194in}{1.333808in}}%
\pgfpathlineto{\pgfqpoint{2.089005in}{1.304865in}}%
\pgfpathlineto{\pgfqpoint{2.091815in}{1.316402in}}%
\pgfpathlineto{\pgfqpoint{2.094626in}{1.317013in}}%
\pgfpathlineto{\pgfqpoint{2.097437in}{1.313703in}}%
\pgfpathlineto{\pgfqpoint{2.100247in}{1.296478in}}%
\pgfpathlineto{\pgfqpoint{2.103058in}{1.243373in}}%
\pgfpathlineto{\pgfqpoint{2.105869in}{1.245944in}}%
\pgfpathlineto{\pgfqpoint{2.108680in}{1.271931in}}%
\pgfpathlineto{\pgfqpoint{2.111490in}{1.247733in}}%
\pgfpathlineto{\pgfqpoint{2.114301in}{1.260131in}}%
\pgfpathlineto{\pgfqpoint{2.117112in}{1.255807in}}%
\pgfpathlineto{\pgfqpoint{2.119922in}{1.254802in}}%
\pgfpathlineto{\pgfqpoint{2.122733in}{1.262701in}}%
\pgfpathlineto{\pgfqpoint{2.125544in}{1.247681in}}%
\pgfpathlineto{\pgfqpoint{2.128354in}{1.283249in}}%
\pgfpathlineto{\pgfqpoint{2.131165in}{1.310706in}}%
\pgfpathlineto{\pgfqpoint{2.133976in}{1.322159in}}%
\pgfpathlineto{\pgfqpoint{2.136786in}{1.309183in}}%
\pgfpathlineto{\pgfqpoint{2.139597in}{1.315314in}}%
\pgfpathlineto{\pgfqpoint{2.142408in}{1.329804in}}%
\pgfpathlineto{\pgfqpoint{2.145218in}{1.313202in}}%
\pgfpathlineto{\pgfqpoint{2.148029in}{1.325976in}}%
\pgfpathlineto{\pgfqpoint{2.150840in}{1.357538in}}%
\pgfpathlineto{\pgfqpoint{2.153651in}{1.329704in}}%
\pgfpathlineto{\pgfqpoint{2.156461in}{1.369037in}}%
\pgfpathlineto{\pgfqpoint{2.159272in}{1.383175in}}%
\pgfpathlineto{\pgfqpoint{2.162083in}{1.377635in}}%
\pgfpathlineto{\pgfqpoint{2.164893in}{1.355569in}}%
\pgfpathlineto{\pgfqpoint{2.167704in}{1.342690in}}%
\pgfpathlineto{\pgfqpoint{2.170515in}{1.338665in}}%
\pgfpathlineto{\pgfqpoint{2.173325in}{1.349784in}}%
\pgfpathlineto{\pgfqpoint{2.176136in}{1.330014in}}%
\pgfpathlineto{\pgfqpoint{2.178947in}{1.319085in}}%
\pgfpathlineto{\pgfqpoint{2.181757in}{1.310665in}}%
\pgfpathlineto{\pgfqpoint{2.184568in}{1.314949in}}%
\pgfpathlineto{\pgfqpoint{2.187379in}{1.316961in}}%
\pgfpathlineto{\pgfqpoint{2.190189in}{1.323932in}}%
\pgfpathlineto{\pgfqpoint{2.193000in}{1.305289in}}%
\pgfpathlineto{\pgfqpoint{2.195811in}{1.307299in}}%
\pgfpathlineto{\pgfqpoint{2.198621in}{1.309258in}}%
\pgfpathlineto{\pgfqpoint{2.201432in}{1.305580in}}%
\pgfpathlineto{\pgfqpoint{2.204243in}{1.309559in}}%
\pgfpathlineto{\pgfqpoint{2.207054in}{1.343073in}}%
\pgfpathlineto{\pgfqpoint{2.209864in}{1.331194in}}%
\pgfpathlineto{\pgfqpoint{2.212675in}{1.345948in}}%
\pgfpathlineto{\pgfqpoint{2.215486in}{1.342455in}}%
\pgfpathlineto{\pgfqpoint{2.218296in}{1.336173in}}%
\pgfpathlineto{\pgfqpoint{2.221107in}{1.351452in}}%
\pgfpathlineto{\pgfqpoint{2.223918in}{1.341828in}}%
\pgfpathlineto{\pgfqpoint{2.226728in}{1.339139in}}%
\pgfpathlineto{\pgfqpoint{2.229539in}{1.345489in}}%
\pgfpathlineto{\pgfqpoint{2.232350in}{1.331401in}}%
\pgfpathlineto{\pgfqpoint{2.235160in}{1.331901in}}%
\pgfpathlineto{\pgfqpoint{2.237971in}{1.324585in}}%
\pgfpathlineto{\pgfqpoint{2.240782in}{1.336335in}}%
\pgfpathlineto{\pgfqpoint{2.243592in}{1.349370in}}%
\pgfpathlineto{\pgfqpoint{2.246403in}{1.346939in}}%
\pgfpathlineto{\pgfqpoint{2.249214in}{1.331468in}}%
\pgfpathlineto{\pgfqpoint{2.252025in}{1.331183in}}%
\pgfpathlineto{\pgfqpoint{2.254835in}{1.333795in}}%
\pgfpathlineto{\pgfqpoint{2.257646in}{1.327753in}}%
\pgfpathlineto{\pgfqpoint{2.260457in}{1.327472in}}%
\pgfpathlineto{\pgfqpoint{2.263267in}{1.331965in}}%
\pgfpathlineto{\pgfqpoint{2.266078in}{1.325481in}}%
\pgfpathlineto{\pgfqpoint{2.268889in}{1.342880in}}%
\pgfpathlineto{\pgfqpoint{2.271699in}{1.339285in}}%
\pgfpathlineto{\pgfqpoint{2.274510in}{1.345809in}}%
\pgfpathlineto{\pgfqpoint{2.277321in}{1.346601in}}%
\pgfpathlineto{\pgfqpoint{2.280131in}{1.349338in}}%
\pgfpathlineto{\pgfqpoint{2.282942in}{1.360260in}}%
\pgfpathlineto{\pgfqpoint{2.285753in}{1.330682in}}%
\pgfpathlineto{\pgfqpoint{2.288563in}{1.343560in}}%
\pgfpathlineto{\pgfqpoint{2.291374in}{1.325655in}}%
\pgfpathlineto{\pgfqpoint{2.294185in}{1.326123in}}%
\pgfpathlineto{\pgfqpoint{2.296996in}{1.300151in}}%
\pgfpathlineto{\pgfqpoint{2.299806in}{1.293077in}}%
\pgfpathlineto{\pgfqpoint{2.302617in}{1.292007in}}%
\pgfpathlineto{\pgfqpoint{2.305428in}{1.294608in}}%
\pgfpathlineto{\pgfqpoint{2.308238in}{1.287066in}}%
\pgfpathlineto{\pgfqpoint{2.311049in}{1.286370in}}%
\pgfpathlineto{\pgfqpoint{2.313860in}{1.282014in}}%
\pgfpathlineto{\pgfqpoint{2.316670in}{1.286622in}}%
\pgfpathlineto{\pgfqpoint{2.319481in}{1.283142in}}%
\pgfpathlineto{\pgfqpoint{2.322292in}{1.277778in}}%
\pgfpathlineto{\pgfqpoint{2.325102in}{1.287206in}}%
\pgfpathlineto{\pgfqpoint{2.327913in}{1.288393in}}%
\pgfpathlineto{\pgfqpoint{2.330724in}{1.296312in}}%
\pgfpathlineto{\pgfqpoint{2.333534in}{1.290665in}}%
\pgfpathlineto{\pgfqpoint{2.336345in}{1.289947in}}%
\pgfpathlineto{\pgfqpoint{2.339156in}{1.293445in}}%
\pgfpathlineto{\pgfqpoint{2.341967in}{1.298830in}}%
\pgfpathlineto{\pgfqpoint{2.344777in}{1.310568in}}%
\pgfpathlineto{\pgfqpoint{2.347588in}{1.294691in}}%
\pgfpathlineto{\pgfqpoint{2.350399in}{1.292478in}}%
\pgfpathlineto{\pgfqpoint{2.353209in}{1.303177in}}%
\pgfpathlineto{\pgfqpoint{2.356020in}{1.255799in}}%
\pgfpathlineto{\pgfqpoint{2.358831in}{1.260504in}}%
\pgfpathlineto{\pgfqpoint{2.361641in}{1.258929in}}%
\pgfpathlineto{\pgfqpoint{2.364452in}{1.279284in}}%
\pgfpathlineto{\pgfqpoint{2.367263in}{1.309564in}}%
\pgfpathlineto{\pgfqpoint{2.370073in}{1.332967in}}%
\pgfpathlineto{\pgfqpoint{2.372884in}{1.311194in}}%
\pgfpathlineto{\pgfqpoint{2.375695in}{1.330983in}}%
\pgfpathlineto{\pgfqpoint{2.378505in}{1.314244in}}%
\pgfpathlineto{\pgfqpoint{2.381316in}{1.341077in}}%
\pgfpathlineto{\pgfqpoint{2.384127in}{1.350896in}}%
\pgfpathlineto{\pgfqpoint{2.386937in}{1.347152in}}%
\pgfpathlineto{\pgfqpoint{2.389748in}{1.343838in}}%
\pgfpathlineto{\pgfqpoint{2.392559in}{1.329366in}}%
\pgfpathlineto{\pgfqpoint{2.395370in}{1.318396in}}%
\pgfpathlineto{\pgfqpoint{2.398180in}{1.326205in}}%
\pgfpathlineto{\pgfqpoint{2.400991in}{1.319543in}}%
\pgfpathlineto{\pgfqpoint{2.403802in}{1.304963in}}%
\pgfpathlineto{\pgfqpoint{2.406612in}{1.307978in}}%
\pgfpathlineto{\pgfqpoint{2.409423in}{1.300697in}}%
\pgfpathlineto{\pgfqpoint{2.412234in}{1.302761in}}%
\pgfpathlineto{\pgfqpoint{2.415044in}{1.314756in}}%
\pgfpathlineto{\pgfqpoint{2.417855in}{1.318475in}}%
\pgfpathlineto{\pgfqpoint{2.420666in}{1.320761in}}%
\pgfpathlineto{\pgfqpoint{2.423476in}{1.309212in}}%
\pgfpathlineto{\pgfqpoint{2.426287in}{1.307152in}}%
\pgfpathlineto{\pgfqpoint{2.429098in}{1.313977in}}%
\pgfpathlineto{\pgfqpoint{2.431908in}{1.315056in}}%
\pgfpathlineto{\pgfqpoint{2.434719in}{1.318635in}}%
\pgfpathlineto{\pgfqpoint{2.437530in}{1.340693in}}%
\pgfpathlineto{\pgfqpoint{2.440341in}{1.322413in}}%
\pgfpathlineto{\pgfqpoint{2.443151in}{1.337093in}}%
\pgfpathlineto{\pgfqpoint{2.445962in}{1.333737in}}%
\pgfpathlineto{\pgfqpoint{2.448773in}{1.327596in}}%
\pgfpathlineto{\pgfqpoint{2.451583in}{1.329851in}}%
\pgfpathlineto{\pgfqpoint{2.454394in}{1.323855in}}%
\pgfpathlineto{\pgfqpoint{2.457205in}{1.330883in}}%
\pgfpathlineto{\pgfqpoint{2.460015in}{1.356271in}}%
\pgfpathlineto{\pgfqpoint{2.462826in}{1.357253in}}%
\pgfpathlineto{\pgfqpoint{2.465637in}{1.344603in}}%
\pgfpathlineto{\pgfqpoint{2.468447in}{1.334329in}}%
\pgfpathlineto{\pgfqpoint{2.471258in}{1.348061in}}%
\pgfpathlineto{\pgfqpoint{2.474069in}{1.358390in}}%
\pgfpathlineto{\pgfqpoint{2.476879in}{1.352633in}}%
\pgfpathlineto{\pgfqpoint{2.479690in}{1.365784in}}%
\pgfpathlineto{\pgfqpoint{2.482501in}{1.376904in}}%
\pgfpathlineto{\pgfqpoint{2.485312in}{1.388367in}}%
\pgfpathlineto{\pgfqpoint{2.488122in}{1.388209in}}%
\pgfpathlineto{\pgfqpoint{2.490933in}{1.401908in}}%
\pgfpathlineto{\pgfqpoint{2.493744in}{1.387262in}}%
\pgfpathlineto{\pgfqpoint{2.496554in}{1.392322in}}%
\pgfpathlineto{\pgfqpoint{2.499365in}{1.392575in}}%
\pgfpathlineto{\pgfqpoint{2.502176in}{1.391473in}}%
\pgfpathlineto{\pgfqpoint{2.504986in}{1.412819in}}%
\pgfpathlineto{\pgfqpoint{2.507797in}{1.418704in}}%
\pgfpathlineto{\pgfqpoint{2.510608in}{1.416599in}}%
\pgfpathlineto{\pgfqpoint{2.513418in}{1.404383in}}%
\pgfpathlineto{\pgfqpoint{2.516229in}{1.356670in}}%
\pgfpathlineto{\pgfqpoint{2.519040in}{1.367140in}}%
\pgfpathlineto{\pgfqpoint{2.521850in}{1.328678in}}%
\pgfpathlineto{\pgfqpoint{2.524661in}{1.309562in}}%
\pgfpathlineto{\pgfqpoint{2.527472in}{1.386848in}}%
\pgfpathlineto{\pgfqpoint{2.530283in}{1.304505in}}%
\pgfpathlineto{\pgfqpoint{2.533093in}{1.385931in}}%
\pgfpathlineto{\pgfqpoint{2.535904in}{1.383485in}}%
\pgfpathlineto{\pgfqpoint{2.538715in}{1.308167in}}%
\pgfpathlineto{\pgfqpoint{2.541525in}{1.307195in}}%
\pgfpathlineto{\pgfqpoint{2.544336in}{1.360323in}}%
\pgfpathlineto{\pgfqpoint{2.547147in}{1.335373in}}%
\pgfpathlineto{\pgfqpoint{2.549957in}{1.322248in}}%
\pgfpathlineto{\pgfqpoint{2.552768in}{1.319419in}}%
\pgfpathlineto{\pgfqpoint{2.555579in}{1.296600in}}%
\pgfpathlineto{\pgfqpoint{2.558389in}{1.308558in}}%
\pgfpathlineto{\pgfqpoint{2.561200in}{1.292249in}}%
\pgfpathlineto{\pgfqpoint{2.564011in}{1.306101in}}%
\pgfpathlineto{\pgfqpoint{2.566821in}{1.286004in}}%
\pgfpathlineto{\pgfqpoint{2.569632in}{1.278141in}}%
\pgfpathlineto{\pgfqpoint{2.572443in}{1.284938in}}%
\pgfpathlineto{\pgfqpoint{2.575253in}{1.284752in}}%
\pgfpathlineto{\pgfqpoint{2.578064in}{1.287660in}}%
\pgfpathlineto{\pgfqpoint{2.580875in}{1.295996in}}%
\pgfpathlineto{\pgfqpoint{2.583686in}{1.312962in}}%
\pgfpathlineto{\pgfqpoint{2.586496in}{1.298540in}}%
\pgfpathlineto{\pgfqpoint{2.589307in}{1.297886in}}%
\pgfpathlineto{\pgfqpoint{2.592118in}{1.314838in}}%
\pgfpathlineto{\pgfqpoint{2.594928in}{1.301230in}}%
\pgfpathlineto{\pgfqpoint{2.597739in}{1.307744in}}%
\pgfpathlineto{\pgfqpoint{2.600550in}{1.292162in}}%
\pgfpathlineto{\pgfqpoint{2.603360in}{1.281892in}}%
\pgfpathlineto{\pgfqpoint{2.606171in}{1.292039in}}%
\pgfpathlineto{\pgfqpoint{2.608982in}{1.291315in}}%
\pgfpathlineto{\pgfqpoint{2.611792in}{1.294817in}}%
\pgfpathlineto{\pgfqpoint{2.614603in}{1.306284in}}%
\pgfpathlineto{\pgfqpoint{2.617414in}{1.320436in}}%
\pgfpathlineto{\pgfqpoint{2.620224in}{1.318477in}}%
\pgfpathlineto{\pgfqpoint{2.623035in}{1.313284in}}%
\pgfpathlineto{\pgfqpoint{2.625846in}{1.308900in}}%
\pgfpathlineto{\pgfqpoint{2.628657in}{1.312620in}}%
\pgfpathlineto{\pgfqpoint{2.631467in}{1.303464in}}%
\pgfpathlineto{\pgfqpoint{2.634278in}{1.314958in}}%
\pgfpathlineto{\pgfqpoint{2.637089in}{1.320330in}}%
\pgfpathlineto{\pgfqpoint{2.639899in}{1.322441in}}%
\pgfpathlineto{\pgfqpoint{2.642710in}{1.327316in}}%
\pgfpathlineto{\pgfqpoint{2.645521in}{1.321673in}}%
\pgfpathlineto{\pgfqpoint{2.648331in}{1.318186in}}%
\pgfpathlineto{\pgfqpoint{2.651142in}{1.323232in}}%
\pgfpathlineto{\pgfqpoint{2.653953in}{1.326549in}}%
\pgfpathlineto{\pgfqpoint{2.656763in}{1.324798in}}%
\pgfpathlineto{\pgfqpoint{2.659574in}{1.332583in}}%
\pgfpathlineto{\pgfqpoint{2.662385in}{1.321890in}}%
\pgfpathlineto{\pgfqpoint{2.665195in}{1.321561in}}%
\pgfpathlineto{\pgfqpoint{2.668006in}{1.318661in}}%
\pgfpathlineto{\pgfqpoint{2.670817in}{1.310070in}}%
\pgfpathlineto{\pgfqpoint{2.673628in}{1.291724in}}%
\pgfpathlineto{\pgfqpoint{2.676438in}{1.293674in}}%
\pgfpathlineto{\pgfqpoint{2.679249in}{1.289243in}}%
\pgfpathlineto{\pgfqpoint{2.682060in}{1.293665in}}%
\pgfpathlineto{\pgfqpoint{2.684870in}{1.306566in}}%
\pgfpathlineto{\pgfqpoint{2.687681in}{1.304094in}}%
\pgfpathlineto{\pgfqpoint{2.690492in}{1.310389in}}%
\pgfpathlineto{\pgfqpoint{2.693302in}{1.304233in}}%
\pgfpathlineto{\pgfqpoint{2.696113in}{1.284483in}}%
\pgfpathlineto{\pgfqpoint{2.698924in}{1.287459in}}%
\pgfpathlineto{\pgfqpoint{2.701734in}{1.289035in}}%
\pgfpathlineto{\pgfqpoint{2.704545in}{1.319978in}}%
\pgfpathlineto{\pgfqpoint{2.707356in}{1.307548in}}%
\pgfpathlineto{\pgfqpoint{2.710166in}{1.315426in}}%
\pgfpathlineto{\pgfqpoint{2.712977in}{1.305212in}}%
\pgfpathlineto{\pgfqpoint{2.715788in}{1.308591in}}%
\pgfpathlineto{\pgfqpoint{2.718599in}{1.302624in}}%
\pgfpathlineto{\pgfqpoint{2.721409in}{1.349160in}}%
\pgfpathlineto{\pgfqpoint{2.724220in}{1.345973in}}%
\pgfpathlineto{\pgfqpoint{2.727031in}{1.345024in}}%
\pgfpathlineto{\pgfqpoint{2.729841in}{1.337308in}}%
\pgfpathlineto{\pgfqpoint{2.732652in}{1.350030in}}%
\pgfpathlineto{\pgfqpoint{2.735463in}{1.340063in}}%
\pgfpathlineto{\pgfqpoint{2.738273in}{1.334394in}}%
\pgfpathlineto{\pgfqpoint{2.741084in}{1.339493in}}%
\pgfpathlineto{\pgfqpoint{2.743895in}{1.334625in}}%
\pgfpathlineto{\pgfqpoint{2.746705in}{1.339650in}}%
\pgfpathlineto{\pgfqpoint{2.749516in}{1.327536in}}%
\pgfpathlineto{\pgfqpoint{2.752327in}{1.325124in}}%
\pgfpathlineto{\pgfqpoint{2.755137in}{1.325026in}}%
\pgfpathlineto{\pgfqpoint{2.757948in}{1.309263in}}%
\pgfpathlineto{\pgfqpoint{2.760759in}{1.305896in}}%
\pgfpathlineto{\pgfqpoint{2.763570in}{1.320906in}}%
\pgfpathlineto{\pgfqpoint{2.766380in}{1.297175in}}%
\pgfpathlineto{\pgfqpoint{2.769191in}{1.298222in}}%
\pgfpathlineto{\pgfqpoint{2.772002in}{1.296148in}}%
\pgfpathlineto{\pgfqpoint{2.774812in}{1.303314in}}%
\pgfpathlineto{\pgfqpoint{2.777623in}{1.304945in}}%
\pgfpathlineto{\pgfqpoint{2.780434in}{1.308617in}}%
\pgfpathlineto{\pgfqpoint{2.783244in}{1.307299in}}%
\pgfpathlineto{\pgfqpoint{2.786055in}{1.307908in}}%
\pgfpathlineto{\pgfqpoint{2.788866in}{1.324439in}}%
\pgfpathlineto{\pgfqpoint{2.791676in}{1.334844in}}%
\pgfpathlineto{\pgfqpoint{2.794487in}{1.320810in}}%
\pgfpathlineto{\pgfqpoint{2.797298in}{1.320084in}}%
\pgfpathlineto{\pgfqpoint{2.800108in}{1.324328in}}%
\pgfpathlineto{\pgfqpoint{2.802919in}{1.324391in}}%
\pgfpathlineto{\pgfqpoint{2.805730in}{1.315743in}}%
\pgfpathlineto{\pgfqpoint{2.808540in}{1.325184in}}%
\pgfpathlineto{\pgfqpoint{2.811351in}{1.310492in}}%
\pgfpathlineto{\pgfqpoint{2.814162in}{1.323524in}}%
\pgfpathlineto{\pgfqpoint{2.816973in}{1.328733in}}%
\pgfpathlineto{\pgfqpoint{2.819783in}{1.324556in}}%
\pgfpathlineto{\pgfqpoint{2.822594in}{1.309480in}}%
\pgfpathlineto{\pgfqpoint{2.825405in}{1.322011in}}%
\pgfpathlineto{\pgfqpoint{2.828215in}{1.317559in}}%
\pgfpathlineto{\pgfqpoint{2.831026in}{1.319952in}}%
\pgfpathlineto{\pgfqpoint{2.833837in}{1.334297in}}%
\pgfpathlineto{\pgfqpoint{2.836647in}{1.340298in}}%
\pgfpathlineto{\pgfqpoint{2.839458in}{1.325994in}}%
\pgfpathlineto{\pgfqpoint{2.842269in}{1.346208in}}%
\pgfpathlineto{\pgfqpoint{2.845079in}{1.334565in}}%
\pgfpathlineto{\pgfqpoint{2.847890in}{1.337991in}}%
\pgfpathlineto{\pgfqpoint{2.850701in}{1.321135in}}%
\pgfpathlineto{\pgfqpoint{2.853511in}{1.340509in}}%
\pgfpathlineto{\pgfqpoint{2.856322in}{1.338357in}}%
\pgfpathlineto{\pgfqpoint{2.859133in}{1.321732in}}%
\pgfpathlineto{\pgfqpoint{2.861944in}{1.323411in}}%
\pgfpathlineto{\pgfqpoint{2.864754in}{1.345283in}}%
\pgfpathlineto{\pgfqpoint{2.867565in}{1.313228in}}%
\pgfpathlineto{\pgfqpoint{2.870376in}{1.343763in}}%
\pgfpathlineto{\pgfqpoint{2.873186in}{1.354934in}}%
\pgfpathlineto{\pgfqpoint{2.875997in}{1.350240in}}%
\pgfpathlineto{\pgfqpoint{2.878808in}{1.352072in}}%
\pgfpathlineto{\pgfqpoint{2.881618in}{1.357687in}}%
\pgfpathlineto{\pgfqpoint{2.884429in}{1.331060in}}%
\pgfpathlineto{\pgfqpoint{2.887240in}{1.310689in}}%
\pgfpathlineto{\pgfqpoint{2.890050in}{1.303016in}}%
\pgfpathlineto{\pgfqpoint{2.892861in}{1.279442in}}%
\pgfpathlineto{\pgfqpoint{2.895672in}{1.299793in}}%
\pgfpathlineto{\pgfqpoint{2.898482in}{1.285642in}}%
\pgfpathlineto{\pgfqpoint{2.901293in}{1.297634in}}%
\pgfpathlineto{\pgfqpoint{2.904104in}{1.303349in}}%
\pgfpathlineto{\pgfqpoint{2.906915in}{1.288635in}}%
\pgfpathlineto{\pgfqpoint{2.909725in}{1.307260in}}%
\pgfpathlineto{\pgfqpoint{2.912536in}{1.169749in}}%
\pgfpathlineto{\pgfqpoint{2.915347in}{1.199718in}}%
\pgfpathlineto{\pgfqpoint{2.918157in}{1.228190in}}%
\pgfpathlineto{\pgfqpoint{2.920968in}{1.248233in}}%
\pgfpathlineto{\pgfqpoint{2.923779in}{1.251487in}}%
\pgfpathlineto{\pgfqpoint{2.926589in}{1.270944in}}%
\pgfpathlineto{\pgfqpoint{2.929400in}{1.277629in}}%
\pgfpathlineto{\pgfqpoint{2.932211in}{1.296484in}}%
\pgfpathlineto{\pgfqpoint{2.935021in}{1.301969in}}%
\pgfpathlineto{\pgfqpoint{2.937832in}{1.287181in}}%
\pgfpathlineto{\pgfqpoint{2.940643in}{1.299366in}}%
\pgfpathlineto{\pgfqpoint{2.943453in}{1.310768in}}%
\pgfpathlineto{\pgfqpoint{2.946264in}{1.305394in}}%
\pgfpathlineto{\pgfqpoint{2.949075in}{1.306563in}}%
\pgfpathlineto{\pgfqpoint{2.951886in}{1.306332in}}%
\pgfpathlineto{\pgfqpoint{2.954696in}{1.299525in}}%
\pgfpathlineto{\pgfqpoint{2.957507in}{1.292502in}}%
\pgfpathlineto{\pgfqpoint{2.960318in}{1.297181in}}%
\pgfpathlineto{\pgfqpoint{2.963128in}{1.291181in}}%
\pgfpathlineto{\pgfqpoint{2.965939in}{1.298332in}}%
\pgfpathlineto{\pgfqpoint{2.968750in}{1.299130in}}%
\pgfpathlineto{\pgfqpoint{2.971560in}{1.303379in}}%
\pgfpathlineto{\pgfqpoint{2.974371in}{1.298753in}}%
\pgfpathlineto{\pgfqpoint{2.977182in}{1.297162in}}%
\pgfpathlineto{\pgfqpoint{2.979992in}{1.304387in}}%
\pgfpathlineto{\pgfqpoint{2.982803in}{1.298315in}}%
\pgfpathlineto{\pgfqpoint{2.985614in}{1.304520in}}%
\pgfpathlineto{\pgfqpoint{2.988424in}{1.303278in}}%
\pgfpathlineto{\pgfqpoint{2.991235in}{1.316048in}}%
\pgfpathlineto{\pgfqpoint{2.994046in}{1.308748in}}%
\pgfpathlineto{\pgfqpoint{2.996856in}{1.334318in}}%
\pgfpathlineto{\pgfqpoint{2.999667in}{1.325051in}}%
\pgfpathlineto{\pgfqpoint{3.002478in}{1.331601in}}%
\pgfpathlineto{\pgfqpoint{3.005289in}{1.300323in}}%
\pgfpathlineto{\pgfqpoint{3.008099in}{1.285072in}}%
\pgfpathlineto{\pgfqpoint{3.010910in}{1.304200in}}%
\pgfpathlineto{\pgfqpoint{3.013721in}{1.296677in}}%
\pgfpathlineto{\pgfqpoint{3.016531in}{1.298212in}}%
\pgfpathlineto{\pgfqpoint{3.019342in}{1.294119in}}%
\pgfpathlineto{\pgfqpoint{3.022153in}{1.304763in}}%
\pgfpathlineto{\pgfqpoint{3.024963in}{1.312152in}}%
\pgfpathlineto{\pgfqpoint{3.027774in}{1.315502in}}%
\pgfpathlineto{\pgfqpoint{3.030585in}{1.323116in}}%
\pgfpathlineto{\pgfqpoint{3.033395in}{1.308156in}}%
\pgfpathlineto{\pgfqpoint{3.036206in}{1.332002in}}%
\pgfpathlineto{\pgfqpoint{3.039017in}{1.330233in}}%
\pgfpathlineto{\pgfqpoint{3.041827in}{1.309392in}}%
\pgfpathlineto{\pgfqpoint{3.044638in}{1.296803in}}%
\pgfpathlineto{\pgfqpoint{3.047449in}{1.317535in}}%
\pgfpathlineto{\pgfqpoint{3.050260in}{1.317421in}}%
\pgfpathlineto{\pgfqpoint{3.053070in}{1.312083in}}%
\pgfpathlineto{\pgfqpoint{3.055881in}{1.332598in}}%
\pgfpathlineto{\pgfqpoint{3.058692in}{1.333295in}}%
\pgfpathlineto{\pgfqpoint{3.061502in}{1.317227in}}%
\pgfpathlineto{\pgfqpoint{3.064313in}{1.307496in}}%
\pgfpathlineto{\pgfqpoint{3.067124in}{1.307227in}}%
\pgfpathlineto{\pgfqpoint{3.069934in}{1.307822in}}%
\pgfpathlineto{\pgfqpoint{3.072745in}{1.306630in}}%
\pgfpathlineto{\pgfqpoint{3.075556in}{1.315755in}}%
\pgfpathlineto{\pgfqpoint{3.078366in}{1.335790in}}%
\pgfpathlineto{\pgfqpoint{3.081177in}{1.342999in}}%
\pgfpathlineto{\pgfqpoint{3.083988in}{1.324342in}}%
\pgfpathlineto{\pgfqpoint{3.086798in}{1.292645in}}%
\pgfpathlineto{\pgfqpoint{3.089609in}{1.297493in}}%
\pgfpathlineto{\pgfqpoint{3.092420in}{1.285393in}}%
\pgfpathlineto{\pgfqpoint{3.095231in}{1.276220in}}%
\pgfpathlineto{\pgfqpoint{3.098041in}{1.265936in}}%
\pgfpathlineto{\pgfqpoint{3.100852in}{1.292871in}}%
\pgfpathlineto{\pgfqpoint{3.103663in}{1.304505in}}%
\pgfpathlineto{\pgfqpoint{3.106473in}{1.309294in}}%
\pgfpathlineto{\pgfqpoint{3.109284in}{1.298593in}}%
\pgfpathlineto{\pgfqpoint{3.112095in}{1.285810in}}%
\pgfpathlineto{\pgfqpoint{3.114905in}{1.298122in}}%
\pgfpathlineto{\pgfqpoint{3.117716in}{1.298304in}}%
\pgfpathlineto{\pgfqpoint{3.120527in}{1.308298in}}%
\pgfpathlineto{\pgfqpoint{3.123337in}{1.309446in}}%
\pgfpathlineto{\pgfqpoint{3.126148in}{1.297057in}}%
\pgfpathlineto{\pgfqpoint{3.128959in}{1.301999in}}%
\pgfpathlineto{\pgfqpoint{3.131769in}{1.307216in}}%
\pgfpathlineto{\pgfqpoint{3.134580in}{1.307235in}}%
\pgfpathlineto{\pgfqpoint{3.137391in}{1.306642in}}%
\pgfpathlineto{\pgfqpoint{3.140202in}{1.316870in}}%
\pgfpathlineto{\pgfqpoint{3.143012in}{1.291191in}}%
\pgfpathlineto{\pgfqpoint{3.145823in}{1.308423in}}%
\pgfpathlineto{\pgfqpoint{3.148634in}{1.315364in}}%
\pgfpathlineto{\pgfqpoint{3.151444in}{1.313350in}}%
\pgfpathlineto{\pgfqpoint{3.154255in}{1.329543in}}%
\pgfpathlineto{\pgfqpoint{3.157066in}{1.316658in}}%
\pgfpathlineto{\pgfqpoint{3.159876in}{1.336445in}}%
\pgfpathlineto{\pgfqpoint{3.162687in}{1.329480in}}%
\pgfpathlineto{\pgfqpoint{3.165498in}{1.306554in}}%
\pgfpathlineto{\pgfqpoint{3.168308in}{1.323950in}}%
\pgfpathlineto{\pgfqpoint{3.171119in}{1.304761in}}%
\pgfpathlineto{\pgfqpoint{3.173930in}{1.323634in}}%
\pgfpathlineto{\pgfqpoint{3.176740in}{1.306909in}}%
\pgfpathlineto{\pgfqpoint{3.179551in}{1.310112in}}%
\pgfpathlineto{\pgfqpoint{3.182362in}{1.301509in}}%
\pgfpathlineto{\pgfqpoint{3.185173in}{1.310291in}}%
\pgfpathlineto{\pgfqpoint{3.187983in}{1.310630in}}%
\pgfpathlineto{\pgfqpoint{3.190794in}{1.331732in}}%
\pgfpathlineto{\pgfqpoint{3.193605in}{1.326243in}}%
\pgfpathlineto{\pgfqpoint{3.196415in}{1.321599in}}%
\pgfpathlineto{\pgfqpoint{3.199226in}{1.316223in}}%
\pgfpathlineto{\pgfqpoint{3.202037in}{1.318307in}}%
\pgfpathlineto{\pgfqpoint{3.204847in}{1.319341in}}%
\pgfpathlineto{\pgfqpoint{3.207658in}{1.314863in}}%
\pgfpathlineto{\pgfqpoint{3.210469in}{1.315274in}}%
\pgfpathlineto{\pgfqpoint{3.213279in}{1.304976in}}%
\pgfpathlineto{\pgfqpoint{3.216090in}{1.297160in}}%
\pgfpathlineto{\pgfqpoint{3.218901in}{1.303875in}}%
\pgfpathlineto{\pgfqpoint{3.221711in}{1.308865in}}%
\pgfpathlineto{\pgfqpoint{3.224522in}{1.321331in}}%
\pgfpathlineto{\pgfqpoint{3.227333in}{1.319651in}}%
\pgfpathlineto{\pgfqpoint{3.230143in}{1.315224in}}%
\pgfpathlineto{\pgfqpoint{3.232954in}{1.315143in}}%
\pgfpathlineto{\pgfqpoint{3.235765in}{1.331350in}}%
\pgfpathlineto{\pgfqpoint{3.238576in}{1.321702in}}%
\pgfpathlineto{\pgfqpoint{3.241386in}{1.309071in}}%
\pgfpathlineto{\pgfqpoint{3.244197in}{1.270333in}}%
\pgfpathlineto{\pgfqpoint{3.247008in}{1.288614in}}%
\pgfpathlineto{\pgfqpoint{3.249818in}{1.300477in}}%
\pgfpathlineto{\pgfqpoint{3.252629in}{1.310150in}}%
\pgfpathlineto{\pgfqpoint{3.255440in}{1.315987in}}%
\pgfpathlineto{\pgfqpoint{3.258250in}{1.306743in}}%
\pgfpathlineto{\pgfqpoint{3.261061in}{1.327805in}}%
\pgfpathlineto{\pgfqpoint{3.263872in}{1.326879in}}%
\pgfpathlineto{\pgfqpoint{3.266682in}{1.324987in}}%
\pgfpathlineto{\pgfqpoint{3.269493in}{1.315608in}}%
\pgfpathlineto{\pgfqpoint{3.272304in}{1.319098in}}%
\pgfpathlineto{\pgfqpoint{3.275114in}{1.303690in}}%
\pgfpathlineto{\pgfqpoint{3.277925in}{1.265176in}}%
\pgfpathlineto{\pgfqpoint{3.280736in}{1.285366in}}%
\pgfpathlineto{\pgfqpoint{3.283547in}{1.298188in}}%
\pgfpathlineto{\pgfqpoint{3.286357in}{1.301737in}}%
\pgfpathlineto{\pgfqpoint{3.289168in}{1.286216in}}%
\pgfpathlineto{\pgfqpoint{3.291979in}{1.300671in}}%
\pgfpathlineto{\pgfqpoint{3.294789in}{1.298693in}}%
\pgfpathlineto{\pgfqpoint{3.297600in}{1.301186in}}%
\pgfpathlineto{\pgfqpoint{3.300411in}{1.310448in}}%
\pgfpathlineto{\pgfqpoint{3.303221in}{1.317172in}}%
\pgfpathlineto{\pgfqpoint{3.306032in}{1.312730in}}%
\pgfpathlineto{\pgfqpoint{3.308843in}{1.327861in}}%
\pgfpathlineto{\pgfqpoint{3.311653in}{1.309931in}}%
\pgfpathlineto{\pgfqpoint{3.314464in}{1.310376in}}%
\pgfpathlineto{\pgfqpoint{3.317275in}{1.323917in}}%
\pgfpathlineto{\pgfqpoint{3.320085in}{1.314525in}}%
\pgfpathlineto{\pgfqpoint{3.322896in}{1.315481in}}%
\pgfpathlineto{\pgfqpoint{3.325707in}{1.316479in}}%
\pgfpathlineto{\pgfqpoint{3.328518in}{1.326225in}}%
\pgfpathlineto{\pgfqpoint{3.331328in}{1.320155in}}%
\pgfpathlineto{\pgfqpoint{3.334139in}{1.328071in}}%
\pgfpathlineto{\pgfqpoint{3.336950in}{1.319700in}}%
\pgfpathlineto{\pgfqpoint{3.339760in}{1.293535in}}%
\pgfpathlineto{\pgfqpoint{3.342571in}{1.300087in}}%
\pgfpathlineto{\pgfqpoint{3.345382in}{1.309185in}}%
\pgfpathlineto{\pgfqpoint{3.348192in}{1.321770in}}%
\pgfpathlineto{\pgfqpoint{3.351003in}{1.313294in}}%
\pgfpathlineto{\pgfqpoint{3.353814in}{1.317618in}}%
\pgfpathlineto{\pgfqpoint{3.356624in}{1.305832in}}%
\pgfpathlineto{\pgfqpoint{3.359435in}{1.320663in}}%
\pgfpathlineto{\pgfqpoint{3.362246in}{1.313011in}}%
\pgfpathlineto{\pgfqpoint{3.365056in}{1.309003in}}%
\pgfpathlineto{\pgfqpoint{3.367867in}{1.319339in}}%
\pgfpathlineto{\pgfqpoint{3.370678in}{1.321828in}}%
\pgfpathlineto{\pgfqpoint{3.373489in}{1.317941in}}%
\pgfpathlineto{\pgfqpoint{3.376299in}{1.346248in}}%
\pgfpathlineto{\pgfqpoint{3.379110in}{1.329280in}}%
\pgfpathlineto{\pgfqpoint{3.381921in}{1.314000in}}%
\pgfpathlineto{\pgfqpoint{3.384731in}{1.308558in}}%
\pgfpathlineto{\pgfqpoint{3.387542in}{1.312560in}}%
\pgfpathlineto{\pgfqpoint{3.390353in}{1.318728in}}%
\pgfpathlineto{\pgfqpoint{3.393163in}{1.333030in}}%
\pgfpathlineto{\pgfqpoint{3.395974in}{1.322220in}}%
\pgfpathlineto{\pgfqpoint{3.398785in}{1.298912in}}%
\pgfpathlineto{\pgfqpoint{3.401595in}{1.288182in}}%
\pgfpathlineto{\pgfqpoint{3.404406in}{1.288696in}}%
\pgfpathlineto{\pgfqpoint{3.407217in}{1.298226in}}%
\pgfpathlineto{\pgfqpoint{3.410027in}{1.296111in}}%
\pgfpathlineto{\pgfqpoint{3.412838in}{1.298146in}}%
\pgfpathlineto{\pgfqpoint{3.415649in}{1.278836in}}%
\pgfpathlineto{\pgfqpoint{3.418459in}{1.299656in}}%
\pgfpathlineto{\pgfqpoint{3.421270in}{1.305672in}}%
\pgfpathlineto{\pgfqpoint{3.424081in}{1.311999in}}%
\pgfpathlineto{\pgfqpoint{3.426892in}{1.269890in}}%
\pgfpathlineto{\pgfqpoint{3.429702in}{1.297991in}}%
\pgfpathlineto{\pgfqpoint{3.432513in}{1.291855in}}%
\pgfpathlineto{\pgfqpoint{3.435324in}{1.282926in}}%
\pgfpathlineto{\pgfqpoint{3.438134in}{1.289962in}}%
\pgfpathlineto{\pgfqpoint{3.440945in}{1.312635in}}%
\pgfpathlineto{\pgfqpoint{3.443756in}{1.307988in}}%
\pgfpathlineto{\pgfqpoint{3.446566in}{1.317468in}}%
\pgfpathlineto{\pgfqpoint{3.449377in}{1.320790in}}%
\pgfpathlineto{\pgfqpoint{3.452188in}{1.337635in}}%
\pgfpathlineto{\pgfqpoint{3.454998in}{1.321276in}}%
\pgfpathlineto{\pgfqpoint{3.457809in}{1.316882in}}%
\pgfpathlineto{\pgfqpoint{3.460620in}{1.330330in}}%
\pgfpathlineto{\pgfqpoint{3.463430in}{1.320862in}}%
\pgfpathlineto{\pgfqpoint{3.466241in}{1.313575in}}%
\pgfpathlineto{\pgfqpoint{3.469052in}{1.310414in}}%
\pgfpathlineto{\pgfqpoint{3.471863in}{1.309882in}}%
\pgfpathlineto{\pgfqpoint{3.474673in}{1.311330in}}%
\pgfpathlineto{\pgfqpoint{3.477484in}{1.314198in}}%
\pgfpathlineto{\pgfqpoint{3.480295in}{1.321223in}}%
\pgfpathlineto{\pgfqpoint{3.483105in}{1.355857in}}%
\pgfpathlineto{\pgfqpoint{3.485916in}{1.370092in}}%
\pgfpathlineto{\pgfqpoint{3.488727in}{1.360124in}}%
\pgfpathlineto{\pgfqpoint{3.491537in}{1.304754in}}%
\pgfpathlineto{\pgfqpoint{3.494348in}{1.290646in}}%
\pgfpathlineto{\pgfqpoint{3.497159in}{1.294475in}}%
\pgfpathlineto{\pgfqpoint{3.499969in}{1.310649in}}%
\pgfpathlineto{\pgfqpoint{3.502780in}{1.338206in}}%
\pgfpathlineto{\pgfqpoint{3.505591in}{1.321443in}}%
\pgfpathlineto{\pgfqpoint{3.508401in}{1.307578in}}%
\pgfpathlineto{\pgfqpoint{3.511212in}{1.322974in}}%
\pgfpathlineto{\pgfqpoint{3.514023in}{1.300194in}}%
\pgfpathlineto{\pgfqpoint{3.516834in}{1.314244in}}%
\pgfpathlineto{\pgfqpoint{3.519644in}{1.303436in}}%
\pgfpathlineto{\pgfqpoint{3.522455in}{1.297387in}}%
\pgfpathlineto{\pgfqpoint{3.525266in}{1.310197in}}%
\pgfpathlineto{\pgfqpoint{3.528076in}{1.302419in}}%
\pgfpathlineto{\pgfqpoint{3.530887in}{1.297395in}}%
\pgfpathlineto{\pgfqpoint{3.533698in}{1.300742in}}%
\pgfpathlineto{\pgfqpoint{3.536508in}{1.318393in}}%
\pgfpathlineto{\pgfqpoint{3.539319in}{1.301490in}}%
\pgfpathlineto{\pgfqpoint{3.542130in}{1.310643in}}%
\pgfpathlineto{\pgfqpoint{3.544940in}{1.299977in}}%
\pgfpathlineto{\pgfqpoint{3.547751in}{1.312090in}}%
\pgfpathlineto{\pgfqpoint{3.550562in}{1.304489in}}%
\pgfpathlineto{\pgfqpoint{3.553372in}{1.352713in}}%
\pgfpathlineto{\pgfqpoint{3.556183in}{1.318927in}}%
\pgfpathlineto{\pgfqpoint{3.558994in}{1.297575in}}%
\pgfpathlineto{\pgfqpoint{3.561805in}{1.294826in}}%
\pgfpathlineto{\pgfqpoint{3.564615in}{1.287757in}}%
\pgfpathlineto{\pgfqpoint{3.567426in}{1.270145in}}%
\pgfpathlineto{\pgfqpoint{3.570237in}{1.282300in}}%
\pgfpathlineto{\pgfqpoint{3.573047in}{1.272872in}}%
\pgfpathlineto{\pgfqpoint{3.575858in}{1.273136in}}%
\pgfpathlineto{\pgfqpoint{3.578669in}{1.281413in}}%
\pgfpathlineto{\pgfqpoint{3.581479in}{1.275302in}}%
\pgfpathlineto{\pgfqpoint{3.584290in}{1.284185in}}%
\pgfpathlineto{\pgfqpoint{3.587101in}{1.300761in}}%
\pgfpathlineto{\pgfqpoint{3.589911in}{1.287317in}}%
\pgfpathlineto{\pgfqpoint{3.592722in}{1.284348in}}%
\pgfpathlineto{\pgfqpoint{3.595533in}{1.278253in}}%
\pgfpathlineto{\pgfqpoint{3.598343in}{1.295817in}}%
\pgfpathlineto{\pgfqpoint{3.601154in}{1.309532in}}%
\pgfpathlineto{\pgfqpoint{3.603965in}{1.296126in}}%
\pgfpathlineto{\pgfqpoint{3.606776in}{1.290836in}}%
\pgfpathlineto{\pgfqpoint{3.609586in}{1.281278in}}%
\pgfpathlineto{\pgfqpoint{3.612397in}{1.297549in}}%
\pgfpathlineto{\pgfqpoint{3.615208in}{1.282918in}}%
\pgfpathlineto{\pgfqpoint{3.618018in}{1.295074in}}%
\pgfpathlineto{\pgfqpoint{3.620829in}{1.310719in}}%
\pgfpathlineto{\pgfqpoint{3.623640in}{1.338482in}}%
\pgfpathlineto{\pgfqpoint{3.626450in}{1.292416in}}%
\pgfpathlineto{\pgfqpoint{3.629261in}{1.290313in}}%
\pgfpathlineto{\pgfqpoint{3.632072in}{1.282807in}}%
\pgfpathlineto{\pgfqpoint{3.634882in}{1.296775in}}%
\pgfpathlineto{\pgfqpoint{3.637693in}{1.309103in}}%
\pgfpathlineto{\pgfqpoint{3.640504in}{1.286358in}}%
\pgfpathlineto{\pgfqpoint{3.643314in}{1.296550in}}%
\pgfpathlineto{\pgfqpoint{3.646125in}{1.311489in}}%
\pgfpathlineto{\pgfqpoint{3.648936in}{1.312602in}}%
\pgfpathlineto{\pgfqpoint{3.651746in}{1.299749in}}%
\pgfpathlineto{\pgfqpoint{3.654557in}{1.302902in}}%
\pgfpathlineto{\pgfqpoint{3.657368in}{1.281512in}}%
\pgfpathlineto{\pgfqpoint{3.660179in}{1.293464in}}%
\pgfpathlineto{\pgfqpoint{3.662989in}{1.297117in}}%
\pgfpathlineto{\pgfqpoint{3.665800in}{1.301392in}}%
\pgfpathlineto{\pgfqpoint{3.668611in}{1.307617in}}%
\pgfpathlineto{\pgfqpoint{3.671421in}{1.310718in}}%
\pgfpathlineto{\pgfqpoint{3.674232in}{1.306042in}}%
\pgfpathlineto{\pgfqpoint{3.677043in}{1.317933in}}%
\pgfpathlineto{\pgfqpoint{3.679853in}{1.302346in}}%
\pgfpathlineto{\pgfqpoint{3.682664in}{1.316052in}}%
\pgfpathlineto{\pgfqpoint{3.685475in}{1.323797in}}%
\pgfpathlineto{\pgfqpoint{3.688285in}{1.288702in}}%
\pgfpathlineto{\pgfqpoint{3.691096in}{1.306386in}}%
\pgfpathlineto{\pgfqpoint{3.693907in}{1.313628in}}%
\pgfpathlineto{\pgfqpoint{3.696717in}{1.328301in}}%
\pgfpathlineto{\pgfqpoint{3.699528in}{1.318339in}}%
\pgfpathlineto{\pgfqpoint{3.702339in}{1.342442in}}%
\pgfpathlineto{\pgfqpoint{3.705150in}{1.320219in}}%
\pgfpathlineto{\pgfqpoint{3.707960in}{1.290575in}}%
\pgfpathlineto{\pgfqpoint{3.710771in}{1.278544in}}%
\pgfpathlineto{\pgfqpoint{3.713582in}{1.303230in}}%
\pgfpathlineto{\pgfqpoint{3.716392in}{1.337654in}}%
\pgfpathlineto{\pgfqpoint{3.719203in}{1.315189in}}%
\pgfpathlineto{\pgfqpoint{3.722014in}{1.309164in}}%
\pgfpathlineto{\pgfqpoint{3.724824in}{1.290626in}}%
\pgfpathlineto{\pgfqpoint{3.727635in}{1.299654in}}%
\pgfpathlineto{\pgfqpoint{3.730446in}{1.297716in}}%
\pgfpathlineto{\pgfqpoint{3.733256in}{1.290420in}}%
\pgfpathlineto{\pgfqpoint{3.736067in}{1.307896in}}%
\pgfpathlineto{\pgfqpoint{3.738878in}{1.318991in}}%
\pgfpathlineto{\pgfqpoint{3.741688in}{1.341290in}}%
\pgfpathlineto{\pgfqpoint{3.744499in}{1.322113in}}%
\pgfpathlineto{\pgfqpoint{3.747310in}{1.331231in}}%
\pgfpathlineto{\pgfqpoint{3.750121in}{1.343503in}}%
\pgfpathlineto{\pgfqpoint{3.752931in}{1.343085in}}%
\pgfpathlineto{\pgfqpoint{3.755742in}{1.316070in}}%
\pgfpathlineto{\pgfqpoint{3.758553in}{1.300724in}}%
\pgfpathlineto{\pgfqpoint{3.761363in}{1.324856in}}%
\pgfpathlineto{\pgfqpoint{3.764174in}{1.307645in}}%
\pgfpathlineto{\pgfqpoint{3.766985in}{1.334919in}}%
\pgfpathlineto{\pgfqpoint{3.769795in}{1.330206in}}%
\pgfpathlineto{\pgfqpoint{3.772606in}{1.330856in}}%
\pgfpathlineto{\pgfqpoint{3.775417in}{1.308222in}}%
\pgfpathlineto{\pgfqpoint{3.778227in}{1.289420in}}%
\pgfpathlineto{\pgfqpoint{3.781038in}{1.307712in}}%
\pgfpathlineto{\pgfqpoint{3.783849in}{1.303887in}}%
\pgfpathlineto{\pgfqpoint{3.786659in}{1.318011in}}%
\pgfpathlineto{\pgfqpoint{3.789470in}{1.336005in}}%
\pgfpathlineto{\pgfqpoint{3.792281in}{1.235492in}}%
\pgfpathlineto{\pgfqpoint{3.795092in}{1.258491in}}%
\pgfpathlineto{\pgfqpoint{3.797902in}{1.287609in}}%
\pgfpathlineto{\pgfqpoint{3.800713in}{1.274810in}}%
\pgfpathlineto{\pgfqpoint{3.803524in}{1.294263in}}%
\pgfpathlineto{\pgfqpoint{3.806334in}{1.330407in}}%
\pgfpathlineto{\pgfqpoint{3.809145in}{1.373195in}}%
\pgfpathlineto{\pgfqpoint{3.811956in}{1.339492in}}%
\pgfpathlineto{\pgfqpoint{3.814766in}{1.293819in}}%
\pgfpathlineto{\pgfqpoint{3.817577in}{1.319939in}}%
\pgfpathlineto{\pgfqpoint{3.820388in}{1.277815in}}%
\pgfpathlineto{\pgfqpoint{3.823198in}{1.271014in}}%
\pgfpathlineto{\pgfqpoint{3.826009in}{1.260239in}}%
\pgfpathlineto{\pgfqpoint{3.828820in}{1.282765in}}%
\pgfpathlineto{\pgfqpoint{3.831630in}{1.280978in}}%
\pgfpathlineto{\pgfqpoint{3.834441in}{1.259037in}}%
\pgfpathlineto{\pgfqpoint{3.837252in}{1.284054in}}%
\pgfpathlineto{\pgfqpoint{3.840062in}{1.291771in}}%
\pgfpathlineto{\pgfqpoint{3.842873in}{1.268205in}}%
\pgfpathlineto{\pgfqpoint{3.845684in}{1.284750in}}%
\pgfpathlineto{\pgfqpoint{3.848495in}{1.292888in}}%
\pgfpathlineto{\pgfqpoint{3.851305in}{1.254687in}}%
\pgfpathlineto{\pgfqpoint{3.854116in}{1.272125in}}%
\pgfpathlineto{\pgfqpoint{3.856927in}{1.285041in}}%
\pgfpathlineto{\pgfqpoint{3.859737in}{1.287714in}}%
\pgfpathlineto{\pgfqpoint{3.862548in}{1.321485in}}%
\pgfpathlineto{\pgfqpoint{3.865359in}{1.333270in}}%
\pgfpathlineto{\pgfqpoint{3.868169in}{1.319569in}}%
\pgfpathlineto{\pgfqpoint{3.870980in}{1.318450in}}%
\pgfpathlineto{\pgfqpoint{3.873791in}{1.287139in}}%
\pgfpathlineto{\pgfqpoint{3.876601in}{1.290841in}}%
\pgfpathlineto{\pgfqpoint{3.879412in}{1.284302in}}%
\pgfpathlineto{\pgfqpoint{3.882223in}{1.279147in}}%
\pgfpathlineto{\pgfqpoint{3.885033in}{1.270656in}}%
\pgfpathlineto{\pgfqpoint{3.887844in}{1.266893in}}%
\pgfpathlineto{\pgfqpoint{3.890655in}{1.274700in}}%
\pgfpathlineto{\pgfqpoint{3.893466in}{1.292860in}}%
\pgfpathlineto{\pgfqpoint{3.896276in}{1.277610in}}%
\pgfpathlineto{\pgfqpoint{3.899087in}{1.278668in}}%
\pgfpathlineto{\pgfqpoint{3.901898in}{1.284035in}}%
\pgfpathlineto{\pgfqpoint{3.904708in}{1.266180in}}%
\pgfpathlineto{\pgfqpoint{3.907519in}{1.252827in}}%
\pgfpathlineto{\pgfqpoint{3.910330in}{1.269559in}}%
\pgfpathlineto{\pgfqpoint{3.913140in}{1.260119in}}%
\pgfpathlineto{\pgfqpoint{3.915951in}{1.266914in}}%
\pgfpathlineto{\pgfqpoint{3.918762in}{1.287633in}}%
\pgfpathlineto{\pgfqpoint{3.921572in}{1.271119in}}%
\pgfpathlineto{\pgfqpoint{3.924383in}{1.287419in}}%
\pgfpathlineto{\pgfqpoint{3.927194in}{1.282827in}}%
\pgfpathlineto{\pgfqpoint{3.930004in}{1.289659in}}%
\pgfpathlineto{\pgfqpoint{3.932815in}{1.280618in}}%
\pgfpathlineto{\pgfqpoint{3.935626in}{1.267294in}}%
\pgfpathlineto{\pgfqpoint{3.938437in}{1.266940in}}%
\pgfpathlineto{\pgfqpoint{3.941247in}{1.279194in}}%
\pgfpathlineto{\pgfqpoint{3.944058in}{1.264377in}}%
\pgfpathlineto{\pgfqpoint{3.946869in}{1.284045in}}%
\pgfpathlineto{\pgfqpoint{3.949679in}{1.281937in}}%
\pgfpathlineto{\pgfqpoint{3.952490in}{1.291414in}}%
\pgfpathlineto{\pgfqpoint{3.955301in}{1.316657in}}%
\pgfpathlineto{\pgfqpoint{3.958111in}{1.323683in}}%
\pgfpathlineto{\pgfqpoint{3.960922in}{1.309625in}}%
\pgfpathlineto{\pgfqpoint{3.963733in}{1.302171in}}%
\pgfpathlineto{\pgfqpoint{3.966543in}{1.315556in}}%
\pgfpathlineto{\pgfqpoint{3.969354in}{1.315718in}}%
\pgfpathlineto{\pgfqpoint{3.972165in}{1.291817in}}%
\pgfpathlineto{\pgfqpoint{3.974975in}{1.311163in}}%
\pgfpathlineto{\pgfqpoint{3.977786in}{1.308033in}}%
\pgfpathlineto{\pgfqpoint{3.980597in}{1.299882in}}%
\pgfpathlineto{\pgfqpoint{3.983408in}{1.293107in}}%
\pgfpathlineto{\pgfqpoint{3.986218in}{1.286690in}}%
\pgfpathlineto{\pgfqpoint{3.989029in}{1.275169in}}%
\pgfpathlineto{\pgfqpoint{3.991840in}{1.307030in}}%
\pgfpathlineto{\pgfqpoint{3.994650in}{1.294600in}}%
\pgfpathlineto{\pgfqpoint{3.997461in}{1.310797in}}%
\pgfpathlineto{\pgfqpoint{4.000272in}{1.291311in}}%
\pgfpathlineto{\pgfqpoint{4.003082in}{1.307907in}}%
\pgfpathlineto{\pgfqpoint{4.005893in}{1.294188in}}%
\pgfpathlineto{\pgfqpoint{4.008704in}{1.303074in}}%
\pgfpathlineto{\pgfqpoint{4.011514in}{1.288670in}}%
\pgfpathlineto{\pgfqpoint{4.014325in}{1.297979in}}%
\pgfpathlineto{\pgfqpoint{4.017136in}{1.263644in}}%
\pgfpathlineto{\pgfqpoint{4.019946in}{1.278455in}}%
\pgfpathlineto{\pgfqpoint{4.022757in}{1.284906in}}%
\pgfpathlineto{\pgfqpoint{4.025568in}{1.278372in}}%
\pgfpathlineto{\pgfqpoint{4.028378in}{1.294955in}}%
\pgfpathlineto{\pgfqpoint{4.031189in}{1.291086in}}%
\pgfpathlineto{\pgfqpoint{4.034000in}{1.281795in}}%
\pgfpathlineto{\pgfqpoint{4.036811in}{1.286706in}}%
\pgfpathlineto{\pgfqpoint{4.039621in}{1.280880in}}%
\pgfpathlineto{\pgfqpoint{4.042432in}{1.285143in}}%
\pgfpathlineto{\pgfqpoint{4.045243in}{1.273456in}}%
\pgfpathlineto{\pgfqpoint{4.048053in}{1.286815in}}%
\pgfpathlineto{\pgfqpoint{4.050864in}{1.306074in}}%
\pgfpathlineto{\pgfqpoint{4.053675in}{1.329366in}}%
\pgfpathlineto{\pgfqpoint{4.056485in}{1.316780in}}%
\pgfpathlineto{\pgfqpoint{4.059296in}{1.316573in}}%
\pgfpathlineto{\pgfqpoint{4.062107in}{1.307845in}}%
\pgfpathlineto{\pgfqpoint{4.064917in}{1.324401in}}%
\pgfpathlineto{\pgfqpoint{4.067728in}{1.310552in}}%
\pgfpathlineto{\pgfqpoint{4.070539in}{1.306405in}}%
\pgfpathlineto{\pgfqpoint{4.073349in}{1.315206in}}%
\pgfpathlineto{\pgfqpoint{4.076160in}{1.283572in}}%
\pgfpathlineto{\pgfqpoint{4.078971in}{1.335901in}}%
\pgfpathlineto{\pgfqpoint{4.081782in}{1.347325in}}%
\pgfpathlineto{\pgfqpoint{4.084592in}{1.301121in}}%
\pgfpathlineto{\pgfqpoint{4.087403in}{1.274193in}}%
\pgfpathlineto{\pgfqpoint{4.090214in}{1.320991in}}%
\pgfpathlineto{\pgfqpoint{4.093024in}{1.306225in}}%
\pgfpathlineto{\pgfqpoint{4.095835in}{1.298416in}}%
\pgfpathlineto{\pgfqpoint{4.098646in}{1.306748in}}%
\pgfpathlineto{\pgfqpoint{4.101456in}{1.294433in}}%
\pgfpathlineto{\pgfqpoint{4.104267in}{1.261776in}}%
\pgfpathlineto{\pgfqpoint{4.107078in}{1.270593in}}%
\pgfpathlineto{\pgfqpoint{4.109888in}{1.262692in}}%
\pgfpathlineto{\pgfqpoint{4.112699in}{1.273082in}}%
\pgfpathlineto{\pgfqpoint{4.115510in}{1.269162in}}%
\pgfpathlineto{\pgfqpoint{4.118320in}{1.274830in}}%
\pgfpathlineto{\pgfqpoint{4.121131in}{1.281096in}}%
\pgfpathlineto{\pgfqpoint{4.123942in}{1.279067in}}%
\pgfpathlineto{\pgfqpoint{4.126753in}{1.274302in}}%
\pgfpathlineto{\pgfqpoint{4.129563in}{1.289875in}}%
\pgfpathlineto{\pgfqpoint{4.132374in}{1.273960in}}%
\pgfpathlineto{\pgfqpoint{4.135185in}{1.296145in}}%
\pgfpathlineto{\pgfqpoint{4.137995in}{1.302758in}}%
\pgfpathlineto{\pgfqpoint{4.140806in}{1.299480in}}%
\pgfpathlineto{\pgfqpoint{4.143617in}{1.287442in}}%
\pgfpathlineto{\pgfqpoint{4.146427in}{1.308109in}}%
\pgfpathlineto{\pgfqpoint{4.149238in}{1.300280in}}%
\pgfpathlineto{\pgfqpoint{4.152049in}{1.298255in}}%
\pgfpathlineto{\pgfqpoint{4.154859in}{1.290430in}}%
\pgfpathlineto{\pgfqpoint{4.157670in}{1.279975in}}%
\pgfpathlineto{\pgfqpoint{4.160481in}{1.274860in}}%
\pgfpathlineto{\pgfqpoint{4.163291in}{1.284953in}}%
\pgfpathlineto{\pgfqpoint{4.166102in}{1.286961in}}%
\pgfpathlineto{\pgfqpoint{4.168913in}{1.296000in}}%
\pgfpathlineto{\pgfqpoint{4.171724in}{1.289277in}}%
\pgfpathlineto{\pgfqpoint{4.174534in}{1.293356in}}%
\pgfpathlineto{\pgfqpoint{4.177345in}{1.280590in}}%
\pgfpathlineto{\pgfqpoint{4.180156in}{1.287680in}}%
\pgfpathlineto{\pgfqpoint{4.182966in}{1.288182in}}%
\pgfpathlineto{\pgfqpoint{4.185777in}{1.300106in}}%
\pgfpathlineto{\pgfqpoint{4.188588in}{1.296038in}}%
\pgfpathlineto{\pgfqpoint{4.191398in}{1.287500in}}%
\pgfpathlineto{\pgfqpoint{4.194209in}{1.294911in}}%
\pgfpathlineto{\pgfqpoint{4.197020in}{1.303839in}}%
\pgfpathlineto{\pgfqpoint{4.199830in}{1.301409in}}%
\pgfpathlineto{\pgfqpoint{4.202641in}{1.296504in}}%
\pgfpathlineto{\pgfqpoint{4.205452in}{1.292358in}}%
\pgfpathlineto{\pgfqpoint{4.208262in}{1.289839in}}%
\pgfpathlineto{\pgfqpoint{4.211073in}{1.296914in}}%
\pgfpathlineto{\pgfqpoint{4.213884in}{1.291391in}}%
\pgfpathlineto{\pgfqpoint{4.216695in}{1.283234in}}%
\pgfpathlineto{\pgfqpoint{4.219505in}{1.280490in}}%
\pgfpathlineto{\pgfqpoint{4.222316in}{1.276035in}}%
\pgfpathlineto{\pgfqpoint{4.225127in}{1.288037in}}%
\pgfpathlineto{\pgfqpoint{4.227937in}{1.316332in}}%
\pgfpathlineto{\pgfqpoint{4.230748in}{1.285440in}}%
\pgfpathlineto{\pgfqpoint{4.233559in}{1.306588in}}%
\pgfpathlineto{\pgfqpoint{4.236369in}{1.310458in}}%
\pgfpathlineto{\pgfqpoint{4.239180in}{1.298625in}}%
\pgfpathlineto{\pgfqpoint{4.241991in}{1.298003in}}%
\pgfpathlineto{\pgfqpoint{4.244801in}{1.291778in}}%
\pgfpathlineto{\pgfqpoint{4.247612in}{1.295003in}}%
\pgfpathlineto{\pgfqpoint{4.250423in}{1.285187in}}%
\pgfpathlineto{\pgfqpoint{4.253233in}{1.288082in}}%
\pgfpathlineto{\pgfqpoint{4.256044in}{1.304891in}}%
\pgfpathlineto{\pgfqpoint{4.258855in}{1.314520in}}%
\pgfpathlineto{\pgfqpoint{4.261665in}{1.304440in}}%
\pgfpathlineto{\pgfqpoint{4.264476in}{1.290656in}}%
\pgfpathlineto{\pgfqpoint{4.267287in}{1.310053in}}%
\pgfpathlineto{\pgfqpoint{4.270098in}{1.294762in}}%
\pgfpathlineto{\pgfqpoint{4.272908in}{1.294284in}}%
\pgfpathlineto{\pgfqpoint{4.275719in}{1.298620in}}%
\pgfpathlineto{\pgfqpoint{4.278530in}{1.289804in}}%
\pgfpathlineto{\pgfqpoint{4.281340in}{1.293546in}}%
\pgfpathlineto{\pgfqpoint{4.284151in}{1.303642in}}%
\pgfpathlineto{\pgfqpoint{4.286962in}{1.299329in}}%
\pgfpathlineto{\pgfqpoint{4.289772in}{1.316848in}}%
\pgfpathlineto{\pgfqpoint{4.292583in}{1.308383in}}%
\pgfpathlineto{\pgfqpoint{4.295394in}{1.312029in}}%
\pgfpathlineto{\pgfqpoint{4.298204in}{1.299490in}}%
\pgfpathlineto{\pgfqpoint{4.301015in}{1.305105in}}%
\pgfpathlineto{\pgfqpoint{4.303826in}{1.313003in}}%
\pgfpathlineto{\pgfqpoint{4.306636in}{1.289799in}}%
\pgfpathlineto{\pgfqpoint{4.309447in}{1.298939in}}%
\pgfpathlineto{\pgfqpoint{4.312258in}{1.302349in}}%
\pgfpathlineto{\pgfqpoint{4.315069in}{1.289539in}}%
\pgfpathlineto{\pgfqpoint{4.317879in}{1.311845in}}%
\pgfpathlineto{\pgfqpoint{4.320690in}{1.313019in}}%
\pgfpathlineto{\pgfqpoint{4.323501in}{1.306731in}}%
\pgfpathlineto{\pgfqpoint{4.326311in}{1.300697in}}%
\pgfpathlineto{\pgfqpoint{4.329122in}{1.296950in}}%
\pgfpathlineto{\pgfqpoint{4.331933in}{1.313023in}}%
\pgfpathlineto{\pgfqpoint{4.334743in}{1.326002in}}%
\pgfpathlineto{\pgfqpoint{4.337554in}{1.316943in}}%
\pgfpathlineto{\pgfqpoint{4.340365in}{1.316667in}}%
\pgfpathlineto{\pgfqpoint{4.343175in}{1.277515in}}%
\pgfpathlineto{\pgfqpoint{4.345986in}{1.280219in}}%
\pgfpathlineto{\pgfqpoint{4.348797in}{1.282345in}}%
\pgfpathlineto{\pgfqpoint{4.351607in}{1.310621in}}%
\pgfpathlineto{\pgfqpoint{4.354418in}{1.307950in}}%
\pgfpathlineto{\pgfqpoint{4.357229in}{1.360425in}}%
\pgfpathlineto{\pgfqpoint{4.360040in}{1.337608in}}%
\pgfpathlineto{\pgfqpoint{4.362850in}{1.298538in}}%
\pgfpathlineto{\pgfqpoint{4.365661in}{1.283805in}}%
\pgfpathlineto{\pgfqpoint{4.368472in}{1.293337in}}%
\pgfpathlineto{\pgfqpoint{4.371282in}{1.282204in}}%
\pgfpathlineto{\pgfqpoint{4.374093in}{1.316465in}}%
\pgfpathlineto{\pgfqpoint{4.376904in}{1.317273in}}%
\pgfpathlineto{\pgfqpoint{4.379714in}{1.302983in}}%
\pgfpathlineto{\pgfqpoint{4.382525in}{1.319453in}}%
\pgfpathlineto{\pgfqpoint{4.385336in}{1.312352in}}%
\pgfpathlineto{\pgfqpoint{4.388146in}{1.338293in}}%
\pgfpathlineto{\pgfqpoint{4.390957in}{1.355240in}}%
\pgfpathlineto{\pgfqpoint{4.393768in}{1.331942in}}%
\pgfpathlineto{\pgfqpoint{4.396578in}{1.295823in}}%
\pgfpathlineto{\pgfqpoint{4.399389in}{1.300323in}}%
\pgfpathlineto{\pgfqpoint{4.402200in}{1.264714in}}%
\pgfpathlineto{\pgfqpoint{4.405011in}{1.276311in}}%
\pgfpathlineto{\pgfqpoint{4.407821in}{1.285900in}}%
\pgfpathlineto{\pgfqpoint{4.410632in}{1.300625in}}%
\pgfpathlineto{\pgfqpoint{4.413443in}{1.289384in}}%
\pgfpathlineto{\pgfqpoint{4.416253in}{1.293866in}}%
\pgfpathlineto{\pgfqpoint{4.419064in}{1.290151in}}%
\pgfpathlineto{\pgfqpoint{4.421875in}{1.311792in}}%
\pgfpathlineto{\pgfqpoint{4.424685in}{1.312333in}}%
\pgfpathlineto{\pgfqpoint{4.427496in}{1.304403in}}%
\pgfpathlineto{\pgfqpoint{4.430307in}{1.306889in}}%
\pgfpathlineto{\pgfqpoint{4.433117in}{1.308782in}}%
\pgfpathlineto{\pgfqpoint{4.435928in}{1.302395in}}%
\pgfpathlineto{\pgfqpoint{4.438739in}{1.298744in}}%
\pgfpathlineto{\pgfqpoint{4.441549in}{1.300399in}}%
\pgfpathlineto{\pgfqpoint{4.444360in}{1.300266in}}%
\pgfpathlineto{\pgfqpoint{4.447171in}{1.305626in}}%
\pgfpathlineto{\pgfqpoint{4.449981in}{1.280409in}}%
\pgfpathlineto{\pgfqpoint{4.452792in}{1.276868in}}%
\pgfpathlineto{\pgfqpoint{4.455603in}{1.270184in}}%
\pgfpathlineto{\pgfqpoint{4.458414in}{1.263228in}}%
\pgfpathlineto{\pgfqpoint{4.461224in}{1.282254in}}%
\pgfpathlineto{\pgfqpoint{4.464035in}{1.295117in}}%
\pgfpathlineto{\pgfqpoint{4.466846in}{1.290019in}}%
\pgfpathlineto{\pgfqpoint{4.469656in}{1.300445in}}%
\pgfpathlineto{\pgfqpoint{4.472467in}{1.304271in}}%
\pgfpathlineto{\pgfqpoint{4.475278in}{1.302459in}}%
\pgfpathlineto{\pgfqpoint{4.478088in}{1.297680in}}%
\pgfpathlineto{\pgfqpoint{4.480899in}{1.297250in}}%
\pgfpathlineto{\pgfqpoint{4.483710in}{1.297475in}}%
\pgfpathlineto{\pgfqpoint{4.486520in}{1.294744in}}%
\pgfpathlineto{\pgfqpoint{4.489331in}{1.281214in}}%
\pgfpathlineto{\pgfqpoint{4.492142in}{1.278316in}}%
\pgfpathlineto{\pgfqpoint{4.494952in}{1.296045in}}%
\pgfpathlineto{\pgfqpoint{4.497763in}{1.290697in}}%
\pgfpathlineto{\pgfqpoint{4.500574in}{1.296004in}}%
\pgfpathlineto{\pgfqpoint{4.503385in}{1.313216in}}%
\pgfpathlineto{\pgfqpoint{4.506195in}{1.314415in}}%
\pgfpathlineto{\pgfqpoint{4.509006in}{1.313476in}}%
\pgfpathlineto{\pgfqpoint{4.511817in}{1.252861in}}%
\pgfpathlineto{\pgfqpoint{4.514627in}{1.272201in}}%
\pgfpathlineto{\pgfqpoint{4.517438in}{1.282703in}}%
\pgfpathlineto{\pgfqpoint{4.520249in}{1.299205in}}%
\pgfpathlineto{\pgfqpoint{4.523059in}{1.293476in}}%
\pgfpathlineto{\pgfqpoint{4.525870in}{1.292142in}}%
\pgfpathlineto{\pgfqpoint{4.528681in}{1.288008in}}%
\pgfpathlineto{\pgfqpoint{4.531491in}{1.287059in}}%
\pgfpathlineto{\pgfqpoint{4.534302in}{1.280179in}}%
\pgfpathlineto{\pgfqpoint{4.537113in}{1.289754in}}%
\pgfpathlineto{\pgfqpoint{4.539923in}{1.293782in}}%
\pgfpathlineto{\pgfqpoint{4.542734in}{1.290783in}}%
\pgfpathlineto{\pgfqpoint{4.545545in}{1.297076in}}%
\pgfpathlineto{\pgfqpoint{4.548356in}{1.294191in}}%
\pgfpathlineto{\pgfqpoint{4.551166in}{1.294313in}}%
\pgfpathlineto{\pgfqpoint{4.553977in}{1.303987in}}%
\pgfpathlineto{\pgfqpoint{4.556788in}{1.305478in}}%
\pgfpathlineto{\pgfqpoint{4.559598in}{1.290662in}}%
\pgfpathlineto{\pgfqpoint{4.562409in}{1.302021in}}%
\pgfpathlineto{\pgfqpoint{4.565220in}{1.299610in}}%
\pgfpathlineto{\pgfqpoint{4.568030in}{1.299752in}}%
\pgfpathlineto{\pgfqpoint{4.570841in}{1.300193in}}%
\pgfpathlineto{\pgfqpoint{4.573652in}{1.303719in}}%
\pgfpathlineto{\pgfqpoint{4.576462in}{1.302579in}}%
\pgfpathlineto{\pgfqpoint{4.579273in}{1.295211in}}%
\pgfpathlineto{\pgfqpoint{4.582084in}{1.293710in}}%
\pgfpathlineto{\pgfqpoint{4.584894in}{1.305858in}}%
\pgfpathlineto{\pgfqpoint{4.587705in}{1.301086in}}%
\pgfpathlineto{\pgfqpoint{4.590516in}{1.301457in}}%
\pgfpathlineto{\pgfqpoint{4.593327in}{1.300764in}}%
\pgfpathlineto{\pgfqpoint{4.596137in}{1.317842in}}%
\pgfpathlineto{\pgfqpoint{4.598948in}{1.330371in}}%
\pgfpathlineto{\pgfqpoint{4.601759in}{1.316841in}}%
\pgfpathlineto{\pgfqpoint{4.604569in}{1.309311in}}%
\pgfpathlineto{\pgfqpoint{4.607380in}{1.304192in}}%
\pgfpathlineto{\pgfqpoint{4.610191in}{1.308894in}}%
\pgfpathlineto{\pgfqpoint{4.613001in}{1.306286in}}%
\pgfpathlineto{\pgfqpoint{4.615812in}{1.306122in}}%
\pgfpathlineto{\pgfqpoint{4.618623in}{1.309355in}}%
\pgfpathlineto{\pgfqpoint{4.621433in}{1.311186in}}%
\pgfpathlineto{\pgfqpoint{4.624244in}{1.303410in}}%
\pgfpathlineto{\pgfqpoint{4.627055in}{1.313338in}}%
\pgfpathlineto{\pgfqpoint{4.629865in}{1.308549in}}%
\pgfpathlineto{\pgfqpoint{4.632676in}{1.307905in}}%
\pgfpathlineto{\pgfqpoint{4.635487in}{1.313257in}}%
\pgfpathlineto{\pgfqpoint{4.638298in}{1.310483in}}%
\pgfpathlineto{\pgfqpoint{4.641108in}{1.310353in}}%
\pgfpathlineto{\pgfqpoint{4.643919in}{1.312464in}}%
\pgfpathlineto{\pgfqpoint{4.646730in}{1.309045in}}%
\pgfpathlineto{\pgfqpoint{4.649540in}{1.295914in}}%
\pgfpathlineto{\pgfqpoint{4.652351in}{1.301507in}}%
\pgfpathlineto{\pgfqpoint{4.655162in}{1.303973in}}%
\pgfpathlineto{\pgfqpoint{4.657972in}{1.286224in}}%
\pgfpathlineto{\pgfqpoint{4.660783in}{1.293831in}}%
\pgfpathlineto{\pgfqpoint{4.663594in}{1.289425in}}%
\pgfpathlineto{\pgfqpoint{4.666404in}{1.292672in}}%
\pgfpathlineto{\pgfqpoint{4.669215in}{1.302301in}}%
\pgfpathlineto{\pgfqpoint{4.672026in}{1.308131in}}%
\pgfpathlineto{\pgfqpoint{4.674836in}{1.314263in}}%
\pgfpathlineto{\pgfqpoint{4.677647in}{1.312810in}}%
\pgfpathlineto{\pgfqpoint{4.680458in}{1.294852in}}%
\pgfpathlineto{\pgfqpoint{4.683268in}{1.301340in}}%
\pgfpathlineto{\pgfqpoint{4.686079in}{1.302249in}}%
\pgfpathlineto{\pgfqpoint{4.688890in}{1.312470in}}%
\pgfpathlineto{\pgfqpoint{4.691701in}{1.314692in}}%
\pgfpathlineto{\pgfqpoint{4.694511in}{1.312783in}}%
\pgfpathlineto{\pgfqpoint{4.697322in}{1.309146in}}%
\pgfpathlineto{\pgfqpoint{4.700133in}{1.312252in}}%
\pgfpathlineto{\pgfqpoint{4.702943in}{1.303712in}}%
\pgfpathlineto{\pgfqpoint{4.705754in}{1.299744in}}%
\pgfpathlineto{\pgfqpoint{4.708565in}{1.304935in}}%
\pgfpathlineto{\pgfqpoint{4.711375in}{1.323940in}}%
\pgfpathlineto{\pgfqpoint{4.714186in}{1.321187in}}%
\pgfpathlineto{\pgfqpoint{4.716997in}{1.308043in}}%
\pgfpathlineto{\pgfqpoint{4.719807in}{1.298415in}}%
\pgfpathlineto{\pgfqpoint{4.722618in}{1.295511in}}%
\pgfpathlineto{\pgfqpoint{4.725429in}{1.288463in}}%
\pgfpathlineto{\pgfqpoint{4.728239in}{1.293176in}}%
\pgfpathlineto{\pgfqpoint{4.731050in}{1.304490in}}%
\pgfpathlineto{\pgfqpoint{4.733861in}{1.306742in}}%
\pgfpathlineto{\pgfqpoint{4.736672in}{1.302172in}}%
\pgfpathlineto{\pgfqpoint{4.739482in}{1.303777in}}%
\pgfpathlineto{\pgfqpoint{4.742293in}{1.297204in}}%
\pgfpathlineto{\pgfqpoint{4.745104in}{1.297310in}}%
\pgfpathlineto{\pgfqpoint{4.747914in}{1.312570in}}%
\pgfpathlineto{\pgfqpoint{4.750725in}{1.309416in}}%
\pgfpathlineto{\pgfqpoint{4.753536in}{1.311134in}}%
\pgfpathlineto{\pgfqpoint{4.756346in}{1.332134in}}%
\pgfpathlineto{\pgfqpoint{4.759157in}{1.340581in}}%
\pgfpathlineto{\pgfqpoint{4.761968in}{1.311628in}}%
\pgfpathlineto{\pgfqpoint{4.764778in}{1.309397in}}%
\pgfpathlineto{\pgfqpoint{4.767589in}{1.326306in}}%
\pgfpathlineto{\pgfqpoint{4.770400in}{1.322748in}}%
\pgfpathlineto{\pgfqpoint{4.773210in}{1.312083in}}%
\pgfpathlineto{\pgfqpoint{4.776021in}{1.318384in}}%
\pgfpathlineto{\pgfqpoint{4.778832in}{1.315222in}}%
\pgfpathlineto{\pgfqpoint{4.781643in}{1.323058in}}%
\pgfpathlineto{\pgfqpoint{4.784453in}{1.298819in}}%
\pgfpathlineto{\pgfqpoint{4.787264in}{1.308645in}}%
\pgfpathlineto{\pgfqpoint{4.790075in}{1.315098in}}%
\pgfpathlineto{\pgfqpoint{4.792885in}{1.297437in}}%
\pgfpathlineto{\pgfqpoint{4.795696in}{1.327059in}}%
\pgfpathlineto{\pgfqpoint{4.798507in}{1.331977in}}%
\pgfpathlineto{\pgfqpoint{4.801317in}{1.332646in}}%
\pgfpathlineto{\pgfqpoint{4.804128in}{1.318345in}}%
\pgfpathlineto{\pgfqpoint{4.806939in}{1.327565in}}%
\pgfpathlineto{\pgfqpoint{4.809749in}{1.314692in}}%
\pgfpathlineto{\pgfqpoint{4.812560in}{1.299217in}}%
\pgfpathlineto{\pgfqpoint{4.815371in}{1.300174in}}%
\pgfpathlineto{\pgfqpoint{4.818181in}{1.294821in}}%
\pgfpathlineto{\pgfqpoint{4.820992in}{1.304479in}}%
\pgfpathlineto{\pgfqpoint{4.823803in}{1.295114in}}%
\pgfpathlineto{\pgfqpoint{4.826614in}{1.303031in}}%
\pgfpathlineto{\pgfqpoint{4.829424in}{1.297407in}}%
\pgfpathlineto{\pgfqpoint{4.832235in}{1.294736in}}%
\pgfpathlineto{\pgfqpoint{4.835046in}{1.303434in}}%
\pgfpathlineto{\pgfqpoint{4.837856in}{1.288475in}}%
\pgfpathlineto{\pgfqpoint{4.840667in}{1.287741in}}%
\pgfpathlineto{\pgfqpoint{4.843478in}{1.301645in}}%
\pgfpathlineto{\pgfqpoint{4.846288in}{1.295905in}}%
\pgfpathlineto{\pgfqpoint{4.849099in}{1.318567in}}%
\pgfpathlineto{\pgfqpoint{4.851910in}{1.323425in}}%
\pgfpathlineto{\pgfqpoint{4.854720in}{1.316451in}}%
\pgfpathlineto{\pgfqpoint{4.857531in}{1.300195in}}%
\pgfpathlineto{\pgfqpoint{4.860342in}{1.300484in}}%
\pgfpathlineto{\pgfqpoint{4.863152in}{1.314544in}}%
\pgfpathlineto{\pgfqpoint{4.865963in}{1.311935in}}%
\pgfpathlineto{\pgfqpoint{4.868774in}{1.306370in}}%
\pgfpathlineto{\pgfqpoint{4.871584in}{1.312643in}}%
\pgfpathlineto{\pgfqpoint{4.874395in}{1.318392in}}%
\pgfpathlineto{\pgfqpoint{4.877206in}{1.336895in}}%
\pgfpathlineto{\pgfqpoint{4.880017in}{1.323432in}}%
\pgfpathlineto{\pgfqpoint{4.882827in}{1.299375in}}%
\pgfpathlineto{\pgfqpoint{4.885638in}{1.295904in}}%
\pgfpathlineto{\pgfqpoint{4.888449in}{1.292521in}}%
\pgfpathlineto{\pgfqpoint{4.891259in}{1.314731in}}%
\pgfpathlineto{\pgfqpoint{4.894070in}{1.311750in}}%
\pgfpathlineto{\pgfqpoint{4.896881in}{1.305234in}}%
\pgfpathlineto{\pgfqpoint{4.899691in}{1.299113in}}%
\pgfpathlineto{\pgfqpoint{4.902502in}{1.316217in}}%
\pgfpathlineto{\pgfqpoint{4.905313in}{1.316695in}}%
\pgfpathlineto{\pgfqpoint{4.908123in}{1.315947in}}%
\pgfpathlineto{\pgfqpoint{4.910934in}{1.311739in}}%
\pgfpathlineto{\pgfqpoint{4.913745in}{1.314359in}}%
\pgfpathlineto{\pgfqpoint{4.916555in}{1.316580in}}%
\pgfpathlineto{\pgfqpoint{4.919366in}{1.320259in}}%
\pgfpathlineto{\pgfqpoint{4.922177in}{1.315631in}}%
\pgfpathlineto{\pgfqpoint{4.924988in}{1.327839in}}%
\pgfpathlineto{\pgfqpoint{4.927798in}{1.323504in}}%
\pgfpathlineto{\pgfqpoint{4.930609in}{1.305425in}}%
\pgfpathlineto{\pgfqpoint{4.933420in}{1.311700in}}%
\pgfpathlineto{\pgfqpoint{4.936230in}{1.293585in}}%
\pgfpathlineto{\pgfqpoint{4.939041in}{1.301251in}}%
\pgfpathlineto{\pgfqpoint{4.941852in}{1.311795in}}%
\pgfpathlineto{\pgfqpoint{4.944662in}{1.311600in}}%
\pgfpathlineto{\pgfqpoint{4.947473in}{1.323880in}}%
\pgfpathlineto{\pgfqpoint{4.950284in}{1.328762in}}%
\pgfpathlineto{\pgfqpoint{4.953094in}{1.319464in}}%
\pgfpathlineto{\pgfqpoint{4.955905in}{1.325302in}}%
\pgfpathlineto{\pgfqpoint{4.958716in}{1.319950in}}%
\pgfpathlineto{\pgfqpoint{4.961526in}{1.317522in}}%
\pgfpathlineto{\pgfqpoint{4.964337in}{1.349225in}}%
\pgfpathlineto{\pgfqpoint{4.967148in}{1.341453in}}%
\pgfpathlineto{\pgfqpoint{4.969959in}{1.323641in}}%
\pgfpathlineto{\pgfqpoint{4.972769in}{1.314408in}}%
\pgfpathlineto{\pgfqpoint{4.975580in}{1.308609in}}%
\pgfpathlineto{\pgfqpoint{4.978391in}{1.310283in}}%
\pgfpathlineto{\pgfqpoint{4.981201in}{1.311978in}}%
\pgfpathlineto{\pgfqpoint{4.984012in}{1.318320in}}%
\pgfpathlineto{\pgfqpoint{4.986823in}{1.308328in}}%
\pgfpathlineto{\pgfqpoint{4.989633in}{1.306640in}}%
\pgfpathlineto{\pgfqpoint{4.992444in}{1.308199in}}%
\pgfpathlineto{\pgfqpoint{4.995255in}{1.308978in}}%
\pgfpathlineto{\pgfqpoint{4.998065in}{1.299774in}}%
\pgfpathlineto{\pgfqpoint{5.000876in}{1.310726in}}%
\pgfpathlineto{\pgfqpoint{5.003687in}{1.308060in}}%
\pgfpathlineto{\pgfqpoint{5.006497in}{1.316793in}}%
\pgfpathlineto{\pgfqpoint{5.009308in}{1.327605in}}%
\pgfpathlineto{\pgfqpoint{5.012119in}{1.323282in}}%
\pgfpathlineto{\pgfqpoint{5.014930in}{1.332434in}}%
\pgfpathlineto{\pgfqpoint{5.017740in}{1.323317in}}%
\pgfpathlineto{\pgfqpoint{5.020551in}{1.323604in}}%
\pgfpathlineto{\pgfqpoint{5.023362in}{1.313212in}}%
\pgfpathlineto{\pgfqpoint{5.026172in}{1.303887in}}%
\pgfpathlineto{\pgfqpoint{5.028983in}{1.306355in}}%
\pgfpathlineto{\pgfqpoint{5.031794in}{1.312783in}}%
\pgfpathlineto{\pgfqpoint{5.034604in}{1.312518in}}%
\pgfpathlineto{\pgfqpoint{5.037415in}{1.316809in}}%
\pgfpathlineto{\pgfqpoint{5.040226in}{1.306559in}}%
\pgfpathlineto{\pgfqpoint{5.043036in}{1.313092in}}%
\pgfpathlineto{\pgfqpoint{5.045847in}{1.312258in}}%
\pgfpathlineto{\pgfqpoint{5.048658in}{1.309789in}}%
\pgfpathlineto{\pgfqpoint{5.051468in}{1.312501in}}%
\pgfpathlineto{\pgfqpoint{5.054279in}{1.312153in}}%
\pgfpathlineto{\pgfqpoint{5.057090in}{1.319810in}}%
\pgfpathlineto{\pgfqpoint{5.059901in}{1.324483in}}%
\pgfpathlineto{\pgfqpoint{5.062711in}{1.330153in}}%
\pgfpathlineto{\pgfqpoint{5.065522in}{1.322229in}}%
\pgfpathlineto{\pgfqpoint{5.068333in}{1.342768in}}%
\pgfpathlineto{\pgfqpoint{5.071143in}{1.326386in}}%
\pgfpathlineto{\pgfqpoint{5.073954in}{1.339828in}}%
\pgfpathlineto{\pgfqpoint{5.076765in}{1.333921in}}%
\pgfpathlineto{\pgfqpoint{5.079575in}{1.314160in}}%
\pgfpathlineto{\pgfqpoint{5.082386in}{1.324823in}}%
\pgfpathlineto{\pgfqpoint{5.085197in}{1.312000in}}%
\pgfpathlineto{\pgfqpoint{5.088007in}{1.311562in}}%
\pgfpathlineto{\pgfqpoint{5.090818in}{1.304926in}}%
\pgfpathlineto{\pgfqpoint{5.093629in}{1.350904in}}%
\pgfpathlineto{\pgfqpoint{5.096439in}{1.310803in}}%
\pgfpathlineto{\pgfqpoint{5.099250in}{1.337123in}}%
\pgfpathlineto{\pgfqpoint{5.102061in}{1.371742in}}%
\pgfpathlineto{\pgfqpoint{5.104871in}{1.343204in}}%
\pgfpathlineto{\pgfqpoint{5.107682in}{1.323309in}}%
\pgfpathlineto{\pgfqpoint{5.110493in}{1.303945in}}%
\pgfpathlineto{\pgfqpoint{5.113304in}{1.296263in}}%
\pgfpathlineto{\pgfqpoint{5.116114in}{1.307565in}}%
\pgfpathlineto{\pgfqpoint{5.118925in}{1.300129in}}%
\pgfpathlineto{\pgfqpoint{5.121736in}{1.309393in}}%
\pgfpathlineto{\pgfqpoint{5.124546in}{1.318669in}}%
\pgfpathlineto{\pgfqpoint{5.127357in}{1.310246in}}%
\pgfpathlineto{\pgfqpoint{5.130168in}{1.317830in}}%
\pgfpathlineto{\pgfqpoint{5.132978in}{1.336339in}}%
\pgfpathlineto{\pgfqpoint{5.135789in}{1.333207in}}%
\pgfpathlineto{\pgfqpoint{5.138600in}{1.327434in}}%
\pgfpathlineto{\pgfqpoint{5.141410in}{1.323569in}}%
\pgfpathlineto{\pgfqpoint{5.144221in}{1.320804in}}%
\pgfpathlineto{\pgfqpoint{5.147032in}{1.310408in}}%
\pgfpathlineto{\pgfqpoint{5.149842in}{1.311282in}}%
\pgfpathlineto{\pgfqpoint{5.149842in}{2.086745in}}%
\pgfpathlineto{\pgfqpoint{5.149842in}{2.086745in}}%
\pgfpathlineto{\pgfqpoint{5.147032in}{2.086119in}}%
\pgfpathlineto{\pgfqpoint{5.144221in}{2.096669in}}%
\pgfpathlineto{\pgfqpoint{5.141410in}{2.099689in}}%
\pgfpathlineto{\pgfqpoint{5.138600in}{2.103811in}}%
\pgfpathlineto{\pgfqpoint{5.135789in}{2.109841in}}%
\pgfpathlineto{\pgfqpoint{5.132978in}{2.113223in}}%
\pgfpathlineto{\pgfqpoint{5.130168in}{2.094683in}}%
\pgfpathlineto{\pgfqpoint{5.127357in}{2.087346in}}%
\pgfpathlineto{\pgfqpoint{5.124546in}{2.095951in}}%
\pgfpathlineto{\pgfqpoint{5.121736in}{2.086910in}}%
\pgfpathlineto{\pgfqpoint{5.118925in}{2.077904in}}%
\pgfpathlineto{\pgfqpoint{5.116114in}{2.085458in}}%
\pgfpathlineto{\pgfqpoint{5.113304in}{2.074412in}}%
\pgfpathlineto{\pgfqpoint{5.110493in}{2.082171in}}%
\pgfpathlineto{\pgfqpoint{5.107682in}{2.101495in}}%
\pgfpathlineto{\pgfqpoint{5.104871in}{2.121546in}}%
\pgfpathlineto{\pgfqpoint{5.102061in}{2.150284in}}%
\pgfpathlineto{\pgfqpoint{5.099250in}{2.114234in}}%
\pgfpathlineto{\pgfqpoint{5.096439in}{2.087701in}}%
\pgfpathlineto{\pgfqpoint{5.093629in}{2.127314in}}%
\pgfpathlineto{\pgfqpoint{5.090818in}{2.080000in}}%
\pgfpathlineto{\pgfqpoint{5.088007in}{2.086787in}}%
\pgfpathlineto{\pgfqpoint{5.085197in}{2.087469in}}%
\pgfpathlineto{\pgfqpoint{5.082386in}{2.100425in}}%
\pgfpathlineto{\pgfqpoint{5.079575in}{2.089958in}}%
\pgfpathlineto{\pgfqpoint{5.076765in}{2.109763in}}%
\pgfpathlineto{\pgfqpoint{5.073954in}{2.115930in}}%
\pgfpathlineto{\pgfqpoint{5.071143in}{2.102502in}}%
\pgfpathlineto{\pgfqpoint{5.068333in}{2.119082in}}%
\pgfpathlineto{\pgfqpoint{5.065522in}{2.098347in}}%
\pgfpathlineto{\pgfqpoint{5.062711in}{2.106514in}}%
\pgfpathlineto{\pgfqpoint{5.059901in}{2.101056in}}%
\pgfpathlineto{\pgfqpoint{5.057090in}{2.096628in}}%
\pgfpathlineto{\pgfqpoint{5.054279in}{2.089215in}}%
\pgfpathlineto{\pgfqpoint{5.051468in}{2.089814in}}%
\pgfpathlineto{\pgfqpoint{5.048658in}{2.087365in}}%
\pgfpathlineto{\pgfqpoint{5.045847in}{2.090066in}}%
\pgfpathlineto{\pgfqpoint{5.043036in}{2.091153in}}%
\pgfpathlineto{\pgfqpoint{5.040226in}{2.084882in}}%
\pgfpathlineto{\pgfqpoint{5.037415in}{2.095265in}}%
\pgfpathlineto{\pgfqpoint{5.034604in}{2.091237in}}%
\pgfpathlineto{\pgfqpoint{5.031794in}{2.091759in}}%
\pgfpathlineto{\pgfqpoint{5.028983in}{2.085595in}}%
\pgfpathlineto{\pgfqpoint{5.026172in}{2.083382in}}%
\pgfpathlineto{\pgfqpoint{5.023362in}{2.092835in}}%
\pgfpathlineto{\pgfqpoint{5.020551in}{2.103419in}}%
\pgfpathlineto{\pgfqpoint{5.017740in}{2.103394in}}%
\pgfpathlineto{\pgfqpoint{5.014930in}{2.112764in}}%
\pgfpathlineto{\pgfqpoint{5.012119in}{2.103756in}}%
\pgfpathlineto{\pgfqpoint{5.009308in}{2.108346in}}%
\pgfpathlineto{\pgfqpoint{5.006497in}{2.097697in}}%
\pgfpathlineto{\pgfqpoint{5.003687in}{2.089210in}}%
\pgfpathlineto{\pgfqpoint{5.000876in}{2.092111in}}%
\pgfpathlineto{\pgfqpoint{4.998065in}{2.081414in}}%
\pgfpathlineto{\pgfqpoint{4.995255in}{2.090722in}}%
\pgfpathlineto{\pgfqpoint{4.992444in}{2.090203in}}%
\pgfpathlineto{\pgfqpoint{4.989633in}{2.088905in}}%
\pgfpathlineto{\pgfqpoint{4.986823in}{2.090835in}}%
\pgfpathlineto{\pgfqpoint{4.984012in}{2.101001in}}%
\pgfpathlineto{\pgfqpoint{4.981201in}{2.094912in}}%
\pgfpathlineto{\pgfqpoint{4.978391in}{2.093486in}}%
\pgfpathlineto{\pgfqpoint{4.975580in}{2.092080in}}%
\pgfpathlineto{\pgfqpoint{4.972769in}{2.098103in}}%
\pgfpathlineto{\pgfqpoint{4.969959in}{2.107563in}}%
\pgfpathlineto{\pgfqpoint{4.967148in}{2.125578in}}%
\pgfpathlineto{\pgfqpoint{4.964337in}{2.133597in}}%
\pgfpathlineto{\pgfqpoint{4.961526in}{2.101098in}}%
\pgfpathlineto{\pgfqpoint{4.958716in}{2.103796in}}%
\pgfpathlineto{\pgfqpoint{4.955905in}{2.109416in}}%
\pgfpathlineto{\pgfqpoint{4.953094in}{2.103803in}}%
\pgfpathlineto{\pgfqpoint{4.950284in}{2.113351in}}%
\pgfpathlineto{\pgfqpoint{4.947473in}{2.108682in}}%
\pgfpathlineto{\pgfqpoint{4.944662in}{2.096565in}}%
\pgfpathlineto{\pgfqpoint{4.941852in}{2.097029in}}%
\pgfpathlineto{\pgfqpoint{4.939041in}{2.086737in}}%
\pgfpathlineto{\pgfqpoint{4.936230in}{2.079345in}}%
\pgfpathlineto{\pgfqpoint{4.933420in}{2.097336in}}%
\pgfpathlineto{\pgfqpoint{4.930609in}{2.091331in}}%
\pgfpathlineto{\pgfqpoint{4.927798in}{2.109454in}}%
\pgfpathlineto{\pgfqpoint{4.924988in}{2.114064in}}%
\pgfpathlineto{\pgfqpoint{4.922177in}{2.101978in}}%
\pgfpathlineto{\pgfqpoint{4.919366in}{2.106873in}}%
\pgfpathlineto{\pgfqpoint{4.916555in}{2.103453in}}%
\pgfpathlineto{\pgfqpoint{4.913745in}{2.101505in}}%
\pgfpathlineto{\pgfqpoint{4.910934in}{2.099161in}}%
\pgfpathlineto{\pgfqpoint{4.908123in}{2.103630in}}%
\pgfpathlineto{\pgfqpoint{4.905313in}{2.104655in}}%
\pgfpathlineto{\pgfqpoint{4.902502in}{2.104453in}}%
\pgfpathlineto{\pgfqpoint{4.899691in}{2.087501in}}%
\pgfpathlineto{\pgfqpoint{4.896881in}{2.093801in}}%
\pgfpathlineto{\pgfqpoint{4.894070in}{2.100535in}}%
\pgfpathlineto{\pgfqpoint{4.891259in}{2.103786in}}%
\pgfpathlineto{\pgfqpoint{4.888449in}{2.081655in}}%
\pgfpathlineto{\pgfqpoint{4.885638in}{2.085201in}}%
\pgfpathlineto{\pgfqpoint{4.882827in}{2.088861in}}%
\pgfpathlineto{\pgfqpoint{4.880017in}{2.112773in}}%
\pgfpathlineto{\pgfqpoint{4.877206in}{2.126494in}}%
\pgfpathlineto{\pgfqpoint{4.874395in}{2.107834in}}%
\pgfpathlineto{\pgfqpoint{4.871584in}{2.102337in}}%
\pgfpathlineto{\pgfqpoint{4.868774in}{2.096333in}}%
\pgfpathlineto{\pgfqpoint{4.865963in}{2.102137in}}%
\pgfpathlineto{\pgfqpoint{4.863152in}{2.105022in}}%
\pgfpathlineto{\pgfqpoint{4.860342in}{2.091160in}}%
\pgfpathlineto{\pgfqpoint{4.857531in}{2.091127in}}%
\pgfpathlineto{\pgfqpoint{4.854720in}{2.107431in}}%
\pgfpathlineto{\pgfqpoint{4.851910in}{2.114678in}}%
\pgfpathlineto{\pgfqpoint{4.849099in}{2.110049in}}%
\pgfpathlineto{\pgfqpoint{4.846288in}{2.087397in}}%
\pgfpathlineto{\pgfqpoint{4.843478in}{2.093308in}}%
\pgfpathlineto{\pgfqpoint{4.840667in}{2.079669in}}%
\pgfpathlineto{\pgfqpoint{4.837856in}{2.080573in}}%
\pgfpathlineto{\pgfqpoint{4.835046in}{2.095455in}}%
\pgfpathlineto{\pgfqpoint{4.832235in}{2.087038in}}%
\pgfpathlineto{\pgfqpoint{4.829424in}{2.089916in}}%
\pgfpathlineto{\pgfqpoint{4.826614in}{2.095736in}}%
\pgfpathlineto{\pgfqpoint{4.823803in}{2.088102in}}%
\pgfpathlineto{\pgfqpoint{4.820992in}{2.097594in}}%
\pgfpathlineto{\pgfqpoint{4.818181in}{2.088212in}}%
\pgfpathlineto{\pgfqpoint{4.815371in}{2.093749in}}%
\pgfpathlineto{\pgfqpoint{4.812560in}{2.093063in}}%
\pgfpathlineto{\pgfqpoint{4.809749in}{2.108616in}}%
\pgfpathlineto{\pgfqpoint{4.806939in}{2.121732in}}%
\pgfpathlineto{\pgfqpoint{4.804128in}{2.112644in}}%
\pgfpathlineto{\pgfqpoint{4.801317in}{2.127196in}}%
\pgfpathlineto{\pgfqpoint{4.798507in}{2.126730in}}%
\pgfpathlineto{\pgfqpoint{4.795696in}{2.121965in}}%
\pgfpathlineto{\pgfqpoint{4.792885in}{2.092008in}}%
\pgfpathlineto{\pgfqpoint{4.790075in}{2.109686in}}%
\pgfpathlineto{\pgfqpoint{4.787264in}{2.103490in}}%
\pgfpathlineto{\pgfqpoint{4.784453in}{2.093925in}}%
\pgfpathlineto{\pgfqpoint{4.781643in}{2.118052in}}%
\pgfpathlineto{\pgfqpoint{4.778832in}{2.110407in}}%
\pgfpathlineto{\pgfqpoint{4.776021in}{2.113859in}}%
\pgfpathlineto{\pgfqpoint{4.773210in}{2.107797in}}%
\pgfpathlineto{\pgfqpoint{4.770400in}{2.118713in}}%
\pgfpathlineto{\pgfqpoint{4.767589in}{2.122554in}}%
\pgfpathlineto{\pgfqpoint{4.764778in}{2.105637in}}%
\pgfpathlineto{\pgfqpoint{4.761968in}{2.108155in}}%
\pgfpathlineto{\pgfqpoint{4.759157in}{2.137074in}}%
\pgfpathlineto{\pgfqpoint{4.756346in}{2.128592in}}%
\pgfpathlineto{\pgfqpoint{4.753536in}{2.107382in}}%
\pgfpathlineto{\pgfqpoint{4.750725in}{2.105954in}}%
\pgfpathlineto{\pgfqpoint{4.747914in}{2.109395in}}%
\pgfpathlineto{\pgfqpoint{4.745104in}{2.094312in}}%
\pgfpathlineto{\pgfqpoint{4.742293in}{2.094472in}}%
\pgfpathlineto{\pgfqpoint{4.739482in}{2.101249in}}%
\pgfpathlineto{\pgfqpoint{4.736672in}{2.099934in}}%
\pgfpathlineto{\pgfqpoint{4.733861in}{2.104762in}}%
\pgfpathlineto{\pgfqpoint{4.731050in}{2.102804in}}%
\pgfpathlineto{\pgfqpoint{4.728239in}{2.091757in}}%
\pgfpathlineto{\pgfqpoint{4.725429in}{2.087326in}}%
\pgfpathlineto{\pgfqpoint{4.722618in}{2.094501in}}%
\pgfpathlineto{\pgfqpoint{4.719807in}{2.097647in}}%
\pgfpathlineto{\pgfqpoint{4.716997in}{2.107464in}}%
\pgfpathlineto{\pgfqpoint{4.714186in}{2.120830in}}%
\pgfpathlineto{\pgfqpoint{4.711375in}{2.123867in}}%
\pgfpathlineto{\pgfqpoint{4.708565in}{2.104819in}}%
\pgfpathlineto{\pgfqpoint{4.705754in}{2.099921in}}%
\pgfpathlineto{\pgfqpoint{4.702943in}{2.104149in}}%
\pgfpathlineto{\pgfqpoint{4.700133in}{2.112934in}}%
\pgfpathlineto{\pgfqpoint{4.697322in}{2.110112in}}%
\pgfpathlineto{\pgfqpoint{4.694511in}{2.114043in}}%
\pgfpathlineto{\pgfqpoint{4.691701in}{2.116250in}}%
\pgfpathlineto{\pgfqpoint{4.688890in}{2.114308in}}%
\pgfpathlineto{\pgfqpoint{4.686079in}{2.104313in}}%
\pgfpathlineto{\pgfqpoint{4.683268in}{2.103700in}}%
\pgfpathlineto{\pgfqpoint{4.680458in}{2.097509in}}%
\pgfpathlineto{\pgfqpoint{4.677647in}{2.115499in}}%
\pgfpathlineto{\pgfqpoint{4.674836in}{2.117250in}}%
\pgfpathlineto{\pgfqpoint{4.672026in}{2.111370in}}%
\pgfpathlineto{\pgfqpoint{4.669215in}{2.105823in}}%
\pgfpathlineto{\pgfqpoint{4.666404in}{2.096476in}}%
\pgfpathlineto{\pgfqpoint{4.663594in}{2.093516in}}%
\pgfpathlineto{\pgfqpoint{4.660783in}{2.098128in}}%
\pgfpathlineto{\pgfqpoint{4.657972in}{2.090823in}}%
\pgfpathlineto{\pgfqpoint{4.655162in}{2.108502in}}%
\pgfpathlineto{\pgfqpoint{4.652351in}{2.106338in}}%
\pgfpathlineto{\pgfqpoint{4.649540in}{2.101045in}}%
\pgfpathlineto{\pgfqpoint{4.646730in}{2.114327in}}%
\pgfpathlineto{\pgfqpoint{4.643919in}{2.118049in}}%
\pgfpathlineto{\pgfqpoint{4.641108in}{2.116224in}}%
\pgfpathlineto{\pgfqpoint{4.638298in}{2.116654in}}%
\pgfpathlineto{\pgfqpoint{4.635487in}{2.119731in}}%
\pgfpathlineto{\pgfqpoint{4.632676in}{2.114639in}}%
\pgfpathlineto{\pgfqpoint{4.629865in}{2.115587in}}%
\pgfpathlineto{\pgfqpoint{4.627055in}{2.120680in}}%
\pgfpathlineto{\pgfqpoint{4.624244in}{2.110959in}}%
\pgfpathlineto{\pgfqpoint{4.621433in}{2.119012in}}%
\pgfpathlineto{\pgfqpoint{4.618623in}{2.117471in}}%
\pgfpathlineto{\pgfqpoint{4.615812in}{2.114528in}}%
\pgfpathlineto{\pgfqpoint{4.613001in}{2.114999in}}%
\pgfpathlineto{\pgfqpoint{4.610191in}{2.117914in}}%
\pgfpathlineto{\pgfqpoint{4.607380in}{2.113494in}}%
\pgfpathlineto{\pgfqpoint{4.604569in}{2.118911in}}%
\pgfpathlineto{\pgfqpoint{4.601759in}{2.126744in}}%
\pgfpathlineto{\pgfqpoint{4.598948in}{2.140573in}}%
\pgfpathlineto{\pgfqpoint{4.596137in}{2.127978in}}%
\pgfpathlineto{\pgfqpoint{4.593327in}{2.110925in}}%
\pgfpathlineto{\pgfqpoint{4.590516in}{2.111925in}}%
\pgfpathlineto{\pgfqpoint{4.587705in}{2.111865in}}%
\pgfpathlineto{\pgfqpoint{4.584894in}{2.116932in}}%
\pgfpathlineto{\pgfqpoint{4.582084in}{2.105015in}}%
\pgfpathlineto{\pgfqpoint{4.579273in}{2.106804in}}%
\pgfpathlineto{\pgfqpoint{4.576462in}{2.114422in}}%
\pgfpathlineto{\pgfqpoint{4.573652in}{2.115873in}}%
\pgfpathlineto{\pgfqpoint{4.570841in}{2.112651in}}%
\pgfpathlineto{\pgfqpoint{4.568030in}{2.112522in}}%
\pgfpathlineto{\pgfqpoint{4.565220in}{2.112692in}}%
\pgfpathlineto{\pgfqpoint{4.562409in}{2.115410in}}%
\pgfpathlineto{\pgfqpoint{4.559598in}{2.104312in}}%
\pgfpathlineto{\pgfqpoint{4.556788in}{2.119244in}}%
\pgfpathlineto{\pgfqpoint{4.553977in}{2.118061in}}%
\pgfpathlineto{\pgfqpoint{4.551166in}{2.108649in}}%
\pgfpathlineto{\pgfqpoint{4.548356in}{2.108833in}}%
\pgfpathlineto{\pgfqpoint{4.545545in}{2.112011in}}%
\pgfpathlineto{\pgfqpoint{4.542734in}{2.106027in}}%
\pgfpathlineto{\pgfqpoint{4.539923in}{2.109305in}}%
\pgfpathlineto{\pgfqpoint{4.537113in}{2.105593in}}%
\pgfpathlineto{\pgfqpoint{4.534302in}{2.096332in}}%
\pgfpathlineto{\pgfqpoint{4.531491in}{2.103365in}}%
\pgfpathlineto{\pgfqpoint{4.528681in}{2.104595in}}%
\pgfpathlineto{\pgfqpoint{4.525870in}{2.108988in}}%
\pgfpathlineto{\pgfqpoint{4.523059in}{2.110623in}}%
\pgfpathlineto{\pgfqpoint{4.520249in}{2.116631in}}%
\pgfpathlineto{\pgfqpoint{4.517438in}{2.100336in}}%
\pgfpathlineto{\pgfqpoint{4.514627in}{2.090152in}}%
\pgfpathlineto{\pgfqpoint{4.511817in}{2.071130in}}%
\pgfpathlineto{\pgfqpoint{4.509006in}{2.128753in}}%
\pgfpathlineto{\pgfqpoint{4.506195in}{2.129987in}}%
\pgfpathlineto{\pgfqpoint{4.503385in}{2.129061in}}%
\pgfpathlineto{\pgfqpoint{4.500574in}{2.111881in}}%
\pgfpathlineto{\pgfqpoint{4.497763in}{2.106891in}}%
\pgfpathlineto{\pgfqpoint{4.494952in}{2.112505in}}%
\pgfpathlineto{\pgfqpoint{4.492142in}{2.094993in}}%
\pgfpathlineto{\pgfqpoint{4.489331in}{2.098091in}}%
\pgfpathlineto{\pgfqpoint{4.486520in}{2.111668in}}%
\pgfpathlineto{\pgfqpoint{4.483710in}{2.114706in}}%
\pgfpathlineto{\pgfqpoint{4.480899in}{2.114803in}}%
\pgfpathlineto{\pgfqpoint{4.478088in}{2.115553in}}%
\pgfpathlineto{\pgfqpoint{4.475278in}{2.120642in}}%
\pgfpathlineto{\pgfqpoint{4.472467in}{2.122776in}}%
\pgfpathlineto{\pgfqpoint{4.469656in}{2.119250in}}%
\pgfpathlineto{\pgfqpoint{4.466846in}{2.109089in}}%
\pgfpathlineto{\pgfqpoint{4.464035in}{2.114463in}}%
\pgfpathlineto{\pgfqpoint{4.461224in}{2.101873in}}%
\pgfpathlineto{\pgfqpoint{4.458414in}{2.083136in}}%
\pgfpathlineto{\pgfqpoint{4.455603in}{2.090038in}}%
\pgfpathlineto{\pgfqpoint{4.452792in}{2.096789in}}%
\pgfpathlineto{\pgfqpoint{4.449981in}{2.100533in}}%
\pgfpathlineto{\pgfqpoint{4.447171in}{2.125536in}}%
\pgfpathlineto{\pgfqpoint{4.444360in}{2.120456in}}%
\pgfpathlineto{\pgfqpoint{4.441549in}{2.120914in}}%
\pgfpathlineto{\pgfqpoint{4.438739in}{2.119581in}}%
\pgfpathlineto{\pgfqpoint{4.435928in}{2.123557in}}%
\pgfpathlineto{\pgfqpoint{4.433117in}{2.130268in}}%
\pgfpathlineto{\pgfqpoint{4.430307in}{2.128664in}}%
\pgfpathlineto{\pgfqpoint{4.427496in}{2.126472in}}%
\pgfpathlineto{\pgfqpoint{4.424685in}{2.134728in}}%
\pgfpathlineto{\pgfqpoint{4.421875in}{2.134466in}}%
\pgfpathlineto{\pgfqpoint{4.419064in}{2.112734in}}%
\pgfpathlineto{\pgfqpoint{4.416253in}{2.116754in}}%
\pgfpathlineto{\pgfqpoint{4.413443in}{2.112598in}}%
\pgfpathlineto{\pgfqpoint{4.410632in}{2.124068in}}%
\pgfpathlineto{\pgfqpoint{4.407821in}{2.109542in}}%
\pgfpathlineto{\pgfqpoint{4.405011in}{2.100281in}}%
\pgfpathlineto{\pgfqpoint{4.402200in}{2.089015in}}%
\pgfpathlineto{\pgfqpoint{4.399389in}{2.123639in}}%
\pgfpathlineto{\pgfqpoint{4.396578in}{2.119447in}}%
\pgfpathlineto{\pgfqpoint{4.393768in}{2.155343in}}%
\pgfpathlineto{\pgfqpoint{4.390957in}{2.178972in}}%
\pgfpathlineto{\pgfqpoint{4.388146in}{2.160870in}}%
\pgfpathlineto{\pgfqpoint{4.385336in}{2.134072in}}%
\pgfpathlineto{\pgfqpoint{4.382525in}{2.141501in}}%
\pgfpathlineto{\pgfqpoint{4.379714in}{2.124972in}}%
\pgfpathlineto{\pgfqpoint{4.376904in}{2.139555in}}%
\pgfpathlineto{\pgfqpoint{4.374093in}{2.138997in}}%
\pgfpathlineto{\pgfqpoint{4.371282in}{2.104186in}}%
\pgfpathlineto{\pgfqpoint{4.368472in}{2.115474in}}%
\pgfpathlineto{\pgfqpoint{4.365661in}{2.106257in}}%
\pgfpathlineto{\pgfqpoint{4.362850in}{2.121097in}}%
\pgfpathlineto{\pgfqpoint{4.360040in}{2.159931in}}%
\pgfpathlineto{\pgfqpoint{4.357229in}{2.183070in}}%
\pgfpathlineto{\pgfqpoint{4.354418in}{2.127109in}}%
\pgfpathlineto{\pgfqpoint{4.351607in}{2.130109in}}%
\pgfpathlineto{\pgfqpoint{4.348797in}{2.101656in}}%
\pgfpathlineto{\pgfqpoint{4.345986in}{2.099829in}}%
\pgfpathlineto{\pgfqpoint{4.343175in}{2.097417in}}%
\pgfpathlineto{\pgfqpoint{4.340365in}{2.135817in}}%
\pgfpathlineto{\pgfqpoint{4.337554in}{2.136367in}}%
\pgfpathlineto{\pgfqpoint{4.334743in}{2.145756in}}%
\pgfpathlineto{\pgfqpoint{4.331933in}{2.132701in}}%
\pgfpathlineto{\pgfqpoint{4.329122in}{2.116706in}}%
\pgfpathlineto{\pgfqpoint{4.326311in}{2.120781in}}%
\pgfpathlineto{\pgfqpoint{4.323501in}{2.127145in}}%
\pgfpathlineto{\pgfqpoint{4.320690in}{2.133772in}}%
\pgfpathlineto{\pgfqpoint{4.317879in}{2.132883in}}%
\pgfpathlineto{\pgfqpoint{4.315069in}{2.110529in}}%
\pgfpathlineto{\pgfqpoint{4.312258in}{2.123542in}}%
\pgfpathlineto{\pgfqpoint{4.309447in}{2.120456in}}%
\pgfpathlineto{\pgfqpoint{4.306636in}{2.111620in}}%
\pgfpathlineto{\pgfqpoint{4.303826in}{2.134847in}}%
\pgfpathlineto{\pgfqpoint{4.301015in}{2.127153in}}%
\pgfpathlineto{\pgfqpoint{4.298204in}{2.121834in}}%
\pgfpathlineto{\pgfqpoint{4.295394in}{2.134665in}}%
\pgfpathlineto{\pgfqpoint{4.292583in}{2.131285in}}%
\pgfpathlineto{\pgfqpoint{4.289772in}{2.140093in}}%
\pgfpathlineto{\pgfqpoint{4.286962in}{2.122529in}}%
\pgfpathlineto{\pgfqpoint{4.284151in}{2.127181in}}%
\pgfpathlineto{\pgfqpoint{4.281340in}{2.117346in}}%
\pgfpathlineto{\pgfqpoint{4.278530in}{2.113947in}}%
\pgfpathlineto{\pgfqpoint{4.275719in}{2.123033in}}%
\pgfpathlineto{\pgfqpoint{4.272908in}{2.119029in}}%
\pgfpathlineto{\pgfqpoint{4.270098in}{2.119851in}}%
\pgfpathlineto{\pgfqpoint{4.267287in}{2.135381in}}%
\pgfpathlineto{\pgfqpoint{4.264476in}{2.115992in}}%
\pgfpathlineto{\pgfqpoint{4.261665in}{2.129997in}}%
\pgfpathlineto{\pgfqpoint{4.258855in}{2.140416in}}%
\pgfpathlineto{\pgfqpoint{4.256044in}{2.130925in}}%
\pgfpathlineto{\pgfqpoint{4.253233in}{2.114252in}}%
\pgfpathlineto{\pgfqpoint{4.250423in}{2.111702in}}%
\pgfpathlineto{\pgfqpoint{4.247612in}{2.121745in}}%
\pgfpathlineto{\pgfqpoint{4.244801in}{2.118867in}}%
\pgfpathlineto{\pgfqpoint{4.241991in}{2.125409in}}%
\pgfpathlineto{\pgfqpoint{4.239180in}{2.126380in}}%
\pgfpathlineto{\pgfqpoint{4.236369in}{2.138528in}}%
\pgfpathlineto{\pgfqpoint{4.233559in}{2.134919in}}%
\pgfpathlineto{\pgfqpoint{4.230748in}{2.113774in}}%
\pgfpathlineto{\pgfqpoint{4.227937in}{2.144437in}}%
\pgfpathlineto{\pgfqpoint{4.225127in}{2.115688in}}%
\pgfpathlineto{\pgfqpoint{4.222316in}{2.104017in}}%
\pgfpathlineto{\pgfqpoint{4.219505in}{2.108687in}}%
\pgfpathlineto{\pgfqpoint{4.216695in}{2.111712in}}%
\pgfpathlineto{\pgfqpoint{4.213884in}{2.120111in}}%
\pgfpathlineto{\pgfqpoint{4.211073in}{2.125961in}}%
\pgfpathlineto{\pgfqpoint{4.208262in}{2.119207in}}%
\pgfpathlineto{\pgfqpoint{4.205452in}{2.122066in}}%
\pgfpathlineto{\pgfqpoint{4.202641in}{2.126555in}}%
\pgfpathlineto{\pgfqpoint{4.199830in}{2.131811in}}%
\pgfpathlineto{\pgfqpoint{4.197020in}{2.134593in}}%
\pgfpathlineto{\pgfqpoint{4.194209in}{2.125912in}}%
\pgfpathlineto{\pgfqpoint{4.191398in}{2.118828in}}%
\pgfpathlineto{\pgfqpoint{4.188588in}{2.127655in}}%
\pgfpathlineto{\pgfqpoint{4.185777in}{2.132078in}}%
\pgfpathlineto{\pgfqpoint{4.182966in}{2.120392in}}%
\pgfpathlineto{\pgfqpoint{4.180156in}{2.120244in}}%
\pgfpathlineto{\pgfqpoint{4.177345in}{2.113508in}}%
\pgfpathlineto{\pgfqpoint{4.174534in}{2.126442in}}%
\pgfpathlineto{\pgfqpoint{4.171724in}{2.122714in}}%
\pgfpathlineto{\pgfqpoint{4.168913in}{2.129763in}}%
\pgfpathlineto{\pgfqpoint{4.166102in}{2.121025in}}%
\pgfpathlineto{\pgfqpoint{4.163291in}{2.119377in}}%
\pgfpathlineto{\pgfqpoint{4.160481in}{2.109634in}}%
\pgfpathlineto{\pgfqpoint{4.157670in}{2.114987in}}%
\pgfpathlineto{\pgfqpoint{4.154859in}{2.125658in}}%
\pgfpathlineto{\pgfqpoint{4.152049in}{2.133815in}}%
\pgfpathlineto{\pgfqpoint{4.149238in}{2.136200in}}%
\pgfpathlineto{\pgfqpoint{4.146427in}{2.144391in}}%
\pgfpathlineto{\pgfqpoint{4.143617in}{2.123625in}}%
\pgfpathlineto{\pgfqpoint{4.140806in}{2.135937in}}%
\pgfpathlineto{\pgfqpoint{4.137995in}{2.139577in}}%
\pgfpathlineto{\pgfqpoint{4.135185in}{2.133228in}}%
\pgfpathlineto{\pgfqpoint{4.132374in}{2.111115in}}%
\pgfpathlineto{\pgfqpoint{4.129563in}{2.127068in}}%
\pgfpathlineto{\pgfqpoint{4.126753in}{2.111760in}}%
\pgfpathlineto{\pgfqpoint{4.123942in}{2.116777in}}%
\pgfpathlineto{\pgfqpoint{4.121131in}{2.119130in}}%
\pgfpathlineto{\pgfqpoint{4.118320in}{2.113232in}}%
\pgfpathlineto{\pgfqpoint{4.115510in}{2.107922in}}%
\pgfpathlineto{\pgfqpoint{4.112699in}{2.112064in}}%
\pgfpathlineto{\pgfqpoint{4.109888in}{2.102042in}}%
\pgfpathlineto{\pgfqpoint{4.107078in}{2.110008in}}%
\pgfpathlineto{\pgfqpoint{4.104267in}{2.101553in}}%
\pgfpathlineto{\pgfqpoint{4.101456in}{2.133457in}}%
\pgfpathlineto{\pgfqpoint{4.098646in}{2.146112in}}%
\pgfpathlineto{\pgfqpoint{4.095835in}{2.137958in}}%
\pgfpathlineto{\pgfqpoint{4.093024in}{2.146138in}}%
\pgfpathlineto{\pgfqpoint{4.090214in}{2.161273in}}%
\pgfpathlineto{\pgfqpoint{4.087403in}{2.112564in}}%
\pgfpathlineto{\pgfqpoint{4.084592in}{2.139157in}}%
\pgfpathlineto{\pgfqpoint{4.081782in}{2.184967in}}%
\pgfpathlineto{\pgfqpoint{4.078971in}{2.172475in}}%
\pgfpathlineto{\pgfqpoint{4.076160in}{2.117011in}}%
\pgfpathlineto{\pgfqpoint{4.073349in}{2.148267in}}%
\pgfpathlineto{\pgfqpoint{4.070539in}{2.139532in}}%
\pgfpathlineto{\pgfqpoint{4.067728in}{2.144039in}}%
\pgfpathlineto{\pgfqpoint{4.064917in}{2.158256in}}%
\pgfpathlineto{\pgfqpoint{4.062107in}{2.141305in}}%
\pgfpathlineto{\pgfqpoint{4.059296in}{2.150405in}}%
\pgfpathlineto{\pgfqpoint{4.056485in}{2.150863in}}%
\pgfpathlineto{\pgfqpoint{4.053675in}{2.163819in}}%
\pgfpathlineto{\pgfqpoint{4.050864in}{2.139693in}}%
\pgfpathlineto{\pgfqpoint{4.048053in}{2.120412in}}%
\pgfpathlineto{\pgfqpoint{4.045243in}{2.107382in}}%
\pgfpathlineto{\pgfqpoint{4.042432in}{2.119144in}}%
\pgfpathlineto{\pgfqpoint{4.039621in}{2.115255in}}%
\pgfpathlineto{\pgfqpoint{4.036811in}{2.121365in}}%
\pgfpathlineto{\pgfqpoint{4.034000in}{2.116829in}}%
\pgfpathlineto{\pgfqpoint{4.031189in}{2.126366in}}%
\pgfpathlineto{\pgfqpoint{4.028378in}{2.130601in}}%
\pgfpathlineto{\pgfqpoint{4.025568in}{2.114216in}}%
\pgfpathlineto{\pgfqpoint{4.022757in}{2.121008in}}%
\pgfpathlineto{\pgfqpoint{4.019946in}{2.114933in}}%
\pgfpathlineto{\pgfqpoint{4.017136in}{2.100479in}}%
\pgfpathlineto{\pgfqpoint{4.014325in}{2.133806in}}%
\pgfpathlineto{\pgfqpoint{4.011514in}{2.124781in}}%
\pgfpathlineto{\pgfqpoint{4.008704in}{2.139425in}}%
\pgfpathlineto{\pgfqpoint{4.005893in}{2.130773in}}%
\pgfpathlineto{\pgfqpoint{4.003082in}{2.144795in}}%
\pgfpathlineto{\pgfqpoint{4.000272in}{2.128168in}}%
\pgfpathlineto{\pgfqpoint{3.997461in}{2.147825in}}%
\pgfpathlineto{\pgfqpoint{3.994650in}{2.131545in}}%
\pgfpathlineto{\pgfqpoint{3.991840in}{2.144296in}}%
\pgfpathlineto{\pgfqpoint{3.989029in}{2.111778in}}%
\pgfpathlineto{\pgfqpoint{3.986218in}{2.123404in}}%
\pgfpathlineto{\pgfqpoint{3.983408in}{2.130152in}}%
\pgfpathlineto{\pgfqpoint{3.980597in}{2.137291in}}%
\pgfpathlineto{\pgfqpoint{3.977786in}{2.145822in}}%
\pgfpathlineto{\pgfqpoint{3.974975in}{2.149303in}}%
\pgfpathlineto{\pgfqpoint{3.972165in}{2.129713in}}%
\pgfpathlineto{\pgfqpoint{3.969354in}{2.153657in}}%
\pgfpathlineto{\pgfqpoint{3.966543in}{2.153736in}}%
\pgfpathlineto{\pgfqpoint{3.963733in}{2.140236in}}%
\pgfpathlineto{\pgfqpoint{3.960922in}{2.148076in}}%
\pgfpathlineto{\pgfqpoint{3.958111in}{2.162515in}}%
\pgfpathlineto{\pgfqpoint{3.955301in}{2.155410in}}%
\pgfpathlineto{\pgfqpoint{3.952490in}{2.129540in}}%
\pgfpathlineto{\pgfqpoint{3.949679in}{2.120399in}}%
\pgfpathlineto{\pgfqpoint{3.946869in}{2.122859in}}%
\pgfpathlineto{\pgfqpoint{3.944058in}{2.103442in}}%
\pgfpathlineto{\pgfqpoint{3.941247in}{2.118080in}}%
\pgfpathlineto{\pgfqpoint{3.938437in}{2.106201in}}%
\pgfpathlineto{\pgfqpoint{3.935626in}{2.106813in}}%
\pgfpathlineto{\pgfqpoint{3.932815in}{2.120093in}}%
\pgfpathlineto{\pgfqpoint{3.930004in}{2.129395in}}%
\pgfpathlineto{\pgfqpoint{3.927194in}{2.122930in}}%
\pgfpathlineto{\pgfqpoint{3.924383in}{2.127866in}}%
\pgfpathlineto{\pgfqpoint{3.921572in}{2.111827in}}%
\pgfpathlineto{\pgfqpoint{3.918762in}{2.128269in}}%
\pgfpathlineto{\pgfqpoint{3.915951in}{2.107700in}}%
\pgfpathlineto{\pgfqpoint{3.913140in}{2.101270in}}%
\pgfpathlineto{\pgfqpoint{3.910330in}{2.110660in}}%
\pgfpathlineto{\pgfqpoint{3.907519in}{2.094306in}}%
\pgfpathlineto{\pgfqpoint{3.904708in}{2.107318in}}%
\pgfpathlineto{\pgfqpoint{3.901898in}{2.124992in}}%
\pgfpathlineto{\pgfqpoint{3.899087in}{2.120018in}}%
\pgfpathlineto{\pgfqpoint{3.896276in}{2.119341in}}%
\pgfpathlineto{\pgfqpoint{3.893466in}{2.134708in}}%
\pgfpathlineto{\pgfqpoint{3.890655in}{2.116680in}}%
\pgfpathlineto{\pgfqpoint{3.887844in}{2.109270in}}%
\pgfpathlineto{\pgfqpoint{3.885033in}{2.113269in}}%
\pgfpathlineto{\pgfqpoint{3.882223in}{2.121949in}}%
\pgfpathlineto{\pgfqpoint{3.879412in}{2.127440in}}%
\pgfpathlineto{\pgfqpoint{3.876601in}{2.134335in}}%
\pgfpathlineto{\pgfqpoint{3.873791in}{2.131014in}}%
\pgfpathlineto{\pgfqpoint{3.870980in}{2.162131in}}%
\pgfpathlineto{\pgfqpoint{3.868169in}{2.163453in}}%
\pgfpathlineto{\pgfqpoint{3.865359in}{2.177534in}}%
\pgfpathlineto{\pgfqpoint{3.862548in}{2.165149in}}%
\pgfpathlineto{\pgfqpoint{3.859737in}{2.130116in}}%
\pgfpathlineto{\pgfqpoint{3.856927in}{2.127843in}}%
\pgfpathlineto{\pgfqpoint{3.854116in}{2.115269in}}%
\pgfpathlineto{\pgfqpoint{3.851305in}{2.098208in}}%
\pgfpathlineto{\pgfqpoint{3.848495in}{2.134996in}}%
\pgfpathlineto{\pgfqpoint{3.845684in}{2.127196in}}%
\pgfpathlineto{\pgfqpoint{3.842873in}{2.110943in}}%
\pgfpathlineto{\pgfqpoint{3.840062in}{2.134201in}}%
\pgfpathlineto{\pgfqpoint{3.837252in}{2.126834in}}%
\pgfpathlineto{\pgfqpoint{3.834441in}{2.101925in}}%
\pgfpathlineto{\pgfqpoint{3.831630in}{2.123364in}}%
\pgfpathlineto{\pgfqpoint{3.828820in}{2.125535in}}%
\pgfpathlineto{\pgfqpoint{3.826009in}{2.103188in}}%
\pgfpathlineto{\pgfqpoint{3.823198in}{2.113898in}}%
\pgfpathlineto{\pgfqpoint{3.820388in}{2.120923in}}%
\pgfpathlineto{\pgfqpoint{3.817577in}{2.162125in}}%
\pgfpathlineto{\pgfqpoint{3.814766in}{2.134996in}}%
\pgfpathlineto{\pgfqpoint{3.811956in}{2.179844in}}%
\pgfpathlineto{\pgfqpoint{3.809145in}{2.213949in}}%
\pgfpathlineto{\pgfqpoint{3.806334in}{2.164407in}}%
\pgfpathlineto{\pgfqpoint{3.803524in}{2.126622in}}%
\pgfpathlineto{\pgfqpoint{3.800713in}{2.107382in}}%
\pgfpathlineto{\pgfqpoint{3.797902in}{2.120246in}}%
\pgfpathlineto{\pgfqpoint{3.795092in}{2.091214in}}%
\pgfpathlineto{\pgfqpoint{3.792281in}{2.068618in}}%
\pgfpathlineto{\pgfqpoint{3.789470in}{2.159308in}}%
\pgfpathlineto{\pgfqpoint{3.786659in}{2.140332in}}%
\pgfpathlineto{\pgfqpoint{3.783849in}{2.126109in}}%
\pgfpathlineto{\pgfqpoint{3.781038in}{2.130337in}}%
\pgfpathlineto{\pgfqpoint{3.778227in}{2.112025in}}%
\pgfpathlineto{\pgfqpoint{3.775417in}{2.130791in}}%
\pgfpathlineto{\pgfqpoint{3.772606in}{2.153646in}}%
\pgfpathlineto{\pgfqpoint{3.769795in}{2.153015in}}%
\pgfpathlineto{\pgfqpoint{3.766985in}{2.157931in}}%
\pgfpathlineto{\pgfqpoint{3.764174in}{2.129205in}}%
\pgfpathlineto{\pgfqpoint{3.761363in}{2.146720in}}%
\pgfpathlineto{\pgfqpoint{3.758553in}{2.121744in}}%
\pgfpathlineto{\pgfqpoint{3.755742in}{2.137315in}}%
\pgfpathlineto{\pgfqpoint{3.752931in}{2.164498in}}%
\pgfpathlineto{\pgfqpoint{3.750121in}{2.164716in}}%
\pgfpathlineto{\pgfqpoint{3.747310in}{2.151654in}}%
\pgfpathlineto{\pgfqpoint{3.744499in}{2.142384in}}%
\pgfpathlineto{\pgfqpoint{3.741688in}{2.161939in}}%
\pgfpathlineto{\pgfqpoint{3.738878in}{2.138456in}}%
\pgfpathlineto{\pgfqpoint{3.736067in}{2.127453in}}%
\pgfpathlineto{\pgfqpoint{3.733256in}{2.110084in}}%
\pgfpathlineto{\pgfqpoint{3.730446in}{2.117658in}}%
\pgfpathlineto{\pgfqpoint{3.727635in}{2.119993in}}%
\pgfpathlineto{\pgfqpoint{3.724824in}{2.111332in}}%
\pgfpathlineto{\pgfqpoint{3.722014in}{2.129872in}}%
\pgfpathlineto{\pgfqpoint{3.719203in}{2.136307in}}%
\pgfpathlineto{\pgfqpoint{3.716392in}{2.159056in}}%
\pgfpathlineto{\pgfqpoint{3.713582in}{2.122722in}}%
\pgfpathlineto{\pgfqpoint{3.710771in}{2.097964in}}%
\pgfpathlineto{\pgfqpoint{3.707960in}{2.109889in}}%
\pgfpathlineto{\pgfqpoint{3.705150in}{2.139010in}}%
\pgfpathlineto{\pgfqpoint{3.702339in}{2.161564in}}%
\pgfpathlineto{\pgfqpoint{3.699528in}{2.136036in}}%
\pgfpathlineto{\pgfqpoint{3.696717in}{2.146410in}}%
\pgfpathlineto{\pgfqpoint{3.693907in}{2.131481in}}%
\pgfpathlineto{\pgfqpoint{3.691096in}{2.124519in}}%
\pgfpathlineto{\pgfqpoint{3.688285in}{2.106968in}}%
\pgfpathlineto{\pgfqpoint{3.685475in}{2.141133in}}%
\pgfpathlineto{\pgfqpoint{3.682664in}{2.133487in}}%
\pgfpathlineto{\pgfqpoint{3.679853in}{2.119850in}}%
\pgfpathlineto{\pgfqpoint{3.677043in}{2.135658in}}%
\pgfpathlineto{\pgfqpoint{3.674232in}{2.123850in}}%
\pgfpathlineto{\pgfqpoint{3.671421in}{2.128936in}}%
\pgfpathlineto{\pgfqpoint{3.668611in}{2.126207in}}%
\pgfpathlineto{\pgfqpoint{3.665800in}{2.120337in}}%
\pgfpathlineto{\pgfqpoint{3.662989in}{2.116471in}}%
\pgfpathlineto{\pgfqpoint{3.660179in}{2.113237in}}%
\pgfpathlineto{\pgfqpoint{3.657368in}{2.101671in}}%
\pgfpathlineto{\pgfqpoint{3.654557in}{2.122632in}}%
\pgfpathlineto{\pgfqpoint{3.651746in}{2.119889in}}%
\pgfpathlineto{\pgfqpoint{3.648936in}{2.133010in}}%
\pgfpathlineto{\pgfqpoint{3.646125in}{2.132273in}}%
\pgfpathlineto{\pgfqpoint{3.643314in}{2.117415in}}%
\pgfpathlineto{\pgfqpoint{3.640504in}{2.107607in}}%
\pgfpathlineto{\pgfqpoint{3.637693in}{2.129995in}}%
\pgfpathlineto{\pgfqpoint{3.634882in}{2.117860in}}%
\pgfpathlineto{\pgfqpoint{3.632072in}{2.104214in}}%
\pgfpathlineto{\pgfqpoint{3.629261in}{2.111865in}}%
\pgfpathlineto{\pgfqpoint{3.626450in}{2.114332in}}%
\pgfpathlineto{\pgfqpoint{3.623640in}{2.158518in}}%
\pgfpathlineto{\pgfqpoint{3.620829in}{2.128658in}}%
\pgfpathlineto{\pgfqpoint{3.618018in}{2.113058in}}%
\pgfpathlineto{\pgfqpoint{3.615208in}{2.101277in}}%
\pgfpathlineto{\pgfqpoint{3.612397in}{2.115701in}}%
\pgfpathlineto{\pgfqpoint{3.609586in}{2.099690in}}%
\pgfpathlineto{\pgfqpoint{3.606776in}{2.109220in}}%
\pgfpathlineto{\pgfqpoint{3.603965in}{2.114800in}}%
\pgfpathlineto{\pgfqpoint{3.601154in}{2.128351in}}%
\pgfpathlineto{\pgfqpoint{3.598343in}{2.114738in}}%
\pgfpathlineto{\pgfqpoint{3.595533in}{2.097410in}}%
\pgfpathlineto{\pgfqpoint{3.592722in}{2.103542in}}%
\pgfpathlineto{\pgfqpoint{3.589911in}{2.106772in}}%
\pgfpathlineto{\pgfqpoint{3.587101in}{2.120170in}}%
\pgfpathlineto{\pgfqpoint{3.584290in}{2.103742in}}%
\pgfpathlineto{\pgfqpoint{3.581479in}{2.095290in}}%
\pgfpathlineto{\pgfqpoint{3.578669in}{2.101389in}}%
\pgfpathlineto{\pgfqpoint{3.575858in}{2.093539in}}%
\pgfpathlineto{\pgfqpoint{3.573047in}{2.093477in}}%
\pgfpathlineto{\pgfqpoint{3.570237in}{2.102744in}}%
\pgfpathlineto{\pgfqpoint{3.567426in}{2.091020in}}%
\pgfpathlineto{\pgfqpoint{3.564615in}{2.107947in}}%
\pgfpathlineto{\pgfqpoint{3.561805in}{2.115263in}}%
\pgfpathlineto{\pgfqpoint{3.558994in}{2.118414in}}%
\pgfpathlineto{\pgfqpoint{3.556183in}{2.139740in}}%
\pgfpathlineto{\pgfqpoint{3.553372in}{2.173410in}}%
\pgfpathlineto{\pgfqpoint{3.550562in}{2.119051in}}%
\pgfpathlineto{\pgfqpoint{3.547751in}{2.127033in}}%
\pgfpathlineto{\pgfqpoint{3.544940in}{2.115056in}}%
\pgfpathlineto{\pgfqpoint{3.542130in}{2.125991in}}%
\pgfpathlineto{\pgfqpoint{3.539319in}{2.117089in}}%
\pgfpathlineto{\pgfqpoint{3.536508in}{2.134083in}}%
\pgfpathlineto{\pgfqpoint{3.533698in}{2.116108in}}%
\pgfpathlineto{\pgfqpoint{3.530887in}{2.113194in}}%
\pgfpathlineto{\pgfqpoint{3.528076in}{2.118582in}}%
\pgfpathlineto{\pgfqpoint{3.525266in}{2.126725in}}%
\pgfpathlineto{\pgfqpoint{3.522455in}{2.114026in}}%
\pgfpathlineto{\pgfqpoint{3.519644in}{2.120421in}}%
\pgfpathlineto{\pgfqpoint{3.516834in}{2.131543in}}%
\pgfpathlineto{\pgfqpoint{3.514023in}{2.117420in}}%
\pgfpathlineto{\pgfqpoint{3.511212in}{2.139920in}}%
\pgfpathlineto{\pgfqpoint{3.508401in}{2.124051in}}%
\pgfpathlineto{\pgfqpoint{3.505591in}{2.138169in}}%
\pgfpathlineto{\pgfqpoint{3.502780in}{2.155286in}}%
\pgfpathlineto{\pgfqpoint{3.499969in}{2.125123in}}%
\pgfpathlineto{\pgfqpoint{3.497159in}{2.108693in}}%
\pgfpathlineto{\pgfqpoint{3.494348in}{2.105307in}}%
\pgfpathlineto{\pgfqpoint{3.491537in}{2.118604in}}%
\pgfpathlineto{\pgfqpoint{3.488727in}{2.168401in}}%
\pgfpathlineto{\pgfqpoint{3.485916in}{2.177164in}}%
\pgfpathlineto{\pgfqpoint{3.483105in}{2.158324in}}%
\pgfpathlineto{\pgfqpoint{3.480295in}{2.120063in}}%
\pgfpathlineto{\pgfqpoint{3.477484in}{2.113337in}}%
\pgfpathlineto{\pgfqpoint{3.474673in}{2.110895in}}%
\pgfpathlineto{\pgfqpoint{3.471863in}{2.109886in}}%
\pgfpathlineto{\pgfqpoint{3.469052in}{2.110854in}}%
\pgfpathlineto{\pgfqpoint{3.466241in}{2.114440in}}%
\pgfpathlineto{\pgfqpoint{3.463430in}{2.122139in}}%
\pgfpathlineto{\pgfqpoint{3.460620in}{2.132042in}}%
\pgfpathlineto{\pgfqpoint{3.457809in}{2.118420in}}%
\pgfpathlineto{\pgfqpoint{3.454998in}{2.123257in}}%
\pgfpathlineto{\pgfqpoint{3.452188in}{2.139982in}}%
\pgfpathlineto{\pgfqpoint{3.449377in}{2.122466in}}%
\pgfpathlineto{\pgfqpoint{3.446566in}{2.119510in}}%
\pgfpathlineto{\pgfqpoint{3.443756in}{2.110324in}}%
\pgfpathlineto{\pgfqpoint{3.440945in}{2.115372in}}%
\pgfpathlineto{\pgfqpoint{3.438134in}{2.092671in}}%
\pgfpathlineto{\pgfqpoint{3.435324in}{2.086007in}}%
\pgfpathlineto{\pgfqpoint{3.432513in}{2.094656in}}%
\pgfpathlineto{\pgfqpoint{3.429702in}{2.100908in}}%
\pgfpathlineto{\pgfqpoint{3.426892in}{2.072925in}}%
\pgfpathlineto{\pgfqpoint{3.424081in}{2.111250in}}%
\pgfpathlineto{\pgfqpoint{3.421270in}{2.105332in}}%
\pgfpathlineto{\pgfqpoint{3.418459in}{2.099761in}}%
\pgfpathlineto{\pgfqpoint{3.415649in}{2.079197in}}%
\pgfpathlineto{\pgfqpoint{3.412838in}{2.097307in}}%
\pgfpathlineto{\pgfqpoint{3.410027in}{2.095676in}}%
\pgfpathlineto{\pgfqpoint{3.407217in}{2.098099in}}%
\pgfpathlineto{\pgfqpoint{3.404406in}{2.089017in}}%
\pgfpathlineto{\pgfqpoint{3.401595in}{2.088765in}}%
\pgfpathlineto{\pgfqpoint{3.398785in}{2.099311in}}%
\pgfpathlineto{\pgfqpoint{3.395974in}{2.122184in}}%
\pgfpathlineto{\pgfqpoint{3.393163in}{2.133436in}}%
\pgfpathlineto{\pgfqpoint{3.390353in}{2.118803in}}%
\pgfpathlineto{\pgfqpoint{3.387542in}{2.112977in}}%
\pgfpathlineto{\pgfqpoint{3.384731in}{2.109408in}}%
\pgfpathlineto{\pgfqpoint{3.381921in}{2.115257in}}%
\pgfpathlineto{\pgfqpoint{3.379110in}{2.130826in}}%
\pgfpathlineto{\pgfqpoint{3.376299in}{2.148228in}}%
\pgfpathlineto{\pgfqpoint{3.373489in}{2.117706in}}%
\pgfpathlineto{\pgfqpoint{3.370678in}{2.122050in}}%
\pgfpathlineto{\pgfqpoint{3.367867in}{2.119942in}}%
\pgfpathlineto{\pgfqpoint{3.365056in}{2.109840in}}%
\pgfpathlineto{\pgfqpoint{3.362246in}{2.114277in}}%
\pgfpathlineto{\pgfqpoint{3.359435in}{2.122348in}}%
\pgfpathlineto{\pgfqpoint{3.356624in}{2.107543in}}%
\pgfpathlineto{\pgfqpoint{3.353814in}{2.119564in}}%
\pgfpathlineto{\pgfqpoint{3.351003in}{2.115630in}}%
\pgfpathlineto{\pgfqpoint{3.348192in}{2.124519in}}%
\pgfpathlineto{\pgfqpoint{3.345382in}{2.112002in}}%
\pgfpathlineto{\pgfqpoint{3.342571in}{2.103300in}}%
\pgfpathlineto{\pgfqpoint{3.339760in}{2.097213in}}%
\pgfpathlineto{\pgfqpoint{3.336950in}{2.122468in}}%
\pgfpathlineto{\pgfqpoint{3.334139in}{2.131299in}}%
\pgfpathlineto{\pgfqpoint{3.331328in}{2.123456in}}%
\pgfpathlineto{\pgfqpoint{3.328518in}{2.129993in}}%
\pgfpathlineto{\pgfqpoint{3.325707in}{2.120301in}}%
\pgfpathlineto{\pgfqpoint{3.322896in}{2.119748in}}%
\pgfpathlineto{\pgfqpoint{3.320085in}{2.119243in}}%
\pgfpathlineto{\pgfqpoint{3.317275in}{2.129060in}}%
\pgfpathlineto{\pgfqpoint{3.314464in}{2.115455in}}%
\pgfpathlineto{\pgfqpoint{3.311653in}{2.115481in}}%
\pgfpathlineto{\pgfqpoint{3.308843in}{2.133539in}}%
\pgfpathlineto{\pgfqpoint{3.306032in}{2.118090in}}%
\pgfpathlineto{\pgfqpoint{3.303221in}{2.122997in}}%
\pgfpathlineto{\pgfqpoint{3.300411in}{2.116601in}}%
\pgfpathlineto{\pgfqpoint{3.297600in}{2.107708in}}%
\pgfpathlineto{\pgfqpoint{3.294789in}{2.105683in}}%
\pgfpathlineto{\pgfqpoint{3.291979in}{2.108058in}}%
\pgfpathlineto{\pgfqpoint{3.289168in}{2.093979in}}%
\pgfpathlineto{\pgfqpoint{3.286357in}{2.109026in}}%
\pgfpathlineto{\pgfqpoint{3.283547in}{2.105956in}}%
\pgfpathlineto{\pgfqpoint{3.280736in}{2.093558in}}%
\pgfpathlineto{\pgfqpoint{3.277925in}{2.073777in}}%
\pgfpathlineto{\pgfqpoint{3.275114in}{2.107991in}}%
\pgfpathlineto{\pgfqpoint{3.272304in}{2.123488in}}%
\pgfpathlineto{\pgfqpoint{3.269493in}{2.120376in}}%
\pgfpathlineto{\pgfqpoint{3.266682in}{2.130203in}}%
\pgfpathlineto{\pgfqpoint{3.263872in}{2.132500in}}%
\pgfpathlineto{\pgfqpoint{3.261061in}{2.133784in}}%
\pgfpathlineto{\pgfqpoint{3.258250in}{2.112037in}}%
\pgfpathlineto{\pgfqpoint{3.255440in}{2.121637in}}%
\pgfpathlineto{\pgfqpoint{3.252629in}{2.116174in}}%
\pgfpathlineto{\pgfqpoint{3.249818in}{2.106876in}}%
\pgfpathlineto{\pgfqpoint{3.247008in}{2.095461in}}%
\pgfpathlineto{\pgfqpoint{3.244197in}{2.077640in}}%
\pgfpathlineto{\pgfqpoint{3.241386in}{2.112655in}}%
\pgfpathlineto{\pgfqpoint{3.238576in}{2.125607in}}%
\pgfpathlineto{\pgfqpoint{3.235765in}{2.135741in}}%
\pgfpathlineto{\pgfqpoint{3.232954in}{2.119014in}}%
\pgfpathlineto{\pgfqpoint{3.230143in}{2.119570in}}%
\pgfpathlineto{\pgfqpoint{3.227333in}{2.124484in}}%
\pgfpathlineto{\pgfqpoint{3.224522in}{2.126623in}}%
\pgfpathlineto{\pgfqpoint{3.221711in}{2.114225in}}%
\pgfpathlineto{\pgfqpoint{3.218901in}{2.109709in}}%
\pgfpathlineto{\pgfqpoint{3.216090in}{2.103479in}}%
\pgfpathlineto{\pgfqpoint{3.213279in}{2.111529in}}%
\pgfpathlineto{\pgfqpoint{3.210469in}{2.122164in}}%
\pgfpathlineto{\pgfqpoint{3.207658in}{2.122219in}}%
\pgfpathlineto{\pgfqpoint{3.204847in}{2.127191in}}%
\pgfpathlineto{\pgfqpoint{3.202037in}{2.126570in}}%
\pgfpathlineto{\pgfqpoint{3.199226in}{2.124893in}}%
\pgfpathlineto{\pgfqpoint{3.196415in}{2.130767in}}%
\pgfpathlineto{\pgfqpoint{3.193605in}{2.135878in}}%
\pgfpathlineto{\pgfqpoint{3.190794in}{2.141799in}}%
\pgfpathlineto{\pgfqpoint{3.187983in}{2.119793in}}%
\pgfpathlineto{\pgfqpoint{3.185173in}{2.119950in}}%
\pgfpathlineto{\pgfqpoint{3.182362in}{2.111572in}}%
\pgfpathlineto{\pgfqpoint{3.179551in}{2.120519in}}%
\pgfpathlineto{\pgfqpoint{3.176740in}{2.117796in}}%
\pgfpathlineto{\pgfqpoint{3.173930in}{2.134762in}}%
\pgfpathlineto{\pgfqpoint{3.171119in}{2.115445in}}%
\pgfpathlineto{\pgfqpoint{3.168308in}{2.134682in}}%
\pgfpathlineto{\pgfqpoint{3.165498in}{2.116834in}}%
\pgfpathlineto{\pgfqpoint{3.162687in}{2.139587in}}%
\pgfpathlineto{\pgfqpoint{3.159876in}{2.146989in}}%
\pgfpathlineto{\pgfqpoint{3.157066in}{2.125883in}}%
\pgfpathlineto{\pgfqpoint{3.154255in}{2.139200in}}%
\pgfpathlineto{\pgfqpoint{3.151444in}{2.122435in}}%
\pgfpathlineto{\pgfqpoint{3.148634in}{2.124957in}}%
\pgfpathlineto{\pgfqpoint{3.145823in}{2.118379in}}%
\pgfpathlineto{\pgfqpoint{3.143012in}{2.101304in}}%
\pgfpathlineto{\pgfqpoint{3.140202in}{2.125880in}}%
\pgfpathlineto{\pgfqpoint{3.137391in}{2.115860in}}%
\pgfpathlineto{\pgfqpoint{3.134580in}{2.116959in}}%
\pgfpathlineto{\pgfqpoint{3.131769in}{2.117453in}}%
\pgfpathlineto{\pgfqpoint{3.128959in}{2.112724in}}%
\pgfpathlineto{\pgfqpoint{3.126148in}{2.108295in}}%
\pgfpathlineto{\pgfqpoint{3.123337in}{2.120716in}}%
\pgfpathlineto{\pgfqpoint{3.120527in}{2.120078in}}%
\pgfpathlineto{\pgfqpoint{3.117716in}{2.110449in}}%
\pgfpathlineto{\pgfqpoint{3.114905in}{2.110744in}}%
\pgfpathlineto{\pgfqpoint{3.112095in}{2.098884in}}%
\pgfpathlineto{\pgfqpoint{3.109284in}{2.111286in}}%
\pgfpathlineto{\pgfqpoint{3.106473in}{2.122142in}}%
\pgfpathlineto{\pgfqpoint{3.103663in}{2.117810in}}%
\pgfpathlineto{\pgfqpoint{3.100852in}{2.106519in}}%
\pgfpathlineto{\pgfqpoint{3.098041in}{2.079451in}}%
\pgfpathlineto{\pgfqpoint{3.095231in}{2.088205in}}%
\pgfpathlineto{\pgfqpoint{3.092420in}{2.096785in}}%
\pgfpathlineto{\pgfqpoint{3.089609in}{2.108591in}}%
\pgfpathlineto{\pgfqpoint{3.086798in}{2.104263in}}%
\pgfpathlineto{\pgfqpoint{3.083988in}{2.134615in}}%
\pgfpathlineto{\pgfqpoint{3.081177in}{2.153768in}}%
\pgfpathlineto{\pgfqpoint{3.078366in}{2.145657in}}%
\pgfpathlineto{\pgfqpoint{3.075556in}{2.124473in}}%
\pgfpathlineto{\pgfqpoint{3.072745in}{2.115681in}}%
\pgfpathlineto{\pgfqpoint{3.069934in}{2.117394in}}%
\pgfpathlineto{\pgfqpoint{3.067124in}{2.117327in}}%
\pgfpathlineto{\pgfqpoint{3.064313in}{2.118124in}}%
\pgfpathlineto{\pgfqpoint{3.061502in}{2.128300in}}%
\pgfpathlineto{\pgfqpoint{3.058692in}{2.144846in}}%
\pgfpathlineto{\pgfqpoint{3.055881in}{2.144213in}}%
\pgfpathlineto{\pgfqpoint{3.053070in}{2.122773in}}%
\pgfpathlineto{\pgfqpoint{3.050260in}{2.128640in}}%
\pgfpathlineto{\pgfqpoint{3.047449in}{2.129238in}}%
\pgfpathlineto{\pgfqpoint{3.044638in}{2.108305in}}%
\pgfpathlineto{\pgfqpoint{3.041827in}{2.121006in}}%
\pgfpathlineto{\pgfqpoint{3.039017in}{2.142048in}}%
\pgfpathlineto{\pgfqpoint{3.036206in}{2.144086in}}%
\pgfpathlineto{\pgfqpoint{3.033395in}{2.119074in}}%
\pgfpathlineto{\pgfqpoint{3.030585in}{2.134365in}}%
\pgfpathlineto{\pgfqpoint{3.027774in}{2.126938in}}%
\pgfpathlineto{\pgfqpoint{3.024963in}{2.124053in}}%
\pgfpathlineto{\pgfqpoint{3.022153in}{2.117103in}}%
\pgfpathlineto{\pgfqpoint{3.019342in}{2.106945in}}%
\pgfpathlineto{\pgfqpoint{3.016531in}{2.111381in}}%
\pgfpathlineto{\pgfqpoint{3.013721in}{2.110358in}}%
\pgfpathlineto{\pgfqpoint{3.010910in}{2.118207in}}%
\pgfpathlineto{\pgfqpoint{3.008099in}{2.099347in}}%
\pgfpathlineto{\pgfqpoint{3.005289in}{2.114101in}}%
\pgfpathlineto{\pgfqpoint{3.002478in}{2.144799in}}%
\pgfpathlineto{\pgfqpoint{2.999667in}{2.138060in}}%
\pgfpathlineto{\pgfqpoint{2.996856in}{2.147839in}}%
\pgfpathlineto{\pgfqpoint{2.994046in}{2.120748in}}%
\pgfpathlineto{\pgfqpoint{2.991235in}{2.128561in}}%
\pgfpathlineto{\pgfqpoint{2.988424in}{2.116009in}}%
\pgfpathlineto{\pgfqpoint{2.985614in}{2.117780in}}%
\pgfpathlineto{\pgfqpoint{2.982803in}{2.112117in}}%
\pgfpathlineto{\pgfqpoint{2.979992in}{2.118582in}}%
\pgfpathlineto{\pgfqpoint{2.977182in}{2.111891in}}%
\pgfpathlineto{\pgfqpoint{2.974371in}{2.113954in}}%
\pgfpathlineto{\pgfqpoint{2.971560in}{2.119030in}}%
\pgfpathlineto{\pgfqpoint{2.968750in}{2.115336in}}%
\pgfpathlineto{\pgfqpoint{2.965939in}{2.115075in}}%
\pgfpathlineto{\pgfqpoint{2.963128in}{2.108484in}}%
\pgfpathlineto{\pgfqpoint{2.960318in}{2.114770in}}%
\pgfpathlineto{\pgfqpoint{2.957507in}{2.110650in}}%
\pgfpathlineto{\pgfqpoint{2.954696in}{2.117983in}}%
\pgfpathlineto{\pgfqpoint{2.951886in}{2.125254in}}%
\pgfpathlineto{\pgfqpoint{2.949075in}{2.126048in}}%
\pgfpathlineto{\pgfqpoint{2.946264in}{2.125435in}}%
\pgfpathlineto{\pgfqpoint{2.943453in}{2.131364in}}%
\pgfpathlineto{\pgfqpoint{2.940643in}{2.120249in}}%
\pgfpathlineto{\pgfqpoint{2.937832in}{2.108570in}}%
\pgfpathlineto{\pgfqpoint{2.935021in}{2.123188in}}%
\pgfpathlineto{\pgfqpoint{2.932211in}{2.118247in}}%
\pgfpathlineto{\pgfqpoint{2.929400in}{2.099732in}}%
\pgfpathlineto{\pgfqpoint{2.926589in}{2.093310in}}%
\pgfpathlineto{\pgfqpoint{2.923779in}{2.074377in}}%
\pgfpathlineto{\pgfqpoint{2.920968in}{2.069141in}}%
\pgfpathlineto{\pgfqpoint{2.918157in}{2.049407in}}%
\pgfpathlineto{\pgfqpoint{2.915347in}{2.021416in}}%
\pgfpathlineto{\pgfqpoint{2.912536in}{1.989788in}}%
\pgfpathlineto{\pgfqpoint{2.909725in}{2.097278in}}%
\pgfpathlineto{\pgfqpoint{2.906915in}{2.078941in}}%
\pgfpathlineto{\pgfqpoint{2.904104in}{2.093460in}}%
\pgfpathlineto{\pgfqpoint{2.901293in}{2.088303in}}%
\pgfpathlineto{\pgfqpoint{2.898482in}{2.076840in}}%
\pgfpathlineto{\pgfqpoint{2.895672in}{2.090741in}}%
\pgfpathlineto{\pgfqpoint{2.892861in}{2.070610in}}%
\pgfpathlineto{\pgfqpoint{2.890050in}{2.092580in}}%
\pgfpathlineto{\pgfqpoint{2.887240in}{2.100599in}}%
\pgfpathlineto{\pgfqpoint{2.884429in}{2.121004in}}%
\pgfpathlineto{\pgfqpoint{2.881618in}{2.147884in}}%
\pgfpathlineto{\pgfqpoint{2.878808in}{2.142014in}}%
\pgfpathlineto{\pgfqpoint{2.875997in}{2.140320in}}%
\pgfpathlineto{\pgfqpoint{2.873186in}{2.145414in}}%
\pgfpathlineto{\pgfqpoint{2.870376in}{2.133919in}}%
\pgfpathlineto{\pgfqpoint{2.867565in}{2.102129in}}%
\pgfpathlineto{\pgfqpoint{2.864754in}{2.133706in}}%
\pgfpathlineto{\pgfqpoint{2.861944in}{2.111117in}}%
\pgfpathlineto{\pgfqpoint{2.859133in}{2.109984in}}%
\pgfpathlineto{\pgfqpoint{2.856322in}{2.126953in}}%
\pgfpathlineto{\pgfqpoint{2.853511in}{2.129617in}}%
\pgfpathlineto{\pgfqpoint{2.850701in}{2.109782in}}%
\pgfpathlineto{\pgfqpoint{2.847890in}{2.126941in}}%
\pgfpathlineto{\pgfqpoint{2.845079in}{2.123913in}}%
\pgfpathlineto{\pgfqpoint{2.842269in}{2.136108in}}%
\pgfpathlineto{\pgfqpoint{2.839458in}{2.115196in}}%
\pgfpathlineto{\pgfqpoint{2.836647in}{2.129932in}}%
\pgfpathlineto{\pgfqpoint{2.833837in}{2.124223in}}%
\pgfpathlineto{\pgfqpoint{2.831026in}{2.109907in}}%
\pgfpathlineto{\pgfqpoint{2.828215in}{2.108079in}}%
\pgfpathlineto{\pgfqpoint{2.825405in}{2.113091in}}%
\pgfpathlineto{\pgfqpoint{2.822594in}{2.100862in}}%
\pgfpathlineto{\pgfqpoint{2.819783in}{2.116129in}}%
\pgfpathlineto{\pgfqpoint{2.816973in}{2.120889in}}%
\pgfpathlineto{\pgfqpoint{2.814162in}{2.116136in}}%
\pgfpathlineto{\pgfqpoint{2.811351in}{2.103350in}}%
\pgfpathlineto{\pgfqpoint{2.808540in}{2.118269in}}%
\pgfpathlineto{\pgfqpoint{2.805730in}{2.109184in}}%
\pgfpathlineto{\pgfqpoint{2.802919in}{2.118323in}}%
\pgfpathlineto{\pgfqpoint{2.800108in}{2.118835in}}%
\pgfpathlineto{\pgfqpoint{2.797298in}{2.115104in}}%
\pgfpathlineto{\pgfqpoint{2.794487in}{2.116423in}}%
\pgfpathlineto{\pgfqpoint{2.791676in}{2.130849in}}%
\pgfpathlineto{\pgfqpoint{2.788866in}{2.120570in}}%
\pgfpathlineto{\pgfqpoint{2.786055in}{2.104064in}}%
\pgfpathlineto{\pgfqpoint{2.783244in}{2.104049in}}%
\pgfpathlineto{\pgfqpoint{2.780434in}{2.105948in}}%
\pgfpathlineto{\pgfqpoint{2.777623in}{2.102873in}}%
\pgfpathlineto{\pgfqpoint{2.774812in}{2.101841in}}%
\pgfpathlineto{\pgfqpoint{2.772002in}{2.095259in}}%
\pgfpathlineto{\pgfqpoint{2.769191in}{2.097841in}}%
\pgfpathlineto{\pgfqpoint{2.766380in}{2.097374in}}%
\pgfpathlineto{\pgfqpoint{2.763570in}{2.120599in}}%
\pgfpathlineto{\pgfqpoint{2.760759in}{2.105688in}}%
\pgfpathlineto{\pgfqpoint{2.757948in}{2.109608in}}%
\pgfpathlineto{\pgfqpoint{2.755137in}{2.125566in}}%
\pgfpathlineto{\pgfqpoint{2.752327in}{2.126235in}}%
\pgfpathlineto{\pgfqpoint{2.749516in}{2.129247in}}%
\pgfpathlineto{\pgfqpoint{2.746705in}{2.141926in}}%
\pgfpathlineto{\pgfqpoint{2.743895in}{2.137112in}}%
\pgfpathlineto{\pgfqpoint{2.741084in}{2.142574in}}%
\pgfpathlineto{\pgfqpoint{2.738273in}{2.137704in}}%
\pgfpathlineto{\pgfqpoint{2.735463in}{2.143984in}}%
\pgfpathlineto{\pgfqpoint{2.732652in}{2.154572in}}%
\pgfpathlineto{\pgfqpoint{2.729841in}{2.141390in}}%
\pgfpathlineto{\pgfqpoint{2.727031in}{2.149726in}}%
\pgfpathlineto{\pgfqpoint{2.724220in}{2.151116in}}%
\pgfpathlineto{\pgfqpoint{2.721409in}{2.154783in}}%
\pgfpathlineto{\pgfqpoint{2.718599in}{2.103900in}}%
\pgfpathlineto{\pgfqpoint{2.715788in}{2.110372in}}%
\pgfpathlineto{\pgfqpoint{2.712977in}{2.107614in}}%
\pgfpathlineto{\pgfqpoint{2.710166in}{2.118244in}}%
\pgfpathlineto{\pgfqpoint{2.707356in}{2.110884in}}%
\pgfpathlineto{\pgfqpoint{2.704545in}{2.123669in}}%
\pgfpathlineto{\pgfqpoint{2.701734in}{2.091657in}}%
\pgfpathlineto{\pgfqpoint{2.698924in}{2.090649in}}%
\pgfpathlineto{\pgfqpoint{2.696113in}{2.088257in}}%
\pgfpathlineto{\pgfqpoint{2.693302in}{2.107498in}}%
\pgfpathlineto{\pgfqpoint{2.690492in}{2.114184in}}%
\pgfpathlineto{\pgfqpoint{2.687681in}{2.108476in}}%
\pgfpathlineto{\pgfqpoint{2.684870in}{2.111552in}}%
\pgfpathlineto{\pgfqpoint{2.682060in}{2.099104in}}%
\pgfpathlineto{\pgfqpoint{2.679249in}{2.095319in}}%
\pgfpathlineto{\pgfqpoint{2.676438in}{2.100185in}}%
\pgfpathlineto{\pgfqpoint{2.673628in}{2.098855in}}%
\pgfpathlineto{\pgfqpoint{2.670817in}{2.117001in}}%
\pgfpathlineto{\pgfqpoint{2.668006in}{2.126150in}}%
\pgfpathlineto{\pgfqpoint{2.665195in}{2.129697in}}%
\pgfpathlineto{\pgfqpoint{2.662385in}{2.130646in}}%
\pgfpathlineto{\pgfqpoint{2.659574in}{2.141952in}}%
\pgfpathlineto{\pgfqpoint{2.656763in}{2.134424in}}%
\pgfpathlineto{\pgfqpoint{2.653953in}{2.136801in}}%
\pgfpathlineto{\pgfqpoint{2.651142in}{2.133997in}}%
\pgfpathlineto{\pgfqpoint{2.648331in}{2.129451in}}%
\pgfpathlineto{\pgfqpoint{2.645521in}{2.133597in}}%
\pgfpathlineto{\pgfqpoint{2.642710in}{2.139899in}}%
\pgfpathlineto{\pgfqpoint{2.639899in}{2.135482in}}%
\pgfpathlineto{\pgfqpoint{2.637089in}{2.133953in}}%
\pgfpathlineto{\pgfqpoint{2.634278in}{2.129105in}}%
\pgfpathlineto{\pgfqpoint{2.631467in}{2.118004in}}%
\pgfpathlineto{\pgfqpoint{2.628657in}{2.127695in}}%
\pgfpathlineto{\pgfqpoint{2.625846in}{2.124596in}}%
\pgfpathlineto{\pgfqpoint{2.623035in}{2.129637in}}%
\pgfpathlineto{\pgfqpoint{2.620224in}{2.135496in}}%
\pgfpathlineto{\pgfqpoint{2.617414in}{2.138116in}}%
\pgfpathlineto{\pgfqpoint{2.614603in}{2.124112in}}%
\pgfpathlineto{\pgfqpoint{2.611792in}{2.113143in}}%
\pgfpathlineto{\pgfqpoint{2.608982in}{2.110319in}}%
\pgfpathlineto{\pgfqpoint{2.606171in}{2.111681in}}%
\pgfpathlineto{\pgfqpoint{2.603360in}{2.102178in}}%
\pgfpathlineto{\pgfqpoint{2.600550in}{2.112703in}}%
\pgfpathlineto{\pgfqpoint{2.597739in}{2.128495in}}%
\pgfpathlineto{\pgfqpoint{2.594928in}{2.122583in}}%
\pgfpathlineto{\pgfqpoint{2.592118in}{2.136624in}}%
\pgfpathlineto{\pgfqpoint{2.589307in}{2.119756in}}%
\pgfpathlineto{\pgfqpoint{2.586496in}{2.121092in}}%
\pgfpathlineto{\pgfqpoint{2.583686in}{2.135878in}}%
\pgfpathlineto{\pgfqpoint{2.580875in}{2.119004in}}%
\pgfpathlineto{\pgfqpoint{2.578064in}{2.111316in}}%
\pgfpathlineto{\pgfqpoint{2.575253in}{2.109096in}}%
\pgfpathlineto{\pgfqpoint{2.572443in}{2.109922in}}%
\pgfpathlineto{\pgfqpoint{2.569632in}{2.103825in}}%
\pgfpathlineto{\pgfqpoint{2.566821in}{2.112014in}}%
\pgfpathlineto{\pgfqpoint{2.564011in}{2.132002in}}%
\pgfpathlineto{\pgfqpoint{2.561200in}{2.118506in}}%
\pgfpathlineto{\pgfqpoint{2.558389in}{2.135025in}}%
\pgfpathlineto{\pgfqpoint{2.555579in}{2.123430in}}%
\pgfpathlineto{\pgfqpoint{2.552768in}{2.146149in}}%
\pgfpathlineto{\pgfqpoint{2.549957in}{2.149653in}}%
\pgfpathlineto{\pgfqpoint{2.547147in}{2.163467in}}%
\pgfpathlineto{\pgfqpoint{2.544336in}{2.188940in}}%
\pgfpathlineto{\pgfqpoint{2.541525in}{2.125746in}}%
\pgfpathlineto{\pgfqpoint{2.538715in}{2.127423in}}%
\pgfpathlineto{\pgfqpoint{2.535904in}{2.199170in}}%
\pgfpathlineto{\pgfqpoint{2.533093in}{2.202296in}}%
\pgfpathlineto{\pgfqpoint{2.530283in}{2.125628in}}%
\pgfpathlineto{\pgfqpoint{2.527472in}{2.203901in}}%
\pgfpathlineto{\pgfqpoint{2.524661in}{2.129759in}}%
\pgfpathlineto{\pgfqpoint{2.521850in}{2.149022in}}%
\pgfpathlineto{\pgfqpoint{2.519040in}{2.187152in}}%
\pgfpathlineto{\pgfqpoint{2.516229in}{2.174011in}}%
\pgfpathlineto{\pgfqpoint{2.513418in}{2.213199in}}%
\pgfpathlineto{\pgfqpoint{2.510608in}{2.225144in}}%
\pgfpathlineto{\pgfqpoint{2.507797in}{2.227300in}}%
\pgfpathlineto{\pgfqpoint{2.504986in}{2.218824in}}%
\pgfpathlineto{\pgfqpoint{2.502176in}{2.191287in}}%
\pgfpathlineto{\pgfqpoint{2.499365in}{2.193011in}}%
\pgfpathlineto{\pgfqpoint{2.496554in}{2.193276in}}%
\pgfpathlineto{\pgfqpoint{2.493744in}{2.188366in}}%
\pgfpathlineto{\pgfqpoint{2.490933in}{2.203290in}}%
\pgfpathlineto{\pgfqpoint{2.488122in}{2.188014in}}%
\pgfpathlineto{\pgfqpoint{2.485312in}{2.188719in}}%
\pgfpathlineto{\pgfqpoint{2.482501in}{2.176663in}}%
\pgfpathlineto{\pgfqpoint{2.479690in}{2.165241in}}%
\pgfpathlineto{\pgfqpoint{2.476879in}{2.151778in}}%
\pgfpathlineto{\pgfqpoint{2.474069in}{2.158248in}}%
\pgfpathlineto{\pgfqpoint{2.471258in}{2.147953in}}%
\pgfpathlineto{\pgfqpoint{2.468447in}{2.134151in}}%
\pgfpathlineto{\pgfqpoint{2.465637in}{2.144847in}}%
\pgfpathlineto{\pgfqpoint{2.462826in}{2.157839in}}%
\pgfpathlineto{\pgfqpoint{2.460015in}{2.157536in}}%
\pgfpathlineto{\pgfqpoint{2.457205in}{2.129940in}}%
\pgfpathlineto{\pgfqpoint{2.454394in}{2.123440in}}%
\pgfpathlineto{\pgfqpoint{2.451583in}{2.130068in}}%
\pgfpathlineto{\pgfqpoint{2.448773in}{2.128513in}}%
\pgfpathlineto{\pgfqpoint{2.445962in}{2.135307in}}%
\pgfpathlineto{\pgfqpoint{2.443151in}{2.139374in}}%
\pgfpathlineto{\pgfqpoint{2.440341in}{2.124591in}}%
\pgfpathlineto{\pgfqpoint{2.437530in}{2.142657in}}%
\pgfpathlineto{\pgfqpoint{2.434719in}{2.119316in}}%
\pgfpathlineto{\pgfqpoint{2.431908in}{2.116448in}}%
\pgfpathlineto{\pgfqpoint{2.429098in}{2.116119in}}%
\pgfpathlineto{\pgfqpoint{2.426287in}{2.109895in}}%
\pgfpathlineto{\pgfqpoint{2.423476in}{2.112661in}}%
\pgfpathlineto{\pgfqpoint{2.420666in}{2.124482in}}%
\pgfpathlineto{\pgfqpoint{2.417855in}{2.122937in}}%
\pgfpathlineto{\pgfqpoint{2.415044in}{2.119915in}}%
\pgfpathlineto{\pgfqpoint{2.412234in}{2.108195in}}%
\pgfpathlineto{\pgfqpoint{2.409423in}{2.106875in}}%
\pgfpathlineto{\pgfqpoint{2.406612in}{2.114677in}}%
\pgfpathlineto{\pgfqpoint{2.403802in}{2.112419in}}%
\pgfpathlineto{\pgfqpoint{2.400991in}{2.126965in}}%
\pgfpathlineto{\pgfqpoint{2.398180in}{2.134253in}}%
\pgfpathlineto{\pgfqpoint{2.395370in}{2.126975in}}%
\pgfpathlineto{\pgfqpoint{2.392559in}{2.138289in}}%
\pgfpathlineto{\pgfqpoint{2.389748in}{2.152920in}}%
\pgfpathlineto{\pgfqpoint{2.386937in}{2.157007in}}%
\pgfpathlineto{\pgfqpoint{2.384127in}{2.161506in}}%
\pgfpathlineto{\pgfqpoint{2.381316in}{2.151795in}}%
\pgfpathlineto{\pgfqpoint{2.378505in}{2.122589in}}%
\pgfpathlineto{\pgfqpoint{2.375695in}{2.139210in}}%
\pgfpathlineto{\pgfqpoint{2.372884in}{2.118606in}}%
\pgfpathlineto{\pgfqpoint{2.370073in}{2.139588in}}%
\pgfpathlineto{\pgfqpoint{2.367263in}{2.114361in}}%
\pgfpathlineto{\pgfqpoint{2.364452in}{2.080772in}}%
\pgfpathlineto{\pgfqpoint{2.361641in}{2.059702in}}%
\pgfpathlineto{\pgfqpoint{2.358831in}{2.061861in}}%
\pgfpathlineto{\pgfqpoint{2.356020in}{2.057940in}}%
\pgfpathlineto{\pgfqpoint{2.353209in}{2.096685in}}%
\pgfpathlineto{\pgfqpoint{2.350399in}{2.086422in}}%
\pgfpathlineto{\pgfqpoint{2.347588in}{2.089321in}}%
\pgfpathlineto{\pgfqpoint{2.344777in}{2.104770in}}%
\pgfpathlineto{\pgfqpoint{2.341967in}{2.093348in}}%
\pgfpathlineto{\pgfqpoint{2.339156in}{2.088659in}}%
\pgfpathlineto{\pgfqpoint{2.336345in}{2.085953in}}%
\pgfpathlineto{\pgfqpoint{2.333534in}{2.087415in}}%
\pgfpathlineto{\pgfqpoint{2.330724in}{2.093577in}}%
\pgfpathlineto{\pgfqpoint{2.327913in}{2.086302in}}%
\pgfpathlineto{\pgfqpoint{2.325102in}{2.085914in}}%
\pgfpathlineto{\pgfqpoint{2.322292in}{2.077076in}}%
\pgfpathlineto{\pgfqpoint{2.319481in}{2.082908in}}%
\pgfpathlineto{\pgfqpoint{2.316670in}{2.087000in}}%
\pgfpathlineto{\pgfqpoint{2.313860in}{2.083197in}}%
\pgfpathlineto{\pgfqpoint{2.311049in}{2.088122in}}%
\pgfpathlineto{\pgfqpoint{2.308238in}{2.089586in}}%
\pgfpathlineto{\pgfqpoint{2.305428in}{2.097519in}}%
\pgfpathlineto{\pgfqpoint{2.302617in}{2.095734in}}%
\pgfpathlineto{\pgfqpoint{2.299806in}{2.097576in}}%
\pgfpathlineto{\pgfqpoint{2.296996in}{2.105089in}}%
\pgfpathlineto{\pgfqpoint{2.294185in}{2.129162in}}%
\pgfpathlineto{\pgfqpoint{2.291374in}{2.129516in}}%
\pgfpathlineto{\pgfqpoint{2.288563in}{2.147118in}}%
\pgfpathlineto{\pgfqpoint{2.285753in}{2.134164in}}%
\pgfpathlineto{\pgfqpoint{2.282942in}{2.161308in}}%
\pgfpathlineto{\pgfqpoint{2.280131in}{2.150234in}}%
\pgfpathlineto{\pgfqpoint{2.277321in}{2.148195in}}%
\pgfpathlineto{\pgfqpoint{2.274510in}{2.148202in}}%
\pgfpathlineto{\pgfqpoint{2.271699in}{2.142171in}}%
\pgfpathlineto{\pgfqpoint{2.268889in}{2.146581in}}%
\pgfpathlineto{\pgfqpoint{2.266078in}{2.128336in}}%
\pgfpathlineto{\pgfqpoint{2.263267in}{2.135520in}}%
\pgfpathlineto{\pgfqpoint{2.260457in}{2.131720in}}%
\pgfpathlineto{\pgfqpoint{2.257646in}{2.132842in}}%
\pgfpathlineto{\pgfqpoint{2.254835in}{2.139614in}}%
\pgfpathlineto{\pgfqpoint{2.252025in}{2.137800in}}%
\pgfpathlineto{\pgfqpoint{2.249214in}{2.138926in}}%
\pgfpathlineto{\pgfqpoint{2.246403in}{2.154366in}}%
\pgfpathlineto{\pgfqpoint{2.243592in}{2.157631in}}%
\pgfpathlineto{\pgfqpoint{2.240782in}{2.144229in}}%
\pgfpathlineto{\pgfqpoint{2.237971in}{2.132561in}}%
\pgfpathlineto{\pgfqpoint{2.235160in}{2.140539in}}%
\pgfpathlineto{\pgfqpoint{2.232350in}{2.140892in}}%
\pgfpathlineto{\pgfqpoint{2.229539in}{2.155153in}}%
\pgfpathlineto{\pgfqpoint{2.226728in}{2.149314in}}%
\pgfpathlineto{\pgfqpoint{2.223918in}{2.152871in}}%
\pgfpathlineto{\pgfqpoint{2.221107in}{2.163123in}}%
\pgfpathlineto{\pgfqpoint{2.218296in}{2.147106in}}%
\pgfpathlineto{\pgfqpoint{2.215486in}{2.154164in}}%
\pgfpathlineto{\pgfqpoint{2.212675in}{2.158515in}}%
\pgfpathlineto{\pgfqpoint{2.209864in}{2.143249in}}%
\pgfpathlineto{\pgfqpoint{2.207054in}{2.155634in}}%
\pgfpathlineto{\pgfqpoint{2.204243in}{2.116074in}}%
\pgfpathlineto{\pgfqpoint{2.201432in}{2.112943in}}%
\pgfpathlineto{\pgfqpoint{2.198621in}{2.117406in}}%
\pgfpathlineto{\pgfqpoint{2.195811in}{2.116325in}}%
\pgfpathlineto{\pgfqpoint{2.193000in}{2.115201in}}%
\pgfpathlineto{\pgfqpoint{2.190189in}{2.133147in}}%
\pgfpathlineto{\pgfqpoint{2.187379in}{2.126807in}}%
\pgfpathlineto{\pgfqpoint{2.184568in}{2.125678in}}%
\pgfpathlineto{\pgfqpoint{2.181757in}{2.122223in}}%
\pgfpathlineto{\pgfqpoint{2.178947in}{2.131151in}}%
\pgfpathlineto{\pgfqpoint{2.176136in}{2.142398in}}%
\pgfpathlineto{\pgfqpoint{2.173325in}{2.161469in}}%
\pgfpathlineto{\pgfqpoint{2.170515in}{2.150224in}}%
\pgfpathlineto{\pgfqpoint{2.167704in}{2.155107in}}%
\pgfpathlineto{\pgfqpoint{2.164893in}{2.168191in}}%
\pgfpathlineto{\pgfqpoint{2.162083in}{2.188737in}}%
\pgfpathlineto{\pgfqpoint{2.159272in}{2.195188in}}%
\pgfpathlineto{\pgfqpoint{2.156461in}{2.178563in}}%
\pgfpathlineto{\pgfqpoint{2.153651in}{2.143035in}}%
\pgfpathlineto{\pgfqpoint{2.150840in}{2.165242in}}%
\pgfpathlineto{\pgfqpoint{2.148029in}{2.139083in}}%
\pgfpathlineto{\pgfqpoint{2.145218in}{2.126795in}}%
\pgfpathlineto{\pgfqpoint{2.142408in}{2.143980in}}%
\pgfpathlineto{\pgfqpoint{2.139597in}{2.129667in}}%
\pgfpathlineto{\pgfqpoint{2.136786in}{2.124386in}}%
\pgfpathlineto{\pgfqpoint{2.133976in}{2.137979in}}%
\pgfpathlineto{\pgfqpoint{2.131165in}{2.126858in}}%
\pgfpathlineto{\pgfqpoint{2.128354in}{2.099205in}}%
\pgfpathlineto{\pgfqpoint{2.125544in}{2.063975in}}%
\pgfpathlineto{\pgfqpoint{2.122733in}{2.077324in}}%
\pgfpathlineto{\pgfqpoint{2.119922in}{2.070095in}}%
\pgfpathlineto{\pgfqpoint{2.117112in}{2.071207in}}%
\pgfpathlineto{\pgfqpoint{2.114301in}{2.075564in}}%
\pgfpathlineto{\pgfqpoint{2.111490in}{2.064020in}}%
\pgfpathlineto{\pgfqpoint{2.108680in}{2.086606in}}%
\pgfpathlineto{\pgfqpoint{2.105869in}{2.061481in}}%
\pgfpathlineto{\pgfqpoint{2.103058in}{2.059115in}}%
\pgfpathlineto{\pgfqpoint{2.100247in}{2.107462in}}%
\pgfpathlineto{\pgfqpoint{2.097437in}{2.125212in}}%
\pgfpathlineto{\pgfqpoint{2.094626in}{2.129487in}}%
\pgfpathlineto{\pgfqpoint{2.091815in}{2.129827in}}%
\pgfpathlineto{\pgfqpoint{2.089005in}{2.119070in}}%
\pgfpathlineto{\pgfqpoint{2.086194in}{2.148249in}}%
\pgfpathlineto{\pgfqpoint{2.083383in}{2.141045in}}%
\pgfpathlineto{\pgfqpoint{2.080573in}{2.145997in}}%
\pgfpathlineto{\pgfqpoint{2.077762in}{2.163175in}}%
\pgfpathlineto{\pgfqpoint{2.074951in}{2.160822in}}%
\pgfpathlineto{\pgfqpoint{2.072141in}{2.117120in}}%
\pgfpathlineto{\pgfqpoint{2.069330in}{2.114988in}}%
\pgfpathlineto{\pgfqpoint{2.066519in}{2.115500in}}%
\pgfpathlineto{\pgfqpoint{2.063709in}{2.167205in}}%
\pgfpathlineto{\pgfqpoint{2.060898in}{2.158121in}}%
\pgfpathlineto{\pgfqpoint{2.058087in}{2.186759in}}%
\pgfpathlineto{\pgfqpoint{2.055276in}{2.193881in}}%
\pgfpathlineto{\pgfqpoint{2.052466in}{2.161873in}}%
\pgfpathlineto{\pgfqpoint{2.049655in}{2.173342in}}%
\pgfpathlineto{\pgfqpoint{2.046844in}{2.179316in}}%
\pgfpathlineto{\pgfqpoint{2.044034in}{2.189588in}}%
\pgfpathlineto{\pgfqpoint{2.041223in}{2.188747in}}%
\pgfpathlineto{\pgfqpoint{2.038412in}{2.184718in}}%
\pgfpathlineto{\pgfqpoint{2.035602in}{2.173982in}}%
\pgfpathlineto{\pgfqpoint{2.032791in}{2.161306in}}%
\pgfpathlineto{\pgfqpoint{2.029980in}{2.163314in}}%
\pgfpathlineto{\pgfqpoint{2.027170in}{2.192289in}}%
\pgfpathlineto{\pgfqpoint{2.024359in}{2.208124in}}%
\pgfpathlineto{\pgfqpoint{2.021548in}{2.092513in}}%
\pgfpathlineto{\pgfqpoint{2.018738in}{2.087894in}}%
\pgfpathlineto{\pgfqpoint{2.015927in}{2.078961in}}%
\pgfpathlineto{\pgfqpoint{2.013116in}{2.067175in}}%
\pgfpathlineto{\pgfqpoint{2.010305in}{2.109901in}}%
\pgfpathlineto{\pgfqpoint{2.007495in}{2.097259in}}%
\pgfpathlineto{\pgfqpoint{2.004684in}{2.078831in}}%
\pgfpathlineto{\pgfqpoint{2.001873in}{2.085206in}}%
\pgfpathlineto{\pgfqpoint{1.999063in}{2.076823in}}%
\pgfpathlineto{\pgfqpoint{1.996252in}{2.083130in}}%
\pgfpathlineto{\pgfqpoint{1.993441in}{2.083318in}}%
\pgfpathlineto{\pgfqpoint{1.990631in}{2.076564in}}%
\pgfpathlineto{\pgfqpoint{1.987820in}{2.072247in}}%
\pgfpathlineto{\pgfqpoint{1.985009in}{2.074266in}}%
\pgfpathlineto{\pgfqpoint{1.982199in}{2.091854in}}%
\pgfpathlineto{\pgfqpoint{1.979388in}{2.087288in}}%
\pgfpathlineto{\pgfqpoint{1.976577in}{2.080904in}}%
\pgfpathlineto{\pgfqpoint{1.973767in}{2.062388in}}%
\pgfpathlineto{\pgfqpoint{1.970956in}{2.081248in}}%
\pgfpathlineto{\pgfqpoint{1.968145in}{2.090436in}}%
\pgfpathlineto{\pgfqpoint{1.965334in}{2.088778in}}%
\pgfpathlineto{\pgfqpoint{1.962524in}{2.102476in}}%
\pgfpathlineto{\pgfqpoint{1.959713in}{2.091265in}}%
\pgfpathlineto{\pgfqpoint{1.956902in}{2.099259in}}%
\pgfpathlineto{\pgfqpoint{1.954092in}{2.111047in}}%
\pgfpathlineto{\pgfqpoint{1.951281in}{2.123669in}}%
\pgfpathlineto{\pgfqpoint{1.948470in}{2.114304in}}%
\pgfpathlineto{\pgfqpoint{1.945660in}{2.120596in}}%
\pgfpathlineto{\pgfqpoint{1.942849in}{2.091513in}}%
\pgfpathlineto{\pgfqpoint{1.940038in}{2.081124in}}%
\pgfpathlineto{\pgfqpoint{1.937228in}{2.089405in}}%
\pgfpathlineto{\pgfqpoint{1.934417in}{2.096736in}}%
\pgfpathlineto{\pgfqpoint{1.931606in}{2.086514in}}%
\pgfpathlineto{\pgfqpoint{1.928796in}{2.103623in}}%
\pgfpathlineto{\pgfqpoint{1.925985in}{2.099566in}}%
\pgfpathlineto{\pgfqpoint{1.923174in}{2.080870in}}%
\pgfpathlineto{\pgfqpoint{1.920364in}{2.093351in}}%
\pgfpathlineto{\pgfqpoint{1.917553in}{2.095027in}}%
\pgfpathlineto{\pgfqpoint{1.914742in}{2.107318in}}%
\pgfpathlineto{\pgfqpoint{1.911931in}{2.086882in}}%
\pgfpathlineto{\pgfqpoint{1.909121in}{2.085053in}}%
\pgfpathlineto{\pgfqpoint{1.906310in}{2.096385in}}%
\pgfpathlineto{\pgfqpoint{1.903499in}{2.076498in}}%
\pgfpathlineto{\pgfqpoint{1.900689in}{2.091838in}}%
\pgfpathlineto{\pgfqpoint{1.897878in}{2.079099in}}%
\pgfpathlineto{\pgfqpoint{1.895067in}{2.074575in}}%
\pgfpathlineto{\pgfqpoint{1.892257in}{2.086880in}}%
\pgfpathlineto{\pgfqpoint{1.889446in}{2.073459in}}%
\pgfpathlineto{\pgfqpoint{1.886635in}{2.064076in}}%
\pgfpathlineto{\pgfqpoint{1.883825in}{2.062736in}}%
\pgfpathlineto{\pgfqpoint{1.881014in}{2.044856in}}%
\pgfpathlineto{\pgfqpoint{1.878203in}{2.065568in}}%
\pgfpathlineto{\pgfqpoint{1.875393in}{2.052649in}}%
\pgfpathlineto{\pgfqpoint{1.872582in}{2.064473in}}%
\pgfpathlineto{\pgfqpoint{1.869771in}{2.070555in}}%
\pgfpathlineto{\pgfqpoint{1.866960in}{2.069918in}}%
\pgfpathlineto{\pgfqpoint{1.864150in}{2.069474in}}%
\pgfpathlineto{\pgfqpoint{1.861339in}{2.064154in}}%
\pgfpathlineto{\pgfqpoint{1.858528in}{2.059000in}}%
\pgfpathlineto{\pgfqpoint{1.855718in}{2.062273in}}%
\pgfpathlineto{\pgfqpoint{1.852907in}{2.055844in}}%
\pgfpathlineto{\pgfqpoint{1.850096in}{2.074460in}}%
\pgfpathlineto{\pgfqpoint{1.847286in}{2.122837in}}%
\pgfpathlineto{\pgfqpoint{1.844475in}{2.107251in}}%
\pgfpathlineto{\pgfqpoint{1.841664in}{2.109774in}}%
\pgfpathlineto{\pgfqpoint{1.838854in}{2.113391in}}%
\pgfpathlineto{\pgfqpoint{1.836043in}{2.102044in}}%
\pgfpathlineto{\pgfqpoint{1.833232in}{2.111144in}}%
\pgfpathlineto{\pgfqpoint{1.830422in}{2.118569in}}%
\pgfpathlineto{\pgfqpoint{1.827611in}{2.131168in}}%
\pgfpathlineto{\pgfqpoint{1.824800in}{2.127054in}}%
\pgfpathlineto{\pgfqpoint{1.821989in}{2.141886in}}%
\pgfpathlineto{\pgfqpoint{1.819179in}{2.136749in}}%
\pgfpathlineto{\pgfqpoint{1.816368in}{2.125653in}}%
\pgfpathlineto{\pgfqpoint{1.813557in}{2.139066in}}%
\pgfpathlineto{\pgfqpoint{1.810747in}{2.107249in}}%
\pgfpathlineto{\pgfqpoint{1.807936in}{2.093980in}}%
\pgfpathlineto{\pgfqpoint{1.805125in}{2.096188in}}%
\pgfpathlineto{\pgfqpoint{1.802315in}{2.109069in}}%
\pgfpathlineto{\pgfqpoint{1.799504in}{2.097823in}}%
\pgfpathlineto{\pgfqpoint{1.796693in}{2.104177in}}%
\pgfpathlineto{\pgfqpoint{1.793883in}{2.094274in}}%
\pgfpathlineto{\pgfqpoint{1.791072in}{2.104447in}}%
\pgfpathlineto{\pgfqpoint{1.788261in}{2.082262in}}%
\pgfpathlineto{\pgfqpoint{1.785451in}{2.085371in}}%
\pgfpathlineto{\pgfqpoint{1.782640in}{2.068822in}}%
\pgfpathlineto{\pgfqpoint{1.779829in}{2.073708in}}%
\pgfpathlineto{\pgfqpoint{1.777018in}{2.078220in}}%
\pgfpathlineto{\pgfqpoint{1.774208in}{2.077638in}}%
\pgfpathlineto{\pgfqpoint{1.771397in}{2.106495in}}%
\pgfpathlineto{\pgfqpoint{1.768586in}{2.128963in}}%
\pgfpathlineto{\pgfqpoint{1.765776in}{2.115053in}}%
\pgfpathlineto{\pgfqpoint{1.762965in}{2.147721in}}%
\pgfpathlineto{\pgfqpoint{1.760154in}{2.104562in}}%
\pgfpathlineto{\pgfqpoint{1.757344in}{2.129873in}}%
\pgfpathlineto{\pgfqpoint{1.754533in}{2.105903in}}%
\pgfpathlineto{\pgfqpoint{1.751722in}{2.106060in}}%
\pgfpathlineto{\pgfqpoint{1.748912in}{2.106861in}}%
\pgfpathlineto{\pgfqpoint{1.746101in}{2.107628in}}%
\pgfpathlineto{\pgfqpoint{1.743290in}{2.108192in}}%
\pgfpathlineto{\pgfqpoint{1.740480in}{2.108896in}}%
\pgfpathlineto{\pgfqpoint{1.737669in}{2.109726in}}%
\pgfpathlineto{\pgfqpoint{1.734858in}{2.110200in}}%
\pgfpathlineto{\pgfqpoint{1.732048in}{2.110653in}}%
\pgfpathlineto{\pgfqpoint{1.729237in}{2.110697in}}%
\pgfpathlineto{\pgfqpoint{1.726426in}{2.114447in}}%
\pgfpathlineto{\pgfqpoint{1.723615in}{2.102407in}}%
\pgfpathlineto{\pgfqpoint{1.720805in}{2.117483in}}%
\pgfpathlineto{\pgfqpoint{1.717994in}{2.136270in}}%
\pgfpathlineto{\pgfqpoint{1.715183in}{2.096718in}}%
\pgfpathlineto{\pgfqpoint{1.712373in}{2.106654in}}%
\pgfpathlineto{\pgfqpoint{1.709562in}{2.136384in}}%
\pgfpathlineto{\pgfqpoint{1.706751in}{2.103569in}}%
\pgfpathlineto{\pgfqpoint{1.703941in}{2.100070in}}%
\pgfpathlineto{\pgfqpoint{1.701130in}{2.110460in}}%
\pgfpathlineto{\pgfqpoint{1.698319in}{2.117002in}}%
\pgfpathlineto{\pgfqpoint{1.695509in}{2.111900in}}%
\pgfpathlineto{\pgfqpoint{1.692698in}{2.114613in}}%
\pgfpathlineto{\pgfqpoint{1.689887in}{2.104329in}}%
\pgfpathlineto{\pgfqpoint{1.687077in}{2.109106in}}%
\pgfpathlineto{\pgfqpoint{1.684266in}{2.084293in}}%
\pgfpathlineto{\pgfqpoint{1.681455in}{2.089303in}}%
\pgfpathlineto{\pgfqpoint{1.678644in}{2.104228in}}%
\pgfpathlineto{\pgfqpoint{1.675834in}{2.091834in}}%
\pgfpathlineto{\pgfqpoint{1.673023in}{2.093788in}}%
\pgfpathlineto{\pgfqpoint{1.670212in}{2.108764in}}%
\pgfpathlineto{\pgfqpoint{1.667402in}{2.098134in}}%
\pgfpathlineto{\pgfqpoint{1.664591in}{2.100225in}}%
\pgfpathlineto{\pgfqpoint{1.661780in}{2.112477in}}%
\pgfpathlineto{\pgfqpoint{1.658970in}{2.113655in}}%
\pgfpathlineto{\pgfqpoint{1.656159in}{2.116050in}}%
\pgfpathlineto{\pgfqpoint{1.653348in}{2.115563in}}%
\pgfpathlineto{\pgfqpoint{1.650538in}{2.119398in}}%
\pgfpathlineto{\pgfqpoint{1.647727in}{2.108123in}}%
\pgfpathlineto{\pgfqpoint{1.644916in}{2.103581in}}%
\pgfpathlineto{\pgfqpoint{1.642106in}{2.098803in}}%
\pgfpathlineto{\pgfqpoint{1.639295in}{2.093103in}}%
\pgfpathlineto{\pgfqpoint{1.636484in}{2.084197in}}%
\pgfpathlineto{\pgfqpoint{1.633673in}{2.121212in}}%
\pgfpathlineto{\pgfqpoint{1.630863in}{2.122153in}}%
\pgfpathlineto{\pgfqpoint{1.628052in}{2.121313in}}%
\pgfpathlineto{\pgfqpoint{1.625241in}{2.121397in}}%
\pgfpathlineto{\pgfqpoint{1.622431in}{2.121711in}}%
\pgfpathlineto{\pgfqpoint{1.619620in}{2.121952in}}%
\pgfpathlineto{\pgfqpoint{1.616809in}{2.122809in}}%
\pgfpathlineto{\pgfqpoint{1.613999in}{2.111945in}}%
\pgfpathlineto{\pgfqpoint{1.611188in}{2.140364in}}%
\pgfpathlineto{\pgfqpoint{1.608377in}{2.134136in}}%
\pgfpathlineto{\pgfqpoint{1.605567in}{2.131140in}}%
\pgfpathlineto{\pgfqpoint{1.602756in}{2.114976in}}%
\pgfpathlineto{\pgfqpoint{1.599945in}{2.116974in}}%
\pgfpathlineto{\pgfqpoint{1.597135in}{2.122460in}}%
\pgfpathlineto{\pgfqpoint{1.594324in}{2.120079in}}%
\pgfpathlineto{\pgfqpoint{1.591513in}{2.120993in}}%
\pgfpathlineto{\pgfqpoint{1.588702in}{2.120863in}}%
\pgfpathlineto{\pgfqpoint{1.585892in}{2.119856in}}%
\pgfpathlineto{\pgfqpoint{1.583081in}{2.120886in}}%
\pgfpathlineto{\pgfqpoint{1.580270in}{2.121957in}}%
\pgfpathlineto{\pgfqpoint{1.577460in}{2.123018in}}%
\pgfpathlineto{\pgfqpoint{1.574649in}{2.123872in}}%
\pgfpathlineto{\pgfqpoint{1.571838in}{2.124804in}}%
\pgfpathlineto{\pgfqpoint{1.569028in}{2.125743in}}%
\pgfpathlineto{\pgfqpoint{1.566217in}{2.126549in}}%
\pgfpathlineto{\pgfqpoint{1.563406in}{2.127310in}}%
\pgfpathlineto{\pgfqpoint{1.560596in}{2.128408in}}%
\pgfpathlineto{\pgfqpoint{1.557785in}{2.129512in}}%
\pgfpathlineto{\pgfqpoint{1.554974in}{2.129444in}}%
\pgfpathlineto{\pgfqpoint{1.552164in}{2.129962in}}%
\pgfpathlineto{\pgfqpoint{1.549353in}{2.130747in}}%
\pgfpathlineto{\pgfqpoint{1.546542in}{2.131877in}}%
\pgfpathlineto{\pgfqpoint{1.543731in}{2.133033in}}%
\pgfpathlineto{\pgfqpoint{1.540921in}{2.133050in}}%
\pgfpathlineto{\pgfqpoint{1.538110in}{2.133953in}}%
\pgfpathlineto{\pgfqpoint{1.535299in}{2.134620in}}%
\pgfpathlineto{\pgfqpoint{1.532489in}{2.120096in}}%
\pgfpathlineto{\pgfqpoint{1.529678in}{2.147441in}}%
\pgfpathlineto{\pgfqpoint{1.526867in}{2.149788in}}%
\pgfpathlineto{\pgfqpoint{1.524057in}{2.137691in}}%
\pgfpathlineto{\pgfqpoint{1.521246in}{2.127540in}}%
\pgfpathlineto{\pgfqpoint{1.518435in}{2.128869in}}%
\pgfpathlineto{\pgfqpoint{1.515625in}{2.136537in}}%
\pgfpathlineto{\pgfqpoint{1.512814in}{2.129596in}}%
\pgfpathlineto{\pgfqpoint{1.510003in}{2.112017in}}%
\pgfpathlineto{\pgfqpoint{1.507193in}{2.132360in}}%
\pgfpathlineto{\pgfqpoint{1.504382in}{2.109654in}}%
\pgfpathlineto{\pgfqpoint{1.501571in}{2.100005in}}%
\pgfpathlineto{\pgfqpoint{1.498761in}{2.139499in}}%
\pgfpathlineto{\pgfqpoint{1.495950in}{2.144925in}}%
\pgfpathlineto{\pgfqpoint{1.493139in}{2.154461in}}%
\pgfpathlineto{\pgfqpoint{1.490328in}{2.168216in}}%
\pgfpathlineto{\pgfqpoint{1.487518in}{2.173723in}}%
\pgfpathlineto{\pgfqpoint{1.484707in}{2.153848in}}%
\pgfpathlineto{\pgfqpoint{1.481896in}{2.144864in}}%
\pgfpathlineto{\pgfqpoint{1.479086in}{2.120444in}}%
\pgfpathlineto{\pgfqpoint{1.476275in}{2.107265in}}%
\pgfpathlineto{\pgfqpoint{1.473464in}{2.116094in}}%
\pgfpathlineto{\pgfqpoint{1.470654in}{2.125490in}}%
\pgfpathlineto{\pgfqpoint{1.467843in}{2.131132in}}%
\pgfpathlineto{\pgfqpoint{1.465032in}{2.129397in}}%
\pgfpathlineto{\pgfqpoint{1.462222in}{2.145798in}}%
\pgfpathlineto{\pgfqpoint{1.459411in}{2.154205in}}%
\pgfpathlineto{\pgfqpoint{1.456600in}{2.130966in}}%
\pgfpathlineto{\pgfqpoint{1.453790in}{2.108465in}}%
\pgfpathlineto{\pgfqpoint{1.450979in}{2.105987in}}%
\pgfpathlineto{\pgfqpoint{1.448168in}{2.109565in}}%
\pgfpathlineto{\pgfqpoint{1.445357in}{2.131718in}}%
\pgfpathlineto{\pgfqpoint{1.442547in}{2.118813in}}%
\pgfpathlineto{\pgfqpoint{1.439736in}{2.142574in}}%
\pgfpathlineto{\pgfqpoint{1.436925in}{2.147218in}}%
\pgfpathlineto{\pgfqpoint{1.434115in}{2.164590in}}%
\pgfpathlineto{\pgfqpoint{1.431304in}{2.140972in}}%
\pgfpathlineto{\pgfqpoint{1.428493in}{2.150175in}}%
\pgfpathlineto{\pgfqpoint{1.425683in}{2.133677in}}%
\pgfpathlineto{\pgfqpoint{1.422872in}{2.135045in}}%
\pgfpathlineto{\pgfqpoint{1.420061in}{2.129157in}}%
\pgfpathlineto{\pgfqpoint{1.417251in}{2.137600in}}%
\pgfpathlineto{\pgfqpoint{1.414440in}{2.132080in}}%
\pgfpathlineto{\pgfqpoint{1.411629in}{2.122121in}}%
\pgfpathlineto{\pgfqpoint{1.408819in}{2.106199in}}%
\pgfpathlineto{\pgfqpoint{1.406008in}{2.110205in}}%
\pgfpathlineto{\pgfqpoint{1.403197in}{2.122910in}}%
\pgfpathlineto{\pgfqpoint{1.400386in}{2.153834in}}%
\pgfpathlineto{\pgfqpoint{1.397576in}{2.139243in}}%
\pgfpathlineto{\pgfqpoint{1.394765in}{2.138359in}}%
\pgfpathlineto{\pgfqpoint{1.391954in}{2.160916in}}%
\pgfpathlineto{\pgfqpoint{1.389144in}{2.149547in}}%
\pgfpathlineto{\pgfqpoint{1.386333in}{2.152201in}}%
\pgfpathlineto{\pgfqpoint{1.383522in}{2.172610in}}%
\pgfpathlineto{\pgfqpoint{1.380712in}{2.158872in}}%
\pgfpathlineto{\pgfqpoint{1.377901in}{2.167453in}}%
\pgfpathlineto{\pgfqpoint{1.375090in}{2.167143in}}%
\pgfpathlineto{\pgfqpoint{1.372280in}{2.182335in}}%
\pgfpathlineto{\pgfqpoint{1.369469in}{2.181626in}}%
\pgfpathlineto{\pgfqpoint{1.366658in}{2.173408in}}%
\pgfpathlineto{\pgfqpoint{1.363848in}{2.183970in}}%
\pgfpathlineto{\pgfqpoint{1.361037in}{2.179350in}}%
\pgfpathlineto{\pgfqpoint{1.358226in}{2.175494in}}%
\pgfpathlineto{\pgfqpoint{1.355415in}{2.162156in}}%
\pgfpathlineto{\pgfqpoint{1.352605in}{2.169039in}}%
\pgfpathlineto{\pgfqpoint{1.349794in}{2.169230in}}%
\pgfpathlineto{\pgfqpoint{1.346983in}{2.189430in}}%
\pgfpathlineto{\pgfqpoint{1.344173in}{2.170770in}}%
\pgfpathlineto{\pgfqpoint{1.341362in}{2.171897in}}%
\pgfpathlineto{\pgfqpoint{1.338551in}{2.172360in}}%
\pgfpathlineto{\pgfqpoint{1.335741in}{2.173320in}}%
\pgfpathlineto{\pgfqpoint{1.332930in}{2.175233in}}%
\pgfpathlineto{\pgfqpoint{1.330119in}{2.176904in}}%
\pgfpathlineto{\pgfqpoint{1.327309in}{2.176338in}}%
\pgfpathlineto{\pgfqpoint{1.324498in}{2.175508in}}%
\pgfpathlineto{\pgfqpoint{1.321687in}{2.158454in}}%
\pgfpathlineto{\pgfqpoint{1.318877in}{2.132891in}}%
\pgfpathlineto{\pgfqpoint{1.316066in}{2.136567in}}%
\pgfpathlineto{\pgfqpoint{1.313255in}{2.151712in}}%
\pgfpathlineto{\pgfqpoint{1.310445in}{2.199182in}}%
\pgfpathlineto{\pgfqpoint{1.307634in}{2.204751in}}%
\pgfpathlineto{\pgfqpoint{1.304823in}{2.190748in}}%
\pgfpathlineto{\pgfqpoint{1.302012in}{2.177153in}}%
\pgfpathlineto{\pgfqpoint{1.299202in}{2.184293in}}%
\pgfpathlineto{\pgfqpoint{1.296391in}{2.149352in}}%
\pgfpathlineto{\pgfqpoint{1.293580in}{2.136200in}}%
\pgfpathlineto{\pgfqpoint{1.290770in}{2.135038in}}%
\pgfpathlineto{\pgfqpoint{1.287959in}{2.161269in}}%
\pgfpathlineto{\pgfqpoint{1.285148in}{2.180304in}}%
\pgfpathlineto{\pgfqpoint{1.282338in}{2.213252in}}%
\pgfpathlineto{\pgfqpoint{1.279527in}{2.209127in}}%
\pgfpathlineto{\pgfqpoint{1.276716in}{2.203021in}}%
\pgfpathlineto{\pgfqpoint{1.273906in}{2.184696in}}%
\pgfpathlineto{\pgfqpoint{1.271095in}{2.150834in}}%
\pgfpathlineto{\pgfqpoint{1.268284in}{2.149314in}}%
\pgfpathlineto{\pgfqpoint{1.265474in}{2.126974in}}%
\pgfpathlineto{\pgfqpoint{1.262663in}{2.156720in}}%
\pgfpathlineto{\pgfqpoint{1.259852in}{2.192045in}}%
\pgfpathlineto{\pgfqpoint{1.257041in}{2.160124in}}%
\pgfpathlineto{\pgfqpoint{1.254231in}{2.157575in}}%
\pgfpathlineto{\pgfqpoint{1.251420in}{2.182942in}}%
\pgfpathlineto{\pgfqpoint{1.248609in}{2.171285in}}%
\pgfpathlineto{\pgfqpoint{1.245799in}{2.150742in}}%
\pgfpathlineto{\pgfqpoint{1.242988in}{2.149437in}}%
\pgfpathlineto{\pgfqpoint{1.240177in}{2.149520in}}%
\pgfpathlineto{\pgfqpoint{1.237367in}{2.149447in}}%
\pgfpathlineto{\pgfqpoint{1.234556in}{2.148011in}}%
\pgfpathlineto{\pgfqpoint{1.231745in}{2.145811in}}%
\pgfpathlineto{\pgfqpoint{1.228935in}{2.144537in}}%
\pgfpathlineto{\pgfqpoint{1.226124in}{2.145263in}}%
\pgfpathlineto{\pgfqpoint{1.223313in}{2.146113in}}%
\pgfpathlineto{\pgfqpoint{1.220503in}{2.148457in}}%
\pgfpathlineto{\pgfqpoint{1.217692in}{2.150834in}}%
\pgfpathlineto{\pgfqpoint{1.214881in}{2.152440in}}%
\pgfpathlineto{\pgfqpoint{1.212070in}{2.146951in}}%
\pgfpathlineto{\pgfqpoint{1.209260in}{2.149479in}}%
\pgfpathlineto{\pgfqpoint{1.206449in}{2.146752in}}%
\pgfpathlineto{\pgfqpoint{1.203638in}{2.145458in}}%
\pgfpathlineto{\pgfqpoint{1.200828in}{2.146096in}}%
\pgfpathlineto{\pgfqpoint{1.198017in}{2.145328in}}%
\pgfpathlineto{\pgfqpoint{1.195206in}{2.146174in}}%
\pgfpathlineto{\pgfqpoint{1.192396in}{2.148846in}}%
\pgfpathlineto{\pgfqpoint{1.189585in}{2.150321in}}%
\pgfpathlineto{\pgfqpoint{1.186774in}{2.151515in}}%
\pgfpathlineto{\pgfqpoint{1.183964in}{2.147836in}}%
\pgfpathlineto{\pgfqpoint{1.181153in}{2.134740in}}%
\pgfpathlineto{\pgfqpoint{1.178342in}{2.134897in}}%
\pgfpathlineto{\pgfqpoint{1.175532in}{2.134624in}}%
\pgfpathlineto{\pgfqpoint{1.172721in}{2.132185in}}%
\pgfpathlineto{\pgfqpoint{1.169910in}{2.134793in}}%
\pgfpathlineto{\pgfqpoint{1.167099in}{2.137418in}}%
\pgfpathlineto{\pgfqpoint{1.164289in}{2.140235in}}%
\pgfpathlineto{\pgfqpoint{1.161478in}{2.141567in}}%
\pgfpathlineto{\pgfqpoint{1.158667in}{2.144627in}}%
\pgfpathlineto{\pgfqpoint{1.155857in}{2.147672in}}%
\pgfpathlineto{\pgfqpoint{1.153046in}{2.149123in}}%
\pgfpathlineto{\pgfqpoint{1.150235in}{2.148545in}}%
\pgfpathlineto{\pgfqpoint{1.147425in}{2.126606in}}%
\pgfpathlineto{\pgfqpoint{1.144614in}{2.129554in}}%
\pgfpathlineto{\pgfqpoint{1.141803in}{2.132233in}}%
\pgfpathlineto{\pgfqpoint{1.138993in}{2.135407in}}%
\pgfpathlineto{\pgfqpoint{1.136182in}{2.137930in}}%
\pgfpathlineto{\pgfqpoint{1.133371in}{2.137493in}}%
\pgfpathlineto{\pgfqpoint{1.130561in}{2.140500in}}%
\pgfpathlineto{\pgfqpoint{1.127750in}{2.130134in}}%
\pgfpathlineto{\pgfqpoint{1.124939in}{2.127615in}}%
\pgfpathlineto{\pgfqpoint{1.122128in}{2.130879in}}%
\pgfpathlineto{\pgfqpoint{1.119318in}{2.134338in}}%
\pgfpathlineto{\pgfqpoint{1.116507in}{2.137819in}}%
\pgfpathlineto{\pgfqpoint{1.113696in}{2.137882in}}%
\pgfpathlineto{\pgfqpoint{1.110886in}{2.137997in}}%
\pgfpathlineto{\pgfqpoint{1.108075in}{2.105634in}}%
\pgfpathlineto{\pgfqpoint{1.105264in}{2.105347in}}%
\pgfpathlineto{\pgfqpoint{1.102454in}{2.131423in}}%
\pgfpathlineto{\pgfqpoint{1.099643in}{2.158215in}}%
\pgfpathlineto{\pgfqpoint{1.096832in}{2.084810in}}%
\pgfpathlineto{\pgfqpoint{1.094022in}{2.128258in}}%
\pgfpathlineto{\pgfqpoint{1.091211in}{2.127174in}}%
\pgfpathlineto{\pgfqpoint{1.088400in}{2.086886in}}%
\pgfpathlineto{\pgfqpoint{1.085590in}{2.116513in}}%
\pgfpathlineto{\pgfqpoint{1.082779in}{2.120032in}}%
\pgfpathlineto{\pgfqpoint{1.079968in}{2.129559in}}%
\pgfpathlineto{\pgfqpoint{1.077158in}{2.130347in}}%
\pgfpathlineto{\pgfqpoint{1.074347in}{2.136448in}}%
\pgfpathlineto{\pgfqpoint{1.071536in}{2.091107in}}%
\pgfpathlineto{\pgfqpoint{1.068725in}{2.143464in}}%
\pgfpathlineto{\pgfqpoint{1.065915in}{2.103163in}}%
\pgfpathlineto{\pgfqpoint{1.063104in}{2.185102in}}%
\pgfpathlineto{\pgfqpoint{1.060293in}{2.137809in}}%
\pgfpathlineto{\pgfqpoint{1.057483in}{2.131686in}}%
\pgfpathlineto{\pgfqpoint{1.054672in}{2.136647in}}%
\pgfpathlineto{\pgfqpoint{1.051861in}{2.140849in}}%
\pgfpathlineto{\pgfqpoint{1.049051in}{2.146139in}}%
\pgfpathlineto{\pgfqpoint{1.046240in}{2.148893in}}%
\pgfpathlineto{\pgfqpoint{1.043429in}{2.154465in}}%
\pgfpathlineto{\pgfqpoint{1.040619in}{2.160011in}}%
\pgfpathlineto{\pgfqpoint{1.037808in}{2.151690in}}%
\pgfpathlineto{\pgfqpoint{1.034997in}{2.154464in}}%
\pgfpathlineto{\pgfqpoint{1.032187in}{2.158980in}}%
\pgfpathlineto{\pgfqpoint{1.029376in}{2.164787in}}%
\pgfpathlineto{\pgfqpoint{1.026565in}{2.170676in}}%
\pgfpathlineto{\pgfqpoint{1.023754in}{2.174304in}}%
\pgfpathlineto{\pgfqpoint{1.020944in}{2.178955in}}%
\pgfpathlineto{\pgfqpoint{1.018133in}{2.045657in}}%
\pgfpathlineto{\pgfqpoint{1.015322in}{2.226930in}}%
\pgfpathlineto{\pgfqpoint{1.012512in}{2.079129in}}%
\pgfpathlineto{\pgfqpoint{1.009701in}{2.215099in}}%
\pgfpathlineto{\pgfqpoint{1.006890in}{2.140115in}}%
\pgfpathlineto{\pgfqpoint{1.004080in}{2.145499in}}%
\pgfpathlineto{\pgfqpoint{1.001269in}{2.195172in}}%
\pgfpathlineto{\pgfqpoint{0.998458in}{2.087593in}}%
\pgfpathlineto{\pgfqpoint{0.995648in}{2.291660in}}%
\pgfpathlineto{\pgfqpoint{0.992837in}{2.075070in}}%
\pgfpathlineto{\pgfqpoint{0.990026in}{2.184549in}}%
\pgfpathlineto{\pgfqpoint{0.987216in}{2.172416in}}%
\pgfpathlineto{\pgfqpoint{0.984405in}{2.143472in}}%
\pgfpathlineto{\pgfqpoint{0.981594in}{2.100793in}}%
\pgfpathlineto{\pgfqpoint{0.978783in}{2.129292in}}%
\pgfpathlineto{\pgfqpoint{0.975973in}{2.034577in}}%
\pgfpathlineto{\pgfqpoint{0.973162in}{2.230769in}}%
\pgfpathlineto{\pgfqpoint{0.970351in}{1.903703in}}%
\pgfpathlineto{\pgfqpoint{0.967541in}{1.982047in}}%
\pgfpathlineto{\pgfqpoint{0.964730in}{2.037780in}}%
\pgfpathlineto{\pgfqpoint{0.961919in}{2.035615in}}%
\pgfpathlineto{\pgfqpoint{0.959109in}{2.047284in}}%
\pgfpathlineto{\pgfqpoint{0.956298in}{2.053234in}}%
\pgfpathlineto{\pgfqpoint{0.953487in}{2.069287in}}%
\pgfpathlineto{\pgfqpoint{0.950677in}{2.046326in}}%
\pgfpathlineto{\pgfqpoint{0.947866in}{2.062399in}}%
\pgfpathlineto{\pgfqpoint{0.945055in}{2.141134in}}%
\pgfpathlineto{\pgfqpoint{0.942245in}{2.061041in}}%
\pgfpathlineto{\pgfqpoint{0.939434in}{2.072788in}}%
\pgfpathlineto{\pgfqpoint{0.936623in}{2.034771in}}%
\pgfpathclose%
\pgfusepath{stroke,fill}%
\end{pgfscope}%
\begin{pgfscope}%
\pgfpathrectangle{\pgfqpoint{0.711206in}{0.331635in}}{\pgfqpoint{4.650000in}{3.020000in}}%
\pgfusepath{clip}%
\pgfsetroundcap%
\pgfsetroundjoin%
\pgfsetlinewidth{1.505625pt}%
\definecolor{currentstroke}{rgb}{0.121569,0.466667,0.705882}%
\pgfsetstrokecolor{currentstroke}%
\pgfsetdash{}{0pt}%
\pgfpathmoveto{\pgfqpoint{0.922570in}{1.564045in}}%
\pgfpathlineto{\pgfqpoint{0.925380in}{1.677738in}}%
\pgfpathlineto{\pgfqpoint{0.928191in}{1.584468in}}%
\pgfpathlineto{\pgfqpoint{0.931002in}{2.033621in}}%
\pgfpathlineto{\pgfqpoint{0.933812in}{1.600243in}}%
\pgfpathlineto{\pgfqpoint{0.936623in}{1.965760in}}%
\pgfpathlineto{\pgfqpoint{0.939434in}{1.885633in}}%
\pgfpathlineto{\pgfqpoint{0.942245in}{1.477526in}}%
\pgfpathlineto{\pgfqpoint{0.945055in}{1.417916in}}%
\pgfpathlineto{\pgfqpoint{0.947866in}{1.535248in}}%
\pgfpathlineto{\pgfqpoint{0.950677in}{1.912482in}}%
\pgfpathlineto{\pgfqpoint{0.953487in}{1.621247in}}%
\pgfpathlineto{\pgfqpoint{0.956298in}{1.741600in}}%
\pgfpathlineto{\pgfqpoint{0.959109in}{1.720282in}}%
\pgfpathlineto{\pgfqpoint{0.961919in}{1.507417in}}%
\pgfpathlineto{\pgfqpoint{0.964730in}{1.812974in}}%
\pgfpathlineto{\pgfqpoint{0.967541in}{1.979928in}}%
\pgfpathlineto{\pgfqpoint{0.970351in}{2.225501in}}%
\pgfpathlineto{\pgfqpoint{0.973162in}{1.832590in}}%
\pgfpathlineto{\pgfqpoint{0.978783in}{1.638782in}}%
\pgfpathlineto{\pgfqpoint{0.981594in}{1.890745in}}%
\pgfpathlineto{\pgfqpoint{0.984405in}{2.267102in}}%
\pgfpathlineto{\pgfqpoint{0.990026in}{1.528343in}}%
\pgfpathlineto{\pgfqpoint{0.992837in}{2.036267in}}%
\pgfpathlineto{\pgfqpoint{0.998458in}{1.512676in}}%
\pgfpathlineto{\pgfqpoint{1.001269in}{1.859067in}}%
\pgfpathlineto{\pgfqpoint{1.004080in}{1.580859in}}%
\pgfpathlineto{\pgfqpoint{1.006890in}{1.986719in}}%
\pgfpathlineto{\pgfqpoint{1.009701in}{1.704182in}}%
\pgfpathlineto{\pgfqpoint{1.012512in}{1.837197in}}%
\pgfpathlineto{\pgfqpoint{1.015322in}{1.601447in}}%
\pgfpathlineto{\pgfqpoint{1.018133in}{1.968139in}}%
\pgfpathlineto{\pgfqpoint{1.020944in}{1.349149in}}%
\pgfpathlineto{\pgfqpoint{1.023754in}{1.801115in}}%
\pgfpathlineto{\pgfqpoint{1.026565in}{1.552492in}}%
\pgfpathlineto{\pgfqpoint{1.029376in}{1.704206in}}%
\pgfpathlineto{\pgfqpoint{1.032187in}{1.502113in}}%
\pgfpathlineto{\pgfqpoint{1.034997in}{1.802575in}}%
\pgfpathlineto{\pgfqpoint{1.037808in}{2.001065in}}%
\pgfpathlineto{\pgfqpoint{1.040619in}{1.583647in}}%
\pgfpathlineto{\pgfqpoint{1.043429in}{1.607207in}}%
\pgfpathlineto{\pgfqpoint{1.046240in}{1.794954in}}%
\pgfpathlineto{\pgfqpoint{1.049051in}{1.619436in}}%
\pgfpathlineto{\pgfqpoint{1.051861in}{1.728359in}}%
\pgfpathlineto{\pgfqpoint{1.054672in}{1.649623in}}%
\pgfpathlineto{\pgfqpoint{1.057483in}{1.974522in}}%
\pgfpathlineto{\pgfqpoint{1.060293in}{1.385244in}}%
\pgfpathlineto{\pgfqpoint{1.063104in}{1.722367in}}%
\pgfpathlineto{\pgfqpoint{1.065915in}{1.764610in}}%
\pgfpathlineto{\pgfqpoint{1.068725in}{1.889913in}}%
\pgfpathlineto{\pgfqpoint{1.071536in}{1.857984in}}%
\pgfpathlineto{\pgfqpoint{1.074347in}{1.662516in}}%
\pgfpathlineto{\pgfqpoint{1.077158in}{1.626911in}}%
\pgfpathlineto{\pgfqpoint{1.079968in}{1.650446in}}%
\pgfpathlineto{\pgfqpoint{1.082779in}{1.560627in}}%
\pgfpathlineto{\pgfqpoint{1.085590in}{1.817470in}}%
\pgfpathlineto{\pgfqpoint{1.088400in}{1.875383in}}%
\pgfpathlineto{\pgfqpoint{1.091211in}{1.508663in}}%
\pgfpathlineto{\pgfqpoint{1.094022in}{1.957490in}}%
\pgfpathlineto{\pgfqpoint{1.096832in}{1.497790in}}%
\pgfpathlineto{\pgfqpoint{1.099643in}{1.326495in}}%
\pgfpathlineto{\pgfqpoint{1.105264in}{2.167132in}}%
\pgfpathlineto{\pgfqpoint{1.108075in}{1.964550in}}%
\pgfpathlineto{\pgfqpoint{1.110886in}{1.396081in}}%
\pgfpathlineto{\pgfqpoint{1.113696in}{1.855195in}}%
\pgfpathlineto{\pgfqpoint{1.116507in}{1.651390in}}%
\pgfpathlineto{\pgfqpoint{1.119318in}{1.616279in}}%
\pgfpathlineto{\pgfqpoint{1.122128in}{1.668643in}}%
\pgfpathlineto{\pgfqpoint{1.124939in}{1.313100in}}%
\pgfpathlineto{\pgfqpoint{1.127750in}{1.799528in}}%
\pgfpathlineto{\pgfqpoint{1.130561in}{2.062565in}}%
\pgfpathlineto{\pgfqpoint{1.133371in}{1.871173in}}%
\pgfpathlineto{\pgfqpoint{1.136182in}{1.732193in}}%
\pgfpathlineto{\pgfqpoint{1.138993in}{1.623096in}}%
\pgfpathlineto{\pgfqpoint{1.141803in}{1.709283in}}%
\pgfpathlineto{\pgfqpoint{1.144614in}{1.576592in}}%
\pgfpathlineto{\pgfqpoint{1.147425in}{0.938416in}}%
\pgfpathlineto{\pgfqpoint{1.150235in}{1.873351in}}%
\pgfpathlineto{\pgfqpoint{1.153046in}{1.787877in}}%
\pgfpathlineto{\pgfqpoint{1.155857in}{1.644170in}}%
\pgfpathlineto{\pgfqpoint{1.158667in}{1.595874in}}%
\pgfpathlineto{\pgfqpoint{1.161478in}{1.788172in}}%
\pgfpathlineto{\pgfqpoint{1.167099in}{1.529112in}}%
\pgfpathlineto{\pgfqpoint{1.169910in}{1.679888in}}%
\pgfpathlineto{\pgfqpoint{1.172721in}{1.932956in}}%
\pgfpathlineto{\pgfqpoint{1.175532in}{1.353665in}}%
\pgfpathlineto{\pgfqpoint{1.178342in}{1.364890in}}%
\pgfpathlineto{\pgfqpoint{1.181153in}{2.183208in}}%
\pgfpathlineto{\pgfqpoint{1.186774in}{1.792574in}}%
\pgfpathlineto{\pgfqpoint{1.189585in}{1.774346in}}%
\pgfpathlineto{\pgfqpoint{1.192396in}{1.639151in}}%
\pgfpathlineto{\pgfqpoint{1.195206in}{1.809275in}}%
\pgfpathlineto{\pgfqpoint{1.198017in}{1.325637in}}%
\pgfpathlineto{\pgfqpoint{1.200828in}{1.384057in}}%
\pgfpathlineto{\pgfqpoint{1.203638in}{1.290497in}}%
\pgfpathlineto{\pgfqpoint{1.206449in}{1.952648in}}%
\pgfpathlineto{\pgfqpoint{1.209260in}{1.642714in}}%
\pgfpathlineto{\pgfqpoint{1.212070in}{2.039993in}}%
\pgfpathlineto{\pgfqpoint{1.214881in}{1.752278in}}%
\pgfpathlineto{\pgfqpoint{1.217692in}{1.661924in}}%
\pgfpathlineto{\pgfqpoint{1.220503in}{1.661866in}}%
\pgfpathlineto{\pgfqpoint{1.223313in}{1.800316in}}%
\pgfpathlineto{\pgfqpoint{1.226124in}{1.389334in}}%
\pgfpathlineto{\pgfqpoint{1.228935in}{1.904520in}}%
\pgfpathlineto{\pgfqpoint{1.231745in}{1.943258in}}%
\pgfpathlineto{\pgfqpoint{1.234556in}{1.915897in}}%
\pgfpathlineto{\pgfqpoint{1.237367in}{1.854952in}}%
\pgfpathlineto{\pgfqpoint{1.240177in}{1.847538in}}%
\pgfpathlineto{\pgfqpoint{1.242988in}{1.291472in}}%
\pgfpathlineto{\pgfqpoint{1.245799in}{2.393411in}}%
\pgfpathlineto{\pgfqpoint{1.248609in}{1.224089in}}%
\pgfpathlineto{\pgfqpoint{1.251420in}{1.933449in}}%
\pgfpathlineto{\pgfqpoint{1.254231in}{1.726209in}}%
\pgfpathlineto{\pgfqpoint{1.257041in}{1.451130in}}%
\pgfpathlineto{\pgfqpoint{1.259852in}{1.961348in}}%
\pgfpathlineto{\pgfqpoint{1.262663in}{2.062546in}}%
\pgfpathlineto{\pgfqpoint{1.265474in}{1.658700in}}%
\pgfpathlineto{\pgfqpoint{1.268284in}{1.775197in}}%
\pgfpathlineto{\pgfqpoint{1.271095in}{1.474842in}}%
\pgfpathlineto{\pgfqpoint{1.273906in}{1.488830in}}%
\pgfpathlineto{\pgfqpoint{1.276716in}{1.520651in}}%
\pgfpathlineto{\pgfqpoint{1.279527in}{1.489681in}}%
\pgfpathlineto{\pgfqpoint{1.282338in}{1.940248in}}%
\pgfpathlineto{\pgfqpoint{1.285148in}{1.874220in}}%
\pgfpathlineto{\pgfqpoint{1.287959in}{2.072270in}}%
\pgfpathlineto{\pgfqpoint{1.290770in}{1.824046in}}%
\pgfpathlineto{\pgfqpoint{1.293580in}{1.642645in}}%
\pgfpathlineto{\pgfqpoint{1.296391in}{1.392429in}}%
\pgfpathlineto{\pgfqpoint{1.299202in}{1.736772in}}%
\pgfpathlineto{\pgfqpoint{1.302012in}{1.529194in}}%
\pgfpathlineto{\pgfqpoint{1.304823in}{1.464607in}}%
\pgfpathlineto{\pgfqpoint{1.307634in}{1.634647in}}%
\pgfpathlineto{\pgfqpoint{1.310445in}{2.262044in}}%
\pgfpathlineto{\pgfqpoint{1.313255in}{1.983016in}}%
\pgfpathlineto{\pgfqpoint{1.316066in}{1.897370in}}%
\pgfpathlineto{\pgfqpoint{1.318877in}{1.475793in}}%
\pgfpathlineto{\pgfqpoint{1.321687in}{1.500333in}}%
\pgfpathlineto{\pgfqpoint{1.324498in}{1.910906in}}%
\pgfpathlineto{\pgfqpoint{1.327309in}{1.897157in}}%
\pgfpathlineto{\pgfqpoint{1.330119in}{1.697385in}}%
\pgfpathlineto{\pgfqpoint{1.332930in}{1.616787in}}%
\pgfpathlineto{\pgfqpoint{1.335741in}{1.783334in}}%
\pgfpathlineto{\pgfqpoint{1.338551in}{1.372802in}}%
\pgfpathlineto{\pgfqpoint{1.341362in}{1.762923in}}%
\pgfpathlineto{\pgfqpoint{1.344173in}{1.599243in}}%
\pgfpathlineto{\pgfqpoint{1.346983in}{1.882443in}}%
\pgfpathlineto{\pgfqpoint{1.349794in}{1.697430in}}%
\pgfpathlineto{\pgfqpoint{1.352605in}{1.767519in}}%
\pgfpathlineto{\pgfqpoint{1.355415in}{1.567800in}}%
\pgfpathlineto{\pgfqpoint{1.358226in}{1.632260in}}%
\pgfpathlineto{\pgfqpoint{1.361037in}{1.610158in}}%
\pgfpathlineto{\pgfqpoint{1.363848in}{1.762997in}}%
\pgfpathlineto{\pgfqpoint{1.366658in}{1.577211in}}%
\pgfpathlineto{\pgfqpoint{1.369469in}{1.631545in}}%
\pgfpathlineto{\pgfqpoint{1.372280in}{1.812746in}}%
\pgfpathlineto{\pgfqpoint{1.375090in}{1.664668in}}%
\pgfpathlineto{\pgfqpoint{1.377901in}{1.763047in}}%
\pgfpathlineto{\pgfqpoint{1.380712in}{1.533139in}}%
\pgfpathlineto{\pgfqpoint{1.383522in}{1.883579in}}%
\pgfpathlineto{\pgfqpoint{1.389144in}{1.577612in}}%
\pgfpathlineto{\pgfqpoint{1.391954in}{1.925400in}}%
\pgfpathlineto{\pgfqpoint{1.394765in}{1.713556in}}%
\pgfpathlineto{\pgfqpoint{1.397576in}{1.567800in}}%
\pgfpathlineto{\pgfqpoint{1.400386in}{2.029959in}}%
\pgfpathlineto{\pgfqpoint{1.403197in}{1.923775in}}%
\pgfpathlineto{\pgfqpoint{1.406008in}{1.868776in}}%
\pgfpathlineto{\pgfqpoint{1.408819in}{1.609094in}}%
\pgfpathlineto{\pgfqpoint{1.411629in}{1.645181in}}%
\pgfpathlineto{\pgfqpoint{1.414440in}{1.671159in}}%
\pgfpathlineto{\pgfqpoint{1.417251in}{1.821966in}}%
\pgfpathlineto{\pgfqpoint{1.420061in}{1.671305in}}%
\pgfpathlineto{\pgfqpoint{1.422872in}{1.738601in}}%
\pgfpathlineto{\pgfqpoint{1.425683in}{1.541439in}}%
\pgfpathlineto{\pgfqpoint{1.428493in}{1.790846in}}%
\pgfpathlineto{\pgfqpoint{1.431304in}{1.451683in}}%
\pgfpathlineto{\pgfqpoint{1.434115in}{1.854397in}}%
\pgfpathlineto{\pgfqpoint{1.436925in}{1.738849in}}%
\pgfpathlineto{\pgfqpoint{1.439736in}{1.970068in}}%
\pgfpathlineto{\pgfqpoint{1.442547in}{1.610155in}}%
\pgfpathlineto{\pgfqpoint{1.445357in}{1.976703in}}%
\pgfpathlineto{\pgfqpoint{1.448168in}{1.822451in}}%
\pgfpathlineto{\pgfqpoint{1.450979in}{1.756830in}}%
\pgfpathlineto{\pgfqpoint{1.453790in}{1.521954in}}%
\pgfpathlineto{\pgfqpoint{1.456600in}{1.469147in}}%
\pgfpathlineto{\pgfqpoint{1.459411in}{1.768362in}}%
\pgfpathlineto{\pgfqpoint{1.462222in}{1.873726in}}%
\pgfpathlineto{\pgfqpoint{1.465032in}{1.702056in}}%
\pgfpathlineto{\pgfqpoint{1.467843in}{1.777043in}}%
\pgfpathlineto{\pgfqpoint{1.470654in}{1.831054in}}%
\pgfpathlineto{\pgfqpoint{1.473464in}{1.849414in}}%
\pgfpathlineto{\pgfqpoint{1.476275in}{1.618444in}}%
\pgfpathlineto{\pgfqpoint{1.479086in}{1.479153in}}%
\pgfpathlineto{\pgfqpoint{1.481896in}{1.581979in}}%
\pgfpathlineto{\pgfqpoint{1.484707in}{1.423687in}}%
\pgfpathlineto{\pgfqpoint{1.487518in}{1.676695in}}%
\pgfpathlineto{\pgfqpoint{1.490328in}{1.814225in}}%
\pgfpathlineto{\pgfqpoint{1.495950in}{1.742382in}}%
\pgfpathlineto{\pgfqpoint{1.498761in}{2.254912in}}%
\pgfpathlineto{\pgfqpoint{1.501571in}{1.638586in}}%
\pgfpathlineto{\pgfqpoint{1.504382in}{1.486289in}}%
\pgfpathlineto{\pgfqpoint{1.507193in}{1.951194in}}%
\pgfpathlineto{\pgfqpoint{1.510003in}{1.550746in}}%
\pgfpathlineto{\pgfqpoint{1.512814in}{1.638099in}}%
\pgfpathlineto{\pgfqpoint{1.515625in}{1.785048in}}%
\pgfpathlineto{\pgfqpoint{1.518435in}{1.726156in}}%
\pgfpathlineto{\pgfqpoint{1.521246in}{1.604101in}}%
\pgfpathlineto{\pgfqpoint{1.524057in}{1.564143in}}%
\pgfpathlineto{\pgfqpoint{1.526867in}{1.696961in}}%
\pgfpathlineto{\pgfqpoint{1.529678in}{1.990316in}}%
\pgfpathlineto{\pgfqpoint{1.532489in}{2.008471in}}%
\pgfpathlineto{\pgfqpoint{1.535299in}{1.767781in}}%
\pgfpathlineto{\pgfqpoint{1.538110in}{1.725052in}}%
\pgfpathlineto{\pgfqpoint{1.540921in}{1.846796in}}%
\pgfpathlineto{\pgfqpoint{1.543731in}{1.598405in}}%
\pgfpathlineto{\pgfqpoint{1.546542in}{1.645002in}}%
\pgfpathlineto{\pgfqpoint{1.549353in}{1.743738in}}%
\pgfpathlineto{\pgfqpoint{1.552164in}{1.785613in}}%
\pgfpathlineto{\pgfqpoint{1.554974in}{1.854467in}}%
\pgfpathlineto{\pgfqpoint{1.557785in}{1.580815in}}%
\pgfpathlineto{\pgfqpoint{1.560596in}{1.580017in}}%
\pgfpathlineto{\pgfqpoint{1.563406in}{1.743459in}}%
\pgfpathlineto{\pgfqpoint{1.566217in}{1.733981in}}%
\pgfpathlineto{\pgfqpoint{1.569028in}{1.701332in}}%
\pgfpathlineto{\pgfqpoint{1.571838in}{1.701326in}}%
\pgfpathlineto{\pgfqpoint{1.574649in}{1.719899in}}%
\pgfpathlineto{\pgfqpoint{1.577460in}{1.584892in}}%
\pgfpathlineto{\pgfqpoint{1.583081in}{1.654406in}}%
\pgfpathlineto{\pgfqpoint{1.585892in}{1.934925in}}%
\pgfpathlineto{\pgfqpoint{1.588702in}{1.857998in}}%
\pgfpathlineto{\pgfqpoint{1.591513in}{1.530682in}}%
\pgfpathlineto{\pgfqpoint{1.594324in}{2.031200in}}%
\pgfpathlineto{\pgfqpoint{1.597135in}{1.518772in}}%
\pgfpathlineto{\pgfqpoint{1.599945in}{1.733230in}}%
\pgfpathlineto{\pgfqpoint{1.602756in}{1.545011in}}%
\pgfpathlineto{\pgfqpoint{1.605567in}{1.655053in}}%
\pgfpathlineto{\pgfqpoint{1.608377in}{1.613169in}}%
\pgfpathlineto{\pgfqpoint{1.611188in}{1.995958in}}%
\pgfpathlineto{\pgfqpoint{1.613999in}{2.079381in}}%
\pgfpathlineto{\pgfqpoint{1.616809in}{1.705363in}}%
\pgfpathlineto{\pgfqpoint{1.619620in}{1.815987in}}%
\pgfpathlineto{\pgfqpoint{1.622431in}{1.806258in}}%
\pgfpathlineto{\pgfqpoint{1.625241in}{1.835819in}}%
\pgfpathlineto{\pgfqpoint{1.628052in}{1.928778in}}%
\pgfpathlineto{\pgfqpoint{1.630863in}{1.567636in}}%
\pgfpathlineto{\pgfqpoint{1.633673in}{1.752195in}}%
\pgfpathlineto{\pgfqpoint{1.636484in}{1.644764in}}%
\pgfpathlineto{\pgfqpoint{1.639295in}{1.670494in}}%
\pgfpathlineto{\pgfqpoint{1.642106in}{1.670465in}}%
\pgfpathlineto{\pgfqpoint{1.644916in}{1.661791in}}%
\pgfpathlineto{\pgfqpoint{1.647727in}{1.531246in}}%
\pgfpathlineto{\pgfqpoint{1.650538in}{1.770446in}}%
\pgfpathlineto{\pgfqpoint{1.653348in}{1.692023in}}%
\pgfpathlineto{\pgfqpoint{1.656159in}{1.739755in}}%
\pgfpathlineto{\pgfqpoint{1.658970in}{1.722322in}}%
\pgfpathlineto{\pgfqpoint{1.661780in}{1.346615in}}%
\pgfpathlineto{\pgfqpoint{1.664591in}{1.722939in}}%
\pgfpathlineto{\pgfqpoint{1.667402in}{1.527910in}}%
\pgfpathlineto{\pgfqpoint{1.670212in}{1.979193in}}%
\pgfpathlineto{\pgfqpoint{1.675834in}{1.503440in}}%
\pgfpathlineto{\pgfqpoint{1.678644in}{1.976216in}}%
\pgfpathlineto{\pgfqpoint{1.681455in}{1.817166in}}%
\pgfpathlineto{\pgfqpoint{1.684266in}{1.322347in}}%
\pgfpathlineto{\pgfqpoint{1.687077in}{1.766662in}}%
\pgfpathlineto{\pgfqpoint{1.689887in}{1.542386in}}%
\pgfpathlineto{\pgfqpoint{1.692698in}{1.740592in}}%
\pgfpathlineto{\pgfqpoint{1.695509in}{1.621327in}}%
\pgfpathlineto{\pgfqpoint{1.698319in}{1.811137in}}%
\pgfpathlineto{\pgfqpoint{1.701130in}{1.892888in}}%
\pgfpathlineto{\pgfqpoint{1.703941in}{1.657275in}}%
\pgfpathlineto{\pgfqpoint{1.706751in}{1.473078in}}%
\pgfpathlineto{\pgfqpoint{1.709562in}{1.893565in}}%
\pgfpathlineto{\pgfqpoint{1.712373in}{1.886697in}}%
\pgfpathlineto{\pgfqpoint{1.715183in}{1.339843in}}%
\pgfpathlineto{\pgfqpoint{1.717994in}{1.888659in}}%
\pgfpathlineto{\pgfqpoint{1.720805in}{1.890500in}}%
\pgfpathlineto{\pgfqpoint{1.723615in}{1.571512in}}%
\pgfpathlineto{\pgfqpoint{1.726426in}{1.635753in}}%
\pgfpathlineto{\pgfqpoint{1.729237in}{1.834467in}}%
\pgfpathlineto{\pgfqpoint{1.732048in}{1.769175in}}%
\pgfpathlineto{\pgfqpoint{1.734858in}{1.764540in}}%
\pgfpathlineto{\pgfqpoint{1.737669in}{1.640869in}}%
\pgfpathlineto{\pgfqpoint{1.740480in}{1.704827in}}%
\pgfpathlineto{\pgfqpoint{1.743290in}{1.743135in}}%
\pgfpathlineto{\pgfqpoint{1.746101in}{1.555351in}}%
\pgfpathlineto{\pgfqpoint{1.748912in}{1.653367in}}%
\pgfpathlineto{\pgfqpoint{1.751722in}{1.816239in}}%
\pgfpathlineto{\pgfqpoint{1.754533in}{1.485885in}}%
\pgfpathlineto{\pgfqpoint{1.757344in}{1.644397in}}%
\pgfpathlineto{\pgfqpoint{1.760154in}{1.486877in}}%
\pgfpathlineto{\pgfqpoint{1.762965in}{2.022416in}}%
\pgfpathlineto{\pgfqpoint{1.765776in}{1.540737in}}%
\pgfpathlineto{\pgfqpoint{1.768586in}{1.916265in}}%
\pgfpathlineto{\pgfqpoint{1.771397in}{2.068500in}}%
\pgfpathlineto{\pgfqpoint{1.774208in}{2.018757in}}%
\pgfpathlineto{\pgfqpoint{1.777018in}{1.802141in}}%
\pgfpathlineto{\pgfqpoint{1.779829in}{1.801370in}}%
\pgfpathlineto{\pgfqpoint{1.782640in}{1.460275in}}%
\pgfpathlineto{\pgfqpoint{1.785451in}{1.749263in}}%
\pgfpathlineto{\pgfqpoint{1.788261in}{1.391222in}}%
\pgfpathlineto{\pgfqpoint{1.791072in}{1.853446in}}%
\pgfpathlineto{\pgfqpoint{1.793883in}{1.568000in}}%
\pgfpathlineto{\pgfqpoint{1.796693in}{1.799566in}}%
\pgfpathlineto{\pgfqpoint{1.799504in}{1.547080in}}%
\pgfpathlineto{\pgfqpoint{1.802315in}{1.902780in}}%
\pgfpathlineto{\pgfqpoint{1.805125in}{1.749383in}}%
\pgfpathlineto{\pgfqpoint{1.810747in}{1.264431in}}%
\pgfpathlineto{\pgfqpoint{1.813557in}{2.020054in}}%
\pgfpathlineto{\pgfqpoint{1.816368in}{1.457214in}}%
\pgfpathlineto{\pgfqpoint{1.819179in}{1.611546in}}%
\pgfpathlineto{\pgfqpoint{1.821989in}{1.936400in}}%
\pgfpathlineto{\pgfqpoint{1.824800in}{1.650161in}}%
\pgfpathlineto{\pgfqpoint{1.827611in}{1.896082in}}%
\pgfpathlineto{\pgfqpoint{1.830422in}{1.823860in}}%
\pgfpathlineto{\pgfqpoint{1.833232in}{1.847147in}}%
\pgfpathlineto{\pgfqpoint{1.836043in}{1.553301in}}%
\pgfpathlineto{\pgfqpoint{1.838854in}{1.753376in}}%
\pgfpathlineto{\pgfqpoint{1.841664in}{1.745001in}}%
\pgfpathlineto{\pgfqpoint{1.844475in}{1.462517in}}%
\pgfpathlineto{\pgfqpoint{1.847286in}{2.551440in}}%
\pgfpathlineto{\pgfqpoint{1.852907in}{1.634654in}}%
\pgfpathlineto{\pgfqpoint{1.855718in}{1.772280in}}%
\pgfpathlineto{\pgfqpoint{1.858528in}{1.642389in}}%
\pgfpathlineto{\pgfqpoint{1.861339in}{1.634556in}}%
\pgfpathlineto{\pgfqpoint{1.864150in}{1.707368in}}%
\pgfpathlineto{\pgfqpoint{1.866960in}{1.707353in}}%
\pgfpathlineto{\pgfqpoint{1.872582in}{1.929656in}}%
\pgfpathlineto{\pgfqpoint{1.875393in}{1.522655in}}%
\pgfpathlineto{\pgfqpoint{1.878203in}{2.099146in}}%
\pgfpathlineto{\pgfqpoint{1.881014in}{1.439287in}}%
\pgfpathlineto{\pgfqpoint{1.883825in}{1.688276in}}%
\pgfpathlineto{\pgfqpoint{1.886635in}{1.571628in}}%
\pgfpathlineto{\pgfqpoint{1.889446in}{1.513492in}}%
\pgfpathlineto{\pgfqpoint{1.892257in}{1.896970in}}%
\pgfpathlineto{\pgfqpoint{1.895067in}{1.654272in}}%
\pgfpathlineto{\pgfqpoint{1.897878in}{1.528646in}}%
\pgfpathlineto{\pgfqpoint{1.900689in}{1.942090in}}%
\pgfpathlineto{\pgfqpoint{1.903499in}{1.441956in}}%
\pgfpathlineto{\pgfqpoint{1.906310in}{1.870511in}}%
\pgfpathlineto{\pgfqpoint{1.909121in}{1.692023in}}%
\pgfpathlineto{\pgfqpoint{1.911931in}{1.433120in}}%
\pgfpathlineto{\pgfqpoint{1.914742in}{1.879016in}}%
\pgfpathlineto{\pgfqpoint{1.917553in}{1.741260in}}%
\pgfpathlineto{\pgfqpoint{1.920364in}{1.898630in}}%
\pgfpathlineto{\pgfqpoint{1.923174in}{1.462710in}}%
\pgfpathlineto{\pgfqpoint{1.925985in}{1.646513in}}%
\pgfpathlineto{\pgfqpoint{1.928796in}{1.959380in}}%
\pgfpathlineto{\pgfqpoint{1.931606in}{1.579613in}}%
\pgfpathlineto{\pgfqpoint{1.934417in}{1.811898in}}%
\pgfpathlineto{\pgfqpoint{1.937228in}{1.836886in}}%
\pgfpathlineto{\pgfqpoint{1.942849in}{1.330213in}}%
\pgfpathlineto{\pgfqpoint{1.945660in}{1.783265in}}%
\pgfpathlineto{\pgfqpoint{1.948470in}{1.585522in}}%
\pgfpathlineto{\pgfqpoint{1.951281in}{1.881696in}}%
\pgfpathlineto{\pgfqpoint{1.954092in}{1.879419in}}%
\pgfpathlineto{\pgfqpoint{1.956902in}{1.829261in}}%
\pgfpathlineto{\pgfqpoint{1.959713in}{1.565957in}}%
\pgfpathlineto{\pgfqpoint{1.962524in}{1.899122in}}%
\pgfpathlineto{\pgfqpoint{1.965334in}{1.695696in}}%
\pgfpathlineto{\pgfqpoint{1.970956in}{2.001779in}}%
\pgfpathlineto{\pgfqpoint{1.973767in}{1.469331in}}%
\pgfpathlineto{\pgfqpoint{1.976577in}{1.619504in}}%
\pgfpathlineto{\pgfqpoint{1.979388in}{1.648349in}}%
\pgfpathlineto{\pgfqpoint{1.982199in}{1.963001in}}%
\pgfpathlineto{\pgfqpoint{1.985009in}{1.752796in}}%
\pgfpathlineto{\pgfqpoint{1.987820in}{1.656303in}}%
\pgfpathlineto{\pgfqpoint{1.990631in}{1.620336in}}%
\pgfpathlineto{\pgfqpoint{1.996252in}{1.802845in}}%
\pgfpathlineto{\pgfqpoint{1.999063in}{1.602712in}}%
\pgfpathlineto{\pgfqpoint{2.001873in}{1.799135in}}%
\pgfpathlineto{\pgfqpoint{2.004684in}{1.473451in}}%
\pgfpathlineto{\pgfqpoint{2.007495in}{1.539731in}}%
\pgfpathlineto{\pgfqpoint{2.010305in}{2.337985in}}%
\pgfpathlineto{\pgfqpoint{2.013116in}{1.576226in}}%
\pgfpathlineto{\pgfqpoint{2.015927in}{1.593093in}}%
\pgfpathlineto{\pgfqpoint{2.018738in}{1.649432in}}%
\pgfpathlineto{\pgfqpoint{2.021548in}{0.468908in}}%
\pgfpathlineto{\pgfqpoint{2.024359in}{1.879245in}}%
\pgfpathlineto{\pgfqpoint{2.027170in}{2.108011in}}%
\pgfpathlineto{\pgfqpoint{2.029980in}{1.740025in}}%
\pgfpathlineto{\pgfqpoint{2.032791in}{1.551293in}}%
\pgfpathlineto{\pgfqpoint{2.035602in}{1.561271in}}%
\pgfpathlineto{\pgfqpoint{2.038412in}{1.639414in}}%
\pgfpathlineto{\pgfqpoint{2.041223in}{1.608991in}}%
\pgfpathlineto{\pgfqpoint{2.044034in}{1.688238in}}%
\pgfpathlineto{\pgfqpoint{2.046844in}{1.646535in}}%
\pgfpathlineto{\pgfqpoint{2.049655in}{1.729939in}}%
\pgfpathlineto{\pgfqpoint{2.052466in}{1.300891in}}%
\pgfpathlineto{\pgfqpoint{2.055276in}{1.618069in}}%
\pgfpathlineto{\pgfqpoint{2.058087in}{1.858894in}}%
\pgfpathlineto{\pgfqpoint{2.060898in}{1.532953in}}%
\pgfpathlineto{\pgfqpoint{2.063709in}{2.153094in}}%
\pgfpathlineto{\pgfqpoint{2.066519in}{1.737390in}}%
\pgfpathlineto{\pgfqpoint{2.069330in}{1.707116in}}%
\pgfpathlineto{\pgfqpoint{2.072141in}{1.333421in}}%
\pgfpathlineto{\pgfqpoint{2.074951in}{1.614650in}}%
\pgfpathlineto{\pgfqpoint{2.077762in}{1.784825in}}%
\pgfpathlineto{\pgfqpoint{2.080573in}{1.707436in}}%
\pgfpathlineto{\pgfqpoint{2.083383in}{1.607060in}}%
\pgfpathlineto{\pgfqpoint{2.086194in}{1.922660in}}%
\pgfpathlineto{\pgfqpoint{2.089005in}{1.619356in}}%
\pgfpathlineto{\pgfqpoint{2.091815in}{1.695856in}}%
\pgfpathlineto{\pgfqpoint{2.094626in}{1.730301in}}%
\pgfpathlineto{\pgfqpoint{2.097437in}{1.859342in}}%
\pgfpathlineto{\pgfqpoint{2.100247in}{2.209113in}}%
\pgfpathlineto{\pgfqpoint{2.103058in}{1.837706in}}%
\pgfpathlineto{\pgfqpoint{2.105869in}{1.586539in}}%
\pgfpathlineto{\pgfqpoint{2.108680in}{2.020661in}}%
\pgfpathlineto{\pgfqpoint{2.111490in}{1.724153in}}%
\pgfpathlineto{\pgfqpoint{2.114301in}{1.872853in}}%
\pgfpathlineto{\pgfqpoint{2.117112in}{1.860323in}}%
\pgfpathlineto{\pgfqpoint{2.119922in}{1.775497in}}%
\pgfpathlineto{\pgfqpoint{2.122733in}{2.018128in}}%
\pgfpathlineto{\pgfqpoint{2.125544in}{1.486856in}}%
\pgfpathlineto{\pgfqpoint{2.128354in}{1.459687in}}%
\pgfpathlineto{\pgfqpoint{2.131165in}{1.548122in}}%
\pgfpathlineto{\pgfqpoint{2.133976in}{1.853382in}}%
\pgfpathlineto{\pgfqpoint{2.136786in}{1.653588in}}%
\pgfpathlineto{\pgfqpoint{2.139597in}{1.533798in}}%
\pgfpathlineto{\pgfqpoint{2.142408in}{1.864235in}}%
\pgfpathlineto{\pgfqpoint{2.145218in}{1.576253in}}%
\pgfpathlineto{\pgfqpoint{2.148029in}{1.432832in}}%
\pgfpathlineto{\pgfqpoint{2.150840in}{1.841672in}}%
\pgfpathlineto{\pgfqpoint{2.153651in}{1.347835in}}%
\pgfpathlineto{\pgfqpoint{2.156461in}{1.373407in}}%
\pgfpathlineto{\pgfqpoint{2.159272in}{1.802627in}}%
\pgfpathlineto{\pgfqpoint{2.162083in}{2.126762in}}%
\pgfpathlineto{\pgfqpoint{2.167704in}{1.793601in}}%
\pgfpathlineto{\pgfqpoint{2.170515in}{1.523534in}}%
\pgfpathlineto{\pgfqpoint{2.173325in}{2.034108in}}%
\pgfpathlineto{\pgfqpoint{2.176136in}{1.904634in}}%
\pgfpathlineto{\pgfqpoint{2.178947in}{1.861399in}}%
\pgfpathlineto{\pgfqpoint{2.181757in}{1.658295in}}%
\pgfpathlineto{\pgfqpoint{2.184568in}{1.688646in}}%
\pgfpathlineto{\pgfqpoint{2.187379in}{1.607363in}}%
\pgfpathlineto{\pgfqpoint{2.190189in}{2.004660in}}%
\pgfpathlineto{\pgfqpoint{2.193000in}{1.702004in}}%
\pgfpathlineto{\pgfqpoint{2.195811in}{1.692023in}}%
\pgfpathlineto{\pgfqpoint{2.198621in}{1.781571in}}%
\pgfpathlineto{\pgfqpoint{2.201432in}{1.662231in}}%
\pgfpathlineto{\pgfqpoint{2.204243in}{1.132071in}}%
\pgfpathlineto{\pgfqpoint{2.207054in}{1.889960in}}%
\pgfpathlineto{\pgfqpoint{2.209864in}{1.476904in}}%
\pgfpathlineto{\pgfqpoint{2.212675in}{1.767481in}}%
\pgfpathlineto{\pgfqpoint{2.215486in}{1.821507in}}%
\pgfpathlineto{\pgfqpoint{2.218296in}{1.459551in}}%
\pgfpathlineto{\pgfqpoint{2.221107in}{1.866701in}}%
\pgfpathlineto{\pgfqpoint{2.223918in}{1.766775in}}%
\pgfpathlineto{\pgfqpoint{2.226728in}{1.613865in}}%
\pgfpathlineto{\pgfqpoint{2.229539in}{1.938752in}}%
\pgfpathlineto{\pgfqpoint{2.232350in}{1.718826in}}%
\pgfpathlineto{\pgfqpoint{2.235160in}{1.835306in}}%
\pgfpathlineto{\pgfqpoint{2.237971in}{1.531993in}}%
\pgfpathlineto{\pgfqpoint{2.240782in}{1.493044in}}%
\pgfpathlineto{\pgfqpoint{2.246403in}{1.963557in}}%
\pgfpathlineto{\pgfqpoint{2.249214in}{1.731851in}}%
\pgfpathlineto{\pgfqpoint{2.252025in}{1.678758in}}%
\pgfpathlineto{\pgfqpoint{2.254835in}{1.814295in}}%
\pgfpathlineto{\pgfqpoint{2.260457in}{1.649271in}}%
\pgfpathlineto{\pgfqpoint{2.263267in}{1.819929in}}%
\pgfpathlineto{\pgfqpoint{2.266078in}{1.435153in}}%
\pgfpathlineto{\pgfqpoint{2.268889in}{1.768207in}}%
\pgfpathlineto{\pgfqpoint{2.271699in}{1.612518in}}%
\pgfpathlineto{\pgfqpoint{2.274510in}{1.701983in}}%
\pgfpathlineto{\pgfqpoint{2.277321in}{1.672096in}}%
\pgfpathlineto{\pgfqpoint{2.280131in}{1.524968in}}%
\pgfpathlineto{\pgfqpoint{2.282942in}{2.171639in}}%
\pgfpathlineto{\pgfqpoint{2.285753in}{1.521713in}}%
\pgfpathlineto{\pgfqpoint{2.288563in}{1.988857in}}%
\pgfpathlineto{\pgfqpoint{2.291374in}{1.721076in}}%
\pgfpathlineto{\pgfqpoint{2.294185in}{2.102609in}}%
\pgfpathlineto{\pgfqpoint{2.296996in}{1.842083in}}%
\pgfpathlineto{\pgfqpoint{2.299806in}{1.744825in}}%
\pgfpathlineto{\pgfqpoint{2.302617in}{1.685820in}}%
\pgfpathlineto{\pgfqpoint{2.305428in}{1.840213in}}%
\pgfpathlineto{\pgfqpoint{2.308238in}{1.741109in}}%
\pgfpathlineto{\pgfqpoint{2.311049in}{1.795820in}}%
\pgfpathlineto{\pgfqpoint{2.313860in}{1.655467in}}%
\pgfpathlineto{\pgfqpoint{2.316670in}{1.780219in}}%
\pgfpathlineto{\pgfqpoint{2.319481in}{1.815870in}}%
\pgfpathlineto{\pgfqpoint{2.322292in}{1.574240in}}%
\pgfpathlineto{\pgfqpoint{2.325102in}{1.704144in}}%
\pgfpathlineto{\pgfqpoint{2.327913in}{1.597836in}}%
\pgfpathlineto{\pgfqpoint{2.330724in}{1.810424in}}%
\pgfpathlineto{\pgfqpoint{2.336345in}{1.670903in}}%
\pgfpathlineto{\pgfqpoint{2.339156in}{1.637583in}}%
\pgfpathlineto{\pgfqpoint{2.341967in}{1.533676in}}%
\pgfpathlineto{\pgfqpoint{2.344777in}{1.959060in}}%
\pgfpathlineto{\pgfqpoint{2.350399in}{1.553675in}}%
\pgfpathlineto{\pgfqpoint{2.353209in}{2.409367in}}%
\pgfpathlineto{\pgfqpoint{2.356020in}{1.674702in}}%
\pgfpathlineto{\pgfqpoint{2.358831in}{1.761191in}}%
\pgfpathlineto{\pgfqpoint{2.361641in}{1.372452in}}%
\pgfpathlineto{\pgfqpoint{2.364452in}{1.206109in}}%
\pgfpathlineto{\pgfqpoint{2.367263in}{1.333668in}}%
\pgfpathlineto{\pgfqpoint{2.370073in}{2.029168in}}%
\pgfpathlineto{\pgfqpoint{2.372884in}{1.419807in}}%
\pgfpathlineto{\pgfqpoint{2.375695in}{1.958173in}}%
\pgfpathlineto{\pgfqpoint{2.378505in}{1.301965in}}%
\pgfpathlineto{\pgfqpoint{2.384127in}{1.770281in}}%
\pgfpathlineto{\pgfqpoint{2.386937in}{1.769890in}}%
\pgfpathlineto{\pgfqpoint{2.389748in}{1.935561in}}%
\pgfpathlineto{\pgfqpoint{2.392559in}{1.886570in}}%
\pgfpathlineto{\pgfqpoint{2.395370in}{1.607207in}}%
\pgfpathlineto{\pgfqpoint{2.398180in}{1.816062in}}%
\pgfpathlineto{\pgfqpoint{2.400991in}{1.934215in}}%
\pgfpathlineto{\pgfqpoint{2.403802in}{1.683120in}}%
\pgfpathlineto{\pgfqpoint{2.406612in}{1.827986in}}%
\pgfpathlineto{\pgfqpoint{2.412234in}{1.541238in}}%
\pgfpathlineto{\pgfqpoint{2.415044in}{1.662286in}}%
\pgfpathlineto{\pgfqpoint{2.417855in}{1.686069in}}%
\pgfpathlineto{\pgfqpoint{2.420666in}{1.887324in}}%
\pgfpathlineto{\pgfqpoint{2.426287in}{1.621579in}}%
\pgfpathlineto{\pgfqpoint{2.429098in}{1.703786in}}%
\pgfpathlineto{\pgfqpoint{2.431908in}{1.668489in}}%
\pgfpathlineto{\pgfqpoint{2.434719in}{1.379801in}}%
\pgfpathlineto{\pgfqpoint{2.437530in}{1.974777in}}%
\pgfpathlineto{\pgfqpoint{2.440341in}{1.505085in}}%
\pgfpathlineto{\pgfqpoint{2.443151in}{1.763502in}}%
\pgfpathlineto{\pgfqpoint{2.445962in}{1.810431in}}%
\pgfpathlineto{\pgfqpoint{2.448773in}{1.689074in}}%
\pgfpathlineto{\pgfqpoint{2.451583in}{1.806634in}}%
\pgfpathlineto{\pgfqpoint{2.454394in}{1.618651in}}%
\pgfpathlineto{\pgfqpoint{2.457205in}{1.319899in}}%
\pgfpathlineto{\pgfqpoint{2.460015in}{1.692023in}}%
\pgfpathlineto{\pgfqpoint{2.462826in}{1.904477in}}%
\pgfpathlineto{\pgfqpoint{2.465637in}{1.872272in}}%
\pgfpathlineto{\pgfqpoint{2.468447in}{1.514746in}}%
\pgfpathlineto{\pgfqpoint{2.471258in}{1.551735in}}%
\pgfpathlineto{\pgfqpoint{2.474069in}{1.799590in}}%
\pgfpathlineto{\pgfqpoint{2.476879in}{1.509318in}}%
\pgfpathlineto{\pgfqpoint{2.479690in}{1.522389in}}%
\pgfpathlineto{\pgfqpoint{2.482501in}{1.498955in}}%
\pgfpathlineto{\pgfqpoint{2.485312in}{1.698189in}}%
\pgfpathlineto{\pgfqpoint{2.488122in}{1.421524in}}%
\pgfpathlineto{\pgfqpoint{2.490933in}{1.968685in}}%
\pgfpathlineto{\pgfqpoint{2.493744in}{1.608609in}}%
\pgfpathlineto{\pgfqpoint{2.496554in}{1.688925in}}%
\pgfpathlineto{\pgfqpoint{2.499365in}{1.716790in}}%
\pgfpathlineto{\pgfqpoint{2.502176in}{1.150679in}}%
\pgfpathlineto{\pgfqpoint{2.507797in}{1.626492in}}%
\pgfpathlineto{\pgfqpoint{2.510608in}{2.071532in}}%
\pgfpathlineto{\pgfqpoint{2.513418in}{1.233512in}}%
\pgfpathlineto{\pgfqpoint{2.516229in}{1.307822in}}%
\pgfpathlineto{\pgfqpoint{2.519040in}{2.033264in}}%
\pgfpathlineto{\pgfqpoint{2.521850in}{1.928484in}}%
\pgfpathlineto{\pgfqpoint{2.524661in}{2.091047in}}%
\pgfpathlineto{\pgfqpoint{2.527472in}{1.586823in}}%
\pgfpathlineto{\pgfqpoint{2.530283in}{1.781329in}}%
\pgfpathlineto{\pgfqpoint{2.533093in}{1.755501in}}%
\pgfpathlineto{\pgfqpoint{2.535904in}{1.606266in}}%
\pgfpathlineto{\pgfqpoint{2.538715in}{1.733372in}}%
\pgfpathlineto{\pgfqpoint{2.541525in}{0.893706in}}%
\pgfpathlineto{\pgfqpoint{2.544336in}{1.884735in}}%
\pgfpathlineto{\pgfqpoint{2.549957in}{1.688741in}}%
\pgfpathlineto{\pgfqpoint{2.552768in}{1.958912in}}%
\pgfpathlineto{\pgfqpoint{2.555579in}{1.562403in}}%
\pgfpathlineto{\pgfqpoint{2.558389in}{1.898901in}}%
\pgfpathlineto{\pgfqpoint{2.561200in}{1.556569in}}%
\pgfpathlineto{\pgfqpoint{2.564011in}{1.945839in}}%
\pgfpathlineto{\pgfqpoint{2.566821in}{1.853713in}}%
\pgfpathlineto{\pgfqpoint{2.569632in}{1.685714in}}%
\pgfpathlineto{\pgfqpoint{2.572443in}{1.758144in}}%
\pgfpathlineto{\pgfqpoint{2.575253in}{1.720275in}}%
\pgfpathlineto{\pgfqpoint{2.578064in}{1.644908in}}%
\pgfpathlineto{\pgfqpoint{2.580875in}{1.511676in}}%
\pgfpathlineto{\pgfqpoint{2.583686in}{1.869223in}}%
\pgfpathlineto{\pgfqpoint{2.586496in}{1.732873in}}%
\pgfpathlineto{\pgfqpoint{2.589307in}{1.512114in}}%
\pgfpathlineto{\pgfqpoint{2.592118in}{1.853091in}}%
\pgfpathlineto{\pgfqpoint{2.594928in}{1.638518in}}%
\pgfpathlineto{\pgfqpoint{2.597739in}{1.892506in}}%
\pgfpathlineto{\pgfqpoint{2.600550in}{1.868435in}}%
\pgfpathlineto{\pgfqpoint{2.603360in}{1.642705in}}%
\pgfpathlineto{\pgfqpoint{2.606171in}{1.753646in}}%
\pgfpathlineto{\pgfqpoint{2.608982in}{1.704318in}}%
\pgfpathlineto{\pgfqpoint{2.611792in}{1.596478in}}%
\pgfpathlineto{\pgfqpoint{2.614603in}{1.530429in}}%
\pgfpathlineto{\pgfqpoint{2.617414in}{1.701391in}}%
\pgfpathlineto{\pgfqpoint{2.620224in}{1.748115in}}%
\pgfpathlineto{\pgfqpoint{2.623035in}{1.751014in}}%
\pgfpathlineto{\pgfqpoint{2.625846in}{1.661003in}}%
\pgfpathlineto{\pgfqpoint{2.628657in}{1.812654in}}%
\pgfpathlineto{\pgfqpoint{2.631467in}{1.577601in}}%
\pgfpathlineto{\pgfqpoint{2.634278in}{1.626706in}}%
\pgfpathlineto{\pgfqpoint{2.637089in}{1.654576in}}%
\pgfpathlineto{\pgfqpoint{2.639899in}{1.613718in}}%
\pgfpathlineto{\pgfqpoint{2.642710in}{1.735922in}}%
\pgfpathlineto{\pgfqpoint{2.645521in}{1.723304in}}%
\pgfpathlineto{\pgfqpoint{2.648331in}{1.623122in}}%
\pgfpathlineto{\pgfqpoint{2.651142in}{1.632271in}}%
\pgfpathlineto{\pgfqpoint{2.653953in}{1.688872in}}%
\pgfpathlineto{\pgfqpoint{2.656763in}{1.568629in}}%
\pgfpathlineto{\pgfqpoint{2.659574in}{1.787028in}}%
\pgfpathlineto{\pgfqpoint{2.662385in}{1.682548in}}%
\pgfpathlineto{\pgfqpoint{2.665195in}{1.717276in}}%
\pgfpathlineto{\pgfqpoint{2.668006in}{1.798893in}}%
\pgfpathlineto{\pgfqpoint{2.670817in}{1.949879in}}%
\pgfpathlineto{\pgfqpoint{2.673628in}{1.735100in}}%
\pgfpathlineto{\pgfqpoint{2.676438in}{1.814449in}}%
\pgfpathlineto{\pgfqpoint{2.682060in}{1.585058in}}%
\pgfpathlineto{\pgfqpoint{2.684870in}{1.756290in}}%
\pgfpathlineto{\pgfqpoint{2.687681in}{1.649208in}}%
\pgfpathlineto{\pgfqpoint{2.690492in}{1.795800in}}%
\pgfpathlineto{\pgfqpoint{2.693302in}{1.981369in}}%
\pgfpathlineto{\pgfqpoint{2.696113in}{1.742707in}}%
\pgfpathlineto{\pgfqpoint{2.698924in}{1.754406in}}%
\pgfpathlineto{\pgfqpoint{2.701734in}{1.353365in}}%
\pgfpathlineto{\pgfqpoint{2.704545in}{1.857775in}}%
\pgfpathlineto{\pgfqpoint{2.707356in}{1.622923in}}%
\pgfpathlineto{\pgfqpoint{2.710166in}{1.835887in}}%
\pgfpathlineto{\pgfqpoint{2.712977in}{1.689039in}}%
\pgfpathlineto{\pgfqpoint{2.715788in}{1.799078in}}%
\pgfpathlineto{\pgfqpoint{2.718599in}{1.124684in}}%
\pgfpathlineto{\pgfqpoint{2.721409in}{1.652019in}}%
\pgfpathlineto{\pgfqpoint{2.724220in}{1.636463in}}%
\pgfpathlineto{\pgfqpoint{2.727031in}{1.735253in}}%
\pgfpathlineto{\pgfqpoint{2.729841in}{1.468409in}}%
\pgfpathlineto{\pgfqpoint{2.732652in}{1.751344in}}%
\pgfpathlineto{\pgfqpoint{2.735463in}{1.716933in}}%
\pgfpathlineto{\pgfqpoint{2.738273in}{1.573345in}}%
\pgfpathlineto{\pgfqpoint{2.741084in}{1.701425in}}%
\pgfpathlineto{\pgfqpoint{2.743895in}{1.569350in}}%
\pgfpathlineto{\pgfqpoint{2.746705in}{1.799022in}}%
\pgfpathlineto{\pgfqpoint{2.749516in}{1.698294in}}%
\pgfpathlineto{\pgfqpoint{2.752327in}{1.673201in}}%
\pgfpathlineto{\pgfqpoint{2.755137in}{1.894723in}}%
\pgfpathlineto{\pgfqpoint{2.757948in}{1.772372in}}%
\pgfpathlineto{\pgfqpoint{2.760759in}{1.518447in}}%
\pgfpathlineto{\pgfqpoint{2.763570in}{1.997581in}}%
\pgfpathlineto{\pgfqpoint{2.766380in}{1.740849in}}%
\pgfpathlineto{\pgfqpoint{2.769191in}{1.780133in}}%
\pgfpathlineto{\pgfqpoint{2.772002in}{1.658660in}}%
\pgfpathlineto{\pgfqpoint{2.774812in}{1.716294in}}%
\pgfpathlineto{\pgfqpoint{2.777623in}{1.685959in}}%
\pgfpathlineto{\pgfqpoint{2.780434in}{1.743493in}}%
\pgfpathlineto{\pgfqpoint{2.783244in}{1.722221in}}%
\pgfpathlineto{\pgfqpoint{2.786055in}{1.506902in}}%
\pgfpathlineto{\pgfqpoint{2.788866in}{1.541712in}}%
\pgfpathlineto{\pgfqpoint{2.791676in}{1.851490in}}%
\pgfpathlineto{\pgfqpoint{2.794487in}{1.707271in}}%
\pgfpathlineto{\pgfqpoint{2.797298in}{1.643177in}}%
\pgfpathlineto{\pgfqpoint{2.800108in}{1.688965in}}%
\pgfpathlineto{\pgfqpoint{2.802919in}{1.804769in}}%
\pgfpathlineto{\pgfqpoint{2.805730in}{1.585392in}}%
\pgfpathlineto{\pgfqpoint{2.808540in}{1.880417in}}%
\pgfpathlineto{\pgfqpoint{2.811351in}{1.552465in}}%
\pgfpathlineto{\pgfqpoint{2.814162in}{1.624834in}}%
\pgfpathlineto{\pgfqpoint{2.816973in}{1.734813in}}%
\pgfpathlineto{\pgfqpoint{2.819783in}{1.886154in}}%
\pgfpathlineto{\pgfqpoint{2.822594in}{1.564897in}}%
\pgfpathlineto{\pgfqpoint{2.825405in}{1.755715in}}%
\pgfpathlineto{\pgfqpoint{2.828215in}{1.679911in}}%
\pgfpathlineto{\pgfqpoint{2.831026in}{1.518400in}}%
\pgfpathlineto{\pgfqpoint{2.833837in}{1.587531in}}%
\pgfpathlineto{\pgfqpoint{2.836647in}{1.836284in}}%
\pgfpathlineto{\pgfqpoint{2.839458in}{1.420820in}}%
\pgfpathlineto{\pgfqpoint{2.842269in}{1.781918in}}%
\pgfpathlineto{\pgfqpoint{2.845079in}{1.617664in}}%
\pgfpathlineto{\pgfqpoint{2.847890in}{1.871122in}}%
\pgfpathlineto{\pgfqpoint{2.850701in}{1.444449in}}%
\pgfpathlineto{\pgfqpoint{2.856322in}{1.859799in}}%
\pgfpathlineto{\pgfqpoint{2.859133in}{1.676564in}}%
\pgfpathlineto{\pgfqpoint{2.861944in}{1.414254in}}%
\pgfpathlineto{\pgfqpoint{2.864754in}{2.013039in}}%
\pgfpathlineto{\pgfqpoint{2.867565in}{1.345795in}}%
\pgfpathlineto{\pgfqpoint{2.873186in}{1.640803in}}%
\pgfpathlineto{\pgfqpoint{2.878808in}{1.503447in}}%
\pgfpathlineto{\pgfqpoint{2.881618in}{1.903205in}}%
\pgfpathlineto{\pgfqpoint{2.884429in}{1.922660in}}%
\pgfpathlineto{\pgfqpoint{2.887240in}{1.831308in}}%
\pgfpathlineto{\pgfqpoint{2.890050in}{2.096430in}}%
\pgfpathlineto{\pgfqpoint{2.892861in}{1.515995in}}%
\pgfpathlineto{\pgfqpoint{2.895672in}{1.938522in}}%
\pgfpathlineto{\pgfqpoint{2.898482in}{1.633831in}}%
\pgfpathlineto{\pgfqpoint{2.901293in}{1.685885in}}%
\pgfpathlineto{\pgfqpoint{2.904104in}{1.932661in}}%
\pgfpathlineto{\pgfqpoint{2.906915in}{1.543204in}}%
\pgfpathlineto{\pgfqpoint{2.909725in}{3.214362in}}%
\pgfpathlineto{\pgfqpoint{2.912536in}{2.004430in}}%
\pgfpathlineto{\pgfqpoint{2.915347in}{1.697448in}}%
\pgfpathlineto{\pgfqpoint{2.918157in}{1.783967in}}%
\pgfpathlineto{\pgfqpoint{2.920968in}{2.109701in}}%
\pgfpathlineto{\pgfqpoint{2.923779in}{1.726112in}}%
\pgfpathlineto{\pgfqpoint{2.926589in}{1.838022in}}%
\pgfpathlineto{\pgfqpoint{2.929400in}{1.553880in}}%
\pgfpathlineto{\pgfqpoint{2.932211in}{1.660575in}}%
\pgfpathlineto{\pgfqpoint{2.935021in}{1.952175in}}%
\pgfpathlineto{\pgfqpoint{2.937832in}{1.627390in}}%
\pgfpathlineto{\pgfqpoint{2.940643in}{1.561949in}}%
\pgfpathlineto{\pgfqpoint{2.943453in}{1.751991in}}%
\pgfpathlineto{\pgfqpoint{2.946264in}{1.686817in}}%
\pgfpathlineto{\pgfqpoint{2.949075in}{1.702432in}}%
\pgfpathlineto{\pgfqpoint{2.951886in}{1.803487in}}%
\pgfpathlineto{\pgfqpoint{2.954696in}{1.848792in}}%
\pgfpathlineto{\pgfqpoint{2.957507in}{1.717573in}}%
\pgfpathlineto{\pgfqpoint{2.960318in}{1.852035in}}%
\pgfpathlineto{\pgfqpoint{2.963128in}{1.692023in}}%
\pgfpathlineto{\pgfqpoint{2.965939in}{1.747516in}}%
\pgfpathlineto{\pgfqpoint{2.968750in}{1.692023in}}%
\pgfpathlineto{\pgfqpoint{2.971560in}{1.802420in}}%
\pgfpathlineto{\pgfqpoint{2.974371in}{1.786742in}}%
\pgfpathlineto{\pgfqpoint{2.977182in}{1.664663in}}%
\pgfpathlineto{\pgfqpoint{2.979992in}{1.823407in}}%
\pgfpathlineto{\pgfqpoint{2.982803in}{1.677205in}}%
\pgfpathlineto{\pgfqpoint{2.985614in}{1.753673in}}%
\pgfpathlineto{\pgfqpoint{2.988424in}{1.548620in}}%
\pgfpathlineto{\pgfqpoint{2.991235in}{1.776247in}}%
\pgfpathlineto{\pgfqpoint{2.994046in}{1.311916in}}%
\pgfpathlineto{\pgfqpoint{2.996856in}{1.686959in}}%
\pgfpathlineto{\pgfqpoint{2.999667in}{1.490684in}}%
\pgfpathlineto{\pgfqpoint{3.002478in}{2.037036in}}%
\pgfpathlineto{\pgfqpoint{3.005289in}{1.995161in}}%
\pgfpathlineto{\pgfqpoint{3.008099in}{1.543675in}}%
\pgfpathlineto{\pgfqpoint{3.010910in}{1.842832in}}%
\pgfpathlineto{\pgfqpoint{3.013721in}{1.753411in}}%
\pgfpathlineto{\pgfqpoint{3.016531in}{1.831092in}}%
\pgfpathlineto{\pgfqpoint{3.019342in}{1.633619in}}%
\pgfpathlineto{\pgfqpoint{3.022153in}{1.621158in}}%
\pgfpathlineto{\pgfqpoint{3.024963in}{1.640505in}}%
\pgfpathlineto{\pgfqpoint{3.027774in}{1.553805in}}%
\pgfpathlineto{\pgfqpoint{3.030585in}{1.857247in}}%
\pgfpathlineto{\pgfqpoint{3.033395in}{1.344765in}}%
\pgfpathlineto{\pgfqpoint{3.036206in}{1.591370in}}%
\pgfpathlineto{\pgfqpoint{3.039017in}{1.900158in}}%
\pgfpathlineto{\pgfqpoint{3.041827in}{1.899877in}}%
\pgfpathlineto{\pgfqpoint{3.044638in}{1.459239in}}%
\pgfpathlineto{\pgfqpoint{3.047449in}{1.659554in}}%
\pgfpathlineto{\pgfqpoint{3.050260in}{1.739454in}}%
\pgfpathlineto{\pgfqpoint{3.053070in}{1.374753in}}%
\pgfpathlineto{\pgfqpoint{3.055881in}{1.543781in}}%
\pgfpathlineto{\pgfqpoint{3.058692in}{1.804611in}}%
\pgfpathlineto{\pgfqpoint{3.061502in}{1.806335in}}%
\pgfpathlineto{\pgfqpoint{3.064313in}{1.719839in}}%
\pgfpathlineto{\pgfqpoint{3.067124in}{1.709698in}}%
\pgfpathlineto{\pgfqpoint{3.069934in}{1.734866in}}%
\pgfpathlineto{\pgfqpoint{3.072745in}{1.585961in}}%
\pgfpathlineto{\pgfqpoint{3.075556in}{1.351326in}}%
\pgfpathlineto{\pgfqpoint{3.078366in}{1.399236in}}%
\pgfpathlineto{\pgfqpoint{3.083988in}{2.121173in}}%
\pgfpathlineto{\pgfqpoint{3.086798in}{1.717502in}}%
\pgfpathlineto{\pgfqpoint{3.089609in}{1.962080in}}%
\pgfpathlineto{\pgfqpoint{3.092420in}{1.994342in}}%
\pgfpathlineto{\pgfqpoint{3.095231in}{2.094190in}}%
\pgfpathlineto{\pgfqpoint{3.098041in}{1.438847in}}%
\pgfpathlineto{\pgfqpoint{3.100852in}{1.577352in}}%
\pgfpathlineto{\pgfqpoint{3.103663in}{1.638056in}}%
\pgfpathlineto{\pgfqpoint{3.106473in}{1.889803in}}%
\pgfpathlineto{\pgfqpoint{3.109284in}{1.978117in}}%
\pgfpathlineto{\pgfqpoint{3.112095in}{1.617998in}}%
\pgfpathlineto{\pgfqpoint{3.114905in}{1.763666in}}%
\pgfpathlineto{\pgfqpoint{3.117716in}{1.591631in}}%
\pgfpathlineto{\pgfqpoint{3.120527in}{1.692023in}}%
\pgfpathlineto{\pgfqpoint{3.123337in}{1.915822in}}%
\pgfpathlineto{\pgfqpoint{3.126148in}{1.692023in}}%
\pgfpathlineto{\pgfqpoint{3.128959in}{1.661263in}}%
\pgfpathlineto{\pgfqpoint{3.131769in}{1.722783in}}%
\pgfpathlineto{\pgfqpoint{3.134580in}{1.734514in}}%
\pgfpathlineto{\pgfqpoint{3.137391in}{1.552312in}}%
\pgfpathlineto{\pgfqpoint{3.140202in}{2.088920in}}%
\pgfpathlineto{\pgfqpoint{3.143012in}{1.524163in}}%
\pgfpathlineto{\pgfqpoint{3.145823in}{1.602697in}}%
\pgfpathlineto{\pgfqpoint{3.148634in}{1.713225in}}%
\pgfpathlineto{\pgfqpoint{3.151444in}{1.416516in}}%
\pgfpathlineto{\pgfqpoint{3.154255in}{1.813751in}}%
\pgfpathlineto{\pgfqpoint{3.157066in}{1.328822in}}%
\pgfpathlineto{\pgfqpoint{3.162687in}{1.974659in}}%
\pgfpathlineto{\pgfqpoint{3.165498in}{1.426442in}}%
\pgfpathlineto{\pgfqpoint{3.168308in}{1.926452in}}%
\pgfpathlineto{\pgfqpoint{3.171119in}{1.425907in}}%
\pgfpathlineto{\pgfqpoint{3.173930in}{1.878780in}}%
\pgfpathlineto{\pgfqpoint{3.176740in}{1.667895in}}%
\pgfpathlineto{\pgfqpoint{3.179551in}{1.831450in}}%
\pgfpathlineto{\pgfqpoint{3.182362in}{1.615252in}}%
\pgfpathlineto{\pgfqpoint{3.185173in}{1.696832in}}%
\pgfpathlineto{\pgfqpoint{3.187983in}{1.368933in}}%
\pgfpathlineto{\pgfqpoint{3.190794in}{1.660081in}}%
\pgfpathlineto{\pgfqpoint{3.193605in}{1.679720in}}%
\pgfpathlineto{\pgfqpoint{3.196415in}{1.719076in}}%
\pgfpathlineto{\pgfqpoint{3.199226in}{1.632937in}}%
\pgfpathlineto{\pgfqpoint{3.202037in}{1.637663in}}%
\pgfpathlineto{\pgfqpoint{3.204847in}{1.719226in}}%
\pgfpathlineto{\pgfqpoint{3.207658in}{1.669769in}}%
\pgfpathlineto{\pgfqpoint{3.210469in}{1.834883in}}%
\pgfpathlineto{\pgfqpoint{3.213279in}{1.857866in}}%
\pgfpathlineto{\pgfqpoint{3.216090in}{1.677460in}}%
\pgfpathlineto{\pgfqpoint{3.218901in}{1.667722in}}%
\pgfpathlineto{\pgfqpoint{3.221711in}{1.520848in}}%
\pgfpathlineto{\pgfqpoint{3.224522in}{1.672340in}}%
\pgfpathlineto{\pgfqpoint{3.227333in}{1.726452in}}%
\pgfpathlineto{\pgfqpoint{3.230143in}{1.684652in}}%
\pgfpathlineto{\pgfqpoint{3.232954in}{1.421812in}}%
\pgfpathlineto{\pgfqpoint{3.235765in}{1.739462in}}%
\pgfpathlineto{\pgfqpoint{3.238576in}{1.843359in}}%
\pgfpathlineto{\pgfqpoint{3.241386in}{2.316351in}}%
\pgfpathlineto{\pgfqpoint{3.244197in}{1.625462in}}%
\pgfpathlineto{\pgfqpoint{3.247008in}{1.634744in}}%
\pgfpathlineto{\pgfqpoint{3.249818in}{1.605708in}}%
\pgfpathlineto{\pgfqpoint{3.252629in}{1.614896in}}%
\pgfpathlineto{\pgfqpoint{3.255440in}{1.824350in}}%
\pgfpathlineto{\pgfqpoint{3.258250in}{1.389639in}}%
\pgfpathlineto{\pgfqpoint{3.261061in}{1.625931in}}%
\pgfpathlineto{\pgfqpoint{3.263872in}{1.647806in}}%
\pgfpathlineto{\pgfqpoint{3.266682in}{1.780332in}}%
\pgfpathlineto{\pgfqpoint{3.269493in}{1.623381in}}%
\pgfpathlineto{\pgfqpoint{3.272304in}{1.906745in}}%
\pgfpathlineto{\pgfqpoint{3.275114in}{2.356214in}}%
\pgfpathlineto{\pgfqpoint{3.277925in}{1.589504in}}%
\pgfpathlineto{\pgfqpoint{3.280736in}{1.619387in}}%
\pgfpathlineto{\pgfqpoint{3.283547in}{1.708454in}}%
\pgfpathlineto{\pgfqpoint{3.286357in}{1.996372in}}%
\pgfpathlineto{\pgfqpoint{3.289168in}{1.595090in}}%
\pgfpathlineto{\pgfqpoint{3.291979in}{1.788956in}}%
\pgfpathlineto{\pgfqpoint{3.294789in}{1.728792in}}%
\pgfpathlineto{\pgfqpoint{3.297600in}{1.606863in}}%
\pgfpathlineto{\pgfqpoint{3.300411in}{1.599432in}}%
\pgfpathlineto{\pgfqpoint{3.303221in}{1.747643in}}%
\pgfpathlineto{\pgfqpoint{3.306032in}{1.449557in}}%
\pgfpathlineto{\pgfqpoint{3.308843in}{1.906704in}}%
\pgfpathlineto{\pgfqpoint{3.314464in}{1.493958in}}%
\pgfpathlineto{\pgfqpoint{3.317275in}{1.790204in}}%
\pgfpathlineto{\pgfqpoint{3.320085in}{1.675702in}}%
\pgfpathlineto{\pgfqpoint{3.322896in}{1.671014in}}%
\pgfpathlineto{\pgfqpoint{3.325707in}{1.525287in}}%
\pgfpathlineto{\pgfqpoint{3.328518in}{1.727398in}}%
\pgfpathlineto{\pgfqpoint{3.331328in}{1.533391in}}%
\pgfpathlineto{\pgfqpoint{3.334139in}{1.753773in}}%
\pgfpathlineto{\pgfqpoint{3.336950in}{2.073679in}}%
\pgfpathlineto{\pgfqpoint{3.339760in}{1.694336in}}%
\pgfpathlineto{\pgfqpoint{3.342571in}{1.622487in}}%
\pgfpathlineto{\pgfqpoint{3.345382in}{1.521513in}}%
\pgfpathlineto{\pgfqpoint{3.348192in}{1.790352in}}%
\pgfpathlineto{\pgfqpoint{3.351003in}{1.633569in}}%
\pgfpathlineto{\pgfqpoint{3.353814in}{1.862100in}}%
\pgfpathlineto{\pgfqpoint{3.356624in}{1.507885in}}%
\pgfpathlineto{\pgfqpoint{3.359435in}{1.783198in}}%
\pgfpathlineto{\pgfqpoint{3.362246in}{1.766438in}}%
\pgfpathlineto{\pgfqpoint{3.365056in}{1.566241in}}%
\pgfpathlineto{\pgfqpoint{3.367867in}{1.635793in}}%
\pgfpathlineto{\pgfqpoint{3.370678in}{1.722506in}}%
\pgfpathlineto{\pgfqpoint{3.373489in}{1.223578in}}%
\pgfpathlineto{\pgfqpoint{3.376299in}{1.793081in}}%
\pgfpathlineto{\pgfqpoint{3.379110in}{1.854265in}}%
\pgfpathlineto{\pgfqpoint{3.381921in}{1.781958in}}%
\pgfpathlineto{\pgfqpoint{3.384731in}{1.663678in}}%
\pgfpathlineto{\pgfqpoint{3.387542in}{1.611432in}}%
\pgfpathlineto{\pgfqpoint{3.390353in}{1.450137in}}%
\pgfpathlineto{\pgfqpoint{3.393163in}{1.766665in}}%
\pgfpathlineto{\pgfqpoint{3.395974in}{2.008270in}}%
\pgfpathlineto{\pgfqpoint{3.398785in}{1.941876in}}%
\pgfpathlineto{\pgfqpoint{3.401595in}{1.821210in}}%
\pgfpathlineto{\pgfqpoint{3.404406in}{1.673633in}}%
\pgfpathlineto{\pgfqpoint{3.407217in}{1.813454in}}%
\pgfpathlineto{\pgfqpoint{3.410027in}{1.760341in}}%
\pgfpathlineto{\pgfqpoint{3.412838in}{2.089124in}}%
\pgfpathlineto{\pgfqpoint{3.415649in}{1.545130in}}%
\pgfpathlineto{\pgfqpoint{3.418459in}{1.680838in}}%
\pgfpathlineto{\pgfqpoint{3.421270in}{1.644959in}}%
\pgfpathlineto{\pgfqpoint{3.424081in}{2.340494in}}%
\pgfpathlineto{\pgfqpoint{3.426892in}{1.494845in}}%
\pgfpathlineto{\pgfqpoint{3.429702in}{1.876277in}}%
\pgfpathlineto{\pgfqpoint{3.432513in}{1.954856in}}%
\pgfpathlineto{\pgfqpoint{3.435324in}{1.768120in}}%
\pgfpathlineto{\pgfqpoint{3.438134in}{1.479720in}}%
\pgfpathlineto{\pgfqpoint{3.440945in}{1.779417in}}%
\pgfpathlineto{\pgfqpoint{3.443756in}{1.589659in}}%
\pgfpathlineto{\pgfqpoint{3.446566in}{1.631999in}}%
\pgfpathlineto{\pgfqpoint{3.449377in}{1.397150in}}%
\pgfpathlineto{\pgfqpoint{3.452188in}{1.824972in}}%
\pgfpathlineto{\pgfqpoint{3.454998in}{1.728873in}}%
\pgfpathlineto{\pgfqpoint{3.457809in}{1.476191in}}%
\pgfpathlineto{\pgfqpoint{3.460620in}{1.755556in}}%
\pgfpathlineto{\pgfqpoint{3.463430in}{1.774884in}}%
\pgfpathlineto{\pgfqpoint{3.471863in}{1.704983in}}%
\pgfpathlineto{\pgfqpoint{3.474673in}{1.676902in}}%
\pgfpathlineto{\pgfqpoint{3.477484in}{1.598813in}}%
\pgfpathlineto{\pgfqpoint{3.480295in}{1.098192in}}%
\pgfpathlineto{\pgfqpoint{3.483105in}{1.058243in}}%
\pgfpathlineto{\pgfqpoint{3.485916in}{1.371216in}}%
\pgfpathlineto{\pgfqpoint{3.488727in}{2.535550in}}%
\pgfpathlineto{\pgfqpoint{3.491537in}{2.065114in}}%
\pgfpathlineto{\pgfqpoint{3.494348in}{1.705343in}}%
\pgfpathlineto{\pgfqpoint{3.499969in}{1.171503in}}%
\pgfpathlineto{\pgfqpoint{3.502780in}{1.840543in}}%
\pgfpathlineto{\pgfqpoint{3.505591in}{1.866575in}}%
\pgfpathlineto{\pgfqpoint{3.508401in}{1.415516in}}%
\pgfpathlineto{\pgfqpoint{3.511212in}{1.998193in}}%
\pgfpathlineto{\pgfqpoint{3.514023in}{1.483188in}}%
\pgfpathlineto{\pgfqpoint{3.516834in}{1.834609in}}%
\pgfpathlineto{\pgfqpoint{3.519644in}{1.808342in}}%
\pgfpathlineto{\pgfqpoint{3.522455in}{1.525257in}}%
\pgfpathlineto{\pgfqpoint{3.525266in}{1.804158in}}%
\pgfpathlineto{\pgfqpoint{3.528076in}{1.796564in}}%
\pgfpathlineto{\pgfqpoint{3.530887in}{1.685226in}}%
\pgfpathlineto{\pgfqpoint{3.533698in}{1.433838in}}%
\pgfpathlineto{\pgfqpoint{3.536508in}{1.911638in}}%
\pgfpathlineto{\pgfqpoint{3.539319in}{1.573449in}}%
\pgfpathlineto{\pgfqpoint{3.542130in}{1.846901in}}%
\pgfpathlineto{\pgfqpoint{3.544940in}{1.534856in}}%
\pgfpathlineto{\pgfqpoint{3.547751in}{1.792428in}}%
\pgfpathlineto{\pgfqpoint{3.550562in}{0.901227in}}%
\pgfpathlineto{\pgfqpoint{3.553372in}{2.000033in}}%
\pgfpathlineto{\pgfqpoint{3.556183in}{1.945673in}}%
\pgfpathlineto{\pgfqpoint{3.558994in}{1.765713in}}%
\pgfpathlineto{\pgfqpoint{3.561805in}{1.842918in}}%
\pgfpathlineto{\pgfqpoint{3.564615in}{2.034089in}}%
\pgfpathlineto{\pgfqpoint{3.567426in}{1.658598in}}%
\pgfpathlineto{\pgfqpoint{3.570237in}{1.937689in}}%
\pgfpathlineto{\pgfqpoint{3.573047in}{1.836282in}}%
\pgfpathlineto{\pgfqpoint{3.575858in}{1.709419in}}%
\pgfpathlineto{\pgfqpoint{3.578669in}{1.903562in}}%
\pgfpathlineto{\pgfqpoint{3.584290in}{1.523912in}}%
\pgfpathlineto{\pgfqpoint{3.587101in}{1.930716in}}%
\pgfpathlineto{\pgfqpoint{3.589911in}{1.828006in}}%
\pgfpathlineto{\pgfqpoint{3.592722in}{1.893802in}}%
\pgfpathlineto{\pgfqpoint{3.595533in}{1.547258in}}%
\pgfpathlineto{\pgfqpoint{3.598343in}{1.522482in}}%
\pgfpathlineto{\pgfqpoint{3.601154in}{1.888942in}}%
\pgfpathlineto{\pgfqpoint{3.603965in}{1.824022in}}%
\pgfpathlineto{\pgfqpoint{3.606776in}{1.915741in}}%
\pgfpathlineto{\pgfqpoint{3.609586in}{1.559837in}}%
\pgfpathlineto{\pgfqpoint{3.612397in}{1.961402in}}%
\pgfpathlineto{\pgfqpoint{3.615208in}{1.620505in}}%
\pgfpathlineto{\pgfqpoint{3.618018in}{1.506601in}}%
\pgfpathlineto{\pgfqpoint{3.620829in}{1.210048in}}%
\pgfpathlineto{\pgfqpoint{3.623640in}{2.238115in}}%
\pgfpathlineto{\pgfqpoint{3.626450in}{1.782577in}}%
\pgfpathlineto{\pgfqpoint{3.629261in}{1.871570in}}%
\pgfpathlineto{\pgfqpoint{3.632072in}{1.590261in}}%
\pgfpathlineto{\pgfqpoint{3.634882in}{1.552553in}}%
\pgfpathlineto{\pgfqpoint{3.637693in}{2.004095in}}%
\pgfpathlineto{\pgfqpoint{3.640504in}{1.633349in}}%
\pgfpathlineto{\pgfqpoint{3.643314in}{1.518767in}}%
\pgfpathlineto{\pgfqpoint{3.646125in}{1.653031in}}%
\pgfpathlineto{\pgfqpoint{3.648936in}{1.849445in}}%
\pgfpathlineto{\pgfqpoint{3.651746in}{1.677777in}}%
\pgfpathlineto{\pgfqpoint{3.654557in}{2.010444in}}%
\pgfpathlineto{\pgfqpoint{3.657368in}{1.632061in}}%
\pgfpathlineto{\pgfqpoint{3.660179in}{1.700031in}}%
\pgfpathlineto{\pgfqpoint{3.662989in}{1.676002in}}%
\pgfpathlineto{\pgfqpoint{3.665800in}{1.629787in}}%
\pgfpathlineto{\pgfqpoint{3.668611in}{1.647704in}}%
\pgfpathlineto{\pgfqpoint{3.671421in}{1.746396in}}%
\pgfpathlineto{\pgfqpoint{3.674232in}{1.528331in}}%
\pgfpathlineto{\pgfqpoint{3.677043in}{1.869780in}}%
\pgfpathlineto{\pgfqpoint{3.679853in}{1.520359in}}%
\pgfpathlineto{\pgfqpoint{3.682664in}{1.543071in}}%
\pgfpathlineto{\pgfqpoint{3.685475in}{2.108751in}}%
\pgfpathlineto{\pgfqpoint{3.688285in}{1.525454in}}%
\pgfpathlineto{\pgfqpoint{3.691096in}{1.596906in}}%
\pgfpathlineto{\pgfqpoint{3.693907in}{1.456817in}}%
\pgfpathlineto{\pgfqpoint{3.696717in}{1.739349in}}%
\pgfpathlineto{\pgfqpoint{3.699528in}{1.286159in}}%
\pgfpathlineto{\pgfqpoint{3.702339in}{1.845200in}}%
\pgfpathlineto{\pgfqpoint{3.705150in}{2.045057in}}%
\pgfpathlineto{\pgfqpoint{3.707960in}{1.935082in}}%
\pgfpathlineto{\pgfqpoint{3.710771in}{1.461207in}}%
\pgfpathlineto{\pgfqpoint{3.713582in}{1.215684in}}%
\pgfpathlineto{\pgfqpoint{3.716392in}{1.865600in}}%
\pgfpathlineto{\pgfqpoint{3.719203in}{1.733562in}}%
\pgfpathlineto{\pgfqpoint{3.722014in}{1.928749in}}%
\pgfpathlineto{\pgfqpoint{3.724824in}{1.632665in}}%
\pgfpathlineto{\pgfqpoint{3.727635in}{1.745251in}}%
\pgfpathlineto{\pgfqpoint{3.730446in}{1.826331in}}%
\pgfpathlineto{\pgfqpoint{3.733256in}{1.520883in}}%
\pgfpathlineto{\pgfqpoint{3.736067in}{1.533479in}}%
\pgfpathlineto{\pgfqpoint{3.738878in}{1.314994in}}%
\pgfpathlineto{\pgfqpoint{3.741688in}{1.808196in}}%
\pgfpathlineto{\pgfqpoint{3.744499in}{1.486551in}}%
\pgfpathlineto{\pgfqpoint{3.747310in}{1.381914in}}%
\pgfpathlineto{\pgfqpoint{3.750121in}{1.497231in}}%
\pgfpathlineto{\pgfqpoint{3.752931in}{1.915065in}}%
\pgfpathlineto{\pgfqpoint{3.755742in}{1.866902in}}%
\pgfpathlineto{\pgfqpoint{3.758553in}{1.342480in}}%
\pgfpathlineto{\pgfqpoint{3.761363in}{1.840612in}}%
\pgfpathlineto{\pgfqpoint{3.764174in}{1.268780in}}%
\pgfpathlineto{\pgfqpoint{3.766985in}{1.597888in}}%
\pgfpathlineto{\pgfqpoint{3.769795in}{1.536130in}}%
\pgfpathlineto{\pgfqpoint{3.772606in}{1.883844in}}%
\pgfpathlineto{\pgfqpoint{3.775417in}{1.939020in}}%
\pgfpathlineto{\pgfqpoint{3.778227in}{1.485346in}}%
\pgfpathlineto{\pgfqpoint{3.781038in}{1.721078in}}%
\pgfpathlineto{\pgfqpoint{3.783849in}{1.480701in}}%
\pgfpathlineto{\pgfqpoint{3.786659in}{1.339508in}}%
\pgfpathlineto{\pgfqpoint{3.789470in}{2.812968in}}%
\pgfpathlineto{\pgfqpoint{3.792281in}{1.668304in}}%
\pgfpathlineto{\pgfqpoint{3.795092in}{1.485661in}}%
\pgfpathlineto{\pgfqpoint{3.797902in}{1.898384in}}%
\pgfpathlineto{\pgfqpoint{3.806334in}{0.848492in}}%
\pgfpathlineto{\pgfqpoint{3.809145in}{1.819382in}}%
\pgfpathlineto{\pgfqpoint{3.811956in}{2.134336in}}%
\pgfpathlineto{\pgfqpoint{3.814766in}{1.313002in}}%
\pgfpathlineto{\pgfqpoint{3.817577in}{2.130297in}}%
\pgfpathlineto{\pgfqpoint{3.820388in}{1.843683in}}%
\pgfpathlineto{\pgfqpoint{3.823198in}{1.924554in}}%
\pgfpathlineto{\pgfqpoint{3.826009in}{1.520194in}}%
\pgfpathlineto{\pgfqpoint{3.831630in}{2.023704in}}%
\pgfpathlineto{\pgfqpoint{3.834441in}{1.496077in}}%
\pgfpathlineto{\pgfqpoint{3.837252in}{1.625415in}}%
\pgfpathlineto{\pgfqpoint{3.840062in}{1.998273in}}%
\pgfpathlineto{\pgfqpoint{3.842873in}{1.575959in}}%
\pgfpathlineto{\pgfqpoint{3.845684in}{1.619318in}}%
\pgfpathlineto{\pgfqpoint{3.848495in}{2.170443in}}%
\pgfpathlineto{\pgfqpoint{3.851305in}{1.621188in}}%
\pgfpathlineto{\pgfqpoint{3.854116in}{1.610056in}}%
\pgfpathlineto{\pgfqpoint{3.856927in}{1.689860in}}%
\pgfpathlineto{\pgfqpoint{3.859737in}{1.271099in}}%
\pgfpathlineto{\pgfqpoint{3.862548in}{1.402701in}}%
\pgfpathlineto{\pgfqpoint{3.865359in}{1.705599in}}%
\pgfpathlineto{\pgfqpoint{3.868169in}{1.589910in}}%
\pgfpathlineto{\pgfqpoint{3.870980in}{2.003074in}}%
\pgfpathlineto{\pgfqpoint{3.873791in}{1.660748in}}%
\pgfpathlineto{\pgfqpoint{3.876601in}{1.783437in}}%
\pgfpathlineto{\pgfqpoint{3.879412in}{1.791743in}}%
\pgfpathlineto{\pgfqpoint{3.882223in}{1.856820in}}%
\pgfpathlineto{\pgfqpoint{3.885033in}{1.829120in}}%
\pgfpathlineto{\pgfqpoint{3.887844in}{1.689856in}}%
\pgfpathlineto{\pgfqpoint{3.890655in}{1.519891in}}%
\pgfpathlineto{\pgfqpoint{3.893466in}{1.894463in}}%
\pgfpathlineto{\pgfqpoint{3.896276in}{1.739530in}}%
\pgfpathlineto{\pgfqpoint{3.899087in}{1.679081in}}%
\pgfpathlineto{\pgfqpoint{3.901898in}{1.963701in}}%
\pgfpathlineto{\pgfqpoint{3.904708in}{1.979886in}}%
\pgfpathlineto{\pgfqpoint{3.907519in}{1.629440in}}%
\pgfpathlineto{\pgfqpoint{3.910330in}{1.918218in}}%
\pgfpathlineto{\pgfqpoint{3.913140in}{1.739336in}}%
\pgfpathlineto{\pgfqpoint{3.915951in}{1.518528in}}%
\pgfpathlineto{\pgfqpoint{3.918762in}{1.941331in}}%
\pgfpathlineto{\pgfqpoint{3.921572in}{1.566837in}}%
\pgfpathlineto{\pgfqpoint{3.924383in}{1.782422in}}%
\pgfpathlineto{\pgfqpoint{3.927194in}{1.648943in}}%
\pgfpathlineto{\pgfqpoint{3.930004in}{1.833126in}}%
\pgfpathlineto{\pgfqpoint{3.932815in}{1.926402in}}%
\pgfpathlineto{\pgfqpoint{3.935626in}{1.811886in}}%
\pgfpathlineto{\pgfqpoint{3.938437in}{1.644188in}}%
\pgfpathlineto{\pgfqpoint{3.941247in}{1.957244in}}%
\pgfpathlineto{\pgfqpoint{3.944058in}{1.556017in}}%
\pgfpathlineto{\pgfqpoint{3.946869in}{1.769042in}}%
\pgfpathlineto{\pgfqpoint{3.949679in}{1.622921in}}%
\pgfpathlineto{\pgfqpoint{3.952490in}{1.364092in}}%
\pgfpathlineto{\pgfqpoint{3.955301in}{1.500908in}}%
\pgfpathlineto{\pgfqpoint{3.958111in}{1.769564in}}%
\pgfpathlineto{\pgfqpoint{3.960922in}{1.736740in}}%
\pgfpathlineto{\pgfqpoint{3.963733in}{1.477439in}}%
\pgfpathlineto{\pgfqpoint{3.966543in}{1.603287in}}%
\pgfpathlineto{\pgfqpoint{3.969354in}{1.936371in}}%
\pgfpathlineto{\pgfqpoint{3.972165in}{1.437324in}}%
\pgfpathlineto{\pgfqpoint{3.974975in}{1.667153in}}%
\pgfpathlineto{\pgfqpoint{3.977786in}{1.749992in}}%
\pgfpathlineto{\pgfqpoint{3.980597in}{1.766240in}}%
\pgfpathlineto{\pgfqpoint{3.983408in}{1.790433in}}%
\pgfpathlineto{\pgfqpoint{3.986218in}{1.887001in}}%
\pgfpathlineto{\pgfqpoint{3.989029in}{1.340942in}}%
\pgfpathlineto{\pgfqpoint{3.991840in}{1.811296in}}%
\pgfpathlineto{\pgfqpoint{3.994650in}{1.477502in}}%
\pgfpathlineto{\pgfqpoint{3.997461in}{1.888095in}}%
\pgfpathlineto{\pgfqpoint{4.000272in}{1.487640in}}%
\pgfpathlineto{\pgfqpoint{4.003082in}{1.822394in}}%
\pgfpathlineto{\pgfqpoint{4.005893in}{1.580343in}}%
\pgfpathlineto{\pgfqpoint{4.008704in}{1.851028in}}%
\pgfpathlineto{\pgfqpoint{4.011514in}{1.599295in}}%
\pgfpathlineto{\pgfqpoint{4.014325in}{2.124139in}}%
\pgfpathlineto{\pgfqpoint{4.017136in}{1.629578in}}%
\pgfpathlineto{\pgfqpoint{4.019946in}{1.679908in}}%
\pgfpathlineto{\pgfqpoint{4.022757in}{1.822770in}}%
\pgfpathlineto{\pgfqpoint{4.025568in}{1.549151in}}%
\pgfpathlineto{\pgfqpoint{4.028378in}{1.748523in}}%
\pgfpathlineto{\pgfqpoint{4.031189in}{1.834381in}}%
\pgfpathlineto{\pgfqpoint{4.034000in}{1.690027in}}%
\pgfpathlineto{\pgfqpoint{4.036811in}{1.809355in}}%
\pgfpathlineto{\pgfqpoint{4.039621in}{1.703906in}}%
\pgfpathlineto{\pgfqpoint{4.042432in}{1.892659in}}%
\pgfpathlineto{\pgfqpoint{4.045243in}{1.617577in}}%
\pgfpathlineto{\pgfqpoint{4.050864in}{1.339816in}}%
\pgfpathlineto{\pgfqpoint{4.053675in}{1.726588in}}%
\pgfpathlineto{\pgfqpoint{4.056485in}{1.612613in}}%
\pgfpathlineto{\pgfqpoint{4.059296in}{1.728724in}}%
\pgfpathlineto{\pgfqpoint{4.062107in}{1.418783in}}%
\pgfpathlineto{\pgfqpoint{4.064917in}{1.762330in}}%
\pgfpathlineto{\pgfqpoint{4.067728in}{1.689960in}}%
\pgfpathlineto{\pgfqpoint{4.070539in}{1.530235in}}%
\pgfpathlineto{\pgfqpoint{4.073349in}{2.034349in}}%
\pgfpathlineto{\pgfqpoint{4.076160in}{1.044407in}}%
\pgfpathlineto{\pgfqpoint{4.078971in}{1.332850in}}%
\pgfpathlineto{\pgfqpoint{4.081782in}{2.074566in}}%
\pgfpathlineto{\pgfqpoint{4.084592in}{2.015587in}}%
\pgfpathlineto{\pgfqpoint{4.087403in}{1.161158in}}%
\pgfpathlineto{\pgfqpoint{4.090214in}{1.756431in}}%
\pgfpathlineto{\pgfqpoint{4.093024in}{1.732678in}}%
\pgfpathlineto{\pgfqpoint{4.095835in}{1.563276in}}%
\pgfpathlineto{\pgfqpoint{4.098646in}{1.786541in}}%
\pgfpathlineto{\pgfqpoint{4.101456in}{2.087457in}}%
\pgfpathlineto{\pgfqpoint{4.104267in}{1.712890in}}%
\pgfpathlineto{\pgfqpoint{4.107078in}{1.884758in}}%
\pgfpathlineto{\pgfqpoint{4.109888in}{1.689963in}}%
\pgfpathlineto{\pgfqpoint{4.112699in}{1.827390in}}%
\pgfpathlineto{\pgfqpoint{4.115510in}{1.724663in}}%
\pgfpathlineto{\pgfqpoint{4.118320in}{1.694061in}}%
\pgfpathlineto{\pgfqpoint{4.121131in}{1.773316in}}%
\pgfpathlineto{\pgfqpoint{4.123942in}{1.817196in}}%
\pgfpathlineto{\pgfqpoint{4.126753in}{1.581033in}}%
\pgfpathlineto{\pgfqpoint{4.129563in}{1.911222in}}%
\pgfpathlineto{\pgfqpoint{4.132374in}{1.501151in}}%
\pgfpathlineto{\pgfqpoint{4.140806in}{1.823550in}}%
\pgfpathlineto{\pgfqpoint{4.143617in}{1.464696in}}%
\pgfpathlineto{\pgfqpoint{4.146427in}{1.734900in}}%
\pgfpathlineto{\pgfqpoint{4.149238in}{1.696100in}}%
\pgfpathlineto{\pgfqpoint{4.154859in}{1.843334in}}%
\pgfpathlineto{\pgfqpoint{4.157670in}{1.821983in}}%
\pgfpathlineto{\pgfqpoint{4.160481in}{1.654146in}}%
\pgfpathlineto{\pgfqpoint{4.163291in}{1.713963in}}%
\pgfpathlineto{\pgfqpoint{4.166102in}{1.618103in}}%
\pgfpathlineto{\pgfqpoint{4.168913in}{1.777897in}}%
\pgfpathlineto{\pgfqpoint{4.171724in}{1.672094in}}%
\pgfpathlineto{\pgfqpoint{4.174534in}{1.864556in}}%
\pgfpathlineto{\pgfqpoint{4.177345in}{1.672288in}}%
\pgfpathlineto{\pgfqpoint{4.180156in}{1.725557in}}%
\pgfpathlineto{\pgfqpoint{4.182966in}{1.581285in}}%
\pgfpathlineto{\pgfqpoint{4.185777in}{1.731663in}}%
\pgfpathlineto{\pgfqpoint{4.188588in}{1.804447in}}%
\pgfpathlineto{\pgfqpoint{4.191398in}{1.642814in}}%
\pgfpathlineto{\pgfqpoint{4.194209in}{1.593137in}}%
\pgfpathlineto{\pgfqpoint{4.197020in}{1.697974in}}%
\pgfpathlineto{\pgfqpoint{4.199830in}{1.739548in}}%
\pgfpathlineto{\pgfqpoint{4.202641in}{1.751227in}}%
\pgfpathlineto{\pgfqpoint{4.205452in}{1.749041in}}%
\pgfpathlineto{\pgfqpoint{4.208262in}{1.640913in}}%
\pgfpathlineto{\pgfqpoint{4.211073in}{1.768625in}}%
\pgfpathlineto{\pgfqpoint{4.213884in}{1.824692in}}%
\pgfpathlineto{\pgfqpoint{4.216695in}{1.792720in}}%
\pgfpathlineto{\pgfqpoint{4.219505in}{1.826559in}}%
\pgfpathlineto{\pgfqpoint{4.222316in}{1.640268in}}%
\pgfpathlineto{\pgfqpoint{4.225127in}{1.381752in}}%
\pgfpathlineto{\pgfqpoint{4.227937in}{2.002294in}}%
\pgfpathlineto{\pgfqpoint{4.230748in}{1.489105in}}%
\pgfpathlineto{\pgfqpoint{4.233559in}{1.614021in}}%
\pgfpathlineto{\pgfqpoint{4.236369in}{1.791407in}}%
\pgfpathlineto{\pgfqpoint{4.239180in}{1.703674in}}%
\pgfpathlineto{\pgfqpoint{4.241991in}{1.775266in}}%
\pgfpathlineto{\pgfqpoint{4.244801in}{1.686230in}}%
\pgfpathlineto{\pgfqpoint{4.247612in}{1.832386in}}%
\pgfpathlineto{\pgfqpoint{4.250423in}{1.718798in}}%
\pgfpathlineto{\pgfqpoint{4.253233in}{1.536469in}}%
\pgfpathlineto{\pgfqpoint{4.256044in}{1.554386in}}%
\pgfpathlineto{\pgfqpoint{4.258855in}{1.758085in}}%
\pgfpathlineto{\pgfqpoint{4.261665in}{1.844449in}}%
\pgfpathlineto{\pgfqpoint{4.264476in}{1.494939in}}%
\pgfpathlineto{\pgfqpoint{4.267287in}{1.839106in}}%
\pgfpathlineto{\pgfqpoint{4.270098in}{1.722812in}}%
\pgfpathlineto{\pgfqpoint{4.272908in}{1.667012in}}%
\pgfpathlineto{\pgfqpoint{4.275719in}{1.809038in}}%
\pgfpathlineto{\pgfqpoint{4.278530in}{1.693934in}}%
\pgfpathlineto{\pgfqpoint{4.281340in}{1.601952in}}%
\pgfpathlineto{\pgfqpoint{4.284151in}{1.736165in}}%
\pgfpathlineto{\pgfqpoint{4.286962in}{1.487539in}}%
\pgfpathlineto{\pgfqpoint{4.289772in}{1.732747in}}%
\pgfpathlineto{\pgfqpoint{4.292583in}{1.620199in}}%
\pgfpathlineto{\pgfqpoint{4.295394in}{1.802533in}}%
\pgfpathlineto{\pgfqpoint{4.298204in}{1.633957in}}%
\pgfpathlineto{\pgfqpoint{4.301015in}{1.583052in}}%
\pgfpathlineto{\pgfqpoint{4.303826in}{1.926531in}}%
\pgfpathlineto{\pgfqpoint{4.306636in}{1.632277in}}%
\pgfpathlineto{\pgfqpoint{4.309447in}{1.664966in}}%
\pgfpathlineto{\pgfqpoint{4.312258in}{1.846007in}}%
\pgfpathlineto{\pgfqpoint{4.315069in}{1.476018in}}%
\pgfpathlineto{\pgfqpoint{4.317879in}{1.639500in}}%
\pgfpathlineto{\pgfqpoint{4.320690in}{1.725114in}}%
\pgfpathlineto{\pgfqpoint{4.323501in}{1.748310in}}%
\pgfpathlineto{\pgfqpoint{4.326311in}{1.746177in}}%
\pgfpathlineto{\pgfqpoint{4.329122in}{1.523138in}}%
\pgfpathlineto{\pgfqpoint{4.331933in}{1.491665in}}%
\pgfpathlineto{\pgfqpoint{4.334743in}{1.705856in}}%
\pgfpathlineto{\pgfqpoint{4.337554in}{1.636619in}}%
\pgfpathlineto{\pgfqpoint{4.340365in}{2.102837in}}%
\pgfpathlineto{\pgfqpoint{4.343175in}{1.763297in}}%
\pgfpathlineto{\pgfqpoint{4.345986in}{1.759146in}}%
\pgfpathlineto{\pgfqpoint{4.348797in}{1.433458in}}%
\pgfpathlineto{\pgfqpoint{4.351607in}{1.693968in}}%
\pgfpathlineto{\pgfqpoint{4.354418in}{1.010059in}}%
\pgfpathlineto{\pgfqpoint{4.357229in}{1.728562in}}%
\pgfpathlineto{\pgfqpoint{4.360040in}{2.023035in}}%
\pgfpathlineto{\pgfqpoint{4.362850in}{1.883391in}}%
\pgfpathlineto{\pgfqpoint{4.365661in}{1.644892in}}%
\pgfpathlineto{\pgfqpoint{4.368472in}{1.858307in}}%
\pgfpathlineto{\pgfqpoint{4.371282in}{1.351698in}}%
\pgfpathlineto{\pgfqpoint{4.374093in}{1.622259in}}%
\pgfpathlineto{\pgfqpoint{4.376904in}{1.801512in}}%
\pgfpathlineto{\pgfqpoint{4.379714in}{1.486348in}}%
\pgfpathlineto{\pgfqpoint{4.382525in}{1.706087in}}%
\pgfpathlineto{\pgfqpoint{4.385336in}{1.326285in}}%
\pgfpathlineto{\pgfqpoint{4.388146in}{1.304868in}}%
\pgfpathlineto{\pgfqpoint{4.390957in}{1.753021in}}%
\pgfpathlineto{\pgfqpoint{4.393768in}{2.016124in}}%
\pgfpathlineto{\pgfqpoint{4.396578in}{1.650846in}}%
\pgfpathlineto{\pgfqpoint{4.399389in}{2.131089in}}%
\pgfpathlineto{\pgfqpoint{4.402200in}{1.686009in}}%
\pgfpathlineto{\pgfqpoint{4.405011in}{1.665934in}}%
\pgfpathlineto{\pgfqpoint{4.407821in}{1.564971in}}%
\pgfpathlineto{\pgfqpoint{4.410632in}{1.829114in}}%
\pgfpathlineto{\pgfqpoint{4.413443in}{1.679975in}}%
\pgfpathlineto{\pgfqpoint{4.416253in}{1.764172in}}%
\pgfpathlineto{\pgfqpoint{4.419064in}{1.464427in}}%
\pgfpathlineto{\pgfqpoint{4.421875in}{1.641218in}}%
\pgfpathlineto{\pgfqpoint{4.424685in}{1.744856in}}%
\pgfpathlineto{\pgfqpoint{4.427496in}{1.647329in}}%
\pgfpathlineto{\pgfqpoint{4.430307in}{1.645161in}}%
\pgfpathlineto{\pgfqpoint{4.433117in}{1.740919in}}%
\pgfpathlineto{\pgfqpoint{4.435928in}{1.732653in}}%
\pgfpathlineto{\pgfqpoint{4.438739in}{1.681875in}}%
\pgfpathlineto{\pgfqpoint{4.441549in}{1.698112in}}%
\pgfpathlineto{\pgfqpoint{4.444360in}{1.631022in}}%
\pgfpathlineto{\pgfqpoint{4.447171in}{1.984678in}}%
\pgfpathlineto{\pgfqpoint{4.449981in}{1.819469in}}%
\pgfpathlineto{\pgfqpoint{4.452792in}{1.873423in}}%
\pgfpathlineto{\pgfqpoint{4.455603in}{1.906182in}}%
\pgfpathlineto{\pgfqpoint{4.458414in}{1.604767in}}%
\pgfpathlineto{\pgfqpoint{4.461224in}{1.608187in}}%
\pgfpathlineto{\pgfqpoint{4.464035in}{1.785578in}}%
\pgfpathlineto{\pgfqpoint{4.466846in}{1.610193in}}%
\pgfpathlineto{\pgfqpoint{4.469656in}{1.652905in}}%
\pgfpathlineto{\pgfqpoint{4.472467in}{1.709638in}}%
\pgfpathlineto{\pgfqpoint{4.475278in}{1.754494in}}%
\pgfpathlineto{\pgfqpoint{4.478088in}{1.719276in}}%
\pgfpathlineto{\pgfqpoint{4.480899in}{1.713403in}}%
\pgfpathlineto{\pgfqpoint{4.483710in}{1.750183in}}%
\pgfpathlineto{\pgfqpoint{4.486520in}{1.895801in}}%
\pgfpathlineto{\pgfqpoint{4.489331in}{1.817573in}}%
\pgfpathlineto{\pgfqpoint{4.492142in}{1.568383in}}%
\pgfpathlineto{\pgfqpoint{4.494952in}{1.791013in}}%
\pgfpathlineto{\pgfqpoint{4.497763in}{1.678734in}}%
\pgfpathlineto{\pgfqpoint{4.500574in}{1.506700in}}%
\pgfpathlineto{\pgfqpoint{4.503385in}{1.641972in}}%
\pgfpathlineto{\pgfqpoint{4.506195in}{1.665006in}}%
\pgfpathlineto{\pgfqpoint{4.509006in}{2.392628in}}%
\pgfpathlineto{\pgfqpoint{4.511817in}{1.645788in}}%
\pgfpathlineto{\pgfqpoint{4.514627in}{1.682759in}}%
\pgfpathlineto{\pgfqpoint{4.517438in}{1.567359in}}%
\pgfpathlineto{\pgfqpoint{4.520249in}{1.781435in}}%
\pgfpathlineto{\pgfqpoint{4.523059in}{1.749497in}}%
\pgfpathlineto{\pgfqpoint{4.525870in}{1.789798in}}%
\pgfpathlineto{\pgfqpoint{4.528681in}{1.767242in}}%
\pgfpathlineto{\pgfqpoint{4.531491in}{1.845008in}}%
\pgfpathlineto{\pgfqpoint{4.534302in}{1.668445in}}%
\pgfpathlineto{\pgfqpoint{4.537113in}{1.701096in}}%
\pgfpathlineto{\pgfqpoint{4.539923in}{1.773442in}}%
\pgfpathlineto{\pgfqpoint{4.542734in}{1.670353in}}%
\pgfpathlineto{\pgfqpoint{4.545545in}{1.760542in}}%
\pgfpathlineto{\pgfqpoint{4.548356in}{1.735144in}}%
\pgfpathlineto{\pgfqpoint{4.551166in}{1.616483in}}%
\pgfpathlineto{\pgfqpoint{4.553977in}{1.681202in}}%
\pgfpathlineto{\pgfqpoint{4.556788in}{1.876755in}}%
\pgfpathlineto{\pgfqpoint{4.559598in}{1.611583in}}%
\pgfpathlineto{\pgfqpoint{4.562409in}{1.738550in}}%
\pgfpathlineto{\pgfqpoint{4.565220in}{1.717018in}}%
\pgfpathlineto{\pgfqpoint{4.568030in}{1.713416in}}%
\pgfpathlineto{\pgfqpoint{4.570841in}{1.674198in}}%
\pgfpathlineto{\pgfqpoint{4.573652in}{1.718753in}}%
\pgfpathlineto{\pgfqpoint{4.576462in}{1.800258in}}%
\pgfpathlineto{\pgfqpoint{4.579273in}{1.757310in}}%
\pgfpathlineto{\pgfqpoint{4.582084in}{1.594876in}}%
\pgfpathlineto{\pgfqpoint{4.584894in}{1.757443in}}%
\pgfpathlineto{\pgfqpoint{4.587705in}{1.713181in}}%
\pgfpathlineto{\pgfqpoint{4.590516in}{1.725465in}}%
\pgfpathlineto{\pgfqpoint{4.593327in}{1.508086in}}%
\pgfpathlineto{\pgfqpoint{4.596137in}{1.495086in}}%
\pgfpathlineto{\pgfqpoint{4.598948in}{1.772889in}}%
\pgfpathlineto{\pgfqpoint{4.601759in}{1.751059in}}%
\pgfpathlineto{\pgfqpoint{4.604569in}{1.750837in}}%
\pgfpathlineto{\pgfqpoint{4.607380in}{1.649271in}}%
\pgfpathlineto{\pgfqpoint{4.610191in}{1.722317in}}%
\pgfpathlineto{\pgfqpoint{4.613001in}{1.702701in}}%
\pgfpathlineto{\pgfqpoint{4.615812in}{1.661749in}}%
\pgfpathlineto{\pgfqpoint{4.618623in}{1.667047in}}%
\pgfpathlineto{\pgfqpoint{4.621433in}{1.779263in}}%
\pgfpathlineto{\pgfqpoint{4.624244in}{1.590493in}}%
\pgfpathlineto{\pgfqpoint{4.627055in}{1.734853in}}%
\pgfpathlineto{\pgfqpoint{4.629865in}{1.702712in}}%
\pgfpathlineto{\pgfqpoint{4.632676in}{1.631354in}}%
\pgfpathlineto{\pgfqpoint{4.635487in}{1.711677in}}%
\pgfpathlineto{\pgfqpoint{4.638298in}{1.690237in}}%
\pgfpathlineto{\pgfqpoint{4.641108in}{1.663424in}}%
\pgfpathlineto{\pgfqpoint{4.643919in}{1.724193in}}%
\pgfpathlineto{\pgfqpoint{4.646730in}{1.857188in}}%
\pgfpathlineto{\pgfqpoint{4.649540in}{1.677884in}}%
\pgfpathlineto{\pgfqpoint{4.652351in}{1.695559in}}%
\pgfpathlineto{\pgfqpoint{4.655162in}{1.934092in}}%
\pgfpathlineto{\pgfqpoint{4.657972in}{1.692023in}}%
\pgfpathlineto{\pgfqpoint{4.660783in}{1.811664in}}%
\pgfpathlineto{\pgfqpoint{4.666404in}{1.643717in}}%
\pgfpathlineto{\pgfqpoint{4.669215in}{1.653963in}}%
\pgfpathlineto{\pgfqpoint{4.672026in}{1.627804in}}%
\pgfpathlineto{\pgfqpoint{4.674836in}{1.698978in}}%
\pgfpathlineto{\pgfqpoint{4.677647in}{1.907831in}}%
\pgfpathlineto{\pgfqpoint{4.680458in}{1.676584in}}%
\pgfpathlineto{\pgfqpoint{4.683268in}{1.721172in}}%
\pgfpathlineto{\pgfqpoint{4.686079in}{1.602688in}}%
\pgfpathlineto{\pgfqpoint{4.688890in}{1.662706in}}%
\pgfpathlineto{\pgfqpoint{4.691701in}{1.705826in}}%
\pgfpathlineto{\pgfqpoint{4.694511in}{1.735079in}}%
\pgfpathlineto{\pgfqpoint{4.697322in}{1.666203in}}%
\pgfpathlineto{\pgfqpoint{4.700133in}{1.798468in}}%
\pgfpathlineto{\pgfqpoint{4.702943in}{1.775638in}}%
\pgfpathlineto{\pgfqpoint{4.705754in}{1.678402in}}%
\pgfpathlineto{\pgfqpoint{4.708565in}{1.486270in}}%
\pgfpathlineto{\pgfqpoint{4.711375in}{1.683391in}}%
\pgfpathlineto{\pgfqpoint{4.714186in}{1.822720in}}%
\pgfpathlineto{\pgfqpoint{4.716997in}{1.830124in}}%
\pgfpathlineto{\pgfqpoint{4.719807in}{1.785103in}}%
\pgfpathlineto{\pgfqpoint{4.722618in}{1.848161in}}%
\pgfpathlineto{\pgfqpoint{4.728239in}{1.630243in}}%
\pgfpathlineto{\pgfqpoint{4.733861in}{1.777070in}}%
\pgfpathlineto{\pgfqpoint{4.736672in}{1.718609in}}%
\pgfpathlineto{\pgfqpoint{4.739482in}{1.814399in}}%
\pgfpathlineto{\pgfqpoint{4.742293in}{1.757775in}}%
\pgfpathlineto{\pgfqpoint{4.745104in}{1.568510in}}%
\pgfpathlineto{\pgfqpoint{4.747914in}{1.739897in}}%
\pgfpathlineto{\pgfqpoint{4.750725in}{1.692023in}}%
\pgfpathlineto{\pgfqpoint{4.753536in}{1.442798in}}%
\pgfpathlineto{\pgfqpoint{4.756346in}{1.515167in}}%
\pgfpathlineto{\pgfqpoint{4.759157in}{1.954063in}}%
\pgfpathlineto{\pgfqpoint{4.764778in}{1.501457in}}%
\pgfpathlineto{\pgfqpoint{4.767589in}{1.692023in}}%
\pgfpathlineto{\pgfqpoint{4.770400in}{1.794282in}}%
\pgfpathlineto{\pgfqpoint{4.773210in}{1.625044in}}%
\pgfpathlineto{\pgfqpoint{4.776021in}{1.718849in}}%
\pgfpathlineto{\pgfqpoint{4.778832in}{1.594558in}}%
\pgfpathlineto{\pgfqpoint{4.781643in}{1.959429in}}%
\pgfpathlineto{\pgfqpoint{4.784453in}{1.633918in}}%
\pgfpathlineto{\pgfqpoint{4.787264in}{1.638708in}}%
\pgfpathlineto{\pgfqpoint{4.790075in}{1.909135in}}%
\pgfpathlineto{\pgfqpoint{4.792885in}{1.396273in}}%
\pgfpathlineto{\pgfqpoint{4.795696in}{1.585990in}}%
\pgfpathlineto{\pgfqpoint{4.798507in}{1.619235in}}%
\pgfpathlineto{\pgfqpoint{4.801317in}{1.801920in}}%
\pgfpathlineto{\pgfqpoint{4.804128in}{1.566848in}}%
\pgfpathlineto{\pgfqpoint{4.806939in}{1.803713in}}%
\pgfpathlineto{\pgfqpoint{4.809749in}{1.884745in}}%
\pgfpathlineto{\pgfqpoint{4.812560in}{1.743570in}}%
\pgfpathlineto{\pgfqpoint{4.815371in}{1.817680in}}%
\pgfpathlineto{\pgfqpoint{4.818181in}{1.654102in}}%
\pgfpathlineto{\pgfqpoint{4.820992in}{1.851329in}}%
\pgfpathlineto{\pgfqpoint{4.823803in}{1.675675in}}%
\pgfpathlineto{\pgfqpoint{4.826614in}{1.812597in}}%
\pgfpathlineto{\pgfqpoint{4.829424in}{1.798782in}}%
\pgfpathlineto{\pgfqpoint{4.832235in}{1.669438in}}%
\pgfpathlineto{\pgfqpoint{4.835046in}{1.927549in}}%
\pgfpathlineto{\pgfqpoint{4.837856in}{1.812412in}}%
\pgfpathlineto{\pgfqpoint{4.840667in}{1.633529in}}%
\pgfpathlineto{\pgfqpoint{4.843478in}{1.824506in}}%
\pgfpathlineto{\pgfqpoint{4.846288in}{1.492874in}}%
\pgfpathlineto{\pgfqpoint{4.854720in}{1.895904in}}%
\pgfpathlineto{\pgfqpoint{4.857531in}{1.756268in}}%
\pgfpathlineto{\pgfqpoint{4.860342in}{1.585325in}}%
\pgfpathlineto{\pgfqpoint{4.863152in}{1.737616in}}%
\pgfpathlineto{\pgfqpoint{4.865963in}{1.784372in}}%
\pgfpathlineto{\pgfqpoint{4.868774in}{1.660779in}}%
\pgfpathlineto{\pgfqpoint{4.871584in}{1.643472in}}%
\pgfpathlineto{\pgfqpoint{4.874395in}{1.461308in}}%
\pgfpathlineto{\pgfqpoint{4.877206in}{1.787250in}}%
\pgfpathlineto{\pgfqpoint{4.880017in}{1.966496in}}%
\pgfpathlineto{\pgfqpoint{4.882827in}{1.808192in}}%
\pgfpathlineto{\pgfqpoint{4.885638in}{1.821114in}}%
\pgfpathlineto{\pgfqpoint{4.888449in}{1.516568in}}%
\pgfpathlineto{\pgfqpoint{4.891259in}{1.744558in}}%
\pgfpathlineto{\pgfqpoint{4.894070in}{1.799631in}}%
\pgfpathlineto{\pgfqpoint{4.896881in}{1.820180in}}%
\pgfpathlineto{\pgfqpoint{4.899691in}{1.557736in}}%
\pgfpathlineto{\pgfqpoint{4.902502in}{1.698152in}}%
\pgfpathlineto{\pgfqpoint{4.905313in}{1.711926in}}%
\pgfpathlineto{\pgfqpoint{4.908123in}{1.757675in}}%
\pgfpathlineto{\pgfqpoint{4.910934in}{1.690499in}}%
\pgfpathlineto{\pgfqpoint{4.913745in}{1.685927in}}%
\pgfpathlineto{\pgfqpoint{4.916555in}{1.659979in}}%
\pgfpathlineto{\pgfqpoint{4.919366in}{1.748437in}}%
\pgfpathlineto{\pgfqpoint{4.922177in}{1.559048in}}%
\pgfpathlineto{\pgfqpoint{4.927798in}{1.900585in}}%
\pgfpathlineto{\pgfqpoint{4.930609in}{1.672352in}}%
\pgfpathlineto{\pgfqpoint{4.933420in}{1.945836in}}%
\pgfpathlineto{\pgfqpoint{4.936230in}{1.700959in}}%
\pgfpathlineto{\pgfqpoint{4.939041in}{1.636838in}}%
\pgfpathlineto{\pgfqpoint{4.941852in}{1.729331in}}%
\pgfpathlineto{\pgfqpoint{4.944662in}{1.576826in}}%
\pgfpathlineto{\pgfqpoint{4.947473in}{1.621288in}}%
\pgfpathlineto{\pgfqpoint{4.950284in}{1.777767in}}%
\pgfpathlineto{\pgfqpoint{4.953094in}{1.627383in}}%
\pgfpathlineto{\pgfqpoint{4.955905in}{1.743156in}}%
\pgfpathlineto{\pgfqpoint{4.958716in}{1.728016in}}%
\pgfpathlineto{\pgfqpoint{4.961526in}{1.312957in}}%
\pgfpathlineto{\pgfqpoint{4.964337in}{1.679739in}}%
\pgfpathlineto{\pgfqpoint{4.967148in}{1.834225in}}%
\pgfpathlineto{\pgfqpoint{4.969959in}{1.796696in}}%
\pgfpathlineto{\pgfqpoint{4.972769in}{1.789990in}}%
\pgfpathlineto{\pgfqpoint{4.975580in}{1.720544in}}%
\pgfpathlineto{\pgfqpoint{4.978391in}{1.714503in}}%
\pgfpathlineto{\pgfqpoint{4.981201in}{1.651536in}}%
\pgfpathlineto{\pgfqpoint{4.984012in}{1.828061in}}%
\pgfpathlineto{\pgfqpoint{4.986823in}{1.764786in}}%
\pgfpathlineto{\pgfqpoint{4.989633in}{1.731972in}}%
\pgfpathlineto{\pgfqpoint{4.992444in}{1.736291in}}%
\pgfpathlineto{\pgfqpoint{4.995255in}{1.856192in}}%
\pgfpathlineto{\pgfqpoint{4.998065in}{1.643831in}}%
\pgfpathlineto{\pgfqpoint{5.000876in}{1.770805in}}%
\pgfpathlineto{\pgfqpoint{5.003687in}{1.641006in}}%
\pgfpathlineto{\pgfqpoint{5.006497in}{1.582136in}}%
\pgfpathlineto{\pgfqpoint{5.009308in}{1.728738in}}%
\pgfpathlineto{\pgfqpoint{5.012119in}{1.578662in}}%
\pgfpathlineto{\pgfqpoint{5.014930in}{1.770139in}}%
\pgfpathlineto{\pgfqpoint{5.017740in}{1.689082in}}%
\pgfpathlineto{\pgfqpoint{5.020551in}{1.819433in}}%
\pgfpathlineto{\pgfqpoint{5.023362in}{1.845865in}}%
\pgfpathlineto{\pgfqpoint{5.026172in}{1.736729in}}%
\pgfpathlineto{\pgfqpoint{5.028983in}{1.679057in}}%
\pgfpathlineto{\pgfqpoint{5.031794in}{1.738076in}}%
\pgfpathlineto{\pgfqpoint{5.034604in}{1.683398in}}%
\pgfpathlineto{\pgfqpoint{5.037415in}{1.846540in}}%
\pgfpathlineto{\pgfqpoint{5.040226in}{1.679205in}}%
\pgfpathlineto{\pgfqpoint{5.043036in}{1.746073in}}%
\pgfpathlineto{\pgfqpoint{5.045847in}{1.769923in}}%
\pgfpathlineto{\pgfqpoint{5.048658in}{1.716023in}}%
\pgfpathlineto{\pgfqpoint{5.051468in}{1.744130in}}%
\pgfpathlineto{\pgfqpoint{5.054279in}{1.646967in}}%
\pgfpathlineto{\pgfqpoint{5.057090in}{1.655319in}}%
\pgfpathlineto{\pgfqpoint{5.059901in}{1.625454in}}%
\pgfpathlineto{\pgfqpoint{5.062711in}{1.772719in}}%
\pgfpathlineto{\pgfqpoint{5.065522in}{1.448673in}}%
\pgfpathlineto{\pgfqpoint{5.068333in}{1.829108in}}%
\pgfpathlineto{\pgfqpoint{5.071143in}{1.521915in}}%
\pgfpathlineto{\pgfqpoint{5.076765in}{1.903055in}}%
\pgfpathlineto{\pgfqpoint{5.079575in}{1.603965in}}%
\pgfpathlineto{\pgfqpoint{5.082386in}{1.852145in}}%
\pgfpathlineto{\pgfqpoint{5.085197in}{1.749717in}}%
\pgfpathlineto{\pgfqpoint{5.088007in}{1.827675in}}%
\pgfpathlineto{\pgfqpoint{5.090818in}{1.202656in}}%
\pgfpathlineto{\pgfqpoint{5.093629in}{2.075206in}}%
\pgfpathlineto{\pgfqpoint{5.096439in}{1.431915in}}%
\pgfpathlineto{\pgfqpoint{5.099250in}{1.220258in}}%
\pgfpathlineto{\pgfqpoint{5.102061in}{1.857169in}}%
\pgfpathlineto{\pgfqpoint{5.104871in}{1.858316in}}%
\pgfpathlineto{\pgfqpoint{5.107682in}{1.926148in}}%
\pgfpathlineto{\pgfqpoint{5.110493in}{1.859734in}}%
\pgfpathlineto{\pgfqpoint{5.113304in}{1.659748in}}%
\pgfpathlineto{\pgfqpoint{5.116114in}{1.842997in}}%
\pgfpathlineto{\pgfqpoint{5.118925in}{1.671141in}}%
\pgfpathlineto{\pgfqpoint{5.121736in}{1.637600in}}%
\pgfpathlineto{\pgfqpoint{5.124546in}{1.815943in}}%
\pgfpathlineto{\pgfqpoint{5.130168in}{1.496192in}}%
\pgfpathlineto{\pgfqpoint{5.132978in}{1.689207in}}%
\pgfpathlineto{\pgfqpoint{5.135789in}{1.732798in}}%
\pgfpathlineto{\pgfqpoint{5.138600in}{1.731291in}}%
\pgfpathlineto{\pgfqpoint{5.141410in}{1.732589in}}%
\pgfpathlineto{\pgfqpoint{5.144221in}{1.833873in}}%
\pgfpathlineto{\pgfqpoint{5.149842in}{1.646405in}}%
\pgfpathlineto{\pgfqpoint{5.149842in}{1.646405in}}%
\pgfusepath{stroke}%
\end{pgfscope}%
\begin{pgfscope}%
\pgfpathrectangle{\pgfqpoint{0.711206in}{0.331635in}}{\pgfqpoint{4.650000in}{3.020000in}}%
\pgfusepath{clip}%
\pgfsetroundcap%
\pgfsetroundjoin%
\pgfsetlinewidth{1.505625pt}%
\definecolor{currentstroke}{rgb}{1.000000,0.647059,0.000000}%
\pgfsetstrokecolor{currentstroke}%
\pgfsetdash{}{0pt}%
\pgfpathmoveto{\pgfqpoint{0.922570in}{1.577636in}}%
\pgfpathlineto{\pgfqpoint{0.925380in}{1.682046in}}%
\pgfpathlineto{\pgfqpoint{0.928191in}{1.594938in}}%
\pgfpathlineto{\pgfqpoint{0.931002in}{1.604087in}}%
\pgfpathlineto{\pgfqpoint{0.933812in}{0.965695in}}%
\pgfpathlineto{\pgfqpoint{0.936623in}{1.713509in}}%
\pgfpathlineto{\pgfqpoint{0.939434in}{1.684309in}}%
\pgfpathlineto{\pgfqpoint{0.942245in}{1.688979in}}%
\pgfpathlineto{\pgfqpoint{0.945055in}{1.769461in}}%
\pgfpathlineto{\pgfqpoint{0.947866in}{1.661655in}}%
\pgfpathlineto{\pgfqpoint{0.950677in}{1.656260in}}%
\pgfpathlineto{\pgfqpoint{0.953487in}{1.670542in}}%
\pgfpathlineto{\pgfqpoint{0.956298in}{1.667256in}}%
\pgfpathlineto{\pgfqpoint{0.959109in}{1.671902in}}%
\pgfpathlineto{\pgfqpoint{0.961919in}{1.673168in}}%
\pgfpathlineto{\pgfqpoint{0.964730in}{1.717988in}}%
\pgfpathlineto{\pgfqpoint{0.967541in}{1.668758in}}%
\pgfpathlineto{\pgfqpoint{0.970351in}{1.574531in}}%
\pgfpathlineto{\pgfqpoint{0.973162in}{1.834383in}}%
\pgfpathlineto{\pgfqpoint{0.975973in}{1.647448in}}%
\pgfpathlineto{\pgfqpoint{0.978783in}{1.723302in}}%
\pgfpathlineto{\pgfqpoint{0.981594in}{1.701985in}}%
\pgfpathlineto{\pgfqpoint{0.984405in}{1.746065in}}%
\pgfpathlineto{\pgfqpoint{0.987216in}{1.767933in}}%
\pgfpathlineto{\pgfqpoint{0.990026in}{1.784553in}}%
\pgfpathlineto{\pgfqpoint{0.992837in}{1.664608in}}%
\pgfpathlineto{\pgfqpoint{0.995648in}{1.865936in}}%
\pgfpathlineto{\pgfqpoint{0.998458in}{1.668107in}}%
\pgfpathlineto{\pgfqpoint{1.001269in}{1.780157in}}%
\pgfpathlineto{\pgfqpoint{1.004080in}{1.736830in}}%
\pgfpathlineto{\pgfqpoint{1.006890in}{1.739096in}}%
\pgfpathlineto{\pgfqpoint{1.009701in}{1.814887in}}%
\pgfpathlineto{\pgfqpoint{1.012512in}{1.683079in}}%
\pgfpathlineto{\pgfqpoint{1.015322in}{1.833198in}}%
\pgfpathlineto{\pgfqpoint{1.018133in}{1.650378in}}%
\pgfpathlineto{\pgfqpoint{1.020944in}{1.762366in}}%
\pgfpathlineto{\pgfqpoint{1.023754in}{1.744078in}}%
\pgfpathlineto{\pgfqpoint{1.026565in}{1.745504in}}%
\pgfpathlineto{\pgfqpoint{1.029376in}{1.740796in}}%
\pgfpathlineto{\pgfqpoint{1.032187in}{1.739925in}}%
\pgfpathlineto{\pgfqpoint{1.034997in}{1.734394in}}%
\pgfpathlineto{\pgfqpoint{1.037808in}{1.735944in}}%
\pgfpathlineto{\pgfqpoint{1.040619in}{1.741835in}}%
\pgfpathlineto{\pgfqpoint{1.046240in}{1.735605in}}%
\pgfpathlineto{\pgfqpoint{1.049051in}{1.736842in}}%
\pgfpathlineto{\pgfqpoint{1.051861in}{1.734446in}}%
\pgfpathlineto{\pgfqpoint{1.054672in}{1.734324in}}%
\pgfpathlineto{\pgfqpoint{1.057483in}{1.732663in}}%
\pgfpathlineto{\pgfqpoint{1.060293in}{1.737314in}}%
\pgfpathlineto{\pgfqpoint{1.063104in}{1.780290in}}%
\pgfpathlineto{\pgfqpoint{1.065915in}{1.701832in}}%
\pgfpathlineto{\pgfqpoint{1.068725in}{1.745458in}}%
\pgfpathlineto{\pgfqpoint{1.071536in}{1.695525in}}%
\pgfpathlineto{\pgfqpoint{1.074347in}{1.741976in}}%
\pgfpathlineto{\pgfqpoint{1.077158in}{1.738776in}}%
\pgfpathlineto{\pgfqpoint{1.079968in}{1.740324in}}%
\pgfpathlineto{\pgfqpoint{1.082779in}{1.733397in}}%
\pgfpathlineto{\pgfqpoint{1.085590in}{1.728868in}}%
\pgfpathlineto{\pgfqpoint{1.088400in}{1.704337in}}%
\pgfpathlineto{\pgfqpoint{1.091211in}{1.745320in}}%
\pgfpathlineto{\pgfqpoint{1.094022in}{1.745281in}}%
\pgfpathlineto{\pgfqpoint{1.096832in}{1.701464in}}%
\pgfpathlineto{\pgfqpoint{1.099643in}{1.774803in}}%
\pgfpathlineto{\pgfqpoint{1.102454in}{1.736862in}}%
\pgfpathlineto{\pgfqpoint{1.105264in}{1.713596in}}%
\pgfpathlineto{\pgfqpoint{1.108075in}{1.703351in}}%
\pgfpathlineto{\pgfqpoint{1.110886in}{1.733609in}}%
\pgfpathlineto{\pgfqpoint{1.113696in}{1.728855in}}%
\pgfpathlineto{\pgfqpoint{1.116507in}{1.730609in}}%
\pgfpathlineto{\pgfqpoint{1.124939in}{1.727203in}}%
\pgfpathlineto{\pgfqpoint{1.127750in}{1.721754in}}%
\pgfpathlineto{\pgfqpoint{1.130561in}{1.735768in}}%
\pgfpathlineto{\pgfqpoint{1.133371in}{1.727120in}}%
\pgfpathlineto{\pgfqpoint{1.136182in}{1.728944in}}%
\pgfpathlineto{\pgfqpoint{1.138993in}{1.728984in}}%
\pgfpathlineto{\pgfqpoint{1.141803in}{1.727677in}}%
\pgfpathlineto{\pgfqpoint{1.144614in}{1.727453in}}%
\pgfpathlineto{\pgfqpoint{1.147425in}{1.725635in}}%
\pgfpathlineto{\pgfqpoint{1.150235in}{1.716264in}}%
\pgfpathlineto{\pgfqpoint{1.153046in}{1.718112in}}%
\pgfpathlineto{\pgfqpoint{1.155857in}{1.718923in}}%
\pgfpathlineto{\pgfqpoint{1.161478in}{1.716675in}}%
\pgfpathlineto{\pgfqpoint{1.164289in}{1.717478in}}%
\pgfpathlineto{\pgfqpoint{1.167099in}{1.716862in}}%
\pgfpathlineto{\pgfqpoint{1.169910in}{1.714799in}}%
\pgfpathlineto{\pgfqpoint{1.172721in}{1.714420in}}%
\pgfpathlineto{\pgfqpoint{1.175532in}{1.716770in}}%
\pgfpathlineto{\pgfqpoint{1.181153in}{1.709244in}}%
\pgfpathlineto{\pgfqpoint{1.183964in}{1.714181in}}%
\pgfpathlineto{\pgfqpoint{1.186774in}{1.716916in}}%
\pgfpathlineto{\pgfqpoint{1.192396in}{1.718260in}}%
\pgfpathlineto{\pgfqpoint{1.195206in}{1.717469in}}%
\pgfpathlineto{\pgfqpoint{1.198017in}{1.718378in}}%
\pgfpathlineto{\pgfqpoint{1.206449in}{1.707273in}}%
\pgfpathlineto{\pgfqpoint{1.209260in}{1.709610in}}%
\pgfpathlineto{\pgfqpoint{1.212070in}{1.708979in}}%
\pgfpathlineto{\pgfqpoint{1.214881in}{1.712072in}}%
\pgfpathlineto{\pgfqpoint{1.217692in}{1.712444in}}%
\pgfpathlineto{\pgfqpoint{1.223313in}{1.711525in}}%
\pgfpathlineto{\pgfqpoint{1.226124in}{1.712325in}}%
\pgfpathlineto{\pgfqpoint{1.228935in}{1.709441in}}%
\pgfpathlineto{\pgfqpoint{1.240177in}{1.716173in}}%
\pgfpathlineto{\pgfqpoint{1.242988in}{1.717296in}}%
\pgfpathlineto{\pgfqpoint{1.245799in}{1.713687in}}%
\pgfpathlineto{\pgfqpoint{1.248609in}{1.719399in}}%
\pgfpathlineto{\pgfqpoint{1.251420in}{1.728228in}}%
\pgfpathlineto{\pgfqpoint{1.254231in}{1.703307in}}%
\pgfpathlineto{\pgfqpoint{1.257041in}{1.707703in}}%
\pgfpathlineto{\pgfqpoint{1.259852in}{1.739220in}}%
\pgfpathlineto{\pgfqpoint{1.265474in}{1.671814in}}%
\pgfpathlineto{\pgfqpoint{1.268284in}{1.695959in}}%
\pgfpathlineto{\pgfqpoint{1.271095in}{1.699058in}}%
\pgfpathlineto{\pgfqpoint{1.273906in}{1.733045in}}%
\pgfpathlineto{\pgfqpoint{1.276716in}{1.751187in}}%
\pgfpathlineto{\pgfqpoint{1.279527in}{1.757329in}}%
\pgfpathlineto{\pgfqpoint{1.282338in}{1.760905in}}%
\pgfpathlineto{\pgfqpoint{1.285148in}{1.728676in}}%
\pgfpathlineto{\pgfqpoint{1.287959in}{1.710678in}}%
\pgfpathlineto{\pgfqpoint{1.290770in}{1.682042in}}%
\pgfpathlineto{\pgfqpoint{1.293580in}{1.684265in}}%
\pgfpathlineto{\pgfqpoint{1.296391in}{1.699029in}}%
\pgfpathlineto{\pgfqpoint{1.299202in}{1.732726in}}%
\pgfpathlineto{\pgfqpoint{1.302012in}{1.727226in}}%
\pgfpathlineto{\pgfqpoint{1.307634in}{1.754546in}}%
\pgfpathlineto{\pgfqpoint{1.310445in}{1.750148in}}%
\pgfpathlineto{\pgfqpoint{1.313255in}{1.696594in}}%
\pgfpathlineto{\pgfqpoint{1.316066in}{1.680670in}}%
\pgfpathlineto{\pgfqpoint{1.318877in}{1.677237in}}%
\pgfpathlineto{\pgfqpoint{1.321687in}{1.703230in}}%
\pgfpathlineto{\pgfqpoint{1.324498in}{1.716471in}}%
\pgfpathlineto{\pgfqpoint{1.330119in}{1.719006in}}%
\pgfpathlineto{\pgfqpoint{1.338551in}{1.718612in}}%
\pgfpathlineto{\pgfqpoint{1.341362in}{1.716337in}}%
\pgfpathlineto{\pgfqpoint{1.344173in}{1.716641in}}%
\pgfpathlineto{\pgfqpoint{1.346983in}{1.740626in}}%
\pgfpathlineto{\pgfqpoint{1.349794in}{1.721327in}}%
\pgfpathlineto{\pgfqpoint{1.352605in}{1.722558in}}%
\pgfpathlineto{\pgfqpoint{1.355415in}{1.717045in}}%
\pgfpathlineto{\pgfqpoint{1.358226in}{1.731187in}}%
\pgfpathlineto{\pgfqpoint{1.363848in}{1.741754in}}%
\pgfpathlineto{\pgfqpoint{1.366658in}{1.732556in}}%
\pgfpathlineto{\pgfqpoint{1.369469in}{1.741489in}}%
\pgfpathlineto{\pgfqpoint{1.372280in}{1.743227in}}%
\pgfpathlineto{\pgfqpoint{1.375090in}{1.729250in}}%
\pgfpathlineto{\pgfqpoint{1.377901in}{1.730777in}}%
\pgfpathlineto{\pgfqpoint{1.380712in}{1.723487in}}%
\pgfpathlineto{\pgfqpoint{1.383522in}{1.737577in}}%
\pgfpathlineto{\pgfqpoint{1.386333in}{1.717913in}}%
\pgfpathlineto{\pgfqpoint{1.389144in}{1.716545in}}%
\pgfpathlineto{\pgfqpoint{1.391954in}{1.728687in}}%
\pgfpathlineto{\pgfqpoint{1.394765in}{1.706400in}}%
\pgfpathlineto{\pgfqpoint{1.397576in}{1.708541in}}%
\pgfpathlineto{\pgfqpoint{1.400386in}{1.723870in}}%
\pgfpathlineto{\pgfqpoint{1.403197in}{1.691796in}}%
\pgfpathlineto{\pgfqpoint{1.406008in}{1.678969in}}%
\pgfpathlineto{\pgfqpoint{1.408819in}{1.675291in}}%
\pgfpathlineto{\pgfqpoint{1.411629in}{1.692325in}}%
\pgfpathlineto{\pgfqpoint{1.414440in}{1.703439in}}%
\pgfpathlineto{\pgfqpoint{1.417251in}{1.710132in}}%
\pgfpathlineto{\pgfqpoint{1.420061in}{1.702566in}}%
\pgfpathlineto{\pgfqpoint{1.422872in}{1.709610in}}%
\pgfpathlineto{\pgfqpoint{1.425683in}{1.709392in}}%
\pgfpathlineto{\pgfqpoint{1.428493in}{1.726359in}}%
\pgfpathlineto{\pgfqpoint{1.431304in}{1.718208in}}%
\pgfpathlineto{\pgfqpoint{1.434115in}{1.741238in}}%
\pgfpathlineto{\pgfqpoint{1.436925in}{1.724697in}}%
\pgfpathlineto{\pgfqpoint{1.439736in}{1.721179in}}%
\pgfpathlineto{\pgfqpoint{1.442547in}{1.697047in}}%
\pgfpathlineto{\pgfqpoint{1.445357in}{1.710889in}}%
\pgfpathlineto{\pgfqpoint{1.448168in}{1.688155in}}%
\pgfpathlineto{\pgfqpoint{1.450979in}{1.685254in}}%
\pgfpathlineto{\pgfqpoint{1.453790in}{1.688708in}}%
\pgfpathlineto{\pgfqpoint{1.459411in}{1.734596in}}%
\pgfpathlineto{\pgfqpoint{1.462222in}{1.727240in}}%
\pgfpathlineto{\pgfqpoint{1.465032in}{1.711408in}}%
\pgfpathlineto{\pgfqpoint{1.467843in}{1.714203in}}%
\pgfpathlineto{\pgfqpoint{1.470654in}{1.709524in}}%
\pgfpathlineto{\pgfqpoint{1.476275in}{1.692533in}}%
\pgfpathlineto{\pgfqpoint{1.479086in}{1.706619in}}%
\pgfpathlineto{\pgfqpoint{1.481896in}{1.730887in}}%
\pgfpathlineto{\pgfqpoint{1.484707in}{1.740390in}}%
\pgfpathlineto{\pgfqpoint{1.487518in}{1.759030in}}%
\pgfpathlineto{\pgfqpoint{1.490328in}{1.754387in}}%
\pgfpathlineto{\pgfqpoint{1.493139in}{1.741558in}}%
\pgfpathlineto{\pgfqpoint{1.495950in}{1.732992in}}%
\pgfpathlineto{\pgfqpoint{1.498761in}{1.728555in}}%
\pgfpathlineto{\pgfqpoint{1.501571in}{1.683940in}}%
\pgfpathlineto{\pgfqpoint{1.504382in}{1.694535in}}%
\pgfpathlineto{\pgfqpoint{1.507193in}{1.717283in}}%
\pgfpathlineto{\pgfqpoint{1.510003in}{1.696731in}}%
\pgfpathlineto{\pgfqpoint{1.512814in}{1.714828in}}%
\pgfpathlineto{\pgfqpoint{1.515625in}{1.722611in}}%
\pgfpathlineto{\pgfqpoint{1.518435in}{1.715823in}}%
\pgfpathlineto{\pgfqpoint{1.521246in}{1.715449in}}%
\pgfpathlineto{\pgfqpoint{1.526867in}{1.738767in}}%
\pgfpathlineto{\pgfqpoint{1.529678in}{1.737322in}}%
\pgfpathlineto{\pgfqpoint{1.532489in}{1.709559in}}%
\pgfpathlineto{\pgfqpoint{1.535299in}{1.718717in}}%
\pgfpathlineto{\pgfqpoint{1.543731in}{1.719536in}}%
\pgfpathlineto{\pgfqpoint{1.549353in}{1.718670in}}%
\pgfpathlineto{\pgfqpoint{1.554974in}{1.719074in}}%
\pgfpathlineto{\pgfqpoint{1.557785in}{1.719665in}}%
\pgfpathlineto{\pgfqpoint{1.563406in}{1.718459in}}%
\pgfpathlineto{\pgfqpoint{1.577460in}{1.718492in}}%
\pgfpathlineto{\pgfqpoint{1.585892in}{1.717241in}}%
\pgfpathlineto{\pgfqpoint{1.591513in}{1.718728in}}%
\pgfpathlineto{\pgfqpoint{1.594324in}{1.717951in}}%
\pgfpathlineto{\pgfqpoint{1.597135in}{1.719240in}}%
\pgfpathlineto{\pgfqpoint{1.602756in}{1.717132in}}%
\pgfpathlineto{\pgfqpoint{1.605567in}{1.733526in}}%
\pgfpathlineto{\pgfqpoint{1.608377in}{1.737207in}}%
\pgfpathlineto{\pgfqpoint{1.611188in}{1.743937in}}%
\pgfpathlineto{\pgfqpoint{1.613999in}{1.715090in}}%
\pgfpathlineto{\pgfqpoint{1.616809in}{1.719663in}}%
\pgfpathlineto{\pgfqpoint{1.622431in}{1.719989in}}%
\pgfpathlineto{\pgfqpoint{1.628052in}{1.720785in}}%
\pgfpathlineto{\pgfqpoint{1.630863in}{1.721600in}}%
\pgfpathlineto{\pgfqpoint{1.633673in}{1.720999in}}%
\pgfpathlineto{\pgfqpoint{1.636484in}{1.688584in}}%
\pgfpathlineto{\pgfqpoint{1.639295in}{1.698224in}}%
\pgfpathlineto{\pgfqpoint{1.644916in}{1.710190in}}%
\pgfpathlineto{\pgfqpoint{1.647727in}{1.715446in}}%
\pgfpathlineto{\pgfqpoint{1.650538in}{1.726865in}}%
\pgfpathlineto{\pgfqpoint{1.653348in}{1.723738in}}%
\pgfpathlineto{\pgfqpoint{1.656159in}{1.724952in}}%
\pgfpathlineto{\pgfqpoint{1.658970in}{1.723291in}}%
\pgfpathlineto{\pgfqpoint{1.661780in}{1.722848in}}%
\pgfpathlineto{\pgfqpoint{1.664591in}{1.711733in}}%
\pgfpathlineto{\pgfqpoint{1.667402in}{1.710359in}}%
\pgfpathlineto{\pgfqpoint{1.670212in}{1.721108in}}%
\pgfpathlineto{\pgfqpoint{1.673023in}{1.705641in}}%
\pgfpathlineto{\pgfqpoint{1.675834in}{1.704356in}}%
\pgfpathlineto{\pgfqpoint{1.678644in}{1.716741in}}%
\pgfpathlineto{\pgfqpoint{1.681455in}{1.701311in}}%
\pgfpathlineto{\pgfqpoint{1.684266in}{1.696775in}}%
\pgfpathlineto{\pgfqpoint{1.687077in}{1.719818in}}%
\pgfpathlineto{\pgfqpoint{1.689887in}{1.715703in}}%
\pgfpathlineto{\pgfqpoint{1.692698in}{1.726160in}}%
\pgfpathlineto{\pgfqpoint{1.695509in}{1.724152in}}%
\pgfpathlineto{\pgfqpoint{1.698319in}{1.729758in}}%
\pgfpathlineto{\pgfqpoint{1.701130in}{1.723798in}}%
\pgfpathlineto{\pgfqpoint{1.703941in}{1.713591in}}%
\pgfpathlineto{\pgfqpoint{1.706751in}{1.717717in}}%
\pgfpathlineto{\pgfqpoint{1.709562in}{1.746975in}}%
\pgfpathlineto{\pgfqpoint{1.712373in}{1.720681in}}%
\pgfpathlineto{\pgfqpoint{1.715183in}{1.710956in}}%
\pgfpathlineto{\pgfqpoint{1.717994in}{1.745707in}}%
\pgfpathlineto{\pgfqpoint{1.723615in}{1.712415in}}%
\pgfpathlineto{\pgfqpoint{1.726426in}{1.724792in}}%
\pgfpathlineto{\pgfqpoint{1.729237in}{1.717229in}}%
\pgfpathlineto{\pgfqpoint{1.737669in}{1.717968in}}%
\pgfpathlineto{\pgfqpoint{1.743290in}{1.717662in}}%
\pgfpathlineto{\pgfqpoint{1.746101in}{1.717748in}}%
\pgfpathlineto{\pgfqpoint{1.751722in}{1.716987in}}%
\pgfpathlineto{\pgfqpoint{1.754533in}{1.717319in}}%
\pgfpathlineto{\pgfqpoint{1.757344in}{1.745343in}}%
\pgfpathlineto{\pgfqpoint{1.760154in}{1.716308in}}%
\pgfpathlineto{\pgfqpoint{1.762965in}{1.763175in}}%
\pgfpathlineto{\pgfqpoint{1.765776in}{1.730052in}}%
\pgfpathlineto{\pgfqpoint{1.768586in}{1.744009in}}%
\pgfpathlineto{\pgfqpoint{1.771397in}{1.721693in}}%
\pgfpathlineto{\pgfqpoint{1.774208in}{1.691537in}}%
\pgfpathlineto{\pgfqpoint{1.777018in}{1.694251in}}%
\pgfpathlineto{\pgfqpoint{1.782640in}{1.685706in}}%
\pgfpathlineto{\pgfqpoint{1.785451in}{1.702081in}}%
\pgfpathlineto{\pgfqpoint{1.788261in}{1.699549in}}%
\pgfpathlineto{\pgfqpoint{1.791072in}{1.720844in}}%
\pgfpathlineto{\pgfqpoint{1.793883in}{1.711002in}}%
\pgfpathlineto{\pgfqpoint{1.796693in}{1.721188in}}%
\pgfpathlineto{\pgfqpoint{1.799504in}{1.715355in}}%
\pgfpathlineto{\pgfqpoint{1.802315in}{1.726752in}}%
\pgfpathlineto{\pgfqpoint{1.805125in}{1.713986in}}%
\pgfpathlineto{\pgfqpoint{1.807936in}{1.712369in}}%
\pgfpathlineto{\pgfqpoint{1.810747in}{1.725655in}}%
\pgfpathlineto{\pgfqpoint{1.813557in}{1.754776in}}%
\pgfpathlineto{\pgfqpoint{1.816368in}{1.737841in}}%
\pgfpathlineto{\pgfqpoint{1.819179in}{1.751397in}}%
\pgfpathlineto{\pgfqpoint{1.821989in}{1.756826in}}%
\pgfpathlineto{\pgfqpoint{1.824800in}{1.742104in}}%
\pgfpathlineto{\pgfqpoint{1.827611in}{1.746678in}}%
\pgfpathlineto{\pgfqpoint{1.830422in}{1.734345in}}%
\pgfpathlineto{\pgfqpoint{1.833232in}{1.727390in}}%
\pgfpathlineto{\pgfqpoint{1.836043in}{1.718661in}}%
\pgfpathlineto{\pgfqpoint{1.838854in}{1.730180in}}%
\pgfpathlineto{\pgfqpoint{1.841664in}{1.727123in}}%
\pgfpathlineto{\pgfqpoint{1.844475in}{1.725185in}}%
\pgfpathlineto{\pgfqpoint{1.847286in}{1.737334in}}%
\pgfpathlineto{\pgfqpoint{1.850096in}{1.682878in}}%
\pgfpathlineto{\pgfqpoint{1.852907in}{1.662853in}}%
\pgfpathlineto{\pgfqpoint{1.855718in}{1.669866in}}%
\pgfpathlineto{\pgfqpoint{1.858528in}{1.667016in}}%
\pgfpathlineto{\pgfqpoint{1.864150in}{1.678621in}}%
\pgfpathlineto{\pgfqpoint{1.869771in}{1.680830in}}%
\pgfpathlineto{\pgfqpoint{1.872582in}{1.675066in}}%
\pgfpathlineto{\pgfqpoint{1.875393in}{1.662877in}}%
\pgfpathlineto{\pgfqpoint{1.878203in}{1.676095in}}%
\pgfpathlineto{\pgfqpoint{1.881014in}{1.653396in}}%
\pgfpathlineto{\pgfqpoint{1.883825in}{1.671217in}}%
\pgfpathlineto{\pgfqpoint{1.886635in}{1.673121in}}%
\pgfpathlineto{\pgfqpoint{1.889446in}{1.682932in}}%
\pgfpathlineto{\pgfqpoint{1.892257in}{1.696523in}}%
\pgfpathlineto{\pgfqpoint{1.895067in}{1.684212in}}%
\pgfpathlineto{\pgfqpoint{1.897878in}{1.689271in}}%
\pgfpathlineto{\pgfqpoint{1.900689in}{1.702215in}}%
\pgfpathlineto{\pgfqpoint{1.903499in}{1.686629in}}%
\pgfpathlineto{\pgfqpoint{1.906310in}{1.706253in}}%
\pgfpathlineto{\pgfqpoint{1.909121in}{1.695092in}}%
\pgfpathlineto{\pgfqpoint{1.911931in}{1.697468in}}%
\pgfpathlineto{\pgfqpoint{1.914742in}{1.717504in}}%
\pgfpathlineto{\pgfqpoint{1.917553in}{1.705398in}}%
\pgfpathlineto{\pgfqpoint{1.920364in}{1.704250in}}%
\pgfpathlineto{\pgfqpoint{1.923174in}{1.691793in}}%
\pgfpathlineto{\pgfqpoint{1.925985in}{1.710315in}}%
\pgfpathlineto{\pgfqpoint{1.928796in}{1.714867in}}%
\pgfpathlineto{\pgfqpoint{1.931606in}{1.697485in}}%
\pgfpathlineto{\pgfqpoint{1.934417in}{1.708051in}}%
\pgfpathlineto{\pgfqpoint{1.940038in}{1.693111in}}%
\pgfpathlineto{\pgfqpoint{1.942849in}{1.703812in}}%
\pgfpathlineto{\pgfqpoint{1.945660in}{1.731602in}}%
\pgfpathlineto{\pgfqpoint{1.948470in}{1.725799in}}%
\pgfpathlineto{\pgfqpoint{1.951281in}{1.735427in}}%
\pgfpathlineto{\pgfqpoint{1.956902in}{1.711467in}}%
\pgfpathlineto{\pgfqpoint{1.959713in}{1.703818in}}%
\pgfpathlineto{\pgfqpoint{1.962524in}{1.715290in}}%
\pgfpathlineto{\pgfqpoint{1.965334in}{1.701668in}}%
\pgfpathlineto{\pgfqpoint{1.968145in}{1.703846in}}%
\pgfpathlineto{\pgfqpoint{1.970956in}{1.694919in}}%
\pgfpathlineto{\pgfqpoint{1.973767in}{1.675336in}}%
\pgfpathlineto{\pgfqpoint{1.976577in}{1.693814in}}%
\pgfpathlineto{\pgfqpoint{1.979388in}{1.700648in}}%
\pgfpathlineto{\pgfqpoint{1.982199in}{1.705690in}}%
\pgfpathlineto{\pgfqpoint{1.985009in}{1.687746in}}%
\pgfpathlineto{\pgfqpoint{1.987820in}{1.686178in}}%
\pgfpathlineto{\pgfqpoint{1.990631in}{1.690990in}}%
\pgfpathlineto{\pgfqpoint{1.993441in}{1.698185in}}%
\pgfpathlineto{\pgfqpoint{1.996252in}{1.698491in}}%
\pgfpathlineto{\pgfqpoint{1.999063in}{1.692542in}}%
\pgfpathlineto{\pgfqpoint{2.001873in}{1.701321in}}%
\pgfpathlineto{\pgfqpoint{2.004684in}{1.695320in}}%
\pgfpathlineto{\pgfqpoint{2.007495in}{1.713619in}}%
\pgfpathlineto{\pgfqpoint{2.010305in}{1.726372in}}%
\pgfpathlineto{\pgfqpoint{2.013116in}{1.679406in}}%
\pgfpathlineto{\pgfqpoint{2.015927in}{1.691551in}}%
\pgfpathlineto{\pgfqpoint{2.018738in}{1.700853in}}%
\pgfpathlineto{\pgfqpoint{2.021548in}{1.705942in}}%
\pgfpathlineto{\pgfqpoint{2.024359in}{1.803429in}}%
\pgfpathlineto{\pgfqpoint{2.027170in}{1.788026in}}%
\pgfpathlineto{\pgfqpoint{2.029980in}{1.758382in}}%
\pgfpathlineto{\pgfqpoint{2.032791in}{1.756878in}}%
\pgfpathlineto{\pgfqpoint{2.038412in}{1.780283in}}%
\pgfpathlineto{\pgfqpoint{2.041223in}{1.784581in}}%
\pgfpathlineto{\pgfqpoint{2.044034in}{1.782264in}}%
\pgfpathlineto{\pgfqpoint{2.046844in}{1.772392in}}%
\pgfpathlineto{\pgfqpoint{2.049655in}{1.766737in}}%
\pgfpathlineto{\pgfqpoint{2.052466in}{1.755755in}}%
\pgfpathlineto{\pgfqpoint{2.055276in}{1.785909in}}%
\pgfpathlineto{\pgfqpoint{2.058087in}{1.778966in}}%
\pgfpathlineto{\pgfqpoint{2.060898in}{1.750756in}}%
\pgfpathlineto{\pgfqpoint{2.063709in}{1.759799in}}%
\pgfpathlineto{\pgfqpoint{2.066519in}{1.706849in}}%
\pgfpathlineto{\pgfqpoint{2.069330in}{1.706825in}}%
\pgfpathlineto{\pgfqpoint{2.072141in}{1.709452in}}%
\pgfpathlineto{\pgfqpoint{2.074951in}{1.752051in}}%
\pgfpathlineto{\pgfqpoint{2.077762in}{1.754684in}}%
\pgfpathlineto{\pgfqpoint{2.080573in}{1.737988in}}%
\pgfpathlineto{\pgfqpoint{2.083383in}{1.733516in}}%
\pgfpathlineto{\pgfqpoint{2.086194in}{1.741028in}}%
\pgfpathlineto{\pgfqpoint{2.089005in}{1.711968in}}%
\pgfpathlineto{\pgfqpoint{2.091815in}{1.723114in}}%
\pgfpathlineto{\pgfqpoint{2.094626in}{1.723250in}}%
\pgfpathlineto{\pgfqpoint{2.097437in}{1.719458in}}%
\pgfpathlineto{\pgfqpoint{2.100247in}{1.701970in}}%
\pgfpathlineto{\pgfqpoint{2.103058in}{1.651244in}}%
\pgfpathlineto{\pgfqpoint{2.105869in}{1.653713in}}%
\pgfpathlineto{\pgfqpoint{2.108680in}{1.679268in}}%
\pgfpathlineto{\pgfqpoint{2.111490in}{1.655877in}}%
\pgfpathlineto{\pgfqpoint{2.114301in}{1.667847in}}%
\pgfpathlineto{\pgfqpoint{2.117112in}{1.663507in}}%
\pgfpathlineto{\pgfqpoint{2.119922in}{1.662448in}}%
\pgfpathlineto{\pgfqpoint{2.122733in}{1.670013in}}%
\pgfpathlineto{\pgfqpoint{2.125544in}{1.655828in}}%
\pgfpathlineto{\pgfqpoint{2.128354in}{1.691227in}}%
\pgfpathlineto{\pgfqpoint{2.131165in}{1.718782in}}%
\pgfpathlineto{\pgfqpoint{2.133976in}{1.730069in}}%
\pgfpathlineto{\pgfqpoint{2.136786in}{1.716784in}}%
\pgfpathlineto{\pgfqpoint{2.139597in}{1.722491in}}%
\pgfpathlineto{\pgfqpoint{2.142408in}{1.736892in}}%
\pgfpathlineto{\pgfqpoint{2.145218in}{1.719998in}}%
\pgfpathlineto{\pgfqpoint{2.148029in}{1.732529in}}%
\pgfpathlineto{\pgfqpoint{2.150840in}{1.761390in}}%
\pgfpathlineto{\pgfqpoint{2.153651in}{1.736370in}}%
\pgfpathlineto{\pgfqpoint{2.156461in}{1.773800in}}%
\pgfpathlineto{\pgfqpoint{2.159272in}{1.789182in}}%
\pgfpathlineto{\pgfqpoint{2.162083in}{1.783186in}}%
\pgfpathlineto{\pgfqpoint{2.164893in}{1.761880in}}%
\pgfpathlineto{\pgfqpoint{2.167704in}{1.748899in}}%
\pgfpathlineto{\pgfqpoint{2.170515in}{1.744444in}}%
\pgfpathlineto{\pgfqpoint{2.173325in}{1.755626in}}%
\pgfpathlineto{\pgfqpoint{2.176136in}{1.736206in}}%
\pgfpathlineto{\pgfqpoint{2.178947in}{1.725118in}}%
\pgfpathlineto{\pgfqpoint{2.181757in}{1.716444in}}%
\pgfpathlineto{\pgfqpoint{2.184568in}{1.720314in}}%
\pgfpathlineto{\pgfqpoint{2.187379in}{1.721884in}}%
\pgfpathlineto{\pgfqpoint{2.190189in}{1.728540in}}%
\pgfpathlineto{\pgfqpoint{2.193000in}{1.710245in}}%
\pgfpathlineto{\pgfqpoint{2.198621in}{1.713332in}}%
\pgfpathlineto{\pgfqpoint{2.201432in}{1.709261in}}%
\pgfpathlineto{\pgfqpoint{2.204243in}{1.712817in}}%
\pgfpathlineto{\pgfqpoint{2.207054in}{1.749353in}}%
\pgfpathlineto{\pgfqpoint{2.209864in}{1.737222in}}%
\pgfpathlineto{\pgfqpoint{2.212675in}{1.752232in}}%
\pgfpathlineto{\pgfqpoint{2.215486in}{1.748310in}}%
\pgfpathlineto{\pgfqpoint{2.218296in}{1.741639in}}%
\pgfpathlineto{\pgfqpoint{2.221107in}{1.757288in}}%
\pgfpathlineto{\pgfqpoint{2.223918in}{1.747349in}}%
\pgfpathlineto{\pgfqpoint{2.226728in}{1.744226in}}%
\pgfpathlineto{\pgfqpoint{2.229539in}{1.750321in}}%
\pgfpathlineto{\pgfqpoint{2.232350in}{1.736146in}}%
\pgfpathlineto{\pgfqpoint{2.235160in}{1.736220in}}%
\pgfpathlineto{\pgfqpoint{2.237971in}{1.728573in}}%
\pgfpathlineto{\pgfqpoint{2.243592in}{1.753500in}}%
\pgfpathlineto{\pgfqpoint{2.246403in}{1.750653in}}%
\pgfpathlineto{\pgfqpoint{2.249214in}{1.735197in}}%
\pgfpathlineto{\pgfqpoint{2.252025in}{1.734492in}}%
\pgfpathlineto{\pgfqpoint{2.254835in}{1.736704in}}%
\pgfpathlineto{\pgfqpoint{2.257646in}{1.730297in}}%
\pgfpathlineto{\pgfqpoint{2.260457in}{1.729596in}}%
\pgfpathlineto{\pgfqpoint{2.263267in}{1.733743in}}%
\pgfpathlineto{\pgfqpoint{2.266078in}{1.726909in}}%
\pgfpathlineto{\pgfqpoint{2.268889in}{1.744731in}}%
\pgfpathlineto{\pgfqpoint{2.271699in}{1.740728in}}%
\pgfpathlineto{\pgfqpoint{2.274510in}{1.747006in}}%
\pgfpathlineto{\pgfqpoint{2.277321in}{1.747398in}}%
\pgfpathlineto{\pgfqpoint{2.280131in}{1.749786in}}%
\pgfpathlineto{\pgfqpoint{2.282942in}{1.760784in}}%
\pgfpathlineto{\pgfqpoint{2.285753in}{1.732423in}}%
\pgfpathlineto{\pgfqpoint{2.288563in}{1.745339in}}%
\pgfpathlineto{\pgfqpoint{2.291374in}{1.727586in}}%
\pgfpathlineto{\pgfqpoint{2.294185in}{1.727643in}}%
\pgfpathlineto{\pgfqpoint{2.296996in}{1.702620in}}%
\pgfpathlineto{\pgfqpoint{2.299806in}{1.695326in}}%
\pgfpathlineto{\pgfqpoint{2.302617in}{1.693870in}}%
\pgfpathlineto{\pgfqpoint{2.305428in}{1.696064in}}%
\pgfpathlineto{\pgfqpoint{2.308238in}{1.688326in}}%
\pgfpathlineto{\pgfqpoint{2.311049in}{1.687246in}}%
\pgfpathlineto{\pgfqpoint{2.313860in}{1.682606in}}%
\pgfpathlineto{\pgfqpoint{2.316670in}{1.686811in}}%
\pgfpathlineto{\pgfqpoint{2.319481in}{1.683025in}}%
\pgfpathlineto{\pgfqpoint{2.322292in}{1.677427in}}%
\pgfpathlineto{\pgfqpoint{2.325102in}{1.686560in}}%
\pgfpathlineto{\pgfqpoint{2.327913in}{1.687348in}}%
\pgfpathlineto{\pgfqpoint{2.330724in}{1.694944in}}%
\pgfpathlineto{\pgfqpoint{2.333534in}{1.689040in}}%
\pgfpathlineto{\pgfqpoint{2.336345in}{1.687950in}}%
\pgfpathlineto{\pgfqpoint{2.339156in}{1.691052in}}%
\pgfpathlineto{\pgfqpoint{2.341967in}{1.696089in}}%
\pgfpathlineto{\pgfqpoint{2.344777in}{1.707669in}}%
\pgfpathlineto{\pgfqpoint{2.347588in}{1.692006in}}%
\pgfpathlineto{\pgfqpoint{2.350399in}{1.689450in}}%
\pgfpathlineto{\pgfqpoint{2.353209in}{1.699931in}}%
\pgfpathlineto{\pgfqpoint{2.356020in}{1.656869in}}%
\pgfpathlineto{\pgfqpoint{2.358831in}{1.661183in}}%
\pgfpathlineto{\pgfqpoint{2.361641in}{1.659315in}}%
\pgfpathlineto{\pgfqpoint{2.364452in}{1.680028in}}%
\pgfpathlineto{\pgfqpoint{2.367263in}{1.711962in}}%
\pgfpathlineto{\pgfqpoint{2.370073in}{1.736277in}}%
\pgfpathlineto{\pgfqpoint{2.372884in}{1.714900in}}%
\pgfpathlineto{\pgfqpoint{2.375695in}{1.735097in}}%
\pgfpathlineto{\pgfqpoint{2.378505in}{1.718417in}}%
\pgfpathlineto{\pgfqpoint{2.381316in}{1.746436in}}%
\pgfpathlineto{\pgfqpoint{2.384127in}{1.756201in}}%
\pgfpathlineto{\pgfqpoint{2.389748in}{1.748379in}}%
\pgfpathlineto{\pgfqpoint{2.392559in}{1.733828in}}%
\pgfpathlineto{\pgfqpoint{2.395370in}{1.722685in}}%
\pgfpathlineto{\pgfqpoint{2.398180in}{1.730229in}}%
\pgfpathlineto{\pgfqpoint{2.400991in}{1.723254in}}%
\pgfpathlineto{\pgfqpoint{2.403802in}{1.708691in}}%
\pgfpathlineto{\pgfqpoint{2.406612in}{1.711327in}}%
\pgfpathlineto{\pgfqpoint{2.409423in}{1.703786in}}%
\pgfpathlineto{\pgfqpoint{2.412234in}{1.705478in}}%
\pgfpathlineto{\pgfqpoint{2.415044in}{1.717335in}}%
\pgfpathlineto{\pgfqpoint{2.417855in}{1.720706in}}%
\pgfpathlineto{\pgfqpoint{2.420666in}{1.722621in}}%
\pgfpathlineto{\pgfqpoint{2.423476in}{1.710936in}}%
\pgfpathlineto{\pgfqpoint{2.426287in}{1.708523in}}%
\pgfpathlineto{\pgfqpoint{2.429098in}{1.715048in}}%
\pgfpathlineto{\pgfqpoint{2.431908in}{1.715752in}}%
\pgfpathlineto{\pgfqpoint{2.434719in}{1.718975in}}%
\pgfpathlineto{\pgfqpoint{2.437530in}{1.741675in}}%
\pgfpathlineto{\pgfqpoint{2.440341in}{1.723502in}}%
\pgfpathlineto{\pgfqpoint{2.443151in}{1.738234in}}%
\pgfpathlineto{\pgfqpoint{2.445962in}{1.734522in}}%
\pgfpathlineto{\pgfqpoint{2.448773in}{1.728055in}}%
\pgfpathlineto{\pgfqpoint{2.451583in}{1.729959in}}%
\pgfpathlineto{\pgfqpoint{2.454394in}{1.723648in}}%
\pgfpathlineto{\pgfqpoint{2.457205in}{1.730411in}}%
\pgfpathlineto{\pgfqpoint{2.460015in}{1.756903in}}%
\pgfpathlineto{\pgfqpoint{2.462826in}{1.757546in}}%
\pgfpathlineto{\pgfqpoint{2.468447in}{1.734240in}}%
\pgfpathlineto{\pgfqpoint{2.471258in}{1.748007in}}%
\pgfpathlineto{\pgfqpoint{2.474069in}{1.758319in}}%
\pgfpathlineto{\pgfqpoint{2.476879in}{1.752206in}}%
\pgfpathlineto{\pgfqpoint{2.482501in}{1.776784in}}%
\pgfpathlineto{\pgfqpoint{2.485312in}{1.788543in}}%
\pgfpathlineto{\pgfqpoint{2.488122in}{1.788111in}}%
\pgfpathlineto{\pgfqpoint{2.490933in}{1.802599in}}%
\pgfpathlineto{\pgfqpoint{2.493744in}{1.787814in}}%
\pgfpathlineto{\pgfqpoint{2.496554in}{1.792799in}}%
\pgfpathlineto{\pgfqpoint{2.499365in}{1.792793in}}%
\pgfpathlineto{\pgfqpoint{2.502176in}{1.791380in}}%
\pgfpathlineto{\pgfqpoint{2.504986in}{1.815822in}}%
\pgfpathlineto{\pgfqpoint{2.507797in}{1.823002in}}%
\pgfpathlineto{\pgfqpoint{2.510608in}{1.820872in}}%
\pgfpathlineto{\pgfqpoint{2.513418in}{1.808791in}}%
\pgfpathlineto{\pgfqpoint{2.516229in}{1.765340in}}%
\pgfpathlineto{\pgfqpoint{2.519040in}{1.777146in}}%
\pgfpathlineto{\pgfqpoint{2.521850in}{1.738850in}}%
\pgfpathlineto{\pgfqpoint{2.524661in}{1.719660in}}%
\pgfpathlineto{\pgfqpoint{2.527472in}{1.795375in}}%
\pgfpathlineto{\pgfqpoint{2.530283in}{1.715066in}}%
\pgfpathlineto{\pgfqpoint{2.533093in}{1.794113in}}%
\pgfpathlineto{\pgfqpoint{2.535904in}{1.791327in}}%
\pgfpathlineto{\pgfqpoint{2.538715in}{1.717795in}}%
\pgfpathlineto{\pgfqpoint{2.541525in}{1.716471in}}%
\pgfpathlineto{\pgfqpoint{2.544336in}{1.774631in}}%
\pgfpathlineto{\pgfqpoint{2.547147in}{1.749420in}}%
\pgfpathlineto{\pgfqpoint{2.549957in}{1.735951in}}%
\pgfpathlineto{\pgfqpoint{2.552768in}{1.732784in}}%
\pgfpathlineto{\pgfqpoint{2.555579in}{1.710015in}}%
\pgfpathlineto{\pgfqpoint{2.558389in}{1.721791in}}%
\pgfpathlineto{\pgfqpoint{2.561200in}{1.705378in}}%
\pgfpathlineto{\pgfqpoint{2.564011in}{1.719052in}}%
\pgfpathlineto{\pgfqpoint{2.566821in}{1.699009in}}%
\pgfpathlineto{\pgfqpoint{2.569632in}{1.690983in}}%
\pgfpathlineto{\pgfqpoint{2.572443in}{1.697430in}}%
\pgfpathlineto{\pgfqpoint{2.575253in}{1.696924in}}%
\pgfpathlineto{\pgfqpoint{2.578064in}{1.699488in}}%
\pgfpathlineto{\pgfqpoint{2.580875in}{1.707500in}}%
\pgfpathlineto{\pgfqpoint{2.583686in}{1.724420in}}%
\pgfpathlineto{\pgfqpoint{2.586496in}{1.709816in}}%
\pgfpathlineto{\pgfqpoint{2.589307in}{1.708821in}}%
\pgfpathlineto{\pgfqpoint{2.592118in}{1.725731in}}%
\pgfpathlineto{\pgfqpoint{2.594928in}{1.711906in}}%
\pgfpathlineto{\pgfqpoint{2.597739in}{1.718120in}}%
\pgfpathlineto{\pgfqpoint{2.600550in}{1.702433in}}%
\pgfpathlineto{\pgfqpoint{2.603360in}{1.692035in}}%
\pgfpathlineto{\pgfqpoint{2.606171in}{1.701860in}}%
\pgfpathlineto{\pgfqpoint{2.608982in}{1.700817in}}%
\pgfpathlineto{\pgfqpoint{2.611792in}{1.703980in}}%
\pgfpathlineto{\pgfqpoint{2.614603in}{1.715198in}}%
\pgfpathlineto{\pgfqpoint{2.617414in}{1.729276in}}%
\pgfpathlineto{\pgfqpoint{2.620224in}{1.726987in}}%
\pgfpathlineto{\pgfqpoint{2.625846in}{1.716748in}}%
\pgfpathlineto{\pgfqpoint{2.628657in}{1.720158in}}%
\pgfpathlineto{\pgfqpoint{2.631467in}{1.710734in}}%
\pgfpathlineto{\pgfqpoint{2.634278in}{1.722031in}}%
\pgfpathlineto{\pgfqpoint{2.637089in}{1.727141in}}%
\pgfpathlineto{\pgfqpoint{2.639899in}{1.728961in}}%
\pgfpathlineto{\pgfqpoint{2.642710in}{1.733607in}}%
\pgfpathlineto{\pgfqpoint{2.645521in}{1.727635in}}%
\pgfpathlineto{\pgfqpoint{2.648331in}{1.723819in}}%
\pgfpathlineto{\pgfqpoint{2.651142in}{1.728615in}}%
\pgfpathlineto{\pgfqpoint{2.653953in}{1.731675in}}%
\pgfpathlineto{\pgfqpoint{2.656763in}{1.729611in}}%
\pgfpathlineto{\pgfqpoint{2.659574in}{1.737268in}}%
\pgfpathlineto{\pgfqpoint{2.662385in}{1.726268in}}%
\pgfpathlineto{\pgfqpoint{2.665195in}{1.725629in}}%
\pgfpathlineto{\pgfqpoint{2.668006in}{1.722405in}}%
\pgfpathlineto{\pgfqpoint{2.670817in}{1.713535in}}%
\pgfpathlineto{\pgfqpoint{2.673628in}{1.695289in}}%
\pgfpathlineto{\pgfqpoint{2.676438in}{1.696930in}}%
\pgfpathlineto{\pgfqpoint{2.679249in}{1.692281in}}%
\pgfpathlineto{\pgfqpoint{2.682060in}{1.696384in}}%
\pgfpathlineto{\pgfqpoint{2.684870in}{1.709059in}}%
\pgfpathlineto{\pgfqpoint{2.687681in}{1.706285in}}%
\pgfpathlineto{\pgfqpoint{2.690492in}{1.712287in}}%
\pgfpathlineto{\pgfqpoint{2.693302in}{1.705866in}}%
\pgfpathlineto{\pgfqpoint{2.696113in}{1.686370in}}%
\pgfpathlineto{\pgfqpoint{2.698924in}{1.689054in}}%
\pgfpathlineto{\pgfqpoint{2.701734in}{1.690346in}}%
\pgfpathlineto{\pgfqpoint{2.704545in}{1.721823in}}%
\pgfpathlineto{\pgfqpoint{2.707356in}{1.709216in}}%
\pgfpathlineto{\pgfqpoint{2.710166in}{1.716835in}}%
\pgfpathlineto{\pgfqpoint{2.712977in}{1.706413in}}%
\pgfpathlineto{\pgfqpoint{2.715788in}{1.709482in}}%
\pgfpathlineto{\pgfqpoint{2.718599in}{1.703262in}}%
\pgfpathlineto{\pgfqpoint{2.721409in}{1.751972in}}%
\pgfpathlineto{\pgfqpoint{2.724220in}{1.748545in}}%
\pgfpathlineto{\pgfqpoint{2.727031in}{1.747375in}}%
\pgfpathlineto{\pgfqpoint{2.729841in}{1.739349in}}%
\pgfpathlineto{\pgfqpoint{2.732652in}{1.752301in}}%
\pgfpathlineto{\pgfqpoint{2.735463in}{1.742024in}}%
\pgfpathlineto{\pgfqpoint{2.738273in}{1.736049in}}%
\pgfpathlineto{\pgfqpoint{2.741084in}{1.741033in}}%
\pgfpathlineto{\pgfqpoint{2.743895in}{1.735869in}}%
\pgfpathlineto{\pgfqpoint{2.746705in}{1.740788in}}%
\pgfpathlineto{\pgfqpoint{2.749516in}{1.728392in}}%
\pgfpathlineto{\pgfqpoint{2.752327in}{1.725679in}}%
\pgfpathlineto{\pgfqpoint{2.755137in}{1.725296in}}%
\pgfpathlineto{\pgfqpoint{2.757948in}{1.709435in}}%
\pgfpathlineto{\pgfqpoint{2.760759in}{1.705792in}}%
\pgfpathlineto{\pgfqpoint{2.763570in}{1.720753in}}%
\pgfpathlineto{\pgfqpoint{2.766380in}{1.697275in}}%
\pgfpathlineto{\pgfqpoint{2.769191in}{1.698031in}}%
\pgfpathlineto{\pgfqpoint{2.772002in}{1.695704in}}%
\pgfpathlineto{\pgfqpoint{2.774812in}{1.702578in}}%
\pgfpathlineto{\pgfqpoint{2.777623in}{1.703909in}}%
\pgfpathlineto{\pgfqpoint{2.780434in}{1.707282in}}%
\pgfpathlineto{\pgfqpoint{2.783244in}{1.705674in}}%
\pgfpathlineto{\pgfqpoint{2.786055in}{1.705986in}}%
\pgfpathlineto{\pgfqpoint{2.788866in}{1.722504in}}%
\pgfpathlineto{\pgfqpoint{2.791676in}{1.732846in}}%
\pgfpathlineto{\pgfqpoint{2.794487in}{1.718616in}}%
\pgfpathlineto{\pgfqpoint{2.797298in}{1.717594in}}%
\pgfpathlineto{\pgfqpoint{2.800108in}{1.721582in}}%
\pgfpathlineto{\pgfqpoint{2.802919in}{1.721357in}}%
\pgfpathlineto{\pgfqpoint{2.805730in}{1.712464in}}%
\pgfpathlineto{\pgfqpoint{2.808540in}{1.721727in}}%
\pgfpathlineto{\pgfqpoint{2.811351in}{1.706921in}}%
\pgfpathlineto{\pgfqpoint{2.814162in}{1.719830in}}%
\pgfpathlineto{\pgfqpoint{2.816973in}{1.724811in}}%
\pgfpathlineto{\pgfqpoint{2.819783in}{1.720343in}}%
\pgfpathlineto{\pgfqpoint{2.822594in}{1.705171in}}%
\pgfpathlineto{\pgfqpoint{2.825405in}{1.717551in}}%
\pgfpathlineto{\pgfqpoint{2.828215in}{1.712819in}}%
\pgfpathlineto{\pgfqpoint{2.831026in}{1.714930in}}%
\pgfpathlineto{\pgfqpoint{2.833837in}{1.729260in}}%
\pgfpathlineto{\pgfqpoint{2.836647in}{1.735115in}}%
\pgfpathlineto{\pgfqpoint{2.839458in}{1.720595in}}%
\pgfpathlineto{\pgfqpoint{2.842269in}{1.741158in}}%
\pgfpathlineto{\pgfqpoint{2.845079in}{1.729239in}}%
\pgfpathlineto{\pgfqpoint{2.847890in}{1.732466in}}%
\pgfpathlineto{\pgfqpoint{2.850701in}{1.715459in}}%
\pgfpathlineto{\pgfqpoint{2.853511in}{1.735063in}}%
\pgfpathlineto{\pgfqpoint{2.856322in}{1.732655in}}%
\pgfpathlineto{\pgfqpoint{2.859133in}{1.715858in}}%
\pgfpathlineto{\pgfqpoint{2.861944in}{1.717264in}}%
\pgfpathlineto{\pgfqpoint{2.864754in}{1.739495in}}%
\pgfpathlineto{\pgfqpoint{2.867565in}{1.707678in}}%
\pgfpathlineto{\pgfqpoint{2.870376in}{1.738841in}}%
\pgfpathlineto{\pgfqpoint{2.873186in}{1.750174in}}%
\pgfpathlineto{\pgfqpoint{2.875997in}{1.745280in}}%
\pgfpathlineto{\pgfqpoint{2.878808in}{1.747043in}}%
\pgfpathlineto{\pgfqpoint{2.881618in}{1.752786in}}%
\pgfpathlineto{\pgfqpoint{2.884429in}{1.726032in}}%
\pgfpathlineto{\pgfqpoint{2.887240in}{1.705644in}}%
\pgfpathlineto{\pgfqpoint{2.890050in}{1.697798in}}%
\pgfpathlineto{\pgfqpoint{2.892861in}{1.675026in}}%
\pgfpathlineto{\pgfqpoint{2.895672in}{1.695267in}}%
\pgfpathlineto{\pgfqpoint{2.898482in}{1.681241in}}%
\pgfpathlineto{\pgfqpoint{2.901293in}{1.692968in}}%
\pgfpathlineto{\pgfqpoint{2.904104in}{1.698405in}}%
\pgfpathlineto{\pgfqpoint{2.906915in}{1.683788in}}%
\pgfpathlineto{\pgfqpoint{2.909725in}{1.702269in}}%
\pgfpathlineto{\pgfqpoint{2.912536in}{1.579768in}}%
\pgfpathlineto{\pgfqpoint{2.918157in}{1.638798in}}%
\pgfpathlineto{\pgfqpoint{2.920968in}{1.658687in}}%
\pgfpathlineto{\pgfqpoint{2.923779in}{1.662932in}}%
\pgfpathlineto{\pgfqpoint{2.926589in}{1.682127in}}%
\pgfpathlineto{\pgfqpoint{2.929400in}{1.688680in}}%
\pgfpathlineto{\pgfqpoint{2.932211in}{1.707365in}}%
\pgfpathlineto{\pgfqpoint{2.935021in}{1.712578in}}%
\pgfpathlineto{\pgfqpoint{2.937832in}{1.697876in}}%
\pgfpathlineto{\pgfqpoint{2.943453in}{1.721066in}}%
\pgfpathlineto{\pgfqpoint{2.946264in}{1.715415in}}%
\pgfpathlineto{\pgfqpoint{2.949075in}{1.716305in}}%
\pgfpathlineto{\pgfqpoint{2.951886in}{1.715793in}}%
\pgfpathlineto{\pgfqpoint{2.957507in}{1.701576in}}%
\pgfpathlineto{\pgfqpoint{2.960318in}{1.705976in}}%
\pgfpathlineto{\pgfqpoint{2.963128in}{1.699832in}}%
\pgfpathlineto{\pgfqpoint{2.965939in}{1.706704in}}%
\pgfpathlineto{\pgfqpoint{2.968750in}{1.707233in}}%
\pgfpathlineto{\pgfqpoint{2.971560in}{1.711205in}}%
\pgfpathlineto{\pgfqpoint{2.974371in}{1.706354in}}%
\pgfpathlineto{\pgfqpoint{2.977182in}{1.704526in}}%
\pgfpathlineto{\pgfqpoint{2.979992in}{1.711484in}}%
\pgfpathlineto{\pgfqpoint{2.982803in}{1.705216in}}%
\pgfpathlineto{\pgfqpoint{2.985614in}{1.711150in}}%
\pgfpathlineto{\pgfqpoint{2.988424in}{1.709643in}}%
\pgfpathlineto{\pgfqpoint{2.991235in}{1.722304in}}%
\pgfpathlineto{\pgfqpoint{2.994046in}{1.714748in}}%
\pgfpathlineto{\pgfqpoint{2.996856in}{1.741078in}}%
\pgfpathlineto{\pgfqpoint{2.999667in}{1.731556in}}%
\pgfpathlineto{\pgfqpoint{3.002478in}{1.738200in}}%
\pgfpathlineto{\pgfqpoint{3.005289in}{1.707212in}}%
\pgfpathlineto{\pgfqpoint{3.008099in}{1.692210in}}%
\pgfpathlineto{\pgfqpoint{3.010910in}{1.711204in}}%
\pgfpathlineto{\pgfqpoint{3.013721in}{1.703517in}}%
\pgfpathlineto{\pgfqpoint{3.016531in}{1.704796in}}%
\pgfpathlineto{\pgfqpoint{3.019342in}{1.700532in}}%
\pgfpathlineto{\pgfqpoint{3.022153in}{1.710933in}}%
\pgfpathlineto{\pgfqpoint{3.024963in}{1.718102in}}%
\pgfpathlineto{\pgfqpoint{3.027774in}{1.721220in}}%
\pgfpathlineto{\pgfqpoint{3.030585in}{1.728740in}}%
\pgfpathlineto{\pgfqpoint{3.033395in}{1.713615in}}%
\pgfpathlineto{\pgfqpoint{3.036206in}{1.738044in}}%
\pgfpathlineto{\pgfqpoint{3.039017in}{1.736140in}}%
\pgfpathlineto{\pgfqpoint{3.041827in}{1.715199in}}%
\pgfpathlineto{\pgfqpoint{3.044638in}{1.702554in}}%
\pgfpathlineto{\pgfqpoint{3.047449in}{1.723387in}}%
\pgfpathlineto{\pgfqpoint{3.050260in}{1.723031in}}%
\pgfpathlineto{\pgfqpoint{3.053070in}{1.717428in}}%
\pgfpathlineto{\pgfqpoint{3.055881in}{1.738405in}}%
\pgfpathlineto{\pgfqpoint{3.058692in}{1.739071in}}%
\pgfpathlineto{\pgfqpoint{3.061502in}{1.722763in}}%
\pgfpathlineto{\pgfqpoint{3.064313in}{1.712810in}}%
\pgfpathlineto{\pgfqpoint{3.067124in}{1.712277in}}%
\pgfpathlineto{\pgfqpoint{3.069934in}{1.712608in}}%
\pgfpathlineto{\pgfqpoint{3.072745in}{1.711156in}}%
\pgfpathlineto{\pgfqpoint{3.075556in}{1.720114in}}%
\pgfpathlineto{\pgfqpoint{3.078366in}{1.740724in}}%
\pgfpathlineto{\pgfqpoint{3.081177in}{1.748384in}}%
\pgfpathlineto{\pgfqpoint{3.083988in}{1.729478in}}%
\pgfpathlineto{\pgfqpoint{3.086798in}{1.698454in}}%
\pgfpathlineto{\pgfqpoint{3.089609in}{1.703042in}}%
\pgfpathlineto{\pgfqpoint{3.092420in}{1.691089in}}%
\pgfpathlineto{\pgfqpoint{3.098041in}{1.672694in}}%
\pgfpathlineto{\pgfqpoint{3.100852in}{1.699695in}}%
\pgfpathlineto{\pgfqpoint{3.103663in}{1.711157in}}%
\pgfpathlineto{\pgfqpoint{3.106473in}{1.715718in}}%
\pgfpathlineto{\pgfqpoint{3.112095in}{1.692347in}}%
\pgfpathlineto{\pgfqpoint{3.114905in}{1.704433in}}%
\pgfpathlineto{\pgfqpoint{3.117716in}{1.704377in}}%
\pgfpathlineto{\pgfqpoint{3.120527in}{1.714188in}}%
\pgfpathlineto{\pgfqpoint{3.123337in}{1.715081in}}%
\pgfpathlineto{\pgfqpoint{3.126148in}{1.702676in}}%
\pgfpathlineto{\pgfqpoint{3.131769in}{1.712334in}}%
\pgfpathlineto{\pgfqpoint{3.134580in}{1.712097in}}%
\pgfpathlineto{\pgfqpoint{3.137391in}{1.711251in}}%
\pgfpathlineto{\pgfqpoint{3.140202in}{1.721375in}}%
\pgfpathlineto{\pgfqpoint{3.143012in}{1.696248in}}%
\pgfpathlineto{\pgfqpoint{3.145823in}{1.713401in}}%
\pgfpathlineto{\pgfqpoint{3.148634in}{1.720161in}}%
\pgfpathlineto{\pgfqpoint{3.151444in}{1.717893in}}%
\pgfpathlineto{\pgfqpoint{3.154255in}{1.734371in}}%
\pgfpathlineto{\pgfqpoint{3.157066in}{1.721270in}}%
\pgfpathlineto{\pgfqpoint{3.159876in}{1.741717in}}%
\pgfpathlineto{\pgfqpoint{3.162687in}{1.734534in}}%
\pgfpathlineto{\pgfqpoint{3.165498in}{1.711694in}}%
\pgfpathlineto{\pgfqpoint{3.168308in}{1.729316in}}%
\pgfpathlineto{\pgfqpoint{3.171119in}{1.710103in}}%
\pgfpathlineto{\pgfqpoint{3.173930in}{1.729198in}}%
\pgfpathlineto{\pgfqpoint{3.176740in}{1.712352in}}%
\pgfpathlineto{\pgfqpoint{3.179551in}{1.715316in}}%
\pgfpathlineto{\pgfqpoint{3.182362in}{1.706540in}}%
\pgfpathlineto{\pgfqpoint{3.185173in}{1.715120in}}%
\pgfpathlineto{\pgfqpoint{3.187983in}{1.715212in}}%
\pgfpathlineto{\pgfqpoint{3.190794in}{1.736766in}}%
\pgfpathlineto{\pgfqpoint{3.199226in}{1.720558in}}%
\pgfpathlineto{\pgfqpoint{3.202037in}{1.722438in}}%
\pgfpathlineto{\pgfqpoint{3.204847in}{1.723266in}}%
\pgfpathlineto{\pgfqpoint{3.207658in}{1.718541in}}%
\pgfpathlineto{\pgfqpoint{3.210469in}{1.718719in}}%
\pgfpathlineto{\pgfqpoint{3.213279in}{1.708253in}}%
\pgfpathlineto{\pgfqpoint{3.216090in}{1.700319in}}%
\pgfpathlineto{\pgfqpoint{3.218901in}{1.706792in}}%
\pgfpathlineto{\pgfqpoint{3.221711in}{1.711545in}}%
\pgfpathlineto{\pgfqpoint{3.224522in}{1.723977in}}%
\pgfpathlineto{\pgfqpoint{3.227333in}{1.722068in}}%
\pgfpathlineto{\pgfqpoint{3.230143in}{1.717397in}}%
\pgfpathlineto{\pgfqpoint{3.232954in}{1.717079in}}%
\pgfpathlineto{\pgfqpoint{3.235765in}{1.733546in}}%
\pgfpathlineto{\pgfqpoint{3.238576in}{1.723655in}}%
\pgfpathlineto{\pgfqpoint{3.241386in}{1.710863in}}%
\pgfpathlineto{\pgfqpoint{3.244197in}{1.673987in}}%
\pgfpathlineto{\pgfqpoint{3.247008in}{1.692037in}}%
\pgfpathlineto{\pgfqpoint{3.252629in}{1.713162in}}%
\pgfpathlineto{\pgfqpoint{3.255440in}{1.718812in}}%
\pgfpathlineto{\pgfqpoint{3.258250in}{1.709390in}}%
\pgfpathlineto{\pgfqpoint{3.261061in}{1.730794in}}%
\pgfpathlineto{\pgfqpoint{3.263872in}{1.729690in}}%
\pgfpathlineto{\pgfqpoint{3.266682in}{1.727595in}}%
\pgfpathlineto{\pgfqpoint{3.269493in}{1.717992in}}%
\pgfpathlineto{\pgfqpoint{3.272304in}{1.721293in}}%
\pgfpathlineto{\pgfqpoint{3.275114in}{1.705840in}}%
\pgfpathlineto{\pgfqpoint{3.277925in}{1.669476in}}%
\pgfpathlineto{\pgfqpoint{3.280736in}{1.689462in}}%
\pgfpathlineto{\pgfqpoint{3.283547in}{1.702072in}}%
\pgfpathlineto{\pgfqpoint{3.286357in}{1.705381in}}%
\pgfpathlineto{\pgfqpoint{3.289168in}{1.690097in}}%
\pgfpathlineto{\pgfqpoint{3.291979in}{1.704365in}}%
\pgfpathlineto{\pgfqpoint{3.294789in}{1.702188in}}%
\pgfpathlineto{\pgfqpoint{3.297600in}{1.704447in}}%
\pgfpathlineto{\pgfqpoint{3.300411in}{1.713524in}}%
\pgfpathlineto{\pgfqpoint{3.303221in}{1.720085in}}%
\pgfpathlineto{\pgfqpoint{3.306032in}{1.715410in}}%
\pgfpathlineto{\pgfqpoint{3.308843in}{1.730700in}}%
\pgfpathlineto{\pgfqpoint{3.311653in}{1.712706in}}%
\pgfpathlineto{\pgfqpoint{3.314464in}{1.712916in}}%
\pgfpathlineto{\pgfqpoint{3.317275in}{1.726489in}}%
\pgfpathlineto{\pgfqpoint{3.320085in}{1.716884in}}%
\pgfpathlineto{\pgfqpoint{3.325707in}{1.718390in}}%
\pgfpathlineto{\pgfqpoint{3.328518in}{1.728109in}}%
\pgfpathlineto{\pgfqpoint{3.331328in}{1.721805in}}%
\pgfpathlineto{\pgfqpoint{3.334139in}{1.729685in}}%
\pgfpathlineto{\pgfqpoint{3.336950in}{1.721084in}}%
\pgfpathlineto{\pgfqpoint{3.339760in}{1.695374in}}%
\pgfpathlineto{\pgfqpoint{3.342571in}{1.701694in}}%
\pgfpathlineto{\pgfqpoint{3.345382in}{1.710593in}}%
\pgfpathlineto{\pgfqpoint{3.348192in}{1.723145in}}%
\pgfpathlineto{\pgfqpoint{3.351003in}{1.714462in}}%
\pgfpathlineto{\pgfqpoint{3.353814in}{1.718591in}}%
\pgfpathlineto{\pgfqpoint{3.356624in}{1.706687in}}%
\pgfpathlineto{\pgfqpoint{3.359435in}{1.721505in}}%
\pgfpathlineto{\pgfqpoint{3.362246in}{1.713644in}}%
\pgfpathlineto{\pgfqpoint{3.365056in}{1.709422in}}%
\pgfpathlineto{\pgfqpoint{3.367867in}{1.719641in}}%
\pgfpathlineto{\pgfqpoint{3.370678in}{1.721939in}}%
\pgfpathlineto{\pgfqpoint{3.373489in}{1.717823in}}%
\pgfpathlineto{\pgfqpoint{3.376299in}{1.747238in}}%
\pgfpathlineto{\pgfqpoint{3.381921in}{1.714629in}}%
\pgfpathlineto{\pgfqpoint{3.384731in}{1.708983in}}%
\pgfpathlineto{\pgfqpoint{3.387542in}{1.712769in}}%
\pgfpathlineto{\pgfqpoint{3.390353in}{1.718765in}}%
\pgfpathlineto{\pgfqpoint{3.393163in}{1.733233in}}%
\pgfpathlineto{\pgfqpoint{3.395974in}{1.722202in}}%
\pgfpathlineto{\pgfqpoint{3.398785in}{1.699111in}}%
\pgfpathlineto{\pgfqpoint{3.401595in}{1.688473in}}%
\pgfpathlineto{\pgfqpoint{3.404406in}{1.688856in}}%
\pgfpathlineto{\pgfqpoint{3.407217in}{1.698163in}}%
\pgfpathlineto{\pgfqpoint{3.410027in}{1.695894in}}%
\pgfpathlineto{\pgfqpoint{3.412838in}{1.697726in}}%
\pgfpathlineto{\pgfqpoint{3.415649in}{1.679017in}}%
\pgfpathlineto{\pgfqpoint{3.418459in}{1.699709in}}%
\pgfpathlineto{\pgfqpoint{3.424081in}{1.711625in}}%
\pgfpathlineto{\pgfqpoint{3.426892in}{1.671408in}}%
\pgfpathlineto{\pgfqpoint{3.429702in}{1.699449in}}%
\pgfpathlineto{\pgfqpoint{3.432513in}{1.693255in}}%
\pgfpathlineto{\pgfqpoint{3.435324in}{1.684466in}}%
\pgfpathlineto{\pgfqpoint{3.438134in}{1.691316in}}%
\pgfpathlineto{\pgfqpoint{3.440945in}{1.714004in}}%
\pgfpathlineto{\pgfqpoint{3.443756in}{1.709156in}}%
\pgfpathlineto{\pgfqpoint{3.446566in}{1.718489in}}%
\pgfpathlineto{\pgfqpoint{3.449377in}{1.721628in}}%
\pgfpathlineto{\pgfqpoint{3.452188in}{1.738809in}}%
\pgfpathlineto{\pgfqpoint{3.454998in}{1.722267in}}%
\pgfpathlineto{\pgfqpoint{3.457809in}{1.717651in}}%
\pgfpathlineto{\pgfqpoint{3.460620in}{1.731186in}}%
\pgfpathlineto{\pgfqpoint{3.463430in}{1.721500in}}%
\pgfpathlineto{\pgfqpoint{3.466241in}{1.714008in}}%
\pgfpathlineto{\pgfqpoint{3.469052in}{1.710634in}}%
\pgfpathlineto{\pgfqpoint{3.471863in}{1.709884in}}%
\pgfpathlineto{\pgfqpoint{3.474673in}{1.711113in}}%
\pgfpathlineto{\pgfqpoint{3.477484in}{1.713767in}}%
\pgfpathlineto{\pgfqpoint{3.480295in}{1.720643in}}%
\pgfpathlineto{\pgfqpoint{3.483105in}{1.757090in}}%
\pgfpathlineto{\pgfqpoint{3.485916in}{1.773628in}}%
\pgfpathlineto{\pgfqpoint{3.488727in}{1.764263in}}%
\pgfpathlineto{\pgfqpoint{3.491537in}{1.711679in}}%
\pgfpathlineto{\pgfqpoint{3.494348in}{1.697977in}}%
\pgfpathlineto{\pgfqpoint{3.497159in}{1.701584in}}%
\pgfpathlineto{\pgfqpoint{3.499969in}{1.717886in}}%
\pgfpathlineto{\pgfqpoint{3.502780in}{1.746746in}}%
\pgfpathlineto{\pgfqpoint{3.508401in}{1.715814in}}%
\pgfpathlineto{\pgfqpoint{3.511212in}{1.731447in}}%
\pgfpathlineto{\pgfqpoint{3.514023in}{1.708807in}}%
\pgfpathlineto{\pgfqpoint{3.516834in}{1.722894in}}%
\pgfpathlineto{\pgfqpoint{3.519644in}{1.711928in}}%
\pgfpathlineto{\pgfqpoint{3.522455in}{1.705706in}}%
\pgfpathlineto{\pgfqpoint{3.525266in}{1.718461in}}%
\pgfpathlineto{\pgfqpoint{3.528076in}{1.710501in}}%
\pgfpathlineto{\pgfqpoint{3.530887in}{1.705295in}}%
\pgfpathlineto{\pgfqpoint{3.533698in}{1.708425in}}%
\pgfpathlineto{\pgfqpoint{3.536508in}{1.726238in}}%
\pgfpathlineto{\pgfqpoint{3.539319in}{1.709289in}}%
\pgfpathlineto{\pgfqpoint{3.542130in}{1.718317in}}%
\pgfpathlineto{\pgfqpoint{3.544940in}{1.707516in}}%
\pgfpathlineto{\pgfqpoint{3.547751in}{1.719562in}}%
\pgfpathlineto{\pgfqpoint{3.550562in}{1.711770in}}%
\pgfpathlineto{\pgfqpoint{3.553372in}{1.763061in}}%
\pgfpathlineto{\pgfqpoint{3.556183in}{1.729334in}}%
\pgfpathlineto{\pgfqpoint{3.558994in}{1.707995in}}%
\pgfpathlineto{\pgfqpoint{3.561805in}{1.705045in}}%
\pgfpathlineto{\pgfqpoint{3.564615in}{1.697852in}}%
\pgfpathlineto{\pgfqpoint{3.567426in}{1.680583in}}%
\pgfpathlineto{\pgfqpoint{3.570237in}{1.692522in}}%
\pgfpathlineto{\pgfqpoint{3.573047in}{1.683175in}}%
\pgfpathlineto{\pgfqpoint{3.575858in}{1.683338in}}%
\pgfpathlineto{\pgfqpoint{3.578669in}{1.691401in}}%
\pgfpathlineto{\pgfqpoint{3.581479in}{1.685296in}}%
\pgfpathlineto{\pgfqpoint{3.584290in}{1.693963in}}%
\pgfpathlineto{\pgfqpoint{3.587101in}{1.710465in}}%
\pgfpathlineto{\pgfqpoint{3.589911in}{1.697044in}}%
\pgfpathlineto{\pgfqpoint{3.592722in}{1.693945in}}%
\pgfpathlineto{\pgfqpoint{3.595533in}{1.687832in}}%
\pgfpathlineto{\pgfqpoint{3.598343in}{1.705278in}}%
\pgfpathlineto{\pgfqpoint{3.601154in}{1.718942in}}%
\pgfpathlineto{\pgfqpoint{3.603965in}{1.705463in}}%
\pgfpathlineto{\pgfqpoint{3.606776in}{1.700028in}}%
\pgfpathlineto{\pgfqpoint{3.609586in}{1.690484in}}%
\pgfpathlineto{\pgfqpoint{3.612397in}{1.706625in}}%
\pgfpathlineto{\pgfqpoint{3.615208in}{1.692098in}}%
\pgfpathlineto{\pgfqpoint{3.618018in}{1.704066in}}%
\pgfpathlineto{\pgfqpoint{3.620829in}{1.719689in}}%
\pgfpathlineto{\pgfqpoint{3.623640in}{1.748500in}}%
\pgfpathlineto{\pgfqpoint{3.626450in}{1.703374in}}%
\pgfpathlineto{\pgfqpoint{3.629261in}{1.701089in}}%
\pgfpathlineto{\pgfqpoint{3.632072in}{1.693510in}}%
\pgfpathlineto{\pgfqpoint{3.637693in}{1.719549in}}%
\pgfpathlineto{\pgfqpoint{3.640504in}{1.696982in}}%
\pgfpathlineto{\pgfqpoint{3.643314in}{1.706982in}}%
\pgfpathlineto{\pgfqpoint{3.646125in}{1.721881in}}%
\pgfpathlineto{\pgfqpoint{3.648936in}{1.722806in}}%
\pgfpathlineto{\pgfqpoint{3.651746in}{1.709819in}}%
\pgfpathlineto{\pgfqpoint{3.654557in}{1.712767in}}%
\pgfpathlineto{\pgfqpoint{3.657368in}{1.691591in}}%
\pgfpathlineto{\pgfqpoint{3.660179in}{1.703351in}}%
\pgfpathlineto{\pgfqpoint{3.665800in}{1.710865in}}%
\pgfpathlineto{\pgfqpoint{3.668611in}{1.716912in}}%
\pgfpathlineto{\pgfqpoint{3.671421in}{1.719827in}}%
\pgfpathlineto{\pgfqpoint{3.674232in}{1.714946in}}%
\pgfpathlineto{\pgfqpoint{3.677043in}{1.726796in}}%
\pgfpathlineto{\pgfqpoint{3.679853in}{1.711098in}}%
\pgfpathlineto{\pgfqpoint{3.682664in}{1.724770in}}%
\pgfpathlineto{\pgfqpoint{3.685475in}{1.732465in}}%
\pgfpathlineto{\pgfqpoint{3.688285in}{1.697835in}}%
\pgfpathlineto{\pgfqpoint{3.691096in}{1.715452in}}%
\pgfpathlineto{\pgfqpoint{3.693907in}{1.722555in}}%
\pgfpathlineto{\pgfqpoint{3.696717in}{1.737355in}}%
\pgfpathlineto{\pgfqpoint{3.699528in}{1.727188in}}%
\pgfpathlineto{\pgfqpoint{3.702339in}{1.752003in}}%
\pgfpathlineto{\pgfqpoint{3.705150in}{1.729614in}}%
\pgfpathlineto{\pgfqpoint{3.707960in}{1.700232in}}%
\pgfpathlineto{\pgfqpoint{3.710771in}{1.688254in}}%
\pgfpathlineto{\pgfqpoint{3.713582in}{1.712976in}}%
\pgfpathlineto{\pgfqpoint{3.716392in}{1.748355in}}%
\pgfpathlineto{\pgfqpoint{3.719203in}{1.725748in}}%
\pgfpathlineto{\pgfqpoint{3.722014in}{1.719518in}}%
\pgfpathlineto{\pgfqpoint{3.724824in}{1.700979in}}%
\pgfpathlineto{\pgfqpoint{3.727635in}{1.709823in}}%
\pgfpathlineto{\pgfqpoint{3.730446in}{1.707687in}}%
\pgfpathlineto{\pgfqpoint{3.733256in}{1.700252in}}%
\pgfpathlineto{\pgfqpoint{3.736067in}{1.717675in}}%
\pgfpathlineto{\pgfqpoint{3.738878in}{1.728724in}}%
\pgfpathlineto{\pgfqpoint{3.741688in}{1.751615in}}%
\pgfpathlineto{\pgfqpoint{3.744499in}{1.732249in}}%
\pgfpathlineto{\pgfqpoint{3.747310in}{1.741442in}}%
\pgfpathlineto{\pgfqpoint{3.750121in}{1.754110in}}%
\pgfpathlineto{\pgfqpoint{3.752931in}{1.753791in}}%
\pgfpathlineto{\pgfqpoint{3.755742in}{1.726692in}}%
\pgfpathlineto{\pgfqpoint{3.758553in}{1.711234in}}%
\pgfpathlineto{\pgfqpoint{3.761363in}{1.735788in}}%
\pgfpathlineto{\pgfqpoint{3.764174in}{1.718425in}}%
\pgfpathlineto{\pgfqpoint{3.766985in}{1.746425in}}%
\pgfpathlineto{\pgfqpoint{3.769795in}{1.741610in}}%
\pgfpathlineto{\pgfqpoint{3.772606in}{1.742251in}}%
\pgfpathlineto{\pgfqpoint{3.778227in}{1.700723in}}%
\pgfpathlineto{\pgfqpoint{3.781038in}{1.719025in}}%
\pgfpathlineto{\pgfqpoint{3.783849in}{1.714998in}}%
\pgfpathlineto{\pgfqpoint{3.786659in}{1.729172in}}%
\pgfpathlineto{\pgfqpoint{3.789470in}{1.747657in}}%
\pgfpathlineto{\pgfqpoint{3.792281in}{1.652055in}}%
\pgfpathlineto{\pgfqpoint{3.795092in}{1.674853in}}%
\pgfpathlineto{\pgfqpoint{3.797902in}{1.703927in}}%
\pgfpathlineto{\pgfqpoint{3.800713in}{1.691096in}}%
\pgfpathlineto{\pgfqpoint{3.803524in}{1.710443in}}%
\pgfpathlineto{\pgfqpoint{3.806334in}{1.747407in}}%
\pgfpathlineto{\pgfqpoint{3.809145in}{1.793572in}}%
\pgfpathlineto{\pgfqpoint{3.811956in}{1.759668in}}%
\pgfpathlineto{\pgfqpoint{3.814766in}{1.714407in}}%
\pgfpathlineto{\pgfqpoint{3.817577in}{1.741032in}}%
\pgfpathlineto{\pgfqpoint{3.820388in}{1.699369in}}%
\pgfpathlineto{\pgfqpoint{3.823198in}{1.692456in}}%
\pgfpathlineto{\pgfqpoint{3.826009in}{1.681713in}}%
\pgfpathlineto{\pgfqpoint{3.828820in}{1.704150in}}%
\pgfpathlineto{\pgfqpoint{3.831630in}{1.702171in}}%
\pgfpathlineto{\pgfqpoint{3.834441in}{1.680481in}}%
\pgfpathlineto{\pgfqpoint{3.837252in}{1.705444in}}%
\pgfpathlineto{\pgfqpoint{3.840062in}{1.712986in}}%
\pgfpathlineto{\pgfqpoint{3.842873in}{1.689574in}}%
\pgfpathlineto{\pgfqpoint{3.845684in}{1.705973in}}%
\pgfpathlineto{\pgfqpoint{3.848495in}{1.713942in}}%
\pgfpathlineto{\pgfqpoint{3.851305in}{1.676448in}}%
\pgfpathlineto{\pgfqpoint{3.854116in}{1.693697in}}%
\pgfpathlineto{\pgfqpoint{3.856927in}{1.706442in}}%
\pgfpathlineto{\pgfqpoint{3.859737in}{1.708915in}}%
\pgfpathlineto{\pgfqpoint{3.862548in}{1.743317in}}%
\pgfpathlineto{\pgfqpoint{3.865359in}{1.755402in}}%
\pgfpathlineto{\pgfqpoint{3.868169in}{1.741511in}}%
\pgfpathlineto{\pgfqpoint{3.870980in}{1.740291in}}%
\pgfpathlineto{\pgfqpoint{3.873791in}{1.709076in}}%
\pgfpathlineto{\pgfqpoint{3.876601in}{1.712588in}}%
\pgfpathlineto{\pgfqpoint{3.885033in}{1.691962in}}%
\pgfpathlineto{\pgfqpoint{3.887844in}{1.688082in}}%
\pgfpathlineto{\pgfqpoint{3.890655in}{1.695690in}}%
\pgfpathlineto{\pgfqpoint{3.893466in}{1.713784in}}%
\pgfpathlineto{\pgfqpoint{3.896276in}{1.698475in}}%
\pgfpathlineto{\pgfqpoint{3.899087in}{1.699343in}}%
\pgfpathlineto{\pgfqpoint{3.901898in}{1.704513in}}%
\pgfpathlineto{\pgfqpoint{3.904708in}{1.686749in}}%
\pgfpathlineto{\pgfqpoint{3.907519in}{1.673567in}}%
\pgfpathlineto{\pgfqpoint{3.910330in}{1.690110in}}%
\pgfpathlineto{\pgfqpoint{3.913140in}{1.680695in}}%
\pgfpathlineto{\pgfqpoint{3.915951in}{1.687307in}}%
\pgfpathlineto{\pgfqpoint{3.918762in}{1.707951in}}%
\pgfpathlineto{\pgfqpoint{3.921572in}{1.691473in}}%
\pgfpathlineto{\pgfqpoint{3.924383in}{1.707642in}}%
\pgfpathlineto{\pgfqpoint{3.927194in}{1.702879in}}%
\pgfpathlineto{\pgfqpoint{3.930004in}{1.709527in}}%
\pgfpathlineto{\pgfqpoint{3.932815in}{1.700356in}}%
\pgfpathlineto{\pgfqpoint{3.935626in}{1.687054in}}%
\pgfpathlineto{\pgfqpoint{3.938437in}{1.686570in}}%
\pgfpathlineto{\pgfqpoint{3.941247in}{1.698637in}}%
\pgfpathlineto{\pgfqpoint{3.944058in}{1.683909in}}%
\pgfpathlineto{\pgfqpoint{3.946869in}{1.703452in}}%
\pgfpathlineto{\pgfqpoint{3.949679in}{1.701168in}}%
\pgfpathlineto{\pgfqpoint{3.952490in}{1.710477in}}%
\pgfpathlineto{\pgfqpoint{3.955301in}{1.736033in}}%
\pgfpathlineto{\pgfqpoint{3.958111in}{1.743099in}}%
\pgfpathlineto{\pgfqpoint{3.960922in}{1.728850in}}%
\pgfpathlineto{\pgfqpoint{3.963733in}{1.721204in}}%
\pgfpathlineto{\pgfqpoint{3.966543in}{1.734646in}}%
\pgfpathlineto{\pgfqpoint{3.969354in}{1.734687in}}%
\pgfpathlineto{\pgfqpoint{3.972165in}{1.710765in}}%
\pgfpathlineto{\pgfqpoint{3.974975in}{1.730233in}}%
\pgfpathlineto{\pgfqpoint{3.977786in}{1.726928in}}%
\pgfpathlineto{\pgfqpoint{3.983408in}{1.711629in}}%
\pgfpathlineto{\pgfqpoint{3.986218in}{1.705047in}}%
\pgfpathlineto{\pgfqpoint{3.989029in}{1.693473in}}%
\pgfpathlineto{\pgfqpoint{3.991840in}{1.725663in}}%
\pgfpathlineto{\pgfqpoint{3.994650in}{1.713073in}}%
\pgfpathlineto{\pgfqpoint{3.997461in}{1.729311in}}%
\pgfpathlineto{\pgfqpoint{4.000272in}{1.709740in}}%
\pgfpathlineto{\pgfqpoint{4.003082in}{1.726351in}}%
\pgfpathlineto{\pgfqpoint{4.005893in}{1.712480in}}%
\pgfpathlineto{\pgfqpoint{4.008704in}{1.721250in}}%
\pgfpathlineto{\pgfqpoint{4.011514in}{1.706725in}}%
\pgfpathlineto{\pgfqpoint{4.014325in}{1.715893in}}%
\pgfpathlineto{\pgfqpoint{4.017136in}{1.682061in}}%
\pgfpathlineto{\pgfqpoint{4.019946in}{1.696694in}}%
\pgfpathlineto{\pgfqpoint{4.022757in}{1.702957in}}%
\pgfpathlineto{\pgfqpoint{4.025568in}{1.696294in}}%
\pgfpathlineto{\pgfqpoint{4.028378in}{1.712778in}}%
\pgfpathlineto{\pgfqpoint{4.031189in}{1.708726in}}%
\pgfpathlineto{\pgfqpoint{4.034000in}{1.699312in}}%
\pgfpathlineto{\pgfqpoint{4.036811in}{1.704036in}}%
\pgfpathlineto{\pgfqpoint{4.039621in}{1.698067in}}%
\pgfpathlineto{\pgfqpoint{4.042432in}{1.702144in}}%
\pgfpathlineto{\pgfqpoint{4.045243in}{1.690419in}}%
\pgfpathlineto{\pgfqpoint{4.048053in}{1.703614in}}%
\pgfpathlineto{\pgfqpoint{4.050864in}{1.722884in}}%
\pgfpathlineto{\pgfqpoint{4.053675in}{1.746593in}}%
\pgfpathlineto{\pgfqpoint{4.056485in}{1.733821in}}%
\pgfpathlineto{\pgfqpoint{4.059296in}{1.733489in}}%
\pgfpathlineto{\pgfqpoint{4.062107in}{1.724575in}}%
\pgfpathlineto{\pgfqpoint{4.064917in}{1.741329in}}%
\pgfpathlineto{\pgfqpoint{4.067728in}{1.727295in}}%
\pgfpathlineto{\pgfqpoint{4.070539in}{1.722969in}}%
\pgfpathlineto{\pgfqpoint{4.073349in}{1.731736in}}%
\pgfpathlineto{\pgfqpoint{4.076160in}{1.700291in}}%
\pgfpathlineto{\pgfqpoint{4.078971in}{1.754188in}}%
\pgfpathlineto{\pgfqpoint{4.081782in}{1.766146in}}%
\pgfpathlineto{\pgfqpoint{4.084592in}{1.720139in}}%
\pgfpathlineto{\pgfqpoint{4.087403in}{1.693378in}}%
\pgfpathlineto{\pgfqpoint{4.090214in}{1.741132in}}%
\pgfpathlineto{\pgfqpoint{4.093024in}{1.726181in}}%
\pgfpathlineto{\pgfqpoint{4.095835in}{1.718187in}}%
\pgfpathlineto{\pgfqpoint{4.098646in}{1.726430in}}%
\pgfpathlineto{\pgfqpoint{4.101456in}{1.713945in}}%
\pgfpathlineto{\pgfqpoint{4.104267in}{1.681665in}}%
\pgfpathlineto{\pgfqpoint{4.107078in}{1.690301in}}%
\pgfpathlineto{\pgfqpoint{4.109888in}{1.682367in}}%
\pgfpathlineto{\pgfqpoint{4.112699in}{1.692573in}}%
\pgfpathlineto{\pgfqpoint{4.115510in}{1.688542in}}%
\pgfpathlineto{\pgfqpoint{4.121131in}{1.700113in}}%
\pgfpathlineto{\pgfqpoint{4.123942in}{1.697922in}}%
\pgfpathlineto{\pgfqpoint{4.126753in}{1.693031in}}%
\pgfpathlineto{\pgfqpoint{4.129563in}{1.708471in}}%
\pgfpathlineto{\pgfqpoint{4.132374in}{1.692538in}}%
\pgfpathlineto{\pgfqpoint{4.135185in}{1.714687in}}%
\pgfpathlineto{\pgfqpoint{4.137995in}{1.721167in}}%
\pgfpathlineto{\pgfqpoint{4.140806in}{1.717708in}}%
\pgfpathlineto{\pgfqpoint{4.143617in}{1.705533in}}%
\pgfpathlineto{\pgfqpoint{4.146427in}{1.726250in}}%
\pgfpathlineto{\pgfqpoint{4.149238in}{1.718240in}}%
\pgfpathlineto{\pgfqpoint{4.152049in}{1.716035in}}%
\pgfpathlineto{\pgfqpoint{4.154859in}{1.708044in}}%
\pgfpathlineto{\pgfqpoint{4.157670in}{1.697481in}}%
\pgfpathlineto{\pgfqpoint{4.160481in}{1.692247in}}%
\pgfpathlineto{\pgfqpoint{4.163291in}{1.702165in}}%
\pgfpathlineto{\pgfqpoint{4.166102in}{1.703993in}}%
\pgfpathlineto{\pgfqpoint{4.168913in}{1.712882in}}%
\pgfpathlineto{\pgfqpoint{4.171724in}{1.705996in}}%
\pgfpathlineto{\pgfqpoint{4.174534in}{1.709899in}}%
\pgfpathlineto{\pgfqpoint{4.177345in}{1.697049in}}%
\pgfpathlineto{\pgfqpoint{4.180156in}{1.703962in}}%
\pgfpathlineto{\pgfqpoint{4.182966in}{1.704287in}}%
\pgfpathlineto{\pgfqpoint{4.185777in}{1.716092in}}%
\pgfpathlineto{\pgfqpoint{4.188588in}{1.711846in}}%
\pgfpathlineto{\pgfqpoint{4.191398in}{1.703164in}}%
\pgfpathlineto{\pgfqpoint{4.194209in}{1.710412in}}%
\pgfpathlineto{\pgfqpoint{4.197020in}{1.719216in}}%
\pgfpathlineto{\pgfqpoint{4.199830in}{1.716610in}}%
\pgfpathlineto{\pgfqpoint{4.205452in}{1.707212in}}%
\pgfpathlineto{\pgfqpoint{4.208262in}{1.704523in}}%
\pgfpathlineto{\pgfqpoint{4.211073in}{1.711438in}}%
\pgfpathlineto{\pgfqpoint{4.213884in}{1.705751in}}%
\pgfpathlineto{\pgfqpoint{4.216695in}{1.697473in}}%
\pgfpathlineto{\pgfqpoint{4.219505in}{1.694588in}}%
\pgfpathlineto{\pgfqpoint{4.222316in}{1.690026in}}%
\pgfpathlineto{\pgfqpoint{4.225127in}{1.701862in}}%
\pgfpathlineto{\pgfqpoint{4.227937in}{1.730384in}}%
\pgfpathlineto{\pgfqpoint{4.230748in}{1.699607in}}%
\pgfpathlineto{\pgfqpoint{4.233559in}{1.720753in}}%
\pgfpathlineto{\pgfqpoint{4.236369in}{1.724493in}}%
\pgfpathlineto{\pgfqpoint{4.239180in}{1.712503in}}%
\pgfpathlineto{\pgfqpoint{4.241991in}{1.711706in}}%
\pgfpathlineto{\pgfqpoint{4.244801in}{1.705322in}}%
\pgfpathlineto{\pgfqpoint{4.247612in}{1.708374in}}%
\pgfpathlineto{\pgfqpoint{4.250423in}{1.698445in}}%
\pgfpathlineto{\pgfqpoint{4.253233in}{1.701167in}}%
\pgfpathlineto{\pgfqpoint{4.256044in}{1.717908in}}%
\pgfpathlineto{\pgfqpoint{4.258855in}{1.727468in}}%
\pgfpathlineto{\pgfqpoint{4.261665in}{1.717218in}}%
\pgfpathlineto{\pgfqpoint{4.264476in}{1.703324in}}%
\pgfpathlineto{\pgfqpoint{4.267287in}{1.722717in}}%
\pgfpathlineto{\pgfqpoint{4.270098in}{1.707306in}}%
\pgfpathlineto{\pgfqpoint{4.272908in}{1.706656in}}%
\pgfpathlineto{\pgfqpoint{4.275719in}{1.710827in}}%
\pgfpathlineto{\pgfqpoint{4.278530in}{1.701876in}}%
\pgfpathlineto{\pgfqpoint{4.281340in}{1.705446in}}%
\pgfpathlineto{\pgfqpoint{4.284151in}{1.715411in}}%
\pgfpathlineto{\pgfqpoint{4.286962in}{1.710929in}}%
\pgfpathlineto{\pgfqpoint{4.289772in}{1.728470in}}%
\pgfpathlineto{\pgfqpoint{4.292583in}{1.719834in}}%
\pgfpathlineto{\pgfqpoint{4.295394in}{1.723347in}}%
\pgfpathlineto{\pgfqpoint{4.298204in}{1.710662in}}%
\pgfpathlineto{\pgfqpoint{4.301015in}{1.716129in}}%
\pgfpathlineto{\pgfqpoint{4.303826in}{1.723925in}}%
\pgfpathlineto{\pgfqpoint{4.306636in}{1.700709in}}%
\pgfpathlineto{\pgfqpoint{4.309447in}{1.709698in}}%
\pgfpathlineto{\pgfqpoint{4.312258in}{1.712946in}}%
\pgfpathlineto{\pgfqpoint{4.315069in}{1.700034in}}%
\pgfpathlineto{\pgfqpoint{4.317879in}{1.722364in}}%
\pgfpathlineto{\pgfqpoint{4.320690in}{1.723395in}}%
\pgfpathlineto{\pgfqpoint{4.326311in}{1.710739in}}%
\pgfpathlineto{\pgfqpoint{4.329122in}{1.706828in}}%
\pgfpathlineto{\pgfqpoint{4.334743in}{1.735879in}}%
\pgfpathlineto{\pgfqpoint{4.337554in}{1.726655in}}%
\pgfpathlineto{\pgfqpoint{4.340365in}{1.726242in}}%
\pgfpathlineto{\pgfqpoint{4.343175in}{1.687466in}}%
\pgfpathlineto{\pgfqpoint{4.348797in}{1.692001in}}%
\pgfpathlineto{\pgfqpoint{4.351607in}{1.720365in}}%
\pgfpathlineto{\pgfqpoint{4.354418in}{1.717529in}}%
\pgfpathlineto{\pgfqpoint{4.357229in}{1.771748in}}%
\pgfpathlineto{\pgfqpoint{4.360040in}{1.748769in}}%
\pgfpathlineto{\pgfqpoint{4.362850in}{1.709817in}}%
\pgfpathlineto{\pgfqpoint{4.365661in}{1.695031in}}%
\pgfpathlineto{\pgfqpoint{4.368472in}{1.704406in}}%
\pgfpathlineto{\pgfqpoint{4.371282in}{1.693195in}}%
\pgfpathlineto{\pgfqpoint{4.374093in}{1.727731in}}%
\pgfpathlineto{\pgfqpoint{4.376904in}{1.728414in}}%
\pgfpathlineto{\pgfqpoint{4.379714in}{1.713978in}}%
\pgfpathlineto{\pgfqpoint{4.382525in}{1.730477in}}%
\pgfpathlineto{\pgfqpoint{4.385336in}{1.723212in}}%
\pgfpathlineto{\pgfqpoint{4.388146in}{1.749581in}}%
\pgfpathlineto{\pgfqpoint{4.390957in}{1.767106in}}%
\pgfpathlineto{\pgfqpoint{4.393768in}{1.743643in}}%
\pgfpathlineto{\pgfqpoint{4.396578in}{1.707635in}}%
\pgfpathlineto{\pgfqpoint{4.399389in}{1.711981in}}%
\pgfpathlineto{\pgfqpoint{4.402200in}{1.676865in}}%
\pgfpathlineto{\pgfqpoint{4.410632in}{1.712347in}}%
\pgfpathlineto{\pgfqpoint{4.413443in}{1.700991in}}%
\pgfpathlineto{\pgfqpoint{4.416253in}{1.705310in}}%
\pgfpathlineto{\pgfqpoint{4.419064in}{1.701442in}}%
\pgfpathlineto{\pgfqpoint{4.421875in}{1.723129in}}%
\pgfpathlineto{\pgfqpoint{4.424685in}{1.723530in}}%
\pgfpathlineto{\pgfqpoint{4.427496in}{1.715437in}}%
\pgfpathlineto{\pgfqpoint{4.433117in}{1.719525in}}%
\pgfpathlineto{\pgfqpoint{4.435928in}{1.712976in}}%
\pgfpathlineto{\pgfqpoint{4.438739in}{1.709163in}}%
\pgfpathlineto{\pgfqpoint{4.441549in}{1.710656in}}%
\pgfpathlineto{\pgfqpoint{4.444360in}{1.710361in}}%
\pgfpathlineto{\pgfqpoint{4.447171in}{1.715581in}}%
\pgfpathlineto{\pgfqpoint{4.449981in}{1.690471in}}%
\pgfpathlineto{\pgfqpoint{4.452792in}{1.686829in}}%
\pgfpathlineto{\pgfqpoint{4.458414in}{1.673182in}}%
\pgfpathlineto{\pgfqpoint{4.461224in}{1.692064in}}%
\pgfpathlineto{\pgfqpoint{4.464035in}{1.704790in}}%
\pgfpathlineto{\pgfqpoint{4.466846in}{1.699554in}}%
\pgfpathlineto{\pgfqpoint{4.469656in}{1.709848in}}%
\pgfpathlineto{\pgfqpoint{4.472467in}{1.713524in}}%
\pgfpathlineto{\pgfqpoint{4.475278in}{1.711550in}}%
\pgfpathlineto{\pgfqpoint{4.478088in}{1.706616in}}%
\pgfpathlineto{\pgfqpoint{4.480899in}{1.706026in}}%
\pgfpathlineto{\pgfqpoint{4.483710in}{1.706090in}}%
\pgfpathlineto{\pgfqpoint{4.486520in}{1.703206in}}%
\pgfpathlineto{\pgfqpoint{4.489331in}{1.689653in}}%
\pgfpathlineto{\pgfqpoint{4.492142in}{1.686655in}}%
\pgfpathlineto{\pgfqpoint{4.494952in}{1.704275in}}%
\pgfpathlineto{\pgfqpoint{4.497763in}{1.698794in}}%
\pgfpathlineto{\pgfqpoint{4.500574in}{1.703942in}}%
\pgfpathlineto{\pgfqpoint{4.503385in}{1.721138in}}%
\pgfpathlineto{\pgfqpoint{4.506195in}{1.722201in}}%
\pgfpathlineto{\pgfqpoint{4.509006in}{1.721114in}}%
\pgfpathlineto{\pgfqpoint{4.511817in}{1.661995in}}%
\pgfpathlineto{\pgfqpoint{4.514627in}{1.681176in}}%
\pgfpathlineto{\pgfqpoint{4.517438in}{1.691519in}}%
\pgfpathlineto{\pgfqpoint{4.520249in}{1.707918in}}%
\pgfpathlineto{\pgfqpoint{4.523059in}{1.702050in}}%
\pgfpathlineto{\pgfqpoint{4.525870in}{1.700565in}}%
\pgfpathlineto{\pgfqpoint{4.528681in}{1.696302in}}%
\pgfpathlineto{\pgfqpoint{4.531491in}{1.695212in}}%
\pgfpathlineto{\pgfqpoint{4.534302in}{1.688255in}}%
\pgfpathlineto{\pgfqpoint{4.537113in}{1.697674in}}%
\pgfpathlineto{\pgfqpoint{4.539923in}{1.701544in}}%
\pgfpathlineto{\pgfqpoint{4.542734in}{1.698405in}}%
\pgfpathlineto{\pgfqpoint{4.545545in}{1.704543in}}%
\pgfpathlineto{\pgfqpoint{4.548356in}{1.701512in}}%
\pgfpathlineto{\pgfqpoint{4.551166in}{1.701481in}}%
\pgfpathlineto{\pgfqpoint{4.553977in}{1.711024in}}%
\pgfpathlineto{\pgfqpoint{4.556788in}{1.712361in}}%
\pgfpathlineto{\pgfqpoint{4.559598in}{1.697487in}}%
\pgfpathlineto{\pgfqpoint{4.562409in}{1.708715in}}%
\pgfpathlineto{\pgfqpoint{4.565220in}{1.706151in}}%
\pgfpathlineto{\pgfqpoint{4.570841in}{1.706422in}}%
\pgfpathlineto{\pgfqpoint{4.573652in}{1.709796in}}%
\pgfpathlineto{\pgfqpoint{4.576462in}{1.708500in}}%
\pgfpathlineto{\pgfqpoint{4.579273in}{1.701008in}}%
\pgfpathlineto{\pgfqpoint{4.582084in}{1.699362in}}%
\pgfpathlineto{\pgfqpoint{4.584894in}{1.711395in}}%
\pgfpathlineto{\pgfqpoint{4.587705in}{1.706476in}}%
\pgfpathlineto{\pgfqpoint{4.590516in}{1.706691in}}%
\pgfpathlineto{\pgfqpoint{4.593327in}{1.705845in}}%
\pgfpathlineto{\pgfqpoint{4.596137in}{1.722910in}}%
\pgfpathlineto{\pgfqpoint{4.598948in}{1.735472in}}%
\pgfpathlineto{\pgfqpoint{4.601759in}{1.721793in}}%
\pgfpathlineto{\pgfqpoint{4.604569in}{1.714111in}}%
\pgfpathlineto{\pgfqpoint{4.607380in}{1.708843in}}%
\pgfpathlineto{\pgfqpoint{4.610191in}{1.713404in}}%
\pgfpathlineto{\pgfqpoint{4.613001in}{1.710642in}}%
\pgfpathlineto{\pgfqpoint{4.615812in}{1.710325in}}%
\pgfpathlineto{\pgfqpoint{4.618623in}{1.713413in}}%
\pgfpathlineto{\pgfqpoint{4.621433in}{1.715099in}}%
\pgfpathlineto{\pgfqpoint{4.624244in}{1.707185in}}%
\pgfpathlineto{\pgfqpoint{4.627055in}{1.717009in}}%
\pgfpathlineto{\pgfqpoint{4.629865in}{1.712068in}}%
\pgfpathlineto{\pgfqpoint{4.632676in}{1.711272in}}%
\pgfpathlineto{\pgfqpoint{4.635487in}{1.716494in}}%
\pgfpathlineto{\pgfqpoint{4.638298in}{1.713568in}}%
\pgfpathlineto{\pgfqpoint{4.641108in}{1.713288in}}%
\pgfpathlineto{\pgfqpoint{4.643919in}{1.715257in}}%
\pgfpathlineto{\pgfqpoint{4.646730in}{1.711686in}}%
\pgfpathlineto{\pgfqpoint{4.649540in}{1.698479in}}%
\pgfpathlineto{\pgfqpoint{4.652351in}{1.703922in}}%
\pgfpathlineto{\pgfqpoint{4.655162in}{1.706238in}}%
\pgfpathlineto{\pgfqpoint{4.657972in}{1.688524in}}%
\pgfpathlineto{\pgfqpoint{4.660783in}{1.695979in}}%
\pgfpathlineto{\pgfqpoint{4.663594in}{1.691471in}}%
\pgfpathlineto{\pgfqpoint{4.666404in}{1.694574in}}%
\pgfpathlineto{\pgfqpoint{4.669215in}{1.704062in}}%
\pgfpathlineto{\pgfqpoint{4.674836in}{1.715757in}}%
\pgfpathlineto{\pgfqpoint{4.677647in}{1.714155in}}%
\pgfpathlineto{\pgfqpoint{4.680458in}{1.696180in}}%
\pgfpathlineto{\pgfqpoint{4.683268in}{1.702520in}}%
\pgfpathlineto{\pgfqpoint{4.686079in}{1.703281in}}%
\pgfpathlineto{\pgfqpoint{4.688890in}{1.713389in}}%
\pgfpathlineto{\pgfqpoint{4.691701in}{1.715471in}}%
\pgfpathlineto{\pgfqpoint{4.694511in}{1.713413in}}%
\pgfpathlineto{\pgfqpoint{4.697322in}{1.709629in}}%
\pgfpathlineto{\pgfqpoint{4.700133in}{1.712593in}}%
\pgfpathlineto{\pgfqpoint{4.702943in}{1.703931in}}%
\pgfpathlineto{\pgfqpoint{4.705754in}{1.699832in}}%
\pgfpathlineto{\pgfqpoint{4.708565in}{1.704877in}}%
\pgfpathlineto{\pgfqpoint{4.711375in}{1.723903in}}%
\pgfpathlineto{\pgfqpoint{4.714186in}{1.721008in}}%
\pgfpathlineto{\pgfqpoint{4.716997in}{1.707753in}}%
\pgfpathlineto{\pgfqpoint{4.719807in}{1.698031in}}%
\pgfpathlineto{\pgfqpoint{4.722618in}{1.695006in}}%
\pgfpathlineto{\pgfqpoint{4.725429in}{1.687894in}}%
\pgfpathlineto{\pgfqpoint{4.728239in}{1.692467in}}%
\pgfpathlineto{\pgfqpoint{4.731050in}{1.703647in}}%
\pgfpathlineto{\pgfqpoint{4.733861in}{1.705752in}}%
\pgfpathlineto{\pgfqpoint{4.736672in}{1.701053in}}%
\pgfpathlineto{\pgfqpoint{4.739482in}{1.702513in}}%
\pgfpathlineto{\pgfqpoint{4.742293in}{1.695838in}}%
\pgfpathlineto{\pgfqpoint{4.745104in}{1.695811in}}%
\pgfpathlineto{\pgfqpoint{4.747914in}{1.710982in}}%
\pgfpathlineto{\pgfqpoint{4.750725in}{1.707685in}}%
\pgfpathlineto{\pgfqpoint{4.753536in}{1.709258in}}%
\pgfpathlineto{\pgfqpoint{4.756346in}{1.730363in}}%
\pgfpathlineto{\pgfqpoint{4.759157in}{1.738827in}}%
\pgfpathlineto{\pgfqpoint{4.761968in}{1.709891in}}%
\pgfpathlineto{\pgfqpoint{4.764778in}{1.707517in}}%
\pgfpathlineto{\pgfqpoint{4.767589in}{1.724430in}}%
\pgfpathlineto{\pgfqpoint{4.770400in}{1.720730in}}%
\pgfpathlineto{\pgfqpoint{4.773210in}{1.709940in}}%
\pgfpathlineto{\pgfqpoint{4.776021in}{1.716122in}}%
\pgfpathlineto{\pgfqpoint{4.778832in}{1.712814in}}%
\pgfpathlineto{\pgfqpoint{4.781643in}{1.720555in}}%
\pgfpathlineto{\pgfqpoint{4.784453in}{1.696372in}}%
\pgfpathlineto{\pgfqpoint{4.787264in}{1.706068in}}%
\pgfpathlineto{\pgfqpoint{4.790075in}{1.712392in}}%
\pgfpathlineto{\pgfqpoint{4.792885in}{1.694722in}}%
\pgfpathlineto{\pgfqpoint{4.795696in}{1.724512in}}%
\pgfpathlineto{\pgfqpoint{4.798507in}{1.729354in}}%
\pgfpathlineto{\pgfqpoint{4.801317in}{1.729921in}}%
\pgfpathlineto{\pgfqpoint{4.804128in}{1.715494in}}%
\pgfpathlineto{\pgfqpoint{4.806939in}{1.724648in}}%
\pgfpathlineto{\pgfqpoint{4.812560in}{1.696140in}}%
\pgfpathlineto{\pgfqpoint{4.815371in}{1.696962in}}%
\pgfpathlineto{\pgfqpoint{4.818181in}{1.691517in}}%
\pgfpathlineto{\pgfqpoint{4.820992in}{1.701036in}}%
\pgfpathlineto{\pgfqpoint{4.823803in}{1.691608in}}%
\pgfpathlineto{\pgfqpoint{4.826614in}{1.699384in}}%
\pgfpathlineto{\pgfqpoint{4.829424in}{1.693662in}}%
\pgfpathlineto{\pgfqpoint{4.832235in}{1.690887in}}%
\pgfpathlineto{\pgfqpoint{4.835046in}{1.699445in}}%
\pgfpathlineto{\pgfqpoint{4.837856in}{1.684524in}}%
\pgfpathlineto{\pgfqpoint{4.840667in}{1.683705in}}%
\pgfpathlineto{\pgfqpoint{4.843478in}{1.697477in}}%
\pgfpathlineto{\pgfqpoint{4.846288in}{1.691651in}}%
\pgfpathlineto{\pgfqpoint{4.849099in}{1.714308in}}%
\pgfpathlineto{\pgfqpoint{4.851910in}{1.719051in}}%
\pgfpathlineto{\pgfqpoint{4.854720in}{1.711941in}}%
\pgfpathlineto{\pgfqpoint{4.857531in}{1.695661in}}%
\pgfpathlineto{\pgfqpoint{4.860342in}{1.695822in}}%
\pgfpathlineto{\pgfqpoint{4.863152in}{1.709783in}}%
\pgfpathlineto{\pgfqpoint{4.865963in}{1.707036in}}%
\pgfpathlineto{\pgfqpoint{4.868774in}{1.701352in}}%
\pgfpathlineto{\pgfqpoint{4.874395in}{1.713113in}}%
\pgfpathlineto{\pgfqpoint{4.877206in}{1.731695in}}%
\pgfpathlineto{\pgfqpoint{4.880017in}{1.718103in}}%
\pgfpathlineto{\pgfqpoint{4.882827in}{1.694118in}}%
\pgfpathlineto{\pgfqpoint{4.888449in}{1.687088in}}%
\pgfpathlineto{\pgfqpoint{4.891259in}{1.709258in}}%
\pgfpathlineto{\pgfqpoint{4.894070in}{1.706143in}}%
\pgfpathlineto{\pgfqpoint{4.899691in}{1.693307in}}%
\pgfpathlineto{\pgfqpoint{4.902502in}{1.710335in}}%
\pgfpathlineto{\pgfqpoint{4.905313in}{1.710675in}}%
\pgfpathlineto{\pgfqpoint{4.908123in}{1.709788in}}%
\pgfpathlineto{\pgfqpoint{4.910934in}{1.705450in}}%
\pgfpathlineto{\pgfqpoint{4.919366in}{1.713566in}}%
\pgfpathlineto{\pgfqpoint{4.922177in}{1.708805in}}%
\pgfpathlineto{\pgfqpoint{4.924988in}{1.720951in}}%
\pgfpathlineto{\pgfqpoint{4.927798in}{1.716479in}}%
\pgfpathlineto{\pgfqpoint{4.930609in}{1.698378in}}%
\pgfpathlineto{\pgfqpoint{4.933420in}{1.704518in}}%
\pgfpathlineto{\pgfqpoint{4.936230in}{1.686465in}}%
\pgfpathlineto{\pgfqpoint{4.939041in}{1.693994in}}%
\pgfpathlineto{\pgfqpoint{4.941852in}{1.704412in}}%
\pgfpathlineto{\pgfqpoint{4.944662in}{1.704083in}}%
\pgfpathlineto{\pgfqpoint{4.947473in}{1.716281in}}%
\pgfpathlineto{\pgfqpoint{4.950284in}{1.721057in}}%
\pgfpathlineto{\pgfqpoint{4.953094in}{1.711633in}}%
\pgfpathlineto{\pgfqpoint{4.955905in}{1.717359in}}%
\pgfpathlineto{\pgfqpoint{4.958716in}{1.711873in}}%
\pgfpathlineto{\pgfqpoint{4.961526in}{1.709310in}}%
\pgfpathlineto{\pgfqpoint{4.964337in}{1.741411in}}%
\pgfpathlineto{\pgfqpoint{4.967148in}{1.733515in}}%
\pgfpathlineto{\pgfqpoint{4.969959in}{1.715602in}}%
\pgfpathlineto{\pgfqpoint{4.972769in}{1.706256in}}%
\pgfpathlineto{\pgfqpoint{4.975580in}{1.700345in}}%
\pgfpathlineto{\pgfqpoint{4.981201in}{1.703445in}}%
\pgfpathlineto{\pgfqpoint{4.984012in}{1.709660in}}%
\pgfpathlineto{\pgfqpoint{4.986823in}{1.699581in}}%
\pgfpathlineto{\pgfqpoint{4.989633in}{1.697773in}}%
\pgfpathlineto{\pgfqpoint{4.992444in}{1.699201in}}%
\pgfpathlineto{\pgfqpoint{4.995255in}{1.699850in}}%
\pgfpathlineto{\pgfqpoint{4.998065in}{1.690594in}}%
\pgfpathlineto{\pgfqpoint{5.000876in}{1.701418in}}%
\pgfpathlineto{\pgfqpoint{5.003687in}{1.698635in}}%
\pgfpathlineto{\pgfqpoint{5.006497in}{1.707245in}}%
\pgfpathlineto{\pgfqpoint{5.009308in}{1.717975in}}%
\pgfpathlineto{\pgfqpoint{5.012119in}{1.713519in}}%
\pgfpathlineto{\pgfqpoint{5.014930in}{1.722599in}}%
\pgfpathlineto{\pgfqpoint{5.017740in}{1.713355in}}%
\pgfpathlineto{\pgfqpoint{5.020551in}{1.713512in}}%
\pgfpathlineto{\pgfqpoint{5.026172in}{1.693635in}}%
\pgfpathlineto{\pgfqpoint{5.028983in}{1.695975in}}%
\pgfpathlineto{\pgfqpoint{5.031794in}{1.702271in}}%
\pgfpathlineto{\pgfqpoint{5.034604in}{1.701877in}}%
\pgfpathlineto{\pgfqpoint{5.037415in}{1.706037in}}%
\pgfpathlineto{\pgfqpoint{5.040226in}{1.695721in}}%
\pgfpathlineto{\pgfqpoint{5.043036in}{1.702122in}}%
\pgfpathlineto{\pgfqpoint{5.045847in}{1.701162in}}%
\pgfpathlineto{\pgfqpoint{5.048658in}{1.698577in}}%
\pgfpathlineto{\pgfqpoint{5.051468in}{1.701157in}}%
\pgfpathlineto{\pgfqpoint{5.054279in}{1.700684in}}%
\pgfpathlineto{\pgfqpoint{5.057090in}{1.708219in}}%
\pgfpathlineto{\pgfqpoint{5.062711in}{1.718333in}}%
\pgfpathlineto{\pgfqpoint{5.065522in}{1.710288in}}%
\pgfpathlineto{\pgfqpoint{5.068333in}{1.730925in}}%
\pgfpathlineto{\pgfqpoint{5.071143in}{1.714444in}}%
\pgfpathlineto{\pgfqpoint{5.073954in}{1.727879in}}%
\pgfpathlineto{\pgfqpoint{5.076765in}{1.721842in}}%
\pgfpathlineto{\pgfqpoint{5.079575in}{1.702059in}}%
\pgfpathlineto{\pgfqpoint{5.082386in}{1.712624in}}%
\pgfpathlineto{\pgfqpoint{5.085197in}{1.699735in}}%
\pgfpathlineto{\pgfqpoint{5.088007in}{1.699174in}}%
\pgfpathlineto{\pgfqpoint{5.090818in}{1.692463in}}%
\pgfpathlineto{\pgfqpoint{5.093629in}{1.739109in}}%
\pgfpathlineto{\pgfqpoint{5.096439in}{1.699252in}}%
\pgfpathlineto{\pgfqpoint{5.099250in}{1.725679in}}%
\pgfpathlineto{\pgfqpoint{5.102061in}{1.761013in}}%
\pgfpathlineto{\pgfqpoint{5.104871in}{1.732375in}}%
\pgfpathlineto{\pgfqpoint{5.110493in}{1.693058in}}%
\pgfpathlineto{\pgfqpoint{5.113304in}{1.685338in}}%
\pgfpathlineto{\pgfqpoint{5.116114in}{1.696511in}}%
\pgfpathlineto{\pgfqpoint{5.118925in}{1.689017in}}%
\pgfpathlineto{\pgfqpoint{5.124546in}{1.707310in}}%
\pgfpathlineto{\pgfqpoint{5.127357in}{1.698796in}}%
\pgfpathlineto{\pgfqpoint{5.130168in}{1.706257in}}%
\pgfpathlineto{\pgfqpoint{5.132978in}{1.724781in}}%
\pgfpathlineto{\pgfqpoint{5.135789in}{1.721524in}}%
\pgfpathlineto{\pgfqpoint{5.138600in}{1.715623in}}%
\pgfpathlineto{\pgfqpoint{5.141410in}{1.711629in}}%
\pgfpathlineto{\pgfqpoint{5.144221in}{1.708736in}}%
\pgfpathlineto{\pgfqpoint{5.147032in}{1.698263in}}%
\pgfpathlineto{\pgfqpoint{5.149842in}{1.699014in}}%
\pgfpathlineto{\pgfqpoint{5.149842in}{1.699014in}}%
\pgfusepath{stroke}%
\end{pgfscope}%
\begin{pgfscope}%
\pgfsetrectcap%
\pgfsetmiterjoin%
\pgfsetlinewidth{0.803000pt}%
\definecolor{currentstroke}{rgb}{1.000000,1.000000,1.000000}%
\pgfsetstrokecolor{currentstroke}%
\pgfsetdash{}{0pt}%
\pgfpathmoveto{\pgfqpoint{0.711206in}{0.331635in}}%
\pgfpathlineto{\pgfqpoint{0.711206in}{3.351635in}}%
\pgfusepath{stroke}%
\end{pgfscope}%
\begin{pgfscope}%
\pgfsetrectcap%
\pgfsetmiterjoin%
\pgfsetlinewidth{0.803000pt}%
\definecolor{currentstroke}{rgb}{1.000000,1.000000,1.000000}%
\pgfsetstrokecolor{currentstroke}%
\pgfsetdash{}{0pt}%
\pgfpathmoveto{\pgfqpoint{5.361206in}{0.331635in}}%
\pgfpathlineto{\pgfqpoint{5.361206in}{3.351635in}}%
\pgfusepath{stroke}%
\end{pgfscope}%
\begin{pgfscope}%
\pgfsetrectcap%
\pgfsetmiterjoin%
\pgfsetlinewidth{0.803000pt}%
\definecolor{currentstroke}{rgb}{1.000000,1.000000,1.000000}%
\pgfsetstrokecolor{currentstroke}%
\pgfsetdash{}{0pt}%
\pgfpathmoveto{\pgfqpoint{0.711206in}{0.331635in}}%
\pgfpathlineto{\pgfqpoint{5.361206in}{0.331635in}}%
\pgfusepath{stroke}%
\end{pgfscope}%
\begin{pgfscope}%
\pgfsetrectcap%
\pgfsetmiterjoin%
\pgfsetlinewidth{0.803000pt}%
\definecolor{currentstroke}{rgb}{1.000000,1.000000,1.000000}%
\pgfsetstrokecolor{currentstroke}%
\pgfsetdash{}{0pt}%
\pgfpathmoveto{\pgfqpoint{0.711206in}{3.351635in}}%
\pgfpathlineto{\pgfqpoint{5.361206in}{3.351635in}}%
\pgfusepath{stroke}%
\end{pgfscope}%
\end{pgfpicture}%
\makeatother%
\endgroup%

    %% Creator: Matplotlib, PGF backend
%%
%% To include the figure in your LaTeX document, write
%%   \input{<filename>.pgf}
%%
%% Make sure the required packages are loaded in your preamble
%%   \usepackage{pgf}
%%
%% Figures using additional raster images can only be included by \input if
%% they are in the same directory as the main LaTeX file. For loading figures
%% from other directories you can use the `import` package
%%   \usepackage{import}
%% and then include the figures with
%%   \import{<path to file>}{<filename>.pgf}
%%
%% Matplotlib used the following preamble
%%   \usepackage{fontspec}
%%   \setmainfont{DejaVuSerif.ttf}[Path=/opt/tljh/user/lib/python3.6/site-packages/matplotlib/mpl-data/fonts/ttf/]
%%   \setsansfont{DejaVuSans.ttf}[Path=/opt/tljh/user/lib/python3.6/site-packages/matplotlib/mpl-data/fonts/ttf/]
%%   \setmonofont{DejaVuSansMono.ttf}[Path=/opt/tljh/user/lib/python3.6/site-packages/matplotlib/mpl-data/fonts/ttf/]
%%
\begingroup%
\makeatletter%
\begin{pgfpicture}%
\pgfpathrectangle{\pgfpointorigin}{\pgfqpoint{5.461206in}{3.451635in}}%
\pgfusepath{use as bounding box, clip}%
\begin{pgfscope}%
\pgfsetbuttcap%
\pgfsetmiterjoin%
\definecolor{currentfill}{rgb}{1.000000,1.000000,1.000000}%
\pgfsetfillcolor{currentfill}%
\pgfsetlinewidth{0.000000pt}%
\definecolor{currentstroke}{rgb}{1.000000,1.000000,1.000000}%
\pgfsetstrokecolor{currentstroke}%
\pgfsetdash{}{0pt}%
\pgfpathmoveto{\pgfqpoint{0.000000in}{0.000000in}}%
\pgfpathlineto{\pgfqpoint{5.461206in}{0.000000in}}%
\pgfpathlineto{\pgfqpoint{5.461206in}{3.451635in}}%
\pgfpathlineto{\pgfqpoint{0.000000in}{3.451635in}}%
\pgfpathclose%
\pgfusepath{fill}%
\end{pgfscope}%
\begin{pgfscope}%
\pgfsetbuttcap%
\pgfsetmiterjoin%
\definecolor{currentfill}{rgb}{0.917647,0.917647,0.949020}%
\pgfsetfillcolor{currentfill}%
\pgfsetlinewidth{0.000000pt}%
\definecolor{currentstroke}{rgb}{0.000000,0.000000,0.000000}%
\pgfsetstrokecolor{currentstroke}%
\pgfsetstrokeopacity{0.000000}%
\pgfsetdash{}{0pt}%
\pgfpathmoveto{\pgfqpoint{0.711206in}{0.331635in}}%
\pgfpathlineto{\pgfqpoint{5.361206in}{0.331635in}}%
\pgfpathlineto{\pgfqpoint{5.361206in}{3.351635in}}%
\pgfpathlineto{\pgfqpoint{0.711206in}{3.351635in}}%
\pgfpathclose%
\pgfusepath{fill}%
\end{pgfscope}%
\begin{pgfscope}%
\pgfpathrectangle{\pgfqpoint{0.711206in}{0.331635in}}{\pgfqpoint{4.650000in}{3.020000in}}%
\pgfusepath{clip}%
\pgfsetroundcap%
\pgfsetroundjoin%
\pgfsetlinewidth{0.803000pt}%
\definecolor{currentstroke}{rgb}{1.000000,1.000000,1.000000}%
\pgfsetstrokecolor{currentstroke}%
\pgfsetdash{}{0pt}%
\pgfpathmoveto{\pgfqpoint{0.922570in}{0.331635in}}%
\pgfpathlineto{\pgfqpoint{0.922570in}{3.351635in}}%
\pgfusepath{stroke}%
\end{pgfscope}%
\begin{pgfscope}%
\definecolor{textcolor}{rgb}{0.150000,0.150000,0.150000}%
\pgfsetstrokecolor{textcolor}%
\pgfsetfillcolor{textcolor}%
\pgftext[x=0.922570in,y=0.234413in,,top]{\color{textcolor}\rmfamily\fontsize{10.000000}{12.000000}\selectfont 0}%
\end{pgfscope}%
\begin{pgfscope}%
\pgfpathrectangle{\pgfqpoint{0.711206in}{0.331635in}}{\pgfqpoint{4.650000in}{3.020000in}}%
\pgfusepath{clip}%
\pgfsetroundcap%
\pgfsetroundjoin%
\pgfsetlinewidth{0.803000pt}%
\definecolor{currentstroke}{rgb}{1.000000,1.000000,1.000000}%
\pgfsetstrokecolor{currentstroke}%
\pgfsetdash{}{0pt}%
\pgfpathmoveto{\pgfqpoint{1.484334in}{0.331635in}}%
\pgfpathlineto{\pgfqpoint{1.484334in}{3.351635in}}%
\pgfusepath{stroke}%
\end{pgfscope}%
\begin{pgfscope}%
\definecolor{textcolor}{rgb}{0.150000,0.150000,0.150000}%
\pgfsetstrokecolor{textcolor}%
\pgfsetfillcolor{textcolor}%
\pgftext[x=1.484334in,y=0.234413in,,top]{\color{textcolor}\rmfamily\fontsize{10.000000}{12.000000}\selectfont 200}%
\end{pgfscope}%
\begin{pgfscope}%
\pgfpathrectangle{\pgfqpoint{0.711206in}{0.331635in}}{\pgfqpoint{4.650000in}{3.020000in}}%
\pgfusepath{clip}%
\pgfsetroundcap%
\pgfsetroundjoin%
\pgfsetlinewidth{0.803000pt}%
\definecolor{currentstroke}{rgb}{1.000000,1.000000,1.000000}%
\pgfsetstrokecolor{currentstroke}%
\pgfsetdash{}{0pt}%
\pgfpathmoveto{\pgfqpoint{2.046097in}{0.331635in}}%
\pgfpathlineto{\pgfqpoint{2.046097in}{3.351635in}}%
\pgfusepath{stroke}%
\end{pgfscope}%
\begin{pgfscope}%
\definecolor{textcolor}{rgb}{0.150000,0.150000,0.150000}%
\pgfsetstrokecolor{textcolor}%
\pgfsetfillcolor{textcolor}%
\pgftext[x=2.046097in,y=0.234413in,,top]{\color{textcolor}\rmfamily\fontsize{10.000000}{12.000000}\selectfont 400}%
\end{pgfscope}%
\begin{pgfscope}%
\pgfpathrectangle{\pgfqpoint{0.711206in}{0.331635in}}{\pgfqpoint{4.650000in}{3.020000in}}%
\pgfusepath{clip}%
\pgfsetroundcap%
\pgfsetroundjoin%
\pgfsetlinewidth{0.803000pt}%
\definecolor{currentstroke}{rgb}{1.000000,1.000000,1.000000}%
\pgfsetstrokecolor{currentstroke}%
\pgfsetdash{}{0pt}%
\pgfpathmoveto{\pgfqpoint{2.607861in}{0.331635in}}%
\pgfpathlineto{\pgfqpoint{2.607861in}{3.351635in}}%
\pgfusepath{stroke}%
\end{pgfscope}%
\begin{pgfscope}%
\definecolor{textcolor}{rgb}{0.150000,0.150000,0.150000}%
\pgfsetstrokecolor{textcolor}%
\pgfsetfillcolor{textcolor}%
\pgftext[x=2.607861in,y=0.234413in,,top]{\color{textcolor}\rmfamily\fontsize{10.000000}{12.000000}\selectfont 600}%
\end{pgfscope}%
\begin{pgfscope}%
\pgfpathrectangle{\pgfqpoint{0.711206in}{0.331635in}}{\pgfqpoint{4.650000in}{3.020000in}}%
\pgfusepath{clip}%
\pgfsetroundcap%
\pgfsetroundjoin%
\pgfsetlinewidth{0.803000pt}%
\definecolor{currentstroke}{rgb}{1.000000,1.000000,1.000000}%
\pgfsetstrokecolor{currentstroke}%
\pgfsetdash{}{0pt}%
\pgfpathmoveto{\pgfqpoint{3.169625in}{0.331635in}}%
\pgfpathlineto{\pgfqpoint{3.169625in}{3.351635in}}%
\pgfusepath{stroke}%
\end{pgfscope}%
\begin{pgfscope}%
\definecolor{textcolor}{rgb}{0.150000,0.150000,0.150000}%
\pgfsetstrokecolor{textcolor}%
\pgfsetfillcolor{textcolor}%
\pgftext[x=3.169625in,y=0.234413in,,top]{\color{textcolor}\rmfamily\fontsize{10.000000}{12.000000}\selectfont 800}%
\end{pgfscope}%
\begin{pgfscope}%
\pgfpathrectangle{\pgfqpoint{0.711206in}{0.331635in}}{\pgfqpoint{4.650000in}{3.020000in}}%
\pgfusepath{clip}%
\pgfsetroundcap%
\pgfsetroundjoin%
\pgfsetlinewidth{0.803000pt}%
\definecolor{currentstroke}{rgb}{1.000000,1.000000,1.000000}%
\pgfsetstrokecolor{currentstroke}%
\pgfsetdash{}{0pt}%
\pgfpathmoveto{\pgfqpoint{3.731389in}{0.331635in}}%
\pgfpathlineto{\pgfqpoint{3.731389in}{3.351635in}}%
\pgfusepath{stroke}%
\end{pgfscope}%
\begin{pgfscope}%
\definecolor{textcolor}{rgb}{0.150000,0.150000,0.150000}%
\pgfsetstrokecolor{textcolor}%
\pgfsetfillcolor{textcolor}%
\pgftext[x=3.731389in,y=0.234413in,,top]{\color{textcolor}\rmfamily\fontsize{10.000000}{12.000000}\selectfont 1000}%
\end{pgfscope}%
\begin{pgfscope}%
\pgfpathrectangle{\pgfqpoint{0.711206in}{0.331635in}}{\pgfqpoint{4.650000in}{3.020000in}}%
\pgfusepath{clip}%
\pgfsetroundcap%
\pgfsetroundjoin%
\pgfsetlinewidth{0.803000pt}%
\definecolor{currentstroke}{rgb}{1.000000,1.000000,1.000000}%
\pgfsetstrokecolor{currentstroke}%
\pgfsetdash{}{0pt}%
\pgfpathmoveto{\pgfqpoint{4.293153in}{0.331635in}}%
\pgfpathlineto{\pgfqpoint{4.293153in}{3.351635in}}%
\pgfusepath{stroke}%
\end{pgfscope}%
\begin{pgfscope}%
\definecolor{textcolor}{rgb}{0.150000,0.150000,0.150000}%
\pgfsetstrokecolor{textcolor}%
\pgfsetfillcolor{textcolor}%
\pgftext[x=4.293153in,y=0.234413in,,top]{\color{textcolor}\rmfamily\fontsize{10.000000}{12.000000}\selectfont 1200}%
\end{pgfscope}%
\begin{pgfscope}%
\pgfpathrectangle{\pgfqpoint{0.711206in}{0.331635in}}{\pgfqpoint{4.650000in}{3.020000in}}%
\pgfusepath{clip}%
\pgfsetroundcap%
\pgfsetroundjoin%
\pgfsetlinewidth{0.803000pt}%
\definecolor{currentstroke}{rgb}{1.000000,1.000000,1.000000}%
\pgfsetstrokecolor{currentstroke}%
\pgfsetdash{}{0pt}%
\pgfpathmoveto{\pgfqpoint{4.854916in}{0.331635in}}%
\pgfpathlineto{\pgfqpoint{4.854916in}{3.351635in}}%
\pgfusepath{stroke}%
\end{pgfscope}%
\begin{pgfscope}%
\definecolor{textcolor}{rgb}{0.150000,0.150000,0.150000}%
\pgfsetstrokecolor{textcolor}%
\pgfsetfillcolor{textcolor}%
\pgftext[x=4.854916in,y=0.234413in,,top]{\color{textcolor}\rmfamily\fontsize{10.000000}{12.000000}\selectfont 1400}%
\end{pgfscope}%
\begin{pgfscope}%
\pgfpathrectangle{\pgfqpoint{0.711206in}{0.331635in}}{\pgfqpoint{4.650000in}{3.020000in}}%
\pgfusepath{clip}%
\pgfsetroundcap%
\pgfsetroundjoin%
\pgfsetlinewidth{0.803000pt}%
\definecolor{currentstroke}{rgb}{1.000000,1.000000,1.000000}%
\pgfsetstrokecolor{currentstroke}%
\pgfsetdash{}{0pt}%
\pgfpathmoveto{\pgfqpoint{0.711206in}{0.398433in}}%
\pgfpathlineto{\pgfqpoint{5.361206in}{0.398433in}}%
\pgfusepath{stroke}%
\end{pgfscope}%
\begin{pgfscope}%
\definecolor{textcolor}{rgb}{0.150000,0.150000,0.150000}%
\pgfsetstrokecolor{textcolor}%
\pgfsetfillcolor{textcolor}%
\pgftext[x=0.100000in,y=0.345672in,left,base]{\color{textcolor}\rmfamily\fontsize{10.000000}{12.000000}\selectfont −0.100}%
\end{pgfscope}%
\begin{pgfscope}%
\pgfpathrectangle{\pgfqpoint{0.711206in}{0.331635in}}{\pgfqpoint{4.650000in}{3.020000in}}%
\pgfusepath{clip}%
\pgfsetroundcap%
\pgfsetroundjoin%
\pgfsetlinewidth{0.803000pt}%
\definecolor{currentstroke}{rgb}{1.000000,1.000000,1.000000}%
\pgfsetstrokecolor{currentstroke}%
\pgfsetdash{}{0pt}%
\pgfpathmoveto{\pgfqpoint{0.711206in}{0.771413in}}%
\pgfpathlineto{\pgfqpoint{5.361206in}{0.771413in}}%
\pgfusepath{stroke}%
\end{pgfscope}%
\begin{pgfscope}%
\definecolor{textcolor}{rgb}{0.150000,0.150000,0.150000}%
\pgfsetstrokecolor{textcolor}%
\pgfsetfillcolor{textcolor}%
\pgftext[x=0.100000in,y=0.718652in,left,base]{\color{textcolor}\rmfamily\fontsize{10.000000}{12.000000}\selectfont −0.075}%
\end{pgfscope}%
\begin{pgfscope}%
\pgfpathrectangle{\pgfqpoint{0.711206in}{0.331635in}}{\pgfqpoint{4.650000in}{3.020000in}}%
\pgfusepath{clip}%
\pgfsetroundcap%
\pgfsetroundjoin%
\pgfsetlinewidth{0.803000pt}%
\definecolor{currentstroke}{rgb}{1.000000,1.000000,1.000000}%
\pgfsetstrokecolor{currentstroke}%
\pgfsetdash{}{0pt}%
\pgfpathmoveto{\pgfqpoint{0.711206in}{1.144393in}}%
\pgfpathlineto{\pgfqpoint{5.361206in}{1.144393in}}%
\pgfusepath{stroke}%
\end{pgfscope}%
\begin{pgfscope}%
\definecolor{textcolor}{rgb}{0.150000,0.150000,0.150000}%
\pgfsetstrokecolor{textcolor}%
\pgfsetfillcolor{textcolor}%
\pgftext[x=0.100000in,y=1.091632in,left,base]{\color{textcolor}\rmfamily\fontsize{10.000000}{12.000000}\selectfont −0.050}%
\end{pgfscope}%
\begin{pgfscope}%
\pgfpathrectangle{\pgfqpoint{0.711206in}{0.331635in}}{\pgfqpoint{4.650000in}{3.020000in}}%
\pgfusepath{clip}%
\pgfsetroundcap%
\pgfsetroundjoin%
\pgfsetlinewidth{0.803000pt}%
\definecolor{currentstroke}{rgb}{1.000000,1.000000,1.000000}%
\pgfsetstrokecolor{currentstroke}%
\pgfsetdash{}{0pt}%
\pgfpathmoveto{\pgfqpoint{0.711206in}{1.517373in}}%
\pgfpathlineto{\pgfqpoint{5.361206in}{1.517373in}}%
\pgfusepath{stroke}%
\end{pgfscope}%
\begin{pgfscope}%
\definecolor{textcolor}{rgb}{0.150000,0.150000,0.150000}%
\pgfsetstrokecolor{textcolor}%
\pgfsetfillcolor{textcolor}%
\pgftext[x=0.100000in,y=1.464612in,left,base]{\color{textcolor}\rmfamily\fontsize{10.000000}{12.000000}\selectfont −0.025}%
\end{pgfscope}%
\begin{pgfscope}%
\pgfpathrectangle{\pgfqpoint{0.711206in}{0.331635in}}{\pgfqpoint{4.650000in}{3.020000in}}%
\pgfusepath{clip}%
\pgfsetroundcap%
\pgfsetroundjoin%
\pgfsetlinewidth{0.803000pt}%
\definecolor{currentstroke}{rgb}{1.000000,1.000000,1.000000}%
\pgfsetstrokecolor{currentstroke}%
\pgfsetdash{}{0pt}%
\pgfpathmoveto{\pgfqpoint{0.711206in}{1.890353in}}%
\pgfpathlineto{\pgfqpoint{5.361206in}{1.890353in}}%
\pgfusepath{stroke}%
\end{pgfscope}%
\begin{pgfscope}%
\definecolor{textcolor}{rgb}{0.150000,0.150000,0.150000}%
\pgfsetstrokecolor{textcolor}%
\pgfsetfillcolor{textcolor}%
\pgftext[x=0.216374in,y=1.837592in,left,base]{\color{textcolor}\rmfamily\fontsize{10.000000}{12.000000}\selectfont 0.000}%
\end{pgfscope}%
\begin{pgfscope}%
\pgfpathrectangle{\pgfqpoint{0.711206in}{0.331635in}}{\pgfqpoint{4.650000in}{3.020000in}}%
\pgfusepath{clip}%
\pgfsetroundcap%
\pgfsetroundjoin%
\pgfsetlinewidth{0.803000pt}%
\definecolor{currentstroke}{rgb}{1.000000,1.000000,1.000000}%
\pgfsetstrokecolor{currentstroke}%
\pgfsetdash{}{0pt}%
\pgfpathmoveto{\pgfqpoint{0.711206in}{2.263333in}}%
\pgfpathlineto{\pgfqpoint{5.361206in}{2.263333in}}%
\pgfusepath{stroke}%
\end{pgfscope}%
\begin{pgfscope}%
\definecolor{textcolor}{rgb}{0.150000,0.150000,0.150000}%
\pgfsetstrokecolor{textcolor}%
\pgfsetfillcolor{textcolor}%
\pgftext[x=0.216374in,y=2.210571in,left,base]{\color{textcolor}\rmfamily\fontsize{10.000000}{12.000000}\selectfont 0.025}%
\end{pgfscope}%
\begin{pgfscope}%
\pgfpathrectangle{\pgfqpoint{0.711206in}{0.331635in}}{\pgfqpoint{4.650000in}{3.020000in}}%
\pgfusepath{clip}%
\pgfsetroundcap%
\pgfsetroundjoin%
\pgfsetlinewidth{0.803000pt}%
\definecolor{currentstroke}{rgb}{1.000000,1.000000,1.000000}%
\pgfsetstrokecolor{currentstroke}%
\pgfsetdash{}{0pt}%
\pgfpathmoveto{\pgfqpoint{0.711206in}{2.636313in}}%
\pgfpathlineto{\pgfqpoint{5.361206in}{2.636313in}}%
\pgfusepath{stroke}%
\end{pgfscope}%
\begin{pgfscope}%
\definecolor{textcolor}{rgb}{0.150000,0.150000,0.150000}%
\pgfsetstrokecolor{textcolor}%
\pgfsetfillcolor{textcolor}%
\pgftext[x=0.216374in,y=2.583551in,left,base]{\color{textcolor}\rmfamily\fontsize{10.000000}{12.000000}\selectfont 0.050}%
\end{pgfscope}%
\begin{pgfscope}%
\pgfpathrectangle{\pgfqpoint{0.711206in}{0.331635in}}{\pgfqpoint{4.650000in}{3.020000in}}%
\pgfusepath{clip}%
\pgfsetroundcap%
\pgfsetroundjoin%
\pgfsetlinewidth{0.803000pt}%
\definecolor{currentstroke}{rgb}{1.000000,1.000000,1.000000}%
\pgfsetstrokecolor{currentstroke}%
\pgfsetdash{}{0pt}%
\pgfpathmoveto{\pgfqpoint{0.711206in}{3.009293in}}%
\pgfpathlineto{\pgfqpoint{5.361206in}{3.009293in}}%
\pgfusepath{stroke}%
\end{pgfscope}%
\begin{pgfscope}%
\definecolor{textcolor}{rgb}{0.150000,0.150000,0.150000}%
\pgfsetstrokecolor{textcolor}%
\pgfsetfillcolor{textcolor}%
\pgftext[x=0.216374in,y=2.956531in,left,base]{\color{textcolor}\rmfamily\fontsize{10.000000}{12.000000}\selectfont 0.075}%
\end{pgfscope}%
\begin{pgfscope}%
\pgfpathrectangle{\pgfqpoint{0.711206in}{0.331635in}}{\pgfqpoint{4.650000in}{3.020000in}}%
\pgfusepath{clip}%
\pgfsetbuttcap%
\pgfsetroundjoin%
\definecolor{currentfill}{rgb}{1.000000,0.000000,0.000000}%
\pgfsetfillcolor{currentfill}%
\pgfsetfillopacity{0.400000}%
\pgfsetlinewidth{1.003750pt}%
\definecolor{currentstroke}{rgb}{1.000000,0.000000,0.000000}%
\pgfsetstrokecolor{currentstroke}%
\pgfsetstrokeopacity{0.400000}%
\pgfsetdash{}{0pt}%
\pgfpathmoveto{\pgfqpoint{0.922570in}{2.381465in}}%
\pgfpathlineto{\pgfqpoint{0.922570in}{1.683291in}}%
\pgfpathlineto{\pgfqpoint{0.925380in}{1.728086in}}%
\pgfpathlineto{\pgfqpoint{0.928191in}{1.739288in}}%
\pgfpathlineto{\pgfqpoint{0.931002in}{1.763352in}}%
\pgfpathlineto{\pgfqpoint{0.933812in}{1.736975in}}%
\pgfpathlineto{\pgfqpoint{0.936623in}{1.553023in}}%
\pgfpathlineto{\pgfqpoint{0.939434in}{1.557274in}}%
\pgfpathlineto{\pgfqpoint{0.942245in}{1.579118in}}%
\pgfpathlineto{\pgfqpoint{0.945055in}{1.600447in}}%
\pgfpathlineto{\pgfqpoint{0.947866in}{1.590490in}}%
\pgfpathlineto{\pgfqpoint{0.950677in}{1.609822in}}%
\pgfpathlineto{\pgfqpoint{0.953487in}{1.624161in}}%
\pgfpathlineto{\pgfqpoint{0.956298in}{1.626738in}}%
\pgfpathlineto{\pgfqpoint{0.959109in}{1.616995in}}%
\pgfpathlineto{\pgfqpoint{0.961919in}{1.620051in}}%
\pgfpathlineto{\pgfqpoint{0.964730in}{1.624609in}}%
\pgfpathlineto{\pgfqpoint{0.967541in}{1.601398in}}%
\pgfpathlineto{\pgfqpoint{0.970351in}{1.611102in}}%
\pgfpathlineto{\pgfqpoint{0.973162in}{1.612389in}}%
\pgfpathlineto{\pgfqpoint{0.975973in}{1.618978in}}%
\pgfpathlineto{\pgfqpoint{0.978783in}{1.622057in}}%
\pgfpathlineto{\pgfqpoint{0.981594in}{1.621544in}}%
\pgfpathlineto{\pgfqpoint{0.984405in}{1.629712in}}%
\pgfpathlineto{\pgfqpoint{0.987216in}{1.633328in}}%
\pgfpathlineto{\pgfqpoint{0.990026in}{1.628899in}}%
\pgfpathlineto{\pgfqpoint{0.992837in}{1.632146in}}%
\pgfpathlineto{\pgfqpoint{0.995648in}{1.637492in}}%
\pgfpathlineto{\pgfqpoint{0.998458in}{1.631764in}}%
\pgfpathlineto{\pgfqpoint{1.001269in}{1.637921in}}%
\pgfpathlineto{\pgfqpoint{1.004080in}{1.638722in}}%
\pgfpathlineto{\pgfqpoint{1.006890in}{1.633359in}}%
\pgfpathlineto{\pgfqpoint{1.009701in}{1.614909in}}%
\pgfpathlineto{\pgfqpoint{1.012512in}{1.615898in}}%
\pgfpathlineto{\pgfqpoint{1.015322in}{1.619737in}}%
\pgfpathlineto{\pgfqpoint{1.018133in}{1.625007in}}%
\pgfpathlineto{\pgfqpoint{1.020944in}{1.629249in}}%
\pgfpathlineto{\pgfqpoint{1.023754in}{1.618143in}}%
\pgfpathlineto{\pgfqpoint{1.026565in}{1.620332in}}%
\pgfpathlineto{\pgfqpoint{1.029376in}{1.624052in}}%
\pgfpathlineto{\pgfqpoint{1.032187in}{1.612787in}}%
\pgfpathlineto{\pgfqpoint{1.034997in}{1.616497in}}%
\pgfpathlineto{\pgfqpoint{1.037808in}{1.620438in}}%
\pgfpathlineto{\pgfqpoint{1.040619in}{1.621345in}}%
\pgfpathlineto{\pgfqpoint{1.043429in}{1.625444in}}%
\pgfpathlineto{\pgfqpoint{1.046240in}{1.625945in}}%
\pgfpathlineto{\pgfqpoint{1.049051in}{1.626346in}}%
\pgfpathlineto{\pgfqpoint{1.051861in}{1.628051in}}%
\pgfpathlineto{\pgfqpoint{1.054672in}{1.631633in}}%
\pgfpathlineto{\pgfqpoint{1.057483in}{1.633187in}}%
\pgfpathlineto{\pgfqpoint{1.060293in}{1.635316in}}%
\pgfpathlineto{\pgfqpoint{1.063104in}{1.637393in}}%
\pgfpathlineto{\pgfqpoint{1.065915in}{1.639590in}}%
\pgfpathlineto{\pgfqpoint{1.068725in}{1.642661in}}%
\pgfpathlineto{\pgfqpoint{1.071536in}{1.644008in}}%
\pgfpathlineto{\pgfqpoint{1.074347in}{1.646769in}}%
\pgfpathlineto{\pgfqpoint{1.077158in}{1.648376in}}%
\pgfpathlineto{\pgfqpoint{1.079968in}{1.639705in}}%
\pgfpathlineto{\pgfqpoint{1.082779in}{1.642009in}}%
\pgfpathlineto{\pgfqpoint{1.085590in}{1.643082in}}%
\pgfpathlineto{\pgfqpoint{1.088400in}{1.645871in}}%
\pgfpathlineto{\pgfqpoint{1.091211in}{1.641617in}}%
\pgfpathlineto{\pgfqpoint{1.094022in}{1.640118in}}%
\pgfpathlineto{\pgfqpoint{1.096832in}{1.642759in}}%
\pgfpathlineto{\pgfqpoint{1.099643in}{1.637714in}}%
\pgfpathlineto{\pgfqpoint{1.102454in}{1.633218in}}%
\pgfpathlineto{\pgfqpoint{1.105264in}{1.634551in}}%
\pgfpathlineto{\pgfqpoint{1.108075in}{1.632225in}}%
\pgfpathlineto{\pgfqpoint{1.110886in}{1.625575in}}%
\pgfpathlineto{\pgfqpoint{1.113696in}{1.627904in}}%
\pgfpathlineto{\pgfqpoint{1.116507in}{1.629967in}}%
\pgfpathlineto{\pgfqpoint{1.119318in}{1.619131in}}%
\pgfpathlineto{\pgfqpoint{1.122128in}{1.616668in}}%
\pgfpathlineto{\pgfqpoint{1.124939in}{1.617233in}}%
\pgfpathlineto{\pgfqpoint{1.127750in}{1.616833in}}%
\pgfpathlineto{\pgfqpoint{1.130561in}{1.616664in}}%
\pgfpathlineto{\pgfqpoint{1.133371in}{1.616047in}}%
\pgfpathlineto{\pgfqpoint{1.136182in}{1.617808in}}%
\pgfpathlineto{\pgfqpoint{1.138993in}{1.620151in}}%
\pgfpathlineto{\pgfqpoint{1.141803in}{1.621691in}}%
\pgfpathlineto{\pgfqpoint{1.144614in}{1.621430in}}%
\pgfpathlineto{\pgfqpoint{1.147425in}{1.623717in}}%
\pgfpathlineto{\pgfqpoint{1.150235in}{1.617984in}}%
\pgfpathlineto{\pgfqpoint{1.153046in}{1.605186in}}%
\pgfpathlineto{\pgfqpoint{1.155857in}{1.604970in}}%
\pgfpathlineto{\pgfqpoint{1.158667in}{1.600211in}}%
\pgfpathlineto{\pgfqpoint{1.161478in}{1.599703in}}%
\pgfpathlineto{\pgfqpoint{1.164289in}{1.601532in}}%
\pgfpathlineto{\pgfqpoint{1.167099in}{1.602884in}}%
\pgfpathlineto{\pgfqpoint{1.169910in}{1.592163in}}%
\pgfpathlineto{\pgfqpoint{1.172721in}{1.592277in}}%
\pgfpathlineto{\pgfqpoint{1.175532in}{1.588201in}}%
\pgfpathlineto{\pgfqpoint{1.178342in}{1.585502in}}%
\pgfpathlineto{\pgfqpoint{1.181153in}{1.585844in}}%
\pgfpathlineto{\pgfqpoint{1.183964in}{1.587802in}}%
\pgfpathlineto{\pgfqpoint{1.186774in}{1.588126in}}%
\pgfpathlineto{\pgfqpoint{1.189585in}{1.578525in}}%
\pgfpathlineto{\pgfqpoint{1.192396in}{1.580502in}}%
\pgfpathlineto{\pgfqpoint{1.195206in}{1.582414in}}%
\pgfpathlineto{\pgfqpoint{1.198017in}{1.583698in}}%
\pgfpathlineto{\pgfqpoint{1.200828in}{1.585354in}}%
\pgfpathlineto{\pgfqpoint{1.203638in}{1.583358in}}%
\pgfpathlineto{\pgfqpoint{1.206449in}{1.570799in}}%
\pgfpathlineto{\pgfqpoint{1.209260in}{1.571494in}}%
\pgfpathlineto{\pgfqpoint{1.212070in}{1.572356in}}%
\pgfpathlineto{\pgfqpoint{1.214881in}{1.570131in}}%
\pgfpathlineto{\pgfqpoint{1.217692in}{1.570542in}}%
\pgfpathlineto{\pgfqpoint{1.220503in}{1.570974in}}%
\pgfpathlineto{\pgfqpoint{1.223313in}{1.567053in}}%
\pgfpathlineto{\pgfqpoint{1.226124in}{1.566880in}}%
\pgfpathlineto{\pgfqpoint{1.228935in}{1.568397in}}%
\pgfpathlineto{\pgfqpoint{1.231745in}{1.569229in}}%
\pgfpathlineto{\pgfqpoint{1.234556in}{1.570624in}}%
\pgfpathlineto{\pgfqpoint{1.237367in}{1.572273in}}%
\pgfpathlineto{\pgfqpoint{1.240177in}{1.573938in}}%
\pgfpathlineto{\pgfqpoint{1.242988in}{1.575695in}}%
\pgfpathlineto{\pgfqpoint{1.245799in}{1.559047in}}%
\pgfpathlineto{\pgfqpoint{1.248609in}{1.560833in}}%
\pgfpathlineto{\pgfqpoint{1.251420in}{1.545953in}}%
\pgfpathlineto{\pgfqpoint{1.254231in}{1.547039in}}%
\pgfpathlineto{\pgfqpoint{1.257041in}{1.548846in}}%
\pgfpathlineto{\pgfqpoint{1.259852in}{1.546013in}}%
\pgfpathlineto{\pgfqpoint{1.262663in}{1.541969in}}%
\pgfpathlineto{\pgfqpoint{1.265474in}{1.543374in}}%
\pgfpathlineto{\pgfqpoint{1.268284in}{1.545172in}}%
\pgfpathlineto{\pgfqpoint{1.271095in}{1.543801in}}%
\pgfpathlineto{\pgfqpoint{1.273906in}{1.540994in}}%
\pgfpathlineto{\pgfqpoint{1.276716in}{1.542370in}}%
\pgfpathlineto{\pgfqpoint{1.279527in}{1.535579in}}%
\pgfpathlineto{\pgfqpoint{1.282338in}{1.535768in}}%
\pgfpathlineto{\pgfqpoint{1.285148in}{1.528011in}}%
\pgfpathlineto{\pgfqpoint{1.287959in}{1.528059in}}%
\pgfpathlineto{\pgfqpoint{1.290770in}{1.528600in}}%
\pgfpathlineto{\pgfqpoint{1.293580in}{1.530225in}}%
\pgfpathlineto{\pgfqpoint{1.296391in}{1.526567in}}%
\pgfpathlineto{\pgfqpoint{1.299202in}{1.526905in}}%
\pgfpathlineto{\pgfqpoint{1.302012in}{1.522156in}}%
\pgfpathlineto{\pgfqpoint{1.304823in}{1.521409in}}%
\pgfpathlineto{\pgfqpoint{1.307634in}{1.520794in}}%
\pgfpathlineto{\pgfqpoint{1.310445in}{1.522440in}}%
\pgfpathlineto{\pgfqpoint{1.313255in}{1.523579in}}%
\pgfpathlineto{\pgfqpoint{1.316066in}{1.523885in}}%
\pgfpathlineto{\pgfqpoint{1.318877in}{1.523185in}}%
\pgfpathlineto{\pgfqpoint{1.321687in}{1.524186in}}%
\pgfpathlineto{\pgfqpoint{1.324498in}{1.525773in}}%
\pgfpathlineto{\pgfqpoint{1.327309in}{1.526876in}}%
\pgfpathlineto{\pgfqpoint{1.330119in}{1.526937in}}%
\pgfpathlineto{\pgfqpoint{1.332930in}{1.528403in}}%
\pgfpathlineto{\pgfqpoint{1.335741in}{1.529966in}}%
\pgfpathlineto{\pgfqpoint{1.338551in}{1.531426in}}%
\pgfpathlineto{\pgfqpoint{1.341362in}{1.532870in}}%
\pgfpathlineto{\pgfqpoint{1.344173in}{1.534389in}}%
\pgfpathlineto{\pgfqpoint{1.346983in}{1.534393in}}%
\pgfpathlineto{\pgfqpoint{1.349794in}{1.534188in}}%
\pgfpathlineto{\pgfqpoint{1.352605in}{1.533981in}}%
\pgfpathlineto{\pgfqpoint{1.355415in}{1.535284in}}%
\pgfpathlineto{\pgfqpoint{1.358226in}{1.534635in}}%
\pgfpathlineto{\pgfqpoint{1.361037in}{1.535233in}}%
\pgfpathlineto{\pgfqpoint{1.363848in}{1.535749in}}%
\pgfpathlineto{\pgfqpoint{1.366658in}{1.533736in}}%
\pgfpathlineto{\pgfqpoint{1.369469in}{1.526749in}}%
\pgfpathlineto{\pgfqpoint{1.372280in}{1.527204in}}%
\pgfpathlineto{\pgfqpoint{1.375090in}{1.527956in}}%
\pgfpathlineto{\pgfqpoint{1.377901in}{1.529355in}}%
\pgfpathlineto{\pgfqpoint{1.380712in}{1.528037in}}%
\pgfpathlineto{\pgfqpoint{1.383522in}{1.525632in}}%
\pgfpathlineto{\pgfqpoint{1.386333in}{1.525195in}}%
\pgfpathlineto{\pgfqpoint{1.389144in}{1.522867in}}%
\pgfpathlineto{\pgfqpoint{1.391954in}{1.523847in}}%
\pgfpathlineto{\pgfqpoint{1.394765in}{1.521967in}}%
\pgfpathlineto{\pgfqpoint{1.397576in}{1.510848in}}%
\pgfpathlineto{\pgfqpoint{1.400386in}{1.498720in}}%
\pgfpathlineto{\pgfqpoint{1.403197in}{1.500057in}}%
\pgfpathlineto{\pgfqpoint{1.406008in}{1.500390in}}%
\pgfpathlineto{\pgfqpoint{1.408819in}{1.501774in}}%
\pgfpathlineto{\pgfqpoint{1.411629in}{1.502922in}}%
\pgfpathlineto{\pgfqpoint{1.414440in}{1.503755in}}%
\pgfpathlineto{\pgfqpoint{1.417251in}{1.505016in}}%
\pgfpathlineto{\pgfqpoint{1.420061in}{1.504747in}}%
\pgfpathlineto{\pgfqpoint{1.422872in}{1.505900in}}%
\pgfpathlineto{\pgfqpoint{1.425683in}{1.506766in}}%
\pgfpathlineto{\pgfqpoint{1.428493in}{1.505509in}}%
\pgfpathlineto{\pgfqpoint{1.431304in}{1.504873in}}%
\pgfpathlineto{\pgfqpoint{1.434115in}{1.506159in}}%
\pgfpathlineto{\pgfqpoint{1.436925in}{1.506492in}}%
\pgfpathlineto{\pgfqpoint{1.439736in}{1.503912in}}%
\pgfpathlineto{\pgfqpoint{1.442547in}{1.505170in}}%
\pgfpathlineto{\pgfqpoint{1.445357in}{1.506403in}}%
\pgfpathlineto{\pgfqpoint{1.448168in}{1.505422in}}%
\pgfpathlineto{\pgfqpoint{1.450979in}{1.506133in}}%
\pgfpathlineto{\pgfqpoint{1.453790in}{1.507326in}}%
\pgfpathlineto{\pgfqpoint{1.456600in}{1.507418in}}%
\pgfpathlineto{\pgfqpoint{1.459411in}{1.502092in}}%
\pgfpathlineto{\pgfqpoint{1.462222in}{1.502437in}}%
\pgfpathlineto{\pgfqpoint{1.465032in}{1.503072in}}%
\pgfpathlineto{\pgfqpoint{1.467843in}{1.502950in}}%
\pgfpathlineto{\pgfqpoint{1.470654in}{1.503994in}}%
\pgfpathlineto{\pgfqpoint{1.473464in}{1.502491in}}%
\pgfpathlineto{\pgfqpoint{1.476275in}{1.498143in}}%
\pgfpathlineto{\pgfqpoint{1.479086in}{1.498563in}}%
\pgfpathlineto{\pgfqpoint{1.481896in}{1.496417in}}%
\pgfpathlineto{\pgfqpoint{1.484707in}{1.497561in}}%
\pgfpathlineto{\pgfqpoint{1.487518in}{1.498750in}}%
\pgfpathlineto{\pgfqpoint{1.490328in}{1.499157in}}%
\pgfpathlineto{\pgfqpoint{1.493139in}{1.500237in}}%
\pgfpathlineto{\pgfqpoint{1.495950in}{1.501230in}}%
\pgfpathlineto{\pgfqpoint{1.498761in}{1.500017in}}%
\pgfpathlineto{\pgfqpoint{1.501571in}{1.498479in}}%
\pgfpathlineto{\pgfqpoint{1.504382in}{1.498402in}}%
\pgfpathlineto{\pgfqpoint{1.507193in}{1.499345in}}%
\pgfpathlineto{\pgfqpoint{1.510003in}{1.499811in}}%
\pgfpathlineto{\pgfqpoint{1.512814in}{1.490800in}}%
\pgfpathlineto{\pgfqpoint{1.515625in}{1.491405in}}%
\pgfpathlineto{\pgfqpoint{1.518435in}{1.492272in}}%
\pgfpathlineto{\pgfqpoint{1.521246in}{1.493086in}}%
\pgfpathlineto{\pgfqpoint{1.524057in}{1.489641in}}%
\pgfpathlineto{\pgfqpoint{1.526867in}{1.488353in}}%
\pgfpathlineto{\pgfqpoint{1.529678in}{1.489450in}}%
\pgfpathlineto{\pgfqpoint{1.532489in}{1.490549in}}%
\pgfpathlineto{\pgfqpoint{1.535299in}{1.491614in}}%
\pgfpathlineto{\pgfqpoint{1.538110in}{1.483311in}}%
\pgfpathlineto{\pgfqpoint{1.540921in}{1.483530in}}%
\pgfpathlineto{\pgfqpoint{1.543731in}{1.483840in}}%
\pgfpathlineto{\pgfqpoint{1.546542in}{1.484927in}}%
\pgfpathlineto{\pgfqpoint{1.549353in}{1.485963in}}%
\pgfpathlineto{\pgfqpoint{1.552164in}{1.487039in}}%
\pgfpathlineto{\pgfqpoint{1.554974in}{1.482429in}}%
\pgfpathlineto{\pgfqpoint{1.557785in}{1.483436in}}%
\pgfpathlineto{\pgfqpoint{1.560596in}{1.484266in}}%
\pgfpathlineto{\pgfqpoint{1.563406in}{1.484193in}}%
\pgfpathlineto{\pgfqpoint{1.566217in}{1.484560in}}%
\pgfpathlineto{\pgfqpoint{1.569028in}{1.485236in}}%
\pgfpathlineto{\pgfqpoint{1.571838in}{1.486123in}}%
\pgfpathlineto{\pgfqpoint{1.574649in}{1.486733in}}%
\pgfpathlineto{\pgfqpoint{1.577460in}{1.485666in}}%
\pgfpathlineto{\pgfqpoint{1.580270in}{1.486600in}}%
\pgfpathlineto{\pgfqpoint{1.583081in}{1.486590in}}%
\pgfpathlineto{\pgfqpoint{1.585892in}{1.487563in}}%
\pgfpathlineto{\pgfqpoint{1.588702in}{1.488507in}}%
\pgfpathlineto{\pgfqpoint{1.591513in}{1.488822in}}%
\pgfpathlineto{\pgfqpoint{1.594324in}{1.489849in}}%
\pgfpathlineto{\pgfqpoint{1.597135in}{1.490434in}}%
\pgfpathlineto{\pgfqpoint{1.599945in}{1.489892in}}%
\pgfpathlineto{\pgfqpoint{1.602756in}{1.490238in}}%
\pgfpathlineto{\pgfqpoint{1.605567in}{1.491067in}}%
\pgfpathlineto{\pgfqpoint{1.608377in}{1.491339in}}%
\pgfpathlineto{\pgfqpoint{1.611188in}{1.490553in}}%
\pgfpathlineto{\pgfqpoint{1.613999in}{1.490800in}}%
\pgfpathlineto{\pgfqpoint{1.616809in}{1.487920in}}%
\pgfpathlineto{\pgfqpoint{1.619620in}{1.488538in}}%
\pgfpathlineto{\pgfqpoint{1.622431in}{1.488689in}}%
\pgfpathlineto{\pgfqpoint{1.625241in}{1.489649in}}%
\pgfpathlineto{\pgfqpoint{1.628052in}{1.489794in}}%
\pgfpathlineto{\pgfqpoint{1.630863in}{1.490267in}}%
\pgfpathlineto{\pgfqpoint{1.633673in}{1.490868in}}%
\pgfpathlineto{\pgfqpoint{1.636484in}{1.491813in}}%
\pgfpathlineto{\pgfqpoint{1.639295in}{1.492594in}}%
\pgfpathlineto{\pgfqpoint{1.642106in}{1.492987in}}%
\pgfpathlineto{\pgfqpoint{1.644916in}{1.493890in}}%
\pgfpathlineto{\pgfqpoint{1.647727in}{1.493320in}}%
\pgfpathlineto{\pgfqpoint{1.650538in}{1.473186in}}%
\pgfpathlineto{\pgfqpoint{1.653348in}{1.473758in}}%
\pgfpathlineto{\pgfqpoint{1.656159in}{1.474372in}}%
\pgfpathlineto{\pgfqpoint{1.658970in}{1.474543in}}%
\pgfpathlineto{\pgfqpoint{1.661780in}{1.475367in}}%
\pgfpathlineto{\pgfqpoint{1.664591in}{1.476305in}}%
\pgfpathlineto{\pgfqpoint{1.667402in}{1.477191in}}%
\pgfpathlineto{\pgfqpoint{1.670212in}{1.478117in}}%
\pgfpathlineto{\pgfqpoint{1.673023in}{1.477128in}}%
\pgfpathlineto{\pgfqpoint{1.675834in}{1.477803in}}%
\pgfpathlineto{\pgfqpoint{1.678644in}{1.477765in}}%
\pgfpathlineto{\pgfqpoint{1.681455in}{1.478614in}}%
\pgfpathlineto{\pgfqpoint{1.684266in}{1.478575in}}%
\pgfpathlineto{\pgfqpoint{1.687077in}{1.478666in}}%
\pgfpathlineto{\pgfqpoint{1.689887in}{1.479576in}}%
\pgfpathlineto{\pgfqpoint{1.692698in}{1.480405in}}%
\pgfpathlineto{\pgfqpoint{1.695509in}{1.481320in}}%
\pgfpathlineto{\pgfqpoint{1.698319in}{1.482177in}}%
\pgfpathlineto{\pgfqpoint{1.701130in}{1.482853in}}%
\pgfpathlineto{\pgfqpoint{1.703941in}{1.483227in}}%
\pgfpathlineto{\pgfqpoint{1.706751in}{1.483896in}}%
\pgfpathlineto{\pgfqpoint{1.709562in}{1.482617in}}%
\pgfpathlineto{\pgfqpoint{1.712373in}{1.480080in}}%
\pgfpathlineto{\pgfqpoint{1.715183in}{1.480959in}}%
\pgfpathlineto{\pgfqpoint{1.717994in}{1.480954in}}%
\pgfpathlineto{\pgfqpoint{1.720805in}{1.481412in}}%
\pgfpathlineto{\pgfqpoint{1.723615in}{1.481940in}}%
\pgfpathlineto{\pgfqpoint{1.726426in}{1.482524in}}%
\pgfpathlineto{\pgfqpoint{1.729237in}{1.483395in}}%
\pgfpathlineto{\pgfqpoint{1.732048in}{1.484180in}}%
\pgfpathlineto{\pgfqpoint{1.734858in}{1.484985in}}%
\pgfpathlineto{\pgfqpoint{1.737669in}{1.485775in}}%
\pgfpathlineto{\pgfqpoint{1.740480in}{1.486634in}}%
\pgfpathlineto{\pgfqpoint{1.743290in}{1.485854in}}%
\pgfpathlineto{\pgfqpoint{1.746101in}{1.486697in}}%
\pgfpathlineto{\pgfqpoint{1.748912in}{1.487258in}}%
\pgfpathlineto{\pgfqpoint{1.751722in}{1.487990in}}%
\pgfpathlineto{\pgfqpoint{1.754533in}{1.488642in}}%
\pgfpathlineto{\pgfqpoint{1.757344in}{1.488175in}}%
\pgfpathlineto{\pgfqpoint{1.760154in}{1.488466in}}%
\pgfpathlineto{\pgfqpoint{1.762965in}{1.488753in}}%
\pgfpathlineto{\pgfqpoint{1.765776in}{1.489495in}}%
\pgfpathlineto{\pgfqpoint{1.768586in}{1.489725in}}%
\pgfpathlineto{\pgfqpoint{1.771397in}{1.490388in}}%
\pgfpathlineto{\pgfqpoint{1.774208in}{1.490451in}}%
\pgfpathlineto{\pgfqpoint{1.777018in}{1.489526in}}%
\pgfpathlineto{\pgfqpoint{1.779829in}{1.490286in}}%
\pgfpathlineto{\pgfqpoint{1.782640in}{1.490940in}}%
\pgfpathlineto{\pgfqpoint{1.785451in}{1.489539in}}%
\pgfpathlineto{\pgfqpoint{1.788261in}{1.490238in}}%
\pgfpathlineto{\pgfqpoint{1.791072in}{1.488695in}}%
\pgfpathlineto{\pgfqpoint{1.793883in}{1.489480in}}%
\pgfpathlineto{\pgfqpoint{1.796693in}{1.489353in}}%
\pgfpathlineto{\pgfqpoint{1.799504in}{1.490142in}}%
\pgfpathlineto{\pgfqpoint{1.802315in}{1.488971in}}%
\pgfpathlineto{\pgfqpoint{1.805125in}{1.488850in}}%
\pgfpathlineto{\pgfqpoint{1.807936in}{1.487356in}}%
\pgfpathlineto{\pgfqpoint{1.810747in}{1.487489in}}%
\pgfpathlineto{\pgfqpoint{1.813557in}{1.486905in}}%
\pgfpathlineto{\pgfqpoint{1.816368in}{1.486618in}}%
\pgfpathlineto{\pgfqpoint{1.819179in}{1.487274in}}%
\pgfpathlineto{\pgfqpoint{1.821989in}{1.487895in}}%
\pgfpathlineto{\pgfqpoint{1.824800in}{1.488655in}}%
\pgfpathlineto{\pgfqpoint{1.827611in}{1.488924in}}%
\pgfpathlineto{\pgfqpoint{1.830422in}{1.489004in}}%
\pgfpathlineto{\pgfqpoint{1.833232in}{1.489701in}}%
\pgfpathlineto{\pgfqpoint{1.836043in}{1.489313in}}%
\pgfpathlineto{\pgfqpoint{1.838854in}{1.489967in}}%
\pgfpathlineto{\pgfqpoint{1.841664in}{1.490501in}}%
\pgfpathlineto{\pgfqpoint{1.844475in}{1.491263in}}%
\pgfpathlineto{\pgfqpoint{1.847286in}{1.491942in}}%
\pgfpathlineto{\pgfqpoint{1.850096in}{1.492686in}}%
\pgfpathlineto{\pgfqpoint{1.852907in}{1.493387in}}%
\pgfpathlineto{\pgfqpoint{1.855718in}{1.493874in}}%
\pgfpathlineto{\pgfqpoint{1.858528in}{1.494593in}}%
\pgfpathlineto{\pgfqpoint{1.861339in}{1.495306in}}%
\pgfpathlineto{\pgfqpoint{1.864150in}{1.496025in}}%
\pgfpathlineto{\pgfqpoint{1.866960in}{1.496759in}}%
\pgfpathlineto{\pgfqpoint{1.869771in}{1.495607in}}%
\pgfpathlineto{\pgfqpoint{1.872582in}{1.495420in}}%
\pgfpathlineto{\pgfqpoint{1.875393in}{1.495948in}}%
\pgfpathlineto{\pgfqpoint{1.878203in}{1.495655in}}%
\pgfpathlineto{\pgfqpoint{1.881014in}{1.496353in}}%
\pgfpathlineto{\pgfqpoint{1.883825in}{1.496998in}}%
\pgfpathlineto{\pgfqpoint{1.886635in}{1.497667in}}%
\pgfpathlineto{\pgfqpoint{1.889446in}{1.498041in}}%
\pgfpathlineto{\pgfqpoint{1.892257in}{1.498556in}}%
\pgfpathlineto{\pgfqpoint{1.895067in}{1.498782in}}%
\pgfpathlineto{\pgfqpoint{1.897878in}{1.499490in}}%
\pgfpathlineto{\pgfqpoint{1.900689in}{1.500192in}}%
\pgfpathlineto{\pgfqpoint{1.903499in}{1.500618in}}%
\pgfpathlineto{\pgfqpoint{1.906310in}{1.501255in}}%
\pgfpathlineto{\pgfqpoint{1.909121in}{1.498839in}}%
\pgfpathlineto{\pgfqpoint{1.911931in}{1.499523in}}%
\pgfpathlineto{\pgfqpoint{1.914742in}{1.496804in}}%
\pgfpathlineto{\pgfqpoint{1.917553in}{1.497256in}}%
\pgfpathlineto{\pgfqpoint{1.920364in}{1.497680in}}%
\pgfpathlineto{\pgfqpoint{1.923174in}{1.498079in}}%
\pgfpathlineto{\pgfqpoint{1.925985in}{1.497666in}}%
\pgfpathlineto{\pgfqpoint{1.928796in}{1.497456in}}%
\pgfpathlineto{\pgfqpoint{1.931606in}{1.497531in}}%
\pgfpathlineto{\pgfqpoint{1.934417in}{1.497949in}}%
\pgfpathlineto{\pgfqpoint{1.937228in}{1.498634in}}%
\pgfpathlineto{\pgfqpoint{1.940038in}{1.499165in}}%
\pgfpathlineto{\pgfqpoint{1.942849in}{1.497891in}}%
\pgfpathlineto{\pgfqpoint{1.945660in}{1.493908in}}%
\pgfpathlineto{\pgfqpoint{1.948470in}{1.494464in}}%
\pgfpathlineto{\pgfqpoint{1.951281in}{1.491935in}}%
\pgfpathlineto{\pgfqpoint{1.954092in}{1.492523in}}%
\pgfpathlineto{\pgfqpoint{1.956902in}{1.493194in}}%
\pgfpathlineto{\pgfqpoint{1.959713in}{1.493781in}}%
\pgfpathlineto{\pgfqpoint{1.962524in}{1.494454in}}%
\pgfpathlineto{\pgfqpoint{1.965334in}{1.493852in}}%
\pgfpathlineto{\pgfqpoint{1.968145in}{1.493952in}}%
\pgfpathlineto{\pgfqpoint{1.970956in}{1.494533in}}%
\pgfpathlineto{\pgfqpoint{1.973767in}{1.495108in}}%
\pgfpathlineto{\pgfqpoint{1.976577in}{1.490357in}}%
\pgfpathlineto{\pgfqpoint{1.979388in}{1.490792in}}%
\pgfpathlineto{\pgfqpoint{1.982199in}{1.491440in}}%
\pgfpathlineto{\pgfqpoint{1.985009in}{1.490480in}}%
\pgfpathlineto{\pgfqpoint{1.987820in}{1.490829in}}%
\pgfpathlineto{\pgfqpoint{1.990631in}{1.491402in}}%
\pgfpathlineto{\pgfqpoint{1.993441in}{1.491963in}}%
\pgfpathlineto{\pgfqpoint{1.996252in}{1.492278in}}%
\pgfpathlineto{\pgfqpoint{1.999063in}{1.487185in}}%
\pgfpathlineto{\pgfqpoint{2.001873in}{1.487152in}}%
\pgfpathlineto{\pgfqpoint{2.004684in}{1.486865in}}%
\pgfpathlineto{\pgfqpoint{2.007495in}{1.487342in}}%
\pgfpathlineto{\pgfqpoint{2.010305in}{1.487990in}}%
\pgfpathlineto{\pgfqpoint{2.013116in}{1.488634in}}%
\pgfpathlineto{\pgfqpoint{2.015927in}{1.489271in}}%
\pgfpathlineto{\pgfqpoint{2.018738in}{1.489739in}}%
\pgfpathlineto{\pgfqpoint{2.021548in}{1.490379in}}%
\pgfpathlineto{\pgfqpoint{2.024359in}{1.490796in}}%
\pgfpathlineto{\pgfqpoint{2.027170in}{1.490999in}}%
\pgfpathlineto{\pgfqpoint{2.029980in}{1.491521in}}%
\pgfpathlineto{\pgfqpoint{2.032791in}{1.491880in}}%
\pgfpathlineto{\pgfqpoint{2.035602in}{1.492112in}}%
\pgfpathlineto{\pgfqpoint{2.038412in}{1.492376in}}%
\pgfpathlineto{\pgfqpoint{2.041223in}{1.492134in}}%
\pgfpathlineto{\pgfqpoint{2.044034in}{1.492710in}}%
\pgfpathlineto{\pgfqpoint{2.046844in}{1.493328in}}%
\pgfpathlineto{\pgfqpoint{2.049655in}{1.493552in}}%
\pgfpathlineto{\pgfqpoint{2.052466in}{1.494121in}}%
\pgfpathlineto{\pgfqpoint{2.055276in}{1.492113in}}%
\pgfpathlineto{\pgfqpoint{2.058087in}{1.492332in}}%
\pgfpathlineto{\pgfqpoint{2.060898in}{1.492686in}}%
\pgfpathlineto{\pgfqpoint{2.063709in}{1.493259in}}%
\pgfpathlineto{\pgfqpoint{2.066519in}{1.492560in}}%
\pgfpathlineto{\pgfqpoint{2.069330in}{1.493143in}}%
\pgfpathlineto{\pgfqpoint{2.072141in}{1.493742in}}%
\pgfpathlineto{\pgfqpoint{2.074951in}{1.493770in}}%
\pgfpathlineto{\pgfqpoint{2.077762in}{1.494110in}}%
\pgfpathlineto{\pgfqpoint{2.080573in}{1.494693in}}%
\pgfpathlineto{\pgfqpoint{2.083383in}{1.494538in}}%
\pgfpathlineto{\pgfqpoint{2.086194in}{1.494844in}}%
\pgfpathlineto{\pgfqpoint{2.089005in}{1.495423in}}%
\pgfpathlineto{\pgfqpoint{2.091815in}{1.495154in}}%
\pgfpathlineto{\pgfqpoint{2.094626in}{1.495564in}}%
\pgfpathlineto{\pgfqpoint{2.097437in}{1.496119in}}%
\pgfpathlineto{\pgfqpoint{2.100247in}{1.496674in}}%
\pgfpathlineto{\pgfqpoint{2.103058in}{1.497225in}}%
\pgfpathlineto{\pgfqpoint{2.105869in}{1.497297in}}%
\pgfpathlineto{\pgfqpoint{2.108680in}{1.497327in}}%
\pgfpathlineto{\pgfqpoint{2.111490in}{1.495870in}}%
\pgfpathlineto{\pgfqpoint{2.114301in}{1.496226in}}%
\pgfpathlineto{\pgfqpoint{2.117112in}{1.496637in}}%
\pgfpathlineto{\pgfqpoint{2.119922in}{1.497217in}}%
\pgfpathlineto{\pgfqpoint{2.122733in}{1.497689in}}%
\pgfpathlineto{\pgfqpoint{2.125544in}{1.497843in}}%
\pgfpathlineto{\pgfqpoint{2.128354in}{1.497995in}}%
\pgfpathlineto{\pgfqpoint{2.131165in}{1.498529in}}%
\pgfpathlineto{\pgfqpoint{2.133976in}{1.498979in}}%
\pgfpathlineto{\pgfqpoint{2.136786in}{1.498640in}}%
\pgfpathlineto{\pgfqpoint{2.139597in}{1.497601in}}%
\pgfpathlineto{\pgfqpoint{2.142408in}{1.497945in}}%
\pgfpathlineto{\pgfqpoint{2.145218in}{1.498209in}}%
\pgfpathlineto{\pgfqpoint{2.148029in}{1.498728in}}%
\pgfpathlineto{\pgfqpoint{2.150840in}{1.498357in}}%
\pgfpathlineto{\pgfqpoint{2.153651in}{1.498899in}}%
\pgfpathlineto{\pgfqpoint{2.156461in}{1.499359in}}%
\pgfpathlineto{\pgfqpoint{2.159272in}{1.498716in}}%
\pgfpathlineto{\pgfqpoint{2.162083in}{1.499262in}}%
\pgfpathlineto{\pgfqpoint{2.164893in}{1.499216in}}%
\pgfpathlineto{\pgfqpoint{2.167704in}{1.499769in}}%
\pgfpathlineto{\pgfqpoint{2.170515in}{1.500316in}}%
\pgfpathlineto{\pgfqpoint{2.173325in}{1.500650in}}%
\pgfpathlineto{\pgfqpoint{2.176136in}{1.501102in}}%
\pgfpathlineto{\pgfqpoint{2.178947in}{1.501632in}}%
\pgfpathlineto{\pgfqpoint{2.181757in}{1.502006in}}%
\pgfpathlineto{\pgfqpoint{2.184568in}{1.502520in}}%
\pgfpathlineto{\pgfqpoint{2.187379in}{1.502826in}}%
\pgfpathlineto{\pgfqpoint{2.190189in}{1.502366in}}%
\pgfpathlineto{\pgfqpoint{2.193000in}{1.502835in}}%
\pgfpathlineto{\pgfqpoint{2.195811in}{1.502998in}}%
\pgfpathlineto{\pgfqpoint{2.198621in}{1.503525in}}%
\pgfpathlineto{\pgfqpoint{2.201432in}{1.504060in}}%
\pgfpathlineto{\pgfqpoint{2.204243in}{1.504462in}}%
\pgfpathlineto{\pgfqpoint{2.207054in}{1.504826in}}%
\pgfpathlineto{\pgfqpoint{2.209864in}{1.504967in}}%
\pgfpathlineto{\pgfqpoint{2.212675in}{1.505265in}}%
\pgfpathlineto{\pgfqpoint{2.215486in}{1.505679in}}%
\pgfpathlineto{\pgfqpoint{2.218296in}{1.506190in}}%
\pgfpathlineto{\pgfqpoint{2.221107in}{1.506201in}}%
\pgfpathlineto{\pgfqpoint{2.223918in}{1.506641in}}%
\pgfpathlineto{\pgfqpoint{2.226728in}{1.507133in}}%
\pgfpathlineto{\pgfqpoint{2.229539in}{1.507618in}}%
\pgfpathlineto{\pgfqpoint{2.232350in}{1.508135in}}%
\pgfpathlineto{\pgfqpoint{2.235160in}{1.508115in}}%
\pgfpathlineto{\pgfqpoint{2.237971in}{1.508624in}}%
\pgfpathlineto{\pgfqpoint{2.240782in}{1.509100in}}%
\pgfpathlineto{\pgfqpoint{2.243592in}{1.509596in}}%
\pgfpathlineto{\pgfqpoint{2.246403in}{1.509732in}}%
\pgfpathlineto{\pgfqpoint{2.249214in}{1.509289in}}%
\pgfpathlineto{\pgfqpoint{2.252025in}{1.500692in}}%
\pgfpathlineto{\pgfqpoint{2.254835in}{1.500886in}}%
\pgfpathlineto{\pgfqpoint{2.257646in}{1.501130in}}%
\pgfpathlineto{\pgfqpoint{2.260457in}{1.501616in}}%
\pgfpathlineto{\pgfqpoint{2.263267in}{1.501928in}}%
\pgfpathlineto{\pgfqpoint{2.266078in}{1.502062in}}%
\pgfpathlineto{\pgfqpoint{2.268889in}{1.502165in}}%
\pgfpathlineto{\pgfqpoint{2.271699in}{1.502670in}}%
\pgfpathlineto{\pgfqpoint{2.274510in}{1.502651in}}%
\pgfpathlineto{\pgfqpoint{2.277321in}{1.502605in}}%
\pgfpathlineto{\pgfqpoint{2.280131in}{1.503095in}}%
\pgfpathlineto{\pgfqpoint{2.282942in}{1.503315in}}%
\pgfpathlineto{\pgfqpoint{2.285753in}{1.502602in}}%
\pgfpathlineto{\pgfqpoint{2.288563in}{1.503048in}}%
\pgfpathlineto{\pgfqpoint{2.291374in}{1.503098in}}%
\pgfpathlineto{\pgfqpoint{2.294185in}{1.503598in}}%
\pgfpathlineto{\pgfqpoint{2.296996in}{1.504093in}}%
\pgfpathlineto{\pgfqpoint{2.299806in}{1.504217in}}%
\pgfpathlineto{\pgfqpoint{2.302617in}{1.504590in}}%
\pgfpathlineto{\pgfqpoint{2.305428in}{1.504851in}}%
\pgfpathlineto{\pgfqpoint{2.308238in}{1.505320in}}%
\pgfpathlineto{\pgfqpoint{2.311049in}{1.505800in}}%
\pgfpathlineto{\pgfqpoint{2.313860in}{1.506262in}}%
\pgfpathlineto{\pgfqpoint{2.316670in}{1.506478in}}%
\pgfpathlineto{\pgfqpoint{2.319481in}{1.506951in}}%
\pgfpathlineto{\pgfqpoint{2.322292in}{1.507425in}}%
\pgfpathlineto{\pgfqpoint{2.325102in}{1.507489in}}%
\pgfpathlineto{\pgfqpoint{2.327913in}{1.507851in}}%
\pgfpathlineto{\pgfqpoint{2.330724in}{1.507506in}}%
\pgfpathlineto{\pgfqpoint{2.333534in}{1.507981in}}%
\pgfpathlineto{\pgfqpoint{2.336345in}{1.508093in}}%
\pgfpathlineto{\pgfqpoint{2.339156in}{1.508278in}}%
\pgfpathlineto{\pgfqpoint{2.341967in}{1.508752in}}%
\pgfpathlineto{\pgfqpoint{2.344777in}{1.509094in}}%
\pgfpathlineto{\pgfqpoint{2.347588in}{1.507404in}}%
\pgfpathlineto{\pgfqpoint{2.350399in}{1.507881in}}%
\pgfpathlineto{\pgfqpoint{2.353209in}{1.508050in}}%
\pgfpathlineto{\pgfqpoint{2.356020in}{1.506095in}}%
\pgfpathlineto{\pgfqpoint{2.358831in}{1.505943in}}%
\pgfpathlineto{\pgfqpoint{2.361641in}{1.505751in}}%
\pgfpathlineto{\pgfqpoint{2.364452in}{1.505779in}}%
\pgfpathlineto{\pgfqpoint{2.367263in}{1.505423in}}%
\pgfpathlineto{\pgfqpoint{2.370073in}{1.505670in}}%
\pgfpathlineto{\pgfqpoint{2.372884in}{1.506140in}}%
\pgfpathlineto{\pgfqpoint{2.375695in}{1.506068in}}%
\pgfpathlineto{\pgfqpoint{2.378505in}{1.506492in}}%
\pgfpathlineto{\pgfqpoint{2.381316in}{1.506480in}}%
\pgfpathlineto{\pgfqpoint{2.384127in}{1.504886in}}%
\pgfpathlineto{\pgfqpoint{2.386937in}{1.505033in}}%
\pgfpathlineto{\pgfqpoint{2.389748in}{1.505295in}}%
\pgfpathlineto{\pgfqpoint{2.392559in}{1.505411in}}%
\pgfpathlineto{\pgfqpoint{2.395370in}{1.505862in}}%
\pgfpathlineto{\pgfqpoint{2.398180in}{1.506298in}}%
\pgfpathlineto{\pgfqpoint{2.400991in}{1.506758in}}%
\pgfpathlineto{\pgfqpoint{2.403802in}{1.507191in}}%
\pgfpathlineto{\pgfqpoint{2.406612in}{1.507648in}}%
\pgfpathlineto{\pgfqpoint{2.409423in}{1.508061in}}%
\pgfpathlineto{\pgfqpoint{2.412234in}{1.508418in}}%
\pgfpathlineto{\pgfqpoint{2.415044in}{1.508247in}}%
\pgfpathlineto{\pgfqpoint{2.417855in}{1.508686in}}%
\pgfpathlineto{\pgfqpoint{2.420666in}{1.508332in}}%
\pgfpathlineto{\pgfqpoint{2.423476in}{1.508777in}}%
\pgfpathlineto{\pgfqpoint{2.426287in}{1.509115in}}%
\pgfpathlineto{\pgfqpoint{2.429098in}{1.509564in}}%
\pgfpathlineto{\pgfqpoint{2.431908in}{1.509870in}}%
\pgfpathlineto{\pgfqpoint{2.434719in}{1.510220in}}%
\pgfpathlineto{\pgfqpoint{2.437530in}{1.510049in}}%
\pgfpathlineto{\pgfqpoint{2.440341in}{1.510480in}}%
\pgfpathlineto{\pgfqpoint{2.443151in}{1.510669in}}%
\pgfpathlineto{\pgfqpoint{2.445962in}{1.511106in}}%
\pgfpathlineto{\pgfqpoint{2.448773in}{1.511464in}}%
\pgfpathlineto{\pgfqpoint{2.451583in}{1.511900in}}%
\pgfpathlineto{\pgfqpoint{2.454394in}{1.512066in}}%
\pgfpathlineto{\pgfqpoint{2.457205in}{1.512444in}}%
\pgfpathlineto{\pgfqpoint{2.460015in}{1.512457in}}%
\pgfpathlineto{\pgfqpoint{2.462826in}{1.512703in}}%
\pgfpathlineto{\pgfqpoint{2.465637in}{1.513133in}}%
\pgfpathlineto{\pgfqpoint{2.468447in}{1.513557in}}%
\pgfpathlineto{\pgfqpoint{2.471258in}{1.513986in}}%
\pgfpathlineto{\pgfqpoint{2.474069in}{1.514237in}}%
\pgfpathlineto{\pgfqpoint{2.476879in}{1.514082in}}%
\pgfpathlineto{\pgfqpoint{2.479690in}{1.514359in}}%
\pgfpathlineto{\pgfqpoint{2.482501in}{1.514700in}}%
\pgfpathlineto{\pgfqpoint{2.485312in}{1.514923in}}%
\pgfpathlineto{\pgfqpoint{2.488122in}{1.515164in}}%
\pgfpathlineto{\pgfqpoint{2.490933in}{1.515527in}}%
\pgfpathlineto{\pgfqpoint{2.493744in}{1.515951in}}%
\pgfpathlineto{\pgfqpoint{2.496554in}{1.516374in}}%
\pgfpathlineto{\pgfqpoint{2.499365in}{1.516577in}}%
\pgfpathlineto{\pgfqpoint{2.502176in}{1.516670in}}%
\pgfpathlineto{\pgfqpoint{2.504986in}{1.516566in}}%
\pgfpathlineto{\pgfqpoint{2.507797in}{1.516924in}}%
\pgfpathlineto{\pgfqpoint{2.510608in}{1.517186in}}%
\pgfpathlineto{\pgfqpoint{2.513418in}{1.517568in}}%
\pgfpathlineto{\pgfqpoint{2.516229in}{1.516488in}}%
\pgfpathlineto{\pgfqpoint{2.519040in}{1.516357in}}%
\pgfpathlineto{\pgfqpoint{2.521850in}{1.516655in}}%
\pgfpathlineto{\pgfqpoint{2.524661in}{1.517068in}}%
\pgfpathlineto{\pgfqpoint{2.527472in}{1.517482in}}%
\pgfpathlineto{\pgfqpoint{2.530283in}{1.517878in}}%
\pgfpathlineto{\pgfqpoint{2.533093in}{1.518080in}}%
\pgfpathlineto{\pgfqpoint{2.535904in}{1.518263in}}%
\pgfpathlineto{\pgfqpoint{2.538715in}{1.518463in}}%
\pgfpathlineto{\pgfqpoint{2.541525in}{1.518780in}}%
\pgfpathlineto{\pgfqpoint{2.544336in}{1.517928in}}%
\pgfpathlineto{\pgfqpoint{2.547147in}{1.518302in}}%
\pgfpathlineto{\pgfqpoint{2.549957in}{1.518707in}}%
\pgfpathlineto{\pgfqpoint{2.552768in}{1.519111in}}%
\pgfpathlineto{\pgfqpoint{2.555579in}{1.519040in}}%
\pgfpathlineto{\pgfqpoint{2.558389in}{1.519301in}}%
\pgfpathlineto{\pgfqpoint{2.561200in}{1.519602in}}%
\pgfpathlineto{\pgfqpoint{2.564011in}{1.519937in}}%
\pgfpathlineto{\pgfqpoint{2.566821in}{1.520338in}}%
\pgfpathlineto{\pgfqpoint{2.569632in}{1.520610in}}%
\pgfpathlineto{\pgfqpoint{2.572443in}{1.520882in}}%
\pgfpathlineto{\pgfqpoint{2.575253in}{1.521247in}}%
\pgfpathlineto{\pgfqpoint{2.578064in}{1.521618in}}%
\pgfpathlineto{\pgfqpoint{2.580875in}{1.521773in}}%
\pgfpathlineto{\pgfqpoint{2.583686in}{1.521459in}}%
\pgfpathlineto{\pgfqpoint{2.586496in}{1.521481in}}%
\pgfpathlineto{\pgfqpoint{2.589307in}{1.521869in}}%
\pgfpathlineto{\pgfqpoint{2.592118in}{1.522174in}}%
\pgfpathlineto{\pgfqpoint{2.594928in}{1.522565in}}%
\pgfpathlineto{\pgfqpoint{2.597739in}{1.522817in}}%
\pgfpathlineto{\pgfqpoint{2.600550in}{1.523203in}}%
\pgfpathlineto{\pgfqpoint{2.603360in}{1.523438in}}%
\pgfpathlineto{\pgfqpoint{2.606171in}{1.523826in}}%
\pgfpathlineto{\pgfqpoint{2.608982in}{1.524187in}}%
\pgfpathlineto{\pgfqpoint{2.611792in}{1.524495in}}%
\pgfpathlineto{\pgfqpoint{2.614603in}{1.524729in}}%
\pgfpathlineto{\pgfqpoint{2.617414in}{1.525007in}}%
\pgfpathlineto{\pgfqpoint{2.620224in}{1.525240in}}%
\pgfpathlineto{\pgfqpoint{2.623035in}{1.525582in}}%
\pgfpathlineto{\pgfqpoint{2.625846in}{1.525717in}}%
\pgfpathlineto{\pgfqpoint{2.628657in}{1.525619in}}%
\pgfpathlineto{\pgfqpoint{2.631467in}{1.525959in}}%
\pgfpathlineto{\pgfqpoint{2.634278in}{1.525758in}}%
\pgfpathlineto{\pgfqpoint{2.637089in}{1.526071in}}%
\pgfpathlineto{\pgfqpoint{2.639899in}{1.519915in}}%
\pgfpathlineto{\pgfqpoint{2.642710in}{1.520291in}}%
\pgfpathlineto{\pgfqpoint{2.645521in}{1.520528in}}%
\pgfpathlineto{\pgfqpoint{2.648331in}{1.520801in}}%
\pgfpathlineto{\pgfqpoint{2.651142in}{1.521180in}}%
\pgfpathlineto{\pgfqpoint{2.653953in}{1.521545in}}%
\pgfpathlineto{\pgfqpoint{2.656763in}{1.521855in}}%
\pgfpathlineto{\pgfqpoint{2.659574in}{1.522228in}}%
\pgfpathlineto{\pgfqpoint{2.662385in}{1.522563in}}%
\pgfpathlineto{\pgfqpoint{2.665195in}{1.522757in}}%
\pgfpathlineto{\pgfqpoint{2.668006in}{1.523127in}}%
\pgfpathlineto{\pgfqpoint{2.670817in}{1.523383in}}%
\pgfpathlineto{\pgfqpoint{2.673628in}{1.523725in}}%
\pgfpathlineto{\pgfqpoint{2.676438in}{1.524011in}}%
\pgfpathlineto{\pgfqpoint{2.679249in}{1.524380in}}%
\pgfpathlineto{\pgfqpoint{2.682060in}{1.524544in}}%
\pgfpathlineto{\pgfqpoint{2.684870in}{1.524534in}}%
\pgfpathlineto{\pgfqpoint{2.687681in}{1.524878in}}%
\pgfpathlineto{\pgfqpoint{2.690492in}{1.525206in}}%
\pgfpathlineto{\pgfqpoint{2.693302in}{1.525478in}}%
\pgfpathlineto{\pgfqpoint{2.696113in}{1.525848in}}%
\pgfpathlineto{\pgfqpoint{2.698924in}{1.526218in}}%
\pgfpathlineto{\pgfqpoint{2.701734in}{1.514801in}}%
\pgfpathlineto{\pgfqpoint{2.704545in}{1.513029in}}%
\pgfpathlineto{\pgfqpoint{2.707356in}{1.513317in}}%
\pgfpathlineto{\pgfqpoint{2.710166in}{1.513680in}}%
\pgfpathlineto{\pgfqpoint{2.712977in}{1.513733in}}%
\pgfpathlineto{\pgfqpoint{2.715788in}{1.513683in}}%
\pgfpathlineto{\pgfqpoint{2.718599in}{1.513684in}}%
\pgfpathlineto{\pgfqpoint{2.721409in}{1.513969in}}%
\pgfpathlineto{\pgfqpoint{2.724220in}{1.514244in}}%
\pgfpathlineto{\pgfqpoint{2.727031in}{1.514490in}}%
\pgfpathlineto{\pgfqpoint{2.729841in}{1.514854in}}%
\pgfpathlineto{\pgfqpoint{2.732652in}{1.514476in}}%
\pgfpathlineto{\pgfqpoint{2.735463in}{1.514612in}}%
\pgfpathlineto{\pgfqpoint{2.738273in}{1.514980in}}%
\pgfpathlineto{\pgfqpoint{2.741084in}{1.512967in}}%
\pgfpathlineto{\pgfqpoint{2.743895in}{1.513266in}}%
\pgfpathlineto{\pgfqpoint{2.746705in}{1.513369in}}%
\pgfpathlineto{\pgfqpoint{2.749516in}{1.513591in}}%
\pgfpathlineto{\pgfqpoint{2.752327in}{1.513914in}}%
\pgfpathlineto{\pgfqpoint{2.755137in}{1.514260in}}%
\pgfpathlineto{\pgfqpoint{2.757948in}{1.513857in}}%
\pgfpathlineto{\pgfqpoint{2.760759in}{1.513979in}}%
\pgfpathlineto{\pgfqpoint{2.763570in}{1.514349in}}%
\pgfpathlineto{\pgfqpoint{2.766380in}{1.514717in}}%
\pgfpathlineto{\pgfqpoint{2.769191in}{1.514937in}}%
\pgfpathlineto{\pgfqpoint{2.772002in}{1.515292in}}%
\pgfpathlineto{\pgfqpoint{2.774812in}{1.515452in}}%
\pgfpathlineto{\pgfqpoint{2.777623in}{1.515521in}}%
\pgfpathlineto{\pgfqpoint{2.780434in}{1.515683in}}%
\pgfpathlineto{\pgfqpoint{2.783244in}{1.515949in}}%
\pgfpathlineto{\pgfqpoint{2.786055in}{1.516213in}}%
\pgfpathlineto{\pgfqpoint{2.788866in}{1.516361in}}%
\pgfpathlineto{\pgfqpoint{2.791676in}{1.516723in}}%
\pgfpathlineto{\pgfqpoint{2.794487in}{1.516586in}}%
\pgfpathlineto{\pgfqpoint{2.797298in}{1.516857in}}%
\pgfpathlineto{\pgfqpoint{2.800108in}{1.517209in}}%
\pgfpathlineto{\pgfqpoint{2.802919in}{1.517531in}}%
\pgfpathlineto{\pgfqpoint{2.805730in}{1.517882in}}%
\pgfpathlineto{\pgfqpoint{2.808540in}{1.517611in}}%
\pgfpathlineto{\pgfqpoint{2.811351in}{1.517941in}}%
\pgfpathlineto{\pgfqpoint{2.814162in}{1.518209in}}%
\pgfpathlineto{\pgfqpoint{2.816973in}{1.517974in}}%
\pgfpathlineto{\pgfqpoint{2.819783in}{1.518176in}}%
\pgfpathlineto{\pgfqpoint{2.822594in}{1.518505in}}%
\pgfpathlineto{\pgfqpoint{2.825405in}{1.518799in}}%
\pgfpathlineto{\pgfqpoint{2.828215in}{1.519147in}}%
\pgfpathlineto{\pgfqpoint{2.831026in}{1.518995in}}%
\pgfpathlineto{\pgfqpoint{2.833837in}{1.519173in}}%
\pgfpathlineto{\pgfqpoint{2.836647in}{1.519124in}}%
\pgfpathlineto{\pgfqpoint{2.839458in}{1.519467in}}%
\pgfpathlineto{\pgfqpoint{2.842269in}{1.518794in}}%
\pgfpathlineto{\pgfqpoint{2.845079in}{1.519123in}}%
\pgfpathlineto{\pgfqpoint{2.847890in}{1.519273in}}%
\pgfpathlineto{\pgfqpoint{2.850701in}{1.519471in}}%
\pgfpathlineto{\pgfqpoint{2.853511in}{1.518218in}}%
\pgfpathlineto{\pgfqpoint{2.856322in}{1.517833in}}%
\pgfpathlineto{\pgfqpoint{2.859133in}{1.518075in}}%
\pgfpathlineto{\pgfqpoint{2.861944in}{1.518383in}}%
\pgfpathlineto{\pgfqpoint{2.864754in}{1.517609in}}%
\pgfpathlineto{\pgfqpoint{2.867565in}{1.517516in}}%
\pgfpathlineto{\pgfqpoint{2.870376in}{1.516724in}}%
\pgfpathlineto{\pgfqpoint{2.873186in}{1.511466in}}%
\pgfpathlineto{\pgfqpoint{2.875997in}{1.511088in}}%
\pgfpathlineto{\pgfqpoint{2.878808in}{1.511123in}}%
\pgfpathlineto{\pgfqpoint{2.881618in}{1.509597in}}%
\pgfpathlineto{\pgfqpoint{2.884429in}{1.509230in}}%
\pgfpathlineto{\pgfqpoint{2.887240in}{1.509418in}}%
\pgfpathlineto{\pgfqpoint{2.890050in}{1.509764in}}%
\pgfpathlineto{\pgfqpoint{2.892861in}{1.509165in}}%
\pgfpathlineto{\pgfqpoint{2.895672in}{1.509029in}}%
\pgfpathlineto{\pgfqpoint{2.898482in}{1.509331in}}%
\pgfpathlineto{\pgfqpoint{2.901293in}{1.509586in}}%
\pgfpathlineto{\pgfqpoint{2.904104in}{1.509864in}}%
\pgfpathlineto{\pgfqpoint{2.906915in}{1.510090in}}%
\pgfpathlineto{\pgfqpoint{2.909725in}{1.510427in}}%
\pgfpathlineto{\pgfqpoint{2.912536in}{1.507287in}}%
\pgfpathlineto{\pgfqpoint{2.915347in}{1.505709in}}%
\pgfpathlineto{\pgfqpoint{2.918157in}{1.506055in}}%
\pgfpathlineto{\pgfqpoint{2.920968in}{1.506401in}}%
\pgfpathlineto{\pgfqpoint{2.923779in}{1.505901in}}%
\pgfpathlineto{\pgfqpoint{2.926589in}{1.506197in}}%
\pgfpathlineto{\pgfqpoint{2.929400in}{1.506220in}}%
\pgfpathlineto{\pgfqpoint{2.932211in}{1.506126in}}%
\pgfpathlineto{\pgfqpoint{2.935021in}{1.506415in}}%
\pgfpathlineto{\pgfqpoint{2.937832in}{1.506715in}}%
\pgfpathlineto{\pgfqpoint{2.940643in}{1.507053in}}%
\pgfpathlineto{\pgfqpoint{2.943453in}{1.507395in}}%
\pgfpathlineto{\pgfqpoint{2.946264in}{1.507735in}}%
\pgfpathlineto{\pgfqpoint{2.949075in}{1.508019in}}%
\pgfpathlineto{\pgfqpoint{2.951886in}{1.507869in}}%
\pgfpathlineto{\pgfqpoint{2.954696in}{1.506038in}}%
\pgfpathlineto{\pgfqpoint{2.957507in}{1.505925in}}%
\pgfpathlineto{\pgfqpoint{2.960318in}{1.506100in}}%
\pgfpathlineto{\pgfqpoint{2.963128in}{1.506393in}}%
\pgfpathlineto{\pgfqpoint{2.965939in}{1.506629in}}%
\pgfpathlineto{\pgfqpoint{2.968750in}{1.506964in}}%
\pgfpathlineto{\pgfqpoint{2.971560in}{1.507162in}}%
\pgfpathlineto{\pgfqpoint{2.974371in}{1.507480in}}%
\pgfpathlineto{\pgfqpoint{2.977182in}{1.507610in}}%
\pgfpathlineto{\pgfqpoint{2.979992in}{1.507876in}}%
\pgfpathlineto{\pgfqpoint{2.982803in}{1.508208in}}%
\pgfpathlineto{\pgfqpoint{2.985614in}{1.507929in}}%
\pgfpathlineto{\pgfqpoint{2.988424in}{1.507893in}}%
\pgfpathlineto{\pgfqpoint{2.991235in}{1.507619in}}%
\pgfpathlineto{\pgfqpoint{2.994046in}{1.507953in}}%
\pgfpathlineto{\pgfqpoint{2.996856in}{1.507676in}}%
\pgfpathlineto{\pgfqpoint{2.999667in}{1.507616in}}%
\pgfpathlineto{\pgfqpoint{3.002478in}{1.507504in}}%
\pgfpathlineto{\pgfqpoint{3.005289in}{1.507650in}}%
\pgfpathlineto{\pgfqpoint{3.008099in}{1.507695in}}%
\pgfpathlineto{\pgfqpoint{3.010910in}{1.507094in}}%
\pgfpathlineto{\pgfqpoint{3.013721in}{1.507092in}}%
\pgfpathlineto{\pgfqpoint{3.016531in}{1.507421in}}%
\pgfpathlineto{\pgfqpoint{3.019342in}{1.507669in}}%
\pgfpathlineto{\pgfqpoint{3.022153in}{1.507970in}}%
\pgfpathlineto{\pgfqpoint{3.024963in}{1.507864in}}%
\pgfpathlineto{\pgfqpoint{3.027774in}{1.507671in}}%
\pgfpathlineto{\pgfqpoint{3.030585in}{1.507401in}}%
\pgfpathlineto{\pgfqpoint{3.033395in}{1.507683in}}%
\pgfpathlineto{\pgfqpoint{3.036206in}{1.507496in}}%
\pgfpathlineto{\pgfqpoint{3.039017in}{1.506817in}}%
\pgfpathlineto{\pgfqpoint{3.041827in}{1.506888in}}%
\pgfpathlineto{\pgfqpoint{3.044638in}{1.507040in}}%
\pgfpathlineto{\pgfqpoint{3.047449in}{1.507321in}}%
\pgfpathlineto{\pgfqpoint{3.050260in}{1.507444in}}%
\pgfpathlineto{\pgfqpoint{3.053070in}{1.507617in}}%
\pgfpathlineto{\pgfqpoint{3.055881in}{1.507750in}}%
\pgfpathlineto{\pgfqpoint{3.058692in}{1.507871in}}%
\pgfpathlineto{\pgfqpoint{3.061502in}{1.508195in}}%
\pgfpathlineto{\pgfqpoint{3.064313in}{1.508077in}}%
\pgfpathlineto{\pgfqpoint{3.067124in}{1.508384in}}%
\pgfpathlineto{\pgfqpoint{3.069934in}{1.508680in}}%
\pgfpathlineto{\pgfqpoint{3.072745in}{1.508434in}}%
\pgfpathlineto{\pgfqpoint{3.075556in}{1.507854in}}%
\pgfpathlineto{\pgfqpoint{3.078366in}{1.504079in}}%
\pgfpathlineto{\pgfqpoint{3.081177in}{1.503864in}}%
\pgfpathlineto{\pgfqpoint{3.083988in}{1.504146in}}%
\pgfpathlineto{\pgfqpoint{3.086798in}{1.501984in}}%
\pgfpathlineto{\pgfqpoint{3.089609in}{1.502147in}}%
\pgfpathlineto{\pgfqpoint{3.092420in}{1.502417in}}%
\pgfpathlineto{\pgfqpoint{3.095231in}{1.502726in}}%
\pgfpathlineto{\pgfqpoint{3.098041in}{1.503038in}}%
\pgfpathlineto{\pgfqpoint{3.100852in}{1.502368in}}%
\pgfpathlineto{\pgfqpoint{3.103663in}{1.502218in}}%
\pgfpathlineto{\pgfqpoint{3.106473in}{1.502211in}}%
\pgfpathlineto{\pgfqpoint{3.109284in}{1.502337in}}%
\pgfpathlineto{\pgfqpoint{3.112095in}{1.502528in}}%
\pgfpathlineto{\pgfqpoint{3.114905in}{1.502846in}}%
\pgfpathlineto{\pgfqpoint{3.117716in}{1.503149in}}%
\pgfpathlineto{\pgfqpoint{3.120527in}{1.502862in}}%
\pgfpathlineto{\pgfqpoint{3.123337in}{1.503056in}}%
\pgfpathlineto{\pgfqpoint{3.126148in}{1.503370in}}%
\pgfpathlineto{\pgfqpoint{3.128959in}{1.502727in}}%
\pgfpathlineto{\pgfqpoint{3.131769in}{1.502862in}}%
\pgfpathlineto{\pgfqpoint{3.134580in}{1.502570in}}%
\pgfpathlineto{\pgfqpoint{3.137391in}{1.502527in}}%
\pgfpathlineto{\pgfqpoint{3.140202in}{1.502333in}}%
\pgfpathlineto{\pgfqpoint{3.143012in}{1.502277in}}%
\pgfpathlineto{\pgfqpoint{3.145823in}{1.502541in}}%
\pgfpathlineto{\pgfqpoint{3.148634in}{1.502786in}}%
\pgfpathlineto{\pgfqpoint{3.151444in}{1.502618in}}%
\pgfpathlineto{\pgfqpoint{3.154255in}{1.502190in}}%
\pgfpathlineto{\pgfqpoint{3.157066in}{1.501854in}}%
\pgfpathlineto{\pgfqpoint{3.159876in}{1.500100in}}%
\pgfpathlineto{\pgfqpoint{3.162687in}{1.500204in}}%
\pgfpathlineto{\pgfqpoint{3.165498in}{1.496323in}}%
\pgfpathlineto{\pgfqpoint{3.168308in}{1.496622in}}%
\pgfpathlineto{\pgfqpoint{3.171119in}{1.496794in}}%
\pgfpathlineto{\pgfqpoint{3.173930in}{1.496802in}}%
\pgfpathlineto{\pgfqpoint{3.176740in}{1.497108in}}%
\pgfpathlineto{\pgfqpoint{3.179551in}{1.497224in}}%
\pgfpathlineto{\pgfqpoint{3.182362in}{1.497378in}}%
\pgfpathlineto{\pgfqpoint{3.185173in}{1.497528in}}%
\pgfpathlineto{\pgfqpoint{3.187983in}{1.497279in}}%
\pgfpathlineto{\pgfqpoint{3.190794in}{1.495788in}}%
\pgfpathlineto{\pgfqpoint{3.193605in}{1.496098in}}%
\pgfpathlineto{\pgfqpoint{3.196415in}{1.492539in}}%
\pgfpathlineto{\pgfqpoint{3.199226in}{1.492061in}}%
\pgfpathlineto{\pgfqpoint{3.202037in}{1.492132in}}%
\pgfpathlineto{\pgfqpoint{3.204847in}{1.491811in}}%
\pgfpathlineto{\pgfqpoint{3.207658in}{1.492051in}}%
\pgfpathlineto{\pgfqpoint{3.210469in}{1.492363in}}%
\pgfpathlineto{\pgfqpoint{3.213279in}{1.492675in}}%
\pgfpathlineto{\pgfqpoint{3.216090in}{1.492939in}}%
\pgfpathlineto{\pgfqpoint{3.218901in}{1.493129in}}%
\pgfpathlineto{\pgfqpoint{3.221711in}{1.493170in}}%
\pgfpathlineto{\pgfqpoint{3.224522in}{1.493230in}}%
\pgfpathlineto{\pgfqpoint{3.227333in}{1.493249in}}%
\pgfpathlineto{\pgfqpoint{3.230143in}{1.492064in}}%
\pgfpathlineto{\pgfqpoint{3.232954in}{1.492319in}}%
\pgfpathlineto{\pgfqpoint{3.235765in}{1.492132in}}%
\pgfpathlineto{\pgfqpoint{3.238576in}{1.492440in}}%
\pgfpathlineto{\pgfqpoint{3.241386in}{1.492397in}}%
\pgfpathlineto{\pgfqpoint{3.244197in}{1.492705in}}%
\pgfpathlineto{\pgfqpoint{3.247008in}{1.492599in}}%
\pgfpathlineto{\pgfqpoint{3.249818in}{1.492582in}}%
\pgfpathlineto{\pgfqpoint{3.252629in}{1.492850in}}%
\pgfpathlineto{\pgfqpoint{3.255440in}{1.493067in}}%
\pgfpathlineto{\pgfqpoint{3.258250in}{1.493199in}}%
\pgfpathlineto{\pgfqpoint{3.261061in}{1.493097in}}%
\pgfpathlineto{\pgfqpoint{3.263872in}{1.492948in}}%
\pgfpathlineto{\pgfqpoint{3.266682in}{1.493187in}}%
\pgfpathlineto{\pgfqpoint{3.269493in}{1.492744in}}%
\pgfpathlineto{\pgfqpoint{3.272304in}{1.492525in}}%
\pgfpathlineto{\pgfqpoint{3.275114in}{1.492764in}}%
\pgfpathlineto{\pgfqpoint{3.277925in}{1.492946in}}%
\pgfpathlineto{\pgfqpoint{3.280736in}{1.493098in}}%
\pgfpathlineto{\pgfqpoint{3.283547in}{1.492843in}}%
\pgfpathlineto{\pgfqpoint{3.286357in}{1.493118in}}%
\pgfpathlineto{\pgfqpoint{3.289168in}{1.493413in}}%
\pgfpathlineto{\pgfqpoint{3.291979in}{1.493655in}}%
\pgfpathlineto{\pgfqpoint{3.294789in}{1.493924in}}%
\pgfpathlineto{\pgfqpoint{3.297600in}{1.493931in}}%
\pgfpathlineto{\pgfqpoint{3.300411in}{1.494230in}}%
\pgfpathlineto{\pgfqpoint{3.303221in}{1.494524in}}%
\pgfpathlineto{\pgfqpoint{3.306032in}{1.494683in}}%
\pgfpathlineto{\pgfqpoint{3.308843in}{1.494577in}}%
\pgfpathlineto{\pgfqpoint{3.311653in}{1.494733in}}%
\pgfpathlineto{\pgfqpoint{3.314464in}{1.495029in}}%
\pgfpathlineto{\pgfqpoint{3.317275in}{1.495287in}}%
\pgfpathlineto{\pgfqpoint{3.320085in}{1.494906in}}%
\pgfpathlineto{\pgfqpoint{3.322896in}{1.494322in}}%
\pgfpathlineto{\pgfqpoint{3.325707in}{1.493912in}}%
\pgfpathlineto{\pgfqpoint{3.328518in}{1.493698in}}%
\pgfpathlineto{\pgfqpoint{3.331328in}{1.493384in}}%
\pgfpathlineto{\pgfqpoint{3.334139in}{1.492934in}}%
\pgfpathlineto{\pgfqpoint{3.336950in}{1.493132in}}%
\pgfpathlineto{\pgfqpoint{3.339760in}{1.493294in}}%
\pgfpathlineto{\pgfqpoint{3.342571in}{1.493537in}}%
\pgfpathlineto{\pgfqpoint{3.345382in}{1.493111in}}%
\pgfpathlineto{\pgfqpoint{3.348192in}{1.493371in}}%
\pgfpathlineto{\pgfqpoint{3.351003in}{1.493664in}}%
\pgfpathlineto{\pgfqpoint{3.353814in}{1.493949in}}%
\pgfpathlineto{\pgfqpoint{3.356624in}{1.494200in}}%
\pgfpathlineto{\pgfqpoint{3.359435in}{1.494128in}}%
\pgfpathlineto{\pgfqpoint{3.362246in}{1.494416in}}%
\pgfpathlineto{\pgfqpoint{3.365056in}{1.494510in}}%
\pgfpathlineto{\pgfqpoint{3.367867in}{1.494592in}}%
\pgfpathlineto{\pgfqpoint{3.370678in}{1.494852in}}%
\pgfpathlineto{\pgfqpoint{3.373489in}{1.493368in}}%
\pgfpathlineto{\pgfqpoint{3.376299in}{1.492710in}}%
\pgfpathlineto{\pgfqpoint{3.379110in}{1.492951in}}%
\pgfpathlineto{\pgfqpoint{3.381921in}{1.492961in}}%
\pgfpathlineto{\pgfqpoint{3.384731in}{1.493221in}}%
\pgfpathlineto{\pgfqpoint{3.387542in}{1.492785in}}%
\pgfpathlineto{\pgfqpoint{3.390353in}{1.492903in}}%
\pgfpathlineto{\pgfqpoint{3.393163in}{1.492657in}}%
\pgfpathlineto{\pgfqpoint{3.395974in}{1.492091in}}%
\pgfpathlineto{\pgfqpoint{3.398785in}{1.492378in}}%
\pgfpathlineto{\pgfqpoint{3.401595in}{1.492499in}}%
\pgfpathlineto{\pgfqpoint{3.404406in}{1.492665in}}%
\pgfpathlineto{\pgfqpoint{3.407217in}{1.492905in}}%
\pgfpathlineto{\pgfqpoint{3.410027in}{1.493190in}}%
\pgfpathlineto{\pgfqpoint{3.412838in}{1.492908in}}%
\pgfpathlineto{\pgfqpoint{3.415649in}{1.492713in}}%
\pgfpathlineto{\pgfqpoint{3.418459in}{1.492491in}}%
\pgfpathlineto{\pgfqpoint{3.421270in}{1.492625in}}%
\pgfpathlineto{\pgfqpoint{3.424081in}{1.492838in}}%
\pgfpathlineto{\pgfqpoint{3.426892in}{1.492317in}}%
\pgfpathlineto{\pgfqpoint{3.429702in}{1.492591in}}%
\pgfpathlineto{\pgfqpoint{3.432513in}{1.492645in}}%
\pgfpathlineto{\pgfqpoint{3.435324in}{1.492893in}}%
\pgfpathlineto{\pgfqpoint{3.438134in}{1.493036in}}%
\pgfpathlineto{\pgfqpoint{3.440945in}{1.493273in}}%
\pgfpathlineto{\pgfqpoint{3.443756in}{1.493537in}}%
\pgfpathlineto{\pgfqpoint{3.446566in}{1.493793in}}%
\pgfpathlineto{\pgfqpoint{3.449377in}{1.494073in}}%
\pgfpathlineto{\pgfqpoint{3.452188in}{1.494205in}}%
\pgfpathlineto{\pgfqpoint{3.454998in}{1.494327in}}%
\pgfpathlineto{\pgfqpoint{3.457809in}{1.494197in}}%
\pgfpathlineto{\pgfqpoint{3.460620in}{1.493402in}}%
\pgfpathlineto{\pgfqpoint{3.463430in}{1.493584in}}%
\pgfpathlineto{\pgfqpoint{3.466241in}{1.492991in}}%
\pgfpathlineto{\pgfqpoint{3.469052in}{1.493264in}}%
\pgfpathlineto{\pgfqpoint{3.471863in}{1.493508in}}%
\pgfpathlineto{\pgfqpoint{3.474673in}{1.493585in}}%
\pgfpathlineto{\pgfqpoint{3.477484in}{1.493046in}}%
\pgfpathlineto{\pgfqpoint{3.480295in}{1.491737in}}%
\pgfpathlineto{\pgfqpoint{3.483105in}{1.489845in}}%
\pgfpathlineto{\pgfqpoint{3.485916in}{1.489700in}}%
\pgfpathlineto{\pgfqpoint{3.488727in}{1.489427in}}%
\pgfpathlineto{\pgfqpoint{3.491537in}{1.487213in}}%
\pgfpathlineto{\pgfqpoint{3.494348in}{1.487418in}}%
\pgfpathlineto{\pgfqpoint{3.497159in}{1.487355in}}%
\pgfpathlineto{\pgfqpoint{3.499969in}{1.487623in}}%
\pgfpathlineto{\pgfqpoint{3.502780in}{1.486677in}}%
\pgfpathlineto{\pgfqpoint{3.505591in}{1.486487in}}%
\pgfpathlineto{\pgfqpoint{3.508401in}{1.486677in}}%
\pgfpathlineto{\pgfqpoint{3.511212in}{1.486139in}}%
\pgfpathlineto{\pgfqpoint{3.514023in}{1.485627in}}%
\pgfpathlineto{\pgfqpoint{3.516834in}{1.485609in}}%
\pgfpathlineto{\pgfqpoint{3.519644in}{1.485836in}}%
\pgfpathlineto{\pgfqpoint{3.522455in}{1.486114in}}%
\pgfpathlineto{\pgfqpoint{3.525266in}{1.486276in}}%
\pgfpathlineto{\pgfqpoint{3.528076in}{1.486533in}}%
\pgfpathlineto{\pgfqpoint{3.530887in}{1.486765in}}%
\pgfpathlineto{\pgfqpoint{3.533698in}{1.486943in}}%
\pgfpathlineto{\pgfqpoint{3.536508in}{1.486137in}}%
\pgfpathlineto{\pgfqpoint{3.539319in}{1.486406in}}%
\pgfpathlineto{\pgfqpoint{3.542130in}{1.486013in}}%
\pgfpathlineto{\pgfqpoint{3.544940in}{1.486259in}}%
\pgfpathlineto{\pgfqpoint{3.547751in}{1.486233in}}%
\pgfpathlineto{\pgfqpoint{3.550562in}{1.486490in}}%
\pgfpathlineto{\pgfqpoint{3.553372in}{1.486675in}}%
\pgfpathlineto{\pgfqpoint{3.556183in}{1.486855in}}%
\pgfpathlineto{\pgfqpoint{3.558994in}{1.486558in}}%
\pgfpathlineto{\pgfqpoint{3.561805in}{1.486674in}}%
\pgfpathlineto{\pgfqpoint{3.564615in}{1.486854in}}%
\pgfpathlineto{\pgfqpoint{3.567426in}{1.486868in}}%
\pgfpathlineto{\pgfqpoint{3.570237in}{1.487049in}}%
\pgfpathlineto{\pgfqpoint{3.573047in}{1.487198in}}%
\pgfpathlineto{\pgfqpoint{3.575858in}{1.487468in}}%
\pgfpathlineto{\pgfqpoint{3.578669in}{1.487329in}}%
\pgfpathlineto{\pgfqpoint{3.581479in}{1.487568in}}%
\pgfpathlineto{\pgfqpoint{3.584290in}{1.487659in}}%
\pgfpathlineto{\pgfqpoint{3.587101in}{1.487653in}}%
\pgfpathlineto{\pgfqpoint{3.589911in}{1.487836in}}%
\pgfpathlineto{\pgfqpoint{3.592722in}{1.488105in}}%
\pgfpathlineto{\pgfqpoint{3.595533in}{1.488289in}}%
\pgfpathlineto{\pgfqpoint{3.598343in}{1.488395in}}%
\pgfpathlineto{\pgfqpoint{3.601154in}{1.488563in}}%
\pgfpathlineto{\pgfqpoint{3.603965in}{1.488267in}}%
\pgfpathlineto{\pgfqpoint{3.606776in}{1.488485in}}%
\pgfpathlineto{\pgfqpoint{3.609586in}{1.488452in}}%
\pgfpathlineto{\pgfqpoint{3.612397in}{1.488592in}}%
\pgfpathlineto{\pgfqpoint{3.615208in}{1.488860in}}%
\pgfpathlineto{\pgfqpoint{3.618018in}{1.488308in}}%
\pgfpathlineto{\pgfqpoint{3.620829in}{1.488405in}}%
\pgfpathlineto{\pgfqpoint{3.623640in}{1.488672in}}%
\pgfpathlineto{\pgfqpoint{3.626450in}{1.488936in}}%
\pgfpathlineto{\pgfqpoint{3.629261in}{1.489175in}}%
\pgfpathlineto{\pgfqpoint{3.632072in}{1.489281in}}%
\pgfpathlineto{\pgfqpoint{3.634882in}{1.489386in}}%
\pgfpathlineto{\pgfqpoint{3.637693in}{1.489106in}}%
\pgfpathlineto{\pgfqpoint{3.640504in}{1.489227in}}%
\pgfpathlineto{\pgfqpoint{3.643314in}{1.489129in}}%
\pgfpathlineto{\pgfqpoint{3.646125in}{1.488895in}}%
\pgfpathlineto{\pgfqpoint{3.648936in}{1.488867in}}%
\pgfpathlineto{\pgfqpoint{3.651746in}{1.489064in}}%
\pgfpathlineto{\pgfqpoint{3.654557in}{1.489241in}}%
\pgfpathlineto{\pgfqpoint{3.657368in}{1.489430in}}%
\pgfpathlineto{\pgfqpoint{3.660179in}{1.488951in}}%
\pgfpathlineto{\pgfqpoint{3.662989in}{1.489206in}}%
\pgfpathlineto{\pgfqpoint{3.665800in}{1.489295in}}%
\pgfpathlineto{\pgfqpoint{3.668611in}{1.489425in}}%
\pgfpathlineto{\pgfqpoint{3.671421in}{1.489663in}}%
\pgfpathlineto{\pgfqpoint{3.674232in}{1.489864in}}%
\pgfpathlineto{\pgfqpoint{3.677043in}{1.490123in}}%
\pgfpathlineto{\pgfqpoint{3.679853in}{1.490381in}}%
\pgfpathlineto{\pgfqpoint{3.682664in}{1.490405in}}%
\pgfpathlineto{\pgfqpoint{3.685475in}{1.489677in}}%
\pgfpathlineto{\pgfqpoint{3.688285in}{1.489575in}}%
\pgfpathlineto{\pgfqpoint{3.691096in}{1.489793in}}%
\pgfpathlineto{\pgfqpoint{3.693907in}{1.489836in}}%
\pgfpathlineto{\pgfqpoint{3.696717in}{1.490058in}}%
\pgfpathlineto{\pgfqpoint{3.699528in}{1.490233in}}%
\pgfpathlineto{\pgfqpoint{3.702339in}{1.489977in}}%
\pgfpathlineto{\pgfqpoint{3.705150in}{1.490233in}}%
\pgfpathlineto{\pgfqpoint{3.707960in}{1.490318in}}%
\pgfpathlineto{\pgfqpoint{3.710771in}{1.490558in}}%
\pgfpathlineto{\pgfqpoint{3.713582in}{1.490448in}}%
\pgfpathlineto{\pgfqpoint{3.716392in}{1.489192in}}%
\pgfpathlineto{\pgfqpoint{3.719203in}{1.489436in}}%
\pgfpathlineto{\pgfqpoint{3.722014in}{1.489648in}}%
\pgfpathlineto{\pgfqpoint{3.724824in}{1.489906in}}%
\pgfpathlineto{\pgfqpoint{3.727635in}{1.490091in}}%
\pgfpathlineto{\pgfqpoint{3.730446in}{1.490265in}}%
\pgfpathlineto{\pgfqpoint{3.733256in}{1.490471in}}%
\pgfpathlineto{\pgfqpoint{3.736067in}{1.490287in}}%
\pgfpathlineto{\pgfqpoint{3.738878in}{1.489993in}}%
\pgfpathlineto{\pgfqpoint{3.741688in}{1.489779in}}%
\pgfpathlineto{\pgfqpoint{3.744499in}{1.489880in}}%
\pgfpathlineto{\pgfqpoint{3.747310in}{1.489191in}}%
\pgfpathlineto{\pgfqpoint{3.750121in}{1.487264in}}%
\pgfpathlineto{\pgfqpoint{3.752931in}{1.487195in}}%
\pgfpathlineto{\pgfqpoint{3.755742in}{1.487355in}}%
\pgfpathlineto{\pgfqpoint{3.758553in}{1.487469in}}%
\pgfpathlineto{\pgfqpoint{3.761363in}{1.486716in}}%
\pgfpathlineto{\pgfqpoint{3.764174in}{1.486642in}}%
\pgfpathlineto{\pgfqpoint{3.766985in}{1.476138in}}%
\pgfpathlineto{\pgfqpoint{3.769795in}{1.476360in}}%
\pgfpathlineto{\pgfqpoint{3.772606in}{1.476402in}}%
\pgfpathlineto{\pgfqpoint{3.775417in}{1.476632in}}%
\pgfpathlineto{\pgfqpoint{3.778227in}{1.476890in}}%
\pgfpathlineto{\pgfqpoint{3.781038in}{1.476807in}}%
\pgfpathlineto{\pgfqpoint{3.783849in}{1.477053in}}%
\pgfpathlineto{\pgfqpoint{3.786659in}{1.477174in}}%
\pgfpathlineto{\pgfqpoint{3.789470in}{1.477425in}}%
\pgfpathlineto{\pgfqpoint{3.792281in}{1.476958in}}%
\pgfpathlineto{\pgfqpoint{3.795092in}{1.477017in}}%
\pgfpathlineto{\pgfqpoint{3.797902in}{1.475574in}}%
\pgfpathlineto{\pgfqpoint{3.800713in}{1.475616in}}%
\pgfpathlineto{\pgfqpoint{3.803524in}{1.475821in}}%
\pgfpathlineto{\pgfqpoint{3.806334in}{1.475019in}}%
\pgfpathlineto{\pgfqpoint{3.809145in}{1.475046in}}%
\pgfpathlineto{\pgfqpoint{3.811956in}{1.475240in}}%
\pgfpathlineto{\pgfqpoint{3.814766in}{1.474740in}}%
\pgfpathlineto{\pgfqpoint{3.817577in}{1.474933in}}%
\pgfpathlineto{\pgfqpoint{3.820388in}{1.475139in}}%
\pgfpathlineto{\pgfqpoint{3.823198in}{1.475388in}}%
\pgfpathlineto{\pgfqpoint{3.826009in}{1.475397in}}%
\pgfpathlineto{\pgfqpoint{3.828820in}{1.475570in}}%
\pgfpathlineto{\pgfqpoint{3.831630in}{1.474805in}}%
\pgfpathlineto{\pgfqpoint{3.834441in}{1.474858in}}%
\pgfpathlineto{\pgfqpoint{3.837252in}{1.474422in}}%
\pgfpathlineto{\pgfqpoint{3.840062in}{1.474641in}}%
\pgfpathlineto{\pgfqpoint{3.842873in}{1.474848in}}%
\pgfpathlineto{\pgfqpoint{3.845684in}{1.475099in}}%
\pgfpathlineto{\pgfqpoint{3.848495in}{1.475142in}}%
\pgfpathlineto{\pgfqpoint{3.851305in}{1.475066in}}%
\pgfpathlineto{\pgfqpoint{3.854116in}{1.475316in}}%
\pgfpathlineto{\pgfqpoint{3.856927in}{1.475527in}}%
\pgfpathlineto{\pgfqpoint{3.859737in}{1.475746in}}%
\pgfpathlineto{\pgfqpoint{3.862548in}{1.475993in}}%
\pgfpathlineto{\pgfqpoint{3.865359in}{1.475849in}}%
\pgfpathlineto{\pgfqpoint{3.868169in}{1.476093in}}%
\pgfpathlineto{\pgfqpoint{3.870980in}{1.476326in}}%
\pgfpathlineto{\pgfqpoint{3.873791in}{1.476509in}}%
\pgfpathlineto{\pgfqpoint{3.876601in}{1.476442in}}%
\pgfpathlineto{\pgfqpoint{3.879412in}{1.476692in}}%
\pgfpathlineto{\pgfqpoint{3.882223in}{1.476904in}}%
\pgfpathlineto{\pgfqpoint{3.885033in}{1.477152in}}%
\pgfpathlineto{\pgfqpoint{3.887844in}{1.477211in}}%
\pgfpathlineto{\pgfqpoint{3.890655in}{1.477141in}}%
\pgfpathlineto{\pgfqpoint{3.893466in}{1.477322in}}%
\pgfpathlineto{\pgfqpoint{3.896276in}{1.477272in}}%
\pgfpathlineto{\pgfqpoint{3.899087in}{1.477394in}}%
\pgfpathlineto{\pgfqpoint{3.901898in}{1.477595in}}%
\pgfpathlineto{\pgfqpoint{3.904708in}{1.477791in}}%
\pgfpathlineto{\pgfqpoint{3.907519in}{1.478033in}}%
\pgfpathlineto{\pgfqpoint{3.910330in}{1.477939in}}%
\pgfpathlineto{\pgfqpoint{3.913140in}{1.478166in}}%
\pgfpathlineto{\pgfqpoint{3.915951in}{1.477962in}}%
\pgfpathlineto{\pgfqpoint{3.918762in}{1.478097in}}%
\pgfpathlineto{\pgfqpoint{3.921572in}{1.478339in}}%
\pgfpathlineto{\pgfqpoint{3.924383in}{1.478022in}}%
\pgfpathlineto{\pgfqpoint{3.927194in}{1.478245in}}%
\pgfpathlineto{\pgfqpoint{3.930004in}{1.478449in}}%
\pgfpathlineto{\pgfqpoint{3.932815in}{1.478691in}}%
\pgfpathlineto{\pgfqpoint{3.935626in}{1.478935in}}%
\pgfpathlineto{\pgfqpoint{3.938437in}{1.478871in}}%
\pgfpathlineto{\pgfqpoint{3.941247in}{1.478791in}}%
\pgfpathlineto{\pgfqpoint{3.944058in}{1.479032in}}%
\pgfpathlineto{\pgfqpoint{3.946869in}{1.479198in}}%
\pgfpathlineto{\pgfqpoint{3.949679in}{1.479419in}}%
\pgfpathlineto{\pgfqpoint{3.952490in}{1.479591in}}%
\pgfpathlineto{\pgfqpoint{3.955301in}{1.479525in}}%
\pgfpathlineto{\pgfqpoint{3.958111in}{1.479545in}}%
\pgfpathlineto{\pgfqpoint{3.960922in}{1.479737in}}%
\pgfpathlineto{\pgfqpoint{3.963733in}{1.479969in}}%
\pgfpathlineto{\pgfqpoint{3.966543in}{1.479472in}}%
\pgfpathlineto{\pgfqpoint{3.969354in}{1.478589in}}%
\pgfpathlineto{\pgfqpoint{3.972165in}{1.478821in}}%
\pgfpathlineto{\pgfqpoint{3.974975in}{1.478824in}}%
\pgfpathlineto{\pgfqpoint{3.977786in}{1.478824in}}%
\pgfpathlineto{\pgfqpoint{3.980597in}{1.479034in}}%
\pgfpathlineto{\pgfqpoint{3.983408in}{1.479266in}}%
\pgfpathlineto{\pgfqpoint{3.986218in}{1.479277in}}%
\pgfpathlineto{\pgfqpoint{3.989029in}{1.479504in}}%
\pgfpathlineto{\pgfqpoint{3.991840in}{1.479640in}}%
\pgfpathlineto{\pgfqpoint{3.994650in}{1.479594in}}%
\pgfpathlineto{\pgfqpoint{3.997461in}{1.479827in}}%
\pgfpathlineto{\pgfqpoint{4.000272in}{1.480000in}}%
\pgfpathlineto{\pgfqpoint{4.003082in}{1.479813in}}%
\pgfpathlineto{\pgfqpoint{4.005893in}{1.480003in}}%
\pgfpathlineto{\pgfqpoint{4.008704in}{1.479879in}}%
\pgfpathlineto{\pgfqpoint{4.011514in}{1.480024in}}%
\pgfpathlineto{\pgfqpoint{4.014325in}{1.480238in}}%
\pgfpathlineto{\pgfqpoint{4.017136in}{1.480121in}}%
\pgfpathlineto{\pgfqpoint{4.019946in}{1.480350in}}%
\pgfpathlineto{\pgfqpoint{4.022757in}{1.480571in}}%
\pgfpathlineto{\pgfqpoint{4.025568in}{1.480782in}}%
\pgfpathlineto{\pgfqpoint{4.028378in}{1.480974in}}%
\pgfpathlineto{\pgfqpoint{4.031189in}{1.481181in}}%
\pgfpathlineto{\pgfqpoint{4.034000in}{1.481398in}}%
\pgfpathlineto{\pgfqpoint{4.036811in}{1.481498in}}%
\pgfpathlineto{\pgfqpoint{4.039621in}{1.481704in}}%
\pgfpathlineto{\pgfqpoint{4.042432in}{1.481937in}}%
\pgfpathlineto{\pgfqpoint{4.045243in}{1.482122in}}%
\pgfpathlineto{\pgfqpoint{4.048053in}{1.482324in}}%
\pgfpathlineto{\pgfqpoint{4.050864in}{1.482539in}}%
\pgfpathlineto{\pgfqpoint{4.053675in}{1.482761in}}%
\pgfpathlineto{\pgfqpoint{4.056485in}{1.482925in}}%
\pgfpathlineto{\pgfqpoint{4.059296in}{1.482625in}}%
\pgfpathlineto{\pgfqpoint{4.062107in}{1.482833in}}%
\pgfpathlineto{\pgfqpoint{4.064917in}{1.483037in}}%
\pgfpathlineto{\pgfqpoint{4.067728in}{1.483242in}}%
\pgfpathlineto{\pgfqpoint{4.070539in}{1.483466in}}%
\pgfpathlineto{\pgfqpoint{4.073349in}{1.483629in}}%
\pgfpathlineto{\pgfqpoint{4.076160in}{1.483685in}}%
\pgfpathlineto{\pgfqpoint{4.078971in}{1.481410in}}%
\pgfpathlineto{\pgfqpoint{4.081782in}{1.480571in}}%
\pgfpathlineto{\pgfqpoint{4.084592in}{1.480750in}}%
\pgfpathlineto{\pgfqpoint{4.087403in}{1.480751in}}%
\pgfpathlineto{\pgfqpoint{4.090214in}{1.480642in}}%
\pgfpathlineto{\pgfqpoint{4.093024in}{1.480795in}}%
\pgfpathlineto{\pgfqpoint{4.095835in}{1.480941in}}%
\pgfpathlineto{\pgfqpoint{4.098646in}{1.481171in}}%
\pgfpathlineto{\pgfqpoint{4.101456in}{1.481402in}}%
\pgfpathlineto{\pgfqpoint{4.104267in}{1.481389in}}%
\pgfpathlineto{\pgfqpoint{4.107078in}{1.481610in}}%
\pgfpathlineto{\pgfqpoint{4.109888in}{1.481772in}}%
\pgfpathlineto{\pgfqpoint{4.112699in}{1.481971in}}%
\pgfpathlineto{\pgfqpoint{4.115510in}{1.482197in}}%
\pgfpathlineto{\pgfqpoint{4.118320in}{1.482309in}}%
\pgfpathlineto{\pgfqpoint{4.121131in}{1.482475in}}%
\pgfpathlineto{\pgfqpoint{4.123942in}{1.482683in}}%
\pgfpathlineto{\pgfqpoint{4.126753in}{1.482857in}}%
\pgfpathlineto{\pgfqpoint{4.129563in}{1.480963in}}%
\pgfpathlineto{\pgfqpoint{4.132374in}{1.481180in}}%
\pgfpathlineto{\pgfqpoint{4.135185in}{1.481368in}}%
\pgfpathlineto{\pgfqpoint{4.137995in}{1.481582in}}%
\pgfpathlineto{\pgfqpoint{4.140806in}{1.481604in}}%
\pgfpathlineto{\pgfqpoint{4.143617in}{1.481750in}}%
\pgfpathlineto{\pgfqpoint{4.146427in}{1.481953in}}%
\pgfpathlineto{\pgfqpoint{4.149238in}{1.482128in}}%
\pgfpathlineto{\pgfqpoint{4.152049in}{1.482121in}}%
\pgfpathlineto{\pgfqpoint{4.154859in}{1.482271in}}%
\pgfpathlineto{\pgfqpoint{4.157670in}{1.482494in}}%
\pgfpathlineto{\pgfqpoint{4.160481in}{1.482707in}}%
\pgfpathlineto{\pgfqpoint{4.163291in}{1.482899in}}%
\pgfpathlineto{\pgfqpoint{4.166102in}{1.483015in}}%
\pgfpathlineto{\pgfqpoint{4.168913in}{1.482922in}}%
\pgfpathlineto{\pgfqpoint{4.171724in}{1.483138in}}%
\pgfpathlineto{\pgfqpoint{4.174534in}{1.483259in}}%
\pgfpathlineto{\pgfqpoint{4.177345in}{1.483480in}}%
\pgfpathlineto{\pgfqpoint{4.180156in}{1.483703in}}%
\pgfpathlineto{\pgfqpoint{4.182966in}{1.483772in}}%
\pgfpathlineto{\pgfqpoint{4.185777in}{1.483926in}}%
\pgfpathlineto{\pgfqpoint{4.188588in}{1.484149in}}%
\pgfpathlineto{\pgfqpoint{4.191398in}{1.484356in}}%
\pgfpathlineto{\pgfqpoint{4.194209in}{1.484542in}}%
\pgfpathlineto{\pgfqpoint{4.197020in}{1.484573in}}%
\pgfpathlineto{\pgfqpoint{4.199830in}{1.484715in}}%
\pgfpathlineto{\pgfqpoint{4.202641in}{1.484931in}}%
\pgfpathlineto{\pgfqpoint{4.205452in}{1.485153in}}%
\pgfpathlineto{\pgfqpoint{4.208262in}{1.485364in}}%
\pgfpathlineto{\pgfqpoint{4.211073in}{1.485581in}}%
\pgfpathlineto{\pgfqpoint{4.213884in}{1.485788in}}%
\pgfpathlineto{\pgfqpoint{4.216695in}{1.485975in}}%
\pgfpathlineto{\pgfqpoint{4.219505in}{1.486163in}}%
\pgfpathlineto{\pgfqpoint{4.222316in}{1.486278in}}%
\pgfpathlineto{\pgfqpoint{4.225127in}{1.486442in}}%
\pgfpathlineto{\pgfqpoint{4.227937in}{1.485541in}}%
\pgfpathlineto{\pgfqpoint{4.230748in}{1.485669in}}%
\pgfpathlineto{\pgfqpoint{4.233559in}{1.485514in}}%
\pgfpathlineto{\pgfqpoint{4.236369in}{1.485686in}}%
\pgfpathlineto{\pgfqpoint{4.239180in}{1.485606in}}%
\pgfpathlineto{\pgfqpoint{4.241991in}{1.485398in}}%
\pgfpathlineto{\pgfqpoint{4.244801in}{1.485221in}}%
\pgfpathlineto{\pgfqpoint{4.247612in}{1.485380in}}%
\pgfpathlineto{\pgfqpoint{4.250423in}{1.485599in}}%
\pgfpathlineto{\pgfqpoint{4.253233in}{1.485795in}}%
\pgfpathlineto{\pgfqpoint{4.256044in}{1.485754in}}%
\pgfpathlineto{\pgfqpoint{4.258855in}{1.485537in}}%
\pgfpathlineto{\pgfqpoint{4.261665in}{1.485716in}}%
\pgfpathlineto{\pgfqpoint{4.264476in}{1.485933in}}%
\pgfpathlineto{\pgfqpoint{4.267287in}{1.486048in}}%
\pgfpathlineto{\pgfqpoint{4.270098in}{1.486252in}}%
\pgfpathlineto{\pgfqpoint{4.272908in}{1.486378in}}%
\pgfpathlineto{\pgfqpoint{4.275719in}{1.486492in}}%
\pgfpathlineto{\pgfqpoint{4.278530in}{1.486692in}}%
\pgfpathlineto{\pgfqpoint{4.281340in}{1.486881in}}%
\pgfpathlineto{\pgfqpoint{4.284151in}{1.487056in}}%
\pgfpathlineto{\pgfqpoint{4.286962in}{1.487191in}}%
\pgfpathlineto{\pgfqpoint{4.289772in}{1.486737in}}%
\pgfpathlineto{\pgfqpoint{4.292583in}{1.486839in}}%
\pgfpathlineto{\pgfqpoint{4.295394in}{1.486935in}}%
\pgfpathlineto{\pgfqpoint{4.298204in}{1.487127in}}%
\pgfpathlineto{\pgfqpoint{4.301015in}{1.487218in}}%
\pgfpathlineto{\pgfqpoint{4.303826in}{1.487412in}}%
\pgfpathlineto{\pgfqpoint{4.306636in}{1.483549in}}%
\pgfpathlineto{\pgfqpoint{4.309447in}{1.483677in}}%
\pgfpathlineto{\pgfqpoint{4.312258in}{1.483692in}}%
\pgfpathlineto{\pgfqpoint{4.315069in}{1.483889in}}%
\pgfpathlineto{\pgfqpoint{4.317879in}{1.483976in}}%
\pgfpathlineto{\pgfqpoint{4.320690in}{1.484049in}}%
\pgfpathlineto{\pgfqpoint{4.323501in}{1.484165in}}%
\pgfpathlineto{\pgfqpoint{4.326311in}{1.484302in}}%
\pgfpathlineto{\pgfqpoint{4.329122in}{1.484503in}}%
\pgfpathlineto{\pgfqpoint{4.331933in}{1.484443in}}%
\pgfpathlineto{\pgfqpoint{4.334743in}{1.484634in}}%
\pgfpathlineto{\pgfqpoint{4.337554in}{1.484521in}}%
\pgfpathlineto{\pgfqpoint{4.340365in}{1.484484in}}%
\pgfpathlineto{\pgfqpoint{4.343175in}{1.484213in}}%
\pgfpathlineto{\pgfqpoint{4.345986in}{1.484394in}}%
\pgfpathlineto{\pgfqpoint{4.348797in}{1.484562in}}%
\pgfpathlineto{\pgfqpoint{4.351607in}{1.484589in}}%
\pgfpathlineto{\pgfqpoint{4.354418in}{1.484784in}}%
\pgfpathlineto{\pgfqpoint{4.357229in}{1.484887in}}%
\pgfpathlineto{\pgfqpoint{4.360040in}{1.485080in}}%
\pgfpathlineto{\pgfqpoint{4.362850in}{1.485210in}}%
\pgfpathlineto{\pgfqpoint{4.365661in}{1.485417in}}%
\pgfpathlineto{\pgfqpoint{4.368472in}{1.485547in}}%
\pgfpathlineto{\pgfqpoint{4.371282in}{1.485720in}}%
\pgfpathlineto{\pgfqpoint{4.374093in}{1.485892in}}%
\pgfpathlineto{\pgfqpoint{4.376904in}{1.485899in}}%
\pgfpathlineto{\pgfqpoint{4.379714in}{1.486109in}}%
\pgfpathlineto{\pgfqpoint{4.382525in}{1.486293in}}%
\pgfpathlineto{\pgfqpoint{4.385336in}{1.486352in}}%
\pgfpathlineto{\pgfqpoint{4.388146in}{1.486031in}}%
\pgfpathlineto{\pgfqpoint{4.390957in}{1.485216in}}%
\pgfpathlineto{\pgfqpoint{4.393768in}{1.485411in}}%
\pgfpathlineto{\pgfqpoint{4.396578in}{1.485619in}}%
\pgfpathlineto{\pgfqpoint{4.399389in}{1.485825in}}%
\pgfpathlineto{\pgfqpoint{4.402200in}{1.485851in}}%
\pgfpathlineto{\pgfqpoint{4.405011in}{1.486055in}}%
\pgfpathlineto{\pgfqpoint{4.407821in}{1.486235in}}%
\pgfpathlineto{\pgfqpoint{4.410632in}{1.486440in}}%
\pgfpathlineto{\pgfqpoint{4.413443in}{1.486449in}}%
\pgfpathlineto{\pgfqpoint{4.416253in}{1.486478in}}%
\pgfpathlineto{\pgfqpoint{4.419064in}{1.486685in}}%
\pgfpathlineto{\pgfqpoint{4.421875in}{1.486529in}}%
\pgfpathlineto{\pgfqpoint{4.424685in}{1.486680in}}%
\pgfpathlineto{\pgfqpoint{4.427496in}{1.486886in}}%
\pgfpathlineto{\pgfqpoint{4.430307in}{1.486929in}}%
\pgfpathlineto{\pgfqpoint{4.433117in}{1.487072in}}%
\pgfpathlineto{\pgfqpoint{4.435928in}{1.487240in}}%
\pgfpathlineto{\pgfqpoint{4.438739in}{1.487428in}}%
\pgfpathlineto{\pgfqpoint{4.441549in}{1.487315in}}%
\pgfpathlineto{\pgfqpoint{4.444360in}{1.487484in}}%
\pgfpathlineto{\pgfqpoint{4.447171in}{1.487405in}}%
\pgfpathlineto{\pgfqpoint{4.449981in}{1.487608in}}%
\pgfpathlineto{\pgfqpoint{4.452792in}{1.487675in}}%
\pgfpathlineto{\pgfqpoint{4.455603in}{1.487812in}}%
\pgfpathlineto{\pgfqpoint{4.458414in}{1.488003in}}%
\pgfpathlineto{\pgfqpoint{4.461224in}{1.488193in}}%
\pgfpathlineto{\pgfqpoint{4.464035in}{1.488320in}}%
\pgfpathlineto{\pgfqpoint{4.466846in}{1.488512in}}%
\pgfpathlineto{\pgfqpoint{4.469656in}{1.488547in}}%
\pgfpathlineto{\pgfqpoint{4.472467in}{1.488727in}}%
\pgfpathlineto{\pgfqpoint{4.475278in}{1.488887in}}%
\pgfpathlineto{\pgfqpoint{4.478088in}{1.489026in}}%
\pgfpathlineto{\pgfqpoint{4.480899in}{1.489093in}}%
\pgfpathlineto{\pgfqpoint{4.483710in}{1.489289in}}%
\pgfpathlineto{\pgfqpoint{4.486520in}{1.489368in}}%
\pgfpathlineto{\pgfqpoint{4.489331in}{1.489379in}}%
\pgfpathlineto{\pgfqpoint{4.492142in}{1.489574in}}%
\pgfpathlineto{\pgfqpoint{4.494952in}{1.489612in}}%
\pgfpathlineto{\pgfqpoint{4.497763in}{1.489803in}}%
\pgfpathlineto{\pgfqpoint{4.500574in}{1.489592in}}%
\pgfpathlineto{\pgfqpoint{4.503385in}{1.489328in}}%
\pgfpathlineto{\pgfqpoint{4.506195in}{1.489333in}}%
\pgfpathlineto{\pgfqpoint{4.509006in}{1.489524in}}%
\pgfpathlineto{\pgfqpoint{4.511817in}{1.489707in}}%
\pgfpathlineto{\pgfqpoint{4.514627in}{1.489733in}}%
\pgfpathlineto{\pgfqpoint{4.517438in}{1.489911in}}%
\pgfpathlineto{\pgfqpoint{4.520249in}{1.490075in}}%
\pgfpathlineto{\pgfqpoint{4.523059in}{1.489363in}}%
\pgfpathlineto{\pgfqpoint{4.525870in}{1.489467in}}%
\pgfpathlineto{\pgfqpoint{4.528681in}{1.489643in}}%
\pgfpathlineto{\pgfqpoint{4.531491in}{1.489831in}}%
\pgfpathlineto{\pgfqpoint{4.534302in}{1.490015in}}%
\pgfpathlineto{\pgfqpoint{4.537113in}{1.490209in}}%
\pgfpathlineto{\pgfqpoint{4.539923in}{1.490381in}}%
\pgfpathlineto{\pgfqpoint{4.542734in}{1.490546in}}%
\pgfpathlineto{\pgfqpoint{4.545545in}{1.490423in}}%
\pgfpathlineto{\pgfqpoint{4.548356in}{1.490604in}}%
\pgfpathlineto{\pgfqpoint{4.551166in}{1.490798in}}%
\pgfpathlineto{\pgfqpoint{4.553977in}{1.490942in}}%
\pgfpathlineto{\pgfqpoint{4.556788in}{1.490931in}}%
\pgfpathlineto{\pgfqpoint{4.559598in}{1.490947in}}%
\pgfpathlineto{\pgfqpoint{4.562409in}{1.491091in}}%
\pgfpathlineto{\pgfqpoint{4.565220in}{1.491238in}}%
\pgfpathlineto{\pgfqpoint{4.568030in}{1.491207in}}%
\pgfpathlineto{\pgfqpoint{4.570841in}{1.491403in}}%
\pgfpathlineto{\pgfqpoint{4.573652in}{1.491471in}}%
\pgfpathlineto{\pgfqpoint{4.576462in}{1.491664in}}%
\pgfpathlineto{\pgfqpoint{4.579273in}{1.491841in}}%
\pgfpathlineto{\pgfqpoint{4.582084in}{1.491365in}}%
\pgfpathlineto{\pgfqpoint{4.584894in}{1.491523in}}%
\pgfpathlineto{\pgfqpoint{4.587705in}{1.491643in}}%
\pgfpathlineto{\pgfqpoint{4.590516in}{1.491804in}}%
\pgfpathlineto{\pgfqpoint{4.593327in}{1.491989in}}%
\pgfpathlineto{\pgfqpoint{4.596137in}{1.492176in}}%
\pgfpathlineto{\pgfqpoint{4.598948in}{1.492089in}}%
\pgfpathlineto{\pgfqpoint{4.601759in}{1.492281in}}%
\pgfpathlineto{\pgfqpoint{4.604569in}{1.492391in}}%
\pgfpathlineto{\pgfqpoint{4.607380in}{1.492491in}}%
\pgfpathlineto{\pgfqpoint{4.610191in}{1.492684in}}%
\pgfpathlineto{\pgfqpoint{4.613001in}{1.492876in}}%
\pgfpathlineto{\pgfqpoint{4.615812in}{1.493013in}}%
\pgfpathlineto{\pgfqpoint{4.618623in}{1.493201in}}%
\pgfpathlineto{\pgfqpoint{4.621433in}{1.493392in}}%
\pgfpathlineto{\pgfqpoint{4.624244in}{1.493564in}}%
\pgfpathlineto{\pgfqpoint{4.627055in}{1.493742in}}%
\pgfpathlineto{\pgfqpoint{4.629865in}{1.493870in}}%
\pgfpathlineto{\pgfqpoint{4.632676in}{1.493931in}}%
\pgfpathlineto{\pgfqpoint{4.635487in}{1.494079in}}%
\pgfpathlineto{\pgfqpoint{4.638298in}{1.494113in}}%
\pgfpathlineto{\pgfqpoint{4.641108in}{1.494241in}}%
\pgfpathlineto{\pgfqpoint{4.643919in}{1.494340in}}%
\pgfpathlineto{\pgfqpoint{4.646730in}{1.494272in}}%
\pgfpathlineto{\pgfqpoint{4.649540in}{1.494462in}}%
\pgfpathlineto{\pgfqpoint{4.652351in}{1.494652in}}%
\pgfpathlineto{\pgfqpoint{4.655162in}{1.494832in}}%
\pgfpathlineto{\pgfqpoint{4.657972in}{1.495022in}}%
\pgfpathlineto{\pgfqpoint{4.660783in}{1.495202in}}%
\pgfpathlineto{\pgfqpoint{4.663594in}{1.495377in}}%
\pgfpathlineto{\pgfqpoint{4.666404in}{1.495554in}}%
\pgfpathlineto{\pgfqpoint{4.669215in}{1.495715in}}%
\pgfpathlineto{\pgfqpoint{4.672026in}{1.495875in}}%
\pgfpathlineto{\pgfqpoint{4.674836in}{1.494643in}}%
\pgfpathlineto{\pgfqpoint{4.677647in}{1.494825in}}%
\pgfpathlineto{\pgfqpoint{4.680458in}{1.494925in}}%
\pgfpathlineto{\pgfqpoint{4.683268in}{1.495114in}}%
\pgfpathlineto{\pgfqpoint{4.686079in}{1.495203in}}%
\pgfpathlineto{\pgfqpoint{4.688890in}{1.495341in}}%
\pgfpathlineto{\pgfqpoint{4.691701in}{1.495345in}}%
\pgfpathlineto{\pgfqpoint{4.694511in}{1.495418in}}%
\pgfpathlineto{\pgfqpoint{4.697322in}{1.495366in}}%
\pgfpathlineto{\pgfqpoint{4.700133in}{1.495345in}}%
\pgfpathlineto{\pgfqpoint{4.702943in}{1.495421in}}%
\pgfpathlineto{\pgfqpoint{4.705754in}{1.495590in}}%
\pgfpathlineto{\pgfqpoint{4.708565in}{1.495774in}}%
\pgfpathlineto{\pgfqpoint{4.711375in}{1.495287in}}%
\pgfpathlineto{\pgfqpoint{4.714186in}{1.495470in}}%
\pgfpathlineto{\pgfqpoint{4.716997in}{1.495652in}}%
\pgfpathlineto{\pgfqpoint{4.719807in}{1.495833in}}%
\pgfpathlineto{\pgfqpoint{4.722618in}{1.496002in}}%
\pgfpathlineto{\pgfqpoint{4.725429in}{1.496188in}}%
\pgfpathlineto{\pgfqpoint{4.728239in}{1.496364in}}%
\pgfpathlineto{\pgfqpoint{4.731050in}{1.496506in}}%
\pgfpathlineto{\pgfqpoint{4.733861in}{1.496617in}}%
\pgfpathlineto{\pgfqpoint{4.736672in}{1.496737in}}%
\pgfpathlineto{\pgfqpoint{4.739482in}{1.496882in}}%
\pgfpathlineto{\pgfqpoint{4.742293in}{1.497064in}}%
\pgfpathlineto{\pgfqpoint{4.745104in}{1.497212in}}%
\pgfpathlineto{\pgfqpoint{4.747914in}{1.497259in}}%
\pgfpathlineto{\pgfqpoint{4.750725in}{1.497433in}}%
\pgfpathlineto{\pgfqpoint{4.753536in}{1.497616in}}%
\pgfpathlineto{\pgfqpoint{4.756346in}{1.497150in}}%
\pgfpathlineto{\pgfqpoint{4.759157in}{1.497298in}}%
\pgfpathlineto{\pgfqpoint{4.761968in}{1.497474in}}%
\pgfpathlineto{\pgfqpoint{4.764778in}{1.497427in}}%
\pgfpathlineto{\pgfqpoint{4.767589in}{1.497465in}}%
\pgfpathlineto{\pgfqpoint{4.770400in}{1.497569in}}%
\pgfpathlineto{\pgfqpoint{4.773210in}{1.497751in}}%
\pgfpathlineto{\pgfqpoint{4.776021in}{1.497418in}}%
\pgfpathlineto{\pgfqpoint{4.778832in}{1.497412in}}%
\pgfpathlineto{\pgfqpoint{4.781643in}{1.497446in}}%
\pgfpathlineto{\pgfqpoint{4.784453in}{1.497511in}}%
\pgfpathlineto{\pgfqpoint{4.787264in}{1.497604in}}%
\pgfpathlineto{\pgfqpoint{4.790075in}{1.497480in}}%
\pgfpathlineto{\pgfqpoint{4.792885in}{1.497600in}}%
\pgfpathlineto{\pgfqpoint{4.795696in}{1.497227in}}%
\pgfpathlineto{\pgfqpoint{4.798507in}{1.497408in}}%
\pgfpathlineto{\pgfqpoint{4.801317in}{1.497395in}}%
\pgfpathlineto{\pgfqpoint{4.804128in}{1.497307in}}%
\pgfpathlineto{\pgfqpoint{4.806939in}{1.496873in}}%
\pgfpathlineto{\pgfqpoint{4.809749in}{1.497054in}}%
\pgfpathlineto{\pgfqpoint{4.812560in}{1.497079in}}%
\pgfpathlineto{\pgfqpoint{4.815371in}{1.497259in}}%
\pgfpathlineto{\pgfqpoint{4.818181in}{1.497436in}}%
\pgfpathlineto{\pgfqpoint{4.820992in}{1.497572in}}%
\pgfpathlineto{\pgfqpoint{4.823803in}{1.497728in}}%
\pgfpathlineto{\pgfqpoint{4.826614in}{1.497769in}}%
\pgfpathlineto{\pgfqpoint{4.829424in}{1.497923in}}%
\pgfpathlineto{\pgfqpoint{4.832235in}{1.498072in}}%
\pgfpathlineto{\pgfqpoint{4.835046in}{1.498249in}}%
\pgfpathlineto{\pgfqpoint{4.837856in}{1.498380in}}%
\pgfpathlineto{\pgfqpoint{4.840667in}{1.498414in}}%
\pgfpathlineto{\pgfqpoint{4.843478in}{1.498589in}}%
\pgfpathlineto{\pgfqpoint{4.846288in}{1.498747in}}%
\pgfpathlineto{\pgfqpoint{4.849099in}{1.498926in}}%
\pgfpathlineto{\pgfqpoint{4.851910in}{1.499100in}}%
\pgfpathlineto{\pgfqpoint{4.854720in}{1.499272in}}%
\pgfpathlineto{\pgfqpoint{4.857531in}{1.499225in}}%
\pgfpathlineto{\pgfqpoint{4.860342in}{1.499403in}}%
\pgfpathlineto{\pgfqpoint{4.863152in}{1.499568in}}%
\pgfpathlineto{\pgfqpoint{4.865963in}{1.499625in}}%
\pgfpathlineto{\pgfqpoint{4.868774in}{1.499793in}}%
\pgfpathlineto{\pgfqpoint{4.871584in}{1.499923in}}%
\pgfpathlineto{\pgfqpoint{4.874395in}{1.500096in}}%
\pgfpathlineto{\pgfqpoint{4.877206in}{1.499974in}}%
\pgfpathlineto{\pgfqpoint{4.880017in}{1.499981in}}%
\pgfpathlineto{\pgfqpoint{4.882827in}{1.500133in}}%
\pgfpathlineto{\pgfqpoint{4.885638in}{1.500098in}}%
\pgfpathlineto{\pgfqpoint{4.888449in}{1.500154in}}%
\pgfpathlineto{\pgfqpoint{4.891259in}{1.499840in}}%
\pgfpathlineto{\pgfqpoint{4.894070in}{1.499910in}}%
\pgfpathlineto{\pgfqpoint{4.896881in}{1.500010in}}%
\pgfpathlineto{\pgfqpoint{4.899691in}{1.500011in}}%
\pgfpathlineto{\pgfqpoint{4.902502in}{1.500150in}}%
\pgfpathlineto{\pgfqpoint{4.905313in}{1.500300in}}%
\pgfpathlineto{\pgfqpoint{4.908123in}{1.500421in}}%
\pgfpathlineto{\pgfqpoint{4.910934in}{1.500549in}}%
\pgfpathlineto{\pgfqpoint{4.913745in}{1.500706in}}%
\pgfpathlineto{\pgfqpoint{4.916555in}{1.500876in}}%
\pgfpathlineto{\pgfqpoint{4.919366in}{1.501047in}}%
\pgfpathlineto{\pgfqpoint{4.922177in}{1.501188in}}%
\pgfpathlineto{\pgfqpoint{4.924988in}{1.501295in}}%
\pgfpathlineto{\pgfqpoint{4.927798in}{1.501325in}}%
\pgfpathlineto{\pgfqpoint{4.930609in}{1.501361in}}%
\pgfpathlineto{\pgfqpoint{4.933420in}{1.501314in}}%
\pgfpathlineto{\pgfqpoint{4.936230in}{1.501427in}}%
\pgfpathlineto{\pgfqpoint{4.939041in}{1.501599in}}%
\pgfpathlineto{\pgfqpoint{4.941852in}{1.501772in}}%
\pgfpathlineto{\pgfqpoint{4.944662in}{1.501938in}}%
\pgfpathlineto{\pgfqpoint{4.947473in}{1.502074in}}%
\pgfpathlineto{\pgfqpoint{4.950284in}{1.502207in}}%
\pgfpathlineto{\pgfqpoint{4.953094in}{1.502379in}}%
\pgfpathlineto{\pgfqpoint{4.955905in}{1.502451in}}%
\pgfpathlineto{\pgfqpoint{4.958716in}{1.502612in}}%
\pgfpathlineto{\pgfqpoint{4.961526in}{1.502738in}}%
\pgfpathlineto{\pgfqpoint{4.964337in}{1.502864in}}%
\pgfpathlineto{\pgfqpoint{4.967148in}{1.503035in}}%
\pgfpathlineto{\pgfqpoint{4.969959in}{1.503183in}}%
\pgfpathlineto{\pgfqpoint{4.972769in}{1.503355in}}%
\pgfpathlineto{\pgfqpoint{4.975580in}{1.503526in}}%
\pgfpathlineto{\pgfqpoint{4.978391in}{1.503468in}}%
\pgfpathlineto{\pgfqpoint{4.981201in}{1.503638in}}%
\pgfpathlineto{\pgfqpoint{4.984012in}{1.503756in}}%
\pgfpathlineto{\pgfqpoint{4.986823in}{1.503924in}}%
\pgfpathlineto{\pgfqpoint{4.989633in}{1.504076in}}%
\pgfpathlineto{\pgfqpoint{4.992444in}{1.504245in}}%
\pgfpathlineto{\pgfqpoint{4.995255in}{1.504297in}}%
\pgfpathlineto{\pgfqpoint{4.998065in}{1.504273in}}%
\pgfpathlineto{\pgfqpoint{5.000876in}{1.504365in}}%
\pgfpathlineto{\pgfqpoint{5.003687in}{1.504517in}}%
\pgfpathlineto{\pgfqpoint{5.006497in}{1.504666in}}%
\pgfpathlineto{\pgfqpoint{5.009308in}{1.504804in}}%
\pgfpathlineto{\pgfqpoint{5.012119in}{1.504961in}}%
\pgfpathlineto{\pgfqpoint{5.014930in}{1.505036in}}%
\pgfpathlineto{\pgfqpoint{5.017740in}{1.505204in}}%
\pgfpathlineto{\pgfqpoint{5.020551in}{1.505369in}}%
\pgfpathlineto{\pgfqpoint{5.023362in}{1.505524in}}%
\pgfpathlineto{\pgfqpoint{5.026172in}{1.505591in}}%
\pgfpathlineto{\pgfqpoint{5.028983in}{1.505726in}}%
\pgfpathlineto{\pgfqpoint{5.031794in}{1.502882in}}%
\pgfpathlineto{\pgfqpoint{5.034604in}{1.503003in}}%
\pgfpathlineto{\pgfqpoint{5.037415in}{1.502945in}}%
\pgfpathlineto{\pgfqpoint{5.040226in}{1.502849in}}%
\pgfpathlineto{\pgfqpoint{5.043036in}{1.503019in}}%
\pgfpathlineto{\pgfqpoint{5.045847in}{1.502769in}}%
\pgfpathlineto{\pgfqpoint{5.048658in}{1.502910in}}%
\pgfpathlineto{\pgfqpoint{5.051468in}{1.503055in}}%
\pgfpathlineto{\pgfqpoint{5.054279in}{1.503162in}}%
\pgfpathlineto{\pgfqpoint{5.057090in}{1.503143in}}%
\pgfpathlineto{\pgfqpoint{5.059901in}{1.502919in}}%
\pgfpathlineto{\pgfqpoint{5.062711in}{1.503078in}}%
\pgfpathlineto{\pgfqpoint{5.065522in}{1.503228in}}%
\pgfpathlineto{\pgfqpoint{5.068333in}{1.503201in}}%
\pgfpathlineto{\pgfqpoint{5.071143in}{1.503361in}}%
\pgfpathlineto{\pgfqpoint{5.073954in}{1.502876in}}%
\pgfpathlineto{\pgfqpoint{5.076765in}{1.503002in}}%
\pgfpathlineto{\pgfqpoint{5.079575in}{1.503170in}}%
\pgfpathlineto{\pgfqpoint{5.082386in}{1.503195in}}%
\pgfpathlineto{\pgfqpoint{5.085197in}{1.503342in}}%
\pgfpathlineto{\pgfqpoint{5.088007in}{1.503382in}}%
\pgfpathlineto{\pgfqpoint{5.090818in}{1.503546in}}%
\pgfpathlineto{\pgfqpoint{5.093629in}{1.503263in}}%
\pgfpathlineto{\pgfqpoint{5.096439in}{1.503315in}}%
\pgfpathlineto{\pgfqpoint{5.099250in}{1.503396in}}%
\pgfpathlineto{\pgfqpoint{5.102061in}{1.503460in}}%
\pgfpathlineto{\pgfqpoint{5.104871in}{1.502921in}}%
\pgfpathlineto{\pgfqpoint{5.107682in}{1.503051in}}%
\pgfpathlineto{\pgfqpoint{5.110493in}{1.503035in}}%
\pgfpathlineto{\pgfqpoint{5.113304in}{1.503200in}}%
\pgfpathlineto{\pgfqpoint{5.116114in}{1.503367in}}%
\pgfpathlineto{\pgfqpoint{5.118925in}{1.503368in}}%
\pgfpathlineto{\pgfqpoint{5.121736in}{1.503497in}}%
\pgfpathlineto{\pgfqpoint{5.124546in}{1.503603in}}%
\pgfpathlineto{\pgfqpoint{5.127357in}{1.503416in}}%
\pgfpathlineto{\pgfqpoint{5.130168in}{1.502929in}}%
\pgfpathlineto{\pgfqpoint{5.132978in}{1.503036in}}%
\pgfpathlineto{\pgfqpoint{5.135789in}{1.503193in}}%
\pgfpathlineto{\pgfqpoint{5.138600in}{1.502934in}}%
\pgfpathlineto{\pgfqpoint{5.141410in}{1.503044in}}%
\pgfpathlineto{\pgfqpoint{5.144221in}{1.502900in}}%
\pgfpathlineto{\pgfqpoint{5.147032in}{1.503031in}}%
\pgfpathlineto{\pgfqpoint{5.149842in}{1.503179in}}%
\pgfpathlineto{\pgfqpoint{5.149842in}{2.293917in}}%
\pgfpathlineto{\pgfqpoint{5.149842in}{2.293917in}}%
\pgfpathlineto{\pgfqpoint{5.147032in}{2.294026in}}%
\pgfpathlineto{\pgfqpoint{5.144221in}{2.294158in}}%
\pgfpathlineto{\pgfqpoint{5.141410in}{2.294289in}}%
\pgfpathlineto{\pgfqpoint{5.138600in}{2.294437in}}%
\pgfpathlineto{\pgfqpoint{5.135789in}{2.294524in}}%
\pgfpathlineto{\pgfqpoint{5.132978in}{2.294473in}}%
\pgfpathlineto{\pgfqpoint{5.130168in}{2.294260in}}%
\pgfpathlineto{\pgfqpoint{5.127357in}{2.293038in}}%
\pgfpathlineto{\pgfqpoint{5.124546in}{2.292273in}}%
\pgfpathlineto{\pgfqpoint{5.121736in}{2.292423in}}%
\pgfpathlineto{\pgfqpoint{5.118925in}{2.292558in}}%
\pgfpathlineto{\pgfqpoint{5.116114in}{2.292723in}}%
\pgfpathlineto{\pgfqpoint{5.113304in}{2.292756in}}%
\pgfpathlineto{\pgfqpoint{5.110493in}{2.292810in}}%
\pgfpathlineto{\pgfqpoint{5.107682in}{2.292974in}}%
\pgfpathlineto{\pgfqpoint{5.104871in}{2.293109in}}%
\pgfpathlineto{\pgfqpoint{5.102061in}{2.293056in}}%
\pgfpathlineto{\pgfqpoint{5.099250in}{2.293220in}}%
\pgfpathlineto{\pgfqpoint{5.096439in}{2.293382in}}%
\pgfpathlineto{\pgfqpoint{5.093629in}{2.293043in}}%
\pgfpathlineto{\pgfqpoint{5.090818in}{2.293121in}}%
\pgfpathlineto{\pgfqpoint{5.088007in}{2.293188in}}%
\pgfpathlineto{\pgfqpoint{5.085197in}{2.293355in}}%
\pgfpathlineto{\pgfqpoint{5.082386in}{2.293471in}}%
\pgfpathlineto{\pgfqpoint{5.079575in}{2.293639in}}%
\pgfpathlineto{\pgfqpoint{5.076765in}{2.293672in}}%
\pgfpathlineto{\pgfqpoint{5.073954in}{2.293813in}}%
\pgfpathlineto{\pgfqpoint{5.071143in}{2.293792in}}%
\pgfpathlineto{\pgfqpoint{5.068333in}{2.293880in}}%
\pgfpathlineto{\pgfqpoint{5.065522in}{2.294044in}}%
\pgfpathlineto{\pgfqpoint{5.062711in}{2.294155in}}%
\pgfpathlineto{\pgfqpoint{5.059901in}{2.294251in}}%
\pgfpathlineto{\pgfqpoint{5.057090in}{2.294356in}}%
\pgfpathlineto{\pgfqpoint{5.054279in}{2.294521in}}%
\pgfpathlineto{\pgfqpoint{5.051468in}{2.294676in}}%
\pgfpathlineto{\pgfqpoint{5.048658in}{2.294797in}}%
\pgfpathlineto{\pgfqpoint{5.045847in}{2.294670in}}%
\pgfpathlineto{\pgfqpoint{5.043036in}{2.294765in}}%
\pgfpathlineto{\pgfqpoint{5.040226in}{2.294778in}}%
\pgfpathlineto{\pgfqpoint{5.037415in}{2.294156in}}%
\pgfpathlineto{\pgfqpoint{5.034604in}{2.293599in}}%
\pgfpathlineto{\pgfqpoint{5.031794in}{2.293746in}}%
\pgfpathlineto{\pgfqpoint{5.028983in}{2.289463in}}%
\pgfpathlineto{\pgfqpoint{5.026172in}{2.289320in}}%
\pgfpathlineto{\pgfqpoint{5.023362in}{2.289486in}}%
\pgfpathlineto{\pgfqpoint{5.020551in}{2.289589in}}%
\pgfpathlineto{\pgfqpoint{5.017740in}{2.289562in}}%
\pgfpathlineto{\pgfqpoint{5.014930in}{2.289569in}}%
\pgfpathlineto{\pgfqpoint{5.012119in}{2.289734in}}%
\pgfpathlineto{\pgfqpoint{5.009308in}{2.289663in}}%
\pgfpathlineto{\pgfqpoint{5.006497in}{2.289795in}}%
\pgfpathlineto{\pgfqpoint{5.003687in}{2.289910in}}%
\pgfpathlineto{\pgfqpoint{5.000876in}{2.289820in}}%
\pgfpathlineto{\pgfqpoint{4.998065in}{2.289981in}}%
\pgfpathlineto{\pgfqpoint{4.995255in}{2.290147in}}%
\pgfpathlineto{\pgfqpoint{4.992444in}{2.290317in}}%
\pgfpathlineto{\pgfqpoint{4.989633in}{2.290376in}}%
\pgfpathlineto{\pgfqpoint{4.986823in}{2.290489in}}%
\pgfpathlineto{\pgfqpoint{4.984012in}{2.290561in}}%
\pgfpathlineto{\pgfqpoint{4.981201in}{2.290711in}}%
\pgfpathlineto{\pgfqpoint{4.978391in}{2.290714in}}%
\pgfpathlineto{\pgfqpoint{4.975580in}{2.290152in}}%
\pgfpathlineto{\pgfqpoint{4.972769in}{2.290195in}}%
\pgfpathlineto{\pgfqpoint{4.969959in}{2.290214in}}%
\pgfpathlineto{\pgfqpoint{4.967148in}{2.290337in}}%
\pgfpathlineto{\pgfqpoint{4.964337in}{2.290341in}}%
\pgfpathlineto{\pgfqpoint{4.961526in}{2.290488in}}%
\pgfpathlineto{\pgfqpoint{4.958716in}{2.290634in}}%
\pgfpathlineto{\pgfqpoint{4.955905in}{2.290733in}}%
\pgfpathlineto{\pgfqpoint{4.953094in}{2.290902in}}%
\pgfpathlineto{\pgfqpoint{4.950284in}{2.290954in}}%
\pgfpathlineto{\pgfqpoint{4.947473in}{2.291095in}}%
\pgfpathlineto{\pgfqpoint{4.944662in}{2.290943in}}%
\pgfpathlineto{\pgfqpoint{4.941852in}{2.291032in}}%
\pgfpathlineto{\pgfqpoint{4.939041in}{2.291075in}}%
\pgfpathlineto{\pgfqpoint{4.936230in}{2.291073in}}%
\pgfpathlineto{\pgfqpoint{4.933420in}{2.290853in}}%
\pgfpathlineto{\pgfqpoint{4.930609in}{2.291019in}}%
\pgfpathlineto{\pgfqpoint{4.927798in}{2.291193in}}%
\pgfpathlineto{\pgfqpoint{4.924988in}{2.290790in}}%
\pgfpathlineto{\pgfqpoint{4.922177in}{2.290950in}}%
\pgfpathlineto{\pgfqpoint{4.919366in}{2.291086in}}%
\pgfpathlineto{\pgfqpoint{4.916555in}{2.291158in}}%
\pgfpathlineto{\pgfqpoint{4.913745in}{2.291240in}}%
\pgfpathlineto{\pgfqpoint{4.910934in}{2.291354in}}%
\pgfpathlineto{\pgfqpoint{4.908123in}{2.291503in}}%
\pgfpathlineto{\pgfqpoint{4.905313in}{2.291657in}}%
\pgfpathlineto{\pgfqpoint{4.902502in}{2.291784in}}%
\pgfpathlineto{\pgfqpoint{4.899691in}{2.291924in}}%
\pgfpathlineto{\pgfqpoint{4.896881in}{2.292099in}}%
\pgfpathlineto{\pgfqpoint{4.894070in}{2.292264in}}%
\pgfpathlineto{\pgfqpoint{4.891259in}{2.292437in}}%
\pgfpathlineto{\pgfqpoint{4.888449in}{2.292512in}}%
\pgfpathlineto{\pgfqpoint{4.885638in}{2.292687in}}%
\pgfpathlineto{\pgfqpoint{4.882827in}{2.292858in}}%
\pgfpathlineto{\pgfqpoint{4.880017in}{2.292741in}}%
\pgfpathlineto{\pgfqpoint{4.877206in}{2.292917in}}%
\pgfpathlineto{\pgfqpoint{4.874395in}{2.293068in}}%
\pgfpathlineto{\pgfqpoint{4.871584in}{2.293146in}}%
\pgfpathlineto{\pgfqpoint{4.868774in}{2.293296in}}%
\pgfpathlineto{\pgfqpoint{4.865963in}{2.293393in}}%
\pgfpathlineto{\pgfqpoint{4.863152in}{2.293570in}}%
\pgfpathlineto{\pgfqpoint{4.860342in}{2.293676in}}%
\pgfpathlineto{\pgfqpoint{4.857531in}{2.293690in}}%
\pgfpathlineto{\pgfqpoint{4.854720in}{2.293130in}}%
\pgfpathlineto{\pgfqpoint{4.851910in}{2.293215in}}%
\pgfpathlineto{\pgfqpoint{4.849099in}{2.293193in}}%
\pgfpathlineto{\pgfqpoint{4.846288in}{2.293243in}}%
\pgfpathlineto{\pgfqpoint{4.843478in}{2.293364in}}%
\pgfpathlineto{\pgfqpoint{4.840667in}{2.293442in}}%
\pgfpathlineto{\pgfqpoint{4.837856in}{2.293621in}}%
\pgfpathlineto{\pgfqpoint{4.835046in}{2.293774in}}%
\pgfpathlineto{\pgfqpoint{4.832235in}{2.293841in}}%
\pgfpathlineto{\pgfqpoint{4.829424in}{2.293978in}}%
\pgfpathlineto{\pgfqpoint{4.826614in}{2.294108in}}%
\pgfpathlineto{\pgfqpoint{4.823803in}{2.294288in}}%
\pgfpathlineto{\pgfqpoint{4.820992in}{2.294175in}}%
\pgfpathlineto{\pgfqpoint{4.818181in}{2.294326in}}%
\pgfpathlineto{\pgfqpoint{4.815371in}{2.294303in}}%
\pgfpathlineto{\pgfqpoint{4.812560in}{2.294316in}}%
\pgfpathlineto{\pgfqpoint{4.809749in}{2.294497in}}%
\pgfpathlineto{\pgfqpoint{4.806939in}{2.294524in}}%
\pgfpathlineto{\pgfqpoint{4.804128in}{2.294547in}}%
\pgfpathlineto{\pgfqpoint{4.801317in}{2.293909in}}%
\pgfpathlineto{\pgfqpoint{4.798507in}{2.294087in}}%
\pgfpathlineto{\pgfqpoint{4.795696in}{2.294146in}}%
\pgfpathlineto{\pgfqpoint{4.792885in}{2.294201in}}%
\pgfpathlineto{\pgfqpoint{4.790075in}{2.293977in}}%
\pgfpathlineto{\pgfqpoint{4.787264in}{2.294132in}}%
\pgfpathlineto{\pgfqpoint{4.784453in}{2.294307in}}%
\pgfpathlineto{\pgfqpoint{4.781643in}{2.294487in}}%
\pgfpathlineto{\pgfqpoint{4.778832in}{2.294669in}}%
\pgfpathlineto{\pgfqpoint{4.776021in}{2.294850in}}%
\pgfpathlineto{\pgfqpoint{4.773210in}{2.294924in}}%
\pgfpathlineto{\pgfqpoint{4.770400in}{2.294933in}}%
\pgfpathlineto{\pgfqpoint{4.767589in}{2.295104in}}%
\pgfpathlineto{\pgfqpoint{4.764778in}{2.295288in}}%
\pgfpathlineto{\pgfqpoint{4.761968in}{2.295463in}}%
\pgfpathlineto{\pgfqpoint{4.759157in}{2.295557in}}%
\pgfpathlineto{\pgfqpoint{4.756346in}{2.295701in}}%
\pgfpathlineto{\pgfqpoint{4.753536in}{2.295712in}}%
\pgfpathlineto{\pgfqpoint{4.750725in}{2.295771in}}%
\pgfpathlineto{\pgfqpoint{4.747914in}{2.295871in}}%
\pgfpathlineto{\pgfqpoint{4.745104in}{2.296055in}}%
\pgfpathlineto{\pgfqpoint{4.742293in}{2.296200in}}%
\pgfpathlineto{\pgfqpoint{4.739482in}{2.296275in}}%
\pgfpathlineto{\pgfqpoint{4.736672in}{2.296423in}}%
\pgfpathlineto{\pgfqpoint{4.733861in}{2.296591in}}%
\pgfpathlineto{\pgfqpoint{4.731050in}{2.296762in}}%
\pgfpathlineto{\pgfqpoint{4.728239in}{2.296914in}}%
\pgfpathlineto{\pgfqpoint{4.725429in}{2.297015in}}%
\pgfpathlineto{\pgfqpoint{4.722618in}{2.297049in}}%
\pgfpathlineto{\pgfqpoint{4.719807in}{2.297168in}}%
\pgfpathlineto{\pgfqpoint{4.716997in}{2.297135in}}%
\pgfpathlineto{\pgfqpoint{4.714186in}{2.297214in}}%
\pgfpathlineto{\pgfqpoint{4.711375in}{2.297292in}}%
\pgfpathlineto{\pgfqpoint{4.708565in}{2.297296in}}%
\pgfpathlineto{\pgfqpoint{4.705754in}{2.297371in}}%
\pgfpathlineto{\pgfqpoint{4.702943in}{2.297490in}}%
\pgfpathlineto{\pgfqpoint{4.700133in}{2.297674in}}%
\pgfpathlineto{\pgfqpoint{4.697322in}{2.297858in}}%
\pgfpathlineto{\pgfqpoint{4.694511in}{2.298037in}}%
\pgfpathlineto{\pgfqpoint{4.691701in}{2.298222in}}%
\pgfpathlineto{\pgfqpoint{4.688890in}{2.298409in}}%
\pgfpathlineto{\pgfqpoint{4.686079in}{2.298569in}}%
\pgfpathlineto{\pgfqpoint{4.683268in}{2.298752in}}%
\pgfpathlineto{\pgfqpoint{4.680458in}{2.298777in}}%
\pgfpathlineto{\pgfqpoint{4.677647in}{2.298484in}}%
\pgfpathlineto{\pgfqpoint{4.674836in}{2.298577in}}%
\pgfpathlineto{\pgfqpoint{4.672026in}{2.298132in}}%
\pgfpathlineto{\pgfqpoint{4.669215in}{2.298001in}}%
\pgfpathlineto{\pgfqpoint{4.666404in}{2.298139in}}%
\pgfpathlineto{\pgfqpoint{4.663594in}{2.298248in}}%
\pgfpathlineto{\pgfqpoint{4.660783in}{2.298171in}}%
\pgfpathlineto{\pgfqpoint{4.657972in}{2.298275in}}%
\pgfpathlineto{\pgfqpoint{4.655162in}{2.298308in}}%
\pgfpathlineto{\pgfqpoint{4.652351in}{2.298411in}}%
\pgfpathlineto{\pgfqpoint{4.649540in}{2.298422in}}%
\pgfpathlineto{\pgfqpoint{4.646730in}{2.298478in}}%
\pgfpathlineto{\pgfqpoint{4.643919in}{2.298657in}}%
\pgfpathlineto{\pgfqpoint{4.641108in}{2.298839in}}%
\pgfpathlineto{\pgfqpoint{4.638298in}{2.299009in}}%
\pgfpathlineto{\pgfqpoint{4.635487in}{2.299201in}}%
\pgfpathlineto{\pgfqpoint{4.632676in}{2.299358in}}%
\pgfpathlineto{\pgfqpoint{4.629865in}{2.299548in}}%
\pgfpathlineto{\pgfqpoint{4.627055in}{2.299719in}}%
\pgfpathlineto{\pgfqpoint{4.624244in}{2.299834in}}%
\pgfpathlineto{\pgfqpoint{4.621433in}{2.299961in}}%
\pgfpathlineto{\pgfqpoint{4.618623in}{2.299960in}}%
\pgfpathlineto{\pgfqpoint{4.615812in}{2.300043in}}%
\pgfpathlineto{\pgfqpoint{4.613001in}{2.300210in}}%
\pgfpathlineto{\pgfqpoint{4.610191in}{2.300275in}}%
\pgfpathlineto{\pgfqpoint{4.607380in}{2.300326in}}%
\pgfpathlineto{\pgfqpoint{4.604569in}{2.300511in}}%
\pgfpathlineto{\pgfqpoint{4.601759in}{2.300692in}}%
\pgfpathlineto{\pgfqpoint{4.598948in}{2.300679in}}%
\pgfpathlineto{\pgfqpoint{4.596137in}{2.300857in}}%
\pgfpathlineto{\pgfqpoint{4.593327in}{2.300951in}}%
\pgfpathlineto{\pgfqpoint{4.590516in}{2.301055in}}%
\pgfpathlineto{\pgfqpoint{4.587705in}{2.301204in}}%
\pgfpathlineto{\pgfqpoint{4.584894in}{2.301382in}}%
\pgfpathlineto{\pgfqpoint{4.582084in}{2.301535in}}%
\pgfpathlineto{\pgfqpoint{4.579273in}{2.301557in}}%
\pgfpathlineto{\pgfqpoint{4.576462in}{2.301682in}}%
\pgfpathlineto{\pgfqpoint{4.573652in}{2.301755in}}%
\pgfpathlineto{\pgfqpoint{4.570841in}{2.301949in}}%
\pgfpathlineto{\pgfqpoint{4.568030in}{2.302010in}}%
\pgfpathlineto{\pgfqpoint{4.565220in}{2.302201in}}%
\pgfpathlineto{\pgfqpoint{4.562409in}{2.302365in}}%
\pgfpathlineto{\pgfqpoint{4.559598in}{2.302533in}}%
\pgfpathlineto{\pgfqpoint{4.556788in}{2.302729in}}%
\pgfpathlineto{\pgfqpoint{4.553977in}{2.302924in}}%
\pgfpathlineto{\pgfqpoint{4.551166in}{2.303092in}}%
\pgfpathlineto{\pgfqpoint{4.548356in}{2.303073in}}%
\pgfpathlineto{\pgfqpoint{4.545545in}{2.303196in}}%
\pgfpathlineto{\pgfqpoint{4.542734in}{2.303367in}}%
\pgfpathlineto{\pgfqpoint{4.539923in}{2.303516in}}%
\pgfpathlineto{\pgfqpoint{4.537113in}{2.303657in}}%
\pgfpathlineto{\pgfqpoint{4.534302in}{2.303629in}}%
\pgfpathlineto{\pgfqpoint{4.531491in}{2.303748in}}%
\pgfpathlineto{\pgfqpoint{4.528681in}{2.303857in}}%
\pgfpathlineto{\pgfqpoint{4.525870in}{2.303743in}}%
\pgfpathlineto{\pgfqpoint{4.523059in}{2.303934in}}%
\pgfpathlineto{\pgfqpoint{4.520249in}{2.303833in}}%
\pgfpathlineto{\pgfqpoint{4.517438in}{2.303986in}}%
\pgfpathlineto{\pgfqpoint{4.514627in}{2.304120in}}%
\pgfpathlineto{\pgfqpoint{4.511817in}{2.304320in}}%
\pgfpathlineto{\pgfqpoint{4.509006in}{2.304445in}}%
\pgfpathlineto{\pgfqpoint{4.506195in}{2.304551in}}%
\pgfpathlineto{\pgfqpoint{4.503385in}{2.304751in}}%
\pgfpathlineto{\pgfqpoint{4.500574in}{2.304877in}}%
\pgfpathlineto{\pgfqpoint{4.497763in}{2.305023in}}%
\pgfpathlineto{\pgfqpoint{4.494952in}{2.304966in}}%
\pgfpathlineto{\pgfqpoint{4.492142in}{2.305168in}}%
\pgfpathlineto{\pgfqpoint{4.489331in}{2.305261in}}%
\pgfpathlineto{\pgfqpoint{4.486520in}{2.304746in}}%
\pgfpathlineto{\pgfqpoint{4.483710in}{2.304945in}}%
\pgfpathlineto{\pgfqpoint{4.480899in}{2.304912in}}%
\pgfpathlineto{\pgfqpoint{4.478088in}{2.305113in}}%
\pgfpathlineto{\pgfqpoint{4.475278in}{2.305291in}}%
\pgfpathlineto{\pgfqpoint{4.472467in}{2.305454in}}%
\pgfpathlineto{\pgfqpoint{4.469656in}{2.305590in}}%
\pgfpathlineto{\pgfqpoint{4.466846in}{2.305794in}}%
\pgfpathlineto{\pgfqpoint{4.464035in}{2.305727in}}%
\pgfpathlineto{\pgfqpoint{4.461224in}{2.305913in}}%
\pgfpathlineto{\pgfqpoint{4.458414in}{2.306031in}}%
\pgfpathlineto{\pgfqpoint{4.455603in}{2.306149in}}%
\pgfpathlineto{\pgfqpoint{4.452792in}{2.306331in}}%
\pgfpathlineto{\pgfqpoint{4.449981in}{2.306534in}}%
\pgfpathlineto{\pgfqpoint{4.447171in}{2.306531in}}%
\pgfpathlineto{\pgfqpoint{4.444360in}{2.306722in}}%
\pgfpathlineto{\pgfqpoint{4.441549in}{2.306880in}}%
\pgfpathlineto{\pgfqpoint{4.438739in}{2.307062in}}%
\pgfpathlineto{\pgfqpoint{4.435928in}{2.307192in}}%
\pgfpathlineto{\pgfqpoint{4.433117in}{2.307351in}}%
\pgfpathlineto{\pgfqpoint{4.430307in}{2.307532in}}%
\pgfpathlineto{\pgfqpoint{4.427496in}{2.307738in}}%
\pgfpathlineto{\pgfqpoint{4.424685in}{2.307748in}}%
\pgfpathlineto{\pgfqpoint{4.421875in}{2.307530in}}%
\pgfpathlineto{\pgfqpoint{4.419064in}{2.307702in}}%
\pgfpathlineto{\pgfqpoint{4.416253in}{2.307759in}}%
\pgfpathlineto{\pgfqpoint{4.413443in}{2.307966in}}%
\pgfpathlineto{\pgfqpoint{4.410632in}{2.307434in}}%
\pgfpathlineto{\pgfqpoint{4.407821in}{2.307513in}}%
\pgfpathlineto{\pgfqpoint{4.405011in}{2.307660in}}%
\pgfpathlineto{\pgfqpoint{4.402200in}{2.307744in}}%
\pgfpathlineto{\pgfqpoint{4.399389in}{2.307243in}}%
\pgfpathlineto{\pgfqpoint{4.396578in}{2.307226in}}%
\pgfpathlineto{\pgfqpoint{4.393768in}{2.307286in}}%
\pgfpathlineto{\pgfqpoint{4.390957in}{2.307208in}}%
\pgfpathlineto{\pgfqpoint{4.388146in}{2.307062in}}%
\pgfpathlineto{\pgfqpoint{4.385336in}{2.307173in}}%
\pgfpathlineto{\pgfqpoint{4.382525in}{2.307382in}}%
\pgfpathlineto{\pgfqpoint{4.379714in}{2.307526in}}%
\pgfpathlineto{\pgfqpoint{4.376904in}{2.307584in}}%
\pgfpathlineto{\pgfqpoint{4.374093in}{2.307793in}}%
\pgfpathlineto{\pgfqpoint{4.371282in}{2.307628in}}%
\pgfpathlineto{\pgfqpoint{4.368472in}{2.307788in}}%
\pgfpathlineto{\pgfqpoint{4.365661in}{2.307981in}}%
\pgfpathlineto{\pgfqpoint{4.362850in}{2.308071in}}%
\pgfpathlineto{\pgfqpoint{4.360040in}{2.308265in}}%
\pgfpathlineto{\pgfqpoint{4.357229in}{2.308165in}}%
\pgfpathlineto{\pgfqpoint{4.354418in}{2.308369in}}%
\pgfpathlineto{\pgfqpoint{4.351607in}{2.308499in}}%
\pgfpathlineto{\pgfqpoint{4.348797in}{2.308711in}}%
\pgfpathlineto{\pgfqpoint{4.345986in}{2.308881in}}%
\pgfpathlineto{\pgfqpoint{4.343175in}{2.309035in}}%
\pgfpathlineto{\pgfqpoint{4.340365in}{2.308003in}}%
\pgfpathlineto{\pgfqpoint{4.337554in}{2.308210in}}%
\pgfpathlineto{\pgfqpoint{4.334743in}{2.308400in}}%
\pgfpathlineto{\pgfqpoint{4.331933in}{2.308540in}}%
\pgfpathlineto{\pgfqpoint{4.329122in}{2.308743in}}%
\pgfpathlineto{\pgfqpoint{4.326311in}{2.308863in}}%
\pgfpathlineto{\pgfqpoint{4.323501in}{2.309056in}}%
\pgfpathlineto{\pgfqpoint{4.320690in}{2.309259in}}%
\pgfpathlineto{\pgfqpoint{4.317879in}{2.309472in}}%
\pgfpathlineto{\pgfqpoint{4.315069in}{2.309683in}}%
\pgfpathlineto{\pgfqpoint{4.312258in}{2.309815in}}%
\pgfpathlineto{\pgfqpoint{4.309447in}{2.310030in}}%
\pgfpathlineto{\pgfqpoint{4.306636in}{2.310229in}}%
\pgfpathlineto{\pgfqpoint{4.303826in}{2.307890in}}%
\pgfpathlineto{\pgfqpoint{4.301015in}{2.307790in}}%
\pgfpathlineto{\pgfqpoint{4.298204in}{2.308000in}}%
\pgfpathlineto{\pgfqpoint{4.295394in}{2.307886in}}%
\pgfpathlineto{\pgfqpoint{4.292583in}{2.308096in}}%
\pgfpathlineto{\pgfqpoint{4.289772in}{2.308304in}}%
\pgfpathlineto{\pgfqpoint{4.286962in}{2.308362in}}%
\pgfpathlineto{\pgfqpoint{4.284151in}{2.308559in}}%
\pgfpathlineto{\pgfqpoint{4.281340in}{2.308726in}}%
\pgfpathlineto{\pgfqpoint{4.278530in}{2.308877in}}%
\pgfpathlineto{\pgfqpoint{4.275719in}{2.308789in}}%
\pgfpathlineto{\pgfqpoint{4.272908in}{2.308995in}}%
\pgfpathlineto{\pgfqpoint{4.270098in}{2.309197in}}%
\pgfpathlineto{\pgfqpoint{4.267287in}{2.309125in}}%
\pgfpathlineto{\pgfqpoint{4.264476in}{2.309332in}}%
\pgfpathlineto{\pgfqpoint{4.261665in}{2.309387in}}%
\pgfpathlineto{\pgfqpoint{4.258855in}{2.309218in}}%
\pgfpathlineto{\pgfqpoint{4.256044in}{2.309381in}}%
\pgfpathlineto{\pgfqpoint{4.253233in}{2.309592in}}%
\pgfpathlineto{\pgfqpoint{4.250423in}{2.309735in}}%
\pgfpathlineto{\pgfqpoint{4.247612in}{2.309754in}}%
\pgfpathlineto{\pgfqpoint{4.244801in}{2.309940in}}%
\pgfpathlineto{\pgfqpoint{4.241991in}{2.310118in}}%
\pgfpathlineto{\pgfqpoint{4.239180in}{2.309172in}}%
\pgfpathlineto{\pgfqpoint{4.236369in}{2.308445in}}%
\pgfpathlineto{\pgfqpoint{4.233559in}{2.308621in}}%
\pgfpathlineto{\pgfqpoint{4.230748in}{2.308806in}}%
\pgfpathlineto{\pgfqpoint{4.227937in}{2.308493in}}%
\pgfpathlineto{\pgfqpoint{4.225127in}{2.308313in}}%
\pgfpathlineto{\pgfqpoint{4.222316in}{2.308497in}}%
\pgfpathlineto{\pgfqpoint{4.219505in}{2.308706in}}%
\pgfpathlineto{\pgfqpoint{4.216695in}{2.308558in}}%
\pgfpathlineto{\pgfqpoint{4.213884in}{2.308719in}}%
\pgfpathlineto{\pgfqpoint{4.211073in}{2.308846in}}%
\pgfpathlineto{\pgfqpoint{4.208262in}{2.308936in}}%
\pgfpathlineto{\pgfqpoint{4.205452in}{2.309050in}}%
\pgfpathlineto{\pgfqpoint{4.202641in}{2.309081in}}%
\pgfpathlineto{\pgfqpoint{4.199830in}{2.309182in}}%
\pgfpathlineto{\pgfqpoint{4.197020in}{2.309383in}}%
\pgfpathlineto{\pgfqpoint{4.194209in}{2.309606in}}%
\pgfpathlineto{\pgfqpoint{4.191398in}{2.309772in}}%
\pgfpathlineto{\pgfqpoint{4.188588in}{2.309904in}}%
\pgfpathlineto{\pgfqpoint{4.185777in}{2.309941in}}%
\pgfpathlineto{\pgfqpoint{4.182966in}{2.310138in}}%
\pgfpathlineto{\pgfqpoint{4.180156in}{2.310361in}}%
\pgfpathlineto{\pgfqpoint{4.177345in}{2.310373in}}%
\pgfpathlineto{\pgfqpoint{4.174534in}{2.310357in}}%
\pgfpathlineto{\pgfqpoint{4.171724in}{2.310570in}}%
\pgfpathlineto{\pgfqpoint{4.168913in}{2.310684in}}%
\pgfpathlineto{\pgfqpoint{4.166102in}{2.310892in}}%
\pgfpathlineto{\pgfqpoint{4.163291in}{2.311107in}}%
\pgfpathlineto{\pgfqpoint{4.160481in}{2.311271in}}%
\pgfpathlineto{\pgfqpoint{4.157670in}{2.311198in}}%
\pgfpathlineto{\pgfqpoint{4.154859in}{2.311179in}}%
\pgfpathlineto{\pgfqpoint{4.152049in}{2.311381in}}%
\pgfpathlineto{\pgfqpoint{4.149238in}{2.311605in}}%
\pgfpathlineto{\pgfqpoint{4.146427in}{2.311791in}}%
\pgfpathlineto{\pgfqpoint{4.143617in}{2.311941in}}%
\pgfpathlineto{\pgfqpoint{4.140806in}{2.312147in}}%
\pgfpathlineto{\pgfqpoint{4.137995in}{2.312375in}}%
\pgfpathlineto{\pgfqpoint{4.135185in}{2.312300in}}%
\pgfpathlineto{\pgfqpoint{4.132374in}{2.312474in}}%
\pgfpathlineto{\pgfqpoint{4.129563in}{2.312405in}}%
\pgfpathlineto{\pgfqpoint{4.126753in}{2.311589in}}%
\pgfpathlineto{\pgfqpoint{4.123942in}{2.311373in}}%
\pgfpathlineto{\pgfqpoint{4.121131in}{2.311519in}}%
\pgfpathlineto{\pgfqpoint{4.118320in}{2.311714in}}%
\pgfpathlineto{\pgfqpoint{4.115510in}{2.311935in}}%
\pgfpathlineto{\pgfqpoint{4.112699in}{2.312025in}}%
\pgfpathlineto{\pgfqpoint{4.109888in}{2.312187in}}%
\pgfpathlineto{\pgfqpoint{4.107078in}{2.311931in}}%
\pgfpathlineto{\pgfqpoint{4.104267in}{2.311878in}}%
\pgfpathlineto{\pgfqpoint{4.101456in}{2.311244in}}%
\pgfpathlineto{\pgfqpoint{4.098646in}{2.311303in}}%
\pgfpathlineto{\pgfqpoint{4.095835in}{2.311315in}}%
\pgfpathlineto{\pgfqpoint{4.093024in}{2.311525in}}%
\pgfpathlineto{\pgfqpoint{4.090214in}{2.311732in}}%
\pgfpathlineto{\pgfqpoint{4.087403in}{2.310921in}}%
\pgfpathlineto{\pgfqpoint{4.084592in}{2.310309in}}%
\pgfpathlineto{\pgfqpoint{4.081782in}{2.310097in}}%
\pgfpathlineto{\pgfqpoint{4.078971in}{2.309972in}}%
\pgfpathlineto{\pgfqpoint{4.076160in}{2.308891in}}%
\pgfpathlineto{\pgfqpoint{4.073349in}{2.308389in}}%
\pgfpathlineto{\pgfqpoint{4.070539in}{2.308589in}}%
\pgfpathlineto{\pgfqpoint{4.067728in}{2.308697in}}%
\pgfpathlineto{\pgfqpoint{4.064917in}{2.308567in}}%
\pgfpathlineto{\pgfqpoint{4.062107in}{2.308726in}}%
\pgfpathlineto{\pgfqpoint{4.059296in}{2.308879in}}%
\pgfpathlineto{\pgfqpoint{4.056485in}{2.309027in}}%
\pgfpathlineto{\pgfqpoint{4.053675in}{2.309229in}}%
\pgfpathlineto{\pgfqpoint{4.050864in}{2.309352in}}%
\pgfpathlineto{\pgfqpoint{4.048053in}{2.309494in}}%
\pgfpathlineto{\pgfqpoint{4.045243in}{2.309660in}}%
\pgfpathlineto{\pgfqpoint{4.042432in}{2.309847in}}%
\pgfpathlineto{\pgfqpoint{4.039621in}{2.309923in}}%
\pgfpathlineto{\pgfqpoint{4.036811in}{2.310084in}}%
\pgfpathlineto{\pgfqpoint{4.034000in}{2.310314in}}%
\pgfpathlineto{\pgfqpoint{4.031189in}{2.310457in}}%
\pgfpathlineto{\pgfqpoint{4.028378in}{2.310619in}}%
\pgfpathlineto{\pgfqpoint{4.025568in}{2.310801in}}%
\pgfpathlineto{\pgfqpoint{4.022757in}{2.310957in}}%
\pgfpathlineto{\pgfqpoint{4.019946in}{2.311094in}}%
\pgfpathlineto{\pgfqpoint{4.017136in}{2.311050in}}%
\pgfpathlineto{\pgfqpoint{4.014325in}{2.310211in}}%
\pgfpathlineto{\pgfqpoint{4.011514in}{2.310365in}}%
\pgfpathlineto{\pgfqpoint{4.008704in}{2.310043in}}%
\pgfpathlineto{\pgfqpoint{4.005893in}{2.310256in}}%
\pgfpathlineto{\pgfqpoint{4.003082in}{2.310445in}}%
\pgfpathlineto{\pgfqpoint{4.000272in}{2.310639in}}%
\pgfpathlineto{\pgfqpoint{3.997461in}{2.310388in}}%
\pgfpathlineto{\pgfqpoint{3.994650in}{2.310491in}}%
\pgfpathlineto{\pgfqpoint{3.991840in}{2.310722in}}%
\pgfpathlineto{\pgfqpoint{3.989029in}{2.310946in}}%
\pgfpathlineto{\pgfqpoint{3.986218in}{2.310873in}}%
\pgfpathlineto{\pgfqpoint{3.983408in}{2.311112in}}%
\pgfpathlineto{\pgfqpoint{3.980597in}{2.311224in}}%
\pgfpathlineto{\pgfqpoint{3.977786in}{2.311392in}}%
\pgfpathlineto{\pgfqpoint{3.974975in}{2.311631in}}%
\pgfpathlineto{\pgfqpoint{3.972165in}{2.311871in}}%
\pgfpathlineto{\pgfqpoint{3.969354in}{2.311814in}}%
\pgfpathlineto{\pgfqpoint{3.966543in}{2.311678in}}%
\pgfpathlineto{\pgfqpoint{3.963733in}{2.311750in}}%
\pgfpathlineto{\pgfqpoint{3.960922in}{2.311695in}}%
\pgfpathlineto{\pgfqpoint{3.958111in}{2.311887in}}%
\pgfpathlineto{\pgfqpoint{3.955301in}{2.312128in}}%
\pgfpathlineto{\pgfqpoint{3.952490in}{2.312359in}}%
\pgfpathlineto{\pgfqpoint{3.949679in}{2.312570in}}%
\pgfpathlineto{\pgfqpoint{3.946869in}{2.312457in}}%
\pgfpathlineto{\pgfqpoint{3.944058in}{2.312671in}}%
\pgfpathlineto{\pgfqpoint{3.941247in}{2.312758in}}%
\pgfpathlineto{\pgfqpoint{3.938437in}{2.312988in}}%
\pgfpathlineto{\pgfqpoint{3.935626in}{2.313221in}}%
\pgfpathlineto{\pgfqpoint{3.932815in}{2.313237in}}%
\pgfpathlineto{\pgfqpoint{3.930004in}{2.313326in}}%
\pgfpathlineto{\pgfqpoint{3.927194in}{2.313511in}}%
\pgfpathlineto{\pgfqpoint{3.924383in}{2.313666in}}%
\pgfpathlineto{\pgfqpoint{3.921572in}{2.313823in}}%
\pgfpathlineto{\pgfqpoint{3.918762in}{2.313921in}}%
\pgfpathlineto{\pgfqpoint{3.915951in}{2.314154in}}%
\pgfpathlineto{\pgfqpoint{3.913140in}{2.314352in}}%
\pgfpathlineto{\pgfqpoint{3.910330in}{2.314505in}}%
\pgfpathlineto{\pgfqpoint{3.907519in}{2.314734in}}%
\pgfpathlineto{\pgfqpoint{3.904708in}{2.314703in}}%
\pgfpathlineto{\pgfqpoint{3.901898in}{2.314487in}}%
\pgfpathlineto{\pgfqpoint{3.899087in}{2.314680in}}%
\pgfpathlineto{\pgfqpoint{3.896276in}{2.314919in}}%
\pgfpathlineto{\pgfqpoint{3.893466in}{2.315159in}}%
\pgfpathlineto{\pgfqpoint{3.890655in}{2.315371in}}%
\pgfpathlineto{\pgfqpoint{3.887844in}{2.315608in}}%
\pgfpathlineto{\pgfqpoint{3.885033in}{2.315063in}}%
\pgfpathlineto{\pgfqpoint{3.882223in}{2.315069in}}%
\pgfpathlineto{\pgfqpoint{3.879412in}{2.315252in}}%
\pgfpathlineto{\pgfqpoint{3.876601in}{2.315315in}}%
\pgfpathlineto{\pgfqpoint{3.873791in}{2.315554in}}%
\pgfpathlineto{\pgfqpoint{3.870980in}{2.315292in}}%
\pgfpathlineto{\pgfqpoint{3.868169in}{2.315194in}}%
\pgfpathlineto{\pgfqpoint{3.865359in}{2.315153in}}%
\pgfpathlineto{\pgfqpoint{3.862548in}{2.315375in}}%
\pgfpathlineto{\pgfqpoint{3.859737in}{2.315347in}}%
\pgfpathlineto{\pgfqpoint{3.856927in}{2.315525in}}%
\pgfpathlineto{\pgfqpoint{3.854116in}{2.315715in}}%
\pgfpathlineto{\pgfqpoint{3.851305in}{2.315810in}}%
\pgfpathlineto{\pgfqpoint{3.848495in}{2.314998in}}%
\pgfpathlineto{\pgfqpoint{3.845684in}{2.315252in}}%
\pgfpathlineto{\pgfqpoint{3.842873in}{2.315342in}}%
\pgfpathlineto{\pgfqpoint{3.840062in}{2.315140in}}%
\pgfpathlineto{\pgfqpoint{3.837252in}{2.314976in}}%
\pgfpathlineto{\pgfqpoint{3.834441in}{2.315092in}}%
\pgfpathlineto{\pgfqpoint{3.831630in}{2.314522in}}%
\pgfpathlineto{\pgfqpoint{3.828820in}{2.314473in}}%
\pgfpathlineto{\pgfqpoint{3.826009in}{2.314698in}}%
\pgfpathlineto{\pgfqpoint{3.823198in}{2.314042in}}%
\pgfpathlineto{\pgfqpoint{3.820388in}{2.314155in}}%
\pgfpathlineto{\pgfqpoint{3.817577in}{2.313944in}}%
\pgfpathlineto{\pgfqpoint{3.814766in}{2.314157in}}%
\pgfpathlineto{\pgfqpoint{3.811956in}{2.314246in}}%
\pgfpathlineto{\pgfqpoint{3.809145in}{2.314458in}}%
\pgfpathlineto{\pgfqpoint{3.806334in}{2.314715in}}%
\pgfpathlineto{\pgfqpoint{3.803524in}{2.314649in}}%
\pgfpathlineto{\pgfqpoint{3.800713in}{2.314431in}}%
\pgfpathlineto{\pgfqpoint{3.797902in}{2.314688in}}%
\pgfpathlineto{\pgfqpoint{3.795092in}{2.314236in}}%
\pgfpathlineto{\pgfqpoint{3.792281in}{2.314493in}}%
\pgfpathlineto{\pgfqpoint{3.789470in}{2.313028in}}%
\pgfpathlineto{\pgfqpoint{3.786659in}{2.313137in}}%
\pgfpathlineto{\pgfqpoint{3.783849in}{2.313387in}}%
\pgfpathlineto{\pgfqpoint{3.781038in}{2.313315in}}%
\pgfpathlineto{\pgfqpoint{3.778227in}{2.313559in}}%
\pgfpathlineto{\pgfqpoint{3.775417in}{2.313563in}}%
\pgfpathlineto{\pgfqpoint{3.772606in}{2.313737in}}%
\pgfpathlineto{\pgfqpoint{3.769795in}{2.313996in}}%
\pgfpathlineto{\pgfqpoint{3.766985in}{2.314184in}}%
\pgfpathlineto{\pgfqpoint{3.764174in}{2.306494in}}%
\pgfpathlineto{\pgfqpoint{3.761363in}{2.305684in}}%
\pgfpathlineto{\pgfqpoint{3.758553in}{2.305641in}}%
\pgfpathlineto{\pgfqpoint{3.755742in}{2.305201in}}%
\pgfpathlineto{\pgfqpoint{3.752931in}{2.304867in}}%
\pgfpathlineto{\pgfqpoint{3.750121in}{2.305110in}}%
\pgfpathlineto{\pgfqpoint{3.747310in}{2.304328in}}%
\pgfpathlineto{\pgfqpoint{3.744499in}{2.304320in}}%
\pgfpathlineto{\pgfqpoint{3.741688in}{2.304571in}}%
\pgfpathlineto{\pgfqpoint{3.738878in}{2.304775in}}%
\pgfpathlineto{\pgfqpoint{3.736067in}{2.304951in}}%
\pgfpathlineto{\pgfqpoint{3.733256in}{2.305165in}}%
\pgfpathlineto{\pgfqpoint{3.730446in}{2.304954in}}%
\pgfpathlineto{\pgfqpoint{3.727635in}{2.305181in}}%
\pgfpathlineto{\pgfqpoint{3.724824in}{2.305400in}}%
\pgfpathlineto{\pgfqpoint{3.722014in}{2.305443in}}%
\pgfpathlineto{\pgfqpoint{3.719203in}{2.305246in}}%
\pgfpathlineto{\pgfqpoint{3.716392in}{2.305174in}}%
\pgfpathlineto{\pgfqpoint{3.713582in}{2.304840in}}%
\pgfpathlineto{\pgfqpoint{3.710771in}{2.305077in}}%
\pgfpathlineto{\pgfqpoint{3.707960in}{2.305228in}}%
\pgfpathlineto{\pgfqpoint{3.705150in}{2.304715in}}%
\pgfpathlineto{\pgfqpoint{3.702339in}{2.304810in}}%
\pgfpathlineto{\pgfqpoint{3.699528in}{2.305003in}}%
\pgfpathlineto{\pgfqpoint{3.696717in}{2.305232in}}%
\pgfpathlineto{\pgfqpoint{3.693907in}{2.305420in}}%
\pgfpathlineto{\pgfqpoint{3.691096in}{2.305680in}}%
\pgfpathlineto{\pgfqpoint{3.688285in}{2.305875in}}%
\pgfpathlineto{\pgfqpoint{3.685475in}{2.304999in}}%
\pgfpathlineto{\pgfqpoint{3.682664in}{2.304978in}}%
\pgfpathlineto{\pgfqpoint{3.679853in}{2.305238in}}%
\pgfpathlineto{\pgfqpoint{3.677043in}{2.305232in}}%
\pgfpathlineto{\pgfqpoint{3.674232in}{2.305233in}}%
\pgfpathlineto{\pgfqpoint{3.671421in}{2.305448in}}%
\pgfpathlineto{\pgfqpoint{3.668611in}{2.305613in}}%
\pgfpathlineto{\pgfqpoint{3.665800in}{2.305866in}}%
\pgfpathlineto{\pgfqpoint{3.662989in}{2.306126in}}%
\pgfpathlineto{\pgfqpoint{3.660179in}{2.306082in}}%
\pgfpathlineto{\pgfqpoint{3.657368in}{2.304581in}}%
\pgfpathlineto{\pgfqpoint{3.654557in}{2.304298in}}%
\pgfpathlineto{\pgfqpoint{3.651746in}{2.303984in}}%
\pgfpathlineto{\pgfqpoint{3.648936in}{2.304205in}}%
\pgfpathlineto{\pgfqpoint{3.646125in}{2.304463in}}%
\pgfpathlineto{\pgfqpoint{3.643314in}{2.304671in}}%
\pgfpathlineto{\pgfqpoint{3.640504in}{2.304918in}}%
\pgfpathlineto{\pgfqpoint{3.637693in}{2.305175in}}%
\pgfpathlineto{\pgfqpoint{3.634882in}{2.305368in}}%
\pgfpathlineto{\pgfqpoint{3.632072in}{2.305629in}}%
\pgfpathlineto{\pgfqpoint{3.629261in}{2.305890in}}%
\pgfpathlineto{\pgfqpoint{3.626450in}{2.306064in}}%
\pgfpathlineto{\pgfqpoint{3.623640in}{2.306161in}}%
\pgfpathlineto{\pgfqpoint{3.620829in}{2.306219in}}%
\pgfpathlineto{\pgfqpoint{3.618018in}{2.306483in}}%
\pgfpathlineto{\pgfqpoint{3.615208in}{2.306562in}}%
\pgfpathlineto{\pgfqpoint{3.612397in}{2.306634in}}%
\pgfpathlineto{\pgfqpoint{3.609586in}{2.306890in}}%
\pgfpathlineto{\pgfqpoint{3.606776in}{2.307153in}}%
\pgfpathlineto{\pgfqpoint{3.603965in}{2.306934in}}%
\pgfpathlineto{\pgfqpoint{3.601154in}{2.305705in}}%
\pgfpathlineto{\pgfqpoint{3.598343in}{2.305951in}}%
\pgfpathlineto{\pgfqpoint{3.595533in}{2.306216in}}%
\pgfpathlineto{\pgfqpoint{3.592722in}{2.305900in}}%
\pgfpathlineto{\pgfqpoint{3.589911in}{2.305906in}}%
\pgfpathlineto{\pgfqpoint{3.587101in}{2.306146in}}%
\pgfpathlineto{\pgfqpoint{3.584290in}{2.305417in}}%
\pgfpathlineto{\pgfqpoint{3.581479in}{2.305686in}}%
\pgfpathlineto{\pgfqpoint{3.578669in}{2.305872in}}%
\pgfpathlineto{\pgfqpoint{3.575858in}{2.306117in}}%
\pgfpathlineto{\pgfqpoint{3.573047in}{2.306205in}}%
\pgfpathlineto{\pgfqpoint{3.570237in}{2.305794in}}%
\pgfpathlineto{\pgfqpoint{3.567426in}{2.305460in}}%
\pgfpathlineto{\pgfqpoint{3.564615in}{2.304764in}}%
\pgfpathlineto{\pgfqpoint{3.561805in}{2.304430in}}%
\pgfpathlineto{\pgfqpoint{3.558994in}{2.304697in}}%
\pgfpathlineto{\pgfqpoint{3.556183in}{2.303455in}}%
\pgfpathlineto{\pgfqpoint{3.553372in}{2.303120in}}%
\pgfpathlineto{\pgfqpoint{3.550562in}{2.303363in}}%
\pgfpathlineto{\pgfqpoint{3.547751in}{2.303270in}}%
\pgfpathlineto{\pgfqpoint{3.544940in}{2.303540in}}%
\pgfpathlineto{\pgfqpoint{3.542130in}{2.303721in}}%
\pgfpathlineto{\pgfqpoint{3.539319in}{2.303881in}}%
\pgfpathlineto{\pgfqpoint{3.536508in}{2.304000in}}%
\pgfpathlineto{\pgfqpoint{3.533698in}{2.303960in}}%
\pgfpathlineto{\pgfqpoint{3.530887in}{2.304209in}}%
\pgfpathlineto{\pgfqpoint{3.528076in}{2.304416in}}%
\pgfpathlineto{\pgfqpoint{3.525266in}{2.304307in}}%
\pgfpathlineto{\pgfqpoint{3.522455in}{2.304566in}}%
\pgfpathlineto{\pgfqpoint{3.519644in}{2.304633in}}%
\pgfpathlineto{\pgfqpoint{3.516834in}{2.304847in}}%
\pgfpathlineto{\pgfqpoint{3.514023in}{2.305122in}}%
\pgfpathlineto{\pgfqpoint{3.511212in}{2.303526in}}%
\pgfpathlineto{\pgfqpoint{3.508401in}{2.303625in}}%
\pgfpathlineto{\pgfqpoint{3.505591in}{2.303299in}}%
\pgfpathlineto{\pgfqpoint{3.502780in}{2.302219in}}%
\pgfpathlineto{\pgfqpoint{3.499969in}{2.302105in}}%
\pgfpathlineto{\pgfqpoint{3.497159in}{2.302243in}}%
\pgfpathlineto{\pgfqpoint{3.494348in}{2.301388in}}%
\pgfpathlineto{\pgfqpoint{3.491537in}{2.301099in}}%
\pgfpathlineto{\pgfqpoint{3.488727in}{2.297140in}}%
\pgfpathlineto{\pgfqpoint{3.485916in}{2.297348in}}%
\pgfpathlineto{\pgfqpoint{3.483105in}{2.297597in}}%
\pgfpathlineto{\pgfqpoint{3.480295in}{2.296888in}}%
\pgfpathlineto{\pgfqpoint{3.477484in}{2.296556in}}%
\pgfpathlineto{\pgfqpoint{3.474673in}{2.296654in}}%
\pgfpathlineto{\pgfqpoint{3.471863in}{2.296931in}}%
\pgfpathlineto{\pgfqpoint{3.469052in}{2.297122in}}%
\pgfpathlineto{\pgfqpoint{3.466241in}{2.297241in}}%
\pgfpathlineto{\pgfqpoint{3.463430in}{2.297314in}}%
\pgfpathlineto{\pgfqpoint{3.460620in}{2.296965in}}%
\pgfpathlineto{\pgfqpoint{3.457809in}{2.296936in}}%
\pgfpathlineto{\pgfqpoint{3.454998in}{2.295960in}}%
\pgfpathlineto{\pgfqpoint{3.452188in}{2.296233in}}%
\pgfpathlineto{\pgfqpoint{3.449377in}{2.296503in}}%
\pgfpathlineto{\pgfqpoint{3.446566in}{2.296526in}}%
\pgfpathlineto{\pgfqpoint{3.443756in}{2.296702in}}%
\pgfpathlineto{\pgfqpoint{3.440945in}{2.296861in}}%
\pgfpathlineto{\pgfqpoint{3.438134in}{2.297070in}}%
\pgfpathlineto{\pgfqpoint{3.435324in}{2.297340in}}%
\pgfpathlineto{\pgfqpoint{3.432513in}{2.297534in}}%
\pgfpathlineto{\pgfqpoint{3.429702in}{2.296890in}}%
\pgfpathlineto{\pgfqpoint{3.426892in}{2.296841in}}%
\pgfpathlineto{\pgfqpoint{3.424081in}{2.296954in}}%
\pgfpathlineto{\pgfqpoint{3.421270in}{2.297190in}}%
\pgfpathlineto{\pgfqpoint{3.418459in}{2.297465in}}%
\pgfpathlineto{\pgfqpoint{3.415649in}{2.297699in}}%
\pgfpathlineto{\pgfqpoint{3.412838in}{2.297941in}}%
\pgfpathlineto{\pgfqpoint{3.410027in}{2.298155in}}%
\pgfpathlineto{\pgfqpoint{3.407217in}{2.298211in}}%
\pgfpathlineto{\pgfqpoint{3.404406in}{2.298424in}}%
\pgfpathlineto{\pgfqpoint{3.401595in}{2.298691in}}%
\pgfpathlineto{\pgfqpoint{3.398785in}{2.298178in}}%
\pgfpathlineto{\pgfqpoint{3.395974in}{2.298201in}}%
\pgfpathlineto{\pgfqpoint{3.393163in}{2.298298in}}%
\pgfpathlineto{\pgfqpoint{3.390353in}{2.298527in}}%
\pgfpathlineto{\pgfqpoint{3.387542in}{2.298810in}}%
\pgfpathlineto{\pgfqpoint{3.384731in}{2.298967in}}%
\pgfpathlineto{\pgfqpoint{3.381921in}{2.298824in}}%
\pgfpathlineto{\pgfqpoint{3.379110in}{2.299112in}}%
\pgfpathlineto{\pgfqpoint{3.376299in}{2.299329in}}%
\pgfpathlineto{\pgfqpoint{3.373489in}{2.299384in}}%
\pgfpathlineto{\pgfqpoint{3.370678in}{2.298965in}}%
\pgfpathlineto{\pgfqpoint{3.367867in}{2.299156in}}%
\pgfpathlineto{\pgfqpoint{3.365056in}{2.299445in}}%
\pgfpathlineto{\pgfqpoint{3.362246in}{2.299734in}}%
\pgfpathlineto{\pgfqpoint{3.359435in}{2.299834in}}%
\pgfpathlineto{\pgfqpoint{3.356624in}{2.300113in}}%
\pgfpathlineto{\pgfqpoint{3.353814in}{2.299922in}}%
\pgfpathlineto{\pgfqpoint{3.351003in}{2.299881in}}%
\pgfpathlineto{\pgfqpoint{3.348192in}{2.299918in}}%
\pgfpathlineto{\pgfqpoint{3.345382in}{2.300117in}}%
\pgfpathlineto{\pgfqpoint{3.342571in}{2.300285in}}%
\pgfpathlineto{\pgfqpoint{3.339760in}{2.300507in}}%
\pgfpathlineto{\pgfqpoint{3.336950in}{2.300066in}}%
\pgfpathlineto{\pgfqpoint{3.334139in}{2.300328in}}%
\pgfpathlineto{\pgfqpoint{3.331328in}{2.300487in}}%
\pgfpathlineto{\pgfqpoint{3.328518in}{2.300701in}}%
\pgfpathlineto{\pgfqpoint{3.325707in}{2.300949in}}%
\pgfpathlineto{\pgfqpoint{3.322896in}{2.301126in}}%
\pgfpathlineto{\pgfqpoint{3.320085in}{2.301226in}}%
\pgfpathlineto{\pgfqpoint{3.317275in}{2.301415in}}%
\pgfpathlineto{\pgfqpoint{3.314464in}{2.301232in}}%
\pgfpathlineto{\pgfqpoint{3.311653in}{2.301244in}}%
\pgfpathlineto{\pgfqpoint{3.308843in}{2.300780in}}%
\pgfpathlineto{\pgfqpoint{3.306032in}{2.301058in}}%
\pgfpathlineto{\pgfqpoint{3.303221in}{2.301342in}}%
\pgfpathlineto{\pgfqpoint{3.300411in}{2.301464in}}%
\pgfpathlineto{\pgfqpoint{3.297600in}{2.301542in}}%
\pgfpathlineto{\pgfqpoint{3.294789in}{2.301841in}}%
\pgfpathlineto{\pgfqpoint{3.291979in}{2.301685in}}%
\pgfpathlineto{\pgfqpoint{3.289168in}{2.301920in}}%
\pgfpathlineto{\pgfqpoint{3.286357in}{2.301880in}}%
\pgfpathlineto{\pgfqpoint{3.283547in}{2.301741in}}%
\pgfpathlineto{\pgfqpoint{3.280736in}{2.301983in}}%
\pgfpathlineto{\pgfqpoint{3.277925in}{2.302274in}}%
\pgfpathlineto{\pgfqpoint{3.275114in}{2.301858in}}%
\pgfpathlineto{\pgfqpoint{3.272304in}{2.302101in}}%
\pgfpathlineto{\pgfqpoint{3.269493in}{2.302356in}}%
\pgfpathlineto{\pgfqpoint{3.266682in}{2.302529in}}%
\pgfpathlineto{\pgfqpoint{3.263872in}{2.302774in}}%
\pgfpathlineto{\pgfqpoint{3.261061in}{2.301696in}}%
\pgfpathlineto{\pgfqpoint{3.258250in}{2.301982in}}%
\pgfpathlineto{\pgfqpoint{3.255440in}{2.302281in}}%
\pgfpathlineto{\pgfqpoint{3.252629in}{2.301943in}}%
\pgfpathlineto{\pgfqpoint{3.249818in}{2.301761in}}%
\pgfpathlineto{\pgfqpoint{3.247008in}{2.302064in}}%
\pgfpathlineto{\pgfqpoint{3.244197in}{2.302352in}}%
\pgfpathlineto{\pgfqpoint{3.241386in}{2.302388in}}%
\pgfpathlineto{\pgfqpoint{3.238576in}{2.302689in}}%
\pgfpathlineto{\pgfqpoint{3.235765in}{2.302726in}}%
\pgfpathlineto{\pgfqpoint{3.232954in}{2.302996in}}%
\pgfpathlineto{\pgfqpoint{3.230143in}{2.303231in}}%
\pgfpathlineto{\pgfqpoint{3.227333in}{2.300553in}}%
\pgfpathlineto{\pgfqpoint{3.224522in}{2.300862in}}%
\pgfpathlineto{\pgfqpoint{3.221711in}{2.301171in}}%
\pgfpathlineto{\pgfqpoint{3.218901in}{2.300432in}}%
\pgfpathlineto{\pgfqpoint{3.216090in}{2.300717in}}%
\pgfpathlineto{\pgfqpoint{3.213279in}{2.300943in}}%
\pgfpathlineto{\pgfqpoint{3.210469in}{2.301005in}}%
\pgfpathlineto{\pgfqpoint{3.207658in}{2.301064in}}%
\pgfpathlineto{\pgfqpoint{3.204847in}{2.301319in}}%
\pgfpathlineto{\pgfqpoint{3.202037in}{2.301550in}}%
\pgfpathlineto{\pgfqpoint{3.199226in}{2.301863in}}%
\pgfpathlineto{\pgfqpoint{3.196415in}{2.302032in}}%
\pgfpathlineto{\pgfqpoint{3.193605in}{2.296216in}}%
\pgfpathlineto{\pgfqpoint{3.190794in}{2.296303in}}%
\pgfpathlineto{\pgfqpoint{3.187983in}{2.295916in}}%
\pgfpathlineto{\pgfqpoint{3.185173in}{2.296169in}}%
\pgfpathlineto{\pgfqpoint{3.182362in}{2.296470in}}%
\pgfpathlineto{\pgfqpoint{3.179551in}{2.295963in}}%
\pgfpathlineto{\pgfqpoint{3.176740in}{2.296272in}}%
\pgfpathlineto{\pgfqpoint{3.173930in}{2.296238in}}%
\pgfpathlineto{\pgfqpoint{3.171119in}{2.296549in}}%
\pgfpathlineto{\pgfqpoint{3.168308in}{2.296846in}}%
\pgfpathlineto{\pgfqpoint{3.165498in}{2.297011in}}%
\pgfpathlineto{\pgfqpoint{3.162687in}{2.294956in}}%
\pgfpathlineto{\pgfqpoint{3.159876in}{2.294337in}}%
\pgfpathlineto{\pgfqpoint{3.157066in}{2.293787in}}%
\pgfpathlineto{\pgfqpoint{3.154255in}{2.294011in}}%
\pgfpathlineto{\pgfqpoint{3.151444in}{2.294201in}}%
\pgfpathlineto{\pgfqpoint{3.148634in}{2.294480in}}%
\pgfpathlineto{\pgfqpoint{3.145823in}{2.294733in}}%
\pgfpathlineto{\pgfqpoint{3.143012in}{2.294967in}}%
\pgfpathlineto{\pgfqpoint{3.140202in}{2.294030in}}%
\pgfpathlineto{\pgfqpoint{3.137391in}{2.294304in}}%
\pgfpathlineto{\pgfqpoint{3.134580in}{2.294612in}}%
\pgfpathlineto{\pgfqpoint{3.131769in}{2.294856in}}%
\pgfpathlineto{\pgfqpoint{3.128959in}{2.294293in}}%
\pgfpathlineto{\pgfqpoint{3.126148in}{2.294390in}}%
\pgfpathlineto{\pgfqpoint{3.123337in}{2.294500in}}%
\pgfpathlineto{\pgfqpoint{3.120527in}{2.294791in}}%
\pgfpathlineto{\pgfqpoint{3.117716in}{2.295039in}}%
\pgfpathlineto{\pgfqpoint{3.114905in}{2.294951in}}%
\pgfpathlineto{\pgfqpoint{3.112095in}{2.295026in}}%
\pgfpathlineto{\pgfqpoint{3.109284in}{2.294585in}}%
\pgfpathlineto{\pgfqpoint{3.106473in}{2.294899in}}%
\pgfpathlineto{\pgfqpoint{3.103663in}{2.294038in}}%
\pgfpathlineto{\pgfqpoint{3.100852in}{2.294329in}}%
\pgfpathlineto{\pgfqpoint{3.098041in}{2.294420in}}%
\pgfpathlineto{\pgfqpoint{3.095231in}{2.294365in}}%
\pgfpathlineto{\pgfqpoint{3.092420in}{2.294525in}}%
\pgfpathlineto{\pgfqpoint{3.089609in}{2.294765in}}%
\pgfpathlineto{\pgfqpoint{3.086798in}{2.294246in}}%
\pgfpathlineto{\pgfqpoint{3.083988in}{2.293445in}}%
\pgfpathlineto{\pgfqpoint{3.081177in}{2.293255in}}%
\pgfpathlineto{\pgfqpoint{3.078366in}{2.293531in}}%
\pgfpathlineto{\pgfqpoint{3.075556in}{2.291578in}}%
\pgfpathlineto{\pgfqpoint{3.072745in}{2.291710in}}%
\pgfpathlineto{\pgfqpoint{3.069934in}{2.291975in}}%
\pgfpathlineto{\pgfqpoint{3.067124in}{2.291835in}}%
\pgfpathlineto{\pgfqpoint{3.064313in}{2.291750in}}%
\pgfpathlineto{\pgfqpoint{3.061502in}{2.292052in}}%
\pgfpathlineto{\pgfqpoint{3.058692in}{2.292118in}}%
\pgfpathlineto{\pgfqpoint{3.055881in}{2.292437in}}%
\pgfpathlineto{\pgfqpoint{3.053070in}{2.292756in}}%
\pgfpathlineto{\pgfqpoint{3.050260in}{2.293065in}}%
\pgfpathlineto{\pgfqpoint{3.047449in}{2.293386in}}%
\pgfpathlineto{\pgfqpoint{3.044638in}{2.293620in}}%
\pgfpathlineto{\pgfqpoint{3.041827in}{2.293067in}}%
\pgfpathlineto{\pgfqpoint{3.039017in}{2.292340in}}%
\pgfpathlineto{\pgfqpoint{3.036206in}{2.292434in}}%
\pgfpathlineto{\pgfqpoint{3.033395in}{2.292723in}}%
\pgfpathlineto{\pgfqpoint{3.030585in}{2.292957in}}%
\pgfpathlineto{\pgfqpoint{3.027774in}{2.293222in}}%
\pgfpathlineto{\pgfqpoint{3.024963in}{2.293511in}}%
\pgfpathlineto{\pgfqpoint{3.022153in}{2.293821in}}%
\pgfpathlineto{\pgfqpoint{3.019342in}{2.294029in}}%
\pgfpathlineto{\pgfqpoint{3.016531in}{2.294305in}}%
\pgfpathlineto{\pgfqpoint{3.013721in}{2.294417in}}%
\pgfpathlineto{\pgfqpoint{3.010910in}{2.293536in}}%
\pgfpathlineto{\pgfqpoint{3.008099in}{2.293673in}}%
\pgfpathlineto{\pgfqpoint{3.005289in}{2.292882in}}%
\pgfpathlineto{\pgfqpoint{3.002478in}{2.292300in}}%
\pgfpathlineto{\pgfqpoint{2.999667in}{2.292613in}}%
\pgfpathlineto{\pgfqpoint{2.996856in}{2.292936in}}%
\pgfpathlineto{\pgfqpoint{2.994046in}{2.293204in}}%
\pgfpathlineto{\pgfqpoint{2.991235in}{2.293251in}}%
\pgfpathlineto{\pgfqpoint{2.988424in}{2.293521in}}%
\pgfpathlineto{\pgfqpoint{2.985614in}{2.293850in}}%
\pgfpathlineto{\pgfqpoint{2.982803in}{2.294120in}}%
\pgfpathlineto{\pgfqpoint{2.979992in}{2.294247in}}%
\pgfpathlineto{\pgfqpoint{2.977182in}{2.294517in}}%
\pgfpathlineto{\pgfqpoint{2.974371in}{2.294850in}}%
\pgfpathlineto{\pgfqpoint{2.971560in}{2.294736in}}%
\pgfpathlineto{\pgfqpoint{2.968750in}{2.295051in}}%
\pgfpathlineto{\pgfqpoint{2.965939in}{2.295024in}}%
\pgfpathlineto{\pgfqpoint{2.963128in}{2.294637in}}%
\pgfpathlineto{\pgfqpoint{2.960318in}{2.294881in}}%
\pgfpathlineto{\pgfqpoint{2.957507in}{2.294338in}}%
\pgfpathlineto{\pgfqpoint{2.954696in}{2.294660in}}%
\pgfpathlineto{\pgfqpoint{2.951886in}{2.290982in}}%
\pgfpathlineto{\pgfqpoint{2.949075in}{2.291294in}}%
\pgfpathlineto{\pgfqpoint{2.946264in}{2.291042in}}%
\pgfpathlineto{\pgfqpoint{2.943453in}{2.291046in}}%
\pgfpathlineto{\pgfqpoint{2.940643in}{2.291091in}}%
\pgfpathlineto{\pgfqpoint{2.937832in}{2.291072in}}%
\pgfpathlineto{\pgfqpoint{2.935021in}{2.291310in}}%
\pgfpathlineto{\pgfqpoint{2.932211in}{2.291566in}}%
\pgfpathlineto{\pgfqpoint{2.929400in}{2.291894in}}%
\pgfpathlineto{\pgfqpoint{2.926589in}{2.292239in}}%
\pgfpathlineto{\pgfqpoint{2.923779in}{2.292489in}}%
\pgfpathlineto{\pgfqpoint{2.920968in}{2.292687in}}%
\pgfpathlineto{\pgfqpoint{2.918157in}{2.292780in}}%
\pgfpathlineto{\pgfqpoint{2.915347in}{2.292784in}}%
\pgfpathlineto{\pgfqpoint{2.912536in}{2.289431in}}%
\pgfpathlineto{\pgfqpoint{2.909725in}{2.288002in}}%
\pgfpathlineto{\pgfqpoint{2.906915in}{2.288149in}}%
\pgfpathlineto{\pgfqpoint{2.904104in}{2.287719in}}%
\pgfpathlineto{\pgfqpoint{2.901293in}{2.287991in}}%
\pgfpathlineto{\pgfqpoint{2.898482in}{2.287636in}}%
\pgfpathlineto{\pgfqpoint{2.895672in}{2.287420in}}%
\pgfpathlineto{\pgfqpoint{2.892861in}{2.287743in}}%
\pgfpathlineto{\pgfqpoint{2.890050in}{2.285812in}}%
\pgfpathlineto{\pgfqpoint{2.887240in}{2.285915in}}%
\pgfpathlineto{\pgfqpoint{2.884429in}{2.285384in}}%
\pgfpathlineto{\pgfqpoint{2.881618in}{2.285637in}}%
\pgfpathlineto{\pgfqpoint{2.878808in}{2.285294in}}%
\pgfpathlineto{\pgfqpoint{2.875997in}{2.284438in}}%
\pgfpathlineto{\pgfqpoint{2.873186in}{2.284688in}}%
\pgfpathlineto{\pgfqpoint{2.870376in}{2.281682in}}%
\pgfpathlineto{\pgfqpoint{2.867565in}{2.281745in}}%
\pgfpathlineto{\pgfqpoint{2.864754in}{2.280656in}}%
\pgfpathlineto{\pgfqpoint{2.861944in}{2.280728in}}%
\pgfpathlineto{\pgfqpoint{2.859133in}{2.280961in}}%
\pgfpathlineto{\pgfqpoint{2.856322in}{2.280567in}}%
\pgfpathlineto{\pgfqpoint{2.853511in}{2.280813in}}%
\pgfpathlineto{\pgfqpoint{2.850701in}{2.280629in}}%
\pgfpathlineto{\pgfqpoint{2.847890in}{2.280956in}}%
\pgfpathlineto{\pgfqpoint{2.845079in}{2.280339in}}%
\pgfpathlineto{\pgfqpoint{2.842269in}{2.280534in}}%
\pgfpathlineto{\pgfqpoint{2.839458in}{2.280660in}}%
\pgfpathlineto{\pgfqpoint{2.836647in}{2.280626in}}%
\pgfpathlineto{\pgfqpoint{2.833837in}{2.280968in}}%
\pgfpathlineto{\pgfqpoint{2.831026in}{2.281305in}}%
\pgfpathlineto{\pgfqpoint{2.828215in}{2.281628in}}%
\pgfpathlineto{\pgfqpoint{2.825405in}{2.281761in}}%
\pgfpathlineto{\pgfqpoint{2.822594in}{2.282027in}}%
\pgfpathlineto{\pgfqpoint{2.819783in}{2.281889in}}%
\pgfpathlineto{\pgfqpoint{2.816973in}{2.282222in}}%
\pgfpathlineto{\pgfqpoint{2.814162in}{2.282527in}}%
\pgfpathlineto{\pgfqpoint{2.811351in}{2.282824in}}%
\pgfpathlineto{\pgfqpoint{2.808540in}{2.283038in}}%
\pgfpathlineto{\pgfqpoint{2.805730in}{2.283334in}}%
\pgfpathlineto{\pgfqpoint{2.802919in}{2.283294in}}%
\pgfpathlineto{\pgfqpoint{2.800108in}{2.283529in}}%
\pgfpathlineto{\pgfqpoint{2.797298in}{2.283483in}}%
\pgfpathlineto{\pgfqpoint{2.794487in}{2.283783in}}%
\pgfpathlineto{\pgfqpoint{2.791676in}{2.284120in}}%
\pgfpathlineto{\pgfqpoint{2.788866in}{2.284172in}}%
\pgfpathlineto{\pgfqpoint{2.786055in}{2.284528in}}%
\pgfpathlineto{\pgfqpoint{2.783244in}{2.284837in}}%
\pgfpathlineto{\pgfqpoint{2.780434in}{2.285147in}}%
\pgfpathlineto{\pgfqpoint{2.777623in}{2.285503in}}%
\pgfpathlineto{\pgfqpoint{2.774812in}{2.285869in}}%
\pgfpathlineto{\pgfqpoint{2.772002in}{2.285225in}}%
\pgfpathlineto{\pgfqpoint{2.769191in}{2.285398in}}%
\pgfpathlineto{\pgfqpoint{2.766380in}{2.285739in}}%
\pgfpathlineto{\pgfqpoint{2.763570in}{2.285818in}}%
\pgfpathlineto{\pgfqpoint{2.760759in}{2.285910in}}%
\pgfpathlineto{\pgfqpoint{2.757948in}{2.286278in}}%
\pgfpathlineto{\pgfqpoint{2.755137in}{2.284593in}}%
\pgfpathlineto{\pgfqpoint{2.752327in}{2.284808in}}%
\pgfpathlineto{\pgfqpoint{2.749516in}{2.284583in}}%
\pgfpathlineto{\pgfqpoint{2.746705in}{2.284928in}}%
\pgfpathlineto{\pgfqpoint{2.743895in}{2.285300in}}%
\pgfpathlineto{\pgfqpoint{2.741084in}{2.285594in}}%
\pgfpathlineto{\pgfqpoint{2.738273in}{2.284990in}}%
\pgfpathlineto{\pgfqpoint{2.735463in}{2.284963in}}%
\pgfpathlineto{\pgfqpoint{2.732652in}{2.285333in}}%
\pgfpathlineto{\pgfqpoint{2.729841in}{2.285612in}}%
\pgfpathlineto{\pgfqpoint{2.727031in}{2.285788in}}%
\pgfpathlineto{\pgfqpoint{2.724220in}{2.286127in}}%
\pgfpathlineto{\pgfqpoint{2.721409in}{2.286449in}}%
\pgfpathlineto{\pgfqpoint{2.718599in}{2.286764in}}%
\pgfpathlineto{\pgfqpoint{2.715788in}{2.287141in}}%
\pgfpathlineto{\pgfqpoint{2.712977in}{2.287512in}}%
\pgfpathlineto{\pgfqpoint{2.710166in}{2.286607in}}%
\pgfpathlineto{\pgfqpoint{2.707356in}{2.286505in}}%
\pgfpathlineto{\pgfqpoint{2.704545in}{2.286823in}}%
\pgfpathlineto{\pgfqpoint{2.701734in}{2.286385in}}%
\pgfpathlineto{\pgfqpoint{2.698924in}{2.270831in}}%
\pgfpathlineto{\pgfqpoint{2.696113in}{2.270898in}}%
\pgfpathlineto{\pgfqpoint{2.693302in}{2.270928in}}%
\pgfpathlineto{\pgfqpoint{2.690492in}{2.271242in}}%
\pgfpathlineto{\pgfqpoint{2.687681in}{2.271032in}}%
\pgfpathlineto{\pgfqpoint{2.684870in}{2.271257in}}%
\pgfpathlineto{\pgfqpoint{2.682060in}{2.271627in}}%
\pgfpathlineto{\pgfqpoint{2.679249in}{2.271992in}}%
\pgfpathlineto{\pgfqpoint{2.676438in}{2.272138in}}%
\pgfpathlineto{\pgfqpoint{2.673628in}{2.272449in}}%
\pgfpathlineto{\pgfqpoint{2.670817in}{2.272690in}}%
\pgfpathlineto{\pgfqpoint{2.668006in}{2.273024in}}%
\pgfpathlineto{\pgfqpoint{2.665195in}{2.273186in}}%
\pgfpathlineto{\pgfqpoint{2.662385in}{2.273549in}}%
\pgfpathlineto{\pgfqpoint{2.659574in}{2.273328in}}%
\pgfpathlineto{\pgfqpoint{2.656763in}{2.273284in}}%
\pgfpathlineto{\pgfqpoint{2.653953in}{2.273581in}}%
\pgfpathlineto{\pgfqpoint{2.651142in}{2.273783in}}%
\pgfpathlineto{\pgfqpoint{2.648331in}{2.273924in}}%
\pgfpathlineto{\pgfqpoint{2.645521in}{2.274256in}}%
\pgfpathlineto{\pgfqpoint{2.642710in}{2.274610in}}%
\pgfpathlineto{\pgfqpoint{2.639899in}{2.274784in}}%
\pgfpathlineto{\pgfqpoint{2.637089in}{2.265431in}}%
\pgfpathlineto{\pgfqpoint{2.634278in}{2.265721in}}%
\pgfpathlineto{\pgfqpoint{2.631467in}{2.266062in}}%
\pgfpathlineto{\pgfqpoint{2.628657in}{2.265855in}}%
\pgfpathlineto{\pgfqpoint{2.625846in}{2.266220in}}%
\pgfpathlineto{\pgfqpoint{2.623035in}{2.265477in}}%
\pgfpathlineto{\pgfqpoint{2.620224in}{2.265733in}}%
\pgfpathlineto{\pgfqpoint{2.617414in}{2.266087in}}%
\pgfpathlineto{\pgfqpoint{2.614603in}{2.265674in}}%
\pgfpathlineto{\pgfqpoint{2.611792in}{2.266029in}}%
\pgfpathlineto{\pgfqpoint{2.608982in}{2.265698in}}%
\pgfpathlineto{\pgfqpoint{2.606171in}{2.265925in}}%
\pgfpathlineto{\pgfqpoint{2.603360in}{2.266007in}}%
\pgfpathlineto{\pgfqpoint{2.600550in}{2.265473in}}%
\pgfpathlineto{\pgfqpoint{2.597739in}{2.265609in}}%
\pgfpathlineto{\pgfqpoint{2.594928in}{2.265962in}}%
\pgfpathlineto{\pgfqpoint{2.592118in}{2.266077in}}%
\pgfpathlineto{\pgfqpoint{2.589307in}{2.266395in}}%
\pgfpathlineto{\pgfqpoint{2.586496in}{2.266370in}}%
\pgfpathlineto{\pgfqpoint{2.583686in}{2.266764in}}%
\pgfpathlineto{\pgfqpoint{2.580875in}{2.267086in}}%
\pgfpathlineto{\pgfqpoint{2.578064in}{2.267476in}}%
\pgfpathlineto{\pgfqpoint{2.575253in}{2.267706in}}%
\pgfpathlineto{\pgfqpoint{2.572443in}{2.267952in}}%
\pgfpathlineto{\pgfqpoint{2.569632in}{2.268304in}}%
\pgfpathlineto{\pgfqpoint{2.566821in}{2.268657in}}%
\pgfpathlineto{\pgfqpoint{2.564011in}{2.268737in}}%
\pgfpathlineto{\pgfqpoint{2.561200in}{2.269039in}}%
\pgfpathlineto{\pgfqpoint{2.558389in}{2.269377in}}%
\pgfpathlineto{\pgfqpoint{2.555579in}{2.269741in}}%
\pgfpathlineto{\pgfqpoint{2.552768in}{2.270135in}}%
\pgfpathlineto{\pgfqpoint{2.549957in}{2.270148in}}%
\pgfpathlineto{\pgfqpoint{2.547147in}{2.270271in}}%
\pgfpathlineto{\pgfqpoint{2.544336in}{2.270522in}}%
\pgfpathlineto{\pgfqpoint{2.541525in}{2.270638in}}%
\pgfpathlineto{\pgfqpoint{2.538715in}{2.270970in}}%
\pgfpathlineto{\pgfqpoint{2.535904in}{2.271365in}}%
\pgfpathlineto{\pgfqpoint{2.533093in}{2.271766in}}%
\pgfpathlineto{\pgfqpoint{2.530283in}{2.272164in}}%
\pgfpathlineto{\pgfqpoint{2.527472in}{2.272373in}}%
\pgfpathlineto{\pgfqpoint{2.524661in}{2.272486in}}%
\pgfpathlineto{\pgfqpoint{2.521850in}{2.272504in}}%
\pgfpathlineto{\pgfqpoint{2.519040in}{2.272052in}}%
\pgfpathlineto{\pgfqpoint{2.516229in}{2.272446in}}%
\pgfpathlineto{\pgfqpoint{2.513418in}{2.272459in}}%
\pgfpathlineto{\pgfqpoint{2.510608in}{2.272720in}}%
\pgfpathlineto{\pgfqpoint{2.507797in}{2.272170in}}%
\pgfpathlineto{\pgfqpoint{2.504986in}{2.271888in}}%
\pgfpathlineto{\pgfqpoint{2.502176in}{2.272290in}}%
\pgfpathlineto{\pgfqpoint{2.499365in}{2.271357in}}%
\pgfpathlineto{\pgfqpoint{2.496554in}{2.271764in}}%
\pgfpathlineto{\pgfqpoint{2.493744in}{2.271841in}}%
\pgfpathlineto{\pgfqpoint{2.490933in}{2.271891in}}%
\pgfpathlineto{\pgfqpoint{2.488122in}{2.271606in}}%
\pgfpathlineto{\pgfqpoint{2.485312in}{2.272005in}}%
\pgfpathlineto{\pgfqpoint{2.482501in}{2.272412in}}%
\pgfpathlineto{\pgfqpoint{2.479690in}{2.272046in}}%
\pgfpathlineto{\pgfqpoint{2.476879in}{2.272433in}}%
\pgfpathlineto{\pgfqpoint{2.474069in}{2.272835in}}%
\pgfpathlineto{\pgfqpoint{2.471258in}{2.272224in}}%
\pgfpathlineto{\pgfqpoint{2.468447in}{2.272233in}}%
\pgfpathlineto{\pgfqpoint{2.465637in}{2.272411in}}%
\pgfpathlineto{\pgfqpoint{2.462826in}{2.272413in}}%
\pgfpathlineto{\pgfqpoint{2.460015in}{2.272822in}}%
\pgfpathlineto{\pgfqpoint{2.457205in}{2.273255in}}%
\pgfpathlineto{\pgfqpoint{2.454394in}{2.273565in}}%
\pgfpathlineto{\pgfqpoint{2.451583in}{2.273997in}}%
\pgfpathlineto{\pgfqpoint{2.448773in}{2.274004in}}%
\pgfpathlineto{\pgfqpoint{2.445962in}{2.274345in}}%
\pgfpathlineto{\pgfqpoint{2.443151in}{2.274504in}}%
\pgfpathlineto{\pgfqpoint{2.440341in}{2.274936in}}%
\pgfpathlineto{\pgfqpoint{2.437530in}{2.275140in}}%
\pgfpathlineto{\pgfqpoint{2.434719in}{2.275554in}}%
\pgfpathlineto{\pgfqpoint{2.431908in}{2.275912in}}%
\pgfpathlineto{\pgfqpoint{2.429098in}{2.276306in}}%
\pgfpathlineto{\pgfqpoint{2.426287in}{2.276371in}}%
\pgfpathlineto{\pgfqpoint{2.423476in}{2.276745in}}%
\pgfpathlineto{\pgfqpoint{2.420666in}{2.276721in}}%
\pgfpathlineto{\pgfqpoint{2.417855in}{2.277091in}}%
\pgfpathlineto{\pgfqpoint{2.415044in}{2.277009in}}%
\pgfpathlineto{\pgfqpoint{2.412234in}{2.277432in}}%
\pgfpathlineto{\pgfqpoint{2.409423in}{2.277798in}}%
\pgfpathlineto{\pgfqpoint{2.406612in}{2.278088in}}%
\pgfpathlineto{\pgfqpoint{2.403802in}{2.278207in}}%
\pgfpathlineto{\pgfqpoint{2.400991in}{2.278461in}}%
\pgfpathlineto{\pgfqpoint{2.398180in}{2.278523in}}%
\pgfpathlineto{\pgfqpoint{2.395370in}{2.278775in}}%
\pgfpathlineto{\pgfqpoint{2.392559in}{2.278711in}}%
\pgfpathlineto{\pgfqpoint{2.389748in}{2.277716in}}%
\pgfpathlineto{\pgfqpoint{2.386937in}{2.278152in}}%
\pgfpathlineto{\pgfqpoint{2.384127in}{2.278614in}}%
\pgfpathlineto{\pgfqpoint{2.381316in}{2.278416in}}%
\pgfpathlineto{\pgfqpoint{2.378505in}{2.278878in}}%
\pgfpathlineto{\pgfqpoint{2.375695in}{2.279174in}}%
\pgfpathlineto{\pgfqpoint{2.372884in}{2.279630in}}%
\pgfpathlineto{\pgfqpoint{2.370073in}{2.279699in}}%
\pgfpathlineto{\pgfqpoint{2.367263in}{2.280148in}}%
\pgfpathlineto{\pgfqpoint{2.364452in}{2.280539in}}%
\pgfpathlineto{\pgfqpoint{2.361641in}{2.281011in}}%
\pgfpathlineto{\pgfqpoint{2.358831in}{2.281450in}}%
\pgfpathlineto{\pgfqpoint{2.356020in}{2.281899in}}%
\pgfpathlineto{\pgfqpoint{2.353209in}{2.281506in}}%
\pgfpathlineto{\pgfqpoint{2.350399in}{2.281977in}}%
\pgfpathlineto{\pgfqpoint{2.347588in}{2.282108in}}%
\pgfpathlineto{\pgfqpoint{2.344777in}{2.278175in}}%
\pgfpathlineto{\pgfqpoint{2.341967in}{2.278584in}}%
\pgfpathlineto{\pgfqpoint{2.339156in}{2.278580in}}%
\pgfpathlineto{\pgfqpoint{2.336345in}{2.279051in}}%
\pgfpathlineto{\pgfqpoint{2.333534in}{2.279532in}}%
\pgfpathlineto{\pgfqpoint{2.330724in}{2.279718in}}%
\pgfpathlineto{\pgfqpoint{2.327913in}{2.280124in}}%
\pgfpathlineto{\pgfqpoint{2.325102in}{2.280529in}}%
\pgfpathlineto{\pgfqpoint{2.322292in}{2.281014in}}%
\pgfpathlineto{\pgfqpoint{2.319481in}{2.281232in}}%
\pgfpathlineto{\pgfqpoint{2.316670in}{2.281144in}}%
\pgfpathlineto{\pgfqpoint{2.313860in}{2.281621in}}%
\pgfpathlineto{\pgfqpoint{2.311049in}{2.281461in}}%
\pgfpathlineto{\pgfqpoint{2.308238in}{2.281675in}}%
\pgfpathlineto{\pgfqpoint{2.305428in}{2.281535in}}%
\pgfpathlineto{\pgfqpoint{2.302617in}{2.282006in}}%
\pgfpathlineto{\pgfqpoint{2.299806in}{2.282420in}}%
\pgfpathlineto{\pgfqpoint{2.296996in}{2.281349in}}%
\pgfpathlineto{\pgfqpoint{2.294185in}{2.281358in}}%
\pgfpathlineto{\pgfqpoint{2.291374in}{2.281453in}}%
\pgfpathlineto{\pgfqpoint{2.288563in}{2.281954in}}%
\pgfpathlineto{\pgfqpoint{2.285753in}{2.282291in}}%
\pgfpathlineto{\pgfqpoint{2.282942in}{2.282585in}}%
\pgfpathlineto{\pgfqpoint{2.280131in}{2.283077in}}%
\pgfpathlineto{\pgfqpoint{2.277321in}{2.283316in}}%
\pgfpathlineto{\pgfqpoint{2.274510in}{2.281888in}}%
\pgfpathlineto{\pgfqpoint{2.271699in}{2.280511in}}%
\pgfpathlineto{\pgfqpoint{2.268889in}{2.280527in}}%
\pgfpathlineto{\pgfqpoint{2.266078in}{2.281036in}}%
\pgfpathlineto{\pgfqpoint{2.263267in}{2.281546in}}%
\pgfpathlineto{\pgfqpoint{2.260457in}{2.282017in}}%
\pgfpathlineto{\pgfqpoint{2.257646in}{2.281859in}}%
\pgfpathlineto{\pgfqpoint{2.254835in}{2.282357in}}%
\pgfpathlineto{\pgfqpoint{2.252025in}{2.282866in}}%
\pgfpathlineto{\pgfqpoint{2.249214in}{2.277778in}}%
\pgfpathlineto{\pgfqpoint{2.246403in}{2.275649in}}%
\pgfpathlineto{\pgfqpoint{2.243592in}{2.276157in}}%
\pgfpathlineto{\pgfqpoint{2.240782in}{2.276392in}}%
\pgfpathlineto{\pgfqpoint{2.237971in}{2.276695in}}%
\pgfpathlineto{\pgfqpoint{2.235160in}{2.276877in}}%
\pgfpathlineto{\pgfqpoint{2.232350in}{2.277387in}}%
\pgfpathlineto{\pgfqpoint{2.229539in}{2.277485in}}%
\pgfpathlineto{\pgfqpoint{2.226728in}{2.277296in}}%
\pgfpathlineto{\pgfqpoint{2.223918in}{2.277578in}}%
\pgfpathlineto{\pgfqpoint{2.221107in}{2.277963in}}%
\pgfpathlineto{\pgfqpoint{2.218296in}{2.278483in}}%
\pgfpathlineto{\pgfqpoint{2.215486in}{2.278404in}}%
\pgfpathlineto{\pgfqpoint{2.212675in}{2.278826in}}%
\pgfpathlineto{\pgfqpoint{2.209864in}{2.279321in}}%
\pgfpathlineto{\pgfqpoint{2.207054in}{2.279850in}}%
\pgfpathlineto{\pgfqpoint{2.204243in}{2.280318in}}%
\pgfpathlineto{\pgfqpoint{2.201432in}{2.280760in}}%
\pgfpathlineto{\pgfqpoint{2.198621in}{2.280890in}}%
\pgfpathlineto{\pgfqpoint{2.195811in}{2.281105in}}%
\pgfpathlineto{\pgfqpoint{2.193000in}{2.280007in}}%
\pgfpathlineto{\pgfqpoint{2.190189in}{2.280382in}}%
\pgfpathlineto{\pgfqpoint{2.187379in}{2.280813in}}%
\pgfpathlineto{\pgfqpoint{2.184568in}{2.281321in}}%
\pgfpathlineto{\pgfqpoint{2.181757in}{2.281152in}}%
\pgfpathlineto{\pgfqpoint{2.178947in}{2.281631in}}%
\pgfpathlineto{\pgfqpoint{2.176136in}{2.281536in}}%
\pgfpathlineto{\pgfqpoint{2.173325in}{2.281115in}}%
\pgfpathlineto{\pgfqpoint{2.170515in}{2.281621in}}%
\pgfpathlineto{\pgfqpoint{2.167704in}{2.281626in}}%
\pgfpathlineto{\pgfqpoint{2.164893in}{2.281733in}}%
\pgfpathlineto{\pgfqpoint{2.162083in}{2.280181in}}%
\pgfpathlineto{\pgfqpoint{2.159272in}{2.280403in}}%
\pgfpathlineto{\pgfqpoint{2.156461in}{2.280786in}}%
\pgfpathlineto{\pgfqpoint{2.153651in}{2.281209in}}%
\pgfpathlineto{\pgfqpoint{2.150840in}{2.281110in}}%
\pgfpathlineto{\pgfqpoint{2.148029in}{2.281590in}}%
\pgfpathlineto{\pgfqpoint{2.145218in}{2.281931in}}%
\pgfpathlineto{\pgfqpoint{2.142408in}{2.282477in}}%
\pgfpathlineto{\pgfqpoint{2.139597in}{2.282999in}}%
\pgfpathlineto{\pgfqpoint{2.136786in}{2.283227in}}%
\pgfpathlineto{\pgfqpoint{2.133976in}{2.283725in}}%
\pgfpathlineto{\pgfqpoint{2.131165in}{2.284180in}}%
\pgfpathlineto{\pgfqpoint{2.128354in}{2.284510in}}%
\pgfpathlineto{\pgfqpoint{2.125544in}{2.285083in}}%
\pgfpathlineto{\pgfqpoint{2.122733in}{2.285658in}}%
\pgfpathlineto{\pgfqpoint{2.119922in}{2.286102in}}%
\pgfpathlineto{\pgfqpoint{2.117112in}{2.286235in}}%
\pgfpathlineto{\pgfqpoint{2.114301in}{2.285592in}}%
\pgfpathlineto{\pgfqpoint{2.111490in}{2.286128in}}%
\pgfpathlineto{\pgfqpoint{2.108680in}{2.282207in}}%
\pgfpathlineto{\pgfqpoint{2.105869in}{2.282788in}}%
\pgfpathlineto{\pgfqpoint{2.103058in}{2.283372in}}%
\pgfpathlineto{\pgfqpoint{2.100247in}{2.283702in}}%
\pgfpathlineto{\pgfqpoint{2.097437in}{2.283516in}}%
\pgfpathlineto{\pgfqpoint{2.094626in}{2.283844in}}%
\pgfpathlineto{\pgfqpoint{2.091815in}{2.284363in}}%
\pgfpathlineto{\pgfqpoint{2.089005in}{2.282305in}}%
\pgfpathlineto{\pgfqpoint{2.086194in}{2.282567in}}%
\pgfpathlineto{\pgfqpoint{2.083383in}{2.283137in}}%
\pgfpathlineto{\pgfqpoint{2.080573in}{2.283707in}}%
\pgfpathlineto{\pgfqpoint{2.077762in}{2.283977in}}%
\pgfpathlineto{\pgfqpoint{2.074951in}{2.284542in}}%
\pgfpathlineto{\pgfqpoint{2.072141in}{2.285143in}}%
\pgfpathlineto{\pgfqpoint{2.069330in}{2.285119in}}%
\pgfpathlineto{\pgfqpoint{2.066519in}{2.285423in}}%
\pgfpathlineto{\pgfqpoint{2.063709in}{2.285844in}}%
\pgfpathlineto{\pgfqpoint{2.060898in}{2.285636in}}%
\pgfpathlineto{\pgfqpoint{2.058087in}{2.284760in}}%
\pgfpathlineto{\pgfqpoint{2.055276in}{2.285367in}}%
\pgfpathlineto{\pgfqpoint{2.052466in}{2.285161in}}%
\pgfpathlineto{\pgfqpoint{2.049655in}{2.285531in}}%
\pgfpathlineto{\pgfqpoint{2.046844in}{2.286142in}}%
\pgfpathlineto{\pgfqpoint{2.044034in}{2.286327in}}%
\pgfpathlineto{\pgfqpoint{2.041223in}{2.286701in}}%
\pgfpathlineto{\pgfqpoint{2.038412in}{2.287281in}}%
\pgfpathlineto{\pgfqpoint{2.035602in}{2.287894in}}%
\pgfpathlineto{\pgfqpoint{2.032791in}{2.288516in}}%
\pgfpathlineto{\pgfqpoint{2.029980in}{2.289107in}}%
\pgfpathlineto{\pgfqpoint{2.027170in}{2.289590in}}%
\pgfpathlineto{\pgfqpoint{2.024359in}{2.290222in}}%
\pgfpathlineto{\pgfqpoint{2.021548in}{2.290796in}}%
\pgfpathlineto{\pgfqpoint{2.018738in}{2.290967in}}%
\pgfpathlineto{\pgfqpoint{2.015927in}{2.291514in}}%
\pgfpathlineto{\pgfqpoint{2.013116in}{2.291477in}}%
\pgfpathlineto{\pgfqpoint{2.010305in}{2.291680in}}%
\pgfpathlineto{\pgfqpoint{2.007495in}{2.291720in}}%
\pgfpathlineto{\pgfqpoint{2.004684in}{2.292273in}}%
\pgfpathlineto{\pgfqpoint{2.001873in}{2.292873in}}%
\pgfpathlineto{\pgfqpoint{1.999063in}{2.293522in}}%
\pgfpathlineto{\pgfqpoint{1.996252in}{2.291417in}}%
\pgfpathlineto{\pgfqpoint{1.993441in}{2.292049in}}%
\pgfpathlineto{\pgfqpoint{1.990631in}{2.291607in}}%
\pgfpathlineto{\pgfqpoint{1.987820in}{2.292069in}}%
\pgfpathlineto{\pgfqpoint{1.985009in}{2.292698in}}%
\pgfpathlineto{\pgfqpoint{1.982199in}{2.289307in}}%
\pgfpathlineto{\pgfqpoint{1.979388in}{2.289588in}}%
\pgfpathlineto{\pgfqpoint{1.976577in}{2.290185in}}%
\pgfpathlineto{\pgfqpoint{1.973767in}{2.288335in}}%
\pgfpathlineto{\pgfqpoint{1.970956in}{2.287919in}}%
\pgfpathlineto{\pgfqpoint{1.968145in}{2.288384in}}%
\pgfpathlineto{\pgfqpoint{1.965334in}{2.289052in}}%
\pgfpathlineto{\pgfqpoint{1.962524in}{2.289576in}}%
\pgfpathlineto{\pgfqpoint{1.959713in}{2.289658in}}%
\pgfpathlineto{\pgfqpoint{1.956902in}{2.290131in}}%
\pgfpathlineto{\pgfqpoint{1.954092in}{2.290365in}}%
\pgfpathlineto{\pgfqpoint{1.951281in}{2.289934in}}%
\pgfpathlineto{\pgfqpoint{1.948470in}{2.289530in}}%
\pgfpathlineto{\pgfqpoint{1.945660in}{2.290055in}}%
\pgfpathlineto{\pgfqpoint{1.942849in}{2.288752in}}%
\pgfpathlineto{\pgfqpoint{1.940038in}{2.289013in}}%
\pgfpathlineto{\pgfqpoint{1.937228in}{2.288395in}}%
\pgfpathlineto{\pgfqpoint{1.934417in}{2.288505in}}%
\pgfpathlineto{\pgfqpoint{1.931606in}{2.289137in}}%
\pgfpathlineto{\pgfqpoint{1.928796in}{2.287460in}}%
\pgfpathlineto{\pgfqpoint{1.925985in}{2.288113in}}%
\pgfpathlineto{\pgfqpoint{1.923174in}{2.288719in}}%
\pgfpathlineto{\pgfqpoint{1.920364in}{2.287720in}}%
\pgfpathlineto{\pgfqpoint{1.917553in}{2.288361in}}%
\pgfpathlineto{\pgfqpoint{1.914742in}{2.288992in}}%
\pgfpathlineto{\pgfqpoint{1.911931in}{2.288507in}}%
\pgfpathlineto{\pgfqpoint{1.909121in}{2.288812in}}%
\pgfpathlineto{\pgfqpoint{1.906310in}{2.283130in}}%
\pgfpathlineto{\pgfqpoint{1.903499in}{2.283568in}}%
\pgfpathlineto{\pgfqpoint{1.900689in}{2.284215in}}%
\pgfpathlineto{\pgfqpoint{1.897878in}{2.284249in}}%
\pgfpathlineto{\pgfqpoint{1.895067in}{2.284366in}}%
\pgfpathlineto{\pgfqpoint{1.892257in}{2.285072in}}%
\pgfpathlineto{\pgfqpoint{1.889446in}{2.285681in}}%
\pgfpathlineto{\pgfqpoint{1.886635in}{2.286364in}}%
\pgfpathlineto{\pgfqpoint{1.883825in}{2.286785in}}%
\pgfpathlineto{\pgfqpoint{1.881014in}{2.287265in}}%
\pgfpathlineto{\pgfqpoint{1.878203in}{2.287622in}}%
\pgfpathlineto{\pgfqpoint{1.875393in}{2.288296in}}%
\pgfpathlineto{\pgfqpoint{1.872582in}{2.287521in}}%
\pgfpathlineto{\pgfqpoint{1.869771in}{2.288221in}}%
\pgfpathlineto{\pgfqpoint{1.866960in}{2.288603in}}%
\pgfpathlineto{\pgfqpoint{1.864150in}{2.288829in}}%
\pgfpathlineto{\pgfqpoint{1.861339in}{2.289166in}}%
\pgfpathlineto{\pgfqpoint{1.858528in}{2.289538in}}%
\pgfpathlineto{\pgfqpoint{1.855718in}{2.289376in}}%
\pgfpathlineto{\pgfqpoint{1.852907in}{2.290050in}}%
\pgfpathlineto{\pgfqpoint{1.850096in}{2.290489in}}%
\pgfpathlineto{\pgfqpoint{1.847286in}{2.290787in}}%
\pgfpathlineto{\pgfqpoint{1.844475in}{2.291289in}}%
\pgfpathlineto{\pgfqpoint{1.841664in}{2.291367in}}%
\pgfpathlineto{\pgfqpoint{1.838854in}{2.290505in}}%
\pgfpathlineto{\pgfqpoint{1.836043in}{2.291063in}}%
\pgfpathlineto{\pgfqpoint{1.833232in}{2.291757in}}%
\pgfpathlineto{\pgfqpoint{1.830422in}{2.291370in}}%
\pgfpathlineto{\pgfqpoint{1.827611in}{2.289473in}}%
\pgfpathlineto{\pgfqpoint{1.824800in}{2.287960in}}%
\pgfpathlineto{\pgfqpoint{1.821989in}{2.287853in}}%
\pgfpathlineto{\pgfqpoint{1.819179in}{2.287186in}}%
\pgfpathlineto{\pgfqpoint{1.816368in}{2.287771in}}%
\pgfpathlineto{\pgfqpoint{1.813557in}{2.285164in}}%
\pgfpathlineto{\pgfqpoint{1.810747in}{2.285819in}}%
\pgfpathlineto{\pgfqpoint{1.807936in}{2.286609in}}%
\pgfpathlineto{\pgfqpoint{1.805125in}{2.286907in}}%
\pgfpathlineto{\pgfqpoint{1.802315in}{2.284583in}}%
\pgfpathlineto{\pgfqpoint{1.799504in}{2.280491in}}%
\pgfpathlineto{\pgfqpoint{1.796693in}{2.280535in}}%
\pgfpathlineto{\pgfqpoint{1.793883in}{2.281308in}}%
\pgfpathlineto{\pgfqpoint{1.791072in}{2.281621in}}%
\pgfpathlineto{\pgfqpoint{1.788261in}{2.281908in}}%
\pgfpathlineto{\pgfqpoint{1.785451in}{2.282470in}}%
\pgfpathlineto{\pgfqpoint{1.782640in}{2.282829in}}%
\pgfpathlineto{\pgfqpoint{1.779829in}{2.283460in}}%
\pgfpathlineto{\pgfqpoint{1.777018in}{2.283927in}}%
\pgfpathlineto{\pgfqpoint{1.774208in}{2.280166in}}%
\pgfpathlineto{\pgfqpoint{1.771397in}{2.280979in}}%
\pgfpathlineto{\pgfqpoint{1.768586in}{2.280317in}}%
\pgfpathlineto{\pgfqpoint{1.765776in}{2.281135in}}%
\pgfpathlineto{\pgfqpoint{1.762965in}{2.281673in}}%
\pgfpathlineto{\pgfqpoint{1.760154in}{2.282494in}}%
\pgfpathlineto{\pgfqpoint{1.757344in}{2.283319in}}%
\pgfpathlineto{\pgfqpoint{1.754533in}{2.284059in}}%
\pgfpathlineto{\pgfqpoint{1.751722in}{2.284740in}}%
\pgfpathlineto{\pgfqpoint{1.748912in}{2.285333in}}%
\pgfpathlineto{\pgfqpoint{1.746101in}{2.286092in}}%
\pgfpathlineto{\pgfqpoint{1.743290in}{2.286397in}}%
\pgfpathlineto{\pgfqpoint{1.740480in}{2.287062in}}%
\pgfpathlineto{\pgfqpoint{1.737669in}{2.287218in}}%
\pgfpathlineto{\pgfqpoint{1.734858in}{2.286834in}}%
\pgfpathlineto{\pgfqpoint{1.732048in}{2.286504in}}%
\pgfpathlineto{\pgfqpoint{1.729237in}{2.286074in}}%
\pgfpathlineto{\pgfqpoint{1.726426in}{2.286128in}}%
\pgfpathlineto{\pgfqpoint{1.723615in}{2.286911in}}%
\pgfpathlineto{\pgfqpoint{1.720805in}{2.285696in}}%
\pgfpathlineto{\pgfqpoint{1.717994in}{2.284306in}}%
\pgfpathlineto{\pgfqpoint{1.715183in}{2.285185in}}%
\pgfpathlineto{\pgfqpoint{1.712373in}{2.285134in}}%
\pgfpathlineto{\pgfqpoint{1.709562in}{2.285042in}}%
\pgfpathlineto{\pgfqpoint{1.706751in}{2.285558in}}%
\pgfpathlineto{\pgfqpoint{1.703941in}{2.286310in}}%
\pgfpathlineto{\pgfqpoint{1.701130in}{2.287194in}}%
\pgfpathlineto{\pgfqpoint{1.698319in}{2.287954in}}%
\pgfpathlineto{\pgfqpoint{1.695509in}{2.288466in}}%
\pgfpathlineto{\pgfqpoint{1.692698in}{2.288536in}}%
\pgfpathlineto{\pgfqpoint{1.689887in}{2.289135in}}%
\pgfpathlineto{\pgfqpoint{1.687077in}{2.289056in}}%
\pgfpathlineto{\pgfqpoint{1.684266in}{2.289985in}}%
\pgfpathlineto{\pgfqpoint{1.681455in}{2.290909in}}%
\pgfpathlineto{\pgfqpoint{1.678644in}{2.290453in}}%
\pgfpathlineto{\pgfqpoint{1.675834in}{2.291384in}}%
\pgfpathlineto{\pgfqpoint{1.673023in}{2.290355in}}%
\pgfpathlineto{\pgfqpoint{1.670212in}{2.291048in}}%
\pgfpathlineto{\pgfqpoint{1.667402in}{2.291484in}}%
\pgfpathlineto{\pgfqpoint{1.664591in}{2.291094in}}%
\pgfpathlineto{\pgfqpoint{1.661780in}{2.291523in}}%
\pgfpathlineto{\pgfqpoint{1.658970in}{2.292229in}}%
\pgfpathlineto{\pgfqpoint{1.656159in}{2.293204in}}%
\pgfpathlineto{\pgfqpoint{1.653348in}{2.294097in}}%
\pgfpathlineto{\pgfqpoint{1.650538in}{2.295015in}}%
\pgfpathlineto{\pgfqpoint{1.647727in}{2.282307in}}%
\pgfpathlineto{\pgfqpoint{1.644916in}{2.278763in}}%
\pgfpathlineto{\pgfqpoint{1.642106in}{2.278477in}}%
\pgfpathlineto{\pgfqpoint{1.639295in}{2.279417in}}%
\pgfpathlineto{\pgfqpoint{1.636484in}{2.280164in}}%
\pgfpathlineto{\pgfqpoint{1.633673in}{2.280030in}}%
\pgfpathlineto{\pgfqpoint{1.630863in}{2.278734in}}%
\pgfpathlineto{\pgfqpoint{1.628052in}{2.277128in}}%
\pgfpathlineto{\pgfqpoint{1.625241in}{2.278108in}}%
\pgfpathlineto{\pgfqpoint{1.622431in}{2.278537in}}%
\pgfpathlineto{\pgfqpoint{1.619620in}{2.279527in}}%
\pgfpathlineto{\pgfqpoint{1.616809in}{2.280437in}}%
\pgfpathlineto{\pgfqpoint{1.613999in}{2.273168in}}%
\pgfpathlineto{\pgfqpoint{1.611188in}{2.271034in}}%
\pgfpathlineto{\pgfqpoint{1.608377in}{2.271845in}}%
\pgfpathlineto{\pgfqpoint{1.605567in}{2.272840in}}%
\pgfpathlineto{\pgfqpoint{1.602756in}{2.273601in}}%
\pgfpathlineto{\pgfqpoint{1.599945in}{2.274602in}}%
\pgfpathlineto{\pgfqpoint{1.597135in}{2.275507in}}%
\pgfpathlineto{\pgfqpoint{1.594324in}{2.276462in}}%
\pgfpathlineto{\pgfqpoint{1.591513in}{2.276635in}}%
\pgfpathlineto{\pgfqpoint{1.588702in}{2.274536in}}%
\pgfpathlineto{\pgfqpoint{1.585892in}{2.275183in}}%
\pgfpathlineto{\pgfqpoint{1.583081in}{2.275781in}}%
\pgfpathlineto{\pgfqpoint{1.580270in}{2.276819in}}%
\pgfpathlineto{\pgfqpoint{1.577460in}{2.277529in}}%
\pgfpathlineto{\pgfqpoint{1.574649in}{2.272869in}}%
\pgfpathlineto{\pgfqpoint{1.571838in}{2.273856in}}%
\pgfpathlineto{\pgfqpoint{1.569028in}{2.274653in}}%
\pgfpathlineto{\pgfqpoint{1.566217in}{2.273268in}}%
\pgfpathlineto{\pgfqpoint{1.563406in}{2.274332in}}%
\pgfpathlineto{\pgfqpoint{1.560596in}{2.271343in}}%
\pgfpathlineto{\pgfqpoint{1.557785in}{2.272227in}}%
\pgfpathlineto{\pgfqpoint{1.554974in}{2.272878in}}%
\pgfpathlineto{\pgfqpoint{1.552164in}{2.271900in}}%
\pgfpathlineto{\pgfqpoint{1.549353in}{2.271827in}}%
\pgfpathlineto{\pgfqpoint{1.546542in}{2.272434in}}%
\pgfpathlineto{\pgfqpoint{1.543731in}{2.272345in}}%
\pgfpathlineto{\pgfqpoint{1.540921in}{2.270025in}}%
\pgfpathlineto{\pgfqpoint{1.538110in}{2.271140in}}%
\pgfpathlineto{\pgfqpoint{1.535299in}{2.267801in}}%
\pgfpathlineto{\pgfqpoint{1.532489in}{2.268358in}}%
\pgfpathlineto{\pgfqpoint{1.529678in}{2.268277in}}%
\pgfpathlineto{\pgfqpoint{1.526867in}{2.268765in}}%
\pgfpathlineto{\pgfqpoint{1.524057in}{2.269549in}}%
\pgfpathlineto{\pgfqpoint{1.521246in}{2.269307in}}%
\pgfpathlineto{\pgfqpoint{1.518435in}{2.270282in}}%
\pgfpathlineto{\pgfqpoint{1.515625in}{2.271227in}}%
\pgfpathlineto{\pgfqpoint{1.512814in}{2.272324in}}%
\pgfpathlineto{\pgfqpoint{1.510003in}{2.268576in}}%
\pgfpathlineto{\pgfqpoint{1.507193in}{2.269700in}}%
\pgfpathlineto{\pgfqpoint{1.504382in}{2.270588in}}%
\pgfpathlineto{\pgfqpoint{1.501571in}{2.271732in}}%
\pgfpathlineto{\pgfqpoint{1.498761in}{2.266011in}}%
\pgfpathlineto{\pgfqpoint{1.495950in}{2.266859in}}%
\pgfpathlineto{\pgfqpoint{1.493139in}{2.266063in}}%
\pgfpathlineto{\pgfqpoint{1.490328in}{2.265568in}}%
\pgfpathlineto{\pgfqpoint{1.487518in}{2.266738in}}%
\pgfpathlineto{\pgfqpoint{1.484707in}{2.267006in}}%
\pgfpathlineto{\pgfqpoint{1.481896in}{2.266700in}}%
\pgfpathlineto{\pgfqpoint{1.479086in}{2.267209in}}%
\pgfpathlineto{\pgfqpoint{1.476275in}{2.268406in}}%
\pgfpathlineto{\pgfqpoint{1.473464in}{2.267776in}}%
\pgfpathlineto{\pgfqpoint{1.470654in}{2.261982in}}%
\pgfpathlineto{\pgfqpoint{1.467843in}{2.261213in}}%
\pgfpathlineto{\pgfqpoint{1.465032in}{2.262406in}}%
\pgfpathlineto{\pgfqpoint{1.462222in}{2.263575in}}%
\pgfpathlineto{\pgfqpoint{1.459411in}{2.264804in}}%
\pgfpathlineto{\pgfqpoint{1.456600in}{2.263635in}}%
\pgfpathlineto{\pgfqpoint{1.453790in}{2.264871in}}%
\pgfpathlineto{\pgfqpoint{1.450979in}{2.264561in}}%
\pgfpathlineto{\pgfqpoint{1.448168in}{2.265738in}}%
\pgfpathlineto{\pgfqpoint{1.445357in}{2.266775in}}%
\pgfpathlineto{\pgfqpoint{1.442547in}{2.267364in}}%
\pgfpathlineto{\pgfqpoint{1.439736in}{2.267882in}}%
\pgfpathlineto{\pgfqpoint{1.436925in}{2.268297in}}%
\pgfpathlineto{\pgfqpoint{1.434115in}{2.265539in}}%
\pgfpathlineto{\pgfqpoint{1.431304in}{2.265976in}}%
\pgfpathlineto{\pgfqpoint{1.428493in}{2.267157in}}%
\pgfpathlineto{\pgfqpoint{1.425683in}{2.268157in}}%
\pgfpathlineto{\pgfqpoint{1.422872in}{2.269344in}}%
\pgfpathlineto{\pgfqpoint{1.420061in}{2.270289in}}%
\pgfpathlineto{\pgfqpoint{1.417251in}{2.271587in}}%
\pgfpathlineto{\pgfqpoint{1.414440in}{2.272379in}}%
\pgfpathlineto{\pgfqpoint{1.411629in}{2.273635in}}%
\pgfpathlineto{\pgfqpoint{1.408819in}{2.274669in}}%
\pgfpathlineto{\pgfqpoint{1.406008in}{2.274757in}}%
\pgfpathlineto{\pgfqpoint{1.403197in}{2.276155in}}%
\pgfpathlineto{\pgfqpoint{1.400386in}{2.276916in}}%
\pgfpathlineto{\pgfqpoint{1.397576in}{2.271554in}}%
\pgfpathlineto{\pgfqpoint{1.394765in}{2.266852in}}%
\pgfpathlineto{\pgfqpoint{1.391954in}{2.260009in}}%
\pgfpathlineto{\pgfqpoint{1.389144in}{2.261186in}}%
\pgfpathlineto{\pgfqpoint{1.386333in}{2.261830in}}%
\pgfpathlineto{\pgfqpoint{1.383522in}{2.257370in}}%
\pgfpathlineto{\pgfqpoint{1.380712in}{2.257983in}}%
\pgfpathlineto{\pgfqpoint{1.377901in}{2.259043in}}%
\pgfpathlineto{\pgfqpoint{1.375090in}{2.259313in}}%
\pgfpathlineto{\pgfqpoint{1.372280in}{2.260652in}}%
\pgfpathlineto{\pgfqpoint{1.369469in}{2.262066in}}%
\pgfpathlineto{\pgfqpoint{1.366658in}{2.260198in}}%
\pgfpathlineto{\pgfqpoint{1.363848in}{2.261031in}}%
\pgfpathlineto{\pgfqpoint{1.361037in}{2.262448in}}%
\pgfpathlineto{\pgfqpoint{1.358226in}{2.263866in}}%
\pgfpathlineto{\pgfqpoint{1.355415in}{2.265201in}}%
\pgfpathlineto{\pgfqpoint{1.352605in}{2.264388in}}%
\pgfpathlineto{\pgfqpoint{1.349794in}{2.265832in}}%
\pgfpathlineto{\pgfqpoint{1.346983in}{2.267288in}}%
\pgfpathlineto{\pgfqpoint{1.344173in}{2.268782in}}%
\pgfpathlineto{\pgfqpoint{1.341362in}{2.269086in}}%
\pgfpathlineto{\pgfqpoint{1.338551in}{2.269936in}}%
\pgfpathlineto{\pgfqpoint{1.335741in}{2.270791in}}%
\pgfpathlineto{\pgfqpoint{1.332930in}{2.271078in}}%
\pgfpathlineto{\pgfqpoint{1.330119in}{2.272006in}}%
\pgfpathlineto{\pgfqpoint{1.327309in}{2.267925in}}%
\pgfpathlineto{\pgfqpoint{1.324498in}{2.269322in}}%
\pgfpathlineto{\pgfqpoint{1.321687in}{2.269215in}}%
\pgfpathlineto{\pgfqpoint{1.318877in}{2.270707in}}%
\pgfpathlineto{\pgfqpoint{1.316066in}{2.272215in}}%
\pgfpathlineto{\pgfqpoint{1.313255in}{2.268448in}}%
\pgfpathlineto{\pgfqpoint{1.310445in}{2.266532in}}%
\pgfpathlineto{\pgfqpoint{1.307634in}{2.267174in}}%
\pgfpathlineto{\pgfqpoint{1.304823in}{2.268748in}}%
\pgfpathlineto{\pgfqpoint{1.302012in}{2.270305in}}%
\pgfpathlineto{\pgfqpoint{1.299202in}{2.270183in}}%
\pgfpathlineto{\pgfqpoint{1.296391in}{2.271901in}}%
\pgfpathlineto{\pgfqpoint{1.293580in}{2.261303in}}%
\pgfpathlineto{\pgfqpoint{1.290770in}{2.260804in}}%
\pgfpathlineto{\pgfqpoint{1.287959in}{2.262524in}}%
\pgfpathlineto{\pgfqpoint{1.285148in}{2.258011in}}%
\pgfpathlineto{\pgfqpoint{1.282338in}{2.256203in}}%
\pgfpathlineto{\pgfqpoint{1.279527in}{2.257942in}}%
\pgfpathlineto{\pgfqpoint{1.276716in}{2.256730in}}%
\pgfpathlineto{\pgfqpoint{1.273906in}{2.258139in}}%
\pgfpathlineto{\pgfqpoint{1.271095in}{2.259044in}}%
\pgfpathlineto{\pgfqpoint{1.268284in}{2.260509in}}%
\pgfpathlineto{\pgfqpoint{1.265474in}{2.260835in}}%
\pgfpathlineto{\pgfqpoint{1.262663in}{2.262316in}}%
\pgfpathlineto{\pgfqpoint{1.259852in}{2.250835in}}%
\pgfpathlineto{\pgfqpoint{1.257041in}{2.251777in}}%
\pgfpathlineto{\pgfqpoint{1.254231in}{2.251741in}}%
\pgfpathlineto{\pgfqpoint{1.251420in}{2.253455in}}%
\pgfpathlineto{\pgfqpoint{1.248609in}{2.247133in}}%
\pgfpathlineto{\pgfqpoint{1.245799in}{2.246911in}}%
\pgfpathlineto{\pgfqpoint{1.242988in}{2.239280in}}%
\pgfpathlineto{\pgfqpoint{1.240177in}{2.240049in}}%
\pgfpathlineto{\pgfqpoint{1.237367in}{2.241177in}}%
\pgfpathlineto{\pgfqpoint{1.234556in}{2.242395in}}%
\pgfpathlineto{\pgfqpoint{1.231745in}{2.240552in}}%
\pgfpathlineto{\pgfqpoint{1.228935in}{2.237287in}}%
\pgfpathlineto{\pgfqpoint{1.226124in}{2.238776in}}%
\pgfpathlineto{\pgfqpoint{1.223313in}{2.233405in}}%
\pgfpathlineto{\pgfqpoint{1.220503in}{2.233976in}}%
\pgfpathlineto{\pgfqpoint{1.217692in}{2.229749in}}%
\pgfpathlineto{\pgfqpoint{1.214881in}{2.231684in}}%
\pgfpathlineto{\pgfqpoint{1.212070in}{2.222687in}}%
\pgfpathlineto{\pgfqpoint{1.209260in}{2.219321in}}%
\pgfpathlineto{\pgfqpoint{1.206449in}{2.221257in}}%
\pgfpathlineto{\pgfqpoint{1.203638in}{2.216765in}}%
\pgfpathlineto{\pgfqpoint{1.200828in}{2.218196in}}%
\pgfpathlineto{\pgfqpoint{1.198017in}{2.219654in}}%
\pgfpathlineto{\pgfqpoint{1.195206in}{2.217134in}}%
\pgfpathlineto{\pgfqpoint{1.192396in}{2.218210in}}%
\pgfpathlineto{\pgfqpoint{1.189585in}{2.217838in}}%
\pgfpathlineto{\pgfqpoint{1.186774in}{2.215422in}}%
\pgfpathlineto{\pgfqpoint{1.183964in}{2.217474in}}%
\pgfpathlineto{\pgfqpoint{1.181153in}{2.218638in}}%
\pgfpathlineto{\pgfqpoint{1.178342in}{2.220751in}}%
\pgfpathlineto{\pgfqpoint{1.175532in}{2.222132in}}%
\pgfpathlineto{\pgfqpoint{1.172721in}{2.222936in}}%
\pgfpathlineto{\pgfqpoint{1.169910in}{2.225103in}}%
\pgfpathlineto{\pgfqpoint{1.167099in}{2.222221in}}%
\pgfpathlineto{\pgfqpoint{1.164289in}{2.219336in}}%
\pgfpathlineto{\pgfqpoint{1.161478in}{2.220999in}}%
\pgfpathlineto{\pgfqpoint{1.158667in}{2.223151in}}%
\pgfpathlineto{\pgfqpoint{1.155857in}{2.223767in}}%
\pgfpathlineto{\pgfqpoint{1.153046in}{2.226008in}}%
\pgfpathlineto{\pgfqpoint{1.150235in}{2.221992in}}%
\pgfpathlineto{\pgfqpoint{1.147425in}{2.222130in}}%
\pgfpathlineto{\pgfqpoint{1.144614in}{2.222374in}}%
\pgfpathlineto{\pgfqpoint{1.141803in}{2.215773in}}%
\pgfpathlineto{\pgfqpoint{1.138993in}{2.217857in}}%
\pgfpathlineto{\pgfqpoint{1.136182in}{2.218741in}}%
\pgfpathlineto{\pgfqpoint{1.133371in}{2.216235in}}%
\pgfpathlineto{\pgfqpoint{1.130561in}{2.208680in}}%
\pgfpathlineto{\pgfqpoint{1.127750in}{2.211085in}}%
\pgfpathlineto{\pgfqpoint{1.124939in}{2.213499in}}%
\pgfpathlineto{\pgfqpoint{1.122128in}{2.216025in}}%
\pgfpathlineto{\pgfqpoint{1.119318in}{2.217975in}}%
\pgfpathlineto{\pgfqpoint{1.116507in}{2.215694in}}%
\pgfpathlineto{\pgfqpoint{1.113696in}{2.217741in}}%
\pgfpathlineto{\pgfqpoint{1.110886in}{2.216247in}}%
\pgfpathlineto{\pgfqpoint{1.108075in}{2.216451in}}%
\pgfpathlineto{\pgfqpoint{1.105264in}{2.205316in}}%
\pgfpathlineto{\pgfqpoint{1.102454in}{2.201042in}}%
\pgfpathlineto{\pgfqpoint{1.099643in}{2.202307in}}%
\pgfpathlineto{\pgfqpoint{1.096832in}{2.203353in}}%
\pgfpathlineto{\pgfqpoint{1.094022in}{2.204710in}}%
\pgfpathlineto{\pgfqpoint{1.091211in}{2.207209in}}%
\pgfpathlineto{\pgfqpoint{1.088400in}{2.208773in}}%
\pgfpathlineto{\pgfqpoint{1.085590in}{2.208196in}}%
\pgfpathlineto{\pgfqpoint{1.082779in}{2.211113in}}%
\pgfpathlineto{\pgfqpoint{1.079968in}{2.208271in}}%
\pgfpathlineto{\pgfqpoint{1.077158in}{2.207933in}}%
\pgfpathlineto{\pgfqpoint{1.074347in}{2.210866in}}%
\pgfpathlineto{\pgfqpoint{1.071536in}{2.209018in}}%
\pgfpathlineto{\pgfqpoint{1.068725in}{2.212169in}}%
\pgfpathlineto{\pgfqpoint{1.065915in}{2.214114in}}%
\pgfpathlineto{\pgfqpoint{1.063104in}{2.217177in}}%
\pgfpathlineto{\pgfqpoint{1.060293in}{2.220415in}}%
\pgfpathlineto{\pgfqpoint{1.057483in}{2.223749in}}%
\pgfpathlineto{\pgfqpoint{1.054672in}{2.227369in}}%
\pgfpathlineto{\pgfqpoint{1.051861in}{2.226162in}}%
\pgfpathlineto{\pgfqpoint{1.049051in}{2.229967in}}%
\pgfpathlineto{\pgfqpoint{1.046240in}{2.220236in}}%
\pgfpathlineto{\pgfqpoint{1.043429in}{2.224276in}}%
\pgfpathlineto{\pgfqpoint{1.040619in}{2.224023in}}%
\pgfpathlineto{\pgfqpoint{1.037808in}{2.228307in}}%
\pgfpathlineto{\pgfqpoint{1.034997in}{2.225993in}}%
\pgfpathlineto{\pgfqpoint{1.032187in}{2.229490in}}%
\pgfpathlineto{\pgfqpoint{1.029376in}{2.230011in}}%
\pgfpathlineto{\pgfqpoint{1.026565in}{2.233818in}}%
\pgfpathlineto{\pgfqpoint{1.023754in}{2.238608in}}%
\pgfpathlineto{\pgfqpoint{1.020944in}{2.239734in}}%
\pgfpathlineto{\pgfqpoint{1.018133in}{2.235855in}}%
\pgfpathlineto{\pgfqpoint{1.015322in}{2.237507in}}%
\pgfpathlineto{\pgfqpoint{1.012512in}{2.242326in}}%
\pgfpathlineto{\pgfqpoint{1.009701in}{2.248078in}}%
\pgfpathlineto{\pgfqpoint{1.006890in}{2.246488in}}%
\pgfpathlineto{\pgfqpoint{1.004080in}{2.251114in}}%
\pgfpathlineto{\pgfqpoint{1.001269in}{2.236649in}}%
\pgfpathlineto{\pgfqpoint{0.998458in}{2.236556in}}%
\pgfpathlineto{\pgfqpoint{0.995648in}{2.241618in}}%
\pgfpathlineto{\pgfqpoint{0.992837in}{2.246965in}}%
\pgfpathlineto{\pgfqpoint{0.990026in}{2.253858in}}%
\pgfpathlineto{\pgfqpoint{0.987216in}{2.260369in}}%
\pgfpathlineto{\pgfqpoint{0.984405in}{2.267945in}}%
\pgfpathlineto{\pgfqpoint{0.981594in}{2.271033in}}%
\pgfpathlineto{\pgfqpoint{0.978783in}{2.279671in}}%
\pgfpathlineto{\pgfqpoint{0.975973in}{2.288890in}}%
\pgfpathlineto{\pgfqpoint{0.973162in}{2.277234in}}%
\pgfpathlineto{\pgfqpoint{0.970351in}{2.287536in}}%
\pgfpathlineto{\pgfqpoint{0.967541in}{2.295167in}}%
\pgfpathlineto{\pgfqpoint{0.964730in}{2.298976in}}%
\pgfpathlineto{\pgfqpoint{0.961919in}{2.311106in}}%
\pgfpathlineto{\pgfqpoint{0.959109in}{2.324450in}}%
\pgfpathlineto{\pgfqpoint{0.956298in}{2.336720in}}%
\pgfpathlineto{\pgfqpoint{0.953487in}{2.352351in}}%
\pgfpathlineto{\pgfqpoint{0.950677in}{2.365446in}}%
\pgfpathlineto{\pgfqpoint{0.947866in}{2.370073in}}%
\pgfpathlineto{\pgfqpoint{0.945055in}{2.298336in}}%
\pgfpathlineto{\pgfqpoint{0.942245in}{2.303260in}}%
\pgfpathlineto{\pgfqpoint{0.939434in}{2.291337in}}%
\pgfpathlineto{\pgfqpoint{0.936623in}{2.319169in}}%
\pgfpathlineto{\pgfqpoint{0.933812in}{2.250340in}}%
\pgfpathlineto{\pgfqpoint{0.931002in}{2.268192in}}%
\pgfpathlineto{\pgfqpoint{0.928191in}{2.292312in}}%
\pgfpathlineto{\pgfqpoint{0.925380in}{2.332861in}}%
\pgfpathlineto{\pgfqpoint{0.922570in}{2.381465in}}%
\pgfpathclose%
\pgfusepath{stroke,fill}%
\end{pgfscope}%
\begin{pgfscope}%
\pgfpathrectangle{\pgfqpoint{0.711206in}{0.331635in}}{\pgfqpoint{4.650000in}{3.020000in}}%
\pgfusepath{clip}%
\pgfsetroundcap%
\pgfsetroundjoin%
\pgfsetlinewidth{1.505625pt}%
\definecolor{currentstroke}{rgb}{0.121569,0.466667,0.705882}%
\pgfsetstrokecolor{currentstroke}%
\pgfsetdash{}{0pt}%
\pgfpathmoveto{\pgfqpoint{0.922570in}{2.024761in}}%
\pgfpathlineto{\pgfqpoint{0.925380in}{1.957106in}}%
\pgfpathlineto{\pgfqpoint{0.928191in}{2.015632in}}%
\pgfpathlineto{\pgfqpoint{0.931002in}{1.860970in}}%
\pgfpathlineto{\pgfqpoint{0.933812in}{1.533168in}}%
\pgfpathlineto{\pgfqpoint{0.939434in}{2.093134in}}%
\pgfpathlineto{\pgfqpoint{0.942245in}{2.031416in}}%
\pgfpathlineto{\pgfqpoint{0.945055in}{2.320079in}}%
\pgfpathlineto{\pgfqpoint{0.947866in}{2.075864in}}%
\pgfpathlineto{\pgfqpoint{0.950677in}{1.996339in}}%
\pgfpathlineto{\pgfqpoint{0.953487in}{1.890353in}}%
\pgfpathlineto{\pgfqpoint{0.956298in}{1.805625in}}%
\pgfpathlineto{\pgfqpoint{0.959109in}{1.883271in}}%
\pgfpathlineto{\pgfqpoint{0.961919in}{1.897436in}}%
\pgfpathlineto{\pgfqpoint{0.964730in}{1.705108in}}%
\pgfpathlineto{\pgfqpoint{0.967541in}{1.969007in}}%
\pgfpathlineto{\pgfqpoint{0.970351in}{1.854653in}}%
\pgfpathlineto{\pgfqpoint{0.973162in}{2.145509in}}%
\pgfpathlineto{\pgfqpoint{0.975973in}{1.883324in}}%
\pgfpathlineto{\pgfqpoint{0.978783in}{1.841057in}}%
\pgfpathlineto{\pgfqpoint{0.981594in}{2.009792in}}%
\pgfpathlineto{\pgfqpoint{0.987216in}{1.799147in}}%
\pgfpathlineto{\pgfqpoint{0.990026in}{1.890353in}}%
\pgfpathlineto{\pgfqpoint{0.992837in}{1.939533in}}%
\pgfpathlineto{\pgfqpoint{0.995648in}{1.777702in}}%
\pgfpathlineto{\pgfqpoint{0.998458in}{2.031035in}}%
\pgfpathlineto{\pgfqpoint{1.001269in}{2.181536in}}%
\pgfpathlineto{\pgfqpoint{1.004080in}{1.780095in}}%
\pgfpathlineto{\pgfqpoint{1.006890in}{1.653314in}}%
\pgfpathlineto{\pgfqpoint{1.009701in}{1.848129in}}%
\pgfpathlineto{\pgfqpoint{1.012512in}{1.911480in}}%
\pgfpathlineto{\pgfqpoint{1.018133in}{2.084743in}}%
\pgfpathlineto{\pgfqpoint{1.020944in}{1.695963in}}%
\pgfpathlineto{\pgfqpoint{1.023754in}{1.876371in}}%
\pgfpathlineto{\pgfqpoint{1.026565in}{1.925284in}}%
\pgfpathlineto{\pgfqpoint{1.029376in}{1.679527in}}%
\pgfpathlineto{\pgfqpoint{1.032187in}{1.925698in}}%
\pgfpathlineto{\pgfqpoint{1.034997in}{2.058854in}}%
\pgfpathlineto{\pgfqpoint{1.037808in}{1.848406in}}%
\pgfpathlineto{\pgfqpoint{1.040619in}{2.022783in}}%
\pgfpathlineto{\pgfqpoint{1.043429in}{1.841700in}}%
\pgfpathlineto{\pgfqpoint{1.046240in}{2.166259in}}%
\pgfpathlineto{\pgfqpoint{1.049051in}{1.876678in}}%
\pgfpathlineto{\pgfqpoint{1.051861in}{2.046861in}}%
\pgfpathlineto{\pgfqpoint{1.054672in}{1.876809in}}%
\pgfpathlineto{\pgfqpoint{1.057483in}{1.897127in}}%
\pgfpathlineto{\pgfqpoint{1.060293in}{1.897124in}}%
\pgfpathlineto{\pgfqpoint{1.063104in}{1.903885in}}%
\pgfpathlineto{\pgfqpoint{1.065915in}{1.957830in}}%
\pgfpathlineto{\pgfqpoint{1.068725in}{1.876882in}}%
\pgfpathlineto{\pgfqpoint{1.071536in}{2.057873in}}%
\pgfpathlineto{\pgfqpoint{1.074347in}{1.890353in}}%
\pgfpathlineto{\pgfqpoint{1.077158in}{1.682346in}}%
\pgfpathlineto{\pgfqpoint{1.079968in}{2.078356in}}%
\pgfpathlineto{\pgfqpoint{1.082779in}{1.870323in}}%
\pgfpathlineto{\pgfqpoint{1.085590in}{2.030003in}}%
\pgfpathlineto{\pgfqpoint{1.088400in}{1.744020in}}%
\pgfpathlineto{\pgfqpoint{1.091211in}{1.796479in}}%
\pgfpathlineto{\pgfqpoint{1.094022in}{1.964161in}}%
\pgfpathlineto{\pgfqpoint{1.096832in}{1.722077in}}%
\pgfpathlineto{\pgfqpoint{1.099643in}{1.727003in}}%
\pgfpathlineto{\pgfqpoint{1.102454in}{2.107759in}}%
\pgfpathlineto{\pgfqpoint{1.105264in}{2.223831in}}%
\pgfpathlineto{\pgfqpoint{1.108075in}{1.684476in}}%
\pgfpathlineto{\pgfqpoint{1.110886in}{2.056604in}}%
\pgfpathlineto{\pgfqpoint{1.113696in}{1.923382in}}%
\pgfpathlineto{\pgfqpoint{1.116507in}{1.610610in}}%
\pgfpathlineto{\pgfqpoint{1.119318in}{1.755276in}}%
\pgfpathlineto{\pgfqpoint{1.122128in}{1.842786in}}%
\pgfpathlineto{\pgfqpoint{1.124939in}{1.808454in}}%
\pgfpathlineto{\pgfqpoint{1.127750in}{1.814882in}}%
\pgfpathlineto{\pgfqpoint{1.130561in}{2.183230in}}%
\pgfpathlineto{\pgfqpoint{1.138993in}{1.896971in}}%
\pgfpathlineto{\pgfqpoint{1.141803in}{2.178656in}}%
\pgfpathlineto{\pgfqpoint{1.144614in}{2.006708in}}%
\pgfpathlineto{\pgfqpoint{1.147425in}{1.676338in}}%
\pgfpathlineto{\pgfqpoint{1.150235in}{1.546760in}}%
\pgfpathlineto{\pgfqpoint{1.153046in}{1.809926in}}%
\pgfpathlineto{\pgfqpoint{1.155857in}{1.680554in}}%
\pgfpathlineto{\pgfqpoint{1.161478in}{1.917765in}}%
\pgfpathlineto{\pgfqpoint{1.164289in}{2.101108in}}%
\pgfpathlineto{\pgfqpoint{1.167099in}{1.555843in}}%
\pgfpathlineto{\pgfqpoint{1.169910in}{1.814217in}}%
\pgfpathlineto{\pgfqpoint{1.172721in}{1.680712in}}%
\pgfpathlineto{\pgfqpoint{1.175532in}{1.713374in}}%
\pgfpathlineto{\pgfqpoint{1.181153in}{1.940358in}}%
\pgfpathlineto{\pgfqpoint{1.183964in}{1.818867in}}%
\pgfpathlineto{\pgfqpoint{1.186774in}{1.549701in}}%
\pgfpathlineto{\pgfqpoint{1.189585in}{2.014467in}}%
\pgfpathlineto{\pgfqpoint{1.192396in}{1.941160in}}%
\pgfpathlineto{\pgfqpoint{1.195206in}{2.091870in}}%
\pgfpathlineto{\pgfqpoint{1.200828in}{1.725260in}}%
\pgfpathlineto{\pgfqpoint{1.203638in}{1.480580in}}%
\pgfpathlineto{\pgfqpoint{1.209260in}{2.119486in}}%
\pgfpathlineto{\pgfqpoint{1.212070in}{2.259820in}}%
\pgfpathlineto{\pgfqpoint{1.214881in}{1.818626in}}%
\pgfpathlineto{\pgfqpoint{1.217692in}{2.154038in}}%
\pgfpathlineto{\pgfqpoint{1.220503in}{1.655400in}}%
\pgfpathlineto{\pgfqpoint{1.223313in}{2.188747in}}%
\pgfpathlineto{\pgfqpoint{1.226124in}{1.904414in}}%
\pgfpathlineto{\pgfqpoint{1.228935in}{2.134307in}}%
\pgfpathlineto{\pgfqpoint{1.231745in}{2.089508in}}%
\pgfpathlineto{\pgfqpoint{1.234556in}{1.931228in}}%
\pgfpathlineto{\pgfqpoint{1.237367in}{1.937899in}}%
\pgfpathlineto{\pgfqpoint{1.240177in}{1.964763in}}%
\pgfpathlineto{\pgfqpoint{1.242988in}{1.375492in}}%
\pgfpathlineto{\pgfqpoint{1.245799in}{2.022474in}}%
\pgfpathlineto{\pgfqpoint{1.248609in}{1.390536in}}%
\pgfpathlineto{\pgfqpoint{1.251420in}{1.861690in}}%
\pgfpathlineto{\pgfqpoint{1.254231in}{2.011793in}}%
\pgfpathlineto{\pgfqpoint{1.257041in}{1.668156in}}%
\pgfpathlineto{\pgfqpoint{1.259852in}{2.359504in}}%
\pgfpathlineto{\pgfqpoint{1.262663in}{1.897349in}}%
\pgfpathlineto{\pgfqpoint{1.265474in}{1.994903in}}%
\pgfpathlineto{\pgfqpoint{1.268284in}{1.722720in}}%
\pgfpathlineto{\pgfqpoint{1.271095in}{1.663871in}}%
\pgfpathlineto{\pgfqpoint{1.273906in}{1.897483in}}%
\pgfpathlineto{\pgfqpoint{1.276716in}{1.536921in}}%
\pgfpathlineto{\pgfqpoint{1.279527in}{1.795163in}}%
\pgfpathlineto{\pgfqpoint{1.282338in}{1.503400in}}%
\pgfpathlineto{\pgfqpoint{1.285148in}{2.196284in}}%
\pgfpathlineto{\pgfqpoint{1.287959in}{1.816312in}}%
\pgfpathlineto{\pgfqpoint{1.290770in}{2.038069in}}%
\pgfpathlineto{\pgfqpoint{1.293580in}{2.367692in}}%
\pgfpathlineto{\pgfqpoint{1.296391in}{1.804692in}}%
\pgfpathlineto{\pgfqpoint{1.299202in}{1.579299in}}%
\pgfpathlineto{\pgfqpoint{1.302012in}{1.736053in}}%
\pgfpathlineto{\pgfqpoint{1.304823in}{1.741902in}}%
\pgfpathlineto{\pgfqpoint{1.307634in}{1.964763in}}%
\pgfpathlineto{\pgfqpoint{1.310445in}{2.111382in}}%
\pgfpathlineto{\pgfqpoint{1.313255in}{2.187226in}}%
\pgfpathlineto{\pgfqpoint{1.316066in}{1.739034in}}%
\pgfpathlineto{\pgfqpoint{1.318877in}{1.861356in}}%
\pgfpathlineto{\pgfqpoint{1.321687in}{2.020400in}}%
\pgfpathlineto{\pgfqpoint{1.324498in}{1.875959in}}%
\pgfpathlineto{\pgfqpoint{1.327309in}{2.203853in}}%
\pgfpathlineto{\pgfqpoint{1.330119in}{1.939626in}}%
\pgfpathlineto{\pgfqpoint{1.332930in}{1.995394in}}%
\pgfpathlineto{\pgfqpoint{1.335741in}{1.946074in}}%
\pgfpathlineto{\pgfqpoint{1.338551in}{1.945866in}}%
\pgfpathlineto{\pgfqpoint{1.341362in}{1.993887in}}%
\pgfpathlineto{\pgfqpoint{1.344173in}{1.786819in}}%
\pgfpathlineto{\pgfqpoint{1.346983in}{1.772139in}}%
\pgfpathlineto{\pgfqpoint{1.349794in}{1.771195in}}%
\pgfpathlineto{\pgfqpoint{1.352605in}{2.065258in}}%
\pgfpathlineto{\pgfqpoint{1.355415in}{1.743571in}}%
\pgfpathlineto{\pgfqpoint{1.358226in}{1.834054in}}%
\pgfpathlineto{\pgfqpoint{1.361037in}{1.826762in}}%
\pgfpathlineto{\pgfqpoint{1.366658in}{1.482381in}}%
\pgfpathlineto{\pgfqpoint{1.369469in}{1.816312in}}%
\pgfpathlineto{\pgfqpoint{1.372280in}{1.845752in}}%
\pgfpathlineto{\pgfqpoint{1.375090in}{1.986822in}}%
\pgfpathlineto{\pgfqpoint{1.377901in}{1.696788in}}%
\pgfpathlineto{\pgfqpoint{1.380712in}{1.641001in}}%
\pgfpathlineto{\pgfqpoint{1.383522in}{2.229355in}}%
\pgfpathlineto{\pgfqpoint{1.386333in}{1.642508in}}%
\pgfpathlineto{\pgfqpoint{1.389144in}{1.875199in}}%
\pgfpathlineto{\pgfqpoint{1.391954in}{2.316324in}}%
\pgfpathlineto{\pgfqpoint{1.394765in}{1.342592in}}%
\pgfpathlineto{\pgfqpoint{1.397576in}{1.305823in}}%
\pgfpathlineto{\pgfqpoint{1.400386in}{1.937968in}}%
\pgfpathlineto{\pgfqpoint{1.403197in}{1.794972in}}%
\pgfpathlineto{\pgfqpoint{1.406008in}{2.001572in}}%
\pgfpathlineto{\pgfqpoint{1.408819in}{1.898266in}}%
\pgfpathlineto{\pgfqpoint{1.411629in}{1.850748in}}%
\pgfpathlineto{\pgfqpoint{1.414440in}{1.929958in}}%
\pgfpathlineto{\pgfqpoint{1.417251in}{1.747280in}}%
\pgfpathlineto{\pgfqpoint{1.420061in}{1.906318in}}%
\pgfpathlineto{\pgfqpoint{1.422872in}{1.858406in}}%
\pgfpathlineto{\pgfqpoint{1.425683in}{1.681014in}}%
\pgfpathlineto{\pgfqpoint{1.428493in}{1.719100in}}%
\pgfpathlineto{\pgfqpoint{1.434115in}{2.173305in}}%
\pgfpathlineto{\pgfqpoint{1.436925in}{1.607401in}}%
\pgfpathlineto{\pgfqpoint{1.439736in}{1.955502in}}%
\pgfpathlineto{\pgfqpoint{1.442547in}{1.947126in}}%
\pgfpathlineto{\pgfqpoint{1.445357in}{1.694796in}}%
\pgfpathlineto{\pgfqpoint{1.448168in}{1.841061in}}%
\pgfpathlineto{\pgfqpoint{1.450979in}{2.029594in}}%
\pgfpathlineto{\pgfqpoint{1.456600in}{1.482267in}}%
\pgfpathlineto{\pgfqpoint{1.459411in}{1.797187in}}%
\pgfpathlineto{\pgfqpoint{1.462222in}{1.830761in}}%
\pgfpathlineto{\pgfqpoint{1.465032in}{1.753243in}}%
\pgfpathlineto{\pgfqpoint{1.470654in}{2.309931in}}%
\pgfpathlineto{\pgfqpoint{1.473464in}{1.513268in}}%
\pgfpathlineto{\pgfqpoint{1.476275in}{1.805246in}}%
\pgfpathlineto{\pgfqpoint{1.479086in}{1.614702in}}%
\pgfpathlineto{\pgfqpoint{1.481896in}{2.028815in}}%
\pgfpathlineto{\pgfqpoint{1.484707in}{1.976244in}}%
\pgfpathlineto{\pgfqpoint{1.487518in}{1.804462in}}%
\pgfpathlineto{\pgfqpoint{1.490328in}{2.044602in}}%
\pgfpathlineto{\pgfqpoint{1.493139in}{2.068318in}}%
\pgfpathlineto{\pgfqpoint{1.495950in}{1.669701in}}%
\pgfpathlineto{\pgfqpoint{1.498761in}{2.320136in}}%
\pgfpathlineto{\pgfqpoint{1.501571in}{1.756847in}}%
\pgfpathlineto{\pgfqpoint{1.504382in}{1.890353in}}%
\pgfpathlineto{\pgfqpoint{1.507193in}{1.814728in}}%
\pgfpathlineto{\pgfqpoint{1.510003in}{1.323733in}}%
\pgfpathlineto{\pgfqpoint{1.512814in}{1.828975in}}%
\pgfpathlineto{\pgfqpoint{1.515625in}{1.872770in}}%
\pgfpathlineto{\pgfqpoint{1.518435in}{1.863940in}}%
\pgfpathlineto{\pgfqpoint{1.521246in}{1.533631in}}%
\pgfpathlineto{\pgfqpoint{1.524057in}{1.653824in}}%
\pgfpathlineto{\pgfqpoint{1.526867in}{1.945270in}}%
\pgfpathlineto{\pgfqpoint{1.529678in}{2.008652in}}%
\pgfpathlineto{\pgfqpoint{1.532489in}{1.935604in}}%
\pgfpathlineto{\pgfqpoint{1.535299in}{1.328688in}}%
\pgfpathlineto{\pgfqpoint{1.540921in}{2.171325in}}%
\pgfpathlineto{\pgfqpoint{1.543731in}{2.010483in}}%
\pgfpathlineto{\pgfqpoint{1.546542in}{1.927122in}}%
\pgfpathlineto{\pgfqpoint{1.549353in}{2.009232in}}%
\pgfpathlineto{\pgfqpoint{1.552164in}{1.465381in}}%
\pgfpathlineto{\pgfqpoint{1.554974in}{1.918441in}}%
\pgfpathlineto{\pgfqpoint{1.557785in}{1.871634in}}%
\pgfpathlineto{\pgfqpoint{1.560596in}{2.214595in}}%
\pgfpathlineto{\pgfqpoint{1.563406in}{1.798429in}}%
\pgfpathlineto{\pgfqpoint{1.566217in}{2.119109in}}%
\pgfpathlineto{\pgfqpoint{1.569028in}{1.890353in}}%
\pgfpathlineto{\pgfqpoint{1.571838in}{1.835771in}}%
\pgfpathlineto{\pgfqpoint{1.574649in}{2.303803in}}%
\pgfpathlineto{\pgfqpoint{1.577460in}{1.908072in}}%
\pgfpathlineto{\pgfqpoint{1.580270in}{1.756947in}}%
\pgfpathlineto{\pgfqpoint{1.583081in}{1.926045in}}%
\pgfpathlineto{\pgfqpoint{1.585892in}{1.917066in}}%
\pgfpathlineto{\pgfqpoint{1.588702in}{2.172355in}}%
\pgfpathlineto{\pgfqpoint{1.591513in}{1.986073in}}%
\pgfpathlineto{\pgfqpoint{1.597135in}{1.706433in}}%
\pgfpathlineto{\pgfqpoint{1.602756in}{1.890353in}}%
\pgfpathlineto{\pgfqpoint{1.608377in}{1.683700in}}%
\pgfpathlineto{\pgfqpoint{1.611188in}{2.177097in}}%
\pgfpathlineto{\pgfqpoint{1.613999in}{2.430710in}}%
\pgfpathlineto{\pgfqpoint{1.616809in}{1.847494in}}%
\pgfpathlineto{\pgfqpoint{1.619620in}{1.778340in}}%
\pgfpathlineto{\pgfqpoint{1.622431in}{1.950772in}}%
\pgfpathlineto{\pgfqpoint{1.625241in}{1.777951in}}%
\pgfpathlineto{\pgfqpoint{1.628052in}{2.148477in}}%
\pgfpathlineto{\pgfqpoint{1.630863in}{2.127305in}}%
\pgfpathlineto{\pgfqpoint{1.633673in}{2.024083in}}%
\pgfpathlineto{\pgfqpoint{1.636484in}{1.890353in}}%
\pgfpathlineto{\pgfqpoint{1.639295in}{1.815277in}}%
\pgfpathlineto{\pgfqpoint{1.644916in}{2.274489in}}%
\pgfpathlineto{\pgfqpoint{1.647727in}{0.915113in}}%
\pgfpathlineto{\pgfqpoint{1.650538in}{1.838580in}}%
\pgfpathlineto{\pgfqpoint{1.653348in}{1.847071in}}%
\pgfpathlineto{\pgfqpoint{1.656159in}{1.777229in}}%
\pgfpathlineto{\pgfqpoint{1.658970in}{1.899085in}}%
\pgfpathlineto{\pgfqpoint{1.661780in}{1.951337in}}%
\pgfpathlineto{\pgfqpoint{1.664591in}{2.054634in}}%
\pgfpathlineto{\pgfqpoint{1.667402in}{1.950425in}}%
\pgfpathlineto{\pgfqpoint{1.670212in}{1.657303in}}%
\pgfpathlineto{\pgfqpoint{1.673023in}{2.114836in}}%
\pgfpathlineto{\pgfqpoint{1.675834in}{1.752610in}}%
\pgfpathlineto{\pgfqpoint{1.678644in}{2.062334in}}%
\pgfpathlineto{\pgfqpoint{1.681455in}{1.752927in}}%
\pgfpathlineto{\pgfqpoint{1.684266in}{1.769058in}}%
\pgfpathlineto{\pgfqpoint{1.687077in}{2.020274in}}%
\pgfpathlineto{\pgfqpoint{1.689887in}{1.916202in}}%
\pgfpathlineto{\pgfqpoint{1.692698in}{2.001851in}}%
\pgfpathlineto{\pgfqpoint{1.698319in}{1.873303in}}%
\pgfpathlineto{\pgfqpoint{1.701130in}{1.813384in}}%
\pgfpathlineto{\pgfqpoint{1.703941in}{1.873195in}}%
\pgfpathlineto{\pgfqpoint{1.706751in}{1.630581in}}%
\pgfpathlineto{\pgfqpoint{1.709562in}{1.536801in}}%
\pgfpathlineto{\pgfqpoint{1.712373in}{2.015052in}}%
\pgfpathlineto{\pgfqpoint{1.715183in}{1.756708in}}%
\pgfpathlineto{\pgfqpoint{1.717994in}{2.147663in}}%
\pgfpathlineto{\pgfqpoint{1.720805in}{2.134649in}}%
\pgfpathlineto{\pgfqpoint{1.723615in}{1.855698in}}%
\pgfpathlineto{\pgfqpoint{1.726426in}{2.002690in}}%
\pgfpathlineto{\pgfqpoint{1.729237in}{2.061545in}}%
\pgfpathlineto{\pgfqpoint{1.732048in}{2.051186in}}%
\pgfpathlineto{\pgfqpoint{1.734858in}{2.057798in}}%
\pgfpathlineto{\pgfqpoint{1.737669in}{1.989925in}}%
\pgfpathlineto{\pgfqpoint{1.740480in}{1.673767in}}%
\pgfpathlineto{\pgfqpoint{1.743290in}{1.965682in}}%
\pgfpathlineto{\pgfqpoint{1.746101in}{1.856921in}}%
\pgfpathlineto{\pgfqpoint{1.748912in}{1.907079in}}%
\pgfpathlineto{\pgfqpoint{1.751722in}{1.881993in}}%
\pgfpathlineto{\pgfqpoint{1.754533in}{1.705228in}}%
\pgfpathlineto{\pgfqpoint{1.757344in}{1.805440in}}%
\pgfpathlineto{\pgfqpoint{1.760154in}{1.804954in}}%
\pgfpathlineto{\pgfqpoint{1.762965in}{1.916024in}}%
\pgfpathlineto{\pgfqpoint{1.765776in}{1.796009in}}%
\pgfpathlineto{\pgfqpoint{1.768586in}{2.086942in}}%
\pgfpathlineto{\pgfqpoint{1.771397in}{1.770999in}}%
\pgfpathlineto{\pgfqpoint{1.774208in}{2.320623in}}%
\pgfpathlineto{\pgfqpoint{1.777018in}{1.931876in}}%
\pgfpathlineto{\pgfqpoint{1.779829in}{1.890353in}}%
\pgfpathlineto{\pgfqpoint{1.782640in}{1.614141in}}%
\pgfpathlineto{\pgfqpoint{1.785451in}{1.907240in}}%
\pgfpathlineto{\pgfqpoint{1.788261in}{1.600651in}}%
\pgfpathlineto{\pgfqpoint{1.791072in}{1.959026in}}%
\pgfpathlineto{\pgfqpoint{1.793883in}{1.744043in}}%
\pgfpathlineto{\pgfqpoint{1.796693in}{2.002366in}}%
\pgfpathlineto{\pgfqpoint{1.799504in}{2.346840in}}%
\pgfpathlineto{\pgfqpoint{1.802315in}{2.235987in}}%
\pgfpathlineto{\pgfqpoint{1.805125in}{1.602884in}}%
\pgfpathlineto{\pgfqpoint{1.807936in}{1.782152in}}%
\pgfpathlineto{\pgfqpoint{1.810747in}{1.688512in}}%
\pgfpathlineto{\pgfqpoint{1.813557in}{2.258334in}}%
\pgfpathlineto{\pgfqpoint{1.816368in}{1.898612in}}%
\pgfpathlineto{\pgfqpoint{1.819179in}{2.095347in}}%
\pgfpathlineto{\pgfqpoint{1.821989in}{2.028157in}}%
\pgfpathlineto{\pgfqpoint{1.824800in}{2.178038in}}%
\pgfpathlineto{\pgfqpoint{1.827611in}{2.211377in}}%
\pgfpathlineto{\pgfqpoint{1.830422in}{2.067460in}}%
\pgfpathlineto{\pgfqpoint{1.833232in}{1.713246in}}%
\pgfpathlineto{\pgfqpoint{1.838854in}{2.120711in}}%
\pgfpathlineto{\pgfqpoint{1.841664in}{2.004212in}}%
\pgfpathlineto{\pgfqpoint{1.844475in}{1.920569in}}%
\pgfpathlineto{\pgfqpoint{1.847286in}{1.965627in}}%
\pgfpathlineto{\pgfqpoint{1.850096in}{1.935336in}}%
\pgfpathlineto{\pgfqpoint{1.852907in}{1.860380in}}%
\pgfpathlineto{\pgfqpoint{1.855718in}{2.039621in}}%
\pgfpathlineto{\pgfqpoint{1.858528in}{1.949645in}}%
\pgfpathlineto{\pgfqpoint{1.861339in}{1.956775in}}%
\pgfpathlineto{\pgfqpoint{1.864150in}{1.978459in}}%
\pgfpathlineto{\pgfqpoint{1.866960in}{1.631910in}}%
\pgfpathlineto{\pgfqpoint{1.869771in}{1.740636in}}%
\pgfpathlineto{\pgfqpoint{1.872582in}{2.114369in}}%
\pgfpathlineto{\pgfqpoint{1.875393in}{1.726404in}}%
\pgfpathlineto{\pgfqpoint{1.878203in}{1.950179in}}%
\pgfpathlineto{\pgfqpoint{1.881014in}{1.920177in}}%
\pgfpathlineto{\pgfqpoint{1.883825in}{1.934977in}}%
\pgfpathlineto{\pgfqpoint{1.886635in}{1.838279in}}%
\pgfpathlineto{\pgfqpoint{1.889446in}{1.875441in}}%
\pgfpathlineto{\pgfqpoint{1.892257in}{1.808071in}}%
\pgfpathlineto{\pgfqpoint{1.895067in}{1.994997in}}%
\pgfpathlineto{\pgfqpoint{1.897878in}{2.009054in}}%
\pgfpathlineto{\pgfqpoint{1.900689in}{1.853360in}}%
\pgfpathlineto{\pgfqpoint{1.903499in}{1.927346in}}%
\pgfpathlineto{\pgfqpoint{1.906310in}{2.470094in}}%
\pgfpathlineto{\pgfqpoint{1.911931in}{1.496116in}}%
\pgfpathlineto{\pgfqpoint{1.914742in}{1.861271in}}%
\pgfpathlineto{\pgfqpoint{1.917553in}{1.853920in}}%
\pgfpathlineto{\pgfqpoint{1.920364in}{2.143533in}}%
\pgfpathlineto{\pgfqpoint{1.923174in}{1.709949in}}%
\pgfpathlineto{\pgfqpoint{1.925985in}{1.737110in}}%
\pgfpathlineto{\pgfqpoint{1.928796in}{2.209648in}}%
\pgfpathlineto{\pgfqpoint{1.931606in}{1.854412in}}%
\pgfpathlineto{\pgfqpoint{1.934417in}{1.997918in}}%
\pgfpathlineto{\pgfqpoint{1.937228in}{2.103184in}}%
\pgfpathlineto{\pgfqpoint{1.940038in}{1.613076in}}%
\pgfpathlineto{\pgfqpoint{1.942849in}{1.401634in}}%
\pgfpathlineto{\pgfqpoint{1.945660in}{1.897766in}}%
\pgfpathlineto{\pgfqpoint{1.948470in}{1.499894in}}%
\pgfpathlineto{\pgfqpoint{1.951281in}{2.079350in}}%
\pgfpathlineto{\pgfqpoint{1.954092in}{1.972759in}}%
\pgfpathlineto{\pgfqpoint{1.956902in}{1.912749in}}%
\pgfpathlineto{\pgfqpoint{1.959713in}{2.001829in}}%
\pgfpathlineto{\pgfqpoint{1.962524in}{1.681585in}}%
\pgfpathlineto{\pgfqpoint{1.965334in}{1.784863in}}%
\pgfpathlineto{\pgfqpoint{1.968145in}{1.913021in}}%
\pgfpathlineto{\pgfqpoint{1.970956in}{2.077924in}}%
\pgfpathlineto{\pgfqpoint{1.973767in}{1.343634in}}%
\pgfpathlineto{\pgfqpoint{1.976577in}{1.859384in}}%
\pgfpathlineto{\pgfqpoint{1.979388in}{1.959943in}}%
\pgfpathlineto{\pgfqpoint{1.982199in}{2.353649in}}%
\pgfpathlineto{\pgfqpoint{1.985009in}{1.837913in}}%
\pgfpathlineto{\pgfqpoint{1.987820in}{1.912850in}}%
\pgfpathlineto{\pgfqpoint{1.990631in}{2.083918in}}%
\pgfpathlineto{\pgfqpoint{1.993441in}{1.831062in}}%
\pgfpathlineto{\pgfqpoint{1.996252in}{1.315011in}}%
\pgfpathlineto{\pgfqpoint{1.999063in}{1.758564in}}%
\pgfpathlineto{\pgfqpoint{2.001873in}{1.718056in}}%
\pgfpathlineto{\pgfqpoint{2.007495in}{2.008168in}}%
\pgfpathlineto{\pgfqpoint{2.010305in}{1.976163in}}%
\pgfpathlineto{\pgfqpoint{2.013116in}{2.022005in}}%
\pgfpathlineto{\pgfqpoint{2.015927in}{1.874925in}}%
\pgfpathlineto{\pgfqpoint{2.018738in}{1.982685in}}%
\pgfpathlineto{\pgfqpoint{2.021548in}{1.859639in}}%
\pgfpathlineto{\pgfqpoint{2.024359in}{1.805563in}}%
\pgfpathlineto{\pgfqpoint{2.027170in}{1.898081in}}%
\pgfpathlineto{\pgfqpoint{2.029980in}{1.843924in}}%
\pgfpathlineto{\pgfqpoint{2.032791in}{1.812649in}}%
\pgfpathlineto{\pgfqpoint{2.035602in}{1.820071in}}%
\pgfpathlineto{\pgfqpoint{2.038412in}{1.725064in}}%
\pgfpathlineto{\pgfqpoint{2.041223in}{1.929874in}}%
\pgfpathlineto{\pgfqpoint{2.044034in}{1.976933in}}%
\pgfpathlineto{\pgfqpoint{2.046844in}{1.811665in}}%
\pgfpathlineto{\pgfqpoint{2.049655in}{1.929749in}}%
\pgfpathlineto{\pgfqpoint{2.052466in}{1.523926in}}%
\pgfpathlineto{\pgfqpoint{2.058087in}{2.139615in}}%
\pgfpathlineto{\pgfqpoint{2.060898in}{2.048985in}}%
\pgfpathlineto{\pgfqpoint{2.063709in}{1.659783in}}%
\pgfpathlineto{\pgfqpoint{2.066519in}{1.946335in}}%
\pgfpathlineto{\pgfqpoint{2.069330in}{2.017529in}}%
\pgfpathlineto{\pgfqpoint{2.072141in}{1.771158in}}%
\pgfpathlineto{\pgfqpoint{2.074951in}{1.842407in}}%
\pgfpathlineto{\pgfqpoint{2.077762in}{1.954247in}}%
\pgfpathlineto{\pgfqpoint{2.080573in}{1.738156in}}%
\pgfpathlineto{\pgfqpoint{2.083383in}{1.833887in}}%
\pgfpathlineto{\pgfqpoint{2.086194in}{1.954869in}}%
\pgfpathlineto{\pgfqpoint{2.089005in}{2.263849in}}%
\pgfpathlineto{\pgfqpoint{2.091815in}{1.866790in}}%
\pgfpathlineto{\pgfqpoint{2.094626in}{1.937442in}}%
\pgfpathlineto{\pgfqpoint{2.097437in}{2.046250in}}%
\pgfpathlineto{\pgfqpoint{2.100247in}{1.936806in}}%
\pgfpathlineto{\pgfqpoint{2.103058in}{1.781736in}}%
\pgfpathlineto{\pgfqpoint{2.105869in}{1.773094in}}%
\pgfpathlineto{\pgfqpoint{2.108680in}{2.414699in}}%
\pgfpathlineto{\pgfqpoint{2.111490in}{1.852420in}}%
\pgfpathlineto{\pgfqpoint{2.114301in}{2.116520in}}%
\pgfpathlineto{\pgfqpoint{2.117112in}{1.987304in}}%
\pgfpathlineto{\pgfqpoint{2.122733in}{1.800927in}}%
\pgfpathlineto{\pgfqpoint{2.125544in}{1.800388in}}%
\pgfpathlineto{\pgfqpoint{2.128354in}{1.935403in}}%
\pgfpathlineto{\pgfqpoint{2.131165in}{1.890353in}}%
\pgfpathlineto{\pgfqpoint{2.133976in}{1.709329in}}%
\pgfpathlineto{\pgfqpoint{2.136786in}{1.614631in}}%
\pgfpathlineto{\pgfqpoint{2.139597in}{1.851652in}}%
\pgfpathlineto{\pgfqpoint{2.142408in}{1.828222in}}%
\pgfpathlineto{\pgfqpoint{2.145218in}{1.929215in}}%
\pgfpathlineto{\pgfqpoint{2.148029in}{1.702884in}}%
\pgfpathlineto{\pgfqpoint{2.150840in}{2.031175in}}%
\pgfpathlineto{\pgfqpoint{2.153651in}{1.898138in}}%
\pgfpathlineto{\pgfqpoint{2.156461in}{1.662934in}}%
\pgfpathlineto{\pgfqpoint{2.162083in}{2.224756in}}%
\pgfpathlineto{\pgfqpoint{2.164893in}{1.989994in}}%
\pgfpathlineto{\pgfqpoint{2.167704in}{2.012081in}}%
\pgfpathlineto{\pgfqpoint{2.170515in}{1.852420in}}%
\pgfpathlineto{\pgfqpoint{2.173325in}{2.086562in}}%
\pgfpathlineto{\pgfqpoint{2.176136in}{2.032122in}}%
\pgfpathlineto{\pgfqpoint{2.178947in}{1.868058in}}%
\pgfpathlineto{\pgfqpoint{2.181757in}{2.045724in}}%
\pgfpathlineto{\pgfqpoint{2.184568in}{1.846126in}}%
\pgfpathlineto{\pgfqpoint{2.187379in}{1.689693in}}%
\pgfpathlineto{\pgfqpoint{2.193000in}{2.178906in}}%
\pgfpathlineto{\pgfqpoint{2.195811in}{1.963451in}}%
\pgfpathlineto{\pgfqpoint{2.198621in}{1.984848in}}%
\pgfpathlineto{\pgfqpoint{2.201432in}{1.883105in}}%
\pgfpathlineto{\pgfqpoint{2.204243in}{1.868589in}}%
\pgfpathlineto{\pgfqpoint{2.207054in}{1.802978in}}%
\pgfpathlineto{\pgfqpoint{2.212675in}{1.890353in}}%
\pgfpathlineto{\pgfqpoint{2.215486in}{2.028866in}}%
\pgfpathlineto{\pgfqpoint{2.218296in}{1.773796in}}%
\pgfpathlineto{\pgfqpoint{2.221107in}{1.904973in}}%
\pgfpathlineto{\pgfqpoint{2.223918in}{1.941409in}}%
\pgfpathlineto{\pgfqpoint{2.226728in}{2.049686in}}%
\pgfpathlineto{\pgfqpoint{2.229539in}{1.990868in}}%
\pgfpathlineto{\pgfqpoint{2.232350in}{1.768211in}}%
\pgfpathlineto{\pgfqpoint{2.235160in}{1.969500in}}%
\pgfpathlineto{\pgfqpoint{2.237971in}{1.933348in}}%
\pgfpathlineto{\pgfqpoint{2.240782in}{1.954614in}}%
\pgfpathlineto{\pgfqpoint{2.243592in}{1.804610in}}%
\pgfpathlineto{\pgfqpoint{2.246403in}{2.293312in}}%
\pgfpathlineto{\pgfqpoint{2.249214in}{1.058505in}}%
\pgfpathlineto{\pgfqpoint{2.252025in}{1.816422in}}%
\pgfpathlineto{\pgfqpoint{2.254835in}{1.830944in}}%
\pgfpathlineto{\pgfqpoint{2.257646in}{2.045801in}}%
\pgfpathlineto{\pgfqpoint{2.260457in}{1.853488in}}%
\pgfpathlineto{\pgfqpoint{2.263267in}{1.801504in}}%
\pgfpathlineto{\pgfqpoint{2.266078in}{1.793499in}}%
\pgfpathlineto{\pgfqpoint{2.271699in}{2.220356in}}%
\pgfpathlineto{\pgfqpoint{2.274510in}{2.227403in}}%
\pgfpathlineto{\pgfqpoint{2.277321in}{1.954035in}}%
\pgfpathlineto{\pgfqpoint{2.280131in}{1.826671in}}%
\pgfpathlineto{\pgfqpoint{2.282942in}{1.647294in}}%
\pgfpathlineto{\pgfqpoint{2.285753in}{1.919155in}}%
\pgfpathlineto{\pgfqpoint{2.288563in}{1.782059in}}%
\pgfpathlineto{\pgfqpoint{2.291374in}{1.991452in}}%
\pgfpathlineto{\pgfqpoint{2.294185in}{2.012202in}}%
\pgfpathlineto{\pgfqpoint{2.296996in}{2.187192in}}%
\pgfpathlineto{\pgfqpoint{2.299806in}{1.883354in}}%
\pgfpathlineto{\pgfqpoint{2.302617in}{1.841265in}}%
\pgfpathlineto{\pgfqpoint{2.305428in}{2.044089in}}%
\pgfpathlineto{\pgfqpoint{2.308238in}{1.959713in}}%
\pgfpathlineto{\pgfqpoint{2.311049in}{2.048667in}}%
\pgfpathlineto{\pgfqpoint{2.313860in}{1.828604in}}%
\pgfpathlineto{\pgfqpoint{2.316670in}{2.034038in}}%
\pgfpathlineto{\pgfqpoint{2.319481in}{1.958291in}}%
\pgfpathlineto{\pgfqpoint{2.322292in}{1.788329in}}%
\pgfpathlineto{\pgfqpoint{2.325102in}{1.883527in}}%
\pgfpathlineto{\pgfqpoint{2.327913in}{1.704848in}}%
\pgfpathlineto{\pgfqpoint{2.330724in}{1.966208in}}%
\pgfpathlineto{\pgfqpoint{2.333534in}{1.800665in}}%
\pgfpathlineto{\pgfqpoint{2.336345in}{1.820994in}}%
\pgfpathlineto{\pgfqpoint{2.339156in}{2.014969in}}%
\pgfpathlineto{\pgfqpoint{2.341967in}{1.876558in}}%
\pgfpathlineto{\pgfqpoint{2.344777in}{2.465671in}}%
\pgfpathlineto{\pgfqpoint{2.347588in}{1.983019in}}%
\pgfpathlineto{\pgfqpoint{2.350399in}{1.817592in}}%
\pgfpathlineto{\pgfqpoint{2.353209in}{1.493918in}}%
\pgfpathlineto{\pgfqpoint{2.356020in}{1.739791in}}%
\pgfpathlineto{\pgfqpoint{2.358831in}{1.731306in}}%
\pgfpathlineto{\pgfqpoint{2.361641in}{1.778703in}}%
\pgfpathlineto{\pgfqpoint{2.364452in}{1.700028in}}%
\pgfpathlineto{\pgfqpoint{2.370073in}{1.996742in}}%
\pgfpathlineto{\pgfqpoint{2.372884in}{1.755465in}}%
\pgfpathlineto{\pgfqpoint{2.375695in}{1.925968in}}%
\pgfpathlineto{\pgfqpoint{2.378505in}{1.768913in}}%
\pgfpathlineto{\pgfqpoint{2.381316in}{1.527338in}}%
\pgfpathlineto{\pgfqpoint{2.384127in}{1.809290in}}%
\pgfpathlineto{\pgfqpoint{2.386937in}{1.845951in}}%
\pgfpathlineto{\pgfqpoint{2.389748in}{2.183903in}}%
\pgfpathlineto{\pgfqpoint{2.392559in}{2.027791in}}%
\pgfpathlineto{\pgfqpoint{2.395370in}{1.940671in}}%
\pgfpathlineto{\pgfqpoint{2.398180in}{1.997609in}}%
\pgfpathlineto{\pgfqpoint{2.400991in}{1.940143in}}%
\pgfpathlineto{\pgfqpoint{2.403802in}{1.982382in}}%
\pgfpathlineto{\pgfqpoint{2.406612in}{1.925598in}}%
\pgfpathlineto{\pgfqpoint{2.409423in}{1.890353in}}%
\pgfpathlineto{\pgfqpoint{2.412234in}{1.734648in}}%
\pgfpathlineto{\pgfqpoint{2.415044in}{2.031970in}}%
\pgfpathlineto{\pgfqpoint{2.417855in}{1.698851in}}%
\pgfpathlineto{\pgfqpoint{2.420666in}{2.018294in}}%
\pgfpathlineto{\pgfqpoint{2.423476in}{1.883274in}}%
\pgfpathlineto{\pgfqpoint{2.426287in}{1.996188in}}%
\pgfpathlineto{\pgfqpoint{2.429098in}{1.869246in}}%
\pgfpathlineto{\pgfqpoint{2.431908in}{1.890353in}}%
\pgfpathlineto{\pgfqpoint{2.434719in}{1.734648in}}%
\pgfpathlineto{\pgfqpoint{2.437530in}{1.954247in}}%
\pgfpathlineto{\pgfqpoint{2.440341in}{1.826459in}}%
\pgfpathlineto{\pgfqpoint{2.443151in}{1.968408in}}%
\pgfpathlineto{\pgfqpoint{2.445962in}{1.897429in}}%
\pgfpathlineto{\pgfqpoint{2.448773in}{2.010130in}}%
\pgfpathlineto{\pgfqpoint{2.451583in}{1.820013in}}%
\pgfpathlineto{\pgfqpoint{2.454394in}{1.911490in}}%
\pgfpathlineto{\pgfqpoint{2.457205in}{1.777275in}}%
\pgfpathlineto{\pgfqpoint{2.460015in}{1.847727in}}%
\pgfpathlineto{\pgfqpoint{2.462826in}{2.010813in}}%
\pgfpathlineto{\pgfqpoint{2.465637in}{1.960760in}}%
\pgfpathlineto{\pgfqpoint{2.468447in}{2.009287in}}%
\pgfpathlineto{\pgfqpoint{2.471258in}{2.132273in}}%
\pgfpathlineto{\pgfqpoint{2.474069in}{1.738748in}}%
\pgfpathlineto{\pgfqpoint{2.476879in}{1.862622in}}%
\pgfpathlineto{\pgfqpoint{2.479690in}{2.090244in}}%
\pgfpathlineto{\pgfqpoint{2.482501in}{1.842348in}}%
\pgfpathlineto{\pgfqpoint{2.485312in}{1.849083in}}%
\pgfpathlineto{\pgfqpoint{2.488122in}{2.075177in}}%
\pgfpathlineto{\pgfqpoint{2.490933in}{1.998807in}}%
\pgfpathlineto{\pgfqpoint{2.493744in}{1.991318in}}%
\pgfpathlineto{\pgfqpoint{2.496554in}{1.836590in}}%
\pgfpathlineto{\pgfqpoint{2.499365in}{2.183680in}}%
\pgfpathlineto{\pgfqpoint{2.502176in}{1.751075in}}%
\pgfpathlineto{\pgfqpoint{2.504986in}{2.075769in}}%
\pgfpathlineto{\pgfqpoint{2.507797in}{2.125409in}}%
\pgfpathlineto{\pgfqpoint{2.510608in}{1.929171in}}%
\pgfpathlineto{\pgfqpoint{2.513418in}{1.583538in}}%
\pgfpathlineto{\pgfqpoint{2.516229in}{1.744540in}}%
\pgfpathlineto{\pgfqpoint{2.519040in}{2.108541in}}%
\pgfpathlineto{\pgfqpoint{2.521850in}{2.008034in}}%
\pgfpathlineto{\pgfqpoint{2.524661in}{1.981245in}}%
\pgfpathlineto{\pgfqpoint{2.527472in}{1.948492in}}%
\pgfpathlineto{\pgfqpoint{2.530283in}{1.838685in}}%
\pgfpathlineto{\pgfqpoint{2.533093in}{1.832012in}}%
\pgfpathlineto{\pgfqpoint{2.535904in}{1.838302in}}%
\pgfpathlineto{\pgfqpoint{2.538715in}{1.890353in}}%
\pgfpathlineto{\pgfqpoint{2.541525in}{1.614064in}}%
\pgfpathlineto{\pgfqpoint{2.544336in}{1.930138in}}%
\pgfpathlineto{\pgfqpoint{2.547147in}{1.976191in}}%
\pgfpathlineto{\pgfqpoint{2.549957in}{2.008396in}}%
\pgfpathlineto{\pgfqpoint{2.552768in}{1.759137in}}%
\pgfpathlineto{\pgfqpoint{2.555579in}{1.863971in}}%
\pgfpathlineto{\pgfqpoint{2.561200in}{1.903556in}}%
\pgfpathlineto{\pgfqpoint{2.564011in}{1.989004in}}%
\pgfpathlineto{\pgfqpoint{2.566821in}{1.870675in}}%
\pgfpathlineto{\pgfqpoint{2.569632in}{1.870649in}}%
\pgfpathlineto{\pgfqpoint{2.572443in}{1.929735in}}%
\pgfpathlineto{\pgfqpoint{2.575253in}{1.936168in}}%
\pgfpathlineto{\pgfqpoint{2.580875in}{1.705429in}}%
\pgfpathlineto{\pgfqpoint{2.583686in}{1.783644in}}%
\pgfpathlineto{\pgfqpoint{2.586496in}{2.016985in}}%
\pgfpathlineto{\pgfqpoint{2.589307in}{1.890353in}}%
\pgfpathlineto{\pgfqpoint{2.592118in}{1.976381in}}%
\pgfpathlineto{\pgfqpoint{2.594928in}{1.863936in}}%
\pgfpathlineto{\pgfqpoint{2.597739in}{1.969465in}}%
\pgfpathlineto{\pgfqpoint{2.600550in}{2.125203in}}%
\pgfpathlineto{\pgfqpoint{2.603360in}{1.987127in}}%
\pgfpathlineto{\pgfqpoint{2.606171in}{1.935300in}}%
\pgfpathlineto{\pgfqpoint{2.608982in}{2.087792in}}%
\pgfpathlineto{\pgfqpoint{2.611792in}{1.858684in}}%
\pgfpathlineto{\pgfqpoint{2.614603in}{2.104387in}}%
\pgfpathlineto{\pgfqpoint{2.617414in}{1.859069in}}%
\pgfpathlineto{\pgfqpoint{2.620224in}{1.921637in}}%
\pgfpathlineto{\pgfqpoint{2.623035in}{2.162857in}}%
\pgfpathlineto{\pgfqpoint{2.625846in}{1.754723in}}%
\pgfpathlineto{\pgfqpoint{2.628657in}{2.062760in}}%
\pgfpathlineto{\pgfqpoint{2.631467in}{1.730327in}}%
\pgfpathlineto{\pgfqpoint{2.634278in}{1.902724in}}%
\pgfpathlineto{\pgfqpoint{2.637089in}{2.877155in}}%
\pgfpathlineto{\pgfqpoint{2.639899in}{1.959637in}}%
\pgfpathlineto{\pgfqpoint{2.642710in}{1.861524in}}%
\pgfpathlineto{\pgfqpoint{2.645521in}{1.878806in}}%
\pgfpathlineto{\pgfqpoint{2.648331in}{1.970998in}}%
\pgfpathlineto{\pgfqpoint{2.651142in}{1.947691in}}%
\pgfpathlineto{\pgfqpoint{2.653953in}{1.901794in}}%
\pgfpathlineto{\pgfqpoint{2.656763in}{2.026966in}}%
\pgfpathlineto{\pgfqpoint{2.659574in}{2.070580in}}%
\pgfpathlineto{\pgfqpoint{2.662385in}{1.845500in}}%
\pgfpathlineto{\pgfqpoint{2.665195in}{1.963171in}}%
\pgfpathlineto{\pgfqpoint{2.668006in}{1.873580in}}%
\pgfpathlineto{\pgfqpoint{2.670817in}{1.929460in}}%
\pgfpathlineto{\pgfqpoint{2.673628in}{1.890353in}}%
\pgfpathlineto{\pgfqpoint{2.676438in}{1.968260in}}%
\pgfpathlineto{\pgfqpoint{2.679249in}{1.834747in}}%
\pgfpathlineto{\pgfqpoint{2.682060in}{1.778514in}}%
\pgfpathlineto{\pgfqpoint{2.687681in}{2.068362in}}%
\pgfpathlineto{\pgfqpoint{2.690492in}{1.884822in}}%
\pgfpathlineto{\pgfqpoint{2.693302in}{2.006070in}}%
\pgfpathlineto{\pgfqpoint{2.696113in}{1.994282in}}%
\pgfpathlineto{\pgfqpoint{2.698924in}{3.214362in}}%
\pgfpathlineto{\pgfqpoint{2.701734in}{1.475623in}}%
\pgfpathlineto{\pgfqpoint{2.704545in}{1.890353in}}%
\pgfpathlineto{\pgfqpoint{2.710166in}{2.206673in}}%
\pgfpathlineto{\pgfqpoint{2.712977in}{1.765631in}}%
\pgfpathlineto{\pgfqpoint{2.715788in}{1.779728in}}%
\pgfpathlineto{\pgfqpoint{2.718599in}{1.890353in}}%
\pgfpathlineto{\pgfqpoint{2.721409in}{1.885305in}}%
\pgfpathlineto{\pgfqpoint{2.724220in}{1.870144in}}%
\pgfpathlineto{\pgfqpoint{2.727031in}{1.960965in}}%
\pgfpathlineto{\pgfqpoint{2.729841in}{1.687712in}}%
\pgfpathlineto{\pgfqpoint{2.732652in}{1.823898in}}%
\pgfpathlineto{\pgfqpoint{2.735463in}{2.028046in}}%
\pgfpathlineto{\pgfqpoint{2.738273in}{1.442048in}}%
\pgfpathlineto{\pgfqpoint{2.741084in}{1.900812in}}%
\pgfpathlineto{\pgfqpoint{2.743895in}{1.811734in}}%
\pgfpathlineto{\pgfqpoint{2.746705in}{1.858789in}}%
\pgfpathlineto{\pgfqpoint{2.749516in}{2.078742in}}%
\pgfpathlineto{\pgfqpoint{2.752327in}{1.942264in}}%
\pgfpathlineto{\pgfqpoint{2.755137in}{2.319418in}}%
\pgfpathlineto{\pgfqpoint{2.757948in}{1.819694in}}%
\pgfpathlineto{\pgfqpoint{2.760759in}{1.991193in}}%
\pgfpathlineto{\pgfqpoint{2.763570in}{1.995506in}}%
\pgfpathlineto{\pgfqpoint{2.766380in}{1.860385in}}%
\pgfpathlineto{\pgfqpoint{2.769191in}{1.960186in}}%
\pgfpathlineto{\pgfqpoint{2.772002in}{2.166460in}}%
\pgfpathlineto{\pgfqpoint{2.774812in}{1.802160in}}%
\pgfpathlineto{\pgfqpoint{2.777623in}{1.836200in}}%
\pgfpathlineto{\pgfqpoint{2.780434in}{1.885420in}}%
\pgfpathlineto{\pgfqpoint{2.783244in}{1.885419in}}%
\pgfpathlineto{\pgfqpoint{2.786055in}{1.831012in}}%
\pgfpathlineto{\pgfqpoint{2.788866in}{2.003883in}}%
\pgfpathlineto{\pgfqpoint{2.791676in}{1.742099in}}%
\pgfpathlineto{\pgfqpoint{2.797298in}{2.033689in}}%
\pgfpathlineto{\pgfqpoint{2.800108in}{1.929653in}}%
\pgfpathlineto{\pgfqpoint{2.802919in}{2.031953in}}%
\pgfpathlineto{\pgfqpoint{2.805730in}{1.709453in}}%
\pgfpathlineto{\pgfqpoint{2.808540in}{1.939462in}}%
\pgfpathlineto{\pgfqpoint{2.811351in}{1.890353in}}%
\pgfpathlineto{\pgfqpoint{2.814162in}{1.717761in}}%
\pgfpathlineto{\pgfqpoint{2.816973in}{1.855594in}}%
\pgfpathlineto{\pgfqpoint{2.819783in}{2.058431in}}%
\pgfpathlineto{\pgfqpoint{2.822594in}{1.910003in}}%
\pgfpathlineto{\pgfqpoint{2.825405in}{1.973578in}}%
\pgfpathlineto{\pgfqpoint{2.828215in}{1.738240in}}%
\pgfpathlineto{\pgfqpoint{2.831026in}{1.845899in}}%
\pgfpathlineto{\pgfqpoint{2.833837in}{1.766171in}}%
\pgfpathlineto{\pgfqpoint{2.836647in}{2.029368in}}%
\pgfpathlineto{\pgfqpoint{2.839458in}{1.626113in}}%
\pgfpathlineto{\pgfqpoint{2.842269in}{1.945581in}}%
\pgfpathlineto{\pgfqpoint{2.845079in}{2.163470in}}%
\pgfpathlineto{\pgfqpoint{2.850701in}{1.530957in}}%
\pgfpathlineto{\pgfqpoint{2.853511in}{1.681761in}}%
\pgfpathlineto{\pgfqpoint{2.856322in}{2.119141in}}%
\pgfpathlineto{\pgfqpoint{2.859133in}{1.925629in}}%
\pgfpathlineto{\pgfqpoint{2.861944in}{1.605782in}}%
\pgfpathlineto{\pgfqpoint{2.864754in}{2.245227in}}%
\pgfpathlineto{\pgfqpoint{2.867565in}{1.602026in}}%
\pgfpathlineto{\pgfqpoint{2.870376in}{1.114611in}}%
\pgfpathlineto{\pgfqpoint{2.875997in}{2.208886in}}%
\pgfpathlineto{\pgfqpoint{2.878808in}{1.484253in}}%
\pgfpathlineto{\pgfqpoint{2.881618in}{1.680298in}}%
\pgfpathlineto{\pgfqpoint{2.884429in}{2.149728in}}%
\pgfpathlineto{\pgfqpoint{2.887240in}{1.983070in}}%
\pgfpathlineto{\pgfqpoint{2.890050in}{2.366566in}}%
\pgfpathlineto{\pgfqpoint{2.892861in}{1.736846in}}%
\pgfpathlineto{\pgfqpoint{2.895672in}{2.080678in}}%
\pgfpathlineto{\pgfqpoint{2.898482in}{2.114549in}}%
\pgfpathlineto{\pgfqpoint{2.901293in}{1.900699in}}%
\pgfpathlineto{\pgfqpoint{2.904104in}{2.131446in}}%
\pgfpathlineto{\pgfqpoint{2.906915in}{1.966485in}}%
\pgfpathlineto{\pgfqpoint{2.909725in}{1.291232in}}%
\pgfpathlineto{\pgfqpoint{2.912536in}{2.529919in}}%
\pgfpathlineto{\pgfqpoint{2.915347in}{2.021048in}}%
\pgfpathlineto{\pgfqpoint{2.918157in}{1.990114in}}%
\pgfpathlineto{\pgfqpoint{2.920968in}{1.649796in}}%
\pgfpathlineto{\pgfqpoint{2.923779in}{1.915593in}}%
\pgfpathlineto{\pgfqpoint{2.926589in}{1.784058in}}%
\pgfpathlineto{\pgfqpoint{2.929400in}{1.747438in}}%
\pgfpathlineto{\pgfqpoint{2.932211in}{1.910853in}}%
\pgfpathlineto{\pgfqpoint{2.935021in}{1.921051in}}%
\pgfpathlineto{\pgfqpoint{2.937832in}{2.027717in}}%
\pgfpathlineto{\pgfqpoint{2.940643in}{2.006378in}}%
\pgfpathlineto{\pgfqpoint{2.943453in}{2.020434in}}%
\pgfpathlineto{\pgfqpoint{2.946264in}{2.093204in}}%
\pgfpathlineto{\pgfqpoint{2.949075in}{1.732267in}}%
\pgfpathlineto{\pgfqpoint{2.951886in}{2.569939in}}%
\pgfpathlineto{\pgfqpoint{2.954696in}{1.742519in}}%
\pgfpathlineto{\pgfqpoint{2.957507in}{2.161057in}}%
\pgfpathlineto{\pgfqpoint{2.960318in}{1.918565in}}%
\pgfpathlineto{\pgfqpoint{2.963128in}{2.128021in}}%
\pgfpathlineto{\pgfqpoint{2.965939in}{2.032989in}}%
\pgfpathlineto{\pgfqpoint{2.968750in}{1.858264in}}%
\pgfpathlineto{\pgfqpoint{2.971560in}{2.059191in}}%
\pgfpathlineto{\pgfqpoint{2.974371in}{1.826693in}}%
\pgfpathlineto{\pgfqpoint{2.979992in}{1.976631in}}%
\pgfpathlineto{\pgfqpoint{2.982803in}{1.698962in}}%
\pgfpathlineto{\pgfqpoint{2.985614in}{1.766007in}}%
\pgfpathlineto{\pgfqpoint{2.988424in}{1.699526in}}%
\pgfpathlineto{\pgfqpoint{2.991235in}{2.007001in}}%
\pgfpathlineto{\pgfqpoint{2.994046in}{1.698569in}}%
\pgfpathlineto{\pgfqpoint{2.996856in}{1.757947in}}%
\pgfpathlineto{\pgfqpoint{2.999667in}{1.742377in}}%
\pgfpathlineto{\pgfqpoint{3.002478in}{2.170735in}}%
\pgfpathlineto{\pgfqpoint{3.005289in}{2.211709in}}%
\pgfpathlineto{\pgfqpoint{3.008099in}{1.625385in}}%
\pgfpathlineto{\pgfqpoint{3.010910in}{2.228861in}}%
\pgfpathlineto{\pgfqpoint{3.013721in}{1.981770in}}%
\pgfpathlineto{\pgfqpoint{3.016531in}{1.890353in}}%
\pgfpathlineto{\pgfqpoint{3.019342in}{1.935852in}}%
\pgfpathlineto{\pgfqpoint{3.022153in}{1.744264in}}%
\pgfpathlineto{\pgfqpoint{3.027774in}{1.698869in}}%
\pgfpathlineto{\pgfqpoint{3.030585in}{1.918529in}}%
\pgfpathlineto{\pgfqpoint{3.033395in}{1.720493in}}%
\pgfpathlineto{\pgfqpoint{3.036206in}{1.607721in}}%
\pgfpathlineto{\pgfqpoint{3.039017in}{2.201430in}}%
\pgfpathlineto{\pgfqpoint{3.041827in}{2.167207in}}%
\pgfpathlineto{\pgfqpoint{3.044638in}{1.918222in}}%
\pgfpathlineto{\pgfqpoint{3.047449in}{1.825244in}}%
\pgfpathlineto{\pgfqpoint{3.050260in}{1.848347in}}%
\pgfpathlineto{\pgfqpoint{3.053070in}{1.829468in}}%
\pgfpathlineto{\pgfqpoint{3.055881in}{1.824505in}}%
\pgfpathlineto{\pgfqpoint{3.058692in}{1.998379in}}%
\pgfpathlineto{\pgfqpoint{3.061502in}{1.739843in}}%
\pgfpathlineto{\pgfqpoint{3.064313in}{2.050220in}}%
\pgfpathlineto{\pgfqpoint{3.067124in}{2.067024in}}%
\pgfpathlineto{\pgfqpoint{3.069934in}{1.704326in}}%
\pgfpathlineto{\pgfqpoint{3.072745in}{1.625955in}}%
\pgfpathlineto{\pgfqpoint{3.075556in}{1.198126in}}%
\pgfpathlineto{\pgfqpoint{3.078366in}{1.709634in}}%
\pgfpathlineto{\pgfqpoint{3.081177in}{2.081048in}}%
\pgfpathlineto{\pgfqpoint{3.083988in}{1.372872in}}%
\pgfpathlineto{\pgfqpoint{3.086798in}{2.161479in}}%
\pgfpathlineto{\pgfqpoint{3.089609in}{1.910617in}}%
\pgfpathlineto{\pgfqpoint{3.092420in}{1.956021in}}%
\pgfpathlineto{\pgfqpoint{3.095231in}{2.040800in}}%
\pgfpathlineto{\pgfqpoint{3.098041in}{1.603194in}}%
\pgfpathlineto{\pgfqpoint{3.100852in}{1.726685in}}%
\pgfpathlineto{\pgfqpoint{3.103663in}{2.231000in}}%
\pgfpathlineto{\pgfqpoint{3.106473in}{1.824863in}}%
\pgfpathlineto{\pgfqpoint{3.109284in}{2.145645in}}%
\pgfpathlineto{\pgfqpoint{3.112095in}{1.994217in}}%
\pgfpathlineto{\pgfqpoint{3.114905in}{2.052120in}}%
\pgfpathlineto{\pgfqpoint{3.117716in}{1.689104in}}%
\pgfpathlineto{\pgfqpoint{3.120527in}{1.860673in}}%
\pgfpathlineto{\pgfqpoint{3.123337in}{1.979217in}}%
\pgfpathlineto{\pgfqpoint{3.126148in}{1.607116in}}%
\pgfpathlineto{\pgfqpoint{3.128959in}{2.173590in}}%
\pgfpathlineto{\pgfqpoint{3.131769in}{1.687165in}}%
\pgfpathlineto{\pgfqpoint{3.134580in}{1.760054in}}%
\pgfpathlineto{\pgfqpoint{3.137391in}{1.713133in}}%
\pgfpathlineto{\pgfqpoint{3.140202in}{2.247686in}}%
\pgfpathlineto{\pgfqpoint{3.143012in}{1.910232in}}%
\pgfpathlineto{\pgfqpoint{3.145823in}{1.895319in}}%
\pgfpathlineto{\pgfqpoint{3.148634in}{1.720588in}}%
\pgfpathlineto{\pgfqpoint{3.151444in}{1.652451in}}%
\pgfpathlineto{\pgfqpoint{3.154255in}{1.674501in}}%
\pgfpathlineto{\pgfqpoint{3.157066in}{1.417022in}}%
\pgfpathlineto{\pgfqpoint{3.159876in}{2.186630in}}%
\pgfpathlineto{\pgfqpoint{3.162687in}{1.165848in}}%
\pgfpathlineto{\pgfqpoint{3.165498in}{1.950721in}}%
\pgfpathlineto{\pgfqpoint{3.168308in}{1.846473in}}%
\pgfpathlineto{\pgfqpoint{3.171119in}{1.774550in}}%
\pgfpathlineto{\pgfqpoint{3.173930in}{2.033596in}}%
\pgfpathlineto{\pgfqpoint{3.176740in}{1.818903in}}%
\pgfpathlineto{\pgfqpoint{3.179551in}{2.163306in}}%
\pgfpathlineto{\pgfqpoint{3.182362in}{1.836160in}}%
\pgfpathlineto{\pgfqpoint{3.185173in}{1.693614in}}%
\pgfpathlineto{\pgfqpoint{3.187983in}{1.449304in}}%
\pgfpathlineto{\pgfqpoint{3.190794in}{1.986369in}}%
\pgfpathlineto{\pgfqpoint{3.193605in}{2.812696in}}%
\pgfpathlineto{\pgfqpoint{3.196415in}{1.634132in}}%
\pgfpathlineto{\pgfqpoint{3.199226in}{1.798542in}}%
\pgfpathlineto{\pgfqpoint{3.202037in}{1.672073in}}%
\pgfpathlineto{\pgfqpoint{3.204847in}{1.890353in}}%
\pgfpathlineto{\pgfqpoint{3.207658in}{1.999892in}}%
\pgfpathlineto{\pgfqpoint{3.210469in}{1.999094in}}%
\pgfpathlineto{\pgfqpoint{3.213279in}{1.912006in}}%
\pgfpathlineto{\pgfqpoint{3.216090in}{1.857861in}}%
\pgfpathlineto{\pgfqpoint{3.218901in}{2.217436in}}%
\pgfpathlineto{\pgfqpoint{3.221711in}{1.794580in}}%
\pgfpathlineto{\pgfqpoint{3.224522in}{1.777836in}}%
\pgfpathlineto{\pgfqpoint{3.227333in}{2.511857in}}%
\pgfpathlineto{\pgfqpoint{3.230143in}{1.905821in}}%
\pgfpathlineto{\pgfqpoint{3.232954in}{1.708884in}}%
\pgfpathlineto{\pgfqpoint{3.235765in}{2.009853in}}%
\pgfpathlineto{\pgfqpoint{3.238576in}{1.755196in}}%
\pgfpathlineto{\pgfqpoint{3.241386in}{2.009978in}}%
\pgfpathlineto{\pgfqpoint{3.244197in}{1.734130in}}%
\pgfpathlineto{\pgfqpoint{3.247008in}{1.764186in}}%
\pgfpathlineto{\pgfqpoint{3.249818in}{2.084418in}}%
\pgfpathlineto{\pgfqpoint{3.252629in}{2.128155in}}%
\pgfpathlineto{\pgfqpoint{3.255440in}{1.828682in}}%
\pgfpathlineto{\pgfqpoint{3.258250in}{1.735051in}}%
\pgfpathlineto{\pgfqpoint{3.261061in}{2.285757in}}%
\pgfpathlineto{\pgfqpoint{3.263872in}{1.895420in}}%
\pgfpathlineto{\pgfqpoint{3.266682in}{1.640034in}}%
\pgfpathlineto{\pgfqpoint{3.269493in}{1.698514in}}%
\pgfpathlineto{\pgfqpoint{3.272304in}{1.895570in}}%
\pgfpathlineto{\pgfqpoint{3.275114in}{2.148924in}}%
\pgfpathlineto{\pgfqpoint{3.277925in}{1.838996in}}%
\pgfpathlineto{\pgfqpoint{3.280736in}{1.688354in}}%
\pgfpathlineto{\pgfqpoint{3.283547in}{2.071759in}}%
\pgfpathlineto{\pgfqpoint{3.286357in}{2.039008in}}%
\pgfpathlineto{\pgfqpoint{3.289168in}{1.900551in}}%
\pgfpathlineto{\pgfqpoint{3.291979in}{2.077763in}}%
\pgfpathlineto{\pgfqpoint{3.294789in}{1.774132in}}%
\pgfpathlineto{\pgfqpoint{3.297600in}{1.991466in}}%
\pgfpathlineto{\pgfqpoint{3.300411in}{1.970753in}}%
\pgfpathlineto{\pgfqpoint{3.306032in}{1.733706in}}%
\pgfpathlineto{\pgfqpoint{3.308843in}{2.162168in}}%
\pgfpathlineto{\pgfqpoint{3.311653in}{2.019481in}}%
\pgfpathlineto{\pgfqpoint{3.314464in}{2.086855in}}%
\pgfpathlineto{\pgfqpoint{3.317275in}{1.654238in}}%
\pgfpathlineto{\pgfqpoint{3.320085in}{1.605025in}}%
\pgfpathlineto{\pgfqpoint{3.322896in}{1.645771in}}%
\pgfpathlineto{\pgfqpoint{3.325707in}{1.699046in}}%
\pgfpathlineto{\pgfqpoint{3.328518in}{1.670178in}}%
\pgfpathlineto{\pgfqpoint{3.331328in}{1.634680in}}%
\pgfpathlineto{\pgfqpoint{3.334139in}{1.868848in}}%
\pgfpathlineto{\pgfqpoint{3.336950in}{2.156965in}}%
\pgfpathlineto{\pgfqpoint{3.342571in}{1.640141in}}%
\pgfpathlineto{\pgfqpoint{3.345382in}{1.922530in}}%
\pgfpathlineto{\pgfqpoint{3.348192in}{2.007744in}}%
\pgfpathlineto{\pgfqpoint{3.351003in}{2.038436in}}%
\pgfpathlineto{\pgfqpoint{3.353814in}{2.089000in}}%
\pgfpathlineto{\pgfqpoint{3.356624in}{1.744239in}}%
\pgfpathlineto{\pgfqpoint{3.359435in}{1.979236in}}%
\pgfpathlineto{\pgfqpoint{3.362246in}{1.811955in}}%
\pgfpathlineto{\pgfqpoint{3.365056in}{1.806271in}}%
\pgfpathlineto{\pgfqpoint{3.367867in}{1.927197in}}%
\pgfpathlineto{\pgfqpoint{3.370678in}{1.431265in}}%
\pgfpathlineto{\pgfqpoint{3.373489in}{1.583633in}}%
\pgfpathlineto{\pgfqpoint{3.376299in}{1.906945in}}%
\pgfpathlineto{\pgfqpoint{3.379110in}{1.773818in}}%
\pgfpathlineto{\pgfqpoint{3.381921in}{2.073073in}}%
\pgfpathlineto{\pgfqpoint{3.384731in}{1.635033in}}%
\pgfpathlineto{\pgfqpoint{3.387542in}{1.823023in}}%
\pgfpathlineto{\pgfqpoint{3.390353in}{1.686520in}}%
\pgfpathlineto{\pgfqpoint{3.393163in}{1.602551in}}%
\pgfpathlineto{\pgfqpoint{3.395974in}{2.011907in}}%
\pgfpathlineto{\pgfqpoint{3.398785in}{2.175842in}}%
\pgfpathlineto{\pgfqpoint{3.401595in}{1.850712in}}%
\pgfpathlineto{\pgfqpoint{3.404406in}{1.907355in}}%
\pgfpathlineto{\pgfqpoint{3.407217in}{1.997584in}}%
\pgfpathlineto{\pgfqpoint{3.410027in}{1.675114in}}%
\pgfpathlineto{\pgfqpoint{3.412838in}{1.700882in}}%
\pgfpathlineto{\pgfqpoint{3.415649in}{1.692591in}}%
\pgfpathlineto{\pgfqpoint{3.418459in}{1.831685in}}%
\pgfpathlineto{\pgfqpoint{3.421270in}{1.884474in}}%
\pgfpathlineto{\pgfqpoint{3.424081in}{1.611370in}}%
\pgfpathlineto{\pgfqpoint{3.426892in}{2.039397in}}%
\pgfpathlineto{\pgfqpoint{3.429702in}{2.207295in}}%
\pgfpathlineto{\pgfqpoint{3.432513in}{1.919362in}}%
\pgfpathlineto{\pgfqpoint{3.435324in}{1.838097in}}%
\pgfpathlineto{\pgfqpoint{3.438134in}{1.907792in}}%
\pgfpathlineto{\pgfqpoint{3.440945in}{1.942549in}}%
\pgfpathlineto{\pgfqpoint{3.443756in}{1.930824in}}%
\pgfpathlineto{\pgfqpoint{3.446566in}{2.011110in}}%
\pgfpathlineto{\pgfqpoint{3.449377in}{1.832971in}}%
\pgfpathlineto{\pgfqpoint{3.452188in}{1.826977in}}%
\pgfpathlineto{\pgfqpoint{3.454998in}{2.277886in}}%
\pgfpathlineto{\pgfqpoint{3.457809in}{1.548938in}}%
\pgfpathlineto{\pgfqpoint{3.460620in}{2.135825in}}%
\pgfpathlineto{\pgfqpoint{3.463430in}{1.592989in}}%
\pgfpathlineto{\pgfqpoint{3.466241in}{1.965252in}}%
\pgfpathlineto{\pgfqpoint{3.469052in}{1.919060in}}%
\pgfpathlineto{\pgfqpoint{3.471863in}{1.804065in}}%
\pgfpathlineto{\pgfqpoint{3.474673in}{1.604949in}}%
\pgfpathlineto{\pgfqpoint{3.477484in}{1.448713in}}%
\pgfpathlineto{\pgfqpoint{3.480295in}{1.353833in}}%
\pgfpathlineto{\pgfqpoint{3.483105in}{1.713493in}}%
\pgfpathlineto{\pgfqpoint{3.485916in}{1.672738in}}%
\pgfpathlineto{\pgfqpoint{3.488727in}{2.693599in}}%
\pgfpathlineto{\pgfqpoint{3.491537in}{2.120723in}}%
\pgfpathlineto{\pgfqpoint{3.494348in}{2.258747in}}%
\pgfpathlineto{\pgfqpoint{3.497159in}{1.954774in}}%
\pgfpathlineto{\pgfqpoint{3.499969in}{1.511583in}}%
\pgfpathlineto{\pgfqpoint{3.502780in}{2.304145in}}%
\pgfpathlineto{\pgfqpoint{3.505591in}{2.133229in}}%
\pgfpathlineto{\pgfqpoint{3.508401in}{1.600763in}}%
\pgfpathlineto{\pgfqpoint{3.511212in}{2.396332in}}%
\pgfpathlineto{\pgfqpoint{3.514023in}{1.759756in}}%
\pgfpathlineto{\pgfqpoint{3.516834in}{1.901755in}}%
\pgfpathlineto{\pgfqpoint{3.519644in}{1.992579in}}%
\pgfpathlineto{\pgfqpoint{3.522455in}{1.850682in}}%
\pgfpathlineto{\pgfqpoint{3.525266in}{2.065244in}}%
\pgfpathlineto{\pgfqpoint{3.528076in}{1.907170in}}%
\pgfpathlineto{\pgfqpoint{3.530887in}{1.862315in}}%
\pgfpathlineto{\pgfqpoint{3.533698in}{1.538228in}}%
\pgfpathlineto{\pgfqpoint{3.536508in}{1.964878in}}%
\pgfpathlineto{\pgfqpoint{3.539319in}{1.636599in}}%
\pgfpathlineto{\pgfqpoint{3.542130in}{1.925211in}}%
\pgfpathlineto{\pgfqpoint{3.544940in}{1.756286in}}%
\pgfpathlineto{\pgfqpoint{3.547751in}{2.059196in}}%
\pgfpathlineto{\pgfqpoint{3.550562in}{1.867178in}}%
\pgfpathlineto{\pgfqpoint{3.553372in}{2.137622in}}%
\pgfpathlineto{\pgfqpoint{3.556183in}{2.339760in}}%
\pgfpathlineto{\pgfqpoint{3.558994in}{1.823799in}}%
\pgfpathlineto{\pgfqpoint{3.561805in}{2.138415in}}%
\pgfpathlineto{\pgfqpoint{3.564615in}{2.230853in}}%
\pgfpathlineto{\pgfqpoint{3.567426in}{2.139410in}}%
\pgfpathlineto{\pgfqpoint{3.570237in}{2.161145in}}%
\pgfpathlineto{\pgfqpoint{3.573047in}{1.982955in}}%
\pgfpathlineto{\pgfqpoint{3.575858in}{1.714952in}}%
\pgfpathlineto{\pgfqpoint{3.578669in}{1.921456in}}%
\pgfpathlineto{\pgfqpoint{3.581479in}{1.812473in}}%
\pgfpathlineto{\pgfqpoint{3.584290in}{2.240198in}}%
\pgfpathlineto{\pgfqpoint{3.587101in}{1.869999in}}%
\pgfpathlineto{\pgfqpoint{3.589911in}{2.022158in}}%
\pgfpathlineto{\pgfqpoint{3.592722in}{2.135633in}}%
\pgfpathlineto{\pgfqpoint{3.595533in}{1.820684in}}%
\pgfpathlineto{\pgfqpoint{3.598343in}{1.860395in}}%
\pgfpathlineto{\pgfqpoint{3.601154in}{2.343230in}}%
\pgfpathlineto{\pgfqpoint{3.603965in}{2.106961in}}%
\pgfpathlineto{\pgfqpoint{3.606776in}{1.755945in}}%
\pgfpathlineto{\pgfqpoint{3.609586in}{1.842055in}}%
\pgfpathlineto{\pgfqpoint{3.612397in}{1.991599in}}%
\pgfpathlineto{\pgfqpoint{3.615208in}{1.594337in}}%
\pgfpathlineto{\pgfqpoint{3.620829in}{1.998321in}}%
\pgfpathlineto{\pgfqpoint{3.623640in}{1.978113in}}%
\pgfpathlineto{\pgfqpoint{3.626450in}{1.929192in}}%
\pgfpathlineto{\pgfqpoint{3.629261in}{1.822318in}}%
\pgfpathlineto{\pgfqpoint{3.632072in}{1.822006in}}%
\pgfpathlineto{\pgfqpoint{3.634882in}{1.668521in}}%
\pgfpathlineto{\pgfqpoint{3.637693in}{1.830636in}}%
\pgfpathlineto{\pgfqpoint{3.640504in}{1.729930in}}%
\pgfpathlineto{\pgfqpoint{3.643314in}{1.682257in}}%
\pgfpathlineto{\pgfqpoint{3.646125in}{1.756870in}}%
\pgfpathlineto{\pgfqpoint{3.648936in}{1.885195in}}%
\pgfpathlineto{\pgfqpoint{3.651746in}{2.135942in}}%
\pgfpathlineto{\pgfqpoint{3.654557in}{2.126971in}}%
\pgfpathlineto{\pgfqpoint{3.657368in}{2.396137in}}%
\pgfpathlineto{\pgfqpoint{3.660179in}{2.044062in}}%
\pgfpathlineto{\pgfqpoint{3.662989in}{1.813697in}}%
\pgfpathlineto{\pgfqpoint{3.665800in}{1.837422in}}%
\pgfpathlineto{\pgfqpoint{3.668611in}{1.933674in}}%
\pgfpathlineto{\pgfqpoint{3.671421in}{1.890353in}}%
\pgfpathlineto{\pgfqpoint{3.674232in}{2.024330in}}%
\pgfpathlineto{\pgfqpoint{3.677043in}{2.027858in}}%
\pgfpathlineto{\pgfqpoint{3.682664in}{1.548916in}}%
\pgfpathlineto{\pgfqpoint{3.685475in}{2.279258in}}%
\pgfpathlineto{\pgfqpoint{3.688285in}{1.909298in}}%
\pgfpathlineto{\pgfqpoint{3.691096in}{1.790622in}}%
\pgfpathlineto{\pgfqpoint{3.693907in}{1.914159in}}%
\pgfpathlineto{\pgfqpoint{3.696717in}{1.871311in}}%
\pgfpathlineto{\pgfqpoint{3.699528in}{1.674442in}}%
\pgfpathlineto{\pgfqpoint{3.702339in}{1.977093in}}%
\pgfpathlineto{\pgfqpoint{3.705150in}{2.194740in}}%
\pgfpathlineto{\pgfqpoint{3.713582in}{1.437638in}}%
\pgfpathlineto{\pgfqpoint{3.716392in}{2.055691in}}%
\pgfpathlineto{\pgfqpoint{3.719203in}{2.101636in}}%
\pgfpathlineto{\pgfqpoint{3.722014in}{2.004349in}}%
\pgfpathlineto{\pgfqpoint{3.724824in}{1.880887in}}%
\pgfpathlineto{\pgfqpoint{3.727635in}{1.871402in}}%
\pgfpathlineto{\pgfqpoint{3.730446in}{2.106850in}}%
\pgfpathlineto{\pgfqpoint{3.733256in}{1.697541in}}%
\pgfpathlineto{\pgfqpoint{3.736067in}{1.661409in}}%
\pgfpathlineto{\pgfqpoint{3.738878in}{1.687104in}}%
\pgfpathlineto{\pgfqpoint{3.741688in}{1.821983in}}%
\pgfpathlineto{\pgfqpoint{3.747310in}{1.318823in}}%
\pgfpathlineto{\pgfqpoint{3.752931in}{2.145655in}}%
\pgfpathlineto{\pgfqpoint{3.755742in}{2.176876in}}%
\pgfpathlineto{\pgfqpoint{3.758553in}{1.536522in}}%
\pgfpathlineto{\pgfqpoint{3.761363in}{2.269501in}}%
\pgfpathlineto{\pgfqpoint{3.764174in}{0.468908in}}%
\pgfpathlineto{\pgfqpoint{3.766985in}{1.912596in}}%
\pgfpathlineto{\pgfqpoint{3.769795in}{1.784404in}}%
\pgfpathlineto{\pgfqpoint{3.772606in}{1.923892in}}%
\pgfpathlineto{\pgfqpoint{3.775417in}{2.023760in}}%
\pgfpathlineto{\pgfqpoint{3.778227in}{1.729003in}}%
\pgfpathlineto{\pgfqpoint{3.781038in}{2.057236in}}%
\pgfpathlineto{\pgfqpoint{3.783849in}{1.829379in}}%
\pgfpathlineto{\pgfqpoint{3.786659in}{1.967913in}}%
\pgfpathlineto{\pgfqpoint{3.789470in}{2.406264in}}%
\pgfpathlineto{\pgfqpoint{3.792281in}{1.793962in}}%
\pgfpathlineto{\pgfqpoint{3.795092in}{1.387718in}}%
\pgfpathlineto{\pgfqpoint{3.800713in}{2.112534in}}%
\pgfpathlineto{\pgfqpoint{3.803524in}{1.516307in}}%
\pgfpathlineto{\pgfqpoint{3.806334in}{1.776855in}}%
\pgfpathlineto{\pgfqpoint{3.809145in}{1.884655in}}%
\pgfpathlineto{\pgfqpoint{3.811956in}{1.591038in}}%
\pgfpathlineto{\pgfqpoint{3.814766in}{1.884538in}}%
\pgfpathlineto{\pgfqpoint{3.817577in}{2.109757in}}%
\pgfpathlineto{\pgfqpoint{3.820388in}{1.964678in}}%
\pgfpathlineto{\pgfqpoint{3.823198in}{2.239818in}}%
\pgfpathlineto{\pgfqpoint{3.828820in}{1.523214in}}%
\pgfpathlineto{\pgfqpoint{3.831630in}{2.218385in}}%
\pgfpathlineto{\pgfqpoint{3.834441in}{1.607998in}}%
\pgfpathlineto{\pgfqpoint{3.837252in}{2.094186in}}%
\pgfpathlineto{\pgfqpoint{3.840062in}{2.108074in}}%
\pgfpathlineto{\pgfqpoint{3.842873in}{1.978763in}}%
\pgfpathlineto{\pgfqpoint{3.845684in}{1.785307in}}%
\pgfpathlineto{\pgfqpoint{3.848495in}{2.279167in}}%
\pgfpathlineto{\pgfqpoint{3.851305in}{1.976591in}}%
\pgfpathlineto{\pgfqpoint{3.854116in}{1.906467in}}%
\pgfpathlineto{\pgfqpoint{3.856927in}{1.917172in}}%
\pgfpathlineto{\pgfqpoint{3.859737in}{2.039653in}}%
\pgfpathlineto{\pgfqpoint{3.862548in}{1.703494in}}%
\pgfpathlineto{\pgfqpoint{3.865359in}{2.045345in}}%
\pgfpathlineto{\pgfqpoint{3.868169in}{2.070041in}}%
\pgfpathlineto{\pgfqpoint{3.870980in}{2.130066in}}%
\pgfpathlineto{\pgfqpoint{3.873791in}{1.734456in}}%
\pgfpathlineto{\pgfqpoint{3.876601in}{1.994465in}}%
\pgfpathlineto{\pgfqpoint{3.879412in}{1.911089in}}%
\pgfpathlineto{\pgfqpoint{3.882223in}{2.024436in}}%
\pgfpathlineto{\pgfqpoint{3.885033in}{2.215359in}}%
\pgfpathlineto{\pgfqpoint{3.887844in}{1.733812in}}%
\pgfpathlineto{\pgfqpoint{3.890655in}{1.880197in}}%
\pgfpathlineto{\pgfqpoint{3.893466in}{1.742308in}}%
\pgfpathlineto{\pgfqpoint{3.896276in}{1.833812in}}%
\pgfpathlineto{\pgfqpoint{3.899087in}{1.900649in}}%
\pgfpathlineto{\pgfqpoint{3.901898in}{2.115090in}}%
\pgfpathlineto{\pgfqpoint{3.904708in}{2.041665in}}%
\pgfpathlineto{\pgfqpoint{3.907519in}{1.723826in}}%
\pgfpathlineto{\pgfqpoint{3.910330in}{1.935954in}}%
\pgfpathlineto{\pgfqpoint{3.913140in}{1.681476in}}%
\pgfpathlineto{\pgfqpoint{3.915951in}{1.844108in}}%
\pgfpathlineto{\pgfqpoint{3.918762in}{1.972468in}}%
\pgfpathlineto{\pgfqpoint{3.921572in}{1.642640in}}%
\pgfpathlineto{\pgfqpoint{3.924383in}{1.931925in}}%
\pgfpathlineto{\pgfqpoint{3.927194in}{1.905913in}}%
\pgfpathlineto{\pgfqpoint{3.930004in}{1.978220in}}%
\pgfpathlineto{\pgfqpoint{3.932815in}{2.018636in}}%
\pgfpathlineto{\pgfqpoint{3.935626in}{1.736281in}}%
\pgfpathlineto{\pgfqpoint{3.938437in}{1.729456in}}%
\pgfpathlineto{\pgfqpoint{3.941247in}{1.978802in}}%
\pgfpathlineto{\pgfqpoint{3.944058in}{1.869589in}}%
\pgfpathlineto{\pgfqpoint{3.946869in}{2.076200in}}%
\pgfpathlineto{\pgfqpoint{3.949679in}{1.874954in}}%
\pgfpathlineto{\pgfqpoint{3.952490in}{1.735481in}}%
\pgfpathlineto{\pgfqpoint{3.955301in}{1.775750in}}%
\pgfpathlineto{\pgfqpoint{3.958111in}{1.895581in}}%
\pgfpathlineto{\pgfqpoint{3.960922in}{2.051531in}}%
\pgfpathlineto{\pgfqpoint{3.963733in}{1.587362in}}%
\pgfpathlineto{\pgfqpoint{3.966543in}{1.489203in}}%
\pgfpathlineto{\pgfqpoint{3.969354in}{2.052110in}}%
\pgfpathlineto{\pgfqpoint{3.972165in}{1.766497in}}%
\pgfpathlineto{\pgfqpoint{3.974975in}{1.765460in}}%
\pgfpathlineto{\pgfqpoint{3.977786in}{1.917593in}}%
\pgfpathlineto{\pgfqpoint{3.980597in}{1.960944in}}%
\pgfpathlineto{\pgfqpoint{3.983408in}{1.770694in}}%
\pgfpathlineto{\pgfqpoint{3.986218in}{2.058688in}}%
\pgfpathlineto{\pgfqpoint{3.989029in}{1.847093in}}%
\pgfpathlineto{\pgfqpoint{3.991840in}{1.743418in}}%
\pgfpathlineto{\pgfqpoint{3.994650in}{1.966722in}}%
\pgfpathlineto{\pgfqpoint{3.997461in}{2.127856in}}%
\pgfpathlineto{\pgfqpoint{4.000272in}{1.685460in}}%
\pgfpathlineto{\pgfqpoint{4.003082in}{1.895781in}}%
\pgfpathlineto{\pgfqpoint{4.005893in}{1.710174in}}%
\pgfpathlineto{\pgfqpoint{4.008704in}{2.151718in}}%
\pgfpathlineto{\pgfqpoint{4.011514in}{1.928089in}}%
\pgfpathlineto{\pgfqpoint{4.014325in}{2.294030in}}%
\pgfpathlineto{\pgfqpoint{4.017136in}{2.046740in}}%
\pgfpathlineto{\pgfqpoint{4.019946in}{1.942120in}}%
\pgfpathlineto{\pgfqpoint{4.022757in}{1.926483in}}%
\pgfpathlineto{\pgfqpoint{4.025568in}{1.900660in}}%
\pgfpathlineto{\pgfqpoint{4.028378in}{1.921231in}}%
\pgfpathlineto{\pgfqpoint{4.031189in}{1.936550in}}%
\pgfpathlineto{\pgfqpoint{4.034000in}{1.823577in}}%
\pgfpathlineto{\pgfqpoint{4.036811in}{1.921210in}}%
\pgfpathlineto{\pgfqpoint{4.039621in}{1.982542in}}%
\pgfpathlineto{\pgfqpoint{4.042432in}{1.895458in}}%
\pgfpathlineto{\pgfqpoint{4.048053in}{1.936141in}}%
\pgfpathlineto{\pgfqpoint{4.050864in}{1.951186in}}%
\pgfpathlineto{\pgfqpoint{4.053675in}{1.875168in}}%
\pgfpathlineto{\pgfqpoint{4.056485in}{1.645268in}}%
\pgfpathlineto{\pgfqpoint{4.059296in}{1.926346in}}%
\pgfpathlineto{\pgfqpoint{4.062107in}{1.921135in}}%
\pgfpathlineto{\pgfqpoint{4.064917in}{2.083847in}}%
\pgfpathlineto{\pgfqpoint{4.067728in}{1.961012in}}%
\pgfpathlineto{\pgfqpoint{4.070539in}{1.875240in}}%
\pgfpathlineto{\pgfqpoint{4.073349in}{2.209492in}}%
\pgfpathlineto{\pgfqpoint{4.076160in}{1.224477in}}%
\pgfpathlineto{\pgfqpoint{4.078971in}{1.493042in}}%
\pgfpathlineto{\pgfqpoint{4.081782in}{2.116446in}}%
\pgfpathlineto{\pgfqpoint{4.084592in}{2.241045in}}%
\pgfpathlineto{\pgfqpoint{4.087403in}{2.292648in}}%
\pgfpathlineto{\pgfqpoint{4.090214in}{1.865525in}}%
\pgfpathlineto{\pgfqpoint{4.093024in}{1.860505in}}%
\pgfpathlineto{\pgfqpoint{4.095835in}{2.019267in}}%
\pgfpathlineto{\pgfqpoint{4.098646in}{1.993669in}}%
\pgfpathlineto{\pgfqpoint{4.101456in}{2.248818in}}%
\pgfpathlineto{\pgfqpoint{4.104267in}{2.052209in}}%
\pgfpathlineto{\pgfqpoint{4.107078in}{2.134551in}}%
\pgfpathlineto{\pgfqpoint{4.109888in}{1.918274in}}%
\pgfpathlineto{\pgfqpoint{4.112699in}{1.973804in}}%
\pgfpathlineto{\pgfqpoint{4.115510in}{1.834771in}}%
\pgfpathlineto{\pgfqpoint{4.121131in}{1.932085in}}%
\pgfpathlineto{\pgfqpoint{4.123942in}{2.120096in}}%
\pgfpathlineto{\pgfqpoint{4.126753in}{1.280752in}}%
\pgfpathlineto{\pgfqpoint{4.129563in}{2.060374in}}%
\pgfpathlineto{\pgfqpoint{4.132374in}{1.904434in}}%
\pgfpathlineto{\pgfqpoint{4.135185in}{2.062939in}}%
\pgfpathlineto{\pgfqpoint{4.137995in}{1.778633in}}%
\pgfpathlineto{\pgfqpoint{4.140806in}{1.862292in}}%
\pgfpathlineto{\pgfqpoint{4.143617in}{1.927756in}}%
\pgfpathlineto{\pgfqpoint{4.146427in}{1.890353in}}%
\pgfpathlineto{\pgfqpoint{4.149238in}{1.763740in}}%
\pgfpathlineto{\pgfqpoint{4.152049in}{1.866788in}}%
\pgfpathlineto{\pgfqpoint{4.154859in}{2.035862in}}%
\pgfpathlineto{\pgfqpoint{4.157670in}{2.062187in}}%
\pgfpathlineto{\pgfqpoint{4.160481in}{1.913423in}}%
\pgfpathlineto{\pgfqpoint{4.163291in}{1.839552in}}%
\pgfpathlineto{\pgfqpoint{4.166102in}{1.722877in}}%
\pgfpathlineto{\pgfqpoint{4.168913in}{1.955706in}}%
\pgfpathlineto{\pgfqpoint{4.171724in}{1.843701in}}%
\pgfpathlineto{\pgfqpoint{4.174534in}{2.034501in}}%
\pgfpathlineto{\pgfqpoint{4.177345in}{2.019365in}}%
\pgfpathlineto{\pgfqpoint{4.180156in}{1.807545in}}%
\pgfpathlineto{\pgfqpoint{4.182966in}{1.871889in}}%
\pgfpathlineto{\pgfqpoint{4.185777in}{2.005382in}}%
\pgfpathlineto{\pgfqpoint{4.188588in}{1.940686in}}%
\pgfpathlineto{\pgfqpoint{4.191398in}{1.908614in}}%
\pgfpathlineto{\pgfqpoint{4.194209in}{1.785046in}}%
\pgfpathlineto{\pgfqpoint{4.197020in}{1.862759in}}%
\pgfpathlineto{\pgfqpoint{4.199830in}{1.963824in}}%
\pgfpathlineto{\pgfqpoint{4.202641in}{2.008977in}}%
\pgfpathlineto{\pgfqpoint{4.205452in}{1.953839in}}%
\pgfpathlineto{\pgfqpoint{4.208262in}{1.971583in}}%
\pgfpathlineto{\pgfqpoint{4.213884in}{1.912758in}}%
\pgfpathlineto{\pgfqpoint{4.216695in}{2.094912in}}%
\pgfpathlineto{\pgfqpoint{4.219505in}{1.841691in}}%
\pgfpathlineto{\pgfqpoint{4.222316in}{1.885921in}}%
\pgfpathlineto{\pgfqpoint{4.225127in}{1.472346in}}%
\pgfpathlineto{\pgfqpoint{4.227937in}{2.156895in}}%
\pgfpathlineto{\pgfqpoint{4.230748in}{1.696565in}}%
\pgfpathlineto{\pgfqpoint{4.233559in}{1.894888in}}%
\pgfpathlineto{\pgfqpoint{4.236369in}{2.279747in}}%
\pgfpathlineto{\pgfqpoint{4.239180in}{2.334306in}}%
\pgfpathlineto{\pgfqpoint{4.241991in}{1.687426in}}%
\pgfpathlineto{\pgfqpoint{4.244801in}{1.881656in}}%
\pgfpathlineto{\pgfqpoint{4.247612in}{2.015962in}}%
\pgfpathlineto{\pgfqpoint{4.250423in}{1.929121in}}%
\pgfpathlineto{\pgfqpoint{4.253233in}{1.747709in}}%
\pgfpathlineto{\pgfqpoint{4.256044in}{1.671593in}}%
\pgfpathlineto{\pgfqpoint{4.258855in}{2.104770in}}%
\pgfpathlineto{\pgfqpoint{4.261665in}{1.994260in}}%
\pgfpathlineto{\pgfqpoint{4.264476in}{1.842819in}}%
\pgfpathlineto{\pgfqpoint{4.267287in}{2.062483in}}%
\pgfpathlineto{\pgfqpoint{4.270098in}{1.851797in}}%
\pgfpathlineto{\pgfqpoint{4.272908in}{1.843093in}}%
\pgfpathlineto{\pgfqpoint{4.275719in}{2.070001in}}%
\pgfpathlineto{\pgfqpoint{4.278530in}{1.920085in}}%
\pgfpathlineto{\pgfqpoint{4.281340in}{1.903077in}}%
\pgfpathlineto{\pgfqpoint{4.284151in}{1.860646in}}%
\pgfpathlineto{\pgfqpoint{4.286962in}{1.589985in}}%
\pgfpathlineto{\pgfqpoint{4.289772in}{1.833898in}}%
\pgfpathlineto{\pgfqpoint{4.292583in}{1.829316in}}%
\pgfpathlineto{\pgfqpoint{4.295394in}{2.081349in}}%
\pgfpathlineto{\pgfqpoint{4.298204in}{1.825515in}}%
\pgfpathlineto{\pgfqpoint{4.301015in}{2.075473in}}%
\pgfpathlineto{\pgfqpoint{4.303826in}{0.977323in}}%
\pgfpathlineto{\pgfqpoint{4.306636in}{1.853920in}}%
\pgfpathlineto{\pgfqpoint{4.309447in}{1.775924in}}%
\pgfpathlineto{\pgfqpoint{4.312258in}{1.936230in}}%
\pgfpathlineto{\pgfqpoint{4.315069in}{1.821485in}}%
\pgfpathlineto{\pgfqpoint{4.317879in}{1.811916in}}%
\pgfpathlineto{\pgfqpoint{4.320690in}{1.844020in}}%
\pgfpathlineto{\pgfqpoint{4.323501in}{1.862484in}}%
\pgfpathlineto{\pgfqpoint{4.326311in}{1.946039in}}%
\pgfpathlineto{\pgfqpoint{4.329122in}{1.736714in}}%
\pgfpathlineto{\pgfqpoint{4.331933in}{1.927745in}}%
\pgfpathlineto{\pgfqpoint{4.334743in}{1.711903in}}%
\pgfpathlineto{\pgfqpoint{4.337554in}{1.747948in}}%
\pgfpathlineto{\pgfqpoint{4.340365in}{2.359844in}}%
\pgfpathlineto{\pgfqpoint{4.343175in}{1.913444in}}%
\pgfpathlineto{\pgfqpoint{4.345986in}{1.894967in}}%
\pgfpathlineto{\pgfqpoint{4.348797in}{1.783870in}}%
\pgfpathlineto{\pgfqpoint{4.351607in}{1.936744in}}%
\pgfpathlineto{\pgfqpoint{4.354418in}{1.834667in}}%
\pgfpathlineto{\pgfqpoint{4.357229in}{2.075170in}}%
\pgfpathlineto{\pgfqpoint{4.360040in}{1.858175in}}%
\pgfpathlineto{\pgfqpoint{4.362850in}{1.968380in}}%
\pgfpathlineto{\pgfqpoint{4.365661in}{1.858274in}}%
\pgfpathlineto{\pgfqpoint{4.368472in}{1.904110in}}%
\pgfpathlineto{\pgfqpoint{4.371282in}{2.104236in}}%
\pgfpathlineto{\pgfqpoint{4.374093in}{1.772414in}}%
\pgfpathlineto{\pgfqpoint{4.376904in}{1.990208in}}%
\pgfpathlineto{\pgfqpoint{4.379714in}{1.921985in}}%
\pgfpathlineto{\pgfqpoint{4.382525in}{1.804337in}}%
\pgfpathlineto{\pgfqpoint{4.388146in}{1.482866in}}%
\pgfpathlineto{\pgfqpoint{4.390957in}{2.065014in}}%
\pgfpathlineto{\pgfqpoint{4.393768in}{1.988583in}}%
\pgfpathlineto{\pgfqpoint{4.396578in}{2.034187in}}%
\pgfpathlineto{\pgfqpoint{4.399389in}{2.223680in}}%
\pgfpathlineto{\pgfqpoint{4.402200in}{1.971411in}}%
\pgfpathlineto{\pgfqpoint{4.405011in}{1.917275in}}%
\pgfpathlineto{\pgfqpoint{4.407821in}{1.975287in}}%
\pgfpathlineto{\pgfqpoint{4.410632in}{2.234047in}}%
\pgfpathlineto{\pgfqpoint{4.413443in}{1.785442in}}%
\pgfpathlineto{\pgfqpoint{4.416253in}{1.990908in}}%
\pgfpathlineto{\pgfqpoint{4.419064in}{1.692977in}}%
\pgfpathlineto{\pgfqpoint{4.421875in}{2.126893in}}%
\pgfpathlineto{\pgfqpoint{4.424685in}{2.020161in}}%
\pgfpathlineto{\pgfqpoint{4.427496in}{1.795271in}}%
\pgfpathlineto{\pgfqpoint{4.430307in}{1.873000in}}%
\pgfpathlineto{\pgfqpoint{4.435928in}{1.933660in}}%
\pgfpathlineto{\pgfqpoint{4.438739in}{1.711991in}}%
\pgfpathlineto{\pgfqpoint{4.441549in}{1.903477in}}%
\pgfpathlineto{\pgfqpoint{4.444360in}{1.727684in}}%
\pgfpathlineto{\pgfqpoint{4.447171in}{2.026763in}}%
\pgfpathlineto{\pgfqpoint{4.449981in}{1.811299in}}%
\pgfpathlineto{\pgfqpoint{4.452792in}{1.868319in}}%
\pgfpathlineto{\pgfqpoint{4.455603in}{1.943180in}}%
\pgfpathlineto{\pgfqpoint{4.458414in}{1.942994in}}%
\pgfpathlineto{\pgfqpoint{4.461224in}{1.859669in}}%
\pgfpathlineto{\pgfqpoint{4.464035in}{2.060511in}}%
\pgfpathlineto{\pgfqpoint{4.466846in}{1.790238in}}%
\pgfpathlineto{\pgfqpoint{4.469656in}{1.925252in}}%
\pgfpathlineto{\pgfqpoint{4.472467in}{1.894710in}}%
\pgfpathlineto{\pgfqpoint{4.475278in}{1.872919in}}%
\pgfpathlineto{\pgfqpoint{4.478088in}{1.811647in}}%
\pgfpathlineto{\pgfqpoint{4.480899in}{2.043014in}}%
\pgfpathlineto{\pgfqpoint{4.483710in}{1.820759in}}%
\pgfpathlineto{\pgfqpoint{4.486520in}{2.230861in}}%
\pgfpathlineto{\pgfqpoint{4.489331in}{1.962622in}}%
\pgfpathlineto{\pgfqpoint{4.492142in}{1.792494in}}%
\pgfpathlineto{\pgfqpoint{4.494952in}{2.055911in}}%
\pgfpathlineto{\pgfqpoint{4.497763in}{1.669198in}}%
\pgfpathlineto{\pgfqpoint{4.500574in}{1.648462in}}%
\pgfpathlineto{\pgfqpoint{4.503385in}{1.772310in}}%
\pgfpathlineto{\pgfqpoint{4.506195in}{1.951677in}}%
\pgfpathlineto{\pgfqpoint{4.509006in}{1.934002in}}%
\pgfpathlineto{\pgfqpoint{4.511817in}{1.785380in}}%
\pgfpathlineto{\pgfqpoint{4.514627in}{1.925426in}}%
\pgfpathlineto{\pgfqpoint{4.517438in}{1.903484in}}%
\pgfpathlineto{\pgfqpoint{4.520249in}{1.504777in}}%
\pgfpathlineto{\pgfqpoint{4.523059in}{1.840885in}}%
\pgfpathlineto{\pgfqpoint{4.525870in}{2.082804in}}%
\pgfpathlineto{\pgfqpoint{4.528681in}{1.948050in}}%
\pgfpathlineto{\pgfqpoint{4.531491in}{1.939000in}}%
\pgfpathlineto{\pgfqpoint{4.534302in}{2.039722in}}%
\pgfpathlineto{\pgfqpoint{4.537113in}{1.916558in}}%
\pgfpathlineto{\pgfqpoint{4.539923in}{1.907797in}}%
\pgfpathlineto{\pgfqpoint{4.542734in}{1.706163in}}%
\pgfpathlineto{\pgfqpoint{4.545545in}{1.934414in}}%
\pgfpathlineto{\pgfqpoint{4.548356in}{2.034839in}}%
\pgfpathlineto{\pgfqpoint{4.551166in}{1.881636in}}%
\pgfpathlineto{\pgfqpoint{4.553977in}{1.763380in}}%
\pgfpathlineto{\pgfqpoint{4.556788in}{1.780020in}}%
\pgfpathlineto{\pgfqpoint{4.559598in}{1.881491in}}%
\pgfpathlineto{\pgfqpoint{4.562409in}{1.885920in}}%
\pgfpathlineto{\pgfqpoint{4.565220in}{1.752273in}}%
\pgfpathlineto{\pgfqpoint{4.568030in}{1.984030in}}%
\pgfpathlineto{\pgfqpoint{4.570841in}{1.814564in}}%
\pgfpathlineto{\pgfqpoint{4.573652in}{1.975033in}}%
\pgfpathlineto{\pgfqpoint{4.576462in}{1.930297in}}%
\pgfpathlineto{\pgfqpoint{4.579273in}{1.572292in}}%
\pgfpathlineto{\pgfqpoint{4.582084in}{1.899406in}}%
\pgfpathlineto{\pgfqpoint{4.584894in}{1.858644in}}%
\pgfpathlineto{\pgfqpoint{4.590516in}{1.949135in}}%
\pgfpathlineto{\pgfqpoint{4.593327in}{1.957891in}}%
\pgfpathlineto{\pgfqpoint{4.596137in}{1.723202in}}%
\pgfpathlineto{\pgfqpoint{4.598948in}{2.030525in}}%
\pgfpathlineto{\pgfqpoint{4.601759in}{1.849793in}}%
\pgfpathlineto{\pgfqpoint{4.604569in}{1.840630in}}%
\pgfpathlineto{\pgfqpoint{4.607380in}{1.989634in}}%
\pgfpathlineto{\pgfqpoint{4.610191in}{1.980039in}}%
\pgfpathlineto{\pgfqpoint{4.613001in}{1.876934in}}%
\pgfpathlineto{\pgfqpoint{4.615812in}{1.966232in}}%
\pgfpathlineto{\pgfqpoint{4.618623in}{2.023324in}}%
\pgfpathlineto{\pgfqpoint{4.621433in}{1.925613in}}%
\pgfpathlineto{\pgfqpoint{4.624244in}{1.938699in}}%
\pgfpathlineto{\pgfqpoint{4.629865in}{1.811042in}}%
\pgfpathlineto{\pgfqpoint{4.632676in}{1.890353in}}%
\pgfpathlineto{\pgfqpoint{4.635487in}{1.792842in}}%
\pgfpathlineto{\pgfqpoint{4.638298in}{1.868102in}}%
\pgfpathlineto{\pgfqpoint{4.641108in}{1.841284in}}%
\pgfpathlineto{\pgfqpoint{4.643919in}{1.733142in}}%
\pgfpathlineto{\pgfqpoint{4.646730in}{1.984878in}}%
\pgfpathlineto{\pgfqpoint{4.649540in}{2.015462in}}%
\pgfpathlineto{\pgfqpoint{4.652351in}{1.948085in}}%
\pgfpathlineto{\pgfqpoint{4.655162in}{2.000752in}}%
\pgfpathlineto{\pgfqpoint{4.657972in}{1.947439in}}%
\pgfpathlineto{\pgfqpoint{4.660783in}{2.064645in}}%
\pgfpathlineto{\pgfqpoint{4.663594in}{1.942246in}}%
\pgfpathlineto{\pgfqpoint{4.666404in}{1.911922in}}%
\pgfpathlineto{\pgfqpoint{4.669215in}{2.091591in}}%
\pgfpathlineto{\pgfqpoint{4.672026in}{1.371071in}}%
\pgfpathlineto{\pgfqpoint{4.674836in}{1.956260in}}%
\pgfpathlineto{\pgfqpoint{4.677647in}{2.159723in}}%
\pgfpathlineto{\pgfqpoint{4.683268in}{1.834708in}}%
\pgfpathlineto{\pgfqpoint{4.686079in}{1.881774in}}%
\pgfpathlineto{\pgfqpoint{4.688890in}{1.774049in}}%
\pgfpathlineto{\pgfqpoint{4.691701in}{1.821002in}}%
\pgfpathlineto{\pgfqpoint{4.694511in}{1.741902in}}%
\pgfpathlineto{\pgfqpoint{4.697322in}{1.758129in}}%
\pgfpathlineto{\pgfqpoint{4.700133in}{1.823799in}}%
\pgfpathlineto{\pgfqpoint{4.702943in}{1.930321in}}%
\pgfpathlineto{\pgfqpoint{4.705754in}{1.969969in}}%
\pgfpathlineto{\pgfqpoint{4.708565in}{1.564798in}}%
\pgfpathlineto{\pgfqpoint{4.711375in}{1.966804in}}%
\pgfpathlineto{\pgfqpoint{4.714186in}{1.966414in}}%
\pgfpathlineto{\pgfqpoint{4.716997in}{2.041322in}}%
\pgfpathlineto{\pgfqpoint{4.719807in}{1.930061in}}%
\pgfpathlineto{\pgfqpoint{4.722618in}{2.000102in}}%
\pgfpathlineto{\pgfqpoint{4.728239in}{1.890353in}}%
\pgfpathlineto{\pgfqpoint{4.731050in}{1.855454in}}%
\pgfpathlineto{\pgfqpoint{4.733861in}{1.864125in}}%
\pgfpathlineto{\pgfqpoint{4.736672in}{1.894728in}}%
\pgfpathlineto{\pgfqpoint{4.739482in}{1.968875in}}%
\pgfpathlineto{\pgfqpoint{4.742293in}{1.899052in}}%
\pgfpathlineto{\pgfqpoint{4.745104in}{1.803132in}}%
\pgfpathlineto{\pgfqpoint{4.747914in}{1.947105in}}%
\pgfpathlineto{\pgfqpoint{4.750725in}{1.981576in}}%
\pgfpathlineto{\pgfqpoint{4.753536in}{1.570816in}}%
\pgfpathlineto{\pgfqpoint{4.756346in}{1.899199in}}%
\pgfpathlineto{\pgfqpoint{4.759157in}{1.952130in}}%
\pgfpathlineto{\pgfqpoint{4.761968in}{1.744324in}}%
\pgfpathlineto{\pgfqpoint{4.767589in}{1.850025in}}%
\pgfpathlineto{\pgfqpoint{4.770400in}{2.015462in}}%
\pgfpathlineto{\pgfqpoint{4.773210in}{1.616434in}}%
\pgfpathlineto{\pgfqpoint{4.776021in}{1.767486in}}%
\pgfpathlineto{\pgfqpoint{4.781643in}{1.816587in}}%
\pgfpathlineto{\pgfqpoint{4.784453in}{1.839426in}}%
\pgfpathlineto{\pgfqpoint{4.787264in}{1.703686in}}%
\pgfpathlineto{\pgfqpoint{4.790075in}{2.132568in}}%
\pgfpathlineto{\pgfqpoint{4.792885in}{1.601104in}}%
\pgfpathlineto{\pgfqpoint{4.795696in}{1.979591in}}%
\pgfpathlineto{\pgfqpoint{4.798507in}{1.763380in}}%
\pgfpathlineto{\pgfqpoint{4.801317in}{2.277281in}}%
\pgfpathlineto{\pgfqpoint{4.804128in}{1.578799in}}%
\pgfpathlineto{\pgfqpoint{4.806939in}{2.002704in}}%
\pgfpathlineto{\pgfqpoint{4.809749in}{1.787397in}}%
\pgfpathlineto{\pgfqpoint{4.812560in}{2.011952in}}%
\pgfpathlineto{\pgfqpoint{4.815371in}{2.034053in}}%
\pgfpathlineto{\pgfqpoint{4.818181in}{1.885739in}}%
\pgfpathlineto{\pgfqpoint{4.820992in}{2.082921in}}%
\pgfpathlineto{\pgfqpoint{4.823803in}{1.798964in}}%
\pgfpathlineto{\pgfqpoint{4.826614in}{1.913253in}}%
\pgfpathlineto{\pgfqpoint{4.829424in}{1.904076in}}%
\pgfpathlineto{\pgfqpoint{4.832235in}{1.972427in}}%
\pgfpathlineto{\pgfqpoint{4.837856in}{1.794497in}}%
\pgfpathlineto{\pgfqpoint{4.840667in}{1.963442in}}%
\pgfpathlineto{\pgfqpoint{4.843478in}{1.922217in}}%
\pgfpathlineto{\pgfqpoint{4.846288in}{1.985539in}}%
\pgfpathlineto{\pgfqpoint{4.849099in}{2.034240in}}%
\pgfpathlineto{\pgfqpoint{4.851910in}{1.957325in}}%
\pgfpathlineto{\pgfqpoint{4.854720in}{2.255596in}}%
\pgfpathlineto{\pgfqpoint{4.857531in}{2.011578in}}%
\pgfpathlineto{\pgfqpoint{4.860342in}{1.937709in}}%
\pgfpathlineto{\pgfqpoint{4.863152in}{1.812783in}}%
\pgfpathlineto{\pgfqpoint{4.865963in}{1.946416in}}%
\pgfpathlineto{\pgfqpoint{4.868774in}{1.881742in}}%
\pgfpathlineto{\pgfqpoint{4.871584in}{1.963392in}}%
\pgfpathlineto{\pgfqpoint{4.874395in}{1.704912in}}%
\pgfpathlineto{\pgfqpoint{4.877206in}{1.777097in}}%
\pgfpathlineto{\pgfqpoint{4.880017in}{2.085832in}}%
\pgfpathlineto{\pgfqpoint{4.882827in}{1.751609in}}%
\pgfpathlineto{\pgfqpoint{4.885638in}{1.811739in}}%
\pgfpathlineto{\pgfqpoint{4.888449in}{1.620815in}}%
\pgfpathlineto{\pgfqpoint{4.891259in}{1.823320in}}%
\pgfpathlineto{\pgfqpoint{4.894070in}{1.849988in}}%
\pgfpathlineto{\pgfqpoint{4.896881in}{1.773127in}}%
\pgfpathlineto{\pgfqpoint{4.899691in}{1.894879in}}%
\pgfpathlineto{\pgfqpoint{4.902502in}{1.912961in}}%
\pgfpathlineto{\pgfqpoint{4.905313in}{1.872269in}}%
\pgfpathlineto{\pgfqpoint{4.908123in}{1.881303in}}%
\pgfpathlineto{\pgfqpoint{4.910934in}{1.926521in}}%
\pgfpathlineto{\pgfqpoint{4.913745in}{1.957932in}}%
\pgfpathlineto{\pgfqpoint{4.916555in}{1.966574in}}%
\pgfpathlineto{\pgfqpoint{4.919366in}{1.899295in}}%
\pgfpathlineto{\pgfqpoint{4.922177in}{1.859034in}}%
\pgfpathlineto{\pgfqpoint{4.924988in}{2.205013in}}%
\pgfpathlineto{\pgfqpoint{4.927798in}{1.797974in}}%
\pgfpathlineto{\pgfqpoint{4.930609in}{1.744020in}}%
\pgfpathlineto{\pgfqpoint{4.933420in}{2.133450in}}%
\pgfpathlineto{\pgfqpoint{4.936230in}{2.021301in}}%
\pgfpathlineto{\pgfqpoint{4.939041in}{1.989974in}}%
\pgfpathlineto{\pgfqpoint{4.941852in}{1.950668in}}%
\pgfpathlineto{\pgfqpoint{4.944662in}{2.103793in}}%
\pgfpathlineto{\pgfqpoint{4.947473in}{1.890353in}}%
\pgfpathlineto{\pgfqpoint{4.950284in}{1.983308in}}%
\pgfpathlineto{\pgfqpoint{4.953094in}{1.827038in}}%
\pgfpathlineto{\pgfqpoint{4.955905in}{1.941027in}}%
\pgfpathlineto{\pgfqpoint{4.958716in}{1.881919in}}%
\pgfpathlineto{\pgfqpoint{4.961526in}{1.881915in}}%
\pgfpathlineto{\pgfqpoint{4.964337in}{2.016432in}}%
\pgfpathlineto{\pgfqpoint{4.967148in}{1.915441in}}%
\pgfpathlineto{\pgfqpoint{4.969959in}{2.006877in}}%
\pgfpathlineto{\pgfqpoint{4.972769in}{1.989512in}}%
\pgfpathlineto{\pgfqpoint{4.975580in}{2.260454in}}%
\pgfpathlineto{\pgfqpoint{4.978391in}{2.018347in}}%
\pgfpathlineto{\pgfqpoint{4.981201in}{1.874414in}}%
\pgfpathlineto{\pgfqpoint{4.984012in}{1.965914in}}%
\pgfpathlineto{\pgfqpoint{4.986823in}{1.926012in}}%
\pgfpathlineto{\pgfqpoint{4.989633in}{1.977162in}}%
\pgfpathlineto{\pgfqpoint{4.992444in}{1.811457in}}%
\pgfpathlineto{\pgfqpoint{4.995255in}{1.759256in}}%
\pgfpathlineto{\pgfqpoint{4.998065in}{1.846397in}}%
\pgfpathlineto{\pgfqpoint{5.000876in}{2.073315in}}%
\pgfpathlineto{\pgfqpoint{5.003687in}{1.921945in}}%
\pgfpathlineto{\pgfqpoint{5.006497in}{1.902183in}}%
\pgfpathlineto{\pgfqpoint{5.009308in}{2.062786in}}%
\pgfpathlineto{\pgfqpoint{5.012119in}{1.831793in}}%
\pgfpathlineto{\pgfqpoint{5.014930in}{2.015005in}}%
\pgfpathlineto{\pgfqpoint{5.017740in}{2.037037in}}%
\pgfpathlineto{\pgfqpoint{5.023362in}{1.825111in}}%
\pgfpathlineto{\pgfqpoint{5.026172in}{2.100405in}}%
\pgfpathlineto{\pgfqpoint{5.028983in}{2.951493in}}%
\pgfpathlineto{\pgfqpoint{5.031794in}{1.879753in}}%
\pgfpathlineto{\pgfqpoint{5.034604in}{2.263832in}}%
\pgfpathlineto{\pgfqpoint{5.037415in}{2.284956in}}%
\pgfpathlineto{\pgfqpoint{5.040226in}{2.014054in}}%
\pgfpathlineto{\pgfqpoint{5.043036in}{1.645303in}}%
\pgfpathlineto{\pgfqpoint{5.045847in}{2.095396in}}%
\pgfpathlineto{\pgfqpoint{5.048658in}{1.917036in}}%
\pgfpathlineto{\pgfqpoint{5.051468in}{1.863670in}}%
\pgfpathlineto{\pgfqpoint{5.057090in}{1.656202in}}%
\pgfpathlineto{\pgfqpoint{5.059901in}{1.944977in}}%
\pgfpathlineto{\pgfqpoint{5.062711in}{1.927791in}}%
\pgfpathlineto{\pgfqpoint{5.065522in}{1.757191in}}%
\pgfpathlineto{\pgfqpoint{5.068333in}{1.951960in}}%
\pgfpathlineto{\pgfqpoint{5.071143in}{1.555309in}}%
\pgfpathlineto{\pgfqpoint{5.073954in}{1.886859in}}%
\pgfpathlineto{\pgfqpoint{5.076765in}{1.998274in}}%
\pgfpathlineto{\pgfqpoint{5.079575in}{1.792910in}}%
\pgfpathlineto{\pgfqpoint{5.082386in}{1.921744in}}%
\pgfpathlineto{\pgfqpoint{5.085197in}{1.802994in}}%
\pgfpathlineto{\pgfqpoint{5.088007in}{1.970742in}}%
\pgfpathlineto{\pgfqpoint{5.090818in}{1.630148in}}%
\pgfpathlineto{\pgfqpoint{5.093629in}{2.188852in}}%
\pgfpathlineto{\pgfqpoint{5.096439in}{1.838109in}}%
\pgfpathlineto{\pgfqpoint{5.099250in}{1.823913in}}%
\pgfpathlineto{\pgfqpoint{5.102061in}{1.535708in}}%
\pgfpathlineto{\pgfqpoint{5.104871in}{1.893942in}}%
\pgfpathlineto{\pgfqpoint{5.107682in}{1.764239in}}%
\pgfpathlineto{\pgfqpoint{5.110493in}{1.980543in}}%
\pgfpathlineto{\pgfqpoint{5.113304in}{1.997866in}}%
\pgfpathlineto{\pgfqpoint{5.116114in}{1.775645in}}%
\pgfpathlineto{\pgfqpoint{5.118925in}{1.893951in}}%
\pgfpathlineto{\pgfqpoint{5.121736in}{1.865149in}}%
\pgfpathlineto{\pgfqpoint{5.124546in}{2.330647in}}%
\pgfpathlineto{\pgfqpoint{5.127357in}{2.450039in}}%
\pgfpathlineto{\pgfqpoint{5.130168in}{2.137675in}}%
\pgfpathlineto{\pgfqpoint{5.132978in}{2.055171in}}%
\pgfpathlineto{\pgfqpoint{5.135789in}{1.639104in}}%
\pgfpathlineto{\pgfqpoint{5.138600in}{1.870336in}}%
\pgfpathlineto{\pgfqpoint{5.141410in}{1.692077in}}%
\pgfpathlineto{\pgfqpoint{5.144221in}{1.897118in}}%
\pgfpathlineto{\pgfqpoint{5.147032in}{1.927503in}}%
\pgfpathlineto{\pgfqpoint{5.149842in}{1.870101in}}%
\pgfpathlineto{\pgfqpoint{5.149842in}{1.870101in}}%
\pgfusepath{stroke}%
\end{pgfscope}%
\begin{pgfscope}%
\pgfpathrectangle{\pgfqpoint{0.711206in}{0.331635in}}{\pgfqpoint{4.650000in}{3.020000in}}%
\pgfusepath{clip}%
\pgfsetroundcap%
\pgfsetroundjoin%
\pgfsetlinewidth{1.505625pt}%
\definecolor{currentstroke}{rgb}{1.000000,0.647059,0.000000}%
\pgfsetstrokecolor{currentstroke}%
\pgfsetdash{}{0pt}%
\pgfpathmoveto{\pgfqpoint{0.922570in}{2.032378in}}%
\pgfpathlineto{\pgfqpoint{0.925380in}{2.030474in}}%
\pgfpathlineto{\pgfqpoint{0.928191in}{2.015800in}}%
\pgfpathlineto{\pgfqpoint{0.931002in}{2.015772in}}%
\pgfpathlineto{\pgfqpoint{0.933812in}{1.993658in}}%
\pgfpathlineto{\pgfqpoint{0.936623in}{1.936096in}}%
\pgfpathlineto{\pgfqpoint{0.939434in}{1.924306in}}%
\pgfpathlineto{\pgfqpoint{0.942245in}{1.941189in}}%
\pgfpathlineto{\pgfqpoint{0.945055in}{1.949391in}}%
\pgfpathlineto{\pgfqpoint{0.947866in}{1.980282in}}%
\pgfpathlineto{\pgfqpoint{0.950677in}{1.987634in}}%
\pgfpathlineto{\pgfqpoint{0.953487in}{1.988256in}}%
\pgfpathlineto{\pgfqpoint{0.956298in}{1.981729in}}%
\pgfpathlineto{\pgfqpoint{0.959109in}{1.970723in}}%
\pgfpathlineto{\pgfqpoint{0.961919in}{1.965578in}}%
\pgfpathlineto{\pgfqpoint{0.964730in}{1.961793in}}%
\pgfpathlineto{\pgfqpoint{0.967541in}{1.948283in}}%
\pgfpathlineto{\pgfqpoint{0.970351in}{1.949319in}}%
\pgfpathlineto{\pgfqpoint{0.973162in}{1.944811in}}%
\pgfpathlineto{\pgfqpoint{0.975973in}{1.953934in}}%
\pgfpathlineto{\pgfqpoint{0.978783in}{1.950864in}}%
\pgfpathlineto{\pgfqpoint{0.981594in}{1.946289in}}%
\pgfpathlineto{\pgfqpoint{0.984405in}{1.948829in}}%
\pgfpathlineto{\pgfqpoint{0.987216in}{1.946849in}}%
\pgfpathlineto{\pgfqpoint{0.990026in}{1.941378in}}%
\pgfpathlineto{\pgfqpoint{0.992837in}{1.939556in}}%
\pgfpathlineto{\pgfqpoint{0.995648in}{1.939555in}}%
\pgfpathlineto{\pgfqpoint{0.998458in}{1.934160in}}%
\pgfpathlineto{\pgfqpoint{1.001269in}{1.937285in}}%
\pgfpathlineto{\pgfqpoint{1.004080in}{1.944918in}}%
\pgfpathlineto{\pgfqpoint{1.006890in}{1.939923in}}%
\pgfpathlineto{\pgfqpoint{1.009701in}{1.931494in}}%
\pgfpathlineto{\pgfqpoint{1.012512in}{1.929112in}}%
\pgfpathlineto{\pgfqpoint{1.015322in}{1.928622in}}%
\pgfpathlineto{\pgfqpoint{1.018133in}{1.930431in}}%
\pgfpathlineto{\pgfqpoint{1.020944in}{1.934491in}}%
\pgfpathlineto{\pgfqpoint{1.023754in}{1.928375in}}%
\pgfpathlineto{\pgfqpoint{1.026565in}{1.927075in}}%
\pgfpathlineto{\pgfqpoint{1.029376in}{1.927032in}}%
\pgfpathlineto{\pgfqpoint{1.032187in}{1.921139in}}%
\pgfpathlineto{\pgfqpoint{1.034997in}{1.921245in}}%
\pgfpathlineto{\pgfqpoint{1.037808in}{1.924372in}}%
\pgfpathlineto{\pgfqpoint{1.040619in}{1.922684in}}%
\pgfpathlineto{\pgfqpoint{1.043429in}{1.924860in}}%
\pgfpathlineto{\pgfqpoint{1.046240in}{1.923091in}}%
\pgfpathlineto{\pgfqpoint{1.049051in}{1.928157in}}%
\pgfpathlineto{\pgfqpoint{1.051861in}{1.927106in}}%
\pgfpathlineto{\pgfqpoint{1.054672in}{1.929501in}}%
\pgfpathlineto{\pgfqpoint{1.060293in}{1.927865in}}%
\pgfpathlineto{\pgfqpoint{1.065915in}{1.926852in}}%
\pgfpathlineto{\pgfqpoint{1.068725in}{1.927415in}}%
\pgfpathlineto{\pgfqpoint{1.071536in}{1.926513in}}%
\pgfpathlineto{\pgfqpoint{1.074347in}{1.928817in}}%
\pgfpathlineto{\pgfqpoint{1.077158in}{1.928154in}}%
\pgfpathlineto{\pgfqpoint{1.079968in}{1.923988in}}%
\pgfpathlineto{\pgfqpoint{1.082779in}{1.926561in}}%
\pgfpathlineto{\pgfqpoint{1.085590in}{1.925639in}}%
\pgfpathlineto{\pgfqpoint{1.088400in}{1.927322in}}%
\pgfpathlineto{\pgfqpoint{1.091211in}{1.924413in}}%
\pgfpathlineto{\pgfqpoint{1.094022in}{1.922414in}}%
\pgfpathlineto{\pgfqpoint{1.096832in}{1.923056in}}%
\pgfpathlineto{\pgfqpoint{1.102454in}{1.917130in}}%
\pgfpathlineto{\pgfqpoint{1.105264in}{1.919933in}}%
\pgfpathlineto{\pgfqpoint{1.108075in}{1.924338in}}%
\pgfpathlineto{\pgfqpoint{1.110886in}{1.920911in}}%
\pgfpathlineto{\pgfqpoint{1.113696in}{1.922822in}}%
\pgfpathlineto{\pgfqpoint{1.116507in}{1.922830in}}%
\pgfpathlineto{\pgfqpoint{1.119318in}{1.918553in}}%
\pgfpathlineto{\pgfqpoint{1.122128in}{1.916347in}}%
\pgfpathlineto{\pgfqpoint{1.127750in}{1.913959in}}%
\pgfpathlineto{\pgfqpoint{1.130561in}{1.912672in}}%
\pgfpathlineto{\pgfqpoint{1.133371in}{1.916141in}}%
\pgfpathlineto{\pgfqpoint{1.136182in}{1.918274in}}%
\pgfpathlineto{\pgfqpoint{1.138993in}{1.919004in}}%
\pgfpathlineto{\pgfqpoint{1.141803in}{1.918732in}}%
\pgfpathlineto{\pgfqpoint{1.144614in}{1.921902in}}%
\pgfpathlineto{\pgfqpoint{1.147425in}{1.922924in}}%
\pgfpathlineto{\pgfqpoint{1.150235in}{1.919988in}}%
\pgfpathlineto{\pgfqpoint{1.153046in}{1.915597in}}%
\pgfpathlineto{\pgfqpoint{1.155857in}{1.914368in}}%
\pgfpathlineto{\pgfqpoint{1.158667in}{1.911681in}}%
\pgfpathlineto{\pgfqpoint{1.161478in}{1.910351in}}%
\pgfpathlineto{\pgfqpoint{1.164289in}{1.910434in}}%
\pgfpathlineto{\pgfqpoint{1.167099in}{1.912553in}}%
\pgfpathlineto{\pgfqpoint{1.169910in}{1.908633in}}%
\pgfpathlineto{\pgfqpoint{1.172721in}{1.907607in}}%
\pgfpathlineto{\pgfqpoint{1.178342in}{1.903126in}}%
\pgfpathlineto{\pgfqpoint{1.181153in}{1.902241in}}%
\pgfpathlineto{\pgfqpoint{1.183964in}{1.902638in}}%
\pgfpathlineto{\pgfqpoint{1.186774in}{1.901774in}}%
\pgfpathlineto{\pgfqpoint{1.189585in}{1.898181in}}%
\pgfpathlineto{\pgfqpoint{1.192396in}{1.899356in}}%
\pgfpathlineto{\pgfqpoint{1.195206in}{1.899774in}}%
\pgfpathlineto{\pgfqpoint{1.198017in}{1.901676in}}%
\pgfpathlineto{\pgfqpoint{1.200828in}{1.901775in}}%
\pgfpathlineto{\pgfqpoint{1.203638in}{1.900061in}}%
\pgfpathlineto{\pgfqpoint{1.206449in}{1.896028in}}%
\pgfpathlineto{\pgfqpoint{1.209260in}{1.895407in}}%
\pgfpathlineto{\pgfqpoint{1.212070in}{1.897521in}}%
\pgfpathlineto{\pgfqpoint{1.214881in}{1.900907in}}%
\pgfpathlineto{\pgfqpoint{1.217692in}{1.900146in}}%
\pgfpathlineto{\pgfqpoint{1.220503in}{1.902475in}}%
\pgfpathlineto{\pgfqpoint{1.223313in}{1.900229in}}%
\pgfpathlineto{\pgfqpoint{1.226124in}{1.902828in}}%
\pgfpathlineto{\pgfqpoint{1.228935in}{1.902842in}}%
\pgfpathlineto{\pgfqpoint{1.234556in}{1.906510in}}%
\pgfpathlineto{\pgfqpoint{1.242988in}{1.907487in}}%
\pgfpathlineto{\pgfqpoint{1.245799in}{1.902979in}}%
\pgfpathlineto{\pgfqpoint{1.248609in}{1.903983in}}%
\pgfpathlineto{\pgfqpoint{1.251420in}{1.899704in}}%
\pgfpathlineto{\pgfqpoint{1.254231in}{1.899390in}}%
\pgfpathlineto{\pgfqpoint{1.257041in}{1.900312in}}%
\pgfpathlineto{\pgfqpoint{1.259852in}{1.898424in}}%
\pgfpathlineto{\pgfqpoint{1.262663in}{1.902142in}}%
\pgfpathlineto{\pgfqpoint{1.265474in}{1.902104in}}%
\pgfpathlineto{\pgfqpoint{1.268284in}{1.902841in}}%
\pgfpathlineto{\pgfqpoint{1.273906in}{1.899567in}}%
\pgfpathlineto{\pgfqpoint{1.276716in}{1.899550in}}%
\pgfpathlineto{\pgfqpoint{1.279527in}{1.896761in}}%
\pgfpathlineto{\pgfqpoint{1.282338in}{1.895985in}}%
\pgfpathlineto{\pgfqpoint{1.285148in}{1.893011in}}%
\pgfpathlineto{\pgfqpoint{1.287959in}{1.895291in}}%
\pgfpathlineto{\pgfqpoint{1.290770in}{1.894702in}}%
\pgfpathlineto{\pgfqpoint{1.293580in}{1.895764in}}%
\pgfpathlineto{\pgfqpoint{1.296391in}{1.899234in}}%
\pgfpathlineto{\pgfqpoint{1.299202in}{1.898544in}}%
\pgfpathlineto{\pgfqpoint{1.302012in}{1.896231in}}%
\pgfpathlineto{\pgfqpoint{1.307634in}{1.893984in}}%
\pgfpathlineto{\pgfqpoint{1.310445in}{1.894486in}}%
\pgfpathlineto{\pgfqpoint{1.316066in}{1.898050in}}%
\pgfpathlineto{\pgfqpoint{1.318877in}{1.896946in}}%
\pgfpathlineto{\pgfqpoint{1.321687in}{1.896700in}}%
\pgfpathlineto{\pgfqpoint{1.324498in}{1.897548in}}%
\pgfpathlineto{\pgfqpoint{1.327309in}{1.897401in}}%
\pgfpathlineto{\pgfqpoint{1.330119in}{1.899471in}}%
\pgfpathlineto{\pgfqpoint{1.341362in}{1.900978in}}%
\pgfpathlineto{\pgfqpoint{1.344173in}{1.901586in}}%
\pgfpathlineto{\pgfqpoint{1.352605in}{1.899184in}}%
\pgfpathlineto{\pgfqpoint{1.355415in}{1.900242in}}%
\pgfpathlineto{\pgfqpoint{1.361037in}{1.898840in}}%
\pgfpathlineto{\pgfqpoint{1.363848in}{1.898390in}}%
\pgfpathlineto{\pgfqpoint{1.366658in}{1.896967in}}%
\pgfpathlineto{\pgfqpoint{1.369469in}{1.894407in}}%
\pgfpathlineto{\pgfqpoint{1.375090in}{1.893635in}}%
\pgfpathlineto{\pgfqpoint{1.377901in}{1.894199in}}%
\pgfpathlineto{\pgfqpoint{1.383522in}{1.891501in}}%
\pgfpathlineto{\pgfqpoint{1.386333in}{1.893512in}}%
\pgfpathlineto{\pgfqpoint{1.389144in}{1.892027in}}%
\pgfpathlineto{\pgfqpoint{1.391954in}{1.891928in}}%
\pgfpathlineto{\pgfqpoint{1.394765in}{1.894410in}}%
\pgfpathlineto{\pgfqpoint{1.400386in}{1.887818in}}%
\pgfpathlineto{\pgfqpoint{1.403197in}{1.888106in}}%
\pgfpathlineto{\pgfqpoint{1.406008in}{1.887574in}}%
\pgfpathlineto{\pgfqpoint{1.408819in}{1.888222in}}%
\pgfpathlineto{\pgfqpoint{1.417251in}{1.888301in}}%
\pgfpathlineto{\pgfqpoint{1.420061in}{1.887518in}}%
\pgfpathlineto{\pgfqpoint{1.425683in}{1.887461in}}%
\pgfpathlineto{\pgfqpoint{1.431304in}{1.885424in}}%
\pgfpathlineto{\pgfqpoint{1.434115in}{1.885849in}}%
\pgfpathlineto{\pgfqpoint{1.436925in}{1.887394in}}%
\pgfpathlineto{\pgfqpoint{1.439736in}{1.885897in}}%
\pgfpathlineto{\pgfqpoint{1.445357in}{1.886589in}}%
\pgfpathlineto{\pgfqpoint{1.448168in}{1.885580in}}%
\pgfpathlineto{\pgfqpoint{1.450979in}{1.885347in}}%
\pgfpathlineto{\pgfqpoint{1.453790in}{1.886098in}}%
\pgfpathlineto{\pgfqpoint{1.456600in}{1.885527in}}%
\pgfpathlineto{\pgfqpoint{1.459411in}{1.883448in}}%
\pgfpathlineto{\pgfqpoint{1.467843in}{1.882082in}}%
\pgfpathlineto{\pgfqpoint{1.470654in}{1.882988in}}%
\pgfpathlineto{\pgfqpoint{1.473464in}{1.885133in}}%
\pgfpathlineto{\pgfqpoint{1.476275in}{1.883274in}}%
\pgfpathlineto{\pgfqpoint{1.479086in}{1.882886in}}%
\pgfpathlineto{\pgfqpoint{1.481896in}{1.881558in}}%
\pgfpathlineto{\pgfqpoint{1.487518in}{1.882744in}}%
\pgfpathlineto{\pgfqpoint{1.490328in}{1.882362in}}%
\pgfpathlineto{\pgfqpoint{1.495950in}{1.884045in}}%
\pgfpathlineto{\pgfqpoint{1.498761in}{1.883014in}}%
\pgfpathlineto{\pgfqpoint{1.501571in}{1.885106in}}%
\pgfpathlineto{\pgfqpoint{1.504382in}{1.884495in}}%
\pgfpathlineto{\pgfqpoint{1.510003in}{1.884193in}}%
\pgfpathlineto{\pgfqpoint{1.512814in}{1.881562in}}%
\pgfpathlineto{\pgfqpoint{1.521246in}{1.881196in}}%
\pgfpathlineto{\pgfqpoint{1.526867in}{1.878559in}}%
\pgfpathlineto{\pgfqpoint{1.535299in}{1.879708in}}%
\pgfpathlineto{\pgfqpoint{1.538110in}{1.877226in}}%
\pgfpathlineto{\pgfqpoint{1.540921in}{1.876778in}}%
\pgfpathlineto{\pgfqpoint{1.543731in}{1.878093in}}%
\pgfpathlineto{\pgfqpoint{1.552164in}{1.879469in}}%
\pgfpathlineto{\pgfqpoint{1.554974in}{1.877653in}}%
\pgfpathlineto{\pgfqpoint{1.560596in}{1.877804in}}%
\pgfpathlineto{\pgfqpoint{1.563406in}{1.879262in}}%
\pgfpathlineto{\pgfqpoint{1.566217in}{1.878914in}}%
\pgfpathlineto{\pgfqpoint{1.569028in}{1.879945in}}%
\pgfpathlineto{\pgfqpoint{1.574649in}{1.879801in}}%
\pgfpathlineto{\pgfqpoint{1.577460in}{1.881598in}}%
\pgfpathlineto{\pgfqpoint{1.580270in}{1.881709in}}%
\pgfpathlineto{\pgfqpoint{1.583081in}{1.881185in}}%
\pgfpathlineto{\pgfqpoint{1.588702in}{1.881522in}}%
\pgfpathlineto{\pgfqpoint{1.591513in}{1.882728in}}%
\pgfpathlineto{\pgfqpoint{1.594324in}{1.883156in}}%
\pgfpathlineto{\pgfqpoint{1.599945in}{1.882247in}}%
\pgfpathlineto{\pgfqpoint{1.611188in}{1.880794in}}%
\pgfpathlineto{\pgfqpoint{1.613999in}{1.881984in}}%
\pgfpathlineto{\pgfqpoint{1.616809in}{1.884179in}}%
\pgfpathlineto{\pgfqpoint{1.628052in}{1.883461in}}%
\pgfpathlineto{\pgfqpoint{1.636484in}{1.885989in}}%
\pgfpathlineto{\pgfqpoint{1.644916in}{1.886326in}}%
\pgfpathlineto{\pgfqpoint{1.647727in}{1.887813in}}%
\pgfpathlineto{\pgfqpoint{1.650538in}{1.884101in}}%
\pgfpathlineto{\pgfqpoint{1.664591in}{1.883699in}}%
\pgfpathlineto{\pgfqpoint{1.670212in}{1.884583in}}%
\pgfpathlineto{\pgfqpoint{1.673023in}{1.883741in}}%
\pgfpathlineto{\pgfqpoint{1.675834in}{1.884594in}}%
\pgfpathlineto{\pgfqpoint{1.678644in}{1.884109in}}%
\pgfpathlineto{\pgfqpoint{1.681455in}{1.884761in}}%
\pgfpathlineto{\pgfqpoint{1.687077in}{1.883861in}}%
\pgfpathlineto{\pgfqpoint{1.695509in}{1.884893in}}%
\pgfpathlineto{\pgfqpoint{1.701130in}{1.885023in}}%
\pgfpathlineto{\pgfqpoint{1.709562in}{1.883829in}}%
\pgfpathlineto{\pgfqpoint{1.712373in}{1.882607in}}%
\pgfpathlineto{\pgfqpoint{1.715183in}{1.883072in}}%
\pgfpathlineto{\pgfqpoint{1.717994in}{1.882630in}}%
\pgfpathlineto{\pgfqpoint{1.723615in}{1.884425in}}%
\pgfpathlineto{\pgfqpoint{1.729237in}{1.884734in}}%
\pgfpathlineto{\pgfqpoint{1.740480in}{1.886848in}}%
\pgfpathlineto{\pgfqpoint{1.743290in}{1.886126in}}%
\pgfpathlineto{\pgfqpoint{1.748912in}{1.886295in}}%
\pgfpathlineto{\pgfqpoint{1.754533in}{1.886351in}}%
\pgfpathlineto{\pgfqpoint{1.760154in}{1.885480in}}%
\pgfpathlineto{\pgfqpoint{1.768586in}{1.885021in}}%
\pgfpathlineto{\pgfqpoint{1.771397in}{1.885683in}}%
\pgfpathlineto{\pgfqpoint{1.774208in}{1.885309in}}%
\pgfpathlineto{\pgfqpoint{1.777018in}{1.886726in}}%
\pgfpathlineto{\pgfqpoint{1.782640in}{1.886884in}}%
\pgfpathlineto{\pgfqpoint{1.785451in}{1.886005in}}%
\pgfpathlineto{\pgfqpoint{1.788261in}{1.886073in}}%
\pgfpathlineto{\pgfqpoint{1.791072in}{1.885158in}}%
\pgfpathlineto{\pgfqpoint{1.793883in}{1.885394in}}%
\pgfpathlineto{\pgfqpoint{1.796693in}{1.884944in}}%
\pgfpathlineto{\pgfqpoint{1.799504in}{1.885317in}}%
\pgfpathlineto{\pgfqpoint{1.805125in}{1.887879in}}%
\pgfpathlineto{\pgfqpoint{1.810747in}{1.886654in}}%
\pgfpathlineto{\pgfqpoint{1.813557in}{1.886035in}}%
\pgfpathlineto{\pgfqpoint{1.816368in}{1.887195in}}%
\pgfpathlineto{\pgfqpoint{1.819179in}{1.887230in}}%
\pgfpathlineto{\pgfqpoint{1.833232in}{1.890729in}}%
\pgfpathlineto{\pgfqpoint{1.836043in}{1.890188in}}%
\pgfpathlineto{\pgfqpoint{1.838854in}{1.890236in}}%
\pgfpathlineto{\pgfqpoint{1.844475in}{1.891276in}}%
\pgfpathlineto{\pgfqpoint{1.864150in}{1.892427in}}%
\pgfpathlineto{\pgfqpoint{1.866960in}{1.892681in}}%
\pgfpathlineto{\pgfqpoint{1.872582in}{1.891470in}}%
\pgfpathlineto{\pgfqpoint{1.875393in}{1.892122in}}%
\pgfpathlineto{\pgfqpoint{1.878203in}{1.891639in}}%
\pgfpathlineto{\pgfqpoint{1.889446in}{1.891861in}}%
\pgfpathlineto{\pgfqpoint{1.903499in}{1.892093in}}%
\pgfpathlineto{\pgfqpoint{1.906310in}{1.892193in}}%
\pgfpathlineto{\pgfqpoint{1.909121in}{1.893825in}}%
\pgfpathlineto{\pgfqpoint{1.911931in}{1.894015in}}%
\pgfpathlineto{\pgfqpoint{1.914742in}{1.892898in}}%
\pgfpathlineto{\pgfqpoint{1.920364in}{1.892700in}}%
\pgfpathlineto{\pgfqpoint{1.923174in}{1.893399in}}%
\pgfpathlineto{\pgfqpoint{1.928796in}{1.892458in}}%
\pgfpathlineto{\pgfqpoint{1.931606in}{1.893334in}}%
\pgfpathlineto{\pgfqpoint{1.937228in}{1.893514in}}%
\pgfpathlineto{\pgfqpoint{1.940038in}{1.894089in}}%
\pgfpathlineto{\pgfqpoint{1.951281in}{1.890934in}}%
\pgfpathlineto{\pgfqpoint{1.956902in}{1.891663in}}%
\pgfpathlineto{\pgfqpoint{1.965334in}{1.891452in}}%
\pgfpathlineto{\pgfqpoint{1.970956in}{1.891226in}}%
\pgfpathlineto{\pgfqpoint{1.973767in}{1.891721in}}%
\pgfpathlineto{\pgfqpoint{1.976577in}{1.890271in}}%
\pgfpathlineto{\pgfqpoint{1.982199in}{1.890373in}}%
\pgfpathlineto{\pgfqpoint{1.985009in}{1.891589in}}%
\pgfpathlineto{\pgfqpoint{1.990631in}{1.891505in}}%
\pgfpathlineto{\pgfqpoint{1.993441in}{1.892006in}}%
\pgfpathlineto{\pgfqpoint{1.996252in}{1.891847in}}%
\pgfpathlineto{\pgfqpoint{1.999063in}{1.890353in}}%
\pgfpathlineto{\pgfqpoint{2.007495in}{1.889531in}}%
\pgfpathlineto{\pgfqpoint{2.021548in}{1.890587in}}%
\pgfpathlineto{\pgfqpoint{2.038412in}{1.889828in}}%
\pgfpathlineto{\pgfqpoint{2.044034in}{1.889518in}}%
\pgfpathlineto{\pgfqpoint{2.052466in}{1.889641in}}%
\pgfpathlineto{\pgfqpoint{2.055276in}{1.888740in}}%
\pgfpathlineto{\pgfqpoint{2.058087in}{1.888546in}}%
\pgfpathlineto{\pgfqpoint{2.063709in}{1.889552in}}%
\pgfpathlineto{\pgfqpoint{2.066519in}{1.888991in}}%
\pgfpathlineto{\pgfqpoint{2.080573in}{1.889200in}}%
\pgfpathlineto{\pgfqpoint{2.086194in}{1.888705in}}%
\pgfpathlineto{\pgfqpoint{2.089005in}{1.888864in}}%
\pgfpathlineto{\pgfqpoint{2.091815in}{1.889759in}}%
\pgfpathlineto{\pgfqpoint{2.097437in}{1.889817in}}%
\pgfpathlineto{\pgfqpoint{2.103058in}{1.890298in}}%
\pgfpathlineto{\pgfqpoint{2.108680in}{1.889767in}}%
\pgfpathlineto{\pgfqpoint{2.111490in}{1.890999in}}%
\pgfpathlineto{\pgfqpoint{2.114301in}{1.890909in}}%
\pgfpathlineto{\pgfqpoint{2.119922in}{1.891660in}}%
\pgfpathlineto{\pgfqpoint{2.133976in}{1.891352in}}%
\pgfpathlineto{\pgfqpoint{2.145218in}{1.890070in}}%
\pgfpathlineto{\pgfqpoint{2.156461in}{1.890072in}}%
\pgfpathlineto{\pgfqpoint{2.159272in}{1.889560in}}%
\pgfpathlineto{\pgfqpoint{2.162083in}{1.889721in}}%
\pgfpathlineto{\pgfqpoint{2.167704in}{1.890697in}}%
\pgfpathlineto{\pgfqpoint{2.184568in}{1.891920in}}%
\pgfpathlineto{\pgfqpoint{2.195811in}{1.892052in}}%
\pgfpathlineto{\pgfqpoint{2.207054in}{1.892338in}}%
\pgfpathlineto{\pgfqpoint{2.215486in}{1.892042in}}%
\pgfpathlineto{\pgfqpoint{2.221107in}{1.892082in}}%
\pgfpathlineto{\pgfqpoint{2.229539in}{1.892551in}}%
\pgfpathlineto{\pgfqpoint{2.237971in}{1.892659in}}%
\pgfpathlineto{\pgfqpoint{2.249214in}{1.893534in}}%
\pgfpathlineto{\pgfqpoint{2.252025in}{1.891779in}}%
\pgfpathlineto{\pgfqpoint{2.260457in}{1.891817in}}%
\pgfpathlineto{\pgfqpoint{2.271699in}{1.891590in}}%
\pgfpathlineto{\pgfqpoint{2.280131in}{1.893086in}}%
\pgfpathlineto{\pgfqpoint{2.296996in}{1.892721in}}%
\pgfpathlineto{\pgfqpoint{2.299806in}{1.893318in}}%
\pgfpathlineto{\pgfqpoint{2.311049in}{1.893631in}}%
\pgfpathlineto{\pgfqpoint{2.319481in}{1.894091in}}%
\pgfpathlineto{\pgfqpoint{2.327913in}{1.893988in}}%
\pgfpathlineto{\pgfqpoint{2.333534in}{1.893756in}}%
\pgfpathlineto{\pgfqpoint{2.341967in}{1.893668in}}%
\pgfpathlineto{\pgfqpoint{2.344777in}{1.893635in}}%
\pgfpathlineto{\pgfqpoint{2.347588in}{1.894756in}}%
\pgfpathlineto{\pgfqpoint{2.353209in}{1.894778in}}%
\pgfpathlineto{\pgfqpoint{2.358831in}{1.893697in}}%
\pgfpathlineto{\pgfqpoint{2.370073in}{1.892685in}}%
\pgfpathlineto{\pgfqpoint{2.375695in}{1.892621in}}%
\pgfpathlineto{\pgfqpoint{2.381316in}{1.892448in}}%
\pgfpathlineto{\pgfqpoint{2.386937in}{1.891592in}}%
\pgfpathlineto{\pgfqpoint{2.389748in}{1.891505in}}%
\pgfpathlineto{\pgfqpoint{2.395370in}{1.892319in}}%
\pgfpathlineto{\pgfqpoint{2.417855in}{1.892889in}}%
\pgfpathlineto{\pgfqpoint{2.420666in}{1.892527in}}%
\pgfpathlineto{\pgfqpoint{2.429098in}{1.892935in}}%
\pgfpathlineto{\pgfqpoint{2.465637in}{1.892772in}}%
\pgfpathlineto{\pgfqpoint{2.516229in}{1.894467in}}%
\pgfpathlineto{\pgfqpoint{2.519040in}{1.894204in}}%
\pgfpathlineto{\pgfqpoint{2.527472in}{1.894928in}}%
\pgfpathlineto{\pgfqpoint{2.535904in}{1.894814in}}%
\pgfpathlineto{\pgfqpoint{2.564011in}{1.894337in}}%
\pgfpathlineto{\pgfqpoint{2.575253in}{1.894477in}}%
\pgfpathlineto{\pgfqpoint{2.580875in}{1.894429in}}%
\pgfpathlineto{\pgfqpoint{2.586496in}{1.893926in}}%
\pgfpathlineto{\pgfqpoint{2.597739in}{1.894213in}}%
\pgfpathlineto{\pgfqpoint{2.637089in}{1.895751in}}%
\pgfpathlineto{\pgfqpoint{2.639899in}{1.897350in}}%
\pgfpathlineto{\pgfqpoint{2.673628in}{1.898087in}}%
\pgfpathlineto{\pgfqpoint{2.698924in}{1.898524in}}%
\pgfpathlineto{\pgfqpoint{2.701734in}{1.900593in}}%
\pgfpathlineto{\pgfqpoint{2.704545in}{1.899926in}}%
\pgfpathlineto{\pgfqpoint{2.710166in}{1.900144in}}%
\pgfpathlineto{\pgfqpoint{2.712977in}{1.900623in}}%
\pgfpathlineto{\pgfqpoint{2.724220in}{1.900185in}}%
\pgfpathlineto{\pgfqpoint{2.738273in}{1.899985in}}%
\pgfpathlineto{\pgfqpoint{2.741084in}{1.899280in}}%
\pgfpathlineto{\pgfqpoint{2.755137in}{1.899427in}}%
\pgfpathlineto{\pgfqpoint{2.757948in}{1.900067in}}%
\pgfpathlineto{\pgfqpoint{2.766380in}{1.900228in}}%
\pgfpathlineto{\pgfqpoint{2.772002in}{1.900259in}}%
\pgfpathlineto{\pgfqpoint{2.777623in}{1.900512in}}%
\pgfpathlineto{\pgfqpoint{2.797298in}{1.900170in}}%
\pgfpathlineto{\pgfqpoint{2.805730in}{1.900608in}}%
\pgfpathlineto{\pgfqpoint{2.811351in}{1.900383in}}%
\pgfpathlineto{\pgfqpoint{2.850701in}{1.900050in}}%
\pgfpathlineto{\pgfqpoint{2.856322in}{1.899200in}}%
\pgfpathlineto{\pgfqpoint{2.861944in}{1.899556in}}%
\pgfpathlineto{\pgfqpoint{2.864754in}{1.899132in}}%
\pgfpathlineto{\pgfqpoint{2.867565in}{1.899630in}}%
\pgfpathlineto{\pgfqpoint{2.870376in}{1.899203in}}%
\pgfpathlineto{\pgfqpoint{2.873186in}{1.898077in}}%
\pgfpathlineto{\pgfqpoint{2.875997in}{1.897763in}}%
\pgfpathlineto{\pgfqpoint{2.878808in}{1.898208in}}%
\pgfpathlineto{\pgfqpoint{2.884429in}{1.897307in}}%
\pgfpathlineto{\pgfqpoint{2.895672in}{1.898225in}}%
\pgfpathlineto{\pgfqpoint{2.909725in}{1.899215in}}%
\pgfpathlineto{\pgfqpoint{2.912536in}{1.898359in}}%
\pgfpathlineto{\pgfqpoint{2.915347in}{1.899246in}}%
\pgfpathlineto{\pgfqpoint{2.920968in}{1.899544in}}%
\pgfpathlineto{\pgfqpoint{2.926589in}{1.899218in}}%
\pgfpathlineto{\pgfqpoint{2.937832in}{1.898894in}}%
\pgfpathlineto{\pgfqpoint{2.979992in}{1.901061in}}%
\pgfpathlineto{\pgfqpoint{2.985614in}{1.900890in}}%
\pgfpathlineto{\pgfqpoint{3.002478in}{1.899902in}}%
\pgfpathlineto{\pgfqpoint{3.008099in}{1.900684in}}%
\pgfpathlineto{\pgfqpoint{3.010910in}{1.900315in}}%
\pgfpathlineto{\pgfqpoint{3.016531in}{1.900863in}}%
\pgfpathlineto{\pgfqpoint{3.027774in}{1.900446in}}%
\pgfpathlineto{\pgfqpoint{3.039017in}{1.899578in}}%
\pgfpathlineto{\pgfqpoint{3.047449in}{1.900353in}}%
\pgfpathlineto{\pgfqpoint{3.067124in}{1.900110in}}%
\pgfpathlineto{\pgfqpoint{3.072745in}{1.900072in}}%
\pgfpathlineto{\pgfqpoint{3.089609in}{1.898456in}}%
\pgfpathlineto{\pgfqpoint{3.137391in}{1.898416in}}%
\pgfpathlineto{\pgfqpoint{3.140202in}{1.898182in}}%
\pgfpathlineto{\pgfqpoint{3.145823in}{1.898637in}}%
\pgfpathlineto{\pgfqpoint{3.154255in}{1.898101in}}%
\pgfpathlineto{\pgfqpoint{3.173930in}{1.896520in}}%
\pgfpathlineto{\pgfqpoint{3.185173in}{1.896849in}}%
\pgfpathlineto{\pgfqpoint{3.193605in}{1.896157in}}%
\pgfpathlineto{\pgfqpoint{3.196415in}{1.897285in}}%
\pgfpathlineto{\pgfqpoint{3.204847in}{1.896565in}}%
\pgfpathlineto{\pgfqpoint{3.218901in}{1.896780in}}%
\pgfpathlineto{\pgfqpoint{3.224522in}{1.897046in}}%
\pgfpathlineto{\pgfqpoint{3.227333in}{1.896901in}}%
\pgfpathlineto{\pgfqpoint{3.230143in}{1.897647in}}%
\pgfpathlineto{\pgfqpoint{3.325707in}{1.897430in}}%
\pgfpathlineto{\pgfqpoint{3.336950in}{1.896599in}}%
\pgfpathlineto{\pgfqpoint{3.342571in}{1.896911in}}%
\pgfpathlineto{\pgfqpoint{3.348192in}{1.896644in}}%
\pgfpathlineto{\pgfqpoint{3.365056in}{1.896978in}}%
\pgfpathlineto{\pgfqpoint{3.393163in}{1.895478in}}%
\pgfpathlineto{\pgfqpoint{3.398785in}{1.895278in}}%
\pgfpathlineto{\pgfqpoint{3.404406in}{1.895544in}}%
\pgfpathlineto{\pgfqpoint{3.415649in}{1.895206in}}%
\pgfpathlineto{\pgfqpoint{3.426892in}{1.894579in}}%
\pgfpathlineto{\pgfqpoint{3.446566in}{1.895159in}}%
\pgfpathlineto{\pgfqpoint{3.457809in}{1.895566in}}%
\pgfpathlineto{\pgfqpoint{3.460620in}{1.895184in}}%
\pgfpathlineto{\pgfqpoint{3.463430in}{1.895449in}}%
\pgfpathlineto{\pgfqpoint{3.469052in}{1.895193in}}%
\pgfpathlineto{\pgfqpoint{3.477484in}{1.894801in}}%
\pgfpathlineto{\pgfqpoint{3.488727in}{1.893283in}}%
\pgfpathlineto{\pgfqpoint{3.491537in}{1.894156in}}%
\pgfpathlineto{\pgfqpoint{3.499969in}{1.894864in}}%
\pgfpathlineto{\pgfqpoint{3.502780in}{1.894448in}}%
\pgfpathlineto{\pgfqpoint{3.508401in}{1.895151in}}%
\pgfpathlineto{\pgfqpoint{3.511212in}{1.894832in}}%
\pgfpathlineto{\pgfqpoint{3.514023in}{1.895374in}}%
\pgfpathlineto{\pgfqpoint{3.525266in}{1.895292in}}%
\pgfpathlineto{\pgfqpoint{3.533698in}{1.895452in}}%
\pgfpathlineto{\pgfqpoint{3.539319in}{1.895143in}}%
\pgfpathlineto{\pgfqpoint{3.547751in}{1.894751in}}%
\pgfpathlineto{\pgfqpoint{3.595533in}{1.897253in}}%
\pgfpathlineto{\pgfqpoint{3.601154in}{1.897134in}}%
\pgfpathlineto{\pgfqpoint{3.606776in}{1.897819in}}%
\pgfpathlineto{\pgfqpoint{3.623640in}{1.897416in}}%
\pgfpathlineto{\pgfqpoint{3.634882in}{1.897377in}}%
\pgfpathlineto{\pgfqpoint{3.648936in}{1.896536in}}%
\pgfpathlineto{\pgfqpoint{3.657368in}{1.897005in}}%
\pgfpathlineto{\pgfqpoint{3.662989in}{1.897666in}}%
\pgfpathlineto{\pgfqpoint{3.677043in}{1.897678in}}%
\pgfpathlineto{\pgfqpoint{3.682664in}{1.897692in}}%
\pgfpathlineto{\pgfqpoint{3.685475in}{1.897338in}}%
\pgfpathlineto{\pgfqpoint{3.691096in}{1.897737in}}%
\pgfpathlineto{\pgfqpoint{3.705150in}{1.897474in}}%
\pgfpathlineto{\pgfqpoint{3.710771in}{1.897817in}}%
\pgfpathlineto{\pgfqpoint{3.722014in}{1.897546in}}%
\pgfpathlineto{\pgfqpoint{3.736067in}{1.897619in}}%
\pgfpathlineto{\pgfqpoint{3.755742in}{1.896278in}}%
\pgfpathlineto{\pgfqpoint{3.758553in}{1.896555in}}%
\pgfpathlineto{\pgfqpoint{3.761363in}{1.896200in}}%
\pgfpathlineto{\pgfqpoint{3.764174in}{1.896568in}}%
\pgfpathlineto{\pgfqpoint{3.766985in}{1.895161in}}%
\pgfpathlineto{\pgfqpoint{3.806334in}{1.894867in}}%
\pgfpathlineto{\pgfqpoint{3.823198in}{1.894715in}}%
\pgfpathlineto{\pgfqpoint{3.828820in}{1.895022in}}%
\pgfpathlineto{\pgfqpoint{3.831630in}{1.894663in}}%
\pgfpathlineto{\pgfqpoint{3.834441in}{1.894975in}}%
\pgfpathlineto{\pgfqpoint{3.840062in}{1.894891in}}%
\pgfpathlineto{\pgfqpoint{3.854116in}{1.895515in}}%
\pgfpathlineto{\pgfqpoint{3.885033in}{1.896108in}}%
\pgfpathlineto{\pgfqpoint{3.890655in}{1.896256in}}%
\pgfpathlineto{\pgfqpoint{3.966543in}{1.895575in}}%
\pgfpathlineto{\pgfqpoint{3.972165in}{1.895346in}}%
\pgfpathlineto{\pgfqpoint{3.986218in}{1.895075in}}%
\pgfpathlineto{\pgfqpoint{3.994650in}{1.895042in}}%
\pgfpathlineto{\pgfqpoint{4.056485in}{1.895976in}}%
\pgfpathlineto{\pgfqpoint{4.064917in}{1.895802in}}%
\pgfpathlineto{\pgfqpoint{4.076160in}{1.896288in}}%
\pgfpathlineto{\pgfqpoint{4.081782in}{1.895334in}}%
\pgfpathlineto{\pgfqpoint{4.112699in}{1.896998in}}%
\pgfpathlineto{\pgfqpoint{4.126753in}{1.897223in}}%
\pgfpathlineto{\pgfqpoint{4.129563in}{1.896684in}}%
\pgfpathlineto{\pgfqpoint{4.143617in}{1.896845in}}%
\pgfpathlineto{\pgfqpoint{4.225127in}{1.897378in}}%
\pgfpathlineto{\pgfqpoint{4.227937in}{1.897017in}}%
\pgfpathlineto{\pgfqpoint{4.233559in}{1.897067in}}%
\pgfpathlineto{\pgfqpoint{4.239180in}{1.897389in}}%
\pgfpathlineto{\pgfqpoint{4.244801in}{1.897581in}}%
\pgfpathlineto{\pgfqpoint{4.317879in}{1.896724in}}%
\pgfpathlineto{\pgfqpoint{4.343175in}{1.896624in}}%
\pgfpathlineto{\pgfqpoint{4.388146in}{1.896546in}}%
\pgfpathlineto{\pgfqpoint{4.393768in}{1.896348in}}%
\pgfpathlineto{\pgfqpoint{4.441549in}{1.897098in}}%
\pgfpathlineto{\pgfqpoint{4.520249in}{1.896954in}}%
\pgfpathlineto{\pgfqpoint{4.525870in}{1.896605in}}%
\pgfpathlineto{\pgfqpoint{4.542734in}{1.896957in}}%
\pgfpathlineto{\pgfqpoint{4.551166in}{1.896945in}}%
\pgfpathlineto{\pgfqpoint{4.669215in}{1.896858in}}%
\pgfpathlineto{\pgfqpoint{4.674836in}{1.896610in}}%
\pgfpathlineto{\pgfqpoint{4.700133in}{1.896509in}}%
\pgfpathlineto{\pgfqpoint{4.722618in}{1.896525in}}%
\pgfpathlineto{\pgfqpoint{4.745104in}{1.896634in}}%
\pgfpathlineto{\pgfqpoint{4.801317in}{1.895652in}}%
\pgfpathlineto{\pgfqpoint{4.806939in}{1.895698in}}%
\pgfpathlineto{\pgfqpoint{4.837856in}{1.896001in}}%
\pgfpathlineto{\pgfqpoint{4.851910in}{1.896158in}}%
\pgfpathlineto{\pgfqpoint{4.888449in}{1.896333in}}%
\pgfpathlineto{\pgfqpoint{4.899691in}{1.895968in}}%
\pgfpathlineto{\pgfqpoint{4.944662in}{1.896440in}}%
\pgfpathlineto{\pgfqpoint{4.958716in}{1.896623in}}%
\pgfpathlineto{\pgfqpoint{4.978391in}{1.897091in}}%
\pgfpathlineto{\pgfqpoint{4.998065in}{1.897127in}}%
\pgfpathlineto{\pgfqpoint{5.017740in}{1.897383in}}%
\pgfpathlineto{\pgfqpoint{5.037415in}{1.898550in}}%
\pgfpathlineto{\pgfqpoint{5.043036in}{1.898892in}}%
\pgfpathlineto{\pgfqpoint{5.051468in}{1.898865in}}%
\pgfpathlineto{\pgfqpoint{5.130168in}{1.898595in}}%
\pgfpathlineto{\pgfqpoint{5.138600in}{1.898685in}}%
\pgfpathlineto{\pgfqpoint{5.149842in}{1.898548in}}%
\pgfpathlineto{\pgfqpoint{5.149842in}{1.898548in}}%
\pgfusepath{stroke}%
\end{pgfscope}%
\begin{pgfscope}%
\pgfsetrectcap%
\pgfsetmiterjoin%
\pgfsetlinewidth{0.803000pt}%
\definecolor{currentstroke}{rgb}{1.000000,1.000000,1.000000}%
\pgfsetstrokecolor{currentstroke}%
\pgfsetdash{}{0pt}%
\pgfpathmoveto{\pgfqpoint{0.711206in}{0.331635in}}%
\pgfpathlineto{\pgfqpoint{0.711206in}{3.351635in}}%
\pgfusepath{stroke}%
\end{pgfscope}%
\begin{pgfscope}%
\pgfsetrectcap%
\pgfsetmiterjoin%
\pgfsetlinewidth{0.803000pt}%
\definecolor{currentstroke}{rgb}{1.000000,1.000000,1.000000}%
\pgfsetstrokecolor{currentstroke}%
\pgfsetdash{}{0pt}%
\pgfpathmoveto{\pgfqpoint{5.361206in}{0.331635in}}%
\pgfpathlineto{\pgfqpoint{5.361206in}{3.351635in}}%
\pgfusepath{stroke}%
\end{pgfscope}%
\begin{pgfscope}%
\pgfsetrectcap%
\pgfsetmiterjoin%
\pgfsetlinewidth{0.803000pt}%
\definecolor{currentstroke}{rgb}{1.000000,1.000000,1.000000}%
\pgfsetstrokecolor{currentstroke}%
\pgfsetdash{}{0pt}%
\pgfpathmoveto{\pgfqpoint{0.711206in}{0.331635in}}%
\pgfpathlineto{\pgfqpoint{5.361206in}{0.331635in}}%
\pgfusepath{stroke}%
\end{pgfscope}%
\begin{pgfscope}%
\pgfsetrectcap%
\pgfsetmiterjoin%
\pgfsetlinewidth{0.803000pt}%
\definecolor{currentstroke}{rgb}{1.000000,1.000000,1.000000}%
\pgfsetstrokecolor{currentstroke}%
\pgfsetdash{}{0pt}%
\pgfpathmoveto{\pgfqpoint{0.711206in}{3.351635in}}%
\pgfpathlineto{\pgfqpoint{5.361206in}{3.351635in}}%
\pgfusepath{stroke}%
\end{pgfscope}%
\end{pgfpicture}%
\makeatother%
\endgroup%

    \end{adjustbox}
    \caption{The figure shows predictions for log-returns plotted (orange) with confidence intervals (red) against the real returns. The left plot shows the the predictions generated by ARMA(0,0) for V, the middle predictions from ARMA(1,1) and the right plot shows predictions for INTC generated by the ARMA(0,0) model.}
    \label{fig:V_INTC_ARMA_predictions_plot}
\end{figure}{}

\subsubsection{\textcolor{red}{Forecasting Volatility}}
While GARCH models are not able to better predict future returns, they can help predict the volatility of future returns. Figure \ref{fig:} gives an intuition to the usefulness of volatility forecasts. In the figure, the volatility of V process is plotted together with the absolute value of the log-returns of the stock. While obviously the scaling does not match and the log-returns themselves do not equal not the actual volatility, this figure serves the point of illustrating how future turbulences in the markets seem at least somewhat predictable. Note that the direction of movement remains still unpredictable, we can merely forecast uncertainty about future returns. 

\begin{figure}[h]
    \centering
    \figuretitle{Volatility Forecasts for V and INTC}
    \begin{adjustbox}{width = 0.95\textwidth}
    %% Creator: Matplotlib, PGF backend
%%
%% To include the figure in your LaTeX document, write
%%   \input{<filename>.pgf}
%%
%% Make sure the required packages are loaded in your preamble
%%   \usepackage{pgf}
%%
%% Figures using additional raster images can only be included by \input if
%% they are in the same directory as the main LaTeX file. For loading figures
%% from other directories you can use the `import` package
%%   \usepackage{import}
%% and then include the figures with
%%   \import{<path to file>}{<filename>.pgf}
%%
%% Matplotlib used the following preamble
%%   \usepackage{fontspec}
%%   \setmainfont{DejaVuSerif.ttf}[Path=/opt/tljh/user/lib/python3.6/site-packages/matplotlib/mpl-data/fonts/ttf/]
%%   \setsansfont{DejaVuSans.ttf}[Path=/opt/tljh/user/lib/python3.6/site-packages/matplotlib/mpl-data/fonts/ttf/]
%%   \setmonofont{DejaVuSansMono.ttf}[Path=/opt/tljh/user/lib/python3.6/site-packages/matplotlib/mpl-data/fonts/ttf/]
%%
\begingroup%
\makeatletter%
\begin{pgfpicture}%
\pgfpathrectangle{\pgfpointorigin}{\pgfqpoint{5.123953in}{3.451635in}}%
\pgfusepath{use as bounding box, clip}%
\begin{pgfscope}%
\pgfsetbuttcap%
\pgfsetmiterjoin%
\definecolor{currentfill}{rgb}{1.000000,1.000000,1.000000}%
\pgfsetfillcolor{currentfill}%
\pgfsetlinewidth{0.000000pt}%
\definecolor{currentstroke}{rgb}{1.000000,1.000000,1.000000}%
\pgfsetstrokecolor{currentstroke}%
\pgfsetdash{}{0pt}%
\pgfpathmoveto{\pgfqpoint{0.000000in}{0.000000in}}%
\pgfpathlineto{\pgfqpoint{5.123953in}{0.000000in}}%
\pgfpathlineto{\pgfqpoint{5.123953in}{3.451635in}}%
\pgfpathlineto{\pgfqpoint{0.000000in}{3.451635in}}%
\pgfpathclose%
\pgfusepath{fill}%
\end{pgfscope}%
\begin{pgfscope}%
\pgfsetbuttcap%
\pgfsetmiterjoin%
\definecolor{currentfill}{rgb}{0.917647,0.917647,0.949020}%
\pgfsetfillcolor{currentfill}%
\pgfsetlinewidth{0.000000pt}%
\definecolor{currentstroke}{rgb}{0.000000,0.000000,0.000000}%
\pgfsetstrokecolor{currentstroke}%
\pgfsetstrokeopacity{0.000000}%
\pgfsetdash{}{0pt}%
\pgfpathmoveto{\pgfqpoint{0.373953in}{0.331635in}}%
\pgfpathlineto{\pgfqpoint{5.023953in}{0.331635in}}%
\pgfpathlineto{\pgfqpoint{5.023953in}{3.351635in}}%
\pgfpathlineto{\pgfqpoint{0.373953in}{3.351635in}}%
\pgfpathclose%
\pgfusepath{fill}%
\end{pgfscope}%
\begin{pgfscope}%
\pgfpathrectangle{\pgfqpoint{0.373953in}{0.331635in}}{\pgfqpoint{4.650000in}{3.020000in}}%
\pgfusepath{clip}%
\pgfsetroundcap%
\pgfsetroundjoin%
\pgfsetlinewidth{0.803000pt}%
\definecolor{currentstroke}{rgb}{1.000000,1.000000,1.000000}%
\pgfsetstrokecolor{currentstroke}%
\pgfsetdash{}{0pt}%
\pgfpathmoveto{\pgfqpoint{0.585317in}{0.331635in}}%
\pgfpathlineto{\pgfqpoint{0.585317in}{3.351635in}}%
\pgfusepath{stroke}%
\end{pgfscope}%
\begin{pgfscope}%
\definecolor{textcolor}{rgb}{0.150000,0.150000,0.150000}%
\pgfsetstrokecolor{textcolor}%
\pgfsetfillcolor{textcolor}%
\pgftext[x=0.585317in,y=0.234413in,,top]{\color{textcolor}\rmfamily\fontsize{10.000000}{12.000000}\selectfont 0}%
\end{pgfscope}%
\begin{pgfscope}%
\pgfpathrectangle{\pgfqpoint{0.373953in}{0.331635in}}{\pgfqpoint{4.650000in}{3.020000in}}%
\pgfusepath{clip}%
\pgfsetroundcap%
\pgfsetroundjoin%
\pgfsetlinewidth{0.803000pt}%
\definecolor{currentstroke}{rgb}{1.000000,1.000000,1.000000}%
\pgfsetstrokecolor{currentstroke}%
\pgfsetdash{}{0pt}%
\pgfpathmoveto{\pgfqpoint{1.145963in}{0.331635in}}%
\pgfpathlineto{\pgfqpoint{1.145963in}{3.351635in}}%
\pgfusepath{stroke}%
\end{pgfscope}%
\begin{pgfscope}%
\definecolor{textcolor}{rgb}{0.150000,0.150000,0.150000}%
\pgfsetstrokecolor{textcolor}%
\pgfsetfillcolor{textcolor}%
\pgftext[x=1.145963in,y=0.234413in,,top]{\color{textcolor}\rmfamily\fontsize{10.000000}{12.000000}\selectfont 200}%
\end{pgfscope}%
\begin{pgfscope}%
\pgfpathrectangle{\pgfqpoint{0.373953in}{0.331635in}}{\pgfqpoint{4.650000in}{3.020000in}}%
\pgfusepath{clip}%
\pgfsetroundcap%
\pgfsetroundjoin%
\pgfsetlinewidth{0.803000pt}%
\definecolor{currentstroke}{rgb}{1.000000,1.000000,1.000000}%
\pgfsetstrokecolor{currentstroke}%
\pgfsetdash{}{0pt}%
\pgfpathmoveto{\pgfqpoint{1.706609in}{0.331635in}}%
\pgfpathlineto{\pgfqpoint{1.706609in}{3.351635in}}%
\pgfusepath{stroke}%
\end{pgfscope}%
\begin{pgfscope}%
\definecolor{textcolor}{rgb}{0.150000,0.150000,0.150000}%
\pgfsetstrokecolor{textcolor}%
\pgfsetfillcolor{textcolor}%
\pgftext[x=1.706609in,y=0.234413in,,top]{\color{textcolor}\rmfamily\fontsize{10.000000}{12.000000}\selectfont 400}%
\end{pgfscope}%
\begin{pgfscope}%
\pgfpathrectangle{\pgfqpoint{0.373953in}{0.331635in}}{\pgfqpoint{4.650000in}{3.020000in}}%
\pgfusepath{clip}%
\pgfsetroundcap%
\pgfsetroundjoin%
\pgfsetlinewidth{0.803000pt}%
\definecolor{currentstroke}{rgb}{1.000000,1.000000,1.000000}%
\pgfsetstrokecolor{currentstroke}%
\pgfsetdash{}{0pt}%
\pgfpathmoveto{\pgfqpoint{2.267255in}{0.331635in}}%
\pgfpathlineto{\pgfqpoint{2.267255in}{3.351635in}}%
\pgfusepath{stroke}%
\end{pgfscope}%
\begin{pgfscope}%
\definecolor{textcolor}{rgb}{0.150000,0.150000,0.150000}%
\pgfsetstrokecolor{textcolor}%
\pgfsetfillcolor{textcolor}%
\pgftext[x=2.267255in,y=0.234413in,,top]{\color{textcolor}\rmfamily\fontsize{10.000000}{12.000000}\selectfont 600}%
\end{pgfscope}%
\begin{pgfscope}%
\pgfpathrectangle{\pgfqpoint{0.373953in}{0.331635in}}{\pgfqpoint{4.650000in}{3.020000in}}%
\pgfusepath{clip}%
\pgfsetroundcap%
\pgfsetroundjoin%
\pgfsetlinewidth{0.803000pt}%
\definecolor{currentstroke}{rgb}{1.000000,1.000000,1.000000}%
\pgfsetstrokecolor{currentstroke}%
\pgfsetdash{}{0pt}%
\pgfpathmoveto{\pgfqpoint{2.827902in}{0.331635in}}%
\pgfpathlineto{\pgfqpoint{2.827902in}{3.351635in}}%
\pgfusepath{stroke}%
\end{pgfscope}%
\begin{pgfscope}%
\definecolor{textcolor}{rgb}{0.150000,0.150000,0.150000}%
\pgfsetstrokecolor{textcolor}%
\pgfsetfillcolor{textcolor}%
\pgftext[x=2.827902in,y=0.234413in,,top]{\color{textcolor}\rmfamily\fontsize{10.000000}{12.000000}\selectfont 800}%
\end{pgfscope}%
\begin{pgfscope}%
\pgfpathrectangle{\pgfqpoint{0.373953in}{0.331635in}}{\pgfqpoint{4.650000in}{3.020000in}}%
\pgfusepath{clip}%
\pgfsetroundcap%
\pgfsetroundjoin%
\pgfsetlinewidth{0.803000pt}%
\definecolor{currentstroke}{rgb}{1.000000,1.000000,1.000000}%
\pgfsetstrokecolor{currentstroke}%
\pgfsetdash{}{0pt}%
\pgfpathmoveto{\pgfqpoint{3.388548in}{0.331635in}}%
\pgfpathlineto{\pgfqpoint{3.388548in}{3.351635in}}%
\pgfusepath{stroke}%
\end{pgfscope}%
\begin{pgfscope}%
\definecolor{textcolor}{rgb}{0.150000,0.150000,0.150000}%
\pgfsetstrokecolor{textcolor}%
\pgfsetfillcolor{textcolor}%
\pgftext[x=3.388548in,y=0.234413in,,top]{\color{textcolor}\rmfamily\fontsize{10.000000}{12.000000}\selectfont 1000}%
\end{pgfscope}%
\begin{pgfscope}%
\pgfpathrectangle{\pgfqpoint{0.373953in}{0.331635in}}{\pgfqpoint{4.650000in}{3.020000in}}%
\pgfusepath{clip}%
\pgfsetroundcap%
\pgfsetroundjoin%
\pgfsetlinewidth{0.803000pt}%
\definecolor{currentstroke}{rgb}{1.000000,1.000000,1.000000}%
\pgfsetstrokecolor{currentstroke}%
\pgfsetdash{}{0pt}%
\pgfpathmoveto{\pgfqpoint{3.949194in}{0.331635in}}%
\pgfpathlineto{\pgfqpoint{3.949194in}{3.351635in}}%
\pgfusepath{stroke}%
\end{pgfscope}%
\begin{pgfscope}%
\definecolor{textcolor}{rgb}{0.150000,0.150000,0.150000}%
\pgfsetstrokecolor{textcolor}%
\pgfsetfillcolor{textcolor}%
\pgftext[x=3.949194in,y=0.234413in,,top]{\color{textcolor}\rmfamily\fontsize{10.000000}{12.000000}\selectfont 1200}%
\end{pgfscope}%
\begin{pgfscope}%
\pgfpathrectangle{\pgfqpoint{0.373953in}{0.331635in}}{\pgfqpoint{4.650000in}{3.020000in}}%
\pgfusepath{clip}%
\pgfsetroundcap%
\pgfsetroundjoin%
\pgfsetlinewidth{0.803000pt}%
\definecolor{currentstroke}{rgb}{1.000000,1.000000,1.000000}%
\pgfsetstrokecolor{currentstroke}%
\pgfsetdash{}{0pt}%
\pgfpathmoveto{\pgfqpoint{4.509840in}{0.331635in}}%
\pgfpathlineto{\pgfqpoint{4.509840in}{3.351635in}}%
\pgfusepath{stroke}%
\end{pgfscope}%
\begin{pgfscope}%
\definecolor{textcolor}{rgb}{0.150000,0.150000,0.150000}%
\pgfsetstrokecolor{textcolor}%
\pgfsetfillcolor{textcolor}%
\pgftext[x=4.509840in,y=0.234413in,,top]{\color{textcolor}\rmfamily\fontsize{10.000000}{12.000000}\selectfont 1400}%
\end{pgfscope}%
\begin{pgfscope}%
\pgfpathrectangle{\pgfqpoint{0.373953in}{0.331635in}}{\pgfqpoint{4.650000in}{3.020000in}}%
\pgfusepath{clip}%
\pgfsetroundcap%
\pgfsetroundjoin%
\pgfsetlinewidth{0.803000pt}%
\definecolor{currentstroke}{rgb}{1.000000,1.000000,1.000000}%
\pgfsetstrokecolor{currentstroke}%
\pgfsetdash{}{0pt}%
\pgfpathmoveto{\pgfqpoint{0.373953in}{0.468908in}}%
\pgfpathlineto{\pgfqpoint{5.023953in}{0.468908in}}%
\pgfusepath{stroke}%
\end{pgfscope}%
\begin{pgfscope}%
\definecolor{textcolor}{rgb}{0.150000,0.150000,0.150000}%
\pgfsetstrokecolor{textcolor}%
\pgfsetfillcolor{textcolor}%
\pgftext[x=0.188365in,y=0.416146in,left,base]{\color{textcolor}\rmfamily\fontsize{10.000000}{12.000000}\selectfont 0}%
\end{pgfscope}%
\begin{pgfscope}%
\pgfpathrectangle{\pgfqpoint{0.373953in}{0.331635in}}{\pgfqpoint{4.650000in}{3.020000in}}%
\pgfusepath{clip}%
\pgfsetroundcap%
\pgfsetroundjoin%
\pgfsetlinewidth{0.803000pt}%
\definecolor{currentstroke}{rgb}{1.000000,1.000000,1.000000}%
\pgfsetstrokecolor{currentstroke}%
\pgfsetdash{}{0pt}%
\pgfpathmoveto{\pgfqpoint{0.373953in}{1.088062in}}%
\pgfpathlineto{\pgfqpoint{5.023953in}{1.088062in}}%
\pgfusepath{stroke}%
\end{pgfscope}%
\begin{pgfscope}%
\definecolor{textcolor}{rgb}{0.150000,0.150000,0.150000}%
\pgfsetstrokecolor{textcolor}%
\pgfsetfillcolor{textcolor}%
\pgftext[x=0.188365in,y=1.035301in,left,base]{\color{textcolor}\rmfamily\fontsize{10.000000}{12.000000}\selectfont 5}%
\end{pgfscope}%
\begin{pgfscope}%
\pgfpathrectangle{\pgfqpoint{0.373953in}{0.331635in}}{\pgfqpoint{4.650000in}{3.020000in}}%
\pgfusepath{clip}%
\pgfsetroundcap%
\pgfsetroundjoin%
\pgfsetlinewidth{0.803000pt}%
\definecolor{currentstroke}{rgb}{1.000000,1.000000,1.000000}%
\pgfsetstrokecolor{currentstroke}%
\pgfsetdash{}{0pt}%
\pgfpathmoveto{\pgfqpoint{0.373953in}{1.707217in}}%
\pgfpathlineto{\pgfqpoint{5.023953in}{1.707217in}}%
\pgfusepath{stroke}%
\end{pgfscope}%
\begin{pgfscope}%
\definecolor{textcolor}{rgb}{0.150000,0.150000,0.150000}%
\pgfsetstrokecolor{textcolor}%
\pgfsetfillcolor{textcolor}%
\pgftext[x=0.100000in,y=1.654456in,left,base]{\color{textcolor}\rmfamily\fontsize{10.000000}{12.000000}\selectfont 10}%
\end{pgfscope}%
\begin{pgfscope}%
\pgfpathrectangle{\pgfqpoint{0.373953in}{0.331635in}}{\pgfqpoint{4.650000in}{3.020000in}}%
\pgfusepath{clip}%
\pgfsetroundcap%
\pgfsetroundjoin%
\pgfsetlinewidth{0.803000pt}%
\definecolor{currentstroke}{rgb}{1.000000,1.000000,1.000000}%
\pgfsetstrokecolor{currentstroke}%
\pgfsetdash{}{0pt}%
\pgfpathmoveto{\pgfqpoint{0.373953in}{2.326372in}}%
\pgfpathlineto{\pgfqpoint{5.023953in}{2.326372in}}%
\pgfusepath{stroke}%
\end{pgfscope}%
\begin{pgfscope}%
\definecolor{textcolor}{rgb}{0.150000,0.150000,0.150000}%
\pgfsetstrokecolor{textcolor}%
\pgfsetfillcolor{textcolor}%
\pgftext[x=0.100000in,y=2.273610in,left,base]{\color{textcolor}\rmfamily\fontsize{10.000000}{12.000000}\selectfont 15}%
\end{pgfscope}%
\begin{pgfscope}%
\pgfpathrectangle{\pgfqpoint{0.373953in}{0.331635in}}{\pgfqpoint{4.650000in}{3.020000in}}%
\pgfusepath{clip}%
\pgfsetroundcap%
\pgfsetroundjoin%
\pgfsetlinewidth{0.803000pt}%
\definecolor{currentstroke}{rgb}{1.000000,1.000000,1.000000}%
\pgfsetstrokecolor{currentstroke}%
\pgfsetdash{}{0pt}%
\pgfpathmoveto{\pgfqpoint{0.373953in}{2.945526in}}%
\pgfpathlineto{\pgfqpoint{5.023953in}{2.945526in}}%
\pgfusepath{stroke}%
\end{pgfscope}%
\begin{pgfscope}%
\definecolor{textcolor}{rgb}{0.150000,0.150000,0.150000}%
\pgfsetstrokecolor{textcolor}%
\pgfsetfillcolor{textcolor}%
\pgftext[x=0.100000in,y=2.892765in,left,base]{\color{textcolor}\rmfamily\fontsize{10.000000}{12.000000}\selectfont 20}%
\end{pgfscope}%
\begin{pgfscope}%
\pgfpathrectangle{\pgfqpoint{0.373953in}{0.331635in}}{\pgfqpoint{4.650000in}{3.020000in}}%
\pgfusepath{clip}%
\pgfsetroundcap%
\pgfsetroundjoin%
\pgfsetlinewidth{1.505625pt}%
\definecolor{currentstroke}{rgb}{0.121569,0.466667,0.705882}%
\pgfsetstrokecolor{currentstroke}%
\pgfsetdash{}{0pt}%
\pgfpathmoveto{\pgfqpoint{0.585317in}{0.468943in}}%
\pgfpathlineto{\pgfqpoint{0.588120in}{0.501815in}}%
\pgfpathlineto{\pgfqpoint{0.590923in}{0.469433in}}%
\pgfpathlineto{\pgfqpoint{0.593726in}{0.699980in}}%
\pgfpathlineto{\pgfqpoint{0.596529in}{0.641101in}}%
\pgfpathlineto{\pgfqpoint{0.599333in}{0.698700in}}%
\pgfpathlineto{\pgfqpoint{0.602136in}{0.659505in}}%
\pgfpathlineto{\pgfqpoint{0.604939in}{0.696742in}}%
\pgfpathlineto{\pgfqpoint{0.607742in}{0.722689in}}%
\pgfpathlineto{\pgfqpoint{0.610546in}{0.698867in}}%
\pgfpathlineto{\pgfqpoint{0.613349in}{0.709574in}}%
\pgfpathlineto{\pgfqpoint{0.616152in}{0.663237in}}%
\pgfpathlineto{\pgfqpoint{0.621759in}{0.644651in}}%
\pgfpathlineto{\pgfqpoint{0.624562in}{0.640224in}}%
\pgfpathlineto{\pgfqpoint{0.627365in}{0.606889in}}%
\pgfpathlineto{\pgfqpoint{0.630168in}{0.649048in}}%
\pgfpathlineto{\pgfqpoint{0.632971in}{1.821315in}}%
\pgfpathlineto{\pgfqpoint{0.635775in}{0.860193in}}%
\pgfpathlineto{\pgfqpoint{0.638578in}{0.655059in}}%
\pgfpathlineto{\pgfqpoint{0.641381in}{0.635871in}}%
\pgfpathlineto{\pgfqpoint{0.644184in}{0.706195in}}%
\pgfpathlineto{\pgfqpoint{0.646988in}{2.146639in}}%
\pgfpathlineto{\pgfqpoint{0.649791in}{0.731104in}}%
\pgfpathlineto{\pgfqpoint{0.652594in}{0.724249in}}%
\pgfpathlineto{\pgfqpoint{0.655397in}{0.730127in}}%
\pgfpathlineto{\pgfqpoint{0.658201in}{0.721431in}}%
\pgfpathlineto{\pgfqpoint{0.661004in}{0.719914in}}%
\pgfpathlineto{\pgfqpoint{0.663807in}{0.720630in}}%
\pgfpathlineto{\pgfqpoint{0.666610in}{0.709180in}}%
\pgfpathlineto{\pgfqpoint{0.669413in}{0.714596in}}%
\pgfpathlineto{\pgfqpoint{0.672217in}{0.704890in}}%
\pgfpathlineto{\pgfqpoint{0.675020in}{0.689882in}}%
\pgfpathlineto{\pgfqpoint{0.677823in}{0.697381in}}%
\pgfpathlineto{\pgfqpoint{0.680626in}{0.697471in}}%
\pgfpathlineto{\pgfqpoint{0.683430in}{0.714978in}}%
\pgfpathlineto{\pgfqpoint{0.686233in}{0.708982in}}%
\pgfpathlineto{\pgfqpoint{0.689036in}{0.706468in}}%
\pgfpathlineto{\pgfqpoint{0.691839in}{0.701672in}}%
\pgfpathlineto{\pgfqpoint{0.694643in}{0.701941in}}%
\pgfpathlineto{\pgfqpoint{0.697446in}{0.701589in}}%
\pgfpathlineto{\pgfqpoint{0.700249in}{0.704364in}}%
\pgfpathlineto{\pgfqpoint{0.703052in}{0.700406in}}%
\pgfpathlineto{\pgfqpoint{0.705855in}{0.695123in}}%
\pgfpathlineto{\pgfqpoint{0.708659in}{0.687297in}}%
\pgfpathlineto{\pgfqpoint{0.711462in}{0.685628in}}%
\pgfpathlineto{\pgfqpoint{0.714265in}{0.680976in}}%
\pgfpathlineto{\pgfqpoint{0.717068in}{0.644577in}}%
\pgfpathlineto{\pgfqpoint{0.719872in}{0.675684in}}%
\pgfpathlineto{\pgfqpoint{0.722675in}{0.644493in}}%
\pgfpathlineto{\pgfqpoint{0.725478in}{0.636454in}}%
\pgfpathlineto{\pgfqpoint{0.728281in}{0.634143in}}%
\pgfpathlineto{\pgfqpoint{0.731085in}{0.711410in}}%
\pgfpathlineto{\pgfqpoint{0.733888in}{0.677864in}}%
\pgfpathlineto{\pgfqpoint{0.736691in}{0.628152in}}%
\pgfpathlineto{\pgfqpoint{0.742298in}{0.619891in}}%
\pgfpathlineto{\pgfqpoint{0.750707in}{0.604582in}}%
\pgfpathlineto{\pgfqpoint{0.756314in}{0.615075in}}%
\pgfpathlineto{\pgfqpoint{0.759117in}{0.620047in}}%
\pgfpathlineto{\pgfqpoint{0.761920in}{0.639716in}}%
\pgfpathlineto{\pgfqpoint{0.764723in}{0.642445in}}%
\pgfpathlineto{\pgfqpoint{0.767527in}{1.023832in}}%
\pgfpathlineto{\pgfqpoint{0.770330in}{0.735072in}}%
\pgfpathlineto{\pgfqpoint{0.773133in}{0.657953in}}%
\pgfpathlineto{\pgfqpoint{0.775936in}{0.684228in}}%
\pgfpathlineto{\pgfqpoint{0.778740in}{0.651971in}}%
\pgfpathlineto{\pgfqpoint{0.781543in}{0.648918in}}%
\pgfpathlineto{\pgfqpoint{0.784346in}{0.644603in}}%
\pgfpathlineto{\pgfqpoint{0.787149in}{0.670271in}}%
\pgfpathlineto{\pgfqpoint{0.789952in}{0.667905in}}%
\pgfpathlineto{\pgfqpoint{0.792756in}{0.829869in}}%
\pgfpathlineto{\pgfqpoint{0.795559in}{0.689715in}}%
\pgfpathlineto{\pgfqpoint{0.798362in}{0.669199in}}%
\pgfpathlineto{\pgfqpoint{0.803969in}{0.660223in}}%
\pgfpathlineto{\pgfqpoint{0.806772in}{0.658198in}}%
\pgfpathlineto{\pgfqpoint{0.809575in}{0.707535in}}%
\pgfpathlineto{\pgfqpoint{0.812378in}{0.715060in}}%
\pgfpathlineto{\pgfqpoint{0.815182in}{0.705532in}}%
\pgfpathlineto{\pgfqpoint{0.817985in}{0.699883in}}%
\pgfpathlineto{\pgfqpoint{0.820788in}{0.696584in}}%
\pgfpathlineto{\pgfqpoint{0.823591in}{0.695752in}}%
\pgfpathlineto{\pgfqpoint{0.826394in}{0.688954in}}%
\pgfpathlineto{\pgfqpoint{0.829198in}{0.688258in}}%
\pgfpathlineto{\pgfqpoint{0.832001in}{0.684043in}}%
\pgfpathlineto{\pgfqpoint{0.834804in}{0.718726in}}%
\pgfpathlineto{\pgfqpoint{0.837607in}{0.684498in}}%
\pgfpathlineto{\pgfqpoint{0.840411in}{0.692294in}}%
\pgfpathlineto{\pgfqpoint{0.843214in}{0.809547in}}%
\pgfpathlineto{\pgfqpoint{0.846017in}{0.745022in}}%
\pgfpathlineto{\pgfqpoint{0.848820in}{0.709393in}}%
\pgfpathlineto{\pgfqpoint{0.854427in}{0.700527in}}%
\pgfpathlineto{\pgfqpoint{0.860033in}{0.704335in}}%
\pgfpathlineto{\pgfqpoint{0.862836in}{0.729886in}}%
\pgfpathlineto{\pgfqpoint{0.865640in}{0.847842in}}%
\pgfpathlineto{\pgfqpoint{0.868443in}{0.729720in}}%
\pgfpathlineto{\pgfqpoint{0.871246in}{0.694015in}}%
\pgfpathlineto{\pgfqpoint{0.874049in}{0.741111in}}%
\pgfpathlineto{\pgfqpoint{0.876853in}{0.704305in}}%
\pgfpathlineto{\pgfqpoint{0.882459in}{0.693257in}}%
\pgfpathlineto{\pgfqpoint{0.885262in}{0.690507in}}%
\pgfpathlineto{\pgfqpoint{0.888066in}{0.777760in}}%
\pgfpathlineto{\pgfqpoint{0.890869in}{0.707394in}}%
\pgfpathlineto{\pgfqpoint{0.893672in}{0.712266in}}%
\pgfpathlineto{\pgfqpoint{0.896475in}{0.707939in}}%
\pgfpathlineto{\pgfqpoint{0.899278in}{0.701262in}}%
\pgfpathlineto{\pgfqpoint{0.902082in}{0.698881in}}%
\pgfpathlineto{\pgfqpoint{0.904885in}{0.798308in}}%
\pgfpathlineto{\pgfqpoint{0.907688in}{0.868354in}}%
\pgfpathlineto{\pgfqpoint{0.910491in}{1.131930in}}%
\pgfpathlineto{\pgfqpoint{0.913295in}{0.727550in}}%
\pgfpathlineto{\pgfqpoint{0.916098in}{0.678141in}}%
\pgfpathlineto{\pgfqpoint{0.918901in}{0.796568in}}%
\pgfpathlineto{\pgfqpoint{0.921704in}{0.735478in}}%
\pgfpathlineto{\pgfqpoint{0.924508in}{0.799693in}}%
\pgfpathlineto{\pgfqpoint{0.927311in}{0.688671in}}%
\pgfpathlineto{\pgfqpoint{0.930114in}{0.683810in}}%
\pgfpathlineto{\pgfqpoint{0.932917in}{0.775689in}}%
\pgfpathlineto{\pgfqpoint{0.935720in}{0.764139in}}%
\pgfpathlineto{\pgfqpoint{0.938524in}{0.740253in}}%
\pgfpathlineto{\pgfqpoint{0.941327in}{0.761369in}}%
\pgfpathlineto{\pgfqpoint{0.944130in}{0.725182in}}%
\pgfpathlineto{\pgfqpoint{0.946933in}{0.702095in}}%
\pgfpathlineto{\pgfqpoint{0.949737in}{0.790289in}}%
\pgfpathlineto{\pgfqpoint{0.952540in}{0.695167in}}%
\pgfpathlineto{\pgfqpoint{0.955343in}{0.691205in}}%
\pgfpathlineto{\pgfqpoint{0.958146in}{0.846721in}}%
\pgfpathlineto{\pgfqpoint{0.960950in}{0.688680in}}%
\pgfpathlineto{\pgfqpoint{0.966556in}{0.774918in}}%
\pgfpathlineto{\pgfqpoint{0.969359in}{0.693709in}}%
\pgfpathlineto{\pgfqpoint{0.972162in}{0.873212in}}%
\pgfpathlineto{\pgfqpoint{0.974966in}{0.743722in}}%
\pgfpathlineto{\pgfqpoint{0.977769in}{0.721129in}}%
\pgfpathlineto{\pgfqpoint{0.980572in}{0.770369in}}%
\pgfpathlineto{\pgfqpoint{0.983375in}{0.756346in}}%
\pgfpathlineto{\pgfqpoint{0.986179in}{0.723902in}}%
\pgfpathlineto{\pgfqpoint{0.988982in}{0.720227in}}%
\pgfpathlineto{\pgfqpoint{0.991785in}{0.700535in}}%
\pgfpathlineto{\pgfqpoint{0.994588in}{0.708934in}}%
\pgfpathlineto{\pgfqpoint{0.997392in}{0.697984in}}%
\pgfpathlineto{\pgfqpoint{1.000195in}{0.838239in}}%
\pgfpathlineto{\pgfqpoint{1.002998in}{0.701656in}}%
\pgfpathlineto{\pgfqpoint{1.005801in}{0.711713in}}%
\pgfpathlineto{\pgfqpoint{1.008604in}{0.713547in}}%
\pgfpathlineto{\pgfqpoint{1.011408in}{0.695329in}}%
\pgfpathlineto{\pgfqpoint{1.014211in}{0.694204in}}%
\pgfpathlineto{\pgfqpoint{1.017014in}{0.715125in}}%
\pgfpathlineto{\pgfqpoint{1.019817in}{0.696060in}}%
\pgfpathlineto{\pgfqpoint{1.022621in}{0.698620in}}%
\pgfpathlineto{\pgfqpoint{1.025424in}{0.686245in}}%
\pgfpathlineto{\pgfqpoint{1.028227in}{0.704435in}}%
\pgfpathlineto{\pgfqpoint{1.031030in}{0.687878in}}%
\pgfpathlineto{\pgfqpoint{1.033834in}{0.686725in}}%
\pgfpathlineto{\pgfqpoint{1.036637in}{0.678694in}}%
\pgfpathlineto{\pgfqpoint{1.039440in}{0.676106in}}%
\pgfpathlineto{\pgfqpoint{1.042243in}{0.714987in}}%
\pgfpathlineto{\pgfqpoint{1.045046in}{0.696979in}}%
\pgfpathlineto{\pgfqpoint{1.047850in}{0.671334in}}%
\pgfpathlineto{\pgfqpoint{1.050653in}{0.693190in}}%
\pgfpathlineto{\pgfqpoint{1.053456in}{0.706901in}}%
\pgfpathlineto{\pgfqpoint{1.056259in}{0.668917in}}%
\pgfpathlineto{\pgfqpoint{1.059063in}{0.695777in}}%
\pgfpathlineto{\pgfqpoint{1.061866in}{0.745083in}}%
\pgfpathlineto{\pgfqpoint{1.064669in}{0.703224in}}%
\pgfpathlineto{\pgfqpoint{1.067472in}{0.687804in}}%
\pgfpathlineto{\pgfqpoint{1.070276in}{0.683660in}}%
\pgfpathlineto{\pgfqpoint{1.073079in}{0.672997in}}%
\pgfpathlineto{\pgfqpoint{1.075882in}{0.666637in}}%
\pgfpathlineto{\pgfqpoint{1.078685in}{0.672783in}}%
\pgfpathlineto{\pgfqpoint{1.081488in}{0.663230in}}%
\pgfpathlineto{\pgfqpoint{1.084292in}{0.659171in}}%
\pgfpathlineto{\pgfqpoint{1.087095in}{0.702554in}}%
\pgfpathlineto{\pgfqpoint{1.089898in}{0.662436in}}%
\pgfpathlineto{\pgfqpoint{1.092701in}{0.761646in}}%
\pgfpathlineto{\pgfqpoint{1.095505in}{0.676685in}}%
\pgfpathlineto{\pgfqpoint{1.098308in}{0.657675in}}%
\pgfpathlineto{\pgfqpoint{1.101111in}{0.712618in}}%
\pgfpathlineto{\pgfqpoint{1.103914in}{0.673873in}}%
\pgfpathlineto{\pgfqpoint{1.106718in}{0.710479in}}%
\pgfpathlineto{\pgfqpoint{1.109521in}{0.669194in}}%
\pgfpathlineto{\pgfqpoint{1.112324in}{0.659172in}}%
\pgfpathlineto{\pgfqpoint{1.115127in}{0.710252in}}%
\pgfpathlineto{\pgfqpoint{1.117930in}{0.747854in}}%
\pgfpathlineto{\pgfqpoint{1.120734in}{0.662605in}}%
\pgfpathlineto{\pgfqpoint{1.123537in}{0.675826in}}%
\pgfpathlineto{\pgfqpoint{1.126340in}{0.656135in}}%
\pgfpathlineto{\pgfqpoint{1.129143in}{0.655874in}}%
\pgfpathlineto{\pgfqpoint{1.131947in}{0.661191in}}%
\pgfpathlineto{\pgfqpoint{1.134750in}{0.618764in}}%
\pgfpathlineto{\pgfqpoint{1.137553in}{0.614895in}}%
\pgfpathlineto{\pgfqpoint{1.140356in}{0.734030in}}%
\pgfpathlineto{\pgfqpoint{1.143160in}{0.633954in}}%
\pgfpathlineto{\pgfqpoint{1.145963in}{0.780412in}}%
\pgfpathlineto{\pgfqpoint{1.148766in}{0.653695in}}%
\pgfpathlineto{\pgfqpoint{1.154372in}{0.638125in}}%
\pgfpathlineto{\pgfqpoint{1.157176in}{0.629143in}}%
\pgfpathlineto{\pgfqpoint{1.159979in}{0.650655in}}%
\pgfpathlineto{\pgfqpoint{1.162782in}{0.646700in}}%
\pgfpathlineto{\pgfqpoint{1.165585in}{0.659170in}}%
\pgfpathlineto{\pgfqpoint{1.168389in}{0.656699in}}%
\pgfpathlineto{\pgfqpoint{1.171192in}{0.659847in}}%
\pgfpathlineto{\pgfqpoint{1.173995in}{0.655656in}}%
\pgfpathlineto{\pgfqpoint{1.179602in}{0.643996in}}%
\pgfpathlineto{\pgfqpoint{1.182405in}{0.642394in}}%
\pgfpathlineto{\pgfqpoint{1.185208in}{0.644440in}}%
\pgfpathlineto{\pgfqpoint{1.188011in}{0.638511in}}%
\pgfpathlineto{\pgfqpoint{1.190815in}{0.638734in}}%
\pgfpathlineto{\pgfqpoint{1.193618in}{0.639843in}}%
\pgfpathlineto{\pgfqpoint{1.202027in}{0.624588in}}%
\pgfpathlineto{\pgfqpoint{1.204831in}{0.624292in}}%
\pgfpathlineto{\pgfqpoint{1.207634in}{0.620399in}}%
\pgfpathlineto{\pgfqpoint{1.213240in}{0.608865in}}%
\pgfpathlineto{\pgfqpoint{1.216044in}{0.605366in}}%
\pgfpathlineto{\pgfqpoint{1.221650in}{0.609922in}}%
\pgfpathlineto{\pgfqpoint{1.230060in}{0.590743in}}%
\pgfpathlineto{\pgfqpoint{1.235666in}{0.575598in}}%
\pgfpathlineto{\pgfqpoint{1.238469in}{0.578583in}}%
\pgfpathlineto{\pgfqpoint{1.241273in}{0.576284in}}%
\pgfpathlineto{\pgfqpoint{1.244076in}{0.569601in}}%
\pgfpathlineto{\pgfqpoint{1.246879in}{0.575977in}}%
\pgfpathlineto{\pgfqpoint{1.249682in}{0.573914in}}%
\pgfpathlineto{\pgfqpoint{1.252486in}{0.586733in}}%
\pgfpathlineto{\pgfqpoint{1.255289in}{0.594081in}}%
\pgfpathlineto{\pgfqpoint{1.258092in}{0.606581in}}%
\pgfpathlineto{\pgfqpoint{1.260895in}{0.600681in}}%
\pgfpathlineto{\pgfqpoint{1.263699in}{0.607948in}}%
\pgfpathlineto{\pgfqpoint{1.266502in}{0.603487in}}%
\pgfpathlineto{\pgfqpoint{1.269305in}{0.602052in}}%
\pgfpathlineto{\pgfqpoint{1.272108in}{0.604475in}}%
\pgfpathlineto{\pgfqpoint{1.274911in}{0.610293in}}%
\pgfpathlineto{\pgfqpoint{1.280518in}{0.600752in}}%
\pgfpathlineto{\pgfqpoint{1.286124in}{0.592685in}}%
\pgfpathlineto{\pgfqpoint{1.288928in}{0.592749in}}%
\pgfpathlineto{\pgfqpoint{1.291731in}{0.598182in}}%
\pgfpathlineto{\pgfqpoint{1.294534in}{0.593084in}}%
\pgfpathlineto{\pgfqpoint{1.297337in}{0.590475in}}%
\pgfpathlineto{\pgfqpoint{1.305747in}{0.578288in}}%
\pgfpathlineto{\pgfqpoint{1.308550in}{0.590140in}}%
\pgfpathlineto{\pgfqpoint{1.319763in}{0.568995in}}%
\pgfpathlineto{\pgfqpoint{1.322566in}{0.626384in}}%
\pgfpathlineto{\pgfqpoint{1.325370in}{0.618890in}}%
\pgfpathlineto{\pgfqpoint{1.328173in}{0.627439in}}%
\pgfpathlineto{\pgfqpoint{1.330976in}{0.626886in}}%
\pgfpathlineto{\pgfqpoint{1.333779in}{0.620140in}}%
\pgfpathlineto{\pgfqpoint{1.336583in}{0.632834in}}%
\pgfpathlineto{\pgfqpoint{1.339386in}{0.631005in}}%
\pgfpathlineto{\pgfqpoint{1.342189in}{0.624965in}}%
\pgfpathlineto{\pgfqpoint{1.344992in}{0.684824in}}%
\pgfpathlineto{\pgfqpoint{1.347795in}{0.672943in}}%
\pgfpathlineto{\pgfqpoint{1.350599in}{0.676143in}}%
\pgfpathlineto{\pgfqpoint{1.353402in}{0.665838in}}%
\pgfpathlineto{\pgfqpoint{1.356205in}{0.660468in}}%
\pgfpathlineto{\pgfqpoint{1.359008in}{0.652893in}}%
\pgfpathlineto{\pgfqpoint{1.364615in}{0.641913in}}%
\pgfpathlineto{\pgfqpoint{1.367418in}{0.657041in}}%
\pgfpathlineto{\pgfqpoint{1.373025in}{0.645928in}}%
\pgfpathlineto{\pgfqpoint{1.375828in}{0.692113in}}%
\pgfpathlineto{\pgfqpoint{1.381434in}{0.676014in}}%
\pgfpathlineto{\pgfqpoint{1.384237in}{0.675307in}}%
\pgfpathlineto{\pgfqpoint{1.389844in}{0.662097in}}%
\pgfpathlineto{\pgfqpoint{1.395450in}{0.648806in}}%
\pgfpathlineto{\pgfqpoint{1.398254in}{0.644718in}}%
\pgfpathlineto{\pgfqpoint{1.403860in}{0.633862in}}%
\pgfpathlineto{\pgfqpoint{1.406663in}{0.636933in}}%
\pgfpathlineto{\pgfqpoint{1.412270in}{0.628309in}}%
\pgfpathlineto{\pgfqpoint{1.415073in}{0.639927in}}%
\pgfpathlineto{\pgfqpoint{1.417876in}{0.636074in}}%
\pgfpathlineto{\pgfqpoint{1.420679in}{0.647099in}}%
\pgfpathlineto{\pgfqpoint{1.423483in}{0.644328in}}%
\pgfpathlineto{\pgfqpoint{1.426286in}{0.648404in}}%
\pgfpathlineto{\pgfqpoint{1.429089in}{0.643794in}}%
\pgfpathlineto{\pgfqpoint{1.431892in}{0.641325in}}%
\pgfpathlineto{\pgfqpoint{1.434696in}{0.638035in}}%
\pgfpathlineto{\pgfqpoint{1.440302in}{0.628236in}}%
\pgfpathlineto{\pgfqpoint{1.443105in}{0.645046in}}%
\pgfpathlineto{\pgfqpoint{1.445909in}{0.639507in}}%
\pgfpathlineto{\pgfqpoint{1.448712in}{0.668913in}}%
\pgfpathlineto{\pgfqpoint{1.451515in}{0.662287in}}%
\pgfpathlineto{\pgfqpoint{1.454318in}{0.662832in}}%
\pgfpathlineto{\pgfqpoint{1.457121in}{0.656363in}}%
\pgfpathlineto{\pgfqpoint{1.459925in}{0.659348in}}%
\pgfpathlineto{\pgfqpoint{1.465531in}{0.648089in}}%
\pgfpathlineto{\pgfqpoint{1.468334in}{0.654638in}}%
\pgfpathlineto{\pgfqpoint{1.471138in}{0.716597in}}%
\pgfpathlineto{\pgfqpoint{1.473941in}{0.711578in}}%
\pgfpathlineto{\pgfqpoint{1.476744in}{0.726625in}}%
\pgfpathlineto{\pgfqpoint{1.493563in}{0.677057in}}%
\pgfpathlineto{\pgfqpoint{1.496367in}{0.678339in}}%
\pgfpathlineto{\pgfqpoint{1.501973in}{0.664476in}}%
\pgfpathlineto{\pgfqpoint{1.504776in}{0.678139in}}%
\pgfpathlineto{\pgfqpoint{1.507580in}{0.683210in}}%
\pgfpathlineto{\pgfqpoint{1.532809in}{0.634550in}}%
\pgfpathlineto{\pgfqpoint{1.535612in}{0.643006in}}%
\pgfpathlineto{\pgfqpoint{1.538415in}{0.640659in}}%
\pgfpathlineto{\pgfqpoint{1.541218in}{0.662333in}}%
\pgfpathlineto{\pgfqpoint{1.544022in}{0.656373in}}%
\pgfpathlineto{\pgfqpoint{1.546825in}{0.657559in}}%
\pgfpathlineto{\pgfqpoint{1.549628in}{0.665590in}}%
\pgfpathlineto{\pgfqpoint{1.555235in}{0.654912in}}%
\pgfpathlineto{\pgfqpoint{1.558038in}{0.660971in}}%
\pgfpathlineto{\pgfqpoint{1.560841in}{0.655717in}}%
\pgfpathlineto{\pgfqpoint{1.563644in}{0.675645in}}%
\pgfpathlineto{\pgfqpoint{1.569251in}{0.662608in}}%
\pgfpathlineto{\pgfqpoint{1.572054in}{0.683801in}}%
\pgfpathlineto{\pgfqpoint{1.580464in}{0.663715in}}%
\pgfpathlineto{\pgfqpoint{1.583267in}{0.679161in}}%
\pgfpathlineto{\pgfqpoint{1.588873in}{0.667815in}}%
\pgfpathlineto{\pgfqpoint{1.591677in}{0.667515in}}%
\pgfpathlineto{\pgfqpoint{1.597283in}{0.655485in}}%
\pgfpathlineto{\pgfqpoint{1.600086in}{0.656392in}}%
\pgfpathlineto{\pgfqpoint{1.602889in}{0.700510in}}%
\pgfpathlineto{\pgfqpoint{1.605693in}{0.691773in}}%
\pgfpathlineto{\pgfqpoint{1.608496in}{0.689405in}}%
\pgfpathlineto{\pgfqpoint{1.616906in}{0.669322in}}%
\pgfpathlineto{\pgfqpoint{1.619709in}{0.670141in}}%
\pgfpathlineto{\pgfqpoint{1.628119in}{0.653114in}}%
\pgfpathlineto{\pgfqpoint{1.630922in}{0.649026in}}%
\pgfpathlineto{\pgfqpoint{1.633725in}{0.662957in}}%
\pgfpathlineto{\pgfqpoint{1.636528in}{0.659988in}}%
\pgfpathlineto{\pgfqpoint{1.642135in}{0.651193in}}%
\pgfpathlineto{\pgfqpoint{1.644938in}{0.645962in}}%
\pgfpathlineto{\pgfqpoint{1.647741in}{0.641969in}}%
\pgfpathlineto{\pgfqpoint{1.650544in}{0.639850in}}%
\pgfpathlineto{\pgfqpoint{1.656151in}{0.630281in}}%
\pgfpathlineto{\pgfqpoint{1.658954in}{0.629729in}}%
\pgfpathlineto{\pgfqpoint{1.661757in}{0.625203in}}%
\pgfpathlineto{\pgfqpoint{1.664561in}{0.640200in}}%
\pgfpathlineto{\pgfqpoint{1.667364in}{0.645100in}}%
\pgfpathlineto{\pgfqpoint{1.670167in}{0.888278in}}%
\pgfpathlineto{\pgfqpoint{1.672970in}{0.691975in}}%
\pgfpathlineto{\pgfqpoint{1.675774in}{0.659087in}}%
\pgfpathlineto{\pgfqpoint{1.678577in}{0.636711in}}%
\pgfpathlineto{\pgfqpoint{1.681380in}{3.214362in}}%
\pgfpathlineto{\pgfqpoint{1.684183in}{0.946375in}}%
\pgfpathlineto{\pgfqpoint{1.686986in}{0.843130in}}%
\pgfpathlineto{\pgfqpoint{1.689790in}{0.671330in}}%
\pgfpathlineto{\pgfqpoint{1.692593in}{0.685502in}}%
\pgfpathlineto{\pgfqpoint{1.695396in}{0.683149in}}%
\pgfpathlineto{\pgfqpoint{1.698199in}{0.656420in}}%
\pgfpathlineto{\pgfqpoint{1.701003in}{0.659009in}}%
\pgfpathlineto{\pgfqpoint{1.703806in}{0.644145in}}%
\pgfpathlineto{\pgfqpoint{1.706609in}{0.645891in}}%
\pgfpathlineto{\pgfqpoint{1.709412in}{0.639522in}}%
\pgfpathlineto{\pgfqpoint{1.712216in}{0.894914in}}%
\pgfpathlineto{\pgfqpoint{1.715019in}{0.695593in}}%
\pgfpathlineto{\pgfqpoint{1.717822in}{0.664150in}}%
\pgfpathlineto{\pgfqpoint{1.720625in}{0.689730in}}%
\pgfpathlineto{\pgfqpoint{1.723428in}{0.753523in}}%
\pgfpathlineto{\pgfqpoint{1.726232in}{0.663009in}}%
\pgfpathlineto{\pgfqpoint{1.729035in}{0.647509in}}%
\pgfpathlineto{\pgfqpoint{1.731838in}{0.846318in}}%
\pgfpathlineto{\pgfqpoint{1.734641in}{0.693515in}}%
\pgfpathlineto{\pgfqpoint{1.737445in}{0.658457in}}%
\pgfpathlineto{\pgfqpoint{1.740248in}{0.649286in}}%
\pgfpathlineto{\pgfqpoint{1.745854in}{0.670288in}}%
\pgfpathlineto{\pgfqpoint{1.748658in}{0.661066in}}%
\pgfpathlineto{\pgfqpoint{1.751461in}{0.648468in}}%
\pgfpathlineto{\pgfqpoint{1.754264in}{0.645401in}}%
\pgfpathlineto{\pgfqpoint{1.757067in}{0.655244in}}%
\pgfpathlineto{\pgfqpoint{1.759870in}{0.768223in}}%
\pgfpathlineto{\pgfqpoint{1.762674in}{0.674766in}}%
\pgfpathlineto{\pgfqpoint{1.765477in}{0.672649in}}%
\pgfpathlineto{\pgfqpoint{1.768280in}{0.694073in}}%
\pgfpathlineto{\pgfqpoint{1.771083in}{0.657514in}}%
\pgfpathlineto{\pgfqpoint{1.773887in}{0.661677in}}%
\pgfpathlineto{\pgfqpoint{1.776690in}{0.660653in}}%
\pgfpathlineto{\pgfqpoint{1.779493in}{0.652911in}}%
\pgfpathlineto{\pgfqpoint{1.785100in}{0.717854in}}%
\pgfpathlineto{\pgfqpoint{1.787903in}{0.740374in}}%
\pgfpathlineto{\pgfqpoint{1.790706in}{0.698117in}}%
\pgfpathlineto{\pgfqpoint{1.793509in}{0.666469in}}%
\pgfpathlineto{\pgfqpoint{1.796312in}{0.656857in}}%
\pgfpathlineto{\pgfqpoint{1.799116in}{0.689640in}}%
\pgfpathlineto{\pgfqpoint{1.801919in}{0.665588in}}%
\pgfpathlineto{\pgfqpoint{1.804722in}{0.674329in}}%
\pgfpathlineto{\pgfqpoint{1.807525in}{0.750042in}}%
\pgfpathlineto{\pgfqpoint{1.810329in}{0.672736in}}%
\pgfpathlineto{\pgfqpoint{1.813132in}{0.812929in}}%
\pgfpathlineto{\pgfqpoint{1.815935in}{0.827658in}}%
\pgfpathlineto{\pgfqpoint{1.818738in}{0.682447in}}%
\pgfpathlineto{\pgfqpoint{1.821542in}{0.717583in}}%
\pgfpathlineto{\pgfqpoint{1.827148in}{0.660853in}}%
\pgfpathlineto{\pgfqpoint{1.829951in}{0.695561in}}%
\pgfpathlineto{\pgfqpoint{1.832754in}{0.695477in}}%
\pgfpathlineto{\pgfqpoint{1.835558in}{0.673251in}}%
\pgfpathlineto{\pgfqpoint{1.838361in}{0.663560in}}%
\pgfpathlineto{\pgfqpoint{1.841164in}{0.657072in}}%
\pgfpathlineto{\pgfqpoint{1.843967in}{0.652039in}}%
\pgfpathlineto{\pgfqpoint{1.846771in}{0.663422in}}%
\pgfpathlineto{\pgfqpoint{1.849574in}{0.681287in}}%
\pgfpathlineto{\pgfqpoint{1.852377in}{0.656561in}}%
\pgfpathlineto{\pgfqpoint{1.855180in}{0.651123in}}%
\pgfpathlineto{\pgfqpoint{1.857984in}{0.650345in}}%
\pgfpathlineto{\pgfqpoint{1.860787in}{0.650961in}}%
\pgfpathlineto{\pgfqpoint{1.863590in}{1.075410in}}%
\pgfpathlineto{\pgfqpoint{1.866393in}{0.746844in}}%
\pgfpathlineto{\pgfqpoint{1.869196in}{0.739532in}}%
\pgfpathlineto{\pgfqpoint{1.872000in}{0.672887in}}%
\pgfpathlineto{\pgfqpoint{1.874803in}{0.661685in}}%
\pgfpathlineto{\pgfqpoint{1.877606in}{0.727398in}}%
\pgfpathlineto{\pgfqpoint{1.880409in}{0.677148in}}%
\pgfpathlineto{\pgfqpoint{1.883213in}{0.660112in}}%
\pgfpathlineto{\pgfqpoint{1.888819in}{0.672103in}}%
\pgfpathlineto{\pgfqpoint{1.891622in}{0.657483in}}%
\pgfpathlineto{\pgfqpoint{1.894426in}{0.658600in}}%
\pgfpathlineto{\pgfqpoint{1.897229in}{0.690549in}}%
\pgfpathlineto{\pgfqpoint{1.900032in}{0.715077in}}%
\pgfpathlineto{\pgfqpoint{1.902835in}{0.666293in}}%
\pgfpathlineto{\pgfqpoint{1.905638in}{0.674477in}}%
\pgfpathlineto{\pgfqpoint{1.908442in}{0.657960in}}%
\pgfpathlineto{\pgfqpoint{1.911245in}{0.654761in}}%
\pgfpathlineto{\pgfqpoint{1.914048in}{0.655366in}}%
\pgfpathlineto{\pgfqpoint{1.916851in}{0.651849in}}%
\pgfpathlineto{\pgfqpoint{1.919655in}{0.654749in}}%
\pgfpathlineto{\pgfqpoint{1.922458in}{0.653937in}}%
\pgfpathlineto{\pgfqpoint{1.925261in}{0.738875in}}%
\pgfpathlineto{\pgfqpoint{1.928064in}{0.670870in}}%
\pgfpathlineto{\pgfqpoint{1.930868in}{0.665621in}}%
\pgfpathlineto{\pgfqpoint{1.933671in}{0.653279in}}%
\pgfpathlineto{\pgfqpoint{1.936474in}{0.651403in}}%
\pgfpathlineto{\pgfqpoint{1.942080in}{0.721861in}}%
\pgfpathlineto{\pgfqpoint{1.944884in}{0.707634in}}%
\pgfpathlineto{\pgfqpoint{1.947687in}{0.684972in}}%
\pgfpathlineto{\pgfqpoint{1.950490in}{0.656803in}}%
\pgfpathlineto{\pgfqpoint{1.953293in}{0.695556in}}%
\pgfpathlineto{\pgfqpoint{1.956097in}{0.665619in}}%
\pgfpathlineto{\pgfqpoint{1.958900in}{0.654822in}}%
\pgfpathlineto{\pgfqpoint{1.961703in}{0.652285in}}%
\pgfpathlineto{\pgfqpoint{1.964506in}{0.654966in}}%
\pgfpathlineto{\pgfqpoint{1.967310in}{0.650738in}}%
\pgfpathlineto{\pgfqpoint{1.970113in}{0.650881in}}%
\pgfpathlineto{\pgfqpoint{1.972916in}{0.652384in}}%
\pgfpathlineto{\pgfqpoint{1.975719in}{0.649370in}}%
\pgfpathlineto{\pgfqpoint{1.978522in}{0.649958in}}%
\pgfpathlineto{\pgfqpoint{1.981326in}{0.670981in}}%
\pgfpathlineto{\pgfqpoint{1.984129in}{0.651417in}}%
\pgfpathlineto{\pgfqpoint{1.986932in}{0.662968in}}%
\pgfpathlineto{\pgfqpoint{1.989735in}{0.651525in}}%
\pgfpathlineto{\pgfqpoint{1.992539in}{0.645629in}}%
\pgfpathlineto{\pgfqpoint{1.995342in}{0.645489in}}%
\pgfpathlineto{\pgfqpoint{1.998145in}{0.649939in}}%
\pgfpathlineto{\pgfqpoint{2.000948in}{0.685890in}}%
\pgfpathlineto{\pgfqpoint{2.003752in}{0.670547in}}%
\pgfpathlineto{\pgfqpoint{2.006555in}{0.649178in}}%
\pgfpathlineto{\pgfqpoint{2.009358in}{0.676992in}}%
\pgfpathlineto{\pgfqpoint{2.012161in}{0.794639in}}%
\pgfpathlineto{\pgfqpoint{2.014964in}{0.675674in}}%
\pgfpathlineto{\pgfqpoint{2.017768in}{0.652954in}}%
\pgfpathlineto{\pgfqpoint{2.020571in}{0.806756in}}%
\pgfpathlineto{\pgfqpoint{2.023374in}{1.099301in}}%
\pgfpathlineto{\pgfqpoint{2.026177in}{0.985596in}}%
\pgfpathlineto{\pgfqpoint{2.028981in}{0.740943in}}%
\pgfpathlineto{\pgfqpoint{2.031784in}{0.813870in}}%
\pgfpathlineto{\pgfqpoint{2.034587in}{0.696191in}}%
\pgfpathlineto{\pgfqpoint{2.037390in}{0.948036in}}%
\pgfpathlineto{\pgfqpoint{2.040194in}{0.762842in}}%
\pgfpathlineto{\pgfqpoint{2.042997in}{0.672991in}}%
\pgfpathlineto{\pgfqpoint{2.045800in}{0.650802in}}%
\pgfpathlineto{\pgfqpoint{2.048603in}{0.657430in}}%
\pgfpathlineto{\pgfqpoint{2.051406in}{0.654318in}}%
\pgfpathlineto{\pgfqpoint{2.054210in}{0.663758in}}%
\pgfpathlineto{\pgfqpoint{2.057013in}{0.650632in}}%
\pgfpathlineto{\pgfqpoint{2.059816in}{0.656571in}}%
\pgfpathlineto{\pgfqpoint{2.062619in}{0.647571in}}%
\pgfpathlineto{\pgfqpoint{2.065423in}{0.646688in}}%
\pgfpathlineto{\pgfqpoint{2.068226in}{0.643257in}}%
\pgfpathlineto{\pgfqpoint{2.071029in}{0.689859in}}%
\pgfpathlineto{\pgfqpoint{2.073832in}{0.657587in}}%
\pgfpathlineto{\pgfqpoint{2.076636in}{0.645879in}}%
\pgfpathlineto{\pgfqpoint{2.079439in}{0.649061in}}%
\pgfpathlineto{\pgfqpoint{2.082242in}{0.642631in}}%
\pgfpathlineto{\pgfqpoint{2.085045in}{0.653533in}}%
\pgfpathlineto{\pgfqpoint{2.087849in}{0.642412in}}%
\pgfpathlineto{\pgfqpoint{2.090652in}{0.641810in}}%
\pgfpathlineto{\pgfqpoint{2.093455in}{0.822845in}}%
\pgfpathlineto{\pgfqpoint{2.096258in}{0.702674in}}%
\pgfpathlineto{\pgfqpoint{2.099061in}{0.727095in}}%
\pgfpathlineto{\pgfqpoint{2.101865in}{0.662237in}}%
\pgfpathlineto{\pgfqpoint{2.104668in}{0.647137in}}%
\pgfpathlineto{\pgfqpoint{2.107471in}{0.641271in}}%
\pgfpathlineto{\pgfqpoint{2.110274in}{0.640776in}}%
\pgfpathlineto{\pgfqpoint{2.113078in}{0.652549in}}%
\pgfpathlineto{\pgfqpoint{2.115881in}{0.902931in}}%
\pgfpathlineto{\pgfqpoint{2.118684in}{0.703909in}}%
\pgfpathlineto{\pgfqpoint{2.121487in}{0.664975in}}%
\pgfpathlineto{\pgfqpoint{2.124291in}{0.652259in}}%
\pgfpathlineto{\pgfqpoint{2.127094in}{0.704567in}}%
\pgfpathlineto{\pgfqpoint{2.129897in}{0.696656in}}%
\pgfpathlineto{\pgfqpoint{2.132700in}{0.656074in}}%
\pgfpathlineto{\pgfqpoint{2.135503in}{0.707872in}}%
\pgfpathlineto{\pgfqpoint{2.138307in}{0.713014in}}%
\pgfpathlineto{\pgfqpoint{2.141110in}{0.729352in}}%
\pgfpathlineto{\pgfqpoint{2.143913in}{0.662450in}}%
\pgfpathlineto{\pgfqpoint{2.146716in}{0.775178in}}%
\pgfpathlineto{\pgfqpoint{2.149520in}{0.690351in}}%
\pgfpathlineto{\pgfqpoint{2.155126in}{0.647771in}}%
\pgfpathlineto{\pgfqpoint{2.157929in}{0.641093in}}%
\pgfpathlineto{\pgfqpoint{2.160733in}{1.123085in}}%
\pgfpathlineto{\pgfqpoint{2.163536in}{0.945778in}}%
\pgfpathlineto{\pgfqpoint{2.166339in}{0.723417in}}%
\pgfpathlineto{\pgfqpoint{2.169142in}{0.695045in}}%
\pgfpathlineto{\pgfqpoint{2.171945in}{0.996961in}}%
\pgfpathlineto{\pgfqpoint{2.174749in}{1.007394in}}%
\pgfpathlineto{\pgfqpoint{2.177552in}{0.778360in}}%
\pgfpathlineto{\pgfqpoint{2.180355in}{0.694248in}}%
\pgfpathlineto{\pgfqpoint{2.183158in}{0.693103in}}%
\pgfpathlineto{\pgfqpoint{2.185962in}{0.681093in}}%
\pgfpathlineto{\pgfqpoint{2.188765in}{0.655415in}}%
\pgfpathlineto{\pgfqpoint{2.191568in}{0.645434in}}%
\pgfpathlineto{\pgfqpoint{2.194371in}{0.657697in}}%
\pgfpathlineto{\pgfqpoint{2.197175in}{0.645092in}}%
\pgfpathlineto{\pgfqpoint{2.199978in}{1.669549in}}%
\pgfpathlineto{\pgfqpoint{2.202781in}{0.997190in}}%
\pgfpathlineto{\pgfqpoint{2.205584in}{0.756986in}}%
\pgfpathlineto{\pgfqpoint{2.208387in}{0.673090in}}%
\pgfpathlineto{\pgfqpoint{2.211191in}{0.671680in}}%
\pgfpathlineto{\pgfqpoint{2.213994in}{0.686192in}}%
\pgfpathlineto{\pgfqpoint{2.216797in}{0.668089in}}%
\pgfpathlineto{\pgfqpoint{2.219600in}{0.687397in}}%
\pgfpathlineto{\pgfqpoint{2.225207in}{0.660445in}}%
\pgfpathlineto{\pgfqpoint{2.228010in}{0.652866in}}%
\pgfpathlineto{\pgfqpoint{2.230813in}{0.649566in}}%
\pgfpathlineto{\pgfqpoint{2.233617in}{0.647352in}}%
\pgfpathlineto{\pgfqpoint{2.236420in}{0.652140in}}%
\pgfpathlineto{\pgfqpoint{2.239223in}{0.704542in}}%
\pgfpathlineto{\pgfqpoint{2.242026in}{0.669772in}}%
\pgfpathlineto{\pgfqpoint{2.244829in}{0.653237in}}%
\pgfpathlineto{\pgfqpoint{2.247633in}{0.704038in}}%
\pgfpathlineto{\pgfqpoint{2.250436in}{0.668369in}}%
\pgfpathlineto{\pgfqpoint{2.253239in}{0.659658in}}%
\pgfpathlineto{\pgfqpoint{2.258846in}{0.653900in}}%
\pgfpathlineto{\pgfqpoint{2.261649in}{0.653801in}}%
\pgfpathlineto{\pgfqpoint{2.264452in}{0.647186in}}%
\pgfpathlineto{\pgfqpoint{2.267255in}{0.643985in}}%
\pgfpathlineto{\pgfqpoint{2.270059in}{0.661423in}}%
\pgfpathlineto{\pgfqpoint{2.272862in}{0.695572in}}%
\pgfpathlineto{\pgfqpoint{2.275665in}{0.659644in}}%
\pgfpathlineto{\pgfqpoint{2.278468in}{0.647518in}}%
\pgfpathlineto{\pgfqpoint{2.281271in}{0.642976in}}%
\pgfpathlineto{\pgfqpoint{2.284075in}{0.643956in}}%
\pgfpathlineto{\pgfqpoint{2.286878in}{0.642995in}}%
\pgfpathlineto{\pgfqpoint{2.289681in}{0.666095in}}%
\pgfpathlineto{\pgfqpoint{2.295288in}{0.648676in}}%
\pgfpathlineto{\pgfqpoint{2.298091in}{0.654477in}}%
\pgfpathlineto{\pgfqpoint{2.300894in}{0.642468in}}%
\pgfpathlineto{\pgfqpoint{2.303697in}{0.637539in}}%
\pgfpathlineto{\pgfqpoint{2.306501in}{0.646662in}}%
\pgfpathlineto{\pgfqpoint{2.309304in}{0.647099in}}%
\pgfpathlineto{\pgfqpoint{2.312107in}{0.638250in}}%
\pgfpathlineto{\pgfqpoint{2.314910in}{0.665149in}}%
\pgfpathlineto{\pgfqpoint{2.317713in}{0.645467in}}%
\pgfpathlineto{\pgfqpoint{2.320517in}{0.637226in}}%
\pgfpathlineto{\pgfqpoint{2.323320in}{0.632588in}}%
\pgfpathlineto{\pgfqpoint{2.326123in}{0.632932in}}%
\pgfpathlineto{\pgfqpoint{2.328926in}{0.648332in}}%
\pgfpathlineto{\pgfqpoint{2.331730in}{0.636601in}}%
\pgfpathlineto{\pgfqpoint{2.334533in}{0.635154in}}%
\pgfpathlineto{\pgfqpoint{2.337336in}{0.630891in}}%
\pgfpathlineto{\pgfqpoint{2.340139in}{0.653708in}}%
\pgfpathlineto{\pgfqpoint{2.342943in}{0.637955in}}%
\pgfpathlineto{\pgfqpoint{2.345746in}{0.636573in}}%
\pgfpathlineto{\pgfqpoint{2.348549in}{0.632228in}}%
\pgfpathlineto{\pgfqpoint{2.351352in}{0.650327in}}%
\pgfpathlineto{\pgfqpoint{2.354155in}{0.635936in}}%
\pgfpathlineto{\pgfqpoint{2.356959in}{0.630492in}}%
\pgfpathlineto{\pgfqpoint{2.359762in}{0.828191in}}%
\pgfpathlineto{\pgfqpoint{2.362565in}{0.710244in}}%
\pgfpathlineto{\pgfqpoint{2.365368in}{0.671295in}}%
\pgfpathlineto{\pgfqpoint{2.368172in}{0.649231in}}%
\pgfpathlineto{\pgfqpoint{2.370975in}{0.636888in}}%
\pgfpathlineto{\pgfqpoint{2.373778in}{0.633532in}}%
\pgfpathlineto{\pgfqpoint{2.376581in}{1.148839in}}%
\pgfpathlineto{\pgfqpoint{2.379385in}{0.825786in}}%
\pgfpathlineto{\pgfqpoint{2.382188in}{0.709495in}}%
\pgfpathlineto{\pgfqpoint{2.384991in}{0.657352in}}%
\pgfpathlineto{\pgfqpoint{2.387794in}{0.724501in}}%
\pgfpathlineto{\pgfqpoint{2.390597in}{0.667566in}}%
\pgfpathlineto{\pgfqpoint{2.393401in}{0.646215in}}%
\pgfpathlineto{\pgfqpoint{2.396204in}{0.663974in}}%
\pgfpathlineto{\pgfqpoint{2.399007in}{0.644719in}}%
\pgfpathlineto{\pgfqpoint{2.401810in}{0.664428in}}%
\pgfpathlineto{\pgfqpoint{2.404614in}{0.646908in}}%
\pgfpathlineto{\pgfqpoint{2.407417in}{0.637939in}}%
\pgfpathlineto{\pgfqpoint{2.410220in}{0.635710in}}%
\pgfpathlineto{\pgfqpoint{2.413023in}{0.644161in}}%
\pgfpathlineto{\pgfqpoint{2.415827in}{0.637286in}}%
\pgfpathlineto{\pgfqpoint{2.418630in}{0.684102in}}%
\pgfpathlineto{\pgfqpoint{2.421433in}{0.675686in}}%
\pgfpathlineto{\pgfqpoint{2.424236in}{0.647756in}}%
\pgfpathlineto{\pgfqpoint{2.427039in}{0.638984in}}%
\pgfpathlineto{\pgfqpoint{2.429843in}{0.637569in}}%
\pgfpathlineto{\pgfqpoint{2.432646in}{0.632891in}}%
\pgfpathlineto{\pgfqpoint{2.441056in}{0.628148in}}%
\pgfpathlineto{\pgfqpoint{2.443859in}{0.688403in}}%
\pgfpathlineto{\pgfqpoint{2.446662in}{0.691098in}}%
\pgfpathlineto{\pgfqpoint{2.449465in}{0.657207in}}%
\pgfpathlineto{\pgfqpoint{2.452269in}{0.638373in}}%
\pgfpathlineto{\pgfqpoint{2.455072in}{0.637383in}}%
\pgfpathlineto{\pgfqpoint{2.457875in}{0.630744in}}%
\pgfpathlineto{\pgfqpoint{2.460678in}{0.630519in}}%
\pgfpathlineto{\pgfqpoint{2.463481in}{0.650058in}}%
\pgfpathlineto{\pgfqpoint{2.466285in}{0.644044in}}%
\pgfpathlineto{\pgfqpoint{2.469088in}{0.668867in}}%
\pgfpathlineto{\pgfqpoint{2.474694in}{0.634688in}}%
\pgfpathlineto{\pgfqpoint{2.477498in}{0.638737in}}%
\pgfpathlineto{\pgfqpoint{2.480301in}{0.660487in}}%
\pgfpathlineto{\pgfqpoint{2.483104in}{0.638392in}}%
\pgfpathlineto{\pgfqpoint{2.485907in}{0.630170in}}%
\pgfpathlineto{\pgfqpoint{2.488711in}{0.679808in}}%
\pgfpathlineto{\pgfqpoint{2.491514in}{0.665983in}}%
\pgfpathlineto{\pgfqpoint{2.494317in}{0.644863in}}%
\pgfpathlineto{\pgfqpoint{2.497120in}{0.752808in}}%
\pgfpathlineto{\pgfqpoint{2.499923in}{0.674686in}}%
\pgfpathlineto{\pgfqpoint{2.502727in}{0.654775in}}%
\pgfpathlineto{\pgfqpoint{2.505530in}{0.644454in}}%
\pgfpathlineto{\pgfqpoint{2.508333in}{0.729749in}}%
\pgfpathlineto{\pgfqpoint{2.511136in}{0.667064in}}%
\pgfpathlineto{\pgfqpoint{2.513940in}{0.648569in}}%
\pgfpathlineto{\pgfqpoint{2.516743in}{0.634844in}}%
\pgfpathlineto{\pgfqpoint{2.519546in}{0.745673in}}%
\pgfpathlineto{\pgfqpoint{2.522349in}{0.701568in}}%
\pgfpathlineto{\pgfqpoint{2.525153in}{0.841408in}}%
\pgfpathlineto{\pgfqpoint{2.527956in}{0.776866in}}%
\pgfpathlineto{\pgfqpoint{2.530759in}{0.686584in}}%
\pgfpathlineto{\pgfqpoint{2.533562in}{0.673635in}}%
\pgfpathlineto{\pgfqpoint{2.536366in}{0.701547in}}%
\pgfpathlineto{\pgfqpoint{2.539169in}{0.666357in}}%
\pgfpathlineto{\pgfqpoint{2.541972in}{0.656178in}}%
\pgfpathlineto{\pgfqpoint{2.544775in}{0.641712in}}%
\pgfpathlineto{\pgfqpoint{2.547578in}{0.683775in}}%
\pgfpathlineto{\pgfqpoint{2.550382in}{0.698031in}}%
\pgfpathlineto{\pgfqpoint{2.555988in}{0.650281in}}%
\pgfpathlineto{\pgfqpoint{2.558791in}{0.635474in}}%
\pgfpathlineto{\pgfqpoint{2.561595in}{0.647187in}}%
\pgfpathlineto{\pgfqpoint{2.564398in}{0.670979in}}%
\pgfpathlineto{\pgfqpoint{2.567201in}{1.535156in}}%
\pgfpathlineto{\pgfqpoint{2.570004in}{0.943691in}}%
\pgfpathlineto{\pgfqpoint{2.572808in}{0.713185in}}%
\pgfpathlineto{\pgfqpoint{2.575611in}{0.655808in}}%
\pgfpathlineto{\pgfqpoint{2.578414in}{0.693121in}}%
\pgfpathlineto{\pgfqpoint{2.581217in}{0.655296in}}%
\pgfpathlineto{\pgfqpoint{2.584020in}{0.647085in}}%
\pgfpathlineto{\pgfqpoint{2.586824in}{0.674780in}}%
\pgfpathlineto{\pgfqpoint{2.589627in}{0.653362in}}%
\pgfpathlineto{\pgfqpoint{2.592430in}{0.656532in}}%
\pgfpathlineto{\pgfqpoint{2.595233in}{0.652811in}}%
\pgfpathlineto{\pgfqpoint{2.598037in}{0.672803in}}%
\pgfpathlineto{\pgfqpoint{2.600840in}{0.649791in}}%
\pgfpathlineto{\pgfqpoint{2.603643in}{0.640775in}}%
\pgfpathlineto{\pgfqpoint{2.606446in}{0.636118in}}%
\pgfpathlineto{\pgfqpoint{2.609250in}{0.636227in}}%
\pgfpathlineto{\pgfqpoint{2.612053in}{0.638882in}}%
\pgfpathlineto{\pgfqpoint{2.614856in}{0.634374in}}%
\pgfpathlineto{\pgfqpoint{2.617659in}{0.637743in}}%
\pgfpathlineto{\pgfqpoint{2.620462in}{0.633733in}}%
\pgfpathlineto{\pgfqpoint{2.626069in}{0.629983in}}%
\pgfpathlineto{\pgfqpoint{2.628872in}{0.630892in}}%
\pgfpathlineto{\pgfqpoint{2.631675in}{0.630128in}}%
\pgfpathlineto{\pgfqpoint{2.637282in}{0.631140in}}%
\pgfpathlineto{\pgfqpoint{2.642888in}{0.626237in}}%
\pgfpathlineto{\pgfqpoint{2.645692in}{0.663328in}}%
\pgfpathlineto{\pgfqpoint{2.648495in}{0.642252in}}%
\pgfpathlineto{\pgfqpoint{2.651298in}{0.865767in}}%
\pgfpathlineto{\pgfqpoint{2.654101in}{0.731945in}}%
\pgfpathlineto{\pgfqpoint{2.656904in}{0.741711in}}%
\pgfpathlineto{\pgfqpoint{2.659708in}{0.706379in}}%
\pgfpathlineto{\pgfqpoint{2.662511in}{0.684250in}}%
\pgfpathlineto{\pgfqpoint{2.665314in}{0.691391in}}%
\pgfpathlineto{\pgfqpoint{2.668117in}{0.660415in}}%
\pgfpathlineto{\pgfqpoint{2.670921in}{0.641514in}}%
\pgfpathlineto{\pgfqpoint{2.673724in}{0.635748in}}%
\pgfpathlineto{\pgfqpoint{2.676527in}{0.637261in}}%
\pgfpathlineto{\pgfqpoint{2.679330in}{0.640629in}}%
\pgfpathlineto{\pgfqpoint{2.682134in}{0.637460in}}%
\pgfpathlineto{\pgfqpoint{2.684937in}{0.664146in}}%
\pgfpathlineto{\pgfqpoint{2.687740in}{0.647475in}}%
\pgfpathlineto{\pgfqpoint{2.690543in}{0.826718in}}%
\pgfpathlineto{\pgfqpoint{2.693346in}{0.736925in}}%
\pgfpathlineto{\pgfqpoint{2.696150in}{0.686532in}}%
\pgfpathlineto{\pgfqpoint{2.698953in}{0.662619in}}%
\pgfpathlineto{\pgfqpoint{2.701756in}{0.728420in}}%
\pgfpathlineto{\pgfqpoint{2.704559in}{0.677445in}}%
\pgfpathlineto{\pgfqpoint{2.707363in}{0.649575in}}%
\pgfpathlineto{\pgfqpoint{2.710166in}{0.785332in}}%
\pgfpathlineto{\pgfqpoint{2.712969in}{0.739428in}}%
\pgfpathlineto{\pgfqpoint{2.715772in}{0.683239in}}%
\pgfpathlineto{\pgfqpoint{2.718576in}{0.655259in}}%
\pgfpathlineto{\pgfqpoint{2.721379in}{0.638764in}}%
\pgfpathlineto{\pgfqpoint{2.724182in}{0.630503in}}%
\pgfpathlineto{\pgfqpoint{2.726985in}{0.626325in}}%
\pgfpathlineto{\pgfqpoint{2.729788in}{0.644472in}}%
\pgfpathlineto{\pgfqpoint{2.732592in}{0.809918in}}%
\pgfpathlineto{\pgfqpoint{2.735395in}{0.847979in}}%
\pgfpathlineto{\pgfqpoint{2.738198in}{0.726219in}}%
\pgfpathlineto{\pgfqpoint{2.741001in}{0.717114in}}%
\pgfpathlineto{\pgfqpoint{2.743805in}{0.666808in}}%
\pgfpathlineto{\pgfqpoint{2.746608in}{0.660377in}}%
\pgfpathlineto{\pgfqpoint{2.749411in}{0.662513in}}%
\pgfpathlineto{\pgfqpoint{2.752214in}{0.686782in}}%
\pgfpathlineto{\pgfqpoint{2.755018in}{0.752236in}}%
\pgfpathlineto{\pgfqpoint{2.760624in}{0.671731in}}%
\pgfpathlineto{\pgfqpoint{2.763427in}{0.657331in}}%
\pgfpathlineto{\pgfqpoint{2.766230in}{0.662528in}}%
\pgfpathlineto{\pgfqpoint{2.769034in}{0.653204in}}%
\pgfpathlineto{\pgfqpoint{2.771837in}{0.639010in}}%
\pgfpathlineto{\pgfqpoint{2.774640in}{0.648281in}}%
\pgfpathlineto{\pgfqpoint{2.777443in}{0.635479in}}%
\pgfpathlineto{\pgfqpoint{2.780247in}{0.641823in}}%
\pgfpathlineto{\pgfqpoint{2.783050in}{0.632133in}}%
\pgfpathlineto{\pgfqpoint{2.785853in}{0.629347in}}%
\pgfpathlineto{\pgfqpoint{2.788656in}{0.624504in}}%
\pgfpathlineto{\pgfqpoint{2.791460in}{0.621535in}}%
\pgfpathlineto{\pgfqpoint{2.797066in}{0.680384in}}%
\pgfpathlineto{\pgfqpoint{2.799869in}{0.696211in}}%
\pgfpathlineto{\pgfqpoint{2.805476in}{0.647046in}}%
\pgfpathlineto{\pgfqpoint{2.808279in}{0.745880in}}%
\pgfpathlineto{\pgfqpoint{2.811082in}{0.686892in}}%
\pgfpathlineto{\pgfqpoint{2.813885in}{0.841882in}}%
\pgfpathlineto{\pgfqpoint{2.816689in}{0.735695in}}%
\pgfpathlineto{\pgfqpoint{2.819492in}{0.701548in}}%
\pgfpathlineto{\pgfqpoint{2.822295in}{0.763303in}}%
\pgfpathlineto{\pgfqpoint{2.825098in}{0.710174in}}%
\pgfpathlineto{\pgfqpoint{2.827902in}{0.768517in}}%
\pgfpathlineto{\pgfqpoint{2.830705in}{0.708522in}}%
\pgfpathlineto{\pgfqpoint{2.833508in}{0.669282in}}%
\pgfpathlineto{\pgfqpoint{2.836311in}{0.650398in}}%
\pgfpathlineto{\pgfqpoint{2.839114in}{0.646800in}}%
\pgfpathlineto{\pgfqpoint{2.841918in}{0.634066in}}%
\pgfpathlineto{\pgfqpoint{2.844721in}{0.771038in}}%
\pgfpathlineto{\pgfqpoint{2.847524in}{0.702726in}}%
\pgfpathlineto{\pgfqpoint{2.850327in}{0.665981in}}%
\pgfpathlineto{\pgfqpoint{2.853131in}{0.644829in}}%
\pgfpathlineto{\pgfqpoint{2.855934in}{0.640035in}}%
\pgfpathlineto{\pgfqpoint{2.858737in}{0.636517in}}%
\pgfpathlineto{\pgfqpoint{2.861540in}{0.628384in}}%
\pgfpathlineto{\pgfqpoint{2.864344in}{0.625291in}}%
\pgfpathlineto{\pgfqpoint{2.867147in}{0.627060in}}%
\pgfpathlineto{\pgfqpoint{2.869950in}{0.630002in}}%
\pgfpathlineto{\pgfqpoint{2.872753in}{0.625019in}}%
\pgfpathlineto{\pgfqpoint{2.875556in}{0.622720in}}%
\pgfpathlineto{\pgfqpoint{2.878360in}{0.663512in}}%
\pgfpathlineto{\pgfqpoint{2.881163in}{0.642911in}}%
\pgfpathlineto{\pgfqpoint{2.883966in}{0.630425in}}%
\pgfpathlineto{\pgfqpoint{2.886769in}{0.623621in}}%
\pgfpathlineto{\pgfqpoint{2.889573in}{0.719806in}}%
\pgfpathlineto{\pgfqpoint{2.892376in}{0.672527in}}%
\pgfpathlineto{\pgfqpoint{2.895179in}{0.653177in}}%
\pgfpathlineto{\pgfqpoint{2.897982in}{0.745451in}}%
\pgfpathlineto{\pgfqpoint{2.900786in}{0.693683in}}%
\pgfpathlineto{\pgfqpoint{2.903589in}{0.666585in}}%
\pgfpathlineto{\pgfqpoint{2.911998in}{0.640137in}}%
\pgfpathlineto{\pgfqpoint{2.914802in}{0.747151in}}%
\pgfpathlineto{\pgfqpoint{2.917605in}{0.699947in}}%
\pgfpathlineto{\pgfqpoint{2.920408in}{0.670245in}}%
\pgfpathlineto{\pgfqpoint{2.923211in}{0.649927in}}%
\pgfpathlineto{\pgfqpoint{2.926015in}{0.644204in}}%
\pgfpathlineto{\pgfqpoint{2.928818in}{0.643256in}}%
\pgfpathlineto{\pgfqpoint{2.931621in}{0.737247in}}%
\pgfpathlineto{\pgfqpoint{2.934424in}{0.699446in}}%
\pgfpathlineto{\pgfqpoint{2.940031in}{0.652895in}}%
\pgfpathlineto{\pgfqpoint{2.942834in}{0.657719in}}%
\pgfpathlineto{\pgfqpoint{2.945637in}{0.656441in}}%
\pgfpathlineto{\pgfqpoint{2.948440in}{0.643920in}}%
\pgfpathlineto{\pgfqpoint{2.951244in}{0.634802in}}%
\pgfpathlineto{\pgfqpoint{2.956850in}{0.643981in}}%
\pgfpathlineto{\pgfqpoint{2.959653in}{0.634202in}}%
\pgfpathlineto{\pgfqpoint{2.962457in}{0.699244in}}%
\pgfpathlineto{\pgfqpoint{2.968063in}{0.654736in}}%
\pgfpathlineto{\pgfqpoint{2.970866in}{0.688965in}}%
\pgfpathlineto{\pgfqpoint{2.973670in}{0.663852in}}%
\pgfpathlineto{\pgfqpoint{2.976473in}{0.648090in}}%
\pgfpathlineto{\pgfqpoint{2.979276in}{0.638111in}}%
\pgfpathlineto{\pgfqpoint{2.982079in}{0.664983in}}%
\pgfpathlineto{\pgfqpoint{2.984883in}{0.647260in}}%
\pgfpathlineto{\pgfqpoint{2.987686in}{0.666989in}}%
\pgfpathlineto{\pgfqpoint{2.990489in}{0.648966in}}%
\pgfpathlineto{\pgfqpoint{2.993292in}{0.661970in}}%
\pgfpathlineto{\pgfqpoint{2.996095in}{0.645631in}}%
\pgfpathlineto{\pgfqpoint{2.998899in}{0.642293in}}%
\pgfpathlineto{\pgfqpoint{3.001702in}{0.667048in}}%
\pgfpathlineto{\pgfqpoint{3.004505in}{0.650303in}}%
\pgfpathlineto{\pgfqpoint{3.010112in}{0.637686in}}%
\pgfpathlineto{\pgfqpoint{3.012915in}{0.667164in}}%
\pgfpathlineto{\pgfqpoint{3.015718in}{0.650509in}}%
\pgfpathlineto{\pgfqpoint{3.018521in}{0.639317in}}%
\pgfpathlineto{\pgfqpoint{3.021325in}{0.648408in}}%
\pgfpathlineto{\pgfqpoint{3.024128in}{0.641521in}}%
\pgfpathlineto{\pgfqpoint{3.026931in}{0.632033in}}%
\pgfpathlineto{\pgfqpoint{3.029734in}{0.835514in}}%
\pgfpathlineto{\pgfqpoint{3.032537in}{0.766445in}}%
\pgfpathlineto{\pgfqpoint{3.035341in}{0.727898in}}%
\pgfpathlineto{\pgfqpoint{3.038144in}{0.697670in}}%
\pgfpathlineto{\pgfqpoint{3.040947in}{0.674673in}}%
\pgfpathlineto{\pgfqpoint{3.043750in}{0.661619in}}%
\pgfpathlineto{\pgfqpoint{3.046554in}{0.704368in}}%
\pgfpathlineto{\pgfqpoint{3.049357in}{0.679831in}}%
\pgfpathlineto{\pgfqpoint{3.052160in}{0.674533in}}%
\pgfpathlineto{\pgfqpoint{3.054963in}{0.665354in}}%
\pgfpathlineto{\pgfqpoint{3.060570in}{0.642772in}}%
\pgfpathlineto{\pgfqpoint{3.066176in}{0.629076in}}%
\pgfpathlineto{\pgfqpoint{3.071783in}{0.655017in}}%
\pgfpathlineto{\pgfqpoint{3.074586in}{0.644012in}}%
\pgfpathlineto{\pgfqpoint{3.077389in}{0.638387in}}%
\pgfpathlineto{\pgfqpoint{3.080192in}{0.681866in}}%
\pgfpathlineto{\pgfqpoint{3.082996in}{0.704732in}}%
\pgfpathlineto{\pgfqpoint{3.085799in}{0.683822in}}%
\pgfpathlineto{\pgfqpoint{3.088602in}{0.675309in}}%
\pgfpathlineto{\pgfqpoint{3.091405in}{0.658363in}}%
\pgfpathlineto{\pgfqpoint{3.094209in}{0.688041in}}%
\pgfpathlineto{\pgfqpoint{3.097012in}{0.668716in}}%
\pgfpathlineto{\pgfqpoint{3.099815in}{0.665101in}}%
\pgfpathlineto{\pgfqpoint{3.102618in}{0.656077in}}%
\pgfpathlineto{\pgfqpoint{3.105421in}{0.719118in}}%
\pgfpathlineto{\pgfqpoint{3.108225in}{0.692846in}}%
\pgfpathlineto{\pgfqpoint{3.111028in}{0.672725in}}%
\pgfpathlineto{\pgfqpoint{3.113831in}{0.697333in}}%
\pgfpathlineto{\pgfqpoint{3.116634in}{0.676470in}}%
\pgfpathlineto{\pgfqpoint{3.119438in}{0.661134in}}%
\pgfpathlineto{\pgfqpoint{3.122241in}{0.648936in}}%
\pgfpathlineto{\pgfqpoint{3.125044in}{0.639319in}}%
\pgfpathlineto{\pgfqpoint{3.127847in}{0.632159in}}%
\pgfpathlineto{\pgfqpoint{3.130651in}{0.627807in}}%
\pgfpathlineto{\pgfqpoint{3.133454in}{0.633636in}}%
\pgfpathlineto{\pgfqpoint{3.139060in}{1.325001in}}%
\pgfpathlineto{\pgfqpoint{3.141863in}{1.136021in}}%
\pgfpathlineto{\pgfqpoint{3.144667in}{1.078728in}}%
\pgfpathlineto{\pgfqpoint{3.150273in}{0.770802in}}%
\pgfpathlineto{\pgfqpoint{3.153076in}{0.802001in}}%
\pgfpathlineto{\pgfqpoint{3.155880in}{1.085251in}}%
\pgfpathlineto{\pgfqpoint{3.158683in}{0.916130in}}%
\pgfpathlineto{\pgfqpoint{3.161486in}{0.805840in}}%
\pgfpathlineto{\pgfqpoint{3.164289in}{0.843968in}}%
\pgfpathlineto{\pgfqpoint{3.167093in}{0.791433in}}%
\pgfpathlineto{\pgfqpoint{3.169896in}{0.789750in}}%
\pgfpathlineto{\pgfqpoint{3.172699in}{0.730416in}}%
\pgfpathlineto{\pgfqpoint{3.175502in}{0.687917in}}%
\pgfpathlineto{\pgfqpoint{3.178305in}{0.699369in}}%
\pgfpathlineto{\pgfqpoint{3.181109in}{0.670955in}}%
\pgfpathlineto{\pgfqpoint{3.183912in}{0.652400in}}%
\pgfpathlineto{\pgfqpoint{3.186715in}{0.639430in}}%
\pgfpathlineto{\pgfqpoint{3.189518in}{0.714300in}}%
\pgfpathlineto{\pgfqpoint{3.192322in}{0.686937in}}%
\pgfpathlineto{\pgfqpoint{3.195125in}{0.681246in}}%
\pgfpathlineto{\pgfqpoint{3.197928in}{0.661767in}}%
\pgfpathlineto{\pgfqpoint{3.200731in}{0.678111in}}%
\pgfpathlineto{\pgfqpoint{3.203535in}{0.657540in}}%
\pgfpathlineto{\pgfqpoint{3.206338in}{1.359966in}}%
\pgfpathlineto{\pgfqpoint{3.209141in}{1.109302in}}%
\pgfpathlineto{\pgfqpoint{3.211944in}{0.946645in}}%
\pgfpathlineto{\pgfqpoint{3.214747in}{0.821455in}}%
\pgfpathlineto{\pgfqpoint{3.217551in}{0.742488in}}%
\pgfpathlineto{\pgfqpoint{3.220354in}{0.737610in}}%
\pgfpathlineto{\pgfqpoint{3.223157in}{0.690066in}}%
\pgfpathlineto{\pgfqpoint{3.225960in}{0.679897in}}%
\pgfpathlineto{\pgfqpoint{3.231567in}{0.644837in}}%
\pgfpathlineto{\pgfqpoint{3.234370in}{0.643403in}}%
\pgfpathlineto{\pgfqpoint{3.237173in}{0.635285in}}%
\pgfpathlineto{\pgfqpoint{3.239977in}{0.665943in}}%
\pgfpathlineto{\pgfqpoint{3.242780in}{0.658198in}}%
\pgfpathlineto{\pgfqpoint{3.245583in}{0.647338in}}%
\pgfpathlineto{\pgfqpoint{3.248386in}{0.643762in}}%
\pgfpathlineto{\pgfqpoint{3.253993in}{0.684196in}}%
\pgfpathlineto{\pgfqpoint{3.259599in}{0.652676in}}%
\pgfpathlineto{\pgfqpoint{3.262402in}{0.648480in}}%
\pgfpathlineto{\pgfqpoint{3.265206in}{0.661708in}}%
\pgfpathlineto{\pgfqpoint{3.268009in}{0.657705in}}%
\pgfpathlineto{\pgfqpoint{3.270812in}{0.652591in}}%
\pgfpathlineto{\pgfqpoint{3.273615in}{0.684093in}}%
\pgfpathlineto{\pgfqpoint{3.276419in}{0.937369in}}%
\pgfpathlineto{\pgfqpoint{3.279222in}{0.868165in}}%
\pgfpathlineto{\pgfqpoint{3.282025in}{0.765219in}}%
\pgfpathlineto{\pgfqpoint{3.284828in}{0.710301in}}%
\pgfpathlineto{\pgfqpoint{3.287631in}{0.689957in}}%
\pgfpathlineto{\pgfqpoint{3.290435in}{0.691562in}}%
\pgfpathlineto{\pgfqpoint{3.296041in}{0.662313in}}%
\pgfpathlineto{\pgfqpoint{3.298844in}{0.690021in}}%
\pgfpathlineto{\pgfqpoint{3.301648in}{0.666426in}}%
\pgfpathlineto{\pgfqpoint{3.304451in}{0.652469in}}%
\pgfpathlineto{\pgfqpoint{3.307254in}{0.642150in}}%
\pgfpathlineto{\pgfqpoint{3.310057in}{0.649522in}}%
\pgfpathlineto{\pgfqpoint{3.312861in}{0.646358in}}%
\pgfpathlineto{\pgfqpoint{3.315664in}{0.637757in}}%
\pgfpathlineto{\pgfqpoint{3.318467in}{0.633567in}}%
\pgfpathlineto{\pgfqpoint{3.321270in}{0.636918in}}%
\pgfpathlineto{\pgfqpoint{3.324073in}{0.635576in}}%
\pgfpathlineto{\pgfqpoint{3.326877in}{0.630486in}}%
\pgfpathlineto{\pgfqpoint{3.329680in}{0.666389in}}%
\pgfpathlineto{\pgfqpoint{3.332483in}{0.652715in}}%
\pgfpathlineto{\pgfqpoint{3.335286in}{0.682957in}}%
\pgfpathlineto{\pgfqpoint{3.338090in}{0.690872in}}%
\pgfpathlineto{\pgfqpoint{3.340893in}{0.686601in}}%
\pgfpathlineto{\pgfqpoint{3.343696in}{0.701329in}}%
\pgfpathlineto{\pgfqpoint{3.346499in}{0.683568in}}%
\pgfpathlineto{\pgfqpoint{3.349303in}{0.736513in}}%
\pgfpathlineto{\pgfqpoint{3.352106in}{0.689014in}}%
\pgfpathlineto{\pgfqpoint{3.354909in}{0.882049in}}%
\pgfpathlineto{\pgfqpoint{3.357712in}{0.776528in}}%
\pgfpathlineto{\pgfqpoint{3.360515in}{0.730856in}}%
\pgfpathlineto{\pgfqpoint{3.363319in}{0.694924in}}%
\pgfpathlineto{\pgfqpoint{3.366122in}{0.739870in}}%
\pgfpathlineto{\pgfqpoint{3.368925in}{0.994268in}}%
\pgfpathlineto{\pgfqpoint{3.371728in}{0.848856in}}%
\pgfpathlineto{\pgfqpoint{3.374532in}{0.758289in}}%
\pgfpathlineto{\pgfqpoint{3.377335in}{0.712958in}}%
\pgfpathlineto{\pgfqpoint{3.380138in}{0.681874in}}%
\pgfpathlineto{\pgfqpoint{3.382941in}{0.657092in}}%
\pgfpathlineto{\pgfqpoint{3.385745in}{0.645312in}}%
\pgfpathlineto{\pgfqpoint{3.388548in}{0.678953in}}%
\pgfpathlineto{\pgfqpoint{3.391351in}{0.692965in}}%
\pgfpathlineto{\pgfqpoint{3.394154in}{0.858172in}}%
\pgfpathlineto{\pgfqpoint{3.396957in}{0.760319in}}%
\pgfpathlineto{\pgfqpoint{3.399761in}{0.763732in}}%
\pgfpathlineto{\pgfqpoint{3.402564in}{0.838464in}}%
\pgfpathlineto{\pgfqpoint{3.405367in}{0.804228in}}%
\pgfpathlineto{\pgfqpoint{3.408170in}{0.735842in}}%
\pgfpathlineto{\pgfqpoint{3.410974in}{0.692928in}}%
\pgfpathlineto{\pgfqpoint{3.413777in}{0.828673in}}%
\pgfpathlineto{\pgfqpoint{3.416580in}{0.749864in}}%
\pgfpathlineto{\pgfqpoint{3.419383in}{0.934133in}}%
\pgfpathlineto{\pgfqpoint{3.422187in}{0.828065in}}%
\pgfpathlineto{\pgfqpoint{3.424990in}{0.783727in}}%
\pgfpathlineto{\pgfqpoint{3.427793in}{0.726556in}}%
\pgfpathlineto{\pgfqpoint{3.430596in}{0.695710in}}%
\pgfpathlineto{\pgfqpoint{3.433400in}{0.725291in}}%
\pgfpathlineto{\pgfqpoint{3.436203in}{0.686723in}}%
\pgfpathlineto{\pgfqpoint{3.439006in}{0.721144in}}%
\pgfpathlineto{\pgfqpoint{3.441809in}{0.841061in}}%
\pgfpathlineto{\pgfqpoint{3.444612in}{0.918212in}}%
\pgfpathlineto{\pgfqpoint{3.447416in}{0.785628in}}%
\pgfpathlineto{\pgfqpoint{3.450219in}{0.787310in}}%
\pgfpathlineto{\pgfqpoint{3.453022in}{0.726985in}}%
\pgfpathlineto{\pgfqpoint{3.455825in}{0.719787in}}%
\pgfpathlineto{\pgfqpoint{3.458629in}{0.976922in}}%
\pgfpathlineto{\pgfqpoint{3.461432in}{1.942414in}}%
\pgfpathlineto{\pgfqpoint{3.464235in}{1.395920in}}%
\pgfpathlineto{\pgfqpoint{3.467038in}{1.121582in}}%
\pgfpathlineto{\pgfqpoint{3.469842in}{1.157281in}}%
\pgfpathlineto{\pgfqpoint{3.475448in}{0.839335in}}%
\pgfpathlineto{\pgfqpoint{3.478251in}{0.761765in}}%
\pgfpathlineto{\pgfqpoint{3.481054in}{0.754094in}}%
\pgfpathlineto{\pgfqpoint{3.483858in}{0.697101in}}%
\pgfpathlineto{\pgfqpoint{3.486661in}{0.686729in}}%
\pgfpathlineto{\pgfqpoint{3.489464in}{0.722696in}}%
\pgfpathlineto{\pgfqpoint{3.492267in}{0.692473in}}%
\pgfpathlineto{\pgfqpoint{3.495071in}{0.676554in}}%
\pgfpathlineto{\pgfqpoint{3.497874in}{0.677443in}}%
\pgfpathlineto{\pgfqpoint{3.500677in}{0.665634in}}%
\pgfpathlineto{\pgfqpoint{3.503480in}{0.672825in}}%
\pgfpathlineto{\pgfqpoint{3.506284in}{0.662186in}}%
\pgfpathlineto{\pgfqpoint{3.509087in}{0.658403in}}%
\pgfpathlineto{\pgfqpoint{3.511890in}{0.644458in}}%
\pgfpathlineto{\pgfqpoint{3.514693in}{0.886236in}}%
\pgfpathlineto{\pgfqpoint{3.517496in}{0.918950in}}%
\pgfpathlineto{\pgfqpoint{3.520300in}{0.809371in}}%
\pgfpathlineto{\pgfqpoint{3.523103in}{0.759030in}}%
\pgfpathlineto{\pgfqpoint{3.528709in}{0.685879in}}%
\pgfpathlineto{\pgfqpoint{3.531513in}{0.661637in}}%
\pgfpathlineto{\pgfqpoint{3.534316in}{0.646794in}}%
\pgfpathlineto{\pgfqpoint{3.537119in}{0.639701in}}%
\pgfpathlineto{\pgfqpoint{3.542726in}{0.629416in}}%
\pgfpathlineto{\pgfqpoint{3.545529in}{0.673328in}}%
\pgfpathlineto{\pgfqpoint{3.551135in}{0.643717in}}%
\pgfpathlineto{\pgfqpoint{3.553938in}{0.635681in}}%
\pgfpathlineto{\pgfqpoint{3.556742in}{0.637064in}}%
\pgfpathlineto{\pgfqpoint{3.559545in}{0.639260in}}%
\pgfpathlineto{\pgfqpoint{3.562348in}{0.640128in}}%
\pgfpathlineto{\pgfqpoint{3.565151in}{0.638025in}}%
\pgfpathlineto{\pgfqpoint{3.567955in}{0.631711in}}%
\pgfpathlineto{\pgfqpoint{3.570758in}{0.675591in}}%
\pgfpathlineto{\pgfqpoint{3.573561in}{0.660978in}}%
\pgfpathlineto{\pgfqpoint{3.576364in}{0.672435in}}%
\pgfpathlineto{\pgfqpoint{3.579168in}{0.653628in}}%
\pgfpathlineto{\pgfqpoint{3.584774in}{0.638062in}}%
\pgfpathlineto{\pgfqpoint{3.587577in}{0.636698in}}%
\pgfpathlineto{\pgfqpoint{3.590380in}{0.631988in}}%
\pgfpathlineto{\pgfqpoint{3.593184in}{0.632708in}}%
\pgfpathlineto{\pgfqpoint{3.595987in}{0.634939in}}%
\pgfpathlineto{\pgfqpoint{3.598790in}{0.660398in}}%
\pgfpathlineto{\pgfqpoint{3.601593in}{0.645669in}}%
\pgfpathlineto{\pgfqpoint{3.604397in}{0.645321in}}%
\pgfpathlineto{\pgfqpoint{3.607200in}{0.795744in}}%
\pgfpathlineto{\pgfqpoint{3.610003in}{0.785828in}}%
\pgfpathlineto{\pgfqpoint{3.612806in}{0.723005in}}%
\pgfpathlineto{\pgfqpoint{3.615610in}{0.684827in}}%
\pgfpathlineto{\pgfqpoint{3.618413in}{0.729935in}}%
\pgfpathlineto{\pgfqpoint{3.624019in}{0.677505in}}%
\pgfpathlineto{\pgfqpoint{3.626822in}{0.750477in}}%
\pgfpathlineto{\pgfqpoint{3.629626in}{0.704650in}}%
\pgfpathlineto{\pgfqpoint{3.632429in}{0.674486in}}%
\pgfpathlineto{\pgfqpoint{3.635232in}{0.655241in}}%
\pgfpathlineto{\pgfqpoint{3.638035in}{0.643329in}}%
\pgfpathlineto{\pgfqpoint{3.640839in}{0.638510in}}%
\pgfpathlineto{\pgfqpoint{3.643642in}{0.800806in}}%
\pgfpathlineto{\pgfqpoint{3.646445in}{0.735215in}}%
\pgfpathlineto{\pgfqpoint{3.649248in}{0.758792in}}%
\pgfpathlineto{\pgfqpoint{3.652052in}{0.712828in}}%
\pgfpathlineto{\pgfqpoint{3.654855in}{0.738573in}}%
\pgfpathlineto{\pgfqpoint{3.657658in}{0.698138in}}%
\pgfpathlineto{\pgfqpoint{3.660461in}{0.689718in}}%
\pgfpathlineto{\pgfqpoint{3.663264in}{0.667333in}}%
\pgfpathlineto{\pgfqpoint{3.666068in}{0.664535in}}%
\pgfpathlineto{\pgfqpoint{3.668871in}{0.665422in}}%
\pgfpathlineto{\pgfqpoint{3.671674in}{0.656750in}}%
\pgfpathlineto{\pgfqpoint{3.674477in}{0.645144in}}%
\pgfpathlineto{\pgfqpoint{3.677281in}{0.638128in}}%
\pgfpathlineto{\pgfqpoint{3.680084in}{0.662402in}}%
\pgfpathlineto{\pgfqpoint{3.682887in}{0.647925in}}%
\pgfpathlineto{\pgfqpoint{3.688494in}{0.633322in}}%
\pgfpathlineto{\pgfqpoint{3.694100in}{0.625641in}}%
\pgfpathlineto{\pgfqpoint{3.696903in}{0.626810in}}%
\pgfpathlineto{\pgfqpoint{3.699706in}{0.633112in}}%
\pgfpathlineto{\pgfqpoint{3.702510in}{0.688637in}}%
\pgfpathlineto{\pgfqpoint{3.705313in}{0.840124in}}%
\pgfpathlineto{\pgfqpoint{3.708116in}{0.755919in}}%
\pgfpathlineto{\pgfqpoint{3.713723in}{0.683286in}}%
\pgfpathlineto{\pgfqpoint{3.716526in}{0.761023in}}%
\pgfpathlineto{\pgfqpoint{3.719329in}{0.712106in}}%
\pgfpathlineto{\pgfqpoint{3.722132in}{0.680573in}}%
\pgfpathlineto{\pgfqpoint{3.724936in}{0.695928in}}%
\pgfpathlineto{\pgfqpoint{3.727739in}{0.679109in}}%
\pgfpathlineto{\pgfqpoint{3.730542in}{1.164257in}}%
\pgfpathlineto{\pgfqpoint{3.733345in}{1.155351in}}%
\pgfpathlineto{\pgfqpoint{3.736148in}{0.987877in}}%
\pgfpathlineto{\pgfqpoint{3.738952in}{0.872664in}}%
\pgfpathlineto{\pgfqpoint{3.741755in}{1.151730in}}%
\pgfpathlineto{\pgfqpoint{3.744558in}{0.985109in}}%
\pgfpathlineto{\pgfqpoint{3.747361in}{0.866624in}}%
\pgfpathlineto{\pgfqpoint{3.752968in}{0.736574in}}%
\pgfpathlineto{\pgfqpoint{3.755771in}{0.731319in}}%
\pgfpathlineto{\pgfqpoint{3.758574in}{0.688533in}}%
\pgfpathlineto{\pgfqpoint{3.764181in}{0.650844in}}%
\pgfpathlineto{\pgfqpoint{3.769787in}{0.633697in}}%
\pgfpathlineto{\pgfqpoint{3.772590in}{0.628818in}}%
\pgfpathlineto{\pgfqpoint{3.775394in}{0.625856in}}%
\pgfpathlineto{\pgfqpoint{3.778197in}{0.624616in}}%
\pgfpathlineto{\pgfqpoint{3.781000in}{0.641448in}}%
\pgfpathlineto{\pgfqpoint{3.783803in}{0.637976in}}%
\pgfpathlineto{\pgfqpoint{3.786607in}{0.683006in}}%
\pgfpathlineto{\pgfqpoint{3.789410in}{0.673535in}}%
\pgfpathlineto{\pgfqpoint{3.792213in}{0.654771in}}%
\pgfpathlineto{\pgfqpoint{3.795016in}{0.643815in}}%
\pgfpathlineto{\pgfqpoint{3.797820in}{0.705191in}}%
\pgfpathlineto{\pgfqpoint{3.800623in}{0.675657in}}%
\pgfpathlineto{\pgfqpoint{3.803426in}{0.656586in}}%
\pgfpathlineto{\pgfqpoint{3.806229in}{0.644208in}}%
\pgfpathlineto{\pgfqpoint{3.809032in}{0.637246in}}%
\pgfpathlineto{\pgfqpoint{3.811836in}{0.631966in}}%
\pgfpathlineto{\pgfqpoint{3.814639in}{0.629933in}}%
\pgfpathlineto{\pgfqpoint{3.817442in}{0.625181in}}%
\pgfpathlineto{\pgfqpoint{3.820245in}{0.630846in}}%
\pgfpathlineto{\pgfqpoint{3.823049in}{0.625954in}}%
\pgfpathlineto{\pgfqpoint{3.825852in}{0.623039in}}%
\pgfpathlineto{\pgfqpoint{3.828655in}{0.622889in}}%
\pgfpathlineto{\pgfqpoint{3.831458in}{0.620693in}}%
\pgfpathlineto{\pgfqpoint{3.834262in}{0.617560in}}%
\pgfpathlineto{\pgfqpoint{3.837065in}{0.633787in}}%
\pgfpathlineto{\pgfqpoint{3.839868in}{0.626162in}}%
\pgfpathlineto{\pgfqpoint{3.842671in}{0.622078in}}%
\pgfpathlineto{\pgfqpoint{3.845474in}{0.622354in}}%
\pgfpathlineto{\pgfqpoint{3.848278in}{0.632700in}}%
\pgfpathlineto{\pgfqpoint{3.851081in}{0.624763in}}%
\pgfpathlineto{\pgfqpoint{3.853884in}{0.619111in}}%
\pgfpathlineto{\pgfqpoint{3.859491in}{0.611669in}}%
\pgfpathlineto{\pgfqpoint{3.862294in}{0.613590in}}%
\pgfpathlineto{\pgfqpoint{3.865097in}{0.610434in}}%
\pgfpathlineto{\pgfqpoint{3.867900in}{0.609710in}}%
\pgfpathlineto{\pgfqpoint{3.870704in}{0.607934in}}%
\pgfpathlineto{\pgfqpoint{3.873507in}{0.607702in}}%
\pgfpathlineto{\pgfqpoint{3.876310in}{0.609883in}}%
\pgfpathlineto{\pgfqpoint{3.879113in}{0.729397in}}%
\pgfpathlineto{\pgfqpoint{3.881917in}{0.706470in}}%
\pgfpathlineto{\pgfqpoint{3.884720in}{0.732126in}}%
\pgfpathlineto{\pgfqpoint{3.890326in}{0.675710in}}%
\pgfpathlineto{\pgfqpoint{3.893129in}{0.654894in}}%
\pgfpathlineto{\pgfqpoint{3.895933in}{0.640573in}}%
\pgfpathlineto{\pgfqpoint{3.898736in}{0.630049in}}%
\pgfpathlineto{\pgfqpoint{3.904342in}{0.617907in}}%
\pgfpathlineto{\pgfqpoint{3.907146in}{0.647424in}}%
\pgfpathlineto{\pgfqpoint{3.909949in}{0.661304in}}%
\pgfpathlineto{\pgfqpoint{3.912752in}{0.644167in}}%
\pgfpathlineto{\pgfqpoint{3.915555in}{0.634505in}}%
\pgfpathlineto{\pgfqpoint{3.918359in}{0.677137in}}%
\pgfpathlineto{\pgfqpoint{3.923965in}{0.641488in}}%
\pgfpathlineto{\pgfqpoint{3.926768in}{0.631658in}}%
\pgfpathlineto{\pgfqpoint{3.932375in}{0.617514in}}%
\pgfpathlineto{\pgfqpoint{3.935178in}{0.624896in}}%
\pgfpathlineto{\pgfqpoint{3.937981in}{0.617532in}}%
\pgfpathlineto{\pgfqpoint{3.940784in}{0.667943in}}%
\pgfpathlineto{\pgfqpoint{3.943588in}{0.647995in}}%
\pgfpathlineto{\pgfqpoint{3.946391in}{0.642314in}}%
\pgfpathlineto{\pgfqpoint{3.949194in}{0.631355in}}%
\pgfpathlineto{\pgfqpoint{3.951997in}{0.627725in}}%
\pgfpathlineto{\pgfqpoint{3.954801in}{0.636216in}}%
\pgfpathlineto{\pgfqpoint{3.960407in}{0.629002in}}%
\pgfpathlineto{\pgfqpoint{3.963210in}{0.621924in}}%
\pgfpathlineto{\pgfqpoint{3.966013in}{0.617906in}}%
\pgfpathlineto{\pgfqpoint{3.968817in}{0.671997in}}%
\pgfpathlineto{\pgfqpoint{3.974423in}{0.639713in}}%
\pgfpathlineto{\pgfqpoint{3.977226in}{0.628213in}}%
\pgfpathlineto{\pgfqpoint{3.980030in}{0.619605in}}%
\pgfpathlineto{\pgfqpoint{3.982833in}{0.649735in}}%
\pgfpathlineto{\pgfqpoint{3.985636in}{0.686072in}}%
\pgfpathlineto{\pgfqpoint{3.988439in}{0.661387in}}%
\pgfpathlineto{\pgfqpoint{3.991243in}{0.648866in}}%
\pgfpathlineto{\pgfqpoint{3.994046in}{0.656567in}}%
\pgfpathlineto{\pgfqpoint{3.996849in}{0.640979in}}%
\pgfpathlineto{\pgfqpoint{3.999652in}{0.629512in}}%
\pgfpathlineto{\pgfqpoint{4.002455in}{0.701102in}}%
\pgfpathlineto{\pgfqpoint{4.005259in}{0.673364in}}%
\pgfpathlineto{\pgfqpoint{4.008062in}{1.192272in}}%
\pgfpathlineto{\pgfqpoint{4.010865in}{1.022997in}}%
\pgfpathlineto{\pgfqpoint{4.016472in}{0.845287in}}%
\pgfpathlineto{\pgfqpoint{4.019275in}{0.782723in}}%
\pgfpathlineto{\pgfqpoint{4.022078in}{0.736049in}}%
\pgfpathlineto{\pgfqpoint{4.024881in}{0.836703in}}%
\pgfpathlineto{\pgfqpoint{4.030488in}{0.735668in}}%
\pgfpathlineto{\pgfqpoint{4.033291in}{0.750846in}}%
\pgfpathlineto{\pgfqpoint{4.036094in}{0.710718in}}%
\pgfpathlineto{\pgfqpoint{4.041701in}{0.942893in}}%
\pgfpathlineto{\pgfqpoint{4.044504in}{0.855158in}}%
\pgfpathlineto{\pgfqpoint{4.050110in}{0.754803in}}%
\pgfpathlineto{\pgfqpoint{4.052914in}{0.741695in}}%
\pgfpathlineto{\pgfqpoint{4.055717in}{0.705188in}}%
\pgfpathlineto{\pgfqpoint{4.058520in}{0.678147in}}%
\pgfpathlineto{\pgfqpoint{4.061323in}{0.677850in}}%
\pgfpathlineto{\pgfqpoint{4.066930in}{0.643441in}}%
\pgfpathlineto{\pgfqpoint{4.069733in}{0.631696in}}%
\pgfpathlineto{\pgfqpoint{4.072536in}{0.684844in}}%
\pgfpathlineto{\pgfqpoint{4.078143in}{0.649540in}}%
\pgfpathlineto{\pgfqpoint{4.080946in}{0.639672in}}%
\pgfpathlineto{\pgfqpoint{4.083749in}{0.632459in}}%
\pgfpathlineto{\pgfqpoint{4.086552in}{0.623434in}}%
\pgfpathlineto{\pgfqpoint{4.089356in}{0.616452in}}%
\pgfpathlineto{\pgfqpoint{4.094962in}{0.607000in}}%
\pgfpathlineto{\pgfqpoint{4.097765in}{0.609358in}}%
\pgfpathlineto{\pgfqpoint{4.100569in}{0.616901in}}%
\pgfpathlineto{\pgfqpoint{4.103372in}{0.613159in}}%
\pgfpathlineto{\pgfqpoint{4.106175in}{0.612643in}}%
\pgfpathlineto{\pgfqpoint{4.108978in}{0.614162in}}%
\pgfpathlineto{\pgfqpoint{4.111781in}{0.620189in}}%
\pgfpathlineto{\pgfqpoint{4.114585in}{0.623809in}}%
\pgfpathlineto{\pgfqpoint{4.117388in}{0.617216in}}%
\pgfpathlineto{\pgfqpoint{4.120191in}{0.620975in}}%
\pgfpathlineto{\pgfqpoint{4.122994in}{0.616808in}}%
\pgfpathlineto{\pgfqpoint{4.125798in}{0.610625in}}%
\pgfpathlineto{\pgfqpoint{4.131404in}{0.602218in}}%
\pgfpathlineto{\pgfqpoint{4.134207in}{0.598901in}}%
\pgfpathlineto{\pgfqpoint{4.137011in}{0.596455in}}%
\pgfpathlineto{\pgfqpoint{4.139814in}{0.600558in}}%
\pgfpathlineto{\pgfqpoint{4.142617in}{0.599674in}}%
\pgfpathlineto{\pgfqpoint{4.145420in}{0.617303in}}%
\pgfpathlineto{\pgfqpoint{4.151027in}{0.606151in}}%
\pgfpathlineto{\pgfqpoint{4.153830in}{0.644152in}}%
\pgfpathlineto{\pgfqpoint{4.159436in}{0.624782in}}%
\pgfpathlineto{\pgfqpoint{4.162240in}{0.679377in}}%
\pgfpathlineto{\pgfqpoint{4.167846in}{0.640943in}}%
\pgfpathlineto{\pgfqpoint{4.170649in}{0.649142in}}%
\pgfpathlineto{\pgfqpoint{4.173453in}{0.635861in}}%
\pgfpathlineto{\pgfqpoint{4.176256in}{0.625409in}}%
\pgfpathlineto{\pgfqpoint{4.181862in}{0.612031in}}%
\pgfpathlineto{\pgfqpoint{4.187469in}{0.606003in}}%
\pgfpathlineto{\pgfqpoint{4.190272in}{0.601960in}}%
\pgfpathlineto{\pgfqpoint{4.193075in}{0.599251in}}%
\pgfpathlineto{\pgfqpoint{4.195878in}{0.597647in}}%
\pgfpathlineto{\pgfqpoint{4.201485in}{0.592912in}}%
\pgfpathlineto{\pgfqpoint{4.204288in}{0.599531in}}%
\pgfpathlineto{\pgfqpoint{4.207091in}{0.596274in}}%
\pgfpathlineto{\pgfqpoint{4.209895in}{0.597428in}}%
\pgfpathlineto{\pgfqpoint{4.212698in}{0.603791in}}%
\pgfpathlineto{\pgfqpoint{4.218304in}{0.594917in}}%
\pgfpathlineto{\pgfqpoint{4.221107in}{0.591451in}}%
\pgfpathlineto{\pgfqpoint{4.223911in}{0.589537in}}%
\pgfpathlineto{\pgfqpoint{4.226714in}{0.586716in}}%
\pgfpathlineto{\pgfqpoint{4.229517in}{0.585834in}}%
\pgfpathlineto{\pgfqpoint{4.232320in}{0.583859in}}%
\pgfpathlineto{\pgfqpoint{4.235124in}{0.595462in}}%
\pgfpathlineto{\pgfqpoint{4.240730in}{0.587221in}}%
\pgfpathlineto{\pgfqpoint{4.243533in}{0.583829in}}%
\pgfpathlineto{\pgfqpoint{4.249140in}{0.662158in}}%
\pgfpathlineto{\pgfqpoint{4.251943in}{0.643952in}}%
\pgfpathlineto{\pgfqpoint{4.254746in}{0.629655in}}%
\pgfpathlineto{\pgfqpoint{4.257549in}{0.618452in}}%
\pgfpathlineto{\pgfqpoint{4.260353in}{0.612507in}}%
\pgfpathlineto{\pgfqpoint{4.265959in}{0.597715in}}%
\pgfpathlineto{\pgfqpoint{4.274369in}{0.587040in}}%
\pgfpathlineto{\pgfqpoint{4.277172in}{0.597344in}}%
\pgfpathlineto{\pgfqpoint{4.282779in}{0.586395in}}%
\pgfpathlineto{\pgfqpoint{4.285582in}{0.588000in}}%
\pgfpathlineto{\pgfqpoint{4.291188in}{0.579327in}}%
\pgfpathlineto{\pgfqpoint{4.293991in}{0.577587in}}%
\pgfpathlineto{\pgfqpoint{4.296795in}{0.574096in}}%
\pgfpathlineto{\pgfqpoint{4.299598in}{0.575887in}}%
\pgfpathlineto{\pgfqpoint{4.305204in}{0.570315in}}%
\pgfpathlineto{\pgfqpoint{4.308008in}{0.577913in}}%
\pgfpathlineto{\pgfqpoint{4.310811in}{0.574346in}}%
\pgfpathlineto{\pgfqpoint{4.313614in}{0.573511in}}%
\pgfpathlineto{\pgfqpoint{4.316417in}{0.570484in}}%
\pgfpathlineto{\pgfqpoint{4.319221in}{0.571873in}}%
\pgfpathlineto{\pgfqpoint{4.322024in}{0.571529in}}%
\pgfpathlineto{\pgfqpoint{4.324827in}{0.574847in}}%
\pgfpathlineto{\pgfqpoint{4.327630in}{0.570730in}}%
\pgfpathlineto{\pgfqpoint{4.330434in}{0.575498in}}%
\pgfpathlineto{\pgfqpoint{4.336040in}{0.568604in}}%
\pgfpathlineto{\pgfqpoint{4.338843in}{0.577067in}}%
\pgfpathlineto{\pgfqpoint{4.341646in}{0.574282in}}%
\pgfpathlineto{\pgfqpoint{4.347253in}{0.566090in}}%
\pgfpathlineto{\pgfqpoint{4.358466in}{0.558500in}}%
\pgfpathlineto{\pgfqpoint{4.361269in}{0.607933in}}%
\pgfpathlineto{\pgfqpoint{4.364072in}{0.597792in}}%
\pgfpathlineto{\pgfqpoint{4.366876in}{0.591742in}}%
\pgfpathlineto{\pgfqpoint{4.372482in}{0.581404in}}%
\pgfpathlineto{\pgfqpoint{4.375285in}{0.579165in}}%
\pgfpathlineto{\pgfqpoint{4.378088in}{0.573736in}}%
\pgfpathlineto{\pgfqpoint{4.380892in}{0.574923in}}%
\pgfpathlineto{\pgfqpoint{4.383695in}{0.569845in}}%
\pgfpathlineto{\pgfqpoint{4.389301in}{0.562335in}}%
\pgfpathlineto{\pgfqpoint{4.392105in}{0.561202in}}%
\pgfpathlineto{\pgfqpoint{4.394908in}{0.558239in}}%
\pgfpathlineto{\pgfqpoint{4.397711in}{0.575179in}}%
\pgfpathlineto{\pgfqpoint{4.403318in}{0.564946in}}%
\pgfpathlineto{\pgfqpoint{4.406121in}{0.631737in}}%
\pgfpathlineto{\pgfqpoint{4.408924in}{0.655013in}}%
\pgfpathlineto{\pgfqpoint{4.411727in}{0.647421in}}%
\pgfpathlineto{\pgfqpoint{4.414530in}{0.630686in}}%
\pgfpathlineto{\pgfqpoint{4.417334in}{0.659389in}}%
\pgfpathlineto{\pgfqpoint{4.420137in}{0.641032in}}%
\pgfpathlineto{\pgfqpoint{4.422940in}{0.626935in}}%
\pgfpathlineto{\pgfqpoint{4.425743in}{0.620460in}}%
\pgfpathlineto{\pgfqpoint{4.428547in}{0.608616in}}%
\pgfpathlineto{\pgfqpoint{4.431350in}{0.610705in}}%
\pgfpathlineto{\pgfqpoint{4.434153in}{0.609290in}}%
\pgfpathlineto{\pgfqpoint{4.439760in}{0.599347in}}%
\pgfpathlineto{\pgfqpoint{4.442563in}{0.596524in}}%
\pgfpathlineto{\pgfqpoint{4.445366in}{0.682840in}}%
\pgfpathlineto{\pgfqpoint{4.448169in}{0.675062in}}%
\pgfpathlineto{\pgfqpoint{4.450972in}{0.661873in}}%
\pgfpathlineto{\pgfqpoint{4.453776in}{0.645061in}}%
\pgfpathlineto{\pgfqpoint{4.456579in}{0.648208in}}%
\pgfpathlineto{\pgfqpoint{4.459382in}{0.633677in}}%
\pgfpathlineto{\pgfqpoint{4.462185in}{0.624368in}}%
\pgfpathlineto{\pgfqpoint{4.464989in}{0.612477in}}%
\pgfpathlineto{\pgfqpoint{4.467792in}{0.603832in}}%
\pgfpathlineto{\pgfqpoint{4.476202in}{0.585593in}}%
\pgfpathlineto{\pgfqpoint{4.481808in}{0.576288in}}%
\pgfpathlineto{\pgfqpoint{4.484611in}{0.572342in}}%
\pgfpathlineto{\pgfqpoint{4.487414in}{0.575163in}}%
\pgfpathlineto{\pgfqpoint{4.490218in}{0.571947in}}%
\pgfpathlineto{\pgfqpoint{4.493021in}{0.572613in}}%
\pgfpathlineto{\pgfqpoint{4.495824in}{0.570154in}}%
\pgfpathlineto{\pgfqpoint{4.498627in}{0.610429in}}%
\pgfpathlineto{\pgfqpoint{4.501431in}{0.606122in}}%
\pgfpathlineto{\pgfqpoint{4.504234in}{0.596804in}}%
\pgfpathlineto{\pgfqpoint{4.507037in}{0.593878in}}%
\pgfpathlineto{\pgfqpoint{4.509840in}{0.586507in}}%
\pgfpathlineto{\pgfqpoint{4.512644in}{0.593935in}}%
\pgfpathlineto{\pgfqpoint{4.515447in}{0.586290in}}%
\pgfpathlineto{\pgfqpoint{4.518250in}{0.580530in}}%
\pgfpathlineto{\pgfqpoint{4.521053in}{0.576620in}}%
\pgfpathlineto{\pgfqpoint{4.523856in}{0.575001in}}%
\pgfpathlineto{\pgfqpoint{4.526660in}{0.627317in}}%
\pgfpathlineto{\pgfqpoint{4.529463in}{0.615235in}}%
\pgfpathlineto{\pgfqpoint{4.532266in}{0.613838in}}%
\pgfpathlineto{\pgfqpoint{4.537873in}{0.597048in}}%
\pgfpathlineto{\pgfqpoint{4.540676in}{0.623108in}}%
\pgfpathlineto{\pgfqpoint{4.543479in}{0.611143in}}%
\pgfpathlineto{\pgfqpoint{4.546282in}{0.602105in}}%
\pgfpathlineto{\pgfqpoint{4.549086in}{0.594992in}}%
\pgfpathlineto{\pgfqpoint{4.551889in}{0.607931in}}%
\pgfpathlineto{\pgfqpoint{4.557495in}{0.589934in}}%
\pgfpathlineto{\pgfqpoint{4.563102in}{0.577093in}}%
\pgfpathlineto{\pgfqpoint{4.565905in}{0.571951in}}%
\pgfpathlineto{\pgfqpoint{4.568708in}{0.569085in}}%
\pgfpathlineto{\pgfqpoint{4.571511in}{0.564741in}}%
\pgfpathlineto{\pgfqpoint{4.574315in}{0.581636in}}%
\pgfpathlineto{\pgfqpoint{4.577118in}{0.575181in}}%
\pgfpathlineto{\pgfqpoint{4.579921in}{0.575554in}}%
\pgfpathlineto{\pgfqpoint{4.582724in}{0.571125in}}%
\pgfpathlineto{\pgfqpoint{4.585528in}{0.574954in}}%
\pgfpathlineto{\pgfqpoint{4.588331in}{0.569546in}}%
\pgfpathlineto{\pgfqpoint{4.591134in}{0.569531in}}%
\pgfpathlineto{\pgfqpoint{4.593937in}{0.564838in}}%
\pgfpathlineto{\pgfqpoint{4.596740in}{0.576878in}}%
\pgfpathlineto{\pgfqpoint{4.599544in}{0.577942in}}%
\pgfpathlineto{\pgfqpoint{4.602347in}{0.572741in}}%
\pgfpathlineto{\pgfqpoint{4.605150in}{0.573346in}}%
\pgfpathlineto{\pgfqpoint{4.610757in}{0.563319in}}%
\pgfpathlineto{\pgfqpoint{4.613560in}{0.704318in}}%
\pgfpathlineto{\pgfqpoint{4.616363in}{0.679294in}}%
\pgfpathlineto{\pgfqpoint{4.621970in}{0.644650in}}%
\pgfpathlineto{\pgfqpoint{4.627576in}{0.617778in}}%
\pgfpathlineto{\pgfqpoint{4.630379in}{0.606778in}}%
\pgfpathlineto{\pgfqpoint{4.644395in}{0.573223in}}%
\pgfpathlineto{\pgfqpoint{4.650002in}{0.569858in}}%
\pgfpathlineto{\pgfqpoint{4.652805in}{0.565588in}}%
\pgfpathlineto{\pgfqpoint{4.655608in}{0.565143in}}%
\pgfpathlineto{\pgfqpoint{4.658412in}{0.574955in}}%
\pgfpathlineto{\pgfqpoint{4.661215in}{0.569310in}}%
\pgfpathlineto{\pgfqpoint{4.664018in}{0.579242in}}%
\pgfpathlineto{\pgfqpoint{4.669624in}{0.568130in}}%
\pgfpathlineto{\pgfqpoint{4.672428in}{0.565262in}}%
\pgfpathlineto{\pgfqpoint{4.675231in}{0.563852in}}%
\pgfpathlineto{\pgfqpoint{4.678034in}{0.559554in}}%
\pgfpathlineto{\pgfqpoint{4.686444in}{0.549860in}}%
\pgfpathlineto{\pgfqpoint{4.689247in}{0.550494in}}%
\pgfpathlineto{\pgfqpoint{4.703263in}{0.538399in}}%
\pgfpathlineto{\pgfqpoint{4.706066in}{0.539628in}}%
\pgfpathlineto{\pgfqpoint{4.708870in}{0.539651in}}%
\pgfpathlineto{\pgfqpoint{4.711673in}{0.543522in}}%
\pgfpathlineto{\pgfqpoint{4.714476in}{0.541298in}}%
\pgfpathlineto{\pgfqpoint{4.717279in}{0.600951in}}%
\pgfpathlineto{\pgfqpoint{4.720083in}{0.593353in}}%
\pgfpathlineto{\pgfqpoint{4.722886in}{0.616291in}}%
\pgfpathlineto{\pgfqpoint{4.725689in}{0.603710in}}%
\pgfpathlineto{\pgfqpoint{4.728492in}{0.599704in}}%
\pgfpathlineto{\pgfqpoint{4.731296in}{0.599626in}}%
\pgfpathlineto{\pgfqpoint{4.734099in}{0.593238in}}%
\pgfpathlineto{\pgfqpoint{4.736902in}{0.584706in}}%
\pgfpathlineto{\pgfqpoint{4.739705in}{0.579266in}}%
\pgfpathlineto{\pgfqpoint{4.742508in}{0.801769in}}%
\pgfpathlineto{\pgfqpoint{4.745312in}{0.789268in}}%
\pgfpathlineto{\pgfqpoint{4.748115in}{0.825978in}}%
\pgfpathlineto{\pgfqpoint{4.750918in}{1.019074in}}%
\pgfpathlineto{\pgfqpoint{4.756525in}{0.891387in}}%
\pgfpathlineto{\pgfqpoint{4.762131in}{0.805389in}}%
\pgfpathlineto{\pgfqpoint{4.767738in}{0.737250in}}%
\pgfpathlineto{\pgfqpoint{4.770541in}{0.708627in}}%
\pgfpathlineto{\pgfqpoint{4.778951in}{0.647724in}}%
\pgfpathlineto{\pgfqpoint{4.781754in}{0.674284in}}%
\pgfpathlineto{\pgfqpoint{4.787360in}{0.634792in}}%
\pgfpathlineto{\pgfqpoint{4.792967in}{0.604857in}}%
\pgfpathlineto{\pgfqpoint{4.795770in}{0.596782in}}%
\pgfpathlineto{\pgfqpoint{4.798573in}{0.586416in}}%
\pgfpathlineto{\pgfqpoint{4.798573in}{0.586416in}}%
\pgfusepath{stroke}%
\end{pgfscope}%
\begin{pgfscope}%
\pgfpathrectangle{\pgfqpoint{0.373953in}{0.331635in}}{\pgfqpoint{4.650000in}{3.020000in}}%
\pgfusepath{clip}%
\pgfsetroundcap%
\pgfsetroundjoin%
\pgfsetlinewidth{1.505625pt}%
\definecolor{currentstroke}{rgb}{1.000000,0.498039,0.054902}%
\pgfsetstrokecolor{currentstroke}%
\pgfsetstrokeopacity{0.600000}%
\pgfsetdash{}{0pt}%
\pgfpathmoveto{\pgfqpoint{0.585317in}{0.468908in}}%
\pgfpathlineto{\pgfqpoint{0.588120in}{0.690734in}}%
\pgfpathlineto{\pgfqpoint{0.590923in}{0.563670in}}%
\pgfpathlineto{\pgfqpoint{0.593726in}{0.614133in}}%
\pgfpathlineto{\pgfqpoint{0.596529in}{0.570455in}}%
\pgfpathlineto{\pgfqpoint{0.599333in}{0.480242in}}%
\pgfpathlineto{\pgfqpoint{0.602136in}{0.554250in}}%
\pgfpathlineto{\pgfqpoint{0.604939in}{0.739958in}}%
\pgfpathlineto{\pgfqpoint{0.607742in}{0.541733in}}%
\pgfpathlineto{\pgfqpoint{0.610546in}{0.686112in}}%
\pgfpathlineto{\pgfqpoint{0.613349in}{0.622532in}}%
\pgfpathlineto{\pgfqpoint{0.616152in}{0.639106in}}%
\pgfpathlineto{\pgfqpoint{0.618955in}{0.686405in}}%
\pgfpathlineto{\pgfqpoint{0.621759in}{0.593305in}}%
\pgfpathlineto{\pgfqpoint{0.624562in}{0.643837in}}%
\pgfpathlineto{\pgfqpoint{0.627365in}{0.525067in}}%
\pgfpathlineto{\pgfqpoint{0.632971in}{0.491331in}}%
\pgfpathlineto{\pgfqpoint{0.635775in}{0.615388in}}%
\pgfpathlineto{\pgfqpoint{0.638578in}{0.564879in}}%
\pgfpathlineto{\pgfqpoint{0.641381in}{0.697354in}}%
\pgfpathlineto{\pgfqpoint{0.644184in}{0.892210in}}%
\pgfpathlineto{\pgfqpoint{0.646988in}{0.580444in}}%
\pgfpathlineto{\pgfqpoint{0.649791in}{0.505864in}}%
\pgfpathlineto{\pgfqpoint{0.652594in}{0.511153in}}%
\pgfpathlineto{\pgfqpoint{0.655397in}{0.626589in}}%
\pgfpathlineto{\pgfqpoint{0.658201in}{0.925219in}}%
\pgfpathlineto{\pgfqpoint{0.661004in}{0.628950in}}%
\pgfpathlineto{\pgfqpoint{0.663807in}{0.598784in}}%
\pgfpathlineto{\pgfqpoint{0.666610in}{0.742057in}}%
\pgfpathlineto{\pgfqpoint{0.669413in}{0.552131in}}%
\pgfpathlineto{\pgfqpoint{0.672217in}{0.611215in}}%
\pgfpathlineto{\pgfqpoint{0.675020in}{0.601453in}}%
\pgfpathlineto{\pgfqpoint{0.677823in}{0.557114in}}%
\pgfpathlineto{\pgfqpoint{0.680626in}{0.702742in}}%
\pgfpathlineto{\pgfqpoint{0.683430in}{0.478556in}}%
\pgfpathlineto{\pgfqpoint{0.686233in}{0.584100in}}%
\pgfpathlineto{\pgfqpoint{0.689036in}{0.540777in}}%
\pgfpathlineto{\pgfqpoint{0.691839in}{0.687999in}}%
\pgfpathlineto{\pgfqpoint{0.694643in}{0.740970in}}%
\pgfpathlineto{\pgfqpoint{0.697446in}{0.555470in}}%
\pgfpathlineto{\pgfqpoint{0.700249in}{0.579622in}}%
\pgfpathlineto{\pgfqpoint{0.703052in}{0.478574in}}%
\pgfpathlineto{\pgfqpoint{0.705855in}{0.619597in}}%
\pgfpathlineto{\pgfqpoint{0.708659in}{0.556628in}}%
\pgfpathlineto{\pgfqpoint{0.711462in}{0.714126in}}%
\pgfpathlineto{\pgfqpoint{0.714265in}{0.554902in}}%
\pgfpathlineto{\pgfqpoint{0.717068in}{0.536207in}}%
\pgfpathlineto{\pgfqpoint{0.719872in}{0.550581in}}%
\pgfpathlineto{\pgfqpoint{0.725478in}{0.497739in}}%
\pgfpathlineto{\pgfqpoint{0.728281in}{0.502551in}}%
\pgfpathlineto{\pgfqpoint{0.731085in}{0.693064in}}%
\pgfpathlineto{\pgfqpoint{0.733888in}{0.712330in}}%
\pgfpathlineto{\pgfqpoint{0.736691in}{0.492985in}}%
\pgfpathlineto{\pgfqpoint{0.739494in}{0.526504in}}%
\pgfpathlineto{\pgfqpoint{0.742298in}{0.625929in}}%
\pgfpathlineto{\pgfqpoint{0.745101in}{0.600594in}}%
\pgfpathlineto{\pgfqpoint{0.747904in}{0.492321in}}%
\pgfpathlineto{\pgfqpoint{0.750707in}{0.520573in}}%
\pgfpathlineto{\pgfqpoint{0.753510in}{0.501898in}}%
\pgfpathlineto{\pgfqpoint{0.756314in}{0.573167in}}%
\pgfpathlineto{\pgfqpoint{0.759117in}{0.568447in}}%
\pgfpathlineto{\pgfqpoint{0.761920in}{0.614400in}}%
\pgfpathlineto{\pgfqpoint{0.764723in}{0.614400in}}%
\pgfpathlineto{\pgfqpoint{0.767527in}{0.679549in}}%
\pgfpathlineto{\pgfqpoint{0.770330in}{0.623027in}}%
\pgfpathlineto{\pgfqpoint{0.773133in}{0.758945in}}%
\pgfpathlineto{\pgfqpoint{0.775936in}{0.536077in}}%
\pgfpathlineto{\pgfqpoint{0.778740in}{0.845896in}}%
\pgfpathlineto{\pgfqpoint{0.781543in}{0.685152in}}%
\pgfpathlineto{\pgfqpoint{0.784346in}{0.703731in}}%
\pgfpathlineto{\pgfqpoint{0.789952in}{0.501149in}}%
\pgfpathlineto{\pgfqpoint{0.792756in}{0.529009in}}%
\pgfpathlineto{\pgfqpoint{0.795559in}{0.487459in}}%
\pgfpathlineto{\pgfqpoint{0.798362in}{0.769574in}}%
\pgfpathlineto{\pgfqpoint{0.801165in}{0.554211in}}%
\pgfpathlineto{\pgfqpoint{0.803969in}{0.762924in}}%
\pgfpathlineto{\pgfqpoint{0.806772in}{0.611059in}}%
\pgfpathlineto{\pgfqpoint{0.809575in}{0.500782in}}%
\pgfpathlineto{\pgfqpoint{0.812378in}{0.523599in}}%
\pgfpathlineto{\pgfqpoint{0.815182in}{0.482603in}}%
\pgfpathlineto{\pgfqpoint{0.817985in}{0.560499in}}%
\pgfpathlineto{\pgfqpoint{0.820788in}{1.066877in}}%
\pgfpathlineto{\pgfqpoint{0.823591in}{0.612787in}}%
\pgfpathlineto{\pgfqpoint{0.826394in}{0.544965in}}%
\pgfpathlineto{\pgfqpoint{0.829198in}{0.506878in}}%
\pgfpathlineto{\pgfqpoint{0.832001in}{0.545200in}}%
\pgfpathlineto{\pgfqpoint{0.834804in}{0.545200in}}%
\pgfpathlineto{\pgfqpoint{0.837607in}{0.492698in}}%
\pgfpathlineto{\pgfqpoint{0.840411in}{0.598174in}}%
\pgfpathlineto{\pgfqpoint{0.843214in}{0.478537in}}%
\pgfpathlineto{\pgfqpoint{0.846017in}{0.660082in}}%
\pgfpathlineto{\pgfqpoint{0.848820in}{0.737386in}}%
\pgfpathlineto{\pgfqpoint{0.851624in}{0.728480in}}%
\pgfpathlineto{\pgfqpoint{0.854427in}{0.858652in}}%
\pgfpathlineto{\pgfqpoint{0.860033in}{0.548693in}}%
\pgfpathlineto{\pgfqpoint{0.862836in}{0.534229in}}%
\pgfpathlineto{\pgfqpoint{0.865640in}{0.510861in}}%
\pgfpathlineto{\pgfqpoint{0.868443in}{0.561944in}}%
\pgfpathlineto{\pgfqpoint{0.871246in}{0.759626in}}%
\pgfpathlineto{\pgfqpoint{0.874049in}{0.713272in}}%
\pgfpathlineto{\pgfqpoint{0.876853in}{0.787509in}}%
\pgfpathlineto{\pgfqpoint{0.879656in}{0.675708in}}%
\pgfpathlineto{\pgfqpoint{0.882459in}{0.508033in}}%
\pgfpathlineto{\pgfqpoint{0.885262in}{0.745014in}}%
\pgfpathlineto{\pgfqpoint{0.888066in}{0.516719in}}%
\pgfpathlineto{\pgfqpoint{0.890869in}{0.492790in}}%
\pgfpathlineto{\pgfqpoint{0.893672in}{0.492836in}}%
\pgfpathlineto{\pgfqpoint{0.896475in}{0.554835in}}%
\pgfpathlineto{\pgfqpoint{0.899278in}{0.709084in}}%
\pgfpathlineto{\pgfqpoint{0.902082in}{0.637519in}}%
\pgfpathlineto{\pgfqpoint{0.904885in}{0.668257in}}%
\pgfpathlineto{\pgfqpoint{0.907688in}{0.646547in}}%
\pgfpathlineto{\pgfqpoint{0.910491in}{0.598188in}}%
\pgfpathlineto{\pgfqpoint{0.913295in}{0.592306in}}%
\pgfpathlineto{\pgfqpoint{0.916098in}{0.786736in}}%
\pgfpathlineto{\pgfqpoint{0.918901in}{1.025442in}}%
\pgfpathlineto{\pgfqpoint{0.927311in}{0.496034in}}%
\pgfpathlineto{\pgfqpoint{0.930114in}{0.660051in}}%
\pgfpathlineto{\pgfqpoint{0.932917in}{0.682611in}}%
\pgfpathlineto{\pgfqpoint{0.935720in}{0.762909in}}%
\pgfpathlineto{\pgfqpoint{0.938524in}{0.495349in}}%
\pgfpathlineto{\pgfqpoint{0.941327in}{0.534904in}}%
\pgfpathlineto{\pgfqpoint{0.944130in}{0.641235in}}%
\pgfpathlineto{\pgfqpoint{0.946933in}{0.630137in}}%
\pgfpathlineto{\pgfqpoint{0.949737in}{0.604887in}}%
\pgfpathlineto{\pgfqpoint{0.952540in}{0.629461in}}%
\pgfpathlineto{\pgfqpoint{0.955343in}{0.665869in}}%
\pgfpathlineto{\pgfqpoint{0.958146in}{0.613477in}}%
\pgfpathlineto{\pgfqpoint{0.960950in}{0.770625in}}%
\pgfpathlineto{\pgfqpoint{0.963753in}{0.573665in}}%
\pgfpathlineto{\pgfqpoint{0.966556in}{0.508088in}}%
\pgfpathlineto{\pgfqpoint{0.969359in}{0.706628in}}%
\pgfpathlineto{\pgfqpoint{0.972162in}{0.504415in}}%
\pgfpathlineto{\pgfqpoint{0.974966in}{0.598108in}}%
\pgfpathlineto{\pgfqpoint{0.977769in}{0.649357in}}%
\pgfpathlineto{\pgfqpoint{0.980572in}{0.514434in}}%
\pgfpathlineto{\pgfqpoint{0.983375in}{0.921206in}}%
\pgfpathlineto{\pgfqpoint{0.986179in}{0.699804in}}%
\pgfpathlineto{\pgfqpoint{0.988982in}{0.631846in}}%
\pgfpathlineto{\pgfqpoint{0.991785in}{0.640481in}}%
\pgfpathlineto{\pgfqpoint{0.994588in}{0.621009in}}%
\pgfpathlineto{\pgfqpoint{0.997392in}{0.642586in}}%
\pgfpathlineto{\pgfqpoint{1.000195in}{0.631676in}}%
\pgfpathlineto{\pgfqpoint{1.002998in}{0.473162in}}%
\pgfpathlineto{\pgfqpoint{1.005801in}{0.528606in}}%
\pgfpathlineto{\pgfqpoint{1.008604in}{0.541361in}}%
\pgfpathlineto{\pgfqpoint{1.011408in}{0.722202in}}%
\pgfpathlineto{\pgfqpoint{1.014211in}{0.525165in}}%
\pgfpathlineto{\pgfqpoint{1.017014in}{0.542527in}}%
\pgfpathlineto{\pgfqpoint{1.019817in}{0.620002in}}%
\pgfpathlineto{\pgfqpoint{1.022621in}{0.473198in}}%
\pgfpathlineto{\pgfqpoint{1.025424in}{0.528812in}}%
\pgfpathlineto{\pgfqpoint{1.028227in}{0.567476in}}%
\pgfpathlineto{\pgfqpoint{1.031030in}{0.516328in}}%
\pgfpathlineto{\pgfqpoint{1.033834in}{0.533866in}}%
\pgfpathlineto{\pgfqpoint{1.036637in}{0.525224in}}%
\pgfpathlineto{\pgfqpoint{1.039440in}{0.560008in}}%
\pgfpathlineto{\pgfqpoint{1.042243in}{0.516896in}}%
\pgfpathlineto{\pgfqpoint{1.045046in}{0.564699in}}%
\pgfpathlineto{\pgfqpoint{1.047850in}{0.490613in}}%
\pgfpathlineto{\pgfqpoint{1.050653in}{0.525264in}}%
\pgfpathlineto{\pgfqpoint{1.053456in}{0.594979in}}%
\pgfpathlineto{\pgfqpoint{1.056259in}{0.620903in}}%
\pgfpathlineto{\pgfqpoint{1.059063in}{0.494778in}}%
\pgfpathlineto{\pgfqpoint{1.061866in}{0.559690in}}%
\pgfpathlineto{\pgfqpoint{1.064669in}{0.654087in}}%
\pgfpathlineto{\pgfqpoint{1.067472in}{0.485994in}}%
\pgfpathlineto{\pgfqpoint{1.070276in}{0.567476in}}%
\pgfpathlineto{\pgfqpoint{1.073079in}{0.737052in}}%
\pgfpathlineto{\pgfqpoint{1.075882in}{0.652798in}}%
\pgfpathlineto{\pgfqpoint{1.078685in}{0.609157in}}%
\pgfpathlineto{\pgfqpoint{1.081488in}{0.534710in}}%
\pgfpathlineto{\pgfqpoint{1.084292in}{0.506076in}}%
\pgfpathlineto{\pgfqpoint{1.087095in}{0.485463in}}%
\pgfpathlineto{\pgfqpoint{1.089898in}{0.572015in}}%
\pgfpathlineto{\pgfqpoint{1.092701in}{0.485347in}}%
\pgfpathlineto{\pgfqpoint{1.095505in}{0.505866in}}%
\pgfpathlineto{\pgfqpoint{1.098308in}{0.588393in}}%
\pgfpathlineto{\pgfqpoint{1.101111in}{0.547321in}}%
\pgfpathlineto{\pgfqpoint{1.103914in}{0.659612in}}%
\pgfpathlineto{\pgfqpoint{1.106718in}{0.597748in}}%
\pgfpathlineto{\pgfqpoint{1.109521in}{0.506063in}}%
\pgfpathlineto{\pgfqpoint{1.112324in}{0.689530in}}%
\pgfpathlineto{\pgfqpoint{1.115127in}{0.533868in}}%
\pgfpathlineto{\pgfqpoint{1.117930in}{0.694795in}}%
\pgfpathlineto{\pgfqpoint{1.120734in}{0.572399in}}%
\pgfpathlineto{\pgfqpoint{1.123537in}{0.520331in}}%
\pgfpathlineto{\pgfqpoint{1.126340in}{0.603853in}}%
\pgfpathlineto{\pgfqpoint{1.129143in}{0.645755in}}%
\pgfpathlineto{\pgfqpoint{1.131947in}{0.529481in}}%
\pgfpathlineto{\pgfqpoint{1.134750in}{0.613085in}}%
\pgfpathlineto{\pgfqpoint{1.137553in}{0.476869in}}%
\pgfpathlineto{\pgfqpoint{1.140356in}{0.536369in}}%
\pgfpathlineto{\pgfqpoint{1.143160in}{0.579226in}}%
\pgfpathlineto{\pgfqpoint{1.145963in}{0.593793in}}%
\pgfpathlineto{\pgfqpoint{1.148766in}{0.527291in}}%
\pgfpathlineto{\pgfqpoint{1.151569in}{0.637815in}}%
\pgfpathlineto{\pgfqpoint{1.154372in}{0.556225in}}%
\pgfpathlineto{\pgfqpoint{1.157176in}{0.681826in}}%
\pgfpathlineto{\pgfqpoint{1.159979in}{0.481070in}}%
\pgfpathlineto{\pgfqpoint{1.162782in}{0.565872in}}%
\pgfpathlineto{\pgfqpoint{1.168389in}{0.508866in}}%
\pgfpathlineto{\pgfqpoint{1.171192in}{0.915547in}}%
\pgfpathlineto{\pgfqpoint{1.173995in}{0.511309in}}%
\pgfpathlineto{\pgfqpoint{1.176798in}{0.632152in}}%
\pgfpathlineto{\pgfqpoint{1.179602in}{0.674554in}}%
\pgfpathlineto{\pgfqpoint{1.182405in}{0.581008in}}%
\pgfpathlineto{\pgfqpoint{1.185208in}{0.511695in}}%
\pgfpathlineto{\pgfqpoint{1.188011in}{0.542721in}}%
\pgfpathlineto{\pgfqpoint{1.190815in}{0.495992in}}%
\pgfpathlineto{\pgfqpoint{1.193618in}{0.538672in}}%
\pgfpathlineto{\pgfqpoint{1.196421in}{0.570377in}}%
\pgfpathlineto{\pgfqpoint{1.199224in}{0.472826in}}%
\pgfpathlineto{\pgfqpoint{1.202027in}{0.705596in}}%
\pgfpathlineto{\pgfqpoint{1.204831in}{0.720002in}}%
\pgfpathlineto{\pgfqpoint{1.207634in}{0.529020in}}%
\pgfpathlineto{\pgfqpoint{1.210437in}{0.495115in}}%
\pgfpathlineto{\pgfqpoint{1.213240in}{0.591716in}}%
\pgfpathlineto{\pgfqpoint{1.216044in}{0.543192in}}%
\pgfpathlineto{\pgfqpoint{1.218847in}{0.506217in}}%
\pgfpathlineto{\pgfqpoint{1.221650in}{0.509942in}}%
\pgfpathlineto{\pgfqpoint{1.224453in}{0.543169in}}%
\pgfpathlineto{\pgfqpoint{1.227257in}{0.597804in}}%
\pgfpathlineto{\pgfqpoint{1.230060in}{0.557149in}}%
\pgfpathlineto{\pgfqpoint{1.232863in}{0.557782in}}%
\pgfpathlineto{\pgfqpoint{1.235666in}{0.509721in}}%
\pgfpathlineto{\pgfqpoint{1.238469in}{0.502201in}}%
\pgfpathlineto{\pgfqpoint{1.241273in}{0.476294in}}%
\pgfpathlineto{\pgfqpoint{1.244076in}{0.476290in}}%
\pgfpathlineto{\pgfqpoint{1.246879in}{0.491027in}}%
\pgfpathlineto{\pgfqpoint{1.249682in}{0.553914in}}%
\pgfpathlineto{\pgfqpoint{1.255289in}{0.498756in}}%
\pgfpathlineto{\pgfqpoint{1.258092in}{0.661645in}}%
\pgfpathlineto{\pgfqpoint{1.260895in}{0.600605in}}%
\pgfpathlineto{\pgfqpoint{1.263699in}{0.596928in}}%
\pgfpathlineto{\pgfqpoint{1.266502in}{0.738037in}}%
\pgfpathlineto{\pgfqpoint{1.269305in}{0.606378in}}%
\pgfpathlineto{\pgfqpoint{1.272108in}{0.501605in}}%
\pgfpathlineto{\pgfqpoint{1.274911in}{0.585558in}}%
\pgfpathlineto{\pgfqpoint{1.277715in}{0.498243in}}%
\pgfpathlineto{\pgfqpoint{1.280518in}{0.531477in}}%
\pgfpathlineto{\pgfqpoint{1.283321in}{0.710073in}}%
\pgfpathlineto{\pgfqpoint{1.286124in}{0.776267in}}%
\pgfpathlineto{\pgfqpoint{1.288928in}{0.479493in}}%
\pgfpathlineto{\pgfqpoint{1.291731in}{0.567270in}}%
\pgfpathlineto{\pgfqpoint{1.294534in}{0.559550in}}%
\pgfpathlineto{\pgfqpoint{1.297337in}{0.583007in}}%
\pgfpathlineto{\pgfqpoint{1.300141in}{0.656767in}}%
\pgfpathlineto{\pgfqpoint{1.302944in}{0.567605in}}%
\pgfpathlineto{\pgfqpoint{1.305747in}{0.516653in}}%
\pgfpathlineto{\pgfqpoint{1.308550in}{0.506406in}}%
\pgfpathlineto{\pgfqpoint{1.311353in}{0.485990in}}%
\pgfpathlineto{\pgfqpoint{1.314157in}{0.486014in}}%
\pgfpathlineto{\pgfqpoint{1.316960in}{0.492896in}}%
\pgfpathlineto{\pgfqpoint{1.319763in}{0.596481in}}%
\pgfpathlineto{\pgfqpoint{1.325370in}{0.468908in}}%
\pgfpathlineto{\pgfqpoint{1.328173in}{0.506782in}}%
\pgfpathlineto{\pgfqpoint{1.330976in}{0.492949in}}%
\pgfpathlineto{\pgfqpoint{1.333779in}{0.742981in}}%
\pgfpathlineto{\pgfqpoint{1.336583in}{0.493439in}}%
\pgfpathlineto{\pgfqpoint{1.342189in}{0.696770in}}%
\pgfpathlineto{\pgfqpoint{1.344992in}{0.520901in}}%
\pgfpathlineto{\pgfqpoint{1.347795in}{0.618544in}}%
\pgfpathlineto{\pgfqpoint{1.350599in}{0.694408in}}%
\pgfpathlineto{\pgfqpoint{1.353402in}{0.568206in}}%
\pgfpathlineto{\pgfqpoint{1.356205in}{0.762237in}}%
\pgfpathlineto{\pgfqpoint{1.359008in}{0.528132in}}%
\pgfpathlineto{\pgfqpoint{1.361812in}{0.587641in}}%
\pgfpathlineto{\pgfqpoint{1.364615in}{0.507446in}}%
\pgfpathlineto{\pgfqpoint{1.367418in}{0.525003in}}%
\pgfpathlineto{\pgfqpoint{1.370221in}{0.563422in}}%
\pgfpathlineto{\pgfqpoint{1.373025in}{0.628289in}}%
\pgfpathlineto{\pgfqpoint{1.375828in}{0.496479in}}%
\pgfpathlineto{\pgfqpoint{1.378631in}{0.642635in}}%
\pgfpathlineto{\pgfqpoint{1.381434in}{0.628826in}}%
\pgfpathlineto{\pgfqpoint{1.384237in}{0.623377in}}%
\pgfpathlineto{\pgfqpoint{1.387041in}{0.748354in}}%
\pgfpathlineto{\pgfqpoint{1.389844in}{0.624934in}}%
\pgfpathlineto{\pgfqpoint{1.392647in}{0.626395in}}%
\pgfpathlineto{\pgfqpoint{1.398254in}{0.513556in}}%
\pgfpathlineto{\pgfqpoint{1.401057in}{0.581934in}}%
\pgfpathlineto{\pgfqpoint{1.403860in}{0.530126in}}%
\pgfpathlineto{\pgfqpoint{1.406663in}{0.526449in}}%
\pgfpathlineto{\pgfqpoint{1.409467in}{0.509497in}}%
\pgfpathlineto{\pgfqpoint{1.412270in}{0.479068in}}%
\pgfpathlineto{\pgfqpoint{1.415073in}{0.509464in}}%
\pgfpathlineto{\pgfqpoint{1.417876in}{0.577354in}}%
\pgfpathlineto{\pgfqpoint{1.420679in}{0.499580in}}%
\pgfpathlineto{\pgfqpoint{1.426286in}{0.632473in}}%
\pgfpathlineto{\pgfqpoint{1.429089in}{0.506698in}}%
\pgfpathlineto{\pgfqpoint{1.431892in}{0.631686in}}%
\pgfpathlineto{\pgfqpoint{1.434696in}{0.731067in}}%
\pgfpathlineto{\pgfqpoint{1.437499in}{0.588950in}}%
\pgfpathlineto{\pgfqpoint{1.440302in}{0.646839in}}%
\pgfpathlineto{\pgfqpoint{1.443105in}{0.767634in}}%
\pgfpathlineto{\pgfqpoint{1.445909in}{0.728163in}}%
\pgfpathlineto{\pgfqpoint{1.448712in}{0.556284in}}%
\pgfpathlineto{\pgfqpoint{1.451515in}{0.555672in}}%
\pgfpathlineto{\pgfqpoint{1.454318in}{0.652794in}}%
\pgfpathlineto{\pgfqpoint{1.457121in}{0.514327in}}%
\pgfpathlineto{\pgfqpoint{1.459925in}{0.707586in}}%
\pgfpathlineto{\pgfqpoint{1.462728in}{0.596993in}}%
\pgfpathlineto{\pgfqpoint{1.465531in}{0.567317in}}%
\pgfpathlineto{\pgfqpoint{1.468334in}{0.554241in}}%
\pgfpathlineto{\pgfqpoint{1.471138in}{0.583917in}}%
\pgfpathlineto{\pgfqpoint{1.473941in}{0.636139in}}%
\pgfpathlineto{\pgfqpoint{1.476744in}{0.514422in}}%
\pgfpathlineto{\pgfqpoint{1.479547in}{0.605955in}}%
\pgfpathlineto{\pgfqpoint{1.482351in}{0.808192in}}%
\pgfpathlineto{\pgfqpoint{1.487957in}{0.655223in}}%
\pgfpathlineto{\pgfqpoint{1.490760in}{0.532764in}}%
\pgfpathlineto{\pgfqpoint{1.493563in}{0.662815in}}%
\pgfpathlineto{\pgfqpoint{1.496367in}{0.502124in}}%
\pgfpathlineto{\pgfqpoint{1.499170in}{0.630824in}}%
\pgfpathlineto{\pgfqpoint{1.501973in}{0.573517in}}%
\pgfpathlineto{\pgfqpoint{1.504776in}{0.591995in}}%
\pgfpathlineto{\pgfqpoint{1.507580in}{0.578980in}}%
\pgfpathlineto{\pgfqpoint{1.510383in}{0.517590in}}%
\pgfpathlineto{\pgfqpoint{1.513186in}{0.510945in}}%
\pgfpathlineto{\pgfqpoint{1.515989in}{0.651015in}}%
\pgfpathlineto{\pgfqpoint{1.518793in}{1.150835in}}%
\pgfpathlineto{\pgfqpoint{1.521596in}{0.756971in}}%
\pgfpathlineto{\pgfqpoint{1.524399in}{0.514428in}}%
\pgfpathlineto{\pgfqpoint{1.527202in}{0.532590in}}%
\pgfpathlineto{\pgfqpoint{1.530005in}{0.508291in}}%
\pgfpathlineto{\pgfqpoint{1.532809in}{0.514507in}}%
\pgfpathlineto{\pgfqpoint{1.535612in}{0.481084in}}%
\pgfpathlineto{\pgfqpoint{1.538415in}{0.481072in}}%
\pgfpathlineto{\pgfqpoint{1.544022in}{0.657464in}}%
\pgfpathlineto{\pgfqpoint{1.546825in}{0.603297in}}%
\pgfpathlineto{\pgfqpoint{1.549628in}{0.791950in}}%
\pgfpathlineto{\pgfqpoint{1.552431in}{0.669448in}}%
\pgfpathlineto{\pgfqpoint{1.555235in}{0.471881in}}%
\pgfpathlineto{\pgfqpoint{1.558038in}{0.564438in}}%
\pgfpathlineto{\pgfqpoint{1.560841in}{0.610568in}}%
\pgfpathlineto{\pgfqpoint{1.563644in}{0.631529in}}%
\pgfpathlineto{\pgfqpoint{1.566447in}{0.498862in}}%
\pgfpathlineto{\pgfqpoint{1.569251in}{0.598543in}}%
\pgfpathlineto{\pgfqpoint{1.572054in}{0.667330in}}%
\pgfpathlineto{\pgfqpoint{1.574857in}{0.667330in}}%
\pgfpathlineto{\pgfqpoint{1.577660in}{0.610534in}}%
\pgfpathlineto{\pgfqpoint{1.580464in}{0.468908in}}%
\pgfpathlineto{\pgfqpoint{1.583267in}{0.674341in}}%
\pgfpathlineto{\pgfqpoint{1.586070in}{0.617283in}}%
\pgfpathlineto{\pgfqpoint{1.588873in}{0.507976in}}%
\pgfpathlineto{\pgfqpoint{1.591677in}{0.632846in}}%
\pgfpathlineto{\pgfqpoint{1.594480in}{0.650862in}}%
\pgfpathlineto{\pgfqpoint{1.597283in}{0.505019in}}%
\pgfpathlineto{\pgfqpoint{1.600086in}{0.681049in}}%
\pgfpathlineto{\pgfqpoint{1.602889in}{0.558102in}}%
\pgfpathlineto{\pgfqpoint{1.605693in}{0.564026in}}%
\pgfpathlineto{\pgfqpoint{1.608496in}{0.583853in}}%
\pgfpathlineto{\pgfqpoint{1.611299in}{0.563143in}}%
\pgfpathlineto{\pgfqpoint{1.614102in}{0.755995in}}%
\pgfpathlineto{\pgfqpoint{1.616906in}{0.541306in}}%
\pgfpathlineto{\pgfqpoint{1.619709in}{0.553414in}}%
\pgfpathlineto{\pgfqpoint{1.622512in}{0.619409in}}%
\pgfpathlineto{\pgfqpoint{1.625315in}{0.617602in}}%
\pgfpathlineto{\pgfqpoint{1.628119in}{0.577803in}}%
\pgfpathlineto{\pgfqpoint{1.630922in}{0.568938in}}%
\pgfpathlineto{\pgfqpoint{1.633725in}{0.633236in}}%
\pgfpathlineto{\pgfqpoint{1.636528in}{0.471822in}}%
\pgfpathlineto{\pgfqpoint{1.642135in}{0.714691in}}%
\pgfpathlineto{\pgfqpoint{1.644938in}{0.645608in}}%
\pgfpathlineto{\pgfqpoint{1.647741in}{0.526450in}}%
\pgfpathlineto{\pgfqpoint{1.650544in}{0.503562in}}%
\pgfpathlineto{\pgfqpoint{1.653348in}{0.683922in}}%
\pgfpathlineto{\pgfqpoint{1.656151in}{0.517130in}}%
\pgfpathlineto{\pgfqpoint{1.658954in}{0.497251in}}%
\pgfpathlineto{\pgfqpoint{1.661757in}{0.525789in}}%
\pgfpathlineto{\pgfqpoint{1.664561in}{0.486000in}}%
\pgfpathlineto{\pgfqpoint{1.667364in}{0.556842in}}%
\pgfpathlineto{\pgfqpoint{1.670167in}{0.539774in}}%
\pgfpathlineto{\pgfqpoint{1.672970in}{0.553898in}}%
\pgfpathlineto{\pgfqpoint{1.675774in}{0.642339in}}%
\pgfpathlineto{\pgfqpoint{1.678577in}{0.589747in}}%
\pgfpathlineto{\pgfqpoint{1.681380in}{0.981463in}}%
\pgfpathlineto{\pgfqpoint{1.684183in}{0.560789in}}%
\pgfpathlineto{\pgfqpoint{1.686986in}{0.547406in}}%
\pgfpathlineto{\pgfqpoint{1.689790in}{0.502703in}}%
\pgfpathlineto{\pgfqpoint{1.692593in}{1.439420in}}%
\pgfpathlineto{\pgfqpoint{1.695396in}{0.617464in}}%
\pgfpathlineto{\pgfqpoint{1.698199in}{0.798985in}}%
\pgfpathlineto{\pgfqpoint{1.701003in}{0.506996in}}%
\pgfpathlineto{\pgfqpoint{1.703806in}{0.580574in}}%
\pgfpathlineto{\pgfqpoint{1.706609in}{0.572657in}}%
\pgfpathlineto{\pgfqpoint{1.709412in}{0.510652in}}%
\pgfpathlineto{\pgfqpoint{1.712216in}{0.534791in}}%
\pgfpathlineto{\pgfqpoint{1.715019in}{0.471911in}}%
\pgfpathlineto{\pgfqpoint{1.717822in}{0.505001in}}%
\pgfpathlineto{\pgfqpoint{1.720625in}{0.498993in}}%
\pgfpathlineto{\pgfqpoint{1.723428in}{0.779262in}}%
\pgfpathlineto{\pgfqpoint{1.726232in}{0.527588in}}%
\pgfpathlineto{\pgfqpoint{1.729035in}{0.601316in}}%
\pgfpathlineto{\pgfqpoint{1.731838in}{0.595126in}}%
\pgfpathlineto{\pgfqpoint{1.734641in}{0.834757in}}%
\pgfpathlineto{\pgfqpoint{1.737445in}{0.504905in}}%
\pgfpathlineto{\pgfqpoint{1.740248in}{0.480884in}}%
\pgfpathlineto{\pgfqpoint{1.743051in}{0.753450in}}%
\pgfpathlineto{\pgfqpoint{1.745854in}{0.530301in}}%
\pgfpathlineto{\pgfqpoint{1.748658in}{0.542544in}}%
\pgfpathlineto{\pgfqpoint{1.751461in}{0.481138in}}%
\pgfpathlineto{\pgfqpoint{1.754264in}{0.536324in}}%
\pgfpathlineto{\pgfqpoint{1.757067in}{0.651912in}}%
\pgfpathlineto{\pgfqpoint{1.759870in}{0.526567in}}%
\pgfpathlineto{\pgfqpoint{1.762674in}{0.471949in}}%
\pgfpathlineto{\pgfqpoint{1.765477in}{0.499281in}}%
\pgfpathlineto{\pgfqpoint{1.768280in}{0.601671in}}%
\pgfpathlineto{\pgfqpoint{1.771083in}{0.879206in}}%
\pgfpathlineto{\pgfqpoint{1.773887in}{0.584503in}}%
\pgfpathlineto{\pgfqpoint{1.776690in}{0.552607in}}%
\pgfpathlineto{\pgfqpoint{1.779493in}{0.729674in}}%
\pgfpathlineto{\pgfqpoint{1.782296in}{0.494402in}}%
\pgfpathlineto{\pgfqpoint{1.785100in}{0.612392in}}%
\pgfpathlineto{\pgfqpoint{1.787903in}{0.602449in}}%
\pgfpathlineto{\pgfqpoint{1.790706in}{0.535142in}}%
\pgfpathlineto{\pgfqpoint{1.793509in}{0.727664in}}%
\pgfpathlineto{\pgfqpoint{1.796312in}{0.631703in}}%
\pgfpathlineto{\pgfqpoint{1.799116in}{0.653261in}}%
\pgfpathlineto{\pgfqpoint{1.801919in}{0.583090in}}%
\pgfpathlineto{\pgfqpoint{1.804722in}{0.596943in}}%
\pgfpathlineto{\pgfqpoint{1.807525in}{0.499405in}}%
\pgfpathlineto{\pgfqpoint{1.810329in}{0.594456in}}%
\pgfpathlineto{\pgfqpoint{1.813132in}{0.605554in}}%
\pgfpathlineto{\pgfqpoint{1.815935in}{0.560769in}}%
\pgfpathlineto{\pgfqpoint{1.818738in}{0.674570in}}%
\pgfpathlineto{\pgfqpoint{1.821542in}{0.587651in}}%
\pgfpathlineto{\pgfqpoint{1.824345in}{0.742013in}}%
\pgfpathlineto{\pgfqpoint{1.827148in}{0.721722in}}%
\pgfpathlineto{\pgfqpoint{1.829951in}{0.556669in}}%
\pgfpathlineto{\pgfqpoint{1.832754in}{0.813863in}}%
\pgfpathlineto{\pgfqpoint{1.838361in}{0.549507in}}%
\pgfpathlineto{\pgfqpoint{1.841164in}{0.602600in}}%
\pgfpathlineto{\pgfqpoint{1.843967in}{0.740344in}}%
\pgfpathlineto{\pgfqpoint{1.846771in}{0.637610in}}%
\pgfpathlineto{\pgfqpoint{1.849574in}{0.603303in}}%
\pgfpathlineto{\pgfqpoint{1.852377in}{0.495670in}}%
\pgfpathlineto{\pgfqpoint{1.855180in}{0.471587in}}%
\pgfpathlineto{\pgfqpoint{1.857984in}{0.536083in}}%
\pgfpathlineto{\pgfqpoint{1.860787in}{0.716978in}}%
\pgfpathlineto{\pgfqpoint{1.863590in}{0.476828in}}%
\pgfpathlineto{\pgfqpoint{1.866393in}{0.468908in}}%
\pgfpathlineto{\pgfqpoint{1.869196in}{0.539962in}}%
\pgfpathlineto{\pgfqpoint{1.872000in}{0.492547in}}%
\pgfpathlineto{\pgfqpoint{1.874803in}{0.913216in}}%
\pgfpathlineto{\pgfqpoint{1.877606in}{0.625966in}}%
\pgfpathlineto{\pgfqpoint{1.880409in}{0.639600in}}%
\pgfpathlineto{\pgfqpoint{1.883213in}{0.528782in}}%
\pgfpathlineto{\pgfqpoint{1.886016in}{0.571650in}}%
\pgfpathlineto{\pgfqpoint{1.888819in}{0.653369in}}%
\pgfpathlineto{\pgfqpoint{1.891622in}{0.607511in}}%
\pgfpathlineto{\pgfqpoint{1.894426in}{0.528222in}}%
\pgfpathlineto{\pgfqpoint{1.897229in}{0.530924in}}%
\pgfpathlineto{\pgfqpoint{1.900032in}{0.664681in}}%
\pgfpathlineto{\pgfqpoint{1.902835in}{0.490175in}}%
\pgfpathlineto{\pgfqpoint{1.905638in}{0.582600in}}%
\pgfpathlineto{\pgfqpoint{1.908442in}{0.595888in}}%
\pgfpathlineto{\pgfqpoint{1.911245in}{0.626793in}}%
\pgfpathlineto{\pgfqpoint{1.914048in}{0.514607in}}%
\pgfpathlineto{\pgfqpoint{1.916851in}{0.684364in}}%
\pgfpathlineto{\pgfqpoint{1.919655in}{0.500511in}}%
\pgfpathlineto{\pgfqpoint{1.922458in}{0.479433in}}%
\pgfpathlineto{\pgfqpoint{1.925261in}{0.565928in}}%
\pgfpathlineto{\pgfqpoint{1.928064in}{0.497606in}}%
\pgfpathlineto{\pgfqpoint{1.930868in}{0.502830in}}%
\pgfpathlineto{\pgfqpoint{1.933671in}{0.570398in}}%
\pgfpathlineto{\pgfqpoint{1.936474in}{0.672728in}}%
\pgfpathlineto{\pgfqpoint{1.939277in}{0.529358in}}%
\pgfpathlineto{\pgfqpoint{1.942080in}{0.531993in}}%
\pgfpathlineto{\pgfqpoint{1.944884in}{0.476811in}}%
\pgfpathlineto{\pgfqpoint{1.947687in}{0.484719in}}%
\pgfpathlineto{\pgfqpoint{1.950490in}{0.601462in}}%
\pgfpathlineto{\pgfqpoint{1.953293in}{0.849471in}}%
\pgfpathlineto{\pgfqpoint{1.956097in}{0.604045in}}%
\pgfpathlineto{\pgfqpoint{1.958900in}{0.704438in}}%
\pgfpathlineto{\pgfqpoint{1.961703in}{0.491960in}}%
\pgfpathlineto{\pgfqpoint{1.964506in}{0.794698in}}%
\pgfpathlineto{\pgfqpoint{1.967310in}{0.587977in}}%
\pgfpathlineto{\pgfqpoint{1.970113in}{0.510805in}}%
\pgfpathlineto{\pgfqpoint{1.972916in}{0.473829in}}%
\pgfpathlineto{\pgfqpoint{1.975719in}{0.586493in}}%
\pgfpathlineto{\pgfqpoint{1.978522in}{0.507856in}}%
\pgfpathlineto{\pgfqpoint{1.981326in}{0.551268in}}%
\pgfpathlineto{\pgfqpoint{1.984129in}{0.497914in}}%
\pgfpathlineto{\pgfqpoint{1.986932in}{0.538889in}}%
\pgfpathlineto{\pgfqpoint{1.989735in}{0.567177in}}%
\pgfpathlineto{\pgfqpoint{1.992539in}{0.562365in}}%
\pgfpathlineto{\pgfqpoint{1.995342in}{0.478526in}}%
\pgfpathlineto{\pgfqpoint{1.998145in}{0.543643in}}%
\pgfpathlineto{\pgfqpoint{2.000948in}{0.562856in}}%
\pgfpathlineto{\pgfqpoint{2.003752in}{0.507246in}}%
\pgfpathlineto{\pgfqpoint{2.006555in}{0.485666in}}%
\pgfpathlineto{\pgfqpoint{2.009358in}{0.512105in}}%
\pgfpathlineto{\pgfqpoint{2.014964in}{0.680795in}}%
\pgfpathlineto{\pgfqpoint{2.017768in}{0.528448in}}%
\pgfpathlineto{\pgfqpoint{2.020571in}{0.578683in}}%
\pgfpathlineto{\pgfqpoint{2.023374in}{1.038103in}}%
\pgfpathlineto{\pgfqpoint{2.026177in}{0.482651in}}%
\pgfpathlineto{\pgfqpoint{2.028981in}{0.523791in}}%
\pgfpathlineto{\pgfqpoint{2.031784in}{0.722480in}}%
\pgfpathlineto{\pgfqpoint{2.034587in}{0.854469in}}%
\pgfpathlineto{\pgfqpoint{2.037390in}{0.753254in}}%
\pgfpathlineto{\pgfqpoint{2.040194in}{0.736424in}}%
\pgfpathlineto{\pgfqpoint{2.042997in}{0.684904in}}%
\pgfpathlineto{\pgfqpoint{2.045800in}{0.680092in}}%
\pgfpathlineto{\pgfqpoint{2.048603in}{0.778410in}}%
\pgfpathlineto{\pgfqpoint{2.051406in}{0.580457in}}%
\pgfpathlineto{\pgfqpoint{2.054210in}{0.531003in}}%
\pgfpathlineto{\pgfqpoint{2.057013in}{0.530694in}}%
\pgfpathlineto{\pgfqpoint{2.059816in}{0.662149in}}%
\pgfpathlineto{\pgfqpoint{2.062619in}{0.623276in}}%
\pgfpathlineto{\pgfqpoint{2.065423in}{0.536207in}}%
\pgfpathlineto{\pgfqpoint{2.068226in}{0.567330in}}%
\pgfpathlineto{\pgfqpoint{2.071029in}{0.661081in}}%
\pgfpathlineto{\pgfqpoint{2.073832in}{0.475972in}}%
\pgfpathlineto{\pgfqpoint{2.076636in}{0.576791in}}%
\pgfpathlineto{\pgfqpoint{2.079439in}{0.471243in}}%
\pgfpathlineto{\pgfqpoint{2.082242in}{0.588552in}}%
\pgfpathlineto{\pgfqpoint{2.085045in}{0.492504in}}%
\pgfpathlineto{\pgfqpoint{2.087849in}{0.473632in}}%
\pgfpathlineto{\pgfqpoint{2.090652in}{0.623875in}}%
\pgfpathlineto{\pgfqpoint{2.093455in}{0.520135in}}%
\pgfpathlineto{\pgfqpoint{2.096258in}{0.524803in}}%
\pgfpathlineto{\pgfqpoint{2.099061in}{0.478241in}}%
\pgfpathlineto{\pgfqpoint{2.101865in}{0.487582in}}%
\pgfpathlineto{\pgfqpoint{2.104668in}{0.716649in}}%
\pgfpathlineto{\pgfqpoint{2.107471in}{0.693267in}}%
\pgfpathlineto{\pgfqpoint{2.110274in}{0.617238in}}%
\pgfpathlineto{\pgfqpoint{2.113078in}{0.525624in}}%
\pgfpathlineto{\pgfqpoint{2.115881in}{0.562862in}}%
\pgfpathlineto{\pgfqpoint{2.118684in}{0.471248in}}%
\pgfpathlineto{\pgfqpoint{2.121487in}{0.559849in}}%
\pgfpathlineto{\pgfqpoint{2.124291in}{0.527126in}}%
\pgfpathlineto{\pgfqpoint{2.127094in}{0.764179in}}%
\pgfpathlineto{\pgfqpoint{2.129897in}{0.468908in}}%
\pgfpathlineto{\pgfqpoint{2.132700in}{0.637485in}}%
\pgfpathlineto{\pgfqpoint{2.135503in}{0.611931in}}%
\pgfpathlineto{\pgfqpoint{2.138307in}{0.609573in}}%
\pgfpathlineto{\pgfqpoint{2.143913in}{0.554260in}}%
\pgfpathlineto{\pgfqpoint{2.146716in}{0.613880in}}%
\pgfpathlineto{\pgfqpoint{2.149520in}{0.603508in}}%
\pgfpathlineto{\pgfqpoint{2.152323in}{0.622103in}}%
\pgfpathlineto{\pgfqpoint{2.155126in}{0.473800in}}%
\pgfpathlineto{\pgfqpoint{2.157929in}{0.683542in}}%
\pgfpathlineto{\pgfqpoint{2.160733in}{0.688433in}}%
\pgfpathlineto{\pgfqpoint{2.163536in}{0.535095in}}%
\pgfpathlineto{\pgfqpoint{2.166339in}{0.471366in}}%
\pgfpathlineto{\pgfqpoint{2.169142in}{0.488560in}}%
\pgfpathlineto{\pgfqpoint{2.171945in}{0.898451in}}%
\pgfpathlineto{\pgfqpoint{2.177552in}{0.520905in}}%
\pgfpathlineto{\pgfqpoint{2.180355in}{0.770039in}}%
\pgfpathlineto{\pgfqpoint{2.183158in}{0.832725in}}%
\pgfpathlineto{\pgfqpoint{2.185962in}{0.773762in}}%
\pgfpathlineto{\pgfqpoint{2.188765in}{0.739674in}}%
\pgfpathlineto{\pgfqpoint{2.191568in}{0.656534in}}%
\pgfpathlineto{\pgfqpoint{2.194371in}{0.785524in}}%
\pgfpathlineto{\pgfqpoint{2.197175in}{0.552381in}}%
\pgfpathlineto{\pgfqpoint{2.199978in}{0.539770in}}%
\pgfpathlineto{\pgfqpoint{2.202781in}{0.519276in}}%
\pgfpathlineto{\pgfqpoint{2.205584in}{0.536954in}}%
\pgfpathlineto{\pgfqpoint{2.208387in}{0.501717in}}%
\pgfpathlineto{\pgfqpoint{2.211191in}{1.102353in}}%
\pgfpathlineto{\pgfqpoint{2.213994in}{0.621820in}}%
\pgfpathlineto{\pgfqpoint{2.219600in}{0.471512in}}%
\pgfpathlineto{\pgfqpoint{2.222404in}{0.680678in}}%
\pgfpathlineto{\pgfqpoint{2.225207in}{0.571758in}}%
\pgfpathlineto{\pgfqpoint{2.228010in}{0.633060in}}%
\pgfpathlineto{\pgfqpoint{2.230813in}{0.576387in}}%
\pgfpathlineto{\pgfqpoint{2.233617in}{0.670305in}}%
\pgfpathlineto{\pgfqpoint{2.236420in}{0.597205in}}%
\pgfpathlineto{\pgfqpoint{2.239223in}{0.473914in}}%
\pgfpathlineto{\pgfqpoint{2.242026in}{0.521373in}}%
\pgfpathlineto{\pgfqpoint{2.244829in}{0.491325in}}%
\pgfpathlineto{\pgfqpoint{2.247633in}{0.506293in}}%
\pgfpathlineto{\pgfqpoint{2.250436in}{0.612008in}}%
\pgfpathlineto{\pgfqpoint{2.253239in}{0.609512in}}%
\pgfpathlineto{\pgfqpoint{2.256042in}{0.501321in}}%
\pgfpathlineto{\pgfqpoint{2.258846in}{0.611661in}}%
\pgfpathlineto{\pgfqpoint{2.261649in}{0.596712in}}%
\pgfpathlineto{\pgfqpoint{2.264452in}{0.511363in}}%
\pgfpathlineto{\pgfqpoint{2.267255in}{0.627986in}}%
\pgfpathlineto{\pgfqpoint{2.270059in}{0.608887in}}%
\pgfpathlineto{\pgfqpoint{2.272862in}{0.508041in}}%
\pgfpathlineto{\pgfqpoint{2.275665in}{0.517805in}}%
\pgfpathlineto{\pgfqpoint{2.278468in}{0.478664in}}%
\pgfpathlineto{\pgfqpoint{2.281271in}{0.544720in}}%
\pgfpathlineto{\pgfqpoint{2.284075in}{0.597129in}}%
\pgfpathlineto{\pgfqpoint{2.286878in}{0.476341in}}%
\pgfpathlineto{\pgfqpoint{2.289681in}{0.513416in}}%
\pgfpathlineto{\pgfqpoint{2.292484in}{0.515715in}}%
\pgfpathlineto{\pgfqpoint{2.295288in}{0.493521in}}%
\pgfpathlineto{\pgfqpoint{2.298091in}{0.564625in}}%
\pgfpathlineto{\pgfqpoint{2.300894in}{0.559699in}}%
\pgfpathlineto{\pgfqpoint{2.303697in}{0.520735in}}%
\pgfpathlineto{\pgfqpoint{2.306501in}{0.498621in}}%
\pgfpathlineto{\pgfqpoint{2.309304in}{0.531041in}}%
\pgfpathlineto{\pgfqpoint{2.312107in}{0.503741in}}%
\pgfpathlineto{\pgfqpoint{2.314910in}{0.493729in}}%
\pgfpathlineto{\pgfqpoint{2.317713in}{0.523579in}}%
\pgfpathlineto{\pgfqpoint{2.320517in}{0.516319in}}%
\pgfpathlineto{\pgfqpoint{2.323320in}{0.471408in}}%
\pgfpathlineto{\pgfqpoint{2.326123in}{0.566818in}}%
\pgfpathlineto{\pgfqpoint{2.328926in}{0.544292in}}%
\pgfpathlineto{\pgfqpoint{2.331730in}{0.476426in}}%
\pgfpathlineto{\pgfqpoint{2.334533in}{0.488945in}}%
\pgfpathlineto{\pgfqpoint{2.337336in}{0.553707in}}%
\pgfpathlineto{\pgfqpoint{2.340139in}{0.673510in}}%
\pgfpathlineto{\pgfqpoint{2.342943in}{0.503088in}}%
\pgfpathlineto{\pgfqpoint{2.345746in}{0.566050in}}%
\pgfpathlineto{\pgfqpoint{2.348549in}{0.480997in}}%
\pgfpathlineto{\pgfqpoint{2.351352in}{0.553782in}}%
\pgfpathlineto{\pgfqpoint{2.354155in}{0.519902in}}%
\pgfpathlineto{\pgfqpoint{2.356959in}{0.502881in}}%
\pgfpathlineto{\pgfqpoint{2.359762in}{0.551252in}}%
\pgfpathlineto{\pgfqpoint{2.362565in}{0.698497in}}%
\pgfpathlineto{\pgfqpoint{2.365368in}{0.509124in}}%
\pgfpathlineto{\pgfqpoint{2.368172in}{0.518407in}}%
\pgfpathlineto{\pgfqpoint{2.370975in}{0.737625in}}%
\pgfpathlineto{\pgfqpoint{2.373778in}{0.600428in}}%
\pgfpathlineto{\pgfqpoint{2.376581in}{0.523737in}}%
\pgfpathlineto{\pgfqpoint{2.379385in}{0.583060in}}%
\pgfpathlineto{\pgfqpoint{2.382188in}{0.471275in}}%
\pgfpathlineto{\pgfqpoint{2.384991in}{0.553853in}}%
\pgfpathlineto{\pgfqpoint{2.387794in}{0.919077in}}%
\pgfpathlineto{\pgfqpoint{2.390597in}{0.500650in}}%
\pgfpathlineto{\pgfqpoint{2.393401in}{0.512993in}}%
\pgfpathlineto{\pgfqpoint{2.396204in}{0.503210in}}%
\pgfpathlineto{\pgfqpoint{2.399007in}{0.646340in}}%
\pgfpathlineto{\pgfqpoint{2.401810in}{0.515978in}}%
\pgfpathlineto{\pgfqpoint{2.404614in}{0.488673in}}%
\pgfpathlineto{\pgfqpoint{2.407417in}{0.563076in}}%
\pgfpathlineto{\pgfqpoint{2.410220in}{0.476368in}}%
\pgfpathlineto{\pgfqpoint{2.413023in}{0.566246in}}%
\pgfpathlineto{\pgfqpoint{2.415827in}{0.553809in}}%
\pgfpathlineto{\pgfqpoint{2.418630in}{0.473884in}}%
\pgfpathlineto{\pgfqpoint{2.421433in}{0.483842in}}%
\pgfpathlineto{\pgfqpoint{2.424236in}{0.629745in}}%
\pgfpathlineto{\pgfqpoint{2.427039in}{0.532662in}}%
\pgfpathlineto{\pgfqpoint{2.429843in}{0.606636in}}%
\pgfpathlineto{\pgfqpoint{2.432646in}{0.711361in}}%
\pgfpathlineto{\pgfqpoint{2.435449in}{0.507650in}}%
\pgfpathlineto{\pgfqpoint{2.438252in}{0.538821in}}%
\pgfpathlineto{\pgfqpoint{2.441056in}{0.495380in}}%
\pgfpathlineto{\pgfqpoint{2.443859in}{0.488166in}}%
\pgfpathlineto{\pgfqpoint{2.446662in}{0.473719in}}%
\pgfpathlineto{\pgfqpoint{2.449465in}{0.509748in}}%
\pgfpathlineto{\pgfqpoint{2.452269in}{0.492869in}}%
\pgfpathlineto{\pgfqpoint{2.455072in}{0.615797in}}%
\pgfpathlineto{\pgfqpoint{2.457875in}{0.588176in}}%
\pgfpathlineto{\pgfqpoint{2.460678in}{0.595441in}}%
\pgfpathlineto{\pgfqpoint{2.463481in}{0.481007in}}%
\pgfpathlineto{\pgfqpoint{2.466285in}{0.507665in}}%
\pgfpathlineto{\pgfqpoint{2.469088in}{0.471334in}}%
\pgfpathlineto{\pgfqpoint{2.471891in}{0.558369in}}%
\pgfpathlineto{\pgfqpoint{2.474694in}{0.553517in}}%
\pgfpathlineto{\pgfqpoint{2.477498in}{0.618394in}}%
\pgfpathlineto{\pgfqpoint{2.480301in}{0.579644in}}%
\pgfpathlineto{\pgfqpoint{2.483104in}{0.522221in}}%
\pgfpathlineto{\pgfqpoint{2.485907in}{0.502861in}}%
\pgfpathlineto{\pgfqpoint{2.488711in}{0.622946in}}%
\pgfpathlineto{\pgfqpoint{2.494317in}{0.519446in}}%
\pgfpathlineto{\pgfqpoint{2.497120in}{0.478518in}}%
\pgfpathlineto{\pgfqpoint{2.499923in}{0.606674in}}%
\pgfpathlineto{\pgfqpoint{2.502727in}{0.551820in}}%
\pgfpathlineto{\pgfqpoint{2.505530in}{0.583375in}}%
\pgfpathlineto{\pgfqpoint{2.508333in}{0.684101in}}%
\pgfpathlineto{\pgfqpoint{2.511136in}{0.540238in}}%
\pgfpathlineto{\pgfqpoint{2.513940in}{0.527910in}}%
\pgfpathlineto{\pgfqpoint{2.516743in}{0.611018in}}%
\pgfpathlineto{\pgfqpoint{2.519546in}{0.665352in}}%
\pgfpathlineto{\pgfqpoint{2.522349in}{0.486246in}}%
\pgfpathlineto{\pgfqpoint{2.525153in}{0.602034in}}%
\pgfpathlineto{\pgfqpoint{2.527956in}{0.481174in}}%
\pgfpathlineto{\pgfqpoint{2.530759in}{0.689311in}}%
\pgfpathlineto{\pgfqpoint{2.533562in}{0.723626in}}%
\pgfpathlineto{\pgfqpoint{2.536366in}{0.743631in}}%
\pgfpathlineto{\pgfqpoint{2.541972in}{0.509550in}}%
\pgfpathlineto{\pgfqpoint{2.547578in}{0.618539in}}%
\pgfpathlineto{\pgfqpoint{2.553185in}{0.651912in}}%
\pgfpathlineto{\pgfqpoint{2.555988in}{0.579427in}}%
\pgfpathlineto{\pgfqpoint{2.558791in}{0.789795in}}%
\pgfpathlineto{\pgfqpoint{2.561595in}{0.608582in}}%
\pgfpathlineto{\pgfqpoint{2.564398in}{0.664499in}}%
\pgfpathlineto{\pgfqpoint{2.567201in}{0.515082in}}%
\pgfpathlineto{\pgfqpoint{2.570004in}{0.473778in}}%
\pgfpathlineto{\pgfqpoint{2.572808in}{0.659848in}}%
\pgfpathlineto{\pgfqpoint{2.575611in}{0.586992in}}%
\pgfpathlineto{\pgfqpoint{2.578414in}{1.676848in}}%
\pgfpathlineto{\pgfqpoint{2.581217in}{0.716795in}}%
\pgfpathlineto{\pgfqpoint{2.584020in}{0.473213in}}%
\pgfpathlineto{\pgfqpoint{2.586824in}{0.541863in}}%
\pgfpathlineto{\pgfqpoint{2.589627in}{0.800325in}}%
\pgfpathlineto{\pgfqpoint{2.592430in}{0.495956in}}%
\pgfpathlineto{\pgfqpoint{2.595233in}{0.584754in}}%
\pgfpathlineto{\pgfqpoint{2.598037in}{0.578521in}}%
\pgfpathlineto{\pgfqpoint{2.600840in}{0.493861in}}%
\pgfpathlineto{\pgfqpoint{2.603643in}{0.675332in}}%
\pgfpathlineto{\pgfqpoint{2.606446in}{0.520192in}}%
\pgfpathlineto{\pgfqpoint{2.609250in}{0.572118in}}%
\pgfpathlineto{\pgfqpoint{2.612053in}{0.516491in}}%
\pgfpathlineto{\pgfqpoint{2.614856in}{0.473038in}}%
\pgfpathlineto{\pgfqpoint{2.617659in}{0.477167in}}%
\pgfpathlineto{\pgfqpoint{2.620462in}{0.557351in}}%
\pgfpathlineto{\pgfqpoint{2.623266in}{0.593300in}}%
\pgfpathlineto{\pgfqpoint{2.626069in}{0.489181in}}%
\pgfpathlineto{\pgfqpoint{2.628872in}{0.595874in}}%
\pgfpathlineto{\pgfqpoint{2.631675in}{0.468908in}}%
\pgfpathlineto{\pgfqpoint{2.634479in}{0.512940in}}%
\pgfpathlineto{\pgfqpoint{2.637282in}{0.468908in}}%
\pgfpathlineto{\pgfqpoint{2.640085in}{0.556506in}}%
\pgfpathlineto{\pgfqpoint{2.642888in}{0.544065in}}%
\pgfpathlineto{\pgfqpoint{2.645692in}{0.490617in}}%
\pgfpathlineto{\pgfqpoint{2.648495in}{0.573158in}}%
\pgfpathlineto{\pgfqpoint{2.651298in}{0.480666in}}%
\pgfpathlineto{\pgfqpoint{2.654101in}{0.517826in}}%
\pgfpathlineto{\pgfqpoint{2.656904in}{0.582695in}}%
\pgfpathlineto{\pgfqpoint{2.659708in}{0.535737in}}%
\pgfpathlineto{\pgfqpoint{2.662511in}{0.770514in}}%
\pgfpathlineto{\pgfqpoint{2.665314in}{0.472926in}}%
\pgfpathlineto{\pgfqpoint{2.668117in}{0.628666in}}%
\pgfpathlineto{\pgfqpoint{2.670921in}{0.742667in}}%
\pgfpathlineto{\pgfqpoint{2.673724in}{0.709440in}}%
\pgfpathlineto{\pgfqpoint{2.676527in}{0.586619in}}%
\pgfpathlineto{\pgfqpoint{2.679330in}{0.588571in}}%
\pgfpathlineto{\pgfqpoint{2.682134in}{0.517618in}}%
\pgfpathlineto{\pgfqpoint{2.684937in}{0.579256in}}%
\pgfpathlineto{\pgfqpoint{2.687740in}{0.515250in}}%
\pgfpathlineto{\pgfqpoint{2.690543in}{0.525137in}}%
\pgfpathlineto{\pgfqpoint{2.693346in}{0.509786in}}%
\pgfpathlineto{\pgfqpoint{2.696150in}{0.578580in}}%
\pgfpathlineto{\pgfqpoint{2.698953in}{0.600009in}}%
\pgfpathlineto{\pgfqpoint{2.701756in}{0.744449in}}%
\pgfpathlineto{\pgfqpoint{2.704559in}{0.548773in}}%
\pgfpathlineto{\pgfqpoint{2.707363in}{0.634058in}}%
\pgfpathlineto{\pgfqpoint{2.710166in}{0.633835in}}%
\pgfpathlineto{\pgfqpoint{2.712969in}{0.653616in}}%
\pgfpathlineto{\pgfqpoint{2.715772in}{0.494671in}}%
\pgfpathlineto{\pgfqpoint{2.718576in}{0.506543in}}%
\pgfpathlineto{\pgfqpoint{2.721379in}{0.720654in}}%
\pgfpathlineto{\pgfqpoint{2.724182in}{0.586534in}}%
\pgfpathlineto{\pgfqpoint{2.726985in}{0.558243in}}%
\pgfpathlineto{\pgfqpoint{2.729788in}{0.559612in}}%
\pgfpathlineto{\pgfqpoint{2.732592in}{0.490979in}}%
\pgfpathlineto{\pgfqpoint{2.735395in}{0.482933in}}%
\pgfpathlineto{\pgfqpoint{2.738198in}{0.502903in}}%
\pgfpathlineto{\pgfqpoint{2.741001in}{0.553065in}}%
\pgfpathlineto{\pgfqpoint{2.743805in}{0.739243in}}%
\pgfpathlineto{\pgfqpoint{2.746608in}{0.701227in}}%
\pgfpathlineto{\pgfqpoint{2.749411in}{0.552380in}}%
\pgfpathlineto{\pgfqpoint{2.752214in}{0.809428in}}%
\pgfpathlineto{\pgfqpoint{2.755018in}{0.489125in}}%
\pgfpathlineto{\pgfqpoint{2.757821in}{0.683192in}}%
\pgfpathlineto{\pgfqpoint{2.760624in}{0.708791in}}%
\pgfpathlineto{\pgfqpoint{2.763427in}{0.788018in}}%
\pgfpathlineto{\pgfqpoint{2.769034in}{0.559896in}}%
\pgfpathlineto{\pgfqpoint{2.771837in}{0.511729in}}%
\pgfpathlineto{\pgfqpoint{2.774640in}{0.625842in}}%
\pgfpathlineto{\pgfqpoint{2.777443in}{0.695917in}}%
\pgfpathlineto{\pgfqpoint{2.780247in}{0.527645in}}%
\pgfpathlineto{\pgfqpoint{2.783050in}{0.525754in}}%
\pgfpathlineto{\pgfqpoint{2.785853in}{0.548566in}}%
\pgfpathlineto{\pgfqpoint{2.788656in}{0.468908in}}%
\pgfpathlineto{\pgfqpoint{2.791460in}{0.646487in}}%
\pgfpathlineto{\pgfqpoint{2.794263in}{0.468908in}}%
\pgfpathlineto{\pgfqpoint{2.797066in}{0.493315in}}%
\pgfpathlineto{\pgfqpoint{2.799869in}{0.493315in}}%
\pgfpathlineto{\pgfqpoint{2.802672in}{0.502624in}}%
\pgfpathlineto{\pgfqpoint{2.805476in}{0.579765in}}%
\pgfpathlineto{\pgfqpoint{2.808279in}{0.783836in}}%
\pgfpathlineto{\pgfqpoint{2.811082in}{0.602100in}}%
\pgfpathlineto{\pgfqpoint{2.816689in}{0.485731in}}%
\pgfpathlineto{\pgfqpoint{2.819492in}{0.687516in}}%
\pgfpathlineto{\pgfqpoint{2.822295in}{0.565496in}}%
\pgfpathlineto{\pgfqpoint{2.825098in}{0.757099in}}%
\pgfpathlineto{\pgfqpoint{2.827902in}{0.490166in}}%
\pgfpathlineto{\pgfqpoint{2.830705in}{0.693172in}}%
\pgfpathlineto{\pgfqpoint{2.833508in}{0.679640in}}%
\pgfpathlineto{\pgfqpoint{2.836311in}{0.654921in}}%
\pgfpathlineto{\pgfqpoint{2.839114in}{0.680065in}}%
\pgfpathlineto{\pgfqpoint{2.841918in}{0.617095in}}%
\pgfpathlineto{\pgfqpoint{2.844721in}{0.488053in}}%
\pgfpathlineto{\pgfqpoint{2.847524in}{0.579540in}}%
\pgfpathlineto{\pgfqpoint{2.853131in}{0.472724in}}%
\pgfpathlineto{\pgfqpoint{2.855934in}{0.725272in}}%
\pgfpathlineto{\pgfqpoint{2.858737in}{0.494253in}}%
\pgfpathlineto{\pgfqpoint{2.861540in}{0.478670in}}%
\pgfpathlineto{\pgfqpoint{2.864344in}{0.490374in}}%
\pgfpathlineto{\pgfqpoint{2.867147in}{0.515791in}}%
\pgfpathlineto{\pgfqpoint{2.869950in}{0.512041in}}%
\pgfpathlineto{\pgfqpoint{2.872753in}{0.490493in}}%
\pgfpathlineto{\pgfqpoint{2.875556in}{0.486566in}}%
\pgfpathlineto{\pgfqpoint{2.878360in}{0.582264in}}%
\pgfpathlineto{\pgfqpoint{2.881163in}{0.600500in}}%
\pgfpathlineto{\pgfqpoint{2.883966in}{0.480463in}}%
\pgfpathlineto{\pgfqpoint{2.886769in}{0.488190in}}%
\pgfpathlineto{\pgfqpoint{2.889573in}{0.604731in}}%
\pgfpathlineto{\pgfqpoint{2.892376in}{0.484526in}}%
\pgfpathlineto{\pgfqpoint{2.895179in}{0.496226in}}%
\pgfpathlineto{\pgfqpoint{2.897982in}{0.474757in}}%
\pgfpathlineto{\pgfqpoint{2.900786in}{0.683313in}}%
\pgfpathlineto{\pgfqpoint{2.903589in}{0.506549in}}%
\pgfpathlineto{\pgfqpoint{2.906392in}{0.588990in}}%
\pgfpathlineto{\pgfqpoint{2.909195in}{0.964298in}}%
\pgfpathlineto{\pgfqpoint{2.911998in}{0.521722in}}%
\pgfpathlineto{\pgfqpoint{2.914802in}{0.514357in}}%
\pgfpathlineto{\pgfqpoint{2.917605in}{0.537396in}}%
\pgfpathlineto{\pgfqpoint{2.920408in}{0.530106in}}%
\pgfpathlineto{\pgfqpoint{2.923211in}{0.573906in}}%
\pgfpathlineto{\pgfqpoint{2.926015in}{0.708842in}}%
\pgfpathlineto{\pgfqpoint{2.928818in}{0.521350in}}%
\pgfpathlineto{\pgfqpoint{2.931621in}{0.503993in}}%
\pgfpathlineto{\pgfqpoint{2.934424in}{0.538979in}}%
\pgfpathlineto{\pgfqpoint{2.937228in}{0.523373in}}%
\pgfpathlineto{\pgfqpoint{2.940031in}{0.639284in}}%
\pgfpathlineto{\pgfqpoint{2.942834in}{0.995927in}}%
\pgfpathlineto{\pgfqpoint{2.945637in}{0.550254in}}%
\pgfpathlineto{\pgfqpoint{2.948440in}{0.526543in}}%
\pgfpathlineto{\pgfqpoint{2.951244in}{0.481946in}}%
\pgfpathlineto{\pgfqpoint{2.954047in}{0.710402in}}%
\pgfpathlineto{\pgfqpoint{2.956850in}{0.545822in}}%
\pgfpathlineto{\pgfqpoint{2.959653in}{0.545822in}}%
\pgfpathlineto{\pgfqpoint{2.962457in}{0.498083in}}%
\pgfpathlineto{\pgfqpoint{2.965260in}{0.536480in}}%
\pgfpathlineto{\pgfqpoint{2.968063in}{0.542376in}}%
\pgfpathlineto{\pgfqpoint{2.970866in}{0.513041in}}%
\pgfpathlineto{\pgfqpoint{2.973670in}{0.661299in}}%
\pgfpathlineto{\pgfqpoint{2.976473in}{0.639252in}}%
\pgfpathlineto{\pgfqpoint{2.979276in}{0.479936in}}%
\pgfpathlineto{\pgfqpoint{2.982079in}{0.626067in}}%
\pgfpathlineto{\pgfqpoint{2.987686in}{0.481858in}}%
\pgfpathlineto{\pgfqpoint{2.990489in}{0.485578in}}%
\pgfpathlineto{\pgfqpoint{2.993292in}{0.601209in}}%
\pgfpathlineto{\pgfqpoint{2.996095in}{0.496977in}}%
\pgfpathlineto{\pgfqpoint{2.998899in}{0.594778in}}%
\pgfpathlineto{\pgfqpoint{3.001702in}{0.517905in}}%
\pgfpathlineto{\pgfqpoint{3.004505in}{0.771742in}}%
\pgfpathlineto{\pgfqpoint{3.007308in}{0.470743in}}%
\pgfpathlineto{\pgfqpoint{3.010112in}{0.524083in}}%
\pgfpathlineto{\pgfqpoint{3.012915in}{0.604203in}}%
\pgfpathlineto{\pgfqpoint{3.015718in}{0.546929in}}%
\pgfpathlineto{\pgfqpoint{3.018521in}{0.515290in}}%
\pgfpathlineto{\pgfqpoint{3.021325in}{0.603860in}}%
\pgfpathlineto{\pgfqpoint{3.024128in}{0.615017in}}%
\pgfpathlineto{\pgfqpoint{3.026931in}{0.541253in}}%
\pgfpathlineto{\pgfqpoint{3.029734in}{0.527954in}}%
\pgfpathlineto{\pgfqpoint{3.032537in}{0.568712in}}%
\pgfpathlineto{\pgfqpoint{3.035341in}{0.513525in}}%
\pgfpathlineto{\pgfqpoint{3.038144in}{0.493095in}}%
\pgfpathlineto{\pgfqpoint{3.040947in}{0.840608in}}%
\pgfpathlineto{\pgfqpoint{3.043750in}{0.549095in}}%
\pgfpathlineto{\pgfqpoint{3.046554in}{0.597643in}}%
\pgfpathlineto{\pgfqpoint{3.052160in}{0.491399in}}%
\pgfpathlineto{\pgfqpoint{3.054963in}{0.532855in}}%
\pgfpathlineto{\pgfqpoint{3.057767in}{0.660839in}}%
\pgfpathlineto{\pgfqpoint{3.060570in}{0.528134in}}%
\pgfpathlineto{\pgfqpoint{3.063373in}{0.719842in}}%
\pgfpathlineto{\pgfqpoint{3.066176in}{0.667160in}}%
\pgfpathlineto{\pgfqpoint{3.071783in}{0.483500in}}%
\pgfpathlineto{\pgfqpoint{3.074586in}{0.565261in}}%
\pgfpathlineto{\pgfqpoint{3.077389in}{0.523117in}}%
\pgfpathlineto{\pgfqpoint{3.080192in}{0.783998in}}%
\pgfpathlineto{\pgfqpoint{3.082996in}{0.585464in}}%
\pgfpathlineto{\pgfqpoint{3.085799in}{0.477783in}}%
\pgfpathlineto{\pgfqpoint{3.088602in}{0.506252in}}%
\pgfpathlineto{\pgfqpoint{3.091405in}{0.983454in}}%
\pgfpathlineto{\pgfqpoint{3.094209in}{0.625364in}}%
\pgfpathlineto{\pgfqpoint{3.097012in}{0.615109in}}%
\pgfpathlineto{\pgfqpoint{3.099815in}{0.677459in}}%
\pgfpathlineto{\pgfqpoint{3.102618in}{0.529289in}}%
\pgfpathlineto{\pgfqpoint{3.105421in}{0.637365in}}%
\pgfpathlineto{\pgfqpoint{3.108225in}{0.538253in}}%
\pgfpathlineto{\pgfqpoint{3.111028in}{0.550131in}}%
\pgfpathlineto{\pgfqpoint{3.113831in}{0.516535in}}%
\pgfpathlineto{\pgfqpoint{3.116634in}{0.702883in}}%
\pgfpathlineto{\pgfqpoint{3.119438in}{0.574399in}}%
\pgfpathlineto{\pgfqpoint{3.122241in}{0.498148in}}%
\pgfpathlineto{\pgfqpoint{3.125044in}{0.640165in}}%
\pgfpathlineto{\pgfqpoint{3.127847in}{0.519320in}}%
\pgfpathlineto{\pgfqpoint{3.130651in}{0.534656in}}%
\pgfpathlineto{\pgfqpoint{3.139060in}{0.479191in}}%
\pgfpathlineto{\pgfqpoint{3.141863in}{0.480906in}}%
\pgfpathlineto{\pgfqpoint{3.144667in}{0.542868in}}%
\pgfpathlineto{\pgfqpoint{3.147470in}{0.940099in}}%
\pgfpathlineto{\pgfqpoint{3.150273in}{0.971797in}}%
\pgfpathlineto{\pgfqpoint{3.153076in}{0.723461in}}%
\pgfpathlineto{\pgfqpoint{3.155880in}{1.138227in}}%
\pgfpathlineto{\pgfqpoint{3.158683in}{0.764946in}}%
\pgfpathlineto{\pgfqpoint{3.161486in}{0.479477in}}%
\pgfpathlineto{\pgfqpoint{3.167093in}{0.881928in}}%
\pgfpathlineto{\pgfqpoint{3.169896in}{0.586755in}}%
\pgfpathlineto{\pgfqpoint{3.172699in}{0.607410in}}%
\pgfpathlineto{\pgfqpoint{3.175502in}{0.688310in}}%
\pgfpathlineto{\pgfqpoint{3.178305in}{0.711847in}}%
\pgfpathlineto{\pgfqpoint{3.181109in}{0.634613in}}%
\pgfpathlineto{\pgfqpoint{3.183912in}{0.582046in}}%
\pgfpathlineto{\pgfqpoint{3.186715in}{0.561205in}}%
\pgfpathlineto{\pgfqpoint{3.189518in}{0.601232in}}%
\pgfpathlineto{\pgfqpoint{3.192322in}{0.557884in}}%
\pgfpathlineto{\pgfqpoint{3.195125in}{0.551859in}}%
\pgfpathlineto{\pgfqpoint{3.197928in}{0.474301in}}%
\pgfpathlineto{\pgfqpoint{3.200731in}{0.673771in}}%
\pgfpathlineto{\pgfqpoint{3.203535in}{0.643167in}}%
\pgfpathlineto{\pgfqpoint{3.206338in}{0.562994in}}%
\pgfpathlineto{\pgfqpoint{3.209141in}{0.591800in}}%
\pgfpathlineto{\pgfqpoint{3.211944in}{0.593616in}}%
\pgfpathlineto{\pgfqpoint{3.214747in}{0.548577in}}%
\pgfpathlineto{\pgfqpoint{3.217551in}{1.096385in}}%
\pgfpathlineto{\pgfqpoint{3.220354in}{0.713306in}}%
\pgfpathlineto{\pgfqpoint{3.223157in}{0.670173in}}%
\pgfpathlineto{\pgfqpoint{3.225960in}{0.527379in}}%
\pgfpathlineto{\pgfqpoint{3.228764in}{0.588640in}}%
\pgfpathlineto{\pgfqpoint{3.231567in}{0.740329in}}%
\pgfpathlineto{\pgfqpoint{3.234370in}{0.495430in}}%
\pgfpathlineto{\pgfqpoint{3.237173in}{0.663838in}}%
\pgfpathlineto{\pgfqpoint{3.239977in}{0.583374in}}%
\pgfpathlineto{\pgfqpoint{3.242780in}{0.482711in}}%
\pgfpathlineto{\pgfqpoint{3.245583in}{0.636759in}}%
\pgfpathlineto{\pgfqpoint{3.248386in}{0.470609in}}%
\pgfpathlineto{\pgfqpoint{3.251189in}{0.602300in}}%
\pgfpathlineto{\pgfqpoint{3.253993in}{0.658305in}}%
\pgfpathlineto{\pgfqpoint{3.256796in}{0.576807in}}%
\pgfpathlineto{\pgfqpoint{3.259599in}{0.629015in}}%
\pgfpathlineto{\pgfqpoint{3.262402in}{0.583775in}}%
\pgfpathlineto{\pgfqpoint{3.268009in}{0.625158in}}%
\pgfpathlineto{\pgfqpoint{3.270812in}{0.573646in}}%
\pgfpathlineto{\pgfqpoint{3.273615in}{0.646423in}}%
\pgfpathlineto{\pgfqpoint{3.276419in}{0.573794in}}%
\pgfpathlineto{\pgfqpoint{3.279222in}{0.682654in}}%
\pgfpathlineto{\pgfqpoint{3.282025in}{0.525655in}}%
\pgfpathlineto{\pgfqpoint{3.284828in}{0.616035in}}%
\pgfpathlineto{\pgfqpoint{3.287631in}{0.851344in}}%
\pgfpathlineto{\pgfqpoint{3.290435in}{0.902219in}}%
\pgfpathlineto{\pgfqpoint{3.293238in}{0.540760in}}%
\pgfpathlineto{\pgfqpoint{3.296041in}{0.611374in}}%
\pgfpathlineto{\pgfqpoint{3.298844in}{0.549653in}}%
\pgfpathlineto{\pgfqpoint{3.301648in}{0.579574in}}%
\pgfpathlineto{\pgfqpoint{3.304451in}{0.716529in}}%
\pgfpathlineto{\pgfqpoint{3.307254in}{0.515464in}}%
\pgfpathlineto{\pgfqpoint{3.310057in}{0.606382in}}%
\pgfpathlineto{\pgfqpoint{3.312861in}{0.499847in}}%
\pgfpathlineto{\pgfqpoint{3.315664in}{0.593818in}}%
\pgfpathlineto{\pgfqpoint{3.318467in}{0.480211in}}%
\pgfpathlineto{\pgfqpoint{3.321270in}{0.721567in}}%
\pgfpathlineto{\pgfqpoint{3.324073in}{0.516486in}}%
\pgfpathlineto{\pgfqpoint{3.326877in}{0.475262in}}%
\pgfpathlineto{\pgfqpoint{3.329680in}{0.481620in}}%
\pgfpathlineto{\pgfqpoint{3.332483in}{0.518290in}}%
\pgfpathlineto{\pgfqpoint{3.335286in}{0.504073in}}%
\pgfpathlineto{\pgfqpoint{3.338090in}{0.512052in}}%
\pgfpathlineto{\pgfqpoint{3.340893in}{0.598793in}}%
\pgfpathlineto{\pgfqpoint{3.343696in}{0.609954in}}%
\pgfpathlineto{\pgfqpoint{3.346499in}{0.605119in}}%
\pgfpathlineto{\pgfqpoint{3.349303in}{0.587097in}}%
\pgfpathlineto{\pgfqpoint{3.352106in}{0.799571in}}%
\pgfpathlineto{\pgfqpoint{3.354909in}{0.601076in}}%
\pgfpathlineto{\pgfqpoint{3.357712in}{0.544380in}}%
\pgfpathlineto{\pgfqpoint{3.360515in}{0.655538in}}%
\pgfpathlineto{\pgfqpoint{3.363319in}{0.506459in}}%
\pgfpathlineto{\pgfqpoint{3.366122in}{0.790951in}}%
\pgfpathlineto{\pgfqpoint{3.368925in}{0.590450in}}%
\pgfpathlineto{\pgfqpoint{3.371728in}{0.749032in}}%
\pgfpathlineto{\pgfqpoint{3.374532in}{0.661770in}}%
\pgfpathlineto{\pgfqpoint{3.377335in}{0.652055in}}%
\pgfpathlineto{\pgfqpoint{3.380138in}{0.846871in}}%
\pgfpathlineto{\pgfqpoint{3.382941in}{0.606637in}}%
\pgfpathlineto{\pgfqpoint{3.385745in}{0.501868in}}%
\pgfpathlineto{\pgfqpoint{3.388548in}{0.656744in}}%
\pgfpathlineto{\pgfqpoint{3.391351in}{0.516007in}}%
\pgfpathlineto{\pgfqpoint{3.394154in}{0.511143in}}%
\pgfpathlineto{\pgfqpoint{3.396957in}{0.575478in}}%
\pgfpathlineto{\pgfqpoint{3.399761in}{0.604703in}}%
\pgfpathlineto{\pgfqpoint{3.402564in}{0.594708in}}%
\pgfpathlineto{\pgfqpoint{3.405367in}{0.768071in}}%
\pgfpathlineto{\pgfqpoint{3.408170in}{0.561088in}}%
\pgfpathlineto{\pgfqpoint{3.413777in}{0.714971in}}%
\pgfpathlineto{\pgfqpoint{3.416580in}{0.623470in}}%
\pgfpathlineto{\pgfqpoint{3.419383in}{0.645886in}}%
\pgfpathlineto{\pgfqpoint{3.422187in}{0.607670in}}%
\pgfpathlineto{\pgfqpoint{3.424990in}{0.746262in}}%
\pgfpathlineto{\pgfqpoint{3.427793in}{0.586810in}}%
\pgfpathlineto{\pgfqpoint{3.430596in}{0.804741in}}%
\pgfpathlineto{\pgfqpoint{3.433400in}{0.543601in}}%
\pgfpathlineto{\pgfqpoint{3.436203in}{0.592605in}}%
\pgfpathlineto{\pgfqpoint{3.439006in}{0.621113in}}%
\pgfpathlineto{\pgfqpoint{3.441809in}{0.664894in}}%
\pgfpathlineto{\pgfqpoint{3.444612in}{0.632901in}}%
\pgfpathlineto{\pgfqpoint{3.447416in}{0.491963in}}%
\pgfpathlineto{\pgfqpoint{3.450219in}{0.636587in}}%
\pgfpathlineto{\pgfqpoint{3.453022in}{0.748620in}}%
\pgfpathlineto{\pgfqpoint{3.455825in}{1.358351in}}%
\pgfpathlineto{\pgfqpoint{3.458629in}{0.487728in}}%
\pgfpathlineto{\pgfqpoint{3.461432in}{0.632651in}}%
\pgfpathlineto{\pgfqpoint{3.464235in}{0.632651in}}%
\pgfpathlineto{\pgfqpoint{3.467038in}{0.585893in}}%
\pgfpathlineto{\pgfqpoint{3.469842in}{0.833762in}}%
\pgfpathlineto{\pgfqpoint{3.472645in}{1.138229in}}%
\pgfpathlineto{\pgfqpoint{3.475448in}{0.569964in}}%
\pgfpathlineto{\pgfqpoint{3.478251in}{0.819873in}}%
\pgfpathlineto{\pgfqpoint{3.481054in}{0.769652in}}%
\pgfpathlineto{\pgfqpoint{3.483858in}{0.816667in}}%
\pgfpathlineto{\pgfqpoint{3.486661in}{0.589246in}}%
\pgfpathlineto{\pgfqpoint{3.489464in}{0.653416in}}%
\pgfpathlineto{\pgfqpoint{3.492267in}{0.605250in}}%
\pgfpathlineto{\pgfqpoint{3.495071in}{0.518661in}}%
\pgfpathlineto{\pgfqpoint{3.497874in}{0.732089in}}%
\pgfpathlineto{\pgfqpoint{3.503480in}{0.521759in}}%
\pgfpathlineto{\pgfqpoint{3.506284in}{0.711910in}}%
\pgfpathlineto{\pgfqpoint{3.509087in}{0.561002in}}%
\pgfpathlineto{\pgfqpoint{3.511890in}{0.526597in}}%
\pgfpathlineto{\pgfqpoint{3.514693in}{0.848522in}}%
\pgfpathlineto{\pgfqpoint{3.517496in}{0.525113in}}%
\pgfpathlineto{\pgfqpoint{3.520300in}{0.533947in}}%
\pgfpathlineto{\pgfqpoint{3.523103in}{0.470624in}}%
\pgfpathlineto{\pgfqpoint{3.525906in}{0.802901in}}%
\pgfpathlineto{\pgfqpoint{3.528709in}{0.698478in}}%
\pgfpathlineto{\pgfqpoint{3.531513in}{0.479680in}}%
\pgfpathlineto{\pgfqpoint{3.534316in}{0.549932in}}%
\pgfpathlineto{\pgfqpoint{3.537119in}{0.715719in}}%
\pgfpathlineto{\pgfqpoint{3.539922in}{0.493724in}}%
\pgfpathlineto{\pgfqpoint{3.542726in}{0.541443in}}%
\pgfpathlineto{\pgfqpoint{3.545529in}{0.548033in}}%
\pgfpathlineto{\pgfqpoint{3.548332in}{0.599670in}}%
\pgfpathlineto{\pgfqpoint{3.551135in}{0.577691in}}%
\pgfpathlineto{\pgfqpoint{3.553938in}{0.470627in}}%
\pgfpathlineto{\pgfqpoint{3.556742in}{0.605491in}}%
\pgfpathlineto{\pgfqpoint{3.559545in}{0.629539in}}%
\pgfpathlineto{\pgfqpoint{3.562348in}{0.506604in}}%
\pgfpathlineto{\pgfqpoint{3.565151in}{0.479177in}}%
\pgfpathlineto{\pgfqpoint{3.567955in}{0.684478in}}%
\pgfpathlineto{\pgfqpoint{3.570758in}{0.697320in}}%
\pgfpathlineto{\pgfqpoint{3.573561in}{0.518566in}}%
\pgfpathlineto{\pgfqpoint{3.576364in}{0.648388in}}%
\pgfpathlineto{\pgfqpoint{3.579168in}{0.506450in}}%
\pgfpathlineto{\pgfqpoint{3.581971in}{0.606572in}}%
\pgfpathlineto{\pgfqpoint{3.584774in}{0.666728in}}%
\pgfpathlineto{\pgfqpoint{3.587577in}{0.568240in}}%
\pgfpathlineto{\pgfqpoint{3.590380in}{0.540637in}}%
\pgfpathlineto{\pgfqpoint{3.593184in}{0.503090in}}%
\pgfpathlineto{\pgfqpoint{3.598790in}{0.654882in}}%
\pgfpathlineto{\pgfqpoint{3.601593in}{0.564016in}}%
\pgfpathlineto{\pgfqpoint{3.604397in}{0.506864in}}%
\pgfpathlineto{\pgfqpoint{3.607200in}{0.679355in}}%
\pgfpathlineto{\pgfqpoint{3.610003in}{0.576825in}}%
\pgfpathlineto{\pgfqpoint{3.612806in}{0.530020in}}%
\pgfpathlineto{\pgfqpoint{3.615610in}{0.523738in}}%
\pgfpathlineto{\pgfqpoint{3.618413in}{0.729113in}}%
\pgfpathlineto{\pgfqpoint{3.624019in}{0.530435in}}%
\pgfpathlineto{\pgfqpoint{3.626822in}{0.504389in}}%
\pgfpathlineto{\pgfqpoint{3.629626in}{0.639175in}}%
\pgfpathlineto{\pgfqpoint{3.632429in}{0.539318in}}%
\pgfpathlineto{\pgfqpoint{3.635232in}{0.662792in}}%
\pgfpathlineto{\pgfqpoint{3.638035in}{0.671005in}}%
\pgfpathlineto{\pgfqpoint{3.640839in}{0.488642in}}%
\pgfpathlineto{\pgfqpoint{3.643642in}{0.514905in}}%
\pgfpathlineto{\pgfqpoint{3.646445in}{0.527797in}}%
\pgfpathlineto{\pgfqpoint{3.649248in}{0.546994in}}%
\pgfpathlineto{\pgfqpoint{3.652052in}{0.623618in}}%
\pgfpathlineto{\pgfqpoint{3.654855in}{0.747482in}}%
\pgfpathlineto{\pgfqpoint{3.657658in}{0.563548in}}%
\pgfpathlineto{\pgfqpoint{3.660461in}{0.639125in}}%
\pgfpathlineto{\pgfqpoint{3.663264in}{0.624487in}}%
\pgfpathlineto{\pgfqpoint{3.666068in}{0.631080in}}%
\pgfpathlineto{\pgfqpoint{3.668871in}{0.572354in}}%
\pgfpathlineto{\pgfqpoint{3.671674in}{0.557523in}}%
\pgfpathlineto{\pgfqpoint{3.674477in}{0.595075in}}%
\pgfpathlineto{\pgfqpoint{3.677281in}{0.542485in}}%
\pgfpathlineto{\pgfqpoint{3.680084in}{0.811781in}}%
\pgfpathlineto{\pgfqpoint{3.682887in}{0.518456in}}%
\pgfpathlineto{\pgfqpoint{3.685690in}{0.478521in}}%
\pgfpathlineto{\pgfqpoint{3.688494in}{0.572653in}}%
\pgfpathlineto{\pgfqpoint{3.691297in}{0.582273in}}%
\pgfpathlineto{\pgfqpoint{3.694100in}{0.513739in}}%
\pgfpathlineto{\pgfqpoint{3.696903in}{0.581865in}}%
\pgfpathlineto{\pgfqpoint{3.699706in}{0.470492in}}%
\pgfpathlineto{\pgfqpoint{3.702510in}{0.562008in}}%
\pgfpathlineto{\pgfqpoint{3.705313in}{0.478336in}}%
\pgfpathlineto{\pgfqpoint{3.708116in}{0.628108in}}%
\pgfpathlineto{\pgfqpoint{3.710919in}{0.527979in}}%
\pgfpathlineto{\pgfqpoint{3.716526in}{0.748376in}}%
\pgfpathlineto{\pgfqpoint{3.719329in}{0.496334in}}%
\pgfpathlineto{\pgfqpoint{3.722132in}{0.531918in}}%
\pgfpathlineto{\pgfqpoint{3.724936in}{0.498029in}}%
\pgfpathlineto{\pgfqpoint{3.727739in}{0.685717in}}%
\pgfpathlineto{\pgfqpoint{3.730542in}{0.524695in}}%
\pgfpathlineto{\pgfqpoint{3.733345in}{0.470545in}}%
\pgfpathlineto{\pgfqpoint{3.738952in}{0.740535in}}%
\pgfpathlineto{\pgfqpoint{3.741755in}{0.982775in}}%
\pgfpathlineto{\pgfqpoint{3.744558in}{0.753903in}}%
\pgfpathlineto{\pgfqpoint{3.747361in}{0.772447in}}%
\pgfpathlineto{\pgfqpoint{3.750165in}{0.725648in}}%
\pgfpathlineto{\pgfqpoint{3.752968in}{0.890136in}}%
\pgfpathlineto{\pgfqpoint{3.755771in}{0.520014in}}%
\pgfpathlineto{\pgfqpoint{3.758574in}{0.501166in}}%
\pgfpathlineto{\pgfqpoint{3.761378in}{0.571065in}}%
\pgfpathlineto{\pgfqpoint{3.764181in}{0.543905in}}%
\pgfpathlineto{\pgfqpoint{3.766984in}{0.782675in}}%
\pgfpathlineto{\pgfqpoint{3.769787in}{0.485465in}}%
\pgfpathlineto{\pgfqpoint{3.772590in}{0.621838in}}%
\pgfpathlineto{\pgfqpoint{3.775394in}{0.470542in}}%
\pgfpathlineto{\pgfqpoint{3.778197in}{0.576319in}}%
\pgfpathlineto{\pgfqpoint{3.781000in}{0.494807in}}%
\pgfpathlineto{\pgfqpoint{3.783803in}{0.470525in}}%
\pgfpathlineto{\pgfqpoint{3.786607in}{0.533412in}}%
\pgfpathlineto{\pgfqpoint{3.789410in}{0.568229in}}%
\pgfpathlineto{\pgfqpoint{3.792213in}{0.556976in}}%
\pgfpathlineto{\pgfqpoint{3.795016in}{0.642837in}}%
\pgfpathlineto{\pgfqpoint{3.797820in}{0.620360in}}%
\pgfpathlineto{\pgfqpoint{3.800623in}{0.539687in}}%
\pgfpathlineto{\pgfqpoint{3.803426in}{0.475359in}}%
\pgfpathlineto{\pgfqpoint{3.806229in}{0.573272in}}%
\pgfpathlineto{\pgfqpoint{3.809032in}{0.649286in}}%
\pgfpathlineto{\pgfqpoint{3.811836in}{0.502929in}}%
\pgfpathlineto{\pgfqpoint{3.814639in}{0.472143in}}%
\pgfpathlineto{\pgfqpoint{3.820245in}{0.588970in}}%
\pgfpathlineto{\pgfqpoint{3.823049in}{0.572028in}}%
\pgfpathlineto{\pgfqpoint{3.825852in}{0.498962in}}%
\pgfpathlineto{\pgfqpoint{3.828655in}{0.486316in}}%
\pgfpathlineto{\pgfqpoint{3.831458in}{0.527562in}}%
\pgfpathlineto{\pgfqpoint{3.834262in}{0.537047in}}%
\pgfpathlineto{\pgfqpoint{3.837065in}{0.484721in}}%
\pgfpathlineto{\pgfqpoint{3.839868in}{0.605808in}}%
\pgfpathlineto{\pgfqpoint{3.842671in}{0.484567in}}%
\pgfpathlineto{\pgfqpoint{3.845474in}{0.495516in}}%
\pgfpathlineto{\pgfqpoint{3.848278in}{0.556776in}}%
\pgfpathlineto{\pgfqpoint{3.851081in}{0.500361in}}%
\pgfpathlineto{\pgfqpoint{3.853884in}{0.558113in}}%
\pgfpathlineto{\pgfqpoint{3.856687in}{0.507954in}}%
\pgfpathlineto{\pgfqpoint{3.859491in}{0.547371in}}%
\pgfpathlineto{\pgfqpoint{3.862294in}{0.473630in}}%
\pgfpathlineto{\pgfqpoint{3.865097in}{0.506618in}}%
\pgfpathlineto{\pgfqpoint{3.867900in}{0.515885in}}%
\pgfpathlineto{\pgfqpoint{3.870704in}{0.514150in}}%
\pgfpathlineto{\pgfqpoint{3.873507in}{0.509462in}}%
\pgfpathlineto{\pgfqpoint{3.876310in}{0.529690in}}%
\pgfpathlineto{\pgfqpoint{3.879113in}{0.574178in}}%
\pgfpathlineto{\pgfqpoint{3.881917in}{0.548809in}}%
\pgfpathlineto{\pgfqpoint{3.884720in}{0.575659in}}%
\pgfpathlineto{\pgfqpoint{3.887523in}{0.509974in}}%
\pgfpathlineto{\pgfqpoint{3.890326in}{0.715101in}}%
\pgfpathlineto{\pgfqpoint{3.893129in}{0.715101in}}%
\pgfpathlineto{\pgfqpoint{3.898736in}{0.530800in}}%
\pgfpathlineto{\pgfqpoint{3.901539in}{0.547766in}}%
\pgfpathlineto{\pgfqpoint{3.904342in}{0.478152in}}%
\pgfpathlineto{\pgfqpoint{3.907146in}{0.534959in}}%
\pgfpathlineto{\pgfqpoint{3.909949in}{0.473505in}}%
\pgfpathlineto{\pgfqpoint{3.912752in}{0.580282in}}%
\pgfpathlineto{\pgfqpoint{3.915555in}{0.490153in}}%
\pgfpathlineto{\pgfqpoint{3.918359in}{0.592336in}}%
\pgfpathlineto{\pgfqpoint{3.921162in}{0.578119in}}%
\pgfpathlineto{\pgfqpoint{3.923965in}{0.521326in}}%
\pgfpathlineto{\pgfqpoint{3.926768in}{0.589854in}}%
\pgfpathlineto{\pgfqpoint{3.929571in}{0.625289in}}%
\pgfpathlineto{\pgfqpoint{3.932375in}{0.585614in}}%
\pgfpathlineto{\pgfqpoint{3.935178in}{0.493338in}}%
\pgfpathlineto{\pgfqpoint{3.937981in}{0.488754in}}%
\pgfpathlineto{\pgfqpoint{3.940784in}{0.561757in}}%
\pgfpathlineto{\pgfqpoint{3.943588in}{0.470424in}}%
\pgfpathlineto{\pgfqpoint{3.946391in}{0.540377in}}%
\pgfpathlineto{\pgfqpoint{3.949194in}{0.503933in}}%
\pgfpathlineto{\pgfqpoint{3.951997in}{0.631161in}}%
\pgfpathlineto{\pgfqpoint{3.954801in}{0.501222in}}%
\pgfpathlineto{\pgfqpoint{3.957604in}{0.525898in}}%
\pgfpathlineto{\pgfqpoint{3.960407in}{0.556595in}}%
\pgfpathlineto{\pgfqpoint{3.963210in}{0.514982in}}%
\pgfpathlineto{\pgfqpoint{3.966013in}{0.555374in}}%
\pgfpathlineto{\pgfqpoint{3.968817in}{0.654984in}}%
\pgfpathlineto{\pgfqpoint{3.971620in}{0.516315in}}%
\pgfpathlineto{\pgfqpoint{3.974423in}{0.490377in}}%
\pgfpathlineto{\pgfqpoint{3.977226in}{0.591090in}}%
\pgfpathlineto{\pgfqpoint{3.980030in}{0.640303in}}%
\pgfpathlineto{\pgfqpoint{3.982833in}{0.510583in}}%
\pgfpathlineto{\pgfqpoint{3.985636in}{0.495164in}}%
\pgfpathlineto{\pgfqpoint{3.988439in}{0.513570in}}%
\pgfpathlineto{\pgfqpoint{3.991243in}{0.511878in}}%
\pgfpathlineto{\pgfqpoint{3.994046in}{0.602914in}}%
\pgfpathlineto{\pgfqpoint{3.996849in}{0.627887in}}%
\pgfpathlineto{\pgfqpoint{3.999652in}{0.479884in}}%
\pgfpathlineto{\pgfqpoint{4.002455in}{0.512870in}}%
\pgfpathlineto{\pgfqpoint{4.005259in}{0.794879in}}%
\pgfpathlineto{\pgfqpoint{4.008062in}{0.525462in}}%
\pgfpathlineto{\pgfqpoint{4.010865in}{0.522168in}}%
\pgfpathlineto{\pgfqpoint{4.013668in}{0.674072in}}%
\pgfpathlineto{\pgfqpoint{4.016472in}{0.470451in}}%
\pgfpathlineto{\pgfqpoint{4.019275in}{1.010030in}}%
\pgfpathlineto{\pgfqpoint{4.022078in}{0.497900in}}%
\pgfpathlineto{\pgfqpoint{4.024881in}{0.731558in}}%
\pgfpathlineto{\pgfqpoint{4.030488in}{0.506305in}}%
\pgfpathlineto{\pgfqpoint{4.033291in}{0.600850in}}%
\pgfpathlineto{\pgfqpoint{4.036094in}{0.738947in}}%
\pgfpathlineto{\pgfqpoint{4.038897in}{0.524264in}}%
\pgfpathlineto{\pgfqpoint{4.041701in}{0.555785in}}%
\pgfpathlineto{\pgfqpoint{4.044504in}{0.632106in}}%
\pgfpathlineto{\pgfqpoint{4.047307in}{0.480067in}}%
\pgfpathlineto{\pgfqpoint{4.050110in}{0.759112in}}%
\pgfpathlineto{\pgfqpoint{4.052914in}{0.776106in}}%
\pgfpathlineto{\pgfqpoint{4.055717in}{0.517309in}}%
\pgfpathlineto{\pgfqpoint{4.058520in}{0.726074in}}%
\pgfpathlineto{\pgfqpoint{4.061323in}{0.501581in}}%
\pgfpathlineto{\pgfqpoint{4.064127in}{0.817296in}}%
\pgfpathlineto{\pgfqpoint{4.066930in}{0.473680in}}%
\pgfpathlineto{\pgfqpoint{4.069733in}{0.489609in}}%
\pgfpathlineto{\pgfqpoint{4.072536in}{0.569720in}}%
\pgfpathlineto{\pgfqpoint{4.075339in}{0.577686in}}%
\pgfpathlineto{\pgfqpoint{4.078143in}{0.478468in}}%
\pgfpathlineto{\pgfqpoint{4.080946in}{0.526156in}}%
\pgfpathlineto{\pgfqpoint{4.083749in}{0.649499in}}%
\pgfpathlineto{\pgfqpoint{4.086552in}{0.509220in}}%
\pgfpathlineto{\pgfqpoint{4.089356in}{0.510830in}}%
\pgfpathlineto{\pgfqpoint{4.092159in}{0.504371in}}%
\pgfpathlineto{\pgfqpoint{4.097765in}{0.507706in}}%
\pgfpathlineto{\pgfqpoint{4.100569in}{0.501147in}}%
\pgfpathlineto{\pgfqpoint{4.103372in}{0.476960in}}%
\pgfpathlineto{\pgfqpoint{4.106175in}{0.473740in}}%
\pgfpathlineto{\pgfqpoint{4.108978in}{0.517311in}}%
\pgfpathlineto{\pgfqpoint{4.111781in}{0.701123in}}%
\pgfpathlineto{\pgfqpoint{4.114585in}{0.570033in}}%
\pgfpathlineto{\pgfqpoint{4.117388in}{0.612844in}}%
\pgfpathlineto{\pgfqpoint{4.120191in}{0.638838in}}%
\pgfpathlineto{\pgfqpoint{4.122994in}{0.538143in}}%
\pgfpathlineto{\pgfqpoint{4.125798in}{0.535430in}}%
\pgfpathlineto{\pgfqpoint{4.128601in}{0.543142in}}%
\pgfpathlineto{\pgfqpoint{4.131404in}{0.533838in}}%
\pgfpathlineto{\pgfqpoint{4.134207in}{0.499947in}}%
\pgfpathlineto{\pgfqpoint{4.137011in}{0.482885in}}%
\pgfpathlineto{\pgfqpoint{4.139814in}{0.518477in}}%
\pgfpathlineto{\pgfqpoint{4.142617in}{0.490532in}}%
\pgfpathlineto{\pgfqpoint{4.145420in}{0.485872in}}%
\pgfpathlineto{\pgfqpoint{4.148223in}{0.515056in}}%
\pgfpathlineto{\pgfqpoint{4.151027in}{0.630601in}}%
\pgfpathlineto{\pgfqpoint{4.153830in}{0.568528in}}%
\pgfpathlineto{\pgfqpoint{4.156633in}{0.567013in}}%
\pgfpathlineto{\pgfqpoint{4.159436in}{0.547454in}}%
\pgfpathlineto{\pgfqpoint{4.162240in}{0.479452in}}%
\pgfpathlineto{\pgfqpoint{4.165043in}{0.615957in}}%
\pgfpathlineto{\pgfqpoint{4.167846in}{0.508622in}}%
\pgfpathlineto{\pgfqpoint{4.170649in}{0.490345in}}%
\pgfpathlineto{\pgfqpoint{4.173453in}{1.024821in}}%
\pgfpathlineto{\pgfqpoint{4.176256in}{0.505594in}}%
\pgfpathlineto{\pgfqpoint{4.179059in}{0.476258in}}%
\pgfpathlineto{\pgfqpoint{4.181862in}{0.567825in}}%
\pgfpathlineto{\pgfqpoint{4.187469in}{0.514512in}}%
\pgfpathlineto{\pgfqpoint{4.190272in}{0.546490in}}%
\pgfpathlineto{\pgfqpoint{4.193075in}{0.528592in}}%
\pgfpathlineto{\pgfqpoint{4.195878in}{0.590297in}}%
\pgfpathlineto{\pgfqpoint{4.198682in}{0.487617in}}%
\pgfpathlineto{\pgfqpoint{4.201485in}{0.476107in}}%
\pgfpathlineto{\pgfqpoint{4.204288in}{0.533512in}}%
\pgfpathlineto{\pgfqpoint{4.207091in}{0.486102in}}%
\pgfpathlineto{\pgfqpoint{4.209895in}{0.523276in}}%
\pgfpathlineto{\pgfqpoint{4.212698in}{0.503123in}}%
\pgfpathlineto{\pgfqpoint{4.215501in}{0.528847in}}%
\pgfpathlineto{\pgfqpoint{4.218304in}{0.477494in}}%
\pgfpathlineto{\pgfqpoint{4.221107in}{0.615489in}}%
\pgfpathlineto{\pgfqpoint{4.223911in}{0.532735in}}%
\pgfpathlineto{\pgfqpoint{4.226714in}{0.505826in}}%
\pgfpathlineto{\pgfqpoint{4.229517in}{0.488741in}}%
\pgfpathlineto{\pgfqpoint{4.235124in}{0.483052in}}%
\pgfpathlineto{\pgfqpoint{4.237927in}{0.490118in}}%
\pgfpathlineto{\pgfqpoint{4.240730in}{0.554789in}}%
\pgfpathlineto{\pgfqpoint{4.243533in}{0.520711in}}%
\pgfpathlineto{\pgfqpoint{4.246337in}{0.545991in}}%
\pgfpathlineto{\pgfqpoint{4.249140in}{0.520817in}}%
\pgfpathlineto{\pgfqpoint{4.251943in}{0.485696in}}%
\pgfpathlineto{\pgfqpoint{4.254746in}{0.495443in}}%
\pgfpathlineto{\pgfqpoint{4.257549in}{0.614857in}}%
\pgfpathlineto{\pgfqpoint{4.260353in}{0.625172in}}%
\pgfpathlineto{\pgfqpoint{4.263156in}{0.533073in}}%
\pgfpathlineto{\pgfqpoint{4.265959in}{0.515752in}}%
\pgfpathlineto{\pgfqpoint{4.268762in}{0.515575in}}%
\pgfpathlineto{\pgfqpoint{4.271566in}{0.502830in}}%
\pgfpathlineto{\pgfqpoint{4.274369in}{0.492946in}}%
\pgfpathlineto{\pgfqpoint{4.277172in}{0.477381in}}%
\pgfpathlineto{\pgfqpoint{4.279975in}{0.492929in}}%
\pgfpathlineto{\pgfqpoint{4.282779in}{0.488725in}}%
\pgfpathlineto{\pgfqpoint{4.285582in}{0.538131in}}%
\pgfpathlineto{\pgfqpoint{4.288385in}{0.549469in}}%
\pgfpathlineto{\pgfqpoint{4.291188in}{0.502892in}}%
\pgfpathlineto{\pgfqpoint{4.293991in}{0.477389in}}%
\pgfpathlineto{\pgfqpoint{4.296795in}{0.517047in}}%
\pgfpathlineto{\pgfqpoint{4.299598in}{0.484503in}}%
\pgfpathlineto{\pgfqpoint{4.302401in}{0.470325in}}%
\pgfpathlineto{\pgfqpoint{4.305204in}{0.491600in}}%
\pgfpathlineto{\pgfqpoint{4.308008in}{0.494434in}}%
\pgfpathlineto{\pgfqpoint{4.310811in}{0.599962in}}%
\pgfpathlineto{\pgfqpoint{4.313614in}{0.480127in}}%
\pgfpathlineto{\pgfqpoint{4.316417in}{0.471713in}}%
\pgfpathlineto{\pgfqpoint{4.319221in}{0.660984in}}%
\pgfpathlineto{\pgfqpoint{4.322024in}{0.468908in}}%
\pgfpathlineto{\pgfqpoint{4.324827in}{0.563840in}}%
\pgfpathlineto{\pgfqpoint{4.327630in}{0.503125in}}%
\pgfpathlineto{\pgfqpoint{4.330434in}{0.507237in}}%
\pgfpathlineto{\pgfqpoint{4.333237in}{0.499107in}}%
\pgfpathlineto{\pgfqpoint{4.336040in}{0.519864in}}%
\pgfpathlineto{\pgfqpoint{4.338843in}{0.474427in}}%
\pgfpathlineto{\pgfqpoint{4.341646in}{0.640146in}}%
\pgfpathlineto{\pgfqpoint{4.344450in}{0.481158in}}%
\pgfpathlineto{\pgfqpoint{4.347253in}{0.492037in}}%
\pgfpathlineto{\pgfqpoint{4.350056in}{0.539793in}}%
\pgfpathlineto{\pgfqpoint{4.352859in}{0.492170in}}%
\pgfpathlineto{\pgfqpoint{4.355663in}{0.479860in}}%
\pgfpathlineto{\pgfqpoint{4.358466in}{0.503072in}}%
\pgfpathlineto{\pgfqpoint{4.361269in}{0.489395in}}%
\pgfpathlineto{\pgfqpoint{4.364072in}{0.553369in}}%
\pgfpathlineto{\pgfqpoint{4.366876in}{0.535254in}}%
\pgfpathlineto{\pgfqpoint{4.369679in}{0.479716in}}%
\pgfpathlineto{\pgfqpoint{4.372482in}{0.632168in}}%
\pgfpathlineto{\pgfqpoint{4.375285in}{0.475757in}}%
\pgfpathlineto{\pgfqpoint{4.378088in}{0.572612in}}%
\pgfpathlineto{\pgfqpoint{4.380892in}{0.578487in}}%
\pgfpathlineto{\pgfqpoint{4.383695in}{0.542765in}}%
\pgfpathlineto{\pgfqpoint{4.386498in}{0.592800in}}%
\pgfpathlineto{\pgfqpoint{4.389301in}{0.499357in}}%
\pgfpathlineto{\pgfqpoint{4.392105in}{0.517929in}}%
\pgfpathlineto{\pgfqpoint{4.394908in}{0.475544in}}%
\pgfpathlineto{\pgfqpoint{4.397711in}{0.536391in}}%
\pgfpathlineto{\pgfqpoint{4.400514in}{0.490003in}}%
\pgfpathlineto{\pgfqpoint{4.403318in}{0.566010in}}%
\pgfpathlineto{\pgfqpoint{4.406121in}{0.521080in}}%
\pgfpathlineto{\pgfqpoint{4.408924in}{0.566912in}}%
\pgfpathlineto{\pgfqpoint{4.411727in}{0.506895in}}%
\pgfpathlineto{\pgfqpoint{4.414530in}{0.468908in}}%
\pgfpathlineto{\pgfqpoint{4.417334in}{0.666662in}}%
\pgfpathlineto{\pgfqpoint{4.420137in}{0.609239in}}%
\pgfpathlineto{\pgfqpoint{4.422940in}{0.676830in}}%
\pgfpathlineto{\pgfqpoint{4.425743in}{0.500588in}}%
\pgfpathlineto{\pgfqpoint{4.428547in}{0.620117in}}%
\pgfpathlineto{\pgfqpoint{4.431350in}{0.468908in}}%
\pgfpathlineto{\pgfqpoint{4.434153in}{0.550048in}}%
\pgfpathlineto{\pgfqpoint{4.439760in}{0.490194in}}%
\pgfpathlineto{\pgfqpoint{4.442563in}{0.546244in}}%
\pgfpathlineto{\pgfqpoint{4.445366in}{0.681088in}}%
\pgfpathlineto{\pgfqpoint{4.448169in}{0.515013in}}%
\pgfpathlineto{\pgfqpoint{4.450972in}{0.511211in}}%
\pgfpathlineto{\pgfqpoint{4.453776in}{0.641181in}}%
\pgfpathlineto{\pgfqpoint{4.456579in}{0.703578in}}%
\pgfpathlineto{\pgfqpoint{4.459382in}{0.553042in}}%
\pgfpathlineto{\pgfqpoint{4.462185in}{0.526663in}}%
\pgfpathlineto{\pgfqpoint{4.464989in}{0.556108in}}%
\pgfpathlineto{\pgfqpoint{4.467792in}{0.568231in}}%
\pgfpathlineto{\pgfqpoint{4.470595in}{0.557531in}}%
\pgfpathlineto{\pgfqpoint{4.473398in}{0.621828in}}%
\pgfpathlineto{\pgfqpoint{4.476202in}{0.509809in}}%
\pgfpathlineto{\pgfqpoint{4.479005in}{0.568613in}}%
\pgfpathlineto{\pgfqpoint{4.481808in}{0.498997in}}%
\pgfpathlineto{\pgfqpoint{4.484611in}{0.595313in}}%
\pgfpathlineto{\pgfqpoint{4.487414in}{0.481880in}}%
\pgfpathlineto{\pgfqpoint{4.490218in}{0.564580in}}%
\pgfpathlineto{\pgfqpoint{4.493021in}{0.553618in}}%
\pgfpathlineto{\pgfqpoint{4.495824in}{0.486828in}}%
\pgfpathlineto{\pgfqpoint{4.498627in}{0.655792in}}%
\pgfpathlineto{\pgfqpoint{4.501431in}{0.564433in}}%
\pgfpathlineto{\pgfqpoint{4.504234in}{0.515322in}}%
\pgfpathlineto{\pgfqpoint{4.509840in}{0.626928in}}%
\pgfpathlineto{\pgfqpoint{4.512644in}{0.520765in}}%
\pgfpathlineto{\pgfqpoint{4.515447in}{0.519503in}}%
\pgfpathlineto{\pgfqpoint{4.518250in}{0.630682in}}%
\pgfpathlineto{\pgfqpoint{4.521053in}{0.519885in}}%
\pgfpathlineto{\pgfqpoint{4.523856in}{0.553570in}}%
\pgfpathlineto{\pgfqpoint{4.526660in}{0.505085in}}%
\pgfpathlineto{\pgfqpoint{4.529463in}{0.542184in}}%
\pgfpathlineto{\pgfqpoint{4.532266in}{0.493699in}}%
\pgfpathlineto{\pgfqpoint{4.535069in}{0.507432in}}%
\pgfpathlineto{\pgfqpoint{4.537873in}{0.651974in}}%
\pgfpathlineto{\pgfqpoint{4.540676in}{0.544468in}}%
\pgfpathlineto{\pgfqpoint{4.543479in}{0.686695in}}%
\pgfpathlineto{\pgfqpoint{4.546282in}{0.561085in}}%
\pgfpathlineto{\pgfqpoint{4.549086in}{0.571339in}}%
\pgfpathlineto{\pgfqpoint{4.551889in}{0.608127in}}%
\pgfpathlineto{\pgfqpoint{4.554692in}{0.510593in}}%
\pgfpathlineto{\pgfqpoint{4.557495in}{0.554292in}}%
\pgfpathlineto{\pgfqpoint{4.560298in}{0.570598in}}%
\pgfpathlineto{\pgfqpoint{4.563102in}{0.575461in}}%
\pgfpathlineto{\pgfqpoint{4.565905in}{0.473771in}}%
\pgfpathlineto{\pgfqpoint{4.568708in}{0.484700in}}%
\pgfpathlineto{\pgfqpoint{4.571511in}{0.521001in}}%
\pgfpathlineto{\pgfqpoint{4.574315in}{0.470117in}}%
\pgfpathlineto{\pgfqpoint{4.577118in}{0.473745in}}%
\pgfpathlineto{\pgfqpoint{4.582724in}{0.513671in}}%
\pgfpathlineto{\pgfqpoint{4.585528in}{0.574420in}}%
\pgfpathlineto{\pgfqpoint{4.588331in}{0.488380in}}%
\pgfpathlineto{\pgfqpoint{4.591134in}{0.634397in}}%
\pgfpathlineto{\pgfqpoint{4.593937in}{0.484516in}}%
\pgfpathlineto{\pgfqpoint{4.596740in}{0.670302in}}%
\pgfpathlineto{\pgfqpoint{4.599544in}{0.475998in}}%
\pgfpathlineto{\pgfqpoint{4.602347in}{0.512696in}}%
\pgfpathlineto{\pgfqpoint{4.605150in}{0.498511in}}%
\pgfpathlineto{\pgfqpoint{4.607953in}{0.560314in}}%
\pgfpathlineto{\pgfqpoint{4.610757in}{0.525034in}}%
\pgfpathlineto{\pgfqpoint{4.613560in}{0.536944in}}%
\pgfpathlineto{\pgfqpoint{4.616363in}{0.520198in}}%
\pgfpathlineto{\pgfqpoint{4.621970in}{0.497468in}}%
\pgfpathlineto{\pgfqpoint{4.624773in}{0.769687in}}%
\pgfpathlineto{\pgfqpoint{4.627576in}{0.478654in}}%
\pgfpathlineto{\pgfqpoint{4.630379in}{0.581741in}}%
\pgfpathlineto{\pgfqpoint{4.633182in}{0.551964in}}%
\pgfpathlineto{\pgfqpoint{4.635986in}{0.546642in}}%
\pgfpathlineto{\pgfqpoint{4.638789in}{0.491538in}}%
\pgfpathlineto{\pgfqpoint{4.641592in}{0.486745in}}%
\pgfpathlineto{\pgfqpoint{4.644395in}{0.501033in}}%
\pgfpathlineto{\pgfqpoint{4.647199in}{0.576851in}}%
\pgfpathlineto{\pgfqpoint{4.650002in}{0.526643in}}%
\pgfpathlineto{\pgfqpoint{4.652805in}{0.500607in}}%
\pgfpathlineto{\pgfqpoint{4.655608in}{0.504034in}}%
\pgfpathlineto{\pgfqpoint{4.658412in}{0.599172in}}%
\pgfpathlineto{\pgfqpoint{4.661215in}{0.507147in}}%
\pgfpathlineto{\pgfqpoint{4.664018in}{0.531420in}}%
\pgfpathlineto{\pgfqpoint{4.666821in}{0.509389in}}%
\pgfpathlineto{\pgfqpoint{4.669624in}{0.556101in}}%
\pgfpathlineto{\pgfqpoint{4.672428in}{0.498040in}}%
\pgfpathlineto{\pgfqpoint{4.675231in}{0.558857in}}%
\pgfpathlineto{\pgfqpoint{4.678034in}{0.530891in}}%
\pgfpathlineto{\pgfqpoint{4.680837in}{0.471241in}}%
\pgfpathlineto{\pgfqpoint{4.683641in}{0.570004in}}%
\pgfpathlineto{\pgfqpoint{4.686444in}{0.590978in}}%
\pgfpathlineto{\pgfqpoint{4.689247in}{0.504381in}}%
\pgfpathlineto{\pgfqpoint{4.692050in}{0.479196in}}%
\pgfpathlineto{\pgfqpoint{4.694854in}{0.505450in}}%
\pgfpathlineto{\pgfqpoint{4.697657in}{0.475751in}}%
\pgfpathlineto{\pgfqpoint{4.700460in}{0.591514in}}%
\pgfpathlineto{\pgfqpoint{4.703263in}{0.479079in}}%
\pgfpathlineto{\pgfqpoint{4.706066in}{0.511795in}}%
\pgfpathlineto{\pgfqpoint{4.708870in}{0.530719in}}%
\pgfpathlineto{\pgfqpoint{4.711673in}{0.487951in}}%
\pgfpathlineto{\pgfqpoint{4.714476in}{0.510254in}}%
\pgfpathlineto{\pgfqpoint{4.720083in}{0.498031in}}%
\pgfpathlineto{\pgfqpoint{4.722886in}{0.521729in}}%
\pgfpathlineto{\pgfqpoint{4.725689in}{0.532938in}}%
\pgfpathlineto{\pgfqpoint{4.728492in}{0.662000in}}%
\pgfpathlineto{\pgfqpoint{4.731296in}{0.577681in}}%
\pgfpathlineto{\pgfqpoint{4.734099in}{0.603884in}}%
\pgfpathlineto{\pgfqpoint{4.736902in}{0.483725in}}%
\pgfpathlineto{\pgfqpoint{4.739705in}{0.636356in}}%
\pgfpathlineto{\pgfqpoint{4.742508in}{0.538780in}}%
\pgfpathlineto{\pgfqpoint{4.745312in}{0.595961in}}%
\pgfpathlineto{\pgfqpoint{4.748115in}{0.514686in}}%
\pgfpathlineto{\pgfqpoint{4.750918in}{0.576544in}}%
\pgfpathlineto{\pgfqpoint{4.753721in}{0.857209in}}%
\pgfpathlineto{\pgfqpoint{4.759328in}{0.675297in}}%
\pgfpathlineto{\pgfqpoint{4.762131in}{0.843242in}}%
\pgfpathlineto{\pgfqpoint{4.764934in}{0.599947in}}%
\pgfpathlineto{\pgfqpoint{4.767738in}{0.600858in}}%
\pgfpathlineto{\pgfqpoint{4.770541in}{0.654680in}}%
\pgfpathlineto{\pgfqpoint{4.773344in}{0.601982in}}%
\pgfpathlineto{\pgfqpoint{4.776147in}{0.494517in}}%
\pgfpathlineto{\pgfqpoint{4.778951in}{0.588702in}}%
\pgfpathlineto{\pgfqpoint{4.781754in}{0.485477in}}%
\pgfpathlineto{\pgfqpoint{4.784557in}{0.512091in}}%
\pgfpathlineto{\pgfqpoint{4.787360in}{0.567235in}}%
\pgfpathlineto{\pgfqpoint{4.790163in}{0.497552in}}%
\pgfpathlineto{\pgfqpoint{4.792967in}{0.624295in}}%
\pgfpathlineto{\pgfqpoint{4.795770in}{0.471142in}}%
\pgfpathlineto{\pgfqpoint{4.798573in}{0.501261in}}%
\pgfpathlineto{\pgfqpoint{4.801376in}{0.500066in}}%
\pgfpathlineto{\pgfqpoint{4.804180in}{0.501096in}}%
\pgfpathlineto{\pgfqpoint{4.806983in}{0.581462in}}%
\pgfpathlineto{\pgfqpoint{4.809786in}{0.505104in}}%
\pgfpathlineto{\pgfqpoint{4.812589in}{0.505104in}}%
\pgfpathlineto{\pgfqpoint{4.812589in}{0.505104in}}%
\pgfusepath{stroke}%
\end{pgfscope}%
\begin{pgfscope}%
\pgfsetrectcap%
\pgfsetmiterjoin%
\pgfsetlinewidth{0.803000pt}%
\definecolor{currentstroke}{rgb}{1.000000,1.000000,1.000000}%
\pgfsetstrokecolor{currentstroke}%
\pgfsetdash{}{0pt}%
\pgfpathmoveto{\pgfqpoint{0.373953in}{0.331635in}}%
\pgfpathlineto{\pgfqpoint{0.373953in}{3.351635in}}%
\pgfusepath{stroke}%
\end{pgfscope}%
\begin{pgfscope}%
\pgfsetrectcap%
\pgfsetmiterjoin%
\pgfsetlinewidth{0.803000pt}%
\definecolor{currentstroke}{rgb}{1.000000,1.000000,1.000000}%
\pgfsetstrokecolor{currentstroke}%
\pgfsetdash{}{0pt}%
\pgfpathmoveto{\pgfqpoint{5.023953in}{0.331635in}}%
\pgfpathlineto{\pgfqpoint{5.023953in}{3.351635in}}%
\pgfusepath{stroke}%
\end{pgfscope}%
\begin{pgfscope}%
\pgfsetrectcap%
\pgfsetmiterjoin%
\pgfsetlinewidth{0.803000pt}%
\definecolor{currentstroke}{rgb}{1.000000,1.000000,1.000000}%
\pgfsetstrokecolor{currentstroke}%
\pgfsetdash{}{0pt}%
\pgfpathmoveto{\pgfqpoint{0.373953in}{0.331635in}}%
\pgfpathlineto{\pgfqpoint{5.023953in}{0.331635in}}%
\pgfusepath{stroke}%
\end{pgfscope}%
\begin{pgfscope}%
\pgfsetrectcap%
\pgfsetmiterjoin%
\pgfsetlinewidth{0.803000pt}%
\definecolor{currentstroke}{rgb}{1.000000,1.000000,1.000000}%
\pgfsetstrokecolor{currentstroke}%
\pgfsetdash{}{0pt}%
\pgfpathmoveto{\pgfqpoint{0.373953in}{3.351635in}}%
\pgfpathlineto{\pgfqpoint{5.023953in}{3.351635in}}%
\pgfusepath{stroke}%
\end{pgfscope}%
\end{pgfpicture}%
\makeatother%
\endgroup%

    %% Creator: Matplotlib, PGF backend
%%
%% To include the figure in your LaTeX document, write
%%   \input{<filename>.pgf}
%%
%% Make sure the required packages are loaded in your preamble
%%   \usepackage{pgf}
%%
%% Figures using additional raster images can only be included by \input if
%% they are in the same directory as the main LaTeX file. For loading figures
%% from other directories you can use the `import` package
%%   \usepackage{import}
%% and then include the figures with
%%   \import{<path to file>}{<filename>.pgf}
%%
%% Matplotlib used the following preamble
%%   \usepackage{fontspec}
%%   \setmainfont{DejaVuSerif.ttf}[Path=/opt/tljh/user/lib/python3.6/site-packages/matplotlib/mpl-data/fonts/ttf/]
%%   \setsansfont{DejaVuSans.ttf}[Path=/opt/tljh/user/lib/python3.6/site-packages/matplotlib/mpl-data/fonts/ttf/]
%%   \setmonofont{DejaVuSansMono.ttf}[Path=/opt/tljh/user/lib/python3.6/site-packages/matplotlib/mpl-data/fonts/ttf/]
%%
\begingroup%
\makeatletter%
\begin{pgfpicture}%
\pgfpathrectangle{\pgfpointorigin}{\pgfqpoint{5.123953in}{3.503241in}}%
\pgfusepath{use as bounding box, clip}%
\begin{pgfscope}%
\pgfsetbuttcap%
\pgfsetmiterjoin%
\definecolor{currentfill}{rgb}{1.000000,1.000000,1.000000}%
\pgfsetfillcolor{currentfill}%
\pgfsetlinewidth{0.000000pt}%
\definecolor{currentstroke}{rgb}{1.000000,1.000000,1.000000}%
\pgfsetstrokecolor{currentstroke}%
\pgfsetdash{}{0pt}%
\pgfpathmoveto{\pgfqpoint{0.000000in}{0.000000in}}%
\pgfpathlineto{\pgfqpoint{5.123953in}{0.000000in}}%
\pgfpathlineto{\pgfqpoint{5.123953in}{3.503241in}}%
\pgfpathlineto{\pgfqpoint{0.000000in}{3.503241in}}%
\pgfpathclose%
\pgfusepath{fill}%
\end{pgfscope}%
\begin{pgfscope}%
\pgfsetbuttcap%
\pgfsetmiterjoin%
\definecolor{currentfill}{rgb}{0.917647,0.917647,0.949020}%
\pgfsetfillcolor{currentfill}%
\pgfsetlinewidth{0.000000pt}%
\definecolor{currentstroke}{rgb}{0.000000,0.000000,0.000000}%
\pgfsetstrokecolor{currentstroke}%
\pgfsetstrokeopacity{0.000000}%
\pgfsetdash{}{0pt}%
\pgfpathmoveto{\pgfqpoint{0.373953in}{0.331635in}}%
\pgfpathlineto{\pgfqpoint{5.023953in}{0.331635in}}%
\pgfpathlineto{\pgfqpoint{5.023953in}{3.351635in}}%
\pgfpathlineto{\pgfqpoint{0.373953in}{3.351635in}}%
\pgfpathclose%
\pgfusepath{fill}%
\end{pgfscope}%
\begin{pgfscope}%
\pgfpathrectangle{\pgfqpoint{0.373953in}{0.331635in}}{\pgfqpoint{4.650000in}{3.020000in}}%
\pgfusepath{clip}%
\pgfsetroundcap%
\pgfsetroundjoin%
\pgfsetlinewidth{0.803000pt}%
\definecolor{currentstroke}{rgb}{1.000000,1.000000,1.000000}%
\pgfsetstrokecolor{currentstroke}%
\pgfsetdash{}{0pt}%
\pgfpathmoveto{\pgfqpoint{0.585317in}{0.331635in}}%
\pgfpathlineto{\pgfqpoint{0.585317in}{3.351635in}}%
\pgfusepath{stroke}%
\end{pgfscope}%
\begin{pgfscope}%
\definecolor{textcolor}{rgb}{0.150000,0.150000,0.150000}%
\pgfsetstrokecolor{textcolor}%
\pgfsetfillcolor{textcolor}%
\pgftext[x=0.585317in,y=0.234413in,,top]{\color{textcolor}\rmfamily\fontsize{10.000000}{12.000000}\selectfont 0}%
\end{pgfscope}%
\begin{pgfscope}%
\pgfpathrectangle{\pgfqpoint{0.373953in}{0.331635in}}{\pgfqpoint{4.650000in}{3.020000in}}%
\pgfusepath{clip}%
\pgfsetroundcap%
\pgfsetroundjoin%
\pgfsetlinewidth{0.803000pt}%
\definecolor{currentstroke}{rgb}{1.000000,1.000000,1.000000}%
\pgfsetstrokecolor{currentstroke}%
\pgfsetdash{}{0pt}%
\pgfpathmoveto{\pgfqpoint{1.148953in}{0.331635in}}%
\pgfpathlineto{\pgfqpoint{1.148953in}{3.351635in}}%
\pgfusepath{stroke}%
\end{pgfscope}%
\begin{pgfscope}%
\definecolor{textcolor}{rgb}{0.150000,0.150000,0.150000}%
\pgfsetstrokecolor{textcolor}%
\pgfsetfillcolor{textcolor}%
\pgftext[x=1.148953in,y=0.234413in,,top]{\color{textcolor}\rmfamily\fontsize{10.000000}{12.000000}\selectfont 200}%
\end{pgfscope}%
\begin{pgfscope}%
\pgfpathrectangle{\pgfqpoint{0.373953in}{0.331635in}}{\pgfqpoint{4.650000in}{3.020000in}}%
\pgfusepath{clip}%
\pgfsetroundcap%
\pgfsetroundjoin%
\pgfsetlinewidth{0.803000pt}%
\definecolor{currentstroke}{rgb}{1.000000,1.000000,1.000000}%
\pgfsetstrokecolor{currentstroke}%
\pgfsetdash{}{0pt}%
\pgfpathmoveto{\pgfqpoint{1.712589in}{0.331635in}}%
\pgfpathlineto{\pgfqpoint{1.712589in}{3.351635in}}%
\pgfusepath{stroke}%
\end{pgfscope}%
\begin{pgfscope}%
\definecolor{textcolor}{rgb}{0.150000,0.150000,0.150000}%
\pgfsetstrokecolor{textcolor}%
\pgfsetfillcolor{textcolor}%
\pgftext[x=1.712589in,y=0.234413in,,top]{\color{textcolor}\rmfamily\fontsize{10.000000}{12.000000}\selectfont 400}%
\end{pgfscope}%
\begin{pgfscope}%
\pgfpathrectangle{\pgfqpoint{0.373953in}{0.331635in}}{\pgfqpoint{4.650000in}{3.020000in}}%
\pgfusepath{clip}%
\pgfsetroundcap%
\pgfsetroundjoin%
\pgfsetlinewidth{0.803000pt}%
\definecolor{currentstroke}{rgb}{1.000000,1.000000,1.000000}%
\pgfsetstrokecolor{currentstroke}%
\pgfsetdash{}{0pt}%
\pgfpathmoveto{\pgfqpoint{2.276226in}{0.331635in}}%
\pgfpathlineto{\pgfqpoint{2.276226in}{3.351635in}}%
\pgfusepath{stroke}%
\end{pgfscope}%
\begin{pgfscope}%
\definecolor{textcolor}{rgb}{0.150000,0.150000,0.150000}%
\pgfsetstrokecolor{textcolor}%
\pgfsetfillcolor{textcolor}%
\pgftext[x=2.276226in,y=0.234413in,,top]{\color{textcolor}\rmfamily\fontsize{10.000000}{12.000000}\selectfont 600}%
\end{pgfscope}%
\begin{pgfscope}%
\pgfpathrectangle{\pgfqpoint{0.373953in}{0.331635in}}{\pgfqpoint{4.650000in}{3.020000in}}%
\pgfusepath{clip}%
\pgfsetroundcap%
\pgfsetroundjoin%
\pgfsetlinewidth{0.803000pt}%
\definecolor{currentstroke}{rgb}{1.000000,1.000000,1.000000}%
\pgfsetstrokecolor{currentstroke}%
\pgfsetdash{}{0pt}%
\pgfpathmoveto{\pgfqpoint{2.839862in}{0.331635in}}%
\pgfpathlineto{\pgfqpoint{2.839862in}{3.351635in}}%
\pgfusepath{stroke}%
\end{pgfscope}%
\begin{pgfscope}%
\definecolor{textcolor}{rgb}{0.150000,0.150000,0.150000}%
\pgfsetstrokecolor{textcolor}%
\pgfsetfillcolor{textcolor}%
\pgftext[x=2.839862in,y=0.234413in,,top]{\color{textcolor}\rmfamily\fontsize{10.000000}{12.000000}\selectfont 800}%
\end{pgfscope}%
\begin{pgfscope}%
\pgfpathrectangle{\pgfqpoint{0.373953in}{0.331635in}}{\pgfqpoint{4.650000in}{3.020000in}}%
\pgfusepath{clip}%
\pgfsetroundcap%
\pgfsetroundjoin%
\pgfsetlinewidth{0.803000pt}%
\definecolor{currentstroke}{rgb}{1.000000,1.000000,1.000000}%
\pgfsetstrokecolor{currentstroke}%
\pgfsetdash{}{0pt}%
\pgfpathmoveto{\pgfqpoint{3.403498in}{0.331635in}}%
\pgfpathlineto{\pgfqpoint{3.403498in}{3.351635in}}%
\pgfusepath{stroke}%
\end{pgfscope}%
\begin{pgfscope}%
\definecolor{textcolor}{rgb}{0.150000,0.150000,0.150000}%
\pgfsetstrokecolor{textcolor}%
\pgfsetfillcolor{textcolor}%
\pgftext[x=3.403498in,y=0.234413in,,top]{\color{textcolor}\rmfamily\fontsize{10.000000}{12.000000}\selectfont 1000}%
\end{pgfscope}%
\begin{pgfscope}%
\pgfpathrectangle{\pgfqpoint{0.373953in}{0.331635in}}{\pgfqpoint{4.650000in}{3.020000in}}%
\pgfusepath{clip}%
\pgfsetroundcap%
\pgfsetroundjoin%
\pgfsetlinewidth{0.803000pt}%
\definecolor{currentstroke}{rgb}{1.000000,1.000000,1.000000}%
\pgfsetstrokecolor{currentstroke}%
\pgfsetdash{}{0pt}%
\pgfpathmoveto{\pgfqpoint{3.967135in}{0.331635in}}%
\pgfpathlineto{\pgfqpoint{3.967135in}{3.351635in}}%
\pgfusepath{stroke}%
\end{pgfscope}%
\begin{pgfscope}%
\definecolor{textcolor}{rgb}{0.150000,0.150000,0.150000}%
\pgfsetstrokecolor{textcolor}%
\pgfsetfillcolor{textcolor}%
\pgftext[x=3.967135in,y=0.234413in,,top]{\color{textcolor}\rmfamily\fontsize{10.000000}{12.000000}\selectfont 1200}%
\end{pgfscope}%
\begin{pgfscope}%
\pgfpathrectangle{\pgfqpoint{0.373953in}{0.331635in}}{\pgfqpoint{4.650000in}{3.020000in}}%
\pgfusepath{clip}%
\pgfsetroundcap%
\pgfsetroundjoin%
\pgfsetlinewidth{0.803000pt}%
\definecolor{currentstroke}{rgb}{1.000000,1.000000,1.000000}%
\pgfsetstrokecolor{currentstroke}%
\pgfsetdash{}{0pt}%
\pgfpathmoveto{\pgfqpoint{4.530771in}{0.331635in}}%
\pgfpathlineto{\pgfqpoint{4.530771in}{3.351635in}}%
\pgfusepath{stroke}%
\end{pgfscope}%
\begin{pgfscope}%
\definecolor{textcolor}{rgb}{0.150000,0.150000,0.150000}%
\pgfsetstrokecolor{textcolor}%
\pgfsetfillcolor{textcolor}%
\pgftext[x=4.530771in,y=0.234413in,,top]{\color{textcolor}\rmfamily\fontsize{10.000000}{12.000000}\selectfont 1400}%
\end{pgfscope}%
\begin{pgfscope}%
\pgfpathrectangle{\pgfqpoint{0.373953in}{0.331635in}}{\pgfqpoint{4.650000in}{3.020000in}}%
\pgfusepath{clip}%
\pgfsetroundcap%
\pgfsetroundjoin%
\pgfsetlinewidth{0.803000pt}%
\definecolor{currentstroke}{rgb}{1.000000,1.000000,1.000000}%
\pgfsetstrokecolor{currentstroke}%
\pgfsetdash{}{0pt}%
\pgfpathmoveto{\pgfqpoint{0.373953in}{0.468908in}}%
\pgfpathlineto{\pgfqpoint{5.023953in}{0.468908in}}%
\pgfusepath{stroke}%
\end{pgfscope}%
\begin{pgfscope}%
\definecolor{textcolor}{rgb}{0.150000,0.150000,0.150000}%
\pgfsetstrokecolor{textcolor}%
\pgfsetfillcolor{textcolor}%
\pgftext[x=0.188365in,y=0.416146in,left,base]{\color{textcolor}\rmfamily\fontsize{10.000000}{12.000000}\selectfont 0}%
\end{pgfscope}%
\begin{pgfscope}%
\pgfpathrectangle{\pgfqpoint{0.373953in}{0.331635in}}{\pgfqpoint{4.650000in}{3.020000in}}%
\pgfusepath{clip}%
\pgfsetroundcap%
\pgfsetroundjoin%
\pgfsetlinewidth{0.803000pt}%
\definecolor{currentstroke}{rgb}{1.000000,1.000000,1.000000}%
\pgfsetstrokecolor{currentstroke}%
\pgfsetdash{}{0pt}%
\pgfpathmoveto{\pgfqpoint{0.373953in}{1.045222in}}%
\pgfpathlineto{\pgfqpoint{5.023953in}{1.045222in}}%
\pgfusepath{stroke}%
\end{pgfscope}%
\begin{pgfscope}%
\definecolor{textcolor}{rgb}{0.150000,0.150000,0.150000}%
\pgfsetstrokecolor{textcolor}%
\pgfsetfillcolor{textcolor}%
\pgftext[x=0.188365in,y=0.992461in,left,base]{\color{textcolor}\rmfamily\fontsize{10.000000}{12.000000}\selectfont 2}%
\end{pgfscope}%
\begin{pgfscope}%
\pgfpathrectangle{\pgfqpoint{0.373953in}{0.331635in}}{\pgfqpoint{4.650000in}{3.020000in}}%
\pgfusepath{clip}%
\pgfsetroundcap%
\pgfsetroundjoin%
\pgfsetlinewidth{0.803000pt}%
\definecolor{currentstroke}{rgb}{1.000000,1.000000,1.000000}%
\pgfsetstrokecolor{currentstroke}%
\pgfsetdash{}{0pt}%
\pgfpathmoveto{\pgfqpoint{0.373953in}{1.621537in}}%
\pgfpathlineto{\pgfqpoint{5.023953in}{1.621537in}}%
\pgfusepath{stroke}%
\end{pgfscope}%
\begin{pgfscope}%
\definecolor{textcolor}{rgb}{0.150000,0.150000,0.150000}%
\pgfsetstrokecolor{textcolor}%
\pgfsetfillcolor{textcolor}%
\pgftext[x=0.188365in,y=1.568775in,left,base]{\color{textcolor}\rmfamily\fontsize{10.000000}{12.000000}\selectfont 4}%
\end{pgfscope}%
\begin{pgfscope}%
\pgfpathrectangle{\pgfqpoint{0.373953in}{0.331635in}}{\pgfqpoint{4.650000in}{3.020000in}}%
\pgfusepath{clip}%
\pgfsetroundcap%
\pgfsetroundjoin%
\pgfsetlinewidth{0.803000pt}%
\definecolor{currentstroke}{rgb}{1.000000,1.000000,1.000000}%
\pgfsetstrokecolor{currentstroke}%
\pgfsetdash{}{0pt}%
\pgfpathmoveto{\pgfqpoint{0.373953in}{2.197851in}}%
\pgfpathlineto{\pgfqpoint{5.023953in}{2.197851in}}%
\pgfusepath{stroke}%
\end{pgfscope}%
\begin{pgfscope}%
\definecolor{textcolor}{rgb}{0.150000,0.150000,0.150000}%
\pgfsetstrokecolor{textcolor}%
\pgfsetfillcolor{textcolor}%
\pgftext[x=0.188365in,y=2.145090in,left,base]{\color{textcolor}\rmfamily\fontsize{10.000000}{12.000000}\selectfont 6}%
\end{pgfscope}%
\begin{pgfscope}%
\pgfpathrectangle{\pgfqpoint{0.373953in}{0.331635in}}{\pgfqpoint{4.650000in}{3.020000in}}%
\pgfusepath{clip}%
\pgfsetroundcap%
\pgfsetroundjoin%
\pgfsetlinewidth{0.803000pt}%
\definecolor{currentstroke}{rgb}{1.000000,1.000000,1.000000}%
\pgfsetstrokecolor{currentstroke}%
\pgfsetdash{}{0pt}%
\pgfpathmoveto{\pgfqpoint{0.373953in}{2.774166in}}%
\pgfpathlineto{\pgfqpoint{5.023953in}{2.774166in}}%
\pgfusepath{stroke}%
\end{pgfscope}%
\begin{pgfscope}%
\definecolor{textcolor}{rgb}{0.150000,0.150000,0.150000}%
\pgfsetstrokecolor{textcolor}%
\pgfsetfillcolor{textcolor}%
\pgftext[x=0.188365in,y=2.721404in,left,base]{\color{textcolor}\rmfamily\fontsize{10.000000}{12.000000}\selectfont 8}%
\end{pgfscope}%
\begin{pgfscope}%
\pgfpathrectangle{\pgfqpoint{0.373953in}{0.331635in}}{\pgfqpoint{4.650000in}{3.020000in}}%
\pgfusepath{clip}%
\pgfsetroundcap%
\pgfsetroundjoin%
\pgfsetlinewidth{0.803000pt}%
\definecolor{currentstroke}{rgb}{1.000000,1.000000,1.000000}%
\pgfsetstrokecolor{currentstroke}%
\pgfsetdash{}{0pt}%
\pgfpathmoveto{\pgfqpoint{0.373953in}{3.350480in}}%
\pgfpathlineto{\pgfqpoint{5.023953in}{3.350480in}}%
\pgfusepath{stroke}%
\end{pgfscope}%
\begin{pgfscope}%
\definecolor{textcolor}{rgb}{0.150000,0.150000,0.150000}%
\pgfsetstrokecolor{textcolor}%
\pgfsetfillcolor{textcolor}%
\pgftext[x=0.100000in,y=3.297718in,left,base]{\color{textcolor}\rmfamily\fontsize{10.000000}{12.000000}\selectfont 10}%
\end{pgfscope}%
\begin{pgfscope}%
\pgfpathrectangle{\pgfqpoint{0.373953in}{0.331635in}}{\pgfqpoint{4.650000in}{3.020000in}}%
\pgfusepath{clip}%
\pgfsetroundcap%
\pgfsetroundjoin%
\pgfsetlinewidth{1.505625pt}%
\definecolor{currentstroke}{rgb}{0.121569,0.466667,0.705882}%
\pgfsetstrokecolor{currentstroke}%
\pgfsetdash{}{0pt}%
\pgfpathmoveto{\pgfqpoint{0.585317in}{0.845918in}}%
\pgfpathlineto{\pgfqpoint{0.588135in}{1.122351in}}%
\pgfpathlineto{\pgfqpoint{0.590953in}{1.071145in}}%
\pgfpathlineto{\pgfqpoint{0.593771in}{1.430501in}}%
\pgfpathlineto{\pgfqpoint{0.596589in}{0.897996in}}%
\pgfpathlineto{\pgfqpoint{0.602226in}{0.854961in}}%
\pgfpathlineto{\pgfqpoint{0.605044in}{0.748284in}}%
\pgfpathlineto{\pgfqpoint{0.607862in}{0.856282in}}%
\pgfpathlineto{\pgfqpoint{0.610680in}{0.768382in}}%
\pgfpathlineto{\pgfqpoint{0.613498in}{0.727284in}}%
\pgfpathlineto{\pgfqpoint{0.616317in}{0.780064in}}%
\pgfpathlineto{\pgfqpoint{0.619135in}{0.740216in}}%
\pgfpathlineto{\pgfqpoint{0.624771in}{0.692077in}}%
\pgfpathlineto{\pgfqpoint{0.627589in}{0.660635in}}%
\pgfpathlineto{\pgfqpoint{0.630407in}{0.662031in}}%
\pgfpathlineto{\pgfqpoint{0.636044in}{0.609776in}}%
\pgfpathlineto{\pgfqpoint{0.638862in}{0.621696in}}%
\pgfpathlineto{\pgfqpoint{0.641680in}{0.623122in}}%
\pgfpathlineto{\pgfqpoint{0.644498in}{0.684791in}}%
\pgfpathlineto{\pgfqpoint{0.647317in}{0.693364in}}%
\pgfpathlineto{\pgfqpoint{0.650135in}{0.753158in}}%
\pgfpathlineto{\pgfqpoint{0.652953in}{0.735565in}}%
\pgfpathlineto{\pgfqpoint{0.655771in}{0.711376in}}%
\pgfpathlineto{\pgfqpoint{0.658589in}{0.692754in}}%
\pgfpathlineto{\pgfqpoint{0.661407in}{0.704455in}}%
\pgfpathlineto{\pgfqpoint{0.664226in}{0.734348in}}%
\pgfpathlineto{\pgfqpoint{0.667044in}{0.716325in}}%
\pgfpathlineto{\pgfqpoint{0.669862in}{0.691895in}}%
\pgfpathlineto{\pgfqpoint{0.672680in}{0.727030in}}%
\pgfpathlineto{\pgfqpoint{0.675498in}{0.709135in}}%
\pgfpathlineto{\pgfqpoint{0.678317in}{0.710390in}}%
\pgfpathlineto{\pgfqpoint{0.681135in}{0.690980in}}%
\pgfpathlineto{\pgfqpoint{0.683953in}{0.679963in}}%
\pgfpathlineto{\pgfqpoint{0.686771in}{0.664802in}}%
\pgfpathlineto{\pgfqpoint{0.689589in}{0.717169in}}%
\pgfpathlineto{\pgfqpoint{0.692407in}{0.699861in}}%
\pgfpathlineto{\pgfqpoint{0.695226in}{0.695162in}}%
\pgfpathlineto{\pgfqpoint{0.698044in}{0.669579in}}%
\pgfpathlineto{\pgfqpoint{0.712135in}{0.612760in}}%
\pgfpathlineto{\pgfqpoint{0.714953in}{0.616255in}}%
\pgfpathlineto{\pgfqpoint{0.717771in}{0.606853in}}%
\pgfpathlineto{\pgfqpoint{0.720589in}{0.635131in}}%
\pgfpathlineto{\pgfqpoint{0.723407in}{0.646259in}}%
\pgfpathlineto{\pgfqpoint{0.726226in}{0.632262in}}%
\pgfpathlineto{\pgfqpoint{0.729044in}{0.629606in}}%
\pgfpathlineto{\pgfqpoint{0.731862in}{0.649414in}}%
\pgfpathlineto{\pgfqpoint{0.734680in}{0.649152in}}%
\pgfpathlineto{\pgfqpoint{0.737498in}{0.633205in}}%
\pgfpathlineto{\pgfqpoint{0.740317in}{0.656924in}}%
\pgfpathlineto{\pgfqpoint{0.745953in}{0.683815in}}%
\pgfpathlineto{\pgfqpoint{0.748771in}{0.744588in}}%
\pgfpathlineto{\pgfqpoint{0.751589in}{0.821738in}}%
\pgfpathlineto{\pgfqpoint{0.757226in}{0.753784in}}%
\pgfpathlineto{\pgfqpoint{0.760044in}{0.916554in}}%
\pgfpathlineto{\pgfqpoint{0.762862in}{0.903643in}}%
\pgfpathlineto{\pgfqpoint{0.765680in}{0.830941in}}%
\pgfpathlineto{\pgfqpoint{0.771317in}{0.785972in}}%
\pgfpathlineto{\pgfqpoint{0.774135in}{0.797961in}}%
\pgfpathlineto{\pgfqpoint{0.779771in}{0.762471in}}%
\pgfpathlineto{\pgfqpoint{0.782589in}{0.744303in}}%
\pgfpathlineto{\pgfqpoint{0.785407in}{0.748599in}}%
\pgfpathlineto{\pgfqpoint{0.788226in}{0.736760in}}%
\pgfpathlineto{\pgfqpoint{0.791044in}{0.784940in}}%
\pgfpathlineto{\pgfqpoint{0.793862in}{0.995391in}}%
\pgfpathlineto{\pgfqpoint{0.796680in}{0.934038in}}%
\pgfpathlineto{\pgfqpoint{0.799498in}{0.995103in}}%
\pgfpathlineto{\pgfqpoint{0.802317in}{0.945841in}}%
\pgfpathlineto{\pgfqpoint{0.805135in}{0.877504in}}%
\pgfpathlineto{\pgfqpoint{0.807953in}{0.857823in}}%
\pgfpathlineto{\pgfqpoint{0.810771in}{1.072852in}}%
\pgfpathlineto{\pgfqpoint{0.813589in}{1.005517in}}%
\pgfpathlineto{\pgfqpoint{0.816407in}{1.045847in}}%
\pgfpathlineto{\pgfqpoint{0.819226in}{1.051642in}}%
\pgfpathlineto{\pgfqpoint{0.824862in}{0.910339in}}%
\pgfpathlineto{\pgfqpoint{0.827680in}{0.868175in}}%
\pgfpathlineto{\pgfqpoint{0.830498in}{1.057282in}}%
\pgfpathlineto{\pgfqpoint{0.836135in}{0.922197in}}%
\pgfpathlineto{\pgfqpoint{0.838953in}{0.904481in}}%
\pgfpathlineto{\pgfqpoint{0.841771in}{0.861297in}}%
\pgfpathlineto{\pgfqpoint{0.844589in}{0.865494in}}%
\pgfpathlineto{\pgfqpoint{0.847407in}{1.127426in}}%
\pgfpathlineto{\pgfqpoint{0.850226in}{1.049642in}}%
\pgfpathlineto{\pgfqpoint{0.853044in}{1.027715in}}%
\pgfpathlineto{\pgfqpoint{0.855862in}{1.065128in}}%
\pgfpathlineto{\pgfqpoint{0.858680in}{1.016875in}}%
\pgfpathlineto{\pgfqpoint{0.861498in}{1.010771in}}%
\pgfpathlineto{\pgfqpoint{0.864317in}{1.057761in}}%
\pgfpathlineto{\pgfqpoint{0.867135in}{1.054110in}}%
\pgfpathlineto{\pgfqpoint{0.869953in}{0.996840in}}%
\pgfpathlineto{\pgfqpoint{0.872771in}{0.983083in}}%
\pgfpathlineto{\pgfqpoint{0.875589in}{0.957557in}}%
\pgfpathlineto{\pgfqpoint{0.881226in}{0.871287in}}%
\pgfpathlineto{\pgfqpoint{0.884044in}{0.834229in}}%
\pgfpathlineto{\pgfqpoint{0.886862in}{1.247082in}}%
\pgfpathlineto{\pgfqpoint{0.889680in}{1.202571in}}%
\pgfpathlineto{\pgfqpoint{0.892498in}{1.517743in}}%
\pgfpathlineto{\pgfqpoint{0.898135in}{1.383637in}}%
\pgfpathlineto{\pgfqpoint{0.900953in}{1.399439in}}%
\pgfpathlineto{\pgfqpoint{0.903771in}{1.407240in}}%
\pgfpathlineto{\pgfqpoint{0.909407in}{1.313265in}}%
\pgfpathlineto{\pgfqpoint{0.912226in}{1.311404in}}%
\pgfpathlineto{\pgfqpoint{0.915044in}{1.334609in}}%
\pgfpathlineto{\pgfqpoint{0.917862in}{1.277655in}}%
\pgfpathlineto{\pgfqpoint{0.920680in}{1.408333in}}%
\pgfpathlineto{\pgfqpoint{0.923498in}{1.355986in}}%
\pgfpathlineto{\pgfqpoint{0.926317in}{1.508325in}}%
\pgfpathlineto{\pgfqpoint{0.929135in}{1.474284in}}%
\pgfpathlineto{\pgfqpoint{0.931953in}{1.427401in}}%
\pgfpathlineto{\pgfqpoint{0.934771in}{1.370152in}}%
\pgfpathlineto{\pgfqpoint{0.937589in}{1.380887in}}%
\pgfpathlineto{\pgfqpoint{0.940407in}{1.354155in}}%
\pgfpathlineto{\pgfqpoint{0.943226in}{1.423862in}}%
\pgfpathlineto{\pgfqpoint{0.946044in}{1.419293in}}%
\pgfpathlineto{\pgfqpoint{0.948862in}{1.383482in}}%
\pgfpathlineto{\pgfqpoint{0.951680in}{1.321093in}}%
\pgfpathlineto{\pgfqpoint{0.954498in}{1.285227in}}%
\pgfpathlineto{\pgfqpoint{0.957317in}{1.274153in}}%
\pgfpathlineto{\pgfqpoint{0.960135in}{1.244806in}}%
\pgfpathlineto{\pgfqpoint{0.962953in}{1.176810in}}%
\pgfpathlineto{\pgfqpoint{0.965771in}{1.135681in}}%
\pgfpathlineto{\pgfqpoint{0.968589in}{1.078415in}}%
\pgfpathlineto{\pgfqpoint{0.971407in}{1.088280in}}%
\pgfpathlineto{\pgfqpoint{0.977044in}{0.986331in}}%
\pgfpathlineto{\pgfqpoint{0.982680in}{0.891156in}}%
\pgfpathlineto{\pgfqpoint{0.985498in}{0.854089in}}%
\pgfpathlineto{\pgfqpoint{0.988317in}{0.837051in}}%
\pgfpathlineto{\pgfqpoint{0.991135in}{0.836759in}}%
\pgfpathlineto{\pgfqpoint{0.993953in}{0.837063in}}%
\pgfpathlineto{\pgfqpoint{0.996771in}{0.818790in}}%
\pgfpathlineto{\pgfqpoint{0.999589in}{0.839285in}}%
\pgfpathlineto{\pgfqpoint{1.002407in}{0.814917in}}%
\pgfpathlineto{\pgfqpoint{1.005226in}{0.804748in}}%
\pgfpathlineto{\pgfqpoint{1.008044in}{0.866611in}}%
\pgfpathlineto{\pgfqpoint{1.010862in}{1.095990in}}%
\pgfpathlineto{\pgfqpoint{1.016498in}{1.016765in}}%
\pgfpathlineto{\pgfqpoint{1.019317in}{0.983329in}}%
\pgfpathlineto{\pgfqpoint{1.022135in}{1.003497in}}%
\pgfpathlineto{\pgfqpoint{1.024953in}{1.056305in}}%
\pgfpathlineto{\pgfqpoint{1.027771in}{1.036819in}}%
\pgfpathlineto{\pgfqpoint{1.030589in}{1.090059in}}%
\pgfpathlineto{\pgfqpoint{1.033407in}{1.046201in}}%
\pgfpathlineto{\pgfqpoint{1.036226in}{1.032119in}}%
\pgfpathlineto{\pgfqpoint{1.039044in}{1.516996in}}%
\pgfpathlineto{\pgfqpoint{1.041862in}{2.280481in}}%
\pgfpathlineto{\pgfqpoint{1.044680in}{2.030381in}}%
\pgfpathlineto{\pgfqpoint{1.050317in}{1.695120in}}%
\pgfpathlineto{\pgfqpoint{1.053135in}{1.568141in}}%
\pgfpathlineto{\pgfqpoint{1.058771in}{1.382054in}}%
\pgfpathlineto{\pgfqpoint{1.061589in}{1.337877in}}%
\pgfpathlineto{\pgfqpoint{1.067226in}{1.192095in}}%
\pgfpathlineto{\pgfqpoint{1.070044in}{1.203121in}}%
\pgfpathlineto{\pgfqpoint{1.072862in}{1.184781in}}%
\pgfpathlineto{\pgfqpoint{1.075680in}{1.115940in}}%
\pgfpathlineto{\pgfqpoint{1.078498in}{1.093570in}}%
\pgfpathlineto{\pgfqpoint{1.081317in}{1.175995in}}%
\pgfpathlineto{\pgfqpoint{1.086953in}{1.061294in}}%
\pgfpathlineto{\pgfqpoint{1.089771in}{1.072143in}}%
\pgfpathlineto{\pgfqpoint{1.095407in}{0.981274in}}%
\pgfpathlineto{\pgfqpoint{1.098226in}{0.959769in}}%
\pgfpathlineto{\pgfqpoint{1.101044in}{1.191157in}}%
\pgfpathlineto{\pgfqpoint{1.106680in}{1.102265in}}%
\pgfpathlineto{\pgfqpoint{1.109498in}{1.080729in}}%
\pgfpathlineto{\pgfqpoint{1.112317in}{1.042856in}}%
\pgfpathlineto{\pgfqpoint{1.115135in}{1.046500in}}%
\pgfpathlineto{\pgfqpoint{1.117953in}{1.197548in}}%
\pgfpathlineto{\pgfqpoint{1.120771in}{1.162624in}}%
\pgfpathlineto{\pgfqpoint{1.123589in}{1.207928in}}%
\pgfpathlineto{\pgfqpoint{1.129226in}{1.125792in}}%
\pgfpathlineto{\pgfqpoint{1.137680in}{1.029221in}}%
\pgfpathlineto{\pgfqpoint{1.140498in}{1.054663in}}%
\pgfpathlineto{\pgfqpoint{1.143317in}{1.066261in}}%
\pgfpathlineto{\pgfqpoint{1.146135in}{1.048990in}}%
\pgfpathlineto{\pgfqpoint{1.151771in}{0.982725in}}%
\pgfpathlineto{\pgfqpoint{1.154589in}{1.290100in}}%
\pgfpathlineto{\pgfqpoint{1.163044in}{1.184976in}}%
\pgfpathlineto{\pgfqpoint{1.165862in}{1.258881in}}%
\pgfpathlineto{\pgfqpoint{1.168680in}{1.269919in}}%
\pgfpathlineto{\pgfqpoint{1.174317in}{1.203783in}}%
\pgfpathlineto{\pgfqpoint{1.177135in}{1.170466in}}%
\pgfpathlineto{\pgfqpoint{1.179953in}{1.392220in}}%
\pgfpathlineto{\pgfqpoint{1.182771in}{1.356720in}}%
\pgfpathlineto{\pgfqpoint{1.185589in}{1.336345in}}%
\pgfpathlineto{\pgfqpoint{1.194044in}{1.234946in}}%
\pgfpathlineto{\pgfqpoint{1.196862in}{1.339400in}}%
\pgfpathlineto{\pgfqpoint{1.202498in}{1.263740in}}%
\pgfpathlineto{\pgfqpoint{1.205317in}{1.258346in}}%
\pgfpathlineto{\pgfqpoint{1.208135in}{1.232843in}}%
\pgfpathlineto{\pgfqpoint{1.210953in}{1.216883in}}%
\pgfpathlineto{\pgfqpoint{1.213771in}{1.195171in}}%
\pgfpathlineto{\pgfqpoint{1.216589in}{1.150875in}}%
\pgfpathlineto{\pgfqpoint{1.219407in}{1.178538in}}%
\pgfpathlineto{\pgfqpoint{1.222226in}{1.195800in}}%
\pgfpathlineto{\pgfqpoint{1.225044in}{1.176686in}}%
\pgfpathlineto{\pgfqpoint{1.227862in}{1.141852in}}%
\pgfpathlineto{\pgfqpoint{1.230680in}{1.070104in}}%
\pgfpathlineto{\pgfqpoint{1.233498in}{1.139941in}}%
\pgfpathlineto{\pgfqpoint{1.236317in}{1.045323in}}%
\pgfpathlineto{\pgfqpoint{1.239135in}{1.012702in}}%
\pgfpathlineto{\pgfqpoint{1.241953in}{1.014214in}}%
\pgfpathlineto{\pgfqpoint{1.244771in}{0.988610in}}%
\pgfpathlineto{\pgfqpoint{1.247589in}{0.954917in}}%
\pgfpathlineto{\pgfqpoint{1.250407in}{0.933250in}}%
\pgfpathlineto{\pgfqpoint{1.253226in}{0.954247in}}%
\pgfpathlineto{\pgfqpoint{1.256044in}{0.959662in}}%
\pgfpathlineto{\pgfqpoint{1.258862in}{1.154371in}}%
\pgfpathlineto{\pgfqpoint{1.261680in}{1.149086in}}%
\pgfpathlineto{\pgfqpoint{1.264498in}{1.003392in}}%
\pgfpathlineto{\pgfqpoint{1.267317in}{0.974350in}}%
\pgfpathlineto{\pgfqpoint{1.270135in}{0.956603in}}%
\pgfpathlineto{\pgfqpoint{1.272953in}{0.966049in}}%
\pgfpathlineto{\pgfqpoint{1.275771in}{0.969841in}}%
\pgfpathlineto{\pgfqpoint{1.278589in}{0.950581in}}%
\pgfpathlineto{\pgfqpoint{1.281407in}{0.919875in}}%
\pgfpathlineto{\pgfqpoint{1.284226in}{0.897575in}}%
\pgfpathlineto{\pgfqpoint{1.287044in}{0.884223in}}%
\pgfpathlineto{\pgfqpoint{1.289862in}{0.943730in}}%
\pgfpathlineto{\pgfqpoint{1.292680in}{1.228411in}}%
\pgfpathlineto{\pgfqpoint{1.301135in}{1.211571in}}%
\pgfpathlineto{\pgfqpoint{1.306771in}{1.182005in}}%
\pgfpathlineto{\pgfqpoint{1.309589in}{1.178271in}}%
\pgfpathlineto{\pgfqpoint{1.312407in}{1.165983in}}%
\pgfpathlineto{\pgfqpoint{1.318044in}{1.174371in}}%
\pgfpathlineto{\pgfqpoint{1.320862in}{1.167379in}}%
\pgfpathlineto{\pgfqpoint{1.323680in}{1.164870in}}%
\pgfpathlineto{\pgfqpoint{1.332135in}{1.145051in}}%
\pgfpathlineto{\pgfqpoint{1.334953in}{1.134643in}}%
\pgfpathlineto{\pgfqpoint{1.337771in}{1.128349in}}%
\pgfpathlineto{\pgfqpoint{1.343407in}{1.111435in}}%
\pgfpathlineto{\pgfqpoint{1.349044in}{1.097732in}}%
\pgfpathlineto{\pgfqpoint{1.351862in}{1.101728in}}%
\pgfpathlineto{\pgfqpoint{1.354680in}{1.114816in}}%
\pgfpathlineto{\pgfqpoint{1.360317in}{1.105957in}}%
\pgfpathlineto{\pgfqpoint{1.365953in}{1.116082in}}%
\pgfpathlineto{\pgfqpoint{1.368771in}{1.108083in}}%
\pgfpathlineto{\pgfqpoint{1.371589in}{1.103221in}}%
\pgfpathlineto{\pgfqpoint{1.374407in}{1.101826in}}%
\pgfpathlineto{\pgfqpoint{1.380044in}{1.097758in}}%
\pgfpathlineto{\pgfqpoint{1.382862in}{1.092796in}}%
\pgfpathlineto{\pgfqpoint{1.385680in}{1.093844in}}%
\pgfpathlineto{\pgfqpoint{1.396953in}{1.071400in}}%
\pgfpathlineto{\pgfqpoint{1.399771in}{1.070410in}}%
\pgfpathlineto{\pgfqpoint{1.402589in}{1.066203in}}%
\pgfpathlineto{\pgfqpoint{1.405407in}{0.950572in}}%
\pgfpathlineto{\pgfqpoint{1.408226in}{0.950057in}}%
\pgfpathlineto{\pgfqpoint{1.411044in}{0.946596in}}%
\pgfpathlineto{\pgfqpoint{1.413862in}{1.048220in}}%
\pgfpathlineto{\pgfqpoint{1.416680in}{0.948277in}}%
\pgfpathlineto{\pgfqpoint{1.419498in}{1.415684in}}%
\pgfpathlineto{\pgfqpoint{1.422317in}{1.061545in}}%
\pgfpathlineto{\pgfqpoint{1.425135in}{1.057244in}}%
\pgfpathlineto{\pgfqpoint{1.427953in}{1.060532in}}%
\pgfpathlineto{\pgfqpoint{1.430771in}{1.056420in}}%
\pgfpathlineto{\pgfqpoint{1.433589in}{1.059883in}}%
\pgfpathlineto{\pgfqpoint{1.436407in}{1.056659in}}%
\pgfpathlineto{\pgfqpoint{1.442044in}{1.051771in}}%
\pgfpathlineto{\pgfqpoint{1.444862in}{1.067942in}}%
\pgfpathlineto{\pgfqpoint{1.447680in}{1.076903in}}%
\pgfpathlineto{\pgfqpoint{1.450498in}{1.082180in}}%
\pgfpathlineto{\pgfqpoint{1.453317in}{1.078023in}}%
\pgfpathlineto{\pgfqpoint{1.456135in}{1.077783in}}%
\pgfpathlineto{\pgfqpoint{1.458953in}{1.090453in}}%
\pgfpathlineto{\pgfqpoint{1.461771in}{1.084253in}}%
\pgfpathlineto{\pgfqpoint{1.464589in}{1.084757in}}%
\pgfpathlineto{\pgfqpoint{1.467407in}{1.081879in}}%
\pgfpathlineto{\pgfqpoint{1.470226in}{1.087620in}}%
\pgfpathlineto{\pgfqpoint{1.473044in}{1.096338in}}%
\pgfpathlineto{\pgfqpoint{1.478680in}{1.093687in}}%
\pgfpathlineto{\pgfqpoint{1.481498in}{1.087329in}}%
\pgfpathlineto{\pgfqpoint{1.484317in}{1.089030in}}%
\pgfpathlineto{\pgfqpoint{1.492771in}{1.074677in}}%
\pgfpathlineto{\pgfqpoint{1.498407in}{1.065407in}}%
\pgfpathlineto{\pgfqpoint{1.501226in}{1.063298in}}%
\pgfpathlineto{\pgfqpoint{1.506862in}{1.055697in}}%
\pgfpathlineto{\pgfqpoint{1.509680in}{1.052558in}}%
\pgfpathlineto{\pgfqpoint{1.512498in}{1.055092in}}%
\pgfpathlineto{\pgfqpoint{1.515317in}{1.053408in}}%
\pgfpathlineto{\pgfqpoint{1.518135in}{1.054168in}}%
\pgfpathlineto{\pgfqpoint{1.520953in}{1.052967in}}%
\pgfpathlineto{\pgfqpoint{1.532226in}{1.039595in}}%
\pgfpathlineto{\pgfqpoint{1.537862in}{1.033952in}}%
\pgfpathlineto{\pgfqpoint{1.546317in}{1.027157in}}%
\pgfpathlineto{\pgfqpoint{1.549135in}{0.945942in}}%
\pgfpathlineto{\pgfqpoint{1.551953in}{1.042925in}}%
\pgfpathlineto{\pgfqpoint{1.554771in}{1.040592in}}%
\pgfpathlineto{\pgfqpoint{1.557589in}{0.970442in}}%
\pgfpathlineto{\pgfqpoint{1.560407in}{1.046015in}}%
\pgfpathlineto{\pgfqpoint{1.563226in}{1.043152in}}%
\pgfpathlineto{\pgfqpoint{1.566044in}{1.044514in}}%
\pgfpathlineto{\pgfqpoint{1.571680in}{1.042629in}}%
\pgfpathlineto{\pgfqpoint{1.574498in}{1.046755in}}%
\pgfpathlineto{\pgfqpoint{1.580135in}{1.041597in}}%
\pgfpathlineto{\pgfqpoint{1.582953in}{1.041698in}}%
\pgfpathlineto{\pgfqpoint{1.585771in}{1.044309in}}%
\pgfpathlineto{\pgfqpoint{1.588589in}{0.960569in}}%
\pgfpathlineto{\pgfqpoint{1.591407in}{1.055297in}}%
\pgfpathlineto{\pgfqpoint{1.594226in}{1.064396in}}%
\pgfpathlineto{\pgfqpoint{1.597044in}{1.063829in}}%
\pgfpathlineto{\pgfqpoint{1.602680in}{1.057191in}}%
\pgfpathlineto{\pgfqpoint{1.605498in}{0.982857in}}%
\pgfpathlineto{\pgfqpoint{1.608317in}{0.963356in}}%
\pgfpathlineto{\pgfqpoint{1.611135in}{0.959151in}}%
\pgfpathlineto{\pgfqpoint{1.613953in}{0.957459in}}%
\pgfpathlineto{\pgfqpoint{1.616771in}{1.023901in}}%
\pgfpathlineto{\pgfqpoint{1.619589in}{1.066974in}}%
\pgfpathlineto{\pgfqpoint{1.625226in}{1.060481in}}%
\pgfpathlineto{\pgfqpoint{1.628044in}{1.073358in}}%
\pgfpathlineto{\pgfqpoint{1.633680in}{1.066430in}}%
\pgfpathlineto{\pgfqpoint{1.636498in}{1.065983in}}%
\pgfpathlineto{\pgfqpoint{1.639317in}{1.062955in}}%
\pgfpathlineto{\pgfqpoint{1.642135in}{1.083392in}}%
\pgfpathlineto{\pgfqpoint{1.644953in}{1.081191in}}%
\pgfpathlineto{\pgfqpoint{1.647771in}{1.080194in}}%
\pgfpathlineto{\pgfqpoint{1.650589in}{1.076373in}}%
\pgfpathlineto{\pgfqpoint{1.659044in}{1.069216in}}%
\pgfpathlineto{\pgfqpoint{1.664680in}{1.063804in}}%
\pgfpathlineto{\pgfqpoint{1.673135in}{1.056268in}}%
\pgfpathlineto{\pgfqpoint{1.678771in}{1.051790in}}%
\pgfpathlineto{\pgfqpoint{1.681589in}{0.969799in}}%
\pgfpathlineto{\pgfqpoint{1.684407in}{0.968962in}}%
\pgfpathlineto{\pgfqpoint{1.687226in}{0.969630in}}%
\pgfpathlineto{\pgfqpoint{1.690044in}{0.979133in}}%
\pgfpathlineto{\pgfqpoint{1.692862in}{0.966394in}}%
\pgfpathlineto{\pgfqpoint{1.695680in}{0.965537in}}%
\pgfpathlineto{\pgfqpoint{1.698498in}{0.968259in}}%
\pgfpathlineto{\pgfqpoint{1.701317in}{0.966371in}}%
\pgfpathlineto{\pgfqpoint{1.704135in}{1.080782in}}%
\pgfpathlineto{\pgfqpoint{1.709771in}{0.974481in}}%
\pgfpathlineto{\pgfqpoint{1.712589in}{0.970865in}}%
\pgfpathlineto{\pgfqpoint{1.715407in}{0.993518in}}%
\pgfpathlineto{\pgfqpoint{1.718226in}{0.966696in}}%
\pgfpathlineto{\pgfqpoint{1.721044in}{0.961531in}}%
\pgfpathlineto{\pgfqpoint{1.723862in}{0.966601in}}%
\pgfpathlineto{\pgfqpoint{1.726680in}{0.959488in}}%
\pgfpathlineto{\pgfqpoint{1.729498in}{0.956973in}}%
\pgfpathlineto{\pgfqpoint{1.732317in}{0.962891in}}%
\pgfpathlineto{\pgfqpoint{1.735135in}{1.233707in}}%
\pgfpathlineto{\pgfqpoint{1.737953in}{1.000109in}}%
\pgfpathlineto{\pgfqpoint{1.740771in}{1.031937in}}%
\pgfpathlineto{\pgfqpoint{1.743589in}{1.030973in}}%
\pgfpathlineto{\pgfqpoint{1.752044in}{1.026355in}}%
\pgfpathlineto{\pgfqpoint{1.754862in}{1.407360in}}%
\pgfpathlineto{\pgfqpoint{1.757680in}{1.035871in}}%
\pgfpathlineto{\pgfqpoint{1.760498in}{1.036355in}}%
\pgfpathlineto{\pgfqpoint{1.774589in}{1.028720in}}%
\pgfpathlineto{\pgfqpoint{1.780226in}{1.026414in}}%
\pgfpathlineto{\pgfqpoint{1.783044in}{1.027390in}}%
\pgfpathlineto{\pgfqpoint{1.791498in}{1.023150in}}%
\pgfpathlineto{\pgfqpoint{1.797135in}{1.021336in}}%
\pgfpathlineto{\pgfqpoint{1.799953in}{1.020027in}}%
\pgfpathlineto{\pgfqpoint{1.802771in}{1.019791in}}%
\pgfpathlineto{\pgfqpoint{1.805589in}{1.018559in}}%
\pgfpathlineto{\pgfqpoint{1.808407in}{1.021187in}}%
\pgfpathlineto{\pgfqpoint{1.816862in}{1.017555in}}%
\pgfpathlineto{\pgfqpoint{1.822498in}{1.016030in}}%
\pgfpathlineto{\pgfqpoint{1.830953in}{1.012536in}}%
\pgfpathlineto{\pgfqpoint{1.833771in}{1.012005in}}%
\pgfpathlineto{\pgfqpoint{1.836589in}{1.010809in}}%
\pgfpathlineto{\pgfqpoint{1.839407in}{1.011984in}}%
\pgfpathlineto{\pgfqpoint{1.850680in}{1.007250in}}%
\pgfpathlineto{\pgfqpoint{1.853498in}{1.001810in}}%
\pgfpathlineto{\pgfqpoint{1.859135in}{0.980089in}}%
\pgfpathlineto{\pgfqpoint{1.864771in}{0.971396in}}%
\pgfpathlineto{\pgfqpoint{1.870407in}{0.950715in}}%
\pgfpathlineto{\pgfqpoint{1.873226in}{0.948221in}}%
\pgfpathlineto{\pgfqpoint{1.876044in}{0.940527in}}%
\pgfpathlineto{\pgfqpoint{1.878862in}{0.938428in}}%
\pgfpathlineto{\pgfqpoint{1.884498in}{0.919631in}}%
\pgfpathlineto{\pgfqpoint{1.887317in}{0.910031in}}%
\pgfpathlineto{\pgfqpoint{1.890135in}{0.904727in}}%
\pgfpathlineto{\pgfqpoint{1.892953in}{0.931087in}}%
\pgfpathlineto{\pgfqpoint{1.895771in}{0.997493in}}%
\pgfpathlineto{\pgfqpoint{1.898589in}{0.957690in}}%
\pgfpathlineto{\pgfqpoint{1.901407in}{0.952743in}}%
\pgfpathlineto{\pgfqpoint{1.904226in}{0.998867in}}%
\pgfpathlineto{\pgfqpoint{1.907044in}{0.956657in}}%
\pgfpathlineto{\pgfqpoint{1.909862in}{0.950768in}}%
\pgfpathlineto{\pgfqpoint{1.912680in}{0.949038in}}%
\pgfpathlineto{\pgfqpoint{1.915498in}{0.975253in}}%
\pgfpathlineto{\pgfqpoint{1.918317in}{1.179488in}}%
\pgfpathlineto{\pgfqpoint{1.921135in}{1.240517in}}%
\pgfpathlineto{\pgfqpoint{1.923953in}{0.982601in}}%
\pgfpathlineto{\pgfqpoint{1.926771in}{0.950757in}}%
\pgfpathlineto{\pgfqpoint{1.929589in}{0.948584in}}%
\pgfpathlineto{\pgfqpoint{1.932407in}{0.948339in}}%
\pgfpathlineto{\pgfqpoint{1.935226in}{0.946202in}}%
\pgfpathlineto{\pgfqpoint{1.938044in}{0.965430in}}%
\pgfpathlineto{\pgfqpoint{1.940862in}{0.976372in}}%
\pgfpathlineto{\pgfqpoint{1.943680in}{1.141853in}}%
\pgfpathlineto{\pgfqpoint{1.946498in}{0.964175in}}%
\pgfpathlineto{\pgfqpoint{1.949317in}{0.947644in}}%
\pgfpathlineto{\pgfqpoint{1.952135in}{0.991877in}}%
\pgfpathlineto{\pgfqpoint{1.954953in}{0.957381in}}%
\pgfpathlineto{\pgfqpoint{1.957771in}{0.993937in}}%
\pgfpathlineto{\pgfqpoint{1.960589in}{0.948555in}}%
\pgfpathlineto{\pgfqpoint{1.963407in}{0.982672in}}%
\pgfpathlineto{\pgfqpoint{1.966226in}{0.953973in}}%
\pgfpathlineto{\pgfqpoint{1.969044in}{0.942988in}}%
\pgfpathlineto{\pgfqpoint{1.971862in}{0.940990in}}%
\pgfpathlineto{\pgfqpoint{1.974680in}{0.940445in}}%
\pgfpathlineto{\pgfqpoint{1.977498in}{0.949262in}}%
\pgfpathlineto{\pgfqpoint{1.980317in}{0.939346in}}%
\pgfpathlineto{\pgfqpoint{1.983135in}{0.937371in}}%
\pgfpathlineto{\pgfqpoint{1.985953in}{0.966172in}}%
\pgfpathlineto{\pgfqpoint{1.988771in}{0.938931in}}%
\pgfpathlineto{\pgfqpoint{1.991589in}{1.612904in}}%
\pgfpathlineto{\pgfqpoint{1.994407in}{1.025759in}}%
\pgfpathlineto{\pgfqpoint{1.997226in}{0.958387in}}%
\pgfpathlineto{\pgfqpoint{2.000044in}{0.959431in}}%
\pgfpathlineto{\pgfqpoint{2.005680in}{0.957170in}}%
\pgfpathlineto{\pgfqpoint{2.008498in}{0.956041in}}%
\pgfpathlineto{\pgfqpoint{2.011317in}{0.955601in}}%
\pgfpathlineto{\pgfqpoint{2.014135in}{0.954401in}}%
\pgfpathlineto{\pgfqpoint{2.016953in}{0.969905in}}%
\pgfpathlineto{\pgfqpoint{2.019771in}{0.953977in}}%
\pgfpathlineto{\pgfqpoint{2.022589in}{0.952959in}}%
\pgfpathlineto{\pgfqpoint{2.025407in}{0.950417in}}%
\pgfpathlineto{\pgfqpoint{2.028226in}{0.954536in}}%
\pgfpathlineto{\pgfqpoint{2.033862in}{0.952165in}}%
\pgfpathlineto{\pgfqpoint{2.036680in}{1.088408in}}%
\pgfpathlineto{\pgfqpoint{2.039498in}{0.993307in}}%
\pgfpathlineto{\pgfqpoint{2.042317in}{0.960425in}}%
\pgfpathlineto{\pgfqpoint{2.045135in}{0.970128in}}%
\pgfpathlineto{\pgfqpoint{2.047953in}{0.957359in}}%
\pgfpathlineto{\pgfqpoint{2.050771in}{0.964038in}}%
\pgfpathlineto{\pgfqpoint{2.053589in}{0.953716in}}%
\pgfpathlineto{\pgfqpoint{2.056407in}{0.950609in}}%
\pgfpathlineto{\pgfqpoint{2.059226in}{0.949586in}}%
\pgfpathlineto{\pgfqpoint{2.062044in}{0.978032in}}%
\pgfpathlineto{\pgfqpoint{2.064862in}{0.950350in}}%
\pgfpathlineto{\pgfqpoint{2.067680in}{0.971231in}}%
\pgfpathlineto{\pgfqpoint{2.070498in}{0.948687in}}%
\pgfpathlineto{\pgfqpoint{2.073317in}{0.961845in}}%
\pgfpathlineto{\pgfqpoint{2.076135in}{0.946468in}}%
\pgfpathlineto{\pgfqpoint{2.078953in}{0.944078in}}%
\pgfpathlineto{\pgfqpoint{2.081771in}{0.943140in}}%
\pgfpathlineto{\pgfqpoint{2.084589in}{0.947632in}}%
\pgfpathlineto{\pgfqpoint{2.087407in}{0.941644in}}%
\pgfpathlineto{\pgfqpoint{2.090226in}{0.948868in}}%
\pgfpathlineto{\pgfqpoint{2.093044in}{0.940169in}}%
\pgfpathlineto{\pgfqpoint{2.095862in}{0.960436in}}%
\pgfpathlineto{\pgfqpoint{2.098680in}{0.940223in}}%
\pgfpathlineto{\pgfqpoint{2.101498in}{0.937302in}}%
\pgfpathlineto{\pgfqpoint{2.107135in}{0.934427in}}%
\pgfpathlineto{\pgfqpoint{2.109953in}{0.956699in}}%
\pgfpathlineto{\pgfqpoint{2.112771in}{0.943370in}}%
\pgfpathlineto{\pgfqpoint{2.115589in}{0.955953in}}%
\pgfpathlineto{\pgfqpoint{2.118407in}{1.034139in}}%
\pgfpathlineto{\pgfqpoint{2.121226in}{0.945151in}}%
\pgfpathlineto{\pgfqpoint{2.124044in}{0.932341in}}%
\pgfpathlineto{\pgfqpoint{2.126862in}{0.997440in}}%
\pgfpathlineto{\pgfqpoint{2.129680in}{0.939183in}}%
\pgfpathlineto{\pgfqpoint{2.132498in}{0.931000in}}%
\pgfpathlineto{\pgfqpoint{2.135317in}{0.984996in}}%
\pgfpathlineto{\pgfqpoint{2.138135in}{0.954002in}}%
\pgfpathlineto{\pgfqpoint{2.140953in}{0.946890in}}%
\pgfpathlineto{\pgfqpoint{2.143771in}{0.930362in}}%
\pgfpathlineto{\pgfqpoint{2.146589in}{1.068302in}}%
\pgfpathlineto{\pgfqpoint{2.149407in}{0.946823in}}%
\pgfpathlineto{\pgfqpoint{2.155044in}{1.023371in}}%
\pgfpathlineto{\pgfqpoint{2.157862in}{0.942050in}}%
\pgfpathlineto{\pgfqpoint{2.160680in}{0.931102in}}%
\pgfpathlineto{\pgfqpoint{2.163498in}{0.928090in}}%
\pgfpathlineto{\pgfqpoint{2.166317in}{0.998573in}}%
\pgfpathlineto{\pgfqpoint{2.169135in}{0.957823in}}%
\pgfpathlineto{\pgfqpoint{2.171953in}{0.942667in}}%
\pgfpathlineto{\pgfqpoint{2.174771in}{0.932998in}}%
\pgfpathlineto{\pgfqpoint{2.177589in}{0.926847in}}%
\pgfpathlineto{\pgfqpoint{2.183226in}{0.923644in}}%
\pgfpathlineto{\pgfqpoint{2.186044in}{0.922462in}}%
\pgfpathlineto{\pgfqpoint{2.188862in}{0.924246in}}%
\pgfpathlineto{\pgfqpoint{2.191680in}{0.924757in}}%
\pgfpathlineto{\pgfqpoint{2.194498in}{0.932231in}}%
\pgfpathlineto{\pgfqpoint{2.197317in}{0.942816in}}%
\pgfpathlineto{\pgfqpoint{2.200135in}{0.924441in}}%
\pgfpathlineto{\pgfqpoint{2.202953in}{0.920243in}}%
\pgfpathlineto{\pgfqpoint{2.208589in}{0.917327in}}%
\pgfpathlineto{\pgfqpoint{2.211407in}{0.930215in}}%
\pgfpathlineto{\pgfqpoint{2.214226in}{0.918279in}}%
\pgfpathlineto{\pgfqpoint{2.217044in}{0.914832in}}%
\pgfpathlineto{\pgfqpoint{2.222680in}{0.914611in}}%
\pgfpathlineto{\pgfqpoint{2.225498in}{0.911220in}}%
\pgfpathlineto{\pgfqpoint{2.231135in}{0.909503in}}%
\pgfpathlineto{\pgfqpoint{2.233953in}{0.933651in}}%
\pgfpathlineto{\pgfqpoint{2.236771in}{0.914302in}}%
\pgfpathlineto{\pgfqpoint{2.239589in}{0.919265in}}%
\pgfpathlineto{\pgfqpoint{2.242407in}{0.909183in}}%
\pgfpathlineto{\pgfqpoint{2.245226in}{0.780860in}}%
\pgfpathlineto{\pgfqpoint{2.248044in}{0.999105in}}%
\pgfpathlineto{\pgfqpoint{2.250862in}{0.797768in}}%
\pgfpathlineto{\pgfqpoint{2.253680in}{0.786430in}}%
\pgfpathlineto{\pgfqpoint{2.256498in}{0.798088in}}%
\pgfpathlineto{\pgfqpoint{2.259317in}{0.788254in}}%
\pgfpathlineto{\pgfqpoint{2.262135in}{0.802187in}}%
\pgfpathlineto{\pgfqpoint{2.267771in}{0.782167in}}%
\pgfpathlineto{\pgfqpoint{2.270589in}{0.808149in}}%
\pgfpathlineto{\pgfqpoint{2.273407in}{0.810555in}}%
\pgfpathlineto{\pgfqpoint{2.276226in}{0.814338in}}%
\pgfpathlineto{\pgfqpoint{2.279044in}{0.819898in}}%
\pgfpathlineto{\pgfqpoint{2.281862in}{0.810574in}}%
\pgfpathlineto{\pgfqpoint{2.284680in}{1.052964in}}%
\pgfpathlineto{\pgfqpoint{2.290317in}{0.868008in}}%
\pgfpathlineto{\pgfqpoint{2.293135in}{0.949427in}}%
\pgfpathlineto{\pgfqpoint{2.295953in}{0.952537in}}%
\pgfpathlineto{\pgfqpoint{2.301589in}{0.947188in}}%
\pgfpathlineto{\pgfqpoint{2.307226in}{0.970107in}}%
\pgfpathlineto{\pgfqpoint{2.310044in}{0.945573in}}%
\pgfpathlineto{\pgfqpoint{2.312862in}{0.947952in}}%
\pgfpathlineto{\pgfqpoint{2.315680in}{0.944061in}}%
\pgfpathlineto{\pgfqpoint{2.318498in}{0.944011in}}%
\pgfpathlineto{\pgfqpoint{2.321317in}{0.942579in}}%
\pgfpathlineto{\pgfqpoint{2.324135in}{0.945958in}}%
\pgfpathlineto{\pgfqpoint{2.326953in}{0.940995in}}%
\pgfpathlineto{\pgfqpoint{2.329771in}{0.940083in}}%
\pgfpathlineto{\pgfqpoint{2.332589in}{0.940486in}}%
\pgfpathlineto{\pgfqpoint{2.335407in}{0.965991in}}%
\pgfpathlineto{\pgfqpoint{2.338226in}{0.939022in}}%
\pgfpathlineto{\pgfqpoint{2.341044in}{0.948617in}}%
\pgfpathlineto{\pgfqpoint{2.343862in}{0.762857in}}%
\pgfpathlineto{\pgfqpoint{2.346680in}{2.285639in}}%
\pgfpathlineto{\pgfqpoint{2.349498in}{0.961619in}}%
\pgfpathlineto{\pgfqpoint{2.352317in}{0.961137in}}%
\pgfpathlineto{\pgfqpoint{2.355135in}{0.988716in}}%
\pgfpathlineto{\pgfqpoint{2.357953in}{1.086329in}}%
\pgfpathlineto{\pgfqpoint{2.360771in}{0.959783in}}%
\pgfpathlineto{\pgfqpoint{2.366407in}{0.958344in}}%
\pgfpathlineto{\pgfqpoint{2.372044in}{0.957297in}}%
\pgfpathlineto{\pgfqpoint{2.374862in}{0.961604in}}%
\pgfpathlineto{\pgfqpoint{2.377680in}{0.956037in}}%
\pgfpathlineto{\pgfqpoint{2.380498in}{0.955297in}}%
\pgfpathlineto{\pgfqpoint{2.383317in}{0.976919in}}%
\pgfpathlineto{\pgfqpoint{2.386135in}{0.959113in}}%
\pgfpathlineto{\pgfqpoint{2.394589in}{0.957204in}}%
\pgfpathlineto{\pgfqpoint{2.397407in}{1.002741in}}%
\pgfpathlineto{\pgfqpoint{2.400226in}{0.959043in}}%
\pgfpathlineto{\pgfqpoint{2.403044in}{1.205489in}}%
\pgfpathlineto{\pgfqpoint{2.405862in}{0.963037in}}%
\pgfpathlineto{\pgfqpoint{2.408680in}{0.972467in}}%
\pgfpathlineto{\pgfqpoint{2.411498in}{0.972720in}}%
\pgfpathlineto{\pgfqpoint{2.414317in}{0.961163in}}%
\pgfpathlineto{\pgfqpoint{2.417135in}{0.964966in}}%
\pgfpathlineto{\pgfqpoint{2.419953in}{1.049989in}}%
\pgfpathlineto{\pgfqpoint{2.422771in}{0.960658in}}%
\pgfpathlineto{\pgfqpoint{2.434044in}{0.958466in}}%
\pgfpathlineto{\pgfqpoint{2.436862in}{0.971228in}}%
\pgfpathlineto{\pgfqpoint{2.439680in}{0.956584in}}%
\pgfpathlineto{\pgfqpoint{2.442498in}{0.956159in}}%
\pgfpathlineto{\pgfqpoint{2.445317in}{0.978211in}}%
\pgfpathlineto{\pgfqpoint{2.448135in}{0.956210in}}%
\pgfpathlineto{\pgfqpoint{2.450953in}{0.976373in}}%
\pgfpathlineto{\pgfqpoint{2.453771in}{0.953185in}}%
\pgfpathlineto{\pgfqpoint{2.456589in}{0.954572in}}%
\pgfpathlineto{\pgfqpoint{2.459407in}{0.952278in}}%
\pgfpathlineto{\pgfqpoint{2.465044in}{0.950981in}}%
\pgfpathlineto{\pgfqpoint{2.467862in}{0.983113in}}%
\pgfpathlineto{\pgfqpoint{2.470680in}{0.950329in}}%
\pgfpathlineto{\pgfqpoint{2.473498in}{0.956449in}}%
\pgfpathlineto{\pgfqpoint{2.476317in}{0.948633in}}%
\pgfpathlineto{\pgfqpoint{2.481953in}{0.947176in}}%
\pgfpathlineto{\pgfqpoint{2.484771in}{0.968270in}}%
\pgfpathlineto{\pgfqpoint{2.487589in}{0.946125in}}%
\pgfpathlineto{\pgfqpoint{2.490407in}{0.948302in}}%
\pgfpathlineto{\pgfqpoint{2.493226in}{1.038622in}}%
\pgfpathlineto{\pgfqpoint{2.496044in}{0.946675in}}%
\pgfpathlineto{\pgfqpoint{2.498862in}{0.949757in}}%
\pgfpathlineto{\pgfqpoint{2.501680in}{0.949373in}}%
\pgfpathlineto{\pgfqpoint{2.504498in}{1.006403in}}%
\pgfpathlineto{\pgfqpoint{2.507317in}{0.950209in}}%
\pgfpathlineto{\pgfqpoint{2.510135in}{0.950494in}}%
\pgfpathlineto{\pgfqpoint{2.512953in}{1.083488in}}%
\pgfpathlineto{\pgfqpoint{2.515771in}{0.950004in}}%
\pgfpathlineto{\pgfqpoint{2.518589in}{0.966444in}}%
\pgfpathlineto{\pgfqpoint{2.521407in}{0.965987in}}%
\pgfpathlineto{\pgfqpoint{2.524226in}{1.078684in}}%
\pgfpathlineto{\pgfqpoint{2.527044in}{0.965959in}}%
\pgfpathlineto{\pgfqpoint{2.529862in}{0.965519in}}%
\pgfpathlineto{\pgfqpoint{2.532680in}{1.298389in}}%
\pgfpathlineto{\pgfqpoint{2.535498in}{1.261488in}}%
\pgfpathlineto{\pgfqpoint{2.538317in}{1.281828in}}%
\pgfpathlineto{\pgfqpoint{2.541135in}{1.269489in}}%
\pgfpathlineto{\pgfqpoint{2.549589in}{1.200734in}}%
\pgfpathlineto{\pgfqpoint{2.552407in}{1.184738in}}%
\pgfpathlineto{\pgfqpoint{2.555226in}{1.162716in}}%
\pgfpathlineto{\pgfqpoint{2.558044in}{1.355101in}}%
\pgfpathlineto{\pgfqpoint{2.560862in}{1.373307in}}%
\pgfpathlineto{\pgfqpoint{2.566498in}{1.302747in}}%
\pgfpathlineto{\pgfqpoint{2.569317in}{1.316716in}}%
\pgfpathlineto{\pgfqpoint{2.572135in}{1.283048in}}%
\pgfpathlineto{\pgfqpoint{2.574953in}{1.262443in}}%
\pgfpathlineto{\pgfqpoint{2.577771in}{1.250060in}}%
\pgfpathlineto{\pgfqpoint{2.583407in}{1.196595in}}%
\pgfpathlineto{\pgfqpoint{2.591862in}{1.131742in}}%
\pgfpathlineto{\pgfqpoint{2.594680in}{1.115936in}}%
\pgfpathlineto{\pgfqpoint{2.597498in}{1.114675in}}%
\pgfpathlineto{\pgfqpoint{2.600317in}{1.122652in}}%
\pgfpathlineto{\pgfqpoint{2.603135in}{1.120118in}}%
\pgfpathlineto{\pgfqpoint{2.605953in}{1.107989in}}%
\pgfpathlineto{\pgfqpoint{2.608771in}{1.089818in}}%
\pgfpathlineto{\pgfqpoint{2.611589in}{1.078161in}}%
\pgfpathlineto{\pgfqpoint{2.620044in}{1.035081in}}%
\pgfpathlineto{\pgfqpoint{2.622862in}{1.022404in}}%
\pgfpathlineto{\pgfqpoint{2.628498in}{0.986848in}}%
\pgfpathlineto{\pgfqpoint{2.631317in}{1.005312in}}%
\pgfpathlineto{\pgfqpoint{2.634135in}{1.003677in}}%
\pgfpathlineto{\pgfqpoint{2.636953in}{1.017741in}}%
\pgfpathlineto{\pgfqpoint{2.639771in}{1.004312in}}%
\pgfpathlineto{\pgfqpoint{2.642589in}{1.017710in}}%
\pgfpathlineto{\pgfqpoint{2.645407in}{1.016705in}}%
\pgfpathlineto{\pgfqpoint{2.648226in}{1.018578in}}%
\pgfpathlineto{\pgfqpoint{2.651044in}{1.010776in}}%
\pgfpathlineto{\pgfqpoint{2.653862in}{1.005162in}}%
\pgfpathlineto{\pgfqpoint{2.656680in}{1.035934in}}%
\pgfpathlineto{\pgfqpoint{2.659498in}{1.028375in}}%
\pgfpathlineto{\pgfqpoint{2.662317in}{1.016451in}}%
\pgfpathlineto{\pgfqpoint{2.665135in}{0.995827in}}%
\pgfpathlineto{\pgfqpoint{2.667953in}{0.992763in}}%
\pgfpathlineto{\pgfqpoint{2.670771in}{0.995442in}}%
\pgfpathlineto{\pgfqpoint{2.673589in}{1.002421in}}%
\pgfpathlineto{\pgfqpoint{2.676407in}{1.013037in}}%
\pgfpathlineto{\pgfqpoint{2.679226in}{1.001698in}}%
\pgfpathlineto{\pgfqpoint{2.682044in}{1.007763in}}%
\pgfpathlineto{\pgfqpoint{2.684862in}{1.041163in}}%
\pgfpathlineto{\pgfqpoint{2.701771in}{0.985168in}}%
\pgfpathlineto{\pgfqpoint{2.704589in}{0.977908in}}%
\pgfpathlineto{\pgfqpoint{2.707407in}{0.968103in}}%
\pgfpathlineto{\pgfqpoint{2.710226in}{0.972231in}}%
\pgfpathlineto{\pgfqpoint{2.715862in}{0.954883in}}%
\pgfpathlineto{\pgfqpoint{2.718680in}{0.966418in}}%
\pgfpathlineto{\pgfqpoint{2.721498in}{0.995819in}}%
\pgfpathlineto{\pgfqpoint{2.724317in}{1.233775in}}%
\pgfpathlineto{\pgfqpoint{2.727135in}{1.232654in}}%
\pgfpathlineto{\pgfqpoint{2.729953in}{1.213390in}}%
\pgfpathlineto{\pgfqpoint{2.732771in}{1.362477in}}%
\pgfpathlineto{\pgfqpoint{2.735589in}{1.344909in}}%
\pgfpathlineto{\pgfqpoint{2.741226in}{1.275632in}}%
\pgfpathlineto{\pgfqpoint{2.744044in}{1.251306in}}%
\pgfpathlineto{\pgfqpoint{2.746862in}{1.281985in}}%
\pgfpathlineto{\pgfqpoint{2.749680in}{1.274222in}}%
\pgfpathlineto{\pgfqpoint{2.752498in}{1.260550in}}%
\pgfpathlineto{\pgfqpoint{2.755317in}{1.237745in}}%
\pgfpathlineto{\pgfqpoint{2.758135in}{1.219489in}}%
\pgfpathlineto{\pgfqpoint{2.763771in}{1.177314in}}%
\pgfpathlineto{\pgfqpoint{2.766589in}{1.183093in}}%
\pgfpathlineto{\pgfqpoint{2.772226in}{1.143878in}}%
\pgfpathlineto{\pgfqpoint{2.775044in}{1.174616in}}%
\pgfpathlineto{\pgfqpoint{2.777862in}{1.159400in}}%
\pgfpathlineto{\pgfqpoint{2.780680in}{1.166318in}}%
\pgfpathlineto{\pgfqpoint{2.783498in}{1.157923in}}%
\pgfpathlineto{\pgfqpoint{2.786317in}{1.158750in}}%
\pgfpathlineto{\pgfqpoint{2.789135in}{1.146716in}}%
\pgfpathlineto{\pgfqpoint{2.794771in}{1.109579in}}%
\pgfpathlineto{\pgfqpoint{2.797589in}{1.111099in}}%
\pgfpathlineto{\pgfqpoint{2.800407in}{1.129416in}}%
\pgfpathlineto{\pgfqpoint{2.803226in}{1.140643in}}%
\pgfpathlineto{\pgfqpoint{2.806044in}{1.255475in}}%
\pgfpathlineto{\pgfqpoint{2.808862in}{1.238268in}}%
\pgfpathlineto{\pgfqpoint{2.811680in}{1.544192in}}%
\pgfpathlineto{\pgfqpoint{2.817317in}{1.455606in}}%
\pgfpathlineto{\pgfqpoint{2.822953in}{1.397067in}}%
\pgfpathlineto{\pgfqpoint{2.825771in}{1.368441in}}%
\pgfpathlineto{\pgfqpoint{2.828589in}{1.347872in}}%
\pgfpathlineto{\pgfqpoint{2.831407in}{1.321171in}}%
\pgfpathlineto{\pgfqpoint{2.834226in}{1.320073in}}%
\pgfpathlineto{\pgfqpoint{2.837044in}{1.414953in}}%
\pgfpathlineto{\pgfqpoint{2.839862in}{1.382807in}}%
\pgfpathlineto{\pgfqpoint{2.845498in}{1.412608in}}%
\pgfpathlineto{\pgfqpoint{2.848317in}{1.386964in}}%
\pgfpathlineto{\pgfqpoint{2.851135in}{1.390477in}}%
\pgfpathlineto{\pgfqpoint{2.856771in}{1.331068in}}%
\pgfpathlineto{\pgfqpoint{2.865226in}{1.252282in}}%
\pgfpathlineto{\pgfqpoint{2.868044in}{1.238540in}}%
\pgfpathlineto{\pgfqpoint{2.873680in}{1.207397in}}%
\pgfpathlineto{\pgfqpoint{2.876498in}{1.210847in}}%
\pgfpathlineto{\pgfqpoint{2.879317in}{1.188017in}}%
\pgfpathlineto{\pgfqpoint{2.882135in}{1.190456in}}%
\pgfpathlineto{\pgfqpoint{2.884953in}{1.170247in}}%
\pgfpathlineto{\pgfqpoint{2.887771in}{1.162939in}}%
\pgfpathlineto{\pgfqpoint{2.890589in}{1.143510in}}%
\pgfpathlineto{\pgfqpoint{2.893407in}{1.141786in}}%
\pgfpathlineto{\pgfqpoint{2.896226in}{1.133667in}}%
\pgfpathlineto{\pgfqpoint{2.899044in}{1.119066in}}%
\pgfpathlineto{\pgfqpoint{2.901862in}{1.107726in}}%
\pgfpathlineto{\pgfqpoint{2.904680in}{1.091542in}}%
\pgfpathlineto{\pgfqpoint{2.907498in}{1.092022in}}%
\pgfpathlineto{\pgfqpoint{2.910317in}{1.090884in}}%
\pgfpathlineto{\pgfqpoint{2.913135in}{1.072430in}}%
\pgfpathlineto{\pgfqpoint{2.915953in}{1.100041in}}%
\pgfpathlineto{\pgfqpoint{2.918771in}{1.108404in}}%
\pgfpathlineto{\pgfqpoint{2.921589in}{1.091017in}}%
\pgfpathlineto{\pgfqpoint{2.924407in}{1.081284in}}%
\pgfpathlineto{\pgfqpoint{2.927226in}{1.067007in}}%
\pgfpathlineto{\pgfqpoint{2.930044in}{1.079374in}}%
\pgfpathlineto{\pgfqpoint{2.935680in}{1.053755in}}%
\pgfpathlineto{\pgfqpoint{2.938498in}{1.038172in}}%
\pgfpathlineto{\pgfqpoint{2.941317in}{1.027510in}}%
\pgfpathlineto{\pgfqpoint{2.944135in}{1.023324in}}%
\pgfpathlineto{\pgfqpoint{2.952589in}{0.984041in}}%
\pgfpathlineto{\pgfqpoint{2.955407in}{0.989844in}}%
\pgfpathlineto{\pgfqpoint{2.958226in}{0.985084in}}%
\pgfpathlineto{\pgfqpoint{2.961044in}{0.974220in}}%
\pgfpathlineto{\pgfqpoint{2.963862in}{0.966764in}}%
\pgfpathlineto{\pgfqpoint{2.966680in}{0.993613in}}%
\pgfpathlineto{\pgfqpoint{2.969498in}{1.031932in}}%
\pgfpathlineto{\pgfqpoint{2.972317in}{1.054499in}}%
\pgfpathlineto{\pgfqpoint{2.977953in}{1.071653in}}%
\pgfpathlineto{\pgfqpoint{2.980771in}{1.099748in}}%
\pgfpathlineto{\pgfqpoint{2.989226in}{1.061911in}}%
\pgfpathlineto{\pgfqpoint{2.992044in}{1.083553in}}%
\pgfpathlineto{\pgfqpoint{3.000498in}{1.046096in}}%
\pgfpathlineto{\pgfqpoint{3.003317in}{1.035234in}}%
\pgfpathlineto{\pgfqpoint{3.006135in}{1.035723in}}%
\pgfpathlineto{\pgfqpoint{3.008953in}{1.024244in}}%
\pgfpathlineto{\pgfqpoint{3.014589in}{1.009915in}}%
\pgfpathlineto{\pgfqpoint{3.017407in}{0.966154in}}%
\pgfpathlineto{\pgfqpoint{3.020226in}{1.101050in}}%
\pgfpathlineto{\pgfqpoint{3.023044in}{1.137505in}}%
\pgfpathlineto{\pgfqpoint{3.025862in}{1.122003in}}%
\pgfpathlineto{\pgfqpoint{3.028680in}{1.115003in}}%
\pgfpathlineto{\pgfqpoint{3.031498in}{1.101502in}}%
\pgfpathlineto{\pgfqpoint{3.034317in}{1.122357in}}%
\pgfpathlineto{\pgfqpoint{3.037135in}{1.110409in}}%
\pgfpathlineto{\pgfqpoint{3.039953in}{1.118442in}}%
\pgfpathlineto{\pgfqpoint{3.042771in}{1.148012in}}%
\pgfpathlineto{\pgfqpoint{3.048407in}{1.118918in}}%
\pgfpathlineto{\pgfqpoint{3.056862in}{1.079259in}}%
\pgfpathlineto{\pgfqpoint{3.059680in}{1.090860in}}%
\pgfpathlineto{\pgfqpoint{3.062498in}{1.096463in}}%
\pgfpathlineto{\pgfqpoint{3.065317in}{1.103513in}}%
\pgfpathlineto{\pgfqpoint{3.070953in}{1.078400in}}%
\pgfpathlineto{\pgfqpoint{3.073771in}{1.105856in}}%
\pgfpathlineto{\pgfqpoint{3.079407in}{1.081785in}}%
\pgfpathlineto{\pgfqpoint{3.085044in}{1.057955in}}%
\pgfpathlineto{\pgfqpoint{3.096317in}{1.011824in}}%
\pgfpathlineto{\pgfqpoint{3.101953in}{0.994698in}}%
\pgfpathlineto{\pgfqpoint{3.104771in}{0.990028in}}%
\pgfpathlineto{\pgfqpoint{3.107589in}{1.041326in}}%
\pgfpathlineto{\pgfqpoint{3.110407in}{1.031327in}}%
\pgfpathlineto{\pgfqpoint{3.113226in}{1.057117in}}%
\pgfpathlineto{\pgfqpoint{3.116044in}{1.053489in}}%
\pgfpathlineto{\pgfqpoint{3.118862in}{1.042063in}}%
\pgfpathlineto{\pgfqpoint{3.121680in}{1.035026in}}%
\pgfpathlineto{\pgfqpoint{3.124498in}{1.064935in}}%
\pgfpathlineto{\pgfqpoint{3.127317in}{1.150405in}}%
\pgfpathlineto{\pgfqpoint{3.130135in}{1.287217in}}%
\pgfpathlineto{\pgfqpoint{3.132953in}{1.282935in}}%
\pgfpathlineto{\pgfqpoint{3.135771in}{1.288865in}}%
\pgfpathlineto{\pgfqpoint{3.138589in}{1.298841in}}%
\pgfpathlineto{\pgfqpoint{3.141407in}{1.278659in}}%
\pgfpathlineto{\pgfqpoint{3.144226in}{1.264634in}}%
\pgfpathlineto{\pgfqpoint{3.147044in}{1.240969in}}%
\pgfpathlineto{\pgfqpoint{3.149862in}{1.310417in}}%
\pgfpathlineto{\pgfqpoint{3.152680in}{1.296690in}}%
\pgfpathlineto{\pgfqpoint{3.155498in}{1.274839in}}%
\pgfpathlineto{\pgfqpoint{3.158317in}{1.307855in}}%
\pgfpathlineto{\pgfqpoint{3.163953in}{1.281105in}}%
\pgfpathlineto{\pgfqpoint{3.169589in}{1.231398in}}%
\pgfpathlineto{\pgfqpoint{3.178044in}{1.168266in}}%
\pgfpathlineto{\pgfqpoint{3.180862in}{1.010379in}}%
\pgfpathlineto{\pgfqpoint{3.183680in}{1.213098in}}%
\pgfpathlineto{\pgfqpoint{3.186498in}{1.191467in}}%
\pgfpathlineto{\pgfqpoint{3.189317in}{1.214011in}}%
\pgfpathlineto{\pgfqpoint{3.192135in}{1.192184in}}%
\pgfpathlineto{\pgfqpoint{3.194953in}{1.183777in}}%
\pgfpathlineto{\pgfqpoint{3.203407in}{1.129019in}}%
\pgfpathlineto{\pgfqpoint{3.206226in}{1.120156in}}%
\pgfpathlineto{\pgfqpoint{3.211862in}{1.092005in}}%
\pgfpathlineto{\pgfqpoint{3.214680in}{1.082327in}}%
\pgfpathlineto{\pgfqpoint{3.223135in}{1.044755in}}%
\pgfpathlineto{\pgfqpoint{3.225953in}{1.051975in}}%
\pgfpathlineto{\pgfqpoint{3.228771in}{1.036596in}}%
\pgfpathlineto{\pgfqpoint{3.231589in}{1.026415in}}%
\pgfpathlineto{\pgfqpoint{3.234407in}{1.021275in}}%
\pgfpathlineto{\pgfqpoint{3.240044in}{0.994819in}}%
\pgfpathlineto{\pgfqpoint{3.242862in}{0.987538in}}%
\pgfpathlineto{\pgfqpoint{3.245680in}{0.978117in}}%
\pgfpathlineto{\pgfqpoint{3.248498in}{0.965268in}}%
\pgfpathlineto{\pgfqpoint{3.251317in}{0.969330in}}%
\pgfpathlineto{\pgfqpoint{3.254135in}{0.962428in}}%
\pgfpathlineto{\pgfqpoint{3.256953in}{0.964333in}}%
\pgfpathlineto{\pgfqpoint{3.262589in}{0.943502in}}%
\pgfpathlineto{\pgfqpoint{3.265407in}{0.992132in}}%
\pgfpathlineto{\pgfqpoint{3.268226in}{0.984745in}}%
\pgfpathlineto{\pgfqpoint{3.276680in}{0.952349in}}%
\pgfpathlineto{\pgfqpoint{3.282317in}{0.938676in}}%
\pgfpathlineto{\pgfqpoint{3.285135in}{0.962241in}}%
\pgfpathlineto{\pgfqpoint{3.287953in}{0.954664in}}%
\pgfpathlineto{\pgfqpoint{3.290771in}{0.961878in}}%
\pgfpathlineto{\pgfqpoint{3.293589in}{0.979512in}}%
\pgfpathlineto{\pgfqpoint{3.296407in}{0.980540in}}%
\pgfpathlineto{\pgfqpoint{3.302044in}{0.961544in}}%
\pgfpathlineto{\pgfqpoint{3.304862in}{0.953294in}}%
\pgfpathlineto{\pgfqpoint{3.307680in}{0.951363in}}%
\pgfpathlineto{\pgfqpoint{3.310498in}{0.942845in}}%
\pgfpathlineto{\pgfqpoint{3.313317in}{0.937819in}}%
\pgfpathlineto{\pgfqpoint{3.316135in}{0.930830in}}%
\pgfpathlineto{\pgfqpoint{3.324589in}{0.905074in}}%
\pgfpathlineto{\pgfqpoint{3.327407in}{0.897627in}}%
\pgfpathlineto{\pgfqpoint{3.330226in}{0.898198in}}%
\pgfpathlineto{\pgfqpoint{3.333044in}{0.963598in}}%
\pgfpathlineto{\pgfqpoint{3.335862in}{0.956992in}}%
\pgfpathlineto{\pgfqpoint{3.338680in}{0.947946in}}%
\pgfpathlineto{\pgfqpoint{3.341498in}{0.945715in}}%
\pgfpathlineto{\pgfqpoint{3.347135in}{0.928802in}}%
\pgfpathlineto{\pgfqpoint{3.349953in}{0.949407in}}%
\pgfpathlineto{\pgfqpoint{3.352771in}{0.940606in}}%
\pgfpathlineto{\pgfqpoint{3.355589in}{0.933854in}}%
\pgfpathlineto{\pgfqpoint{3.358407in}{0.925467in}}%
\pgfpathlineto{\pgfqpoint{3.361226in}{0.934566in}}%
\pgfpathlineto{\pgfqpoint{3.364044in}{1.045564in}}%
\pgfpathlineto{\pgfqpoint{3.372498in}{1.011068in}}%
\pgfpathlineto{\pgfqpoint{3.380953in}{0.980803in}}%
\pgfpathlineto{\pgfqpoint{3.383771in}{0.993131in}}%
\pgfpathlineto{\pgfqpoint{3.386589in}{1.013678in}}%
\pgfpathlineto{\pgfqpoint{3.389407in}{1.027159in}}%
\pgfpathlineto{\pgfqpoint{3.392226in}{1.018735in}}%
\pgfpathlineto{\pgfqpoint{3.395044in}{1.073594in}}%
\pgfpathlineto{\pgfqpoint{3.397862in}{1.260736in}}%
\pgfpathlineto{\pgfqpoint{3.400680in}{1.251773in}}%
\pgfpathlineto{\pgfqpoint{3.406317in}{1.220612in}}%
\pgfpathlineto{\pgfqpoint{3.409135in}{1.290479in}}%
\pgfpathlineto{\pgfqpoint{3.411953in}{1.280180in}}%
\pgfpathlineto{\pgfqpoint{3.414771in}{2.898246in}}%
\pgfpathlineto{\pgfqpoint{3.417589in}{2.632489in}}%
\pgfpathlineto{\pgfqpoint{3.423226in}{2.315929in}}%
\pgfpathlineto{\pgfqpoint{3.426044in}{2.204390in}}%
\pgfpathlineto{\pgfqpoint{3.431680in}{2.055265in}}%
\pgfpathlineto{\pgfqpoint{3.437317in}{1.904553in}}%
\pgfpathlineto{\pgfqpoint{3.440135in}{1.906113in}}%
\pgfpathlineto{\pgfqpoint{3.442953in}{1.851775in}}%
\pgfpathlineto{\pgfqpoint{3.445771in}{1.989085in}}%
\pgfpathlineto{\pgfqpoint{3.451407in}{1.883005in}}%
\pgfpathlineto{\pgfqpoint{3.454226in}{1.935824in}}%
\pgfpathlineto{\pgfqpoint{3.459862in}{1.828797in}}%
\pgfpathlineto{\pgfqpoint{3.462680in}{1.844690in}}%
\pgfpathlineto{\pgfqpoint{3.465498in}{1.791110in}}%
\pgfpathlineto{\pgfqpoint{3.471135in}{1.704316in}}%
\pgfpathlineto{\pgfqpoint{3.473953in}{1.677976in}}%
\pgfpathlineto{\pgfqpoint{3.476771in}{1.635611in}}%
\pgfpathlineto{\pgfqpoint{3.479589in}{1.685356in}}%
\pgfpathlineto{\pgfqpoint{3.482407in}{1.657583in}}%
\pgfpathlineto{\pgfqpoint{3.485226in}{1.671711in}}%
\pgfpathlineto{\pgfqpoint{3.496498in}{1.534351in}}%
\pgfpathlineto{\pgfqpoint{3.499317in}{1.517458in}}%
\pgfpathlineto{\pgfqpoint{3.510589in}{1.372738in}}%
\pgfpathlineto{\pgfqpoint{3.513407in}{1.367084in}}%
\pgfpathlineto{\pgfqpoint{3.519044in}{1.306904in}}%
\pgfpathlineto{\pgfqpoint{3.524680in}{1.278563in}}%
\pgfpathlineto{\pgfqpoint{3.527498in}{1.239343in}}%
\pgfpathlineto{\pgfqpoint{3.530317in}{1.181999in}}%
\pgfpathlineto{\pgfqpoint{3.533135in}{1.137791in}}%
\pgfpathlineto{\pgfqpoint{3.535953in}{1.176052in}}%
\pgfpathlineto{\pgfqpoint{3.538771in}{1.169087in}}%
\pgfpathlineto{\pgfqpoint{3.541589in}{1.116617in}}%
\pgfpathlineto{\pgfqpoint{3.544407in}{1.108760in}}%
\pgfpathlineto{\pgfqpoint{3.550044in}{1.015466in}}%
\pgfpathlineto{\pgfqpoint{3.552862in}{1.019288in}}%
\pgfpathlineto{\pgfqpoint{3.555680in}{1.000598in}}%
\pgfpathlineto{\pgfqpoint{3.558498in}{1.014222in}}%
\pgfpathlineto{\pgfqpoint{3.561317in}{0.982546in}}%
\pgfpathlineto{\pgfqpoint{3.564135in}{1.015848in}}%
\pgfpathlineto{\pgfqpoint{3.566953in}{0.988022in}}%
\pgfpathlineto{\pgfqpoint{3.569771in}{0.974325in}}%
\pgfpathlineto{\pgfqpoint{3.572589in}{1.020290in}}%
\pgfpathlineto{\pgfqpoint{3.578226in}{0.983364in}}%
\pgfpathlineto{\pgfqpoint{3.583862in}{0.955237in}}%
\pgfpathlineto{\pgfqpoint{3.589498in}{0.975788in}}%
\pgfpathlineto{\pgfqpoint{3.595135in}{0.950073in}}%
\pgfpathlineto{\pgfqpoint{3.600771in}{0.928392in}}%
\pgfpathlineto{\pgfqpoint{3.603589in}{0.940439in}}%
\pgfpathlineto{\pgfqpoint{3.606407in}{0.941830in}}%
\pgfpathlineto{\pgfqpoint{3.612044in}{0.920853in}}%
\pgfpathlineto{\pgfqpoint{3.614862in}{0.991131in}}%
\pgfpathlineto{\pgfqpoint{3.617680in}{1.112563in}}%
\pgfpathlineto{\pgfqpoint{3.620498in}{1.089463in}}%
\pgfpathlineto{\pgfqpoint{3.626135in}{1.074625in}}%
\pgfpathlineto{\pgfqpoint{3.631771in}{1.036148in}}%
\pgfpathlineto{\pgfqpoint{3.634589in}{1.031809in}}%
\pgfpathlineto{\pgfqpoint{3.640226in}{1.002423in}}%
\pgfpathlineto{\pgfqpoint{3.643044in}{1.006214in}}%
\pgfpathlineto{\pgfqpoint{3.648680in}{0.978624in}}%
\pgfpathlineto{\pgfqpoint{3.651498in}{1.000402in}}%
\pgfpathlineto{\pgfqpoint{3.654317in}{0.986332in}}%
\pgfpathlineto{\pgfqpoint{3.657135in}{0.999851in}}%
\pgfpathlineto{\pgfqpoint{3.662771in}{0.973113in}}%
\pgfpathlineto{\pgfqpoint{3.668407in}{0.952009in}}%
\pgfpathlineto{\pgfqpoint{3.676862in}{0.920252in}}%
\pgfpathlineto{\pgfqpoint{3.682498in}{0.901317in}}%
\pgfpathlineto{\pgfqpoint{3.685317in}{0.897288in}}%
\pgfpathlineto{\pgfqpoint{3.693771in}{0.872659in}}%
\pgfpathlineto{\pgfqpoint{3.702226in}{0.850743in}}%
\pgfpathlineto{\pgfqpoint{3.705044in}{0.844471in}}%
\pgfpathlineto{\pgfqpoint{3.707862in}{0.898827in}}%
\pgfpathlineto{\pgfqpoint{3.713498in}{0.880844in}}%
\pgfpathlineto{\pgfqpoint{3.719135in}{0.866440in}}%
\pgfpathlineto{\pgfqpoint{3.721953in}{0.859191in}}%
\pgfpathlineto{\pgfqpoint{3.724771in}{0.857245in}}%
\pgfpathlineto{\pgfqpoint{3.727589in}{1.239925in}}%
\pgfpathlineto{\pgfqpoint{3.730407in}{1.389664in}}%
\pgfpathlineto{\pgfqpoint{3.733226in}{1.343367in}}%
\pgfpathlineto{\pgfqpoint{3.738862in}{1.292385in}}%
\pgfpathlineto{\pgfqpoint{3.741680in}{1.238974in}}%
\pgfpathlineto{\pgfqpoint{3.747317in}{1.159731in}}%
\pgfpathlineto{\pgfqpoint{3.752953in}{1.100038in}}%
\pgfpathlineto{\pgfqpoint{3.758589in}{1.049024in}}%
\pgfpathlineto{\pgfqpoint{3.764226in}{1.005519in}}%
\pgfpathlineto{\pgfqpoint{3.767044in}{0.990465in}}%
\pgfpathlineto{\pgfqpoint{3.772680in}{0.954643in}}%
\pgfpathlineto{\pgfqpoint{3.775498in}{0.941879in}}%
\pgfpathlineto{\pgfqpoint{3.778317in}{1.304471in}}%
\pgfpathlineto{\pgfqpoint{3.783953in}{1.215752in}}%
\pgfpathlineto{\pgfqpoint{3.786771in}{1.180795in}}%
\pgfpathlineto{\pgfqpoint{3.789589in}{1.162136in}}%
\pgfpathlineto{\pgfqpoint{3.795226in}{1.105645in}}%
\pgfpathlineto{\pgfqpoint{3.798044in}{0.951842in}}%
\pgfpathlineto{\pgfqpoint{3.800862in}{1.074651in}}%
\pgfpathlineto{\pgfqpoint{3.809317in}{1.014078in}}%
\pgfpathlineto{\pgfqpoint{3.812135in}{0.995909in}}%
\pgfpathlineto{\pgfqpoint{3.814953in}{0.982219in}}%
\pgfpathlineto{\pgfqpoint{3.817771in}{0.995689in}}%
\pgfpathlineto{\pgfqpoint{3.820589in}{0.978965in}}%
\pgfpathlineto{\pgfqpoint{3.829044in}{0.938854in}}%
\pgfpathlineto{\pgfqpoint{3.831862in}{0.934228in}}%
\pgfpathlineto{\pgfqpoint{3.840317in}{0.897592in}}%
\pgfpathlineto{\pgfqpoint{3.843135in}{0.885852in}}%
\pgfpathlineto{\pgfqpoint{3.845953in}{0.890489in}}%
\pgfpathlineto{\pgfqpoint{3.857226in}{0.855135in}}%
\pgfpathlineto{\pgfqpoint{3.865680in}{0.832189in}}%
\pgfpathlineto{\pgfqpoint{3.868498in}{0.831600in}}%
\pgfpathlineto{\pgfqpoint{3.871317in}{0.829604in}}%
\pgfpathlineto{\pgfqpoint{3.874135in}{0.823303in}}%
\pgfpathlineto{\pgfqpoint{3.876953in}{1.028253in}}%
\pgfpathlineto{\pgfqpoint{3.879771in}{1.011931in}}%
\pgfpathlineto{\pgfqpoint{3.882589in}{1.038378in}}%
\pgfpathlineto{\pgfqpoint{3.885407in}{1.009644in}}%
\pgfpathlineto{\pgfqpoint{3.888226in}{0.998167in}}%
\pgfpathlineto{\pgfqpoint{3.891044in}{0.990938in}}%
\pgfpathlineto{\pgfqpoint{3.893862in}{1.026690in}}%
\pgfpathlineto{\pgfqpoint{3.896680in}{0.998758in}}%
\pgfpathlineto{\pgfqpoint{3.902317in}{0.956095in}}%
\pgfpathlineto{\pgfqpoint{3.905135in}{0.966565in}}%
\pgfpathlineto{\pgfqpoint{3.907953in}{1.010832in}}%
\pgfpathlineto{\pgfqpoint{3.913589in}{0.968212in}}%
\pgfpathlineto{\pgfqpoint{3.922044in}{0.924338in}}%
\pgfpathlineto{\pgfqpoint{3.927680in}{0.902402in}}%
\pgfpathlineto{\pgfqpoint{3.933317in}{0.878676in}}%
\pgfpathlineto{\pgfqpoint{3.936135in}{0.870292in}}%
\pgfpathlineto{\pgfqpoint{3.938953in}{0.980688in}}%
\pgfpathlineto{\pgfqpoint{3.947407in}{0.936231in}}%
\pgfpathlineto{\pgfqpoint{3.950226in}{0.927199in}}%
\pgfpathlineto{\pgfqpoint{3.953044in}{0.914181in}}%
\pgfpathlineto{\pgfqpoint{3.955862in}{1.907285in}}%
\pgfpathlineto{\pgfqpoint{3.958680in}{1.733535in}}%
\pgfpathlineto{\pgfqpoint{3.964317in}{1.523740in}}%
\pgfpathlineto{\pgfqpoint{3.967135in}{1.449321in}}%
\pgfpathlineto{\pgfqpoint{3.972771in}{1.337440in}}%
\pgfpathlineto{\pgfqpoint{3.978407in}{1.249267in}}%
\pgfpathlineto{\pgfqpoint{3.981226in}{1.235928in}}%
\pgfpathlineto{\pgfqpoint{3.984044in}{1.200732in}}%
\pgfpathlineto{\pgfqpoint{3.986862in}{1.198649in}}%
\pgfpathlineto{\pgfqpoint{3.989680in}{1.186240in}}%
\pgfpathlineto{\pgfqpoint{3.992498in}{1.163270in}}%
\pgfpathlineto{\pgfqpoint{3.998135in}{1.107560in}}%
\pgfpathlineto{\pgfqpoint{4.000953in}{1.093815in}}%
\pgfpathlineto{\pgfqpoint{4.006589in}{1.045620in}}%
\pgfpathlineto{\pgfqpoint{4.009407in}{1.027004in}}%
\pgfpathlineto{\pgfqpoint{4.020680in}{0.936526in}}%
\pgfpathlineto{\pgfqpoint{4.023498in}{0.930675in}}%
\pgfpathlineto{\pgfqpoint{4.026317in}{0.934510in}}%
\pgfpathlineto{\pgfqpoint{4.031953in}{0.905078in}}%
\pgfpathlineto{\pgfqpoint{4.034771in}{0.903673in}}%
\pgfpathlineto{\pgfqpoint{4.037589in}{0.974785in}}%
\pgfpathlineto{\pgfqpoint{4.040407in}{1.135578in}}%
\pgfpathlineto{\pgfqpoint{4.046044in}{1.073307in}}%
\pgfpathlineto{\pgfqpoint{4.048862in}{1.047851in}}%
\pgfpathlineto{\pgfqpoint{4.060135in}{0.970447in}}%
\pgfpathlineto{\pgfqpoint{4.062953in}{0.959853in}}%
\pgfpathlineto{\pgfqpoint{4.065771in}{0.958546in}}%
\pgfpathlineto{\pgfqpoint{4.068589in}{0.944343in}}%
\pgfpathlineto{\pgfqpoint{4.071407in}{0.974297in}}%
\pgfpathlineto{\pgfqpoint{4.077044in}{0.946149in}}%
\pgfpathlineto{\pgfqpoint{4.079862in}{0.944069in}}%
\pgfpathlineto{\pgfqpoint{4.088317in}{0.908219in}}%
\pgfpathlineto{\pgfqpoint{4.091135in}{0.935242in}}%
\pgfpathlineto{\pgfqpoint{4.093953in}{0.922679in}}%
\pgfpathlineto{\pgfqpoint{4.096771in}{0.941947in}}%
\pgfpathlineto{\pgfqpoint{4.099589in}{0.929280in}}%
\pgfpathlineto{\pgfqpoint{4.102407in}{0.925361in}}%
\pgfpathlineto{\pgfqpoint{4.110862in}{0.893110in}}%
\pgfpathlineto{\pgfqpoint{4.116498in}{0.878602in}}%
\pgfpathlineto{\pgfqpoint{4.119317in}{0.883715in}}%
\pgfpathlineto{\pgfqpoint{4.124953in}{0.866195in}}%
\pgfpathlineto{\pgfqpoint{4.127771in}{0.859045in}}%
\pgfpathlineto{\pgfqpoint{4.130589in}{0.860839in}}%
\pgfpathlineto{\pgfqpoint{4.133407in}{0.855793in}}%
\pgfpathlineto{\pgfqpoint{4.136226in}{0.856062in}}%
\pgfpathlineto{\pgfqpoint{4.139044in}{0.859976in}}%
\pgfpathlineto{\pgfqpoint{4.141862in}{0.853408in}}%
\pgfpathlineto{\pgfqpoint{4.144680in}{0.860562in}}%
\pgfpathlineto{\pgfqpoint{4.147498in}{0.856038in}}%
\pgfpathlineto{\pgfqpoint{4.153135in}{0.970406in}}%
\pgfpathlineto{\pgfqpoint{4.155953in}{0.969587in}}%
\pgfpathlineto{\pgfqpoint{4.161589in}{0.933385in}}%
\pgfpathlineto{\pgfqpoint{4.164407in}{0.933372in}}%
\pgfpathlineto{\pgfqpoint{4.170044in}{0.906411in}}%
\pgfpathlineto{\pgfqpoint{4.172862in}{1.059135in}}%
\pgfpathlineto{\pgfqpoint{4.175680in}{1.032187in}}%
\pgfpathlineto{\pgfqpoint{4.184135in}{0.970644in}}%
\pgfpathlineto{\pgfqpoint{4.192589in}{0.926901in}}%
\pgfpathlineto{\pgfqpoint{4.195407in}{0.950965in}}%
\pgfpathlineto{\pgfqpoint{4.201044in}{0.924354in}}%
\pgfpathlineto{\pgfqpoint{4.203862in}{0.912573in}}%
\pgfpathlineto{\pgfqpoint{4.206680in}{0.919494in}}%
\pgfpathlineto{\pgfqpoint{4.209498in}{0.921564in}}%
\pgfpathlineto{\pgfqpoint{4.215135in}{0.898659in}}%
\pgfpathlineto{\pgfqpoint{4.217953in}{0.909612in}}%
\pgfpathlineto{\pgfqpoint{4.220771in}{0.898795in}}%
\pgfpathlineto{\pgfqpoint{4.223589in}{0.895377in}}%
\pgfpathlineto{\pgfqpoint{4.229226in}{0.875631in}}%
\pgfpathlineto{\pgfqpoint{4.232044in}{0.971171in}}%
\pgfpathlineto{\pgfqpoint{4.234862in}{0.952759in}}%
\pgfpathlineto{\pgfqpoint{4.243317in}{0.912349in}}%
\pgfpathlineto{\pgfqpoint{4.246135in}{0.901123in}}%
\pgfpathlineto{\pgfqpoint{4.248953in}{0.918757in}}%
\pgfpathlineto{\pgfqpoint{4.254589in}{0.899424in}}%
\pgfpathlineto{\pgfqpoint{4.257407in}{0.891987in}}%
\pgfpathlineto{\pgfqpoint{4.265862in}{0.865321in}}%
\pgfpathlineto{\pgfqpoint{4.268680in}{0.857240in}}%
\pgfpathlineto{\pgfqpoint{4.271498in}{0.851233in}}%
\pgfpathlineto{\pgfqpoint{4.277135in}{0.835845in}}%
\pgfpathlineto{\pgfqpoint{4.279953in}{0.829409in}}%
\pgfpathlineto{\pgfqpoint{4.282771in}{0.830244in}}%
\pgfpathlineto{\pgfqpoint{4.285589in}{0.822830in}}%
\pgfpathlineto{\pgfqpoint{4.288407in}{0.827489in}}%
\pgfpathlineto{\pgfqpoint{4.291226in}{0.820870in}}%
\pgfpathlineto{\pgfqpoint{4.294044in}{0.816982in}}%
\pgfpathlineto{\pgfqpoint{4.296862in}{0.838002in}}%
\pgfpathlineto{\pgfqpoint{4.299680in}{0.830709in}}%
\pgfpathlineto{\pgfqpoint{4.302498in}{0.825505in}}%
\pgfpathlineto{\pgfqpoint{4.305317in}{0.817723in}}%
\pgfpathlineto{\pgfqpoint{4.308135in}{0.812504in}}%
\pgfpathlineto{\pgfqpoint{4.310953in}{0.805213in}}%
\pgfpathlineto{\pgfqpoint{4.313771in}{0.804898in}}%
\pgfpathlineto{\pgfqpoint{4.319407in}{0.790477in}}%
\pgfpathlineto{\pgfqpoint{4.322226in}{0.793369in}}%
\pgfpathlineto{\pgfqpoint{4.325044in}{1.042355in}}%
\pgfpathlineto{\pgfqpoint{4.327862in}{1.000441in}}%
\pgfpathlineto{\pgfqpoint{4.330680in}{1.009201in}}%
\pgfpathlineto{\pgfqpoint{4.336317in}{0.963243in}}%
\pgfpathlineto{\pgfqpoint{4.339135in}{0.941826in}}%
\pgfpathlineto{\pgfqpoint{4.341953in}{0.940482in}}%
\pgfpathlineto{\pgfqpoint{4.344771in}{0.929502in}}%
\pgfpathlineto{\pgfqpoint{4.347589in}{0.937450in}}%
\pgfpathlineto{\pgfqpoint{4.350407in}{0.940224in}}%
\pgfpathlineto{\pgfqpoint{4.353226in}{0.929211in}}%
\pgfpathlineto{\pgfqpoint{4.358862in}{0.901288in}}%
\pgfpathlineto{\pgfqpoint{4.361680in}{0.978892in}}%
\pgfpathlineto{\pgfqpoint{4.367317in}{0.943483in}}%
\pgfpathlineto{\pgfqpoint{4.370135in}{0.932899in}}%
\pgfpathlineto{\pgfqpoint{4.372953in}{0.918334in}}%
\pgfpathlineto{\pgfqpoint{4.395498in}{0.827802in}}%
\pgfpathlineto{\pgfqpoint{4.398317in}{0.824492in}}%
\pgfpathlineto{\pgfqpoint{4.403953in}{0.798523in}}%
\pgfpathlineto{\pgfqpoint{4.406771in}{0.872993in}}%
\pgfpathlineto{\pgfqpoint{4.412407in}{0.833947in}}%
\pgfpathlineto{\pgfqpoint{4.415226in}{0.846875in}}%
\pgfpathlineto{\pgfqpoint{4.418044in}{0.841438in}}%
\pgfpathlineto{\pgfqpoint{4.420862in}{0.819908in}}%
\pgfpathlineto{\pgfqpoint{4.423680in}{0.814066in}}%
\pgfpathlineto{\pgfqpoint{4.426498in}{0.876799in}}%
\pgfpathlineto{\pgfqpoint{4.429317in}{0.878478in}}%
\pgfpathlineto{\pgfqpoint{4.432135in}{0.872546in}}%
\pgfpathlineto{\pgfqpoint{4.434953in}{0.860795in}}%
\pgfpathlineto{\pgfqpoint{4.437771in}{0.842834in}}%
\pgfpathlineto{\pgfqpoint{4.440589in}{0.868882in}}%
\pgfpathlineto{\pgfqpoint{4.443407in}{0.881955in}}%
\pgfpathlineto{\pgfqpoint{4.446226in}{0.937143in}}%
\pgfpathlineto{\pgfqpoint{4.449044in}{0.921566in}}%
\pgfpathlineto{\pgfqpoint{4.451862in}{0.921803in}}%
\pgfpathlineto{\pgfqpoint{4.454680in}{0.940586in}}%
\pgfpathlineto{\pgfqpoint{4.457498in}{1.003938in}}%
\pgfpathlineto{\pgfqpoint{4.460317in}{0.986288in}}%
\pgfpathlineto{\pgfqpoint{4.463135in}{0.978271in}}%
\pgfpathlineto{\pgfqpoint{4.471589in}{0.937702in}}%
\pgfpathlineto{\pgfqpoint{4.477226in}{0.925105in}}%
\pgfpathlineto{\pgfqpoint{4.482862in}{0.901113in}}%
\pgfpathlineto{\pgfqpoint{4.488498in}{0.880786in}}%
\pgfpathlineto{\pgfqpoint{4.491317in}{0.878059in}}%
\pgfpathlineto{\pgfqpoint{4.499771in}{0.850976in}}%
\pgfpathlineto{\pgfqpoint{4.502589in}{0.845957in}}%
\pgfpathlineto{\pgfqpoint{4.505407in}{0.837723in}}%
\pgfpathlineto{\pgfqpoint{4.508226in}{0.850225in}}%
\pgfpathlineto{\pgfqpoint{4.511044in}{0.844200in}}%
\pgfpathlineto{\pgfqpoint{4.513862in}{0.836578in}}%
\pgfpathlineto{\pgfqpoint{4.516680in}{0.833748in}}%
\pgfpathlineto{\pgfqpoint{4.525135in}{0.812445in}}%
\pgfpathlineto{\pgfqpoint{4.527953in}{0.828799in}}%
\pgfpathlineto{\pgfqpoint{4.530771in}{0.830608in}}%
\pgfpathlineto{\pgfqpoint{4.533589in}{0.829564in}}%
\pgfpathlineto{\pgfqpoint{4.536407in}{0.835937in}}%
\pgfpathlineto{\pgfqpoint{4.539226in}{0.832792in}}%
\pgfpathlineto{\pgfqpoint{4.542044in}{0.874275in}}%
\pgfpathlineto{\pgfqpoint{4.544862in}{0.867131in}}%
\pgfpathlineto{\pgfqpoint{4.547680in}{0.857706in}}%
\pgfpathlineto{\pgfqpoint{4.550498in}{0.858868in}}%
\pgfpathlineto{\pgfqpoint{4.558953in}{0.826540in}}%
\pgfpathlineto{\pgfqpoint{4.567407in}{0.796994in}}%
\pgfpathlineto{\pgfqpoint{4.570226in}{0.788993in}}%
\pgfpathlineto{\pgfqpoint{4.575862in}{0.769730in}}%
\pgfpathlineto{\pgfqpoint{4.578680in}{0.791128in}}%
\pgfpathlineto{\pgfqpoint{4.581498in}{0.789585in}}%
\pgfpathlineto{\pgfqpoint{4.587135in}{0.805038in}}%
\pgfpathlineto{\pgfqpoint{4.589953in}{0.801537in}}%
\pgfpathlineto{\pgfqpoint{4.592771in}{0.796028in}}%
\pgfpathlineto{\pgfqpoint{4.595589in}{0.788791in}}%
\pgfpathlineto{\pgfqpoint{4.598407in}{0.792729in}}%
\pgfpathlineto{\pgfqpoint{4.601226in}{0.785203in}}%
\pgfpathlineto{\pgfqpoint{4.604044in}{0.779950in}}%
\pgfpathlineto{\pgfqpoint{4.606862in}{0.776108in}}%
\pgfpathlineto{\pgfqpoint{4.615317in}{0.754651in}}%
\pgfpathlineto{\pgfqpoint{4.618135in}{0.752416in}}%
\pgfpathlineto{\pgfqpoint{4.620953in}{0.745089in}}%
\pgfpathlineto{\pgfqpoint{4.623771in}{0.742347in}}%
\pgfpathlineto{\pgfqpoint{4.626589in}{0.738382in}}%
\pgfpathlineto{\pgfqpoint{4.629407in}{0.762881in}}%
\pgfpathlineto{\pgfqpoint{4.632226in}{0.760404in}}%
\pgfpathlineto{\pgfqpoint{4.635044in}{0.754171in}}%
\pgfpathlineto{\pgfqpoint{4.637862in}{0.829827in}}%
\pgfpathlineto{\pgfqpoint{4.640680in}{0.742270in}}%
\pgfpathlineto{\pgfqpoint{4.643498in}{0.737632in}}%
\pgfpathlineto{\pgfqpoint{4.646317in}{0.736477in}}%
\pgfpathlineto{\pgfqpoint{4.649135in}{0.742644in}}%
\pgfpathlineto{\pgfqpoint{4.651953in}{0.737820in}}%
\pgfpathlineto{\pgfqpoint{4.654771in}{0.741047in}}%
\pgfpathlineto{\pgfqpoint{4.660407in}{0.727295in}}%
\pgfpathlineto{\pgfqpoint{4.663226in}{0.729888in}}%
\pgfpathlineto{\pgfqpoint{4.666044in}{0.726662in}}%
\pgfpathlineto{\pgfqpoint{4.668862in}{0.725012in}}%
\pgfpathlineto{\pgfqpoint{4.671680in}{0.725281in}}%
\pgfpathlineto{\pgfqpoint{4.674498in}{0.718994in}}%
\pgfpathlineto{\pgfqpoint{4.677317in}{0.716707in}}%
\pgfpathlineto{\pgfqpoint{4.680135in}{0.723615in}}%
\pgfpathlineto{\pgfqpoint{4.682953in}{0.891206in}}%
\pgfpathlineto{\pgfqpoint{4.685771in}{0.867513in}}%
\pgfpathlineto{\pgfqpoint{4.691407in}{0.969229in}}%
\pgfpathlineto{\pgfqpoint{4.694226in}{0.955759in}}%
\pgfpathlineto{\pgfqpoint{4.697044in}{0.988196in}}%
\pgfpathlineto{\pgfqpoint{4.699862in}{0.990353in}}%
\pgfpathlineto{\pgfqpoint{4.705498in}{0.949019in}}%
\pgfpathlineto{\pgfqpoint{4.708317in}{0.946592in}}%
\pgfpathlineto{\pgfqpoint{4.711135in}{0.970115in}}%
\pgfpathlineto{\pgfqpoint{4.716771in}{0.936887in}}%
\pgfpathlineto{\pgfqpoint{4.719589in}{0.935920in}}%
\pgfpathlineto{\pgfqpoint{4.722407in}{0.921764in}}%
\pgfpathlineto{\pgfqpoint{4.725226in}{0.969780in}}%
\pgfpathlineto{\pgfqpoint{4.728044in}{0.955647in}}%
\pgfpathlineto{\pgfqpoint{4.730862in}{0.945626in}}%
\pgfpathlineto{\pgfqpoint{4.733680in}{0.939167in}}%
\pgfpathlineto{\pgfqpoint{4.736498in}{0.926709in}}%
\pgfpathlineto{\pgfqpoint{4.739317in}{0.920011in}}%
\pgfpathlineto{\pgfqpoint{4.742135in}{0.910008in}}%
\pgfpathlineto{\pgfqpoint{4.744953in}{0.932945in}}%
\pgfpathlineto{\pgfqpoint{4.747771in}{0.942858in}}%
\pgfpathlineto{\pgfqpoint{4.753407in}{0.924570in}}%
\pgfpathlineto{\pgfqpoint{4.756226in}{0.969389in}}%
\pgfpathlineto{\pgfqpoint{4.759044in}{0.957328in}}%
\pgfpathlineto{\pgfqpoint{4.761862in}{0.954026in}}%
\pgfpathlineto{\pgfqpoint{4.767498in}{0.936218in}}%
\pgfpathlineto{\pgfqpoint{4.770317in}{0.932404in}}%
\pgfpathlineto{\pgfqpoint{4.775953in}{0.913330in}}%
\pgfpathlineto{\pgfqpoint{4.778771in}{0.942938in}}%
\pgfpathlineto{\pgfqpoint{4.781589in}{1.000532in}}%
\pgfpathlineto{\pgfqpoint{4.784407in}{1.004797in}}%
\pgfpathlineto{\pgfqpoint{4.787226in}{0.998721in}}%
\pgfpathlineto{\pgfqpoint{4.790044in}{1.016253in}}%
\pgfpathlineto{\pgfqpoint{4.792862in}{1.002371in}}%
\pgfpathlineto{\pgfqpoint{4.795680in}{1.008128in}}%
\pgfpathlineto{\pgfqpoint{4.801317in}{0.981783in}}%
\pgfpathlineto{\pgfqpoint{4.801317in}{0.981783in}}%
\pgfusepath{stroke}%
\end{pgfscope}%
\begin{pgfscope}%
\pgfpathrectangle{\pgfqpoint{0.373953in}{0.331635in}}{\pgfqpoint{4.650000in}{3.020000in}}%
\pgfusepath{clip}%
\pgfsetroundcap%
\pgfsetroundjoin%
\pgfsetlinewidth{1.505625pt}%
\definecolor{currentstroke}{rgb}{1.000000,0.498039,0.054902}%
\pgfsetstrokecolor{currentstroke}%
\pgfsetstrokeopacity{0.600000}%
\pgfsetdash{}{0pt}%
\pgfpathmoveto{\pgfqpoint{0.585317in}{1.158793in}}%
\pgfpathlineto{\pgfqpoint{0.588135in}{0.585512in}}%
\pgfpathlineto{\pgfqpoint{0.590953in}{0.860570in}}%
\pgfpathlineto{\pgfqpoint{0.593771in}{0.741365in}}%
\pgfpathlineto{\pgfqpoint{0.596589in}{1.298903in}}%
\pgfpathlineto{\pgfqpoint{0.599407in}{0.827214in}}%
\pgfpathlineto{\pgfqpoint{0.602226in}{0.673614in}}%
\pgfpathlineto{\pgfqpoint{0.605044in}{0.468908in}}%
\pgfpathlineto{\pgfqpoint{0.607862in}{0.632556in}}%
\pgfpathlineto{\pgfqpoint{0.610680in}{0.482587in}}%
\pgfpathlineto{\pgfqpoint{0.613498in}{0.482587in}}%
\pgfpathlineto{\pgfqpoint{0.616317in}{0.826701in}}%
\pgfpathlineto{\pgfqpoint{0.619135in}{0.620824in}}%
\pgfpathlineto{\pgfqpoint{0.621953in}{0.537861in}}%
\pgfpathlineto{\pgfqpoint{0.624771in}{0.961730in}}%
\pgfpathlineto{\pgfqpoint{0.627589in}{0.482484in}}%
\pgfpathlineto{\pgfqpoint{0.630407in}{0.564121in}}%
\pgfpathlineto{\pgfqpoint{0.633226in}{0.699598in}}%
\pgfpathlineto{\pgfqpoint{0.636044in}{0.482420in}}%
\pgfpathlineto{\pgfqpoint{0.638862in}{0.645068in}}%
\pgfpathlineto{\pgfqpoint{0.641680in}{0.468908in}}%
\pgfpathlineto{\pgfqpoint{0.644498in}{0.563897in}}%
\pgfpathlineto{\pgfqpoint{0.647317in}{0.686489in}}%
\pgfpathlineto{\pgfqpoint{0.650135in}{0.740628in}}%
\pgfpathlineto{\pgfqpoint{0.652953in}{1.031314in}}%
\pgfpathlineto{\pgfqpoint{0.655771in}{0.681866in}}%
\pgfpathlineto{\pgfqpoint{0.658589in}{0.926737in}}%
\pgfpathlineto{\pgfqpoint{0.661407in}{0.550462in}}%
\pgfpathlineto{\pgfqpoint{0.664226in}{0.509714in}}%
\pgfpathlineto{\pgfqpoint{0.669862in}{0.844362in}}%
\pgfpathlineto{\pgfqpoint{0.672680in}{0.844362in}}%
\pgfpathlineto{\pgfqpoint{0.675498in}{0.495914in}}%
\pgfpathlineto{\pgfqpoint{0.678317in}{0.536376in}}%
\pgfpathlineto{\pgfqpoint{0.681135in}{0.876108in}}%
\pgfpathlineto{\pgfqpoint{0.683953in}{0.537175in}}%
\pgfpathlineto{\pgfqpoint{0.686771in}{0.794359in}}%
\pgfpathlineto{\pgfqpoint{0.689589in}{0.549927in}}%
\pgfpathlineto{\pgfqpoint{0.692407in}{0.724690in}}%
\pgfpathlineto{\pgfqpoint{0.695226in}{0.562879in}}%
\pgfpathlineto{\pgfqpoint{0.698044in}{1.001807in}}%
\pgfpathlineto{\pgfqpoint{0.700862in}{0.495320in}}%
\pgfpathlineto{\pgfqpoint{0.703680in}{0.771196in}}%
\pgfpathlineto{\pgfqpoint{0.706498in}{0.495068in}}%
\pgfpathlineto{\pgfqpoint{0.709317in}{0.481991in}}%
\pgfpathlineto{\pgfqpoint{0.712135in}{0.481985in}}%
\pgfpathlineto{\pgfqpoint{0.714953in}{0.495044in}}%
\pgfpathlineto{\pgfqpoint{0.717771in}{0.599237in}}%
\pgfpathlineto{\pgfqpoint{0.720589in}{0.494926in}}%
\pgfpathlineto{\pgfqpoint{0.723407in}{0.792465in}}%
\pgfpathlineto{\pgfqpoint{0.726226in}{0.468908in}}%
\pgfpathlineto{\pgfqpoint{0.729044in}{0.870663in}}%
\pgfpathlineto{\pgfqpoint{0.731862in}{0.832027in}}%
\pgfpathlineto{\pgfqpoint{0.734680in}{0.507595in}}%
\pgfpathlineto{\pgfqpoint{0.737498in}{0.738636in}}%
\pgfpathlineto{\pgfqpoint{0.740317in}{0.751543in}}%
\pgfpathlineto{\pgfqpoint{0.743135in}{0.650221in}}%
\pgfpathlineto{\pgfqpoint{0.745953in}{0.611464in}}%
\pgfpathlineto{\pgfqpoint{0.748771in}{0.793925in}}%
\pgfpathlineto{\pgfqpoint{0.751589in}{0.784411in}}%
\pgfpathlineto{\pgfqpoint{0.754407in}{0.888816in}}%
\pgfpathlineto{\pgfqpoint{0.757226in}{1.113005in}}%
\pgfpathlineto{\pgfqpoint{0.760044in}{0.866549in}}%
\pgfpathlineto{\pgfqpoint{0.762862in}{0.790013in}}%
\pgfpathlineto{\pgfqpoint{0.765680in}{0.532702in}}%
\pgfpathlineto{\pgfqpoint{0.768498in}{1.009218in}}%
\pgfpathlineto{\pgfqpoint{0.771317in}{0.729803in}}%
\pgfpathlineto{\pgfqpoint{0.774135in}{0.560782in}}%
\pgfpathlineto{\pgfqpoint{0.776953in}{0.627091in}}%
\pgfpathlineto{\pgfqpoint{0.779771in}{0.614676in}}%
\pgfpathlineto{\pgfqpoint{0.782589in}{1.034585in}}%
\pgfpathlineto{\pgfqpoint{0.791044in}{0.481689in}}%
\pgfpathlineto{\pgfqpoint{0.793862in}{1.025751in}}%
\pgfpathlineto{\pgfqpoint{0.796680in}{0.693642in}}%
\pgfpathlineto{\pgfqpoint{0.799498in}{0.882268in}}%
\pgfpathlineto{\pgfqpoint{0.802317in}{1.132541in}}%
\pgfpathlineto{\pgfqpoint{0.805135in}{0.624249in}}%
\pgfpathlineto{\pgfqpoint{0.807953in}{0.874125in}}%
\pgfpathlineto{\pgfqpoint{0.810771in}{0.653794in}}%
\pgfpathlineto{\pgfqpoint{0.813589in}{0.521853in}}%
\pgfpathlineto{\pgfqpoint{0.816407in}{0.875972in}}%
\pgfpathlineto{\pgfqpoint{0.819226in}{1.114998in}}%
\pgfpathlineto{\pgfqpoint{0.822044in}{0.615961in}}%
\pgfpathlineto{\pgfqpoint{0.824862in}{0.873819in}}%
\pgfpathlineto{\pgfqpoint{0.827680in}{0.810735in}}%
\pgfpathlineto{\pgfqpoint{0.830498in}{0.606782in}}%
\pgfpathlineto{\pgfqpoint{0.833317in}{0.565489in}}%
\pgfpathlineto{\pgfqpoint{0.836135in}{0.606980in}}%
\pgfpathlineto{\pgfqpoint{0.838953in}{1.126861in}}%
\pgfpathlineto{\pgfqpoint{0.841771in}{0.708629in}}%
\pgfpathlineto{\pgfqpoint{0.844589in}{0.567040in}}%
\pgfpathlineto{\pgfqpoint{0.847407in}{0.858127in}}%
\pgfpathlineto{\pgfqpoint{0.850226in}{0.510300in}}%
\pgfpathlineto{\pgfqpoint{0.853044in}{0.787776in}}%
\pgfpathlineto{\pgfqpoint{0.855862in}{1.260365in}}%
\pgfpathlineto{\pgfqpoint{0.858680in}{0.583769in}}%
\pgfpathlineto{\pgfqpoint{0.864317in}{1.182516in}}%
\pgfpathlineto{\pgfqpoint{0.867135in}{0.607445in}}%
\pgfpathlineto{\pgfqpoint{0.869953in}{0.978203in}}%
\pgfpathlineto{\pgfqpoint{0.872771in}{0.922708in}}%
\pgfpathlineto{\pgfqpoint{0.875589in}{1.045241in}}%
\pgfpathlineto{\pgfqpoint{0.878407in}{0.496067in}}%
\pgfpathlineto{\pgfqpoint{0.881226in}{0.940093in}}%
\pgfpathlineto{\pgfqpoint{0.884044in}{0.853565in}}%
\pgfpathlineto{\pgfqpoint{0.886862in}{0.547855in}}%
\pgfpathlineto{\pgfqpoint{0.889680in}{0.560740in}}%
\pgfpathlineto{\pgfqpoint{0.892498in}{0.612627in}}%
\pgfpathlineto{\pgfqpoint{0.895317in}{1.463338in}}%
\pgfpathlineto{\pgfqpoint{0.898135in}{0.724094in}}%
\pgfpathlineto{\pgfqpoint{0.900953in}{1.434281in}}%
\pgfpathlineto{\pgfqpoint{0.903771in}{0.524269in}}%
\pgfpathlineto{\pgfqpoint{0.909407in}{0.898071in}}%
\pgfpathlineto{\pgfqpoint{0.912226in}{1.375051in}}%
\pgfpathlineto{\pgfqpoint{0.915044in}{0.482420in}}%
\pgfpathlineto{\pgfqpoint{0.917862in}{0.670841in}}%
\pgfpathlineto{\pgfqpoint{0.923498in}{0.906346in}}%
\pgfpathlineto{\pgfqpoint{0.926317in}{0.482679in}}%
\pgfpathlineto{\pgfqpoint{0.929135in}{1.151546in}}%
\pgfpathlineto{\pgfqpoint{0.931953in}{0.652763in}}%
\pgfpathlineto{\pgfqpoint{0.934771in}{1.216290in}}%
\pgfpathlineto{\pgfqpoint{0.937589in}{1.059798in}}%
\pgfpathlineto{\pgfqpoint{0.940407in}{0.611914in}}%
\pgfpathlineto{\pgfqpoint{0.943226in}{0.754214in}}%
\pgfpathlineto{\pgfqpoint{0.946044in}{1.390866in}}%
\pgfpathlineto{\pgfqpoint{0.948862in}{0.634357in}}%
\pgfpathlineto{\pgfqpoint{0.951680in}{1.069694in}}%
\pgfpathlineto{\pgfqpoint{0.954498in}{0.766931in}}%
\pgfpathlineto{\pgfqpoint{0.957317in}{0.755634in}}%
\pgfpathlineto{\pgfqpoint{0.960135in}{0.612627in}}%
\pgfpathlineto{\pgfqpoint{0.962953in}{0.895815in}}%
\pgfpathlineto{\pgfqpoint{0.965771in}{1.042304in}}%
\pgfpathlineto{\pgfqpoint{0.971407in}{0.524915in}}%
\pgfpathlineto{\pgfqpoint{0.974226in}{0.720088in}}%
\pgfpathlineto{\pgfqpoint{0.977044in}{0.496709in}}%
\pgfpathlineto{\pgfqpoint{0.979862in}{1.074418in}}%
\pgfpathlineto{\pgfqpoint{0.982680in}{0.564077in}}%
\pgfpathlineto{\pgfqpoint{0.985498in}{0.671789in}}%
\pgfpathlineto{\pgfqpoint{0.988317in}{0.576530in}}%
\pgfpathlineto{\pgfqpoint{0.991135in}{0.576129in}}%
\pgfpathlineto{\pgfqpoint{0.993953in}{0.668879in}}%
\pgfpathlineto{\pgfqpoint{0.996771in}{0.668879in}}%
\pgfpathlineto{\pgfqpoint{0.999589in}{0.697232in}}%
\pgfpathlineto{\pgfqpoint{1.002407in}{0.699056in}}%
\pgfpathlineto{\pgfqpoint{1.005226in}{0.806728in}}%
\pgfpathlineto{\pgfqpoint{1.008044in}{0.752410in}}%
\pgfpathlineto{\pgfqpoint{1.010862in}{0.577646in}}%
\pgfpathlineto{\pgfqpoint{1.013680in}{0.591731in}}%
\pgfpathlineto{\pgfqpoint{1.019317in}{1.256887in}}%
\pgfpathlineto{\pgfqpoint{1.022135in}{0.611914in}}%
\pgfpathlineto{\pgfqpoint{1.024953in}{0.555053in}}%
\pgfpathlineto{\pgfqpoint{1.027771in}{0.655232in}}%
\pgfpathlineto{\pgfqpoint{1.030589in}{0.842770in}}%
\pgfpathlineto{\pgfqpoint{1.033407in}{0.950519in}}%
\pgfpathlineto{\pgfqpoint{1.036226in}{1.123673in}}%
\pgfpathlineto{\pgfqpoint{1.039044in}{0.947609in}}%
\pgfpathlineto{\pgfqpoint{1.041862in}{0.498177in}}%
\pgfpathlineto{\pgfqpoint{1.044680in}{1.291651in}}%
\pgfpathlineto{\pgfqpoint{1.047498in}{1.526883in}}%
\pgfpathlineto{\pgfqpoint{1.050317in}{1.597900in}}%
\pgfpathlineto{\pgfqpoint{1.053135in}{0.560873in}}%
\pgfpathlineto{\pgfqpoint{1.055953in}{0.653132in}}%
\pgfpathlineto{\pgfqpoint{1.058771in}{0.683722in}}%
\pgfpathlineto{\pgfqpoint{1.061589in}{0.484191in}}%
\pgfpathlineto{\pgfqpoint{1.064407in}{0.545403in}}%
\pgfpathlineto{\pgfqpoint{1.067226in}{0.545403in}}%
\pgfpathlineto{\pgfqpoint{1.070044in}{0.745246in}}%
\pgfpathlineto{\pgfqpoint{1.072862in}{0.499743in}}%
\pgfpathlineto{\pgfqpoint{1.075680in}{0.530612in}}%
\pgfpathlineto{\pgfqpoint{1.078498in}{0.873236in}}%
\pgfpathlineto{\pgfqpoint{1.081317in}{0.799674in}}%
\pgfpathlineto{\pgfqpoint{1.084135in}{0.611130in}}%
\pgfpathlineto{\pgfqpoint{1.086953in}{1.015416in}}%
\pgfpathlineto{\pgfqpoint{1.089771in}{1.015416in}}%
\pgfpathlineto{\pgfqpoint{1.092589in}{0.594741in}}%
\pgfpathlineto{\pgfqpoint{1.095407in}{0.578563in}}%
\pgfpathlineto{\pgfqpoint{1.098226in}{0.846618in}}%
\pgfpathlineto{\pgfqpoint{1.101044in}{0.564114in}}%
\pgfpathlineto{\pgfqpoint{1.103862in}{0.737846in}}%
\pgfpathlineto{\pgfqpoint{1.106680in}{0.690203in}}%
\pgfpathlineto{\pgfqpoint{1.109498in}{1.257107in}}%
\pgfpathlineto{\pgfqpoint{1.112317in}{0.648853in}}%
\pgfpathlineto{\pgfqpoint{1.115135in}{0.584006in}}%
\pgfpathlineto{\pgfqpoint{1.117953in}{0.733730in}}%
\pgfpathlineto{\pgfqpoint{1.120771in}{0.799556in}}%
\pgfpathlineto{\pgfqpoint{1.123589in}{1.279302in}}%
\pgfpathlineto{\pgfqpoint{1.126407in}{1.197229in}}%
\pgfpathlineto{\pgfqpoint{1.129226in}{0.633288in}}%
\pgfpathlineto{\pgfqpoint{1.132044in}{1.001315in}}%
\pgfpathlineto{\pgfqpoint{1.134862in}{0.736341in}}%
\pgfpathlineto{\pgfqpoint{1.137680in}{0.634802in}}%
\pgfpathlineto{\pgfqpoint{1.140498in}{0.634802in}}%
\pgfpathlineto{\pgfqpoint{1.143317in}{0.766833in}}%
\pgfpathlineto{\pgfqpoint{1.146135in}{0.812638in}}%
\pgfpathlineto{\pgfqpoint{1.148953in}{0.895087in}}%
\pgfpathlineto{\pgfqpoint{1.151771in}{1.299014in}}%
\pgfpathlineto{\pgfqpoint{1.154589in}{0.726768in}}%
\pgfpathlineto{\pgfqpoint{1.157407in}{0.468908in}}%
\pgfpathlineto{\pgfqpoint{1.160226in}{0.614975in}}%
\pgfpathlineto{\pgfqpoint{1.163044in}{1.563307in}}%
\pgfpathlineto{\pgfqpoint{1.165862in}{0.587456in}}%
\pgfpathlineto{\pgfqpoint{1.168680in}{0.502869in}}%
\pgfpathlineto{\pgfqpoint{1.171498in}{0.519924in}}%
\pgfpathlineto{\pgfqpoint{1.174317in}{1.157900in}}%
\pgfpathlineto{\pgfqpoint{1.177135in}{0.925752in}}%
\pgfpathlineto{\pgfqpoint{1.179953in}{0.574978in}}%
\pgfpathlineto{\pgfqpoint{1.182771in}{0.697396in}}%
\pgfpathlineto{\pgfqpoint{1.185589in}{0.556308in}}%
\pgfpathlineto{\pgfqpoint{1.188407in}{1.553738in}}%
\pgfpathlineto{\pgfqpoint{1.191226in}{0.687210in}}%
\pgfpathlineto{\pgfqpoint{1.194044in}{1.011593in}}%
\pgfpathlineto{\pgfqpoint{1.196862in}{0.700934in}}%
\pgfpathlineto{\pgfqpoint{1.199680in}{0.539926in}}%
\pgfpathlineto{\pgfqpoint{1.202498in}{0.698516in}}%
\pgfpathlineto{\pgfqpoint{1.205317in}{1.289721in}}%
\pgfpathlineto{\pgfqpoint{1.208135in}{0.523158in}}%
\pgfpathlineto{\pgfqpoint{1.210953in}{0.505063in}}%
\pgfpathlineto{\pgfqpoint{1.213771in}{1.095167in}}%
\pgfpathlineto{\pgfqpoint{1.216589in}{0.646454in}}%
\pgfpathlineto{\pgfqpoint{1.219407in}{0.910740in}}%
\pgfpathlineto{\pgfqpoint{1.222226in}{0.468908in}}%
\pgfpathlineto{\pgfqpoint{1.225044in}{0.574331in}}%
\pgfpathlineto{\pgfqpoint{1.227862in}{1.267467in}}%
\pgfpathlineto{\pgfqpoint{1.230680in}{0.503131in}}%
\pgfpathlineto{\pgfqpoint{1.233498in}{0.726576in}}%
\pgfpathlineto{\pgfqpoint{1.236317in}{0.537845in}}%
\pgfpathlineto{\pgfqpoint{1.239135in}{0.520503in}}%
\pgfpathlineto{\pgfqpoint{1.241953in}{1.013580in}}%
\pgfpathlineto{\pgfqpoint{1.244771in}{0.653786in}}%
\pgfpathlineto{\pgfqpoint{1.247589in}{0.569603in}}%
\pgfpathlineto{\pgfqpoint{1.250407in}{0.824140in}}%
\pgfpathlineto{\pgfqpoint{1.256044in}{0.468908in}}%
\pgfpathlineto{\pgfqpoint{1.264498in}{1.022740in}}%
\pgfpathlineto{\pgfqpoint{1.267317in}{1.512582in}}%
\pgfpathlineto{\pgfqpoint{1.270135in}{0.551688in}}%
\pgfpathlineto{\pgfqpoint{1.272953in}{0.685255in}}%
\pgfpathlineto{\pgfqpoint{1.275771in}{0.585604in}}%
\pgfpathlineto{\pgfqpoint{1.278589in}{0.686008in}}%
\pgfpathlineto{\pgfqpoint{1.281407in}{0.967462in}}%
\pgfpathlineto{\pgfqpoint{1.284226in}{0.926569in}}%
\pgfpathlineto{\pgfqpoint{1.287044in}{0.727202in}}%
\pgfpathlineto{\pgfqpoint{1.289862in}{0.468908in}}%
\pgfpathlineto{\pgfqpoint{1.295498in}{0.758192in}}%
\pgfpathlineto{\pgfqpoint{1.298317in}{1.210849in}}%
\pgfpathlineto{\pgfqpoint{1.301135in}{2.352538in}}%
\pgfpathlineto{\pgfqpoint{1.303953in}{0.568905in}}%
\pgfpathlineto{\pgfqpoint{1.306771in}{0.552504in}}%
\pgfpathlineto{\pgfqpoint{1.309589in}{0.687401in}}%
\pgfpathlineto{\pgfqpoint{1.312407in}{0.485774in}}%
\pgfpathlineto{\pgfqpoint{1.315226in}{0.586695in}}%
\pgfpathlineto{\pgfqpoint{1.318044in}{0.786210in}}%
\pgfpathlineto{\pgfqpoint{1.320862in}{0.584933in}}%
\pgfpathlineto{\pgfqpoint{1.323680in}{0.919032in}}%
\pgfpathlineto{\pgfqpoint{1.326498in}{0.902486in}}%
\pgfpathlineto{\pgfqpoint{1.329317in}{0.734952in}}%
\pgfpathlineto{\pgfqpoint{1.332135in}{0.801081in}}%
\pgfpathlineto{\pgfqpoint{1.334953in}{0.734339in}}%
\pgfpathlineto{\pgfqpoint{1.337771in}{0.703183in}}%
\pgfpathlineto{\pgfqpoint{1.340589in}{0.719844in}}%
\pgfpathlineto{\pgfqpoint{1.343407in}{0.518834in}}%
\pgfpathlineto{\pgfqpoint{1.346226in}{0.684260in}}%
\pgfpathlineto{\pgfqpoint{1.349044in}{0.551309in}}%
\pgfpathlineto{\pgfqpoint{1.351862in}{0.501840in}}%
\pgfpathlineto{\pgfqpoint{1.354680in}{0.617570in}}%
\pgfpathlineto{\pgfqpoint{1.357498in}{0.502048in}}%
\pgfpathlineto{\pgfqpoint{1.360317in}{0.970645in}}%
\pgfpathlineto{\pgfqpoint{1.363135in}{1.151777in}}%
\pgfpathlineto{\pgfqpoint{1.365953in}{0.709757in}}%
\pgfpathlineto{\pgfqpoint{1.368771in}{0.727038in}}%
\pgfpathlineto{\pgfqpoint{1.371589in}{0.965891in}}%
\pgfpathlineto{\pgfqpoint{1.374407in}{0.940755in}}%
\pgfpathlineto{\pgfqpoint{1.377226in}{0.535843in}}%
\pgfpathlineto{\pgfqpoint{1.380044in}{0.685882in}}%
\pgfpathlineto{\pgfqpoint{1.382862in}{0.799556in}}%
\pgfpathlineto{\pgfqpoint{1.385680in}{0.779548in}}%
\pgfpathlineto{\pgfqpoint{1.388498in}{0.792320in}}%
\pgfpathlineto{\pgfqpoint{1.391317in}{0.661227in}}%
\pgfpathlineto{\pgfqpoint{1.394135in}{0.887234in}}%
\pgfpathlineto{\pgfqpoint{1.396953in}{0.614401in}}%
\pgfpathlineto{\pgfqpoint{1.399771in}{0.533481in}}%
\pgfpathlineto{\pgfqpoint{1.402589in}{0.501212in}}%
\pgfpathlineto{\pgfqpoint{1.405407in}{0.485056in}}%
\pgfpathlineto{\pgfqpoint{1.408226in}{0.826468in}}%
\pgfpathlineto{\pgfqpoint{1.411044in}{0.632913in}}%
\pgfpathlineto{\pgfqpoint{1.413862in}{0.633852in}}%
\pgfpathlineto{\pgfqpoint{1.416680in}{0.518490in}}%
\pgfpathlineto{\pgfqpoint{1.419498in}{0.651129in}}%
\pgfpathlineto{\pgfqpoint{1.422317in}{0.848611in}}%
\pgfpathlineto{\pgfqpoint{1.425135in}{0.699435in}}%
\pgfpathlineto{\pgfqpoint{1.427953in}{1.299953in}}%
\pgfpathlineto{\pgfqpoint{1.430771in}{0.549107in}}%
\pgfpathlineto{\pgfqpoint{1.433589in}{0.468908in}}%
\pgfpathlineto{\pgfqpoint{1.436407in}{1.002398in}}%
\pgfpathlineto{\pgfqpoint{1.439226in}{0.501523in}}%
\pgfpathlineto{\pgfqpoint{1.442044in}{1.028454in}}%
\pgfpathlineto{\pgfqpoint{1.444862in}{0.601546in}}%
\pgfpathlineto{\pgfqpoint{1.447680in}{0.751499in}}%
\pgfpathlineto{\pgfqpoint{1.450498in}{0.685255in}}%
\pgfpathlineto{\pgfqpoint{1.453317in}{1.350591in}}%
\pgfpathlineto{\pgfqpoint{1.456135in}{1.136483in}}%
\pgfpathlineto{\pgfqpoint{1.458953in}{1.024141in}}%
\pgfpathlineto{\pgfqpoint{1.461771in}{0.677893in}}%
\pgfpathlineto{\pgfqpoint{1.464589in}{0.858754in}}%
\pgfpathlineto{\pgfqpoint{1.467407in}{1.179645in}}%
\pgfpathlineto{\pgfqpoint{1.470226in}{0.484859in}}%
\pgfpathlineto{\pgfqpoint{1.473044in}{0.864843in}}%
\pgfpathlineto{\pgfqpoint{1.475862in}{0.735070in}}%
\pgfpathlineto{\pgfqpoint{1.478680in}{1.024558in}}%
\pgfpathlineto{\pgfqpoint{1.481498in}{1.088950in}}%
\pgfpathlineto{\pgfqpoint{1.484317in}{0.810982in}}%
\pgfpathlineto{\pgfqpoint{1.487135in}{0.810982in}}%
\pgfpathlineto{\pgfqpoint{1.489953in}{0.498815in}}%
\pgfpathlineto{\pgfqpoint{1.492771in}{0.913833in}}%
\pgfpathlineto{\pgfqpoint{1.495589in}{0.688820in}}%
\pgfpathlineto{\pgfqpoint{1.498407in}{0.527269in}}%
\pgfpathlineto{\pgfqpoint{1.501226in}{0.614295in}}%
\pgfpathlineto{\pgfqpoint{1.504044in}{0.555789in}}%
\pgfpathlineto{\pgfqpoint{1.506862in}{0.526800in}}%
\pgfpathlineto{\pgfqpoint{1.509680in}{0.757212in}}%
\pgfpathlineto{\pgfqpoint{1.512498in}{0.583426in}}%
\pgfpathlineto{\pgfqpoint{1.515317in}{0.597200in}}%
\pgfpathlineto{\pgfqpoint{1.518135in}{0.639080in}}%
\pgfpathlineto{\pgfqpoint{1.520953in}{0.968079in}}%
\pgfpathlineto{\pgfqpoint{1.523771in}{0.758079in}}%
\pgfpathlineto{\pgfqpoint{1.526589in}{0.901584in}}%
\pgfpathlineto{\pgfqpoint{1.529407in}{0.785567in}}%
\pgfpathlineto{\pgfqpoint{1.532226in}{0.584460in}}%
\pgfpathlineto{\pgfqpoint{1.535044in}{0.526510in}}%
\pgfpathlineto{\pgfqpoint{1.537862in}{0.555096in}}%
\pgfpathlineto{\pgfqpoint{1.540680in}{0.569486in}}%
\pgfpathlineto{\pgfqpoint{1.543498in}{0.497709in}}%
\pgfpathlineto{\pgfqpoint{1.546317in}{0.627832in}}%
\pgfpathlineto{\pgfqpoint{1.549135in}{0.671023in}}%
\pgfpathlineto{\pgfqpoint{1.551953in}{0.698174in}}%
\pgfpathlineto{\pgfqpoint{1.554771in}{0.540358in}}%
\pgfpathlineto{\pgfqpoint{1.557589in}{0.540358in}}%
\pgfpathlineto{\pgfqpoint{1.560407in}{1.588649in}}%
\pgfpathlineto{\pgfqpoint{1.563226in}{0.605865in}}%
\pgfpathlineto{\pgfqpoint{1.566044in}{1.230357in}}%
\pgfpathlineto{\pgfqpoint{1.568862in}{0.525079in}}%
\pgfpathlineto{\pgfqpoint{1.571680in}{0.539276in}}%
\pgfpathlineto{\pgfqpoint{1.574498in}{0.957914in}}%
\pgfpathlineto{\pgfqpoint{1.577317in}{0.817349in}}%
\pgfpathlineto{\pgfqpoint{1.580135in}{0.764890in}}%
\pgfpathlineto{\pgfqpoint{1.582953in}{1.085610in}}%
\pgfpathlineto{\pgfqpoint{1.585771in}{0.538327in}}%
\pgfpathlineto{\pgfqpoint{1.588589in}{0.676664in}}%
\pgfpathlineto{\pgfqpoint{1.591407in}{0.879981in}}%
\pgfpathlineto{\pgfqpoint{1.594226in}{1.004455in}}%
\pgfpathlineto{\pgfqpoint{1.597044in}{1.412846in}}%
\pgfpathlineto{\pgfqpoint{1.599862in}{0.483226in}}%
\pgfpathlineto{\pgfqpoint{1.602680in}{1.223061in}}%
\pgfpathlineto{\pgfqpoint{1.605498in}{0.833946in}}%
\pgfpathlineto{\pgfqpoint{1.608317in}{0.628071in}}%
\pgfpathlineto{\pgfqpoint{1.611135in}{0.512164in}}%
\pgfpathlineto{\pgfqpoint{1.616771in}{0.872133in}}%
\pgfpathlineto{\pgfqpoint{1.619589in}{0.672656in}}%
\pgfpathlineto{\pgfqpoint{1.622407in}{0.512690in}}%
\pgfpathlineto{\pgfqpoint{1.625226in}{0.831192in}}%
\pgfpathlineto{\pgfqpoint{1.628044in}{1.524869in}}%
\pgfpathlineto{\pgfqpoint{1.630862in}{0.528722in}}%
\pgfpathlineto{\pgfqpoint{1.633680in}{0.603317in}}%
\pgfpathlineto{\pgfqpoint{1.636498in}{1.363741in}}%
\pgfpathlineto{\pgfqpoint{1.639317in}{0.570193in}}%
\pgfpathlineto{\pgfqpoint{1.642135in}{0.512359in}}%
\pgfpathlineto{\pgfqpoint{1.644953in}{0.842770in}}%
\pgfpathlineto{\pgfqpoint{1.647771in}{0.583426in}}%
\pgfpathlineto{\pgfqpoint{1.650589in}{1.580154in}}%
\pgfpathlineto{\pgfqpoint{1.653407in}{0.723452in}}%
\pgfpathlineto{\pgfqpoint{1.656226in}{0.801691in}}%
\pgfpathlineto{\pgfqpoint{1.659044in}{0.499352in}}%
\pgfpathlineto{\pgfqpoint{1.661862in}{0.696461in}}%
\pgfpathlineto{\pgfqpoint{1.664680in}{0.634646in}}%
\pgfpathlineto{\pgfqpoint{1.667498in}{0.723188in}}%
\pgfpathlineto{\pgfqpoint{1.670317in}{0.498707in}}%
\pgfpathlineto{\pgfqpoint{1.673135in}{0.647242in}}%
\pgfpathlineto{\pgfqpoint{1.675953in}{0.528230in}}%
\pgfpathlineto{\pgfqpoint{1.678771in}{0.632676in}}%
\pgfpathlineto{\pgfqpoint{1.681589in}{0.483834in}}%
\pgfpathlineto{\pgfqpoint{1.687226in}{0.618990in}}%
\pgfpathlineto{\pgfqpoint{1.690044in}{0.604653in}}%
\pgfpathlineto{\pgfqpoint{1.692862in}{0.788157in}}%
\pgfpathlineto{\pgfqpoint{1.695680in}{0.545241in}}%
\pgfpathlineto{\pgfqpoint{1.698498in}{0.636132in}}%
\pgfpathlineto{\pgfqpoint{1.701317in}{0.620890in}}%
\pgfpathlineto{\pgfqpoint{1.704135in}{0.544999in}}%
\pgfpathlineto{\pgfqpoint{1.706953in}{1.176644in}}%
\pgfpathlineto{\pgfqpoint{1.709771in}{0.625091in}}%
\pgfpathlineto{\pgfqpoint{1.712589in}{0.950346in}}%
\pgfpathlineto{\pgfqpoint{1.715407in}{0.775298in}}%
\pgfpathlineto{\pgfqpoint{1.718226in}{0.914243in}}%
\pgfpathlineto{\pgfqpoint{1.721044in}{0.577034in}}%
\pgfpathlineto{\pgfqpoint{1.723862in}{0.714541in}}%
\pgfpathlineto{\pgfqpoint{1.726680in}{0.699128in}}%
\pgfpathlineto{\pgfqpoint{1.729498in}{0.561513in}}%
\pgfpathlineto{\pgfqpoint{1.732317in}{0.592316in}}%
\pgfpathlineto{\pgfqpoint{1.735135in}{0.762869in}}%
\pgfpathlineto{\pgfqpoint{1.737953in}{0.577970in}}%
\pgfpathlineto{\pgfqpoint{1.740771in}{0.593516in}}%
\pgfpathlineto{\pgfqpoint{1.743589in}{1.190298in}}%
\pgfpathlineto{\pgfqpoint{1.746407in}{0.514418in}}%
\pgfpathlineto{\pgfqpoint{1.749226in}{0.559857in}}%
\pgfpathlineto{\pgfqpoint{1.752044in}{0.770015in}}%
\pgfpathlineto{\pgfqpoint{1.754862in}{0.558630in}}%
\pgfpathlineto{\pgfqpoint{1.757680in}{0.678696in}}%
\pgfpathlineto{\pgfqpoint{1.760498in}{0.695388in}}%
\pgfpathlineto{\pgfqpoint{1.763317in}{1.481657in}}%
\pgfpathlineto{\pgfqpoint{1.766135in}{0.542174in}}%
\pgfpathlineto{\pgfqpoint{1.768953in}{0.905739in}}%
\pgfpathlineto{\pgfqpoint{1.771771in}{0.656164in}}%
\pgfpathlineto{\pgfqpoint{1.774589in}{0.483262in}}%
\pgfpathlineto{\pgfqpoint{1.777407in}{0.641630in}}%
\pgfpathlineto{\pgfqpoint{1.780226in}{0.642671in}}%
\pgfpathlineto{\pgfqpoint{1.785862in}{0.468908in}}%
\pgfpathlineto{\pgfqpoint{1.788680in}{0.818547in}}%
\pgfpathlineto{\pgfqpoint{1.791498in}{1.001452in}}%
\pgfpathlineto{\pgfqpoint{1.794317in}{0.543657in}}%
\pgfpathlineto{\pgfqpoint{1.797135in}{0.588911in}}%
\pgfpathlineto{\pgfqpoint{1.799953in}{0.543968in}}%
\pgfpathlineto{\pgfqpoint{1.802771in}{0.830995in}}%
\pgfpathlineto{\pgfqpoint{1.805589in}{0.740899in}}%
\pgfpathlineto{\pgfqpoint{1.808407in}{0.483943in}}%
\pgfpathlineto{\pgfqpoint{1.811226in}{0.908157in}}%
\pgfpathlineto{\pgfqpoint{1.814044in}{0.605944in}}%
\pgfpathlineto{\pgfqpoint{1.816862in}{1.114791in}}%
\pgfpathlineto{\pgfqpoint{1.819680in}{0.661359in}}%
\pgfpathlineto{\pgfqpoint{1.822498in}{0.704019in}}%
\pgfpathlineto{\pgfqpoint{1.825317in}{0.542174in}}%
\pgfpathlineto{\pgfqpoint{1.828135in}{0.847876in}}%
\pgfpathlineto{\pgfqpoint{1.830953in}{0.742728in}}%
\pgfpathlineto{\pgfqpoint{1.833771in}{0.511970in}}%
\pgfpathlineto{\pgfqpoint{1.836589in}{0.769000in}}%
\pgfpathlineto{\pgfqpoint{1.839407in}{0.554330in}}%
\pgfpathlineto{\pgfqpoint{1.842226in}{0.856472in}}%
\pgfpathlineto{\pgfqpoint{1.845044in}{0.512229in}}%
\pgfpathlineto{\pgfqpoint{1.847862in}{1.026234in}}%
\pgfpathlineto{\pgfqpoint{1.850680in}{0.610092in}}%
\pgfpathlineto{\pgfqpoint{1.853498in}{0.651420in}}%
\pgfpathlineto{\pgfqpoint{1.856317in}{0.482906in}}%
\pgfpathlineto{\pgfqpoint{1.859135in}{0.510944in}}%
\pgfpathlineto{\pgfqpoint{1.861953in}{0.637668in}}%
\pgfpathlineto{\pgfqpoint{1.867589in}{0.468908in}}%
\pgfpathlineto{\pgfqpoint{1.870407in}{0.736439in}}%
\pgfpathlineto{\pgfqpoint{1.873226in}{0.694032in}}%
\pgfpathlineto{\pgfqpoint{1.876044in}{0.497145in}}%
\pgfpathlineto{\pgfqpoint{1.878862in}{0.567519in}}%
\pgfpathlineto{\pgfqpoint{1.881680in}{0.776651in}}%
\pgfpathlineto{\pgfqpoint{1.884498in}{0.663047in}}%
\pgfpathlineto{\pgfqpoint{1.887317in}{0.704819in}}%
\pgfpathlineto{\pgfqpoint{1.892953in}{0.551950in}}%
\pgfpathlineto{\pgfqpoint{1.898589in}{0.634516in}}%
\pgfpathlineto{\pgfqpoint{1.901407in}{1.247205in}}%
\pgfpathlineto{\pgfqpoint{1.904226in}{2.075582in}}%
\pgfpathlineto{\pgfqpoint{1.907044in}{0.611701in}}%
\pgfpathlineto{\pgfqpoint{1.909862in}{0.583654in}}%
\pgfpathlineto{\pgfqpoint{1.912680in}{0.769149in}}%
\pgfpathlineto{\pgfqpoint{1.915498in}{0.540110in}}%
\pgfpathlineto{\pgfqpoint{1.918317in}{0.640516in}}%
\pgfpathlineto{\pgfqpoint{1.921135in}{0.655977in}}%
\pgfpathlineto{\pgfqpoint{1.923953in}{0.698975in}}%
\pgfpathlineto{\pgfqpoint{1.926771in}{1.106293in}}%
\pgfpathlineto{\pgfqpoint{1.929589in}{1.119903in}}%
\pgfpathlineto{\pgfqpoint{1.932407in}{0.591906in}}%
\pgfpathlineto{\pgfqpoint{1.935226in}{0.591906in}}%
\pgfpathlineto{\pgfqpoint{1.938044in}{0.938364in}}%
\pgfpathlineto{\pgfqpoint{1.940862in}{0.524537in}}%
\pgfpathlineto{\pgfqpoint{1.943680in}{0.678072in}}%
\pgfpathlineto{\pgfqpoint{1.946498in}{0.664175in}}%
\pgfpathlineto{\pgfqpoint{1.949317in}{0.704252in}}%
\pgfpathlineto{\pgfqpoint{1.952135in}{1.042238in}}%
\pgfpathlineto{\pgfqpoint{1.954953in}{0.482427in}}%
\pgfpathlineto{\pgfqpoint{1.957771in}{0.563719in}}%
\pgfpathlineto{\pgfqpoint{1.960589in}{0.765841in}}%
\pgfpathlineto{\pgfqpoint{1.963407in}{0.602872in}}%
\pgfpathlineto{\pgfqpoint{1.966226in}{0.774683in}}%
\pgfpathlineto{\pgfqpoint{1.969044in}{0.588173in}}%
\pgfpathlineto{\pgfqpoint{1.971862in}{0.746429in}}%
\pgfpathlineto{\pgfqpoint{1.974680in}{0.600127in}}%
\pgfpathlineto{\pgfqpoint{1.977498in}{0.665962in}}%
\pgfpathlineto{\pgfqpoint{1.980317in}{0.482093in}}%
\pgfpathlineto{\pgfqpoint{1.983135in}{0.827201in}}%
\pgfpathlineto{\pgfqpoint{1.985953in}{0.615417in}}%
\pgfpathlineto{\pgfqpoint{1.988771in}{0.642136in}}%
\pgfpathlineto{\pgfqpoint{1.991589in}{0.602872in}}%
\pgfpathlineto{\pgfqpoint{1.994407in}{0.709597in}}%
\pgfpathlineto{\pgfqpoint{1.997226in}{0.495552in}}%
\pgfpathlineto{\pgfqpoint{2.000044in}{1.580108in}}%
\pgfpathlineto{\pgfqpoint{2.002862in}{0.647888in}}%
\pgfpathlineto{\pgfqpoint{2.005680in}{0.609441in}}%
\pgfpathlineto{\pgfqpoint{2.008498in}{1.234604in}}%
\pgfpathlineto{\pgfqpoint{2.011317in}{0.759711in}}%
\pgfpathlineto{\pgfqpoint{2.014135in}{0.776100in}}%
\pgfpathlineto{\pgfqpoint{2.016953in}{0.684555in}}%
\pgfpathlineto{\pgfqpoint{2.019771in}{0.836512in}}%
\pgfpathlineto{\pgfqpoint{2.022589in}{0.564983in}}%
\pgfpathlineto{\pgfqpoint{2.025407in}{0.674393in}}%
\pgfpathlineto{\pgfqpoint{2.028226in}{0.729437in}}%
\pgfpathlineto{\pgfqpoint{2.031044in}{0.537697in}}%
\pgfpathlineto{\pgfqpoint{2.033862in}{0.703464in}}%
\pgfpathlineto{\pgfqpoint{2.036680in}{1.170055in}}%
\pgfpathlineto{\pgfqpoint{2.039498in}{0.625477in}}%
\pgfpathlineto{\pgfqpoint{2.042317in}{0.554669in}}%
\pgfpathlineto{\pgfqpoint{2.045135in}{1.035886in}}%
\pgfpathlineto{\pgfqpoint{2.047953in}{0.734364in}}%
\pgfpathlineto{\pgfqpoint{2.050771in}{0.566094in}}%
\pgfpathlineto{\pgfqpoint{2.053589in}{0.676067in}}%
\pgfpathlineto{\pgfqpoint{2.056407in}{0.565075in}}%
\pgfpathlineto{\pgfqpoint{2.059226in}{0.646657in}}%
\pgfpathlineto{\pgfqpoint{2.062044in}{0.536982in}}%
\pgfpathlineto{\pgfqpoint{2.064862in}{0.468908in}}%
\pgfpathlineto{\pgfqpoint{2.067680in}{0.769644in}}%
\pgfpathlineto{\pgfqpoint{2.070498in}{0.742433in}}%
\pgfpathlineto{\pgfqpoint{2.073317in}{0.838784in}}%
\pgfpathlineto{\pgfqpoint{2.076135in}{0.716019in}}%
\pgfpathlineto{\pgfqpoint{2.078953in}{0.482581in}}%
\pgfpathlineto{\pgfqpoint{2.081771in}{0.673324in}}%
\pgfpathlineto{\pgfqpoint{2.084589in}{0.509675in}}%
\pgfpathlineto{\pgfqpoint{2.087407in}{0.468908in}}%
\pgfpathlineto{\pgfqpoint{2.090226in}{0.769644in}}%
\pgfpathlineto{\pgfqpoint{2.093044in}{0.592316in}}%
\pgfpathlineto{\pgfqpoint{2.095862in}{0.592316in}}%
\pgfpathlineto{\pgfqpoint{2.098680in}{0.619668in}}%
\pgfpathlineto{\pgfqpoint{2.101498in}{0.482574in}}%
\pgfpathlineto{\pgfqpoint{2.104317in}{0.700251in}}%
\pgfpathlineto{\pgfqpoint{2.109953in}{0.509733in}}%
\pgfpathlineto{\pgfqpoint{2.112771in}{0.687313in}}%
\pgfpathlineto{\pgfqpoint{2.115589in}{0.551238in}}%
\pgfpathlineto{\pgfqpoint{2.118407in}{0.701570in}}%
\pgfpathlineto{\pgfqpoint{2.121226in}{0.604895in}}%
\pgfpathlineto{\pgfqpoint{2.124044in}{0.698624in}}%
\pgfpathlineto{\pgfqpoint{2.126862in}{0.936164in}}%
\pgfpathlineto{\pgfqpoint{2.129680in}{0.761726in}}%
\pgfpathlineto{\pgfqpoint{2.132498in}{0.522469in}}%
\pgfpathlineto{\pgfqpoint{2.135317in}{0.854987in}}%
\pgfpathlineto{\pgfqpoint{2.138135in}{0.561627in}}%
\pgfpathlineto{\pgfqpoint{2.140953in}{0.548619in}}%
\pgfpathlineto{\pgfqpoint{2.143771in}{0.825886in}}%
\pgfpathlineto{\pgfqpoint{2.146589in}{0.678382in}}%
\pgfpathlineto{\pgfqpoint{2.149407in}{0.663917in}}%
\pgfpathlineto{\pgfqpoint{2.152226in}{0.572748in}}%
\pgfpathlineto{\pgfqpoint{2.155044in}{1.035455in}}%
\pgfpathlineto{\pgfqpoint{2.157862in}{0.737916in}}%
\pgfpathlineto{\pgfqpoint{2.163498in}{0.922907in}}%
\pgfpathlineto{\pgfqpoint{2.166317in}{0.543884in}}%
\pgfpathlineto{\pgfqpoint{2.169135in}{1.061507in}}%
\pgfpathlineto{\pgfqpoint{2.171953in}{0.750539in}}%
\pgfpathlineto{\pgfqpoint{2.174771in}{0.890328in}}%
\pgfpathlineto{\pgfqpoint{2.177589in}{0.696202in}}%
\pgfpathlineto{\pgfqpoint{2.180407in}{0.644461in}}%
\pgfpathlineto{\pgfqpoint{2.183226in}{0.581201in}}%
\pgfpathlineto{\pgfqpoint{2.186044in}{0.568703in}}%
\pgfpathlineto{\pgfqpoint{2.188862in}{0.581592in}}%
\pgfpathlineto{\pgfqpoint{2.191680in}{0.569442in}}%
\pgfpathlineto{\pgfqpoint{2.194498in}{0.468908in}}%
\pgfpathlineto{\pgfqpoint{2.197317in}{1.002547in}}%
\pgfpathlineto{\pgfqpoint{2.200135in}{0.545750in}}%
\pgfpathlineto{\pgfqpoint{2.202953in}{0.634699in}}%
\pgfpathlineto{\pgfqpoint{2.205771in}{0.696901in}}%
\pgfpathlineto{\pgfqpoint{2.208589in}{0.722346in}}%
\pgfpathlineto{\pgfqpoint{2.211407in}{0.519864in}}%
\pgfpathlineto{\pgfqpoint{2.214226in}{0.481661in}}%
\pgfpathlineto{\pgfqpoint{2.217044in}{0.494408in}}%
\pgfpathlineto{\pgfqpoint{2.219862in}{0.659447in}}%
\pgfpathlineto{\pgfqpoint{2.222680in}{0.506915in}}%
\pgfpathlineto{\pgfqpoint{2.225498in}{0.506965in}}%
\pgfpathlineto{\pgfqpoint{2.228317in}{0.544972in}}%
\pgfpathlineto{\pgfqpoint{2.231135in}{0.557397in}}%
\pgfpathlineto{\pgfqpoint{2.233953in}{0.595404in}}%
\pgfpathlineto{\pgfqpoint{2.236771in}{0.826079in}}%
\pgfpathlineto{\pgfqpoint{2.239589in}{0.675011in}}%
\pgfpathlineto{\pgfqpoint{2.242407in}{0.713492in}}%
\pgfpathlineto{\pgfqpoint{2.245226in}{0.468908in}}%
\pgfpathlineto{\pgfqpoint{2.248044in}{0.635067in}}%
\pgfpathlineto{\pgfqpoint{2.250862in}{0.519932in}}%
\pgfpathlineto{\pgfqpoint{2.253680in}{0.621709in}}%
\pgfpathlineto{\pgfqpoint{2.256498in}{0.922510in}}%
\pgfpathlineto{\pgfqpoint{2.259317in}{0.655821in}}%
\pgfpathlineto{\pgfqpoint{2.262135in}{0.555721in}}%
\pgfpathlineto{\pgfqpoint{2.264953in}{0.850253in}}%
\pgfpathlineto{\pgfqpoint{2.267771in}{0.530075in}}%
\pgfpathlineto{\pgfqpoint{2.270589in}{0.882305in}}%
\pgfpathlineto{\pgfqpoint{2.273407in}{0.529331in}}%
\pgfpathlineto{\pgfqpoint{2.276226in}{0.529331in}}%
\pgfpathlineto{\pgfqpoint{2.279044in}{0.995237in}}%
\pgfpathlineto{\pgfqpoint{2.281862in}{0.730871in}}%
\pgfpathlineto{\pgfqpoint{2.284680in}{0.801904in}}%
\pgfpathlineto{\pgfqpoint{2.287498in}{0.777990in}}%
\pgfpathlineto{\pgfqpoint{2.290317in}{0.492801in}}%
\pgfpathlineto{\pgfqpoint{2.293135in}{2.374869in}}%
\pgfpathlineto{\pgfqpoint{2.295953in}{0.602727in}}%
\pgfpathlineto{\pgfqpoint{2.298771in}{0.524590in}}%
\pgfpathlineto{\pgfqpoint{2.301589in}{0.491211in}}%
\pgfpathlineto{\pgfqpoint{2.304407in}{0.624669in}}%
\pgfpathlineto{\pgfqpoint{2.307226in}{0.579652in}}%
\pgfpathlineto{\pgfqpoint{2.310044in}{0.491006in}}%
\pgfpathlineto{\pgfqpoint{2.312862in}{0.732770in}}%
\pgfpathlineto{\pgfqpoint{2.315680in}{0.817007in}}%
\pgfpathlineto{\pgfqpoint{2.318498in}{0.555539in}}%
\pgfpathlineto{\pgfqpoint{2.321317in}{0.609552in}}%
\pgfpathlineto{\pgfqpoint{2.324135in}{0.501303in}}%
\pgfpathlineto{\pgfqpoint{2.326953in}{0.544440in}}%
\pgfpathlineto{\pgfqpoint{2.329771in}{0.468908in}}%
\pgfpathlineto{\pgfqpoint{2.332589in}{0.619382in}}%
\pgfpathlineto{\pgfqpoint{2.335407in}{0.576309in}}%
\pgfpathlineto{\pgfqpoint{2.338226in}{0.684919in}}%
\pgfpathlineto{\pgfqpoint{2.341044in}{0.555507in}}%
\pgfpathlineto{\pgfqpoint{2.343862in}{0.812724in}}%
\pgfpathlineto{\pgfqpoint{2.346680in}{0.479590in}}%
\pgfpathlineto{\pgfqpoint{2.349498in}{0.692410in}}%
\pgfpathlineto{\pgfqpoint{2.352317in}{0.669641in}}%
\pgfpathlineto{\pgfqpoint{2.355135in}{3.026169in}}%
\pgfpathlineto{\pgfqpoint{2.357953in}{1.269939in}}%
\pgfpathlineto{\pgfqpoint{2.360771in}{0.468908in}}%
\pgfpathlineto{\pgfqpoint{2.366407in}{1.079865in}}%
\pgfpathlineto{\pgfqpoint{2.369226in}{0.709803in}}%
\pgfpathlineto{\pgfqpoint{2.372044in}{0.682574in}}%
\pgfpathlineto{\pgfqpoint{2.374862in}{0.468908in}}%
\pgfpathlineto{\pgfqpoint{2.377680in}{0.478658in}}%
\pgfpathlineto{\pgfqpoint{2.380498in}{0.507940in}}%
\pgfpathlineto{\pgfqpoint{2.383317in}{0.605291in}}%
\pgfpathlineto{\pgfqpoint{2.386135in}{0.860299in}}%
\pgfpathlineto{\pgfqpoint{2.388953in}{0.597263in}}%
\pgfpathlineto{\pgfqpoint{2.391771in}{0.734855in}}%
\pgfpathlineto{\pgfqpoint{2.394589in}{1.334788in}}%
\pgfpathlineto{\pgfqpoint{2.397407in}{0.489108in}}%
\pgfpathlineto{\pgfqpoint{2.400226in}{0.620756in}}%
\pgfpathlineto{\pgfqpoint{2.403044in}{0.529872in}}%
\pgfpathlineto{\pgfqpoint{2.405862in}{0.832773in}}%
\pgfpathlineto{\pgfqpoint{2.408680in}{0.569171in}}%
\pgfpathlineto{\pgfqpoint{2.411498in}{1.297626in}}%
\pgfpathlineto{\pgfqpoint{2.414317in}{0.605383in}}%
\pgfpathlineto{\pgfqpoint{2.417135in}{0.663675in}}%
\pgfpathlineto{\pgfqpoint{2.419953in}{0.672006in}}%
\pgfpathlineto{\pgfqpoint{2.422771in}{0.526790in}}%
\pgfpathlineto{\pgfqpoint{2.425589in}{0.603786in}}%
\pgfpathlineto{\pgfqpoint{2.428407in}{1.002195in}}%
\pgfpathlineto{\pgfqpoint{2.431226in}{0.639247in}}%
\pgfpathlineto{\pgfqpoint{2.434044in}{0.573502in}}%
\pgfpathlineto{\pgfqpoint{2.436862in}{0.478435in}}%
\pgfpathlineto{\pgfqpoint{2.439680in}{0.478438in}}%
\pgfpathlineto{\pgfqpoint{2.445317in}{0.688185in}}%
\pgfpathlineto{\pgfqpoint{2.448135in}{0.755254in}}%
\pgfpathlineto{\pgfqpoint{2.450953in}{0.468908in}}%
\pgfpathlineto{\pgfqpoint{2.453771in}{0.745755in}}%
\pgfpathlineto{\pgfqpoint{2.456589in}{0.544814in}}%
\pgfpathlineto{\pgfqpoint{2.459407in}{0.742402in}}%
\pgfpathlineto{\pgfqpoint{2.462226in}{0.818308in}}%
\pgfpathlineto{\pgfqpoint{2.465044in}{0.563759in}}%
\pgfpathlineto{\pgfqpoint{2.467862in}{0.468908in}}%
\pgfpathlineto{\pgfqpoint{2.470680in}{0.802261in}}%
\pgfpathlineto{\pgfqpoint{2.473498in}{0.536044in}}%
\pgfpathlineto{\pgfqpoint{2.476317in}{0.793542in}}%
\pgfpathlineto{\pgfqpoint{2.479135in}{0.506861in}}%
\pgfpathlineto{\pgfqpoint{2.484771in}{0.762706in}}%
\pgfpathlineto{\pgfqpoint{2.487589in}{0.554768in}}%
\pgfpathlineto{\pgfqpoint{2.490407in}{0.708760in}}%
\pgfpathlineto{\pgfqpoint{2.493226in}{0.737409in}}%
\pgfpathlineto{\pgfqpoint{2.496044in}{0.979275in}}%
\pgfpathlineto{\pgfqpoint{2.498862in}{0.575579in}}%
\pgfpathlineto{\pgfqpoint{2.501680in}{0.996419in}}%
\pgfpathlineto{\pgfqpoint{2.504498in}{0.535512in}}%
\pgfpathlineto{\pgfqpoint{2.507317in}{1.163064in}}%
\pgfpathlineto{\pgfqpoint{2.510135in}{0.871794in}}%
\pgfpathlineto{\pgfqpoint{2.512953in}{0.910800in}}%
\pgfpathlineto{\pgfqpoint{2.515771in}{0.537042in}}%
\pgfpathlineto{\pgfqpoint{2.518589in}{1.018543in}}%
\pgfpathlineto{\pgfqpoint{2.521407in}{1.154329in}}%
\pgfpathlineto{\pgfqpoint{2.524226in}{1.025798in}}%
\pgfpathlineto{\pgfqpoint{2.527044in}{1.967217in}}%
\pgfpathlineto{\pgfqpoint{2.529862in}{0.877054in}}%
\pgfpathlineto{\pgfqpoint{2.535498in}{1.253271in}}%
\pgfpathlineto{\pgfqpoint{2.538317in}{0.874619in}}%
\pgfpathlineto{\pgfqpoint{2.541135in}{0.969878in}}%
\pgfpathlineto{\pgfqpoint{2.543953in}{0.647986in}}%
\pgfpathlineto{\pgfqpoint{2.546771in}{1.388691in}}%
\pgfpathlineto{\pgfqpoint{2.549589in}{0.765400in}}%
\pgfpathlineto{\pgfqpoint{2.555226in}{0.901932in}}%
\pgfpathlineto{\pgfqpoint{2.558044in}{0.488891in}}%
\pgfpathlineto{\pgfqpoint{2.560862in}{0.934566in}}%
\pgfpathlineto{\pgfqpoint{2.563680in}{0.615952in}}%
\pgfpathlineto{\pgfqpoint{2.566498in}{1.626081in}}%
\pgfpathlineto{\pgfqpoint{2.569317in}{1.704199in}}%
\pgfpathlineto{\pgfqpoint{2.572135in}{0.721338in}}%
\pgfpathlineto{\pgfqpoint{2.574953in}{0.661591in}}%
\pgfpathlineto{\pgfqpoint{2.577771in}{0.933531in}}%
\pgfpathlineto{\pgfqpoint{2.580589in}{0.517657in}}%
\pgfpathlineto{\pgfqpoint{2.583407in}{0.674211in}}%
\pgfpathlineto{\pgfqpoint{2.586226in}{0.744941in}}%
\pgfpathlineto{\pgfqpoint{2.589044in}{0.508503in}}%
\pgfpathlineto{\pgfqpoint{2.591862in}{0.528199in}}%
\pgfpathlineto{\pgfqpoint{2.594680in}{0.734221in}}%
\pgfpathlineto{\pgfqpoint{2.597498in}{0.693004in}}%
\pgfpathlineto{\pgfqpoint{2.600317in}{0.720153in}}%
\pgfpathlineto{\pgfqpoint{2.603135in}{0.860704in}}%
\pgfpathlineto{\pgfqpoint{2.605953in}{0.774243in}}%
\pgfpathlineto{\pgfqpoint{2.608771in}{1.781497in}}%
\pgfpathlineto{\pgfqpoint{2.611589in}{0.754442in}}%
\pgfpathlineto{\pgfqpoint{2.614407in}{0.991760in}}%
\pgfpathlineto{\pgfqpoint{2.617226in}{0.523397in}}%
\pgfpathlineto{\pgfqpoint{2.620044in}{0.927953in}}%
\pgfpathlineto{\pgfqpoint{2.625680in}{0.530887in}}%
\pgfpathlineto{\pgfqpoint{2.628498in}{0.795011in}}%
\pgfpathlineto{\pgfqpoint{2.631317in}{0.591864in}}%
\pgfpathlineto{\pgfqpoint{2.634135in}{0.486505in}}%
\pgfpathlineto{\pgfqpoint{2.636953in}{0.635549in}}%
\pgfpathlineto{\pgfqpoint{2.639771in}{0.838571in}}%
\pgfpathlineto{\pgfqpoint{2.642589in}{0.709077in}}%
\pgfpathlineto{\pgfqpoint{2.645407in}{0.837481in}}%
\pgfpathlineto{\pgfqpoint{2.648226in}{0.694208in}}%
\pgfpathlineto{\pgfqpoint{2.651044in}{0.839330in}}%
\pgfpathlineto{\pgfqpoint{2.653862in}{0.724643in}}%
\pgfpathlineto{\pgfqpoint{2.656680in}{0.754716in}}%
\pgfpathlineto{\pgfqpoint{2.659498in}{1.010452in}}%
\pgfpathlineto{\pgfqpoint{2.662317in}{1.089592in}}%
\pgfpathlineto{\pgfqpoint{2.665135in}{0.980682in}}%
\pgfpathlineto{\pgfqpoint{2.667953in}{1.122719in}}%
\pgfpathlineto{\pgfqpoint{2.670771in}{0.645475in}}%
\pgfpathlineto{\pgfqpoint{2.673589in}{0.468908in}}%
\pgfpathlineto{\pgfqpoint{2.676407in}{0.556787in}}%
\pgfpathlineto{\pgfqpoint{2.679226in}{0.751071in}}%
\pgfpathlineto{\pgfqpoint{2.684862in}{0.838750in}}%
\pgfpathlineto{\pgfqpoint{2.687680in}{0.523328in}}%
\pgfpathlineto{\pgfqpoint{2.690498in}{0.796984in}}%
\pgfpathlineto{\pgfqpoint{2.693317in}{1.014799in}}%
\pgfpathlineto{\pgfqpoint{2.696135in}{1.069738in}}%
\pgfpathlineto{\pgfqpoint{2.698953in}{1.003638in}}%
\pgfpathlineto{\pgfqpoint{2.701771in}{0.522735in}}%
\pgfpathlineto{\pgfqpoint{2.704589in}{0.594662in}}%
\pgfpathlineto{\pgfqpoint{2.707407in}{0.550041in}}%
\pgfpathlineto{\pgfqpoint{2.710226in}{0.586505in}}%
\pgfpathlineto{\pgfqpoint{2.713044in}{0.596090in}}%
\pgfpathlineto{\pgfqpoint{2.718680in}{0.759611in}}%
\pgfpathlineto{\pgfqpoint{2.721498in}{0.777683in}}%
\pgfpathlineto{\pgfqpoint{2.724317in}{0.810139in}}%
\pgfpathlineto{\pgfqpoint{2.727135in}{0.828211in}}%
\pgfpathlineto{\pgfqpoint{2.729953in}{0.979579in}}%
\pgfpathlineto{\pgfqpoint{2.732771in}{1.805911in}}%
\pgfpathlineto{\pgfqpoint{2.735589in}{0.817958in}}%
\pgfpathlineto{\pgfqpoint{2.738407in}{0.837227in}}%
\pgfpathlineto{\pgfqpoint{2.741226in}{1.468397in}}%
\pgfpathlineto{\pgfqpoint{2.746862in}{0.508046in}}%
\pgfpathlineto{\pgfqpoint{2.749680in}{0.595742in}}%
\pgfpathlineto{\pgfqpoint{2.752498in}{0.759489in}}%
\pgfpathlineto{\pgfqpoint{2.755317in}{1.023541in}}%
\pgfpathlineto{\pgfqpoint{2.758135in}{0.785024in}}%
\pgfpathlineto{\pgfqpoint{2.760953in}{1.126852in}}%
\pgfpathlineto{\pgfqpoint{2.763771in}{0.595399in}}%
\pgfpathlineto{\pgfqpoint{2.766589in}{0.961992in}}%
\pgfpathlineto{\pgfqpoint{2.769407in}{0.669515in}}%
\pgfpathlineto{\pgfqpoint{2.772226in}{0.781353in}}%
\pgfpathlineto{\pgfqpoint{2.775044in}{0.857611in}}%
\pgfpathlineto{\pgfqpoint{2.777862in}{0.526233in}}%
\pgfpathlineto{\pgfqpoint{2.780680in}{0.640544in}}%
\pgfpathlineto{\pgfqpoint{2.783498in}{1.015967in}}%
\pgfpathlineto{\pgfqpoint{2.786317in}{1.015967in}}%
\pgfpathlineto{\pgfqpoint{2.791953in}{0.720575in}}%
\pgfpathlineto{\pgfqpoint{2.794771in}{0.811199in}}%
\pgfpathlineto{\pgfqpoint{2.797589in}{1.159079in}}%
\pgfpathlineto{\pgfqpoint{2.800407in}{0.507303in}}%
\pgfpathlineto{\pgfqpoint{2.803226in}{0.478499in}}%
\pgfpathlineto{\pgfqpoint{2.806044in}{0.796801in}}%
\pgfpathlineto{\pgfqpoint{2.808862in}{0.928405in}}%
\pgfpathlineto{\pgfqpoint{2.811680in}{0.885815in}}%
\pgfpathlineto{\pgfqpoint{2.814498in}{1.383124in}}%
\pgfpathlineto{\pgfqpoint{2.817317in}{1.041154in}}%
\pgfpathlineto{\pgfqpoint{2.820135in}{1.868256in}}%
\pgfpathlineto{\pgfqpoint{2.822953in}{0.585506in}}%
\pgfpathlineto{\pgfqpoint{2.825771in}{0.553660in}}%
\pgfpathlineto{\pgfqpoint{2.828589in}{0.692576in}}%
\pgfpathlineto{\pgfqpoint{2.831407in}{0.745575in}}%
\pgfpathlineto{\pgfqpoint{2.834226in}{0.606909in}}%
\pgfpathlineto{\pgfqpoint{2.837044in}{0.996104in}}%
\pgfpathlineto{\pgfqpoint{2.839862in}{0.573578in}}%
\pgfpathlineto{\pgfqpoint{2.842680in}{0.848901in}}%
\pgfpathlineto{\pgfqpoint{2.845498in}{1.320772in}}%
\pgfpathlineto{\pgfqpoint{2.848317in}{0.654359in}}%
\pgfpathlineto{\pgfqpoint{2.851135in}{2.250370in}}%
\pgfpathlineto{\pgfqpoint{2.853953in}{0.963787in}}%
\pgfpathlineto{\pgfqpoint{2.856771in}{0.646236in}}%
\pgfpathlineto{\pgfqpoint{2.859589in}{0.890506in}}%
\pgfpathlineto{\pgfqpoint{2.862407in}{0.468908in}}%
\pgfpathlineto{\pgfqpoint{2.865226in}{0.680478in}}%
\pgfpathlineto{\pgfqpoint{2.868044in}{0.678936in}}%
\pgfpathlineto{\pgfqpoint{2.870862in}{0.510730in}}%
\pgfpathlineto{\pgfqpoint{2.873680in}{0.531664in}}%
\pgfpathlineto{\pgfqpoint{2.876498in}{1.100652in}}%
\pgfpathlineto{\pgfqpoint{2.879317in}{0.653888in}}%
\pgfpathlineto{\pgfqpoint{2.882135in}{0.686230in}}%
\pgfpathlineto{\pgfqpoint{2.884953in}{1.669313in}}%
\pgfpathlineto{\pgfqpoint{2.887771in}{0.498784in}}%
\pgfpathlineto{\pgfqpoint{2.890589in}{0.819407in}}%
\pgfpathlineto{\pgfqpoint{2.893407in}{0.699716in}}%
\pgfpathlineto{\pgfqpoint{2.896226in}{0.729958in}}%
\pgfpathlineto{\pgfqpoint{2.899044in}{0.699957in}}%
\pgfpathlineto{\pgfqpoint{2.901862in}{0.770646in}}%
\pgfpathlineto{\pgfqpoint{2.904680in}{0.712594in}}%
\pgfpathlineto{\pgfqpoint{2.907498in}{0.843735in}}%
\pgfpathlineto{\pgfqpoint{2.910317in}{0.928212in}}%
\pgfpathlineto{\pgfqpoint{2.913135in}{0.588022in}}%
\pgfpathlineto{\pgfqpoint{2.915953in}{0.768866in}}%
\pgfpathlineto{\pgfqpoint{2.918771in}{1.232611in}}%
\pgfpathlineto{\pgfqpoint{2.921589in}{0.478694in}}%
\pgfpathlineto{\pgfqpoint{2.924407in}{0.952388in}}%
\pgfpathlineto{\pgfqpoint{2.927226in}{0.839436in}}%
\pgfpathlineto{\pgfqpoint{2.930044in}{0.478985in}}%
\pgfpathlineto{\pgfqpoint{2.932862in}{0.968326in}}%
\pgfpathlineto{\pgfqpoint{2.935680in}{0.568101in}}%
\pgfpathlineto{\pgfqpoint{2.938498in}{0.859059in}}%
\pgfpathlineto{\pgfqpoint{2.941317in}{0.819286in}}%
\pgfpathlineto{\pgfqpoint{2.944135in}{0.756028in}}%
\pgfpathlineto{\pgfqpoint{2.946953in}{0.488604in}}%
\pgfpathlineto{\pgfqpoint{2.949771in}{0.830882in}}%
\pgfpathlineto{\pgfqpoint{2.952589in}{0.693384in}}%
\pgfpathlineto{\pgfqpoint{2.955407in}{0.664203in}}%
\pgfpathlineto{\pgfqpoint{2.958226in}{0.624197in}}%
\pgfpathlineto{\pgfqpoint{2.961044in}{0.556155in}}%
\pgfpathlineto{\pgfqpoint{2.966680in}{0.993905in}}%
\pgfpathlineto{\pgfqpoint{2.969498in}{0.718313in}}%
\pgfpathlineto{\pgfqpoint{2.972317in}{0.848442in}}%
\pgfpathlineto{\pgfqpoint{2.975135in}{0.924952in}}%
\pgfpathlineto{\pgfqpoint{2.977953in}{1.020005in}}%
\pgfpathlineto{\pgfqpoint{2.980771in}{0.941307in}}%
\pgfpathlineto{\pgfqpoint{2.983589in}{0.838409in}}%
\pgfpathlineto{\pgfqpoint{2.986407in}{0.894165in}}%
\pgfpathlineto{\pgfqpoint{2.989226in}{0.962727in}}%
\pgfpathlineto{\pgfqpoint{2.992044in}{0.510444in}}%
\pgfpathlineto{\pgfqpoint{2.994862in}{0.983856in}}%
\pgfpathlineto{\pgfqpoint{2.997680in}{0.499514in}}%
\pgfpathlineto{\pgfqpoint{3.000498in}{0.952180in}}%
\pgfpathlineto{\pgfqpoint{3.003317in}{0.531055in}}%
\pgfpathlineto{\pgfqpoint{3.006135in}{0.695642in}}%
\pgfpathlineto{\pgfqpoint{3.008953in}{0.754922in}}%
\pgfpathlineto{\pgfqpoint{3.011771in}{0.852585in}}%
\pgfpathlineto{\pgfqpoint{3.017407in}{0.640581in}}%
\pgfpathlineto{\pgfqpoint{3.020226in}{0.620331in}}%
\pgfpathlineto{\pgfqpoint{3.023044in}{0.631307in}}%
\pgfpathlineto{\pgfqpoint{3.025862in}{0.540070in}}%
\pgfpathlineto{\pgfqpoint{3.028680in}{1.355615in}}%
\pgfpathlineto{\pgfqpoint{3.031498in}{1.061322in}}%
\pgfpathlineto{\pgfqpoint{3.034317in}{0.500955in}}%
\pgfpathlineto{\pgfqpoint{3.037135in}{0.693990in}}%
\pgfpathlineto{\pgfqpoint{3.042771in}{0.962046in}}%
\pgfpathlineto{\pgfqpoint{3.045589in}{0.598953in}}%
\pgfpathlineto{\pgfqpoint{3.048407in}{0.862601in}}%
\pgfpathlineto{\pgfqpoint{3.051226in}{1.024784in}}%
\pgfpathlineto{\pgfqpoint{3.054044in}{0.703683in}}%
\pgfpathlineto{\pgfqpoint{3.056862in}{1.020316in}}%
\pgfpathlineto{\pgfqpoint{3.059680in}{0.545473in}}%
\pgfpathlineto{\pgfqpoint{3.062498in}{0.501746in}}%
\pgfpathlineto{\pgfqpoint{3.068135in}{0.884632in}}%
\pgfpathlineto{\pgfqpoint{3.070953in}{0.834862in}}%
\pgfpathlineto{\pgfqpoint{3.073771in}{0.850876in}}%
\pgfpathlineto{\pgfqpoint{3.076589in}{0.582222in}}%
\pgfpathlineto{\pgfqpoint{3.079407in}{0.480264in}}%
\pgfpathlineto{\pgfqpoint{3.082226in}{1.007749in}}%
\pgfpathlineto{\pgfqpoint{3.085044in}{0.756779in}}%
\pgfpathlineto{\pgfqpoint{3.087862in}{1.081066in}}%
\pgfpathlineto{\pgfqpoint{3.090680in}{0.524937in}}%
\pgfpathlineto{\pgfqpoint{3.093498in}{0.569839in}}%
\pgfpathlineto{\pgfqpoint{3.096317in}{0.502591in}}%
\pgfpathlineto{\pgfqpoint{3.099135in}{0.569721in}}%
\pgfpathlineto{\pgfqpoint{3.101953in}{0.547075in}}%
\pgfpathlineto{\pgfqpoint{3.104771in}{0.702145in}}%
\pgfpathlineto{\pgfqpoint{3.107589in}{0.579738in}}%
\pgfpathlineto{\pgfqpoint{3.110407in}{0.591315in}}%
\pgfpathlineto{\pgfqpoint{3.113226in}{1.217410in}}%
\pgfpathlineto{\pgfqpoint{3.116044in}{1.128335in}}%
\pgfpathlineto{\pgfqpoint{3.118862in}{0.943024in}}%
\pgfpathlineto{\pgfqpoint{3.121680in}{1.043253in}}%
\pgfpathlineto{\pgfqpoint{3.124498in}{0.613571in}}%
\pgfpathlineto{\pgfqpoint{3.127317in}{0.524355in}}%
\pgfpathlineto{\pgfqpoint{3.130135in}{0.635569in}}%
\pgfpathlineto{\pgfqpoint{3.132953in}{1.020152in}}%
\pgfpathlineto{\pgfqpoint{3.135771in}{1.321915in}}%
\pgfpathlineto{\pgfqpoint{3.138589in}{1.505171in}}%
\pgfpathlineto{\pgfqpoint{3.141407in}{0.810504in}}%
\pgfpathlineto{\pgfqpoint{3.144226in}{0.889222in}}%
\pgfpathlineto{\pgfqpoint{3.147044in}{2.020339in}}%
\pgfpathlineto{\pgfqpoint{3.149862in}{0.913857in}}%
\pgfpathlineto{\pgfqpoint{3.152680in}{1.180443in}}%
\pgfpathlineto{\pgfqpoint{3.155498in}{0.593333in}}%
\pgfpathlineto{\pgfqpoint{3.158317in}{1.200485in}}%
\pgfpathlineto{\pgfqpoint{3.161135in}{1.268127in}}%
\pgfpathlineto{\pgfqpoint{3.163953in}{0.938011in}}%
\pgfpathlineto{\pgfqpoint{3.166771in}{1.028237in}}%
\pgfpathlineto{\pgfqpoint{3.169589in}{1.446182in}}%
\pgfpathlineto{\pgfqpoint{3.172407in}{0.721150in}}%
\pgfpathlineto{\pgfqpoint{3.175226in}{0.490930in}}%
\pgfpathlineto{\pgfqpoint{3.178044in}{0.666352in}}%
\pgfpathlineto{\pgfqpoint{3.180862in}{0.545531in}}%
\pgfpathlineto{\pgfqpoint{3.183680in}{0.806702in}}%
\pgfpathlineto{\pgfqpoint{3.186498in}{0.501388in}}%
\pgfpathlineto{\pgfqpoint{3.189317in}{0.523062in}}%
\pgfpathlineto{\pgfqpoint{3.192135in}{1.149021in}}%
\pgfpathlineto{\pgfqpoint{3.194953in}{0.612848in}}%
\pgfpathlineto{\pgfqpoint{3.197771in}{0.959021in}}%
\pgfpathlineto{\pgfqpoint{3.200589in}{0.536234in}}%
\pgfpathlineto{\pgfqpoint{3.203407in}{0.727851in}}%
\pgfpathlineto{\pgfqpoint{3.206226in}{0.795021in}}%
\pgfpathlineto{\pgfqpoint{3.209044in}{0.513670in}}%
\pgfpathlineto{\pgfqpoint{3.214680in}{1.336917in}}%
\pgfpathlineto{\pgfqpoint{3.217498in}{0.597454in}}%
\pgfpathlineto{\pgfqpoint{3.220317in}{0.948027in}}%
\pgfpathlineto{\pgfqpoint{3.223135in}{1.126567in}}%
\pgfpathlineto{\pgfqpoint{3.225953in}{0.949949in}}%
\pgfpathlineto{\pgfqpoint{3.228771in}{0.991930in}}%
\pgfpathlineto{\pgfqpoint{3.231589in}{0.647765in}}%
\pgfpathlineto{\pgfqpoint{3.234407in}{0.807686in}}%
\pgfpathlineto{\pgfqpoint{3.237226in}{0.528982in}}%
\pgfpathlineto{\pgfqpoint{3.240044in}{0.619329in}}%
\pgfpathlineto{\pgfqpoint{3.242862in}{1.144618in}}%
\pgfpathlineto{\pgfqpoint{3.245680in}{0.508220in}}%
\pgfpathlineto{\pgfqpoint{3.251317in}{0.942654in}}%
\pgfpathlineto{\pgfqpoint{3.254135in}{0.603471in}}%
\pgfpathlineto{\pgfqpoint{3.256953in}{0.526771in}}%
\pgfpathlineto{\pgfqpoint{3.259771in}{1.343619in}}%
\pgfpathlineto{\pgfqpoint{3.262589in}{0.887275in}}%
\pgfpathlineto{\pgfqpoint{3.268226in}{0.562193in}}%
\pgfpathlineto{\pgfqpoint{3.271044in}{0.664459in}}%
\pgfpathlineto{\pgfqpoint{3.273862in}{1.040648in}}%
\pgfpathlineto{\pgfqpoint{3.276680in}{0.611255in}}%
\pgfpathlineto{\pgfqpoint{3.279498in}{0.677443in}}%
\pgfpathlineto{\pgfqpoint{3.282317in}{0.638412in}}%
\pgfpathlineto{\pgfqpoint{3.285135in}{0.543924in}}%
\pgfpathlineto{\pgfqpoint{3.287953in}{0.600315in}}%
\pgfpathlineto{\pgfqpoint{3.290771in}{0.600917in}}%
\pgfpathlineto{\pgfqpoint{3.293589in}{0.897367in}}%
\pgfpathlineto{\pgfqpoint{3.296407in}{0.584248in}}%
\pgfpathlineto{\pgfqpoint{3.299226in}{0.778757in}}%
\pgfpathlineto{\pgfqpoint{3.302044in}{0.870835in}}%
\pgfpathlineto{\pgfqpoint{3.304862in}{0.726724in}}%
\pgfpathlineto{\pgfqpoint{3.307680in}{0.478870in}}%
\pgfpathlineto{\pgfqpoint{3.310498in}{0.943251in}}%
\pgfpathlineto{\pgfqpoint{3.313317in}{0.925924in}}%
\pgfpathlineto{\pgfqpoint{3.316135in}{1.445805in}}%
\pgfpathlineto{\pgfqpoint{3.318953in}{0.765788in}}%
\pgfpathlineto{\pgfqpoint{3.321771in}{0.616966in}}%
\pgfpathlineto{\pgfqpoint{3.324589in}{0.571141in}}%
\pgfpathlineto{\pgfqpoint{3.327407in}{0.552580in}}%
\pgfpathlineto{\pgfqpoint{3.330226in}{0.468908in}}%
\pgfpathlineto{\pgfqpoint{3.333044in}{0.727678in}}%
\pgfpathlineto{\pgfqpoint{3.335862in}{0.734492in}}%
\pgfpathlineto{\pgfqpoint{3.338680in}{0.679343in}}%
\pgfpathlineto{\pgfqpoint{3.341498in}{1.128378in}}%
\pgfpathlineto{\pgfqpoint{3.344317in}{1.220060in}}%
\pgfpathlineto{\pgfqpoint{3.347135in}{0.505499in}}%
\pgfpathlineto{\pgfqpoint{3.349953in}{0.661533in}}%
\pgfpathlineto{\pgfqpoint{3.352771in}{0.514888in}}%
\pgfpathlineto{\pgfqpoint{3.355589in}{0.505686in}}%
\pgfpathlineto{\pgfqpoint{3.358407in}{0.885930in}}%
\pgfpathlineto{\pgfqpoint{3.361226in}{0.636441in}}%
\pgfpathlineto{\pgfqpoint{3.364044in}{1.056817in}}%
\pgfpathlineto{\pgfqpoint{3.366862in}{0.568758in}}%
\pgfpathlineto{\pgfqpoint{3.369680in}{0.787820in}}%
\pgfpathlineto{\pgfqpoint{3.372498in}{1.343306in}}%
\pgfpathlineto{\pgfqpoint{3.375317in}{0.788251in}}%
\pgfpathlineto{\pgfqpoint{3.378135in}{0.876990in}}%
\pgfpathlineto{\pgfqpoint{3.383771in}{0.487192in}}%
\pgfpathlineto{\pgfqpoint{3.386589in}{0.505511in}}%
\pgfpathlineto{\pgfqpoint{3.389407in}{0.887061in}}%
\pgfpathlineto{\pgfqpoint{3.392226in}{0.841315in}}%
\pgfpathlineto{\pgfqpoint{3.395044in}{0.911101in}}%
\pgfpathlineto{\pgfqpoint{3.397862in}{0.861473in}}%
\pgfpathlineto{\pgfqpoint{3.400680in}{0.600960in}}%
\pgfpathlineto{\pgfqpoint{3.406317in}{1.572791in}}%
\pgfpathlineto{\pgfqpoint{3.409135in}{0.761763in}}%
\pgfpathlineto{\pgfqpoint{3.411953in}{0.962012in}}%
\pgfpathlineto{\pgfqpoint{3.414771in}{1.022313in}}%
\pgfpathlineto{\pgfqpoint{3.417589in}{1.152317in}}%
\pgfpathlineto{\pgfqpoint{3.420407in}{1.201215in}}%
\pgfpathlineto{\pgfqpoint{3.423226in}{3.214362in}}%
\pgfpathlineto{\pgfqpoint{3.426044in}{0.511868in}}%
\pgfpathlineto{\pgfqpoint{3.428862in}{0.673543in}}%
\pgfpathlineto{\pgfqpoint{3.431680in}{0.533686in}}%
\pgfpathlineto{\pgfqpoint{3.434498in}{0.726576in}}%
\pgfpathlineto{\pgfqpoint{3.437317in}{0.780548in}}%
\pgfpathlineto{\pgfqpoint{3.440135in}{0.791235in}}%
\pgfpathlineto{\pgfqpoint{3.442953in}{0.586676in}}%
\pgfpathlineto{\pgfqpoint{3.445771in}{0.618711in}}%
\pgfpathlineto{\pgfqpoint{3.448589in}{1.465365in}}%
\pgfpathlineto{\pgfqpoint{3.451407in}{0.655083in}}%
\pgfpathlineto{\pgfqpoint{3.454226in}{1.439723in}}%
\pgfpathlineto{\pgfqpoint{3.457044in}{0.673543in}}%
\pgfpathlineto{\pgfqpoint{3.459862in}{0.898040in}}%
\pgfpathlineto{\pgfqpoint{3.462680in}{1.191360in}}%
\pgfpathlineto{\pgfqpoint{3.465498in}{0.688123in}}%
\pgfpathlineto{\pgfqpoint{3.468317in}{0.479912in}}%
\pgfpathlineto{\pgfqpoint{3.471135in}{1.047020in}}%
\pgfpathlineto{\pgfqpoint{3.473953in}{0.480140in}}%
\pgfpathlineto{\pgfqpoint{3.476771in}{0.892676in}}%
\pgfpathlineto{\pgfqpoint{3.479589in}{0.612462in}}%
\pgfpathlineto{\pgfqpoint{3.482407in}{1.143882in}}%
\pgfpathlineto{\pgfqpoint{3.485226in}{0.511981in}}%
\pgfpathlineto{\pgfqpoint{3.488044in}{1.178019in}}%
\pgfpathlineto{\pgfqpoint{3.493680in}{1.014263in}}%
\pgfpathlineto{\pgfqpoint{3.496498in}{0.862601in}}%
\pgfpathlineto{\pgfqpoint{3.499317in}{0.889425in}}%
\pgfpathlineto{\pgfqpoint{3.502135in}{0.639668in}}%
\pgfpathlineto{\pgfqpoint{3.504953in}{0.671799in}}%
\pgfpathlineto{\pgfqpoint{3.507771in}{1.219883in}}%
\pgfpathlineto{\pgfqpoint{3.510589in}{0.635473in}}%
\pgfpathlineto{\pgfqpoint{3.513407in}{0.500032in}}%
\pgfpathlineto{\pgfqpoint{3.516226in}{0.520707in}}%
\pgfpathlineto{\pgfqpoint{3.519044in}{0.757273in}}%
\pgfpathlineto{\pgfqpoint{3.521862in}{0.829818in}}%
\pgfpathlineto{\pgfqpoint{3.524680in}{0.768268in}}%
\pgfpathlineto{\pgfqpoint{3.527498in}{0.815966in}}%
\pgfpathlineto{\pgfqpoint{3.530317in}{0.931902in}}%
\pgfpathlineto{\pgfqpoint{3.533135in}{0.770015in}}%
\pgfpathlineto{\pgfqpoint{3.535953in}{0.669995in}}%
\pgfpathlineto{\pgfqpoint{3.538771in}{0.508957in}}%
\pgfpathlineto{\pgfqpoint{3.541589in}{0.727882in}}%
\pgfpathlineto{\pgfqpoint{3.544407in}{1.096641in}}%
\pgfpathlineto{\pgfqpoint{3.550044in}{0.488524in}}%
\pgfpathlineto{\pgfqpoint{3.552862in}{0.754849in}}%
\pgfpathlineto{\pgfqpoint{3.555680in}{0.578115in}}%
\pgfpathlineto{\pgfqpoint{3.558498in}{0.488794in}}%
\pgfpathlineto{\pgfqpoint{3.561317in}{0.902977in}}%
\pgfpathlineto{\pgfqpoint{3.564135in}{0.761159in}}%
\pgfpathlineto{\pgfqpoint{3.566953in}{0.790548in}}%
\pgfpathlineto{\pgfqpoint{3.569771in}{0.556984in}}%
\pgfpathlineto{\pgfqpoint{3.572589in}{0.872344in}}%
\pgfpathlineto{\pgfqpoint{3.575407in}{0.558228in}}%
\pgfpathlineto{\pgfqpoint{3.578226in}{0.627509in}}%
\pgfpathlineto{\pgfqpoint{3.581044in}{0.947353in}}%
\pgfpathlineto{\pgfqpoint{3.583862in}{0.549202in}}%
\pgfpathlineto{\pgfqpoint{3.586680in}{0.498961in}}%
\pgfpathlineto{\pgfqpoint{3.589498in}{0.638618in}}%
\pgfpathlineto{\pgfqpoint{3.592317in}{0.716680in}}%
\pgfpathlineto{\pgfqpoint{3.595135in}{0.766491in}}%
\pgfpathlineto{\pgfqpoint{3.597953in}{0.779673in}}%
\pgfpathlineto{\pgfqpoint{3.603589in}{0.509013in}}%
\pgfpathlineto{\pgfqpoint{3.606407in}{0.827863in}}%
\pgfpathlineto{\pgfqpoint{3.609226in}{0.498650in}}%
\pgfpathlineto{\pgfqpoint{3.612044in}{0.768036in}}%
\pgfpathlineto{\pgfqpoint{3.614862in}{0.690259in}}%
\pgfpathlineto{\pgfqpoint{3.617680in}{0.479006in}}%
\pgfpathlineto{\pgfqpoint{3.623317in}{1.054121in}}%
\pgfpathlineto{\pgfqpoint{3.626135in}{1.243709in}}%
\pgfpathlineto{\pgfqpoint{3.628953in}{0.781333in}}%
\pgfpathlineto{\pgfqpoint{3.631771in}{0.708130in}}%
\pgfpathlineto{\pgfqpoint{3.634589in}{0.710133in}}%
\pgfpathlineto{\pgfqpoint{3.637407in}{0.521520in}}%
\pgfpathlineto{\pgfqpoint{3.643044in}{0.700023in}}%
\pgfpathlineto{\pgfqpoint{3.645862in}{0.794039in}}%
\pgfpathlineto{\pgfqpoint{3.648680in}{0.552462in}}%
\pgfpathlineto{\pgfqpoint{3.651498in}{0.752705in}}%
\pgfpathlineto{\pgfqpoint{3.654317in}{0.616411in}}%
\pgfpathlineto{\pgfqpoint{3.657135in}{0.927634in}}%
\pgfpathlineto{\pgfqpoint{3.659953in}{0.864648in}}%
\pgfpathlineto{\pgfqpoint{3.662771in}{0.479392in}}%
\pgfpathlineto{\pgfqpoint{3.665589in}{0.816915in}}%
\pgfpathlineto{\pgfqpoint{3.668407in}{0.973722in}}%
\pgfpathlineto{\pgfqpoint{3.671226in}{0.541793in}}%
\pgfpathlineto{\pgfqpoint{3.674044in}{1.248590in}}%
\pgfpathlineto{\pgfqpoint{3.676862in}{0.770962in}}%
\pgfpathlineto{\pgfqpoint{3.679680in}{0.568893in}}%
\pgfpathlineto{\pgfqpoint{3.682498in}{0.538692in}}%
\pgfpathlineto{\pgfqpoint{3.685317in}{0.488815in}}%
\pgfpathlineto{\pgfqpoint{3.688135in}{0.528547in}}%
\pgfpathlineto{\pgfqpoint{3.690953in}{0.558136in}}%
\pgfpathlineto{\pgfqpoint{3.693771in}{0.597882in}}%
\pgfpathlineto{\pgfqpoint{3.696589in}{0.528506in}}%
\pgfpathlineto{\pgfqpoint{3.699407in}{0.646966in}}%
\pgfpathlineto{\pgfqpoint{3.702226in}{0.478768in}}%
\pgfpathlineto{\pgfqpoint{3.707862in}{0.557345in}}%
\pgfpathlineto{\pgfqpoint{3.710680in}{0.586403in}}%
\pgfpathlineto{\pgfqpoint{3.713498in}{0.498237in}}%
\pgfpathlineto{\pgfqpoint{3.716317in}{0.942277in}}%
\pgfpathlineto{\pgfqpoint{3.719135in}{0.538427in}}%
\pgfpathlineto{\pgfqpoint{3.721953in}{0.528362in}}%
\pgfpathlineto{\pgfqpoint{3.724771in}{0.842632in}}%
\pgfpathlineto{\pgfqpoint{3.727589in}{0.605383in}}%
\pgfpathlineto{\pgfqpoint{3.730407in}{0.498098in}}%
\pgfpathlineto{\pgfqpoint{3.736044in}{1.755015in}}%
\pgfpathlineto{\pgfqpoint{3.738862in}{1.236295in}}%
\pgfpathlineto{\pgfqpoint{3.741680in}{0.905595in}}%
\pgfpathlineto{\pgfqpoint{3.744498in}{1.146253in}}%
\pgfpathlineto{\pgfqpoint{3.747317in}{1.245921in}}%
\pgfpathlineto{\pgfqpoint{3.750135in}{0.516862in}}%
\pgfpathlineto{\pgfqpoint{3.752953in}{0.526558in}}%
\pgfpathlineto{\pgfqpoint{3.755771in}{0.717899in}}%
\pgfpathlineto{\pgfqpoint{3.758589in}{0.668457in}}%
\pgfpathlineto{\pgfqpoint{3.761407in}{1.161267in}}%
\pgfpathlineto{\pgfqpoint{3.764226in}{0.781526in}}%
\pgfpathlineto{\pgfqpoint{3.767044in}{0.940564in}}%
\pgfpathlineto{\pgfqpoint{3.769862in}{0.522836in}}%
\pgfpathlineto{\pgfqpoint{3.772680in}{0.630090in}}%
\pgfpathlineto{\pgfqpoint{3.775498in}{0.576262in}}%
\pgfpathlineto{\pgfqpoint{3.778317in}{0.486839in}}%
\pgfpathlineto{\pgfqpoint{3.781135in}{0.549511in}}%
\pgfpathlineto{\pgfqpoint{3.783953in}{0.912645in}}%
\pgfpathlineto{\pgfqpoint{3.786771in}{1.646324in}}%
\pgfpathlineto{\pgfqpoint{3.789589in}{0.797296in}}%
\pgfpathlineto{\pgfqpoint{3.792407in}{0.496105in}}%
\pgfpathlineto{\pgfqpoint{3.795226in}{0.802249in}}%
\pgfpathlineto{\pgfqpoint{3.798044in}{0.684689in}}%
\pgfpathlineto{\pgfqpoint{3.800862in}{0.523107in}}%
\pgfpathlineto{\pgfqpoint{3.803680in}{0.541150in}}%
\pgfpathlineto{\pgfqpoint{3.806498in}{0.468908in}}%
\pgfpathlineto{\pgfqpoint{3.809317in}{0.713456in}}%
\pgfpathlineto{\pgfqpoint{3.812135in}{0.514423in}}%
\pgfpathlineto{\pgfqpoint{3.814953in}{0.749951in}}%
\pgfpathlineto{\pgfqpoint{3.817771in}{0.800797in}}%
\pgfpathlineto{\pgfqpoint{3.820589in}{0.513466in}}%
\pgfpathlineto{\pgfqpoint{3.823407in}{0.567027in}}%
\pgfpathlineto{\pgfqpoint{3.826226in}{0.792380in}}%
\pgfpathlineto{\pgfqpoint{3.829044in}{0.595134in}}%
\pgfpathlineto{\pgfqpoint{3.831862in}{0.559013in}}%
\pgfpathlineto{\pgfqpoint{3.834680in}{0.747323in}}%
\pgfpathlineto{\pgfqpoint{3.837498in}{0.718088in}}%
\pgfpathlineto{\pgfqpoint{3.840317in}{0.628847in}}%
\pgfpathlineto{\pgfqpoint{3.843135in}{0.504571in}}%
\pgfpathlineto{\pgfqpoint{3.845953in}{0.691081in}}%
\pgfpathlineto{\pgfqpoint{3.848771in}{0.566124in}}%
\pgfpathlineto{\pgfqpoint{3.851589in}{0.504178in}}%
\pgfpathlineto{\pgfqpoint{3.854407in}{0.672303in}}%
\pgfpathlineto{\pgfqpoint{3.857226in}{0.522204in}}%
\pgfpathlineto{\pgfqpoint{3.862862in}{0.698025in}}%
\pgfpathlineto{\pgfqpoint{3.865680in}{0.591528in}}%
\pgfpathlineto{\pgfqpoint{3.868498in}{0.625800in}}%
\pgfpathlineto{\pgfqpoint{3.874135in}{0.512181in}}%
\pgfpathlineto{\pgfqpoint{3.876953in}{0.864003in}}%
\pgfpathlineto{\pgfqpoint{3.879771in}{0.562895in}}%
\pgfpathlineto{\pgfqpoint{3.882589in}{0.477467in}}%
\pgfpathlineto{\pgfqpoint{3.885407in}{1.276270in}}%
\pgfpathlineto{\pgfqpoint{3.888226in}{0.983722in}}%
\pgfpathlineto{\pgfqpoint{3.891044in}{0.843200in}}%
\pgfpathlineto{\pgfqpoint{3.893862in}{0.477668in}}%
\pgfpathlineto{\pgfqpoint{3.896680in}{1.221003in}}%
\pgfpathlineto{\pgfqpoint{3.899498in}{1.326382in}}%
\pgfpathlineto{\pgfqpoint{3.905135in}{0.485705in}}%
\pgfpathlineto{\pgfqpoint{3.907953in}{0.711515in}}%
\pgfpathlineto{\pgfqpoint{3.910771in}{0.543786in}}%
\pgfpathlineto{\pgfqpoint{3.913589in}{0.744419in}}%
\pgfpathlineto{\pgfqpoint{3.916407in}{0.891433in}}%
\pgfpathlineto{\pgfqpoint{3.919226in}{0.883043in}}%
\pgfpathlineto{\pgfqpoint{3.922044in}{0.669598in}}%
\pgfpathlineto{\pgfqpoint{3.924862in}{0.560717in}}%
\pgfpathlineto{\pgfqpoint{3.927680in}{0.801369in}}%
\pgfpathlineto{\pgfqpoint{3.930498in}{0.543378in}}%
\pgfpathlineto{\pgfqpoint{3.933317in}{0.560189in}}%
\pgfpathlineto{\pgfqpoint{3.936135in}{0.815890in}}%
\pgfpathlineto{\pgfqpoint{3.938953in}{0.526334in}}%
\pgfpathlineto{\pgfqpoint{3.941771in}{0.493484in}}%
\pgfpathlineto{\pgfqpoint{3.944589in}{0.526285in}}%
\pgfpathlineto{\pgfqpoint{3.947407in}{1.049054in}}%
\pgfpathlineto{\pgfqpoint{3.950226in}{0.577947in}}%
\pgfpathlineto{\pgfqpoint{3.953044in}{0.586798in}}%
\pgfpathlineto{\pgfqpoint{3.955862in}{0.837808in}}%
\pgfpathlineto{\pgfqpoint{3.958680in}{0.594139in}}%
\pgfpathlineto{\pgfqpoint{3.961498in}{0.826458in}}%
\pgfpathlineto{\pgfqpoint{3.964317in}{2.232382in}}%
\pgfpathlineto{\pgfqpoint{3.967135in}{0.539276in}}%
\pgfpathlineto{\pgfqpoint{3.969953in}{0.689922in}}%
\pgfpathlineto{\pgfqpoint{3.972771in}{0.557517in}}%
\pgfpathlineto{\pgfqpoint{3.975589in}{0.601924in}}%
\pgfpathlineto{\pgfqpoint{3.978407in}{0.620406in}}%
\pgfpathlineto{\pgfqpoint{3.981226in}{0.558398in}}%
\pgfpathlineto{\pgfqpoint{3.984044in}{0.522735in}}%
\pgfpathlineto{\pgfqpoint{3.986862in}{0.576463in}}%
\pgfpathlineto{\pgfqpoint{3.989680in}{0.765655in}}%
\pgfpathlineto{\pgfqpoint{3.992498in}{0.541128in}}%
\pgfpathlineto{\pgfqpoint{3.995317in}{0.813576in}}%
\pgfpathlineto{\pgfqpoint{3.998135in}{0.743956in}}%
\pgfpathlineto{\pgfqpoint{4.000953in}{1.375708in}}%
\pgfpathlineto{\pgfqpoint{4.003771in}{0.513507in}}%
\pgfpathlineto{\pgfqpoint{4.006589in}{0.477819in}}%
\pgfpathlineto{\pgfqpoint{4.009407in}{0.674576in}}%
\pgfpathlineto{\pgfqpoint{4.012226in}{0.558509in}}%
\pgfpathlineto{\pgfqpoint{4.015044in}{0.576463in}}%
\pgfpathlineto{\pgfqpoint{4.017862in}{0.825874in}}%
\pgfpathlineto{\pgfqpoint{4.020680in}{0.531058in}}%
\pgfpathlineto{\pgfqpoint{4.023498in}{0.619613in}}%
\pgfpathlineto{\pgfqpoint{4.026317in}{0.530868in}}%
\pgfpathlineto{\pgfqpoint{4.029135in}{0.495478in}}%
\pgfpathlineto{\pgfqpoint{4.031953in}{0.882012in}}%
\pgfpathlineto{\pgfqpoint{4.034771in}{0.696701in}}%
\pgfpathlineto{\pgfqpoint{4.037589in}{0.661773in}}%
\pgfpathlineto{\pgfqpoint{4.040407in}{0.530004in}}%
\pgfpathlineto{\pgfqpoint{4.043226in}{0.635043in}}%
\pgfpathlineto{\pgfqpoint{4.048862in}{1.255950in}}%
\pgfpathlineto{\pgfqpoint{4.051680in}{0.806257in}}%
\pgfpathlineto{\pgfqpoint{4.054498in}{0.658634in}}%
\pgfpathlineto{\pgfqpoint{4.057317in}{0.746717in}}%
\pgfpathlineto{\pgfqpoint{4.060135in}{1.112713in}}%
\pgfpathlineto{\pgfqpoint{4.062953in}{0.625468in}}%
\pgfpathlineto{\pgfqpoint{4.065771in}{0.520906in}}%
\pgfpathlineto{\pgfqpoint{4.068589in}{0.632953in}}%
\pgfpathlineto{\pgfqpoint{4.071407in}{1.132737in}}%
\pgfpathlineto{\pgfqpoint{4.074226in}{0.671539in}}%
\pgfpathlineto{\pgfqpoint{4.077044in}{0.663124in}}%
\pgfpathlineto{\pgfqpoint{4.079862in}{0.850130in}}%
\pgfpathlineto{\pgfqpoint{4.082680in}{0.925773in}}%
\pgfpathlineto{\pgfqpoint{4.085498in}{0.719626in}}%
\pgfpathlineto{\pgfqpoint{4.088317in}{0.652554in}}%
\pgfpathlineto{\pgfqpoint{4.091135in}{0.502424in}}%
\pgfpathlineto{\pgfqpoint{4.093953in}{0.494049in}}%
\pgfpathlineto{\pgfqpoint{4.096771in}{0.552553in}}%
\pgfpathlineto{\pgfqpoint{4.099589in}{0.813407in}}%
\pgfpathlineto{\pgfqpoint{4.102407in}{0.494255in}}%
\pgfpathlineto{\pgfqpoint{4.105226in}{0.783094in}}%
\pgfpathlineto{\pgfqpoint{4.108044in}{0.732377in}}%
\pgfpathlineto{\pgfqpoint{4.113680in}{0.511465in}}%
\pgfpathlineto{\pgfqpoint{4.116498in}{0.570940in}}%
\pgfpathlineto{\pgfqpoint{4.119317in}{0.570580in}}%
\pgfpathlineto{\pgfqpoint{4.122135in}{0.528173in}}%
\pgfpathlineto{\pgfqpoint{4.124953in}{0.797560in}}%
\pgfpathlineto{\pgfqpoint{4.130589in}{0.536313in}}%
\pgfpathlineto{\pgfqpoint{4.133407in}{0.477322in}}%
\pgfpathlineto{\pgfqpoint{4.136226in}{0.502581in}}%
\pgfpathlineto{\pgfqpoint{4.139044in}{0.620925in}}%
\pgfpathlineto{\pgfqpoint{4.141862in}{0.763765in}}%
\pgfpathlineto{\pgfqpoint{4.144680in}{0.603325in}}%
\pgfpathlineto{\pgfqpoint{4.147498in}{1.126582in}}%
\pgfpathlineto{\pgfqpoint{4.150317in}{0.608491in}}%
\pgfpathlineto{\pgfqpoint{4.153135in}{0.657918in}}%
\pgfpathlineto{\pgfqpoint{4.158771in}{0.896059in}}%
\pgfpathlineto{\pgfqpoint{4.161589in}{0.936110in}}%
\pgfpathlineto{\pgfqpoint{4.164407in}{0.696901in}}%
\pgfpathlineto{\pgfqpoint{4.167226in}{0.587352in}}%
\pgfpathlineto{\pgfqpoint{4.170044in}{0.553214in}}%
\pgfpathlineto{\pgfqpoint{4.172862in}{0.671658in}}%
\pgfpathlineto{\pgfqpoint{4.175680in}{0.536650in}}%
\pgfpathlineto{\pgfqpoint{4.178498in}{0.494270in}}%
\pgfpathlineto{\pgfqpoint{4.181317in}{1.213630in}}%
\pgfpathlineto{\pgfqpoint{4.184135in}{0.564454in}}%
\pgfpathlineto{\pgfqpoint{4.186953in}{0.840617in}}%
\pgfpathlineto{\pgfqpoint{4.189771in}{0.580348in}}%
\pgfpathlineto{\pgfqpoint{4.192589in}{0.562867in}}%
\pgfpathlineto{\pgfqpoint{4.195407in}{0.757407in}}%
\pgfpathlineto{\pgfqpoint{4.198226in}{0.519521in}}%
\pgfpathlineto{\pgfqpoint{4.201044in}{0.502600in}}%
\pgfpathlineto{\pgfqpoint{4.203862in}{0.824662in}}%
\pgfpathlineto{\pgfqpoint{4.206680in}{0.554010in}}%
\pgfpathlineto{\pgfqpoint{4.209498in}{0.747976in}}%
\pgfpathlineto{\pgfqpoint{4.212317in}{0.485744in}}%
\pgfpathlineto{\pgfqpoint{4.215135in}{0.714149in}}%
\pgfpathlineto{\pgfqpoint{4.217953in}{0.682011in}}%
\pgfpathlineto{\pgfqpoint{4.220771in}{0.486024in}}%
\pgfpathlineto{\pgfqpoint{4.223589in}{0.477470in}}%
\pgfpathlineto{\pgfqpoint{4.226407in}{0.735602in}}%
\pgfpathlineto{\pgfqpoint{4.229226in}{0.649841in}}%
\pgfpathlineto{\pgfqpoint{4.232044in}{0.615290in}}%
\pgfpathlineto{\pgfqpoint{4.234862in}{0.632463in}}%
\pgfpathlineto{\pgfqpoint{4.237680in}{0.546058in}}%
\pgfpathlineto{\pgfqpoint{4.240498in}{1.083228in}}%
\pgfpathlineto{\pgfqpoint{4.243317in}{0.486393in}}%
\pgfpathlineto{\pgfqpoint{4.246135in}{0.530153in}}%
\pgfpathlineto{\pgfqpoint{4.248953in}{0.495171in}}%
\pgfpathlineto{\pgfqpoint{4.251771in}{0.582442in}}%
\pgfpathlineto{\pgfqpoint{4.254589in}{0.599355in}}%
\pgfpathlineto{\pgfqpoint{4.257407in}{0.791752in}}%
\pgfpathlineto{\pgfqpoint{4.260226in}{0.739644in}}%
\pgfpathlineto{\pgfqpoint{4.263044in}{0.547247in}}%
\pgfpathlineto{\pgfqpoint{4.265862in}{0.564946in}}%
\pgfpathlineto{\pgfqpoint{4.268680in}{0.660665in}}%
\pgfpathlineto{\pgfqpoint{4.271498in}{0.642132in}}%
\pgfpathlineto{\pgfqpoint{4.274317in}{0.494825in}}%
\pgfpathlineto{\pgfqpoint{4.279953in}{0.725734in}}%
\pgfpathlineto{\pgfqpoint{4.282771in}{0.537010in}}%
\pgfpathlineto{\pgfqpoint{4.285589in}{0.562286in}}%
\pgfpathlineto{\pgfqpoint{4.288407in}{0.511315in}}%
\pgfpathlineto{\pgfqpoint{4.291226in}{0.622093in}}%
\pgfpathlineto{\pgfqpoint{4.294044in}{0.468908in}}%
\pgfpathlineto{\pgfqpoint{4.296862in}{0.657246in}}%
\pgfpathlineto{\pgfqpoint{4.299680in}{0.511884in}}%
\pgfpathlineto{\pgfqpoint{4.302498in}{0.563682in}}%
\pgfpathlineto{\pgfqpoint{4.305317in}{0.772554in}}%
\pgfpathlineto{\pgfqpoint{4.308135in}{0.651479in}}%
\pgfpathlineto{\pgfqpoint{4.310953in}{0.710550in}}%
\pgfpathlineto{\pgfqpoint{4.313771in}{0.580414in}}%
\pgfpathlineto{\pgfqpoint{4.316589in}{0.682137in}}%
\pgfpathlineto{\pgfqpoint{4.319407in}{0.579167in}}%
\pgfpathlineto{\pgfqpoint{4.322226in}{0.805544in}}%
\pgfpathlineto{\pgfqpoint{4.325044in}{0.569136in}}%
\pgfpathlineto{\pgfqpoint{4.327862in}{0.510567in}}%
\pgfpathlineto{\pgfqpoint{4.330680in}{0.857589in}}%
\pgfpathlineto{\pgfqpoint{4.333498in}{1.471876in}}%
\pgfpathlineto{\pgfqpoint{4.336317in}{0.596205in}}%
\pgfpathlineto{\pgfqpoint{4.339135in}{0.989182in}}%
\pgfpathlineto{\pgfqpoint{4.341953in}{0.692576in}}%
\pgfpathlineto{\pgfqpoint{4.344771in}{0.576383in}}%
\pgfpathlineto{\pgfqpoint{4.347589in}{0.485478in}}%
\pgfpathlineto{\pgfqpoint{4.350407in}{0.693544in}}%
\pgfpathlineto{\pgfqpoint{4.353226in}{0.602857in}}%
\pgfpathlineto{\pgfqpoint{4.356044in}{0.755634in}}%
\pgfpathlineto{\pgfqpoint{4.358862in}{0.724292in}}%
\pgfpathlineto{\pgfqpoint{4.361680in}{0.597454in}}%
\pgfpathlineto{\pgfqpoint{4.364498in}{0.546104in}}%
\pgfpathlineto{\pgfqpoint{4.367317in}{0.622683in}}%
\pgfpathlineto{\pgfqpoint{4.370135in}{1.097702in}}%
\pgfpathlineto{\pgfqpoint{4.372953in}{0.616570in}}%
\pgfpathlineto{\pgfqpoint{4.375771in}{0.615817in}}%
\pgfpathlineto{\pgfqpoint{4.378589in}{0.760498in}}%
\pgfpathlineto{\pgfqpoint{4.381407in}{0.545602in}}%
\pgfpathlineto{\pgfqpoint{4.384226in}{0.680883in}}%
\pgfpathlineto{\pgfqpoint{4.389862in}{0.468908in}}%
\pgfpathlineto{\pgfqpoint{4.392680in}{0.536313in}}%
\pgfpathlineto{\pgfqpoint{4.395498in}{0.519565in}}%
\pgfpathlineto{\pgfqpoint{4.398317in}{0.477357in}}%
\pgfpathlineto{\pgfqpoint{4.401135in}{0.620570in}}%
\pgfpathlineto{\pgfqpoint{4.403953in}{0.485710in}}%
\pgfpathlineto{\pgfqpoint{4.406771in}{0.637372in}}%
\pgfpathlineto{\pgfqpoint{4.409589in}{0.578521in}}%
\pgfpathlineto{\pgfqpoint{4.412407in}{0.645100in}}%
\pgfpathlineto{\pgfqpoint{4.415226in}{1.086079in}}%
\pgfpathlineto{\pgfqpoint{4.418044in}{0.485994in}}%
\pgfpathlineto{\pgfqpoint{4.420862in}{0.588228in}}%
\pgfpathlineto{\pgfqpoint{4.423680in}{0.750956in}}%
\pgfpathlineto{\pgfqpoint{4.429317in}{0.546800in}}%
\pgfpathlineto{\pgfqpoint{4.432135in}{0.710550in}}%
\pgfpathlineto{\pgfqpoint{4.434953in}{0.997970in}}%
\pgfpathlineto{\pgfqpoint{4.437771in}{0.706220in}}%
\pgfpathlineto{\pgfqpoint{4.443407in}{0.611384in}}%
\pgfpathlineto{\pgfqpoint{4.446226in}{0.567270in}}%
\pgfpathlineto{\pgfqpoint{4.449044in}{0.829447in}}%
\pgfpathlineto{\pgfqpoint{4.454680in}{1.027579in}}%
\pgfpathlineto{\pgfqpoint{4.457498in}{0.641267in}}%
\pgfpathlineto{\pgfqpoint{4.460317in}{0.714149in}}%
\pgfpathlineto{\pgfqpoint{4.463135in}{1.216241in}}%
\pgfpathlineto{\pgfqpoint{4.465953in}{1.070660in}}%
\pgfpathlineto{\pgfqpoint{4.468771in}{0.685908in}}%
\pgfpathlineto{\pgfqpoint{4.471589in}{0.667762in}}%
\pgfpathlineto{\pgfqpoint{4.477226in}{0.746458in}}%
\pgfpathlineto{\pgfqpoint{4.480044in}{0.477819in}}%
\pgfpathlineto{\pgfqpoint{4.482862in}{0.840843in}}%
\pgfpathlineto{\pgfqpoint{4.485680in}{0.645421in}}%
\pgfpathlineto{\pgfqpoint{4.488498in}{0.513138in}}%
\pgfpathlineto{\pgfqpoint{4.491317in}{0.495413in}}%
\pgfpathlineto{\pgfqpoint{4.494135in}{0.627430in}}%
\pgfpathlineto{\pgfqpoint{4.496953in}{0.486478in}}%
\pgfpathlineto{\pgfqpoint{4.499771in}{0.654048in}}%
\pgfpathlineto{\pgfqpoint{4.502589in}{0.610075in}}%
\pgfpathlineto{\pgfqpoint{4.505407in}{0.530452in}}%
\pgfpathlineto{\pgfqpoint{4.508226in}{0.652755in}}%
\pgfpathlineto{\pgfqpoint{4.511044in}{0.746819in}}%
\pgfpathlineto{\pgfqpoint{4.513862in}{0.598262in}}%
\pgfpathlineto{\pgfqpoint{4.516680in}{1.174357in}}%
\pgfpathlineto{\pgfqpoint{4.519498in}{0.703047in}}%
\pgfpathlineto{\pgfqpoint{4.522317in}{0.560373in}}%
\pgfpathlineto{\pgfqpoint{4.525135in}{0.618730in}}%
\pgfpathlineto{\pgfqpoint{4.527953in}{0.577191in}}%
\pgfpathlineto{\pgfqpoint{4.530771in}{0.485540in}}%
\pgfpathlineto{\pgfqpoint{4.533589in}{0.609978in}}%
\pgfpathlineto{\pgfqpoint{4.536407in}{0.827079in}}%
\pgfpathlineto{\pgfqpoint{4.539226in}{0.687656in}}%
\pgfpathlineto{\pgfqpoint{4.542044in}{0.846467in}}%
\pgfpathlineto{\pgfqpoint{4.547680in}{0.620747in}}%
\pgfpathlineto{\pgfqpoint{4.550498in}{0.989508in}}%
\pgfpathlineto{\pgfqpoint{4.553317in}{0.598378in}}%
\pgfpathlineto{\pgfqpoint{4.556135in}{0.546870in}}%
\pgfpathlineto{\pgfqpoint{4.558953in}{0.695324in}}%
\pgfpathlineto{\pgfqpoint{4.561771in}{0.477649in}}%
\pgfpathlineto{\pgfqpoint{4.564589in}{0.512575in}}%
\pgfpathlineto{\pgfqpoint{4.567407in}{0.503836in}}%
\pgfpathlineto{\pgfqpoint{4.570226in}{0.486388in}}%
\pgfpathlineto{\pgfqpoint{4.575862in}{0.599434in}}%
\pgfpathlineto{\pgfqpoint{4.578680in}{0.616126in}}%
\pgfpathlineto{\pgfqpoint{4.581498in}{0.486178in}}%
\pgfpathlineto{\pgfqpoint{4.584317in}{0.529400in}}%
\pgfpathlineto{\pgfqpoint{4.587135in}{1.076658in}}%
\pgfpathlineto{\pgfqpoint{4.589953in}{0.647334in}}%
\pgfpathlineto{\pgfqpoint{4.592771in}{0.751543in}}%
\pgfpathlineto{\pgfqpoint{4.595589in}{0.938438in}}%
\pgfpathlineto{\pgfqpoint{4.598407in}{0.721827in}}%
\pgfpathlineto{\pgfqpoint{4.601226in}{0.661320in}}%
\pgfpathlineto{\pgfqpoint{4.604044in}{0.585402in}}%
\pgfpathlineto{\pgfqpoint{4.606862in}{0.881157in}}%
\pgfpathlineto{\pgfqpoint{4.609680in}{0.468908in}}%
\pgfpathlineto{\pgfqpoint{4.612498in}{0.648446in}}%
\pgfpathlineto{\pgfqpoint{4.615317in}{0.591198in}}%
\pgfpathlineto{\pgfqpoint{4.618135in}{0.566782in}}%
\pgfpathlineto{\pgfqpoint{4.620953in}{0.485197in}}%
\pgfpathlineto{\pgfqpoint{4.623771in}{0.485206in}}%
\pgfpathlineto{\pgfqpoint{4.626589in}{0.712422in}}%
\pgfpathlineto{\pgfqpoint{4.629407in}{0.517365in}}%
\pgfpathlineto{\pgfqpoint{4.632226in}{0.693969in}}%
\pgfpathlineto{\pgfqpoint{4.635044in}{0.660428in}}%
\pgfpathlineto{\pgfqpoint{4.637862in}{1.183741in}}%
\pgfpathlineto{\pgfqpoint{4.640680in}{0.716122in}}%
\pgfpathlineto{\pgfqpoint{4.643498in}{0.499694in}}%
\pgfpathlineto{\pgfqpoint{4.646317in}{0.614849in}}%
\pgfpathlineto{\pgfqpoint{4.649135in}{0.537781in}}%
\pgfpathlineto{\pgfqpoint{4.651953in}{0.636574in}}%
\pgfpathlineto{\pgfqpoint{4.654771in}{0.621292in}}%
\pgfpathlineto{\pgfqpoint{4.657589in}{0.722116in}}%
\pgfpathlineto{\pgfqpoint{4.660407in}{0.553807in}}%
\pgfpathlineto{\pgfqpoint{4.663226in}{0.822291in}}%
\pgfpathlineto{\pgfqpoint{4.666044in}{0.529926in}}%
\pgfpathlineto{\pgfqpoint{4.668862in}{0.491756in}}%
\pgfpathlineto{\pgfqpoint{4.671680in}{0.801954in}}%
\pgfpathlineto{\pgfqpoint{4.674498in}{0.582014in}}%
\pgfpathlineto{\pgfqpoint{4.677317in}{0.709667in}}%
\pgfpathlineto{\pgfqpoint{4.680135in}{0.752221in}}%
\pgfpathlineto{\pgfqpoint{4.682953in}{0.557799in}}%
\pgfpathlineto{\pgfqpoint{4.685771in}{0.594919in}}%
\pgfpathlineto{\pgfqpoint{4.688589in}{0.874614in}}%
\pgfpathlineto{\pgfqpoint{4.691407in}{2.518450in}}%
\pgfpathlineto{\pgfqpoint{4.694226in}{0.489381in}}%
\pgfpathlineto{\pgfqpoint{4.697044in}{1.190265in}}%
\pgfpathlineto{\pgfqpoint{4.699862in}{1.231064in}}%
\pgfpathlineto{\pgfqpoint{4.702680in}{0.707829in}}%
\pgfpathlineto{\pgfqpoint{4.705498in}{0.942211in}}%
\pgfpathlineto{\pgfqpoint{4.708317in}{0.864939in}}%
\pgfpathlineto{\pgfqpoint{4.711135in}{0.520445in}}%
\pgfpathlineto{\pgfqpoint{4.713953in}{0.520445in}}%
\pgfpathlineto{\pgfqpoint{4.719589in}{0.921160in}}%
\pgfpathlineto{\pgfqpoint{4.722407in}{0.574412in}}%
\pgfpathlineto{\pgfqpoint{4.725226in}{0.541218in}}%
\pgfpathlineto{\pgfqpoint{4.728044in}{0.726104in}}%
\pgfpathlineto{\pgfqpoint{4.730862in}{0.587899in}}%
\pgfpathlineto{\pgfqpoint{4.733680in}{1.116030in}}%
\pgfpathlineto{\pgfqpoint{4.736498in}{0.475655in}}%
\pgfpathlineto{\pgfqpoint{4.739317in}{0.677353in}}%
\pgfpathlineto{\pgfqpoint{4.742135in}{0.657115in}}%
\pgfpathlineto{\pgfqpoint{4.744953in}{0.529537in}}%
\pgfpathlineto{\pgfqpoint{4.747771in}{0.637638in}}%
\pgfpathlineto{\pgfqpoint{4.750589in}{0.624176in}}%
\pgfpathlineto{\pgfqpoint{4.753407in}{0.971481in}}%
\pgfpathlineto{\pgfqpoint{4.756226in}{1.045445in}}%
\pgfpathlineto{\pgfqpoint{4.759044in}{0.569815in}}%
\pgfpathlineto{\pgfqpoint{4.761862in}{0.597233in}}%
\pgfpathlineto{\pgfqpoint{4.764680in}{1.153887in}}%
\pgfpathlineto{\pgfqpoint{4.767498in}{0.475839in}}%
\pgfpathlineto{\pgfqpoint{4.770317in}{0.712491in}}%
\pgfpathlineto{\pgfqpoint{4.773135in}{0.643106in}}%
\pgfpathlineto{\pgfqpoint{4.775953in}{0.676565in}}%
\pgfpathlineto{\pgfqpoint{4.778771in}{0.690462in}}%
\pgfpathlineto{\pgfqpoint{4.781589in}{0.475857in}}%
\pgfpathlineto{\pgfqpoint{4.784407in}{0.517589in}}%
\pgfpathlineto{\pgfqpoint{4.787226in}{1.319314in}}%
\pgfpathlineto{\pgfqpoint{4.790044in}{1.549914in}}%
\pgfpathlineto{\pgfqpoint{4.792862in}{0.946599in}}%
\pgfpathlineto{\pgfqpoint{4.795680in}{0.787246in}}%
\pgfpathlineto{\pgfqpoint{4.798498in}{0.954183in}}%
\pgfpathlineto{\pgfqpoint{4.801317in}{0.507569in}}%
\pgfpathlineto{\pgfqpoint{4.804135in}{0.851868in}}%
\pgfpathlineto{\pgfqpoint{4.806953in}{0.481973in}}%
\pgfpathlineto{\pgfqpoint{4.809771in}{0.540662in}}%
\pgfpathlineto{\pgfqpoint{4.812589in}{0.508024in}}%
\pgfpathlineto{\pgfqpoint{4.812589in}{0.508024in}}%
\pgfusepath{stroke}%
\end{pgfscope}%
\begin{pgfscope}%
\pgfsetrectcap%
\pgfsetmiterjoin%
\pgfsetlinewidth{0.803000pt}%
\definecolor{currentstroke}{rgb}{1.000000,1.000000,1.000000}%
\pgfsetstrokecolor{currentstroke}%
\pgfsetdash{}{0pt}%
\pgfpathmoveto{\pgfqpoint{0.373953in}{0.331635in}}%
\pgfpathlineto{\pgfqpoint{0.373953in}{3.351635in}}%
\pgfusepath{stroke}%
\end{pgfscope}%
\begin{pgfscope}%
\pgfsetrectcap%
\pgfsetmiterjoin%
\pgfsetlinewidth{0.803000pt}%
\definecolor{currentstroke}{rgb}{1.000000,1.000000,1.000000}%
\pgfsetstrokecolor{currentstroke}%
\pgfsetdash{}{0pt}%
\pgfpathmoveto{\pgfqpoint{5.023953in}{0.331635in}}%
\pgfpathlineto{\pgfqpoint{5.023953in}{3.351635in}}%
\pgfusepath{stroke}%
\end{pgfscope}%
\begin{pgfscope}%
\pgfsetrectcap%
\pgfsetmiterjoin%
\pgfsetlinewidth{0.803000pt}%
\definecolor{currentstroke}{rgb}{1.000000,1.000000,1.000000}%
\pgfsetstrokecolor{currentstroke}%
\pgfsetdash{}{0pt}%
\pgfpathmoveto{\pgfqpoint{0.373953in}{0.331635in}}%
\pgfpathlineto{\pgfqpoint{5.023953in}{0.331635in}}%
\pgfusepath{stroke}%
\end{pgfscope}%
\begin{pgfscope}%
\pgfsetrectcap%
\pgfsetmiterjoin%
\pgfsetlinewidth{0.803000pt}%
\definecolor{currentstroke}{rgb}{1.000000,1.000000,1.000000}%
\pgfsetstrokecolor{currentstroke}%
\pgfsetdash{}{0pt}%
\pgfpathmoveto{\pgfqpoint{0.373953in}{3.351635in}}%
\pgfpathlineto{\pgfqpoint{5.023953in}{3.351635in}}%
\pgfusepath{stroke}%
\end{pgfscope}%
\end{pgfpicture}%
\makeatother%
\endgroup%

    \end{adjustbox}
    \caption{Volatility forecasts (shown in blue) are laid over the absolute value of log-returns (in orange) for V (left) and INTC (right). For V, the GJR-GARCH(1,1) model was used, for INTC, the GARCH(1,1) model was used (both using students-t-distributed resiudals).}
    \label{fig:V_INTC_ARMA_predictions_plot}
\end{figure}{}


This is useful for balancing portfolios, pricing options and XXX [THEORY!]. As the main focus of this paper is the prediction of returns, volatility forecasts will not be further discussed in detail. 

\subsection{\textcolor{red}{Weighted Average of Predictions}}
To stabilize predictions and in trying to deal with the inherent uncertainty about the correct model to apply we also considered averaging over predictions generated by different model. In the specific case, we only had identified two feasible models for V over which to average. The resulting SSE, 0.2451 was in between the SSE for ARMA(0,0), 0.2456 and the SSE for ARMA(1,1), 0.2445. The binary prediction accuracy was by chance exactly the same as the prediction accuracy for the naive strategy of always predicting positive returns. 

\section{\textcolor{red}{Time Series Predictions with the Remaining Stocks - Nikos}}

As we are trying to approach the problem from the perspective of a (presumably a bit naive) trader who is trying to predict future stocks, we are faced with the following problem: We cannot know beforehand which model we shall apply before looking at the data. But if we rely too much on past observations we risk applying fancy models to plain white noise. 

First we could apply different models to a pre-specified past of a time series, see which one fits best and then hope that the future will behave similar to the past. Secondly we could do the same, but iteratively refit the time series every day to produce the next forecast with the current best guess. However, it remains unclear why the future of the time series should behave like its past. Thirdly, we could use a weighted average of different plausible models with the idea in mind that we cannot know which model should apply. Seen from the angle of economic theory, all three methods somewhat resemble reading the leaves or looking into the crystal ball. Alas the last one at least has the beauty of approaching the problem from a Bayesian perspective on future paths. However, it remains unclear which models should be included and what weight they should be given and it seems quite impossible to correctly figure that out. As traders all over the world are indeed approaching the problem like this, we are going to follow their lead and see what comes out. 

Figure\ref{fig:SSE_all_predictions} shows the SSE for a variety of different models for all 10 stocks. To assess the performance of the different models, those were given a rank according to their performance on a single stock (judged by SSE). Those ranks were summed up to produce figure \ref{fig:SSE_models_ranked}. We can see that none of the models consistently beat ARMA(0,0), not even the auto-ARMA model that iteratively adjusts its fit on past observations.

\begin{figure}[h!]
    \centering
    \figuretitle{SSE of Predcitions from Different Models}
    \begin{adjustbox}{width = 0.95\linewidth}
    %% Creator: Matplotlib, PGF backend
%%
%% To include the figure in your LaTeX document, write
%%   \input{<filename>.pgf}
%%
%% Make sure the required packages are loaded in your preamble
%%   \usepackage{pgf}
%%
%% Figures using additional raster images can only be included by \input if
%% they are in the same directory as the main LaTeX file. For loading figures
%% from other directories you can use the `import` package
%%   \usepackage{import}
%% and then include the figures with
%%   \import{<path to file>}{<filename>.pgf}
%%
%% Matplotlib used the following preamble
%%   \usepackage{fontspec}
%%   \setmainfont{DejaVuSerif.ttf}[Path=/opt/tljh/user/lib/python3.6/site-packages/matplotlib/mpl-data/fonts/ttf/]
%%   \setsansfont{DejaVuSans.ttf}[Path=/opt/tljh/user/lib/python3.6/site-packages/matplotlib/mpl-data/fonts/ttf/]
%%   \setmonofont{DejaVuSansMono.ttf}[Path=/opt/tljh/user/lib/python3.6/site-packages/matplotlib/mpl-data/fonts/ttf/]
%%
\begingroup%
\makeatletter%
\begin{pgfpicture}%
\pgfpathrectangle{\pgfpointorigin}{\pgfqpoint{6.894832in}{7.159506in}}%
\pgfusepath{use as bounding box, clip}%
\begin{pgfscope}%
\pgfsetbuttcap%
\pgfsetmiterjoin%
\definecolor{currentfill}{rgb}{1.000000,1.000000,1.000000}%
\pgfsetfillcolor{currentfill}%
\pgfsetlinewidth{0.000000pt}%
\definecolor{currentstroke}{rgb}{1.000000,1.000000,1.000000}%
\pgfsetstrokecolor{currentstroke}%
\pgfsetdash{}{0pt}%
\pgfpathmoveto{\pgfqpoint{0.000000in}{0.000000in}}%
\pgfpathlineto{\pgfqpoint{6.894832in}{0.000000in}}%
\pgfpathlineto{\pgfqpoint{6.894832in}{7.159506in}}%
\pgfpathlineto{\pgfqpoint{0.000000in}{7.159506in}}%
\pgfpathclose%
\pgfusepath{fill}%
\end{pgfscope}%
\begin{pgfscope}%
\pgfsetbuttcap%
\pgfsetmiterjoin%
\definecolor{currentfill}{rgb}{0.917647,0.917647,0.949020}%
\pgfsetfillcolor{currentfill}%
\pgfsetlinewidth{0.000000pt}%
\definecolor{currentstroke}{rgb}{0.000000,0.000000,0.000000}%
\pgfsetstrokecolor{currentstroke}%
\pgfsetstrokeopacity{0.000000}%
\pgfsetdash{}{0pt}%
\pgfpathmoveto{\pgfqpoint{0.594832in}{6.048788in}}%
\pgfpathlineto{\pgfqpoint{3.413014in}{6.048788in}}%
\pgfpathlineto{\pgfqpoint{3.413014in}{6.849545in}}%
\pgfpathlineto{\pgfqpoint{0.594832in}{6.849545in}}%
\pgfpathclose%
\pgfusepath{fill}%
\end{pgfscope}%
\begin{pgfscope}%
\pgfpathrectangle{\pgfqpoint{0.594832in}{6.048788in}}{\pgfqpoint{2.818182in}{0.800758in}}%
\pgfusepath{clip}%
\pgfsetroundcap%
\pgfsetroundjoin%
\pgfsetlinewidth{0.803000pt}%
\definecolor{currentstroke}{rgb}{1.000000,1.000000,1.000000}%
\pgfsetstrokecolor{currentstroke}%
\pgfsetdash{}{0pt}%
\pgfpathmoveto{\pgfqpoint{0.722932in}{6.048788in}}%
\pgfpathlineto{\pgfqpoint{0.722932in}{6.849545in}}%
\pgfusepath{stroke}%
\end{pgfscope}%
\begin{pgfscope}%
\pgfpathrectangle{\pgfqpoint{0.594832in}{6.048788in}}{\pgfqpoint{2.818182in}{0.800758in}}%
\pgfusepath{clip}%
\pgfsetroundcap%
\pgfsetroundjoin%
\pgfsetlinewidth{0.803000pt}%
\definecolor{currentstroke}{rgb}{1.000000,1.000000,1.000000}%
\pgfsetstrokecolor{currentstroke}%
\pgfsetdash{}{0pt}%
\pgfpathmoveto{\pgfqpoint{0.920007in}{6.048788in}}%
\pgfpathlineto{\pgfqpoint{0.920007in}{6.849545in}}%
\pgfusepath{stroke}%
\end{pgfscope}%
\begin{pgfscope}%
\pgfpathrectangle{\pgfqpoint{0.594832in}{6.048788in}}{\pgfqpoint{2.818182in}{0.800758in}}%
\pgfusepath{clip}%
\pgfsetroundcap%
\pgfsetroundjoin%
\pgfsetlinewidth{0.803000pt}%
\definecolor{currentstroke}{rgb}{1.000000,1.000000,1.000000}%
\pgfsetstrokecolor{currentstroke}%
\pgfsetdash{}{0pt}%
\pgfpathmoveto{\pgfqpoint{1.117083in}{6.048788in}}%
\pgfpathlineto{\pgfqpoint{1.117083in}{6.849545in}}%
\pgfusepath{stroke}%
\end{pgfscope}%
\begin{pgfscope}%
\pgfpathrectangle{\pgfqpoint{0.594832in}{6.048788in}}{\pgfqpoint{2.818182in}{0.800758in}}%
\pgfusepath{clip}%
\pgfsetroundcap%
\pgfsetroundjoin%
\pgfsetlinewidth{0.803000pt}%
\definecolor{currentstroke}{rgb}{1.000000,1.000000,1.000000}%
\pgfsetstrokecolor{currentstroke}%
\pgfsetdash{}{0pt}%
\pgfpathmoveto{\pgfqpoint{1.314158in}{6.048788in}}%
\pgfpathlineto{\pgfqpoint{1.314158in}{6.849545in}}%
\pgfusepath{stroke}%
\end{pgfscope}%
\begin{pgfscope}%
\pgfpathrectangle{\pgfqpoint{0.594832in}{6.048788in}}{\pgfqpoint{2.818182in}{0.800758in}}%
\pgfusepath{clip}%
\pgfsetroundcap%
\pgfsetroundjoin%
\pgfsetlinewidth{0.803000pt}%
\definecolor{currentstroke}{rgb}{1.000000,1.000000,1.000000}%
\pgfsetstrokecolor{currentstroke}%
\pgfsetdash{}{0pt}%
\pgfpathmoveto{\pgfqpoint{1.511234in}{6.048788in}}%
\pgfpathlineto{\pgfqpoint{1.511234in}{6.849545in}}%
\pgfusepath{stroke}%
\end{pgfscope}%
\begin{pgfscope}%
\pgfpathrectangle{\pgfqpoint{0.594832in}{6.048788in}}{\pgfqpoint{2.818182in}{0.800758in}}%
\pgfusepath{clip}%
\pgfsetroundcap%
\pgfsetroundjoin%
\pgfsetlinewidth{0.803000pt}%
\definecolor{currentstroke}{rgb}{1.000000,1.000000,1.000000}%
\pgfsetstrokecolor{currentstroke}%
\pgfsetdash{}{0pt}%
\pgfpathmoveto{\pgfqpoint{1.708310in}{6.048788in}}%
\pgfpathlineto{\pgfqpoint{1.708310in}{6.849545in}}%
\pgfusepath{stroke}%
\end{pgfscope}%
\begin{pgfscope}%
\pgfpathrectangle{\pgfqpoint{0.594832in}{6.048788in}}{\pgfqpoint{2.818182in}{0.800758in}}%
\pgfusepath{clip}%
\pgfsetroundcap%
\pgfsetroundjoin%
\pgfsetlinewidth{0.803000pt}%
\definecolor{currentstroke}{rgb}{1.000000,1.000000,1.000000}%
\pgfsetstrokecolor{currentstroke}%
\pgfsetdash{}{0pt}%
\pgfpathmoveto{\pgfqpoint{1.905385in}{6.048788in}}%
\pgfpathlineto{\pgfqpoint{1.905385in}{6.849545in}}%
\pgfusepath{stroke}%
\end{pgfscope}%
\begin{pgfscope}%
\pgfpathrectangle{\pgfqpoint{0.594832in}{6.048788in}}{\pgfqpoint{2.818182in}{0.800758in}}%
\pgfusepath{clip}%
\pgfsetroundcap%
\pgfsetroundjoin%
\pgfsetlinewidth{0.803000pt}%
\definecolor{currentstroke}{rgb}{1.000000,1.000000,1.000000}%
\pgfsetstrokecolor{currentstroke}%
\pgfsetdash{}{0pt}%
\pgfpathmoveto{\pgfqpoint{2.102461in}{6.048788in}}%
\pgfpathlineto{\pgfqpoint{2.102461in}{6.849545in}}%
\pgfusepath{stroke}%
\end{pgfscope}%
\begin{pgfscope}%
\pgfpathrectangle{\pgfqpoint{0.594832in}{6.048788in}}{\pgfqpoint{2.818182in}{0.800758in}}%
\pgfusepath{clip}%
\pgfsetroundcap%
\pgfsetroundjoin%
\pgfsetlinewidth{0.803000pt}%
\definecolor{currentstroke}{rgb}{1.000000,1.000000,1.000000}%
\pgfsetstrokecolor{currentstroke}%
\pgfsetdash{}{0pt}%
\pgfpathmoveto{\pgfqpoint{2.299537in}{6.048788in}}%
\pgfpathlineto{\pgfqpoint{2.299537in}{6.849545in}}%
\pgfusepath{stroke}%
\end{pgfscope}%
\begin{pgfscope}%
\pgfpathrectangle{\pgfqpoint{0.594832in}{6.048788in}}{\pgfqpoint{2.818182in}{0.800758in}}%
\pgfusepath{clip}%
\pgfsetroundcap%
\pgfsetroundjoin%
\pgfsetlinewidth{0.803000pt}%
\definecolor{currentstroke}{rgb}{1.000000,1.000000,1.000000}%
\pgfsetstrokecolor{currentstroke}%
\pgfsetdash{}{0pt}%
\pgfpathmoveto{\pgfqpoint{2.496612in}{6.048788in}}%
\pgfpathlineto{\pgfqpoint{2.496612in}{6.849545in}}%
\pgfusepath{stroke}%
\end{pgfscope}%
\begin{pgfscope}%
\pgfpathrectangle{\pgfqpoint{0.594832in}{6.048788in}}{\pgfqpoint{2.818182in}{0.800758in}}%
\pgfusepath{clip}%
\pgfsetroundcap%
\pgfsetroundjoin%
\pgfsetlinewidth{0.803000pt}%
\definecolor{currentstroke}{rgb}{1.000000,1.000000,1.000000}%
\pgfsetstrokecolor{currentstroke}%
\pgfsetdash{}{0pt}%
\pgfpathmoveto{\pgfqpoint{2.693688in}{6.048788in}}%
\pgfpathlineto{\pgfqpoint{2.693688in}{6.849545in}}%
\pgfusepath{stroke}%
\end{pgfscope}%
\begin{pgfscope}%
\pgfpathrectangle{\pgfqpoint{0.594832in}{6.048788in}}{\pgfqpoint{2.818182in}{0.800758in}}%
\pgfusepath{clip}%
\pgfsetroundcap%
\pgfsetroundjoin%
\pgfsetlinewidth{0.803000pt}%
\definecolor{currentstroke}{rgb}{1.000000,1.000000,1.000000}%
\pgfsetstrokecolor{currentstroke}%
\pgfsetdash{}{0pt}%
\pgfpathmoveto{\pgfqpoint{2.890764in}{6.048788in}}%
\pgfpathlineto{\pgfqpoint{2.890764in}{6.849545in}}%
\pgfusepath{stroke}%
\end{pgfscope}%
\begin{pgfscope}%
\pgfpathrectangle{\pgfqpoint{0.594832in}{6.048788in}}{\pgfqpoint{2.818182in}{0.800758in}}%
\pgfusepath{clip}%
\pgfsetroundcap%
\pgfsetroundjoin%
\pgfsetlinewidth{0.803000pt}%
\definecolor{currentstroke}{rgb}{1.000000,1.000000,1.000000}%
\pgfsetstrokecolor{currentstroke}%
\pgfsetdash{}{0pt}%
\pgfpathmoveto{\pgfqpoint{3.087839in}{6.048788in}}%
\pgfpathlineto{\pgfqpoint{3.087839in}{6.849545in}}%
\pgfusepath{stroke}%
\end{pgfscope}%
\begin{pgfscope}%
\pgfpathrectangle{\pgfqpoint{0.594832in}{6.048788in}}{\pgfqpoint{2.818182in}{0.800758in}}%
\pgfusepath{clip}%
\pgfsetroundcap%
\pgfsetroundjoin%
\pgfsetlinewidth{0.803000pt}%
\definecolor{currentstroke}{rgb}{1.000000,1.000000,1.000000}%
\pgfsetstrokecolor{currentstroke}%
\pgfsetdash{}{0pt}%
\pgfpathmoveto{\pgfqpoint{3.284915in}{6.048788in}}%
\pgfpathlineto{\pgfqpoint{3.284915in}{6.849545in}}%
\pgfusepath{stroke}%
\end{pgfscope}%
\begin{pgfscope}%
\pgfpathrectangle{\pgfqpoint{0.594832in}{6.048788in}}{\pgfqpoint{2.818182in}{0.800758in}}%
\pgfusepath{clip}%
\pgfsetroundcap%
\pgfsetroundjoin%
\pgfsetlinewidth{0.803000pt}%
\definecolor{currentstroke}{rgb}{1.000000,1.000000,1.000000}%
\pgfsetstrokecolor{currentstroke}%
\pgfsetdash{}{0pt}%
\pgfpathmoveto{\pgfqpoint{0.594832in}{6.162266in}}%
\pgfpathlineto{\pgfqpoint{3.413014in}{6.162266in}}%
\pgfusepath{stroke}%
\end{pgfscope}%
\begin{pgfscope}%
\definecolor{textcolor}{rgb}{0.150000,0.150000,0.150000}%
\pgfsetstrokecolor{textcolor}%
\pgfsetfillcolor{textcolor}%
\pgftext[x=0.100000in,y=6.109505in,left,base]{\color{textcolor}\rmfamily\fontsize{10.000000}{12.000000}\selectfont 0.134}%
\end{pgfscope}%
\begin{pgfscope}%
\pgfpathrectangle{\pgfqpoint{0.594832in}{6.048788in}}{\pgfqpoint{2.818182in}{0.800758in}}%
\pgfusepath{clip}%
\pgfsetroundcap%
\pgfsetroundjoin%
\pgfsetlinewidth{0.803000pt}%
\definecolor{currentstroke}{rgb}{1.000000,1.000000,1.000000}%
\pgfsetstrokecolor{currentstroke}%
\pgfsetdash{}{0pt}%
\pgfpathmoveto{\pgfqpoint{0.594832in}{6.572310in}}%
\pgfpathlineto{\pgfqpoint{3.413014in}{6.572310in}}%
\pgfusepath{stroke}%
\end{pgfscope}%
\begin{pgfscope}%
\definecolor{textcolor}{rgb}{0.150000,0.150000,0.150000}%
\pgfsetstrokecolor{textcolor}%
\pgfsetfillcolor{textcolor}%
\pgftext[x=0.100000in,y=6.519548in,left,base]{\color{textcolor}\rmfamily\fontsize{10.000000}{12.000000}\selectfont 0.135}%
\end{pgfscope}%
\begin{pgfscope}%
\pgfpathrectangle{\pgfqpoint{0.594832in}{6.048788in}}{\pgfqpoint{2.818182in}{0.800758in}}%
\pgfusepath{clip}%
\pgfsetroundcap%
\pgfsetroundjoin%
\pgfsetlinewidth{1.505625pt}%
\definecolor{currentstroke}{rgb}{0.121569,0.466667,0.705882}%
\pgfsetstrokecolor{currentstroke}%
\pgfsetdash{}{0pt}%
\pgfpathmoveto{\pgfqpoint{0.722932in}{6.085186in}}%
\pgfpathlineto{\pgfqpoint{0.920007in}{6.274983in}}%
\pgfpathlineto{\pgfqpoint{1.117083in}{6.708244in}}%
\pgfpathlineto{\pgfqpoint{1.314158in}{6.303104in}}%
\pgfpathlineto{\pgfqpoint{1.511234in}{6.245508in}}%
\pgfpathlineto{\pgfqpoint{1.708310in}{6.584745in}}%
\pgfpathlineto{\pgfqpoint{1.905385in}{6.396406in}}%
\pgfpathlineto{\pgfqpoint{2.102461in}{6.303104in}}%
\pgfpathlineto{\pgfqpoint{2.299537in}{6.521878in}}%
\pgfpathlineto{\pgfqpoint{2.496612in}{6.813147in}}%
\pgfpathlineto{\pgfqpoint{2.693688in}{6.547536in}}%
\pgfpathlineto{\pgfqpoint{2.890764in}{6.303104in}}%
\pgfpathlineto{\pgfqpoint{3.087839in}{6.303104in}}%
\pgfpathlineto{\pgfqpoint{3.284915in}{6.149519in}}%
\pgfusepath{stroke}%
\end{pgfscope}%
\begin{pgfscope}%
\pgfpathrectangle{\pgfqpoint{0.594832in}{6.048788in}}{\pgfqpoint{2.818182in}{0.800758in}}%
\pgfusepath{clip}%
\pgfsetbuttcap%
\pgfsetroundjoin%
\definecolor{currentfill}{rgb}{0.121569,0.466667,0.705882}%
\pgfsetfillcolor{currentfill}%
\pgfsetlinewidth{1.003750pt}%
\definecolor{currentstroke}{rgb}{0.121569,0.466667,0.705882}%
\pgfsetstrokecolor{currentstroke}%
\pgfsetdash{}{0pt}%
\pgfsys@defobject{currentmarker}{\pgfqpoint{-0.041667in}{-0.041667in}}{\pgfqpoint{0.041667in}{0.041667in}}{%
\pgfpathmoveto{\pgfqpoint{0.000000in}{-0.041667in}}%
\pgfpathcurveto{\pgfqpoint{0.011050in}{-0.041667in}}{\pgfqpoint{0.021649in}{-0.037276in}}{\pgfqpoint{0.029463in}{-0.029463in}}%
\pgfpathcurveto{\pgfqpoint{0.037276in}{-0.021649in}}{\pgfqpoint{0.041667in}{-0.011050in}}{\pgfqpoint{0.041667in}{0.000000in}}%
\pgfpathcurveto{\pgfqpoint{0.041667in}{0.011050in}}{\pgfqpoint{0.037276in}{0.021649in}}{\pgfqpoint{0.029463in}{0.029463in}}%
\pgfpathcurveto{\pgfqpoint{0.021649in}{0.037276in}}{\pgfqpoint{0.011050in}{0.041667in}}{\pgfqpoint{0.000000in}{0.041667in}}%
\pgfpathcurveto{\pgfqpoint{-0.011050in}{0.041667in}}{\pgfqpoint{-0.021649in}{0.037276in}}{\pgfqpoint{-0.029463in}{0.029463in}}%
\pgfpathcurveto{\pgfqpoint{-0.037276in}{0.021649in}}{\pgfqpoint{-0.041667in}{0.011050in}}{\pgfqpoint{-0.041667in}{0.000000in}}%
\pgfpathcurveto{\pgfqpoint{-0.041667in}{-0.011050in}}{\pgfqpoint{-0.037276in}{-0.021649in}}{\pgfqpoint{-0.029463in}{-0.029463in}}%
\pgfpathcurveto{\pgfqpoint{-0.021649in}{-0.037276in}}{\pgfqpoint{-0.011050in}{-0.041667in}}{\pgfqpoint{0.000000in}{-0.041667in}}%
\pgfpathclose%
\pgfusepath{stroke,fill}%
}%
\begin{pgfscope}%
\pgfsys@transformshift{0.722932in}{6.085186in}%
\pgfsys@useobject{currentmarker}{}%
\end{pgfscope}%
\begin{pgfscope}%
\pgfsys@transformshift{0.920007in}{6.274983in}%
\pgfsys@useobject{currentmarker}{}%
\end{pgfscope}%
\begin{pgfscope}%
\pgfsys@transformshift{1.117083in}{6.708244in}%
\pgfsys@useobject{currentmarker}{}%
\end{pgfscope}%
\begin{pgfscope}%
\pgfsys@transformshift{1.314158in}{6.303104in}%
\pgfsys@useobject{currentmarker}{}%
\end{pgfscope}%
\begin{pgfscope}%
\pgfsys@transformshift{1.511234in}{6.245508in}%
\pgfsys@useobject{currentmarker}{}%
\end{pgfscope}%
\begin{pgfscope}%
\pgfsys@transformshift{1.708310in}{6.584745in}%
\pgfsys@useobject{currentmarker}{}%
\end{pgfscope}%
\begin{pgfscope}%
\pgfsys@transformshift{1.905385in}{6.396406in}%
\pgfsys@useobject{currentmarker}{}%
\end{pgfscope}%
\begin{pgfscope}%
\pgfsys@transformshift{2.102461in}{6.303104in}%
\pgfsys@useobject{currentmarker}{}%
\end{pgfscope}%
\begin{pgfscope}%
\pgfsys@transformshift{2.299537in}{6.521878in}%
\pgfsys@useobject{currentmarker}{}%
\end{pgfscope}%
\begin{pgfscope}%
\pgfsys@transformshift{2.496612in}{6.813147in}%
\pgfsys@useobject{currentmarker}{}%
\end{pgfscope}%
\begin{pgfscope}%
\pgfsys@transformshift{2.693688in}{6.547536in}%
\pgfsys@useobject{currentmarker}{}%
\end{pgfscope}%
\begin{pgfscope}%
\pgfsys@transformshift{2.890764in}{6.303104in}%
\pgfsys@useobject{currentmarker}{}%
\end{pgfscope}%
\begin{pgfscope}%
\pgfsys@transformshift{3.087839in}{6.303104in}%
\pgfsys@useobject{currentmarker}{}%
\end{pgfscope}%
\begin{pgfscope}%
\pgfsys@transformshift{3.284915in}{6.149519in}%
\pgfsys@useobject{currentmarker}{}%
\end{pgfscope}%
\end{pgfscope}%
\begin{pgfscope}%
\pgfsetrectcap%
\pgfsetmiterjoin%
\pgfsetlinewidth{0.803000pt}%
\definecolor{currentstroke}{rgb}{1.000000,1.000000,1.000000}%
\pgfsetstrokecolor{currentstroke}%
\pgfsetdash{}{0pt}%
\pgfpathmoveto{\pgfqpoint{0.594832in}{6.048788in}}%
\pgfpathlineto{\pgfqpoint{0.594832in}{6.849545in}}%
\pgfusepath{stroke}%
\end{pgfscope}%
\begin{pgfscope}%
\pgfsetrectcap%
\pgfsetmiterjoin%
\pgfsetlinewidth{0.803000pt}%
\definecolor{currentstroke}{rgb}{1.000000,1.000000,1.000000}%
\pgfsetstrokecolor{currentstroke}%
\pgfsetdash{}{0pt}%
\pgfpathmoveto{\pgfqpoint{3.413014in}{6.048788in}}%
\pgfpathlineto{\pgfqpoint{3.413014in}{6.849545in}}%
\pgfusepath{stroke}%
\end{pgfscope}%
\begin{pgfscope}%
\pgfsetrectcap%
\pgfsetmiterjoin%
\pgfsetlinewidth{0.803000pt}%
\definecolor{currentstroke}{rgb}{1.000000,1.000000,1.000000}%
\pgfsetstrokecolor{currentstroke}%
\pgfsetdash{}{0pt}%
\pgfpathmoveto{\pgfqpoint{0.594832in}{6.048788in}}%
\pgfpathlineto{\pgfqpoint{3.413014in}{6.048788in}}%
\pgfusepath{stroke}%
\end{pgfscope}%
\begin{pgfscope}%
\pgfsetrectcap%
\pgfsetmiterjoin%
\pgfsetlinewidth{0.803000pt}%
\definecolor{currentstroke}{rgb}{1.000000,1.000000,1.000000}%
\pgfsetstrokecolor{currentstroke}%
\pgfsetdash{}{0pt}%
\pgfpathmoveto{\pgfqpoint{0.594832in}{6.849545in}}%
\pgfpathlineto{\pgfqpoint{3.413014in}{6.849545in}}%
\pgfusepath{stroke}%
\end{pgfscope}%
\begin{pgfscope}%
\definecolor{textcolor}{rgb}{0.150000,0.150000,0.150000}%
\pgfsetstrokecolor{textcolor}%
\pgfsetfillcolor{textcolor}%
\pgftext[x=2.003923in,y=6.932879in,,base]{\color{textcolor}\rmfamily\fontsize{12.000000}{14.400000}\selectfont MMM}%
\end{pgfscope}%
\begin{pgfscope}%
\pgfsetbuttcap%
\pgfsetmiterjoin%
\definecolor{currentfill}{rgb}{0.917647,0.917647,0.949020}%
\pgfsetfillcolor{currentfill}%
\pgfsetlinewidth{0.000000pt}%
\definecolor{currentstroke}{rgb}{0.000000,0.000000,0.000000}%
\pgfsetstrokecolor{currentstroke}%
\pgfsetstrokeopacity{0.000000}%
\pgfsetdash{}{0pt}%
\pgfpathmoveto{\pgfqpoint{3.976651in}{6.048788in}}%
\pgfpathlineto{\pgfqpoint{6.794832in}{6.048788in}}%
\pgfpathlineto{\pgfqpoint{6.794832in}{6.849545in}}%
\pgfpathlineto{\pgfqpoint{3.976651in}{6.849545in}}%
\pgfpathclose%
\pgfusepath{fill}%
\end{pgfscope}%
\begin{pgfscope}%
\pgfpathrectangle{\pgfqpoint{3.976651in}{6.048788in}}{\pgfqpoint{2.818182in}{0.800758in}}%
\pgfusepath{clip}%
\pgfsetroundcap%
\pgfsetroundjoin%
\pgfsetlinewidth{0.803000pt}%
\definecolor{currentstroke}{rgb}{1.000000,1.000000,1.000000}%
\pgfsetstrokecolor{currentstroke}%
\pgfsetdash{}{0pt}%
\pgfpathmoveto{\pgfqpoint{4.104750in}{6.048788in}}%
\pgfpathlineto{\pgfqpoint{4.104750in}{6.849545in}}%
\pgfusepath{stroke}%
\end{pgfscope}%
\begin{pgfscope}%
\pgfpathrectangle{\pgfqpoint{3.976651in}{6.048788in}}{\pgfqpoint{2.818182in}{0.800758in}}%
\pgfusepath{clip}%
\pgfsetroundcap%
\pgfsetroundjoin%
\pgfsetlinewidth{0.803000pt}%
\definecolor{currentstroke}{rgb}{1.000000,1.000000,1.000000}%
\pgfsetstrokecolor{currentstroke}%
\pgfsetdash{}{0pt}%
\pgfpathmoveto{\pgfqpoint{4.301825in}{6.048788in}}%
\pgfpathlineto{\pgfqpoint{4.301825in}{6.849545in}}%
\pgfusepath{stroke}%
\end{pgfscope}%
\begin{pgfscope}%
\pgfpathrectangle{\pgfqpoint{3.976651in}{6.048788in}}{\pgfqpoint{2.818182in}{0.800758in}}%
\pgfusepath{clip}%
\pgfsetroundcap%
\pgfsetroundjoin%
\pgfsetlinewidth{0.803000pt}%
\definecolor{currentstroke}{rgb}{1.000000,1.000000,1.000000}%
\pgfsetstrokecolor{currentstroke}%
\pgfsetdash{}{0pt}%
\pgfpathmoveto{\pgfqpoint{4.498901in}{6.048788in}}%
\pgfpathlineto{\pgfqpoint{4.498901in}{6.849545in}}%
\pgfusepath{stroke}%
\end{pgfscope}%
\begin{pgfscope}%
\pgfpathrectangle{\pgfqpoint{3.976651in}{6.048788in}}{\pgfqpoint{2.818182in}{0.800758in}}%
\pgfusepath{clip}%
\pgfsetroundcap%
\pgfsetroundjoin%
\pgfsetlinewidth{0.803000pt}%
\definecolor{currentstroke}{rgb}{1.000000,1.000000,1.000000}%
\pgfsetstrokecolor{currentstroke}%
\pgfsetdash{}{0pt}%
\pgfpathmoveto{\pgfqpoint{4.695977in}{6.048788in}}%
\pgfpathlineto{\pgfqpoint{4.695977in}{6.849545in}}%
\pgfusepath{stroke}%
\end{pgfscope}%
\begin{pgfscope}%
\pgfpathrectangle{\pgfqpoint{3.976651in}{6.048788in}}{\pgfqpoint{2.818182in}{0.800758in}}%
\pgfusepath{clip}%
\pgfsetroundcap%
\pgfsetroundjoin%
\pgfsetlinewidth{0.803000pt}%
\definecolor{currentstroke}{rgb}{1.000000,1.000000,1.000000}%
\pgfsetstrokecolor{currentstroke}%
\pgfsetdash{}{0pt}%
\pgfpathmoveto{\pgfqpoint{4.893052in}{6.048788in}}%
\pgfpathlineto{\pgfqpoint{4.893052in}{6.849545in}}%
\pgfusepath{stroke}%
\end{pgfscope}%
\begin{pgfscope}%
\pgfpathrectangle{\pgfqpoint{3.976651in}{6.048788in}}{\pgfqpoint{2.818182in}{0.800758in}}%
\pgfusepath{clip}%
\pgfsetroundcap%
\pgfsetroundjoin%
\pgfsetlinewidth{0.803000pt}%
\definecolor{currentstroke}{rgb}{1.000000,1.000000,1.000000}%
\pgfsetstrokecolor{currentstroke}%
\pgfsetdash{}{0pt}%
\pgfpathmoveto{\pgfqpoint{5.090128in}{6.048788in}}%
\pgfpathlineto{\pgfqpoint{5.090128in}{6.849545in}}%
\pgfusepath{stroke}%
\end{pgfscope}%
\begin{pgfscope}%
\pgfpathrectangle{\pgfqpoint{3.976651in}{6.048788in}}{\pgfqpoint{2.818182in}{0.800758in}}%
\pgfusepath{clip}%
\pgfsetroundcap%
\pgfsetroundjoin%
\pgfsetlinewidth{0.803000pt}%
\definecolor{currentstroke}{rgb}{1.000000,1.000000,1.000000}%
\pgfsetstrokecolor{currentstroke}%
\pgfsetdash{}{0pt}%
\pgfpathmoveto{\pgfqpoint{5.287204in}{6.048788in}}%
\pgfpathlineto{\pgfqpoint{5.287204in}{6.849545in}}%
\pgfusepath{stroke}%
\end{pgfscope}%
\begin{pgfscope}%
\pgfpathrectangle{\pgfqpoint{3.976651in}{6.048788in}}{\pgfqpoint{2.818182in}{0.800758in}}%
\pgfusepath{clip}%
\pgfsetroundcap%
\pgfsetroundjoin%
\pgfsetlinewidth{0.803000pt}%
\definecolor{currentstroke}{rgb}{1.000000,1.000000,1.000000}%
\pgfsetstrokecolor{currentstroke}%
\pgfsetdash{}{0pt}%
\pgfpathmoveto{\pgfqpoint{5.484279in}{6.048788in}}%
\pgfpathlineto{\pgfqpoint{5.484279in}{6.849545in}}%
\pgfusepath{stroke}%
\end{pgfscope}%
\begin{pgfscope}%
\pgfpathrectangle{\pgfqpoint{3.976651in}{6.048788in}}{\pgfqpoint{2.818182in}{0.800758in}}%
\pgfusepath{clip}%
\pgfsetroundcap%
\pgfsetroundjoin%
\pgfsetlinewidth{0.803000pt}%
\definecolor{currentstroke}{rgb}{1.000000,1.000000,1.000000}%
\pgfsetstrokecolor{currentstroke}%
\pgfsetdash{}{0pt}%
\pgfpathmoveto{\pgfqpoint{5.681355in}{6.048788in}}%
\pgfpathlineto{\pgfqpoint{5.681355in}{6.849545in}}%
\pgfusepath{stroke}%
\end{pgfscope}%
\begin{pgfscope}%
\pgfpathrectangle{\pgfqpoint{3.976651in}{6.048788in}}{\pgfqpoint{2.818182in}{0.800758in}}%
\pgfusepath{clip}%
\pgfsetroundcap%
\pgfsetroundjoin%
\pgfsetlinewidth{0.803000pt}%
\definecolor{currentstroke}{rgb}{1.000000,1.000000,1.000000}%
\pgfsetstrokecolor{currentstroke}%
\pgfsetdash{}{0pt}%
\pgfpathmoveto{\pgfqpoint{5.878431in}{6.048788in}}%
\pgfpathlineto{\pgfqpoint{5.878431in}{6.849545in}}%
\pgfusepath{stroke}%
\end{pgfscope}%
\begin{pgfscope}%
\pgfpathrectangle{\pgfqpoint{3.976651in}{6.048788in}}{\pgfqpoint{2.818182in}{0.800758in}}%
\pgfusepath{clip}%
\pgfsetroundcap%
\pgfsetroundjoin%
\pgfsetlinewidth{0.803000pt}%
\definecolor{currentstroke}{rgb}{1.000000,1.000000,1.000000}%
\pgfsetstrokecolor{currentstroke}%
\pgfsetdash{}{0pt}%
\pgfpathmoveto{\pgfqpoint{6.075506in}{6.048788in}}%
\pgfpathlineto{\pgfqpoint{6.075506in}{6.849545in}}%
\pgfusepath{stroke}%
\end{pgfscope}%
\begin{pgfscope}%
\pgfpathrectangle{\pgfqpoint{3.976651in}{6.048788in}}{\pgfqpoint{2.818182in}{0.800758in}}%
\pgfusepath{clip}%
\pgfsetroundcap%
\pgfsetroundjoin%
\pgfsetlinewidth{0.803000pt}%
\definecolor{currentstroke}{rgb}{1.000000,1.000000,1.000000}%
\pgfsetstrokecolor{currentstroke}%
\pgfsetdash{}{0pt}%
\pgfpathmoveto{\pgfqpoint{6.272582in}{6.048788in}}%
\pgfpathlineto{\pgfqpoint{6.272582in}{6.849545in}}%
\pgfusepath{stroke}%
\end{pgfscope}%
\begin{pgfscope}%
\pgfpathrectangle{\pgfqpoint{3.976651in}{6.048788in}}{\pgfqpoint{2.818182in}{0.800758in}}%
\pgfusepath{clip}%
\pgfsetroundcap%
\pgfsetroundjoin%
\pgfsetlinewidth{0.803000pt}%
\definecolor{currentstroke}{rgb}{1.000000,1.000000,1.000000}%
\pgfsetstrokecolor{currentstroke}%
\pgfsetdash{}{0pt}%
\pgfpathmoveto{\pgfqpoint{6.469658in}{6.048788in}}%
\pgfpathlineto{\pgfqpoint{6.469658in}{6.849545in}}%
\pgfusepath{stroke}%
\end{pgfscope}%
\begin{pgfscope}%
\pgfpathrectangle{\pgfqpoint{3.976651in}{6.048788in}}{\pgfqpoint{2.818182in}{0.800758in}}%
\pgfusepath{clip}%
\pgfsetroundcap%
\pgfsetroundjoin%
\pgfsetlinewidth{0.803000pt}%
\definecolor{currentstroke}{rgb}{1.000000,1.000000,1.000000}%
\pgfsetstrokecolor{currentstroke}%
\pgfsetdash{}{0pt}%
\pgfpathmoveto{\pgfqpoint{6.666733in}{6.048788in}}%
\pgfpathlineto{\pgfqpoint{6.666733in}{6.849545in}}%
\pgfusepath{stroke}%
\end{pgfscope}%
\begin{pgfscope}%
\pgfpathrectangle{\pgfqpoint{3.976651in}{6.048788in}}{\pgfqpoint{2.818182in}{0.800758in}}%
\pgfusepath{clip}%
\pgfsetroundcap%
\pgfsetroundjoin%
\pgfsetlinewidth{0.803000pt}%
\definecolor{currentstroke}{rgb}{1.000000,1.000000,1.000000}%
\pgfsetstrokecolor{currentstroke}%
\pgfsetdash{}{0pt}%
\pgfpathmoveto{\pgfqpoint{3.976651in}{6.301123in}}%
\pgfpathlineto{\pgfqpoint{6.794832in}{6.301123in}}%
\pgfusepath{stroke}%
\end{pgfscope}%
\begin{pgfscope}%
\definecolor{textcolor}{rgb}{0.150000,0.150000,0.150000}%
\pgfsetstrokecolor{textcolor}%
\pgfsetfillcolor{textcolor}%
\pgftext[x=3.791063in,y=6.248362in,left,base]{\color{textcolor}\rmfamily\fontsize{10.000000}{12.000000}\selectfont 1}%
\end{pgfscope}%
\begin{pgfscope}%
\pgfpathrectangle{\pgfqpoint{3.976651in}{6.048788in}}{\pgfqpoint{2.818182in}{0.800758in}}%
\pgfusepath{clip}%
\pgfsetroundcap%
\pgfsetroundjoin%
\pgfsetlinewidth{0.803000pt}%
\definecolor{currentstroke}{rgb}{1.000000,1.000000,1.000000}%
\pgfsetstrokecolor{currentstroke}%
\pgfsetdash{}{0pt}%
\pgfpathmoveto{\pgfqpoint{3.976651in}{6.581565in}}%
\pgfpathlineto{\pgfqpoint{6.794832in}{6.581565in}}%
\pgfusepath{stroke}%
\end{pgfscope}%
\begin{pgfscope}%
\definecolor{textcolor}{rgb}{0.150000,0.150000,0.150000}%
\pgfsetstrokecolor{textcolor}%
\pgfsetfillcolor{textcolor}%
\pgftext[x=3.791063in,y=6.528804in,left,base]{\color{textcolor}\rmfamily\fontsize{10.000000}{12.000000}\selectfont 2}%
\end{pgfscope}%
\begin{pgfscope}%
\pgfpathrectangle{\pgfqpoint{3.976651in}{6.048788in}}{\pgfqpoint{2.818182in}{0.800758in}}%
\pgfusepath{clip}%
\pgfsetroundcap%
\pgfsetroundjoin%
\pgfsetlinewidth{1.505625pt}%
\definecolor{currentstroke}{rgb}{1.000000,0.498039,0.054902}%
\pgfsetstrokecolor{currentstroke}%
\pgfsetdash{}{0pt}%
\pgfpathmoveto{\pgfqpoint{4.104750in}{6.085296in}}%
\pgfpathlineto{\pgfqpoint{4.301825in}{6.085611in}}%
\pgfpathlineto{\pgfqpoint{4.498901in}{6.085855in}}%
\pgfpathlineto{\pgfqpoint{4.695977in}{6.085186in}}%
\pgfpathlineto{\pgfqpoint{4.893052in}{6.085584in}}%
\pgfpathlineto{\pgfqpoint{5.090128in}{6.085726in}}%
\pgfpathlineto{\pgfqpoint{5.287204in}{6.085395in}}%
\pgfpathlineto{\pgfqpoint{5.484279in}{6.085186in}}%
\pgfpathlineto{\pgfqpoint{5.681355in}{6.085805in}}%
\pgfpathlineto{\pgfqpoint{5.878431in}{6.087082in}}%
\pgfpathlineto{\pgfqpoint{6.075506in}{6.813147in}}%
\pgfpathlineto{\pgfqpoint{6.272582in}{6.085186in}}%
\pgfpathlineto{\pgfqpoint{6.469658in}{6.085186in}}%
\pgfpathlineto{\pgfqpoint{6.666733in}{6.085369in}}%
\pgfusepath{stroke}%
\end{pgfscope}%
\begin{pgfscope}%
\pgfpathrectangle{\pgfqpoint{3.976651in}{6.048788in}}{\pgfqpoint{2.818182in}{0.800758in}}%
\pgfusepath{clip}%
\pgfsetbuttcap%
\pgfsetroundjoin%
\definecolor{currentfill}{rgb}{1.000000,0.498039,0.054902}%
\pgfsetfillcolor{currentfill}%
\pgfsetlinewidth{1.003750pt}%
\definecolor{currentstroke}{rgb}{1.000000,0.498039,0.054902}%
\pgfsetstrokecolor{currentstroke}%
\pgfsetdash{}{0pt}%
\pgfsys@defobject{currentmarker}{\pgfqpoint{-0.041667in}{-0.041667in}}{\pgfqpoint{0.041667in}{0.041667in}}{%
\pgfpathmoveto{\pgfqpoint{0.000000in}{-0.041667in}}%
\pgfpathcurveto{\pgfqpoint{0.011050in}{-0.041667in}}{\pgfqpoint{0.021649in}{-0.037276in}}{\pgfqpoint{0.029463in}{-0.029463in}}%
\pgfpathcurveto{\pgfqpoint{0.037276in}{-0.021649in}}{\pgfqpoint{0.041667in}{-0.011050in}}{\pgfqpoint{0.041667in}{0.000000in}}%
\pgfpathcurveto{\pgfqpoint{0.041667in}{0.011050in}}{\pgfqpoint{0.037276in}{0.021649in}}{\pgfqpoint{0.029463in}{0.029463in}}%
\pgfpathcurveto{\pgfqpoint{0.021649in}{0.037276in}}{\pgfqpoint{0.011050in}{0.041667in}}{\pgfqpoint{0.000000in}{0.041667in}}%
\pgfpathcurveto{\pgfqpoint{-0.011050in}{0.041667in}}{\pgfqpoint{-0.021649in}{0.037276in}}{\pgfqpoint{-0.029463in}{0.029463in}}%
\pgfpathcurveto{\pgfqpoint{-0.037276in}{0.021649in}}{\pgfqpoint{-0.041667in}{0.011050in}}{\pgfqpoint{-0.041667in}{0.000000in}}%
\pgfpathcurveto{\pgfqpoint{-0.041667in}{-0.011050in}}{\pgfqpoint{-0.037276in}{-0.021649in}}{\pgfqpoint{-0.029463in}{-0.029463in}}%
\pgfpathcurveto{\pgfqpoint{-0.021649in}{-0.037276in}}{\pgfqpoint{-0.011050in}{-0.041667in}}{\pgfqpoint{0.000000in}{-0.041667in}}%
\pgfpathclose%
\pgfusepath{stroke,fill}%
}%
\begin{pgfscope}%
\pgfsys@transformshift{4.104750in}{6.085296in}%
\pgfsys@useobject{currentmarker}{}%
\end{pgfscope}%
\begin{pgfscope}%
\pgfsys@transformshift{4.301825in}{6.085611in}%
\pgfsys@useobject{currentmarker}{}%
\end{pgfscope}%
\begin{pgfscope}%
\pgfsys@transformshift{4.498901in}{6.085855in}%
\pgfsys@useobject{currentmarker}{}%
\end{pgfscope}%
\begin{pgfscope}%
\pgfsys@transformshift{4.695977in}{6.085186in}%
\pgfsys@useobject{currentmarker}{}%
\end{pgfscope}%
\begin{pgfscope}%
\pgfsys@transformshift{4.893052in}{6.085584in}%
\pgfsys@useobject{currentmarker}{}%
\end{pgfscope}%
\begin{pgfscope}%
\pgfsys@transformshift{5.090128in}{6.085726in}%
\pgfsys@useobject{currentmarker}{}%
\end{pgfscope}%
\begin{pgfscope}%
\pgfsys@transformshift{5.287204in}{6.085395in}%
\pgfsys@useobject{currentmarker}{}%
\end{pgfscope}%
\begin{pgfscope}%
\pgfsys@transformshift{5.484279in}{6.085186in}%
\pgfsys@useobject{currentmarker}{}%
\end{pgfscope}%
\begin{pgfscope}%
\pgfsys@transformshift{5.681355in}{6.085805in}%
\pgfsys@useobject{currentmarker}{}%
\end{pgfscope}%
\begin{pgfscope}%
\pgfsys@transformshift{5.878431in}{6.087082in}%
\pgfsys@useobject{currentmarker}{}%
\end{pgfscope}%
\begin{pgfscope}%
\pgfsys@transformshift{6.075506in}{6.813147in}%
\pgfsys@useobject{currentmarker}{}%
\end{pgfscope}%
\begin{pgfscope}%
\pgfsys@transformshift{6.272582in}{6.085186in}%
\pgfsys@useobject{currentmarker}{}%
\end{pgfscope}%
\begin{pgfscope}%
\pgfsys@transformshift{6.469658in}{6.085186in}%
\pgfsys@useobject{currentmarker}{}%
\end{pgfscope}%
\begin{pgfscope}%
\pgfsys@transformshift{6.666733in}{6.085369in}%
\pgfsys@useobject{currentmarker}{}%
\end{pgfscope}%
\end{pgfscope}%
\begin{pgfscope}%
\pgfsetrectcap%
\pgfsetmiterjoin%
\pgfsetlinewidth{0.803000pt}%
\definecolor{currentstroke}{rgb}{1.000000,1.000000,1.000000}%
\pgfsetstrokecolor{currentstroke}%
\pgfsetdash{}{0pt}%
\pgfpathmoveto{\pgfqpoint{3.976651in}{6.048788in}}%
\pgfpathlineto{\pgfqpoint{3.976651in}{6.849545in}}%
\pgfusepath{stroke}%
\end{pgfscope}%
\begin{pgfscope}%
\pgfsetrectcap%
\pgfsetmiterjoin%
\pgfsetlinewidth{0.803000pt}%
\definecolor{currentstroke}{rgb}{1.000000,1.000000,1.000000}%
\pgfsetstrokecolor{currentstroke}%
\pgfsetdash{}{0pt}%
\pgfpathmoveto{\pgfqpoint{6.794832in}{6.048788in}}%
\pgfpathlineto{\pgfqpoint{6.794832in}{6.849545in}}%
\pgfusepath{stroke}%
\end{pgfscope}%
\begin{pgfscope}%
\pgfsetrectcap%
\pgfsetmiterjoin%
\pgfsetlinewidth{0.803000pt}%
\definecolor{currentstroke}{rgb}{1.000000,1.000000,1.000000}%
\pgfsetstrokecolor{currentstroke}%
\pgfsetdash{}{0pt}%
\pgfpathmoveto{\pgfqpoint{3.976651in}{6.048788in}}%
\pgfpathlineto{\pgfqpoint{6.794832in}{6.048788in}}%
\pgfusepath{stroke}%
\end{pgfscope}%
\begin{pgfscope}%
\pgfsetrectcap%
\pgfsetmiterjoin%
\pgfsetlinewidth{0.803000pt}%
\definecolor{currentstroke}{rgb}{1.000000,1.000000,1.000000}%
\pgfsetstrokecolor{currentstroke}%
\pgfsetdash{}{0pt}%
\pgfpathmoveto{\pgfqpoint{3.976651in}{6.849545in}}%
\pgfpathlineto{\pgfqpoint{6.794832in}{6.849545in}}%
\pgfusepath{stroke}%
\end{pgfscope}%
\begin{pgfscope}%
\definecolor{textcolor}{rgb}{0.150000,0.150000,0.150000}%
\pgfsetstrokecolor{textcolor}%
\pgfsetfillcolor{textcolor}%
\pgftext[x=5.385741in,y=6.932879in,,base]{\color{textcolor}\rmfamily\fontsize{12.000000}{14.400000}\selectfont AXP}%
\end{pgfscope}%
\begin{pgfscope}%
\pgfsetbuttcap%
\pgfsetmiterjoin%
\definecolor{currentfill}{rgb}{0.917647,0.917647,0.949020}%
\pgfsetfillcolor{currentfill}%
\pgfsetlinewidth{0.000000pt}%
\definecolor{currentstroke}{rgb}{0.000000,0.000000,0.000000}%
\pgfsetstrokecolor{currentstroke}%
\pgfsetstrokeopacity{0.000000}%
\pgfsetdash{}{0pt}%
\pgfpathmoveto{\pgfqpoint{0.594832in}{4.927727in}}%
\pgfpathlineto{\pgfqpoint{3.413014in}{4.927727in}}%
\pgfpathlineto{\pgfqpoint{3.413014in}{5.728485in}}%
\pgfpathlineto{\pgfqpoint{0.594832in}{5.728485in}}%
\pgfpathclose%
\pgfusepath{fill}%
\end{pgfscope}%
\begin{pgfscope}%
\pgfpathrectangle{\pgfqpoint{0.594832in}{4.927727in}}{\pgfqpoint{2.818182in}{0.800758in}}%
\pgfusepath{clip}%
\pgfsetroundcap%
\pgfsetroundjoin%
\pgfsetlinewidth{0.803000pt}%
\definecolor{currentstroke}{rgb}{1.000000,1.000000,1.000000}%
\pgfsetstrokecolor{currentstroke}%
\pgfsetdash{}{0pt}%
\pgfpathmoveto{\pgfqpoint{0.722932in}{4.927727in}}%
\pgfpathlineto{\pgfqpoint{0.722932in}{5.728485in}}%
\pgfusepath{stroke}%
\end{pgfscope}%
\begin{pgfscope}%
\pgfpathrectangle{\pgfqpoint{0.594832in}{4.927727in}}{\pgfqpoint{2.818182in}{0.800758in}}%
\pgfusepath{clip}%
\pgfsetroundcap%
\pgfsetroundjoin%
\pgfsetlinewidth{0.803000pt}%
\definecolor{currentstroke}{rgb}{1.000000,1.000000,1.000000}%
\pgfsetstrokecolor{currentstroke}%
\pgfsetdash{}{0pt}%
\pgfpathmoveto{\pgfqpoint{0.920007in}{4.927727in}}%
\pgfpathlineto{\pgfqpoint{0.920007in}{5.728485in}}%
\pgfusepath{stroke}%
\end{pgfscope}%
\begin{pgfscope}%
\pgfpathrectangle{\pgfqpoint{0.594832in}{4.927727in}}{\pgfqpoint{2.818182in}{0.800758in}}%
\pgfusepath{clip}%
\pgfsetroundcap%
\pgfsetroundjoin%
\pgfsetlinewidth{0.803000pt}%
\definecolor{currentstroke}{rgb}{1.000000,1.000000,1.000000}%
\pgfsetstrokecolor{currentstroke}%
\pgfsetdash{}{0pt}%
\pgfpathmoveto{\pgfqpoint{1.117083in}{4.927727in}}%
\pgfpathlineto{\pgfqpoint{1.117083in}{5.728485in}}%
\pgfusepath{stroke}%
\end{pgfscope}%
\begin{pgfscope}%
\pgfpathrectangle{\pgfqpoint{0.594832in}{4.927727in}}{\pgfqpoint{2.818182in}{0.800758in}}%
\pgfusepath{clip}%
\pgfsetroundcap%
\pgfsetroundjoin%
\pgfsetlinewidth{0.803000pt}%
\definecolor{currentstroke}{rgb}{1.000000,1.000000,1.000000}%
\pgfsetstrokecolor{currentstroke}%
\pgfsetdash{}{0pt}%
\pgfpathmoveto{\pgfqpoint{1.314158in}{4.927727in}}%
\pgfpathlineto{\pgfqpoint{1.314158in}{5.728485in}}%
\pgfusepath{stroke}%
\end{pgfscope}%
\begin{pgfscope}%
\pgfpathrectangle{\pgfqpoint{0.594832in}{4.927727in}}{\pgfqpoint{2.818182in}{0.800758in}}%
\pgfusepath{clip}%
\pgfsetroundcap%
\pgfsetroundjoin%
\pgfsetlinewidth{0.803000pt}%
\definecolor{currentstroke}{rgb}{1.000000,1.000000,1.000000}%
\pgfsetstrokecolor{currentstroke}%
\pgfsetdash{}{0pt}%
\pgfpathmoveto{\pgfqpoint{1.511234in}{4.927727in}}%
\pgfpathlineto{\pgfqpoint{1.511234in}{5.728485in}}%
\pgfusepath{stroke}%
\end{pgfscope}%
\begin{pgfscope}%
\pgfpathrectangle{\pgfqpoint{0.594832in}{4.927727in}}{\pgfqpoint{2.818182in}{0.800758in}}%
\pgfusepath{clip}%
\pgfsetroundcap%
\pgfsetroundjoin%
\pgfsetlinewidth{0.803000pt}%
\definecolor{currentstroke}{rgb}{1.000000,1.000000,1.000000}%
\pgfsetstrokecolor{currentstroke}%
\pgfsetdash{}{0pt}%
\pgfpathmoveto{\pgfqpoint{1.708310in}{4.927727in}}%
\pgfpathlineto{\pgfqpoint{1.708310in}{5.728485in}}%
\pgfusepath{stroke}%
\end{pgfscope}%
\begin{pgfscope}%
\pgfpathrectangle{\pgfqpoint{0.594832in}{4.927727in}}{\pgfqpoint{2.818182in}{0.800758in}}%
\pgfusepath{clip}%
\pgfsetroundcap%
\pgfsetroundjoin%
\pgfsetlinewidth{0.803000pt}%
\definecolor{currentstroke}{rgb}{1.000000,1.000000,1.000000}%
\pgfsetstrokecolor{currentstroke}%
\pgfsetdash{}{0pt}%
\pgfpathmoveto{\pgfqpoint{1.905385in}{4.927727in}}%
\pgfpathlineto{\pgfqpoint{1.905385in}{5.728485in}}%
\pgfusepath{stroke}%
\end{pgfscope}%
\begin{pgfscope}%
\pgfpathrectangle{\pgfqpoint{0.594832in}{4.927727in}}{\pgfqpoint{2.818182in}{0.800758in}}%
\pgfusepath{clip}%
\pgfsetroundcap%
\pgfsetroundjoin%
\pgfsetlinewidth{0.803000pt}%
\definecolor{currentstroke}{rgb}{1.000000,1.000000,1.000000}%
\pgfsetstrokecolor{currentstroke}%
\pgfsetdash{}{0pt}%
\pgfpathmoveto{\pgfqpoint{2.102461in}{4.927727in}}%
\pgfpathlineto{\pgfqpoint{2.102461in}{5.728485in}}%
\pgfusepath{stroke}%
\end{pgfscope}%
\begin{pgfscope}%
\pgfpathrectangle{\pgfqpoint{0.594832in}{4.927727in}}{\pgfqpoint{2.818182in}{0.800758in}}%
\pgfusepath{clip}%
\pgfsetroundcap%
\pgfsetroundjoin%
\pgfsetlinewidth{0.803000pt}%
\definecolor{currentstroke}{rgb}{1.000000,1.000000,1.000000}%
\pgfsetstrokecolor{currentstroke}%
\pgfsetdash{}{0pt}%
\pgfpathmoveto{\pgfqpoint{2.299537in}{4.927727in}}%
\pgfpathlineto{\pgfqpoint{2.299537in}{5.728485in}}%
\pgfusepath{stroke}%
\end{pgfscope}%
\begin{pgfscope}%
\pgfpathrectangle{\pgfqpoint{0.594832in}{4.927727in}}{\pgfqpoint{2.818182in}{0.800758in}}%
\pgfusepath{clip}%
\pgfsetroundcap%
\pgfsetroundjoin%
\pgfsetlinewidth{0.803000pt}%
\definecolor{currentstroke}{rgb}{1.000000,1.000000,1.000000}%
\pgfsetstrokecolor{currentstroke}%
\pgfsetdash{}{0pt}%
\pgfpathmoveto{\pgfqpoint{2.496612in}{4.927727in}}%
\pgfpathlineto{\pgfqpoint{2.496612in}{5.728485in}}%
\pgfusepath{stroke}%
\end{pgfscope}%
\begin{pgfscope}%
\pgfpathrectangle{\pgfqpoint{0.594832in}{4.927727in}}{\pgfqpoint{2.818182in}{0.800758in}}%
\pgfusepath{clip}%
\pgfsetroundcap%
\pgfsetroundjoin%
\pgfsetlinewidth{0.803000pt}%
\definecolor{currentstroke}{rgb}{1.000000,1.000000,1.000000}%
\pgfsetstrokecolor{currentstroke}%
\pgfsetdash{}{0pt}%
\pgfpathmoveto{\pgfqpoint{2.693688in}{4.927727in}}%
\pgfpathlineto{\pgfqpoint{2.693688in}{5.728485in}}%
\pgfusepath{stroke}%
\end{pgfscope}%
\begin{pgfscope}%
\pgfpathrectangle{\pgfqpoint{0.594832in}{4.927727in}}{\pgfqpoint{2.818182in}{0.800758in}}%
\pgfusepath{clip}%
\pgfsetroundcap%
\pgfsetroundjoin%
\pgfsetlinewidth{0.803000pt}%
\definecolor{currentstroke}{rgb}{1.000000,1.000000,1.000000}%
\pgfsetstrokecolor{currentstroke}%
\pgfsetdash{}{0pt}%
\pgfpathmoveto{\pgfqpoint{2.890764in}{4.927727in}}%
\pgfpathlineto{\pgfqpoint{2.890764in}{5.728485in}}%
\pgfusepath{stroke}%
\end{pgfscope}%
\begin{pgfscope}%
\pgfpathrectangle{\pgfqpoint{0.594832in}{4.927727in}}{\pgfqpoint{2.818182in}{0.800758in}}%
\pgfusepath{clip}%
\pgfsetroundcap%
\pgfsetroundjoin%
\pgfsetlinewidth{0.803000pt}%
\definecolor{currentstroke}{rgb}{1.000000,1.000000,1.000000}%
\pgfsetstrokecolor{currentstroke}%
\pgfsetdash{}{0pt}%
\pgfpathmoveto{\pgfqpoint{3.087839in}{4.927727in}}%
\pgfpathlineto{\pgfqpoint{3.087839in}{5.728485in}}%
\pgfusepath{stroke}%
\end{pgfscope}%
\begin{pgfscope}%
\pgfpathrectangle{\pgfqpoint{0.594832in}{4.927727in}}{\pgfqpoint{2.818182in}{0.800758in}}%
\pgfusepath{clip}%
\pgfsetroundcap%
\pgfsetroundjoin%
\pgfsetlinewidth{0.803000pt}%
\definecolor{currentstroke}{rgb}{1.000000,1.000000,1.000000}%
\pgfsetstrokecolor{currentstroke}%
\pgfsetdash{}{0pt}%
\pgfpathmoveto{\pgfqpoint{3.284915in}{4.927727in}}%
\pgfpathlineto{\pgfqpoint{3.284915in}{5.728485in}}%
\pgfusepath{stroke}%
\end{pgfscope}%
\begin{pgfscope}%
\pgfpathrectangle{\pgfqpoint{0.594832in}{4.927727in}}{\pgfqpoint{2.818182in}{0.800758in}}%
\pgfusepath{clip}%
\pgfsetroundcap%
\pgfsetroundjoin%
\pgfsetlinewidth{0.803000pt}%
\definecolor{currentstroke}{rgb}{1.000000,1.000000,1.000000}%
\pgfsetstrokecolor{currentstroke}%
\pgfsetdash{}{0pt}%
\pgfpathmoveto{\pgfqpoint{0.594832in}{5.121073in}}%
\pgfpathlineto{\pgfqpoint{3.413014in}{5.121073in}}%
\pgfusepath{stroke}%
\end{pgfscope}%
\begin{pgfscope}%
\definecolor{textcolor}{rgb}{0.150000,0.150000,0.150000}%
\pgfsetstrokecolor{textcolor}%
\pgfsetfillcolor{textcolor}%
\pgftext[x=0.100000in,y=5.068312in,left,base]{\color{textcolor}\rmfamily\fontsize{10.000000}{12.000000}\selectfont 0.205}%
\end{pgfscope}%
\begin{pgfscope}%
\pgfpathrectangle{\pgfqpoint{0.594832in}{4.927727in}}{\pgfqpoint{2.818182in}{0.800758in}}%
\pgfusepath{clip}%
\pgfsetroundcap%
\pgfsetroundjoin%
\pgfsetlinewidth{0.803000pt}%
\definecolor{currentstroke}{rgb}{1.000000,1.000000,1.000000}%
\pgfsetstrokecolor{currentstroke}%
\pgfsetdash{}{0pt}%
\pgfpathmoveto{\pgfqpoint{0.594832in}{5.432472in}}%
\pgfpathlineto{\pgfqpoint{3.413014in}{5.432472in}}%
\pgfusepath{stroke}%
\end{pgfscope}%
\begin{pgfscope}%
\definecolor{textcolor}{rgb}{0.150000,0.150000,0.150000}%
\pgfsetstrokecolor{textcolor}%
\pgfsetfillcolor{textcolor}%
\pgftext[x=0.100000in,y=5.379711in,left,base]{\color{textcolor}\rmfamily\fontsize{10.000000}{12.000000}\selectfont 0.206}%
\end{pgfscope}%
\begin{pgfscope}%
\pgfpathrectangle{\pgfqpoint{0.594832in}{4.927727in}}{\pgfqpoint{2.818182in}{0.800758in}}%
\pgfusepath{clip}%
\pgfsetroundcap%
\pgfsetroundjoin%
\pgfsetlinewidth{1.505625pt}%
\definecolor{currentstroke}{rgb}{0.172549,0.627451,0.172549}%
\pgfsetstrokecolor{currentstroke}%
\pgfsetdash{}{0pt}%
\pgfpathmoveto{\pgfqpoint{0.722932in}{5.188777in}}%
\pgfpathlineto{\pgfqpoint{0.920007in}{5.572778in}}%
\pgfpathlineto{\pgfqpoint{1.117083in}{5.509083in}}%
\pgfpathlineto{\pgfqpoint{1.314158in}{4.964125in}}%
\pgfpathlineto{\pgfqpoint{1.511234in}{5.470858in}}%
\pgfpathlineto{\pgfqpoint{1.708310in}{5.671545in}}%
\pgfpathlineto{\pgfqpoint{1.905385in}{5.271929in}}%
\pgfpathlineto{\pgfqpoint{2.102461in}{4.964125in}}%
\pgfpathlineto{\pgfqpoint{2.299537in}{5.565339in}}%
\pgfpathlineto{\pgfqpoint{2.496612in}{5.619427in}}%
\pgfpathlineto{\pgfqpoint{2.693688in}{5.692087in}}%
\pgfpathlineto{\pgfqpoint{2.890764in}{4.964125in}}%
\pgfpathlineto{\pgfqpoint{3.087839in}{4.964125in}}%
\pgfpathlineto{\pgfqpoint{3.284915in}{5.346006in}}%
\pgfusepath{stroke}%
\end{pgfscope}%
\begin{pgfscope}%
\pgfpathrectangle{\pgfqpoint{0.594832in}{4.927727in}}{\pgfqpoint{2.818182in}{0.800758in}}%
\pgfusepath{clip}%
\pgfsetbuttcap%
\pgfsetroundjoin%
\definecolor{currentfill}{rgb}{0.172549,0.627451,0.172549}%
\pgfsetfillcolor{currentfill}%
\pgfsetlinewidth{1.003750pt}%
\definecolor{currentstroke}{rgb}{0.172549,0.627451,0.172549}%
\pgfsetstrokecolor{currentstroke}%
\pgfsetdash{}{0pt}%
\pgfsys@defobject{currentmarker}{\pgfqpoint{-0.041667in}{-0.041667in}}{\pgfqpoint{0.041667in}{0.041667in}}{%
\pgfpathmoveto{\pgfqpoint{0.000000in}{-0.041667in}}%
\pgfpathcurveto{\pgfqpoint{0.011050in}{-0.041667in}}{\pgfqpoint{0.021649in}{-0.037276in}}{\pgfqpoint{0.029463in}{-0.029463in}}%
\pgfpathcurveto{\pgfqpoint{0.037276in}{-0.021649in}}{\pgfqpoint{0.041667in}{-0.011050in}}{\pgfqpoint{0.041667in}{0.000000in}}%
\pgfpathcurveto{\pgfqpoint{0.041667in}{0.011050in}}{\pgfqpoint{0.037276in}{0.021649in}}{\pgfqpoint{0.029463in}{0.029463in}}%
\pgfpathcurveto{\pgfqpoint{0.021649in}{0.037276in}}{\pgfqpoint{0.011050in}{0.041667in}}{\pgfqpoint{0.000000in}{0.041667in}}%
\pgfpathcurveto{\pgfqpoint{-0.011050in}{0.041667in}}{\pgfqpoint{-0.021649in}{0.037276in}}{\pgfqpoint{-0.029463in}{0.029463in}}%
\pgfpathcurveto{\pgfqpoint{-0.037276in}{0.021649in}}{\pgfqpoint{-0.041667in}{0.011050in}}{\pgfqpoint{-0.041667in}{0.000000in}}%
\pgfpathcurveto{\pgfqpoint{-0.041667in}{-0.011050in}}{\pgfqpoint{-0.037276in}{-0.021649in}}{\pgfqpoint{-0.029463in}{-0.029463in}}%
\pgfpathcurveto{\pgfqpoint{-0.021649in}{-0.037276in}}{\pgfqpoint{-0.011050in}{-0.041667in}}{\pgfqpoint{0.000000in}{-0.041667in}}%
\pgfpathclose%
\pgfusepath{stroke,fill}%
}%
\begin{pgfscope}%
\pgfsys@transformshift{0.722932in}{5.188777in}%
\pgfsys@useobject{currentmarker}{}%
\end{pgfscope}%
\begin{pgfscope}%
\pgfsys@transformshift{0.920007in}{5.572778in}%
\pgfsys@useobject{currentmarker}{}%
\end{pgfscope}%
\begin{pgfscope}%
\pgfsys@transformshift{1.117083in}{5.509083in}%
\pgfsys@useobject{currentmarker}{}%
\end{pgfscope}%
\begin{pgfscope}%
\pgfsys@transformshift{1.314158in}{4.964125in}%
\pgfsys@useobject{currentmarker}{}%
\end{pgfscope}%
\begin{pgfscope}%
\pgfsys@transformshift{1.511234in}{5.470858in}%
\pgfsys@useobject{currentmarker}{}%
\end{pgfscope}%
\begin{pgfscope}%
\pgfsys@transformshift{1.708310in}{5.671545in}%
\pgfsys@useobject{currentmarker}{}%
\end{pgfscope}%
\begin{pgfscope}%
\pgfsys@transformshift{1.905385in}{5.271929in}%
\pgfsys@useobject{currentmarker}{}%
\end{pgfscope}%
\begin{pgfscope}%
\pgfsys@transformshift{2.102461in}{4.964125in}%
\pgfsys@useobject{currentmarker}{}%
\end{pgfscope}%
\begin{pgfscope}%
\pgfsys@transformshift{2.299537in}{5.565339in}%
\pgfsys@useobject{currentmarker}{}%
\end{pgfscope}%
\begin{pgfscope}%
\pgfsys@transformshift{2.496612in}{5.619427in}%
\pgfsys@useobject{currentmarker}{}%
\end{pgfscope}%
\begin{pgfscope}%
\pgfsys@transformshift{2.693688in}{5.692087in}%
\pgfsys@useobject{currentmarker}{}%
\end{pgfscope}%
\begin{pgfscope}%
\pgfsys@transformshift{2.890764in}{4.964125in}%
\pgfsys@useobject{currentmarker}{}%
\end{pgfscope}%
\begin{pgfscope}%
\pgfsys@transformshift{3.087839in}{4.964125in}%
\pgfsys@useobject{currentmarker}{}%
\end{pgfscope}%
\begin{pgfscope}%
\pgfsys@transformshift{3.284915in}{5.346006in}%
\pgfsys@useobject{currentmarker}{}%
\end{pgfscope}%
\end{pgfscope}%
\begin{pgfscope}%
\pgfsetrectcap%
\pgfsetmiterjoin%
\pgfsetlinewidth{0.803000pt}%
\definecolor{currentstroke}{rgb}{1.000000,1.000000,1.000000}%
\pgfsetstrokecolor{currentstroke}%
\pgfsetdash{}{0pt}%
\pgfpathmoveto{\pgfqpoint{0.594832in}{4.927727in}}%
\pgfpathlineto{\pgfqpoint{0.594832in}{5.728485in}}%
\pgfusepath{stroke}%
\end{pgfscope}%
\begin{pgfscope}%
\pgfsetrectcap%
\pgfsetmiterjoin%
\pgfsetlinewidth{0.803000pt}%
\definecolor{currentstroke}{rgb}{1.000000,1.000000,1.000000}%
\pgfsetstrokecolor{currentstroke}%
\pgfsetdash{}{0pt}%
\pgfpathmoveto{\pgfqpoint{3.413014in}{4.927727in}}%
\pgfpathlineto{\pgfqpoint{3.413014in}{5.728485in}}%
\pgfusepath{stroke}%
\end{pgfscope}%
\begin{pgfscope}%
\pgfsetrectcap%
\pgfsetmiterjoin%
\pgfsetlinewidth{0.803000pt}%
\definecolor{currentstroke}{rgb}{1.000000,1.000000,1.000000}%
\pgfsetstrokecolor{currentstroke}%
\pgfsetdash{}{0pt}%
\pgfpathmoveto{\pgfqpoint{0.594832in}{4.927727in}}%
\pgfpathlineto{\pgfqpoint{3.413014in}{4.927727in}}%
\pgfusepath{stroke}%
\end{pgfscope}%
\begin{pgfscope}%
\pgfsetrectcap%
\pgfsetmiterjoin%
\pgfsetlinewidth{0.803000pt}%
\definecolor{currentstroke}{rgb}{1.000000,1.000000,1.000000}%
\pgfsetstrokecolor{currentstroke}%
\pgfsetdash{}{0pt}%
\pgfpathmoveto{\pgfqpoint{0.594832in}{5.728485in}}%
\pgfpathlineto{\pgfqpoint{3.413014in}{5.728485in}}%
\pgfusepath{stroke}%
\end{pgfscope}%
\begin{pgfscope}%
\definecolor{textcolor}{rgb}{0.150000,0.150000,0.150000}%
\pgfsetstrokecolor{textcolor}%
\pgfsetfillcolor{textcolor}%
\pgftext[x=2.003923in,y=5.811818in,,base]{\color{textcolor}\rmfamily\fontsize{12.000000}{14.400000}\selectfont GE}%
\end{pgfscope}%
\begin{pgfscope}%
\pgfsetbuttcap%
\pgfsetmiterjoin%
\definecolor{currentfill}{rgb}{0.917647,0.917647,0.949020}%
\pgfsetfillcolor{currentfill}%
\pgfsetlinewidth{0.000000pt}%
\definecolor{currentstroke}{rgb}{0.000000,0.000000,0.000000}%
\pgfsetstrokecolor{currentstroke}%
\pgfsetstrokeopacity{0.000000}%
\pgfsetdash{}{0pt}%
\pgfpathmoveto{\pgfqpoint{3.976651in}{4.927727in}}%
\pgfpathlineto{\pgfqpoint{6.794832in}{4.927727in}}%
\pgfpathlineto{\pgfqpoint{6.794832in}{5.728485in}}%
\pgfpathlineto{\pgfqpoint{3.976651in}{5.728485in}}%
\pgfpathclose%
\pgfusepath{fill}%
\end{pgfscope}%
\begin{pgfscope}%
\pgfpathrectangle{\pgfqpoint{3.976651in}{4.927727in}}{\pgfqpoint{2.818182in}{0.800758in}}%
\pgfusepath{clip}%
\pgfsetroundcap%
\pgfsetroundjoin%
\pgfsetlinewidth{0.803000pt}%
\definecolor{currentstroke}{rgb}{1.000000,1.000000,1.000000}%
\pgfsetstrokecolor{currentstroke}%
\pgfsetdash{}{0pt}%
\pgfpathmoveto{\pgfqpoint{4.104750in}{4.927727in}}%
\pgfpathlineto{\pgfqpoint{4.104750in}{5.728485in}}%
\pgfusepath{stroke}%
\end{pgfscope}%
\begin{pgfscope}%
\pgfpathrectangle{\pgfqpoint{3.976651in}{4.927727in}}{\pgfqpoint{2.818182in}{0.800758in}}%
\pgfusepath{clip}%
\pgfsetroundcap%
\pgfsetroundjoin%
\pgfsetlinewidth{0.803000pt}%
\definecolor{currentstroke}{rgb}{1.000000,1.000000,1.000000}%
\pgfsetstrokecolor{currentstroke}%
\pgfsetdash{}{0pt}%
\pgfpathmoveto{\pgfqpoint{4.301825in}{4.927727in}}%
\pgfpathlineto{\pgfqpoint{4.301825in}{5.728485in}}%
\pgfusepath{stroke}%
\end{pgfscope}%
\begin{pgfscope}%
\pgfpathrectangle{\pgfqpoint{3.976651in}{4.927727in}}{\pgfqpoint{2.818182in}{0.800758in}}%
\pgfusepath{clip}%
\pgfsetroundcap%
\pgfsetroundjoin%
\pgfsetlinewidth{0.803000pt}%
\definecolor{currentstroke}{rgb}{1.000000,1.000000,1.000000}%
\pgfsetstrokecolor{currentstroke}%
\pgfsetdash{}{0pt}%
\pgfpathmoveto{\pgfqpoint{4.498901in}{4.927727in}}%
\pgfpathlineto{\pgfqpoint{4.498901in}{5.728485in}}%
\pgfusepath{stroke}%
\end{pgfscope}%
\begin{pgfscope}%
\pgfpathrectangle{\pgfqpoint{3.976651in}{4.927727in}}{\pgfqpoint{2.818182in}{0.800758in}}%
\pgfusepath{clip}%
\pgfsetroundcap%
\pgfsetroundjoin%
\pgfsetlinewidth{0.803000pt}%
\definecolor{currentstroke}{rgb}{1.000000,1.000000,1.000000}%
\pgfsetstrokecolor{currentstroke}%
\pgfsetdash{}{0pt}%
\pgfpathmoveto{\pgfqpoint{4.695977in}{4.927727in}}%
\pgfpathlineto{\pgfqpoint{4.695977in}{5.728485in}}%
\pgfusepath{stroke}%
\end{pgfscope}%
\begin{pgfscope}%
\pgfpathrectangle{\pgfqpoint{3.976651in}{4.927727in}}{\pgfqpoint{2.818182in}{0.800758in}}%
\pgfusepath{clip}%
\pgfsetroundcap%
\pgfsetroundjoin%
\pgfsetlinewidth{0.803000pt}%
\definecolor{currentstroke}{rgb}{1.000000,1.000000,1.000000}%
\pgfsetstrokecolor{currentstroke}%
\pgfsetdash{}{0pt}%
\pgfpathmoveto{\pgfqpoint{4.893052in}{4.927727in}}%
\pgfpathlineto{\pgfqpoint{4.893052in}{5.728485in}}%
\pgfusepath{stroke}%
\end{pgfscope}%
\begin{pgfscope}%
\pgfpathrectangle{\pgfqpoint{3.976651in}{4.927727in}}{\pgfqpoint{2.818182in}{0.800758in}}%
\pgfusepath{clip}%
\pgfsetroundcap%
\pgfsetroundjoin%
\pgfsetlinewidth{0.803000pt}%
\definecolor{currentstroke}{rgb}{1.000000,1.000000,1.000000}%
\pgfsetstrokecolor{currentstroke}%
\pgfsetdash{}{0pt}%
\pgfpathmoveto{\pgfqpoint{5.090128in}{4.927727in}}%
\pgfpathlineto{\pgfqpoint{5.090128in}{5.728485in}}%
\pgfusepath{stroke}%
\end{pgfscope}%
\begin{pgfscope}%
\pgfpathrectangle{\pgfqpoint{3.976651in}{4.927727in}}{\pgfqpoint{2.818182in}{0.800758in}}%
\pgfusepath{clip}%
\pgfsetroundcap%
\pgfsetroundjoin%
\pgfsetlinewidth{0.803000pt}%
\definecolor{currentstroke}{rgb}{1.000000,1.000000,1.000000}%
\pgfsetstrokecolor{currentstroke}%
\pgfsetdash{}{0pt}%
\pgfpathmoveto{\pgfqpoint{5.287204in}{4.927727in}}%
\pgfpathlineto{\pgfqpoint{5.287204in}{5.728485in}}%
\pgfusepath{stroke}%
\end{pgfscope}%
\begin{pgfscope}%
\pgfpathrectangle{\pgfqpoint{3.976651in}{4.927727in}}{\pgfqpoint{2.818182in}{0.800758in}}%
\pgfusepath{clip}%
\pgfsetroundcap%
\pgfsetroundjoin%
\pgfsetlinewidth{0.803000pt}%
\definecolor{currentstroke}{rgb}{1.000000,1.000000,1.000000}%
\pgfsetstrokecolor{currentstroke}%
\pgfsetdash{}{0pt}%
\pgfpathmoveto{\pgfqpoint{5.484279in}{4.927727in}}%
\pgfpathlineto{\pgfqpoint{5.484279in}{5.728485in}}%
\pgfusepath{stroke}%
\end{pgfscope}%
\begin{pgfscope}%
\pgfpathrectangle{\pgfqpoint{3.976651in}{4.927727in}}{\pgfqpoint{2.818182in}{0.800758in}}%
\pgfusepath{clip}%
\pgfsetroundcap%
\pgfsetroundjoin%
\pgfsetlinewidth{0.803000pt}%
\definecolor{currentstroke}{rgb}{1.000000,1.000000,1.000000}%
\pgfsetstrokecolor{currentstroke}%
\pgfsetdash{}{0pt}%
\pgfpathmoveto{\pgfqpoint{5.681355in}{4.927727in}}%
\pgfpathlineto{\pgfqpoint{5.681355in}{5.728485in}}%
\pgfusepath{stroke}%
\end{pgfscope}%
\begin{pgfscope}%
\pgfpathrectangle{\pgfqpoint{3.976651in}{4.927727in}}{\pgfqpoint{2.818182in}{0.800758in}}%
\pgfusepath{clip}%
\pgfsetroundcap%
\pgfsetroundjoin%
\pgfsetlinewidth{0.803000pt}%
\definecolor{currentstroke}{rgb}{1.000000,1.000000,1.000000}%
\pgfsetstrokecolor{currentstroke}%
\pgfsetdash{}{0pt}%
\pgfpathmoveto{\pgfqpoint{5.878431in}{4.927727in}}%
\pgfpathlineto{\pgfqpoint{5.878431in}{5.728485in}}%
\pgfusepath{stroke}%
\end{pgfscope}%
\begin{pgfscope}%
\pgfpathrectangle{\pgfqpoint{3.976651in}{4.927727in}}{\pgfqpoint{2.818182in}{0.800758in}}%
\pgfusepath{clip}%
\pgfsetroundcap%
\pgfsetroundjoin%
\pgfsetlinewidth{0.803000pt}%
\definecolor{currentstroke}{rgb}{1.000000,1.000000,1.000000}%
\pgfsetstrokecolor{currentstroke}%
\pgfsetdash{}{0pt}%
\pgfpathmoveto{\pgfqpoint{6.075506in}{4.927727in}}%
\pgfpathlineto{\pgfqpoint{6.075506in}{5.728485in}}%
\pgfusepath{stroke}%
\end{pgfscope}%
\begin{pgfscope}%
\pgfpathrectangle{\pgfqpoint{3.976651in}{4.927727in}}{\pgfqpoint{2.818182in}{0.800758in}}%
\pgfusepath{clip}%
\pgfsetroundcap%
\pgfsetroundjoin%
\pgfsetlinewidth{0.803000pt}%
\definecolor{currentstroke}{rgb}{1.000000,1.000000,1.000000}%
\pgfsetstrokecolor{currentstroke}%
\pgfsetdash{}{0pt}%
\pgfpathmoveto{\pgfqpoint{6.272582in}{4.927727in}}%
\pgfpathlineto{\pgfqpoint{6.272582in}{5.728485in}}%
\pgfusepath{stroke}%
\end{pgfscope}%
\begin{pgfscope}%
\pgfpathrectangle{\pgfqpoint{3.976651in}{4.927727in}}{\pgfqpoint{2.818182in}{0.800758in}}%
\pgfusepath{clip}%
\pgfsetroundcap%
\pgfsetroundjoin%
\pgfsetlinewidth{0.803000pt}%
\definecolor{currentstroke}{rgb}{1.000000,1.000000,1.000000}%
\pgfsetstrokecolor{currentstroke}%
\pgfsetdash{}{0pt}%
\pgfpathmoveto{\pgfqpoint{6.469658in}{4.927727in}}%
\pgfpathlineto{\pgfqpoint{6.469658in}{5.728485in}}%
\pgfusepath{stroke}%
\end{pgfscope}%
\begin{pgfscope}%
\pgfpathrectangle{\pgfqpoint{3.976651in}{4.927727in}}{\pgfqpoint{2.818182in}{0.800758in}}%
\pgfusepath{clip}%
\pgfsetroundcap%
\pgfsetroundjoin%
\pgfsetlinewidth{0.803000pt}%
\definecolor{currentstroke}{rgb}{1.000000,1.000000,1.000000}%
\pgfsetstrokecolor{currentstroke}%
\pgfsetdash{}{0pt}%
\pgfpathmoveto{\pgfqpoint{6.666733in}{4.927727in}}%
\pgfpathlineto{\pgfqpoint{6.666733in}{5.728485in}}%
\pgfusepath{stroke}%
\end{pgfscope}%
\begin{pgfscope}%
\pgfpathrectangle{\pgfqpoint{3.976651in}{4.927727in}}{\pgfqpoint{2.818182in}{0.800758in}}%
\pgfusepath{clip}%
\pgfsetroundcap%
\pgfsetroundjoin%
\pgfsetlinewidth{0.803000pt}%
\definecolor{currentstroke}{rgb}{1.000000,1.000000,1.000000}%
\pgfsetstrokecolor{currentstroke}%
\pgfsetdash{}{0pt}%
\pgfpathmoveto{\pgfqpoint{3.976651in}{5.310467in}}%
\pgfpathlineto{\pgfqpoint{6.794832in}{5.310467in}}%
\pgfusepath{stroke}%
\end{pgfscope}%
\begin{pgfscope}%
\definecolor{textcolor}{rgb}{0.150000,0.150000,0.150000}%
\pgfsetstrokecolor{textcolor}%
\pgfsetfillcolor{textcolor}%
\pgftext[x=3.393453in,y=5.257706in,left,base]{\color{textcolor}\rmfamily\fontsize{10.000000}{12.000000}\selectfont 0.2775}%
\end{pgfscope}%
\begin{pgfscope}%
\pgfpathrectangle{\pgfqpoint{3.976651in}{4.927727in}}{\pgfqpoint{2.818182in}{0.800758in}}%
\pgfusepath{clip}%
\pgfsetroundcap%
\pgfsetroundjoin%
\pgfsetlinewidth{0.803000pt}%
\definecolor{currentstroke}{rgb}{1.000000,1.000000,1.000000}%
\pgfsetstrokecolor{currentstroke}%
\pgfsetdash{}{0pt}%
\pgfpathmoveto{\pgfqpoint{3.976651in}{5.697598in}}%
\pgfpathlineto{\pgfqpoint{6.794832in}{5.697598in}}%
\pgfusepath{stroke}%
\end{pgfscope}%
\begin{pgfscope}%
\definecolor{textcolor}{rgb}{0.150000,0.150000,0.150000}%
\pgfsetstrokecolor{textcolor}%
\pgfsetfillcolor{textcolor}%
\pgftext[x=3.393453in,y=5.644836in,left,base]{\color{textcolor}\rmfamily\fontsize{10.000000}{12.000000}\selectfont 0.2800}%
\end{pgfscope}%
\begin{pgfscope}%
\pgfpathrectangle{\pgfqpoint{3.976651in}{4.927727in}}{\pgfqpoint{2.818182in}{0.800758in}}%
\pgfusepath{clip}%
\pgfsetroundcap%
\pgfsetroundjoin%
\pgfsetlinewidth{1.505625pt}%
\definecolor{currentstroke}{rgb}{0.839216,0.152941,0.156863}%
\pgfsetstrokecolor{currentstroke}%
\pgfsetdash{}{0pt}%
\pgfpathmoveto{\pgfqpoint{4.104750in}{5.066607in}}%
\pgfpathlineto{\pgfqpoint{4.301825in}{5.430263in}}%
\pgfpathlineto{\pgfqpoint{4.498901in}{5.569282in}}%
\pgfpathlineto{\pgfqpoint{4.695977in}{4.964125in}}%
\pgfpathlineto{\pgfqpoint{4.893052in}{5.238445in}}%
\pgfpathlineto{\pgfqpoint{5.090128in}{5.029072in}}%
\pgfpathlineto{\pgfqpoint{5.287204in}{5.017866in}}%
\pgfpathlineto{\pgfqpoint{5.484279in}{4.964125in}}%
\pgfpathlineto{\pgfqpoint{5.681355in}{5.475299in}}%
\pgfpathlineto{\pgfqpoint{5.878431in}{5.692087in}}%
\pgfpathlineto{\pgfqpoint{6.075506in}{5.458151in}}%
\pgfpathlineto{\pgfqpoint{6.272582in}{4.964125in}}%
\pgfpathlineto{\pgfqpoint{6.469658in}{4.964125in}}%
\pgfpathlineto{\pgfqpoint{6.666733in}{5.142939in}}%
\pgfusepath{stroke}%
\end{pgfscope}%
\begin{pgfscope}%
\pgfpathrectangle{\pgfqpoint{3.976651in}{4.927727in}}{\pgfqpoint{2.818182in}{0.800758in}}%
\pgfusepath{clip}%
\pgfsetbuttcap%
\pgfsetroundjoin%
\definecolor{currentfill}{rgb}{0.839216,0.152941,0.156863}%
\pgfsetfillcolor{currentfill}%
\pgfsetlinewidth{1.003750pt}%
\definecolor{currentstroke}{rgb}{0.839216,0.152941,0.156863}%
\pgfsetstrokecolor{currentstroke}%
\pgfsetdash{}{0pt}%
\pgfsys@defobject{currentmarker}{\pgfqpoint{-0.041667in}{-0.041667in}}{\pgfqpoint{0.041667in}{0.041667in}}{%
\pgfpathmoveto{\pgfqpoint{0.000000in}{-0.041667in}}%
\pgfpathcurveto{\pgfqpoint{0.011050in}{-0.041667in}}{\pgfqpoint{0.021649in}{-0.037276in}}{\pgfqpoint{0.029463in}{-0.029463in}}%
\pgfpathcurveto{\pgfqpoint{0.037276in}{-0.021649in}}{\pgfqpoint{0.041667in}{-0.011050in}}{\pgfqpoint{0.041667in}{0.000000in}}%
\pgfpathcurveto{\pgfqpoint{0.041667in}{0.011050in}}{\pgfqpoint{0.037276in}{0.021649in}}{\pgfqpoint{0.029463in}{0.029463in}}%
\pgfpathcurveto{\pgfqpoint{0.021649in}{0.037276in}}{\pgfqpoint{0.011050in}{0.041667in}}{\pgfqpoint{0.000000in}{0.041667in}}%
\pgfpathcurveto{\pgfqpoint{-0.011050in}{0.041667in}}{\pgfqpoint{-0.021649in}{0.037276in}}{\pgfqpoint{-0.029463in}{0.029463in}}%
\pgfpathcurveto{\pgfqpoint{-0.037276in}{0.021649in}}{\pgfqpoint{-0.041667in}{0.011050in}}{\pgfqpoint{-0.041667in}{0.000000in}}%
\pgfpathcurveto{\pgfqpoint{-0.041667in}{-0.011050in}}{\pgfqpoint{-0.037276in}{-0.021649in}}{\pgfqpoint{-0.029463in}{-0.029463in}}%
\pgfpathcurveto{\pgfqpoint{-0.021649in}{-0.037276in}}{\pgfqpoint{-0.011050in}{-0.041667in}}{\pgfqpoint{0.000000in}{-0.041667in}}%
\pgfpathclose%
\pgfusepath{stroke,fill}%
}%
\begin{pgfscope}%
\pgfsys@transformshift{4.104750in}{5.066607in}%
\pgfsys@useobject{currentmarker}{}%
\end{pgfscope}%
\begin{pgfscope}%
\pgfsys@transformshift{4.301825in}{5.430263in}%
\pgfsys@useobject{currentmarker}{}%
\end{pgfscope}%
\begin{pgfscope}%
\pgfsys@transformshift{4.498901in}{5.569282in}%
\pgfsys@useobject{currentmarker}{}%
\end{pgfscope}%
\begin{pgfscope}%
\pgfsys@transformshift{4.695977in}{4.964125in}%
\pgfsys@useobject{currentmarker}{}%
\end{pgfscope}%
\begin{pgfscope}%
\pgfsys@transformshift{4.893052in}{5.238445in}%
\pgfsys@useobject{currentmarker}{}%
\end{pgfscope}%
\begin{pgfscope}%
\pgfsys@transformshift{5.090128in}{5.029072in}%
\pgfsys@useobject{currentmarker}{}%
\end{pgfscope}%
\begin{pgfscope}%
\pgfsys@transformshift{5.287204in}{5.017866in}%
\pgfsys@useobject{currentmarker}{}%
\end{pgfscope}%
\begin{pgfscope}%
\pgfsys@transformshift{5.484279in}{4.964125in}%
\pgfsys@useobject{currentmarker}{}%
\end{pgfscope}%
\begin{pgfscope}%
\pgfsys@transformshift{5.681355in}{5.475299in}%
\pgfsys@useobject{currentmarker}{}%
\end{pgfscope}%
\begin{pgfscope}%
\pgfsys@transformshift{5.878431in}{5.692087in}%
\pgfsys@useobject{currentmarker}{}%
\end{pgfscope}%
\begin{pgfscope}%
\pgfsys@transformshift{6.075506in}{5.458151in}%
\pgfsys@useobject{currentmarker}{}%
\end{pgfscope}%
\begin{pgfscope}%
\pgfsys@transformshift{6.272582in}{4.964125in}%
\pgfsys@useobject{currentmarker}{}%
\end{pgfscope}%
\begin{pgfscope}%
\pgfsys@transformshift{6.469658in}{4.964125in}%
\pgfsys@useobject{currentmarker}{}%
\end{pgfscope}%
\begin{pgfscope}%
\pgfsys@transformshift{6.666733in}{5.142939in}%
\pgfsys@useobject{currentmarker}{}%
\end{pgfscope}%
\end{pgfscope}%
\begin{pgfscope}%
\pgfsetrectcap%
\pgfsetmiterjoin%
\pgfsetlinewidth{0.803000pt}%
\definecolor{currentstroke}{rgb}{1.000000,1.000000,1.000000}%
\pgfsetstrokecolor{currentstroke}%
\pgfsetdash{}{0pt}%
\pgfpathmoveto{\pgfqpoint{3.976651in}{4.927727in}}%
\pgfpathlineto{\pgfqpoint{3.976651in}{5.728485in}}%
\pgfusepath{stroke}%
\end{pgfscope}%
\begin{pgfscope}%
\pgfsetrectcap%
\pgfsetmiterjoin%
\pgfsetlinewidth{0.803000pt}%
\definecolor{currentstroke}{rgb}{1.000000,1.000000,1.000000}%
\pgfsetstrokecolor{currentstroke}%
\pgfsetdash{}{0pt}%
\pgfpathmoveto{\pgfqpoint{6.794832in}{4.927727in}}%
\pgfpathlineto{\pgfqpoint{6.794832in}{5.728485in}}%
\pgfusepath{stroke}%
\end{pgfscope}%
\begin{pgfscope}%
\pgfsetrectcap%
\pgfsetmiterjoin%
\pgfsetlinewidth{0.803000pt}%
\definecolor{currentstroke}{rgb}{1.000000,1.000000,1.000000}%
\pgfsetstrokecolor{currentstroke}%
\pgfsetdash{}{0pt}%
\pgfpathmoveto{\pgfqpoint{3.976651in}{4.927727in}}%
\pgfpathlineto{\pgfqpoint{6.794832in}{4.927727in}}%
\pgfusepath{stroke}%
\end{pgfscope}%
\begin{pgfscope}%
\pgfsetrectcap%
\pgfsetmiterjoin%
\pgfsetlinewidth{0.803000pt}%
\definecolor{currentstroke}{rgb}{1.000000,1.000000,1.000000}%
\pgfsetstrokecolor{currentstroke}%
\pgfsetdash{}{0pt}%
\pgfpathmoveto{\pgfqpoint{3.976651in}{5.728485in}}%
\pgfpathlineto{\pgfqpoint{6.794832in}{5.728485in}}%
\pgfusepath{stroke}%
\end{pgfscope}%
\begin{pgfscope}%
\definecolor{textcolor}{rgb}{0.150000,0.150000,0.150000}%
\pgfsetstrokecolor{textcolor}%
\pgfsetfillcolor{textcolor}%
\pgftext[x=5.385741in,y=5.811818in,,base]{\color{textcolor}\rmfamily\fontsize{12.000000}{14.400000}\selectfont INTC}%
\end{pgfscope}%
\begin{pgfscope}%
\pgfsetbuttcap%
\pgfsetmiterjoin%
\definecolor{currentfill}{rgb}{0.917647,0.917647,0.949020}%
\pgfsetfillcolor{currentfill}%
\pgfsetlinewidth{0.000000pt}%
\definecolor{currentstroke}{rgb}{0.000000,0.000000,0.000000}%
\pgfsetstrokecolor{currentstroke}%
\pgfsetstrokeopacity{0.000000}%
\pgfsetdash{}{0pt}%
\pgfpathmoveto{\pgfqpoint{0.594832in}{3.806667in}}%
\pgfpathlineto{\pgfqpoint{3.413014in}{3.806667in}}%
\pgfpathlineto{\pgfqpoint{3.413014in}{4.607424in}}%
\pgfpathlineto{\pgfqpoint{0.594832in}{4.607424in}}%
\pgfpathclose%
\pgfusepath{fill}%
\end{pgfscope}%
\begin{pgfscope}%
\pgfpathrectangle{\pgfqpoint{0.594832in}{3.806667in}}{\pgfqpoint{2.818182in}{0.800758in}}%
\pgfusepath{clip}%
\pgfsetroundcap%
\pgfsetroundjoin%
\pgfsetlinewidth{0.803000pt}%
\definecolor{currentstroke}{rgb}{1.000000,1.000000,1.000000}%
\pgfsetstrokecolor{currentstroke}%
\pgfsetdash{}{0pt}%
\pgfpathmoveto{\pgfqpoint{0.722932in}{3.806667in}}%
\pgfpathlineto{\pgfqpoint{0.722932in}{4.607424in}}%
\pgfusepath{stroke}%
\end{pgfscope}%
\begin{pgfscope}%
\pgfpathrectangle{\pgfqpoint{0.594832in}{3.806667in}}{\pgfqpoint{2.818182in}{0.800758in}}%
\pgfusepath{clip}%
\pgfsetroundcap%
\pgfsetroundjoin%
\pgfsetlinewidth{0.803000pt}%
\definecolor{currentstroke}{rgb}{1.000000,1.000000,1.000000}%
\pgfsetstrokecolor{currentstroke}%
\pgfsetdash{}{0pt}%
\pgfpathmoveto{\pgfqpoint{0.920007in}{3.806667in}}%
\pgfpathlineto{\pgfqpoint{0.920007in}{4.607424in}}%
\pgfusepath{stroke}%
\end{pgfscope}%
\begin{pgfscope}%
\pgfpathrectangle{\pgfqpoint{0.594832in}{3.806667in}}{\pgfqpoint{2.818182in}{0.800758in}}%
\pgfusepath{clip}%
\pgfsetroundcap%
\pgfsetroundjoin%
\pgfsetlinewidth{0.803000pt}%
\definecolor{currentstroke}{rgb}{1.000000,1.000000,1.000000}%
\pgfsetstrokecolor{currentstroke}%
\pgfsetdash{}{0pt}%
\pgfpathmoveto{\pgfqpoint{1.117083in}{3.806667in}}%
\pgfpathlineto{\pgfqpoint{1.117083in}{4.607424in}}%
\pgfusepath{stroke}%
\end{pgfscope}%
\begin{pgfscope}%
\pgfpathrectangle{\pgfqpoint{0.594832in}{3.806667in}}{\pgfqpoint{2.818182in}{0.800758in}}%
\pgfusepath{clip}%
\pgfsetroundcap%
\pgfsetroundjoin%
\pgfsetlinewidth{0.803000pt}%
\definecolor{currentstroke}{rgb}{1.000000,1.000000,1.000000}%
\pgfsetstrokecolor{currentstroke}%
\pgfsetdash{}{0pt}%
\pgfpathmoveto{\pgfqpoint{1.314158in}{3.806667in}}%
\pgfpathlineto{\pgfqpoint{1.314158in}{4.607424in}}%
\pgfusepath{stroke}%
\end{pgfscope}%
\begin{pgfscope}%
\pgfpathrectangle{\pgfqpoint{0.594832in}{3.806667in}}{\pgfqpoint{2.818182in}{0.800758in}}%
\pgfusepath{clip}%
\pgfsetroundcap%
\pgfsetroundjoin%
\pgfsetlinewidth{0.803000pt}%
\definecolor{currentstroke}{rgb}{1.000000,1.000000,1.000000}%
\pgfsetstrokecolor{currentstroke}%
\pgfsetdash{}{0pt}%
\pgfpathmoveto{\pgfqpoint{1.511234in}{3.806667in}}%
\pgfpathlineto{\pgfqpoint{1.511234in}{4.607424in}}%
\pgfusepath{stroke}%
\end{pgfscope}%
\begin{pgfscope}%
\pgfpathrectangle{\pgfqpoint{0.594832in}{3.806667in}}{\pgfqpoint{2.818182in}{0.800758in}}%
\pgfusepath{clip}%
\pgfsetroundcap%
\pgfsetroundjoin%
\pgfsetlinewidth{0.803000pt}%
\definecolor{currentstroke}{rgb}{1.000000,1.000000,1.000000}%
\pgfsetstrokecolor{currentstroke}%
\pgfsetdash{}{0pt}%
\pgfpathmoveto{\pgfqpoint{1.708310in}{3.806667in}}%
\pgfpathlineto{\pgfqpoint{1.708310in}{4.607424in}}%
\pgfusepath{stroke}%
\end{pgfscope}%
\begin{pgfscope}%
\pgfpathrectangle{\pgfqpoint{0.594832in}{3.806667in}}{\pgfqpoint{2.818182in}{0.800758in}}%
\pgfusepath{clip}%
\pgfsetroundcap%
\pgfsetroundjoin%
\pgfsetlinewidth{0.803000pt}%
\definecolor{currentstroke}{rgb}{1.000000,1.000000,1.000000}%
\pgfsetstrokecolor{currentstroke}%
\pgfsetdash{}{0pt}%
\pgfpathmoveto{\pgfqpoint{1.905385in}{3.806667in}}%
\pgfpathlineto{\pgfqpoint{1.905385in}{4.607424in}}%
\pgfusepath{stroke}%
\end{pgfscope}%
\begin{pgfscope}%
\pgfpathrectangle{\pgfqpoint{0.594832in}{3.806667in}}{\pgfqpoint{2.818182in}{0.800758in}}%
\pgfusepath{clip}%
\pgfsetroundcap%
\pgfsetroundjoin%
\pgfsetlinewidth{0.803000pt}%
\definecolor{currentstroke}{rgb}{1.000000,1.000000,1.000000}%
\pgfsetstrokecolor{currentstroke}%
\pgfsetdash{}{0pt}%
\pgfpathmoveto{\pgfqpoint{2.102461in}{3.806667in}}%
\pgfpathlineto{\pgfqpoint{2.102461in}{4.607424in}}%
\pgfusepath{stroke}%
\end{pgfscope}%
\begin{pgfscope}%
\pgfpathrectangle{\pgfqpoint{0.594832in}{3.806667in}}{\pgfqpoint{2.818182in}{0.800758in}}%
\pgfusepath{clip}%
\pgfsetroundcap%
\pgfsetroundjoin%
\pgfsetlinewidth{0.803000pt}%
\definecolor{currentstroke}{rgb}{1.000000,1.000000,1.000000}%
\pgfsetstrokecolor{currentstroke}%
\pgfsetdash{}{0pt}%
\pgfpathmoveto{\pgfqpoint{2.299537in}{3.806667in}}%
\pgfpathlineto{\pgfqpoint{2.299537in}{4.607424in}}%
\pgfusepath{stroke}%
\end{pgfscope}%
\begin{pgfscope}%
\pgfpathrectangle{\pgfqpoint{0.594832in}{3.806667in}}{\pgfqpoint{2.818182in}{0.800758in}}%
\pgfusepath{clip}%
\pgfsetroundcap%
\pgfsetroundjoin%
\pgfsetlinewidth{0.803000pt}%
\definecolor{currentstroke}{rgb}{1.000000,1.000000,1.000000}%
\pgfsetstrokecolor{currentstroke}%
\pgfsetdash{}{0pt}%
\pgfpathmoveto{\pgfqpoint{2.496612in}{3.806667in}}%
\pgfpathlineto{\pgfqpoint{2.496612in}{4.607424in}}%
\pgfusepath{stroke}%
\end{pgfscope}%
\begin{pgfscope}%
\pgfpathrectangle{\pgfqpoint{0.594832in}{3.806667in}}{\pgfqpoint{2.818182in}{0.800758in}}%
\pgfusepath{clip}%
\pgfsetroundcap%
\pgfsetroundjoin%
\pgfsetlinewidth{0.803000pt}%
\definecolor{currentstroke}{rgb}{1.000000,1.000000,1.000000}%
\pgfsetstrokecolor{currentstroke}%
\pgfsetdash{}{0pt}%
\pgfpathmoveto{\pgfqpoint{2.693688in}{3.806667in}}%
\pgfpathlineto{\pgfqpoint{2.693688in}{4.607424in}}%
\pgfusepath{stroke}%
\end{pgfscope}%
\begin{pgfscope}%
\pgfpathrectangle{\pgfqpoint{0.594832in}{3.806667in}}{\pgfqpoint{2.818182in}{0.800758in}}%
\pgfusepath{clip}%
\pgfsetroundcap%
\pgfsetroundjoin%
\pgfsetlinewidth{0.803000pt}%
\definecolor{currentstroke}{rgb}{1.000000,1.000000,1.000000}%
\pgfsetstrokecolor{currentstroke}%
\pgfsetdash{}{0pt}%
\pgfpathmoveto{\pgfqpoint{2.890764in}{3.806667in}}%
\pgfpathlineto{\pgfqpoint{2.890764in}{4.607424in}}%
\pgfusepath{stroke}%
\end{pgfscope}%
\begin{pgfscope}%
\pgfpathrectangle{\pgfqpoint{0.594832in}{3.806667in}}{\pgfqpoint{2.818182in}{0.800758in}}%
\pgfusepath{clip}%
\pgfsetroundcap%
\pgfsetroundjoin%
\pgfsetlinewidth{0.803000pt}%
\definecolor{currentstroke}{rgb}{1.000000,1.000000,1.000000}%
\pgfsetstrokecolor{currentstroke}%
\pgfsetdash{}{0pt}%
\pgfpathmoveto{\pgfqpoint{3.087839in}{3.806667in}}%
\pgfpathlineto{\pgfqpoint{3.087839in}{4.607424in}}%
\pgfusepath{stroke}%
\end{pgfscope}%
\begin{pgfscope}%
\pgfpathrectangle{\pgfqpoint{0.594832in}{3.806667in}}{\pgfqpoint{2.818182in}{0.800758in}}%
\pgfusepath{clip}%
\pgfsetroundcap%
\pgfsetroundjoin%
\pgfsetlinewidth{0.803000pt}%
\definecolor{currentstroke}{rgb}{1.000000,1.000000,1.000000}%
\pgfsetstrokecolor{currentstroke}%
\pgfsetdash{}{0pt}%
\pgfpathmoveto{\pgfqpoint{3.284915in}{3.806667in}}%
\pgfpathlineto{\pgfqpoint{3.284915in}{4.607424in}}%
\pgfusepath{stroke}%
\end{pgfscope}%
\begin{pgfscope}%
\pgfpathrectangle{\pgfqpoint{0.594832in}{3.806667in}}{\pgfqpoint{2.818182in}{0.800758in}}%
\pgfusepath{clip}%
\pgfsetroundcap%
\pgfsetroundjoin%
\pgfsetlinewidth{0.803000pt}%
\definecolor{currentstroke}{rgb}{1.000000,1.000000,1.000000}%
\pgfsetstrokecolor{currentstroke}%
\pgfsetdash{}{0pt}%
\pgfpathmoveto{\pgfqpoint{0.594832in}{3.819921in}}%
\pgfpathlineto{\pgfqpoint{3.413014in}{3.819921in}}%
\pgfusepath{stroke}%
\end{pgfscope}%
\begin{pgfscope}%
\definecolor{textcolor}{rgb}{0.150000,0.150000,0.150000}%
\pgfsetstrokecolor{textcolor}%
\pgfsetfillcolor{textcolor}%
\pgftext[x=0.100000in,y=3.767159in,left,base]{\color{textcolor}\rmfamily\fontsize{10.000000}{12.000000}\selectfont 0.104}%
\end{pgfscope}%
\begin{pgfscope}%
\pgfpathrectangle{\pgfqpoint{0.594832in}{3.806667in}}{\pgfqpoint{2.818182in}{0.800758in}}%
\pgfusepath{clip}%
\pgfsetroundcap%
\pgfsetroundjoin%
\pgfsetlinewidth{0.803000pt}%
\definecolor{currentstroke}{rgb}{1.000000,1.000000,1.000000}%
\pgfsetstrokecolor{currentstroke}%
\pgfsetdash{}{0pt}%
\pgfpathmoveto{\pgfqpoint{0.594832in}{4.432280in}}%
\pgfpathlineto{\pgfqpoint{3.413014in}{4.432280in}}%
\pgfusepath{stroke}%
\end{pgfscope}%
\begin{pgfscope}%
\definecolor{textcolor}{rgb}{0.150000,0.150000,0.150000}%
\pgfsetstrokecolor{textcolor}%
\pgfsetfillcolor{textcolor}%
\pgftext[x=0.100000in,y=4.379519in,left,base]{\color{textcolor}\rmfamily\fontsize{10.000000}{12.000000}\selectfont 0.105}%
\end{pgfscope}%
\begin{pgfscope}%
\pgfpathrectangle{\pgfqpoint{0.594832in}{3.806667in}}{\pgfqpoint{2.818182in}{0.800758in}}%
\pgfusepath{clip}%
\pgfsetroundcap%
\pgfsetroundjoin%
\pgfsetlinewidth{1.505625pt}%
\definecolor{currentstroke}{rgb}{0.580392,0.403922,0.741176}%
\pgfsetstrokecolor{currentstroke}%
\pgfsetdash{}{0pt}%
\pgfpathmoveto{\pgfqpoint{0.722932in}{3.843065in}}%
\pgfpathlineto{\pgfqpoint{0.920007in}{4.062354in}}%
\pgfpathlineto{\pgfqpoint{1.117083in}{4.449262in}}%
\pgfpathlineto{\pgfqpoint{1.314158in}{4.007341in}}%
\pgfpathlineto{\pgfqpoint{1.511234in}{4.100491in}}%
\pgfpathlineto{\pgfqpoint{1.708310in}{3.975434in}}%
\pgfpathlineto{\pgfqpoint{1.905385in}{4.225685in}}%
\pgfpathlineto{\pgfqpoint{2.102461in}{4.007341in}}%
\pgfpathlineto{\pgfqpoint{2.299537in}{4.484071in}}%
\pgfpathlineto{\pgfqpoint{2.496612in}{4.571026in}}%
\pgfpathlineto{\pgfqpoint{2.693688in}{4.361690in}}%
\pgfpathlineto{\pgfqpoint{2.890764in}{4.007341in}}%
\pgfpathlineto{\pgfqpoint{3.087839in}{4.007341in}}%
\pgfpathlineto{\pgfqpoint{3.284915in}{4.028300in}}%
\pgfusepath{stroke}%
\end{pgfscope}%
\begin{pgfscope}%
\pgfpathrectangle{\pgfqpoint{0.594832in}{3.806667in}}{\pgfqpoint{2.818182in}{0.800758in}}%
\pgfusepath{clip}%
\pgfsetbuttcap%
\pgfsetroundjoin%
\definecolor{currentfill}{rgb}{0.580392,0.403922,0.741176}%
\pgfsetfillcolor{currentfill}%
\pgfsetlinewidth{1.003750pt}%
\definecolor{currentstroke}{rgb}{0.580392,0.403922,0.741176}%
\pgfsetstrokecolor{currentstroke}%
\pgfsetdash{}{0pt}%
\pgfsys@defobject{currentmarker}{\pgfqpoint{-0.041667in}{-0.041667in}}{\pgfqpoint{0.041667in}{0.041667in}}{%
\pgfpathmoveto{\pgfqpoint{0.000000in}{-0.041667in}}%
\pgfpathcurveto{\pgfqpoint{0.011050in}{-0.041667in}}{\pgfqpoint{0.021649in}{-0.037276in}}{\pgfqpoint{0.029463in}{-0.029463in}}%
\pgfpathcurveto{\pgfqpoint{0.037276in}{-0.021649in}}{\pgfqpoint{0.041667in}{-0.011050in}}{\pgfqpoint{0.041667in}{0.000000in}}%
\pgfpathcurveto{\pgfqpoint{0.041667in}{0.011050in}}{\pgfqpoint{0.037276in}{0.021649in}}{\pgfqpoint{0.029463in}{0.029463in}}%
\pgfpathcurveto{\pgfqpoint{0.021649in}{0.037276in}}{\pgfqpoint{0.011050in}{0.041667in}}{\pgfqpoint{0.000000in}{0.041667in}}%
\pgfpathcurveto{\pgfqpoint{-0.011050in}{0.041667in}}{\pgfqpoint{-0.021649in}{0.037276in}}{\pgfqpoint{-0.029463in}{0.029463in}}%
\pgfpathcurveto{\pgfqpoint{-0.037276in}{0.021649in}}{\pgfqpoint{-0.041667in}{0.011050in}}{\pgfqpoint{-0.041667in}{0.000000in}}%
\pgfpathcurveto{\pgfqpoint{-0.041667in}{-0.011050in}}{\pgfqpoint{-0.037276in}{-0.021649in}}{\pgfqpoint{-0.029463in}{-0.029463in}}%
\pgfpathcurveto{\pgfqpoint{-0.021649in}{-0.037276in}}{\pgfqpoint{-0.011050in}{-0.041667in}}{\pgfqpoint{0.000000in}{-0.041667in}}%
\pgfpathclose%
\pgfusepath{stroke,fill}%
}%
\begin{pgfscope}%
\pgfsys@transformshift{0.722932in}{3.843065in}%
\pgfsys@useobject{currentmarker}{}%
\end{pgfscope}%
\begin{pgfscope}%
\pgfsys@transformshift{0.920007in}{4.062354in}%
\pgfsys@useobject{currentmarker}{}%
\end{pgfscope}%
\begin{pgfscope}%
\pgfsys@transformshift{1.117083in}{4.449262in}%
\pgfsys@useobject{currentmarker}{}%
\end{pgfscope}%
\begin{pgfscope}%
\pgfsys@transformshift{1.314158in}{4.007341in}%
\pgfsys@useobject{currentmarker}{}%
\end{pgfscope}%
\begin{pgfscope}%
\pgfsys@transformshift{1.511234in}{4.100491in}%
\pgfsys@useobject{currentmarker}{}%
\end{pgfscope}%
\begin{pgfscope}%
\pgfsys@transformshift{1.708310in}{3.975434in}%
\pgfsys@useobject{currentmarker}{}%
\end{pgfscope}%
\begin{pgfscope}%
\pgfsys@transformshift{1.905385in}{4.225685in}%
\pgfsys@useobject{currentmarker}{}%
\end{pgfscope}%
\begin{pgfscope}%
\pgfsys@transformshift{2.102461in}{4.007341in}%
\pgfsys@useobject{currentmarker}{}%
\end{pgfscope}%
\begin{pgfscope}%
\pgfsys@transformshift{2.299537in}{4.484071in}%
\pgfsys@useobject{currentmarker}{}%
\end{pgfscope}%
\begin{pgfscope}%
\pgfsys@transformshift{2.496612in}{4.571026in}%
\pgfsys@useobject{currentmarker}{}%
\end{pgfscope}%
\begin{pgfscope}%
\pgfsys@transformshift{2.693688in}{4.361690in}%
\pgfsys@useobject{currentmarker}{}%
\end{pgfscope}%
\begin{pgfscope}%
\pgfsys@transformshift{2.890764in}{4.007341in}%
\pgfsys@useobject{currentmarker}{}%
\end{pgfscope}%
\begin{pgfscope}%
\pgfsys@transformshift{3.087839in}{4.007341in}%
\pgfsys@useobject{currentmarker}{}%
\end{pgfscope}%
\begin{pgfscope}%
\pgfsys@transformshift{3.284915in}{4.028300in}%
\pgfsys@useobject{currentmarker}{}%
\end{pgfscope}%
\end{pgfscope}%
\begin{pgfscope}%
\pgfsetrectcap%
\pgfsetmiterjoin%
\pgfsetlinewidth{0.803000pt}%
\definecolor{currentstroke}{rgb}{1.000000,1.000000,1.000000}%
\pgfsetstrokecolor{currentstroke}%
\pgfsetdash{}{0pt}%
\pgfpathmoveto{\pgfqpoint{0.594832in}{3.806667in}}%
\pgfpathlineto{\pgfqpoint{0.594832in}{4.607424in}}%
\pgfusepath{stroke}%
\end{pgfscope}%
\begin{pgfscope}%
\pgfsetrectcap%
\pgfsetmiterjoin%
\pgfsetlinewidth{0.803000pt}%
\definecolor{currentstroke}{rgb}{1.000000,1.000000,1.000000}%
\pgfsetstrokecolor{currentstroke}%
\pgfsetdash{}{0pt}%
\pgfpathmoveto{\pgfqpoint{3.413014in}{3.806667in}}%
\pgfpathlineto{\pgfqpoint{3.413014in}{4.607424in}}%
\pgfusepath{stroke}%
\end{pgfscope}%
\begin{pgfscope}%
\pgfsetrectcap%
\pgfsetmiterjoin%
\pgfsetlinewidth{0.803000pt}%
\definecolor{currentstroke}{rgb}{1.000000,1.000000,1.000000}%
\pgfsetstrokecolor{currentstroke}%
\pgfsetdash{}{0pt}%
\pgfpathmoveto{\pgfqpoint{0.594832in}{3.806667in}}%
\pgfpathlineto{\pgfqpoint{3.413014in}{3.806667in}}%
\pgfusepath{stroke}%
\end{pgfscope}%
\begin{pgfscope}%
\pgfsetrectcap%
\pgfsetmiterjoin%
\pgfsetlinewidth{0.803000pt}%
\definecolor{currentstroke}{rgb}{1.000000,1.000000,1.000000}%
\pgfsetstrokecolor{currentstroke}%
\pgfsetdash{}{0pt}%
\pgfpathmoveto{\pgfqpoint{0.594832in}{4.607424in}}%
\pgfpathlineto{\pgfqpoint{3.413014in}{4.607424in}}%
\pgfusepath{stroke}%
\end{pgfscope}%
\begin{pgfscope}%
\definecolor{textcolor}{rgb}{0.150000,0.150000,0.150000}%
\pgfsetstrokecolor{textcolor}%
\pgfsetfillcolor{textcolor}%
\pgftext[x=2.003923in,y=4.690758in,,base]{\color{textcolor}\rmfamily\fontsize{12.000000}{14.400000}\selectfont JNJ}%
\end{pgfscope}%
\begin{pgfscope}%
\pgfsetbuttcap%
\pgfsetmiterjoin%
\definecolor{currentfill}{rgb}{0.917647,0.917647,0.949020}%
\pgfsetfillcolor{currentfill}%
\pgfsetlinewidth{0.000000pt}%
\definecolor{currentstroke}{rgb}{0.000000,0.000000,0.000000}%
\pgfsetstrokecolor{currentstroke}%
\pgfsetstrokeopacity{0.000000}%
\pgfsetdash{}{0pt}%
\pgfpathmoveto{\pgfqpoint{3.976651in}{3.806667in}}%
\pgfpathlineto{\pgfqpoint{6.794832in}{3.806667in}}%
\pgfpathlineto{\pgfqpoint{6.794832in}{4.607424in}}%
\pgfpathlineto{\pgfqpoint{3.976651in}{4.607424in}}%
\pgfpathclose%
\pgfusepath{fill}%
\end{pgfscope}%
\begin{pgfscope}%
\pgfpathrectangle{\pgfqpoint{3.976651in}{3.806667in}}{\pgfqpoint{2.818182in}{0.800758in}}%
\pgfusepath{clip}%
\pgfsetroundcap%
\pgfsetroundjoin%
\pgfsetlinewidth{0.803000pt}%
\definecolor{currentstroke}{rgb}{1.000000,1.000000,1.000000}%
\pgfsetstrokecolor{currentstroke}%
\pgfsetdash{}{0pt}%
\pgfpathmoveto{\pgfqpoint{4.104750in}{3.806667in}}%
\pgfpathlineto{\pgfqpoint{4.104750in}{4.607424in}}%
\pgfusepath{stroke}%
\end{pgfscope}%
\begin{pgfscope}%
\pgfpathrectangle{\pgfqpoint{3.976651in}{3.806667in}}{\pgfqpoint{2.818182in}{0.800758in}}%
\pgfusepath{clip}%
\pgfsetroundcap%
\pgfsetroundjoin%
\pgfsetlinewidth{0.803000pt}%
\definecolor{currentstroke}{rgb}{1.000000,1.000000,1.000000}%
\pgfsetstrokecolor{currentstroke}%
\pgfsetdash{}{0pt}%
\pgfpathmoveto{\pgfqpoint{4.301825in}{3.806667in}}%
\pgfpathlineto{\pgfqpoint{4.301825in}{4.607424in}}%
\pgfusepath{stroke}%
\end{pgfscope}%
\begin{pgfscope}%
\pgfpathrectangle{\pgfqpoint{3.976651in}{3.806667in}}{\pgfqpoint{2.818182in}{0.800758in}}%
\pgfusepath{clip}%
\pgfsetroundcap%
\pgfsetroundjoin%
\pgfsetlinewidth{0.803000pt}%
\definecolor{currentstroke}{rgb}{1.000000,1.000000,1.000000}%
\pgfsetstrokecolor{currentstroke}%
\pgfsetdash{}{0pt}%
\pgfpathmoveto{\pgfqpoint{4.498901in}{3.806667in}}%
\pgfpathlineto{\pgfqpoint{4.498901in}{4.607424in}}%
\pgfusepath{stroke}%
\end{pgfscope}%
\begin{pgfscope}%
\pgfpathrectangle{\pgfqpoint{3.976651in}{3.806667in}}{\pgfqpoint{2.818182in}{0.800758in}}%
\pgfusepath{clip}%
\pgfsetroundcap%
\pgfsetroundjoin%
\pgfsetlinewidth{0.803000pt}%
\definecolor{currentstroke}{rgb}{1.000000,1.000000,1.000000}%
\pgfsetstrokecolor{currentstroke}%
\pgfsetdash{}{0pt}%
\pgfpathmoveto{\pgfqpoint{4.695977in}{3.806667in}}%
\pgfpathlineto{\pgfqpoint{4.695977in}{4.607424in}}%
\pgfusepath{stroke}%
\end{pgfscope}%
\begin{pgfscope}%
\pgfpathrectangle{\pgfqpoint{3.976651in}{3.806667in}}{\pgfqpoint{2.818182in}{0.800758in}}%
\pgfusepath{clip}%
\pgfsetroundcap%
\pgfsetroundjoin%
\pgfsetlinewidth{0.803000pt}%
\definecolor{currentstroke}{rgb}{1.000000,1.000000,1.000000}%
\pgfsetstrokecolor{currentstroke}%
\pgfsetdash{}{0pt}%
\pgfpathmoveto{\pgfqpoint{4.893052in}{3.806667in}}%
\pgfpathlineto{\pgfqpoint{4.893052in}{4.607424in}}%
\pgfusepath{stroke}%
\end{pgfscope}%
\begin{pgfscope}%
\pgfpathrectangle{\pgfqpoint{3.976651in}{3.806667in}}{\pgfqpoint{2.818182in}{0.800758in}}%
\pgfusepath{clip}%
\pgfsetroundcap%
\pgfsetroundjoin%
\pgfsetlinewidth{0.803000pt}%
\definecolor{currentstroke}{rgb}{1.000000,1.000000,1.000000}%
\pgfsetstrokecolor{currentstroke}%
\pgfsetdash{}{0pt}%
\pgfpathmoveto{\pgfqpoint{5.090128in}{3.806667in}}%
\pgfpathlineto{\pgfqpoint{5.090128in}{4.607424in}}%
\pgfusepath{stroke}%
\end{pgfscope}%
\begin{pgfscope}%
\pgfpathrectangle{\pgfqpoint{3.976651in}{3.806667in}}{\pgfqpoint{2.818182in}{0.800758in}}%
\pgfusepath{clip}%
\pgfsetroundcap%
\pgfsetroundjoin%
\pgfsetlinewidth{0.803000pt}%
\definecolor{currentstroke}{rgb}{1.000000,1.000000,1.000000}%
\pgfsetstrokecolor{currentstroke}%
\pgfsetdash{}{0pt}%
\pgfpathmoveto{\pgfqpoint{5.287204in}{3.806667in}}%
\pgfpathlineto{\pgfqpoint{5.287204in}{4.607424in}}%
\pgfusepath{stroke}%
\end{pgfscope}%
\begin{pgfscope}%
\pgfpathrectangle{\pgfqpoint{3.976651in}{3.806667in}}{\pgfqpoint{2.818182in}{0.800758in}}%
\pgfusepath{clip}%
\pgfsetroundcap%
\pgfsetroundjoin%
\pgfsetlinewidth{0.803000pt}%
\definecolor{currentstroke}{rgb}{1.000000,1.000000,1.000000}%
\pgfsetstrokecolor{currentstroke}%
\pgfsetdash{}{0pt}%
\pgfpathmoveto{\pgfqpoint{5.484279in}{3.806667in}}%
\pgfpathlineto{\pgfqpoint{5.484279in}{4.607424in}}%
\pgfusepath{stroke}%
\end{pgfscope}%
\begin{pgfscope}%
\pgfpathrectangle{\pgfqpoint{3.976651in}{3.806667in}}{\pgfqpoint{2.818182in}{0.800758in}}%
\pgfusepath{clip}%
\pgfsetroundcap%
\pgfsetroundjoin%
\pgfsetlinewidth{0.803000pt}%
\definecolor{currentstroke}{rgb}{1.000000,1.000000,1.000000}%
\pgfsetstrokecolor{currentstroke}%
\pgfsetdash{}{0pt}%
\pgfpathmoveto{\pgfqpoint{5.681355in}{3.806667in}}%
\pgfpathlineto{\pgfqpoint{5.681355in}{4.607424in}}%
\pgfusepath{stroke}%
\end{pgfscope}%
\begin{pgfscope}%
\pgfpathrectangle{\pgfqpoint{3.976651in}{3.806667in}}{\pgfqpoint{2.818182in}{0.800758in}}%
\pgfusepath{clip}%
\pgfsetroundcap%
\pgfsetroundjoin%
\pgfsetlinewidth{0.803000pt}%
\definecolor{currentstroke}{rgb}{1.000000,1.000000,1.000000}%
\pgfsetstrokecolor{currentstroke}%
\pgfsetdash{}{0pt}%
\pgfpathmoveto{\pgfqpoint{5.878431in}{3.806667in}}%
\pgfpathlineto{\pgfqpoint{5.878431in}{4.607424in}}%
\pgfusepath{stroke}%
\end{pgfscope}%
\begin{pgfscope}%
\pgfpathrectangle{\pgfqpoint{3.976651in}{3.806667in}}{\pgfqpoint{2.818182in}{0.800758in}}%
\pgfusepath{clip}%
\pgfsetroundcap%
\pgfsetroundjoin%
\pgfsetlinewidth{0.803000pt}%
\definecolor{currentstroke}{rgb}{1.000000,1.000000,1.000000}%
\pgfsetstrokecolor{currentstroke}%
\pgfsetdash{}{0pt}%
\pgfpathmoveto{\pgfqpoint{6.075506in}{3.806667in}}%
\pgfpathlineto{\pgfqpoint{6.075506in}{4.607424in}}%
\pgfusepath{stroke}%
\end{pgfscope}%
\begin{pgfscope}%
\pgfpathrectangle{\pgfqpoint{3.976651in}{3.806667in}}{\pgfqpoint{2.818182in}{0.800758in}}%
\pgfusepath{clip}%
\pgfsetroundcap%
\pgfsetroundjoin%
\pgfsetlinewidth{0.803000pt}%
\definecolor{currentstroke}{rgb}{1.000000,1.000000,1.000000}%
\pgfsetstrokecolor{currentstroke}%
\pgfsetdash{}{0pt}%
\pgfpathmoveto{\pgfqpoint{6.272582in}{3.806667in}}%
\pgfpathlineto{\pgfqpoint{6.272582in}{4.607424in}}%
\pgfusepath{stroke}%
\end{pgfscope}%
\begin{pgfscope}%
\pgfpathrectangle{\pgfqpoint{3.976651in}{3.806667in}}{\pgfqpoint{2.818182in}{0.800758in}}%
\pgfusepath{clip}%
\pgfsetroundcap%
\pgfsetroundjoin%
\pgfsetlinewidth{0.803000pt}%
\definecolor{currentstroke}{rgb}{1.000000,1.000000,1.000000}%
\pgfsetstrokecolor{currentstroke}%
\pgfsetdash{}{0pt}%
\pgfpathmoveto{\pgfqpoint{6.469658in}{3.806667in}}%
\pgfpathlineto{\pgfqpoint{6.469658in}{4.607424in}}%
\pgfusepath{stroke}%
\end{pgfscope}%
\begin{pgfscope}%
\pgfpathrectangle{\pgfqpoint{3.976651in}{3.806667in}}{\pgfqpoint{2.818182in}{0.800758in}}%
\pgfusepath{clip}%
\pgfsetroundcap%
\pgfsetroundjoin%
\pgfsetlinewidth{0.803000pt}%
\definecolor{currentstroke}{rgb}{1.000000,1.000000,1.000000}%
\pgfsetstrokecolor{currentstroke}%
\pgfsetdash{}{0pt}%
\pgfpathmoveto{\pgfqpoint{6.666733in}{3.806667in}}%
\pgfpathlineto{\pgfqpoint{6.666733in}{4.607424in}}%
\pgfusepath{stroke}%
\end{pgfscope}%
\begin{pgfscope}%
\pgfpathrectangle{\pgfqpoint{3.976651in}{3.806667in}}{\pgfqpoint{2.818182in}{0.800758in}}%
\pgfusepath{clip}%
\pgfsetroundcap%
\pgfsetroundjoin%
\pgfsetlinewidth{0.803000pt}%
\definecolor{currentstroke}{rgb}{1.000000,1.000000,1.000000}%
\pgfsetstrokecolor{currentstroke}%
\pgfsetdash{}{0pt}%
\pgfpathmoveto{\pgfqpoint{3.976651in}{3.843014in}}%
\pgfpathlineto{\pgfqpoint{6.794832in}{3.843014in}}%
\pgfusepath{stroke}%
\end{pgfscope}%
\begin{pgfscope}%
\definecolor{textcolor}{rgb}{0.150000,0.150000,0.150000}%
\pgfsetstrokecolor{textcolor}%
\pgfsetfillcolor{textcolor}%
\pgftext[x=3.791063in,y=3.790252in,left,base]{\color{textcolor}\rmfamily\fontsize{10.000000}{12.000000}\selectfont 0}%
\end{pgfscope}%
\begin{pgfscope}%
\pgfpathrectangle{\pgfqpoint{3.976651in}{3.806667in}}{\pgfqpoint{2.818182in}{0.800758in}}%
\pgfusepath{clip}%
\pgfsetroundcap%
\pgfsetroundjoin%
\pgfsetlinewidth{0.803000pt}%
\definecolor{currentstroke}{rgb}{1.000000,1.000000,1.000000}%
\pgfsetstrokecolor{currentstroke}%
\pgfsetdash{}{0pt}%
\pgfpathmoveto{\pgfqpoint{3.976651in}{4.288430in}}%
\pgfpathlineto{\pgfqpoint{6.794832in}{4.288430in}}%
\pgfusepath{stroke}%
\end{pgfscope}%
\begin{pgfscope}%
\definecolor{textcolor}{rgb}{0.150000,0.150000,0.150000}%
\pgfsetstrokecolor{textcolor}%
\pgfsetfillcolor{textcolor}%
\pgftext[x=3.525967in,y=4.235668in,left,base]{\color{textcolor}\rmfamily\fontsize{10.000000}{12.000000}\selectfont 1000}%
\end{pgfscope}%
\begin{pgfscope}%
\pgfpathrectangle{\pgfqpoint{3.976651in}{3.806667in}}{\pgfqpoint{2.818182in}{0.800758in}}%
\pgfusepath{clip}%
\pgfsetroundcap%
\pgfsetroundjoin%
\pgfsetlinewidth{1.505625pt}%
\definecolor{currentstroke}{rgb}{0.549020,0.337255,0.294118}%
\pgfsetstrokecolor{currentstroke}%
\pgfsetdash{}{0pt}%
\pgfpathmoveto{\pgfqpoint{4.104750in}{3.843065in}}%
\pgfpathlineto{\pgfqpoint{4.301825in}{3.843065in}}%
\pgfpathlineto{\pgfqpoint{4.498901in}{3.843065in}}%
\pgfpathlineto{\pgfqpoint{4.695977in}{3.843065in}}%
\pgfpathlineto{\pgfqpoint{4.893052in}{3.843065in}}%
\pgfpathlineto{\pgfqpoint{5.090128in}{3.843065in}}%
\pgfpathlineto{\pgfqpoint{5.287204in}{3.843065in}}%
\pgfpathlineto{\pgfqpoint{5.484279in}{3.843065in}}%
\pgfpathlineto{\pgfqpoint{5.681355in}{3.843065in}}%
\pgfpathlineto{\pgfqpoint{5.878431in}{3.843065in}}%
\pgfpathlineto{\pgfqpoint{6.075506in}{4.571026in}}%
\pgfpathlineto{\pgfqpoint{6.272582in}{3.843065in}}%
\pgfpathlineto{\pgfqpoint{6.469658in}{3.843065in}}%
\pgfpathlineto{\pgfqpoint{6.666733in}{3.843065in}}%
\pgfusepath{stroke}%
\end{pgfscope}%
\begin{pgfscope}%
\pgfpathrectangle{\pgfqpoint{3.976651in}{3.806667in}}{\pgfqpoint{2.818182in}{0.800758in}}%
\pgfusepath{clip}%
\pgfsetbuttcap%
\pgfsetroundjoin%
\definecolor{currentfill}{rgb}{0.549020,0.337255,0.294118}%
\pgfsetfillcolor{currentfill}%
\pgfsetlinewidth{1.003750pt}%
\definecolor{currentstroke}{rgb}{0.549020,0.337255,0.294118}%
\pgfsetstrokecolor{currentstroke}%
\pgfsetdash{}{0pt}%
\pgfsys@defobject{currentmarker}{\pgfqpoint{-0.041667in}{-0.041667in}}{\pgfqpoint{0.041667in}{0.041667in}}{%
\pgfpathmoveto{\pgfqpoint{0.000000in}{-0.041667in}}%
\pgfpathcurveto{\pgfqpoint{0.011050in}{-0.041667in}}{\pgfqpoint{0.021649in}{-0.037276in}}{\pgfqpoint{0.029463in}{-0.029463in}}%
\pgfpathcurveto{\pgfqpoint{0.037276in}{-0.021649in}}{\pgfqpoint{0.041667in}{-0.011050in}}{\pgfqpoint{0.041667in}{0.000000in}}%
\pgfpathcurveto{\pgfqpoint{0.041667in}{0.011050in}}{\pgfqpoint{0.037276in}{0.021649in}}{\pgfqpoint{0.029463in}{0.029463in}}%
\pgfpathcurveto{\pgfqpoint{0.021649in}{0.037276in}}{\pgfqpoint{0.011050in}{0.041667in}}{\pgfqpoint{0.000000in}{0.041667in}}%
\pgfpathcurveto{\pgfqpoint{-0.011050in}{0.041667in}}{\pgfqpoint{-0.021649in}{0.037276in}}{\pgfqpoint{-0.029463in}{0.029463in}}%
\pgfpathcurveto{\pgfqpoint{-0.037276in}{0.021649in}}{\pgfqpoint{-0.041667in}{0.011050in}}{\pgfqpoint{-0.041667in}{0.000000in}}%
\pgfpathcurveto{\pgfqpoint{-0.041667in}{-0.011050in}}{\pgfqpoint{-0.037276in}{-0.021649in}}{\pgfqpoint{-0.029463in}{-0.029463in}}%
\pgfpathcurveto{\pgfqpoint{-0.021649in}{-0.037276in}}{\pgfqpoint{-0.011050in}{-0.041667in}}{\pgfqpoint{0.000000in}{-0.041667in}}%
\pgfpathclose%
\pgfusepath{stroke,fill}%
}%
\begin{pgfscope}%
\pgfsys@transformshift{4.104750in}{3.843065in}%
\pgfsys@useobject{currentmarker}{}%
\end{pgfscope}%
\begin{pgfscope}%
\pgfsys@transformshift{4.301825in}{3.843065in}%
\pgfsys@useobject{currentmarker}{}%
\end{pgfscope}%
\begin{pgfscope}%
\pgfsys@transformshift{4.498901in}{3.843065in}%
\pgfsys@useobject{currentmarker}{}%
\end{pgfscope}%
\begin{pgfscope}%
\pgfsys@transformshift{4.695977in}{3.843065in}%
\pgfsys@useobject{currentmarker}{}%
\end{pgfscope}%
\begin{pgfscope}%
\pgfsys@transformshift{4.893052in}{3.843065in}%
\pgfsys@useobject{currentmarker}{}%
\end{pgfscope}%
\begin{pgfscope}%
\pgfsys@transformshift{5.090128in}{3.843065in}%
\pgfsys@useobject{currentmarker}{}%
\end{pgfscope}%
\begin{pgfscope}%
\pgfsys@transformshift{5.287204in}{3.843065in}%
\pgfsys@useobject{currentmarker}{}%
\end{pgfscope}%
\begin{pgfscope}%
\pgfsys@transformshift{5.484279in}{3.843065in}%
\pgfsys@useobject{currentmarker}{}%
\end{pgfscope}%
\begin{pgfscope}%
\pgfsys@transformshift{5.681355in}{3.843065in}%
\pgfsys@useobject{currentmarker}{}%
\end{pgfscope}%
\begin{pgfscope}%
\pgfsys@transformshift{5.878431in}{3.843065in}%
\pgfsys@useobject{currentmarker}{}%
\end{pgfscope}%
\begin{pgfscope}%
\pgfsys@transformshift{6.075506in}{4.571026in}%
\pgfsys@useobject{currentmarker}{}%
\end{pgfscope}%
\begin{pgfscope}%
\pgfsys@transformshift{6.272582in}{3.843065in}%
\pgfsys@useobject{currentmarker}{}%
\end{pgfscope}%
\begin{pgfscope}%
\pgfsys@transformshift{6.469658in}{3.843065in}%
\pgfsys@useobject{currentmarker}{}%
\end{pgfscope}%
\begin{pgfscope}%
\pgfsys@transformshift{6.666733in}{3.843065in}%
\pgfsys@useobject{currentmarker}{}%
\end{pgfscope}%
\end{pgfscope}%
\begin{pgfscope}%
\pgfsetrectcap%
\pgfsetmiterjoin%
\pgfsetlinewidth{0.803000pt}%
\definecolor{currentstroke}{rgb}{1.000000,1.000000,1.000000}%
\pgfsetstrokecolor{currentstroke}%
\pgfsetdash{}{0pt}%
\pgfpathmoveto{\pgfqpoint{3.976651in}{3.806667in}}%
\pgfpathlineto{\pgfqpoint{3.976651in}{4.607424in}}%
\pgfusepath{stroke}%
\end{pgfscope}%
\begin{pgfscope}%
\pgfsetrectcap%
\pgfsetmiterjoin%
\pgfsetlinewidth{0.803000pt}%
\definecolor{currentstroke}{rgb}{1.000000,1.000000,1.000000}%
\pgfsetstrokecolor{currentstroke}%
\pgfsetdash{}{0pt}%
\pgfpathmoveto{\pgfqpoint{6.794832in}{3.806667in}}%
\pgfpathlineto{\pgfqpoint{6.794832in}{4.607424in}}%
\pgfusepath{stroke}%
\end{pgfscope}%
\begin{pgfscope}%
\pgfsetrectcap%
\pgfsetmiterjoin%
\pgfsetlinewidth{0.803000pt}%
\definecolor{currentstroke}{rgb}{1.000000,1.000000,1.000000}%
\pgfsetstrokecolor{currentstroke}%
\pgfsetdash{}{0pt}%
\pgfpathmoveto{\pgfqpoint{3.976651in}{3.806667in}}%
\pgfpathlineto{\pgfqpoint{6.794832in}{3.806667in}}%
\pgfusepath{stroke}%
\end{pgfscope}%
\begin{pgfscope}%
\pgfsetrectcap%
\pgfsetmiterjoin%
\pgfsetlinewidth{0.803000pt}%
\definecolor{currentstroke}{rgb}{1.000000,1.000000,1.000000}%
\pgfsetstrokecolor{currentstroke}%
\pgfsetdash{}{0pt}%
\pgfpathmoveto{\pgfqpoint{3.976651in}{4.607424in}}%
\pgfpathlineto{\pgfqpoint{6.794832in}{4.607424in}}%
\pgfusepath{stroke}%
\end{pgfscope}%
\begin{pgfscope}%
\definecolor{textcolor}{rgb}{0.150000,0.150000,0.150000}%
\pgfsetstrokecolor{textcolor}%
\pgfsetfillcolor{textcolor}%
\pgftext[x=5.385741in,y=4.690758in,,base]{\color{textcolor}\rmfamily\fontsize{12.000000}{14.400000}\selectfont PG}%
\end{pgfscope}%
\begin{pgfscope}%
\pgfsetbuttcap%
\pgfsetmiterjoin%
\definecolor{currentfill}{rgb}{0.917647,0.917647,0.949020}%
\pgfsetfillcolor{currentfill}%
\pgfsetlinewidth{0.000000pt}%
\definecolor{currentstroke}{rgb}{0.000000,0.000000,0.000000}%
\pgfsetstrokecolor{currentstroke}%
\pgfsetstrokeopacity{0.000000}%
\pgfsetdash{}{0pt}%
\pgfpathmoveto{\pgfqpoint{0.594832in}{2.685606in}}%
\pgfpathlineto{\pgfqpoint{3.413014in}{2.685606in}}%
\pgfpathlineto{\pgfqpoint{3.413014in}{3.486364in}}%
\pgfpathlineto{\pgfqpoint{0.594832in}{3.486364in}}%
\pgfpathclose%
\pgfusepath{fill}%
\end{pgfscope}%
\begin{pgfscope}%
\pgfpathrectangle{\pgfqpoint{0.594832in}{2.685606in}}{\pgfqpoint{2.818182in}{0.800758in}}%
\pgfusepath{clip}%
\pgfsetroundcap%
\pgfsetroundjoin%
\pgfsetlinewidth{0.803000pt}%
\definecolor{currentstroke}{rgb}{1.000000,1.000000,1.000000}%
\pgfsetstrokecolor{currentstroke}%
\pgfsetdash{}{0pt}%
\pgfpathmoveto{\pgfqpoint{0.722932in}{2.685606in}}%
\pgfpathlineto{\pgfqpoint{0.722932in}{3.486364in}}%
\pgfusepath{stroke}%
\end{pgfscope}%
\begin{pgfscope}%
\pgfpathrectangle{\pgfqpoint{0.594832in}{2.685606in}}{\pgfqpoint{2.818182in}{0.800758in}}%
\pgfusepath{clip}%
\pgfsetroundcap%
\pgfsetroundjoin%
\pgfsetlinewidth{0.803000pt}%
\definecolor{currentstroke}{rgb}{1.000000,1.000000,1.000000}%
\pgfsetstrokecolor{currentstroke}%
\pgfsetdash{}{0pt}%
\pgfpathmoveto{\pgfqpoint{0.920007in}{2.685606in}}%
\pgfpathlineto{\pgfqpoint{0.920007in}{3.486364in}}%
\pgfusepath{stroke}%
\end{pgfscope}%
\begin{pgfscope}%
\pgfpathrectangle{\pgfqpoint{0.594832in}{2.685606in}}{\pgfqpoint{2.818182in}{0.800758in}}%
\pgfusepath{clip}%
\pgfsetroundcap%
\pgfsetroundjoin%
\pgfsetlinewidth{0.803000pt}%
\definecolor{currentstroke}{rgb}{1.000000,1.000000,1.000000}%
\pgfsetstrokecolor{currentstroke}%
\pgfsetdash{}{0pt}%
\pgfpathmoveto{\pgfqpoint{1.117083in}{2.685606in}}%
\pgfpathlineto{\pgfqpoint{1.117083in}{3.486364in}}%
\pgfusepath{stroke}%
\end{pgfscope}%
\begin{pgfscope}%
\pgfpathrectangle{\pgfqpoint{0.594832in}{2.685606in}}{\pgfqpoint{2.818182in}{0.800758in}}%
\pgfusepath{clip}%
\pgfsetroundcap%
\pgfsetroundjoin%
\pgfsetlinewidth{0.803000pt}%
\definecolor{currentstroke}{rgb}{1.000000,1.000000,1.000000}%
\pgfsetstrokecolor{currentstroke}%
\pgfsetdash{}{0pt}%
\pgfpathmoveto{\pgfqpoint{1.314158in}{2.685606in}}%
\pgfpathlineto{\pgfqpoint{1.314158in}{3.486364in}}%
\pgfusepath{stroke}%
\end{pgfscope}%
\begin{pgfscope}%
\pgfpathrectangle{\pgfqpoint{0.594832in}{2.685606in}}{\pgfqpoint{2.818182in}{0.800758in}}%
\pgfusepath{clip}%
\pgfsetroundcap%
\pgfsetroundjoin%
\pgfsetlinewidth{0.803000pt}%
\definecolor{currentstroke}{rgb}{1.000000,1.000000,1.000000}%
\pgfsetstrokecolor{currentstroke}%
\pgfsetdash{}{0pt}%
\pgfpathmoveto{\pgfqpoint{1.511234in}{2.685606in}}%
\pgfpathlineto{\pgfqpoint{1.511234in}{3.486364in}}%
\pgfusepath{stroke}%
\end{pgfscope}%
\begin{pgfscope}%
\pgfpathrectangle{\pgfqpoint{0.594832in}{2.685606in}}{\pgfqpoint{2.818182in}{0.800758in}}%
\pgfusepath{clip}%
\pgfsetroundcap%
\pgfsetroundjoin%
\pgfsetlinewidth{0.803000pt}%
\definecolor{currentstroke}{rgb}{1.000000,1.000000,1.000000}%
\pgfsetstrokecolor{currentstroke}%
\pgfsetdash{}{0pt}%
\pgfpathmoveto{\pgfqpoint{1.708310in}{2.685606in}}%
\pgfpathlineto{\pgfqpoint{1.708310in}{3.486364in}}%
\pgfusepath{stroke}%
\end{pgfscope}%
\begin{pgfscope}%
\pgfpathrectangle{\pgfqpoint{0.594832in}{2.685606in}}{\pgfqpoint{2.818182in}{0.800758in}}%
\pgfusepath{clip}%
\pgfsetroundcap%
\pgfsetroundjoin%
\pgfsetlinewidth{0.803000pt}%
\definecolor{currentstroke}{rgb}{1.000000,1.000000,1.000000}%
\pgfsetstrokecolor{currentstroke}%
\pgfsetdash{}{0pt}%
\pgfpathmoveto{\pgfqpoint{1.905385in}{2.685606in}}%
\pgfpathlineto{\pgfqpoint{1.905385in}{3.486364in}}%
\pgfusepath{stroke}%
\end{pgfscope}%
\begin{pgfscope}%
\pgfpathrectangle{\pgfqpoint{0.594832in}{2.685606in}}{\pgfqpoint{2.818182in}{0.800758in}}%
\pgfusepath{clip}%
\pgfsetroundcap%
\pgfsetroundjoin%
\pgfsetlinewidth{0.803000pt}%
\definecolor{currentstroke}{rgb}{1.000000,1.000000,1.000000}%
\pgfsetstrokecolor{currentstroke}%
\pgfsetdash{}{0pt}%
\pgfpathmoveto{\pgfqpoint{2.102461in}{2.685606in}}%
\pgfpathlineto{\pgfqpoint{2.102461in}{3.486364in}}%
\pgfusepath{stroke}%
\end{pgfscope}%
\begin{pgfscope}%
\pgfpathrectangle{\pgfqpoint{0.594832in}{2.685606in}}{\pgfqpoint{2.818182in}{0.800758in}}%
\pgfusepath{clip}%
\pgfsetroundcap%
\pgfsetroundjoin%
\pgfsetlinewidth{0.803000pt}%
\definecolor{currentstroke}{rgb}{1.000000,1.000000,1.000000}%
\pgfsetstrokecolor{currentstroke}%
\pgfsetdash{}{0pt}%
\pgfpathmoveto{\pgfqpoint{2.299537in}{2.685606in}}%
\pgfpathlineto{\pgfqpoint{2.299537in}{3.486364in}}%
\pgfusepath{stroke}%
\end{pgfscope}%
\begin{pgfscope}%
\pgfpathrectangle{\pgfqpoint{0.594832in}{2.685606in}}{\pgfqpoint{2.818182in}{0.800758in}}%
\pgfusepath{clip}%
\pgfsetroundcap%
\pgfsetroundjoin%
\pgfsetlinewidth{0.803000pt}%
\definecolor{currentstroke}{rgb}{1.000000,1.000000,1.000000}%
\pgfsetstrokecolor{currentstroke}%
\pgfsetdash{}{0pt}%
\pgfpathmoveto{\pgfqpoint{2.496612in}{2.685606in}}%
\pgfpathlineto{\pgfqpoint{2.496612in}{3.486364in}}%
\pgfusepath{stroke}%
\end{pgfscope}%
\begin{pgfscope}%
\pgfpathrectangle{\pgfqpoint{0.594832in}{2.685606in}}{\pgfqpoint{2.818182in}{0.800758in}}%
\pgfusepath{clip}%
\pgfsetroundcap%
\pgfsetroundjoin%
\pgfsetlinewidth{0.803000pt}%
\definecolor{currentstroke}{rgb}{1.000000,1.000000,1.000000}%
\pgfsetstrokecolor{currentstroke}%
\pgfsetdash{}{0pt}%
\pgfpathmoveto{\pgfqpoint{2.693688in}{2.685606in}}%
\pgfpathlineto{\pgfqpoint{2.693688in}{3.486364in}}%
\pgfusepath{stroke}%
\end{pgfscope}%
\begin{pgfscope}%
\pgfpathrectangle{\pgfqpoint{0.594832in}{2.685606in}}{\pgfqpoint{2.818182in}{0.800758in}}%
\pgfusepath{clip}%
\pgfsetroundcap%
\pgfsetroundjoin%
\pgfsetlinewidth{0.803000pt}%
\definecolor{currentstroke}{rgb}{1.000000,1.000000,1.000000}%
\pgfsetstrokecolor{currentstroke}%
\pgfsetdash{}{0pt}%
\pgfpathmoveto{\pgfqpoint{2.890764in}{2.685606in}}%
\pgfpathlineto{\pgfqpoint{2.890764in}{3.486364in}}%
\pgfusepath{stroke}%
\end{pgfscope}%
\begin{pgfscope}%
\pgfpathrectangle{\pgfqpoint{0.594832in}{2.685606in}}{\pgfqpoint{2.818182in}{0.800758in}}%
\pgfusepath{clip}%
\pgfsetroundcap%
\pgfsetroundjoin%
\pgfsetlinewidth{0.803000pt}%
\definecolor{currentstroke}{rgb}{1.000000,1.000000,1.000000}%
\pgfsetstrokecolor{currentstroke}%
\pgfsetdash{}{0pt}%
\pgfpathmoveto{\pgfqpoint{3.087839in}{2.685606in}}%
\pgfpathlineto{\pgfqpoint{3.087839in}{3.486364in}}%
\pgfusepath{stroke}%
\end{pgfscope}%
\begin{pgfscope}%
\pgfpathrectangle{\pgfqpoint{0.594832in}{2.685606in}}{\pgfqpoint{2.818182in}{0.800758in}}%
\pgfusepath{clip}%
\pgfsetroundcap%
\pgfsetroundjoin%
\pgfsetlinewidth{0.803000pt}%
\definecolor{currentstroke}{rgb}{1.000000,1.000000,1.000000}%
\pgfsetstrokecolor{currentstroke}%
\pgfsetdash{}{0pt}%
\pgfpathmoveto{\pgfqpoint{3.284915in}{2.685606in}}%
\pgfpathlineto{\pgfqpoint{3.284915in}{3.486364in}}%
\pgfusepath{stroke}%
\end{pgfscope}%
\begin{pgfscope}%
\pgfpathrectangle{\pgfqpoint{0.594832in}{2.685606in}}{\pgfqpoint{2.818182in}{0.800758in}}%
\pgfusepath{clip}%
\pgfsetroundcap%
\pgfsetroundjoin%
\pgfsetlinewidth{0.803000pt}%
\definecolor{currentstroke}{rgb}{1.000000,1.000000,1.000000}%
\pgfsetstrokecolor{currentstroke}%
\pgfsetdash{}{0pt}%
\pgfpathmoveto{\pgfqpoint{0.594832in}{2.803684in}}%
\pgfpathlineto{\pgfqpoint{3.413014in}{2.803684in}}%
\pgfusepath{stroke}%
\end{pgfscope}%
\begin{pgfscope}%
\definecolor{textcolor}{rgb}{0.150000,0.150000,0.150000}%
\pgfsetstrokecolor{textcolor}%
\pgfsetfillcolor{textcolor}%
\pgftext[x=0.100000in,y=2.750922in,left,base]{\color{textcolor}\rmfamily\fontsize{10.000000}{12.000000}\selectfont 0.175}%
\end{pgfscope}%
\begin{pgfscope}%
\pgfpathrectangle{\pgfqpoint{0.594832in}{2.685606in}}{\pgfqpoint{2.818182in}{0.800758in}}%
\pgfusepath{clip}%
\pgfsetroundcap%
\pgfsetroundjoin%
\pgfsetlinewidth{0.803000pt}%
\definecolor{currentstroke}{rgb}{1.000000,1.000000,1.000000}%
\pgfsetstrokecolor{currentstroke}%
\pgfsetdash{}{0pt}%
\pgfpathmoveto{\pgfqpoint{0.594832in}{3.225311in}}%
\pgfpathlineto{\pgfqpoint{3.413014in}{3.225311in}}%
\pgfusepath{stroke}%
\end{pgfscope}%
\begin{pgfscope}%
\definecolor{textcolor}{rgb}{0.150000,0.150000,0.150000}%
\pgfsetstrokecolor{textcolor}%
\pgfsetfillcolor{textcolor}%
\pgftext[x=0.100000in,y=3.172549in,left,base]{\color{textcolor}\rmfamily\fontsize{10.000000}{12.000000}\selectfont 0.200}%
\end{pgfscope}%
\begin{pgfscope}%
\pgfpathrectangle{\pgfqpoint{0.594832in}{2.685606in}}{\pgfqpoint{2.818182in}{0.800758in}}%
\pgfusepath{clip}%
\pgfsetroundcap%
\pgfsetroundjoin%
\pgfsetlinewidth{1.505625pt}%
\definecolor{currentstroke}{rgb}{0.890196,0.466667,0.760784}%
\pgfsetstrokecolor{currentstroke}%
\pgfsetdash{}{0pt}%
\pgfpathmoveto{\pgfqpoint{0.722932in}{2.732051in}}%
\pgfpathlineto{\pgfqpoint{0.920007in}{2.749585in}}%
\pgfpathlineto{\pgfqpoint{1.117083in}{2.765521in}}%
\pgfpathlineto{\pgfqpoint{1.314158in}{2.723213in}}%
\pgfpathlineto{\pgfqpoint{1.511234in}{2.744768in}}%
\pgfpathlineto{\pgfqpoint{1.708310in}{2.722004in}}%
\pgfpathlineto{\pgfqpoint{1.905385in}{2.725262in}}%
\pgfpathlineto{\pgfqpoint{2.102461in}{2.723213in}}%
\pgfpathlineto{\pgfqpoint{2.299537in}{2.768667in}}%
\pgfpathlineto{\pgfqpoint{2.496612in}{3.449966in}}%
\pgfpathlineto{\pgfqpoint{2.693688in}{2.757418in}}%
\pgfpathlineto{\pgfqpoint{2.890764in}{2.723213in}}%
\pgfpathlineto{\pgfqpoint{3.087839in}{2.723213in}}%
\pgfpathlineto{\pgfqpoint{3.284915in}{2.733536in}}%
\pgfusepath{stroke}%
\end{pgfscope}%
\begin{pgfscope}%
\pgfpathrectangle{\pgfqpoint{0.594832in}{2.685606in}}{\pgfqpoint{2.818182in}{0.800758in}}%
\pgfusepath{clip}%
\pgfsetbuttcap%
\pgfsetroundjoin%
\definecolor{currentfill}{rgb}{0.890196,0.466667,0.760784}%
\pgfsetfillcolor{currentfill}%
\pgfsetlinewidth{1.003750pt}%
\definecolor{currentstroke}{rgb}{0.890196,0.466667,0.760784}%
\pgfsetstrokecolor{currentstroke}%
\pgfsetdash{}{0pt}%
\pgfsys@defobject{currentmarker}{\pgfqpoint{-0.041667in}{-0.041667in}}{\pgfqpoint{0.041667in}{0.041667in}}{%
\pgfpathmoveto{\pgfqpoint{0.000000in}{-0.041667in}}%
\pgfpathcurveto{\pgfqpoint{0.011050in}{-0.041667in}}{\pgfqpoint{0.021649in}{-0.037276in}}{\pgfqpoint{0.029463in}{-0.029463in}}%
\pgfpathcurveto{\pgfqpoint{0.037276in}{-0.021649in}}{\pgfqpoint{0.041667in}{-0.011050in}}{\pgfqpoint{0.041667in}{0.000000in}}%
\pgfpathcurveto{\pgfqpoint{0.041667in}{0.011050in}}{\pgfqpoint{0.037276in}{0.021649in}}{\pgfqpoint{0.029463in}{0.029463in}}%
\pgfpathcurveto{\pgfqpoint{0.021649in}{0.037276in}}{\pgfqpoint{0.011050in}{0.041667in}}{\pgfqpoint{0.000000in}{0.041667in}}%
\pgfpathcurveto{\pgfqpoint{-0.011050in}{0.041667in}}{\pgfqpoint{-0.021649in}{0.037276in}}{\pgfqpoint{-0.029463in}{0.029463in}}%
\pgfpathcurveto{\pgfqpoint{-0.037276in}{0.021649in}}{\pgfqpoint{-0.041667in}{0.011050in}}{\pgfqpoint{-0.041667in}{0.000000in}}%
\pgfpathcurveto{\pgfqpoint{-0.041667in}{-0.011050in}}{\pgfqpoint{-0.037276in}{-0.021649in}}{\pgfqpoint{-0.029463in}{-0.029463in}}%
\pgfpathcurveto{\pgfqpoint{-0.021649in}{-0.037276in}}{\pgfqpoint{-0.011050in}{-0.041667in}}{\pgfqpoint{0.000000in}{-0.041667in}}%
\pgfpathclose%
\pgfusepath{stroke,fill}%
}%
\begin{pgfscope}%
\pgfsys@transformshift{0.722932in}{2.732051in}%
\pgfsys@useobject{currentmarker}{}%
\end{pgfscope}%
\begin{pgfscope}%
\pgfsys@transformshift{0.920007in}{2.749585in}%
\pgfsys@useobject{currentmarker}{}%
\end{pgfscope}%
\begin{pgfscope}%
\pgfsys@transformshift{1.117083in}{2.765521in}%
\pgfsys@useobject{currentmarker}{}%
\end{pgfscope}%
\begin{pgfscope}%
\pgfsys@transformshift{1.314158in}{2.723213in}%
\pgfsys@useobject{currentmarker}{}%
\end{pgfscope}%
\begin{pgfscope}%
\pgfsys@transformshift{1.511234in}{2.744768in}%
\pgfsys@useobject{currentmarker}{}%
\end{pgfscope}%
\begin{pgfscope}%
\pgfsys@transformshift{1.708310in}{2.722004in}%
\pgfsys@useobject{currentmarker}{}%
\end{pgfscope}%
\begin{pgfscope}%
\pgfsys@transformshift{1.905385in}{2.725262in}%
\pgfsys@useobject{currentmarker}{}%
\end{pgfscope}%
\begin{pgfscope}%
\pgfsys@transformshift{2.102461in}{2.723213in}%
\pgfsys@useobject{currentmarker}{}%
\end{pgfscope}%
\begin{pgfscope}%
\pgfsys@transformshift{2.299537in}{2.768667in}%
\pgfsys@useobject{currentmarker}{}%
\end{pgfscope}%
\begin{pgfscope}%
\pgfsys@transformshift{2.496612in}{3.449966in}%
\pgfsys@useobject{currentmarker}{}%
\end{pgfscope}%
\begin{pgfscope}%
\pgfsys@transformshift{2.693688in}{2.757418in}%
\pgfsys@useobject{currentmarker}{}%
\end{pgfscope}%
\begin{pgfscope}%
\pgfsys@transformshift{2.890764in}{2.723213in}%
\pgfsys@useobject{currentmarker}{}%
\end{pgfscope}%
\begin{pgfscope}%
\pgfsys@transformshift{3.087839in}{2.723213in}%
\pgfsys@useobject{currentmarker}{}%
\end{pgfscope}%
\begin{pgfscope}%
\pgfsys@transformshift{3.284915in}{2.733536in}%
\pgfsys@useobject{currentmarker}{}%
\end{pgfscope}%
\end{pgfscope}%
\begin{pgfscope}%
\pgfsetrectcap%
\pgfsetmiterjoin%
\pgfsetlinewidth{0.803000pt}%
\definecolor{currentstroke}{rgb}{1.000000,1.000000,1.000000}%
\pgfsetstrokecolor{currentstroke}%
\pgfsetdash{}{0pt}%
\pgfpathmoveto{\pgfqpoint{0.594832in}{2.685606in}}%
\pgfpathlineto{\pgfqpoint{0.594832in}{3.486364in}}%
\pgfusepath{stroke}%
\end{pgfscope}%
\begin{pgfscope}%
\pgfsetrectcap%
\pgfsetmiterjoin%
\pgfsetlinewidth{0.803000pt}%
\definecolor{currentstroke}{rgb}{1.000000,1.000000,1.000000}%
\pgfsetstrokecolor{currentstroke}%
\pgfsetdash{}{0pt}%
\pgfpathmoveto{\pgfqpoint{3.413014in}{2.685606in}}%
\pgfpathlineto{\pgfqpoint{3.413014in}{3.486364in}}%
\pgfusepath{stroke}%
\end{pgfscope}%
\begin{pgfscope}%
\pgfsetrectcap%
\pgfsetmiterjoin%
\pgfsetlinewidth{0.803000pt}%
\definecolor{currentstroke}{rgb}{1.000000,1.000000,1.000000}%
\pgfsetstrokecolor{currentstroke}%
\pgfsetdash{}{0pt}%
\pgfpathmoveto{\pgfqpoint{0.594832in}{2.685606in}}%
\pgfpathlineto{\pgfqpoint{3.413014in}{2.685606in}}%
\pgfusepath{stroke}%
\end{pgfscope}%
\begin{pgfscope}%
\pgfsetrectcap%
\pgfsetmiterjoin%
\pgfsetlinewidth{0.803000pt}%
\definecolor{currentstroke}{rgb}{1.000000,1.000000,1.000000}%
\pgfsetstrokecolor{currentstroke}%
\pgfsetdash{}{0pt}%
\pgfpathmoveto{\pgfqpoint{0.594832in}{3.486364in}}%
\pgfpathlineto{\pgfqpoint{3.413014in}{3.486364in}}%
\pgfusepath{stroke}%
\end{pgfscope}%
\begin{pgfscope}%
\definecolor{textcolor}{rgb}{0.150000,0.150000,0.150000}%
\pgfsetstrokecolor{textcolor}%
\pgfsetfillcolor{textcolor}%
\pgftext[x=2.003923in,y=3.569697in,,base]{\color{textcolor}\rmfamily\fontsize{12.000000}{14.400000}\selectfont UTX}%
\end{pgfscope}%
\begin{pgfscope}%
\pgfsetbuttcap%
\pgfsetmiterjoin%
\definecolor{currentfill}{rgb}{0.917647,0.917647,0.949020}%
\pgfsetfillcolor{currentfill}%
\pgfsetlinewidth{0.000000pt}%
\definecolor{currentstroke}{rgb}{0.000000,0.000000,0.000000}%
\pgfsetstrokecolor{currentstroke}%
\pgfsetstrokeopacity{0.000000}%
\pgfsetdash{}{0pt}%
\pgfpathmoveto{\pgfqpoint{3.976651in}{2.685606in}}%
\pgfpathlineto{\pgfqpoint{6.794832in}{2.685606in}}%
\pgfpathlineto{\pgfqpoint{6.794832in}{3.486364in}}%
\pgfpathlineto{\pgfqpoint{3.976651in}{3.486364in}}%
\pgfpathclose%
\pgfusepath{fill}%
\end{pgfscope}%
\begin{pgfscope}%
\pgfpathrectangle{\pgfqpoint{3.976651in}{2.685606in}}{\pgfqpoint{2.818182in}{0.800758in}}%
\pgfusepath{clip}%
\pgfsetroundcap%
\pgfsetroundjoin%
\pgfsetlinewidth{0.803000pt}%
\definecolor{currentstroke}{rgb}{1.000000,1.000000,1.000000}%
\pgfsetstrokecolor{currentstroke}%
\pgfsetdash{}{0pt}%
\pgfpathmoveto{\pgfqpoint{4.104750in}{2.685606in}}%
\pgfpathlineto{\pgfqpoint{4.104750in}{3.486364in}}%
\pgfusepath{stroke}%
\end{pgfscope}%
\begin{pgfscope}%
\pgfpathrectangle{\pgfqpoint{3.976651in}{2.685606in}}{\pgfqpoint{2.818182in}{0.800758in}}%
\pgfusepath{clip}%
\pgfsetroundcap%
\pgfsetroundjoin%
\pgfsetlinewidth{0.803000pt}%
\definecolor{currentstroke}{rgb}{1.000000,1.000000,1.000000}%
\pgfsetstrokecolor{currentstroke}%
\pgfsetdash{}{0pt}%
\pgfpathmoveto{\pgfqpoint{4.301825in}{2.685606in}}%
\pgfpathlineto{\pgfqpoint{4.301825in}{3.486364in}}%
\pgfusepath{stroke}%
\end{pgfscope}%
\begin{pgfscope}%
\pgfpathrectangle{\pgfqpoint{3.976651in}{2.685606in}}{\pgfqpoint{2.818182in}{0.800758in}}%
\pgfusepath{clip}%
\pgfsetroundcap%
\pgfsetroundjoin%
\pgfsetlinewidth{0.803000pt}%
\definecolor{currentstroke}{rgb}{1.000000,1.000000,1.000000}%
\pgfsetstrokecolor{currentstroke}%
\pgfsetdash{}{0pt}%
\pgfpathmoveto{\pgfqpoint{4.498901in}{2.685606in}}%
\pgfpathlineto{\pgfqpoint{4.498901in}{3.486364in}}%
\pgfusepath{stroke}%
\end{pgfscope}%
\begin{pgfscope}%
\pgfpathrectangle{\pgfqpoint{3.976651in}{2.685606in}}{\pgfqpoint{2.818182in}{0.800758in}}%
\pgfusepath{clip}%
\pgfsetroundcap%
\pgfsetroundjoin%
\pgfsetlinewidth{0.803000pt}%
\definecolor{currentstroke}{rgb}{1.000000,1.000000,1.000000}%
\pgfsetstrokecolor{currentstroke}%
\pgfsetdash{}{0pt}%
\pgfpathmoveto{\pgfqpoint{4.695977in}{2.685606in}}%
\pgfpathlineto{\pgfqpoint{4.695977in}{3.486364in}}%
\pgfusepath{stroke}%
\end{pgfscope}%
\begin{pgfscope}%
\pgfpathrectangle{\pgfqpoint{3.976651in}{2.685606in}}{\pgfqpoint{2.818182in}{0.800758in}}%
\pgfusepath{clip}%
\pgfsetroundcap%
\pgfsetroundjoin%
\pgfsetlinewidth{0.803000pt}%
\definecolor{currentstroke}{rgb}{1.000000,1.000000,1.000000}%
\pgfsetstrokecolor{currentstroke}%
\pgfsetdash{}{0pt}%
\pgfpathmoveto{\pgfqpoint{4.893052in}{2.685606in}}%
\pgfpathlineto{\pgfqpoint{4.893052in}{3.486364in}}%
\pgfusepath{stroke}%
\end{pgfscope}%
\begin{pgfscope}%
\pgfpathrectangle{\pgfqpoint{3.976651in}{2.685606in}}{\pgfqpoint{2.818182in}{0.800758in}}%
\pgfusepath{clip}%
\pgfsetroundcap%
\pgfsetroundjoin%
\pgfsetlinewidth{0.803000pt}%
\definecolor{currentstroke}{rgb}{1.000000,1.000000,1.000000}%
\pgfsetstrokecolor{currentstroke}%
\pgfsetdash{}{0pt}%
\pgfpathmoveto{\pgfqpoint{5.090128in}{2.685606in}}%
\pgfpathlineto{\pgfqpoint{5.090128in}{3.486364in}}%
\pgfusepath{stroke}%
\end{pgfscope}%
\begin{pgfscope}%
\pgfpathrectangle{\pgfqpoint{3.976651in}{2.685606in}}{\pgfqpoint{2.818182in}{0.800758in}}%
\pgfusepath{clip}%
\pgfsetroundcap%
\pgfsetroundjoin%
\pgfsetlinewidth{0.803000pt}%
\definecolor{currentstroke}{rgb}{1.000000,1.000000,1.000000}%
\pgfsetstrokecolor{currentstroke}%
\pgfsetdash{}{0pt}%
\pgfpathmoveto{\pgfqpoint{5.287204in}{2.685606in}}%
\pgfpathlineto{\pgfqpoint{5.287204in}{3.486364in}}%
\pgfusepath{stroke}%
\end{pgfscope}%
\begin{pgfscope}%
\pgfpathrectangle{\pgfqpoint{3.976651in}{2.685606in}}{\pgfqpoint{2.818182in}{0.800758in}}%
\pgfusepath{clip}%
\pgfsetroundcap%
\pgfsetroundjoin%
\pgfsetlinewidth{0.803000pt}%
\definecolor{currentstroke}{rgb}{1.000000,1.000000,1.000000}%
\pgfsetstrokecolor{currentstroke}%
\pgfsetdash{}{0pt}%
\pgfpathmoveto{\pgfqpoint{5.484279in}{2.685606in}}%
\pgfpathlineto{\pgfqpoint{5.484279in}{3.486364in}}%
\pgfusepath{stroke}%
\end{pgfscope}%
\begin{pgfscope}%
\pgfpathrectangle{\pgfqpoint{3.976651in}{2.685606in}}{\pgfqpoint{2.818182in}{0.800758in}}%
\pgfusepath{clip}%
\pgfsetroundcap%
\pgfsetroundjoin%
\pgfsetlinewidth{0.803000pt}%
\definecolor{currentstroke}{rgb}{1.000000,1.000000,1.000000}%
\pgfsetstrokecolor{currentstroke}%
\pgfsetdash{}{0pt}%
\pgfpathmoveto{\pgfqpoint{5.681355in}{2.685606in}}%
\pgfpathlineto{\pgfqpoint{5.681355in}{3.486364in}}%
\pgfusepath{stroke}%
\end{pgfscope}%
\begin{pgfscope}%
\pgfpathrectangle{\pgfqpoint{3.976651in}{2.685606in}}{\pgfqpoint{2.818182in}{0.800758in}}%
\pgfusepath{clip}%
\pgfsetroundcap%
\pgfsetroundjoin%
\pgfsetlinewidth{0.803000pt}%
\definecolor{currentstroke}{rgb}{1.000000,1.000000,1.000000}%
\pgfsetstrokecolor{currentstroke}%
\pgfsetdash{}{0pt}%
\pgfpathmoveto{\pgfqpoint{5.878431in}{2.685606in}}%
\pgfpathlineto{\pgfqpoint{5.878431in}{3.486364in}}%
\pgfusepath{stroke}%
\end{pgfscope}%
\begin{pgfscope}%
\pgfpathrectangle{\pgfqpoint{3.976651in}{2.685606in}}{\pgfqpoint{2.818182in}{0.800758in}}%
\pgfusepath{clip}%
\pgfsetroundcap%
\pgfsetroundjoin%
\pgfsetlinewidth{0.803000pt}%
\definecolor{currentstroke}{rgb}{1.000000,1.000000,1.000000}%
\pgfsetstrokecolor{currentstroke}%
\pgfsetdash{}{0pt}%
\pgfpathmoveto{\pgfqpoint{6.075506in}{2.685606in}}%
\pgfpathlineto{\pgfqpoint{6.075506in}{3.486364in}}%
\pgfusepath{stroke}%
\end{pgfscope}%
\begin{pgfscope}%
\pgfpathrectangle{\pgfqpoint{3.976651in}{2.685606in}}{\pgfqpoint{2.818182in}{0.800758in}}%
\pgfusepath{clip}%
\pgfsetroundcap%
\pgfsetroundjoin%
\pgfsetlinewidth{0.803000pt}%
\definecolor{currentstroke}{rgb}{1.000000,1.000000,1.000000}%
\pgfsetstrokecolor{currentstroke}%
\pgfsetdash{}{0pt}%
\pgfpathmoveto{\pgfqpoint{6.272582in}{2.685606in}}%
\pgfpathlineto{\pgfqpoint{6.272582in}{3.486364in}}%
\pgfusepath{stroke}%
\end{pgfscope}%
\begin{pgfscope}%
\pgfpathrectangle{\pgfqpoint{3.976651in}{2.685606in}}{\pgfqpoint{2.818182in}{0.800758in}}%
\pgfusepath{clip}%
\pgfsetroundcap%
\pgfsetroundjoin%
\pgfsetlinewidth{0.803000pt}%
\definecolor{currentstroke}{rgb}{1.000000,1.000000,1.000000}%
\pgfsetstrokecolor{currentstroke}%
\pgfsetdash{}{0pt}%
\pgfpathmoveto{\pgfqpoint{6.469658in}{2.685606in}}%
\pgfpathlineto{\pgfqpoint{6.469658in}{3.486364in}}%
\pgfusepath{stroke}%
\end{pgfscope}%
\begin{pgfscope}%
\pgfpathrectangle{\pgfqpoint{3.976651in}{2.685606in}}{\pgfqpoint{2.818182in}{0.800758in}}%
\pgfusepath{clip}%
\pgfsetroundcap%
\pgfsetroundjoin%
\pgfsetlinewidth{0.803000pt}%
\definecolor{currentstroke}{rgb}{1.000000,1.000000,1.000000}%
\pgfsetstrokecolor{currentstroke}%
\pgfsetdash{}{0pt}%
\pgfpathmoveto{\pgfqpoint{6.666733in}{2.685606in}}%
\pgfpathlineto{\pgfqpoint{6.666733in}{3.486364in}}%
\pgfusepath{stroke}%
\end{pgfscope}%
\begin{pgfscope}%
\pgfpathrectangle{\pgfqpoint{3.976651in}{2.685606in}}{\pgfqpoint{2.818182in}{0.800758in}}%
\pgfusepath{clip}%
\pgfsetroundcap%
\pgfsetroundjoin%
\pgfsetlinewidth{0.803000pt}%
\definecolor{currentstroke}{rgb}{1.000000,1.000000,1.000000}%
\pgfsetstrokecolor{currentstroke}%
\pgfsetdash{}{0pt}%
\pgfpathmoveto{\pgfqpoint{3.976651in}{2.737260in}}%
\pgfpathlineto{\pgfqpoint{6.794832in}{2.737260in}}%
\pgfusepath{stroke}%
\end{pgfscope}%
\begin{pgfscope}%
\definecolor{textcolor}{rgb}{0.150000,0.150000,0.150000}%
\pgfsetstrokecolor{textcolor}%
\pgfsetfillcolor{textcolor}%
\pgftext[x=3.481818in,y=2.684499in,left,base]{\color{textcolor}\rmfamily\fontsize{10.000000}{12.000000}\selectfont 0.150}%
\end{pgfscope}%
\begin{pgfscope}%
\pgfpathrectangle{\pgfqpoint{3.976651in}{2.685606in}}{\pgfqpoint{2.818182in}{0.800758in}}%
\pgfusepath{clip}%
\pgfsetroundcap%
\pgfsetroundjoin%
\pgfsetlinewidth{0.803000pt}%
\definecolor{currentstroke}{rgb}{1.000000,1.000000,1.000000}%
\pgfsetstrokecolor{currentstroke}%
\pgfsetdash{}{0pt}%
\pgfpathmoveto{\pgfqpoint{3.976651in}{3.413195in}}%
\pgfpathlineto{\pgfqpoint{6.794832in}{3.413195in}}%
\pgfusepath{stroke}%
\end{pgfscope}%
\begin{pgfscope}%
\definecolor{textcolor}{rgb}{0.150000,0.150000,0.150000}%
\pgfsetstrokecolor{textcolor}%
\pgfsetfillcolor{textcolor}%
\pgftext[x=3.481818in,y=3.360434in,left,base]{\color{textcolor}\rmfamily\fontsize{10.000000}{12.000000}\selectfont 0.152}%
\end{pgfscope}%
\begin{pgfscope}%
\pgfpathrectangle{\pgfqpoint{3.976651in}{2.685606in}}{\pgfqpoint{2.818182in}{0.800758in}}%
\pgfusepath{clip}%
\pgfsetroundcap%
\pgfsetroundjoin%
\pgfsetlinewidth{1.505625pt}%
\definecolor{currentstroke}{rgb}{0.498039,0.498039,0.498039}%
\pgfsetstrokecolor{currentstroke}%
\pgfsetdash{}{0pt}%
\pgfpathmoveto{\pgfqpoint{4.104750in}{2.844637in}}%
\pgfpathlineto{\pgfqpoint{4.301825in}{3.075241in}}%
\pgfpathlineto{\pgfqpoint{4.498901in}{3.230401in}}%
\pgfpathlineto{\pgfqpoint{4.695977in}{2.722004in}}%
\pgfpathlineto{\pgfqpoint{4.893052in}{3.008368in}}%
\pgfpathlineto{\pgfqpoint{5.090128in}{3.207894in}}%
\pgfpathlineto{\pgfqpoint{5.287204in}{2.958443in}}%
\pgfpathlineto{\pgfqpoint{5.484279in}{2.722004in}}%
\pgfpathlineto{\pgfqpoint{5.681355in}{3.190798in}}%
\pgfpathlineto{\pgfqpoint{5.878431in}{3.393405in}}%
\pgfpathlineto{\pgfqpoint{6.075506in}{3.449966in}}%
\pgfpathlineto{\pgfqpoint{6.272582in}{2.722004in}}%
\pgfpathlineto{\pgfqpoint{6.469658in}{2.722004in}}%
\pgfpathlineto{\pgfqpoint{6.666733in}{3.026732in}}%
\pgfusepath{stroke}%
\end{pgfscope}%
\begin{pgfscope}%
\pgfpathrectangle{\pgfqpoint{3.976651in}{2.685606in}}{\pgfqpoint{2.818182in}{0.800758in}}%
\pgfusepath{clip}%
\pgfsetbuttcap%
\pgfsetroundjoin%
\definecolor{currentfill}{rgb}{0.498039,0.498039,0.498039}%
\pgfsetfillcolor{currentfill}%
\pgfsetlinewidth{1.003750pt}%
\definecolor{currentstroke}{rgb}{0.498039,0.498039,0.498039}%
\pgfsetstrokecolor{currentstroke}%
\pgfsetdash{}{0pt}%
\pgfsys@defobject{currentmarker}{\pgfqpoint{-0.041667in}{-0.041667in}}{\pgfqpoint{0.041667in}{0.041667in}}{%
\pgfpathmoveto{\pgfqpoint{0.000000in}{-0.041667in}}%
\pgfpathcurveto{\pgfqpoint{0.011050in}{-0.041667in}}{\pgfqpoint{0.021649in}{-0.037276in}}{\pgfqpoint{0.029463in}{-0.029463in}}%
\pgfpathcurveto{\pgfqpoint{0.037276in}{-0.021649in}}{\pgfqpoint{0.041667in}{-0.011050in}}{\pgfqpoint{0.041667in}{0.000000in}}%
\pgfpathcurveto{\pgfqpoint{0.041667in}{0.011050in}}{\pgfqpoint{0.037276in}{0.021649in}}{\pgfqpoint{0.029463in}{0.029463in}}%
\pgfpathcurveto{\pgfqpoint{0.021649in}{0.037276in}}{\pgfqpoint{0.011050in}{0.041667in}}{\pgfqpoint{0.000000in}{0.041667in}}%
\pgfpathcurveto{\pgfqpoint{-0.011050in}{0.041667in}}{\pgfqpoint{-0.021649in}{0.037276in}}{\pgfqpoint{-0.029463in}{0.029463in}}%
\pgfpathcurveto{\pgfqpoint{-0.037276in}{0.021649in}}{\pgfqpoint{-0.041667in}{0.011050in}}{\pgfqpoint{-0.041667in}{0.000000in}}%
\pgfpathcurveto{\pgfqpoint{-0.041667in}{-0.011050in}}{\pgfqpoint{-0.037276in}{-0.021649in}}{\pgfqpoint{-0.029463in}{-0.029463in}}%
\pgfpathcurveto{\pgfqpoint{-0.021649in}{-0.037276in}}{\pgfqpoint{-0.011050in}{-0.041667in}}{\pgfqpoint{0.000000in}{-0.041667in}}%
\pgfpathclose%
\pgfusepath{stroke,fill}%
}%
\begin{pgfscope}%
\pgfsys@transformshift{4.104750in}{2.844637in}%
\pgfsys@useobject{currentmarker}{}%
\end{pgfscope}%
\begin{pgfscope}%
\pgfsys@transformshift{4.301825in}{3.075241in}%
\pgfsys@useobject{currentmarker}{}%
\end{pgfscope}%
\begin{pgfscope}%
\pgfsys@transformshift{4.498901in}{3.230401in}%
\pgfsys@useobject{currentmarker}{}%
\end{pgfscope}%
\begin{pgfscope}%
\pgfsys@transformshift{4.695977in}{2.722004in}%
\pgfsys@useobject{currentmarker}{}%
\end{pgfscope}%
\begin{pgfscope}%
\pgfsys@transformshift{4.893052in}{3.008368in}%
\pgfsys@useobject{currentmarker}{}%
\end{pgfscope}%
\begin{pgfscope}%
\pgfsys@transformshift{5.090128in}{3.207894in}%
\pgfsys@useobject{currentmarker}{}%
\end{pgfscope}%
\begin{pgfscope}%
\pgfsys@transformshift{5.287204in}{2.958443in}%
\pgfsys@useobject{currentmarker}{}%
\end{pgfscope}%
\begin{pgfscope}%
\pgfsys@transformshift{5.484279in}{2.722004in}%
\pgfsys@useobject{currentmarker}{}%
\end{pgfscope}%
\begin{pgfscope}%
\pgfsys@transformshift{5.681355in}{3.190798in}%
\pgfsys@useobject{currentmarker}{}%
\end{pgfscope}%
\begin{pgfscope}%
\pgfsys@transformshift{5.878431in}{3.393405in}%
\pgfsys@useobject{currentmarker}{}%
\end{pgfscope}%
\begin{pgfscope}%
\pgfsys@transformshift{6.075506in}{3.449966in}%
\pgfsys@useobject{currentmarker}{}%
\end{pgfscope}%
\begin{pgfscope}%
\pgfsys@transformshift{6.272582in}{2.722004in}%
\pgfsys@useobject{currentmarker}{}%
\end{pgfscope}%
\begin{pgfscope}%
\pgfsys@transformshift{6.469658in}{2.722004in}%
\pgfsys@useobject{currentmarker}{}%
\end{pgfscope}%
\begin{pgfscope}%
\pgfsys@transformshift{6.666733in}{3.026732in}%
\pgfsys@useobject{currentmarker}{}%
\end{pgfscope}%
\end{pgfscope}%
\begin{pgfscope}%
\pgfsetrectcap%
\pgfsetmiterjoin%
\pgfsetlinewidth{0.803000pt}%
\definecolor{currentstroke}{rgb}{1.000000,1.000000,1.000000}%
\pgfsetstrokecolor{currentstroke}%
\pgfsetdash{}{0pt}%
\pgfpathmoveto{\pgfqpoint{3.976651in}{2.685606in}}%
\pgfpathlineto{\pgfqpoint{3.976651in}{3.486364in}}%
\pgfusepath{stroke}%
\end{pgfscope}%
\begin{pgfscope}%
\pgfsetrectcap%
\pgfsetmiterjoin%
\pgfsetlinewidth{0.803000pt}%
\definecolor{currentstroke}{rgb}{1.000000,1.000000,1.000000}%
\pgfsetstrokecolor{currentstroke}%
\pgfsetdash{}{0pt}%
\pgfpathmoveto{\pgfqpoint{6.794832in}{2.685606in}}%
\pgfpathlineto{\pgfqpoint{6.794832in}{3.486364in}}%
\pgfusepath{stroke}%
\end{pgfscope}%
\begin{pgfscope}%
\pgfsetrectcap%
\pgfsetmiterjoin%
\pgfsetlinewidth{0.803000pt}%
\definecolor{currentstroke}{rgb}{1.000000,1.000000,1.000000}%
\pgfsetstrokecolor{currentstroke}%
\pgfsetdash{}{0pt}%
\pgfpathmoveto{\pgfqpoint{3.976651in}{2.685606in}}%
\pgfpathlineto{\pgfqpoint{6.794832in}{2.685606in}}%
\pgfusepath{stroke}%
\end{pgfscope}%
\begin{pgfscope}%
\pgfsetrectcap%
\pgfsetmiterjoin%
\pgfsetlinewidth{0.803000pt}%
\definecolor{currentstroke}{rgb}{1.000000,1.000000,1.000000}%
\pgfsetstrokecolor{currentstroke}%
\pgfsetdash{}{0pt}%
\pgfpathmoveto{\pgfqpoint{3.976651in}{3.486364in}}%
\pgfpathlineto{\pgfqpoint{6.794832in}{3.486364in}}%
\pgfusepath{stroke}%
\end{pgfscope}%
\begin{pgfscope}%
\definecolor{textcolor}{rgb}{0.150000,0.150000,0.150000}%
\pgfsetstrokecolor{textcolor}%
\pgfsetfillcolor{textcolor}%
\pgftext[x=5.385741in,y=3.569697in,,base]{\color{textcolor}\rmfamily\fontsize{12.000000}{14.400000}\selectfont VZ}%
\end{pgfscope}%
\begin{pgfscope}%
\pgfsetbuttcap%
\pgfsetmiterjoin%
\definecolor{currentfill}{rgb}{0.917647,0.917647,0.949020}%
\pgfsetfillcolor{currentfill}%
\pgfsetlinewidth{0.000000pt}%
\definecolor{currentstroke}{rgb}{0.000000,0.000000,0.000000}%
\pgfsetstrokecolor{currentstroke}%
\pgfsetstrokeopacity{0.000000}%
\pgfsetdash{}{0pt}%
\pgfpathmoveto{\pgfqpoint{0.594832in}{1.564545in}}%
\pgfpathlineto{\pgfqpoint{3.413014in}{1.564545in}}%
\pgfpathlineto{\pgfqpoint{3.413014in}{2.365303in}}%
\pgfpathlineto{\pgfqpoint{0.594832in}{2.365303in}}%
\pgfpathclose%
\pgfusepath{fill}%
\end{pgfscope}%
\begin{pgfscope}%
\pgfpathrectangle{\pgfqpoint{0.594832in}{1.564545in}}{\pgfqpoint{2.818182in}{0.800758in}}%
\pgfusepath{clip}%
\pgfsetroundcap%
\pgfsetroundjoin%
\pgfsetlinewidth{0.803000pt}%
\definecolor{currentstroke}{rgb}{1.000000,1.000000,1.000000}%
\pgfsetstrokecolor{currentstroke}%
\pgfsetdash{}{0pt}%
\pgfpathmoveto{\pgfqpoint{0.722932in}{1.564545in}}%
\pgfpathlineto{\pgfqpoint{0.722932in}{2.365303in}}%
\pgfusepath{stroke}%
\end{pgfscope}%
\begin{pgfscope}%
\definecolor{textcolor}{rgb}{0.150000,0.150000,0.150000}%
\pgfsetstrokecolor{textcolor}%
\pgfsetfillcolor{textcolor}%
\pgftext[x=0.758095in,y=0.277680in,left,base,rotate=90.000000]{\color{textcolor}\rmfamily\fontsize{10.000000}{12.000000}\selectfont ARMA\_00\_MMM}%
\end{pgfscope}%
\begin{pgfscope}%
\pgfpathrectangle{\pgfqpoint{0.594832in}{1.564545in}}{\pgfqpoint{2.818182in}{0.800758in}}%
\pgfusepath{clip}%
\pgfsetroundcap%
\pgfsetroundjoin%
\pgfsetlinewidth{0.803000pt}%
\definecolor{currentstroke}{rgb}{1.000000,1.000000,1.000000}%
\pgfsetstrokecolor{currentstroke}%
\pgfsetdash{}{0pt}%
\pgfpathmoveto{\pgfqpoint{0.920007in}{1.564545in}}%
\pgfpathlineto{\pgfqpoint{0.920007in}{2.365303in}}%
\pgfusepath{stroke}%
\end{pgfscope}%
\begin{pgfscope}%
\definecolor{textcolor}{rgb}{0.150000,0.150000,0.150000}%
\pgfsetstrokecolor{textcolor}%
\pgfsetfillcolor{textcolor}%
\pgftext[x=0.955170in,y=0.277680in,left,base,rotate=90.000000]{\color{textcolor}\rmfamily\fontsize{10.000000}{12.000000}\selectfont ARMA\_01\_MMM}%
\end{pgfscope}%
\begin{pgfscope}%
\pgfpathrectangle{\pgfqpoint{0.594832in}{1.564545in}}{\pgfqpoint{2.818182in}{0.800758in}}%
\pgfusepath{clip}%
\pgfsetroundcap%
\pgfsetroundjoin%
\pgfsetlinewidth{0.803000pt}%
\definecolor{currentstroke}{rgb}{1.000000,1.000000,1.000000}%
\pgfsetstrokecolor{currentstroke}%
\pgfsetdash{}{0pt}%
\pgfpathmoveto{\pgfqpoint{1.117083in}{1.564545in}}%
\pgfpathlineto{\pgfqpoint{1.117083in}{2.365303in}}%
\pgfusepath{stroke}%
\end{pgfscope}%
\begin{pgfscope}%
\definecolor{textcolor}{rgb}{0.150000,0.150000,0.150000}%
\pgfsetstrokecolor{textcolor}%
\pgfsetfillcolor{textcolor}%
\pgftext[x=1.152246in,y=0.277680in,left,base,rotate=90.000000]{\color{textcolor}\rmfamily\fontsize{10.000000}{12.000000}\selectfont ARMA\_02\_MMM}%
\end{pgfscope}%
\begin{pgfscope}%
\pgfpathrectangle{\pgfqpoint{0.594832in}{1.564545in}}{\pgfqpoint{2.818182in}{0.800758in}}%
\pgfusepath{clip}%
\pgfsetroundcap%
\pgfsetroundjoin%
\pgfsetlinewidth{0.803000pt}%
\definecolor{currentstroke}{rgb}{1.000000,1.000000,1.000000}%
\pgfsetstrokecolor{currentstroke}%
\pgfsetdash{}{0pt}%
\pgfpathmoveto{\pgfqpoint{1.314158in}{1.564545in}}%
\pgfpathlineto{\pgfqpoint{1.314158in}{2.365303in}}%
\pgfusepath{stroke}%
\end{pgfscope}%
\begin{pgfscope}%
\definecolor{textcolor}{rgb}{0.150000,0.150000,0.150000}%
\pgfsetstrokecolor{textcolor}%
\pgfsetfillcolor{textcolor}%
\pgftext[x=1.349322in,y=0.277680in,left,base,rotate=90.000000]{\color{textcolor}\rmfamily\fontsize{10.000000}{12.000000}\selectfont ARMA\_03\_MMM}%
\end{pgfscope}%
\begin{pgfscope}%
\pgfpathrectangle{\pgfqpoint{0.594832in}{1.564545in}}{\pgfqpoint{2.818182in}{0.800758in}}%
\pgfusepath{clip}%
\pgfsetroundcap%
\pgfsetroundjoin%
\pgfsetlinewidth{0.803000pt}%
\definecolor{currentstroke}{rgb}{1.000000,1.000000,1.000000}%
\pgfsetstrokecolor{currentstroke}%
\pgfsetdash{}{0pt}%
\pgfpathmoveto{\pgfqpoint{1.511234in}{1.564545in}}%
\pgfpathlineto{\pgfqpoint{1.511234in}{2.365303in}}%
\pgfusepath{stroke}%
\end{pgfscope}%
\begin{pgfscope}%
\definecolor{textcolor}{rgb}{0.150000,0.150000,0.150000}%
\pgfsetstrokecolor{textcolor}%
\pgfsetfillcolor{textcolor}%
\pgftext[x=1.546397in,y=0.277680in,left,base,rotate=90.000000]{\color{textcolor}\rmfamily\fontsize{10.000000}{12.000000}\selectfont ARMA\_10\_MMM}%
\end{pgfscope}%
\begin{pgfscope}%
\pgfpathrectangle{\pgfqpoint{0.594832in}{1.564545in}}{\pgfqpoint{2.818182in}{0.800758in}}%
\pgfusepath{clip}%
\pgfsetroundcap%
\pgfsetroundjoin%
\pgfsetlinewidth{0.803000pt}%
\definecolor{currentstroke}{rgb}{1.000000,1.000000,1.000000}%
\pgfsetstrokecolor{currentstroke}%
\pgfsetdash{}{0pt}%
\pgfpathmoveto{\pgfqpoint{1.708310in}{1.564545in}}%
\pgfpathlineto{\pgfqpoint{1.708310in}{2.365303in}}%
\pgfusepath{stroke}%
\end{pgfscope}%
\begin{pgfscope}%
\definecolor{textcolor}{rgb}{0.150000,0.150000,0.150000}%
\pgfsetstrokecolor{textcolor}%
\pgfsetfillcolor{textcolor}%
\pgftext[x=1.743473in,y=0.277680in,left,base,rotate=90.000000]{\color{textcolor}\rmfamily\fontsize{10.000000}{12.000000}\selectfont ARMA\_11\_MMM}%
\end{pgfscope}%
\begin{pgfscope}%
\pgfpathrectangle{\pgfqpoint{0.594832in}{1.564545in}}{\pgfqpoint{2.818182in}{0.800758in}}%
\pgfusepath{clip}%
\pgfsetroundcap%
\pgfsetroundjoin%
\pgfsetlinewidth{0.803000pt}%
\definecolor{currentstroke}{rgb}{1.000000,1.000000,1.000000}%
\pgfsetstrokecolor{currentstroke}%
\pgfsetdash{}{0pt}%
\pgfpathmoveto{\pgfqpoint{1.905385in}{1.564545in}}%
\pgfpathlineto{\pgfqpoint{1.905385in}{2.365303in}}%
\pgfusepath{stroke}%
\end{pgfscope}%
\begin{pgfscope}%
\definecolor{textcolor}{rgb}{0.150000,0.150000,0.150000}%
\pgfsetstrokecolor{textcolor}%
\pgfsetfillcolor{textcolor}%
\pgftext[x=1.940549in,y=0.277680in,left,base,rotate=90.000000]{\color{textcolor}\rmfamily\fontsize{10.000000}{12.000000}\selectfont ARMA\_12\_MMM}%
\end{pgfscope}%
\begin{pgfscope}%
\pgfpathrectangle{\pgfqpoint{0.594832in}{1.564545in}}{\pgfqpoint{2.818182in}{0.800758in}}%
\pgfusepath{clip}%
\pgfsetroundcap%
\pgfsetroundjoin%
\pgfsetlinewidth{0.803000pt}%
\definecolor{currentstroke}{rgb}{1.000000,1.000000,1.000000}%
\pgfsetstrokecolor{currentstroke}%
\pgfsetdash{}{0pt}%
\pgfpathmoveto{\pgfqpoint{2.102461in}{1.564545in}}%
\pgfpathlineto{\pgfqpoint{2.102461in}{2.365303in}}%
\pgfusepath{stroke}%
\end{pgfscope}%
\begin{pgfscope}%
\definecolor{textcolor}{rgb}{0.150000,0.150000,0.150000}%
\pgfsetstrokecolor{textcolor}%
\pgfsetfillcolor{textcolor}%
\pgftext[x=2.137624in,y=0.277680in,left,base,rotate=90.000000]{\color{textcolor}\rmfamily\fontsize{10.000000}{12.000000}\selectfont ARMA\_13\_MMM}%
\end{pgfscope}%
\begin{pgfscope}%
\pgfpathrectangle{\pgfqpoint{0.594832in}{1.564545in}}{\pgfqpoint{2.818182in}{0.800758in}}%
\pgfusepath{clip}%
\pgfsetroundcap%
\pgfsetroundjoin%
\pgfsetlinewidth{0.803000pt}%
\definecolor{currentstroke}{rgb}{1.000000,1.000000,1.000000}%
\pgfsetstrokecolor{currentstroke}%
\pgfsetdash{}{0pt}%
\pgfpathmoveto{\pgfqpoint{2.299537in}{1.564545in}}%
\pgfpathlineto{\pgfqpoint{2.299537in}{2.365303in}}%
\pgfusepath{stroke}%
\end{pgfscope}%
\begin{pgfscope}%
\definecolor{textcolor}{rgb}{0.150000,0.150000,0.150000}%
\pgfsetstrokecolor{textcolor}%
\pgfsetfillcolor{textcolor}%
\pgftext[x=2.334700in,y=0.277680in,left,base,rotate=90.000000]{\color{textcolor}\rmfamily\fontsize{10.000000}{12.000000}\selectfont ARMA\_20\_MMM}%
\end{pgfscope}%
\begin{pgfscope}%
\pgfpathrectangle{\pgfqpoint{0.594832in}{1.564545in}}{\pgfqpoint{2.818182in}{0.800758in}}%
\pgfusepath{clip}%
\pgfsetroundcap%
\pgfsetroundjoin%
\pgfsetlinewidth{0.803000pt}%
\definecolor{currentstroke}{rgb}{1.000000,1.000000,1.000000}%
\pgfsetstrokecolor{currentstroke}%
\pgfsetdash{}{0pt}%
\pgfpathmoveto{\pgfqpoint{2.496612in}{1.564545in}}%
\pgfpathlineto{\pgfqpoint{2.496612in}{2.365303in}}%
\pgfusepath{stroke}%
\end{pgfscope}%
\begin{pgfscope}%
\definecolor{textcolor}{rgb}{0.150000,0.150000,0.150000}%
\pgfsetstrokecolor{textcolor}%
\pgfsetfillcolor{textcolor}%
\pgftext[x=2.531776in,y=0.277680in,left,base,rotate=90.000000]{\color{textcolor}\rmfamily\fontsize{10.000000}{12.000000}\selectfont ARMA\_21\_MMM}%
\end{pgfscope}%
\begin{pgfscope}%
\pgfpathrectangle{\pgfqpoint{0.594832in}{1.564545in}}{\pgfqpoint{2.818182in}{0.800758in}}%
\pgfusepath{clip}%
\pgfsetroundcap%
\pgfsetroundjoin%
\pgfsetlinewidth{0.803000pt}%
\definecolor{currentstroke}{rgb}{1.000000,1.000000,1.000000}%
\pgfsetstrokecolor{currentstroke}%
\pgfsetdash{}{0pt}%
\pgfpathmoveto{\pgfqpoint{2.693688in}{1.564545in}}%
\pgfpathlineto{\pgfqpoint{2.693688in}{2.365303in}}%
\pgfusepath{stroke}%
\end{pgfscope}%
\begin{pgfscope}%
\definecolor{textcolor}{rgb}{0.150000,0.150000,0.150000}%
\pgfsetstrokecolor{textcolor}%
\pgfsetfillcolor{textcolor}%
\pgftext[x=2.728851in,y=0.277680in,left,base,rotate=90.000000]{\color{textcolor}\rmfamily\fontsize{10.000000}{12.000000}\selectfont ARMA\_22\_MMM}%
\end{pgfscope}%
\begin{pgfscope}%
\pgfpathrectangle{\pgfqpoint{0.594832in}{1.564545in}}{\pgfqpoint{2.818182in}{0.800758in}}%
\pgfusepath{clip}%
\pgfsetroundcap%
\pgfsetroundjoin%
\pgfsetlinewidth{0.803000pt}%
\definecolor{currentstroke}{rgb}{1.000000,1.000000,1.000000}%
\pgfsetstrokecolor{currentstroke}%
\pgfsetdash{}{0pt}%
\pgfpathmoveto{\pgfqpoint{2.890764in}{1.564545in}}%
\pgfpathlineto{\pgfqpoint{2.890764in}{2.365303in}}%
\pgfusepath{stroke}%
\end{pgfscope}%
\begin{pgfscope}%
\definecolor{textcolor}{rgb}{0.150000,0.150000,0.150000}%
\pgfsetstrokecolor{textcolor}%
\pgfsetfillcolor{textcolor}%
\pgftext[x=2.925927in,y=0.277680in,left,base,rotate=90.000000]{\color{textcolor}\rmfamily\fontsize{10.000000}{12.000000}\selectfont ARMA\_30\_MMM}%
\end{pgfscope}%
\begin{pgfscope}%
\pgfpathrectangle{\pgfqpoint{0.594832in}{1.564545in}}{\pgfqpoint{2.818182in}{0.800758in}}%
\pgfusepath{clip}%
\pgfsetroundcap%
\pgfsetroundjoin%
\pgfsetlinewidth{0.803000pt}%
\definecolor{currentstroke}{rgb}{1.000000,1.000000,1.000000}%
\pgfsetstrokecolor{currentstroke}%
\pgfsetdash{}{0pt}%
\pgfpathmoveto{\pgfqpoint{3.087839in}{1.564545in}}%
\pgfpathlineto{\pgfqpoint{3.087839in}{2.365303in}}%
\pgfusepath{stroke}%
\end{pgfscope}%
\begin{pgfscope}%
\definecolor{textcolor}{rgb}{0.150000,0.150000,0.150000}%
\pgfsetstrokecolor{textcolor}%
\pgfsetfillcolor{textcolor}%
\pgftext[x=3.123003in,y=0.277680in,left,base,rotate=90.000000]{\color{textcolor}\rmfamily\fontsize{10.000000}{12.000000}\selectfont ARMA\_31\_MMM}%
\end{pgfscope}%
\begin{pgfscope}%
\pgfpathrectangle{\pgfqpoint{0.594832in}{1.564545in}}{\pgfqpoint{2.818182in}{0.800758in}}%
\pgfusepath{clip}%
\pgfsetroundcap%
\pgfsetroundjoin%
\pgfsetlinewidth{0.803000pt}%
\definecolor{currentstroke}{rgb}{1.000000,1.000000,1.000000}%
\pgfsetstrokecolor{currentstroke}%
\pgfsetdash{}{0pt}%
\pgfpathmoveto{\pgfqpoint{3.284915in}{1.564545in}}%
\pgfpathlineto{\pgfqpoint{3.284915in}{2.365303in}}%
\pgfusepath{stroke}%
\end{pgfscope}%
\begin{pgfscope}%
\definecolor{textcolor}{rgb}{0.150000,0.150000,0.150000}%
\pgfsetstrokecolor{textcolor}%
\pgfsetfillcolor{textcolor}%
\pgftext[x=3.320078in,y=0.100000in,left,base,rotate=90.000000]{\color{textcolor}\rmfamily\fontsize{10.000000}{12.000000}\selectfont AUTOARMA\_MMM}%
\end{pgfscope}%
\begin{pgfscope}%
\pgfpathrectangle{\pgfqpoint{0.594832in}{1.564545in}}{\pgfqpoint{2.818182in}{0.800758in}}%
\pgfusepath{clip}%
\pgfsetroundcap%
\pgfsetroundjoin%
\pgfsetlinewidth{0.803000pt}%
\definecolor{currentstroke}{rgb}{1.000000,1.000000,1.000000}%
\pgfsetstrokecolor{currentstroke}%
\pgfsetdash{}{0pt}%
\pgfpathmoveto{\pgfqpoint{0.594832in}{1.714235in}}%
\pgfpathlineto{\pgfqpoint{3.413014in}{1.714235in}}%
\pgfusepath{stroke}%
\end{pgfscope}%
\begin{pgfscope}%
\definecolor{textcolor}{rgb}{0.150000,0.150000,0.150000}%
\pgfsetstrokecolor{textcolor}%
\pgfsetfillcolor{textcolor}%
\pgftext[x=0.100000in,y=1.661473in,left,base]{\color{textcolor}\rmfamily\fontsize{10.000000}{12.000000}\selectfont 0.245}%
\end{pgfscope}%
\begin{pgfscope}%
\pgfpathrectangle{\pgfqpoint{0.594832in}{1.564545in}}{\pgfqpoint{2.818182in}{0.800758in}}%
\pgfusepath{clip}%
\pgfsetroundcap%
\pgfsetroundjoin%
\pgfsetlinewidth{0.803000pt}%
\definecolor{currentstroke}{rgb}{1.000000,1.000000,1.000000}%
\pgfsetstrokecolor{currentstroke}%
\pgfsetdash{}{0pt}%
\pgfpathmoveto{\pgfqpoint{0.594832in}{1.935814in}}%
\pgfpathlineto{\pgfqpoint{3.413014in}{1.935814in}}%
\pgfusepath{stroke}%
\end{pgfscope}%
\begin{pgfscope}%
\definecolor{textcolor}{rgb}{0.150000,0.150000,0.150000}%
\pgfsetstrokecolor{textcolor}%
\pgfsetfillcolor{textcolor}%
\pgftext[x=0.100000in,y=1.883052in,left,base]{\color{textcolor}\rmfamily\fontsize{10.000000}{12.000000}\selectfont 0.246}%
\end{pgfscope}%
\begin{pgfscope}%
\pgfpathrectangle{\pgfqpoint{0.594832in}{1.564545in}}{\pgfqpoint{2.818182in}{0.800758in}}%
\pgfusepath{clip}%
\pgfsetroundcap%
\pgfsetroundjoin%
\pgfsetlinewidth{0.803000pt}%
\definecolor{currentstroke}{rgb}{1.000000,1.000000,1.000000}%
\pgfsetstrokecolor{currentstroke}%
\pgfsetdash{}{0pt}%
\pgfpathmoveto{\pgfqpoint{0.594832in}{2.157393in}}%
\pgfpathlineto{\pgfqpoint{3.413014in}{2.157393in}}%
\pgfusepath{stroke}%
\end{pgfscope}%
\begin{pgfscope}%
\definecolor{textcolor}{rgb}{0.150000,0.150000,0.150000}%
\pgfsetstrokecolor{textcolor}%
\pgfsetfillcolor{textcolor}%
\pgftext[x=0.100000in,y=2.104631in,left,base]{\color{textcolor}\rmfamily\fontsize{10.000000}{12.000000}\selectfont 0.247}%
\end{pgfscope}%
\begin{pgfscope}%
\pgfpathrectangle{\pgfqpoint{0.594832in}{1.564545in}}{\pgfqpoint{2.818182in}{0.800758in}}%
\pgfusepath{clip}%
\pgfsetroundcap%
\pgfsetroundjoin%
\pgfsetlinewidth{1.505625pt}%
\definecolor{currentstroke}{rgb}{0.737255,0.741176,0.133333}%
\pgfsetstrokecolor{currentstroke}%
\pgfsetdash{}{0pt}%
\pgfpathmoveto{\pgfqpoint{0.722932in}{1.844819in}}%
\pgfpathlineto{\pgfqpoint{0.920007in}{1.934672in}}%
\pgfpathlineto{\pgfqpoint{1.117083in}{2.182615in}}%
\pgfpathlineto{\pgfqpoint{1.314158in}{1.964619in}}%
\pgfpathlineto{\pgfqpoint{1.511234in}{1.908091in}}%
\pgfpathlineto{\pgfqpoint{1.708310in}{1.600943in}}%
\pgfpathlineto{\pgfqpoint{1.905385in}{2.328905in}}%
\pgfpathlineto{\pgfqpoint{2.102461in}{1.964619in}}%
\pgfpathlineto{\pgfqpoint{2.299537in}{2.046911in}}%
\pgfpathlineto{\pgfqpoint{2.496612in}{2.186104in}}%
\pgfpathlineto{\pgfqpoint{2.693688in}{2.081505in}}%
\pgfpathlineto{\pgfqpoint{2.890764in}{1.964619in}}%
\pgfpathlineto{\pgfqpoint{3.087839in}{1.964619in}}%
\pgfpathlineto{\pgfqpoint{3.284915in}{1.891066in}}%
\pgfusepath{stroke}%
\end{pgfscope}%
\begin{pgfscope}%
\pgfpathrectangle{\pgfqpoint{0.594832in}{1.564545in}}{\pgfqpoint{2.818182in}{0.800758in}}%
\pgfusepath{clip}%
\pgfsetbuttcap%
\pgfsetroundjoin%
\definecolor{currentfill}{rgb}{0.737255,0.741176,0.133333}%
\pgfsetfillcolor{currentfill}%
\pgfsetlinewidth{1.003750pt}%
\definecolor{currentstroke}{rgb}{0.737255,0.741176,0.133333}%
\pgfsetstrokecolor{currentstroke}%
\pgfsetdash{}{0pt}%
\pgfsys@defobject{currentmarker}{\pgfqpoint{-0.041667in}{-0.041667in}}{\pgfqpoint{0.041667in}{0.041667in}}{%
\pgfpathmoveto{\pgfqpoint{0.000000in}{-0.041667in}}%
\pgfpathcurveto{\pgfqpoint{0.011050in}{-0.041667in}}{\pgfqpoint{0.021649in}{-0.037276in}}{\pgfqpoint{0.029463in}{-0.029463in}}%
\pgfpathcurveto{\pgfqpoint{0.037276in}{-0.021649in}}{\pgfqpoint{0.041667in}{-0.011050in}}{\pgfqpoint{0.041667in}{0.000000in}}%
\pgfpathcurveto{\pgfqpoint{0.041667in}{0.011050in}}{\pgfqpoint{0.037276in}{0.021649in}}{\pgfqpoint{0.029463in}{0.029463in}}%
\pgfpathcurveto{\pgfqpoint{0.021649in}{0.037276in}}{\pgfqpoint{0.011050in}{0.041667in}}{\pgfqpoint{0.000000in}{0.041667in}}%
\pgfpathcurveto{\pgfqpoint{-0.011050in}{0.041667in}}{\pgfqpoint{-0.021649in}{0.037276in}}{\pgfqpoint{-0.029463in}{0.029463in}}%
\pgfpathcurveto{\pgfqpoint{-0.037276in}{0.021649in}}{\pgfqpoint{-0.041667in}{0.011050in}}{\pgfqpoint{-0.041667in}{0.000000in}}%
\pgfpathcurveto{\pgfqpoint{-0.041667in}{-0.011050in}}{\pgfqpoint{-0.037276in}{-0.021649in}}{\pgfqpoint{-0.029463in}{-0.029463in}}%
\pgfpathcurveto{\pgfqpoint{-0.021649in}{-0.037276in}}{\pgfqpoint{-0.011050in}{-0.041667in}}{\pgfqpoint{0.000000in}{-0.041667in}}%
\pgfpathclose%
\pgfusepath{stroke,fill}%
}%
\begin{pgfscope}%
\pgfsys@transformshift{0.722932in}{1.844819in}%
\pgfsys@useobject{currentmarker}{}%
\end{pgfscope}%
\begin{pgfscope}%
\pgfsys@transformshift{0.920007in}{1.934672in}%
\pgfsys@useobject{currentmarker}{}%
\end{pgfscope}%
\begin{pgfscope}%
\pgfsys@transformshift{1.117083in}{2.182615in}%
\pgfsys@useobject{currentmarker}{}%
\end{pgfscope}%
\begin{pgfscope}%
\pgfsys@transformshift{1.314158in}{1.964619in}%
\pgfsys@useobject{currentmarker}{}%
\end{pgfscope}%
\begin{pgfscope}%
\pgfsys@transformshift{1.511234in}{1.908091in}%
\pgfsys@useobject{currentmarker}{}%
\end{pgfscope}%
\begin{pgfscope}%
\pgfsys@transformshift{1.708310in}{1.600943in}%
\pgfsys@useobject{currentmarker}{}%
\end{pgfscope}%
\begin{pgfscope}%
\pgfsys@transformshift{1.905385in}{2.328905in}%
\pgfsys@useobject{currentmarker}{}%
\end{pgfscope}%
\begin{pgfscope}%
\pgfsys@transformshift{2.102461in}{1.964619in}%
\pgfsys@useobject{currentmarker}{}%
\end{pgfscope}%
\begin{pgfscope}%
\pgfsys@transformshift{2.299537in}{2.046911in}%
\pgfsys@useobject{currentmarker}{}%
\end{pgfscope}%
\begin{pgfscope}%
\pgfsys@transformshift{2.496612in}{2.186104in}%
\pgfsys@useobject{currentmarker}{}%
\end{pgfscope}%
\begin{pgfscope}%
\pgfsys@transformshift{2.693688in}{2.081505in}%
\pgfsys@useobject{currentmarker}{}%
\end{pgfscope}%
\begin{pgfscope}%
\pgfsys@transformshift{2.890764in}{1.964619in}%
\pgfsys@useobject{currentmarker}{}%
\end{pgfscope}%
\begin{pgfscope}%
\pgfsys@transformshift{3.087839in}{1.964619in}%
\pgfsys@useobject{currentmarker}{}%
\end{pgfscope}%
\begin{pgfscope}%
\pgfsys@transformshift{3.284915in}{1.891066in}%
\pgfsys@useobject{currentmarker}{}%
\end{pgfscope}%
\end{pgfscope}%
\begin{pgfscope}%
\pgfsetrectcap%
\pgfsetmiterjoin%
\pgfsetlinewidth{0.803000pt}%
\definecolor{currentstroke}{rgb}{1.000000,1.000000,1.000000}%
\pgfsetstrokecolor{currentstroke}%
\pgfsetdash{}{0pt}%
\pgfpathmoveto{\pgfqpoint{0.594832in}{1.564545in}}%
\pgfpathlineto{\pgfqpoint{0.594832in}{2.365303in}}%
\pgfusepath{stroke}%
\end{pgfscope}%
\begin{pgfscope}%
\pgfsetrectcap%
\pgfsetmiterjoin%
\pgfsetlinewidth{0.803000pt}%
\definecolor{currentstroke}{rgb}{1.000000,1.000000,1.000000}%
\pgfsetstrokecolor{currentstroke}%
\pgfsetdash{}{0pt}%
\pgfpathmoveto{\pgfqpoint{3.413014in}{1.564545in}}%
\pgfpathlineto{\pgfqpoint{3.413014in}{2.365303in}}%
\pgfusepath{stroke}%
\end{pgfscope}%
\begin{pgfscope}%
\pgfsetrectcap%
\pgfsetmiterjoin%
\pgfsetlinewidth{0.803000pt}%
\definecolor{currentstroke}{rgb}{1.000000,1.000000,1.000000}%
\pgfsetstrokecolor{currentstroke}%
\pgfsetdash{}{0pt}%
\pgfpathmoveto{\pgfqpoint{0.594832in}{1.564545in}}%
\pgfpathlineto{\pgfqpoint{3.413014in}{1.564545in}}%
\pgfusepath{stroke}%
\end{pgfscope}%
\begin{pgfscope}%
\pgfsetrectcap%
\pgfsetmiterjoin%
\pgfsetlinewidth{0.803000pt}%
\definecolor{currentstroke}{rgb}{1.000000,1.000000,1.000000}%
\pgfsetstrokecolor{currentstroke}%
\pgfsetdash{}{0pt}%
\pgfpathmoveto{\pgfqpoint{0.594832in}{2.365303in}}%
\pgfpathlineto{\pgfqpoint{3.413014in}{2.365303in}}%
\pgfusepath{stroke}%
\end{pgfscope}%
\begin{pgfscope}%
\definecolor{textcolor}{rgb}{0.150000,0.150000,0.150000}%
\pgfsetstrokecolor{textcolor}%
\pgfsetfillcolor{textcolor}%
\pgftext[x=2.003923in,y=2.448636in,,base]{\color{textcolor}\rmfamily\fontsize{12.000000}{14.400000}\selectfont V}%
\end{pgfscope}%
\begin{pgfscope}%
\pgfsetbuttcap%
\pgfsetmiterjoin%
\definecolor{currentfill}{rgb}{0.917647,0.917647,0.949020}%
\pgfsetfillcolor{currentfill}%
\pgfsetlinewidth{0.000000pt}%
\definecolor{currentstroke}{rgb}{0.000000,0.000000,0.000000}%
\pgfsetstrokecolor{currentstroke}%
\pgfsetstrokeopacity{0.000000}%
\pgfsetdash{}{0pt}%
\pgfpathmoveto{\pgfqpoint{3.976651in}{1.564545in}}%
\pgfpathlineto{\pgfqpoint{6.794832in}{1.564545in}}%
\pgfpathlineto{\pgfqpoint{6.794832in}{2.365303in}}%
\pgfpathlineto{\pgfqpoint{3.976651in}{2.365303in}}%
\pgfpathclose%
\pgfusepath{fill}%
\end{pgfscope}%
\begin{pgfscope}%
\pgfpathrectangle{\pgfqpoint{3.976651in}{1.564545in}}{\pgfqpoint{2.818182in}{0.800758in}}%
\pgfusepath{clip}%
\pgfsetroundcap%
\pgfsetroundjoin%
\pgfsetlinewidth{0.803000pt}%
\definecolor{currentstroke}{rgb}{1.000000,1.000000,1.000000}%
\pgfsetstrokecolor{currentstroke}%
\pgfsetdash{}{0pt}%
\pgfpathmoveto{\pgfqpoint{4.104750in}{1.564545in}}%
\pgfpathlineto{\pgfqpoint{4.104750in}{2.365303in}}%
\pgfusepath{stroke}%
\end{pgfscope}%
\begin{pgfscope}%
\definecolor{textcolor}{rgb}{0.150000,0.150000,0.150000}%
\pgfsetstrokecolor{textcolor}%
\pgfsetfillcolor{textcolor}%
\pgftext[x=4.139913in,y=0.277680in,left,base,rotate=90.000000]{\color{textcolor}\rmfamily\fontsize{10.000000}{12.000000}\selectfont ARMA\_00\_MMM}%
\end{pgfscope}%
\begin{pgfscope}%
\pgfpathrectangle{\pgfqpoint{3.976651in}{1.564545in}}{\pgfqpoint{2.818182in}{0.800758in}}%
\pgfusepath{clip}%
\pgfsetroundcap%
\pgfsetroundjoin%
\pgfsetlinewidth{0.803000pt}%
\definecolor{currentstroke}{rgb}{1.000000,1.000000,1.000000}%
\pgfsetstrokecolor{currentstroke}%
\pgfsetdash{}{0pt}%
\pgfpathmoveto{\pgfqpoint{4.301825in}{1.564545in}}%
\pgfpathlineto{\pgfqpoint{4.301825in}{2.365303in}}%
\pgfusepath{stroke}%
\end{pgfscope}%
\begin{pgfscope}%
\definecolor{textcolor}{rgb}{0.150000,0.150000,0.150000}%
\pgfsetstrokecolor{textcolor}%
\pgfsetfillcolor{textcolor}%
\pgftext[x=4.336989in,y=0.277680in,left,base,rotate=90.000000]{\color{textcolor}\rmfamily\fontsize{10.000000}{12.000000}\selectfont ARMA\_01\_MMM}%
\end{pgfscope}%
\begin{pgfscope}%
\pgfpathrectangle{\pgfqpoint{3.976651in}{1.564545in}}{\pgfqpoint{2.818182in}{0.800758in}}%
\pgfusepath{clip}%
\pgfsetroundcap%
\pgfsetroundjoin%
\pgfsetlinewidth{0.803000pt}%
\definecolor{currentstroke}{rgb}{1.000000,1.000000,1.000000}%
\pgfsetstrokecolor{currentstroke}%
\pgfsetdash{}{0pt}%
\pgfpathmoveto{\pgfqpoint{4.498901in}{1.564545in}}%
\pgfpathlineto{\pgfqpoint{4.498901in}{2.365303in}}%
\pgfusepath{stroke}%
\end{pgfscope}%
\begin{pgfscope}%
\definecolor{textcolor}{rgb}{0.150000,0.150000,0.150000}%
\pgfsetstrokecolor{textcolor}%
\pgfsetfillcolor{textcolor}%
\pgftext[x=4.534064in,y=0.277680in,left,base,rotate=90.000000]{\color{textcolor}\rmfamily\fontsize{10.000000}{12.000000}\selectfont ARMA\_02\_MMM}%
\end{pgfscope}%
\begin{pgfscope}%
\pgfpathrectangle{\pgfqpoint{3.976651in}{1.564545in}}{\pgfqpoint{2.818182in}{0.800758in}}%
\pgfusepath{clip}%
\pgfsetroundcap%
\pgfsetroundjoin%
\pgfsetlinewidth{0.803000pt}%
\definecolor{currentstroke}{rgb}{1.000000,1.000000,1.000000}%
\pgfsetstrokecolor{currentstroke}%
\pgfsetdash{}{0pt}%
\pgfpathmoveto{\pgfqpoint{4.695977in}{1.564545in}}%
\pgfpathlineto{\pgfqpoint{4.695977in}{2.365303in}}%
\pgfusepath{stroke}%
\end{pgfscope}%
\begin{pgfscope}%
\definecolor{textcolor}{rgb}{0.150000,0.150000,0.150000}%
\pgfsetstrokecolor{textcolor}%
\pgfsetfillcolor{textcolor}%
\pgftext[x=4.731140in,y=0.277680in,left,base,rotate=90.000000]{\color{textcolor}\rmfamily\fontsize{10.000000}{12.000000}\selectfont ARMA\_03\_MMM}%
\end{pgfscope}%
\begin{pgfscope}%
\pgfpathrectangle{\pgfqpoint{3.976651in}{1.564545in}}{\pgfqpoint{2.818182in}{0.800758in}}%
\pgfusepath{clip}%
\pgfsetroundcap%
\pgfsetroundjoin%
\pgfsetlinewidth{0.803000pt}%
\definecolor{currentstroke}{rgb}{1.000000,1.000000,1.000000}%
\pgfsetstrokecolor{currentstroke}%
\pgfsetdash{}{0pt}%
\pgfpathmoveto{\pgfqpoint{4.893052in}{1.564545in}}%
\pgfpathlineto{\pgfqpoint{4.893052in}{2.365303in}}%
\pgfusepath{stroke}%
\end{pgfscope}%
\begin{pgfscope}%
\definecolor{textcolor}{rgb}{0.150000,0.150000,0.150000}%
\pgfsetstrokecolor{textcolor}%
\pgfsetfillcolor{textcolor}%
\pgftext[x=4.928216in,y=0.277680in,left,base,rotate=90.000000]{\color{textcolor}\rmfamily\fontsize{10.000000}{12.000000}\selectfont ARMA\_10\_MMM}%
\end{pgfscope}%
\begin{pgfscope}%
\pgfpathrectangle{\pgfqpoint{3.976651in}{1.564545in}}{\pgfqpoint{2.818182in}{0.800758in}}%
\pgfusepath{clip}%
\pgfsetroundcap%
\pgfsetroundjoin%
\pgfsetlinewidth{0.803000pt}%
\definecolor{currentstroke}{rgb}{1.000000,1.000000,1.000000}%
\pgfsetstrokecolor{currentstroke}%
\pgfsetdash{}{0pt}%
\pgfpathmoveto{\pgfqpoint{5.090128in}{1.564545in}}%
\pgfpathlineto{\pgfqpoint{5.090128in}{2.365303in}}%
\pgfusepath{stroke}%
\end{pgfscope}%
\begin{pgfscope}%
\definecolor{textcolor}{rgb}{0.150000,0.150000,0.150000}%
\pgfsetstrokecolor{textcolor}%
\pgfsetfillcolor{textcolor}%
\pgftext[x=5.125291in,y=0.277680in,left,base,rotate=90.000000]{\color{textcolor}\rmfamily\fontsize{10.000000}{12.000000}\selectfont ARMA\_11\_MMM}%
\end{pgfscope}%
\begin{pgfscope}%
\pgfpathrectangle{\pgfqpoint{3.976651in}{1.564545in}}{\pgfqpoint{2.818182in}{0.800758in}}%
\pgfusepath{clip}%
\pgfsetroundcap%
\pgfsetroundjoin%
\pgfsetlinewidth{0.803000pt}%
\definecolor{currentstroke}{rgb}{1.000000,1.000000,1.000000}%
\pgfsetstrokecolor{currentstroke}%
\pgfsetdash{}{0pt}%
\pgfpathmoveto{\pgfqpoint{5.287204in}{1.564545in}}%
\pgfpathlineto{\pgfqpoint{5.287204in}{2.365303in}}%
\pgfusepath{stroke}%
\end{pgfscope}%
\begin{pgfscope}%
\definecolor{textcolor}{rgb}{0.150000,0.150000,0.150000}%
\pgfsetstrokecolor{textcolor}%
\pgfsetfillcolor{textcolor}%
\pgftext[x=5.322367in,y=0.277680in,left,base,rotate=90.000000]{\color{textcolor}\rmfamily\fontsize{10.000000}{12.000000}\selectfont ARMA\_12\_MMM}%
\end{pgfscope}%
\begin{pgfscope}%
\pgfpathrectangle{\pgfqpoint{3.976651in}{1.564545in}}{\pgfqpoint{2.818182in}{0.800758in}}%
\pgfusepath{clip}%
\pgfsetroundcap%
\pgfsetroundjoin%
\pgfsetlinewidth{0.803000pt}%
\definecolor{currentstroke}{rgb}{1.000000,1.000000,1.000000}%
\pgfsetstrokecolor{currentstroke}%
\pgfsetdash{}{0pt}%
\pgfpathmoveto{\pgfqpoint{5.484279in}{1.564545in}}%
\pgfpathlineto{\pgfqpoint{5.484279in}{2.365303in}}%
\pgfusepath{stroke}%
\end{pgfscope}%
\begin{pgfscope}%
\definecolor{textcolor}{rgb}{0.150000,0.150000,0.150000}%
\pgfsetstrokecolor{textcolor}%
\pgfsetfillcolor{textcolor}%
\pgftext[x=5.519442in,y=0.277680in,left,base,rotate=90.000000]{\color{textcolor}\rmfamily\fontsize{10.000000}{12.000000}\selectfont ARMA\_13\_MMM}%
\end{pgfscope}%
\begin{pgfscope}%
\pgfpathrectangle{\pgfqpoint{3.976651in}{1.564545in}}{\pgfqpoint{2.818182in}{0.800758in}}%
\pgfusepath{clip}%
\pgfsetroundcap%
\pgfsetroundjoin%
\pgfsetlinewidth{0.803000pt}%
\definecolor{currentstroke}{rgb}{1.000000,1.000000,1.000000}%
\pgfsetstrokecolor{currentstroke}%
\pgfsetdash{}{0pt}%
\pgfpathmoveto{\pgfqpoint{5.681355in}{1.564545in}}%
\pgfpathlineto{\pgfqpoint{5.681355in}{2.365303in}}%
\pgfusepath{stroke}%
\end{pgfscope}%
\begin{pgfscope}%
\definecolor{textcolor}{rgb}{0.150000,0.150000,0.150000}%
\pgfsetstrokecolor{textcolor}%
\pgfsetfillcolor{textcolor}%
\pgftext[x=5.716518in,y=0.277680in,left,base,rotate=90.000000]{\color{textcolor}\rmfamily\fontsize{10.000000}{12.000000}\selectfont ARMA\_20\_MMM}%
\end{pgfscope}%
\begin{pgfscope}%
\pgfpathrectangle{\pgfqpoint{3.976651in}{1.564545in}}{\pgfqpoint{2.818182in}{0.800758in}}%
\pgfusepath{clip}%
\pgfsetroundcap%
\pgfsetroundjoin%
\pgfsetlinewidth{0.803000pt}%
\definecolor{currentstroke}{rgb}{1.000000,1.000000,1.000000}%
\pgfsetstrokecolor{currentstroke}%
\pgfsetdash{}{0pt}%
\pgfpathmoveto{\pgfqpoint{5.878431in}{1.564545in}}%
\pgfpathlineto{\pgfqpoint{5.878431in}{2.365303in}}%
\pgfusepath{stroke}%
\end{pgfscope}%
\begin{pgfscope}%
\definecolor{textcolor}{rgb}{0.150000,0.150000,0.150000}%
\pgfsetstrokecolor{textcolor}%
\pgfsetfillcolor{textcolor}%
\pgftext[x=5.913594in,y=0.277680in,left,base,rotate=90.000000]{\color{textcolor}\rmfamily\fontsize{10.000000}{12.000000}\selectfont ARMA\_21\_MMM}%
\end{pgfscope}%
\begin{pgfscope}%
\pgfpathrectangle{\pgfqpoint{3.976651in}{1.564545in}}{\pgfqpoint{2.818182in}{0.800758in}}%
\pgfusepath{clip}%
\pgfsetroundcap%
\pgfsetroundjoin%
\pgfsetlinewidth{0.803000pt}%
\definecolor{currentstroke}{rgb}{1.000000,1.000000,1.000000}%
\pgfsetstrokecolor{currentstroke}%
\pgfsetdash{}{0pt}%
\pgfpathmoveto{\pgfqpoint{6.075506in}{1.564545in}}%
\pgfpathlineto{\pgfqpoint{6.075506in}{2.365303in}}%
\pgfusepath{stroke}%
\end{pgfscope}%
\begin{pgfscope}%
\definecolor{textcolor}{rgb}{0.150000,0.150000,0.150000}%
\pgfsetstrokecolor{textcolor}%
\pgfsetfillcolor{textcolor}%
\pgftext[x=6.110669in,y=0.277680in,left,base,rotate=90.000000]{\color{textcolor}\rmfamily\fontsize{10.000000}{12.000000}\selectfont ARMA\_22\_MMM}%
\end{pgfscope}%
\begin{pgfscope}%
\pgfpathrectangle{\pgfqpoint{3.976651in}{1.564545in}}{\pgfqpoint{2.818182in}{0.800758in}}%
\pgfusepath{clip}%
\pgfsetroundcap%
\pgfsetroundjoin%
\pgfsetlinewidth{0.803000pt}%
\definecolor{currentstroke}{rgb}{1.000000,1.000000,1.000000}%
\pgfsetstrokecolor{currentstroke}%
\pgfsetdash{}{0pt}%
\pgfpathmoveto{\pgfqpoint{6.272582in}{1.564545in}}%
\pgfpathlineto{\pgfqpoint{6.272582in}{2.365303in}}%
\pgfusepath{stroke}%
\end{pgfscope}%
\begin{pgfscope}%
\definecolor{textcolor}{rgb}{0.150000,0.150000,0.150000}%
\pgfsetstrokecolor{textcolor}%
\pgfsetfillcolor{textcolor}%
\pgftext[x=6.307745in,y=0.277680in,left,base,rotate=90.000000]{\color{textcolor}\rmfamily\fontsize{10.000000}{12.000000}\selectfont ARMA\_30\_MMM}%
\end{pgfscope}%
\begin{pgfscope}%
\pgfpathrectangle{\pgfqpoint{3.976651in}{1.564545in}}{\pgfqpoint{2.818182in}{0.800758in}}%
\pgfusepath{clip}%
\pgfsetroundcap%
\pgfsetroundjoin%
\pgfsetlinewidth{0.803000pt}%
\definecolor{currentstroke}{rgb}{1.000000,1.000000,1.000000}%
\pgfsetstrokecolor{currentstroke}%
\pgfsetdash{}{0pt}%
\pgfpathmoveto{\pgfqpoint{6.469658in}{1.564545in}}%
\pgfpathlineto{\pgfqpoint{6.469658in}{2.365303in}}%
\pgfusepath{stroke}%
\end{pgfscope}%
\begin{pgfscope}%
\definecolor{textcolor}{rgb}{0.150000,0.150000,0.150000}%
\pgfsetstrokecolor{textcolor}%
\pgfsetfillcolor{textcolor}%
\pgftext[x=6.504821in,y=0.277680in,left,base,rotate=90.000000]{\color{textcolor}\rmfamily\fontsize{10.000000}{12.000000}\selectfont ARMA\_31\_MMM}%
\end{pgfscope}%
\begin{pgfscope}%
\pgfpathrectangle{\pgfqpoint{3.976651in}{1.564545in}}{\pgfqpoint{2.818182in}{0.800758in}}%
\pgfusepath{clip}%
\pgfsetroundcap%
\pgfsetroundjoin%
\pgfsetlinewidth{0.803000pt}%
\definecolor{currentstroke}{rgb}{1.000000,1.000000,1.000000}%
\pgfsetstrokecolor{currentstroke}%
\pgfsetdash{}{0pt}%
\pgfpathmoveto{\pgfqpoint{6.666733in}{1.564545in}}%
\pgfpathlineto{\pgfqpoint{6.666733in}{2.365303in}}%
\pgfusepath{stroke}%
\end{pgfscope}%
\begin{pgfscope}%
\definecolor{textcolor}{rgb}{0.150000,0.150000,0.150000}%
\pgfsetstrokecolor{textcolor}%
\pgfsetfillcolor{textcolor}%
\pgftext[x=6.701896in,y=0.100000in,left,base,rotate=90.000000]{\color{textcolor}\rmfamily\fontsize{10.000000}{12.000000}\selectfont AUTOARMA\_MMM}%
\end{pgfscope}%
\begin{pgfscope}%
\pgfpathrectangle{\pgfqpoint{3.976651in}{1.564545in}}{\pgfqpoint{2.818182in}{0.800758in}}%
\pgfusepath{clip}%
\pgfsetroundcap%
\pgfsetroundjoin%
\pgfsetlinewidth{0.803000pt}%
\definecolor{currentstroke}{rgb}{1.000000,1.000000,1.000000}%
\pgfsetstrokecolor{currentstroke}%
\pgfsetdash{}{0pt}%
\pgfpathmoveto{\pgfqpoint{3.976651in}{1.614125in}}%
\pgfpathlineto{\pgfqpoint{6.794832in}{1.614125in}}%
\pgfusepath{stroke}%
\end{pgfscope}%
\begin{pgfscope}%
\definecolor{textcolor}{rgb}{0.150000,0.150000,0.150000}%
\pgfsetstrokecolor{textcolor}%
\pgfsetfillcolor{textcolor}%
\pgftext[x=3.570183in,y=1.561364in,left,base]{\color{textcolor}\rmfamily\fontsize{10.000000}{12.000000}\selectfont 0.20}%
\end{pgfscope}%
\begin{pgfscope}%
\pgfpathrectangle{\pgfqpoint{3.976651in}{1.564545in}}{\pgfqpoint{2.818182in}{0.800758in}}%
\pgfusepath{clip}%
\pgfsetroundcap%
\pgfsetroundjoin%
\pgfsetlinewidth{0.803000pt}%
\definecolor{currentstroke}{rgb}{1.000000,1.000000,1.000000}%
\pgfsetstrokecolor{currentstroke}%
\pgfsetdash{}{0pt}%
\pgfpathmoveto{\pgfqpoint{3.976651in}{2.306980in}}%
\pgfpathlineto{\pgfqpoint{6.794832in}{2.306980in}}%
\pgfusepath{stroke}%
\end{pgfscope}%
\begin{pgfscope}%
\definecolor{textcolor}{rgb}{0.150000,0.150000,0.150000}%
\pgfsetstrokecolor{textcolor}%
\pgfsetfillcolor{textcolor}%
\pgftext[x=3.570183in,y=2.254218in,left,base]{\color{textcolor}\rmfamily\fontsize{10.000000}{12.000000}\selectfont 0.25}%
\end{pgfscope}%
\begin{pgfscope}%
\pgfpathrectangle{\pgfqpoint{3.976651in}{1.564545in}}{\pgfqpoint{2.818182in}{0.800758in}}%
\pgfusepath{clip}%
\pgfsetroundcap%
\pgfsetroundjoin%
\pgfsetlinewidth{1.505625pt}%
\definecolor{currentstroke}{rgb}{0.090196,0.745098,0.811765}%
\pgfsetstrokecolor{currentstroke}%
\pgfsetdash{}{0pt}%
\pgfpathmoveto{\pgfqpoint{4.104750in}{1.610326in}}%
\pgfpathlineto{\pgfqpoint{4.301825in}{1.621147in}}%
\pgfpathlineto{\pgfqpoint{4.498901in}{1.635598in}}%
\pgfpathlineto{\pgfqpoint{4.695977in}{1.604889in}}%
\pgfpathlineto{\pgfqpoint{4.893052in}{1.622714in}}%
\pgfpathlineto{\pgfqpoint{5.090128in}{1.600943in}}%
\pgfpathlineto{\pgfqpoint{5.287204in}{1.609696in}}%
\pgfpathlineto{\pgfqpoint{5.484279in}{1.604889in}}%
\pgfpathlineto{\pgfqpoint{5.681355in}{1.634157in}}%
\pgfpathlineto{\pgfqpoint{5.878431in}{2.328905in}}%
\pgfpathlineto{\pgfqpoint{6.075506in}{1.601422in}}%
\pgfpathlineto{\pgfqpoint{6.272582in}{1.604889in}}%
\pgfpathlineto{\pgfqpoint{6.469658in}{1.604889in}}%
\pgfpathlineto{\pgfqpoint{6.666733in}{1.605044in}}%
\pgfusepath{stroke}%
\end{pgfscope}%
\begin{pgfscope}%
\pgfpathrectangle{\pgfqpoint{3.976651in}{1.564545in}}{\pgfqpoint{2.818182in}{0.800758in}}%
\pgfusepath{clip}%
\pgfsetbuttcap%
\pgfsetroundjoin%
\definecolor{currentfill}{rgb}{0.090196,0.745098,0.811765}%
\pgfsetfillcolor{currentfill}%
\pgfsetlinewidth{1.003750pt}%
\definecolor{currentstroke}{rgb}{0.090196,0.745098,0.811765}%
\pgfsetstrokecolor{currentstroke}%
\pgfsetdash{}{0pt}%
\pgfsys@defobject{currentmarker}{\pgfqpoint{-0.041667in}{-0.041667in}}{\pgfqpoint{0.041667in}{0.041667in}}{%
\pgfpathmoveto{\pgfqpoint{0.000000in}{-0.041667in}}%
\pgfpathcurveto{\pgfqpoint{0.011050in}{-0.041667in}}{\pgfqpoint{0.021649in}{-0.037276in}}{\pgfqpoint{0.029463in}{-0.029463in}}%
\pgfpathcurveto{\pgfqpoint{0.037276in}{-0.021649in}}{\pgfqpoint{0.041667in}{-0.011050in}}{\pgfqpoint{0.041667in}{0.000000in}}%
\pgfpathcurveto{\pgfqpoint{0.041667in}{0.011050in}}{\pgfqpoint{0.037276in}{0.021649in}}{\pgfqpoint{0.029463in}{0.029463in}}%
\pgfpathcurveto{\pgfqpoint{0.021649in}{0.037276in}}{\pgfqpoint{0.011050in}{0.041667in}}{\pgfqpoint{0.000000in}{0.041667in}}%
\pgfpathcurveto{\pgfqpoint{-0.011050in}{0.041667in}}{\pgfqpoint{-0.021649in}{0.037276in}}{\pgfqpoint{-0.029463in}{0.029463in}}%
\pgfpathcurveto{\pgfqpoint{-0.037276in}{0.021649in}}{\pgfqpoint{-0.041667in}{0.011050in}}{\pgfqpoint{-0.041667in}{0.000000in}}%
\pgfpathcurveto{\pgfqpoint{-0.041667in}{-0.011050in}}{\pgfqpoint{-0.037276in}{-0.021649in}}{\pgfqpoint{-0.029463in}{-0.029463in}}%
\pgfpathcurveto{\pgfqpoint{-0.021649in}{-0.037276in}}{\pgfqpoint{-0.011050in}{-0.041667in}}{\pgfqpoint{0.000000in}{-0.041667in}}%
\pgfpathclose%
\pgfusepath{stroke,fill}%
}%
\begin{pgfscope}%
\pgfsys@transformshift{4.104750in}{1.610326in}%
\pgfsys@useobject{currentmarker}{}%
\end{pgfscope}%
\begin{pgfscope}%
\pgfsys@transformshift{4.301825in}{1.621147in}%
\pgfsys@useobject{currentmarker}{}%
\end{pgfscope}%
\begin{pgfscope}%
\pgfsys@transformshift{4.498901in}{1.635598in}%
\pgfsys@useobject{currentmarker}{}%
\end{pgfscope}%
\begin{pgfscope}%
\pgfsys@transformshift{4.695977in}{1.604889in}%
\pgfsys@useobject{currentmarker}{}%
\end{pgfscope}%
\begin{pgfscope}%
\pgfsys@transformshift{4.893052in}{1.622714in}%
\pgfsys@useobject{currentmarker}{}%
\end{pgfscope}%
\begin{pgfscope}%
\pgfsys@transformshift{5.090128in}{1.600943in}%
\pgfsys@useobject{currentmarker}{}%
\end{pgfscope}%
\begin{pgfscope}%
\pgfsys@transformshift{5.287204in}{1.609696in}%
\pgfsys@useobject{currentmarker}{}%
\end{pgfscope}%
\begin{pgfscope}%
\pgfsys@transformshift{5.484279in}{1.604889in}%
\pgfsys@useobject{currentmarker}{}%
\end{pgfscope}%
\begin{pgfscope}%
\pgfsys@transformshift{5.681355in}{1.634157in}%
\pgfsys@useobject{currentmarker}{}%
\end{pgfscope}%
\begin{pgfscope}%
\pgfsys@transformshift{5.878431in}{2.328905in}%
\pgfsys@useobject{currentmarker}{}%
\end{pgfscope}%
\begin{pgfscope}%
\pgfsys@transformshift{6.075506in}{1.601422in}%
\pgfsys@useobject{currentmarker}{}%
\end{pgfscope}%
\begin{pgfscope}%
\pgfsys@transformshift{6.272582in}{1.604889in}%
\pgfsys@useobject{currentmarker}{}%
\end{pgfscope}%
\begin{pgfscope}%
\pgfsys@transformshift{6.469658in}{1.604889in}%
\pgfsys@useobject{currentmarker}{}%
\end{pgfscope}%
\begin{pgfscope}%
\pgfsys@transformshift{6.666733in}{1.605044in}%
\pgfsys@useobject{currentmarker}{}%
\end{pgfscope}%
\end{pgfscope}%
\begin{pgfscope}%
\pgfsetrectcap%
\pgfsetmiterjoin%
\pgfsetlinewidth{0.803000pt}%
\definecolor{currentstroke}{rgb}{1.000000,1.000000,1.000000}%
\pgfsetstrokecolor{currentstroke}%
\pgfsetdash{}{0pt}%
\pgfpathmoveto{\pgfqpoint{3.976651in}{1.564545in}}%
\pgfpathlineto{\pgfqpoint{3.976651in}{2.365303in}}%
\pgfusepath{stroke}%
\end{pgfscope}%
\begin{pgfscope}%
\pgfsetrectcap%
\pgfsetmiterjoin%
\pgfsetlinewidth{0.803000pt}%
\definecolor{currentstroke}{rgb}{1.000000,1.000000,1.000000}%
\pgfsetstrokecolor{currentstroke}%
\pgfsetdash{}{0pt}%
\pgfpathmoveto{\pgfqpoint{6.794832in}{1.564545in}}%
\pgfpathlineto{\pgfqpoint{6.794832in}{2.365303in}}%
\pgfusepath{stroke}%
\end{pgfscope}%
\begin{pgfscope}%
\pgfsetrectcap%
\pgfsetmiterjoin%
\pgfsetlinewidth{0.803000pt}%
\definecolor{currentstroke}{rgb}{1.000000,1.000000,1.000000}%
\pgfsetstrokecolor{currentstroke}%
\pgfsetdash{}{0pt}%
\pgfpathmoveto{\pgfqpoint{3.976651in}{1.564545in}}%
\pgfpathlineto{\pgfqpoint{6.794832in}{1.564545in}}%
\pgfusepath{stroke}%
\end{pgfscope}%
\begin{pgfscope}%
\pgfsetrectcap%
\pgfsetmiterjoin%
\pgfsetlinewidth{0.803000pt}%
\definecolor{currentstroke}{rgb}{1.000000,1.000000,1.000000}%
\pgfsetstrokecolor{currentstroke}%
\pgfsetdash{}{0pt}%
\pgfpathmoveto{\pgfqpoint{3.976651in}{2.365303in}}%
\pgfpathlineto{\pgfqpoint{6.794832in}{2.365303in}}%
\pgfusepath{stroke}%
\end{pgfscope}%
\begin{pgfscope}%
\definecolor{textcolor}{rgb}{0.150000,0.150000,0.150000}%
\pgfsetstrokecolor{textcolor}%
\pgfsetfillcolor{textcolor}%
\pgftext[x=5.385741in,y=2.448636in,,base]{\color{textcolor}\rmfamily\fontsize{12.000000}{14.400000}\selectfont DIS}%
\end{pgfscope}%
\end{pgfpicture}%
\makeatother%
\endgroup%

    \end{adjustbox}
    \caption{SSE for the predictions generated by all models tested on all stocks.}
    \label{fig:SSE_all_predictions}
\end{figure}{}

\begin{figure}[h!]
    \centering
    \figuretitle{Models Ranked According to Performance}
    \begin{adjustbox}{width = 0.95\linewidth}
    %% Creator: Matplotlib, PGF backend
%%
%% To include the figure in your LaTeX document, write
%%   \input{<filename>.pgf}
%%
%% Make sure the required packages are loaded in your preamble
%%   \usepackage{pgf}
%%
%% Figures using additional raster images can only be included by \input if
%% they are in the same directory as the main LaTeX file. For loading figures
%% from other directories you can use the `import` package
%%   \usepackage{import}
%% and then include the figures with
%%   \import{<path to file>}{<filename>.pgf}
%%
%% Matplotlib used the following preamble
%%   \usepackage{fontspec}
%%   \setmainfont{DejaVuSerif.ttf}[Path=/opt/tljh/user/lib/python3.6/site-packages/matplotlib/mpl-data/fonts/ttf/]
%%   \setsansfont{DejaVuSans.ttf}[Path=/opt/tljh/user/lib/python3.6/site-packages/matplotlib/mpl-data/fonts/ttf/]
%%   \setmonofont{DejaVuSansMono.ttf}[Path=/opt/tljh/user/lib/python3.6/site-packages/matplotlib/mpl-data/fonts/ttf/]
%%
\begingroup%
\makeatletter%
\begin{pgfpicture}%
\pgfpathrectangle{\pgfpointorigin}{\pgfqpoint{6.762318in}{3.929545in}}%
\pgfusepath{use as bounding box, clip}%
\begin{pgfscope}%
\pgfsetbuttcap%
\pgfsetmiterjoin%
\definecolor{currentfill}{rgb}{1.000000,1.000000,1.000000}%
\pgfsetfillcolor{currentfill}%
\pgfsetlinewidth{0.000000pt}%
\definecolor{currentstroke}{rgb}{1.000000,1.000000,1.000000}%
\pgfsetstrokecolor{currentstroke}%
\pgfsetdash{}{0pt}%
\pgfpathmoveto{\pgfqpoint{0.000000in}{0.000000in}}%
\pgfpathlineto{\pgfqpoint{6.762318in}{0.000000in}}%
\pgfpathlineto{\pgfqpoint{6.762318in}{3.929545in}}%
\pgfpathlineto{\pgfqpoint{0.000000in}{3.929545in}}%
\pgfpathclose%
\pgfusepath{fill}%
\end{pgfscope}%
\begin{pgfscope}%
\pgfsetbuttcap%
\pgfsetmiterjoin%
\definecolor{currentfill}{rgb}{0.917647,0.917647,0.949020}%
\pgfsetfillcolor{currentfill}%
\pgfsetlinewidth{0.000000pt}%
\definecolor{currentstroke}{rgb}{0.000000,0.000000,0.000000}%
\pgfsetstrokecolor{currentstroke}%
\pgfsetstrokeopacity{0.000000}%
\pgfsetdash{}{0pt}%
\pgfpathmoveto{\pgfqpoint{0.462318in}{1.564545in}}%
\pgfpathlineto{\pgfqpoint{6.662318in}{1.564545in}}%
\pgfpathlineto{\pgfqpoint{6.662318in}{3.829545in}}%
\pgfpathlineto{\pgfqpoint{0.462318in}{3.829545in}}%
\pgfpathclose%
\pgfusepath{fill}%
\end{pgfscope}%
\begin{pgfscope}%
\pgfpathrectangle{\pgfqpoint{0.462318in}{1.564545in}}{\pgfqpoint{6.200000in}{2.265000in}}%
\pgfusepath{clip}%
\pgfsetroundcap%
\pgfsetroundjoin%
\pgfsetlinewidth{0.803000pt}%
\definecolor{currentstroke}{rgb}{1.000000,1.000000,1.000000}%
\pgfsetstrokecolor{currentstroke}%
\pgfsetdash{}{0pt}%
\pgfpathmoveto{\pgfqpoint{0.744136in}{1.564545in}}%
\pgfpathlineto{\pgfqpoint{0.744136in}{3.829545in}}%
\pgfusepath{stroke}%
\end{pgfscope}%
\begin{pgfscope}%
\definecolor{textcolor}{rgb}{0.150000,0.150000,0.150000}%
\pgfsetstrokecolor{textcolor}%
\pgfsetfillcolor{textcolor}%
\pgftext[x=0.779300in,y=0.277680in,left,base,rotate=90.000000]{\color{textcolor}\rmfamily\fontsize{10.000000}{12.000000}\selectfont ARMA\_00\_MMM}%
\end{pgfscope}%
\begin{pgfscope}%
\pgfpathrectangle{\pgfqpoint{0.462318in}{1.564545in}}{\pgfqpoint{6.200000in}{2.265000in}}%
\pgfusepath{clip}%
\pgfsetroundcap%
\pgfsetroundjoin%
\pgfsetlinewidth{0.803000pt}%
\definecolor{currentstroke}{rgb}{1.000000,1.000000,1.000000}%
\pgfsetstrokecolor{currentstroke}%
\pgfsetdash{}{0pt}%
\pgfpathmoveto{\pgfqpoint{1.177703in}{1.564545in}}%
\pgfpathlineto{\pgfqpoint{1.177703in}{3.829545in}}%
\pgfusepath{stroke}%
\end{pgfscope}%
\begin{pgfscope}%
\definecolor{textcolor}{rgb}{0.150000,0.150000,0.150000}%
\pgfsetstrokecolor{textcolor}%
\pgfsetfillcolor{textcolor}%
\pgftext[x=1.212866in,y=0.277680in,left,base,rotate=90.000000]{\color{textcolor}\rmfamily\fontsize{10.000000}{12.000000}\selectfont ARMA\_01\_MMM}%
\end{pgfscope}%
\begin{pgfscope}%
\pgfpathrectangle{\pgfqpoint{0.462318in}{1.564545in}}{\pgfqpoint{6.200000in}{2.265000in}}%
\pgfusepath{clip}%
\pgfsetroundcap%
\pgfsetroundjoin%
\pgfsetlinewidth{0.803000pt}%
\definecolor{currentstroke}{rgb}{1.000000,1.000000,1.000000}%
\pgfsetstrokecolor{currentstroke}%
\pgfsetdash{}{0pt}%
\pgfpathmoveto{\pgfqpoint{1.611269in}{1.564545in}}%
\pgfpathlineto{\pgfqpoint{1.611269in}{3.829545in}}%
\pgfusepath{stroke}%
\end{pgfscope}%
\begin{pgfscope}%
\definecolor{textcolor}{rgb}{0.150000,0.150000,0.150000}%
\pgfsetstrokecolor{textcolor}%
\pgfsetfillcolor{textcolor}%
\pgftext[x=1.646432in,y=0.277680in,left,base,rotate=90.000000]{\color{textcolor}\rmfamily\fontsize{10.000000}{12.000000}\selectfont ARMA\_02\_MMM}%
\end{pgfscope}%
\begin{pgfscope}%
\pgfpathrectangle{\pgfqpoint{0.462318in}{1.564545in}}{\pgfqpoint{6.200000in}{2.265000in}}%
\pgfusepath{clip}%
\pgfsetroundcap%
\pgfsetroundjoin%
\pgfsetlinewidth{0.803000pt}%
\definecolor{currentstroke}{rgb}{1.000000,1.000000,1.000000}%
\pgfsetstrokecolor{currentstroke}%
\pgfsetdash{}{0pt}%
\pgfpathmoveto{\pgfqpoint{2.044836in}{1.564545in}}%
\pgfpathlineto{\pgfqpoint{2.044836in}{3.829545in}}%
\pgfusepath{stroke}%
\end{pgfscope}%
\begin{pgfscope}%
\definecolor{textcolor}{rgb}{0.150000,0.150000,0.150000}%
\pgfsetstrokecolor{textcolor}%
\pgfsetfillcolor{textcolor}%
\pgftext[x=2.079999in,y=0.277680in,left,base,rotate=90.000000]{\color{textcolor}\rmfamily\fontsize{10.000000}{12.000000}\selectfont ARMA\_03\_MMM}%
\end{pgfscope}%
\begin{pgfscope}%
\pgfpathrectangle{\pgfqpoint{0.462318in}{1.564545in}}{\pgfqpoint{6.200000in}{2.265000in}}%
\pgfusepath{clip}%
\pgfsetroundcap%
\pgfsetroundjoin%
\pgfsetlinewidth{0.803000pt}%
\definecolor{currentstroke}{rgb}{1.000000,1.000000,1.000000}%
\pgfsetstrokecolor{currentstroke}%
\pgfsetdash{}{0pt}%
\pgfpathmoveto{\pgfqpoint{2.478402in}{1.564545in}}%
\pgfpathlineto{\pgfqpoint{2.478402in}{3.829545in}}%
\pgfusepath{stroke}%
\end{pgfscope}%
\begin{pgfscope}%
\definecolor{textcolor}{rgb}{0.150000,0.150000,0.150000}%
\pgfsetstrokecolor{textcolor}%
\pgfsetfillcolor{textcolor}%
\pgftext[x=2.513565in,y=0.277680in,left,base,rotate=90.000000]{\color{textcolor}\rmfamily\fontsize{10.000000}{12.000000}\selectfont ARMA\_10\_MMM}%
\end{pgfscope}%
\begin{pgfscope}%
\pgfpathrectangle{\pgfqpoint{0.462318in}{1.564545in}}{\pgfqpoint{6.200000in}{2.265000in}}%
\pgfusepath{clip}%
\pgfsetroundcap%
\pgfsetroundjoin%
\pgfsetlinewidth{0.803000pt}%
\definecolor{currentstroke}{rgb}{1.000000,1.000000,1.000000}%
\pgfsetstrokecolor{currentstroke}%
\pgfsetdash{}{0pt}%
\pgfpathmoveto{\pgfqpoint{2.911969in}{1.564545in}}%
\pgfpathlineto{\pgfqpoint{2.911969in}{3.829545in}}%
\pgfusepath{stroke}%
\end{pgfscope}%
\begin{pgfscope}%
\definecolor{textcolor}{rgb}{0.150000,0.150000,0.150000}%
\pgfsetstrokecolor{textcolor}%
\pgfsetfillcolor{textcolor}%
\pgftext[x=2.947132in,y=0.277680in,left,base,rotate=90.000000]{\color{textcolor}\rmfamily\fontsize{10.000000}{12.000000}\selectfont ARMA\_11\_MMM}%
\end{pgfscope}%
\begin{pgfscope}%
\pgfpathrectangle{\pgfqpoint{0.462318in}{1.564545in}}{\pgfqpoint{6.200000in}{2.265000in}}%
\pgfusepath{clip}%
\pgfsetroundcap%
\pgfsetroundjoin%
\pgfsetlinewidth{0.803000pt}%
\definecolor{currentstroke}{rgb}{1.000000,1.000000,1.000000}%
\pgfsetstrokecolor{currentstroke}%
\pgfsetdash{}{0pt}%
\pgfpathmoveto{\pgfqpoint{3.345535in}{1.564545in}}%
\pgfpathlineto{\pgfqpoint{3.345535in}{3.829545in}}%
\pgfusepath{stroke}%
\end{pgfscope}%
\begin{pgfscope}%
\definecolor{textcolor}{rgb}{0.150000,0.150000,0.150000}%
\pgfsetstrokecolor{textcolor}%
\pgfsetfillcolor{textcolor}%
\pgftext[x=3.380698in,y=0.277680in,left,base,rotate=90.000000]{\color{textcolor}\rmfamily\fontsize{10.000000}{12.000000}\selectfont ARMA\_12\_MMM}%
\end{pgfscope}%
\begin{pgfscope}%
\pgfpathrectangle{\pgfqpoint{0.462318in}{1.564545in}}{\pgfqpoint{6.200000in}{2.265000in}}%
\pgfusepath{clip}%
\pgfsetroundcap%
\pgfsetroundjoin%
\pgfsetlinewidth{0.803000pt}%
\definecolor{currentstroke}{rgb}{1.000000,1.000000,1.000000}%
\pgfsetstrokecolor{currentstroke}%
\pgfsetdash{}{0pt}%
\pgfpathmoveto{\pgfqpoint{3.779101in}{1.564545in}}%
\pgfpathlineto{\pgfqpoint{3.779101in}{3.829545in}}%
\pgfusepath{stroke}%
\end{pgfscope}%
\begin{pgfscope}%
\definecolor{textcolor}{rgb}{0.150000,0.150000,0.150000}%
\pgfsetstrokecolor{textcolor}%
\pgfsetfillcolor{textcolor}%
\pgftext[x=3.814265in,y=0.277680in,left,base,rotate=90.000000]{\color{textcolor}\rmfamily\fontsize{10.000000}{12.000000}\selectfont ARMA\_13\_MMM}%
\end{pgfscope}%
\begin{pgfscope}%
\pgfpathrectangle{\pgfqpoint{0.462318in}{1.564545in}}{\pgfqpoint{6.200000in}{2.265000in}}%
\pgfusepath{clip}%
\pgfsetroundcap%
\pgfsetroundjoin%
\pgfsetlinewidth{0.803000pt}%
\definecolor{currentstroke}{rgb}{1.000000,1.000000,1.000000}%
\pgfsetstrokecolor{currentstroke}%
\pgfsetdash{}{0pt}%
\pgfpathmoveto{\pgfqpoint{4.212668in}{1.564545in}}%
\pgfpathlineto{\pgfqpoint{4.212668in}{3.829545in}}%
\pgfusepath{stroke}%
\end{pgfscope}%
\begin{pgfscope}%
\definecolor{textcolor}{rgb}{0.150000,0.150000,0.150000}%
\pgfsetstrokecolor{textcolor}%
\pgfsetfillcolor{textcolor}%
\pgftext[x=4.247831in,y=0.277680in,left,base,rotate=90.000000]{\color{textcolor}\rmfamily\fontsize{10.000000}{12.000000}\selectfont ARMA\_20\_MMM}%
\end{pgfscope}%
\begin{pgfscope}%
\pgfpathrectangle{\pgfqpoint{0.462318in}{1.564545in}}{\pgfqpoint{6.200000in}{2.265000in}}%
\pgfusepath{clip}%
\pgfsetroundcap%
\pgfsetroundjoin%
\pgfsetlinewidth{0.803000pt}%
\definecolor{currentstroke}{rgb}{1.000000,1.000000,1.000000}%
\pgfsetstrokecolor{currentstroke}%
\pgfsetdash{}{0pt}%
\pgfpathmoveto{\pgfqpoint{4.646234in}{1.564545in}}%
\pgfpathlineto{\pgfqpoint{4.646234in}{3.829545in}}%
\pgfusepath{stroke}%
\end{pgfscope}%
\begin{pgfscope}%
\definecolor{textcolor}{rgb}{0.150000,0.150000,0.150000}%
\pgfsetstrokecolor{textcolor}%
\pgfsetfillcolor{textcolor}%
\pgftext[x=4.681397in,y=0.277680in,left,base,rotate=90.000000]{\color{textcolor}\rmfamily\fontsize{10.000000}{12.000000}\selectfont ARMA\_21\_MMM}%
\end{pgfscope}%
\begin{pgfscope}%
\pgfpathrectangle{\pgfqpoint{0.462318in}{1.564545in}}{\pgfqpoint{6.200000in}{2.265000in}}%
\pgfusepath{clip}%
\pgfsetroundcap%
\pgfsetroundjoin%
\pgfsetlinewidth{0.803000pt}%
\definecolor{currentstroke}{rgb}{1.000000,1.000000,1.000000}%
\pgfsetstrokecolor{currentstroke}%
\pgfsetdash{}{0pt}%
\pgfpathmoveto{\pgfqpoint{5.079801in}{1.564545in}}%
\pgfpathlineto{\pgfqpoint{5.079801in}{3.829545in}}%
\pgfusepath{stroke}%
\end{pgfscope}%
\begin{pgfscope}%
\definecolor{textcolor}{rgb}{0.150000,0.150000,0.150000}%
\pgfsetstrokecolor{textcolor}%
\pgfsetfillcolor{textcolor}%
\pgftext[x=5.114964in,y=0.277680in,left,base,rotate=90.000000]{\color{textcolor}\rmfamily\fontsize{10.000000}{12.000000}\selectfont ARMA\_22\_MMM}%
\end{pgfscope}%
\begin{pgfscope}%
\pgfpathrectangle{\pgfqpoint{0.462318in}{1.564545in}}{\pgfqpoint{6.200000in}{2.265000in}}%
\pgfusepath{clip}%
\pgfsetroundcap%
\pgfsetroundjoin%
\pgfsetlinewidth{0.803000pt}%
\definecolor{currentstroke}{rgb}{1.000000,1.000000,1.000000}%
\pgfsetstrokecolor{currentstroke}%
\pgfsetdash{}{0pt}%
\pgfpathmoveto{\pgfqpoint{5.513367in}{1.564545in}}%
\pgfpathlineto{\pgfqpoint{5.513367in}{3.829545in}}%
\pgfusepath{stroke}%
\end{pgfscope}%
\begin{pgfscope}%
\definecolor{textcolor}{rgb}{0.150000,0.150000,0.150000}%
\pgfsetstrokecolor{textcolor}%
\pgfsetfillcolor{textcolor}%
\pgftext[x=5.548530in,y=0.277680in,left,base,rotate=90.000000]{\color{textcolor}\rmfamily\fontsize{10.000000}{12.000000}\selectfont ARMA\_30\_MMM}%
\end{pgfscope}%
\begin{pgfscope}%
\pgfpathrectangle{\pgfqpoint{0.462318in}{1.564545in}}{\pgfqpoint{6.200000in}{2.265000in}}%
\pgfusepath{clip}%
\pgfsetroundcap%
\pgfsetroundjoin%
\pgfsetlinewidth{0.803000pt}%
\definecolor{currentstroke}{rgb}{1.000000,1.000000,1.000000}%
\pgfsetstrokecolor{currentstroke}%
\pgfsetdash{}{0pt}%
\pgfpathmoveto{\pgfqpoint{5.946934in}{1.564545in}}%
\pgfpathlineto{\pgfqpoint{5.946934in}{3.829545in}}%
\pgfusepath{stroke}%
\end{pgfscope}%
\begin{pgfscope}%
\definecolor{textcolor}{rgb}{0.150000,0.150000,0.150000}%
\pgfsetstrokecolor{textcolor}%
\pgfsetfillcolor{textcolor}%
\pgftext[x=5.982097in,y=0.277680in,left,base,rotate=90.000000]{\color{textcolor}\rmfamily\fontsize{10.000000}{12.000000}\selectfont ARMA\_31\_MMM}%
\end{pgfscope}%
\begin{pgfscope}%
\pgfpathrectangle{\pgfqpoint{0.462318in}{1.564545in}}{\pgfqpoint{6.200000in}{2.265000in}}%
\pgfusepath{clip}%
\pgfsetroundcap%
\pgfsetroundjoin%
\pgfsetlinewidth{0.803000pt}%
\definecolor{currentstroke}{rgb}{1.000000,1.000000,1.000000}%
\pgfsetstrokecolor{currentstroke}%
\pgfsetdash{}{0pt}%
\pgfpathmoveto{\pgfqpoint{6.380500in}{1.564545in}}%
\pgfpathlineto{\pgfqpoint{6.380500in}{3.829545in}}%
\pgfusepath{stroke}%
\end{pgfscope}%
\begin{pgfscope}%
\definecolor{textcolor}{rgb}{0.150000,0.150000,0.150000}%
\pgfsetstrokecolor{textcolor}%
\pgfsetfillcolor{textcolor}%
\pgftext[x=6.415663in,y=0.100000in,left,base,rotate=90.000000]{\color{textcolor}\rmfamily\fontsize{10.000000}{12.000000}\selectfont AUTOARMA\_MMM}%
\end{pgfscope}%
\begin{pgfscope}%
\pgfpathrectangle{\pgfqpoint{0.462318in}{1.564545in}}{\pgfqpoint{6.200000in}{2.265000in}}%
\pgfusepath{clip}%
\pgfsetroundcap%
\pgfsetroundjoin%
\pgfsetlinewidth{0.803000pt}%
\definecolor{currentstroke}{rgb}{1.000000,1.000000,1.000000}%
\pgfsetstrokecolor{currentstroke}%
\pgfsetdash{}{0pt}%
\pgfpathmoveto{\pgfqpoint{0.462318in}{1.958685in}}%
\pgfpathlineto{\pgfqpoint{6.662318in}{1.958685in}}%
\pgfusepath{stroke}%
\end{pgfscope}%
\begin{pgfscope}%
\definecolor{textcolor}{rgb}{0.150000,0.150000,0.150000}%
\pgfsetstrokecolor{textcolor}%
\pgfsetfillcolor{textcolor}%
\pgftext[x=0.188365in,y=1.905923in,left,base]{\color{textcolor}\rmfamily\fontsize{10.000000}{12.000000}\selectfont 40}%
\end{pgfscope}%
\begin{pgfscope}%
\pgfpathrectangle{\pgfqpoint{0.462318in}{1.564545in}}{\pgfqpoint{6.200000in}{2.265000in}}%
\pgfusepath{clip}%
\pgfsetroundcap%
\pgfsetroundjoin%
\pgfsetlinewidth{0.803000pt}%
\definecolor{currentstroke}{rgb}{1.000000,1.000000,1.000000}%
\pgfsetstrokecolor{currentstroke}%
\pgfsetdash{}{0pt}%
\pgfpathmoveto{\pgfqpoint{0.462318in}{2.374662in}}%
\pgfpathlineto{\pgfqpoint{6.662318in}{2.374662in}}%
\pgfusepath{stroke}%
\end{pgfscope}%
\begin{pgfscope}%
\definecolor{textcolor}{rgb}{0.150000,0.150000,0.150000}%
\pgfsetstrokecolor{textcolor}%
\pgfsetfillcolor{textcolor}%
\pgftext[x=0.188365in,y=2.321901in,left,base]{\color{textcolor}\rmfamily\fontsize{10.000000}{12.000000}\selectfont 60}%
\end{pgfscope}%
\begin{pgfscope}%
\pgfpathrectangle{\pgfqpoint{0.462318in}{1.564545in}}{\pgfqpoint{6.200000in}{2.265000in}}%
\pgfusepath{clip}%
\pgfsetroundcap%
\pgfsetroundjoin%
\pgfsetlinewidth{0.803000pt}%
\definecolor{currentstroke}{rgb}{1.000000,1.000000,1.000000}%
\pgfsetstrokecolor{currentstroke}%
\pgfsetdash{}{0pt}%
\pgfpathmoveto{\pgfqpoint{0.462318in}{2.790640in}}%
\pgfpathlineto{\pgfqpoint{6.662318in}{2.790640in}}%
\pgfusepath{stroke}%
\end{pgfscope}%
\begin{pgfscope}%
\definecolor{textcolor}{rgb}{0.150000,0.150000,0.150000}%
\pgfsetstrokecolor{textcolor}%
\pgfsetfillcolor{textcolor}%
\pgftext[x=0.188365in,y=2.737879in,left,base]{\color{textcolor}\rmfamily\fontsize{10.000000}{12.000000}\selectfont 80}%
\end{pgfscope}%
\begin{pgfscope}%
\pgfpathrectangle{\pgfqpoint{0.462318in}{1.564545in}}{\pgfqpoint{6.200000in}{2.265000in}}%
\pgfusepath{clip}%
\pgfsetroundcap%
\pgfsetroundjoin%
\pgfsetlinewidth{0.803000pt}%
\definecolor{currentstroke}{rgb}{1.000000,1.000000,1.000000}%
\pgfsetstrokecolor{currentstroke}%
\pgfsetdash{}{0pt}%
\pgfpathmoveto{\pgfqpoint{0.462318in}{3.206618in}}%
\pgfpathlineto{\pgfqpoint{6.662318in}{3.206618in}}%
\pgfusepath{stroke}%
\end{pgfscope}%
\begin{pgfscope}%
\definecolor{textcolor}{rgb}{0.150000,0.150000,0.150000}%
\pgfsetstrokecolor{textcolor}%
\pgfsetfillcolor{textcolor}%
\pgftext[x=0.100000in,y=3.153857in,left,base]{\color{textcolor}\rmfamily\fontsize{10.000000}{12.000000}\selectfont 100}%
\end{pgfscope}%
\begin{pgfscope}%
\pgfpathrectangle{\pgfqpoint{0.462318in}{1.564545in}}{\pgfqpoint{6.200000in}{2.265000in}}%
\pgfusepath{clip}%
\pgfsetroundcap%
\pgfsetroundjoin%
\pgfsetlinewidth{0.803000pt}%
\definecolor{currentstroke}{rgb}{1.000000,1.000000,1.000000}%
\pgfsetstrokecolor{currentstroke}%
\pgfsetdash{}{0pt}%
\pgfpathmoveto{\pgfqpoint{0.462318in}{3.622596in}}%
\pgfpathlineto{\pgfqpoint{6.662318in}{3.622596in}}%
\pgfusepath{stroke}%
\end{pgfscope}%
\begin{pgfscope}%
\definecolor{textcolor}{rgb}{0.150000,0.150000,0.150000}%
\pgfsetstrokecolor{textcolor}%
\pgfsetfillcolor{textcolor}%
\pgftext[x=0.100000in,y=3.569835in,left,base]{\color{textcolor}\rmfamily\fontsize{10.000000}{12.000000}\selectfont 120}%
\end{pgfscope}%
\begin{pgfscope}%
\pgfpathrectangle{\pgfqpoint{0.462318in}{1.564545in}}{\pgfqpoint{6.200000in}{2.265000in}}%
\pgfusepath{clip}%
\pgfsetroundcap%
\pgfsetroundjoin%
\pgfsetlinewidth{1.505625pt}%
\definecolor{currentstroke}{rgb}{0.000000,0.000000,1.000000}%
\pgfsetstrokecolor{currentstroke}%
\pgfsetdash{}{0pt}%
\pgfpathmoveto{\pgfqpoint{0.744136in}{1.667500in}}%
\pgfpathlineto{\pgfqpoint{1.177703in}{2.395461in}}%
\pgfpathlineto{\pgfqpoint{1.611269in}{3.081825in}}%
\pgfpathlineto{\pgfqpoint{2.044836in}{3.726591in}}%
\pgfpathlineto{\pgfqpoint{2.478402in}{2.208271in}}%
\pgfpathlineto{\pgfqpoint{2.911969in}{2.104277in}}%
\pgfpathlineto{\pgfqpoint{3.345535in}{2.104277in}}%
\pgfpathlineto{\pgfqpoint{3.779101in}{1.750696in}}%
\pgfpathlineto{\pgfqpoint{4.212668in}{2.873836in}}%
\pgfpathlineto{\pgfqpoint{4.646234in}{3.684993in}}%
\pgfpathlineto{\pgfqpoint{5.079801in}{3.123423in}}%
\pgfpathlineto{\pgfqpoint{5.513367in}{3.456205in}}%
\pgfpathlineto{\pgfqpoint{5.946934in}{3.539401in}}%
\pgfpathlineto{\pgfqpoint{6.380500in}{1.896288in}}%
\pgfusepath{stroke}%
\end{pgfscope}%
\begin{pgfscope}%
\pgfpathrectangle{\pgfqpoint{0.462318in}{1.564545in}}{\pgfqpoint{6.200000in}{2.265000in}}%
\pgfusepath{clip}%
\pgfsetbuttcap%
\pgfsetroundjoin%
\definecolor{currentfill}{rgb}{0.000000,0.000000,1.000000}%
\pgfsetfillcolor{currentfill}%
\pgfsetlinewidth{1.003750pt}%
\definecolor{currentstroke}{rgb}{0.000000,0.000000,1.000000}%
\pgfsetstrokecolor{currentstroke}%
\pgfsetdash{}{0pt}%
\pgfsys@defobject{currentmarker}{\pgfqpoint{-0.041667in}{-0.041667in}}{\pgfqpoint{0.041667in}{0.041667in}}{%
\pgfpathmoveto{\pgfqpoint{0.000000in}{-0.041667in}}%
\pgfpathcurveto{\pgfqpoint{0.011050in}{-0.041667in}}{\pgfqpoint{0.021649in}{-0.037276in}}{\pgfqpoint{0.029463in}{-0.029463in}}%
\pgfpathcurveto{\pgfqpoint{0.037276in}{-0.021649in}}{\pgfqpoint{0.041667in}{-0.011050in}}{\pgfqpoint{0.041667in}{0.000000in}}%
\pgfpathcurveto{\pgfqpoint{0.041667in}{0.011050in}}{\pgfqpoint{0.037276in}{0.021649in}}{\pgfqpoint{0.029463in}{0.029463in}}%
\pgfpathcurveto{\pgfqpoint{0.021649in}{0.037276in}}{\pgfqpoint{0.011050in}{0.041667in}}{\pgfqpoint{0.000000in}{0.041667in}}%
\pgfpathcurveto{\pgfqpoint{-0.011050in}{0.041667in}}{\pgfqpoint{-0.021649in}{0.037276in}}{\pgfqpoint{-0.029463in}{0.029463in}}%
\pgfpathcurveto{\pgfqpoint{-0.037276in}{0.021649in}}{\pgfqpoint{-0.041667in}{0.011050in}}{\pgfqpoint{-0.041667in}{0.000000in}}%
\pgfpathcurveto{\pgfqpoint{-0.041667in}{-0.011050in}}{\pgfqpoint{-0.037276in}{-0.021649in}}{\pgfqpoint{-0.029463in}{-0.029463in}}%
\pgfpathcurveto{\pgfqpoint{-0.021649in}{-0.037276in}}{\pgfqpoint{-0.011050in}{-0.041667in}}{\pgfqpoint{0.000000in}{-0.041667in}}%
\pgfpathclose%
\pgfusepath{stroke,fill}%
}%
\begin{pgfscope}%
\pgfsys@transformshift{0.744136in}{1.667500in}%
\pgfsys@useobject{currentmarker}{}%
\end{pgfscope}%
\begin{pgfscope}%
\pgfsys@transformshift{1.177703in}{2.395461in}%
\pgfsys@useobject{currentmarker}{}%
\end{pgfscope}%
\begin{pgfscope}%
\pgfsys@transformshift{1.611269in}{3.081825in}%
\pgfsys@useobject{currentmarker}{}%
\end{pgfscope}%
\begin{pgfscope}%
\pgfsys@transformshift{2.044836in}{3.726591in}%
\pgfsys@useobject{currentmarker}{}%
\end{pgfscope}%
\begin{pgfscope}%
\pgfsys@transformshift{2.478402in}{2.208271in}%
\pgfsys@useobject{currentmarker}{}%
\end{pgfscope}%
\begin{pgfscope}%
\pgfsys@transformshift{2.911969in}{2.104277in}%
\pgfsys@useobject{currentmarker}{}%
\end{pgfscope}%
\begin{pgfscope}%
\pgfsys@transformshift{3.345535in}{2.104277in}%
\pgfsys@useobject{currentmarker}{}%
\end{pgfscope}%
\begin{pgfscope}%
\pgfsys@transformshift{3.779101in}{1.750696in}%
\pgfsys@useobject{currentmarker}{}%
\end{pgfscope}%
\begin{pgfscope}%
\pgfsys@transformshift{4.212668in}{2.873836in}%
\pgfsys@useobject{currentmarker}{}%
\end{pgfscope}%
\begin{pgfscope}%
\pgfsys@transformshift{4.646234in}{3.684993in}%
\pgfsys@useobject{currentmarker}{}%
\end{pgfscope}%
\begin{pgfscope}%
\pgfsys@transformshift{5.079801in}{3.123423in}%
\pgfsys@useobject{currentmarker}{}%
\end{pgfscope}%
\begin{pgfscope}%
\pgfsys@transformshift{5.513367in}{3.456205in}%
\pgfsys@useobject{currentmarker}{}%
\end{pgfscope}%
\begin{pgfscope}%
\pgfsys@transformshift{5.946934in}{3.539401in}%
\pgfsys@useobject{currentmarker}{}%
\end{pgfscope}%
\begin{pgfscope}%
\pgfsys@transformshift{6.380500in}{1.896288in}%
\pgfsys@useobject{currentmarker}{}%
\end{pgfscope}%
\end{pgfscope}%
\begin{pgfscope}%
\pgfsetrectcap%
\pgfsetmiterjoin%
\pgfsetlinewidth{0.803000pt}%
\definecolor{currentstroke}{rgb}{1.000000,1.000000,1.000000}%
\pgfsetstrokecolor{currentstroke}%
\pgfsetdash{}{0pt}%
\pgfpathmoveto{\pgfqpoint{0.462318in}{1.564545in}}%
\pgfpathlineto{\pgfqpoint{0.462318in}{3.829545in}}%
\pgfusepath{stroke}%
\end{pgfscope}%
\begin{pgfscope}%
\pgfsetrectcap%
\pgfsetmiterjoin%
\pgfsetlinewidth{0.803000pt}%
\definecolor{currentstroke}{rgb}{1.000000,1.000000,1.000000}%
\pgfsetstrokecolor{currentstroke}%
\pgfsetdash{}{0pt}%
\pgfpathmoveto{\pgfqpoint{6.662318in}{1.564545in}}%
\pgfpathlineto{\pgfqpoint{6.662318in}{3.829545in}}%
\pgfusepath{stroke}%
\end{pgfscope}%
\begin{pgfscope}%
\pgfsetrectcap%
\pgfsetmiterjoin%
\pgfsetlinewidth{0.803000pt}%
\definecolor{currentstroke}{rgb}{1.000000,1.000000,1.000000}%
\pgfsetstrokecolor{currentstroke}%
\pgfsetdash{}{0pt}%
\pgfpathmoveto{\pgfqpoint{0.462318in}{1.564545in}}%
\pgfpathlineto{\pgfqpoint{6.662318in}{1.564545in}}%
\pgfusepath{stroke}%
\end{pgfscope}%
\begin{pgfscope}%
\pgfsetrectcap%
\pgfsetmiterjoin%
\pgfsetlinewidth{0.803000pt}%
\definecolor{currentstroke}{rgb}{1.000000,1.000000,1.000000}%
\pgfsetstrokecolor{currentstroke}%
\pgfsetdash{}{0pt}%
\pgfpathmoveto{\pgfqpoint{0.462318in}{3.829545in}}%
\pgfpathlineto{\pgfqpoint{6.662318in}{3.829545in}}%
\pgfusepath{stroke}%
\end{pgfscope}%
\end{pgfpicture}%
\makeatother%
\endgroup%

    \end{adjustbox}
    \caption{For every stock shown in figure \ref{fig:SSE_all_predictions} the different models were given a rank. This graph shows the sum of the rank achieved by the model across all stocks.}
    \label{fig:SSE_models_ranked}
\end{figure}{}






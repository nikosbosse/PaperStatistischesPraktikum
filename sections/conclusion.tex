\chapter{Conclusion}\label{ch:conclusion}

In this report we have made a sincere attempt to predict stocks by using text analysis, machine learning and time series. We have firstly looked at two different ways to generate sentiments ourselves from any body of text data, the Bag of Words methods and the Joint Sentiment Topic method. Because of the limitations imposed by the analyst reports data set, we also worked with the much richer data set of precomputed sentiment scores from RavenPack. We secondly tried to use those two data sets in conjuncture with the machine learning method XGBoost to generate stock predictions. Thirdly, we examined a variety of different time series models and applied them to the stock data. We then used those model to generate predictions and compared them against different baselines. Lastly, we looked at a whole range of theory guided trading and compared their effectiveness on our data set.

While we were ultimately (and unsurprisingly) not successful in predicting future stocks returns, we still made some interesting findings. First, we learned that a custom sentiment analysis using the Bag of Words is possible, but somewhat difficult, because it depends on good sentiment libraries that are often not freely available. Secondly, JST is a very interesting and powerful model, yet the results are very hard to interpret. Thirdly we learned that the Analyst Reports don't contain information useful for forecasting that we were able to extract. In light of the efficient market hypothesis, this is not surprising, as the reports usually don't contain new information that has not been priced in at the time of writing of the report. We cannot, however, explicitly assess the usefulness of the  recommendations made in the reports, because these have been meshed up with everything else in the report during preprocessing. We fourthly learned that the RavenPack sentiments do seem to contain information that is new and relevant to the market. This of course does not necessarily mean that the underlying news articles themselves influence the stock directly. It does, however, mean that the news collected by RavenPack are informed by events that really do influence the market. We were unfortunately not able to exploit this information, as we only have daily closing prices. By the time the market closes, all information has already been priced in and the RavenPack sentiments do not inform future returns. If one had access to the information and sentiment scores in real time, however, it might be possible to use that information for forecasting, even with the methods we have employed in this report. Fifthly, we learned that financial time series often look like they exhibit significant autocorrelation, but that these patterns may be due to random chance and do not allow reliably predicting future returns. Lastly, we saw that other trading strategies were also not successful in reliably generating excess returns - at least not in the way we implemented them. All of this does not definitively mean that predicting stocks is impossible. It is merely a lot harder than many popular internet sites and forums suggest. One can think of an intraday trader who is quicker than others in adapting to new information. One can also conceive that the trading strategies we discussed may work with some tweaking and over different time frames This, however, is left for future work. 

%they actually affect stocks, it merely means that information that is moving the market is also captured in the Ravenpack data. 






